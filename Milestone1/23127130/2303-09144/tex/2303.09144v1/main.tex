\documentclass[10pt]{article}
\usepackage{geometry}
\geometry{
    a4paper,
    total={171mm,248mm},
    left=19.5mm,
    top=20mm
}
\usepackage{graphicx} % Required for inserting images
\usepackage{xcolor}
\usepackage{bm}
\usepackage{mathtools}
\usepackage{amssymb,amsfonts,amsmath}
\usepackage{units}

\usepackage{tikz}
\usepackage{cite}
\usepackage{pgfplots,pgfplotstable}
\pgfplotsset{compat=newest}
\newcommand*\thankagain[1][\value{footnote}]{\footnotemark[#1]}

\usepackage{hyperref}
\hypersetup{pdfauthor={Lea Bold, Hannes Eschmann, Mario Rosenfelder, Henrik Ebel, Karl Worthmann},pdftitle={On Koopman-based surrogate models for non-holonomic robots},colorlinks=false, hidelinks}

\title{On Koopman-based surrogate models for non-holonomic robots}
\makeatletter
\let\@fnsymbol\@arabic
\makeatother
\author{Lea Bold\thanks{Optimization-based Control Group, Institute of Mathematics, Technische Universit\"at Ilmenau, Germany,
{\tt\small [lea.bold, karl.worthmann]@tu-ilmenau.de}.\newline K.\ Worthmann gratefully acknowledges funding by the German Research Foundation (DFG, project-ID 507037103).} ,
Hannes Eschmann\thanks{Institute of Engineering and Computational Mechanics~(ITM), University of Stuttgart, Germany, 
{\tt\small [hannes.eschmann, mario.rosenfelder, henrik.ebel]@itm.uni-stuttgart.de}.\newline The ITM acknowledges the support by the Deutsche Forschungsgemeinschaft (DFG, German Research Foundation) under Germany’s Excellence Strategy – EXC 2075 – 390740016, project PN4-4 “Learning from Data - Predictive Control in Adaptive Multi-Agent Scenarios” and project EB195/32-1, 433183605 “Research on Multibody Dynamics and Control for Collaborative Elastic Object Transportation by a Heterogeneous Swarm with Aerial and Land-Based Mobile Robots”.
} , 
Mario Rosenfelder\thankagain {} , 
Henrik Ebel\thankagain {} , 
Karl Worthmann\thankagain[1] 
}

\date{February 2023}


\begin{document}

\maketitle

\begin{abstract}
    Data-driven surrogate models of dynamical systems based on the extended dynamic mode decomposition are nowadays well-established and widespread in applications. Further, for non-holonomic systems exhibiting a multiplicative coupling between states and controls, the usage of bi-linear surrogate models has proven beneficial. However, an in-depth analysis of the approximation quality and its dependence on different hyperparameters based on both simulation and experimental data is still missing. We investigate a differential-drive mobile robot to close this gap and provide first guidelines on the systematic design of data-efficient surrogate models.
\end{abstract}


\section{Introduction}

Non-holonomic vehicles are of indispensible practical value in transportation and robotics. 
To automate their behavior, accurate models are key for tasks such as motion planning and model-based  
control. 
Often, in robotics, simple %nominal 
kinematic models based on first principles are employed because it can be arduous to  
take into account hardware imperfections and effects beyond kinematics, and because it fits typical cascade-type control approaches. 
An alternative are data-driven techniques, which 
need to strike a balance between data efficiency, model expressiveness, efficient and reliable numerical realizations, and, at best, should have a theoretical underpinning that may bring about beneficial theoretical properties such as quantifiable error bounds with finite data. 
With regard to these requirements, a very popular method  
is the extended Dynamic Mode Decomposition (eDMD), whose theoretical foundation is the Koopman framework. 
The Koopman operator lifts the nonlinear dynamics to linear but infinite-dimensional dynamics, which are then approximated using eDMD to generate a data-based surrogate model~\cite{BrunKutz22}. 
This approach has been recently generalized to the setting with inputs~\cite{ProcBrun18} to apply linear techniques for the controller design~\cite{BevaSosn21}. 
In this paper, we 
show, based on real-world data and hardware experiments with a non-holonomic (differential-drive) mobile robot, that and how eDMD in a Koopman framework can be used to learn a model more accurate than the nominal kinematic  
model.  
Moreover, we show how it is possible to improve data efficiency and model accuracy by incorporating physical a-priori knowledge. 

Even with  
an accurate model, controller design for non-holonomic systems remains challenging~\cite{Asto96} since, e.g., Brockett's condition is violated meaning that there does not exist a continuous time-invariant state-feedback law. 
For instance, as rigorously shown in~\cite{MullWort17, RoseEbel22}, techniques like model predictive control based on quadratic costs do not  
successfully solve the set-point stabilization problem. 
A remedy are more sophisticated schemes using structural insight, e.g., based on the homogeneous approximation and privileged coordinates, see~\cite{WortMehr15,WortMehr16,CoroGrun20,RoseEbel22}. 
This insight is key to understand whether a linear surrogate model as proposed in eDMDc suffices or a bilinear one is required~\cite{BrudFu21, FolkBurd21, OttoRowl21}. 

Extended DMD with control (eDMDc) has already been explored for robotic systems, e.g., for an inverted pendulum or a tail-actuated robotic fish~\cite{MamaCast21}, 
or within simulations for non-holonomic mobile robots~\cite{ShiKary21}. 
Even a first experimental validation of Koopman-based LQR control utilizing structural knowledge  
has been explored for a tail-actuated robotic fish~\cite{MamaCast19}. 
However, determining an optimal dimensionality of the Koopman-based surrogate model remains challenging~\cite{RenJian22}. 
A rare experimental work, in which eDMDc is applied to non-holonomic robots, can be found in~\cite{ShiKary21ACD}. 
Therein, eDMDc is used to identify a model based on simulated data using a dictionary consisting of Hermite polynomials, and the prediction of that model is also compared with the behavior of a hardware robot. 
However, the authors do not identify a model based on data from real-world hardware and, hence, only the nominal dynamics is replicated. 
Moreover, a bi-linear surrogate model seems to be advantageous as shown in~\cite{BrudFu21,FolkBurd21} on a simulated robot arm and a planar quadrotor, respectively --~a claim, which is further supported in~\cite{OttoRowl21,NuskPeit23} for control-affine systems exhibiting a state-control coupling 
since lifted linear models of finite dimension cannot capture nonlinear actuation effects inherent in many robotic systems~\cite{FolkBurd21}. 

The contribution of this manuscript is the experimental investigation of the Koopman-based, bi-linear surrogate model in simulation \textit{and} experiment, which, to the knowledge of the authors, is novel in itself and in the depth of the conducted analysis. 
In that regard, we consider the so-called one-step error to analyze and compare the prediction accuracy for various reference trajectories in dependence of the key hyperparameters like the composition of the dictionary, the amount of data points, and the control basis employed for the bilinear approach. 
In particular, we outperform nominal models using surrogate models generated from random real-world data. 

Section~\ref{sec:eDMD} recaps eDMD in the Koopman framework before the problem setup is given in Section~\ref{sec:robot}. Then, simulation and experimental results are presented in Sections~\ref{sec:simulation} and~\ref{sec:exp}, respectively, before the results are discussed and conclusions are drawn.

\bigskip\noindent\textbf{Notation}: For integers~$n,m \in \mathbb{Z}$ with~$n \leq m$, we define~$[n:m] \coloneqq \mathbb{Z} \cap [n,m]$.




\section{Recap: eDMD in the Koopman framework}\label{sec:eDMD}

We consider the 
nonlinear dynamical system governed by~$\dot{x}(t) = f(x(t))$ with a locally-Lipschitz continuous vector field~$f: \mathbb{R}^{n_x} \rightarrow \mathbb{R}^{n_x}$.
Then, for observables~$\varphi \in L^2(\mathbb{R}^{n_x},\mathbb{R})$, the Koopman operator is defined by the identify
\begin{equation}\label{eq:Koopman_operator}
    (\mathcal{K}^t \varphi)(x^0) = \varphi(x(t;x^0)) \qquad\forall\,(t,x^0) \in \mathbb{R}_{\geq 0} \times \mathbb{R}^{n_x},
\end{equation}
i.e., instead of evaluating the observable~$\varphi$ at the flow~$x(t;x^0)$ emanating from the initial condition~$x(0;x^0) = x^0$ at time~$t$, the Koopman operator propagates the observable forward in time~$\mathcal{K}^t \varphi$ and, then, evaluates the propagated observable at the initial value~$x^0 \in \mathbb{R}^{n_x}$. 
Alternatively, one may also work with the  
generator~$\mathcal{L}$ of the Koopman semigroup~$(\mathcal{K}^t)_{t \in \mathbb{R}_{\geq 0}}$, which satisfies the abstract Cauchy problem~$\dot{z}(t) = \mathcal{L}z(t)$,~$z(0) = \varphi$, see, e.g.,~\cite{SchaWort22}.
For details on DMD~\cite{Tu13} and its variants, we refer to~\cite{Schmi22} and the references therein. The connection to the Koopman framework is treated in~\cite{BrunKutz22}. Here, we restrict ourselves to a compact set~$\mathbb{X} \subset \mathbb{R}^{n_x}$, see~\cite{SchaWort22} for a detailed discussion. 

For the dictionary~$\mathbb{V} \coloneqq \operatorname{span} \{ ( \psi_j )_{j=1}^N \}$ with~$\psi_j: \mathbb{X} \to \mathbb{R}$, the data-based surrogate model of the Koopman generator using the i.i.d.\ data points~$x^{[1]}, .., x^{[d]} \in \mathbb{X}$ is given by 
\begin{align*}
    \tilde{\mathcal{L}}_{d} = \tilde{C}^{-1} \tilde{A} %
    \quad\text{ with }\quad \tilde{C} = \tfrac{1}{d} \Psi_X \Psi_X^\top\text{ and }\tilde{A} = \tfrac{1}{d} \Psi_X \Psi_Y^\top,
\end{align*}
where the matrices~$\Psi_X, \Psi_Y \in \mathbb{R}^{N \times d}$ are defined by
\begin{align*}
    \Psi_X & \coloneqq \left[ \left. \left[\begin{smallmatrix}
            \psi_1(x^{[1]}) \\ 
            : \\ 
            \psi_N(x^{[1]})
        \end{smallmatrix}\right]\right| \ldots \left| \left[\begin{smallmatrix}
            \psi_1(x^{[d]})\\
            : \\
            \psi_N(x^{[d]})
        \end{smallmatrix}\right]\right. \right], \\
        \Psi_Y &\coloneqq \left[ \left. \left[\begin{smallmatrix}
            (\mathcal{L}\psi_1)(x^{[1]})\\
            : \\
            (\mathcal{L}\psi_N)(x^{[1]})
        \end{smallmatrix}\right]\right| \ldots \left| \left[\begin{smallmatrix}
            (\mathcal{L}\psi_1)(x^{[d]})\\
            : \\
            (\mathcal{L}\psi_N)(x^{[d]})
        \end{smallmatrix}\right]\right. \right].
    \end{align*}
Note that~$(\mathcal{L}\psi_j)(x^{[i]}) = f(x^{[i]}) \cdot \nabla \psi_j(x^{[i]})$ holds for all~$(i,j) \in [1:d] \times [1:N]$. 
Since one cannot expect invariance of~$\mathbb{V}$ w.r.t.\ the approximated Koopman operator, one projects the outcome to the coordinate functions, which are tacitly assumed to be contained in the dictionary, e.g.,~$\psi_i(x) = x_i$ for all~$i \in [1:n_x]$. In the operator setting, a time shift~$\delta > 0$ is fixed and the data matrix~$\Psi_Y$ contains the entries~$\psi_j(x(\delta;x_i))$ instead of~$(\mathcal{L}\psi_j)(x_i)$. 




For control-affine systems $\dot{x}(t) = f(x(t)) + \sum_{i=1}^{n_u} g_i(x(t)) u_i(t)$, there are two different options to deduce eDMD-based surrogate models. In~\cite{KordMezi18}, a linear surrogate model~$\dot{\psi} = \mathcal{L}\psi + \mathcal{B}u(t)$ (eDMDc) is proposed. To this end, the state is augmented by the control, i.e.,~$\tilde x = [x^\top\ u^\top]^\top$. 
An alternative are bi-linear surrogate models that explicitly leverage the control-affine structure, i.e., the identity~$\mathcal{L}^{u(t)} = \mathcal{L}^{0} + \sum_{i=1}^{n_u} u_i(t) (\mathcal{L}^{e_i} - \mathcal{L}^0)$, where~$\mathcal{L}^{e_i}$ is the generator for the autonomous dynamics with~$u\equiv e_i$. This yields~$\dot{\psi} = \mathcal{L}^{u(t)}\psi$, 
see, e.g.,~\cite{PeitOtto20} and the references therein. This approach seems to be preferable. 
On the one hand, it alleviates the curse of dimensionality resulting from the state augmentation in eDMDc. 
On the other hand, bilinear models seem to be superior if state-control couplings are present, i.e., one of the vector fields~$g_i$ depends on the state~$x$, see~\cite{BrudFu21,FolkBurd21,OttoRowl21,NuskPeit23}. 
For further details on the Koopman theory for control systems, see, e.g.,~\cite{BevaSosn21} and the references therein.



The approximation error can be split up into its two sources of error, i.e., the estimation~\cite{NuskPeit23} and the projection error~\cite{SchaWort22}. 
While the latter results from only finitely many observables in the dictionary~$\mathbb{V}$ and, thus, approximating the Koopman generator/operator on the respective finite-dimensional subspace, the former is a consequence of using only finitely many data points~$x^{[i]}$,~$i \in [1:d]$. 
While the convergence in the infinite-data limit also holds for eDMDc~\cite{KordMezi18convergence}, finite-data error bounds are presently only available for the bilinear approach, see~\cite{NuskPeit23,SchaWort22}.




\section{Problem Setup}\label{sec:robot}

The nominal kinematics of the 
differential-drive robot is given in terms of the driftless control-affine system
\begin{align}\label{eq:nominal_kinematics}
    \dot{x} (t) = \begin{bmatrix} \cos \theta(t) \\ \sin \theta(t) \\ 0 \end{bmatrix} v(t) + \begin{bmatrix} 0 \\ 0 \\ 1 \end{bmatrix} \omega (t), % \eqqcolon \bm{G}(\bm{x}(t)) \, \bm{u}(t),
\end{align}
$x(0)=x^0$. The state~$x = [ x_1\ x_2\ \theta ]^\top \in \mathbb{X} \subset \mathbb{R}^3$ consists of its position~$[ x_1\ x_2]^\top$ in the plane and its orientation~$\theta$ measured relative to the~$x_1$-axis. 
Nominally, it is assumed that the robot can instantaneously attain any admissible translational velocity~$v$ in forward direction and angular yaw velocity~$\omega$, so that these act as the system's control input~$u= [ v\ \omega ]^\top\in\mathbb{U}\subset\mathbb{R}^2$, where~$\mathbb{U}$ is compact, convex, and~$0\in\text{int}(\mathbb{U})$.
In general, the nominal kinematics does not perfectly describe the dynamics of the physical robot since inertia effects, motor dynamics, and manufacturing imperfections are not accounted for. 
From a mechanical point of view, the dynamics~\eqref{eq:nominal_kinematics} 
describe the kinematics of a differential-drive mobile robot in the plane under the common assumption that the wheels roll without slipping with the wheel-floor contact point sticking perfectly to the ground, preventing instantaneous lateral motions of the robot and thereby giving rise to a non-holonomic kinematic constraint. 
A physical robot with such a kinematic setup is employed throughout this contribution.  
On the nominal kinematic level, the robot's configuration is completely described by means of its pose, hence it is sufficient to formulate the observables based on~$x$.
Thus, in general, the learning procedure from Section~\ref{sec:eDMD} receives as data recorded pairs of states and corresponding successor states, but not any prior information on the dynamics of the robot.
However, in Sec.~\ref{sec:simulation}, we show how some mechanical prior knowledge can be incorporated, e.g., when choosing the observables of the dictionary~$\mathbb{V}$.

\section{Simulation results}\label{sec:simulation}
In this section, eDMD is applied to the simulated, nominal non-holonomic robot. 
First, we generate i.i.d.\ random data matrices~$X_i \in \mathbb{R}^{n_x \times d}$,~$i \in \lbrace 0,\dots,n_u\rbrace$, with~$n_x = 3$ and~$d=10000$ data points each, where each column is in the set~$\mathbb{X}$ and serves as an initial condition for the dynamical system~\eqref{eq:nominal_kinematics}. 
Each data point contained in~$X_i$ is simulated forward 
$\delta = 0.02\,\textnormal{s}$ with the Runge-Kutta method of fourth order  
using a specific constant control input~$u_i$. 
For~$i=0$, the latter is chosen to~$u_0 = 0$. 
For~$i>0$, it is selected to be the~$i$th vector of a basis~$B$ of~$\mathbb{R}^{n_u}$. 
Here, with~$n_u=2$ and the basis~$B=\lbrace u_1, u_2\rbrace$, this yields the matrices~$Y_0$,~$Y_1$, and~$Y_2$ containing in each column the successor states of the states in~$X_0$,~$X_1$, and $X_2$ for the inputs~$u_0$,~$u_1$, and~$u_2$, respectively. 
Nominally, the system is free of drift, i.e.,~$Y_0 = X_0$. 
In simulations, different from experiments, it is possible to choose~$X_0 = X_1 = X_2$,  
which is done in this section. 
In the following, the dictionary~$\mathbb{V}$ is spanned by the monomials of~$x_1$,~$x_2$, and~$\theta$ of degree less or equal than 
7, which yields 
$N=120$ 
observables in total, yielding the set of observables~$\mathbb{O}_{120}$. 
By lifting the matrices~$X_i, Y_i$,~$i \in [0:n_u]$, with those observables, the matrices~$\Psi_{X_i}, \Psi_{Y_i}$ are computed, see Section~\ref{sec:eDMD}.  
Now, an approximation of the Koopman operator for step size~$\delta$  
is computed by 
$K_i^{\delta} = ( %\frac{1}{d}
     (\Psi_{X_i} \Psi_{X_i}^\top )^{-1} %\frac{1}{d} 
     \Psi_{X_i} \Psi_{Y_i}^\top )^\top$,~$i \in [0:n_u]$. %\lbrace 0, 1, 2 \rbrace,
Using the bilinear  
approach, we approximate the Koopman operator  
for a control value~$u \in \mathbb{U}\subset\mathbb{R}^{n_u}$ by %\begin{align*} 
$K_u^{\delta} = K_0^{\delta} + \sum_{i = 1}^{n_u} g_i \cdot \left(K_i^{\delta} - K_0^{\delta}\right)$
for factors~$g_i$,~$i \in[ 1 : n_u ]$, which, here, solve the linear system~$g_1 u_1 + g_2 u_2 = u$.
There are two different ways to use the approximated Koopman operator to obtain the approximate values of the coordinates at a time step~$k>0$. 
In the first surrogate model variant proposed in~\cite{PeitOtto20}, subsequently referred to as SUR$_1$, one projects after %in 
each  
time step, i.e.,~$x_j[k] = ( \mathcal{K}^{\delta}_{u[k-1]} \Psi ( x[k-1] ) )_{j}$ for~$j \in [1:n_x]$, 
with the number inside square brackets denoting the time step, where one step is of duration~$\delta$. 
Between time steps, the new values of the observables are calculated based on the new coordinate values. 
In the second variant, called SUR$_2$ in the following, one projects once at the end, i.e.,
$x_j[k] =  (( \prod_{i=0}^{k-1} \mathcal{K}^{\delta}_{u[i]} ) \Psi(x^0) )_{\! j}$. 
To analyze their influence, Fig.~\ref{fig: modelbased circle} shows prediction results for the two variants and, as a reference, the result of the time integration of the nominal kinematic model using the Runge-Kutta method of fourth order. 
\def\lineWidthSUR{1.5}%
\def\lineWidthSURError{1.25}%
\definecolor{ODE}{RGB}{0,0,255}%
\definecolor{SUR1}{RGB}{255,127,14}%
\definecolor{SUR2}{RGB}{0,128,0}%
\begin{figure}
    \centering%
    % This file was created with tikzplotlib v0.10.1.
\begin{tikzpicture}%
\begin{axis}[%
width = .5\textwidth, % scaled down due to axis equal image
height = 6.0cm,
at={(0.0, 0)},
axis equal image=true,
%legend cell align={left},
legend style={
  fill opacity=1,
  draw opacity=1,
  text opacity=1,
  at={(1,0)},
  anchor=south east,
  %draw=lightgray204
},%
%tick align=outside,
%tick pos=left,
%x grid style={darkgray176},
xlabel = {$x_1$ position (m)}, 
ylabel = {$x_2$ position (m)},
xmin=-0, xmax=2.5,
%xtick style={color=black},
%y grid style={darkgray176},
%ylabel={\(\displaystyle y\)},
%ymin=-2.52637688506353, ymax=1.16792262955725,
ymin=-1.7, ymax=1.2,
%ytick style={color=black}
]%
\addplot [line width = \lineWidthSUR, color = ODE]
table {%
0.2 0
0.200007999989333 -0.00399998933334222
0.200031999829334 -0.00799991466694045
0.200071999136004 -0.0119997120020747
0.200127997269357 -0.0159993173420728
0.200199993333422 -0.0199986666933349
0.200287986176265 -0.0239976960663564
0.200391974390003 -0.0279963414767529
0.200511956310825 -0.031994538946283
0.200647930019023 -0.0359922245038725
0.200799893339022 -0.0399893341866377
0.200967843839411 -0.0439858040409091
0.201151778832986 -0.0479815701232542
0.201351695376791 -0.0519765685015009
0.201567590272166 -0.0559707352557605
0.201799460064796 -0.0599640064794499
0.20204730104477 -0.063956318280315
0.202311109246638 -0.067947606781452
0.202590880449474 -0.07193780812233
0.202886610176945 -0.0759268584598127
0.203198293697381 -0.0799146939691798
0.203525926023853 -0.0839012508451483
0.20386950191425 -0.0878864653028931
0.204229015871366 -0.091870273579068
0.204604462142985 -0.0958526119328256
0.204995834721975 -0.099833416646837
0.205403127346382 -0.103812624028312
0.205826333499534 -0.107790170410017
0.20626544641014 -0.111765992151295
0.206720459052405 -0.115740025639083
0.207191364146134 -0.11971220728893
0.207678154156858 -0.123682473546014
0.208180821295945 -0.12765076088616
0.208699357520733 -0.131617005816855
0.209233754534653 -0.135581144878265
0.209784003787364 -0.139543114644249
0.210350096474889 -0.143502851723376
0.210932023539759 -0.147460292759936
0.211529775671152 -0.151415374434958
0.212143343305047 -0.15536803346722
0.212772716624374 -0.15931820661426
0.213417885559175 -0.163265830673394
0.21407883978676 -0.167210842482719
0.214755568731875 -0.171153178922132
0.215448061566872 -0.175092776914334
0.21615630721188 -0.17902957342584
0.216880294334984 -0.182963505467991
0.217620011352406 -0.186894510097957
0.218375446428688 -0.190822524419749
0.219146587476886 -0.194747485585221
0.21993342215876 -0.198669330795079
0.220735937884971 -0.202587997299882
0.221554121815285 -0.20650342240105
0.222387960858777 -0.210415543451865
0.223237441674039 -0.214324297858474
0.224102550669397 -0.218229623080889
0.224983274003125 -0.22213145663399
0.225879597583668 -0.226029736088524
0.226791507069869 -0.229924399072103
0.227718987871195 -0.233815383270202
0.228662025147973 -0.237702626427156
0.229620603811627 -0.241586066347158
0.230594708524918 -0.245465640895253
0.231584323702193 -0.24934128799833
0.232589433509628 -0.253212945646118
0.23361002186549 -0.257080551892178
0.234646072440385 -0.260944044854892
0.235697568657527 -0.264803362718455
0.236764493692999 -0.268658443733864
0.237846830476023 -0.272509226219904
0.238944561689233 -0.276355648564138
0.240057669768953 -0.280197649223891
0.241186136905479 -0.284035166727234
0.242329945043359 -0.287868139673969
0.243489075881688 -0.29169650673661
0.244663510874398 -0.295520206661366
0.245853231230553 -0.29933917826912
0.247058217914654 -0.303153360456407
0.248278451646938 -0.306962692196395
0.249513912903693 -0.310767112539856
0.250764581917564 -0.314566560616146
0.252030438677874 -0.318360975634177
0.253311462930942 -0.322150296883389
0.254607634180407 -0.325934463734724
0.255918931687557 -0.329713415641592
0.257245334471659 -0.333487092140844
0.258586821310298 -0.337255432853736
0.259943370739712 -0.341018377486897
0.26131496105514 -0.344775865833294
0.262701570311166 -0.348527837773192
0.264103176322071 -0.352274233275121
0.265519756662189 -0.356014992396833
0.266951288666265 -0.359750055286262
0.268397749429818 -0.363479362182481
0.269859115809505 -0.367202853416659
0.271335364423496 -0.370920469413016
0.272826471651845 -0.374632150689775
0.274332413636868 -0.378337837860116
0.275853166283526 -0.382037471633121
0.277388705259807 -0.385730992814731
0.278939005997122 -0.389418342308685
0.280504043690692 -0.39309946111747
0.282083793299947 -0.396774290343262
0.283678229548927 -0.400442771188874
0.285287326926685 -0.40410484495869
0.2869110596877 -0.407760453059607
0.28854940185228 -0.411409537001974
0.290202327206986 -0.415052038400525
0.291869809305047 -0.418687898975317
0.293551821466787 -0.422317060552657
0.295248336780045 -0.425939465066038
0.296959328100614 -0.429555054557064
0.298684768052668 -0.433163771176381
0.300424629029205 -0.4367655571846
0.302178883192487 -0.440360354953222
0.303947502474484 -0.443948106965559
0.305730458577326 -0.447528755817656
0.307527722973753 -0.451102244219207
0.309339266907573 -0.454668514994473
0.311165061394122 -0.4582275110832
0.313005077220726 -0.461779175541524
0.314859284947173 -0.465323451542891
0.316727654906177 -0.468860282378961
0.318610157203859 -0.472389611460514
0.320506761720224 -0.475911382318362
0.322417438109638 -0.479425538604246
0.324342155801321 -0.482932024091739
0.326280883999831 -0.48643078267715
0.328233591685557 -0.489921758380415
0.330200247615218 -0.493404895345998
0.332180820322362 -0.496880137843781
0.334175278117867 -0.500347430269959
0.336183589090452 -0.503806717147926
0.338205721107183 -0.507257943129167
0.340241641813991 -0.51070105299414
0.342291318636189 -0.514135991653159
0.344354718778991 -0.51756270414728
0.346431809228038 -0.520981135649176
0.348522556749929 -0.524391231464016
0.350626927892746 -0.52779293703034
0.352744888986598 -0.531186197920931
0.354876406144151 -0.534570959843687
0.357021445261176 -0.537947168642489
0.359179972017093 -0.54131477029807
0.361351951875521 -0.544673710928874
0.363537350084828 -0.548023936791923
0.365736131678689 -0.551365394283674
0.367948261476645 -0.554698029940879
0.370173704084667 -0.558021790441438
0.372412423895721 -0.561336622605255
0.374664385090337 -0.564642473395086
0.376929551637185 -0.567939289917388
0.379207887293648 -0.571227019423167
0.381499355606404 -0.574505609308821
0.383803919912009 -0.577775007116983
0.386121543337483 -0.581035160537357
0.388452188800901 -0.584286017407558
0.390795819011985 -0.587527525713944
0.393152396472702 -0.590759633592454
0.395521883477862 -0.593982289329428
0.397904242115725 -0.597195441362445
0.400299434268603 -0.600399038281141
0.402707421613472 -0.603593028828032
0.405128165622586 -0.606777361899339
0.407561627564093 -0.6099519865458
0.410007768502654 -0.613116851973489
0.412466549300065 -0.616271907544625
0.414937930615887 -0.619417102778389
0.417421872908072 -0.622552387351722
0.419918336433594 -0.625677711100138
0.422427281249092 -0.628793024018525
0.424948667211502 -0.631898276261941
0.427482453978702 -0.634993418146418
0.430028601010158 -0.638078400149751
0.432587067567572 -0.641153172912294
0.435157812715533 -0.644217687237749
0.437740795322174 -0.64727189409395
0.440335974059828 -0.650315744613655
0.442943307405692 -0.65334919009532
0.445562753642486 -0.656372182003881
0.448194270859128 -0.659384671971532
0.450837816951397 -0.662386611798499
0.453493349622612 -0.665377953453808
0.456160826384309 -0.668358649076056
0.458840204556914 -0.671328650974178
0.461531441270436 -0.674287911628205
0.464234493465143 -0.677236383690032
0.466949317892258 -0.680174019984167
0.469675871114649 -0.683100773508492
0.472414109507521 -0.686016597435015
0.47516398925912 -0.688921445110613
0.477925466371427 -0.691815270057787
0.480698496660868 -0.694698025975398
0.483483035759018 -0.697569666739414
0.486279039113309 -0.700430146403643
0.489086461987749 -0.703279419200473
0.491905259463631 -0.706117439541599
0.494735386440255 -0.708944162018756
0.497576797635651 -0.711759541404444
0.500429447587299 -0.714563532652655
0.503293290652862 -0.717356090899587
0.506168281010911 -0.720137171464368
0.509054372661663 -0.722906729849769
0.511951519427712 -0.725664721742914
0.514859674954771 -0.728411103015992
0.517778792712415 -0.731145829726961
0.520708825994822 -0.733868858120252
0.523649727921521 -0.73658014462747
0.526601451438144 -0.739279645868087
0.529563949317179 -0.741967318650141
0.532537174158722 -0.744643119970926
0.535521078391239 -0.747307007017677
0.538515614272327 -0.749958937168258
0.541520733889474 -0.752598867991843
0.544536389160832 -0.755226757249596
0.54756253183598 -0.757842562895345
0.5505991134967 -0.760446243076254
0.55364608555775 -0.763037756133497
0.556703399267642 -0.76561706060292
0.559771005709422 -0.768184115215707
0.562848855801453 -0.770738878899038
0.565936900298198 -0.773281310776749
0.569035089791012 -0.775811370169985
0.572143374708929 -0.778329016597848
0.575261705319456 -0.78083420977805
0.57839003172937 -0.783326909627553
0.581528303885515 -0.785807076263214
0.584676471575605 -0.788274670002418
0.587834484429022 -0.790729651363718
0.591002291917629 -0.793171981067466
0.594179843356573 -0.795601620036437
0.597367087905099 -0.798018529396461
0.600563974567363 -0.800422670477039
0.603770452193246 -0.802814004811964
0.606986469479174 -0.80519249413994
0.610211974968939 -0.807558100405186
0.613446917054523 -0.809910785758054
0.61669124397692 -0.812250512555628
0.619944903826971 -0.81457724336233
0.623207844546185 -0.816890940950515
0.626480013927582 -0.819191568301071
0.629761359616521 -0.821479088604011
0.633051829111541 -0.823753465259059
0.6363513697652 -0.826014661876235
0.639659928784919 -0.828262642276443
0.642977453233823 -0.830497370492045
0.646303890031592 -0.832718810767435
0.649639185955309 -0.834926927559618
0.652983287640309 -0.837121685538772
0.656336141581037 -0.839303049588816
0.659697694131902 -0.841470984807972
0.663067891508134 -0.843625456509322
0.666446679786648 -0.845766430221365
0.669834004906902 -0.847893871688567
0.673229812671768 -0.850007746871911
0.676634048748394 -0.852108021949439
0.680046658669074 -0.854194663316794
0.683467587832123 -0.856267637587758
0.686896781502748 -0.858326911594788
0.690334184813922 -0.860372452389544
0.693779742767266 -0.862404227243415
0.697233400233927 -0.864422203648049
0.700695101955458 -0.866426349315866
0.704164792544707 -0.868416632180577
0.707642416486696 -0.8703930203977
0.711127918139519 -0.872355482345064
0.714621241735222 -0.874303986623322
0.718122331380702 -0.876238502056445
0.7216311310586 -0.878158997692228
0.725147584628196 -0.880065442802782
0.728671635826308 -0.881957806885026
0.732203228268192 -0.883836059661175
0.735742305448443 -0.885700171079225
0.739288810741904 -0.887550111313432
0.742842687404563 -0.889385850764793
0.746403878574472 -0.891207360061515
0.749972327272647 -0.893014610059488
0.753547976403985 -0.894807571842752
0.757130768758176 -0.896586216723955
0.760720647010619 -0.898350516244818
0.764317553723339 -0.900100442176585
0.767921431345905 -0.90183596652048
0.771532222216351 -0.90355706150815
0.775149868562101 -0.905263699602112
0.778774312500891 -0.906955853496192
0.782405496041695 -0.908633496115965
0.786043361085653 -0.910296600619183
0.789687849427003 -0.911945140396212
0.793338902754009 -0.913579089070449
0.796996462649895 -0.915198420498751
0.800660470593781 -0.916803108771849
0.804330867961619 -0.918393128214765
0.808007596027128 -0.919968453387222
0.811690595962739 -0.921529059084051
0.81537980884053 -0.923074920335593
0.819075175633174 -0.924606012408103
0.822776637214882 -0.926122310804139
0.826484134362347 -0.927623791262959
0.830197607755694 -0.929110429760909
0.833916997979428 -0.930582202511803
0.837642245523384 -0.93203908596731
0.841373290783682 -0.933481056817324
0.845110074063676 -0.934908091990344
0.848852535574915 -0.936320168653836
0.852600615438093 -0.937717264214605
0.856354253684013 -0.939099356319151
0.860113390254542 -0.94046642285403
0.863877965003577 -0.941818441946207
0.867647917698001 -0.943155391963405
0.871423188018652 -0.944477251514452
0.875203715561285 -0.945783999449623
0.87898943983754 -0.94707561486098
0.882780300275909 -0.948352077082704
0.886576236222707 -0.949613365691426
0.890377186943038 -0.950859460506555
0.894183091621774 -0.9520903415906
0.897993889364519 -0.953305989249492
0.901809519198591 -0.954506384032893
0.905629920073995 -0.955691506734512
0.909455030864398 -0.956861338392412
0.913284790368109 -0.95801586028931
0.917119137309057 -0.959155053952881
0.920958010337774 -0.960278901156051
0.924801348032372 -0.961387383917289
0.928649088899531 -0.962480484500893
0.932501171375479 -0.963558185417278
0.936357533826978 -0.964620469423253
0.940218114552311 -0.965667319522295
0.94408285178227 -0.966698718964826
0.94795168368114 -0.967714651248476
0.951824548347695 -0.968715100118351
0.955701383816182 -0.96970004956729
0.959582128057315 -0.970669483836123
0.963466718979271 -0.971623387413922
0.967355094428678 -0.97256174503825
0.97124719219161 -0.973484541695406
0.975142949994588 -0.974391762620662
0.979042305505568 -0.975283393298503
0.982945196334946 -0.976159419462858
0.986851560036553 -0.977019827097326
0.990761334108652 -0.977864602435403
0.994674455994942 -0.978693731960702
0.998590863085555 -0.979507202407169
1.00251049271806 -0.980305000759294
1.00643328217847 -0.98108711425232
1.01035916870224 -0.981853530372447
1.01428808947526 -0.982604236857035
1.01821998163489 -0.983339221694795
1.02215478227093 -0.984058473125986
1.02609242842667 -0.9847619796426
1.03003285709983 -0.985449729988548
1.03397600524366 -0.986121713159839
1.03792180976787 -0.986777918404757
1.04187020753966 -0.987418335224031
1.04582113538475 -0.988042953371007
1.04977453008839 -0.988651762851808
1.05373032839634 -0.989244753925493
1.05768846701592 -0.98982191710422
1.06164888261699 -0.99038324315339
1.06561151183299 -0.990928723091797
1.06957629126193 -0.991458348191774
1.07354315746743 -0.991972109979332
1.07751204697971 -0.992470000234292
1.08148289629663 -0.992952010990421
1.08545564188468 -0.993418134535557
1.08943022018001 -0.993868363411733
1.09340656758946 -0.994302690415297
1.09738462049156 -0.994721108597027
1.10136431523754 -0.995123611262239
1.10534558815237 -0.995510191970901
1.10932837553577 -0.995880844537728
1.11331261366323 -0.996235563032289
1.11729823878702 -0.996574341779094
1.12128518713723 -0.996897175357691
1.12527339492276 -0.997204058602749
1.12926279833238 -0.997494986604143
1.13325333353572 -0.997769954707031
1.13724493668429 -0.99802895851193
1.14123754391255 -0.998271993874784
1.14523109133884 -0.998499056907032
1.14922551506651 -0.998710143975672
1.15322075118485 -0.998905251703313
1.15721673577017 -0.999084376968237
1.16121340488681 -0.999247516904443
1.16521069458814 -0.999394668901696
1.16920854091762 -0.999525830605568
1.17320687990979 -0.999640999917472
1.1772056475913 -0.999740174994704
1.18120477998197 -0.999823354250463
1.18520421309576 -0.999890536353884
1.18920388294182 -0.999941720230055
1.19320372552553 -0.999976905060034
1.19720367684948 -0.999996090280864
1.20120367291454 -0.999999275585584
1.20520364972086 -0.999986460923227
1.2092035432689 -0.999957646498829
1.21320328956044 -0.999912832773419
1.21720282459963 -0.999852020464016
1.22120208439399 -0.999775210543616
1.22520100495546 -0.999682404241175
1.22919952230138 -0.999573603041594
1.23319757245557 -0.999448808685688
1.23719509144931 -0.999308023170164
1.24119201532238 -0.999151248747588
1.24518828012408 -0.998978487926348
1.24918382191426 -0.998789743470612
1.25317857676435 -0.998585018400289
1.25717248075834 -0.998364315990974
1.26116546999385 -0.998127639773901
1.26515748058316 -0.997874993535886
1.26914844865416 -0.997606381319262
1.27313831035146 -0.997321807421819
1.27712700183735 -0.997021276396734
1.28111445929286 -0.996704793052496
1.28510061891875 -0.996372362452832
1.28908541693656 -0.996023989916625
1.29306878958959 -0.995659681017828
1.29705067314398 -0.995279441585375
1.30103100388966 -0.994883277703089
1.30500971814144 -0.994471195709585
1.30898675223997 -0.994043202198164
1.31296204255279 -0.993599304016714
1.31693552547534 -0.993139508267597
1.32090713743197 -0.992663822307535
1.32487681487698 -0.992172253747493
1.32884449429563 -0.991664810452556
1.33281011220511 -0.991141500541808
1.33677360515563 -0.990602332388196
1.34073490973138 -0.990047314618399
1.34469396255158 -0.989476456112688
1.34865070027147 -0.988889766004789
1.35260505958332 -0.98828725368173
1.35655697721748 -0.987668928783695
1.36050638994334 -0.98703480120387
1.36445323457038 -0.986384881088282
1.36839744794918 -0.98571917883564
1.37233896697241 -0.985037705097166
1.37627772857584 -0.984340470776424
1.38021366973937 -0.983627487029148
1.38414672748804 -0.982898765263064
1.38807683889299 -0.982154317137705
1.39200394107253 -0.981394154564224
1.39592797119311 -0.980618289705206
1.39984886647033 -0.979826734974473
1.40376656416995 -0.979019503036883
1.40768100160889 -0.978196606808131
1.41159211615624 -0.977358059454537
1.41549984523424 -0.976503874392843
1.41940412631931 -0.975634065289991
1.42330489694305 -0.974748646062907
1.4272020946932 -0.973847630878281
1.43109565721468 -0.972931034152336
1.43498552221057 -0.971998870550601
1.43887162744312 -0.971051154987673
1.44275391073473 -0.970087902626981
1.44663230996895 -0.969109128880542
1.45050676309146 -0.968114849408715
1.45437720811112 -0.967105080119951
1.45824358310086 -0.966079837170536
1.46210582619879 -0.965039136964337
1.46596387560909 -0.963982996152533
1.46981766960306 -0.962911431633356
1.47366714652007 -0.961824460551815
1.47751224476857 -0.960722100299424
1.48135290282707 -0.959604368513924
1.48518905924514 -0.958471283078999
1.48902065264433 -0.957322862123992
1.49284762171926 -0.956159124023614
1.49666990523848 -0.954980087397649
1.50048744204555 -0.953785771110659
1.50430017105995 -0.952576194271679
1.50810803127811 -0.951351376233912
1.51191096177434 -0.950111336594421
1.51570890170184 -0.948856095193814
1.51950179029364 -0.947585672115925
1.52328956686362 -0.946300087687498
1.52707217080743 -0.944999362477855
1.53084954160347 -0.943683517298573
1.53462161881392 -0.942352573203146
1.53838834208559 -0.941006551486651
1.54214965115102 -0.939645473685408
1.54590548582932 -0.938269361576631
1.54965578602723 -0.936878237178085
1.55340049174003 -0.935472122747731
1.55713954305249 -0.93405104078337
1.56087288013989 -0.932615014022283
1.5646004432689 -0.931164065440867
1.5683221727986 -0.929698218254269
1.57203800918139 -0.928217495916012
1.57574789296397 -0.926721922117623
1.57945176478828 -0.92521152078825
1.58314956539245 -0.923686316094282
1.58684123561174 -0.922146332438962
1.59052671637952 -0.920591594461995
1.59420594872816 -0.919022127039156
1.59787887379004 -0.917437955281891
1.60154543279843 -0.915839104536912
1.60520556708845 -0.9142256003858
1.60885921809805 -0.912597468644584
1.61250632736888 -0.910954735363339
1.61614683654727 -0.909297426825762
1.61978068738515 -0.907625569548754
1.623407821741 -0.905939190281995
1.62702818158072 -0.904238316007519
1.63064170897865 -0.902522973939277
1.63424834611843 -0.900793191522707
1.63784803529393 -0.899048996434288
1.64144071891021 -0.897290416581106
1.6450263394844 -0.895517480100402
1.64860483964665 -0.89373021535912
1.65217616214104 -0.891928650953458
1.65574024982648 -0.890112815708409
1.65929704567765 -0.888282738677298
1.66284649278589 -0.886438449141318
1.66638853436011 -0.884579976609062
1.66992311372773 -0.882707350816052
1.67345017433556 -0.880820601724259
1.6769696597507 -0.87891975952163
1.68048151366146 -0.877004854621599
1.68398567987825 -0.875075917662603
1.68748210233449 -0.873132979507593
1.69097072508749 -0.871176071243538
1.69445149231937 -0.869205224180928
1.69792434833791 -0.867220469853274
1.7013892375775 -0.865221840016603
1.70484610459999 -0.86320936664895
1.70829489409558 -0.861183081949845
1.7117355508837 -0.859143018339801
1.71516801991393 -0.857089208459793
1.71859224626682 -0.855021685170733
1.72200817515484 -0.852940481552951
1.7254157519232 -0.85084563090566
1.72881492205073 -0.848737166746427
1.73220563115079 -0.846615122810631
1.73558782497211 -0.844479533050932
1.73896144939965 -0.84233043163672
1.7423264504555 -0.840167852953571
1.74568277429971 -0.837991831602698
1.74903036723118 -0.835802402400397
1.75236917568848 -0.833599600377488
1.75569914625075 -0.831383460778756
1.75902022563854 -0.829154019062388
1.76233236071465 -0.826911310899404
1.76563549848498 -0.824655372173086
1.7689295860994 -0.822386238978406
1.77221457085258 -0.820103947621446
1.77549040018483 -0.81780853461882
1.77875702168296 -0.815500036697087
1.78201438308109 -0.813178490792164
1.7852624322615 -0.810843934048736
1.78850111725549 -0.808496403819661
1.79173038624415 -0.806135937665373
1.79495018755926 -0.80376257335328
1.79816046968406 -0.80137634885716
1.8013611812541 -0.798977302356555
1.80455227105807 -0.796565472236156
1.80773368803861 -0.794140897085196
1.8109053812931 -0.791703615696823
1.81406730007452 -0.789253667067489
1.81721939379224 -0.78679109039632
1.82036161201282 -0.784315925084489
1.82349390446085 -0.781828210734588
1.82661622101971 -0.779327987149995
1.8297285117324 -0.776815294334233
1.83283072680234 -0.774290172490334
1.83592281659415 -0.771752662020193
1.83900473163446 -0.769202803523924
1.84207642261269 -0.766640637799208
1.84513784038187 -0.764066205840643
1.84818893595935 -0.761479548839084
1.85122966052769 -0.758880708180988
1.85425996543536 -0.756269725447751
1.85727980219753 -0.753646642415041
1.8602891224969 -0.75101150105213
1.86328787818439 -0.748364343521223
1.86627602127997 -0.745705212176785
1.86925350397343 -0.743034149564861
1.87222027862511 -0.740351198422395
1.87517629776667 -0.737656401676549
1.87812151410186 -0.734949802444012
1.8810558805073 -0.732231444030315
1.88397935003318 -0.729501369929135
1.88689187590405 -0.726759623821598
1.88979341151956 -0.724006249575585
1.8926839104552 -0.721241291245024
1.89556332646305 -0.718464793069189
1.89843161347252 -0.715676799471992
1.90128872559107 -0.712877355061272
1.90413461710497 -0.710066504628079
1.90696924248002 -0.707244293145959
1.90979255636227 -0.704410765770238
1.91260451357877 -0.701565967837291
1.91540506913824 -0.698709944863826
1.91819417823188 -0.695842742546148
1.92097179623398 -0.692964406759435
1.92373787870272 -0.690074983556996
1.92649238138085 -0.687174519169543
1.92923526019637 -0.684263060004442
1.93196647126328 -0.681340652644979
1.93468597088227 -0.678407343849609
1.9373937155414 -0.67546318055121
1.94008966191681 -0.672508209856332
1.94277376687342 -0.669542479044443
1.94544598746561 -0.666566035567173
1.94810628093789 -0.663578927047554
1.95075460472565 -0.660581201279258
1.95339091645574 -0.657572906225835
1.95601517394724 -0.65455409001994
1.95862733521209 -0.651524800962568
1.96122735845576 -0.648485087522281
1.96381520207793 -0.645434998334427
1.96639082467317 -0.642374582200369
1.96895418503155 -0.6393038880867
1.97150524213939 -0.63622296512446
1.9740439551798 -0.633131862608351
1.97657028353345 -0.630030629995947
1.97908418677913 -0.626919316906904
1.98158562469444 -0.623797973122164
1.98407455725643 -0.620666648583162
1.98655094464223 -0.617525393391024
1.98901474722969 -0.614374257805765
1.99146592559803 -0.611213292245488
1.99390444052845 -0.608042547285574
1.99633025300475 -0.604862073657876
1.998743324214 -0.601671922249902
2.00114361554709 -0.598472144104008
2.00353108859943 -0.595262790416576
2.0059057051715 -0.592043912537196
2.00826742726947 -0.588815561967846
2.01061621710585 -0.585577790362066
2.01295203710005 -0.582330649524132
2.015274849879 -0.579074191408229
2.01758461827775 -0.575808468117617
2.01988130534004 -0.572533531903799
2.02216487431894 -0.569249435165685
2.02443528867739 -0.565956230448752
2.0266925120888 -0.562653970444205
2.02893650843766 -0.559342707988135
2.03116724182007 -0.556022496060669
2.03338467654435 -0.552693387785129
2.03558877713157 -0.549355436427174
2.03777950831619 -0.546008695393956
2.03995683504655 -0.542653218233261
2.04212072248546 -0.53928905863265
2.04427113601078 -0.535916270418605
2.04640804121594 -0.532534907555667
2.04853140391049 -0.529145024145569
2.05064119012069 -0.525746674426373
2.05273736608999 -0.522339912771602
2.05481989827962 -0.51892479368937
2.05688875336911 -0.515501371821509
2.05894389825684 -0.512069701942696
2.06098530006051 -0.508629838959577
2.06301292611775 -0.505181837909885
2.06502674398659 -0.501725753961563
2.06702672144597 -0.498261642411882
2.06901282649631 -0.494789558686551
2.07098502735996 -0.491309558338837
2.07294329248176 -0.487821697048671
2.0748875905295 -0.484326030621759
2.07681789039445 -0.48082261498869
2.07873416119186 -0.477311506204038
2.08063637226144 -0.473792760445469
2.08252449316786 -0.470266434012841
2.08439849370121 -0.466732583327301
2.08625834387752 -0.463191264930385
2.08810401393924 -0.459642535483111
2.08993547435569 -0.456086451765076
2.09175269582353 -0.452523070673541
2.09355564926726 -0.44895244922253
2.09534430583966 -0.445374644541909
2.09711863692227 -0.441789713876477
2.09887861412583 -0.438197714585048
2.10062420929074 -0.434598704139533
2.10235539448752 -0.430992740124023
2.10407214201723 -0.427379880233866
2.10577442441196 -0.423760182274743
2.10746221443522 -0.420133704161744
2.10913548508241 -0.416500503918441
2.11079420958123 -0.41286063967596
2.11243836139213 -0.409214169672052
2.11406791420871 -0.40556115225016
2.11568284195817 -0.401901645858483
2.11728311880168 -0.398235709049047
2.11886871913487 -0.394563400476761
2.12043961758815 -0.390884778898486
2.12199578902719 -0.387199903172086
2.12353720855327 -0.383508832255497
2.12506385150372 -0.379811625205772
2.12657569345228 -0.376108341178147
2.12807271020951 -0.372399039425087
2.12955487782317 -0.368683779295341
2.13102217257862 -0.364962620232991
2.13247457099917 -0.361235621776503
2.13391204984648 -0.357502843557773
2.13533458612091 -0.353764345301173
2.13674215706192 -0.350020186822595
2.1381347401484 -0.346270428028495
2.13951231309905 -0.342515128914934
2.14087485387274 -0.338754349566617
2.14222234066883 -0.334988150155933
2.14355475192758 -0.331216590941994
2.14487206633042 -0.327439732269665
2.14617426280036 -0.323657634568606
2.14746132050228 -0.319870358352298
2.14873321884328 -0.31607796421708
2.14998993747302 -0.312280512841178
2.15123145628403 -0.308478064983733
2.15245775541203 -0.304670681483828
2.15366881523627 -0.30085842325952
2.1548646163798 -0.297041351306857
2.15604513970985 -0.293219526698911
2.15721036633805 -0.289393010584792
2.15836027762081 -0.285561864188679
2.15949485515958 -0.28172614880883
2.16061408080113 -0.27788592581661
2.16171793663788 -0.274041256655505
2.16280640500816 -0.27019220284014
2.16387946849651 -0.266338825955294
2.16493710993392 -0.262481187654915
2.16597931239815 -0.258619349661132
2.167006059214 -0.254753373763272
2.16801733395352 -0.250883321816867
2.16901312043636 -0.247009255742665
2.16999340272994 -0.24313123752564
2.17095816514977 -0.239249329214002
2.17190739225967 -0.2353635929182
2.17284106887203 -0.231474090809933
2.17375918004804 -0.227580885121151
2.17466171109794 -0.223684038143062
2.17554864758126 -0.219783612225135
2.17641997530703 -0.2158796697741
2.17727568033402 -0.211972273252955
2.17811574897098 -0.20806148517996
2.17894016777682 -0.204147368127641
2.17974892356086 -0.200229984721787
2.18054200338303 -0.196309397640448
2.18131939455406 -0.192385669612936
2.18208108463571 -0.188458863418814
2.18282706144095 -0.184529041886897
2.18355731303419 -0.180596267894247
2.18427182773139 -0.176660604365164
2.18497059410035 -0.172722114270179
2.18565360096082 -0.168780860625051
2.18632083738469 -0.164836906489753
2.18697229269622 -0.160890314967468
2.18760795647211 -0.156941149203576
2.18822781854177 -0.152989472384645
2.18883186898742 -0.149035347737419
2.18942009814425 -0.145078838527809
2.18999249660063 -0.141120008059878
2.19054905519817 -0.137158919674828
2.19108976503197 -0.13319563674999
2.19161461745067 -0.129230222697805
2.19212360405664 -0.125262740964813
2.19261671670612 -0.121293255030639
2.1930939475093 -0.117321828406971
2.19355528883051 -0.113348524636552
2.19400073328829 -0.109373407292157
2.19443027375555 -0.105396539975579
2.19484390335964 -0.101417986316609
2.1952416154825 -0.0974378099720221
2.19562340376074 -0.0934560746245541
2.19598926208577 -0.0894728439818858
2.19633918460385 -0.0854881817756226
2.19667316571623 -0.0815021517602747
2.19699120007922 -0.0775148177122373
2.19729328260428 -0.0735262434287702
2.19757940845809 -0.0695364927269769
2.19784957306265 -0.0655456294427833
2.19810377209533 -0.0615537174299169
2.19834200148894 -0.0575608205588849
2.19856425743183 -0.0535670027159518
2.19877053636791 -0.049572327802118
2.19896083499671 -0.045576859732097
2.19913515027346 -0.0415806624332925
2.19929347940912 -0.0375837998447762
2.19943581987043 -0.0335863359162641
2.19956216937994 -0.0295883346070939
2.19967252591607 -0.0255898598852012
2.19976688771311 -0.0215909757260962
2.19984525326127 -0.0175917461118403
2.19990762130671 -0.0135922350300218
2.19995399085154 -0.0095925064727328
2.19998436115385 -0.00559262443554489
2.19999873172772 -0.00159265291648533
2.19999710234321 0.00240734408498693
2.1999794730264 0.00640730256900518
2.19994584405935 0.010407158536319
2.19989621598012 0.0144068479893183
2.19983058958278 0.0184063069330571
2.19974896591733 0.0224054713762776
2.19965134628975 0.0264042773324341
2.19953773226196 0.0304026608207166
2.19940812565177 0.0344005578670745
2.1992625285329 0.0383979045052404
2.19910094323489 0.0423946367777534
2.1989233723431 0.0463906907369824
2.19872981869866 0.0503860024461492
2.19852028539843 0.0543805079803518
2.19829477579493 0.0583741434275869
2.19805329349632 0.0623668448897724
2.19779584236631 0.0663585484837702
2.1975224265241 0.0703491903424078
2.19723305034435 0.0743387066155008
2.19692771845707 0.0783270334708738
2.19660643574756 0.0823141070953823
2.19626920735634 0.0862998636959333
2.19591603867905 0.0902842395005062
2.19554693536639 0.0942671707591731
2.19516190332401 0.098248593745119
2.1947609487124 0.102228444755661
2.19434407794683 0.106206660113268
2.19391129769723 0.110183176166578
2.19346261488807 0.114157929291421
2.19299803669827 0.11813085589183
2.19251757056106 0.122101892401065
2.1920212241639 0.126070975282626
2.19150900544831 0.130038041031273
2.19098092260979 0.134003026174037
2.19043698409765 0.137965867271241
2.18987719861489 0.141926500917512
2.18930157511807 0.145884863742796
2.18871012281716 0.149840892413373
2.18810285117537 0.153794523632867
2.18747976990904 0.157745694143264
2.18684088898746 0.16169434072592
2.18618621863271 0.165640400202574
2.1855157693195 0.169583809436359
2.18482955177501 0.17352450533281
2.18412757697869 0.177462424840878
2.18340985616214 0.181397504953934
2.18267640080887 0.185329682710781
2.18192722265415 0.189258895196658
2.18116233368482 0.193185079544249
2.18038174613908 0.197108172934689
2.17958547250631 0.201028112598567
2.17877352552688 0.204944835816932
2.17794591819193 0.208858279922296
2.17710266374314 0.212768382299637
2.17624377567259 0.216675080387401
2.17536926772244 0.2205783116785
2.17447915388482 0.224478013721318
2.17357344840153 0.228374124120706
2.17265216576384 0.23226658053898
2.17171532071224 0.23615532069692
2.17076292823624 0.240040282374767
2.16979500357409 0.243921403413216
2.16881156221258 0.247798621714415
2.16781261988674 0.251671875242953
2.16679819257963 0.255541102026856
2.16576829652207 0.259406240158578
2.16472294819236 0.263267227795991
2.16366216431606 0.267124003163376
2.1625859618657 0.270976504552409
2.16149435806047 0.27482467032315
2.16038737036603 0.278668438905029
2.15926501649415 0.28250774879783
2.15812731440248 0.286342538572677
2.15697428229421 0.290172746873016
2.15580593861784 0.293998312415595
2.15462230206684 0.297819173991448
2.15342339157938 0.30163527046687
2.15220922633799 0.305446540784399
2.15097982576929 0.309252923963793
2.14973520954367 0.313054359103
2.14847539757495 0.31685078537914
2.14720041002011 0.320642142049473
2.14591026727892 0.324428368452374
2.14460498999363 0.328209404008301
2.14328459904865 0.331985188220765
2.14194911557021 0.335755660677301
2.14059856092601 0.339520761050428
2.13923295672491 0.343280429098621
2.13785232481653 0.347034604667273
2.13645668729097 0.350783227689653
2.13504606647838 0.354526238187873
2.13362048494868 0.358263576273846
2.13217996551114 0.361995182150241
2.13072453121404 0.365720996111444
2.12925420534429 0.369440958544512
2.12776901142708 0.373155009930124
2.12626897322549 0.376863090843538
2.12475411474008 0.380565141955538
2.12322446020856 0.384261104033386
2.12168003410537 0.387950917941766
2.1201208611413 0.391634524643736
2.11854696626309 0.395311865201667
2.116958374653 0.398982880778187
2.11535511172849 0.402647512637127
2.11373720314171 0.406305702144454
2.11210467477918 0.409957390769215
2.11045755276131 0.41360252008447
2.10879586344202 0.417241031768226
2.1071196334083 0.420872867604376
2.1054288894798 0.424497969483622
2.10372365870838 0.428116279404412
2.1020039683777 0.431727739473866
2.10026984600277 0.435332291908697
2.09852131932951 0.438929879036146
2.09675841633431 0.442520443294893
2.09498116522358 0.446103927235989
2.0931895944333 0.449680273523765
2.09138373262856 0.453249424936758
2.08956360870312 0.456811324368622
2.08772925177892 0.460365914829041
2.08588069120562 0.463913139444644
2.08401795656016 0.467452941459912
2.08214107764626 0.470985264238089
2.08025008449394 0.474510051262087
2.07834500735904 0.478027246135387
2.07642587672276 0.481536792582947
2.07449272329115 0.485038634452098
2.07254557799463 0.488532715713445
2.07058447198747 0.492018980461763
2.06860943664733 0.49549737291689
2.06662050357473 0.498967837424621
2.06461770459257 0.502430318457598
2.06260107174558 0.505884760616198
2.06057063729984 0.509331108629421
2.05852643374226 0.512769307355771
2.05646849378006 0.516199301784142
2.05439685034023 0.519621037034697
2.05231153656903 0.523034458359745
2.05021258583142 0.526439511144617
2.04810003171057 0.529836140908542
2.04597390800731 0.533224293305515
2.04383424873957 0.536603914125171
2.04168108814185 0.539974949293648
2.03951446066467 0.543337344874456
2.03733440097404 0.546691047069337
2.03514094395085 0.550036002219128
2.03293412469038 0.553372156804617
2.03071397850168 0.556699457447403
2.02848054090705 0.560017850910746
2.02623384764143 0.563327284100422
2.02397393465188 0.566627704065569
2.02170083809695 0.569919057999539
2.01941459434615 0.57320129324074
2.01711523997931 0.576474357273477
2.01480281178607 0.579738197728796
2.01247734676522 0.582992762385319
2.01013888212416 0.586237999170081
2.00778745527826 0.589473856159362
2.00542310385031 0.592700281579521
2.00304586566989 0.595917223807818
2.00065577877274 0.599124631373248
1.99825288140022 0.602322452957357
1.99583721199862 0.605510637395068
1.99340880921861 0.608689133675498
1.99096771191457 0.611857890942775
1.98851395914402 0.61501685849685
1.98604759016695 0.618165985794309
1.9835686444452 0.621305222449183
1.98107716164185 0.624434518233752
1.97857318162059 0.627553823079351
1.97605674444502 0.630663087077168
1.9735278903781 0.633762260479046
1.97098665988144 0.636851293698277
1.96843309361467 0.639930137310395
1.96586723243479 0.642998742053967
1.96328911739553 0.646057058831385
1.96069878974668 0.649105038709643
1.95809629093342 0.652142632921129
1.95548166259567 0.655169792864402
1.95285494656745 0.658186470104965
1.95021618487614 0.661192616376048
1.94756541974187 0.664188183579374
1.94490269357684 0.667173123785933
1.9422280489846 0.670147389236743
1.93954152875941 0.673110932343623
1.93684317588554 0.676063705689945
1.93413303353657 0.679005662031398
1.93141114507473 0.681936754296744
1.92867755405017 0.68485693558857
1.92593230420029 0.687766159184036
1.92317543944903 0.690664378535628
1.92040700390617 0.693551547271897
1.91762704186661 0.696427619198206
1.91483559780969 0.699292548297465
1.91203271639846 0.702146288730869
1.90921844247895 0.704988794838633
1.9063928210795 0.707820021140719
1.90355589740998 0.710639922337566
1.9007077168611 0.713448453310817
1.89784832500371 0.716245569124035
1.89497776758801 0.719031225023427
1.89209609054284 0.721805376438558
1.889203339975 0.724567978983064
1.88629956216842 0.727318988455365
1.88338480358348 0.730058360839365
1.88045911085627 0.732786052305168
1.8775225307978 0.735502019209766
1.87457511039329 0.738206218097747
1.8716168968014 0.740898605701987
1.8686479373535 0.743579138944342
1.86566827955286 0.746247774936336
1.86267797107394 0.748904470979852
1.85967705976163 0.751549184567809
1.85666559363044 0.754181873384845
1.85364362086376 0.756802495307996
1.85061118981308 0.759411008407368
1.84756834899725 0.762007370946806
1.84451514710165 0.764591541384565
1.84145163297745 0.767163478373973
1.8383778556408 0.769723140764093
1.83529386427208 0.772270487600382
1.83219970821509 0.774805478125345
1.82909543697625 0.777328071779186
1.82598110022384 0.779838228200462
1.82285674778718 0.782335907226723
1.81972242965585 0.784821068895159
1.81657819597886 0.787293673443235
1.81342409706389 0.789753681309331
1.81026018337645 0.792201053133375
1.80708650553909 0.794635749757469
1.8039031143306 0.797057732226518
1.80071006068517 0.799466961788857
1.79750739569157 0.801863399896861
1.7942951705924 0.804247008207574
1.79107343678317 0.806617748583313
1.78784224581156 0.808975583092282
1.78460164937656 0.81132047400918
1.78135169932764 0.813652383815802
1.77809244766394 0.815971275201641
1.77482394653341 0.818277111064484
1.77154624823199 0.820569854511007
1.76825940520279 0.822849468857362
1.76496347003524 0.825115917629771
1.76165849546421 0.827369164565099
1.75834453436924 0.829609173611445
1.75502163977364 0.831835908928712
1.75168986484363 0.834049334889182
1.74834926288756 0.836249416078086
1.74499988735498 0.838436117294173
1.74164179183583 0.84060940355027
1.73827503005957 0.842769240073843
1.7348996558943 0.844915592307554
1.73151572334596 0.847048425909813
1.72812328655738 0.849167706755328
1.72472239980748 0.851273400935651
1.72131311751038 0.853365474759718
1.71789549421452 0.855443894754395
1.7144695846018 0.857508627665006
1.7110354434867 0.859559640455867
1.70759312581541 0.861596900310818
1.70414268666493 0.863620374633745
1.70068418124222 0.865630031049101
1.6972176648833 0.867625837402428
1.69374319305234 0.869607761760865
1.69026082134083 0.871575772413666
1.68677060546663 0.873529837872702
1.68327260127313 0.875469926872967
1.67976686472833 0.877396008373079
1.67625345192392 0.879308051555775
1.67273241907444 0.881206025828404
1.66920382251634 0.883089900823419
1.66566771870709 0.884959646398861
1.66212416422427 0.886815232638839
1.65857321576468 0.888656629854013
1.65501493014343 0.890483808582068
1.651449364293 0.892296739588183
1.64787657526237 0.894095393865499
1.6442966202161 0.895879742635588
1.64070955643337 0.897649757348907
1.63711544130715 0.899405409685258
1.63351433234319 0.901146671554242
1.62990628715916 0.902873515095705
1.62629136348371 0.904585912680188
1.62266961915554 0.906283836909365
1.61904111212248 0.907967260616486
1.61540590044056 0.909636156866808
1.6117640422731 0.911290498958025
1.60811559588975 0.912930260420699
1.60446061966556 0.914555415018683
1.6007991720801 0.916165936749537
1.59713131171642 0.917761799844948
1.59345709726024 0.91934297877114
1.58977658749889 0.920909448229284
1.58608984132046 0.922461183155903
1.5823969177128 0.92399815872327
1.57869787576262 0.92552035033981
1.5749927746545 0.927027733650489
1.57128167366999 0.928520284537206
1.56756463218662 0.929997979119179
1.56384170967697 0.931460793753326
1.56011296570774 0.932908705034644
1.55637845993873 0.934341689796583
1.55263825212197 0.935759725111418
1.5488924021007 0.937162788290615
1.54514096980844 0.938550856885191
1.54138401526802 0.93992390868608
1.53762159859064 0.941281921724481
1.53385377997489 0.942624874272216
1.53008061970578 0.943952744842072
1.5263021781538 0.945265512188147
1.52251851577392 0.946563155306194
1.51872969310468 0.947845653433949
1.51493577076714 0.949112986051469
1.51113680946399 0.950365132881461
1.50733286997853 0.9516020738896
1.5035240131737 0.952823789284859
1.49971029999114 0.954030259519816
1.49589179145018 0.955221465290973
1.49206854864686 0.956397387539063
1.48824063275299 0.957558007449356
1.48440810501515 0.958703306451958
1.4805710267537 0.959833266222109
1.4767294593618 0.960947868680476
1.47288346430446 0.962047095993445
1.46903310311751 0.963130930573402
1.46517843740665 0.964199355079016
1.46131952884644 0.965252352415519
1.45745643917936 0.966289905734975
1.45358923021473 0.967311998436554
1.44971796382784 0.968318614166793
1.44584270195885 0.969309736819863
1.44196350661187 0.970285350537823
1.43808043985395 0.971245439710874
1.43419356381407 0.972189988977609
1.43030294068216 0.97311898322526
1.42640863270812 0.97403240758994
1.42251070220079 0.974930247456877
1.41860921152696 0.975812488460654
1.41470422311042 0.976679116485433
1.41079579943088 0.977530117665184
1.40688400302305 0.978365478383906
1.40296889647559 0.979185185275846
1.39905054243011 0.97998922522571
1.39512900358021 0.980777585368878
1.3912043426704 0.981550253091602
1.3872766224952 0.982307216031217
1.38334590589802 0.983048462076332
1.37941225577027 0.983773979367025
1.37547573505025 0.984483756295035
1.37153640672221 0.985177781503947
1.36759433381533 0.985856043889371
1.36364957940268 0.986518532599126
1.35970220660024 0.987165237033404
1.35575227856592 0.987796146844949
1.35179985849845 0.988411251939218
1.34784500963649 0.989010542474542
1.34388779525753 0.989594008862285
1.33992827867692 0.990161641766997
1.33596652324683 0.990713432106565
1.33200259235527 0.991249371052354
1.32803654942504 0.991769450029354
1.32406845791276 0.99227366071631
1.3200983813078 0.992761995045863
1.3161263831313 0.993234445204675
1.31215252693515 0.993691003633552
1.30817687630096 0.99413166302757
1.30419949483905 0.994556416336188
1.30022044618745 0.994965256763361
1.29623979401085 0.995358177767652
1.2922576019996 0.995735173062333
1.28827393386869 0.996096236615488
1.28428885335672 0.996441362650106
1.28030242422489 0.99677054564418
1.27631471025598 0.997083780330788
1.27232577525334 0.997381061698181
1.26833568303984 0.997662384989865
1.26434449745687 0.997927745704672
1.26035228236331 0.998177139596838
1.25635910163453 0.998410562676064
1.25236501916131 0.998628011207587
1.24837009884891 0.998829481712234
1.24437440461596 0.999014970966483
1.24037800039348 0.999184476002508
1.23638095012386 0.999337994108233
1.23238331775981 0.999475522827372
1.22838516726337 0.999597059959467
1.22438656260486 0.999702603559928
1.22038756776186 0.999792151940059
1.21638824671822 0.999865703667088
1.21238866346298 0.999923257564189
1.20838888198938 0.9999648127105
1.20438896629386 0.99999036844114
1.20038898037496 0.999999924347219
1.19638898823239 0.999993480275841
1.19238905386593 0.999971036330112
1.18838924127445 0.999932592869135
1.18438961445486 0.999878150508003
1.1803902374011 0.999807710117794
1.17639117410313 0.999721272825552
1.17239248854587 0.999618840014273
1.1683942447082 0.999500413322878
1.16439650656195 0.999365994646193
1.16039933807082 0.999215586134914
1.15640280318945 0.999049190195573
1.15240696586229 0.998866809490502
1.14841189002267 0.998668446937788
1.1444176395917 0.998454105711228
1.14042427847731 0.998223789240277
1.1364318705732 0.997977501209993
1.1324404797578 0.997715245560981
1.12845016989329 0.997437026489324
1.12446100482453 0.997142848446521
1.12047304837808 0.996832716139416
1.11648636436116 0.996506634530119
1.11250101656063 0.996164608835928
1.10851706874197 0.995806644529247
1.10453458464825 0.995432747337497
1.10055362799915 0.995042923243026
1.09657426248987 0.99463717848301
1.09259655179019 0.994215519549358
1.08862055954339 0.993777953188603
1.08464634936526 0.993324486401797
1.08067398484309 0.9928551264444
1.0767035295346 0.992369880826161
1.07273504696702 0.991868757310999
1.06876860063596 0.99135176391688
1.0648042540045 0.990818908915687
1.06084207050209 0.990270200833089
1.05688211352358 0.989705648448403
1.05292444642821 0.989125260794456
1.04896913253855 0.988529047157438
1.04501623513955 0.987917017076753
1.04106581747748 0.987289180344871
1.03711794275894 0.986645547007165
1.03317267414985 0.985986127361756
1.0292300747744 0.985310931959344
1.02529020771412 0.98461997160304
1.02135313600679 0.983913257348195
1.01741892264547 0.983190800502223
1.01348763057749 0.982452612624418
1.00955932270345 0.98169870552577
1.00563406187617 0.980929091268776
1.00171191089977 0.980143782167249
0.997792932528554 0.979342790786116
0.993877189466108 0.978526129941224
0.989964744364232 0.977693812699127
0.986055659821966 0.976845852376885
0.982149998384578 0.975982262541845
0.978247822542569 0.975103057011424
0.974349194730669 0.974208249852894
0.970454177326839 0.973297855383149
0.966562832651275 0.972371888168482
0.962675222965408 0.971430363024347
0.958791410470911 0.970473295015129
0.954911457308701 0.969500699453893
0.951035425557945 0.968512591902149
0.947163377235069 0.967508988169596
0.943295374292763 0.966489904313872
0.939431478618993 0.965455356640298
0.935571752036005 0.964405361701614
0.931716256299344 0.963339936297717
0.927865053096859 0.96225909747539
0.924018204047719 0.961162862528032
0.920175770701426 0.960051248995378
0.916337814536833 0.958924274663221
0.912504396961155 0.957781957563127
0.908675579308993 0.956624315972145
0.904851422841347 0.955451368412514
0.901031988744639 0.954263133651373
0.897217338129733 0.95305963070045
0.893407532030958 0.951840878815768
0.88960263140513 0.950606897497331
0.885802697130577 0.949357706488814
0.882007790006167 0.948093325777245
0.878217970750334 0.946813775592691
0.874433300000104 0.945519076407925
0.870653838310128 0.944209248938108
0.866879646151713 0.942884314140451
0.863110783911854 0.941544293213883
0.859347311892264 0.94018920759871
0.855589290308418 0.938819078976272
0.851836779288579 0.937433929268599
0.848089838872843 0.936033780638056
0.844348529012179 0.934618655486992
0.840612909567463 0.933188576457378
0.836883040308527 0.931743566430449
0.833158980913199 0.930283648526335
0.829440790966351 0.92880884610369
0.825728529958943 0.927319182759322
0.82202225728707 0.925814682327812
0.818322032251018 0.924295368881136
0.814627914054307 0.922761266728276
0.810939961802749 0.921212400414835
0.807258234503503 0.919648794722638
0.803582791064127 0.918070474669346
0.799913690291636 0.916477465508045
0.796250990891566 0.914869792726846
0.792594751467028 0.913247482048482
0.788945030517776 0.911610559429887
0.785301886439266 0.909959051061789
0.781665377521726 0.908292983368286
0.778035561949221 0.906612383006427
0.774412497798723 0.90491727686578
0.770796243039181 0.903207692068009
0.767186855530594 0.901483655966432
0.763584393023085 0.899745196145592
0.759988913155978 0.897992340420809
0.756400473456873 0.896225116837735
0.75281913134073 0.894443553671912
0.749244944108945 0.892647679428311
0.745677968948439 0.890837522840883
0.742118262930737 0.889013112872094
0.73856588301106 0.887174478712464
0.735020886027411 0.885321649780101
0.731483328699666 0.883454655720229
0.727953267628667 0.881573526404712
0.724430759295315 0.87967829193158
0.720915860059669 0.877768982624543
0.717408626160042 0.87584562903251
0.713909113712101 0.873908261929097
0.71041737870797 0.871956912312137
0.706933477015336 0.869991611403181
0.703457464376551 0.868012390647003
0.699989396407743 0.866019281711092
0.696529328597925 0.864012316485149
0.69307731630811 0.861991527080574
0.689633414770419 0.859956945829955
0.686197679087204 0.857908605286548
0.682770164230162 0.855846538223759
0.679350925039458 0.853770777634616
0.675940016222847 0.851681356731245
0.672537492354796 0.849578308944336
0.669143407875615 0.847461667922608
0.665757817090583 0.845331467532272
0.662380774169081 0.843187741856489
0.659012333143722 0.841030525194825
0.655652547909492 0.838859852062699
0.652301472222883 0.836675757190836
0.648959159701034 0.834478275524707
0.645625663820874 0.832267442223972
0.642301037918267 0.830043292661917
0.638985335187155 0.827805862424886
0.635678608678712 0.825555187311717
0.632380911300491 0.823291303333162
0.62909229581558 0.821014246711317
0.625812814841757 0.81872405387904
0.622542520850647 0.816420761479368
0.619281466166885 0.814104406364928
0.616029702967275 0.811775025597354
0.612787283279959 0.809432656446689
0.609554258983584 0.807077336390788
0.606330681806469 0.804709103114722
0.60311660332578 0.802327994510174
0.599912074966704 0.79993404867483
0.596717148001627 0.797527303911772
0.593531873549312 0.795107798728866
0.590356302574083 0.792675571838143
0.587190485885006 0.79023066215518
0.584034474135082 0.787773108798481
0.580888317820431 0.785302951088847
0.577752067279487 0.782820228548749
0.574625772692193 0.780324980901694
0.571509484079193 0.777817248071592
0.568403251301041 0.775297070182114
0.565307124057394 0.772764487556053
0.562221151886222 0.770219540714676
0.559145384163014 0.76766227037708
0.556079870099987 0.765092717459534
0.553024658745302 0.762510923074831
0.549979798982275 0.759916928531625
0.546945339528597 0.757310775333775
0.543921328935555 0.754692505179675
0.540907815587254 0.752062159961592
0.537904847699842 0.749419781764994
0.534912473320743 0.746765412867876
0.531930740327882 0.744099095740082
0.528959696428924 0.741420873042631
0.525999389160507 0.738730787627028
0.523049865887485 0.736028882534582
0.520111173802167 0.733315200995718
0.517183359923565 0.730589786429282
0.514266471096636 0.727852682441849
0.511360553991541 0.725103932827025
0.508465655102891 0.722343581564744
0.505581820749007 0.719571672820568
0.502709097071177 0.716788250944978
0.499847530032918 0.713993360472664
0.496997165419242 0.711187046121815
0.494158048835922 0.7083693527934
0.491330225708762 0.705540325570451
0.488513741282873 0.702700009717345
0.485708640621945 0.699848450679075
0.482914968607529 0.696985694080524
0.480132769938317 0.694111785725736
0.477362089129429 0.691226771597185
0.474602970511699 0.688330697855034
0.471855458230965 0.685423610836402
0.469119596247366 0.682505557054619
0.466395428334635 0.679576583198483
0.4636829980794 0.676636736131514
0.460982348880488 0.673686062891201
0.458293523948229 0.670724610688254
0.455616566303763 0.667752426905845
0.452951518778356 0.664769559098851
0.450298424012711 0.661776054993093
0.447657324456289 0.658771962484573
0.445028262366625 0.655757329638707
0.442411279808658 0.652732204689556
0.439806418654051 0.649696636039055
0.437213720580529 0.646650672256238
0.434633227071205 0.643594362076459
0.432064979413919 0.640527754400618
0.429509018700581 0.637450898294371
0.426965385826506 0.63436384298735
0.424434121489767 0.631266637872374
0.421915266190538 0.628159332504659
0.419408860230451 0.625041976601025
0.416914943711948 0.6219146200391
0.414433556537639 0.618777312856521
0.411964738409668 0.615630105250138
0.40950852882907 0.612473047575203
0.407064967095147 0.609306190344574
0.404634092304835 0.606129584227898
0.402215943352078 0.602943280050805
0.399810558927208 0.599747328794093
0.397417977516325 0.596541781592916
0.395038237400679 0.593326689735959
0.392671376656063 0.590102104664625
0.390317433152197 0.586868077972204
0.387976444552128 0.583624661403056
0.385648448311622 0.580371906851776
0.383333481678571 0.577109866362367
0.38103158169239 0.573838592127409
0.378742785183432 0.570558136487218
0.37646712877239 0.567268551929015
0.374204648869719 0.563969891086084
0.371955381675049 0.560662206736926
0.369719363176608 0.557345551804421
0.367496629150643 0.554019979354978
0.365287215160851 0.550685542597683
0.36309115655781 0.547342294883456
0.36090848847841 0.543990289704186
0.358739245845293 0.540629580691887
0.356583463366296 0.53726022161783
0.354441175533893 0.533882266391688
0.352312416624642 0.530495769060674
0.350197220698641 0.527100783808672
0.34809562159898 0.523697364955374
0.3460076529512 0.520285566955409
0.343933348162753 0.516865444397472
0.341872740422474 0.513437052003451
0.339825862700041 0.510000444627551
0.337792747745454 0.506555677255418
0.33577342808851 0.503102805003254
0.333767936038279 0.499641883116944
0.331776303682592 0.496172966971162
0.329798562887524 0.492696112068495
0.327834745296886 0.489211374038544
0.325884882331717 0.485718808637046
0.323949005189783 0.482218471744972
0.322027144845076 0.478710419367637
0.320119332047322 0.475194707633804
0.318225597321484 0.471671392794787
0.316345970967279 0.468140531223548
0.314480483058686 0.464602179413796
0.312629163443473 0.461056393979085
0.310792041742715 0.457503231651906
0.308969147350319 0.45394274928278
0.307160509432557 0.450375003839351
0.305366156927596 0.446800052405468
0.30358611854504 0.443217952180278
0.301820422765463 0.439628760477308
0.30006909783996 0.43603253472355
0.298332171789694 0.432429332458538
0.296609672405444 0.428819211333432
0.294901627247162 0.425202229110092
0.293208063643536 0.421578443660158
0.291529008691547 0.417947912964119
0.289864489256038 0.414310695110389
0.288214531969285 0.410666848294376
0.286579163230569 0.40701643081755
0.284958409205755 0.403359501086515
0.283352295826872 0.399696117612068
0.281760848791701 0.396026339008265
0.280184093563361 0.392350223991487
0.278622055369901 0.388667831379495
0.277074759203899 0.384979220090493
0.27554222982206 0.381284449142182
0.274024491744823 0.377583577650818
0.272521569255963 0.373876664830268
0.271033486402209 0.370163769991056
0.269560266992855 0.366444952539421
0.26810193459938 0.362720271976364
0.26665851255507 0.358989787896692
0.265230023954649 0.355253559988073
0.263816491653902 0.351511648030072
0.262417938269317 0.347764111893202
0.261034386177718 0.344011011537961
0.259665857515909 0.340252407013874
0.258312374180319 0.336488358458533
0.256973957826653 0.332718926096636
0.255650629869544 0.328944170239019
0.25434241148221 0.325164151281695
0.253049323596119 0.321378929704888
0.251771386900649 0.317588566072062
0.25050862184276 0.313793121028954
0.249261048626665 0.309992655302604
0.24802868721351 0.306187229700381
0.246811557321051 0.302376905109016
0.24560967842334 0.29856174249362
0.244423069750414 0.294741802896713
0.243251750287986 0.290917147437248
0.242095738777143 0.287087837309629
0.240955053714044 0.283253933782739
0.239829713349626 0.27941549819895
0.238719735689312 0.275572591973152
0.237625138492718 0.27172527659176
0.236545939273378 0.26787361361174
0.235482155298456 0.264017664659617
0.234433803588473 0.260157491430491
0.233400900917034 0.256293155687053
0.23238346381056 0.252424719258592
0.231381508548022 0.248552244040007
0.230395051160683 0.24467579199082
0.229424107431841 0.240795425134181
0.228468692896574 0.236911205555876
0.227528822841495 0.233023195403337
0.226604512304504 0.229131456884643
0.225695776074551 0.225236052267527
0.224802628691396 0.221337043878379
0.223925084445377 0.217434494101251
0.223063157377184 0.213528465376856
0.222216861277632 0.209619020201571
0.22138620968744 0.205706221126434
0.220571215897016 0.201790130756147
0.219771892946243 0.197870811748074
0.218988253624272 0.193948326811233
0.218220310469314 0.190022738705301
0.217468075768444 0.186094110239604
0.216731561557401 0.182162504272112
0.216010779620397 0.178227983708439
0.215305741489927 0.174290611500828
0.214616458446587 0.170350450647152
0.21394294151889 0.166407564189899
0.213285201483092 0.16246201521517
0.212643248863021 0.158513866851662
0.212017093929905 0.154563182269667
0.211406746702209 0.150610024680053
0.210812216945475 0.146654457333257
0.210233514172168 0.142696543518272
0.209670647641519 0.138736346561635
0.209123626359381 0.134773929826413
0.208592459078082 0.130809356711189
0.208077154296288 0.126842690649048
0.207577720258865 0.122873995106562
0.207094164956745 0.118903333582777
0.206626496126804 0.11493076960819
0.206174721251734 0.110956366743742
0.205738847559921 0.106980188579792
0.205318882025337 0.103002298735108
0.204914831367421 0.0990227608558415
0.204526702050974 0.0950416386145136
0.204154500286057 0.0910589957089955
0.203798232027892 0.0870748958614886
0.203457902976761 0.0830894028175055
0.203133518577924 0.0791025803448499
0.202825084021522 0.0751144922325963
0.202532604242504 0.0711252022900694
0.202256083920538 0.0671347743458233
0.201995527479944 0.0631432722466198
0.20175093908962 0.0591507598564074
0.201522322662974 0.0551573010552991
0.201309681857865 0.0511629597385507
0.201113020076541 0.047167799815538
0.200932340465586 0.0431718852087344
0.20076764591587 0.0391752798526885
0.200618939062502 0.0351780476930007
0.20048622228479 0.0311802526853003
0.200369497706197 0.0271819587942221
0.200268767194317 0.023183229992383
0.200184032360833 0.0191841302593587
0.200115294561502 0.0151847235806594
0.200062554896128 0.0111850739467069
0.200025814208543 0.00718524535180974
0.200005073086598 0.00318530179314027
0.20000033186215 -0.000814692730289934
0.200011590611059 -0.00481467421865384
0.200038849153186 -0.00881457867233295
0.200082107052393 -0.0128143420929414
0.200141363616556 -0.0168139004843497
0.200216617897571 -0.0208131898537089
0.200307868691371 -0.0248121462124745
0.200415114537945 -0.02881070557743
0.200538353721362 -0.0328088039717109
0.200677584269797 -0.0368063774258282
0.200832803955566 -0.0408033619786919
0.201004010295155 -0.0447996936786344
0.201191200549268 -0.0487953085844337
0.201394371722864 -0.0527901427663367
0.20161352056521 -0.0567841323070816
0.201848643569927 -0.060777213302921
0.202099736975053 -0.0647693218646442
0.202366796763099 -0.0687603941185992
0.202649818661114 -0.0727503662077152
0.202948798140753 -0.0767391742925239
0.203263730418352 -0.0807267545521811
0.203594610455001 -0.0847130431854875
0.203941432956625 -0.0886979764119102
0.204304192374073 -0.0926814904726026
0.204682882903202 -0.0966635216314246
0.205077498484971 -0.100644006175963
0.205488032805539 -0.104622880418549
0.205914479296365 -0.10860008069728
0.206356831134316 -0.112575543377038
0.206815081241771 -0.116549204850502
0.207289222286738 -0.120521001539176
0.207779246682971 -0.124490869894396
0.20828514659009 -0.128458746398354
0.208806913913706 -0.132424567565109
0.209344540305555 -0.136388269941609
0.209898017163624 -0.140349790108699
0.210467335632297 -0.144309064682142
0.21105248660249 -0.148266030313628
0.211653460711799 -0.152220623691792
0.212270248344653 -0.156172781543225
0.212902839632461 -0.160122440633484
0.213551224453777 -0.16406953776811
0.214215392434458 -0.168014009793631
0.21489533294783 -0.17195579359858
0.215591035114859 -0.175894826114499
0.216302487804325 -0.179831044316954
0.217029679633001 -0.183764385226535
0.217772598965833 -0.187694785909873
0.218531233916127 -0.191622183480641
0.219305572345741 -0.195546515100561
0.220095601865275 -0.199467717980411
0.220901309834275 -0.203385729381029
0.221722683361431 -0.207300486614316
0.222559709304782 -0.211211927044239
0.223412374271933 -0.215119988087835
0.224280664620261 -0.219024607216211
0.225164566457141 -0.222925721955544
0.22606406564016 -0.226823269888082
0.226979147777352 -0.23071718865314
0.227909798227422 -0.234607415948102
0.228856002099982 -0.238493889529414
0.22981774425579 -0.242376547213581
0.230795009306994 -0.246255326878163
0.231787781617372 -0.250130166462769
0.232796045302589 -0.254001003970047
0.233819784230448 -0.257867777466681
0.234858982021147 -0.261730425084376
0.235913622047544 -0.265588885020853
0.236983687435422 -0.269443095540836
0.238069161063755 -0.273292994977039
0.239170025564991 -0.277138521731152
0.24028626332532 -0.28097961427483
0.241417856484962 -0.284816211150673
0.24256478693845 -0.288648250973215
0.243727036334922 -0.292475672429898
0.244904586078413 -0.296298414282062
0.24609741732815 -0.300116415365919
0.247305510998861 -0.303929614593532
0.248528847761071 -0.307737950953796
0.249767408041419 -0.311541363513409
0.251021172022967 -0.315339791417852
0.252290119645518 -0.31913317389236
0.253574230605936 -0.322921450242894
0.254873484358475 -0.326704559857113
0.2561878601151 -0.330482442205344
0.25751733684583 -0.33425503684155
0.258861893279064 -0.338022283404297
0.260221507901927 -0.341784121617721
0.261596158960616 -0.345540491292491
0.262985824460742 -0.349291332326771
0.264390482167687 -0.353036584707185
0.265810109606957 -0.356776188509776
0.267244684064545 -0.360510083900962
0.268694182587288 -0.364238211138496
0.270158581983243 -0.367960510572423
0.271637858822049 -0.37167692264603
0.273131989435309 -0.375387387896804
0.274640949916965 -0.379091846957379
0.276164716123681 -0.382790240556491
0.27770326367523 -0.38648250951992
0.279256567954885 -0.390168594771442
0.280824604109811 -0.393848437333771
0.282407347051461 -0.397521978329505
0.284004771455984 -0.401189158982066
0.285616851764622 -0.404849920616641
0.287243562184125 -0.408504204661124
0.28888487668716 -0.412151952647047
0.290540769012732 -0.41579310621052
0.292211212666598 -0.419427607093165
0.293896180921695 -0.423055397143043
0.295595646818567 -0.426676418315592
0.297309583165795 -0.430290612674551
0.299037962540436 -0.433897922392886
0.300780757288456 -0.43749828975372
0.302537939525176 -0.441091657151251
0.304309481135717 -0.444677967091677
0.306095353775452 -0.448257162194116
0.307895528870457 -0.451829185191524
0.309709977617969 -0.455393978931607
0.311538670986845 -0.458951486377742
0.313381579718033 -0.462501650609886
0.31523867432503 -0.466044414825487
0.317109925094363 -0.469579722340393
0.318995302086059 -0.473107516589759
0.320894775134127 -0.476627741128953
0.322808313847039 -0.480140339634457
0.324735887608216 -0.48364525590477
0.326677465576519 -0.487142433861306
0.328633016686742 -0.490631817549293
0.330602509650109 -0.494113351138665
0.332585912954774 -0.497586978924961
0.334583194866327 -0.501052645330209
0.3365943234283 -0.50451029490382
0.338619266462679 -0.507959872323477
0.340657991570418 -0.511401322396012
0.34271046613196 -0.514834590058299
0.344776657307754 -0.518259620378128
0.346856532038787 -0.521676358555087
0.348950057047107 -0.525084749921439
0.351057198836359 -0.528484739942994
0.353177923692318 -0.531876274219984
0.355312197683433 -0.535259298487933
0.357459986661365 -0.538633758618525
0.359621256261536 -0.54199960062047
0.361795971903678 -0.545356770640367
0.363984098792389 -0.548705214963569
0.366185601917684 -0.552044880015037
0.36840044605556 -0.555375712360202
0.370628595768559 -0.558697658705817
0.372870015406332 -0.562010665900812
0.375124669106213 -0.565314680937142
0.377392520793791 -0.568609650950637
0.379673534183488 -0.571895523221846
0.381967672779137 -0.575172245176885
0.38427489987457 -0.578439764388272
0.386595178554203 -0.581698028575768
0.388928471693627 -0.584946985607216
0.391274741960201 -0.588186583499374
0.393633951813651 -0.591416770418744
0.396006063506669 -0.594637494682404
0.398391039085519 -0.597848704758834
0.400788840390643 -0.601050349268743
0.403199429057271 -0.604242376985885
0.405622766516035 -0.607424736837886
0.408058813993589 -0.610597377907057
0.410507532513223 -0.613760249431207
0.412968882895496 -0.616913300804459
0.415442825758851 -0.620056481578059
0.417929321520258 -0.623189741461183
0.420428330395836 -0.626313030321737
0.422939812401497 -0.629426298187168
0.425463727353582 -0.632529495245255
0.428000034869506 -0.635622571844912
0.430548694368403 -0.63870547849698
0.433109665071775 -0.641778165875018
0.435682906004145 -0.644840584816093
0.438268375993714 -0.647892686321567
0.440866033673017 -0.650934421557882
0.443475837479586 -0.653965741857339
0.446097745656616 -0.656986598718877
};
\addlegendentry{ODE}
\addplot [line width = \lineWidthSUR, color = SUR1, dashed]
table {%
0.2 0
0.200001186061397 -0.00400071280052999
0.200018424612073 -0.00800139582427851
0.200051715146851 -0.012001984925729
0.200101056897584 -0.0160024159609204
0.20016644883324 -0.0200026247884981
0.200247889659978 -0.024002547270764
0.20034537782124 -0.0280021192747275
0.200458911497841 -0.0320012766731548
0.200588488608063 -0.0359999553456193
0.200734106807761 -0.0399980911795506
0.20089576349046 -0.0439956200712842
0.201073455787472 -0.0479924779271096
0.201267180568004 -0.0519886006643196
0.201476934439281 -0.0559839242122578
0.201702713746663 -0.0599783845133665
0.201944514573777 -0.0639719175242345
0.202202332742646 -0.0679644592166433
0.202476163813824 -0.0719559455786138
0.202766003086536 -0.0759463126154528
0.203071845598825 -0.0799354963507977
0.203393686127696 -0.0839234328276622
0.203731519189275 -0.0879100581094807
0.204085339038959 -0.0918953082811519
0.204455139671582 -0.0958791194500829
0.204840914821581 -0.0998614277472314
0.205242657963165 -0.103842169328148
0.205660362310488 -0.10782128037402
0.206094020817829 -0.111798697092707
0.206543626179774 -0.115774355719788
0.207009170831405 -0.119748192519598
0.207490646948487 -0.123720143786266
0.207988046447669 -0.127690145844755
0.208501360986678 -0.131658135051898
0.209030581964526 -0.135624047797439
0.209575700521718 -0.139587820505063
0.210136707540463 -0.143549389633438
0.21071359364489 -0.147508691677244
0.211306349201269 -0.151465663168211
0.211914964318236 -0.155420240676149
0.212539428847019 -0.159372360809982
0.213179732381676 -0.163321960218779
0.213835864259326 -0.167268975592784
0.214507813560393 -0.171213343664447
0.21519556910885 -0.175155001209449
0.215899119472469 -0.179093885047736
0.216618452963074 -0.183029932044539
0.217353557636797 -0.186963079111405
0.218104421294339 -0.190893263207218
0.218871031481237 -0.194820421339226
0.219653375488132 -0.198744490564064
0.220451440351043 -0.202665407988772
0.221265212851643 -0.20658311077182
0.22209467951754 -0.210497536124125
0.222939826622564 -0.214408621310072
0.223800640187056 -0.218316303648531
0.224677105978158 -0.22222052051387
0.225569209510113 -0.226121209336976
0.226476936044566 -0.230018307606266
0.227400270590867 -0.2339117528687
0.228339197906382 -0.237801482730794
0.229293702496802 -0.241687434859631
0.230263768616465 -0.245569546983868
0.231249380268672 -0.249447756894747
0.232250521206012 -0.253322002447101
0.233267174930693 -0.25719222156036
0.234299324694871 -0.261058352219554
0.235346953500986 -0.264920332476319
0.236410044102105 -0.268778100449898
0.237488579002262 -0.272631594328138
0.238582540456805 -0.276480752368497
0.23969191047275 -0.280325512899034
0.240816670809137 -0.284165814319408
0.241956802977382 -0.288001595101877
0.243112288241648 -0.291832793792286
0.244283107619203 -0.295659349011065
0.245469241880797 -0.299481199454213
0.246670671551032 -0.303298283894295
0.247887376908741 -0.307110541181425
0.249119337987367 -0.310917910244253
0.250366534575351 -0.314720330090953
0.251628946216519 -0.318517739810203
0.252906552210471 -0.322310078572169
0.254199331612986 -0.326097285629484
0.255507263236411 -0.329879300318231
0.256830325650071 -0.333656062058914
0.258168497180676 -0.337427510357438
0.259521755912727 -0.341193584806081
0.260890079688933 -0.344954225084471
0.262273446110632 -0.348709370960549
0.263671832538206 -0.352458962291548
0.265085216091511 -0.35620293902495
0.266513573650303 -0.359941241199461
0.267956881854673 -0.36367380894597
0.269415117105477 -0.367400582488513
0.270888255564785 -0.371121502145236
0.272376273156313 -0.37483650832935
0.273879145565877 -0.378545541550089
0.275396848241841 -0.38224854241367
0.276929356395569 -0.385945451624242
0.278476645001883 -0.38963620998484
0.280038688799524 -0.393320758398333
0.281615462291613 -0.396999037868376
0.283206939746125 -0.400670989500356
0.284813095196353 -0.404336554502332
0.286433902441385 -0.407995674185982
0.288069335046582 -0.411648289967546
0.289719366344059 -0.415294343368758
0.291383969433171 -0.41893377601779
0.293063117180998 -0.422566529650183
0.294756782222838 -0.426192546109782
0.296464936962704 -0.429811767349667
0.298187553573818 -0.433424135433083
0.299924603999114 -0.437029592534367
0.301676059951748 -0.440628080939871
0.303441892915597 -0.444219543048889
0.305222074145779 -0.447803921374577
0.307016574669162 -0.451381158544871
0.308825365284887 -0.454951197303407
0.310648416564886 -0.458513980510435
0.312485698854409 -0.462069451143728
0.314337182272551 -0.465617552299503
0.316202836712785 -0.469158227193318
0.318082631843492 -0.472691419160989
0.319976537108505 -0.476217071659488
0.321884521727649 -0.479735128267849
0.32380655469728 -0.483245532688067
0.325742604790841 -0.486748228745997
0.327692640559404 -0.49024316039225
0.329656630332231 -0.493730271703086
0.331634542217327 -0.497209506881307
0.333626344102005 -0.500680810257147
0.335632003653441 -0.504144126289155
0.337651488319252 -0.507599399565086
0.339684765328055 -0.51104657480278
0.341731801690048 -0.514485596851042
0.343792564197581 -0.517916410690523
0.345867019425738 -0.521338961434594
0.347955133732918 -0.52475319433022
0.35005687326142 -0.528159054758831
0.352172203938033 -0.531556488237192
0.354301091474626 -0.534945440418266
0.356443501368741 -0.538325857092084
0.358599398904196 -0.541697684186603
0.360768749151679 -0.545060867768565
0.362951516969356 -0.548415354044358
0.365147667003476 -0.551761089360865
0.367357163688981 -0.55509802020632
0.369579971250117 -0.55842609321116
0.371816053701052 -0.561745255148865
0.374065374846492 -0.565055452936811
0.376327898282305 -0.568356633637107
0.378603587396143 -0.571648744457436
0.380892405368069 -0.574931732751896
0.383194315171192 -0.578205546021831
0.385509279572293 -0.581470131916664
0.387837261132464 -0.584725438234729
0.39017822220775 -0.587971412924095
0.392532124949785 -0.591208004083396
0.394898931306442 -0.594435159962647
0.397278603022474 -0.59765282896407
0.39967110164017 -0.600860959642905
0.402076388500007 -0.604059500708231
0.404494424741301 -0.607248401023772
0.406925171302872 -0.610427609608709
0.409368588923701 -0.61359707563849
0.411824638143595 -0.616756748445625
0.414293279303857 -0.619906577520497
0.416774472547949 -0.623046512512156
0.419268177822171 -0.626176503229115
0.421774354876333 -0.629296499640146
0.424292963264431 -0.632406451875065
0.426823962345332 -0.635506310225528
0.429367311283451 -0.63859602514581
0.431922969049445 -0.64167554725359
0.434490894420892 -0.644744827330731
0.437071045982988 -0.647803816324054
0.439663382129241 -0.650852465346118
0.442267861062163 -0.653890725675985
0.444884440793972 -0.656918548759996
0.447513079147295 -0.659935886212527
0.450153733755866 -0.662942689816764
0.452806362065241 -0.665938911525453
0.455470921333498 -0.668924503461665
0.458147368631957 -0.671899417919544
0.460835660845889 -0.674863607365065
0.463535754675234 -0.677817024436778
0.46624760663532 -0.680759621946559
0.468971173057587 -0.683691352880348
0.471706410090308 -0.686612170398894
0.474453273699316 -0.689522027838487
0.477211719668737 -0.6924208787117
0.479981703601716 -0.695308676708114
0.482763180921155 -0.698185375695048
0.485556106870446 -0.70105092971829
0.488360436514214 -0.703905293002812
0.491176124739055 -0.706748419953495
0.494003126254279 -0.709580265155844
0.496841395592659 -0.712400783376704
0.499690887111176 -0.715209929564968
0.502551554991771 -0.718007658852286
0.5054233532421 -0.720793926553772
0.508306235696282 -0.723568688168702
0.511200156015667 -0.726331899381219
0.514105067689587 -0.729083516061021
0.51702092403612 -0.73182349426406
0.51994767820286 -0.734551790233232
0.522885283167672 -0.737268360399055
0.525833691739475 -0.739973161380368
0.528792856559 -0.742666149984996
0.531762730099571 -0.745347283210437
0.534743264667878 -0.748016518244536
0.537734412404753 -0.750673812466149
0.540736125285953 -0.753319123445823
0.543748355122938 -0.75595240894645
0.546771053563659 -0.758573626923934
0.549804172093339 -0.761182735527853
0.552847662035268 -0.763779693102106
0.555901474551586 -0.766364458185575
0.55896556064408 -0.768936989512769
0.562039871154977 -0.77149724601447
0.56512435676774 -0.77404518681838
0.568218968007868 -0.776580771249755
0.571323655243695 -0.779103958832048
0.574438368687194 -0.781614709287538
0.577563058394778 -0.784112982537961
0.580697674268112 -0.78659873870514
0.583842166054919 -0.789071938111605
0.586996483349787 -0.791532541281214
0.590160575594988 -0.793980508939775
0.593334392081287 -0.796415802015653
0.596517881948762 -0.798838381640385
0.599710994187618 -0.801248209149289
0.60291367763901 -0.803645246082063
0.606125880995866 -0.806029454183392
0.609347552803709 -0.80840079540354
0.612578641461481 -0.810759231898949
0.615819095222374 -0.813104726032829
0.619068862194659 -0.815437240375743
0.622327890342512 -0.817756737706195
0.625596127486854 -0.82006318101121
0.628873521306183 -0.822356533486909
0.632160019337408 -0.824636758539089
0.635455568976689 -0.826903819783789
0.638760117480281 -0.82915768104786
0.642073611965368 -0.831398306369529
0.645395999410914 -0.833625659998961
0.648727226658503 -0.835839706398815
0.652067240413187 -0.838040410244798
0.655415987244335 -0.840227736426218
0.658773413586486 -0.84240165004653
0.662139465740196 -0.844562116423881
0.665514089872894 -0.846709101091647
0.668897232019737 -0.848842569798977
0.672288838084468 -0.850962488511318
0.675688853840269 -0.853068823410954
0.679097224930629 -0.855161540897528
0.682513896870198 -0.857240607588565
0.685938815045657 -0.859305990319994
0.689371924716575 -0.861357656146665
0.69281317101628 -0.86339557234286
0.696262498952728 -0.865419706402802
0.699719853409366 -0.867430026041166
0.70318517914601 -0.869426499193575
0.70665842079971 -0.871409094017103
0.710139522885629 -0.873377778890771
0.713628429797917 -0.875332522416038
0.717125085810582 -0.877273293417288
0.720629435078378 -0.879200060942316
0.724141421637674 -0.881112794262814
0.727660989407343 -0.883011462874841
0.731188082189635 -0.884896036499303
0.734722643671072 -0.886766485082423
0.738264617423322 -0.888622778796209
0.74181394690409 -0.890464888038913
0.745370575458007 -0.892292783435499
0.748934446317516 -0.894106435838093
0.752505502603761 -0.895905816326441
0.756083687327483 -0.897690896208354
0.759668943389907 -0.899461647020158
0.763261213583641 -0.901218040527134
0.766860440593569 -0.902960048723956
0.770466566997745 -0.904687643835131
0.774079535268294 -0.906400798315424
0.777699287772309 -0.908099484850291
0.781325766772754 -0.9097836763563
0.78495891442936 -0.911453345981556
0.788598672799531 -0.91310846710611
0.792244983839246 -0.914749013342382
0.795897789403967 -0.916374958535567
0.799557031249541 -0.917986276764037
0.803222651033105 -0.91958294233975
0.806894590314001 -0.921164929808645
0.810572790554681 -0.922732213951039
0.814257193121613 -0.924284769782016
0.817947739286203 -0.925822572551819
0.821644370225696 -0.92734559774623
0.825347027024097 -0.928853821086954
0.829055650673081 -0.930347218531995
0.832770182072912 -0.931825766276025
0.836490562033355 -0.933289440750758
0.840216731274599 -0.934738218625318
0.84394863042817 -0.936172076806594
0.847686200037852 -0.937590992439603
0.851429380560611 -0.938994942907844
0.855178112367509 -0.940383905833649
0.858932335744635 -0.941757859078528
0.862691990894018 -0.943116780743517
0.866457017934561 -0.944460649169513
0.870227356902958 -0.945789442937614
0.874002947754625 -0.947103140869445
0.877783730364623 -0.948401722027496
0.881569644528587 -0.949685165715441
0.885360629963656 -0.950953451478456
0.889156626309398 -0.952206559103544
0.892957573128743 -0.953444468619842
0.896763409908915 -0.954667160298935
0.90057407606236 -0.955874614655159
0.904389510927679 -0.957066812445907
0.908209653770562 -0.95824373467192
0.912034443784723 -0.959405362577592
0.915863820092833 -0.960551677651252
0.919697721747457 -0.961682661625458
0.923536087731984 -0.962798296477273
0.927378856961575 -0.96389856442855
0.931225968284089 -0.964983447946209
0.935077360481027 -0.966052929742503
0.938932972268471 -0.96710699277529
0.942792742298023 -0.968145620248296
0.94665660915774 -0.969168795611376
0.950524511373083 -0.970176502560775
0.954396387407853 -0.971168725039369
0.958272175665136 -0.972145447236929
0.962151814488241 -0.973106653590354
0.966035242161647 -0.974052328783919
0.969922396911948 -0.974982457749511
0.973813216908794 -0.975897025666863
0.977707640265836 -0.976796017963781
0.981605605041674 -0.977679420316375
0.985507049240799 -0.978547218649278
0.989411910814546 -0.979399399135865
0.993320127662033 -0.980235948198467
0.997231637631115 -0.981056852508581
1.00114637851933 -0.981862098987082
1.00506428807484 -0.982651674804416
1.0089853039974 -0.98342556738081
1.01290936393928 -0.984183764386459
1.01683640550624 -0.984926253741723
1.02076636625847 -0.985653023617308
1.02469918371152 -0.986364062434458
1.02863479533731 -0.987059358865126
1.032573138565 -0.987738901832157
1.03651415078201 -0.988402680509455
1.04045776933494 -0.989050684322155
1.04440393153054 -0.989682902946783
1.04835257463664 -0.990299326311419
1.05230363588312 -0.990899944595853
1.05625705246287 -0.991484748231739
1.06021276153271 -0.992053727902738
1.0641707002144 -0.992606874544672
1.06813080559554 -0.993144179345654
1.07209301473056 -0.993665633746233
1.07605726464166 -0.994171229439524
1.08002349231978 -0.994660958371337
1.08399163472552 -0.995134812740304
1.08796162879014 -0.995592784997997
1.09193341141651 -0.996034867849049
1.09590691948004 -0.996461054251264
1.09988208982964 -0.996871337415728
1.10385885928872 -0.997265710806916
1.10783716465609 -0.997644168142791
1.11181694270695 -0.998006703394906
1.11579813019386 -0.998353310788491
1.11978066384765 -0.998683984802551
1.12376448037844 -0.998998720169943
1.12774951647655 -0.999297511877469
1.13173570881346 -0.999580355165942
1.13572299404283 -0.999847245530271
1.13971130880135 -1.00009817871952
1.14370058970983 -1.00033315073699
1.14769077337403 -1.00055215784027
1.15168179638573 -1.00075519654129
1.1556735953236 -1.00094226360639
1.15966610675423 -1.00111335605636
1.16365926723304 -1.0012684711665
1.16765301330527 -1.00140760646663
1.17164728150691 -1.00153075974118
1.1756420083657 -1.00163792902917
1.17963713040207 -1.0017291126243
1.18363258413007 -1.00180430907491
1.18762830605839 -1.00186351718406
1.19162423269127 -1.00190673600952
1.19562030052949 -1.00193396486378
1.1996164460713 -1.00194520331408
1.20361260581342 -1.00194045118243
1.20760871625196 -1.00191970854555
1.21160471388342 -1.00188297573496
1.21560053520562 -1.00183025333688
1.21959611671866 -1.00176154219231
1.2235913949259 -1.00167684339696
1.22758630633491 -1.00157615830123
1.23158078745843 -1.00145948851023
1.23557477481532 -1.00132683588374
1.23956820493155 -1.00117820253617
1.24356101434112 -1.00101359083651
1.24755313958705 -1.00083300340836
1.25154451722232 -1.00063644312981
1.25553508381086 -1.00042391313346
1.25952477592846 -1.00019541680632
1.26351353016379 -0.999950957789793
1.26750128311931 -0.999690539979621
1.27148797141224 -0.999414167525796
1.27547353167555 -0.999121844832515
1.27945790055888 -0.998813576558115
1.28344101472951 -0.998489367614989
1.28742281087334 -0.998149223169519
1.29140322569583 -0.997793148641997
1.29538219592294 -0.997421149706532
1.29935965830213 -0.997033232290971
1.3033355496033 -0.996629402576806
1.30730980661973 -0.996209666999075
1.31128236616906 -0.995774032246264
1.31525316509426 -0.995322505260203
1.31922214026454 -0.994855093235963
1.32318922857635 -0.994371803621736
1.32715436695432 -0.993872644118727
1.33111749235223 -0.993357622681034
1.33507854175394 -0.992826747515518
1.33903745217439 -0.992280027081686
1.34299416066051 -0.991717470091551
1.34694860429219 -0.991139085509502
1.35090072018326 -0.990544882552163
1.35485044548242 -0.98993487068825
1.3587977173742 -0.989309059638427
1.36274247307991 -0.988667459375149
1.36668464985863 -0.988010080122516
1.37062418500811 -0.987336932356105
1.37456101586575 -0.986648026802814
1.37849507980958 -0.985943374440694
1.38242631425917 -0.985222986498775
1.3863546566766 -0.984486874456898
1.39028004456741 -0.98373505004553
1.39420241548159 -0.982967525245588
1.39812170701446 -0.982184312288247
1.40203785680767 -0.981385423654753
1.40595080255018 -0.980570872076232
1.40986048197911 -0.979740670533485
1.41376683288082 -0.978894832256793
1.41766979309175 -0.978033370725706
1.42156930049944 -0.977156299668839
1.42546529304344 -0.976263633063653
1.4293577087163 -0.97535538513624
1.43324648556447 -0.974431570361103
1.43713156168928 -0.97349220346093
1.44101287524789 -0.972537299406362
1.44489036445421 -0.971566873415766
1.4487639675799 -0.970580940954991
1.45263362295525 -0.969579517737135
1.45649926897016 -0.968562619722294
1.46036084407513 -0.967530263117317
1.4642182867821 -0.966482464375552
1.4680715356655 -0.965419240196592
1.47192052936314 -0.964340607526011
1.47576520657716 -0.963246583555106
1.47960550607499 -0.962137185720622
1.48344136669028 -0.961012431704489
1.48727272732383 -0.959872339433537
1.49109952694456 -0.958716927079224
1.49492170459044 -0.95754621305735
1.49873919936944 -0.956360216027769
1.50255195046044 -0.955158954894103
1.50635989711421 -0.953942448803439
1.51016297865431 -0.95271071714604
1.51396113447807 -0.951463779555035
1.5177543040575 -0.950201655906116
1.52154242694023 -0.948924366317231
1.52532544275045 -0.947631931148263
1.52910329118988 -0.946324371000721
1.53287591203863 -0.945001706717412
1.53664324515622 -0.943663959382118
1.54040523048246 -0.94231115031927
1.54416180803838 -0.940943301093608
1.54791291792723 -0.939560433509855
1.55165850033531 -0.938162569612366
1.55539849553301 -0.936749731684792
1.55913284387565 -0.935321942249728
1.56286148580446 -0.933879224068363
1.56658436184753 -0.932421600140125
1.57030141262066 -0.930949093702318
1.57401257882838 -0.929461728229765
1.57771780126482 -0.92795952743444
1.58141702081463 -0.926442515265093
1.58511017845398 -0.924910715906885
1.58879721525139 -0.923364153780999
1.59247807236872 -0.921802853544267
1.59615269106207 -0.920226840088783
1.5998210126827 -0.918636138541513
1.60348297867799 -0.917030774263902
1.60713853059228 -0.915410772851479
1.61078761006788 -0.913776160133456
1.61443015884595 -0.912126962172327
1.61806611876739 -0.910463205263457
1.62169543177383 -0.908784915934674
1.62531803990848 -0.907092120945852
1.62893388531707 -0.905384847288494
1.63254291024879 -0.903663122185311
1.63614505705716 -0.901926973089793
1.63974026820097 -0.900176427685786
1.6433284862452 -0.89841151388705
1.64690965386191 -0.896632259836833
1.65048371383115 -0.894838693907424
1.65405060904192 -0.893030844699707
1.65761028249299 -0.891208741042724
1.66116267729388 -0.889372411993213
1.66470773666575 -0.887521886835161
1.66824540394229 -0.885657195079341
1.67177562257063 -0.883778366462855
1.67529833611224 -0.881885430948666
1.67881348824387 -0.879978418725127
1.68232102275837 -0.878057360205513
1.6858208835657 -0.876122286027543
1.68931301469371 -0.874173227052902
1.69279736028913 -0.872210214366755
1.69627386461843 -0.870233279277263
1.69974247206872 -0.868242453315094
1.70320312714862 -0.866237768232925
1.7066557744892 -0.864219256004949
1.71010035884485 -0.862186948826376
1.71353682509414 -0.860140879112922
1.71696511824076 -0.85808107950031
1.72038518341438 -0.856007582843752
1.72379696587153 -0.853920422217438
1.72720041099652 -0.851819630914018
1.73059546430227 -0.849705242444082
1.73398207143124 -0.847577290535631
1.73736017815629 -0.845435809133552
1.74072973038158 -0.843280832399087
1.74409067414339 -0.841112394709296
1.74744295561109 -0.838930530656519
1.75078652108792 -0.836735275047835
1.75412131701194 -0.834526662904515
1.75744728995685 -0.832304729461478
1.76076438663289 -0.830069510166732
1.7640725538877 -0.827821040680824
1.7673717387072 -0.825559356876278
1.77066188821643 -0.823284494837039
1.77394294968042 -0.820996490857899
1.7772148705051 -0.818695381443934
1.7804775982381 -0.816381203309931
1.78373108056962 -0.814053993379811
1.78697526533334 -0.811713788786053
1.79021010050722 -0.80936062686911
1.79343553421436 -0.806994545176822
1.79665151472391 -0.804615581463835
1.79985799045183 -0.802223773691
1.80305490996182 -0.799819160024783
1.80624222196613 -0.797401778836665
1.80941987532642 -0.794971668702545
1.81258781905458 -0.792528868402126
1.8157460023136 -0.790073416918318
1.8188943744184 -0.787605353436618
1.82203288483668 -0.785124717344498
1.82516148318972 -0.782631548230792
1.82828011925327 -0.780125885885069
1.83138874295833 -0.777607770297014
1.83448730439203 -0.775077241655796
1.83757575379841 -0.772534340349441
1.84065404157927 -0.769979106964198
1.84372211829503 -0.767411582283904
1.84677993466548 -0.764831807289341
1.84982744157067 -0.762239823157595
1.85286459005167 -0.75963567126141
1.85589133131144 -0.757019393168539
1.85890761671561 -0.754391030641092
1.86191339779329 -0.751750625634881
1.86490862623793 -0.749098220298761
1.86789325390804 -0.74643385697397
1.87086723282809 -0.743757578193461
1.87383051518925 -0.74106942668124
1.87678305335021 -0.738369445351694
1.879724799838 -0.735657677308912
1.88265570734873 -0.732934165846017
1.88557572874846 -0.730198954444482
1.88848481707393 -0.727452086773445
1.89138292553338 -0.724693606689031
1.89427000750734 -0.721923558233655
1.89714601654939 -0.719141985635338
1.90001090638696 -0.716348933307008
1.90286463092213 -0.713544445845804
1.90570714423235 -0.710728568032376
1.90853840057129 -0.707901344830183
1.91135835436953 -0.705062821384785
1.91416696023541 -0.702213043023134
1.91696417295573 -0.699352055252865
1.91974994749657 -0.696479903761576
1.92252423900402 -0.693596634416117
1.92528700280494 -0.690702293261865
1.92803819440773 -0.687796926522001
1.93077776950307 -0.684880580596786
1.93350568396468 -0.68195330206283
1.93622189385008 -0.679015137672361
1.93892635540132 -0.676066134352489
1.94161902504572 -0.673106339204473
1.94429985939661 -0.670135799502975
1.94696881525411 -0.667154562695321
1.94962584960579 -0.664162676400754
1.95227091962746 -0.661160188409683
1.9549039826839 -0.658147146682937
1.95752499632953 -0.655123599351006
1.96013391830919 -0.652089594713287
1.96273070655886 -0.649045181237325
1.96531531920634 -0.645990407558048
1.96788771457197 -0.642925322477004
1.97044785116938 -0.639849974961594
1.97299568770617 -0.6367644141443
1.97553118308461 -0.633668689321913
1.97805429640235 -0.63056284995476
1.98056498695313 -0.627446945665919
1.98306321422747 -0.624321026240446
1.98554893791335 -0.621185141624586
1.98802211789692 -0.618039341924988
1.99048271426316 -0.61488367740792
1.99293068729661 -0.611718198498472
1.99536599748199 -0.608542955779765
1.99778860550493 -0.605357999992156
2.00019847225263 -0.602163382032438
2.00259555881449 -0.598959152953035
2.00497982648285 -0.595745363961203
2.00735123675359 -0.59252206641822
2.00970975132683 -0.589289311838578
2.01205533210757 -0.586047151889171
2.01438794120634 -0.582795638388481
2.01670754093988 -0.579534823305761
2.01901409383175 -0.576264758760217
2.021307562613 -0.572985497020184
2.02358791022279 -0.569697090502305
2.02585509980907 -0.566399591770703
2.02810909472916 -0.563093053536148
2.0303498585504 -0.559777528655235
2.03257735505079 -0.55645307012954
2.03479154821961 -0.553119731104791
2.03699240225804 -0.549777564870026
2.03917988157975 -0.546426624856753
2.04135395081155 -0.543066964638105
2.04351457479399 -0.539698637927997
2.04566171858194 -0.536321698580276
2.04779534744524 -0.53293620058787
2.04991542686924 -0.529542198081937
2.05202192255545 -0.52613974533101
2.0541148004221 -0.522728896740138
2.05619402660473 -0.519309706850025
2.0582595674568 -0.515882230336173
2.06031138955022 -0.512446522008014
2.06234945967597 -0.509002636808042
2.06437374484469 -0.505550629810948
2.06638421228717 -0.502090556222749
2.06838082945499 -0.498622471379908
2.07036356402107 -0.495146430748468
2.0723323838802 -0.491662489923168
2.07428725714961 -0.488170704626565
2.07622815216953 -0.484671130708151
2.07815503750372 -0.481163824143471
2.08006788194003 -0.477648841033232
2.08196665449092 -0.474126237602422
2.08385132439402 -0.470596070199409
2.08572186111264 -0.467058395295057
2.08757823433631 -0.463513269481826
2.08942041398129 -0.459960749472879
2.09124837019112 -0.456400892101177
2.0930620733371 -0.452833754318581
2.09486149401884 -0.449259393194951
2.09664660306472 -0.445677865917236
2.09841737153244 -0.442089229788568
2.10017377070951 -0.438493542227352
2.10191577211373 -0.434890860766358
2.10364334749371 -0.431281243051802
2.10535646882931 -0.427664746842434
2.1070551083322 -0.424041430008617
2.10873923844627 -0.420411350531412
2.11040883184815 -0.416774566501653
2.11206386144768 -0.413131136119022
2.11370430038834 -0.409481117691128
2.11533012204778 -0.405824569632577
2.11694130003822 -0.402161550464041
2.11853780820697 -0.398492118811329
2.1201196206368 -0.394816333404454
2.12168671164647 -0.391134253076696
2.12323905579115 -0.387445936763665
2.12477662786283 -0.383751443502363
2.12629940289081 -0.380050832430245
2.12780735614209 -0.376344162784273
2.12930046312181 -0.372631493899973
2.1307786995737 -0.368912885210488
2.13224204148047 -0.365188396245631
2.13369046506426 -0.361458086630935
2.135123946787 -0.357722016086702
2.13654246335089 -0.353980244427047
2.13794599169876 -0.350232831558942
2.13933450901447 -0.346479837481268
2.14070799272333 -0.342721322283842
2.14206642049247 -0.338957346146468
2.14340977023126 -0.33518796933797
2.14473802009166 -0.331413252215231
2.14605114846863 -0.327633255222222
2.14734913400048 -0.323848038889041
2.14863195556928 -0.320057663830939
2.1498995923012 -0.316262190747354
2.15115202356687 -0.312461680420935
2.15238922898177 -0.308656193716571
2.15361118840657 -0.304845791580414
2.15481788194747 -0.301030535038901
2.15600928995657 -0.297210485197779
2.15718539303222 -0.293385703241122
2.15834617201935 -0.289556250430347
2.15949160800977 -0.285722188103238
2.16062168234258 -0.281883577672954
2.16173637660443 -0.278040480627045
2.16283567262989 -0.274192958526466
2.16391955250174 -0.270341073004586
2.1649879985513 -0.266484885766197
2.16604099335874 -0.262624458586522
2.1670785197534 -0.258759853310222
2.16810056081408 -0.254891131850401
2.16910709986933 -0.251018356187606
2.17009812049777 -0.247141588368834
2.17107360652839 -0.24326089050653
2.17203354204079 -0.239376324777586
2.17297791136551 -0.235487953422341
2.1739066990843 -0.231595838743574
2.17481989003038 -0.227700043105502
2.17571746928875 -0.223800628932774
2.17659942219639 -0.21989765870946
2.1774657343426 -0.215991194978049
2.1783163915692 -0.212081300338431
2.17915137997084 -0.208168037446891
2.1799706858952 -0.204251469015095
2.18077429594327 -0.200331657809078
2.18156219696958 -0.196408666648226
2.18233437608245 -0.19248255840426
2.18309082064422 -0.188553396000222
2.18383151827148 -0.184621242409453
2.18455645683532 -0.180686160654576
2.18526562446149 -0.176748213806473
2.18595900953071 -0.172807464983263
2.18663660067881 -0.168863977349282
2.187298386797 -0.164917814114054
2.18794435703202 -0.160969038531272
2.18857450078639 -0.157017713897766
2.18918880771858 -0.153063903552478
2.18978726774325 -0.149107670875433
2.19036987103137 -0.145149079286713
2.19093660801047 -0.14118819224542
2.19148746936482 -0.137225073248652
2.19202244603556 -0.133259785830464
2.19254152922092 -0.129292393560841
2.19304471037639 -0.12532296004466
2.19353198121487 -0.121351548920653
2.19400333370685 -0.117378223860378
2.19445876008054 -0.113403048567175
2.19489825282207 -0.109426086775132
2.19532180467563 -0.105447402248046
2.19572940864358 -0.101467058778382
2.19612105798664 -0.0974851201862364
2.19649674622402 -0.0935016503182911
2.19685646713352 -0.0895167130467759
2.19720021475172 -0.085530372268424
2.19752798337408 -0.0815426919034291
2.19783976755504 -0.0775537358944019
2.19813556210819 -0.0735635682053245
2.19841536210635 -0.0695722528205061
2.19867916288169 -0.0655798537435362
2.19892696002584 -0.0615864349962385
2.19915874939001 -0.0575920606176232
2.19937452708507 -0.0535967946628394
2.19957428948166 -0.0496007012021264
2.19975803321026 -0.0456038443197648
2.19992575516131 -0.0416062881130268
2.20007745248527 -0.0376080966911263
2.20021312259272 -0.0336093341741678
2.20033276315444 -0.0296100646920956
2.20043637210143 -0.0256103523836423
2.20052394762506 -0.0216102613952762
2.20059548817708 -0.0176098558801493
2.2006509924697 -0.013609199997044
2.20069045947563 -0.00960835790932009
2.20071388842815 -0.00560739378386077
2.20072127882118 -0.00160637179001872
2.20071263040927 0.00239464390143836
2.20068794320769 0.00639558911938245
2.20064721749244 0.0103963996933802
2.20059045380029 0.0143970114547484
2.20051765292882 0.0183973602376093
2.20042881593644 0.0223973818799466
2.20032394414239 0.0263970122246612
2.2002030391268 0.0303961871206279
2.20006610273067 0.0343948424237513
2.19991313705588 0.0383929139980226
2.19974414446521 0.0423903377165761
2.19955912758234 0.0463870494627462
2.19935808929184 0.0503829851311242
2.19914103273917 0.0543780806286151
2.19890796133065 0.0583722718754949
2.19865887873348 0.0623654948064673
2.19839378887571 0.066357685371721
2.19811269594617 0.0703487795379865
2.19781560439452 0.0743387132895935
2.19750251893117 0.0783274226295273
2.19717344452726 0.0823148435804859
2.19682838641462 0.0863009121859367
2.19646735008571 0.0902855645111734
2.19609034129362 0.094268736644372
2.19569736605195 0.0982503646976475
2.19528843063482 0.10223038480811
2.19486354157677 0.10620873313892
2.19442270567272 0.110185345880345
2.19396592997788 0.114160159250815
2.1934932218077 0.118133109497976
2.19300458873777 0.122104132899746
2.19250003860378 0.126073165765372
2.19197957950139 0.130040144436478
2.19144321978616 0.134005005288125
2.19089096807349 0.137967684729861
2.19032283323845 0.141928119206776
2.18973882441576 0.145886245200551
2.18913895099963 0.149841999230516
2.18852322264368 0.153795317854695
2.18789164926078 0.157746137670862
2.18724424102303 0.16169439531759
2.18658100836152 0.165640027475299
2.18590196196629 0.169582970867311
2.18520711278616 0.173523162260891
2.18449647202859 0.177460538468302
2.18377005115959 0.181395036347849
2.18302786190352 0.185326592804929
2.18226991624295 0.189255144793071
2.18149622641856 0.19318062931499
2.18070680492891 0.197102983423624
2.17990166453035 0.201022144223183
2.17908081823678 0.204938048870191
2.17824427931956 0.208850634574526
2.17739206130727 0.212759838600465
2.17652417798557 0.216665598267722
2.17564064339701 0.22056785095249
2.17474147184083 0.224466534088479
2.17382667787279 0.228361585167952
2.17289627630493 0.232252941742764
2.17195028220544 0.2361405414254
2.17098871089838 0.240024321890006
2.17001157796353 0.243904220873424
2.16901889923615 0.247780176176226
2.16801069080673 0.251652125663747
2.16698696902084 0.255520007267112
2.16594775047883 0.259383758984268
2.16489305203563 0.263243318881013
2.16382289080053 0.267098625092021
2.16273728413689 0.270949615821871
2.16163624966195 0.27479622934607
2.16051980524652 0.278638404012075
2.15938796901477 0.282476078240318
2.15824075934394 0.286309190525227
2.15707819486411 0.290137679436243
2.15590029445787 0.293961483618841
2.1547070772601 0.297780541795545
2.15349856265768 0.301594792766945
2.15227477028916 0.30540417541271
2.15103572004454 0.3092086286926
2.14978143206492 0.313008091647481
2.14851192674223 0.316802503400329
2.1472272247189 0.320591803157242
2.14592734688758 0.324375930208444
2.1446123143908 0.328154823929294
2.14328214862068 0.331928423781283
2.14193687121855 0.335696669313044
2.14057650407469 0.339459500161343
2.13920106932795 0.343216856052084
2.13781058936542 0.346968676801304
2.13640508682208 0.350714902316167
2.13498458458047 0.354455472595956
2.13354910577032 0.358190327733068
2.13209867376818 0.361919407914
2.13063331219706 0.365642653420341
2.12915304492608 0.369360004629752
2.12765789607008 0.373071402016954
2.12614788998921 0.376776786154711
2.12462305128859 0.380476097714807
2.1230834048179 0.384169277469026
2.12152897567095 0.387856266290126
2.11995978918535 0.391537005152818
2.11837587094203 0.395211435134733
2.11677724676484 0.398879497417395
2.11516394272018 0.402541133287187
2.11353598511653 0.40619628413632
2.11189340050401 0.409844891463791
2.11023621567398 0.413486896876353
2.10856445765859 0.417122242089465
2.10687815373031 0.420750868928257
2.1051773314015 0.42437271932848
2.10346201842396 0.427987735337459
2.10173224278843 0.431595859115046
2.09998803272415 0.435197032934566
2.09822941669839 0.438791199183763
2.09645642341594 0.442378300365743
2.09466908181864 0.445958279099914
2.0928674210849 0.449531078122929
2.09105147062919 0.453096640289613
2.08922126010151 0.456654908573907
2.08737681938695 0.46020582606979
2.08551817860508 0.463749335992212
2.08364536810952 0.467285381678023
2.08175841848735 0.470813906586886
2.0798573605586 0.474334854302209
2.0779422253757 0.477848168532056
2.07601304422296 0.481353793110066
2.07406984861599 0.484851671996361
2.07211267030116 0.488341749278462
2.07014154125501 0.491823969172193
2.06815649368372 0.495298276022585
2.0661575600225 0.498764614304779
2.06414477293503 0.502222928624923
2.06211816531287 0.505673163721072
2.06007777027484 0.509115264464077
2.05802362116644 0.512549175858474
2.05595575155927 0.515974843043379
2.05387419525034 0.519392211293364
2.05177898626154 0.522801226019342
2.04967015883895 0.526201832769447
2.04754774745223 0.529593977229905
2.04541178679399 0.53297760522591
2.04326231177914 0.536352662722491
2.04109935754421 0.539719095825381
2.03892295944674 0.543076850781877
2.03673315306459 0.546425873981699
2.03452997419526 0.549766111957851
2.03231345885522 0.553097511387471
2.03008364327926 0.556420019092681
2.02784056391972 0.559733582041437
2.02558425744589 0.563038147348368
2.02331476074323 0.566333662275622
2.0210321109127 0.569620074233697
2.01873634527 0.572897330782278
2.01642750134493 0.576165379631068
2.01410561688055 0.579424168640612
2.01177072983253 0.582673645823122
2.00942287836837 0.585913759343296
2.00706210086665 0.589144457519137
2.00468843591627 0.59236568882276
2.00230192231571 0.595577401881208
1.9999025990722 0.598779545477255
1.99749050540102 0.601972068550204
1.99506568072467 0.605154920196692
1.99262816467205 0.608328049671483
1.99017799707775 0.611491406388254
1.98771521798113 0.61464493992039
1.98523986762561 0.617788600001762
1.98275198645779 0.620922336527509
1.98025161512663 0.624046099554815
1.97773879448264 0.627159839303679
1.975213565577 0.630263506157685
1.97267596966076 0.633357050664766
1.97012604818395 0.636440423537966
1.96756384279472 0.639513575656193
1.96498939533849 0.642576458064977
1.96240274785704 0.645629021977213
1.95980394258767 0.648671218773911
1.95719302196227 0.651703000004934
1.95457002860643 0.654724317389734
1.95193500533856 0.657735122818088
1.94928799516896 0.660735368350821
1.94662904129886 0.663725006220537
1.94395818711957 0.666703988832333
1.94127547621149 0.66967226876452
1.93858095234318 0.67262979876933
1.93587465947041 0.675576531773628
1.93315664173521 0.678512420879612
1.9304269434649 0.681437419365512
1.92768560917113 0.684351480686287
1.92493268354887 0.687254558474314
1.92216821147546 0.690146606540074
1.91939223800959 0.693027578872832
1.91660480839031 0.695897429641315
1.91380596803602 0.698756113194386
1.91099576254344 0.70160358406171
1.90817423768661 0.704439796954417
1.90534143941586 0.707264706765764
1.90249741385671 0.710078268571788
1.89964220730892 0.712880437631951
1.89677586624535 0.715671169389795
1.89389843731094 0.718450419473573
1.89100996732165 0.721218143696891
1.88811050326334 0.723974298059335
1.88520009229074 0.7267188387471
1.88227878172629 0.729451722133612
1.87934661905911 0.732172904780143
1.87640365194383 0.734882343436421
1.87344992819955 0.737579995041243
1.87048549580861 0.740265816723072
1.86751040291556 0.742939765800637
1.86452469782597 0.745601799783524
1.86152842900529 0.748251876372763
1.85852164507769 0.750889953461413
1.85550439482493 0.753515989135137
1.85247672718514 0.756129941672776
1.84943869125168 0.758731769546913
1.84639033627195 0.761321431424441
1.84333171164617 0.763898886167114
1.84026286692621 0.766464092832102
1.83718385181437 0.769017010672536
1.83409471616216 0.771557599138053
1.83099550996906 0.774085817875328
1.82788628338132 0.776601626728608
1.82476708669071 0.779104985740234
1.82163797033324 0.781595855151168
1.81849898488794 0.784074195401499
1.81535018107559 0.786539967130958
1.81219160975743 0.788993131179424
1.80902332193389 0.791433648587416
1.8058453687433 0.793861480596593
1.80265780146061 0.796276588650239
1.79946067149605 0.798678934393742
1.79625403039386 0.801068479675076
1.79303792983092 0.803445186545268
1.78981242161548 0.805809017258865
1.78657755768577 0.808159934274394
1.78333339010871 0.810497900254815
1.78007997107849 0.812822878067971
1.77681735291527 0.815134830787029
1.77354558806378 0.817433721690915
1.77026472909197 0.819719514264751
1.76697482868959 0.821992172200274
1.7636759396668 0.824251659396256
1.76036811495283 0.826497939958922
1.75705140759449 0.828730978202354
1.75372587075479 0.830950738648892
1.75039155771154 0.83315718602953
1.74704852185587 0.83535028528431
1.74369681669081 0.837530001562696
1.74033649582985 0.839696300223961
1.73696761299547 0.841849146837554
1.73359022201769 0.843988507183464
1.73020437683257 0.846114347252582
1.72681013148075 0.848226633247051
1.72340754010596 0.850325331580617
1.7199966569535 0.852410408878962
1.71657753636877 0.854481831980047
1.71315023279571 0.856539567934432
1.70971480077534 0.858583584005603
1.70627129494416 0.860613847670283
1.70281977003266 0.862630326618742
1.69936028086376 0.864632988755102
1.69589288235124 0.866621802197627
1.6924176294982 0.868596735279018
1.68893457739548 0.870557756546692
1.68544378122006 0.87250483476306
1.68194529623351 0.874437938905793
1.67843917778037 0.876357038168092
1.67492548128654 0.878262101958938
1.6714042622577 0.880153099903341
1.66787557627764 0.882030001842588
1.66433947900668 0.883892777834477
1.66079602618 0.885741398153542
1.65724527360601 0.887575833291283
1.65368727716469 0.889396053956374
1.65012209280593 0.891202031074878
1.64654977654788 0.892993735790447
1.64297038447521 0.894771139464511
1.63938397273752 0.896534213676477
1.63579059754754 0.898282930223901
1.63219031517951 0.900017261122665
1.62858318196744 0.901737178607142
1.62496925430336 0.903442655130359
1.62134858863565 0.905133663364148
1.61772124146726 0.906810176199291
1.61408726935397 0.908472166745655
1.61044672890266 0.91011960833233
1.60679967676954 0.911752474507747
1.60314616965837 0.913370739039796
1.59948626431868 0.914974375915935
1.59582001754401 0.916563359343292
1.59214748617009 0.91813766374876
1.58846872707305 0.919697263779079
1.58478379716761 0.921242134300923
1.58109275340526 0.922772250400966
1.57739565277241 0.924287587385949
1.57369255228861 0.925788120782735
1.56998350900463 0.92727382633836
1.56626858000069 0.928744680020072
1.56254782238453 0.930200658015367
1.55882129328957 0.931641736732015
1.55508904987306 0.933067892798077
1.55135114931413 0.934479103061917
1.54760764881196 0.935875344592204
1.54385860558385 0.937256594677907
1.54010407686328 0.938622830828283
1.53634411989807 0.939974030772855
1.53257879194835 0.941310172461383
1.52880815028473 0.942631234063829
1.52503225218627 0.943937193970311
1.52125115493858 0.945228030791051
1.51746491583183 0.946503723356313
1.5136735921588 0.947764250716336
1.50987724121289 0.949009592141254
1.50607592028615 0.950239727121014
1.50226968666726 0.95145463536528
1.49845859763956 0.952654296803331
1.494642710479 0.953838691583956
1.49082208245216 0.955007800075327
1.48699677081421 0.956161602864881
1.48316683280683 0.95730008075918
1.47933232565626 0.958423214783767
1.47549330657115 0.959530986183019
1.47164983274054 0.960623376419979
1.46780196133181 0.961700367176195
1.46394974948857 0.962761940351538
1.46009325432859 0.963808078064014
1.45623253294168 0.964838762649576
1.45236764238763 0.965853976661913
1.44849863969406 0.966853702872243
1.44462558185431 0.967837924269091
1.44074852582533 0.968806624058057
1.4368675285255 0.96975978566158
1.43298264683254 0.970697392718692
1.42909393758131 0.971619429084759
1.42520145756165 0.972525878831214
1.42130526351626 0.973416726245292
1.41740541213843 0.974291955829734
1.41350196006996 0.975151552302508
1.40959496389886 0.975995500596498
1.40568448015721 0.976823785859198
1.40177056531893 0.977636393452394
1.39785327579755 0.978433308951831
1.39393266794398 0.979214518146882
1.39000879804428 0.979980007040194
1.3860817223174 0.980729761847335
1.38215149691295 0.981463768996431
1.37821817790888 0.982182015127787
1.37428182130927 0.982884487093505
1.37034248304199 0.98357117195709
1.36640021895647 0.984242056993045
1.36245508482134 0.984897129686463
1.35850713632216 0.9855363777326
1.3545564290591 0.986159789036446
1.35060301854461 0.986767351712282
1.34664696020105 0.987359054083234
1.34268830935842 0.987934884680803
1.33872712125195 0.988494832244405
1.33476345101977 0.989038885720884
1.33079735370052 0.989567034264026
1.326828884231 0.990079267234055
1.32285809744376 0.990575574197129
1.3188850480647 0.991055944924819
1.31490979071071 0.991520369393574
1.31093237988721 0.991968837784192
1.30695286998575 0.99240134048126
1.30297131528159 0.992817868072602
1.29898776993123 0.993218411348709
1.29500228797001 0.993602961302155
1.29101492330962 0.993971509127011
1.28702572973563 0.994324046218245
1.28303476090506 0.994660564171111
1.27904207034385 0.994981054780529
1.27504771144439 0.995285510040454
1.27105173746304 0.995573922143234
1.26705420151758 0.995846283478962
1.26305515658472 0.99610258663481
1.2590546554976 0.996342824394361
1.25505275094318 0.996566989736922
1.25104949545977 0.996775075836835
1.24704494143446 0.996967076062769
1.24303914110052 0.997142983977009
1.23903214653488 0.99730279333473
1.23502400965554 0.997446498083259
1.23101478221894 0.997574092361334
1.22700451581742 0.997685570498339
1.22299326187661 0.997780927013543
1.21898107165275 0.997860156615319
1.21496799623016 0.997923254200352
1.21095408651854 0.997970214852842
1.20693939325035 0.99800103384369
1.20292396697816 0.998015706629677
1.19890785807202 0.998014228852628
1.19489539932959 0.99800213818648
1.19088363159804 0.997973904517783
1.18687260554264 0.997929532329984
1.18286237200777 0.997869026292535
1.17885298201427 0.997792391261716
1.17484448675681 0.99769963228145
1.17083693760122 0.997590754584113
1.16683038608189 0.997465763591323
1.16282488389917 0.997324664914724
1.15882048291669 0.997167464356762
1.15481723515886 0.996994167911441
1.1508151928082 0.996804781765077
1.1468144082028 0.996599312297036
1.14281493383375 0.996377766080464
1.13881682234255 0.996140149883005
1.13482012651861 0.995886470667505
1.13082489929666 0.995616735592716
1.12683119375425 0.995330952013975
1.12283906310918 0.995029127483885
1.11884856071704 0.994711269752977
1.11485974006871 0.994377386770366
1.11087265478779 0.994027486684394
1.10688735862819 0.993661577843266
1.10290390547162 0.993279668795671
1.09892234932513 0.992881768291397
1.09494274431865 0.992467885281931
1.09096514470253 0.992038028921056
1.08698960484513 0.991592208565428
1.08301617923036 0.991130433775153
1.07904492245526 0.990652714314343
1.07507588922762 0.990159060151674
1.07110913436352 0.989649481460924
1.067144712785 0.989123988621506
1.06318267951763 0.988582592218989
1.05922308968815 0.988025303045608
1.05526599852211 0.98745213210077
1.0513114613415 0.986863090591542
1.0473595335624 0.986258189933134
1.04341027069267 0.985637441749371
1.03946372832957 0.985000857873157
1.03551996215747 0.984348450346925
1.03157902794557 0.983680231423081
1.02764098154552 0.982996213564438
1.02370587888918 0.982296409444638
1.01977377598633 0.981580831948567
1.01584472892238 0.980849494172757
1.01191879385611 0.980102409425785
1.00799602701742 0.979339591228656
1.00407648470507 0.978561053315177
1.00016022328443 0.977766809632326
0.996247299185277 0.976956874340609
0.992337768899547 0.976131261814407
0.988431688979137 0.975289986642312
0.984529116033696 0.974433063627461
0.980630106728428 0.973560507787852
0.976734717781906 0.972672334356655
0.972843005963895 0.971768558782515
0.968955028093183 0.970849196729842
0.965070841035421 0.969914264079099
0.96119050170097 0.968963776927069
0.95731406704276 0.967997751587126
0.953441594054156 0.967016204589488
0.949573139766834 0.966019152681465
0.945708761248663 0.965006612827698
0.941848515601599 0.963978602210385
0.937992459959585 0.962935138229507
0.934140651486463 0.961876238503037
0.93029314737389 0.96080192086714
0.926450004839269 0.95971220337637
0.922611281123682 0.958607104303857
0.918777033489834 0.95748664214148
0.914947319220011 0.956350835600038
0.911122195614039 0.955199703609407
0.907301719987256 0.954033265318694
0.903485949668487 0.952851540096377
0.899674941998039 0.951654547530439
0.895868754325691 0.950442307428495
0.892067444008706 0.949214839817907
0.888271068409836 0.947972164945895
0.884479684895351 0.946714303279633
0.880693350833069 0.945441275506347
0.876912123590389 0.944153102533395
0.873136060532349 0.942849805488341
0.869365219019674 0.941531405719025
0.865599656406842 0.940197924793621
0.861839430040161 0.938849384500688
0.858084597255847 0.937485806849211
0.854335215378116 0.936107214068639
0.850591341717281 0.93471362860891
0.84685303356786 0.933305073140466
0.84312034820669 0.931881570554272
0.839393342891051 0.930443143961809
0.8356720748568 0.928989816695075
0.831956601316508 0.92752161230657
0.828246979457611 0.926038554569278
0.824543266440568 0.92454066747663
0.820845519397022 0.923027975242479
0.817153795427978 0.921500502301045
0.813468151601982 0.919958273306869
0.809788644953314 0.918401313134754
0.806115332480182 0.916829646879695
0.802448271142935 0.915243299856804
0.798787517862271 0.913642297601229
0.795133129517464 0.912026665868065
0.791485162944595 0.910396430632253
0.787843674934791 0.908751618088479
0.784208722232473 0.907092254651059
0.78058036153361 0.905418366953821
0.776958649483983 0.903729981849978
0.773343642677462 0.90202712641199
0.769735397654279 0.900309827931427
0.766133970899321 0.898578113918818
0.762539418840423 0.896832012103494
0.758951797846677 0.895071550433425
0.75537116422674 0.893296757075052
0.751797574227156 0.891507660413106
0.748231084030684 0.889704289050424
0.744671749754638 0.887886671807758
0.741119627449224 0.886054837723576
0.737574773095902 0.884208816053856
0.734037242605738 0.882348636271871
0.730507091817778 0.880474328067976
0.726984376497421 0.878585921349371
0.723469152334806 0.876683446239878
0.719961474943202 0.874766933079693
0.716461399857409 0.872836412425144
0.712968982532167 0.870891915048433
0.709484278340569 0.868933471937378
0.70600734257249 0.866961114295146
0.702538230433011 0.864974873539976
0.699076997040866 0.862974781304904
0.695623697426886 0.860960869437468
0.692178386532452 0.858933169999421
0.68874111920796 0.856891715266429
0.685311950211294 0.854836537727762
0.6818909342063 0.852767670085983
0.678478125761275 0.850685145256627
0.67507357934746 0.848588996367876
0.67167734933754 0.846479256760226
0.668289490004158 0.844355959986147
0.664910055518426 0.842219139809742
0.661539099948455 0.84006883020639
0.658176677257885 0.837905065362392
0.654822841304425 0.835727879674606
0.651477645838399 0.833537307750078
0.648141144501308 0.831333384405666
0.644813390824384 0.829116144667655
0.641494438227169 0.82688562377137
0.638184340016084 0.824641857160785
0.634883149383025 0.822384880488118
0.631590919403949 0.820114729613426
0.628307703037476 0.817831440604194
0.625033553123501 0.815535049734913
0.621768522381807 0.813225593486663
0.618512663410688 0.810903108546676
0.615266028685583 0.808567631807902
0.612028670557711 0.806219200368568
0.608800641252718 0.803857851531732
0.605581992869334 0.801483622804828
0.602372777378029 0.799096551899207
0.599173046619681 0.796696676729673
0.595982852304258 0.794284035414014
0.592802246009496 0.791858666272521
0.58963127917959 0.789420607827517
0.586470003123894 0.786969898802858
0.583318469015623 0.78450657812345
0.580176727890569 0.782030684914744
0.577044830645822 0.77954225850224
0.573922828038492 0.777041338410972
0.57081077068445 0.774527964364995
0.567708709057066 0.772002176286868
0.56461669348596 0.769464014297127
0.561534774155757 0.766913518713756
0.558463001104854 0.76435073005165
0.555401424224189 0.761775689022076
0.552350093256019 0.759188436532125
0.549309057792707 0.756589013684163
0.546278367275515 0.753977461775274
0.543258070993406 0.751353822296698
0.540248218081844 0.748718136933266
0.537248857521619 0.746070447562826
0.534260038137659 0.743410796255668
0.531281808597867 0.740739225273945
0.528314217411952 0.738055777071081
0.525357312930274 0.735360494291184
0.522411143342692 0.732653419768447
0.519475756677428 0.729934596526548
0.516551200799924 0.727204067778045
0.513637523411718 0.724461876923762
0.510734772049321 0.721708067552175
0.507842994083104 0.718942683438793
0.50496223671619 0.71616576854553
0.502092546983356 0.713377367020076
0.499233971749936 0.710577523195265
0.496386557710741 0.707766281588431
0.493550351388973 0.704943686900768
0.490725399135158 0.702109784016679
0.487911747126081 0.699264618003127
0.485109441363725 0.696408234108971
0.482318527674221 0.693540677764311
0.479539051706809 0.690661994579815
0.476771058932791 0.687772230346054
0.474014594644511 0.684871431032822
0.471269703954325 0.681959642788458
0.468536431793586 0.679036911939163
0.465814822911638 0.676103284988312
0.463104921874809 0.673158808615756
0.460406773065418 0.670203529677133
0.457720420680785 0.66723749520316
0.455045908732249 0.664260752398932
0.452383281044196 0.661273348643208
0.449732581253084 0.658275331487703
0.447093852806488 0.655266748656365
0.444467138962143 0.652247648044657
0.441852482786995 0.649218077718825
0.439249927156261 0.646178085915176
0.436659514752497 0.643127721039337
0.434081288064663 0.640067031665522
0.431515289387212 0.636996066535784
0.428961560819167 0.633914874559275
0.426420144263222 0.630823504811495
0.423891081424834 0.627722006533533
0.421374413811332 0.624610429131315
0.418870182731031 0.621488822174843
0.41637842929235 0.618357235397424
0.413899194402937 0.615215718694908
0.411432518768803 0.612064322124908
0.408978442893463 0.608903095906032
0.406537007077079 0.605732090417094
0.404108251415613 0.602551356196338
0.401692215799987 0.599360943940645
0.399288939915252 0.596160904504745
0.396898463239752 0.592951288900422
0.394520825044311 0.589732148295715
0.392156064391415 0.586503534014117
0.389804220134404 0.583265497533772
0.387465330916673 0.580018090486659
0.385139435170872 0.576761364657792
0.382826571118122 0.573495371984392
0.380526776767232 0.570220164555078
0.378240089913925 0.566935794609039
0.375966548140063 0.563642314535212
0.373706188812894 0.560339776871449
0.371459049084286 0.55702823430369
0.369225165889986 0.553707739665126
0.367004575948871 0.550378345935356
0.364797315762212 0.547040106239554
0.362603421612949 0.543693073847616
0.360422929564957 0.540337302173315
0.358255875462339 0.536972844773453
0.356102294928707 0.533599755347
0.353962223366482 0.530218087734244
0.35183569595619 0.526827895915924
0.349722747655778 0.523429234012373
0.347623413199918 0.520022156282646
0.345537727099336 0.516606717123654
0.343465723640134 0.513182971069292
0.341407436883123 0.509750972789562
0.339362900663166 0.506310777089695
0.337332148588518 0.502862438909273
0.335315214040182 0.499406013321342
0.333312130171264 0.495941555531527
0.331322929906339 0.492469120877142
0.32934764594082 0.4889887648263
0.327386310740338 0.485500542977016
0.325438956540117 0.482004511056309
0.323505615344373 0.478500724919304
0.321586318925699 0.47498924054833
0.319681098824474 0.471470114052011
0.317789986348267 0.467943401664361
0.315913012571248 0.464409159743871
0.314050208333612 0.460867444772598
0.312201604241002 0.457318313355248
0.310367230663942 0.453761822218259
0.308547117737273 0.450198028208881
0.306741295359596 0.446626988294249
0.304949793192727 0.443048759560461
0.303172640661147 0.43946339921165
0.301409866951466 0.435870964569052
0.29966150101189 0.432271513070075
0.297927571551698 0.428665102267362
0.296208107040718 0.425051789827856
0.294503135708813 0.42143163353186
0.292812685545377 0.417804691272094
0.291136784298827 0.414171021052751
0.289475459476108 0.410530680988551
0.287828738342206 0.406883729303796
0.286196647919659 0.403230224331412
0.284579214988081 0.399570224512004
0.282976466083688 0.395903788392896
0.281388427498829 0.392230974627178
0.279815125281528 0.388551841972743
0.278256585235029 0.384866449291329
0.276712832917343 0.381174855547556
0.275183893640805 0.377477119807957
0.273669792471639 0.373773301240018
0.272170554229523 0.370063459111202
0.270686203487162 0.366347652787983
0.269216764569871 0.362625941734873
0.267762261555157 0.358898385513443
0.266322718272308 0.355165043781354
0.264898158301996 0.35142597629137
0.263488604975871 0.347681242890387
0.262094081376175 0.343930903518443
0.260714610335353 0.340175018207741
0.259350214435673 0.33641364708166
0.258000916008849 0.332646850353769
0.256666737135675 0.328874688326839
0.255347699645656 0.32509722139185
0.254043825116655 0.321314510027005
0.252755134874538 0.317526614796729
0.251481649992822 0.313733596350679
0.250223391292345 0.309935515422746
0.248980379340917 0.306132432830056
0.247752634452998 0.302324409471971
0.246540176689371 0.298511506329089
0.24534302585682 0.294693784462236
0.244161201507819 0.29087130501147
0.242994722940222 0.28704412919507
0.241843609196961 0.283212318308527
0.240707879065747 0.279375933723542
0.239587551078779 0.275535036887013
0.238482643512456 0.271689689320021
0.237393174387098 0.267839952616824
0.236319161466667 0.263985888443838
0.235260622258499 0.260127558538623
0.234217574013038 0.25626502470887
0.233190033723577 0.252398348831379
0.232178018126001 0.248527592851043
0.23118154369854 0.244652818779831
0.230200626661527 0.240774088695759
0.229235282977155 0.236891464741876
0.228285528349247 0.233005009125236
0.227351378223028 0.229114784115877
0.226432847784901 0.225220852045793
0.225529951962231 0.221323275307907
0.224642705423132 0.217422116355047
0.22377112257626 0.213517437698916
0.222915217570613 0.209609301909061
0.222075004295333 0.205697771611843
0.221250496379512 0.201782909489409
0.220441707192014 0.197864778278655
0.219648649841284 0.193943440770197
0.218871337175181 0.190018959807333
0.218109781780801 0.186091398285012
0.217363995984315 0.182160819148795
0.216633991850808 0.178227285393823
0.215919781184121 0.174290860063773
0.215221375526707 0.170351606249826
0.214538786159476 0.166409587089627
0.213872024101666 0.162464865766241
0.213221100110701 0.158517505507119
0.212586024682062 0.154567569583053
0.211966808049162 0.150615121307136
0.211363460183228 0.146660224033721
0.210775990793185 0.142702941157374
0.210204409325543 0.138743336111835
0.209648724964298 0.134781472368972
0.209108946630825 0.130817413437736
0.208585082983788 0.126851222863119
0.208077142419049 0.122882964225102
0.20758513306958 0.118912701137616
0.207109062805384 0.114940497247492
0.206648939233423 0.110966416233414
0.206204769697539 0.106990521804871
0.205776561278398 0.10301287770111
0.205364320793423 0.0990335476900899
0.204968054796738 0.0950525955674284
0.204587769579121 0.0910700851553564
0.204223471167952 0.0870860803016673
0.203875165327175 0.0831006448786678
0.203542857557261 0.0791138427821277
0.203226553095176 0.0751257379302303
0.202926256914351 0.0711363942625212
0.202641973724661 0.0671458757388582
0.202373707972412 0.0631542463383602
0.202121463840319 0.0591615700583562
0.201885245247504 0.0551679109133343
0.201665055849492 0.0511733329338897
0.201460899038208 0.0471779001656739
0.201272777941989 0.0431816766683428
0.201100695425588 0.0391847265145048
0.200944654090194 0.0351871137886695
0.200804656273448 0.0311889025861953
0.200680704049472 0.0271901570122381
0.200572799228892 0.0231909411806991
0.200480943358877 0.0191913192131729
0.200405137723174 0.015191355237896
0.200345383342152 0.0111911133886947
0.200301680972849 0.0071906578039336
0.200274031109025 0.00319005262546384
0.200262433981216 -0.000810638002428379
0.200266889556797 -0.00481134993407318
0.200287397540049 -0.00881201902346828
0.200323957372226 -0.0128125811253301
0.200376568231629 -0.0168129720961445
0.200445229033689 -0.0208131277952179
0.200529938431048 -0.0248129840857273
0.200630694813648 -0.0288124768357707
0.200747496308823 -0.0328115419194172
0.200880340781394 -0.0368101152177568
0.201029225833774 -0.0408081326199492
0.201194148806072 -0.0448055300242738
0.201375106776203 -0.0488022433391778
0.201572096560004 -0.0527982084843248
0.20178511471135 -0.056793361391643
0.202014157522281 -0.0607876380063728
0.202259221023127 -0.0647809742881136
0.202520300982643 -0.0687733062118717
0.202797392908142 -0.0727645697691052
0.203090492045638 -0.0767547009687714
0.203399593379992 -0.0807436358383711
0.20372469163506 -0.0847313104249941
0.204065781273844 -0.0887176607963635
0.204422856498658 -0.0927026230418796
0.204795911251282 -0.0966861332736633
0.205184939213133 -0.100668127627599
0.205589933805435 -0.104648542264375
0.206010888189394 -0.108627313370529
0.206447795266378 -0.112604377159484
0.206900647678099 -0.116579669872594
0.207369437806802 -0.120553127780178
0.207854157775459 -0.124524687182563
0.208354799447961 -0.128494284411123
0.208871354429321 -0.13246185582931
0.209403814065879 -0.136427337833699
0.209952169445509 -0.140390666855019
0.210516411397834 -0.14435177935919
0.211096530494442 -0.148310611848356
0.211692517049105 -0.152267100861921
0.21230436111801 -0.156221182977581
0.212932052499982 -0.160172794812353
0.213575580736722 -0.164121873023613
0.214234935113043 -0.168068354310118
0.214910104657112 -0.172012175413041
0.215601078140696 -0.175953273117
0.216307844079409 -0.179891584251081
0.217030390732973 -0.183827045689868
0.217768706105468 -0.187759594354468
0.218522777945598 -0.191689167213535
0.219292593746957 -0.195615701284294
0.220078140748299 -0.199539133633566
0.220879405933811 -0.203459401378786
0.221696376033391 -0.207376441689024
0.222529037522931 -0.211290191786008
0.223377376624603 -0.215200588945139
0.224241379307147 -0.219107570496509
0.225121031286167 -0.22301107382592
0.226016318024429 -0.226911036375895
0.226927224732161 -0.230807395646694
0.227853736367359 -0.234700089197328
0.228795837636099 -0.238589054646569
0.229753512992845 -0.242474229673961
0.230726746640775 -0.246355552020828
0.231715522532093 -0.250232959491285
0.232719824368361 -0.254106389953243
0.233739635600823 -0.257975781339412
0.234774939430742 -0.26184107164831
0.235825718809734 -0.265702198945265
0.236891956440109 -0.269559101363411
0.237973634775216 -0.273411717104697
0.239070736019792 -0.277259984440879
0.240183242130311 -0.281103841714523
0.24131113481534 -0.284943227339995
0.242454395535904 -0.288778079804462
0.243613005505843 -0.292608337668883
0.244786945692179 -0.296433939569
0.245976196815491 -0.30025482421633
0.24718073935029 -0.304070930399154
0.248400553525389 -0.307882196983503
0.249635619324294 -0.311688562914148
0.250885916485588 -0.315489967215582
0.252151424503314 -0.319286348993002
0.253432122627376 -0.323077647433292
0.254727989863932 -0.326863801806008
0.256039004975793 -0.330644751464347
0.257365146482832 -0.33442043584613
0.258706392662385 -0.338190794474779
0.260062721549668 -0.341955766960287
0.261434110938189 -0.345715293000191
0.262820538380168 -0.349469312380542
0.264221981186959 -0.353217764976878
0.265638416429474 -0.356960590755186
0.267069820938614 -0.360697729772872
0.268516171305702 -0.364429122179719
0.26997744388292 -0.368154708218856
0.271453614783747 -0.371874428227716
0.272944659883403 -0.37558822263899
0.2744505548193 -0.379296031981594
0.275971274991487 -0.382997796881612
0.277506795563108 -0.38669345806326
0.279057091460857 -0.390382956349831
0.280622137375443 -0.394066232664649
0.282201907762052 -0.397743228032013
0.283796376840812 -0.401413883578147
0.285405518597272 -0.40507814053214
0.287029306782871 -0.408735940226896
0.288667714915421 -0.412387224100062
0.290320716279583 -0.416031933694981
0.291988283927357 -0.419670010661618
0.293670390678572 -0.4233013967575
0.295367009121372 -0.426926033848647
0.297078111612718 -0.430543863910503
0.298803670278882 -0.434154829028866
0.300543657015954 -0.437758871400815
0.302298043490341 -0.441355933335634
0.304066801139284 -0.444945957255734
0.305849901171364 -0.448528885697576
0.30764731456702 -0.452104661312591
0.309459012079069 -0.455673226868092
0.311284964233226 -0.459234525248193
0.313125141328631 -0.462788499454722
0.314979513438378 -0.466335092608128
0.316848050410044 -0.469874247948393
0.318730721866231 -0.473405908835938
0.320627497205096 -0.476930018752523
0.3225383456009 -0.480446521302157
0.324463236004551 -0.483955360211991
0.326402137144147 -0.487456479333218
0.328355017525538 -0.490949822641971
0.330321845432869 -0.494435334240213
0.332302588929148 -0.49791295835663
0.334297215856801 -0.50138263934752
0.336305693838235 -0.504844321697679
0.338327990276412 -0.508297950021288
0.340364072355414 -0.51174346906279
0.342413907041016 -0.515180823697776
0.344477461081266 -0.518609958933859
0.346554701007066 -0.522030819911551
0.348645593132748 -0.525443351905135
0.350750103556667 -0.528847500323537
0.352868198161786 -0.532243210711192
0.354999842616269 -0.535630428748916
0.357145002374079 -0.53900910025476
0.35930364267557 -0.542379171184882
0.361475728548095 -0.545740587634399
0.363661224806605 -0.549093295838245
0.365860096054259 -0.552437242172026
0.368072306683033 -0.555772373152874
0.370297820874334 -0.559098635440291
0.372536602599614 -0.562415975836999
0.374788615620993 -0.565724341289785
0.377053823491876 -0.569023678890342
0.379332189557582 -0.572313935876108
0.381623676955972 -0.575595059631106
0.383928248618074 -0.578866997686772
0.386245867268726 -0.582129697722794
0.388576495427202 -0.585383107567938
0.39092009540786 -0.588627175200873
0.393276629320781 -0.591861848751001
0.395646059072412 -0.595087076499274
0.398028346366216 -0.598302806879013
0.400423452703326 -0.601508988476728
0.40283133938319 -0.604705570032931
0.405251967504238 -0.607892500442945
0.407685297964531 -0.611069728757713
0.41013129146243 -0.614237204184609
0.412589908497259 -0.617394876088237
0.415061109369968 -0.620542693991231
0.417544854183811 -0.623680607575056
0.420041102845012 -0.626808566680804
0.422549815063443 -0.629926521309984
0.425070950353305 -0.633034421625313
0.427604468033803 -0.636132217951506
0.430150327229835 -0.639219860776057
0.432708486872678 -0.642297300750022
0.43527890570067 -0.645364488688802
0.437861542259912 -0.648421375572913
0.440456354904955 -0.651467912548764
0.443063301799498 -0.654504050929427
0.445682340917089 -0.657529742195407
};
%\addlegendentry{SUR$_1$}
\addplot [line width = \lineWidthSUR, color = SUR2, densely dotted]
table {%
0.2 0
0.200001186061397 -0.00400071280052999
0.200018422054509 -0.00800126762160506
0.20005170642087 -0.012001600331461
0.200101037339546 -0.0160016468061935
0.200166412727245 -0.0200013429308064
0.200247830238444 -0.0240006246002609
0.200345287265503 -0.0279994277205228
0.200458780938801 -0.03199768820961
0.200588308126864 -0.0359953419986404
0.200733865436499 -0.0399923250328774
0.20089544921294 -0.0439885732727763
0.201073055539991 -0.0479840226950301
0.201266680240175 -0.0519786092936145
0.201476318874886 -0.055972269080832
0.201701966744553 -0.0599649380883559
0.201943618888797 -0.0639565523682739
0.2022012700866 -0.0679470479941301
0.202474914856477 -0.0719363610619677
0.202764547456649 -0.0759244276913702
0.203070161885223 -0.0799111840265025
0.203391751880381 -0.0838965662371498
0.20372931092056 -0.0878805105197583
0.204082832224651 -0.091862953098473
0.204452308752193 -0.0958438302261767
0.204837733203573 -0.0998230781855255
0.205239098020233 -0.103800633289987
0.205656395384879 -0.107776431884874
0.206089617221692 -0.111750410348381
0.20653875519655 -0.115722505092617
0.207003800717246 -0.119692652564642
0.207484744933714 -0.123660789247492
0.207981578738262 -0.127626851661218
0.208494292765805 -0.131590776363913
0.209022877394099 -0.135552499952743
0.20956732274399 -0.139511959064972
0.210127618679657 -0.143469090378996
0.210703754808862 -0.147423830615365
0.211295720483207 -0.151376116537811
0.211903504798389 -0.155325884954271
0.212527096594467 -0.159273072717914
0.213166484456125 -0.163217616728159
0.213821656712943 -0.167159453931704
0.214492601439676 -0.171098521323538
0.215179306456527 -0.175034755947968
0.215881759329434 -0.178968094899634
0.216599947370354 -0.182898475324526
0.217333857637555 -0.186825834421005
0.218083476935908 -0.190750109440811
0.218848791817192 -0.194671237690081
0.219629788580387 -0.198589156530364
0.22042645327199 -0.202503803379629
0.221238771686316 -0.206415115713277
0.222066729365818 -0.210323031065149
0.222910311601407 -0.214227487028536
0.223769503432767 -0.218128421257186
0.224644289648684 -0.222025771466306
0.225534654787376 -0.22591947543357
0.226440583136829 -0.229809471000121
0.227362058735127 -0.233695696071571
0.2282990653708 -0.237578088619003
0.229251586583165 -0.241456586679968
0.230219605662675 -0.245331128359486
0.231203105651273 -0.249201651831038
0.232202069342746 -0.253068095337564
0.233216479283088 -0.256930397192455
0.234246317770859 -0.260788495780544
0.235291566857555 -0.2646423295591
0.23635220834798 -0.268491837058811
0.237428223800619 -0.272336956884778
0.238519594528016 -0.276177627717499
0.239626301597158 -0.280013788313849
0.24074832582986 -0.283845377508071
0.241885647803153 -0.287672334212749
0.243038247849679 -0.291494597419798
0.244206106058085 -0.295312106201432
0.245389202273429 -0.299124799711149
0.246587516097576 -0.302932617184699
0.247801026889612 -0.306735497941066
0.24902971376625 -0.310533381383431
0.250273555602252 -0.31432620700015
0.251532531030839 -0.31811391436572
0.252806618444116 -0.321896443141742
0.254095795993499 -0.325673733077894
0.255400041590143 -0.329445724012887
0.256719332905373 -0.333212355875434
0.25805364737112 -0.336973568685203
0.259402962180361 -0.340729302553783
0.260767254287565 -0.344479497685635
0.262146500409133 -0.348224094379048
0.26354067702385 -0.351963033027095
0.264949760373341 -0.355696254118583
0.266373726462527 -0.359423698238999
0.267812551060081 -0.363145306071463
0.269266209698896 -0.36686101839767
0.27073467767655 -0.370570776098838
0.272217930055776 -0.374274520156645
0.273715941664935 -0.377972191654175
0.275228687098497 -0.381663731776852
0.276756140717515 -0.385349081813377
0.278298276650115 -0.389028183156664
0.279855068791978 -0.392700977304772
0.281426490806833 -0.396367405861837
0.28301251612695 -0.400027410538997
0.284613117953636 -0.403680933155324
0.28622826925774 -0.407327915638743
0.287857942780147 -0.410968300026959
0.289502111032296 -0.414602028468375
0.291160746296681 -0.418229043223013
0.292833820627367 -0.421849286663428
0.294521305850512 -0.425462701275626
0.296223173564875 -0.429069229659972
0.297939395142352 -0.432668814532104
0.299669941728491 -0.436261398723839
0.301414784243029 -0.439846925184082
0.303173893380419 -0.443425336979725
0.30494723961037 -0.446996577296555
0.306734793178383 -0.450560589440149
0.308536524106292 -0.454117316836775
0.310352402192815 -0.457666703034282
0.312182397014091 -0.461208691703002
0.314026477924242 -0.464743226636631
0.315884614055922 -0.468270251753126
0.317756774320876 -0.471789711095586
0.319642927410498 -0.47530154883314
0.321543041796397 -0.478805709261825
0.323457085730962 -0.482302136805471
0.325385027247935 -0.485790776016574
0.327326834162975 -0.489271571577174
0.329282474074245 -0.492744468299724
0.331251914362979 -0.496209411127967
0.333235122194075 -0.499666345137798
0.335232064516669 -0.503115215538132
0.337242708064727 -0.50655596767177
0.339267019357641 -0.509988547016257
0.34130496470081 -0.513412899184741
0.343356510186246 -0.516828969926829
0.34542162169317 -0.520236705129447
0.347500264888613 -0.523636050817684
0.349592405228026 -0.527026953155642
0.351698007955877 -0.530409358447288
0.353817038106273 -0.533783213137288
0.355949460503571 -0.537148463811868
0.358095239762987 -0.540505057199627
0.360254340291224 -0.543852940172399
0.362426726287088 -0.547192059746068
0.364612361742119 -0.550522363081414
0.366811210441215 -0.553843797484931
0.369023235963259 -0.55715631040966
0.371248401681759 -0.560459849456014
0.373486670765477 -0.563754362372596
0.375738006179075 -0.56703979705702
0.378002370683751 -0.570316101556726
0.380279726837884 -0.5735832240698
0.382570036997682 -0.576841112945773
0.384873263317835 -0.580089716686443
0.387189367752157 -0.583328983946673
0.38951831205425 -0.586558863535196
0.391860057778164 -0.589779304415419
0.394214566279046 -0.592990255706214
0.39658179871381 -0.596191666682721
0.398961716041805 -0.599383486777138
0.401354279025479 -0.60256566557951
0.403759448231048 -0.605738152838517
0.406177184029174 -0.608900898462258
0.408607446595639 -0.612053852519035
0.411050195912024 -0.615196965238131
0.413505391766391 -0.618330187010586
0.41597299375396 -0.621453468389972
0.418452961277804 -0.624566760093161
0.420945253549534 -0.627670013001097
0.423449829589992 -0.63076317815956
0.425966648229941 -0.633846206779928
0.428495668110764 -0.636919050239935
0.431036847685162 -0.639981660084434
0.433590145217856 -0.643033988026141
0.436155518786291 -0.646075985946398
0.438732926281336 -0.649107605895916
0.441322325408002 -0.652128800095516
0.44392367368615 -0.655139520936881
0.446536928451199 -0.658139720983288
0.449162046854847 -0.661129352970348
0.45179898586579 -0.664108369806744
0.454447702270437 -0.667076724574952
0.457108152673637 -0.67003437053198
0.459780293499404 -0.67298126111009
0.462464080991646 -0.675917349917513
0.46515947121489 -0.678842590739181
0.467866420055021 -0.681756937537435
0.47058488322001 -0.68466034445274
0.473314816240655 -0.687552765804396
0.476056174471317 -0.690434156091247
0.478808913090665 -0.693304469992384
0.481572987102419 -0.696163662367846
0.484348351336093 -0.69901168825932
0.487134960447748 -0.701848502890838
0.489932768920735 -0.704674061669465
0.492741731066457 -0.707488320185988
0.495561801025116 -0.710291234215611
0.498392932766474 -0.713082759718631
0.501235080090611 -0.715862852841118
0.504088196628685 -0.718631469915597
0.506952235843703 -0.721388567461716
0.509827151031273 -0.724134102186926
0.512712895320385 -0.726868030987136
0.515609421674171 -0.72959031094739
0.51851668289069 -0.732300899342521
0.521434631603682 -0.734999753637816
0.524363220283365 -0.737686831489662
0.5273024012372 -0.740362090746212
0.53025212661068 -0.743025489448021
0.533212348388101 -0.745676985828698
0.53618301839336 -0.748316538315548
0.539164088290733 -0.75094410553021
0.542155509585666 -0.753559646289301
0.545157233625564 -0.756163119605035
0.548169211600586 -0.758754484685869
0.551191394544437 -0.761333700937118
0.554223733335165 -0.763900727961585
0.557266178695962 -0.766455525560176
0.560318681195962 -0.768998053732523
0.563381191251045 -0.771528272677595
0.566453659124639 -0.774046142794309
0.569536034928525 -0.776551624682134
0.572628268623653 -0.779044679141704
0.575730310020941 -0.78152526717541
0.578842108782094 -0.783993349988002
0.581963614420418 -0.78644888898718
0.585094776301633 -0.788891845784188
0.588235543644687 -0.791322182194401
0.591385865522587 -0.793739860237911
0.594545690863203 -0.796144842140102
0.597714968450107 -0.798537090332237
0.600893646923391 -0.800916567452022
0.604081674780486 -0.803283236344185
0.607279000377011 -0.805637060061041
0.610485571927577 -0.807978001863056
0.613701337506638 -0.810306025219405
0.616926245049316 -0.812621093808539
0.620160242352242 -0.814923171518727
0.623403277074383 -0.817212222448618
0.626655296737893 -0.819488210907778
0.629916248728943 -0.821751101417241
0.633186080298573 -0.824000858710051
0.636464738563524 -0.826237447731793
0.639752170507092 -0.828460833641127
0.643048322979976 -0.830670981810325
0.646353142701117 -0.832867857825795
0.649666576258561 -0.8350514274886
0.6529885701103 -0.837221656814987
0.656319070585134 -0.839378512036891
0.659658023883517 -0.841521959602459
0.663005376078425 -0.843651966176556
0.666361073116204 -0.845768498641273
0.669725060817436 -0.847871524096423
0.673097284877798 -0.849961009860046
0.676477690868925 -0.852036923468908
0.679866224239277 -0.854099232678986
0.683262830314999 -0.856147905465959
0.686667454300796 -0.858182910025694
0.690080041280795 -0.860204214774728
0.693500536219422 -0.862211788350745
0.696928883962267 -0.864205599613053
0.700365029236963 -0.866185617643053
0.703808916654057 -0.868151811744703
0.707260490707891 -0.870104151444993
0.710719695777471 -0.872042606494394
0.714186476127356 -0.873967146867318
0.717660775908532 -0.875877742762578
0.721142539159297 -0.877774364603826
0.724631709806141 -0.87965698304001
0.728128231664632 -0.881525568945813
0.731632048440305 -0.883380093422086
0.735143103729544 -0.885220527796294
0.738661341020475 -0.887046843622942
0.742186703693853 -0.888859012683997
0.745719135023955 -0.890657006989331
0.749258578179467 -0.892440798777123
0.75280497622439 -0.894210360514281
0.756358272118923 -0.895965664896866
0.759918408720363 -0.897706684850494
0.763485328784005 -0.899433393530736
0.767058974964038 -0.901145764323534
0.770639289814447 -0.902843770845595
0.774226215789911 -0.904527386944784
0.777819695246707 -0.906196586700516
0.781419670443618 -0.907851344424151
0.785026083542832 -0.909491634659371
0.788638876610847 -0.911117432182561
0.792257991619385 -0.912728712003195
0.795883370446295 -0.914325449364198
0.79951495487646 -0.915907619742328
0.803152686602715 -0.917475198848529
0.806796507226751 -0.919028162628306
0.810446358260036 -0.920566487262073
0.81410218112472 -0.922090149165515
0.817763917154553 -0.92359912498994
0.821431507595803 -0.925093391622623
0.825104893608173 -0.926572926187152
0.828784016265717 -0.928037706043767
0.832468816557757 -0.929487708789698
0.836159235389808 -0.930922912259494
0.839855213584496 -0.932343294525361
0.84355669188248 -0.933748833897482
0.847263610943376 -0.935139508924332
0.850975911346681 -0.936515298393008
0.854693533592698 -0.937876181329536
0.858416418103459 -0.939222136999182
0.862144505223655 -0.940553144906763
0.865877735221565 -0.941869184796944
0.869616048289982 -0.943170236654539
0.873359384547141 -0.944456280704811
0.877107684037653 -0.945727297413757
0.880860886733434 -0.9469832674884
0.884618932534639 -0.948224171877071
0.888381761270595 -0.949449991769691
0.892149312700732 -0.950660708598046
0.895921526515522 -0.951856304036064
0.899698342337408 -0.953036760000077
0.90347969972175 -0.954202058649088
0.907265538157754 -0.955352182385041
0.911055797069409 -0.956487113853066
0.914850415816437 -0.95760683594174
0.918649333695222 -0.958711331783338
0.922452489939744 -0.959800584754074
0.926259823722543 -0.960874578474347
0.930071274155639 -0.961933296808979
0.933886780291479 -0.962976723867452
0.937706281123894 -0.964004844004131
0.941529715589021 -0.965017641818503
0.945357022566265 -0.966015102155388
0.949188140879233 -0.966997210105168
0.953023009296688 -0.967963951004003
0.956861566533488 -0.968915310434033
0.960703751251542 -0.969851274223592
0.964549502060746 -0.970771828447415
0.968398757519945 -0.971676959426836
0.972251456137869 -0.972566653729983
0.976107536374092 -0.973440898171971
0.979966936639978 -0.974299679815094
0.983829595299631 -0.975142985969003
0.987695450670847 -0.975970804190899
0.991564441026069 -0.976783122285696
0.995436504593332 -0.977579928306203
0.999311579557223 -0.978361210553295
1.00318960405983 -0.979126957576075
1.0070705162017 -0.979877158172033
1.0109542540428 -0.98061180138721
1.01484075560343 -0.981330876516353
1.01872995886526 -0.982034373103054
1.02262180177218 -0.982722280939912
1.02651622223136 -0.983394590068665
1.03041315811415 -0.984051290780333
1.03431254725701 -0.98469237361535
1.03821432746256 -0.985317829363696
1.04211843650043 -0.985927649065026
1.04602481210831 -0.98652182400879
1.04993339199284 -0.987100345734357
1.05384411383061 -0.987663206031123
1.05775691526911 -0.988210396938633
1.06167173392767 -0.988741910746676
1.06558850739846 -0.989257739995402
1.0695071732474 -0.989757877475412
1.07342766901517 -0.990242316227857
1.07734993221813 -0.990711049544534
1.08127390034931 -0.991164070967968
1.08519951087936 -0.9916013742915
1.0891267012575 -0.99202295355937
1.09305540891251 -0.992428803066785
1.09698557125366 -0.992818917360001
1.1009171256717 -0.993193291236389
1.10485000953981 -0.993551919744497
1.10878416021456 -0.993894798184114
1.11271951503687 -0.994221922106327
1.116656011333 -0.994533287313572
1.12059358641547 -0.994828889859689
1.12453217758406 -0.995108726049956
1.12847172212676 -0.995372792441149
1.13241215732074 -0.995621085841559
1.13635342043327 -0.995853603311045
1.14029544872278 -0.996070342161047
1.14423817943973 -0.996271299954622
1.14818154982761 -0.996456474506466
1.15212549712391 -0.996625863882929
1.15606995856108 -0.996779466402035
1.16001487136751 -0.996917280633482
1.16396017276844 -0.997039305398653
1.16790579998698 -0.997145539770628
1.17185169024508 -0.997235983074173
1.17579778076443 -0.997310634885732
1.17974400876748 -0.997369495033432
1.18369031147841 -0.997412563597055
1.18763662612405 -0.997439840908028
1.19158288993487 -0.997451327549403
1.19552904014596 -0.997447024355833
1.19947501399796 -0.99742693241354
1.20342074873804 -0.997391053060286
1.20736618162087 -0.997339387885336
1.2113112499096 -0.997271938729411
1.21525589087677 -0.997188707684659
1.21920004180531 -0.997089697094599
1.22314363998954 -0.996974909554059
1.22708662273603 -0.996844347909138
1.23102892736469 -0.996698015257138
1.23497049120964 -0.996535914946495
1.23891125162019 -0.996358050576723
1.24285114596185 -0.996164425998341
1.24679011161724 -0.995955045312791
1.25072808598706 -0.99572991287236
1.25466500649109 -0.995489033280104
1.25860081056912 -0.995232411389753
1.26253543568189 -0.99496005230563
1.26646881931211 -0.994671961382549
1.27040089896538 -0.994368144225715
1.27433161217116 -0.994048606690633
1.27826089648373 -0.993713354882992
1.28218868948315 -0.993362395158553
1.28611492877624 -0.992995734123044
1.29003955199748 -0.992613378632035
1.29396249681006 -0.992215335790826
1.29788370090677 -0.991801612954312
1.30180310201096 -0.991372217726855
1.30572063787755 -0.990927157962158
1.30963624629393 -0.990466441763114
1.31354986508094 -0.989990077481686
1.31746143209386 -0.989498073718743
1.32137088522331 -0.988990439323923
1.32527816239624 -0.988467183395474
1.32918320157687 -0.987928315280101
1.33308594076767 -0.987373844572803
1.33698631801028 -0.98680378111672
1.3408842713865 -0.986218135002944
1.34477973901922 -0.985616916570371
1.34867265907338 -0.98500013640551
1.35256296975692 -0.984367805342312
1.35645060932174 -0.98371993446198
1.36033551606466 -0.983056535092787
1.36421762832833 -0.982377618809878
1.36809688450223 -0.981683197435092
1.3719732230236 -0.980973283036749
1.37584658237839 -0.980247887929449
1.37971690110221 -0.979507024673869
1.38358411778127 -0.978750706076553
1.38744817105333 -0.977978945189702
1.39130899960869 -0.977191755310948
1.39516654219105 -0.976389149983144
1.39902073759854 -0.975571142994124
1.40287152468461 -0.974737748376489
1.40671884235902 -0.973888980407362
1.41056262958873 -0.973024853608168
1.4144028253989 -0.972145382744375
1.41823936887379 -0.97125058282526
1.42207219915772 -0.97034046910366
1.42590125545602 -0.969415057075718
1.42972647703594 -0.968474362480637
1.43354780322763 -0.967518401300404
1.43736517342504 -0.966547189759542
1.4411785270869 -0.965560744324841
1.4449878037376 -0.964559081705076
1.44879294296819 -0.963542218850754
1.45259388443727 -0.96251017295381
1.45639056787195 -0.961462961447349
1.46018293306877 -0.960400602005344
1.46397091989465 -0.959323112542358
1.4677544682878 -0.958230511213246
1.47153351825866 -0.957122816412855
1.47530800989085 -0.95600004677573
1.47907788334209 -0.954862221175813
1.48284307884508 -0.953709358726118
1.48660353670851 -0.952541478778437
1.49035919731795 -0.951358600923024
1.49411000113676 -0.950160744988262
1.49785588870702 -0.948947931040354
1.5015968006505 -0.947720179382989
1.50533267766951 -0.946477510557024
1.50906346054789 -0.945219945340122
1.51278909015188 -0.943947504746454
1.51650950743109 -0.942660210026324
1.52022465341936 -0.941358082665842
1.52393446923572 -0.940041144386573
1.52763889608532 -0.93870941714518
1.53133787526032 -0.937362923133074
1.53503134814077 -0.936001684776044
1.53871925619562 -0.934625724733911
1.54240154098353 -0.933235065900148
1.54607814415387 -0.931829731401509
1.54974900744758 -0.930409744597666
1.55341407269806 -0.928975129080819
1.55707328183215 -0.927525908675323
1.56072657687099 -0.926062107437302
1.56437389993091 -0.92458374965425
1.56801519322439 -0.923090859844661
1.57165039906092 -0.921583462757606
1.57527945984792 -0.920061583372351
1.57890231809163 -0.918525246897948
1.58251891639804 -0.916974478772834
1.58612919747376 -0.915409304664423
1.58973310412691 -0.913829750468669
1.59333057926806 -0.91223584230968
1.5969215659111 -0.910627606539278
1.60050600717412 -0.909005069736577
1.60408384628033 -0.907368258707561
1.60765502655893 -0.905717200484643
1.61121949144603 -0.904051922326244
1.61477718448551 -0.902372451716337
1.61832804932992 -0.900678816364014
1.62187202974136 -0.898971044203045
1.62540906959239 -0.897249163391418
1.62893911286688 -0.895513202310888
1.63246210366091 -0.893763189566532
1.63597798618367 -0.891999153986277
1.63948670475829 -0.890221124620432
1.64298820382277 -0.888429130741241
1.64648242793084 -0.886623201842396
1.64996932175282 -0.884803367638568
1.65344883007651 -0.882969658064932
1.65692089780806 -0.881122103276677
1.66038546997284 -0.879260733648533
1.66384249171633 -0.877385579774277
1.66729190830495 -0.87549667246625
1.67073366512696 -0.873594042754846
1.67416770769329 -0.871677721888028
1.67759398163848 -0.869747741330823
1.68101243272143 -0.867804132764813
1.68442300682636 -0.865846928087634
1.68782564996361 -0.863876159412456
1.69122030827055 -0.861891859067472
1.69460692801238 -0.859894059595376
1.69798545558302 -0.857882793752851
1.70135583750596 -0.855858094510031
1.7047180204351 -0.853819995049981
1.7080719511556 -0.851768528768156
1.71141757658475 -0.849703729271879
1.71475484377279 -0.847625630379784
1.7180836999038 -0.845534266121299
1.72140409229647 -0.843429670736068
1.72471596840501 -0.841311878673432
1.72801927581995 -0.839180924591859
1.73131396226901 -0.837036843358392
1.7345999756179 -0.834879670048097
1.73787726387119 -0.832709439943492
1.74114577517314 -0.830526188533985
1.74440545780849 -0.828329951515309
1.74765626020335 -0.826120764788938
1.750898130926 -0.823898664461526
1.75413101868769 -0.821663686844322
1.75735487234352 -0.819415868452592
1.76056964089323 -0.81715524600502
1.763775273482 -0.81488185642314
1.76697171940135 -0.812595736830729
1.77015892808985 -0.810296924553222
1.77333684913401 -0.807985457117106
1.77650543226909 -0.805661372249324
1.77966462737988 -0.803324707876664
1.78281438450151 -0.800975502125166
1.78595465382032 -0.798613793319492
1.78908538567459 -0.796239619982328
1.79220653055537 -0.793853020833759
1.79531803910731 -0.791454034790644
1.79841986212944 -0.789042700966004
1.80151195057596 -0.786619058668389
1.80459425555705 -0.784183147401251
1.80766672833966 -0.7817350068623
1.81072932034831 -0.77927467694289
1.81378198316588 -0.776802197727356
1.81682466853439 -0.774317609492387
1.81985732835579 -0.77182095270638
1.82287991469275 -0.769312268028782
1.82589237976945 -0.766791596309448
1.82889467597235 -0.764258978587978
1.83188675585095 -0.761714456093068
1.83486857211864 -0.75915807024185
1.83784007765338 -0.756589862639209
1.84080122549853 -0.75400987507715
1.8437519688636 -0.751418149534098
1.84669226112506 -0.748814728174237
1.84962205582705 -0.746199653346836
1.85254130668215 -0.743572967585567
1.85544996757221 -0.740934713607823
1.85834799254903 -0.738284934314043
1.86123533583514 -0.735623672787002
1.8641119518246 -0.732950972291142
1.86697779508369 -0.730266876271872
1.86983282035171 -0.727571428354868
1.87267698254171 -0.724864672345371
1.87551023674124 -0.722146652227499
1.87833253821309 -0.71941741216352
1.88114384239603 -0.716676996493166
1.88394410490558 -0.713925449732906
1.8867332815347 -0.711162816575245
1.88951132825458 -0.708389141887988
1.89227820121533 -0.705604470713541
1.89503385674673 -0.702808848268176
1.89777825135896 -0.700002319941313
1.90051134174333 -0.697184931294783
1.903233084773 -0.694356728062105
1.90594343750368 -0.691517756147751
1.90864235717441 -0.688668061626405
1.91132980120821 -0.685807690742227
1.91400572721284 -0.682936689908109
1.9166700929815 -0.680055105704935
1.91932285649353 -0.677162984880828
1.92196397591514 -0.674260374350398
1.9245934096001 -0.671347321193994
1.92721111609043 -0.668423872656947
1.92981705411714 -0.665490076148809
1.93241118260091 -0.662545979242598
1.93499346065276 -0.65959162967402
1.93756384757479 -0.656627075340714
1.94012230286084 -0.653652364301489
1.94266878619718 -0.650667544775534
1.94520325746322 -0.647672665141651
1.94772567673215 -0.644667773937487
1.95023600427168 -0.641652919858742
1.95273420054467 -0.638628151758388
1.95522022620984 -0.63559351864589
1.95769404212238 -0.632549069686419
1.96015560933474 -0.629494854200046
1.96260488909718 -0.626430921660968
1.96504184285848 -0.623357321696706
1.96746643226664 -0.620274104087308
1.96987861916948 -0.617181318764537
1.97227836561534 -0.614079015811085
1.97466563385371 -0.610967245459765
1.9770403863359 -0.607846058092685
1.97940258571568 -0.604715504240458
1.98175219484995 -0.601575634581383
1.98408917679935 -0.598426499940628
1.98641349482892 -0.595268151289404
1.98872511240876 -0.592100639744163
1.99102399321462 -0.588924016565761
1.99331010112857 -0.585738333158631
1.99558340023964 -0.582543641069971
1.9978438548444 -0.579339991988894
2.00009142944763 -0.576127437745613
2.00232608876296 -0.572906030310599
2.00454779771341 -0.569675821793737
2.00675652143211 -0.566436864443497
2.00895222526281 -0.563189210646089
2.0111348747606 -0.559932912924616
2.01330443569243 -0.556668023938231
2.01546087403776 -0.553394596481288
2.01760415598917 -0.550112683482487
2.01973424795294 -0.546822338004026
2.02185111654963 -0.543523613240745
2.02395472861474 -0.540216562519263
2.02604505119925 -0.536901239297123
2.02812205157021 -0.533577697161925
2.03018569721134 -0.530245989830471
2.03223595582363 -0.526906171147883
2.03427279532589 -0.523558295086751
2.03629618385534 -0.520202415746252
2.03830608976819 -0.516838587351276
2.04030248164022 -0.513466864251559
2.0422853282673 -0.510087300920792
2.04425459866604 -0.506699951955755
2.04621026207427 -0.503304872075436
2.04815228795166 -0.49990211612013
2.05008064598024 -0.496491739050569
2.05199530606496 -0.493073795947032
2.05389623833427 -0.489648342008451
2.05578341314064 -0.486215432551523
2.05765680106109 -0.482775123009804
2.05951637289778 -0.479327468932837
2.06136209967851 -0.47587252598523
2.06319395265726 -0.472410349945768
2.06501190331474 -0.468940996706512
2.06681592335892 -0.465464522271888
2.06860598472552 -0.461980982757784
2.07038205957859 -0.458490434390647
2.07214412031097 -0.45499293350657
2.07389213954484 -0.451488536550379
2.07562609013227 -0.447977300074717
2.07734594515565 -0.444459280739143
2.07905167792823 -0.440934535309187
2.08074326199469 -0.437403120655462
2.08242067113152 -0.433865093752726
2.08408387934763 -0.43032051167896
2.08573286088476 -0.426769431614449
2.08736759021804 -0.42321191084085
2.08898804205643 -0.419648006740268
2.09059419134322 -0.416077776794323
2.09218601325653 -0.412501278583224
2.09376348320975 -0.40891856978483
2.09532657685204 -0.405329708173718
2.09687527006882 -0.401734751620247
2.0984095389822 -0.398133758089615
2.09992935995148 -0.394526785640924
2.10143470957355 -0.390913892426234
2.10292556468346 -0.387295136689624
2.10440190235475 -0.383670576766245
2.10586369989999 -0.380040271081375
2.1073109348712 -0.376404278149464
2.10874358506029 -0.372762656573199
2.11016162849949 -0.369115465042534
2.11156504346183 -0.36546276233375
2.11295380846154 -0.361804607308497
2.11432790225447 -0.358141058912844
2.11568730383858 -0.354472176176302
2.11703199245428 -0.350798018210892
2.11836194758493 -0.347118644210164
2.1196771489572 -0.343434113448248
2.12097757654151 -0.339744485278881
2.12226321055246 -0.336049819134443
2.12353403144917 -0.332350174525003
2.12479001993577 -0.328645611037341
2.12603115696174 -0.324936188333978
2.12725742372232 -0.321221966152208
2.12846880165892 -0.317503004303132
2.1296652724595 -0.313779362670672
2.13084681805894 -0.310051101210614
2.13201342063944 -0.30631827994962
2.13316506263092 -0.302580958984252
2.13430172671132 -0.298839198479998
2.13542339580707 -0.295093058670291
2.13653005309338 -0.291342599855529
2.13762168199465 -0.287587882402091
2.13869826618482 -0.283828966741354
2.13975978958769 -0.280065913368716
2.14080623637736 -0.276298782842598
2.14183759097847 -0.27252763578347
2.14285383806663 -0.268752532872852
2.14385496256874 -0.264973534852331
2.14484094966331 -0.261190702522572
2.14581178478081 -0.257404096742332
2.146767453604 -0.25361377842745
2.14770794206825 -0.249819808549874
2.14863323636188 -0.246022248136651
2.14954332292648 -0.242221158268949
2.1504381884572 -0.238416600081039
2.15131781990307 -0.234608634759316
2.15218220446735 -0.230797323541292
2.15303132960777 -0.226982727714601
2.15386518303688 -0.223164908615993
2.15468375272235 -0.219343927630342
2.1554870268872 -0.215519846189635
2.15627499401017 -0.211692725771973
2.15704764282596 -0.207862627900568
2.15780496232552 -0.204029614142741
2.15854694175634 -0.200193746108908
2.15927357062269 -0.196355085451578
2.15998483868594 -0.192513693864356
2.16068073596479 -0.18866963308092
2.16136125273554 -0.184822964874018
2.16202637953236 -0.180973751054461
2.16267610714754 -0.177122053470114
2.16331042663172 -0.173267934004878
2.16392932929417 -0.169411454577685
2.16453280670301 -0.165552677141484
2.16512085068545 -0.161691663682229
2.16569345332804 -0.157828476217865
2.16625060697688 -0.153963176797312
2.16679230423787 -0.150095827499452
2.16731853797692 -0.146226490432111
2.16782930132016 -0.142355227731053
2.16832458765415 -0.138482101558949
2.16880439062614 -0.134607174104373
2.16926870414422 -0.130730507580774
2.16971752237754 -0.126852164225468
2.17015083975651 -0.12297220629861
2.17056865097302 -0.119090696082184
2.17097095098057 -0.115207695878976
2.17135773499452 -0.111323268011558
2.17172899849225 -0.107437474821268
2.17208473721329 -0.103550378667188
2.17242494715958 -0.0996620419251237
2.17274962459558 -0.0957725269865872
2.17305876604845 -0.091881896257767
2.1733523683082 -0.0879902121585109
2.17363042842788 -0.084097537121305
2.1738929437237 -0.0802039335902465
2.1741399117752 -0.0763094640200286
2.17437133042537 -0.0724141908749091
2.17458719778083 -0.0685181766276928
2.1747875122119 -0.0646214837587039
2.1749722723528 -0.0607241747547679
2.17514147710177 -0.0568263121081822
2.17529512562113 -0.052927958315696
2.17543321733746 -0.0490291758774833
2.17555575194172 -0.0451300272961213
2.17566272938931 -0.0412305750755619
2.1757541499002 -0.0373308817201141
2.17583001395906 -0.0334310097334144
2.17589032231531 -0.0295310216174029
2.17593507598324 -0.0256309798713079
2.17596427624211 -0.0217309469906019
2.1759779246362 -0.0178309854659948
2.17597602297491 -0.0139311577824026
2.17595857333284 -0.010031526417924
2.17592557804985 -0.00613215384281973
2.17587703973114 -0.00223310251848036
2.17581296124731 0.0016655651035862
2.17573334573439 0.00556378658279402
2.17563819659392 0.00946149949048333
2.17552751749302 0.0133586414109661
2.17540131236439 0.0172551499425265
2.17525958540636 0.021150962698462
2.17510234108295 0.0250460173080942
2.1749295841239 0.0289402514178011
2.17474131952465 0.0328336026920264
2.17453755254642 0.0367260088143211
2.17431828871621 0.0406174074883413
2.17408353382677 0.0445077364388856
2.17383329393669 0.0483969334129154
2.17356757537032 0.0522849361805611
2.17328638471785 0.0561716825361659
2.17298972883522 0.060057110299279
2.17267761484419 0.0639411573156982
2.17235005013228 0.067823761458472
2.17200704235275 0.0717048606289226
2.1716485994246 0.0755843927576707
2.17127472953254 0.0794622958056396
2.17088544112692 0.0833385077650838
2.17048074292375 0.0872129666605971
2.17006064390461 0.0910856105501313
2.16962515331664 0.0949563775260084
2.16917428067245 0.0988252057159401
2.1687080357501 0.102692033284036
2.16822642859304 0.106556798431814
2.16772946950999 0.110419439399227
2.16721716907495 0.114279894465652
2.16668953812706 0.118138101950919
2.16614658777056 0.121994000216314
2.1655883293747 0.125847527665586
2.16501477457363 0.129698622745956
2.16442593526632 0.13354722394913
2.16382182361647 0.137393269812296
2.1632024520524 0.141236698919139
2.16256783326693 0.145077449900838
2.1619179802173 0.14891546143708
2.16125290612501 0.152750672257048
2.16057262447573 0.156583021140436
2.15987714901918 0.160412446918442
2.15916649376895 0.164238888474777
2.15844067300241 0.168062284746653
2.15769970126056 0.171882574725782
2.15694359334785 0.175699697459384
2.15617236433209 0.179513592051174
2.15538602954424 0.183324197662352
2.15458460457827 0.187131453512608
2.15376810529099 0.190935298881108
2.1529365478019 0.194735673107478
2.15208994849297 0.198532515592811
2.15122832400851 0.20232576580064
2.15035169125495 0.20611536325793
2.14946006740067 0.209901247556072
2.14855346987577 0.213683358351855
2.14763191637191 0.217461635368461
2.1466954248421 0.221236018396431
2.14574401350047 0.22500644729467
2.14477770082206 0.228772861991404
2.1437965055426 0.232535202485169
2.14280044665831 0.236293408845786
2.14178954342563 0.240047421215334
2.140763815361 0.243797179809125
2.13972328224064 0.247542624916677
2.13866796410027 0.251283696902682
2.13759788123489 0.255020336207981
2.1365130541985 0.258752483350521
2.13541350380387 0.262480078926333
2.13429925112224 0.266203063610483
2.13317031748307 0.269921378158045
2.13202672447376 0.27363496340506
2.13086849393937 0.277343760269489
2.12969564798233 0.281047709752181
2.12850820896214 0.284746752937822
2.12730619949511 0.288440830995887
2.12608964245402 0.292129885181599
2.12485856096783 0.295813856836869
2.12361297842137 0.299492687391257
2.12235291845502 0.303166318362916
2.12107840496436 0.306834691359527
2.11978946209992 0.31049774807925
2.11848611426675 0.314155430311667
2.11716838612417 0.317807679938712
2.11583630258535 0.321454438935622
2.11448988881703 0.325095649371856
2.11312917023913 0.328731253412042
2.11175417252438 0.3323611933169
2.11036492159799 0.33598541144417
2.10896144363727 0.339603850249547
2.10754376507123 0.343216452287597
2.10611191258025 0.346823160212684
2.10466591309565 0.350423916779895
2.10320579379932 0.354018664845942
2.10173158212332 0.357607347370098
2.10024330574947 0.361189907415096
2.09874099260898 0.364766288148052
2.097224670882 0.36833643284136
2.09569436899722 0.371900284873615
2.09415011563144 0.375457787730505
2.09259193970915 0.379008885005718
2.0910198704021 0.382553520401848
2.08943393712886 0.386091637731277
2.08783416955435 0.389623180917081
2.08622059758945 0.393148093993925
2.08459325139049 0.396666321108943
2.08295216135883 0.400177806522637
2.08129735814034 0.403682494609749
2.079628872625 0.407180329860155
2.07794673594638 0.410671256879733
2.07625097948117 0.414155220391254
2.07454163484869 0.417632165235249
2.0728187339104 0.421102036370879
2.07108230876941 0.424564778876801
2.06933239176997 0.42802033795205
2.06756901549694 0.431468658916884
2.06579221277534 0.434909687213658
2.06400201666974 0.438343368407673
2.06219846048382 0.441769648188038
2.06038157775978 0.445188472368521
2.05855140227785 0.448599786888395
2.05670796805569 0.452003537813291
2.05485130934792 0.455399671336034
2.05298146064549 0.458788133777484
2.05109845667519 0.462168871587388
2.04920233239902 0.465541831345191
2.04729312301368 0.468906959760893
2.04537086394997 0.472264203675855
2.04343559087219 0.475613510063641
2.0414873396776 0.478954826030836
2.03952614649578 0.482288098817855
2.03755204768807 0.485613275799779
2.03556507984695 0.488930304487149
2.03356527979543 0.49223913252679
2.03155268458645 0.495539707702606
2.02952733150224 0.49883197793639
2.02748925805371 0.502115891288622
2.02543850197981 0.505391395959269
2.02337510124693 0.508658440288573
2.02129909404818 0.511916972757845
2.01921051880282 0.515166941990248
2.01710941415557 0.518408296751585
2.01499581897595 0.521640985951069
2.01286977235763 0.524864958642116
2.01073131361774 0.528080164023095
2.00858048229623 0.531286551438121
2.00641731815513 0.534484070377803
2.00424186117792 0.537672670480014
2.00205415156881 0.540852301530646
1.99985422975202 0.544022913464365
1.99764213637114 0.547184456365366
1.99541791228835 0.550336880468116
1.99318159858373 0.553480136158096
1.99093323655456 0.556614173972547
1.98867286771455 0.559738944601208
1.98640053379315 0.562854398887025
1.98411627673478 0.565960487826919
1.9818201386981 0.569057162572475
1.97951216205524 0.572144374430682
1.97719238939106 0.575222074864639
1.97486086350242 0.578290215494279
1.97251762739732 0.581348748097065
1.97016272429423 0.584397624608706
1.96779619762124 0.587436797123855
1.96541809101531 0.590466217896798
1.96302844832145 0.593485839342154
1.96062731359195 0.596495614035563
1.95821473108556 0.599495494714366
1.9557907452667 0.602485434278298
1.95335540080461 0.605465385790139
1.95090874257258 0.608435302476412
1.94845081564706 0.611395137728032
1.94598166530689 0.614344845100987
1.94350133703242 0.617284378316971
1.94100987650469 0.620213691264065
1.93850732960455 0.623132737997373
1.93599374241186 0.626041472739664
1.93346916120454 0.628939849882031
1.9309336324578 0.631827823984509
1.9283872028432 0.634705349776711
1.9258299192278 0.63757238215847
1.92326182867325 0.640428876200446
1.92068297843494 0.643274787144748
1.91809341596104 0.646110070405553
1.91549318889166 0.648934681569713
1.91288234505791 0.651748576397361
1.91026093248099 0.654551710822509
1.90762899937126 0.657344040953641
1.90498659412733 0.660125523074312
1.90233376533511 0.662896113643722
1.89967056176687 0.665655769297309
1.89699703238033 0.668404446847319
1.89431322631767 0.671142103283373
1.89161919290457 0.673868695773045
1.88891498164929 0.676584181662411
1.88620064224165 0.679288518476613
1.8834762245521 0.681981663920412
1.8807417786307 0.684663575878726
1.87799735470616 0.687334212417183
1.87524300318484 0.689993531782646
1.87247877464975 0.692641492403748
1.86970471985953 0.69527805289143
1.86692088974748 0.697903172039436
1.86412733542048 0.700516808824856
1.86132410815805 0.703118922408617
1.85851125941123 0.705709472136007
1.85568884080161 0.708288417537154
1.85285690412028 0.710855718327532
1.85001550132674 0.713411334408457
1.8471646845479 0.715955225867553
1.844304506077 0.718487352979245
1.84143501837253 0.721007676205224
1.83855627405717 0.723516156194927
1.83566832591674 0.726012753785984
1.83277122689906 0.72849743000468
1.82986503011291 0.730970146066408
1.82694978882691 0.733430863376123
1.82402555646841 0.735879543528755
1.82109238662242 0.738316148309672
1.81815033303045 0.740740639695105
1.81519944958944 0.743152979852544
1.81223979035056 0.745553131141178
1.80927140951817 0.747941056112309
1.8062943614486 0.750316717509758
1.80330870064907 0.752680078270242
1.80031448177649 0.755031101523793
1.79731175963633 0.757369750594125
1.79430058918145 0.759695988999042
1.79128102551093 0.762009780450781
1.7882531238689 0.76431108885641
1.78521693964336 0.766599878318186
1.78217252836497 0.768876113133897
1.7791199457059 0.771139757797231
1.77605924747859 0.773390776998116
1.77299048963456 0.775629135623058
1.76991372826322 0.777854798755481
1.7668290195906 0.780067731676046
1.76373641997817 0.782267899862965
1.76063598592161 0.784455268992341
1.75752777404955 0.786629804938448
1.75441184112232 0.78879147377405
1.75128824403074 0.790940241770695
1.74815703979486 0.793076075398997
1.74501828556264 0.795198941328918
1.74187203860878 0.797308806430062
1.73871835633335 0.799405637771922
1.73555729626058 0.801489402624157
1.73238891603754 0.803560068456842
1.72921327343286 0.805617602940724
1.72603042633543 0.80766197394746
1.72284043275311 0.80969314954985
1.71964335081138 0.811711098022085
1.71643923875207 0.813715787839945
1.71322815493203 0.815707187681043
1.71001015782176 0.81768526642501
1.70678530600413 0.819649993153712
1.70355365817299 0.821601337151438
1.70031527313187 0.823539267905095
1.69707020979261 0.825463755104402
1.69381852717397 0.82737476864203
1.69056028440031 0.829272278613806
1.6872955407002 0.831156255318862
1.68402435540503 0.833026669259781
1.68074678794763 0.834883491142757
1.67746289786092 0.836726691877728
1.67417274477646 0.838556242578507
1.67087638842306 0.840372114562908
1.66757388862539 0.842174279352874
1.66426530530258 0.843962708674578
1.66095069846673 0.845737374458531
1.65763012822157 0.847498248839677
1.65430365476096 0.849245304157487
1.65097133836749 0.850978512956031
1.64763323941102 0.852697847984064
1.64428941834722 0.85440328219508
1.64093993571614 0.856094788747386
1.63758485214071 0.857772341004136
1.63422422832531 0.8594359125334
1.63085812505424 0.861085477108186
1.62748660319031 0.862721008706459
1.62410972367328 0.864342481511184
1.62072754751843 0.865949869910319
1.61734013581499 0.867543148496845
1.61394754972471 0.869122292068739
1.61054985048027 0.870687275628981
1.60714709938384 0.872238074385532
1.60373935780547 0.873774663751306
1.60032668718164 0.875297019344151
1.59690914901368 0.87680511698679
1.59348680486621 0.878298932706773
1.59005971636563 0.87977844273644
1.58662794519855 0.881243623512832
1.58319155311022 0.882694451677625
1.57975060190298 0.88413090407707
1.57630515343465 0.88555295776187
1.57285526961699 0.886960589987098
1.5694010124141 0.888353778212098
1.56594244384083 0.889732500100374
1.56247962596116 0.891096733519439
1.55901262088662 0.89244645654072
1.5555414907747 0.893781647439402
1.55206629782716 0.895102284694267
1.54858710428849 0.896408346987578
1.54510397244424 0.897699813204877
1.54161696461938 0.898976662434823
1.53812614317667 0.900238873969023
1.53463157051504 0.901486427301839
1.53113330906786 0.90271930213016
1.52763142130139 0.903937478353247
1.52412596971302 0.905140936072478
1.52061701682964 0.906329655591141
1.51710462520599 0.90750361741418
1.5135888574229 0.908662802247995
1.5100697760857 0.909807191000153
1.50654744382241 0.910936764779136
1.50302192328217 0.912051504894093
1.49949327713342 0.913151392854533
1.49596156806223 0.91423641037007
1.49242685877063 0.915306539350105
1.48888921197478 0.916361761903527
1.48534869040335 0.917402060338407
1.4818053567957 0.91842741716167
1.47825927390017 0.919437815078771
1.47471050447235 0.920433236993329
1.47115911127328 0.921413666006821
1.46760515706774 0.922379085418189
1.46404870462245 0.923329478723483
1.46048981670429 0.92426482961548
1.45692855607855 0.925185121983307
1.45336498550715 0.926090339912033
1.44979916774681 0.926980467682261
1.4462311655473 0.927855489769732
1.4426610416496 0.928715390844891
1.43908885878412 0.929560155772438
1.43551467966887 0.930389769610913
1.43193856700764 0.931204217612208
1.42836058348819 0.932003485221137
1.42478079178042 0.932787558074949
1.42119925453449 0.933556422002842
1.41761603437906 0.934310063025483
1.41403119391934 0.935048467354497
1.41044479573534 0.935771621391955
1.40685690237991 0.936479511729855
1.40326757637696 0.937172125149592
1.39967688021953 0.937849448621418
1.39608487636794 0.938511469303887
1.39249162724791 0.939158174543307
1.38889719524864 0.939789551873137
1.38530164272095 0.940405589013446
1.38170503197539 0.941006273870299
1.37810742528028 0.941591594535153
1.37450888485984 0.942161539284248
1.37090947289226 0.942716096577998
1.36730925150779 0.943255255060348
1.36370828278682 0.943779003558131
1.36010662875788 0.944287331080417
1.35650435139579 0.944780226817847
1.35290151261965 0.945257680141976
1.34929817429093 0.945719680604555
1.34569439821147 0.946166217936875
1.34209024612157 0.946597282049042
1.33848577969798 0.947012863029259
1.33488106055196 0.947412951143107
1.33127615022726 0.947797536832798
1.32767111019817 0.948166610716451
1.32406600186753 0.948520163587305
1.32046088656472 0.948858186412979
1.31685582554365 0.949180670334665
1.31325087998078 0.949487606666345
1.30964611097311 0.949778986894006
1.30604157953613 0.950054802674813
1.30243734660182 0.950315045836279
1.29883347301661 0.950559708375455
1.29523001953937 0.95078878245807
1.29162704683939 0.951002260417658
1.28802461549423 0.951200134754724
1.2844227859878 0.951382398135847
1.28082161870823 0.951549043392796
1.27722117394583 0.951700063521609
1.27362151189104 0.951835451681719
1.27002269263231 0.951955201194995
1.2664247761541 0.952059305544831
1.26282782233473 0.952147758375177
1.25923189094433 0.952220553489623
1.25563704164273 0.952277684850365
1.2520433339774 0.952319146577294
1.24845082738128 0.952344932946954
1.24485958117077 0.952355038391572
1.2412696545435 0.952349457498013
1.23768110657632 0.952328185006775
1.23409399622312 0.952291215810945
1.23050838231274 0.952238544955144
1.22692432354676 0.952170167634476
1.22334187849745 0.952086079193434
1.21976110560562 0.951986275124842
1.2161820631784 0.951870751068734
1.21260480938716 0.951739502811261
1.20902940226533 0.951592526283558
1.20545589970621 0.951429817560623
1.20188435946087 0.951251372860153
1.1983148391359 0.951057188541412
1.19474739619127 0.950847261104034
1.19118208793818 0.950621587186877
1.18761897153682 0.950380163566786
1.18405810399418 0.950122987157421
1.18049954216192 0.949850055008031
1.17694334273408 0.949561364302201
1.17338956224493 0.949256912356656
1.16983825706677 0.948936696619957
1.16628948340767 0.94860071467125
1.16274329730928 0.948248964219007
1.15919975464458 0.947881443099704
1.1556589111157 0.947498149276534
1.15212082225165 0.947099080838072
1.14858554340605 0.946684235996978
1.14505312975497 0.946253613088611
1.14152363629464 0.945807210569699
1.13799711783916 0.945345027016971
1.13447362901827 0.944867061125771
1.13095322427517 0.944373311708664
1.12743595786414 0.943863777694017
1.12392188384831 0.943338458124598
1.12041105609741 0.942797352156133
1.11690352828547 0.942240459055865
1.11339935388852 0.941667778201085
1.10989858618238 0.941079309077695
1.10640127824025 0.940475051278655
1.10290748293053 0.939855004502554
1.09941725291442 0.939219168552061
1.09593064064373 0.938567543332403
1.09244769835847 0.937900128849853
1.08896847808456 0.937216925210138
1.0854930316316 0.9365179326169
1.08202141059042 0.935803151370129
1.07855366633084 0.935072581864544
1.07508984999932 0.934326224587995
1.07163001251664 0.933564080119855
1.06817420457552 0.932786149129385
1.06472247663837 0.93199243237409
1.06127487893487 0.931182930698034
1.05783146145959 0.930357645030225
1.05439227396978 0.929516576382855
1.05095736598287 0.928659725849657
1.0475267867742 0.927787094604164
1.0441005853746 0.926898683897992
1.04067881056808 0.92599449505909
1.03726151088942 0.925074529489981
1.03384873462181 0.924138788666009
1.03044052979447 0.923187274133539
1.0270369441803 0.922219987508154
1.02363802529341 0.921236930472865
1.02024382038686 0.920238104776269
1.01685437645016 0.919223512230721
1.01346974020692 0.91819315471044
1.01008995811246 0.917147034149704
1.00671507635142 0.916085152540917
1.00334514083531 0.915007511932717
0.999980197200136 0.913914114428096
0.996620290803984 0.912804962182435
0.993265466724607 0.911680057401572
0.989915769757026 0.910539402339868
0.986571244411037 0.909382999298219
0.98323193490892 0.908210850622079
0.979897885182872 0.907022958699436
0.976569138872692 0.905819325958834
0.973245739323271 0.904599954867338
0.969927729582206 0.90336484792846
0.966615152397315 0.902114007680112
0.963308050214254 0.900847436692562
0.960006465174036 0.899565137566308
0.956710439110596 0.898267112930019
0.953420013548369 0.896953365438356
0.950135229699754 0.895623897769888
0.94685612846278 0.894278712624924
0.943582750418566 0.892917812723364
0.940315135828873 0.891541200802481
0.937053324633679 0.890148879614775
0.933797356448684 0.88874085192573
0.930547270562871 0.887317120511629
0.927303105935978 0.885877688157268
0.924064901196116 0.884422557653717
0.920832694637232 0.882951731796052
0.917606524216646 0.881465213381099
0.914386427552598 0.87996300520507
0.91117244192176 0.878445110061296
0.907964604256716 0.87691153073785
0.904762951143532 0.875362270015266
0.901567518819244 0.873797330664109
0.898378343169389 0.872216715442631
0.895195459725487 0.870620427094359
0.892018903662608 0.869008468345671
0.888848709796788 0.867380841903429
0.885684912582619 0.865737550452437
0.882527546110715 0.864078596653085
0.879376644105232 0.862403983138803
0.876232239921348 0.86071371251359
0.873094366542791 0.859007787349519
0.869963056579289 0.857286210184208
0.866838342264105 0.855548983518254
0.863720255451577 0.853796109812722
0.860608827614524 0.852027591486532
0.85750408984179 0.850243430913892
0.854406072835703 0.848443630421706
0.851314806909601 0.846628192286882
0.848230321985335 0.844797118733808
0.845152647590687 0.842950411931586
0.842081812856955 0.841088073991457
0.839017846516334 0.839210106964044
0.835960776899475 0.837316512836678
0.832910631932993 0.835407293530638
0.829867439136846 0.833482450898497
0.826831225621923 0.831541986721287
0.8238020180875 0.829585902705757
0.820779842818668 0.827614200481618
0.817764725683919 0.825626881598652
0.814756692132518 0.823623947523991
0.811755767192082 0.821605399639202
0.808761975465973 0.819571239237453
0.805775341130873 0.81752146752069
0.802795887934172 0.815456085596676
0.79982363919151 0.813375094476095
0.796858617784242 0.811278495069672
0.793900846156911 0.8091662881852
0.790950346314775 0.80703847452455
0.788007139821206 0.804895054680777
0.785071247795226 0.802736029135049
0.782142690908998 0.800561398253665
0.779221489385279 0.798371162285056
0.776307662994911 0.796165321356657
0.773401231054349 0.793943875471937
0.770502212423049 0.79170682450728
0.767610625501021 0.789454168208819
0.764726488226295 0.787185906189439
0.761849818072466 0.784902037925573
0.758980632046097 0.78260256275405
0.756118946684218 0.780287479868949
0.753264778051879 0.777956788318356
0.750418141739566 0.775610487001275
0.747579052860768 0.773248574664239
0.744747526049348 0.770871049898229
0.741923575457196 0.76847791113528
0.73910721475162 0.766069156645306
0.736298457112831 0.763644784532744
0.733497315231528 0.761204792733244
0.730703801306309 0.758749179010373
0.72791792704124 0.756277940952216
0.725139703643336 0.753791075968028
0.722369141819989 0.751288581284852
0.719606251776611 0.748770453944076
0.716851043214117 0.746236690798087
0.714103525326291 0.743687288506752
0.711363706797499 0.74112224353396
0.708631595799996 0.738541552144182
0.705907199991763 0.735945210398926
0.703190526513584 0.733333214153276
0.700481581987013 0.730705559052293
0.697780372511563 0.728062240527482
0.695086903662414 0.725403253793257
0.692401180487927 0.722728593843263
0.689723207507097 0.720038255446818
0.687052988707176 0.717332233145306
0.684390527541169 0.714610521248453
0.681735826925362 0.711873113830695
0.679088889236954 0.709120004727536
0.676449716311495 0.706351187531741
0.673818309440488 0.703566655589681
0.671194669368997 0.700766401997551
0.668578796293083 0.697950419597621
0.66597068985752 0.695118700974437
0.663370349153205 0.692271238451035
0.660777772714887 0.689408024085107
0.658192958518626 0.686529049665099
0.655615903979456 0.683634306706452
0.653046605948774 0.680723786447656
0.650485060712235 0.677797479846358
0.647931263987168 0.674855377575387
0.6453852109201 0.671897470018918
0.642846896084524 0.668923747268435
0.640316313478408 0.665934199118709
0.637793456521816 0.662928815063919
0.635278318054553 0.659907584293526
0.632770890333706 0.65687049568826
0.630271165031417 0.653817537816035
0.627779133232359 0.650748698927936
0.625294785431461 0.647663966954022
0.622818111531526 0.644563329499258
0.620349100840848 0.641446773839382
0.61788774207092 0.638314286916678
0.615434023334075 0.63516585533586
0.612987932141095 0.632001465359824
0.610549455398949 0.628821102905462
0.608118579408451 0.625624753539338
0.605695289861949 0.622412402473474
0.603279571840961 0.619184034561103
0.600871409813877 0.615939634292292
0.598470787633801 0.612679185789649
0.596077688536045 0.609402672803911
0.593692095135989 0.606110078709669
0.591313989426698 0.602801386500939
0.588943352776806 0.599476578786703
0.586580165928087 0.596135637786542
0.584224408993247 0.592778545326109
0.581876061453755 0.589405282832686
0.57953510215745 0.586015831330815
0.577201509316467 0.582610171437519
0.574875260504871 0.579188283357997
0.57255633265655 0.575750146880885
0.570244702062897 0.572295741373814
0.567940344370729 0.568825045778734
0.565643234579895 0.565338038607319
0.563353347041314 0.561834697936295
0.561070655454692 0.558315001402789
0.558795132866326 0.554778926199621
0.556526751666947 0.55122644907059
0.55426548358956 0.547657546305828
0.552011299707445 0.544072193736877
0.5497641704318 0.540470366732059
0.547524065509734 0.536852040191629
0.545290954022121 0.533217188542924
0.54306480438159 0.529565785735549
0.540845584330281 0.525897805236543
0.538633260937758 0.522213220025417
0.536427800599085 0.518512002589322
0.534229169032606 0.514794124918062
0.532037331277959 0.511059558499198
0.529852251694031 0.507308274313012
0.527673893956926 0.503540242827512
0.525502221057781 0.49975543399346
0.523337195301053 0.49595381723929
0.521178778302264 0.492135361466065
0.519026930986094 0.48830003504241
0.516881613584378 0.484447805799334
0.514742785634152 0.480578641025129
0.512610405975714 0.47669250746026
0.510484432750641 0.472789371292087
0.508364823399855 0.468869198149783
0.506251534661676 0.464931953099008
0.50414452257003 0.460977600636763
0.502043742452354 0.45700610468597
0.499949148927858 0.453017428590361
0.497860695905629 0.449011535109019
0.495778336582681 0.444988386411087
0.49370202344226 0.440947944070455
0.491631708251834 0.436890169060295
0.489567342061367 0.4328150217477
0.487508875201549 0.428722461888253
0.48545625728185 0.424612448620469
0.48340943718889 0.42048494046055
0.481368363084623 0.416339895296645
0.479332982404554 0.412177270383374
0.477303241855978 0.407997022336406
0.475279087416361 0.403799107126797
0.473260464331535 0.399583480075329
0.47124731711401 0.395350095847
0.469239589541299 0.391098908445326
0.467237224654305 0.386829871206693
0.46524016475561 0.382542936794614
0.463248351407781 0.378238057194082
0.461261725431946 0.37391518370579
0.459280226906028 0.369574266940386
0.457303795163156 0.365215256812693
0.455332368790135 0.360838102535794
0.453365885625914 0.356442752615369
0.451404282760024 0.352029154843677
0.449447496531064 0.347597256293763
0.447495462525168 0.343147003313526
0.445548115574496 0.338678341519749
0.443605389755909 0.334191215792195
0.441667218389284 0.329685570267614
0.439733534036272 0.325161348333666
0.437804268498759 0.320618492622945
0.435879352817672 0.316056945006942
0.433958717271189 0.311476646589906
0.432042291373826 0.306877537702793
0.430130003874915 0.302259557897084
0.428221782757014 0.29762264593862
0.42631755523518 0.292966739801532
0.424417247755152 0.288291776661904
0.422520785992369 0.283597692891572
0.420628094850507 0.27888442405181
0.418739098460584 0.274151904887265
0.41685372017934 0.269400069319413
0.414971882588468 0.264628850440317
0.413093507493144 0.259838180506232
0.411218515920943 0.255027990931278
0.409346828120761 0.250198212281031
0.40747836356185 0.245348774266024
0.405613040932359 0.24047960573526
0.403750778138793 0.235590634669933
0.401891492304408 0.230681788176582
0.400035099768601 0.225752992480857
0.398181516085799 0.220804172920754
0.396330656024531 0.215835253940178
0.39448243356636 0.210846159082195
0.392636761905109 0.205836810982426
0.390793553445732 0.200807131362453
0.388952719803811 0.195757041023001
0.387114171804271 0.190686459837316
0.385277819480947 0.185595306744357
0.383443572075429 0.180483499742042
0.381611338036635 0.175350955880482
0.379781025019724 0.170197591255128
0.377952539885541 0.165023320999875
0.376125788699866 0.15982805928024
0.374300676732759 0.154611719286386
0.37247710845795 0.14937421322627
0.370654987552086 0.144115452318683
0.368834216894328 0.138835346786205
0.367014698565505 0.133533805848121
0.365196333847777 0.128210737713577
0.363379023224148 0.12286604957443
0.361562666377765 0.117499647598027
0.359747162191454 0.112111436920273
0.357932408747629 0.10670132163829
0.356118303327555 0.101269204803479
0.354304742411031 0.0958149884140589
0.352491621676101 0.0903385734080686
0.350678835998679 0.0848398596558795
0.348866279452267 0.0793187459531062
0.347053845307755 0.0737751300131961
0.345241426032999 0.0682089084600683
0.343428913292904 0.0626199768207911
0.341616197949023 0.057008229518253
0.33980317005949 0.0513735598635492
0.337989718878887 0.0457158600486665
0.336175732858266 0.0400350211390617
0.334361099644902 0.0343309330658812
0.332545706082385 0.0286034846188628
0.330729438210824 0.0228525634383203
0.328912181266602 0.0170780560077688
0.327093819682432 0.0112798476463425
0.325274237087825 0.00545782250106441
0.323453316308964 -0.000388136460792055
0.321630939368852 -0.00625814745941966
0.319806987487617 -0.0121523299103572
0.317981341082913 -0.0180708044318436
0.31615387976985 -0.0240136928531385
0.314324482361734 -0.0299811182219827
0.312493026869959 -0.0359732048125494
0.310659390504952 -0.0419900781332885
0.308823449675906 -0.0480318649348774
0.306985079992025 -0.0540986932182435
0.305144156262408 -0.0601906922423652
0.303300552496758 -0.0663079925323018
0.301454141906113 -0.0724507258872507
0.299604796903274 -0.0786190253886332
0.297752389103657 -0.084813025408117
0.295896789325688 -0.0910328616157301
0.294037867591797 -0.097278670988068
0.292175493128966 -0.103550591816415
0.290309534369896 -0.109848763714822
0.288439858953289 -0.116173327628573
0.28656633372524 -0.122524425842307
0.284688824739945 -0.128902201988325
0.282807197260681 -0.13530680105481
0.280921315760821 -0.141738369394403
0.27903104392486 -0.148197054732393
0.27713624464954 -0.154683006175098
0.275236780045034 -0.16119637421852
0.273332511436074 -0.167737310756564
0.271423299363072 -0.174305969089758
0.269509003583522 -0.180902503933552
0.267589483073365 -0.187527071427027
0.265664596027939 -0.194179829141483
0.263734199863947 -0.200860936088944
0.261798151220596 -0.207570552730949
0.259856305960881 -0.214308840987151
0.257908519173654 -0.221075964244108
0.25595464517469 -0.227872087363821
0.253994537508568 -0.234697376692758
0.252028048950308 -0.241552000070463
0.250055031506889 -0.24843612683857
0.248075336419426 -0.255349927849394
0.246088814164686 -0.262293575475205
0.244095314456949 -0.269267243616852
0.242094686250084 -0.27627110771278
0.2400867777393 -0.283305344748115
0.238071436363384 -0.290370133263572
0.236048508806618 -0.297465653364661
0.234017841000927 -0.304592086730601
0.231979278128023 -0.311749616623416
0.229932664621707 -0.318938427897294
0.227877844169939 -0.326158707007529
0.225814659717543 -0.333410642019949
0.223742953468047 -0.340694422620032
0.2216625668866 -0.348010240122065
0.219573340702084 -0.355358287478737
0.217475114910016 -0.362738759290266
0.215367728774851 -0.370151851813723
0.213251020832658 -0.37759776297262
0.21112482889388 -0.385076692366297
0.208988990046059 -0.392588841279206
0.206843340656867 -0.400134412690683
0.204687716376576 -0.407713611284365
0.202521952141339 -0.415326643457433
0.200345882176087 -0.422973717330883
0.198159339997311 -0.430655042758545
0.195962158416613 -0.438370831337004
0.193754169543503 -0.446121296415356
0.191535204788842 -0.45390665310466
0.189305094868025 -0.461727118288103
0.187063669804331 -0.469582910630415
0.184810758932436 -0.477474250587846
0.182546190901817 -0.485401360418166
0.180269793680132 -0.493364464190236
0.177981394557065 -0.501363787794276
0.175680820147768 -0.509399558951493
0.173367896396925 -0.517472007224427
0.171042448582178 -0.525581364026642
0.168704301318058 -0.533727862632951
0.166353278560052 -0.541911738189519
0.163989203608537 -0.550133227723983
0.161611899112714 -0.558392570155625
0.15922118707482 -0.566690006305507
0.156816888854394 -0.575025778906664
0.154398825172365 -0.583400132614379
0.151966816115589 -0.591813314016669
0.149520681141169 -0.600265571644456
0.147060239080673 -0.608757155981806
0.144585308145111 -0.617288319476671
0.142095705929222 -0.625859316551015
0.139591249416284 -0.634470403611395
0.137071754982671 -0.643121839059454
0.134537038403195 -0.651813883302452
0.131986914855361 -0.660546798763903
0.129421198924973 -0.669320849894092
0.12683970461056 -0.678136303180722
0.124242245328929 -0.686993427159713
0.121628633920139 -0.695892492425742
0.118998682653256 -0.704833771643081
0.116352203230889 -0.713817539556352
0.113689006795397 -0.722844073001511
0.111008903933978 -0.731913650916312
0.108311704684439 -0.741026554351613
0.105597218540964 -0.75018306648213
0.102865254459275 -0.759383472617252
0.100115620863235 -0.76862806021218
0.0973481256505266 -0.777917118879181
0.0945625761985696 -0.787250940398138
0.0917587793704655 -0.796629818728128
0.0889365415213206 -0.806054050018311
0.0860956685046403 -0.815523932619151
0.0832359656784831 -0.825039767093557
0.0803572379118265 -0.83460185622841
0.0774592895914736 -0.844210505045567
0.074541924628079 -0.853866020813271
0.0716049464633244 -0.863568713057411
0.0686481580763427 -0.873318893573106
0.0656713619910221 -0.88311687643602
0.0626743602825144 -0.892962978013863
0.0596569545844261 -0.902857516977868
0.0566189460959805 -0.912800814314245
0.0535601355894073 -0.922793193336105
0.0504803234172897 -0.932834979694555
0.0473793095193713 -0.942926501390623
0.0442568934308838 -0.953068088787177
0.0411128742896665 -0.963260074620251
0.0379470508440392 -0.973502794011132
0.0347592214607459 -0.983796584477716
0.0315491841324871 -0.994141785946852
0.0283167364862891 -1.00453874076599
0.0250616757912354 -1.01498779371512
0.0217837989670642 -1.02548929201861
0.0184829025923676 -1.03604358535731
0.0151587829127777 -1.04665102588072
0.0118112358494642 -1.05731196821884
0.0084400570084 -1.06802676949428
0.00504504168796416 -1.07879578933445
0.00162598488883248 -1.08961938988395
-0.00181731867773749 -1.10049793581629
-0.0052850745807973 -1.11143179434683
-0.00877748866027162 -1.12242133524445
-0.0122947670178775 -1.13346693084415
-0.0158371160080293 -1.1445689560595
-0.0194047422280477 -1.15572778839498
-0.0229978525084675 -1.16694380795835
-0.0266166539035027 -1.17821739747329
-0.0302613536819933 -1.18954894229195
-0.0339321593157393 -1.20093883040718
-0.0376292784713144 -1.21238745246568
-0.0413529189987827 -1.22389520178011
-0.0451032889214531 -1.23546247434211
-0.048880596426514 -1.24708966883512
-0.0526850498533094 -1.2587771866467
-0.0565168576833202 -1.27052543188178
-0.0603762285294209 -1.28233481137549
-0.0642633711251648 -1.29420573470559
-0.0681784943130737 -1.30613861420596
-0.0721218070344065 -1.31813386497929
-0.0760935183174496 -1.33019190491005
-0.0800938372663325 -1.34231315467774
-0.0841229730495883 -1.35449803776973
-0.088181134888103 -1.36674698049455
-0.092268532044244 -1.37906041199518
-0.0963853738088858 -1.39143876426198
-0.100531869490624 -1.40388247214601
-0.1047082284027 -1.41639197337267
-0.108914659851195 -1.42896770855451
-0.113151373122463 -1.44161012120499
-0.117418577470858 -1.4543196577517
-0.121716482105811 -1.46709676755003
-0.126045296179143 -1.47994190289634
-0.130405228772318 -1.49285551904197
-0.134796488882657 -1.50583807420627
-0.139219285411443 -1.51889002959061
-0.143673827149371 -1.53201184939203
-0.148160322763388 -1.54520400081711
-0.15267898078335 -1.558466954095
-0.157230009588105 -1.57180118249188
-0.16181361739136 -1.58520716232476
-0.166430012227991 -1.5986853729753
-0.171079401939565 -1.61223629690335
-0.175761994160098 -1.62586041966146
-0.180477996301505 -1.63955822990847
-0.185227615538921 -1.65333021942408
-0.190011058795875 -1.66717688312251
-0.194828532729431 -1.68109871906644
-0.199680243715008 -1.6950962284819
-0.204566397831144 -1.70916991577158
-0.209487200844038 -1.72332028853037
-0.214442858191816 -1.73754785755825
-0.219433574969287 -1.75185313687577
-0.224459555911466 -1.7662366437375
-0.229521005377777 -1.78069889864749
-0.23461812733575 -1.79524042537278
-0.239751125345038 -1.8098617509587
-0.244920202540172 -1.82456340574301
-0.250125561614126 -1.8393459233707
-0.25536740480166 -1.85420984080835
-0.260645933862065 -1.86915569835951
-0.265961350062128 -1.88418403967874
-0.271313854158294 -1.8992954117868
-0.276703646379701 -1.91449036508519
-0.282130926409863 -1.92976945337139
-0.287595893368987 -1.94513323385381
-0.293098745796186 -1.96058226716647
-0.298639681630675 -1.97611711738405
-0.304218898193866 -1.99173835203729
-0.309836592170228 -2.00744654212771
-0.315492959588653 -2.02324226214269
-0.321188195803771 -2.03912609007155
-0.326922495476239 -2.05509860741984
-0.332696052554127 -2.07116039922504
-0.338509060252008 -2.08731205407177
-0.344361711033031 -2.10355416410731
-0.350254196587429 -2.11988732505706
-0.356186707813336 -2.13631213623994
-0.362159434796169 -2.1528292005838
-0.368172566787905 -2.16943912464134
-0.374226292186734 -2.18614251860542
-0.380320798515896 -2.2029399963249
-0.386456272402526 -2.21983217532015
-0.392632899556872 -2.23681967679926
-0.398850864750329 -2.25390312567323
-0.405110351793297 -2.27108315057244
-0.411411543514717 -2.28836038386223
-0.417754621738069 -2.3057354616588
-0.424139767260613 -2.32320902384544
-0.430567159829735 -2.34078171408855
-0.437036978121171 -2.3584541798535
};
%\addlegendentry{SUR$_2$}
\end{axis}


\begin{axis}[
width = .32\textwidth, % scaled down due to axis equal image
height = 6.0cm,
ymode=log,
at={(.32\textwidth, 0)},
legend style={
  fill opacity=1,
  draw opacity=1,
  text opacity=1,
  at={(0.0,1)},
  anchor=north west,
  %draw=lightgray204
},
%x grid style={darkgray176},
xlabel = {time (s)}, 
%ylabel = {to do},
xmin=-0, xmax=35,
%ylabel={\(\displaystyle y\)},
%ymin=-2.52637688506353, ymax=1.16792262955725,
ymin=3.66009327069889e-06, ymax=3.58891776470632,
%ytick style={color=black}
]
\addplot [line width = \lineWidthSURError, color = SUR1, dashed]
table {%
0 0
0.02 6.85222728310329e-06
0.04 1.36557808599718e-05
0.06 2.04109382017601e-05
0.08 2.71179842657477e-05
0.1 3.37772114124798e-05
0.12 4.03889193223796e-05
0.14 4.69534149111147e-05
0.16 5.34710122441333e-05
0.18 5.99420324496547e-05
0.2 6.63668036308715e-05
0.22 7.27456607773731e-05
0.24 7.90789456755874e-05
0.26 8.53670068168513e-05
0.28 9.16101993065789e-05
0.3 9.78088847707142e-05
0.32 0.000103963431261371
0.34 0.000110074213162607
0.36 0.000116141611093325
0.38 0.00012216601181099
0.4 0.000128147808112481
0.42 0.000134087398735492
0.44 0.000139985188258013
0.46 0.000145841586997451
0.48 0.000151657010907575
0.5 0.000157431881475972
0.52 0.000163166625618844
0.54 0.000168861675576637
0.56 0.000174517468807202
0.58 0.00018013444787878
0.6 0.000185713060361544
0.62 0.000191253758718222
0.64 0.000196757000194556
0.66 0.000202223246707887
0.68 0.00020765296473538
0.7 0.000213046625200784
0.72 0.000218404703360968
0.74 0.000223727678691407
0.76 0.000229016034770669
0.78 0.000234270259163708
0.8 0.000239490843305498
0.82 0.000244678282382645
0.84 0.000249833075215118
0.86 0.000254955724136485
0.88 0.000260046734874339
0.9 0.00026510661642899
0.92 0.000270135880952536
0.94 0.000275135043626691
0.96 0.000280104622540065
0.98 0.000285045138564984
1 0.000289957115233897
1.02 0.000294841078615067
1.04 0.000299697557187925
1.06 0.000304527081718015
1.08 0.00030933018513159
1.1 0.000314107402389495
1.12 0.000318859270361311
1.14 0.000323586327699128
1.16 0.00032828911471008
1.18 0.000332968173230995
1.2 0.000337624046500302
1.22 0.00034225727903154
1.24 0.000346868416485954
1.26 0.000351458005545917
1.28 0.000356026593788258
1.3 0.000360574729556545
1.32 0.000365102961834973
1.34 0.000369611840122213
1.36 0.000374101914304561
1.38 0.000378573734530704
1.4 0.000383027851086011
1.42 0.000387464814267591
1.44 0.000391885174259742
1.46 0.000396289481010097
1.48 0.00040067828410619
1.5 0.00040505213265292
1.52 0.000409411575150568
1.54 0.000413757159373561
1.56 0.000418089432250482
1.58 0.000422408939744157
1.6 0.00042671622673368
1.62 0.000431011836896803
1.64 0.000435296312593966
1.66 0.000439570194752104
1.68 0.0004438340227523
1.7 0.000448088334315684
1.72 0.000452333665392229
1.74 0.000456570550051677
1.76 0.000460799520373664
1.78 0.000465021106341127
1.8 0.000469235835735078
1.82 0.000473444234030162
1.84 0.000477646824292299
1.86 0.000481844127077871
1.88 0.000486036660335134
1.9 0.000490224939307218
1.92 0.000494409476436191
1.94 0.000498590781270206
1.96 0.000502769360371512
1.98 0.000506945717227001
2 0.000511120352161718
2.02 0.00051529376225182
2.04 0.000519466441241932
2.06 0.000523638879464025
2.08 0.000527811563757582
2.1 0.000531984977393572
2.12 0.000536159599999543
2.14 0.0005403359074862
2.16 0.000544514371978187
2.18 0.000548695461745716
2.2 0.000552879641139126
2.22 0.000557067370524468
2.24 0.000561259106225181
2.26 0.000565455300460225
2.28 0.000569656401289346
2.3 0.000573862852559302
2.32 0.000578075093852215
2.34 0.0005822935604355
2.36 0.000586518683216375
2.38 0.00059075088869561
2.4 0.000594990598927198
2.42 0.000599238231478374
2.44 0.000603494199390353
2.46 0.000607758911145824
2.48 0.000612032770634512
2.5 0.00061631617712299
2.52 0.000620609525226646
2.54 0.000624913204883871
2.56 0.000629227601332846
2.58 0.000633553095088592
2.6 0.00063789006192531
2.62 0.000642238872858409
2.64 0.000646599894129798
2.66 0.000650973487194503
2.68 0.000655360008710003
2.7 0.000659759810527322
2.72 0.000664173239684976
2.74 0.00066860063840295
2.76 0.000673042344080156
2.78 0.000677498689292304
2.8 0.000681970001793916
2.82 0.000686456604519855
2.84 0.000690958815589084
2.86 0.000695476948310865
2.88 0.000700011311192542
2.9 0.000704562207947129
2.92 0.000709129937504659
2.94 0.000713714794023547
2.96 0.000718317066904157
2.98 0.000722937040803022
3 0.000727574995648794
3.02 0.000732231206659001
3.04 0.000736905944359034
3.06 0.000741599474601342
3.08 0.000746312058585189
3.1 0.000751043952879502
3.12 0.000755795409443344
3.14 0.000760566675650768
3.16 0.000765357994315119
3.18 0.000770169603712173
3.2 0.000775001737608519
3.22 0.000779854625285004
3.24 0.000784728491566333
3.26 0.000789623556847428
3.28 0.000794540037120906
3.3 0.00079947814400778
3.32 0.000804438084783804
3.34 0.00080942006241009
3.36 0.000814424275563629
3.38 0.000819450918665863
3.4 0.000824500181913656
3.42 0.000829572251308899
3.44 0.000834667308689738
3.46 0.000839785531760657
3.48 0.000844927094123282
3.5 0.000850092165307008
3.52 0.000855280910799099
3.54 0.000860493492075813
3.56 0.000865730066632071
3.58 0.000870990788012498
3.6 0.000876275805839776
3.62 0.000881585265846821
3.64 0.000886919309904153
3.66 0.000892278076049417
3.68 0.000897661698517201
3.7 0.000903070307767015
3.72 0.000908504030510938
3.74 0.000913962989742521
3.76 0.000919447304762704
3.78 0.000924957091207135
3.8 0.00093049246107236
3.82 0.000936053522742847
3.84 0.000941640381014725
3.86 0.000947253137120458
3.88 0.000952891888755001
3.9 0.000958556730096951
3.92 0.000964247751834021
3.94 0.000969965041182686
3.96 0.000975708681912843
3.98 0.000981478754367175
4 0.000987275335482549
4.02 0.000993098498809387
4.04 0.00099894831453194
4.06 0.00100482484948569
4.08 0.0010107281671766
4.1 0.00101665832779731
4.12 0.00102261538824442
4.14 0.00102859940213458
4.16 0.00103461041982017
4.18 0.00104064848840259
4.2 0.00104671365174827
4.22 0.00105280595050062
4.24 0.00105892542209377
4.26 0.00106507210076424
4.28 0.0010712460175626
4.3 0.00107744720036344
4.32 0.00108367567387744
4.34 0.00108993145965901
4.36 0.00109621457611739
4.38 0.00110252503852404
4.4 0.00110886285901982
4.42 0.0011152280466239
4.44 0.00112162060723856
4.46 0.00112804054365835
4.48 0.00113448785557268
4.5 0.00114096253957308
4.52 0.00114746458915722
4.54 0.00115399399473201
4.56 0.00116055074362084
4.58 0.00116713482006243
4.6 0.00117374620521732
4.62 0.00118038487716783
4.64 0.00118705081092155
4.66 0.00119374397841346
4.68 0.00120046434850526
4.7 0.00120721188698832
4.72 0.00121398655658323
4.74 0.00122078831694111
4.76 0.0012276171246425
4.78 0.00123447293319882
4.8 0.00124135569304968
4.82 0.00124826535156454
4.84 0.0012552018530395
4.86 0.00126216513869949
4.88 0.00126915514669352
4.9 0.0012761718120947
4.92 0.00128321506689921
4.94 0.00129028484002474
4.96 0.00129738105730779
4.98 0.0013045036415035
5 0.00131165251228156
5.02 0.00131882758622644
5.04 0.00132602877683384
5.06 0.00133325599451027
5.08 0.00134050914656961
5.1 0.00134778813723354
5.12 0.0013550928676285
5.14 0.00136242323578196
5.16 0.00136977913662494
5.18 0.0013771604619881
5.2 0.00138456710060098
5.22 0.00139199893808988
5.24 0.00139945585697886
5.26 0.00140693773668727
5.28 0.00141444445352998
5.3 0.00142197588071698
5.32 0.00142953188835339
5.34 0.00143711234343804
5.36 0.00144471710986607
5.38 0.00145234604842934
5.4 0.00145999901681498
5.42 0.00146767586960995
5.44 0.00147537645830073
5.46 0.00148310063127619
5.48 0.00149084823382928
5.5 0.00149861910816074
5.52 0.00150641309338009
5.54 0.00151423002551237
5.56 0.00152206973749882
5.58 0.00152993205920422
5.6 0.0015378168174193
5.62 0.00154572383586655
5.64 0.00155365293520656
5.66 0.00156160393304454
5.68 0.00156957664393659
5.7 0.00157757087939486
5.72 0.00158558644789965
5.74 0.0015936231549022
5.76 0.00160168080283787
5.78 0.00160975919113314
5.8 0.00161785811621481
5.82 0.00162597737152257
5.84 0.00163411674751808
5.86 0.00164227603169603
5.88 0.0016504550085989
5.9 0.00165865345982599
5.92 0.00166687116404744
5.94 0.00167510789702046
5.96 0.00168336343160177
5.98 0.00169163753776101
6 0.00169992998259835
6.02 0.00170824053036066
6.04 0.0017165689424567
6.06 0.0017249149774763
6.08 0.00173327839120648
6.1 0.00174165893665131
6.12 0.00175005636405048
6.14 0.00175847042089941
6.16 0.00176690085197006
6.18 0.00177534739933143
6.2 0.00178380980237009
6.22 0.00179228779781546
6.24 0.00180078111976051
6.26 0.00180928949968642
6.28 0.00181781266648501
6.3 0.00182635034648641
6.32 0.00183490226348212
6.34 0.00184346813875258
6.36 0.00185204769109408
6.38 0.00186064063684442
6.4 0.00186924668991333
6.42 0.00187786556180813
6.44 0.00188649696166645
6.46 0.00189514059628358
6.48 0.00190379617014313
6.5 0.00191246338545055
6.52 0.00192114194216151
6.54 0.00192983153801642
6.56 0.00193853186857264
6.58 0.00194724262723905
6.6 0.0019559635053089
6.62 0.00196469419199452
6.64 0.00197343437446366
6.66 0.00198218373787593
6.68 0.00199094196541628
6.7 0.00199970873833632
6.72 0.00200848373598778
6.74 0.0020172666358643
6.76 0.00202605711363787
6.78 0.00203485484319918
6.8 0.00204365949669711
6.82 0.00205247074458004
6.84 0.00206128825563488
6.86 0.00207011169703008
6.88 0.00207894073435695
6.9 0.00208777503167328
6.92 0.0020966142515456
6.94 0.00210545805509068
6.96 0.00211430610202411
6.98 0.00212315805069956
7 0.00213201355815751
7.02 0.00214087228016987
7.04 0.00214973387128408
7.06 0.00215859798487175
7.08 0.00216746427317357
7.1 0.00217633238734669
7.12 0.00218520197751503
7.14 0.00219407269281311
7.16 0.00220294418143771
7.18 0.002211816090695
7.2 0.00222068806704991
7.22 0.00222955975617528
7.24 0.00223843080300435
7.26 0.00224730085177767
7.28 0.00225616954609549
7.3 0.00226503652897021
7.32 0.00227390144287299
7.34 0.00228276392978957
7.36 0.00229162363127169
7.38 0.00230048018848716
7.4 0.0023093332422731
7.42 0.00231818243319031
7.44 0.00232702740157611
7.46 0.00233586778759409
7.48 0.00234470323129111
7.5 0.00235353337265016
7.52 0.00236235785164431
7.54 0.00237117630828924
7.56 0.0023799883827002
7.58 0.00238879371514512
7.6 0.00239759194609937
7.62 0.00240638271629778
7.64 0.00241516566679637
7.66 0.00242394043902041
7.68 0.00243270667482358
7.7 0.00244146401654285
7.72 0.00245021210705398
7.74 0.00245895058982152
7.76 0.00246767910896543
7.78 0.00247639730930664
7.8 0.00248510483642695
7.82 0.00249380133672389
7.84 0.00250248645746732
7.86 0.00251115984685452
7.88 0.00251982115406449
7.9 0.00252847002931626
7.92 0.00253710612392386
7.94 0.00254572909035032
7.96 0.00255433858226278
7.98 0.00256293425459188
8 0.00257151576358216
8.02 0.00258008276685191
8.04 0.00258863492344413
8.06 0.00259717189388591
8.08 0.00260569334023791
8.1 0.00261419892615503
8.12 0.00262268831693693
8.14 0.00263116117958295
8.16 0.00263961718284941
8.18 0.00264805599729906
8.2 0.00265647729535761
8.22 0.00266488075136585
8.24 0.00267326604163526
8.26 0.00268163284449932
8.28 0.00268998084036628
8.3 0.00269830971177107
8.32 0.00270661914343081
8.34 0.00271490882229112
8.36 0.00272317843758282
8.38 0.00273142768087144
8.4 0.00273965624610753
8.42 0.00274786382967575
8.44 0.00275605013044809
8.46 0.0027642148498321
8.48 0.00277235769181944
8.5 0.00278047836303621
8.52 0.00278857657279202
8.54 0.0027966520331244
8.56 0.0028047044588493
8.58 0.00281273356760819
8.6 0.00282073907991384
8.62 0.00282872071919609
8.64 0.00283667821184923
8.66 0.0028446112872735
8.68 0.00285251967792521
8.7 0.00286040311935415
8.72 0.00286826135025095
8.74 0.00287609411249088
8.76 0.00288390115117252
8.78 0.002891682214662
8.8 0.00289943705463241
8.82 0.00290716542610637
8.84 0.00291486708749421
8.86 0.00292254180063142
8.88 0.00293018933082095
8.9 0.00293780944686717
8.92 0.00294540192111725
8.94 0.00295296652949288
8.96 0.00296050305152802
8.98 0.00296801127040483
9 0.00297549097298652
9.02 0.00298294194985013
9.04 0.00299036399532127
9.06 0.00299775690750513
9.08 0.00300512048831641
9.1 0.003012454543512
9.12 0.00301975888271768
9.14 0.00302703331945921
9.16 0.00303427767118943
9.18 0.003041491759316
9.2 0.00304867540922626
9.22 0.00305582845031258
9.24 0.00306295071599825
9.26 0.00307004204375965
9.28 0.00307710227515092
9.3 0.00308413125582387
9.32 0.00309112883555072
9.34 0.0030980948682422
9.36 0.00310502921196745
9.38 0.00311193172897352
9.4 0.00311880228570056
9.42 0.00312564075280029
9.44 0.0031324470051478
9.46 0.00313922092186169
9.48 0.00314596238631106
9.5 0.00315267128613423
9.52 0.00315934751324461
9.54 0.00316599096384522
9.56 0.00317260153843636
9.58 0.00317917914182351
9.6 0.00318572368312742
9.62 0.00319223507578723
9.64 0.00319871323756923
9.66 0.0032051580905701
9.68 0.00321156956121952
9.7 0.0032179475802824
9.72 0.00322429208286273
9.74 0.00323060300840103
9.76 0.00323688030067486
9.78 0.00324312390779662
9.8 0.00324933378221269
9.82 0.00325550988069636
9.84 0.00326165216434603
9.86 0.00326776059857774
9.88 0.00327383515311716
9.9 0.0032798758019928
9.92 0.0032858825235275
9.94 0.00329185530032633
9.96 0.00329779411926679
9.98 0.00330369897148684
10 0.0033095698523697
10.02 0.0033154067615311
10.04 0.0033212097028029
10.06 0.00332697868421669
10.08 0.00333271371798465
10.1 0.00333841482048561
10.12 0.00334408201223952
10.14 0.00334971531789019
10.16 0.00335531476618217
10.18 0.00336088038993815
10.2 0.00336641222603525
10.22 0.00337191031537922
10.24 0.00337737470287969
10.26 0.0033828054374214
10.28 0.00338820257183901
10.3 0.00339356616288436
10.32 0.00339889627119943
10.34 0.00340419296128329
10.36 0.0034094563014618
10.38 0.00341468636385212
10.4 0.00341988322433058
10.42 0.00342504696249734
10.44 0.00343017766164029
10.46 0.0034352754086955
10.48 0.00344034029421444
10.5 0.00344537241232165
10.52 0.00345037186067358
10.54 0.00345533874042235
10.56 0.00346027315617046
10.58 0.00346517521592917
10.6 0.00347004503107544
10.62 0.00347488271630725
10.64 0.00347968838959864
10.66 0.00348446217215393
10.68 0.0034892041883601
10.7 0.00349391456573899
10.72 0.00349859343490032
10.74 0.00350324092949191
10.76 0.00350785718614696
10.78 0.00351244234443517
10.8 0.00351699654681309
10.82 0.00352151993856852
10.84 0.00352601266776954
10.86 0.00353047488520748
10.88 0.00353490674434659
10.9 0.00353930840126709
10.92 0.00354368001460761
10.94 0.00354802174551207
10.96 0.00355233375757007
10.98 0.00355661621675902
11 0.00356086929138744
11.02 0.00356509315203445
11.04 0.0035692879714912
11.06 0.00357345392470191
11.08 0.00357759118870134
11.1 0.00358169994255378
11.12 0.00358578036729378
11.14 0.00358983264586273
11.16 0.00359385696304641
11.18 0.00359785350541238
11.2 0.00360182246124706
11.22 0.00360576402049184
11.24 0.00360967837467946
11.26 0.00361356571687055
11.28 0.00361742624158761
11.3 0.00362126014475099
11.32 0.00362506762361325
11.34 0.00362884887669465
11.36 0.00363260410371726
11.38 0.00363633350554012
11.4 0.00364003728409045
11.42 0.00364371564230177
11.44 0.00364736878404473
11.46 0.00365099691406112
11.48 0.00365460023789884
11.5 0.00365817896184489
11.52 0.00366173329285832
11.54 0.00366526343850341
11.56 0.0036687696068849
11.58 0.00367225200658122
11.6 0.00367571084657686
11.62 0.00367914633619692
11.64 0.00368255868504041
11.66 0.00368594810291544
11.68 0.00368931479977288
11.7 0.00369265898563908
11.72 0.00369598087055332
11.74 0.00369928066449889
11.76 0.00370255857734165
11.78 0.00370581481876292
11.8 0.00370904959819631
11.82 0.00371226312476261
11.84 0.0037154556072068
11.86 0.00371862725383386
11.88 0.00372177827244604
11.9 0.00372490887028067
11.92 0.00372801925394684
11.94 0.00373110962936575
11.96 0.00373418020170676
11.98 0.00373723117532778
12 0.00374026275371546
12.02 0.0037432751394246
12.04 0.00374626853402069
12.06 0.00374924313801914
12.08 0.00375219915082806
12.1 0.0037551367706915
12.12 0.00375805619463115
12.14 0.00376095761839269
12.16 0.00376384123638813
12.18 0.0037667072416419
12.2 0.00376955582573784
12.22 0.00377238717876512
12.24 0.00377520148926448
12.26 0.00377799894417872
12.28 0.00378077972880054
12.3 0.00378354402672248
12.32 0.00378629201978873
12.34 0.00378902388804406
12.36 0.00379173980968993
12.38 0.0037944399610331
12.4 0.00379712451644419
12.42 0.00379979364830985
12.44 0.00380244752699003
12.46 0.0038050863207738
12.48 0.0038077101958389
12.5 0.00381031931620831
12.52 0.00381291384371198
12.54 0.00381549393794487
12.56 0.00381805975623113
12.58 0.00382061145358589
12.6 0.00382314918267846
12.62 0.00382567309379634
12.64 0.00382818333481251
12.66 0.00383068005115217
12.68 0.00383316338575814
12.7 0.00383563347906312
12.72 0.00383809046895585
12.74 0.00384053449075344
12.76 0.00384296567717377
12.78 0.00384538415830963
12.8 0.00384779006159919
12.82 0.00385018351180473
12.84 0.00385256463098687
12.86 0.00385493353848257
12.88 0.00385729035088254
12.9 0.00385963518201244
12.92 0.0038619681429109
12.94 0.00386428934181423
12.96 0.00386659888413643
12.98 0.00386889687245596
13 0.0038711834064958
13.02 0.00387345858311599
13.04 0.00387572249629351
13.06 0.00387797523711662
13.08 0.00388021689377049
13.1 0.00388244755152817
13.12 0.00388466729274192
13.14 0.00388687619683739
13.16 0.00388907434030458
13.18 0.00389126179669349
13.2 0.00389343863661073
13.22 0.00389560492771388
13.24 0.00389776073471115
13.26 0.00389990611935881
13.28 0.00390204114046289
13.3 0.00390416585387675
13.32 0.00390628031250615
13.34 0.0039083845663097
13.36 0.00391047866230414
13.38 0.0039125626445676
13.4 0.00391463655424737
13.42 0.00391670042956453
13.44 0.00391875430582287
13.46 0.00392079821541605
13.48 0.00392283218783973
13.5 0.00392485624969928
13.52 0.00392687042472247
13.54 0.00392887473377226
13.56 0.00393086919485749
13.58 0.0039328538231502
13.6 0.00393482863099795
13.62 0.00393679362794091
13.64 0.0039387488207275
13.66 0.00394069421333218
13.68 0.00394262980697307
13.7 0.00394455560013279
13.72 0.0039464715885758
13.74 0.00394837776536981
13.76 0.00395027412090702
13.78 0.00395216064292705
13.8 0.00395403731653786
13.82 0.00395590412424029
13.84 0.00395776104595199
13.86 0.003959608059032
13.88 0.00396144513830645
13.9 0.00396327225609432
13.92 0.00396508938223428
13.94 0.0039668964841115
13.96 0.0039686935266877
13.98 0.00397048047252609
14 0.0039722572818233
14.02 0.00397402391243796
14.04 0.00397578031992116
14.06 0.00397752645754648
14.08 0.0039792622763416
14.1 0.00398098772512034
14.12 0.00398270275051449
14.14 0.00398440729700591
14.16 0.00398610130696078
14.18 0.00398778472066191
14.2 0.00398945747634308
14.22 0.00399111951022364
14.24 0.00399277075654239
14.26 0.00399441114759387
14.28 0.00399604061376276
14.3 0.00399765908356019
14.32 0.00399926648365908
14.34 0.00400086273893148
14.36 0.00400244777248487
14.38 0.00400402150569924
14.4 0.00400558385826348
14.42 0.00400713474821456
14.44 0.00400867409197425
14.46 0.00401020180438671
14.48 0.00401171779875777
14.5 0.00401322198689257
14.52 0.00401471427913459
14.54 0.00401619458440398
14.56 0.00401766281023693
14.58 0.0040191188628249
14.6 0.0040205626470533
14.62 0.00402199406654129
14.64 0.0040234130236813
14.66 0.00402481941967864
14.68 0.00402621315459084
14.7 0.00402759412736805
14.72 0.00402896223589255
14.74 0.004030317377019
14.76 0.00403165944661397
14.78 0.00403298833959666
14.8 0.0040343039499782
14.82 0.00403560617090261
14.84 0.0040368948946863
14.86 0.00403817001285858
14.88 0.00403943141620173
14.9 0.00404067899479137
14.92 0.00404191263803589
14.94 0.00404313223471729
14.96 0.00404433767303154
14.98 0.00404552884062783
15 0.00404670562464829
15.02 0.00404786791176907
15.04 0.00404901558823995
15.06 0.00405014853992365
15.08 0.00405126665233558
15.1 0.00405236981068409
15.12 0.00405345789990961
15.14 0.00405453080472407
15.16 0.00405558840965087
15.18 0.00405663059906346
15.2 0.00405765725722522
15.22 0.0040586682683283
15.24 0.00405966351653163
15.26 0.00406064288600127
15.28 0.00406160626094838
15.3 0.00406255352566743
15.32 0.00406348456457534
15.34 0.00406439926224952
15.36 0.00406529750346554
15.38 0.00406617917323597
15.4 0.00406704415684727
15.42 0.00406789233989811
15.44 0.00406872360833734
15.46 0.00406953784849997
15.48 0.00407033494714532
15.5 0.00407111479149411
15.52 0.00407187726926491
15.54 0.00407262226871057
15.56 0.00407334967865542
15.58 0.00407405938853121
15.6 0.00407475128841284
15.62 0.00407542526905481
15.64 0.00407608122192655
15.66 0.00407671903924911
15.68 0.00407733861402781
15.7 0.00407793984009058
15.72 0.00407852261212075
15.74 0.0040790868256925
15.76 0.0040796323773054
15.78 0.00408015916441821
15.8 0.00408066708548343
15.82 0.00408115603998105
15.84 0.00408162592845249
15.86 0.00408207665253313
15.88 0.00408250811498708
15.9 0.00408292021973786
15.92 0.00408331287190363
15.94 0.00408368597782763
15.96 0.00408403944511187
15.98 0.0040843731826482
16 0.00408468710065089
16.02 0.0040849811106875
16.04 0.00408525512571096
16.06 0.00408550906009033
16.08 0.00408574282964048
16.1 0.00408595635165454
16.12 0.00408614954493225
16.14 0.00408632232981227
16.16 0.00408647462819981
16.18 0.00408660636359732
16.2 0.00408671746113334
16.22 0.00408680784759159
16.24 0.00408687745143995
16.26 0.00408692620285813
16.28 0.00408695403376572
16.3 0.00408696087785222
16.32 0.00408694667060125
16.34 0.00408691134931931
16.36 0.00408685485316475
16.38 0.00408677712317084
16.4 0.00408667810227463
16.42 0.0040865577353416
16.44 0.00408641596919302
16.46 0.0040862527526291
16.48 0.0040860680364567
16.5 0.00408586177351232
16.52 0.00408563391868668
16.54 0.00408538442895079
16.56 0.00408511326337601
16.58 0.004084820383161
16.6 0.00408450575165441
16.62 0.0040841693343741
16.64 0.00408381109903472
16.66 0.00408343101556616
16.68 0.00408302905613672
16.7 0.00408260519517476
16.72 0.00408215940938792
16.74 0.00408169167778493
16.76 0.0040812019816968
16.78 0.00408069030479501
16.8 0.00408015663311123
16.82 0.00407960095505809
16.84 0.00407902326144564
16.86 0.00407842354550061
16.88 0.00407780180288443
16.9 0.00407715803171113
16.92 0.00407649223256331
16.94 0.00407580440850971
16.96 0.00407509456512078
16.98 0.00407436271048602
17 0.00407360885522765
17.02 0.00407283301251489
17.04 0.0040720351980836
17.06 0.0040712154302425
17.08 0.00407037372989497
17.1 0.00406951012054664
17.12 0.00406862462832154
17.14 0.00406771728197153
17.16 0.00406678811289075
17.18 0.00406583715512665
17.2 0.00406486444539055
17.22 0.00406387002306981
17.24 0.00406285393023506
17.26 0.00406181621165177
17.28 0.00406075691479073
17.3 0.00405967608983523
17.32 0.00405857378968946
17.34 0.00405745006998751
17.36 0.00405630498910032
17.38 0.00405513860814211
17.4 0.00405395099097856
17.42 0.00405274220423229
17.44 0.00405151231728771
17.46 0.00405026140229785
17.48 0.00404898953418833
17.5 0.00404769679066033
17.52 0.00404638325219826
17.54 0.00404504900207082
17.56 0.00404369412633254
17.58 0.00404231871383214
17.6 0.00404092285620816
17.62 0.00403950664789521
17.64 0.00403807018612359
17.66 0.00403661357092228
17.68 0.00403513690511667
17.7 0.0040336402943316
17.72 0.00403212384699014
17.74 0.00403058767431157
17.76 0.00402903189031373
17.78 0.00402745661181011
17.8 0.00402586195840652
17.82 0.00402424805250366
17.84 0.0040226150192913
17.86 0.00402096298674673
17.88 0.00401929208563046
17.9 0.00401760244948742
17.92 0.00401589421463853
17.94 0.00401416752017903
17.96 0.00401242250797531
17.98 0.00401065932265881
18 0.00400887811162419
18.02 0.004007079025021
18.04 0.00400526221574913
18.06 0.0040034278394586
18.08 0.00400157605453715
18.1 0.00399970702211056
18.12 0.00399782090603248
18.14 0.00399591787288186
18.16 0.00399399809195405
18.18 0.00399206173525621
18.2 0.00399010897750091
18.22 0.00398813999610057
18.24 0.00398615497115584
18.26 0.00398415408545846
18.28 0.00398213752447578
18.3 0.00398010547634933
18.32 0.00397805813188416
18.34 0.00397599568454533
18.36 0.00397391833045169
18.38 0.00397182626836577
18.4 0.00396971969968955
18.42 0.00396759882845736
18.44 0.00396546386132881
18.46 0.00396331500758326
18.48 0.00396115247911336
18.5 0.00395897649041487
18.52 0.00395678725858764
18.54 0.00395458500332356
18.56 0.00395236994690574
18.58 0.00395014231419504
18.6 0.00394790233263159
18.62 0.00394565023222546
18.64 0.00394338624555385
18.66 0.00394111060775452
18.68 0.0039388235565196
18.7 0.0039365253320933
18.72 0.00393421617726545
18.74 0.00393189633736715
18.76 0.00392956606026794
18.78 0.00392722559637131
18.8 0.00392487519860983
18.82 0.00392251512244253
18.84 0.00392014562584997
18.86 0.00391776696933368
18.88 0.00391537941591207
18.9 0.00391298323111446
18.92 0.00391057868298417
18.94 0.00390816604207043
18.96 0.00390574558142939
18.98 0.00390331757661955
19 0.00390088230570098
19.02 0.00389844004923275
19.04 0.00389599109027208
19.06 0.00389353571436805
19.08 0.00389107420956628
19.1 0.00388860686639889
19.12 0.00388613397788952
19.14 0.00388365583954525
19.16 0.00388117274935871
19.18 0.00387868500780229
19.2 0.00387619291782799
19.22 0.00387369678486054
19.24 0.00387119691679755
19.26 0.00386869362400238
19.28 0.00386618721930227
19.3 0.00386367801798318
19.32 0.00386116633778279
19.34 0.00385865249888583
19.36 0.003856136823916
19.38 0.003853619637928
19.4 0.00385110126839963
19.42 0.00384858204522318
19.44 0.00384606230069265
19.46 0.00384354236949265
19.48 0.00384102258868536
19.5 0.00383850329769571
19.52 0.00383598483829436
19.54 0.00383346755458122
19.56 0.00383095179296542
19.58 0.00382843790214253
19.6 0.0038259262330739
19.62 0.00382341713895746
19.64 0.00382091097520295
19.66 0.00381840809939977
19.68 0.0038159088712836
19.7 0.00381341365270237
19.72 0.00381092280757677
19.74 0.00380843670185615
19.76 0.00380595570347795
19.78 0.00380348018231557
19.8 0.00380101051012661
19.82 0.00379854706049835
19.84 0.00379609020878686
19.86 0.00379364033205188
19.88 0.00379119780898807
19.9 0.00378876301985311
19.92 0.00378633634638832
19.94 0.00378391817173649
19.96 0.0037815088803492
19.98 0.00377910885789767
20 0.00377671849116773
20.02 0.00377433816795745
20.04 0.00377196827696075
20.06 0.00376960920765159
20.08 0.00376726135015608
20.1 0.00376492509511942
20.12 0.00376260083356601
20.14 0.00376028895674985
20.16 0.00375798985599937
20.18 0.00375570392255252
20.2 0.00375343154738454
20.22 0.00375117312102463
20.24 0.00374892903336605
20.26 0.00374669967346729
20.28 0.00374448542933925
20.3 0.00374228668772743
20.32 0.00374010383387891
20.34 0.0037379372513046
20.36 0.00373578732152509
20.38 0.00373365442380913
20.4 0.00373153893489754
20.42 0.00372944122871706
20.44 0.00372736167608187
20.46 0.00372530064438341
20.48 0.00372325849726251
20.5 0.00372123559427576
20.52 0.00371923229054477
20.54 0.00371724893639042
20.56 0.00371528587695569
20.58 0.00371334345181227
20.6 0.00371142199455304
20.62 0.00370952183237103
20.64 0.00370764328562166
20.66 0.00370578666736845
20.68 0.00370395228291571
20.7 0.00370214042932536
20.72 0.00370035139491451
20.74 0.00369858545873702
20.76 0.00369684289005039
20.78 0.00369512394776552
20.8 0.00369342887987872
20.82 0.00369175792288201
20.84 0.00369011130116496
20.86 0.00368848922639052
20.88 0.00368689189685435
20.9 0.00368531949683161
20.92 0.00368377219589989
20.94 0.00368225014824226
20.96 0.00368075349193703
20.98 0.00367928234822687
21 0.0036778368207691
21.02 0.00367641699486331
21.04 0.00367502293666854
21.06 0.00367365469239426
21.08 0.00367231228747432
21.1 0.00367099572572371
21.12 0.00366970498847664
21.14 0.00366844003370214
21.16 0.00366720079510579
21.18 0.00366598718120907
21.2 0.00366479907441169
21.22 0.00366363633003889
21.24 0.00366249877536235
21.26 0.00366138620860919
21.28 0.00366029839795631
21.3 0.00365923508049832
21.32 0.00365819596120541
21.34 0.00365718071186207
21.36 0.00365618896998713
21.38 0.00365522033774182
21.4 0.0036542743808179
21.42 0.00365335062731197
21.44 0.00365244856658784
21.46 0.00365156764811662
21.48 0.00365070728031287
21.5 0.00364986682934849
21.52 0.00364904561795554
21.54 0.00364824292422424
21.56 0.0036474579803766
21.58 0.00364668997154078
21.6 0.00364593803450565
21.62 0.0036452012564714
21.64 0.00364447867378938
21.66 0.00364376927069377
21.68 0.00364307197802425
21.7 0.00364238567194627
21.72 0.00364170917265844
21.74 0.00364104124310102
21.76 0.0036403805876586
21.78 0.00363972585086105
21.8 0.00363907561607501
21.82 0.00363842840420562
21.84 0.00363778267238981
21.86 0.00363713681269244
21.88 0.00363648915080539
21.9 0.00363583794474924
21.92 0.00363518138357967
21.94 0.00363451758609652
21.96 0.00363384459956376
21.98 0.00363316039843302
22 0.00363246288308207
22.02 0.00363174987855534
22.04 0.00363101913332345
22.06 0.0036302683180502
22.08 0.00362949502438133
22.1 0.00362869676374078
22.12 0.00362787096615139
22.14 0.00362701497907036
22.16 0.0036261260662478
22.18 0.00362520140660354
22.2 0.00362423809313546
22.22 0.00362323313184505
22.24 0.00362218344069593
22.26 0.00362108584859971
22.28 0.0036199370944321
22.3 0.00361873382608749
22.32 0.00361747259956133
22.34 0.00361614987807605
22.36 0.00361476203124152
22.38 0.00361330533426754
22.4 0.00361177596720834
22.42 0.00361017001425998
22.44 0.00360848346311301
22.46 0.00360671220434873
22.48 0.00360485203089802
22.5 0.00360289863755395
22.52 0.00360084762055195
22.54 0.0035986944772114
22.56 0.00359643460564915
22.58 0.00359406330456408
22.6 0.00359157577310371
22.62 0.00358896711080688
22.64 0.00358623231763609
22.66 0.00358336629410385
22.68 0.00358036384148858
22.7 0.0035772196621615
22.72 0.0035739283600169
22.74 0.00357048444102071
22.76 0.00356688231387455
22.78 0.00356311629081949
22.8 0.00355918058856911
22.82 0.0035550693293889
22.84 0.00355077654233325
22.86 0.00354629616464969
22.88 0.00354162204334937
22.9 0.00353674793697401
22.92 0.00353166751755646
22.94 0.00352637437279435
22.96 0.00352086200845104
22.98 0.00351512385099615
23 0.00350915325050686
23.02 0.00350294348384419
23.04 0.00349648775812483
23.06 0.00348977921451075
23.08 0.00348281093233564
23.1 0.00347557593359738
23.12 0.00346806718784171
23.14 0.00346027761746327
23.16 0.00345220010346126
23.18 0.00344382749167501
23.2 0.00343515259954659
23.22 0.00342616822344533
23.24 0.00341686714659881
23.26 0.00340724214767247
23.28 0.00339728601006745
23.3 0.00338699153197345
23.32 0.0033763515372446
23.34 0.00336535888717721
23.36 0.00335400649324552
23.38 0.00334228733088668
23.4 0.0033301944544257
23.42 0.00331772101322163
23.44 0.00330486026916079
23.46 0.00329160561559261
23.48 0.00327795059785145
23.5 0.00326388893548711
23.52 0.00324941454636716
23.54 0.003234521572811
23.56 0.00321920440994229
23.58 0.00320345773645232
23.6 0.00318721312233464
23.62 0.00317150487121679
23.64 0.00315632025461895
23.66 0.00314164678024284
23.68 0.00312747218735606
23.7 0.00311378444215472
23.72 0.0031005717331123
23.74 0.00308782246633089
23.76 0.00307552526089764
23.78 0.00306366894426533
23.8 0.00305224254765234
23.82 0.00304123530148975
23.84 0.00303063663090684
23.86 0.00302043615127352
23.88 0.00301062366380138
23.9 0.00300118915121677
23.92 0.00299212277349711
23.94 0.00298341486370563
23.96 0.0029750559238916
23.98 0.00296703662109556
24 0.00295934778344222
24.02 0.00295198039633554
24.04 0.00294492559874463
24.06 0.00293817467961255
24.08 0.00293171907435299
24.1 0.00292555036146536
24.12 0.0029196602592557
24.14 0.00291404062266478
24.16 0.0029086834402156
24.18 0.00290358083105656
24.2 0.00289872504213039
24.22 0.00289410844543442
24.24 0.00288972353540859
24.26 0.00288556292641024
24.28 0.00288161935030462
24.3 0.00287788565415558
24.32 0.002874354798017
24.34 0.00287101985281642
24.36 0.00286787399834473
24.38 0.00286491052132409
24.4 0.0028621228135781
24.42 0.00285950437027975
24.44 0.00285704878828739
24.46 0.00285474976455596
24.48 0.00285260109463041
24.5 0.00285059667120927
24.52 0.00284873048277431
24.54 0.00284699661229652
24.56 0.00284538923599754
24.58 0.00284390262217068
24.6 0.00284253113006613
24.62 0.00284126920882001
24.64 0.00284011139644153
24.66 0.00283905231884325
24.68 0.00283808668891628
24.7 0.00283720930564653
24.72 0.00283641505327009
24.74 0.00283569890046353
24.76 0.00283505589956574
24.78 0.00283448118583281
24.8 0.00283396997671813
24.82 0.00283351757117966
24.84 0.00283311934901027
24.86 0.0028327707701858
24.88 0.00283246737423741
24.9 0.00283220477963527
24.92 0.00283197868319281
24.94 0.0028317848594793
24.96 0.00283161916024852
24.98 0.00283147751387405
25 0.0028313559247943
25.02 0.00283125047296772
25.04 0.00283115731333037
25.06 0.00283107267525988
25.08 0.00283099286204337
25.1 0.00283091425034942
25.12 0.00283083328969957
25.14 0.00283074650194645
25.16 0.0028306504807425
25.18 0.00283054189101764
25.2 0.00283041746845463
25.22 0.0028302740189576
25.24 0.00283010841812549
25.26 0.00282991761071895
25.28 0.00282969861012591
25.3 0.00282944849782442
25.32 0.0028291644228412
25.34 0.00282884360120904
25.36 0.0028284833154172
25.38 0.00282808091386099
25.4 0.0028276338102852
25.42 0.00282713948322485
25.44 0.00282659547544211
25.46 0.00282599939335763
25.48 0.00282534890647965
25.5 0.00282464174682901
25.52 0.00282387570835666
25.54 0.00282304864636254
25.56 0.00282215847690794
25.58 0.00282120317622414
25.6 0.00282018078011576
25.62 0.00281908938336669
25.64 0.00281792713913547
25.66 0.0028166922583522
25.68 0.00281538300911168
25.7 0.00281399771606229
25.72 0.00281253475979444
25.74 0.00281099257622398
25.76 0.0028093696559754
25.78 0.00280766454376347
25.8 0.00280587583777156
25.82 0.00280400218902968
25.84 0.00280204230078714
25.86 0.00279999492789221
25.88 0.00279785887616205
25.9 0.00279563300175547
25.92 0.00279331621054891
25.94 0.00279090745750488
25.96 0.00278840574604254
25.98 0.00278581012741271
26 0.00278311970006757
26.02 0.00278033360903225
26.04 0.00277745104527824
26.06 0.00277447124509605
26.08 0.00277139348946916
26.1 0.00276821710345024
26.12 0.00276494145553496
26.14 0.00276156595704265
26.16 0.00275809006149211
26.18 0.00275451326398499
26.2 0.00275083510058738
26.22 0.00274705514771376
26.24 0.00274317302151485
26.26 0.00273918837726403
26.28 0.00273510090875077
26.3 0.0027309103476733
26.32 0.00272661646303473
26.34 0.00272221906054175
26.36 0.00271771798200668
26.38 0.0027131131047505
26.4 0.00270840434101183
26.42 0.00270359163735809
26.44 0.00269867497409767
26.46 0.00269365436469742
26.48 0.00268852985520384
26.5 0.0026833015236668
26.52 0.00267796947956607
26.54 0.00267253386324359
26.56 0.00266699484533606
26.58 0.002661352626216
26.6 0.00265560743543154
26.62 0.00264975953115456
26.64 0.00264380919962957
26.66 0.00263775675462591
26.68 0.0026316025368987
26.7 0.0026253469136498
26.72 0.0026189902779906
26.74 0.00261253304841678
26.76 0.0026059756682803
26.78 0.00259931860526817
26.8 0.00259256235088399
26.82 0.00258570741993747
26.84 0.00257875435003329
26.86 0.00257170370106801
26.88 0.00256455605472923
26.9 0.00255731201399981
26.92 0.00254997220266683
26.94 0.0025425372648342
26.96 0.00253500786444078
26.98 0.00252738468478186
27 0.00251966842803392
27.02 0.00251185981478757
27.04 0.00250395958358169
27.06 0.00249596849044103
27.08 0.00248788730842397
27.1 0.00247971682716767
27.12 0.00247145785244214
27.14 0.00246311120570763
27.16 0.00245467772367704
27.18 0.00244615825788069
27.2 0.00243755367423706
27.22 0.00242886485262926
27.24 0.00242009268648367
27.26 0.00241123808235268
27.28 0.00240230195950591
27.3 0.00239328524952007
27.32 0.00238418889587779
27.34 0.0023750138535688
27.36 0.00236576108869614
27.38 0.00235643157808503
27.4 0.00234702630889897
27.42 0.00233754627825802
27.44 0.00232799249286291
27.46 0.00231836596862175
27.48 0.00230866773028192
27.5 0.00229889881106652
27.52 0.00228906025231487
27.54 0.00227915310312675
27.56 0.00226917842001194
27.58 0.00225913726654248
27.6 0.00224903071301101
27.62 0.00223885983609124
27.64 0.00222862571850385
27.66 0.00221832944868648
27.68 0.00220797212046637
27.7 0.00219755483274075
27.72 0.0021870786891561
27.74 0.0021765447977957
27.76 0.00216595427086947
27.78 0.00215530822440738
27.8 0.00214460777795889
27.82 0.00213385405429304
27.84 0.00212304817910632
27.86 0.00211219128073073
27.88 0.0021012844898491
27.9 0.00209032893921161
27.92 0.00207932576335782
27.94 0.00206827609834129
27.96 0.00205718108145944
27.98 0.00204604185098572
28 0.00203485954590705
28.02 0.0020236353056633
28.04 0.00201237026989186
28.06 0.00200106557817496
28.08 0.00198972236979144
28.1 0.00197834178347174
28.12 0.0019669249571555
28.14 0.00195547302775557
28.16 0.0019439871309226
28.18 0.00193246840081442
28.2 0.00192091796986924
28.22 0.00190933696858172
28.24 0.00189772652528277
28.26 0.00188608776592222
28.28 0.00187442181385694
28.3 0.00186272978963916
28.32 0.0018510128108113
28.34 0.00183927199170137
28.36 0.0018275084432242
28.38 0.00181572327268408
28.4 0.00180391758358145
28.42 0.00179209247542347
28.44 0.00178024904353598
28.46 0.00176838837888063
28.48 0.00175651156787446
28.5 0.00174461969221126
28.52 0.00173271382868953
28.54 0.00172079504903924
28.56 0.00170886441975505
28.58 0.00169692300193045
28.6 0.00168497185109671
28.62 0.00167301201706291
28.64 0.00166104454376136
28.66 0.00164907046909305
28.68 0.00163709082477875
28.7 0.00162510663621125
28.72 0.00161311892231187
28.74 0.00160112869538785
28.76 0.0015891369609955
28.78 0.00157714471780294
28.8 0.00156515295745804
28.82 0.0015531626644586
28.84 0.00154117481602463
28.86 0.00152919038197326
28.88 0.00151721032459822
28.9 0.00150523559854973
28.92 0.00149326715071808
28.94 0.00148130592012029
28.96 0.00146935283778853
28.98 0.00145740882666159
29 0.00144547480147937
29.02 0.00143355166867884
29.04 0.00142164032629397
29.06 0.00140974166385646
29.08 0.0013978565623014
29.1 0.00138598589387267
29.12 0.0013741305220327
29.14 0.00136229130137462
29.16 0.00135046907753526
29.18 0.00133866468711287
29.2 0.00132687895758559
29.22 0.0013151127072324
29.24 0.00130336674505834
29.26 0.00129164187071865
29.28 0.00127993887444854
29.3 0.00126825853699369
29.32 0.00125660162954331
29.34 0.00124496891366524
29.36 0.00123336114124431
29.38 0.00122177905442199
29.4 0.00121022338553804
29.42 0.00119869485707605
29.44 0.0011871941816098
29.46 0.00117572206175174
29.48 0.00116427919010423
29.5 0.00115286624921372
29.52 0.00114148391152607
29.54 0.00113013283934282
29.56 0.00111881368478402
29.58 0.00110752708974727
29.6 0.00109627368587454
29.62 0.00108505409451727
29.64 0.00107386892670501
29.66 0.00106271878311727
29.68 0.0010516042540555
29.7 0.00104052591941895
29.72 0.00102948434868127
29.74 0.00101848010087036
29.76 0.00100751372455052
29.78 0.000996585757805208
29.8 0.000985696728224345
29.82 0.000974847152890959
29.84 0.000964037538373537
29.86 0.000953268380717519
29.88 0.000942540165439894
29.9 0.000931853367527663
29.92 0.000921208451435844
29.94 0.000910605871090541
29.96 0.00090004606989207
29.98 0.00088952948072165
30 0.000879056525950387
30.02 0.000868627617450649
30.04 0.000858243156609239
30.06 0.0008479035343444
30.08 0.000837609131123772
30.1 0.000827360316986348
30.12 0.000817157451566455
30.14 0.000807000884120562
30.16 0.000796890953556431
30.18 0.000786827988466164
30.2 0.0007768123071614
30.22 0.000766844217711589
30.24 0.000756924017985926
30.26 0.000747051995697801
30.28 0.000737228428453421
30.3 0.000727453583803155
30.32 0.000717727719297143
30.34 0.000708051082544168
30.36 0.000698423911274593
30.38 0.000688846433407933
30.4 0.000679318867123495
30.42 0.000669841420937047
30.44 0.000660414293781051
30.46 0.000651037675090611
30.48 0.00064171174489387
30.5 0.000632436673908932
30.52 0.000623212623646117
30.54 0.000614039746515851
30.56 0.000604918185944739
30.58 0.000595848076496428
30.6 0.000586829544002657
30.62 0.000577862705699856
30.64 0.000568947670375738
30.66 0.000560084538524074
30.68 0.000551273402510273
30.7 0.000542514346746838
30.72 0.00053380744788038
30.74 0.00052515277499054
30.76 0.000516550389802694
30.78 0.000508000346914806
30.8 0.000499502694039183
30.82 0.00049105747226222
30.84 0.000482664716321415
30.86 0.000474324454902235
30.88 0.000466036710958038
30.9 0.000457801502052394
30.92 0.000449618840728127
30.94 0.000441488734904363
30.96 0.000433411188305627
30.98 0.000425386200924742
31 0.000417413769523936
31.02 0.000409493888178071
31.04 0.000401626548864204
31.06 0.000393811742101699
31.08 0.000386049457650591
31.1 0.000378339685272563
31.12 0.000370682415562456
31.14 0.00036307764085819
31.16 0.000355525356239643
31.18 0.000348025560625101
31.2 0.000340578257978501
31.22 0.000333183458640322
31.24 0.000325841180800432
31.26 0.000318551452125972
31.28 0.0003113143115701
31.3 0.000304129811382393
31.32 0.000296998019349175
31.34 0.000289919021296417
31.36 0.000282892923890133
31.38 0.000275919857779882
31.4 0.000268999981132871
31.42 0.000262133483618938
31.44 0.000255320590913204
31.46 0.000248561569797511
31.48 0.000241856733955277
31.5 0.00023520645057033
31.52 0.000228611147861816
31.54 0.00022207132371174
31.56 0.000215587555569795
31.58 0.000209160511857813
31.6 0.000202790965137978
31.62 0.000196479807360915
31.64 0.000190228067575238
31.66 0.000184036932554403
31.68 0.000177907770892948
31.7 0.000171842161237536
31.72 0.000165841925457585
31.74 0.000159909167726151
31.76 0.000154046320686622
31.78 0.000148256200124569
31.8 0.000142542069849953
31.82 0.000136907718834588
31.84 0.000131357553035404
31.86 0.000125896704757227
31.88 0.000120531162848506
31.9 0.000115267927421952
31.92 0.000110115193060478
31.94 0.000105082564418829
31.96 0.000100181307484305
31.98 9.54246380255795e-05
32 9.08280452053615e-05
32.02 8.64096418683204e-05
32.04 8.21905221631644e-05
32.06 7.81950901721603e-05
32.08 7.44512985439447e-05
32.1 7.09907036154717e-05
32.12 6.7848206734483e-05
32.14 6.50613212450378e-05
32.16 6.26688022116304e-05
32.18 6.07085326015874e-05
32.2 5.92147046577364e-05
32.22 5.82145689889228e-05
32.24 5.77252852928947e-05
32.26 5.77515698816804e-05
32.28 5.8284761758554e-05
32.3 5.93035877191983e-05
32.32 6.07764291901607e-05
32.34 6.26645063624062e-05
32.36 6.49252637523652e-05
32.38 6.75153680446231e-05
32.4 7.03929900619625e-05
32.42 7.35192975845432e-05
32.44 7.68592551138812e-05
32.46 8.0381900090955e-05
32.48 8.40602699140042e-05
32.5 8.78711242729431e-05
32.52 9.17945677100974e-05
32.54 9.58136411879534e-05
32.56 9.99139236198239e-05
32.58 0.0001040831650312
32.6 0.000108310960681754
32.62 0.000112588468142815
32.64 0.000116908165326043
32.66 0.000121263645539037
32.68 0.000125649444958507
32.7 0.000130060897905039
32.72 0.000134494015646124
32.74 0.000138945384937387
32.76 0.00014341208303334
32.78 0.00014789160639911
32.8 0.000152381810800952
32.82 0.000156880860846104
32.84 0.000161387187372652
32.86 0.000165899451371666
32.88 0.000170416513351288
32.9 0.000174937407246416
32.92 0.000179461318133879
32.94 0.000183987563139442
32.96 0.000188515575031503
32.98 0.00019304488808248
33 0.000197575125848161
33.02 0.000202105990572969
33.04 0.000206637253981464
33.06 0.00021116874925065
33.08 0.000215700363994201
33.1 0.000220232034112346
33.12 0.000224763738389551
33.14 0.000229295493736733
33.16 0.000233827350988178
33.18 0.000238359391183433
33.2 0.000242891722267755
33.22 0.000247424476160134
33.24 0.000251957806139064
33.26 0.000256491884509971
33.28 0.000261026900517836
33.3 0.000265563058476196
33.32 0.000270100576086712
33.34 0.00027463968292804
33.36 0.00027918061909387
33.38 0.000283723633962313
33.4 0.000288268985085535
33.42 0.00029281693718253
33.44 0.000297367761226575
33.46 0.00030192173361691
33.48 0.000306479135425886
33.5 0.000311040251712344
33.52 0.000315605370898771
33.54 0.00032017478419997
33.56 0.000324748785104876
33.58 0.000329327668900337
33.6 0.000333911732237175
33.62 0.000338501272733117
33.64 0.000343096588608974
33.66 0.000347697978356079
33.68 0.000352305740430864
33.7 0.000356920172975969
33.72 0.000361541573563007
33.74 0.000366170238959365
33.76 0.000370806464912789
33.78 0.000375450545954176
33.8 0.000380102775217852
33.82 0.00038476344427652
33.84 0.000389432842990111
33.86 0.000394111259369835
33.88 0.000398798979449967
33.9 0.00040349628717547
33.92 0.000408203464297411
33.94 0.00041292079027772
33.96 0.000417648542204566
33.98 0.000422386994714581
34 0.000427136419923381
34.02 0.000431897087361862
34.04 0.000436669263922074
34.06 0.000441453213806314
34.08 0.000446249198482635
34.1 0.000451057476647374
34.12 0.000455878304189245
34.14 0.000460711934162532
34.16 0.000465558616760155
34.18 0.00047041859929405
34.2 0.000475292126176117
34.22 0.000480179438906784
34.24 0.000485080776062233
34.26 0.000489996373286682
34.28 0.00049492646328936
34.3 0.000499871275839007
34.32 0.000504831037765967
34.34 0.000509805972963268
34.36 0.000514796302389947
34.38 0.000519802244077563
34.4 0.000524824013137208
34.42 0.000529861821768447
34.44 0.000534915879268391
34.46 0.000539986392045271
34.48 0.000545073563629029
34.5 0.000550177594684801
34.52 0.000555298683028305
34.54 0.000560437023640963
34.56 0.000565592808684689
34.58 0.00057076622751854
34.6 0.000575957466716422
34.62 0.000581166710083017
34.64 0.000586394138672259
34.66 0.000591639930804232
34.68 0.000596904262084124
34.7 0.000602187305419168
34.72 0.000607489231036843
34.74 0.000612810206505003
34.76 0.000618150396746198
34.78 0.000623509964057786
34.8 0.000628889068128698
34.82 0.000634287866057443
34.84 0.000639706512367814
34.86 0.00064514515902754
34.88 0.000650603955462113
34.9 0.000656083048571863
34.92 0.000661582582748817
34.94 0.000667102699888319
34.96 0.000672643539407002
34.98 0.000678205238253057
35 0.000683787930922934
};
\addlegendentry{SUR$_1$}
\addplot [line width = \lineWidthSURError, color = SUR2, densely dotted]
table {%
0 0
0.02 6.85222728310329e-06
0.04 1.36450157816953e-05
0.06 2.03803845828284e-05
0.08 2.70603809869551e-05
0.1 3.36870800047963e-05
0.12 4.02625838051677e-05
0.14 4.67890211126661e-05
0.16 5.3268546553886e-05
0.18 5.97033399540599e-05
0.2 6.60956055801471e-05
0.22 7.24475713327994e-05
0.24 7.87614878875018e-05
0.26 8.50396277826511e-05
0.28 9.12842844573195e-05
0.3 9.74977712384191e-05
0.32 0.000103682420279454
0.34 0.000109840581449169
0.36 0.000115974621173308
0.38 0.000122086921233635
0.4 0.000128179877520006
0.42 0.000134255898741564
0.44 0.000140317405100729
0.46 0.000146366826928458
0.48 0.000152406603286305
0.5 0.000158439180537332
0.52 0.000164467010888983
0.54 0.000170492550910857
0.56 0.000176518260033479
0.58 0.000182546599026537
0.6 0.000188580028465985
0.62 0.000194621007190375
0.64 0.000200671990751864
0.66 0.000206735429864504
0.68 0.000212813768856929
0.7 0.00021890944412962
0.72 0.000225024882624877
0.74 0.000231162500309763
0.76 0.000237324700680383
0.78 0.000243513873287305
0.8 0.000249732392288664
0.82 0.00025598261503621
0.84 0.000262266880691542
0.86 0.000268587508884157
0.88 0.000274946798410228
0.9 0.000281347025975498
0.92 0.00028779044498641
0.94 0.000294279284390005
0.96 0.000300815747569641
0.98 0.000307402011293989
1 0.000314040224722514
1.02 0.000320732508470445
1.04 0.000327480953737646
1.06 0.000334287621492152
1.08 0.000341154541720208
1.1 0.000348083712740811
1.12 0.000355077100580445
1.14 0.000362136638413607
1.16 0.000369264226065311
1.18 0.000376461729578111
1.2 0.000383730980840745
1.22 0.000391073777278303
1.24 0.000398491881603772
1.26 0.000405987021628352
1.28 0.0004135608901303
1.3 0.000421215144778937
1.32 0.0004289514081134
1.34 0.000436771267578482
1.36 0.000444676275603958
1.38 0.000452667949735783
1.4 0.000460747772818639
1.42 0.000468917193214302
1.44 0.000477177625066345
1.46 0.000485530448605202
1.48 0.000493977010489265
1.5 0.000502518624180939
1.52 0.000511156570349751
1.54 0.00051989209731923
1.56 0.000528726421525493
1.58 0.000537660728012304
1.6 0.000546696170939994
1.62 0.000555833874124253
1.64 0.00056507493158865
1.66 0.000574420408131441
1.68 0.00058387133991129
1.7 0.000593428735040376
1.72 0.00060309357419294
1.74 0.000612866811218429
1.76 0.000622749373761086
1.78 0.000632742163888174
1.8 0.000642846058719497
1.82 0.000653061911054471
1.84 0.000663390550008616
1.86 0.000673832781641061
1.88 0.00068438938958408
1.9 0.000695061135668111
1.92 0.000705848760543007
1.94 0.00071675298429222
1.96 0.000727774507042338
1.98 0.000738914009564706
2 0.000750172153869098
2.02 0.000761549583789544
2.04 0.000773046925557725
2.06 0.000784664788371189
2.08 0.000796403764948558
2.1 0.000808264432072009
2.12 0.000820247351122176
2.14 0.000832353068600267
2.16 0.000844582116636798
2.18 0.00085693501349036
2.2 0.000869412264032127
2.22 0.000882014360220255
2.24 0.000894741781559509
2.26 0.000907594995549774
2.28 0.000920574458119545
2.3 0.000933680614052777
2.32 0.000946913897394036
2.34 0.000960274731849784
2.36 0.000973763531168823
2.38 0.000987380699520117
2.4 0.00100112663184995
2.42 0.00101500171423118
2.44 0.0010290063241961
2.46 0.00104314083106475
2.48 0.00105740559625564
2.5 0.00107180097358621
2.52 0.00108632730956258
2.54 0.00110098494365869
2.56 0.0011157742085841
2.58 0.00113069543054005
2.6 0.00114574892946821
2.62 0.00116093501928268
2.64 0.00117625400810167
2.66 0.00119170619846125
2.68 0.00120729188752143
2.7 0.00122301136726783
2.72 0.00123886492469773
2.74 0.00125485284199941
2.76 0.00127097539672675
2.78 0.00128723286195775
2.8 0.00130362550645933
2.82 0.00132015359482355
2.84 0.00133681738760957
2.86 0.00135361714148379
2.88 0.00137055310933486
2.9 0.00138762554040187
2.92 0.00140483468037463
2.94 0.00142218077150539
2.96 0.00143966405271084
2.98 0.0014572847596557
3 0.00147504312484617
3.02 0.00149293937770857
3.04 0.00151097374466528
3.06 0.00152914644920183
3.08 0.00154745771193872
3.1 0.00156590775068271
3.12 0.00158449678048851
3.14 0.00160322501370956
3.16 0.00162209266003643
3.18 0.00164109992654875
3.2 0.00166024701774957
3.22 0.00167953413559292
3.24 0.0016989614795222
3.26 0.00171852924649277
3.28 0.00173823763099666
3.3 0.00175808682507876
3.32 0.00177807701835412
3.34 0.00179820839802268
3.36 0.00181848114887921
3.38 0.00183889545331968
3.4 0.00185945149134383
3.42 0.00188014944055958
3.44 0.00190098947618002
3.46 0.00192197177102324
3.48 0.00194309649550639
3.5 0.00196436381763516
3.52 0.00198577390299495
3.54 0.00200732691474418
3.56 0.00202902301359346
3.58 0.00205086235779214
3.6 0.00207284510311574
3.62 0.0020949714028422
3.64 0.00211724140773149
3.66 0.00213965526600976
3.68 0.00216221312333754
3.7 0.00218491512279254
3.72 0.00220776140483941
3.74 0.00223075210730386
3.76 0.00225388736534447
3.78 0.00227716731142577
3.8 0.00230059207528691
3.82 0.00232416178391077
3.84 0.00234787656148884
3.86 0.00237173652939568
3.88 0.00239574180614824
3.9 0.00241989250737645
3.92 0.00244418874578952
3.94 0.00246863063113772
3.96 0.00249321827017614
3.98 0.0025179517666317
4 0.00254283122116659
4.02 0.00256785673133912
4.04 0.00259302839156532
4.06 0.00261834629308601
4.08 0.00264381052392647
4.1 0.00266942116885939
4.12 0.00269517830936175
4.14 0.00272108202358739
4.16 0.00274713238631954
4.18 0.00277332946893142
4.2 0.00279967333935476
4.22 0.00282616406204104
4.24 0.00285280169791572
4.26 0.00287958630434694
4.28 0.00290651793510387
4.3 0.00293359664032227
4.32 0.00296082246646125
4.34 0.0029881954562723
4.36 0.0030157156487562
4.38 0.00304338307913057
4.4 0.00307119777878675
4.42 0.00309915977525857
4.44 0.00312726909218458
4.46 0.00315552574927544
4.48 0.0031839297622674
4.5 0.00321248114290175
4.52 0.00324117989887932
4.54 0.00327002603382918
4.56 0.00329901954727782
4.58 0.00332816043461174
4.6 0.00335744868704298
4.62 0.00338688429158531
4.64 0.0034164672310109
4.66 0.00344619748382524
4.68 0.0034760750242388
4.7 0.00350609982212629
4.72 0.00353627184300728
4.74 0.00356659104801438
4.76 0.00359705739386006
4.78 0.00362767083281228
4.8 0.00365843131266713
4.82 0.00368933877672006
4.84 0.0037203931637425
4.86 0.00375159440794966
4.88 0.00378294243898581
4.9 0.00381443718189271
4.92 0.00384607855708567
4.94 0.00387786648033488
4.96 0.00390980086273851
4.98 0.0039418816107035
5 0.00397410862592692
5.02 0.00400648180537231
5.04 0.00403900104124994
5.06 0.00407166622099922
5.08 0.0041044772272714
5.1 0.00413743393791236
5.12 0.00417053622594324
5.14 0.00420378395954556
5.16 0.00423717700204915
5.18 0.00427071521191395
5.2 0.00430439844271928
5.22 0.00433822654314676
5.24 0.00437219935697266
5.26 0.00440631672305353
5.28 0.00444057847531908
5.3 0.00447498444275406
5.32 0.00450953444940011
5.34 0.00454422831434075
5.36 0.00457906585169225
5.38 0.00461404687060037
5.4 0.004649171175236
5.42 0.0046844385647842
5.44 0.00471984883344361
5.46 0.00475540177041991
5.48 0.00479109715992416
5.5 0.00482693478117411
5.52 0.00486291440838209
5.54 0.00489903581076871
5.56 0.00493529875255052
5.58 0.00497170299294513
5.6 0.00500824828617457
5.62 0.00504493438146606
5.64 0.00508176102305367
5.66 0.00511872795018334
5.68 0.00515583489711697
5.7 0.00519308159314035
5.72 0.00523046776256153
5.74 0.00526799312472575
5.76 0.00530565739402048
5.78 0.00534346027987892
5.8 0.00538140148679587
5.82 0.00541948071433283
5.84 0.00545769765713042
5.86 0.00549605200491755
5.88 0.00553454344252534
5.9 0.00557317164989808
5.92 0.00561193630210758
5.94 0.00565083706936795
5.96 0.00568987361704322
5.98 0.00572904560567297
6 0.00576835269098367
6.02 0.0058077945239056
6.04 0.00584737075058569
6.06 0.00588708101241365
6.08 0.00592692494603636
6.1 0.0059669021833778
6.12 0.0060070123516571
6.14 0.0060472550734155
6.16 0.00608762996653166
6.18 0.00612813664424343
6.2 0.00616877471517656
6.22 0.00620954378336038
6.24 0.00625044344825992
6.26 0.00629147330479497
6.28 0.00633263294336596
6.3 0.00637392194987991
6.32 0.00641533990577846
6.34 0.00645688638806347
6.36 0.00649856096932735
6.38 0.00654036321777757
6.4 0.00658229269726459
6.42 0.00662434896732143
6.44 0.00666653158317818
6.46 0.00670884009580284
6.48 0.0067512740519373
6.5 0.00679383299411203
6.52 0.0068365164606947
6.54 0.00687932398591542
6.56 0.00692225509989861
6.58 0.00696530932870333
6.6 0.00700848619435254
6.62 0.00705178521487099
6.64 0.0070952059043174
6.66 0.00713874777282538
6.68 0.00718241032663268
6.7 0.00722619306812921
6.72 0.00727009549587981
6.74 0.00731411710467664
6.76 0.00735825738556716
6.78 0.00740251582589929
6.8 0.00744689190935738
6.82 0.00749138511600454
6.84 0.00753599492231992
6.86 0.00758072080124359
6.88 0.00762556222221449
6.9 0.0076705186512103
6.92 0.00771558955079446
6.94 0.00776077438015398
6.96 0.00780607259514907
6.98 0.00785148364834571
7 0.00789700698907051
7.02 0.00794264206344322
7.04 0.00798838831443289
7.06 0.00803424518189595
7.08 0.00808021210262284
7.1 0.00812628851037998
7.12 0.00817247383596178
7.14 0.00821876750723234
7.16 0.00826516894917555
7.18 0.00831167758393711
7.2 0.00835829283087938
7.22 0.00840501410661862
7.24 0.0084518408250794
7.26 0.00849877239754295
7.28 0.00854580823269038
7.3 0.00859294773666232
7.32 0.00864019031309452
7.34 0.0086875353631742
7.36 0.00873498228569192
7.38 0.00878253047708565
7.4 0.00883017933149258
7.42 0.00887792824079735
7.44 0.00892577659469223
7.46 0.00897372378071657
7.48 0.00902176918431456
7.5 0.00906991218887639
7.52 0.00911815217580601
7.54 0.00916648852455816
7.56 0.00921492061269846
7.58 0.00926344781595192
7.6 0.00931206950825455
7.62 0.00936078506181024
7.64 0.00940959384713627
7.66 0.00945849523312087
7.68 0.00950748858707437
7.7 0.00955657327478291
7.72 0.0096057486605608
7.74 0.00965501410730611
7.76 0.00970436897654717
7.78 0.00975381262850569
7.8 0.00980334442214033
7.82 0.00985296371521068
7.84 0.0099026698643177
7.86 0.00995246222497578
7.88 0.0100023401516434
7.9 0.0100523029978042
7.92 0.0101023501159947
7.94 0.0101524808578735
7.96 0.0102026945742763
7.98 0.0102529906152632
8 0.0103033683301743
8.02 0.0103538270676844
8.04 0.0104043661758657
8.06 0.0104549850022215
8.08 0.0105056828937656
8.1 0.0105564591970587
8.12 0.0106073132582672
8.14 0.0106582444232205
8.16 0.0107092520374626
8.18 0.0107603354463054
8.2 0.0108114939948879
8.22 0.0108627270282228
8.24 0.0109140338912545
8.26 0.0109654139289144
8.28 0.011016866486172
8.3 0.0110683909080962
8.32 0.0111199865398931
8.34 0.0111716527269754
8.36 0.0112233888150111
8.38 0.0112751941499708
8.4 0.0113270680781853
8.42 0.0113790099464086
8.44 0.0114310191018532
8.46 0.0114830948922547
8.48 0.0115352366659291
8.5 0.0115874437718052
8.52 0.0116397155594947
8.54 0.0116920513793458
8.56 0.0117444505824848
8.58 0.0117969125208733
8.6 0.0118494365473631
8.62 0.0119020220157412
8.64 0.0119546682807844
8.66 0.0120073746983161
8.68 0.0120601406252453
8.7 0.0121129654196302
8.72 0.0121658484407182
8.74 0.0122187890490038
8.76 0.012271786606279
8.78 0.0123248404756763
8.8 0.0123779500217183
8.82 0.0124311146103848
8.84 0.0124843336091402
8.86 0.0125376063869877
8.88 0.0125909323145223
8.9 0.0126443107639848
8.92 0.0126977411092942
8.94 0.0127512227261136
8.96 0.0128047549918784
8.98 0.0128583372858619
9 0.0129119689892104
9.02 0.0129656494849901
9.04 0.013019378158243
9.06 0.0130731543960208
9.08 0.0131269775874336
9.1 0.013180847123707
9.12 0.0132347623982049
9.14 0.0132887228064942
9.16 0.0133427277463799
9.18 0.0133967766179466
9.2 0.0134508688236134
9.22 0.0135050037681589
9.24 0.0135591808587772
9.26 0.0136133995051203
9.28 0.0136676591193283
9.3 0.0137219591160832
9.32 0.0137762989126481
9.34 0.0138306779288943
9.36 0.013885095587366
9.38 0.0139395513132973
9.4 0.0139940445346634
9.42 0.0140485746822108
9.44 0.0141031411895075
9.46 0.0141577434929714
9.48 0.0142123810319112
9.5 0.0142670532485606
9.52 0.0143217595881228
9.54 0.0143764994987947
9.56 0.0144312724318094
9.58 0.0144860778414755
9.6 0.0145409151852003
9.62 0.0145957839235284
9.64 0.0146506835201846
9.66 0.0147056134420978
9.68 0.0147605731594271
9.7 0.0148155621456091
9.72 0.0148705798773749
9.74 0.0149256258348
9.76 0.0149806995013082
9.78 0.0150358003637194
9.8 0.0150909279122808
9.82 0.0151460816406866
9.84 0.0152012610461073
9.86 0.015256465629217
9.88 0.0153116948942355
9.9 0.0153669483489241
9.92 0.0154222255046421
9.94 0.0154775258763528
9.96 0.0155328489826647
9.98 0.0155881943458324
10 0.0156435614917984
10.02 0.0156989499502139
10.04 0.0157543592544538
10.06 0.0158097889416441
10.08 0.0158652385526788
10.1 0.0159207076322477
10.12 0.0159761957288472
10.14 0.0160317023948035
10.16 0.016087227186291
10.18 0.0161427696633544
10.2 0.0161983293899158
10.22 0.0162539059338009
10.24 0.0163094988667462
10.26 0.0163651077644268
10.28 0.0164207322064584
10.3 0.0164763717764165
10.32 0.0165320260618531
10.34 0.0165876946543006
10.36 0.0166433771492908
10.38 0.0166990731463632
10.4 0.0167547822490795
10.42 0.0168105040650259
10.44 0.0168662382058291
10.46 0.0169219842871652
10.48 0.0169777419287584
10.5 0.0170335107544028
10.52 0.0170892903919565
10.54 0.017145080473347
10.56 0.0172008806345847
10.58 0.0172566905157624
10.6 0.0173125097610623
10.62 0.0173683380187537
10.64 0.0174241749411954
10.66 0.017480020184844
10.68 0.017535873410241
10.7 0.0175917342820306
10.72 0.0176476024689516
10.74 0.0177034776438265
10.76 0.0177593594835702
10.78 0.0178152476691866
10.8 0.0178711418857613
10.82 0.0179270418224534
10.84 0.0179829471725071
10.86 0.0180388576332217
10.88 0.0180947729059582
10.9 0.0181506926961419
10.92 0.0182066167132345
10.94 0.0182625446707382
10.96 0.0183184762861811
10.98 0.0183744112811026
11 0.0184303493810608
11.02 0.0184862903156012
11.04 0.0185422338182431
11.06 0.0185981796264878
11.08 0.0186541274817811
11.1 0.0187100771295129
11.12 0.0187660283189933
11.14 0.0188219808034479
11.16 0.0188779343399873
11.18 0.0189338886895919
11.2 0.0189898436171037
11.22 0.0190457988911972
11.24 0.0191017542843625
11.26 0.0191577095728844
11.28 0.0192136645368194
11.3 0.019269618959977
11.32 0.0193255726298954
11.34 0.0193815253378171
11.36 0.0194374768786631
11.38 0.0194934270510084
11.4 0.0195493756570652
11.42 0.0196053225026379
11.44 0.0196612673971047
11.46 0.0197172101534075
11.48 0.0197731505879932
11.5 0.019829088520798
11.52 0.0198850237752199
11.54 0.019940956178084
11.56 0.0199968855596099
11.58 0.0200528117533859
11.6 0.0201087345963263
11.62 0.0201646539286451
11.64 0.0202205695938208
11.66 0.0202764814385523
11.68 0.0203323893127338
11.7 0.0203882930694143
11.72 0.0204441925647643
11.74 0.0205000876580218
11.76 0.0205559782114831
11.78 0.0206118640904343
11.8 0.0206677451631267
11.82 0.0207236213007303
11.84 0.0207794923773012
11.86 0.0208353582697256
11.88 0.0208912188576915
11.9 0.0209470740236356
11.92 0.0210029236527058
11.94 0.0210587676327055
11.96 0.0211146058540684
11.98 0.0211704382097933
12 0.0212262645954085
12.02 0.0212820849089163
12.04 0.0213378990507531
12.06 0.0213937069237418
12.08 0.0214495084330411
12.1 0.0215053034860881
12.12 0.0215610919925599
12.14 0.0216168738643127
12.16 0.0216726490153438
12.18 0.0217284173617229
12.2 0.0217841788215609
12.22 0.0218399333149334
12.24 0.0218956807638435
12.26 0.0219514210921642
12.28 0.0220071542255845
12.3 0.0220628800915495
12.32 0.0221185986192155
12.34 0.0221743097393786
12.36 0.0222300133844263
12.38 0.0222857094882919
12.4 0.0223413979863767
12.42 0.0223970788155028
12.44 0.0224527519138586
12.46 0.022508417220933
12.48 0.022564074677457
12.5 0.0226197242253456
12.52 0.022675365807638
12.54 0.0227309993684397
12.56 0.022786624852852
12.58 0.022842242206918
12.6 0.0228978513775711
12.62 0.0229534523125512
12.64 0.0230090449603543
12.66 0.0230646292701718
12.68 0.0231202051918238
12.7 0.0231757726756951
12.72 0.0232313316726767
12.74 0.0232868821340932
12.76 0.0233424240116449
12.78 0.0233979572573407
12.8 0.0234534818234352
12.82 0.023508997662369
12.84 0.0235645047266858
12.86 0.023620002968979
12.88 0.0236754923418349
12.9 0.0237309727977516
12.92 0.0237864442890701
12.94 0.0238419067679255
12.96 0.0238973601861743
12.98 0.0239528044953094
13 0.02400823964642
13.02 0.0240636655901045
13.04 0.0241190822764198
13.06 0.0241744896548031
13.08 0.0242298876740037
13.1 0.0242852762820246
13.12 0.0243406554260461
13.14 0.0243960250523655
13.16 0.0244513851063187
13.18 0.0245067355322285
13.2 0.024562076273328
13.22 0.0246174072716917
13.24 0.0246727284681643
13.26 0.0247280398023154
13.28 0.0247833412123385
13.3 0.0248386326350124
13.32 0.0248939140056192
13.34 0.0249491852578773
13.36 0.0250044463238842
13.38 0.0250596971340391
13.4 0.0251149376169855
13.42 0.0251701676995359
13.44 0.0252253873066057
13.46 0.0252805963611515
13.48 0.025335794784114
13.5 0.0253909824943294
13.52 0.0254461594084838
13.54 0.0255013254410415
13.56 0.0255564805041749
13.58 0.0256116245077142
13.6 0.0256667573590612
13.62 0.0257218789631471
13.64 0.0257769892223533
13.66 0.0258320880364614
13.68 0.0258871753025795
13.7 0.0259422509150774
13.72 0.0259973147655429
13.74 0.0260523667426973
13.76 0.0261074067323549
13.78 0.026162434617339
13.8 0.0262174502774479
13.82 0.0262724535893764
13.84 0.0263274444266626
13.86 0.0263824226596325
13.88 0.0264373881553327
13.9 0.0264923407774833
13.92 0.0265472803864195
13.94 0.0266022068390263
13.96 0.0266571199886929
13.98 0.0267120196852547
14 0.0267669057749355
14.02 0.0268217781003051
14.04 0.0268766365002081
14.06 0.0269314808097257
14.08 0.0269863108601231
14.1 0.0270411264787966
14.12 0.0270959274892196
14.14 0.0271507137109058
14.16 0.0272054849593454
14.18 0.0272602410459713
14.2 0.0273149817781005
14.22 0.0273697069589054
14.24 0.0274244163873521
14.26 0.0274791098581694
14.28 0.0275337871617934
14.3 0.0275884480843424
14.32 0.0276430924075584
14.34 0.0276977199087813
14.36 0.0277523303609079
14.38 0.0278069235323471
14.4 0.0278614991869888
14.42 0.0279160570841708
14.44 0.0279705969786408
14.46 0.0280251186205209
14.48 0.0280796217552866
14.5 0.0281341061237226
14.52 0.0281885714619026
14.54 0.0282430175011598
14.56 0.0282974439680556
14.58 0.0283518505843576
14.6 0.0284062370670139
14.62 0.0284606031281343
14.64 0.0285149484749615
14.66 0.0285692728098525
14.68 0.0286235758302667
14.7 0.0286778572287433
14.72 0.0287321166928814
14.74 0.0287863539053299
14.76 0.0288405685437785
14.78 0.0288947602809367
14.8 0.0289489287845285
14.82 0.0290030737172886
14.84 0.0290571947369481
14.86 0.0291112914962331
14.88 0.0291653636428579
14.9 0.029219410819529
14.92 0.0292734326639399
14.94 0.029327428808771
14.96 0.0293813988816983
14.98 0.0294353425053921
15 0.0294892592975337
15.02 0.0295431488708113
15.04 0.0295970108329357
15.06 0.0296508447866527
15.08 0.0297046503297564
15.1 0.029758427055111
15.12 0.0298121745506589
15.14 0.0298658923994453
15.16 0.0299195801796397
15.18 0.0299732374645597
15.2 0.0300268638226955
15.22 0.0300804588177399
15.24 0.0301340220086137
15.26 0.0301875529495043
15.28 0.0302410511898923
15.3 0.0302945162745975
15.32 0.0303479477438074
15.34 0.0304013451331205
15.36 0.0304547079735955
15.38 0.0305080357917839
15.4 0.0305613281097904
15.42 0.0306145844453177
15.44 0.0306678043117128
15.46 0.0307209872180301
15.48 0.0307741326690884
15.5 0.0308272401655209
15.52 0.0308803092038423
15.54 0.0309333392765131
15.56 0.03098632987201
15.58 0.0310392804748875
15.6 0.0310921905658546
15.62 0.0311450596218432
15.64 0.0311978871160999
15.66 0.0312506725182467
15.68 0.0313034152943812
15.7 0.0313561149071419
15.72 0.0314087708158088
15.74 0.031461382476399
15.76 0.0315139493417473
15.78 0.0315664708616064
15.8 0.0316189464827496
15.82 0.0316713756490688
15.84 0.031723757801685
15.86 0.031776092379047
15.88 0.0318283788170472
15.9 0.0318806165491363
15.92 0.0319328050064389
15.94 0.0319849436178731
15.96 0.0320370318102724
15.98 0.0320890690085156
16 0.0321410546356472
16.02 0.032192988113023
16.04 0.0322448688604411
16.06 0.0322966962962694
16.08 0.0323484698376031
16.1 0.0324001889004014
16.12 0.0324518528996411
16.14 0.032503461249466
16.16 0.0325550133633426
16.18 0.0326065086542263
16.2 0.032657946534709
16.22 0.0327093264171999
16.24 0.0327606477140925
16.26 0.0328119098379313
16.28 0.032863112201593
16.3 0.0329142542184733
16.32 0.0329653353026572
16.34 0.0330163548691216
16.36 0.0330673123339195
16.38 0.0331182071143747
16.4 0.0331690386292903
16.42 0.0332198062991409
16.44 0.0332705095462863
16.46 0.03332114779518
16.48 0.0333717204725913
16.5 0.0334222270078154
16.52 0.0334726668329067
16.54 0.033523039382892
16.56 0.0335733440960234
16.58 0.0336235804139915
16.6 0.0336737477821811
16.62 0.033723845649909
16.64 0.0337738734706756
16.66 0.0338238307024143
16.68 0.0338737168077535
16.7 0.0339235312542766
16.72 0.0339732735147886
16.74 0.0340229430675839
16.76 0.0340725393967278
16.78 0.0341220619923231
16.8 0.034171510350812
16.82 0.034220883975259
16.84 0.0342701823756335
16.86 0.0343194050691239
16.88 0.0343685515804335
16.9 0.0344176214420852
16.92 0.0344666141947448
16.94 0.0345155293875246
16.96 0.0345643665783277
16.98 0.0346131253341473
17 0.0346618052314203
17.02 0.034710405856359
17.04 0.0347589268052894
17.06 0.0348073676850078
17.08 0.0348557281131261
17.1 0.0349040077184355
17.12 0.034952206141264
17.14 0.0350003230338523
17.16 0.0350483580607156
17.18 0.0350963108990388
17.2 0.0351441812390538
17.22 0.0351919687844248
17.24 0.0352396732526499
17.26 0.0352872943754668
17.28 0.0353348318992437
17.3 0.0353822855854055
17.32 0.0354296552108472
17.34 0.0354769405683554
17.36 0.0355241414670426
17.38 0.0355712577327769
17.4 0.0356182892086181
17.42 0.0356652357552821
17.44 0.0357120972515714
17.46 0.0357588735948466
17.48 0.0358055647014888
17.5 0.0358521705073587
17.52 0.0358986909682925
17.54 0.035945126060558
17.56 0.0359914757813673
17.58 0.0360377401493553
17.6 0.0360839192050772
17.62 0.0361300130115309
17.64 0.036176021654651
17.66 0.0362219452438325
17.68 0.0362677839124645
17.7 0.0363135378184473
17.72 0.0363592071447375
17.74 0.0364047920998844
17.76 0.0364502929185863
17.78 0.0364957098622394
17.8 0.0365410432195048
17.82 0.0365862933068697
17.84 0.0366314604692269
17.86 0.0366765450804579
17.88 0.0367215475440127
17.9 0.0367664682935054
17.92 0.0368113077933154
17.94 0.0368560665391868
17.96 0.0369007450588596
17.98 0.03694534391266
18 0.0369898636941508
18.02 0.0370343050307455
18.04 0.0370786685843526
18.06 0.0371229550520202
18.08 0.0371671651665886
18.1 0.0372112996973362
18.12 0.037255359450657
18.14 0.037299345270713
18.16 0.0373432580401289
18.18 0.0373870986806563
18.2 0.0374308681538719
18.22 0.0374745674618673
18.24 0.0375181976479444
18.26 0.0375617597973369
18.28 0.0376052550379097
18.3 0.0376486845408725
18.32 0.0376920495215222
18.34 0.0377353512399557
18.36 0.0377785910018177
18.38 0.0378217701590319
18.4 0.0378648901105571
18.42 0.0379079523031392
18.44 0.0379509582320622
18.46 0.0379939094419266
18.48 0.0380368075274019
18.5 0.0380796541340185
18.52 0.0381224509589387
18.54 0.0381651997517439
18.56 0.0382079023152228
18.58 0.0382505605061711
18.6 0.0382931762361905
18.62 0.0383357514724967
18.64 0.0383782882387195
18.66 0.0384207886157313
18.68 0.0384632547424527
18.7 0.0385056888166861
18.72 0.03854809309593
18.74 0.0385904698982239
18.76 0.0386328216029706
18.78 0.0386751506517833
18.8 0.0387174595493234
18.82 0.0387597508641411
18.84 0.038802027229531
18.86 0.038844291344376
18.88 0.0388865459740057
18.9 0.038928793951042
18.92 0.0389710381762765
18.94 0.0390132816195053
18.96 0.0390555273204111
18.98 0.0390977783894089
19 0.0391400380085291
19.02 0.0391823094322634
19.04 0.039224595988445
19.06 0.0392669010791047
19.08 0.039309228181339
19.1 0.0393515808481817
19.12 0.0393939627094656
19.14 0.0394363774726856
19.16 0.0394788289238647
19.18 0.0395213209284068
19.2 0.0395638574319749
19.22 0.0396064424613315
19.24 0.039649080125204
19.26 0.0396917746151265
19.28 0.0397345302063064
19.3 0.0397773512584496
19.32 0.0398202422166109
19.34 0.0398632076120347
19.36 0.0399062520629722
19.38 0.0399493802755202
19.4 0.0399925970444401
19.42 0.0400359072539559
19.44 0.0400793158785781
19.46 0.040122827983897
19.48 0.040166448727368
19.5 0.0402101833590992
19.52 0.0402540372226239
19.54 0.0402980157556604
19.56 0.0403421244908754
19.58 0.0403863690566132
19.6 0.0404307551776421
19.62 0.040475288675863
19.64 0.0405199754710212
19.66 0.0405648215814018
19.68 0.0406098331245065
19.7 0.0406550163177288
19.72 0.0407003774789935
19.74 0.0407459230273991
19.76 0.0407916594838403
19.78 0.0408375934716007
19.8 0.0408837317169492
19.82 0.0409300810497005
19.84 0.0409766484037548
19.86 0.0410234408176479
19.88 0.0410704654350311
19.9 0.0411177295051766
19.92 0.0411652403834271
19.94 0.0412130055316486
19.96 0.0412610325186431
19.98 0.0413093290205339
20 0.04135790282115
20.02 0.0414067618123524
20.04 0.0414559139943585
20.06 0.0415053674760199
20.08 0.0415551304750919
20.1 0.0416052113184608
20.12 0.0416556184423404
20.14 0.0417063603924375
20.16 0.0417574458241068
20.18 0.041808883502439
20.2 0.0418606823023258
20.22 0.0419128512085285
20.24 0.0419653993156615
20.26 0.0420183358281482
20.28 0.0420716700601593
20.3 0.0421254114355216
20.32 0.0421795694875506
20.34 0.0422341538588868
20.36 0.0422891743012447
20.38 0.0423446406751776
20.4 0.0424005629497498
20.42 0.0424569512021869
20.44 0.0425138156175036
20.46 0.0425711664880477
20.48 0.0426290142130275
20.5 0.0426873692979854
20.52 0.0427462423542198
20.54 0.0428056440981773
20.56 0.0428655853507855
20.58 0.0429260770367218
20.6 0.0429871301836705
20.62 0.0430487559214989
20.64 0.0431109654813911
20.66 0.0431737701949398
20.68 0.0432371814931885
20.7 0.043301210905589
20.72 0.0433658700589567
20.74 0.0434311706763344
20.76 0.0434971245758248
20.78 0.0435637436693478
20.8 0.0436310399613701
20.82 0.0436990255475611
20.84 0.0437677126133894
20.86 0.043837113432692
20.88 0.0439072403661452
20.9 0.0439781058597267
20.92 0.0440497224430872
20.94 0.0441221027278847
20.96 0.0441952594060468
20.98 0.0442692052479793
21 0.0443439531007535
21.02 0.044419515886175
21.04 0.0444959065988445
21.06 0.0445731383041495
21.08 0.0446512241361917
21.1 0.0447301772956646
21.12 0.0448100110476854
21.14 0.0448907387195559
21.16 0.0449723736984683
21.18 0.0450549294291738
21.2 0.0451384194115795
21.22 0.0452228571983074
21.24 0.0453082563921765
21.26 0.0453946306436748
21.28 0.0454819936483246
21.3 0.0455703591440519
21.32 0.0456597409084631
21.34 0.0457501527561024
21.36 0.045841608535633
21.38 0.0459341221269942
21.4 0.0460277074385166
21.42 0.0461223784039637
21.44 0.0462181489795641
21.46 0.0463150331409561
21.48 0.046413044880149
21.5 0.0465121982023988
21.52 0.0466125071230524
21.54 0.0467139856643806
21.56 0.0468166478523296
21.58 0.0469205077132917
21.6 0.0470255792708021
21.62 0.0471318765422123
21.64 0.0472394135353588
21.66 0.0473482042451682
21.68 0.0474582626502465
21.7 0.0475696027094687
21.72 0.0476822383585191
21.74 0.0477961835064075
21.76 0.0479114520319778
21.78 0.0480280577804163
21.8 0.0481460145597014
21.82 0.0482653361370795
21.84 0.0483860362355185
21.86 0.0485081285301214
21.88 0.0486316266445896
21.9 0.0487565441476367
21.92 0.0488828945494064
21.94 0.0490106912978912
21.96 0.0491399477753851
21.98 0.0492706772948656
22 0.0494028930964525
22.02 0.0495366083438328
22.04 0.0496718361207256
22.06 0.0498085894272951
22.08 0.0499468811766651
22.1 0.0500867241913847
22.12 0.0502281311999244
22.14 0.050371114833209
22.16 0.050515687621141
22.18 0.0506618619891838
22.2 0.0508096502549406
22.22 0.050959064624778
22.24 0.051110117190454
22.26 0.051262819925801
22.28 0.0514171846834663
22.3 0.0515732231915741
22.32 0.0517309470505802
22.34 0.0518903677300505
22.36 0.0520514965655175
22.38 0.0522143447553718
22.4 0.0523789233578132
22.42 0.0525452432878202
22.44 0.0527133153141706
22.46 0.0528831500565073
22.48 0.0530547579824833
22.5 0.0532281494049011
22.52 0.0534033344789511
22.54 0.0535803231994735
22.56 0.053759125398271
22.58 0.0539397507415031
22.6 0.0541222087270969
22.62 0.0543065086822444
22.64 0.0544926597609446
22.66 0.0546806709415918
22.68 0.0548705510246398
22.7 0.0550623086303084
22.72 0.0552559521963586
22.74 0.0554514899759182
22.76 0.0556489300353911
22.78 0.0558482802523719
22.8 0.0560495483136978
22.82 0.0562527417134893
22.84 0.0564578677513001
22.86 0.0566649335302859
22.88 0.0568739459554809
22.9 0.0570849117321016
22.92 0.0572978373639069
22.94 0.0575127291516517
22.96 0.0577295931915655
22.98 0.0579484353739113
23 0.0581692613815869
23.02 0.0583920766888102
23.04 0.0586168865598398
23.06 0.0588436960477645
23.08 0.05907250999335
23.1 0.0593033330239147
23.12 0.0595361695523357
23.14 0.0597710237760011
23.16 0.0600078996759298
23.18 0.0602468010158567
23.2 0.060487731341426
23.22 0.0607306939793916
23.24 0.060975692036925
23.26 0.0612227284009052
23.28 0.0614718057373215
23.3 0.0617229264907087
23.32 0.0619760928835597
23.34 0.0622313069159069
23.36 0.0624885703648378
23.38 0.0627478847841415
23.4 0.0630092515039138
23.42 0.0632726716302718
23.44 0.0635381460450902
23.46 0.0638056754057558
23.48 0.0640752601449714
23.5 0.0643469004706291
23.52 0.0646205963656539
23.54 0.0648963475879476
23.56 0.0651741536703276
23.58 0.0654540139204998
23.6 0.0657359274210963
23.62 0.0660198930296854
23.64 0.066305909378885
23.66 0.0665939748763981
23.68 0.0668840877052116
23.7 0.0671762458236976
23.72 0.067470446965807
23.74 0.0677666886412544
23.76 0.0680649681357672
23.78 0.0683652825112718
23.8 0.0686676286062413
23.82 0.0689720030358788
23.84 0.0692784021924848
23.86 0.0695868222457361
23.88 0.069897259143034
23.9 0.070209708609837
23.92 0.0705241661500246
23.94 0.0708406270462784
23.96 0.0711590863604349
23.98 0.0714795389339337
24 0.0718019793881718
24.02 0.0721264021249463
24.04 0.0724528013268633
24.06 0.0727811709577896
24.08 0.0731115047632805
24.1 0.0734437962709995
24.12 0.0737780387912195
24.14 0.0741142254172666
24.16 0.074452349025947
24.18 0.0747924022780265
24.2 0.0751343776187507
24.22 0.0754782672782673
24.24 0.0758240632721042
24.26 0.0761717574016633
24.28 0.0765213412547004
24.3 0.0768728062057788
24.32 0.0772261434167991
24.34 0.0775813438374447
24.36 0.077938398205681
24.38 0.0782972970482137
24.4 0.078658030681047
24.42 0.0790205892098723
24.44 0.0793849625305965
24.46 0.0797511403298686
24.48 0.0801191120854651
24.5 0.0804888670668438
24.52 0.0808603943356206
24.54 0.0812336827460404
24.56 0.0816087209454225
24.58 0.0819854973747126
24.6 0.0823640002689222
24.62 0.0827442176575566
24.64 0.0831261373652263
24.66 0.0835097470119916
24.68 0.0838950340139323
24.7 0.0842819855835729
24.72 0.0846705887304053
24.74 0.0850608302613462
24.76 0.0854526967812297
24.78 0.0858461746932762
24.8 0.0862412501996335
24.82 0.0866379093017925
24.84 0.0870361378011428
24.86 0.0874359212994333
24.88 0.0878372451992805
24.9 0.0882400947046933
24.92 0.0886444548215821
24.94 0.0890503103582447
24.96 0.0894576459259058
24.98 0.0898664459392645
25 0.0902766946170127
25.02 0.0906883759823958
25.04 0.091101473863709
25.06 0.0915159718949729
25.08 0.0919318535163961
25.1 0.0923491019750423
25.12 0.0927677003253622
25.14 0.0931876314298623
25.16 0.0936088779596795
25.18 0.0940314223952565
25.2 0.0944552470269821
25.22 0.0948803339558449
25.24 0.095306665094179
25.26 0.0957342221662687
25.28 0.0961629867091899
25.3 0.0965929400734721
25.32 0.0970240634239002
25.34 0.0974563377403044
25.36 0.0978897438183729
25.38 0.0983242622704858
25.4 0.098759873526556
25.42 0.0991965578349854
25.44 0.0996342952635316
25.46 0.100073065700255
25.48 0.100512848854544
25.5 0.100953624258107
25.52 0.101395371265995
25.54 0.101838069057709
25.56 0.10228169663832
25.58 0.102726232839628
25.6 0.103171656321337
25.62 0.103617945572352
25.64 0.104065078911948
25.66 0.104513034491216
25.68 0.10496179029435
25.7 0.105411324140107
25.72 0.105861613683256
25.74 0.106312636416101
25.76 0.106764369670021
25.78 0.107216790617133
25.8 0.107669876271993
25.82 0.108123603493272
25.84 0.108577948985594
25.86 0.109032889301356
25.88 0.109488400842712
25.9 0.109944459863533
25.92 0.110401042471443
25.94 0.110858124629993
25.96 0.111315682160799
25.98 0.111773690745868
26 0.112232125929985
26.02 0.112690963123016
26.04 0.113150177602545
26.06 0.113609744516438
26.08 0.114069638885457
26.1 0.114529835606167
26.12 0.114990309453654
26.14 0.115451035084607
26.16 0.115911987040279
26.18 0.116373139749707
26.2 0.11683446753292
26.22 0.117295944604346
26.24 0.117757545076253
26.26 0.11821924296234
26.28 0.118681012181486
26.3 0.119142826561457
26.32 0.119604659842928
26.34 0.120066485683534
26.36 0.120528277662026
26.38 0.12099000928263
26.4 0.121451653979502
26.42 0.121913185121253
26.44 0.122374576015773
26.46 0.122835799915046
26.48 0.123296830020265
26.5 0.123757639486936
26.52 0.12421820143023
26.54 0.124678488930561
26.56 0.125138475039146
26.58 0.125598132783985
26.6 0.126057435175699
26.62 0.126516355213929
26.64 0.126974865893581
26.66 0.12743294021141
26.68 0.127890551172862
26.7 0.128347671798939
26.72 0.128804275133454
26.74 0.129260334250374
26.76 0.12971582226134
26.78 0.130170712323619
26.8 0.130624977648087
26.82 0.131078591507394
26.84 0.131531527244651
26.86 0.131983758281973
26.88 0.13243525812979
26.9 0.132886000395671
26.92 0.133335958794271
26.94 0.133785107156802
26.96 0.134233419441239
26.98 0.134680869742681
27 0.135127432303888
27.02 0.135573081526276
27.04 0.136017791981136
27.06 0.136461538421148
27.08 0.136904295792301
27.1 0.137346039246014
27.12 0.137786744151674
27.14 0.138226386109557
27.16 0.138664940963924
27.18 0.139102384816734
27.2 0.139538694041431
27.22 0.13997384529741
27.24 0.14040781554463
27.26 0.140840582058745
27.28 0.141272122446449
27.3 0.141702414661663
27.32 0.14213143702173
27.34 0.142559168224073
27.36 0.142985587363582
27.38 0.143410673950208
27.4 0.14383440792705
27.42 0.144256769689018
27.44 0.144677740101836
27.46 0.145097300521764
27.48 0.145515432815521
27.5 0.145932119381
27.52 0.146347343168371
27.54 0.146761087701714
27.56 0.147173337101336
27.58 0.147584076106515
27.6 0.147993290098846
27.62 0.148400965126249
27.64 0.148807087927549
27.66 0.149211645957642
27.68 0.149614627413248
27.7 0.150016021259384
27.72 0.150415817256597
27.74 0.150814005988558
27.76 0.151210578890615
27.78 0.151605528278864
27.8 0.151998847380164
27.82 0.152390530362544
27.84 0.152780572366592
27.86 0.153168969537578
27.88 0.1535557190581
27.9 0.153940819181943
27.92 0.154324269268254
27.94 0.154706069816918
27.96 0.155086222504438
27.98 0.155464730220922
28 0.155841597107599
28.02 0.156216828595604
28.04 0.156590431445304
28.06 0.156962413786618
28.08 0.157332785160243
28.1 0.157701556559752
28.12 0.158068740474863
28.14 0.158434350935182
28.16 0.158798403555283
28.18 0.159160915580616
28.2 0.159521905934467
28.22 0.159881395265697
28.24 0.160239405997634
28.26 0.160595962378064
28.28 0.160951090529989
28.3 0.161304818503585
28.32 0.161657176329162
28.34 0.162008196071134
28.36 0.162357911882869
28.38 0.16270636006303
28.4 0.163053579112446
28.42 0.163399609792331
28.44 0.163744495183506
28.46 0.16408828074665
28.48 0.164431014383609
28.5 0.164772746499761
28.52 0.165113530067358
28.54 0.165453420689977
28.56 0.165792476668043
28.58 0.166130759065186
28.6 0.166468331775765
28.62 0.16680526159338
28.64 0.167141618280194
28.66 0.167477474637433
28.68 0.167812906576574
28.7 0.168147993191622
28.72 0.168482816832228
28.74 0.168817463177535
28.76 0.169152021310927
28.78 0.169486583795625
28.8 0.16982124675085
28.82 0.170156109928683
28.84 0.170491276791732
28.86 0.170826854591232
28.88 0.17116295444564
28.9 0.171499691419794
28.92 0.171837184604453
28.94 0.172175557196116
28.96 0.172514936577018
28.98 0.172855454395565
29 0.173197246646529
29.02 0.173540453751358
29.04 0.17388522063849
29.06 0.174231696823283
29.08 0.174580036487774
29.1 0.174930398559894
29.12 0.175282946792166
29.14 0.175637849839744
29.16 0.175995281337727
29.18 0.176355419977258
29.2 0.176718449580908
29.22 0.177084559176507
29.24 0.177453943069847
29.26 0.177826800915396
29.28 0.178203337785903
29.3 0.178583764239658
29.32 0.178968296385596
29.34 0.179357155946729
29.36 0.179750570320381
29.38 0.180148772636348
29.4 0.180552001811845
29.42 0.180960502603673
29.44 0.181374525656819
29.46 0.181794327550032
29.48 0.182220170837221
29.5 0.182652324085176
29.52 0.183091061907186
29.54 0.183536664992415
29.56 0.183989420130296
29.58 0.184449620230656
29.6 0.184917564338364
29.62 0.185393557643055
29.64 0.18587791148334
29.66 0.186370943345245
29.68 0.18687297685483
29.7 0.187384341764832
29.72 0.187905373934718
29.74 0.188436415304848
29.76 0.188977813863257
29.78 0.189529923606328
29.8 0.190093104491893
29.82 0.190667722385706
29.84 0.191254149000214
29.86 0.191852761826331
29.88 0.192463944057501
29.9 0.193088084506347
29.92 0.193725577513598
29.94 0.194376822849287
29.96 0.195042225606521
29.98 0.195722196087255
30 0.1964171496805
30.02 0.197127506733005
30.04 0.197853692412253
30.06 0.198596136561903
30.08 0.19935527355039
30.1 0.200131542111587
30.12 0.200925385179096
30.14 0.201737249713342
30.16 0.202567586522398
30.18 0.203416850076208
30.2 0.204285498315043
30.22 0.205173992452072
30.24 0.206082796770778
30.26 0.207012378417087
30.28 0.207963207187291
30.3 0.208935755311474
30.32 0.209930497233347
30.34 0.210947909386942
30.36 0.211988469970083
30.38 0.213052658716389
30.4 0.214140956664691
30.42 0.215253845927535
30.44 0.216391809459288
30.46 0.217555330823707
30.48 0.21874489396212
30.5 0.219960982962783
30.52 0.221204081831354
30.54 0.222474674264105
30.56 0.223773243423415
30.58 0.225100271716528
30.6 0.226456240578474
30.62 0.227841630258858
30.64 0.22925691961421
30.66 0.230702585904898
30.68 0.232179104598429
30.7 0.233686949179087
30.72 0.235226590964215
30.74 0.236798498927774
30.76 0.238403139531359
30.78 0.240040976563447
30.8 0.241712470986642
30.82 0.243418080793649
30.84 0.245158260872093
30.86 0.24693346287847
30.88 0.248744135121237
30.9 0.250590722453335
30.92 0.25247366617457
30.94 0.25439340394324
30.96 0.25635036969766
30.98 0.258344993587626
31 0.260377701915146
31.02 0.262448917085257
31.04 0.264559057566083
31.06 0.266708537858694
31.08 0.268897768476061
31.1 0.271127155931307
31.12 0.273397102734966
31.14 0.275708007400819
31.16 0.278060264460567
31.18 0.280454264486243
31.2 0.282890394121212
31.22 0.285369036118338
31.24 0.287890569385847
31.26 0.290455369039884
31.28 0.293063806464258
31.3 0.295716249375927
31.32 0.298413061896831
31.34 0.301154604631334
31.36 0.303941234748774
31.38 0.306773306071111
31.4 0.3096511691648
31.42 0.312575171436963
31.44 0.315545657235584
31.46 0.318562967952672
31.48 0.321627442130991
31.5 0.324739415573118
31.52 0.327899221452968
31.54 0.331107190429732
31.56 0.334363650763099
31.58 0.33766892843003
31.6 0.341023347242932
31.62 0.344427228968429
31.64 0.347880893446568
31.66 0.351384658710801
31.68 0.354938841107618
31.7 0.358543755415815
31.72 0.362199714966343
31.74 0.365907031760278
31.76 0.369666016586843
31.78 0.373476979140247
31.8 0.377340228135023
31.82 0.381256071420741
31.84 0.385224816094466
31.86 0.389246768612154
31.88 0.393322234898319
31.9 0.397451520453439
31.92 0.401634930460068
31.94 0.405872769886307
31.96 0.410165343587827
31.98 0.414512956407114
32 0.418915913270973
32.02 0.423374519285628
32.04 0.427889079829504
32.06 0.432459900643743
32.08 0.437087287920259
32.1 0.441771548387475
32.12 0.446512989393749
32.14 0.451311918988565
32.16 0.456168646001042
32.18 0.461083480116135
32.2 0.466056731948872
32.22 0.471088713115701
32.24 0.476179736303902
32.26 0.481330115338633
32.28 0.486540165247629
32.3 0.491810202323919
32.32 0.497140544186023
32.34 0.502531509836286
32.36 0.507983419717065
32.38 0.513496595764662
32.4 0.519071361461386
32.42 0.524708041885751
32.44 0.530406963760652
32.46 0.536168455499343
32.48 0.541992847250264
32.5 0.547880470939512
32.52 0.553831660311774
32.54 0.559846750969556
32.56 0.565926080411078
32.58 0.572069988065827
32.6 0.578278815329564
32.62 0.584552905597078
32.64 0.590892604293781
32.66 0.597298258905901
32.68 0.60377021900962
32.7 0.610308836298294
32.72 0.616914464609003
32.74 0.623587459947469
32.76 0.630328180512301
32.78 0.637136986717634
32.8 0.64401424121507
32.82 0.650960308914384
32.84 0.657975557003215
32.86 0.665060354966185
32.88 0.672215074602321
32.9 0.679440090042552
32.92 0.686735777765917
32.94 0.694102516614753
32.96 0.70154068780962
32.98 0.709050674962774
33 0.716632864091997
33.02 0.724287643632857
33.04 0.732015404450759
33.06 0.739816539852098
33.08 0.747691445595145
33.1 0.755640519900459
33.12 0.763664163460378
33.14 0.771762779448048
33.16 0.779936773526534
33.18 0.788186553856971
33.2 0.796512531106161
33.22 0.804915118454642
33.24 0.813394731603125
33.26 0.821951788779463
33.28 0.83058671074526
33.3 0.839299920801719
33.32 0.848091844795293
33.34 0.856962911123393
33.36 0.865913550739825
33.38 0.874944197158868
33.4 0.884055286461114
33.42 0.893247257297111
33.44 0.902520550891922
33.46 0.911875611049693
33.48 0.921312884156749
33.5 0.930832819185918
33.52 0.940435867700128
33.54 0.950122483855478
33.56 0.959893124404837
33.58 0.969748248701009
33.6 0.979688318699693
33.62 0.989713798962609
33.64 0.99982515666021
33.66 1.01002286157441
33.68 1.02030738610171
33.7 1.03067920525516
33.72 1.04113879666746
33.74 1.05168664059337
33.76 1.06232321991185
33.78 1.07304902012881
33.8 1.08386452937929
33.82 1.09477023842999
33.84 1.10576664068119
33.86 1.11685423216962
33.88 1.12803351156989
33.9 1.13930498019761
33.92 1.15066914201113
33.94 1.162126503614
33.96 1.17367757425676
33.98 1.1853228658396
34 1.19706289291454
34.02 1.20889817268775
34.04 1.22082922502135
34.06 1.23285657243624
34.08 1.24498074011394
34.1 1.2572022558994
34.12 1.26952165030294
34.14 1.28193945650242
34.16 1.29445621034627
34.18 1.30707245035508
34.2 1.31978871772519
34.22 1.33260555632979
34.24 1.34552351272268
34.26 1.35854313613973
34.28 1.37166497850243
34.3 1.38488959441967
34.32 1.39821754119125
34.34 1.41164937880985
34.36 1.42518566996415
34.38 1.43882698004137
34.4 1.45257387713066
34.42 1.46642693202542
34.44 1.48038671822639
34.46 1.49445381194462
34.48 1.50862879210458
34.5 1.52291224034729
34.52 1.53730474103341
34.54 1.55180688124613
34.56 1.56641925079496
34.58 1.58114244221845
34.6 1.5959770507875
34.62 1.61092367450969
34.64 1.62598291413161
34.66 1.64115537314305
34.68 1.65644165777983
34.7 1.67184237702828
34.72 1.68735814262833
34.74 1.70298956907743
34.76 1.7187372736342
34.78 1.7346018763224
34.8 1.75058399993466
34.82 1.76668427003654
34.84 1.78290331497026
34.86 1.79924176585939
34.88 1.81570025661218
34.9 1.83227942392611
34.92 1.84897990729245
34.94 1.86580234899966
34.96 1.88274739413861
34.98 1.89981569060647
35 1.91700788911139
};%
\addlegendentry{SUR$_2$}%
\end{axis}%
\end{tikzpicture}%
    \caption{Results from two Koopman-based surrogate models based on first-principles data, with the trajectory emanating from~$x^0 = [0.2\ 0\ -\pi/2 ]^\top$ on the left, and the norm of the prediction error on the right.}
    \label{fig: modelbased circle}%
\end{figure}%
In the depicted scenario, the control input is set to the constant value~$u \equiv \begin{bmatrix} 0.2 & 0.2 \end{bmatrix}^\top$, i.e., the robot will move in a circle and the basis is~$B = \{ e_1, e_2\}$ for the unit vectors~$e_1, e_2 \in \mathbb{R}^{n_u}$. 
As can be seen, SUR$_1$ leads to a trajectory whose error remains comparatively small over the whole trajectory. For the model SUR$_2$, however, we receive a trajectory that visibly deviates from the reference after about one quater of simulated time, which can also be seen in the error plot. 
In the second half of the simulation, the prediction based on SUR$_2$ becomes increasingly inaccurate and quickly unusable.  
Consequently, from now on, we will only use SUR$_1$ for Koopman-based surrogate models.

The basis employed for~$\mathbb{R}^{n_u}$ need not consist of unit vectors. 
In the following, we use the bases~$B_1 = \{ [0.2\ 0]^\top, [0\ 2]^\top \}$ and~$B_2 = \{ [0.2\ -0.4]^\top, [0.2\ 0.6]^\top \}$ instead. 
Basis~$B_1$ contains scaled variants of the unit vectors that, in absolute value, fit better to the usual operating points of the employed hardware robot; for instance, it cannot attain translational velocities of~$1\,\textnormal{m}/\textnormal{s}$. 
Still, training with~$B_1$ only captures the robot driving a straight line or rotating on the spot. 
In contrast, to study the influence of the usage of different training motions for learning, the inputs contained in~$B_2$ let the robot drive arcs of different radii.  
In Fig.~\ref{fig: firstprinciple}, the results for those two bases are illustrated. 
\definecolor{ODEcolor}{RGB}{0,0,255}
\definecolor{SUR1B1color}{RGB}{230,97,1}  %230 97 1
\definecolor{SUR1B2color}{RGB}{44,160,44} %{26,150,65}
\definecolor{lightblue}{RGB}{135,206,235}
\def\lineWidthStep{1.0}
\def\lineWidthPlane{1.5}
\def\heightODEtop{5.4cm}
\def\heightODEbottom{5.4cm}
\def\yBottomPlots{-5.0cm}
\begin{figure}%[h!]
    \centering
    % This file was created with tikzplotlib v0.10.1.
\begin{tikzpicture}

%\definecolor{darkgray176}{RGB}{176,176,176}
%\definecolor{lightgray204}{RGB}{204,204,204}
%\definecolor{magenta}{RGB}{255,0,255}
%\definecolor{orange}{RGB}{255,165,0}


\begin{axis}[
width = \textwidth,
height = \heightODEtop,
at={(0cm,0cm)},
axis equal image=true,
legend columns = 2,
legend cell align={left},
legend style={fill opacity=1, draw opacity=1, text opacity=1, draw=black, at={(1,0)}, anchor=south east, inner sep=1pt},
%tick align=outside,
%tick pos=left,
%x grid style={darkgray176},
%xmin=-1.38204497018448, xmax=0.960557025574019,
xmin=-1.4, 
xmax=1,
xtick style={color=black},
xlabel = {$x_1$ position (m)}, 
ylabel = {$x_2$ position (m)},
%y grid style={darkgray176},
%ymin=-1.26703103777094, ymax=0.666171581738761,
ymin=-1.3, ymax=0.7,
ytick style={color=black},
]
\addplot [line width = \lineWidthPlane, color = ODEcolor, dash pattern=on 1pt off 3pt on 3pt off 3pt]
table {%
0.2 0
0.198517124546097 -0.0365791989912238
0.194078229969446 -0.0729183367546423
0.186712447757655 -0.108778927531424
0.176468117917349 -0.143925626161229
0.163412471729345 -0.178127772600671
0.147631190524568 -0.211160905694385
0.129227843376338 -0.242808236264168
0.108323207399354 -0.272862069848606
0.0850544751160836 -0.301125169756097
0.0595743540924508 -0.327412051485819
0.00884585713153947 -0.37454395213327
-0.0429809438509488 -0.420465397694968
-0.095877172986572 -0.465150802535766
-0.149813358563308 -0.508575269694113
-0.20475944944604 -0.550714604753662
-0.260684831819877 -0.591545329323459
-0.317558346246971 -0.631044694119181
-0.375348305027343 -0.669190691638151
-0.43402250985403 -0.705962068421057
-0.49354826975272 -0.74133833689355
-0.414920146137837 -0.696903060924483
-0.33541652392102 -0.654053911247386
-0.255069395044883 -0.612808130190338
-0.173911090875778 -0.573182314892233
-0.0919742691937809 -0.535192410624163
-0.00929190105133053 -0.498853704373103
0.0741027424941918 -0.464180818690487
0.158176103768605 -0.43118770580815
0.242894351984401 -0.399887642024011
0.328223396854124 -0.370293222359734
0.380028281613288 -0.352726670257407
0.431722976099682 -0.334838450680475
0.48330548303355 -0.316629254759416
0.534773809469624 -0.298099786025969
0.58612596687412 -0.279250760385955
0.637359971201571 -0.260082906091621
0.688473842971478 -0.240596963713497
0.739465607344792 -0.220793686111785
0.790333294200215 -0.200673838407276
0.841074938210317 -0.180238197951781
0.762553801299677 -0.214240856635862
0.685655721650586 -0.251769979801843
0.610538977625982 -0.292748321721297
0.537358181087799 -0.337091537181641
0.466263959161286 -0.38470835509285
0.397402644201067 -0.435500766349601
0.330915972597068 -0.489364225562201
0.26694079304027 -0.546187866241067
0.205608784848741 -0.605854728991841
0.147046186933736 -0.668242002251464
0.17784317725774 -0.633374781200382
0.208106030673824 -0.598042962396952
0.237827739287499 -0.562254727543504
0.267001420515847 -0.526018364033508
0.295620318681301 -0.489342263032485
0.323677806576043 -0.452234917534886
0.351167386996651 -0.414704920397386
0.378082694248637 -0.376760962349062
0.404417495620541 -0.338411829978901
0.430165692827217 -0.299666403701107
0.431423996560572 -0.297789497412019
0.432725747849716 -0.29594245949021
0.434070238009615 -0.294126295479629
0.435456735087809 -0.292341994116196
0.436884484262898 -0.290590526789517
0.438352708255473 -0.288872847014058
0.439860607751271 -0.28718988991004
0.441407361836329 -0.28554257169435
0.442992128443903 -0.283931789181745
0.444614044812887 -0.282358419296622
0.410580001838084 -0.314590180280073
0.37652453054594 -0.34679929967777
0.34244764599906 -0.378985763243743
0.308349363269517 -0.411149556742042
0.274229697438852 -0.443290665946747
0.240088663598059 -0.475409076641969
0.205926276847587 -0.507504774621859
0.171742552297328 -0.539577745690612
0.137537505066612 -0.571627975662479
0.103311150284198 -0.603655450361765
0.132299691717159 -0.574415028350043
0.158994783892079 -0.543066824771422
0.183243647106727 -0.509790249727837
0.204907501765322 -0.474775749657646
0.223862362633342 -0.438223717384349
0.239999748422155 -0.400343345240756
0.253227302642433 -0.361351427832353
0.263469322173101 -0.321471121291812
0.270667190520764 -0.280930666125597
0.274779713290002 -0.239962080962004
0.273464771116331 -0.266506481611987
0.272825230084183 -0.293075735761382
0.272861503782951 -0.319652661147997
0.273573568754535 -0.346220070548674
0.274960964508516 -0.372760782894215
0.277022793819946 -0.399257634380334
0.279757723309589 -0.42569348956744
0.283163984306209 -0.452051252462075
0.287239373990367 -0.478313877572852
0.291981256818977 -0.504464380933718
0.284601699339857 -0.474288981850678
0.274378336041097 -0.444954779607051
0.261404339679124 -0.416729117581017
0.245797951479218 -0.389869236237976
0.22770140351941 -0.364619928715853
0.207279622468739 -0.341211309851707
0.184718726493607 -0.319856718982115
0.160224329031006 -0.300750775630567
0.134019664887517 -0.284067605801804
0.106343555742284 -0.269959255048102
0.0537826562138015 -0.249350028025017
-0.032672267162414 -0.225080129336013
0.0232451567411312 -0.234890662415295
-0.0015868653646627 -0.233057742023989
-0.0265479362476038 -0.233714890220042
0.0156491187802505 -0.232475204673005
-0.0373744608034394 -0.233488711235877
-0.0591813379832165 -0.234532119948891
0.0221642749974167 -0.227588466789729
0.0877361085995754 -0.219595041115094
0.00729855408534909 -0.227018325791587
-0.0291954311332132 -0.22930570749543
-0.0530195221626042 -0.231286256874566
-0.0943717311389766 -0.236249854952337
-0.180431583849563 -0.24872298378463
-0.085666821987665 -0.235066719789639
-0.00846650892546544 -0.223795439455245
-0.0137789177066219 -0.224430323291782
0.0326610521762742 -0.222620152391723
-0.0523214854455751 -0.225529406728928
-0.130204288711922 -0.231631656928558
-0.207868103034782 -0.240075616127165
-0.285242536140194 -0.250853630960122
-0.362257458041109 -0.263955932541115
-0.438843064601215 -0.279370645316382
-0.514929940803424 -0.297083797828376
-0.590449123665667 -0.317079335379099
-0.665332164746987 -0.339339134581624
-0.739511192187255 -0.363843019786616
-0.812918972224296 -0.390568781368968
-0.786040175569343 -0.378878731537354
-0.760349883676293 -0.364767180342188
-0.736065675583545 -0.348353642880931
-0.713393221770329 -0.329777130485664
-0.692524542269714 -0.309194973393063
-0.673636380391482 -0.286781488265898
-0.656888705828298 -0.262726501851235
-0.642423359822843 -0.237233743278919
-0.630362853870397 -0.210519118616481
-0.620809332131045 -0.182808882293785
-0.599811241522468 -0.118211442547968
-0.575190752826379 -0.0549059880650698
-0.547026795671116 0.0069045332745612
-0.51540965951836 0.0670219661240984
-0.480440704208545 0.125253582926834
-0.442232035016264 0.181412701772088
-0.40090614325742 0.235319284869673
-0.356595513600271 0.286800515724301
-0.309442199339263 0.335691353159582
-0.259597366993281 0.381835060415518
-0.224582307795341 0.410111455129252
-0.187459291014527 0.43555728963578
-0.148456587681409 0.458016096401743
-0.107814027100197 0.477349775256133
-0.0657815221280175 0.493439442574194
-0.0226175324499772 0.506186162297958
0.0214125247005829 0.515511554298328
0.0660379066040965 0.521358276337874
0.110984209863427 0.523690376670721
0.155975057722185 0.522493515111386
0.111990056995563 0.523279001978228
0.0681320033409612 0.519848490621453
0.0248040740714056 0.512233516967441
-0.017595426815405 0.500504083771917
-0.0586767301569847 0.484768017100222
-0.098062184687107 0.465169975106324
-0.135389728699732 0.441890118222648
-0.17031621840158 0.415142452985367
-0.20252058235655 0.385172864720016
-0.231706773023881 0.352256857172484
-0.284084372424199 0.290730735801249
-0.339800026117977 0.232210290968669
-0.398681551661743 0.176876372983704
-0.460546982870567 0.1248999845821
-0.525205132162398 0.0764417524632607
-0.59245618140001 0.031651430892988
-0.662092299404618 -0.00933256109383952
-0.733898284232836 -0.04638356746044
-0.807652228232072 -0.0793870865831872
-0.883126203819115 -0.108241125104936
-0.886772881748823 -0.109638588185593
-0.890333724610153 -0.111242196960119
-0.893797090220671 -0.113046708426612
-0.897151655099999 -0.115046222731436
-0.900386451491955 -0.117234202458811
-0.903490903223652 -0.119603494004904
-0.906454860284313 -0.122146350966547
-0.909268632010753 -0.124854459468123
-0.911923018771021 -0.12771896534379
-0.914409342042611 -0.130730503086202
-0.96009179798419 -0.195558892442965
-0.999389734891474 -0.264444877971791
-1.03194559934778 -0.336761698538994
-1.05746318087781 -0.411851377458967
-1.07571030702724 -0.4890307091141
-1.08652095579053 -0.567597475120875
-1.08979676616691 -0.646836833484155
-1.08550793310084 -0.726027822607778
-1.07369347866413 -0.804449920984507
-1.05446089701252 -0.881389602881827
-1.04680533835596 -0.911460182738604
-1.04178365815916 -0.942080928635268
-1.03943361625666 -0.97302159207876
-1.03977288346576 -1.00404951899997
-1.04279890871328 -1.03493139916404
-1.04848893821799 -1.06543502051467
-1.05680018658455 -1.09533101526073
-1.06767015852241 -1.12439458457589
-1.08101711877054 -1.15240718894252
-1.09674070669456 -1.17915819142959
-1.08507512853553 -1.16076012025305
-1.07376721626141 -1.1421400811338
-1.06282119197136 -1.12330502634112
-1.05224114264434 -1.10426198842593
-1.0420310186132 -1.08501807759497
-1.03219463208968 -1.06558047905602
-1.02273565574103 -1.04595645033516
-1.01365762131873 -1.02615331856695
-1.00496391833983 -1.00617847775867
-0.996657792821356 -0.986039386029579
-0.978067715864121 -0.943690140531675
-0.956797072435074 -0.902621791604161
-0.932933629327474 -0.863003795206462
-0.906575851728079 -0.824999622862645
-0.87783249693072 -0.788766087146256
-0.846822165582644 -0.754452694641284
-0.813672812315288 -0.722201029049043
-0.778521217778687 -0.69214416698641
-0.741512424258034 -0.664406128885935
-0.702799137201125 -0.639101367263524
-0.760415969394293 -0.674590184263305
-0.817656913746895 -0.710682146992449
-0.8745156566379 -0.747373274542229
-0.930985926602729 -0.784659519916615
-0.987061495024995 -0.822536770478653
-1.04273617682351 -0.861000848404087
-1.09800383113449 -0.90004751114217
-1.1528583619889 -0.939672451883607
-1.20729371898481 -0.979871300035599
-1.26130389795479 -1.02063962170391
};
\addlegendentry{ODE}
\addplot [line width = \lineWidthPlane, color = SUR1B1color]
table {%
0.2 0
0.200010858519103 -0.0366257862197932
0.197047751850266 -0.073130682410825
0.191129869125494 -0.109274629276544
0.182296398030644 -0.144819999534326
0.170606325543447 -0.179533202029126
0.156138076945434 -0.21318625872892
0.138988985813439 -0.24555834060099
0.119274587695788 -0.276437247741833
0.0971277301765668 -0.305620818507562
0.0726974920284524 -0.332918251766372
0.0224929076274164 -0.380633342453223
-0.0288171956158655 -0.427151526043847
-0.0812038720783324 -0.47244705638541
-0.134637618307462 -0.516494870491088
-0.189088393225097 -0.559270600403662
-0.244525638824437 -0.600750584391485
-0.300918301480808 -0.640911877460766
-0.358234854013211 -0.6797322611714
-0.416443318646931 -0.717190252747272
-0.475511291037007 -0.753265113475786
-0.397391290243462 -0.708043119920956
-0.318391672750903 -0.664395983650456
-0.238543964429752 -0.622341170698956
-0.157879839891858 -0.581895535687462
-0.0764311055920778 -0.543075320184002
0.00577031568454835 -0.505896151225403
0.0886923993316514 -0.47037303995736
0.172303032815911 -0.436520380369644
0.256570029371551 -0.404351948126691
0.341461141953163 -0.373880899520284
0.393235611743484 -0.356461139885014
0.444901000185112 -0.338719615986576
0.49645547221764 -0.32065703540027
0.547897185085265 -0.302274116382572
0.599224287233719 -0.283571587790488
0.650434917377799 -0.264550189009151
0.701527203764201 -0.245210669890553
0.752499263653863 -0.225553790706633
0.803349203047335 -0.205580322120201
0.854075116675906 -0.185291045177473
0.774918078643302 -0.217523768327681
0.697289530732243 -0.253314086613599
0.62134528676764 -0.292588073774788
0.54723926894319 -0.335264998321152
0.475122915670017 -0.381257398626439
0.40514462696461 -0.430471185303879
0.337449245487999 -0.482805769198761
0.272177571461658 -0.538154213410534
0.20946590979386 -0.59640340783923
0.149445647854679 -0.657434264831622
0.180542980156041 -0.622811534274782
0.211111799788195 -0.587720093071839
0.241144893814566 -0.552168025180911
0.270635170862307 -0.516163523091935
0.299575662988536 -0.479714886016423
0.327959527539559 -0.442830518036923
0.355780049004775 -0.405518926217489
0.3830306408661 -0.367788718676553
0.409704847442814 -0.329648602623604
0.435796345730895 -0.291107382361145
0.437035109664764 -0.289216229340877
0.438317629233841 -0.287354462666365
0.439643187397773 -0.285523092828677
0.44101104329361 -0.283723113421326
0.442420432754745 -0.281955500609139
0.443870568854469 -0.280221212610492
0.445360642474419 -0.278521189193516
0.446889822898168 -0.276856351186887
0.448457258430228 -0.275227600005817
0.450062077040705 -0.273635817193862
0.416015697419842 -0.305870120033667
0.381948193098884 -0.338081768825477
0.347859565159741 -0.370270749770857
0.313749811379933 -0.402437049197003
0.279618926487925 -0.434580653582068
0.245466902451108 -0.466701549578693
0.211293728794077 -0.498799724035288
0.177099392944682 -0.530875164014655
0.142883880605152 -0.562927856809556
0.10864717614536 -0.59495778995485
0.138746558983351 -0.566835942741431
0.166645211934064 -0.536521106966623
0.192179882913892 -0.504185839967754
0.215200751346095 -0.47001443712636
0.235572562883031 -0.434201912461746
0.253175604290304 -0.396952892986481
0.267906523945569 -0.358480439275353
0.279679003404137 -0.319004804265532
0.288424285482856 -0.278752141871314
0.294091564312254 -0.237953176561837
0.292441650493632 -0.264483751342672
0.291469115531815 -0.291047941715671
0.291174546168687 -0.317628533123242
0.29155807925326 -0.344208300131664
0.292619402326967 -0.370770017845114
0.294357754521875 -0.397296473307762
0.296771927767313 -0.423770476885616
0.299860268300496 -0.450174873619882
0.303620678476766 -0.476492554543791
0.30805061887514 -0.502706467954903
0.302091445318661 -0.472196524629406
0.293245407159696 -0.442392366960841
0.281595260335114 -0.413565991752568
0.267249374150498 -0.385980344820125
0.25034058719981 -0.359886915132408
0.23102486182549 -0.335523446118485
0.209479753892698 -0.313111783314504
0.185902714675597 -0.29285587584911
0.160509241680288 -0.274939947589698
0.133530895256604 -0.25952685209564
0.0820643691673602 -0.236317363028674
-0.00305149058027791 -0.207759607149974
0.0522839757736429 -0.220350753277698
0.0275871375148868 -0.217279641375184
0.00263841109149525 -0.216688697681952
0.0447055947163972 -0.213470257028563
-0.00827950087916957 -0.212538629380031
-0.0300563400256825 -0.213755792296721
0.0514684372891558 -0.21050601942602
0.116904036707068 -0.201914904470061
0.0367022944624329 -0.210949070865164
0.000265984177031733 -0.213575806115985
-0.0235866271216921 -0.214848136900584
-0.0649329704683359 -0.219526495527313
-0.151122879577578 -0.230475383468978
-0.0567070216424635 -0.215080563481194
0.0206387383280396 -0.205382746878822
0.0153643019267971 -0.206262852328255
0.0616809450750889 -0.202891907984797
-0.0232786189910269 -0.203343152884411
-0.101171525329747 -0.208270302437956
-0.178881655172685 -0.215540985063261
-0.256338899149723 -0.225148637695631
-0.333473392462205 -0.237084588977045
-0.410215579052059 -0.251338066833645
-0.48649627428997 -0.267896208032606
-0.562246726159989 -0.286744069704593
-0.637398674969731 -0.307864642799876
-0.71188441165476 -0.331238867428542
-0.785636834770867 -0.35684565001931
-0.758256655784257 -0.346405820329151
-0.731946030364611 -0.333491988646008
-0.706927236627286 -0.318213157293055
-0.683412000954365 -0.300698345075406
-0.661599727488898 -0.281095523036024
-0.641675810152166 -0.259570389921903
-0.623810040001204 -0.236304995879348
-0.608155120750336 -0.211496224118036
-0.594845305284618 -0.185354141504132
-0.583995165994148 -0.158100230263352
-0.564772839381294 -0.0929155559111651
-0.54188829407499 -0.028921722620081
-0.515416590835357 0.0336759054657816
-0.485444284592185 0.0946765066344269
-0.452069106416819 0.153884446363329
-0.41539961276056 0.211109903603817
-0.375554803630912 0.266169476776604
-0.332663711447076 0.318886767729427
-0.286864962349406 0.369092941975264
-0.238306311731789 0.416627263599621
-0.204402154759661 0.446255928581785
-0.1682895297236 0.473137238913218
-0.130190751220088 0.497106312002582
-0.0903399565542775 0.518016189255596
-0.0489816832979245 0.535738716722714
-0.00636939674188275 0.550165311162861
0.0372360253579655 0.561207607732712
0.0815678433328148 0.568797986129306
0.126355034020323 0.572889972632716
0.171323889587916 0.573458516114195
0.127454376041007 0.576350561094712
0.0835101662700133 0.575026115289121
0.0398926897452445 0.569497299984531
-0.00299950200139588 0.55981471994314
-0.044774246524976 0.546067014165092
-0.0850492364910589 0.528380068217442
-0.123455416034403 0.506915893793731
-0.15964028298978 0.481871183625528
-0.193271072592703 0.453475552325304
-0.224037798284091 0.42198947619791
-0.274752305860748 0.359028561180563
-0.328878405233553 0.29898206102255
-0.386247743422468 0.242035302227353
-0.446682171626567 0.188363981762373
-0.509994308856459 0.138133634377551
-0.575988130102285 0.0914991318759864
-0.644459576835979 0.048604215543946
-0.715197187450425 0.00958106288091325
-0.78798274512952 -0.0254501102873047
-0.862591940646032 -0.0563814112833604
-0.866274995108951 -0.0576736260337067
-0.86987864378114 -0.0591743280392355
-0.873391146354152 -0.0608786286307482
-0.876801049091909 -0.0627809746321689
-0.880097221402454 -0.0648751659891816
-0.883268891589961 -0.0671543754840959
-0.886305681689155 -0.0696111704851361
-0.88919764128427 -0.0722375366725279
-0.89193528021461 -0.0750249036779456
-0.89450960006878 -0.0779641725680597
-0.943311157899979 -0.140558981516172
-0.985937450133598 -0.207517296533717
-1.02199583005264 -0.278230069912313
-1.05115280762363 -0.35205372164254
-1.07313742616149 -0.428315984012475
-1.0877441226137 -0.506322039288565
-1.09483503061061 -0.585360897949421
-1.09434168652014 -0.664711960804186
-1.08626609912823 -0.743651704062157
-1.07068114287366 -0.821460421993656
-1.06170193002695 -0.851182016784376
-1.05533907012279 -0.881569417067105
-1.0516411394524 -0.912393776688545
-1.05063600945993 -0.943422874996476
-1.05233071433636 -0.974422911231569
-1.056711498556 -1.00516030673942
-1.06374403487621 -1.03540349923016
-1.07337380330729 -1.06492471416548
-1.08552662154929 -1.09350169920381
-1.100109317387 -1.12091940848418
-1.08824720202478 -1.10263466570016
-1.07674020236336 -1.08412425767059
-1.06559266604427 -1.06539509484247
-1.05480880816177 -1.04645417015798
-1.04439270956086 -1.027308556454
-1.03434831517536 -1.00796540382925
-1.02467943240691 -0.988431936980218
-1.01538972954617 -0.968715452506561
-1.00648273423714 -0.948823316187169
-0.997961831985811 -0.928762960227638
-0.980695491136352 -0.885826089202704
-0.960707524105232 -0.844085237673767
-0.9380819233276 -0.803712731292385
-0.912913474346385 -0.764875179365728
-0.885307325328335 -0.727732789770615
-0.855378520060314 -0.692438712586827
-0.823251496605614 -0.659138414833974
-0.789059553803206 -0.627969088540691
-0.752944287804544 -0.599059094221328
-0.71505500085871 -0.572527441683309
-0.772848994286378 -0.607701190744387
-0.830272026474911 -0.643480133373305
-0.887317853448579 -0.6798603434035
-0.943980284060202 -0.716837829305961
-1.00025317872096 -0.754408534397387
-1.05613044760942 -0.792568336938098
-1.11160604831524 -0.831313050106948
-1.16667398286859 -0.870638421841837
-1.22132829410189 -0.910540134536059
-1.27556306128636 -0.951013804582576
};\label{plot:Sur1B1}
%\addlegendentry{SUR$_1 \, B_1$}
\addplot [line width = \lineWidthPlane, color = SUR1B2color, dashed]
table {%
0.2 0
0.199848764150061 -0.0366254643918402
0.196724128872202 -0.0731168781688429
0.190646345351635 -0.109234269370875
0.181655652309 -0.144740186532029
0.16981206964744 -0.17940130189628
0.155195030497457 -0.212989986799322
0.137902844427168 -0.245285845183487
0.118051984587737 -0.276077190586371
0.0957761915647961 -0.305162451317957
0.0712253867064695 -0.332351487915725
0.0209890184163811 -0.380031629216541
-0.0303520392968986 -0.42651412756818
-0.0827688238179298 -0.471773256910561
-0.136231815089993 -0.515783974717129
-0.190710955817315 -0.558521933844409
-0.246175672162234 -0.599963493713766
-0.302594895058244 -0.640085730809275
-0.359937082276281 -0.678866448478936
-0.418170241394737 -0.716284186030119
-0.477261953833202 -0.752318227113973
-0.399163287992337 -0.707062289313428
-0.320184318858822 -0.663380791956787
-0.24035656536221 -0.621291214688958
-0.159711695418994 -0.580810425799479
-0.0782815090006009 -0.541954680579439
0.00390207741225708 -0.504739619841191
0.0868070453694719 -0.469180268557514
0.170401288373818 -0.435291034596471
0.254652625598419 -0.403085707551554
0.339528815855418 -0.372577457693097
0.391306440718799 -0.355169738657023
0.442975056489779 -0.337440235256536
0.494532828162372 -0.319389654668167
0.545977913077992 -0.301018714750106
0.597308459818126 -0.282328143960938
0.648522607266685 -0.263318681286581
0.699618483866741 -0.243991076178315
0.750594207095862 -0.224346088505128
0.801447883183577 -0.204384488523848
0.852177607093698 -0.184107056870826
0.773056133648437 -0.216423158870497
0.695466894567609 -0.252295172228723
0.619565622862516 -0.291649006931491
0.545506152427514 -0.334403769831443
0.47343982611575 -0.380471840380151
0.403514941350566 -0.429758973597431
0.335876231386912 -0.482164428615229
0.270664380450532 -0.537581121211533
0.208015571090673 -0.595895798831917
0.148061062186003 -0.656989236676828
0.179180580867464 -0.6223873751928
0.209771900168757 -0.587316464343016
0.239827802040566 -0.551784583248933
0.269341189934657 -0.515799919641283
0.298305090669729 -0.479370768051139
0.326712656290245 -0.442505527960668
0.354557165919969 -0.405212701914889
0.381832027611038 -0.367500893595822
0.408530780188529 -0.329378805860449
0.434647095089558 -0.290855238743937
0.43584675049714 -0.288938586225493
0.437090766503528 -0.287050415171784
0.438378446968385 -0.28519175083751
0.439709071580817 -0.283363602072511
0.441081896367308 -0.281566960779934
0.442496154224556 -0.27980280138727
0.443951055477481 -0.278072080330836
0.445445788462682 -0.276375735554331
0.446979520137608 -0.274714686022065
0.448551396715723 -0.273089831247467
0.414504646869533 -0.305322652508215
0.380436772668021 -0.337532819453053
0.346347775077738 -0.369720318279361
0.312237651771133 -0.401885135310978
0.278106397382792 -0.434027257023498
0.243954003798232 -0.466146670067767
0.209780460472901 -0.498243361291113
0.175585754778855 -0.530317317755914
0.14136987237641 -0.5623685267551
0.10713279760784 -0.594396975824227
0.137335192081971 -0.566386046428416
0.165344937049246 -0.536174065687845
0.190998148481613 -0.503932989175395
0.214144330608645 -0.469846559724101
0.234647514358597 -0.434109292203451
0.252387236281213 -0.396925371471314
0.26725936335924 -0.358507475972761
0.279176769116633 -0.319075539023324
0.288069866432256 -0.27885545937987
0.293887002468068 -0.238077772267345
0.292287891070217 -0.264611012252761
0.291366210115085 -0.291176570924681
0.291122513387808 -0.317757232922138
0.291556904768253 -0.344335772869316
0.292669038838005 -0.370894966788237
0.294458121800268 -0.397417603498939
0.296922912708155 -0.423886495998793
0.300061724996949 -0.450284492812775
0.303872428315933 -0.476594489306599
0.308352450655464 -0.502799438954795
0.302580845559615 -0.472253107365005
0.293918061413465 -0.442394812624367
0.282445181759344 -0.413497058056822
0.268268958881813 -0.385823453494481
0.25152068452937 -0.35962630240551
0.23235485846072 -0.335144304948478
0.210947671470863 -0.312600396230614
0.187495319586554 -0.292199737371544
0.162212166157523 -0.274127875298128
0.135328768607616 -0.258549085520022
0.0839450189150393 -0.235157614365148
-0.00110613907281513 -0.206410736963832
0.0542725147085373 -0.218806764636189
0.0296004683590293 -0.215537381913139
0.00465660935703506 -0.214746694620683
0.0467376768412359 -0.211715360380325
-0.00624418103631097 -0.210637140919623
-0.0280188443151067 -0.211886174442502
0.053497163935397 -0.208455443437733
0.118936532246824 -0.199900850270258
0.038745185064904 -0.209014658753166
0.00231219472515769 -0.211678552835339
-0.0215465074683427 -0.21283243904925
-0.0628952945006417 -0.217483566969338
-0.149092791624311 -0.228362388612021
-0.0546664864953893 -0.213040548568394
0.0226678308764609 -0.203262412257275
0.0174241965337326 -0.204325089041668
0.0637500345013223 -0.201088327669276
-0.0212089244740713 -0.201424009498564
-0.0991044546188825 -0.206291050004161
-0.176819021604207 -0.213501729375417
-0.254282511629099 -0.223049538677558
-0.33142505369786 -0.234925860521572
-0.408177083819665 -0.249119976587083
-0.484469407729598 -0.265619077126097
-0.560233262107301 -0.284408272434377
-0.635400374321505 -0.305470606258956
-0.709903020768522 -0.328787071092672
-0.783674083898292 -0.35433662529049
-0.756360126399891 -0.343725118243611
-0.730131231956972 -0.330646385742244
-0.705208988737817 -0.315210823218086
-0.681804307738563 -0.297548775835644
-0.660115659003967 -0.277809463157703
-0.640327391376088 -0.256159744043474
-0.622608148534727 -0.232782730378355
-0.607109394092207 -0.207876259455089
-0.593964058501336 -0.18165123604667
-0.58328532052852 -0.154329856431974
-0.563954580460896 -0.0891781267626706
-0.540963657971994 -0.0252233267838159
-0.514387954947708 0.0373293073446392
-0.484314359786765 0.0982791009529816
-0.45084092827669 0.157430585087177
-0.414076531768571 0.214594122209345
-0.374140474328877 0.269586511559148
-0.331162080615272 0.322231572428937
-0.285280256254887 0.37236070367309
-0.236643022496184 0.419813417842168
-0.202636002457949 0.449323604958103
-0.166430112753422 0.476078772887622
-0.128248240087253 0.499914815575823
-0.0883250313178959 0.520685591969267
-0.0469054679621439 0.538263801695782
-0.00424339098994543 0.552541745986209
0.0394000166784977 0.563431970073644
0.0837577825533359 0.570867783924192
0.128558724392779 0.574803658772196
0.173529049986484 0.575215497551677
0.129672591436322 0.578298735397412
0.0857230565221482 0.577165850988673
0.0420819236398774 0.571827215689231
-0.000852038916047911 0.562331696486326
-0.0426862860490117 0.54876622206481
-0.0830379653085555 0.531255010254991
-0.121537318123304 0.509958462411338
-0.157830988317494 0.485071732736193
-0.19158521346107 0.456822983020863
-0.222488874629922 0.425471335734435
-0.273115852949425 0.362441760133998
-0.32715837179813 0.302321852864702
-0.384448336411538 0.245297168884891
-0.444807843153363 0.191543647244163
-0.508049742347569 0.141227077637033
-0.573978225699248 0.0945025988771827
-0.642389436112856 0.051514230502069
-0.713072097518206 0.0123944386506997
-0.785808162203004 -0.0227362627002915
-0.860373473149976 -0.053769674111246
-0.864094452710589 -0.0549540678596349
-0.86774212973802 -0.0563492789009202
-0.871304620501509 -0.0579507622441618
-0.87477030839138 -0.0597532999845156
-0.878127881007798 -0.0617510178613524
-0.881366366490839 -0.0639374039266587
-0.884475168992518 -0.0663053292747785
-0.887444103191187 -0.0688470707787376
-0.890263427748539 -0.0715543357725761
-0.89292387760924 -0.0744182886132953
-0.941875383675277 -0.13689490567135
-0.984662198429437 -0.203749826240658
-1.02089021551732 -0.274374954857105
-1.05022440179878 -0.348127522062877
-1.07239218813957 -0.424335920829555
-1.08718634447657 -0.502305837278024
-1.09446729812752 -0.581326623260044
-1.09416485541282 -0.66067785422841
-1.08627928705923 -0.739636011561487
-1.07088173719101 -0.81748122408834
-1.06173664037746 -0.847151938251201
-1.05520411284932 -0.877503064876297
-1.05133400857051 -0.908306034678869
-1.05015549600584 -0.939328788245467
-1.05167691568942 -0.970337569643374
-1.05588581768841 -1.00109872885486
-1.06274916957184 -1.03138051721102
-1.07221372547604 -1.0609548608524
-1.08420654684228 -1.08959909809454
-1.09863566539249 -1.11709766742639
-1.08680620394216 -1.09879221760956
-1.07533225144136 -1.08026173672639
-1.06421814322528 -1.06151314280534
-1.05346808193948 -1.04255343613334
-1.04308613584205 -1.02338969665219
-1.0330762371458 -1.00402908132297
-1.0234421804016 -0.984478821459029
-1.01418762092387 -0.964746220028808
-1.00531607325943 -0.944838648929155
-0.996830909700733 -0.924763546230282
-0.97944550441143 -0.881875165995206
-0.959341831618862 -0.840190351016238
-0.936604360762329 -0.799881191386463
-0.91132833835517 -0.761114030943383
-0.883619355646216 -0.72404878339485
-0.853592879831739 -0.688838277263615
-0.821373751004643 -0.655627632028648
-0.78709564703097 -0.624553667685937
-0.750900518555349 -0.595744349797982
-0.712937996352927 -0.569318271949211
-0.770742013369494 -0.604473530875789
-0.828175260582048 -0.640234088698064
-0.885231492241595 -0.676596021222598
-0.941904515455819 -0.713555338887335
-0.998188188945161 -0.751107986974982
-1.05407642128102 -0.789249845716844
-1.10956316856179 -0.827976730274303
-1.16464243147794 -0.867284390586517
-1.21930825171278 -0.907168511074474
-1.27355470762184 -0.94762471019346
};\label{plot:Sur1B2}
%\addlegendentry{SUR$_1 \, B_2$}
\end{axis}

\begin{axis}[
width = .36\textwidth,
height = \heightODEtop,
at={(.37\textwidth,0cm)},
legend cell align={left},
legend columns =2,
legend style={
  fill opacity=1,
  draw opacity=1,
  text opacity=1,
  at={(0,1)},
  anchor=north west,
  column sep = 0.2cm,
  %draw=lightgray204
},
%tick align=outside,
%tick pos=left,
%x grid style={darkgray176},
%xmin=-0.25, 
xmin=0,
xmax=5,
%xtick style={color=black},
%y grid style={darkgray176},
%ymin=-5.1798346369035, ymax=5.48475403032874,
ymin=-5, ymax=8,
%ytick style={color=black},
xlabel = {time (s)},
ylabel = {control},
ylabel style={yshift=-.2cm}
]
\addplot [line width = \lineWidthStep, blue, const plot mark right]
table {%
0 1.83096306185539
0.02 1.83096306185539
0.04 1.83096306185539
0.06 1.83096306185539
0.08 1.83096306185539
0.1 1.83096306185539
0.12 1.83096306185539
0.14 1.83096306185539
0.16 1.83096306185539
0.18 1.83096306185539
0.2 3.46230382145696
0.22 3.46230382145696
0.24 3.46230382145696
0.26 3.46230382145696
0.28 3.46230382145696
0.3 3.46230382145696
0.32 3.46230382145696
0.34 3.46230382145696
0.36 3.46230382145696
0.38 3.46230382145696
0.4 -4.51584685079614
0.42 -4.51584685079614
0.44 -4.51584685079614
0.46 -4.51584685079614
0.48 -4.51584685079614
0.5 -4.51584685079614
0.52 -4.51584685079614
0.54 -4.51584685079614
0.56 -4.51584685079614
0.58 -4.51584685079614
0.6 -2.73511401592434
0.62 -2.73511401592434
0.64 -2.73511401592434
0.66 -2.73511401592434
0.68 -2.73511401592434
0.7 -2.73511401592434
0.72 -2.73511401592434
0.74 -2.73511401592434
0.76 -2.73511401592434
0.78 -2.73511401592434
0.8 4.27872817496404
0.82 4.27872817496404
0.84 4.27872817496404
0.86 4.27872817496404
0.88 4.27872817496404
0.9 4.27872817496404
0.92 4.27872817496404
0.94 4.27872817496404
0.96 4.27872817496404
0.98 4.27872817496404
1 -2.32605861965384
1.02 -2.32605861965384
1.04 -2.32605861965384
1.06 -2.32605861965384
1.08 -2.32605861965384
1.1 -2.32605861965384
1.12 -2.32605861965384
1.14 -2.32605861965384
1.16 -2.32605861965384
1.18 -2.32605861965384
1.2 -0.112986029777426
1.22 -0.112986029777426
1.24 -0.112986029777426
1.26 -0.112986029777426
1.28 -0.112986029777426
1.3 -0.112986029777426
1.32 -0.112986029777426
1.34 -0.112986029777426
1.36 -0.112986029777426
1.38 -0.112986029777426
1.4 2.34371637484658
1.42 2.34371637484658
1.44 2.34371637484658
1.46 2.34371637484658
1.48 2.34371637484658
1.5 2.34371637484658
1.52 2.34371637484658
1.54 2.34371637484658
1.56 2.34371637484658
1.58 2.34371637484658
1.6 -2.05921523107713
1.62 -2.05921523107713
1.64 -2.05921523107713
1.66 -2.05921523107713
1.68 -2.05921523107713
1.7 -2.05921523107713
1.72 -2.05921523107713
1.74 -2.05921523107713
1.76 -2.05921523107713
1.78 -2.05921523107713
1.8 1.32888331620092
1.82 1.32888331620092
1.84 1.32888331620092
1.86 1.32888331620092
1.88 1.32888331620092
1.9 1.32888331620092
1.92 1.32888331620092
1.94 1.32888331620092
1.96 1.32888331620092
1.98 1.32888331620092
2 -1.55382296139641
2.02 -1.55382296139641
2.04 -1.55382296139641
2.06 -1.55382296139641
2.08 -1.55382296139641
2.1 -1.55382296139641
2.12 -1.55382296139641
2.14 -1.55382296139641
2.16 -1.55382296139641
2.18 -1.55382296139641
2.2 -2.8240244606497
2.22 -4.49171574318041
2.24 2.83975870984957
2.26 -1.24549768344202
2.28 -1.24900629735892
2.3 2.11153868623588
2.32 -2.65225832555581
2.34 -1.09160288740635
2.36 4.08347006157089
2.38 3.30290891068872
2.4 -4.03923993612508
2.42 -1.82830652854057
2.44 -1.19548914874326
2.46 -2.08246821337696
2.48 -4.34817804647696
2.5 4.7874557952138
2.52 3.90120692050462
2.54 -0.267606297122881
2.56 2.3242005439542
2.58 -4.25220862372126
2.6 -3.90622244548858
2.62 -3.90622244548858
2.64 -3.90622244548858
2.66 -3.90622244548858
2.68 -3.90622244548858
2.7 -3.90622244548858
2.72 -3.90622244548858
2.74 -3.90622244548858
2.76 -3.90622244548858
2.78 -3.90622244548858
2.8 1.46606102232547
2.82 1.46606102232547
2.84 1.46606102232547
2.86 1.46606102232547
2.88 1.46606102232547
2.9 1.46606102232547
2.92 1.46606102232547
2.94 1.46606102232547
2.96 1.46606102232547
2.98 1.46606102232547
3 3.39668298046635
3.02 3.39668298046635
3.04 3.39668298046635
3.06 3.39668298046635
3.08 3.39668298046635
3.1 3.39668298046635
3.12 3.39668298046635
3.14 3.39668298046635
3.16 3.39668298046635
3.18 3.39668298046635
3.2 2.25091516400766
3.22 2.25091516400766
3.24 2.25091516400766
3.26 2.25091516400766
3.28 2.25091516400766
3.3 2.25091516400766
3.32 2.25091516400766
3.34 2.25091516400766
3.36 2.25091516400766
3.38 2.25091516400766
3.4 -2.20044401632527
3.42 -2.20044401632527
3.44 -2.20044401632527
3.46 -2.20044401632527
3.48 -2.20044401632527
3.5 -2.20044401632527
3.52 -2.20044401632527
3.54 -2.20044401632527
3.56 -2.20044401632527
3.58 -2.20044401632527
3.6 -4.04059357373558
3.62 -4.04059357373558
3.64 -4.04059357373558
3.66 -4.04059357373558
3.68 -4.04059357373558
3.7 -4.04059357373558
3.72 -4.04059357373558
3.74 -4.04059357373558
3.76 -4.04059357373558
3.78 -4.04059357373558
3.8 -0.195290296691431
3.82 -0.195290296691431
3.84 -0.195290296691431
3.86 -0.195290296691431
3.88 -0.195290296691431
3.9 -0.195290296691431
3.92 -0.195290296691431
3.94 -0.195290296691431
3.96 -0.195290296691431
3.98 -0.195290296691431
4 -3.96685679291051
4.02 -3.96685679291051
4.04 -3.96685679291051
4.06 -3.96685679291051
4.08 -3.96685679291051
4.1 -3.96685679291051
4.12 -3.96685679291051
4.14 -3.96685679291051
4.16 -3.96685679291051
4.18 -3.96685679291051
4.2 -1.55197555731833
4.22 -1.55197555731833
4.24 -1.55197555731833
4.26 -1.55197555731833
4.28 -1.55197555731833
4.3 -1.55197555731833
4.32 -1.55197555731833
4.34 -1.55197555731833
4.36 -1.55197555731833
4.38 -1.55197555731833
4.4 1.08925376257295
4.42 1.08925376257295
4.44 1.08925376257295
4.46 1.08925376257295
4.48 1.08925376257295
4.5 1.08925376257295
4.52 1.08925376257295
4.54 1.08925376257295
4.56 1.08925376257295
4.58 1.08925376257295
4.6 2.31289074869928
4.62 2.31289074869928
4.64 2.31289074869928
4.66 2.31289074869928
4.68 2.31289074869928
4.7 2.31289074869928
4.72 2.31289074869928225
4.74 2.31289074869928
4.76 2.31289074869928
4.78 2.31289074869928
4.8 -3.38348842701702
4.82 -3.38348842701702
4.84 -3.38348842701702
4.86 -3.38348842701702
4.88 -3.38348842701702
4.9 -3.38348842701702
4.92 -3.38348842701702
4.94 -3.38348842701702
4.96 -3.38348842701702
4.98 -3.38348842701702
5 4.84180416886178
};
%\addlegendentry{first component}
\addlegendentry{$v$} % (m/s)}
\addplot [line width = \lineWidthStep, lightblue, const plot mark right]
table {%
0 -4.0516577179317
0.02 -4.0516577179317
0.04 -4.0516577179317
0.06 -4.0516577179317
0.08 -4.0516577179317
0.1 -4.0516577179317
0.12 -4.0516577179317
0.14 -4.0516577179317
0.16 -4.0516577179317
0.18 -4.0516577179317
0.2 -1.180240226855
0.22 -1.180240226855
0.24 -1.180240226855
0.26 -1.180240226855
0.28 -1.180240226855
0.3 -1.180240226855
0.32 -1.180240226855
0.34 -1.180240226855
0.36 -1.180240226855
0.38 -1.180240226855
0.4 -1.00300739355721
0.42 -1.00300739355721
0.44 -1.00300739355721
0.46 -1.00300739355721
0.48 -1.00300739355721
0.5 -1.00300739355721
0.52 -1.00300739355721
0.54 -1.00300739355721
0.56 -1.00300739355721
0.58 -1.00300739355721
0.6 0.310790093332731
0.62 0.310790093332731
0.64 0.310790093332731
0.66 0.310790093332731
0.68 0.310790093332731
0.7 0.310790093332731
0.72 0.310790093332731
0.74 0.310790093332731
0.76 0.310790093332731
0.78 0.310790093332731
0.8 2.26861231335311
0.82 2.26861231335311
0.84 2.26861231335311
0.86 2.26861231335311
0.88 2.26861231335311
0.9 2.26861231335311
0.92 2.26861231335311
0.94 2.26861231335311
0.96 2.26861231335311
0.98 2.26861231335311
1 0.760874509666037
1.02 0.760874509666037
1.04 0.760874509666037
1.06 0.760874509666037
1.08 0.760874509666037
1.1 0.760874509666037
1.12 0.760874509666037
1.14 0.760874509666037
1.16 0.760874509666037
1.18 0.760874509666037
1.2 -1.16665508714342
1.22 -1.16665508714342
1.24 -1.16665508714342
1.26 -1.16665508714342
1.28 -1.16665508714342
1.3 -1.16665508714342
1.32 -1.16665508714342
1.34 -1.16665508714342
1.36 -1.16665508714342
1.38 -1.16665508714342
1.4 -0.0332526750470885
1.42 -0.0332526750470885
1.44 -0.0332526750470885
1.46 -0.0332526750470885
1.48 -0.0332526750470885
1.5 -0.0332526750470885
1.52 -0.0332526750470885
1.54 -0.0332526750470885
1.56 -0.0332526750470885
1.58 -0.0332526750470885
1.6 3.7834738758291
1.62 3.7834738758291
1.64 3.7834738758291
1.66 3.7834738758291
1.68 3.7834738758291
1.7 3.7834738758291
1.72 3.7834738758291
1.74 3.7834738758291
1.76 3.7834738758291
1.78 3.7834738758291
1.8 1.27154635354052
1.82 1.27154635354052
1.84 1.27154635354052
1.86 1.27154635354052
1.88 1.27154635354052
1.9 1.27154635354052
1.92 1.27154635354052
1.94 1.27154635354052
1.96 1.27154635354052
1.98 1.27154635354052
2 4.7751018051278
2.02 4.7751018051278
2.04 4.7751018051278
2.06 4.7751018051278
2.08 4.7751018051278
2.1 4.7751018051278
2.12 4.7751018051278
2.14 4.7751018051278
2.16 4.7751018051278
2.18 4.7751018051278
2.2 5
2.22 5
2.24 5
2.26 5
2.28 5
2.3 -4.69508060657476
2.32 3.66927029875222
2.34 -0.799341059262959
2.36 4.53356632935111
2.38 -0.918431785896697
2.4 -2.00948647444005
2.42 -0.933434258423635
2.44 2.96794018458355
2.46 0.683980776228521
2.48 1.76327499102879
2.5 -1.84443274190389
2.52 2.02987731137769
2.54 -4.63294452565045
2.56 -3.3657007906448
2.58 2.89181945226332
2.6 1.50535713838971
2.62 1.50535713838971
2.64 1.50535713838971
2.66 1.50535713838971
2.68 1.50535713838971
2.7 1.50535713838971
2.72 1.50535713838971
2.74 1.50535713838971
2.76 1.50535713838971
2.78 1.50535713838971
2.8 4.60306797038857
2.82 4.60306797038857
2.84 4.60306797038857
2.86 4.60306797038857
2.88 4.60306797038857
2.9 4.60306797038857
2.92 4.60306797038857
2.94 4.60306797038857
2.96 4.60306797038857
2.98 4.60306797038857
3 -2.83139021974241
3.02 -2.83139021974241
3.04 -2.83139021974241
3.06 -2.83139021974241
3.08 -2.83139021974241
3.1 -2.83139021974241
3.12 -2.83139021974241
3.14 -2.83139021974241
3.16 -2.83139021974241
3.18 -2.83139021974241
3.2 -3.92179697027303
3.22 -3.92179697027303
3.24 -3.92179697027303
3.26 -3.92179697027303
3.28 -3.92179697027303
3.3 -3.92179697027303
3.32 -3.92179697027303
3.34 -3.92179697027303
3.36 -3.92179697027303
3.38 -3.92179697027303
3.4 4.79578670470296
3.42 4.79578670470296
3.44 4.79578670470296
3.46 4.79578670470296
3.48 4.79578670470296
3.5 4.79578670470296
3.52 4.79578670470296
3.54 4.79578670470296
3.56 4.79578670470296
3.58 4.79578670470296
3.6 -2.77991655989354
3.62 -2.77991655989354
3.64 -2.77991655989354
3.66 -2.77991655989354
3.68 -2.77991655989354
3.7 -2.77991655989354
3.72 -2.77991655989354
3.74 -2.77991655989354
3.76 -2.77991655989354
3.78 -2.77991655989354
3.8 2.85936753189501
3.82 2.85936753189501
3.84 2.85936753189501
3.86 2.85936753189501
3.88 2.85936753189501
3.9 2.85936753189501
3.92 2.85936753189501
3.94 2.85936753189501
3.96 2.85936753189501
3.98 2.85936753189501
4 4.77112039648313
4.02 4.77112039648313
4.04 4.77112039648313
4.06 4.77112039648313
4.08 4.77112039648313
4.1 4.77112039648313
4.12 4.77112039648313
4.14 4.77112039648313
4.16 4.77112039648313
4.18 4.77112039648313
4.2 -4.33707249067705
4.22 -4.33707249067705
4.24 -4.33707249067705
4.26 -4.33707249067705
4.28 -4.33707249067705
4.3 -4.33707249067705
4.32 -4.33707249067705
4.34 -4.33707249067705
4.36 -4.33707249067705
4.38 -4.33707249067705
4.4 0.966161570640592
4.42 0.966161570640592
4.44 0.966161570640592
4.46 0.966161570640592
4.48 0.966161570640592
4.5 0.966161570640592
4.52 0.966161570640592
4.54 0.966161570640592
4.56 0.966161570640592
4.58 0.966161570640592
4.6 -3.2123251675775
4.62 -3.2123251675775
4.64 -3.2123251675775
4.66 -3.2123251675775
4.68 -3.2123251675775
4.7 -3.2123251675775
4.72 -3.2123251675775
4.74 -3.2123251675775
4.76 -3.2123251675775
4.78 -3.2123251675775
4.8 0.525119154814711
4.82 0.525119154814711
4.84 0.525119154814711
4.86 0.525119154814711
4.88 0.525119154814711
4.9 0.525119154814711
4.92 0.525119154814711
4.94 0.525119154814711
4.96 0.525119154814711
4.98 0.525119154814711
5 5
};
%\addlegendentry{second component}
\addlegendentry{$\omega$} % (rad/s)}
\end{axis}

\begin{axis}[
width = .36\textwidth,
height = \heightODEbottom,
at={(0cm, \yBottomPlots)},
legend cell align={left},
%legend style={fill opacity=1, draw opacity=1, text opacity=1, draw=black, at={(1,1)}, anchor=north east},
%log basis y={10},
%tick align=outside,
%tick pos=left,
%x grid style={darkgray176},
%xmin=-0.249, 
xmin=0,
xmax=5,
xtick style={color=black},
%y grid style={darkgray176},
ymin=5.8601356445684e-06, ymax=2.87252944261343,
ymode=log,
%ytick style={color=black},
xlabel = {time (s)}, 
%ylabel = {to do}
]

\addplot [line width = \lineWidthStep, color = SUR1B1color]
table {%
0 1.58347753387042
0.02 0.00149446028785516
0.04 0.00148507399086975
0.06 0.0014753765130362
0.08 0.00146611732329072
0.1 0.00145808326443384
0.12 0.00145202867834363
0.14 0.0014485968133025
0.16 0.00144823985767335
0.18 0.00145114659229985
0.2 0.00145718646076246
0.22 0.000783889455926503
0.24 0.00078925683626517
0.26 0.00079480485513154
0.28 0.000800492170041208
0.3 0.000806275745664014
0.32 0.000812110651492387
0.34 0.000817949777514173
0.36 0.000823743479459648
0.38 0.000829439169977937
0.4 0.000834980877374436
0.42 0.000936209759378704
0.44 0.000943501877806315
0.46 0.00095046132433912
0.48 0.00095696616508412
0.5 0.000962894624040342
0.52 0.00096812673276968
0.54 0.000972546073828551
0.56 0.000976041645418274
0.58 0.000978509874922779
0.6 0.000979856819220319
0.62 0.000149906836123739
0.64 0.000149592201694194
0.66 0.000149270453227587
0.68 0.000148946084659336
0.7 0.000148623831290454
0.72 0.0001483084686339
0.74 0.00014800461390234
0.76 0.00014771653924505
0.78 0.000147448005914777
0.8 0.000147202128109391
0.82 0.0018806761889775
0.84 0.00188598886213219
0.86 0.00189460337415345
0.88 0.00190591293246686
0.9 0.0019192256996105
0.92 0.00193381048524277
0.94 0.00194894055329313
0.96 0.00196393060790598
0.98 0.00197816483771978
1 0.00199111586769927
1.02 0.000387256034348722
1.04 0.000389078432616476
1.06 0.000390793612382997
1.08 0.000392396326857214
1.1 0.000393881689300013
1.12 0.000395245183605617
1.14 0.000396482678456645
1.16 0.000397590444314475
1.18 0.000398565172429622
1.2 0.000399403995008672
1.22 2.41840225906587e-05
1.24 2.42257038963868e-05
1.26 2.42842115325698e-05
1.28 2.43593076294707e-05
1.3 2.44506207555879e-05
1.32 2.45576466948673e-05
1.34 2.46797496060255e-05
1.36 2.48161632847996e-05
1.38 2.49659922556282e-05
1.4 2.51282124293632e-05
1.42 1.25547175908661e-05
1.44 1.22568229142381e-05
1.46 1.19734855586056e-05
1.48 1.17079101810897e-05
1.5 1.14630416634165e-05
1.52 1.1241535229907e-05
1.54 1.10457293855902e-05
1.56 1.08776216964167e-05
1.58 1.0738847375617e-05
1.6 1.06306605661692e-05
1.62 0.00157644613701175
1.64 0.00158632036485842
1.66 0.00159353657038787
1.68 0.00159758493514983
1.7 0.00159830260691401
1.72 0.00159582775306827
1.74 0.00159054049803131
1.76 0.0015829986768289
1.78 0.00157387343438472
1.8 0.00156388784088725
1.82 0.000335184907466118
1.84 0.000332961534604493
1.86 0.000330793589973574
1.88 0.000328695596065422
1.9 0.000326681129834742
1.92 0.000324762738294699
1.94 0.000322951862668836
1.96 0.000321258772026467
1.98 0.000319692507325708
2 0.000318260836740707
2.02 0.00145917990809807
2.04 0.00145521856838356
2.06 0.0014536858572373
2.08 0.00145443051691393
2.1 0.001457083549735
2.12 0.00146112522905478
2.14 0.00146595951771153
2.16 0.00147098668209612
2.18 0.00147566709403204
2.2 0.00147957174480461
2.22 0.00282113117476739
2.24 0.00449205003129529
2.26 0.00284085937549583
2.28 0.00124554938524666
2.3 0.00124815575019138
2.32 0.00198300672126186
2.34 0.00194552129025725
2.36 0.000176325299563645
2.38 0.00369823792531303
2.4 0.000613000892788778
2.42 0.00162802324533069
2.44 0.000344211880854302
2.46 0.000708795493161517
2.48 0.000285303775767706
2.5 0.00152978574468058
2.52 0.001773228336107
2.54 0.00158018264064764
2.56 0.000248143116384766
2.58 0.00156562990945408
2.6 0.0024581286369714
2.62 0.0011751580538539
2.64 0.00117420465149301
2.66 0.00117329967923222
2.68 0.00117246947913492
2.7 0.0011717430611346
2.72 0.00117115186595011
2.74 0.00117072929737746
2.76 0.00117051005628814
2.78 0.00117052931779751
2.8 0.00117082179712571
2.82 0.00134699800167308
2.84 0.00134881693604686
2.86 0.00135171261041659
2.88 0.00135551904692358
2.9 0.00135992617740729
2.92 0.00136450687771062
2.94 0.00136875886515794
2.96 0.00137215701295315
2.98 0.00137421079454524
3 0.00137452148196457
3.02 0.00187047449872082
3.04 0.00186756420610423
3.06 0.00186645145723204
3.08 0.00186702677831611
3.1 0.00186912553627228
3.12 0.00187253858839433
3.14 0.00187702397670385
3.16 0.00188231906660618
3.18 0.00188815255129969
3.2 0.00189425580714349
3.22 0.00175008670407556
3.24 0.00175542731301481
3.26 0.00176011560082299
3.28 0.00176388746164539
3.3 0.00176658125079866
3.32 0.0017681434263073
3.34 0.00176862723608424
3.36 0.0017681847084106
3.38 0.00176705229545089
3.4 0.00176553068952365
3.42 0.00210973050353564
3.44 0.00210783903344116
3.46 0.00210616898871065
3.48 0.00210532711171444
3.5 0.0021058211267404
3.52 0.00210798382404006
3.54 0.00211191477530175
3.56 0.00211744890928272
3.58 0.00212415801521204
3.6 0.00213138665370984
3.62 0.00219646620038231
3.64 0.00220350502597942
3.66 0.00221084748413218
3.68 0.002218207059606
3.7 0.0022253171739399
3.72 0.00223193881631204
3.74 0.00223786746644433
3.76 0.00224293919948572
3.78 0.00224703574369201
3.8 0.00225008816153256
3.82 0.000111356964697898
3.84 0.000111453974918812
3.86 0.000111608504627121
3.88 0.000111820018752572
3.9 0.000112084753012354
3.92 0.000112395759795493
3.94 0.000112743114488377
3.96 0.000113114272832313
3.98 0.000113494560882382
4 0.000113867772379025
4.02 0.00383633423232417
4.04 0.00384624322840275
4.06 0.00385229905158846
4.08 0.00385324190696368
4.1 0.00384824434732844
4.12 0.0038370503904132
4.14 0.00382007172671207
4.16 0.00379842844452942
4.18 0.00377392078164717
4.2 0.00374891824335849
4.22 0.00136927704358275
4.24 0.00136170980576214
4.26 0.00135326620963989
4.28 0.00134474902839181
4.3 0.00133688236887116
4.32 0.00133025499711639
4.34 0.00132528190634252
4.36 0.00132218573176415
4.38 0.00132099796991599
4.4 0.00132157798861704
4.42 0.000226774312927103
4.44 0.000227177260258225
4.46 0.00022753794433974
4.48 0.000227853306630375
4.5 0.000228120428573421
4.52 0.000228336550275541
4.54 0.000228499088513412
4.56 0.000228605653996293
4.58 0.000228654067795633
4.6 0.00022864237688286
4.62 0.00144853136326707
4.64 0.0014484717055585
4.66 0.00144982260380429
4.68 0.00145236771337253
4.7 0.00145585318081655
4.72 0.00146000433743423
4.74 0.00146454154415616
4.76 0.00146919452486518
4.78 0.00147371476329341
4.8 0.00147788573887356
4.82 0.000361524953612575
4.84 0.000362174076098811
4.86 0.000362890234432077
4.88 0.000363682386775017
4.9 0.000364559142030462
4.92 0.000365528399234528
4.94 0.000366596917233856
4.96 0.000367769805080334
4.98 0.000369049924511745
};
%\addlegendentry{one-step $B_1$}
\addplot [line width = \lineWidthStep, color = SUR1B2color, dashed]
table {%
0 1.58347753387042
0.02 0.00133244306525194
0.04 0.00132305726025289
0.06 0.00131335928640531
0.08 0.00130409877385577
0.1 0.00129606311575781
0.12 0.00129000730695937
0.14 0.00128657493575109
0.16 0.00128621797302984
0.18 0.00128912469986125
0.2 0.00129516471123983
0.22 0.000736657597725507
0.24 0.000742026832408609
0.26 0.000747578461142196
0.28 0.000753271489204822
0.3 0.00075906320567694
0.32 0.000764908966786829
0.34 0.000770761896847398
0.36 0.000776572518154721
0.38 0.000782288325931291
0.4 0.000787853329571413
0.42 0.000976330973590175
0.44 0.000983612756955404
0.46 0.000990558280477714
0.48 0.000997046073631522
0.5 0.00100295506238835
0.52 0.0010081662003897
0.54 0.00101256418454609
0.56 0.00101603928307108
0.58 0.00101848930331059
0.6 0.00101982173569995
0.62 0.000161154768545207
0.64 0.000160844472870254
0.66 0.000160532855959256
0.68 0.00016022397519209
0.7 0.000159922054690173
0.72 0.000159631297895632
0.74 0.000159355704466234
0.76 0.000159098900200085
0.78 0.000158863988642189
0.8 0.000158653432543871
0.82 0.00179019603558359
0.84 0.00179545527274859
0.86 0.00180401250995169
0.88 0.00181526932233958
0.9 0.0018285383757354
0.92 0.00184308988107254
0.94 0.00185819644460244
0.96 0.00187317103391597
0.98 0.00188739577162112
1 0.00190034136573872
1.02 0.000417638124924653
1.04 0.000419456695849117
1.06 0.000421168460438834
1.08 0.000422768194388105
1.1 0.000424251024845417
1.12 0.000425612441816805
1.14 0.000426848313096206
1.16 0.000427954902001441
1.18 0.000428928887100774
1.2 0.000429767383070186
1.22 7.08494620149445e-05
1.24 7.08910663501663e-05
1.26 7.09496545234184e-05
1.28 7.10249982170738e-05
1.3 7.11167205538546e-05
1.32 7.12242935063016e-05
1.34 7.13470362207599e-05
1.36 7.14841142253426e-05
1.38 7.16345394850886e-05
1.4 7.17971712756302e-05
1.42 1.27115814918572e-05
1.44 1.24088882048562e-05
1.46 1.21207725591219e-05
1.48 1.18505028694436e-05
1.5 1.16010887392368e-05
1.52 1.13752490373151e-05
1.54 1.11753822096089e-05
1.56 1.10035390839369e-05
1.58 1.08613983469017e-05
1.6 1.07502448610061e-05
1.62 0.0017277371432707
1.64 0.00173761472220253
1.66 0.00174483186843585
1.68 0.00174888033917163
1.7 0.00174959770317421
1.72 0.00174712208983091
1.74 0.001741833630925
1.76 0.00173429041022795
1.78 0.00172516400133353
1.8 0.00171517788957858
1.82 0.000284317336317363
1.84 0.000282094029270762
1.86 0.000279926199002567
1.88 0.000277828403453126
1.9 0.000275814261508673
1.92 0.000273896365640607
1.94 0.000272086202603947
1.96 0.000270394083321609
1.98 0.000268829083106768
2 0.000267398993367718
2.02 0.0016501112961334
2.04 0.00164615412419098
2.06 0.00164462435782963
2.08 0.0016453704982608
2.1 0.00164802348178193
2.12 0.00165206329890207
2.14 0.00165689319553565
2.16 0.00166191240313343
2.18 0.00166658036516331
2.2 0.00167046780640161
2.22 0.00302097542611595
2.24 0.00469186078132347
2.26 0.00264103230402798
2.28 0.00144534325272639
2.3 0.0014479410785818
2.32 0.001795392394562
2.34 0.00209214931557289
2.36 0.000208126244679819
2.38 0.00351706892709612
2.4 0.00057656805234693
2.42 0.00170832808010949
2.44 0.000381411926390931
2.46 0.000827390820582867
2.48 0.000312493533842657
2.5 0.00160024606747499
2.52 0.00169962309035626
2.54 0.00149915769096609
2.56 0.000433284801498092
2.58 0.00143113895205381
2.6 0.00257369374149942
2.62 0.00123529068938782
2.64 0.00123433933944943
2.66 0.00123343732759222
2.68 0.00123261097169955
2.7 0.00123188923090854
2.72 0.00123130346328546
2.74 0.00123088695436798
2.76 0.00123067424958016
2.78 0.00123070033248664
2.8 0.00123099969470544
2.82 0.00116300944940845
2.84 0.00116480547470721
2.86 0.00116768184501109
2.88 0.00117147407308513
2.9 0.00117587212110082
2.92 0.0011804478470816
2.94 0.00118469754998371
2.96 0.00118809488274968
2.98 0.00119014862725155
3 0.00119045988056248
3.02 0.00175720709624944
3.04 0.00175429669875881
3.06 0.00175318404968419
3.08 0.00175375950983314
3.1 0.001755858300171
3.12 0.00175927121917976
3.14 0.00176375639005297
3.16 0.00176905142461372
3.18 0.00177488541889388
3.2 0.00178099026616808
3.22 0.00159325147879738
3.24 0.0015985991531557
3.26 0.00160329847069665
3.28 0.00160708536580412
3.3 0.00160979743180365
3.32 0.00161137957418859
3.34 0.00161188289029962
3.36 0.00161145700205538
3.38 0.00161033612689394
3.4 0.00160881931978828
3.42 0.00230135325459538
3.44 0.00229945995363701
3.46 0.00229779743237258
3.48 0.00229697098883302
3.5 0.00229748548799363
3.52 0.00229967023671409
3.54 0.00230362161862268
3.56 0.00230917239608885
3.58 0.00231589351188073
3.6 0.00232312982316768
3.62 0.00230760104044789
3.64 0.0023146440719821
3.66 0.00232199095313163
3.68 0.0023293545259985
3.7 0.00233646755567495
3.72 0.00234309044448133
3.74 0.0023490182338629
3.76 0.00235408677387171
3.78 0.00235817782221014
3.8 0.00236122273694439
3.82 0.000225652389111819
3.84 0.000225764037264692
3.86 0.00022593303836557
3.88 0.000226158233011946
3.9 0.000226435380784866
3.92 0.000226757224940703
3.94 0.000227113695172003
3.96 0.000227492242259677
3.98 0.000227878292161057
4 0.000228255801832839
4.02 0.00402706886820168
4.04 0.00403697857446372
4.06 0.00404303438909585
4.08 0.00404397671615668
4.1 0.00403897808004224
4.12 0.00402778257948993
4.14 0.00401080231200806
4.16 0.00398915804643722
4.18 0.00396465064795207
4.2 0.00393964976898128
4.22 0.00154268680010073
4.24 0.00153511763254431
4.26 0.00152667310306753
4.28 0.00151815631740257
4.3 0.00151029122684125
4.32 0.0015036660577221
4.34 0.00149869515322606
4.36 0.00149560066816766
4.38 0.00149441392142062
4.4 0.00149499433044381
4.42 0.000188148847666855
4.44 0.00018855238323918
4.46 0.000188913394654759
4.48 0.000189228827348009
4.5 0.000189495771748643
4.52 0.000189711481437242
4.54 0.000189873390586321
4.56 0.000189979130596044
4.58 0.000190026545885078
4.6 0.000190013708772957
4.62 0.0013200478416053
4.64 0.00131998853537427
4.66 0.00132133955170511
4.68 0.00132388443581286
4.7 0.00132736945057536
4.72 0.00133152030817117
4.74 0.0013360579833444
4.76 0.0013407129509752
4.78 0.00134523744307775
4.8 0.00134941552874235
4.82 0.000382550274117595
4.84 0.000383198388237999
4.86 0.000383913142822616
4.88 0.000384703438097657
4.9 0.000385577818911419
4.92 0.000386544117418846
4.94 0.000387609027295609
4.96 0.000388777600028776
4.98 0.000390052654722644
};
%\addlegendentry{one-step $B_2$}
\end{axis}



\begin{axis}[
width = .36\textwidth,
height = \heightODEbottom,
at={(.37\textwidth,\yBottomPlots)},
legend cell align={left},
%legend style={
%  fill opacity=1,
%  draw opacity=1,
%  text opacity=1,
%  at={(1,0)},
%  anchor=south east,
%},
%log basis y={10},
%tick align=outside,
%tick pos=left,
%x grid style={darkgray176},
%xmin=-0.25,
xmin=0,
xmax=5,
%xtick style={color=black},
%y grid style={darkgray176},
%ymin=0.00108995629643699, ymax=0.0905064985589633,
ymin=0.0007, ymax=0.12,
ymode=log,
ytick style={color=black},
xlabel = {time (s)},
%ylabel = {to do}
]
\addplot [line width = \lineWidthStep, color = SUR1B1color]
table {%
0 0
0.02 0.00149446028785516
0.04 0.00297710444532399
0.06 0.0044451469898257
0.08 0.00589650343928863
0.1 0.00732985433532172
0.12 0.00874466531618083
0.14 0.0101411525745743
0.16 0.0115201834779377
0.18 0.0128831030987115
0.2 0.014231478869636
0.22 0.0149439841309389
0.24 0.0156625692772453
0.26 0.0163872230650036
0.28 0.0171179195317756
0.3 0.0178546090588055
0.32 0.0185972105201633
0.34 0.0193456040721488
0.36 0.0200996242212989
0.38 0.0208590528730136
0.4 0.0216236121134661
0.42 0.020769248985234
0.44 0.0199199402347237
0.46 0.0190779642088579
0.48 0.0182461290342875
0.5 0.0174278569502365
0.52 0.0166272797133523
0.54 0.0158493435464297
0.56 0.0150999186185366
0.58 0.0143859021606514
0.6 0.0137152952106563
0.62 0.0137251532802911
0.64 0.0137376767660857
0.66 0.0137530081230447
0.68 0.0137712752305395
0.7 0.0137925902463046
0.72 0.0138170486782279
0.74 0.0138447286962662
0.76 0.0138756907048575
0.78 0.0139099771941483
0.8 0.0139476128862019
0.82 0.0127926878882567
0.84 0.0117358331444579
0.86 0.0108074972435551
0.88 0.0100484895091164
0.9 0.0095073766600802
0.92 0.00923227938303832
0.94 0.00925726766052086
0.96 0.00958975630744832
0.98 0.0102080793674447
1 0.0110708897947458
1.02 0.0109028033688994
1.04 0.0107515710048476
1.06 0.0106181485536779
1.08 0.0105034295144059
1.1 0.0104082244786605
1.12 0.0103332407465766
1.14 0.0102790631807719
1.16 0.0102461374306681
1.18 0.0102347566152356
1.2 0.010245052387354
1.22 0.0102462439795133
1.24 0.0102480645421702
1.26 0.0102505053893126
1.28 0.0102535576302785
1.3 0.0102572122353105
1.32 0.0102614601028767
1.34 0.0102662921285394
1.36 0.0102716992751371
1.38 0.0102776726440268
1.4 0.0102842035471118
1.42 0.010275516393764
1.44 0.0102670083105492
1.46 0.010258670909739
1.48 0.0102504940132496
1.5 0.0102424657742293
1.52 0.0102345728156162
1.54 0.0102268003849022
1.56 0.0102191325242336
1.58 0.0102115522548775
1.6 0.0102040417749545
1.62 0.00995010734668606
1.64 0.0100685386628672
1.66 0.0105482566882003
1.68 0.0113411235754076
1.7 0.0123815873264819
1.72 0.0136050852384968
1.74 0.0149573431250627
1.76 0.0163962338750675
1.78 0.0178902316759518
1.8 0.0194160574472433
1.82 0.0190843755109529
1.84 0.0187538372947048
1.86 0.0184245655493525
1.88 0.0180966803057771
1.9 0.0177703002695184
1.92 0.0174455442818504
1.94 0.0171225328472079
1.96 0.0168013897275586
1.98 0.0164822436049312
2 0.0161652298138056
2.02 0.0176144710856149
2.04 0.0190402818036854
2.06 0.020437187720992
2.08 0.0218010782091559
2.1 0.0231286414494226
2.12 0.0244169651258531
2.14 0.025663254567937
2.16 0.0268646409220672
2.18 0.0280180624436651
2.2 0.0291202070945497
2.22 0.0311400970533875
2.24 0.0343131300543373
2.26 0.0324755595568814
2.28 0.0331673168058526
2.3 0.0337895561282279
2.32 0.034719833363596
2.34 0.0358527910037909
2.36 0.0357759877115462
2.38 0.033919668859741
2.4 0.0341079941050615
2.42 0.0335082214098767
2.44 0.0333976764085437
2.46 0.0337121209473446
2.48 0.0338572205043739
2.5 0.0345249918665085
2.52 0.035186879373135
2.54 0.0344404219718853
2.56 0.034342164343738
2.58 0.0350907083319516
2.6 0.0365474753102052
2.62 0.0372646512561291
2.64 0.0379758117899265
2.66 0.0386803168291635
2.68 0.0393775511705466
2.7 0.0400669238307358
2.72 0.0407478685268871
2.74 0.0414198451060744
2.76 0.0420823417130691
2.78 0.0427348774840783
2.8 0.0433770055680908
2.82 0.0427365643680459
2.84 0.0422482721372645
2.86 0.0419225177712494
2.88 0.0417666058299155
2.9 0.0417842467874623
2.92 0.0419752536991608
2.94 0.0423354946908985
2.96 0.0428571126149142
2.98 0.0435289826507394
3 0.0443373467126324
3.02 0.0432154081937072
3.04 0.0422402155487373
3.06 0.041423561154971
3.08 0.0407761855168403
3.1 0.0403072437953135
3.12 0.0400237764427969
3.14 0.039930248703873
3.16 0.0400282240679789
3.18 0.0403162221135394
3.2 0.0407897834825335
3.22 0.041396395225967
3.24 0.0421868739733833
3.26 0.0431472564290237
3.28 0.044261725779952
3.3 0.0455133296761387
3.32 0.0468846184677887
3.34 0.0483581686226066
3.36 0.0499169810775196
3.38 0.0515447618879534
3.4 0.0532261023075981
3.42 0.0552787079316611
3.44 0.0572805216372769
3.46 0.0592183009582451
3.48 0.0610802143432022
3.5 0.0628557562383686
3.52 0.064535669918807
3.54 0.0661118696408961
3.56 0.0675773563389698
3.58 0.0689261239660054
3.6 0.0701530564530494
3.62 0.0689324337151096
3.64 0.0676590797962214
3.66 0.0663346489212839
3.68 0.0649608492011109
3.7 0.0635394164261159
3.72 0.062072087337989
3.74 0.0605605729196754
3.76 0.0590065324359046
3.78 0.0574115491644691
3.8 0.055777108940113
3.82 0.0558616205291975
3.84 0.0559416955917844
3.86 0.0560170607464474
3.88 0.0560874613775175
3.9 0.0561526629613664
3.92 0.0562124522514629
3.94 0.0562666383061411
3.96 0.0563150533447339
3.98 0.0563575534196029
4 0.0563940188936289
4.02 0.0575028702204605
4.04 0.0584954142954434
4.06 0.0593712847992903
4.08 0.0601296969204658
4.1 0.0607692155627788
4.12 0.0612876429104351
4.14 0.0616820456735643
4.16 0.061948931453177
4.18 0.062084571080677
4.2 0.0620854499679522
4.22 0.0620915914936805
4.24 0.0620112266107336
4.26 0.0618446086701551
4.28 0.0615921867568566
4.3 0.0612546523156464
4.32 0.0608329813440132
4.34 0.0603284692884875
4.36 0.059742757305494
4.38 0.0590778497972727
4.4 0.0583361241167056
4.42 0.0582119447983084
4.44 0.0580919479659255
4.46 0.0579762126627448
4.48 0.0578648145172024
4.5 0.0577578255689423
4.52 0.0576553141021655
4.54 0.0575573444874737
4.56 0.0574639770332937
4.58 0.057375267847944
4.6 0.057291268713372
4.62 0.0579236880652746
4.64 0.0586670246246617
4.66 0.0595141595855557
4.68 0.0604575401936449
4.7 0.0614893197890281
4.72 0.0626014789936962
4.74 0.0637859271655788
4.76 0.0650345847967067
4.78 0.0663394486229993
4.8 0.0676926418538497
4.82 0.0680346791125031
4.84 0.0683758122701912
4.86 0.06871602516242
4.88 0.069055304235229
4.9 0.0693936385433089
4.92 0.0697310197503908
4.94 0.0700674421256762
4.96 0.0704029025261016
4.98 0.0707374003497571
5 0.071070937440917
};
%\addlegendentry{$B_1$}
\addplot [line width = \lineWidthStep, color = SUR1B2color, dashed]
table {%
0 0
0.02 0.00133244306525194
0.04 0.00265333746379145
0.06 0.00396016242983482
0.08 0.00525109718651705
0.1 0.00652508470290298
0.12 0.00778185150358936
0.14 0.0090218728204847
0.16 0.0102462728316013
0.18 0.0114566507329292
0.2 0.0126548248987496
0.22 0.013325575625838
0.24 0.0140027269956388
0.26 0.0146862425376023
0.28 0.0153760772379174
0.3 0.0160721672703946
0.32 0.0167744211725237
0.34 0.0174827119151762
0.36 0.0181968694319892
0.38 0.0189166732604017
0.4 0.0196418450121891
0.42 0.0187480265648454
0.44 0.0178608727341143
0.46 0.016983229371415
0.48 0.016118650915985
0.5 0.015271541397513
0.52 0.0144473204641929
0.54 0.0136526118570688
0.56 0.0128954428297986
0.58 0.0121854266031424
0.6 0.0115338731505515
0.62 0.011539733793048
0.64 0.0115489651521442
0.66 0.0115617077587148
0.68 0.0115780863083522
0.7 0.0115982084008834
0.72 0.011622163526405
0.74 0.0116500223222192
0.76 0.0116818361221065
0.78 0.0117176368162132
0.8 0.0117574370365938
0.82 0.0107266689986732
0.84 0.00982521964604328
0.86 0.0090933391684975
0.88 0.00857983276536338
0.9 0.00833313407182879
0.92 0.00838620059697194
0.94 0.00874306830236688
0.96 0.00937769498359096
0.98 0.0102456288213364
1 0.0112984381600024
1.02 0.011068502120466
1.04 0.010855085521323
1.06 0.0106594639905267
1.08 0.0104828968672877
1.1 0.0103266021901137
1.12 0.0101917287186566
1.14 0.0100793260126079
1.16 0.00999031400793555
1.18 0.00992545387686509
1.2 0.00988532215658075
1.22 0.00989441161541229
1.24 0.00990564687483729
1.26 0.00991900964786535
1.28 0.0099344798223096
1.3 0.00995203556582967
1.32 0.00997165343214107
1.34 0.00999330846734432
1.36 0.0100169743153502
1.38 0.0100426233213848
1.4 0.0100702266325894
1.42 0.0100642888285787
1.44 0.0100584936335845
1.46 0.0100528342878182
1.48 0.010047302718848
1.5 0.0100418896296806
1.52 0.0100365845988016
1.54 0.0100313761917569
1.56 0.010026252083793
1.58 0.0100211991929901
1.6 0.0100162038232326
1.62 0.00947738437660081
1.64 0.00937201007808464
1.66 0.0097180138275425
1.68 0.0104697621979585
1.7 0.0115433094174068
1.72 0.0128503852139496
1.74 0.0143173597475171
1.76 0.0158890749853155
1.78 0.0175259696427312
1.8 0.0191999770570433
1.82 0.0189183151704948
1.84 0.0186379925845917
1.86 0.0183591154456877
1.88 0.0180817850868409
1.9 0.0178060991292326
1.92 0.0175321526181439
1.94 0.017260039190158
1.96 0.0169898522686308
1.98 0.0167216862848029
2 0.0164556379221538
2.02 0.0180940455319663
2.04 0.0197067069440843
2.06 0.0212876312494535
2.08 0.022832313323337
2.1 0.0243371003112277
2.12 0.0257987598258011
2.14 0.0272141833332152
2.16 0.0285801886135848
2.18 0.0298934001134861
2.2 0.0311501931536253
2.22 0.0333345575823927
2.24 0.036673787016279
2.26 0.0349483720680419
2.28 0.0357716759910244
2.3 0.036517339863242
2.32 0.0373827443798287
2.34 0.0386172058064586
2.36 0.0385219399757245
2.38 0.0367127023220552
2.4 0.0368961730915365
2.42 0.036235654082611
2.44 0.0361032833498817
2.46 0.0364841615810911
2.48 0.0366461406986869
2.5 0.0373720983344581
2.52 0.0380285816104319
2.54 0.0372954731946709
2.56 0.0371194663559243
2.58 0.0378172486821203
2.6 0.0393581202020519
2.62 0.0401166553928839
2.64 0.0408682873971635
2.66 0.0416123859612132
2.68 0.0423483439941342
2.7 0.0430755770229981
2.72 0.043793523767951
2.74 0.0445016476509855
2.76 0.045199439032121
2.78 0.0458864179642905
2.8 0.0465621372715614
2.82 0.0460074107759521
2.84 0.0455784547331879
2.86 0.0452833481348199
2.88 0.0451278886405498
2.9 0.0451152520470267
2.92 0.0452457729346208
2.94 0.0455168766324974
2.96 0.0459231753267466
2.98 0.0464567212401313
3 0.0471073917849871
3.02 0.0461371170314466
3.04 0.0453051255699497
3.06 0.0446201837847338
3.08 0.044089977807988
3.1 0.0437207754190481
3.12 0.0435171118683787
3.14 0.0434815346806016
3.16 0.0436144390381187
3.18 0.0439140155382682
3.2 0.0443763176043029
3.22 0.0449358766706783
3.24 0.045653224894489
3.26 0.0465175233746898
3.28 0.0475164599564932
3.3 0.0486367000477441
3.32 0.0498643080115616
3.34 0.0511851161543552
3.36 0.0525850296671615
3.38 0.0540502651994176
3.4 0.0555675271795453
3.42 0.0577913755656223
3.44 0.0599560251475126
3.46 0.0620478284314143
3.48 0.0640546230362132
3.5 0.0659656315171749
3.52 0.0677713732968086
3.54 0.0694635786536222
3.56 0.0710350979215337
3.58 0.072479802327309
3.6 0.0737924760520998
3.62 0.0725450165781258
3.64 0.0712421401660829
3.66 0.0698855473478649
3.68 0.0684769909681538
3.7 0.067018249436778
3.72 0.0655110995469992
3.74 0.0639572894157488
3.76 0.0623585122936226
3.78 0.0607163821986555
3.8 0.0590324125058464
3.82 0.0592005735354175
3.84 0.0593600253699458
3.86 0.0595102888566919
3.88 0.0596509177329935
3.9 0.0597814999232538
3.92 0.0599016586601823
3.94 0.0600110534223763
3.96 0.0601093806822197
3.98 0.0601963744600371
4 0.0602718066823982
4.02 0.0614272015821522
4.04 0.0624563018028257
4.06 0.0633587192016107
4.08 0.0641336859154135
4.1 0.0647798235883288
4.12 0.0652950282510517
4.14 0.0656764913222575
4.16 0.0659208667350514
4.18 0.0660245819952987
4.2 0.0659842774619781
4.22 0.0660188919106434
4.24 0.0659576310351048
4.26 0.0658006284681407
4.28 0.0655482389737063
4.3 0.065201087567723
4.32 0.0647601131132148
4.34 0.0642266035174586
4.36 0.063602221240886
4.38 0.0628890191320706
4.4 0.0620894476382166
4.42 0.0619920767525661
4.44 0.061898132780534
4.46 0.0618076722074637
4.48 0.0617207485865328
4.5 0.0616374124213079
4.52 0.0615577110552967
4.54 0.0614816885692586
4.56 0.0614093856870001
4.58 0.0613408396903888
4.6 0.0612760843442467
4.62 0.0618303273339668
4.64 0.062483282349657
4.66 0.0632292446758899
4.68 0.0640621181667859
4.7 0.0649755118382689
4.72 0.0659628254341335
4.74 0.0670173234882783
4.76 0.068132198208261
4.78 0.0693006220899808
4.8 0.0705157915402414
4.82 0.070872930420881
4.84 0.0712289585603627
4.86 0.07158385795804
4.88 0.0719376128488263
4.9 0.0722902097825666
4.92 0.0726416377223228
4.94 0.0729918881565794
4.96 0.0733409552165374
4.98 0.0736888357853314
5 0.0740355295812621
};
%\addlegendentry{$B_2$}
\end{axis}

\end{tikzpicture}
 
    \caption{Results using SUR$_1$ and the basis~$B_1$~\eqref{plot:Sur1B1} or~$B_2$~\eqref{plot:Sur1B2}. From left to right, top to bottom, the resulting trajectories in the motion plane, the applied control values, the one-step prediction errors, and total error norms are shown.}
    \label{fig: firstprinciple}
\end{figure}
In the plotted scenario, the same random control sequence~$u$ is applied to the models. 
Once again, the prediction results of the surrogate models are compared to time integrations of the nominal model, which is used as a reference.  
In addition to the error norm, the one-step prediction error is considered. 
To calculate the latter, in each time step, starting from the same reference value, the following time step is predicted using the model of choice and the result is compared with the corresponding, subsequent value of the reference. 
As the results in Fig.~\ref{fig: firstprinciple} show, using random control values leads to a higher error than using the constant control input from Fig.~\ref{fig: modelbased circle}, motivating the subsequent analysis using test trajectories where a wider variety of inputs are applied. 
Moreover, here, the difference between the two bases is negligible. 
However, it is not a priori clear whether the latter also holds when using data from an imperfect hardware robot. 
Hence, real-world data is considered subsequently. 

\section{Experimental results}\label{sec:exp}
We use a custom-built mobile robot as depicted on the right of 
Fig.~\ref{fig:trainingData}. 
Its pose is tracked by an external tracking system consisting of five Optitrack Prime 13W cameras. 
The robot receives its inputs, the desired forward translational velocity and the desired angular yaw velocity, wirelessly. 
On-board software kinematically calculates the angular velocities of the wheels corresponding to the inputs under the assumption of rolling without slipping. 
Two independent PID controllers operating at a frequency of~$100\,\textnormal{Hz}$ control the motors so that the wheels quickly attain the desired angular velocities.  
Naturally, 
due to imperfections, the actual robot velocities may not match the sent ones. 
The time step is set to~$\delta = 0.1\,\textnormal{s}$ subsequently. 

\subsection{Data Generation}
Generating uniformly distributed training samples is possible by driving the robot to each corresponding point in the state space individually, applying one of the~$n_u$ inputs, and potentially driving back to that point to apply another input. 
However, this way of generating training data is notoriously time-inefficient. 
The more efficient procedure used in this paper works as follows. 
For the considered robot, holding any input for several time steps nominally results in a circular motion with the radius being determined by the quotient of the translational and angular velocities.  
The basis vectors of~$B_1$, which consist of driving in a straight line and turning on the spot, correspond to circles with infinite and vanishing radii, respectively. 
Therefore, slightly different sampling strategies for the two input bases~$B_1$ and~$B_2$ are used.
Starting from an initial position on the admissible motion plane~$\mathbb{P}=[0.0, 1.5]\,\textnormal{m}\times[-0.75, 0.75]\,\textnormal{m}$  with~$\mathbb{X}= \mathbb{P}\times \mathbb{R}$, a new point is drawn i.i.d.  
For~$B_1$, the robot turns using the corresponding input of the input basis until it faces this generated point. 
In order to collect as many data points as possible, the robot does at least one full rotation. 
Subsequently, the robot drives in a straight line toward this generated point. 
This way, the necessary input of~$B_1$ is held for several time steps, generating training samples along the way, making the procedure very time efficient. 
This procedure is repeated until a sufficient amount of training data is generated.
Due to the reasons stated above, for the input basis~$B_2$, the sampling strategy is adjusted slightly.
The robot, again, turns and drives towards the uniformly randomly generated point. 
Then, each time alternating between the two basis vectors of~$B_2$, the inputs are applied either until a full circle is driven or until the nominal state prediction of the robot leaves~$\mathbb{X}$. 
Generally, while time efficient and effective, this way of generating samples does not lead to a perfectly uniform distribution.
In Fig.~\ref{fig:trainingData}, some of the trajectories used during the data generation 
are depicted.\footnote{We do not immediately apply the full magnitude of the basis vectors for a more controlled behavior. 
Data points during acceleration and deceleration are not used since they do not correspond to any input basis vector. 
In case the random point is too close to the previous one to reach the desired speed, the point is discarded and a new one is generated so that enough data can be gathered in-between.}
\begin{figure}
    \centering
    %\footnotesize
    \begin{tikzpicture}

\newlength{\dataWidth}
\setlength{\dataWidth}{28mm}
\begin{axis}[%
height=\dataWidth,
width =\dataWidth,
at={(0cm,0cm)},
scale only axis,
%separate axis lines,
tick label style={/pgf/number format/fixed},
%axis equal image=true,
%ytick distance=1.0,
xtick distance=0.5,
xmin=-0.05,
xmax=1.55,
ymin=-0.8,
ymax=0.8,
xlabel = {$x_1$ position (m)},
ylabel = {$x_2$ position (m)},
ylabel shift = -2pt,
every axis y label/.style={at={(ticklabel cs:0.5)},rotate=90,anchor=near ticklabel},
]

\addplot[color = blue,solid,thin,line join=round,unbounded coords=jump] table[x = x, y = y] {data.txt};

\end{axis}

\begin{axis}[%
height=\dataWidth,
width =\dataWidth,
at={(\dataWidth+4.4mm,0cm)},
scale only axis,
separate axis lines,
tick label style={/pgf/number format/fixed},
axis equal image=true,
%ytick distance=1.0,
yticklabels = {,,},
xtick distance=0.5,
xmin=-0.05,
xmax=1.55,
ymin=-0.8,
ymax=0.8,
xlabel = {$x_1$ position (m)},
% ylabel = {$x_2$ position (m)},
every axis y label/.style={at={(ticklabel cs:0.5)},rotate=90,anchor=near ticklabel},
]

\addplot[color = blue,solid,line width = 0.5,line join=round,unbounded coords=jump,opacity=1.0] table[x = x, y = y] {data2.txt};
\coordinate (diana) at (axis cs:1.55,0); 
\end{axis}

\node[anchor=west,inner xsep = 0, inner ysep = 0, xshift=4.4mm] (image) at (diana){\includegraphics[height=\dataWidth]{diana_cropped.jpg}};
\end{tikzpicture}%
    \caption{From left to right, training trajectories used to generate the samples for the input bases~$B_1$ and~$B_2$, and a photograph of the employed type of custom-built robot are shown.}
    \label{fig:trainingData}
\end{figure}
Another practical consideration concerns the measurement of the robot's orientation. 
The optical tracking system steadily continues the angular measurements such that the orientation angle may lie outside of~$(-\pi, \pi]$. 
While it would be possible to use the raw data for training, instead, we leverage the periodicity of the orientation.%
\footnote{Each orientation in~$X_i$,~$i\in\{0,1,2\}$, is shifted to its equivalent value within~$(-\pi, \pi]$. 
The entries of~$Y_i$ are shifted by the same amount as the corresponding entries of~$X_i$. 
However, after that, some orientations in~$Y_i$ may still lie outside of~$(-\pi, \pi]$, namely if the angle left the interval between the sampling instants. 
The matrices with shifted entries are then used to compute the surrogate model. 
Before each evaluation of the model, 
the orientation is shifted to~$(-\pi, \pi]$. 
Subsequently, the output is then shifted back, resulting in the surrogate model being periodic (but not necessarily continuous) in the orientation.}

\subsection{Results}
Two main scenarios are considered.   
In the first scenario, the robot follows an~$\infty$-shaped trajectory. 
As the terminal and initial velocities are zero, at the start as well as the end of the trajectory, the speed is increased or decreased linearly to obtain a smoother motion. 
In the second scenario, the robot shall follow a square-shaped trajectory. To that end, the robot drives trapezoidal velocity profiles on each edge of the square with a top speed of about \unit[0.2]{m/s}. At each corner, the robot makes a quarter counter-clockwise turn with a maximum absolute angular velocity of \unit[1.0]{rad/s}, with the angular velocities being increased or decreased linearly. 

First, we look at Koopman-based models in which we do not incorporate further a-priori knowledge. 
The training data was generated as described above for the constant controls contained in the bases~$B_1$ or~$B_2$, for which~$4626$ or~$5182$ training data points were recorded, respectively.  
Because of the results from Section~\ref{sec:simulation}, only the Koopman-based surrogate model with projection in each step (SUR$_1$) is employed.   
%
Results for the~$\infty$-shaped trajectory can be seen in Fig.~\ref{fig: all observables}, where for the two bases~$B_1$ and~$B_2$ as well as for different observable sets, the resulting predicted trajectories are plotted on the left-hand side and the absolute errors are compared on the right-hand side. 
The errors are measured relative to a representative lap of the hardware robot. 
Due to imperfections, when supplied with inputs that should lead to a perfect~$\infty$-trajectory for the nominal kinematics, the real robot's trajectory is not of perfect shape. 
Three different surrogate models differing in their dictionaries are considered. 
Firstly, the set of observables~$\mathbb{O}_{120}$ from~Section~\ref{sec:simulation} is used. 
Secondly, in~$\mathbb{O}_{32}$, compared to~$\mathbb{O}_{120}$, we exclude monomials for which~$x_1$ and~$x_2$ have a degree larger than~$1$, yielding~$32$ observables in total. 
Finally, we further reduce the number of observables by omitting monomials where~$x_1$ or~$x_2$ appear multiplied with~$\theta$, leading to~$\mathbb{O}_{11}$ with~$11$ observables. 
This is motivated by the physical insight that the robot's dynamics is translation invariant,  
so it is interesting to see whether incorporating this knowledge improves model quality. 
In that regard, as can be seen in the upper part of Fig.~\ref{fig: all observables}, the predictions of the surrogate model using~$B_1$ with~$\mathbb{O}_{120}$ follow the reference rather well for some time but then completely deviate and even leave the experiment area. 
The paths for~$\mathbb{O}_{32}$ and~$\mathbb{O}_{11}$, however, are nearly indistinguishable and follow the reference well; only the error plot suggests that~$\mathbb{O}_{11}$ might perform a bit better. 

To study the influence of the input basis, the same scenario is plotted in the bottom part of Fig.~\ref{fig: all observables} for basis~$B_2$. 
\definecolor{dic120}{RGB}{31,119,180}%
\definecolor{dic32}{RGB}{255,127,14}%
\definecolor{dic11}{RGB}{44,160,44}%
\definecolor{reference}{RGB}{0,0,255}%
\def\linewidthEight{1.5}%
\def\linewidthError{1.0}%
\begin{figure}%[h!]
    \centering
    % This file was created with tikzplotlib v0.10.1.
\begin{tikzpicture}

\begin{axis}[
width = \textwidth, % scaled down due to axis equal image
height = 4.4cm,
at={(0.0, 0)},
axis equal image=true,
legend cell align={left},
legend style={
  fill opacity=1,
  draw opacity=1,
  text opacity=1,
  at={(0.5,0.91)},
  anchor=north,
  %draw=lightgray204
},
%tick align=outside,
%tick pos=left,
unbounded coords=jump,
%x grid style={darkgray176},
%xlabel = {$x_1$ position (m)}, 
ylabel = {$x_2$ position (m)},
xmin=0, 
xmax=1.5,
%xtick style={color=black},
%y grid style={darkgray176},
%ylabel={\(\displaystyle y\)},
ymin=-0.3, ymax=0.4,
xticklabels = {},
%ytick style={color=black}
]
\addplot [line width = \linewidthEight, color = reference]
table {%
0.75 0
0.753796775174918 0.00275082027626007
0.770253583658299 0.0146086690284934
0.788958864934086 0.0278626298724842
0.805758023793338 0.0400302499510602
0.823931007064239 0.0531532313053401
0.84106617676365 0.0651272462437049
0.858337230948657 0.0771219912744552
0.875920686143595 0.0890536810683165
0.892999619106597 0.100689025982836
0.910268216607307 0.111810868722874
0.927155369485107 0.122692123646889
0.944271947890646 0.133248126872312
0.961126069264421 0.143627188027739
0.977642979860227 0.15383448530992
0.994383897910681 0.163352502484412
1.01044548809055 0.172451162379304
1.02668443483368 0.181312738972183
1.04256818618544 0.189747796838266
1.05849129738013 0.197535145736718
1.07411177001412 0.204666427354084
1.08960910037061 0.21157055004819
1.10495636876448 0.217634415202293
1.11972672620237 0.223091919572481
1.13464406954527 0.228115743673971
1.14898414845007 0.232579183288128
1.16313524000647 0.236048403043336
1.17731712629478 0.239498375097147
1.19115277544504 0.241987916087121
1.20475078691893 0.24357341141794
1.2179051788757 0.245202814841758
1.23101137495675 0.245758006855585
1.24362380516706 0.245734565070592
1.25596427000739 0.245035586045094
1.26780296261423 0.243915690097645
1.27942969766319 0.242067323137536
1.29076669615838 0.23966799166844
1.30144802323049 0.236506712301561
1.31206798872445 0.232653669484729
1.32183779339603 0.228592384801218
1.33177828963827 0.223548734377477
1.34106333859447 0.218129184320505
1.35015693385091 0.212083593544392
1.35888263509532 0.205514426208893
1.36732721401058 0.198237703677997
1.37490374677743 0.190581538771692
1.38255513625004 0.18240019011285
1.38961325950765 0.173826808095369
1.39606593780378 0.16462031674418
1.40209413268775 0.155475013292566
1.40766144142924 0.145817165049195
1.41298972922002 0.135430711689483
1.4177980383817 0.125150704222236
1.42217530643287 0.113940480045831
1.42605722005813 0.103059482016292
1.42956849107093 0.0918251912690216
1.43257333591272 0.0803632587732614
1.43525005723557 0.0680997321441579
1.43747354279763 0.056305557615111
1.43903200164914 0.0443137666138458
1.44057442009313 0.0316776217651214
1.4413510801794 0.0194597121678604
1.44173985473391 0.00685847184277516
1.44175304710854 -0.00526920267813812
1.44104028644395 -0.0178002690956425
1.44009964222474 -0.0300979887959264
1.43873929662625 -0.0420597707266236
1.43683527860183 -0.0542540613878999
1.43454414915032 -0.0661993618164278
1.43178063868126 -0.07793275011297
1.42860939001972 -0.0902036039954828
1.42497929453919 -0.101367767761983
1.42084578597473 -0.112995734074035
1.41640300826551 -0.124306882350474
1.41152032108025 -0.134729554563296
1.40607079201313 -0.144853166840076
1.40030595459819 -0.154610482940984
1.39414354790019 -0.164024485370985
1.38748427065227 -0.173231382393035
1.38064408729499 -0.181896666884964
1.37331841508833 -0.190087400057888
1.36519484608907 -0.19812860388081
1.35697623393093 -0.205033044251742
1.34840153732322 -0.211514626079049
1.33928113004562 -0.21763730663906
1.33004900187448 -0.222735976435501
1.32028685217834 -0.227468875379833
1.31019782424788 -0.231938597992863
1.29972367023761 -0.235445423517443
1.2887328899364 -0.238377260466694
1.2774754396843 -0.240988537620381
1.26622519835259 -0.242443616017929
1.25423811689992 -0.243485353393633
1.24204165110139 -0.243984107342374
1.22950006048786 -0.24384429208382
1.21680621065999 -0.242965802234227
1.20378526795112 -0.241685128168492
1.19022085168209 -0.239581051822156
1.17642516281212 -0.236929456370965
1.16246464426886 -0.233589810942679
1.14828949361601 -0.229559269769245
1.134244114945 -0.225021139127758
1.1193916826543 -0.21962483904429
1.10483500153046 -0.21375219196959
1.08949301887148 -0.207426340723266
1.07427915545535 -0.200272296491305
1.05865073551656 -0.193145164392875
1.04282953663211 -0.185027373494884
1.02697988363446 -0.176541280579012
1.01101989520926 -0.167639400673986
0.994823684556384 -0.157985107306108
0.978611547483571 -0.148252113890068
0.962245616000544 -0.13813228039391
0.945289138613748 -0.127489288913879
0.92866688816798 -0.11678880795474
0.911699999758336 -0.105312068616864
0.894785005680457 -0.0934958016648549
0.877304752830891 -0.0810753361113209
0.859867338507263 -0.0690099709019099
0.842552953393209 -0.0565972164378705
0.825243828042276 -0.0441147310471777
0.807750702555605 -0.0312097420590962
0.790618156787901 -0.018184012181872
0.773448995829779 -0.00598021264680999
0.756125635736799 0.00675405645469632
0.738662857324652 0.0200698846503974
0.72119675701568 0.0334214396986119
0.703874584573642 0.0461723698235121
0.686417632616456 0.0593272292403158
0.669268676231481 0.0717358909373415
0.651945408052231 0.0846213768862322
0.634658611133318 0.0966457356254571
0.61766691011985 0.10929461111541
0.600962770468281 0.120736890028959
0.583916214443415 0.132609432334643
0.566703364898398 0.143355113283038
0.549526794264573 0.154919069656636
0.53298265702123 0.165684786365669
0.516280586189622 0.176441724076514
0.499469657201165 0.186297143282059
0.483143366453001 0.196123708388658
0.466547232727129 0.205095875426217
0.4503128526377 0.214073094476273
0.434499592218361 0.222468217516916
0.418779066207624 0.230542812331306
0.403243011226651 0.237473628045056
0.387793402240573 0.244479897043496
0.372395287160294 0.251032532682219
0.357362958539635 0.256485837349139
0.342959940828114 0.261406488246182
0.32815830763551 0.266208681304595
0.313283646731331 0.270303645502696
0.299008764566356 0.273703922777576
0.285596019227139 0.276229464479511
0.271667658663907 0.278876844869545
0.258391825439994 0.280148293958511
0.245954683337346 0.281045899586786
0.233297043080318 0.281166438800653
0.221065882947653 0.280435278965782
0.209213960006697 0.2792423328193
0.197711313341009 0.27745153093713
0.186467060395309 0.274933700120144
0.175647337802095 0.271997391117996
0.165485819498891 0.268414334553556
0.154920655621453 0.264120951359659
0.14557267250259 0.259464577309265
0.1359054564482 0.253809489621943
0.127182186643735 0.247898300503122
0.118715498185119 0.241484242043756
0.110599614202892 0.234627650065451
0.103066671467977 0.227209711390963
0.0953756606293399 0.218714717118634
0.0883116077532903 0.210301099842098
0.0816862019022073 0.201563439530642
0.0754303047598111 0.192040200227576
0.0696280035789295 0.182162510473038
0.0642727671113544 0.171854125577518
0.0592887993296431 0.161295496791183
0.0550401350221935 0.150292025379403
0.0509807490685192 0.139045437299692
0.0473874463212024 0.127624872627466
0.0442485567154118 0.115876687414116
0.04167948049022 0.104035118062325
0.0394988342888216 0.0921296980175037
0.0379006039360912 0.0799267762717973
0.0364917112037286 0.0678905527750915
0.0358931555007101 0.0555642260921366
0.0355289500473749 0.0429731447797864
0.0356926207873237 0.0304970528307748
0.0362636666860241 0.0179069982691894
0.0373684888681004 0.00563301356630583
0.0387615798631966 -0.00676904035650516
0.0406380369959789 -0.0189102629938579
0.0429542857015246 -0.0308705294137852
0.0457203908464014 -0.042746116689976
0.049035766503309 -0.0543490855274337
0.052763997030933 -0.0653774059979454
0.0569627461396097 -0.0767176672610195
0.0615888955937969 -0.0878239387828349
0.0664585976910542 -0.0982801562147958
0.072085609900655 -0.10863675691435
0.0780571592834234 -0.118599914077354
0.08432516434307 -0.128134045396256
0.0908131418279783 -0.136908599720991
0.0979353085759198 -0.145456841571941
0.105635261341533 -0.153920420196278
0.113314668270683 -0.161252708922069
0.121626519720217 -0.168399394022092
0.130199888894095 -0.174833639592176
0.139250621245577 -0.180825566954727
0.149194628828992 -0.18645514023376
0.158780448148931 -0.191196354812289
0.168971315456279 -0.195397950905863
0.179708378760775 -0.199150050909943
0.19045426687616 -0.202081472168663
0.201668373287374 -0.204582922691799
0.213406497742117 -0.206472869228286
0.225121874030311 -0.207235496554042
0.237331823289546 -0.20798471560131
0.249822155919747 -0.207816417771425
0.262786827072768 -0.206841133619445
0.275758723840688 -0.205357544128999
0.288951756143576 -0.203342196331986
0.302258118704561 -0.200601801321886
0.316710130363563 -0.197087998921957
0.330746388986938 -0.192957733094927
0.34554437580787 -0.188212280913441
0.360708133789849 -0.182801467457379
0.375228275071718 -0.176923508666162
0.390282767095232 -0.170709825662335
0.405108136218587 -0.163840010634489
0.42109976226801 -0.156280976465882
0.436573141590362 -0.147809019543375
0.452063988189108 -0.139410920745875
0.46813231140413 -0.130648654008784
0.484585950401663 -0.121010874179629
0.500662408967092 -0.110673419499024
0.517274358043854 -0.100190409204962
0.534167962758998 -0.0893077986447352
0.550610601565416 -0.0777901609279178
0.567378381435418 -0.0667074341001863
0.583918585955594 -0.054701018996261
0.601204930188776 -0.042521053347433
0.61779393188959 -0.0299697736916889
0.635312389786556 -0.0175333854783752
0.652425968524568 -0.00422174438776487
0.669577078744047 0.00880297813329933
0.68642543439641 0.0217815482795204
0.70362399878802 0.0348638435875829
0.720901407938547 0.0482750034504682
};
%\addlegendentry{Reference}
\addplot [line width = \linewidthEight, color = dic120, dashed]
table {%
0.75 0
0.76750050338627 0.0128167117228163
0.784989179581414 0.0256227373486651
0.802457549634473 0.0383817965481219
0.819897186483863 0.0510577897815576
0.837299619047005 0.0636149571701773
0.854656232796904 0.0760180270220949
0.871958170961449 0.0882323527436548
0.889196240973021 0.100224037612808
0.906360831054376 0.111960047580428
0.923441841811851 0.123408312851652
0.940428637396846 0.134537819448696
0.957310020189823 0.145318692248428
0.974074232078098 0.155722271114954
0.990708984281569 0.165721181715671
1.00720151639105 0.17528940243617
1.02353868389984 0.184402328522588
1.03970707211889 0.193036834214039
1.05569313306063 0.201171333221795
1.07148334074399 0.208785837506345
1.08706435948549 0.215862013937218
1.10242321915488 0.222383238127979
1.11754749111416 0.228334644548177
1.13242545862763 0.233703171945112
1.1470462759003 0.238477603172162
1.16140011052026 0.242648598719162
1.1754782648824 0.246208723566293
1.18927327308016 0.249152467418465
1.20277897069714 0.251476258891606
1.21599053585108 0.253178474768469
1.22890450069335 0.254259445950742
1.2415187333211 0.254721462112103
1.25383239070387 0.254568777185903
1.26584584374946 0.253807617574928
1.27756057601803 0.252446194246441
1.28897905780228 0.250494718647129
1.30010459727625 0.247965420751316
1.31094117013795 0.244872565835986
1.32149322863927 0.241232465222324
1.33176549021022 0.237063475769328
1.34176270524435 0.232385983765868
1.3514894032914 0.227222371127576
1.36094961715514 0.221596965084355
1.37014658534269 0.21553597604324
1.37908243487593 0.209067431043851
1.38775784837026 0.202221111347095
1.39617172112218 0.195028501781371
1.40432081536552 0.187522756579437
1.41219941965357 0.179738682020386
1.41979902149866 0.171712730847245
1.42710800112379 0.16348299770606
1.43411135371811 0.155089199121664
1.44079044721272 0.146572616007638
1.44712282252691 0.137975971577403
1.45308204362499 0.12934321307155
1.45863760566173 0.120719162520937
1.46375491105121 0.112149000832974
1.46839532556873 0.103677552339926
1.47251632975173 0.0953483455898849
1.47607178512059 0.0872024428755945
1.47901234027621 0.0792770578444572
1.48128600865904 0.0716040184843647
1.48283895688892 0.0642081804578758
1.48361654808463 0.0571059479916924
1.48356468447262 0.0503041062304785
1.48263148205886 0.0437991952216889
1.48076928028502 0.0375776441759443
1.47793693624598 0.0316168209751693
1.47410227704272 0.0258870323852017
1.46924449649653 0.0203543492940574
1.46335620714443 0.0149839628462522
1.45644482600375 0.00974364891147796
1.44853300965893 0.00460687390568225
1.43965796866828 -0.000444864211881654
1.42986966322252 -0.0054207134793338
1.41922806758045 -0.0103202183849356
1.40779983901448 -0.0151340089837035
1.39565480048061 -0.0198451637041433
1.38286263476111 -0.0244309291679763
1.36949010790123 -0.0288645011032172
1.35559902298329 -0.033116639229863
1.34124498382957 -0.03715697704595
1.32647694470278 -0.0409549717091796
1.31133744653782 -0.0444805064321691
1.29586339284506 -0.0477042034724032
1.28478359089857 -0.0485259432782703
1.27367425245104 -0.048438157983855
1.26254913774251 -0.047470835555357
1.25142172834608 -0.045650975431156
1.24030525040546 -0.0430026429197807
1.22921261326679 -0.0395471840457737
1.21815628878445 -0.03530355625708
1.20714815917687 -0.0302887297050327
1.19619935875796 -0.0245181183592881
1.18532012900534 -0.018006008163146
1.1745196993476 -0.0107659587254583
1.16380619946391 -0.00281116397872001
1.15318660379252 0.00584523528946647
1.14266670565012 0.0151898885749367
1.13225111669927 0.0252090203736358
1.12194328705574 0.0358882583586593
1.11174554164516 0.0472125000631956
1.10165912911096 0.0591658130527992
1.09168428036438 0.071731363937898
1.08182027458078 0.0848913721200917
1.07206551100005 0.0986270847153924
1.06241758525798 0.112918769556596
1.05287336917373 0.127745723501083
1.04342909298255 0.143086293452762
1.03408042897908 0.158917907566447
1.02482257547504 0.175217114075147
1.01565033992516 0.191959625110838
1.00655822007815 0.209120362826883
0.997540482099382 0.226673505124584
0.988591234809438 0.244592528381764
0.979704499493372 0.262850244813201
0.970874275150715 0.281418832484414
0.962094599548807 0.300269856559531
0.95335960697142 0.319374281081791
0.944663584066815 0.338702471434937
0.936001025631971 0.358224188571699
0.927366692457172 0.377908577062975
0.918755673434872 0.397724149947713
0.910163453956692 0.417638774170702
0.901585992148005 0.437619661003721
0.893019803711774 0.457633366179156
0.884462055094246 0.477645804460561
0.875910663402748 0.497622282986787
0.867364400095085 0.517527556936606
0.858822994049138 0.537325909882553
0.850287228363546 0.556981259685824
0.841759024298977 0.576457289015418
0.833241505298794 0.595717597676729
0.824739034149434 0.614725872057878
0.816257217121529 0.633446065315021
0.8078028703663 0.651842580593433
0.799383945838712 0.669880448779702
0.79100941641051 0.68752549213368
0.782689122391162 0.704744465743177
0.774433584129231 0.721505170107859
0.76625378746391 0.737776530251625
0.758160950326849 0.753528639473226
0.750166279630555 0.76873276899171
0.742280727697156 0.783361348084631
0.734514756960735 0.797387922566674
0.726878120685274 0.810787102307206
0.719379666196955 0.823534510637768
0.712027165857013 0.835606749694966
0.70482717987828 0.846981395793764
0.6977849542093 0.857637037744657
0.690904356062813 0.867553368650223
0.684187849134735 0.876711338301381
0.677636509957315 0.885093369111555
0.671250085951095 0.892683633920859
0.665027094437486 0.899468389331401
0.658964960133172 0.905436353802297
0.653060186644473 0.910579115722
0.647308555590431 0.914891553115302
0.64170534572924 0.918372243439829
0.636245564380367 0.921023838985329
0.630924184887185 0.922853380805998
0.625736386846541 0.923872522391722
0.620677799846979 0.924097634415728
0.615744755552843 0.923549765202984
0.610934555948887 0.922254439176277
0.597107823212512 0.92526911437852
0.584197654588893 0.927808980986119
0.572267721567644 0.92982260797634
0.56137542607168 0.931261336657516
0.551569187553568 0.932079544093923
0.542886452065009 0.932235725004983
0.535352451288779 0.931694189491223
0.528979706397883 0.930427094570276
0.523768229482549 0.928416499765019
0.519706329941721 0.925656154851734
0.516771893024043 0.92215278028763
0.514933970900568 0.91792667631498
0.514154518942477 0.913011584568808
0.514390122713903 0.907453815637807
0.515593591432744 0.901310736707923
0.517715334568556 0.89464877522299
0.520704481525824 0.887541130267749
0.524509742425295 0.880065390560779
0.529080035790915 0.872301239217406
0.534364924678741 0.8643283877787
0.540314907545488 0.856224834576104
0.546881606862623 0.848065494308989
0.5540178906195 0.839921203727857
0.561677952497133 0.831858076605415
0.569817367776939 0.823937160959459
0.578393135109351 0.816214341819442
0.587363709419202 0.808740431505225
0.596689028258617 0.801561393794845
0.606330532378856 0.794718656050182
0.616251180680006 0.788249472382808
0.626415459579647 0.782187309895424
0.636789386900017 0.776562238080046
0.647340510403504 0.771401308165131
0.658037901001067 0.766728914442135
0.668852140383163 0.762567133404438
0.679755302399596 0.758936039022794
0.690720927009136 0.75585399382056
0.701723985138821 0.753337915768681
0.712740832482373 0.751403520578942
0.72374915030692 0.750065537958892
0.734727871922242 0.74933789910494
0.745657094768863 0.749233891563008
0.756517980189321 0.749766277098917
0.767292645794958 0.75094736895755
0.777964058653346 0.752789067312263
0.788515940815354 0.755302855964562
0.798932701375062 0.758499769098148
0.809199410777554 0.762390343219866
0.819301833217735 0.766984575072867
0.829226531924227 0.772291910090834
0.838961060583562 0.778321287246547
0.848494253125301 0.785081265193966
0.857816624593464 0.792580252543435
0.866920898723061 0.800826863611712
0.875802683722105 0.809830421800656
0.884461327124831 0.819601637403594
0.892900994141231 0.830153496416918
0.901132033050459 0.841502413260446
0.909172718462766 0.853669725334486
0.917051503320599 0.866683644861929
0.92480997126688 0.88058184029153
0.93250677610254 0.895414907785971
0.940223008253015 0.911251134607689
0.948069682127384 0.928183189903105
0.956198472364795 0.94633777837189
0.964817595189168 0.965890002939081
0.974216146556243 0.987085499190645
0.984802943787334 1.01027596485397
0.997171509429477 1.03597897892389
1.01221507297572 1.06498462109804
1.03134455444605 1.09855920876779
1.05693914536356 1.13887025453673
1.09338960971854 1.18998018153043
1.14992125096055 1.26056342176956
1.25019046338744 1.3732450079962
1.47951033262543 1.61111399165254
2.46297324377544 2.59800689531089
};
%\addlegendentry{$SUR_1$, N = 120}
\addplot [line width = \linewidthEight, color = dic32, dashed, dash pattern=on 3pt off 3pt, dash phase=1.5pt]
table {%
0.75 0
0.767572493274627 0.0125887644482
0.78513832762467 0.0251549574393686
0.802686511241967 0.0376665526408805
0.820206053607766 0.0500916621204411
0.837685974093816 0.0623986205503788
0.855115310393093 0.0745560694480587
0.872483126769547 0.0865330411645192
0.88977852214061 0.0982990423184026
0.906990638029718 0.109824136358237
0.924108666447945 0.121079024923045
0.941121857783146 0.132035127658959
0.958019528790851 0.142664660137892
0.974791070792545 0.152940709513065
0.991425958192798 0.162837307535295
1.00791375742569 0.172329500543107
1.02424413643158 0.181393416029137
1.04040687474588 0.190006325375103
1.05639187424957 0.198146702338644
1.07218917058507 0.205794276869002
1.0877889451771 0.212930083827342
1.10318153771408 0.219536506195457
1.11835745883887 0.225597312379785
1.1333074026662 0.231097687264555
1.14802225858947 0.236024256750017
1.16249312166545 0.24036510564365
1.17671130068393 0.244109788970385
1.19066832286071 0.247249337048837
1.20435593396938 0.24977625505586
1.21776609269793 0.251684518271193
1.23089095814087 0.252969564734591
1.24372286968259 0.253628287602135
1.25625431915137 0.253659029953944
1.26847791605008 0.25306158503318
1.28038634785574 0.25183720470654
1.29197233868791 0.249988618161579
1.30322861082566 0.247520061408913
1.314147854275 0.244437316111416
1.32472270951578 0.240747753910625
1.33494576747141 0.236460380263536
1.34480958866799 0.231585870439394
1.35430674081532 0.226136590254901
1.36342985124133 0.220126595535615
1.37217166842304 0.213571606939579
1.38052512581324 0.206488960045239
1.38848340145717 0.198897533701568
1.39603996833653 0.190817661884122
1.4031886324902 0.182271035333686
1.40992355817631 0.173280599087073
1.41623928118782 0.163870450944389
1.4221307126517 0.154065744383238
1.42759313618431 0.14389259783172
1.43262220125228 0.133378010834257
1.43721391518451 0.122549786629724
1.44136463568558 0.11143646003347
1.44507106506396 0.100067229221726
1.44833024680909 0.0884718899732177
1.45113956467939 0.0766807710406904
1.45349674411377 0.0647246695298818
1.4553998555433 0.05263478540056
1.45684731904008 0.0404426544376761
1.45783790967363 0.0281800792499434
1.45837076293126 0.0158790580285693
1.45844537957993 0.00357171093763664
1.45806162938941 -0.00870979588924659
1.45721975319074 -0.0209333287014179
1.45592036280334 -0.0330668667486405
1.45416443842499 -0.0450785850687939
1.45195332314154 -0.0569369393689278
1.44928871427692 -0.0686107529712599
1.44617265137302 -0.080069305878734
1.44260750066637 -0.0912824261387949
1.43859593601722 -0.10222058385741
1.43414091634929 -0.112854988446862
1.42924565976999 -0.123157689986841
1.42391361465064 -0.133101685937117
1.41814842802692 -0.142661034842688
1.41195391169034 -0.151810979068801
1.40533400622248 -0.160528078896044
1.39829274291232 -0.168790360332479
1.39083420295074 -0.176577478526544
1.38296247253986 -0.183870897402639
1.37468159174059 -0.190654083808237
1.36599549434006 -0.196912710896821
1.35690793625484 -0.202634860802513
1.34742241156679 -0.20781121146895
1.33754205860621 -0.212435187877385
1.32726956343843 -0.216503055363684
1.31660707380178 -0.220013933697966
1.305556141379 -0.222969716013129
1.29411771234138 -0.225374886283069
1.28229218389899 -0.227236241343033
1.27007953777771 -0.228562535924479
1.25747955123176 -0.229364079151875
1.24495555104456 -0.22924463359768
1.23216139332131 -0.22844376041655
1.21909764551662 -0.226967558877643
1.20576725495828 -0.224823061738261
1.19217504166548 -0.222018358953006
1.1783272717797 -0.21856269328926
1.16423130876091 -0.214466530196333
1.14989533465044 -0.20974160520501
1.13532813155964 -0.204400952291503
1.12053891317676 -0.198458916383121
1.10553719674929 -0.191931152748263
1.09033270716555 -0.184834615539724
1.07493530608736 -0.177187537319639
1.05935494037709 -0.169009401017036
1.04360160521839 -0.160320905461511
1.02768531831275 -0.151143925394139
1.01161610234073 -0.141501466669657
0.995403973523735 -0.131417617221604
0.979058934631622 -0.120917494254909
0.962590971177719 -0.110027188050001
0.946010049848555 -0.0987737027025252
0.929326118449383 -0.0871848940780721
0.9125491068251 -0.0752894052281303
0.895688928351679 -0.0631165994889703
0.878755481695945 -0.0506964914672121
0.861758652619263 -0.0380596761029027
0.844708315659313 -0.0252372559918653
0.827614335568289 -0.0122607671430652
0.81048656841891 0.000837896656912291
0.793334862314301 0.014026560701911
0.776169057656084 0.0272728472766489
0.758998986938354 0.0405442544484555
0.741834474044958 0.053808235567292
0.724685333034356 0.0670322792984267
0.707561366401188 0.0801839900090121
0.69047236280694 0.0932311683263611
0.673428094274232 0.106141891682107
0.656438312840519 0.118884594652812
0.639512746667626 0.131428148904065
0.622661095603571 0.143741942541855
0.605893026192636 0.155795958671968
0.589218166128496 0.16756085296557
0.572646098143279 0.179008030026801
0.556186353322429 0.190109718356224
0.539848403830807 0.200839043702138
0.523641655029178 0.21117010058994
0.507575436951408 0.221078021817599
0.491658995100685 0.230539045702471
0.475901480506919 0.239530580860734
0.46031193896596 0.248031268294852
0.444899299352963 0.256021040556003
0.429672360865177 0.263481177736188
0.414639779000977 0.270394360027507
0.399810050018771 0.276744716562118
0.385191493536583 0.282517870213281
0.37079223282435 0.287700977992456
0.356620172197256 0.292282766614723
0.342682970727835 0.29625356271765
0.328988011241223 0.299605317096343
0.315542363221655 0.302331622143746
0.302352737815401 0.304427721438179
0.289425432541942 0.305890510070694
0.276766262606471 0.30671852382075
0.264380474854586 0.306911914643055
0.253270944944717 0.307028230530859
0.242267202473253 0.306426799898214
0.231385269549823 0.305112205928708
0.220644704702479 0.303092561454624
0.210067897499067 0.300378988115792
0.199679301285357 0.296985116979615
0.189504653624207 0.292926648000574
0.179570228663601 0.288220992078075
0.169902157392643 0.28288700402833
0.16052584206791 0.276944800703885
0.151465481070859 0.270415648345515
0.142743710770073 0.263321898295439
0.134381362153975 0.255686950195997
0.126397322637929 0.247535225357598
0.118808488075782 0.238892138251825
0.111629786952039 0.229784059411419
0.104874257973748 0.220238267325302
0.0985531634391161 0.210282889784343
0.0926761232226401 0.199946836637915
0.0872512573247834 0.189259726398459
0.0822853281270071 0.178251808975979
0.0777838763831209 0.1669538863702
0.0737513473595063 0.155397232622707
0.0701912053499916 0.143613513861776
0.067106036072465 0.131634708910592
0.064497637290087 0.119493030681569
0.062367098492446 0.107220848429112
0.0607148707184469 0.094850610856629
0.0595408276851702 0.0824147700482709
0.0588443193691922 0.0699457062033241
0.0586242191153968 0.0574756531787096
0.058878965255012 0.0450366248845729
0.0596065981199074 0.0326603426255691
0.0608047932558007 0.0203781635348576
0.0624708915681818 0.0082210103099617
0.0646019270816191 -0.00378069746861192
0.067194652951566 -0.0155971100713854
0.0702455663297292 -0.0271990119443719
0.0737509326373135 -0.0385578830502028
0.0777068097285856 -0.0496459565294353
0.0821090723097481 -0.0604362718602431
0.0869534367917481 -0.0709027225834368
0.0922354864772968 -0.0810200975908687
0.0979506965950231 -0.090764114991342
0.104094458196068 -0.100111447727173
0.110662099348001 -0.109039740485676
0.11764890146856 -0.117527618101011
0.125050108162129 -0.125554686614805
0.132860923731204 -0.133101529436603
0.14107649883426 -0.140149702487879
0.149691901716925 -0.146681733557264
0.158702075102423 -0.152681131942968
0.168101781032341 -0.15813241436739
0.177885538300695 -0.163021151785657
0.188047559042115 -0.167334039039407
0.198581691924888 -0.171058985718204
0.209481378883945 -0.174185222859119
0.220739630425719 -0.17670341717331
0.232349021688469 -0.178605783089641
0.244301708361212 -0.179886183339018
0.256589458973474 -0.180540210813757
0.26920369845708 -0.180565247369011
0.282135557395534 -0.179960498311686
0.295375921852438 -0.17872700391124
0.308915479766101 -0.176867631016365
0.322744761240888 -0.174387048713278
0.336854171358788 -0.171291692062343
0.351234015196173 -0.167589717540937
0.365874515490557 -0.16329095314439
0.380765823869448 -0.158406845346193
0.395898026779809 -0.152950404418779
0.411261147308531 -0.146936149032243
0.426845144023471 -0.140380050600731
0.442639907841082 -0.13329947752821
0.458635257775867 -0.125713139296684
0.474820936271797 -0.117641030216369
0.491186604669668 -0.109104372595635
0.507721839233425 -0.100125559069377
0.524416128045018 -0.0907280938327255
0.541258868981005 -0.0809365325517793
0.558239368903458 -0.0707764207566319
0.575346844130899 -0.0602742305596102
0.592570422200121 -0.0494572955801059
0.609899144885169 -0.0383537439950698
0.627321972404214 -0.0269924296701977
0.644827788717354 -0.015402861360786
0.662405407797808 -0.00361513000312263
0.680043580744831 0.00834016585274494
0.697731003598476 0.0204319963897626
0.715456325713582 0.0326289785442768
0.733208158552664 0.0448994555688323
};
%\addlegendentry{$SUR_1$, N = 32}
\addplot [line width = \linewidthEight, color = dic11, dashed]
table {%
0.75 0
0.767552419237039 0.0125770360729294
0.785094588443711 0.0251301206984076
0.802615768584575 0.0376274406922686
0.820105220050578 0.0500373518060185
0.837552207952489 0.0623284592340111
0.854946007926699 0.0744696976296679
0.872275912484984 0.0864304104036071
0.889531237942396 0.0981804280704805
0.906701331959218 0.109690145405009
0.923775581733498 0.120930597160884
0.940743422879924 0.131873532098567
0.957594349028129 0.142491485059294
0.974317922168498 0.152757846812421
0.990903783765256 0.16264693139141
1.00734166664403 0.172134040620038
1.02362140764278 0.181195525514795
1.03973296098908 0.189808844232233
1.05566641233085 0.197952616212069
1.07141199329894 0.205606672149664
1.08696009641507 0.212752099417957
1.1023012900738 0.219371282553404
1.11742633321925 0.225447938429605
1.13232618920292 0.230967145775677
1.14699203814883 0.235915368766715
1.16141528696912 0.240280474537011
1.17558757597916 0.244051744661703
1.18950078087648 0.247219880937732
1.20314700870874 0.249777006184309
1.2165185864142 0.251716661277788
1.22960804064234 0.253033800213251
1.24240806793155 0.253724785585804
1.25491149500828 0.253787387401362
1.2671112300206 0.253220788402203
1.27900020690238 0.252025598936502
1.29057132665315 0.250203883632725
1.30181740085646 0.247759200659529
1.31273110388058 0.244696652222948
1.32330494050458 0.241022942461017
1.33353123487205 0.236746436541585
1.34340214460698 0.231877213159278
1.35290970087742 0.226427102285592
1.36204587173516 0.220409701183376
1.37080264294878 0.213840364170188
1.37917210847884 0.206736164841391
1.38714656212653 0.199115832690519
1.39471858271581 0.190999668605264
1.4018811070753 0.182409445158957
1.40862748749946 0.173368297893824
1.41495153272728 0.163900613124733
1.42084753337482 0.154031916551479
1.4263102740081 0.14378876553215
1.43133503464672 0.133198646531321
1.43591758457122 0.122289878183967
1.44005417103923 0.111091519667427
1.44374150506203 0.099633283630718
1.44697674588128 0.0879454527335627
1.4497574852991 0.0760587988255379
1.45208173259904 0.0640045038834385
1.45394790046634 0.0518140819706422
1.45535479207246 0.0395193016499098
1.45630158932106 0.0271521084480636
1.45678784214552 0.0147445471252748
1.45681345869059 0.00232868363846325
1.45637869618927 -0.0100634731925699
1.45548415235574 -0.02240005020373
1.45413075714891 -0.0346493883848711
1.4523197648202 -0.0467801210675815
1.45005274624401 -0.058761251431331
1.44733158164549 -0.0705622288774479
1.44415845399439 -0.082153023742433
1.44053584353353 -0.093504199748063
1.43646652416238 -0.104586983522016
1.43195356270159 -0.115373330480779
1.42700032241306 -0.125835986363586
1.42161047251249 -0.135948543764994
1.41578800572589 -0.145685493160955
1.40953726610438 -0.155022268184131
1.40286298916679 -0.163935285293271
1.39577035578837 -0.172401978489893
1.38826505988185 -0.180400830315744
1.38035338766742 -0.187911400920366
1.3720423032097 -0.194914357377224
1.36333953122009 -0.201391505487906
1.35425362457964 -0.207325825919863
1.34479400168999 -0.212701515652305
1.33497093879517 -0.217504034496944
1.32479550571278 -0.221720155219237
1.3142794400419 -0.225338014899756
1.30343496384498 -0.22834716496801
1.29227455604006 -0.230738617918771
1.28081070096378 -0.232504889899587
1.26905563696978 -0.233640039723182
1.25702112788256 -0.234139705934185
1.24441010699887 -0.233998287942886
1.23150987243108 -0.233198341747588
1.21832656181558 -0.231742514140598
1.2048674231025 -0.229634992451
1.19114053386897 -0.226881537706303
1.17715457861385 -0.223489493646155
1.16291867678558 -0.21946777824048
1.14844225316242 -0.214826862904619
1.13373494230279 -0.209578743212939
1.11880651958335 -0.203736903736233
1.10366685245884 -0.1973162787133
1.08832586675613 -0.19033320960065
1.07279352391305 -0.182805400086781
1.05707980602253 -0.174751868861652
1.04119470632414 -0.166192900253894
1.02514822340626 -0.157149992751486
1.00895035786231 -0.147645805378356
0.992611110507802 -0.137704101889404
0.976140481535093 -0.127349692755964
0.959548470180833 -0.116608374933587
0.942845074623443 -0.105506869428301
0.926040291929591 -0.0940727567028772
0.90914411793937 -0.0823344099889363
0.892166547028268 -0.0703209265931927
0.875117571716139 -0.0580620573061456
0.858007182113613 -0.0455881340392369
0.840845365208251 -0.032929995831892
0.823642103998414 -0.0201189133832688
0.806407376484299 -0.00718651227514739
0.789151154523904 0.00583530493751029
0.771883402557889 0.0189144375830046
0.754614076202023 0.0320186664833124
0.737353120699476 0.0451157341186331
0.720110469218176 0.0581734251401627
0.702896040970757 0.0711596470197001
0.68571973912661 0.0840425106196356
0.668591448477135 0.0967904104620616
0.651521032806485 0.10937210447104
0.634518331910881 0.121756792957187
0.617593158199645 0.133914196608515
0.600755292800429 0.145814633245471
0.584014481079086 0.157429093091065
0.567380427471107 0.168729312298249
0.550862789505642 0.179687844465752
0.534471170884271 0.190278129859561
0.518215113453807 0.200474562039187
0.502104087884289 0.210252551564603
0.486147482828281 0.21958858642977
0.470354592293754 0.228460288830384
0.45473460090737 0.236846467824773
0.43929656667491 0.244727167385716
0.424049400756495 0.252083709264932
0.409001843661085 0.258898729999116
0.394162437120984 0.265156211274996
0.379539490724845 0.270841502740711
0.365141042157602 0.275941336203774
0.350974809607781 0.280443829997795
0.337048134546317 0.284338482142606
0.323367912649368 0.287616150785525
0.309940510130953 0.290269020326745
0.296771662187317 0.292290551644306
0.283866349683107 0.293675415003758
0.271228649732987 0.294419404635913
0.25886155563669 0.294519334667588
0.247031600492077 0.293953490442361
0.235471432420903 0.292757306992053
0.224189900114305 0.290936052858028
0.213196070455386 0.288495719755516
0.202499267176604 0.285443106402306
0.1921090882629 0.281785935934394
0.182035394326665 0.277532997291895
0.172288262935697 0.272694296866955
0.162877908577617 0.267281204571278
0.153814573523183 0.261306578956782
0.14510839976745 0.254784859079474
0.136769295091618 0.247732115696052
0.128806806419396 0.240166059934778
0.121230011273643 0.232106012563315
0.114047434206134 0.223572840509863
0.107266990752081 0.214588869039244
0.100895957735178 0.20517777809811
0.0949409661722696 0.195364490287881
0.0894080117120683 0.185175056253462
0.0843024772934957 0.174636541466534
0.0796291631904882 0.163776916755713
0.0753923204816409 0.152624953652937
0.0715956849757341 0.141210124723463
0.0682425095625004 0.129562508487183
0.0653355937505048 0.117712698248882
0.0628773097705277 0.105691714056341
0.0608696250706444 0.0935309170290948
0.0593141213341872 0.0812619253943938
0.0582120103458394 0.0689165316936081
0.0575641471448301 0.056526620758616
0.0573710409634769 0.0441240881907252
0.0576328644742503 0.0317407591986482
0.0583494618735731 0.0194083077662363
0.0595203563251178 0.00715817622714904
0.061144757274453 -0.00497850457356751
0.0632215681316174 -0.0169709942471797
0.0657493947959461 -0.0287890205647135
0.0687265554618783 -0.0404028553829006
0.0721510920852508 -0.0517833881475033
0.0760207837924747 -0.0629021961860533
0.08033316236227 -0.0737316108552776
0.0850855296816575 -0.0842447784735137
0.09027497675678 -0.0944157148733956
0.0958984034350151 -0.104219352398547
0.101952537476044 -0.113631578300406
0.108433951035244 -0.122629263840775
0.115339072076679 -0.131190284045638
0.122664187851213 -0.139293529033685
0.130405437539841 -0.146918909141301
0.138558791668677 -0.154047357555679
0.147120017095265 -0.160660835576403
0.156084628252752 -0.166742346546426
0.165447827707044 -0.17227596446974
0.175204441476903 -0.177246882020886
0.18534885636857 -0.181641480012137
0.195874967190733 -0.185447416811499
0.206776140811049 -0.188653732486008
0.218045201695605 -0.191250959524419
0.229674440377944 -0.19323123062714
0.241655643024906 -0.19458837452111
0.253980137656145 -0.195317992792939
0.266638851103333 -0.195417513666235
0.279622370552355 -0.194886221695946
0.292921004263075 -0.193725264882594
0.306524837404454 -0.19193764238436
0.32042378046813 -0.189528176780892
0.334607609120382 -0.186503474879086
0.349065995446525 -0.182871880595976
0.363788531280743 -0.178643422752389
0.378764744727796 -0.173829759853745
0.393984111141471 -0.168444123242825
0.409436059808132 -0.162501259443716
0.425109977463255 -0.156017372092492
0.440995209599027 -0.149010063558309
0.457081060338916 -0.141498276175645
0.47335679148297 -0.133502232908498
0.489811621176979 -0.12504337722644
0.506434722533601 -0.116144311970731
0.523215222434045 -0.106828737010936
0.540142200662188 -0.0971213855280892
0.557204689465915 -0.0870479588016831
0.574391673599 -0.0766350594202853
0.591692090868042 -0.0659101228762616
0.609094833189654 -0.0549013475426582
0.626588748151069 -0.0436376230639512
0.644162641060678 -0.0321484572220703
0.661805277472208 -0.0204639013649749
0.679505386166303 -0.00861447450746096
0.697251662575212 0.00336891376681576
0.715032772639585 0.0154550414564559
0.732837357090498 0.0276124531865891
};
%\addlegendentry{$SUR_1$, N = 11}
\end{axis}



\begin{axis}[
width = .3325\textwidth,
height = 4.4cm,
at={(0.45\textwidth, 0)},
legend cell align={left},
legend style={
  fill opacity=1,
  draw opacity=1,
  text opacity=1,
  at={(0.5,0.09)},
  anchor=south,
  %draw=lightgray204
},
%log basis y={10},
%tick align=outside,
%tick pos=left,
%x grid style={darkgray176},
%xlabel = {time (s)},
%xmin=-1.24, 
xmin=0,
xmax=26.0400000000001,
xticklabels={},
%xtick style={color=black},
%y grid style={darkgray176},
ymin=2.97952544004433e-05, ymax=0.0753931451700119,
ymode=log,
%ytick style={color=black}
]
\addplot [line width = \linewidthError, color = dic32]
table {%
0 0
0.1 0.0169279518753107
0.2 0.0182422532103294
0.3 0.0168690596893528
0.4 0.0176061801751764
0.5 0.0165733623708958
0.6 0.0169198363690937
0.7 0.0169904157868122
0.8 0.0166588211817272
0.9 0.0167092445094887
1 0.0166570336837613
1.1 0.01680340796829
1.2 0.0166633453828895
1.3 0.0165370477782845
1.4 0.0164627245539082
1.5 0.0162371054268187
1.6 0.0164428281137268
1.7 0.016244500659883
1.8 0.0161751650779363
1.9 0.0159951548100961
2 0.0159797728002235
2.1 0.015737455727741
2.2 0.0155883593312821
2.3 0.0157647419755171
2.4 0.015540930511922
2.5 0.0155920795361037
2.6 0.0157890901125681
2.7 0.0154380005479592
2.8 0.0153291101800091
2.9 0.0153358481638799
3 0.0151311885528414
3.1 0.0149506995553593
3.2 0.0149106347012952
3.3 0.0148663377143971
3.4 0.0148691619748808
3.5 0.0148345798300028
3.6 0.014729369178065
3.7 0.0149726479070282
3.8 0.0150218561103462
3.9 0.0152880455571
4 0.015310464084396
4.1 0.0154759895380298
4.2 0.0155196719386608
4.3 0.0155408033290464
4.4 0.0155649641049551
4.5 0.0159236551361443
4.6 0.0158963683720912
4.7 0.0159873614229311
4.8 0.0163411790470559
4.9 0.0164489695560149
5 0.0166552955764525
5.1 0.0168778852955183
5.2 0.0169541847017649
5.3 0.0173285865851829
5.4 0.0174496628589653
5.5 0.017557362843644
5.6 0.0177208956018899
5.7 0.0180585346955641
5.8 0.0181003984997828
5.9 0.0183615357404953
5.99999999999999 0.0184833178268538
6.09999999999999 0.0186510146860178
6.19999999999999 0.0189197801943938
6.29999999999999 0.0188890369499687
6.39999999999999 0.0192967048685254
6.49999999999999 0.019418784568976
6.59999999999999 0.0193923014863844
6.69999999999999 0.0196083947802676
6.79999999999999 0.0197198328758786
6.89999999999999 0.0198351289830259
6.99999999999999 0.0202773801977383
7.09999999999999 0.0203093024724829
7.19999999999999 0.0207646740576781
7.29999999999999 0.0211134851970987
7.39999999999999 0.0211682706294564
7.49999999999999 0.0213650093163149
7.59999999999999 0.0214742442919087
7.69999999999999 0.021595805021601
7.79999999999999 0.0219086051510144
7.89999999999999 0.0219829551368043
7.99999999999999 0.0221205968324603
8.09999999999999 0.0227809293864917
8.19999999999999 0.02280864306801
8.29999999999999 0.0228640164997209
8.39999999999999 0.0231468718883599
8.49999999999999 0.0229037982361791
8.59999999999999 0.0228856703810421
8.69999999999999 0.0230152179057101
8.79999999999998 0.0228731325786231
8.89999999999998 0.0228125889628862
8.99999999999998 0.0228199769341636
9.09999999999998 0.0221226641490298
9.19999999999998 0.0217632970076947
9.29999999999998 0.0212620315286777
9.39999999999998 0.021260814101539
9.49999999999998 0.0211346003775062
9.59999999999998 0.0212385440305872
9.69999999999998 0.0214356928076098
9.79999999999998 0.021688695444937
9.89999999999998 0.0218503367598187
9.99999999999998 0.0219529555625241
10.1 0.0218729247053169
10.2 0.0220394449016773
10.3 0.0219202445049867
10.4 0.0223050777661838
10.5 0.0222719221369616
10.6 0.0227998486943237
10.7 0.0230144391901178
10.8 0.0232246032768322
10.9 0.0234486040506928
11 0.0235307395479938
11.1 0.0237778436868209
11.2 0.0240631782024256
11.3 0.0245820742951427
11.4 0.0250065846193941
11.5 0.0252838785962656
11.6 0.0254369054856967
11.7 0.0257000803627169
11.8 0.0263086579308269
11.9 0.0266926822262618
12 0.0271150390495332
12.1 0.0274522779898308
12.2 0.027506122893098
12.3 0.0282084857031948
12.4 0.0286837852981351
12.5 0.0288575464357753
12.6 0.0290092538678919
12.7 0.0294656253080898
12.8 0.0296995277424112
12.9 0.0301934309812781
13 0.0304078668175268
13.1 0.0311275160797201
13.2 0.0310987792286772
13.3 0.0316235632843278
13.4 0.0319467566155339
13.5 0.0330580415029278
13.6 0.0333883246592481
13.7 0.0336895952006181
13.8 0.0339215452667328
13.9 0.0346835621618826
14 0.0349233994271591
14.1 0.0357485037898418
14.2 0.0360951726584645
14.3 0.0363283196651541
14.4 0.0364884460211265
14.5 0.0370796866660454
14.6 0.037313366732463
14.7 0.0375857103492834
14.8 0.0381063524247092
14.9 0.0382888582851543
15 0.0385996847019638
15.1 0.0392137533897791
15.2 0.039618650391561
15.3 0.0397253842444164
15.4 0.0399302063099984
15.5 0.0403205463768094
15.6 0.0401053244858575
15.7 0.040360986821649
15.8 0.0417654293785821
15.9 0.0427961690799251
16 0.0435780709502251
16.1 0.0442835504678864
16.2 0.0446126661162287
16.3 0.044558767993725
16.4 0.0450091221512142
16.5 0.0445282518776221
16.6 0.0447356403682737
16.7 0.044221019108101
16.8 0.0436988289272574
16.9 0.0430883142580953
17 0.0423268588808413
17.1 0.0423434199011762
17.2 0.0418031958915931
17.3 0.0411463445775176
17.4 0.0407685831114155
17.5 0.040341301485562
17.6 0.0399493561037907
17.7 0.0395461400728116
17.8 0.0390390835712872
17.9 0.038694820885538
18 0.0382930706220164
18.1 0.0379783169659332
18.2 0.0375266191358712
18.3 0.037063352870192
18.4 0.0366548186573721
18.5 0.0362437055036507
18.6 0.0357793810802253
18.7 0.0356528472405678
18.8 0.0354076697710209
18.9 0.0353195182534331
19 0.0350001431307502
19.1 0.0349696144644333
19.2 0.0348249842764863
19.3 0.0346767843396545
19.4 0.034615206099886
19.5 0.0344526066393022
19.6 0.034054930293437
19.7 0.0341056255787524
19.8 0.0342222435254718
19.9 0.034198864045503
20 0.034186216495002
20.1 0.0342138062414111
20.2 0.0342941825907756
20.3 0.0342148276293991
20.4 0.0341857758201554
20.5 0.0343736980652696
20.6 0.0342716356328499
20.7 0.0342979118798053
20.8 0.0342413255772005
20.9 0.0342121072030958
21 0.0340537399552842
21.1 0.0340418350753741
21.2 0.0339335559379558
21.3 0.0338424274158513
21.4 0.0337673172890155
21.5 0.0337783906779512
21.6 0.0336956036200094
21.7 0.0334043556784782
21.8 0.0335269648964671
21.9 0.0334405514223051
22 0.0331201135211193
22.1 0.0330759751197437
22.2 0.0331579985754211
22.3 0.0332703438025779
22.4 0.0327296782020949
22.5 0.0326079597772817
22.6000000000001 0.0321618898277143
22.7000000000001 0.0315817751128208
22.8000000000001 0.0316535679336477
22.9000000000001 0.031706152981495
23.0000000000001 0.0319822956195312
23.1000000000001 0.0314980501734845
23.2000000000001 0.0312244278483826
23.3000000000001 0.0314929646064787
23.4000000000001 0.0315540250932015
23.5000000000001 0.0311683451484065
23.6000000000001 0.0310170144592888
23.7000000000001 0.0307566014226265
23.8000000000001 0.0303783567725973
23.9000000000001 0.0303098913380733
24.0000000000001 0.0305320519594152
24.1000000000001 0.0306956485883215
24.2000000000001 0.0303848325220364
24.3000000000001 0.0307087015329573
24.4000000000001 0.0304589799637552
24.5000000000001 0.0303403047716964
24.6000000000001 0.0304610825532847
24.7000000000001 0.030991279322064
24.8000000000001 0.0312399746609726
};
%\addlegendentry{position, N = 32}
\addplot [line width = \linewidthError, color = dic32, dashed]
table {%
0 0
0.1 0.000857958833268779
0.2 0.000210346992173371
0.3 0.00043404713697992
0.4 0.00587603956742921
0.5 0.000739631986810485
0.6 0.00224409142315563
0.7 0.00361953933562864
0.8 0.000720251013578088
0.9 0.00297387719329156
1 0.00352197941262888
1.1 0.00589473776751148
1.2 0.00568558097560612
1.3 0.00702892257104692
1.4 0.00632470578850441
1.5 0.00693679234319117
1.6 0.0069877937928437
1.7 0.00816015412474214
1.8 0.0074205445721448
1.9 0.00954256957917493
2 0.00895404880234385
2.1 0.00894038507325318
2.2 0.00715894871511302
2.3 0.00662340226944141
2.4 0.0089580792666184
2.5 0.0112009806386739
2.6 0.0154707178795206
2.7 0.0134373787804195
2.8 0.0133459585777878
2.9 0.014235538345258
3 0.0173387495894983
3.1 0.0161398279369887
3.2 0.0156453684840788
3.3 0.016384516973876
3.4 0.0175080140405618
3.5 0.0185582860948532
3.6 0.0200155336159638
3.7 0.0210834656869507
3.8 0.0201052949410681
3.9 0.0213764396880632
4 0.0207205814018686
4.1 0.0207662647056067
4.2 0.0195829447839935
4.3 0.0187077693233628
4.4 0.0184804170964706
4.5 0.0194843552441122
4.6 0.0189972036907214
4.7 0.0168645237843565
4.8 0.0175843717761317
4.9 0.0163314439401079
5 0.0154561028339506
5.1 0.0140779140041265
5.2 0.0167104956817685
5.3 0.0109353206359888
5.4 0.0115092047954557
5.5 0.0136687062314169
5.6 0.0119396371177962
5.7 0.0111043359645577
5.8 0.0103954915254147
5.9 0.011508188177586
5.99999999999999 0.00927554810902698
6.09999999999999 0.0131629957617774
6.19999999999999 0.0142265901722687
6.29999999999999 0.0152060353514407
6.39999999999999 0.0156199498138234
6.49999999999999 0.0159259170208592
6.59999999999999 0.0188230226751036
6.69999999999999 0.0208615218522068
6.79999999999999 0.0227997041776447
6.89999999999999 0.0236185728372964
6.99999999999999 0.0230736021921303
7.09999999999999 0.0230197606773961
7.19999999999999 0.0252965707611528
7.29999999999999 0.0263102170011531
7.39999999999999 0.0261130117343038
7.49999999999999 0.0278714683217931
7.59999999999999 0.0303489844598741
7.69999999999999 0.0337829586729903
7.79999999999999 0.0362487854092057
7.89999999999999 0.0407884181257416
7.99999999999999 0.0423792372848486
8.09999999999999 0.0437229464597277
8.19999999999999 0.046116512257302
8.29999999999999 0.0444615021124699
8.39999999999999 0.04377785655163
8.49999999999999 0.0441379762767236
8.59999999999999 0.0437305998864876
8.69999999999999 0.0397390878216966
8.79999999999998 0.0384416463167643
8.89999999999998 0.0374386365993349
8.99999999999998 0.0348529868616803
9.09999999999998 0.0344213791129548
9.19999999999998 0.0269137878228616
9.29999999999998 0.0214280593239815
9.39999999999998 0.0192988021967211
9.49999999999998 0.017406333718204
9.59999999999998 0.0187801325156935
9.69999999999998 0.0167786451670393
9.79999999999998 0.016677385957927
9.89999999999998 0.0187852011388308
9.99999999999998 0.017734289614642
10.1 0.0198398464926748
10.2 0.016989596016618
10.3 0.0179162747145694
10.4 0.0171859368735245
10.5 0.0175395884411476
10.6 0.0187891875929234
10.7 0.0173264079844491
10.8 0.0183492670087975
10.9 0.016845178288964
11 0.0168817203392906
11.1 0.0174760009897019
11.2 0.0189603871384545
11.3 0.0179516032823708
11.4 0.0191899563756301
11.5 0.0181314993965507
11.6 0.0175325374206046
11.7 0.0174760323217442
11.8 0.0173073933229642
11.9 0.0182195344041185
12 0.0204402195898048
12.1 0.0190638382169928
12.2 0.0209184007208827
12.3 0.0188951684241845
12.4 0.019188609556676
12.5 0.0159892299164288
12.6 0.0161340816357325
12.7 0.0161660220939801
12.8 0.0173304727250976
12.9 0.0182098515889009
13 0.0167315052313226
13.1 0.0172830224076383
13.2 0.0176732495155596
13.3 0.0168861759690495
13.4 0.0187276679112252
13.5 0.0164153985053241
13.6 0.0162181595325537
13.7 0.014641643453845
13.8 0.0132516144162151
13.9 0.0139572112645752
14 0.0111102037510067
14.1 0.0120676337762275
14.2 0.00857527521781298
14.3 0.00941217786155768
14.4 0.00615195989513495
14.5 0.00930210130103015
14.6 0.00524301241087821
14.7 0.0027341569387751
14.8 0.00381303311958536
14.9 0.000502446028630477
15 0.00055981650466963
15.1 0.00304411145196593
15.2 0.00559636148746456
15.3 0.00628156313015449
15.4 0.010344635082506
15.5 0.01103977425706
15.6 0.00900173359138545
15.7 0.0182737545516307
15.8 0.0105270900217218
15.9 0.00709305909356095
16 0.00527514789027794
16.1 0.00116517396264193
16.2 0.00220833428008049
16.3 0.00211260507068545
16.4 0.00171571181820829
16.5 0.00344609036098653
16.6 0.00129509251310322
16.7 0.00182270122753669
16.8 0.00195485608500112
16.9 0.00256570045432936
17 0.00273310920284464
17.1 4.25439014932749e-05
17.2 0.0019793442802416
17.3 0.00401451221274529
17.4 0.00549457242293006
17.5 0.00547066929882378
17.6 0.00589481040404349
17.7 0.00651800472747954
17.8 0.00822022521590182
17.9 0.00599171379331542
18 0.00472290824193355
18.1 0.00413237882803164
18.2 0.00376225353346404
18.3 0.00441796476818812
18.4 0.00658495548352467
18.5 0.00704606526727281
18.6 0.00728303583868395
18.7 0.00792811745779098
18.8 0.00771358970111513
18.9 0.00888882653080869
19 0.00672019011179592
19.1 0.0052544494483715
19.2 0.00521173881153758
19.3 0.00281440622175388
19.4 0.00363688377495364
19.5 0.00308666372712652
19.6 0.00097225696679093
19.7 0.0019711619037639
19.8 0.00483921308094093
19.9 0.00354027864164563
20 0.005139743021362
20.1 0.00802781986891143
20.2 0.00876262122355265
20.3 0.010101619170668
20.4 0.0108091368458297
20.5 0.011325575558369
20.6 0.0110061674375805
20.7 0.0128727307128631
20.8 0.0161616397734579
20.9 0.0177348687853595
21 0.0160413588163066
21.1 0.0159873383534573
21.2 0.0142643500094145
21.3 0.0152987238507546
21.4 0.0127120241350449
21.5 0.00925778621884105
21.6 0.0056608965124138
21.7 0.00357020722698334
21.8 0.00389047641671695
21.9 0.000211911641826414
22 0.000456135284652409
22.1 0.00206315666205689
22.2 0.00843621073189937
22.3 0.0119736930783739
22.4 0.0138609628386686
22.5 0.0157344560158258
22.6000000000001 0.0178362977196625
22.7000000000001 0.0182617432678435
22.8000000000001 0.0188727220080291
22.9000000000001 0.0232201824794326
23.0000000000001 0.0248263663654563
23.1000000000001 0.0304297415558072
23.2000000000001 0.0266072665252735
23.3000000000001 0.02948981094354
23.4000000000001 0.0310297006811996
23.5000000000001 0.0296490157078064
23.6000000000001 0.0352221880644773
23.7000000000001 0.0394278792360693
23.8000000000001 0.0350426635547585
23.9000000000001 0.0396930655005195
24.0000000000001 0.0412898642483219
24.1000000000001 0.0425541191158009
24.2000000000001 0.0433903432065745
24.3000000000001 0.0413586625694276
24.4000000000001 0.0391360573707484
24.5000000000001 0.0450584634024097
24.6000000000001 0.0448772533422472
24.7000000000001 0.0466036679621531
24.8000000000001 0.0469783724078681
};
%\addlegendentry{$\phi$, N = 32}
\addplot [line width = \linewidthError, color = dic11]
table {%
0 0
0.1 0.0169047999232898
0.2 0.0181922062511114
0.3 0.0167887625412038
0.4 0.0174924020071367
0.5 0.0164232128763257
0.6 0.0167311419518764
0.7 0.0167610712561913
0.8 0.0163873314212342
0.9 0.0163938124287147
1 0.0162978022885072
1.1 0.0163991907183531
1.2 0.0162149943768374
1.3 0.0160435006430982
1.4 0.0159219385172409
1.5 0.0156530885953649
1.6 0.0158135619862248
1.7 0.0155707366369685
1.8 0.0154558270277891
1.9 0.0152346290293335
2 0.015180829490832
2.1 0.0148977551020125
2.2 0.0147156773111841
2.3 0.0148581848034606
2.4 0.0146050155844446
2.5 0.0146233748423588
2.6 0.014802504796052
2.7 0.0144243922371956
2.8 0.014301452937591
2.9 0.0143106122594626
3 0.0140812394844264
3.1 0.0139051851515245
3.2 0.0138657812416949
3.3 0.0138293982212883
3.4 0.0138256609615322
3.5 0.0137963586950229
3.6 0.0136961942619681
3.7 0.0139421312887387
3.8 0.0140112032124482
3.9 0.0142557052907815
4 0.0142995651385785
4.1 0.0144634623065856
4.2 0.0145145069648122
4.3 0.0145398703626187
4.4 0.0145782497958371
4.5 0.0149238299496153
4.6 0.0148963237165159
4.7 0.0149720320233472
4.8 0.0153075048571477
4.9 0.0153721653747365
5 0.0155356095162566
5.1 0.0157255835341018
5.2 0.0157486394107407
5.3 0.0160798836345583
5.4 0.0161377899848897
5.5 0.0161814904109848
5.6 0.0162772197863787
5.7 0.0165472720246243
5.8 0.0165128126894392
5.9 0.0166954714810985
5.99999999999999 0.0167317464498724
6.09999999999999 0.0168134078677256
6.19999999999999 0.0169891762160918
6.29999999999999 0.0168684283055911
6.39999999999999 0.017179197430645
6.49999999999999 0.0172029477282313
6.59999999999999 0.0170824712343284
6.69999999999999 0.0171938680162016
6.79999999999999 0.0172000601718573
6.89999999999999 0.0172091955175178
6.99999999999999 0.0175095754631389
7.09999999999999 0.0174310618874148
7.19999999999999 0.0177401958098803
7.29999999999999 0.0179339926474251
7.39999999999999 0.0178528988286744
7.49999999999999 0.0179101642128559
7.59999999999999 0.017870348337235
7.69999999999999 0.017832736033183
7.79999999999999 0.0179700418596339
7.89999999999999 0.0178592582783614
7.99999999999999 0.0178110028901565
8.09999999999999 0.0182803889213948
8.19999999999999 0.0181486712344782
8.29999999999999 0.0180449780317346
8.39999999999999 0.0181797202176985
8.49999999999999 0.0178355101628462
8.59999999999999 0.0177459982416288
8.69999999999999 0.0178187787704042
8.79999999999998 0.01772089567602
8.89999999999998 0.0177975782882391
8.99999999999998 0.0180020748471067
9.09999999999998 0.017649792159577
9.19999999999998 0.0177901405922012
9.29999999999998 0.017924758400693
9.39999999999998 0.0178676602974112
9.49999999999998 0.0176522223483827
9.59999999999998 0.017615470500675
9.69999999999998 0.0177044105066435
9.79999999999998 0.0178186086671303
9.89999999999998 0.0178272426513677
9.99999999999998 0.0177722030565702
10.1 0.0174788556992357
10.2 0.0175115144632526
10.3 0.0171903843790473
10.4 0.0174100807697067
10.5 0.0172074270763384
10.6 0.0175193946680138
10.7 0.0175686132966499
10.8 0.0175826664488772
10.9 0.0175965148880848
11 0.0175061150186342
11.1 0.0175285002791326
11.2 0.0175878220480307
11.3 0.0179366336246966
11.4 0.0181191365158551
11.5 0.0182199372078187
11.6 0.0181868295479809
11.7 0.0183447608555907
11.8 0.0187730238647218
11.9 0.0189745376914028
12 0.0191965169699439
12.1 0.0193789349690082
12.2 0.0192417375208044
12.3 0.0196510621453616
12.4 0.019904323258367
12.5 0.0199302476393145
12.6 0.0199445383367286
12.7 0.0201898310766566
12.8 0.0202865485580223
12.9 0.020544837755118
13 0.0206197972831259
13.1 0.0211258545076296
13.2 0.0209589215523308
13.3 0.0212181809961914
13.4 0.021399343237796
13.5 0.0223103485194207
13.6 0.0225715533469381
13.7 0.0227108954868453
13.8 0.0228548352986041
13.9 0.0235030068870923
14 0.0236459968427071
14.1 0.0243763917809878
14.2 0.0246710901628603
14.3 0.0248231677785018
14.4 0.0249432105739152
14.5 0.0254236173374087
14.6 0.0256456766588321
14.7 0.0259477766755307
14.8 0.0264174886882997
14.9 0.0265191074762953
15 0.0268929771607414
15.1 0.0275994115991189
15.2 0.0280520620182413
15.3 0.0281027287955015
15.4 0.0284629324269779
15.5 0.028843272858428
15.6 0.0285941255272301
15.7 0.028841708171988
15.8 0.0292738881790984
15.9 0.0295314981752193
16 0.0297144390649754
16.1 0.0299727935865493
16.2 0.0300302076356526
16.3 0.0297925857080036
16.4 0.0302504881140843
16.5 0.0298118810504246
16.6 0.0301496314662359
16.7 0.0298172091582609
16.8 0.0295548924455063
16.9 0.0292673746222882
17 0.02881703515121
17.1 0.0291165629998679
17.2 0.0289563786334424
17.3 0.0287060720417364
17.4 0.0286547628460987
17.5 0.0285488764452011
17.6 0.0284469280452392
17.7 0.028349030994181
17.8 0.0280439405521419
17.9 0.0279343531023357
18 0.0277596450493617
18.1 0.0276226622842969
18.2 0.0273255904665045
18.3 0.0270274193160651
18.4 0.0266954785988923
18.5 0.0264510115097152
18.6 0.0260079862056454
18.7 0.0258698013566477
18.8 0.0256056633195702
18.9 0.0254561496610495
19 0.0250990031385912
19.1 0.0249978831496594
19.2 0.0247915201079101
19.3 0.0245755938880193
19.4 0.0244123233523705
19.5 0.0241293287319643
19.6 0.0236781919910134
19.7 0.0235388198557458
19.8 0.0234508260644587
19.9 0.0233228305255395
20 0.023088765739149
20.1 0.0229152906827443
20.2 0.0228264284423845
20.3 0.0226802193617895
20.4 0.022503902942416
20.5 0.0224483915493836
20.6 0.0223058783814209
20.7 0.0221964588182399
20.8 0.0220716812230866
20.9 0.021948140900459
21 0.0215688548401734
21.1 0.021548441892521
21.2 0.0213884159718218
21.3 0.0211924694809139
21.4 0.0211354621840057
21.5 0.0211173329426092
21.6 0.0209758659045563
21.7 0.0208162247509606
21.8 0.0209191834915521
21.9 0.0208933975194877
22 0.0206483762885048
22.1 0.0207329156117264
22.2 0.0209493922524194
22.3 0.0212747838480979
22.4 0.0207930732472354
22.5 0.0209124939083009
22.6000000000001 0.0206012682965073
22.7000000000001 0.0201626569650302
22.8000000000001 0.0205835216576983
22.9000000000001 0.0208381664527246
23.0000000000001 0.0214771349796854
23.1000000000001 0.0211824219344904
23.2000000000001 0.021456938580577
23.3000000000001 0.0220974221136923
23.4000000000001 0.0223922218646079
23.5000000000001 0.0223841969122518
23.6000000000001 0.022878176909653
23.7000000000001 0.0230728657159473
23.8000000000001 0.0231473033739448
23.9000000000001 0.0238091084788873
24.0000000000001 0.0243267788981993
24.1000000000001 0.0251770442331311
24.2000000000001 0.0254083636284338
24.3000000000001 0.0264585616630037
24.4000000000001 0.0266544746953717
24.5000000000001 0.027433390926387
24.6000000000001 0.0282030432004643
24.7000000000001 0.0292985407476499
24.8000000000001 0.0300998831568597
};
%\addlegendentry{position, N = 11}
\addplot [line width = \linewidthError, color = dic11, dashed]
table {%
0 0
0.1 0.000846457816919388
0.2 0.000180364091963536
0.3 0.00048861679177381
0.4 0.00596036050025495
0.5 0.000857884286488586
0.6 0.0023994473423371
0.7 0.00381416068234841
0.8 0.000485194897020924
0.9 0.00324958300908684
1 0.00383765275860659
1.1 0.00624887485270353
1.2 0.00607594865439942
1.3 0.0074526657769467
1.4 0.0067784682807448
1.5 0.00741684871187243
1.6 0.00749019050310556
1.7 0.00868085725947493
1.8 0.00795559186707828
1.9 0.0100882243480434
2 0.0095069534404964
2.1 0.00949771202541694
2.2 0.00771854711732017
2.3 0.0071839389426715
2.4 0.00951917335434999
2.5 0.0117633319201422
2.6 0.0160362295868713
2.7 0.0140092764611006
2.8 0.0139289069339848
2.9 0.0148357576923602
3 0.0179641321396291
3.1 0.0168000510246581
3.2 0.0163519967171857
3.3 0.0171510780973936
3.4 0.0183500247287281
3.5 0.0194931956115324
3.6 0.0210625479828095
3.7 0.022263215903799
3.8 0.021439322642765
3.9 0.0228864932243386
4 0.0224277424716425
4.1 0.0226899549835549
4.2 0.0217398862564585
4.3 0.0211109847650679
4.4 0.0211383548381675
4.5 0.0224002059982022
4.6 0.0221684513497101
4.7 0.0202827432019468
4.8 0.0212352605258884
4.9 0.0201950558245951
5 0.0195072398856804
5.1 0.0182866302195697
5.2 0.0210426679204259
5.3 0.0153532478262806
5.4 0.0159722038953556
5.5 0.0181336808892478
5.6 0.016361601812489
5.7 0.0154368841735479
5.8 0.0145912010865061
5.9 0.0155189605808397
5.99999999999999 0.0130528811707016
6.09999999999999 0.0166581912208581
6.19999999999999 0.0173908994368788
6.29999999999999 0.0179907481206774
6.39999999999999 0.0179764318121676
6.49999999999999 0.01780560480604
6.59999999999999 0.0201773856237288
6.69999999999999 0.0216420042193555
6.79999999999999 0.0229576638273861
6.89999999999999 0.0231052381478039
6.99999999999999 0.0218400750791234
7.09999999999999 0.0210171019854291
7.19999999999999 0.0224760147235537
7.29999999999999 0.0226235741874454
7.39999999999999 0.021513334678366
7.49999999999999 0.0223140700577251
7.59999999999999 0.0237929157673449
7.69999999999999 0.0261930619384989
7.79999999999999 0.0275984470430855
7.89999999999999 0.0310631312330125
7.99999999999999 0.0315810545385888
8.09999999999999 0.0318758430865183
8.19999999999999 0.0332725869102655
8.29999999999999 0.0307078215930678
8.39999999999999 0.0292436103843787
8.49999999999999 0.0290014705491872
8.59999999999999 0.0282255151055306
8.69999999999999 0.0241593812886092
8.79999999999998 0.0231445145743319
8.89999999999998 0.0228450692725035
8.99999999999998 0.0214456216704413
9.09999999999998 0.0227395770815408
9.19999999999998 0.0175460527073379
9.29999999999998 0.0150020985260491
9.39999999999998 0.0133502309755453
9.49999999999998 0.0117429482443265
9.59999999999998 0.0132513570390786
9.69999999999998 0.0112682041569627
9.79999999999998 0.0110970780577451
9.89999999999998 0.0130696809149478
9.99999999999998 0.0118367365116772
10.1 0.0137283941470869
10.2 0.0106444009789786
10.3 0.0113271157998791
10.4 0.0103502509355478
10.5 0.0104608630418697
10.6 0.0114756445832458
10.7 0.00978992388918343
10.8 0.0106044861873786
10.9 0.00890878321178734
11 0.00877183016782146
11.1 0.00921167519948929
11.2 0.0105612153969581
11.3 0.00943736638601367
11.4 0.0105803459330769
11.5 0.00944588625990317
11.6 0.00878978072919212
11.7 0.00869432252809466
11.8 0.00850412379210663
11.9 0.00941119717479522
12 0.0116423217061548
12.1 0.0102908357184348
12.2 0.0121836479448771
12.3 0.0102108807536769
12.4 0.0105658381486067
12.5 0.00743784729332697
12.6 0.00766277731950238
12.7 0.00778230771780786
12.8 0.00904069628248516
12.9 0.0100192206734913
13 0.00864411929199349
13.1 0.00930191419454118
13.2 0.00980043602855396
13.3 0.00912271934491748
13.4 0.0110737468054927
13.5 0.00887039064497097
13.6 0.00878073624552833
13.7 0.00730987713039255
13.8 0.00602310003860129
13.9 0.00682920337215931
14 0.00407977050665664
14.1 0.00513183019187746
14.2 0.00173133810889237
14.3 0.00265774576163968
14.4 0.000514677998757485
14.5 0.00272248055547442
14.6 0.00124911094614699
14.7 0.003668359760995
14.8 0.0024957096214373
14.9 0.00570580090885775
15 0.00553808358877328
15.1 0.00901802217225578
15.2 0.0114280933161948
15.3 0.0119475338296837
15.4 0.0158149523181357
15.5 0.0162772907599313
15.6 0.0139611576748466
15.7 0.0229009525614869
15.8 0.0200632637233014
15.9 0.0206050490719281
16 0.0218865306735809
16.1 0.0200684761820189
16.2 0.0226741347327533
16.3 0.0192700778988812
16.4 0.0200244542834205
16.5 0.0181776841884478
16.6 0.0198205820101331
16.7 0.018469875674731
16.8 0.0172694153906519
16.9 0.0154070808316158
17 0.0138589582319311
17.1 0.0150856330640998
17.2 0.011640366996938
17.3 0.00808398434988433
17.4 0.00509578234752084
17.5 0.00364522752751872
17.6 0.00179647342869282
17.7 0.000189081606329022
17.8 0.00318214251921756
17.9 0.00216629428009707
18 0.00202748353439075
18.1 0.0024815855224285
18.2 0.00306952559760632
18.3 0.00459677821093396
18.4 0.00754987172253463
18.5 0.00871361098520995
18.6 0.00957244958585468
18.7 0.0107619868309727
18.8 0.0110183975033893
18.9 0.0125954197459874
19 0.0107641816436217
19.1 0.00957655921669098
19.2 0.00975807781895588
19.3 0.00753670448463462
19.4 0.00849266046783237
19.5 0.00803933548418012
19.6 0.00404666411441279
19.7 0.00308926417509303
19.8 0.000243752805969555
19.9 0.00155182238239626
20 4.66755303010213e-05
20.1 0.00293715599596811
20.2 0.00367348488381003
20.3 0.00500955010516801
20.4 0.00570684681162992
20.5 0.00620377178894094
20.6 0.00585440508488022
20.7 0.00768024726822003
20.8 0.0109181211086844
20.9 0.0124311005090608
21 0.0106697285046549
21.1 0.0105421720287571
21.2 0.00874208612111177
21.3 0.0096979452628681
21.4 0.00703337943722862
21.5 0.00350383452409816
21.6 0.000164094137484305
21.7 0.00232007131342082
21.8 0.00205809280114211
21.9 0.00578694635008405
22 0.00558424862550248
22.1 0.00813578775265698
22.2 0.0145315393405454
22.3 0.0180821406959803
22.4 0.0199731572190502
22.5 0.0218414543784166
22.6000000000001 0.0239297928863935
22.7000000000001 0.024334247927984
22.8000000000001 0.024917727072397
22.9000000000001 0.029232286488639
23.0000000000001 0.0308013740017913
23.1000000000001 0.036364730363767
23.2000000000001 0.0325006219031441
23.3000000000001 0.035341230294009
23.4000000000001 0.0368401682929725
23.5000000000001 0.0354207504877781
23.6000000000001 0.0409585666639333
23.7000000000001 0.0451333374303138
23.8000000000001 0.0407225789590474
23.9000000000001 0.045353624850235
24.0000000000001 0.0469379185497859
24.1000000000001 0.0481970299956207
24.2000000000001 0.0490358237595895
24.3000000000001 0.0470146159342401
24.4000000000001 0.0448104161522784
24.5000000000001 0.0507590329168671
24.6000000000001 0.0506115614847388
24.7000000000001 0.0523788242938664
24.8000000000001 0.0528009388312717
};
%\addlegendentry{$\phi$, N = 11}
\end{axis}




\begin{axis}[
width = \textwidth, % scaled down due to axis equal image
height = 4.4cm,
at={(0, -4.135cm)},
axis equal image=true,
legend cell align={left},
legend columns = 2,
legend style={
  fill opacity=1,
  draw opacity=1,
  text opacity=1,
  at={(0.0,1.05)},
  anchor=south west,
  column sep = 0.05cm
  %inner sep = 2pt,
  %draw=lightgray204
},
%tick align=outside,
%tick pos=left,
unbounded coords=jump,
%x grid style={darkgray176},
xlabel = {$x_1$ position (m)}, 
ylabel = {$x_2$ position (m)},
xmin=0, 
xmax=1.5,
%xtick style={color=black},
%y grid style={darkgray176},
%ylabel={\(\displaystyle y\)},
ymin=-0.3, ymax=0.4,
%ytick style={color=black}
]

\addplot [line width = \linewidthEight, blue]
table {%
0.75 0
0.753796775174918 0.00275082027626007
0.770253583658299 0.0146086690284934
0.788958864934086 0.0278626298724842
0.805758023793338 0.0400302499510602
0.823931007064239 0.0531532313053401
0.84106617676365 0.0651272462437049
0.858337230948657 0.0771219912744552
0.875920686143595 0.0890536810683165
0.892999619106597 0.100689025982836
0.910268216607307 0.111810868722874
0.927155369485107 0.122692123646889
0.944271947890646 0.133248126872312
0.961126069264421 0.143627188027739
0.977642979860227 0.15383448530992
0.994383897910681 0.163352502484412
1.01044548809055 0.172451162379304
1.02668443483368 0.181312738972183
1.04256818618544 0.189747796838266
1.05849129738013 0.197535145736718
1.07411177001412 0.204666427354084
1.08960910037061 0.21157055004819
1.10495636876448 0.217634415202293
1.11972672620237 0.223091919572481
1.13464406954527 0.228115743673971
1.14898414845007 0.232579183288128
1.16313524000647 0.236048403043336
1.17731712629478 0.239498375097147
1.19115277544504 0.241987916087121
1.20475078691893 0.24357341141794
1.2179051788757 0.245202814841758
1.23101137495675 0.245758006855585
1.24362380516706 0.245734565070592
1.25596427000739 0.245035586045094
1.26780296261423 0.243915690097645
1.27942969766319 0.242067323137536
1.29076669615838 0.23966799166844
1.30144802323049 0.236506712301561
1.31206798872445 0.232653669484729
1.32183779339603 0.228592384801218
1.33177828963827 0.223548734377477
1.34106333859447 0.218129184320505
1.35015693385091 0.212083593544392
1.35888263509532 0.205514426208893
1.36732721401058 0.198237703677997
1.37490374677743 0.190581538771692
1.38255513625004 0.18240019011285
1.38961325950765 0.173826808095369
1.39606593780378 0.16462031674418
1.40209413268775 0.155475013292566
1.40766144142924 0.145817165049195
1.41298972922002 0.135430711689483
1.4177980383817 0.125150704222236
1.42217530643287 0.113940480045831
1.42605722005813 0.103059482016292
1.42956849107093 0.0918251912690216
1.43257333591272 0.0803632587732614
1.43525005723557 0.0680997321441579
1.43747354279763 0.056305557615111
1.43903200164914 0.0443137666138458
1.44057442009313 0.0316776217651214
1.4413510801794 0.0194597121678604
1.44173985473391 0.00685847184277516
1.44175304710854 -0.00526920267813812
1.44104028644395 -0.0178002690956425
1.44009964222474 -0.0300979887959264
1.43873929662625 -0.0420597707266236
1.43683527860183 -0.0542540613878999
1.43454414915032 -0.0661993618164278
1.43178063868126 -0.07793275011297
1.42860939001972 -0.0902036039954828
1.42497929453919 -0.101367767761983
1.42084578597473 -0.112995734074035
1.41640300826551 -0.124306882350474
1.41152032108025 -0.134729554563296
1.40607079201313 -0.144853166840076
1.40030595459819 -0.154610482940984
1.39414354790019 -0.164024485370985
1.38748427065227 -0.173231382393035
1.38064408729499 -0.181896666884964
1.37331841508833 -0.190087400057888
1.36519484608907 -0.19812860388081
1.35697623393093 -0.205033044251742
1.34840153732322 -0.211514626079049
1.33928113004562 -0.21763730663906
1.33004900187448 -0.222735976435501
1.32028685217834 -0.227468875379833
1.31019782424788 -0.231938597992863
1.29972367023761 -0.235445423517443
1.2887328899364 -0.238377260466694
1.2774754396843 -0.240988537620381
1.26622519835259 -0.242443616017929
1.25423811689992 -0.243485353393633
1.24204165110139 -0.243984107342374
1.22950006048786 -0.24384429208382
1.21680621065999 -0.242965802234227
1.20378526795112 -0.241685128168492
1.19022085168209 -0.239581051822156
1.17642516281212 -0.236929456370965
1.16246464426886 -0.233589810942679
1.14828949361601 -0.229559269769245
1.134244114945 -0.225021139127758
1.1193916826543 -0.21962483904429
1.10483500153046 -0.21375219196959
1.08949301887148 -0.207426340723266
1.07427915545535 -0.200272296491305
1.05865073551656 -0.193145164392875
1.04282953663211 -0.185027373494884
1.02697988363446 -0.176541280579012
1.01101989520926 -0.167639400673986
0.994823684556384 -0.157985107306108
0.978611547483571 -0.148252113890068
0.962245616000544 -0.13813228039391
0.945289138613748 -0.127489288913879
0.92866688816798 -0.11678880795474
0.911699999758336 -0.105312068616864
0.894785005680457 -0.0934958016648549
0.877304752830891 -0.0810753361113209
0.859867338507263 -0.0690099709019099
0.842552953393209 -0.0565972164378705
0.825243828042276 -0.0441147310471777
0.807750702555605 -0.0312097420590962
0.790618156787901 -0.018184012181872
0.773448995829779 -0.00598021264680999
0.756125635736799 0.00675405645469632
0.738662857324652 0.0200698846503974
0.72119675701568 0.0334214396986119
0.703874584573642 0.0461723698235121
0.686417632616456 0.0593272292403158
0.669268676231481 0.0717358909373415
0.651945408052231 0.0846213768862322
0.634658611133318 0.0966457356254571
0.61766691011985 0.10929461111541
0.600962770468281 0.120736890028959
0.583916214443415 0.132609432334643
0.566703364898398 0.143355113283038
0.549526794264573 0.154919069656636
0.53298265702123 0.165684786365669
0.516280586189622 0.176441724076514
0.499469657201165 0.186297143282059
0.483143366453001 0.196123708388658
0.466547232727129 0.205095875426217
0.4503128526377 0.214073094476273
0.434499592218361 0.222468217516916
0.418779066207624 0.230542812331306
0.403243011226651 0.237473628045056
0.387793402240573 0.244479897043496
0.372395287160294 0.251032532682219
0.357362958539635 0.256485837349139
0.342959940828114 0.261406488246182
0.32815830763551 0.266208681304595
0.313283646731331 0.270303645502696
0.299008764566356 0.273703922777576
0.285596019227139 0.276229464479511
0.271667658663907 0.278876844869545
0.258391825439994 0.280148293958511
0.245954683337346 0.281045899586786
0.233297043080318 0.281166438800653
0.221065882947653 0.280435278965782
0.209213960006697 0.2792423328193
0.197711313341009 0.27745153093713
0.186467060395309 0.274933700120144
0.175647337802095 0.271997391117996
0.165485819498891 0.268414334553556
0.154920655621453 0.264120951359659
0.14557267250259 0.259464577309265
0.1359054564482 0.253809489621943
0.127182186643735 0.247898300503122
0.118715498185119 0.241484242043756
0.110599614202892 0.234627650065451
0.103066671467977 0.227209711390963
0.0953756606293399 0.218714717118634
0.0883116077532903 0.210301099842098
0.0816862019022073 0.201563439530642
0.0754303047598111 0.192040200227576
0.0696280035789295 0.182162510473038
0.0642727671113544 0.171854125577518
0.0592887993296431 0.161295496791183
0.0550401350221935 0.150292025379403
0.0509807490685192 0.139045437299692
0.0473874463212024 0.127624872627466
0.0442485567154118 0.115876687414116
0.04167948049022 0.104035118062325
0.0394988342888216 0.0921296980175037
0.0379006039360912 0.0799267762717973
0.0364917112037286 0.0678905527750915
0.0358931555007101 0.0555642260921366
0.0355289500473749 0.0429731447797864
0.0356926207873237 0.0304970528307748
0.0362636666860241 0.0179069982691894
0.0373684888681004 0.00563301356630583
0.0387615798631966 -0.00676904035650516
0.0406380369959789 -0.0189102629938579
0.0429542857015246 -0.0308705294137852
0.0457203908464014 -0.042746116689976
0.049035766503309 -0.0543490855274337
0.052763997030933 -0.0653774059979454
0.0569627461396097 -0.0767176672610195
0.0615888955937969 -0.0878239387828349
0.0664585976910542 -0.0982801562147958
0.072085609900655 -0.10863675691435
0.0780571592834234 -0.118599914077354
0.08432516434307 -0.128134045396256
0.0908131418279783 -0.136908599720991
0.0979353085759198 -0.145456841571941
0.105635261341533 -0.153920420196278
0.113314668270683 -0.161252708922069
0.121626519720217 -0.168399394022092
0.130199888894095 -0.174833639592176
0.139250621245577 -0.180825566954727
0.149194628828992 -0.18645514023376
0.158780448148931 -0.191196354812289
0.168971315456279 -0.195397950905863
0.179708378760775 -0.199150050909943
0.19045426687616 -0.202081472168663
0.201668373287374 -0.204582922691799
0.213406497742117 -0.206472869228286
0.225121874030311 -0.207235496554042
0.237331823289546 -0.20798471560131
0.249822155919747 -0.207816417771425
0.262786827072768 -0.206841133619445
0.275758723840688 -0.205357544128999
0.288951756143576 -0.203342196331986
0.302258118704561 -0.200601801321886
0.316710130363563 -0.197087998921957
0.330746388986938 -0.192957733094927
0.34554437580787 -0.188212280913441
0.360708133789849 -0.182801467457379
0.375228275071718 -0.176923508666162
0.390282767095232 -0.170709825662335
0.405108136218587 -0.163840010634489
0.42109976226801 -0.156280976465882
0.436573141590362 -0.147809019543375
0.452063988189108 -0.139410920745875
0.46813231140413 -0.130648654008784
0.484585950401663 -0.121010874179629
0.500662408967092 -0.110673419499024
0.517274358043854 -0.100190409204962
0.534167962758998 -0.0893077986447352
0.550610601565416 -0.0777901609279178
0.567378381435418 -0.0667074341001863
0.583918585955594 -0.054701018996261
0.601204930188776 -0.042521053347433
0.61779393188959 -0.0299697736916889
0.635312389786556 -0.0175333854783752
0.652425968524568 -0.00422174438776487
0.669577078744047 0.00880297813329933
0.68642543439641 0.0217815482795204
0.70362399878802 0.0348638435875829
0.720901407938547 0.0482750034504682
};
\addlegendentry{exp.}
\addplot [line width = \linewidthEight, color = dic120, dashed]
table {%
0.75 0
0.767599002151886 0.0126118295361513
0.785182076711963 0.0251927698026548
0.802738873641326 0.0377115068777836
0.820259101973201 0.0501368685360719
0.837732525413022 0.0624378893480144
0.855148959905304 0.0745838750873902
0.872498272934211 0.0865444667613269
0.889770384190053 0.0982897044905932
0.906955267145973 0.109790091370864
0.924042951049187 0.121016657345764
0.941023522836129 0.131941023024549
0.957887128523581 0.142535463286
0.974623973697891 0.152772970429767
0.991224322808639 0.162627316571876
1.00767849705758 0.172073114937833
1.02397687074434 0.181085879692057
1.04010986597486 0.18964208396553
1.05606794564842 0.197719215816371
1.07184160461047 0.20529583199478
1.0874213587948 0.212351609600311
1.1027977320939 0.218867396031276
1.11796124061384 0.224825258045363
1.1329023739286 0.230208531279583
1.14761157300169 0.235001872200806
1.16207920465611 0.239191315128712
1.17629553292448 0.242764337597736
1.19025068837289 0.24570993774619
1.20393463761358 0.2480187274048
1.21733715669254 0.249683043793909
1.23044781373787 0.250697080877804
1.2432559679033 0.251057038161107
1.25575079276136 0.250761279941664
1.2679213322437 0.249810492067978
1.27975659528989 0.248207817028307
1.2912456910528 0.245958943384921
1.30237799987261 0.243072124335573
1.31314336720191 0.23955810458989
1.32353230011397 0.235429945731524
1.33353614138857 0.230702756645824
1.34314719657697 0.225393353845511
1.35235879570375 0.219519891428973
1.361165282207 0.213101507060749
1.36956193443684 0.206158026334177
1.37754483596221 0.198709754384194
1.38511071728253 0.190777365031617
1.3922567922834 0.182381879625515
1.39898060877134 0.173544714622572
1.4052799257193 0.164287770866179
1.41115262270325 0.154633538038753
1.41659664108885 0.144605192772597
1.42160995263983 0.134226675896298
1.42619054938879 0.123522741198326
1.43033644839039 0.11251897367358
1.43404570575893 0.10124177904257
1.43731643562956 0.089718348501707
1.44014683097903 0.0779766035489409
1.44253518437255 0.066045125762206
1.4444799075733 0.053953075960251
1.4459795495538 0.0417301065155371
1.44703281282448 0.0294062698850458
1.44763856820124 0.0170119257654602
1.44779586822831 0.00457764869846867
1.44750395950543 -0.00786586254807691
1.44676229417717 -0.0202878729310175
1.445570540856 -0.0326576959100664
1.44392859528677 -0.0449447749100815
1.44183659113088 -0.0571187616764378
1.43929491135703 -0.0691495920865092
1.43630420087047 -0.081007560338855
1.43286538118413 -0.0926633927959661
1.42897966811407 -0.104088323080027
1.42464859363457 -0.11525417026249
1.41987403310293 -0.126133422060934
1.41465823898584 -0.136699324731265
1.40900388188451 -0.146925980646585
1.40291409894294 -0.156788453186859
1.39639254850007 -0.166262876352777
1.38944346801492 -0.175326563418449
1.38207172984115 -0.183958105192638
1.3742828865168 -0.192137444779162
1.36608319427682 -0.19984591342002
1.35747960120622 -0.207066212876103
1.3484796857924 -0.213782335670199
1.33909153363258 -0.219979426272997
1.32932354544522 -0.225643602746066
1.31918417836303 -0.230761775428051
1.30868163376518 -0.235321510638954
1.29782351660503 -0.239310986725136
1.28661650062569 -0.242719073970229
1.27506603846045 -0.245535541626782
1.26317615379328 -0.247751362894449
1.25094934448564 -0.249359062783987
1.23838661246932 -0.250353042812888
1.22738950572926 -0.250597679607199
1.21577229904072 -0.250153858121271
1.20359104394845 -0.249034152828369
1.19089750039393 -0.247250776642735
1.17773855347735 -0.244815763050321
1.16415612807932 -0.24174120240617
1.150187431093 -0.238039496234036
1.13586538277373 -0.233723603609954
1.12121913478012 -0.228807262907792
1.10627460544977 -0.2233051794575
1.09105498943454 -0.217233174888789
1.07558121839533 -0.210608297382284
1.05987236281182 -0.20344889415464
1.04394597341119 -0.195774648686815
1.02781836560698 -0.187606585813159
1.0115048528049 -0.178967048075161
0.995019935351556 -0.169879646871763
0.978377451896059 -0.160369192002709
0.961590699422093 -0.150461603247063
0.944672527458205 -0.140183807657752
0.927635411150638 -0.129563626277221
0.910491507083134 -0.118629653971776
0.89325269500445 -0.107411136022943
0.875930608003678 -0.0959378449846593
0.858536653166242 -0.0842399611016227
0.841082024349779 -0.0723479592790892
0.823577708432618 -0.0602925051976831
0.806034486196691 -0.0481043626855426
0.788462928896273 -0.0358143139087676
0.770873391516196 -0.0234530933404277
0.75327600371836 -0.0110513358442397
0.735680659493779 0.00136046140893167
0.718097006559716 0.0137519640539258
0.70053443655107 0.026093014046948
0.683002077038459 0.0383536333704101
0.665508786353323 0.0505040200023717
0.648063152108465 0.0625145383710708
0.630673494171498 0.0743557065023941
0.613347872684144 0.0859981818908045
0.596094101532102 0.0974127477818721
0.578919767471759 0.108570301052279
0.561832254927714 0.119441842220392
0.544838776307208 0.129998467328698
0.527946407553817 0.140211360520294
0.511162128603039 0.150051785095377
0.494492868426478 0.159491069687733
0.477945554477323 0.168500584951136
0.461527166592779 0.177051704797796
0.445244795778268 0.185115744800269
0.429105708791986 0.192663868890781
0.41311742004664 0.199666954047926
0.397287772996752 0.20609540140966
0.381625033782839 0.211918881485459
0.366138000278504 0.217106001357234
0.350836129538667 0.221623883758035
0.335729685526748 0.225437652907726
0.320829906263796 0.228509831669525
0.306149184359481 0.230799671167072
0.291701246313505 0.232262459932909
0.277501303230102 0.232848897022153
0.2635661285769 0.232504662654968
0.249913998855701 0.231170377181652
0.236564414861045 0.228782193790993
0.223465872028058 0.227603800579862
0.210886918249435 0.225765918299871
0.198818944331069 0.223297229677958
0.187252279497745 0.220223597759796
0.176178632051558 0.216566484620367
0.16559338114662 0.212341914499179
0.155497336155258 0.207560187071471
0.14589763991998 0.202226440710252
0.136807635934948 0.19634201628131
0.128245716252445 0.18990641704581
0.120233364049813 0.182919545252881
0.112792747243572 0.175383855944676
0.105944271353602 0.167306112684876
0.0997044568950219 0.158698539016521
0.0940843944765696 0.149579295634235
0.0890888905882633 0.139972337050033
0.0847162880231188 0.129906786023334
0.0809588523753875 0.119416000514178
0.0778035682675506 0.108536503067326
0.0752331805419137 0.0973069110667022
0.0732273343509241 0.0857669637872482
0.0717637006531057 0.0739567005372327
0.0708190094480633 0.0619158102641239
0.0703699454024078 0.0496831491900838
0.0703938861018392 0.0372964088198433
0.0708694813182412 0.024791910059516
0.0717770831741079 0.0122044978762414
0.0730990433177937 -0.000432487161856959
0.0748198956828281 -0.0130871805876362
0.0769264433971859 -0.0257291250154132
0.0794077669471858 -0.0383291724611107
0.0822551685045579 -0.0508593665465868
0.0854620648633699 -0.0632928102225007
0.0890238389856315 -0.0756035223728062
0.0929376578696846 -0.0877662850117753
0.0972022624041592 -0.099756481721854
0.101817733086478 -0.111549927496885
0.106785234002525 -0.123122690274651
0.112106736347091 -0.134450905231027
0.117784722131967 -0.145510584466398
0.123821868770651 -0.156277427156227
0.130220716208941 -0.166726638638055
0.136983320498228 -0.176832771224494
0.144110901464818 -0.186569604487807
0.15160349753217 -0.195910087698674
0.159459647554614 -0.204826370854128
0.167676126859191 -0.213289951587098
0.176247770892646 -0.221271961133389
0.185167422444122 -0.228743601504278
0.194426034449965 -0.235676727169402
0.204012947476111 -0.242044539128256
0.213916338564831 -0.247822331459941
0.224123808841763 -0.252988207386797
0.234623047322485 -0.257523671878736
0.245402486340879 -0.261414017267733
0.256451858007463 -0.26464844833234
0.267762575038841 -0.267219937856317
0.279327890672997 -0.269124851037972
0.291142832712626 -0.270362414198137
0.303203944841617 -0.270934120679113
0.315508894772245 -0.270843162450378
0.328056018966132 -0.270093954554142
0.340843868669652 -0.268691789933324
0.353870806744181 -0.266642633250101
0.367134685413867 -0.263953040110804
0.380632616761341 -0.260630175050101
0.394360833655924 -0.256681897051386
0.408314629798883 -0.252116882988681
0.422488363250244 -0.246944764490651
0.43687550697832 -0.241176260107039
0.451468731331776 -0.234823290806387
0.466260005764806 -0.227899071972124
0.481240709865287 -0.220418178943029
0.496401746276009 -0.212396585832489
0.511733650238251 -0.20385167907489
0.527226692166379 -0.194802248137803
0.542870970916354 -0.185268456331052
0.558656496317619 -0.175271794814685
0.574573260184733 -0.164835022881967
0.590611295493085 -0.15398209745121
0.606760723756808 -0.14273809448965
0.623011790933444 -0.131129124839591
0.639354892428207 -0.119182246635456
0.655780587995642 -0.106925376197878
0.672279607541467 -0.0943871989734832
0.688842849007993 -0.0815970817636981
0.705461369673834 -0.0685849871610777
0.722126372302192 -0.0553813907976371
0.738829187622178 -0.0420172017172108
0.755561254617587 -0.0285236859236244
0.772314100024278 -0.014932392937585
0.7890793183027 -0.00127508502444418
};
%\addlegendentry{SUR$_1$, $N = 120$}
\addlegendentry{$\mathbb{O}_{120}$}
\addplot [line width = \linewidthEight, color = dic32, dashed, dash pattern=on 3pt off 3pt, dash phase=1.5pt]
table {%
0.75 0
0.767603961089545 0.0126187860301889
0.785194534063579 0.025209037174046
0.80276087558598 0.0377389592461405
0.82029219361597 0.0501768908168978
0.837777753714533 0.0624913803269537
0.855206884995068 0.0746512622804318
0.872568985673112 0.0866257323212445
0.889853528163299 0.0983844209938551
0.907050063663176 0.109897465989302
0.924148226152593 0.121135582679448
0.941137735723498 0.132070132748396
0.958008401137627 0.142673190741313
0.974750121488214 0.152917608369337
0.99135288681636 0.162777076437529
1.00780677750294 0.172226184304153
1.02410196222386 0.181240476838157
1.04022869422167 0.189796508922851
1.05617730561307 0.19787189766323
1.07193819942593 0.205445372599024
1.08750183904874 0.212496824411554
1.10285873479451 0.219007352844729
1.11799942734733 0.224959314839344
1.13291446799952 0.230336374197494
1.14759439583131 0.235123554427971
1.16202971236954 0.239307296729248
1.17621085482216 0.242875525267159
1.19012816974325 0.245817721882258
1.20377188993111 0.248125011955317
1.21713211843268 0.249790262173605
1.23019882456684 0.250808189181437
1.24296185761435 0.251175475445555
1.2554109838656 0.25089088518302
1.26753595161337 0.249955369270758
1.27932658603341 0.24837214449227
1.29077291159529 0.246146730501398
1.30186529408582 0.243286928892802
1.31259458857595 0.239802733828824
1.32295227534987 0.235706172850894
1.33293056468898 0.231011088360046
1.34252245459258 0.22573288188361
1.35172173285332 0.219888251126664
1.36052292470842 0.213494951189625
1.36892119686435 0.206571605532653
1.37691223538391 0.199137581066897
1.38449211715774 0.191212928642007
1.39165719247591 0.182818378874666
1.39840399093594 0.173975376228019
1.40472915652785 0.164706132187847
1.4106294119733 0.155033680397987
1.41610154831555 0.144981921021366
1.42114243363828 0.134575646676025
1.42574903435045 0.123840546826957
1.42991844316392 0.112803190876732
1.43364790915169 0.101490992278181
1.43693486667151 0.0899321569824453
1.43977696120407 0.0781556197465006
1.44217207116151 0.0661909715626729
1.44411832544701 0.0540683809819717
1.44561411702228 0.0418185115481817
1.44665811302288 0.0294724370365002
1.44724926211031 0.0170615557443521
1.44738679981594 0.00461750472534827
1.44707025265585 -0.00782792541575849
1.44629944180463 -0.0202428747534879
1.44507448713026 -0.0325954970365358
1.4433958124196 -0.0448540413043829
1.44126415266676 -0.0569869320808701
1.43868056434795 -0.0689628481893204
1.43564643964897 -0.0807508003216072
1.43216352561531 -0.0923202075817613
1.42823394911349 -0.10364097330776
1.42386024825771 -0.114683560537608
1.41904541047879 -0.125419067499311
1.41379291658317 -0.135819303425735
1.40810678885585 -0.145856864768471
1.40199163941529 -0.155505211450185
1.39545271262932 -0.164738742111702
1.38849591261257 -0.173532866395918
1.38112780407127 -0.181864071294383
1.37335557279452 -0.189709977755444
1.36518693197314 -0.197049383579905
1.35662996343192 -0.203862289666168
1.34769288965449 -0.210129909334935
1.3383837831379 -0.215834664714455
1.32871023269077 -0.220960179157918
1.3186789988348 -0.225491278710083
1.30829569867713 -0.229414016673634
1.29756456121028 -0.232715731847748
1.28648828575572 -0.235385143108279
1.27506802090421 -0.237412472752015
1.26330346297178 -0.238789581861215
1.25119305662568 -0.239510096127051
1.23873426971061 -0.239569501704718
1.2262262687852 -0.239220599954395
1.21340380958663 -0.238215373433309
1.20027896981645 -0.23656336799035
1.18686330210173 -0.23427389494142
1.17316787013732 -0.231356342389694
1.15920332964147 -0.227820462689496
1.14498002311414 -0.223676621165882
1.13050807034399 -0.218935999962568
1.11579744588851 -0.213610756865795
1.10085804067476 -0.207714142508891
1.08569970821346 -0.201260581097911
1.07033229750911 -0.194265720310746
1.05476567526366 -0.186746455802102
1.03900973990022 -0.178720935151226
1.0230744295923 -0.170208545352095
1.00696972605909 -0.161229887205502
0.990705655476174 -0.151806739301295
0.974292287496413 -0.141962013707102
0.957739733090865 -0.131719705012245
0.941058141700757 -0.121104834004733
0.924257698029797 -0.110143386971846
0.907348618689703 -0.0988622513962332
0.890341148830372 -0.0872891486558428
0.873245558830663 -0.0754525642152462
0.856072141088955 -0.0633816757080621
0.838831206929286 -0.0511062792471669
0.82153308362471 -0.0386567142549123
0.804188111531782 -0.0260637870748376
0.786806641326645 -0.013358693605637
0.769399031332658 -0.000572941184700908
0.751975644930959 0.0122617300597357
0.734546848048078 0.0251134261734412
0.717123006718427 0.0379501784458298
0.699714484723892 0.0507400223728141
0.682331641317798 0.0634510763010384
0.664984829046121 0.0760516195411641
0.647684391685043 0.0885101697376505
0.630440662320848 0.10079555928313
0.6132639616057 0.112877010567625
0.596164596231089 0.1247242098574
0.579152857669583 0.136307379606169
0.562239021244623 0.147597349014261
0.545433345597039 0.158565622671103
0.528746072624704 0.169184447145653
0.512187427976644 0.179426875431961
0.495767622182479 0.189266829217479
0.479496852488087 0.198679159026499
0.463385305442769 0.207639702408207
0.447443160232842 0.216125340498353
0.431680592668178 0.224114053497554
0.41610777958322 0.231584975891391
0.400734903186862 0.238518452601612
0.385572154552653 0.244896097715243
0.370629734939741 0.250700857992658
0.355917852926374 0.255917083993698
0.341446714370405 0.260530612338777
0.327226500944038 0.264528863245021
0.313267331415004 0.267900957873902
0.299579198025106 0.270637859917452
0.286171868434453 0.272732544825494
0.273054742128891 0.274180197597555
0.260236649551048 0.274978435519825
0.247725583400659 0.27512754509387
0.235528355627541 0.274630712530307
0.223490098984227 0.273545022799698
0.211788699507364 0.271786324581331
0.200433396119853 0.269364717125874
0.189434329390565 0.266291947630525
0.178801957619949 0.262581425751649
0.168546459131219 0.258248193967919
0.158677208628593 0.253308838465741
0.149202401447575 0.247781340055293
0.140128867834907 0.24168488060605
0.131462079288666 0.235039632212504
0.123206312468317 0.227866559959281
0.115364913040819 0.220187264181355
0.107940596042399 0.212023877127352
0.100935728590272 0.203399016187598
0.0943525587639486 0.194335785244296
0.0881933742026251 0.184857809441107
0.0824605902924178 0.174989287007695
0.0771567783868573 0.164755043605492
0.0722846493321336 0.154180578374774
0.0678470080277173 0.143292095022365
0.0638466925971645 0.132116514947322
0.0602865084880174 0.120681472126562
0.0571691644945233 0.109015291220116
0.0544972148568642 0.0971469512627959
0.0522730094359071 0.0851060376156609
0.0504986524794862 0.0729226847780096
0.0491759695696964 0.0606275123832684
0.0483064818268612 0.0482515563402883
0.0478913862113365 0.0358261967075895
0.0479315407021177 0.0233830835383664
0.0484274531634223 0.0109540616206231
0.0493792727848656 -0.00142890524372527
0.0507867830658496 -0.0137338100366046
0.0526493953941828 -0.0259286782697503
0.0549661423384266 -0.0379816438197968
0.0577356698382698 -0.0498610252943047
0.0609562275513393 -0.061535403716201
0.0646256567208894 -0.0729737027096246
0.0687413750984074 -0.0841452728058846
0.0733003587283647 -0.0950199819301981
0.0782991208246971 -0.105568314508181
0.0837336885836123 -0.115761481827224
0.0895995796116991 -0.125571546126968
0.0958917806880078 -0.134971560148759
0.102604732736638 -0.143935722301196
0.109732326967889 -0.152439545007452
0.117267917829289 -0.160460030178626
0.125204358263788 -0.167975841429898
0.133534061355337 -0.17496745841822
0.142249089463972 -0.181417295809001
0.151341267512394 -0.187309769403624
0.160802311858367 -0.192631296131566
0.170623961427044 -0.197370223196492
0.180798095025119 -0.201516693383015
0.191316819271885 -0.205062465562829
0.202172515718929 -0.208000718253499
0.213357842631987 -0.21032586682518
0.224865694727461 -0.2120334207159
0.236689130762723 -0.213119897271126
0.248821282690013 -0.213582796716055
0.261255260579885 -0.213420631748317
0.273984065243454 -0.212632997762559
0.287000516613474 -0.211220666643709
0.300297201751396 -0.20918568782202
0.313866442780607 -0.206531483508044
0.327700282578633 -0.203262929245318
0.341790484746566 -0.199386414992875
0.356128544003002 -0.194909885227786
0.370705703413176 -0.189842858821888
0.385512975465833 -0.184196430765358
0.40054116472514 -0.177983258389023
0.41578089046844 -0.171217534814054
0.431222608299098 -0.163914952138945
0.446856630171952 -0.156092656514566
0.46267314259168 -0.147769196860917
0.478662222960428 -0.138964468603669
0.494813854183746 -0.129699653483794
0.511117937715741 -0.119997156229081
0.527564305254067 -0.109880538670597
0.54414272929805 -0.0993744517331205
0.56084293276941 -0.0885045656172072
0.577654597872549 -0.0772974984129554
0.594567374345082 -0.0657807433336114
0.611570887222572 -0.0539825947243614
0.628654744216015 -0.0419320729826101
0.645808542777668 -0.0296588485167436
0.663021876910647 -0.0171931648677433
0.680284343760517 -0.00456576111983453
0.697585550012609 0.00819220626897231
0.714915118106904 0.0210492420791758
0.732262692272557 0.0339735912402899
};
%\addlegendentry{SUR$_1$, $N = 32$}
\addlegendentry{$\mathbb{O}_{32}$}
\addplot [line width = \linewidthEight, color = dic11, dashed]
table {%
0.75 0
0.767617956326385 0.0126152320166181
0.785221863811509 0.0252027597449751
0.802800914877479 0.037730678023013
0.820344353354502 0.0501672114829468
0.837841479402121 0.0624807937498254
0.855281654212752 0.0746401456246323
0.8726543044656 0.0866143520297042
0.889948926489333 0.0983729374936874
0.907155090080218 0.109885939954903
0.924262441908226 0.121123982666573
0.941260708426491 0.132058343995875
0.958139698178854 0.142661024922621
0.974889303375735 0.152904814064391
0.991499500580103 0.16276335008561
1.00796035031295 0.172211181391685
1.02426199535241 0.18122382306993
1.04039465746435 0.189777811121737
1.05634863226828 0.197850754141243
1.07211428191705 0.205421382741197
1.08768202526063 0.21246959721333
1.10304232518768 0.218976514143235
1.11818567291197 0.224924512979598
1.13310256912115 0.230297283877598
1.14778350216287 0.235079878475666
1.16221892384545 0.239258765581085
1.17639922401035 0.242821893959482
1.19031470581062 0.245758764432686
1.20395556459181 0.248060513131333
1.21731187434358 0.249720006830931
1.23037358670628 0.250731949628373
1.24313054819378 0.251092997660852
1.25557254123141 0.250801875173767
1.26768935334999 0.249859481355763
1.27947087604527 0.248268973736809
1.29090723032719 0.246035811758442
1.30198891030508 0.243167744739263
1.31270693042084 0.2396747329622
1.3230529578381 0.235568799178242
1.33301941076826 0.230863819202118
1.34259950721967 0.2255752717836
1.35178725648959 0.219719976196704
1.36057739584553 0.21331584823695
1.36896528446889 0.206381700660446
1.37694677319549 0.198937103982443
1.38451807040133 0.191002311139298
1.39167562170647 0.182598238323915
1.39841601549682 0.173746486864112
1.40473591962011 0.164469388202287
1.41063204875531 0.154790055219354
1.4161011578983 0.144732426846829
1.42114005538066 0.134321297536615
1.42574562851262 0.123582327494227
1.42991487574103 0.112542032959374
1.43364494056566 0.101227758047146
1.43693314392358 0.0896676308607259
1.43977701306657 0.0778905070082937
1.44217430599946 0.065925903577098
1.44412303129885 0.0538039262662512
1.44562146362683 0.0415551919203837
1.44666815555294 0.02921074824029
1.4472619464584 0.0168019920270088
1.44740196937308 0.00436058696207476
1.44708765662805 -0.00808161836160665
1.44631874522255 -0.0204926716783687
1.44509528282376 -0.0328405972756649
1.44341763534724 -0.0450934751043747
1.44128649710733 -0.0572195188501123
1.4387029045691 -0.069187153238354
1.43566825475699 -0.0809650910728741
1.43218432934404 -0.0925224107720339
1.42825332530744 -0.103828635468959
1.42387789271467 -0.114853815061277
1.41906117960491 -0.125568612890349
1.41380688293672 -0.135944398919663
1.40811930307716 -0.145953351243658
1.40200339723687 -0.155568567322336
1.39546482464671 -0.164764185305358
1.38850997335112 -0.173515514002739
1.38114595578049 -0.181799167407816
1.373380558622 -0.18959319635043
1.36522213309472 -0.196877206395879
1.35667941574561 -0.203632448485574
1.34776127806877 -0.209841868332139
1.33847641527527 -0.215490103456259
1.32883299848108 -0.220563423511392
1.31883832696273 -0.225049619402342
1.30849852381291 -0.228937857372683
1.29781831612272 -0.232218522470601
1.28680092912735 -0.234883078669367
1.2754481052745 -0.23692396917894
1.26376023905526 -0.238334571285396
1.2517366021967 -0.239109208443864
1.23937562516597 -0.239243211887987
1.22680213184893 -0.238807395257225
1.21393543906645 -0.237739020556248
1.20078240593775 -0.236042071151146
1.18735022241762 -0.233721342429676
1.17364640105727 -0.230782623175358
1.15967877145654 -0.227232843878014
1.14545547101988 -0.223080191428105
1.13098493007484 -0.218334192059861
1.11627585190563 -0.21300576583718
1.10133718933183 -0.20710725660433
1.0861781196654 -0.200652441384407
1.07080801962987 -0.193656522919943
1.05523644140659 -0.186136108583467
1.03947309054353 -0.178109178360904
1.02352780610049 -0.169595044101661
1.00741054313292 -0.160614301775444
0.991131357433502 -0.151188778093633
0.974700392339589 -0.141341472543863
0.958127867358239 -0.131096495644315
0.941424068341971 -0.120479004039594
0.924599338954019 -0.109515132922426
0.907664073181879 -0.0982319261652385
0.890628708685075 -0.0866572644742216
0.873503720792848 -0.0748197918289904
0.856299616996874 -0.0627488404375812
0.839026931811664 -0.0504743544150802
0.821696221899991 -0.038026812381212
0.804318061382522 -0.0254371491653104
0.786903037269569 -0.012736676804332
0.769461744969041 4.29949803440691e-05
0.752004783838301 0.0128700386396732
0.734542752759352 0.0257124887965431
0.71708624572666 0.0385383221938064
0.699645847445474 0.0513155373923869
0.682232128945876 0.064012234017151
0.664855643224288 0.0765966913445353
0.647526920929781 0.0890374460226655
0.630256466117516 0.101303368712142
0.61305475209577 0.113363739434046
0.595932217396104 0.125188321411462
0.578899261897891 0.136747433192328
0.561966243137731 0.148012018845483
0.545143472830079 0.158953716029221
0.528441213615535 0.169544921743741
0.511869676034824 0.179758855597171
0.495439015694943 0.189569620441466
0.479159330542715 0.198952260271905
0.463040658080903 0.20788281533532
0.447092972239798 0.216338374460998
0.4313261794349 0.22429712471809
0.415750113074706 0.231738398617376
0.400374525400191 0.23864271921491
0.385209075000232 0.244991843638095
0.37026330760938 0.25076880573242
0.35554662680863 0.255957958699172
0.341068249976972 0.260545018723718
0.326837143270305 0.264517110618539
0.312861927583319 0.267862816331934
0.299150745536848 0.270572226674045
0.285711077857378 0.272636995630047
0.272549496646724 0.274050395005081
0.259671343832192 0.27480736476585
0.247080326648207 0.27490455133842
0.234778029508259 0.274340322581253
0.22282112310928 0.273147174338355
0.211212006942925 0.271312180977891
0.199956097386643 0.268842225647395
0.189059517723023 0.265745762795268
0.178528821566307 0.262032850864109
0.168370691684656 0.257715152718163
0.158591659987174 0.252805894637805
0.149197888842735 0.247319783831592
0.140195038860517 0.241272894147138
0.131588228282444 0.23468253677266
0.123382070319116 0.227567134917284
0.11558076226556 0.219946118392384
0.108188196358089 0.211839847285336
0.101208066147946 0.203269566132713
0.0946439506080792 0.194257383564486
0.0884993677377421 0.184826268657233
0.0827777974339671 0.175000054343478
0.0774826785991974 0.164803439468996
0.0726173878066766 0.154261983453973
0.0681852070170777 0.143402090097941
0.0641892867099407 0.132250979290197
0.0606326091279178 0.120836646993391
0.0575179546524513 0.1091878148431
0.0548478729145555 0.097333871164382
0.052624659190043 0.0853048053025321
0.0508503359213517 0.073131137043802
0.0495266387832368 0.0608438426733765
0.0486550064886511 0.0484742789557472
0.0482365734424979 0.0360541060686445
0.0482721643377446 0.0236152102941211
0.0487622898098838 0.0111896270720786
0.0497071422954459 -0.0011905351545308
0.0511065912634432 -0.0134931700399674
0.0529601770000401 -0.0256862472913527
0.0552671021295165 -0.0377378855864492
0.0580262200600968 -0.0496164247994674
0.0612360195719958 -0.0612904981790808
0.0648946048483175 -0.072729105492355
0.0689996704306535 -0.0839016885909927
0.0735484709159618 -0.0947782113548628
0.0785377857638054 -0.105329246473417
0.0839638804158891 -0.115526071942525
0.0898224660832313 -0.125340780325227
0.0961086620137767 -0.134746403524825
0.102816965694759 -0.143717054770436
0.109941237995478 -0.152228087444185
0.117474711258269 -0.160256267124249
0.12541002818095 -0.167779948894335
0.133739317356627 -0.174779247148807
0.142454307123939 -0.181236180931272
0.151546473070516 -0.187134775857536
0.161007207101115 -0.192461105486032
0.170827989254072 -0.197203261514197
0.181000539661948 -0.201351252882427
0.191516929010557 -0.204896846521299
0.202369632020359 -0.207833373621355
0.213551518526099 -0.210155531511126
0.225055787903675 -0.211859210562968
0.236875861773815 -0.212941368413592
0.249005254865772 -0.213399962685728
0.261437444034497 -0.213233941751724
0.274165751632416 -0.212443283842078
0.287183253563903 -0.211029069545602
0.30048271626674 -0.208993571532234
0.314056561900442 -0.206340347160422
0.327896857773206 -0.203074323123288
0.341995324490664 -0.199201865196372
0.356343357086388 -0.194730829634678
0.370932054010246 -0.189670595422765
0.385752249869232 -0.184032078335921
0.400794548920187 -0.177827728743891
0.416049357316204 -0.17107151547963
0.431506912922096 -0.163778898104052
0.447157312120673 -0.15596678968902
0.462990533450136 -0.147653511933735
0.478996458179052 -0.138858744101162
0.495164888077317 -0.12960346695478
0.511485560713425 -0.11990990261259
0.527948162627694 -0.109801451021317
0.544542340718306 -0.0993026235872079
0.561257712146769 -0.0884389743749686
0.578083873031052 -0.0772370291960328
0.595010406154612 -0.0657242128443121
0.612026887881121 -0.0539287746955709
0.629122894429922 -0.0418797128602069
0.646288007636884 -0.0296066970642742
0.663511820299399 -0.0171399904267211
0.680783941182679 -0.00451037029951237
0.69809399974657 0.00825095166029797
0.715431650637487 0.0211124100127436
0.732786577978069 0.0340421664715104
};
%\addlegendentry{SUR$_1$, $N = 11$}
\addlegendentry{$\mathbb{O}_{11}$}
\end{axis}


\begin{axis}[
width = .3325\textwidth,
height = 4.4cm,
at={(0.45\textwidth, -4.135cm)},
legend cell align={left},
legend columns = 2,
legend style={
  fill opacity=1,
  draw opacity=1,
  text opacity=1,
  at={(1,1.05)},
  anchor=south east,
  column sep = 0.05cm
  %draw=lightgray204
},
%log basis y={10},
%tick align=outside,
%tick pos=left,
%x grid style={darkgray176},
xlabel = {time (s)},
%xmin=-1.24, 
xmin=0,
xmax=26.0400000000001,
%xtick style={color=black},
%y grid style={darkgray176},
ymin=9.42097666194827e-06, ymax=0.0664155651605412,
ymode=log,
%ytick style={color=black}
]

\addplot [line width = \linewidthError, color = dic32]
table {%
0 0
0.1 0.0169710085440369
0.2 0.0183193832820399
0.3 0.0169716640295505
0.4 0.0177255864019613
0.5 0.0167012999478274
0.6 0.0170489445642226
0.7 0.0171132678475353
0.8 0.0167686252955949
0.9 0.016799117821784
1 0.0167213921474599
1.1 0.016836080907754
1.2 0.0166589908680344
1.3 0.0164902003966059
1.4 0.0163686127946576
1.5 0.0160908646624775
1.6 0.0162404228519591
1.7 0.015981905860841
1.8 0.0158495787265647
1.9 0.0156009891879512
2 0.0155115784764013
2.1 0.0151940398907489
2.2 0.0149591287142156
2.3 0.0150465496550552
2.4 0.01472482130549
2.5 0.0146783598693329
2.6 0.0147506372969869
2.7 0.0142848513478182
2.8 0.014032319698039
2.9 0.0138544795417019
3 0.0135112525647068
3.1 0.0131210899575022
3.2 0.0128656604483915
3.3 0.012574103633172
3.4 0.0123553179837111
3.5 0.0120544625242956
3.6 0.0116737133161996
3.7 0.0116236688230884
3.8 0.0113042236418369
3.9 0.011353400453753
4 0.0109639217843803
4.1 0.0108025776648706
4.2 0.0104616297420993
4.3 0.0100940750201812
4.4 0.0096271706145742
4.5 0.00960913626286408
4.6 0.00911165788314547
4.7 0.00879198677969288
4.8 0.00866364374564498
4.9 0.00854668164881082
5 0.00848133461421598
5.1 0.00819742188182943
5.2 0.00805821626015394
5.3 0.00782621192467122
5.4 0.00775104644378007
5.5 0.00760572602050709
5.6 0.00753431399734947
5.7 0.00718036515424223
5.8 0.00701128345920359
5.9 0.00703921448941707
5.99999999999999 0.00647102459822642
6.09999999999999 0.00636707973263413
6.19999999999999 0.00607535368360325
6.29999999999999 0.00590082510166369
6.39999999999999 0.00579870998649445
6.49999999999999 0.00556656350416839
6.59999999999999 0.0054305697118695
6.69999999999999 0.00520418175196376
6.79999999999999 0.00497461431876051
6.89999999999999 0.00478391305318098
6.99999999999999 0.00413665209720606
7.09999999999999 0.00396991685698191
7.19999999999999 0.00345481418120169
7.29999999999999 0.00286692261179919
7.39999999999999 0.00252036562884751
7.49999999999999 0.00226995433324859
7.59999999999999 0.00190842144395919
7.69999999999999 0.0014913332892598
7.79999999999999 0.00105560980472733
7.89999999999999 0.000484813770622354
7.99999999999999 0.000379247003299281
8.09999999999999 0.00107924931833045
8.19999999999999 0.00122088883938209
8.29999999999999 0.00155551341357802
8.39999999999999 0.00201364082724076
8.49999999999999 0.00222390622516337
8.59999999999999 0.00254874105443207
8.69999999999999 0.00316094804201204
8.79999999999998 0.00348036900393267
8.89999999999998 0.00374045641793291
8.99999999999998 0.00431090537189863
9.09999999999998 0.00467851506932722
9.19999999999998 0.00500750061500152
9.29999999999998 0.00551611408512191
9.39999999999998 0.00566535444787658
9.49999999999998 0.00584319320719339
9.59999999999998 0.00620697623095432
9.69999999999998 0.00628005201740654
9.79999999999998 0.00645519597050092
9.89999999999998 0.00662733372973905
9.99999999999998 0.00674967773996359
10.1 0.00714051454171439
10.2 0.00700626308228763
10.3 0.00723009397847647
10.4 0.00723918486473812
10.5 0.00718725573919776
10.6 0.00748579753922623
10.7 0.00737305985702866
10.8 0.00744016846110452
10.9 0.00758193465071178
11 0.00742498449212967
11.1 0.00763029273019156
11.2 0.00783735311321875
11.3 0.00765915134754846
11.4 0.00797512242630386
11.5 0.00778040226363695
11.6 0.0076335054384163
11.7 0.00693488426195946
11.8 0.00678831571400602
11.9 0.00663338435295123
12 0.00659996753303772
12.1 0.00625882636786175
12.2 0.00614909338630948
12.3 0.00675579729512964
12.4 0.00689615053694492
12.5 0.00650990349079813
12.6 0.00609138047776695
12.7 0.00617817777107764
12.8 0.00580529409013311
12.9 0.0060808601438754
13 0.00576879279595678
13.1 0.00591710474593691
13.2 0.00567622598682261
13.3 0.00623868539235107
13.4 0.00603028869771524
13.5 0.00615848425006295
13.6 0.00548212284439596
13.7 0.00549511354967656
13.8 0.00506607074283422
13.9 0.00474595594528748
14 0.00445279598216152
14.1 0.00405818184238685
14.2 0.00352800910339658
14.3 0.00326428162680515
14.4 0.00286738157831995
14.5 0.00271703225841019
14.6 0.0022599036019818
14.7 0.00179643612277162
14.8 0.00155300051925392
14.9 0.00174843155927937
15 0.00192095091648478
15.1 0.00240274302227824
15.2 0.00311867532665608
15.3 0.00354401599547911
15.4 0.00489719267910957
15.5 0.00548915405839384
15.6 0.00617762146283375
15.7 0.00690611855996418
15.8 0.00730427637937119
15.9 0.00788805060461452
16 0.00853266032798697
16.1 0.00913698918583455
16.2 0.00993035900536509
16.3 0.0106168700362168
16.4 0.0114461118169566
16.5 0.0122340902786701
16.6 0.0128391334415384
16.7 0.0135522260629128
16.8 0.0143390612761094
16.9 0.0152063413580106
17 0.0159488150982767
17.1 0.0162937120000061
17.2 0.0170699841628497
17.3 0.0179282282458616
17.4 0.0184433878748421
17.5 0.0189658206480293
17.6 0.0194047552792419
17.7 0.0199340264715701
17.8 0.0201966491227621
17.9 0.0205871895909454
18 0.0210237610949864
18.1 0.0213503632287231
18.2 0.0216917712987986
18.3 0.0221338057687732
18.4 0.0223516320838224
18.5 0.022918965636421
18.6 0.023098643773405
18.7 0.0231860896744513
18.8 0.0233260468379748
18.9 0.0233644235537861
19 0.0235610796790724
19.1 0.0236635404195856
19.2 0.023853975964444
19.3 0.0240650836438616
19.4 0.0241902465273382
19.5 0.0242882902275843
19.6 0.0246477064740628
19.7 0.0245334935822413
19.8 0.0243740537693194
19.9 0.0245769304534535
20 0.0243623935752452
20.1 0.0242095955301245
20.2 0.0241626905303039
20.3 0.0244774555019361
20.4 0.024471318904203
20.5 0.0240936593403137
20.6 0.0244319096767718
20.7 0.0243876228806674
20.8 0.0245481507440869
20.9 0.0245733719636982
21 0.0240490193496796
21.1 0.0243163764195841
21.2 0.0243459315765064
21.3 0.0241448081727931
21.4 0.0243422231042133
21.5 0.0243644339725321
21.6 0.0242128887614538
21.7 0.0245346732231395
21.8 0.0245332434900495
21.9 0.0246373152956654
22 0.0246065656011628
22.1 0.0248352890522479
22.2 0.025117984849761
22.3 0.0255809558933988
22.4 0.0251854500140863
22.5 0.0254571147463597
22.6000000000001 0.0252141069874984
22.7000000000001 0.0248440353646756
22.8000000000001 0.0253350636881268
22.9000000000001 0.0255031775302231
23.0000000000001 0.0261145796112608
23.1000000000001 0.0257575563403275
23.2000000000001 0.0261000313814923
23.3000000000001 0.0266019813635286
23.4000000000001 0.0266984143180256
23.5000000000001 0.026551346009528
23.6000000000001 0.026913578057242
23.7000000000001 0.0268807581821864
23.8000000000001 0.0266870606915206
23.9000000000001 0.0270484833699385
24.0000000000001 0.0272047806684527
24.1000000000001 0.0276616322510842
24.2000000000001 0.0274561320655367
24.3000000000001 0.0280163362643082
24.4000000000001 0.0277115756813663
24.5000000000001 0.0278604992471218
24.6000000000001 0.0280151299313805
24.7000000000001 0.0284990938504203
24.8000000000001 0.0286525271789967
};
%\addlegendentry{position, N = 32}
\addlegendentry{pos. $\mathbb{O}_{32}$}
%\addlegendentry{$e_$pos. $\mathbb{O}_{32}$}
\addplot [line width = \linewidthError, color = dic32, dashed]
table {%
0 0
0.1 0.00116213139064947
0.2 0.000812457920769449
0.3 0.000460107027594137
0.4 0.00469536920411917
0.5 0.000722404608430538
0.6 0.000505472743735802
0.7 0.00160878819719601
0.8 0.00299896866994653
0.9 0.000431144835698838
1 0.000719057197755224
1.1 0.00283542688371496
1.2 0.00237377672891925
1.3 0.00346874463732594
1.4 0.00252063871674701
1.5 0.00289383385033049
1.6 0.00271160817834376
1.7 0.00365722794282136
1.8 0.0026983415322403
1.9 0.00460968150349594
2 0.00382033935553511
2.1 0.00361712289570293
2.2 0.00165892653958594
2.3 0.000961038578644602
2.4 0.00314949904857847
2.5 0.00526407179717525
2.6 0.00942515981953709
2.7 0.00730463954897539
2.8 0.00714925660464893
2.9 0.00799976417600354
3 0.0110903419155533
3.1 0.00990659663578621
3.2 0.00945625536495864
3.3 0.0102692828245979
3.4 0.0114968391065631
3.5 0.0126812714243666
3.6 0.0143020941369118
3.7 0.0155616136318829
3.8 0.0148008246640156
3.9 0.0163120409235655
4 0.0159149248045255
4.1 0.0162331170673644
4.2 0.0153303776504953
4.3 0.0147375213224387
4.4 0.0147874572177554
4.5 0.0160566742469519
4.6 0.0158158380732254
4.7 0.0139037957588037
4.8 0.0148123127437357
4.9 0.0137103595378281
5 0.0129432345542473
5.1 0.0116261543630509
5.2 0.0142691191805562
5.3 0.0084506996557896
5.4 0.00892548678800176
5.5 0.0109284352832479
5.6 0.00898430874499057
5.7 0.0078748787149403
5.8 0.0068326726943988
5.9 0.00755294385317251
5.99999999999999 0.00486924006369982
6.09999999999999 0.00824759962376898
6.19999999999999 0.00874482104735574
6.29999999999999 0.00910141882989035
6.39999999999999 0.00883684828056519
6.49999999999999 0.00840952358823888
6.59999999999999 0.0105193380858561
6.69999999999999 0.0117173335816649
6.79999999999999 0.0127625919417387
6.89999999999999 0.0126369710354235
6.99999999999999 0.0110969577070805
7.09999999999999 0.00999883406425806
7.19999999999999 0.0111839401542042
7.29999999999999 0.0110610567001186
7.39999999999999 0.00968623051771722
7.49999999999999 0.010231304407148
7.59999999999999 0.0114671619626017
7.69999999999999 0.0136415104392533
7.79999999999999 0.0148436340552793
7.89999999999999 0.0181337752076409
7.99999999999999 0.0185128245601747
8.09999999999999 0.0187119607094828
8.19999999999999 0.0200641489862385
8.29999999999999 0.0175137342390999
8.39999999999999 0.0161300508881377
8.49999999999999 0.0160408479038625
8.59999999999999 0.0154949765511345
8.69999999999999 0.0117389980264524
8.79999999999998 0.011115354020629
8.89999999999998 0.0112874023230698
8.99999999999998 0.0104375072141014
9.09999999999998 0.012356011501987
9.19999999999998 0.00785878255637451
9.29999999999998 0.0060800922899289
9.39999999999998 0.00379795138685601
9.49999999999998 0.00168248253464043
9.59999999999998 0.00277949720860482
9.69999999999998 0.000461066926673137
9.79999999999998 1.40933074215255e-05
9.89999999999998 0.00175694374357205
9.99999999999998 0.00032976625410841
10.1 0.00205436915892721
10.2 0.00117605279059418
10.3 0.000624273995483815
10.4 0.00172059555721171
10.5 0.00172109351255179
10.6 0.000811516007934276
10.7 0.00259843095200241
10.8 0.00188252004240708
10.9 0.00367545854995965
11 0.00390908778155996
11.1 0.00356599963809945
11.2 0.00231376281962259
11.3 0.00353578643036023
11.4 0.00249206270488989
11.5 0.003726975811571
11.6 0.00448477232835565
11.7 0.00468313671506992
11.8 0.00497738405543879
11.9 0.0041753885981497
12 0.00205022664976662
12.1 0.00350839132918068
12.2 0.00172278828400874
12.3 0.0038030985483446
12.4 0.00355581600202015
12.5 0.00679141072633804
12.6 0.00667381451645799
12.7 0.00666115624567709
12.8 0.00550899708356489
12.9 0.00463589145760812
13 0.00611544799920871
13.1 0.00556100766915035
13.2 0.00516461898423604
13.3 0.00594314187704903
13.4 0.00409150154102988
13.5 0.00639274375202037
13.6 0.00657870663702864
13.7 0.00814421938988152
13.8 0.00952390905179268
13.9 0.00880887181953494
14 0.0116473871179199
14.1 0.0106822419151222
14.2 0.0141672309044365
14.3 0.0133225660646432
14.4 0.0165735300161622
14.5 0.0134111195810633
14.6 0.0174529039609959
14.7 0.0199368169539671
14.8 0.0188220786244409
14.9 0.0220818130947635
15 0.0219536181177209
15.1 0.0254607279195365
15.2 0.0278829729562644
15.3 0.0283964980924338
15.4 0.032236356493486
15.5 0.0326454690504523
15.6 0.0302459725173878
15.7 0.0390673734457399
15.8 0.0366811498851645
15.9 0.0377558620875811
16 0.0396283984249064
16.1 0.0384375695491648
16.2 0.0416868966373092
16.3 0.0389261494613944
16.4 0.0403100553299947
16.5 0.0390691546887814
16.6 0.0412877896731914
16.7 0.0404792418220303
16.8 0.0397863923173687
16.9 0.0383980076635537
17 0.037292307661787
17.1 0.0389327238401638
17.2 0.0358756513554477
17.3 0.0326850283566218
17.4 0.0300430475304276
17.5 0.0289217424965562
17.6 0.0273874613346408
17.7 0.0257034247038379
17.8 0.0230004050115646
17.9 0.0242959908358422
18 0.0247051514892755
18.1 0.0245127158521614
18.2 0.0241782931121142
18.3 0.0228968111714918
18.4 0.020182044028842
18.5 0.0192494164415051
18.6 0.0186146545186463
18.7 0.0176423077873253
18.8 0.0175963353180904
18.9 0.0162231310964609
19 0.0182517085169012
19.1 0.0196303516355592
19.2 0.0196337214482558
19.3 0.0220340698296144
19.4 0.0212514320078061
19.5 0.0218727161339054
19.6 0.0260283295197388
19.7 0.0271440417478919
19.8 0.0301436688948338
19.9 0.0289859967313446
20 0.0307316946324181
20.1 0.0337667379269101
20.2 0.0346455871403037
20.3 0.0361227526582051
20.4 0.0369603715133882
20.5 0.0375974913743912
20.6 0.0373888397688777
20.7 0.0393565472281254
20.8 0.042738033857212
20.9 0.0443969233518526
21 0.0427842478407713
21.1 0.0428085929120141
21.2 0.0411639568568397
21.3 0.04227908248154
21.4 0.0397777988462438
21.5 0.0364156914412281
21.6 0.0329194479769065
21.7 0.0309394616987471
21.8 0.0313817724493468
21.9 0.0278376157974727
22 0.0282293936870174
22.1 0.0258713370381741
22.2 0.0196734926830807
22.3 0.0163252503590963
22.4 0.0146410762305672
22.5 0.0129841399535501
22.6000000000001 0.0111117179927703
22.7000000000001 0.0109277725767375
22.8000000000001 0.0105694318241951
22.9000000000001 0.0064846715206513
23.0000000000001 0.00515007006471602
23.1000000000001 0.000174094762483645
23.2000000000001 0.0039339196397114
23.3000000000001 0.00134192905532571
23.4000000000001 9.63048878078743e-05
23.5000000000001 0.00177370161825918
23.6000000000001 0.00350151832639678
23.7000000000001 0.00740914194005671
23.8000000000001 0.00272677209543182
23.9000000000001 0.00708185110234738
24.0000000000001 0.00838595479015913
24.1000000000001 0.00936081078227635
24.2000000000001 0.00991146822023636
24.3000000000001 0.00759845617411858
24.4000000000001 0.00509902711260068
24.5000000000001 0.0107492635943243
24.6000000000001 0.0103005674044317
24.7000000000001 0.0117641011085853
24.8000000000001 0.0118803568996311
};
%\addlegendentry{$\theta$, N = 32}
\addlegendentry{$\theta$ $\mathbb{O}_{32}$}
\addplot [line width = \linewidthError, color = dic11]
table {%
0 0
0.1 0.0169803317813576
0.2 0.0183380524825849
0.3 0.0169994329592651
0.4 0.0177628544768525
0.5 0.0167482733981372
0.6 0.0171048254518305
0.7 0.017177994842162
0.8 0.016841617128994
0.9 0.016880775607022
1 0.0168098909306751
1.1 0.0169318241863382
1.2 0.0167605831569673
1.3 0.01659822153361
1.4 0.0164841679390536
1.5 0.0162109916974017
1.6 0.0163662900044609
1.7 0.0161129653099518
1.8 0.0159862006523091
1.9 0.0157409796958125
2 0.0156537946616061
2.1 0.0153394860788401
2.2 0.0151049665285745
2.3 0.0151931052967255
2.4 0.0148709065980701
2.5 0.0148248473642993
2.6 0.014893402930369
2.7 0.0144266956682082
2.8 0.0141699627662932
2.9 0.0139843323923823
3 0.0136393740808815
3.1 0.0132414684548758
3.2 0.0129788260338664
3.3 0.0126786255311549
3.4 0.0124535650428808
3.5 0.0121442438211394
3.6 0.0117552695380011
3.7 0.0116961252569817
3.8 0.0113651892882492
3.9 0.0114099947982725
4 0.0110093416607104
4.1 0.0108412653234968
4.2 0.0104930681789746
4.3 0.0101198806003127
4.4 0.00964495099509727
4.5 0.00962352680312114
4.6 0.0091226354774803
4.7 0.0088031224292807
4.8 0.00867129541189083
4.9 0.00856534753169098
5 0.00850914044106052
5.1 0.00822548577820333
5.2 0.00810086380888223
5.3 0.00786489588801787
5.4 0.00780568480016194
5.5 0.00767418911388101
5.6 0.00761626324571683
5.7 0.00725746178654381
5.8 0.00710449552983091
5.9 0.00714358057943019
5.99999999999999 0.00657412175439847
6.09999999999999 0.00648088084408907
6.19999999999999 0.00618861624794974
6.29999999999999 0.00603056719573697
6.39999999999999 0.00592546696428439
6.49999999999999 0.00569897589639973
6.59999999999999 0.00557586006279897
6.69999999999999 0.0053485777683271
6.79999999999999 0.00512075621507268
6.89999999999999 0.00493038035544573
6.99999999999999 0.00426110971926236
7.09999999999999 0.00409574747056877
7.19999999999999 0.0035561406380383
7.29999999999999 0.00294242057239208
7.39999999999999 0.00258924929924716
7.49999999999999 0.00232525338432666
7.59999999999999 0.00194916320337672
7.69999999999999 0.00151424180162726
7.79999999999999 0.00106432927144616
7.89999999999999 0.000511251527902465
7.99999999999999 0.000498095496107776
8.09999999999999 0.00125169494924612
8.19999999999999 0.00143170162231715
8.29999999999999 0.00179110312175223
8.39999999999999 0.00229304325511936
8.49999999999999 0.0024897089108602
8.59999999999999 0.00281975615699122
8.69999999999999 0.0034484875290343
8.79999999999998 0.00374743441160149
8.89999999999998 0.00399271574250835
8.99999999999998 0.00454211420203403
9.09999999999998 0.00479168790213539
9.19999999999998 0.00504065674608153
9.29999999999998 0.00543909771908302
9.39999999999998 0.00571394334785843
9.49999999999998 0.00596326892329461
9.59999999999998 0.0063922822819702
9.69999999999998 0.00652508287597452
9.79999999999998 0.00674574497182136
9.89999999999998 0.00694061363192628
9.99999999999998 0.00707178479777219
10.1 0.00743892109829856
10.2 0.00731577278046685
10.3 0.00750931795740041
10.4 0.00754150309947906
10.5 0.00747109388686939
10.6 0.0077964265922775
10.7 0.0076894183302371
10.8 0.0077567416161555
10.9 0.00789806539256896
11 0.00773455689401231
11.1 0.00794066107929189
11.2 0.00815218503984285
11.3 0.00800517721380339
11.4 0.0083337449615219
11.5 0.00814966996054809
11.6 0.00800252432807892
11.7 0.00731981412485511
11.8 0.00720627445127071
11.9 0.00706556915921511
12 0.0070461523560161
12.1 0.00671608919986446
12.2 0.00659360112160953
12.3 0.00722337867941729
12.4 0.00737473107688071
12.5 0.00698671905683244
12.6 0.00656344341994224
12.7 0.00665840749283755
12.8 0.00628233322384299
12.9 0.00656522969294673
13 0.00624697487340997
13.1 0.00640877731789195
13.2 0.00615059402924658
13.3 0.00671726920445472
13.4 0.00650329639046523
13.5 0.00664282259231065
13.6 0.00595750603439892
13.7 0.00596031486954987
13.8 0.00551901167955264
13.9 0.00519183760440542
14 0.00488602577670926
14.1 0.0044791851721605
14.2 0.0039368925799411
14.3 0.00366271077819963
14.4 0.00325637584576054
14.5 0.00309757729524781
14.6 0.00263454675435638
14.7 0.00214822920321995
14.8 0.00189148534917546
14.9 0.00207861107820358
15 0.00214636592076226
15.1 0.00247699295126985
15.2 0.00313491293034668
15.3 0.0035943109106979
15.4 0.00490634859362031
15.5 0.00549205716985859
15.6 0.00624365527323438
15.7 0.00698492544272961
15.8 0.00749648831689263
15.9 0.00817798873728571
16 0.00889714521537074
16.1 0.00954667624264508
16.2 0.0103728015094647
16.3 0.0110812896124505
16.4 0.0118956623048879
16.5 0.0126743126885908
16.6 0.0132501601262385
16.7 0.0139308870464679
16.8 0.0146786500146133
16.9 0.0155035224478013
17 0.0162007018312627
17.1 0.0165096833134791
17.2 0.0172481708603286
17.3 0.018070753098535
17.4 0.0185567297859327
17.5 0.0190534318640348
17.6 0.0194709054500246
17.7 0.0199829945911715
17.8 0.0202283543719763
17.9 0.0206086983340785
18 0.0210369270282892
18.1 0.0213584067374141
18.2 0.0216938137000093
18.3 0.0221314687353731
18.4 0.0223455373842023
18.5 0.0229115132895226
18.6 0.0230868956297012
18.7 0.0231758309020753
18.8 0.0233150796221423
18.9 0.0233547172821944
19 0.0235488079691053
19.1 0.0236529253765834
19.2 0.0238430056656106
19.3 0.0240535194747058
19.4 0.0241790988162078
19.5 0.024276076445011
19.6 0.0246322174464555
19.7 0.0245207976893606
19.8 0.0243659737202719
19.9 0.02457349247796
20 0.0243643277221537
20.1 0.0242191220382004
20.2 0.024182160746842
20.3 0.0245065453928988
20.4 0.0245114514841796
20.5 0.0241480421418707
20.6 0.0244976228782179
20.7 0.0244658910789356
20.8 0.0246372605908207
20.9 0.0246725512463314
21 0.02415624959178
21.1 0.0244306042017334
21.2 0.0244649485683456
21.3 0.0242679313623047
21.4 0.0244678047247033
21.5 0.0244931730818794
21.6 0.0243444558321334
21.7 0.0246661006813068
21.8 0.0246705355977038
21.9 0.0247793975281402
22 0.0247532711547172
22.1 0.0249899279062693
22.2 0.0252831995696752
22.3 0.0257576843891003
22.4 0.0253734008889488
22.5 0.0256583056187176
22.6000000000001 0.0254295278311547
22.7000000000001 0.0250743325600504
22.8000000000001 0.025582258939427
22.9000000000001 0.0257691286453268
23.0000000000001 0.0263988474404303
23.1000000000001 0.0260594439245115
23.2000000000001 0.0264178495583864
23.3000000000001 0.0269381298312675
23.4000000000001 0.0270527746741363
23.5000000000001 0.0269221316636723
23.6000000000001 0.0272996828159805
23.7000000000001 0.027282431021398
23.8000000000001 0.0271036783021048
23.9000000000001 0.0274788391264497
24.0000000000001 0.0276495120050991
24.1000000000001 0.0281189082006492
24.2000000000001 0.0279253298098788
24.3000000000001 0.0284963888471832
24.4000000000001 0.0282021744004847
24.5000000000001 0.0283594414295333
24.6000000000001 0.0285222635267792
24.7000000000001 0.0290139333190829
24.8000000000001 0.029174152572079
};
%\addlegendentry{position, N = 11}
\addlegendentry{pos. $\mathbb{O}_{11}$}
\addplot [line width = \linewidthError, color = dic11, dashed]
table {%
0 0
0.1 0.00114664190872216
0.2 0.000780702832106295
0.3 0.000411499357226175
0.4 0.00476121772785909
0.5 0.000639133420713311
0.6 0.000606136184147621
0.7 0.00172659767259753
0.8 0.00286447643502219
0.9 0.000581640712369991
1 0.000884665547424834
1.1 0.00301505093674237
1.2 0.00256612327779782
1.3 0.00367233593490579
1.4 0.00273382709556069
1.5 0.00311481900695276
1.6 0.00293845701771744
1.7 0.00388789682008051
1.8 0.00293070066259404
1.9 0.00484154123358932
2 0.00404947790285998
2.1 0.0038413150684028
2.2 0.00187597285247676
2.3 0.00116879409146536
2.4 0.00334590118338335
2.5 0.00544716636304587
2.6 0.00959312450926389
2.7 0.00745580430148085
2.8 0.00728212047279519
2.9 0.00811300871254447
3 0.0111828415728864
3.1 0.00997742720659763
3.2 0.00950470195526665
3.3 0.0102948490835488
3.4 0.0114992592581563
3.5 0.0126605259244861
3.6 0.0142584285199095
3.7 0.0154955585248387
3.8 0.0147132134484258
3.9 0.016204020624008
4 0.0157879558883415
4.1 0.0160889590487407
4.2 0.0151710594155801
4.3 0.014565297349021
4.4 0.0146047527293254
4.5 0.0158660234001999
4.6 0.0156198194461528
4.7 0.0137049688864979
4.8 0.0146131585765319
4.9 0.0135132267824858
5 0.0127502930965142
5.1 0.0114393564251787
5.2 0.0140901685168267
5.3 0.00828102887184357
5.4 0.00876624272463733
5.5 0.0107804722890033
5.6 0.00884818955093669
5.7 0.00775088251080724
5.8 0.00672080988586865
5.9 0.00745297685560931
5.99999999999999 0.00478070935570685
6.09999999999999 0.00816985413926075
6.19999999999999 0.00867705181854461
6.29999999999999 0.00904269459417661
6.39999999999999 0.00878615151862827
6.49999999999999 0.00836578515539022
6.59999999999999 0.0104814682226255
6.69999999999999 0.0116842466368019
6.79999999999999 0.0127332216160749
6.89999999999999 0.0126102723817554
6.99999999999999 0.0110718915530104
7.09999999999999 0.00997432906137252
7.19999999999999 0.011158827463092
7.29999999999999 0.011033972647573
7.39999999999999 0.00965548376243186
7.49999999999999 0.0101947066742527
7.59999999999999 0.0114218257643408
7.69999999999999 0.0135836238487141
7.79999999999999 0.0147682317537514
7.89999999999999 0.0180345363440262
7.99999999999999 0.0183819401861367
8.09999999999999 0.0185401242065932
8.19999999999999 0.019840728257456
8.29999999999999 0.0172271800738431
8.39999999999999 0.015768578000448
8.49999999999999 0.0155933923477747
8.59999999999999 0.0149523885170653
8.69999999999999 0.0110953804133112
8.79999999999998 0.0103694162867893
8.89999999999998 0.0104436776839281
8.99999999999998 0.00950728673557588
9.09999999999998 0.0113578810641486
9.19999999999998 0.00681870039882959
9.29999999999998 0.00503100348391472
9.39999999999998 0.00314446081948505
9.49999999999998 0.00131558129221787
9.59999999999998 0.00261429319583506
9.69999999999998 0.000431897083402077
9.79999999999998 7.04387129442097e-05
9.89999999999998 0.00186009778119578
9.99999999999998 0.000450169467554495
10.1 0.00216951014275768
10.2 0.00108327076980963
10.3 0.000566792235722424
10.4 0.00170816406402352
10.5 0.0017610056539068
10.6 0.00090916660603968
10.7 0.00275773961960724
10.8 0.00210625227314276
10.9 0.00396546954553445
11 0.00426650804669215
11.1 0.00399137674746131
11.2 0.00280717021013421
11.3 0.00409690737249901
11.4 0.00312025556862583
11.5 0.00442132521193939
11.6 0.00524412959074061
11.7 0.00550615224631157
11.8 0.00586253357154121
11.9 0.00512099468777283
12 0.00305447680354609
12.1 0.00456935330736563
12.2 0.00283842350593755
12.3 0.00497127404977915
12.4 0.00477431540767048
12.5 0.00805794461565457
12.6 0.00798603051397384
12.7 0.00801664913774758
12.8 0.00690531924739179
12.9 0.00607056388309424
13 0.00758597218721668
13.1 0.00706487869927575
13.2 0.00669934005963801
13.3 0.00750624067159045
13.4 0.00568054869304779
13.5 0.0080053738672401
13.6 0.00821264217734097
13.7 0.00979729783070526
13.8 0.0111941136828975
13.9 0.0104943665343975
14 0.0133465554742549
14.1 0.0123937306395332
14.2 0.015889997838062
14.3 0.0150559322794717
14.4 0.0183172354426389
14.5 0.0151653809401626
14.6 0.0192184730474625
14.7 0.0217150363325822
14.8 0.0206149305493168
14.9 0.0238919559434776
15 0.0237844029067844
15.1 0.0273161856977304
15.2 0.029767762385088
15.3 0.0303158018838943
15.4 0.0341957164600841
15.5 0.0346505535470345
15.6 0.0323022785096714
15.7 0.0411798798338583
15.8 0.0386225676684968
15.9 0.0395346558575502
16 0.0412572341195476
16.1 0.0399325591201967
16.2 0.0430665917356929
16.3 0.0402103785498658
16.4 0.0415187233509764
16.5 0.0402211285552787
16.6 0.0423999748370782
16.7 0.0415659197261196
16.8 0.0408588449853431
16.9 0.0394644145203773
17 0.038357869049201
17.1 0.0399999493512109
17.2 0.0369447484090522
17.3 0.033754333787718
17.4 0.031109457800063
17.5 0.0299811123706779
17.6 0.0284349537705628
17.7 0.0267338048282129
17.8 0.0240082781537823
17.9 0.025275988926927
18 0.0256520743629407
18.1 0.0254216347436169
18.2 0.0250446235018549
18.3 0.0237163614887086
18.4 0.0209510459008788
18.5 0.0199645413304075
18.6 0.0192730208207967
18.7 0.018241481829941
18.8 0.0181343291744887
18.9 0.0166983997464001
19 0.0186631485717197
19.1 0.0199773035941551
19.2 0.0199159772057746
19.3 0.0222518868758399
19.4 0.0214055553693262
19.5 0.0219644085698769
19.6 0.026059409709744
19.7 0.0271169291521989
19.8 0.0300614325837749
19.9 0.0288524037783691
20 0.0305512517301869
20.1 0.0335447179201812
20.2 0.0343880305259311
20.3 0.0358364332909938
20.4 0.0366527163962943
20.5 0.0372764471018976
20.6 0.0370626832904738
20.7 0.039033645942624
20.8 0.0424265687274247
20.9 0.0441045970552045
21 0.0425180057882154
21.1 0.0425743863004835
21.2 0.040966572765805
21.3 0.042122056906266
21.4 0.0396634165152716
21.5 0.0363450641814194
21.6 0.0328926539893583
21.7 0.03095572306677
21.8 0.0314396477716378
21.9 0.0279351880829717
22 0.028364439001148
22.1 0.0260414652849023
22.2 0.0198762571765139
22.3 0.016558226854772
22.4 0.0149019146951201
22.5 0.0132705937074823
22.6000000000001 0.0114216552645721
22.7000000000001 0.01125917528006
22.8000000000001 0.0109203857893989
22.9000000000001 0.00685335168235984
23.0000000000001 0.00553472336176208
23.1000000000001 0.000224833297334637
23.2000000000001 0.00434546273137371
23.3000000000001 0.00176445231354483
23.4000000000001 0.000528187428975091
23.5000000000001 0.00221332887642578
23.6000000000001 0.00305575893558796
23.7000000000001 0.00695886208067187
23.8000000000001 0.00227358046631665
23.9000000000001 0.00662734853478697
24.0000000000001 0.00793172679081999
24.1000000000001 0.0089084178583313
24.2000000000001 0.00946243434328486
24.3000000000001 0.00715425574860407
24.4000000000001 0.00466107081577471
24.5000000000001 0.0103188834505363
24.6000000000001 0.00987900144818832
24.7000000000001 0.011352477979547
24.8000000000001 0.0114796806993405
};
%\addlegendentry{$\theta$, N = 11}
\addlegendentry{$\theta$ $\mathbb{O}_{11}$}
\end{axis}


\end{tikzpicture}
 
    \caption{Results using~$B_1$ (top) and~$B_2$ (bottom) based on real data, where trajectories for different sets of observables are compared with the result of an experiment run. Absolute errors are depicted on the right, independently for position (pos., norm) and orientation ($\theta$). }
    \label{fig: all observables}
\end{figure} %
Again, the trajectories for~$\mathbb{O}_{32}$ and~$\mathbb{O}_{11}$ are very close to each other, even in the error plot. % there is no clear difference. 
However, with~$B_2$, using the observables~$\mathbb{O}_{120}$ results in a trajectory that is close to the reference for much longer before the error becomes visible. 
Therefore, these results seem to suggest that using the basis~$B_2$ is a lot better if the set of observables~$\mathbb{O}_{120}$, which does not use any physical insight, is used and slightly better if~$\mathbb{O}_{32}$ and~$\mathbb{O}_{11}$ are employed, which partly or fully presume translational invariance. 
Due to these findings, we subsequently use the set of observables~$\mathbb{O}_{11}$ since it seems to yield the best predictions but, due to less elements, is also the most computationally efficient. %
In particular, as Fig.~\ref{fig: mixed} shows, the surrogate models using~$\mathbb{O}_{11}$ beat the predictions of the nominal model as well as (naturally) of the surrogate model from Section~\ref{sec:simulation}. 
\definecolor{KoopData}{RGB}{255,0,0}%
\definecolor{RungeKutta}{RGB}{0,128,0}%
%\definecolor{KoopFP}{RGB}{221,160,221}%
\definecolor{KoopFP}{RGB}{255,127,14}%
\definecolor{reference}{RGB}{0,0,255}%
\def\linewidthEightB{\linewidthEight}%
\def\linewidthError{1.0}%
\def\heightEight{4.2cm}%
\begin{figure}%[h!]
    \centering%
    %\tikzset{every picture/.style={scale=0.65}}%
    % This file was created with tikzplotlib v0.10.1.
\begin{tikzpicture}
\begin{axis}[
width = \textwidth, % scaled down due to axis equal image
height = \heightEight,
axis equal image=true,
at={(0.0, 0)},
legend cell align={left},
legend columns = 2,
legend style={
  fill opacity=1,
  draw opacity=1,
  text opacity=1,
  at={(0.34,1.1)},
  anchor=south west,
  column sep = 0.25cm
  %draw=lightgray204
},
%tick align=outside,
%tick pos=left,
%x grid style={darkgray176},
xlabel = {$x_1$ position (m)}, 
ylabel = {$x_2$ position (m)},
xmin=-0, xmax=1.5,
%xticklabel style={xshift=-.05cm},
%xtick style={color=black},
%y grid style={darkgray176},
ymin=-0.3, ymax=0.4,
ylabel style={yshift=-.1cm},
%ytick style={color=black}
]
\addplot [line width = \linewidthEightB, color = reference]
table {%
0.75 0
0.753796775174918 0.00275082027626007
0.770253583658299 0.0146086690284934
0.788958864934086 0.0278626298724842
0.805758023793338 0.0400302499510602
0.823931007064239 0.0531532313053401
0.84106617676365 0.0651272462437049
0.858337230948657 0.0771219912744552
0.875920686143595 0.0890536810683165
0.892999619106597 0.100689025982836
0.910268216607307 0.111810868722874
0.927155369485107 0.122692123646889
0.944271947890646 0.133248126872312
0.961126069264421 0.143627188027739
0.977642979860227 0.15383448530992
0.994383897910681 0.163352502484412
1.01044548809055 0.172451162379304
1.02668443483368 0.181312738972183
1.04256818618544 0.189747796838266
1.05849129738013 0.197535145736718
1.07411177001412 0.204666427354084
1.08960910037061 0.21157055004819
1.10495636876448 0.217634415202293
1.11972672620237 0.223091919572481
1.13464406954527 0.228115743673971
1.14898414845007 0.232579183288128
1.16313524000647 0.236048403043336
1.17731712629478 0.239498375097147
1.19115277544504 0.241987916087121
1.20475078691893 0.24357341141794
1.2179051788757 0.245202814841758
1.23101137495675 0.245758006855585
1.24362380516706 0.245734565070592
1.25596427000739 0.245035586045094
1.26780296261423 0.243915690097645
1.27942969766319 0.242067323137536
1.29076669615838 0.23966799166844
1.30144802323049 0.236506712301561
1.31206798872445 0.232653669484729
1.32183779339603 0.228592384801218
1.33177828963827 0.223548734377477
1.34106333859447 0.218129184320505
1.35015693385091 0.212083593544392
1.35888263509532 0.205514426208893
1.36732721401058 0.198237703677997
1.37490374677743 0.190581538771692
1.38255513625004 0.18240019011285
1.38961325950765 0.173826808095369
1.39606593780378 0.16462031674418
1.40209413268775 0.155475013292566
1.40766144142924 0.145817165049195
1.41298972922002 0.135430711689483
1.4177980383817 0.125150704222236
1.42217530643287 0.113940480045831
1.42605722005813 0.103059482016292
1.42956849107093 0.0918251912690216
1.43257333591272 0.0803632587732614
1.43525005723557 0.0680997321441579
1.43747354279763 0.056305557615111
1.43903200164914 0.0443137666138458
1.44057442009313 0.0316776217651214
1.4413510801794 0.0194597121678604
1.44173985473391 0.00685847184277516
1.44175304710854 -0.00526920267813812
1.44104028644395 -0.0178002690956425
1.44009964222474 -0.0300979887959264
1.43873929662625 -0.0420597707266236
1.43683527860183 -0.0542540613878999
1.43454414915032 -0.0661993618164278
1.43178063868126 -0.07793275011297
1.42860939001972 -0.0902036039954828
1.42497929453919 -0.101367767761983
1.42084578597473 -0.112995734074035
1.41640300826551 -0.124306882350474
1.41152032108025 -0.134729554563296
1.40607079201313 -0.144853166840076
1.40030595459819 -0.154610482940984
1.39414354790019 -0.164024485370985
1.38748427065227 -0.173231382393035
1.38064408729499 -0.181896666884964
1.37331841508833 -0.190087400057888
1.36519484608907 -0.19812860388081
1.35697623393093 -0.205033044251742
1.34840153732322 -0.211514626079049
1.33928113004562 -0.21763730663906
1.33004900187448 -0.222735976435501
1.32028685217834 -0.227468875379833
1.31019782424788 -0.231938597992863
1.29972367023761 -0.235445423517443
1.2887328899364 -0.238377260466694
1.2774754396843 -0.240988537620381
1.26622519835259 -0.242443616017929
1.25423811689992 -0.243485353393633
1.24204165110139 -0.243984107342374
1.22950006048786 -0.24384429208382
1.21680621065999 -0.242965802234227
1.20378526795112 -0.241685128168492
1.19022085168209 -0.239581051822156
1.17642516281212 -0.236929456370965
1.16246464426886 -0.233589810942679
1.14828949361601 -0.229559269769245
1.134244114945 -0.225021139127758
1.1193916826543 -0.21962483904429
1.10483500153046 -0.21375219196959
1.08949301887148 -0.207426340723266
1.07427915545535 -0.200272296491305
1.05865073551656 -0.193145164392875
1.04282953663211 -0.185027373494884
1.02697988363446 -0.176541280579012
1.01101989520926 -0.167639400673986
0.994823684556384 -0.157985107306108
0.978611547483571 -0.148252113890068
0.962245616000544 -0.13813228039391
0.945289138613748 -0.127489288913879
0.92866688816798 -0.11678880795474
0.911699999758336 -0.105312068616864
0.894785005680457 -0.0934958016648549
0.877304752830891 -0.0810753361113209
0.859867338507263 -0.0690099709019099
0.842552953393209 -0.0565972164378705
0.825243828042276 -0.0441147310471777
0.807750702555605 -0.0312097420590962
0.790618156787901 -0.018184012181872
0.773448995829779 -0.00598021264680999
0.756125635736799 0.00675405645469632
0.738662857324652 0.0200698846503974
0.72119675701568 0.0334214396986119
0.703874584573642 0.0461723698235121
0.686417632616456 0.0593272292403158
0.669268676231481 0.0717358909373415
0.651945408052231 0.0846213768862322
0.634658611133318 0.0966457356254571
0.61766691011985 0.10929461111541
0.600962770468281 0.120736890028959
0.583916214443415 0.132609432334643
0.566703364898398 0.143355113283038
0.549526794264573 0.154919069656636
0.53298265702123 0.165684786365669
0.516280586189622 0.176441724076514
0.499469657201165 0.186297143282059
0.483143366453001 0.196123708388658
0.466547232727129 0.205095875426217
0.4503128526377 0.214073094476273
0.434499592218361 0.222468217516916
0.418779066207624 0.230542812331306
0.403243011226651 0.237473628045056
0.387793402240573 0.244479897043496
0.372395287160294 0.251032532682219
0.357362958539635 0.256485837349139
0.342959940828114 0.261406488246182
0.32815830763551 0.266208681304595
0.313283646731331 0.270303645502696
0.299008764566356 0.273703922777576
0.285596019227139 0.276229464479511
0.271667658663907 0.278876844869545
0.258391825439994 0.280148293958511
0.245954683337346 0.281045899586786
0.233297043080318 0.281166438800653
0.221065882947653 0.280435278965782
0.209213960006697 0.2792423328193
0.197711313341009 0.27745153093713
0.186467060395309 0.274933700120144
0.175647337802095 0.271997391117996
0.165485819498891 0.268414334553556
0.154920655621453 0.264120951359659
0.14557267250259 0.259464577309265
0.1359054564482 0.253809489621943
0.127182186643735 0.247898300503122
0.118715498185119 0.241484242043756
0.110599614202892 0.234627650065451
0.103066671467977 0.227209711390963
0.0953756606293399 0.218714717118634
0.0883116077532903 0.210301099842098
0.0816862019022073 0.201563439530642
0.0754303047598111 0.192040200227576
0.0696280035789295 0.182162510473038
0.0642727671113544 0.171854125577518
0.0592887993296431 0.161295496791183
0.0550401350221935 0.150292025379403
0.0509807490685192 0.139045437299692
0.0473874463212024 0.127624872627466
0.0442485567154118 0.115876687414116
0.04167948049022 0.104035118062325
0.0394988342888216 0.0921296980175037
0.0379006039360912 0.0799267762717973
0.0364917112037286 0.0678905527750915
0.0358931555007101 0.0555642260921366
0.0355289500473749 0.0429731447797864
0.0356926207873237 0.0304970528307748
0.0362636666860241 0.0179069982691894
0.0373684888681004 0.00563301356630583
0.0387615798631966 -0.00676904035650516
0.0406380369959789 -0.0189102629938579
0.0429542857015246 -0.0308705294137852
0.0457203908464014 -0.042746116689976
0.049035766503309 -0.0543490855274337
0.052763997030933 -0.0653774059979454
0.0569627461396097 -0.0767176672610195
0.0615888955937969 -0.0878239387828349
0.0664585976910542 -0.0982801562147958
0.072085609900655 -0.10863675691435
0.0780571592834234 -0.118599914077354
0.08432516434307 -0.128134045396256
0.0908131418279783 -0.136908599720991
0.0979353085759198 -0.145456841571941
0.105635261341533 -0.153920420196278
0.113314668270683 -0.161252708922069
0.121626519720217 -0.168399394022092
0.130199888894095 -0.174833639592176
0.139250621245577 -0.180825566954727
0.149194628828992 -0.18645514023376
0.158780448148931 -0.191196354812289
0.168971315456279 -0.195397950905863
0.179708378760775 -0.199150050909943
0.19045426687616 -0.202081472168663
0.201668373287374 -0.204582922691799
0.213406497742117 -0.206472869228286
0.225121874030311 -0.207235496554042
0.237331823289546 -0.20798471560131
0.249822155919747 -0.207816417771425
0.262786827072768 -0.206841133619445
0.275758723840688 -0.205357544128999
0.288951756143576 -0.203342196331986
0.302258118704561 -0.200601801321886
0.316710130363563 -0.197087998921957
0.330746388986938 -0.192957733094927
0.34554437580787 -0.188212280913441
0.360708133789849 -0.182801467457379
0.375228275071718 -0.176923508666162
0.390282767095232 -0.170709825662335
0.405108136218587 -0.163840010634489
0.42109976226801 -0.156280976465882
0.436573141590362 -0.147809019543375
0.452063988189108 -0.139410920745875
0.46813231140413 -0.130648654008784
0.484585950401663 -0.121010874179629
0.500662408967092 -0.110673419499024
0.517274358043854 -0.100190409204962
0.534167962758998 -0.0893077986447352
0.550610601565416 -0.0777901609279178
0.567378381435418 -0.0667074341001863
0.583918585955594 -0.054701018996261
0.601204930188776 -0.042521053347433
0.61779393188959 -0.0299697736916889
0.635312389786556 -0.0175333854783752
0.652425968524568 -0.00422174438776487
0.669577078744047 0.00880297813329933
0.68642543439641 0.0217815482795204
0.70362399878802 0.0348638435875829
};
\addlegendentry{experiment}
\addplot [line width = \linewidthEightB, color = KoopData, dashed]
table {%
0.75 0
0.76755241918154 0.0125770360734689
0.785094588334309 0.0251301206995445
0.802615768422832 0.0376274406940617
0.820105219838029 0.0500373518085269
0.837552207690645 0.0623284592372939
0.854946007617056 0.074469697633784
0.872275912129028 0.086430410408615
0.889531237541612 0.0981804280764377
0.9067013315151 0.109690145411972
0.92377558124756 0.120930597168908
0.94074342235371 0.131873532107707
0.957594348463223 0.142491485069601
0.974317921566538 0.152757846823946
0.990903783127944 0.1626469314042
1.00734166597315 0.17213404063414
1.02362140694019 0.181195525530252
1.03973296025677 0.189808844249087
1.05566641157089 0.197952616230358
1.07141199251356 0.205606672169423
1.0869600956066 0.21275209943922
1.10230128924473 0.219371282576201
1.11742633237225 0.225447938453961
1.1323261883408 0.230967145801618
1.14699203727458 0.23591536879426
1.16141528608592 0.240280474566176
1.17558757509037 0.244051744692501
1.18950077998566 0.247219880970171
1.20314700781965 0.249777006218392
1.21651858553078 0.251716661313515
1.22960803976876 0.253033800250613
1.24240806707214 0.25372478562479
1.25491149416758 0.253787387441952
1.26711122920332 0.25322078844437
1.27900020611338 0.252025598980213
1.29057132589744 0.250203883677939
1.30181740013914 0.247759200706198
1.31273110320684 0.244696652271017
1.32330493987965 0.241022942510425
1.33353123430114 0.236746436592264
1.3434021440953 0.231877213211155
1.35290970043004 0.226427102338591
1.36204587135707 0.22040970123742
1.37080264264475 0.213840364225197
1.37917210825347 0.206736164897287
1.38714656198417 0.199115832747226
1.39471858266055 0.190999668662709
1.40188110711095 0.182409445217072
1.40862748762955 0.173368297952542
1.41495153295503 0.163900613183994
1.42084753370315 0.154031916611227
1.42631027443962 0.143788765592335
1.43133503518375 0.133198646591897
1.43591758521579 0.122289878244891
1.44005417179306 0.11109151972866
1.44374150592659 0.0996332836922258
1.44697674685773 0.0879454527953125
1.44975748638836 0.0760587988874999
1.45208173380175 0.0640045039455847
1.45394790178286 0.0518140820329462
1.45535479350293 0.0395193017123464
1.45630159086532 0.0271521085106083
1.4567878438032 0.0147445471879033
1.45681346046101 0.00232868370115145
1.45637869807152 -0.0100634731298463
1.45548415434864 -0.0224000501409959
1.45413075925101 -0.0346493883221522
1.45231976702979 -0.0467801210049043
1.4500527485591 -0.0587612513687231
1.44733158406382 -0.0705622288149383
1.44415845651342 -0.0821530236800519
1.44053584615044 -0.0935041996858419
1.43646652687411 -0.104586983459988
1.43195356550482 -0.115373330418978
1.42700032530424 -0.125835986302048
1.42161047548785 -0.135948543703755
1.41578800878154 -0.145685493100053
1.40953726923632 -0.155022268123604
1.40286299237103 -0.163935285233157
1.39577035906101 -0.17240197843023
1.38826506321929 -0.180400830256571
1.38035339106651 -0.187911400861719
1.37204230666799 -0.194914357319135
1.36333953473614 -0.201391505430404
1.3542536281534 -0.207325825862971
1.34479400532321 -0.212701515596036
1.33497094249192 -0.217504034441296
1.32479550948 -0.221720155164188
1.31427944388999 -0.225338014845258
1.30343496778845 -0.228347164913979
1.29227456009816 -0.230738617865077
1.28081070516113 -0.232504889846041
1.26905564133691 -0.233640039669527
1.25702113245639 -0.234139705880083
1.24441011162052 -0.233998287891216
1.23150987709577 -0.233198341697921
1.21832656651924 -0.231742514092606
1.20486742784165 -0.229634992404436
1.19114053864056 -0.226881537660985
1.17715458341523 -0.223489493601953
1.16291868161436 -0.219467778197301
1.14844225801647 -0.214826862862404
1.13373494718014 -0.20957874317165
1.11880652448217 -0.20373690369585
1.10366685737742 -0.197316278673818
1.08832587169285 -0.190333209562072
1.07279352886637 -0.182805400049116
1.05707981099097 -0.174751868824916
1.04119471130626 -0.166192900218104
1.02514822840069 -0.15714999271666
1.00895036286773 -0.147645805344513
0.992611115522913 -0.137704101856562
0.976140486558654 -0.12734969272414
0.959548475211641 -0.116608374902796
0.942845079660339 -0.105506869398556
0.926040296971458 -0.0940727566741864
0.909144122985133 -0.0823344099613076
0.892166552076897 -0.0703209265666302
0.875117576766641 -0.0580620572806502
0.85800718716504 -0.0455881340148065
0.840845370259688 -0.032929995808521
0.823642109048982 -0.0201189133609487
0.806407381533151 -0.00718651225386622
0.789151159570214 0.00583530495776762
0.771883407600853 0.0189144376022564
0.754614081240848 0.0320186665015801
0.737353125733372 0.0451157341359413
0.72011047424635 0.0581734251565391
0.702896045992399 0.0711596470351756
0.685719744140889 0.084042510634244
0.668591453483182 0.0967904104758398
0.651521037803386 0.109372104484028
0.634518336897664 0.121756792969428
0.617593163175272 0.133914196620055
0.60075529776378 0.145814633256359
0.584014486028954 0.157429093101353
0.567380432406184 0.168729312307992
0.550862794424515 0.179687844475008
0.53447117578541 0.190278129868391
0.518215118335564 0.200474562047656
0.502104092744889 0.210252551572776
0.486147487665828 0.219588586437715
0.470354597106223 0.228460288838171
0.454734605692619 0.236846467832474
0.439296571430687 0.244727167393402
0.424049405480447 0.252083709272674
0.409001848350778 0.258898730006982
0.394162441773925 0.26515621128305
0.379539495338511 0.27084150274901
0.365141046729479 0.275941336212366
0.350974814135408 0.280443830006716
0.337048139027343 0.284338482151869
0.323367917081623 0.28761615079512
0.309940514512528 0.290269020336629
0.296771666516665 0.292290551654394
0.283866353959162 0.293675415013908
0.271228653955301 0.294419404645913
0.258861559805596 0.294519334677139
0.247031604760098 0.29395349045429
0.23547143674765 0.292757307006016
0.2241899044659 0.290936052873756
0.213196074803946 0.288495719772803
0.202499271499672 0.285443106420999
0.192109092542846 0.28178593595438
0.182035398550079 0.27753299731309
0.17228826709279 0.272694296889299
0.162877912661648 0.267281204594729
0.153814577529934 0.261306578981305
0.145108403694737 0.254784859105045
0.136769298938872 0.247732115722647
0.128806810187289 0.240166059962376
0.121230014963775 0.232106012591895
0.114047437820772 0.223572840539405
0.107266994293942 0.214588869069725
0.100895961207256 0.205177778129504
0.0949409695776963 0.195364490320161
0.089408015054005 0.185175056286597
0.0843024805750487 0.17463654150049
0.0796291664146447 0.163776916790454
0.0753923236512213 0.152624953688422
0.0715956880933569 0.141210124759649
0.0682425126305576 0.129562508524023
0.0653355967711454 0.117712698286325
0.0628773127456472 0.105691714094334
0.06086962800188 0.0935309170675819
0.059314124222916 0.0812619254333148
0.05821201319318 0.0689165317329006
0.0575641499516463 0.0565266207982147
0.0573710437303835 0.0441240882305625
0.0576328672016192 0.0317407592386542
0.0583494645615412 0.0194083078063393
0.0595203589735943 0.00715817626727523
0.0611447598831278 -0.00497850453349301
0.0632215706999686 -0.0169709942072329
0.0657493973232483 -0.0287890205249709
0.0687265579472097 -0.0404028553434391
0.0721510945275002 -0.0517833881083993
0.0760207861903484 -0.0629021961473827
0.0803331647142979 -0.0737316108171151
0.0850855319861997 -0.084244778435932
0.090274979012033 -0.0944157148364653
0.0958984056390185 -0.104219352362336
0.101952539626688 -0.11363157826498
0.10843395313028 -0.122629263806195
0.115339074113728 -0.131190284011962
0.122664189827784 -0.13929352900097
0.130405439453343 -0.146918909109598
0.138558793516447 -0.154047357525038
0.147120018874589 -0.160660835546871
0.156084629960897 -0.16674234651805
0.165447829341293 -0.172275964442567
0.175204443034591 -0.177246881994962
0.185348857847125 -0.181641479987509
0.195874968587712 -0.185447416788218
0.206776142124176 -0.188653732464125
0.218045202922803 -0.191250959503987
0.229674441517362 -0.193231230608216
0.24165564407494 -0.194588374503751
0.253980138615449 -0.195317992777205
0.266638851970831 -0.195417513652186
0.279622371327238 -0.194886221683642
0.292921004944799 -0.193725264872096
0.306524837992732 -0.191937642375725
0.320423780962919 -0.189528176774178
0.33460760952187 -0.186503474874346
0.349065995755118 -0.182871880593262
0.363788531497047 -0.178643422751748
0.378764744852599 -0.173829759855221
0.393984111175732 -0.168444123246458
0.409436059752959 -0.162501259449541
0.425109977319895 -0.156017372100539
0.440995209368846 -0.149010063568604
0.457081060023385 -0.14149827618821
0.473356791083654 -0.133502232923346
0.489811620695517 -0.125043377243583
0.5064347219717 -0.116144311990172
0.523215221793459 -0.106828737032676
0.540142199944715 -0.0971213855521212
0.557204688673377 -0.0870479588279965
0.574391672733236 -0.0766350594488637
0.591692089930899 -0.0659101229070835
0.609094832182975 -0.054901347575697
0.626588747076685 -0.0436376230991757
0.644162639920403 -0.0321484572594443
0.661805276267829 -0.0204639014044578
0.679505384899576 -0.008614474549008
0.69725166124786 0.00336891372325352
0.715032771253293 0.0154550414109311
0.732837355646911 0.0276124531391578
};
\addlegendentry{Koopman (real data)}
\addplot [line width = \linewidthEightB, color = RungeKutta, dashdotted]
table {%
0.75 0
0.767347219519284 0.012137253483567
0.784683394455509 0.0242435861667548
0.801997486582194 0.036288155514001
0.819278471531621 0.0482402761288358
0.836515345700615 0.0600694978228124
0.853697133135913 0.0717456830897325
0.870812892392119 0.0832390837898383
0.887851723355235 0.0945204168511733
0.904802774024683 0.105560938798497
0.921655247246804 0.116332518924018
0.938398407392915 0.126807710918802
0.955021586975309 0.136959822789028
0.971514193195148 0.146762984887382
0.987865714417124 0.156192215896701
1.0040657265673 0.165223486610677
1.02010389945277 0.173833781364724
1.03597000300535 0.182001156979107
1.05165391345605 0.189704799085743
1.06714561945411 0.196925075719347
1.08243522815337 0.203643588062188
1.09751297130037 0.209843218238558
1.1123692113737 0.215508174058654
1.12699444784162 0.220624030609957
1.14137932362429 0.225177768584601
1.1555146318663 0.22915780921071
1.16939132313901 0.232554045621132
1.18300051319534 0.235357870442502
1.19633349138035 0.237562199321834
1.20938172974852 0.23916149003295
1.22213689283924 0.240151756735333
1.23459084790774 0.240530578918402
1.24673567520343 0.24029710459028
1.25856367765532 0.239452047402427
1.27006738911946 0.237997677671498
1.28123958024499 0.235937807669022
1.29207326111455 0.233277772045945
1.30256168017249 0.230024404727089
1.31269831956044 0.226186013888674
1.32247688771958 0.221772356568193
1.33189131078413 0.216794613983565
1.34093572465289 0.211265367835258
1.34960446953272 0.205198576949498
1.35789208820684 0.198609552872807
1.36579332846363 0.191514932670796
1.37330314928634 0.18393264729003
1.38041672978698 0.175881884318464
1.38712947958844 0.167383044631953
1.39343704941026 0.158457693037802
1.39933534088867 0.149128503482261
1.40482051503024 0.139419199628507
1.40988899904863 0.129354491659788
1.41453749160152 0.118960010081206
1.41876296661194 0.108262237150103
1.42256267593454 0.0972884364110782
1.42593415113889 0.086066580678514
1.42887520465437 0.0746252787098322
1.43138393047621 0.0629937007473294
1.43345870458359 0.0512015030695605
1.43509818517756 0.0392787516771522
1.43630131281055 0.0272558452352367
1.43706731045357 0.0151634373994425
1.43739568352819 0.00303235866024866
1.43728621991849 -0.00910646215137196
1.43673898997073 -0.0212220765475008
1.43575434648365 -0.0332835951024502
1.43433292468914 -0.0452602660529673
1.43247564221903 -0.0571215536269705
1.43018369904855 -0.0688372159533007
1.42745857739782 -0.0803773824121825
1.4243020415584 -0.0917126302942692
1.42071613759078 -0.102814060643163
1.41670319280875 -0.113653373158817
1.41226581492764 -0.124202940032363
1.40740689070718 -0.134435878560225
1.40212958387022 -0.144326122339201
1.3964373320371 -0.153848490767307
1.39033384239815 -0.162978756463708
1.38382308587764 -0.171693710079292
1.37690928964889 -0.179971221816331
1.36959692806383 -0.187790298849553
1.36189071236441 -0.195131137800147
1.35379557991389 -0.201975171526458
1.34531668404164 -0.20830510981277
1.33645938581161 -0.214104974062127
1.32722924897007 -0.219360126749639
1.31763203892423 -0.224057297001799
1.30767372588241 -0.22818460402503
1.2973604914176 -0.231731580045289
1.28669873695777 -0.234689193904312
1.27569509230234 -0.237049875620184
1.26435642233521 -0.238807541316647
1.25268983059913 -0.239957617226964
1.24070265912099 -0.240497061156197
1.22840248459827 -0.24042437986202
1.21579711159144 -0.239739641182038
1.20289456363565 -0.238444480231447
1.1897030732081 -0.236542099469318
1.17623107134149 -0.234037262792463
1.16248717745014 -0.230936284032406
1.14848018970572 -0.227247010319085
1.13421907610845 -0.222978800771493
1.11971296626323 -0.218142500920566
1.10497114378721 -0.212750413195198
1.09000303923492 -0.206816263729265
1.07481822341698 -0.200355165686855
1.05942640099553 -0.193383579258294
1.04383740425637 -0.185919268450478
1.02806118697728 -0.177981254778958
1.01210781833129 -0.169589767962932
0.995987476780275 -0.160766193724954
0.979710443928391 -0.151533018802209
0.963287098315684 -0.141913773283855
0.946727909140567 -0.131932970397911
0.930043429905878 -0.121616043880467
0.91324429198767 -0.110989283069269
0.896341198128817 -0.100079765872392
0.879344915861624 -0.0889152897708354
0.862266270864871 -0.0775243010211479
0.845116140261585 -0.0659358222307342
0.827905445864243 -0.0541793784842489
0.810645147374399 -0.042284922204474
0.793346235543765 -0.0302827569353333
0.776019725303823 -0.0182034602382608
0.758676648870971 -0.00607780589600706
0.74132804883419 0.00606331437980999
0.723984971232126 0.0181889695395348
0.706658458626486 0.0302682678858693
0.689359543178601 0.0422704356646973
0.672099239736024 0.0541648953575106
0.654888538936098 0.0659213434775336
0.637738400333409 0.077509827673207
0.620659745558167 0.0889008229448788
0.603663451512546 0.10006530678342
0.586760343612044 0.110974833042994
0.569961189078862 0.121601604364455
0.55327669029407 0.131918542970798
0.536717478214957 0.141899359661781
0.520294105863188 0.151518620841353
0.504017041888231 0.160751813418715
0.487896664208613 0.169575407431921
0.471943253730781 0.177966916251536
0.456166988141264 0.18590495423109
0.440577935762111 0.193369291680399
0.425186049451712 0.200340907046881
0.410001160522753 0.206802036197834
0.395032972635676 0.212736218702091
0.38029105560979 0.218128341010777
0.36578483907538 0.222964676431608
0.351523605870586 0.227232921776542
0.337516485069478 0.23092223053541
0.323772444518238 0.234023242385741
0.310300282763089 0.236528108790522
0.297108620287721 0.238430514364134
0.284205890052545 0.239725693611976
0.27160032745433 0.240410442590784
0.259299960006205 0.240483125024951
0.247312597261469 0.239943672488383
0.2356458217318 0.238793578459958
0.224306981713199 0.237035886402939
0.21330318694315 0.234675172482025
0.202641307788403 0.231717524032126
0.192327978173203 0.228170515287301
0.1823696017639 0.22404318200088
0.172772360193836 0.219345996319791
0.163542221572307 0.214090842618848
0.154684947376684 0.208290994107505
0.146206096155645 0.201961089155653
0.138111023180157 0.195117105712544
0.130404876038488 0.187776332069152
0.12309258692145 0.179957332525672
0.116178862798813 0.171679907117356
0.109668174793245 0.162965045212245
0.10356474787866 0.153834873345794
0.0978725516915446 0.144312598005936
0.0925952928735381 0.134422444219678
0.0877364090495248 0.124189590767097
0.0832990643281257 0.113640102728524
0.0792861460924091 0.102800861916777
0.0757002628080311 0.0916994956001386
0.0725437425870994 0.0803643038046822
0.069818632284583 0.0688241854024739
0.0675266969522104 0.0571085631419612
0.0656694195214907 0.0452473077513642
0.0642480006271671 0.0332706612373471
0.0632633585132331 0.0212091595029022
0.0627161289858991 0.00909355441514592
0.0626066653929575 -0.00304526453801165
0.0629350386185861 -0.0151763488558262
0.0637010370885092 -0.0272687697454203
0.0649041667842718 -0.0392916968007361
0.0665436512687222 -0.0512144764858195
0.0686184317292081 -0.0630067102720941
0.0711271670520408 -0.0746383322854955
0.0740682339532131 -0.0860796863272447
0.0774397272079407 -0.0973016021399556
0.0812394600470219 -0.108275470796062
0.0854649648223936 -0.118973319084073
0.0901134940873028 -0.12936788275431
0.0951820222852561 -0.139432678452301
0.100667248289028 -0.149142074107302
0.10656559906303 -0.158471357449301
0.112873234718821 -0.167396802199979
0.119586055168304 -0.175895731332128
0.126699708426136 -0.183946576646328
0.13420960035749 -0.191528933821879
0.142110905322839 -0.198623612126504
0.150398576795015 -0.205212678179984
0.159067356720643 -0.211279493592426
0.16811178230224 -0.216808746901301
0.177526189102462 -0.221786480886398
0.187304709947427 -0.226200116851138
0.197441269929379 -0.230038477621279
0.207929578656845 -0.233291810717099
0.218763121506754 -0.235951812451675
0.229935151803763 -0.238011652801433
0.241438685552319 -0.239466000064469
0.253266499708268 -0.240311043799429
0.265411134226006 -0.240544514421553
0.277864897472831 -0.240165698073282
0.290619874197545 -0.23917544583971
0.303667935101709 -0.237576176879925
0.317000747134867 -0.235371875470324
0.330609783830706 -0.232568082244952
0.344486335235128 -0.229171880066449
0.358621517191145 -0.225191874998346
0.373006279910045 -0.220638172815656
0.387631415867 -0.215522351422736
0.402487567119295 -0.209857429471725
0.417565232169048 -0.203657831407191
0.432854772492479 -0.196939349109845
0.448346418844808 -0.189719100275549
0.46403027743119 -0.182015483643688
0.47989633601438 -0.173848131178146
0.495934470011695 -0.165237857301697
0.512134448618413 -0.156206605287707
0.528485940982374 -0.146777390919585
0.544978522444945 -0.136974243536887
0.561601680856407 -0.126822144596177
0.578344822968638 -0.116346963884112
0.595197280904322 -0.105575393529208
0.61214831869946 -0.094534879967153
0.629187138914278 -0.0832535540222168
0.646302889306651 -0.071760159274263
0.663484669561462 -0.0600839788869683
0.680721538069075 -0.0482547610782518
0.698002518745858 -0.0363026434185232
0.715316607889725 -0.0242580761462783
0.732652781063624 -0.012151744693772
};
\addlegendentry{first principles}
\addplot [line width = \linewidthEightB, color = KoopFP, dashed]
table {%
0.75 0
0.767347054978347 0.0121373833630193
0.784677602060703 0.0242516439523981
0.801980571179635 0.0363119684500103
0.819244924555801 0.0482877203131606
0.836459664396623 0.0601485180224483
0.853613840868092 0.071864312763047
0.870696560354337 0.0834054653296501
0.887696994025053 0.0947428220418834
0.904604386736512 0.105847789452861
0.921408066297426 0.116692407628364
0.938097453136357 0.127249421767403
0.954662070412182 0.137492351926154
0.971091554613034 0.147395560595838
0.987375666691188 0.156934317870403
1.00350430378066 0.166084863921137
1.01946751153911 0.174824468471981
1.03525549714407 0.183131486940725
1.0508586429523 0.190985412877308
1.06626752079696 0.198366926291414
1.08147290684468 0.205257937418939
1.09646579685822 0.211641625433762
1.1112374216029 0.21750247157313
1.12577926198915 0.222826286121221
1.14008306335317 0.22760022870005
1.15414084803853 0.23181282137034
1.16794492515786 0.235453954174507
1.1814878960983 0.238514882993358
1.19476265402202 0.240988219974778
1.2077623753616 0.242867917359917
1.22048050121002 0.244149246296079
1.2329107066792 0.244828773163726
1.24504685689018 0.244904336972368
1.25688294938906 0.244375032323923
1.26841304451468 0.243241203034675
1.2796311874911 0.241504451412514
1.29053132848551 0.239167667081363
1.30110724903819 0.236235076951994
1.31135250445939 0.232712314578619
1.3212603913661 0.228606503231458
1.3308239471747 0.223926343429178
1.34003598428659 0.218682193410168
1.34888915670603 0.212886130843688
1.35737605209074 0.206551986191669
1.36548929890997 0.199695342040466
1.37322167718164 0.192333497418731
1.38056622222815 0.184485400452706
1.38751631348133 0.1761715557943
1.39406574371726 0.167413914698674
1.4002087673685 0.158235755535207
1.4059401291556 0.14866156132798
1.41125507594654 0.138716899184472
1.41614935551906 0.128428304658086
1.4206192059609 0.11782317251623
1.42466133904661 0.106929654199161
1.42827292030648 0.0957765614755542
1.43145154782291 0.084393275373029
1.43419523115883 0.0728096592989896
1.43650237129347 0.0610559752814652
1.43837174202818 0.0491628023780043
1.43980247302221 0.037160956469416
1.44079403440686 0.0250814108390321
1.44134622278715 0.0129552171169125
1.44145914835271 0.000813426332619842
1.44113322276889 -0.011312990032289
1.44036914749319 -0.0233932189677048
1.43916790215328 -0.0353966846860806
1.43753073262703 -0.0472931269321206
1.43545913848171 -0.059052678457586
1.43295485946215 -0.0706459411338684
1.43001986077099 -0.0820440600271581
1.42665631696913 -0.0932187945978508
1.42286659445074 -0.10414258601106
1.41865323262776 -0.114788619370402
1.41401892420313 -0.125130879535788
1.40896649522163 -0.135144199095972
1.4034988859479 -0.144804297093682
1.39761913399219 -0.154087807316814
1.3913303614051 -0.162972295452156
1.38463576757066 -0.171436265214902
1.37753862948051 -0.179459154738344
1.3700423102048 -0.187021325965854
1.36215027498192 -0.194104051334669
1.3538661123781 -0.200689503340555
1.34519355571479 -0.206760753191263
1.33613649800646 -0.212301784289242
1.3266989927643 -0.217297524531136
1.31688523392337 -0.221733898546293
1.30669951116539 -0.225597897612439
1.29614614168486 -0.228877661955082
1.28522938491881 -0.231562568296413
1.2739533514596 -0.233643315336791
1.26232191995253 -0.235112001236638
1.25033867554517 -0.235962189538053
1.23800688056869 -0.236188962525886
1.22535530162818 -0.235806596379587
1.21241108799083 -0.234800320626771
1.19917953297074 -0.233173662490519
1.18566713397227 -0.230931629953662
1.17188133327006 -0.228080749490437
1.1578303091684 -0.22462907866385
1.14352281110701 -0.220586200014247
1.12896803164881 -0.215963201369397
1.1141755084805 -0.210772646440631
1.09915505022711 -0.205028538473181
1.08391668076584 -0.198746278843701
1.06847059765082 -0.191942621841865
1.05282714112779 -0.18463562640679
1.03699677097733 -0.17684460527438
1.02099004906013 -0.168590071790554
1.00481762595181 -0.159893684525334
0.988490230460268 -0.150778189758196
0.972018661132604 -0.141267361876786
0.955413779098169 -0.131385941725412
0.938686501774324 -0.121159572947238
0.921847797095791 -0.110614736378655
0.904908678027327 -0.0997786825718992
0.887880197191637 -0.0886793625401684
0.870773441496676 -0.0773453568370064
0.853599526683711 -0.0658058030976513
0.836369591743756 -0.0540903221842539
0.819094793168043 -0.0422289430891916
0.801786299010178 -0.0302520267612395
0.784455282745178 -0.0181901890282879
0.767112916914725 -0.00607422279777744
0.749770366549592 0.00606498027771305
0.732438782359734 0.0181965084770487
0.715129293680487 0.0302895088321661
0.697853001159686 0.0423132658937509
0.680620969165621 0.0542372802364514
0.663444217889342 0.0660313464956735
0.646333715106968 0.077665630724372
0.62930036755807 0.0891107468540641
0.612355011884591 0.100337832039176
0.595508405060859 0.111318620657244
0.57877121422842 0.122025516728889
0.562154005829219 0.132431664510053
0.545667233906249 0.142511016993873
0.529321227411337 0.152238402039612
0.513126176324167 0.161589585819943
0.497092116343534 0.170541333244005
0.481228911859792 0.179071464970092
0.465546236854292 0.187158910566643
0.450053553295271 0.194783757310962
0.434760086507028 0.201927294029801
0.419674796877053 0.208572049282563
0.404806347129678 0.214701823065565
0.390163064229739 0.220301711075022
0.3757528947794 0.225358120410665
0.361583352528383 0.229858775438469
0.347661456324375 0.233792712373341
0.333993656478486 0.237150261011582
0.320585747105712 0.239923011968294
0.307442761525848 0.242103767795487
0.294568847298227 0.243686476517607
0.281967116970193 0.244666146468884
0.269639470258095 0.245038741888252
0.257609186010719 0.244817533899865
0.245901399933677 0.243986910658693
0.234516753126707 0.242549409532126
0.223456707969872 0.240508544229666
0.212723665481129 0.23786884972081
0.202321029913742 0.234635962038441
0.192253212857014 0.230816726674294
0.182525574486551 0.226419324289144
0.173144306163025 0.22145339886659
0.164116265134771 0.215930172271658
0.155448777290745 0.20986253088828
0.147149426599873 0.203265074248231
0.139225849552868 0.196154121259358
0.131685549880995 0.188547675397098
0.12453574400103 0.180465354807253
0.117783242247029 0.171928295970874
0.111434366095059 0.16295904031871
0.10549489797027 0.153581412351688
0.0999700580861348 0.143820396051152
0.0948645019704017 0.133702014261661
0.0901823325408759 0.123253213755743
0.0859271214139484 0.112501757099468
0.082101935218733 0.101476121313332
0.0787093638055608 0.0902054026341887
0.0757515482349136 0.0787192263436368
0.0732302072470627 0.0670476605351242
0.0711466615328075 0.0552211327547731
0.0695018555710937 0.0432703485988681
0.068296377102251 0.0312262115350228
0.0675304745007113 0.0191197434040398
0.0672040724294083 0.00698200523925239
0.0673167862247707 -0.0051559817965022
0.0678679354950424 -0.0172633154091155
0.0688565574283197 -0.0293092910248553
0.0702814203075648 -0.0412634802855324
0.0721410377205165 -0.0530958088488365
0.0744336839309413 -0.0647766330200938
0.0771574108376325 -0.0762768145938376
0.080310066877845 -0.0875677931201561
0.0838893181165073 -0.0986216546331728
0.0878926715811209 -0.109411195695586
0.092317500631241 -0.119909981443113
0.0971610717678999 -0.130092396189907
0.102420571776795 -0.139933685133161
0.108093133462666 -0.149409985844674
0.114175857508555 -0.158498348647097
0.12066582727242 -0.167176745731478
0.127560112771697 -0.175424070040615
0.134855759927475 -0.183220126506273
0.142549761605577 -0.190545620048193
0.15063900833137 -0.197382146513268
0.159120218855544 -0.203712193986738
0.16798985383409 -0.209519162098616
0.177244019247762 -0.214787405629269
0.186878369030618 -0.219502305760343
0.196888017834081 -0.223650368087699
0.207267474297313 -0.227219341877283
0.218010602542386 -0.230198351139615
0.229110615425264 -0.232578025936489
0.240560098387395 -0.234350622421377
0.252351058701079 -0.235510122278639
0.264474992317481 -0.236052305726898
0.276922959710988 -0.235974796108093
0.289685662851306 -0.235277077430042
0.30275351718526 -0.233960488544756
0.316116714658043 -0.232028198796135
0.329765275847062 -0.229485170124563
0.343689090917888 -0.226338110085313
0.357877950232066 -0.222595419351347
0.372321566069414 -0.218267136293317
0.387009587174264 -0.213364880331941
0.401931607817156 -0.207901795026849
0.417077172890188 -0.20189249132472
0.432435780309396 -0.195352991024026
0.447996881737398 -0.188300670292018
0.463749882397286 -0.180754202955754
0.479684140540365 -0.172733503249844
0.495788966960878 -0.164259667712575
0.51205362481888 -0.15535491595896
0.528467329933241 -0.146042530110185
0.545019251634789 -0.136346792714638
0.561698514218957 -0.126292923050579
0.578494199003119 -0.115907011751378
0.595395346971731 -0.105215953739731
0.612390961979189 -0.094247379496698
0.629470014473327 -0.0830295847251056
0.64662144569983 -0.0715914584952316
0.663834172348055 -0.0599624099842478
0.681097091600941 -0.0481722939404261
0.698399086555007 -0.0362513350190936
0.715729031980509 -0.0242300511503997
0.733075800396316 -0.0121391761095536
};
\addlegendentry{Koopman (sim.\ data)}
\end{axis}

\begin{axis}[
width = \textwidth, % scaled down due to axis equal image
height = \heightEight,
axis equal image=true,
at={(.37\textwidth, 0)},
%legend cell align={left},
legend style={
  fill opacity=1,
  draw opacity=1,
  text opacity=1,
  at={(0.5,0.91)},
  anchor=north,
  %draw=lightgray204
},
%tick align=outside,
%tick pos=left,
%x grid style={darkgray176},
xlabel = {$x_1$ position (m)}, 
%ylabel = {$x_2$ position (m)},
xmin=-0, xmax=1.5,
%xtick style={color=black},
%y grid style={darkgray176},
ymin=-0.3, ymax=0.4,
yticklabels = {},
%ytick style={color=black}
]

\addplot [line width = \linewidthEightB, color = reference]
table {%
0.75 0
0.753796775174918 0.00275082027626007
0.770253583658299 0.0146086690284934
0.788958864934086 0.0278626298724842
0.805758023793338 0.0400302499510602
0.823931007064239 0.0531532313053401
0.84106617676365 0.0651272462437049
0.858337230948657 0.0771219912744552
0.875920686143595 0.0890536810683165
0.892999619106597 0.100689025982836
0.910268216607307 0.111810868722874
0.927155369485107 0.122692123646889
0.944271947890646 0.133248126872312
0.961126069264421 0.143627188027739
0.977642979860227 0.15383448530992
0.994383897910681 0.163352502484412
1.01044548809055 0.172451162379304
1.02668443483368 0.181312738972183
1.04256818618544 0.189747796838266
1.05849129738013 0.197535145736718
1.07411177001412 0.204666427354084
1.08960910037061 0.21157055004819
1.10495636876448 0.217634415202293
1.11972672620237 0.223091919572481
1.13464406954527 0.228115743673971
1.14898414845007 0.232579183288128
1.16313524000647 0.236048403043336
1.17731712629478 0.239498375097147
1.19115277544504 0.241987916087121
1.20475078691893 0.24357341141794
1.2179051788757 0.245202814841758
1.23101137495675 0.245758006855585
1.24362380516706 0.245734565070592
1.25596427000739 0.245035586045094
1.26780296261423 0.243915690097645
1.27942969766319 0.242067323137536
1.29076669615838 0.23966799166844
1.30144802323049 0.236506712301561
1.31206798872445 0.232653669484729
1.32183779339603 0.228592384801218
1.33177828963827 0.223548734377477
1.34106333859447 0.218129184320505
1.35015693385091 0.212083593544392
1.35888263509532 0.205514426208893
1.36732721401058 0.198237703677997
1.37490374677743 0.190581538771692
1.38255513625004 0.18240019011285
1.38961325950765 0.173826808095369
1.39606593780378 0.16462031674418
1.40209413268775 0.155475013292566
1.40766144142924 0.145817165049195
1.41298972922002 0.135430711689483
1.4177980383817 0.125150704222236
1.42217530643287 0.113940480045831
1.42605722005813 0.103059482016292
1.42956849107093 0.0918251912690216
1.43257333591272 0.0803632587732614
1.43525005723557 0.0680997321441579
1.43747354279763 0.056305557615111
1.43903200164914 0.0443137666138458
1.44057442009313 0.0316776217651214
1.4413510801794 0.0194597121678604
1.44173985473391 0.00685847184277516
1.44175304710854 -0.00526920267813812
1.44104028644395 -0.0178002690956425
1.44009964222474 -0.0300979887959264
1.43873929662625 -0.0420597707266236
1.43683527860183 -0.0542540613878999
1.43454414915032 -0.0661993618164278
1.43178063868126 -0.07793275011297
1.42860939001972 -0.0902036039954828
1.42497929453919 -0.101367767761983
1.42084578597473 -0.112995734074035
1.41640300826551 -0.124306882350474
1.41152032108025 -0.134729554563296
1.40607079201313 -0.144853166840076
1.40030595459819 -0.154610482940984
1.39414354790019 -0.164024485370985
1.38748427065227 -0.173231382393035
1.38064408729499 -0.181896666884964
1.37331841508833 -0.190087400057888
1.36519484608907 -0.19812860388081
1.35697623393093 -0.205033044251742
1.34840153732322 -0.211514626079049
1.33928113004562 -0.21763730663906
1.33004900187448 -0.222735976435501
1.32028685217834 -0.227468875379833
1.31019782424788 -0.231938597992863
1.29972367023761 -0.235445423517443
1.2887328899364 -0.238377260466694
1.2774754396843 -0.240988537620381
1.26622519835259 -0.242443616017929
1.25423811689992 -0.243485353393633
1.24204165110139 -0.243984107342374
1.22950006048786 -0.24384429208382
1.21680621065999 -0.242965802234227
1.20378526795112 -0.241685128168492
1.19022085168209 -0.239581051822156
1.17642516281212 -0.236929456370965
1.16246464426886 -0.233589810942679
1.14828949361601 -0.229559269769245
1.134244114945 -0.225021139127758
1.1193916826543 -0.21962483904429
1.10483500153046 -0.21375219196959
1.08949301887148 -0.207426340723266
1.07427915545535 -0.200272296491305
1.05865073551656 -0.193145164392875
1.04282953663211 -0.185027373494884
1.02697988363446 -0.176541280579012
1.01101989520926 -0.167639400673986
0.994823684556384 -0.157985107306108
0.978611547483571 -0.148252113890068
0.962245616000544 -0.13813228039391
0.945289138613748 -0.127489288913879
0.92866688816798 -0.11678880795474
0.911699999758336 -0.105312068616864
0.894785005680457 -0.0934958016648549
0.877304752830891 -0.0810753361113209
0.859867338507263 -0.0690099709019099
0.842552953393209 -0.0565972164378705
0.825243828042276 -0.0441147310471777
0.807750702555605 -0.0312097420590962
0.790618156787901 -0.018184012181872
0.773448995829779 -0.00598021264680999
0.756125635736799 0.00675405645469632
0.738662857324652 0.0200698846503974
0.72119675701568 0.0334214396986119
0.703874584573642 0.0461723698235121
0.686417632616456 0.0593272292403158
0.669268676231481 0.0717358909373415
0.651945408052231 0.0846213768862322
0.634658611133318 0.0966457356254571
0.61766691011985 0.10929461111541
0.600962770468281 0.120736890028959
0.583916214443415 0.132609432334643
0.566703364898398 0.143355113283038
0.549526794264573 0.154919069656636
0.53298265702123 0.165684786365669
0.516280586189622 0.176441724076514
0.499469657201165 0.186297143282059
0.483143366453001 0.196123708388658
0.466547232727129 0.205095875426217
0.4503128526377 0.214073094476273
0.434499592218361 0.222468217516916
0.418779066207624 0.230542812331306
0.403243011226651 0.237473628045056
0.387793402240573 0.244479897043496
0.372395287160294 0.251032532682219
0.357362958539635 0.256485837349139
0.342959940828114 0.261406488246182
0.32815830763551 0.266208681304595
0.313283646731331 0.270303645502696
0.299008764566356 0.273703922777576
0.285596019227139 0.276229464479511
0.271667658663907 0.278876844869545
0.258391825439994 0.280148293958511
0.245954683337346 0.281045899586786
0.233297043080318 0.281166438800653
0.221065882947653 0.280435278965782
0.209213960006697 0.2792423328193
0.197711313341009 0.27745153093713
0.186467060395309 0.274933700120144
0.175647337802095 0.271997391117996
0.165485819498891 0.268414334553556
0.154920655621453 0.264120951359659
0.14557267250259 0.259464577309265
0.1359054564482 0.253809489621943
0.127182186643735 0.247898300503122
0.118715498185119 0.241484242043756
0.110599614202892 0.234627650065451
0.103066671467977 0.227209711390963
0.0953756606293399 0.218714717118634
0.0883116077532903 0.210301099842098
0.0816862019022073 0.201563439530642
0.0754303047598111 0.192040200227576
0.0696280035789295 0.182162510473038
0.0642727671113544 0.171854125577518
0.0592887993296431 0.161295496791183
0.0550401350221935 0.150292025379403
0.0509807490685192 0.139045437299692
0.0473874463212024 0.127624872627466
0.0442485567154118 0.115876687414116
0.04167948049022 0.104035118062325
0.0394988342888216 0.0921296980175037
0.0379006039360912 0.0799267762717973
0.0364917112037286 0.0678905527750915
0.0358931555007101 0.0555642260921366
0.0355289500473749 0.0429731447797864
0.0356926207873237 0.0304970528307748
0.0362636666860241 0.0179069982691894
0.0373684888681004 0.00563301356630583
0.0387615798631966 -0.00676904035650516
0.0406380369959789 -0.0189102629938579
0.0429542857015246 -0.0308705294137852
0.0457203908464014 -0.042746116689976
0.049035766503309 -0.0543490855274337
0.052763997030933 -0.0653774059979454
0.0569627461396097 -0.0767176672610195
0.0615888955937969 -0.0878239387828349
0.0664585976910542 -0.0982801562147958
0.072085609900655 -0.10863675691435
0.0780571592834234 -0.118599914077354
0.08432516434307 -0.128134045396256
0.0908131418279783 -0.136908599720991
0.0979353085759198 -0.145456841571941
0.105635261341533 -0.153920420196278
0.113314668270683 -0.161252708922069
0.121626519720217 -0.168399394022092
0.130199888894095 -0.174833639592176
0.139250621245577 -0.180825566954727
0.149194628828992 -0.18645514023376
0.158780448148931 -0.191196354812289
0.168971315456279 -0.195397950905863
0.179708378760775 -0.199150050909943
0.19045426687616 -0.202081472168663
0.201668373287374 -0.204582922691799
0.213406497742117 -0.206472869228286
0.225121874030311 -0.207235496554042
0.237331823289546 -0.20798471560131
0.249822155919747 -0.207816417771425
0.262786827072768 -0.206841133619445
0.275758723840688 -0.205357544128999
0.288951756143576 -0.203342196331986
0.302258118704561 -0.200601801321886
0.316710130363563 -0.197087998921957
0.330746388986938 -0.192957733094927
0.34554437580787 -0.188212280913441
0.360708133789849 -0.182801467457379
0.375228275071718 -0.176923508666162
0.390282767095232 -0.170709825662335
0.405108136218587 -0.163840010634489
0.42109976226801 -0.156280976465882
0.436573141590362 -0.147809019543375
0.452063988189108 -0.139410920745875
0.46813231140413 -0.130648654008784
0.484585950401663 -0.121010874179629
0.500662408967092 -0.110673419499024
0.517274358043854 -0.100190409204962
0.534167962758998 -0.0893077986447352
0.550610601565416 -0.0777901609279178
0.567378381435418 -0.0667074341001863
0.583918585955594 -0.054701018996261
0.601204930188776 -0.042521053347433
0.61779393188959 -0.0299697736916889
0.635312389786556 -0.0175333854783752
0.652425968524568 -0.00422174438776487
0.669577078744047 0.00880297813329933
0.68642543439641 0.0217815482795204
0.70362399878802 0.0348638435875829
};
%\addlegendentry{Reference}
\addplot [line width = \linewidthEightB, color = KoopData, dashed]
table {%
0.75 0
0.767617956255685 0.0126152320166048
0.785221863675294 0.025202759744927
0.802800914680889 0.0377306780229099
0.820344353102624 0.0501672114827695
0.837841479099983 0.0624807937495562
0.855281653865309 0.0746401456242545
0.872654304077727 0.0866143520292023
0.889948926065816 0.0983729374930471
0.90715508962574 0.109885939954111
0.924262441427357 0.121123982665618
0.941260707923679 0.132058343994744
0.958139697658408 0.142661024921306
0.974889302841815 0.152904814062882
0.991499500036706 0.162763350083898
1.00796034976389 0.172211181389763
1.02426199480132 0.18122382306779
1.04039465691463 0.189777811119373
1.0563486317231 0.197850754138649
1.07211428137931 0.205421382738366
1.08768202473299 0.212469597210257
1.10304232467243 0.218976514139914
1.11818567241108 0.224924512976024
1.1331025686362 0.230297283873765
1.147783501695 0.23507987847157
1.16221892339532 0.239258765576722
1.17639922357808 0.242821893954846
1.19031470539574 0.245758764427774
1.20395556419321 0.248060513126141
1.21731187395941 0.249720006825457
1.23037358633393 0.250731949622618
1.24313054782987 0.251092997654816
1.2555725408717 0.250801875167454
1.26768935298946 0.24985948134918
1.27947087567813 0.248268973729967
1.29090722994702 0.246035811751358
1.30198890990496 0.243167744731957
1.31270692999354 0.2396747329547
1.32305295737637 0.235568799170581
1.33301941026506 0.230863819194334
1.34259950666846 0.225575271775738
1.3517872558846 0.219719976188812
1.36057739518197 0.213315848229081
1.36896528374308 0.206381700652654
1.37694677240494 0.198937103974783
1.38451806954475 0.191002311131823
1.39167562078374 0.182598238316676
1.39841601450881 0.173746486857157
1.4047359185686 0.164469388195659
1.41063204764278 0.154790055213092
1.41610115672778 0.144732426840967
1.42114005415556 0.134321297531182
1.42574562723652 0.123582327489247
1.42991487441761 0.112542032954869
1.43364493919855 0.101227758043132
1.43693314251624 0.089667630857217
1.4397770116222 0.0778905070053025
1.44217430452097 0.0659259035746353
1.44412302978876 0.0538039262643269
1.44562146208725 0.0415551919190076
1.44666815398553 0.0292107482394724
1.44726194486432 0.016801992026761
1.44740196775303 0.00436058696240971
1.44708765498222 -0.00808161836067393
1.44631874355061 -0.0204926716768206
1.4450952811249 -0.0328405972734807
1.44341763362016 -0.0450934751015305
1.44128649535024 -0.0572195188465802
1.4387029027798 -0.0691871532341023
1.43566825293285 -0.0809650910678666
1.4321843274821 -0.0925224107662305
1.42825332340439 -0.103828635462315
1.42387789076703 -0.114853815053745
1.41906117760902 -0.125568612881875
1.41380688088891 -0.135944398910193
1.40811930097384 -0.145953351233135
1.40200339507468 -0.155568567310703
1.39546482242266 -0.164764185292558
1.38850997106277 -0.173515513988717
1.38114595342609 -0.181799167392522
1.37338055620066 -0.189593196333819
1.36522213060654 -0.196877206377916
1.35667941319178 -0.203632448466232
1.34776127545165 -0.209841868311401
1.33847641259836 -0.215490103434114
1.328832995749 -0.220563423487838
1.3188383241811 -0.225049619377379
1.30849852098817 -0.228937857346307
1.29781831326193 -0.232218522442798
1.28680092623794 -0.234883078640104
1.27544810236402 -0.23692396914816
1.26376023613114 -0.238334571253006
1.25173659926596 -0.239109208409727
1.23937562223503 -0.239243211851919
1.22680212890194 -0.238807395220756
1.21393543612189 -0.237739020519201
1.2007824030094 -0.236042071113405
1.1873502195155 -0.233721342391173
1.17364639818837 -0.230782623136066
1.15967876862552 -0.227232843837933
1.14545546822952 -0.223080191387264
1.13098492732646 -0.218334192018305
1.11627584919941 -0.213005765794969
1.10133718666704 -0.207107256561536
1.08617811704058 -0.200652441341106
1.070808017043 -0.19365652287622
1.05523643885516 -0.186136108539406
1.03947308802467 -0.178109178316594
1.02352780361101 -0.169595044057188
1.00741054066936 -0.160614301730894
0.991131354992172 -0.151188778049092
0.974700389916618 -0.141341472499412
0.958127864949596 -0.131096495600034
0.941424065943487 -0.120479003995558
0.924599336561409 -0.109515132878708
0.90766407079076 -0.0982319261219058
0.890628706290982 -0.086657264431339
0.873503718391248 -0.074819791786618
0.856299614583182 -0.0627488403957748
0.839026929381251 -0.0504743543738914
0.8216962194482 -0.0380268123406879
0.804318058904672 -0.0254371491254941
0.786903034760972 -0.0127366767652624
0.769461742425004 4.29950186324069e-05
0.752004781254135 0.0128700386771497
0.734542750130382 0.0257124888331812
0.717086243048227 0.0385383222295833
0.699645844712938 0.0513155374272836
0.682232126154625 0.0640122340511521
0.664855640369735 0.0765966913776288
0.647526918007365 0.0890374460548427
0.630256463122704 0.101303368743397
0.613054749024055 0.113363739464377
0.595932214243 0.125188321440869
0.578899258658929 0.136747433220813
0.561966239808458 0.148012018873052
0.545143469406046 0.158953716055882
0.528441210092295 0.169544921769502
0.511869672407924 0.179758855622043
0.495439011959919 0.189569620465462
0.479159326695086 0.198952260295039
0.463040654116175 0.207882815357607
0.447092968153461 0.216338374482455
0.431326175222446 0.224297124738732
0.41575010873165 0.231738398637223
0.400374520922108 0.238642719233979
0.38520907038282 0.244991843656404
0.37026330284855 0.250768805749989
0.355546621900637 0.255957958716022
0.341068244918601 0.260545018739871
0.326837138059129 0.264517110634022
0.312861922218046 0.267862816346776
0.299150740017787 0.270572226688282
0.285711072187049 0.272636995643726
0.272549490830637 0.274050395018263
0.259671337879835 0.274807364778616
0.247080320574249 0.274904551350881
0.234778023333987 0.274340322593555
0.22282111677237 0.273147174350289
0.211212000441657 0.271312180989548
0.199956090718924 0.268842225658862
0.18905951088657 0.265745762806622
0.178528814558833 0.262032850875418
0.168370684504055 0.257715152729482
0.158591652631679 0.252805894649178
0.149197881311021 0.247319783843054
0.140195031151771 0.241272894158711
0.131588220396365 0.234682536784361
0.123382062255884 0.227567134929123
0.115580754025762 0.219946118404367
0.108188187942631 0.211839847297465
0.101208057557953 0.203269566144989
0.0946439418448079 0.19425738357691
0.088499358802492 0.184826268669802
0.0827777883280161 0.17500005435619
0.0774826693237491 0.164803439481848
0.0726173783628224 0.154261983466959
0.0681851974057714 0.143402090111052
0.0641892769319838 0.132250979303424
0.0606325991839522 0.12083664700672
0.0575179445429576 0.109187814856514
0.0548478626398547 0.0973338711778623
0.0526246487503009 0.0853048053160545
0.0508503253165833 0.0731311370573394
0.0495266280133111 0.0608438426868979
0.0486549955532945 0.0484742789692179
0.0482365623412967 0.036054106082026
0.0482721530701452 0.0236152103073717
0.0487622783751904 0.0111896270851532
0.0497071306928156 -0.00119053514168032
0.0511065794918781 -0.0134931700273918
0.0529601650583767 -0.0256862472791051
0.0552670900164109 -0.0377378855745846
0.0580262077740067 -0.0496164247880426
0.061236007111159 -0.0612904981681536
0.0648945922107272 -0.0727291054819839
0.0689996576140302 -0.0839016885812368
0.0735484579177227 -0.0947782113457813
0.0785377725810335 -0.10532924646507
0.0839638670453019 -0.11552607193497
0.0898224525211534 -0.125340780318524
0.0961086482561191 -0.134746403519034
0.10281695173701 -0.143717054765618
0.109941223832714 -0.152228087440402
0.117474696885182 -0.160256267121569
0.125410013591915 -0.167779948892829
0.133739302545796 -0.174779247148552
0.14245429208538 -0.181236180932355
0.151546457798383 -0.187134775860053
0.161007191589854 -0.192461105490089
0.170827973498639 -0.197203261519913
0.181000523658031 -0.201351252889932
0.191516912754772 -0.204896846530735
0.202369615510414 -0.207833373632873
0.213551501760898 -0.210155531524887
0.22505577088336 -0.211859210579139
0.236875844499746 -0.212941368432345
0.249005237340447 -0.213399962707235
0.261437426261433 -0.213233941776159
0.274165733616003 -0.212443283869614
0.28718323530924 -0.211029069576408
0.300482697779478 -0.208993571566475
0.314056543186631 -0.206340347198259
0.327896838839168 -0.203074323164878
0.341995305342881 -0.199201865241865
0.356343337731412 -0.194730829684218
0.370932034454633 -0.189670595476494
0.385752230119493 -0.184032078393973
0.400794528982751 -0.177827728806396
0.416049337197401 -0.171071515546711
0.431506892628147 -0.16377889817583
0.447157291657684 -0.15596678976561
0.462990512824106 -0.147653512015245
0.478996437395876 -0.138858744187699
0.495164867142796 -0.129603467046443
0.511485539633279 -0.119909902709476
0.527948141407569 -0.109801451123518
0.544542319363787 -0.0993026236948103
0.561257690663388 -0.0884389744880547
0.578083851424296 -0.0772370293146803
0.595010384429933 -0.0657242129685941
0.612026866043939 -0.0539287748255556
0.62912287248563 -0.0418797129959582
0.646287985590849 -0.029606697205851
0.663511798156965 -0.0171399905741775
0.680783918949166 -0.00451037045289776
0.698093977427272 0.00825095150093907
0.715431628237671 0.0211124098473716
0.732786555502968 0.0340421663000904
};
%\addlegendentry{Koopman data}
\addplot [line width = \linewidthEightB, color = RungeKutta, dashdotted]
table {%
0.75 0
0.767347219519284 0.012137253483567
0.784683394455509 0.0242435861667548
0.801997486582194 0.036288155514001
0.819278471531621 0.0482402761288358
0.836515345700615 0.0600694978228124
0.853697133135913 0.0717456830897325
0.870812892392119 0.0832390837898383
0.887851723355235 0.0945204168511733
0.904802774024683 0.105560938798497
0.921655247246804 0.116332518924018
0.938398407392915 0.126807710918802
0.955021586975309 0.136959822789028
0.971514193195148 0.146762984887382
0.987865714417124 0.156192215896701
1.0040657265673 0.165223486610677
1.02010389945277 0.173833781364724
1.03597000300535 0.182001156979107
1.05165391345605 0.189704799085743
1.06714561945411 0.196925075719347
1.08243522815337 0.203643588062188
1.09751297130037 0.209843218238558
1.1123692113737 0.215508174058654
1.12699444784162 0.220624030609957
1.14137932362429 0.225177768584601
1.1555146318663 0.22915780921071
1.16939132313901 0.232554045621132
1.18300051319534 0.235357870442502
1.19633349138035 0.237562199321834
1.20938172974852 0.23916149003295
1.22213689283924 0.240151756735333
1.23459084790774 0.240530578918402
1.24673567520343 0.24029710459028
1.25856367765532 0.239452047402427
1.27006738911946 0.237997677671498
1.28123958024499 0.235937807669022
1.29207326111455 0.233277772045945
1.30256168017249 0.230024404727089
1.31269831956044 0.226186013888674
1.32247688771958 0.221772356568193
1.33189131078413 0.216794613983565
1.34093572465289 0.211265367835258
1.34960446953272 0.205198576949498
1.35789208820684 0.198609552872807
1.36579332846363 0.191514932670796
1.37330314928634 0.18393264729003
1.38041672978698 0.175881884318464
1.38712947958844 0.167383044631953
1.39343704941026 0.158457693037802
1.39933534088867 0.149128503482261
1.40482051503024 0.139419199628507
1.40988899904863 0.129354491659788
1.41453749160152 0.118960010081206
1.41876296661194 0.108262237150103
1.42256267593454 0.0972884364110782
1.42593415113889 0.086066580678514
1.42887520465437 0.0746252787098322
1.43138393047621 0.0629937007473294
1.43345870458359 0.0512015030695605
1.43509818517756 0.0392787516771522
1.43630131281055 0.0272558452352367
1.43706731045357 0.0151634373994425
1.43739568352819 0.00303235866024866
1.43728621991849 -0.00910646215137196
1.43673898997073 -0.0212220765475008
1.43575434648365 -0.0332835951024502
1.43433292468914 -0.0452602660529673
1.43247564221903 -0.0571215536269705
1.43018369904855 -0.0688372159533007
1.42745857739782 -0.0803773824121825
1.4243020415584 -0.0917126302942692
1.42071613759078 -0.102814060643163
1.41670319280875 -0.113653373158817
1.41226581492764 -0.124202940032363
1.40740689070718 -0.134435878560225
1.40212958387022 -0.144326122339201
1.3964373320371 -0.153848490767307
1.39033384239815 -0.162978756463708
1.38382308587764 -0.171693710079292
1.37690928964889 -0.179971221816331
1.36959692806383 -0.187790298849553
1.36189071236441 -0.195131137800147
1.35379557991389 -0.201975171526458
1.34531668404164 -0.20830510981277
1.33645938581161 -0.214104974062127
1.32722924897007 -0.219360126749639
1.31763203892423 -0.224057297001799
1.30767372588241 -0.22818460402503
1.2973604914176 -0.231731580045289
1.28669873695777 -0.234689193904312
1.27569509230234 -0.237049875620184
1.26435642233521 -0.238807541316647
1.25268983059913 -0.239957617226964
1.24070265912099 -0.240497061156197
1.22840248459827 -0.24042437986202
1.21579711159144 -0.239739641182038
1.20289456363565 -0.238444480231447
1.1897030732081 -0.236542099469318
1.17623107134149 -0.234037262792463
1.16248717745014 -0.230936284032406
1.14848018970572 -0.227247010319085
1.13421907610845 -0.222978800771493
1.11971296626323 -0.218142500920566
1.10497114378721 -0.212750413195198
1.09000303923492 -0.206816263729265
1.07481822341698 -0.200355165686855
1.05942640099553 -0.193383579258294
1.04383740425637 -0.185919268450478
1.02806118697728 -0.177981254778958
1.01210781833129 -0.169589767962932
0.995987476780275 -0.160766193724954
0.979710443928391 -0.151533018802209
0.963287098315684 -0.141913773283855
0.946727909140567 -0.131932970397911
0.930043429905878 -0.121616043880467
0.91324429198767 -0.110989283069269
0.896341198128817 -0.100079765872392
0.879344915861624 -0.0889152897708354
0.862266270864871 -0.0775243010211479
0.845116140261585 -0.0659358222307342
0.827905445864243 -0.0541793784842489
0.810645147374399 -0.042284922204474
0.793346235543765 -0.0302827569353333
0.776019725303823 -0.0182034602382608
0.758676648870971 -0.00607780589600706
0.74132804883419 0.00606331437980999
0.723984971232126 0.0181889695395348
0.706658458626486 0.0302682678858693
0.689359543178601 0.0422704356646973
0.672099239736024 0.0541648953575106
0.654888538936098 0.0659213434775336
0.637738400333409 0.077509827673207
0.620659745558167 0.0889008229448788
0.603663451512546 0.10006530678342
0.586760343612044 0.110974833042994
0.569961189078862 0.121601604364455
0.55327669029407 0.131918542970798
0.536717478214957 0.141899359661781
0.520294105863188 0.151518620841353
0.504017041888231 0.160751813418715
0.487896664208613 0.169575407431921
0.471943253730781 0.177966916251536
0.456166988141264 0.18590495423109
0.440577935762111 0.193369291680399
0.425186049451712 0.200340907046881
0.410001160522753 0.206802036197834
0.395032972635676 0.212736218702091
0.38029105560979 0.218128341010777
0.36578483907538 0.222964676431608
0.351523605870586 0.227232921776542
0.337516485069478 0.23092223053541
0.323772444518238 0.234023242385741
0.310300282763089 0.236528108790522
0.297108620287721 0.238430514364134
0.284205890052545 0.239725693611976
0.27160032745433 0.240410442590784
0.259299960006205 0.240483125024951
0.247312597261469 0.239943672488383
0.2356458217318 0.238793578459958
0.224306981713199 0.237035886402939
0.21330318694315 0.234675172482025
0.202641307788403 0.231717524032126
0.192327978173203 0.228170515287301
0.1823696017639 0.22404318200088
0.172772360193836 0.219345996319791
0.163542221572307 0.214090842618848
0.154684947376684 0.208290994107505
0.146206096155645 0.201961089155653
0.138111023180157 0.195117105712544
0.130404876038488 0.187776332069152
0.12309258692145 0.179957332525672
0.116178862798813 0.171679907117356
0.109668174793245 0.162965045212245
0.10356474787866 0.153834873345794
0.0978725516915446 0.144312598005936
0.0925952928735381 0.134422444219678
0.0877364090495248 0.124189590767097
0.0832990643281257 0.113640102728524
0.0792861460924091 0.102800861916777
0.0757002628080311 0.0916994956001386
0.0725437425870994 0.0803643038046822
0.069818632284583 0.0688241854024739
0.0675266969522104 0.0571085631419612
0.0656694195214907 0.0452473077513642
0.0642480006271671 0.0332706612373471
0.0632633585132331 0.0212091595029022
0.0627161289858991 0.00909355441514592
0.0626066653929575 -0.00304526453801165
0.0629350386185861 -0.0151763488558262
0.0637010370885092 -0.0272687697454203
0.0649041667842718 -0.0392916968007361
0.0665436512687222 -0.0512144764858195
0.0686184317292081 -0.0630067102720941
0.0711271670520408 -0.0746383322854955
0.0740682339532131 -0.0860796863272447
0.0774397272079407 -0.0973016021399556
0.0812394600470219 -0.108275470796062
0.0854649648223936 -0.118973319084073
0.0901134940873028 -0.12936788275431
0.0951820222852561 -0.139432678452301
0.100667248289028 -0.149142074107302
0.10656559906303 -0.158471357449301
0.112873234718821 -0.167396802199979
0.119586055168304 -0.175895731332128
0.126699708426136 -0.183946576646328
0.13420960035749 -0.191528933821879
0.142110905322839 -0.198623612126504
0.150398576795015 -0.205212678179984
0.159067356720643 -0.211279493592426
0.16811178230224 -0.216808746901301
0.177526189102462 -0.221786480886398
0.187304709947427 -0.226200116851138
0.197441269929379 -0.230038477621279
0.207929578656845 -0.233291810717099
0.218763121506754 -0.235951812451675
0.229935151803763 -0.238011652801433
0.241438685552319 -0.239466000064469
0.253266499708268 -0.240311043799429
0.265411134226006 -0.240544514421553
0.277864897472831 -0.240165698073282
0.290619874197545 -0.23917544583971
0.303667935101709 -0.237576176879925
0.317000747134867 -0.235371875470324
0.330609783830706 -0.232568082244952
0.344486335235128 -0.229171880066449
0.358621517191145 -0.225191874998346
0.373006279910045 -0.220638172815656
0.387631415867 -0.215522351422736
0.402487567119295 -0.209857429471725
0.417565232169048 -0.203657831407191
0.432854772492479 -0.196939349109845
0.448346418844808 -0.189719100275549
0.46403027743119 -0.182015483643688
0.47989633601438 -0.173848131178146
0.495934470011695 -0.165237857301697
0.512134448618413 -0.156206605287707
0.528485940982374 -0.146777390919585
0.544978522444945 -0.136974243536887
0.561601680856407 -0.126822144596177
0.578344822968638 -0.116346963884112
0.595197280904322 -0.105575393529208
0.61214831869946 -0.094534879967153
0.629187138914278 -0.0832535540222168
0.646302889306651 -0.071760159274263
0.663484669561462 -0.0600839788869683
0.680721538069075 -0.0482547610782518
0.698002518745858 -0.0363026434185232
0.715316607889725 -0.0242580761462783
0.732652781063624 -0.012151744693772
};
%\addlegendentry{Runge-Kutta}
\addplot [line width = \linewidthEightB, color = KoopFP, dashed]
table {%
0.75 0
0.767342738670302 0.0121288345850962
0.784674176976675 0.0242272163576458
0.801983260469178 0.0362642794681834
0.819258969703945 0.048209287734508
0.836490327219107 0.0600317117741455
0.853666404394351 0.071701305117237
0.870776328159247 0.0831881790914687
0.887809287502988 0.0944628762709964
0.904754539723828 0.105496442283927
0.92160141633949 0.116260495778608
0.938339328559869 0.12672729635879
0.954957772199854 0.136869810313116
0.971446331882787 0.146661773987267
0.987794684353839 0.156077754679941
1.00399260068791 0.165093208989919
1.02002994713962 0.173684538604677
1.03589668434616 0.181829143606327
1.05158286456114 0.189505473483398
1.06707862657713 0.1966930761832
1.08237418799538 0.203372645724279
1.09745983454175 0.209526069115235
1.11232590622732 0.215136473593235
1.12696278034247 0.220188275492847
1.1413608515871 0.224667232358352
1.15551051011769 0.22856050017352
1.16940211896693 0.231856697724189
1.1830259931781 0.234545980015069
1.19637238406804 0.236620122178272
1.20943147318642 0.23807261425554
1.22219338156418 0.238898765437726
1.23464820039506 0.239095813713015
1.2467860488994 0.238663033506209
1.25859716326395 0.237601830191938
1.27007201684762 0.235915807141483
1.281201466301 0.233610789390237
1.29197691154688 0.230694789388857
1.30239045119216 0.227177905560569
1.31243501096052 0.223072153486699
1.32210442316316 0.218391241015121
1.33139344108689 0.21315030962957
1.34029768274901 0.207365671711406
1.34881351118412 0.201054574342825
1.3569378697233 0.194235014477784
1.36466809744649 0.186925619381062
1.37200175064236 0.179145593514906
1.37893645130629 0.170914722022894
1.38546977561529 0.162253413998033
1.39159918659252 0.153182766508132
1.39732200794372 0.143724632094479
1.40263543140952 0.133901676593129
1.407536548018 0.123737419037599
1.41202239377462 0.113256249892277
1.41609000177855 0.102483427303731
1.41973645475976 0.0914450532864847
1.42295893404936 0.0801680329249064
1.4257547627244 0.0686800200476554
1.42812144198838 0.0570093527011574
1.43005668076427 0.0451849813495401
1.43155841905236 0.0332363922269635
1.43262484592759 0.021193527766854
1.43325441320077 0.0090867055840526
1.43344584581526 -0.0030534628906674
1.43319815004516 -0.015196153330606
1.43251062053873 -0.0273104084439714
1.43138284723314 -0.0393652165833242
1.42981472316585 -0.0513295888962859
1.42780645422398 -0.0631726350805876
1.42535857190042 -0.0748636380342393
1.4224719501453 -0.08637212796684
1.4191478273837 -0.0976679568772265
1.41538783467046 -0.108721374714102
1.41119403070496 -0.119503109013028
1.40656894395159 -0.129984450311975
1.40151562130696 -0.140137346114384
1.39603768152715 -0.14993450646219
1.39013936991326 -0.159349524100437
1.38382560856843 -0.168357011491619
1.37710203405511 -0.176932755269084
1.36997501189179 -0.185053885851805
1.36245161569934 -0.192699055819563
1.35453955882701 -0.19984861559553
1.34624706888584 -0.206484769886453
1.33758270139734 -0.212591694659011
1.32855509755771 -0.218155593985356
1.31917270162364 -0.223164680398353
1.30944346325013 -0.227609071888021
1.29937455637369 -0.231480612020361
1.28897214664191 -0.234772633778052
1.27824123330375 -0.237479698647542
1.26718558035488 -0.239597346902704
1.25580773862524 -0.241121891712858
1.24410914879423 -0.242050279935308
1.23209030747879 -0.242380029417724
1.21975097539385 -0.242109239980754
1.20711556430166 -0.241255258917613
1.19420807159852 -0.23980529036844
1.18103013948123 -0.237763244151176
1.16758488772436 -0.235133970133518
1.15387671063272 -0.231923407717805
1.13991109283931 -0.22813870556363
1.12569444514975 -0.223788311507012
1.11123396062323 -0.218882034577034
1.09653749020947 -0.213431082133163
1.08161343661007 -0.207448075610522
1.06647066461707 -0.200947048363665
1.05111842597905 -0.193943428822559
1.03556629680941 -0.186454011758584
1.01982412563506 -0.178496920000781
1.00390199034022 -0.170091558503138
0.987810162455139 -0.161258562273843
0.971559077446063 -0.152019739349729
0.955159309864849 -0.142398009734244
0.938621552402843 -0.132417341009921
0.921956598059664 -0.122102681178124
0.905175324781305 -0.111479889161116
0.888288682044005 -0.100575663315877
0.871307678962372 -0.0894174682483654
0.854243373584511 -0.078033460175092
0.837106863105818 -0.0664524110511631
0.819909274789156 -0.0547036316664789
0.802661757424395 -0.0428168939017978
0.785375473196819 -0.0308223523315758
0.768061589863252 -0.0187504653593152
0.750731273158479 -0.00663191607236588
0.733395679373645 0.00550246699414097
0.716065948064014 0.0176217889909566
0.698753194856323 0.0296951675887053
0.681468504336936 0.0416918113109007
0.664222923011353 0.0535810971609196
0.647027452334105 0.0653326473346074
0.629893041815786 0.0769164048077401
0.612830582221448 0.0883027075861983
0.595850898881817 0.0994623614068735
0.578964745145982 0.110366710679472
0.562182796011185 0.120987707464294
0.545515641971764 0.131297978289573
0.528973783134485 0.141270888625386
0.512567623650034 0.150880604850975
0.496307466508432 0.160102153580588
0.480203508735915 0.168911478252033
0.464265837007601 0.177285492934774
0.448504423646183 0.185202133383527
0.432929122900956 0.1926404054515
0.41754966727801 0.199580431086261
0.402375663499665 0.206003492259465
0.387416587380347 0.211892073323555
0.37268177648025 0.217229902429504
0.358180418793089 0.222001992752159
0.343921534891792 0.226194684306691
0.329913949852297 0.229795687027749
0.316166249876807 0.232794125418047
0.30268671686715 0.235180584320286
0.289483232365095 0.236947154074381
0.276563140523431 0.238087471358879
0.263933058527119 0.238596749336966
0.25159862278447 0.238471787480646
0.239586232720238 0.237728401143018
0.227925319978886 0.23635471193537
0.216619845596333 0.234356902296625
0.205674061418761 0.231743054457713
0.195092394378097 0.228523098132604
0.184879305229413 0.224708703833585
0.175039145884716 0.220313123178963
0.165576040373081 0.215350988421937
0.1564938094083 0.209838093062401
0.147795948560348 0.203791180617961
0.139485657937308 0.197227767466947
0.131565910677059 0.190166018362055
0.124039541302297 0.182624682103466
0.116909334208419 0.174623083378632
0.110178096396864 0.166181157975776
0.103848704965701 0.157319514099719
0.0979241265525636 0.148059502356991
0.092407411232432 0.138423279952698
0.0873016665708271 0.128433859148491
0.0826100186886249 0.11811513466025
0.0783355668352908 0.107491888524355
0.0744813367448362 0.0965897736609872
0.0710502365205892 0.0854352789284506
0.0680450173349473 0.0740556791189423
0.0654682400320748 0.0624789733806443
0.0633222478437596 0.0507338152244627
0.0616091448491177 0.0388494367839646
0.0603307794679407 0.0268555694734292
0.0594887321069013 0.0147823627040401
0.0590843060158269 0.00266030190380961
0.0591185204088151 -0.00947987325169962
0.059592104926776 -0.0216072497629538
0.0605054945408478 -0.0336908209726259
0.0618588240070401 -0.0456995647575687
0.0636519209769469 -0.0576025202183408
0.0658842968509189 -0.0693688630419502
0.0685551344405922 -0.0809679799535469
0.0716632715090077 -0.0923695429672175
0.0752071793129297 -0.103543584507428
0.0791849354323853 -0.114460574902875
0.0835941905018018 -0.125091504235846
0.0884321290337197 -0.135407971013537
0.0936954254314924 -0.145382280517536
0.0993801975844711 -0.154987555827692
0.105481962132903 -0.164197864184309
0.111995597471148 -0.172988360274389
0.118915322538167 -0.181335445933074
0.126234700905851 -0.18921694248102
0.133946679883267 -0.196612267584923
0.142043672485031 -0.203502603693347
0.150517685540669 -0.209871040876738
0.159360489935761 -0.215702674852652
0.168563819958609 -0.220984642706575
0.178119580057749 -0.225706085268203
0.188020031740514 -0.229858035828292
0.198257933292526 -0.233433247722101
0.208826611428202 -0.236425984835705
0.219719955663284 -0.238831805833242
0.230932339979276 -0.240647372778785
0.242458488363713 -0.241870308057703
0.254293308030976 -0.24249911244598
0.266431715413338 -0.242533145222284
0.278868476198419 -0.241972657409315
0.29159807386582 -0.240818863306619
0.30461461366347 -0.239074033664189
0.317911762531434 -0.236741595264559
0.331482721024548 -0.233826224983836
0.345320220919533 -0.230333930318834
0.359416541539352 -0.22627211199741
0.37376353832131 -0.221649607160015
0.388352678254857 -0.216476713583442
0.403175078109507 -0.210765196597266
0.418221542603658 -0.204528280899378
0.433482600700438 -0.197780629604959
0.448948539015497 -0.190538312732279
0.464609431894666 -0.182818767065665
0.480455168102412 -0.174640749026493
0.496475474297996 -0.166024281878713
0.512659935605586 -0.156990598323564
0.528998013641088 -0.147562079309727
0.545479062368139 -0.137762189701873
0.56209234213737 -0.127615411308515
0.578827032229807 -0.117147173663459
0.595672242185659 -0.106383782877481
0.612617022159029 -0.095352348821972
0.629650372500424 -0.0840807108689903
0.646761252733805 -0.0725973623879606
0.663938590064038 -0.0609313741847298
0.681171287523945 -0.0491123170610559
0.698448231847461 -0.0371701836698217
0.715758301136157 -0.0251353098417733
0.733090372370179 -0.0130382955622058
};
%\addlegendentry{Koopman first principle}
\end{axis}

\end{tikzpicture}
%
    %\input{figures/tikz/mixed_b2_full}%
    \caption{Comparison of the surrogate models using~$B_1$ and~$B_2$ with models based on first principles and on data generated from a first principles model.}
    \label{fig: mixed}%
\end{figure}

From now on, due to the beneficial performance, if not stated otherwise, we employ the basis~$B_2$ with the observables~$\mathbb{O}_{11}$ and compare the prediction performance of the corresponding surrogate model with the nominal model and experiment runs.  
In particular, we include~$15$ experiment runs for each scenario since subsequent experiment realizations generally do not yield identical results due to disturbances, meaning that perfect prediction performance is impossible. 
Results for the~$\infty$-trajectory and for the square-shaped trajectory are plotted in Fig.~\ref{fig: data B2 ref8}. 
From the trajectory plots in the upper part, it becomes evident that the Koopman-based prediction outperforms the nominal model, better representing the systematic skewedness of the physical robot's trajectories. 
Similarly, in the lower part, the minimum, maximum and average Euclidean norms of the error  
between Koopman-based prediction and the family of hardware robot trajectories show that prediction quality is consistently good. 
    
\definecolor{green}{RGB}{0,128,0}
\definecolor{Koopman}{RGB}{255,0,0}
\definecolor{RungeKutta}{RGB}{0,128,0}
\definecolor{reference}{RGB}{0,0,255}
\def\linewidthEightC{\linewidthEight}
\def\linewidthErrorC{1.0}
\def\linewidthErrorStdVar{0.8}
\definecolor{gray}{RGB}{128,128,128}
\definecolor{orange}{RGB}{255,165,0}
\definecolor{MaxError}{RGB}{0,0,0255}
%\definecolor{AvgErr}{RGB}{255,165,0}
\definecolor{AvgErr}{RGB}{44,160,44}%
\def\opacityRef{0.2}
\def\heightRealData{4.6cm}
\def\heightRealDataError{5.33cm}
\begin{figure}%[h!]
    \centering%
    % This file was created with tikzplotlib v0.10.1.
\begin{tikzpicture}
\begin{axis}[
width = \textwidth, % scaled down due to axis equal image
height = \heightRealData,
axis equal image=true,
at={(0.0, 0)},
legend cell align={left},
legend columns = 3,
legend style={
  fill opacity=1,
  draw opacity=1,
  text opacity=1,
  at={(0.1415,1.05)},
  anchor=south west,
  column sep = 0.25cm
  %draw=lightgray204
},
%tick align=outside,
%tick pos=left,
%x grid style={darkgray176},
xlabel = {$x_1$ position (m)}, 
ylabel = {$x_2$ position (m)},
xmin=-0, xmax=1.5,
%xticklabel style={xshift=-.05cm},
ylabel style={yshift=-.1cm},
%xtick style={color=black},
%y grid style={darkgray176},
ymin=-0.3, ymax=0.4,
ylabel style={yshift=-.1cm},
%ytick style={color=black}
]
\addplot [line width = \linewidthEightC, color = reference, opacity=\opacityRef]
table {%
0.75 0
0.753796775174918 0.00275082027626007
0.770253583658299 0.0146086690284934
0.788958864934086 0.0278626298724842
0.805758023793338 0.0400302499510602
0.823931007064239 0.0531532313053401
0.84106617676365 0.0651272462437049
0.858337230948657 0.0771219912744552
0.875920686143595 0.0890536810683165
0.892999619106597 0.100689025982836
0.910268216607307 0.111810868722874
0.927155369485107 0.122692123646889
0.944271947890646 0.133248126872312
0.961126069264421 0.143627188027739
0.977642979860227 0.15383448530992
0.994383897910681 0.163352502484412
1.01044548809055 0.172451162379304
1.02668443483368 0.181312738972183
1.04256818618544 0.189747796838266
1.05849129738013 0.197535145736718
1.07411177001412 0.204666427354084
1.08960910037061 0.21157055004819
1.10495636876448 0.217634415202293
1.11972672620237 0.223091919572481
1.13464406954527 0.228115743673971
1.14898414845007 0.232579183288128
1.16313524000647 0.236048403043336
1.17731712629478 0.239498375097147
1.19115277544504 0.241987916087121
1.20475078691893 0.24357341141794
1.2179051788757 0.245202814841758
1.23101137495675 0.245758006855585
1.24362380516706 0.245734565070592
1.25596427000739 0.245035586045094
1.26780296261423 0.243915690097645
1.27942969766319 0.242067323137536
1.29076669615838 0.23966799166844
1.30144802323049 0.236506712301561
1.31206798872445 0.232653669484729
1.32183779339603 0.228592384801218
1.33177828963827 0.223548734377477
1.34106333859447 0.218129184320505
1.35015693385091 0.212083593544392
1.35888263509532 0.205514426208893
1.36732721401058 0.198237703677997
1.37490374677743 0.190581538771692
1.38255513625004 0.18240019011285
1.38961325950765 0.173826808095369
1.39606593780378 0.16462031674418
1.40209413268775 0.155475013292566
1.40766144142924 0.145817165049195
1.41298972922002 0.135430711689483
1.4177980383817 0.125150704222236
1.42217530643287 0.113940480045831
1.42605722005813 0.103059482016292
1.42956849107093 0.0918251912690216
1.43257333591272 0.0803632587732614
1.43525005723557 0.0680997321441579
1.43747354279763 0.056305557615111
1.43903200164914 0.0443137666138458
1.44057442009313 0.0316776217651214
1.4413510801794 0.0194597121678604
1.44173985473391 0.00685847184277516
1.44175304710854 -0.00526920267813812
1.44104028644395 -0.0178002690956425
1.44009964222474 -0.0300979887959264
1.43873929662625 -0.0420597707266236
1.43683527860183 -0.0542540613878999
1.43454414915032 -0.0661993618164278
1.43178063868126 -0.07793275011297
1.42860939001972 -0.0902036039954828
1.42497929453919 -0.101367767761983
1.42084578597473 -0.112995734074035
1.41640300826551 -0.124306882350474
1.41152032108025 -0.134729554563296
1.40607079201313 -0.144853166840076
1.40030595459819 -0.154610482940984
1.39414354790019 -0.164024485370985
1.38748427065227 -0.173231382393035
1.38064408729499 -0.181896666884964
1.37331841508833 -0.190087400057888
1.36519484608907 -0.19812860388081
1.35697623393093 -0.205033044251742
1.34840153732322 -0.211514626079049
1.33928113004562 -0.21763730663906
1.33004900187448 -0.222735976435501
1.32028685217834 -0.227468875379833
1.31019782424788 -0.231938597992863
1.29972367023761 -0.235445423517443
1.2887328899364 -0.238377260466694
1.2774754396843 -0.240988537620381
1.26622519835259 -0.242443616017929
1.25423811689992 -0.243485353393633
1.24204165110139 -0.243984107342374
1.22950006048786 -0.24384429208382
1.21680621065999 -0.242965802234227
1.20378526795112 -0.241685128168492
1.19022085168209 -0.239581051822156
1.17642516281212 -0.236929456370965
1.16246464426886 -0.233589810942679
1.14828949361601 -0.229559269769245
1.134244114945 -0.225021139127758
1.1193916826543 -0.21962483904429
1.10483500153046 -0.21375219196959
1.08949301887148 -0.207426340723266
1.07427915545535 -0.200272296491305
1.05865073551656 -0.193145164392875
1.04282953663211 -0.185027373494884
1.02697988363446 -0.176541280579012
1.01101989520926 -0.167639400673986
0.994823684556384 -0.157985107306108
0.978611547483571 -0.148252113890068
0.962245616000544 -0.13813228039391
0.945289138613748 -0.127489288913879
0.92866688816798 -0.11678880795474
0.911699999758336 -0.105312068616864
0.894785005680457 -0.0934958016648549
0.877304752830891 -0.0810753361113209
0.859867338507263 -0.0690099709019099
0.842552953393209 -0.0565972164378705
0.825243828042276 -0.0441147310471777
0.807750702555605 -0.0312097420590962
0.790618156787901 -0.018184012181872
0.773448995829779 -0.00598021264680999
0.756125635736799 0.00675405645469632
0.738662857324652 0.0200698846503974
0.72119675701568 0.0334214396986119
0.703874584573642 0.0461723698235121
0.686417632616456 0.0593272292403158
0.669268676231481 0.0717358909373415
0.651945408052231 0.0846213768862322
0.634658611133318 0.0966457356254571
0.61766691011985 0.10929461111541
0.600962770468281 0.120736890028959
0.583916214443415 0.132609432334643
0.566703364898398 0.143355113283038
0.549526794264573 0.154919069656636
0.53298265702123 0.165684786365669
0.516280586189622 0.176441724076514
0.499469657201165 0.186297143282059
0.483143366453001 0.196123708388658
0.466547232727129 0.205095875426217
0.4503128526377 0.214073094476273
0.434499592218361 0.222468217516916
0.418779066207624 0.230542812331306
0.403243011226651 0.237473628045056
0.387793402240573 0.244479897043496
0.372395287160294 0.251032532682219
0.357362958539635 0.256485837349139
0.342959940828114 0.261406488246182
0.32815830763551 0.266208681304595
0.313283646731331 0.270303645502696
0.299008764566356 0.273703922777576
0.285596019227139 0.276229464479511
0.271667658663907 0.278876844869545
0.258391825439994 0.280148293958511
0.245954683337346 0.281045899586786
0.233297043080318 0.281166438800653
0.221065882947653 0.280435278965782
0.209213960006697 0.2792423328193
0.197711313341009 0.27745153093713
0.186467060395309 0.274933700120144
0.175647337802095 0.271997391117996
0.165485819498891 0.268414334553556
0.154920655621453 0.264120951359659
0.14557267250259 0.259464577309265
0.1359054564482 0.253809489621943
0.127182186643735 0.247898300503122
0.118715498185119 0.241484242043756
0.110599614202892 0.234627650065451
0.103066671467977 0.227209711390963
0.0953756606293399 0.218714717118634
0.0883116077532903 0.210301099842098
0.0816862019022073 0.201563439530642
0.0754303047598111 0.192040200227576
0.0696280035789295 0.182162510473038
0.0642727671113544 0.171854125577518
0.0592887993296431 0.161295496791183
0.0550401350221935 0.150292025379403
0.0509807490685192 0.139045437299692
0.0473874463212024 0.127624872627466
0.0442485567154118 0.115876687414116
0.04167948049022 0.104035118062325
0.0394988342888216 0.0921296980175037
0.0379006039360912 0.0799267762717973
0.0364917112037286 0.0678905527750915
0.0358931555007101 0.0555642260921366
0.0355289500473749 0.0429731447797864
0.0356926207873237 0.0304970528307748
0.0362636666860241 0.0179069982691894
0.0373684888681004 0.00563301356630583
0.0387615798631966 -0.00676904035650516
0.0406380369959789 -0.0189102629938579
0.0429542857015246 -0.0308705294137852
0.0457203908464014 -0.042746116689976
0.049035766503309 -0.0543490855274337
0.052763997030933 -0.0653774059979454
0.0569627461396097 -0.0767176672610195
0.0615888955937969 -0.0878239387828349
0.0664585976910542 -0.0982801562147958
0.072085609900655 -0.10863675691435
0.0780571592834234 -0.118599914077354
0.08432516434307 -0.128134045396256
0.0908131418279783 -0.136908599720991
0.0979353085759198 -0.145456841571941
0.105635261341533 -0.153920420196278
0.113314668270683 -0.161252708922069
0.121626519720217 -0.168399394022092
0.130199888894095 -0.174833639592176
0.139250621245577 -0.180825566954727
0.149194628828992 -0.18645514023376
0.158780448148931 -0.191196354812289
0.168971315456279 -0.195397950905863
0.179708378760775 -0.199150050909943
0.19045426687616 -0.202081472168663
0.201668373287374 -0.204582922691799
0.213406497742117 -0.206472869228286
0.225121874030311 -0.207235496554042
0.237331823289546 -0.20798471560131
0.249822155919747 -0.207816417771425
0.262786827072768 -0.206841133619445
0.275758723840688 -0.205357544128999
0.288951756143576 -0.203342196331986
0.302258118704561 -0.200601801321886
0.316710130363563 -0.197087998921957
0.330746388986938 -0.192957733094927
0.34554437580787 -0.188212280913441
0.360708133789849 -0.182801467457379
0.375228275071718 -0.176923508666162
0.390282767095232 -0.170709825662335
0.405108136218587 -0.163840010634489
0.42109976226801 -0.156280976465882
0.436573141590362 -0.147809019543375
0.452063988189108 -0.139410920745875
0.46813231140413 -0.130648654008784
0.484585950401663 -0.121010874179629
0.500662408967092 -0.110673419499024
0.517274358043854 -0.100190409204962
0.534167962758998 -0.0893077986447352
0.550610601565416 -0.0777901609279178
0.567378381435418 -0.0667074341001863
0.583918585955594 -0.054701018996261
0.601204930188776 -0.042521053347433
0.61779393188959 -0.0299697736916889
0.635312389786556 -0.0175333854783752
0.652425968524568 -0.00422174438776487
0.669577078744047 0.00880297813329933
0.68642543439641 0.0217815482795204
0.70362399878802 0.0348638435875829
0.720901407938547 0.0482750034504682
};%
\addlegendentry{experiments}
\addplot [line width = \linewidthEightC, color = Koopman, dashed]
table {%
0.75 0
0.767617956255685 0.0126152320166048
0.785221863675294 0.025202759744927
0.802800914680889 0.0377306780229099
0.820344353102624 0.0501672114827695
0.837841479099983 0.0624807937495562
0.855281653865309 0.0746401456242545
0.872654304077727 0.0866143520292023
0.889948926065816 0.0983729374930471
0.90715508962574 0.109885939954111
0.924262441427357 0.121123982665618
0.941260707923679 0.132058343994744
0.958139697658408 0.142661024921306
0.974889302841815 0.152904814062882
0.991499500036706 0.162763350083898
1.00796034976389 0.172211181389763
1.02426199480132 0.18122382306779
1.04039465691463 0.189777811119373
1.0563486317231 0.197850754138649
1.07211428137931 0.205421382738366
1.08768202473299 0.212469597210257
1.10304232467243 0.218976514139914
1.11818567241108 0.224924512976024
1.1331025686362 0.230297283873765
1.147783501695 0.23507987847157
1.16221892339532 0.239258765576722
1.17639922357808 0.242821893954846
1.19031470539574 0.245758764427774
1.20395556419321 0.248060513126141
1.21731187395941 0.249720006825457
1.23037358633393 0.250731949622618
1.24313054782987 0.251092997654816
1.2555725408717 0.250801875167454
1.26768935298946 0.24985948134918
1.27947087567813 0.248268973729967
1.29090722994702 0.246035811751358
1.30198890990496 0.243167744731957
1.31270692999354 0.2396747329547
1.32305295737637 0.235568799170581
1.33301941026506 0.230863819194334
1.34259950666846 0.225575271775738
1.3517872558846 0.219719976188812
1.36057739518197 0.213315848229081
1.36896528374308 0.206381700652654
1.37694677240494 0.198937103974783
1.38451806954475 0.191002311131823
1.39167562078374 0.182598238316676
1.39841601450881 0.173746486857157
1.4047359185686 0.164469388195659
1.41063204764278 0.154790055213092
1.41610115672778 0.144732426840967
1.42114005415556 0.134321297531182
1.42574562723652 0.123582327489247
1.42991487441761 0.112542032954869
1.43364493919855 0.101227758043132
1.43693314251624 0.089667630857217
1.4397770116222 0.0778905070053025
1.44217430452097 0.0659259035746353
1.44412302978876 0.0538039262643269
1.44562146208725 0.0415551919190076
1.44666815398553 0.0292107482394724
1.44726194486432 0.016801992026761
1.44740196775303 0.00436058696240971
1.44708765498222 -0.00808161836067393
1.44631874355061 -0.0204926716768206
1.4450952811249 -0.0328405972734807
1.44341763362016 -0.0450934751015305
1.44128649535024 -0.0572195188465802
1.4387029027798 -0.0691871532341023
1.43566825293285 -0.0809650910678666
1.4321843274821 -0.0925224107662305
1.42825332340439 -0.103828635462315
1.42387789076703 -0.114853815053745
1.41906117760902 -0.125568612881875
1.41380688088891 -0.135944398910193
1.40811930097384 -0.145953351233135
1.40200339507468 -0.155568567310703
1.39546482242266 -0.164764185292558
1.38850997106277 -0.173515513988717
1.38114595342609 -0.181799167392522
1.37338055620066 -0.189593196333819
1.36522213060654 -0.196877206377916
1.35667941319178 -0.203632448466232
1.34776127545165 -0.209841868311401
1.33847641259836 -0.215490103434114
1.328832995749 -0.220563423487838
1.3188383241811 -0.225049619377379
1.30849852098817 -0.228937857346307
1.29781831326193 -0.232218522442798
1.28680092623794 -0.234883078640104
1.27544810236402 -0.23692396914816
1.26376023613114 -0.238334571253006
1.25173659926596 -0.239109208409727
1.23937562223503 -0.239243211851919
1.22680212890194 -0.238807395220756
1.21393543612189 -0.237739020519201
1.2007824030094 -0.236042071113405
1.1873502195155 -0.233721342391173
1.17364639818837 -0.230782623136066
1.15967876862552 -0.227232843837933
1.14545546822952 -0.223080191387264
1.13098492732646 -0.218334192018305
1.11627584919941 -0.213005765794969
1.10133718666704 -0.207107256561536
1.08617811704058 -0.200652441341106
1.070808017043 -0.19365652287622
1.05523643885516 -0.186136108539406
1.03947308802467 -0.178109178316594
1.02352780361101 -0.169595044057188
1.00741054066936 -0.160614301730894
0.991131354992172 -0.151188778049092
0.974700389916618 -0.141341472499412
0.958127864949596 -0.131096495600034
0.941424065943487 -0.120479003995558
0.924599336561409 -0.109515132878708
0.90766407079076 -0.0982319261219058
0.890628706290982 -0.086657264431339
0.873503718391248 -0.074819791786618
0.856299614583182 -0.0627488403957748
0.839026929381251 -0.0504743543738914
0.8216962194482 -0.0380268123406879
0.804318058904672 -0.0254371491254941
0.786903034760972 -0.0127366767652624
0.769461742425004 4.29950186324069e-05
0.752004781254135 0.0128700386771497
0.734542750130382 0.0257124888331812
0.717086243048227 0.0385383222295833
0.699645844712938 0.0513155374272836
0.682232126154625 0.0640122340511521
0.664855640369735 0.0765966913776288
0.647526918007365 0.0890374460548427
0.630256463122704 0.101303368743397
0.613054749024055 0.113363739464377
0.595932214243 0.125188321440869
0.578899258658929 0.136747433220813
0.561966239808458 0.148012018873052
0.545143469406046 0.158953716055882
0.528441210092295 0.169544921769502
0.511869672407924 0.179758855622043
0.495439011959919 0.189569620465462
0.479159326695086 0.198952260295039
0.463040654116175 0.207882815357607
0.447092968153461 0.216338374482455
0.431326175222446 0.224297124738732
0.41575010873165 0.231738398637223
0.400374520922108 0.238642719233979
0.38520907038282 0.244991843656404
0.37026330284855 0.250768805749989
0.355546621900637 0.255957958716022
0.341068244918601 0.260545018739871
0.326837138059129 0.264517110634022
0.312861922218046 0.267862816346776
0.299150740017787 0.270572226688282
0.285711072187049 0.272636995643726
0.272549490830637 0.274050395018263
0.259671337879835 0.274807364778616
0.247080320574249 0.274904551350881
0.234778023333987 0.274340322593555
0.22282111677237 0.273147174350289
0.211212000441657 0.271312180989548
0.199956090718924 0.268842225658862
0.18905951088657 0.265745762806622
0.178528814558833 0.262032850875418
0.168370684504055 0.257715152729482
0.158591652631679 0.252805894649178
0.149197881311021 0.247319783843054
0.140195031151771 0.241272894158711
0.131588220396365 0.234682536784361
0.123382062255884 0.227567134929123
0.115580754025762 0.219946118404367
0.108188187942631 0.211839847297465
0.101208057557953 0.203269566144989
0.0946439418448079 0.19425738357691
0.088499358802492 0.184826268669802
0.0827777883280161 0.17500005435619
0.0774826693237491 0.164803439481848
0.0726173783628224 0.154261983466959
0.0681851974057714 0.143402090111052
0.0641892769319838 0.132250979303424
0.0606325991839522 0.12083664700672
0.0575179445429576 0.109187814856514
0.0548478626398547 0.0973338711778623
0.0526246487503009 0.0853048053160545
0.0508503253165833 0.0731311370573394
0.0495266280133111 0.0608438426868979
0.0486549955532945 0.0484742789692179
0.0482365623412967 0.036054106082026
0.0482721530701452 0.0236152103073717
0.0487622783751904 0.0111896270851532
0.0497071306928156 -0.00119053514168032
0.0511065794918781 -0.0134931700273918
0.0529601650583767 -0.0256862472791051
0.0552670900164109 -0.0377378855745846
0.0580262077740067 -0.0496164247880426
0.061236007111159 -0.0612904981681536
0.0648945922107272 -0.0727291054819839
0.0689996576140302 -0.0839016885812368
0.0735484579177227 -0.0947782113457813
0.0785377725810335 -0.10532924646507
0.0839638670453019 -0.11552607193497
0.0898224525211534 -0.125340780318524
0.0961086482561191 -0.134746403519034
0.10281695173701 -0.143717054765618
0.109941223832714 -0.152228087440402
0.117474696885182 -0.160256267121569
0.125410013591915 -0.167779948892829
0.133739302545796 -0.174779247148552
0.14245429208538 -0.181236180932355
0.151546457798383 -0.187134775860053
0.161007191589854 -0.192461105490089
0.170827973498639 -0.197203261519913
0.181000523658031 -0.201351252889932
0.191516912754772 -0.204896846530735
0.202369615510414 -0.207833373632873
0.213551501760898 -0.210155531524887
0.22505577088336 -0.211859210579139
0.236875844499746 -0.212941368432345
0.249005237340447 -0.213399962707235
0.261437426261433 -0.213233941776159
0.274165733616003 -0.212443283869614
0.28718323530924 -0.211029069576408
0.300482697779478 -0.208993571566475
0.314056543186631 -0.206340347198259
0.327896838839168 -0.203074323164878
0.341995305342881 -0.199201865241865
0.356343337731412 -0.194730829684218
0.370932034454633 -0.189670595476494
0.385752230119493 -0.184032078393973
0.400794528982751 -0.177827728806396
0.416049337197401 -0.171071515546711
0.431506892628147 -0.16377889817583
0.447157291657684 -0.15596678976561
0.462990512824106 -0.147653512015245
0.478996437395876 -0.138858744187699
0.495164867142796 -0.129603467046443
0.511485539633279 -0.119909902709476
0.527948141407569 -0.109801451123518
0.544542319363787 -0.0993026236948103
0.561257690663388 -0.0884389744880547
0.578083851424296 -0.0772370293146803
0.595010384429933 -0.0657242129685941
0.612026866043939 -0.0539287748255556
0.62912287248563 -0.0418797129959582
0.646287985590849 -0.029606697205851
0.663511798156965 -0.0171399905741775
0.680783918949166 -0.00451037045289776
0.698093977427272 0.00825095150093907
0.715431628237671 0.0211124098473716
0.732786555502968 0.0340421663000904
};%
\addlegendentry{Koopman}
\addplot [line width = \linewidthEightC, color = RungeKutta, dash pattern=on 1pt off 3pt on 3pt off 3pt]
table {%
0.75 0
0.767347219519284 0.012137253483567
0.784683394455509 0.0242435861667548
0.801997486582194 0.036288155514001
0.819278471531621 0.0482402761288358
0.836515345700615 0.0600694978228124
0.853697133135913 0.0717456830897325
0.870812892392119 0.0832390837898383
0.887851723355235 0.0945204168511733
0.904802774024683 0.105560938798497
0.921655247246804 0.116332518924018
0.938398407392915 0.126807710918802
0.955021586975309 0.136959822789028
0.971514193195148 0.146762984887382
0.987865714417124 0.156192215896701
1.0040657265673 0.165223486610677
1.02010389945277 0.173833781364724
1.03597000300535 0.182001156979107
1.05165391345605 0.189704799085743
1.06714561945411 0.196925075719347
1.08243522815337 0.203643588062188
1.09751297130037 0.209843218238558
1.1123692113737 0.215508174058654
1.12699444784162 0.220624030609957
1.14137932362429 0.225177768584601
1.1555146318663 0.22915780921071
1.16939132313901 0.232554045621132
1.18300051319534 0.235357870442502
1.19633349138035 0.237562199321834
1.20938172974852 0.23916149003295
1.22213689283924 0.240151756735333
1.23459084790774 0.240530578918402
1.24673567520343 0.24029710459028
1.25856367765532 0.239452047402427
1.27006738911946 0.237997677671498
1.28123958024499 0.235937807669022
1.29207326111455 0.233277772045945
1.30256168017249 0.230024404727089
1.31269831956044 0.226186013888674
1.32247688771958 0.221772356568193
1.33189131078413 0.216794613983565
1.34093572465289 0.211265367835258
1.34960446953272 0.205198576949498
1.35789208820684 0.198609552872807
1.36579332846363 0.191514932670796
1.37330314928634 0.18393264729003
1.38041672978698 0.175881884318464
1.38712947958844 0.167383044631953
1.39343704941026 0.158457693037802
1.39933534088867 0.149128503482261
1.40482051503024 0.139419199628507
1.40988899904863 0.129354491659788
1.41453749160152 0.118960010081206
1.41876296661194 0.108262237150103
1.42256267593454 0.0972884364110782
1.42593415113889 0.086066580678514
1.42887520465437 0.0746252787098322
1.43138393047621 0.0629937007473294
1.43345870458359 0.0512015030695605
1.43509818517756 0.0392787516771522
1.43630131281055 0.0272558452352367
1.43706731045357 0.0151634373994425
1.43739568352819 0.00303235866024866
1.43728621991849 -0.00910646215137196
1.43673898997073 -0.0212220765475008
1.43575434648365 -0.0332835951024502
1.43433292468914 -0.0452602660529673
1.43247564221903 -0.0571215536269705
1.43018369904855 -0.0688372159533007
1.42745857739782 -0.0803773824121825
1.4243020415584 -0.0917126302942692
1.42071613759078 -0.102814060643163
1.41670319280875 -0.113653373158817
1.41226581492764 -0.124202940032363
1.40740689070718 -0.134435878560225
1.40212958387022 -0.144326122339201
1.3964373320371 -0.153848490767307
1.39033384239815 -0.162978756463708
1.38382308587764 -0.171693710079292
1.37690928964889 -0.179971221816331
1.36959692806383 -0.187790298849553
1.36189071236441 -0.195131137800147
1.35379557991389 -0.201975171526458
1.34531668404164 -0.20830510981277
1.33645938581161 -0.214104974062127
1.32722924897007 -0.219360126749639
1.31763203892423 -0.224057297001799
1.30767372588241 -0.22818460402503
1.2973604914176 -0.231731580045289
1.28669873695777 -0.234689193904312
1.27569509230234 -0.237049875620184
1.26435642233521 -0.238807541316647
1.25268983059913 -0.239957617226964
1.24070265912099 -0.240497061156197
1.22840248459827 -0.24042437986202
1.21579711159144 -0.239739641182038
1.20289456363565 -0.238444480231447
1.1897030732081 -0.236542099469318
1.17623107134149 -0.234037262792463
1.16248717745014 -0.230936284032406
1.14848018970572 -0.227247010319085
1.13421907610845 -0.222978800771493
1.11971296626323 -0.218142500920566
1.10497114378721 -0.212750413195198
1.09000303923492 -0.206816263729265
1.07481822341698 -0.200355165686855
1.05942640099553 -0.193383579258294
1.04383740425637 -0.185919268450478
1.02806118697728 -0.177981254778958
1.01210781833129 -0.169589767962932
0.995987476780275 -0.160766193724954
0.979710443928391 -0.151533018802209
0.963287098315684 -0.141913773283855
0.946727909140567 -0.131932970397911
0.930043429905878 -0.121616043880467
0.91324429198767 -0.110989283069269
0.896341198128817 -0.100079765872392
0.879344915861624 -0.0889152897708354
0.862266270864871 -0.0775243010211479
0.845116140261585 -0.0659358222307342
0.827905445864243 -0.0541793784842489
0.810645147374399 -0.042284922204474
0.793346235543765 -0.0302827569353333
0.776019725303823 -0.0182034602382608
0.758676648870971 -0.00607780589600706
0.74132804883419 0.00606331437980999
0.723984971232126 0.0181889695395348
0.706658458626486 0.0302682678858693
0.689359543178601 0.0422704356646973
0.672099239736024 0.0541648953575106
0.654888538936098 0.0659213434775336
0.637738400333409 0.077509827673207
0.620659745558167 0.0889008229448788
0.603663451512546 0.10006530678342
0.586760343612044 0.110974833042994
0.569961189078862 0.121601604364455
0.55327669029407 0.131918542970798
0.536717478214957 0.141899359661781
0.520294105863188 0.151518620841353
0.504017041888231 0.160751813418715
0.487896664208613 0.169575407431921
0.471943253730781 0.177966916251536
0.456166988141264 0.18590495423109
0.440577935762111 0.193369291680399
0.425186049451712 0.200340907046881
0.410001160522753 0.206802036197834
0.395032972635676 0.212736218702091
0.38029105560979 0.218128341010777
0.36578483907538 0.222964676431608
0.351523605870586 0.227232921776542
0.337516485069478 0.23092223053541
0.323772444518238 0.234023242385741
0.310300282763089 0.236528108790522
0.297108620287721 0.238430514364134
0.284205890052545 0.239725693611976
0.27160032745433 0.240410442590784
0.259299960006205 0.240483125024951
0.247312597261469 0.239943672488383
0.2356458217318 0.238793578459958
0.224306981713199 0.237035886402939
0.21330318694315 0.234675172482025
0.202641307788403 0.231717524032126
0.192327978173203 0.228170515287301
0.1823696017639 0.22404318200088
0.172772360193836 0.219345996319791
0.163542221572307 0.214090842618848
0.154684947376684 0.208290994107505
0.146206096155645 0.201961089155653
0.138111023180157 0.195117105712544
0.130404876038488 0.187776332069152
0.12309258692145 0.179957332525672
0.116178862798813 0.171679907117356
0.109668174793245 0.162965045212245
0.10356474787866 0.153834873345794
0.0978725516915446 0.144312598005936
0.0925952928735381 0.134422444219678
0.0877364090495248 0.124189590767097
0.0832990643281257 0.113640102728524
0.0792861460924091 0.102800861916777
0.0757002628080311 0.0916994956001386
0.0725437425870994 0.0803643038046822
0.069818632284583 0.0688241854024739
0.0675266969522104 0.0571085631419612
0.0656694195214907 0.0452473077513642
0.0642480006271671 0.0332706612373471
0.0632633585132331 0.0212091595029022
0.0627161289858991 0.00909355441514592
0.0626066653929575 -0.00304526453801165
0.0629350386185861 -0.0151763488558262
0.0637010370885092 -0.0272687697454203
0.0649041667842718 -0.0392916968007361
0.0665436512687222 -0.0512144764858195
0.0686184317292081 -0.0630067102720941
0.0711271670520408 -0.0746383322854955
0.0740682339532131 -0.0860796863272447
0.0774397272079407 -0.0973016021399556
0.0812394600470219 -0.108275470796062
0.0854649648223936 -0.118973319084073
0.0901134940873028 -0.12936788275431
0.0951820222852561 -0.139432678452301
0.100667248289028 -0.149142074107302
0.10656559906303 -0.158471357449301
0.112873234718821 -0.167396802199979
0.119586055168304 -0.175895731332128
0.126699708426136 -0.183946576646328
0.13420960035749 -0.191528933821879
0.142110905322839 -0.198623612126504
0.150398576795015 -0.205212678179984
0.159067356720643 -0.211279493592426
0.16811178230224 -0.216808746901301
0.177526189102462 -0.221786480886398
0.187304709947427 -0.226200116851138
0.197441269929379 -0.230038477621279
0.207929578656845 -0.233291810717099
0.218763121506754 -0.235951812451675
0.229935151803763 -0.238011652801433
0.241438685552319 -0.239466000064469
0.253266499708268 -0.240311043799429
0.265411134226006 -0.240544514421553
0.277864897472831 -0.240165698073282
0.290619874197545 -0.23917544583971
0.303667935101709 -0.237576176879925
0.317000747134867 -0.235371875470324
0.330609783830706 -0.232568082244952
0.344486335235128 -0.229171880066449
0.358621517191145 -0.225191874998346
0.373006279910045 -0.220638172815656
0.387631415867 -0.215522351422736
0.402487567119295 -0.209857429471725
0.417565232169048 -0.203657831407191
0.432854772492479 -0.196939349109845
0.448346418844808 -0.189719100275549
0.46403027743119 -0.182015483643688
0.47989633601438 -0.173848131178146
0.495934470011695 -0.165237857301697
0.512134448618413 -0.156206605287707
0.528485940982374 -0.146777390919585
0.544978522444945 -0.136974243536887
0.561601680856407 -0.126822144596177
0.578344822968638 -0.116346963884112
0.595197280904322 -0.105575393529208
0.61214831869946 -0.094534879967153
0.629187138914278 -0.0832535540222168
0.646302889306651 -0.071760159274263
0.663484669561462 -0.0600839788869683
0.680721538069075 -0.0482547610782518
0.698002518745858 -0.0363026434185232
0.715316607889725 -0.0242580761462783
0.732652781063624 -0.012151744693772
};
\addlegendentry{first principles}
\addplot [line width = \linewidthEightC, color = reference, opacity=\opacityRef, forget plot]
table {%
0.75 0
0.753718912601471 0.00242429971694946
0.770212352275848 0.0134586095809937
0.789323568344116 0.0265947580337524
0.806368410587311 0.0381941646337509
0.824847817420959 0.0503321886062622
0.842366218566895 0.0617480725049973
0.859987139701843 0.0731307715177536
0.877748429775238 0.0849964171648026
0.895124673843384 0.0960681438446045
0.913122892379761 0.107316583395004
0.93078351020813 0.118293762207031
0.947595000267029 0.128591924905777
0.964929461479187 0.13916677236557
0.981981873512268 0.149004951119423
0.998650431632996 0.157995745539665
1.01541018486023 0.167062401771545
1.03239750862122 0.17563758790493
1.04923164844513 0.183914437890053
1.06475079059601 0.191342487931252
1.08089053630829 0.198263838887215
1.09715759754181 0.204974785447121
1.11295092105865 0.211227610707283
1.12768888473511 0.216315433382988
1.14281558990479 0.221330478787422
1.15722739696503 0.225320085883141
1.17169892787933 0.228864744305611
1.18580090999603 0.23160745203495
1.1995804309845 0.23436413705349
1.21321308612823 0.235962435603142
1.22622585296631 0.237236872315407
1.23907363414764 0.23741640150547
1.2515857219696 0.237435951828957
1.26365149021149 0.236642554402351
1.27536141872406 0.235256537795067
1.28702819347382 0.23333902657032
1.29814374446869 0.230727210640907
1.30875945091248 0.227544739842415
1.31939446926117 0.223754897713661
1.32922947406769 0.219133988022804
1.33896088600159 0.214116051793098
1.34816455841064 0.208807572722435
1.35690975189209 0.202443465590477
1.3652229309082 0.196036919951439
1.37331628799438 0.188743129372597
1.38093841075897 0.181094661355019
1.38821494579315 0.172646269202232
1.39475667476654 0.164254739880562
1.40123760700226 0.155717775225639
1.40713751316071 0.145796492695808
1.41266846656799 0.136052891612053
1.41758561134338 0.125909015536308
1.42236387729645 0.115142315626144
1.42658770084381 0.103971615433693
1.43046450614929 0.0929473489522934
1.43382585048676 0.0812477469444275
1.43696331977844 0.0697202384471893
1.43913269042969 0.0578404664993286
1.4411860704422 0.0457369983196259
1.4428004026413 0.0337170138955116
1.44364190101624 0.0214479938149452
1.44417023658752 0.00872129201889038
1.44456052780151 -0.00376687943935394
1.44426286220551 -0.0156141631305218
1.44349718093872 -0.02806581184268
1.44260001182556 -0.040539775043726
1.44104206562042 -0.0526782125234604
1.43909907341003 -0.0651116389781237
1.43661344051361 -0.0770855918526649
1.43368828296661 -0.0886896280571818
1.43042039871216 -0.100536113604903
1.42655539512634 -0.111840512603521
1.4225081205368 -0.122851498425007
1.41787934303284 -0.134381428360939
1.41282856464386 -0.144347697496414
1.40747952461243 -0.154774695634842
1.40165996551514 -0.164202868938446
1.39536166191101 -0.173553019762039
1.38837373256683 -0.182547375559807
1.38145959377289 -0.19086255133152
1.37385940551758 -0.198835171759129
1.36580967903137 -0.206451192498207
1.35758364200592 -0.213272988796234
1.34881186485291 -0.219960153102875
1.3397878408432 -0.225531592965126
1.33023571968079 -0.230652406811714
1.32049202919006 -0.235137313604355
1.31029725074768 -0.239362835884094
1.30030155181885 -0.242626890540123
1.28932690620422 -0.24485644698143
1.27797436714172 -0.247304037213326
1.26649379730225 -0.248599871993065
1.25419747829437 -0.249305322766304
1.24222350120544 -0.249616235494614
1.22961914539337 -0.249537467956543
1.21665143966675 -0.24809929728508
1.20377123355865 -0.246283277869225
1.19040763378143 -0.244004741311073
1.1769152879715 -0.240832522511482
1.16300272941589 -0.237234011292458
1.14894711971283 -0.232919171452522
1.13467156887054 -0.2280233502388
1.12009465694427 -0.222499579191208
1.10526943206787 -0.216294705867767
1.09001219272614 -0.209486663341522
1.07490885257721 -0.202274814248085
1.05968141555786 -0.194313056766987
1.04417324066162 -0.186224088072777
1.02862167358398 -0.177295841276646
1.01289522647858 -0.167821124196053
0.996960639953613 -0.158159807324409
0.980685591697693 -0.148069202899933
0.964592337608337 -0.13747327029705
0.948221564292908 -0.126186370849609
0.931687474250793 -0.114853199571371
0.915010333061218 -0.103031110018492
0.898247718811035 -0.0906929392367601
0.881550014019012 -0.0784856379032135
0.864428162574768 -0.0659199524670839
0.847152829170227 -0.0532354824244976
0.830408275127411 -0.0400973930954933
0.813312113285065 -0.0271125212311745
0.796311557292938 -0.0142120681703091
0.778890073299408 -0.000622719526290894
0.76205974817276 0.0124123618006706
0.744917333126068 0.0258218497037888
0.727669775485992 0.039159819483757
0.710483312606812 0.0526659041643143
0.693443775177002 0.0660786926746368
0.676377654075623 0.0789817124605179
0.659431576728821 0.0918970704078674
0.642633318901062 0.104910492897034
0.625959396362305 0.117601409554482
0.609083771705627 0.129351139068604
0.592366576194763 0.142021834850311
0.575754106044769 0.154023259878159
0.559485912322998 0.165491461753845
0.542809903621674 0.176464095711708
0.52660471200943 0.187428250908852
0.510689616203308 0.197872415184975
0.494461357593536 0.207829996943474
0.478566825389862 0.217379346489906
0.463127434253693 0.22639511525631
0.447281777858734 0.23519529402256
0.43188863992691 0.243261680006981
0.416683733463287 0.250534936785698
0.401248961687088 0.257817819714546
0.38664048910141 0.264186218380928
0.371667057275772 0.270101293921471
0.357158631086349 0.275359287858009
0.342733144760132 0.279933497309685
0.328635543584824 0.284174993634224
0.314697563648224 0.287533774971962
0.300990015268326 0.290349826216698
0.288122177124023 0.292503640055656
0.275072872638702 0.294003412127495
0.262523859739304 0.294744864106178
0.250071972608566 0.295193269848824
0.237708151340485 0.294850841164589
0.225887477397919 0.293832048773766
0.214226931333542 0.291856870055199
0.203227549791336 0.289549872279167
0.1922946870327 0.286594465374947
0.181893080472946 0.282819792628288
0.171391159296036 0.27877913415432
0.161658197641373 0.273651763796806
0.152567967772484 0.268529668450356
0.143379971385002 0.262409761548042
0.135153114795685 0.255966052412987
0.126830443739891 0.248980596661568
0.118843644857407 0.241222038865089
0.111569032073021 0.233345672488213
0.104426443576813 0.224568143486977
0.0981230437755585 0.215527758002281
0.091672882437706 0.206323519349098
0.08586385846138 0.196149930357933
0.0805563628673553 0.186116442084312
0.0752256363630295 0.175130948424339
0.0709647834300995 0.16464464366436
0.0669632107019424 0.1536755412817
0.0634450018405914 0.142153069376945
0.0604579001665115 0.130531504750252
0.0576128959655762 0.118661269545555
0.0552339255809784 0.106500536203384
0.0532824248075485 0.0943268835544586
0.0518424659967422 0.0820179134607315
0.0509601682424545 0.0693861246109009
0.0502793937921524 0.0568328201770782
0.0501424074172974 0.04461470246315
0.050503596663475 0.0324117317795753
0.0516011416912079 0.0192458629608154
0.052910715341568 0.00736457109451294
0.054675355553627 -0.00492463260889053
0.0567812919616699 -0.0170576348900795
0.059564083814621 -0.0288318991661072
0.0625138580799103 -0.0400620438158512
0.0659487694501877 -0.0515238456428051
0.0700125098228455 -0.0625142939388752
0.0744896829128265 -0.073791628703475
0.0793655216693878 -0.0842238767072558
0.0846089124679565 -0.0943614486604929
0.0905614793300629 -0.104039743542671
0.096687063574791 -0.113539785146713
0.103571683168411 -0.122788548469543
0.110575646162033 -0.131163910031319
0.117918133735657 -0.139348357915878
0.125692531466484 -0.146777428686619
0.13396991789341 -0.153551951050758
0.142736479640007 -0.160058550536633
0.151705592870712 -0.166077464818954
0.161301761865616 -0.171692401170731
0.171207189559937 -0.176636159420013
0.181274443864822 -0.180351883172989
0.192060023546219 -0.184194765985012
0.203426957130432 -0.18717648088932
0.2148118019104 -0.189745128154755
0.226283699274063 -0.191540703177452
0.238179594278336 -0.192663371562958
0.250413060188293 -0.192603655159473
0.262946337461472 -0.192625150084496
0.275385171175003 -0.191834770143032
0.288414627313614 -0.190404050052166
0.302370667457581 -0.188122496008873
0.316254109144211 -0.185476265847683
0.330432921648026 -0.181842051446438
0.34482279419899 -0.177787967026234
0.359332084655762 -0.173099257051945
0.37461793422699 -0.167893461883068
0.389579743146896 -0.16203498095274
0.405168354511261 -0.155595555901527
0.42081606388092 -0.148530602455139
0.43662691116333 -0.141376465559006
0.452610015869141 -0.133058231323957
0.469050884246826 -0.124966330826283
0.485199928283691 -0.115748234093189
0.501727163791656 -0.105858040973544
0.518296122550964 -0.0963074807077646
0.535101711750031 -0.0860297996550798
0.551869511604309 -0.0749665778130293
0.568884074687958 -0.0641769841313362
0.585899412631989 -0.0531577970832586
0.602847099304199 -0.0408559329807758
0.620223343372345 -0.0294299647212029
0.637332916259766 -0.0170636549592018
0.655044913291931 -0.00501717627048492
0.672139167785645 0.00751902163028717
0.689942240715027 0.0203956067562103
0.707553327083588 0.0328872948884964
0.724772036075592 0.0455862507224083
0.742378532886505 0.058309331536293
};
\addplot [semithick, red, dashed, forget plot]
table {%
0.75 0
0.767617956255685 0.0126152320166048
0.785221863675294 0.025202759744927
0.802800914680889 0.0377306780229099
0.820344353102624 0.0501672114827695
0.837841479099983 0.0624807937495562
0.855281653865309 0.0746401456242545
0.872654304077727 0.0866143520292023
0.889948926065816 0.0983729374930471
0.90715508962574 0.109885939954111
0.924262441427357 0.121123982665618
0.941260707923679 0.132058343994744
0.958139697658408 0.142661024921306
0.974889302841815 0.152904814062882
0.991499500036706 0.162763350083898
1.00796034976389 0.172211181389763
1.02426199480132 0.18122382306779
1.04039465691463 0.189777811119373
1.0563486317231 0.197850754138649
1.07211428137931 0.205421382738366
1.08768202473299 0.212469597210257
1.10304232467243 0.218976514139914
1.11818567241108 0.224924512976024
1.1331025686362 0.230297283873765
1.147783501695 0.23507987847157
1.16221892339532 0.239258765576722
1.17639922357808 0.242821893954846
1.19031470539574 0.245758764427774
1.20395556419321 0.248060513126141
1.21731187395941 0.249720006825457
1.23037358633393 0.250731949622618
1.24313054782987 0.251092997654816
1.2555725408717 0.250801875167454
1.26768935298946 0.24985948134918
1.27947087567813 0.248268973729967
1.29090722994702 0.246035811751358
1.30198890990496 0.243167744731957
1.31270692999354 0.2396747329547
1.32305295737637 0.235568799170581
1.33301941026506 0.230863819194334
1.34259950666846 0.225575271775738
1.3517872558846 0.219719976188812
1.36057739518197 0.213315848229081
1.36896528374308 0.206381700652654
1.37694677240494 0.198937103974783
1.38451806954475 0.191002311131823
1.39167562078374 0.182598238316676
1.39841601450881 0.173746486857157
1.4047359185686 0.164469388195659
1.41063204764278 0.154790055213092
1.41610115672778 0.144732426840967
1.42114005415556 0.134321297531182
1.42574562723652 0.123582327489247
1.42991487441761 0.112542032954869
1.43364493919855 0.101227758043132
1.43693314251624 0.089667630857217
1.4397770116222 0.0778905070053025
1.44217430452097 0.0659259035746353
1.44412302978876 0.0538039262643269
1.44562146208725 0.0415551919190076
1.44666815398553 0.0292107482394724
1.44726194486432 0.016801992026761
1.44740196775303 0.00436058696240971
1.44708765498222 -0.00808161836067393
1.44631874355061 -0.0204926716768206
1.4450952811249 -0.0328405972734807
1.44341763362016 -0.0450934751015305
1.44128649535024 -0.0572195188465802
1.4387029027798 -0.0691871532341023
1.43566825293285 -0.0809650910678666
1.4321843274821 -0.0925224107662305
1.42825332340439 -0.103828635462315
1.42387789076703 -0.114853815053745
1.41906117760902 -0.125568612881875
1.41380688088891 -0.135944398910193
1.40811930097384 -0.145953351233135
1.40200339507468 -0.155568567310703
1.39546482242266 -0.164764185292558
1.38850997106277 -0.173515513988717
1.38114595342609 -0.181799167392522
1.37338055620066 -0.189593196333819
1.36522213060654 -0.196877206377916
1.35667941319178 -0.203632448466232
1.34776127545165 -0.209841868311401
1.33847641259836 -0.215490103434114
1.328832995749 -0.220563423487838
1.3188383241811 -0.225049619377379
1.30849852098817 -0.228937857346307
1.29781831326193 -0.232218522442798
1.28680092623794 -0.234883078640104
1.27544810236402 -0.23692396914816
1.26376023613114 -0.238334571253006
1.25173659926596 -0.239109208409727
1.23937562223503 -0.239243211851919
1.22680212890194 -0.238807395220756
1.21393543612189 -0.237739020519201
1.2007824030094 -0.236042071113405
1.1873502195155 -0.233721342391173
1.17364639818837 -0.230782623136066
1.15967876862552 -0.227232843837933
1.14545546822952 -0.223080191387264
1.13098492732646 -0.218334192018305
1.11627584919941 -0.213005765794969
1.10133718666704 -0.207107256561536
1.08617811704058 -0.200652441341106
1.070808017043 -0.19365652287622
1.05523643885516 -0.186136108539406
1.03947308802467 -0.178109178316594
1.02352780361101 -0.169595044057188
1.00741054066936 -0.160614301730894
0.991131354992172 -0.151188778049092
0.974700389916618 -0.141341472499412
0.958127864949596 -0.131096495600034
0.941424065943487 -0.120479003995558
0.924599336561409 -0.109515132878708
0.90766407079076 -0.0982319261219058
0.890628706290982 -0.086657264431339
0.873503718391248 -0.074819791786618
0.856299614583182 -0.0627488403957748
0.839026929381251 -0.0504743543738914
0.8216962194482 -0.0380268123406879
0.804318058904672 -0.0254371491254941
0.786903034760972 -0.0127366767652624
0.769461742425004 4.29950186324069e-05
0.752004781254135 0.0128700386771497
0.734542750130382 0.0257124888331812
0.717086243048227 0.0385383222295833
0.699645844712938 0.0513155374272836
0.682232126154625 0.0640122340511521
0.664855640369735 0.0765966913776288
0.647526918007365 0.0890374460548427
0.630256463122704 0.101303368743397
0.613054749024055 0.113363739464377
0.595932214243 0.125188321440869
0.578899258658929 0.136747433220813
0.561966239808458 0.148012018873052
0.545143469406046 0.158953716055882
0.528441210092295 0.169544921769502
0.511869672407924 0.179758855622043
0.495439011959919 0.189569620465462
0.479159326695086 0.198952260295039
0.463040654116175 0.207882815357607
0.447092968153461 0.216338374482455
0.431326175222446 0.224297124738732
0.41575010873165 0.231738398637223
0.400374520922108 0.238642719233979
0.38520907038282 0.244991843656404
0.37026330284855 0.250768805749989
0.355546621900637 0.255957958716022
0.341068244918601 0.260545018739871
0.326837138059129 0.264517110634022
0.312861922218046 0.267862816346776
0.299150740017787 0.270572226688282
0.285711072187049 0.272636995643726
0.272549490830637 0.274050395018263
0.259671337879835 0.274807364778616
0.247080320574249 0.274904551350881
0.234778023333987 0.274340322593555
0.22282111677237 0.273147174350289
0.211212000441657 0.271312180989548
0.199956090718924 0.268842225658862
0.18905951088657 0.265745762806622
0.178528814558833 0.262032850875418
0.168370684504055 0.257715152729482
0.158591652631679 0.252805894649178
0.149197881311021 0.247319783843054
0.140195031151771 0.241272894158711
0.131588220396365 0.234682536784361
0.123382062255884 0.227567134929123
0.115580754025762 0.219946118404367
0.108188187942631 0.211839847297465
0.101208057557953 0.203269566144989
0.0946439418448079 0.19425738357691
0.088499358802492 0.184826268669802
0.0827777883280161 0.17500005435619
0.0774826693237491 0.164803439481848
0.0726173783628224 0.154261983466959
0.0681851974057714 0.143402090111052
0.0641892769319838 0.132250979303424
0.0606325991839522 0.12083664700672
0.0575179445429576 0.109187814856514
0.0548478626398547 0.0973338711778623
0.0526246487503009 0.0853048053160545
0.0508503253165833 0.0731311370573394
0.0495266280133111 0.0608438426868979
0.0486549955532945 0.0484742789692179
0.0482365623412967 0.036054106082026
0.0482721530701452 0.0236152103073717
0.0487622783751904 0.0111896270851532
0.0497071306928156 -0.00119053514168032
0.0511065794918781 -0.0134931700273918
0.0529601650583767 -0.0256862472791051
0.0552670900164109 -0.0377378855745846
0.0580262077740067 -0.0496164247880426
0.061236007111159 -0.0612904981681536
0.0648945922107272 -0.0727291054819839
0.0689996576140302 -0.0839016885812368
0.0735484579177227 -0.0947782113457813
0.0785377725810335 -0.10532924646507
0.0839638670453019 -0.11552607193497
0.0898224525211534 -0.125340780318524
0.0961086482561191 -0.134746403519034
0.10281695173701 -0.143717054765618
0.109941223832714 -0.152228087440402
0.117474696885182 -0.160256267121569
0.125410013591915 -0.167779948892829
0.133739302545796 -0.174779247148552
0.14245429208538 -0.181236180932355
0.151546457798383 -0.187134775860053
0.161007191589854 -0.192461105490089
0.170827973498639 -0.197203261519913
0.181000523658031 -0.201351252889932
0.191516912754772 -0.204896846530735
0.202369615510414 -0.207833373632873
0.213551501760898 -0.210155531524887
0.22505577088336 -0.211859210579139
0.236875844499746 -0.212941368432345
0.249005237340447 -0.213399962707235
0.261437426261433 -0.213233941776159
0.274165733616003 -0.212443283869614
0.28718323530924 -0.211029069576408
0.300482697779478 -0.208993571566475
0.314056543186631 -0.206340347198259
0.327896838839168 -0.203074323164878
0.341995305342881 -0.199201865241865
0.356343337731412 -0.194730829684218
0.370932034454633 -0.189670595476494
0.385752230119493 -0.184032078393973
0.400794528982751 -0.177827728806396
0.416049337197401 -0.171071515546711
0.431506892628147 -0.16377889817583
0.447157291657684 -0.15596678976561
0.462990512824106 -0.147653512015245
0.478996437395876 -0.138858744187699
0.495164867142796 -0.129603467046443
0.511485539633279 -0.119909902709476
0.527948141407569 -0.109801451123518
0.544542319363787 -0.0993026236948103
0.561257690663388 -0.0884389744880547
0.578083851424296 -0.0772370293146803
0.595010384429933 -0.0657242129685941
0.612026866043939 -0.0539287748255556
0.62912287248563 -0.0418797129959582
0.646287985590849 -0.029606697205851
0.663511798156965 -0.0171399905741775
0.680783918949166 -0.00451037045289776
0.698093977427272 0.00825095150093907
0.715431628237671 0.0211124098473716
0.732786555502968 0.0340421663000904
};
\addplot [semithick, green, dash pattern=on 1pt off 3pt on 3pt off 3pt, forget plot]
table {%
0.75 0
0.767347219519284 0.012137253483567
0.784683394455509 0.0242435861667548
0.801997486582194 0.036288155514001
0.819278471531621 0.0482402761288358
0.836515345700615 0.0600694978228124
0.853697133135913 0.0717456830897325
0.870812892392119 0.0832390837898383
0.887851723355235 0.0945204168511733
0.904802774024683 0.105560938798497
0.921655247246804 0.116332518924018
0.938398407392915 0.126807710918802
0.955021586975309 0.136959822789028
0.971514193195148 0.146762984887382
0.987865714417124 0.156192215896701
1.0040657265673 0.165223486610677
1.02010389945277 0.173833781364724
1.03597000300535 0.182001156979107
1.05165391345605 0.189704799085743
1.06714561945411 0.196925075719347
1.08243522815337 0.203643588062188
1.09751297130037 0.209843218238558
1.1123692113737 0.215508174058654
1.12699444784162 0.220624030609957
1.14137932362429 0.225177768584601
1.1555146318663 0.22915780921071
1.16939132313901 0.232554045621132
1.18300051319534 0.235357870442502
1.19633349138035 0.237562199321834
1.20938172974852 0.23916149003295
1.22213689283924 0.240151756735333
1.23459084790774 0.240530578918402
1.24673567520343 0.24029710459028
1.25856367765532 0.239452047402427
1.27006738911946 0.237997677671498
1.28123958024499 0.235937807669022
1.29207326111455 0.233277772045945
1.30256168017249 0.230024404727089
1.31269831956044 0.226186013888674
1.32247688771958 0.221772356568193
1.33189131078413 0.216794613983565
1.34093572465289 0.211265367835258
1.34960446953272 0.205198576949498
1.35789208820684 0.198609552872807
1.36579332846363 0.191514932670796
1.37330314928634 0.18393264729003
1.38041672978698 0.175881884318464
1.38712947958844 0.167383044631953
1.39343704941026 0.158457693037802
1.39933534088867 0.149128503482261
1.40482051503024 0.139419199628507
1.40988899904863 0.129354491659788
1.41453749160152 0.118960010081206
1.41876296661194 0.108262237150103
1.42256267593454 0.0972884364110782
1.42593415113889 0.086066580678514
1.42887520465437 0.0746252787098322
1.43138393047621 0.0629937007473294
1.43345870458359 0.0512015030695605
1.43509818517756 0.0392787516771522
1.43630131281055 0.0272558452352367
1.43706731045357 0.0151634373994425
1.43739568352819 0.00303235866024866
1.43728621991849 -0.00910646215137196
1.43673898997073 -0.0212220765475008
1.43575434648365 -0.0332835951024502
1.43433292468914 -0.0452602660529673
1.43247564221903 -0.0571215536269705
1.43018369904855 -0.0688372159533007
1.42745857739782 -0.0803773824121825
1.4243020415584 -0.0917126302942692
1.42071613759078 -0.102814060643163
1.41670319280875 -0.113653373158817
1.41226581492764 -0.124202940032363
1.40740689070718 -0.134435878560225
1.40212958387022 -0.144326122339201
1.3964373320371 -0.153848490767307
1.39033384239815 -0.162978756463708
1.38382308587764 -0.171693710079292
1.37690928964889 -0.179971221816331
1.36959692806383 -0.187790298849553
1.36189071236441 -0.195131137800147
1.35379557991389 -0.201975171526458
1.34531668404164 -0.20830510981277
1.33645938581161 -0.214104974062127
1.32722924897007 -0.219360126749639
1.31763203892423 -0.224057297001799
1.30767372588241 -0.22818460402503
1.2973604914176 -0.231731580045289
1.28669873695777 -0.234689193904312
1.27569509230234 -0.237049875620184
1.26435642233521 -0.238807541316647
1.25268983059913 -0.239957617226964
1.24070265912099 -0.240497061156197
1.22840248459827 -0.24042437986202
1.21579711159144 -0.239739641182038
1.20289456363565 -0.238444480231447
1.1897030732081 -0.236542099469318
1.17623107134149 -0.234037262792463
1.16248717745014 -0.230936284032406
1.14848018970572 -0.227247010319085
1.13421907610845 -0.222978800771493
1.11971296626323 -0.218142500920566
1.10497114378721 -0.212750413195198
1.09000303923492 -0.206816263729265
1.07481822341698 -0.200355165686855
1.05942640099553 -0.193383579258294
1.04383740425637 -0.185919268450478
1.02806118697728 -0.177981254778958
1.01210781833129 -0.169589767962932
0.995987476780275 -0.160766193724954
0.979710443928391 -0.151533018802209
0.963287098315684 -0.141913773283855
0.946727909140567 -0.131932970397911
0.930043429905878 -0.121616043880467
0.91324429198767 -0.110989283069269
0.896341198128817 -0.100079765872392
0.879344915861624 -0.0889152897708354
0.862266270864871 -0.0775243010211479
0.845116140261585 -0.0659358222307342
0.827905445864243 -0.0541793784842489
0.810645147374399 -0.042284922204474
0.793346235543765 -0.0302827569353333
0.776019725303823 -0.0182034602382608
0.758676648870971 -0.00607780589600706
0.74132804883419 0.00606331437980999
0.723984971232126 0.0181889695395348
0.706658458626486 0.0302682678858693
0.689359543178601 0.0422704356646973
0.672099239736024 0.0541648953575106
0.654888538936098 0.0659213434775336
0.637738400333409 0.077509827673207
0.620659745558167 0.0889008229448788
0.603663451512546 0.10006530678342
0.586760343612044 0.110974833042994
0.569961189078862 0.121601604364455
0.55327669029407 0.131918542970798
0.536717478214957 0.141899359661781
0.520294105863188 0.151518620841353
0.504017041888231 0.160751813418715
0.487896664208613 0.169575407431921
0.471943253730781 0.177966916251536
0.456166988141264 0.18590495423109
0.440577935762111 0.193369291680399
0.425186049451712 0.200340907046881
0.410001160522753 0.206802036197834
0.395032972635676 0.212736218702091
0.38029105560979 0.218128341010777
0.36578483907538 0.222964676431608
0.351523605870586 0.227232921776542
0.337516485069478 0.23092223053541
0.323772444518238 0.234023242385741
0.310300282763089 0.236528108790522
0.297108620287721 0.238430514364134
0.284205890052545 0.239725693611976
0.27160032745433 0.240410442590784
0.259299960006205 0.240483125024951
0.247312597261469 0.239943672488383
0.2356458217318 0.238793578459958
0.224306981713199 0.237035886402939
0.21330318694315 0.234675172482025
0.202641307788403 0.231717524032126
0.192327978173203 0.228170515287301
0.1823696017639 0.22404318200088
0.172772360193836 0.219345996319791
0.163542221572307 0.214090842618848
0.154684947376684 0.208290994107505
0.146206096155645 0.201961089155653
0.138111023180157 0.195117105712544
0.130404876038488 0.187776332069152
0.12309258692145 0.179957332525672
0.116178862798813 0.171679907117356
0.109668174793245 0.162965045212245
0.10356474787866 0.153834873345794
0.0978725516915446 0.144312598005936
0.0925952928735381 0.134422444219678
0.0877364090495248 0.124189590767097
0.0832990643281257 0.113640102728524
0.0792861460924091 0.102800861916777
0.0757002628080311 0.0916994956001386
0.0725437425870994 0.0803643038046822
0.069818632284583 0.0688241854024739
0.0675266969522104 0.0571085631419612
0.0656694195214907 0.0452473077513642
0.0642480006271671 0.0332706612373471
0.0632633585132331 0.0212091595029022
0.0627161289858991 0.00909355441514592
0.0626066653929575 -0.00304526453801165
0.0629350386185861 -0.0151763488558262
0.0637010370885092 -0.0272687697454203
0.0649041667842718 -0.0392916968007361
0.0665436512687222 -0.0512144764858195
0.0686184317292081 -0.0630067102720941
0.0711271670520408 -0.0746383322854955
0.0740682339532131 -0.0860796863272447
0.0774397272079407 -0.0973016021399556
0.0812394600470219 -0.108275470796062
0.0854649648223936 -0.118973319084073
0.0901134940873028 -0.12936788275431
0.0951820222852561 -0.139432678452301
0.100667248289028 -0.149142074107302
0.10656559906303 -0.158471357449301
0.112873234718821 -0.167396802199979
0.119586055168304 -0.175895731332128
0.126699708426136 -0.183946576646328
0.13420960035749 -0.191528933821879
0.142110905322839 -0.198623612126504
0.150398576795015 -0.205212678179984
0.159067356720643 -0.211279493592426
0.16811178230224 -0.216808746901301
0.177526189102462 -0.221786480886398
0.187304709947427 -0.226200116851138
0.197441269929379 -0.230038477621279
0.207929578656845 -0.233291810717099
0.218763121506754 -0.235951812451675
0.229935151803763 -0.238011652801433
0.241438685552319 -0.239466000064469
0.253266499708268 -0.240311043799429
0.265411134226006 -0.240544514421553
0.277864897472831 -0.240165698073282
0.290619874197545 -0.23917544583971
0.303667935101709 -0.237576176879925
0.317000747134867 -0.235371875470324
0.330609783830706 -0.232568082244952
0.344486335235128 -0.229171880066449
0.358621517191145 -0.225191874998346
0.373006279910045 -0.220638172815656
0.387631415867 -0.215522351422736
0.402487567119295 -0.209857429471725
0.417565232169048 -0.203657831407191
0.432854772492479 -0.196939349109845
0.448346418844808 -0.189719100275549
0.46403027743119 -0.182015483643688
0.47989633601438 -0.173848131178146
0.495934470011695 -0.165237857301697
0.512134448618413 -0.156206605287707
0.528485940982374 -0.146777390919585
0.544978522444945 -0.136974243536887
0.561601680856407 -0.126822144596177
0.578344822968638 -0.116346963884112
0.595197280904322 -0.105575393529208
0.61214831869946 -0.094534879967153
0.629187138914278 -0.0832535540222168
0.646302889306651 -0.071760159274263
0.663484669561462 -0.0600839788869683
0.680721538069075 -0.0482547610782518
0.698002518745858 -0.0363026434185232
0.715316607889725 -0.0242580761462783
0.732652781063624 -0.012151744693772
};
\addplot [line width = \linewidthEightC, color = reference, opacity=\opacityRef, forget plot]
table {%
0.75 0
0.752575755119324 0.00215128064155579
0.767205417156219 0.0133222192525864
0.786280751228333 0.0277370363473892
0.802484512329102 0.039739266037941
0.819901168346405 0.0531161427497864
0.836963653564453 0.0656682103872299
0.853932023048401 0.0781396776437759
0.870823681354523 0.0907098054885864
0.887600004673004 0.102336779236794
0.904155731201172 0.114123836159706
0.921015739440918 0.126084879040718
0.937650561332703 0.137537330389023
0.954344034194946 0.148466646671295
0.971022725105286 0.158906936645508
0.987065196037292 0.168976068496704
1.00321662425995 0.1787239164114
1.0192574262619 0.188109681010246
1.03494954109192 0.196743175387383
1.05058300495148 0.204712256789207
1.06620192527771 0.212437406182289
1.08144772052765 0.219630405306816
1.09667301177979 0.226312443614006
1.11123645305634 0.232065781950951
1.12593460083008 0.237188145518303
1.14045739173889 0.242261931300163
1.15454590320587 0.246430054306984
1.1685129404068 0.249817296862602
1.18212342262268 0.252737477421761
1.19541525840759 0.254952773451805
1.20853435993195 0.25651328265667
1.22131133079529 0.257282510399818
1.23376750946045 0.257598385214806
1.24552917480469 0.257408753037453
1.2572637796402 0.256569251418114
1.26900029182434 0.254414483904839
1.2804741859436 0.252135083079338
1.29137170314789 0.249488428235054
1.30192601680756 0.246128842234612
1.31231641769409 0.242138311266899
1.32198166847229 0.237636461853981
1.33181297779083 0.23249576985836
1.34108078479767 0.226458504796028
1.349818110466 0.220275983214378
1.35846853256226 0.213245883584023
1.36646783351898 0.205963596701622
1.37441527843475 0.198157414793968
1.38157606124878 0.18964596092701
1.3883593082428 0.180902794003487
1.39493143558502 0.171535521745682
1.4008002281189 0.162054285407066
1.40640771389008 0.151747211813927
1.4117032289505 0.141722470521927
1.41664588451385 0.130396112799644
1.42080342769623 0.120192348957062
1.42457354068756 0.108611986041069
1.42798900604248 0.0967676192522049
1.43072855472565 0.0855403244495392
1.43321228027344 0.0733678042888641
1.43530011177063 0.0612891316413879
1.43680238723755 0.0493988245725632
1.43785965442657 0.0372400656342506
1.43883919715881 0.0247961804270744
1.43914926052094 0.0123106464743614
1.43921983242035 -0.000528618693351746
1.43855512142181 -0.0126301646232605
1.4375673532486 -0.0248252190649509
1.43604934215546 -0.0367438979446888
1.43430006504059 -0.0486404206603765
1.43179094791412 -0.0607579406350851
1.42900323867798 -0.0728645352646708
1.42576360702515 -0.084130484610796
1.42181777954102 -0.0952397678047419
1.41770577430725 -0.106280330568552
1.41297829151154 -0.116966638714075
1.407874584198 -0.12752740085125
1.40238511562347 -0.1378138884902
1.39668214321136 -0.147436983883381
1.39046490192413 -0.156838148832321
1.38356518745422 -0.165770962834358
1.37651085853577 -0.174153544008732
1.36883199214935 -0.182004369795322
1.36060976982117 -0.189779721200466
1.3523143529892 -0.196069076657295
1.34340834617615 -0.202672585844994
1.33426523208618 -0.208261102437973
1.32449877262115 -0.213658854365349
1.31435525417328 -0.217982262372971
1.3043760061264 -0.22172237932682
1.29369795322418 -0.225186422467232
1.28282177448273 -0.227548375725746
1.27115416526794 -0.229821071028709
1.25929045677185 -0.231475636363029
1.24722838401794 -0.232361301779747
1.23464751243591 -0.232167288661003
1.22224175930023 -0.231791511178017
1.20908117294312 -0.230533704161644
1.19543385505676 -0.229195460677147
1.18178677558899 -0.226671382784843
1.16766858100891 -0.223388105630875
1.1535427570343 -0.220086336135864
1.13886141777039 -0.215995922684669
1.12400317192078 -0.21132729947567
1.10894203186035 -0.20592787861824
1.09370028972626 -0.200038745999336
1.07867014408112 -0.193489030003548
1.06276106834412 -0.186380863189697
1.04667675495148 -0.178585663437843
1.03064823150635 -0.170721888542175
1.01431965827942 -0.162336878478527
0.997876763343811 -0.153170943260193
0.98102867603302 -0.143960289657116
0.964348912239075 -0.133790850639343
0.947823166847229 -0.123615562915802
0.930828213691711 -0.113031156361103
0.913803577423096 -0.102256823331118
0.896429061889648 -0.0911583751440048
0.879045903682709 -0.0793363256379962
0.861408889293671 -0.0678172241896391
0.843998849391937 -0.0559111703187227
0.826127767562866 -0.0438387151807547
0.808366358280182 -0.0315168797969818
0.790504395961761 -0.0190286226570606
0.772754073143005 -0.00656995922327042
0.75520396232605 0.0054924488067627
0.737384378910065 0.0182593315839767
0.719463229179382 0.0305039659142494
0.702076733112335 0.0430185198783875
0.68452924489975 0.055399090051651
0.666764199733734 0.0676377862691879
0.649022817611694 0.0799886584281921
0.631596505641937 0.0918839424848557
0.613989949226379 0.1036227196455
0.596686661243439 0.115065217018127
0.579477548599243 0.126677438616753
0.562368810176849 0.137739300727844
0.54533052444458 0.147948607802391
0.528612911701202 0.158274605870247
0.511838138103485 0.168383568525314
0.495180487632751 0.177890285849571
0.478566408157349 0.187091246247292
0.462112665176392 0.195955917239189
0.445851624011993 0.204297170042992
0.429760694503784 0.2121192663908
0.41402006149292 0.21938781440258
0.398187041282654 0.226285800337791
0.382898896932602 0.232457861304283
0.367429465055466 0.238171949982643
0.352806478738785 0.242861434817314
0.337705761194229 0.247955098748207
0.323271185159683 0.252045854926109
0.309147387742996 0.254921481013298
0.29488343000412 0.25798986852169
0.281221896409988 0.259919866919518
0.267925173044205 0.261412516236305
0.254652887582779 0.262266620993614
0.241702288389206 0.262612774968147
0.22926464676857 0.262324050068855
0.217319846153259 0.261357381939888
0.205511420965195 0.259707853198051
0.194227904081345 0.257082685828209
0.182897388935089 0.254405722022057
0.172272056341171 0.250927641987801
0.161999583244324 0.24676214158535
0.15170231461525 0.241967186331749
0.142345070838928 0.236579224467278
0.13313364982605 0.230642035603523
0.124746888875961 0.224150434136391
0.116496235132217 0.217211529612541
0.108591377735138 0.210139855742455
0.101213708519936 0.201845392584801
0.0940813720226288 0.19351689517498
0.0878640860319138 0.185026273131371
0.0817079097032547 0.175606846809387
0.0756756812334061 0.165730983018875
0.0702836215496063 0.155595391988754
0.0653177946805954 0.145141035318375
0.0611104667186737 0.13427846133709
0.056995153427124 0.1226766705513
0.0534923374652863 0.111726269125938
0.0503890663385391 0.0999629497528076
0.0477133691310883 0.0880504846572876
0.0453380346298218 0.0763079971075058
0.0436965525150299 0.0638146847486496
0.0425624400377274 0.0520575940608978
0.0418506562709808 0.0395622551441193
0.0414179563522339 0.027031198143959
0.0413252115249634 0.0151798874139786
0.041848823428154 0.0024307519197464
0.0430923700332642 -0.0105791091918945
0.0445681810379028 -0.0224859714508057
0.0467149466276169 -0.0348117016255856
0.0490246713161469 -0.0469049382954836
0.0519413650035858 -0.0582971479743719
0.0550997257232666 -0.0704701044596732
0.0589020252227783 -0.0816687345504761
0.0629703253507614 -0.0929249040782452
0.0675316900014877 -0.104129485785961
0.0725371092557907 -0.114205978810787
0.0778444111347198 -0.124735150486231
0.0837338119745255 -0.134733699262142
0.0899235159158707 -0.144064374268055
0.0966654568910599 -0.153286181390285
0.103846162557602 -0.161697596311569
0.111178055405617 -0.169978901743889
0.119257673621178 -0.177585430443287
0.12748196721077 -0.184751696884632
0.136418521404266 -0.191873259842396
0.145147070288658 -0.197547659277916
0.154542565345764 -0.202983319759369
0.164455115795135 -0.208080038428307
0.174046218395233 -0.212053671479225
0.184972584247589 -0.216125249862671
0.195491641759872 -0.218932345509529
0.206821739673615 -0.221703484654427
0.218109399080276 -0.223292663693428
0.230260640382767 -0.224675074219704
0.242429584264755 -0.225142404437065
0.25468122959137 -0.224913030862808
0.267553359270096 -0.224219486117363
0.280453264713287 -0.222812429070473
0.293543308973312 -0.221136227250099
0.306498199701309 -0.217942014336586
0.320220142602921 -0.214426413178444
0.334239363670349 -0.21036085486412
0.348620295524597 -0.205623835325241
0.363394141197205 -0.20053568482399
0.377981781959534 -0.194517776370049
0.392620861530304 -0.187698550522327
0.407895803451538 -0.180756539106369
0.423239171504974 -0.173181563615799
0.438636541366577 -0.165069989860058
0.454354107379913 -0.156046964228153
0.469962239265442 -0.147136002779007
0.4861119389534 -0.136845767498016
0.501993715763092 -0.126747455447912
0.518263578414917 -0.116114132106304
0.534630715847015 -0.105251006782055
0.550857305526733 -0.0938975177705288
0.56750762462616 -0.0819668844342232
0.583892583847046 -0.0699179023504257
0.600501477718353 -0.0573937743902206
0.617007255554199 -0.0446676295250654
0.633709132671356 -0.0323262214660645
0.650585949420929 -0.0190182849764824
0.667313754558563 -0.00569628179073334
0.68428635597229 0.00763339549303055
0.70129269361496 0.0211636871099472
0.718463540077209 0.0351196303963661
0.73504775762558 0.0482795163989067
};
\addplot [semithick, red, dashed, forget plot]
table {%
0.75 0
0.767617956255685 0.0126152320166048
0.785221863675294 0.025202759744927
0.802800914680889 0.0377306780229099
0.820344353102624 0.0501672114827695
0.837841479099983 0.0624807937495562
0.855281653865309 0.0746401456242545
0.872654304077727 0.0866143520292023
0.889948926065816 0.0983729374930471
0.90715508962574 0.109885939954111
0.924262441427357 0.121123982665618
0.941260707923679 0.132058343994744
0.958139697658408 0.142661024921306
0.974889302841815 0.152904814062882
0.991499500036706 0.162763350083898
1.00796034976389 0.172211181389763
1.02426199480132 0.18122382306779
1.04039465691463 0.189777811119373
1.0563486317231 0.197850754138649
1.07211428137931 0.205421382738366
1.08768202473299 0.212469597210257
1.10304232467243 0.218976514139914
1.11818567241108 0.224924512976024
1.1331025686362 0.230297283873765
1.147783501695 0.23507987847157
1.16221892339532 0.239258765576722
1.17639922357808 0.242821893954846
1.19031470539574 0.245758764427774
1.20395556419321 0.248060513126141
1.21731187395941 0.249720006825457
1.23037358633393 0.250731949622618
1.24313054782987 0.251092997654816
1.2555725408717 0.250801875167454
1.26768935298946 0.24985948134918
1.27947087567813 0.248268973729967
1.29090722994702 0.246035811751358
1.30198890990496 0.243167744731957
1.31270692999354 0.2396747329547
1.32305295737637 0.235568799170581
1.33301941026506 0.230863819194334
1.34259950666846 0.225575271775738
1.3517872558846 0.219719976188812
1.36057739518197 0.213315848229081
1.36896528374308 0.206381700652654
1.37694677240494 0.198937103974783
1.38451806954475 0.191002311131823
1.39167562078374 0.182598238316676
1.39841601450881 0.173746486857157
1.4047359185686 0.164469388195659
1.41063204764278 0.154790055213092
1.41610115672778 0.144732426840967
1.42114005415556 0.134321297531182
1.42574562723652 0.123582327489247
1.42991487441761 0.112542032954869
1.43364493919855 0.101227758043132
1.43693314251624 0.089667630857217
1.4397770116222 0.0778905070053025
1.44217430452097 0.0659259035746353
1.44412302978876 0.0538039262643269
1.44562146208725 0.0415551919190076
1.44666815398553 0.0292107482394724
1.44726194486432 0.016801992026761
1.44740196775303 0.00436058696240971
1.44708765498222 -0.00808161836067393
1.44631874355061 -0.0204926716768206
1.4450952811249 -0.0328405972734807
1.44341763362016 -0.0450934751015305
1.44128649535024 -0.0572195188465802
1.4387029027798 -0.0691871532341023
1.43566825293285 -0.0809650910678666
1.4321843274821 -0.0925224107662305
1.42825332340439 -0.103828635462315
1.42387789076703 -0.114853815053745
1.41906117760902 -0.125568612881875
1.41380688088891 -0.135944398910193
1.40811930097384 -0.145953351233135
1.40200339507468 -0.155568567310703
1.39546482242266 -0.164764185292558
1.38850997106277 -0.173515513988717
1.38114595342609 -0.181799167392522
1.37338055620066 -0.189593196333819
1.36522213060654 -0.196877206377916
1.35667941319178 -0.203632448466232
1.34776127545165 -0.209841868311401
1.33847641259836 -0.215490103434114
1.328832995749 -0.220563423487838
1.3188383241811 -0.225049619377379
1.30849852098817 -0.228937857346307
1.29781831326193 -0.232218522442798
1.28680092623794 -0.234883078640104
1.27544810236402 -0.23692396914816
1.26376023613114 -0.238334571253006
1.25173659926596 -0.239109208409727
1.23937562223503 -0.239243211851919
1.22680212890194 -0.238807395220756
1.21393543612189 -0.237739020519201
1.2007824030094 -0.236042071113405
1.1873502195155 -0.233721342391173
1.17364639818837 -0.230782623136066
1.15967876862552 -0.227232843837933
1.14545546822952 -0.223080191387264
1.13098492732646 -0.218334192018305
1.11627584919941 -0.213005765794969
1.10133718666704 -0.207107256561536
1.08617811704058 -0.200652441341106
1.070808017043 -0.19365652287622
1.05523643885516 -0.186136108539406
1.03947308802467 -0.178109178316594
1.02352780361101 -0.169595044057188
1.00741054066936 -0.160614301730894
0.991131354992172 -0.151188778049092
0.974700389916618 -0.141341472499412
0.958127864949596 -0.131096495600034
0.941424065943487 -0.120479003995558
0.924599336561409 -0.109515132878708
0.90766407079076 -0.0982319261219058
0.890628706290982 -0.086657264431339
0.873503718391248 -0.074819791786618
0.856299614583182 -0.0627488403957748
0.839026929381251 -0.0504743543738914
0.8216962194482 -0.0380268123406879
0.804318058904672 -0.0254371491254941
0.786903034760972 -0.0127366767652624
0.769461742425004 4.29950186324069e-05
0.752004781254135 0.0128700386771497
0.734542750130382 0.0257124888331812
0.717086243048227 0.0385383222295833
0.699645844712938 0.0513155374272836
0.682232126154625 0.0640122340511521
0.664855640369735 0.0765966913776288
0.647526918007365 0.0890374460548427
0.630256463122704 0.101303368743397
0.613054749024055 0.113363739464377
0.595932214243 0.125188321440869
0.578899258658929 0.136747433220813
0.561966239808458 0.148012018873052
0.545143469406046 0.158953716055882
0.528441210092295 0.169544921769502
0.511869672407924 0.179758855622043
0.495439011959919 0.189569620465462
0.479159326695086 0.198952260295039
0.463040654116175 0.207882815357607
0.447092968153461 0.216338374482455
0.431326175222446 0.224297124738732
0.41575010873165 0.231738398637223
0.400374520922108 0.238642719233979
0.38520907038282 0.244991843656404
0.37026330284855 0.250768805749989
0.355546621900637 0.255957958716022
0.341068244918601 0.260545018739871
0.326837138059129 0.264517110634022
0.312861922218046 0.267862816346776
0.299150740017787 0.270572226688282
0.285711072187049 0.272636995643726
0.272549490830637 0.274050395018263
0.259671337879835 0.274807364778616
0.247080320574249 0.274904551350881
0.234778023333987 0.274340322593555
0.22282111677237 0.273147174350289
0.211212000441657 0.271312180989548
0.199956090718924 0.268842225658862
0.18905951088657 0.265745762806622
0.178528814558833 0.262032850875418
0.168370684504055 0.257715152729482
0.158591652631679 0.252805894649178
0.149197881311021 0.247319783843054
0.140195031151771 0.241272894158711
0.131588220396365 0.234682536784361
0.123382062255884 0.227567134929123
0.115580754025762 0.219946118404367
0.108188187942631 0.211839847297465
0.101208057557953 0.203269566144989
0.0946439418448079 0.19425738357691
0.088499358802492 0.184826268669802
0.0827777883280161 0.17500005435619
0.0774826693237491 0.164803439481848
0.0726173783628224 0.154261983466959
0.0681851974057714 0.143402090111052
0.0641892769319838 0.132250979303424
0.0606325991839522 0.12083664700672
0.0575179445429576 0.109187814856514
0.0548478626398547 0.0973338711778623
0.0526246487503009 0.0853048053160545
0.0508503253165833 0.0731311370573394
0.0495266280133111 0.0608438426868979
0.0486549955532945 0.0484742789692179
0.0482365623412967 0.036054106082026
0.0482721530701452 0.0236152103073717
0.0487622783751904 0.0111896270851532
0.0497071306928156 -0.00119053514168032
0.0511065794918781 -0.0134931700273918
0.0529601650583767 -0.0256862472791051
0.0552670900164109 -0.0377378855745846
0.0580262077740067 -0.0496164247880426
0.061236007111159 -0.0612904981681536
0.0648945922107272 -0.0727291054819839
0.0689996576140302 -0.0839016885812368
0.0735484579177227 -0.0947782113457813
0.0785377725810335 -0.10532924646507
0.0839638670453019 -0.11552607193497
0.0898224525211534 -0.125340780318524
0.0961086482561191 -0.134746403519034
0.10281695173701 -0.143717054765618
0.109941223832714 -0.152228087440402
0.117474696885182 -0.160256267121569
0.125410013591915 -0.167779948892829
0.133739302545796 -0.174779247148552
0.14245429208538 -0.181236180932355
0.151546457798383 -0.187134775860053
0.161007191589854 -0.192461105490089
0.170827973498639 -0.197203261519913
0.181000523658031 -0.201351252889932
0.191516912754772 -0.204896846530735
0.202369615510414 -0.207833373632873
0.213551501760898 -0.210155531524887
0.22505577088336 -0.211859210579139
0.236875844499746 -0.212941368432345
0.249005237340447 -0.213399962707235
0.261437426261433 -0.213233941776159
0.274165733616003 -0.212443283869614
0.28718323530924 -0.211029069576408
0.300482697779478 -0.208993571566475
0.314056543186631 -0.206340347198259
0.327896838839168 -0.203074323164878
0.341995305342881 -0.199201865241865
0.356343337731412 -0.194730829684218
0.370932034454633 -0.189670595476494
0.385752230119493 -0.184032078393973
0.400794528982751 -0.177827728806396
0.416049337197401 -0.171071515546711
0.431506892628147 -0.16377889817583
0.447157291657684 -0.15596678976561
0.462990512824106 -0.147653512015245
0.478996437395876 -0.138858744187699
0.495164867142796 -0.129603467046443
0.511485539633279 -0.119909902709476
0.527948141407569 -0.109801451123518
0.544542319363787 -0.0993026236948103
0.561257690663388 -0.0884389744880547
0.578083851424296 -0.0772370293146803
0.595010384429933 -0.0657242129685941
0.612026866043939 -0.0539287748255556
0.62912287248563 -0.0418797129959582
0.646287985590849 -0.029606697205851
0.663511798156965 -0.0171399905741775
0.680783918949166 -0.00451037045289776
0.698093977427272 0.00825095150093907
0.715431628237671 0.0211124098473716
0.732786555502968 0.0340421663000904
};
\addplot [semithick, green, dash pattern=on 1pt off 3pt on 3pt off 3pt, forget plot]
table {%
0.75 0
0.767347219519284 0.012137253483567
0.784683394455509 0.0242435861667548
0.801997486582194 0.036288155514001
0.819278471531621 0.0482402761288358
0.836515345700615 0.0600694978228124
0.853697133135913 0.0717456830897325
0.870812892392119 0.0832390837898383
0.887851723355235 0.0945204168511733
0.904802774024683 0.105560938798497
0.921655247246804 0.116332518924018
0.938398407392915 0.126807710918802
0.955021586975309 0.136959822789028
0.971514193195148 0.146762984887382
0.987865714417124 0.156192215896701
1.0040657265673 0.165223486610677
1.02010389945277 0.173833781364724
1.03597000300535 0.182001156979107
1.05165391345605 0.189704799085743
1.06714561945411 0.196925075719347
1.08243522815337 0.203643588062188
1.09751297130037 0.209843218238558
1.1123692113737 0.215508174058654
1.12699444784162 0.220624030609957
1.14137932362429 0.225177768584601
1.1555146318663 0.22915780921071
1.16939132313901 0.232554045621132
1.18300051319534 0.235357870442502
1.19633349138035 0.237562199321834
1.20938172974852 0.23916149003295
1.22213689283924 0.240151756735333
1.23459084790774 0.240530578918402
1.24673567520343 0.24029710459028
1.25856367765532 0.239452047402427
1.27006738911946 0.237997677671498
1.28123958024499 0.235937807669022
1.29207326111455 0.233277772045945
1.30256168017249 0.230024404727089
1.31269831956044 0.226186013888674
1.32247688771958 0.221772356568193
1.33189131078413 0.216794613983565
1.34093572465289 0.211265367835258
1.34960446953272 0.205198576949498
1.35789208820684 0.198609552872807
1.36579332846363 0.191514932670796
1.37330314928634 0.18393264729003
1.38041672978698 0.175881884318464
1.38712947958844 0.167383044631953
1.39343704941026 0.158457693037802
1.39933534088867 0.149128503482261
1.40482051503024 0.139419199628507
1.40988899904863 0.129354491659788
1.41453749160152 0.118960010081206
1.41876296661194 0.108262237150103
1.42256267593454 0.0972884364110782
1.42593415113889 0.086066580678514
1.42887520465437 0.0746252787098322
1.43138393047621 0.0629937007473294
1.43345870458359 0.0512015030695605
1.43509818517756 0.0392787516771522
1.43630131281055 0.0272558452352367
1.43706731045357 0.0151634373994425
1.43739568352819 0.00303235866024866
1.43728621991849 -0.00910646215137196
1.43673898997073 -0.0212220765475008
1.43575434648365 -0.0332835951024502
1.43433292468914 -0.0452602660529673
1.43247564221903 -0.0571215536269705
1.43018369904855 -0.0688372159533007
1.42745857739782 -0.0803773824121825
1.4243020415584 -0.0917126302942692
1.42071613759078 -0.102814060643163
1.41670319280875 -0.113653373158817
1.41226581492764 -0.124202940032363
1.40740689070718 -0.134435878560225
1.40212958387022 -0.144326122339201
1.3964373320371 -0.153848490767307
1.39033384239815 -0.162978756463708
1.38382308587764 -0.171693710079292
1.37690928964889 -0.179971221816331
1.36959692806383 -0.187790298849553
1.36189071236441 -0.195131137800147
1.35379557991389 -0.201975171526458
1.34531668404164 -0.20830510981277
1.33645938581161 -0.214104974062127
1.32722924897007 -0.219360126749639
1.31763203892423 -0.224057297001799
1.30767372588241 -0.22818460402503
1.2973604914176 -0.231731580045289
1.28669873695777 -0.234689193904312
1.27569509230234 -0.237049875620184
1.26435642233521 -0.238807541316647
1.25268983059913 -0.239957617226964
1.24070265912099 -0.240497061156197
1.22840248459827 -0.24042437986202
1.21579711159144 -0.239739641182038
1.20289456363565 -0.238444480231447
1.1897030732081 -0.236542099469318
1.17623107134149 -0.234037262792463
1.16248717745014 -0.230936284032406
1.14848018970572 -0.227247010319085
1.13421907610845 -0.222978800771493
1.11971296626323 -0.218142500920566
1.10497114378721 -0.212750413195198
1.09000303923492 -0.206816263729265
1.07481822341698 -0.200355165686855
1.05942640099553 -0.193383579258294
1.04383740425637 -0.185919268450478
1.02806118697728 -0.177981254778958
1.01210781833129 -0.169589767962932
0.995987476780275 -0.160766193724954
0.979710443928391 -0.151533018802209
0.963287098315684 -0.141913773283855
0.946727909140567 -0.131932970397911
0.930043429905878 -0.121616043880467
0.91324429198767 -0.110989283069269
0.896341198128817 -0.100079765872392
0.879344915861624 -0.0889152897708354
0.862266270864871 -0.0775243010211479
0.845116140261585 -0.0659358222307342
0.827905445864243 -0.0541793784842489
0.810645147374399 -0.042284922204474
0.793346235543765 -0.0302827569353333
0.776019725303823 -0.0182034602382608
0.758676648870971 -0.00607780589600706
0.74132804883419 0.00606331437980999
0.723984971232126 0.0181889695395348
0.706658458626486 0.0302682678858693
0.689359543178601 0.0422704356646973
0.672099239736024 0.0541648953575106
0.654888538936098 0.0659213434775336
0.637738400333409 0.077509827673207
0.620659745558167 0.0889008229448788
0.603663451512546 0.10006530678342
0.586760343612044 0.110974833042994
0.569961189078862 0.121601604364455
0.55327669029407 0.131918542970798
0.536717478214957 0.141899359661781
0.520294105863188 0.151518620841353
0.504017041888231 0.160751813418715
0.487896664208613 0.169575407431921
0.471943253730781 0.177966916251536
0.456166988141264 0.18590495423109
0.440577935762111 0.193369291680399
0.425186049451712 0.200340907046881
0.410001160522753 0.206802036197834
0.395032972635676 0.212736218702091
0.38029105560979 0.218128341010777
0.36578483907538 0.222964676431608
0.351523605870586 0.227232921776542
0.337516485069478 0.23092223053541
0.323772444518238 0.234023242385741
0.310300282763089 0.236528108790522
0.297108620287721 0.238430514364134
0.284205890052545 0.239725693611976
0.27160032745433 0.240410442590784
0.259299960006205 0.240483125024951
0.247312597261469 0.239943672488383
0.2356458217318 0.238793578459958
0.224306981713199 0.237035886402939
0.21330318694315 0.234675172482025
0.202641307788403 0.231717524032126
0.192327978173203 0.228170515287301
0.1823696017639 0.22404318200088
0.172772360193836 0.219345996319791
0.163542221572307 0.214090842618848
0.154684947376684 0.208290994107505
0.146206096155645 0.201961089155653
0.138111023180157 0.195117105712544
0.130404876038488 0.187776332069152
0.12309258692145 0.179957332525672
0.116178862798813 0.171679907117356
0.109668174793245 0.162965045212245
0.10356474787866 0.153834873345794
0.0978725516915446 0.144312598005936
0.0925952928735381 0.134422444219678
0.0877364090495248 0.124189590767097
0.0832990643281257 0.113640102728524
0.0792861460924091 0.102800861916777
0.0757002628080311 0.0916994956001386
0.0725437425870994 0.0803643038046822
0.069818632284583 0.0688241854024739
0.0675266969522104 0.0571085631419612
0.0656694195214907 0.0452473077513642
0.0642480006271671 0.0332706612373471
0.0632633585132331 0.0212091595029022
0.0627161289858991 0.00909355441514592
0.0626066653929575 -0.00304526453801165
0.0629350386185861 -0.0151763488558262
0.0637010370885092 -0.0272687697454203
0.0649041667842718 -0.0392916968007361
0.0665436512687222 -0.0512144764858195
0.0686184317292081 -0.0630067102720941
0.0711271670520408 -0.0746383322854955
0.0740682339532131 -0.0860796863272447
0.0774397272079407 -0.0973016021399556
0.0812394600470219 -0.108275470796062
0.0854649648223936 -0.118973319084073
0.0901134940873028 -0.12936788275431
0.0951820222852561 -0.139432678452301
0.100667248289028 -0.149142074107302
0.10656559906303 -0.158471357449301
0.112873234718821 -0.167396802199979
0.119586055168304 -0.175895731332128
0.126699708426136 -0.183946576646328
0.13420960035749 -0.191528933821879
0.142110905322839 -0.198623612126504
0.150398576795015 -0.205212678179984
0.159067356720643 -0.211279493592426
0.16811178230224 -0.216808746901301
0.177526189102462 -0.221786480886398
0.187304709947427 -0.226200116851138
0.197441269929379 -0.230038477621279
0.207929578656845 -0.233291810717099
0.218763121506754 -0.235951812451675
0.229935151803763 -0.238011652801433
0.241438685552319 -0.239466000064469
0.253266499708268 -0.240311043799429
0.265411134226006 -0.240544514421553
0.277864897472831 -0.240165698073282
0.290619874197545 -0.23917544583971
0.303667935101709 -0.237576176879925
0.317000747134867 -0.235371875470324
0.330609783830706 -0.232568082244952
0.344486335235128 -0.229171880066449
0.358621517191145 -0.225191874998346
0.373006279910045 -0.220638172815656
0.387631415867 -0.215522351422736
0.402487567119295 -0.209857429471725
0.417565232169048 -0.203657831407191
0.432854772492479 -0.196939349109845
0.448346418844808 -0.189719100275549
0.46403027743119 -0.182015483643688
0.47989633601438 -0.173848131178146
0.495934470011695 -0.165237857301697
0.512134448618413 -0.156206605287707
0.528485940982374 -0.146777390919585
0.544978522444945 -0.136974243536887
0.561601680856407 -0.126822144596177
0.578344822968638 -0.116346963884112
0.595197280904322 -0.105575393529208
0.61214831869946 -0.094534879967153
0.629187138914278 -0.0832535540222168
0.646302889306651 -0.071760159274263
0.663484669561462 -0.0600839788869683
0.680721538069075 -0.0482547610782518
0.698002518745858 -0.0363026434185232
0.715316607889725 -0.0242580761462783
0.732652781063624 -0.012151744693772
};
\addplot [line width = \linewidthEightC, color = reference, opacity=\opacityRef, forget plot]
table {%
0.75 0
0.752633810043335 0.00239118933677673
0.767792522907257 0.0132435262203217
0.786620378494263 0.0262514352798462
0.803508639335632 0.0381441116333008
0.821437895298004 0.0507908463478088
0.838925659656525 0.0624682307243347
0.856595516204834 0.0747478753328323
0.874005377292633 0.0862622410058975
0.891082882881165 0.0977051854133606
0.908187985420227 0.108892172574997
0.925426125526428 0.120275020599365
0.942245721817017 0.131172969937325
0.959248661994934 0.141595125198364
0.976168751716614 0.151405841112137
0.992928266525269 0.160816729068756
1.0092362165451 0.169821977615356
1.02554762363434 0.178417399525642
1.04178428649902 0.186493799090385
1.0576194524765 0.194066509604454
1.07343685626984 0.201079800724983
1.08893203735352 0.207648977637291
1.10415840148926 0.213754937052727
1.11941981315613 0.218960925936699
1.13440465927124 0.224097952246666
1.14896380901337 0.228377357125282
1.16323292255402 0.23188354074955
1.17728555202484 0.234770074486732
1.19096732139587 0.237272754311562
1.20468974113464 0.239181414246559
1.2175600528717 0.240280404686928
1.23033702373505 0.240765437483788
1.24255657196045 0.240452274680138
1.25479280948639 0.239632025361061
1.26656579971313 0.238486841320992
1.27784693241119 0.236332580447197
1.28899097442627 0.233495488762856
1.29970240592957 0.230320140719414
1.31004798412323 0.226645246148109
1.32025742530823 0.221464917063713
1.32977986335754 0.21699084341526
1.3389630317688 0.211618825793266
1.34827327728271 0.205248489975929
1.35657584667206 0.198604866862297
1.36493396759033 0.191435232758522
1.37269163131714 0.183598086237907
1.38006377220154 0.17564956843853
1.38648724555969 0.167174115777016
1.39345383644104 0.158048078417778
1.39957010746002 0.148156985640526
1.40521514415741 0.138415321707726
1.41063058376312 0.127971068024635
1.41511940956116 0.117246925830841
1.41955387592316 0.106340050697327
1.42305147647858 0.0954943001270294
1.42649459838867 0.083749532699585
1.42936182022095 0.0723376721143723
1.43171679973602 0.0605492889881134
1.43367052078247 0.0484755486249924
1.43560838699341 0.0364347994327545
1.4367595911026 0.0238184332847595
1.4372900724411 0.0109444260597229
1.43778610229492 -0.00115317851305008
1.43739128112793 -0.0134061202406883
1.43682479858398 -0.0257412381470203
1.43564343452454 -0.038092527538538
1.43416321277618 -0.0505985449999571
1.43219888210297 -0.0623053163290024
1.42985534667969 -0.0743778953328729
1.42691922187805 -0.0861758254468441
1.42363655567169 -0.09805360250175
1.41985881328583 -0.108881670981646
1.41589558124542 -0.119992010295391
1.41128480434418 -0.13076027482748
1.40619504451752 -0.141419753432274
1.40087056159973 -0.151385933160782
1.39505743980408 -0.161045290529728
1.38881242275238 -0.170637592673302
1.38188993930817 -0.179680988192558
1.3747935295105 -0.188294462859631
1.36712515354156 -0.196540609002113
1.3594388961792 -0.204202070832253
1.35095798969269 -0.211112469434738
1.34234702587128 -0.217620551586151
1.33338391780853 -0.22372479736805
1.32394659519196 -0.229094639420509
1.31426906585693 -0.233497142791748
1.30426466464996 -0.237587288022041
1.29375600814819 -0.241017639636993
1.28279459476471 -0.243849754333496
1.2716588973999 -0.24605867266655
1.26004409790039 -0.247364118695259
1.24823009967804 -0.248392581939697
1.23627603054047 -0.248657390475273
1.22384679317474 -0.248403385281563
1.21090877056122 -0.247370287775993
1.19790995121002 -0.24586521089077
1.18440806865692 -0.243734210729599
1.17095828056335 -0.240943908691406
1.15692603588104 -0.237302988767624
1.14253842830658 -0.233215525746346
1.12811410427094 -0.228292316198349
1.11347699165344 -0.223129317164421
1.09870433807373 -0.217082977294922
1.08374309539795 -0.210530549287796
1.06860733032227 -0.203181236982346
1.05346179008484 -0.195800706744194
1.03815162181854 -0.187580913305283
1.02230393886566 -0.178853414952755
1.00646758079529 -0.169918790459633
0.990123271942139 -0.160154528915882
0.973961353302002 -0.149704933166504
0.957704305648804 -0.139216430485249
0.941193222999573 -0.128285229206085
0.924591660499573 -0.11700776219368
0.90765917301178 -0.105323579162359
0.890868782997131 -0.0934291183948517
0.8738893866539 -0.0813637282699347
0.857009291648865 -0.0690190047025681
0.839958250522614 -0.0563636235892773
0.822821140289307 -0.0437034890055656
0.805786728858948 -0.0311339050531387
0.788353502750397 -0.0176520720124245
0.771072208881378 -0.00432975590229034
0.753897607326508 0.00883330404758453
0.736701667308807 0.022136278450489
0.719219207763672 0.0354094058275223
0.702241241931915 0.0483253002166748
0.685175955295563 0.0614152401685715
0.667941927909851 0.0744083970785141
0.65108460187912 0.0870069861412048
0.63425487279892 0.0997144728899002
0.617207050323486 0.112391591072083
0.60044652223587 0.12432062625885
0.583423376083374 0.136880904436111
0.566727578639984 0.148403346538544
0.550312280654907 0.159586638212204
0.533609747886658 0.170812517404556
0.517342150211334 0.181201294064522
0.501252889633179 0.191488191485405
0.485179424285889 0.201540872454643
0.469333589076996 0.210719600319862
0.453409612178802 0.219530954957008
0.437322795391083 0.22822667658329
0.421781897544861 0.236315622925758
0.406530976295471 0.243370369076729
0.391164213418961 0.250629231333733
0.375976830720901 0.25690196454525
0.361043959856033 0.2627042979002
0.346401035785675 0.267843171954155
0.331965774297714 0.272469982504845
0.318088948726654 0.276431366801262
0.30431205034256 0.279800042510033
0.290749728679657 0.282241240143776
0.277199536561966 0.284426763653755
0.264228165149689 0.285717949271202
0.251374810934067 0.286550268530846
0.238919526338577 0.286757871508598
0.227029144763947 0.285998448729515
0.214786320924759 0.285119161009789
0.203588247299194 0.283120140433311
0.192256033420563 0.280711010098457
0.181350648403168 0.277593478560448
0.170866757631302 0.274061813950539
0.160586953163147 0.26950441300869
0.151314660906792 0.264440074563026
0.141717746853828 0.259258076548576
0.132902532815933 0.253093883395195
0.12421427667141 0.246554687619209
0.115907967090607 0.239400818943977
0.108348250389099 0.231681272387505
0.101003706455231 0.223704323172569
0.0941406935453415 0.215211912989616
0.0875402987003326 0.206129476428032
0.0814526230096817 0.196789935231209
0.0757084488868713 0.186676785349846
0.0703461915254593 0.176084592938423
0.0655478537082672 0.165886044502258
0.0611743032932281 0.155345171689987
0.0571587979793549 0.143430158495903
0.0534646064043045 0.132404133677483
0.050258606672287 0.120704427361488
0.0474266111850739 0.108982801437378
0.0453597903251648 0.0966313928365707
0.0437350869178772 0.0840993374586105
0.0422695875167847 0.0718294233083725
0.0413115471601486 0.0592411160469055
0.0410134345293045 0.0463525056838989
0.0408090353012085 0.0344531387090683
0.0411886274814606 0.022054634988308
0.0421821177005768 0.00965015590190887
0.0435965061187744 -0.00221201032400131
0.0456737279891968 -0.0152500420808792
0.0478999018669128 -0.0268780514597893
0.0505745410919189 -0.0383910909295082
0.0538726150989532 -0.0500266384333372
0.0577450692653656 -0.0614226497709751
0.0618980973958969 -0.0726633872836828
0.0664279609918594 -0.0836761724203825
0.0714015364646912 -0.094080151990056
0.0767652094364166 -0.104090025648475
0.0825167447328568 -0.113976135849953
0.0886355042457581 -0.123298782855272
0.0954150855541229 -0.131836540997028
0.102740257978439 -0.140511341392994
0.110182613134384 -0.148406460881233
0.117861747741699 -0.155782409012318
0.125947281718254 -0.162603840231895
0.135187432169914 -0.169473066926003
0.144238039851189 -0.175047308206558
0.153779193758965 -0.180496737360954
0.163571745157242 -0.184703961014748
0.174005836248398 -0.189089827239513
0.184571146965027 -0.192616693675518
0.195204377174377 -0.1955187022686
0.206843346357346 -0.197928115725517
0.218167513608932 -0.199469991028309
0.230112761259079 -0.200183190405369
0.242794692516327 -0.201209202408791
0.25558015704155 -0.200496345758438
0.268798023462296 -0.199763022363186
0.282143622636795 -0.198157593607903
0.295737236738205 -0.195824272930622
0.309526830911636 -0.192901238799095
0.324174076318741 -0.189634643495083
0.338372200727463 -0.185476399958134
0.352517038583755 -0.180654264986515
0.367588639259338 -0.175382614135742
0.382658243179321 -0.16948563605547
0.39822593331337 -0.162908278405666
0.414019465446472 -0.156158596277237
0.429653406143188 -0.148217886686325
0.445575714111328 -0.140262447297573
0.461922407150269 -0.131556913256645
0.478192746639252 -0.122294194996357
0.494631350040436 -0.112682104110718
0.511377334594727 -0.102979747578502
0.528168678283691 -0.0925100743770599
0.544598340988159 -0.0816568359732628
0.561732769012451 -0.0705525130033493
0.578792095184326 -0.0596139915287495
0.59576952457428 -0.0474897772073746
0.613142848014832 -0.0357812121510506
0.630119204521179 -0.0232245922088623
0.647511005401611 -0.0111461579799652
0.664871037006378 0.00179444253444672
0.682158827781677 0.0143090263009071
0.699631869792938 0.0271831750869751
0.71708619594574 0.0396513044834137
0.734466433525085 0.0524169951677322
};
\addplot [semithick, red, dashed, forget plot]
table {%
0.75 0
0.767617956255685 0.0126152320166048
0.785221863675294 0.025202759744927
0.802800914680889 0.0377306780229099
0.820344353102624 0.0501672114827695
0.837841479099983 0.0624807937495562
0.855281653865309 0.0746401456242545
0.872654304077727 0.0866143520292023
0.889948926065816 0.0983729374930471
0.90715508962574 0.109885939954111
0.924262441427357 0.121123982665618
0.941260707923679 0.132058343994744
0.958139697658408 0.142661024921306
0.974889302841815 0.152904814062882
0.991499500036706 0.162763350083898
1.00796034976389 0.172211181389763
1.02426199480132 0.18122382306779
1.04039465691463 0.189777811119373
1.0563486317231 0.197850754138649
1.07211428137931 0.205421382738366
1.08768202473299 0.212469597210257
1.10304232467243 0.218976514139914
1.11818567241108 0.224924512976024
1.1331025686362 0.230297283873765
1.147783501695 0.23507987847157
1.16221892339532 0.239258765576722
1.17639922357808 0.242821893954846
1.19031470539574 0.245758764427774
1.20395556419321 0.248060513126141
1.21731187395941 0.249720006825457
1.23037358633393 0.250731949622618
1.24313054782987 0.251092997654816
1.2555725408717 0.250801875167454
1.26768935298946 0.24985948134918
1.27947087567813 0.248268973729967
1.29090722994702 0.246035811751358
1.30198890990496 0.243167744731957
1.31270692999354 0.2396747329547
1.32305295737637 0.235568799170581
1.33301941026506 0.230863819194334
1.34259950666846 0.225575271775738
1.3517872558846 0.219719976188812
1.36057739518197 0.213315848229081
1.36896528374308 0.206381700652654
1.37694677240494 0.198937103974783
1.38451806954475 0.191002311131823
1.39167562078374 0.182598238316676
1.39841601450881 0.173746486857157
1.4047359185686 0.164469388195659
1.41063204764278 0.154790055213092
1.41610115672778 0.144732426840967
1.42114005415556 0.134321297531182
1.42574562723652 0.123582327489247
1.42991487441761 0.112542032954869
1.43364493919855 0.101227758043132
1.43693314251624 0.089667630857217
1.4397770116222 0.0778905070053025
1.44217430452097 0.0659259035746353
1.44412302978876 0.0538039262643269
1.44562146208725 0.0415551919190076
1.44666815398553 0.0292107482394724
1.44726194486432 0.016801992026761
1.44740196775303 0.00436058696240971
1.44708765498222 -0.00808161836067393
1.44631874355061 -0.0204926716768206
1.4450952811249 -0.0328405972734807
1.44341763362016 -0.0450934751015305
1.44128649535024 -0.0572195188465802
1.4387029027798 -0.0691871532341023
1.43566825293285 -0.0809650910678666
1.4321843274821 -0.0925224107662305
1.42825332340439 -0.103828635462315
1.42387789076703 -0.114853815053745
1.41906117760902 -0.125568612881875
1.41380688088891 -0.135944398910193
1.40811930097384 -0.145953351233135
1.40200339507468 -0.155568567310703
1.39546482242266 -0.164764185292558
1.38850997106277 -0.173515513988717
1.38114595342609 -0.181799167392522
1.37338055620066 -0.189593196333819
1.36522213060654 -0.196877206377916
1.35667941319178 -0.203632448466232
1.34776127545165 -0.209841868311401
1.33847641259836 -0.215490103434114
1.328832995749 -0.220563423487838
1.3188383241811 -0.225049619377379
1.30849852098817 -0.228937857346307
1.29781831326193 -0.232218522442798
1.28680092623794 -0.234883078640104
1.27544810236402 -0.23692396914816
1.26376023613114 -0.238334571253006
1.25173659926596 -0.239109208409727
1.23937562223503 -0.239243211851919
1.22680212890194 -0.238807395220756
1.21393543612189 -0.237739020519201
1.2007824030094 -0.236042071113405
1.1873502195155 -0.233721342391173
1.17364639818837 -0.230782623136066
1.15967876862552 -0.227232843837933
1.14545546822952 -0.223080191387264
1.13098492732646 -0.218334192018305
1.11627584919941 -0.213005765794969
1.10133718666704 -0.207107256561536
1.08617811704058 -0.200652441341106
1.070808017043 -0.19365652287622
1.05523643885516 -0.186136108539406
1.03947308802467 -0.178109178316594
1.02352780361101 -0.169595044057188
1.00741054066936 -0.160614301730894
0.991131354992172 -0.151188778049092
0.974700389916618 -0.141341472499412
0.958127864949596 -0.131096495600034
0.941424065943487 -0.120479003995558
0.924599336561409 -0.109515132878708
0.90766407079076 -0.0982319261219058
0.890628706290982 -0.086657264431339
0.873503718391248 -0.074819791786618
0.856299614583182 -0.0627488403957748
0.839026929381251 -0.0504743543738914
0.8216962194482 -0.0380268123406879
0.804318058904672 -0.0254371491254941
0.786903034760972 -0.0127366767652624
0.769461742425004 4.29950186324069e-05
0.752004781254135 0.0128700386771497
0.734542750130382 0.0257124888331812
0.717086243048227 0.0385383222295833
0.699645844712938 0.0513155374272836
0.682232126154625 0.0640122340511521
0.664855640369735 0.0765966913776288
0.647526918007365 0.0890374460548427
0.630256463122704 0.101303368743397
0.613054749024055 0.113363739464377
0.595932214243 0.125188321440869
0.578899258658929 0.136747433220813
0.561966239808458 0.148012018873052
0.545143469406046 0.158953716055882
0.528441210092295 0.169544921769502
0.511869672407924 0.179758855622043
0.495439011959919 0.189569620465462
0.479159326695086 0.198952260295039
0.463040654116175 0.207882815357607
0.447092968153461 0.216338374482455
0.431326175222446 0.224297124738732
0.41575010873165 0.231738398637223
0.400374520922108 0.238642719233979
0.38520907038282 0.244991843656404
0.37026330284855 0.250768805749989
0.355546621900637 0.255957958716022
0.341068244918601 0.260545018739871
0.326837138059129 0.264517110634022
0.312861922218046 0.267862816346776
0.299150740017787 0.270572226688282
0.285711072187049 0.272636995643726
0.272549490830637 0.274050395018263
0.259671337879835 0.274807364778616
0.247080320574249 0.274904551350881
0.234778023333987 0.274340322593555
0.22282111677237 0.273147174350289
0.211212000441657 0.271312180989548
0.199956090718924 0.268842225658862
0.18905951088657 0.265745762806622
0.178528814558833 0.262032850875418
0.168370684504055 0.257715152729482
0.158591652631679 0.252805894649178
0.149197881311021 0.247319783843054
0.140195031151771 0.241272894158711
0.131588220396365 0.234682536784361
0.123382062255884 0.227567134929123
0.115580754025762 0.219946118404367
0.108188187942631 0.211839847297465
0.101208057557953 0.203269566144989
0.0946439418448079 0.19425738357691
0.088499358802492 0.184826268669802
0.0827777883280161 0.17500005435619
0.0774826693237491 0.164803439481848
0.0726173783628224 0.154261983466959
0.0681851974057714 0.143402090111052
0.0641892769319838 0.132250979303424
0.0606325991839522 0.12083664700672
0.0575179445429576 0.109187814856514
0.0548478626398547 0.0973338711778623
0.0526246487503009 0.0853048053160545
0.0508503253165833 0.0731311370573394
0.0495266280133111 0.0608438426868979
0.0486549955532945 0.0484742789692179
0.0482365623412967 0.036054106082026
0.0482721530701452 0.0236152103073717
0.0487622783751904 0.0111896270851532
0.0497071306928156 -0.00119053514168032
0.0511065794918781 -0.0134931700273918
0.0529601650583767 -0.0256862472791051
0.0552670900164109 -0.0377378855745846
0.0580262077740067 -0.0496164247880426
0.061236007111159 -0.0612904981681536
0.0648945922107272 -0.0727291054819839
0.0689996576140302 -0.0839016885812368
0.0735484579177227 -0.0947782113457813
0.0785377725810335 -0.10532924646507
0.0839638670453019 -0.11552607193497
0.0898224525211534 -0.125340780318524
0.0961086482561191 -0.134746403519034
0.10281695173701 -0.143717054765618
0.109941223832714 -0.152228087440402
0.117474696885182 -0.160256267121569
0.125410013591915 -0.167779948892829
0.133739302545796 -0.174779247148552
0.14245429208538 -0.181236180932355
0.151546457798383 -0.187134775860053
0.161007191589854 -0.192461105490089
0.170827973498639 -0.197203261519913
0.181000523658031 -0.201351252889932
0.191516912754772 -0.204896846530735
0.202369615510414 -0.207833373632873
0.213551501760898 -0.210155531524887
0.22505577088336 -0.211859210579139
0.236875844499746 -0.212941368432345
0.249005237340447 -0.213399962707235
0.261437426261433 -0.213233941776159
0.274165733616003 -0.212443283869614
0.28718323530924 -0.211029069576408
0.300482697779478 -0.208993571566475
0.314056543186631 -0.206340347198259
0.327896838839168 -0.203074323164878
0.341995305342881 -0.199201865241865
0.356343337731412 -0.194730829684218
0.370932034454633 -0.189670595476494
0.385752230119493 -0.184032078393973
0.400794528982751 -0.177827728806396
0.416049337197401 -0.171071515546711
0.431506892628147 -0.16377889817583
0.447157291657684 -0.15596678976561
0.462990512824106 -0.147653512015245
0.478996437395876 -0.138858744187699
0.495164867142796 -0.129603467046443
0.511485539633279 -0.119909902709476
0.527948141407569 -0.109801451123518
0.544542319363787 -0.0993026236948103
0.561257690663388 -0.0884389744880547
0.578083851424296 -0.0772370293146803
0.595010384429933 -0.0657242129685941
0.612026866043939 -0.0539287748255556
0.62912287248563 -0.0418797129959582
0.646287985590849 -0.029606697205851
0.663511798156965 -0.0171399905741775
0.680783918949166 -0.00451037045289776
0.698093977427272 0.00825095150093907
0.715431628237671 0.0211124098473716
0.732786555502968 0.0340421663000904
};
\addplot [semithick, green, dash pattern=on 1pt off 3pt on 3pt off 3pt, forget plot]
table {%
0.75 0
0.767347219519284 0.012137253483567
0.784683394455509 0.0242435861667548
0.801997486582194 0.036288155514001
0.819278471531621 0.0482402761288358
0.836515345700615 0.0600694978228124
0.853697133135913 0.0717456830897325
0.870812892392119 0.0832390837898383
0.887851723355235 0.0945204168511733
0.904802774024683 0.105560938798497
0.921655247246804 0.116332518924018
0.938398407392915 0.126807710918802
0.955021586975309 0.136959822789028
0.971514193195148 0.146762984887382
0.987865714417124 0.156192215896701
1.0040657265673 0.165223486610677
1.02010389945277 0.173833781364724
1.03597000300535 0.182001156979107
1.05165391345605 0.189704799085743
1.06714561945411 0.196925075719347
1.08243522815337 0.203643588062188
1.09751297130037 0.209843218238558
1.1123692113737 0.215508174058654
1.12699444784162 0.220624030609957
1.14137932362429 0.225177768584601
1.1555146318663 0.22915780921071
1.16939132313901 0.232554045621132
1.18300051319534 0.235357870442502
1.19633349138035 0.237562199321834
1.20938172974852 0.23916149003295
1.22213689283924 0.240151756735333
1.23459084790774 0.240530578918402
1.24673567520343 0.24029710459028
1.25856367765532 0.239452047402427
1.27006738911946 0.237997677671498
1.28123958024499 0.235937807669022
1.29207326111455 0.233277772045945
1.30256168017249 0.230024404727089
1.31269831956044 0.226186013888674
1.32247688771958 0.221772356568193
1.33189131078413 0.216794613983565
1.34093572465289 0.211265367835258
1.34960446953272 0.205198576949498
1.35789208820684 0.198609552872807
1.36579332846363 0.191514932670796
1.37330314928634 0.18393264729003
1.38041672978698 0.175881884318464
1.38712947958844 0.167383044631953
1.39343704941026 0.158457693037802
1.39933534088867 0.149128503482261
1.40482051503024 0.139419199628507
1.40988899904863 0.129354491659788
1.41453749160152 0.118960010081206
1.41876296661194 0.108262237150103
1.42256267593454 0.0972884364110782
1.42593415113889 0.086066580678514
1.42887520465437 0.0746252787098322
1.43138393047621 0.0629937007473294
1.43345870458359 0.0512015030695605
1.43509818517756 0.0392787516771522
1.43630131281055 0.0272558452352367
1.43706731045357 0.0151634373994425
1.43739568352819 0.00303235866024866
1.43728621991849 -0.00910646215137196
1.43673898997073 -0.0212220765475008
1.43575434648365 -0.0332835951024502
1.43433292468914 -0.0452602660529673
1.43247564221903 -0.0571215536269705
1.43018369904855 -0.0688372159533007
1.42745857739782 -0.0803773824121825
1.4243020415584 -0.0917126302942692
1.42071613759078 -0.102814060643163
1.41670319280875 -0.113653373158817
1.41226581492764 -0.124202940032363
1.40740689070718 -0.134435878560225
1.40212958387022 -0.144326122339201
1.3964373320371 -0.153848490767307
1.39033384239815 -0.162978756463708
1.38382308587764 -0.171693710079292
1.37690928964889 -0.179971221816331
1.36959692806383 -0.187790298849553
1.36189071236441 -0.195131137800147
1.35379557991389 -0.201975171526458
1.34531668404164 -0.20830510981277
1.33645938581161 -0.214104974062127
1.32722924897007 -0.219360126749639
1.31763203892423 -0.224057297001799
1.30767372588241 -0.22818460402503
1.2973604914176 -0.231731580045289
1.28669873695777 -0.234689193904312
1.27569509230234 -0.237049875620184
1.26435642233521 -0.238807541316647
1.25268983059913 -0.239957617226964
1.24070265912099 -0.240497061156197
1.22840248459827 -0.24042437986202
1.21579711159144 -0.239739641182038
1.20289456363565 -0.238444480231447
1.1897030732081 -0.236542099469318
1.17623107134149 -0.234037262792463
1.16248717745014 -0.230936284032406
1.14848018970572 -0.227247010319085
1.13421907610845 -0.222978800771493
1.11971296626323 -0.218142500920566
1.10497114378721 -0.212750413195198
1.09000303923492 -0.206816263729265
1.07481822341698 -0.200355165686855
1.05942640099553 -0.193383579258294
1.04383740425637 -0.185919268450478
1.02806118697728 -0.177981254778958
1.01210781833129 -0.169589767962932
0.995987476780275 -0.160766193724954
0.979710443928391 -0.151533018802209
0.963287098315684 -0.141913773283855
0.946727909140567 -0.131932970397911
0.930043429905878 -0.121616043880467
0.91324429198767 -0.110989283069269
0.896341198128817 -0.100079765872392
0.879344915861624 -0.0889152897708354
0.862266270864871 -0.0775243010211479
0.845116140261585 -0.0659358222307342
0.827905445864243 -0.0541793784842489
0.810645147374399 -0.042284922204474
0.793346235543765 -0.0302827569353333
0.776019725303823 -0.0182034602382608
0.758676648870971 -0.00607780589600706
0.74132804883419 0.00606331437980999
0.723984971232126 0.0181889695395348
0.706658458626486 0.0302682678858693
0.689359543178601 0.0422704356646973
0.672099239736024 0.0541648953575106
0.654888538936098 0.0659213434775336
0.637738400333409 0.077509827673207
0.620659745558167 0.0889008229448788
0.603663451512546 0.10006530678342
0.586760343612044 0.110974833042994
0.569961189078862 0.121601604364455
0.55327669029407 0.131918542970798
0.536717478214957 0.141899359661781
0.520294105863188 0.151518620841353
0.504017041888231 0.160751813418715
0.487896664208613 0.169575407431921
0.471943253730781 0.177966916251536
0.456166988141264 0.18590495423109
0.440577935762111 0.193369291680399
0.425186049451712 0.200340907046881
0.410001160522753 0.206802036197834
0.395032972635676 0.212736218702091
0.38029105560979 0.218128341010777
0.36578483907538 0.222964676431608
0.351523605870586 0.227232921776542
0.337516485069478 0.23092223053541
0.323772444518238 0.234023242385741
0.310300282763089 0.236528108790522
0.297108620287721 0.238430514364134
0.284205890052545 0.239725693611976
0.27160032745433 0.240410442590784
0.259299960006205 0.240483125024951
0.247312597261469 0.239943672488383
0.2356458217318 0.238793578459958
0.224306981713199 0.237035886402939
0.21330318694315 0.234675172482025
0.202641307788403 0.231717524032126
0.192327978173203 0.228170515287301
0.1823696017639 0.22404318200088
0.172772360193836 0.219345996319791
0.163542221572307 0.214090842618848
0.154684947376684 0.208290994107505
0.146206096155645 0.201961089155653
0.138111023180157 0.195117105712544
0.130404876038488 0.187776332069152
0.12309258692145 0.179957332525672
0.116178862798813 0.171679907117356
0.109668174793245 0.162965045212245
0.10356474787866 0.153834873345794
0.0978725516915446 0.144312598005936
0.0925952928735381 0.134422444219678
0.0877364090495248 0.124189590767097
0.0832990643281257 0.113640102728524
0.0792861460924091 0.102800861916777
0.0757002628080311 0.0916994956001386
0.0725437425870994 0.0803643038046822
0.069818632284583 0.0688241854024739
0.0675266969522104 0.0571085631419612
0.0656694195214907 0.0452473077513642
0.0642480006271671 0.0332706612373471
0.0632633585132331 0.0212091595029022
0.0627161289858991 0.00909355441514592
0.0626066653929575 -0.00304526453801165
0.0629350386185861 -0.0151763488558262
0.0637010370885092 -0.0272687697454203
0.0649041667842718 -0.0392916968007361
0.0665436512687222 -0.0512144764858195
0.0686184317292081 -0.0630067102720941
0.0711271670520408 -0.0746383322854955
0.0740682339532131 -0.0860796863272447
0.0774397272079407 -0.0973016021399556
0.0812394600470219 -0.108275470796062
0.0854649648223936 -0.118973319084073
0.0901134940873028 -0.12936788275431
0.0951820222852561 -0.139432678452301
0.100667248289028 -0.149142074107302
0.10656559906303 -0.158471357449301
0.112873234718821 -0.167396802199979
0.119586055168304 -0.175895731332128
0.126699708426136 -0.183946576646328
0.13420960035749 -0.191528933821879
0.142110905322839 -0.198623612126504
0.150398576795015 -0.205212678179984
0.159067356720643 -0.211279493592426
0.16811178230224 -0.216808746901301
0.177526189102462 -0.221786480886398
0.187304709947427 -0.226200116851138
0.197441269929379 -0.230038477621279
0.207929578656845 -0.233291810717099
0.218763121506754 -0.235951812451675
0.229935151803763 -0.238011652801433
0.241438685552319 -0.239466000064469
0.253266499708268 -0.240311043799429
0.265411134226006 -0.240544514421553
0.277864897472831 -0.240165698073282
0.290619874197545 -0.23917544583971
0.303667935101709 -0.237576176879925
0.317000747134867 -0.235371875470324
0.330609783830706 -0.232568082244952
0.344486335235128 -0.229171880066449
0.358621517191145 -0.225191874998346
0.373006279910045 -0.220638172815656
0.387631415867 -0.215522351422736
0.402487567119295 -0.209857429471725
0.417565232169048 -0.203657831407191
0.432854772492479 -0.196939349109845
0.448346418844808 -0.189719100275549
0.46403027743119 -0.182015483643688
0.47989633601438 -0.173848131178146
0.495934470011695 -0.165237857301697
0.512134448618413 -0.156206605287707
0.528485940982374 -0.146777390919585
0.544978522444945 -0.136974243536887
0.561601680856407 -0.126822144596177
0.578344822968638 -0.116346963884112
0.595197280904322 -0.105575393529208
0.61214831869946 -0.094534879967153
0.629187138914278 -0.0832535540222168
0.646302889306651 -0.071760159274263
0.663484669561462 -0.0600839788869683
0.680721538069075 -0.0482547610782518
0.698002518745858 -0.0363026434185232
0.715316607889725 -0.0242580761462783
0.732652781063624 -0.012151744693772
};
\addplot [line width = \linewidthEightC, color = reference, opacity=\opacityRef, forget plot]
table {%
0.75 0
0.75421541929245 0.0029488205909729
0.77053701877594 0.0152587965130806
0.787493348121643 0.0277034193277359
0.805577456951141 0.0408641695976257
0.821425020694733 0.0527719408273697
0.8403400182724 0.0662410110235214
0.857643365859985 0.0787167102098465
0.874771654605865 0.0906331390142441
0.891798555850983 0.102761775255203
0.908758223056793 0.114153876900673
0.925576508045197 0.125570833683014
0.942409217357635 0.136616706848145
0.959320604801178 0.147432565689087
0.97600394487381 0.157837331295013
0.992426335811615 0.167767107486725
1.00691312551498 0.175976112484932
1.02415627241135 0.186106085777283
1.04044204950333 0.194418430328369
1.05611246824265 0.2023084461689
1.07061690092087 0.208860993385315
1.08751946687698 0.216304689645767
1.10125917196274 0.222281605005264
1.11746770143509 0.228197306394577
1.13213390111923 0.233722895383835
1.14632648229599 0.237961679697037
1.16093820333481 0.24182465672493
1.17284804582596 0.24482473731041
1.18825680017471 0.247727543115616
1.20180624723434 0.249729096889496
1.2148500084877 0.251031041145325
1.22810763120651 0.251729995012283
1.24067348241806 0.252210080623627
1.25130039453506 0.251750349998474
1.26460117101669 0.25054058432579
1.27611607313156 0.249001413583755
1.28740018606186 0.246537655591965
1.29800659418106 0.24380961060524
1.30867046117783 0.240069985389709
1.31851071119308 0.235908985137939
1.32823723554611 0.231029748916626
1.33765059709549 0.225876092910767
1.34690994024277 0.219777971506119
1.35560077428818 0.213274627923965
1.36382430791855 0.206232666969299
1.37187224626541 0.198216021060944
1.37942081689835 0.190397799015045
1.38669162988663 0.181994676589966
1.39336806535721 0.173233404755592
1.39950639009476 0.164007216691971
1.40543061494827 0.154484674334526
1.41072076559067 0.143915131688118
1.41559332609177 0.133691385388374
1.42017656564713 0.122498959302902
1.42405658960342 0.111938893795013
1.42761784791946 0.100701913237572
1.43069785833359 0.0890136063098907
1.4335578083992 0.077183797955513
1.4358486533165 0.0650627613067627
1.43769603967667 0.0530935227870941
1.43929940462112 0.0403734445571899
1.44024783372879 0.0278445854783058
1.44079691171646 0.0161374136805534
1.44099277257919 0.0035066083073616
1.44074064493179 -0.00902753323316574
1.43971866369247 -0.0212217085063457
1.43836957216263 -0.033147431910038
1.4366666674614 -0.0454998426139355
1.43456739187241 -0.0574229694902897
1.43202918767929 -0.0695854108780622
1.42939227819443 -0.0802723090164363
1.42540234327316 -0.0927700027823448
1.42174500226974 -0.104812320321798
1.41739517450333 -0.115104522556067
1.41319423913956 -0.124955099076033
1.40742510557175 -0.136054821312428
1.40153115987778 -0.146007612347603
1.39569467306137 -0.155558556318283
1.38916426897049 -0.164530277252197
1.38235372304916 -0.173503324389458
1.37578743696213 -0.180841475725174
1.36711519956589 -0.189493909478188
1.35879653692245 -0.196948871016502
1.3501803278923 -0.203477799892426
1.34132820367813 -0.209762170910835
1.3329501748085 -0.214876294136047
1.32337635755539 -0.219874769449234
1.31323915719986 -0.224069520831108
1.30317276716232 -0.227650582790375
1.29139715433121 -0.231056153774261
1.28044158220291 -0.233300507068634
1.26904708147049 -0.23494328558445
1.25718706846237 -0.236053943634033
1.24629479646683 -0.236542105674744
1.23240000009537 -0.236656472086906
1.21987611055374 -0.236023277044296
1.20658940076828 -0.234422504901886
1.19310706853867 -0.232594072818756
1.17926996946335 -0.230083346366882
1.16543847322464 -0.22704005241394
1.15235728025436 -0.223545461893082
1.13643902540207 -0.21847403049469
1.12176412343979 -0.213458061218262
1.10708171129227 -0.207805410027504
1.09173268079758 -0.201379284262657
1.07651156187057 -0.194618634879589
1.06063860654831 -0.187261044979095
1.04650968313217 -0.180087700486183
1.03056734800339 -0.171931073069572
1.01432257890701 -0.163050651550293
0.996536672115326 -0.153021067380905
0.981856524944305 -0.144190736114979
0.965341031551361 -0.133995305746794
0.946696102619171 -0.122459672391415
0.930096685886383 -0.11172141879797
0.914776146411896 -0.101419821381569
0.897595942020416 -0.0897806994616985
0.87852156162262 -0.0767547581344843
0.861134469509125 -0.0645777657628059
0.843874096870422 -0.052588876336813
0.82621431350708 -0.0399243198335171
0.80872642993927 -0.0276575349271297
0.791434645652771 -0.0147440955042839
0.77378922700882 -0.00221717357635498
0.756431102752686 0.0102654546499252
0.738986253738403 0.0230483487248421
0.721069157123566 0.0361561179161072
0.70378041267395 0.0488365590572357
0.686372518539429 0.0617630034685135
0.669017255306244 0.0741308629512787
0.651377260684967 0.0865159183740616
0.634215652942657 0.0989797413349152
0.616955041885376 0.110996901988983
0.600113987922668 0.12306921184063
0.583011090755463 0.134117856621742
0.565973341464996 0.145778939127922
0.549158811569214 0.156584069132805
0.534050226211548 0.166490942239761
0.516099333763123 0.17780613899231
0.49929815530777 0.187614321708679
0.483075678348541 0.19723704457283
0.466586291790009 0.206264436244965
0.450701653957367 0.214817136526108
0.435023307800293 0.222521245479584
0.419368982315063 0.230382412672043
0.403676629066467 0.237435042858124
0.388340711593628 0.243592768907547
0.373213022947311 0.24972328543663
0.35822868347168 0.25511646270752
0.343570321798325 0.260081619024277
0.328994333744049 0.264363706111908
0.316109389066696 0.267436921596527
0.300860673189163 0.270830661058426
0.287340432405472 0.27299565076828
0.273732632398605 0.274883419275284
0.262123048305511 0.275856614112854
0.247858256101608 0.276542246341705
0.235483855009079 0.276087880134583
0.224560022354126 0.275414347648621
0.211488306522369 0.273920446634293
0.20002007484436 0.271887063980103
0.188631981611252 0.269151717424393
0.178183346986771 0.265760213136673
0.168057143688202 0.261819094419479
0.157636642456055 0.257274240255356
0.148172169923782 0.252058267593384
0.138851642608643 0.246164441108704
0.130169987678528 0.239912986755371
0.121905580163002 0.233202368021011
0.114229142665863 0.225776046514511
0.106473028659821 0.217867970466614
0.0993590503931046 0.209721386432648
0.0925293862819672 0.201090902090073
0.0862037092447281 0.192118018865585
0.0802911221981049 0.18233123421669
0.0749434530735016 0.172086209058762
0.0699098855257034 0.161679074168205
0.0652569383382797 0.151079431176186
0.0611734688282013 0.139958649873734
0.0572208762168884 0.128593072295189
0.0537698268890381 0.117376580834389
0.0510749220848083 0.105294674634933
0.0487240999937057 0.0932778716087341
0.0469802618026733 0.0808463096618652
0.0456023663282394 0.0688017606735229
0.0445530116558075 0.0567060708999634
0.0438252538442612 0.0441992431879044
0.0436929017305374 0.0319249629974365
0.0439489483833313 0.0195717662572861
0.0450771450996399 0.00648088008165359
0.0461819171905518 -0.00476055592298508
0.0479282885789871 -0.0181072056293488
0.0498403608798981 -0.0291296765208244
0.0528151094913483 -0.0421386808156967
0.055712878704071 -0.0540377497673035
0.0592333525419235 -0.0652471561916173
0.0634133517742157 -0.0767055964097381
0.0675414204597473 -0.0873536812141538
0.0725554823875427 -0.0984434634447098
0.0773501545190811 -0.108606830239296
0.0831011533737183 -0.11868305131793
0.0893369317054749 -0.128315195441246
0.0957031846046448 -0.137546345591545
0.103028535842896 -0.146704539656639
0.109101057052612 -0.153568349778652
0.117908716201782 -0.162576094269753
0.12508961558342 -0.169240519404411
0.134055316448212 -0.176396869122982
0.143268004059792 -0.182843796908855
0.152176737785339 -0.188110277056694
0.162164866924286 -0.193360865116119
0.172075510025024 -0.197789534926414
0.18257263302803 -0.202016279101372
0.193042546510696 -0.204779446125031
0.204382508993149 -0.207515567541122
0.21546944975853 -0.209910690784454
0.227461278438568 -0.210891962051392
0.23965984582901 -0.211359396576881
0.252080053091049 -0.212168619036674
0.264749854803085 -0.211416095495224
0.277574986219406 -0.209796562790871
0.290921956300735 -0.207863435149193
0.304460167884827 -0.205243051052094
0.318424195051193 -0.201987847685814
0.332793533802032 -0.198717847466469
0.346846520900726 -0.19398058205843
0.361431539058685 -0.188539788126945
0.376425206661224 -0.183002114295959
0.391384601593018 -0.176570817828178
0.406621932983398 -0.169584408402443
0.422347128391266 -0.162516936659813
0.43766051530838 -0.154479667544365
0.453426539897919 -0.145493298768997
0.469427525997162 -0.136847198009491
0.485554814338684 -0.127413105219603
0.501539826393127 -0.11760813370347
0.518173635005951 -0.107292681932449
0.534520626068115 -0.0964057184755802
0.549252331256866 -0.0863416753709316
0.567556381225586 -0.0735088884830475
0.582541525363922 -0.0632802620530128
0.59920209646225 -0.0507793258875608
0.616171061992645 -0.0385284498333931
0.634790658950806 -0.0245984569191933
0.651655912399292 -0.0119211226701736
0.669055342674255 0.0015803724527359
0.684480547904968 0.0130320116877556
0.703211843967438 0.0276269614696503
0.720421552658081 0.0407083034515381
0.737642407417297 0.0538696050643921
};
\addplot [semithick, red, dashed, forget plot]
table {%
0.75 0
0.767617956255685 0.0126152320166048
0.785221863675294 0.025202759744927
0.802800914680889 0.0377306780229099
0.820344353102624 0.0501672114827695
0.837841479099983 0.0624807937495562
0.855281653865309 0.0746401456242545
0.872654304077727 0.0866143520292023
0.889948926065816 0.0983729374930471
0.90715508962574 0.109885939954111
0.924262441427357 0.121123982665618
0.941260707923679 0.132058343994744
0.958139697658408 0.142661024921306
0.974889302841815 0.152904814062882
0.991499500036706 0.162763350083898
1.00796034976389 0.172211181389763
1.02426199480132 0.18122382306779
1.04039465691463 0.189777811119373
1.0563486317231 0.197850754138649
1.07211428137931 0.205421382738366
1.08768202473299 0.212469597210257
1.10304232467243 0.218976514139914
1.11818567241108 0.224924512976024
1.1331025686362 0.230297283873765
1.147783501695 0.23507987847157
1.16221892339532 0.239258765576722
1.17639922357808 0.242821893954846
1.19031470539574 0.245758764427774
1.20395556419321 0.248060513126141
1.21731187395941 0.249720006825457
1.23037358633393 0.250731949622618
1.24313054782987 0.251092997654816
1.2555725408717 0.250801875167454
1.26768935298946 0.24985948134918
1.27947087567813 0.248268973729967
1.29090722994702 0.246035811751358
1.30198890990496 0.243167744731957
1.31270692999354 0.2396747329547
1.32305295737637 0.235568799170581
1.33301941026506 0.230863819194334
1.34259950666846 0.225575271775738
1.3517872558846 0.219719976188812
1.36057739518197 0.213315848229081
1.36896528374308 0.206381700652654
1.37694677240494 0.198937103974783
1.38451806954475 0.191002311131823
1.39167562078374 0.182598238316676
1.39841601450881 0.173746486857157
1.4047359185686 0.164469388195659
1.41063204764278 0.154790055213092
1.41610115672778 0.144732426840967
1.42114005415556 0.134321297531182
1.42574562723652 0.123582327489247
1.42991487441761 0.112542032954869
1.43364493919855 0.101227758043132
1.43693314251624 0.089667630857217
1.4397770116222 0.0778905070053025
1.44217430452097 0.0659259035746353
1.44412302978876 0.0538039262643269
1.44562146208725 0.0415551919190076
1.44666815398553 0.0292107482394724
1.44726194486432 0.016801992026761
1.44740196775303 0.00436058696240971
1.44708765498222 -0.00808161836067393
1.44631874355061 -0.0204926716768206
1.4450952811249 -0.0328405972734807
1.44341763362016 -0.0450934751015305
1.44128649535024 -0.0572195188465802
1.4387029027798 -0.0691871532341023
1.43566825293285 -0.0809650910678666
1.4321843274821 -0.0925224107662305
1.42825332340439 -0.103828635462315
1.42387789076703 -0.114853815053745
1.41906117760902 -0.125568612881875
1.41380688088891 -0.135944398910193
1.40811930097384 -0.145953351233135
1.40200339507468 -0.155568567310703
1.39546482242266 -0.164764185292558
1.38850997106277 -0.173515513988717
1.38114595342609 -0.181799167392522
1.37338055620066 -0.189593196333819
1.36522213060654 -0.196877206377916
1.35667941319178 -0.203632448466232
1.34776127545165 -0.209841868311401
1.33847641259836 -0.215490103434114
1.328832995749 -0.220563423487838
1.3188383241811 -0.225049619377379
1.30849852098817 -0.228937857346307
1.29781831326193 -0.232218522442798
1.28680092623794 -0.234883078640104
1.27544810236402 -0.23692396914816
1.26376023613114 -0.238334571253006
1.25173659926596 -0.239109208409727
1.23937562223503 -0.239243211851919
1.22680212890194 -0.238807395220756
1.21393543612189 -0.237739020519201
1.2007824030094 -0.236042071113405
1.1873502195155 -0.233721342391173
1.17364639818837 -0.230782623136066
1.15967876862552 -0.227232843837933
1.14545546822952 -0.223080191387264
1.13098492732646 -0.218334192018305
1.11627584919941 -0.213005765794969
1.10133718666704 -0.207107256561536
1.08617811704058 -0.200652441341106
1.070808017043 -0.19365652287622
1.05523643885516 -0.186136108539406
1.03947308802467 -0.178109178316594
1.02352780361101 -0.169595044057188
1.00741054066936 -0.160614301730894
0.991131354992172 -0.151188778049092
0.974700389916618 -0.141341472499412
0.958127864949596 -0.131096495600034
0.941424065943487 -0.120479003995558
0.924599336561409 -0.109515132878708
0.90766407079076 -0.0982319261219058
0.890628706290982 -0.086657264431339
0.873503718391248 -0.074819791786618
0.856299614583182 -0.0627488403957748
0.839026929381251 -0.0504743543738914
0.8216962194482 -0.0380268123406879
0.804318058904672 -0.0254371491254941
0.786903034760972 -0.0127366767652624
0.769461742425004 4.29950186324069e-05
0.752004781254135 0.0128700386771497
0.734542750130382 0.0257124888331812
0.717086243048227 0.0385383222295833
0.699645844712938 0.0513155374272836
0.682232126154625 0.0640122340511521
0.664855640369735 0.0765966913776288
0.647526918007365 0.0890374460548427
0.630256463122704 0.101303368743397
0.613054749024055 0.113363739464377
0.595932214243 0.125188321440869
0.578899258658929 0.136747433220813
0.561966239808458 0.148012018873052
0.545143469406046 0.158953716055882
0.528441210092295 0.169544921769502
0.511869672407924 0.179758855622043
0.495439011959919 0.189569620465462
0.479159326695086 0.198952260295039
0.463040654116175 0.207882815357607
0.447092968153461 0.216338374482455
0.431326175222446 0.224297124738732
0.41575010873165 0.231738398637223
0.400374520922108 0.238642719233979
0.38520907038282 0.244991843656404
0.37026330284855 0.250768805749989
0.355546621900637 0.255957958716022
0.341068244918601 0.260545018739871
0.326837138059129 0.264517110634022
0.312861922218046 0.267862816346776
0.299150740017787 0.270572226688282
0.285711072187049 0.272636995643726
0.272549490830637 0.274050395018263
0.259671337879835 0.274807364778616
0.247080320574249 0.274904551350881
0.234778023333987 0.274340322593555
0.22282111677237 0.273147174350289
0.211212000441657 0.271312180989548
0.199956090718924 0.268842225658862
0.18905951088657 0.265745762806622
0.178528814558833 0.262032850875418
0.168370684504055 0.257715152729482
0.158591652631679 0.252805894649178
0.149197881311021 0.247319783843054
0.140195031151771 0.241272894158711
0.131588220396365 0.234682536784361
0.123382062255884 0.227567134929123
0.115580754025762 0.219946118404367
0.108188187942631 0.211839847297465
0.101208057557953 0.203269566144989
0.0946439418448079 0.19425738357691
0.088499358802492 0.184826268669802
0.0827777883280161 0.17500005435619
0.0774826693237491 0.164803439481848
0.0726173783628224 0.154261983466959
0.0681851974057714 0.143402090111052
0.0641892769319838 0.132250979303424
0.0606325991839522 0.12083664700672
0.0575179445429576 0.109187814856514
0.0548478626398547 0.0973338711778623
0.0526246487503009 0.0853048053160545
0.0508503253165833 0.0731311370573394
0.0495266280133111 0.0608438426868979
0.0486549955532945 0.0484742789692179
0.0482365623412967 0.036054106082026
0.0482721530701452 0.0236152103073717
0.0487622783751904 0.0111896270851532
0.0497071306928156 -0.00119053514168032
0.0511065794918781 -0.0134931700273918
0.0529601650583767 -0.0256862472791051
0.0552670900164109 -0.0377378855745846
0.0580262077740067 -0.0496164247880426
0.061236007111159 -0.0612904981681536
0.0648945922107272 -0.0727291054819839
0.0689996576140302 -0.0839016885812368
0.0735484579177227 -0.0947782113457813
0.0785377725810335 -0.10532924646507
0.0839638670453019 -0.11552607193497
0.0898224525211534 -0.125340780318524
0.0961086482561191 -0.134746403519034
0.10281695173701 -0.143717054765618
0.109941223832714 -0.152228087440402
0.117474696885182 -0.160256267121569
0.125410013591915 -0.167779948892829
0.133739302545796 -0.174779247148552
0.14245429208538 -0.181236180932355
0.151546457798383 -0.187134775860053
0.161007191589854 -0.192461105490089
0.170827973498639 -0.197203261519913
0.181000523658031 -0.201351252889932
0.191516912754772 -0.204896846530735
0.202369615510414 -0.207833373632873
0.213551501760898 -0.210155531524887
0.22505577088336 -0.211859210579139
0.236875844499746 -0.212941368432345
0.249005237340447 -0.213399962707235
0.261437426261433 -0.213233941776159
0.274165733616003 -0.212443283869614
0.28718323530924 -0.211029069576408
0.300482697779478 -0.208993571566475
0.314056543186631 -0.206340347198259
0.327896838839168 -0.203074323164878
0.341995305342881 -0.199201865241865
0.356343337731412 -0.194730829684218
0.370932034454633 -0.189670595476494
0.385752230119493 -0.184032078393973
0.400794528982751 -0.177827728806396
0.416049337197401 -0.171071515546711
0.431506892628147 -0.16377889817583
0.447157291657684 -0.15596678976561
0.462990512824106 -0.147653512015245
0.478996437395876 -0.138858744187699
0.495164867142796 -0.129603467046443
0.511485539633279 -0.119909902709476
0.527948141407569 -0.109801451123518
0.544542319363787 -0.0993026236948103
0.561257690663388 -0.0884389744880547
0.578083851424296 -0.0772370293146803
0.595010384429933 -0.0657242129685941
0.612026866043939 -0.0539287748255556
0.62912287248563 -0.0418797129959582
0.646287985590849 -0.029606697205851
0.663511798156965 -0.0171399905741775
0.680783918949166 -0.00451037045289776
0.698093977427272 0.00825095150093907
0.715431628237671 0.0211124098473716
0.732786555502968 0.0340421663000904
};
\addplot [semithick, green, dash pattern=on 1pt off 3pt on 3pt off 3pt, forget plot]
table {%
0.75 0
0.767347219519284 0.012137253483567
0.784683394455509 0.0242435861667548
0.801997486582194 0.036288155514001
0.819278471531621 0.0482402761288358
0.836515345700615 0.0600694978228124
0.853697133135913 0.0717456830897325
0.870812892392119 0.0832390837898383
0.887851723355235 0.0945204168511733
0.904802774024683 0.105560938798497
0.921655247246804 0.116332518924018
0.938398407392915 0.126807710918802
0.955021586975309 0.136959822789028
0.971514193195148 0.146762984887382
0.987865714417124 0.156192215896701
1.0040657265673 0.165223486610677
1.02010389945277 0.173833781364724
1.03597000300535 0.182001156979107
1.05165391345605 0.189704799085743
1.06714561945411 0.196925075719347
1.08243522815337 0.203643588062188
1.09751297130037 0.209843218238558
1.1123692113737 0.215508174058654
1.12699444784162 0.220624030609957
1.14137932362429 0.225177768584601
1.1555146318663 0.22915780921071
1.16939132313901 0.232554045621132
1.18300051319534 0.235357870442502
1.19633349138035 0.237562199321834
1.20938172974852 0.23916149003295
1.22213689283924 0.240151756735333
1.23459084790774 0.240530578918402
1.24673567520343 0.24029710459028
1.25856367765532 0.239452047402427
1.27006738911946 0.237997677671498
1.28123958024499 0.235937807669022
1.29207326111455 0.233277772045945
1.30256168017249 0.230024404727089
1.31269831956044 0.226186013888674
1.32247688771958 0.221772356568193
1.33189131078413 0.216794613983565
1.34093572465289 0.211265367835258
1.34960446953272 0.205198576949498
1.35789208820684 0.198609552872807
1.36579332846363 0.191514932670796
1.37330314928634 0.18393264729003
1.38041672978698 0.175881884318464
1.38712947958844 0.167383044631953
1.39343704941026 0.158457693037802
1.39933534088867 0.149128503482261
1.40482051503024 0.139419199628507
1.40988899904863 0.129354491659788
1.41453749160152 0.118960010081206
1.41876296661194 0.108262237150103
1.42256267593454 0.0972884364110782
1.42593415113889 0.086066580678514
1.42887520465437 0.0746252787098322
1.43138393047621 0.0629937007473294
1.43345870458359 0.0512015030695605
1.43509818517756 0.0392787516771522
1.43630131281055 0.0272558452352367
1.43706731045357 0.0151634373994425
1.43739568352819 0.00303235866024866
1.43728621991849 -0.00910646215137196
1.43673898997073 -0.0212220765475008
1.43575434648365 -0.0332835951024502
1.43433292468914 -0.0452602660529673
1.43247564221903 -0.0571215536269705
1.43018369904855 -0.0688372159533007
1.42745857739782 -0.0803773824121825
1.4243020415584 -0.0917126302942692
1.42071613759078 -0.102814060643163
1.41670319280875 -0.113653373158817
1.41226581492764 -0.124202940032363
1.40740689070718 -0.134435878560225
1.40212958387022 -0.144326122339201
1.3964373320371 -0.153848490767307
1.39033384239815 -0.162978756463708
1.38382308587764 -0.171693710079292
1.37690928964889 -0.179971221816331
1.36959692806383 -0.187790298849553
1.36189071236441 -0.195131137800147
1.35379557991389 -0.201975171526458
1.34531668404164 -0.20830510981277
1.33645938581161 -0.214104974062127
1.32722924897007 -0.219360126749639
1.31763203892423 -0.224057297001799
1.30767372588241 -0.22818460402503
1.2973604914176 -0.231731580045289
1.28669873695777 -0.234689193904312
1.27569509230234 -0.237049875620184
1.26435642233521 -0.238807541316647
1.25268983059913 -0.239957617226964
1.24070265912099 -0.240497061156197
1.22840248459827 -0.24042437986202
1.21579711159144 -0.239739641182038
1.20289456363565 -0.238444480231447
1.1897030732081 -0.236542099469318
1.17623107134149 -0.234037262792463
1.16248717745014 -0.230936284032406
1.14848018970572 -0.227247010319085
1.13421907610845 -0.222978800771493
1.11971296626323 -0.218142500920566
1.10497114378721 -0.212750413195198
1.09000303923492 -0.206816263729265
1.07481822341698 -0.200355165686855
1.05942640099553 -0.193383579258294
1.04383740425637 -0.185919268450478
1.02806118697728 -0.177981254778958
1.01210781833129 -0.169589767962932
0.995987476780275 -0.160766193724954
0.979710443928391 -0.151533018802209
0.963287098315684 -0.141913773283855
0.946727909140567 -0.131932970397911
0.930043429905878 -0.121616043880467
0.91324429198767 -0.110989283069269
0.896341198128817 -0.100079765872392
0.879344915861624 -0.0889152897708354
0.862266270864871 -0.0775243010211479
0.845116140261585 -0.0659358222307342
0.827905445864243 -0.0541793784842489
0.810645147374399 -0.042284922204474
0.793346235543765 -0.0302827569353333
0.776019725303823 -0.0182034602382608
0.758676648870971 -0.00607780589600706
0.74132804883419 0.00606331437980999
0.723984971232126 0.0181889695395348
0.706658458626486 0.0302682678858693
0.689359543178601 0.0422704356646973
0.672099239736024 0.0541648953575106
0.654888538936098 0.0659213434775336
0.637738400333409 0.077509827673207
0.620659745558167 0.0889008229448788
0.603663451512546 0.10006530678342
0.586760343612044 0.110974833042994
0.569961189078862 0.121601604364455
0.55327669029407 0.131918542970798
0.536717478214957 0.141899359661781
0.520294105863188 0.151518620841353
0.504017041888231 0.160751813418715
0.487896664208613 0.169575407431921
0.471943253730781 0.177966916251536
0.456166988141264 0.18590495423109
0.440577935762111 0.193369291680399
0.425186049451712 0.200340907046881
0.410001160522753 0.206802036197834
0.395032972635676 0.212736218702091
0.38029105560979 0.218128341010777
0.36578483907538 0.222964676431608
0.351523605870586 0.227232921776542
0.337516485069478 0.23092223053541
0.323772444518238 0.234023242385741
0.310300282763089 0.236528108790522
0.297108620287721 0.238430514364134
0.284205890052545 0.239725693611976
0.27160032745433 0.240410442590784
0.259299960006205 0.240483125024951
0.247312597261469 0.239943672488383
0.2356458217318 0.238793578459958
0.224306981713199 0.237035886402939
0.21330318694315 0.234675172482025
0.202641307788403 0.231717524032126
0.192327978173203 0.228170515287301
0.1823696017639 0.22404318200088
0.172772360193836 0.219345996319791
0.163542221572307 0.214090842618848
0.154684947376684 0.208290994107505
0.146206096155645 0.201961089155653
0.138111023180157 0.195117105712544
0.130404876038488 0.187776332069152
0.12309258692145 0.179957332525672
0.116178862798813 0.171679907117356
0.109668174793245 0.162965045212245
0.10356474787866 0.153834873345794
0.0978725516915446 0.144312598005936
0.0925952928735381 0.134422444219678
0.0877364090495248 0.124189590767097
0.0832990643281257 0.113640102728524
0.0792861460924091 0.102800861916777
0.0757002628080311 0.0916994956001386
0.0725437425870994 0.0803643038046822
0.069818632284583 0.0688241854024739
0.0675266969522104 0.0571085631419612
0.0656694195214907 0.0452473077513642
0.0642480006271671 0.0332706612373471
0.0632633585132331 0.0212091595029022
0.0627161289858991 0.00909355441514592
0.0626066653929575 -0.00304526453801165
0.0629350386185861 -0.0151763488558262
0.0637010370885092 -0.0272687697454203
0.0649041667842718 -0.0392916968007361
0.0665436512687222 -0.0512144764858195
0.0686184317292081 -0.0630067102720941
0.0711271670520408 -0.0746383322854955
0.0740682339532131 -0.0860796863272447
0.0774397272079407 -0.0973016021399556
0.0812394600470219 -0.108275470796062
0.0854649648223936 -0.118973319084073
0.0901134940873028 -0.12936788275431
0.0951820222852561 -0.139432678452301
0.100667248289028 -0.149142074107302
0.10656559906303 -0.158471357449301
0.112873234718821 -0.167396802199979
0.119586055168304 -0.175895731332128
0.126699708426136 -0.183946576646328
0.13420960035749 -0.191528933821879
0.142110905322839 -0.198623612126504
0.150398576795015 -0.205212678179984
0.159067356720643 -0.211279493592426
0.16811178230224 -0.216808746901301
0.177526189102462 -0.221786480886398
0.187304709947427 -0.226200116851138
0.197441269929379 -0.230038477621279
0.207929578656845 -0.233291810717099
0.218763121506754 -0.235951812451675
0.229935151803763 -0.238011652801433
0.241438685552319 -0.239466000064469
0.253266499708268 -0.240311043799429
0.265411134226006 -0.240544514421553
0.277864897472831 -0.240165698073282
0.290619874197545 -0.23917544583971
0.303667935101709 -0.237576176879925
0.317000747134867 -0.235371875470324
0.330609783830706 -0.232568082244952
0.344486335235128 -0.229171880066449
0.358621517191145 -0.225191874998346
0.373006279910045 -0.220638172815656
0.387631415867 -0.215522351422736
0.402487567119295 -0.209857429471725
0.417565232169048 -0.203657831407191
0.432854772492479 -0.196939349109845
0.448346418844808 -0.189719100275549
0.46403027743119 -0.182015483643688
0.47989633601438 -0.173848131178146
0.495934470011695 -0.165237857301697
0.512134448618413 -0.156206605287707
0.528485940982374 -0.146777390919585
0.544978522444945 -0.136974243536887
0.561601680856407 -0.126822144596177
0.578344822968638 -0.116346963884112
0.595197280904322 -0.105575393529208
0.61214831869946 -0.094534879967153
0.629187138914278 -0.0832535540222168
0.646302889306651 -0.071760159274263
0.663484669561462 -0.0600839788869683
0.680721538069075 -0.0482547610782518
0.698002518745858 -0.0363026434185232
0.715316607889725 -0.0242580761462783
0.732652781063624 -0.012151744693772
};
\addplot [line width = \linewidthEightC, color = reference, opacity=\opacityRef, forget plot]
table {%
0.75 0
0.753219425678253 0.00282227247953415
0.768264353275299 0.0144888311624527
0.787357091903687 0.0282700955867767
0.803906142711639 0.0401750579476357
0.821577787399292 0.0530469864606857
0.838893592357635 0.0652852952480316
0.856254458427429 0.077686995267868
0.873232364654541 0.0898440033197403
0.890207707881927 0.101744025945663
0.907214045524597 0.113173350691795
0.924317836761475 0.124558001756668
0.941046237945557 0.135283306241035
0.957745671272278 0.14604440331459
0.974259853363037 0.156508967280388
0.990740180015564 0.166351839900017
1.00712156295776 0.175325185060501
1.02338039875031 0.184304609894753
1.03949272632599 0.192713424563408
1.05488228797913 0.200727954506874
1.07053160667419 0.207978770136833
1.08635675907135 0.215009942650795
1.10158145427704 0.221244320273399
1.11670899391174 0.227169260382652
1.13130104541779 0.232124045491219
1.14573705196381 0.236572548747063
1.15995454788208 0.240259632468224
1.17371261119843 0.24353913962841
1.18762159347534 0.245904549956322
1.20077836513519 0.248086646199226
1.21374678611755 0.249730661511421
1.22661054134369 0.250491335988045
1.23917734622955 0.250695452094078
1.25130546092987 0.250075981020927
1.26333284378052 0.249011263251305
1.27498912811279 0.247244253754616
1.28613972663879 0.244893237948418
1.29690420627594 0.241567239165306
1.30760955810547 0.23803685605526
1.3175562620163 0.233666822314262
1.32719933986664 0.229221418499947
1.33661735057831 0.22356091439724
1.34551179409027 0.217689529061317
1.35398554801941 0.211254879832268
1.36210572719574 0.204283282160759
1.36982548236847 0.196847513318062
1.37757062911987 0.188900098204613
1.38456606864929 0.180093362927437
1.39064657688141 0.170979976654053
1.39750349521637 0.161291345953941
1.40330135822296 0.151655778288841
1.40874397754669 0.141456812620163
1.41364717483521 0.130760356783867
1.41816544532776 0.12042647600174
1.42228257656097 0.109095990657806
1.42594885826111 0.0973074436187744
1.42902600765228 0.0861741453409195
1.431685090065 0.0744103789329529
1.43384349346161 0.0620910674333572
1.43591725826263 0.049898236989975
1.4372044801712 0.0377107188105583
1.43798542022705 0.0255243927240372
1.4383453130722 0.0130179673433304
1.43845045566559 0.000657141208648682
1.43805241584778 -0.0114500261843204
1.43716073036194 -0.0234780386090279
1.43586933612823 -0.036049623042345
1.43408679962158 -0.0481833163648844
1.43212258815765 -0.0600184928625822
1.42949235439301 -0.071689426433295
1.42646646499634 -0.0838565863668919
1.42284095287323 -0.0951475296169519
1.41887319087982 -0.10615212097764
1.41451478004456 -0.117642987519503
1.40964496135712 -0.127800468355417
1.40443062782288 -0.138312436640263
1.39871942996979 -0.148051053285599
1.3924503326416 -0.157518647611141
1.38575041294098 -0.166685178875923
1.37914609909058 -0.174922563135624
1.37138545513153 -0.182903826236725
1.36373794078827 -0.190742157399654
1.35541725158691 -0.197643056511879
1.34682607650757 -0.204408690333366
1.33778429031372 -0.210497096180916
1.32840013504028 -0.215801239013672
1.31845486164093 -0.221048668026924
1.30844521522522 -0.225158929824829
1.29790771007538 -0.229101151227951
1.28706312179565 -0.231885150074959
1.27596974372864 -0.234228283166885
1.26427567005157 -0.236313417553902
1.25243723392487 -0.237535700201988
1.23999261856079 -0.238178715109825
1.22767245769501 -0.238166749477386
1.21500754356384 -0.237572520971298
1.2020183801651 -0.235982954502106
1.18838953971863 -0.234027415513992
1.17442047595978 -0.231461152434349
1.16036474704742 -0.228291466832161
1.14621591567993 -0.224309980869293
1.13194358348846 -0.219761610031128
1.11720108985901 -0.214707568287849
1.10245478153229 -0.209115982055664
1.08710289001465 -0.202732115983963
1.072270154953 -0.196297615766525
1.05661499500275 -0.189092740416527
1.04081296920776 -0.180741250514984
1.02507734298706 -0.172570124268532
1.00848352909088 -0.163183264434338
0.992175459861755 -0.153972923755646
0.975838661193848 -0.144384518265724
0.959239363670349 -0.134118191897869
0.942350029945374 -0.123393125832081
0.925597071647644 -0.112457919865847
0.908788442611694 -0.101113608106971
0.891852796077728 -0.0897607877850533
0.87463790178299 -0.0781912617385387
0.857455134391785 -0.0657973363995552
0.839917659759521 -0.0535224974155426
0.822533965110779 -0.0409857928752899
0.805143475532532 -0.0284120962023735
0.787655651569366 -0.0156597718596458
0.769910156726837 -0.00282222777605057
0.752511024475098 0.00989833474159241
0.735025703907013 0.0226972997188568
0.717345893383026 0.0357032120227814
0.7003493309021 0.0481403395533562
0.6829634308815 0.0606399327516556
0.665531992912292 0.073452427983284
0.648036241531372 0.0858319997787476
0.630913972854614 0.0986083149909973
0.613390028476715 0.110783413052559
0.596416056156158 0.12284642457962
0.579208850860596 0.134024396538734
0.562507390975952 0.145600289106369
0.545593678951263 0.156690448522568
0.52908319234848 0.167355731129646
0.512574017047882 0.177472099661827
0.496243119239807 0.187525197863579
0.480102598667145 0.196745678782463
0.463822782039642 0.205899640917778
0.447519242763519 0.214934006333351
0.431434094905853 0.222941055893898
0.415804028511047 0.230621621012688
0.400284200906754 0.237710043787956
0.384793341159821 0.244530722498894
0.369536817073822 0.250518962740898
0.354782313108444 0.255882814526558
0.33979544043541 0.260819599032402
0.325609058141708 0.264874890446663
0.311463296413422 0.268656000494957
0.297501474618912 0.271796122193336
0.283797085285187 0.274230405688286
0.27027615904808 0.275876894593239
0.25747275352478 0.276770636439323
0.244571626186371 0.277424737811089
0.232126027345657 0.277117595076561
0.22000652551651 0.276305928826332
0.207948267459869 0.274942293763161
0.196915477514267 0.272859290242195
0.185570567846298 0.270317688584328
0.175210058689117 0.267018809914589
0.16449248790741 0.262846812605858
0.154316693544388 0.258146867156029
0.145003005862236 0.253102943301201
0.135448411107063 0.247480049729347
0.126851871609688 0.241291627287865
0.118314549326897 0.23441369831562
0.110212564468384 0.226944848895073
0.102904319763184 0.219339534640312
0.0956519991159439 0.210969671607018
0.0889656543731689 0.202314600348473
0.0826373100280762 0.193017825484276
0.0769284218549728 0.183601036667824
0.0712794959545135 0.172996178269386
0.0662429630756378 0.163207247853279
0.0614666342735291 0.152104690670967
0.0574173480272293 0.14127404987812
0.0533283054828644 0.130115672945976
0.0501252412796021 0.118188366293907
0.0470818281173706 0.10662941634655
0.0446700900793076 0.094816192984581
0.0426394194364548 0.0828088372945786
0.0408998727798462 0.0702202916145325
0.0402624309062958 0.0576857775449753
0.0394079983234406 0.0455942749977112
0.0393157601356506 0.0331542119383812
0.0394310802221298 0.0207674205303192
0.0401614308357239 0.00827185064554214
0.0414164662361145 -0.00415612757205963
0.0430413782596588 -0.0166443921625614
0.0451442152261734 -0.0286236442625523
0.047884613275528 -0.0404741056263447
0.0510403662919998 -0.0522587317973375
0.054355725646019 -0.0636100033298135
0.0584911555051804 -0.0747940521687269
0.0626969635486603 -0.0862113768234849
0.0673839896917343 -0.0966970585286617
0.072578638792038 -0.10718847438693
0.0781356692314148 -0.117091633379459
0.0842623710632324 -0.126533336937428
0.0907906889915466 -0.13593702763319
0.0976491570472717 -0.144735306501389
0.104778319597244 -0.153014823794365
0.112606227397919 -0.16102634370327
0.120658189058304 -0.168112762272358
0.129240840673447 -0.174930445849895
0.138577118515968 -0.181742556393147
0.147591099143028 -0.18696104735136
0.156866014003754 -0.192066498100758
0.167185574769974 -0.196678794920444
0.17750546336174 -0.200597405433655
0.188106626272202 -0.203647077083588
0.19894540309906 -0.205925971269608
0.210417360067368 -0.208320677280426
0.221957057714462 -0.209612846374512
0.234289169311523 -0.210365459322929
0.246683299541473 -0.210330992937088
0.259532123804092 -0.209819942712784
0.272712767124176 -0.208961725234985
0.285980761051178 -0.206716626882553
0.299455553293228 -0.204137936234474
0.313170552253723 -0.201200157403946
0.327081561088562 -0.196785591542721
0.341317147016525 -0.192615054547787
0.355810821056366 -0.187426969408989
0.370577216148376 -0.181722730398178
0.385668903589249 -0.175405561923981
0.400949507951736 -0.168798625469208
0.416382253170013 -0.160946115851402
0.43194717168808 -0.153184399008751
0.447824358940125 -0.144240722060204
0.463916182518005 -0.135654918849468
0.480051815509796 -0.12605819478631
0.496239483356476 -0.115927115082741
0.512479662895203 -0.105850394815207
0.528972566127777 -0.0949736405164003
0.545610368251801 -0.0834642034024
0.562217593193054 -0.0721800066530704
0.579058110713959 -0.0607350766658783
0.595789194107056 -0.0485132746398449
0.612574934959412 -0.0360715389251709
0.629422843456268 -0.023150485008955
0.646574258804321 -0.0109780356287956
0.663613855838776 0.00265426933765411
0.680622398853302 0.0152624398469925
0.697747051715851 0.0289444923400879
0.71500825881958 0.0421025305986404
0.732076168060303 0.0551946014165878
};
\addplot [semithick, red, dashed, forget plot]
table {%
0.75 0
0.767617956255685 0.0126152320166048
0.785221863675294 0.025202759744927
0.802800914680889 0.0377306780229099
0.820344353102624 0.0501672114827695
0.837841479099983 0.0624807937495562
0.855281653865309 0.0746401456242545
0.872654304077727 0.0866143520292023
0.889948926065816 0.0983729374930471
0.90715508962574 0.109885939954111
0.924262441427357 0.121123982665618
0.941260707923679 0.132058343994744
0.958139697658408 0.142661024921306
0.974889302841815 0.152904814062882
0.991499500036706 0.162763350083898
1.00796034976389 0.172211181389763
1.02426199480132 0.18122382306779
1.04039465691463 0.189777811119373
1.0563486317231 0.197850754138649
1.07211428137931 0.205421382738366
1.08768202473299 0.212469597210257
1.10304232467243 0.218976514139914
1.11818567241108 0.224924512976024
1.1331025686362 0.230297283873765
1.147783501695 0.23507987847157
1.16221892339532 0.239258765576722
1.17639922357808 0.242821893954846
1.19031470539574 0.245758764427774
1.20395556419321 0.248060513126141
1.21731187395941 0.249720006825457
1.23037358633393 0.250731949622618
1.24313054782987 0.251092997654816
1.2555725408717 0.250801875167454
1.26768935298946 0.24985948134918
1.27947087567813 0.248268973729967
1.29090722994702 0.246035811751358
1.30198890990496 0.243167744731957
1.31270692999354 0.2396747329547
1.32305295737637 0.235568799170581
1.33301941026506 0.230863819194334
1.34259950666846 0.225575271775738
1.3517872558846 0.219719976188812
1.36057739518197 0.213315848229081
1.36896528374308 0.206381700652654
1.37694677240494 0.198937103974783
1.38451806954475 0.191002311131823
1.39167562078374 0.182598238316676
1.39841601450881 0.173746486857157
1.4047359185686 0.164469388195659
1.41063204764278 0.154790055213092
1.41610115672778 0.144732426840967
1.42114005415556 0.134321297531182
1.42574562723652 0.123582327489247
1.42991487441761 0.112542032954869
1.43364493919855 0.101227758043132
1.43693314251624 0.089667630857217
1.4397770116222 0.0778905070053025
1.44217430452097 0.0659259035746353
1.44412302978876 0.0538039262643269
1.44562146208725 0.0415551919190076
1.44666815398553 0.0292107482394724
1.44726194486432 0.016801992026761
1.44740196775303 0.00436058696240971
1.44708765498222 -0.00808161836067393
1.44631874355061 -0.0204926716768206
1.4450952811249 -0.0328405972734807
1.44341763362016 -0.0450934751015305
1.44128649535024 -0.0572195188465802
1.4387029027798 -0.0691871532341023
1.43566825293285 -0.0809650910678666
1.4321843274821 -0.0925224107662305
1.42825332340439 -0.103828635462315
1.42387789076703 -0.114853815053745
1.41906117760902 -0.125568612881875
1.41380688088891 -0.135944398910193
1.40811930097384 -0.145953351233135
1.40200339507468 -0.155568567310703
1.39546482242266 -0.164764185292558
1.38850997106277 -0.173515513988717
1.38114595342609 -0.181799167392522
1.37338055620066 -0.189593196333819
1.36522213060654 -0.196877206377916
1.35667941319178 -0.203632448466232
1.34776127545165 -0.209841868311401
1.33847641259836 -0.215490103434114
1.328832995749 -0.220563423487838
1.3188383241811 -0.225049619377379
1.30849852098817 -0.228937857346307
1.29781831326193 -0.232218522442798
1.28680092623794 -0.234883078640104
1.27544810236402 -0.23692396914816
1.26376023613114 -0.238334571253006
1.25173659926596 -0.239109208409727
1.23937562223503 -0.239243211851919
1.22680212890194 -0.238807395220756
1.21393543612189 -0.237739020519201
1.2007824030094 -0.236042071113405
1.1873502195155 -0.233721342391173
1.17364639818837 -0.230782623136066
1.15967876862552 -0.227232843837933
1.14545546822952 -0.223080191387264
1.13098492732646 -0.218334192018305
1.11627584919941 -0.213005765794969
1.10133718666704 -0.207107256561536
1.08617811704058 -0.200652441341106
1.070808017043 -0.19365652287622
1.05523643885516 -0.186136108539406
1.03947308802467 -0.178109178316594
1.02352780361101 -0.169595044057188
1.00741054066936 -0.160614301730894
0.991131354992172 -0.151188778049092
0.974700389916618 -0.141341472499412
0.958127864949596 -0.131096495600034
0.941424065943487 -0.120479003995558
0.924599336561409 -0.109515132878708
0.90766407079076 -0.0982319261219058
0.890628706290982 -0.086657264431339
0.873503718391248 -0.074819791786618
0.856299614583182 -0.0627488403957748
0.839026929381251 -0.0504743543738914
0.8216962194482 -0.0380268123406879
0.804318058904672 -0.0254371491254941
0.786903034760972 -0.0127366767652624
0.769461742425004 4.29950186324069e-05
0.752004781254135 0.0128700386771497
0.734542750130382 0.0257124888331812
0.717086243048227 0.0385383222295833
0.699645844712938 0.0513155374272836
0.682232126154625 0.0640122340511521
0.664855640369735 0.0765966913776288
0.647526918007365 0.0890374460548427
0.630256463122704 0.101303368743397
0.613054749024055 0.113363739464377
0.595932214243 0.125188321440869
0.578899258658929 0.136747433220813
0.561966239808458 0.148012018873052
0.545143469406046 0.158953716055882
0.528441210092295 0.169544921769502
0.511869672407924 0.179758855622043
0.495439011959919 0.189569620465462
0.479159326695086 0.198952260295039
0.463040654116175 0.207882815357607
0.447092968153461 0.216338374482455
0.431326175222446 0.224297124738732
0.41575010873165 0.231738398637223
0.400374520922108 0.238642719233979
0.38520907038282 0.244991843656404
0.37026330284855 0.250768805749989
0.355546621900637 0.255957958716022
0.341068244918601 0.260545018739871
0.326837138059129 0.264517110634022
0.312861922218046 0.267862816346776
0.299150740017787 0.270572226688282
0.285711072187049 0.272636995643726
0.272549490830637 0.274050395018263
0.259671337879835 0.274807364778616
0.247080320574249 0.274904551350881
0.234778023333987 0.274340322593555
0.22282111677237 0.273147174350289
0.211212000441657 0.271312180989548
0.199956090718924 0.268842225658862
0.18905951088657 0.265745762806622
0.178528814558833 0.262032850875418
0.168370684504055 0.257715152729482
0.158591652631679 0.252805894649178
0.149197881311021 0.247319783843054
0.140195031151771 0.241272894158711
0.131588220396365 0.234682536784361
0.123382062255884 0.227567134929123
0.115580754025762 0.219946118404367
0.108188187942631 0.211839847297465
0.101208057557953 0.203269566144989
0.0946439418448079 0.19425738357691
0.088499358802492 0.184826268669802
0.0827777883280161 0.17500005435619
0.0774826693237491 0.164803439481848
0.0726173783628224 0.154261983466959
0.0681851974057714 0.143402090111052
0.0641892769319838 0.132250979303424
0.0606325991839522 0.12083664700672
0.0575179445429576 0.109187814856514
0.0548478626398547 0.0973338711778623
0.0526246487503009 0.0853048053160545
0.0508503253165833 0.0731311370573394
0.0495266280133111 0.0608438426868979
0.0486549955532945 0.0484742789692179
0.0482365623412967 0.036054106082026
0.0482721530701452 0.0236152103073717
0.0487622783751904 0.0111896270851532
0.0497071306928156 -0.00119053514168032
0.0511065794918781 -0.0134931700273918
0.0529601650583767 -0.0256862472791051
0.0552670900164109 -0.0377378855745846
0.0580262077740067 -0.0496164247880426
0.061236007111159 -0.0612904981681536
0.0648945922107272 -0.0727291054819839
0.0689996576140302 -0.0839016885812368
0.0735484579177227 -0.0947782113457813
0.0785377725810335 -0.10532924646507
0.0839638670453019 -0.11552607193497
0.0898224525211534 -0.125340780318524
0.0961086482561191 -0.134746403519034
0.10281695173701 -0.143717054765618
0.109941223832714 -0.152228087440402
0.117474696885182 -0.160256267121569
0.125410013591915 -0.167779948892829
0.133739302545796 -0.174779247148552
0.14245429208538 -0.181236180932355
0.151546457798383 -0.187134775860053
0.161007191589854 -0.192461105490089
0.170827973498639 -0.197203261519913
0.181000523658031 -0.201351252889932
0.191516912754772 -0.204896846530735
0.202369615510414 -0.207833373632873
0.213551501760898 -0.210155531524887
0.22505577088336 -0.211859210579139
0.236875844499746 -0.212941368432345
0.249005237340447 -0.213399962707235
0.261437426261433 -0.213233941776159
0.274165733616003 -0.212443283869614
0.28718323530924 -0.211029069576408
0.300482697779478 -0.208993571566475
0.314056543186631 -0.206340347198259
0.327896838839168 -0.203074323164878
0.341995305342881 -0.199201865241865
0.356343337731412 -0.194730829684218
0.370932034454633 -0.189670595476494
0.385752230119493 -0.184032078393973
0.400794528982751 -0.177827728806396
0.416049337197401 -0.171071515546711
0.431506892628147 -0.16377889817583
0.447157291657684 -0.15596678976561
0.462990512824106 -0.147653512015245
0.478996437395876 -0.138858744187699
0.495164867142796 -0.129603467046443
0.511485539633279 -0.119909902709476
0.527948141407569 -0.109801451123518
0.544542319363787 -0.0993026236948103
0.561257690663388 -0.0884389744880547
0.578083851424296 -0.0772370293146803
0.595010384429933 -0.0657242129685941
0.612026866043939 -0.0539287748255556
0.62912287248563 -0.0418797129959582
0.646287985590849 -0.029606697205851
0.663511798156965 -0.0171399905741775
0.680783918949166 -0.00451037045289776
0.698093977427272 0.00825095150093907
0.715431628237671 0.0211124098473716
0.732786555502968 0.0340421663000904
};
\addplot [semithick, green, dash pattern=on 1pt off 3pt on 3pt off 3pt, forget plot]
table {%
0.75 0
0.767347219519284 0.012137253483567
0.784683394455509 0.0242435861667548
0.801997486582194 0.036288155514001
0.819278471531621 0.0482402761288358
0.836515345700615 0.0600694978228124
0.853697133135913 0.0717456830897325
0.870812892392119 0.0832390837898383
0.887851723355235 0.0945204168511733
0.904802774024683 0.105560938798497
0.921655247246804 0.116332518924018
0.938398407392915 0.126807710918802
0.955021586975309 0.136959822789028
0.971514193195148 0.146762984887382
0.987865714417124 0.156192215896701
1.0040657265673 0.165223486610677
1.02010389945277 0.173833781364724
1.03597000300535 0.182001156979107
1.05165391345605 0.189704799085743
1.06714561945411 0.196925075719347
1.08243522815337 0.203643588062188
1.09751297130037 0.209843218238558
1.1123692113737 0.215508174058654
1.12699444784162 0.220624030609957
1.14137932362429 0.225177768584601
1.1555146318663 0.22915780921071
1.16939132313901 0.232554045621132
1.18300051319534 0.235357870442502
1.19633349138035 0.237562199321834
1.20938172974852 0.23916149003295
1.22213689283924 0.240151756735333
1.23459084790774 0.240530578918402
1.24673567520343 0.24029710459028
1.25856367765532 0.239452047402427
1.27006738911946 0.237997677671498
1.28123958024499 0.235937807669022
1.29207326111455 0.233277772045945
1.30256168017249 0.230024404727089
1.31269831956044 0.226186013888674
1.32247688771958 0.221772356568193
1.33189131078413 0.216794613983565
1.34093572465289 0.211265367835258
1.34960446953272 0.205198576949498
1.35789208820684 0.198609552872807
1.36579332846363 0.191514932670796
1.37330314928634 0.18393264729003
1.38041672978698 0.175881884318464
1.38712947958844 0.167383044631953
1.39343704941026 0.158457693037802
1.39933534088867 0.149128503482261
1.40482051503024 0.139419199628507
1.40988899904863 0.129354491659788
1.41453749160152 0.118960010081206
1.41876296661194 0.108262237150103
1.42256267593454 0.0972884364110782
1.42593415113889 0.086066580678514
1.42887520465437 0.0746252787098322
1.43138393047621 0.0629937007473294
1.43345870458359 0.0512015030695605
1.43509818517756 0.0392787516771522
1.43630131281055 0.0272558452352367
1.43706731045357 0.0151634373994425
1.43739568352819 0.00303235866024866
1.43728621991849 -0.00910646215137196
1.43673898997073 -0.0212220765475008
1.43575434648365 -0.0332835951024502
1.43433292468914 -0.0452602660529673
1.43247564221903 -0.0571215536269705
1.43018369904855 -0.0688372159533007
1.42745857739782 -0.0803773824121825
1.4243020415584 -0.0917126302942692
1.42071613759078 -0.102814060643163
1.41670319280875 -0.113653373158817
1.41226581492764 -0.124202940032363
1.40740689070718 -0.134435878560225
1.40212958387022 -0.144326122339201
1.3964373320371 -0.153848490767307
1.39033384239815 -0.162978756463708
1.38382308587764 -0.171693710079292
1.37690928964889 -0.179971221816331
1.36959692806383 -0.187790298849553
1.36189071236441 -0.195131137800147
1.35379557991389 -0.201975171526458
1.34531668404164 -0.20830510981277
1.33645938581161 -0.214104974062127
1.32722924897007 -0.219360126749639
1.31763203892423 -0.224057297001799
1.30767372588241 -0.22818460402503
1.2973604914176 -0.231731580045289
1.28669873695777 -0.234689193904312
1.27569509230234 -0.237049875620184
1.26435642233521 -0.238807541316647
1.25268983059913 -0.239957617226964
1.24070265912099 -0.240497061156197
1.22840248459827 -0.24042437986202
1.21579711159144 -0.239739641182038
1.20289456363565 -0.238444480231447
1.1897030732081 -0.236542099469318
1.17623107134149 -0.234037262792463
1.16248717745014 -0.230936284032406
1.14848018970572 -0.227247010319085
1.13421907610845 -0.222978800771493
1.11971296626323 -0.218142500920566
1.10497114378721 -0.212750413195198
1.09000303923492 -0.206816263729265
1.07481822341698 -0.200355165686855
1.05942640099553 -0.193383579258294
1.04383740425637 -0.185919268450478
1.02806118697728 -0.177981254778958
1.01210781833129 -0.169589767962932
0.995987476780275 -0.160766193724954
0.979710443928391 -0.151533018802209
0.963287098315684 -0.141913773283855
0.946727909140567 -0.131932970397911
0.930043429905878 -0.121616043880467
0.91324429198767 -0.110989283069269
0.896341198128817 -0.100079765872392
0.879344915861624 -0.0889152897708354
0.862266270864871 -0.0775243010211479
0.845116140261585 -0.0659358222307342
0.827905445864243 -0.0541793784842489
0.810645147374399 -0.042284922204474
0.793346235543765 -0.0302827569353333
0.776019725303823 -0.0182034602382608
0.758676648870971 -0.00607780589600706
0.74132804883419 0.00606331437980999
0.723984971232126 0.0181889695395348
0.706658458626486 0.0302682678858693
0.689359543178601 0.0422704356646973
0.672099239736024 0.0541648953575106
0.654888538936098 0.0659213434775336
0.637738400333409 0.077509827673207
0.620659745558167 0.0889008229448788
0.603663451512546 0.10006530678342
0.586760343612044 0.110974833042994
0.569961189078862 0.121601604364455
0.55327669029407 0.131918542970798
0.536717478214957 0.141899359661781
0.520294105863188 0.151518620841353
0.504017041888231 0.160751813418715
0.487896664208613 0.169575407431921
0.471943253730781 0.177966916251536
0.456166988141264 0.18590495423109
0.440577935762111 0.193369291680399
0.425186049451712 0.200340907046881
0.410001160522753 0.206802036197834
0.395032972635676 0.212736218702091
0.38029105560979 0.218128341010777
0.36578483907538 0.222964676431608
0.351523605870586 0.227232921776542
0.337516485069478 0.23092223053541
0.323772444518238 0.234023242385741
0.310300282763089 0.236528108790522
0.297108620287721 0.238430514364134
0.284205890052545 0.239725693611976
0.27160032745433 0.240410442590784
0.259299960006205 0.240483125024951
0.247312597261469 0.239943672488383
0.2356458217318 0.238793578459958
0.224306981713199 0.237035886402939
0.21330318694315 0.234675172482025
0.202641307788403 0.231717524032126
0.192327978173203 0.228170515287301
0.1823696017639 0.22404318200088
0.172772360193836 0.219345996319791
0.163542221572307 0.214090842618848
0.154684947376684 0.208290994107505
0.146206096155645 0.201961089155653
0.138111023180157 0.195117105712544
0.130404876038488 0.187776332069152
0.12309258692145 0.179957332525672
0.116178862798813 0.171679907117356
0.109668174793245 0.162965045212245
0.10356474787866 0.153834873345794
0.0978725516915446 0.144312598005936
0.0925952928735381 0.134422444219678
0.0877364090495248 0.124189590767097
0.0832990643281257 0.113640102728524
0.0792861460924091 0.102800861916777
0.0757002628080311 0.0916994956001386
0.0725437425870994 0.0803643038046822
0.069818632284583 0.0688241854024739
0.0675266969522104 0.0571085631419612
0.0656694195214907 0.0452473077513642
0.0642480006271671 0.0332706612373471
0.0632633585132331 0.0212091595029022
0.0627161289858991 0.00909355441514592
0.0626066653929575 -0.00304526453801165
0.0629350386185861 -0.0151763488558262
0.0637010370885092 -0.0272687697454203
0.0649041667842718 -0.0392916968007361
0.0665436512687222 -0.0512144764858195
0.0686184317292081 -0.0630067102720941
0.0711271670520408 -0.0746383322854955
0.0740682339532131 -0.0860796863272447
0.0774397272079407 -0.0973016021399556
0.0812394600470219 -0.108275470796062
0.0854649648223936 -0.118973319084073
0.0901134940873028 -0.12936788275431
0.0951820222852561 -0.139432678452301
0.100667248289028 -0.149142074107302
0.10656559906303 -0.158471357449301
0.112873234718821 -0.167396802199979
0.119586055168304 -0.175895731332128
0.126699708426136 -0.183946576646328
0.13420960035749 -0.191528933821879
0.142110905322839 -0.198623612126504
0.150398576795015 -0.205212678179984
0.159067356720643 -0.211279493592426
0.16811178230224 -0.216808746901301
0.177526189102462 -0.221786480886398
0.187304709947427 -0.226200116851138
0.197441269929379 -0.230038477621279
0.207929578656845 -0.233291810717099
0.218763121506754 -0.235951812451675
0.229935151803763 -0.238011652801433
0.241438685552319 -0.239466000064469
0.253266499708268 -0.240311043799429
0.265411134226006 -0.240544514421553
0.277864897472831 -0.240165698073282
0.290619874197545 -0.23917544583971
0.303667935101709 -0.237576176879925
0.317000747134867 -0.235371875470324
0.330609783830706 -0.232568082244952
0.344486335235128 -0.229171880066449
0.358621517191145 -0.225191874998346
0.373006279910045 -0.220638172815656
0.387631415867 -0.215522351422736
0.402487567119295 -0.209857429471725
0.417565232169048 -0.203657831407191
0.432854772492479 -0.196939349109845
0.448346418844808 -0.189719100275549
0.46403027743119 -0.182015483643688
0.47989633601438 -0.173848131178146
0.495934470011695 -0.165237857301697
0.512134448618413 -0.156206605287707
0.528485940982374 -0.146777390919585
0.544978522444945 -0.136974243536887
0.561601680856407 -0.126822144596177
0.578344822968638 -0.116346963884112
0.595197280904322 -0.105575393529208
0.61214831869946 -0.094534879967153
0.629187138914278 -0.0832535540222168
0.646302889306651 -0.071760159274263
0.663484669561462 -0.0600839788869683
0.680721538069075 -0.0482547610782518
0.698002518745858 -0.0363026434185232
0.715316607889725 -0.0242580761462783
0.732652781063624 -0.012151744693772
};
\addplot [line width = \linewidthEightC, color = reference, opacity=\opacityRef, forget plot]
table {%
0.75 0
0.75294017791748 0.00237509608268738
0.768159806728363 0.0137370452284813
0.787286400794983 0.027612492442131
0.80359011888504 0.0394390001893044
0.821533620357513 0.051962174475193
0.838860690593719 0.0644481852650642
0.855940937995911 0.0763964280486107
0.873260617256165 0.0882028713822365
0.890427529811859 0.0999084040522575
0.907608866691589 0.111399836838245
0.924478769302368 0.122843377292156
0.94128155708313 0.133893869817257
0.958072543144226 0.144275166094303
0.975005626678467 0.154694758355618
0.991557240486145 0.16423537582159
1.00795030593872 0.173708550632
1.02395749092102 0.182243369519711
1.03999888896942 0.190217517316341
1.05562460422516 0.198446206748486
1.07160830497742 0.205872170627117
1.08707475662231 0.21245788782835
1.10221028327942 0.218515358865261
1.11716163158417 0.224298290908337
1.13198029994965 0.229186378419399
1.14688897132874 0.233549050986767
1.16086065769196 0.237465016543865
1.17503499984741 0.240555487573147
1.18872964382172 0.24330922216177
1.20229959487915 0.245204500854015
1.21544873714447 0.246397130191326
1.22817444801331 0.246982745826244
1.2406553030014 0.247403793036938
1.25272440910339 0.246761612594128
1.26461219787598 0.245623074471951
1.27606642246246 0.244084946811199
1.28716170787811 0.241434000432491
1.29786157608032 0.238531522452831
1.30824422836304 0.234632335603237
1.31826210021973 0.230279080569744
1.32780110836029 0.225536189973354
1.33677089214325 0.219979397952557
1.34620463848114 0.213832996785641
1.35454559326172 0.207702599465847
1.36251962184906 0.200578980147839
1.37034797668457 0.192839972674847
1.37703931331635 0.184960596263409
1.38459610939026 0.176415227353573
1.39104700088501 0.167523898184299
1.39711999893188 0.158063031733036
1.40292966365814 0.147779159247875
1.40846836566925 0.137624688446522
1.41324532032013 0.126867972314358
1.4177839756012 0.116383798420429
1.42155647277832 0.104965843260288
1.42501783370972 0.0932668223977089
1.42814338207245 0.0820035561919212
1.43043613433838 0.0699286386370659
1.43245804309845 0.0578133389353752
1.43399858474731 0.0458051338791847
1.43528831005096 0.0338477790355682
1.4359484910965 0.0211388096213341
1.4362907409668 0.00838829576969147
1.43631136417389 -0.00372567027807236
1.43580174446106 -0.0160589963197708
1.43493318557739 -0.028281282633543
1.43338453769684 -0.040928591042757
1.43149328231812 -0.0527229700237513
1.42910599708557 -0.0646350234746933
1.42645978927612 -0.0770538584329188
1.42332124710083 -0.0883679753169417
1.41965770721436 -0.0994634628295898
1.41572630405426 -0.110652320086956
1.41129493713379 -0.121757119894028
1.40635752677917 -0.131892628967762
1.40082514286041 -0.14273627102375
1.39509034156799 -0.152044199407101
1.38882493972778 -0.161253422498703
1.38219368457794 -0.170228816568851
1.37510108947754 -0.178672321140766
1.36742150783539 -0.187066823244095
1.35968327522278 -0.19458743929863
1.35143828392029 -0.201422847807407
1.34280586242676 -0.207816831767559
1.33381867408752 -0.213798142969608
1.32410490512848 -0.219563521444798
1.31467890739441 -0.224183849990368
1.3044251203537 -0.228262804448605
1.2939088344574 -0.231961451470852
1.28306591510773 -0.234605826437473
1.27172458171844 -0.237101994454861
1.26006972789764 -0.238849796354771
1.24814701080322 -0.239693008363247
1.2358580827713 -0.240049310028553
1.22351276874542 -0.240107305347919
1.21082818508148 -0.239089332520962
1.19770109653473 -0.237672694027424
1.18453013896942 -0.235307909548283
1.17088329792023 -0.232687421143055
1.1567051410675 -0.229024015367031
1.14238381385803 -0.225041784346104
1.12785720825195 -0.220265813171864
1.11316776275635 -0.215268589556217
1.09818339347839 -0.209237299859524
1.08338356018066 -0.202820651233196
1.06794583797455 -0.195916198194027
1.05236792564392 -0.188208840787411
1.03703284263611 -0.180393539369106
1.02124416828156 -0.171867035329342
1.00545716285706 -0.16310016810894
0.989356756210327 -0.15352563560009
0.972778916358948 -0.143434226512909
0.956225633621216 -0.133039638400078
0.939662218093872 -0.122062340378761
0.923109769821167 -0.111082822084427
0.906195163726807 -0.0993808973580599
0.889241993427277 -0.0875657815486193
0.872110307216644 -0.0755821075290442
0.854916989803314 -0.0635634567588568
0.837713479995728 -0.050781374797225
0.820423364639282 -0.0385473147034645
0.803413510322571 -0.0260055214166641
0.785965919494629 -0.013111911714077
0.768358469009399 0.000106543302536011
0.751115500926971 0.0129247754812241
0.73374992609024 0.0261640474200249
0.716224789619446 0.0388737097382545
0.698909163475037 0.0522861704230309
0.681838810443878 0.0651128515601158
0.664553284645081 0.0777009502053261
0.647286415100098 0.090551070868969
0.630425214767456 0.102790541946888
0.613296627998352 0.115175895392895
0.596336126327515 0.127441965043545
0.579163908958435 0.139155246317387
0.562465250492096 0.15093844383955
0.545766651630402 0.162269197404385
0.529273390769958 0.173190660774708
0.51268196105957 0.183695875108242
0.4964399933815 0.193465910851955
0.480053424835205 0.203395329415798
0.464224576950073 0.212629728019238
0.448258519172668 0.221284441649914
0.432740092277527 0.229295335710049
0.416834771633148 0.237202219665051
0.401243805885315 0.244738094508648
0.385787725448608 0.251547031104565
0.371005147695541 0.257692597806454
0.355980902910233 0.263131968677044
0.341145485639572 0.268218599259853
0.326730161905289 0.272647045552731
0.312474966049194 0.276453785598278
0.298882275819778 0.279494993388653
0.28520068526268 0.282219104468822
0.272087067365646 0.284008346498013
0.258950054645538 0.285131268203259
0.245983600616455 0.285887114703655
0.233611136674881 0.28609161823988
0.221405982971191 0.285494647920132
0.20970156788826 0.284044526517391
0.1980120241642 0.28198217600584
0.186769247055054 0.279410324990749
0.176144570112228 0.276036165654659
0.165479987859726 0.272286407649517
0.15565624833107 0.2680509313941
0.146040633320808 0.263224177062511
0.136874347925186 0.257612310349941
0.12812452018261 0.251376025378704
0.119499862194061 0.244791649281979
0.111412525177002 0.237570963799953
0.103574514389038 0.229903779923916
0.0962880253791809 0.221309326589108
0.089401051402092 0.212685637176037
0.0830144286155701 0.203794501721859
0.0772256255149841 0.194383762776852
0.07145756483078 0.184206299483776
0.0662515163421631 0.173591040074825
0.0614678114652634 0.163370959460735
0.0571825951337814 0.152453206479549
0.0533227324485779 0.141241855919361
0.0497516691684723 0.12981004267931
0.0467032641172409 0.118126250803471
0.0440839380025864 0.105835683643818
0.0418252050876617 0.0934018865227699
0.0400712490081787 0.0820050463080406
0.0391712635755539 0.0687207505106926
0.0381738990545273 0.0567026361823082
0.0379651039838791 0.0440792962908745
0.038131520152092 0.0310261994600296
0.0385419279336929 0.0190669521689415
0.0395954549312592 0.00681092590093613
0.0411359667778015 -0.00554528087377548
0.0431600511074066 -0.017744991928339
0.0455216318368912 -0.0292831137776375
0.0484930574893951 -0.0413609109818935
0.051739901304245 -0.0527868866920471
0.0553841441869736 -0.0640644393861294
0.0596607029438019 -0.0750972889363766
0.0643247961997986 -0.0857468256726861
0.0694724470376968 -0.0963285267353058
0.0749500691890717 -0.106518965214491
0.0810398757457733 -0.116136658936739
0.0873371958732605 -0.125286072492599
0.0942026376724243 -0.134336654096842
0.101256474852562 -0.142514295876026
0.109077632427216 -0.15071589499712
0.117127746343613 -0.158033512532711
0.125378668308258 -0.164948351681232
0.134114637970924 -0.17105970531702
0.143123865127563 -0.177005529403687
0.152937799692154 -0.18222401291132
0.163103193044662 -0.186953403055668
0.173221260309219 -0.190920412540436
0.183625072240829 -0.194200716912746
0.194965660572052 -0.196951784193516
0.206088751554489 -0.199146248400211
0.217803478240967 -0.200490795075893
0.229935109615326 -0.201741345226765
0.24194547533989 -0.201418749988079
0.254892319440842 -0.201113872230053
0.26743221282959 -0.200192578136921
0.281167894601822 -0.198392532765865
0.294760167598724 -0.195902235805988
0.30863356590271 -0.192821599543095
0.322874188423157 -0.18914021551609
0.337162673473358 -0.184776149690151
0.351760894060135 -0.179963678121567
0.36681604385376 -0.17448402941227
0.381659954786301 -0.168497569859028
0.396955609321594 -0.161860823631287
0.412329614162445 -0.154562972486019
0.428416669368744 -0.146825082600117
0.444254100322723 -0.138614512979984
0.46024763584137 -0.129717249423265
0.476293623447418 -0.120018996298313
0.492700934410095 -0.109937772154808
0.509267210960388 -0.099963366985321
0.526154398918152 -0.0890907291322947
0.542466402053833 -0.0781746823340654
0.559093475341797 -0.0676257507875562
0.575746059417725 -0.0558312721550465
0.592531442642212 -0.0440070275217295
0.609867513179779 -0.0319154262542725
0.626823544502258 -0.019564937800169
0.643898963928223 -0.00686319917440414
0.661305546760559 0.00596896559000015
0.678427219390869 0.0188850536942482
0.695666074752808 0.0321779176592827
0.713037848472595 0.0450780540704727
0.730046212673187 0.0576290115714073
};
\addplot [semithick, red, dashed, forget plot]
table {%
0.75 0
0.767617956255685 0.0126152320166048
0.785221863675294 0.025202759744927
0.802800914680889 0.0377306780229099
0.820344353102624 0.0501672114827695
0.837841479099983 0.0624807937495562
0.855281653865309 0.0746401456242545
0.872654304077727 0.0866143520292023
0.889948926065816 0.0983729374930471
0.90715508962574 0.109885939954111
0.924262441427357 0.121123982665618
0.941260707923679 0.132058343994744
0.958139697658408 0.142661024921306
0.974889302841815 0.152904814062882
0.991499500036706 0.162763350083898
1.00796034976389 0.172211181389763
1.02426199480132 0.18122382306779
1.04039465691463 0.189777811119373
1.0563486317231 0.197850754138649
1.07211428137931 0.205421382738366
1.08768202473299 0.212469597210257
1.10304232467243 0.218976514139914
1.11818567241108 0.224924512976024
1.1331025686362 0.230297283873765
1.147783501695 0.23507987847157
1.16221892339532 0.239258765576722
1.17639922357808 0.242821893954846
1.19031470539574 0.245758764427774
1.20395556419321 0.248060513126141
1.21731187395941 0.249720006825457
1.23037358633393 0.250731949622618
1.24313054782987 0.251092997654816
1.2555725408717 0.250801875167454
1.26768935298946 0.24985948134918
1.27947087567813 0.248268973729967
1.29090722994702 0.246035811751358
1.30198890990496 0.243167744731957
1.31270692999354 0.2396747329547
1.32305295737637 0.235568799170581
1.33301941026506 0.230863819194334
1.34259950666846 0.225575271775738
1.3517872558846 0.219719976188812
1.36057739518197 0.213315848229081
1.36896528374308 0.206381700652654
1.37694677240494 0.198937103974783
1.38451806954475 0.191002311131823
1.39167562078374 0.182598238316676
1.39841601450881 0.173746486857157
1.4047359185686 0.164469388195659
1.41063204764278 0.154790055213092
1.41610115672778 0.144732426840967
1.42114005415556 0.134321297531182
1.42574562723652 0.123582327489247
1.42991487441761 0.112542032954869
1.43364493919855 0.101227758043132
1.43693314251624 0.089667630857217
1.4397770116222 0.0778905070053025
1.44217430452097 0.0659259035746353
1.44412302978876 0.0538039262643269
1.44562146208725 0.0415551919190076
1.44666815398553 0.0292107482394724
1.44726194486432 0.016801992026761
1.44740196775303 0.00436058696240971
1.44708765498222 -0.00808161836067393
1.44631874355061 -0.0204926716768206
1.4450952811249 -0.0328405972734807
1.44341763362016 -0.0450934751015305
1.44128649535024 -0.0572195188465802
1.4387029027798 -0.0691871532341023
1.43566825293285 -0.0809650910678666
1.4321843274821 -0.0925224107662305
1.42825332340439 -0.103828635462315
1.42387789076703 -0.114853815053745
1.41906117760902 -0.125568612881875
1.41380688088891 -0.135944398910193
1.40811930097384 -0.145953351233135
1.40200339507468 -0.155568567310703
1.39546482242266 -0.164764185292558
1.38850997106277 -0.173515513988717
1.38114595342609 -0.181799167392522
1.37338055620066 -0.189593196333819
1.36522213060654 -0.196877206377916
1.35667941319178 -0.203632448466232
1.34776127545165 -0.209841868311401
1.33847641259836 -0.215490103434114
1.328832995749 -0.220563423487838
1.3188383241811 -0.225049619377379
1.30849852098817 -0.228937857346307
1.29781831326193 -0.232218522442798
1.28680092623794 -0.234883078640104
1.27544810236402 -0.23692396914816
1.26376023613114 -0.238334571253006
1.25173659926596 -0.239109208409727
1.23937562223503 -0.239243211851919
1.22680212890194 -0.238807395220756
1.21393543612189 -0.237739020519201
1.2007824030094 -0.236042071113405
1.1873502195155 -0.233721342391173
1.17364639818837 -0.230782623136066
1.15967876862552 -0.227232843837933
1.14545546822952 -0.223080191387264
1.13098492732646 -0.218334192018305
1.11627584919941 -0.213005765794969
1.10133718666704 -0.207107256561536
1.08617811704058 -0.200652441341106
1.070808017043 -0.19365652287622
1.05523643885516 -0.186136108539406
1.03947308802467 -0.178109178316594
1.02352780361101 -0.169595044057188
1.00741054066936 -0.160614301730894
0.991131354992172 -0.151188778049092
0.974700389916618 -0.141341472499412
0.958127864949596 -0.131096495600034
0.941424065943487 -0.120479003995558
0.924599336561409 -0.109515132878708
0.90766407079076 -0.0982319261219058
0.890628706290982 -0.086657264431339
0.873503718391248 -0.074819791786618
0.856299614583182 -0.0627488403957748
0.839026929381251 -0.0504743543738914
0.8216962194482 -0.0380268123406879
0.804318058904672 -0.0254371491254941
0.786903034760972 -0.0127366767652624
0.769461742425004 4.29950186324069e-05
0.752004781254135 0.0128700386771497
0.734542750130382 0.0257124888331812
0.717086243048227 0.0385383222295833
0.699645844712938 0.0513155374272836
0.682232126154625 0.0640122340511521
0.664855640369735 0.0765966913776288
0.647526918007365 0.0890374460548427
0.630256463122704 0.101303368743397
0.613054749024055 0.113363739464377
0.595932214243 0.125188321440869
0.578899258658929 0.136747433220813
0.561966239808458 0.148012018873052
0.545143469406046 0.158953716055882
0.528441210092295 0.169544921769502
0.511869672407924 0.179758855622043
0.495439011959919 0.189569620465462
0.479159326695086 0.198952260295039
0.463040654116175 0.207882815357607
0.447092968153461 0.216338374482455
0.431326175222446 0.224297124738732
0.41575010873165 0.231738398637223
0.400374520922108 0.238642719233979
0.38520907038282 0.244991843656404
0.37026330284855 0.250768805749989
0.355546621900637 0.255957958716022
0.341068244918601 0.260545018739871
0.326837138059129 0.264517110634022
0.312861922218046 0.267862816346776
0.299150740017787 0.270572226688282
0.285711072187049 0.272636995643726
0.272549490830637 0.274050395018263
0.259671337879835 0.274807364778616
0.247080320574249 0.274904551350881
0.234778023333987 0.274340322593555
0.22282111677237 0.273147174350289
0.211212000441657 0.271312180989548
0.199956090718924 0.268842225658862
0.18905951088657 0.265745762806622
0.178528814558833 0.262032850875418
0.168370684504055 0.257715152729482
0.158591652631679 0.252805894649178
0.149197881311021 0.247319783843054
0.140195031151771 0.241272894158711
0.131588220396365 0.234682536784361
0.123382062255884 0.227567134929123
0.115580754025762 0.219946118404367
0.108188187942631 0.211839847297465
0.101208057557953 0.203269566144989
0.0946439418448079 0.19425738357691
0.088499358802492 0.184826268669802
0.0827777883280161 0.17500005435619
0.0774826693237491 0.164803439481848
0.0726173783628224 0.154261983466959
0.0681851974057714 0.143402090111052
0.0641892769319838 0.132250979303424
0.0606325991839522 0.12083664700672
0.0575179445429576 0.109187814856514
0.0548478626398547 0.0973338711778623
0.0526246487503009 0.0853048053160545
0.0508503253165833 0.0731311370573394
0.0495266280133111 0.0608438426868979
0.0486549955532945 0.0484742789692179
0.0482365623412967 0.036054106082026
0.0482721530701452 0.0236152103073717
0.0487622783751904 0.0111896270851532
0.0497071306928156 -0.00119053514168032
0.0511065794918781 -0.0134931700273918
0.0529601650583767 -0.0256862472791051
0.0552670900164109 -0.0377378855745846
0.0580262077740067 -0.0496164247880426
0.061236007111159 -0.0612904981681536
0.0648945922107272 -0.0727291054819839
0.0689996576140302 -0.0839016885812368
0.0735484579177227 -0.0947782113457813
0.0785377725810335 -0.10532924646507
0.0839638670453019 -0.11552607193497
0.0898224525211534 -0.125340780318524
0.0961086482561191 -0.134746403519034
0.10281695173701 -0.143717054765618
0.109941223832714 -0.152228087440402
0.117474696885182 -0.160256267121569
0.125410013591915 -0.167779948892829
0.133739302545796 -0.174779247148552
0.14245429208538 -0.181236180932355
0.151546457798383 -0.187134775860053
0.161007191589854 -0.192461105490089
0.170827973498639 -0.197203261519913
0.181000523658031 -0.201351252889932
0.191516912754772 -0.204896846530735
0.202369615510414 -0.207833373632873
0.213551501760898 -0.210155531524887
0.22505577088336 -0.211859210579139
0.236875844499746 -0.212941368432345
0.249005237340447 -0.213399962707235
0.261437426261433 -0.213233941776159
0.274165733616003 -0.212443283869614
0.28718323530924 -0.211029069576408
0.300482697779478 -0.208993571566475
0.314056543186631 -0.206340347198259
0.327896838839168 -0.203074323164878
0.341995305342881 -0.199201865241865
0.356343337731412 -0.194730829684218
0.370932034454633 -0.189670595476494
0.385752230119493 -0.184032078393973
0.400794528982751 -0.177827728806396
0.416049337197401 -0.171071515546711
0.431506892628147 -0.16377889817583
0.447157291657684 -0.15596678976561
0.462990512824106 -0.147653512015245
0.478996437395876 -0.138858744187699
0.495164867142796 -0.129603467046443
0.511485539633279 -0.119909902709476
0.527948141407569 -0.109801451123518
0.544542319363787 -0.0993026236948103
0.561257690663388 -0.0884389744880547
0.578083851424296 -0.0772370293146803
0.595010384429933 -0.0657242129685941
0.612026866043939 -0.0539287748255556
0.62912287248563 -0.0418797129959582
0.646287985590849 -0.029606697205851
0.663511798156965 -0.0171399905741775
0.680783918949166 -0.00451037045289776
0.698093977427272 0.00825095150093907
0.715431628237671 0.0211124098473716
0.732786555502968 0.0340421663000904
};
\addplot [semithick, green, dash pattern=on 1pt off 3pt on 3pt off 3pt, forget plot]
table {%
0.75 0
0.767347219519284 0.012137253483567
0.784683394455509 0.0242435861667548
0.801997486582194 0.036288155514001
0.819278471531621 0.0482402761288358
0.836515345700615 0.0600694978228124
0.853697133135913 0.0717456830897325
0.870812892392119 0.0832390837898383
0.887851723355235 0.0945204168511733
0.904802774024683 0.105560938798497
0.921655247246804 0.116332518924018
0.938398407392915 0.126807710918802
0.955021586975309 0.136959822789028
0.971514193195148 0.146762984887382
0.987865714417124 0.156192215896701
1.0040657265673 0.165223486610677
1.02010389945277 0.173833781364724
1.03597000300535 0.182001156979107
1.05165391345605 0.189704799085743
1.06714561945411 0.196925075719347
1.08243522815337 0.203643588062188
1.09751297130037 0.209843218238558
1.1123692113737 0.215508174058654
1.12699444784162 0.220624030609957
1.14137932362429 0.225177768584601
1.1555146318663 0.22915780921071
1.16939132313901 0.232554045621132
1.18300051319534 0.235357870442502
1.19633349138035 0.237562199321834
1.20938172974852 0.23916149003295
1.22213689283924 0.240151756735333
1.23459084790774 0.240530578918402
1.24673567520343 0.24029710459028
1.25856367765532 0.239452047402427
1.27006738911946 0.237997677671498
1.28123958024499 0.235937807669022
1.29207326111455 0.233277772045945
1.30256168017249 0.230024404727089
1.31269831956044 0.226186013888674
1.32247688771958 0.221772356568193
1.33189131078413 0.216794613983565
1.34093572465289 0.211265367835258
1.34960446953272 0.205198576949498
1.35789208820684 0.198609552872807
1.36579332846363 0.191514932670796
1.37330314928634 0.18393264729003
1.38041672978698 0.175881884318464
1.38712947958844 0.167383044631953
1.39343704941026 0.158457693037802
1.39933534088867 0.149128503482261
1.40482051503024 0.139419199628507
1.40988899904863 0.129354491659788
1.41453749160152 0.118960010081206
1.41876296661194 0.108262237150103
1.42256267593454 0.0972884364110782
1.42593415113889 0.086066580678514
1.42887520465437 0.0746252787098322
1.43138393047621 0.0629937007473294
1.43345870458359 0.0512015030695605
1.43509818517756 0.0392787516771522
1.43630131281055 0.0272558452352367
1.43706731045357 0.0151634373994425
1.43739568352819 0.00303235866024866
1.43728621991849 -0.00910646215137196
1.43673898997073 -0.0212220765475008
1.43575434648365 -0.0332835951024502
1.43433292468914 -0.0452602660529673
1.43247564221903 -0.0571215536269705
1.43018369904855 -0.0688372159533007
1.42745857739782 -0.0803773824121825
1.4243020415584 -0.0917126302942692
1.42071613759078 -0.102814060643163
1.41670319280875 -0.113653373158817
1.41226581492764 -0.124202940032363
1.40740689070718 -0.134435878560225
1.40212958387022 -0.144326122339201
1.3964373320371 -0.153848490767307
1.39033384239815 -0.162978756463708
1.38382308587764 -0.171693710079292
1.37690928964889 -0.179971221816331
1.36959692806383 -0.187790298849553
1.36189071236441 -0.195131137800147
1.35379557991389 -0.201975171526458
1.34531668404164 -0.20830510981277
1.33645938581161 -0.214104974062127
1.32722924897007 -0.219360126749639
1.31763203892423 -0.224057297001799
1.30767372588241 -0.22818460402503
1.2973604914176 -0.231731580045289
1.28669873695777 -0.234689193904312
1.27569509230234 -0.237049875620184
1.26435642233521 -0.238807541316647
1.25268983059913 -0.239957617226964
1.24070265912099 -0.240497061156197
1.22840248459827 -0.24042437986202
1.21579711159144 -0.239739641182038
1.20289456363565 -0.238444480231447
1.1897030732081 -0.236542099469318
1.17623107134149 -0.234037262792463
1.16248717745014 -0.230936284032406
1.14848018970572 -0.227247010319085
1.13421907610845 -0.222978800771493
1.11971296626323 -0.218142500920566
1.10497114378721 -0.212750413195198
1.09000303923492 -0.206816263729265
1.07481822341698 -0.200355165686855
1.05942640099553 -0.193383579258294
1.04383740425637 -0.185919268450478
1.02806118697728 -0.177981254778958
1.01210781833129 -0.169589767962932
0.995987476780275 -0.160766193724954
0.979710443928391 -0.151533018802209
0.963287098315684 -0.141913773283855
0.946727909140567 -0.131932970397911
0.930043429905878 -0.121616043880467
0.91324429198767 -0.110989283069269
0.896341198128817 -0.100079765872392
0.879344915861624 -0.0889152897708354
0.862266270864871 -0.0775243010211479
0.845116140261585 -0.0659358222307342
0.827905445864243 -0.0541793784842489
0.810645147374399 -0.042284922204474
0.793346235543765 -0.0302827569353333
0.776019725303823 -0.0182034602382608
0.758676648870971 -0.00607780589600706
0.74132804883419 0.00606331437980999
0.723984971232126 0.0181889695395348
0.706658458626486 0.0302682678858693
0.689359543178601 0.0422704356646973
0.672099239736024 0.0541648953575106
0.654888538936098 0.0659213434775336
0.637738400333409 0.077509827673207
0.620659745558167 0.0889008229448788
0.603663451512546 0.10006530678342
0.586760343612044 0.110974833042994
0.569961189078862 0.121601604364455
0.55327669029407 0.131918542970798
0.536717478214957 0.141899359661781
0.520294105863188 0.151518620841353
0.504017041888231 0.160751813418715
0.487896664208613 0.169575407431921
0.471943253730781 0.177966916251536
0.456166988141264 0.18590495423109
0.440577935762111 0.193369291680399
0.425186049451712 0.200340907046881
0.410001160522753 0.206802036197834
0.395032972635676 0.212736218702091
0.38029105560979 0.218128341010777
0.36578483907538 0.222964676431608
0.351523605870586 0.227232921776542
0.337516485069478 0.23092223053541
0.323772444518238 0.234023242385741
0.310300282763089 0.236528108790522
0.297108620287721 0.238430514364134
0.284205890052545 0.239725693611976
0.27160032745433 0.240410442590784
0.259299960006205 0.240483125024951
0.247312597261469 0.239943672488383
0.2356458217318 0.238793578459958
0.224306981713199 0.237035886402939
0.21330318694315 0.234675172482025
0.202641307788403 0.231717524032126
0.192327978173203 0.228170515287301
0.1823696017639 0.22404318200088
0.172772360193836 0.219345996319791
0.163542221572307 0.214090842618848
0.154684947376684 0.208290994107505
0.146206096155645 0.201961089155653
0.138111023180157 0.195117105712544
0.130404876038488 0.187776332069152
0.12309258692145 0.179957332525672
0.116178862798813 0.171679907117356
0.109668174793245 0.162965045212245
0.10356474787866 0.153834873345794
0.0978725516915446 0.144312598005936
0.0925952928735381 0.134422444219678
0.0877364090495248 0.124189590767097
0.0832990643281257 0.113640102728524
0.0792861460924091 0.102800861916777
0.0757002628080311 0.0916994956001386
0.0725437425870994 0.0803643038046822
0.069818632284583 0.0688241854024739
0.0675266969522104 0.0571085631419612
0.0656694195214907 0.0452473077513642
0.0642480006271671 0.0332706612373471
0.0632633585132331 0.0212091595029022
0.0627161289858991 0.00909355441514592
0.0626066653929575 -0.00304526453801165
0.0629350386185861 -0.0151763488558262
0.0637010370885092 -0.0272687697454203
0.0649041667842718 -0.0392916968007361
0.0665436512687222 -0.0512144764858195
0.0686184317292081 -0.0630067102720941
0.0711271670520408 -0.0746383322854955
0.0740682339532131 -0.0860796863272447
0.0774397272079407 -0.0973016021399556
0.0812394600470219 -0.108275470796062
0.0854649648223936 -0.118973319084073
0.0901134940873028 -0.12936788275431
0.0951820222852561 -0.139432678452301
0.100667248289028 -0.149142074107302
0.10656559906303 -0.158471357449301
0.112873234718821 -0.167396802199979
0.119586055168304 -0.175895731332128
0.126699708426136 -0.183946576646328
0.13420960035749 -0.191528933821879
0.142110905322839 -0.198623612126504
0.150398576795015 -0.205212678179984
0.159067356720643 -0.211279493592426
0.16811178230224 -0.216808746901301
0.177526189102462 -0.221786480886398
0.187304709947427 -0.226200116851138
0.197441269929379 -0.230038477621279
0.207929578656845 -0.233291810717099
0.218763121506754 -0.235951812451675
0.229935151803763 -0.238011652801433
0.241438685552319 -0.239466000064469
0.253266499708268 -0.240311043799429
0.265411134226006 -0.240544514421553
0.277864897472831 -0.240165698073282
0.290619874197545 -0.23917544583971
0.303667935101709 -0.237576176879925
0.317000747134867 -0.235371875470324
0.330609783830706 -0.232568082244952
0.344486335235128 -0.229171880066449
0.358621517191145 -0.225191874998346
0.373006279910045 -0.220638172815656
0.387631415867 -0.215522351422736
0.402487567119295 -0.209857429471725
0.417565232169048 -0.203657831407191
0.432854772492479 -0.196939349109845
0.448346418844808 -0.189719100275549
0.46403027743119 -0.182015483643688
0.47989633601438 -0.173848131178146
0.495934470011695 -0.165237857301697
0.512134448618413 -0.156206605287707
0.528485940982374 -0.146777390919585
0.544978522444945 -0.136974243536887
0.561601680856407 -0.126822144596177
0.578344822968638 -0.116346963884112
0.595197280904322 -0.105575393529208
0.61214831869946 -0.094534879967153
0.629187138914278 -0.0832535540222168
0.646302889306651 -0.071760159274263
0.663484669561462 -0.0600839788869683
0.680721538069075 -0.0482547610782518
0.698002518745858 -0.0363026434185232
0.715316607889725 -0.0242580761462783
0.732652781063624 -0.012151744693772
};
\addplot [line width = \linewidthEightC, color = reference, opacity=\opacityRef, forget plot]
table {%
0.75 0
0.752448260784149 0.00186432152986526
0.767692565917969 0.013188511133194
0.786628901958466 0.0269793719053268
0.803270816802979 0.0391807928681374
0.820759832859039 0.0519931614398956
0.838096976280212 0.06429573148489
0.855640232563019 0.076635979115963
0.872642397880554 0.088462196290493
0.889690637588501 0.100438185036182
0.906408965587616 0.111664973199368
0.923453748226166 0.123299457132816
0.940416634082794 0.134168110787868
0.957193791866302 0.144905112683773
0.973605453968048 0.155109606683254
0.990091621875763 0.164784260094166
1.0062113404274 0.17444109171629
1.02251344919205 0.183357052505016
1.03866010904312 0.191898755729198
1.0542865395546 0.199658505618572
1.07011049985886 0.207179717719555
1.0854052901268 0.213995568454266
1.10074430704117 0.220082156360149
1.1155623793602 0.22585778683424
1.13019210100174 0.230741791427135
1.14477568864822 0.235133789479733
1.15898185968399 0.239342890679836
1.17288678884506 0.242311589419842
1.1865354180336 0.245260111987591
1.1998183131218 0.247248344123363
1.21302539110184 0.24836903065443
1.22565227746964 0.248893909156322
1.23820835351944 0.249203108251095
1.2503964304924 0.24867957085371
1.26223999261856 0.2475730702281
1.27385073900223 0.245720081031322
1.28497928380966 0.243422321975231
1.2959857583046 0.240427754819393
1.30660420656204 0.236791037023067
1.31668084859848 0.232774220407009
1.3266333937645 0.227345161139965
1.336086332798 0.222508780658245
1.3452610373497 0.216456018388271
1.35435086488724 0.209683321416378
1.36255246400833 0.202669136226177
1.37037414312363 0.195217184722424
1.37787586450577 0.187310330569744
1.38502138853073 0.178704284131527
1.39184457063675 0.169662527740002
1.39784246683121 0.160510547459126
1.40358799695969 0.151010178029537
1.40916281938553 0.140698246657848
1.41450768709183 0.130608193576336
1.4188112616539 0.119290269911289
1.4229376912117 0.108593620359898
1.42633265256882 0.0969605967402458
1.42963200807571 0.0853637233376503
1.43230372667313 0.0735167935490608
1.43454784154892 0.0614591911435127
1.43636745214462 0.0496458560228348
1.4376659989357 0.0370959341526031
1.43862551450729 0.0251325964927673
1.43928807973862 0.0123290866613388
1.43948525190353 0.000153683125972748
1.43890410661697 -0.0117331482470036
1.43850511312485 -0.0243759267032146
1.43714398145676 -0.0367401912808418
1.43532353639603 -0.048648415133357
1.43317526578903 -0.0607271362096071
1.4305300116539 -0.072985699865967
1.42741721868515 -0.0841385740786791
1.42396157979965 -0.0956730172038078
1.41967540979385 -0.107242468744516
1.41556948423386 -0.118095718324184
1.41071826219559 -0.128583580255508
1.40574258565903 -0.139033675193787
1.40016132593155 -0.149000950157642
1.39409273862839 -0.158841915428638
1.38771122694016 -0.16780199110508
1.38063985109329 -0.176938258111477
1.37325888872147 -0.184988811612129
1.3653878569603 -0.192874878644943
1.35713750123978 -0.19993007928133
1.34868508577347 -0.206780306994915
1.33969455957413 -0.212892331182957
1.33037453889847 -0.218076445162296
1.32070201635361 -0.223310045897961
1.31062823534012 -0.22752632945776
1.30032402276993 -0.231116659939289
1.28957349061966 -0.234287194907665
1.27862328290939 -0.236607737839222
1.26705688238144 -0.238612778484821
1.25510865449905 -0.23974833637476
1.24303501844406 -0.240207962691784
1.23046606779099 -0.240408785641193
1.21764522790909 -0.239358507096767
1.20489698648453 -0.238355614244938
1.19154876470566 -0.23610395938158
1.17775422334671 -0.233410693705082
1.16387122869492 -0.229993216693401
1.14959079027176 -0.226762376725674
1.13505798578262 -0.222060851752758
1.12039464712143 -0.217072077095509
1.10503906011581 -0.211476109921932
1.09020966291428 -0.204992868006229
1.0750977396965 -0.198072694242001
1.05954951047897 -0.190629005432129
1.0436218380928 -0.182936482131481
1.02755600214005 -0.174482651054859
1.01160663366318 -0.165494449436665
0.99504417181015 -0.156054943799973
0.978596746921539 -0.146657474339008
0.962052524089813 -0.136414237320423
0.945398390293121 -0.125682070851326
0.928694665431976 -0.114906199276447
0.911785781383514 -0.103742051869631
0.894676089286804 -0.0921173971146345
0.877399981021881 -0.0799487633630633
0.860067367553711 -0.0678684208542109
0.842768430709839 -0.0554965194314718
0.825365960597992 -0.0431031342595816
0.807794272899628 -0.0311176516115665
0.790390312671661 -0.0181304886937141
0.772673368453979 -0.00525432080030441
0.755475282669067 0.00717263668775558
0.737839877605438 0.0197974815964699
0.720390319824219 0.0325229838490486
0.702657163143158 0.0457612723112106
0.685384094715118 0.0584832951426506
0.668122708797455 0.0711886659264565
0.650667130947113 0.0839790776371956
0.633416354656219 0.0959983095526695
0.616419196128845 0.108062900602818
0.599114239215851 0.119916252791882
0.58198082447052 0.132008351385593
0.565402090549469 0.142881579697132
0.548509120941162 0.153776608407497
0.531765699386597 0.164538450539112
0.515072286128998 0.175025261938572
0.498573303222656 0.185034476220608
0.482424795627594 0.19464760273695
0.466087400913239 0.203601263463497
0.449943900108337 0.212367497384548
0.434045553207397 0.220456026494503
0.418405532836914 0.228040657937527
0.40290093421936 0.235172502696514
0.387726724147797 0.241687588393688
0.372685194015503 0.247666411101818
0.357413053512573 0.253048412501812
0.342724621295929 0.257976345717907
0.328426003456116 0.262200944125652
0.314091712236404 0.265951506793499
0.300215780735016 0.268957726657391
0.286358177661896 0.271325819194317
0.273099482059479 0.273187093436718
0.260122835636139 0.274211995303631
0.246941030025482 0.274872832000256
0.234827041625977 0.274733774363995
0.222512453794479 0.273993693292141
0.21079283952713 0.272395424544811
0.199358522891998 0.270233295857906
0.187918365001678 0.267664186656475
0.177441775798798 0.264554224908352
0.166946142911911 0.260196946561337
0.156875431537628 0.256046943366528
0.14730504155159 0.25081779807806
0.137886136770248 0.244958661496639
0.12932077050209 0.239094607532024
0.120884582400322 0.232094876468182
0.112686857581139 0.224958620965481
0.105255335569382 0.217016391456127
0.0981751531362534 0.208946906030178
0.0914290547370911 0.200278125703335
0.0850926041603088 0.191067419946194
0.0789925009012222 0.181266449391842
0.0734453350305557 0.171586789190769
0.0683787018060684 0.16060458868742
0.0638919621706009 0.150502882897854
0.0594776272773743 0.139311112463474
0.0560705065727234 0.127469338476658
0.0524932146072388 0.116437636315823
0.0495588034391403 0.104321844875813
0.0472214967012405 0.0928196236491203
0.045373797416687 0.0797565206885338
0.0435472726821899 0.0684461519122124
0.0425935834646225 0.0561008974909782
0.0420941263437271 0.0430908724665642
0.0418698191642761 0.0308560207486153
0.0422055572271347 0.018915168941021
0.0429875254631042 0.00631475448608398
0.0440777391195297 -0.00579138100147247
0.0457342267036438 -0.0184538140892982
0.048002153635025 -0.0307075791060925
0.0509804636240005 -0.0430194661021233
0.05373215675354 -0.0544291473925114
0.0571579933166504 -0.0660050939768553
0.0611091405153275 -0.0772973531857133
0.0655638426542282 -0.0883469916880131
0.0700876116752625 -0.0989357158541679
0.0755803287029266 -0.109352145344019
0.0807357132434845 -0.118998423218727
0.0870888084173203 -0.1289390809834
0.0933108031749725 -0.138114154338837
0.100297942757607 -0.146704278886318
0.107932612299919 -0.155525580048561
0.11530214548111 -0.163188718259335
0.123619586229324 -0.170545235276222
0.131956696510315 -0.177218697965145
0.140959724783897 -0.18365029245615
0.150139480829239 -0.189277194440365
0.159653693437576 -0.194403581321239
0.16952458024025 -0.198595978319645
0.179984778165817 -0.202692948281765
0.190655887126923 -0.205929629504681
0.201853722333908 -0.209089927375317
0.212958246469498 -0.21059126406908
0.224993020296097 -0.21191430836916
0.236909717321396 -0.212481670081615
0.249397993087769 -0.212527267634869
0.262375980615616 -0.212541200220585
0.275290340185165 -0.211037136614323
0.288451373577118 -0.209578536450863
0.301920711994171 -0.206600420176983
0.315682709217072 -0.203377686440945
0.32976707816124 -0.199478201568127
0.344371229410172 -0.195360250771046
0.358522713184357 -0.190038926899433
0.373623937368393 -0.184232823550701
0.388633459806442 -0.177969209849834
0.404065370559692 -0.171096883714199
0.419461488723755 -0.163684129714966
0.435102880001068 -0.155731521546841
0.450741767883301 -0.147564545273781
0.466744899749756 -0.138213522732258
0.482981622219086 -0.12885981425643
0.499328911304474 -0.11923798546195
0.515535295009613 -0.108815491199493
0.532034635543823 -0.0979229360818863
0.548713386058807 -0.0870249178260565
0.565361142158508 -0.0749873779714108
0.581887543201447 -0.0636401856318116
0.598730683326721 -0.0514998212456703
0.615701258182526 -0.0392691865563393
0.632486939430237 -0.026871420443058
0.649638295173645 -0.0138055086135864
0.666694581508636 -0.00109214335680008
0.684074223041534 0.0123008266091347
0.701330840587616 0.0250077545642853
0.718584477901459 0.0385493189096451
0.735585570335388 0.0520108714699745
};
\addplot [semithick, red, dashed, forget plot]
table {%
0.75 0
0.767617956255685 0.0126152320166048
0.785221863675294 0.025202759744927
0.802800914680889 0.0377306780229099
0.820344353102624 0.0501672114827695
0.837841479099983 0.0624807937495562
0.855281653865309 0.0746401456242545
0.872654304077727 0.0866143520292023
0.889948926065816 0.0983729374930471
0.90715508962574 0.109885939954111
0.924262441427357 0.121123982665618
0.941260707923679 0.132058343994744
0.958139697658408 0.142661024921306
0.974889302841815 0.152904814062882
0.991499500036706 0.162763350083898
1.00796034976389 0.172211181389763
1.02426199480132 0.18122382306779
1.04039465691463 0.189777811119373
1.0563486317231 0.197850754138649
1.07211428137931 0.205421382738366
1.08768202473299 0.212469597210257
1.10304232467243 0.218976514139914
1.11818567241108 0.224924512976024
1.1331025686362 0.230297283873765
1.147783501695 0.23507987847157
1.16221892339532 0.239258765576722
1.17639922357808 0.242821893954846
1.19031470539574 0.245758764427774
1.20395556419321 0.248060513126141
1.21731187395941 0.249720006825457
1.23037358633393 0.250731949622618
1.24313054782987 0.251092997654816
1.2555725408717 0.250801875167454
1.26768935298946 0.24985948134918
1.27947087567813 0.248268973729967
1.29090722994702 0.246035811751358
1.30198890990496 0.243167744731957
1.31270692999354 0.2396747329547
1.32305295737637 0.235568799170581
1.33301941026506 0.230863819194334
1.34259950666846 0.225575271775738
1.3517872558846 0.219719976188812
1.36057739518197 0.213315848229081
1.36896528374308 0.206381700652654
1.37694677240494 0.198937103974783
1.38451806954475 0.191002311131823
1.39167562078374 0.182598238316676
1.39841601450881 0.173746486857157
1.4047359185686 0.164469388195659
1.41063204764278 0.154790055213092
1.41610115672778 0.144732426840967
1.42114005415556 0.134321297531182
1.42574562723652 0.123582327489247
1.42991487441761 0.112542032954869
1.43364493919855 0.101227758043132
1.43693314251624 0.089667630857217
1.4397770116222 0.0778905070053025
1.44217430452097 0.0659259035746353
1.44412302978876 0.0538039262643269
1.44562146208725 0.0415551919190076
1.44666815398553 0.0292107482394724
1.44726194486432 0.016801992026761
1.44740196775303 0.00436058696240971
1.44708765498222 -0.00808161836067393
1.44631874355061 -0.0204926716768206
1.4450952811249 -0.0328405972734807
1.44341763362016 -0.0450934751015305
1.44128649535024 -0.0572195188465802
1.4387029027798 -0.0691871532341023
1.43566825293285 -0.0809650910678666
1.4321843274821 -0.0925224107662305
1.42825332340439 -0.103828635462315
1.42387789076703 -0.114853815053745
1.41906117760902 -0.125568612881875
1.41380688088891 -0.135944398910193
1.40811930097384 -0.145953351233135
1.40200339507468 -0.155568567310703
1.39546482242266 -0.164764185292558
1.38850997106277 -0.173515513988717
1.38114595342609 -0.181799167392522
1.37338055620066 -0.189593196333819
1.36522213060654 -0.196877206377916
1.35667941319178 -0.203632448466232
1.34776127545165 -0.209841868311401
1.33847641259836 -0.215490103434114
1.328832995749 -0.220563423487838
1.3188383241811 -0.225049619377379
1.30849852098817 -0.228937857346307
1.29781831326193 -0.232218522442798
1.28680092623794 -0.234883078640104
1.27544810236402 -0.23692396914816
1.26376023613114 -0.238334571253006
1.25173659926596 -0.239109208409727
1.23937562223503 -0.239243211851919
1.22680212890194 -0.238807395220756
1.21393543612189 -0.237739020519201
1.2007824030094 -0.236042071113405
1.1873502195155 -0.233721342391173
1.17364639818837 -0.230782623136066
1.15967876862552 -0.227232843837933
1.14545546822952 -0.223080191387264
1.13098492732646 -0.218334192018305
1.11627584919941 -0.213005765794969
1.10133718666704 -0.207107256561536
1.08617811704058 -0.200652441341106
1.070808017043 -0.19365652287622
1.05523643885516 -0.186136108539406
1.03947308802467 -0.178109178316594
1.02352780361101 -0.169595044057188
1.00741054066936 -0.160614301730894
0.991131354992172 -0.151188778049092
0.974700389916618 -0.141341472499412
0.958127864949596 -0.131096495600034
0.941424065943487 -0.120479003995558
0.924599336561409 -0.109515132878708
0.90766407079076 -0.0982319261219058
0.890628706290982 -0.086657264431339
0.873503718391248 -0.074819791786618
0.856299614583182 -0.0627488403957748
0.839026929381251 -0.0504743543738914
0.8216962194482 -0.0380268123406879
0.804318058904672 -0.0254371491254941
0.786903034760972 -0.0127366767652624
0.769461742425004 4.29950186324069e-05
0.752004781254135 0.0128700386771497
0.734542750130382 0.0257124888331812
0.717086243048227 0.0385383222295833
0.699645844712938 0.0513155374272836
0.682232126154625 0.0640122340511521
0.664855640369735 0.0765966913776288
0.647526918007365 0.0890374460548427
0.630256463122704 0.101303368743397
0.613054749024055 0.113363739464377
0.595932214243 0.125188321440869
0.578899258658929 0.136747433220813
0.561966239808458 0.148012018873052
0.545143469406046 0.158953716055882
0.528441210092295 0.169544921769502
0.511869672407924 0.179758855622043
0.495439011959919 0.189569620465462
0.479159326695086 0.198952260295039
0.463040654116175 0.207882815357607
0.447092968153461 0.216338374482455
0.431326175222446 0.224297124738732
0.41575010873165 0.231738398637223
0.400374520922108 0.238642719233979
0.38520907038282 0.244991843656404
0.37026330284855 0.250768805749989
0.355546621900637 0.255957958716022
0.341068244918601 0.260545018739871
0.326837138059129 0.264517110634022
0.312861922218046 0.267862816346776
0.299150740017787 0.270572226688282
0.285711072187049 0.272636995643726
0.272549490830637 0.274050395018263
0.259671337879835 0.274807364778616
0.247080320574249 0.274904551350881
0.234778023333987 0.274340322593555
0.22282111677237 0.273147174350289
0.211212000441657 0.271312180989548
0.199956090718924 0.268842225658862
0.18905951088657 0.265745762806622
0.178528814558833 0.262032850875418
0.168370684504055 0.257715152729482
0.158591652631679 0.252805894649178
0.149197881311021 0.247319783843054
0.140195031151771 0.241272894158711
0.131588220396365 0.234682536784361
0.123382062255884 0.227567134929123
0.115580754025762 0.219946118404367
0.108188187942631 0.211839847297465
0.101208057557953 0.203269566144989
0.0946439418448079 0.19425738357691
0.088499358802492 0.184826268669802
0.0827777883280161 0.17500005435619
0.0774826693237491 0.164803439481848
0.0726173783628224 0.154261983466959
0.0681851974057714 0.143402090111052
0.0641892769319838 0.132250979303424
0.0606325991839522 0.12083664700672
0.0575179445429576 0.109187814856514
0.0548478626398547 0.0973338711778623
0.0526246487503009 0.0853048053160545
0.0508503253165833 0.0731311370573394
0.0495266280133111 0.0608438426868979
0.0486549955532945 0.0484742789692179
0.0482365623412967 0.036054106082026
0.0482721530701452 0.0236152103073717
0.0487622783751904 0.0111896270851532
0.0497071306928156 -0.00119053514168032
0.0511065794918781 -0.0134931700273918
0.0529601650583767 -0.0256862472791051
0.0552670900164109 -0.0377378855745846
0.0580262077740067 -0.0496164247880426
0.061236007111159 -0.0612904981681536
0.0648945922107272 -0.0727291054819839
0.0689996576140302 -0.0839016885812368
0.0735484579177227 -0.0947782113457813
0.0785377725810335 -0.10532924646507
0.0839638670453019 -0.11552607193497
0.0898224525211534 -0.125340780318524
0.0961086482561191 -0.134746403519034
0.10281695173701 -0.143717054765618
0.109941223832714 -0.152228087440402
0.117474696885182 -0.160256267121569
0.125410013591915 -0.167779948892829
0.133739302545796 -0.174779247148552
0.14245429208538 -0.181236180932355
0.151546457798383 -0.187134775860053
0.161007191589854 -0.192461105490089
0.170827973498639 -0.197203261519913
0.181000523658031 -0.201351252889932
0.191516912754772 -0.204896846530735
0.202369615510414 -0.207833373632873
0.213551501760898 -0.210155531524887
0.22505577088336 -0.211859210579139
0.236875844499746 -0.212941368432345
0.249005237340447 -0.213399962707235
0.261437426261433 -0.213233941776159
0.274165733616003 -0.212443283869614
0.28718323530924 -0.211029069576408
0.300482697779478 -0.208993571566475
0.314056543186631 -0.206340347198259
0.327896838839168 -0.203074323164878
0.341995305342881 -0.199201865241865
0.356343337731412 -0.194730829684218
0.370932034454633 -0.189670595476494
0.385752230119493 -0.184032078393973
0.400794528982751 -0.177827728806396
0.416049337197401 -0.171071515546711
0.431506892628147 -0.16377889817583
0.447157291657684 -0.15596678976561
0.462990512824106 -0.147653512015245
0.478996437395876 -0.138858744187699
0.495164867142796 -0.129603467046443
0.511485539633279 -0.119909902709476
0.527948141407569 -0.109801451123518
0.544542319363787 -0.0993026236948103
0.561257690663388 -0.0884389744880547
0.578083851424296 -0.0772370293146803
0.595010384429933 -0.0657242129685941
0.612026866043939 -0.0539287748255556
0.62912287248563 -0.0418797129959582
0.646287985590849 -0.029606697205851
0.663511798156965 -0.0171399905741775
0.680783918949166 -0.00451037045289776
0.698093977427272 0.00825095150093907
0.715431628237671 0.0211124098473716
0.732786555502968 0.0340421663000904
};
\addplot [semithick, green, dash pattern=on 1pt off 3pt on 3pt off 3pt, forget plot]
table {%
0.75 0
0.767347219519284 0.012137253483567
0.784683394455509 0.0242435861667548
0.801997486582194 0.036288155514001
0.819278471531621 0.0482402761288358
0.836515345700615 0.0600694978228124
0.853697133135913 0.0717456830897325
0.870812892392119 0.0832390837898383
0.887851723355235 0.0945204168511733
0.904802774024683 0.105560938798497
0.921655247246804 0.116332518924018
0.938398407392915 0.126807710918802
0.955021586975309 0.136959822789028
0.971514193195148 0.146762984887382
0.987865714417124 0.156192215896701
1.0040657265673 0.165223486610677
1.02010389945277 0.173833781364724
1.03597000300535 0.182001156979107
1.05165391345605 0.189704799085743
1.06714561945411 0.196925075719347
1.08243522815337 0.203643588062188
1.09751297130037 0.209843218238558
1.1123692113737 0.215508174058654
1.12699444784162 0.220624030609957
1.14137932362429 0.225177768584601
1.1555146318663 0.22915780921071
1.16939132313901 0.232554045621132
1.18300051319534 0.235357870442502
1.19633349138035 0.237562199321834
1.20938172974852 0.23916149003295
1.22213689283924 0.240151756735333
1.23459084790774 0.240530578918402
1.24673567520343 0.24029710459028
1.25856367765532 0.239452047402427
1.27006738911946 0.237997677671498
1.28123958024499 0.235937807669022
1.29207326111455 0.233277772045945
1.30256168017249 0.230024404727089
1.31269831956044 0.226186013888674
1.32247688771958 0.221772356568193
1.33189131078413 0.216794613983565
1.34093572465289 0.211265367835258
1.34960446953272 0.205198576949498
1.35789208820684 0.198609552872807
1.36579332846363 0.191514932670796
1.37330314928634 0.18393264729003
1.38041672978698 0.175881884318464
1.38712947958844 0.167383044631953
1.39343704941026 0.158457693037802
1.39933534088867 0.149128503482261
1.40482051503024 0.139419199628507
1.40988899904863 0.129354491659788
1.41453749160152 0.118960010081206
1.41876296661194 0.108262237150103
1.42256267593454 0.0972884364110782
1.42593415113889 0.086066580678514
1.42887520465437 0.0746252787098322
1.43138393047621 0.0629937007473294
1.43345870458359 0.0512015030695605
1.43509818517756 0.0392787516771522
1.43630131281055 0.0272558452352367
1.43706731045357 0.0151634373994425
1.43739568352819 0.00303235866024866
1.43728621991849 -0.00910646215137196
1.43673898997073 -0.0212220765475008
1.43575434648365 -0.0332835951024502
1.43433292468914 -0.0452602660529673
1.43247564221903 -0.0571215536269705
1.43018369904855 -0.0688372159533007
1.42745857739782 -0.0803773824121825
1.4243020415584 -0.0917126302942692
1.42071613759078 -0.102814060643163
1.41670319280875 -0.113653373158817
1.41226581492764 -0.124202940032363
1.40740689070718 -0.134435878560225
1.40212958387022 -0.144326122339201
1.3964373320371 -0.153848490767307
1.39033384239815 -0.162978756463708
1.38382308587764 -0.171693710079292
1.37690928964889 -0.179971221816331
1.36959692806383 -0.187790298849553
1.36189071236441 -0.195131137800147
1.35379557991389 -0.201975171526458
1.34531668404164 -0.20830510981277
1.33645938581161 -0.214104974062127
1.32722924897007 -0.219360126749639
1.31763203892423 -0.224057297001799
1.30767372588241 -0.22818460402503
1.2973604914176 -0.231731580045289
1.28669873695777 -0.234689193904312
1.27569509230234 -0.237049875620184
1.26435642233521 -0.238807541316647
1.25268983059913 -0.239957617226964
1.24070265912099 -0.240497061156197
1.22840248459827 -0.24042437986202
1.21579711159144 -0.239739641182038
1.20289456363565 -0.238444480231447
1.1897030732081 -0.236542099469318
1.17623107134149 -0.234037262792463
1.16248717745014 -0.230936284032406
1.14848018970572 -0.227247010319085
1.13421907610845 -0.222978800771493
1.11971296626323 -0.218142500920566
1.10497114378721 -0.212750413195198
1.09000303923492 -0.206816263729265
1.07481822341698 -0.200355165686855
1.05942640099553 -0.193383579258294
1.04383740425637 -0.185919268450478
1.02806118697728 -0.177981254778958
1.01210781833129 -0.169589767962932
0.995987476780275 -0.160766193724954
0.979710443928391 -0.151533018802209
0.963287098315684 -0.141913773283855
0.946727909140567 -0.131932970397911
0.930043429905878 -0.121616043880467
0.91324429198767 -0.110989283069269
0.896341198128817 -0.100079765872392
0.879344915861624 -0.0889152897708354
0.862266270864871 -0.0775243010211479
0.845116140261585 -0.0659358222307342
0.827905445864243 -0.0541793784842489
0.810645147374399 -0.042284922204474
0.793346235543765 -0.0302827569353333
0.776019725303823 -0.0182034602382608
0.758676648870971 -0.00607780589600706
0.74132804883419 0.00606331437980999
0.723984971232126 0.0181889695395348
0.706658458626486 0.0302682678858693
0.689359543178601 0.0422704356646973
0.672099239736024 0.0541648953575106
0.654888538936098 0.0659213434775336
0.637738400333409 0.077509827673207
0.620659745558167 0.0889008229448788
0.603663451512546 0.10006530678342
0.586760343612044 0.110974833042994
0.569961189078862 0.121601604364455
0.55327669029407 0.131918542970798
0.536717478214957 0.141899359661781
0.520294105863188 0.151518620841353
0.504017041888231 0.160751813418715
0.487896664208613 0.169575407431921
0.471943253730781 0.177966916251536
0.456166988141264 0.18590495423109
0.440577935762111 0.193369291680399
0.425186049451712 0.200340907046881
0.410001160522753 0.206802036197834
0.395032972635676 0.212736218702091
0.38029105560979 0.218128341010777
0.36578483907538 0.222964676431608
0.351523605870586 0.227232921776542
0.337516485069478 0.23092223053541
0.323772444518238 0.234023242385741
0.310300282763089 0.236528108790522
0.297108620287721 0.238430514364134
0.284205890052545 0.239725693611976
0.27160032745433 0.240410442590784
0.259299960006205 0.240483125024951
0.247312597261469 0.239943672488383
0.2356458217318 0.238793578459958
0.224306981713199 0.237035886402939
0.21330318694315 0.234675172482025
0.202641307788403 0.231717524032126
0.192327978173203 0.228170515287301
0.1823696017639 0.22404318200088
0.172772360193836 0.219345996319791
0.163542221572307 0.214090842618848
0.154684947376684 0.208290994107505
0.146206096155645 0.201961089155653
0.138111023180157 0.195117105712544
0.130404876038488 0.187776332069152
0.12309258692145 0.179957332525672
0.116178862798813 0.171679907117356
0.109668174793245 0.162965045212245
0.10356474787866 0.153834873345794
0.0978725516915446 0.144312598005936
0.0925952928735381 0.134422444219678
0.0877364090495248 0.124189590767097
0.0832990643281257 0.113640102728524
0.0792861460924091 0.102800861916777
0.0757002628080311 0.0916994956001386
0.0725437425870994 0.0803643038046822
0.069818632284583 0.0688241854024739
0.0675266969522104 0.0571085631419612
0.0656694195214907 0.0452473077513642
0.0642480006271671 0.0332706612373471
0.0632633585132331 0.0212091595029022
0.0627161289858991 0.00909355441514592
0.0626066653929575 -0.00304526453801165
0.0629350386185861 -0.0151763488558262
0.0637010370885092 -0.0272687697454203
0.0649041667842718 -0.0392916968007361
0.0665436512687222 -0.0512144764858195
0.0686184317292081 -0.0630067102720941
0.0711271670520408 -0.0746383322854955
0.0740682339532131 -0.0860796863272447
0.0774397272079407 -0.0973016021399556
0.0812394600470219 -0.108275470796062
0.0854649648223936 -0.118973319084073
0.0901134940873028 -0.12936788275431
0.0951820222852561 -0.139432678452301
0.100667248289028 -0.149142074107302
0.10656559906303 -0.158471357449301
0.112873234718821 -0.167396802199979
0.119586055168304 -0.175895731332128
0.126699708426136 -0.183946576646328
0.13420960035749 -0.191528933821879
0.142110905322839 -0.198623612126504
0.150398576795015 -0.205212678179984
0.159067356720643 -0.211279493592426
0.16811178230224 -0.216808746901301
0.177526189102462 -0.221786480886398
0.187304709947427 -0.226200116851138
0.197441269929379 -0.230038477621279
0.207929578656845 -0.233291810717099
0.218763121506754 -0.235951812451675
0.229935151803763 -0.238011652801433
0.241438685552319 -0.239466000064469
0.253266499708268 -0.240311043799429
0.265411134226006 -0.240544514421553
0.277864897472831 -0.240165698073282
0.290619874197545 -0.23917544583971
0.303667935101709 -0.237576176879925
0.317000747134867 -0.235371875470324
0.330609783830706 -0.232568082244952
0.344486335235128 -0.229171880066449
0.358621517191145 -0.225191874998346
0.373006279910045 -0.220638172815656
0.387631415867 -0.215522351422736
0.402487567119295 -0.209857429471725
0.417565232169048 -0.203657831407191
0.432854772492479 -0.196939349109845
0.448346418844808 -0.189719100275549
0.46403027743119 -0.182015483643688
0.47989633601438 -0.173848131178146
0.495934470011695 -0.165237857301697
0.512134448618413 -0.156206605287707
0.528485940982374 -0.146777390919585
0.544978522444945 -0.136974243536887
0.561601680856407 -0.126822144596177
0.578344822968638 -0.116346963884112
0.595197280904322 -0.105575393529208
0.61214831869946 -0.094534879967153
0.629187138914278 -0.0832535540222168
0.646302889306651 -0.071760159274263
0.663484669561462 -0.0600839788869683
0.680721538069075 -0.0482547610782518
0.698002518745858 -0.0363026434185232
0.715316607889725 -0.0242580761462783
0.732652781063624 -0.012151744693772
};
\addplot [line width = \linewidthEightC, color = reference, opacity=\opacityRef, forget plot]
table {%
0.75 0
0.753348767757416 0.00279992818832397
0.768736958503723 0.0141798034310341
0.787420153617859 0.0279095470905304
0.803844094276428 0.0402091890573502
0.821752548217773 0.05330790579319
0.838759005069733 0.0654503405094147
0.855746686458588 0.0778264254331589
0.873024880886078 0.0899381190538406
0.88998007774353 0.102347761392593
0.906977891921997 0.113763347268105
0.923702120780945 0.125359714031219
0.940325260162354 0.136509045958519
0.957154393196106 0.147243902087212
0.973600029945374 0.157484263181686
0.990172982215881 0.167461469769478
1.00595128536224 0.176902160048485
1.02218055725098 0.185715571045876
1.03793323040009 0.194395020604134
1.05372726917267 0.202489331364632
1.06934237480164 0.209903284907341
1.08489727973938 0.216940954327583
1.09997320175171 0.223202630877495
1.11496126651764 0.229168340563774
1.12986373901367 0.234200820326805
1.14397490024567 0.238667741417885
1.15840804576874 0.242956981062889
1.17249250411987 0.246222093701363
1.18602693080902 0.248867049813271
1.1994481086731 0.25094847381115
1.21232640743256 0.251697167754173
1.22497570514679 0.253097131848335
1.23729681968689 0.253200188279152
1.2492390871048 0.252903953194618
1.26116216182709 0.251687183976173
1.27274513244629 0.249823108315468
1.28396272659302 0.247698560357094
1.29495596885681 0.244701310992241
1.30540251731873 0.241232201457024
1.31592178344727 0.236928567290306
1.32565236091614 0.232424899935722
1.33529102802277 0.226776346564293
1.34417712688446 0.221134915947914
1.35312724113464 0.214493796229362
1.3613064289093 0.207875654101372
1.36972749233246 0.200139299035072
1.37696051597595 0.192252770066261
1.38397192955017 0.183484062552452
1.39087581634521 0.17466975748539
1.39705276489258 0.165231436491013
1.4028377532959 0.155498072504997
1.40802586078644 0.145370006561279
1.41321265697479 0.13456217944622
1.41791462898254 0.124558866024017
1.42200088500977 0.113378271460533
1.42534363269806 0.101887673139572
1.42878770828247 0.0903080999851227
1.43171870708466 0.0783904939889908
1.4339816570282 0.0664803236722946
1.43580567836761 0.0545702427625656
1.43753135204315 0.042056068778038
1.43853259086609 0.0297842398285866
1.43917679786682 0.0176300704479218
1.43935823440552 0.00474463403224945
1.4391473531723 -0.0075075626373291
1.43840336799622 -0.0200107470154762
1.4371246099472 -0.0318865217268467
1.4355753660202 -0.0439049787819386
1.43351459503174 -0.0559778492897749
1.43105041980743 -0.067850508261472
1.42815661430359 -0.0792810278944671
1.42466878890991 -0.0909876208752394
1.4207820892334 -0.101938113570213
1.41670775413513 -0.112884525209665
1.41170597076416 -0.124175235629082
1.40673053264618 -0.134136073291302
1.40109527111053 -0.144213177263737
1.39512240886688 -0.153874397277832
1.38863909244537 -0.163384921848774
1.38156259059906 -0.172102980315685
1.37429463863373 -0.180270455777645
1.36659979820251 -0.187937080860138
1.35846877098083 -0.195036470890045
1.35009968280792 -0.20183327794075
1.34091877937317 -0.207906812429428
1.33194708824158 -0.213283553719521
1.32208585739136 -0.218380063772202
1.31216156482697 -0.222533285617828
1.30168974399567 -0.226245388388634
1.29094398021698 -0.229519203305244
1.2798638343811 -0.2320946007967
1.2683914899826 -0.234004214406013
1.25671219825745 -0.234733030200005
1.24443316459656 -0.235641732811928
1.23201668262482 -0.235542878508568
1.2190465927124 -0.234822139143944
1.20587742328644 -0.233733788132668
1.19236946105957 -0.232078105211258
1.1784485578537 -0.22952276468277
1.16437435150146 -0.226255849003792
1.14997863769531 -0.222187399864197
1.13549530506134 -0.218056812882423
1.12079799175262 -0.213113769888878
1.10586822032928 -0.207348868250847
1.09097254276276 -0.20150537788868
1.07565987110138 -0.194524988532066
1.06001174449921 -0.187290504574776
1.0444016456604 -0.17980045825243
1.02843391895294 -0.171568974852562
1.01222491264343 -0.162577956914902
0.99602210521698 -0.153334505856037
0.979780435562134 -0.143487095832825
0.9630126953125 -0.133045878261328
0.946327328681946 -0.122711900621653
0.929186820983887 -0.11173503100872
0.912253618240356 -0.10060453787446
0.89504611492157 -0.089035926386714
0.87800258398056 -0.077463049441576
0.860688626766205 -0.0655054990202188
0.843393921852112 -0.0534401200711727
0.825896024703979 -0.0411189720034599
0.808219373226166 -0.0284057334065437
0.790866494178772 -0.0155665948987007
0.773031234741211 -0.00278812646865845
0.755572497844696 0.0099039152264595
0.73806095123291 0.0224206075072289
0.720412909984589 0.0353933274745941
0.702817559242249 0.047999382019043
0.685607254505157 0.0601795315742493
0.668289840221405 0.0730832815170288
0.650801777839661 0.0852250456809998
0.633421182632446 0.0979907214641571
0.616094768047333 0.109761133790016
0.599206447601318 0.122019872069359
0.581835687160492 0.133296772837639
0.564846336841583 0.144805505871773
0.548121988773346 0.155679419636726
0.531381130218506 0.166261866688728
0.514777839183807 0.176681384444237
0.498566567897797 0.186400219798088
0.482201457023621 0.195794656872749
0.46599555015564 0.204834058880806
0.449990630149841 0.213317885994911
0.433945834636688 0.221477463841438
0.417994916439056 0.229223653674126
0.402373850345612 0.236197516322136
0.38694965839386 0.242818400263786
0.371860384941101 0.248692199587822
0.356659322977066 0.254120513796806
0.342126816511154 0.258952781558037
0.327700257301331 0.263177260756493
0.31339368224144 0.266766473650932
0.299720197916031 0.269639983773232
0.285983741283417 0.271851912140846
0.272616028785706 0.273819997906685
0.259516328573227 0.275001749396324
0.246645361185074 0.275487020611763
0.234323918819427 0.275129303336143
0.222025871276855 0.2742989808321
0.210066705942154 0.272525951266289
0.198801159858704 0.270686909556389
0.187689334154129 0.267913445830345
0.17706510424614 0.264497593045235
0.166989475488663 0.260655537247658
0.15701499581337 0.255992755293846
0.147439017891884 0.250809982419014
0.138172507286072 0.24508048593998
0.129364281892776 0.238679692149162
0.121021598577499 0.231902047991753
0.113051027059555 0.224487647414207
0.105482950806618 0.216301426291466
0.0981867611408234 0.208202764391899
0.0913344323635101 0.199223175644875
0.0852174162864685 0.1904127150774
0.0792795568704605 0.1800227612257
0.0737996399402618 0.170738354325294
0.0689031630754471 0.160112157464027
0.0642506927251816 0.149435386061668
0.0602010935544968 0.138241186738014
0.0561963319778442 0.127141699194908
0.0528441220521927 0.115402102470398
0.0501388013362885 0.103316903114319
0.0477314293384552 0.0916914939880371
0.0456123501062393 0.0795239061117172
0.0442077964544296 0.0671844035387039
0.0431932508945465 0.0547529011964798
0.0424330234527588 0.0423663854598999
0.0426092594861984 0.0296947956085205
0.0427621901035309 0.0175190567970276
0.0436653047800064 0.00522004812955856
0.0449957549571991 -0.00719299167394638
0.0466644465923309 -0.0194320790469646
0.0487649738788605 -0.0314002111554146
0.0515512377023697 -0.0436953511089087
0.0544883161783218 -0.0551623553037643
0.0581651777029037 -0.0665701539255679
0.0621625483036041 -0.0778933325782418
0.0665482431650162 -0.0888071237131953
0.0713849514722824 -0.099587133154273
0.0764375478029251 -0.109695490449667
0.082368478178978 -0.120039314031601
0.0882786810398102 -0.12961358949542
0.0946259051561356 -0.13848876953125
0.101918786764145 -0.147478424012661
0.109425663948059 -0.155859172344208
0.11695197224617 -0.163481183350086
0.125275850296021 -0.171065308153629
0.133626788854599 -0.177474580705166
0.142725050449371 -0.183801390230656
0.15180715918541 -0.189303874969482
0.161539435386658 -0.194652542471886
0.171810120344162 -0.198869422078133
0.182115107774734 -0.20299831032753
0.192732840776443 -0.206015944480896
0.203625649213791 -0.208562940359116
0.215116947889328 -0.210646942257881
0.226781994104385 -0.211579650640488
0.239035069942474 -0.212405726313591
0.251265317201614 -0.212492898106575
0.264183074235916 -0.212375223636627
0.27729195356369 -0.210787296295166
0.290606021881104 -0.208585664629936
0.304558247327805 -0.206328228116035
0.318114221096039 -0.202941760420799
0.332295119762421 -0.199276760220528
0.34655225276947 -0.194906778633595
0.360736459493637 -0.189321361482143
0.375697195529938 -0.184157058596611
0.390417873859406 -0.177737712860107
0.405778586864471 -0.170555479824543
0.421539485454559 -0.163268975913525
0.43708997964859 -0.155086427927017
0.452734112739563 -0.146307304501534
0.46878969669342 -0.137559100985527
0.484854757785797 -0.127840146422386
0.501293241977692 -0.117792539298534
0.517617523670197 -0.107802845537663
0.533947885036469 -0.0966541673988104
0.550172865390778 -0.0856099054217339
0.566897869110107 -0.0733188018202782
0.583615243434906 -0.0620446968823671
0.600487351417542 -0.0499633718281984
0.617169260978699 -0.0375501438975334
0.634001433849335 -0.0247449651360512
0.650830149650574 -0.0122876837849617
0.667950987815857 0.00135047733783722
0.685162484645844 0.0141021534800529
0.702174782752991 0.027332715690136
0.719263017177582 0.0408385097980499
0.736500084400177 0.0535925477743149
};
\addplot [semithick, red, dashed, forget plot]
table {%
0.75 0
0.767617956255685 0.0126152320166048
0.785221863675294 0.025202759744927
0.802800914680889 0.0377306780229099
0.820344353102624 0.0501672114827695
0.837841479099983 0.0624807937495562
0.855281653865309 0.0746401456242545
0.872654304077727 0.0866143520292023
0.889948926065816 0.0983729374930471
0.90715508962574 0.109885939954111
0.924262441427357 0.121123982665618
0.941260707923679 0.132058343994744
0.958139697658408 0.142661024921306
0.974889302841815 0.152904814062882
0.991499500036706 0.162763350083898
1.00796034976389 0.172211181389763
1.02426199480132 0.18122382306779
1.04039465691463 0.189777811119373
1.0563486317231 0.197850754138649
1.07211428137931 0.205421382738366
1.08768202473299 0.212469597210257
1.10304232467243 0.218976514139914
1.11818567241108 0.224924512976024
1.1331025686362 0.230297283873765
1.147783501695 0.23507987847157
1.16221892339532 0.239258765576722
1.17639922357808 0.242821893954846
1.19031470539574 0.245758764427774
1.20395556419321 0.248060513126141
1.21731187395941 0.249720006825457
1.23037358633393 0.250731949622618
1.24313054782987 0.251092997654816
1.2555725408717 0.250801875167454
1.26768935298946 0.24985948134918
1.27947087567813 0.248268973729967
1.29090722994702 0.246035811751358
1.30198890990496 0.243167744731957
1.31270692999354 0.2396747329547
1.32305295737637 0.235568799170581
1.33301941026506 0.230863819194334
1.34259950666846 0.225575271775738
1.3517872558846 0.219719976188812
1.36057739518197 0.213315848229081
1.36896528374308 0.206381700652654
1.37694677240494 0.198937103974783
1.38451806954475 0.191002311131823
1.39167562078374 0.182598238316676
1.39841601450881 0.173746486857157
1.4047359185686 0.164469388195659
1.41063204764278 0.154790055213092
1.41610115672778 0.144732426840967
1.42114005415556 0.134321297531182
1.42574562723652 0.123582327489247
1.42991487441761 0.112542032954869
1.43364493919855 0.101227758043132
1.43693314251624 0.089667630857217
1.4397770116222 0.0778905070053025
1.44217430452097 0.0659259035746353
1.44412302978876 0.0538039262643269
1.44562146208725 0.0415551919190076
1.44666815398553 0.0292107482394724
1.44726194486432 0.016801992026761
1.44740196775303 0.00436058696240971
1.44708765498222 -0.00808161836067393
1.44631874355061 -0.0204926716768206
1.4450952811249 -0.0328405972734807
1.44341763362016 -0.0450934751015305
1.44128649535024 -0.0572195188465802
1.4387029027798 -0.0691871532341023
1.43566825293285 -0.0809650910678666
1.4321843274821 -0.0925224107662305
1.42825332340439 -0.103828635462315
1.42387789076703 -0.114853815053745
1.41906117760902 -0.125568612881875
1.41380688088891 -0.135944398910193
1.40811930097384 -0.145953351233135
1.40200339507468 -0.155568567310703
1.39546482242266 -0.164764185292558
1.38850997106277 -0.173515513988717
1.38114595342609 -0.181799167392522
1.37338055620066 -0.189593196333819
1.36522213060654 -0.196877206377916
1.35667941319178 -0.203632448466232
1.34776127545165 -0.209841868311401
1.33847641259836 -0.215490103434114
1.328832995749 -0.220563423487838
1.3188383241811 -0.225049619377379
1.30849852098817 -0.228937857346307
1.29781831326193 -0.232218522442798
1.28680092623794 -0.234883078640104
1.27544810236402 -0.23692396914816
1.26376023613114 -0.238334571253006
1.25173659926596 -0.239109208409727
1.23937562223503 -0.239243211851919
1.22680212890194 -0.238807395220756
1.21393543612189 -0.237739020519201
1.2007824030094 -0.236042071113405
1.1873502195155 -0.233721342391173
1.17364639818837 -0.230782623136066
1.15967876862552 -0.227232843837933
1.14545546822952 -0.223080191387264
1.13098492732646 -0.218334192018305
1.11627584919941 -0.213005765794969
1.10133718666704 -0.207107256561536
1.08617811704058 -0.200652441341106
1.070808017043 -0.19365652287622
1.05523643885516 -0.186136108539406
1.03947308802467 -0.178109178316594
1.02352780361101 -0.169595044057188
1.00741054066936 -0.160614301730894
0.991131354992172 -0.151188778049092
0.974700389916618 -0.141341472499412
0.958127864949596 -0.131096495600034
0.941424065943487 -0.120479003995558
0.924599336561409 -0.109515132878708
0.90766407079076 -0.0982319261219058
0.890628706290982 -0.086657264431339
0.873503718391248 -0.074819791786618
0.856299614583182 -0.0627488403957748
0.839026929381251 -0.0504743543738914
0.8216962194482 -0.0380268123406879
0.804318058904672 -0.0254371491254941
0.786903034760972 -0.0127366767652624
0.769461742425004 4.29950186324069e-05
0.752004781254135 0.0128700386771497
0.734542750130382 0.0257124888331812
0.717086243048227 0.0385383222295833
0.699645844712938 0.0513155374272836
0.682232126154625 0.0640122340511521
0.664855640369735 0.0765966913776288
0.647526918007365 0.0890374460548427
0.630256463122704 0.101303368743397
0.613054749024055 0.113363739464377
0.595932214243 0.125188321440869
0.578899258658929 0.136747433220813
0.561966239808458 0.148012018873052
0.545143469406046 0.158953716055882
0.528441210092295 0.169544921769502
0.511869672407924 0.179758855622043
0.495439011959919 0.189569620465462
0.479159326695086 0.198952260295039
0.463040654116175 0.207882815357607
0.447092968153461 0.216338374482455
0.431326175222446 0.224297124738732
0.41575010873165 0.231738398637223
0.400374520922108 0.238642719233979
0.38520907038282 0.244991843656404
0.37026330284855 0.250768805749989
0.355546621900637 0.255957958716022
0.341068244918601 0.260545018739871
0.326837138059129 0.264517110634022
0.312861922218046 0.267862816346776
0.299150740017787 0.270572226688282
0.285711072187049 0.272636995643726
0.272549490830637 0.274050395018263
0.259671337879835 0.274807364778616
0.247080320574249 0.274904551350881
0.234778023333987 0.274340322593555
0.22282111677237 0.273147174350289
0.211212000441657 0.271312180989548
0.199956090718924 0.268842225658862
0.18905951088657 0.265745762806622
0.178528814558833 0.262032850875418
0.168370684504055 0.257715152729482
0.158591652631679 0.252805894649178
0.149197881311021 0.247319783843054
0.140195031151771 0.241272894158711
0.131588220396365 0.234682536784361
0.123382062255884 0.227567134929123
0.115580754025762 0.219946118404367
0.108188187942631 0.211839847297465
0.101208057557953 0.203269566144989
0.0946439418448079 0.19425738357691
0.088499358802492 0.184826268669802
0.0827777883280161 0.17500005435619
0.0774826693237491 0.164803439481848
0.0726173783628224 0.154261983466959
0.0681851974057714 0.143402090111052
0.0641892769319838 0.132250979303424
0.0606325991839522 0.12083664700672
0.0575179445429576 0.109187814856514
0.0548478626398547 0.0973338711778623
0.0526246487503009 0.0853048053160545
0.0508503253165833 0.0731311370573394
0.0495266280133111 0.0608438426868979
0.0486549955532945 0.0484742789692179
0.0482365623412967 0.036054106082026
0.0482721530701452 0.0236152103073717
0.0487622783751904 0.0111896270851532
0.0497071306928156 -0.00119053514168032
0.0511065794918781 -0.0134931700273918
0.0529601650583767 -0.0256862472791051
0.0552670900164109 -0.0377378855745846
0.0580262077740067 -0.0496164247880426
0.061236007111159 -0.0612904981681536
0.0648945922107272 -0.0727291054819839
0.0689996576140302 -0.0839016885812368
0.0735484579177227 -0.0947782113457813
0.0785377725810335 -0.10532924646507
0.0839638670453019 -0.11552607193497
0.0898224525211534 -0.125340780318524
0.0961086482561191 -0.134746403519034
0.10281695173701 -0.143717054765618
0.109941223832714 -0.152228087440402
0.117474696885182 -0.160256267121569
0.125410013591915 -0.167779948892829
0.133739302545796 -0.174779247148552
0.14245429208538 -0.181236180932355
0.151546457798383 -0.187134775860053
0.161007191589854 -0.192461105490089
0.170827973498639 -0.197203261519913
0.181000523658031 -0.201351252889932
0.191516912754772 -0.204896846530735
0.202369615510414 -0.207833373632873
0.213551501760898 -0.210155531524887
0.22505577088336 -0.211859210579139
0.236875844499746 -0.212941368432345
0.249005237340447 -0.213399962707235
0.261437426261433 -0.213233941776159
0.274165733616003 -0.212443283869614
0.28718323530924 -0.211029069576408
0.300482697779478 -0.208993571566475
0.314056543186631 -0.206340347198259
0.327896838839168 -0.203074323164878
0.341995305342881 -0.199201865241865
0.356343337731412 -0.194730829684218
0.370932034454633 -0.189670595476494
0.385752230119493 -0.184032078393973
0.400794528982751 -0.177827728806396
0.416049337197401 -0.171071515546711
0.431506892628147 -0.16377889817583
0.447157291657684 -0.15596678976561
0.462990512824106 -0.147653512015245
0.478996437395876 -0.138858744187699
0.495164867142796 -0.129603467046443
0.511485539633279 -0.119909902709476
0.527948141407569 -0.109801451123518
0.544542319363787 -0.0993026236948103
0.561257690663388 -0.0884389744880547
0.578083851424296 -0.0772370293146803
0.595010384429933 -0.0657242129685941
0.612026866043939 -0.0539287748255556
0.62912287248563 -0.0418797129959582
0.646287985590849 -0.029606697205851
0.663511798156965 -0.0171399905741775
0.680783918949166 -0.00451037045289776
0.698093977427272 0.00825095150093907
0.715431628237671 0.0211124098473716
0.732786555502968 0.0340421663000904
};
\addplot [semithick, green, dash pattern=on 1pt off 3pt on 3pt off 3pt, forget plot]
table {%
0.75 0
0.767347219519284 0.012137253483567
0.784683394455509 0.0242435861667548
0.801997486582194 0.036288155514001
0.819278471531621 0.0482402761288358
0.836515345700615 0.0600694978228124
0.853697133135913 0.0717456830897325
0.870812892392119 0.0832390837898383
0.887851723355235 0.0945204168511733
0.904802774024683 0.105560938798497
0.921655247246804 0.116332518924018
0.938398407392915 0.126807710918802
0.955021586975309 0.136959822789028
0.971514193195148 0.146762984887382
0.987865714417124 0.156192215896701
1.0040657265673 0.165223486610677
1.02010389945277 0.173833781364724
1.03597000300535 0.182001156979107
1.05165391345605 0.189704799085743
1.06714561945411 0.196925075719347
1.08243522815337 0.203643588062188
1.09751297130037 0.209843218238558
1.1123692113737 0.215508174058654
1.12699444784162 0.220624030609957
1.14137932362429 0.225177768584601
1.1555146318663 0.22915780921071
1.16939132313901 0.232554045621132
1.18300051319534 0.235357870442502
1.19633349138035 0.237562199321834
1.20938172974852 0.23916149003295
1.22213689283924 0.240151756735333
1.23459084790774 0.240530578918402
1.24673567520343 0.24029710459028
1.25856367765532 0.239452047402427
1.27006738911946 0.237997677671498
1.28123958024499 0.235937807669022
1.29207326111455 0.233277772045945
1.30256168017249 0.230024404727089
1.31269831956044 0.226186013888674
1.32247688771958 0.221772356568193
1.33189131078413 0.216794613983565
1.34093572465289 0.211265367835258
1.34960446953272 0.205198576949498
1.35789208820684 0.198609552872807
1.36579332846363 0.191514932670796
1.37330314928634 0.18393264729003
1.38041672978698 0.175881884318464
1.38712947958844 0.167383044631953
1.39343704941026 0.158457693037802
1.39933534088867 0.149128503482261
1.40482051503024 0.139419199628507
1.40988899904863 0.129354491659788
1.41453749160152 0.118960010081206
1.41876296661194 0.108262237150103
1.42256267593454 0.0972884364110782
1.42593415113889 0.086066580678514
1.42887520465437 0.0746252787098322
1.43138393047621 0.0629937007473294
1.43345870458359 0.0512015030695605
1.43509818517756 0.0392787516771522
1.43630131281055 0.0272558452352367
1.43706731045357 0.0151634373994425
1.43739568352819 0.00303235866024866
1.43728621991849 -0.00910646215137196
1.43673898997073 -0.0212220765475008
1.43575434648365 -0.0332835951024502
1.43433292468914 -0.0452602660529673
1.43247564221903 -0.0571215536269705
1.43018369904855 -0.0688372159533007
1.42745857739782 -0.0803773824121825
1.4243020415584 -0.0917126302942692
1.42071613759078 -0.102814060643163
1.41670319280875 -0.113653373158817
1.41226581492764 -0.124202940032363
1.40740689070718 -0.134435878560225
1.40212958387022 -0.144326122339201
1.3964373320371 -0.153848490767307
1.39033384239815 -0.162978756463708
1.38382308587764 -0.171693710079292
1.37690928964889 -0.179971221816331
1.36959692806383 -0.187790298849553
1.36189071236441 -0.195131137800147
1.35379557991389 -0.201975171526458
1.34531668404164 -0.20830510981277
1.33645938581161 -0.214104974062127
1.32722924897007 -0.219360126749639
1.31763203892423 -0.224057297001799
1.30767372588241 -0.22818460402503
1.2973604914176 -0.231731580045289
1.28669873695777 -0.234689193904312
1.27569509230234 -0.237049875620184
1.26435642233521 -0.238807541316647
1.25268983059913 -0.239957617226964
1.24070265912099 -0.240497061156197
1.22840248459827 -0.24042437986202
1.21579711159144 -0.239739641182038
1.20289456363565 -0.238444480231447
1.1897030732081 -0.236542099469318
1.17623107134149 -0.234037262792463
1.16248717745014 -0.230936284032406
1.14848018970572 -0.227247010319085
1.13421907610845 -0.222978800771493
1.11971296626323 -0.218142500920566
1.10497114378721 -0.212750413195198
1.09000303923492 -0.206816263729265
1.07481822341698 -0.200355165686855
1.05942640099553 -0.193383579258294
1.04383740425637 -0.185919268450478
1.02806118697728 -0.177981254778958
1.01210781833129 -0.169589767962932
0.995987476780275 -0.160766193724954
0.979710443928391 -0.151533018802209
0.963287098315684 -0.141913773283855
0.946727909140567 -0.131932970397911
0.930043429905878 -0.121616043880467
0.91324429198767 -0.110989283069269
0.896341198128817 -0.100079765872392
0.879344915861624 -0.0889152897708354
0.862266270864871 -0.0775243010211479
0.845116140261585 -0.0659358222307342
0.827905445864243 -0.0541793784842489
0.810645147374399 -0.042284922204474
0.793346235543765 -0.0302827569353333
0.776019725303823 -0.0182034602382608
0.758676648870971 -0.00607780589600706
0.74132804883419 0.00606331437980999
0.723984971232126 0.0181889695395348
0.706658458626486 0.0302682678858693
0.689359543178601 0.0422704356646973
0.672099239736024 0.0541648953575106
0.654888538936098 0.0659213434775336
0.637738400333409 0.077509827673207
0.620659745558167 0.0889008229448788
0.603663451512546 0.10006530678342
0.586760343612044 0.110974833042994
0.569961189078862 0.121601604364455
0.55327669029407 0.131918542970798
0.536717478214957 0.141899359661781
0.520294105863188 0.151518620841353
0.504017041888231 0.160751813418715
0.487896664208613 0.169575407431921
0.471943253730781 0.177966916251536
0.456166988141264 0.18590495423109
0.440577935762111 0.193369291680399
0.425186049451712 0.200340907046881
0.410001160522753 0.206802036197834
0.395032972635676 0.212736218702091
0.38029105560979 0.218128341010777
0.36578483907538 0.222964676431608
0.351523605870586 0.227232921776542
0.337516485069478 0.23092223053541
0.323772444518238 0.234023242385741
0.310300282763089 0.236528108790522
0.297108620287721 0.238430514364134
0.284205890052545 0.239725693611976
0.27160032745433 0.240410442590784
0.259299960006205 0.240483125024951
0.247312597261469 0.239943672488383
0.2356458217318 0.238793578459958
0.224306981713199 0.237035886402939
0.21330318694315 0.234675172482025
0.202641307788403 0.231717524032126
0.192327978173203 0.228170515287301
0.1823696017639 0.22404318200088
0.172772360193836 0.219345996319791
0.163542221572307 0.214090842618848
0.154684947376684 0.208290994107505
0.146206096155645 0.201961089155653
0.138111023180157 0.195117105712544
0.130404876038488 0.187776332069152
0.12309258692145 0.179957332525672
0.116178862798813 0.171679907117356
0.109668174793245 0.162965045212245
0.10356474787866 0.153834873345794
0.0978725516915446 0.144312598005936
0.0925952928735381 0.134422444219678
0.0877364090495248 0.124189590767097
0.0832990643281257 0.113640102728524
0.0792861460924091 0.102800861916777
0.0757002628080311 0.0916994956001386
0.0725437425870994 0.0803643038046822
0.069818632284583 0.0688241854024739
0.0675266969522104 0.0571085631419612
0.0656694195214907 0.0452473077513642
0.0642480006271671 0.0332706612373471
0.0632633585132331 0.0212091595029022
0.0627161289858991 0.00909355441514592
0.0626066653929575 -0.00304526453801165
0.0629350386185861 -0.0151763488558262
0.0637010370885092 -0.0272687697454203
0.0649041667842718 -0.0392916968007361
0.0665436512687222 -0.0512144764858195
0.0686184317292081 -0.0630067102720941
0.0711271670520408 -0.0746383322854955
0.0740682339532131 -0.0860796863272447
0.0774397272079407 -0.0973016021399556
0.0812394600470219 -0.108275470796062
0.0854649648223936 -0.118973319084073
0.0901134940873028 -0.12936788275431
0.0951820222852561 -0.139432678452301
0.100667248289028 -0.149142074107302
0.10656559906303 -0.158471357449301
0.112873234718821 -0.167396802199979
0.119586055168304 -0.175895731332128
0.126699708426136 -0.183946576646328
0.13420960035749 -0.191528933821879
0.142110905322839 -0.198623612126504
0.150398576795015 -0.205212678179984
0.159067356720643 -0.211279493592426
0.16811178230224 -0.216808746901301
0.177526189102462 -0.221786480886398
0.187304709947427 -0.226200116851138
0.197441269929379 -0.230038477621279
0.207929578656845 -0.233291810717099
0.218763121506754 -0.235951812451675
0.229935151803763 -0.238011652801433
0.241438685552319 -0.239466000064469
0.253266499708268 -0.240311043799429
0.265411134226006 -0.240544514421553
0.277864897472831 -0.240165698073282
0.290619874197545 -0.23917544583971
0.303667935101709 -0.237576176879925
0.317000747134867 -0.235371875470324
0.330609783830706 -0.232568082244952
0.344486335235128 -0.229171880066449
0.358621517191145 -0.225191874998346
0.373006279910045 -0.220638172815656
0.387631415867 -0.215522351422736
0.402487567119295 -0.209857429471725
0.417565232169048 -0.203657831407191
0.432854772492479 -0.196939349109845
0.448346418844808 -0.189719100275549
0.46403027743119 -0.182015483643688
0.47989633601438 -0.173848131178146
0.495934470011695 -0.165237857301697
0.512134448618413 -0.156206605287707
0.528485940982374 -0.146777390919585
0.544978522444945 -0.136974243536887
0.561601680856407 -0.126822144596177
0.578344822968638 -0.116346963884112
0.595197280904322 -0.105575393529208
0.61214831869946 -0.094534879967153
0.629187138914278 -0.0832535540222168
0.646302889306651 -0.071760159274263
0.663484669561462 -0.0600839788869683
0.680721538069075 -0.0482547610782518
0.698002518745858 -0.0363026434185232
0.715316607889725 -0.0242580761462783
0.732652781063624 -0.012151744693772
};
\addplot [line width = \linewidthEightC, color = reference, opacity=\opacityRef, forget plot]
table {%
0.75 0
0.753523170948029 0.00279280543327332
0.768925845623016 0.0145238190889359
0.787816882133484 0.0282577723264694
0.80415552854538 0.0402376875281334
0.821899116039276 0.0531042963266373
0.839288473129272 0.0654929578304291
0.856522560119629 0.0775240659713745
0.873598456382751 0.0897126942873001
0.890544474124908 0.101200386881828
0.90800678730011 0.113106474280357
0.924808859825134 0.124033555388451
0.941638588905334 0.13529634475708
0.957958698272705 0.145673379302025
0.974843621253967 0.156306192278862
0.99134886264801 0.165954440832138
1.0076094865799 0.175437346100807
1.02410554885864 0.184471592307091
1.03982329368591 0.192770108580589
1.05537116527557 0.200397476553917
1.07110345363617 0.207764431834221
1.08664035797119 0.214753642678261
1.10186910629272 0.221051290631294
1.11671090126038 0.226568773388863
1.13137102127075 0.231674239039421
1.14577257633209 0.236207917332649
1.16016435623169 0.23989762365818
1.17421805858612 0.243143782019615
1.18791270256042 0.245294526219368
1.20142912864685 0.247762069106102
1.21463334560394 0.24887053668499
1.22741222381592 0.249909624457359
1.24012339115143 0.250114068388939
1.25202333927155 0.249380841851234
1.26388072967529 0.248331233859062
1.27532088756561 0.246677353978157
1.28605711460114 0.244481906294823
1.29728984832764 0.241131618618965
1.30784451961517 0.237725660204887
1.31821441650391 0.233360126614571
1.32800304889679 0.228812143206596
1.33755373954773 0.22309522330761
1.34667325019836 0.217099145054817
1.35544300079346 0.210521087050438
1.36346781253815 0.203774467110634
1.37150704860687 0.196189925074577
1.3785685300827 0.188550010323524
1.38564586639404 0.179883643984795
1.39231109619141 0.170665696263313
1.39832055568695 0.161737948656082
1.40419983863831 0.151760697364807
1.40931105613708 0.142088368535042
1.41433250904083 0.131272464990616
1.41886603832245 0.120177134871483
1.42279601097107 0.108847290277481
1.42635631561279 0.097615972161293
1.42962181568146 0.0861897468566895
1.43235814571381 0.0745447129011154
1.43463635444641 0.0618470758199692
1.43656992912292 0.0498790517449379
1.43786013126373 0.0380153059959412
1.4387925863266 0.0255251750349998
1.43930912017822 0.0135747715830803
1.4392523765564 0.00119618326425552
1.43879592418671 -0.0111459419131279
1.43777620792389 -0.0235336311161518
1.43651914596558 -0.0358590595424175
1.43486225605011 -0.0482927728444338
1.43278419971466 -0.0600577611476183
1.4304393529892 -0.0720855165272951
1.4271228313446 -0.084222448989749
1.42375731468201 -0.0957241542637348
1.41967737674713 -0.106477122753859
1.41527080535889 -0.117312215268612
1.41062927246094 -0.127986159175634
1.40547788143158 -0.138408347964287
1.39974355697632 -0.148424834012985
1.39366400241852 -0.15779297798872
1.38722693920135 -0.167378723621368
1.38020944595337 -0.175638616085052
1.37272560596466 -0.183970600366592
1.36492526531219 -0.19171454012394
1.35663866996765 -0.198583081364632
1.348064661026 -0.205265015363693
1.33931970596313 -0.211059078574181
1.32992684841156 -0.21644701063633
1.32024800777435 -0.221380144357681
1.31052160263062 -0.225584849715233
1.29988324642181 -0.229364201426506
1.2894229888916 -0.232347518205643
1.27818298339844 -0.234245434403419
1.2665878534317 -0.236352786421776
1.25473606586456 -0.237420871853828
1.24256551265717 -0.237729430198669
1.23032128810883 -0.23777349293232
1.21724438667297 -0.237215206027031
1.20450127124786 -0.235643416643143
1.19064486026764 -0.233625844120979
1.17712533473969 -0.230930835008621
1.16301393508911 -0.22785472869873
1.14889550209045 -0.223915502429008
1.13439702987671 -0.218928471207619
1.11972260475159 -0.213751181960106
1.10449635982513 -0.208093762397766
1.08948266506195 -0.201771467924118
1.07430446147919 -0.195095390081406
1.05854690074921 -0.18755978345871
1.04304146766663 -0.179741188883781
1.02714931964874 -0.171193659305573
1.01123416423798 -0.161972567439079
0.994935274124146 -0.152672529220581
0.978619813919067 -0.142867252230644
0.961926698684692 -0.132490646094084
0.945356607437134 -0.121917188167572
0.928461074829102 -0.110724173486233
0.911699295043945 -0.0993626322597265
0.894670903682709 -0.0875726919621229
0.877436101436615 -0.0758579690009356
0.860209047794342 -0.0638755857944489
0.842860579490662 -0.0510765369981527
0.825295627117157 -0.0387571156024933
0.807929694652557 -0.0263183265924454
0.790891408920288 -0.0136761516332626
0.773231983184814 -0.000791072845458984
0.756015241146088 0.0119439288973808
0.738057136535645 0.0250508338212967
0.720848202705383 0.037895455956459
0.703425645828247 0.051177553832531
0.686194777488708 0.0636991858482361
0.668795943260193 0.0760532468557358
0.651349663734436 0.0890210419893265
0.634273409843445 0.100983992218971
0.617281138896942 0.113018572330475
0.600155770778656 0.125255435705185
0.583060204982758 0.137026876211166
0.566046059131622 0.14812932908535
0.549233317375183 0.15972812473774
0.532855153083801 0.170420333743095
0.516194760799408 0.180778726935387
0.49957001209259 0.190965041518211
0.483509659767151 0.200426891446114
0.467245876789093 0.209605231881142
0.451349079608917 0.21832437813282
0.435436010360718 0.226440742611885
0.419892549514771 0.233941987156868
0.404565930366516 0.240940079092979
0.388980329036713 0.247701093554497
0.373833328485489 0.253909304738045
0.358835101127625 0.259296730160713
0.344098836183548 0.264382496476173
0.329507082700729 0.268702730536461
0.315252661705017 0.27226747572422
0.301427721977234 0.275462850928307
0.287785470485687 0.277898833155632
0.274553060531616 0.279671683907509
0.261250048875809 0.280859664082527
0.248908489942551 0.281454220414162
0.236331045627594 0.281314477324486
0.224010825157166 0.280637219548225
0.212161928415298 0.27923347055912
0.200592130422592 0.27721680700779
0.189521074295044 0.274612680077553
0.178839534521103 0.271292433142662
0.168328791856766 0.267465487122536
0.158427238464355 0.262949392199516
0.14859214425087 0.257743433117867
0.13961635529995 0.252292737364769
0.130757719278336 0.24597530066967
0.122524976730347 0.239387467503548
0.114593341946602 0.231932356953621
0.106964409351349 0.224073335528374
0.0997644066810608 0.215966537594795
0.0928100198507309 0.207478418946266
0.0864613205194473 0.197811737656593
0.0805961042642593 0.188384160399437
0.0749218314886093 0.178331151604652
0.0697550475597382 0.168197453022003
0.0653387904167175 0.157422542572021
0.0609834790229797 0.146422535181046
0.0572390109300613 0.135033935308456
0.053767129778862 0.123482570052147
0.0510748028755188 0.111469149589539
0.0484179556369781 0.0999787598848343
0.0462907701730728 0.0881332755088806
0.0445092618465424 0.0757099837064743
0.0433739721775055 0.0629844665527344
0.0427898466587067 0.0506128072738647
0.0426718592643738 0.0381730645895004
0.0428430140018463 0.0256267115473747
0.0435186475515366 0.0132057145237923
0.0449233949184418 0.000639155507087708
0.0463907271623611 -0.011755894869566
0.0486573725938797 -0.0241974890232086
0.0509359985589981 -0.0356634892523289
0.0540373474359512 -0.0473245326429605
0.0572926998138428 -0.0585837373510003
0.0612345039844513 -0.0698185227811337
0.0655556470155716 -0.0809560399502516
0.070042684674263 -0.0917527135461569
0.0753155499696732 -0.102031271904707
0.0807932317256927 -0.112181715667248
0.0872464776039124 -0.12203061580658
0.0934737324714661 -0.131032936275005
0.100644871592522 -0.140071183443069
0.107474207878113 -0.148268699645996
0.115461006760597 -0.156241156160831
0.123347744345665 -0.163607887923717
0.131583422422409 -0.170142985880375
0.140686899423599 -0.176600612699986
0.150220543146133 -0.182259418070316
0.159670382738113 -0.187441393733025
0.169857740402222 -0.192236512899399
0.180265784263611 -0.196216151118279
0.190937012434006 -0.19922399520874
0.202081441879272 -0.202182725071907
0.213205277919769 -0.20404514670372
0.224694848060608 -0.205373972654343
0.237096905708313 -0.20658016204834
0.249050796031952 -0.206162512302399
0.262043297290802 -0.205901101231575
0.274767935276031 -0.204449474811554
0.288257092237473 -0.203080162405968
0.301804393529892 -0.200261443853378
0.315744459629059 -0.19753398001194
0.329882264137268 -0.193378433585167
0.344353228807449 -0.188963443040848
0.358802407979965 -0.183855444192886
0.373736798763275 -0.178195767104626
0.388810962438583 -0.172125980257988
0.404093861579895 -0.165192469954491
0.419581115245819 -0.157898187637329
0.435053944587708 -0.150248296558857
0.450871527194977 -0.141360007226467
0.467097640037537 -0.132854625582695
0.483379304409027 -0.123316027224064
0.499529004096985 -0.113884422928095
0.515993416309357 -0.103290678933263
0.532474100589752 -0.0922192130237818
0.549250364303589 -0.0817647241055965
0.566079556941986 -0.0700159519910812
0.582710325717926 -0.0586193408817053
0.599467515945435 -0.0462303385138512
0.616545021533966 -0.03408382833004
0.633659839630127 -0.0217410065233707
0.650806307792664 -0.00917717814445496
0.667735397815704 0.00337307900190353
0.685066819190979 0.0168889611959457
0.702424228191376 0.0296528413891792
0.719577074050903 0.0431286171078682
0.736728429794312 0.0563033223152161
};
\addplot [semithick, red, dashed, forget plot]
table {%
0.75 0
0.767617956255685 0.0126152320166048
0.785221863675294 0.025202759744927
0.802800914680889 0.0377306780229099
0.820344353102624 0.0501672114827695
0.837841479099983 0.0624807937495562
0.855281653865309 0.0746401456242545
0.872654304077727 0.0866143520292023
0.889948926065816 0.0983729374930471
0.90715508962574 0.109885939954111
0.924262441427357 0.121123982665618
0.941260707923679 0.132058343994744
0.958139697658408 0.142661024921306
0.974889302841815 0.152904814062882
0.991499500036706 0.162763350083898
1.00796034976389 0.172211181389763
1.02426199480132 0.18122382306779
1.04039465691463 0.189777811119373
1.0563486317231 0.197850754138649
1.07211428137931 0.205421382738366
1.08768202473299 0.212469597210257
1.10304232467243 0.218976514139914
1.11818567241108 0.224924512976024
1.1331025686362 0.230297283873765
1.147783501695 0.23507987847157
1.16221892339532 0.239258765576722
1.17639922357808 0.242821893954846
1.19031470539574 0.245758764427774
1.20395556419321 0.248060513126141
1.21731187395941 0.249720006825457
1.23037358633393 0.250731949622618
1.24313054782987 0.251092997654816
1.2555725408717 0.250801875167454
1.26768935298946 0.24985948134918
1.27947087567813 0.248268973729967
1.29090722994702 0.246035811751358
1.30198890990496 0.243167744731957
1.31270692999354 0.2396747329547
1.32305295737637 0.235568799170581
1.33301941026506 0.230863819194334
1.34259950666846 0.225575271775738
1.3517872558846 0.219719976188812
1.36057739518197 0.213315848229081
1.36896528374308 0.206381700652654
1.37694677240494 0.198937103974783
1.38451806954475 0.191002311131823
1.39167562078374 0.182598238316676
1.39841601450881 0.173746486857157
1.4047359185686 0.164469388195659
1.41063204764278 0.154790055213092
1.41610115672778 0.144732426840967
1.42114005415556 0.134321297531182
1.42574562723652 0.123582327489247
1.42991487441761 0.112542032954869
1.43364493919855 0.101227758043132
1.43693314251624 0.089667630857217
1.4397770116222 0.0778905070053025
1.44217430452097 0.0659259035746353
1.44412302978876 0.0538039262643269
1.44562146208725 0.0415551919190076
1.44666815398553 0.0292107482394724
1.44726194486432 0.016801992026761
1.44740196775303 0.00436058696240971
1.44708765498222 -0.00808161836067393
1.44631874355061 -0.0204926716768206
1.4450952811249 -0.0328405972734807
1.44341763362016 -0.0450934751015305
1.44128649535024 -0.0572195188465802
1.4387029027798 -0.0691871532341023
1.43566825293285 -0.0809650910678666
1.4321843274821 -0.0925224107662305
1.42825332340439 -0.103828635462315
1.42387789076703 -0.114853815053745
1.41906117760902 -0.125568612881875
1.41380688088891 -0.135944398910193
1.40811930097384 -0.145953351233135
1.40200339507468 -0.155568567310703
1.39546482242266 -0.164764185292558
1.38850997106277 -0.173515513988717
1.38114595342609 -0.181799167392522
1.37338055620066 -0.189593196333819
1.36522213060654 -0.196877206377916
1.35667941319178 -0.203632448466232
1.34776127545165 -0.209841868311401
1.33847641259836 -0.215490103434114
1.328832995749 -0.220563423487838
1.3188383241811 -0.225049619377379
1.30849852098817 -0.228937857346307
1.29781831326193 -0.232218522442798
1.28680092623794 -0.234883078640104
1.27544810236402 -0.23692396914816
1.26376023613114 -0.238334571253006
1.25173659926596 -0.239109208409727
1.23937562223503 -0.239243211851919
1.22680212890194 -0.238807395220756
1.21393543612189 -0.237739020519201
1.2007824030094 -0.236042071113405
1.1873502195155 -0.233721342391173
1.17364639818837 -0.230782623136066
1.15967876862552 -0.227232843837933
1.14545546822952 -0.223080191387264
1.13098492732646 -0.218334192018305
1.11627584919941 -0.213005765794969
1.10133718666704 -0.207107256561536
1.08617811704058 -0.200652441341106
1.070808017043 -0.19365652287622
1.05523643885516 -0.186136108539406
1.03947308802467 -0.178109178316594
1.02352780361101 -0.169595044057188
1.00741054066936 -0.160614301730894
0.991131354992172 -0.151188778049092
0.974700389916618 -0.141341472499412
0.958127864949596 -0.131096495600034
0.941424065943487 -0.120479003995558
0.924599336561409 -0.109515132878708
0.90766407079076 -0.0982319261219058
0.890628706290982 -0.086657264431339
0.873503718391248 -0.074819791786618
0.856299614583182 -0.0627488403957748
0.839026929381251 -0.0504743543738914
0.8216962194482 -0.0380268123406879
0.804318058904672 -0.0254371491254941
0.786903034760972 -0.0127366767652624
0.769461742425004 4.29950186324069e-05
0.752004781254135 0.0128700386771497
0.734542750130382 0.0257124888331812
0.717086243048227 0.0385383222295833
0.699645844712938 0.0513155374272836
0.682232126154625 0.0640122340511521
0.664855640369735 0.0765966913776288
0.647526918007365 0.0890374460548427
0.630256463122704 0.101303368743397
0.613054749024055 0.113363739464377
0.595932214243 0.125188321440869
0.578899258658929 0.136747433220813
0.561966239808458 0.148012018873052
0.545143469406046 0.158953716055882
0.528441210092295 0.169544921769502
0.511869672407924 0.179758855622043
0.495439011959919 0.189569620465462
0.479159326695086 0.198952260295039
0.463040654116175 0.207882815357607
0.447092968153461 0.216338374482455
0.431326175222446 0.224297124738732
0.41575010873165 0.231738398637223
0.400374520922108 0.238642719233979
0.38520907038282 0.244991843656404
0.37026330284855 0.250768805749989
0.355546621900637 0.255957958716022
0.341068244918601 0.260545018739871
0.326837138059129 0.264517110634022
0.312861922218046 0.267862816346776
0.299150740017787 0.270572226688282
0.285711072187049 0.272636995643726
0.272549490830637 0.274050395018263
0.259671337879835 0.274807364778616
0.247080320574249 0.274904551350881
0.234778023333987 0.274340322593555
0.22282111677237 0.273147174350289
0.211212000441657 0.271312180989548
0.199956090718924 0.268842225658862
0.18905951088657 0.265745762806622
0.178528814558833 0.262032850875418
0.168370684504055 0.257715152729482
0.158591652631679 0.252805894649178
0.149197881311021 0.247319783843054
0.140195031151771 0.241272894158711
0.131588220396365 0.234682536784361
0.123382062255884 0.227567134929123
0.115580754025762 0.219946118404367
0.108188187942631 0.211839847297465
0.101208057557953 0.203269566144989
0.0946439418448079 0.19425738357691
0.088499358802492 0.184826268669802
0.0827777883280161 0.17500005435619
0.0774826693237491 0.164803439481848
0.0726173783628224 0.154261983466959
0.0681851974057714 0.143402090111052
0.0641892769319838 0.132250979303424
0.0606325991839522 0.12083664700672
0.0575179445429576 0.109187814856514
0.0548478626398547 0.0973338711778623
0.0526246487503009 0.0853048053160545
0.0508503253165833 0.0731311370573394
0.0495266280133111 0.0608438426868979
0.0486549955532945 0.0484742789692179
0.0482365623412967 0.036054106082026
0.0482721530701452 0.0236152103073717
0.0487622783751904 0.0111896270851532
0.0497071306928156 -0.00119053514168032
0.0511065794918781 -0.0134931700273918
0.0529601650583767 -0.0256862472791051
0.0552670900164109 -0.0377378855745846
0.0580262077740067 -0.0496164247880426
0.061236007111159 -0.0612904981681536
0.0648945922107272 -0.0727291054819839
0.0689996576140302 -0.0839016885812368
0.0735484579177227 -0.0947782113457813
0.0785377725810335 -0.10532924646507
0.0839638670453019 -0.11552607193497
0.0898224525211534 -0.125340780318524
0.0961086482561191 -0.134746403519034
0.10281695173701 -0.143717054765618
0.109941223832714 -0.152228087440402
0.117474696885182 -0.160256267121569
0.125410013591915 -0.167779948892829
0.133739302545796 -0.174779247148552
0.14245429208538 -0.181236180932355
0.151546457798383 -0.187134775860053
0.161007191589854 -0.192461105490089
0.170827973498639 -0.197203261519913
0.181000523658031 -0.201351252889932
0.191516912754772 -0.204896846530735
0.202369615510414 -0.207833373632873
0.213551501760898 -0.210155531524887
0.22505577088336 -0.211859210579139
0.236875844499746 -0.212941368432345
0.249005237340447 -0.213399962707235
0.261437426261433 -0.213233941776159
0.274165733616003 -0.212443283869614
0.28718323530924 -0.211029069576408
0.300482697779478 -0.208993571566475
0.314056543186631 -0.206340347198259
0.327896838839168 -0.203074323164878
0.341995305342881 -0.199201865241865
0.356343337731412 -0.194730829684218
0.370932034454633 -0.189670595476494
0.385752230119493 -0.184032078393973
0.400794528982751 -0.177827728806396
0.416049337197401 -0.171071515546711
0.431506892628147 -0.16377889817583
0.447157291657684 -0.15596678976561
0.462990512824106 -0.147653512015245
0.478996437395876 -0.138858744187699
0.495164867142796 -0.129603467046443
0.511485539633279 -0.119909902709476
0.527948141407569 -0.109801451123518
0.544542319363787 -0.0993026236948103
0.561257690663388 -0.0884389744880547
0.578083851424296 -0.0772370293146803
0.595010384429933 -0.0657242129685941
0.612026866043939 -0.0539287748255556
0.62912287248563 -0.0418797129959582
0.646287985590849 -0.029606697205851
0.663511798156965 -0.0171399905741775
0.680783918949166 -0.00451037045289776
0.698093977427272 0.00825095150093907
0.715431628237671 0.0211124098473716
0.732786555502968 0.0340421663000904
};
\addplot [semithick, green, dash pattern=on 1pt off 3pt on 3pt off 3pt, forget plot]
table {%
0.75 0
0.767347219519284 0.012137253483567
0.784683394455509 0.0242435861667548
0.801997486582194 0.036288155514001
0.819278471531621 0.0482402761288358
0.836515345700615 0.0600694978228124
0.853697133135913 0.0717456830897325
0.870812892392119 0.0832390837898383
0.887851723355235 0.0945204168511733
0.904802774024683 0.105560938798497
0.921655247246804 0.116332518924018
0.938398407392915 0.126807710918802
0.955021586975309 0.136959822789028
0.971514193195148 0.146762984887382
0.987865714417124 0.156192215896701
1.0040657265673 0.165223486610677
1.02010389945277 0.173833781364724
1.03597000300535 0.182001156979107
1.05165391345605 0.189704799085743
1.06714561945411 0.196925075719347
1.08243522815337 0.203643588062188
1.09751297130037 0.209843218238558
1.1123692113737 0.215508174058654
1.12699444784162 0.220624030609957
1.14137932362429 0.225177768584601
1.1555146318663 0.22915780921071
1.16939132313901 0.232554045621132
1.18300051319534 0.235357870442502
1.19633349138035 0.237562199321834
1.20938172974852 0.23916149003295
1.22213689283924 0.240151756735333
1.23459084790774 0.240530578918402
1.24673567520343 0.24029710459028
1.25856367765532 0.239452047402427
1.27006738911946 0.237997677671498
1.28123958024499 0.235937807669022
1.29207326111455 0.233277772045945
1.30256168017249 0.230024404727089
1.31269831956044 0.226186013888674
1.32247688771958 0.221772356568193
1.33189131078413 0.216794613983565
1.34093572465289 0.211265367835258
1.34960446953272 0.205198576949498
1.35789208820684 0.198609552872807
1.36579332846363 0.191514932670796
1.37330314928634 0.18393264729003
1.38041672978698 0.175881884318464
1.38712947958844 0.167383044631953
1.39343704941026 0.158457693037802
1.39933534088867 0.149128503482261
1.40482051503024 0.139419199628507
1.40988899904863 0.129354491659788
1.41453749160152 0.118960010081206
1.41876296661194 0.108262237150103
1.42256267593454 0.0972884364110782
1.42593415113889 0.086066580678514
1.42887520465437 0.0746252787098322
1.43138393047621 0.0629937007473294
1.43345870458359 0.0512015030695605
1.43509818517756 0.0392787516771522
1.43630131281055 0.0272558452352367
1.43706731045357 0.0151634373994425
1.43739568352819 0.00303235866024866
1.43728621991849 -0.00910646215137196
1.43673898997073 -0.0212220765475008
1.43575434648365 -0.0332835951024502
1.43433292468914 -0.0452602660529673
1.43247564221903 -0.0571215536269705
1.43018369904855 -0.0688372159533007
1.42745857739782 -0.0803773824121825
1.4243020415584 -0.0917126302942692
1.42071613759078 -0.102814060643163
1.41670319280875 -0.113653373158817
1.41226581492764 -0.124202940032363
1.40740689070718 -0.134435878560225
1.40212958387022 -0.144326122339201
1.3964373320371 -0.153848490767307
1.39033384239815 -0.162978756463708
1.38382308587764 -0.171693710079292
1.37690928964889 -0.179971221816331
1.36959692806383 -0.187790298849553
1.36189071236441 -0.195131137800147
1.35379557991389 -0.201975171526458
1.34531668404164 -0.20830510981277
1.33645938581161 -0.214104974062127
1.32722924897007 -0.219360126749639
1.31763203892423 -0.224057297001799
1.30767372588241 -0.22818460402503
1.2973604914176 -0.231731580045289
1.28669873695777 -0.234689193904312
1.27569509230234 -0.237049875620184
1.26435642233521 -0.238807541316647
1.25268983059913 -0.239957617226964
1.24070265912099 -0.240497061156197
1.22840248459827 -0.24042437986202
1.21579711159144 -0.239739641182038
1.20289456363565 -0.238444480231447
1.1897030732081 -0.236542099469318
1.17623107134149 -0.234037262792463
1.16248717745014 -0.230936284032406
1.14848018970572 -0.227247010319085
1.13421907610845 -0.222978800771493
1.11971296626323 -0.218142500920566
1.10497114378721 -0.212750413195198
1.09000303923492 -0.206816263729265
1.07481822341698 -0.200355165686855
1.05942640099553 -0.193383579258294
1.04383740425637 -0.185919268450478
1.02806118697728 -0.177981254778958
1.01210781833129 -0.169589767962932
0.995987476780275 -0.160766193724954
0.979710443928391 -0.151533018802209
0.963287098315684 -0.141913773283855
0.946727909140567 -0.131932970397911
0.930043429905878 -0.121616043880467
0.91324429198767 -0.110989283069269
0.896341198128817 -0.100079765872392
0.879344915861624 -0.0889152897708354
0.862266270864871 -0.0775243010211479
0.845116140261585 -0.0659358222307342
0.827905445864243 -0.0541793784842489
0.810645147374399 -0.042284922204474
0.793346235543765 -0.0302827569353333
0.776019725303823 -0.0182034602382608
0.758676648870971 -0.00607780589600706
0.74132804883419 0.00606331437980999
0.723984971232126 0.0181889695395348
0.706658458626486 0.0302682678858693
0.689359543178601 0.0422704356646973
0.672099239736024 0.0541648953575106
0.654888538936098 0.0659213434775336
0.637738400333409 0.077509827673207
0.620659745558167 0.0889008229448788
0.603663451512546 0.10006530678342
0.586760343612044 0.110974833042994
0.569961189078862 0.121601604364455
0.55327669029407 0.131918542970798
0.536717478214957 0.141899359661781
0.520294105863188 0.151518620841353
0.504017041888231 0.160751813418715
0.487896664208613 0.169575407431921
0.471943253730781 0.177966916251536
0.456166988141264 0.18590495423109
0.440577935762111 0.193369291680399
0.425186049451712 0.200340907046881
0.410001160522753 0.206802036197834
0.395032972635676 0.212736218702091
0.38029105560979 0.218128341010777
0.36578483907538 0.222964676431608
0.351523605870586 0.227232921776542
0.337516485069478 0.23092223053541
0.323772444518238 0.234023242385741
0.310300282763089 0.236528108790522
0.297108620287721 0.238430514364134
0.284205890052545 0.239725693611976
0.27160032745433 0.240410442590784
0.259299960006205 0.240483125024951
0.247312597261469 0.239943672488383
0.2356458217318 0.238793578459958
0.224306981713199 0.237035886402939
0.21330318694315 0.234675172482025
0.202641307788403 0.231717524032126
0.192327978173203 0.228170515287301
0.1823696017639 0.22404318200088
0.172772360193836 0.219345996319791
0.163542221572307 0.214090842618848
0.154684947376684 0.208290994107505
0.146206096155645 0.201961089155653
0.138111023180157 0.195117105712544
0.130404876038488 0.187776332069152
0.12309258692145 0.179957332525672
0.116178862798813 0.171679907117356
0.109668174793245 0.162965045212245
0.10356474787866 0.153834873345794
0.0978725516915446 0.144312598005936
0.0925952928735381 0.134422444219678
0.0877364090495248 0.124189590767097
0.0832990643281257 0.113640102728524
0.0792861460924091 0.102800861916777
0.0757002628080311 0.0916994956001386
0.0725437425870994 0.0803643038046822
0.069818632284583 0.0688241854024739
0.0675266969522104 0.0571085631419612
0.0656694195214907 0.0452473077513642
0.0642480006271671 0.0332706612373471
0.0632633585132331 0.0212091595029022
0.0627161289858991 0.00909355441514592
0.0626066653929575 -0.00304526453801165
0.0629350386185861 -0.0151763488558262
0.0637010370885092 -0.0272687697454203
0.0649041667842718 -0.0392916968007361
0.0665436512687222 -0.0512144764858195
0.0686184317292081 -0.0630067102720941
0.0711271670520408 -0.0746383322854955
0.0740682339532131 -0.0860796863272447
0.0774397272079407 -0.0973016021399556
0.0812394600470219 -0.108275470796062
0.0854649648223936 -0.118973319084073
0.0901134940873028 -0.12936788275431
0.0951820222852561 -0.139432678452301
0.100667248289028 -0.149142074107302
0.10656559906303 -0.158471357449301
0.112873234718821 -0.167396802199979
0.119586055168304 -0.175895731332128
0.126699708426136 -0.183946576646328
0.13420960035749 -0.191528933821879
0.142110905322839 -0.198623612126504
0.150398576795015 -0.205212678179984
0.159067356720643 -0.211279493592426
0.16811178230224 -0.216808746901301
0.177526189102462 -0.221786480886398
0.187304709947427 -0.226200116851138
0.197441269929379 -0.230038477621279
0.207929578656845 -0.233291810717099
0.218763121506754 -0.235951812451675
0.229935151803763 -0.238011652801433
0.241438685552319 -0.239466000064469
0.253266499708268 -0.240311043799429
0.265411134226006 -0.240544514421553
0.277864897472831 -0.240165698073282
0.290619874197545 -0.23917544583971
0.303667935101709 -0.237576176879925
0.317000747134867 -0.235371875470324
0.330609783830706 -0.232568082244952
0.344486335235128 -0.229171880066449
0.358621517191145 -0.225191874998346
0.373006279910045 -0.220638172815656
0.387631415867 -0.215522351422736
0.402487567119295 -0.209857429471725
0.417565232169048 -0.203657831407191
0.432854772492479 -0.196939349109845
0.448346418844808 -0.189719100275549
0.46403027743119 -0.182015483643688
0.47989633601438 -0.173848131178146
0.495934470011695 -0.165237857301697
0.512134448618413 -0.156206605287707
0.528485940982374 -0.146777390919585
0.544978522444945 -0.136974243536887
0.561601680856407 -0.126822144596177
0.578344822968638 -0.116346963884112
0.595197280904322 -0.105575393529208
0.61214831869946 -0.094534879967153
0.629187138914278 -0.0832535540222168
0.646302889306651 -0.071760159274263
0.663484669561462 -0.0600839788869683
0.680721538069075 -0.0482547610782518
0.698002518745858 -0.0363026434185232
0.715316607889725 -0.0242580761462783
0.732652781063624 -0.012151744693772
};
\addplot [line width = \linewidthEightC, color = reference, opacity=\opacityRef, forget plot]
table {%
0.75 0
0.753872394561768 0.00296931713819504
0.769841015338898 0.0152800679206848
0.788493454456329 0.0293015763163567
0.804714441299438 0.0408521145582199
0.822663545608521 0.0546151995658875
0.839469373226166 0.0669476538896561
0.856497883796692 0.0794207900762558
0.87327241897583 0.0917330086231232
0.890529334545135 0.104056358337402
0.907194793224335 0.115714624524117
0.923994243144989 0.127321138978004
0.940577685832977 0.138546094298363
0.956953823566437 0.149493724107742
0.973733603954315 0.160101965069771
0.989760220050812 0.169872552156448
1.00588268041611 0.179315626621246
1.02190202474594 0.188569515943527
1.0377259850502 0.197275161743164
1.05352276563644 0.205339014530182
1.06920772790909 0.213177859783173
1.08433622121811 0.219960987567902
1.09964543581009 0.226837664842606
1.11455518007278 0.232605963945389
1.12919586896896 0.237649232149124
1.14362555742264 0.242571264505386
1.15752929449081 0.246482372283936
1.17139905691147 0.24996030330658
1.18522673845291 0.252812802791595
1.19848269224167 0.255065679550171
1.21177059412003 0.25626277923584
1.22420090436935 0.257461160421371
1.23684638738632 0.257648319005966
1.24885576963425 0.257589936256409
1.2608123421669 0.256369680166245
1.27200001478195 0.254667520523071
1.28332442045212 0.252602100372314
1.29427093267441 0.249559730291367
1.30443471670151 0.246131598949432
1.31472533941269 0.241986006498337
1.32430344820023 0.236804127693176
1.3337693810463 0.232218146324158
1.34275513887405 0.226316064596176
1.35180753469467 0.219817489385605
1.36007553339005 0.212672114372253
1.36818891763687 0.205383598804474
1.37571340799332 0.197313606739044
1.38288635015488 0.188903331756592
1.38975268602371 0.180093884468079
1.39641505479813 0.170361652970314
1.40236634016037 0.160396933555603
1.40807050466537 0.150725960731506
1.41342955827713 0.140263006091118
1.41788250207901 0.129864528775215
1.42209476232529 0.118277236819267
1.42568439245224 0.107168734073639
1.42925864458084 0.0957269221544266
1.43210631608963 0.083839163184166
1.43460613489151 0.0717771649360657
1.43657451868057 0.0596280544996262
1.43807882070541 0.0474385246634483
1.43915241956711 0.0350436642765999
1.43994587659836 0.0229079499840736
1.44012814760208 0.0104316771030426
1.43989807367325 -0.00162634998559952
1.43920189142227 -0.0140606984496117
1.43826860189438 -0.0262347050011158
1.43681186437607 -0.0383102484047413
1.43478125333786 -0.0505476258695126
1.43239384889603 -0.0627208575606346
1.42951458692551 -0.0741274133324623
1.42632108926773 -0.085835050791502
1.42234629392624 -0.0972288325428963
1.41807502508163 -0.108124792575836
1.41338485479355 -0.118488803505898
1.4081489443779 -0.128621406853199
1.40279883146286 -0.138786554336548
1.3967097401619 -0.148754306137562
1.39033287763596 -0.157727539539337
1.38333147764206 -0.166433803737164
1.37607675790787 -0.174492001533508
1.36853498220444 -0.18255640566349
1.36029106378555 -0.189799346029758
1.35184401273727 -0.196774318814278
1.34271401166916 -0.203288376331329
1.33337062597275 -0.209014520049095
1.32372921705246 -0.21407151222229
1.3133493065834 -0.218725681304932
1.30315655469894 -0.222533911466599
1.2922003865242 -0.225998312234879
1.28145462274551 -0.228702187538147
1.27005964517593 -0.230831608176231
1.25827759504318 -0.232086986303329
1.24603861570358 -0.232694894075394
1.23357087373734 -0.233212426304817
1.22073239088058 -0.232559159398079
1.2074823975563 -0.231681451201439
1.19404035806656 -0.229693070054054
1.18050843477249 -0.227316081523895
1.16620832681656 -0.224141612648964
1.15198558568954 -0.220559492707253
1.13772720098495 -0.216436386108398
1.12288373708725 -0.211538389325142
1.10793477296829 -0.206481009721756
1.09262079000473 -0.200659170746803
1.07703965902328 -0.193901732563972
1.06144136190414 -0.186651051044464
1.04564720392227 -0.17900125682354
1.02965492010117 -0.170992463827133
1.01341634988785 -0.16224117577076
0.996581256389618 -0.153244256973267
0.979956090450287 -0.143578670918941
0.963174164295197 -0.13376097381115
0.946201741695404 -0.123280361294746
0.929143726825714 -0.112548194825649
0.912477791309357 -0.101766012609005
0.895073592662811 -0.0904295686632395
0.87775707244873 -0.0791907617822289
0.859839379787445 -0.0672308020293713
0.842418789863586 -0.0550942849367857
0.824903070926666 -0.0427803471684456
0.807085573673248 -0.0307742618024349
0.789504826068878 -0.0183024182915688
0.771585643291473 -0.00563688576221466
0.753989219665527 0.00664906203746796
0.736419677734375 0.0186057090759277
0.718670427799225 0.0315524488687515
0.700838804244995 0.0442024767398834
0.683323442935944 0.0562011152505875
0.665604174137115 0.0689208954572678
0.648389399051666 0.0808267742395401
0.630715310573578 0.0931119918823242
0.613253057003021 0.104955345392227
0.596066474914551 0.116564348340034
0.578798592090607 0.127805218100548
0.561539947986603 0.138801619410515
0.544806659221649 0.149569422006607
0.528018116950989 0.159734547138214
0.511102437973022 0.169827118515968
0.494551062583923 0.17961198091507
0.478132247924805 0.188523024320602
0.461697816848755 0.197627246379852
0.445467472076416 0.206011563539505
0.429276823997498 0.213814675807953
0.413314163684845 0.221316874027252
0.397540211677551 0.227813214063644
0.382380127906799 0.234394431114197
0.367304682731628 0.240090847015381
0.352190852165222 0.245130062103271
0.337321311235428 0.249691992998123
0.322803348302841 0.253638833761215
0.308711349964142 0.257093489170074
0.294688314199448 0.259886652231216
0.280828237533569 0.261959493160248
0.26747253537178 0.263486206531525
0.254233062267303 0.264392554759979
0.241452842950821 0.264722675085068
0.22906693816185 0.264391303062439
0.216577291488647 0.263258367776871
0.205240994691849 0.261858880519867
0.19328099489212 0.25945657491684
0.182593554258347 0.256741583347321
0.171980738639832 0.253198504447937
0.161550343036652 0.248969078063965
0.151442885398865 0.244262516498566
0.141721367835999 0.238904356956482
0.132843345403671 0.233042150735855
0.124135226011276 0.226142048835754
0.115805476903915 0.219831794500351
0.108031675219536 0.212160646915436
0.100419729948044 0.204530626535416
0.0933786332607269 0.196088492870331
0.0868934541940689 0.187331706285477
0.0807279050350189 0.177957534790039
0.0748851448297501 0.168466448783875
0.0695424973964691 0.157659590244293
0.0645518451929092 0.147757276892662
0.0600231140851974 0.137085288763046
0.0559238493442535 0.125929206609726
0.0524058490991592 0.114621087908745
0.0494729429483414 0.102615624666214
0.0466537624597549 0.0906480699777603
0.0443973392248154 0.0795188248157501
0.0425682067871094 0.0672145336866379
0.0412109196186066 0.0546046942472458
0.0405267924070358 0.0419324859976768
0.0399402529001236 0.0298537909984589
0.040125235915184 0.0171201899647713
0.0406114012002945 0.00519850850105286
0.0414679497480392 -0.00754618644714355
0.0430020987987518 -0.019913524389267
0.0449895858764648 -0.0321720279753208
0.0474427938461304 -0.0442282743752003
0.0501406639814377 -0.0561283137649298
0.0535188466310501 -0.0678538521751761
0.0570553541183472 -0.0792318619787693
0.0610103011131287 -0.0904370509088039
0.0656085014343262 -0.101647263392806
0.0706855356693268 -0.112122390419245
0.0762521326541901 -0.123013310134411
0.0816887021064758 -0.132480062544346
0.0879796743392944 -0.142002150416374
0.0947590172290802 -0.151315078139305
0.101638972759247 -0.159855872392654
0.109116926789284 -0.168159693479538
0.117069125175476 -0.17610864341259
0.125141456723213 -0.182985693216324
0.133796185255051 -0.18969415128231
0.142743274569511 -0.196165233850479
0.152204781770706 -0.201891034841537
0.161871701478958 -0.207250967621803
0.172001123428345 -0.211410805583
0.18205127120018 -0.214577436447144
0.19274765253067 -0.218232333660126
0.20387077331543 -0.220970332622528
0.215286552906036 -0.222751289606094
0.227169066667557 -0.224276214838028
0.238853871822357 -0.224803671240807
0.250970125198364 -0.224655568599701
0.263860523700714 -0.223551362752914
0.276638507843018 -0.222397029399872
0.290043860673904 -0.22059366106987
0.303374618291855 -0.217770516872406
0.316636711359024 -0.214374512434006
0.330443799495697 -0.21027959883213
0.344644486904144 -0.205522939562798
0.359060555696487 -0.200260668992996
0.373477816581726 -0.194866739213467
0.388372957706451 -0.187787689268589
0.403304934501648 -0.180687285959721
0.41848486661911 -0.173147551715374
0.434324562549591 -0.164632923901081
0.44950133562088 -0.156071752309799
0.465089797973633 -0.146662354469299
0.481183171272278 -0.137003615498543
0.497013926506042 -0.127051383256912
0.512510657310486 -0.116320677101612
0.528296709060669 -0.105709701776505
0.544742166996002 -0.0940365977585316
0.561196982860565 -0.082288084551692
0.577502131462097 -0.0691286595538259
0.59410697221756 -0.0570499859750271
0.610350668430328 -0.0443370193243027
0.626858532428741 -0.0314974710345268
0.643454611301422 -0.0184329710900784
0.660348832607269 -0.00499363243579865
0.676748991012573 0.00888577103614807
0.693520486354828 0.0225076973438263
0.710557699203491 0.0359528660774231
0.727058529853821 0.0498010590672493
};
\addplot [semithick, red, dashed, forget plot]
table {%
0.75 0
0.767617956255685 0.0126152320166048
0.785221863675294 0.025202759744927
0.802800914680889 0.0377306780229099
0.820344353102624 0.0501672114827695
0.837841479099983 0.0624807937495562
0.855281653865309 0.0746401456242545
0.872654304077727 0.0866143520292023
0.889948926065816 0.0983729374930471
0.90715508962574 0.109885939954111
0.924262441427357 0.121123982665618
0.941260707923679 0.132058343994744
0.958139697658408 0.142661024921306
0.974889302841815 0.152904814062882
0.991499500036706 0.162763350083898
1.00796034976389 0.172211181389763
1.02426199480132 0.18122382306779
1.04039465691463 0.189777811119373
1.0563486317231 0.197850754138649
1.07211428137931 0.205421382738366
1.08768202473299 0.212469597210257
1.10304232467243 0.218976514139914
1.11818567241108 0.224924512976024
1.1331025686362 0.230297283873765
1.147783501695 0.23507987847157
1.16221892339532 0.239258765576722
1.17639922357808 0.242821893954846
1.19031470539574 0.245758764427774
1.20395556419321 0.248060513126141
1.21731187395941 0.249720006825457
1.23037358633393 0.250731949622618
1.24313054782987 0.251092997654816
1.2555725408717 0.250801875167454
1.26768935298946 0.24985948134918
1.27947087567813 0.248268973729967
1.29090722994702 0.246035811751358
1.30198890990496 0.243167744731957
1.31270692999354 0.2396747329547
1.32305295737637 0.235568799170581
1.33301941026506 0.230863819194334
1.34259950666846 0.225575271775738
1.3517872558846 0.219719976188812
1.36057739518197 0.213315848229081
1.36896528374308 0.206381700652654
1.37694677240494 0.198937103974783
1.38451806954475 0.191002311131823
1.39167562078374 0.182598238316676
1.39841601450881 0.173746486857157
1.4047359185686 0.164469388195659
1.41063204764278 0.154790055213092
1.41610115672778 0.144732426840967
1.42114005415556 0.134321297531182
1.42574562723652 0.123582327489247
1.42991487441761 0.112542032954869
1.43364493919855 0.101227758043132
1.43693314251624 0.089667630857217
1.4397770116222 0.0778905070053025
1.44217430452097 0.0659259035746353
1.44412302978876 0.0538039262643269
1.44562146208725 0.0415551919190076
1.44666815398553 0.0292107482394724
1.44726194486432 0.016801992026761
1.44740196775303 0.00436058696240971
1.44708765498222 -0.00808161836067393
1.44631874355061 -0.0204926716768206
1.4450952811249 -0.0328405972734807
1.44341763362016 -0.0450934751015305
1.44128649535024 -0.0572195188465802
1.4387029027798 -0.0691871532341023
1.43566825293285 -0.0809650910678666
1.4321843274821 -0.0925224107662305
1.42825332340439 -0.103828635462315
1.42387789076703 -0.114853815053745
1.41906117760902 -0.125568612881875
1.41380688088891 -0.135944398910193
1.40811930097384 -0.145953351233135
1.40200339507468 -0.155568567310703
1.39546482242266 -0.164764185292558
1.38850997106277 -0.173515513988717
1.38114595342609 -0.181799167392522
1.37338055620066 -0.189593196333819
1.36522213060654 -0.196877206377916
1.35667941319178 -0.203632448466232
1.34776127545165 -0.209841868311401
1.33847641259836 -0.215490103434114
1.328832995749 -0.220563423487838
1.3188383241811 -0.225049619377379
1.30849852098817 -0.228937857346307
1.29781831326193 -0.232218522442798
1.28680092623794 -0.234883078640104
1.27544810236402 -0.23692396914816
1.26376023613114 -0.238334571253006
1.25173659926596 -0.239109208409727
1.23937562223503 -0.239243211851919
1.22680212890194 -0.238807395220756
1.21393543612189 -0.237739020519201
1.2007824030094 -0.236042071113405
1.1873502195155 -0.233721342391173
1.17364639818837 -0.230782623136066
1.15967876862552 -0.227232843837933
1.14545546822952 -0.223080191387264
1.13098492732646 -0.218334192018305
1.11627584919941 -0.213005765794969
1.10133718666704 -0.207107256561536
1.08617811704058 -0.200652441341106
1.070808017043 -0.19365652287622
1.05523643885516 -0.186136108539406
1.03947308802467 -0.178109178316594
1.02352780361101 -0.169595044057188
1.00741054066936 -0.160614301730894
0.991131354992172 -0.151188778049092
0.974700389916618 -0.141341472499412
0.958127864949596 -0.131096495600034
0.941424065943487 -0.120479003995558
0.924599336561409 -0.109515132878708
0.90766407079076 -0.0982319261219058
0.890628706290982 -0.086657264431339
0.873503718391248 -0.074819791786618
0.856299614583182 -0.0627488403957748
0.839026929381251 -0.0504743543738914
0.8216962194482 -0.0380268123406879
0.804318058904672 -0.0254371491254941
0.786903034760972 -0.0127366767652624
0.769461742425004 4.29950186324069e-05
0.752004781254135 0.0128700386771497
0.734542750130382 0.0257124888331812
0.717086243048227 0.0385383222295833
0.699645844712938 0.0513155374272836
0.682232126154625 0.0640122340511521
0.664855640369735 0.0765966913776288
0.647526918007365 0.0890374460548427
0.630256463122704 0.101303368743397
0.613054749024055 0.113363739464377
0.595932214243 0.125188321440869
0.578899258658929 0.136747433220813
0.561966239808458 0.148012018873052
0.545143469406046 0.158953716055882
0.528441210092295 0.169544921769502
0.511869672407924 0.179758855622043
0.495439011959919 0.189569620465462
0.479159326695086 0.198952260295039
0.463040654116175 0.207882815357607
0.447092968153461 0.216338374482455
0.431326175222446 0.224297124738732
0.41575010873165 0.231738398637223
0.400374520922108 0.238642719233979
0.38520907038282 0.244991843656404
0.37026330284855 0.250768805749989
0.355546621900637 0.255957958716022
0.341068244918601 0.260545018739871
0.326837138059129 0.264517110634022
0.312861922218046 0.267862816346776
0.299150740017787 0.270572226688282
0.285711072187049 0.272636995643726
0.272549490830637 0.274050395018263
0.259671337879835 0.274807364778616
0.247080320574249 0.274904551350881
0.234778023333987 0.274340322593555
0.22282111677237 0.273147174350289
0.211212000441657 0.271312180989548
0.199956090718924 0.268842225658862
0.18905951088657 0.265745762806622
0.178528814558833 0.262032850875418
0.168370684504055 0.257715152729482
0.158591652631679 0.252805894649178
0.149197881311021 0.247319783843054
0.140195031151771 0.241272894158711
0.131588220396365 0.234682536784361
0.123382062255884 0.227567134929123
0.115580754025762 0.219946118404367
0.108188187942631 0.211839847297465
0.101208057557953 0.203269566144989
0.0946439418448079 0.19425738357691
0.088499358802492 0.184826268669802
0.0827777883280161 0.17500005435619
0.0774826693237491 0.164803439481848
0.0726173783628224 0.154261983466959
0.0681851974057714 0.143402090111052
0.0641892769319838 0.132250979303424
0.0606325991839522 0.12083664700672
0.0575179445429576 0.109187814856514
0.0548478626398547 0.0973338711778623
0.0526246487503009 0.0853048053160545
0.0508503253165833 0.0731311370573394
0.0495266280133111 0.0608438426868979
0.0486549955532945 0.0484742789692179
0.0482365623412967 0.036054106082026
0.0482721530701452 0.0236152103073717
0.0487622783751904 0.0111896270851532
0.0497071306928156 -0.00119053514168032
0.0511065794918781 -0.0134931700273918
0.0529601650583767 -0.0256862472791051
0.0552670900164109 -0.0377378855745846
0.0580262077740067 -0.0496164247880426
0.061236007111159 -0.0612904981681536
0.0648945922107272 -0.0727291054819839
0.0689996576140302 -0.0839016885812368
0.0735484579177227 -0.0947782113457813
0.0785377725810335 -0.10532924646507
0.0839638670453019 -0.11552607193497
0.0898224525211534 -0.125340780318524
0.0961086482561191 -0.134746403519034
0.10281695173701 -0.143717054765618
0.109941223832714 -0.152228087440402
0.117474696885182 -0.160256267121569
0.125410013591915 -0.167779948892829
0.133739302545796 -0.174779247148552
0.14245429208538 -0.181236180932355
0.151546457798383 -0.187134775860053
0.161007191589854 -0.192461105490089
0.170827973498639 -0.197203261519913
0.181000523658031 -0.201351252889932
0.191516912754772 -0.204896846530735
0.202369615510414 -0.207833373632873
0.213551501760898 -0.210155531524887
0.22505577088336 -0.211859210579139
0.236875844499746 -0.212941368432345
0.249005237340447 -0.213399962707235
0.261437426261433 -0.213233941776159
0.274165733616003 -0.212443283869614
0.28718323530924 -0.211029069576408
0.300482697779478 -0.208993571566475
0.314056543186631 -0.206340347198259
0.327896838839168 -0.203074323164878
0.341995305342881 -0.199201865241865
0.356343337731412 -0.194730829684218
0.370932034454633 -0.189670595476494
0.385752230119493 -0.184032078393973
0.400794528982751 -0.177827728806396
0.416049337197401 -0.171071515546711
0.431506892628147 -0.16377889817583
0.447157291657684 -0.15596678976561
0.462990512824106 -0.147653512015245
0.478996437395876 -0.138858744187699
0.495164867142796 -0.129603467046443
0.511485539633279 -0.119909902709476
0.527948141407569 -0.109801451123518
0.544542319363787 -0.0993026236948103
0.561257690663388 -0.0884389744880547
0.578083851424296 -0.0772370293146803
0.595010384429933 -0.0657242129685941
0.612026866043939 -0.0539287748255556
0.62912287248563 -0.0418797129959582
0.646287985590849 -0.029606697205851
0.663511798156965 -0.0171399905741775
0.680783918949166 -0.00451037045289776
0.698093977427272 0.00825095150093907
0.715431628237671 0.0211124098473716
0.732786555502968 0.0340421663000904
};
\addplot [semithick, green, dash pattern=on 1pt off 3pt on 3pt off 3pt, forget plot]
table {%
0.75 0
0.767347219519284 0.012137253483567
0.784683394455509 0.0242435861667548
0.801997486582194 0.036288155514001
0.819278471531621 0.0482402761288358
0.836515345700615 0.0600694978228124
0.853697133135913 0.0717456830897325
0.870812892392119 0.0832390837898383
0.887851723355235 0.0945204168511733
0.904802774024683 0.105560938798497
0.921655247246804 0.116332518924018
0.938398407392915 0.126807710918802
0.955021586975309 0.136959822789028
0.971514193195148 0.146762984887382
0.987865714417124 0.156192215896701
1.0040657265673 0.165223486610677
1.02010389945277 0.173833781364724
1.03597000300535 0.182001156979107
1.05165391345605 0.189704799085743
1.06714561945411 0.196925075719347
1.08243522815337 0.203643588062188
1.09751297130037 0.209843218238558
1.1123692113737 0.215508174058654
1.12699444784162 0.220624030609957
1.14137932362429 0.225177768584601
1.1555146318663 0.22915780921071
1.16939132313901 0.232554045621132
1.18300051319534 0.235357870442502
1.19633349138035 0.237562199321834
1.20938172974852 0.23916149003295
1.22213689283924 0.240151756735333
1.23459084790774 0.240530578918402
1.24673567520343 0.24029710459028
1.25856367765532 0.239452047402427
1.27006738911946 0.237997677671498
1.28123958024499 0.235937807669022
1.29207326111455 0.233277772045945
1.30256168017249 0.230024404727089
1.31269831956044 0.226186013888674
1.32247688771958 0.221772356568193
1.33189131078413 0.216794613983565
1.34093572465289 0.211265367835258
1.34960446953272 0.205198576949498
1.35789208820684 0.198609552872807
1.36579332846363 0.191514932670796
1.37330314928634 0.18393264729003
1.38041672978698 0.175881884318464
1.38712947958844 0.167383044631953
1.39343704941026 0.158457693037802
1.39933534088867 0.149128503482261
1.40482051503024 0.139419199628507
1.40988899904863 0.129354491659788
1.41453749160152 0.118960010081206
1.41876296661194 0.108262237150103
1.42256267593454 0.0972884364110782
1.42593415113889 0.086066580678514
1.42887520465437 0.0746252787098322
1.43138393047621 0.0629937007473294
1.43345870458359 0.0512015030695605
1.43509818517756 0.0392787516771522
1.43630131281055 0.0272558452352367
1.43706731045357 0.0151634373994425
1.43739568352819 0.00303235866024866
1.43728621991849 -0.00910646215137196
1.43673898997073 -0.0212220765475008
1.43575434648365 -0.0332835951024502
1.43433292468914 -0.0452602660529673
1.43247564221903 -0.0571215536269705
1.43018369904855 -0.0688372159533007
1.42745857739782 -0.0803773824121825
1.4243020415584 -0.0917126302942692
1.42071613759078 -0.102814060643163
1.41670319280875 -0.113653373158817
1.41226581492764 -0.124202940032363
1.40740689070718 -0.134435878560225
1.40212958387022 -0.144326122339201
1.3964373320371 -0.153848490767307
1.39033384239815 -0.162978756463708
1.38382308587764 -0.171693710079292
1.37690928964889 -0.179971221816331
1.36959692806383 -0.187790298849553
1.36189071236441 -0.195131137800147
1.35379557991389 -0.201975171526458
1.34531668404164 -0.20830510981277
1.33645938581161 -0.214104974062127
1.32722924897007 -0.219360126749639
1.31763203892423 -0.224057297001799
1.30767372588241 -0.22818460402503
1.2973604914176 -0.231731580045289
1.28669873695777 -0.234689193904312
1.27569509230234 -0.237049875620184
1.26435642233521 -0.238807541316647
1.25268983059913 -0.239957617226964
1.24070265912099 -0.240497061156197
1.22840248459827 -0.24042437986202
1.21579711159144 -0.239739641182038
1.20289456363565 -0.238444480231447
1.1897030732081 -0.236542099469318
1.17623107134149 -0.234037262792463
1.16248717745014 -0.230936284032406
1.14848018970572 -0.227247010319085
1.13421907610845 -0.222978800771493
1.11971296626323 -0.218142500920566
1.10497114378721 -0.212750413195198
1.09000303923492 -0.206816263729265
1.07481822341698 -0.200355165686855
1.05942640099553 -0.193383579258294
1.04383740425637 -0.185919268450478
1.02806118697728 -0.177981254778958
1.01210781833129 -0.169589767962932
0.995987476780275 -0.160766193724954
0.979710443928391 -0.151533018802209
0.963287098315684 -0.141913773283855
0.946727909140567 -0.131932970397911
0.930043429905878 -0.121616043880467
0.91324429198767 -0.110989283069269
0.896341198128817 -0.100079765872392
0.879344915861624 -0.0889152897708354
0.862266270864871 -0.0775243010211479
0.845116140261585 -0.0659358222307342
0.827905445864243 -0.0541793784842489
0.810645147374399 -0.042284922204474
0.793346235543765 -0.0302827569353333
0.776019725303823 -0.0182034602382608
0.758676648870971 -0.00607780589600706
0.74132804883419 0.00606331437980999
0.723984971232126 0.0181889695395348
0.706658458626486 0.0302682678858693
0.689359543178601 0.0422704356646973
0.672099239736024 0.0541648953575106
0.654888538936098 0.0659213434775336
0.637738400333409 0.077509827673207
0.620659745558167 0.0889008229448788
0.603663451512546 0.10006530678342
0.586760343612044 0.110974833042994
0.569961189078862 0.121601604364455
0.55327669029407 0.131918542970798
0.536717478214957 0.141899359661781
0.520294105863188 0.151518620841353
0.504017041888231 0.160751813418715
0.487896664208613 0.169575407431921
0.471943253730781 0.177966916251536
0.456166988141264 0.18590495423109
0.440577935762111 0.193369291680399
0.425186049451712 0.200340907046881
0.410001160522753 0.206802036197834
0.395032972635676 0.212736218702091
0.38029105560979 0.218128341010777
0.36578483907538 0.222964676431608
0.351523605870586 0.227232921776542
0.337516485069478 0.23092223053541
0.323772444518238 0.234023242385741
0.310300282763089 0.236528108790522
0.297108620287721 0.238430514364134
0.284205890052545 0.239725693611976
0.27160032745433 0.240410442590784
0.259299960006205 0.240483125024951
0.247312597261469 0.239943672488383
0.2356458217318 0.238793578459958
0.224306981713199 0.237035886402939
0.21330318694315 0.234675172482025
0.202641307788403 0.231717524032126
0.192327978173203 0.228170515287301
0.1823696017639 0.22404318200088
0.172772360193836 0.219345996319791
0.163542221572307 0.214090842618848
0.154684947376684 0.208290994107505
0.146206096155645 0.201961089155653
0.138111023180157 0.195117105712544
0.130404876038488 0.187776332069152
0.12309258692145 0.179957332525672
0.116178862798813 0.171679907117356
0.109668174793245 0.162965045212245
0.10356474787866 0.153834873345794
0.0978725516915446 0.144312598005936
0.0925952928735381 0.134422444219678
0.0877364090495248 0.124189590767097
0.0832990643281257 0.113640102728524
0.0792861460924091 0.102800861916777
0.0757002628080311 0.0916994956001386
0.0725437425870994 0.0803643038046822
0.069818632284583 0.0688241854024739
0.0675266969522104 0.0571085631419612
0.0656694195214907 0.0452473077513642
0.0642480006271671 0.0332706612373471
0.0632633585132331 0.0212091595029022
0.0627161289858991 0.00909355441514592
0.0626066653929575 -0.00304526453801165
0.0629350386185861 -0.0151763488558262
0.0637010370885092 -0.0272687697454203
0.0649041667842718 -0.0392916968007361
0.0665436512687222 -0.0512144764858195
0.0686184317292081 -0.0630067102720941
0.0711271670520408 -0.0746383322854955
0.0740682339532131 -0.0860796863272447
0.0774397272079407 -0.0973016021399556
0.0812394600470219 -0.108275470796062
0.0854649648223936 -0.118973319084073
0.0901134940873028 -0.12936788275431
0.0951820222852561 -0.139432678452301
0.100667248289028 -0.149142074107302
0.10656559906303 -0.158471357449301
0.112873234718821 -0.167396802199979
0.119586055168304 -0.175895731332128
0.126699708426136 -0.183946576646328
0.13420960035749 -0.191528933821879
0.142110905322839 -0.198623612126504
0.150398576795015 -0.205212678179984
0.159067356720643 -0.211279493592426
0.16811178230224 -0.216808746901301
0.177526189102462 -0.221786480886398
0.187304709947427 -0.226200116851138
0.197441269929379 -0.230038477621279
0.207929578656845 -0.233291810717099
0.218763121506754 -0.235951812451675
0.229935151803763 -0.238011652801433
0.241438685552319 -0.239466000064469
0.253266499708268 -0.240311043799429
0.265411134226006 -0.240544514421553
0.277864897472831 -0.240165698073282
0.290619874197545 -0.23917544583971
0.303667935101709 -0.237576176879925
0.317000747134867 -0.235371875470324
0.330609783830706 -0.232568082244952
0.344486335235128 -0.229171880066449
0.358621517191145 -0.225191874998346
0.373006279910045 -0.220638172815656
0.387631415867 -0.215522351422736
0.402487567119295 -0.209857429471725
0.417565232169048 -0.203657831407191
0.432854772492479 -0.196939349109845
0.448346418844808 -0.189719100275549
0.46403027743119 -0.182015483643688
0.47989633601438 -0.173848131178146
0.495934470011695 -0.165237857301697
0.512134448618413 -0.156206605287707
0.528485940982374 -0.146777390919585
0.544978522444945 -0.136974243536887
0.561601680856407 -0.126822144596177
0.578344822968638 -0.116346963884112
0.595197280904322 -0.105575393529208
0.61214831869946 -0.094534879967153
0.629187138914278 -0.0832535540222168
0.646302889306651 -0.071760159274263
0.663484669561462 -0.0600839788869683
0.680721538069075 -0.0482547610782518
0.698002518745858 -0.0363026434185232
0.715316607889725 -0.0242580761462783
0.732652781063624 -0.012151744693772
};
\addplot [line width = \linewidthEightC, color = reference, opacity=\opacityRef, forget plot]
table {%
0.75 0
0.74994432926178 -0.000166334211826324
0.753935277462006 0.00283723324537277
0.770404279232025 0.0149831548333168
0.78904664516449 0.0281670168042183
0.806044340133667 0.039854995906353
0.823701798915863 0.0522806420922279
0.841057777404785 0.0643094852566719
0.858526408672333 0.0762001350522041
0.876199305057526 0.0882692411541939
0.892983734607697 0.0992692336440086
0.910333812236786 0.110677443444729
0.927153289318085 0.121703825891018
0.944168508052826 0.132359005510807
0.961075127124786 0.142425455152988
0.978095352649689 0.152651719748974
0.994906604290009 0.16217178851366
1.01074451208115 0.170405946671963
1.02702015638351 0.179087929427624
1.04325073957443 0.187124483287334
1.05827730894089 0.19431184977293
1.0739694237709 0.20128195732832
1.09013015031815 0.207949720323086
1.10533159971237 0.213621370494366
1.11983090639114 0.21866213530302
1.13447171449661 0.222801201045513
1.14972299337387 0.227280639111996
1.16456073522568 0.230917803943157
1.17775124311447 0.23370536416769
1.1916041970253 0.235892169177532
1.20484119653702 0.237503610551357
1.21740525960922 0.238110713660717
1.23007994890213 0.238461785018444
1.24286490678787 0.238067917525768
1.2549107670784 0.237415336072445
1.2663602232933 0.235883049666882
1.27790719270706 0.23352562636137
1.28905838727951 0.230731301009655
1.29955416917801 0.227283649146557
1.30974477529526 0.223074428737164
1.31968730688095 0.218446396291256
1.32918268442154 0.213073782622814
1.33815842866898 0.207574240863323
1.34685033559799 0.201239757239819
1.35499578714371 0.1944545134902
1.36300760507584 0.186870269477367
1.37023311853409 0.179431103169918
1.37765151262283 0.171053700149059
1.38413971662521 0.162081755697727
1.39034861326218 0.152603410184383
1.3962305188179 0.143406219780445
1.40142267942429 0.133027039468288
1.40676540136337 0.122541211545467
1.41144436597824 0.111877985298634
1.41548377275467 0.101018093526363
1.41918116807938 0.089299164712429
1.42237991094589 0.0774164870381355
1.42512410879135 0.0660805627703667
1.4272808432579 0.0541270449757576
1.42896419763565 0.0423922017216682
1.43028861284256 0.0296073108911514
1.43114739656448 0.0177308171987534
1.4315145611763 0.00536339730024338
1.43168085813522 -0.00748487561941147
1.43115895986557 -0.0195490494370461
1.43010777235031 -0.0316851027309895
1.42894560098648 -0.0443805605173111
1.42721599340439 -0.0567844118922949
1.42492002248764 -0.0685225650668144
1.42243081331253 -0.0805330015718937
1.41954177618027 -0.0920580420643091
1.41587942838669 -0.103544052690268
1.41195994615555 -0.114880666136742
1.40781611204147 -0.12628136947751
1.40307909250259 -0.13672836124897
1.39761513471603 -0.147296648472548
1.39186412096024 -0.156877741217613
1.38564711809158 -0.166725970804691
1.37938731908798 -0.17576852440834
1.37248557806015 -0.184454269707203
1.36527913808823 -0.192390285432339
1.35750836133957 -0.200238257646561
1.34946399927139 -0.207212962210178
1.34102362394333 -0.21371752768755
1.33201438188553 -0.220295898616314
1.32313448190689 -0.225215457379818
1.3132603764534 -0.230485551059246
1.30337792634964 -0.234798051416874
1.29270035028458 -0.238263078033924
1.28211361169815 -0.241624899208546
1.27084022760391 -0.244272939860821
1.25928467512131 -0.245921723544598
1.24739533662796 -0.247300319373608
1.23485642671585 -0.248202092945576
1.22246080636978 -0.248206861317158
1.20979708433151 -0.247285477817059
1.19653874635696 -0.246194399893284
1.18313401937485 -0.244700558483601
1.16926056146622 -0.241882018744946
1.15537041425705 -0.238649733364582
1.14113193750381 -0.234581299126148
1.12650662660599 -0.230404160916805
1.11177641153336 -0.225347198545933
1.09661990404129 -0.219668917357922
1.08153146505356 -0.213638119399548
1.06614059209824 -0.206863433122635
1.05052322149277 -0.199726574122906
1.03492420911789 -0.192164085805416
1.01910680532455 -0.18372705578804
1.00289195775986 -0.174707897007465
0.986519277095795 -0.165787495672703
0.970188677310944 -0.156013078987598
0.953298628330231 -0.146054334938526
0.936546146869659 -0.135369502007961
0.919784486293793 -0.124420743435621
0.902825057506561 -0.113197026774287
0.885461151599884 -0.101637072861195
0.86795711517334 -0.089991606771946
0.851056575775146 -0.0778366941958666
0.833344638347626 -0.0660912226885557
0.815872490406036 -0.0536787360906601
0.798582434654236 -0.0412196889519691
0.780572652816772 -0.0285150408744812
0.76326721906662 -0.0158147588372231
0.745602786540985 -0.00325313955545425
0.728276312351227 0.00983588397502899
0.710568070411682 0.0222982466220856
0.693021774291992 0.0353532060980797
0.67568302154541 0.0474314913153648
0.65825754404068 0.059855617582798
0.640643715858459 0.0723982527852058
0.623472452163696 0.084489606320858
0.606347382068634 0.097026489675045
0.589189529418945 0.10827261954546
0.571917712688446 0.120053984224796
0.555027306079865 0.131161995232105
0.53807657957077 0.142386771738529
0.521364212036133 0.153095655143261
0.504639804363251 0.163210101425648
0.488102793693542 0.172890387475491
0.471790432929993 0.182386688888073
0.455429077148438 0.19140937179327
0.439331352710724 0.199787758290768
0.423658907413483 0.207658253610134
0.407796263694763 0.215132586658001
0.392226099967957 0.222176067531109
0.376782953739166 0.228691570460796
0.361459612846375 0.234801463782787
0.346244245767593 0.240281634032726
0.33129471540451 0.244941167533398
0.317037135362625 0.249164871871471
0.302640080451965 0.252743028104305
0.288818210363388 0.255752943456173
0.275267064571381 0.258094660937786
0.261905193328857 0.259720049798489
0.248762667179108 0.260827593505383
0.236248970031738 0.260936103761196
0.223743051290512 0.260814182460308
0.211264759302139 0.260047994554043
0.199773609638214 0.258645616471767
0.188214153051376 0.256147526204586
0.177141040563583 0.253717087209225
0.166237264871597 0.250410966575146
0.155986726284027 0.246233932673931
0.145887032151222 0.241750173270702
0.136131972074509 0.236341021955013
0.126998364925385 0.230802200734615
0.118628978729248 0.224339567124844
0.110184520483017 0.217792086303234
0.102257043123245 0.210105232894421
0.0946383476257324 0.202293954789639
0.0877418965101242 0.194342367351055
0.0810320228338242 0.18547510355711
0.074737474322319 0.176204197108746
0.0689598470926285 0.166614376008511
0.0632151216268539 0.156314827501774
0.0583051592111588 0.146668113768101
0.0536421090364456 0.135719560086727
0.0494255870580673 0.124604053795338
0.0457631945610046 0.113142006099224
0.042590394616127 0.101627208292484
0.0397409349679947 0.0899434164166451
0.0372330844402313 0.0778988227248192
0.0351645946502686 0.0659013465046883
0.0336356610059738 0.0531866922974586
0.032712996006012 0.0406075939536095
0.0320702642202377 0.0283075049519539
0.0321088284254074 0.0153782740235329
0.0323068201541901 0.00325240939855576
0.0330123901367188 -0.00914529711008072
0.0344112515449524 -0.0215482264757156
0.0360832065343857 -0.0337924025952816
0.0380750000476837 -0.0456532128155231
0.0408352464437485 -0.0576257482171059
0.0438751876354218 -0.0690968595445156
0.0472944676876068 -0.0805404130369425
0.0513169318437576 -0.092033757828176
0.0556735694408417 -0.103158995509148
0.0606140345335007 -0.113926868885756
0.0657841265201569 -0.124341312795877
0.0713877826929092 -0.134162433445454
0.0774706453084946 -0.143869273364544
0.0839152336120605 -0.152800723910332
0.091064840555191 -0.161865599453449
0.0983286499977112 -0.169945932924747
0.106246069073677 -0.177814103662968
0.114152923226357 -0.184662215411663
0.122761845588684 -0.191653080284595
0.131499007344246 -0.197731085121632
0.140761092305183 -0.203329168260098
0.150445654988289 -0.208629064261913
0.16038253903389 -0.212980963289738
0.170757949352264 -0.216676615178585
0.181652188301086 -0.2200693115592
0.192489683628082 -0.222411744296551
0.204196870326996 -0.22471084445715
0.216065227985382 -0.226137958467007
0.227964967489243 -0.226941086351871
0.24039351940155 -0.226724363863468
0.25292494893074 -0.226139031350613
0.266271829605103 -0.225206173956394
0.279626220464706 -0.222946129739285
0.293174088001251 -0.220822669565678
0.307278901338577 -0.217563696205616
0.321228891611099 -0.213353894650936
0.335544556379318 -0.208859749138355
0.349918603897095 -0.203554853796959
0.364793956279755 -0.197840712964535
0.379798114299774 -0.191569909453392
0.394831538200378 -0.185489609837532
0.410400390625 -0.177325583994389
0.425804436206818 -0.169442176818848
0.441570401191711 -0.160985596477985
0.457581520080566 -0.151443421840668
0.473621368408203 -0.142401039600372
0.490110814571381 -0.132344640791416
0.506330668926239 -0.121754456311464
0.522758185863495 -0.110714176669717
0.539553701877594 -0.0995809752494097
0.55608081817627 -0.0882964693009853
0.572792887687683 -0.0762261161580682
0.589311063289642 -0.06474594399333
0.606237709522247 -0.0522401556372643
0.623142778873444 -0.0397425815463066
0.640331625938416 -0.026921771466732
0.657269835472107 -0.0135484784841537
0.674160122871399 -0.000157624483108521
0.691278874874115 0.0128285363316536
0.708619773387909 0.0263111293315887
0.725998818874359 0.0395679324865341
};
\addplot [semithick, red, dashed, forget plot]
table {%
0.75 0
0.767617956255685 0.0126152320166048
0.785221863675294 0.025202759744927
0.802800914680889 0.0377306780229099
0.820344353102624 0.0501672114827695
0.837841479099983 0.0624807937495562
0.855281653865309 0.0746401456242545
0.872654304077727 0.0866143520292023
0.889948926065816 0.0983729374930471
0.90715508962574 0.109885939954111
0.924262441427357 0.121123982665618
0.941260707923679 0.132058343994744
0.958139697658408 0.142661024921306
0.974889302841815 0.152904814062882
0.991499500036706 0.162763350083898
1.00796034976389 0.172211181389763
1.02426199480132 0.18122382306779
1.04039465691463 0.189777811119373
1.0563486317231 0.197850754138649
1.07211428137931 0.205421382738366
1.08768202473299 0.212469597210257
1.10304232467243 0.218976514139914
1.11818567241108 0.224924512976024
1.1331025686362 0.230297283873765
1.147783501695 0.23507987847157
1.16221892339532 0.239258765576722
1.17639922357808 0.242821893954846
1.19031470539574 0.245758764427774
1.20395556419321 0.248060513126141
1.21731187395941 0.249720006825457
1.23037358633393 0.250731949622618
1.24313054782987 0.251092997654816
1.2555725408717 0.250801875167454
1.26768935298946 0.24985948134918
1.27947087567813 0.248268973729967
1.29090722994702 0.246035811751358
1.30198890990496 0.243167744731957
1.31270692999354 0.2396747329547
1.32305295737637 0.235568799170581
1.33301941026506 0.230863819194334
1.34259950666846 0.225575271775738
1.3517872558846 0.219719976188812
1.36057739518197 0.213315848229081
1.36896528374308 0.206381700652654
1.37694677240494 0.198937103974783
1.38451806954475 0.191002311131823
1.39167562078374 0.182598238316676
1.39841601450881 0.173746486857157
1.4047359185686 0.164469388195659
1.41063204764278 0.154790055213092
1.41610115672778 0.144732426840967
1.42114005415556 0.134321297531182
1.42574562723652 0.123582327489247
1.42991487441761 0.112542032954869
1.43364493919855 0.101227758043132
1.43693314251624 0.089667630857217
1.4397770116222 0.0778905070053025
1.44217430452097 0.0659259035746353
1.44412302978876 0.0538039262643269
1.44562146208725 0.0415551919190076
1.44666815398553 0.0292107482394724
1.44726194486432 0.016801992026761
1.44740196775303 0.00436058696240971
1.44708765498222 -0.00808161836067393
1.44631874355061 -0.0204926716768206
1.4450952811249 -0.0328405972734807
1.44341763362016 -0.0450934751015305
1.44128649535024 -0.0572195188465802
1.4387029027798 -0.0691871532341023
1.43566825293285 -0.0809650910678666
1.4321843274821 -0.0925224107662305
1.42825332340439 -0.103828635462315
1.42387789076703 -0.114853815053745
1.41906117760902 -0.125568612881875
1.41380688088891 -0.135944398910193
1.40811930097384 -0.145953351233135
1.40200339507468 -0.155568567310703
1.39546482242266 -0.164764185292558
1.38850997106277 -0.173515513988717
1.38114595342609 -0.181799167392522
1.37338055620066 -0.189593196333819
1.36522213060654 -0.196877206377916
1.35667941319178 -0.203632448466232
1.34776127545165 -0.209841868311401
1.33847641259836 -0.215490103434114
1.328832995749 -0.220563423487838
1.3188383241811 -0.225049619377379
1.30849852098817 -0.228937857346307
1.29781831326193 -0.232218522442798
1.28680092623794 -0.234883078640104
1.27544810236402 -0.23692396914816
1.26376023613114 -0.238334571253006
1.25173659926596 -0.239109208409727
1.23937562223503 -0.239243211851919
1.22680212890194 -0.238807395220756
1.21393543612189 -0.237739020519201
1.2007824030094 -0.236042071113405
1.1873502195155 -0.233721342391173
1.17364639818837 -0.230782623136066
1.15967876862552 -0.227232843837933
1.14545546822952 -0.223080191387264
1.13098492732646 -0.218334192018305
1.11627584919941 -0.213005765794969
1.10133718666704 -0.207107256561536
1.08617811704058 -0.200652441341106
1.070808017043 -0.19365652287622
1.05523643885516 -0.186136108539406
1.03947308802467 -0.178109178316594
1.02352780361101 -0.169595044057188
1.00741054066936 -0.160614301730894
0.991131354992172 -0.151188778049092
0.974700389916618 -0.141341472499412
0.958127864949596 -0.131096495600034
0.941424065943487 -0.120479003995558
0.924599336561409 -0.109515132878708
0.90766407079076 -0.0982319261219058
0.890628706290982 -0.086657264431339
0.873503718391248 -0.074819791786618
0.856299614583182 -0.0627488403957748
0.839026929381251 -0.0504743543738914
0.8216962194482 -0.0380268123406879
0.804318058904672 -0.0254371491254941
0.786903034760972 -0.0127366767652624
0.769461742425004 4.29950186324069e-05
0.752004781254135 0.0128700386771497
0.734542750130382 0.0257124888331812
0.717086243048227 0.0385383222295833
0.699645844712938 0.0513155374272836
0.682232126154625 0.0640122340511521
0.664855640369735 0.0765966913776288
0.647526918007365 0.0890374460548427
0.630256463122704 0.101303368743397
0.613054749024055 0.113363739464377
0.595932214243 0.125188321440869
0.578899258658929 0.136747433220813
0.561966239808458 0.148012018873052
0.545143469406046 0.158953716055882
0.528441210092295 0.169544921769502
0.511869672407924 0.179758855622043
0.495439011959919 0.189569620465462
0.479159326695086 0.198952260295039
0.463040654116175 0.207882815357607
0.447092968153461 0.216338374482455
0.431326175222446 0.224297124738732
0.41575010873165 0.231738398637223
0.400374520922108 0.238642719233979
0.38520907038282 0.244991843656404
0.37026330284855 0.250768805749989
0.355546621900637 0.255957958716022
0.341068244918601 0.260545018739871
0.326837138059129 0.264517110634022
0.312861922218046 0.267862816346776
0.299150740017787 0.270572226688282
0.285711072187049 0.272636995643726
0.272549490830637 0.274050395018263
0.259671337879835 0.274807364778616
0.247080320574249 0.274904551350881
0.234778023333987 0.274340322593555
0.22282111677237 0.273147174350289
0.211212000441657 0.271312180989548
0.199956090718924 0.268842225658862
0.18905951088657 0.265745762806622
0.178528814558833 0.262032850875418
0.168370684504055 0.257715152729482
0.158591652631679 0.252805894649178
0.149197881311021 0.247319783843054
0.140195031151771 0.241272894158711
0.131588220396365 0.234682536784361
0.123382062255884 0.227567134929123
0.115580754025762 0.219946118404367
0.108188187942631 0.211839847297465
0.101208057557953 0.203269566144989
0.0946439418448079 0.19425738357691
0.088499358802492 0.184826268669802
0.0827777883280161 0.17500005435619
0.0774826693237491 0.164803439481848
0.0726173783628224 0.154261983466959
0.0681851974057714 0.143402090111052
0.0641892769319838 0.132250979303424
0.0606325991839522 0.12083664700672
0.0575179445429576 0.109187814856514
0.0548478626398547 0.0973338711778623
0.0526246487503009 0.0853048053160545
0.0508503253165833 0.0731311370573394
0.0495266280133111 0.0608438426868979
0.0486549955532945 0.0484742789692179
0.0482365623412967 0.036054106082026
0.0482721530701452 0.0236152103073717
0.0487622783751904 0.0111896270851532
0.0497071306928156 -0.00119053514168032
0.0511065794918781 -0.0134931700273918
0.0529601650583767 -0.0256862472791051
0.0552670900164109 -0.0377378855745846
0.0580262077740067 -0.0496164247880426
0.061236007111159 -0.0612904981681536
0.0648945922107272 -0.0727291054819839
0.0689996576140302 -0.0839016885812368
0.0735484579177227 -0.0947782113457813
0.0785377725810335 -0.10532924646507
0.0839638670453019 -0.11552607193497
0.0898224525211534 -0.125340780318524
0.0961086482561191 -0.134746403519034
0.10281695173701 -0.143717054765618
0.109941223832714 -0.152228087440402
0.117474696885182 -0.160256267121569
0.125410013591915 -0.167779948892829
0.133739302545796 -0.174779247148552
0.14245429208538 -0.181236180932355
0.151546457798383 -0.187134775860053
0.161007191589854 -0.192461105490089
0.170827973498639 -0.197203261519913
0.181000523658031 -0.201351252889932
0.191516912754772 -0.204896846530735
0.202369615510414 -0.207833373632873
0.213551501760898 -0.210155531524887
0.22505577088336 -0.211859210579139
0.236875844499746 -0.212941368432345
0.249005237340447 -0.213399962707235
0.261437426261433 -0.213233941776159
0.274165733616003 -0.212443283869614
0.28718323530924 -0.211029069576408
0.300482697779478 -0.208993571566475
0.314056543186631 -0.206340347198259
0.327896838839168 -0.203074323164878
0.341995305342881 -0.199201865241865
0.356343337731412 -0.194730829684218
0.370932034454633 -0.189670595476494
0.385752230119493 -0.184032078393973
0.400794528982751 -0.177827728806396
0.416049337197401 -0.171071515546711
0.431506892628147 -0.16377889817583
0.447157291657684 -0.15596678976561
0.462990512824106 -0.147653512015245
0.478996437395876 -0.138858744187699
0.495164867142796 -0.129603467046443
0.511485539633279 -0.119909902709476
0.527948141407569 -0.109801451123518
0.544542319363787 -0.0993026236948103
0.561257690663388 -0.0884389744880547
0.578083851424296 -0.0772370293146803
0.595010384429933 -0.0657242129685941
0.612026866043939 -0.0539287748255556
0.62912287248563 -0.0418797129959582
0.646287985590849 -0.029606697205851
0.663511798156965 -0.0171399905741775
0.680783918949166 -0.00451037045289776
0.698093977427272 0.00825095150093907
0.715431628237671 0.0211124098473716
0.732786555502968 0.0340421663000904
};
\addplot [semithick, green, dash pattern=on 1pt off 3pt on 3pt off 3pt, forget plot]
table {%
0.75 0
0.767347219519284 0.012137253483567
0.784683394455509 0.0242435861667548
0.801997486582194 0.036288155514001
0.819278471531621 0.0482402761288358
0.836515345700615 0.0600694978228124
0.853697133135913 0.0717456830897325
0.870812892392119 0.0832390837898383
0.887851723355235 0.0945204168511733
0.904802774024683 0.105560938798497
0.921655247246804 0.116332518924018
0.938398407392915 0.126807710918802
0.955021586975309 0.136959822789028
0.971514193195148 0.146762984887382
0.987865714417124 0.156192215896701
1.0040657265673 0.165223486610677
1.02010389945277 0.173833781364724
1.03597000300535 0.182001156979107
1.05165391345605 0.189704799085743
1.06714561945411 0.196925075719347
1.08243522815337 0.203643588062188
1.09751297130037 0.209843218238558
1.1123692113737 0.215508174058654
1.12699444784162 0.220624030609957
1.14137932362429 0.225177768584601
1.1555146318663 0.22915780921071
1.16939132313901 0.232554045621132
1.18300051319534 0.235357870442502
1.19633349138035 0.237562199321834
1.20938172974852 0.23916149003295
1.22213689283924 0.240151756735333
1.23459084790774 0.240530578918402
1.24673567520343 0.24029710459028
1.25856367765532 0.239452047402427
1.27006738911946 0.237997677671498
1.28123958024499 0.235937807669022
1.29207326111455 0.233277772045945
1.30256168017249 0.230024404727089
1.31269831956044 0.226186013888674
1.32247688771958 0.221772356568193
1.33189131078413 0.216794613983565
1.34093572465289 0.211265367835258
1.34960446953272 0.205198576949498
1.35789208820684 0.198609552872807
1.36579332846363 0.191514932670796
1.37330314928634 0.18393264729003
1.38041672978698 0.175881884318464
1.38712947958844 0.167383044631953
1.39343704941026 0.158457693037802
1.39933534088867 0.149128503482261
1.40482051503024 0.139419199628507
1.40988899904863 0.129354491659788
1.41453749160152 0.118960010081206
1.41876296661194 0.108262237150103
1.42256267593454 0.0972884364110782
1.42593415113889 0.086066580678514
1.42887520465437 0.0746252787098322
1.43138393047621 0.0629937007473294
1.43345870458359 0.0512015030695605
1.43509818517756 0.0392787516771522
1.43630131281055 0.0272558452352367
1.43706731045357 0.0151634373994425
1.43739568352819 0.00303235866024866
1.43728621991849 -0.00910646215137196
1.43673898997073 -0.0212220765475008
1.43575434648365 -0.0332835951024502
1.43433292468914 -0.0452602660529673
1.43247564221903 -0.0571215536269705
1.43018369904855 -0.0688372159533007
1.42745857739782 -0.0803773824121825
1.4243020415584 -0.0917126302942692
1.42071613759078 -0.102814060643163
1.41670319280875 -0.113653373158817
1.41226581492764 -0.124202940032363
1.40740689070718 -0.134435878560225
1.40212958387022 -0.144326122339201
1.3964373320371 -0.153848490767307
1.39033384239815 -0.162978756463708
1.38382308587764 -0.171693710079292
1.37690928964889 -0.179971221816331
1.36959692806383 -0.187790298849553
1.36189071236441 -0.195131137800147
1.35379557991389 -0.201975171526458
1.34531668404164 -0.20830510981277
1.33645938581161 -0.214104974062127
1.32722924897007 -0.219360126749639
1.31763203892423 -0.224057297001799
1.30767372588241 -0.22818460402503
1.2973604914176 -0.231731580045289
1.28669873695777 -0.234689193904312
1.27569509230234 -0.237049875620184
1.26435642233521 -0.238807541316647
1.25268983059913 -0.239957617226964
1.24070265912099 -0.240497061156197
1.22840248459827 -0.24042437986202
1.21579711159144 -0.239739641182038
1.20289456363565 -0.238444480231447
1.1897030732081 -0.236542099469318
1.17623107134149 -0.234037262792463
1.16248717745014 -0.230936284032406
1.14848018970572 -0.227247010319085
1.13421907610845 -0.222978800771493
1.11971296626323 -0.218142500920566
1.10497114378721 -0.212750413195198
1.09000303923492 -0.206816263729265
1.07481822341698 -0.200355165686855
1.05942640099553 -0.193383579258294
1.04383740425637 -0.185919268450478
1.02806118697728 -0.177981254778958
1.01210781833129 -0.169589767962932
0.995987476780275 -0.160766193724954
0.979710443928391 -0.151533018802209
0.963287098315684 -0.141913773283855
0.946727909140567 -0.131932970397911
0.930043429905878 -0.121616043880467
0.91324429198767 -0.110989283069269
0.896341198128817 -0.100079765872392
0.879344915861624 -0.0889152897708354
0.862266270864871 -0.0775243010211479
0.845116140261585 -0.0659358222307342
0.827905445864243 -0.0541793784842489
0.810645147374399 -0.042284922204474
0.793346235543765 -0.0302827569353333
0.776019725303823 -0.0182034602382608
0.758676648870971 -0.00607780589600706
0.74132804883419 0.00606331437980999
0.723984971232126 0.0181889695395348
0.706658458626486 0.0302682678858693
0.689359543178601 0.0422704356646973
0.672099239736024 0.0541648953575106
0.654888538936098 0.0659213434775336
0.637738400333409 0.077509827673207
0.620659745558167 0.0889008229448788
0.603663451512546 0.10006530678342
0.586760343612044 0.110974833042994
0.569961189078862 0.121601604364455
0.55327669029407 0.131918542970798
0.536717478214957 0.141899359661781
0.520294105863188 0.151518620841353
0.504017041888231 0.160751813418715
0.487896664208613 0.169575407431921
0.471943253730781 0.177966916251536
0.456166988141264 0.18590495423109
0.440577935762111 0.193369291680399
0.425186049451712 0.200340907046881
0.410001160522753 0.206802036197834
0.395032972635676 0.212736218702091
0.38029105560979 0.218128341010777
0.36578483907538 0.222964676431608
0.351523605870586 0.227232921776542
0.337516485069478 0.23092223053541
0.323772444518238 0.234023242385741
0.310300282763089 0.236528108790522
0.297108620287721 0.238430514364134
0.284205890052545 0.239725693611976
0.27160032745433 0.240410442590784
0.259299960006205 0.240483125024951
0.247312597261469 0.239943672488383
0.2356458217318 0.238793578459958
0.224306981713199 0.237035886402939
0.21330318694315 0.234675172482025
0.202641307788403 0.231717524032126
0.192327978173203 0.228170515287301
0.1823696017639 0.22404318200088
0.172772360193836 0.219345996319791
0.163542221572307 0.214090842618848
0.154684947376684 0.208290994107505
0.146206096155645 0.201961089155653
0.138111023180157 0.195117105712544
0.130404876038488 0.187776332069152
0.12309258692145 0.179957332525672
0.116178862798813 0.171679907117356
0.109668174793245 0.162965045212245
0.10356474787866 0.153834873345794
0.0978725516915446 0.144312598005936
0.0925952928735381 0.134422444219678
0.0877364090495248 0.124189590767097
0.0832990643281257 0.113640102728524
0.0792861460924091 0.102800861916777
0.0757002628080311 0.0916994956001386
0.0725437425870994 0.0803643038046822
0.069818632284583 0.0688241854024739
0.0675266969522104 0.0571085631419612
0.0656694195214907 0.0452473077513642
0.0642480006271671 0.0332706612373471
0.0632633585132331 0.0212091595029022
0.0627161289858991 0.00909355441514592
0.0626066653929575 -0.00304526453801165
0.0629350386185861 -0.0151763488558262
0.0637010370885092 -0.0272687697454203
0.0649041667842718 -0.0392916968007361
0.0665436512687222 -0.0512144764858195
0.0686184317292081 -0.0630067102720941
0.0711271670520408 -0.0746383322854955
0.0740682339532131 -0.0860796863272447
0.0774397272079407 -0.0973016021399556
0.0812394600470219 -0.108275470796062
0.0854649648223936 -0.118973319084073
0.0901134940873028 -0.12936788275431
0.0951820222852561 -0.139432678452301
0.100667248289028 -0.149142074107302
0.10656559906303 -0.158471357449301
0.112873234718821 -0.167396802199979
0.119586055168304 -0.175895731332128
0.126699708426136 -0.183946576646328
0.13420960035749 -0.191528933821879
0.142110905322839 -0.198623612126504
0.150398576795015 -0.205212678179984
0.159067356720643 -0.211279493592426
0.16811178230224 -0.216808746901301
0.177526189102462 -0.221786480886398
0.187304709947427 -0.226200116851138
0.197441269929379 -0.230038477621279
0.207929578656845 -0.233291810717099
0.218763121506754 -0.235951812451675
0.229935151803763 -0.238011652801433
0.241438685552319 -0.239466000064469
0.253266499708268 -0.240311043799429
0.265411134226006 -0.240544514421553
0.277864897472831 -0.240165698073282
0.290619874197545 -0.23917544583971
0.303667935101709 -0.237576176879925
0.317000747134867 -0.235371875470324
0.330609783830706 -0.232568082244952
0.344486335235128 -0.229171880066449
0.358621517191145 -0.225191874998346
0.373006279910045 -0.220638172815656
0.387631415867 -0.215522351422736
0.402487567119295 -0.209857429471725
0.417565232169048 -0.203657831407191
0.432854772492479 -0.196939349109845
0.448346418844808 -0.189719100275549
0.46403027743119 -0.182015483643688
0.47989633601438 -0.173848131178146
0.495934470011695 -0.165237857301697
0.512134448618413 -0.156206605287707
0.528485940982374 -0.146777390919585
0.544978522444945 -0.136974243536887
0.561601680856407 -0.126822144596177
0.578344822968638 -0.116346963884112
0.595197280904322 -0.105575393529208
0.61214831869946 -0.094534879967153
0.629187138914278 -0.0832535540222168
0.646302889306651 -0.071760159274263
0.663484669561462 -0.0600839788869683
0.680721538069075 -0.0482547610782518
0.698002518745858 -0.0363026434185232
0.715316607889725 -0.0242580761462783
0.732652781063624 -0.012151744693772
};
\addplot [line width = \linewidthEightC, color = reference, opacity=\opacityRef, forget plot]
table {%
0.75 0
0.752686679363251 0.00235426425933838
0.767670333385468 0.0133496597409248
0.786537766456604 0.0267196372151375
0.803400158882141 0.0386031270027161
0.821255743503571 0.0511330738663673
0.838517844676971 0.0635882243514061
0.856337249279022 0.0757034495472908
0.873269081115723 0.0874350592494011
0.890589773654938 0.0990615412592888
0.907520532608032 0.110253237187862
0.924994587898254 0.121761597692966
0.941486239433289 0.132419683039188
0.958290219306946 0.142861165106297
0.974858164787292 0.15283726900816
0.991547703742981 0.162405170500278
1.00806367397308 0.171405978500843
1.02423202991486 0.180502958595753
1.04034316539764 0.188760586082935
1.05626118183136 0.196513392031193
1.07211327552795 0.20373473316431
1.08774638175964 0.210451431572437
1.10312795639038 0.216512151062489
1.11801135540009 0.222008652985096
1.13273441791534 0.226749546825886
1.14711570739746 0.231003917753696
1.16137742996216 0.234781242907047
1.17524087429047 0.237947799265385
1.18889129161835 0.24069169908762
1.20245897769928 0.242416836321354
1.21567189693451 0.243791706860065
1.22862708568573 0.244385875761509
1.24089992046356 0.244396813213825
1.25317132472992 0.243681021034718
1.26483738422394 0.242394007742405
1.27663290500641 0.240455217659473
1.28755235671997 0.237653858959675
1.29806041717529 0.23467306047678
1.30852603912354 0.230556763708591
1.31862282752991 0.22671053558588
1.32824444770813 0.221714593470097
1.33759820461273 0.216147132217884
1.34631097316742 0.209935791790485
1.3549325466156 0.203374870121479
1.3629949092865 0.196323372423649
1.37058234214783 0.188384093344212
1.37859618663788 0.180449552834034
1.38553392887115 0.171440400183201
1.39200556278229 0.162800751626492
1.39834153652191 0.153251580893993
1.40418684482574 0.143166549503803
1.40940082073212 0.132794193923473
1.4142951965332 0.122248820960522
1.41814649105072 0.111823417246342
1.42238831520081 0.100423164665699
1.42601811885834 0.0886774137616158
1.42906606197357 0.0772930011153221
1.43168139457703 0.0652098581194878
1.43376874923706 0.0528533980250359
1.43536412715912 0.0411625131964684
1.43653655052185 0.0292599573731422
1.43732893466949 0.0167915895581245
1.43772065639496 0.00439900904893875
1.43756711483002 -0.00814098864793777
1.43724942207336 -0.0199725367128849
1.43609702587128 -0.0319916754961014
1.43477487564087 -0.0443665161728859
1.43293857574463 -0.0565433409065008
1.43055438995361 -0.0687530315481126
1.42784953117371 -0.0810887711122632
1.42474567890167 -0.0926464758813381
1.42096948623657 -0.103788275271654
1.41690099239349 -0.114722069352865
1.41254913806915 -0.125694192945957
1.40731763839722 -0.136380888521671
1.40208148956299 -0.146459884941578
1.39629149436951 -0.156266152858734
1.38991141319275 -0.165456727147102
1.38355433940887 -0.174627237021923
1.37636303901672 -0.183165408670902
1.3687801361084 -0.191342107951641
1.36086821556091 -0.199265129864216
1.35271430015564 -0.20622306317091
1.34377288818359 -0.21287664026022
1.33461976051331 -0.219120882451534
1.32502388954163 -0.224676676094532
1.31490063667297 -0.229497544467449
1.30500841140747 -0.233559809625149
1.29462742805481 -0.237047962844372
1.28383672237396 -0.239928387105465
1.27291882038116 -0.242387391626835
1.26127457618713 -0.243994019925594
1.24930024147034 -0.245046846568584
1.23733341693878 -0.245356060564518
1.22496140003204 -0.245314888656139
1.21189093589783 -0.244496338069439
1.19872117042542 -0.243042834103107
1.18537998199463 -0.240726225078106
1.17175710201263 -0.237954966723919
1.15800404548645 -0.234514214098454
1.14398145675659 -0.230662249028683
1.1294983625412 -0.225938893854618
1.11489951610565 -0.220804147422314
1.10013008117676 -0.215139500796795
1.08515095710754 -0.208704061806202
1.06996011734009 -0.201751194894314
1.05452632904053 -0.193914704024792
1.0389575958252 -0.185809589922428
1.02314424514771 -0.177212871611118
1.00691783428192 -0.168294697999954
0.990699529647827 -0.15866881608963
0.974381685256958 -0.148620255291462
0.957812428474426 -0.138222612440586
0.94119656085968 -0.127475712448359
0.924859642982483 -0.116460800170898
0.907824993133545 -0.1049994379282
0.890980660915375 -0.093344496563077
0.873949229717255 -0.081390799023211
0.85678768157959 -0.0689826253801584
0.83961033821106 -0.0566053576767445
0.82206666469574 -0.0433554667979479
0.804952442646027 -0.0311817936599255
0.787632167339325 -0.0181028805673122
0.770071089267731 -0.00502180308103561
0.752668857574463 0.00789569318294525
0.735328614711761 0.0206034258008003
0.718165218830109 0.0337625369429588
0.700918436050415 0.0468962714076042
0.683494806289673 0.0596886649727821
0.666327595710754 0.0728030577301979
0.649206101894379 0.0853240713477135
0.632150769233704 0.0978515669703484
0.614963352680206 0.1098478063941
0.5980104804039 0.122455753386021
0.580920875072479 0.133839346468449
0.564331829547882 0.145670972764492
0.547769784927368 0.156755320727825
0.531213283538818 0.167551688849926
0.514614045619965 0.178133226931095
0.498177528381348 0.188403077423573
0.482008397579193 0.198174543678761
0.465974032878876 0.207387335598469
0.449828803539276 0.216157980263233
0.434050381183624 0.224419362843037
0.41853791475296 0.231931127607822
0.402718544006348 0.239552386105061
0.387763112783432 0.246277429163456
0.372560203075409 0.252571232616901
0.357657700777054 0.258033074438572
0.343278616666794 0.262749470770359
0.328878194093704 0.267256982624531
0.31453076004982 0.271106071770191
0.300446301698685 0.274414010345936
0.286946684122086 0.276887841522694
0.273630112409592 0.278736926615238
0.260394096374512 0.279879726469517
0.247474372386932 0.280658222734928
0.234654009342194 0.280606962740421
0.222812592983246 0.279926516115665
0.211037814617157 0.278748966753483
0.199756532907486 0.276984311640263
0.188590407371521 0.274280019104481
0.178073018789291 0.271235145628452
0.167202830314636 0.267319150269032
0.157322406768799 0.262999601662159
0.147683203220367 0.257979936897755
0.138390734791756 0.252254314720631
0.129624933004379 0.24619997292757
0.12097355723381 0.239306576550007
0.112968489527702 0.232115544378757
0.105211690068245 0.224639721214771
0.0979541838169098 0.216474689543247
0.0912751704454422 0.207759447395802
0.0849283784627914 0.19846535474062
0.0787024348974228 0.189289577305317
0.0731927454471588 0.179212011396885
0.0679395347833633 0.169060908257961
0.0629898011684418 0.157890252768993
0.058830127120018 0.146967299282551
0.0548283755779266 0.135777734220028
0.0514572560787201 0.124038092792034
0.0484348684549332 0.112685658037663
0.0457364022731781 0.100700654089451
0.0436145216226578 0.08807023614645
0.0418326705694199 0.07644372433424
0.0410454273223877 0.0635270997881889
0.0402086973190308 0.051710419356823
0.0400756150484085 0.0392010286450386
0.0401783436536789 0.0263507887721062
0.0407700389623642 0.0137149468064308
0.0418418347835541 0.00169689953327179
0.0432973057031631 -0.0107914730906487
0.0454006046056747 -0.0228959955275059
0.047787070274353 -0.0346193909645081
0.0509209483861923 -0.0465610399842262
0.0541141480207443 -0.0579825732856989
0.0579249560832977 -0.0696671376936138
0.0620927214622498 -0.0803492600098252
0.0665475726127625 -0.0909734610468149
0.0718906670808792 -0.101396158337593
0.0773633271455765 -0.111415222287178
0.0835018754005432 -0.121079102158546
0.0898781567811966 -0.130671795457602
0.0965224057435989 -0.139466889202595
0.103709891438484 -0.147927924990654
0.111652716994286 -0.155932299792767
0.11977057158947 -0.163232512772083
0.128207996487617 -0.17029046267271
0.136961221694946 -0.176543585956097
0.14613188803196 -0.182201437652111
0.155357033014297 -0.187255084514618
0.165519773960114 -0.192200146615505
0.17556956410408 -0.195779412984848
0.186401903629303 -0.199235163629055
0.197603762149811 -0.202370591461658
0.20885095000267 -0.204613707959652
0.220463573932648 -0.20547292381525
0.232657909393311 -0.206516869366169
0.245303213596344 -0.207026980817318
0.257891327142715 -0.206148646771908
0.270581066608429 -0.205118484795094
0.284099996089935 -0.20337850600481
0.2978335916996 -0.201090894639492
0.311456918716431 -0.197557009756565
0.325618237257004 -0.194303721189499
0.339774996042252 -0.189846903085709
0.354513615369797 -0.184844568371773
0.369086861610413 -0.179637357592583
0.384562432765961 -0.173162005841732
0.3996941447258 -0.166349031031132
0.41553121805191 -0.159198015928268
0.431342899799347 -0.151182860136032
0.447285056114197 -0.143600031733513
0.463203012943268 -0.134173586964607
0.479315400123596 -0.124991118907928
0.495656967163086 -0.114947110414505
0.512104332447052 -0.105036057531834
0.528763592243195 -0.0944466274231672
0.54542738199234 -0.0835790075361729
0.562080800533295 -0.0723324231803417
0.578933596611023 -0.0605102032423019
0.595979154109955 -0.0488610602915287
0.612966775894165 -0.0362044461071491
0.630474328994751 -0.0241931676864624
0.647192239761353 -0.0113899707794189
0.664496660232544 0.0014713779091835
0.681457936763763 0.0142377987504005
0.699226260185242 0.0273034945130348
0.716439008712769 0.0399238094687462
0.733730912208557 0.0531855598092079
};
\addplot [semithick, red, dashed, forget plot]
table {%
0.75 0
0.767617956255685 0.0126152320166048
0.785221863675294 0.025202759744927
0.802800914680889 0.0377306780229099
0.820344353102624 0.0501672114827695
0.837841479099983 0.0624807937495562
0.855281653865309 0.0746401456242545
0.872654304077727 0.0866143520292023
0.889948926065816 0.0983729374930471
0.90715508962574 0.109885939954111
0.924262441427357 0.121123982665618
0.941260707923679 0.132058343994744
0.958139697658408 0.142661024921306
0.974889302841815 0.152904814062882
0.991499500036706 0.162763350083898
1.00796034976389 0.172211181389763
1.02426199480132 0.18122382306779
1.04039465691463 0.189777811119373
1.0563486317231 0.197850754138649
1.07211428137931 0.205421382738366
1.08768202473299 0.212469597210257
1.10304232467243 0.218976514139914
1.11818567241108 0.224924512976024
1.1331025686362 0.230297283873765
1.147783501695 0.23507987847157
1.16221892339532 0.239258765576722
1.17639922357808 0.242821893954846
1.19031470539574 0.245758764427774
1.20395556419321 0.248060513126141
1.21731187395941 0.249720006825457
1.23037358633393 0.250731949622618
1.24313054782987 0.251092997654816
1.2555725408717 0.250801875167454
1.26768935298946 0.24985948134918
1.27947087567813 0.248268973729967
1.29090722994702 0.246035811751358
1.30198890990496 0.243167744731957
1.31270692999354 0.2396747329547
1.32305295737637 0.235568799170581
1.33301941026506 0.230863819194334
1.34259950666846 0.225575271775738
1.3517872558846 0.219719976188812
1.36057739518197 0.213315848229081
1.36896528374308 0.206381700652654
1.37694677240494 0.198937103974783
1.38451806954475 0.191002311131823
1.39167562078374 0.182598238316676
1.39841601450881 0.173746486857157
1.4047359185686 0.164469388195659
1.41063204764278 0.154790055213092
1.41610115672778 0.144732426840967
1.42114005415556 0.134321297531182
1.42574562723652 0.123582327489247
1.42991487441761 0.112542032954869
1.43364493919855 0.101227758043132
1.43693314251624 0.089667630857217
1.4397770116222 0.0778905070053025
1.44217430452097 0.0659259035746353
1.44412302978876 0.0538039262643269
1.44562146208725 0.0415551919190076
1.44666815398553 0.0292107482394724
1.44726194486432 0.016801992026761
1.44740196775303 0.00436058696240971
1.44708765498222 -0.00808161836067393
1.44631874355061 -0.0204926716768206
1.4450952811249 -0.0328405972734807
1.44341763362016 -0.0450934751015305
1.44128649535024 -0.0572195188465802
1.4387029027798 -0.0691871532341023
1.43566825293285 -0.0809650910678666
1.4321843274821 -0.0925224107662305
1.42825332340439 -0.103828635462315
1.42387789076703 -0.114853815053745
1.41906117760902 -0.125568612881875
1.41380688088891 -0.135944398910193
1.40811930097384 -0.145953351233135
1.40200339507468 -0.155568567310703
1.39546482242266 -0.164764185292558
1.38850997106277 -0.173515513988717
1.38114595342609 -0.181799167392522
1.37338055620066 -0.189593196333819
1.36522213060654 -0.196877206377916
1.35667941319178 -0.203632448466232
1.34776127545165 -0.209841868311401
1.33847641259836 -0.215490103434114
1.328832995749 -0.220563423487838
1.3188383241811 -0.225049619377379
1.30849852098817 -0.228937857346307
1.29781831326193 -0.232218522442798
1.28680092623794 -0.234883078640104
1.27544810236402 -0.23692396914816
1.26376023613114 -0.238334571253006
1.25173659926596 -0.239109208409727
1.23937562223503 -0.239243211851919
1.22680212890194 -0.238807395220756
1.21393543612189 -0.237739020519201
1.2007824030094 -0.236042071113405
1.1873502195155 -0.233721342391173
1.17364639818837 -0.230782623136066
1.15967876862552 -0.227232843837933
1.14545546822952 -0.223080191387264
1.13098492732646 -0.218334192018305
1.11627584919941 -0.213005765794969
1.10133718666704 -0.207107256561536
1.08617811704058 -0.200652441341106
1.070808017043 -0.19365652287622
1.05523643885516 -0.186136108539406
1.03947308802467 -0.178109178316594
1.02352780361101 -0.169595044057188
1.00741054066936 -0.160614301730894
0.991131354992172 -0.151188778049092
0.974700389916618 -0.141341472499412
0.958127864949596 -0.131096495600034
0.941424065943487 -0.120479003995558
0.924599336561409 -0.109515132878708
0.90766407079076 -0.0982319261219058
0.890628706290982 -0.086657264431339
0.873503718391248 -0.074819791786618
0.856299614583182 -0.0627488403957748
0.839026929381251 -0.0504743543738914
0.8216962194482 -0.0380268123406879
0.804318058904672 -0.0254371491254941
0.786903034760972 -0.0127366767652624
0.769461742425004 4.29950186324069e-05
0.752004781254135 0.0128700386771497
0.734542750130382 0.0257124888331812
0.717086243048227 0.0385383222295833
0.699645844712938 0.0513155374272836
0.682232126154625 0.0640122340511521
0.664855640369735 0.0765966913776288
0.647526918007365 0.0890374460548427
0.630256463122704 0.101303368743397
0.613054749024055 0.113363739464377
0.595932214243 0.125188321440869
0.578899258658929 0.136747433220813
0.561966239808458 0.148012018873052
0.545143469406046 0.158953716055882
0.528441210092295 0.169544921769502
0.511869672407924 0.179758855622043
0.495439011959919 0.189569620465462
0.479159326695086 0.198952260295039
0.463040654116175 0.207882815357607
0.447092968153461 0.216338374482455
0.431326175222446 0.224297124738732
0.41575010873165 0.231738398637223
0.400374520922108 0.238642719233979
0.38520907038282 0.244991843656404
0.37026330284855 0.250768805749989
0.355546621900637 0.255957958716022
0.341068244918601 0.260545018739871
0.326837138059129 0.264517110634022
0.312861922218046 0.267862816346776
0.299150740017787 0.270572226688282
0.285711072187049 0.272636995643726
0.272549490830637 0.274050395018263
0.259671337879835 0.274807364778616
0.247080320574249 0.274904551350881
0.234778023333987 0.274340322593555
0.22282111677237 0.273147174350289
0.211212000441657 0.271312180989548
0.199956090718924 0.268842225658862
0.18905951088657 0.265745762806622
0.178528814558833 0.262032850875418
0.168370684504055 0.257715152729482
0.158591652631679 0.252805894649178
0.149197881311021 0.247319783843054
0.140195031151771 0.241272894158711
0.131588220396365 0.234682536784361
0.123382062255884 0.227567134929123
0.115580754025762 0.219946118404367
0.108188187942631 0.211839847297465
0.101208057557953 0.203269566144989
0.0946439418448079 0.19425738357691
0.088499358802492 0.184826268669802
0.0827777883280161 0.17500005435619
0.0774826693237491 0.164803439481848
0.0726173783628224 0.154261983466959
0.0681851974057714 0.143402090111052
0.0641892769319838 0.132250979303424
0.0606325991839522 0.12083664700672
0.0575179445429576 0.109187814856514
0.0548478626398547 0.0973338711778623
0.0526246487503009 0.0853048053160545
0.0508503253165833 0.0731311370573394
0.0495266280133111 0.0608438426868979
0.0486549955532945 0.0484742789692179
0.0482365623412967 0.036054106082026
0.0482721530701452 0.0236152103073717
0.0487622783751904 0.0111896270851532
0.0497071306928156 -0.00119053514168032
0.0511065794918781 -0.0134931700273918
0.0529601650583767 -0.0256862472791051
0.0552670900164109 -0.0377378855745846
0.0580262077740067 -0.0496164247880426
0.061236007111159 -0.0612904981681536
0.0648945922107272 -0.0727291054819839
0.0689996576140302 -0.0839016885812368
0.0735484579177227 -0.0947782113457813
0.0785377725810335 -0.10532924646507
0.0839638670453019 -0.11552607193497
0.0898224525211534 -0.125340780318524
0.0961086482561191 -0.134746403519034
0.10281695173701 -0.143717054765618
0.109941223832714 -0.152228087440402
0.117474696885182 -0.160256267121569
0.125410013591915 -0.167779948892829
0.133739302545796 -0.174779247148552
0.14245429208538 -0.181236180932355
0.151546457798383 -0.187134775860053
0.161007191589854 -0.192461105490089
0.170827973498639 -0.197203261519913
0.181000523658031 -0.201351252889932
0.191516912754772 -0.204896846530735
0.202369615510414 -0.207833373632873
0.213551501760898 -0.210155531524887
0.22505577088336 -0.211859210579139
0.236875844499746 -0.212941368432345
0.249005237340447 -0.213399962707235
0.261437426261433 -0.213233941776159
0.274165733616003 -0.212443283869614
0.28718323530924 -0.211029069576408
0.300482697779478 -0.208993571566475
0.314056543186631 -0.206340347198259
0.327896838839168 -0.203074323164878
0.341995305342881 -0.199201865241865
0.356343337731412 -0.194730829684218
0.370932034454633 -0.189670595476494
0.385752230119493 -0.184032078393973
0.400794528982751 -0.177827728806396
0.416049337197401 -0.171071515546711
0.431506892628147 -0.16377889817583
0.447157291657684 -0.15596678976561
0.462990512824106 -0.147653512015245
0.478996437395876 -0.138858744187699
0.495164867142796 -0.129603467046443
0.511485539633279 -0.119909902709476
0.527948141407569 -0.109801451123518
0.544542319363787 -0.0993026236948103
0.561257690663388 -0.0884389744880547
0.578083851424296 -0.0772370293146803
0.595010384429933 -0.0657242129685941
0.612026866043939 -0.0539287748255556
0.62912287248563 -0.0418797129959582
0.646287985590849 -0.029606697205851
0.663511798156965 -0.0171399905741775
0.680783918949166 -0.00451037045289776
0.698093977427272 0.00825095150093907
0.715431628237671 0.0211124098473716
0.732786555502968 0.0340421663000904
};
\addplot [semithick, green, dash pattern=on 1pt off 3pt on 3pt off 3pt, forget plot]
table {%
0.75 0
0.767347219519284 0.012137253483567
0.784683394455509 0.0242435861667548
0.801997486582194 0.036288155514001
0.819278471531621 0.0482402761288358
0.836515345700615 0.0600694978228124
0.853697133135913 0.0717456830897325
0.870812892392119 0.0832390837898383
0.887851723355235 0.0945204168511733
0.904802774024683 0.105560938798497
0.921655247246804 0.116332518924018
0.938398407392915 0.126807710918802
0.955021586975309 0.136959822789028
0.971514193195148 0.146762984887382
0.987865714417124 0.156192215896701
1.0040657265673 0.165223486610677
1.02010389945277 0.173833781364724
1.03597000300535 0.182001156979107
1.05165391345605 0.189704799085743
1.06714561945411 0.196925075719347
1.08243522815337 0.203643588062188
1.09751297130037 0.209843218238558
1.1123692113737 0.215508174058654
1.12699444784162 0.220624030609957
1.14137932362429 0.225177768584601
1.1555146318663 0.22915780921071
1.16939132313901 0.232554045621132
1.18300051319534 0.235357870442502
1.19633349138035 0.237562199321834
1.20938172974852 0.23916149003295
1.22213689283924 0.240151756735333
1.23459084790774 0.240530578918402
1.24673567520343 0.24029710459028
1.25856367765532 0.239452047402427
1.27006738911946 0.237997677671498
1.28123958024499 0.235937807669022
1.29207326111455 0.233277772045945
1.30256168017249 0.230024404727089
1.31269831956044 0.226186013888674
1.32247688771958 0.221772356568193
1.33189131078413 0.216794613983565
1.34093572465289 0.211265367835258
1.34960446953272 0.205198576949498
1.35789208820684 0.198609552872807
1.36579332846363 0.191514932670796
1.37330314928634 0.18393264729003
1.38041672978698 0.175881884318464
1.38712947958844 0.167383044631953
1.39343704941026 0.158457693037802
1.39933534088867 0.149128503482261
1.40482051503024 0.139419199628507
1.40988899904863 0.129354491659788
1.41453749160152 0.118960010081206
1.41876296661194 0.108262237150103
1.42256267593454 0.0972884364110782
1.42593415113889 0.086066580678514
1.42887520465437 0.0746252787098322
1.43138393047621 0.0629937007473294
1.43345870458359 0.0512015030695605
1.43509818517756 0.0392787516771522
1.43630131281055 0.0272558452352367
1.43706731045357 0.0151634373994425
1.43739568352819 0.00303235866024866
1.43728621991849 -0.00910646215137196
1.43673898997073 -0.0212220765475008
1.43575434648365 -0.0332835951024502
1.43433292468914 -0.0452602660529673
1.43247564221903 -0.0571215536269705
1.43018369904855 -0.0688372159533007
1.42745857739782 -0.0803773824121825
1.4243020415584 -0.0917126302942692
1.42071613759078 -0.102814060643163
1.41670319280875 -0.113653373158817
1.41226581492764 -0.124202940032363
1.40740689070718 -0.134435878560225
1.40212958387022 -0.144326122339201
1.3964373320371 -0.153848490767307
1.39033384239815 -0.162978756463708
1.38382308587764 -0.171693710079292
1.37690928964889 -0.179971221816331
1.36959692806383 -0.187790298849553
1.36189071236441 -0.195131137800147
1.35379557991389 -0.201975171526458
1.34531668404164 -0.20830510981277
1.33645938581161 -0.214104974062127
1.32722924897007 -0.219360126749639
1.31763203892423 -0.224057297001799
1.30767372588241 -0.22818460402503
1.2973604914176 -0.231731580045289
1.28669873695777 -0.234689193904312
1.27569509230234 -0.237049875620184
1.26435642233521 -0.238807541316647
1.25268983059913 -0.239957617226964
1.24070265912099 -0.240497061156197
1.22840248459827 -0.24042437986202
1.21579711159144 -0.239739641182038
1.20289456363565 -0.238444480231447
1.1897030732081 -0.236542099469318
1.17623107134149 -0.234037262792463
1.16248717745014 -0.230936284032406
1.14848018970572 -0.227247010319085
1.13421907610845 -0.222978800771493
1.11971296626323 -0.218142500920566
1.10497114378721 -0.212750413195198
1.09000303923492 -0.206816263729265
1.07481822341698 -0.200355165686855
1.05942640099553 -0.193383579258294
1.04383740425637 -0.185919268450478
1.02806118697728 -0.177981254778958
1.01210781833129 -0.169589767962932
0.995987476780275 -0.160766193724954
0.979710443928391 -0.151533018802209
0.963287098315684 -0.141913773283855
0.946727909140567 -0.131932970397911
0.930043429905878 -0.121616043880467
0.91324429198767 -0.110989283069269
0.896341198128817 -0.100079765872392
0.879344915861624 -0.0889152897708354
0.862266270864871 -0.0775243010211479
0.845116140261585 -0.0659358222307342
0.827905445864243 -0.0541793784842489
0.810645147374399 -0.042284922204474
0.793346235543765 -0.0302827569353333
0.776019725303823 -0.0182034602382608
0.758676648870971 -0.00607780589600706
0.74132804883419 0.00606331437980999
0.723984971232126 0.0181889695395348
0.706658458626486 0.0302682678858693
0.689359543178601 0.0422704356646973
0.672099239736024 0.0541648953575106
0.654888538936098 0.0659213434775336
0.637738400333409 0.077509827673207
0.620659745558167 0.0889008229448788
0.603663451512546 0.10006530678342
0.586760343612044 0.110974833042994
0.569961189078862 0.121601604364455
0.55327669029407 0.131918542970798
0.536717478214957 0.141899359661781
0.520294105863188 0.151518620841353
0.504017041888231 0.160751813418715
0.487896664208613 0.169575407431921
0.471943253730781 0.177966916251536
0.456166988141264 0.18590495423109
0.440577935762111 0.193369291680399
0.425186049451712 0.200340907046881
0.410001160522753 0.206802036197834
0.395032972635676 0.212736218702091
0.38029105560979 0.218128341010777
0.36578483907538 0.222964676431608
0.351523605870586 0.227232921776542
0.337516485069478 0.23092223053541
0.323772444518238 0.234023242385741
0.310300282763089 0.236528108790522
0.297108620287721 0.238430514364134
0.284205890052545 0.239725693611976
0.27160032745433 0.240410442590784
0.259299960006205 0.240483125024951
0.247312597261469 0.239943672488383
0.2356458217318 0.238793578459958
0.224306981713199 0.237035886402939
0.21330318694315 0.234675172482025
0.202641307788403 0.231717524032126
0.192327978173203 0.228170515287301
0.1823696017639 0.22404318200088
0.172772360193836 0.219345996319791
0.163542221572307 0.214090842618848
0.154684947376684 0.208290994107505
0.146206096155645 0.201961089155653
0.138111023180157 0.195117105712544
0.130404876038488 0.187776332069152
0.12309258692145 0.179957332525672
0.116178862798813 0.171679907117356
0.109668174793245 0.162965045212245
0.10356474787866 0.153834873345794
0.0978725516915446 0.144312598005936
0.0925952928735381 0.134422444219678
0.0877364090495248 0.124189590767097
0.0832990643281257 0.113640102728524
0.0792861460924091 0.102800861916777
0.0757002628080311 0.0916994956001386
0.0725437425870994 0.0803643038046822
0.069818632284583 0.0688241854024739
0.0675266969522104 0.0571085631419612
0.0656694195214907 0.0452473077513642
0.0642480006271671 0.0332706612373471
0.0632633585132331 0.0212091595029022
0.0627161289858991 0.00909355441514592
0.0626066653929575 -0.00304526453801165
0.0629350386185861 -0.0151763488558262
0.0637010370885092 -0.0272687697454203
0.0649041667842718 -0.0392916968007361
0.0665436512687222 -0.0512144764858195
0.0686184317292081 -0.0630067102720941
0.0711271670520408 -0.0746383322854955
0.0740682339532131 -0.0860796863272447
0.0774397272079407 -0.0973016021399556
0.0812394600470219 -0.108275470796062
0.0854649648223936 -0.118973319084073
0.0901134940873028 -0.12936788275431
0.0951820222852561 -0.139432678452301
0.100667248289028 -0.149142074107302
0.10656559906303 -0.158471357449301
0.112873234718821 -0.167396802199979
0.119586055168304 -0.175895731332128
0.126699708426136 -0.183946576646328
0.13420960035749 -0.191528933821879
0.142110905322839 -0.198623612126504
0.150398576795015 -0.205212678179984
0.159067356720643 -0.211279493592426
0.16811178230224 -0.216808746901301
0.177526189102462 -0.221786480886398
0.187304709947427 -0.226200116851138
0.197441269929379 -0.230038477621279
0.207929578656845 -0.233291810717099
0.218763121506754 -0.235951812451675
0.229935151803763 -0.238011652801433
0.241438685552319 -0.239466000064469
0.253266499708268 -0.240311043799429
0.265411134226006 -0.240544514421553
0.277864897472831 -0.240165698073282
0.290619874197545 -0.23917544583971
0.303667935101709 -0.237576176879925
0.317000747134867 -0.235371875470324
0.330609783830706 -0.232568082244952
0.344486335235128 -0.229171880066449
0.358621517191145 -0.225191874998346
0.373006279910045 -0.220638172815656
0.387631415867 -0.215522351422736
0.402487567119295 -0.209857429471725
0.417565232169048 -0.203657831407191
0.432854772492479 -0.196939349109845
0.448346418844808 -0.189719100275549
0.46403027743119 -0.182015483643688
0.47989633601438 -0.173848131178146
0.495934470011695 -0.165237857301697
0.512134448618413 -0.156206605287707
0.528485940982374 -0.146777390919585
0.544978522444945 -0.136974243536887
0.561601680856407 -0.126822144596177
0.578344822968638 -0.116346963884112
0.595197280904322 -0.105575393529208
0.61214831869946 -0.094534879967153
0.629187138914278 -0.0832535540222168
0.646302889306651 -0.071760159274263
0.663484669561462 -0.0600839788869683
0.680721538069075 -0.0482547610782518
0.698002518745858 -0.0363026434185232
0.715316607889725 -0.0242580761462783
0.732652781063624 -0.012151744693772
};
\addplot [line width = \linewidthEightC, color = reference, opacity=\opacityRef, forget plot]
table {%
0.75 0
0.752746224403381 0.00276989489793777
0.767382621765137 0.0137227177619934
0.786403238773346 0.0279699489474297
0.803128898143768 0.039859876036644
0.820545732975006 0.0525600761175156
0.837767422199249 0.0651347115635872
0.854939520359039 0.0771684274077415
0.872315526008606 0.0890500321984291
0.889448583126068 0.101308293640614
0.90601658821106 0.112868495285511
0.922985315322876 0.124406836926937
0.939806580543518 0.135316766798496
0.95695436000824 0.146328337490559
0.973560452461243 0.156819112598896
0.989642739295959 0.166218437254429
1.00593137741089 0.1757587864995
1.02188158035278 0.1847138479352
1.03796017169952 0.193224973976612
1.05377352237701 0.201006181538105
1.069176197052 0.208513028919697
1.08464348316193 0.215218015015125
1.10003590583801 0.221610553562641
1.11496901512146 0.227431751787663
1.12963974475861 0.232371486723423
1.14431118965149 0.237132020294666
1.15853989124298 0.241094835102558
1.17263245582581 0.244369156658649
1.18639957904816 0.247164614498615
1.19983172416687 0.249178506433964
1.21302056312561 0.250674940645695
1.22535514831543 0.25141916424036
1.23799562454224 0.251380987465382
1.25015151500702 0.251093663275242
1.261674284935 0.249999441206455
1.27329838275909 0.248124368488789
1.28430187702179 0.245834447443485
1.29531574249268 0.242987610399723
1.30575084686279 0.23914796859026
1.31571125984192 0.235013820230961
1.32526421546936 0.22985915094614
1.33467245101929 0.224914290010929
1.34365999698639 0.218598134815693
1.35238301753998 0.212121553719044
1.36059284210205 0.205331869423389
1.36880207061768 0.197764046490192
1.37609839439392 0.189546294510365
1.38326466083527 0.181144542992115
1.39028692245483 0.171707890927792
1.39647161960602 0.162569738924503
1.40231430530548 0.152909882366657
1.40769016742706 0.142211906611919
1.41270196437836 0.131715588271618
1.41711854934692 0.121043153107166
1.42111790180206 0.110538564622402
1.42463672161102 0.0982163622975349
1.4277617931366 0.0875211283564568
1.43039166927338 0.0754534676671028
1.4325635433197 0.0636188313364983
1.43431031703949 0.0512158200144768
1.43588709831238 0.0391308888792992
1.43696022033691 0.0265461057424545
1.43735265731812 0.0135792717337608
1.43762767314911 0.00129878520965576
1.4371737241745 -0.0106265433132648
1.43634295463562 -0.0232704542577267
1.43498170375824 -0.0352934785187244
1.43334114551544 -0.0473477076739073
1.4310314655304 -0.0592242917045951
1.42862939834595 -0.0714223217219114
1.42558038234711 -0.0828465316444635
1.42207765579224 -0.0942854937165976
1.41817355155945 -0.105401888489723
1.41376042366028 -0.116375230252743
1.4089777469635 -0.126755759119987
1.40372371673584 -0.137074090540409
1.39788043498993 -0.146906264126301
1.39179182052612 -0.156408794224262
1.38516998291016 -0.165816240012646
1.37825202941895 -0.174604572355747
1.37104904651642 -0.182941056787968
1.36335372924805 -0.190618902444839
1.35513353347778 -0.197763957083225
1.34649932384491 -0.204771660268307
1.33743524551392 -0.210661537945271
1.32821977138519 -0.216311641037464
1.31834650039673 -0.221401058137417
1.3081316947937 -0.225754790008068
1.29779267311096 -0.229254238307476
1.28717124462128 -0.231974445283413
1.27604913711548 -0.234656892716885
1.2645355463028 -0.236127994954586
1.25283193588257 -0.237416096031666
1.24039840698242 -0.23803748935461
1.22825026512146 -0.237930439412594
1.21563827991486 -0.237089864909649
1.20233106613159 -0.235776148736477
1.18877959251404 -0.233791999518871
1.17494857311249 -0.231128104031086
1.1606433391571 -0.227776505053043
1.14645957946777 -0.224040426313877
1.13193726539612 -0.219428323209286
1.11745500564575 -0.214431129395962
1.10229337215424 -0.208717219531536
1.0871467590332 -0.202596686780453
1.07201838493347 -0.195712834596634
1.05646669864655 -0.188376247882843
1.04095709323883 -0.180652469396591
1.02527356147766 -0.172389589250088
1.00908899307251 -0.163291983306408
0.99269425868988 -0.153895191848278
0.976383090019226 -0.144046396017075
0.959768652915955 -0.133703455328941
0.942758798599243 -0.123026296496391
0.926076412200928 -0.112180553376675
0.908983826637268 -0.10056284442544
0.892112672328949 -0.0889298859983683
0.874735534191132 -0.0772947017103434
0.857449591159821 -0.0651915967464447
0.84043824672699 -0.0528755132108927
0.822993993759155 -0.0404927916824818
0.805537104606628 -0.0284068882465363
0.787881076335907 -0.0151269249618053
0.770521342754364 -0.00215449929237366
0.753136217594147 0.0107240378856659
0.735750257968903 0.0233164951205254
0.717977464199066 0.0362808555364609
0.700494885444641 0.0494198128581047
0.683324337005615 0.0617815628647804
0.665916264057159 0.0746392831206322
0.648715436458588 0.0869651958346367
0.631373047828674 0.0994339063763618
0.614122986793518 0.111670963466167
0.597113370895386 0.123711504042149
0.580237567424774 0.135215260088444
0.563145816326141 0.146747998893261
0.546229958534241 0.157843373715878
0.52986878156662 0.168568201363087
0.513054668903351 0.179124481976032
0.496880352497101 0.188926853239536
0.480743706226349 0.19819513708353
0.464168071746826 0.20770949870348
0.448500037193298 0.215685941278934
0.432330667972565 0.22427187114954
0.41669112443924 0.232203401625156
0.401033878326416 0.239420540630817
0.385882347822189 0.246030069887638
0.370468080043793 0.252132721245289
0.355479300022125 0.25773710757494
0.340694278478622 0.262619532644749
0.326160699129105 0.266991294920444
0.312323361635208 0.270615674555302
0.298438191413879 0.273682929575443
0.284827888011932 0.276078380644321
0.271400392055511 0.277788288891315
0.258086264133453 0.278940953314304
0.245523631572723 0.279799915850163
0.23319610953331 0.279616542160511
0.221036344766617 0.278595842421055
0.208950817584991 0.277286000549793
0.19757005572319 0.275415010750294
0.186373263597488 0.272665865719318
0.175681501626968 0.269543386995792
0.165177881717682 0.265370585024357
0.155269265174866 0.260961838066578
0.145691454410553 0.255953438580036
0.136425763368607 0.250240333378315
0.127730220556259 0.243909157812595
0.118977919220924 0.237194962799549
0.111161798238754 0.230037160217762
0.103393614292145 0.22224835306406
0.0958970338106155 0.213695295155048
0.0894226580858231 0.205149240791798
0.0830798745155334 0.195595361292362
0.0770703554153442 0.186225689947605
0.0714504420757294 0.176180340349674
0.0662611573934555 0.166237987577915
0.061749279499054 0.154989756643772
0.0575418472290039 0.144552014768124
0.0537373721599579 0.133000411093235
0.0502457767724991 0.121525339782238
0.0471789538860321 0.109896875917912
0.0445543229579926 0.0976339802145958
0.0425477623939514 0.0854775384068489
0.0407989323139191 0.0736758336424828
0.0397763103246689 0.0608945265412331
0.0391044616699219 0.0483879670500755
0.0389225482940674 0.0358841940760612
0.0391547083854675 0.0233058407902718
0.0397042632102966 0.011098824441433
0.0409049838781357 -0.000746972858905792
0.0426268875598907 -0.0134227499365807
0.0448943674564362 -0.025658018887043
0.0472966432571411 -0.0379143916070461
0.0501177161931992 -0.0494759920984507
0.0534798353910446 -0.0607673199847341
0.0574156939983368 -0.0722640566527843
0.0617814064025879 -0.0830301698297262
0.0664498209953308 -0.0937887039035559
0.0718301236629486 -0.104113593697548
0.0772365033626556 -0.114192381501198
0.0834285318851471 -0.123746074736118
0.0898142457008362 -0.133125703781843
0.0965054333209991 -0.14186005294323
0.103777170181274 -0.149995692074299
0.111650332808495 -0.157716788351536
0.119324371218681 -0.16510746628046
0.1280457675457 -0.172203101217747
0.136747255921364 -0.178423695266247
0.146372959017754 -0.184212997555733
0.155874490737915 -0.189152590930462
0.165993332862854 -0.193832986056805
0.176267981529236 -0.197477556765079
0.187044411897659 -0.201075382530689
0.198216617107391 -0.203858651220798
0.209522575139999 -0.206126056611538
0.220868229866028 -0.206994690001011
0.232742667198181 -0.207763470709324
0.245437681674957 -0.207853235304356
0.258345901966095 -0.20757869631052
0.2711341381073 -0.206170417368412
0.28468856215477 -0.204320125281811
0.298110276460648 -0.201789699494839
0.312136709690094 -0.198660664260387
0.326041460037231 -0.194798521697521
0.340424209833145 -0.190525203943253
0.355323404073715 -0.185456998646259
0.369595736265182 -0.17968862503767
0.384531378746033 -0.173517189919949
0.399722158908844 -0.166624642908573
0.415230214595795 -0.159169100224972
0.431006371974945 -0.151753917336464
0.44681578874588 -0.142586044967175
0.462859451770782 -0.134116351604462
0.479057967662811 -0.124397918581963
0.495400667190552 -0.114205408841372
0.512000739574432 -0.104174241423607
0.528263688087463 -0.0933702830225229
0.54470282793045 -0.0820394102483988
0.561456799507141 -0.0705757439136505
0.578239440917969 -0.0590128600597382
0.595012009143829 -0.0468037277460098
0.611644864082336 -0.0348718464374542
0.628807008266449 -0.0221992135047913
0.645757973194122 -0.00921417027711868
0.663113594055176 0.00412672013044357
0.680012106895447 0.0166400298476219
0.697355151176453 0.0299689695239067
0.714488327503204 0.0431260392069817
0.731702387332916 0.0559300407767296
};
\addplot [semithick, red, dashed, forget plot]
table {%
0.75 0
0.767617956255685 0.0126152320166048
0.785221863675294 0.025202759744927
0.802800914680889 0.0377306780229099
0.820344353102624 0.0501672114827695
0.837841479099983 0.0624807937495562
0.855281653865309 0.0746401456242545
0.872654304077727 0.0866143520292023
0.889948926065816 0.0983729374930471
0.90715508962574 0.109885939954111
0.924262441427357 0.121123982665618
0.941260707923679 0.132058343994744
0.958139697658408 0.142661024921306
0.974889302841815 0.152904814062882
0.991499500036706 0.162763350083898
1.00796034976389 0.172211181389763
1.02426199480132 0.18122382306779
1.04039465691463 0.189777811119373
1.0563486317231 0.197850754138649
1.07211428137931 0.205421382738366
1.08768202473299 0.212469597210257
1.10304232467243 0.218976514139914
1.11818567241108 0.224924512976024
1.1331025686362 0.230297283873765
1.147783501695 0.23507987847157
1.16221892339532 0.239258765576722
1.17639922357808 0.242821893954846
1.19031470539574 0.245758764427774
1.20395556419321 0.248060513126141
1.21731187395941 0.249720006825457
1.23037358633393 0.250731949622618
1.24313054782987 0.251092997654816
1.2555725408717 0.250801875167454
1.26768935298946 0.24985948134918
1.27947087567813 0.248268973729967
1.29090722994702 0.246035811751358
1.30198890990496 0.243167744731957
1.31270692999354 0.2396747329547
1.32305295737637 0.235568799170581
1.33301941026506 0.230863819194334
1.34259950666846 0.225575271775738
1.3517872558846 0.219719976188812
1.36057739518197 0.213315848229081
1.36896528374308 0.206381700652654
1.37694677240494 0.198937103974783
1.38451806954475 0.191002311131823
1.39167562078374 0.182598238316676
1.39841601450881 0.173746486857157
1.4047359185686 0.164469388195659
1.41063204764278 0.154790055213092
1.41610115672778 0.144732426840967
1.42114005415556 0.134321297531182
1.42574562723652 0.123582327489247
1.42991487441761 0.112542032954869
1.43364493919855 0.101227758043132
1.43693314251624 0.089667630857217
1.4397770116222 0.0778905070053025
1.44217430452097 0.0659259035746353
1.44412302978876 0.0538039262643269
1.44562146208725 0.0415551919190076
1.44666815398553 0.0292107482394724
1.44726194486432 0.016801992026761
1.44740196775303 0.00436058696240971
1.44708765498222 -0.00808161836067393
1.44631874355061 -0.0204926716768206
1.4450952811249 -0.0328405972734807
1.44341763362016 -0.0450934751015305
1.44128649535024 -0.0572195188465802
1.4387029027798 -0.0691871532341023
1.43566825293285 -0.0809650910678666
1.4321843274821 -0.0925224107662305
1.42825332340439 -0.103828635462315
1.42387789076703 -0.114853815053745
1.41906117760902 -0.125568612881875
1.41380688088891 -0.135944398910193
1.40811930097384 -0.145953351233135
1.40200339507468 -0.155568567310703
1.39546482242266 -0.164764185292558
1.38850997106277 -0.173515513988717
1.38114595342609 -0.181799167392522
1.37338055620066 -0.189593196333819
1.36522213060654 -0.196877206377916
1.35667941319178 -0.203632448466232
1.34776127545165 -0.209841868311401
1.33847641259836 -0.215490103434114
1.328832995749 -0.220563423487838
1.3188383241811 -0.225049619377379
1.30849852098817 -0.228937857346307
1.29781831326193 -0.232218522442798
1.28680092623794 -0.234883078640104
1.27544810236402 -0.23692396914816
1.26376023613114 -0.238334571253006
1.25173659926596 -0.239109208409727
1.23937562223503 -0.239243211851919
1.22680212890194 -0.238807395220756
1.21393543612189 -0.237739020519201
1.2007824030094 -0.236042071113405
1.1873502195155 -0.233721342391173
1.17364639818837 -0.230782623136066
1.15967876862552 -0.227232843837933
1.14545546822952 -0.223080191387264
1.13098492732646 -0.218334192018305
1.11627584919941 -0.213005765794969
1.10133718666704 -0.207107256561536
1.08617811704058 -0.200652441341106
1.070808017043 -0.19365652287622
1.05523643885516 -0.186136108539406
1.03947308802467 -0.178109178316594
1.02352780361101 -0.169595044057188
1.00741054066936 -0.160614301730894
0.991131354992172 -0.151188778049092
0.974700389916618 -0.141341472499412
0.958127864949596 -0.131096495600034
0.941424065943487 -0.120479003995558
0.924599336561409 -0.109515132878708
0.90766407079076 -0.0982319261219058
0.890628706290982 -0.086657264431339
0.873503718391248 -0.074819791786618
0.856299614583182 -0.0627488403957748
0.839026929381251 -0.0504743543738914
0.8216962194482 -0.0380268123406879
0.804318058904672 -0.0254371491254941
0.786903034760972 -0.0127366767652624
0.769461742425004 4.29950186324069e-05
0.752004781254135 0.0128700386771497
0.734542750130382 0.0257124888331812
0.717086243048227 0.0385383222295833
0.699645844712938 0.0513155374272836
0.682232126154625 0.0640122340511521
0.664855640369735 0.0765966913776288
0.647526918007365 0.0890374460548427
0.630256463122704 0.101303368743397
0.613054749024055 0.113363739464377
0.595932214243 0.125188321440869
0.578899258658929 0.136747433220813
0.561966239808458 0.148012018873052
0.545143469406046 0.158953716055882
0.528441210092295 0.169544921769502
0.511869672407924 0.179758855622043
0.495439011959919 0.189569620465462
0.479159326695086 0.198952260295039
0.463040654116175 0.207882815357607
0.447092968153461 0.216338374482455
0.431326175222446 0.224297124738732
0.41575010873165 0.231738398637223
0.400374520922108 0.238642719233979
0.38520907038282 0.244991843656404
0.37026330284855 0.250768805749989
0.355546621900637 0.255957958716022
0.341068244918601 0.260545018739871
0.326837138059129 0.264517110634022
0.312861922218046 0.267862816346776
0.299150740017787 0.270572226688282
0.285711072187049 0.272636995643726
0.272549490830637 0.274050395018263
0.259671337879835 0.274807364778616
0.247080320574249 0.274904551350881
0.234778023333987 0.274340322593555
0.22282111677237 0.273147174350289
0.211212000441657 0.271312180989548
0.199956090718924 0.268842225658862
0.18905951088657 0.265745762806622
0.178528814558833 0.262032850875418
0.168370684504055 0.257715152729482
0.158591652631679 0.252805894649178
0.149197881311021 0.247319783843054
0.140195031151771 0.241272894158711
0.131588220396365 0.234682536784361
0.123382062255884 0.227567134929123
0.115580754025762 0.219946118404367
0.108188187942631 0.211839847297465
0.101208057557953 0.203269566144989
0.0946439418448079 0.19425738357691
0.088499358802492 0.184826268669802
0.0827777883280161 0.17500005435619
0.0774826693237491 0.164803439481848
0.0726173783628224 0.154261983466959
0.0681851974057714 0.143402090111052
0.0641892769319838 0.132250979303424
0.0606325991839522 0.12083664700672
0.0575179445429576 0.109187814856514
0.0548478626398547 0.0973338711778623
0.0526246487503009 0.0853048053160545
0.0508503253165833 0.0731311370573394
0.0495266280133111 0.0608438426868979
0.0486549955532945 0.0484742789692179
0.0482365623412967 0.036054106082026
0.0482721530701452 0.0236152103073717
0.0487622783751904 0.0111896270851532
0.0497071306928156 -0.00119053514168032
0.0511065794918781 -0.0134931700273918
0.0529601650583767 -0.0256862472791051
0.0552670900164109 -0.0377378855745846
0.0580262077740067 -0.0496164247880426
0.061236007111159 -0.0612904981681536
0.0648945922107272 -0.0727291054819839
0.0689996576140302 -0.0839016885812368
0.0735484579177227 -0.0947782113457813
0.0785377725810335 -0.10532924646507
0.0839638670453019 -0.11552607193497
0.0898224525211534 -0.125340780318524
0.0961086482561191 -0.134746403519034
0.10281695173701 -0.143717054765618
0.109941223832714 -0.152228087440402
0.117474696885182 -0.160256267121569
0.125410013591915 -0.167779948892829
0.133739302545796 -0.174779247148552
0.14245429208538 -0.181236180932355
0.151546457798383 -0.187134775860053
0.161007191589854 -0.192461105490089
0.170827973498639 -0.197203261519913
0.181000523658031 -0.201351252889932
0.191516912754772 -0.204896846530735
0.202369615510414 -0.207833373632873
0.213551501760898 -0.210155531524887
0.22505577088336 -0.211859210579139
0.236875844499746 -0.212941368432345
0.249005237340447 -0.213399962707235
0.261437426261433 -0.213233941776159
0.274165733616003 -0.212443283869614
0.28718323530924 -0.211029069576408
0.300482697779478 -0.208993571566475
0.314056543186631 -0.206340347198259
0.327896838839168 -0.203074323164878
0.341995305342881 -0.199201865241865
0.356343337731412 -0.194730829684218
0.370932034454633 -0.189670595476494
0.385752230119493 -0.184032078393973
0.400794528982751 -0.177827728806396
0.416049337197401 -0.171071515546711
0.431506892628147 -0.16377889817583
0.447157291657684 -0.15596678976561
0.462990512824106 -0.147653512015245
0.478996437395876 -0.138858744187699
0.495164867142796 -0.129603467046443
0.511485539633279 -0.119909902709476
0.527948141407569 -0.109801451123518
0.544542319363787 -0.0993026236948103
0.561257690663388 -0.0884389744880547
0.578083851424296 -0.0772370293146803
0.595010384429933 -0.0657242129685941
0.612026866043939 -0.0539287748255556
0.62912287248563 -0.0418797129959582
0.646287985590849 -0.029606697205851
0.663511798156965 -0.0171399905741775
0.680783918949166 -0.00451037045289776
0.698093977427272 0.00825095150093907
0.715431628237671 0.0211124098473716
0.732786555502968 0.0340421663000904
};
\addplot [semithick, green, dash pattern=on 1pt off 3pt on 3pt off 3pt, forget plot]
table {%
0.75 0
0.767347219519284 0.012137253483567
0.784683394455509 0.0242435861667548
0.801997486582194 0.036288155514001
0.819278471531621 0.0482402761288358
0.836515345700615 0.0600694978228124
0.853697133135913 0.0717456830897325
0.870812892392119 0.0832390837898383
0.887851723355235 0.0945204168511733
0.904802774024683 0.105560938798497
0.921655247246804 0.116332518924018
0.938398407392915 0.126807710918802
0.955021586975309 0.136959822789028
0.971514193195148 0.146762984887382
0.987865714417124 0.156192215896701
1.0040657265673 0.165223486610677
1.02010389945277 0.173833781364724
1.03597000300535 0.182001156979107
1.05165391345605 0.189704799085743
1.06714561945411 0.196925075719347
1.08243522815337 0.203643588062188
1.09751297130037 0.209843218238558
1.1123692113737 0.215508174058654
1.12699444784162 0.220624030609957
1.14137932362429 0.225177768584601
1.1555146318663 0.22915780921071
1.16939132313901 0.232554045621132
1.18300051319534 0.235357870442502
1.19633349138035 0.237562199321834
1.20938172974852 0.23916149003295
1.22213689283924 0.240151756735333
1.23459084790774 0.240530578918402
1.24673567520343 0.24029710459028
1.25856367765532 0.239452047402427
1.27006738911946 0.237997677671498
1.28123958024499 0.235937807669022
1.29207326111455 0.233277772045945
1.30256168017249 0.230024404727089
1.31269831956044 0.226186013888674
1.32247688771958 0.221772356568193
1.33189131078413 0.216794613983565
1.34093572465289 0.211265367835258
1.34960446953272 0.205198576949498
1.35789208820684 0.198609552872807
1.36579332846363 0.191514932670796
1.37330314928634 0.18393264729003
1.38041672978698 0.175881884318464
1.38712947958844 0.167383044631953
1.39343704941026 0.158457693037802
1.39933534088867 0.149128503482261
1.40482051503024 0.139419199628507
1.40988899904863 0.129354491659788
1.41453749160152 0.118960010081206
1.41876296661194 0.108262237150103
1.42256267593454 0.0972884364110782
1.42593415113889 0.086066580678514
1.42887520465437 0.0746252787098322
1.43138393047621 0.0629937007473294
1.43345870458359 0.0512015030695605
1.43509818517756 0.0392787516771522
1.43630131281055 0.0272558452352367
1.43706731045357 0.0151634373994425
1.43739568352819 0.00303235866024866
1.43728621991849 -0.00910646215137196
1.43673898997073 -0.0212220765475008
1.43575434648365 -0.0332835951024502
1.43433292468914 -0.0452602660529673
1.43247564221903 -0.0571215536269705
1.43018369904855 -0.0688372159533007
1.42745857739782 -0.0803773824121825
1.4243020415584 -0.0917126302942692
1.42071613759078 -0.102814060643163
1.41670319280875 -0.113653373158817
1.41226581492764 -0.124202940032363
1.40740689070718 -0.134435878560225
1.40212958387022 -0.144326122339201
1.3964373320371 -0.153848490767307
1.39033384239815 -0.162978756463708
1.38382308587764 -0.171693710079292
1.37690928964889 -0.179971221816331
1.36959692806383 -0.187790298849553
1.36189071236441 -0.195131137800147
1.35379557991389 -0.201975171526458
1.34531668404164 -0.20830510981277
1.33645938581161 -0.214104974062127
1.32722924897007 -0.219360126749639
1.31763203892423 -0.224057297001799
1.30767372588241 -0.22818460402503
1.2973604914176 -0.231731580045289
1.28669873695777 -0.234689193904312
1.27569509230234 -0.237049875620184
1.26435642233521 -0.238807541316647
1.25268983059913 -0.239957617226964
1.24070265912099 -0.240497061156197
1.22840248459827 -0.24042437986202
1.21579711159144 -0.239739641182038
1.20289456363565 -0.238444480231447
1.1897030732081 -0.236542099469318
1.17623107134149 -0.234037262792463
1.16248717745014 -0.230936284032406
1.14848018970572 -0.227247010319085
1.13421907610845 -0.222978800771493
1.11971296626323 -0.218142500920566
1.10497114378721 -0.212750413195198
1.09000303923492 -0.206816263729265
1.07481822341698 -0.200355165686855
1.05942640099553 -0.193383579258294
1.04383740425637 -0.185919268450478
1.02806118697728 -0.177981254778958
1.01210781833129 -0.169589767962932
0.995987476780275 -0.160766193724954
0.979710443928391 -0.151533018802209
0.963287098315684 -0.141913773283855
0.946727909140567 -0.131932970397911
0.930043429905878 -0.121616043880467
0.91324429198767 -0.110989283069269
0.896341198128817 -0.100079765872392
0.879344915861624 -0.0889152897708354
0.862266270864871 -0.0775243010211479
0.845116140261585 -0.0659358222307342
0.827905445864243 -0.0541793784842489
0.810645147374399 -0.042284922204474
0.793346235543765 -0.0302827569353333
0.776019725303823 -0.0182034602382608
0.758676648870971 -0.00607780589600706
0.74132804883419 0.00606331437980999
0.723984971232126 0.0181889695395348
0.706658458626486 0.0302682678858693
0.689359543178601 0.0422704356646973
0.672099239736024 0.0541648953575106
0.654888538936098 0.0659213434775336
0.637738400333409 0.077509827673207
0.620659745558167 0.0889008229448788
0.603663451512546 0.10006530678342
0.586760343612044 0.110974833042994
0.569961189078862 0.121601604364455
0.55327669029407 0.131918542970798
0.536717478214957 0.141899359661781
0.520294105863188 0.151518620841353
0.504017041888231 0.160751813418715
0.487896664208613 0.169575407431921
0.471943253730781 0.177966916251536
0.456166988141264 0.18590495423109
0.440577935762111 0.193369291680399
0.425186049451712 0.200340907046881
0.410001160522753 0.206802036197834
0.395032972635676 0.212736218702091
0.38029105560979 0.218128341010777
0.36578483907538 0.222964676431608
0.351523605870586 0.227232921776542
0.337516485069478 0.23092223053541
0.323772444518238 0.234023242385741
0.310300282763089 0.236528108790522
0.297108620287721 0.238430514364134
0.284205890052545 0.239725693611976
0.27160032745433 0.240410442590784
0.259299960006205 0.240483125024951
0.247312597261469 0.239943672488383
0.2356458217318 0.238793578459958
0.224306981713199 0.237035886402939
0.21330318694315 0.234675172482025
0.202641307788403 0.231717524032126
0.192327978173203 0.228170515287301
0.1823696017639 0.22404318200088
0.172772360193836 0.219345996319791
0.163542221572307 0.214090842618848
0.154684947376684 0.208290994107505
0.146206096155645 0.201961089155653
0.138111023180157 0.195117105712544
0.130404876038488 0.187776332069152
0.12309258692145 0.179957332525672
0.116178862798813 0.171679907117356
0.109668174793245 0.162965045212245
0.10356474787866 0.153834873345794
0.0978725516915446 0.144312598005936
0.0925952928735381 0.134422444219678
0.0877364090495248 0.124189590767097
0.0832990643281257 0.113640102728524
0.0792861460924091 0.102800861916777
0.0757002628080311 0.0916994956001386
0.0725437425870994 0.0803643038046822
0.069818632284583 0.0688241854024739
0.0675266969522104 0.0571085631419612
0.0656694195214907 0.0452473077513642
0.0642480006271671 0.0332706612373471
0.0632633585132331 0.0212091595029022
0.0627161289858991 0.00909355441514592
0.0626066653929575 -0.00304526453801165
0.0629350386185861 -0.0151763488558262
0.0637010370885092 -0.0272687697454203
0.0649041667842718 -0.0392916968007361
0.0665436512687222 -0.0512144764858195
0.0686184317292081 -0.0630067102720941
0.0711271670520408 -0.0746383322854955
0.0740682339532131 -0.0860796863272447
0.0774397272079407 -0.0973016021399556
0.0812394600470219 -0.108275470796062
0.0854649648223936 -0.118973319084073
0.0901134940873028 -0.12936788275431
0.0951820222852561 -0.139432678452301
0.100667248289028 -0.149142074107302
0.10656559906303 -0.158471357449301
0.112873234718821 -0.167396802199979
0.119586055168304 -0.175895731332128
0.126699708426136 -0.183946576646328
0.13420960035749 -0.191528933821879
0.142110905322839 -0.198623612126504
0.150398576795015 -0.205212678179984
0.159067356720643 -0.211279493592426
0.16811178230224 -0.216808746901301
0.177526189102462 -0.221786480886398
0.187304709947427 -0.226200116851138
0.197441269929379 -0.230038477621279
0.207929578656845 -0.233291810717099
0.218763121506754 -0.235951812451675
0.229935151803763 -0.238011652801433
0.241438685552319 -0.239466000064469
0.253266499708268 -0.240311043799429
0.265411134226006 -0.240544514421553
0.277864897472831 -0.240165698073282
0.290619874197545 -0.23917544583971
0.303667935101709 -0.237576176879925
0.317000747134867 -0.235371875470324
0.330609783830706 -0.232568082244952
0.344486335235128 -0.229171880066449
0.358621517191145 -0.225191874998346
0.373006279910045 -0.220638172815656
0.387631415867 -0.215522351422736
0.402487567119295 -0.209857429471725
0.417565232169048 -0.203657831407191
0.432854772492479 -0.196939349109845
0.448346418844808 -0.189719100275549
0.46403027743119 -0.182015483643688
0.47989633601438 -0.173848131178146
0.495934470011695 -0.165237857301697
0.512134448618413 -0.156206605287707
0.528485940982374 -0.146777390919585
0.544978522444945 -0.136974243536887
0.561601680856407 -0.126822144596177
0.578344822968638 -0.116346963884112
0.595197280904322 -0.105575393529208
0.61214831869946 -0.094534879967153
0.629187138914278 -0.0832535540222168
0.646302889306651 -0.071760159274263
0.663484669561462 -0.0600839788869683
0.680721538069075 -0.0482547610782518
0.698002518745858 -0.0363026434185232
0.715316607889725 -0.0242580761462783
0.732652781063624 -0.012151744693772
};
\addplot [line width = \linewidthEightC, color = reference, opacity=\opacityRef, forget plot]
table {%
0.75 0
0.753239095211029 0.00253768265247345
0.767933547496796 0.013369970023632
0.786693155765533 0.0271937251091003
0.803527593612671 0.0391305908560753
0.821137547492981 0.051875427365303
0.839008331298828 0.0644083172082901
0.856212973594666 0.0763905346393585
0.873286187648773 0.0880061388015747
0.890205562114716 0.0997696071863174
0.907287895679474 0.111332938075066
0.924486458301544 0.122647628188133
0.94121128320694 0.133281484246254
0.957839071750641 0.143894448876381
0.974659502506256 0.154067724943161
0.99138730764389 0.163972049951553
1.00800484418869 0.173401236534119
1.02415293455124 0.182137832045555
1.04023963212967 0.190422847867012
1.05574005842209 0.198315873742104
1.07160919904709 0.205645754933357
1.08711296319962 0.211979404091835
1.10221832990646 0.218266770243645
1.11716717481613 0.22398267686367
1.13188916444778 0.229129180312157
1.14640003442764 0.233250841498375
1.16057807207108 0.237387672066689
1.17471784353256 0.240382000803947
1.1884930729866 0.242903664708138
1.20183819532394 0.244689539074898
1.21522396802902 0.245989724993706
1.22794681787491 0.246619090437889
1.24055308103561 0.246651992201805
1.25277334451675 0.24621956050396
1.26463013887405 0.245182737708092
1.2764156460762 0.243587002158165
1.28774458169937 0.241200372576714
1.29861098527908 0.23824317753315
1.30933386087418 0.233940944075584
1.31931573152542 0.230188563466072
1.32934361696243 0.225380703806877
1.33806520700455 0.220090702176094
1.34764629602432 0.213927194476128
1.35593718290329 0.207767471671104
1.36421900987625 0.200588628649712
1.37219494581223 0.193265452980995
1.37941735982895 0.185063675045967
1.38646513223648 0.176695272326469
1.39333695173264 0.167782813310623
1.39977771043777 0.15851117670536
1.4053413271904 0.148058161139488
1.41077488660812 0.138033911585808
1.41559284925461 0.127701759338379
1.4196235537529 0.117055907845497
1.4240158200264 0.105388164520264
1.42761534452438 0.0941447019577026
1.43052810430527 0.0822708606719971
1.43323916196823 0.0708247125148773
1.43538361787796 0.058703824877739
1.4372096657753 0.0466984510421753
1.4387554526329 0.0346819534897804
1.43977397680283 0.0224432200193405
1.44020849466324 0.0100249424576759
1.44029313325882 -0.00243283808231354
1.44004327058792 -0.0148460678756237
1.438860476017 -0.0272964648902416
1.43748933076859 -0.0392404943704605
1.43571537733078 -0.0520545244216919
1.433374106884 -0.0637296801432967
1.43062514066696 -0.0754935548175126
1.42760366201401 -0.0874227331951261
1.42401081323624 -0.0988353379070759
1.42011445760727 -0.10985042899847
1.41577047109604 -0.120738547295332
1.41090482473373 -0.131521712988615
1.40567737817764 -0.142419256269932
1.40004700422287 -0.152169726788998
1.39394491910934 -0.161680899560452
1.38751584291458 -0.170563980937004
1.3803579211235 -0.179361864924431
1.37282174825668 -0.187521778047085
1.36508125066757 -0.195161893963814
1.35675257444382 -0.202249228954315
1.34823054075241 -0.209106370806694
1.33933311700821 -0.215201452374458
1.32957357168198 -0.220746964216232
1.3199308514595 -0.225477635860443
1.31033509969711 -0.229783847928047
1.2997584939003 -0.233531624078751
1.28916829824448 -0.236433789134026
1.2779261469841 -0.238397777080536
1.26640123128891 -0.240280374884605
1.25458174943924 -0.241610810160637
1.24253696203232 -0.241837918758392
1.23043686151505 -0.241563111543655
1.21749037504196 -0.241020247340202
1.20456749200821 -0.239236533641815
1.19139665365219 -0.237509369850159
1.17766004800797 -0.234463751316071
1.16359680891037 -0.23152083158493
1.14935296773911 -0.227566167712212
1.13453763723373 -0.222807422280312
1.11998373270035 -0.21746563911438
1.10511988401413 -0.211723670363426
1.09028774499893 -0.205528989434242
1.07495874166489 -0.198616191744804
1.05923110246658 -0.190831750631332
1.04344707727432 -0.182769000530243
1.02782303094864 -0.174582310020924
1.01175910234451 -0.165421344339848
0.995215475559235 -0.156009584665298
0.979063212871552 -0.146188899874687
0.962888538837433 -0.135858882218599
0.946227133274078 -0.125042095780373
0.929413616657257 -0.113944292068481
0.912314116954803 -0.102407176047564
0.895676493644714 -0.0910323467105627
0.87847900390625 -0.078969812951982
0.861458957195282 -0.0665267370641232
0.843896508216858 -0.0542561374604702
0.826511681079865 -0.0416169725358486
0.809164524078369 -0.0291422083973885
0.791875958442688 -0.0165668129920959
0.774488866329193 -0.00357750058174133
0.757133781909943 0.0095110610127449
0.73957371711731 0.0222045928239822
0.722129285335541 0.0353544950485229
0.705018877983093 0.047969788312912
0.687700271606445 0.0613633245229721
0.670115888118744 0.073985293507576
0.652946591377258 0.0865507423877716
0.635872364044189 0.0988799184560776
0.618727505207062 0.110906764864922
0.60142058134079 0.123387530446053
0.584641754627228 0.135177373886108
0.567867577075958 0.146511256694794
0.550968766212463 0.15743687748909
0.534585237503052 0.168280974030495
0.518022477626801 0.179116144776344
0.501647591590881 0.188955441117287
0.485316455364227 0.198691979050636
0.469256699085236 0.207779660820961
0.45318078994751 0.216462269425392
0.437209725379944 0.224751964211464
0.421703636646271 0.23236183822155
0.40613329410553 0.239525362849236
0.39081808924675 0.24604569375515
0.37565141916275 0.25224994122982
0.360481202602386 0.25790299475193
0.345844507217407 0.263021275401115
0.331279158592224 0.267507418990135
0.31706178188324 0.271142944693565
0.303024083375931 0.274281904101372
0.288946598768234 0.276549503207207
0.276043593883514 0.278488203883171
0.262887239456177 0.279850617051125
0.249910354614258 0.280177935957909
0.237533360719681 0.280356094241142
0.225644946098328 0.279666617512703
0.213705450296402 0.278119161725044
0.202270060777664 0.276243850588799
0.191168427467346 0.273844018578529
0.180174320936203 0.27051617205143
0.169907629489899 0.266462698578835
0.159749507904053 0.261863991618156
0.150304079055786 0.256763890385628
0.140822574496269 0.25127224624157
0.132119029760361 0.245141223073006
0.123656660318375 0.238314554095268
0.115644335746765 0.231211051344872
0.108334898948669 0.223334118723869
0.100958421826363 0.214969381690025
0.0943275690078735 0.206182762980461
0.0876276791095734 0.1971185952425
0.0814271420240402 0.187777146697044
0.0762212127447128 0.177455857396126
0.0709253549575806 0.167199358344078
0.0661797970533371 0.156733483076096
0.061949223279953 0.145784750580788
0.0581098943948746 0.134111180901527
0.0548661947250366 0.122430130839348
0.0520368814468384 0.111096084117889
0.0496130436658859 0.0985787510871887
0.0472984164953232 0.087568074464798
0.0458300411701202 0.0750955790281296
0.0446675270795822 0.0624333471059799
0.0438239276409149 0.0499419271945953
0.0436710864305496 0.0375024154782295
0.0438536256551743 0.0247569307684898
0.0444370210170746 0.0124810636043549
0.0458152741193771 -0.000127322971820831
0.0474891811609268 -0.0130462273955345
0.0494669526815414 -0.0244505405426025
0.0520472079515457 -0.0364738218486309
0.0550171434879303 -0.0481881890445948
0.0582164376974106 -0.0596030838787556
0.0622129142284393 -0.0706984819844365
0.066273421049118 -0.0815856251865625
0.0710213482379913 -0.0924181137233973
0.0761547982692719 -0.103106930851936
0.0818294435739517 -0.113011971116066
0.087813675403595 -0.122992318123579
0.0944161266088486 -0.132160693407059
0.101482719182968 -0.141269773244858
0.108633175492287 -0.149343311786652
0.116222783923149 -0.157184340059757
0.124386578798294 -0.164673268795013
0.132448509335518 -0.171245887875557
0.141644954681396 -0.177740961313248
0.150662899017334 -0.183018878102303
0.160209029912949 -0.188294805586338
0.170479983091354 -0.193168297410011
0.180945515632629 -0.197426348924637
0.191608637571335 -0.20039701461792
0.202764540910721 -0.203372403979301
0.213969677686691 -0.205397456884384
0.225459575653076 -0.206654593348503
0.237398952245712 -0.207314372062683
0.249783515930176 -0.207249999046326
0.262563765048981 -0.207044422626495
0.275457978248596 -0.205690279603004
0.288891047239304 -0.204015299677849
0.302105158567429 -0.20139017701149
0.316396206617355 -0.198679268360138
0.330376863479614 -0.194617860019207
0.345187187194824 -0.190432399511337
0.359416961669922 -0.185141548514366
0.374143779277802 -0.179473958909512
0.389514058828354 -0.173458360135555
0.404536008834839 -0.166455656290054
0.420125961303711 -0.15929339826107
0.435654640197754 -0.151446036994457
0.451492428779602 -0.14321581274271
0.467078566551208 -0.133860755711794
0.483495473861694 -0.124598622322083
0.499645113945007 -0.114663608372211
0.516069948673248 -0.104355735704303
0.532678723335266 -0.0933284014463425
0.549385488033295 -0.0827021319419146
0.565859019756317 -0.0710677616298199
0.582656621932983 -0.0593014545738697
0.599505543708801 -0.0476973485201597
0.616595447063446 -0.0352187640964985
0.63344019651413 -0.0223492234945297
0.650260627269745 -0.0103643238544464
0.6674924492836 0.00308728963136673
0.684529840946198 0.0158190876245499
0.70204371213913 0.0292671024799347
0.718858003616333 0.0422031134366989
0.736257314682007 0.0557236969470978
};
\addplot [semithick, red, dashed, forget plot]
table {%
0.75 0
0.767617956255685 0.0126152320166048
0.785221863675294 0.025202759744927
0.802800914680889 0.0377306780229099
0.820344353102624 0.0501672114827695
0.837841479099983 0.0624807937495562
0.855281653865309 0.0746401456242545
0.872654304077727 0.0866143520292023
0.889948926065816 0.0983729374930471
0.90715508962574 0.109885939954111
0.924262441427357 0.121123982665618
0.941260707923679 0.132058343994744
0.958139697658408 0.142661024921306
0.974889302841815 0.152904814062882
0.991499500036706 0.162763350083898
1.00796034976389 0.172211181389763
1.02426199480132 0.18122382306779
1.04039465691463 0.189777811119373
1.0563486317231 0.197850754138649
1.07211428137931 0.205421382738366
1.08768202473299 0.212469597210257
1.10304232467243 0.218976514139914
1.11818567241108 0.224924512976024
1.1331025686362 0.230297283873765
1.147783501695 0.23507987847157
1.16221892339532 0.239258765576722
1.17639922357808 0.242821893954846
1.19031470539574 0.245758764427774
1.20395556419321 0.248060513126141
1.21731187395941 0.249720006825457
1.23037358633393 0.250731949622618
1.24313054782987 0.251092997654816
1.2555725408717 0.250801875167454
1.26768935298946 0.24985948134918
1.27947087567813 0.248268973729967
1.29090722994702 0.246035811751358
1.30198890990496 0.243167744731957
1.31270692999354 0.2396747329547
1.32305295737637 0.235568799170581
1.33301941026506 0.230863819194334
1.34259950666846 0.225575271775738
1.3517872558846 0.219719976188812
1.36057739518197 0.213315848229081
1.36896528374308 0.206381700652654
1.37694677240494 0.198937103974783
1.38451806954475 0.191002311131823
1.39167562078374 0.182598238316676
1.39841601450881 0.173746486857157
1.4047359185686 0.164469388195659
1.41063204764278 0.154790055213092
1.41610115672778 0.144732426840967
1.42114005415556 0.134321297531182
1.42574562723652 0.123582327489247
1.42991487441761 0.112542032954869
1.43364493919855 0.101227758043132
1.43693314251624 0.089667630857217
1.4397770116222 0.0778905070053025
1.44217430452097 0.0659259035746353
1.44412302978876 0.0538039262643269
1.44562146208725 0.0415551919190076
1.44666815398553 0.0292107482394724
1.44726194486432 0.016801992026761
1.44740196775303 0.00436058696240971
1.44708765498222 -0.00808161836067393
1.44631874355061 -0.0204926716768206
1.4450952811249 -0.0328405972734807
1.44341763362016 -0.0450934751015305
1.44128649535024 -0.0572195188465802
1.4387029027798 -0.0691871532341023
1.43566825293285 -0.0809650910678666
1.4321843274821 -0.0925224107662305
1.42825332340439 -0.103828635462315
1.42387789076703 -0.114853815053745
1.41906117760902 -0.125568612881875
1.41380688088891 -0.135944398910193
1.40811930097384 -0.145953351233135
1.40200339507468 -0.155568567310703
1.39546482242266 -0.164764185292558
1.38850997106277 -0.173515513988717
1.38114595342609 -0.181799167392522
1.37338055620066 -0.189593196333819
1.36522213060654 -0.196877206377916
1.35667941319178 -0.203632448466232
1.34776127545165 -0.209841868311401
1.33847641259836 -0.215490103434114
1.328832995749 -0.220563423487838
1.3188383241811 -0.225049619377379
1.30849852098817 -0.228937857346307
1.29781831326193 -0.232218522442798
1.28680092623794 -0.234883078640104
1.27544810236402 -0.23692396914816
1.26376023613114 -0.238334571253006
1.25173659926596 -0.239109208409727
1.23937562223503 -0.239243211851919
1.22680212890194 -0.238807395220756
1.21393543612189 -0.237739020519201
1.2007824030094 -0.236042071113405
1.1873502195155 -0.233721342391173
1.17364639818837 -0.230782623136066
1.15967876862552 -0.227232843837933
1.14545546822952 -0.223080191387264
1.13098492732646 -0.218334192018305
1.11627584919941 -0.213005765794969
1.10133718666704 -0.207107256561536
1.08617811704058 -0.200652441341106
1.070808017043 -0.19365652287622
1.05523643885516 -0.186136108539406
1.03947308802467 -0.178109178316594
1.02352780361101 -0.169595044057188
1.00741054066936 -0.160614301730894
0.991131354992172 -0.151188778049092
0.974700389916618 -0.141341472499412
0.958127864949596 -0.131096495600034
0.941424065943487 -0.120479003995558
0.924599336561409 -0.109515132878708
0.90766407079076 -0.0982319261219058
0.890628706290982 -0.086657264431339
0.873503718391248 -0.074819791786618
0.856299614583182 -0.0627488403957748
0.839026929381251 -0.0504743543738914
0.8216962194482 -0.0380268123406879
0.804318058904672 -0.0254371491254941
0.786903034760972 -0.0127366767652624
0.769461742425004 4.29950186324069e-05
0.752004781254135 0.0128700386771497
0.734542750130382 0.0257124888331812
0.717086243048227 0.0385383222295833
0.699645844712938 0.0513155374272836
0.682232126154625 0.0640122340511521
0.664855640369735 0.0765966913776288
0.647526918007365 0.0890374460548427
0.630256463122704 0.101303368743397
0.613054749024055 0.113363739464377
0.595932214243 0.125188321440869
0.578899258658929 0.136747433220813
0.561966239808458 0.148012018873052
0.545143469406046 0.158953716055882
0.528441210092295 0.169544921769502
0.511869672407924 0.179758855622043
0.495439011959919 0.189569620465462
0.479159326695086 0.198952260295039
0.463040654116175 0.207882815357607
0.447092968153461 0.216338374482455
0.431326175222446 0.224297124738732
0.41575010873165 0.231738398637223
0.400374520922108 0.238642719233979
0.38520907038282 0.244991843656404
0.37026330284855 0.250768805749989
0.355546621900637 0.255957958716022
0.341068244918601 0.260545018739871
0.326837138059129 0.264517110634022
0.312861922218046 0.267862816346776
0.299150740017787 0.270572226688282
0.285711072187049 0.272636995643726
0.272549490830637 0.274050395018263
0.259671337879835 0.274807364778616
0.247080320574249 0.274904551350881
0.234778023333987 0.274340322593555
0.22282111677237 0.273147174350289
0.211212000441657 0.271312180989548
0.199956090718924 0.268842225658862
0.18905951088657 0.265745762806622
0.178528814558833 0.262032850875418
0.168370684504055 0.257715152729482
0.158591652631679 0.252805894649178
0.149197881311021 0.247319783843054
0.140195031151771 0.241272894158711
0.131588220396365 0.234682536784361
0.123382062255884 0.227567134929123
0.115580754025762 0.219946118404367
0.108188187942631 0.211839847297465
0.101208057557953 0.203269566144989
0.0946439418448079 0.19425738357691
0.088499358802492 0.184826268669802
0.0827777883280161 0.17500005435619
0.0774826693237491 0.164803439481848
0.0726173783628224 0.154261983466959
0.0681851974057714 0.143402090111052
0.0641892769319838 0.132250979303424
0.0606325991839522 0.12083664700672
0.0575179445429576 0.109187814856514
0.0548478626398547 0.0973338711778623
0.0526246487503009 0.0853048053160545
0.0508503253165833 0.0731311370573394
0.0495266280133111 0.0608438426868979
0.0486549955532945 0.0484742789692179
0.0482365623412967 0.036054106082026
0.0482721530701452 0.0236152103073717
0.0487622783751904 0.0111896270851532
0.0497071306928156 -0.00119053514168032
0.0511065794918781 -0.0134931700273918
0.0529601650583767 -0.0256862472791051
0.0552670900164109 -0.0377378855745846
0.0580262077740067 -0.0496164247880426
0.061236007111159 -0.0612904981681536
0.0648945922107272 -0.0727291054819839
0.0689996576140302 -0.0839016885812368
0.0735484579177227 -0.0947782113457813
0.0785377725810335 -0.10532924646507
0.0839638670453019 -0.11552607193497
0.0898224525211534 -0.125340780318524
0.0961086482561191 -0.134746403519034
0.10281695173701 -0.143717054765618
0.109941223832714 -0.152228087440402
0.117474696885182 -0.160256267121569
0.125410013591915 -0.167779948892829
0.133739302545796 -0.174779247148552
0.14245429208538 -0.181236180932355
0.151546457798383 -0.187134775860053
0.161007191589854 -0.192461105490089
0.170827973498639 -0.197203261519913
0.181000523658031 -0.201351252889932
0.191516912754772 -0.204896846530735
0.202369615510414 -0.207833373632873
0.213551501760898 -0.210155531524887
0.22505577088336 -0.211859210579139
0.236875844499746 -0.212941368432345
0.249005237340447 -0.213399962707235
0.261437426261433 -0.213233941776159
0.274165733616003 -0.212443283869614
0.28718323530924 -0.211029069576408
0.300482697779478 -0.208993571566475
0.314056543186631 -0.206340347198259
0.327896838839168 -0.203074323164878
0.341995305342881 -0.199201865241865
0.356343337731412 -0.194730829684218
0.370932034454633 -0.189670595476494
0.385752230119493 -0.184032078393973
0.400794528982751 -0.177827728806396
0.416049337197401 -0.171071515546711
0.431506892628147 -0.16377889817583
0.447157291657684 -0.15596678976561
0.462990512824106 -0.147653512015245
0.478996437395876 -0.138858744187699
0.495164867142796 -0.129603467046443
0.511485539633279 -0.119909902709476
0.527948141407569 -0.109801451123518
0.544542319363787 -0.0993026236948103
0.561257690663388 -0.0884389744880547
0.578083851424296 -0.0772370293146803
0.595010384429933 -0.0657242129685941
0.612026866043939 -0.0539287748255556
0.62912287248563 -0.0418797129959582
0.646287985590849 -0.029606697205851
0.663511798156965 -0.0171399905741775
0.680783918949166 -0.00451037045289776
0.698093977427272 0.00825095150093907
0.715431628237671 0.0211124098473716
0.732786555502968 0.0340421663000904
};
\addplot [semithick, green, dash pattern=on 1pt off 3pt on 3pt off 3pt, forget plot]
table {%
0.75 0
0.767347219519284 0.012137253483567
0.784683394455509 0.0242435861667548
0.801997486582194 0.036288155514001
0.819278471531621 0.0482402761288358
0.836515345700615 0.0600694978228124
0.853697133135913 0.0717456830897325
0.870812892392119 0.0832390837898383
0.887851723355235 0.0945204168511733
0.904802774024683 0.105560938798497
0.921655247246804 0.116332518924018
0.938398407392915 0.126807710918802
0.955021586975309 0.136959822789028
0.971514193195148 0.146762984887382
0.987865714417124 0.156192215896701
1.0040657265673 0.165223486610677
1.02010389945277 0.173833781364724
1.03597000300535 0.182001156979107
1.05165391345605 0.189704799085743
1.06714561945411 0.196925075719347
1.08243522815337 0.203643588062188
1.09751297130037 0.209843218238558
1.1123692113737 0.215508174058654
1.12699444784162 0.220624030609957
1.14137932362429 0.225177768584601
1.1555146318663 0.22915780921071
1.16939132313901 0.232554045621132
1.18300051319534 0.235357870442502
1.19633349138035 0.237562199321834
1.20938172974852 0.23916149003295
1.22213689283924 0.240151756735333
1.23459084790774 0.240530578918402
1.24673567520343 0.24029710459028
1.25856367765532 0.239452047402427
1.27006738911946 0.237997677671498
1.28123958024499 0.235937807669022
1.29207326111455 0.233277772045945
1.30256168017249 0.230024404727089
1.31269831956044 0.226186013888674
1.32247688771958 0.221772356568193
1.33189131078413 0.216794613983565
1.34093572465289 0.211265367835258
1.34960446953272 0.205198576949498
1.35789208820684 0.198609552872807
1.36579332846363 0.191514932670796
1.37330314928634 0.18393264729003
1.38041672978698 0.175881884318464
1.38712947958844 0.167383044631953
1.39343704941026 0.158457693037802
1.39933534088867 0.149128503482261
1.40482051503024 0.139419199628507
1.40988899904863 0.129354491659788
1.41453749160152 0.118960010081206
1.41876296661194 0.108262237150103
1.42256267593454 0.0972884364110782
1.42593415113889 0.086066580678514
1.42887520465437 0.0746252787098322
1.43138393047621 0.0629937007473294
1.43345870458359 0.0512015030695605
1.43509818517756 0.0392787516771522
1.43630131281055 0.0272558452352367
1.43706731045357 0.0151634373994425
1.43739568352819 0.00303235866024866
1.43728621991849 -0.00910646215137196
1.43673898997073 -0.0212220765475008
1.43575434648365 -0.0332835951024502
1.43433292468914 -0.0452602660529673
1.43247564221903 -0.0571215536269705
1.43018369904855 -0.0688372159533007
1.42745857739782 -0.0803773824121825
1.4243020415584 -0.0917126302942692
1.42071613759078 -0.102814060643163
1.41670319280875 -0.113653373158817
1.41226581492764 -0.124202940032363
1.40740689070718 -0.134435878560225
1.40212958387022 -0.144326122339201
1.3964373320371 -0.153848490767307
1.39033384239815 -0.162978756463708
1.38382308587764 -0.171693710079292
1.37690928964889 -0.179971221816331
1.36959692806383 -0.187790298849553
1.36189071236441 -0.195131137800147
1.35379557991389 -0.201975171526458
1.34531668404164 -0.20830510981277
1.33645938581161 -0.214104974062127
1.32722924897007 -0.219360126749639
1.31763203892423 -0.224057297001799
1.30767372588241 -0.22818460402503
1.2973604914176 -0.231731580045289
1.28669873695777 -0.234689193904312
1.27569509230234 -0.237049875620184
1.26435642233521 -0.238807541316647
1.25268983059913 -0.239957617226964
1.24070265912099 -0.240497061156197
1.22840248459827 -0.24042437986202
1.21579711159144 -0.239739641182038
1.20289456363565 -0.238444480231447
1.1897030732081 -0.236542099469318
1.17623107134149 -0.234037262792463
1.16248717745014 -0.230936284032406
1.14848018970572 -0.227247010319085
1.13421907610845 -0.222978800771493
1.11971296626323 -0.218142500920566
1.10497114378721 -0.212750413195198
1.09000303923492 -0.206816263729265
1.07481822341698 -0.200355165686855
1.05942640099553 -0.193383579258294
1.04383740425637 -0.185919268450478
1.02806118697728 -0.177981254778958
1.01210781833129 -0.169589767962932
0.995987476780275 -0.160766193724954
0.979710443928391 -0.151533018802209
0.963287098315684 -0.141913773283855
0.946727909140567 -0.131932970397911
0.930043429905878 -0.121616043880467
0.91324429198767 -0.110989283069269
0.896341198128817 -0.100079765872392
0.879344915861624 -0.0889152897708354
0.862266270864871 -0.0775243010211479
0.845116140261585 -0.0659358222307342
0.827905445864243 -0.0541793784842489
0.810645147374399 -0.042284922204474
0.793346235543765 -0.0302827569353333
0.776019725303823 -0.0182034602382608
0.758676648870971 -0.00607780589600706
0.74132804883419 0.00606331437980999
0.723984971232126 0.0181889695395348
0.706658458626486 0.0302682678858693
0.689359543178601 0.0422704356646973
0.672099239736024 0.0541648953575106
0.654888538936098 0.0659213434775336
0.637738400333409 0.077509827673207
0.620659745558167 0.0889008229448788
0.603663451512546 0.10006530678342
0.586760343612044 0.110974833042994
0.569961189078862 0.121601604364455
0.55327669029407 0.131918542970798
0.536717478214957 0.141899359661781
0.520294105863188 0.151518620841353
0.504017041888231 0.160751813418715
0.487896664208613 0.169575407431921
0.471943253730781 0.177966916251536
0.456166988141264 0.18590495423109
0.440577935762111 0.193369291680399
0.425186049451712 0.200340907046881
0.410001160522753 0.206802036197834
0.395032972635676 0.212736218702091
0.38029105560979 0.218128341010777
0.36578483907538 0.222964676431608
0.351523605870586 0.227232921776542
0.337516485069478 0.23092223053541
0.323772444518238 0.234023242385741
0.310300282763089 0.236528108790522
0.297108620287721 0.238430514364134
0.284205890052545 0.239725693611976
0.27160032745433 0.240410442590784
0.259299960006205 0.240483125024951
0.247312597261469 0.239943672488383
0.2356458217318 0.238793578459958
0.224306981713199 0.237035886402939
0.21330318694315 0.234675172482025
0.202641307788403 0.231717524032126
0.192327978173203 0.228170515287301
0.1823696017639 0.22404318200088
0.172772360193836 0.219345996319791
0.163542221572307 0.214090842618848
0.154684947376684 0.208290994107505
0.146206096155645 0.201961089155653
0.138111023180157 0.195117105712544
0.130404876038488 0.187776332069152
0.12309258692145 0.179957332525672
0.116178862798813 0.171679907117356
0.109668174793245 0.162965045212245
0.10356474787866 0.153834873345794
0.0978725516915446 0.144312598005936
0.0925952928735381 0.134422444219678
0.0877364090495248 0.124189590767097
0.0832990643281257 0.113640102728524
0.0792861460924091 0.102800861916777
0.0757002628080311 0.0916994956001386
0.0725437425870994 0.0803643038046822
0.069818632284583 0.0688241854024739
0.0675266969522104 0.0571085631419612
0.0656694195214907 0.0452473077513642
0.0642480006271671 0.0332706612373471
0.0632633585132331 0.0212091595029022
0.0627161289858991 0.00909355441514592
0.0626066653929575 -0.00304526453801165
0.0629350386185861 -0.0151763488558262
0.0637010370885092 -0.0272687697454203
0.0649041667842718 -0.0392916968007361
0.0665436512687222 -0.0512144764858195
0.0686184317292081 -0.0630067102720941
0.0711271670520408 -0.0746383322854955
0.0740682339532131 -0.0860796863272447
0.0774397272079407 -0.0973016021399556
0.0812394600470219 -0.108275470796062
0.0854649648223936 -0.118973319084073
0.0901134940873028 -0.12936788275431
0.0951820222852561 -0.139432678452301
0.100667248289028 -0.149142074107302
0.10656559906303 -0.158471357449301
0.112873234718821 -0.167396802199979
0.119586055168304 -0.175895731332128
0.126699708426136 -0.183946576646328
0.13420960035749 -0.191528933821879
0.142110905322839 -0.198623612126504
0.150398576795015 -0.205212678179984
0.159067356720643 -0.211279493592426
0.16811178230224 -0.216808746901301
0.177526189102462 -0.221786480886398
0.187304709947427 -0.226200116851138
0.197441269929379 -0.230038477621279
0.207929578656845 -0.233291810717099
0.218763121506754 -0.235951812451675
0.229935151803763 -0.238011652801433
0.241438685552319 -0.239466000064469
0.253266499708268 -0.240311043799429
0.265411134226006 -0.240544514421553
0.277864897472831 -0.240165698073282
0.290619874197545 -0.23917544583971
0.303667935101709 -0.237576176879925
0.317000747134867 -0.235371875470324
0.330609783830706 -0.232568082244952
0.344486335235128 -0.229171880066449
0.358621517191145 -0.225191874998346
0.373006279910045 -0.220638172815656
0.387631415867 -0.215522351422736
0.402487567119295 -0.209857429471725
0.417565232169048 -0.203657831407191
0.432854772492479 -0.196939349109845
0.448346418844808 -0.189719100275549
0.46403027743119 -0.182015483643688
0.47989633601438 -0.173848131178146
0.495934470011695 -0.165237857301697
0.512134448618413 -0.156206605287707
0.528485940982374 -0.146777390919585
0.544978522444945 -0.136974243536887
0.561601680856407 -0.126822144596177
0.578344822968638 -0.116346963884112
0.595197280904322 -0.105575393529208
0.61214831869946 -0.094534879967153
0.629187138914278 -0.0832535540222168
0.646302889306651 -0.071760159274263
0.663484669561462 -0.0600839788869683
0.680721538069075 -0.0482547610782518
0.698002518745858 -0.0363026434185232
0.715316607889725 -0.0242580761462783
0.732652781063624 -0.012151744693772
};
\end{axis}


\begin{axis}[
width = \textwidth, % scaled down due to axis equal image
height = \heightRealData,
axis equal image=true,
at={(.437\textwidth, 0)},
legend cell align={left},
legend columns = 3,
legend style={
  fill opacity=1,
  draw opacity=1,
  text opacity=1,
  at={(0.0,1.18)},
  anchor=south west,
  %column sep = 0.2cm
  %draw=lightgray204
},
%tick align=outside,
%tick pos=left,
%x grid style={darkgray176},
xlabel = {$x_1$ position (m)}, 
%ylabel = {$x_2$ position (m)},
xmin=-0, xmax=1.5,
yticklabel style={xshift=.05cm},
xticklabel style={xshift=-.05cm},
%xtick style={color=black},
%y grid style={darkgray176},
ymin=-0.6, ymax=0.6,
ylabel style={yshift=-.1cm},
%ytick style={color=black}
]

\addplot [semithick, blue, opacity=\opacityRef]
table {%
0.25 -0.5
0.25044121395935 -0.499931164443115
0.254235104154476 -0.499935670346237
0.261047989235748 -0.49973607685031
0.270937379073635 -0.499938075754066
0.283079399057204 -0.499971202910929
0.297742259701645 -0.499928478263135
0.315186069277752 -0.499831912475605
0.335245616673558 -0.499905997311402
0.356763318559074 -0.499955775075133
0.376892341487477 -0.499887295557554
0.397718437572356 -0.49985236036052
0.418400380921097 -0.499871877131587
0.438928403368646 -0.499761713869845
0.459671009026121 -0.499718164143513
0.479795997094954 -0.499641389763891
0.500233158521943 -0.499556653541238
0.520951953234452 -0.499453527574414
0.541547382430445 -0.499162394589773
0.562096130970068 -0.499276363588803
0.582642085159055 -0.499367815665508
0.603412099021107 -0.499376075569593
0.624141966067834 -0.49925811234016
0.644780442686348 -0.499340580191728
0.665159903297451 -0.499392955410691
0.685781291207459 -0.499467074182603
0.706044891127332 -0.499390205729213
0.726669479767943 -0.49940085863439
0.747399242222832 -0.499345690197263
0.767562419711172 -0.499367134166475
0.788146629175432 -0.499324611492552
0.808994460439453 -0.499267543890106
0.829594562082717 -0.499232482172426
0.850373868380592 -0.499083903849469
0.870726911967299 -0.49908601424305
0.891236337787481 -0.498880127917086
0.911725770960004 -0.49907629003541
0.932202203548224 -0.498923944598756
0.952716945201064 -0.498828003738689
0.973151166531998 -0.498721406978574
0.993429917329528 -0.498742543749851
1.01419349420618 -0.498655109456756
1.03490889205821 -0.498384509274646
1.05537295736489 -0.498453808426137
1.07593036852868 -0.49826049834581
1.09647078594801 -0.498197629848586
1.11695991697414 -0.498201417095879
1.13765018716129 -0.498283180034664
1.1579945802596 -0.498427402929394
1.17835357440043 -0.498535253449149
1.19904434230778 -0.498333888338051
1.21833900797383 -0.498630324543936
1.23397752336295 -0.498791927351119
1.24752007022813 -0.498689184852918
1.25840389564815 -0.498928939868456
1.26659812158848 -0.498888171849933
1.27217141098038 -0.498877649234301
1.27538512554037 -0.498716211495967
1.27582567262694 -0.498896529762535
1.27642545057435 -0.498854382821161
1.27760352259114 -0.498669105172935
1.27747023003317 -0.498354843772649
1.27741710750697 -0.498387921025373
1.27766098855796 -0.498335177924151
1.27785861890442 -0.498043617707562
1.27778352810039 -0.497854476256444
1.27712543189221 -0.497819933202846
1.27672990755797 -0.498278271078214
1.27674330345899 -0.497640695002392
1.27684786948328 -0.497644346760698
1.27678098390378 -0.498025766793617
1.27659080690463 -0.49785096718299
1.27666596913708 -0.497910794398889
1.27647121019244 -0.49801288460762
1.27647541181842 -0.498330023934166
1.27654660819505 -0.497641197475524
1.27641550357323 -0.494007262951012
1.27633934650537 -0.486756140572346
1.27623535876846 -0.477291263800307
1.27612662954464 -0.46471915218884
1.27592949047656 -0.449665870218458
1.27559921082984 -0.432269716467741
1.27561992178159 -0.412629728882999
1.27531665250728 -0.391775583339136
1.27490175142969 -0.371581705978278
1.27474045952563 -0.351277726505063
1.2747487027877 -0.330378667070507
1.27474524760369 -0.309859530990052
1.2746282528379 -0.289162561293204
1.27445690871684 -0.268443808201729
1.27427087882979 -0.247864127982893
1.27412537089997 -0.226899403852493
1.27397544940757 -0.206436936658258
1.27397510535463 -0.186518726564789
1.27390561627959 -0.165594334424876
1.2736607294854 -0.144901607608948
1.27345751265594 -0.125107924844626
1.27340785572329 -0.104168256942669
1.27321684716118 -0.0837857835713284
1.27310224217498 -0.0630659880155764
1.27265821345109 -0.042265446795116
1.27262064456396 -0.0217184660314062
1.27241518056014 -0.00137766546391405
1.27211020337582 0.0195800787717896
1.27174621848703 0.0398670804122687
1.27154398937519 0.060563778478895
1.27110021580736 0.0806665989767141
1.27087446346148 0.101418959387672
1.27042710949174 0.122073298123766
1.27046604024927 0.142890418832278
1.26970551046406 0.163570534832012
1.26975899183855 0.183572192226935
1.26945710544206 0.204531682269999
1.26899626052133 0.2251182346703
1.26866133198007 0.245243874603269
1.26851478498389 0.265890539233749
1.26808755588741 0.285711225514297
1.26769579927839 0.306763535630952
1.266764220473 0.326937137732032
1.26699907184699 0.348197669760554
1.26702632266304 0.368463546433459
1.26675902916349 0.389059511560355
1.26644086423255 0.409122381079883
1.26637781501646 0.430453313998622
1.26602542046431 0.450974348690726
1.26581038873196 0.469788969107138
1.26530384769739 0.485283280584339
1.26516159501434 0.498722424012642
1.26497906439172 0.509476829836607
1.2651590981371 0.517756139101187
1.2648225395583 0.523388185462486
1.26474836692776 0.526218237410823
1.26477140334952 0.526787106675439
1.26468484357254 0.527603746304852
1.26473495969102 0.528504037116924
1.26382624720854 0.528162563540702
1.26417226369437 0.527263231721898
1.26415700535745 0.527670141311732
1.26421235831998 0.527614387845299
1.26429802626657 0.528030383107182
1.26355933418358 0.52789920935477
1.26447282370996 0.527568536805007
1.26399483822163 0.527438839887755
1.26388609752089 0.527371123216877
1.26405848572832 0.5272734538129
1.26364644815422 0.527330411669359
1.26317700224124 0.527305245506714
1.263511929959 0.527220425650042
1.26340135170057 0.527210517052273
1.26280815326971 0.527259539500755
1.25881925194007 0.527257501369344
1.25176586449545 0.527218449656476
1.24161000427083 0.527062204807536
1.22955402802707 0.526693602840528
1.21508364230843 0.526379678315266
1.19792607135961 0.526148696997819
1.17817218800577 0.525921955381006
1.15690639556186 0.525758795418295
1.13692305577877 0.525311682771699
1.11618674944975 0.524990582137637
1.09556223897855 0.524674270100605
1.07510238040546 0.524090195982437
1.05440749257247 0.523926182934326
1.03404862378691 0.523577874077982
1.01380072275464 0.523280961105587
0.99276641912348 0.52288208189675
0.97239890608026 0.522510126270063
0.951996959590685 0.522266150779024
0.931862699763244 0.521974512434433
0.911344067568518 0.521474785335848
0.890956282566006 0.520972371804319
0.870347186493167 0.520624843686905
0.84972010458429 0.520034870531243
0.82949212694022 0.519746878351171
0.809053241045391 0.519451815946835
0.78811144994577 0.519171654424949
0.768118653276595 0.518732422209949
0.747577411999032 0.518451930355557
0.727122510887284 0.518356645064864
0.706436722096172 0.518005218388287
0.68578105841284 0.517672836606682
0.665164874415143 0.517432026441472
0.644701633344229 0.517060428674672
0.624274045753437 0.516459468089189
0.603505958299411 0.516320244063003
0.582903308141305 0.516186492944202
0.562457807029593 0.515831374982469
0.5420124463138 0.515519888860497
0.521521879084308 0.515225522828687
0.500797938420331 0.515042589435076
0.480286081445615 0.514756342947748
0.459568970891155 0.514380798960628
0.439051381407716 0.513976922510278
0.418388295472861 0.513626860580869
0.397750375511885 0.513136612876216
0.376972488914046 0.51290042170793
0.356397125119867 0.512625601290236
0.335719111455571 0.512292810204512
0.315032239701296 0.511506437652339
0.295692505235749 0.511469097741151
0.280352888427886 0.511521471719591
0.266838080170327 0.51139034615173
0.254592669540338 0.511074326653761
0.243273393701455 0.510793972646751
0.238865486371971 0.51075801907254
0.237216172191913 0.510740747366459
0.236058193406843 0.510802634074187
0.23569081975805 0.510762895951992
0.234801745672998 0.510694095571328
0.235111528625774 0.510568618861838
0.235126146943014 0.510266313533552
0.235160227808112 0.510038552230412
0.23507934623983 0.509927223914455
0.235282576871388 0.509890088323805
0.235811752752607 0.509986005436341
0.236181090714991 0.510409373587015
0.236240568416719 0.510078859838925
0.236297535750736 0.510135828190477
0.236573564052762 0.510331218017388
0.236747183508096 0.510465121658659
0.236626762934077 0.510444733115358
0.23663882857826 0.510403725079131
0.236650140830294 0.510433226460604
0.236594650921904 0.509678465239842
0.236615595419796 0.505830922801145
0.236721528552964 0.498504339244019
0.236972911386683 0.488927542155158
0.237331458068624 0.476577631504104
0.237592545427271 0.461651716797677
0.237881998504306 0.443988566080876
0.23809789898201 0.424406858506362
0.238473300609546 0.403038945862945
0.238500660878703 0.383099769973389
0.238741098239614 0.362280121142846
0.239170978811472 0.342163328518174
0.23960638015272 0.3216500087639
0.239887763046509 0.300705332884205
0.240450757402941 0.28052268340506
0.240984269745545 0.25919218547935
0.241215547320128 0.23910489219384
0.241666873543569 0.21854162790903
0.24202506259597 0.197895316649328
0.242527892450624 0.176675680986957
0.242922094660054 0.15558709826935
0.24307236172467 0.13615063516387
0.243420903224856 0.115032722779931
0.243600925869028 0.0949801999173174
0.244027105455573 0.0746101032393195
0.244179775549788 0.0541073713920343
0.244506885124118 0.0337414455805753
0.245006566885302 0.0130523436258675
0.245464385475532 -0.00874607253648652
0.245732949267671 -0.0282862723798538
0.246078394717731 -0.0497370887949536
0.246376346834473 -0.0693048015445604
0.24650550492063 -0.0902329944715395
0.24712266633364 -0.111172988488511
0.247855609216236 -0.132040752925234
0.247724339092715 -0.151829164473388
0.24795418178514 -0.17264885115185
0.248585706342311 -0.192974942515852
0.249243452678447 -0.21439253122103
0.24944673696563 -0.23464271638067
0.249737444277056 -0.255299302819537
0.250205939771807 -0.275905054766315
0.250144372760529 -0.296234105184807
0.250808023827233 -0.316979047540672
0.251199117946172 -0.337347597798498
0.251696494843897 -0.357747358757053
0.252050651909336 -0.37816148241849
0.251957314694554 -0.399130912440649
0.252479345495718 -0.419385441344894
0.252866549211786 -0.440334388591971
0.253227739944743 -0.459862651826368
0.253367716192196 -0.47489473066109
0.253709089732487 -0.488551187413739
0.253812152474169 -0.499542234834015
0.253846589730107 -0.507458582193508
0.253921589347567 -0.513199250310685
0.253709379073851 -0.515473835018118
0.253750870692531 -0.516199383587546
};
%\addlegendentry{Reference}
\addplot [semithick, red, dashed]
table {%
0.25 -0.5
0.252558268640609 -0.499971215983182
0.257674800204547 -0.499913561943335
0.265349580546156 -0.499826822527541
0.275582586992214 -0.499710651911554
0.288373788282223 -0.499564571828345
0.303723144483551 -0.499387968751125
0.321630606880243 -0.499180090231924
0.342096117833915 -0.498940040397106
0.362561444756502 -0.498697137467738
0.383026586783779 -0.498451352589885
0.403491543037366 -0.498202656922469
0.423956312624558 -0.49795102163749
0.444420894638161 -0.497696417920264
0.464885288156327 -0.497438816969653
0.485349492242387 -0.497178189998295
0.505813505944689 -0.496914508232844
0.526277328296427 -0.4966477429142
0.546740958315482 -0.49637786529775
0.567204395004253 -0.496104846653605
0.587667637349495 -0.495828658266839
0.608130684322154 -0.495549271437727
0.628593534877201 -0.495266657481993
0.649056187953474 -0.494980787731044
0.669518642473505 -0.494691633532224
0.689980897343365 -0.49439916624905
0.710442951452497 -0.494103357261464
0.730904803673553 -0.493804177966079
0.751366452862231 -0.493501599776428
0.771827897857116 -0.493195594123215
0.792289137479511 -0.492886132454562
0.812750170533283 -0.492573186236266
0.833210995804693 -0.492256726952051
0.853671612062244 -0.491936726103822
0.874132018056508 -0.491613155211922
0.894592212519978 -0.491285985815386
0.915052194166895 -0.490955189472205
0.935511961693098 -0.49062073775958
0.955971513775856 -0.490282602274183
0.976430849073713 -0.489940754632423
0.996889966226327 -0.489595166470704
1.01734886385431 -0.489245809445691
1.03780754055907 -0.488892655234577
1.05826599492265 -0.488535675535346
1.07872422550757 -0.488174842067041
1.09918223085668 -0.487810126570036
1.11964000949299 -0.487441500806302
1.1400975599195 -0.48706893655968
1.16055488061908 -0.486692405636151
1.18101197005428 -0.486311879864113
1.20146882666719 -0.485927331094652
1.21936837110133 -0.485587306188302
1.23471068292331 -0.485293179242986
1.24749583154271 -0.485046149752488
1.25772387591864 -0.484847237098433
1.26539486430542 -0.484697275874471
1.27050883404033 -0.484596912039359
1.27306581137596 -0.484546599896135
1.2730658113589 -0.484546599895144
1.27305541155906 -0.484320911054072
1.2730238336127 -0.483870450269343
1.27294421772612 -0.483198810692961
1.27277401858219 -0.482315683800082
1.27245616754417 -0.481242223159833
1.27192174548361 -0.480017905211023
1.27109518948705 -0.478707989424014
1.26987791322201 -0.477382389353572
1.26861868448422 -0.476491262912181
1.26741310591075 -0.475930505649088
1.26634153930566 -0.475618318765773
1.26545420209816 -0.475475522024619
1.26477827798535 -0.475433459898834
1.26432554654506 -0.475438002031128
1.26409923112047 -0.475451102493
1.26409923107294 -0.475451102492838
1.26400510754809 -0.472893953643021
1.26381676720545 -0.467779675905825
1.26353397696839 -0.460108319156843
1.26315636493238 -0.449879963098326
1.26268342195054 -0.437094717113806
1.26211450389605 -0.421752720062896
1.26144883460045 -0.403854140014764
1.2606855094668 -0.383399173918339
1.25991927860775 -0.362944839063709
1.25915016511684 -0.342491133403313
1.25837819207707 -0.322038054872471
1.25760338256082 -0.301585601389565
1.25682575962964 -0.281133770856217
1.25604534633415 -0.260682561157473
1.25526216571386 -0.240231970161977
1.25447624079705 -0.219781995722154
1.25368759460061 -0.199332635674387
1.25289625012991 -0.178883887839198
1.25210223037863 -0.158435750021426
1.25130555832866 -0.137988220010405
1.25050625694995 -0.117541295580143
1.24970434920032 -0.0970949744895014
1.2488998580254 -0.0766492544823687
1.24809280635845 -0.0562041332878428
1.24728321712021 -0.0357596086204065
1.24647111321881 -0.0153156781801049
1.24565651754959 0.00512766034727754
1.244839452995 0.0255704092900399
1.24401994242446 0.0460125709903884
1.24319800869422 0.0664541478042603
1.24237367464724 0.0868951421011472
1.24154696311305 0.10733555626392
1.24071789690763 0.127775392688651
1.23988649883331 0.148214653784443
1.23905279167858 0.168653341973249
1.23821679821801 0.189091459689702
1.23737854121215 0.209529009380936
1.23653804340733 0.229965993506414
1.23569532753562 0.250402414537754
1.23485041631467 0.270838274958555
1.23400333244757 0.291273577264223
1.23315409862277 0.311708323961797
1.23230273751397 0.332142517569777
1.23144927177993 0.352576160617952
1.23059372406446 0.373009255647224
1.22973611699622 0.393441805209441
1.22887647318863 0.413873811867219
1.22801481523979 0.434305278193775
1.22715116573232 0.454736206772755
1.22628554723329 0.475166600198059
1.22552642790666 0.493042728464227
1.22487449160254 0.508364776283352
1.22433031870982 0.52113290203806
1.22389439056138 0.531347238274115
1.22356709317291 0.539007892120091
1.22334872031756 0.544114945632642
1.22323947593845 0.546668456066198
1.22323947590048 0.546668456066214
1.2230123362303 0.54664910964162
1.22255893634353 0.546599663012994
1.22188286957468 0.546493423413052
1.22099418533407 0.546288449972454
1.21991546682544 0.545929219980312
1.21868995460871 0.545349990456186
1.21739061938076 0.544480715471633
1.21610030698399 0.543230273381219
1.21526412455335 0.541961848316668
1.21476024252823 0.540762508740427
1.21448767411105 0.539705040879324
1.21435620536168 0.538833702471361
1.21429925392167 0.538171836563276
1.21427642250785 0.537729171531928
1.21429899313131 0.537497396084245
1.21429899310163 0.53749739608416
1.21173554316399 0.537351398434274
1.2066086675668 0.537059357198718
1.19891842689941 0.536621157703719
1.18866491799536 0.536036617118684
1.17584827376684 0.535305485487585
1.16046866297006 0.534427447211652
1.1425262899005 0.533402122984755
1.12202139401749 0.532229072183377
1.10151726555175 0.531054604210161
1.08101390213675 0.529878734481104
1.06051130139776 0.528701478429746
1.04000946095186 0.5275228515073
1.01950837840803 0.526342869182777
0.999008051367172 0.525161546943109
0.978508477422155 0.523978900293278
0.958009654157879 0.522794944756435
0.937511579151311 0.521609695874025
0.917014249971538 0.520423169205916
0.896517664179812 0.519235380330514
0.876021819329601 0.518046344844887
0.855526712966634 0.516856078364893
0.835032342628955 0.515664596525293
0.814538705846972 0.514471914979876
0.794045800143499 0.513278049401579
0.773553623033816 0.512083015482608
0.753062172025712 0.510886828934553
0.732571444619537 0.509689505488512
0.712081438308253 0.508491060895207
0.691592150577485 0.507291510925102
0.671103578905572 0.506090871368521
0.650615720763616 0.504889158035764
0.630128573615536 0.503686386757223
0.609642134918118 0.502482573383502
0.589156402121068 0.501277733785526
0.568671372667066 0.500071883854662
0.548187043991813 0.49886503950283
0.527703413524089 0.497657216662618
0.507220478685805 0.496448431287396
0.486738236892053 0.495238699351429
0.466256685551165 0.494028036849988
0.445775822064762 0.492816459799465
0.425295643827809 0.491603984237482
0.404816148228671 0.490390626223003
0.384337332649168 0.489176401836445
0.363859194464625 0.487961327179788
0.343381731043935 0.486745418376685
0.322904939749608 0.48552869157257
0.302428817937827 0.484311162934768
0.281953362958509 0.483092848652601
0.261478572155355 0.4818737649375
0.243563709029771 0.480806407637357
0.228208544470985 0.479891046434884
0.215412881797103 0.479127906813159
0.20517655624396 0.4785171732374
0.197499434529847 0.478058991869279
0.192381414497009 0.477753472810477
0.189822424830656 0.477600691872751
0.189822424856044 0.477600691872319
0.18984329041129 0.477361217804826
0.189891281488213 0.476888179768742
0.18998422190601 0.476195304214649
0.190156086957441 0.475302243681175
0.19046107666499 0.47423395694596
0.190973511005364 0.47302370084793
0.1917804370421 0.471720939032194
0.192990859960637 0.470375213510415
0.194255394688102 0.469435488118
0.195465103825654 0.468815206001496
0.196534139830494 0.468445902524195
0.19741290176854 0.468255645457359
0.19807765083531 0.468178447005524
0.198520413044486 0.468160326901788
0.198740901208049 0.468162462817004
0.198740901269321 0.46816246281638
0.198945440782965 0.465612270025241
0.199354508471781 0.460511864695189
0.199968076306772 0.452861197525416
0.200786100039452 0.442660189857331
0.201808520127061 0.429908733995519
0.203035263052485 0.414606693655117
0.20446624304152 0.396753904535665
0.206101364180888 0.376350175021588
0.207736171974856 0.355945830121864
0.209370679915717 0.335540874152152
0.211004901527408 0.315135311429408
0.212638850365535 0.294729146271967
0.214272540017396 0.27432238299961
0.215905984101993 0.253915025933645
0.217539196270065 0.233507079396979
0.219172190204101 0.2130985477142
0.220804979618368 0.192689435211644
0.222437578258934 0.172279746217481
0.224069999903693 0.15186948506179
0.225702258362386 0.131458656076634
0.227334367476629 0.111047263596139
0.228966341119943 0.0906353119565765
0.230598193197773 0.0702228054964383
0.232229937647521 0.0498097485565179
0.233861588438572 0.0293961454799898
0.235493159572323 0.00898200061249
0.237124665082211 -0.0114326816978031
0.238756119033746 -0.0318478971000879
0.240387535524538 -0.0522636412408578
0.24201892868433 -0.0726799097638194
0.243650312675029 -0.0930966983098097
0.245281701690739 -0.113514002516714
0.246913109957795 -0.133931818019382
0.248544551734793 -0.154350140449545
0.250176041312631 -0.174768965435729
0.251807593014535 -0.195188288603176
0.253439221196103 -0.215608105573753
0.255070940245335 -0.236028411965869
0.256702764582677 -0.256449203394391
0.258334708661049 -0.276870475470555
0.259966786965892 -0.29729222380188
0.261599014015202 -0.317714443992083
0.263231404359571 -0.338137131640988
0.264863972582227 -0.358560282344441
0.266496733299076 -0.378983891694222
0.26812970115874 -0.399407955277952
0.269762890842605 -0.419832468679009
0.271396317064857 -0.440257427476434
0.27302999457253 -0.460682827244845
0.27466393814555 -0.481108663554342
0.276093884549649 -0.498981648418472
0.277319746923832 -0.514301630525306
0.278341443563315 -0.527068481124652
0.279158900790641 -0.537282093189924
0.279772055404735 -0.544942380708266
0.280180856703051 -0.550049278099581
0.280385268072754 -0.552602739765039
};
%\addlegendentry{Koopman}
\addplot [semithick, green, dash pattern=on 1pt off 3pt on 3pt off 3pt]
table {%
0.25 -0.5
0.2525 -0.5
0.2575 -0.5
0.265 -0.5
0.275 -0.5
0.2875 -0.5
0.3025 -0.5
0.32 -0.5
0.34 -0.5
0.36 -0.5
0.38 -0.5
0.4 -0.5
0.42 -0.5
0.44 -0.5
0.46 -0.5
0.48 -0.5
0.5 -0.5
0.52 -0.5
0.54 -0.5
0.56 -0.5
0.58 -0.5
0.6 -0.5
0.62 -0.5
0.64 -0.5
0.66 -0.5
0.68 -0.5
0.7 -0.5
0.72 -0.5
0.74 -0.5
0.76 -0.5
0.78 -0.5
0.8 -0.5
0.82 -0.5
0.84 -0.5
0.86 -0.5
0.88 -0.5
0.9 -0.5
0.92 -0.5
0.94 -0.5
0.96 -0.5
0.98 -0.5
1 -0.5
1.02 -0.5
1.04 -0.5
1.06 -0.5
1.08 -0.5
1.1 -0.5
1.12 -0.5
1.14 -0.5
1.16 -0.5
1.18 -0.5
1.1975 -0.5
1.2125 -0.5
1.225 -0.5
1.235 -0.5
1.2425 -0.5
1.2475 -0.5
1.25 -0.5
1.25 -0.5
1.25 -0.5
1.25 -0.5
1.25 -0.5
1.25 -0.5
1.25 -0.5
1.25 -0.5
1.25 -0.5
1.25 -0.5
1.25 -0.5
1.25 -0.5
1.25 -0.5
1.25 -0.5
1.25 -0.5
1.25 -0.5
1.25 -0.5
1.25 -0.5
1.25 -0.4975
1.25 -0.4925
1.25 -0.485
1.25 -0.475
1.25 -0.4625
1.25 -0.4475
1.25 -0.43
1.25 -0.41
1.25 -0.39
1.25 -0.37
1.25 -0.35
1.25 -0.33
1.25 -0.31
1.25 -0.29
1.25 -0.27
1.25 -0.25
1.25 -0.23
1.25 -0.21
1.25 -0.19
1.25 -0.17
1.25 -0.15
1.25 -0.13
1.25 -0.11
1.25 -0.09
1.25 -0.07
1.25 -0.05
1.25 -0.03
1.25 -0.01
1.25 0.00999999999999998
1.25 0.03
1.25 0.05
1.25 0.07
1.25 0.09
1.25 0.11
1.25 0.13
1.25 0.15
1.25 0.17
1.25 0.19
1.25 0.21
1.25 0.23
1.25 0.25
1.25 0.27
1.25 0.29
1.25 0.31
1.25 0.33
1.25 0.35
1.25 0.37
1.25 0.39
1.25 0.41
1.25 0.43
1.25 0.4475
1.25 0.4625
1.25 0.475
1.25 0.485
1.25 0.4925
1.25 0.4975
1.25 0.5
1.25 0.5
1.25 0.5
1.25 0.5
1.25 0.5
1.25 0.5
1.25 0.5
1.25 0.5
1.25 0.5
1.25 0.5
1.25 0.5
1.25 0.5
1.25 0.5
1.25 0.5
1.25 0.5
1.25 0.5
1.25 0.5
1.25 0.5
1.2475 0.5
1.2425 0.5
1.235 0.5
1.225 0.5
1.2125 0.5
1.1975 0.5
1.18 0.5
1.16 0.5
1.14 0.5
1.12 0.5
1.1 0.5
1.08 0.5
1.06 0.5
1.04 0.5
1.02 0.5
1 0.5
0.98 0.5
0.96 0.5
0.94 0.5
0.92 0.5
0.9 0.5
0.88 0.5
0.86 0.5
0.84 0.5
0.82 0.5
0.8 0.5
0.78 0.5
0.76 0.5
0.74 0.5
0.72 0.5
0.7 0.5
0.68 0.5
0.66 0.5
0.64 0.5
0.62 0.5
0.6 0.5
0.58 0.5
0.56 0.5
0.54 0.5
0.52 0.5
0.5 0.5
0.48 0.5
0.46 0.5
0.44 0.5
0.42 0.5
0.4 0.5
0.38 0.5
0.36 0.5
0.34 0.5
0.32 0.5
0.3025 0.5
0.2875 0.5
0.275 0.5
0.265 0.5
0.2575 0.5
0.2525 0.5
0.25 0.5
0.25 0.5
0.25 0.5
0.25 0.5
0.25 0.5
0.25 0.5
0.25 0.5
0.25 0.5
0.25 0.5
0.25 0.5
0.25 0.5
0.25 0.5
0.25 0.5
0.25 0.5
0.25 0.5
0.25 0.5
0.25 0.5
0.25 0.5
0.25 0.4975
0.25 0.4925
0.25 0.485
0.25 0.475
0.25 0.4625
0.25 0.4475
0.25 0.43
0.25 0.41
0.25 0.39
0.25 0.37
0.25 0.35
0.25 0.33
0.25 0.31
0.25 0.29
0.25 0.27
0.25 0.25
0.25 0.23
0.25 0.21
0.25 0.19
0.25 0.17
0.25 0.15
0.249999999999999 0.13
0.249999999999999 0.11
0.249999999999999 0.09
0.249999999999999 0.07
0.249999999999999 0.05
0.249999999999999 0.03
0.249999999999999 0.01
0.249999999999999 -0.00999999999999998
0.249999999999999 -0.03
0.249999999999999 -0.05
0.249999999999999 -0.07
0.249999999999999 -0.09
0.249999999999999 -0.11
0.249999999999999 -0.13
0.249999999999999 -0.15
0.249999999999999 -0.17
0.249999999999999 -0.19
0.249999999999999 -0.21
0.249999999999999 -0.23
0.249999999999999 -0.25
0.249999999999999 -0.27
0.249999999999999 -0.29
0.249999999999999 -0.31
0.249999999999999 -0.33
0.249999999999999 -0.35
0.249999999999999 -0.37
0.249999999999999 -0.39
0.249999999999999 -0.41
0.249999999999999 -0.43
0.249999999999999 -0.4475
0.249999999999999 -0.4625
0.249999999999999 -0.475
0.249999999999999 -0.485
0.249999999999999 -0.4925
0.249999999999999 -0.4975
0.249999999999999 -0.5
};
%\addlegendentry{Runge-Kutta}
\addplot [semithick, blue, opacity=\opacityRef, forget plot]
table {%
0.25 -0.5
0.250626623630524 -0.4999640583992
0.254143327474594 -0.499834805727005
0.261647284030914 -0.499781876802444
0.271050184965134 -0.499684393405914
0.283128052949905 -0.499547809362411
0.298117965459824 -0.499487787485123
0.315663725137711 -0.499065339565277
0.335427910089493 -0.498745113611221
0.356708586215973 -0.498395651578903
0.376834154129028 -0.49838599562645
0.397669285535812 -0.497951716184616
0.41855126619339 -0.497398406267166
0.43925604224205 -0.497284948825836
0.459813147783279 -0.496974140405655
0.480303317308426 -0.496602952480316
0.500866502523422 -0.49622443318367
0.521093875169754 -0.495945930480957
0.541718810796738 -0.495497435331345
0.562221497297287 -0.495077878236771
0.582824140787125 -0.494835823774338
0.603334754705429 -0.494807153940201
0.624194294214249 -0.494503647089005
0.644715279340744 -0.494147837162018
0.665623873472214 -0.493835031986237
0.685807377099991 -0.49360778927803
0.706791788339615 -0.493293046951294
0.727184325456619 -0.492925822734833
0.747520178556442 -0.492770165205002
0.768008261919022 -0.492363154888153
0.788260668516159 -0.491896241903305
0.808965712785721 -0.491626143455505
0.829519957304001 -0.491620659828186
0.850416213274002 -0.491290330886841
0.870734423398972 -0.490930408239365
0.891558200120926 -0.490548998117447
0.912192851305008 -0.49023312330246
0.932792872190475 -0.48986142873764
0.953147679567337 -0.489463239908218
0.973350077867508 -0.489341855049133
0.993815273046494 -0.489013433456421
1.01432123780251 -0.488179057836533
1.03478077054024 -0.488025933504105
1.05544748902321 -0.487651526927948
1.0760124027729 -0.487186670303345
1.09666863083839 -0.487088084220886
1.11745718121529 -0.486630618572235
1.1379317343235 -0.486422836780548
1.15859189629555 -0.486338883638382
1.17875507473946 -0.486253321170807
1.19946792721748 -0.485877931118011
1.21861937642097 -0.485662549734116
1.23404923081398 -0.485652446746826
1.24776795506477 -0.485187500715256
1.25812175869942 -0.485116988420486
1.26648584008217 -0.485285639762878
1.27218607068062 -0.485255122184753
1.27520814538002 -0.485251098871231
1.27562084794044 -0.485235989093781
1.27638164162636 -0.485047996044159
1.27739146351814 -0.485070586204529
1.27736583352089 -0.484908252954483
1.27740588784218 -0.484728306531906
1.27766624093056 -0.484787344932556
1.27750506997108 -0.484193086624146
1.27734604477882 -0.484226614236832
1.27711048722267 -0.483972519636154
1.27693405747414 -0.484230518341064
1.27689066529274 -0.484090864658356
1.27680531144142 -0.484015762805939
1.27659991383553 -0.484249621629715
1.27652052044868 -0.484301090240479
1.27647343277931 -0.484401613473892
1.27639964222908 -0.484391152858734
1.27643147110939 -0.484559386968613
1.27641335129738 -0.484334588050842
1.2762795984745 -0.48060730099678
1.27602997422218 -0.472728401422501
1.27583363652229 -0.46323361992836
1.27538219094276 -0.450839191675186
1.27500334382057 -0.436138212680817
1.27438524365425 -0.418473154306412
1.27368953824043 -0.398340255022049
1.27302923798561 -0.377564251422882
1.27230432629585 -0.357200980186462
1.27165046334267 -0.337186634540558
1.27113941311836 -0.31614288687706
1.27056947350502 -0.295828610658646
1.26987591385841 -0.275413930416107
1.26933470368385 -0.254779815673828
1.26883962750435 -0.234159916639328
1.2683496773243 -0.213058680295944
1.26772680878639 -0.192717403173447
1.2671745121479 -0.172182992100716
1.26653125882149 -0.151792481541634
1.26580455899239 -0.131056264042854
1.26521852612495 -0.110953733325005
1.26468542218208 -0.09096410125494
1.26408544182777 -0.0698892846703529
1.26338282227516 -0.0494239032268524
1.26263120770454 -0.028490049764514
1.26204958558083 -0.00787987373769283
1.26145032048225 0.0127129359170794
1.26068916916847 0.0330536998808384
1.25982597470284 0.0539086908102036
1.25934639573097 0.0745700001716614
1.25818791985512 0.0942444205284119
1.25753799080849 0.11544231325388
1.25675871968269 0.135453969240189
1.25629606842995 0.156645938754082
1.25537350773811 0.17669814825058
1.25459614396095 0.197236001491547
1.25364282727242 0.21828131377697
1.25323036313057 0.238531857728958
1.25212529301643 0.259153932332993
1.25151005387306 0.279964983463287
1.25059071183205 0.300346255302429
1.24981489777565 0.320940971374512
1.24867704510689 0.341317862272263
1.24833682179451 0.362009644508362
1.2476010620594 0.381608009338379
1.24681225419044 0.402668863534927
1.24604347348213 0.423369139432907
1.24516108632088 0.443871706724167
1.24451461434364 0.464342027902603
1.2437826693058 0.483703851699829
1.24355688691139 0.499095916748047
1.24269226193428 0.51423704624176
1.24251583218575 0.528313517570496
1.24163070321083 0.53537392616272
1.24140980839729 0.539457619190216
1.24129071831703 0.543660581111908
1.24129143357277 0.543900430202484
1.24160757660866 0.544277608394623
1.24184218049049 0.545334458351135
1.24112048745155 0.544964075088501
1.24144986271858 0.545125782489777
1.24140766263008 0.544974446296692
1.24122121930122 0.544937431812286
1.24169543385506 0.54449850320816
1.2413811981678 0.544275999069214
1.2416355907917 0.544522702693939
1.24148717522621 0.544396817684174
1.24125161767006 0.544159591197968
1.24111714959145 0.544052481651306
1.2410184442997 0.543937861919403
1.24105575680733 0.543873369693756
1.24129298329353 0.543893218040466
1.24114230275154 0.543918311595917
1.2407206594944 0.543901026248932
1.23710384964943 0.543823778629303
1.22973111271858 0.543312072753906
1.21997573971748 0.542805314064026
1.20736292004585 0.542218863964081
1.19236591458321 0.541527032852173
1.17530909180641 0.540697813034058
1.15540292859077 0.539658725261688
1.13499453663826 0.53869241476059
1.11489060521126 0.537730395793915
1.0940959751606 0.536899209022522
1.07353332638741 0.535995006561279
1.05227974057198 0.535142540931702
1.03177264332771 0.5339115858078
1.01226213574409 0.532896339893341
0.991688102483749 0.531981706619263
0.970780402421951 0.531027972698212
0.950629144906998 0.530097544193268
0.930080026388168 0.529099941253662
0.909932523965836 0.528094470500946
0.889512449502945 0.527039170265198
0.868990868330002 0.526078164577484
0.848793834447861 0.525151431560516
0.828342765569687 0.524191975593567
0.807693272829056 0.522945463657379
0.787055939435959 0.522022306919098
0.766432136297226 0.521006524562836
0.745935708284378 0.520236194133759
0.725773483514786 0.519186198711395
0.705125600099564 0.518240392208099
0.684522598981857 0.517357110977173
0.664170533418655 0.516431450843811
0.643560200929642 0.515549600124359
0.623092621564865 0.514686465263367
0.602387279272079 0.51365715265274
0.581707030534744 0.512577533721924
0.561537593603134 0.511792242527008
0.540475398302078 0.510728180408478
0.52023121714592 0.509848415851593
0.499731212854385 0.509034872055054
0.479135274887085 0.508061528205872
0.458688199520111 0.506969332695007
0.438110798597336 0.506107866764069
0.417646616697311 0.505182445049286
0.397079676389694 0.504226088523865
0.376451283693314 0.503288388252258
0.356210917234421 0.501943290233612
0.334821581840515 0.50134152173996
0.314330250024796 0.500298321247101
0.293708324432373 0.499351501464844
0.274019211530685 0.498499572277069
0.258903026580811 0.497682183980942
0.245129689574242 0.496951282024384
0.234368741512299 0.496295303106308
0.226341605186462 0.49594059586525
0.220549017190933 0.495828539133072
0.217578321695328 0.495528101921082
0.216456174850464 0.495686322450638
0.216101661324501 0.495550066232681
0.21531680226326 0.495361983776093
0.215485200285912 0.495296269655228
0.215603470802307 0.495336979627609
0.21526837348938 0.495314985513687
0.215267866849899 0.495107412338257
0.215479135513306 0.495046675205231
0.215885400772095 0.495135009288788
0.216280862689018 0.495393961668015
0.216382697224617 0.495136320590973
0.216351106762886 0.494924873113632
0.216455727815628 0.495169579982758
0.216500520706177 0.495239078998566
0.216571867465973 0.495138168334961
0.21662849187851 0.495208084583282
0.216702371835709 0.495233148336411
0.216725572943687 0.494929254055023
0.216948851943016 0.491341561079025
0.217357978224754 0.483906656503677
0.217821598052979 0.474306136369705
0.218451097607613 0.462025582790375
0.219455450773239 0.447092890739441
0.22027787566185 0.42986923456192
0.221378177404404 0.409976720809937
0.222699657082558 0.388913244009018
0.223804831504822 0.368585705757141
0.22507306933403 0.348034203052521
0.226461857557297 0.327520459890366
0.22760097682476 0.30724424123764
0.228943765163422 0.28706169128418
0.230409726500511 0.266277194023132
0.231518611311913 0.246032297611237
0.233083993196487 0.224420636892319
0.234180569648743 0.205027222633362
0.235480517148972 0.183470144867897
0.236717969179153 0.162961259484291
0.238030061125755 0.141980037093163
0.239460453391075 0.121898606419563
0.240506768226624 0.101452693343163
0.241758778691292 0.0808084383606911
0.24299094080925 0.0603051036596298
0.244483157992363 0.039212916046381
0.245459228754044 0.0194344762712717
0.246727466583252 -0.000837862258777022
0.247780233621597 -0.0213230811059475
0.2491215467453 -0.0421212762594223
0.250642359256744 -0.0635988488793373
0.251408636569977 -0.0830889791250229
0.252860933542252 -0.103880919516087
0.254020899534225 -0.124866724014282
0.255651354789734 -0.146131694316864
0.2564717233181 -0.165398985147476
0.257876396179199 -0.186467424035072
0.259179919958115 -0.206335827708244
0.260364949703217 -0.227949485182762
0.261355489492416 -0.247603759169579
0.262839823961258 -0.268183946609497
0.264208853244781 -0.288938909769058
0.265331625938416 -0.309380769729614
0.266708374023438 -0.329894155263901
0.267769992351532 -0.350864857435226
0.269173264503479 -0.371044129133224
0.270285248756409 -0.391378462314606
0.271676391363144 -0.412104427814484
0.27273815870285 -0.432635307312012
0.274118006229401 -0.453247755765915
0.275076866149902 -0.472338557243347
0.275904506444931 -0.48834303021431
0.276771157979965 -0.501835346221924
0.277361661195755 -0.51248162984848
0.277772098779678 -0.520788967609406
0.27814057469368 -0.526734232902527
0.278336614370346 -0.52966833114624
0.278412491083145 -0.530814170837402
};
\addplot [semithick, red, dashed, forget plot]
table {%
0.25 -0.5
0.252558268640609 -0.499971215983182
0.257674800204547 -0.499913561943335
0.265349580546156 -0.499826822527541
0.275582586992214 -0.499710651911554
0.288373788282223 -0.499564571828345
0.303723144483551 -0.499387968751125
0.321630606880243 -0.499180090231924
0.342096117833915 -0.498940040397106
0.362561444756502 -0.498697137467738
0.383026586783779 -0.498451352589885
0.403491543037366 -0.498202656922469
0.423956312624558 -0.49795102163749
0.444420894638161 -0.497696417920264
0.464885288156327 -0.497438816969653
0.485349492242387 -0.497178189998295
0.505813505944689 -0.496914508232844
0.526277328296427 -0.4966477429142
0.546740958315482 -0.49637786529775
0.567204395004253 -0.496104846653605
0.587667637349495 -0.495828658266839
0.608130684322154 -0.495549271437727
0.628593534877201 -0.495266657481993
0.649056187953474 -0.494980787731044
0.669518642473505 -0.494691633532224
0.689980897343365 -0.49439916624905
0.710442951452497 -0.494103357261464
0.730904803673553 -0.493804177966079
0.751366452862231 -0.493501599776428
0.771827897857116 -0.493195594123215
0.792289137479511 -0.492886132454562
0.812750170533283 -0.492573186236266
0.833210995804693 -0.492256726952051
0.853671612062244 -0.491936726103822
0.874132018056508 -0.491613155211922
0.894592212519978 -0.491285985815386
0.915052194166895 -0.490955189472205
0.935511961693098 -0.49062073775958
0.955971513775856 -0.490282602274183
0.976430849073713 -0.489940754632423
0.996889966226327 -0.489595166470704
1.01734886385431 -0.489245809445691
1.03780754055907 -0.488892655234577
1.05826599492265 -0.488535675535346
1.07872422550757 -0.488174842067041
1.09918223085668 -0.487810126570036
1.11964000949299 -0.487441500806302
1.1400975599195 -0.48706893655968
1.16055488061908 -0.486692405636151
1.18101197005428 -0.486311879864113
1.20146882666719 -0.485927331094652
1.21936837110133 -0.485587306188302
1.23471068292331 -0.485293179242986
1.24749583154271 -0.485046149752488
1.25772387591864 -0.484847237098433
1.26539486430542 -0.484697275874471
1.27050883404033 -0.484596912039359
1.27306581137596 -0.484546599896135
1.2730658113589 -0.484546599895144
1.27305541155906 -0.484320911054072
1.2730238336127 -0.483870450269343
1.27294421772612 -0.483198810692961
1.27277401858219 -0.482315683800082
1.27245616754417 -0.481242223159833
1.27192174548361 -0.480017905211023
1.27109518948705 -0.478707989424014
1.26987791322201 -0.477382389353572
1.26861868448422 -0.476491262912181
1.26741310591075 -0.475930505649088
1.26634153930566 -0.475618318765773
1.26545420209816 -0.475475522024619
1.26477827798535 -0.475433459898834
1.26432554654506 -0.475438002031128
1.26409923112047 -0.475451102493
1.26409923107294 -0.475451102492838
1.26400510754809 -0.472893953643021
1.26381676720545 -0.467779675905825
1.26353397696839 -0.460108319156843
1.26315636493238 -0.449879963098326
1.26268342195054 -0.437094717113806
1.26211450389605 -0.421752720062896
1.26144883460045 -0.403854140014764
1.2606855094668 -0.383399173918339
1.25991927860775 -0.362944839063709
1.25915016511684 -0.342491133403313
1.25837819207707 -0.322038054872471
1.25760338256082 -0.301585601389565
1.25682575962964 -0.281133770856217
1.25604534633415 -0.260682561157473
1.25526216571386 -0.240231970161977
1.25447624079705 -0.219781995722154
1.25368759460061 -0.199332635674387
1.25289625012991 -0.178883887839198
1.25210223037863 -0.158435750021426
1.25130555832866 -0.137988220010405
1.25050625694995 -0.117541295580143
1.24970434920032 -0.0970949744895014
1.2488998580254 -0.0766492544823687
1.24809280635845 -0.0562041332878428
1.24728321712021 -0.0357596086204065
1.24647111321881 -0.0153156781801049
1.24565651754959 0.00512766034727754
1.244839452995 0.0255704092900399
1.24401994242446 0.0460125709903884
1.24319800869422 0.0664541478042603
1.24237367464724 0.0868951421011472
1.24154696311305 0.10733555626392
1.24071789690763 0.127775392688651
1.23988649883331 0.148214653784443
1.23905279167858 0.168653341973249
1.23821679821801 0.189091459689702
1.23737854121215 0.209529009380936
1.23653804340733 0.229965993506414
1.23569532753562 0.250402414537754
1.23485041631467 0.270838274958555
1.23400333244757 0.291273577264223
1.23315409862277 0.311708323961797
1.23230273751397 0.332142517569777
1.23144927177993 0.352576160617952
1.23059372406446 0.373009255647224
1.22973611699622 0.393441805209441
1.22887647318863 0.413873811867219
1.22801481523979 0.434305278193775
1.22715116573232 0.454736206772755
1.22628554723329 0.475166600198059
1.22552642790666 0.493042728464227
1.22487449160254 0.508364776283352
1.22433031870982 0.52113290203806
1.22389439056138 0.531347238274115
1.22356709317291 0.539007892120091
1.22334872031756 0.544114945632642
1.22323947593845 0.546668456066198
1.22323947590048 0.546668456066214
1.2230123362303 0.54664910964162
1.22255893634353 0.546599663012994
1.22188286957468 0.546493423413052
1.22099418533407 0.546288449972454
1.21991546682544 0.545929219980312
1.21868995460871 0.545349990456186
1.21739061938076 0.544480715471633
1.21610030698399 0.543230273381219
1.21526412455335 0.541961848316668
1.21476024252823 0.540762508740427
1.21448767411105 0.539705040879324
1.21435620536168 0.538833702471361
1.21429925392167 0.538171836563276
1.21427642250785 0.537729171531928
1.21429899313131 0.537497396084245
1.21429899310163 0.53749739608416
1.21173554316399 0.537351398434274
1.2066086675668 0.537059357198718
1.19891842689941 0.536621157703719
1.18866491799536 0.536036617118684
1.17584827376684 0.535305485487585
1.16046866297006 0.534427447211652
1.1425262899005 0.533402122984755
1.12202139401749 0.532229072183377
1.10151726555175 0.531054604210161
1.08101390213675 0.529878734481104
1.06051130139776 0.528701478429746
1.04000946095186 0.5275228515073
1.01950837840803 0.526342869182777
0.999008051367172 0.525161546943109
0.978508477422155 0.523978900293278
0.958009654157879 0.522794944756435
0.937511579151311 0.521609695874025
0.917014249971538 0.520423169205916
0.896517664179812 0.519235380330514
0.876021819329601 0.518046344844887
0.855526712966634 0.516856078364893
0.835032342628955 0.515664596525293
0.814538705846972 0.514471914979876
0.794045800143499 0.513278049401579
0.773553623033816 0.512083015482608
0.753062172025712 0.510886828934553
0.732571444619537 0.509689505488512
0.712081438308253 0.508491060895207
0.691592150577485 0.507291510925102
0.671103578905572 0.506090871368521
0.650615720763616 0.504889158035764
0.630128573615536 0.503686386757223
0.609642134918118 0.502482573383502
0.589156402121068 0.501277733785526
0.568671372667066 0.500071883854662
0.548187043991813 0.49886503950283
0.527703413524089 0.497657216662618
0.507220478685805 0.496448431287396
0.486738236892053 0.495238699351429
0.466256685551165 0.494028036849988
0.445775822064762 0.492816459799465
0.425295643827809 0.491603984237482
0.404816148228671 0.490390626223003
0.384337332649168 0.489176401836445
0.363859194464625 0.487961327179788
0.343381731043935 0.486745418376685
0.322904939749608 0.48552869157257
0.302428817937827 0.484311162934768
0.281953362958509 0.483092848652601
0.261478572155355 0.4818737649375
0.243563709029771 0.480806407637357
0.228208544470985 0.479891046434884
0.215412881797103 0.479127906813159
0.20517655624396 0.4785171732374
0.197499434529847 0.478058991869279
0.192381414497009 0.477753472810477
0.189822424830656 0.477600691872751
0.189822424856044 0.477600691872319
0.18984329041129 0.477361217804826
0.189891281488213 0.476888179768742
0.18998422190601 0.476195304214649
0.190156086957441 0.475302243681175
0.19046107666499 0.47423395694596
0.190973511005364 0.47302370084793
0.1917804370421 0.471720939032194
0.192990859960637 0.470375213510415
0.194255394688102 0.469435488118
0.195465103825654 0.468815206001496
0.196534139830494 0.468445902524195
0.19741290176854 0.468255645457359
0.19807765083531 0.468178447005524
0.198520413044486 0.468160326901788
0.198740901208049 0.468162462817004
0.198740901269321 0.46816246281638
0.198945440782965 0.465612270025241
0.199354508471781 0.460511864695189
0.199968076306772 0.452861197525416
0.200786100039452 0.442660189857331
0.201808520127061 0.429908733995519
0.203035263052485 0.414606693655117
0.20446624304152 0.396753904535665
0.206101364180888 0.376350175021588
0.207736171974856 0.355945830121864
0.209370679915717 0.335540874152152
0.211004901527408 0.315135311429408
0.212638850365535 0.294729146271967
0.214272540017396 0.27432238299961
0.215905984101993 0.253915025933645
0.217539196270065 0.233507079396979
0.219172190204101 0.2130985477142
0.220804979618368 0.192689435211644
0.222437578258934 0.172279746217481
0.224069999903693 0.15186948506179
0.225702258362386 0.131458656076634
0.227334367476629 0.111047263596139
0.228966341119943 0.0906353119565765
0.230598193197773 0.0702228054964383
0.232229937647521 0.0498097485565179
0.233861588438572 0.0293961454799898
0.235493159572323 0.00898200061249
0.237124665082211 -0.0114326816978031
0.238756119033746 -0.0318478971000879
0.240387535524538 -0.0522636412408578
0.24201892868433 -0.0726799097638194
0.243650312675029 -0.0930966983098097
0.245281701690739 -0.113514002516714
0.246913109957795 -0.133931818019382
0.248544551734793 -0.154350140449545
0.250176041312631 -0.174768965435729
0.251807593014535 -0.195188288603176
0.253439221196103 -0.215608105573753
0.255070940245335 -0.236028411965869
0.256702764582677 -0.256449203394391
0.258334708661049 -0.276870475470555
0.259966786965892 -0.29729222380188
0.261599014015202 -0.317714443992083
0.263231404359571 -0.338137131640988
0.264863972582227 -0.358560282344441
0.266496733299076 -0.378983891694222
0.26812970115874 -0.399407955277952
0.269762890842605 -0.419832468679009
0.271396317064857 -0.440257427476434
0.27302999457253 -0.460682827244845
0.27466393814555 -0.481108663554342
0.276093884549649 -0.498981648418472
0.277319746923832 -0.514301630525306
0.278341443563315 -0.527068481124652
0.279158900790641 -0.537282093189924
0.279772055404735 -0.544942380708266
0.280180856703051 -0.550049278099581
0.280385268072754 -0.552602739765039
};
\addplot [semithick, green, dash pattern=on 1pt off 3pt on 3pt off 3pt, forget plot]
table {%
0.25 -0.5
0.2525 -0.5
0.2575 -0.5
0.265 -0.5
0.275 -0.5
0.2875 -0.5
0.3025 -0.5
0.32 -0.5
0.34 -0.5
0.36 -0.5
0.38 -0.5
0.4 -0.5
0.42 -0.5
0.44 -0.5
0.46 -0.5
0.48 -0.5
0.5 -0.5
0.52 -0.5
0.54 -0.5
0.56 -0.5
0.58 -0.5
0.6 -0.5
0.62 -0.5
0.64 -0.5
0.66 -0.5
0.68 -0.5
0.7 -0.5
0.72 -0.5
0.74 -0.5
0.76 -0.5
0.78 -0.5
0.8 -0.5
0.82 -0.5
0.84 -0.5
0.86 -0.5
0.88 -0.5
0.9 -0.5
0.92 -0.5
0.94 -0.5
0.96 -0.5
0.98 -0.5
1 -0.5
1.02 -0.5
1.04 -0.5
1.06 -0.5
1.08 -0.5
1.1 -0.5
1.12 -0.5
1.14 -0.5
1.16 -0.5
1.18 -0.5
1.1975 -0.5
1.2125 -0.5
1.225 -0.5
1.235 -0.5
1.2425 -0.5
1.2475 -0.5
1.25 -0.5
1.25 -0.5
1.25 -0.5
1.25 -0.5
1.25 -0.5
1.25 -0.5
1.25 -0.5
1.25 -0.5
1.25 -0.5
1.25 -0.5
1.25 -0.5
1.25 -0.5
1.25 -0.5
1.25 -0.5
1.25 -0.5
1.25 -0.5
1.25 -0.5
1.25 -0.5
1.25 -0.4975
1.25 -0.4925
1.25 -0.485
1.25 -0.475
1.25 -0.4625
1.25 -0.4475
1.25 -0.43
1.25 -0.41
1.25 -0.39
1.25 -0.37
1.25 -0.35
1.25 -0.33
1.25 -0.31
1.25 -0.29
1.25 -0.27
1.25 -0.25
1.25 -0.23
1.25 -0.21
1.25 -0.19
1.25 -0.17
1.25 -0.15
1.25 -0.13
1.25 -0.11
1.25 -0.09
1.25 -0.07
1.25 -0.05
1.25 -0.03
1.25 -0.01
1.25 0.00999999999999998
1.25 0.03
1.25 0.05
1.25 0.07
1.25 0.09
1.25 0.11
1.25 0.13
1.25 0.15
1.25 0.17
1.25 0.19
1.25 0.21
1.25 0.23
1.25 0.25
1.25 0.27
1.25 0.29
1.25 0.31
1.25 0.33
1.25 0.35
1.25 0.37
1.25 0.39
1.25 0.41
1.25 0.43
1.25 0.4475
1.25 0.4625
1.25 0.475
1.25 0.485
1.25 0.4925
1.25 0.4975
1.25 0.5
1.25 0.5
1.25 0.5
1.25 0.5
1.25 0.5
1.25 0.5
1.25 0.5
1.25 0.5
1.25 0.5
1.25 0.5
1.25 0.5
1.25 0.5
1.25 0.5
1.25 0.5
1.25 0.5
1.25 0.5
1.25 0.5
1.25 0.5
1.2475 0.5
1.2425 0.5
1.235 0.5
1.225 0.5
1.2125 0.5
1.1975 0.5
1.18 0.5
1.16 0.5
1.14 0.5
1.12 0.5
1.1 0.5
1.08 0.5
1.06 0.5
1.04 0.5
1.02 0.5
1 0.5
0.98 0.5
0.96 0.5
0.94 0.5
0.92 0.5
0.9 0.5
0.88 0.5
0.86 0.5
0.84 0.5
0.82 0.5
0.8 0.5
0.78 0.5
0.76 0.5
0.74 0.5
0.72 0.5
0.7 0.5
0.68 0.5
0.66 0.5
0.64 0.5
0.62 0.5
0.6 0.5
0.58 0.5
0.56 0.5
0.54 0.5
0.52 0.5
0.5 0.5
0.48 0.5
0.46 0.5
0.44 0.5
0.42 0.5
0.4 0.5
0.38 0.5
0.36 0.5
0.34 0.5
0.32 0.5
0.3025 0.5
0.2875 0.5
0.275 0.5
0.265 0.5
0.2575 0.5
0.2525 0.5
0.25 0.5
0.25 0.5
0.25 0.5
0.25 0.5
0.25 0.5
0.25 0.5
0.25 0.5
0.25 0.5
0.25 0.5
0.25 0.5
0.25 0.5
0.25 0.5
0.25 0.5
0.25 0.5
0.25 0.5
0.25 0.5
0.25 0.5
0.25 0.5
0.25 0.4975
0.25 0.4925
0.25 0.485
0.25 0.475
0.25 0.4625
0.25 0.4475
0.25 0.43
0.25 0.41
0.25 0.39
0.25 0.37
0.25 0.35
0.25 0.33
0.25 0.31
0.25 0.29
0.25 0.27
0.25 0.25
0.25 0.23
0.25 0.21
0.25 0.19
0.25 0.17
0.25 0.15
0.249999999999999 0.13
0.249999999999999 0.11
0.249999999999999 0.09
0.249999999999999 0.07
0.249999999999999 0.05
0.249999999999999 0.03
0.249999999999999 0.01
0.249999999999999 -0.00999999999999998
0.249999999999999 -0.03
0.249999999999999 -0.05
0.249999999999999 -0.07
0.249999999999999 -0.09
0.249999999999999 -0.11
0.249999999999999 -0.13
0.249999999999999 -0.15
0.249999999999999 -0.17
0.249999999999999 -0.19
0.249999999999999 -0.21
0.249999999999999 -0.23
0.249999999999999 -0.25
0.249999999999999 -0.27
0.249999999999999 -0.29
0.249999999999999 -0.31
0.249999999999999 -0.33
0.249999999999999 -0.35
0.249999999999999 -0.37
0.249999999999999 -0.39
0.249999999999999 -0.41
0.249999999999999 -0.43
0.249999999999999 -0.4475
0.249999999999999 -0.4625
0.249999999999999 -0.475
0.249999999999999 -0.485
0.249999999999999 -0.4925
0.249999999999999 -0.4975
0.249999999999999 -0.5
};
\addplot [semithick, blue, opacity=\opacityRef, forget plot]
table {%
0.25 -0.5
0.250387400388718 -0.499840557575226
0.254600703716278 -0.500066995620728
0.262050539255142 -0.500173151493073
0.271828055381775 -0.500495672225952
0.284392893314362 -0.500795960426331
0.299484193325043 -0.50110137462616
0.316835165023804 -0.501103460788727
0.336975961923599 -0.501315355300903
0.358016818761826 -0.501693725585938
0.378679066896439 -0.501844167709351
0.399204462766647 -0.502166628837585
0.419626981019974 -0.502508819103241
0.440134823322296 -0.502709925174713
0.460389643907547 -0.502659559249878
0.481351971626282 -0.502916157245636
0.501570403575897 -0.503284811973572
0.522484838962555 -0.503642857074738
0.54284930229187 -0.5036700963974
0.563834130764008 -0.504003822803497
0.584381103515625 -0.504437804222107
0.604971051216125 -0.504667043685913
0.625414252281189 -0.504969894886017
0.645705759525299 -0.505327224731445
0.666398048400879 -0.50558203458786
0.686713099479675 -0.505653262138367
0.707380175590515 -0.506099581718445
0.727990329265594 -0.506622552871704
0.748577058315277 -0.5069220662117
0.769176125526428 -0.50719553232193
0.789992868900299 -0.507420063018799
0.810353398323059 -0.507832586765289
0.830967009067535 -0.507993459701538
0.851495802402496 -0.50809234380722
0.871890366077423 -0.508449256420135
0.892439603805542 -0.50881952047348
0.912704229354858 -0.508566796779633
0.933314442634583 -0.509141981601715
0.953807830810547 -0.509569764137268
0.974637389183044 -0.509565472602844
0.994934022426605 -0.509932935237885
1.01578336954117 -0.510145902633667
1.03590506315231 -0.510392725467682
1.05671435594559 -0.510371387004852
1.07730263471603 -0.510595500469208
1.09770482778549 -0.511246681213379
1.1183003783226 -0.511480510234833
1.13838678598404 -0.511619210243225
1.15937894582748 -0.512004733085632
1.17995971441269 -0.512618064880371
1.20063143968582 -0.513133943080902
1.21983188390732 -0.51331353187561
1.23546665906906 -0.513472318649292
1.24892181158066 -0.51392537355423
1.25946897268295 -0.513965725898743
1.26788467168808 -0.514159321784973
1.27362304925919 -0.514162659645081
1.27692919969559 -0.51420384645462
1.27751249074936 -0.514217495918274
1.27797681093216 -0.51424503326416
1.27875000238419 -0.514188945293427
1.27850049734116 -0.51412045955658
1.27859979867935 -0.514146447181702
1.27858680486679 -0.513854384422302
1.27838677167892 -0.513781309127808
1.27812570333481 -0.513472139835358
1.27794605493546 -0.513176620006561
1.27766197919846 -0.513734877109528
1.27768367528915 -0.513251602649689
1.27729862928391 -0.513548016548157
1.27713960409164 -0.513691306114197
1.27719444036484 -0.513572692871094
1.27708727121353 -0.51356965303421
1.27709144353867 -0.513332068920135
1.27715963125229 -0.513802945613861
1.27709430456161 -0.513152778148651
1.27710801362991 -0.509629368782043
1.27712053060532 -0.502204239368439
1.27712959051132 -0.492639601230621
1.27702158689499 -0.480175048112869
1.2770511507988 -0.465470880270004
1.27716225385666 -0.448033362627029
1.27732747793198 -0.428329139947891
1.27739471197128 -0.407021969556808
1.27751690149307 -0.386865854263306
1.27767926454544 -0.366162180900574
1.27796405553818 -0.345758855342865
1.27814251184464 -0.324733436107635
1.27820163965225 -0.304685264825821
1.27840667963028 -0.283993929624557
1.27852112054825 -0.26375824213028
1.27860504388809 -0.242760643362999
1.27878576517105 -0.2225221991539
1.27912026643753 -0.202024921774864
1.27920109033585 -0.18130287528038
1.27934736013412 -0.160998627543449
1.27956026792526 -0.140187188982964
1.28000646829605 -0.119596414268017
1.28015583753586 -0.0987944230437279
1.28007119894028 -0.0786774829030037
1.280328810215 -0.0580364614725113
1.28028863668442 -0.0374215245246887
1.28026396036148 -0.0167231149971485
1.28050845861435 0.00333848083391786
1.28072720766068 0.023687582463026
1.2808056473732 0.0447225235402584
1.28077918291092 0.0654603913426399
1.28103810548782 0.0860754549503326
1.28108388185501 0.10629265755415
1.28113371133804 0.127279072999954
1.28111881017685 0.147937700152397
1.28097099065781 0.168234884738922
1.28088289499283 0.188856825232506
1.28046780824661 0.209297686815262
1.28070765733719 0.22962374985218
1.28078359365463 0.249933749437332
1.28075450658798 0.270790785551071
1.28082972764969 0.291614443063736
1.28075164556503 0.311969369649887
1.28117948770523 0.332645893096924
1.28113049268723 0.353386372327805
1.28099495172501 0.374091148376465
1.28123885393143 0.394736468791962
1.28090196847916 0.414587587118149
1.28053003549576 0.435371816158295
1.28067487478256 0.454240560531616
1.28072327375412 0.46985000371933
1.28071564435959 0.483144551515579
1.28072661161423 0.493906021118164
1.28065711259842 0.501982688903809
1.28035491704941 0.507666349411011
1.28099936246872 0.510650813579559
1.28067511320114 0.511268734931946
1.28065377473831 0.511698126792908
1.28028923273087 0.51272988319397
1.28092247247696 0.512296378612518
1.28039222955704 0.512360870838165
1.28026515245438 0.512438893318176
1.28024357557297 0.512456655502319
1.28035968542099 0.512169480323792
1.27984696626663 0.511659562587738
1.28081041574478 0.511506795883179
1.27948075532913 0.511743068695068
1.27932292222977 0.511886119842529
1.27984148263931 0.511780619621277
1.27955776453018 0.511788368225098
1.27987712621689 0.511691033840179
1.27976948022842 0.511823415756226
1.27934199571609 0.511771976947784
1.27914196252823 0.511767089366913
1.27539855241776 0.51166969537735
1.26839417219162 0.511552214622498
1.25883394479752 0.511410474777222
1.24645394086838 0.51129949092865
1.23138576745987 0.511269271373749
1.21351903676987 0.511024057865143
1.19412487745285 0.510756969451904
1.17310279607773 0.510403633117676
1.15278083086014 0.510125637054443
1.13204902410507 0.509831428527832
1.11199349164963 0.509574353694916
1.09075504541397 0.509481847286224
1.07044607400894 0.509246826171875
1.05024594068527 0.509027183055878
1.02996426820755 0.508741915225983
1.00910896062851 0.5085089802742
0.988598465919495 0.508460640907288
0.968506872653961 0.508184731006622
0.94752824306488 0.507818818092346
0.92749148607254 0.507445871829987
0.90681403875351 0.507117033004761
0.886633813381195 0.506811857223511
0.866109490394592 0.506547927856445
0.845836937427521 0.506334722042084
0.82554018497467 0.50598680973053
0.805068373680115 0.505641222000122
0.784481346607208 0.505358517169952
0.763901770114899 0.505216479301453
0.74342954158783 0.505094468593597
0.722444295883179 0.504583954811096
0.70225065946579 0.504220247268677
0.681869566440582 0.504124462604523
0.660947024822235 0.50392609834671
0.640338718891144 0.503713488578796
0.619884371757507 0.50348287820816
0.59954571723938 0.503272294998169
0.578979730606079 0.503086447715759
0.558203935623169 0.502851665019989
0.537805438041687 0.502806961536407
0.517001569271088 0.502549469470978
0.496619164943695 0.502233624458313
0.47610729932785 0.501756846904755
0.455085933208466 0.501451313495636
0.434957236051559 0.501496255397797
0.413946837186813 0.501015603542328
0.393733203411102 0.500746160745621
0.372968375682831 0.500692874193192
0.352732181549072 0.500136256217957
0.3313989341259 0.500101238489151
0.311771184206009 0.500062257051468
0.297127068042755 0.499796718358994
0.282990902662277 0.499649345874786
0.2723228931427 0.499427884817123
0.264343798160553 0.499265164136887
0.258468121290207 0.498957335948944
0.255582928657532 0.499014973640442
0.255061417818069 0.499005228281021
0.254145354032516 0.49872088432312
0.253501743078232 0.498799830675125
0.253797769546509 0.498739212751389
0.253734081983566 0.498683422803879
0.253815054893494 0.498642385005951
0.253860384225845 0.498519659042358
0.253817111253738 0.498417258262634
0.253887534141541 0.498273372650146
0.254427641630173 0.498745948076248
0.254474103450775 0.4984130859375
0.254721522331238 0.498524367809296
0.254691570997238 0.498532831668854
0.254779309034348 0.498364120721817
0.254801750183105 0.498525857925415
0.254918903112411 0.498571991920471
0.254897952079773 0.498416423797607
0.255042523145676 0.497828900814056
0.254974275827408 0.494719952344894
0.25538244843483 0.487217783927917
0.255636870861053 0.477500796318054
0.255687385797501 0.46464216709137
0.25633442401886 0.450232177972794
0.256933122873306 0.432678908109665
0.257462024688721 0.41269388794899
0.257849365472794 0.391940861940384
0.258343428373337 0.3716821372509
0.259060502052307 0.351359724998474
0.259696334600449 0.330708861351013
0.260282516479492 0.310189694166183
0.260831981897354 0.28953954577446
0.261810630559921 0.26875302195549
0.262276649475098 0.248434782028198
0.262956380844116 0.226995319128036
0.264116108417511 0.205417037010193
0.264437466859818 0.186062380671501
0.264987736940384 0.165739580988884
0.2657171189785 0.144137114286423
0.266265034675598 0.124433308839798
0.266904026269913 0.10291863232851
0.267447352409363 0.0837689638137817
0.268109321594238 0.0622237399220467
0.268395900726318 0.0421570986509323
0.268766611814499 0.0213258489966393
0.269494026899338 0.000787951750680804
0.270337790250778 -0.0201824102550745
0.270982056856155 -0.0412418432533741
0.271383732557297 -0.0609812326729298
0.271996647119522 -0.0807274505496025
0.272738188505173 -0.101847529411316
0.273459613323212 -0.122545398771763
0.273885130882263 -0.14381255209446
0.27413871884346 -0.16313536465168
0.274959057569504 -0.184235066175461
0.275496453046799 -0.205879271030426
0.276041448116302 -0.225384667515755
0.276714712381363 -0.246156811714172
0.277274370193481 -0.267337143421173
0.277730017900467 -0.287159383296967
0.278329163789749 -0.307492047548294
0.278929471969604 -0.327891707420349
0.279387980699539 -0.348368316888809
0.279970526695251 -0.369268536567688
0.280564457178116 -0.3897924721241
0.28120955824852 -0.410248547792435
0.281796187162399 -0.430965781211853
0.282339304685593 -0.451630353927612
0.282897800207138 -0.471030205488205
0.283262491226196 -0.486683160066605
0.283549576997757 -0.50006628036499
0.283889949321747 -0.510864794254303
0.283975601196289 -0.518911361694336
0.284251093864441 -0.524605870246887
0.284453481435776 -0.527721643447876
0.284456104040146 -0.528373658657074
};
\addplot [semithick, red, dashed, forget plot]
table {%
0.25 -0.5
0.252558268640609 -0.499971215983182
0.257674800204547 -0.499913561943335
0.265349580546156 -0.499826822527541
0.275582586992214 -0.499710651911554
0.288373788282223 -0.499564571828345
0.303723144483551 -0.499387968751125
0.321630606880243 -0.499180090231924
0.342096117833915 -0.498940040397106
0.362561444756502 -0.498697137467738
0.383026586783779 -0.498451352589885
0.403491543037366 -0.498202656922469
0.423956312624558 -0.49795102163749
0.444420894638161 -0.497696417920264
0.464885288156327 -0.497438816969653
0.485349492242387 -0.497178189998295
0.505813505944689 -0.496914508232844
0.526277328296427 -0.4966477429142
0.546740958315482 -0.49637786529775
0.567204395004253 -0.496104846653605
0.587667637349495 -0.495828658266839
0.608130684322154 -0.495549271437727
0.628593534877201 -0.495266657481993
0.649056187953474 -0.494980787731044
0.669518642473505 -0.494691633532224
0.689980897343365 -0.49439916624905
0.710442951452497 -0.494103357261464
0.730904803673553 -0.493804177966079
0.751366452862231 -0.493501599776428
0.771827897857116 -0.493195594123215
0.792289137479511 -0.492886132454562
0.812750170533283 -0.492573186236266
0.833210995804693 -0.492256726952051
0.853671612062244 -0.491936726103822
0.874132018056508 -0.491613155211922
0.894592212519978 -0.491285985815386
0.915052194166895 -0.490955189472205
0.935511961693098 -0.49062073775958
0.955971513775856 -0.490282602274183
0.976430849073713 -0.489940754632423
0.996889966226327 -0.489595166470704
1.01734886385431 -0.489245809445691
1.03780754055907 -0.488892655234577
1.05826599492265 -0.488535675535346
1.07872422550757 -0.488174842067041
1.09918223085668 -0.487810126570036
1.11964000949299 -0.487441500806302
1.1400975599195 -0.48706893655968
1.16055488061908 -0.486692405636151
1.18101197005428 -0.486311879864113
1.20146882666719 -0.485927331094652
1.21936837110133 -0.485587306188302
1.23471068292331 -0.485293179242986
1.24749583154271 -0.485046149752488
1.25772387591864 -0.484847237098433
1.26539486430542 -0.484697275874471
1.27050883404033 -0.484596912039359
1.27306581137596 -0.484546599896135
1.2730658113589 -0.484546599895144
1.27305541155906 -0.484320911054072
1.2730238336127 -0.483870450269343
1.27294421772612 -0.483198810692961
1.27277401858219 -0.482315683800082
1.27245616754417 -0.481242223159833
1.27192174548361 -0.480017905211023
1.27109518948705 -0.478707989424014
1.26987791322201 -0.477382389353572
1.26861868448422 -0.476491262912181
1.26741310591075 -0.475930505649088
1.26634153930566 -0.475618318765773
1.26545420209816 -0.475475522024619
1.26477827798535 -0.475433459898834
1.26432554654506 -0.475438002031128
1.26409923112047 -0.475451102493
1.26409923107294 -0.475451102492838
1.26400510754809 -0.472893953643021
1.26381676720545 -0.467779675905825
1.26353397696839 -0.460108319156843
1.26315636493238 -0.449879963098326
1.26268342195054 -0.437094717113806
1.26211450389605 -0.421752720062896
1.26144883460045 -0.403854140014764
1.2606855094668 -0.383399173918339
1.25991927860775 -0.362944839063709
1.25915016511684 -0.342491133403313
1.25837819207707 -0.322038054872471
1.25760338256082 -0.301585601389565
1.25682575962964 -0.281133770856217
1.25604534633415 -0.260682561157473
1.25526216571386 -0.240231970161977
1.25447624079705 -0.219781995722154
1.25368759460061 -0.199332635674387
1.25289625012991 -0.178883887839198
1.25210223037863 -0.158435750021426
1.25130555832866 -0.137988220010405
1.25050625694995 -0.117541295580143
1.24970434920032 -0.0970949744895014
1.2488998580254 -0.0766492544823687
1.24809280635845 -0.0562041332878428
1.24728321712021 -0.0357596086204065
1.24647111321881 -0.0153156781801049
1.24565651754959 0.00512766034727754
1.244839452995 0.0255704092900399
1.24401994242446 0.0460125709903884
1.24319800869422 0.0664541478042603
1.24237367464724 0.0868951421011472
1.24154696311305 0.10733555626392
1.24071789690763 0.127775392688651
1.23988649883331 0.148214653784443
1.23905279167858 0.168653341973249
1.23821679821801 0.189091459689702
1.23737854121215 0.209529009380936
1.23653804340733 0.229965993506414
1.23569532753562 0.250402414537754
1.23485041631467 0.270838274958555
1.23400333244757 0.291273577264223
1.23315409862277 0.311708323961797
1.23230273751397 0.332142517569777
1.23144927177993 0.352576160617952
1.23059372406446 0.373009255647224
1.22973611699622 0.393441805209441
1.22887647318863 0.413873811867219
1.22801481523979 0.434305278193775
1.22715116573232 0.454736206772755
1.22628554723329 0.475166600198059
1.22552642790666 0.493042728464227
1.22487449160254 0.508364776283352
1.22433031870982 0.52113290203806
1.22389439056138 0.531347238274115
1.22356709317291 0.539007892120091
1.22334872031756 0.544114945632642
1.22323947593845 0.546668456066198
1.22323947590048 0.546668456066214
1.2230123362303 0.54664910964162
1.22255893634353 0.546599663012994
1.22188286957468 0.546493423413052
1.22099418533407 0.546288449972454
1.21991546682544 0.545929219980312
1.21868995460871 0.545349990456186
1.21739061938076 0.544480715471633
1.21610030698399 0.543230273381219
1.21526412455335 0.541961848316668
1.21476024252823 0.540762508740427
1.21448767411105 0.539705040879324
1.21435620536168 0.538833702471361
1.21429925392167 0.538171836563276
1.21427642250785 0.537729171531928
1.21429899313131 0.537497396084245
1.21429899310163 0.53749739608416
1.21173554316399 0.537351398434274
1.2066086675668 0.537059357198718
1.19891842689941 0.536621157703719
1.18866491799536 0.536036617118684
1.17584827376684 0.535305485487585
1.16046866297006 0.534427447211652
1.1425262899005 0.533402122984755
1.12202139401749 0.532229072183377
1.10151726555175 0.531054604210161
1.08101390213675 0.529878734481104
1.06051130139776 0.528701478429746
1.04000946095186 0.5275228515073
1.01950837840803 0.526342869182777
0.999008051367172 0.525161546943109
0.978508477422155 0.523978900293278
0.958009654157879 0.522794944756435
0.937511579151311 0.521609695874025
0.917014249971538 0.520423169205916
0.896517664179812 0.519235380330514
0.876021819329601 0.518046344844887
0.855526712966634 0.516856078364893
0.835032342628955 0.515664596525293
0.814538705846972 0.514471914979876
0.794045800143499 0.513278049401579
0.773553623033816 0.512083015482608
0.753062172025712 0.510886828934553
0.732571444619537 0.509689505488512
0.712081438308253 0.508491060895207
0.691592150577485 0.507291510925102
0.671103578905572 0.506090871368521
0.650615720763616 0.504889158035764
0.630128573615536 0.503686386757223
0.609642134918118 0.502482573383502
0.589156402121068 0.501277733785526
0.568671372667066 0.500071883854662
0.548187043991813 0.49886503950283
0.527703413524089 0.497657216662618
0.507220478685805 0.496448431287396
0.486738236892053 0.495238699351429
0.466256685551165 0.494028036849988
0.445775822064762 0.492816459799465
0.425295643827809 0.491603984237482
0.404816148228671 0.490390626223003
0.384337332649168 0.489176401836445
0.363859194464625 0.487961327179788
0.343381731043935 0.486745418376685
0.322904939749608 0.48552869157257
0.302428817937827 0.484311162934768
0.281953362958509 0.483092848652601
0.261478572155355 0.4818737649375
0.243563709029771 0.480806407637357
0.228208544470985 0.479891046434884
0.215412881797103 0.479127906813159
0.20517655624396 0.4785171732374
0.197499434529847 0.478058991869279
0.192381414497009 0.477753472810477
0.189822424830656 0.477600691872751
0.189822424856044 0.477600691872319
0.18984329041129 0.477361217804826
0.189891281488213 0.476888179768742
0.18998422190601 0.476195304214649
0.190156086957441 0.475302243681175
0.19046107666499 0.47423395694596
0.190973511005364 0.47302370084793
0.1917804370421 0.471720939032194
0.192990859960637 0.470375213510415
0.194255394688102 0.469435488118
0.195465103825654 0.468815206001496
0.196534139830494 0.468445902524195
0.19741290176854 0.468255645457359
0.19807765083531 0.468178447005524
0.198520413044486 0.468160326901788
0.198740901208049 0.468162462817004
0.198740901269321 0.46816246281638
0.198945440782965 0.465612270025241
0.199354508471781 0.460511864695189
0.199968076306772 0.452861197525416
0.200786100039452 0.442660189857331
0.201808520127061 0.429908733995519
0.203035263052485 0.414606693655117
0.20446624304152 0.396753904535665
0.206101364180888 0.376350175021588
0.207736171974856 0.355945830121864
0.209370679915717 0.335540874152152
0.211004901527408 0.315135311429408
0.212638850365535 0.294729146271967
0.214272540017396 0.27432238299961
0.215905984101993 0.253915025933645
0.217539196270065 0.233507079396979
0.219172190204101 0.2130985477142
0.220804979618368 0.192689435211644
0.222437578258934 0.172279746217481
0.224069999903693 0.15186948506179
0.225702258362386 0.131458656076634
0.227334367476629 0.111047263596139
0.228966341119943 0.0906353119565765
0.230598193197773 0.0702228054964383
0.232229937647521 0.0498097485565179
0.233861588438572 0.0293961454799898
0.235493159572323 0.00898200061249
0.237124665082211 -0.0114326816978031
0.238756119033746 -0.0318478971000879
0.240387535524538 -0.0522636412408578
0.24201892868433 -0.0726799097638194
0.243650312675029 -0.0930966983098097
0.245281701690739 -0.113514002516714
0.246913109957795 -0.133931818019382
0.248544551734793 -0.154350140449545
0.250176041312631 -0.174768965435729
0.251807593014535 -0.195188288603176
0.253439221196103 -0.215608105573753
0.255070940245335 -0.236028411965869
0.256702764582677 -0.256449203394391
0.258334708661049 -0.276870475470555
0.259966786965892 -0.29729222380188
0.261599014015202 -0.317714443992083
0.263231404359571 -0.338137131640988
0.264863972582227 -0.358560282344441
0.266496733299076 -0.378983891694222
0.26812970115874 -0.399407955277952
0.269762890842605 -0.419832468679009
0.271396317064857 -0.440257427476434
0.27302999457253 -0.460682827244845
0.27466393814555 -0.481108663554342
0.276093884549649 -0.498981648418472
0.277319746923832 -0.514301630525306
0.278341443563315 -0.527068481124652
0.279158900790641 -0.537282093189924
0.279772055404735 -0.544942380708266
0.280180856703051 -0.550049278099581
0.280385268072754 -0.552602739765039
};
\addplot [semithick, green, dash pattern=on 1pt off 3pt on 3pt off 3pt, forget plot]
table {%
0.25 -0.5
0.2525 -0.5
0.2575 -0.5
0.265 -0.5
0.275 -0.5
0.2875 -0.5
0.3025 -0.5
0.32 -0.5
0.34 -0.5
0.36 -0.5
0.38 -0.5
0.4 -0.5
0.42 -0.5
0.44 -0.5
0.46 -0.5
0.48 -0.5
0.5 -0.5
0.52 -0.5
0.54 -0.5
0.56 -0.5
0.58 -0.5
0.6 -0.5
0.62 -0.5
0.64 -0.5
0.66 -0.5
0.68 -0.5
0.7 -0.5
0.72 -0.5
0.74 -0.5
0.76 -0.5
0.78 -0.5
0.8 -0.5
0.82 -0.5
0.84 -0.5
0.86 -0.5
0.88 -0.5
0.9 -0.5
0.92 -0.5
0.94 -0.5
0.96 -0.5
0.98 -0.5
1 -0.5
1.02 -0.5
1.04 -0.5
1.06 -0.5
1.08 -0.5
1.1 -0.5
1.12 -0.5
1.14 -0.5
1.16 -0.5
1.18 -0.5
1.1975 -0.5
1.2125 -0.5
1.225 -0.5
1.235 -0.5
1.2425 -0.5
1.2475 -0.5
1.25 -0.5
1.25 -0.5
1.25 -0.5
1.25 -0.5
1.25 -0.5
1.25 -0.5
1.25 -0.5
1.25 -0.5
1.25 -0.5
1.25 -0.5
1.25 -0.5
1.25 -0.5
1.25 -0.5
1.25 -0.5
1.25 -0.5
1.25 -0.5
1.25 -0.5
1.25 -0.5
1.25 -0.4975
1.25 -0.4925
1.25 -0.485
1.25 -0.475
1.25 -0.4625
1.25 -0.4475
1.25 -0.43
1.25 -0.41
1.25 -0.39
1.25 -0.37
1.25 -0.35
1.25 -0.33
1.25 -0.31
1.25 -0.29
1.25 -0.27
1.25 -0.25
1.25 -0.23
1.25 -0.21
1.25 -0.19
1.25 -0.17
1.25 -0.15
1.25 -0.13
1.25 -0.11
1.25 -0.09
1.25 -0.07
1.25 -0.05
1.25 -0.03
1.25 -0.01
1.25 0.00999999999999998
1.25 0.03
1.25 0.05
1.25 0.07
1.25 0.09
1.25 0.11
1.25 0.13
1.25 0.15
1.25 0.17
1.25 0.19
1.25 0.21
1.25 0.23
1.25 0.25
1.25 0.27
1.25 0.29
1.25 0.31
1.25 0.33
1.25 0.35
1.25 0.37
1.25 0.39
1.25 0.41
1.25 0.43
1.25 0.4475
1.25 0.4625
1.25 0.475
1.25 0.485
1.25 0.4925
1.25 0.4975
1.25 0.5
1.25 0.5
1.25 0.5
1.25 0.5
1.25 0.5
1.25 0.5
1.25 0.5
1.25 0.5
1.25 0.5
1.25 0.5
1.25 0.5
1.25 0.5
1.25 0.5
1.25 0.5
1.25 0.5
1.25 0.5
1.25 0.5
1.25 0.5
1.2475 0.5
1.2425 0.5
1.235 0.5
1.225 0.5
1.2125 0.5
1.1975 0.5
1.18 0.5
1.16 0.5
1.14 0.5
1.12 0.5
1.1 0.5
1.08 0.5
1.06 0.5
1.04 0.5
1.02 0.5
1 0.5
0.98 0.5
0.96 0.5
0.94 0.5
0.92 0.5
0.9 0.5
0.88 0.5
0.86 0.5
0.84 0.5
0.82 0.5
0.8 0.5
0.78 0.5
0.76 0.5
0.74 0.5
0.72 0.5
0.7 0.5
0.68 0.5
0.66 0.5
0.64 0.5
0.62 0.5
0.6 0.5
0.58 0.5
0.56 0.5
0.54 0.5
0.52 0.5
0.5 0.5
0.48 0.5
0.46 0.5
0.44 0.5
0.42 0.5
0.4 0.5
0.38 0.5
0.36 0.5
0.34 0.5
0.32 0.5
0.3025 0.5
0.2875 0.5
0.275 0.5
0.265 0.5
0.2575 0.5
0.2525 0.5
0.25 0.5
0.25 0.5
0.25 0.5
0.25 0.5
0.25 0.5
0.25 0.5
0.25 0.5
0.25 0.5
0.25 0.5
0.25 0.5
0.25 0.5
0.25 0.5
0.25 0.5
0.25 0.5
0.25 0.5
0.25 0.5
0.25 0.5
0.25 0.5
0.25 0.4975
0.25 0.4925
0.25 0.485
0.25 0.475
0.25 0.4625
0.25 0.4475
0.25 0.43
0.25 0.41
0.25 0.39
0.25 0.37
0.25 0.35
0.25 0.33
0.25 0.31
0.25 0.29
0.25 0.27
0.25 0.25
0.25 0.23
0.25 0.21
0.25 0.19
0.25 0.17
0.25 0.15
0.249999999999999 0.13
0.249999999999999 0.11
0.249999999999999 0.09
0.249999999999999 0.07
0.249999999999999 0.05
0.249999999999999 0.03
0.249999999999999 0.01
0.249999999999999 -0.00999999999999998
0.249999999999999 -0.03
0.249999999999999 -0.05
0.249999999999999 -0.07
0.249999999999999 -0.09
0.249999999999999 -0.11
0.249999999999999 -0.13
0.249999999999999 -0.15
0.249999999999999 -0.17
0.249999999999999 -0.19
0.249999999999999 -0.21
0.249999999999999 -0.23
0.249999999999999 -0.25
0.249999999999999 -0.27
0.249999999999999 -0.29
0.249999999999999 -0.31
0.249999999999999 -0.33
0.249999999999999 -0.35
0.249999999999999 -0.37
0.249999999999999 -0.39
0.249999999999999 -0.41
0.249999999999999 -0.43
0.249999999999999 -0.4475
0.249999999999999 -0.4625
0.249999999999999 -0.475
0.249999999999999 -0.485
0.249999999999999 -0.4925
0.249999999999999 -0.4975
0.249999999999999 -0.5
};
\addplot [semithick, blue, opacity=\opacityRef, forget plot]
table {%
0.25 -0.5
0.250134587287903 -0.499975740909576
0.254364520311356 -0.500102579593658
0.261599600315094 -0.500107169151306
0.271652609109879 -0.500190675258636
0.283783078193665 -0.500302195549011
0.298230290412903 -0.500412464141846
0.315541923046112 -0.500544548034668
0.335578590631485 -0.500884354114532
0.35632985830307 -0.500516414642334
0.376694440841675 -0.500788450241089
0.397453784942627 -0.501067519187927
0.41796013712883 -0.501211941242218
0.43890044093132 -0.501327395439148
0.459446012973785 -0.501241624355316
0.480003148317337 -0.501211404800415
0.500290781259537 -0.501124918460846
0.521199077367783 -0.501396894454956
0.54151514172554 -0.501492321491241
0.562147051095963 -0.501669049263
0.582711189985275 -0.501598060131073
0.602959364652634 -0.501742660999298
0.623842149972916 -0.502025425434113
0.64429435133934 -0.502254605293274
0.665196448564529 -0.502356588840485
0.685860902070999 -0.502340018749237
0.706250935792923 -0.502545237541199
0.727059751749039 -0.502572536468506
0.747389823198318 -0.502674221992493
0.767833977937698 -0.502841174602509
0.788216441869736 -0.502996802330017
0.808488994836807 -0.502866744995117
0.829352170228958 -0.503052413463593
0.849757641553879 -0.503376483917236
0.870628744363785 -0.503478944301605
0.891002625226974 -0.503502666950226
0.911931782960892 -0.50354415178299
0.932406336069107 -0.503706932067871
0.952825576066971 -0.503655016422272
0.973445922136307 -0.503691613674164
0.993304759263992 -0.503610134124756
1.01411005854607 -0.503715991973877
1.03449186682701 -0.50353741645813
1.05497059226036 -0.503421485424042
1.07584854960442 -0.503655433654785
1.09624227881432 -0.504056572914124
1.11711248755455 -0.504051089286804
1.13772854208946 -0.504298806190491
1.15827652812004 -0.504721283912659
1.17907127737999 -0.504731237888336
1.1993645131588 -0.504829943180084
1.21874222159386 -0.505171775817871
1.23402556777 -0.505479037761688
1.24745413661003 -0.505399823188782
1.2582103908062 -0.505542993545532
1.26648482680321 -0.505407273769379
1.27185532450676 -0.505461037158966
1.27532586455345 -0.505309998989105
1.27588793635368 -0.505458474159241
1.27640268206596 -0.505327165126801
1.27718517184258 -0.505186021327972
1.2772658765316 -0.505307257175446
1.27748367190361 -0.505163729190826
1.27741822600365 -0.505099058151245
1.27749285101891 -0.504942178726196
1.27737507224083 -0.504625797271729
1.27705261111259 -0.504451990127563
1.27666601538658 -0.504635393619537
1.27658578753471 -0.504563689231873
1.27657422423363 -0.504620671272278
1.27650579810143 -0.504778206348419
1.27645847201347 -0.504693329334259
1.27635416388512 -0.504892468452454
1.27630552649498 -0.504746437072754
1.27628681063652 -0.504792213439941
1.27626404166222 -0.504226684570312
1.27623602747917 -0.50121545791626
1.27601191401482 -0.493828892707825
1.27580425143242 -0.484300822019577
1.27561458945274 -0.471954256296158
1.27552017569542 -0.45722571015358
1.27544388175011 -0.439793199300766
1.27512690424919 -0.419501841068268
1.27487060427666 -0.398504972457886
1.27469310164452 -0.378581404685974
1.27448257803917 -0.357628524303436
1.2741627395153 -0.337194889783859
1.27392038702965 -0.316699028015137
1.27394101023674 -0.296485722064972
1.27376601099968 -0.275597095489502
1.27344581484795 -0.25583016872406
1.27345916628838 -0.234443128108978
1.2731781899929 -0.214407309889793
1.27320298552513 -0.19313357770443
1.27298173308372 -0.172748327255249
1.27284786105156 -0.151974260807037
1.27270922064781 -0.131899058818817
1.27246496081352 -0.111078023910522
1.27231845259666 -0.0912491232156754
1.27233693003654 -0.0701435729861259
1.27192088961601 -0.0503905788064003
1.27183720469475 -0.0294062290340662
1.27130696177483 -0.00874903611838818
1.27148720622063 0.0121132787317038
1.27118381857872 0.0327140614390373
1.27094468474388 0.0531526058912277
1.27071532607079 0.0734186992049217
1.2702222764492 0.0938008576631546
1.26983973383904 0.11490710824728
1.26966544985771 0.135428339242935
1.26924416422844 0.15543232858181
1.26896521449089 0.176327347755432
1.26847633719444 0.196783065795898
1.26839908957481 0.217912599444389
1.26833578944206 0.238198205828667
1.26780614256859 0.25879368185997
1.26758000254631 0.279611498117447
1.26704582571983 0.300206780433655
1.26669737696648 0.320525676012039
1.26629647612572 0.341260880231857
1.2660296857357 0.361968815326691
1.26588103175163 0.382240295410156
1.26584550738335 0.402667045593262
1.26530811190605 0.423333644866943
1.26515385508537 0.444013863801956
1.26469644904137 0.464210540056229
1.26443693041801 0.483510255813599
1.26353988051414 0.495914965867996
1.26403376460075 0.506396353244781
1.26389583945274 0.520410180091858
1.26334890723228 0.530100584030151
1.26385352015495 0.530811429023743
1.2638575732708 0.531947493553162
1.26306268572807 0.534497082233429
1.26359960436821 0.534720957279205
1.26321038603783 0.534544765949249
1.26331183314323 0.534810423851013
1.26306459307671 0.534356772899628
1.26290974020958 0.534022927284241
1.26296409964561 0.534163236618042
1.26331028342247 0.534097671508789
1.26343390345573 0.533294975757599
1.26306411623955 0.533267796039581
1.262665361166 0.533523797988892
1.26344558596611 0.533410251140594
1.2626186311245 0.53341418504715
1.26242634654045 0.533198654651642
1.26311632990837 0.533085584640503
1.26311442255974 0.533047199249268
1.26314243674278 0.533043622970581
1.26198348402977 0.533029973506927
1.25388512015343 0.532839059829712
1.24639388918877 0.532619178295135
1.23839196562767 0.532619059085846
1.22674974799156 0.532370269298553
1.2081535756588 0.53195333480835
1.18991681933403 0.531617641448975
1.17190167307854 0.531213998794556
1.1527781188488 0.530911505222321
1.13161870837212 0.530400097370148
1.11186155676842 0.530045509338379
1.09046706557274 0.529693901538849
1.07051667571068 0.529365003108978
1.04973062872887 0.528951406478882
1.02942606806755 0.52834689617157
1.0087579190731 0.527958273887634
0.987912625074387 0.527447581291199
0.967967301607132 0.527189910411835
0.947023123502731 0.52673351764679
0.927059262990952 0.526389718055725
0.90680393576622 0.525787532329559
0.886254876852036 0.525317788124084
0.865073472261429 0.524889230728149
0.84497657418251 0.52461850643158
0.82443842291832 0.524082958698273
0.804586738348007 0.523464202880859
0.783658057451248 0.523112595081329
0.763218134641647 0.522701859474182
0.742673009634018 0.522395730018616
0.722050577402115 0.521953284740448
0.701502829790115 0.521670341491699
0.681198567152023 0.521314382553101
0.660657316446304 0.520998597145081
0.639705568552017 0.520614504814148
0.619709521532059 0.520388841629028
0.598721414804459 0.520051300525665
0.578375667333603 0.519471228122711
0.557724803686142 0.519199848175049
0.536719530820847 0.518800675868988
0.51650008559227 0.518453299999237
0.495961219072342 0.518078088760376
0.475473761558533 0.517839312553406
0.454958707094193 0.517263174057007
0.434264451265335 0.516940534114838
0.41351592540741 0.516423344612122
0.393231809139252 0.516191482543945
0.372134447097778 0.51578426361084
0.351651608943939 0.515261113643646
0.330936372280121 0.51482355594635
0.310291141271591 0.514433324337006
0.289405554533005 0.513801872730255
0.272056758403778 0.513529002666473
0.264025807380676 0.513509809970856
0.253979444503784 0.512894809246063
0.244669646024704 0.512766003608704
0.243256464600563 0.512556135654449
0.242041885852814 0.512913286685944
0.239295020699501 0.512692451477051
0.23687469959259 0.512611925601959
0.236648306250572 0.512437403202057
0.236951068043709 0.512229442596436
0.236799091100693 0.512362360954285
0.237096145749092 0.511899590492249
0.236902326345444 0.511906802654266
0.237000212073326 0.51183021068573
0.237344428896904 0.511774897575378
0.237611725926399 0.511761069297791
0.237875759601593 0.511535167694092
0.237898394465446 0.511673033237457
0.237900257110596 0.51174408197403
0.23793038725853 0.511673152446747
0.237971380352974 0.511723935604095
0.238060623407364 0.512074708938599
0.238067924976349 0.511841177940369
0.238114565610886 0.509295463562012
0.238030806183815 0.506714284420013
0.238113358616829 0.500233948230743
0.238138124346733 0.486373782157898
0.238222241401672 0.474018812179565
0.23837411403656 0.458301067352295
0.238655641674995 0.437078207731247
0.238592326641083 0.416740953922272
0.238361969590187 0.396077513694763
0.238630995154381 0.37514454126358
0.238962441682816 0.355087280273438
0.239040166139603 0.334346622228622
0.239332675933838 0.313692003488541
0.239720538258553 0.293406546115875
0.239759877324104 0.272992610931396
0.239960193634033 0.252501457929611
0.240271389484406 0.231834530830383
0.240479186177254 0.211377695202827
0.240850239992142 0.190396964550018
0.24080154299736 0.170131206512451
0.240954726934433 0.148987829685211
0.241066455841064 0.128528565168381
0.241374686360359 0.107518896460533
0.241708755493164 0.0867738574743271
0.241640582680702 0.0666625052690506
0.241736829280853 0.0469138883054256
0.241781055927277 0.0260544754564762
0.241966262459755 0.0053991787135601
0.242395505309105 -0.0146730504930019
0.242622315883636 -0.0361780561506748
0.242674380540848 -0.0563565343618393
0.242975443601608 -0.0779163688421249
0.242989599704742 -0.0979859456419945
0.243122920393944 -0.118330843746662
0.243540838360786 -0.139212548732758
0.24359904229641 -0.159436985850334
0.243718534708023 -0.17982605099678
0.243897393345833 -0.200731471180916
0.244203001260757 -0.221120223402977
0.244454383850098 -0.24180743098259
0.244598463177681 -0.26235756278038
0.244719713926315 -0.282900124788284
0.245138958096504 -0.303754895925522
0.245357647538185 -0.324305057525635
0.245519071817398 -0.344911754131317
0.245634406805038 -0.365250885486603
0.245750144124031 -0.386120915412903
0.246036008000374 -0.406508177518845
0.246105507016182 -0.427008420228958
0.246242761611938 -0.44646018743515
0.246336609125137 -0.463376462459564
0.246450200676918 -0.481611043214798
0.246598243713379 -0.497351050376892
0.246637120842934 -0.503044366836548
0.246688827872276 -0.508935868740082
0.246709808707237 -0.516293883323669
0.246621876955032 -0.516304850578308
};
\addplot [semithick, red, dashed, forget plot]
table {%
0.25 -0.5
0.252558268640609 -0.499971215983182
0.257674800204547 -0.499913561943335
0.265349580546156 -0.499826822527541
0.275582586992214 -0.499710651911554
0.288373788282223 -0.499564571828345
0.303723144483551 -0.499387968751125
0.321630606880243 -0.499180090231924
0.342096117833915 -0.498940040397106
0.362561444756502 -0.498697137467738
0.383026586783779 -0.498451352589885
0.403491543037366 -0.498202656922469
0.423956312624558 -0.49795102163749
0.444420894638161 -0.497696417920264
0.464885288156327 -0.497438816969653
0.485349492242387 -0.497178189998295
0.505813505944689 -0.496914508232844
0.526277328296427 -0.4966477429142
0.546740958315482 -0.49637786529775
0.567204395004253 -0.496104846653605
0.587667637349495 -0.495828658266839
0.608130684322154 -0.495549271437727
0.628593534877201 -0.495266657481993
0.649056187953474 -0.494980787731044
0.669518642473505 -0.494691633532224
0.689980897343365 -0.49439916624905
0.710442951452497 -0.494103357261464
0.730904803673553 -0.493804177966079
0.751366452862231 -0.493501599776428
0.771827897857116 -0.493195594123215
0.792289137479511 -0.492886132454562
0.812750170533283 -0.492573186236266
0.833210995804693 -0.492256726952051
0.853671612062244 -0.491936726103822
0.874132018056508 -0.491613155211922
0.894592212519978 -0.491285985815386
0.915052194166895 -0.490955189472205
0.935511961693098 -0.49062073775958
0.955971513775856 -0.490282602274183
0.976430849073713 -0.489940754632423
0.996889966226327 -0.489595166470704
1.01734886385431 -0.489245809445691
1.03780754055907 -0.488892655234577
1.05826599492265 -0.488535675535346
1.07872422550757 -0.488174842067041
1.09918223085668 -0.487810126570036
1.11964000949299 -0.487441500806302
1.1400975599195 -0.48706893655968
1.16055488061908 -0.486692405636151
1.18101197005428 -0.486311879864113
1.20146882666719 -0.485927331094652
1.21936837110133 -0.485587306188302
1.23471068292331 -0.485293179242986
1.24749583154271 -0.485046149752488
1.25772387591864 -0.484847237098433
1.26539486430542 -0.484697275874471
1.27050883404033 -0.484596912039359
1.27306581137596 -0.484546599896135
1.2730658113589 -0.484546599895144
1.27305541155906 -0.484320911054072
1.2730238336127 -0.483870450269343
1.27294421772612 -0.483198810692961
1.27277401858219 -0.482315683800082
1.27245616754417 -0.481242223159833
1.27192174548361 -0.480017905211023
1.27109518948705 -0.478707989424014
1.26987791322201 -0.477382389353572
1.26861868448422 -0.476491262912181
1.26741310591075 -0.475930505649088
1.26634153930566 -0.475618318765773
1.26545420209816 -0.475475522024619
1.26477827798535 -0.475433459898834
1.26432554654506 -0.475438002031128
1.26409923112047 -0.475451102493
1.26409923107294 -0.475451102492838
1.26400510754809 -0.472893953643021
1.26381676720545 -0.467779675905825
1.26353397696839 -0.460108319156843
1.26315636493238 -0.449879963098326
1.26268342195054 -0.437094717113806
1.26211450389605 -0.421752720062896
1.26144883460045 -0.403854140014764
1.2606855094668 -0.383399173918339
1.25991927860775 -0.362944839063709
1.25915016511684 -0.342491133403313
1.25837819207707 -0.322038054872471
1.25760338256082 -0.301585601389565
1.25682575962964 -0.281133770856217
1.25604534633415 -0.260682561157473
1.25526216571386 -0.240231970161977
1.25447624079705 -0.219781995722154
1.25368759460061 -0.199332635674387
1.25289625012991 -0.178883887839198
1.25210223037863 -0.158435750021426
1.25130555832866 -0.137988220010405
1.25050625694995 -0.117541295580143
1.24970434920032 -0.0970949744895014
1.2488998580254 -0.0766492544823687
1.24809280635845 -0.0562041332878428
1.24728321712021 -0.0357596086204065
1.24647111321881 -0.0153156781801049
1.24565651754959 0.00512766034727754
1.244839452995 0.0255704092900399
1.24401994242446 0.0460125709903884
1.24319800869422 0.0664541478042603
1.24237367464724 0.0868951421011472
1.24154696311305 0.10733555626392
1.24071789690763 0.127775392688651
1.23988649883331 0.148214653784443
1.23905279167858 0.168653341973249
1.23821679821801 0.189091459689702
1.23737854121215 0.209529009380936
1.23653804340733 0.229965993506414
1.23569532753562 0.250402414537754
1.23485041631467 0.270838274958555
1.23400333244757 0.291273577264223
1.23315409862277 0.311708323961797
1.23230273751397 0.332142517569777
1.23144927177993 0.352576160617952
1.23059372406446 0.373009255647224
1.22973611699622 0.393441805209441
1.22887647318863 0.413873811867219
1.22801481523979 0.434305278193775
1.22715116573232 0.454736206772755
1.22628554723329 0.475166600198059
1.22552642790666 0.493042728464227
1.22487449160254 0.508364776283352
1.22433031870982 0.52113290203806
1.22389439056138 0.531347238274115
1.22356709317291 0.539007892120091
1.22334872031756 0.544114945632642
1.22323947593845 0.546668456066198
1.22323947590048 0.546668456066214
1.2230123362303 0.54664910964162
1.22255893634353 0.546599663012994
1.22188286957468 0.546493423413052
1.22099418533407 0.546288449972454
1.21991546682544 0.545929219980312
1.21868995460871 0.545349990456186
1.21739061938076 0.544480715471633
1.21610030698399 0.543230273381219
1.21526412455335 0.541961848316668
1.21476024252823 0.540762508740427
1.21448767411105 0.539705040879324
1.21435620536168 0.538833702471361
1.21429925392167 0.538171836563276
1.21427642250785 0.537729171531928
1.21429899313131 0.537497396084245
1.21429899310163 0.53749739608416
1.21173554316399 0.537351398434274
1.2066086675668 0.537059357198718
1.19891842689941 0.536621157703719
1.18866491799536 0.536036617118684
1.17584827376684 0.535305485487585
1.16046866297006 0.534427447211652
1.1425262899005 0.533402122984755
1.12202139401749 0.532229072183377
1.10151726555175 0.531054604210161
1.08101390213675 0.529878734481104
1.06051130139776 0.528701478429746
1.04000946095186 0.5275228515073
1.01950837840803 0.526342869182777
0.999008051367172 0.525161546943109
0.978508477422155 0.523978900293278
0.958009654157879 0.522794944756435
0.937511579151311 0.521609695874025
0.917014249971538 0.520423169205916
0.896517664179812 0.519235380330514
0.876021819329601 0.518046344844887
0.855526712966634 0.516856078364893
0.835032342628955 0.515664596525293
0.814538705846972 0.514471914979876
0.794045800143499 0.513278049401579
0.773553623033816 0.512083015482608
0.753062172025712 0.510886828934553
0.732571444619537 0.509689505488512
0.712081438308253 0.508491060895207
0.691592150577485 0.507291510925102
0.671103578905572 0.506090871368521
0.650615720763616 0.504889158035764
0.630128573615536 0.503686386757223
0.609642134918118 0.502482573383502
0.589156402121068 0.501277733785526
0.568671372667066 0.500071883854662
0.548187043991813 0.49886503950283
0.527703413524089 0.497657216662618
0.507220478685805 0.496448431287396
0.486738236892053 0.495238699351429
0.466256685551165 0.494028036849988
0.445775822064762 0.492816459799465
0.425295643827809 0.491603984237482
0.404816148228671 0.490390626223003
0.384337332649168 0.489176401836445
0.363859194464625 0.487961327179788
0.343381731043935 0.486745418376685
0.322904939749608 0.48552869157257
0.302428817937827 0.484311162934768
0.281953362958509 0.483092848652601
0.261478572155355 0.4818737649375
0.243563709029771 0.480806407637357
0.228208544470985 0.479891046434884
0.215412881797103 0.479127906813159
0.20517655624396 0.4785171732374
0.197499434529847 0.478058991869279
0.192381414497009 0.477753472810477
0.189822424830656 0.477600691872751
0.189822424856044 0.477600691872319
0.18984329041129 0.477361217804826
0.189891281488213 0.476888179768742
0.18998422190601 0.476195304214649
0.190156086957441 0.475302243681175
0.19046107666499 0.47423395694596
0.190973511005364 0.47302370084793
0.1917804370421 0.471720939032194
0.192990859960637 0.470375213510415
0.194255394688102 0.469435488118
0.195465103825654 0.468815206001496
0.196534139830494 0.468445902524195
0.19741290176854 0.468255645457359
0.19807765083531 0.468178447005524
0.198520413044486 0.468160326901788
0.198740901208049 0.468162462817004
0.198740901269321 0.46816246281638
0.198945440782965 0.465612270025241
0.199354508471781 0.460511864695189
0.199968076306772 0.452861197525416
0.200786100039452 0.442660189857331
0.201808520127061 0.429908733995519
0.203035263052485 0.414606693655117
0.20446624304152 0.396753904535665
0.206101364180888 0.376350175021588
0.207736171974856 0.355945830121864
0.209370679915717 0.335540874152152
0.211004901527408 0.315135311429408
0.212638850365535 0.294729146271967
0.214272540017396 0.27432238299961
0.215905984101993 0.253915025933645
0.217539196270065 0.233507079396979
0.219172190204101 0.2130985477142
0.220804979618368 0.192689435211644
0.222437578258934 0.172279746217481
0.224069999903693 0.15186948506179
0.225702258362386 0.131458656076634
0.227334367476629 0.111047263596139
0.228966341119943 0.0906353119565765
0.230598193197773 0.0702228054964383
0.232229937647521 0.0498097485565179
0.233861588438572 0.0293961454799898
0.235493159572323 0.00898200061249
0.237124665082211 -0.0114326816978031
0.238756119033746 -0.0318478971000879
0.240387535524538 -0.0522636412408578
0.24201892868433 -0.0726799097638194
0.243650312675029 -0.0930966983098097
0.245281701690739 -0.113514002516714
0.246913109957795 -0.133931818019382
0.248544551734793 -0.154350140449545
0.250176041312631 -0.174768965435729
0.251807593014535 -0.195188288603176
0.253439221196103 -0.215608105573753
0.255070940245335 -0.236028411965869
0.256702764582677 -0.256449203394391
0.258334708661049 -0.276870475470555
0.259966786965892 -0.29729222380188
0.261599014015202 -0.317714443992083
0.263231404359571 -0.338137131640988
0.264863972582227 -0.358560282344441
0.266496733299076 -0.378983891694222
0.26812970115874 -0.399407955277952
0.269762890842605 -0.419832468679009
0.271396317064857 -0.440257427476434
0.27302999457253 -0.460682827244845
0.27466393814555 -0.481108663554342
0.276093884549649 -0.498981648418472
0.277319746923832 -0.514301630525306
0.278341443563315 -0.527068481124652
0.279158900790641 -0.537282093189924
0.279772055404735 -0.544942380708266
0.280180856703051 -0.550049278099581
0.280385268072754 -0.552602739765039
};
\addplot [semithick, green, dash pattern=on 1pt off 3pt on 3pt off 3pt, forget plot]
table {%
0.25 -0.5
0.2525 -0.5
0.2575 -0.5
0.265 -0.5
0.275 -0.5
0.2875 -0.5
0.3025 -0.5
0.32 -0.5
0.34 -0.5
0.36 -0.5
0.38 -0.5
0.4 -0.5
0.42 -0.5
0.44 -0.5
0.46 -0.5
0.48 -0.5
0.5 -0.5
0.52 -0.5
0.54 -0.5
0.56 -0.5
0.58 -0.5
0.6 -0.5
0.62 -0.5
0.64 -0.5
0.66 -0.5
0.68 -0.5
0.7 -0.5
0.72 -0.5
0.74 -0.5
0.76 -0.5
0.78 -0.5
0.8 -0.5
0.82 -0.5
0.84 -0.5
0.86 -0.5
0.88 -0.5
0.9 -0.5
0.92 -0.5
0.94 -0.5
0.96 -0.5
0.98 -0.5
1 -0.5
1.02 -0.5
1.04 -0.5
1.06 -0.5
1.08 -0.5
1.1 -0.5
1.12 -0.5
1.14 -0.5
1.16 -0.5
1.18 -0.5
1.1975 -0.5
1.2125 -0.5
1.225 -0.5
1.235 -0.5
1.2425 -0.5
1.2475 -0.5
1.25 -0.5
1.25 -0.5
1.25 -0.5
1.25 -0.5
1.25 -0.5
1.25 -0.5
1.25 -0.5
1.25 -0.5
1.25 -0.5
1.25 -0.5
1.25 -0.5
1.25 -0.5
1.25 -0.5
1.25 -0.5
1.25 -0.5
1.25 -0.5
1.25 -0.5
1.25 -0.5
1.25 -0.4975
1.25 -0.4925
1.25 -0.485
1.25 -0.475
1.25 -0.4625
1.25 -0.4475
1.25 -0.43
1.25 -0.41
1.25 -0.39
1.25 -0.37
1.25 -0.35
1.25 -0.33
1.25 -0.31
1.25 -0.29
1.25 -0.27
1.25 -0.25
1.25 -0.23
1.25 -0.21
1.25 -0.19
1.25 -0.17
1.25 -0.15
1.25 -0.13
1.25 -0.11
1.25 -0.09
1.25 -0.07
1.25 -0.05
1.25 -0.03
1.25 -0.01
1.25 0.00999999999999998
1.25 0.03
1.25 0.05
1.25 0.07
1.25 0.09
1.25 0.11
1.25 0.13
1.25 0.15
1.25 0.17
1.25 0.19
1.25 0.21
1.25 0.23
1.25 0.25
1.25 0.27
1.25 0.29
1.25 0.31
1.25 0.33
1.25 0.35
1.25 0.37
1.25 0.39
1.25 0.41
1.25 0.43
1.25 0.4475
1.25 0.4625
1.25 0.475
1.25 0.485
1.25 0.4925
1.25 0.4975
1.25 0.5
1.25 0.5
1.25 0.5
1.25 0.5
1.25 0.5
1.25 0.5
1.25 0.5
1.25 0.5
1.25 0.5
1.25 0.5
1.25 0.5
1.25 0.5
1.25 0.5
1.25 0.5
1.25 0.5
1.25 0.5
1.25 0.5
1.25 0.5
1.2475 0.5
1.2425 0.5
1.235 0.5
1.225 0.5
1.2125 0.5
1.1975 0.5
1.18 0.5
1.16 0.5
1.14 0.5
1.12 0.5
1.1 0.5
1.08 0.5
1.06 0.5
1.04 0.5
1.02 0.5
1 0.5
0.98 0.5
0.96 0.5
0.94 0.5
0.92 0.5
0.9 0.5
0.88 0.5
0.86 0.5
0.84 0.5
0.82 0.5
0.8 0.5
0.78 0.5
0.76 0.5
0.74 0.5
0.72 0.5
0.7 0.5
0.68 0.5
0.66 0.5
0.64 0.5
0.62 0.5
0.6 0.5
0.58 0.5
0.56 0.5
0.54 0.5
0.52 0.5
0.5 0.5
0.48 0.5
0.46 0.5
0.44 0.5
0.42 0.5
0.4 0.5
0.38 0.5
0.36 0.5
0.34 0.5
0.32 0.5
0.3025 0.5
0.2875 0.5
0.275 0.5
0.265 0.5
0.2575 0.5
0.2525 0.5
0.25 0.5
0.25 0.5
0.25 0.5
0.25 0.5
0.25 0.5
0.25 0.5
0.25 0.5
0.25 0.5
0.25 0.5
0.25 0.5
0.25 0.5
0.25 0.5
0.25 0.5
0.25 0.5
0.25 0.5
0.25 0.5
0.25 0.5
0.25 0.5
0.25 0.4975
0.25 0.4925
0.25 0.485
0.25 0.475
0.25 0.4625
0.25 0.4475
0.25 0.43
0.25 0.41
0.25 0.39
0.25 0.37
0.25 0.35
0.25 0.33
0.25 0.31
0.25 0.29
0.25 0.27
0.25 0.25
0.25 0.23
0.25 0.21
0.25 0.19
0.25 0.17
0.25 0.15
0.249999999999999 0.13
0.249999999999999 0.11
0.249999999999999 0.09
0.249999999999999 0.07
0.249999999999999 0.05
0.249999999999999 0.03
0.249999999999999 0.01
0.249999999999999 -0.00999999999999998
0.249999999999999 -0.03
0.249999999999999 -0.05
0.249999999999999 -0.07
0.249999999999999 -0.09
0.249999999999999 -0.11
0.249999999999999 -0.13
0.249999999999999 -0.15
0.249999999999999 -0.17
0.249999999999999 -0.19
0.249999999999999 -0.21
0.249999999999999 -0.23
0.249999999999999 -0.25
0.249999999999999 -0.27
0.249999999999999 -0.29
0.249999999999999 -0.31
0.249999999999999 -0.33
0.249999999999999 -0.35
0.249999999999999 -0.37
0.249999999999999 -0.39
0.249999999999999 -0.41
0.249999999999999 -0.43
0.249999999999999 -0.4475
0.249999999999999 -0.4625
0.249999999999999 -0.475
0.249999999999999 -0.485
0.249999999999999 -0.4925
0.249999999999999 -0.4975
0.249999999999999 -0.5
};
\addplot [semithick, blue, opacity=\opacityRef, forget plot]
table {%
0.25 -0.5
0.250626727938652 -0.500109165906906
0.253930762410164 -0.500120759010315
0.261408343911171 -0.499985665082932
0.2711191624403 -0.499955028295517
0.283589825034142 -0.499994188547134
0.298410221934319 -0.499925553798676
0.315663680434227 -0.49979105591774
0.33572556078434 -0.499666899442673
0.356736376881599 -0.499502450227737
0.376953765749931 -0.499644428491592
0.397610828280449 -0.499555170536041
0.418334320187569 -0.499559849500656
0.438662305474281 -0.499334990978241
0.4595937281847 -0.499438315629959
0.480146840214729 -0.49924173951149
0.500791683793068 -0.499105215072632
0.521337524056435 -0.498945653438568
0.541751816868782 -0.498882114887238
0.562454834580421 -0.498837798833847
0.583011701703072 -0.498778015375137
0.603434756398201 -0.49867171049118
0.623824432492256 -0.498669296503067
0.6444441229105 -0.498543053865433
0.665056005120277 -0.498632550239563
0.685824885964394 -0.498715758323669
0.706390932202339 -0.498730838298798
0.726990893483162 -0.498647689819336
0.747565224766731 -0.498544961214066
0.768130257725716 -0.498431414365768
0.788667693734169 -0.498429864645004
0.809083178639412 -0.498414069414139
0.829409971833229 -0.498354434967041
0.850012794137001 -0.497988492250443
0.870565012097359 -0.497951239347458
0.891200676560402 -0.49797385931015
0.911851480603218 -0.498222053050995
0.932262435555458 -0.498111546039581
0.952922537922859 -0.497938811779022
0.973527982831001 -0.497783660888672
0.993690088391304 -0.497255384922028
1.0144215375185 -0.497356623411179
1.03457207977772 -0.497148931026459
1.05514760315418 -0.497154057025909
1.07562310993671 -0.496897101402283
1.09605748951435 -0.496825814247131
1.11691279709339 -0.496989011764526
1.13750453293324 -0.497251600027084
1.15806539356709 -0.497157871723175
1.17896766960621 -0.497269719839096
1.19938452541828 -0.497373461723328
1.21861465275288 -0.497366666793823
1.23403425514698 -0.497423082590103
1.24760015308857 -0.497501760721207
1.25840218365192 -0.497450470924377
1.26643164455891 -0.497451961040497
1.27206249535084 -0.497294455766678
1.27525027096272 -0.497320234775543
1.2761916667223 -0.497327953577042
1.2762783318758 -0.497409522533417
1.27747042477131 -0.497049361467361
1.27702911198139 -0.497015923261642
1.27703876793385 -0.496631324291229
1.27693767845631 -0.496688693761826
1.27703507244587 -0.496379941701889
1.2769537717104 -0.496288359165192
1.27690000832081 -0.496233701705933
1.27618236839771 -0.496470630168915
1.27606935799122 -0.496438980102539
1.27610690891743 -0.496235460042953
1.27591522037983 -0.496202290058136
1.27586300671101 -0.4964719414711
1.27587743103504 -0.496448874473572
1.27587051689625 -0.496645957231522
1.27585490047932 -0.496454298496246
1.27586431801319 -0.495944559574127
1.27577598392963 -0.492514997720718
1.27555437386036 -0.485428392887115
1.27535875141621 -0.475402474403381
1.27523250877857 -0.463547229766846
1.27494485676289 -0.448869198560715
1.27457399666309 -0.431164920330048
1.27424295246601 -0.411091536283493
1.27397830784321 -0.390353322029114
1.2736007720232 -0.37054580450058
1.27328915894032 -0.349501430988312
1.27294011414051 -0.32863649725914
1.27263362705708 -0.307807207107544
1.27210862934589 -0.287405014038086
1.27201993763447 -0.266967505216599
1.27177937328815 -0.24614155292511
1.27148766815662 -0.225588500499725
1.27103681862354 -0.205540135502815
1.27091486752033 -0.184447377920151
1.27081568539143 -0.163840085268021
1.27063031494617 -0.143600985407829
1.27034075558186 -0.122890621423721
1.26991184055805 -0.102557398378849
1.26940293610096 -0.081987701356411
1.26912827789783 -0.0615051053464413
1.26879222691059 -0.0412132740020752
1.26858659088612 -0.0203484371304512
1.26800556480885 7.35805369913578e-05
1.26775224506855 0.0211013704538345
1.26733739674091 0.041436955332756
1.26701100170612 0.0616360008716583
1.26655860245228 0.0821394249796867
1.26623351871967 0.103343732655048
1.26589031517506 0.123975306749344
1.26523144543171 0.144330129027367
1.26460058987141 0.16467396914959
1.26427884399891 0.185541599988937
1.26401670277119 0.205992802977562
1.26338894665241 0.226284816861153
1.26281960308552 0.247092351317406
1.26238544285297 0.267547458410263
1.26193495094776 0.287764817476273
1.26129288971424 0.308682173490524
1.26080413162708 0.329073160886765
1.26078493893147 0.349850475788116
1.26007457077503 0.369876712560654
1.25964231789112 0.390777349472046
1.25917883217335 0.411500751972198
1.25917179882526 0.431707888841629
1.25868792831898 0.452225565910339
1.25829906761646 0.471344411373138
1.25746496021748 0.48675325512886
1.25804205238819 0.500512480735779
1.25770504772663 0.511119246482849
1.25774390995502 0.51933342218399
1.25760813057423 0.524908065795898
1.25702197849751 0.52818500995636
1.2573209553957 0.52883380651474
1.25697799026966 0.52907121181488
1.25741727650166 0.53017258644104
1.25736756622791 0.529694676399231
1.25684793293476 0.529619693756104
1.2570493966341 0.529704630374908
1.2571384459734 0.529374122619629
1.25701983273029 0.529357135295868
1.25654788315296 0.529218554496765
1.25698514282703 0.52918940782547
1.25602908432484 0.529407799243927
1.25656659901142 0.529013335704803
1.2567185908556 0.528837203979492
1.25688321888447 0.528790950775146
1.25704069435596 0.528708159923553
1.25623519718647 0.528730690479279
1.25627525150776 0.528679728507996
1.25601287186146 0.528663337230682
1.25236244499683 0.528572261333466
1.24527557194233 0.528495013713837
1.23578281700611 0.528221130371094
1.22312898933887 0.527863681316376
1.2074820548296 0.527405738830566
1.19131596386433 0.526967346668243
1.17089922726154 0.526366889476776
1.15046258270741 0.525712788105011
1.13015063107014 0.525189280509949
1.10943634808064 0.524721145629883
1.08919306099415 0.524261236190796
1.06887049973011 0.523652613162994
1.04807801544666 0.523017406463623
1.02728505432606 0.522569000720978
1.00704355537891 0.521982848644257
0.986598804593086 0.521467447280884
0.966175094246864 0.520862638950348
0.945756748318672 0.520196974277496
0.925076618790627 0.5196533203125
0.90426941215992 0.518948674201965
0.884594276547432 0.518480777740479
0.863947048783302 0.51778918504715
0.843497112393379 0.517180263996124
0.822697356343269 0.516501009464264
0.802271321415901 0.516083896160126
0.782036378979683 0.515575706958771
0.761836603283882 0.514893412590027
0.74073089659214 0.514338910579681
0.720308735966682 0.513883769512177
0.699801102280617 0.51317310333252
0.679178848862648 0.5128373503685
0.658891335129738 0.512322187423706
0.638086751103401 0.5118008852005
0.61765743792057 0.511313617229462
0.597199872136116 0.510612547397614
0.576671198010445 0.510184764862061
0.556242361664772 0.509791553020477
0.535558834671974 0.509208083152771
0.514917984604836 0.508735299110413
0.494505301117897 0.508155941963196
0.47379769384861 0.50754725933075
0.453082486987114 0.507060885429382
0.432395175099373 0.506533622741699
0.411999806761742 0.505971789360046
0.391605690121651 0.505320429801941
0.371051147580147 0.504874646663666
0.350245580077171 0.50434011220932
0.329275563359261 0.503590941429138
0.308709338307381 0.503160297870636
0.288937017321587 0.50249856710434
0.273634657263756 0.502223491668701
0.26013071835041 0.501804828643799
0.248681411147118 0.501448631286621
0.240811437368393 0.501255571842194
0.234866410493851 0.500842332839966
0.231956213712692 0.500983893871307
0.231500864028931 0.500938177108765
0.23091547191143 0.500937402248383
0.229954674839973 0.500731229782104
0.230210244655609 0.500568449497223
0.230064570903778 0.500529825687408
0.230257526040077 0.500553667545319
0.230201929807663 0.50039142370224
0.230213925242424 0.500161349773407
0.230769380927086 0.500175595283508
0.230988010764122 0.500623822212219
0.230903178453445 0.500211477279663
0.231047466397285 0.500122606754303
0.231159433722496 0.500071227550507
0.231229245662689 0.50004631280899
0.231147423386574 0.500020861625671
0.231262341141701 0.500013947486877
0.231321945786476 0.500117659568787
0.231305465102196 0.499542117118835
0.231395319104195 0.496139883995056
0.231455743312836 0.488808751106262
0.23164501786232 0.479110032320023
0.231759607791901 0.466872841119766
0.231826156377792 0.452161341905594
0.232043921947479 0.434945493936539
0.232273668050766 0.414870142936707
0.232497170567513 0.394018203020096
0.232931926846504 0.373947232961655
0.232938706874847 0.353376775979996
0.232991248369217 0.332647979259491
0.233455121517181 0.311857491731644
0.233964666724205 0.290747225284576
0.234190300107002 0.270915925502777
0.234692797064781 0.250218510627747
0.234912037849426 0.229499906301498
0.235398545861244 0.207838162779808
0.235749796032906 0.187613800168037
0.235801354050636 0.168276011943817
0.2357217669487 0.147550284862518
0.235967323184013 0.126374825835228
0.236580237746239 0.105250403285027
0.236749023199081 0.0846251100301743
0.23676560819149 0.0652167797088623
0.237125992774963 0.0434707961976528
0.23761174082756 0.023018840700388
0.237513571977615 0.0026105847209692
0.237895265221596 -0.0179649461060762
0.238069742918015 -0.0379515700042248
0.23820073902607 -0.0581621862947941
0.238670542836189 -0.0797369480133057
0.238713145256042 -0.0992669388651848
0.238885775208473 -0.120091654360294
0.239202350378036 -0.141157731413841
0.239465445280075 -0.162223502993584
0.239778116345406 -0.182481333613396
0.240171208977699 -0.203136697411537
0.240357011556625 -0.223878353834152
0.240372374653816 -0.244020581245422
0.240875631570816 -0.264306396245956
0.241301730275154 -0.284986287355423
0.241402253508568 -0.305514007806778
0.241571441292763 -0.325919479131699
0.241784423589706 -0.346541196107864
0.242184996604919 -0.367580860853195
0.242250621318817 -0.387956738471985
0.242488890886307 -0.408572971820831
0.242848590016365 -0.42947182059288
0.242991641163826 -0.449509799480438
0.243098706007004 -0.46895045042038
0.243275061249733 -0.48475444316864
0.243510708212852 -0.497910350561142
0.24373160302639 -0.508797764778137
0.243690177798271 -0.516930460929871
0.243992194533348 -0.522664785385132
0.244014963507652 -0.525942862033844
0.243841230869293 -0.52623039484024
};
\addplot [semithick, red, dashed, forget plot]
table {%
0.25 -0.5
0.252558268640609 -0.499971215983182
0.257674800204547 -0.499913561943335
0.265349580546156 -0.499826822527541
0.275582586992214 -0.499710651911554
0.288373788282223 -0.499564571828345
0.303723144483551 -0.499387968751125
0.321630606880243 -0.499180090231924
0.342096117833915 -0.498940040397106
0.362561444756502 -0.498697137467738
0.383026586783779 -0.498451352589885
0.403491543037366 -0.498202656922469
0.423956312624558 -0.49795102163749
0.444420894638161 -0.497696417920264
0.464885288156327 -0.497438816969653
0.485349492242387 -0.497178189998295
0.505813505944689 -0.496914508232844
0.526277328296427 -0.4966477429142
0.546740958315482 -0.49637786529775
0.567204395004253 -0.496104846653605
0.587667637349495 -0.495828658266839
0.608130684322154 -0.495549271437727
0.628593534877201 -0.495266657481993
0.649056187953474 -0.494980787731044
0.669518642473505 -0.494691633532224
0.689980897343365 -0.49439916624905
0.710442951452497 -0.494103357261464
0.730904803673553 -0.493804177966079
0.751366452862231 -0.493501599776428
0.771827897857116 -0.493195594123215
0.792289137479511 -0.492886132454562
0.812750170533283 -0.492573186236266
0.833210995804693 -0.492256726952051
0.853671612062244 -0.491936726103822
0.874132018056508 -0.491613155211922
0.894592212519978 -0.491285985815386
0.915052194166895 -0.490955189472205
0.935511961693098 -0.49062073775958
0.955971513775856 -0.490282602274183
0.976430849073713 -0.489940754632423
0.996889966226327 -0.489595166470704
1.01734886385431 -0.489245809445691
1.03780754055907 -0.488892655234577
1.05826599492265 -0.488535675535346
1.07872422550757 -0.488174842067041
1.09918223085668 -0.487810126570036
1.11964000949299 -0.487441500806302
1.1400975599195 -0.48706893655968
1.16055488061908 -0.486692405636151
1.18101197005428 -0.486311879864113
1.20146882666719 -0.485927331094652
1.21936837110133 -0.485587306188302
1.23471068292331 -0.485293179242986
1.24749583154271 -0.485046149752488
1.25772387591864 -0.484847237098433
1.26539486430542 -0.484697275874471
1.27050883404033 -0.484596912039359
1.27306581137596 -0.484546599896135
1.2730658113589 -0.484546599895144
1.27305541155906 -0.484320911054072
1.2730238336127 -0.483870450269343
1.27294421772612 -0.483198810692961
1.27277401858219 -0.482315683800082
1.27245616754417 -0.481242223159833
1.27192174548361 -0.480017905211023
1.27109518948705 -0.478707989424014
1.26987791322201 -0.477382389353572
1.26861868448422 -0.476491262912181
1.26741310591075 -0.475930505649088
1.26634153930566 -0.475618318765773
1.26545420209816 -0.475475522024619
1.26477827798535 -0.475433459898834
1.26432554654506 -0.475438002031128
1.26409923112047 -0.475451102493
1.26409923107294 -0.475451102492838
1.26400510754809 -0.472893953643021
1.26381676720545 -0.467779675905825
1.26353397696839 -0.460108319156843
1.26315636493238 -0.449879963098326
1.26268342195054 -0.437094717113806
1.26211450389605 -0.421752720062896
1.26144883460045 -0.403854140014764
1.2606855094668 -0.383399173918339
1.25991927860775 -0.362944839063709
1.25915016511684 -0.342491133403313
1.25837819207707 -0.322038054872471
1.25760338256082 -0.301585601389565
1.25682575962964 -0.281133770856217
1.25604534633415 -0.260682561157473
1.25526216571386 -0.240231970161977
1.25447624079705 -0.219781995722154
1.25368759460061 -0.199332635674387
1.25289625012991 -0.178883887839198
1.25210223037863 -0.158435750021426
1.25130555832866 -0.137988220010405
1.25050625694995 -0.117541295580143
1.24970434920032 -0.0970949744895014
1.2488998580254 -0.0766492544823687
1.24809280635845 -0.0562041332878428
1.24728321712021 -0.0357596086204065
1.24647111321881 -0.0153156781801049
1.24565651754959 0.00512766034727754
1.244839452995 0.0255704092900399
1.24401994242446 0.0460125709903884
1.24319800869422 0.0664541478042603
1.24237367464724 0.0868951421011472
1.24154696311305 0.10733555626392
1.24071789690763 0.127775392688651
1.23988649883331 0.148214653784443
1.23905279167858 0.168653341973249
1.23821679821801 0.189091459689702
1.23737854121215 0.209529009380936
1.23653804340733 0.229965993506414
1.23569532753562 0.250402414537754
1.23485041631467 0.270838274958555
1.23400333244757 0.291273577264223
1.23315409862277 0.311708323961797
1.23230273751397 0.332142517569777
1.23144927177993 0.352576160617952
1.23059372406446 0.373009255647224
1.22973611699622 0.393441805209441
1.22887647318863 0.413873811867219
1.22801481523979 0.434305278193775
1.22715116573232 0.454736206772755
1.22628554723329 0.475166600198059
1.22552642790666 0.493042728464227
1.22487449160254 0.508364776283352
1.22433031870982 0.52113290203806
1.22389439056138 0.531347238274115
1.22356709317291 0.539007892120091
1.22334872031756 0.544114945632642
1.22323947593845 0.546668456066198
1.22323947590048 0.546668456066214
1.2230123362303 0.54664910964162
1.22255893634353 0.546599663012994
1.22188286957468 0.546493423413052
1.22099418533407 0.546288449972454
1.21991546682544 0.545929219980312
1.21868995460871 0.545349990456186
1.21739061938076 0.544480715471633
1.21610030698399 0.543230273381219
1.21526412455335 0.541961848316668
1.21476024252823 0.540762508740427
1.21448767411105 0.539705040879324
1.21435620536168 0.538833702471361
1.21429925392167 0.538171836563276
1.21427642250785 0.537729171531928
1.21429899313131 0.537497396084245
1.21429899310163 0.53749739608416
1.21173554316399 0.537351398434274
1.2066086675668 0.537059357198718
1.19891842689941 0.536621157703719
1.18866491799536 0.536036617118684
1.17584827376684 0.535305485487585
1.16046866297006 0.534427447211652
1.1425262899005 0.533402122984755
1.12202139401749 0.532229072183377
1.10151726555175 0.531054604210161
1.08101390213675 0.529878734481104
1.06051130139776 0.528701478429746
1.04000946095186 0.5275228515073
1.01950837840803 0.526342869182777
0.999008051367172 0.525161546943109
0.978508477422155 0.523978900293278
0.958009654157879 0.522794944756435
0.937511579151311 0.521609695874025
0.917014249971538 0.520423169205916
0.896517664179812 0.519235380330514
0.876021819329601 0.518046344844887
0.855526712966634 0.516856078364893
0.835032342628955 0.515664596525293
0.814538705846972 0.514471914979876
0.794045800143499 0.513278049401579
0.773553623033816 0.512083015482608
0.753062172025712 0.510886828934553
0.732571444619537 0.509689505488512
0.712081438308253 0.508491060895207
0.691592150577485 0.507291510925102
0.671103578905572 0.506090871368521
0.650615720763616 0.504889158035764
0.630128573615536 0.503686386757223
0.609642134918118 0.502482573383502
0.589156402121068 0.501277733785526
0.568671372667066 0.500071883854662
0.548187043991813 0.49886503950283
0.527703413524089 0.497657216662618
0.507220478685805 0.496448431287396
0.486738236892053 0.495238699351429
0.466256685551165 0.494028036849988
0.445775822064762 0.492816459799465
0.425295643827809 0.491603984237482
0.404816148228671 0.490390626223003
0.384337332649168 0.489176401836445
0.363859194464625 0.487961327179788
0.343381731043935 0.486745418376685
0.322904939749608 0.48552869157257
0.302428817937827 0.484311162934768
0.281953362958509 0.483092848652601
0.261478572155355 0.4818737649375
0.243563709029771 0.480806407637357
0.228208544470985 0.479891046434884
0.215412881797103 0.479127906813159
0.20517655624396 0.4785171732374
0.197499434529847 0.478058991869279
0.192381414497009 0.477753472810477
0.189822424830656 0.477600691872751
0.189822424856044 0.477600691872319
0.18984329041129 0.477361217804826
0.189891281488213 0.476888179768742
0.18998422190601 0.476195304214649
0.190156086957441 0.475302243681175
0.19046107666499 0.47423395694596
0.190973511005364 0.47302370084793
0.1917804370421 0.471720939032194
0.192990859960637 0.470375213510415
0.194255394688102 0.469435488118
0.195465103825654 0.468815206001496
0.196534139830494 0.468445902524195
0.19741290176854 0.468255645457359
0.19807765083531 0.468178447005524
0.198520413044486 0.468160326901788
0.198740901208049 0.468162462817004
0.198740901269321 0.46816246281638
0.198945440782965 0.465612270025241
0.199354508471781 0.460511864695189
0.199968076306772 0.452861197525416
0.200786100039452 0.442660189857331
0.201808520127061 0.429908733995519
0.203035263052485 0.414606693655117
0.20446624304152 0.396753904535665
0.206101364180888 0.376350175021588
0.207736171974856 0.355945830121864
0.209370679915717 0.335540874152152
0.211004901527408 0.315135311429408
0.212638850365535 0.294729146271967
0.214272540017396 0.27432238299961
0.215905984101993 0.253915025933645
0.217539196270065 0.233507079396979
0.219172190204101 0.2130985477142
0.220804979618368 0.192689435211644
0.222437578258934 0.172279746217481
0.224069999903693 0.15186948506179
0.225702258362386 0.131458656076634
0.227334367476629 0.111047263596139
0.228966341119943 0.0906353119565765
0.230598193197773 0.0702228054964383
0.232229937647521 0.0498097485565179
0.233861588438572 0.0293961454799898
0.235493159572323 0.00898200061249
0.237124665082211 -0.0114326816978031
0.238756119033746 -0.0318478971000879
0.240387535524538 -0.0522636412408578
0.24201892868433 -0.0726799097638194
0.243650312675029 -0.0930966983098097
0.245281701690739 -0.113514002516714
0.246913109957795 -0.133931818019382
0.248544551734793 -0.154350140449545
0.250176041312631 -0.174768965435729
0.251807593014535 -0.195188288603176
0.253439221196103 -0.215608105573753
0.255070940245335 -0.236028411965869
0.256702764582677 -0.256449203394391
0.258334708661049 -0.276870475470555
0.259966786965892 -0.29729222380188
0.261599014015202 -0.317714443992083
0.263231404359571 -0.338137131640988
0.264863972582227 -0.358560282344441
0.266496733299076 -0.378983891694222
0.26812970115874 -0.399407955277952
0.269762890842605 -0.419832468679009
0.271396317064857 -0.440257427476434
0.27302999457253 -0.460682827244845
0.27466393814555 -0.481108663554342
0.276093884549649 -0.498981648418472
0.277319746923832 -0.514301630525306
0.278341443563315 -0.527068481124652
0.279158900790641 -0.537282093189924
0.279772055404735 -0.544942380708266
0.280180856703051 -0.550049278099581
0.280385268072754 -0.552602739765039
};
\addplot [semithick, green, dash pattern=on 1pt off 3pt on 3pt off 3pt, forget plot]
table {%
0.25 -0.5
0.2525 -0.5
0.2575 -0.5
0.265 -0.5
0.275 -0.5
0.2875 -0.5
0.3025 -0.5
0.32 -0.5
0.34 -0.5
0.36 -0.5
0.38 -0.5
0.4 -0.5
0.42 -0.5
0.44 -0.5
0.46 -0.5
0.48 -0.5
0.5 -0.5
0.52 -0.5
0.54 -0.5
0.56 -0.5
0.58 -0.5
0.6 -0.5
0.62 -0.5
0.64 -0.5
0.66 -0.5
0.68 -0.5
0.7 -0.5
0.72 -0.5
0.74 -0.5
0.76 -0.5
0.78 -0.5
0.8 -0.5
0.82 -0.5
0.84 -0.5
0.86 -0.5
0.88 -0.5
0.9 -0.5
0.92 -0.5
0.94 -0.5
0.96 -0.5
0.98 -0.5
1 -0.5
1.02 -0.5
1.04 -0.5
1.06 -0.5
1.08 -0.5
1.1 -0.5
1.12 -0.5
1.14 -0.5
1.16 -0.5
1.18 -0.5
1.1975 -0.5
1.2125 -0.5
1.225 -0.5
1.235 -0.5
1.2425 -0.5
1.2475 -0.5
1.25 -0.5
1.25 -0.5
1.25 -0.5
1.25 -0.5
1.25 -0.5
1.25 -0.5
1.25 -0.5
1.25 -0.5
1.25 -0.5
1.25 -0.5
1.25 -0.5
1.25 -0.5
1.25 -0.5
1.25 -0.5
1.25 -0.5
1.25 -0.5
1.25 -0.5
1.25 -0.5
1.25 -0.4975
1.25 -0.4925
1.25 -0.485
1.25 -0.475
1.25 -0.4625
1.25 -0.4475
1.25 -0.43
1.25 -0.41
1.25 -0.39
1.25 -0.37
1.25 -0.35
1.25 -0.33
1.25 -0.31
1.25 -0.29
1.25 -0.27
1.25 -0.25
1.25 -0.23
1.25 -0.21
1.25 -0.19
1.25 -0.17
1.25 -0.15
1.25 -0.13
1.25 -0.11
1.25 -0.09
1.25 -0.07
1.25 -0.05
1.25 -0.03
1.25 -0.01
1.25 0.00999999999999998
1.25 0.03
1.25 0.05
1.25 0.07
1.25 0.09
1.25 0.11
1.25 0.13
1.25 0.15
1.25 0.17
1.25 0.19
1.25 0.21
1.25 0.23
1.25 0.25
1.25 0.27
1.25 0.29
1.25 0.31
1.25 0.33
1.25 0.35
1.25 0.37
1.25 0.39
1.25 0.41
1.25 0.43
1.25 0.4475
1.25 0.4625
1.25 0.475
1.25 0.485
1.25 0.4925
1.25 0.4975
1.25 0.5
1.25 0.5
1.25 0.5
1.25 0.5
1.25 0.5
1.25 0.5
1.25 0.5
1.25 0.5
1.25 0.5
1.25 0.5
1.25 0.5
1.25 0.5
1.25 0.5
1.25 0.5
1.25 0.5
1.25 0.5
1.25 0.5
1.25 0.5
1.2475 0.5
1.2425 0.5
1.235 0.5
1.225 0.5
1.2125 0.5
1.1975 0.5
1.18 0.5
1.16 0.5
1.14 0.5
1.12 0.5
1.1 0.5
1.08 0.5
1.06 0.5
1.04 0.5
1.02 0.5
1 0.5
0.98 0.5
0.96 0.5
0.94 0.5
0.92 0.5
0.9 0.5
0.88 0.5
0.86 0.5
0.84 0.5
0.82 0.5
0.8 0.5
0.78 0.5
0.76 0.5
0.74 0.5
0.72 0.5
0.7 0.5
0.68 0.5
0.66 0.5
0.64 0.5
0.62 0.5
0.6 0.5
0.58 0.5
0.56 0.5
0.54 0.5
0.52 0.5
0.5 0.5
0.48 0.5
0.46 0.5
0.44 0.5
0.42 0.5
0.4 0.5
0.38 0.5
0.36 0.5
0.34 0.5
0.32 0.5
0.3025 0.5
0.2875 0.5
0.275 0.5
0.265 0.5
0.2575 0.5
0.2525 0.5
0.25 0.5
0.25 0.5
0.25 0.5
0.25 0.5
0.25 0.5
0.25 0.5
0.25 0.5
0.25 0.5
0.25 0.5
0.25 0.5
0.25 0.5
0.25 0.5
0.25 0.5
0.25 0.5
0.25 0.5
0.25 0.5
0.25 0.5
0.25 0.5
0.25 0.4975
0.25 0.4925
0.25 0.485
0.25 0.475
0.25 0.4625
0.25 0.4475
0.25 0.43
0.25 0.41
0.25 0.39
0.25 0.37
0.25 0.35
0.25 0.33
0.25 0.31
0.25 0.29
0.25 0.27
0.25 0.25
0.25 0.23
0.25 0.21
0.25 0.19
0.25 0.17
0.25 0.15
0.249999999999999 0.13
0.249999999999999 0.11
0.249999999999999 0.09
0.249999999999999 0.07
0.249999999999999 0.05
0.249999999999999 0.03
0.249999999999999 0.01
0.249999999999999 -0.00999999999999998
0.249999999999999 -0.03
0.249999999999999 -0.05
0.249999999999999 -0.07
0.249999999999999 -0.09
0.249999999999999 -0.11
0.249999999999999 -0.13
0.249999999999999 -0.15
0.249999999999999 -0.17
0.249999999999999 -0.19
0.249999999999999 -0.21
0.249999999999999 -0.23
0.249999999999999 -0.25
0.249999999999999 -0.27
0.249999999999999 -0.29
0.249999999999999 -0.31
0.249999999999999 -0.33
0.249999999999999 -0.35
0.249999999999999 -0.37
0.249999999999999 -0.39
0.249999999999999 -0.41
0.249999999999999 -0.43
0.249999999999999 -0.4475
0.249999999999999 -0.4625
0.249999999999999 -0.475
0.249999999999999 -0.485
0.249999999999999 -0.4925
0.249999999999999 -0.4975
0.249999999999999 -0.5
};
\addplot [semithick, blue, opacity=\opacityRef, forget plot]
table {%
0.25 -0.5
0.250390768051147 -0.500023603439331
0.253915131092072 -0.500058352947235
0.26162251830101 -0.499871253967285
0.270942986011505 -0.499658048152924
0.283197999000549 -0.499421119689941
0.297886908054352 -0.498934149742126
0.315260738134384 -0.498727262020111
0.334920346736908 -0.497913956642151
0.356570571660995 -0.497562170028687
0.376802951097488 -0.497385501861572
0.397537618875504 -0.496871799230576
0.418197929859161 -0.496526300907135
0.439166963100433 -0.49579131603241
0.459499359130859 -0.495327800512314
0.480120331048965 -0.494732737541199
0.500540196895599 -0.494413644075394
0.520801484584808 -0.49377316236496
0.54145359992981 -0.493302136659622
0.5619255900383 -0.492872834205627
0.582590341567993 -0.492431342601776
0.603226363658905 -0.492111891508102
0.623650014400482 -0.49168598651886
0.644309878349304 -0.491280287504196
0.665281891822815 -0.490774691104889
0.685523867607117 -0.490406155586243
0.706162691116333 -0.489846885204315
0.72660768032074 -0.489371865987778
0.746979951858521 -0.488839983940125
0.76740962266922 -0.48838010430336
0.788018584251404 -0.487800747156143
0.808342158794403 -0.487429469823837
0.829219102859497 -0.487112283706665
0.849871933460236 -0.486551254987717
0.870466232299805 -0.486187249422073
0.891040921211243 -0.485591471195221
0.911631882190704 -0.485120475292206
0.931986689567566 -0.484164208173752
0.952650249004364 -0.483993738889694
0.972759902477264 -0.483469039201736
0.993205308914185 -0.482884913682938
1.01376336812973 -0.48239877820015
1.03399938344955 -0.481800824403763
1.0549584031105 -0.481281489133835
1.07534676790237 -0.480885416269302
1.09581297636032 -0.480392336845398
1.11665862798691 -0.479689329862595
1.13711649179459 -0.47935226559639
1.15785306692123 -0.478848844766617
1.17822617292404 -0.478567510843277
1.1986957192421 -0.477841317653656
1.21739274263382 -0.477858483791351
1.23324126005173 -0.477439731359482
1.24661356210709 -0.477053731679916
1.25742262601852 -0.47686904668808
1.26582962274551 -0.476829946041107
1.27144223451614 -0.476772487163544
1.27427095174789 -0.476889163255692
1.27521401643753 -0.476874709129333
1.27564030885696 -0.476680189371109
1.27682656049728 -0.47662878036499
1.27652627229691 -0.476655274629593
1.27682727575302 -0.476512104272842
1.27688091993332 -0.476343035697937
1.27686458826065 -0.476044684648514
1.27673262357712 -0.475793242454529
1.27621310949326 -0.475573182106018
1.27601450681686 -0.476174116134644
1.27605336904526 -0.475634455680847
1.27594441175461 -0.47558319568634
1.27578514814377 -0.476019442081451
1.27570909261703 -0.476018905639648
1.27556782960892 -0.476106643676758
1.27561789751053 -0.476196318864822
1.27559095621109 -0.476400643587112
1.27557724714279 -0.475497990846634
1.27543181180954 -0.471969813108444
1.27509838342667 -0.464199185371399
1.27481681108475 -0.454809904098511
1.27436536550522 -0.442336618900299
1.27372008562088 -0.427544236183167
1.27300518751144 -0.409918487071991
1.27230542898178 -0.390164524316788
1.27138191461563 -0.369435220956802
1.27054768800735 -0.348899036645889
1.26979929208755 -0.328280329704285
1.26903134584427 -0.30764165520668
1.26844841241837 -0.287390142679214
1.26750308275223 -0.266557604074478
1.26681512594223 -0.246159344911575
1.26606017351151 -0.225307270884514
1.26535350084305 -0.204826101660728
1.26476782560349 -0.184198841452599
1.26391464471817 -0.164307966828346
1.26308566331863 -0.143425151705742
1.26228040456772 -0.122849404811859
1.26147264242172 -0.102235451340675
1.26071673631668 -0.0819951370358467
1.26002067327499 -0.0617710873484612
1.25906985998154 -0.0410020537674427
1.25809270143509 -0.0209505800157785
1.25730925798416 0.000533735146746039
1.2565661072731 0.0207181163132191
1.25567299127579 0.0410289876163006
1.25473302602768 0.061835553497076
1.25360375642776 0.0826123878359795
1.25256329774857 0.102382235229015
1.25169080495834 0.123533226549625
1.25095361471176 0.144054934382439
1.24986678361893 0.164059609174728
1.24879270792007 0.184826165437698
1.24787801504135 0.205343842506409
1.24694031476974 0.226192876696587
1.24599343538284 0.246158733963966
1.24506515264511 0.2670558989048
1.24412852525711 0.287898451089859
1.24264615774155 0.308235079050064
1.24210649728775 0.328571051359177
1.24076682329178 0.349419623613358
1.23980838060379 0.369700938463211
1.23944085836411 0.390330106019974
1.23831397294998 0.410484790802002
1.23710519075394 0.431143015623093
1.23642951250076 0.451704680919647
1.23473101854324 0.472475826740265
1.2348844408989 0.491348296403885
1.23393791913986 0.508064746856689
1.2329643368721 0.524531304836273
1.23219043016434 0.53478080034256
1.23267179727554 0.54090029001236
1.232312977314 0.547465622425079
1.23186510801315 0.549713611602783
1.23188728094101 0.550713181495667
1.23174566030502 0.550886154174805
1.23088723421097 0.552553653717041
1.23167842626572 0.552108585834503
1.23108726739883 0.552101671695709
1.23118215799332 0.552107751369476
1.23095494508743 0.551857590675354
1.2308457493782 0.551715493202209
1.23127418756485 0.551409423351288
1.23151308298111 0.551154971122742
1.23127883672714 0.551370024681091
1.23085981607437 0.55120187997818
1.23093086481094 0.551010429859161
1.2308492064476 0.550842046737671
1.23081332445145 0.55077338218689
1.2309085726738 0.550741314888
1.23061352968216 0.550679564476013
1.23042756319046 0.550715446472168
1.22722810506821 0.550623953342438
1.21982640028 0.550172984600067
1.21050781011581 0.549754500389099
1.1978914141655 0.549104511737823
1.18314224481583 0.548472225666046
1.16610306501389 0.547818779945374
1.1465328335762 0.546816051006317
1.125268638134 0.54604697227478
1.10458213090897 0.545133888721466
1.08443516492844 0.544279873371124
1.06429237127304 0.543389081954956
1.0437029004097 0.542460799217224
1.0225116610527 0.541455268859863
1.00199609994888 0.540579557418823
0.981163561344147 0.539676308631897
0.961474657058716 0.538621664047241
0.941210389137268 0.537849187850952
0.920483589172363 0.536937594413757
0.900176763534546 0.536064624786377
0.879982352256775 0.534925997257233
0.859249949455261 0.534151554107666
0.839112222194672 0.533083558082581
0.81887274980545 0.532267332077026
0.798316776752472 0.531288743019104
0.777638614177704 0.530353128910065
0.757460713386536 0.529505670070648
0.736276268959045 0.528560400009155
0.715944170951843 0.527793765068054
0.695368766784668 0.526905357837677
0.675037205219269 0.526027202606201
0.654758453369141 0.525141894817352
0.634100019931793 0.524364411830902
0.613171756267548 0.523376941680908
0.593203008174896 0.522698938846588
0.572415709495544 0.521831035614014
0.551926016807556 0.520887970924377
0.531070291996002 0.520081102848053
0.510594666004181 0.519195258617401
0.490296095609665 0.518608450889587
0.46976312994957 0.517700612545013
0.449220091104507 0.516698718070984
0.42851322889328 0.515509009361267
0.407941967248917 0.514783501625061
0.387477844953537 0.514086067676544
0.366867959499359 0.51328307390213
0.345916241407394 0.512377738952637
0.324941962957382 0.511292695999146
0.304481536149979 0.510410606861115
0.283475667238235 0.509306371212006
0.264488279819489 0.508769512176514
0.249024450778961 0.508231043815613
0.235157981514931 0.507446467876434
0.22459676861763 0.507049798965454
0.216472789645195 0.506647229194641
0.210975110530853 0.506217539310455
0.207727447152138 0.506079196929932
0.207064837217331 0.506064236164093
0.206510633230209 0.506008207798004
0.205765172839165 0.505706965923309
0.205784231424332 0.505871295928955
0.205866456031799 0.50580358505249
0.205863043665886 0.505796372890472
0.205853506922722 0.505465090274811
0.205941453576088 0.505313456058502
0.206344336271286 0.505206525325775
0.206804126501083 0.505824565887451
0.2067691385746 0.505668580532074
0.20684702694416 0.505570113658905
0.207015067338943 0.50564843416214
0.207211822271347 0.505902886390686
0.207319021224976 0.505985915660858
0.207372963428497 0.505967140197754
0.207451552152634 0.505956768989563
0.207459822297096 0.505321025848389
0.207665666937828 0.502206444740295
0.208146080374718 0.494780123233795
0.208697840571404 0.485260814428329
0.209245309233665 0.472836762666702
0.209740668535233 0.457923471927643
0.21052160859108 0.440411567687988
0.211399674415588 0.42079821228981
0.212577968835831 0.399726957082748
0.213301151990891 0.37916973233223
0.214574858546257 0.359017014503479
0.215493574738503 0.338275045156479
0.216671675443649 0.317892998456955
0.217575341463089 0.297498524188995
0.218708649277687 0.276914417743683
0.220105990767479 0.256716817617416
0.221140399575233 0.235771670937538
0.221973225474358 0.215264543890953
0.22310845553875 0.194357499480247
0.224125728011131 0.173708081245422
0.22519488632679 0.152793630957603
0.226392641663551 0.131629347801208
0.227152496576309 0.111779861152172
0.228361234068871 0.0907197296619415
0.229348689317703 0.0707378312945366
0.230248242616653 0.0508815385401249
0.231613531708717 0.0299147702753544
0.232483088970184 0.00934671284630895
0.233526334166527 -0.0114196930080652
0.234370246529579 -0.0315202698111534
0.235094904899597 -0.0522434376180172
0.236419275403023 -0.0726857334375381
0.237478032708168 -0.0937373861670494
0.238766267895699 -0.114918820559978
0.23958483338356 -0.134240478277206
0.240363195538521 -0.155013501644135
0.241795092821121 -0.175611302256584
0.242976263165474 -0.19572077691555
0.244212463498116 -0.216671258211136
0.24494706094265 -0.236939564347267
0.245981246232986 -0.257776021957397
0.247048020362854 -0.278559118509293
0.248385399580002 -0.298690378665924
0.24943083524704 -0.319422364234924
0.25058913230896 -0.340041995048523
0.251587390899658 -0.360584318637848
0.252610981464386 -0.381166487932205
0.25380876660347 -0.401351392269135
0.255087971687317 -0.421599715948105
0.255951136350632 -0.44275489449501
0.256529420614243 -0.462118834257126
0.25763350725174 -0.477540850639343
0.25813215970993 -0.491204082965851
0.258839040994644 -0.501919269561768
0.259425103664398 -0.510193884372711
0.259692549705505 -0.515949308872223
0.259707242250443 -0.519077897071838
0.259880900382996 -0.519667625427246
};
\addplot [semithick, red, dashed, forget plot]
table {%
0.25 -0.5
0.252558268640609 -0.499971215983182
0.257674800204547 -0.499913561943335
0.265349580546156 -0.499826822527541
0.275582586992214 -0.499710651911554
0.288373788282223 -0.499564571828345
0.303723144483551 -0.499387968751125
0.321630606880243 -0.499180090231924
0.342096117833915 -0.498940040397106
0.362561444756502 -0.498697137467738
0.383026586783779 -0.498451352589885
0.403491543037366 -0.498202656922469
0.423956312624558 -0.49795102163749
0.444420894638161 -0.497696417920264
0.464885288156327 -0.497438816969653
0.485349492242387 -0.497178189998295
0.505813505944689 -0.496914508232844
0.526277328296427 -0.4966477429142
0.546740958315482 -0.49637786529775
0.567204395004253 -0.496104846653605
0.587667637349495 -0.495828658266839
0.608130684322154 -0.495549271437727
0.628593534877201 -0.495266657481993
0.649056187953474 -0.494980787731044
0.669518642473505 -0.494691633532224
0.689980897343365 -0.49439916624905
0.710442951452497 -0.494103357261464
0.730904803673553 -0.493804177966079
0.751366452862231 -0.493501599776428
0.771827897857116 -0.493195594123215
0.792289137479511 -0.492886132454562
0.812750170533283 -0.492573186236266
0.833210995804693 -0.492256726952051
0.853671612062244 -0.491936726103822
0.874132018056508 -0.491613155211922
0.894592212519978 -0.491285985815386
0.915052194166895 -0.490955189472205
0.935511961693098 -0.49062073775958
0.955971513775856 -0.490282602274183
0.976430849073713 -0.489940754632423
0.996889966226327 -0.489595166470704
1.01734886385431 -0.489245809445691
1.03780754055907 -0.488892655234577
1.05826599492265 -0.488535675535346
1.07872422550757 -0.488174842067041
1.09918223085668 -0.487810126570036
1.11964000949299 -0.487441500806302
1.1400975599195 -0.48706893655968
1.16055488061908 -0.486692405636151
1.18101197005428 -0.486311879864113
1.20146882666719 -0.485927331094652
1.21936837110133 -0.485587306188302
1.23471068292331 -0.485293179242986
1.24749583154271 -0.485046149752488
1.25772387591864 -0.484847237098433
1.26539486430542 -0.484697275874471
1.27050883404033 -0.484596912039359
1.27306581137596 -0.484546599896135
1.2730658113589 -0.484546599895144
1.27305541155906 -0.484320911054072
1.2730238336127 -0.483870450269343
1.27294421772612 -0.483198810692961
1.27277401858219 -0.482315683800082
1.27245616754417 -0.481242223159833
1.27192174548361 -0.480017905211023
1.27109518948705 -0.478707989424014
1.26987791322201 -0.477382389353572
1.26861868448422 -0.476491262912181
1.26741310591075 -0.475930505649088
1.26634153930566 -0.475618318765773
1.26545420209816 -0.475475522024619
1.26477827798535 -0.475433459898834
1.26432554654506 -0.475438002031128
1.26409923112047 -0.475451102493
1.26409923107294 -0.475451102492838
1.26400510754809 -0.472893953643021
1.26381676720545 -0.467779675905825
1.26353397696839 -0.460108319156843
1.26315636493238 -0.449879963098326
1.26268342195054 -0.437094717113806
1.26211450389605 -0.421752720062896
1.26144883460045 -0.403854140014764
1.2606855094668 -0.383399173918339
1.25991927860775 -0.362944839063709
1.25915016511684 -0.342491133403313
1.25837819207707 -0.322038054872471
1.25760338256082 -0.301585601389565
1.25682575962964 -0.281133770856217
1.25604534633415 -0.260682561157473
1.25526216571386 -0.240231970161977
1.25447624079705 -0.219781995722154
1.25368759460061 -0.199332635674387
1.25289625012991 -0.178883887839198
1.25210223037863 -0.158435750021426
1.25130555832866 -0.137988220010405
1.25050625694995 -0.117541295580143
1.24970434920032 -0.0970949744895014
1.2488998580254 -0.0766492544823687
1.24809280635845 -0.0562041332878428
1.24728321712021 -0.0357596086204065
1.24647111321881 -0.0153156781801049
1.24565651754959 0.00512766034727754
1.244839452995 0.0255704092900399
1.24401994242446 0.0460125709903884
1.24319800869422 0.0664541478042603
1.24237367464724 0.0868951421011472
1.24154696311305 0.10733555626392
1.24071789690763 0.127775392688651
1.23988649883331 0.148214653784443
1.23905279167858 0.168653341973249
1.23821679821801 0.189091459689702
1.23737854121215 0.209529009380936
1.23653804340733 0.229965993506414
1.23569532753562 0.250402414537754
1.23485041631467 0.270838274958555
1.23400333244757 0.291273577264223
1.23315409862277 0.311708323961797
1.23230273751397 0.332142517569777
1.23144927177993 0.352576160617952
1.23059372406446 0.373009255647224
1.22973611699622 0.393441805209441
1.22887647318863 0.413873811867219
1.22801481523979 0.434305278193775
1.22715116573232 0.454736206772755
1.22628554723329 0.475166600198059
1.22552642790666 0.493042728464227
1.22487449160254 0.508364776283352
1.22433031870982 0.52113290203806
1.22389439056138 0.531347238274115
1.22356709317291 0.539007892120091
1.22334872031756 0.544114945632642
1.22323947593845 0.546668456066198
1.22323947590048 0.546668456066214
1.2230123362303 0.54664910964162
1.22255893634353 0.546599663012994
1.22188286957468 0.546493423413052
1.22099418533407 0.546288449972454
1.21991546682544 0.545929219980312
1.21868995460871 0.545349990456186
1.21739061938076 0.544480715471633
1.21610030698399 0.543230273381219
1.21526412455335 0.541961848316668
1.21476024252823 0.540762508740427
1.21448767411105 0.539705040879324
1.21435620536168 0.538833702471361
1.21429925392167 0.538171836563276
1.21427642250785 0.537729171531928
1.21429899313131 0.537497396084245
1.21429899310163 0.53749739608416
1.21173554316399 0.537351398434274
1.2066086675668 0.537059357198718
1.19891842689941 0.536621157703719
1.18866491799536 0.536036617118684
1.17584827376684 0.535305485487585
1.16046866297006 0.534427447211652
1.1425262899005 0.533402122984755
1.12202139401749 0.532229072183377
1.10151726555175 0.531054604210161
1.08101390213675 0.529878734481104
1.06051130139776 0.528701478429746
1.04000946095186 0.5275228515073
1.01950837840803 0.526342869182777
0.999008051367172 0.525161546943109
0.978508477422155 0.523978900293278
0.958009654157879 0.522794944756435
0.937511579151311 0.521609695874025
0.917014249971538 0.520423169205916
0.896517664179812 0.519235380330514
0.876021819329601 0.518046344844887
0.855526712966634 0.516856078364893
0.835032342628955 0.515664596525293
0.814538705846972 0.514471914979876
0.794045800143499 0.513278049401579
0.773553623033816 0.512083015482608
0.753062172025712 0.510886828934553
0.732571444619537 0.509689505488512
0.712081438308253 0.508491060895207
0.691592150577485 0.507291510925102
0.671103578905572 0.506090871368521
0.650615720763616 0.504889158035764
0.630128573615536 0.503686386757223
0.609642134918118 0.502482573383502
0.589156402121068 0.501277733785526
0.568671372667066 0.500071883854662
0.548187043991813 0.49886503950283
0.527703413524089 0.497657216662618
0.507220478685805 0.496448431287396
0.486738236892053 0.495238699351429
0.466256685551165 0.494028036849988
0.445775822064762 0.492816459799465
0.425295643827809 0.491603984237482
0.404816148228671 0.490390626223003
0.384337332649168 0.489176401836445
0.363859194464625 0.487961327179788
0.343381731043935 0.486745418376685
0.322904939749608 0.48552869157257
0.302428817937827 0.484311162934768
0.281953362958509 0.483092848652601
0.261478572155355 0.4818737649375
0.243563709029771 0.480806407637357
0.228208544470985 0.479891046434884
0.215412881797103 0.479127906813159
0.20517655624396 0.4785171732374
0.197499434529847 0.478058991869279
0.192381414497009 0.477753472810477
0.189822424830656 0.477600691872751
0.189822424856044 0.477600691872319
0.18984329041129 0.477361217804826
0.189891281488213 0.476888179768742
0.18998422190601 0.476195304214649
0.190156086957441 0.475302243681175
0.19046107666499 0.47423395694596
0.190973511005364 0.47302370084793
0.1917804370421 0.471720939032194
0.192990859960637 0.470375213510415
0.194255394688102 0.469435488118
0.195465103825654 0.468815206001496
0.196534139830494 0.468445902524195
0.19741290176854 0.468255645457359
0.19807765083531 0.468178447005524
0.198520413044486 0.468160326901788
0.198740901208049 0.468162462817004
0.198740901269321 0.46816246281638
0.198945440782965 0.465612270025241
0.199354508471781 0.460511864695189
0.199968076306772 0.452861197525416
0.200786100039452 0.442660189857331
0.201808520127061 0.429908733995519
0.203035263052485 0.414606693655117
0.20446624304152 0.396753904535665
0.206101364180888 0.376350175021588
0.207736171974856 0.355945830121864
0.209370679915717 0.335540874152152
0.211004901527408 0.315135311429408
0.212638850365535 0.294729146271967
0.214272540017396 0.27432238299961
0.215905984101993 0.253915025933645
0.217539196270065 0.233507079396979
0.219172190204101 0.2130985477142
0.220804979618368 0.192689435211644
0.222437578258934 0.172279746217481
0.224069999903693 0.15186948506179
0.225702258362386 0.131458656076634
0.227334367476629 0.111047263596139
0.228966341119943 0.0906353119565765
0.230598193197773 0.0702228054964383
0.232229937647521 0.0498097485565179
0.233861588438572 0.0293961454799898
0.235493159572323 0.00898200061249
0.237124665082211 -0.0114326816978031
0.238756119033746 -0.0318478971000879
0.240387535524538 -0.0522636412408578
0.24201892868433 -0.0726799097638194
0.243650312675029 -0.0930966983098097
0.245281701690739 -0.113514002516714
0.246913109957795 -0.133931818019382
0.248544551734793 -0.154350140449545
0.250176041312631 -0.174768965435729
0.251807593014535 -0.195188288603176
0.253439221196103 -0.215608105573753
0.255070940245335 -0.236028411965869
0.256702764582677 -0.256449203394391
0.258334708661049 -0.276870475470555
0.259966786965892 -0.29729222380188
0.261599014015202 -0.317714443992083
0.263231404359571 -0.338137131640988
0.264863972582227 -0.358560282344441
0.266496733299076 -0.378983891694222
0.26812970115874 -0.399407955277952
0.269762890842605 -0.419832468679009
0.271396317064857 -0.440257427476434
0.27302999457253 -0.460682827244845
0.27466393814555 -0.481108663554342
0.276093884549649 -0.498981648418472
0.277319746923832 -0.514301630525306
0.278341443563315 -0.527068481124652
0.279158900790641 -0.537282093189924
0.279772055404735 -0.544942380708266
0.280180856703051 -0.550049278099581
0.280385268072754 -0.552602739765039
};
\addplot [semithick, green, dash pattern=on 1pt off 3pt on 3pt off 3pt, forget plot]
table {%
0.25 -0.5
0.2525 -0.5
0.2575 -0.5
0.265 -0.5
0.275 -0.5
0.2875 -0.5
0.3025 -0.5
0.32 -0.5
0.34 -0.5
0.36 -0.5
0.38 -0.5
0.4 -0.5
0.42 -0.5
0.44 -0.5
0.46 -0.5
0.48 -0.5
0.5 -0.5
0.52 -0.5
0.54 -0.5
0.56 -0.5
0.58 -0.5
0.6 -0.5
0.62 -0.5
0.64 -0.5
0.66 -0.5
0.68 -0.5
0.7 -0.5
0.72 -0.5
0.74 -0.5
0.76 -0.5
0.78 -0.5
0.8 -0.5
0.82 -0.5
0.84 -0.5
0.86 -0.5
0.88 -0.5
0.9 -0.5
0.92 -0.5
0.94 -0.5
0.96 -0.5
0.98 -0.5
1 -0.5
1.02 -0.5
1.04 -0.5
1.06 -0.5
1.08 -0.5
1.1 -0.5
1.12 -0.5
1.14 -0.5
1.16 -0.5
1.18 -0.5
1.1975 -0.5
1.2125 -0.5
1.225 -0.5
1.235 -0.5
1.2425 -0.5
1.2475 -0.5
1.25 -0.5
1.25 -0.5
1.25 -0.5
1.25 -0.5
1.25 -0.5
1.25 -0.5
1.25 -0.5
1.25 -0.5
1.25 -0.5
1.25 -0.5
1.25 -0.5
1.25 -0.5
1.25 -0.5
1.25 -0.5
1.25 -0.5
1.25 -0.5
1.25 -0.5
1.25 -0.5
1.25 -0.4975
1.25 -0.4925
1.25 -0.485
1.25 -0.475
1.25 -0.4625
1.25 -0.4475
1.25 -0.43
1.25 -0.41
1.25 -0.39
1.25 -0.37
1.25 -0.35
1.25 -0.33
1.25 -0.31
1.25 -0.29
1.25 -0.27
1.25 -0.25
1.25 -0.23
1.25 -0.21
1.25 -0.19
1.25 -0.17
1.25 -0.15
1.25 -0.13
1.25 -0.11
1.25 -0.09
1.25 -0.07
1.25 -0.05
1.25 -0.03
1.25 -0.01
1.25 0.00999999999999998
1.25 0.03
1.25 0.05
1.25 0.07
1.25 0.09
1.25 0.11
1.25 0.13
1.25 0.15
1.25 0.17
1.25 0.19
1.25 0.21
1.25 0.23
1.25 0.25
1.25 0.27
1.25 0.29
1.25 0.31
1.25 0.33
1.25 0.35
1.25 0.37
1.25 0.39
1.25 0.41
1.25 0.43
1.25 0.4475
1.25 0.4625
1.25 0.475
1.25 0.485
1.25 0.4925
1.25 0.4975
1.25 0.5
1.25 0.5
1.25 0.5
1.25 0.5
1.25 0.5
1.25 0.5
1.25 0.5
1.25 0.5
1.25 0.5
1.25 0.5
1.25 0.5
1.25 0.5
1.25 0.5
1.25 0.5
1.25 0.5
1.25 0.5
1.25 0.5
1.25 0.5
1.2475 0.5
1.2425 0.5
1.235 0.5
1.225 0.5
1.2125 0.5
1.1975 0.5
1.18 0.5
1.16 0.5
1.14 0.5
1.12 0.5
1.1 0.5
1.08 0.5
1.06 0.5
1.04 0.5
1.02 0.5
1 0.5
0.98 0.5
0.96 0.5
0.94 0.5
0.92 0.5
0.9 0.5
0.88 0.5
0.86 0.5
0.84 0.5
0.82 0.5
0.8 0.5
0.78 0.5
0.76 0.5
0.74 0.5
0.72 0.5
0.7 0.5
0.68 0.5
0.66 0.5
0.64 0.5
0.62 0.5
0.6 0.5
0.58 0.5
0.56 0.5
0.54 0.5
0.52 0.5
0.5 0.5
0.48 0.5
0.46 0.5
0.44 0.5
0.42 0.5
0.4 0.5
0.38 0.5
0.36 0.5
0.34 0.5
0.32 0.5
0.3025 0.5
0.2875 0.5
0.275 0.5
0.265 0.5
0.2575 0.5
0.2525 0.5
0.25 0.5
0.25 0.5
0.25 0.5
0.25 0.5
0.25 0.5
0.25 0.5
0.25 0.5
0.25 0.5
0.25 0.5
0.25 0.5
0.25 0.5
0.25 0.5
0.25 0.5
0.25 0.5
0.25 0.5
0.25 0.5
0.25 0.5
0.25 0.5
0.25 0.4975
0.25 0.4925
0.25 0.485
0.25 0.475
0.25 0.4625
0.25 0.4475
0.25 0.43
0.25 0.41
0.25 0.39
0.25 0.37
0.25 0.35
0.25 0.33
0.25 0.31
0.25 0.29
0.25 0.27
0.25 0.25
0.25 0.23
0.25 0.21
0.25 0.19
0.25 0.17
0.25 0.15
0.249999999999999 0.13
0.249999999999999 0.11
0.249999999999999 0.09
0.249999999999999 0.07
0.249999999999999 0.05
0.249999999999999 0.03
0.249999999999999 0.01
0.249999999999999 -0.00999999999999998
0.249999999999999 -0.03
0.249999999999999 -0.05
0.249999999999999 -0.07
0.249999999999999 -0.09
0.249999999999999 -0.11
0.249999999999999 -0.13
0.249999999999999 -0.15
0.249999999999999 -0.17
0.249999999999999 -0.19
0.249999999999999 -0.21
0.249999999999999 -0.23
0.249999999999999 -0.25
0.249999999999999 -0.27
0.249999999999999 -0.29
0.249999999999999 -0.31
0.249999999999999 -0.33
0.249999999999999 -0.35
0.249999999999999 -0.37
0.249999999999999 -0.39
0.249999999999999 -0.41
0.249999999999999 -0.43
0.249999999999999 -0.4475
0.249999999999999 -0.4625
0.249999999999999 -0.475
0.249999999999999 -0.485
0.249999999999999 -0.4925
0.249999999999999 -0.4975
0.249999999999999 -0.5
};
\addplot [semithick, blue, opacity=\opacityRef, forget plot]
table {%
0.25 -0.5
0.250656127929688 -0.50003856420517
0.253857731819153 -0.499936997890472
0.261424660682678 -0.499895453453064
0.271072596311569 -0.499988377094269
0.283211946487427 -0.50006091594696
0.298185437917709 -0.500015616416931
0.315571486949921 -0.499980390071869
0.335693120956421 -0.500084757804871
0.357114642858505 -0.499843060970306
0.377084761857986 -0.499844431877136
0.397879868745804 -0.499709725379944
0.418456703424454 -0.499656856060028
0.439139634370804 -0.499501883983612
0.459440141916275 -0.499338388442993
0.480033755302429 -0.499295175075531
0.500453501939774 -0.49912166595459
0.521309167146683 -0.499074339866638
0.541937738656998 -0.498938322067261
0.562595933675766 -0.499104022979736
0.583279877901077 -0.499027609825134
0.603658586740494 -0.498975098133087
0.62438628077507 -0.498869299888611
0.644908398389816 -0.498860955238342
0.665395885705948 -0.498790562152863
0.685700446367264 -0.498838007450104
0.706320196390152 -0.498789608478546
0.726944357156754 -0.49871289730072
0.747729808092117 -0.498720169067383
0.768103212118149 -0.499025940895081
0.788774222135544 -0.498765707015991
0.809267491102219 -0.498779714107513
0.82995018362999 -0.498710036277771
0.850572198629379 -0.498639345169067
0.870963126420975 -0.498612940311432
0.89144429564476 -0.498421490192413
0.911776095628738 -0.49849534034729
0.932665139436722 -0.49831759929657
0.952987462282181 -0.49803301692009
0.973696559667587 -0.49795937538147
0.993828028440475 -0.49789434671402
1.01438567042351 -0.498036593198776
1.03538092970848 -0.49792468547821
1.05554601550102 -0.497680723667145
1.07625398039818 -0.497593730688095
1.0966964662075 -0.49761825799942
1.11715051531792 -0.497532159090042
1.13770446181297 -0.497644275426865
1.15820416808128 -0.497712880373001
1.17878183722496 -0.497696220874786
1.19949862360954 -0.497611463069916
1.21889099478722 -0.497857958078384
1.23441025614738 -0.497850149869919
1.24784585833549 -0.497821122407913
1.2587818801403 -0.498163014650345
1.26712617278099 -0.498070955276489
1.27263256907463 -0.498105227947235
1.27578064799309 -0.497982114553452
1.27608236670494 -0.497948974370956
1.27670022845268 -0.498081535100937
1.27785620093346 -0.49790433049202
1.27761861681938 -0.497751832008362
1.2778328359127 -0.497647345066071
1.27793964743614 -0.497434824705124
1.27794679999352 -0.497462123632431
1.277757614851 -0.49721822142601
1.2774176299572 -0.496771931648254
1.27690860629082 -0.497430622577667
1.27685841917992 -0.497176200151443
1.27686497569084 -0.497220039367676
1.27667173743248 -0.497327238321304
1.27656337618828 -0.497465401887894
1.27653166651726 -0.49754324555397
1.27644297480583 -0.497409850358963
1.2763646543026 -0.49745774269104
1.27638229727745 -0.49742192029953
1.27628466486931 -0.493366330862045
1.27607008814812 -0.486060589551926
1.27574798464775 -0.476734966039658
1.27544268965721 -0.464220076799393
1.27510234713554 -0.449286073446274
1.27476891875267 -0.431702613830566
1.27449843287468 -0.411877542734146
1.27420601248741 -0.391213119029999
1.27374169230461 -0.370546162128448
1.27319392561913 -0.350075751543045
1.2729866206646 -0.329397171735764
1.27274259924889 -0.309066474437714
1.2725758254528 -0.288422882556915
1.27204510569572 -0.26775187253952
1.27161046862602 -0.247480481863022
1.27134510874748 -0.226611316204071
1.27100989222527 -0.205731242895126
1.27074274420738 -0.185672268271446
1.27056011557579 -0.165024027228355
1.27012822031975 -0.144435957074165
1.26980456709862 -0.124052718281746
1.26950380206108 -0.103499852120876
1.26927635073662 -0.0830783322453499
1.26889941096306 -0.0621067062020302
1.26851031184196 -0.041886780411005
1.26798424124718 -0.0212876535952091
1.26732477545738 -0.00114883040077984
1.267015427351 0.0200595557689667
1.26658448576927 0.0406086556613445
1.26624736189842 0.0608082748949528
1.26591047644615 0.0815570428967476
1.26524850726128 0.101940527558327
1.26474687457085 0.122160546481609
1.26447150111198 0.143121033906937
1.26401206851006 0.163892835378647
1.26316794753075 0.183549776673317
1.26262184977531 0.204519093036652
1.26198622584343 0.22542342543602
1.26134976744652 0.245479136705399
1.26085528731346 0.26605024933815
1.26043567061424 0.286878943443298
1.25946959853172 0.30741024017334
1.25928422808647 0.328012555837631
1.25852021574974 0.348715782165527
1.2583734691143 0.368601083755493
1.25759586691856 0.38959476351738
1.25768467783928 0.41021528840065
1.25699922442436 0.430850684642792
1.25615319609642 0.451171338558197
1.25567862391472 0.470344930887222
1.2554986178875 0.48569643497467
1.25519773364067 0.499242782592773
1.25507733225822 0.510013341903687
1.25492545962334 0.518379390239716
1.25488123297691 0.524040400981903
1.25423392653465 0.526967406272888
1.2541945874691 0.527779936790466
1.25416192412376 0.52798455953598
1.25428387522697 0.528805375099182
1.25429424643517 0.528315424919128
1.2537839114666 0.528353154659271
1.25407680869102 0.528173804283142
1.25357374548912 0.528098165988922
1.25387701392174 0.528439164161682
1.2539094388485 0.528365194797516
1.25377497076988 0.527592420578003
1.25382539629936 0.527656078338623
1.25285992026329 0.528097212314606
1.25300714373589 0.52815043926239
1.25318321585655 0.528144717216492
1.25312694907188 0.528088510036469
1.25303813815117 0.528161942958832
1.253019541502 0.52802711725235
1.25268289446831 0.528016746044159
1.24799606204033 0.527962923049927
1.24193844199181 0.527876019477844
1.23214313387871 0.527866303920746
1.21976801753044 0.527752995491028
1.20559379458427 0.527576208114624
1.18782207369804 0.527447640895844
1.16825303435326 0.527226448059082
1.14724585413933 0.527006447315216
1.12719413638115 0.526872515678406
1.10612055659294 0.526706635951996
1.08567866683006 0.526532471179962
1.06499180197716 0.526305198669434
1.04399046301842 0.526175439357758
1.02412900328636 0.526020050048828
1.00307688117027 0.525991857051849
0.983227103948593 0.525656819343567
0.962808221578598 0.525493860244751
0.941653043031693 0.525297582149506
0.921907156705856 0.525102853775024
0.901384443044662 0.524894535541534
0.880701750516891 0.52471250295639
0.860292345285416 0.52445924282074
0.839658170938492 0.524151623249054
0.819567173719406 0.524106383323669
0.799138277769089 0.523875176906586
0.778501063585281 0.52372533082962
0.758177906274796 0.523550629615784
0.737842351198196 0.523411273956299
0.717185705900192 0.523225426673889
0.696631222963333 0.523113250732422
0.675699144601822 0.522947430610657
0.655068129301071 0.522918105125427
0.634478718042374 0.522773206233978
0.613642364740372 0.522466361522675
0.593169361352921 0.522491097450256
0.572460323572159 0.522489845752716
0.552210122346878 0.522352337837219
0.531907945871353 0.522122025489807
0.511148065328598 0.522052049636841
0.490937501192093 0.521834671497345
0.470148205757141 0.521786212921143
0.449273645877838 0.521532833576202
0.428599327802658 0.521311223506927
0.408126264810562 0.521232724189758
0.387457489967346 0.521056830883026
0.366472989320755 0.521072685718536
0.346257597208023 0.521068334579468
0.325402498245239 0.520760118961334
0.304997533559799 0.520229458808899
0.285572171211243 0.520352005958557
0.269810169935226 0.520211279392242
0.256212323904037 0.520102798938751
0.245506539940834 0.520184397697449
0.237203940749168 0.520156145095825
0.231500133872032 0.520144999027252
0.22826711833477 0.519933044910431
0.227694034576416 0.519914269447327
0.227323964238167 0.52000480890274
0.225979506969452 0.519715309143066
0.226444751024246 0.519832909107208
0.226373583078384 0.519535183906555
0.226400122046471 0.519202649593353
0.226532354950905 0.51900327205658
0.226533651351929 0.51907080411911
0.226801231503487 0.51903772354126
0.227331429719925 0.519446492195129
0.227414429187775 0.519291281700134
0.22756227850914 0.51940792798996
0.227700784802437 0.519648849964142
0.227725684642792 0.519654273986816
0.227733194828033 0.519565641880035
0.227766007184982 0.519379556179047
0.227813139557838 0.519492387771606
0.227773323655128 0.518850684165955
0.2277762144804 0.515487492084503
0.227891057729721 0.50826221704483
0.227992877364159 0.498562067747116
0.228255718946457 0.486154705286026
0.228483408689499 0.471096634864807
0.228561848402023 0.453577995300293
0.228918835520744 0.433900952339172
0.229412287473679 0.412765443325043
0.229383215308189 0.39287343621254
0.229677632451057 0.372029334306717
0.230215892195702 0.351458638906479
0.230406299233437 0.331342130899429
0.230713710188866 0.310921221971512
0.231140062212944 0.29022604227066
0.231516465544701 0.269463062286377
0.231870993971825 0.248939380049706
0.232194200158119 0.227724775671959
0.232532665133476 0.2072444409132
0.23331880569458 0.186215579509735
0.233444303274155 0.165562361478806
0.233430236577988 0.146146804094315
0.234049275517464 0.12589742988348
0.234506651759148 0.104115262627602
0.234686881303787 0.0838550478219986
0.234545528888702 0.063685841858387
0.234736949205399 0.0435168109834194
0.235194772481918 0.0226905271410942
0.235589876770973 0.0022104864474386
0.235907688736916 -0.0191186685115099
0.236106663942337 -0.0394637435674667
0.236267477273941 -0.0598168522119522
0.236842527985573 -0.0813687667250633
0.237349778413773 -0.101112060248852
0.237454131245613 -0.12161498516798
0.237677574157715 -0.142173364758492
0.237783789634705 -0.162970259785652
0.238097086548805 -0.183531045913696
0.238727629184723 -0.204522952437401
0.238976985216141 -0.225501298904419
0.239145621657372 -0.245588719844818
0.239592283964157 -0.265733480453491
0.239792302250862 -0.286548137664795
0.240043014287949 -0.306607455015182
0.240698099136353 -0.327544689178467
0.240947842597961 -0.347690373659134
0.240918427705765 -0.368605643510818
0.241147458553314 -0.388997167348862
0.241519808769226 -0.40998238325119
0.241893336176872 -0.430488437414169
0.242079377174377 -0.449777394533157
0.242129802703857 -0.465172082185745
0.242517992854118 -0.478695303201675
0.242472037672997 -0.489531934261322
0.242792651057243 -0.49803626537323
0.242654800415039 -0.503581404685974
0.242922306060791 -0.506524443626404
0.242937505245209 -0.507153391838074
};
\addplot [semithick, red, dashed, forget plot]
table {%
0.25 -0.5
0.252558268640609 -0.499971215983182
0.257674800204547 -0.499913561943335
0.265349580546156 -0.499826822527541
0.275582586992214 -0.499710651911554
0.288373788282223 -0.499564571828345
0.303723144483551 -0.499387968751125
0.321630606880243 -0.499180090231924
0.342096117833915 -0.498940040397106
0.362561444756502 -0.498697137467738
0.383026586783779 -0.498451352589885
0.403491543037366 -0.498202656922469
0.423956312624558 -0.49795102163749
0.444420894638161 -0.497696417920264
0.464885288156327 -0.497438816969653
0.485349492242387 -0.497178189998295
0.505813505944689 -0.496914508232844
0.526277328296427 -0.4966477429142
0.546740958315482 -0.49637786529775
0.567204395004253 -0.496104846653605
0.587667637349495 -0.495828658266839
0.608130684322154 -0.495549271437727
0.628593534877201 -0.495266657481993
0.649056187953474 -0.494980787731044
0.669518642473505 -0.494691633532224
0.689980897343365 -0.49439916624905
0.710442951452497 -0.494103357261464
0.730904803673553 -0.493804177966079
0.751366452862231 -0.493501599776428
0.771827897857116 -0.493195594123215
0.792289137479511 -0.492886132454562
0.812750170533283 -0.492573186236266
0.833210995804693 -0.492256726952051
0.853671612062244 -0.491936726103822
0.874132018056508 -0.491613155211922
0.894592212519978 -0.491285985815386
0.915052194166895 -0.490955189472205
0.935511961693098 -0.49062073775958
0.955971513775856 -0.490282602274183
0.976430849073713 -0.489940754632423
0.996889966226327 -0.489595166470704
1.01734886385431 -0.489245809445691
1.03780754055907 -0.488892655234577
1.05826599492265 -0.488535675535346
1.07872422550757 -0.488174842067041
1.09918223085668 -0.487810126570036
1.11964000949299 -0.487441500806302
1.1400975599195 -0.48706893655968
1.16055488061908 -0.486692405636151
1.18101197005428 -0.486311879864113
1.20146882666719 -0.485927331094652
1.21936837110133 -0.485587306188302
1.23471068292331 -0.485293179242986
1.24749583154271 -0.485046149752488
1.25772387591864 -0.484847237098433
1.26539486430542 -0.484697275874471
1.27050883404033 -0.484596912039359
1.27306581137596 -0.484546599896135
1.2730658113589 -0.484546599895144
1.27305541155906 -0.484320911054072
1.2730238336127 -0.483870450269343
1.27294421772612 -0.483198810692961
1.27277401858219 -0.482315683800082
1.27245616754417 -0.481242223159833
1.27192174548361 -0.480017905211023
1.27109518948705 -0.478707989424014
1.26987791322201 -0.477382389353572
1.26861868448422 -0.476491262912181
1.26741310591075 -0.475930505649088
1.26634153930566 -0.475618318765773
1.26545420209816 -0.475475522024619
1.26477827798535 -0.475433459898834
1.26432554654506 -0.475438002031128
1.26409923112047 -0.475451102493
1.26409923107294 -0.475451102492838
1.26400510754809 -0.472893953643021
1.26381676720545 -0.467779675905825
1.26353397696839 -0.460108319156843
1.26315636493238 -0.449879963098326
1.26268342195054 -0.437094717113806
1.26211450389605 -0.421752720062896
1.26144883460045 -0.403854140014764
1.2606855094668 -0.383399173918339
1.25991927860775 -0.362944839063709
1.25915016511684 -0.342491133403313
1.25837819207707 -0.322038054872471
1.25760338256082 -0.301585601389565
1.25682575962964 -0.281133770856217
1.25604534633415 -0.260682561157473
1.25526216571386 -0.240231970161977
1.25447624079705 -0.219781995722154
1.25368759460061 -0.199332635674387
1.25289625012991 -0.178883887839198
1.25210223037863 -0.158435750021426
1.25130555832866 -0.137988220010405
1.25050625694995 -0.117541295580143
1.24970434920032 -0.0970949744895014
1.2488998580254 -0.0766492544823687
1.24809280635845 -0.0562041332878428
1.24728321712021 -0.0357596086204065
1.24647111321881 -0.0153156781801049
1.24565651754959 0.00512766034727754
1.244839452995 0.0255704092900399
1.24401994242446 0.0460125709903884
1.24319800869422 0.0664541478042603
1.24237367464724 0.0868951421011472
1.24154696311305 0.10733555626392
1.24071789690763 0.127775392688651
1.23988649883331 0.148214653784443
1.23905279167858 0.168653341973249
1.23821679821801 0.189091459689702
1.23737854121215 0.209529009380936
1.23653804340733 0.229965993506414
1.23569532753562 0.250402414537754
1.23485041631467 0.270838274958555
1.23400333244757 0.291273577264223
1.23315409862277 0.311708323961797
1.23230273751397 0.332142517569777
1.23144927177993 0.352576160617952
1.23059372406446 0.373009255647224
1.22973611699622 0.393441805209441
1.22887647318863 0.413873811867219
1.22801481523979 0.434305278193775
1.22715116573232 0.454736206772755
1.22628554723329 0.475166600198059
1.22552642790666 0.493042728464227
1.22487449160254 0.508364776283352
1.22433031870982 0.52113290203806
1.22389439056138 0.531347238274115
1.22356709317291 0.539007892120091
1.22334872031756 0.544114945632642
1.22323947593845 0.546668456066198
1.22323947590048 0.546668456066214
1.2230123362303 0.54664910964162
1.22255893634353 0.546599663012994
1.22188286957468 0.546493423413052
1.22099418533407 0.546288449972454
1.21991546682544 0.545929219980312
1.21868995460871 0.545349990456186
1.21739061938076 0.544480715471633
1.21610030698399 0.543230273381219
1.21526412455335 0.541961848316668
1.21476024252823 0.540762508740427
1.21448767411105 0.539705040879324
1.21435620536168 0.538833702471361
1.21429925392167 0.538171836563276
1.21427642250785 0.537729171531928
1.21429899313131 0.537497396084245
1.21429899310163 0.53749739608416
1.21173554316399 0.537351398434274
1.2066086675668 0.537059357198718
1.19891842689941 0.536621157703719
1.18866491799536 0.536036617118684
1.17584827376684 0.535305485487585
1.16046866297006 0.534427447211652
1.1425262899005 0.533402122984755
1.12202139401749 0.532229072183377
1.10151726555175 0.531054604210161
1.08101390213675 0.529878734481104
1.06051130139776 0.528701478429746
1.04000946095186 0.5275228515073
1.01950837840803 0.526342869182777
0.999008051367172 0.525161546943109
0.978508477422155 0.523978900293278
0.958009654157879 0.522794944756435
0.937511579151311 0.521609695874025
0.917014249971538 0.520423169205916
0.896517664179812 0.519235380330514
0.876021819329601 0.518046344844887
0.855526712966634 0.516856078364893
0.835032342628955 0.515664596525293
0.814538705846972 0.514471914979876
0.794045800143499 0.513278049401579
0.773553623033816 0.512083015482608
0.753062172025712 0.510886828934553
0.732571444619537 0.509689505488512
0.712081438308253 0.508491060895207
0.691592150577485 0.507291510925102
0.671103578905572 0.506090871368521
0.650615720763616 0.504889158035764
0.630128573615536 0.503686386757223
0.609642134918118 0.502482573383502
0.589156402121068 0.501277733785526
0.568671372667066 0.500071883854662
0.548187043991813 0.49886503950283
0.527703413524089 0.497657216662618
0.507220478685805 0.496448431287396
0.486738236892053 0.495238699351429
0.466256685551165 0.494028036849988
0.445775822064762 0.492816459799465
0.425295643827809 0.491603984237482
0.404816148228671 0.490390626223003
0.384337332649168 0.489176401836445
0.363859194464625 0.487961327179788
0.343381731043935 0.486745418376685
0.322904939749608 0.48552869157257
0.302428817937827 0.484311162934768
0.281953362958509 0.483092848652601
0.261478572155355 0.4818737649375
0.243563709029771 0.480806407637357
0.228208544470985 0.479891046434884
0.215412881797103 0.479127906813159
0.20517655624396 0.4785171732374
0.197499434529847 0.478058991869279
0.192381414497009 0.477753472810477
0.189822424830656 0.477600691872751
0.189822424856044 0.477600691872319
0.18984329041129 0.477361217804826
0.189891281488213 0.476888179768742
0.18998422190601 0.476195304214649
0.190156086957441 0.475302243681175
0.19046107666499 0.47423395694596
0.190973511005364 0.47302370084793
0.1917804370421 0.471720939032194
0.192990859960637 0.470375213510415
0.194255394688102 0.469435488118
0.195465103825654 0.468815206001496
0.196534139830494 0.468445902524195
0.19741290176854 0.468255645457359
0.19807765083531 0.468178447005524
0.198520413044486 0.468160326901788
0.198740901208049 0.468162462817004
0.198740901269321 0.46816246281638
0.198945440782965 0.465612270025241
0.199354508471781 0.460511864695189
0.199968076306772 0.452861197525416
0.200786100039452 0.442660189857331
0.201808520127061 0.429908733995519
0.203035263052485 0.414606693655117
0.20446624304152 0.396753904535665
0.206101364180888 0.376350175021588
0.207736171974856 0.355945830121864
0.209370679915717 0.335540874152152
0.211004901527408 0.315135311429408
0.212638850365535 0.294729146271967
0.214272540017396 0.27432238299961
0.215905984101993 0.253915025933645
0.217539196270065 0.233507079396979
0.219172190204101 0.2130985477142
0.220804979618368 0.192689435211644
0.222437578258934 0.172279746217481
0.224069999903693 0.15186948506179
0.225702258362386 0.131458656076634
0.227334367476629 0.111047263596139
0.228966341119943 0.0906353119565765
0.230598193197773 0.0702228054964383
0.232229937647521 0.0498097485565179
0.233861588438572 0.0293961454799898
0.235493159572323 0.00898200061249
0.237124665082211 -0.0114326816978031
0.238756119033746 -0.0318478971000879
0.240387535524538 -0.0522636412408578
0.24201892868433 -0.0726799097638194
0.243650312675029 -0.0930966983098097
0.245281701690739 -0.113514002516714
0.246913109957795 -0.133931818019382
0.248544551734793 -0.154350140449545
0.250176041312631 -0.174768965435729
0.251807593014535 -0.195188288603176
0.253439221196103 -0.215608105573753
0.255070940245335 -0.236028411965869
0.256702764582677 -0.256449203394391
0.258334708661049 -0.276870475470555
0.259966786965892 -0.29729222380188
0.261599014015202 -0.317714443992083
0.263231404359571 -0.338137131640988
0.264863972582227 -0.358560282344441
0.266496733299076 -0.378983891694222
0.26812970115874 -0.399407955277952
0.269762890842605 -0.419832468679009
0.271396317064857 -0.440257427476434
0.27302999457253 -0.460682827244845
0.27466393814555 -0.481108663554342
0.276093884549649 -0.498981648418472
0.277319746923832 -0.514301630525306
0.278341443563315 -0.527068481124652
0.279158900790641 -0.537282093189924
0.279772055404735 -0.544942380708266
0.280180856703051 -0.550049278099581
0.280385268072754 -0.552602739765039
};
\addplot [semithick, green, dash pattern=on 1pt off 3pt on 3pt off 3pt, forget plot]
table {%
0.25 -0.5
0.2525 -0.5
0.2575 -0.5
0.265 -0.5
0.275 -0.5
0.2875 -0.5
0.3025 -0.5
0.32 -0.5
0.34 -0.5
0.36 -0.5
0.38 -0.5
0.4 -0.5
0.42 -0.5
0.44 -0.5
0.46 -0.5
0.48 -0.5
0.5 -0.5
0.52 -0.5
0.54 -0.5
0.56 -0.5
0.58 -0.5
0.6 -0.5
0.62 -0.5
0.64 -0.5
0.66 -0.5
0.68 -0.5
0.7 -0.5
0.72 -0.5
0.74 -0.5
0.76 -0.5
0.78 -0.5
0.8 -0.5
0.82 -0.5
0.84 -0.5
0.86 -0.5
0.88 -0.5
0.9 -0.5
0.92 -0.5
0.94 -0.5
0.96 -0.5
0.98 -0.5
1 -0.5
1.02 -0.5
1.04 -0.5
1.06 -0.5
1.08 -0.5
1.1 -0.5
1.12 -0.5
1.14 -0.5
1.16 -0.5
1.18 -0.5
1.1975 -0.5
1.2125 -0.5
1.225 -0.5
1.235 -0.5
1.2425 -0.5
1.2475 -0.5
1.25 -0.5
1.25 -0.5
1.25 -0.5
1.25 -0.5
1.25 -0.5
1.25 -0.5
1.25 -0.5
1.25 -0.5
1.25 -0.5
1.25 -0.5
1.25 -0.5
1.25 -0.5
1.25 -0.5
1.25 -0.5
1.25 -0.5
1.25 -0.5
1.25 -0.5
1.25 -0.5
1.25 -0.4975
1.25 -0.4925
1.25 -0.485
1.25 -0.475
1.25 -0.4625
1.25 -0.4475
1.25 -0.43
1.25 -0.41
1.25 -0.39
1.25 -0.37
1.25 -0.35
1.25 -0.33
1.25 -0.31
1.25 -0.29
1.25 -0.27
1.25 -0.25
1.25 -0.23
1.25 -0.21
1.25 -0.19
1.25 -0.17
1.25 -0.15
1.25 -0.13
1.25 -0.11
1.25 -0.09
1.25 -0.07
1.25 -0.05
1.25 -0.03
1.25 -0.01
1.25 0.00999999999999998
1.25 0.03
1.25 0.05
1.25 0.07
1.25 0.09
1.25 0.11
1.25 0.13
1.25 0.15
1.25 0.17
1.25 0.19
1.25 0.21
1.25 0.23
1.25 0.25
1.25 0.27
1.25 0.29
1.25 0.31
1.25 0.33
1.25 0.35
1.25 0.37
1.25 0.39
1.25 0.41
1.25 0.43
1.25 0.4475
1.25 0.4625
1.25 0.475
1.25 0.485
1.25 0.4925
1.25 0.4975
1.25 0.5
1.25 0.5
1.25 0.5
1.25 0.5
1.25 0.5
1.25 0.5
1.25 0.5
1.25 0.5
1.25 0.5
1.25 0.5
1.25 0.5
1.25 0.5
1.25 0.5
1.25 0.5
1.25 0.5
1.25 0.5
1.25 0.5
1.25 0.5
1.2475 0.5
1.2425 0.5
1.235 0.5
1.225 0.5
1.2125 0.5
1.1975 0.5
1.18 0.5
1.16 0.5
1.14 0.5
1.12 0.5
1.1 0.5
1.08 0.5
1.06 0.5
1.04 0.5
1.02 0.5
1 0.5
0.98 0.5
0.96 0.5
0.94 0.5
0.92 0.5
0.9 0.5
0.88 0.5
0.86 0.5
0.84 0.5
0.82 0.5
0.8 0.5
0.78 0.5
0.76 0.5
0.74 0.5
0.72 0.5
0.7 0.5
0.68 0.5
0.66 0.5
0.64 0.5
0.62 0.5
0.6 0.5
0.58 0.5
0.56 0.5
0.54 0.5
0.52 0.5
0.5 0.5
0.48 0.5
0.46 0.5
0.44 0.5
0.42 0.5
0.4 0.5
0.38 0.5
0.36 0.5
0.34 0.5
0.32 0.5
0.3025 0.5
0.2875 0.5
0.275 0.5
0.265 0.5
0.2575 0.5
0.2525 0.5
0.25 0.5
0.25 0.5
0.25 0.5
0.25 0.5
0.25 0.5
0.25 0.5
0.25 0.5
0.25 0.5
0.25 0.5
0.25 0.5
0.25 0.5
0.25 0.5
0.25 0.5
0.25 0.5
0.25 0.5
0.25 0.5
0.25 0.5
0.25 0.5
0.25 0.4975
0.25 0.4925
0.25 0.485
0.25 0.475
0.25 0.4625
0.25 0.4475
0.25 0.43
0.25 0.41
0.25 0.39
0.25 0.37
0.25 0.35
0.25 0.33
0.25 0.31
0.25 0.29
0.25 0.27
0.25 0.25
0.25 0.23
0.25 0.21
0.25 0.19
0.25 0.17
0.25 0.15
0.249999999999999 0.13
0.249999999999999 0.11
0.249999999999999 0.09
0.249999999999999 0.07
0.249999999999999 0.05
0.249999999999999 0.03
0.249999999999999 0.01
0.249999999999999 -0.00999999999999998
0.249999999999999 -0.03
0.249999999999999 -0.05
0.249999999999999 -0.07
0.249999999999999 -0.09
0.249999999999999 -0.11
0.249999999999999 -0.13
0.249999999999999 -0.15
0.249999999999999 -0.17
0.249999999999999 -0.19
0.249999999999999 -0.21
0.249999999999999 -0.23
0.249999999999999 -0.25
0.249999999999999 -0.27
0.249999999999999 -0.29
0.249999999999999 -0.31
0.249999999999999 -0.33
0.249999999999999 -0.35
0.249999999999999 -0.37
0.249999999999999 -0.39
0.249999999999999 -0.41
0.249999999999999 -0.43
0.249999999999999 -0.4475
0.249999999999999 -0.4625
0.249999999999999 -0.475
0.249999999999999 -0.485
0.249999999999999 -0.4925
0.249999999999999 -0.4975
0.249999999999999 -0.5
};
\addplot [semithick, blue, opacity=\opacityRef, forget plot]
table {%
0.25 -0.5
0.250131398439407 -0.500100314617157
0.254033744335175 -0.500012278556824
0.261586964130402 -0.49991899728775
0.271065503358841 -0.499614298343658
0.283550411462784 -0.499401390552521
0.298296928405762 -0.498960614204407
0.315689831972122 -0.498388886451721
0.335707694292068 -0.497886002063751
0.356870830059052 -0.497471779584885
0.376881927251816 -0.496915757656097
0.397470951080322 -0.49621707201004
0.418248355388641 -0.495763510465622
0.43885526061058 -0.495138645172119
0.459438771009445 -0.494537770748138
0.48010390996933 -0.494229257106781
0.500693023204803 -0.493669480085373
0.521329462528229 -0.492980062961578
0.54190468788147 -0.492443293333054
0.562441647052765 -0.491681307554245
0.582989931106567 -0.491385698318481
0.603334069252014 -0.490597248077393
0.623929381370544 -0.490413725376129
0.644616186618805 -0.489908784627914
0.665059208869934 -0.489443182945251
0.68588387966156 -0.489042997360229
0.706227898597717 -0.488578677177429
0.727100014686584 -0.488081634044647
0.747715532779694 -0.487512737512589
0.767892181873322 -0.486996531486511
0.788653910160065 -0.486427187919617
0.809097766876221 -0.485968589782715
0.829631090164185 -0.485367953777313
0.850112318992615 -0.484789669513702
0.870528221130371 -0.484327793121338
0.890982091426849 -0.483682155609131
0.912028014659882 -0.483229130506516
0.932370245456696 -0.482667863368988
0.952920794487 -0.482010900974274
0.9734246134758 -0.481351882219315
0.994009256362915 -0.48063337802887
1.01440703868866 -0.480235397815704
1.03506445884705 -0.479773461818695
1.05516970157623 -0.479019045829773
1.0758216381073 -0.478536754846573
1.09600651264191 -0.477975398302078
1.11679255962372 -0.477402478456497
1.13781630992889 -0.477002054452896
1.15801334381104 -0.47636666893959
1.17879045009613 -0.476102590560913
1.19946384429932 -0.475269436836243
1.21859681606293 -0.474933117628098
1.23414170742035 -0.474858641624451
1.2475358247757 -0.474482148885727
1.25869596004486 -0.474250972270966
1.26646542549133 -0.47396919131279
1.27213060855865 -0.47381916642189
1.27503800392151 -0.473835498094559
1.27588522434235 -0.47361958026886
1.27643311023712 -0.473747789859772
1.27722442150116 -0.473539263010025
1.27734625339508 -0.473421692848206
1.27725839614868 -0.473366320133209
1.27737557888031 -0.473094284534454
1.2771635055542 -0.473150730133057
1.2768965959549 -0.472784757614136
1.27680516242981 -0.472703486680984
1.27667915821075 -0.472662746906281
1.27644073963165 -0.472639709711075
1.27623343467712 -0.472660005092621
1.2761447429657 -0.47287192940712
1.27604675292969 -0.47284808754921
1.27604997158051 -0.473096430301666
1.27600967884064 -0.473028421401978
1.27606117725372 -0.47275123000145
1.27603697776794 -0.472529888153076
1.27593898773193 -0.468957275152206
1.2756005525589 -0.461708754301071
1.27523672580719 -0.451830118894577
1.27466058731079 -0.439323157072067
1.27402663230896 -0.424603939056396
1.27324020862579 -0.407430708408356
1.2724152803421 -0.387473165988922
1.27151429653168 -0.366443276405334
1.27074778079987 -0.346664100885391
1.26991677284241 -0.325317233800888
1.2690681219101 -0.304880917072296
1.26819753646851 -0.28414922952652
1.26730704307556 -0.263959467411041
1.26644885540009 -0.242995366454124
1.26565134525299 -0.222408890724182
1.26496684551239 -0.202129527926445
1.26426136493683 -0.181461200118065
1.26340055465698 -0.161247879266739
1.26261532306671 -0.140508770942688
1.26194047927856 -0.120178319513798
1.26107490062714 -0.0996773093938828
1.26023852825165 -0.07907634973526
1.25914573669434 -0.0586293302476406
1.25825297832489 -0.0377599261701107
1.25728476047516 -0.0171806886792183
1.2565530538559 0.00323964806739241
1.25562906265259 0.0237636920064688
1.25464868545532 0.0442411750555038
1.25361239910126 0.0649921074509621
1.25262343883514 0.0854188948869705
1.25192475318909 0.106069281697273
1.25090229511261 0.126551389694214
1.24993503093719 0.147195801138878
1.248743891716 0.167689561843872
1.24782860279083 0.188230499625206
1.24687874317169 0.208892896771431
1.24579513072968 0.229318261146545
1.2449232339859 0.249570652842522
1.24398589134216 0.270420879125595
1.2430317401886 0.290624678134918
1.24198496341705 0.310946315526962
1.24088001251221 0.331510007381439
1.23976039886475 0.3524569272995
1.23920571804047 0.372817814350128
1.23818123340607 0.393390715122223
1.23720359802246 0.41374197602272
1.23614752292633 0.43439319729805
1.23462641239166 0.45479080080986
1.23425936698914 0.475286394357681
1.23355555534363 0.494142651557922
1.23282599449158 0.511012375354767
1.23168444633484 0.528049409389496
1.23115348815918 0.537279486656189
1.23099029064178 0.544628024101257
1.23037481307983 0.550857245922089
1.23063457012177 0.555217742919922
1.23044729232788 0.559043407440186
1.23039448261261 0.558800101280212
1.23044300079346 0.558394551277161
1.23070001602173 0.558462083339691
1.23049592971802 0.558438241481781
1.22970485687256 0.558251082897186
1.22988414764404 0.558266818523407
1.22981691360474 0.558104515075684
1.22966504096985 0.557993948459625
1.23024547100067 0.557570815086365
1.22974479198456 0.557683765888214
1.22955811023712 0.557571947574615
1.22970604896545 0.557297348976135
1.2294328212738 0.557221472263336
1.22933804988861 0.557153165340424
1.2301242351532 0.557080686092377
1.2302463054657 0.55710357427597
1.22946262359619 0.557071447372437
1.22583150863647 0.556811451911926
1.21869492530823 0.556385040283203
1.20899498462677 0.555826246738434
1.19715619087219 0.555038273334503
1.18192231655121 0.554127633571625
1.1645427942276 0.553043067455292
1.14443099498749 0.551722288131714
1.12352585792542 0.550505101680756
1.10333073139191 0.549231767654419
1.08256506919861 0.548090040683746
1.06192326545715 0.546924829483032
1.04138219356537 0.545632719993591
1.02089512348175 0.544247031211853
1.00091183185577 0.543099522590637
0.980520367622375 0.541785776615143
0.960005939006805 0.540525376796722
0.93957245349884 0.539120197296143
0.918813288211823 0.537816107273102
0.89896696805954 0.536615133285522
0.878457725048065 0.535245656967163
0.857747495174408 0.533986568450928
0.83759468793869 0.532731413841248
0.817436873912811 0.53144633769989
0.79709804058075 0.530122816562653
0.776336908340454 0.528840780258179
0.755742907524109 0.527654349803925
0.735178470611572 0.526443302631378
0.71456116437912 0.525223314762115
0.69382232427597 0.52396821975708
0.673705220222473 0.522695362567902
0.652902901172638 0.521453320980072
0.632296085357666 0.520219385623932
0.612104177474976 0.519111454486847
0.591688930988312 0.517774879932404
0.5713010430336 0.516472578048706
0.550489723682404 0.515326321125031
0.530151605606079 0.514173328876495
0.509720981121063 0.512782752513885
0.489196538925171 0.511630594730377
0.468432247638702 0.510288298130035
0.44776263833046 0.509017586708069
0.426895886659622 0.507396161556244
0.406715005636215 0.506404161453247
0.386120766401291 0.505240678787231
0.365713864564896 0.504008710384369
0.34523668885231 0.502626955509186
0.323813647031784 0.50135925412178
0.304041385650635 0.500293582677841
0.283029586076736 0.498633503913879
0.263471364974976 0.49776229262352
0.247836694121361 0.496551990509033
0.234380185604095 0.49579194188118
0.223593920469284 0.495226442813873
0.215191975235939 0.494510889053345
0.209633246064186 0.494031488895416
0.206341758370399 0.493927925825119
0.205514773726463 0.493834555149078
0.205321833491325 0.493754833936691
0.204247027635574 0.493693947792053
0.204700633883476 0.493685573339462
0.204459652304649 0.493596464395523
0.204637110233307 0.493535339832306
0.204812377691269 0.493459671735764
0.205303281545639 0.493620425462723
0.205664619803429 0.493491351604462
0.205554127693176 0.493493258953094
0.205424770712852 0.493109375238419
0.205588564276695 0.493458598852158
0.20576810836792 0.493719011545181
0.205842405557632 0.493517637252808
0.205891817808151 0.493723422288895
0.205949008464813 0.493622660636902
0.205965086817741 0.493573695421219
0.205981463193893 0.493188291788101
0.20614717900753 0.48942905664444
0.206731379032135 0.482025861740112
0.207364082336426 0.472240775823593
0.20823110640049 0.46012744307518
0.20917309820652 0.445233047008514
0.210028856992722 0.428007334470749
0.211675301194191 0.408537060022354
0.213016405701637 0.38762754201889
0.214626997709274 0.367590665817261
0.21592666208744 0.346726804971695
0.217355281114578 0.32587605714798
0.218798652291298 0.305938601493835
0.22015792131424 0.285179287195206
0.221831381320953 0.264693051576614
0.223323464393616 0.243756651878357
0.224841818213463 0.222750529646873
0.226290881633759 0.202620476484299
0.227806448936462 0.182048290967941
0.229259923100471 0.161976203322411
0.230821147561073 0.14043191075325
0.231940984725952 0.120352022349834
0.233012333512306 0.100823916494846
0.234554514288902 0.0800052583217621
0.236188679933548 0.058783270418644
0.237580612301826 0.0379825420677662
0.238940477371216 0.0173061117529869
0.240468889474869 -0.00252957595512271
0.241447687149048 -0.0229066610336304
0.243264406919479 -0.0432933419942856
0.244375169277191 -0.0636487603187561
0.245962858200073 -0.084276370704174
0.247564285993576 -0.105650365352631
0.248916268348694 -0.126169249415398
0.250116676092148 -0.146369844675064
0.252102762460709 -0.166987732052803
0.253404527902603 -0.187628969550133
0.254943281412125 -0.207913294434547
0.256359338760376 -0.228826552629471
0.257762134075165 -0.248847782611847
0.259436011314392 -0.269835025072098
0.260765433311462 -0.289695650339127
0.262547940015793 -0.310449212789536
0.263696759939194 -0.331332266330719
0.265192240476608 -0.351478755474091
0.266608595848083 -0.372218817472458
0.268103063106537 -0.39284759759903
0.269478797912598 -0.413242012262344
0.271078318357468 -0.43437322974205
0.272198230028152 -0.454129070043564
0.273499846458435 -0.473375618457794
0.274472892284393 -0.488934874534607
0.275755047798157 -0.502459406852722
0.276317656040192 -0.513142108917236
0.27699202299118 -0.521204829216003
0.277134209871292 -0.526971280574799
0.277548223733902 -0.530503451824188
0.277577430009842 -0.530832588672638
};
\addplot [semithick, red, dashed, forget plot]
table {%
0.25 -0.5
0.252558268640609 -0.499971215983182
0.257674800204547 -0.499913561943335
0.265349580546156 -0.499826822527541
0.275582586992214 -0.499710651911554
0.288373788282223 -0.499564571828345
0.303723144483551 -0.499387968751125
0.321630606880243 -0.499180090231924
0.342096117833915 -0.498940040397106
0.362561444756502 -0.498697137467738
0.383026586783779 -0.498451352589885
0.403491543037366 -0.498202656922469
0.423956312624558 -0.49795102163749
0.444420894638161 -0.497696417920264
0.464885288156327 -0.497438816969653
0.485349492242387 -0.497178189998295
0.505813505944689 -0.496914508232844
0.526277328296427 -0.4966477429142
0.546740958315482 -0.49637786529775
0.567204395004253 -0.496104846653605
0.587667637349495 -0.495828658266839
0.608130684322154 -0.495549271437727
0.628593534877201 -0.495266657481993
0.649056187953474 -0.494980787731044
0.669518642473505 -0.494691633532224
0.689980897343365 -0.49439916624905
0.710442951452497 -0.494103357261464
0.730904803673553 -0.493804177966079
0.751366452862231 -0.493501599776428
0.771827897857116 -0.493195594123215
0.792289137479511 -0.492886132454562
0.812750170533283 -0.492573186236266
0.833210995804693 -0.492256726952051
0.853671612062244 -0.491936726103822
0.874132018056508 -0.491613155211922
0.894592212519978 -0.491285985815386
0.915052194166895 -0.490955189472205
0.935511961693098 -0.49062073775958
0.955971513775856 -0.490282602274183
0.976430849073713 -0.489940754632423
0.996889966226327 -0.489595166470704
1.01734886385431 -0.489245809445691
1.03780754055907 -0.488892655234577
1.05826599492265 -0.488535675535346
1.07872422550757 -0.488174842067041
1.09918223085668 -0.487810126570036
1.11964000949299 -0.487441500806302
1.1400975599195 -0.48706893655968
1.16055488061908 -0.486692405636151
1.18101197005428 -0.486311879864113
1.20146882666719 -0.485927331094652
1.21936837110133 -0.485587306188302
1.23471068292331 -0.485293179242986
1.24749583154271 -0.485046149752488
1.25772387591864 -0.484847237098433
1.26539486430542 -0.484697275874471
1.27050883404033 -0.484596912039359
1.27306581137596 -0.484546599896135
1.2730658113589 -0.484546599895144
1.27305541155906 -0.484320911054072
1.2730238336127 -0.483870450269343
1.27294421772612 -0.483198810692961
1.27277401858219 -0.482315683800082
1.27245616754417 -0.481242223159833
1.27192174548361 -0.480017905211023
1.27109518948705 -0.478707989424014
1.26987791322201 -0.477382389353572
1.26861868448422 -0.476491262912181
1.26741310591075 -0.475930505649088
1.26634153930566 -0.475618318765773
1.26545420209816 -0.475475522024619
1.26477827798535 -0.475433459898834
1.26432554654506 -0.475438002031128
1.26409923112047 -0.475451102493
1.26409923107294 -0.475451102492838
1.26400510754809 -0.472893953643021
1.26381676720545 -0.467779675905825
1.26353397696839 -0.460108319156843
1.26315636493238 -0.449879963098326
1.26268342195054 -0.437094717113806
1.26211450389605 -0.421752720062896
1.26144883460045 -0.403854140014764
1.2606855094668 -0.383399173918339
1.25991927860775 -0.362944839063709
1.25915016511684 -0.342491133403313
1.25837819207707 -0.322038054872471
1.25760338256082 -0.301585601389565
1.25682575962964 -0.281133770856217
1.25604534633415 -0.260682561157473
1.25526216571386 -0.240231970161977
1.25447624079705 -0.219781995722154
1.25368759460061 -0.199332635674387
1.25289625012991 -0.178883887839198
1.25210223037863 -0.158435750021426
1.25130555832866 -0.137988220010405
1.25050625694995 -0.117541295580143
1.24970434920032 -0.0970949744895014
1.2488998580254 -0.0766492544823687
1.24809280635845 -0.0562041332878428
1.24728321712021 -0.0357596086204065
1.24647111321881 -0.0153156781801049
1.24565651754959 0.00512766034727754
1.244839452995 0.0255704092900399
1.24401994242446 0.0460125709903884
1.24319800869422 0.0664541478042603
1.24237367464724 0.0868951421011472
1.24154696311305 0.10733555626392
1.24071789690763 0.127775392688651
1.23988649883331 0.148214653784443
1.23905279167858 0.168653341973249
1.23821679821801 0.189091459689702
1.23737854121215 0.209529009380936
1.23653804340733 0.229965993506414
1.23569532753562 0.250402414537754
1.23485041631467 0.270838274958555
1.23400333244757 0.291273577264223
1.23315409862277 0.311708323961797
1.23230273751397 0.332142517569777
1.23144927177993 0.352576160617952
1.23059372406446 0.373009255647224
1.22973611699622 0.393441805209441
1.22887647318863 0.413873811867219
1.22801481523979 0.434305278193775
1.22715116573232 0.454736206772755
1.22628554723329 0.475166600198059
1.22552642790666 0.493042728464227
1.22487449160254 0.508364776283352
1.22433031870982 0.52113290203806
1.22389439056138 0.531347238274115
1.22356709317291 0.539007892120091
1.22334872031756 0.544114945632642
1.22323947593845 0.546668456066198
1.22323947590048 0.546668456066214
1.2230123362303 0.54664910964162
1.22255893634353 0.546599663012994
1.22188286957468 0.546493423413052
1.22099418533407 0.546288449972454
1.21991546682544 0.545929219980312
1.21868995460871 0.545349990456186
1.21739061938076 0.544480715471633
1.21610030698399 0.543230273381219
1.21526412455335 0.541961848316668
1.21476024252823 0.540762508740427
1.21448767411105 0.539705040879324
1.21435620536168 0.538833702471361
1.21429925392167 0.538171836563276
1.21427642250785 0.537729171531928
1.21429899313131 0.537497396084245
1.21429899310163 0.53749739608416
1.21173554316399 0.537351398434274
1.2066086675668 0.537059357198718
1.19891842689941 0.536621157703719
1.18866491799536 0.536036617118684
1.17584827376684 0.535305485487585
1.16046866297006 0.534427447211652
1.1425262899005 0.533402122984755
1.12202139401749 0.532229072183377
1.10151726555175 0.531054604210161
1.08101390213675 0.529878734481104
1.06051130139776 0.528701478429746
1.04000946095186 0.5275228515073
1.01950837840803 0.526342869182777
0.999008051367172 0.525161546943109
0.978508477422155 0.523978900293278
0.958009654157879 0.522794944756435
0.937511579151311 0.521609695874025
0.917014249971538 0.520423169205916
0.896517664179812 0.519235380330514
0.876021819329601 0.518046344844887
0.855526712966634 0.516856078364893
0.835032342628955 0.515664596525293
0.814538705846972 0.514471914979876
0.794045800143499 0.513278049401579
0.773553623033816 0.512083015482608
0.753062172025712 0.510886828934553
0.732571444619537 0.509689505488512
0.712081438308253 0.508491060895207
0.691592150577485 0.507291510925102
0.671103578905572 0.506090871368521
0.650615720763616 0.504889158035764
0.630128573615536 0.503686386757223
0.609642134918118 0.502482573383502
0.589156402121068 0.501277733785526
0.568671372667066 0.500071883854662
0.548187043991813 0.49886503950283
0.527703413524089 0.497657216662618
0.507220478685805 0.496448431287396
0.486738236892053 0.495238699351429
0.466256685551165 0.494028036849988
0.445775822064762 0.492816459799465
0.425295643827809 0.491603984237482
0.404816148228671 0.490390626223003
0.384337332649168 0.489176401836445
0.363859194464625 0.487961327179788
0.343381731043935 0.486745418376685
0.322904939749608 0.48552869157257
0.302428817937827 0.484311162934768
0.281953362958509 0.483092848652601
0.261478572155355 0.4818737649375
0.243563709029771 0.480806407637357
0.228208544470985 0.479891046434884
0.215412881797103 0.479127906813159
0.20517655624396 0.4785171732374
0.197499434529847 0.478058991869279
0.192381414497009 0.477753472810477
0.189822424830656 0.477600691872751
0.189822424856044 0.477600691872319
0.18984329041129 0.477361217804826
0.189891281488213 0.476888179768742
0.18998422190601 0.476195304214649
0.190156086957441 0.475302243681175
0.19046107666499 0.47423395694596
0.190973511005364 0.47302370084793
0.1917804370421 0.471720939032194
0.192990859960637 0.470375213510415
0.194255394688102 0.469435488118
0.195465103825654 0.468815206001496
0.196534139830494 0.468445902524195
0.19741290176854 0.468255645457359
0.19807765083531 0.468178447005524
0.198520413044486 0.468160326901788
0.198740901208049 0.468162462817004
0.198740901269321 0.46816246281638
0.198945440782965 0.465612270025241
0.199354508471781 0.460511864695189
0.199968076306772 0.452861197525416
0.200786100039452 0.442660189857331
0.201808520127061 0.429908733995519
0.203035263052485 0.414606693655117
0.20446624304152 0.396753904535665
0.206101364180888 0.376350175021588
0.207736171974856 0.355945830121864
0.209370679915717 0.335540874152152
0.211004901527408 0.315135311429408
0.212638850365535 0.294729146271967
0.214272540017396 0.27432238299961
0.215905984101993 0.253915025933645
0.217539196270065 0.233507079396979
0.219172190204101 0.2130985477142
0.220804979618368 0.192689435211644
0.222437578258934 0.172279746217481
0.224069999903693 0.15186948506179
0.225702258362386 0.131458656076634
0.227334367476629 0.111047263596139
0.228966341119943 0.0906353119565765
0.230598193197773 0.0702228054964383
0.232229937647521 0.0498097485565179
0.233861588438572 0.0293961454799898
0.235493159572323 0.00898200061249
0.237124665082211 -0.0114326816978031
0.238756119033746 -0.0318478971000879
0.240387535524538 -0.0522636412408578
0.24201892868433 -0.0726799097638194
0.243650312675029 -0.0930966983098097
0.245281701690739 -0.113514002516714
0.246913109957795 -0.133931818019382
0.248544551734793 -0.154350140449545
0.250176041312631 -0.174768965435729
0.251807593014535 -0.195188288603176
0.253439221196103 -0.215608105573753
0.255070940245335 -0.236028411965869
0.256702764582677 -0.256449203394391
0.258334708661049 -0.276870475470555
0.259966786965892 -0.29729222380188
0.261599014015202 -0.317714443992083
0.263231404359571 -0.338137131640988
0.264863972582227 -0.358560282344441
0.266496733299076 -0.378983891694222
0.26812970115874 -0.399407955277952
0.269762890842605 -0.419832468679009
0.271396317064857 -0.440257427476434
0.27302999457253 -0.460682827244845
0.27466393814555 -0.481108663554342
0.276093884549649 -0.498981648418472
0.277319746923832 -0.514301630525306
0.278341443563315 -0.527068481124652
0.279158900790641 -0.537282093189924
0.279772055404735 -0.544942380708266
0.280180856703051 -0.550049278099581
0.280385268072754 -0.552602739765039
};
\addplot [semithick, green, dash pattern=on 1pt off 3pt on 3pt off 3pt, forget plot]
table {%
0.25 -0.5
0.2525 -0.5
0.2575 -0.5
0.265 -0.5
0.275 -0.5
0.2875 -0.5
0.3025 -0.5
0.32 -0.5
0.34 -0.5
0.36 -0.5
0.38 -0.5
0.4 -0.5
0.42 -0.5
0.44 -0.5
0.46 -0.5
0.48 -0.5
0.5 -0.5
0.52 -0.5
0.54 -0.5
0.56 -0.5
0.58 -0.5
0.6 -0.5
0.62 -0.5
0.64 -0.5
0.66 -0.5
0.68 -0.5
0.7 -0.5
0.72 -0.5
0.74 -0.5
0.76 -0.5
0.78 -0.5
0.8 -0.5
0.82 -0.5
0.84 -0.5
0.86 -0.5
0.88 -0.5
0.9 -0.5
0.92 -0.5
0.94 -0.5
0.96 -0.5
0.98 -0.5
1 -0.5
1.02 -0.5
1.04 -0.5
1.06 -0.5
1.08 -0.5
1.1 -0.5
1.12 -0.5
1.14 -0.5
1.16 -0.5
1.18 -0.5
1.1975 -0.5
1.2125 -0.5
1.225 -0.5
1.235 -0.5
1.2425 -0.5
1.2475 -0.5
1.25 -0.5
1.25 -0.5
1.25 -0.5
1.25 -0.5
1.25 -0.5
1.25 -0.5
1.25 -0.5
1.25 -0.5
1.25 -0.5
1.25 -0.5
1.25 -0.5
1.25 -0.5
1.25 -0.5
1.25 -0.5
1.25 -0.5
1.25 -0.5
1.25 -0.5
1.25 -0.5
1.25 -0.4975
1.25 -0.4925
1.25 -0.485
1.25 -0.475
1.25 -0.4625
1.25 -0.4475
1.25 -0.43
1.25 -0.41
1.25 -0.39
1.25 -0.37
1.25 -0.35
1.25 -0.33
1.25 -0.31
1.25 -0.29
1.25 -0.27
1.25 -0.25
1.25 -0.23
1.25 -0.21
1.25 -0.19
1.25 -0.17
1.25 -0.15
1.25 -0.13
1.25 -0.11
1.25 -0.09
1.25 -0.07
1.25 -0.05
1.25 -0.03
1.25 -0.01
1.25 0.00999999999999998
1.25 0.03
1.25 0.05
1.25 0.07
1.25 0.09
1.25 0.11
1.25 0.13
1.25 0.15
1.25 0.17
1.25 0.19
1.25 0.21
1.25 0.23
1.25 0.25
1.25 0.27
1.25 0.29
1.25 0.31
1.25 0.33
1.25 0.35
1.25 0.37
1.25 0.39
1.25 0.41
1.25 0.43
1.25 0.4475
1.25 0.4625
1.25 0.475
1.25 0.485
1.25 0.4925
1.25 0.4975
1.25 0.5
1.25 0.5
1.25 0.5
1.25 0.5
1.25 0.5
1.25 0.5
1.25 0.5
1.25 0.5
1.25 0.5
1.25 0.5
1.25 0.5
1.25 0.5
1.25 0.5
1.25 0.5
1.25 0.5
1.25 0.5
1.25 0.5
1.25 0.5
1.2475 0.5
1.2425 0.5
1.235 0.5
1.225 0.5
1.2125 0.5
1.1975 0.5
1.18 0.5
1.16 0.5
1.14 0.5
1.12 0.5
1.1 0.5
1.08 0.5
1.06 0.5
1.04 0.5
1.02 0.5
1 0.5
0.98 0.5
0.96 0.5
0.94 0.5
0.92 0.5
0.9 0.5
0.88 0.5
0.86 0.5
0.84 0.5
0.82 0.5
0.8 0.5
0.78 0.5
0.76 0.5
0.74 0.5
0.72 0.5
0.7 0.5
0.68 0.5
0.66 0.5
0.64 0.5
0.62 0.5
0.6 0.5
0.58 0.5
0.56 0.5
0.54 0.5
0.52 0.5
0.5 0.5
0.48 0.5
0.46 0.5
0.44 0.5
0.42 0.5
0.4 0.5
0.38 0.5
0.36 0.5
0.34 0.5
0.32 0.5
0.3025 0.5
0.2875 0.5
0.275 0.5
0.265 0.5
0.2575 0.5
0.2525 0.5
0.25 0.5
0.25 0.5
0.25 0.5
0.25 0.5
0.25 0.5
0.25 0.5
0.25 0.5
0.25 0.5
0.25 0.5
0.25 0.5
0.25 0.5
0.25 0.5
0.25 0.5
0.25 0.5
0.25 0.5
0.25 0.5
0.25 0.5
0.25 0.5
0.25 0.4975
0.25 0.4925
0.25 0.485
0.25 0.475
0.25 0.4625
0.25 0.4475
0.25 0.43
0.25 0.41
0.25 0.39
0.25 0.37
0.25 0.35
0.25 0.33
0.25 0.31
0.25 0.29
0.25 0.27
0.25 0.25
0.25 0.23
0.25 0.21
0.25 0.19
0.25 0.17
0.25 0.15
0.249999999999999 0.13
0.249999999999999 0.11
0.249999999999999 0.09
0.249999999999999 0.07
0.249999999999999 0.05
0.249999999999999 0.03
0.249999999999999 0.01
0.249999999999999 -0.00999999999999998
0.249999999999999 -0.03
0.249999999999999 -0.05
0.249999999999999 -0.07
0.249999999999999 -0.09
0.249999999999999 -0.11
0.249999999999999 -0.13
0.249999999999999 -0.15
0.249999999999999 -0.17
0.249999999999999 -0.19
0.249999999999999 -0.21
0.249999999999999 -0.23
0.249999999999999 -0.25
0.249999999999999 -0.27
0.249999999999999 -0.29
0.249999999999999 -0.31
0.249999999999999 -0.33
0.249999999999999 -0.35
0.249999999999999 -0.37
0.249999999999999 -0.39
0.249999999999999 -0.41
0.249999999999999 -0.43
0.249999999999999 -0.4475
0.249999999999999 -0.4625
0.249999999999999 -0.475
0.249999999999999 -0.485
0.249999999999999 -0.4925
0.249999999999999 -0.4975
0.249999999999999 -0.5
};
\addplot [semithick, blue, opacity=\opacityRef, forget plot]
table {%
0.25 -0.5
0.250292241573334 -0.500043928623199
0.253514021635056 -0.499782562255859
0.261505663394928 -0.499890506267548
0.270901262760162 -0.499894201755524
0.281957775354385 -0.49994033575058
0.298155069351196 -0.499843955039978
0.31569242477417 -0.499866127967834
0.335551887750626 -0.499709069728851
0.354797661304474 -0.499486207962036
0.377147674560547 -0.499717831611633
0.398112565279007 -0.499444186687469
0.416756063699722 -0.499407231807709
0.437171846628189 -0.499383568763733
0.459804564714432 -0.499194622039795
0.478387832641602 -0.499046862125397
0.498891919851303 -0.498785018920898
0.521146386861801 -0.498627901077271
0.541790813207626 -0.498498916625977
0.562064379453659 -0.498436331748962
0.583291321992874 -0.49822074174881
0.601738303899765 -0.498485743999481
0.622179180383682 -0.498523771762848
0.644780188798904 -0.498334527015686
0.663350373506546 -0.498209893703461
0.684185951948166 -0.498193144798279
0.706849962472916 -0.498116731643677
0.725100785493851 -0.498011350631714
0.745426148176193 -0.498004764318466
0.768139392137527 -0.49784654378891
0.788637787103653 -0.497784674167633
0.807321697473526 -0.497831702232361
0.829711645841599 -0.497888505458832
0.850516885519028 -0.497911870479584
0.869023770093918 -0.497717052698135
0.889875560998917 -0.497638911008835
0.912213414907455 -0.497595012187958
0.930836111307144 -0.497276157140732
0.953084677457809 -0.497198611497879
0.973781377077103 -0.497181713581085
0.994150370359421 -0.496965229511261
1.01416835188866 -0.496706485748291
1.03540655970573 -0.49687260389328
1.05342862010002 -0.496806979179382
1.07655677199364 -0.496546775102615
1.09715804457664 -0.49650964140892
1.11766621470451 -0.496633887290955
1.13596197962761 -0.49666428565979
1.15869423747063 -0.496625602245331
1.17924353480339 -0.496620714664459
1.19959530234337 -0.49668824672699
1.21710225939751 -0.496796578168869
1.23388120532036 -0.496820986270905
1.24621316790581 -0.496818602085114
1.25788649916649 -0.496936619281769
1.26642498373985 -0.496824860572815
1.27123579382896 -0.496898353099823
1.2742917239666 -0.496814399957657
1.27514454722404 -0.496799856424332
1.27536091208458 -0.496719479560852
1.27653869986534 -0.496761739253998
1.27632281184196 -0.496725499629974
1.27636632323265 -0.496665894985199
1.27631077170372 -0.496389538049698
1.27653726935387 -0.496053636074066
1.27641841769218 -0.496034741401672
1.27612480521202 -0.495754361152649
1.27577754855156 -0.49618124961853
1.27575573325157 -0.495869278907776
1.2756595313549 -0.495846182107925
1.27549836039543 -0.495873361825943
1.2753594815731 -0.496150881052017
1.2752580344677 -0.496403187513351
1.27522978186607 -0.496242582798004
1.27527090907097 -0.496171534061432
1.27520725131035 -0.495639562606812
1.27520868182182 -0.491389334201813
1.27502879500389 -0.483797550201416
1.2750196158886 -0.474211126565933
1.27492544054985 -0.463193237781525
1.2747103869915 -0.446944445371628
1.27446088194847 -0.429481893777847
1.27405020594597 -0.411260396242142
1.27394267916679 -0.390201091766357
1.27366778254509 -0.369845420122147
1.27332648634911 -0.349204361438751
1.27328154444695 -0.326710134744644
1.27299496531487 -0.305939704179764
1.27290281653404 -0.285495162010193
1.27278432250023 -0.266958385705948
1.27259227633476 -0.244469434022903
1.27246078848839 -0.223828196525574
1.27232381701469 -0.20516411960125
1.27196642756462 -0.183037206530571
1.271862834692 -0.164258733391762
1.27156302332878 -0.141674801707268
1.27147969603539 -0.12123180180788
1.27137061953545 -0.102740429341793
1.2712182700634 -0.0801259502768517
1.27100560069084 -0.0598755516111851
1.27073773741722 -0.0415524765849113
1.27058836817741 -0.0183177702128887
1.2702359855175 -0.000463132979348302
1.27020546793938 0.0207759737968445
1.26965722441673 0.0411987900733948
1.26936885714531 0.0637635290622711
1.26896771788597 0.0824567750096321
1.26886603236198 0.103114269673824
1.26853713393211 0.125532098114491
1.26815256476402 0.143901899456978
1.26775595545769 0.166479706764221
1.2675047814846 0.185219749808311
1.26721522212029 0.205669477581978
1.2668579518795 0.228102385997772
1.26662883162498 0.246954292058945
1.26600202918053 0.269410133361816
1.2656190097332 0.289931535720825
1.26491555571556 0.307851940393448
1.26471969485283 0.330726325511932
1.26487597823143 0.35152992606163
1.26453122496605 0.372218638658524
1.26418229937553 0.392752915620804
1.26402518153191 0.411401301622391
1.26381024718285 0.433685749769211
1.26326915621758 0.452017605304718
1.26322481036186 0.471258610486984
1.26290509104729 0.488799840211868
1.26239225268364 0.501817345619202
1.26224526762962 0.512609839439392
1.26216623187065 0.52045875787735
1.26208445429802 0.526326894760132
1.26209565997124 0.529417634010315
1.26204892992973 0.529904305934906
1.26198348402977 0.530846536159515
1.26225516200066 0.531270980834961
1.26175805926323 0.531471610069275
1.26127335429192 0.531428337097168
1.26117572188377 0.531330823898315
1.26176390051842 0.531042337417603
1.26158210635185 0.530764281749725
1.26163980364799 0.530689716339111
1.26156160235405 0.530771136283875
1.26148101687431 0.530873000621796
1.26093718409538 0.530786216259003
1.26119622588158 0.530563354492188
1.26139411330223 0.530549824237823
1.26056632399559 0.530534744262695
1.26093766093254 0.530482470989227
1.26090189814568 0.530454814434052
1.26071056723595 0.530423521995544
1.25634321570396 0.530301213264465
1.24949631094933 0.5303053855896
1.24065634608269 0.530012369155884
1.22673568129539 0.529694080352783
1.21282431483269 0.52940034866333
1.19404026865959 0.528931260108948
1.17456743121147 0.528400540351868
1.15341970324516 0.527868628501892
1.13331684470177 0.527385532855988
1.11498674750328 0.526851117610931
1.091642588377 0.526327967643738
1.07143160700798 0.525888502597809
1.05113872885704 0.525282025337219
1.03038772940636 0.524727344512939
1.00964543223381 0.524028241634369
0.991125553846359 0.523593246936798
0.971251517534256 0.523155093193054
0.948164373636246 0.522622108459473
0.928324311971664 0.52192211151123
0.909521490335464 0.521350085735321
0.887482017278671 0.52082371711731
0.866983681917191 0.520204424858093
0.846250027418137 0.519600391387939
0.825492471456528 0.518994867801666
0.805173844099045 0.5185387134552
0.784901231527328 0.517765164375305
0.766423314809799 0.517407536506653
0.746000081300735 0.51708447933197
0.723179489374161 0.516525208950043
0.704795986413956 0.516001880168915
0.682332307100296 0.515433490276337
0.663731008768082 0.515017092227936
0.641281634569168 0.514619708061218
0.620337516069412 0.513912379741669
0.599706023931503 0.513415217399597
0.579334884881973 0.513154625892639
0.560649245977402 0.512536942958832
0.538051158189774 0.512046694755554
0.519422084093094 0.511755347251892
0.497188955545425 0.511211454868317
0.476724833250046 0.510719835758209
0.456185698509216 0.510140359401703
0.435387283563614 0.509476482868195
0.416756451129913 0.509032070636749
0.396149188280106 0.508618950843811
0.37355250120163 0.5079705119133
0.354713708162308 0.507560849189758
0.331608206033707 0.506673812866211
0.310765892267227 0.506359398365021
0.293963491916656 0.505960941314697
0.276670932769775 0.505689144134521
0.264586627483368 0.505184769630432
0.252947866916656 0.50480842590332
0.244580954313278 0.504572868347168
0.239383369684219 0.504593729972839
0.236221224069595 0.50426185131073
0.235734775662422 0.504380524158478
0.23494179546833 0.504469931125641
0.233839064836502 0.504157066345215
0.234099745750427 0.504312813282013
0.23402501642704 0.504195988178253
0.234075859189034 0.50416487455368
0.233538180589676 0.503707766532898
0.233913466334343 0.503780126571655
0.234154373407364 0.503561556339264
0.234644293785095 0.503957033157349
0.234570622444153 0.503876984119415
0.234542414546013 0.503770589828491
0.234798640012741 0.503896594047546
0.234795466065407 0.503828048706055
0.234916642308235 0.503890454769135
0.234936520457268 0.503880858421326
0.234977841377258 0.503911018371582
0.235016003251076 0.503419399261475
0.235086902976036 0.500708699226379
0.235344007611275 0.493243306875229
0.235646590590477 0.482659101486206
0.235820472240448 0.471936047077179
0.236074641346931 0.457124620676041
0.236223801970482 0.43971985578537
0.236916497349739 0.418154776096344
0.237321153283119 0.398995041847229
0.237831741571426 0.37863889336586
0.238476246595383 0.356437265872955
0.23913536965847 0.337854981422424
0.239538103342056 0.317153334617615
0.240321487188339 0.295035719871521
0.240931957960129 0.276310056447983
0.241678670048714 0.2537821829319
0.242258116602898 0.235324874520302
0.242999196052551 0.211222589015961
0.2434121966362 0.193286493420601
0.243947565555573 0.170987442135811
0.244435355067253 0.150164797902107
0.244923532009125 0.129812717437744
0.245678395032883 0.111053332686424
0.245989784598351 0.0882409811019897
0.246694073081017 0.0685292929410934
0.247471526265144 0.0496980100870132
0.248027071356773 0.0287666525691748
0.248308390378952 0.00694138463586569
0.248933464288712 -0.0143415667116642
0.249208509922028 -0.0345835722982883
0.250079482793808 -0.055282786488533
0.250765740871429 -0.0769993588328362
0.251022756099701 -0.0951161906123161
0.251529723405838 -0.116897732019424
0.251879662275314 -0.135463580489159
0.252776563167572 -0.158564150333405
0.253287255764008 -0.178640469908714
0.254013359546661 -0.199782282114029
0.254457026720047 -0.218239158391953
0.254784166812897 -0.23878338932991
0.25510311126709 -0.258965820074081
0.256020963191986 -0.280200898647308
0.256401538848877 -0.300518721342087
0.257308006286621 -0.323107242584229
0.257881343364716 -0.343518882989883
0.258243501186371 -0.362092345952988
0.258973985910416 -0.384718865156174
0.259627163410187 -0.40508434176445
0.260054022073746 -0.425373733043671
0.260625392198563 -0.444475293159485
0.26088285446167 -0.464022070169449
0.26117417216301 -0.480044096708298
0.261533856391907 -0.495161652565002
0.261809259653091 -0.50486147403717
0.262034088373184 -0.513637006282806
0.262223184108734 -0.519711852073669
0.262349605560303 -0.522591650485992
0.262432217597961 -0.523023962974548
};
\addplot [semithick, red, dashed, forget plot]
table {%
0.25 -0.5
0.252558268640609 -0.499971215983182
0.257674800204547 -0.499913561943335
0.265349580546156 -0.499826822527541
0.275582586992214 -0.499710651911554
0.288373788282223 -0.499564571828345
0.303723144483551 -0.499387968751125
0.321630606880243 -0.499180090231924
0.342096117833915 -0.498940040397106
0.362561444756502 -0.498697137467738
0.383026586783779 -0.498451352589885
0.403491543037366 -0.498202656922469
0.423956312624558 -0.49795102163749
0.444420894638161 -0.497696417920264
0.464885288156327 -0.497438816969653
0.485349492242387 -0.497178189998295
0.505813505944689 -0.496914508232844
0.526277328296427 -0.4966477429142
0.546740958315482 -0.49637786529775
0.567204395004253 -0.496104846653605
0.587667637349495 -0.495828658266839
0.608130684322154 -0.495549271437727
0.628593534877201 -0.495266657481993
0.649056187953474 -0.494980787731044
0.669518642473505 -0.494691633532224
0.689980897343365 -0.49439916624905
0.710442951452497 -0.494103357261464
0.730904803673553 -0.493804177966079
0.751366452862231 -0.493501599776428
0.771827897857116 -0.493195594123215
0.792289137479511 -0.492886132454562
0.812750170533283 -0.492573186236266
0.833210995804693 -0.492256726952051
0.853671612062244 -0.491936726103822
0.874132018056508 -0.491613155211922
0.894592212519978 -0.491285985815386
0.915052194166895 -0.490955189472205
0.935511961693098 -0.49062073775958
0.955971513775856 -0.490282602274183
0.976430849073713 -0.489940754632423
0.996889966226327 -0.489595166470704
1.01734886385431 -0.489245809445691
1.03780754055907 -0.488892655234577
1.05826599492265 -0.488535675535346
1.07872422550757 -0.488174842067041
1.09918223085668 -0.487810126570036
1.11964000949299 -0.487441500806302
1.1400975599195 -0.48706893655968
1.16055488061908 -0.486692405636151
1.18101197005428 -0.486311879864113
1.20146882666719 -0.485927331094652
1.21936837110133 -0.485587306188302
1.23471068292331 -0.485293179242986
1.24749583154271 -0.485046149752488
1.25772387591864 -0.484847237098433
1.26539486430542 -0.484697275874471
1.27050883404033 -0.484596912039359
1.27306581137596 -0.484546599896135
1.2730658113589 -0.484546599895144
1.27305541155906 -0.484320911054072
1.2730238336127 -0.483870450269343
1.27294421772612 -0.483198810692961
1.27277401858219 -0.482315683800082
1.27245616754417 -0.481242223159833
1.27192174548361 -0.480017905211023
1.27109518948705 -0.478707989424014
1.26987791322201 -0.477382389353572
1.26861868448422 -0.476491262912181
1.26741310591075 -0.475930505649088
1.26634153930566 -0.475618318765773
1.26545420209816 -0.475475522024619
1.26477827798535 -0.475433459898834
1.26432554654506 -0.475438002031128
1.26409923112047 -0.475451102493
1.26409923107294 -0.475451102492838
1.26400510754809 -0.472893953643021
1.26381676720545 -0.467779675905825
1.26353397696839 -0.460108319156843
1.26315636493238 -0.449879963098326
1.26268342195054 -0.437094717113806
1.26211450389605 -0.421752720062896
1.26144883460045 -0.403854140014764
1.2606855094668 -0.383399173918339
1.25991927860775 -0.362944839063709
1.25915016511684 -0.342491133403313
1.25837819207707 -0.322038054872471
1.25760338256082 -0.301585601389565
1.25682575962964 -0.281133770856217
1.25604534633415 -0.260682561157473
1.25526216571386 -0.240231970161977
1.25447624079705 -0.219781995722154
1.25368759460061 -0.199332635674387
1.25289625012991 -0.178883887839198
1.25210223037863 -0.158435750021426
1.25130555832866 -0.137988220010405
1.25050625694995 -0.117541295580143
1.24970434920032 -0.0970949744895014
1.2488998580254 -0.0766492544823687
1.24809280635845 -0.0562041332878428
1.24728321712021 -0.0357596086204065
1.24647111321881 -0.0153156781801049
1.24565651754959 0.00512766034727754
1.244839452995 0.0255704092900399
1.24401994242446 0.0460125709903884
1.24319800869422 0.0664541478042603
1.24237367464724 0.0868951421011472
1.24154696311305 0.10733555626392
1.24071789690763 0.127775392688651
1.23988649883331 0.148214653784443
1.23905279167858 0.168653341973249
1.23821679821801 0.189091459689702
1.23737854121215 0.209529009380936
1.23653804340733 0.229965993506414
1.23569532753562 0.250402414537754
1.23485041631467 0.270838274958555
1.23400333244757 0.291273577264223
1.23315409862277 0.311708323961797
1.23230273751397 0.332142517569777
1.23144927177993 0.352576160617952
1.23059372406446 0.373009255647224
1.22973611699622 0.393441805209441
1.22887647318863 0.413873811867219
1.22801481523979 0.434305278193775
1.22715116573232 0.454736206772755
1.22628554723329 0.475166600198059
1.22552642790666 0.493042728464227
1.22487449160254 0.508364776283352
1.22433031870982 0.52113290203806
1.22389439056138 0.531347238274115
1.22356709317291 0.539007892120091
1.22334872031756 0.544114945632642
1.22323947593845 0.546668456066198
1.22323947590048 0.546668456066214
1.2230123362303 0.54664910964162
1.22255893634353 0.546599663012994
1.22188286957468 0.546493423413052
1.22099418533407 0.546288449972454
1.21991546682544 0.545929219980312
1.21868995460871 0.545349990456186
1.21739061938076 0.544480715471633
1.21610030698399 0.543230273381219
1.21526412455335 0.541961848316668
1.21476024252823 0.540762508740427
1.21448767411105 0.539705040879324
1.21435620536168 0.538833702471361
1.21429925392167 0.538171836563276
1.21427642250785 0.537729171531928
1.21429899313131 0.537497396084245
1.21429899310163 0.53749739608416
1.21173554316399 0.537351398434274
1.2066086675668 0.537059357198718
1.19891842689941 0.536621157703719
1.18866491799536 0.536036617118684
1.17584827376684 0.535305485487585
1.16046866297006 0.534427447211652
1.1425262899005 0.533402122984755
1.12202139401749 0.532229072183377
1.10151726555175 0.531054604210161
1.08101390213675 0.529878734481104
1.06051130139776 0.528701478429746
1.04000946095186 0.5275228515073
1.01950837840803 0.526342869182777
0.999008051367172 0.525161546943109
0.978508477422155 0.523978900293278
0.958009654157879 0.522794944756435
0.937511579151311 0.521609695874025
0.917014249971538 0.520423169205916
0.896517664179812 0.519235380330514
0.876021819329601 0.518046344844887
0.855526712966634 0.516856078364893
0.835032342628955 0.515664596525293
0.814538705846972 0.514471914979876
0.794045800143499 0.513278049401579
0.773553623033816 0.512083015482608
0.753062172025712 0.510886828934553
0.732571444619537 0.509689505488512
0.712081438308253 0.508491060895207
0.691592150577485 0.507291510925102
0.671103578905572 0.506090871368521
0.650615720763616 0.504889158035764
0.630128573615536 0.503686386757223
0.609642134918118 0.502482573383502
0.589156402121068 0.501277733785526
0.568671372667066 0.500071883854662
0.548187043991813 0.49886503950283
0.527703413524089 0.497657216662618
0.507220478685805 0.496448431287396
0.486738236892053 0.495238699351429
0.466256685551165 0.494028036849988
0.445775822064762 0.492816459799465
0.425295643827809 0.491603984237482
0.404816148228671 0.490390626223003
0.384337332649168 0.489176401836445
0.363859194464625 0.487961327179788
0.343381731043935 0.486745418376685
0.322904939749608 0.48552869157257
0.302428817937827 0.484311162934768
0.281953362958509 0.483092848652601
0.261478572155355 0.4818737649375
0.243563709029771 0.480806407637357
0.228208544470985 0.479891046434884
0.215412881797103 0.479127906813159
0.20517655624396 0.4785171732374
0.197499434529847 0.478058991869279
0.192381414497009 0.477753472810477
0.189822424830656 0.477600691872751
0.189822424856044 0.477600691872319
0.18984329041129 0.477361217804826
0.189891281488213 0.476888179768742
0.18998422190601 0.476195304214649
0.190156086957441 0.475302243681175
0.19046107666499 0.47423395694596
0.190973511005364 0.47302370084793
0.1917804370421 0.471720939032194
0.192990859960637 0.470375213510415
0.194255394688102 0.469435488118
0.195465103825654 0.468815206001496
0.196534139830494 0.468445902524195
0.19741290176854 0.468255645457359
0.19807765083531 0.468178447005524
0.198520413044486 0.468160326901788
0.198740901208049 0.468162462817004
0.198740901269321 0.46816246281638
0.198945440782965 0.465612270025241
0.199354508471781 0.460511864695189
0.199968076306772 0.452861197525416
0.200786100039452 0.442660189857331
0.201808520127061 0.429908733995519
0.203035263052485 0.414606693655117
0.20446624304152 0.396753904535665
0.206101364180888 0.376350175021588
0.207736171974856 0.355945830121864
0.209370679915717 0.335540874152152
0.211004901527408 0.315135311429408
0.212638850365535 0.294729146271967
0.214272540017396 0.27432238299961
0.215905984101993 0.253915025933645
0.217539196270065 0.233507079396979
0.219172190204101 0.2130985477142
0.220804979618368 0.192689435211644
0.222437578258934 0.172279746217481
0.224069999903693 0.15186948506179
0.225702258362386 0.131458656076634
0.227334367476629 0.111047263596139
0.228966341119943 0.0906353119565765
0.230598193197773 0.0702228054964383
0.232229937647521 0.0498097485565179
0.233861588438572 0.0293961454799898
0.235493159572323 0.00898200061249
0.237124665082211 -0.0114326816978031
0.238756119033746 -0.0318478971000879
0.240387535524538 -0.0522636412408578
0.24201892868433 -0.0726799097638194
0.243650312675029 -0.0930966983098097
0.245281701690739 -0.113514002516714
0.246913109957795 -0.133931818019382
0.248544551734793 -0.154350140449545
0.250176041312631 -0.174768965435729
0.251807593014535 -0.195188288603176
0.253439221196103 -0.215608105573753
0.255070940245335 -0.236028411965869
0.256702764582677 -0.256449203394391
0.258334708661049 -0.276870475470555
0.259966786965892 -0.29729222380188
0.261599014015202 -0.317714443992083
0.263231404359571 -0.338137131640988
0.264863972582227 -0.358560282344441
0.266496733299076 -0.378983891694222
0.26812970115874 -0.399407955277952
0.269762890842605 -0.419832468679009
0.271396317064857 -0.440257427476434
0.27302999457253 -0.460682827244845
0.27466393814555 -0.481108663554342
0.276093884549649 -0.498981648418472
0.277319746923832 -0.514301630525306
0.278341443563315 -0.527068481124652
0.279158900790641 -0.537282093189924
0.279772055404735 -0.544942380708266
0.280180856703051 -0.550049278099581
0.280385268072754 -0.552602739765039
};
\addplot [semithick, green, dash pattern=on 1pt off 3pt on 3pt off 3pt, forget plot]
table {%
0.25 -0.5
0.2525 -0.5
0.2575 -0.5
0.265 -0.5
0.275 -0.5
0.2875 -0.5
0.3025 -0.5
0.32 -0.5
0.34 -0.5
0.36 -0.5
0.38 -0.5
0.4 -0.5
0.42 -0.5
0.44 -0.5
0.46 -0.5
0.48 -0.5
0.5 -0.5
0.52 -0.5
0.54 -0.5
0.56 -0.5
0.58 -0.5
0.6 -0.5
0.62 -0.5
0.64 -0.5
0.66 -0.5
0.68 -0.5
0.7 -0.5
0.72 -0.5
0.74 -0.5
0.76 -0.5
0.78 -0.5
0.8 -0.5
0.82 -0.5
0.84 -0.5
0.86 -0.5
0.88 -0.5
0.9 -0.5
0.92 -0.5
0.94 -0.5
0.96 -0.5
0.98 -0.5
1 -0.5
1.02 -0.5
1.04 -0.5
1.06 -0.5
1.08 -0.5
1.1 -0.5
1.12 -0.5
1.14 -0.5
1.16 -0.5
1.18 -0.5
1.1975 -0.5
1.2125 -0.5
1.225 -0.5
1.235 -0.5
1.2425 -0.5
1.2475 -0.5
1.25 -0.5
1.25 -0.5
1.25 -0.5
1.25 -0.5
1.25 -0.5
1.25 -0.5
1.25 -0.5
1.25 -0.5
1.25 -0.5
1.25 -0.5
1.25 -0.5
1.25 -0.5
1.25 -0.5
1.25 -0.5
1.25 -0.5
1.25 -0.5
1.25 -0.5
1.25 -0.5
1.25 -0.4975
1.25 -0.4925
1.25 -0.485
1.25 -0.475
1.25 -0.4625
1.25 -0.4475
1.25 -0.43
1.25 -0.41
1.25 -0.39
1.25 -0.37
1.25 -0.35
1.25 -0.33
1.25 -0.31
1.25 -0.29
1.25 -0.27
1.25 -0.25
1.25 -0.23
1.25 -0.21
1.25 -0.19
1.25 -0.17
1.25 -0.15
1.25 -0.13
1.25 -0.11
1.25 -0.09
1.25 -0.07
1.25 -0.05
1.25 -0.03
1.25 -0.01
1.25 0.00999999999999998
1.25 0.03
1.25 0.05
1.25 0.07
1.25 0.09
1.25 0.11
1.25 0.13
1.25 0.15
1.25 0.17
1.25 0.19
1.25 0.21
1.25 0.23
1.25 0.25
1.25 0.27
1.25 0.29
1.25 0.31
1.25 0.33
1.25 0.35
1.25 0.37
1.25 0.39
1.25 0.41
1.25 0.43
1.25 0.4475
1.25 0.4625
1.25 0.475
1.25 0.485
1.25 0.4925
1.25 0.4975
1.25 0.5
1.25 0.5
1.25 0.5
1.25 0.5
1.25 0.5
1.25 0.5
1.25 0.5
1.25 0.5
1.25 0.5
1.25 0.5
1.25 0.5
1.25 0.5
1.25 0.5
1.25 0.5
1.25 0.5
1.25 0.5
1.25 0.5
1.25 0.5
1.2475 0.5
1.2425 0.5
1.235 0.5
1.225 0.5
1.2125 0.5
1.1975 0.5
1.18 0.5
1.16 0.5
1.14 0.5
1.12 0.5
1.1 0.5
1.08 0.5
1.06 0.5
1.04 0.5
1.02 0.5
1 0.5
0.98 0.5
0.96 0.5
0.94 0.5
0.92 0.5
0.9 0.5
0.88 0.5
0.86 0.5
0.84 0.5
0.82 0.5
0.8 0.5
0.78 0.5
0.76 0.5
0.74 0.5
0.72 0.5
0.7 0.5
0.68 0.5
0.66 0.5
0.64 0.5
0.62 0.5
0.6 0.5
0.58 0.5
0.56 0.5
0.54 0.5
0.52 0.5
0.5 0.5
0.48 0.5
0.46 0.5
0.44 0.5
0.42 0.5
0.4 0.5
0.38 0.5
0.36 0.5
0.34 0.5
0.32 0.5
0.3025 0.5
0.2875 0.5
0.275 0.5
0.265 0.5
0.2575 0.5
0.2525 0.5
0.25 0.5
0.25 0.5
0.25 0.5
0.25 0.5
0.25 0.5
0.25 0.5
0.25 0.5
0.25 0.5
0.25 0.5
0.25 0.5
0.25 0.5
0.25 0.5
0.25 0.5
0.25 0.5
0.25 0.5
0.25 0.5
0.25 0.5
0.25 0.5
0.25 0.4975
0.25 0.4925
0.25 0.485
0.25 0.475
0.25 0.4625
0.25 0.4475
0.25 0.43
0.25 0.41
0.25 0.39
0.25 0.37
0.25 0.35
0.25 0.33
0.25 0.31
0.25 0.29
0.25 0.27
0.25 0.25
0.25 0.23
0.25 0.21
0.25 0.19
0.25 0.17
0.25 0.15
0.249999999999999 0.13
0.249999999999999 0.11
0.249999999999999 0.09
0.249999999999999 0.07
0.249999999999999 0.05
0.249999999999999 0.03
0.249999999999999 0.01
0.249999999999999 -0.00999999999999998
0.249999999999999 -0.03
0.249999999999999 -0.05
0.249999999999999 -0.07
0.249999999999999 -0.09
0.249999999999999 -0.11
0.249999999999999 -0.13
0.249999999999999 -0.15
0.249999999999999 -0.17
0.249999999999999 -0.19
0.249999999999999 -0.21
0.249999999999999 -0.23
0.249999999999999 -0.25
0.249999999999999 -0.27
0.249999999999999 -0.29
0.249999999999999 -0.31
0.249999999999999 -0.33
0.249999999999999 -0.35
0.249999999999999 -0.37
0.249999999999999 -0.39
0.249999999999999 -0.41
0.249999999999999 -0.43
0.249999999999999 -0.4475
0.249999999999999 -0.4625
0.249999999999999 -0.475
0.249999999999999 -0.485
0.249999999999999 -0.4925
0.249999999999999 -0.4975
0.249999999999999 -0.5
};
\addplot [semithick, blue, opacity=\opacityRef, forget plot]
table {%
0.25 -0.5
0.250280737876892 -0.500106394290924
0.254356980323792 -0.499754250049591
0.262064397335052 -0.499844193458557
0.271713018417358 -0.499760389328003
0.284218430519104 -0.499681770801544
0.29900735616684 -0.499491095542908
0.316877275705338 -0.499148488044739
0.336816966533661 -0.498802810907364
0.358074516057968 -0.498356223106384
0.378538757562637 -0.497951686382294
0.39897820353508 -0.497435569763184
0.419487565755844 -0.497125446796417
0.439901977777481 -0.496647298336029
0.46051150560379 -0.496194839477539
0.481178879737854 -0.495776355266571
0.501916944980621 -0.495491355657578
0.522202432155609 -0.495184510946274
0.542912304401398 -0.49466210603714
0.563467442989349 -0.494416415691376
0.584466636180878 -0.494044214487076
0.604529917240143 -0.493492364883423
0.625318586826324 -0.493321299552917
0.645613670349121 -0.492967665195465
0.666333794593811 -0.492667555809021
0.686869323253632 -0.492313504219055
0.707302629947662 -0.491935431957245
0.728071987628937 -0.491492658853531
0.748493015766144 -0.491296142339706
0.769352197647095 -0.490975201129913
0.789832770824432 -0.49049785733223
0.810291826725006 -0.490115612745285
0.830882489681244 -0.489732295274734
0.851446807384491 -0.489381492137909
0.871988296508789 -0.488953173160553
0.892326354980469 -0.488619178533554
0.912856638431549 -0.48818227648735
0.933341026306152 -0.48778423666954
0.954060018062592 -0.487312078475952
0.97458291053772 -0.487149566411972
0.995022714138031 -0.486462414264679
1.01563340425491 -0.48613902926445
1.03608709573746 -0.485725045204163
1.05655354261398 -0.485331416130066
1.0774160027504 -0.484827220439911
1.09749263525009 -0.484602481126785
1.11806207895279 -0.484284251928329
1.13844805955887 -0.484036028385162
1.15929645299911 -0.483709305524826
1.18001633882523 -0.483491897583008
1.2006214261055 -0.483324706554413
1.21977287530899 -0.483062475919724
1.23530870676041 -0.482956647872925
1.24922055006027 -0.482729107141495
1.25963646173477 -0.482646375894547
1.26792806386948 -0.482554316520691
1.27330511808395 -0.482519239187241
1.27617102861404 -0.482467472553253
1.27682453393936 -0.482491552829742
1.27737540006638 -0.482439070940018
1.27835136651993 -0.482055187225342
1.27827507257462 -0.482080906629562
1.27826648950577 -0.482180088758469
1.27839380502701 -0.481982886791229
1.27832728624344 -0.481825619935989
1.27835029363632 -0.481573641300201
1.27785211801529 -0.481439679861069
1.27742463350296 -0.481757909059525
1.27757257223129 -0.48179766535759
1.27760344743729 -0.481246143579483
1.27755099534988 -0.481280267238617
1.27734965085983 -0.481331616640091
1.27720123529434 -0.481453627347946
1.27708297967911 -0.48141947388649
1.27720063924789 -0.481438100337982
1.27717024087906 -0.481301546096802
1.27692610025406 -0.477720201015472
1.27659446001053 -0.470097213983536
1.27629119157791 -0.46086722612381
1.2757356762886 -0.448078393936157
1.27520531415939 -0.433191329240799
1.27460032701492 -0.415555268526077
1.27393954992294 -0.395040601491928
1.27326363325119 -0.374481469392776
1.27242106199265 -0.354243844747543
1.27169400453568 -0.334173053503036
1.27114588022232 -0.313457429409027
1.27055579423904 -0.292454212903976
1.26981884241104 -0.271825730800629
1.26907962560654 -0.251038402318954
1.26836377382278 -0.230817660689354
1.26761871576309 -0.209483906626701
1.26709944009781 -0.189185783267021
1.26654547452927 -0.168870866298676
1.2659472823143 -0.148660585284233
1.26510661840439 -0.128162354230881
1.26444834470749 -0.107355184853077
1.2640705704689 -0.0868785232305527
1.26329952478409 -0.0664488822221756
1.26259690523148 -0.0456597395241261
1.261578977108 -0.0253757517784834
1.26083105802536 -0.00503821484744549
1.25995939970016 0.0162620190531015
1.25932985544205 0.0362987034022808
1.25851184129715 0.0567420646548271
1.25764185190201 0.0776568800210953
1.25692540407181 0.098129890859127
1.25602585077286 0.117880329489708
1.25542277097702 0.13884299993515
1.25421994924545 0.159677714109421
1.25321501493454 0.18032830953598
1.25252860784531 0.200523942708969
1.2517803311348 0.221250131726265
1.25089412927628 0.241966769099236
1.25004750490189 0.262156218290329
1.24924749135971 0.28290668129921
1.24813109636307 0.302505224943161
1.24765592813492 0.32369464635849
1.24629217386246 0.344384968280792
1.24606984853745 0.365218132734299
1.24492341279984 0.385470747947693
1.24418848752975 0.406194359064102
1.24353760480881 0.426535367965698
1.24273782968521 0.44684910774231
1.24169880151749 0.467778295278549
1.24112957715988 0.48663455247879
1.23996561765671 0.502124965190887
1.23992699384689 0.516915738582611
1.23890489339828 0.530900120735168
1.23871701955795 0.537140667438507
1.23856049776077 0.541998147964478
1.23864418268204 0.545354664325714
1.23866873979568 0.545611917972565
1.23836225271225 0.546148777008057
1.23804992437363 0.547331750392914
1.23805409669876 0.547004163265228
1.23788172006607 0.546962738037109
1.23837321996689 0.547211766242981
1.2379087805748 0.546908497810364
1.23808068037033 0.546979129314423
1.23790627717972 0.546760618686676
1.23806899785995 0.546574354171753
1.23760265111923 0.5461665391922
1.2374581694603 0.546191096305847
1.23790675401688 0.546203911304474
1.23741418123245 0.546134114265442
1.23744505643845 0.546199202537537
1.23776239156723 0.546197175979614
1.23747235536575 0.546136736869812
1.23755341768265 0.546124815940857
1.2332392334938 0.545832633972168
1.22659856081009 0.545502245426178
1.21699243783951 0.544840037822723
1.20459419488907 0.54417222738266
1.18975847959518 0.543426096439362
1.17190724611282 0.542521893978119
1.15283268690109 0.541463553905487
1.13198059797287 0.54045957326889
1.11143499612808 0.539326727390289
1.09028905630112 0.538364112377167
1.07012289762497 0.537222027778625
1.04938322305679 0.536303520202637
1.02904576063156 0.535319447517395
1.00879782438278 0.534184694290161
0.988197207450867 0.532998740673065
0.967958450317383 0.532031714916229
0.946955978870392 0.53098726272583
0.927237451076508 0.529869437217712
0.906736969947815 0.528810560703278
0.886179029941559 0.527630865573883
0.865913808345795 0.526654183864594
0.844879508018494 0.52543967962265
0.824427604675293 0.524380385875702
0.804125726222992 0.523389101028442
0.784121930599213 0.522279739379883
0.763745248317719 0.521257758140564
0.743297100067139 0.520263314247131
0.722567737102509 0.519139528274536
0.701853215694427 0.51808112859726
0.68125718832016 0.517179608345032
0.660834491252899 0.51629501581192
0.639941573143005 0.515223324298859
0.619331300258636 0.514015018939972
0.598941802978516 0.513219118118286
0.578567504882812 0.512227416038513
0.55807888507843 0.51128876209259
0.537688851356506 0.510239243507385
0.516911089420319 0.509101390838623
0.496627986431122 0.508203625679016
0.476188033819199 0.507380247116089
0.455304563045502 0.506191670894623
0.434605479240417 0.505145788192749
0.413683295249939 0.504027247428894
0.393322050571442 0.502750217914581
0.372770607471466 0.502097249031067
0.352304339408875 0.501107633113861
0.331282645463943 0.500015765428543
0.310910016298294 0.499090194702148
0.290341913700104 0.497932285070419
0.270738571882248 0.496805161237717
0.255841493606567 0.496185719966888
0.242235779762268 0.495742738246918
0.231308311223984 0.495083391666412
0.223039627075195 0.494658678770065
0.217223063111305 0.494207352399826
0.214062556624413 0.494193643331528
0.213552162051201 0.494022876024246
0.213112279772758 0.493901193141937
0.211955890059471 0.493858963251114
0.21234442293644 0.493829965591431
0.212216809391975 0.493798434734344
0.212328344583511 0.4937484562397
0.212701737880707 0.493699938058853
0.212906837463379 0.493703007698059
0.213155955076218 0.493745505809784
0.213469967246056 0.493930488824844
0.213680565357208 0.493965119123459
0.213591158390045 0.49388587474823
0.213690474629402 0.493890672922134
0.213741764426231 0.49371013045311
0.213767021894455 0.493771284818649
0.213705375790596 0.493994504213333
0.213618069887161 0.493908524513245
0.213799446821213 0.49348446726799
0.213873639702797 0.489512920379639
0.214440315961838 0.482181340456009
0.214881643652916 0.472772359848022
0.21557229757309 0.460411339998245
0.21623794734478 0.445493906736374
0.21703565120697 0.428066223859787
0.218261301517487 0.408266484737396
0.219178766012192 0.387164741754532
0.220459431409836 0.367492854595184
0.221602737903595 0.346965938806534
0.222824156284332 0.326600193977356
0.223851382732391 0.305748611688614
0.224907293915749 0.285341769456863
0.226119250059128 0.264220654964447
0.227384656667709 0.244251891970634
0.228514701128006 0.223284929990768
0.229673475027084 0.202362298965454
0.23099522292614 0.181324154138565
0.231742039322853 0.161815285682678
0.233021572232246 0.141110479831696
0.234304070472717 0.120115727186203
0.235467880964279 0.100261174142361
0.236358910799026 0.0792772397398949
0.237335756421089 0.0593261085450649
0.238522693514824 0.0374630875885487
0.23968817293644 0.0174125004559755
0.240769729018211 -0.00346526969224215
0.241673573851585 -0.0233295410871506
0.243015617132187 -0.0440549850463867
0.24359142780304 -0.064533419907093
0.245054930448532 -0.0846668556332588
0.246529042720795 -0.105873815715313
0.247612476348877 -0.126205116510391
0.248457193374634 -0.146445363759995
0.249460399150848 -0.167632460594177
0.25082203745842 -0.188185632228851
0.25189334154129 -0.208203211426735
0.252956360578537 -0.229202181100845
0.254194974899292 -0.25021880865097
0.255318909883499 -0.270748615264893
0.256427228450775 -0.290782481431961
0.257725298404694 -0.311315506696701
0.258945167064667 -0.331228047609329
0.259862393140793 -0.351700454950333
0.260961413383484 -0.372767627239227
0.262171775102615 -0.393754452466965
0.263226807117462 -0.413885146379471
0.264218479394913 -0.434645175933838
0.26544177532196 -0.455401420593262
0.266306966543198 -0.474621593952179
0.267148494720459 -0.489957481622696
0.267893493175507 -0.503658711910248
0.268453121185303 -0.514330089092255
0.268890798091888 -0.522484362125397
0.269135743379593 -0.528214037418365
0.269276559352875 -0.531374514102936
0.269343793392181 -0.532299399375916
};
\addplot [semithick, red, dashed, forget plot]
table {%
0.25 -0.5
0.252558268640609 -0.499971215983182
0.257674800204547 -0.499913561943335
0.265349580546156 -0.499826822527541
0.275582586992214 -0.499710651911554
0.288373788282223 -0.499564571828345
0.303723144483551 -0.499387968751125
0.321630606880243 -0.499180090231924
0.342096117833915 -0.498940040397106
0.362561444756502 -0.498697137467738
0.383026586783779 -0.498451352589885
0.403491543037366 -0.498202656922469
0.423956312624558 -0.49795102163749
0.444420894638161 -0.497696417920264
0.464885288156327 -0.497438816969653
0.485349492242387 -0.497178189998295
0.505813505944689 -0.496914508232844
0.526277328296427 -0.4966477429142
0.546740958315482 -0.49637786529775
0.567204395004253 -0.496104846653605
0.587667637349495 -0.495828658266839
0.608130684322154 -0.495549271437727
0.628593534877201 -0.495266657481993
0.649056187953474 -0.494980787731044
0.669518642473505 -0.494691633532224
0.689980897343365 -0.49439916624905
0.710442951452497 -0.494103357261464
0.730904803673553 -0.493804177966079
0.751366452862231 -0.493501599776428
0.771827897857116 -0.493195594123215
0.792289137479511 -0.492886132454562
0.812750170533283 -0.492573186236266
0.833210995804693 -0.492256726952051
0.853671612062244 -0.491936726103822
0.874132018056508 -0.491613155211922
0.894592212519978 -0.491285985815386
0.915052194166895 -0.490955189472205
0.935511961693098 -0.49062073775958
0.955971513775856 -0.490282602274183
0.976430849073713 -0.489940754632423
0.996889966226327 -0.489595166470704
1.01734886385431 -0.489245809445691
1.03780754055907 -0.488892655234577
1.05826599492265 -0.488535675535346
1.07872422550757 -0.488174842067041
1.09918223085668 -0.487810126570036
1.11964000949299 -0.487441500806302
1.1400975599195 -0.48706893655968
1.16055488061908 -0.486692405636151
1.18101197005428 -0.486311879864113
1.20146882666719 -0.485927331094652
1.21936837110133 -0.485587306188302
1.23471068292331 -0.485293179242986
1.24749583154271 -0.485046149752488
1.25772387591864 -0.484847237098433
1.26539486430542 -0.484697275874471
1.27050883404033 -0.484596912039359
1.27306581137596 -0.484546599896135
1.2730658113589 -0.484546599895144
1.27305541155906 -0.484320911054072
1.2730238336127 -0.483870450269343
1.27294421772612 -0.483198810692961
1.27277401858219 -0.482315683800082
1.27245616754417 -0.481242223159833
1.27192174548361 -0.480017905211023
1.27109518948705 -0.478707989424014
1.26987791322201 -0.477382389353572
1.26861868448422 -0.476491262912181
1.26741310591075 -0.475930505649088
1.26634153930566 -0.475618318765773
1.26545420209816 -0.475475522024619
1.26477827798535 -0.475433459898834
1.26432554654506 -0.475438002031128
1.26409923112047 -0.475451102493
1.26409923107294 -0.475451102492838
1.26400510754809 -0.472893953643021
1.26381676720545 -0.467779675905825
1.26353397696839 -0.460108319156843
1.26315636493238 -0.449879963098326
1.26268342195054 -0.437094717113806
1.26211450389605 -0.421752720062896
1.26144883460045 -0.403854140014764
1.2606855094668 -0.383399173918339
1.25991927860775 -0.362944839063709
1.25915016511684 -0.342491133403313
1.25837819207707 -0.322038054872471
1.25760338256082 -0.301585601389565
1.25682575962964 -0.281133770856217
1.25604534633415 -0.260682561157473
1.25526216571386 -0.240231970161977
1.25447624079705 -0.219781995722154
1.25368759460061 -0.199332635674387
1.25289625012991 -0.178883887839198
1.25210223037863 -0.158435750021426
1.25130555832866 -0.137988220010405
1.25050625694995 -0.117541295580143
1.24970434920032 -0.0970949744895014
1.2488998580254 -0.0766492544823687
1.24809280635845 -0.0562041332878428
1.24728321712021 -0.0357596086204065
1.24647111321881 -0.0153156781801049
1.24565651754959 0.00512766034727754
1.244839452995 0.0255704092900399
1.24401994242446 0.0460125709903884
1.24319800869422 0.0664541478042603
1.24237367464724 0.0868951421011472
1.24154696311305 0.10733555626392
1.24071789690763 0.127775392688651
1.23988649883331 0.148214653784443
1.23905279167858 0.168653341973249
1.23821679821801 0.189091459689702
1.23737854121215 0.209529009380936
1.23653804340733 0.229965993506414
1.23569532753562 0.250402414537754
1.23485041631467 0.270838274958555
1.23400333244757 0.291273577264223
1.23315409862277 0.311708323961797
1.23230273751397 0.332142517569777
1.23144927177993 0.352576160617952
1.23059372406446 0.373009255647224
1.22973611699622 0.393441805209441
1.22887647318863 0.413873811867219
1.22801481523979 0.434305278193775
1.22715116573232 0.454736206772755
1.22628554723329 0.475166600198059
1.22552642790666 0.493042728464227
1.22487449160254 0.508364776283352
1.22433031870982 0.52113290203806
1.22389439056138 0.531347238274115
1.22356709317291 0.539007892120091
1.22334872031756 0.544114945632642
1.22323947593845 0.546668456066198
1.22323947590048 0.546668456066214
1.2230123362303 0.54664910964162
1.22255893634353 0.546599663012994
1.22188286957468 0.546493423413052
1.22099418533407 0.546288449972454
1.21991546682544 0.545929219980312
1.21868995460871 0.545349990456186
1.21739061938076 0.544480715471633
1.21610030698399 0.543230273381219
1.21526412455335 0.541961848316668
1.21476024252823 0.540762508740427
1.21448767411105 0.539705040879324
1.21435620536168 0.538833702471361
1.21429925392167 0.538171836563276
1.21427642250785 0.537729171531928
1.21429899313131 0.537497396084245
1.21429899310163 0.53749739608416
1.21173554316399 0.537351398434274
1.2066086675668 0.537059357198718
1.19891842689941 0.536621157703719
1.18866491799536 0.536036617118684
1.17584827376684 0.535305485487585
1.16046866297006 0.534427447211652
1.1425262899005 0.533402122984755
1.12202139401749 0.532229072183377
1.10151726555175 0.531054604210161
1.08101390213675 0.529878734481104
1.06051130139776 0.528701478429746
1.04000946095186 0.5275228515073
1.01950837840803 0.526342869182777
0.999008051367172 0.525161546943109
0.978508477422155 0.523978900293278
0.958009654157879 0.522794944756435
0.937511579151311 0.521609695874025
0.917014249971538 0.520423169205916
0.896517664179812 0.519235380330514
0.876021819329601 0.518046344844887
0.855526712966634 0.516856078364893
0.835032342628955 0.515664596525293
0.814538705846972 0.514471914979876
0.794045800143499 0.513278049401579
0.773553623033816 0.512083015482608
0.753062172025712 0.510886828934553
0.732571444619537 0.509689505488512
0.712081438308253 0.508491060895207
0.691592150577485 0.507291510925102
0.671103578905572 0.506090871368521
0.650615720763616 0.504889158035764
0.630128573615536 0.503686386757223
0.609642134918118 0.502482573383502
0.589156402121068 0.501277733785526
0.568671372667066 0.500071883854662
0.548187043991813 0.49886503950283
0.527703413524089 0.497657216662618
0.507220478685805 0.496448431287396
0.486738236892053 0.495238699351429
0.466256685551165 0.494028036849988
0.445775822064762 0.492816459799465
0.425295643827809 0.491603984237482
0.404816148228671 0.490390626223003
0.384337332649168 0.489176401836445
0.363859194464625 0.487961327179788
0.343381731043935 0.486745418376685
0.322904939749608 0.48552869157257
0.302428817937827 0.484311162934768
0.281953362958509 0.483092848652601
0.261478572155355 0.4818737649375
0.243563709029771 0.480806407637357
0.228208544470985 0.479891046434884
0.215412881797103 0.479127906813159
0.20517655624396 0.4785171732374
0.197499434529847 0.478058991869279
0.192381414497009 0.477753472810477
0.189822424830656 0.477600691872751
0.189822424856044 0.477600691872319
0.18984329041129 0.477361217804826
0.189891281488213 0.476888179768742
0.18998422190601 0.476195304214649
0.190156086957441 0.475302243681175
0.19046107666499 0.47423395694596
0.190973511005364 0.47302370084793
0.1917804370421 0.471720939032194
0.192990859960637 0.470375213510415
0.194255394688102 0.469435488118
0.195465103825654 0.468815206001496
0.196534139830494 0.468445902524195
0.19741290176854 0.468255645457359
0.19807765083531 0.468178447005524
0.198520413044486 0.468160326901788
0.198740901208049 0.468162462817004
0.198740901269321 0.46816246281638
0.198945440782965 0.465612270025241
0.199354508471781 0.460511864695189
0.199968076306772 0.452861197525416
0.200786100039452 0.442660189857331
0.201808520127061 0.429908733995519
0.203035263052485 0.414606693655117
0.20446624304152 0.396753904535665
0.206101364180888 0.376350175021588
0.207736171974856 0.355945830121864
0.209370679915717 0.335540874152152
0.211004901527408 0.315135311429408
0.212638850365535 0.294729146271967
0.214272540017396 0.27432238299961
0.215905984101993 0.253915025933645
0.217539196270065 0.233507079396979
0.219172190204101 0.2130985477142
0.220804979618368 0.192689435211644
0.222437578258934 0.172279746217481
0.224069999903693 0.15186948506179
0.225702258362386 0.131458656076634
0.227334367476629 0.111047263596139
0.228966341119943 0.0906353119565765
0.230598193197773 0.0702228054964383
0.232229937647521 0.0498097485565179
0.233861588438572 0.0293961454799898
0.235493159572323 0.00898200061249
0.237124665082211 -0.0114326816978031
0.238756119033746 -0.0318478971000879
0.240387535524538 -0.0522636412408578
0.24201892868433 -0.0726799097638194
0.243650312675029 -0.0930966983098097
0.245281701690739 -0.113514002516714
0.246913109957795 -0.133931818019382
0.248544551734793 -0.154350140449545
0.250176041312631 -0.174768965435729
0.251807593014535 -0.195188288603176
0.253439221196103 -0.215608105573753
0.255070940245335 -0.236028411965869
0.256702764582677 -0.256449203394391
0.258334708661049 -0.276870475470555
0.259966786965892 -0.29729222380188
0.261599014015202 -0.317714443992083
0.263231404359571 -0.338137131640988
0.264863972582227 -0.358560282344441
0.266496733299076 -0.378983891694222
0.26812970115874 -0.399407955277952
0.269762890842605 -0.419832468679009
0.271396317064857 -0.440257427476434
0.27302999457253 -0.460682827244845
0.27466393814555 -0.481108663554342
0.276093884549649 -0.498981648418472
0.277319746923832 -0.514301630525306
0.278341443563315 -0.527068481124652
0.279158900790641 -0.537282093189924
0.279772055404735 -0.544942380708266
0.280180856703051 -0.550049278099581
0.280385268072754 -0.552602739765039
};
\addplot [semithick, green, dash pattern=on 1pt off 3pt on 3pt off 3pt, forget plot]
table {%
0.25 -0.5
0.2525 -0.5
0.2575 -0.5
0.265 -0.5
0.275 -0.5
0.2875 -0.5
0.3025 -0.5
0.32 -0.5
0.34 -0.5
0.36 -0.5
0.38 -0.5
0.4 -0.5
0.42 -0.5
0.44 -0.5
0.46 -0.5
0.48 -0.5
0.5 -0.5
0.52 -0.5
0.54 -0.5
0.56 -0.5
0.58 -0.5
0.6 -0.5
0.62 -0.5
0.64 -0.5
0.66 -0.5
0.68 -0.5
0.7 -0.5
0.72 -0.5
0.74 -0.5
0.76 -0.5
0.78 -0.5
0.8 -0.5
0.82 -0.5
0.84 -0.5
0.86 -0.5
0.88 -0.5
0.9 -0.5
0.92 -0.5
0.94 -0.5
0.96 -0.5
0.98 -0.5
1 -0.5
1.02 -0.5
1.04 -0.5
1.06 -0.5
1.08 -0.5
1.1 -0.5
1.12 -0.5
1.14 -0.5
1.16 -0.5
1.18 -0.5
1.1975 -0.5
1.2125 -0.5
1.225 -0.5
1.235 -0.5
1.2425 -0.5
1.2475 -0.5
1.25 -0.5
1.25 -0.5
1.25 -0.5
1.25 -0.5
1.25 -0.5
1.25 -0.5
1.25 -0.5
1.25 -0.5
1.25 -0.5
1.25 -0.5
1.25 -0.5
1.25 -0.5
1.25 -0.5
1.25 -0.5
1.25 -0.5
1.25 -0.5
1.25 -0.5
1.25 -0.5
1.25 -0.4975
1.25 -0.4925
1.25 -0.485
1.25 -0.475
1.25 -0.4625
1.25 -0.4475
1.25 -0.43
1.25 -0.41
1.25 -0.39
1.25 -0.37
1.25 -0.35
1.25 -0.33
1.25 -0.31
1.25 -0.29
1.25 -0.27
1.25 -0.25
1.25 -0.23
1.25 -0.21
1.25 -0.19
1.25 -0.17
1.25 -0.15
1.25 -0.13
1.25 -0.11
1.25 -0.09
1.25 -0.07
1.25 -0.05
1.25 -0.03
1.25 -0.01
1.25 0.00999999999999998
1.25 0.03
1.25 0.05
1.25 0.07
1.25 0.09
1.25 0.11
1.25 0.13
1.25 0.15
1.25 0.17
1.25 0.19
1.25 0.21
1.25 0.23
1.25 0.25
1.25 0.27
1.25 0.29
1.25 0.31
1.25 0.33
1.25 0.35
1.25 0.37
1.25 0.39
1.25 0.41
1.25 0.43
1.25 0.4475
1.25 0.4625
1.25 0.475
1.25 0.485
1.25 0.4925
1.25 0.4975
1.25 0.5
1.25 0.5
1.25 0.5
1.25 0.5
1.25 0.5
1.25 0.5
1.25 0.5
1.25 0.5
1.25 0.5
1.25 0.5
1.25 0.5
1.25 0.5
1.25 0.5
1.25 0.5
1.25 0.5
1.25 0.5
1.25 0.5
1.25 0.5
1.2475 0.5
1.2425 0.5
1.235 0.5
1.225 0.5
1.2125 0.5
1.1975 0.5
1.18 0.5
1.16 0.5
1.14 0.5
1.12 0.5
1.1 0.5
1.08 0.5
1.06 0.5
1.04 0.5
1.02 0.5
1 0.5
0.98 0.5
0.96 0.5
0.94 0.5
0.92 0.5
0.9 0.5
0.88 0.5
0.86 0.5
0.84 0.5
0.82 0.5
0.8 0.5
0.78 0.5
0.76 0.5
0.74 0.5
0.72 0.5
0.7 0.5
0.68 0.5
0.66 0.5
0.64 0.5
0.62 0.5
0.6 0.5
0.58 0.5
0.56 0.5
0.54 0.5
0.52 0.5
0.5 0.5
0.48 0.5
0.46 0.5
0.44 0.5
0.42 0.5
0.4 0.5
0.38 0.5
0.36 0.5
0.34 0.5
0.32 0.5
0.3025 0.5
0.2875 0.5
0.275 0.5
0.265 0.5
0.2575 0.5
0.2525 0.5
0.25 0.5
0.25 0.5
0.25 0.5
0.25 0.5
0.25 0.5
0.25 0.5
0.25 0.5
0.25 0.5
0.25 0.5
0.25 0.5
0.25 0.5
0.25 0.5
0.25 0.5
0.25 0.5
0.25 0.5
0.25 0.5
0.25 0.5
0.25 0.5
0.25 0.4975
0.25 0.4925
0.25 0.485
0.25 0.475
0.25 0.4625
0.25 0.4475
0.25 0.43
0.25 0.41
0.25 0.39
0.25 0.37
0.25 0.35
0.25 0.33
0.25 0.31
0.25 0.29
0.25 0.27
0.25 0.25
0.25 0.23
0.25 0.21
0.25 0.19
0.25 0.17
0.25 0.15
0.249999999999999 0.13
0.249999999999999 0.11
0.249999999999999 0.09
0.249999999999999 0.07
0.249999999999999 0.05
0.249999999999999 0.03
0.249999999999999 0.01
0.249999999999999 -0.00999999999999998
0.249999999999999 -0.03
0.249999999999999 -0.05
0.249999999999999 -0.07
0.249999999999999 -0.09
0.249999999999999 -0.11
0.249999999999999 -0.13
0.249999999999999 -0.15
0.249999999999999 -0.17
0.249999999999999 -0.19
0.249999999999999 -0.21
0.249999999999999 -0.23
0.249999999999999 -0.25
0.249999999999999 -0.27
0.249999999999999 -0.29
0.249999999999999 -0.31
0.249999999999999 -0.33
0.249999999999999 -0.35
0.249999999999999 -0.37
0.249999999999999 -0.39
0.249999999999999 -0.41
0.249999999999999 -0.43
0.249999999999999 -0.4475
0.249999999999999 -0.4625
0.249999999999999 -0.475
0.249999999999999 -0.485
0.249999999999999 -0.4925
0.249999999999999 -0.4975
0.249999999999999 -0.5
};
\addplot [semithick, blue, opacity=\opacityRef, forget plot]
table {%
0.25 -0.5
0.250483870506287 -0.499836266040802
0.253716975450516 -0.499737739562988
0.260877013206482 -0.499832451343536
0.270609587430954 -0.499945223331451
0.28277000784874 -0.500046908855438
0.297772943973541 -0.499780356884003
0.315113872289658 -0.499751448631287
0.335145860910416 -0.499718606472015
0.356252402067184 -0.499409139156342
0.376153767108917 -0.499573230743408
0.397225886583328 -0.499227702617645
0.418017119169235 -0.499524056911469
0.438529908657074 -0.499329805374146
0.459273725748062 -0.499297499656677
0.479789704084396 -0.499048054218292
0.500352084636688 -0.499000608921051
0.520805478096008 -0.498747110366821
0.541672646999359 -0.498673856258392
0.561685025691986 -0.498569488525391
0.58249294757843 -0.498639345169067
0.602801382541656 -0.498510122299194
0.623732805252075 -0.498447299003601
0.644278466701508 -0.498478591442108
0.664875864982605 -0.498591363430023
0.685400426387787 -0.498487293720245
0.706092774868011 -0.498355001211166
0.726780295372009 -0.498313665390015
0.747339904308319 -0.498213350772858
0.767729878425598 -0.498084038496017
0.788036406040192 -0.498057037591934
0.808623015880585 -0.498047709465027
0.829141855239868 -0.497966378927231
0.849948823451996 -0.497901171445847
0.870375156402588 -0.497890383005142
0.89098459482193 -0.498254626989365
0.911631941795349 -0.497883409261703
0.932579517364502 -0.497698128223419
0.952785611152649 -0.497436195611954
0.973345875740051 -0.497344613075256
0.993647038936615 -0.497147500514984
1.01370251178741 -0.496798038482666
1.03483533859253 -0.49693101644516
1.05484640598297 -0.496352285146713
1.07578563690186 -0.496450006961823
1.09612715244293 -0.496748208999634
1.1169399023056 -0.49682292342186
1.13753426074982 -0.496756732463837
1.15806150436401 -0.496780186891556
1.17859268188477 -0.496855527162552
1.19919490814209 -0.496632933616638
1.21888601779938 -0.496787875890732
1.2338924407959 -0.496862471103668
1.24721002578735 -0.496907860040665
1.25805854797363 -0.496946781873703
1.26614928245544 -0.496797680854797
1.27182757854462 -0.496809095144272
1.27494645118713 -0.496649473905563
1.27545619010925 -0.496775686740875
1.27613520622253 -0.496751517057419
1.27723515033722 -0.496498584747314
1.27704012393951 -0.496608674526215
1.27728343009949 -0.496484845876694
1.27730703353882 -0.49644860625267
1.2776620388031 -0.496116101741791
1.27697145938873 -0.49611222743988
1.27691280841827 -0.495848059654236
1.27661275863647 -0.495979934930801
1.2764880657196 -0.495844453573227
1.27635145187378 -0.495986670255661
1.2764003276825 -0.495929539203644
1.27634453773499 -0.496025145053864
1.27626097202301 -0.496272951364517
1.27625477313995 -0.49600088596344
1.27621173858643 -0.496005237102509
1.27618634700775 -0.495565503835678
1.27615427970886 -0.492361903190613
1.27579414844513 -0.484641253948212
1.27555000782013 -0.474888116121292
1.27531933784485 -0.46252965927124
1.27480947971344 -0.447694689035416
1.27449464797974 -0.430272817611694
1.27417707443237 -0.410215854644775
1.2739315032959 -0.389323621988297
1.27338349819183 -0.369144111871719
1.27272343635559 -0.348220586776733
1.27222073078156 -0.327884197235107
1.27190780639648 -0.30730339884758
1.2717844247818 -0.286951005458832
1.27143359184265 -0.266150236129761
1.2708865404129 -0.245489507913589
1.27055263519287 -0.225325092673302
1.27033460140228 -0.204356133937836
1.2700856924057 -0.184014216065407
1.26973628997803 -0.163610383868217
1.26919651031494 -0.142851278185844
1.2688125371933 -0.122412815690041
1.26831221580505 -0.101462386548519
1.26794302463531 -0.0814569219946861
1.26756775379181 -0.0605224519968033
1.2671936750412 -0.0399056002497673
1.2666472196579 -0.0199417639523745
1.26595067977905 0.000758258742280304
1.26568937301636 0.021542277187109
1.26516580581665 0.0421480908989906
1.26483082771301 0.0627353824675083
1.26386404037476 0.0832195356488228
1.26337337493896 0.103678584098816
1.26280558109283 0.124352365732193
1.26228761672974 0.145199790596962
1.2613822221756 0.165264591574669
1.26109182834625 0.185896724462509
1.26062941551208 0.206269845366478
1.25979912281036 0.226691126823425
1.2593514919281 0.247707799077034
1.25886654853821 0.268082737922668
1.25846862792969 0.288561522960663
1.2577451467514 0.308916807174683
1.25655686855316 0.329890578985214
1.25657403469086 0.350737452507019
1.2555730342865 0.371034622192383
1.25549554824829 0.39136078953743
1.25495111942291 0.41152024269104
1.25404822826385 0.432623744010925
1.25395715236664 0.452745765447617
1.25359511375427 0.472050875425339
1.25295507907867 0.487116008996964
1.25308728218079 0.500851064920425
1.2524516582489 0.511758983135223
1.25218832492828 0.5199373960495
1.25200653076172 0.52560955286026
1.25172185897827 0.528615713119507
1.25190091133118 0.529056549072266
1.25221824645996 0.529882550239563
1.25174474716187 0.530978083610535
1.25174868106842 0.530204474925995
1.25220835208893 0.530525386333466
1.25190794467926 0.530449688434601
1.25151264667511 0.530194222927094
1.25142967700958 0.530468821525574
1.25123882293701 0.53028130531311
1.25175154209137 0.529717922210693
1.25188314914703 0.52951967716217
1.25111401081085 0.529717624187469
1.25145542621613 0.529764592647552
1.25091671943665 0.529930591583252
1.25048196315765 0.529935717582703
1.25110256671906 0.529906332492828
1.25085508823395 0.529911696910858
1.25068426132202 0.529915332794189
1.24692976474762 0.5298171043396
1.23960745334625 0.529886484146118
1.23018181324005 0.529710829257965
1.21810734272003 0.529517889022827
1.20259261131287 0.529403686523438
1.18571984767914 0.529320955276489
1.16517663002014 0.528982698917389
1.1440966129303 0.528855502605438
1.12455129623413 0.528700172901154
1.10391819477081 0.528523445129395
1.08328986167908 0.528336703777313
1.06234812736511 0.528021335601807
1.04214632511139 0.527978181838989
1.02232551574707 0.527869284152985
1.00172054767609 0.527571797370911
0.980817496776581 0.527230799198151
0.960290133953094 0.527137935161591
0.939865529537201 0.52688455581665
0.919452250003815 0.526792943477631
0.899110913276672 0.526454091072083
0.87880951166153 0.52614289522171
0.858191728591919 0.525892913341522
0.83810019493103 0.525707483291626
0.817663013935089 0.525432944297791
0.797246754169464 0.525275766849518
0.776411592960358 0.525198221206665
0.755868136882782 0.52484130859375
0.735302269458771 0.524731159210205
0.714726328849792 0.524558544158936
0.693811595439911 0.524373352527618
0.673502504825592 0.524301469326019
0.652758896350861 0.524103462696075
0.632657408714294 0.523885071277618
0.612196564674377 0.523737251758575
0.591527462005615 0.52356094121933
0.570662140846252 0.523387968540192
0.550093054771423 0.523458003997803
0.529677927494049 0.523126006126404
0.50896942615509 0.522896826267242
0.488607376813889 0.522863030433655
0.467568397521973 0.522599697113037
0.447284519672394 0.522321045398712
0.426875591278076 0.522293865680695
0.406168848276138 0.521790623664856
0.385536253452301 0.521699905395508
0.364819496870041 0.521669685840607
0.344118654727936 0.521505832672119
0.323143362998962 0.521172881126404
0.302531212568283 0.520682334899902
0.282715350389481 0.520256102085114
0.267072826623917 0.520341038703918
0.253442645072937 0.520302355289459
0.242574751377106 0.520383596420288
0.234329283237457 0.520314395427704
0.22881056368351 0.520314931869507
0.225653395056725 0.5202756524086
0.224800541996956 0.519982695579529
0.224248334765434 0.519980311393738
0.22371968626976 0.520095586776733
0.223872244358063 0.519919514656067
0.223957449197769 0.519778668880463
0.224061444401741 0.519568264484406
0.223973721265793 0.519438028335571
0.224235400557518 0.519498884677887
0.224404513835907 0.519410252571106
0.224744603037834 0.519605815410614
0.224881514906883 0.51943027973175
0.225143373012543 0.51936662197113
0.225278407335281 0.519758999347687
0.225304022431374 0.519619226455688
0.225359961390495 0.519716918468475
0.225281417369843 0.519500017166138
0.225322023034096 0.519545793533325
0.225330621004105 0.519154012203217
0.225366607308388 0.515400826931
0.225396126508713 0.507868111133575
0.225421741604805 0.498555183410645
0.225786730647087 0.486437618732452
0.225935831665993 0.471486836671829
0.226392954587936 0.45458135008812
0.226444363594055 0.434735685586929
0.226773247122765 0.413571059703827
0.226859137415886 0.393432885408401
0.227191418409348 0.372446417808533
0.227246254682541 0.352099299430847
0.227671504020691 0.331287205219269
0.228225395083427 0.310591727495193
0.228796526789665 0.289456516504288
0.228813096880913 0.269874215126038
0.229401618242264 0.249690055847168
0.229805380105972 0.229036182165146
0.230273902416229 0.208058416843414
0.230324149131775 0.187969595193863
0.230764508247375 0.16643925011158
0.231146052479744 0.145302921533585
0.231310829520226 0.125896014273167
0.231613174080849 0.104402139782906
0.231847941875458 0.0841227546334267
0.232446491718292 0.0631110742688179
0.232484012842178 0.0433390960097313
0.232787743210793 0.0225997995585203
0.23318612575531 0.00222139549441636
0.233628273010254 -0.0190286170691252
0.233745127916336 -0.0390503145754337
0.233991697430611 -0.0598182529211044
0.234112083911896 -0.0797668248414993
0.234511286020279 -0.100333280861378
0.234912768006325 -0.121678702533245
0.23536504805088 -0.143067359924316
0.235311582684517 -0.162392050027847
0.235617250204086 -0.182990983128548
0.236041575670242 -0.203253984451294
0.236677780747414 -0.223913073539734
0.236985966563225 -0.244611889123917
0.237162590026855 -0.264981806278229
0.237526416778564 -0.286180168390274
0.23755019903183 -0.306212365627289
0.238063007593155 -0.327083766460419
0.238432869315147 -0.347544848918915
0.238929510116577 -0.36811488866806
0.239092990756035 -0.388865500688553
0.239329263567924 -0.409213870763779
0.239719673991203 -0.429887592792511
0.239831015467644 -0.448518395423889
0.240269094705582 -0.464328169822693
0.240517884492874 -0.478263139724731
0.240452453494072 -0.488967746496201
0.240589708089828 -0.49748957157135
0.240377724170685 -0.502894639968872
0.240700572729111 -0.506259143352509
0.240473955869675 -0.506531178951263
};
\addplot [semithick, red, dashed, forget plot]
table {%
0.25 -0.5
0.252558268640609 -0.499971215983182
0.257674800204547 -0.499913561943335
0.265349580546156 -0.499826822527541
0.275582586992214 -0.499710651911554
0.288373788282223 -0.499564571828345
0.303723144483551 -0.499387968751125
0.321630606880243 -0.499180090231924
0.342096117833915 -0.498940040397106
0.362561444756502 -0.498697137467738
0.383026586783779 -0.498451352589885
0.403491543037366 -0.498202656922469
0.423956312624558 -0.49795102163749
0.444420894638161 -0.497696417920264
0.464885288156327 -0.497438816969653
0.485349492242387 -0.497178189998295
0.505813505944689 -0.496914508232844
0.526277328296427 -0.4966477429142
0.546740958315482 -0.49637786529775
0.567204395004253 -0.496104846653605
0.587667637349495 -0.495828658266839
0.608130684322154 -0.495549271437727
0.628593534877201 -0.495266657481993
0.649056187953474 -0.494980787731044
0.669518642473505 -0.494691633532224
0.689980897343365 -0.49439916624905
0.710442951452497 -0.494103357261464
0.730904803673553 -0.493804177966079
0.751366452862231 -0.493501599776428
0.771827897857116 -0.493195594123215
0.792289137479511 -0.492886132454562
0.812750170533283 -0.492573186236266
0.833210995804693 -0.492256726952051
0.853671612062244 -0.491936726103822
0.874132018056508 -0.491613155211922
0.894592212519978 -0.491285985815386
0.915052194166895 -0.490955189472205
0.935511961693098 -0.49062073775958
0.955971513775856 -0.490282602274183
0.976430849073713 -0.489940754632423
0.996889966226327 -0.489595166470704
1.01734886385431 -0.489245809445691
1.03780754055907 -0.488892655234577
1.05826599492265 -0.488535675535346
1.07872422550757 -0.488174842067041
1.09918223085668 -0.487810126570036
1.11964000949299 -0.487441500806302
1.1400975599195 -0.48706893655968
1.16055488061908 -0.486692405636151
1.18101197005428 -0.486311879864113
1.20146882666719 -0.485927331094652
1.21936837110133 -0.485587306188302
1.23471068292331 -0.485293179242986
1.24749583154271 -0.485046149752488
1.25772387591864 -0.484847237098433
1.26539486430542 -0.484697275874471
1.27050883404033 -0.484596912039359
1.27306581137596 -0.484546599896135
1.2730658113589 -0.484546599895144
1.27305541155906 -0.484320911054072
1.2730238336127 -0.483870450269343
1.27294421772612 -0.483198810692961
1.27277401858219 -0.482315683800082
1.27245616754417 -0.481242223159833
1.27192174548361 -0.480017905211023
1.27109518948705 -0.478707989424014
1.26987791322201 -0.477382389353572
1.26861868448422 -0.476491262912181
1.26741310591075 -0.475930505649088
1.26634153930566 -0.475618318765773
1.26545420209816 -0.475475522024619
1.26477827798535 -0.475433459898834
1.26432554654506 -0.475438002031128
1.26409923112047 -0.475451102493
1.26409923107294 -0.475451102492838
1.26400510754809 -0.472893953643021
1.26381676720545 -0.467779675905825
1.26353397696839 -0.460108319156843
1.26315636493238 -0.449879963098326
1.26268342195054 -0.437094717113806
1.26211450389605 -0.421752720062896
1.26144883460045 -0.403854140014764
1.2606855094668 -0.383399173918339
1.25991927860775 -0.362944839063709
1.25915016511684 -0.342491133403313
1.25837819207707 -0.322038054872471
1.25760338256082 -0.301585601389565
1.25682575962964 -0.281133770856217
1.25604534633415 -0.260682561157473
1.25526216571386 -0.240231970161977
1.25447624079705 -0.219781995722154
1.25368759460061 -0.199332635674387
1.25289625012991 -0.178883887839198
1.25210223037863 -0.158435750021426
1.25130555832866 -0.137988220010405
1.25050625694995 -0.117541295580143
1.24970434920032 -0.0970949744895014
1.2488998580254 -0.0766492544823687
1.24809280635845 -0.0562041332878428
1.24728321712021 -0.0357596086204065
1.24647111321881 -0.0153156781801049
1.24565651754959 0.00512766034727754
1.244839452995 0.0255704092900399
1.24401994242446 0.0460125709903884
1.24319800869422 0.0664541478042603
1.24237367464724 0.0868951421011472
1.24154696311305 0.10733555626392
1.24071789690763 0.127775392688651
1.23988649883331 0.148214653784443
1.23905279167858 0.168653341973249
1.23821679821801 0.189091459689702
1.23737854121215 0.209529009380936
1.23653804340733 0.229965993506414
1.23569532753562 0.250402414537754
1.23485041631467 0.270838274958555
1.23400333244757 0.291273577264223
1.23315409862277 0.311708323961797
1.23230273751397 0.332142517569777
1.23144927177993 0.352576160617952
1.23059372406446 0.373009255647224
1.22973611699622 0.393441805209441
1.22887647318863 0.413873811867219
1.22801481523979 0.434305278193775
1.22715116573232 0.454736206772755
1.22628554723329 0.475166600198059
1.22552642790666 0.493042728464227
1.22487449160254 0.508364776283352
1.22433031870982 0.52113290203806
1.22389439056138 0.531347238274115
1.22356709317291 0.539007892120091
1.22334872031756 0.544114945632642
1.22323947593845 0.546668456066198
1.22323947590048 0.546668456066214
1.2230123362303 0.54664910964162
1.22255893634353 0.546599663012994
1.22188286957468 0.546493423413052
1.22099418533407 0.546288449972454
1.21991546682544 0.545929219980312
1.21868995460871 0.545349990456186
1.21739061938076 0.544480715471633
1.21610030698399 0.543230273381219
1.21526412455335 0.541961848316668
1.21476024252823 0.540762508740427
1.21448767411105 0.539705040879324
1.21435620536168 0.538833702471361
1.21429925392167 0.538171836563276
1.21427642250785 0.537729171531928
1.21429899313131 0.537497396084245
1.21429899310163 0.53749739608416
1.21173554316399 0.537351398434274
1.2066086675668 0.537059357198718
1.19891842689941 0.536621157703719
1.18866491799536 0.536036617118684
1.17584827376684 0.535305485487585
1.16046866297006 0.534427447211652
1.1425262899005 0.533402122984755
1.12202139401749 0.532229072183377
1.10151726555175 0.531054604210161
1.08101390213675 0.529878734481104
1.06051130139776 0.528701478429746
1.04000946095186 0.5275228515073
1.01950837840803 0.526342869182777
0.999008051367172 0.525161546943109
0.978508477422155 0.523978900293278
0.958009654157879 0.522794944756435
0.937511579151311 0.521609695874025
0.917014249971538 0.520423169205916
0.896517664179812 0.519235380330514
0.876021819329601 0.518046344844887
0.855526712966634 0.516856078364893
0.835032342628955 0.515664596525293
0.814538705846972 0.514471914979876
0.794045800143499 0.513278049401579
0.773553623033816 0.512083015482608
0.753062172025712 0.510886828934553
0.732571444619537 0.509689505488512
0.712081438308253 0.508491060895207
0.691592150577485 0.507291510925102
0.671103578905572 0.506090871368521
0.650615720763616 0.504889158035764
0.630128573615536 0.503686386757223
0.609642134918118 0.502482573383502
0.589156402121068 0.501277733785526
0.568671372667066 0.500071883854662
0.548187043991813 0.49886503950283
0.527703413524089 0.497657216662618
0.507220478685805 0.496448431287396
0.486738236892053 0.495238699351429
0.466256685551165 0.494028036849988
0.445775822064762 0.492816459799465
0.425295643827809 0.491603984237482
0.404816148228671 0.490390626223003
0.384337332649168 0.489176401836445
0.363859194464625 0.487961327179788
0.343381731043935 0.486745418376685
0.322904939749608 0.48552869157257
0.302428817937827 0.484311162934768
0.281953362958509 0.483092848652601
0.261478572155355 0.4818737649375
0.243563709029771 0.480806407637357
0.228208544470985 0.479891046434884
0.215412881797103 0.479127906813159
0.20517655624396 0.4785171732374
0.197499434529847 0.478058991869279
0.192381414497009 0.477753472810477
0.189822424830656 0.477600691872751
0.189822424856044 0.477600691872319
0.18984329041129 0.477361217804826
0.189891281488213 0.476888179768742
0.18998422190601 0.476195304214649
0.190156086957441 0.475302243681175
0.19046107666499 0.47423395694596
0.190973511005364 0.47302370084793
0.1917804370421 0.471720939032194
0.192990859960637 0.470375213510415
0.194255394688102 0.469435488118
0.195465103825654 0.468815206001496
0.196534139830494 0.468445902524195
0.19741290176854 0.468255645457359
0.19807765083531 0.468178447005524
0.198520413044486 0.468160326901788
0.198740901208049 0.468162462817004
0.198740901269321 0.46816246281638
0.198945440782965 0.465612270025241
0.199354508471781 0.460511864695189
0.199968076306772 0.452861197525416
0.200786100039452 0.442660189857331
0.201808520127061 0.429908733995519
0.203035263052485 0.414606693655117
0.20446624304152 0.396753904535665
0.206101364180888 0.376350175021588
0.207736171974856 0.355945830121864
0.209370679915717 0.335540874152152
0.211004901527408 0.315135311429408
0.212638850365535 0.294729146271967
0.214272540017396 0.27432238299961
0.215905984101993 0.253915025933645
0.217539196270065 0.233507079396979
0.219172190204101 0.2130985477142
0.220804979618368 0.192689435211644
0.222437578258934 0.172279746217481
0.224069999903693 0.15186948506179
0.225702258362386 0.131458656076634
0.227334367476629 0.111047263596139
0.228966341119943 0.0906353119565765
0.230598193197773 0.0702228054964383
0.232229937647521 0.0498097485565179
0.233861588438572 0.0293961454799898
0.235493159572323 0.00898200061249
0.237124665082211 -0.0114326816978031
0.238756119033746 -0.0318478971000879
0.240387535524538 -0.0522636412408578
0.24201892868433 -0.0726799097638194
0.243650312675029 -0.0930966983098097
0.245281701690739 -0.113514002516714
0.246913109957795 -0.133931818019382
0.248544551734793 -0.154350140449545
0.250176041312631 -0.174768965435729
0.251807593014535 -0.195188288603176
0.253439221196103 -0.215608105573753
0.255070940245335 -0.236028411965869
0.256702764582677 -0.256449203394391
0.258334708661049 -0.276870475470555
0.259966786965892 -0.29729222380188
0.261599014015202 -0.317714443992083
0.263231404359571 -0.338137131640988
0.264863972582227 -0.358560282344441
0.266496733299076 -0.378983891694222
0.26812970115874 -0.399407955277952
0.269762890842605 -0.419832468679009
0.271396317064857 -0.440257427476434
0.27302999457253 -0.460682827244845
0.27466393814555 -0.481108663554342
0.276093884549649 -0.498981648418472
0.277319746923832 -0.514301630525306
0.278341443563315 -0.527068481124652
0.279158900790641 -0.537282093189924
0.279772055404735 -0.544942380708266
0.280180856703051 -0.550049278099581
0.280385268072754 -0.552602739765039
};
\addplot [semithick, green, dash pattern=on 1pt off 3pt on 3pt off 3pt, forget plot]
table {%
0.25 -0.5
0.2525 -0.5
0.2575 -0.5
0.265 -0.5
0.275 -0.5
0.2875 -0.5
0.3025 -0.5
0.32 -0.5
0.34 -0.5
0.36 -0.5
0.38 -0.5
0.4 -0.5
0.42 -0.5
0.44 -0.5
0.46 -0.5
0.48 -0.5
0.5 -0.5
0.52 -0.5
0.54 -0.5
0.56 -0.5
0.58 -0.5
0.6 -0.5
0.62 -0.5
0.64 -0.5
0.66 -0.5
0.68 -0.5
0.7 -0.5
0.72 -0.5
0.74 -0.5
0.76 -0.5
0.78 -0.5
0.8 -0.5
0.82 -0.5
0.84 -0.5
0.86 -0.5
0.88 -0.5
0.9 -0.5
0.92 -0.5
0.94 -0.5
0.96 -0.5
0.98 -0.5
1 -0.5
1.02 -0.5
1.04 -0.5
1.06 -0.5
1.08 -0.5
1.1 -0.5
1.12 -0.5
1.14 -0.5
1.16 -0.5
1.18 -0.5
1.1975 -0.5
1.2125 -0.5
1.225 -0.5
1.235 -0.5
1.2425 -0.5
1.2475 -0.5
1.25 -0.5
1.25 -0.5
1.25 -0.5
1.25 -0.5
1.25 -0.5
1.25 -0.5
1.25 -0.5
1.25 -0.5
1.25 -0.5
1.25 -0.5
1.25 -0.5
1.25 -0.5
1.25 -0.5
1.25 -0.5
1.25 -0.5
1.25 -0.5
1.25 -0.5
1.25 -0.5
1.25 -0.4975
1.25 -0.4925
1.25 -0.485
1.25 -0.475
1.25 -0.4625
1.25 -0.4475
1.25 -0.43
1.25 -0.41
1.25 -0.39
1.25 -0.37
1.25 -0.35
1.25 -0.33
1.25 -0.31
1.25 -0.29
1.25 -0.27
1.25 -0.25
1.25 -0.23
1.25 -0.21
1.25 -0.19
1.25 -0.17
1.25 -0.15
1.25 -0.13
1.25 -0.11
1.25 -0.09
1.25 -0.07
1.25 -0.05
1.25 -0.03
1.25 -0.01
1.25 0.00999999999999998
1.25 0.03
1.25 0.05
1.25 0.07
1.25 0.09
1.25 0.11
1.25 0.13
1.25 0.15
1.25 0.17
1.25 0.19
1.25 0.21
1.25 0.23
1.25 0.25
1.25 0.27
1.25 0.29
1.25 0.31
1.25 0.33
1.25 0.35
1.25 0.37
1.25 0.39
1.25 0.41
1.25 0.43
1.25 0.4475
1.25 0.4625
1.25 0.475
1.25 0.485
1.25 0.4925
1.25 0.4975
1.25 0.5
1.25 0.5
1.25 0.5
1.25 0.5
1.25 0.5
1.25 0.5
1.25 0.5
1.25 0.5
1.25 0.5
1.25 0.5
1.25 0.5
1.25 0.5
1.25 0.5
1.25 0.5
1.25 0.5
1.25 0.5
1.25 0.5
1.25 0.5
1.2475 0.5
1.2425 0.5
1.235 0.5
1.225 0.5
1.2125 0.5
1.1975 0.5
1.18 0.5
1.16 0.5
1.14 0.5
1.12 0.5
1.1 0.5
1.08 0.5
1.06 0.5
1.04 0.5
1.02 0.5
1 0.5
0.98 0.5
0.96 0.5
0.94 0.5
0.92 0.5
0.9 0.5
0.88 0.5
0.86 0.5
0.84 0.5
0.82 0.5
0.8 0.5
0.78 0.5
0.76 0.5
0.74 0.5
0.72 0.5
0.7 0.5
0.68 0.5
0.66 0.5
0.64 0.5
0.62 0.5
0.6 0.5
0.58 0.5
0.56 0.5
0.54 0.5
0.52 0.5
0.5 0.5
0.48 0.5
0.46 0.5
0.44 0.5
0.42 0.5
0.4 0.5
0.38 0.5
0.36 0.5
0.34 0.5
0.32 0.5
0.3025 0.5
0.2875 0.5
0.275 0.5
0.265 0.5
0.2575 0.5
0.2525 0.5
0.25 0.5
0.25 0.5
0.25 0.5
0.25 0.5
0.25 0.5
0.25 0.5
0.25 0.5
0.25 0.5
0.25 0.5
0.25 0.5
0.25 0.5
0.25 0.5
0.25 0.5
0.25 0.5
0.25 0.5
0.25 0.5
0.25 0.5
0.25 0.5
0.25 0.4975
0.25 0.4925
0.25 0.485
0.25 0.475
0.25 0.4625
0.25 0.4475
0.25 0.43
0.25 0.41
0.25 0.39
0.25 0.37
0.25 0.35
0.25 0.33
0.25 0.31
0.25 0.29
0.25 0.27
0.25 0.25
0.25 0.23
0.25 0.21
0.25 0.19
0.25 0.17
0.25 0.15
0.249999999999999 0.13
0.249999999999999 0.11
0.249999999999999 0.09
0.249999999999999 0.07
0.249999999999999 0.05
0.249999999999999 0.03
0.249999999999999 0.01
0.249999999999999 -0.00999999999999998
0.249999999999999 -0.03
0.249999999999999 -0.05
0.249999999999999 -0.07
0.249999999999999 -0.09
0.249999999999999 -0.11
0.249999999999999 -0.13
0.249999999999999 -0.15
0.249999999999999 -0.17
0.249999999999999 -0.19
0.249999999999999 -0.21
0.249999999999999 -0.23
0.249999999999999 -0.25
0.249999999999999 -0.27
0.249999999999999 -0.29
0.249999999999999 -0.31
0.249999999999999 -0.33
0.249999999999999 -0.35
0.249999999999999 -0.37
0.249999999999999 -0.39
0.249999999999999 -0.41
0.249999999999999 -0.43
0.249999999999999 -0.4475
0.249999999999999 -0.4625
0.249999999999999 -0.475
0.249999999999999 -0.485
0.249999999999999 -0.4925
0.249999999999999 -0.4975
0.249999999999999 -0.5
};
\addplot [semithick, blue, opacity=\opacityRef, forget plot]
table {%
0.25 -0.5
0.250405013561249 -0.500023066997528
0.254420608282089 -0.499927639961243
0.261954605579376 -0.499700367450714
0.271322548389435 -0.499603152275085
0.282460540533066 -0.499481558799744
0.297058910131454 -0.498842775821686
0.316285997629166 -0.498398542404175
0.334478825330734 -0.498049199581146
0.355755716562271 -0.497657835483551
0.375745445489883 -0.49736687541008
0.396663725376129 -0.496834069490433
0.419246792793274 -0.496127396821976
0.437622845172882 -0.495314061641693
0.458408296108246 -0.495179653167725
0.478865176439285 -0.494543254375458
0.501222372055054 -0.494086354970932
0.519940137863159 -0.493625670671463
0.540383517742157 -0.493088722229004
0.561086118221283 -0.492678701877594
0.581511080265045 -0.492289066314697
0.602226436138153 -0.491940975189209
0.622854828834534 -0.491407513618469
0.643664002418518 -0.490944623947144
0.66401344537735 -0.49043020606041
0.684694647789001 -0.489972025156021
0.707064211368561 -0.489382743835449
0.725705325603485 -0.488891750574112
0.746301651000977 -0.488492369651794
0.766418159008026 -0.487979143857956
0.787356913089752 -0.4874227643013
0.807545244693756 -0.487110167741776
0.82855612039566 -0.486719459295273
0.848987877368927 -0.485918939113617
0.869812726974487 -0.485574781894684
0.890043914318085 -0.485114067792892
0.912867248058319 -0.484339028596878
0.931557953357697 -0.484092086553574
0.951545536518097 -0.4835564494133
0.974116325378418 -0.483134359121323
0.992350518703461 -0.482499897480011
1.01332145929337 -0.481981098651886
1.03366833925247 -0.480967164039612
1.05418509244919 -0.481074720621109
1.07479411363602 -0.48045739531517
1.0954664349556 -0.480043709278107
1.1182364821434 -0.479491651058197
1.13900703191757 -0.478790313005447
1.15717619657516 -0.478797435760498
1.17761224508286 -0.478190630674362
1.1981583237648 -0.477915912866592
1.21720153093338 -0.477585047483444
1.23348838090897 -0.477313905954361
1.2465438246727 -0.476924866437912
1.25846070051193 -0.476816743612289
1.26622003316879 -0.476943820714951
1.27166253328323 -0.476728081703186
1.27464681863785 -0.476845979690552
1.27546983957291 -0.476686179637909
1.27591186761856 -0.476627856492996
1.2768822312355 -0.476442366838455
1.27680283784866 -0.476370275020599
1.27671879529953 -0.476382374763489
1.27683717012405 -0.476087003946304
1.27680593729019 -0.475720077753067
1.27649861574173 -0.475638657808304
1.27644318342209 -0.475566148757935
1.27615684270859 -0.475615262985229
1.27596670389175 -0.47555935382843
1.27581161260605 -0.475869685411453
1.27564007043839 -0.475975215435028
1.27559059858322 -0.476120859384537
1.27551192045212 -0.476067781448364
1.27554172277451 -0.476144731044769
1.27558392286301 -0.476221174001694
1.27547627687454 -0.475810080766678
1.27536183595657 -0.472586363554001
1.27502292394638 -0.464759021997452
1.27455395460129 -0.4556705057621
1.27396482229233 -0.442953795194626
1.27314525842667 -0.426750421524048
1.27223283052444 -0.410802632570267
1.27126330137253 -0.391420930624008
1.27029579877853 -0.370241850614548
1.26928633451462 -0.347978919744492
1.26843076944351 -0.328938275575638
1.26756674051285 -0.309010058641434
1.26659160852432 -0.288347631692886
1.26570862531662 -0.267679899930954
1.26477247476578 -0.246971353888512
1.26408964395523 -0.226082220673561
1.26324981451035 -0.206023097038269
1.26214402914047 -0.185177132487297
1.26121860742569 -0.164786830544472
1.26029020547867 -0.144162848591805
1.25933033227921 -0.123692139983177
1.25838547945023 -0.101642206311226
1.25758105516434 -0.0831583365797997
1.25654619932175 -0.0631818771362305
1.25552183389664 -0.0417923554778099
1.25463742017746 -0.0212850943207741
1.25359803438187 -0.000822959234938025
1.25246030092239 0.0189864318817854
1.25111120939255 0.0420508570969105
1.25011938810349 0.0605214908719063
1.24904185533524 0.0812976732850075
1.247986972332 0.104082264006138
1.24716752767563 0.122631594538689
1.24613291025162 0.143185287714005
1.24482387304306 0.163244932889938
1.24395591020584 0.183842197060585
1.24289447069168 0.204689085483551
1.24185329675674 0.224708020687103
1.24071699380875 0.245794057846069
1.2396132349968 0.266452729701996
1.23811095952988 0.286722958087921
1.23748058080673 0.307469993829727
1.23635345697403 0.329949349164963
1.23519784212112 0.350634634494781
1.2340789437294 0.370582580566406
1.23304969072342 0.391537815332413
1.23193484544754 0.41190892457962
1.23111754655838 0.430507689714432
1.22994560003281 0.45110684633255
1.22907584905624 0.471523255109787
1.22708636522293 0.492044359445572
1.22700697183609 0.50619900226593
1.22598630189896 0.521031737327576
1.22534209489822 0.534884750843048
1.22524267435074 0.542483389377594
1.22426110506058 0.547447144985199
1.22446805238724 0.551150321960449
1.22444504499435 0.552067399024963
1.22432273626328 0.555235505104065
1.22416716814041 0.55544114112854
1.22419041395187 0.555162668228149
1.22409063577652 0.555026948451996
1.2236060500145 0.555117666721344
1.22393995523453 0.554936945438385
1.22368687391281 0.554542601108551
1.22358471155167 0.55460798740387
1.22407311201096 0.554387032985687
1.2236995100975 0.554421305656433
1.22314816713333 0.554179906845093
1.22307592630386 0.554120779037476
1.22330349683762 0.554040372371674
1.22338539361954 0.554027259349823
1.22333782911301 0.553985714912415
1.22357279062271 0.553953349590302
1.22287565469742 0.553925096988678
1.21927720308304 0.553742647171021
1.21236389875412 0.55318146944046
1.20291417837143 0.552609622478485
1.19085973501205 0.55180561542511
1.17620819807053 0.550891160964966
1.15878301858902 0.549797415733337
1.13704079389572 0.548290431499481
1.11754280328751 0.547088980674744
1.09732586145401 0.545969903469086
1.07701307535172 0.544615030288696
1.05669206380844 0.543317139148712
1.03585749864578 0.541944205760956
1.01468902826309 0.540464997291565
0.995077073574066 0.539143085479736
0.974256277084351 0.537813246250153
0.954100370407104 0.536531329154968
0.93382340669632 0.535237610340118
0.913250088691711 0.533855199813843
0.893536031246185 0.532512009143829
0.872769236564636 0.530993700027466
0.852360665798187 0.529611885547638
0.831867694854736 0.528278350830078
0.811572730541229 0.526968240737915
0.791047930717468 0.525576055049896
0.770317912101746 0.524270117282867
0.749398648738861 0.522916674613953
0.72893887758255 0.52166610956192
0.708939731121063 0.520600497722626
0.688148200511932 0.519164264202118
0.6681067943573 0.51801210641861
0.647082388401031 0.516603052616119
0.624525129795074 0.51519501209259
0.606439590454102 0.514194846153259
0.585701048374176 0.512824594974518
0.565021574497223 0.511545956134796
0.54415899515152 0.510084867477417
0.524090170860291 0.508989214897156
0.503620505332947 0.507618308067322
0.483295679092407 0.506556451320648
0.462621539831161 0.505100011825562
0.442002505064011 0.503693342208862
0.421523720026016 0.502368122339249
0.401281207799911 0.501079618930817
0.380549430847168 0.499803572893143
0.359789431095123 0.498413413763046
0.338726878166199 0.497045606374741
0.318436980247498 0.49587869644165
0.297567784786224 0.494448274374008
0.276962727308273 0.493222385644913
0.257530212402344 0.491707891225815
0.241846635937691 0.490875422954559
0.228206425905228 0.48963937163353
0.217488780617714 0.489216804504395
0.20884570479393 0.488132476806641
0.203275829553604 0.488158941268921
0.200075998902321 0.487814694643021
0.199211493134499 0.487732082605362
0.198726534843445 0.487664103507996
0.197370693087578 0.487481296062469
0.197795391082764 0.487543135881424
0.197431519627571 0.487424790859222
0.197262421250343 0.487189263105392
0.197662368416786 0.487110584974289
0.198174178600311 0.487399160861969
0.198350459337234 0.487475663423538
0.198462158441544 0.487362325191498
0.198548391461372 0.487262845039368
0.198916703462601 0.487906157970428
0.199057146906853 0.487703651189804
0.199148952960968 0.487830400466919
0.199225768446922 0.487980008125305
0.199218899011612 0.487969249486923
0.199206829071045 0.488063454627991
0.199307158589363 0.48768675327301
0.199550181627274 0.483470439910889
0.200021088123322 0.4766506254673
0.200907617807388 0.466315239667892
0.201578423380852 0.455271601676941
0.202570885419846 0.440516442060471
0.204002141952515 0.421742230653763
0.205581068992615 0.401713848114014
0.207086831331253 0.382670938968658
0.208537742495537 0.362453371286392
0.210437342524529 0.339809119701385
0.212022885680199 0.319703012704849
0.213412001729012 0.29923877120018
0.21537147462368 0.280386477708817
0.217275828123093 0.258447259664536
0.218808904290199 0.239829421043396
0.220663800835609 0.216891899704933
0.221976563334465 0.198811590671539
0.224077194929123 0.17542465031147
0.225460052490234 0.155136376619339
0.227088928222656 0.135767698287964
0.228398457169533 0.114172510802746
0.230142191052437 0.0953218415379524
0.231698170304298 0.07463738322258
0.233631804585457 0.0542909018695354
0.235106691718102 0.0343251116573811
0.236794054508209 0.0134788732975721
0.238105833530426 -0.00615298561751842
0.239772662520409 -0.0272009447216988
0.241917118430138 -0.0507972054183483
0.243190854787827 -0.070477694272995
0.244959503412247 -0.0911861211061478
0.24654883146286 -0.111342132091522
0.248328655958176 -0.132207557559013
0.249510735273361 -0.150453552603722
0.25169113278389 -0.172697201371193
0.253420531749725 -0.191114336252213
0.255062878131866 -0.212059661746025
0.256706207990646 -0.234564825892448
0.25817134976387 -0.253149628639221
0.259933918714523 -0.27397957444191
0.261791676282883 -0.294694840908051
0.263263076543808 -0.314810812473297
0.265123933553696 -0.335502237081528
0.266619056463242 -0.355856597423553
0.268120914697647 -0.376241475343704
0.270071744918823 -0.39642745256424
0.271701574325562 -0.416890382766724
0.273327261209488 -0.438217610120773
0.274704366922379 -0.458396553993225
0.276198595762253 -0.477949172258377
0.277414828538895 -0.493446916341782
0.278549194335938 -0.507106184959412
0.279365390539169 -0.518185138702393
0.280031770467758 -0.526600778102875
0.280489236116409 -0.532608568668365
0.280815124511719 -0.535681486129761
0.280628949403763 -0.536182045936584
};
\addplot [semithick, red, dashed, forget plot]
table {%
0.25 -0.5
0.252558268640609 -0.499971215983182
0.257674800204547 -0.499913561943335
0.265349580546156 -0.499826822527541
0.275582586992214 -0.499710651911554
0.288373788282223 -0.499564571828345
0.303723144483551 -0.499387968751125
0.321630606880243 -0.499180090231924
0.342096117833915 -0.498940040397106
0.362561444756502 -0.498697137467738
0.383026586783779 -0.498451352589885
0.403491543037366 -0.498202656922469
0.423956312624558 -0.49795102163749
0.444420894638161 -0.497696417920264
0.464885288156327 -0.497438816969653
0.485349492242387 -0.497178189998295
0.505813505944689 -0.496914508232844
0.526277328296427 -0.4966477429142
0.546740958315482 -0.49637786529775
0.567204395004253 -0.496104846653605
0.587667637349495 -0.495828658266839
0.608130684322154 -0.495549271437727
0.628593534877201 -0.495266657481993
0.649056187953474 -0.494980787731044
0.669518642473505 -0.494691633532224
0.689980897343365 -0.49439916624905
0.710442951452497 -0.494103357261464
0.730904803673553 -0.493804177966079
0.751366452862231 -0.493501599776428
0.771827897857116 -0.493195594123215
0.792289137479511 -0.492886132454562
0.812750170533283 -0.492573186236266
0.833210995804693 -0.492256726952051
0.853671612062244 -0.491936726103822
0.874132018056508 -0.491613155211922
0.894592212519978 -0.491285985815386
0.915052194166895 -0.490955189472205
0.935511961693098 -0.49062073775958
0.955971513775856 -0.490282602274183
0.976430849073713 -0.489940754632423
0.996889966226327 -0.489595166470704
1.01734886385431 -0.489245809445691
1.03780754055907 -0.488892655234577
1.05826599492265 -0.488535675535346
1.07872422550757 -0.488174842067041
1.09918223085668 -0.487810126570036
1.11964000949299 -0.487441500806302
1.1400975599195 -0.48706893655968
1.16055488061908 -0.486692405636151
1.18101197005428 -0.486311879864113
1.20146882666719 -0.485927331094652
1.21936837110133 -0.485587306188302
1.23471068292331 -0.485293179242986
1.24749583154271 -0.485046149752488
1.25772387591864 -0.484847237098433
1.26539486430542 -0.484697275874471
1.27050883404033 -0.484596912039359
1.27306581137596 -0.484546599896135
1.2730658113589 -0.484546599895144
1.27305541155906 -0.484320911054072
1.2730238336127 -0.483870450269343
1.27294421772612 -0.483198810692961
1.27277401858219 -0.482315683800082
1.27245616754417 -0.481242223159833
1.27192174548361 -0.480017905211023
1.27109518948705 -0.478707989424014
1.26987791322201 -0.477382389353572
1.26861868448422 -0.476491262912181
1.26741310591075 -0.475930505649088
1.26634153930566 -0.475618318765773
1.26545420209816 -0.475475522024619
1.26477827798535 -0.475433459898834
1.26432554654506 -0.475438002031128
1.26409923112047 -0.475451102493
1.26409923107294 -0.475451102492838
1.26400510754809 -0.472893953643021
1.26381676720545 -0.467779675905825
1.26353397696839 -0.460108319156843
1.26315636493238 -0.449879963098326
1.26268342195054 -0.437094717113806
1.26211450389605 -0.421752720062896
1.26144883460045 -0.403854140014764
1.2606855094668 -0.383399173918339
1.25991927860775 -0.362944839063709
1.25915016511684 -0.342491133403313
1.25837819207707 -0.322038054872471
1.25760338256082 -0.301585601389565
1.25682575962964 -0.281133770856217
1.25604534633415 -0.260682561157473
1.25526216571386 -0.240231970161977
1.25447624079705 -0.219781995722154
1.25368759460061 -0.199332635674387
1.25289625012991 -0.178883887839198
1.25210223037863 -0.158435750021426
1.25130555832866 -0.137988220010405
1.25050625694995 -0.117541295580143
1.24970434920032 -0.0970949744895014
1.2488998580254 -0.0766492544823687
1.24809280635845 -0.0562041332878428
1.24728321712021 -0.0357596086204065
1.24647111321881 -0.0153156781801049
1.24565651754959 0.00512766034727754
1.244839452995 0.0255704092900399
1.24401994242446 0.0460125709903884
1.24319800869422 0.0664541478042603
1.24237367464724 0.0868951421011472
1.24154696311305 0.10733555626392
1.24071789690763 0.127775392688651
1.23988649883331 0.148214653784443
1.23905279167858 0.168653341973249
1.23821679821801 0.189091459689702
1.23737854121215 0.209529009380936
1.23653804340733 0.229965993506414
1.23569532753562 0.250402414537754
1.23485041631467 0.270838274958555
1.23400333244757 0.291273577264223
1.23315409862277 0.311708323961797
1.23230273751397 0.332142517569777
1.23144927177993 0.352576160617952
1.23059372406446 0.373009255647224
1.22973611699622 0.393441805209441
1.22887647318863 0.413873811867219
1.22801481523979 0.434305278193775
1.22715116573232 0.454736206772755
1.22628554723329 0.475166600198059
1.22552642790666 0.493042728464227
1.22487449160254 0.508364776283352
1.22433031870982 0.52113290203806
1.22389439056138 0.531347238274115
1.22356709317291 0.539007892120091
1.22334872031756 0.544114945632642
1.22323947593845 0.546668456066198
1.22323947590048 0.546668456066214
1.2230123362303 0.54664910964162
1.22255893634353 0.546599663012994
1.22188286957468 0.546493423413052
1.22099418533407 0.546288449972454
1.21991546682544 0.545929219980312
1.21868995460871 0.545349990456186
1.21739061938076 0.544480715471633
1.21610030698399 0.543230273381219
1.21526412455335 0.541961848316668
1.21476024252823 0.540762508740427
1.21448767411105 0.539705040879324
1.21435620536168 0.538833702471361
1.21429925392167 0.538171836563276
1.21427642250785 0.537729171531928
1.21429899313131 0.537497396084245
1.21429899310163 0.53749739608416
1.21173554316399 0.537351398434274
1.2066086675668 0.537059357198718
1.19891842689941 0.536621157703719
1.18866491799536 0.536036617118684
1.17584827376684 0.535305485487585
1.16046866297006 0.534427447211652
1.1425262899005 0.533402122984755
1.12202139401749 0.532229072183377
1.10151726555175 0.531054604210161
1.08101390213675 0.529878734481104
1.06051130139776 0.528701478429746
1.04000946095186 0.5275228515073
1.01950837840803 0.526342869182777
0.999008051367172 0.525161546943109
0.978508477422155 0.523978900293278
0.958009654157879 0.522794944756435
0.937511579151311 0.521609695874025
0.917014249971538 0.520423169205916
0.896517664179812 0.519235380330514
0.876021819329601 0.518046344844887
0.855526712966634 0.516856078364893
0.835032342628955 0.515664596525293
0.814538705846972 0.514471914979876
0.794045800143499 0.513278049401579
0.773553623033816 0.512083015482608
0.753062172025712 0.510886828934553
0.732571444619537 0.509689505488512
0.712081438308253 0.508491060895207
0.691592150577485 0.507291510925102
0.671103578905572 0.506090871368521
0.650615720763616 0.504889158035764
0.630128573615536 0.503686386757223
0.609642134918118 0.502482573383502
0.589156402121068 0.501277733785526
0.568671372667066 0.500071883854662
0.548187043991813 0.49886503950283
0.527703413524089 0.497657216662618
0.507220478685805 0.496448431287396
0.486738236892053 0.495238699351429
0.466256685551165 0.494028036849988
0.445775822064762 0.492816459799465
0.425295643827809 0.491603984237482
0.404816148228671 0.490390626223003
0.384337332649168 0.489176401836445
0.363859194464625 0.487961327179788
0.343381731043935 0.486745418376685
0.322904939749608 0.48552869157257
0.302428817937827 0.484311162934768
0.281953362958509 0.483092848652601
0.261478572155355 0.4818737649375
0.243563709029771 0.480806407637357
0.228208544470985 0.479891046434884
0.215412881797103 0.479127906813159
0.20517655624396 0.4785171732374
0.197499434529847 0.478058991869279
0.192381414497009 0.477753472810477
0.189822424830656 0.477600691872751
0.189822424856044 0.477600691872319
0.18984329041129 0.477361217804826
0.189891281488213 0.476888179768742
0.18998422190601 0.476195304214649
0.190156086957441 0.475302243681175
0.19046107666499 0.47423395694596
0.190973511005364 0.47302370084793
0.1917804370421 0.471720939032194
0.192990859960637 0.470375213510415
0.194255394688102 0.469435488118
0.195465103825654 0.468815206001496
0.196534139830494 0.468445902524195
0.19741290176854 0.468255645457359
0.19807765083531 0.468178447005524
0.198520413044486 0.468160326901788
0.198740901208049 0.468162462817004
0.198740901269321 0.46816246281638
0.198945440782965 0.465612270025241
0.199354508471781 0.460511864695189
0.199968076306772 0.452861197525416
0.200786100039452 0.442660189857331
0.201808520127061 0.429908733995519
0.203035263052485 0.414606693655117
0.20446624304152 0.396753904535665
0.206101364180888 0.376350175021588
0.207736171974856 0.355945830121864
0.209370679915717 0.335540874152152
0.211004901527408 0.315135311429408
0.212638850365535 0.294729146271967
0.214272540017396 0.27432238299961
0.215905984101993 0.253915025933645
0.217539196270065 0.233507079396979
0.219172190204101 0.2130985477142
0.220804979618368 0.192689435211644
0.222437578258934 0.172279746217481
0.224069999903693 0.15186948506179
0.225702258362386 0.131458656076634
0.227334367476629 0.111047263596139
0.228966341119943 0.0906353119565765
0.230598193197773 0.0702228054964383
0.232229937647521 0.0498097485565179
0.233861588438572 0.0293961454799898
0.235493159572323 0.00898200061249
0.237124665082211 -0.0114326816978031
0.238756119033746 -0.0318478971000879
0.240387535524538 -0.0522636412408578
0.24201892868433 -0.0726799097638194
0.243650312675029 -0.0930966983098097
0.245281701690739 -0.113514002516714
0.246913109957795 -0.133931818019382
0.248544551734793 -0.154350140449545
0.250176041312631 -0.174768965435729
0.251807593014535 -0.195188288603176
0.253439221196103 -0.215608105573753
0.255070940245335 -0.236028411965869
0.256702764582677 -0.256449203394391
0.258334708661049 -0.276870475470555
0.259966786965892 -0.29729222380188
0.261599014015202 -0.317714443992083
0.263231404359571 -0.338137131640988
0.264863972582227 -0.358560282344441
0.266496733299076 -0.378983891694222
0.26812970115874 -0.399407955277952
0.269762890842605 -0.419832468679009
0.271396317064857 -0.440257427476434
0.27302999457253 -0.460682827244845
0.27466393814555 -0.481108663554342
0.276093884549649 -0.498981648418472
0.277319746923832 -0.514301630525306
0.278341443563315 -0.527068481124652
0.279158900790641 -0.537282093189924
0.279772055404735 -0.544942380708266
0.280180856703051 -0.550049278099581
0.280385268072754 -0.552602739765039
};
\addplot [semithick, green, dash pattern=on 1pt off 3pt on 3pt off 3pt, forget plot]
table {%
0.25 -0.5
0.2525 -0.5
0.2575 -0.5
0.265 -0.5
0.275 -0.5
0.2875 -0.5
0.3025 -0.5
0.32 -0.5
0.34 -0.5
0.36 -0.5
0.38 -0.5
0.4 -0.5
0.42 -0.5
0.44 -0.5
0.46 -0.5
0.48 -0.5
0.5 -0.5
0.52 -0.5
0.54 -0.5
0.56 -0.5
0.58 -0.5
0.6 -0.5
0.62 -0.5
0.64 -0.5
0.66 -0.5
0.68 -0.5
0.7 -0.5
0.72 -0.5
0.74 -0.5
0.76 -0.5
0.78 -0.5
0.8 -0.5
0.82 -0.5
0.84 -0.5
0.86 -0.5
0.88 -0.5
0.9 -0.5
0.92 -0.5
0.94 -0.5
0.96 -0.5
0.98 -0.5
1 -0.5
1.02 -0.5
1.04 -0.5
1.06 -0.5
1.08 -0.5
1.1 -0.5
1.12 -0.5
1.14 -0.5
1.16 -0.5
1.18 -0.5
1.1975 -0.5
1.2125 -0.5
1.225 -0.5
1.235 -0.5
1.2425 -0.5
1.2475 -0.5
1.25 -0.5
1.25 -0.5
1.25 -0.5
1.25 -0.5
1.25 -0.5
1.25 -0.5
1.25 -0.5
1.25 -0.5
1.25 -0.5
1.25 -0.5
1.25 -0.5
1.25 -0.5
1.25 -0.5
1.25 -0.5
1.25 -0.5
1.25 -0.5
1.25 -0.5
1.25 -0.5
1.25 -0.4975
1.25 -0.4925
1.25 -0.485
1.25 -0.475
1.25 -0.4625
1.25 -0.4475
1.25 -0.43
1.25 -0.41
1.25 -0.39
1.25 -0.37
1.25 -0.35
1.25 -0.33
1.25 -0.31
1.25 -0.29
1.25 -0.27
1.25 -0.25
1.25 -0.23
1.25 -0.21
1.25 -0.19
1.25 -0.17
1.25 -0.15
1.25 -0.13
1.25 -0.11
1.25 -0.09
1.25 -0.07
1.25 -0.05
1.25 -0.03
1.25 -0.01
1.25 0.00999999999999998
1.25 0.03
1.25 0.05
1.25 0.07
1.25 0.09
1.25 0.11
1.25 0.13
1.25 0.15
1.25 0.17
1.25 0.19
1.25 0.21
1.25 0.23
1.25 0.25
1.25 0.27
1.25 0.29
1.25 0.31
1.25 0.33
1.25 0.35
1.25 0.37
1.25 0.39
1.25 0.41
1.25 0.43
1.25 0.4475
1.25 0.4625
1.25 0.475
1.25 0.485
1.25 0.4925
1.25 0.4975
1.25 0.5
1.25 0.5
1.25 0.5
1.25 0.5
1.25 0.5
1.25 0.5
1.25 0.5
1.25 0.5
1.25 0.5
1.25 0.5
1.25 0.5
1.25 0.5
1.25 0.5
1.25 0.5
1.25 0.5
1.25 0.5
1.25 0.5
1.25 0.5
1.2475 0.5
1.2425 0.5
1.235 0.5
1.225 0.5
1.2125 0.5
1.1975 0.5
1.18 0.5
1.16 0.5
1.14 0.5
1.12 0.5
1.1 0.5
1.08 0.5
1.06 0.5
1.04 0.5
1.02 0.5
1 0.5
0.98 0.5
0.96 0.5
0.94 0.5
0.92 0.5
0.9 0.5
0.88 0.5
0.86 0.5
0.84 0.5
0.82 0.5
0.8 0.5
0.78 0.5
0.76 0.5
0.74 0.5
0.72 0.5
0.7 0.5
0.68 0.5
0.66 0.5
0.64 0.5
0.62 0.5
0.6 0.5
0.58 0.5
0.56 0.5
0.54 0.5
0.52 0.5
0.5 0.5
0.48 0.5
0.46 0.5
0.44 0.5
0.42 0.5
0.4 0.5
0.38 0.5
0.36 0.5
0.34 0.5
0.32 0.5
0.3025 0.5
0.2875 0.5
0.275 0.5
0.265 0.5
0.2575 0.5
0.2525 0.5
0.25 0.5
0.25 0.5
0.25 0.5
0.25 0.5
0.25 0.5
0.25 0.5
0.25 0.5
0.25 0.5
0.25 0.5
0.25 0.5
0.25 0.5
0.25 0.5
0.25 0.5
0.25 0.5
0.25 0.5
0.25 0.5
0.25 0.5
0.25 0.5
0.25 0.4975
0.25 0.4925
0.25 0.485
0.25 0.475
0.25 0.4625
0.25 0.4475
0.25 0.43
0.25 0.41
0.25 0.39
0.25 0.37
0.25 0.35
0.25 0.33
0.25 0.31
0.25 0.29
0.25 0.27
0.25 0.25
0.25 0.23
0.25 0.21
0.25 0.19
0.25 0.17
0.25 0.15
0.249999999999999 0.13
0.249999999999999 0.11
0.249999999999999 0.09
0.249999999999999 0.07
0.249999999999999 0.05
0.249999999999999 0.03
0.249999999999999 0.01
0.249999999999999 -0.00999999999999998
0.249999999999999 -0.03
0.249999999999999 -0.05
0.249999999999999 -0.07
0.249999999999999 -0.09
0.249999999999999 -0.11
0.249999999999999 -0.13
0.249999999999999 -0.15
0.249999999999999 -0.17
0.249999999999999 -0.19
0.249999999999999 -0.21
0.249999999999999 -0.23
0.249999999999999 -0.25
0.249999999999999 -0.27
0.249999999999999 -0.29
0.249999999999999 -0.31
0.249999999999999 -0.33
0.249999999999999 -0.35
0.249999999999999 -0.37
0.249999999999999 -0.39
0.249999999999999 -0.41
0.249999999999999 -0.43
0.249999999999999 -0.4475
0.249999999999999 -0.4625
0.249999999999999 -0.475
0.249999999999999 -0.485
0.249999999999999 -0.4925
0.249999999999999 -0.4975
0.249999999999999 -0.5
};
\addplot [semithick, blue, opacity=\opacityRef, forget plot]
table {%
0.25 -0.5
0.250493794679642 -0.500044107437134
0.254497438669205 -0.500073552131653
0.2617467045784 -0.500244200229645
0.271251410245895 -0.500403940677643
0.283750206232071 -0.500654697418213
0.298750668764114 -0.500828146934509
0.316048920154572 -0.501041889190674
0.335848093032837 -0.5013747215271
0.356958389282227 -0.501484334468842
0.377288728952408 -0.501756727695465
0.398212909698486 -0.502101182937622
0.418293118476868 -0.502291202545166
0.439023226499557 -0.502704322338104
0.45981365442276 -0.502954423427582
0.480458408594131 -0.503316938877106
0.500864267349243 -0.503709197044373
0.521413266658783 -0.503917694091797
0.54207980632782 -0.504068911075592
0.562440931797028 -0.504433989524841
0.583510875701904 -0.504666566848755
0.603656113147736 -0.50497305393219
0.624098718166351 -0.505257964134216
0.644565165042877 -0.505567789077759
0.665202617645264 -0.505933701992035
0.68608021736145 -0.506260752677917
0.70647519826889 -0.506764113903046
0.726966679096222 -0.507162690162659
0.747678756713867 -0.507455825805664
0.768226802349091 -0.507739543914795
0.78892058134079 -0.508082568645477
0.809338986873627 -0.50829142332077
0.829634189605713 -0.508603453636169
0.850364625453949 -0.508957743644714
0.870981097221375 -0.509194374084473
0.891472637653351 -0.509177446365356
0.91192626953125 -0.509504377841949
0.93254154920578 -0.509916305541992
0.952906131744385 -0.510626018047333
0.973629593849182 -0.510521709918976
0.994059443473816 -0.510686457157135
1.0144208073616 -0.510975301265717
1.03497940301895 -0.511101603507996
1.05535632371902 -0.511349022388458
1.07624417543411 -0.511491656303406
1.09654766321182 -0.511735379695892
1.11699539422989 -0.512184739112854
1.13769668340683 -0.512677133083344
1.15839725732803 -0.513203620910645
1.1791198849678 -0.513587534427643
1.19969457387924 -0.513695359230042
1.21887093782425 -0.514069616794586
1.2343253493309 -0.514293909072876
1.2478124499321 -0.514506280422211
1.25835329294205 -0.514796495437622
1.26656883955002 -0.515105128288269
1.27208560705185 -0.515057146549225
1.2749360203743 -0.515085697174072
1.27531772851944 -0.515131294727325
1.27585762739182 -0.515028417110443
1.277084171772 -0.514778017997742
1.27660423517227 -0.514675199985504
1.27643471956253 -0.5145223736763
1.27660065889359 -0.514375627040863
1.27647870779037 -0.514101028442383
1.27631622552872 -0.51408851146698
1.27616971731186 -0.513923823833466
1.27584832906723 -0.514256596565247
1.27562922239304 -0.514087080955505
1.2754448056221 -0.513987183570862
1.27524763345718 -0.514329791069031
1.27523022890091 -0.514289081096649
1.27522963285446 -0.514034390449524
1.27507132291794 -0.514133036136627
1.27517825365067 -0.51422518491745
1.27518600225449 -0.513830006122589
1.27516132593155 -0.510152220726013
1.27510505914688 -0.502259373664856
1.27506858110428 -0.492513388395309
1.27505999803543 -0.480252653360367
1.27516263723373 -0.465642035007477
1.27517992258072 -0.448350310325623
1.27506893873215 -0.428025960922241
1.27514463663101 -0.406988322734833
1.27528578042984 -0.387241393327713
1.27532202005386 -0.366342604160309
1.27541214227676 -0.345595389604568
1.27521997690201 -0.325232803821564
1.27522903680801 -0.304706007242203
1.27549201250076 -0.283950656652451
1.27559167146683 -0.263469517230988
1.27569276094437 -0.243383571505547
1.27583771944046 -0.222339481115341
1.27606493234634 -0.201770097017288
1.27597838640213 -0.181585967540741
1.27612382173538 -0.161198168992996
1.27629953622818 -0.140343770384789
1.27646142244339 -0.119400396943092
1.27632111310959 -0.099356472492218
1.27638465166092 -0.0786500722169876
1.27640372514725 -0.0579799152910709
1.27629488706589 -0.037228174507618
1.27650278806686 -0.0167531203478575
1.2764955163002 0.00316412304528058
1.27668780088425 0.0239723194390535
1.27658766508102 0.0447618588805199
1.27661913633347 0.0651076585054398
1.27676385641098 0.0853714421391487
1.27685052156448 0.105887584388256
1.27642422914505 0.12651738524437
1.27636152505875 0.14760048687458
1.27629047632217 0.168318003416061
1.27635079622269 0.188960671424866
1.27616959810257 0.209060996770859
1.27604609727859 0.229280605912209
1.27576607465744 0.250416964292526
1.27562218904495 0.271012127399445
1.2752565741539 0.291207849979401
1.27560323476791 0.311829447746277
1.27549535036087 0.332646906375885
1.2751926779747 0.353123962879181
1.27528291940689 0.373931646347046
1.27489751577377 0.394204258918762
1.27490311861038 0.415003180503845
1.27493196725845 0.435292571783066
1.27485460042953 0.454333156347275
1.27545636892319 0.470082193613052
1.27536994218826 0.483378797769547
1.27511996030807 0.493712782859802
1.2747568488121 0.501792073249817
1.27510207891464 0.507584810256958
1.27474921941757 0.510246217250824
1.2748606801033 0.510692417621613
1.27503246068954 0.511300325393677
1.27460306882858 0.512200593948364
1.27422672510147 0.512048482894897
1.27470046281815 0.512050330638885
1.27432030439377 0.511867463588715
1.27449136972427 0.511883795261383
1.27418440580368 0.511524319648743
1.27435165643692 0.51144677400589
1.2745069861412 0.511546015739441
1.2737004160881 0.5118288397789
1.27390497922897 0.511912226676941
1.2739571928978 0.511581182479858
1.27415210008621 0.511477530002594
1.27410215139389 0.511308670043945
1.27401131391525 0.511312425136566
1.2739126086235 0.511375606060028
1.27341169118881 0.511363983154297
1.26964110136032 0.511172354221344
1.26193600893021 0.511080205440521
1.252221763134 0.51089745759964
1.23943656682968 0.510741472244263
1.22493809461594 0.510726749897003
1.20741778612137 0.510480940341949
1.18737834692001 0.510023891925812
1.1661234498024 0.50970447063446
1.14590948820114 0.509423434734344
1.12545078992844 0.509256660938263
1.10460644960403 0.509099006652832
1.08517128229141 0.50871467590332
1.06401818990707 0.508441865444183
1.04388278722763 0.50822377204895
1.02352613210678 0.5076904296875
1.00330942869186 0.507523655891418
0.982425808906555 0.507315456867218
0.961390972137451 0.506963193416595
0.941540777683258 0.506490409374237
0.920495748519897 0.506212532520294
0.900856912136078 0.505857110023499
0.880062878131866 0.505636274814606
0.859566867351532 0.505179166793823
0.839065492153168 0.504863739013672
0.818741381168365 0.504488825798035
0.798568665981293 0.504047930240631
0.777900397777557 0.50381988286972
0.757187724113464 0.503541767597198
0.736325919628143 0.503157913684845
0.716116964817047 0.502903699874878
0.695210933685303 0.502642452716827
0.675115466117859 0.502345860004425
0.654126524925232 0.502123415470123
0.633657455444336 0.502008318901062
0.613304674625397 0.501662760972977
0.593111217021942 0.501452893018723
0.572306036949158 0.501102119684219
0.551566123962402 0.5009406208992
0.531083285808563 0.500759482383728
0.510742127895355 0.50057253241539
0.489738166332245 0.500271409749985
0.469378620386124 0.49999076128006
0.448416382074356 0.499597758054733
0.427897989749908 0.499505579471588
0.407934039831161 0.499157845973969
0.387248575687408 0.498878329992294
0.366516381502151 0.498573035001755
0.345468074083328 0.498171210289001
0.324986398220062 0.498058080673218
0.30423891544342 0.497800320386887
0.284980833530426 0.497650742530823
0.272079050540924 0.497265607118607
0.261940181255341 0.497331827878952
0.249212831258774 0.49669685959816
0.245514765381813 0.496901452541351
0.243667095899582 0.497031360864639
0.242316424846649 0.497089087963104
0.241673290729523 0.49662372469902
0.241168990731239 0.496788859367371
0.241378024220467 0.496776074171066
0.241753071546555 0.496632754802704
0.241627782583237 0.496603935956955
0.241870164871216 0.496469527482986
0.241669669747353 0.49615278840065
0.241971909999847 0.496000915765762
0.242606818675995 0.496309041976929
0.242711961269379 0.496307909488678
0.242598921060562 0.496006786823273
0.242771923542023 0.49606391787529
0.242832750082016 0.496239930391312
0.242969393730164 0.496277242898941
0.243054628372192 0.496231257915497
0.2431331127882 0.49604856967926
0.243073791265488 0.496272027492523
0.243207737803459 0.494525849819183
0.243539944291115 0.487343102693558
0.244141310453415 0.479179978370667
0.244710445404053 0.471967279911041
0.245605230331421 0.459939479827881
0.246825411915779 0.440770924091339
0.247879385948181 0.423308998346329
0.249402746558189 0.403653383255005
0.250712633132935 0.382871359586716
0.252095460891724 0.362752884626389
0.253508031368256 0.342090725898743
0.255107015371323 0.321616381406784
0.256781667470932 0.30085363984108
0.258128732442856 0.280504167079926
0.259763389825821 0.259840250015259
0.2611203789711 0.239598467946053
0.262632101774216 0.218364611268044
0.26421245932579 0.197928071022034
0.265465259552002 0.177986174821854
0.267167150974274 0.157000198960304
0.268375962972641 0.136379063129425
0.2697933614254 0.116206370294094
0.271325826644897 0.0944200456142426
0.2727070748806 0.0750001221895218
0.273782402276993 0.0544215738773346
0.275684803724289 0.0327688921242952
0.2769615650177 0.0126469098031521
0.278319180011749 -0.00728271342813969
0.279720097780228 -0.0279258098453283
0.281437307596207 -0.0480307824909687
0.283074468374252 -0.0689738020300865
0.284306943416595 -0.0900707840919495
0.285602480173111 -0.110053651034832
0.28702387213707 -0.130251377820969
0.288687169551849 -0.151221364736557
0.290145993232727 -0.172396272420883
0.291504144668579 -0.192216351628304
0.292989492416382 -0.212838500738144
0.294230312108994 -0.233130127191544
0.295586615800858 -0.253370046615601
0.2972811460495 -0.27406370639801
0.298909336328506 -0.294895201921463
0.300302356481552 -0.315097898244858
0.301769703626633 -0.336407750844955
0.302996069192886 -0.356083601713181
0.304398626089096 -0.377037674188614
0.305869549512863 -0.397575557231903
0.307332932949066 -0.41822150349617
0.308751404285431 -0.438584864139557
0.310341686010361 -0.459373563528061
0.311559289693832 -0.478299885988235
0.312345385551453 -0.490271627902985
0.313109427690506 -0.500602781772614
0.314103990793228 -0.512511909008026
0.314503401517868 -0.517237842082977
0.314738810062408 -0.519930839538574
0.315100967884064 -0.526087760925293
};
\addplot [semithick, red, dashed, forget plot]
table {%
0.25 -0.5
0.252558268640609 -0.499971215983182
0.257674800204547 -0.499913561943335
0.265349580546156 -0.499826822527541
0.275582586992214 -0.499710651911554
0.288373788282223 -0.499564571828345
0.303723144483551 -0.499387968751125
0.321630606880243 -0.499180090231924
0.342096117833915 -0.498940040397106
0.362561444756502 -0.498697137467738
0.383026586783779 -0.498451352589885
0.403491543037366 -0.498202656922469
0.423956312624558 -0.49795102163749
0.444420894638161 -0.497696417920264
0.464885288156327 -0.497438816969653
0.485349492242387 -0.497178189998295
0.505813505944689 -0.496914508232844
0.526277328296427 -0.4966477429142
0.546740958315482 -0.49637786529775
0.567204395004253 -0.496104846653605
0.587667637349495 -0.495828658266839
0.608130684322154 -0.495549271437727
0.628593534877201 -0.495266657481993
0.649056187953474 -0.494980787731044
0.669518642473505 -0.494691633532224
0.689980897343365 -0.49439916624905
0.710442951452497 -0.494103357261464
0.730904803673553 -0.493804177966079
0.751366452862231 -0.493501599776428
0.771827897857116 -0.493195594123215
0.792289137479511 -0.492886132454562
0.812750170533283 -0.492573186236266
0.833210995804693 -0.492256726952051
0.853671612062244 -0.491936726103822
0.874132018056508 -0.491613155211922
0.894592212519978 -0.491285985815386
0.915052194166895 -0.490955189472205
0.935511961693098 -0.49062073775958
0.955971513775856 -0.490282602274183
0.976430849073713 -0.489940754632423
0.996889966226327 -0.489595166470704
1.01734886385431 -0.489245809445691
1.03780754055907 -0.488892655234577
1.05826599492265 -0.488535675535346
1.07872422550757 -0.488174842067041
1.09918223085668 -0.487810126570036
1.11964000949299 -0.487441500806302
1.1400975599195 -0.48706893655968
1.16055488061908 -0.486692405636151
1.18101197005428 -0.486311879864113
1.20146882666719 -0.485927331094652
1.21936837110133 -0.485587306188302
1.23471068292331 -0.485293179242986
1.24749583154271 -0.485046149752488
1.25772387591864 -0.484847237098433
1.26539486430542 -0.484697275874471
1.27050883404033 -0.484596912039359
1.27306581137596 -0.484546599896135
1.2730658113589 -0.484546599895144
1.27305541155906 -0.484320911054072
1.2730238336127 -0.483870450269343
1.27294421772612 -0.483198810692961
1.27277401858219 -0.482315683800082
1.27245616754417 -0.481242223159833
1.27192174548361 -0.480017905211023
1.27109518948705 -0.478707989424014
1.26987791322201 -0.477382389353572
1.26861868448422 -0.476491262912181
1.26741310591075 -0.475930505649088
1.26634153930566 -0.475618318765773
1.26545420209816 -0.475475522024619
1.26477827798535 -0.475433459898834
1.26432554654506 -0.475438002031128
1.26409923112047 -0.475451102493
1.26409923107294 -0.475451102492838
1.26400510754809 -0.472893953643021
1.26381676720545 -0.467779675905825
1.26353397696839 -0.460108319156843
1.26315636493238 -0.449879963098326
1.26268342195054 -0.437094717113806
1.26211450389605 -0.421752720062896
1.26144883460045 -0.403854140014764
1.2606855094668 -0.383399173918339
1.25991927860775 -0.362944839063709
1.25915016511684 -0.342491133403313
1.25837819207707 -0.322038054872471
1.25760338256082 -0.301585601389565
1.25682575962964 -0.281133770856217
1.25604534633415 -0.260682561157473
1.25526216571386 -0.240231970161977
1.25447624079705 -0.219781995722154
1.25368759460061 -0.199332635674387
1.25289625012991 -0.178883887839198
1.25210223037863 -0.158435750021426
1.25130555832866 -0.137988220010405
1.25050625694995 -0.117541295580143
1.24970434920032 -0.0970949744895014
1.2488998580254 -0.0766492544823687
1.24809280635845 -0.0562041332878428
1.24728321712021 -0.0357596086204065
1.24647111321881 -0.0153156781801049
1.24565651754959 0.00512766034727754
1.244839452995 0.0255704092900399
1.24401994242446 0.0460125709903884
1.24319800869422 0.0664541478042603
1.24237367464724 0.0868951421011472
1.24154696311305 0.10733555626392
1.24071789690763 0.127775392688651
1.23988649883331 0.148214653784443
1.23905279167858 0.168653341973249
1.23821679821801 0.189091459689702
1.23737854121215 0.209529009380936
1.23653804340733 0.229965993506414
1.23569532753562 0.250402414537754
1.23485041631467 0.270838274958555
1.23400333244757 0.291273577264223
1.23315409862277 0.311708323961797
1.23230273751397 0.332142517569777
1.23144927177993 0.352576160617952
1.23059372406446 0.373009255647224
1.22973611699622 0.393441805209441
1.22887647318863 0.413873811867219
1.22801481523979 0.434305278193775
1.22715116573232 0.454736206772755
1.22628554723329 0.475166600198059
1.22552642790666 0.493042728464227
1.22487449160254 0.508364776283352
1.22433031870982 0.52113290203806
1.22389439056138 0.531347238274115
1.22356709317291 0.539007892120091
1.22334872031756 0.544114945632642
1.22323947593845 0.546668456066198
1.22323947590048 0.546668456066214
1.2230123362303 0.54664910964162
1.22255893634353 0.546599663012994
1.22188286957468 0.546493423413052
1.22099418533407 0.546288449972454
1.21991546682544 0.545929219980312
1.21868995460871 0.545349990456186
1.21739061938076 0.544480715471633
1.21610030698399 0.543230273381219
1.21526412455335 0.541961848316668
1.21476024252823 0.540762508740427
1.21448767411105 0.539705040879324
1.21435620536168 0.538833702471361
1.21429925392167 0.538171836563276
1.21427642250785 0.537729171531928
1.21429899313131 0.537497396084245
1.21429899310163 0.53749739608416
1.21173554316399 0.537351398434274
1.2066086675668 0.537059357198718
1.19891842689941 0.536621157703719
1.18866491799536 0.536036617118684
1.17584827376684 0.535305485487585
1.16046866297006 0.534427447211652
1.1425262899005 0.533402122984755
1.12202139401749 0.532229072183377
1.10151726555175 0.531054604210161
1.08101390213675 0.529878734481104
1.06051130139776 0.528701478429746
1.04000946095186 0.5275228515073
1.01950837840803 0.526342869182777
0.999008051367172 0.525161546943109
0.978508477422155 0.523978900293278
0.958009654157879 0.522794944756435
0.937511579151311 0.521609695874025
0.917014249971538 0.520423169205916
0.896517664179812 0.519235380330514
0.876021819329601 0.518046344844887
0.855526712966634 0.516856078364893
0.835032342628955 0.515664596525293
0.814538705846972 0.514471914979876
0.794045800143499 0.513278049401579
0.773553623033816 0.512083015482608
0.753062172025712 0.510886828934553
0.732571444619537 0.509689505488512
0.712081438308253 0.508491060895207
0.691592150577485 0.507291510925102
0.671103578905572 0.506090871368521
0.650615720763616 0.504889158035764
0.630128573615536 0.503686386757223
0.609642134918118 0.502482573383502
0.589156402121068 0.501277733785526
0.568671372667066 0.500071883854662
0.548187043991813 0.49886503950283
0.527703413524089 0.497657216662618
0.507220478685805 0.496448431287396
0.486738236892053 0.495238699351429
0.466256685551165 0.494028036849988
0.445775822064762 0.492816459799465
0.425295643827809 0.491603984237482
0.404816148228671 0.490390626223003
0.384337332649168 0.489176401836445
0.363859194464625 0.487961327179788
0.343381731043935 0.486745418376685
0.322904939749608 0.48552869157257
0.302428817937827 0.484311162934768
0.281953362958509 0.483092848652601
0.261478572155355 0.4818737649375
0.243563709029771 0.480806407637357
0.228208544470985 0.479891046434884
0.215412881797103 0.479127906813159
0.20517655624396 0.4785171732374
0.197499434529847 0.478058991869279
0.192381414497009 0.477753472810477
0.189822424830656 0.477600691872751
0.189822424856044 0.477600691872319
0.18984329041129 0.477361217804826
0.189891281488213 0.476888179768742
0.18998422190601 0.476195304214649
0.190156086957441 0.475302243681175
0.19046107666499 0.47423395694596
0.190973511005364 0.47302370084793
0.1917804370421 0.471720939032194
0.192990859960637 0.470375213510415
0.194255394688102 0.469435488118
0.195465103825654 0.468815206001496
0.196534139830494 0.468445902524195
0.19741290176854 0.468255645457359
0.19807765083531 0.468178447005524
0.198520413044486 0.468160326901788
0.198740901208049 0.468162462817004
0.198740901269321 0.46816246281638
0.198945440782965 0.465612270025241
0.199354508471781 0.460511864695189
0.199968076306772 0.452861197525416
0.200786100039452 0.442660189857331
0.201808520127061 0.429908733995519
0.203035263052485 0.414606693655117
0.20446624304152 0.396753904535665
0.206101364180888 0.376350175021588
0.207736171974856 0.355945830121864
0.209370679915717 0.335540874152152
0.211004901527408 0.315135311429408
0.212638850365535 0.294729146271967
0.214272540017396 0.27432238299961
0.215905984101993 0.253915025933645
0.217539196270065 0.233507079396979
0.219172190204101 0.2130985477142
0.220804979618368 0.192689435211644
0.222437578258934 0.172279746217481
0.224069999903693 0.15186948506179
0.225702258362386 0.131458656076634
0.227334367476629 0.111047263596139
0.228966341119943 0.0906353119565765
0.230598193197773 0.0702228054964383
0.232229937647521 0.0498097485565179
0.233861588438572 0.0293961454799898
0.235493159572323 0.00898200061249
0.237124665082211 -0.0114326816978031
0.238756119033746 -0.0318478971000879
0.240387535524538 -0.0522636412408578
0.24201892868433 -0.0726799097638194
0.243650312675029 -0.0930966983098097
0.245281701690739 -0.113514002516714
0.246913109957795 -0.133931818019382
0.248544551734793 -0.154350140449545
0.250176041312631 -0.174768965435729
0.251807593014535 -0.195188288603176
0.253439221196103 -0.215608105573753
0.255070940245335 -0.236028411965869
0.256702764582677 -0.256449203394391
0.258334708661049 -0.276870475470555
0.259966786965892 -0.29729222380188
0.261599014015202 -0.317714443992083
0.263231404359571 -0.338137131640988
0.264863972582227 -0.358560282344441
0.266496733299076 -0.378983891694222
0.26812970115874 -0.399407955277952
0.269762890842605 -0.419832468679009
0.271396317064857 -0.440257427476434
0.27302999457253 -0.460682827244845
0.27466393814555 -0.481108663554342
0.276093884549649 -0.498981648418472
0.277319746923832 -0.514301630525306
0.278341443563315 -0.527068481124652
0.279158900790641 -0.537282093189924
0.279772055404735 -0.544942380708266
0.280180856703051 -0.550049278099581
0.280385268072754 -0.552602739765039
};
\addplot [semithick, green, dash pattern=on 1pt off 3pt on 3pt off 3pt, forget plot]
table {%
0.25 -0.5
0.2525 -0.5
0.2575 -0.5
0.265 -0.5
0.275 -0.5
0.2875 -0.5
0.3025 -0.5
0.32 -0.5
0.34 -0.5
0.36 -0.5
0.38 -0.5
0.4 -0.5
0.42 -0.5
0.44 -0.5
0.46 -0.5
0.48 -0.5
0.5 -0.5
0.52 -0.5
0.54 -0.5
0.56 -0.5
0.58 -0.5
0.6 -0.5
0.62 -0.5
0.64 -0.5
0.66 -0.5
0.68 -0.5
0.7 -0.5
0.72 -0.5
0.74 -0.5
0.76 -0.5
0.78 -0.5
0.8 -0.5
0.82 -0.5
0.84 -0.5
0.86 -0.5
0.88 -0.5
0.9 -0.5
0.92 -0.5
0.94 -0.5
0.96 -0.5
0.98 -0.5
1 -0.5
1.02 -0.5
1.04 -0.5
1.06 -0.5
1.08 -0.5
1.1 -0.5
1.12 -0.5
1.14 -0.5
1.16 -0.5
1.18 -0.5
1.1975 -0.5
1.2125 -0.5
1.225 -0.5
1.235 -0.5
1.2425 -0.5
1.2475 -0.5
1.25 -0.5
1.25 -0.5
1.25 -0.5
1.25 -0.5
1.25 -0.5
1.25 -0.5
1.25 -0.5
1.25 -0.5
1.25 -0.5
1.25 -0.5
1.25 -0.5
1.25 -0.5
1.25 -0.5
1.25 -0.5
1.25 -0.5
1.25 -0.5
1.25 -0.5
1.25 -0.5
1.25 -0.4975
1.25 -0.4925
1.25 -0.485
1.25 -0.475
1.25 -0.4625
1.25 -0.4475
1.25 -0.43
1.25 -0.41
1.25 -0.39
1.25 -0.37
1.25 -0.35
1.25 -0.33
1.25 -0.31
1.25 -0.29
1.25 -0.27
1.25 -0.25
1.25 -0.23
1.25 -0.21
1.25 -0.19
1.25 -0.17
1.25 -0.15
1.25 -0.13
1.25 -0.11
1.25 -0.09
1.25 -0.07
1.25 -0.05
1.25 -0.03
1.25 -0.01
1.25 0.00999999999999998
1.25 0.03
1.25 0.05
1.25 0.07
1.25 0.09
1.25 0.11
1.25 0.13
1.25 0.15
1.25 0.17
1.25 0.19
1.25 0.21
1.25 0.23
1.25 0.25
1.25 0.27
1.25 0.29
1.25 0.31
1.25 0.33
1.25 0.35
1.25 0.37
1.25 0.39
1.25 0.41
1.25 0.43
1.25 0.4475
1.25 0.4625
1.25 0.475
1.25 0.485
1.25 0.4925
1.25 0.4975
1.25 0.5
1.25 0.5
1.25 0.5
1.25 0.5
1.25 0.5
1.25 0.5
1.25 0.5
1.25 0.5
1.25 0.5
1.25 0.5
1.25 0.5
1.25 0.5
1.25 0.5
1.25 0.5
1.25 0.5
1.25 0.5
1.25 0.5
1.25 0.5
1.2475 0.5
1.2425 0.5
1.235 0.5
1.225 0.5
1.2125 0.5
1.1975 0.5
1.18 0.5
1.16 0.5
1.14 0.5
1.12 0.5
1.1 0.5
1.08 0.5
1.06 0.5
1.04 0.5
1.02 0.5
1 0.5
0.98 0.5
0.96 0.5
0.94 0.5
0.92 0.5
0.9 0.5
0.88 0.5
0.86 0.5
0.84 0.5
0.82 0.5
0.8 0.5
0.78 0.5
0.76 0.5
0.74 0.5
0.72 0.5
0.7 0.5
0.68 0.5
0.66 0.5
0.64 0.5
0.62 0.5
0.6 0.5
0.58 0.5
0.56 0.5
0.54 0.5
0.52 0.5
0.5 0.5
0.48 0.5
0.46 0.5
0.44 0.5
0.42 0.5
0.4 0.5
0.38 0.5
0.36 0.5
0.34 0.5
0.32 0.5
0.3025 0.5
0.2875 0.5
0.275 0.5
0.265 0.5
0.2575 0.5
0.2525 0.5
0.25 0.5
0.25 0.5
0.25 0.5
0.25 0.5
0.25 0.5
0.25 0.5
0.25 0.5
0.25 0.5
0.25 0.5
0.25 0.5
0.25 0.5
0.25 0.5
0.25 0.5
0.25 0.5
0.25 0.5
0.25 0.5
0.25 0.5
0.25 0.5
0.25 0.4975
0.25 0.4925
0.25 0.485
0.25 0.475
0.25 0.4625
0.25 0.4475
0.25 0.43
0.25 0.41
0.25 0.39
0.25 0.37
0.25 0.35
0.25 0.33
0.25 0.31
0.25 0.29
0.25 0.27
0.25 0.25
0.25 0.23
0.25 0.21
0.25 0.19
0.25 0.17
0.25 0.15
0.249999999999999 0.13
0.249999999999999 0.11
0.249999999999999 0.09
0.249999999999999 0.07
0.249999999999999 0.05
0.249999999999999 0.03
0.249999999999999 0.01
0.249999999999999 -0.00999999999999998
0.249999999999999 -0.03
0.249999999999999 -0.05
0.249999999999999 -0.07
0.249999999999999 -0.09
0.249999999999999 -0.11
0.249999999999999 -0.13
0.249999999999999 -0.15
0.249999999999999 -0.17
0.249999999999999 -0.19
0.249999999999999 -0.21
0.249999999999999 -0.23
0.249999999999999 -0.25
0.249999999999999 -0.27
0.249999999999999 -0.29
0.249999999999999 -0.31
0.249999999999999 -0.33
0.249999999999999 -0.35
0.249999999999999 -0.37
0.249999999999999 -0.39
0.249999999999999 -0.41
0.249999999999999 -0.43
0.249999999999999 -0.4475
0.249999999999999 -0.4625
0.249999999999999 -0.475
0.249999999999999 -0.485
0.249999999999999 -0.4925
0.249999999999999 -0.4975
0.249999999999999 -0.5
};
\addplot [semithick, blue, opacity=\opacityRef, forget plot]
table {%
0.25 -0.5
0.250557988882065 -0.499968707561493
0.254411071538925 -0.500163912773132
0.262099504470825 -0.500716626644135
0.270861357450485 -0.501002848148346
0.283630937337875 -0.501982808113098
0.298274010419846 -0.502451062202454
0.315875679254532 -0.503057539463043
0.335843235254288 -0.504007697105408
0.356790572404861 -0.505324304103851
0.376916587352753 -0.50628924369812
0.398080766201019 -0.507321059703827
0.418591022491455 -0.508136212825775
0.439080953598022 -0.509061574935913
0.459406584501266 -0.509778022766113
0.479748398065567 -0.510706603527069
0.500260829925537 -0.511809051036835
0.52105700969696 -0.512844920158386
0.541355013847351 -0.513696670532227
0.561919271945953 -0.514661967754364
0.582505106925964 -0.515798807144165
0.603079676628113 -0.516850590705872
0.623943507671356 -0.517794191837311
0.644135117530823 -0.518778204917908
0.664786875247955 -0.519918084144592
0.685331344604492 -0.520718038082123
0.705947518348694 -0.521746575832367
0.726427555084229 -0.522948622703552
0.746725499629974 -0.523962080478668
0.767204821109772 -0.524873971939087
0.787912845611572 -0.525911033153534
0.80865603685379 -0.527000844478607
0.829085290431976 -0.52813732624054
0.849607646465302 -0.528957903385162
0.870140075683594 -0.529826402664185
0.890612185001373 -0.530845642089844
0.911466419696808 -0.531573295593262
0.931665241718292 -0.532603621482849
0.951985120773315 -0.533402562141418
0.972350299358368 -0.53446239233017
0.992799997329712 -0.53535670042038
1.01364892721176 -0.536264181137085
1.03409522771835 -0.537349879741669
1.05423885583878 -0.538373172283173
1.07492262125015 -0.539351940155029
1.09575098752975 -0.540011763572693
1.11633676290512 -0.541502058506012
1.13672024011612 -0.542266249656677
1.15738075971603 -0.543476939201355
1.17804712057114 -0.544548034667969
1.19861298799515 -0.545581698417664
1.21811348199844 -0.546623647212982
1.23336726427078 -0.547548770904541
1.24671131372452 -0.548361539840698
1.25762182474136 -0.548964738845825
1.26571410894394 -0.549422979354858
1.27151483297348 -0.549870848655701
1.27444261312485 -0.550102114677429
1.27507251501083 -0.550116002559662
1.27564960718155 -0.550132513046265
1.27680236101151 -0.549857139587402
1.27643269300461 -0.549602270126343
1.27669936418533 -0.549797058105469
1.27685457468033 -0.549320757389069
1.27656298875809 -0.548904955387115
1.27642434835434 -0.548710227012634
1.27623444795609 -0.548829913139343
1.27589684724808 -0.549115478992462
1.27577131986618 -0.54877632856369
1.27526313066483 -0.549218833446503
1.27520579099655 -0.549347341060638
1.27512377500534 -0.549501240253448
1.27506810426712 -0.549680292606354
1.27502530813217 -0.549729704856873
1.2750124335289 -0.549564719200134
1.27505594491959 -0.549153923988342
1.27517873048782 -0.545571088790894
1.27551919221878 -0.537738621234894
1.27586299180984 -0.528184652328491
1.27627176046371 -0.515857517719269
1.27661198377609 -0.501197814941406
1.27699762582779 -0.483643889427185
1.27757328748703 -0.463871449232101
1.27837520837784 -0.442396402359009
1.27905660867691 -0.422391146421432
1.27976304292679 -0.401323616504669
1.28054744005203 -0.381450086832047
1.28120166063309 -0.36094456911087
1.2817410826683 -0.340153425931931
1.2826743721962 -0.31957346200943
1.28349953889847 -0.299708366394043
1.28421741724014 -0.278636336326599
1.28476220369339 -0.258201122283936
1.28547781705856 -0.237296164035797
1.28623431921005 -0.217263787984848
1.28706461191177 -0.196375265717506
1.2878902554512 -0.176427781581879
1.28875213861465 -0.155477568507195
1.28948777914047 -0.134898364543915
1.29022008180618 -0.114074170589447
1.29108852148056 -0.0939716622233391
1.29196244478226 -0.0736533775925636
1.29266375303268 -0.0527499616146088
1.29331833124161 -0.0320376679301262
1.29397767782211 -0.0115231834352016
1.29477506875992 0.0089070089161396
1.2955077290535 0.0295686796307564
1.29634636640549 0.0503309965133667
1.29713147878647 0.0702872574329376
1.29776388406754 0.0910497754812241
1.29837244749069 0.111535020172596
1.29912441968918 0.132282361388206
1.2998611330986 0.152681156992912
1.30041271448135 0.172838345170021
1.30086010694504 0.19355720281601
1.30142432451248 0.21406777203083
1.30205625295639 0.234990566968918
1.30240315198898 0.255443811416626
1.30318635702133 0.275986313819885
1.30387777090073 0.296195834875107
1.30402606725693 0.316650152206421
1.30489367246628 0.336894750595093
1.3047816157341 0.358103215694427
1.30613666772842 0.378630757331848
1.30723613500595 0.398990333080292
1.30743843317032 0.418592840433121
1.30814915895462 0.434183746576309
1.30858486890793 0.44777774810791
1.30903774499893 0.458549112081528
1.30908769369125 0.466660261154175
1.30901747941971 0.472291588783264
1.30911356210709 0.475542157888412
1.30950182676315 0.475999385118484
1.30913347005844 0.476502120494843
1.30914586782455 0.477198988199234
1.30924278497696 0.47732612490654
1.30903881788254 0.476541608572006
1.30872708559036 0.476881206035614
1.30891352891922 0.476710915565491
1.30873793363571 0.476209163665771
1.30843275785446 0.475833564996719
1.30876082181931 0.47573384642601
1.30841225385666 0.475794434547424
1.30823904275894 0.47573921084404
1.30868774652481 0.475568830966949
1.30843776464462 0.475500732660294
1.30831760168076 0.475526839494705
1.30814701318741 0.475528031587601
1.30804592370987 0.475525468587875
1.30760771036148 0.475603669881821
1.30407756567001 0.475749790668488
1.29651528596878 0.476180970668793
1.28692263364792 0.476645320653915
1.27464836835861 0.477187037467957
1.25940591096878 0.478010416030884
1.24188774824142 0.478944033384323
1.22213143110275 0.4799445271492
1.20168823003769 0.480849325656891
1.18169122934341 0.481682986021042
1.16088503599167 0.482942789793015
1.14017921686172 0.483995884656906
1.11931163072586 0.484952419996262
1.09868544340134 0.485895484685898
1.07864207029343 0.486888855695724
1.05794304609299 0.487967133522034
1.03676921129227 0.489028811454773
1.0168200135231 0.48982560634613
0.996870696544647 0.490827441215515
0.976148545742035 0.491673171520233
0.955514192581177 0.49254047870636
0.935023188591003 0.493689984083176
0.914903581142426 0.494650632143021
0.894773423671722 0.495493412017822
0.873569428920746 0.496370404958725
0.853182852268219 0.497313261032104
0.833121478557587 0.49813523888588
0.812546730041504 0.499097347259521
0.792400717735291 0.499989479780197
0.77167534828186 0.501016318798065
0.751469016075134 0.501725703477859
0.730691015720367 0.502540081739426
0.710508346557617 0.503779768943787
0.689988255500793 0.504930436611176
0.669147193431854 0.505915820598602
0.648479104042053 0.506810009479523
0.627851784229279 0.507973432540894
0.607438802719116 0.508839726448059
0.586697578430176 0.509893238544464
0.566323101520538 0.510913550853729
0.545874893665314 0.512023329734802
0.525525808334351 0.512926697731018
0.505177855491638 0.513875186443329
0.484474182128906 0.514971256256104
0.463737696409225 0.516161561012268
0.443160742521286 0.517107784748077
0.422152549028397 0.517846167087555
0.40172877907753 0.518969237804413
0.381043642759323 0.519921362400055
0.360238134860992 0.520972669124603
0.341003686189651 0.522059917449951
0.325528711080551 0.522801578044891
0.312139630317688 0.523549616336823
0.301209300756454 0.523963570594788
0.292941778898239 0.524319648742676
0.287254989147186 0.524547159671783
0.284045606851578 0.52452290058136
0.283484935760498 0.524563252925873
0.282805770635605 0.524567127227783
0.281828016042709 0.524586021900177
0.282351821660995 0.524484157562256
0.28217676281929 0.524358570575714
0.281997740268707 0.524267017841339
0.281997203826904 0.524167656898499
0.282039076089859 0.523936331272125
0.282324522733688 0.523876667022705
0.282701373100281 0.52414482831955
0.282527327537537 0.523605942726135
0.282758980989456 0.523676335811615
0.282936155796051 0.523837924003601
0.28314134478569 0.523839294910431
0.283300250768661 0.524064719676971
0.283275961875916 0.52403312921524
0.283298999071121 0.524028182029724
0.283289819955826 0.523814380168915
0.283223092556 0.520027935504913
0.282942146062851 0.512533068656921
0.282854229211807 0.50452321767807
0.282516956329346 0.494243294000626
0.28212833404541 0.478032857179642
0.281552970409393 0.461326748132706
0.280994176864624 0.441448450088501
0.280375927686691 0.420009762048721
0.279959857463837 0.399949312210083
0.279385477304459 0.379364520311356
0.278951019048691 0.358936846256256
0.278380453586578 0.338615119457245
0.277785390615463 0.318081766366959
0.277397066354752 0.296852856874466
0.27705717086792 0.277044057846069
0.276248961687088 0.256304115056992
0.275735169649124 0.235596016049385
0.275148719549179 0.214233383536339
0.274775147438049 0.193844884634018
0.274206936359406 0.173236325383186
0.273490965366364 0.1527389138937
0.272831320762634 0.131875172257423
0.272486597299576 0.110808327794075
0.271613597869873 0.0916711315512657
0.271139353513718 0.071215994656086
0.270530581474304 0.0511507950723171
0.270275115966797 0.0289886314421892
0.269375920295715 0.00850944966077805
0.268788158893585 -0.0112309167161584
0.268330007791519 -0.0320054963231087
0.267687648534775 -0.0529673583805561
0.267076820135117 -0.0738661363720894
0.266342729330063 -0.0937515944242477
0.266045868396759 -0.114012278616428
0.265438824892044 -0.134413406252861
0.264768868684769 -0.15542858839035
0.264258861541748 -0.176429584622383
0.263769179582596 -0.196637764573097
0.262828230857849 -0.217230081558228
0.262236773967743 -0.238074719905853
0.26181036233902 -0.258662611246109
0.261175870895386 -0.279510855674744
0.260758817195892 -0.299705028533936
0.260118454694748 -0.320132702589035
0.259589999914169 -0.340174108743668
0.25914192199707 -0.361140608787537
0.258512407541275 -0.381511360406876
0.258088946342468 -0.402296751737595
0.257364749908447 -0.422832757234573
0.256527841091156 -0.441640645265579
0.256003171205521 -0.457966864109039
0.255562484264374 -0.471664100885391
0.255228400230408 -0.48249763250351
0.254851698875427 -0.490871727466583
0.254881352186203 -0.496460318565369
0.254825264215469 -0.499390244483948
0.254553020000458 -0.500230312347412
};
\addplot [semithick, red, dashed, forget plot]
table {%
0.25 -0.5
0.252558268640609 -0.499971215983182
0.257674800204547 -0.499913561943335
0.265349580546156 -0.499826822527541
0.275582586992214 -0.499710651911554
0.288373788282223 -0.499564571828345
0.303723144483551 -0.499387968751125
0.321630606880243 -0.499180090231924
0.342096117833915 -0.498940040397106
0.362561444756502 -0.498697137467738
0.383026586783779 -0.498451352589885
0.403491543037366 -0.498202656922469
0.423956312624558 -0.49795102163749
0.444420894638161 -0.497696417920264
0.464885288156327 -0.497438816969653
0.485349492242387 -0.497178189998295
0.505813505944689 -0.496914508232844
0.526277328296427 -0.4966477429142
0.546740958315482 -0.49637786529775
0.567204395004253 -0.496104846653605
0.587667637349495 -0.495828658266839
0.608130684322154 -0.495549271437727
0.628593534877201 -0.495266657481993
0.649056187953474 -0.494980787731044
0.669518642473505 -0.494691633532224
0.689980897343365 -0.49439916624905
0.710442951452497 -0.494103357261464
0.730904803673553 -0.493804177966079
0.751366452862231 -0.493501599776428
0.771827897857116 -0.493195594123215
0.792289137479511 -0.492886132454562
0.812750170533283 -0.492573186236266
0.833210995804693 -0.492256726952051
0.853671612062244 -0.491936726103822
0.874132018056508 -0.491613155211922
0.894592212519978 -0.491285985815386
0.915052194166895 -0.490955189472205
0.935511961693098 -0.49062073775958
0.955971513775856 -0.490282602274183
0.976430849073713 -0.489940754632423
0.996889966226327 -0.489595166470704
1.01734886385431 -0.489245809445691
1.03780754055907 -0.488892655234577
1.05826599492265 -0.488535675535346
1.07872422550757 -0.488174842067041
1.09918223085668 -0.487810126570036
1.11964000949299 -0.487441500806302
1.1400975599195 -0.48706893655968
1.16055488061908 -0.486692405636151
1.18101197005428 -0.486311879864113
1.20146882666719 -0.485927331094652
1.21936837110133 -0.485587306188302
1.23471068292331 -0.485293179242986
1.24749583154271 -0.485046149752488
1.25772387591864 -0.484847237098433
1.26539486430542 -0.484697275874471
1.27050883404033 -0.484596912039359
1.27306581137596 -0.484546599896135
1.2730658113589 -0.484546599895144
1.27305541155906 -0.484320911054072
1.2730238336127 -0.483870450269343
1.27294421772612 -0.483198810692961
1.27277401858219 -0.482315683800082
1.27245616754417 -0.481242223159833
1.27192174548361 -0.480017905211023
1.27109518948705 -0.478707989424014
1.26987791322201 -0.477382389353572
1.26861868448422 -0.476491262912181
1.26741310591075 -0.475930505649088
1.26634153930566 -0.475618318765773
1.26545420209816 -0.475475522024619
1.26477827798535 -0.475433459898834
1.26432554654506 -0.475438002031128
1.26409923112047 -0.475451102493
1.26409923107294 -0.475451102492838
1.26400510754809 -0.472893953643021
1.26381676720545 -0.467779675905825
1.26353397696839 -0.460108319156843
1.26315636493238 -0.449879963098326
1.26268342195054 -0.437094717113806
1.26211450389605 -0.421752720062896
1.26144883460045 -0.403854140014764
1.2606855094668 -0.383399173918339
1.25991927860775 -0.362944839063709
1.25915016511684 -0.342491133403313
1.25837819207707 -0.322038054872471
1.25760338256082 -0.301585601389565
1.25682575962964 -0.281133770856217
1.25604534633415 -0.260682561157473
1.25526216571386 -0.240231970161977
1.25447624079705 -0.219781995722154
1.25368759460061 -0.199332635674387
1.25289625012991 -0.178883887839198
1.25210223037863 -0.158435750021426
1.25130555832866 -0.137988220010405
1.25050625694995 -0.117541295580143
1.24970434920032 -0.0970949744895014
1.2488998580254 -0.0766492544823687
1.24809280635845 -0.0562041332878428
1.24728321712021 -0.0357596086204065
1.24647111321881 -0.0153156781801049
1.24565651754959 0.00512766034727754
1.244839452995 0.0255704092900399
1.24401994242446 0.0460125709903884
1.24319800869422 0.0664541478042603
1.24237367464724 0.0868951421011472
1.24154696311305 0.10733555626392
1.24071789690763 0.127775392688651
1.23988649883331 0.148214653784443
1.23905279167858 0.168653341973249
1.23821679821801 0.189091459689702
1.23737854121215 0.209529009380936
1.23653804340733 0.229965993506414
1.23569532753562 0.250402414537754
1.23485041631467 0.270838274958555
1.23400333244757 0.291273577264223
1.23315409862277 0.311708323961797
1.23230273751397 0.332142517569777
1.23144927177993 0.352576160617952
1.23059372406446 0.373009255647224
1.22973611699622 0.393441805209441
1.22887647318863 0.413873811867219
1.22801481523979 0.434305278193775
1.22715116573232 0.454736206772755
1.22628554723329 0.475166600198059
1.22552642790666 0.493042728464227
1.22487449160254 0.508364776283352
1.22433031870982 0.52113290203806
1.22389439056138 0.531347238274115
1.22356709317291 0.539007892120091
1.22334872031756 0.544114945632642
1.22323947593845 0.546668456066198
1.22323947590048 0.546668456066214
1.2230123362303 0.54664910964162
1.22255893634353 0.546599663012994
1.22188286957468 0.546493423413052
1.22099418533407 0.546288449972454
1.21991546682544 0.545929219980312
1.21868995460871 0.545349990456186
1.21739061938076 0.544480715471633
1.21610030698399 0.543230273381219
1.21526412455335 0.541961848316668
1.21476024252823 0.540762508740427
1.21448767411105 0.539705040879324
1.21435620536168 0.538833702471361
1.21429925392167 0.538171836563276
1.21427642250785 0.537729171531928
1.21429899313131 0.537497396084245
1.21429899310163 0.53749739608416
1.21173554316399 0.537351398434274
1.2066086675668 0.537059357198718
1.19891842689941 0.536621157703719
1.18866491799536 0.536036617118684
1.17584827376684 0.535305485487585
1.16046866297006 0.534427447211652
1.1425262899005 0.533402122984755
1.12202139401749 0.532229072183377
1.10151726555175 0.531054604210161
1.08101390213675 0.529878734481104
1.06051130139776 0.528701478429746
1.04000946095186 0.5275228515073
1.01950837840803 0.526342869182777
0.999008051367172 0.525161546943109
0.978508477422155 0.523978900293278
0.958009654157879 0.522794944756435
0.937511579151311 0.521609695874025
0.917014249971538 0.520423169205916
0.896517664179812 0.519235380330514
0.876021819329601 0.518046344844887
0.855526712966634 0.516856078364893
0.835032342628955 0.515664596525293
0.814538705846972 0.514471914979876
0.794045800143499 0.513278049401579
0.773553623033816 0.512083015482608
0.753062172025712 0.510886828934553
0.732571444619537 0.509689505488512
0.712081438308253 0.508491060895207
0.691592150577485 0.507291510925102
0.671103578905572 0.506090871368521
0.650615720763616 0.504889158035764
0.630128573615536 0.503686386757223
0.609642134918118 0.502482573383502
0.589156402121068 0.501277733785526
0.568671372667066 0.500071883854662
0.548187043991813 0.49886503950283
0.527703413524089 0.497657216662618
0.507220478685805 0.496448431287396
0.486738236892053 0.495238699351429
0.466256685551165 0.494028036849988
0.445775822064762 0.492816459799465
0.425295643827809 0.491603984237482
0.404816148228671 0.490390626223003
0.384337332649168 0.489176401836445
0.363859194464625 0.487961327179788
0.343381731043935 0.486745418376685
0.322904939749608 0.48552869157257
0.302428817937827 0.484311162934768
0.281953362958509 0.483092848652601
0.261478572155355 0.4818737649375
0.243563709029771 0.480806407637357
0.228208544470985 0.479891046434884
0.215412881797103 0.479127906813159
0.20517655624396 0.4785171732374
0.197499434529847 0.478058991869279
0.192381414497009 0.477753472810477
0.189822424830656 0.477600691872751
0.189822424856044 0.477600691872319
0.18984329041129 0.477361217804826
0.189891281488213 0.476888179768742
0.18998422190601 0.476195304214649
0.190156086957441 0.475302243681175
0.19046107666499 0.47423395694596
0.190973511005364 0.47302370084793
0.1917804370421 0.471720939032194
0.192990859960637 0.470375213510415
0.194255394688102 0.469435488118
0.195465103825654 0.468815206001496
0.196534139830494 0.468445902524195
0.19741290176854 0.468255645457359
0.19807765083531 0.468178447005524
0.198520413044486 0.468160326901788
0.198740901208049 0.468162462817004
0.198740901269321 0.46816246281638
0.198945440782965 0.465612270025241
0.199354508471781 0.460511864695189
0.199968076306772 0.452861197525416
0.200786100039452 0.442660189857331
0.201808520127061 0.429908733995519
0.203035263052485 0.414606693655117
0.20446624304152 0.396753904535665
0.206101364180888 0.376350175021588
0.207736171974856 0.355945830121864
0.209370679915717 0.335540874152152
0.211004901527408 0.315135311429408
0.212638850365535 0.294729146271967
0.214272540017396 0.27432238299961
0.215905984101993 0.253915025933645
0.217539196270065 0.233507079396979
0.219172190204101 0.2130985477142
0.220804979618368 0.192689435211644
0.222437578258934 0.172279746217481
0.224069999903693 0.15186948506179
0.225702258362386 0.131458656076634
0.227334367476629 0.111047263596139
0.228966341119943 0.0906353119565765
0.230598193197773 0.0702228054964383
0.232229937647521 0.0498097485565179
0.233861588438572 0.0293961454799898
0.235493159572323 0.00898200061249
0.237124665082211 -0.0114326816978031
0.238756119033746 -0.0318478971000879
0.240387535524538 -0.0522636412408578
0.24201892868433 -0.0726799097638194
0.243650312675029 -0.0930966983098097
0.245281701690739 -0.113514002516714
0.246913109957795 -0.133931818019382
0.248544551734793 -0.154350140449545
0.250176041312631 -0.174768965435729
0.251807593014535 -0.195188288603176
0.253439221196103 -0.215608105573753
0.255070940245335 -0.236028411965869
0.256702764582677 -0.256449203394391
0.258334708661049 -0.276870475470555
0.259966786965892 -0.29729222380188
0.261599014015202 -0.317714443992083
0.263231404359571 -0.338137131640988
0.264863972582227 -0.358560282344441
0.266496733299076 -0.378983891694222
0.26812970115874 -0.399407955277952
0.269762890842605 -0.419832468679009
0.271396317064857 -0.440257427476434
0.27302999457253 -0.460682827244845
0.27466393814555 -0.481108663554342
0.276093884549649 -0.498981648418472
0.277319746923832 -0.514301630525306
0.278341443563315 -0.527068481124652
0.279158900790641 -0.537282093189924
0.279772055404735 -0.544942380708266
0.280180856703051 -0.550049278099581
0.280385268072754 -0.552602739765039
};
\addplot [semithick, green, dash pattern=on 1pt off 3pt on 3pt off 3pt, forget plot]
table {%
0.25 -0.5
0.2525 -0.5
0.2575 -0.5
0.265 -0.5
0.275 -0.5
0.2875 -0.5
0.3025 -0.5
0.32 -0.5
0.34 -0.5
0.36 -0.5
0.38 -0.5
0.4 -0.5
0.42 -0.5
0.44 -0.5
0.46 -0.5
0.48 -0.5
0.5 -0.5
0.52 -0.5
0.54 -0.5
0.56 -0.5
0.58 -0.5
0.6 -0.5
0.62 -0.5
0.64 -0.5
0.66 -0.5
0.68 -0.5
0.7 -0.5
0.72 -0.5
0.74 -0.5
0.76 -0.5
0.78 -0.5
0.8 -0.5
0.82 -0.5
0.84 -0.5
0.86 -0.5
0.88 -0.5
0.9 -0.5
0.92 -0.5
0.94 -0.5
0.96 -0.5
0.98 -0.5
1 -0.5
1.02 -0.5
1.04 -0.5
1.06 -0.5
1.08 -0.5
1.1 -0.5
1.12 -0.5
1.14 -0.5
1.16 -0.5
1.18 -0.5
1.1975 -0.5
1.2125 -0.5
1.225 -0.5
1.235 -0.5
1.2425 -0.5
1.2475 -0.5
1.25 -0.5
1.25 -0.5
1.25 -0.5
1.25 -0.5
1.25 -0.5
1.25 -0.5
1.25 -0.5
1.25 -0.5
1.25 -0.5
1.25 -0.5
1.25 -0.5
1.25 -0.5
1.25 -0.5
1.25 -0.5
1.25 -0.5
1.25 -0.5
1.25 -0.5
1.25 -0.5
1.25 -0.4975
1.25 -0.4925
1.25 -0.485
1.25 -0.475
1.25 -0.4625
1.25 -0.4475
1.25 -0.43
1.25 -0.41
1.25 -0.39
1.25 -0.37
1.25 -0.35
1.25 -0.33
1.25 -0.31
1.25 -0.29
1.25 -0.27
1.25 -0.25
1.25 -0.23
1.25 -0.21
1.25 -0.19
1.25 -0.17
1.25 -0.15
1.25 -0.13
1.25 -0.11
1.25 -0.09
1.25 -0.07
1.25 -0.05
1.25 -0.03
1.25 -0.01
1.25 0.00999999999999998
1.25 0.03
1.25 0.05
1.25 0.07
1.25 0.09
1.25 0.11
1.25 0.13
1.25 0.15
1.25 0.17
1.25 0.19
1.25 0.21
1.25 0.23
1.25 0.25
1.25 0.27
1.25 0.29
1.25 0.31
1.25 0.33
1.25 0.35
1.25 0.37
1.25 0.39
1.25 0.41
1.25 0.43
1.25 0.4475
1.25 0.4625
1.25 0.475
1.25 0.485
1.25 0.4925
1.25 0.4975
1.25 0.5
1.25 0.5
1.25 0.5
1.25 0.5
1.25 0.5
1.25 0.5
1.25 0.5
1.25 0.5
1.25 0.5
1.25 0.5
1.25 0.5
1.25 0.5
1.25 0.5
1.25 0.5
1.25 0.5
1.25 0.5
1.25 0.5
1.25 0.5
1.2475 0.5
1.2425 0.5
1.235 0.5
1.225 0.5
1.2125 0.5
1.1975 0.5
1.18 0.5
1.16 0.5
1.14 0.5
1.12 0.5
1.1 0.5
1.08 0.5
1.06 0.5
1.04 0.5
1.02 0.5
1 0.5
0.98 0.5
0.96 0.5
0.94 0.5
0.92 0.5
0.9 0.5
0.88 0.5
0.86 0.5
0.84 0.5
0.82 0.5
0.8 0.5
0.78 0.5
0.76 0.5
0.74 0.5
0.72 0.5
0.7 0.5
0.68 0.5
0.66 0.5
0.64 0.5
0.62 0.5
0.6 0.5
0.58 0.5
0.56 0.5
0.54 0.5
0.52 0.5
0.5 0.5
0.48 0.5
0.46 0.5
0.44 0.5
0.42 0.5
0.4 0.5
0.38 0.5
0.36 0.5
0.34 0.5
0.32 0.5
0.3025 0.5
0.2875 0.5
0.275 0.5
0.265 0.5
0.2575 0.5
0.2525 0.5
0.25 0.5
0.25 0.5
0.25 0.5
0.25 0.5
0.25 0.5
0.25 0.5
0.25 0.5
0.25 0.5
0.25 0.5
0.25 0.5
0.25 0.5
0.25 0.5
0.25 0.5
0.25 0.5
0.25 0.5
0.25 0.5
0.25 0.5
0.25 0.5
0.25 0.4975
0.25 0.4925
0.25 0.485
0.25 0.475
0.25 0.4625
0.25 0.4475
0.25 0.43
0.25 0.41
0.25 0.39
0.25 0.37
0.25 0.35
0.25 0.33
0.25 0.31
0.25 0.29
0.25 0.27
0.25 0.25
0.25 0.23
0.25 0.21
0.25 0.19
0.25 0.17
0.25 0.15
0.249999999999999 0.13
0.249999999999999 0.11
0.249999999999999 0.09
0.249999999999999 0.07
0.249999999999999 0.05
0.249999999999999 0.03
0.249999999999999 0.01
0.249999999999999 -0.00999999999999998
0.249999999999999 -0.03
0.249999999999999 -0.05
0.249999999999999 -0.07
0.249999999999999 -0.09
0.249999999999999 -0.11
0.249999999999999 -0.13
0.249999999999999 -0.15
0.249999999999999 -0.17
0.249999999999999 -0.19
0.249999999999999 -0.21
0.249999999999999 -0.23
0.249999999999999 -0.25
0.249999999999999 -0.27
0.249999999999999 -0.29
0.249999999999999 -0.31
0.249999999999999 -0.33
0.249999999999999 -0.35
0.249999999999999 -0.37
0.249999999999999 -0.39
0.249999999999999 -0.41
0.249999999999999 -0.43
0.249999999999999 -0.4475
0.249999999999999 -0.4625
0.249999999999999 -0.475
0.249999999999999 -0.485
0.249999999999999 -0.4925
0.249999999999999 -0.4975
0.249999999999999 -0.5
};
\addplot [semithick, blue, opacity=\opacityRef, forget plot]
table {%
0.25 -0.5
0.250242352485657 -0.499995172023773
0.253990322351456 -0.49972277879715
0.261390835046768 -0.499847114086151
0.270874321460724 -0.499948918819427
0.283222883939743 -0.500098884105682
0.298190176486969 -0.500260949134827
0.315810978412628 -0.500281751155853
0.335687339305878 -0.500123083591461
0.356977432966232 -0.500108897686005
0.376978427171707 -0.500151693820953
0.398091703653336 -0.500061929225922
0.418724954128265 -0.500089049339294
0.439139276742935 -0.500062763690948
0.459527283906937 -0.500048995018005
0.480037569999695 -0.499874711036682
0.500589698553085 -0.499665856361389
0.521509021520615 -0.500000298023224
0.541824251413345 -0.499867856502533
0.562534958124161 -0.500112473964691
0.583153933286667 -0.500399827957153
0.603774398565292 -0.500191152095795
0.624560564756393 -0.500116169452667
0.644621759653091 -0.500149190425873
0.665264517068863 -0.500240981578827
0.685735791921616 -0.500290334224701
0.706357568502426 -0.500250458717346
0.727081507444382 -0.500173926353455
0.747555404901505 -0.500285983085632
0.767845302820206 -0.500442504882812
0.788789600133896 -0.50038069486618
0.809245079755783 -0.500357210636139
0.829869657754898 -0.500441133975983
0.850336819887161 -0.500294148921967
0.87082913517952 -0.500352919101715
0.891225606203079 -0.500332176685333
0.912041395902634 -0.500328540802002
0.932275205850601 -0.500064849853516
0.952837020158768 -0.500167667865753
0.973520189523697 -0.500254452228546
0.99381759762764 -0.500257253646851
1.01461353898048 -0.500048577785492
1.03499582409859 -0.499990463256836
1.05555883049965 -0.499773621559143
1.07623353600502 -0.499589920043945
1.09668555855751 -0.499793410301208
1.11735036969185 -0.499622583389282
1.1375278532505 -0.499911367893219
1.15813234448433 -0.499930620193481
1.17876115441322 -0.499916732311249
1.19950905442238 -0.500144422054291
1.21877935528755 -0.500191569328308
1.23421171307564 -0.500408530235291
1.24789407849312 -0.500331223011017
1.25856068730354 -0.500192046165466
1.26669934391975 -0.500232696533203
1.27245715260506 -0.500159800052643
1.27560117840767 -0.500165045261383
1.2759704887867 -0.500150084495544
1.27662816643715 -0.500255525112152
1.27799239754677 -0.499996542930603
1.27767243981361 -0.500076949596405
1.27783253788948 -0.499964654445648
1.27786007523537 -0.499915242195129
1.27789774537086 -0.499777555465698
1.27772834897041 -0.499587833881378
1.27728107571602 -0.499495983123779
1.27676442265511 -0.500014066696167
1.27676239609718 -0.499819695949554
1.27686706185341 -0.499572098255157
1.27670457959175 -0.499633550643921
1.27655008435249 -0.499638378620148
1.27632036805153 -0.500023424625397
1.2762990295887 -0.49915999174118
1.27621617913246 -0.499664604663849
1.276237398386 -0.499211132526398
1.2761028110981 -0.495770514011383
1.27581503987312 -0.488080501556396
1.27560916543007 -0.478320598602295
1.2752528488636 -0.466180473566055
1.27482512593269 -0.451487183570862
1.27442601323128 -0.433787375688553
1.27391019463539 -0.413704812526703
1.27351823449135 -0.393004924058914
1.27305009961128 -0.372889071702957
1.2724235355854 -0.35242235660553
1.27200952172279 -0.331925362348557
1.27173176407814 -0.311462700366974
1.27122619748116 -0.291103363037109
1.27054849267006 -0.269901663064957
1.27007463574409 -0.249285519123077
1.26960673928261 -0.229130119085312
1.26916208863258 -0.208150550723076
1.26886847615242 -0.187305673956871
1.26851096749306 -0.166979566216469
1.26797273755074 -0.147110745310783
1.26762083172798 -0.126208648085594
1.26698842644691 -0.105765007436275
1.26670137047768 -0.0851579308509827
1.26617053151131 -0.0644799098372459
1.26555052399635 -0.0440981611609459
1.26479879021645 -0.0235636960715055
1.26431480050087 -0.00294940266758204
1.26364949345589 0.0172771532088518
1.26334574818611 0.0382870100438595
1.26267495751381 0.0586144998669624
1.26200023293495 0.0789402425289154
1.26144877076149 0.0999024957418442
1.260812073946 0.120713762938976
1.26045528054237 0.141005218029022
1.25980260968208 0.161369740962982
1.25884118676186 0.181962251663208
1.25797834992409 0.202723622322083
1.25748124718666 0.223193496465683
1.25700011849403 0.243719398975372
1.25618532299995 0.264317005872726
1.25571504235268 0.284654170274734
1.25493982434273 0.305257052183151
1.25410774350166 0.325364738702774
1.25365963578224 0.346344083547592
1.25316289067268 0.366345047950745
1.25259891152382 0.387077748775482
1.25184145569801 0.408231049776077
1.25102320313454 0.428626418113708
1.25030937790871 0.449001997709274
1.24984493851662 0.468338131904602
1.24940910935402 0.483503878116608
1.24923518300056 0.497090727090836
1.24884667992592 0.507957637310028
1.24853351712227 0.516094982624054
1.24838772416115 0.521762073040009
1.2474801838398 0.524605572223663
1.24781957268715 0.525340259075165
1.24817684292793 0.525983989238739
1.24804332852364 0.526594936847687
1.2480221092701 0.526154220104218
1.24790087342262 0.526210129261017
1.24769786000252 0.52637243270874
1.24775412678719 0.526106536388397
1.24788084626198 0.526262104511261
1.24741938710213 0.526058197021484
1.2478691637516 0.525658190250397
1.24765935540199 0.525655031204224
1.24719932675362 0.525730609893799
1.24701479077339 0.525881588459015
1.24704959988594 0.525917828083038
1.24659314751625 0.525856494903564
1.24690106511116 0.525883197784424
1.24646320939064 0.525879800319672
1.24621227383614 0.525857269763947
1.24218299984932 0.525663256645203
1.23546901345253 0.525382339954376
1.22635021805763 0.524864792823792
1.21417978405952 0.524192869663239
1.19938889145851 0.523527443408966
1.18148723244667 0.522652268409729
1.16205754876137 0.52182525396347
1.14081171154976 0.520870745182037
1.12048676609993 0.520029187202454
1.09919515252113 0.519083797931671
1.07848814129829 0.518167436122894
1.05860486626625 0.517185747623444
1.03832331299782 0.51635217666626
1.01784470677376 0.515279650688171
0.997328490018845 0.51435923576355
0.976930528879166 0.513395190238953
0.95618525147438 0.512254953384399
0.935608059167862 0.511538445949554
0.915779322385788 0.510573387145996
0.895369738340378 0.509559154510498
0.874380975961685 0.508422374725342
0.854405134916306 0.507478713989258
0.83343717455864 0.506443679332733
0.813209742307663 0.505517601966858
0.793067246675491 0.504580438137054
0.772470444440842 0.50353068113327
0.752405196428299 0.502603113651276
0.73154804110527 0.50152787566185
0.711222499608994 0.500749409198761
0.690432101488113 0.499812185764313
0.669496327638626 0.498862683773041
0.649145811796188 0.497973024845123
0.628402858972549 0.496975004673004
0.607732146978378 0.496122896671295
0.587115079164505 0.495291084051132
0.566598504781723 0.494380176067352
0.54657170176506 0.493571072816849
0.526306182146072 0.492704629898071
0.505607932806015 0.491710394620895
0.485096246004105 0.490875214338303
0.464402139186859 0.490010768175125
0.443914115428925 0.489032506942749
0.422871977090836 0.4880251288414
0.402512311935425 0.487133711576462
0.38147759437561 0.486097663640976
0.361050069332123 0.485319256782532
0.340638786554337 0.484560817480087
0.319819957017899 0.483466774225235
0.299469500780106 0.482504487037659
0.280118495225906 0.481419295072556
0.264800041913986 0.480844378471375
0.250570744276047 0.480024844408035
0.239589348435402 0.479747772216797
0.231683447957039 0.479289025068283
0.2254329174757 0.479012459516525
0.222306534647942 0.478909879922867
0.22162264585495 0.478809893131256
0.221240893006325 0.47890704870224
0.220177471637726 0.478442311286926
0.220410570502281 0.478564977645874
0.220373958349228 0.478448301553726
0.220201015472412 0.478124916553497
0.220560550689697 0.477959811687469
0.220717459917068 0.478027820587158
0.2211634516716 0.478230684995651
0.22162352502346 0.478743493556976
0.221603244543076 0.478077441453934
0.221703678369522 0.47814479470253
0.221666887402534 0.478280514478683
0.221778184175491 0.478223621845245
0.221757411956787 0.478252947330475
0.221860483288765 0.478240162134171
0.221849411725998 0.478330016136169
0.221856132149696 0.477818727493286
0.222075819969177 0.474061191082001
0.222496092319489 0.466727167367935
0.223220080137253 0.45715668797493
0.22390840947628 0.444712728261948
0.224718049168587 0.429908871650696
0.225650191307068 0.412630587816238
0.226908206939697 0.392686188220978
0.228342026472092 0.371807724237442
0.22949206829071 0.351671129465103
0.23087739944458 0.331296324729919
0.232483848929405 0.311126589775085
0.233603522181511 0.290607333183289
0.235042616724968 0.26905369758606
0.236381709575653 0.249107182025909
0.237565621733665 0.228035897016525
0.239009171724319 0.207920923829079
0.240579575300217 0.186336949467659
0.241844087839127 0.166888430714607
0.242945373058319 0.145569071173668
0.244471788406372 0.125356264412403
0.245824664831161 0.105234883725643
0.24735726416111 0.0846609175205231
0.248518422245979 0.0640843659639359
0.24992048740387 0.0432498194277287
0.251017093658447 0.0228986199945211
0.252392619848251 0.00158992991782725
0.253424048423767 -0.0180474165827036
0.254976034164429 -0.0387592799961567
0.256533950567245 -0.0599786005914211
0.25770428776741 -0.0795753821730614
0.259016573429108 -0.100400909781456
0.260392695665359 -0.120834670960903
0.261874169111252 -0.141849905252457
0.262989670038223 -0.161493703722954
0.264216035604477 -0.182992190122604
0.265355199575424 -0.203007325530052
0.266701340675354 -0.223851636052132
0.26801723241806 -0.244661912322044
0.269274801015854 -0.264745503664017
0.270528316497803 -0.28519144654274
0.271806359291077 -0.305943220853806
0.273088186979294 -0.326375782489777
0.274620413780212 -0.346836030483246
0.275769919157028 -0.366909682750702
0.277254641056061 -0.387656331062317
0.278390407562256 -0.408303290605545
0.279731065034866 -0.4293352663517
0.280902534723282 -0.449758112430573
0.282217025756836 -0.470382243394852
0.283249944448471 -0.489826828241348
0.284441590309143 -0.504981219768524
0.285270154476166 -0.51855993270874
0.285889327526093 -0.529626905918121
0.286391198635101 -0.537556767463684
0.286833018064499 -0.543513476848602
0.287053048610687 -0.546893537044525
0.287083655595779 -0.547254323959351
};
\addplot [semithick, red, dashed, forget plot]
table {%
0.25 -0.5
0.252558268640609 -0.499971215983182
0.257674800204547 -0.499913561943335
0.265349580546156 -0.499826822527541
0.275582586992214 -0.499710651911554
0.288373788282223 -0.499564571828345
0.303723144483551 -0.499387968751125
0.321630606880243 -0.499180090231924
0.342096117833915 -0.498940040397106
0.362561444756502 -0.498697137467738
0.383026586783779 -0.498451352589885
0.403491543037366 -0.498202656922469
0.423956312624558 -0.49795102163749
0.444420894638161 -0.497696417920264
0.464885288156327 -0.497438816969653
0.485349492242387 -0.497178189998295
0.505813505944689 -0.496914508232844
0.526277328296427 -0.4966477429142
0.546740958315482 -0.49637786529775
0.567204395004253 -0.496104846653605
0.587667637349495 -0.495828658266839
0.608130684322154 -0.495549271437727
0.628593534877201 -0.495266657481993
0.649056187953474 -0.494980787731044
0.669518642473505 -0.494691633532224
0.689980897343365 -0.49439916624905
0.710442951452497 -0.494103357261464
0.730904803673553 -0.493804177966079
0.751366452862231 -0.493501599776428
0.771827897857116 -0.493195594123215
0.792289137479511 -0.492886132454562
0.812750170533283 -0.492573186236266
0.833210995804693 -0.492256726952051
0.853671612062244 -0.491936726103822
0.874132018056508 -0.491613155211922
0.894592212519978 -0.491285985815386
0.915052194166895 -0.490955189472205
0.935511961693098 -0.49062073775958
0.955971513775856 -0.490282602274183
0.976430849073713 -0.489940754632423
0.996889966226327 -0.489595166470704
1.01734886385431 -0.489245809445691
1.03780754055907 -0.488892655234577
1.05826599492265 -0.488535675535346
1.07872422550757 -0.488174842067041
1.09918223085668 -0.487810126570036
1.11964000949299 -0.487441500806302
1.1400975599195 -0.48706893655968
1.16055488061908 -0.486692405636151
1.18101197005428 -0.486311879864113
1.20146882666719 -0.485927331094652
1.21936837110133 -0.485587306188302
1.23471068292331 -0.485293179242986
1.24749583154271 -0.485046149752488
1.25772387591864 -0.484847237098433
1.26539486430542 -0.484697275874471
1.27050883404033 -0.484596912039359
1.27306581137596 -0.484546599896135
1.2730658113589 -0.484546599895144
1.27305541155906 -0.484320911054072
1.2730238336127 -0.483870450269343
1.27294421772612 -0.483198810692961
1.27277401858219 -0.482315683800082
1.27245616754417 -0.481242223159833
1.27192174548361 -0.480017905211023
1.27109518948705 -0.478707989424014
1.26987791322201 -0.477382389353572
1.26861868448422 -0.476491262912181
1.26741310591075 -0.475930505649088
1.26634153930566 -0.475618318765773
1.26545420209816 -0.475475522024619
1.26477827798535 -0.475433459898834
1.26432554654506 -0.475438002031128
1.26409923112047 -0.475451102493
1.26409923107294 -0.475451102492838
1.26400510754809 -0.472893953643021
1.26381676720545 -0.467779675905825
1.26353397696839 -0.460108319156843
1.26315636493238 -0.449879963098326
1.26268342195054 -0.437094717113806
1.26211450389605 -0.421752720062896
1.26144883460045 -0.403854140014764
1.2606855094668 -0.383399173918339
1.25991927860775 -0.362944839063709
1.25915016511684 -0.342491133403313
1.25837819207707 -0.322038054872471
1.25760338256082 -0.301585601389565
1.25682575962964 -0.281133770856217
1.25604534633415 -0.260682561157473
1.25526216571386 -0.240231970161977
1.25447624079705 -0.219781995722154
1.25368759460061 -0.199332635674387
1.25289625012991 -0.178883887839198
1.25210223037863 -0.158435750021426
1.25130555832866 -0.137988220010405
1.25050625694995 -0.117541295580143
1.24970434920032 -0.0970949744895014
1.2488998580254 -0.0766492544823687
1.24809280635845 -0.0562041332878428
1.24728321712021 -0.0357596086204065
1.24647111321881 -0.0153156781801049
1.24565651754959 0.00512766034727754
1.244839452995 0.0255704092900399
1.24401994242446 0.0460125709903884
1.24319800869422 0.0664541478042603
1.24237367464724 0.0868951421011472
1.24154696311305 0.10733555626392
1.24071789690763 0.127775392688651
1.23988649883331 0.148214653784443
1.23905279167858 0.168653341973249
1.23821679821801 0.189091459689702
1.23737854121215 0.209529009380936
1.23653804340733 0.229965993506414
1.23569532753562 0.250402414537754
1.23485041631467 0.270838274958555
1.23400333244757 0.291273577264223
1.23315409862277 0.311708323961797
1.23230273751397 0.332142517569777
1.23144927177993 0.352576160617952
1.23059372406446 0.373009255647224
1.22973611699622 0.393441805209441
1.22887647318863 0.413873811867219
1.22801481523979 0.434305278193775
1.22715116573232 0.454736206772755
1.22628554723329 0.475166600198059
1.22552642790666 0.493042728464227
1.22487449160254 0.508364776283352
1.22433031870982 0.52113290203806
1.22389439056138 0.531347238274115
1.22356709317291 0.539007892120091
1.22334872031756 0.544114945632642
1.22323947593845 0.546668456066198
1.22323947590048 0.546668456066214
1.2230123362303 0.54664910964162
1.22255893634353 0.546599663012994
1.22188286957468 0.546493423413052
1.22099418533407 0.546288449972454
1.21991546682544 0.545929219980312
1.21868995460871 0.545349990456186
1.21739061938076 0.544480715471633
1.21610030698399 0.543230273381219
1.21526412455335 0.541961848316668
1.21476024252823 0.540762508740427
1.21448767411105 0.539705040879324
1.21435620536168 0.538833702471361
1.21429925392167 0.538171836563276
1.21427642250785 0.537729171531928
1.21429899313131 0.537497396084245
1.21429899310163 0.53749739608416
1.21173554316399 0.537351398434274
1.2066086675668 0.537059357198718
1.19891842689941 0.536621157703719
1.18866491799536 0.536036617118684
1.17584827376684 0.535305485487585
1.16046866297006 0.534427447211652
1.1425262899005 0.533402122984755
1.12202139401749 0.532229072183377
1.10151726555175 0.531054604210161
1.08101390213675 0.529878734481104
1.06051130139776 0.528701478429746
1.04000946095186 0.5275228515073
1.01950837840803 0.526342869182777
0.999008051367172 0.525161546943109
0.978508477422155 0.523978900293278
0.958009654157879 0.522794944756435
0.937511579151311 0.521609695874025
0.917014249971538 0.520423169205916
0.896517664179812 0.519235380330514
0.876021819329601 0.518046344844887
0.855526712966634 0.516856078364893
0.835032342628955 0.515664596525293
0.814538705846972 0.514471914979876
0.794045800143499 0.513278049401579
0.773553623033816 0.512083015482608
0.753062172025712 0.510886828934553
0.732571444619537 0.509689505488512
0.712081438308253 0.508491060895207
0.691592150577485 0.507291510925102
0.671103578905572 0.506090871368521
0.650615720763616 0.504889158035764
0.630128573615536 0.503686386757223
0.609642134918118 0.502482573383502
0.589156402121068 0.501277733785526
0.568671372667066 0.500071883854662
0.548187043991813 0.49886503950283
0.527703413524089 0.497657216662618
0.507220478685805 0.496448431287396
0.486738236892053 0.495238699351429
0.466256685551165 0.494028036849988
0.445775822064762 0.492816459799465
0.425295643827809 0.491603984237482
0.404816148228671 0.490390626223003
0.384337332649168 0.489176401836445
0.363859194464625 0.487961327179788
0.343381731043935 0.486745418376685
0.322904939749608 0.48552869157257
0.302428817937827 0.484311162934768
0.281953362958509 0.483092848652601
0.261478572155355 0.4818737649375
0.243563709029771 0.480806407637357
0.228208544470985 0.479891046434884
0.215412881797103 0.479127906813159
0.20517655624396 0.4785171732374
0.197499434529847 0.478058991869279
0.192381414497009 0.477753472810477
0.189822424830656 0.477600691872751
0.189822424856044 0.477600691872319
0.18984329041129 0.477361217804826
0.189891281488213 0.476888179768742
0.18998422190601 0.476195304214649
0.190156086957441 0.475302243681175
0.19046107666499 0.47423395694596
0.190973511005364 0.47302370084793
0.1917804370421 0.471720939032194
0.192990859960637 0.470375213510415
0.194255394688102 0.469435488118
0.195465103825654 0.468815206001496
0.196534139830494 0.468445902524195
0.19741290176854 0.468255645457359
0.19807765083531 0.468178447005524
0.198520413044486 0.468160326901788
0.198740901208049 0.468162462817004
0.198740901269321 0.46816246281638
0.198945440782965 0.465612270025241
0.199354508471781 0.460511864695189
0.199968076306772 0.452861197525416
0.200786100039452 0.442660189857331
0.201808520127061 0.429908733995519
0.203035263052485 0.414606693655117
0.20446624304152 0.396753904535665
0.206101364180888 0.376350175021588
0.207736171974856 0.355945830121864
0.209370679915717 0.335540874152152
0.211004901527408 0.315135311429408
0.212638850365535 0.294729146271967
0.214272540017396 0.27432238299961
0.215905984101993 0.253915025933645
0.217539196270065 0.233507079396979
0.219172190204101 0.2130985477142
0.220804979618368 0.192689435211644
0.222437578258934 0.172279746217481
0.224069999903693 0.15186948506179
0.225702258362386 0.131458656076634
0.227334367476629 0.111047263596139
0.228966341119943 0.0906353119565765
0.230598193197773 0.0702228054964383
0.232229937647521 0.0498097485565179
0.233861588438572 0.0293961454799898
0.235493159572323 0.00898200061249
0.237124665082211 -0.0114326816978031
0.238756119033746 -0.0318478971000879
0.240387535524538 -0.0522636412408578
0.24201892868433 -0.0726799097638194
0.243650312675029 -0.0930966983098097
0.245281701690739 -0.113514002516714
0.246913109957795 -0.133931818019382
0.248544551734793 -0.154350140449545
0.250176041312631 -0.174768965435729
0.251807593014535 -0.195188288603176
0.253439221196103 -0.215608105573753
0.255070940245335 -0.236028411965869
0.256702764582677 -0.256449203394391
0.258334708661049 -0.276870475470555
0.259966786965892 -0.29729222380188
0.261599014015202 -0.317714443992083
0.263231404359571 -0.338137131640988
0.264863972582227 -0.358560282344441
0.266496733299076 -0.378983891694222
0.26812970115874 -0.399407955277952
0.269762890842605 -0.419832468679009
0.271396317064857 -0.440257427476434
0.27302999457253 -0.460682827244845
0.27466393814555 -0.481108663554342
0.276093884549649 -0.498981648418472
0.277319746923832 -0.514301630525306
0.278341443563315 -0.527068481124652
0.279158900790641 -0.537282093189924
0.279772055404735 -0.544942380708266
0.280180856703051 -0.550049278099581
0.280385268072754 -0.552602739765039
};
\addplot [semithick, green, dash pattern=on 1pt off 3pt on 3pt off 3pt, forget plot]
table {%
0.25 -0.5
0.2525 -0.5
0.2575 -0.5
0.265 -0.5
0.275 -0.5
0.2875 -0.5
0.3025 -0.5
0.32 -0.5
0.34 -0.5
0.36 -0.5
0.38 -0.5
0.4 -0.5
0.42 -0.5
0.44 -0.5
0.46 -0.5
0.48 -0.5
0.5 -0.5
0.52 -0.5
0.54 -0.5
0.56 -0.5
0.58 -0.5
0.6 -0.5
0.62 -0.5
0.64 -0.5
0.66 -0.5
0.68 -0.5
0.7 -0.5
0.72 -0.5
0.74 -0.5
0.76 -0.5
0.78 -0.5
0.8 -0.5
0.82 -0.5
0.84 -0.5
0.86 -0.5
0.88 -0.5
0.9 -0.5
0.92 -0.5
0.94 -0.5
0.96 -0.5
0.98 -0.5
1 -0.5
1.02 -0.5
1.04 -0.5
1.06 -0.5
1.08 -0.5
1.1 -0.5
1.12 -0.5
1.14 -0.5
1.16 -0.5
1.18 -0.5
1.1975 -0.5
1.2125 -0.5
1.225 -0.5
1.235 -0.5
1.2425 -0.5
1.2475 -0.5
1.25 -0.5
1.25 -0.5
1.25 -0.5
1.25 -0.5
1.25 -0.5
1.25 -0.5
1.25 -0.5
1.25 -0.5
1.25 -0.5
1.25 -0.5
1.25 -0.5
1.25 -0.5
1.25 -0.5
1.25 -0.5
1.25 -0.5
1.25 -0.5
1.25 -0.5
1.25 -0.5
1.25 -0.4975
1.25 -0.4925
1.25 -0.485
1.25 -0.475
1.25 -0.4625
1.25 -0.4475
1.25 -0.43
1.25 -0.41
1.25 -0.39
1.25 -0.37
1.25 -0.35
1.25 -0.33
1.25 -0.31
1.25 -0.29
1.25 -0.27
1.25 -0.25
1.25 -0.23
1.25 -0.21
1.25 -0.19
1.25 -0.17
1.25 -0.15
1.25 -0.13
1.25 -0.11
1.25 -0.09
1.25 -0.07
1.25 -0.05
1.25 -0.03
1.25 -0.01
1.25 0.00999999999999998
1.25 0.03
1.25 0.05
1.25 0.07
1.25 0.09
1.25 0.11
1.25 0.13
1.25 0.15
1.25 0.17
1.25 0.19
1.25 0.21
1.25 0.23
1.25 0.25
1.25 0.27
1.25 0.29
1.25 0.31
1.25 0.33
1.25 0.35
1.25 0.37
1.25 0.39
1.25 0.41
1.25 0.43
1.25 0.4475
1.25 0.4625
1.25 0.475
1.25 0.485
1.25 0.4925
1.25 0.4975
1.25 0.5
1.25 0.5
1.25 0.5
1.25 0.5
1.25 0.5
1.25 0.5
1.25 0.5
1.25 0.5
1.25 0.5
1.25 0.5
1.25 0.5
1.25 0.5
1.25 0.5
1.25 0.5
1.25 0.5
1.25 0.5
1.25 0.5
1.25 0.5
1.2475 0.5
1.2425 0.5
1.235 0.5
1.225 0.5
1.2125 0.5
1.1975 0.5
1.18 0.5
1.16 0.5
1.14 0.5
1.12 0.5
1.1 0.5
1.08 0.5
1.06 0.5
1.04 0.5
1.02 0.5
1 0.5
0.98 0.5
0.96 0.5
0.94 0.5
0.92 0.5
0.9 0.5
0.88 0.5
0.86 0.5
0.84 0.5
0.82 0.5
0.8 0.5
0.78 0.5
0.76 0.5
0.74 0.5
0.72 0.5
0.7 0.5
0.68 0.5
0.66 0.5
0.64 0.5
0.62 0.5
0.6 0.5
0.58 0.5
0.56 0.5
0.54 0.5
0.52 0.5
0.5 0.5
0.48 0.5
0.46 0.5
0.44 0.5
0.42 0.5
0.4 0.5
0.38 0.5
0.36 0.5
0.34 0.5
0.32 0.5
0.3025 0.5
0.2875 0.5
0.275 0.5
0.265 0.5
0.2575 0.5
0.2525 0.5
0.25 0.5
0.25 0.5
0.25 0.5
0.25 0.5
0.25 0.5
0.25 0.5
0.25 0.5
0.25 0.5
0.25 0.5
0.25 0.5
0.25 0.5
0.25 0.5
0.25 0.5
0.25 0.5
0.25 0.5
0.25 0.5
0.25 0.5
0.25 0.5
0.25 0.4975
0.25 0.4925
0.25 0.485
0.25 0.475
0.25 0.4625
0.25 0.4475
0.25 0.43
0.25 0.41
0.25 0.39
0.25 0.37
0.25 0.35
0.25 0.33
0.25 0.31
0.25 0.29
0.25 0.27
0.25 0.25
0.25 0.23
0.25 0.21
0.25 0.19
0.25 0.17
0.25 0.15
0.249999999999999 0.13
0.249999999999999 0.11
0.249999999999999 0.09
0.249999999999999 0.07
0.249999999999999 0.05
0.249999999999999 0.03
0.249999999999999 0.01
0.249999999999999 -0.00999999999999998
0.249999999999999 -0.03
0.249999999999999 -0.05
0.249999999999999 -0.07
0.249999999999999 -0.09
0.249999999999999 -0.11
0.249999999999999 -0.13
0.249999999999999 -0.15
0.249999999999999 -0.17
0.249999999999999 -0.19
0.249999999999999 -0.21
0.249999999999999 -0.23
0.249999999999999 -0.25
0.249999999999999 -0.27
0.249999999999999 -0.29
0.249999999999999 -0.31
0.249999999999999 -0.33
0.249999999999999 -0.35
0.249999999999999 -0.37
0.249999999999999 -0.39
0.249999999999999 -0.41
0.249999999999999 -0.43
0.249999999999999 -0.4475
0.249999999999999 -0.4625
0.249999999999999 -0.475
0.249999999999999 -0.485
0.249999999999999 -0.4925
0.249999999999999 -0.4975
0.249999999999999 -0.5
};
\end{axis}%
\end{tikzpicture}
%
    % This file was created with tikzplotlib v0.10.1.
\begin{tikzpicture}
\hspace{4pt}
\begin{axis}[
width = .405\textwidth,
height = \heightRealDataError,
at={(0.0, 0)},
legend cell align={left},
legend columns = 2,
legend style={
  fill opacity=1,
  draw opacity=1,
  text opacity=1,
  at={(1,1)},
  anchor=north east,
  column sep = 0.15cm
  %draw=lightgray204
},
xlabel = {time (s)}, 
ylabel = {error},
%xmin=-1.24, xmax=26.04,
xmin=-0, xmax=25,
%xticklabel style={xshift=-.05cm},
ymin=-1e-05, ymax=0.005,
%ylabel style={yshift=-.1cm},
xshift=0.19cm
]
\addplot [line width = \linewidthErrorC, color = MaxError]
table {%
0 0
0.1 0.0218111330831636
0.2 0.0166472248805543
0.3 0.00237514206843605
0.4 0.00182170178453521
0.5 0.00164579073236717
0.6 0.00196000337950324
0.7 0.00113874255138009
0.8 0.00118168943661031
0.9 0.001160172931261
1 0.00122910121785218
1.1 0.00111081766217359
1.2 0.000858986488234555
1.3 0.00110773844630138
1.4 0.00119802088284138
1.5 0.00116935190467242
1.6 0.0019776421881197
1.7 0.00188687710664118
1.8 0.00093771466741924
1.9 0.00112615450499511
2 0.0011780655172553
2.1 0.00179470409992574
2.2 0.00147525362428379
2.3 0.00139894744667057
2.4 0.000819139504006065
2.5 0.00120920414319461
2.6 0.00140535390933018
2.7 0.00199599238447611
2.8 0.00185120341359877
2.9 0.000862463783887557
3 0.000601878517427685
3.1 0.000937473825799147
3.2 0.000696142325656951
3.3 0.00154049779804585
3.4 0.00150571912218871
3.5 0.000634425814140833
3.6 0.000667721772331938
3.7 0.000744632704745629
3.8 0.000615193939489468
3.9 0.00082220426978209
4 0.000848794282432759
4.1 0.00114440624315436
4.2 0.000698643711936882
4.3 0.000772726978441963
4.4 0.000664073251126432
4.5 0.000703441422968412
4.6 0.000918802170107213
4.7 0.000877287668899846
4.8 0.000692255669268265
4.9 0.000877904909881829
5 0.000555316114073289
5.1 0.000817334853110838
5.2 0.000835604530605776
5.3 0.0010664845591555
5.4 0.00106355815259028
5.5 0.000810109319470303
5.6 0.00103923566666399
5.7 0.000710252079863388
5.8 0.000673651708606017
5.9 0.00077613827449373
6 0.000808478237266405
6.1 0.000771293014645576
6.2 0.000820404289902603
6.3 0.000614788047022066
6.4 0.000777718111615948
6.5 0.000759756919552243
6.6 0.000628088093348934
6.7 0.000728338126275741
6.8 0.000693358227370197
6.9 0.000749205417638388
7 0.00118091528993192
7.1 0.0011590752213565
7.2 0.00115185915938922
7.3 0.000819431266890017
7.4 0.00110154293992482
7.5 0.00102460405937304
7.6 0.000799398114850884
7.7 0.000760792172164481
7.8 0.000846896035761227
7.9 0.000742296489287405
8 0.00122279941114923
8.1 0.00137293261394708
8.2 0.000810226480900624
8.3 0.000697674976649158
8.4 0.000745288842816009
8.5 0.00119846841920537
8.6 0.000740240646524749
8.7 0.000772921588417582
8.8 0.000714659783497115
8.9 0.00099181165354226
9 0.000739289932915314
9.1 0.000692945989677082
9.2 0.000711016777556627
9.3 0.00149795434969083
9.4 0.00139886349669039
9.5 0.000822840561000893
9.6 0.000744540838832495
9.7 0.000847148301396
9.8 0.000912370398483527
9.9 0.000746554300264008
10 0.00132921496233553
10.1 0.00148171118167327
10.2 0.000785310296790808
10.3 0.000809779553151078
10.4 0.000770628462325511
10.5 0.000910059184315224
10.6 0.000778920553660853
10.7 0.00184995117223489
10.8 0.0012298482705877
10.9 0.00119022435436737
11 0.00164299566278309
11.1 0.00202733599503063
11.2 0.000838711797071759
11.3 0.00217386104988614
11.4 0.0012333802140042
11.5 0.00189238549807984
11.6 0.0010879664575399
11.7 0.00230606401196725
11.8 0.0015766531716285
11.9 0.000618100910581359
12 0.00121613108246024
12.1 0.00128013018491497
12.2 0.00110259818799942
12.3 0.00130508326151106
12.4 0.000875945817304124
12.5 0.00149161012416394
12.6 0.000845522104393796
12.7 0.00144503109514916
12.8 0.000878380558363015
12.9 0.00111058310280176
13 0.00125020365462378
13.1 0.00117653798182924
13.2 0.00167261761307526
13.3 0.000706254580971073
13.4 0.00124848406152861
13.5 0.00100861398260138
13.6 0.00129079090248657
13.7 0.00171234480910751
13.8 0.00175539592106978
13.9 0.000890539130872647
14 0.00113268966958203
14.1 0.00104536475711799
14.2 0.000839373409885747
14.3 0.000855888971494867
14.4 0.000860975958597921
14.5 0.000962882131407204
14.6 0.000967954458361552
14.7 0.00110905231084328
14.8 0.00106372424521503
14.9 0.000826308320354236
15 0.00101990266120665
15.1 0.00115604399624758
15.2 0.00164676320413975
15.3 0.000929545497489082
15.4 0.00116936951785769
15.5 0.00124384607253603
15.6 0.00171070392573958
15.7 0.00106765678032582
15.8 0.00118231574128314
15.9 0.00141131087891961
16 0.000680310797916312
16.1 0.0012520256879146
16.2 0.00104163868575545
16.3 0.000654455363419246
16.4 0.00103432289092331
16.5 0.000818533939149639
16.6 0.0010126848606802
16.7 0.000540852645001082
16.8 0.00112483285466042
16.9 0.000561379653397216
17 0.000863098135906454
17.1 0.00100814126578749
17.2 0.000792091094689208
17.3 0.000801836501969912
17.4 0.000756639942120807
17.5 0.00101485717720009
17.6 0.00131399204623153
17.7 0.000834961685914142
17.8 0.0008664539901077
17.9 0.000701567022655454
18 0.000684086756828024
18.1 0.000823827113785157
18.2 0.000923066866633415
18.3 0.00114420130416271
18.4 0.000974713180786147
18.5 0.000975679347363912
18.6 0.000850070834138813
18.7 0.000836566984504858
18.8 0.000955138947456466
18.9 0.000917611817943651
19 0.00108634755477606
19.1 0.00112730293463918
19.2 0.00108351929691235
19.3 0.00117049437419037
19.4 0.000807565662042005
19.5 0.000873083218083089
19.6 0.000865410003871241
19.7 0.000913135349093921
19.8 0.000760760521778913
19.9 0.000935465583664372
20 0.000820498552890802
20.1 0.000917520844378701
20.2 0.000771878891951089
20.3 0.000997368640354455
20.4 0.00184888836018966
20.5 0.00166946068983158
20.6 0.00112651446631485
20.7 0.00107210024386617
20.8 0.000915537878561935
20.9 0.000910702761685879
21 0.00115334968575375
21.1 0.000890204377010191
21.2 0.000804008786428191
21.3 0.00106454290202765
21.4 0.000795773304781374
21.5 0.00125189571371278
21.6 0.00104055052614912
21.7 0.00108006238185511
21.8 0.000910216949063225
21.9 0.000955947709258221
22 0.00130232038045185
22.1 0.000648755411043843
22.2 0.00136544416518835
22.3 0.00097494877579632
22.4 0.000582040725109847
22.5 0.000910709067860922
22.6 0.000738551009741783
22.7 0.000998407269776071
22.8 0.000969878703478162
22.9 0.00168146758319692
23 0.000751994745763455
23.1 0.00100768668364692
23.2 0.00102413963640494
23.3 0.00083405054556639
23.4 0.00134840860550717
23.5 0.00117395378508981
23.6 0.000935592207059303
23.7 0.000839910285277137
23.8 0.00214663118502847
23.9 0.00219298516068275
24 0.00232767289758298
24.1 0.00149881087644298
24.2 0.00104957540397154
24.3 0.00221402854508365
24.4 0.00124645959062284
24.5 0.000953242754624308
24.6 0.00229528977055165
24.7 0.00222197648060372
24.8 0.00121233139270325
};\label{plot:MaxErrorEight}
%\addlegendentry{max. err.}
\addlegendentry{$e_{\text{max}}$}
\addplot [line width = \linewidthErrorC, color = AvgErr]
table {%
0 0
0.1 0.0178480138587105
0.2 0.00346136960533238
0.3 0.00168251853907658
0.4 0.00112301906284021
0.5 0.000832727710936349
0.6 0.000510055158639128
0.7 0.00039842563906734
0.8 0.000393303143266708
0.9 0.000459975456877203
1 0.000406231109289026
1.1 0.000555252553064137
1.2 0.000430449743368592
1.3 0.000508456284717768
1.4 0.000515604725540065
1.5 0.000412531841441262
1.6 0.000548886118901324
1.7 0.000572254182847836
1.8 0.000413668561277504
1.9 0.000350319824151475
2 0.000466797808874442
2.1 0.000508889721853971
2.2 0.000500003219953263
2.3 0.000445229705934446
2.4 0.000310428674926053
2.5 0.000394313043558444
2.6 0.000406539918632542
2.7 0.000460363173280814
2.8 0.000448415712219654
2.9 0.000384005029749448
3 0.000316303124456396
3.1 0.000425818101627574
3.2 0.000371456514660979
3.3 0.000462551244798035
3.4 0.000436692489068627
3.5 0.000381163138117614
3.6 0.00039176839023812
3.7 0.000406415030087471
3.8 0.00037487122958732
3.9 0.000501486518207436
4 0.000410233138983916
4.1 0.000520552759753522
4.2 0.00040432676457731
4.3 0.000469595926971492
4.4 0.000388266973502595
4.5 0.000410571129221132
4.6 0.000493032029914097
4.7 0.000369370421270014
4.8 0.000472103466127895
4.9 0.000472659554082058
5 0.000402803669194487
5.1 0.000420398327752655
5.2 0.000381906627062597
5.3 0.000489934800507741
5.4 0.000484875759501379
5.5 0.000350414149606082
5.6 0.000460121261847649
5.7 0.000348656066347556
5.8 0.000289323376865398
5.9 0.000409936315273775
6 0.00046787049413993
6.1 0.000369304919060326
6.2 0.000446892262918518
6.3 0.000407756887910023
6.4 0.000449127108704602
6.5 0.000341993446420941
6.6 0.000389066866927044
6.7 0.000395046575115106
6.8 0.000356410059560883
6.9 0.00042916620920751
7 0.000526995891733802
7.1 0.000431426161140984
7.2 0.000429585333430335
7.3 0.000502138091763193
7.4 0.000478222509314336
7.5 0.000559873957220904
7.6 0.000438081908879151
7.7 0.00043099523650966
7.8 0.000485906921630406
7.9 0.000448709840090899
8 0.000456239863236378
8.1 0.000546282285420562
8.2 0.000342275242033338
8.3 0.000469713719297109
8.4 0.000441286446444214
8.5 0.000495415053529066
8.6 0.000446450024972111
8.7 0.000395001223000835
8.8 0.000459130673366819
8.9 0.000426615865692915
9 0.000407747417659968
9.1 0.000429676976032896
9.2 0.000352723823080096
9.3 0.000472008980913212
9.4 0.000513556058988567
9.5 0.00039848666886367
9.6 0.000378136393244121
9.7 0.000363628479878839
9.8 0.000368575564416556
9.9 0.000350282378496879
10 0.000486508656354637
10.1 0.000387293499335707
10.2 0.000369794272236367
10.3 0.000393579468817532
10.4 0.000386048597799702
10.5 0.000363003541448739
10.6 0.000352980111642184
10.7 0.000477367395163282
10.8 0.000394622629634395
10.9 0.000364558190993254
11 0.000409089240999483
11.1 0.000472634520783616
11.2 0.000270291584728241
11.3 0.000455967871704817
11.4 0.000363674070763994
11.5 0.000470122167192538
11.6 0.000348864229052387
11.7 0.000549366138935868
11.8 0.000409974164874536
11.9 0.000325056663457393
12 0.000466858269415193
12.1 0.000558189816159613
12.2 0.000369011016748082
12.3 0.00049120527176261
12.4 0.000412259493327199
12.5 0.000511784845452854
12.6 0.000418524372737589
12.7 0.000445136250651688
12.8 0.000432996129376773
12.9 0.000379596788669839
13 0.000444876207956507
13.1 0.000415344756153192
13.2 0.000487818652709673
13.3 0.000342253485153672
13.4 0.000543839373042666
13.5 0.000418155145669528
13.6 0.000447382943225051
13.7 0.000398282718790385
13.8 0.000482306053659726
13.9 0.000387381493957682
14 0.000362125104589586
14.1 0.000416322086396805
14.2 0.000423200480849395
14.3 0.000479641963938867
14.4 0.000376648615696125
14.5 0.000453031813520734
14.6 0.000439166948181694
14.7 0.000452266151005869
14.8 0.000524598956297956
14.9 0.000405980821282593
15 0.000431857558438903
15.1 0.000511277443467781
15.2 0.000477582582202853
15.3 0.000388135186018543
15.4 0.000401566719309941
15.5 0.000446522271405705
15.6 0.000503569354327938
15.7 0.000346293264076215
15.8 0.000400164149454075
15.9 0.000457355318216753
16 0.000351460846695983
16.1 0.000416329965784737
16.2 0.000307078470562202
16.3 0.000337597155558062
16.4 0.000423300138727769
16.5 0.000274128730662479
16.6 0.000372023919929724
16.7 0.000274101619856441
16.8 0.000351328322707309
16.9 0.000300200108256463
17 0.000329771789689699
17.1 0.000329839822688807
17.2 0.000330350917527844
17.3 0.000294142621253281
17.4 0.000373085039688761
17.5 0.000359044034294392
17.6 0.000391947828394907
17.7 0.000414418912398849
17.8 0.000303851429429684
17.9 0.000346581854604036
18 0.00031640577176365
18.1 0.000303341624172758
18.2 0.000401556537244306
18.3 0.000423566518241002
18.4 0.000464025416571
18.5 0.000407249068921094
18.6 0.000320415946483372
18.7 0.000289999772282618
18.8 0.00038411575699688
18.9 0.000356588302546137
19 0.000370467764026846
19.1 0.000420119793670597
19.2 0.000386689379562959
19.3 0.000422341091129088
19.4 0.000291526663758252
19.5 0.000271109287898287
19.6 0.000253238999638801
19.7 0.000298875123203227
19.8 0.000347245253199604
19.9 0.000322733729003605
20 0.000329835995168627
20.1 0.000290223530729577
20.2 0.000335551201198197
20.3 0.000376481898032925
20.4 0.00046458626980805
20.5 0.000484482708865416
20.6 0.000402459820004954
20.7 0.000445126201743358
20.8 0.000461805980666489
20.9 0.000373072512368866
21 0.000467752076588497
21.1 0.000430563844468952
21.2 0.000435266356434795
21.3 0.00040836033747623
21.4 0.000428583503506233
21.5 0.000472270633043583
21.6 0.000401556867256419
21.7 0.000382620727688219
21.8 0.000424168736243815
21.9 0.000428736733461563
22 0.000410154826258137
22.1 0.000344937418226348
22.2 0.000398575762192661
22.3 0.000382415433816121
22.4 0.000348613270782797
22.5 0.00032815032326095
22.6 0.000361925036810857
22.7 0.000374005753111173
22.8 0.000376402951282942
22.9 0.000374583268279
23 0.000325324934572279
23.1 0.00039975599296835
23.2 0.000469548810699636
23.3 0.000442858059960306
23.4 0.000414090548263624
23.5 0.000390415846222797
23.6 0.00039288643650547
23.7 0.000451894185899233
23.8 0.000543890945595062
23.9 0.000658471065129718
24 0.000614187978679585
24.1 0.000513351091487459
24.2 0.000463115324423361
24.3 0.000632837339420454
24.4 0.000531969845428524
24.5 0.000589684477361628
24.6 0.000605776008354775
24.7 0.000628512252509024
24.8 0.000434691345317759
};\label{plot:AvgErrorEight}
%\addlegendentry{avg. err.}
\addlegendentry{$e_{\text{avg}}$}
\addplot [line width = \linewidthErrorStdVar, color = Koopman] % densely dashdotted]
table {%
0 0
0.1 0.0190491623042049
0.2 0.00704799793596951
0.3 0.00209433207062622
0.4 0.00141757715673368
0.5 0.00126606084347488
0.6 0.000960984675418757
0.7 0.000641400990496132
0.8 0.00066883080750079
0.9 0.000717208566771342
1 0.000726985037606839
1.1 0.000798676786369337
1.2 0.000653141302798547
1.3 0.000756874725582265
1.4 0.000759747238749241
1.5 0.000659551314029731
1.6 0.00100469887992618
1.7 0.00104661852248414
1.8 0.000634745004462205
1.9 0.000601637456269032
2 0.000744077690348807
2.1 0.000936294595837474
2.2 0.000911664389946073
2.3 0.000779135713400769
2.4 0.000549072589617891
2.5 0.000686711600912299
2.6 0.000720681459784784
2.7 0.000949195029965004
2.8 0.000845793796689429
2.9 0.00059050598788353
3 0.000479733914386366
3.1 0.000689694619494556
3.2 0.00052726287327279
3.3 0.000823669505887278
3.4 0.000746468967988175
3.5 0.000522660558264304
3.6 0.000596470542404951
3.7 0.00057935387432754
3.8 0.000552009051272775
3.9 0.000671367465401712
4 0.000643582748701368
4.1 0.000785011477059047
4.2 0.000560775512567449
4.3 0.000613629363907817
4.4 0.000539554672064954
4.5 0.00059997072697477
4.6 0.000736118025849699
4.7 0.000555668625347132
4.8 0.000643223978540583
4.9 0.000645120084680304
5 0.000500573044597864
5.1 0.000624814278455164
5.2 0.000602188841672137
5.3 0.000726969353428992
5.4 0.000732803060714139
5.5 0.000526786313767674
5.6 0.000686239836265107
5.7 0.000526426780774891
5.8 0.00046666234411077
5.9 0.000559075909999121
6 0.00061391540606144
6.1 0.000510359012800975
6.2 0.000649285063093051
6.3 0.000548234602110887
6.4 0.000645037651629012
6.5 0.000543136218786295
6.6 0.000503403229479756
6.7 0.000533273733853097
6.8 0.000534230701530469
6.9 0.000619267267535731
7 0.000743757749824733
7.1 0.000675381957968618
7.2 0.000646276695772685
7.3 0.000663292072722796
7.4 0.000694274744761798
7.5 0.000794629969010058
7.6 0.000613799294867365
7.7 0.000640852096864321
7.8 0.00066547277057286
7.9 0.000611785820290137
8 0.000720057143023211
8.1 0.000816686704953236
8.2 0.000508799338256045
8.3 0.000607513023361248
8.4 0.000612291875276559
8.5 0.000775763671470604
8.6 0.000612188971036919
8.7 0.000553563807920814
8.8 0.000621905387896033
8.9 0.000659109432182751
9 0.000551601839298667
9.1 0.000620893652175312
9.2 0.000497737426112242
9.3 0.000766705295092424
9.4 0.00080693001109161
9.5 0.000601963945476729
9.6 0.000543125769688944
9.7 0.000589988772867731
9.8 0.000553228935050132
9.9 0.000557907471136746
10 0.000831444669174632
10.1 0.000740643326633164
10.2 0.000526815249595294
10.3 0.000598604560258954
10.4 0.000567481981642056
10.5 0.000586054201195979
10.6 0.000518233360285903
10.7 0.000890074981821544
10.8 0.000673642318372208
10.9 0.000650398447281175
11 0.000792749094923946
11.1 0.000977880371079086
11.2 0.000461879857413873
11.3 0.00100417593850639
11.4 0.000616496389142681
11.5 0.000929134098199974
11.6 0.00061355858435514
11.7 0.00105814811647695
11.8 0.000781654144705779
11.9 0.000491279108492736
12 0.000747485905768025
12.1 0.000822585800409748
12.2 0.000621448204458177
12.3 0.000769011509197561
12.4 0.000647782961953572
12.5 0.000842319601431871
12.6 0.000623350923600542
12.7 0.000746533840858154
12.8 0.000654978789557511
12.9 0.000634524700572168
13 0.000701602799976788
13.1 0.000681762106502567
13.2 0.000832581537980088
13.3 0.000541192616943161
13.4 0.000837912469927652
13.5 0.000653210949514678
13.6 0.000746920979754525
13.7 0.000839372569722776
13.8 0.000861769412222676
13.9 0.000591991100559362
14 0.000626435891966447
14.1 0.000646156553307237
14.2 0.000622022785708915
14.3 0.000649987545280624
14.4 0.000551577450550621
14.5 0.000677100499293426
14.6 0.00068094352182683
14.7 0.00068606814308832
14.8 0.000722305566344855
14.9 0.000610689440812786
15 0.00064545596406583
15.1 0.000832824889111897
15.2 0.000863432543917598
15.3 0.000607931148874746
15.4 0.000681425773467903
15.5 0.000736693843302295
15.6 0.000847485238137124
15.7 0.00060706670052829
15.8 0.000726666129415896
15.9 0.000808291342990077
16 0.000502339513068011
16.1 0.000668091195694862
16.2 0.00055875525650616
16.3 0.000501236295340802
16.4 0.000678281152339475
16.5 0.000457789636647539
16.6 0.00061942054517356
16.7 0.000438487568638623
16.8 0.00060783193890024
16.9 0.00045381109571069
17 0.000517924223929636
17.1 0.000549043180377354
17.2 0.000506996848825398
17.3 0.000496346673167572
17.4 0.000568318301756044
17.5 0.00063807028379851
17.6 0.000692705097697475
17.7 0.000622927400017558
17.8 0.000540944164608012
17.9 0.0005320688714957
18 0.000497834412193976
18.1 0.000476286042149753
18.2 0.000640553296915761
18.3 0.000746550881252391
18.4 0.000741187358285568
18.5 0.000629851435742356
18.6 0.000539604046365171
18.7 0.00051569523079599
18.8 0.000615845293594637
18.9 0.000609490255031382
19 0.000660391458538922
19.1 0.000694009033884897
19.2 0.000728441375060119
19.3 0.000689015111416167
19.4 0.000463447465891365
19.5 0.000469330004903184
19.6 0.000464847490359082
19.7 0.000504010071642718
19.8 0.000545738944002571
19.9 0.000540012333695694
20 0.000532721388344323
20.1 0.000508567248209636
20.2 0.000493810946005179
20.3 0.000602200887856431
20.4 0.000876491982517506
20.5 0.000856026577168347
20.6 0.000675643954390011
20.7 0.000664216059162964
20.8 0.000665924981696889
20.9 0.000575347268177347
21 0.000690067797652509
21.1 0.000621181605321642
21.2 0.000609344606379007
21.3 0.00063491509839819
21.4 0.00059085629890106
21.5 0.000720487744503473
21.6 0.000617710305254994
21.7 0.000624787291581173
21.8 0.000695000704547913
21.9 0.000657199970251283
22 0.000702514541826663
22.1 0.000514260290490728
22.2 0.000733410729314145
22.3 0.000634392992172463
22.4 0.000524889539428506
22.5 0.000521319453181333
22.6 0.000566973256150851
22.7 0.000602670181775924
22.8 0.000648425372702121
22.9 0.000749539161431323
23 0.000507764692119787
23.1 0.000621377325213472
23.2 0.000741372481497649
23.3 0.000681557697444227
23.4 0.000750424692364996
23.5 0.000647469246695626
23.6 0.000630187916473506
23.7 0.000647260038736539
23.8 0.00105240629026901
23.9 0.00115693686448696
24 0.00116904977700385
24.1 0.000830646641539638
24.2 0.000731951067800125
24.3 0.0011105309084948
24.4 0.00077699742736547
24.5 0.000834306068700037
24.6 0.00113043197927471
24.7 0.00114667297698243
24.8 0.0007687305639908
};\label{plot:hErrorEight}
\addlegendentry{$\sigma$}
\end{axis}
\begin{axis}[
width = .405\textwidth,
height = \heightRealDataError,
at={(.354\textwidth, 0)},
%legend cell align={left},
legend columns = 1,
legend style={
  fill opacity=1,
  draw opacity=1,
  text opacity=1,
  at={(1,1.1)},
  anchor=north east,
  column sep = 0.2cm
  %draw=lightgray204
},
xlabel = {time (s)}, 
%ylabel = {$x_2$ position (m)},
%xmin=-1.24, xmax=26.04,
xmin=-0, xmax=28,
%xticklabel style={xshift=-.05cm},
ymin=-0.000317420907404805, ymax=0.00666583905550091,
%ylabel style={yshift=-.1cm},
%xshift=0.012cm
]
\addplot [line width = \linewidthErrorC, color = MaxError]
table {%
0 0
0.1 0.00243030151026045
0.2 0.00191673452960251
0.3 0.000887512761918704
0.4 0.00147929005628157
0.5 0.00173792875028588
0.6 0.000984666627024665
0.7 0.00138936251650066
0.8 0.00229776125775171
0.9 0.00124260889840826
1 0.00189845088152088
1.1 0.00101863278442778
1.2 0.00224207344283344
1.3 0.0022258720196888
1.4 0.0021716530098494
1.5 0.00188511936984768
1.6 0.00193513933205989
1.7 0.00179741673935735
1.8 0.000593499900620881
1.9 0.000710950602873612
2 0.000859365209289473
2.1 0.0020469892391188
2.2 0.000764688589694938
2.3 0.00214234926890111
2.4 0.00189678924696314
2.5 0.000563620497323481
2.6 0.00219978509915878
2.7 0.00221348212684458
2.8 0.000810499743372499
2.9 0.00225232865072449
3 0.000809805997244321
3.1 0.00178150792933667
3.2 0.00192789585074392
3.3 0.000782017002823046
3.4 0.0019620956325225
3.5 0.000824355191515104
3.6 0.00247622572403226
3.7 0.00185794230652155
3.8 0.00178629763675698
3.9 0.00214665061286236
4 0.00231085572237006
4.1 0.000807539876157421
4.2 0.00101025433243081
4.3 0.00243985277031156
4.4 0.00267844233203417
4.5 0.000632158133444315
4.6 0.00236179027461731
4.7 0.00216664205742035
4.8 0.00229167212816204
4.9 0.000887367341716449
5 0.000816247027397059
5.1 0.00180220105999559
5.2 0.00143405782346682
5.3 0.00114237921381472
5.4 0.0016905605643704
5.5 0.000872297501729055
5.6 0.000786497432161249
5.7 0.000923499840409351
5.8 0.000943175467467437
5.9 0.000768974558825933
6 0.00139514376858888
6.1 0.00080717308555432
6.2 0.00115309380050881
6.3 0.00126300518599934
6.4 0.00136787680624442
6.5 0.00151434580993336
6.6 0.00179654804122978
6.7 0.00187071119531102
6.8 0.00147605557992692
6.9 0.00121232198022002
7 0.000992209322900209
7.1 0.000727030637053469
7.2 0.000532347953306163
7.3 0.00089391807343616
7.4 0.000511369134792497
7.5 0.002523359497698
7.6 0.00210656671591806
7.7 0.000597782799362139
7.8 0.0011630886536476
7.9 0.00177604083530233
8 0.000980652255753125
8.1 0.00200732865558027
8.2 0.00225753547755002
8.3 0.00132908370495918
8.4 0.00185179998061135
8.5 0.0014528064799741
8.6 0.00203479128465893
8.7 0.000726661022930454
8.8 0.000683253822841503
8.9 0.00193014234571764
9 0.00202448586206326
9.1 0.00100440579503102
9.2 0.00180037667394974
9.3 0.00167056279268262
9.4 0.00169140486977675
9.5 0.00212435172756484
9.6 0.00164306240893193
9.7 0.00198166297198967
9.8 0.00215660553635352
9.9 0.00103633750949908
10 0.00213614381702186
10.1 0.00277807559546614
10.2 0.00260630113932305
10.3 0.00271626359823287
10.4 0.00200373018716632
10.5 0.00210952693212198
10.6 0.00238031225727075
10.7 0.00191676998334233
10.8 0.00196454227784894
10.9 0.00209056192580351
11 0.00213089736853111
11.1 0.00171399170471313
11.2 0.000779955871224051
11.3 0.00198517913518675
11.4 0.00159932840592942
11.5 0.00204663745300815
11.6 0.00108185929836322
11.7 0.0025654797646767
11.8 0.00244275181842232
11.9 0.0010207147104581
12 0.000873297947506
12.1 0.0010207906807331
12.2 0.00186393047835346
12.3 0.00183776738581444
12.4 0.00214156724563992
12.5 0.00303388734482888
12.6 0.00396860868117939
12.7 0.00435221970923234
12.8 0.00385769220572877
12.9 0.0063484181480961
13 0.00460070297429933
13.1 0.00192794149846424
13.2 0.00383024568133048
13.3 0.00318734336005937
13.4 0.00175081086988246
13.5 0.00152045307790448
13.6 0.00145744674204857
13.7 0.00168174721564103
13.8 0.00186233607346212
13.9 0.00200740716275265
14 0.00208275421749735
14.1 0.00217996866459722
14.2 0.00157380481020825
14.3 0.00158492694011368
14.4 0.00123631953629882
14.5 0.000960540180434862
14.6 0.000900047915273402
14.7 0.000830389347365177
14.8 0.000437868869268062
14.9 0.00265153735603369
15 0.00398114002134728
15.1 0.00162419091478261
15.2 0.00278662117386217
15.3 0.00485917900057668
15.4 0.0037783147545486
15.5 0.00182475866674973
15.6 0.00232995110054523
15.7 0.00254695445922885
15.8 0.00145704664756429
15.9 0.002241757458959
16 0.00278628900598818
16.1 0.00113262940424731
16.2 0.0009194331613674
16.3 0.00108647130571415
16.4 0.000887106629106828
16.5 0.00203407739287681
16.6 0.000713281947606964
16.7 0.00253301138017987
16.8 0.000982420423433222
16.9 0.00175504518469364
17 0.0014870533802038
17.1 0.000811596726628912
17.2 0.000713671015137884
17.3 0.000999677246703432
17.4 0.000722599665164494
17.5 0.000709777821357932
17.6 0.00207483829897343
17.7 0.000758448776913675
17.8 0.00227394507767697
17.9 0.00216715332566875
18 0.00191911220194457
18.1 0.0021065810019956
18.2 0.00244716642800937
18.3 0.000856227162847044
18.4 0.000725173073547468
18.5 0.000964527619822043
18.6 0.00186314666903763
18.7 0.00205928450844084
18.8 0.00192272024685532
18.9 0.00169332589346587
19 0.000735786607242953
19.1 0.000851677116652567
19.2 0.000921440732309084
19.3 0.00190681945953216
19.4 0.00072744868301746
19.5 0.0020659866614276
19.6 0.00169946990664866
19.7 0.00260046455878947
19.8 0.000967945926301732
19.9 0.00279365511543098
20 0.00549225916542742
20.1 0.00451843718541041
20.2 0.00224661906006302
20.3 0.00504711102305296
20.4 0.00417957480572267
20.5 0.00116080372431168
20.6 0.00135190412730632
20.7 0.00276397813032723
20.8 0.00248798923109946
20.9 0.000932350012909134
21 0.000957704745632271
21.1 0.00130583791146355
21.2 0.00124959969795433
21.3 0.00165879907310727
21.4 0.00176299017569233
21.5 0.00181362965863251
21.6 0.00138068246829359
21.7 0.00128304359004505
21.8 0.000986483816859094
21.9 0.000694766090812802
22 0.000564037451134075
22.1 0.000368499422788179
22.2 0.00036194784841371
22.3 0.00278442971441178
22.4 0.00336578500244648
22.5 0.00509657721859998
22.6 0.00375243135686919
22.7 0.00555778335433584
22.8 0.00329322626027611
22.9 0.00219074291230735
23 0.00301357022174283
23.1 0.00143931114164639
23.2 0.000931945075180504
23.3 0.0023225871822788
23.4 0.00188611669195171
23.5 0.000740280745198771
23.6 0.00177053954195161
23.7 0.00174170258962898
23.8 0.00207945674732425
23.9 0.00259061309482629
24 0.00364455038119097
24.1 0.00309620782487069
24.2 0.00183604988116839
24.3 0.00124248464761039
24.4 0.00118280863751591
24.5 0.00173520511362815
24.6 0.00234706845673169
24.7 0.00159751522145684
24.8 0.00167149590959325
24.9 0.00126164160975007
25 0.0017374205979871
25.1 0.00132544310008675
25.2 0.00330046265004049
25.3 0.00115380073160819
25.4 0.00125602088620519
25.5 0.00236295226741401
25.6 0.00130648748224772
25.7 0.00220262993872144
25.8 0.00265942026206182
25.9 0.00211014041459084
26 0.00122530593807025
26.1 0.0021134944749626
26.2 0.00188353321088364
26.3 0.000884879942476731
26.4 0.0008965119956245
26.5 0.000922555471395546
26.6 0.00214286757233242
26.7 0.00092153225345721
26.8 0.00191558358185165
26.9 0.00215299109539737
27 0.000811119688002781
27.1 0.000970815985005502
27.2 0.00138913004191141
27.3 0.00292941649705783
27.4 0.00362685701682557
27.5 0.00542868401636261
27.6 0.00549505358274129
27.7 0.00426660418179206
27.8 0.00095564338549811
27.9 0.00479671384427148
};
%\addlegendentry{max error}
\addplot [line width = \linewidthErrorC, color = AvgErr]
table {%
0 0
0.1 0.00214995191340496
0.2 0.00140390878492252
0.3 0.000325771334712744
0.4 0.000707136938050266
0.5 0.000686346920103303
0.6 0.000612988560202573
0.7 0.000583400299982956
0.8 0.000719823939759715
0.9 0.000864997931102958
1 0.00052815281364469
1.1 0.000538933338336908
1.2 0.000571053669130864
1.3 0.000479849296754965
1.4 0.000451481135906668
1.5 0.000472910954232216
1.6 0.000427455983453428
1.7 0.000625226386542763
1.8 0.000337441923252853
1.9 0.00038065001456982
2 0.000395764417683126
2.1 0.000464963769876949
2.2 0.00035266522871744
2.3 0.000464799198525325
2.4 0.000481319712855152
2.5 0.000313318408054501
2.6 0.000605097125228146
2.7 0.00061001529701726
2.8 0.000301494217292217
2.9 0.000478250630457454
3 0.000392712016535797
3.1 0.000395206497805362
3.2 0.000455046590051422
3.3 0.00038966441653259
3.4 0.000409261791102267
3.5 0.000344181665219991
3.6 0.000635458756570036
3.7 0.00063378832933252
3.8 0.000469756944061987
3.9 0.000437539783843503
4 0.000521003711479189
4.1 0.00044677503997516
4.2 0.000456797602215318
4.3 0.000518227294795556
4.4 0.000565740307489386
4.5 0.000327625306384091
4.6 0.000536159850105096
4.7 0.000486118642494934
4.8 0.000650935366937868
4.9 0.000353064760505279
5 0.000373711609313808
5.1 0.00130586769924227
5.2 0.000435607652595198
5.3 0.000737649638800371
5.4 0.000684542273140462
5.5 0.000561942285627596
5.6 0.000486208606081046
5.7 0.000522777197744279
5.8 0.000624716647932447
5.9 0.000571820735528701
6 0.00112529799493495
6.1 0.000652499637793528
6.2 0.000856087441237497
6.3 0.000995388253914678
6.4 0.00112440181795449
6.5 0.00133042389068265
6.6 0.0015191581409735
6.7 0.001571226656355
6.8 0.00118946739823819
6.9 0.00105206680055683
7 0.000835633079482644
7.1 0.000611691987573463
7.2 0.000431137244228736
7.3 0.000266044857639398
7.4 0.000195703850793922
7.5 0.00212078048774427
7.6 0.00155642261956804
7.7 0.00028489318032669
7.8 0.000687356006550015
7.9 0.00053559434407874
8 0.000664513043564771
8.1 0.000576843838990278
8.2 0.000695568203658888
8.3 0.000559727837042746
8.4 0.000566073788240878
8.5 0.000531366922874246
8.6 0.000516480345840153
8.7 0.000347040172453653
8.8 0.000365827678300548
8.9 0.000566542487376736
9 0.000508736585392571
9.1 0.000511174741753438
9.2 0.000516029254541366
9.3 0.000587911605012655
9.4 0.000450407186849406
9.5 0.000512026476999501
9.6 0.000520099723185061
9.7 0.000633628321411928
9.8 0.000560893066014673
9.9 0.000482168326467034
10 0.000564006607477693
10.1 0.000592411507606343
10.2 0.000584972828531718
10.3 0.000695769729959216
10.4 0.000527287440905794
10.5 0.000535528891005092
10.6 0.000665615724815263
10.7 0.000607125881588774
10.8 0.000541780533218091
10.9 0.000652285046095762
11 0.00058188119151537
11.1 0.000563154701898057
11.2 0.000419523060531403
11.3 0.000559617099971077
11.4 0.00056311691335904
11.5 0.000511718721579497
11.6 0.000470395280087649
11.7 0.000675736861901787
11.8 0.000716467016882296
11.9 0.000538437159376277
12 0.000455502673828448
12.1 0.000477535363149772
12.2 0.000643668064973019
12.3 0.00067624147241258
12.4 0.000548425057815169
12.5 0.00143661106995938
12.6 0.0010271696462158
12.7 0.00150554114747026
12.8 0.00123743288084334
12.9 0.00102404185217932
13 0.000992617555657845
13.1 0.000901318471839698
13.2 0.000867369894888715
13.3 0.000903443027006259
13.4 0.00102486851174129
13.5 0.000731886613897782
13.6 0.000843636407792331
13.7 0.00103808270038527
13.8 0.00129738101032276
13.9 0.00155455636741119
14 0.00158765744782914
14.1 0.00164914197219794
14.2 0.00125002434928474
14.3 0.00107764799194429
14.4 0.000877825884688674
14.5 0.000691004159938887
14.6 0.000469921142148943
14.7 0.000435227796879541
14.8 0.000189447317226511
14.9 0.00223967419259794
15 0.00146739993371378
15.1 0.00067336244993143
15.2 0.000885448347150335
15.3 0.000878303847451836
15.4 0.000899321704972797
15.5 0.000706762391558523
15.6 0.000968143705687901
15.7 0.000726866571646136
15.8 0.000565703507336729
15.9 0.000589880319663505
16 0.000577132008488073
16.1 0.000559187951263492
16.2 0.000497075761598006
16.3 0.000542686484735261
16.4 0.000368554820265425
16.5 0.000592639915394481
16.6 0.000406815337941095
16.7 0.000633421975816608
16.8 0.000553955311748541
16.9 0.000543830183296829
17 0.000527270447049267
17.1 0.000406570422506521
17.2 0.000394742442383238
17.3 0.000443165705924016
17.4 0.00033337753123386
17.5 0.000392973397729035
17.6 0.000498237758218976
17.7 0.000366655628030927
17.8 0.000483910914068369
17.9 0.000471149627714541
18 0.000493433784956355
18.1 0.000595360587991296
18.2 0.000634694089613901
18.3 0.000396120589897357
18.4 0.000349707483941994
18.5 0.000385564908649255
18.6 0.000471218520208343
18.7 0.000486233407519572
18.8 0.000417256346699489
18.9 0.000455494018989065
19 0.000390969839153773
19.1 0.000352950107890155
19.2 0.000363033863201353
19.3 0.000476729456512644
19.4 0.00034957101634229
19.5 0.000544746683701971
19.6 0.000576327251673644
19.7 0.000578303065939675
19.8 0.000486749382779984
19.9 0.00172176512305615
20 0.00107171717672001
20.1 0.00111396626057107
20.2 0.000862708812150197
20.3 0.00118331864325502
20.4 0.000919972921973535
20.5 0.000660063793320172
20.6 0.000801389524190682
20.7 0.000727312084813533
20.8 0.00115309898085549
20.9 0.000745892930231987
21 0.000838766433658393
21.1 0.00105064703348448
21.2 0.00115050115523215
21.3 0.00146319061685302
21.4 0.00161381918108652
21.5 0.00158208788964914
21.6 0.00123038367526442
21.7 0.00105650889448498
21.8 0.000842921694452999
21.9 0.000610425014588016
22 0.000427306693801245
22.1 0.000205696557186717
22.2 0.000109601770136235
22.3 0.00213836109943829
22.4 0.00173787250266421
22.5 0.000686603272001525
22.6 0.00101524112456659
22.7 0.0012008115897805
22.8 0.000898667029377483
22.9 0.000809805951367347
23 0.000911801767471536
23.1 0.000772987156153566
23.2 0.000477697557820781
23.3 0.000621723774102673
23.4 0.000502408801737035
23.5 0.000357214728462955
23.6 0.000635341186399885
23.7 0.000708239961005285
23.8 0.000762649638474294
23.9 0.000799934839281067
24 0.00100875594907718
24.1 0.00092373611982632
24.2 0.000713370648012384
24.3 0.000670016054757896
24.4 0.000725215970362317
24.5 0.000773678889085173
24.6 0.000833851000474213
24.7 0.000703661225977375
24.8 0.000733109271269768
24.9 0.000662343736118036
25 0.000616055991436885
25.1 0.000528064309980171
25.2 0.000843626052030132
25.3 0.000582707419971831
25.4 0.000710313997601959
25.5 0.000870970078469133
25.6 0.000623217066965274
25.7 0.000818480109092624
25.8 0.000974934826186047
25.9 0.00072314818249674
26 0.00054684077052434
26.1 0.000813097302476729
26.2 0.000572341583729559
26.3 0.000456125882261828
26.4 0.000462274464398028
26.5 0.000420423730109992
26.6 0.000564418597401537
26.7 0.000418416490197651
26.8 0.000554565646367333
26.9 0.000535733010018129
27 0.000406891273923051
27.1 0.000427969118694948
27.2 0.000504302439952392
27.3 0.00151345106125084
27.4 0.000750185351145669
27.5 0.00124593247517885
27.6 0.000911737847670965
27.7 0.000956417165105982
27.8 0.000652922516071572
27.9 0.000836827092230456
};
%\addlegendentry{average error}
\addplot [line width = \linewidthErrorStdVar, color = Koopman] %, dotted]
table {%
0 0
0.1 0.00231382155809625
0.2 0.00174986460561837
0.3 0.000521473999119349
0.4 0.000991335148159158
0.5 0.0011123898652819
0.6 0.000826073506929425
0.7 0.000830010117497585
0.8 0.00118579388253674
0.9 0.00109054961755824
1 0.00092272141404117
1.1 0.000735840113387553
1.2 0.00116620234622212
1.3 0.00100038067579605
1.4 0.000930608722818219
1.5 0.000890634100684767
1.6 0.000874882364762989
1.7 0.00111799988969234
1.8 0.000500681534138414
1.9 0.000550055482228143
2 0.000629157126586424
2.1 0.000938304045859337
2.2 0.000533932167282896
2.3 0.000945623610745909
2.4 0.000919544215033855
2.5 0.000466442106561947
2.6 0.00122047429037796
2.7 0.0012105009133116
2.8 0.000486485945157758
2.9 0.000977131453425499
3 0.000613894969847019
3.1 0.000817722344022048
3.2 0.000903684503877579
3.3 0.000568164246540783
3.4 0.000856107706360912
3.5 0.000558153181636606
3.6 0.00126359047389572
3.7 0.00114533691041616
3.8 0.000874080601113791
3.9 0.00093454598721081
4 0.00103476348650648
4.1 0.000624812344385531
4.2 0.000723151201333599
4.3 0.00108206762044156
4.4 0.00115242509184273
4.5 0.000509198290058789
4.6 0.00110380351125568
4.7 0.000970567109691654
4.8 0.00133973966453398
4.9 0.000559141977077853
5 0.000597800514839613
5.1 0.00163858340003746
5.2 0.000792414061259948
5.3 0.000919313893704079
5.4 0.00109598962196249
5.5 0.000749247100236153
5.6 0.000630490658182267
5.7 0.000697438100003723
5.8 0.000835914006116052
5.9 0.000716131693396821
6 0.00128902088055644
6.1 0.000774245591486976
6.2 0.000988878122143022
6.3 0.00113345094009276
6.4 0.00128873180672814
6.5 0.00144432310637114
6.6 0.00166927755457942
6.7 0.00173019436296492
6.8 0.00128291320360566
6.9 0.00113998991396086
7 0.000918835658622885
7.1 0.000679578542263456
7.2 0.000494312204127911
7.3 0.000450104258328163
7.4 0.000334947693737067
7.5 0.00233697149816379
7.6 0.00185552855490247
7.7 0.000420624822008737
7.8 0.000927546068707483
7.9 0.000913894837169301
8 0.000830914025253391
8.1 0.000992114342055241
8.2 0.00119893235420779
8.3 0.000844849186910553
8.4 0.00097157893006511
8.5 0.000876176546543679
8.6 0.000979884411707082
8.7 0.000541959659784739
8.8 0.000552737010915882
8.9 0.000981541763191271
9 0.000972605996428251
9.1 0.000786760301073484
9.2 0.000908656885660565
9.3 0.000954639142338446
9.4 0.000839342969118711
9.5 0.000986830749668205
9.6 0.000901500198226887
9.7 0.00122096069891553
9.8 0.001035423542903
9.9 0.000738657930110164
10 0.00103897941989842
10.1 0.00123339343960432
10.2 0.00117879152336734
10.3 0.00126013673036793
10.4 0.000974661923062277
10.5 0.0010184098526295
10.6 0.00128482964123976
10.7 0.00105129832221835
10.8 0.000968090669500494
10.9 0.00109647353713529
11 0.00103326065460577
11.1 0.000930256850486395
11.2 0.000672720938816891
11.3 0.000992094092460991
11.4 0.000875099242841079
11.5 0.000987518477205598
11.6 0.000784275225117406
11.7 0.00135456697713248
11.8 0.00125993897128347
11.9 0.000753105525091796
12 0.000675029777917687
12.1 0.000705358927470156
12.2 0.00114772775785879
12.3 0.00114092390052541
12.4 0.00107677923464777
12.5 0.00198479136163936
12.6 0.00200495676479348
12.7 0.00267455378242174
12.8 0.00252522101137611
12.9 0.00250428334377271
13 0.00200298209420446
13.1 0.00140775671622728
13.2 0.00169670188171006
13.3 0.00173407531352689
13.4 0.00134607234503036
13.5 0.0010955544101341
13.6 0.00123446299106776
13.7 0.00135578971820703
13.8 0.00158616425942461
13.9 0.00181866400998842
14 0.00187715461462363
14.1 0.00193825994187203
14.2 0.00144310362383119
14.3 0.00127941593597263
14.4 0.00105268301433719
14.5 0.000814022211225082
14.6 0.000622488821800507
14.7 0.000641586439733255
14.8 0.000314938721211812
14.9 0.00247850102864788
15 0.00223214359343289
15.1 0.00102844243315233
15.2 0.00147051601771364
15.3 0.00197454569100926
15.4 0.00174461916787
15.5 0.00111925423496654
15.6 0.00149783053863594
15.7 0.00125952442926746
15.8 0.000862842844598773
15.9 0.00110250646726383
16 0.00121539660201696
16.1 0.000830968613504923
16.2 0.000722503710457852
16.3 0.000847761675673719
16.4 0.000570058875205066
16.5 0.00106452387915394
16.6 0.000601459590367437
16.7 0.00120254834380059
16.8 0.000794470229155346
16.9 0.000937184902292423
17 0.000882883052882641
17.1 0.000660592404660236
17.2 0.000570500660050605
17.3 0.000644114727357958
17.4 0.000550791963056205
17.5 0.000564626037781922
17.6 0.000962605350165296
17.7 0.000560863123892184
17.8 0.00100356340686651
17.9 0.000958999150272623
18 0.000919675335132375
18.1 0.00120269890371762
18.2 0.00128748430524133
18.3 0.000606464433023574
18.4 0.000533676637588932
18.5 0.000614330750395674
18.6 0.000875093933570785
18.7 0.000952594970414312
18.8 0.000859723338646776
18.9 0.000851478021904575
19 0.000607520763651768
19.1 0.000607153500922389
19.2 0.000585628426338073
19.3 0.000930621225656116
19.4 0.00054848005075443
19.5 0.000999132965025807
19.6 0.000987622520888055
19.7 0.00115768499120896
19.8 0.000760857848185578
19.9 0.00217868443749937
20 0.0025757532457817
20.1 0.00207432134123099
20.2 0.00142866174139104
20.3 0.00254784611964488
20.4 0.00184142839867851
20.5 0.000853156570506987
20.6 0.00109232285508926
20.7 0.0012973481281092
20.8 0.00155508978655704
20.9 0.000861571217779736
21 0.00093263536608656
21.1 0.00118309981572182
21.2 0.00125364798566768
21.3 0.00156947201832797
21.4 0.00170204760274928
21.5 0.00171366580158903
21.6 0.00130483811482201
21.7 0.00118086179402456
21.8 0.000931936017224753
21.9 0.000664528382793357
22 0.000490760807422744
22.1 0.000281347199176296
22.2 0.000191017192758325
22.3 0.00236696155268246
22.4 0.00236328733502591
22.5 0.00187616405394143
22.6 0.00192645536123725
22.7 0.00253863459847572
22.8 0.00179965746473059
22.9 0.00130425785357984
23 0.00151577686891308
23.1 0.00108714647039781
23.2 0.00070834692040687
23.3 0.00121358785845256
23.4 0.000888372037969257
23.5 0.000550482608614709
23.6 0.00112809963945043
23.7 0.00114982277362499
23.8 0.00129473823190493
23.9 0.00148186688506181
24 0.00190206407880203
24.1 0.00176082792094983
24.2 0.00110483423312575
24.3 0.000999349114697908
24.4 0.000998093868544542
24.5 0.00124792654650288
24.6 0.0013920839234567
24.7 0.00111135082035635
24.8 0.00115880031633385
24.9 0.000926167006732821
25 0.00105521472216144
25.1 0.000868361381014774
25.2 0.00153992767564172
25.3 0.000919609780331292
25.4 0.00101711837589698
25.5 0.00135878133609676
25.6 0.000923107221516305
25.7 0.00139683838152107
25.8 0.00161383105905065
25.9 0.00118038272201498
26 0.000818681400112886
26.1 0.00136885193230068
26.2 0.00101882430201544
26.3 0.000691494193709063
26.4 0.000686111480391842
26.5 0.000709339391666605
26.6 0.00101982176323454
26.7 0.00068081109359308
26.8 0.000974395440697961
26.9 0.00103464579775581
27 0.000619928796904854
27.1 0.000690288691454762
27.2 0.000812848642398674
27.3 0.001957096074384
27.4 0.00159860627015604
27.5 0.00243437390615917
27.6 0.00214864291646018
27.7 0.00193049644072864
27.8 0.000830450552688482
27.9 0.00191806377103793
};
\end{axis}
\end{tikzpicture}
%  
    \caption{Comparison of the surrogate model using~$B_2$ with 15 experiment runs, where~$e_{\textnormal{max}}$,~$e_{\textnormal{avg}}$, and~$\sigma$ denote maximum, average, and standard deviation of the error norms, respectively.}  
    \label{fig: data B2 ref8}%
\end{figure}

A remaining concern is data efficiency. 
Hence, subsequently, training data points are removed systematically to obtain smaller training data sets. 
The original training data was generated by choosing, for~$B_1$,~$m_1 = 50$ and, for~$B_2$,~$m_1 = 39$ random initial conditions in~$\mathbb{X}$. 
The relevant trajectory pieces driven for each sampled point are all of different lengths, e.g., depending on the distance to the boundary of~$\mathbb{X}$. 
To systematically reduce the number of data points, first, these lengths are unified by taking the length of the shortest trajectory, which, here, consists of~$m_2 = 20$ steps, and discarding the data points beyond that for each trajectory segment. 
This leads to a new training data set of~$m_1 \cdot m_2 = 1000$ or~$780$ per basis. %, that means~$2000$ points overall \textcolor{blue}{(Stuttgart versteht diese Rechnung noch nicht :-O)}.} 
Then, every~$n$th data point,~$n \in \{1, 20, 50, 100\}$, is used to create training data sets of lower cardinality.  %while retaining something similar to independency in the data. 
%The resulting data sets have the sizes~$2000, 100, 40, 20$ and~$1520, 76, 32, 16$ for~$B_1$ and~$B_2$, respectively \textcolor{red}{(s.\ oben bzgl.\ Zahlen)}. 
Using~$\mathbb{O}_{11}$,  the different resulting surrogate models' average one-step prediction errors w.r.t.\ the~$15$~recorded trajectories in the~$\infty$-scenario are given in Fig.~\ref{fig: lessdata}. 
\definecolor{RungeKuttaError}{RGB}{0,128,0}%
\definecolor{n1}{RGB}{0,0,255}%
\definecolor{n20}{RGB}{255,165,0}%
\definecolor{n50}{RGB}{85,107,47}%
\definecolor{n100}{RGB}{221,160,221}%
\def\lineWidthErrorN{1.0}%
\def\heightErrorN{5.34cm}%
\begin{figure}[t!]
    \centering%
    % This file was created with tikzplotlib v0.10.1.
\begin{tikzpicture}%
\begin{axis}[%
width = .38\textwidth,
height = \heightErrorN,
at={(0, 0)},
legend cell align={left},
legend columns = 3,
legend style={
  fill opacity=1,
  draw opacity=1,
  text opacity=1,
  at={(0.2,1.05)},
  anchor=south west,
  column sep = 0.25cm
  %draw=lightgray204
},%
xlabel = {time (s)}, 
%ylabel = {to do},
%xmin=-1.24, xmax=26.04,
xmin=-0, xmax=3,
%xticklabel style={xshift=-.05cm},
%ymin=0.000271409542571133, ymax=0.0758614433353828,
ymin=0.000145108448524775, ymax=0.233723638032451,
ymode=log,
%ylabel style={yshift=-.1cm},
]%
\addplot [line width = \lineWidthErrorN, color = RungeKuttaError]%
table {%
0 0
0.1 0.0165135818582296
0.2 0.0173573550131051
0.3 0.0155240468737607
0.4 0.016691397105289
0.5 0.0143597427283743
0.6 0.0143277101587927
0.7 0.0141366060316006
0.8 0.0132617453957916
0.9 0.0128695136926716
1 0.0124076789840141
1.1 0.0126642841967782
1.2 0.0119596603709604
1.3 0.0118711563344089
1.4 0.0111805312022035
1.5 0.0107251605082315
1.6 0.0105427094466726
1.7 0.0105103942337942
1.8 0.00985982763058332
1.9 0.0103415019491021
2 0.00961337937132115
2.1 0.009176374240606
2.2 0.00801289848433475
2.3 0.00777665982657768
2.4 0.0080071078808638
2.5 0.0089096573833126
2.6 0.0113768144136755
2.7 0.00947614752641055
2.8 0.0089458000759625
2.9 0.00891442844297832
3 0.0110135746945919
};
\addlegendentry{first principles}%
\addplot [line width = \lineWidthErrorN, color = n1]%
table {%
0 0
0.1 0.0168547775726986
0.2 0.00148356996263524
0.3 0.00169140700049326
0.4 0.00567886035490619
0.5 0.00506867470821116
0.6 0.00172323321283047
0.7 0.0015779221033442
0.8 0.00416152555713289
0.9 0.00389955994174455
1 0.000798252110969428
1.1 0.00256938661463039
1.2 0.000373631108651846
1.3 0.00154784972300235
1.4 0.000576092227241258
1.5 0.00088091019309275
1.6 0.000264028222304142
1.7 0.00136843922082343
1.8 0.000609183899763228
1.9 0.00229624541683824
2 0.000478179499443402
2.1 0.000321676864699139
2.2 0.00166068600174685
2.3 0.000419506047125796
2.4 0.00250094497523073
2.5 0.00239357229708892
2.6 0.00444473103999932
2.7 0.00190549954896417
2.8 0.000333395126520599
2.9 0.00120727362785229
3 0.00328798935183676
};
\addlegendentry{$n = 1$}%%
\addplot [line width = \lineWidthErrorN, color = n20]
table {%
0 0
0.1 0.0107520046142247
0.2 0.00703012533295705
0.3 0.00914622109847614
0.4 0.0106978338302807
0.5 0.0080884136978301
0.6 0.00780451593980268
0.7 0.00782457410451835
0.8 0.00722562085079703
0.9 0.00891159770822631
1 0.00706118253863211
1.1 0.00759381786545972
1.2 0.00662928014698138
1.3 0.00695113504294955
1.4 0.00630578552458954
1.5 0.00642374990273454
1.6 0.00577892116526725
1.7 0.00619664180256355
1.8 0.00577753317339912
1.9 0.00601060039346849
2 0.00540674142439366
2.1 0.00544416750963064
2.2 0.00573971102864734
2.3 0.0048921279110506
2.4 0.00495427396964455
2.5 0.00444867188324502
2.6 0.00479381797469929
2.7 0.00590327777469894
2.8 0.00454789990000331
2.9 0.00399834964862631
3 0.00372148446618804
};
\addlegendentry{$n = 20$}%
\addplot [line width = \lineWidthErrorN, color = n50]%
table {%
0 0
0.1 0.0203182924000263
0.2 0.0113276944775643
0.3 0.011849451266052
0.4 0.00769612610285905
0.5 0.0174108700990504
0.6 0.0115449522486172
0.7 0.0118855032277033
0.8 0.0174574096169832
0.9 0.0103252520981193
1 0.0134727784348086
1.1 0.0119737329604545
1.2 0.0146000918001689
1.3 0.0133250697103859
1.4 0.0152715300774054
1.5 0.0144506685252092
1.6 0.0149753041238364
1.7 0.0141513867404197
1.8 0.0159680451492941
1.9 0.0136039863195457
2 0.0161461804984792
2.1 0.0156359940126427
2.2 0.0174030960440711
2.3 0.0161872730270556
2.4 0.0136759546937059
2.5 0.0135912132715082
2.6 0.0119497986521269
2.7 0.0173607996267529
2.8 0.0155252933484477
2.9 0.0145914329956792
3 0.012211805863653
};
\addlegendentry{$n = 50$}%
\addplot [line width = \lineWidthErrorN, color = n100]%
table {%
0 0
0.1 0.0933999981288589
0.2 0.0926791017161514
0.3 0.0902161505732473
0.4 0.0902438387460173
0.5 0.0825383448184232
0.6 0.0839399792925418
0.7 0.0818470967294791
0.8 0.0776295250010234
0.9 0.0798208064035864
1 0.0760140054005865
1.1 0.0752048315891476
1.2 0.0721635656360341
1.3 0.0713763125336862
1.4 0.0691227358920897
1.5 0.0676327364980061
1.6 0.0659148212444659
1.7 0.0645714371656367
1.8 0.0619536830491499
1.9 0.0607896154753264
2 0.0573091656383364
2.1 0.0552321375810049
2.2 0.0516499860037895
2.3 0.0492233768548466
2.4 0.0467331101025748
2.5 0.0436695235227793
2.6 0.0404854600722299
2.7 0.0390111508653
2.8 0.0379974974058325
2.9 0.038801886467359
3 0.0412726391072552
};
\addlegendentry{$n = 100$}%
\end{axis}%
\begin{axis}[%
width = .38\textwidth,
height = \heightErrorN,
at={(.31\textwidth, 0)},
legend cell align={left},
legend columns = 2,
legend style={
  fill opacity=1,
  draw opacity=1,
  text opacity=1,
  at={(1,1)},
  anchor=south east,
  column sep = 0.25cm
  %draw=lightgray204
},
xlabel = {time (s)}, 
%ylabel = {to do},
%xmin=-1.24, xmax=26.04,
xmin=-0, xmax=3,
%xticklabel style={xshift=-.05cm},
ymin=0.000145108448524775, ymax=0.233723638032451,
yticklabels = {},
ymode=log,
%ylabel style={yshift=-.1cm},
]%
\addplot [line width = \lineWidthErrorN, color = n1]%
table {%
0 0
0.1 0.0170069580861405
0.2 0.00140108710254583
0.3 0.00140166355322037
0.4 0.00524505512800558
0.5 0.00548379538129153
0.6 0.00131104362399937
0.7 0.00114401085311314
0.8 0.00458180649427533
0.9 0.00347765639867595
1 0.000393577703062499
1.1 0.00216158993567225
1.2 0.00051132053399086
1.3 0.00114818171993019
1.4 0.000950680860953724
1.5 0.0005169422718385
1.6 0.000254150965122212
1.7 0.00102315000557671
1.8 0.000968693191825949
1.9 0.00195887117092178
2 0.000765285060260696
2.1 0.000403483349547753
2.2 0.00194988024257816
2.3 0.000692710960952163
2.4 0.00222684337625356
2.5 0.00213746120278164
2.6 0.00417121081270337
2.7 0.00217147326910141
2.8 0.000282310699715259
2.9 0.000903478356395921
3 0.0031183003184554
};
%\addlegendentry{$n = 1$}
\addplot [line width = \lineWidthErrorN, color = n20]%
table {%
0 0
0.1 0.016324256353984
0.2 0.00074922458318092
0.3 0.00202276598665036
0.4 0.0049800220301007
0.5 0.00588510757008295
0.6 0.00100287896277084
0.7 0.000964578089686334
0.8 0.00503298303191267
0.9 0.00310395844945226
1 0.000603396347684195
1.1 0.00170677595678879
1.2 0.00114498278106672
1.3 0.000831252902677732
1.4 0.00161108352750722
1.5 0.000681212381770717
1.6 0.000822934354859154
1.7 0.000662718346706381
1.8 0.00168515529805249
1.9 0.00132523769164896
2 0.00152087521345923
2.1 0.00105973505875112
2.2 0.00273167306878487
2.3 0.00146453098478398
2.4 0.00145657790291938
2.5 0.00131695625243886
2.6 0.00334100766291158
2.7 0.00300686523940952
2.8 0.00106200234801933
2.9 0.000410397904973214
3 0.00217766362144465
};
%\addlegendentry{$n = 20$}
\addplot [line width = \lineWidthErrorN, color = n50]%
table {%
0 0
0.1 0.0172869416888037
0.2 0.00170236706334129
0.3 0.00115936929110603
0.4 0.0053263463032189
0.5 0.00544258401478348
0.6 0.00143192537455004
0.7 0.00120832232729362
0.8 0.00457674112606484
0.9 0.00349655148232675
1 0.00038333026982136
1.1 0.00216397541600475
1.2 0.000504354055491986
1.3 0.00108091732532949
1.4 0.00103254962259857
1.5 0.000355549301347473
1.6 0.000456801227495615
1.7 0.000824868786196053
1.8 0.00115783179291279
1.9 0.00167984667289002
2 0.0011099642095187
2.1 0.000650509355301443
2.2 0.00246572174851809
2.3 0.00130732071227655
2.4 0.00149285133885671
2.5 0.00123402791476715
2.6 0.00313382906577376
2.7 0.00337774788247286
2.8 0.00159590064532075
2.9 0.000891028175052442
3 0.00136466777763771
};
%\addlegendentry{$n = 50$}
\addplot [line width = \lineWidthErrorN, color = n100]%
table {%
0 0
0.1 0.0772943787246779
0.2 0.0665255209801766
0.3 0.0575907266458193
0.4 0.0534691317695794
0.5 0.0381210714228748
0.6 0.0340932957962354
0.7 0.0264599696272329
0.8 0.0143969567419319
0.9 0.0161006729696299
1 0.0135212180707904
1.1 0.0195525981449623
1.2 0.0269650897918591
1.3 0.0367237923427581
1.4 0.0475616380025682
1.5 0.058880175717317
1.6 0.0712161881753133
1.7 0.0843582460620498
1.8 0.0989536781702462
1.9 0.113715127842012
2 0.130171352583139
2.1 0.148101330900094
2.2 0.167564727912161
2.3 0.188205592105269
2.4 0.209211921561547
2.5 0.231721016990809
2.6 0.254517103896584
2.7 0.281428636644663
2.8 0.309246509298716
2.9 0.337780221263813
3 0.367296165046544
};
%\addlegendentry{$n = 100$}
\addplot [line width = \lineWidthErrorN, color = RungeKuttaError]%
table {%
0 0
0.1 0.0165135818582296
0.2 0.0173573550131051
0.3 0.0155240468737607
0.4 0.016691397105289
0.5 0.0143597427283743
0.6 0.0143277101587927
0.7 0.0141366060316006
0.8 0.0132617453957916
0.9 0.0128695136926716
1 0.0124076789840141
1.1 0.0126642841967782
1.2 0.0119596603709604
1.3 0.0118711563344089
1.4 0.0111805312022035
1.5 0.0107251605082315
1.6 0.0105427094466726
1.7 0.0105103942337942
1.8 0.00985982763058332
1.9 0.0103415019491021
2 0.00961337937132115
2.1 0.009176374240606
2.2 0.00801289848433475
2.3 0.00777665982657768
2.4 0.0080071078808638
2.5 0.0089096573833126
2.6 0.0113768144136755
2.7 0.00947614752641055
2.8 0.0089458000759625
2.9 0.00891442844297832
3 0.0110135746945919
};%
%\addlegendentry{Runge-Kutta}
\end{axis}%
\end{tikzpicture}%%
    \caption{Average one-step prediction errors for surrogate models using~$B_1$ (left) and~$B_2$ (right) when only using every~$n$th data point.}%
    \label{fig: lessdata}%
\end{figure}
These show that basis~$B_2$ seems to be more data efficient since the errors remain lower in data-sparse settings.  
Moreover, comparatively small training data sets can suffice in this scenario to achieve one-step prediction errors that are consistently smaller than that of the nominal model, especially when using basis~$B_2$. 

\section{Summary and Outlook}
This contribution showed with a detailed analysis that a bilinear eDMD approach in the Koopman framework can be a very powerful data-driven modeling tool in mobile robotics.  
Even with a modest amount of data and a calculation time in the second range, %\textcolor{blue}{@Lea: Wie lange rechnet es denn im Trainingschritt?}, 
the approach can be used to learn a dynamical model that is on average more accurate in predictions than the common nominal kinematic model of a differential-drive robot. 
Moreover, we have seen that and how physical a-priori knowledge can be successfully incorporated into the model, which is interesting beyond the considered application scenario. 
In particular, we have shown how the dictionary of observables can be modified to account for translation invariance. 
Still, there are many remaining topics that we will cover in subsequent research. 
This includes data-driven modeling that strives to include second-order effects such as actuator dynamics and inertia, complicating especially practical considerations such as measuring and sampling of training data. 
Similarly, we will look at non-holonomic vehicles of higher degree of non-holonomy. 
Moreover, we intend to use the learned models for data-based predictive control.
\\


\noindent\textbf{Acknowledgement}: We sincerely thank Manuel Schaller (TU Ilmenau) for his support w.r.t.\ implementation details and fruitful discussions, which improved our manuscript.

\bibliographystyle{IEEEtran}
\bibliography{references_Koopman_robotics, references_nonholonomic_robot}



\end{document}


