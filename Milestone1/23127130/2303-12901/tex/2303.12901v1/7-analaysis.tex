% \section{Performance Analysis}

\section{Implementation details}
\label{sec:Implementation details}

We implement the proposed accelerator on a state-of-the-art FPGA board -- Xilinx Alveo U250, which
has four Super Logic Regions (SLR) \cite{ref-alvelu250}.  As shown in Figure \ref{fig:schematic}, we implement two Computation Cores (CC) in each SLR except for SLR1, because the FPGA shell (which handles the CPU-FPGA communication) and soft processor is placed in SLR1. For each CC, $p_{sys}=16$. We develop the CC using Verilog HDL, and implement the soft processor using Xilinx Microblaze Soft IP core \cite{Microblaze-link}. Each CC is connected to the soft processor through the AXI4-Stream interface \cite{Microblaze-link}, through which the soft processor sends the control signals to CC and the CC sends the sparsity information to the soft processor. 
{ We develop the compiler using Python. The IR of a kernel is implemented as a Python object that stores the meta data of a kernel and its execution scheme.}
We develop the Runtime system on the soft processor using C in Xilinx Vitis Unified Software Platform (version 2020.1). The Index Shuffle Network and Data Shuffle Network are implemented using a butterfly network with buffering to handle the routing congestion. We perform synthesis and Place\&Route using Vivado 2020.1.  The resource utilization is shown in Figure \ref{fig:schematic}. The CCs run at 250 MHz.
% The reported power consumption of FPGA design is 48 Watt.
\begin{figure}[h]
     \centering
     \includegraphics[width=8.7cm]{pic/schematic.pdf}
     \caption{The layout (FPGA chip) and resource utilization of the proposed design on Xilinx Alveo U250. The Computation Cores (CC0-CC6) are represented using different colors.}
     \label{fig:schematic}
\end{figure}


\noindent \textbf{Soft processor}: Our implementation achieves 370 MHz and  around 500 Million Instructions Per Second \cite{Microblaze-link} performance. It has two caches -- an Instruction Cache (I-Cache) and a Data Cache (D-Cache). I-Cache has size 32 KB which is sufficient to hold the binary code of the runtime system after a warm-up execution. D-Cache has the size 64 KB which stores the sparsity of the data partitions. For large graphs that D-Cache is not enough to hold the sparsity information of all the data partitions, we store it in the external memory and prefetch the sparsity information to the D-Cache. The soft processor reads/writes the data from/to the AXI-stream interface through the $\verb|get|$ and $\verb|put|$ instructions \cite{Microblaze-link}, which have one or two clock cycles latency.


% \begin{table}[ht]
% \centering
% \vspace{-0.2cm}
% \caption{Resource Utilization on Xilinx Alveo U250}
% \vspace{-0.2cm}
% \begin{adjustbox}{max width=0.42\textwidth}
% \begin{tabular}{ccccc}
% \toprule
%   & \textbf{LUTs} & \textbf{DSPs} & \textbf{BRAMs} & \textbf{URAMs}\\
% \midrule
% \midrule
% Soft Processor & 5550 & 6 & 26 & 0 \\
% \rowcolor{LightCyan}
% One Computation Core  & 118140 &  1024 & 96 & 120  \\
% FPGA Shell  & 181045 &  13 & 447 & 0  \\ \midrule
% Total  & 1011311 &  7187 & 894 & 840 \\
% \rowcolor{LightCyan}
% Available on U250  & 1728000  & 12288 & 1145 & 960\\
% Utilization  & 58.6\%  & 58.4\% & 42.6\% & 87.5\%\\
% \bottomrule
% \end{tabular}
% \end{adjustbox}
% \label{tab:Resource Utilization}
% \vspace{-0.2cm}
% \end{table}



