%%%%%%%%%%%%%%%%%%%%%%%%%%%%%%%%%%%%%%%%%%%%%%%%%%%%%%%%%%%%%%%%%%%%%%%%%%%%%%%%
%2345678901234567890123456789012345678901234567890123456789012345678901234567890
%        1         2         3         4         5         6         7         8

\documentclass[letterpaper, 10 pt, conference]{ieeeconf}  % Comment this line out if you need a4paper

%\documentclass[a4paper, 10pt, conference]{ieeeconf}      % Use this line for a4 paper

\IEEEoverridecommandlockouts                              % This command is only needed if 
                                                          % you want to use the \thanks command

\overrideIEEEmargins                                      % Needed to meet printer requirements.

\usepackage[normalem]{ulem}
\usepackage[utf8]{inputenc}
\usepackage[T1]{fontenc}
% My packages
\setlength {\marginparwidth }{1.5cm}
\usepackage[textsize=scriptsize]{todonotes}
\usepackage{lipsum}
\usepackage{algorithm}
\usepackage{algpseudocode}
\usepackage{url}
\usepackage{multirow}
% \usepackage{graphicx}
\usepackage{blindtext}
\usepackage[separate-uncertainty = true, free-standing-units, space-before-unit, use-xspace]{siunitx}
\usepackage{color,soul}
\usepackage{svg}
% The following packages can be found on http:\\www.ctan.org
%\usepackage{graphics} % for pdf, bitmapped graphics files
%\usepackage{epsfig} % for postscript graphics files
%\usepackage{mathptmx} % assumes new font selection scheme installed
%\usepackage{mathptmx} % assumes new font selection scheme installed
\usepackage{amsmath} % assumes amsmath package installed
\usepackage{amsfonts}
\usepackage{amssymb}  % assumes amsmath package installed

\newcommand{\explainindetail}[1]{\todo[color=blue!40]{#1}}
\newcommand{\explainindetailline}[1]{\todo[inline, color=blue!40]{#1}}
\newcommand{\insertref}[1]{\todo[color=green!40]{#1}}
\newcommand{\bds}[1]{\boldsymbol{#1}}
\newcommand{\vok}[1]{\textbf{\{vok:} \textcolor{cyan!70!black}{#1}\textbf{\}}}
\newcommand{\fas}[1]{\textbf{\{fas:} \textcolor{cyan!70!black}{#1}\textbf{\}}}


\section{Review of ML-based ride-hailing planning}
\label{sec:review}
\revise{In this section, we review matching, repositioning, and joint matching and repositioning in Sec.~\ref{sec:review-matching} Sec.~\ref{sec:review-repositioning}, and Sec.~\ref{sec:review-joint}, respectively.}
In each part, we discuss the collective and the distributed strategy separately.
Fig.~\ref{fig:review-outline} gives an outline of the review.

\begin{figure*}[h]
	\centering
	\includegraphics[width=0.8\linewidth]{figs/survey-taxonomy.pdf}
	\caption{\revise{A taxonomy of the ride-hailing planning literature. %is summarized.
	In each category, we discuss three works as representative examples.}}
	\label{fig:review-outline}
\end{figure*}

\subsection{Matching}
\label{sec:review-matching}

\subsubsection{Collective Matching}
\revise{RL is a promising technique for solving the matching problem.
Chen et al.~\cite{chen2020order} propose an RL-based solution in which
a deep evaluation network, which is a plain feed-forward neural network, is used to calculate a score for each pair of driver and rider based on the predicted detour distance, vehicle's seat utilization rate, and profit achieved if they get matched.
For each new ride request, the vehicle with the highest score will be assigned to serve the rider.
When the trip of the ride request is finished, the observed reward, i.e., the sum of the increased profit of the driver and, if any, the reduced cost of the rider through sharing the ride with others, is used to guide the learning process of the deep evaluation network.
Agussurja et al.~\cite{agussurja2019state} formulate the matching problem as a two-stage planning process.
In the first stage, ride requests to be scheduled are selected from all the unserved ones, the problem of which is modeled as a Markov Decision Process.
An approximated value iteration algorithm is used to learn the value function for the matching actions.
In the second stage, the final matching decision is made between the selected ride requests and all vehicles based on the learned value function.
\revise{Kullman et al.~\cite{kullman2022dynamic} apply deep RL to develop matching policies whose decisions leverage the Q-value approximations learned by deep neural networks.}
Multi-hop ride-hailing can improve the efficiency of a ride-hailing system.
To find the transfer points for each transferring trip in the multi-hop ride-hailing service, Xu et al.~\cite{xu2020highly} use a multi-layer feed-forward network to predict the reachable areas of vehicles, based on which the search space of possible vehicle pairs and transfer points for transferring riders is pruned.
In this way, the transfer points searching process can be more efficient.
	Wang et al.~\cite{wang2023optimization} also consider the scenario where riders are allowed to transfer between vehicles.
	They leverage RL to learn a policy that estimates the values of all the vehicles, which are then used to compute the optimal matching decisions by integer-linear programming.
The lengths of the time-intervals between the matching decisions can have critical impact in the matching outcomes.
	Specifically, the efficiency of matching may be improved substantially if the matching is delayed by adaptively adjusting the matching time-intervals according to the real-time situation of the riders and drivers.
	Wang et al.~\cite{wang2019adaptive} find that, if riders are willing to wait for a certain amount of time even if there are available vehicles that can serve them right away, the ride-hailing system can achieve better results, for example, in terms of the total vehicle miles traveled.
	In their solution, 
	 they propose to use an RL policy to decide for each rider, at each time step, whether to conduct matching for her/him, or %leave her/him alone and
	wait for the next time step. 
	Similarly, Qin et al.~\cite{qin2021optimizing} leverage RL in solving the ride-hailing matching problem with dynamic matching time-intervals.}

\revise{Clustering techniques are frequently used in ride-hailing planning.
Hong et al.~\cite{hong2017commuter} propose to use a density-based clustering algorithm, specifically DBSCAN \cite{parimala2011survey}, to identify riders that share similar itineraries based on their historical traveling trajectories. 
To alleviate the computational overhead caused by the large number of distance queries in the matching process, Zhang et al.~\cite{zheng2018order} propose a new clustering algorithm that groups the geographical locations in the road network into different clusters.
Then, the distance between any two nodes is approximated by the distance between the centers of the clusters they belong to. 
Shen et al.~\cite{shen2019roo} propose a spatial-temporal distance metric that measures the similarity of each pair of ride requests.
The ride requests are grouped by a clustering process based on the proposed distance metric.
Then, shared-rides are computed within each group of ride requests.
Another clustering algorithm is proposed by \cite{trasarti2011mining} to extract the mobility profiles from riders' and drivers' historical itineraries.
The matching between riders and drivers is determined based on the similarities between their profiles.}


\revise{An increasing number of collective matching solutions leverage various other ML techniques in planning.
Most of them take social factors of drivers and riders into consideration \cite{mitropoulos2021systematic}.
To mitigate the social barriers in the ride-hailing process, especially in shared-rides, Yatnalkar et al.~\cite{yatnalkar2020enhanced} and Narman et al.~\cite{narman2021enhanced} use Support Vector Machine (SVM) to predict the user social types, e.g., chatty, safety, or punctuality, based on their registered user characteristics.
Riders with similar social characteristics %are more likely
would be more willing to share a trip.
%on their closest available vehicle.
Levinger et al.~\cite{levinger2020human} use a feed-forward neural network to predict rider satisfaction levels according to their profile and trip information.
They proposed a stochastic algorithm to compute the matching decision with rider satisfaction level maximization as the objective.
Montazery and Wilson \cite{montazery2016learning, montazery2018new} propose to take into account the user preference in evaluating the weight (benefit) of the matching between each pair of rider and driver, which is given by their proposed support vector machine-based score function.
With the value calculated, the final matching can be obtained by solving an optimization problem in which the sum of the weights of those matched pairs is maximized.
Tang et al.~\cite{tang2020efficient} model various types of information (e.g., driver, rider, travel time, and activity) and their relationships within a ride-hailing system using a Heterogeneous Information Network (HIN) \cite{sun2012mining}.
Each driver or rider is projected to a multi-dimensional embedding (vector) using the skip-gram model \cite{mikolov2013efficient}.
Moreover, the skip-gram is conducted on node sequences obtained by meta path-based random walks originating from the corresponding node within the HIN \cite{dong2017metapath2vec}.
The cosine similarity between the embeddings of each driver-rider pair is then used to identify possible matching.
Zhang et al.~\cite{zhang2017taxi} consider a scenario where each rider is assigned to multiple drivers (to improve the order answer rate), and riders are free from having to enter the details of destinations (to improve the user experience).
They first leverage historical data to model the probability distribution of destinations of each rider based on his/her departure time and location with Bayesian rules, which is followed by predicting the acceptance probability between the rider and available drivers with logistic regression \cite{friedman2001elements} and gradient boosted decision tree \cite{mason1999boosting}.
They propose a hill climbing-based algorithm to solve the matching problem, which is formulated as an NP-hard combinatorial optimization with maximizing the success rate of matching as the objective.
Schleibaum and M{\"u}ller \cite{schleibaum2020human} advocate taking the determinants of user satisfaction and explainable matching decisions into consideration.
One of their future studies is to find out whether increasing the explainability can improve user satisfaction level or not.}

\revise{It is worth mentioning that many ML-based collective matching strategies take advantage of the Kuhn-Munkres (KM) bipartite matching algorithm as a component of their decision-making pipelines \cite{jonker1986improving}.
Drivers and riders are usually regarded as the two sets of vertices in the target bipartite graph.
To guide the matching between ride requests and ride offers,  Guo et al.~\cite{guo2020spatiotemporal} propose spatial-temporal Thermo, which is used to reflect the demand density of different places and times.
They use Random Forest Regression \cite{breiman2001random} to map multiple features of spatial, temporal, and meteorological dimensions to Thermo.
The weight of each pair of driver and rider in the bipartite graph is estimated by Thermo.
A KM algorithm is then used to calculate the final matching decisions according to the constructed bipartite graph.
Similarly, Xu et al.~\cite{xu2018large} derive their matching decisions using the KM algorithm.
In contrast to \cite{guo2020spatiotemporal}, Xu et al.~\cite{xu2018large} leverage a policy evaluation algorithm to learn a value function which maps each pair of driver and rider to a score.
The KM algorithm calculates the final matching between drivers and riders based on the scores.
Guo and Xu \cite{guo2020deep} also conduct the matching planning using the KM algorithm.
The weight between each pair of driver and rider is obtained from a value function learned by a convolutional neural network-based Double Q-learning (Double DQN) algorithm \cite{van2016deep}.}

\subsubsection{Distributed Matching}
\revise{RL is also a powerful technique for distributed matching \cite{sutton1999reinforcement}.
Gu{\'e}riau and Dusparic \cite{gueriau2018samod} use the Q-learning algorithm to train a policy for each agent (driver) to choose the pickup or rebalancing action based on the environment state, including the status of itself and current distribution of supply and demand.
If pickup action is chosen, then the vehicle will go and pick up the nearest rider.
In their follow-up work \cite{gueriau2020shared}, they extend the method to consider traffic congestion when agents are making decisions.
Wang et al.~\cite{wang2018deep} propose to use the DQN \cite{mnih2015human}, in which a deep neural network is employed to estimate the state-action value function from a single driver's perspective.
Many methods of distributed matching allow the decisions to be determined individually while the matching policy is trained collectively.
For example, De Lima et al.~\cite{de2020efficient} follow the QMIX framework proposed in \cite{rashid2018qmix}, in which the coordinated planning policies are trained by learning a joint action-value function for multiple vehicles and riders aiming at optimizing a global objective.
In the execution process, the matching decision of each vehicle is made in a distributed  manner following its own component in the learned action-value function.
By ensuring the monotonicity of the relationship between the global action-value and the action-value of each passenger, the objectives of distributed planning decisions are ensured to coincide with the centralized decisions during the training process.
Similar to \cite{de2020efficient}, Li et al.~\cite{li2019efficient} adopt the framework where the matching policy is trained in a centralized manner and executed in a distributed manner.
Specifically, they adopt the actor-critic RL framework, where actor and critic are two different networks used to decide and evaluate the action for each driver, respectively.
The coordination among drivers in the matching policy is enabled by the critic network.
It adopts the mean field approximation to model the interactions of drivers by calculating an average on the actions taken by their neighborhoods, which is then considered in the process of evaluating each driver's action.
\revise{In \cite{zhou2019multi}, another centralized training process is proposed, in which a Kullback–Leibler divergence optimization is used to balance the supply and demand and to enable coordination among the vehicles.}
In the execution phase, each driver chooses an action based on their own action-value functions.}

\revise{Some distributed matching strategies leverage other ML techniques.
They mostly determine the matching decisions based on the similarities between the riders and drivers in ride-hailing.
For example, Bicocchi and Mamei \cite{bicocchi2014investigating} use the bag-of-words model to summarize users' frequently visited places as vector representations, which are then fed to the Latent Dirichlet Allocation (LDA) \cite{blei2003latent} model to identify their patterns of daily travel routine behaviors.
Given a rider or a driver, his/her potential participants of shared-rides can be found by calculating the similarities between his/her daily travel routine and those of the other riders and drivers.
Lasmar et al.~\cite{lasmar2019rsrs} propose to leverage a multi-layer Perceptron model to learn user preferences based on their responses to the questionnaires.
For each rider, a ranking list of potential partners for shared-rides is generated according to the similarities between the predicted preferences of her/him and other riders.}

    
\subsection{Repositioning}
\label{sec:review-repositioning}
\subsubsection{Collective Repositioning}
\revise{Some collective repositioning methods leverage RL techniques.
Ride-hailing repositioning for electric vehicles is studied in \cite{liang2020mobility, tang2020online}, in which the state of charge of the electric vehicles is an important factor to be considered.
Liang et al.~\cite{liang2020mobility} develop a solution method utilizing deep RL combined with binary linear programming to obtain a regional joint planning policy for electric vehicles with their state of charge considered.
Using binary linear programming, each vehicle repositioning action is modeled as a binary decision variable, and its weight in the objective is obtained by the value function learned by the policy iteration method.
Similarly, Tang et al.~\cite{tang2020online} also combine RL with combinatorial optimization, in which the RL learned policy is used to advise decision making in the optimization step.
Liang et al.~\cite{liang2021integrated} adopt temporal-difference (TD) learning to obtain action-value function.
Different from \cite{de2020efficient}, the settings in \cite{liang2021integrated} do not allow factorization of the joint action-value function into individual ones while guaranteeing global maximization.
Thus, they formulate two linear programming instances to collectively find the decisions for the vehicles.
To improve the stability of the training process in RL, Fluri et al.~\cite{fluri2019learning} propose a cascading multi-level learning model.
In this model, the area concerned is split in halves as the number of levels of learning increases.
The policy training process proceeds in a top-down manner, i.e., from less to more fine-grained area partitioning. 
The motivation behind is that the policy trained from a coarse level can serve as guidance to the finer levels, which avoids the instability caused by directly training a policy with a large state size (w.r.t. the number of regions).
Fluri et al.~\cite{fluri2019learning} propose to leverage the Lloyd K-means algorithm \cite{lloyd1982least} to partition the area concerned into multiple smaller regions.
Deng et al.~\cite{deng2020multi} leverage the Proximal Policy Optimization algorithm (PPO) \cite{schulman2017proximal} to learn the joint repositioning policy for vehicles, in which the value- and policy-function are approximated by neural networks.
Shi et al.~\cite{shi2019optimal} use Deep Deterministic Policy Gradient (DDPG) \cite{silver2014deterministic} to learn the grid-based multiple vehicles repositioning policy with the objective of total profits maximization.
\revise{In \cite{shou2020reward}, a mean-filed multi-agent RL approach is leveraged to collectively relocate the vehicles in ride-hailing.}}


%In collective repositioning, 
\revise{Some other collective repositioning solutions leverage various ML techniques to predict future information of a ride-hailing system, which plays an important role in guiding the platforms to make better repositioning decisions \cite{chen2022h}.
Riley et al.~\cite{riley2020real} leverage Vector autoregression to forecast the future demand from region to region.
The predicted demand and current system status are then fed into two mixed-integer programming instances to find the desired distribution of vehicles and the assignment of vehicles to regions, respectively.
Iglesias et al.~\cite{iglesias2018data} use a Long Short-Term Memory (LSTM) neural network to predict the future ride requests for each pair of origin and destination within a certain time period.
The predicted information is then used as input to their proposed mixed-integer linear programming instance, which is solved to find the optimal rebalancing actions.
Xu et al.~\cite{xu2018taxi} use two LSTM-based and Mixture Density Network (MDN)-based models to predict the distributions of origins and destinations of future requests, respectively.
With a prediction on the distributions, the repositioning decisions are then obtained by solving a mixed-integer programming problem with total idle driving distance minimization as the objective.
Cheng et al.~\cite{cheng18taxis} leverage a multilevel logistic regression model to predict the likelihood of ride requests occurring at different times and places.
The online repositioning planning decisions of drivers are obtained by leveraging a centralized multi-period stochastic optimization model with both the real-time and predicted demand considered.
Li et al.~\cite{li2020data} and Gao et al.~\cite{gao2020learning} formulate the repositioning task as a two-stage stochastic programming problem.
The source of the stochasticity is the underlying uncertainty of the future demands, the probability distribution of which is obtained by kernel density estimation and a deep learning model combining the LSTM and MDN in \cite{li2020data} and \cite{gao2020learning}, respectively.
Pouls et al.~\cite{pouls2020idle} propose a forecast-driven repositioning solution framework, the core of which is a mixed-integer programming problem with the demand predictions as inputs.
Moreover, it is solved by an off-the-shelf solver called Gurobi \cite{gurobi}.
Note that, in practice, not all planning decisions can be successfully executed by the drivers at the end.
Xu et al.~\cite{xu2020recommender} take the first step to predict the failure possibility of repositioning tasks in the decision-making process, including situations where drivers disobey the planning or end up being unmatched for an unexpectedly long time even though they follow the repositioning planning decisions accordingly.
In the latter case, drivers will be compensated.
The failure rate of each repositioning task is predicted by XGBoost \cite{chen2016xgboost} with both driver- and environment-related features as inputs.
\revise{The problem of multi-vehicle collaboration optimization aiming at maximizing the platform's profit is converted into a minimum cost flow problem, which is solved by an off-the-shelf method called GNU Linear Programming Kit (GLPK) \cite{makhorin2008glpk}. }}


\subsubsection{Distributed Repositioning}
Geographical regions or grids (i.e., abstracts of individual locations) are usually used to model the road networks in the problem of ride-hailing repositioning.
Different from most of the repositioning methods (e.g., \cite{lin2018efficient, riley2020real, ke2019optimizing, li2019efficient, zhou2019multi}) in which the region of interest is divided into predefined and static geographic zones, Castagna et al.~\cite{castagna2020demand, castagna2021demand} leverage the Expectation-Maximization clustering algorithm to derive zones for rebalancing vehicles in an online manner.
They leverage the Proximal Policy Optimization algorithm (PPO) \cite{schulman2017proximal} to train a policy for each vehicle to decide whether to make a pick-up, drop-off, or repositioning action.
Specifically, similar to \cite{tang2021value}, the repositioning destination is also sampled from a probability distribution over all potential positions, which is determined by the number of unserved requests.
Different from \cite{castagna2020demand, castagna2021demand}, Verma et al.~\cite{verma2017augmenting} propose an iterative method to dynamically split the zones based on their expected revenue (Q-values).
The iterative splitting process does not terminate until the historical data is exhausted for the Q-values learning.
\revise{Different from most of the works that model the drivers as agents, Jin et al.~\cite{jin2019coride} regard each geographical region as an agent.}
By hierarchically partitioning the target areas into regions with different granularities, they perform hierarchical RL where the multi-head attention mechanism is used to capture the impacts among the neighboring agents.
Guo et al.~\cite{guo2021multi} try various methods (e.g., Support Vector Regression, Random Forest Regression, and k-Nearest Neighbors regression) to predict future demand density, which is then used to evaluate each region for their spatial-temporal value.
Each available vehicle chooses to stay still or relocate to a neighbor region in a probabilistic manner based on their spatial-temporal values, which can help avoid over-saturation of supply.
In \cite{provoostdemandprop}, the region of interest is represented as a graph.
They build two neural networks to predict the demand on vertices and the passenger flows on edges, respectively.
The proposed repositioning algorithm aims at satisfying the demand on edges in the decreasing order with the nearest vehicles found by backward traversing.

\revise{However, in spite of the various grid-based methods as discussed in most of the related works mentioned above, e.g., \cite{lin2018efficient, guo2021multi}, Jiao et al.~\cite{jiao20deep, jiao2021real} argue that grid-based repositioning policies are not satisfactory in practice because of the excessively-simplified and overlooked non-stationarity in the environment caused by the dynamic environment and the large number of vehicles when coarse-grained region-wise decisions are considered.}
They put forward the process of carrying out repositioning %algorithms
in industrial production by combining offline learning, i.e., batch RL, and online planning stages, i.e., decision-time planning \cite{sutton1999reinforcement}.
To counter the issues of coarse-grained decisions, Kim and Kim \cite{kim2020optimizing} uses a graph to model the road networks which is more realistic.
They build a Graph Neural Network to predict the future demands.
The repositioning destination of each driver is decided greedily based on a function of the predicted demand, the number of excessive vehicles, and the distance information to each candidate position.

\revise{Some other works also spend special effort on tackling the non-stationarity.
With the observation that the actions of drivers are independent (based on self interests), Chaudhari et al.~\cite{chaudhari2020learn} propose a vanilla RL framework where each driver, based on a probabilistic value denoting the extent to which coordination is needed, stochastically chooses to perform an action guided by the independent or coordinated policy.
Note that, although vehicles execute repositioning decisions sequentially in this solution framework, coordination in the latter policy is explicitly considered by solving a minimum cost flow problem for the optimal rebalancing flow of vehicles among all the regions (which is similar to \cite{xu2020recommender}).
In addition, the independence between different repositioning policies learned by the drivers concurrently also contributes to the non-stationarity of the environment.
In this regard, Verma et al.~\cite{verma2019entropy} propose a method for each driver to learn the information of other vehicles in order to make a better planning decision.}
%Concretely,
The principle of maximum entropy \cite{jaynes1957information} is leveraged to improve the predictability of the distribution of drivers even with only limited knowledge available, e.g., the local density of supply.
To tackle the non-stationary challenge in online ride-hailing as well as the catastrophic forgetting of RL \cite{kemker2018measuring}, Haliem et al.~\cite{haliem2020adapool, haliem2021adapool} propose to learn multiple repositioning policies to deal with different contexts of environments (e.g., peak/non-peak hours and weekends/weekdays).
When changes in the distribution of experiences are identified by their proposed change point detection algorithm, switching among those different policies is enabled so as to enhance adaptability to the dynamic environment.
Lei et al.~\cite{lei2019optimal} define the concept of stochastic relocation matrix.
The element in the $i$-th row and the $j$-th column within the matrix represents the probability that an empty vehicle located in the $i$-th region should relocate to the $j$-th region.
\revise{To circumvent the curse of dimensionality, they leverage low-rank approximation to project the original matrix onto a low-dimensional vector.}
They propose a deep convolution-LSTM model to learn how to predict the approximation vector based on the system status.
To alleviate the instability of the state-value function approximator caused by the large scale of its states, Tang et al.~\cite{tang2019deep} propose to bound its outputs by regularizing its worst-case variation w.r.t. %any
changes in its inputs (i.e., states).
Transfer learning proposed in \cite{wang2018deep} is applied to increase the adaptability of the trained model across different cities.

\revise{Besides the traditional ML techniques discussed above, RL, being another well-known technique for decision making in non-stationary environments, has been a key technology in distributed repositioning \cite{khetarpal2022towards,xie2021deep,mao2021near}.}
\revise{Liu et al.~\cite{liu2022deep} propose a single-agent deep RL approach which relocates vacant vehicles to regions with a large demand gap in advance.}
Nguyen et al.~\cite{nguyen2018policy} propose to use the RL framework to train a homogeneous repositioning policy for all agents, i.e., vehicles.
\revise{The policy is trained in a centralized manner with collective behaviors of drivers considered while executing in a distributed manner.}
He and Shin \cite{he2019spatio} leverage Double DQN with their proposed spatial-temporal capsule-based neural network as the state-action value approximator.
The inputs of the network proposed include the location of the vehicle to be relocated, distribution of other vehicles and riders, ride preferences, and some external factors that have impacts on supply and demand, e.g., weather conditions and holiday events.
With all those information processed, the estimated value for each candidate position given the current state of the target vehicle is obtained, and the final decision can be decided in a probabilistic manner.
A more elaborate analysis is presented in their follow-up study \cite{he2020spatio}.
Yu et al.~\cite{yu2019markov} formulate the single-vehicle repositioning planning problem as a Markov Decision Process.
They propose to leverage parallelized matrix operations to re-formulate the Bellman equation \cite{sutton1999reinforcement}, thus reducing the computational complexity in finding optimal planning policy.
Multi-hop ride-hailing repositioning is considered in \cite{singh2019reinforcement, singh2021distributed}.
Similar to \cite{al2019deeppool}, they predict the number of vehicles in each region for certain time slots ahead of time using an estimated time of arrival (ETA) model.
Double DQN is adopted for each vehicle to choose the best neighbor region to move forward based on the current status of all the vehicles and the predicted demand and supply.
    
\subsection{Joint Matching and Repositioning}
\label{sec:review-joint}
In this part, we review methods that jointly optimize matching and repositioning with ML techniques. 
Note that all of them belong to the category of distributed planning. The research works in this part leverage RL to guide the decision making process.
Different from the review given by Qin et al.~\cite{qin2021reinforcement}, we focus on the works that jointly decide matching and repositioning.

Haliem et al.~\cite{haliem2020distributed-a, haliem2021distributed} propose to consider both the matching and repositioning in the ride-hailing planning process.
In their ride-hailing systems, each vehicle conduct initial matching by greedily searching the nearest requests, after which an insertion-based method is used to finalize the potential request list.
Then each driver, based on the value function learned by the DQN, weighs the requests in the final list.
The riders who receive those proposed ride offers can decide whether to accept the offers and join the trips where shared-rides are allowed.
The trips can be solo-ride or shared-ride.
Drivers are repositioned in parallel with the matching process.
Each driver takes actions indicated by his/her trained RL agent, i.e., the decision-making policy, independently.
Their proposed solution framework learns an optimal policy for each driver as opposed to those RL-based methods with collective planning scheme where a central policy is used, e.g., \cite{oda2018movi}.
Note that, in some works, although each driver makes decisions independently (e.g., \cite{haliem2020distributed-a, haliem2021distributed}), all drivers share one trained policy (e.g., \cite{manchella2020passgoodpool, manchella2021flexpool}).
Manchella et al.~\cite{manchella2020passgoodpool, manchella2021flexpool} propose to collectively optimize the system objectives, e.g., minimizing the waiting times and routing times.
Nevertheless, they allow distributed inference at the level of individual drivers. 
Their proposed model can be used by each vehicle independently.
It helps decrease computational costs associated with the growth of distributed systems. 
Specifically, they utilize a Double DQN with the experience relay mechanism.
Their model learns a probabilistic dependence between drivers' actions and the reward function.
The trained policy indicates a destination for each driver if s/he is not matched with any rider according to their proposed heuristic matching algorithm. 
Similar to \cite{xu2020highly, singh2019reinforcement, singh2021distributed}, multi-hop transit is enabled in their solutions.
Wang et al.~\cite{wang2018deep} model the matching and repositioning problems as a Markov Decision Process and propose learning solutions based on DQNs to optimize the trained policy for the drivers.
\revise{Their solution uses a temporal and spatial expanded action search strategy to accommodate the scenarios where there is only sparse training data, e.g., certain remote regions in the middle of the night.}
Besides, to increase the learning adaptability and
efficiency, they propose to use a transfer learning method to leverage the knowledge across both spatial and temporal spaces.

Besides \cite{haliem2020distributed-a, haliem2021distributed, manchella2020passgoodpool, manchella2021flexpool, wang2018deep},
DQN is used in other works as well, e.g., \cite{al2019deeppool, guo2022deep, tang2021value, li2020balancing}.
In \cite{al2019deeppool}, each vehicle decides its action by learning the impact of its action on the reward using a DQN model without coordinating with other vehicles.
In \cite{guo2022deep}, the vehicle repositioning procedure is formulated as a Markov Decision Process.
By sampling the future riders based on the historical probability distribution, the proactive relocation of vehicles is realized via a deep RL framework, which is composed of a Convolutional Neural Network and a Double DQN module. \revise{Then a request-vehicle assignment scheme is presented based on the value function attained from the vehicle repositioning process.}
\revise{Similarly, Tang et al.~\cite{tang2021value} propose a planning framework for tackling both the matching and repositioning tasks, the core of which is a unified value function which is trained offline using abundant historical data and is updated during the online phase.}
With the value function learned, the matching problem is then solved by the method proposed in \cite{xu2018large}, while the reposition destination of each idle vehicle is determined in a probabilistic manner following the distribution given by the discounted long-term values of all the candidate positions.
Li and Allan \cite{li2020balancing} also leverage a global value function for both the tasks of matching and repositioning, which is learned by the value iteration algorithm with historical data of ride requests.


 %% for proof-reading; comment this line before officially submitting the draft.

\title{\LARGE \bf

Learning to Adapt the Parameters of Behavior Trees and Motion Generators (BTMGs) to Task Variations 
}

\author{Faseeh Ahmad$^{1}$, Matthias Mayr$^{1}$, and Volker Krueger$^{1}$% <-this % stops a space
	\thanks{*This work was partially supported by the Wallenberg AI, Autonomous Systems and Software Program (WASP) funded by Knut and Alice Wallenberg Foundation.}% <-this % stops a space
	\thanks{$^{1}$Department of Computer Science, Faculty of Engineering (LTH), Lund University, SE~221~00 Lund, Sweden. E-mail: <firstname>.<lastname>@cs.lth.se.
	}%
}

\begin{document}

\maketitle
\thispagestyle{empty}
\pagestyle{empty}

%%%%%%%%%%%%%%%%%%%%%%%%%%%%%%%%%%%%%%%%%%%%%%%%%%%%%%%%%%%%%%%%%%%%%%%%%%%%%%%% 
\begin{abstract}
The ability to learn new tasks and quickly adapt to different variations or dimensions is an important attribute in agile robotics.
In our previous work, we have explored Behavior Trees and Motion Generators (BTMGs) as a robot arm policy representation to facilitate the learning and execution of assembly tasks. 
The current implementation of the BTMGs for a specific task may not be robust to the changes in the environment and  may not generalize well to different variations of tasks.
We propose to extend the BTMG policy representation with a module that predicts BTMG parameters for a new task variation. 
To achieve this, we propose a model that combines a Gaussian process and a weighted support vector machine classifier. This model predicts the performance measure and the feasibility of the predicted policy with BTMG parameters and task variations as inputs. Using the outputs of the model, we then construct a surrogate reward function that is utilized within an optimizer to maximize the performance of a task over BTMG parameters for a fixed task variation.
To demonstrate the effectiveness of our proposed approach, we conducted experimental evaluations on push and obstacle avoidance tasks in simulation and with a real \textit{KUKA iiwa} robot. Furthermore, we compared the performance of our approach with four baseline methods.
\end{abstract}

%%%%%%%%%%%%%%%%%%%%%%%%%%%%%%%%%%%%%%%%%%%%%%%%%%%%%%%%%%%%%%%%%%%%%%%%%%%%%%%%
\section{Introduction}

Robots have been utilized effectively for many years in repetitive and automated industrial processes. However, despite the shift towards smaller batch sizes and increased demand for customization, many robot systems still require a lengthy and expensive reconfiguration process. To keep up with the demands of society and modern industrial production, robots should have the ability to adapt quickly to different situations. 
In these situations, the task formulations should be robust to failures, interpretable, and possibly reactive to failures. Additionally, the task formulations should also be adaptable to different variations or dimensions of the same task, such as pushing an object to different locations, picking an object from any location in the space, and avoiding an obstacle with different shapes and positions.

To overcome the challenges, Rovida F. et al.~\cite{rovida18btmg} have suggested a representation that combines behavior trees (BT)~\cite{colledanchise142iicirsa, colledanchise17bt} and motion generators (MG), (BTMG). 
In our previous work, we used BTMGs to model skills for contact-rich tasks such as inserting a peg into the hole to mimic engine assembly~\cite{rovida18btmg, mayr21iros} and pushing an object to a target location~\cite{mayr2022combining,mayr2022skill}.

\begin{figure}
    \centering
    \includegraphics[width=\columnwidth]{fig/experimental-setup-small.jpg}
    \caption{The experimental setup. It shows the object with the skewed weight distribution that is pushed with a \SI{45}{\milli\meter} wide peg. On the table the different start and goal positions for the object can be seen in different colours. On the sides, some example sizes for obstacles are shown.}
    \label{fig:exp-setup}
\end{figure}

A BTMG is a parameterized policy representation that combines the strengths of both behavior trees and motion generators. Behavior trees provide a clear and intuitive way to describe the high-level logic of the robot's behavior, while motion generators generate the low-level motion commands by controlling the end-effector in Cartesian space. For a more concrete definition of motion generators, refer to ~\cite{rovida18btmg}. The parameters of a BTMG can be used to specify the structure of the behavior tree as well as values such as controller stiffness. 

BTMGs are easy to interpret and can be designed to be \textbf{robust} to faults and failures that can occur during execution~\cite{rovida18btmg}. Furthermore, they have the ability to be \textbf{reactive}~\cite{colledanchise142iicirsa}, allowing the robot to adapt and respond to current circumstances.
Simple BTs can also be systematically combined with more complex ones to solve complex tasks ~\cite{rovida18btmg, mayr21iros, rovida172iicirsi}.

BTMGs are a promising technique for motion modeling because of their explicitness, robustness, and reactiveness.
% This can be addressed by properly writing skill description, execution strategy, and knowledge integration, but it can be challenging and also requires an expert. Another solution could be to use a reasoner, but that also requires expertise. 
There are mainly three ways to set the parameters of BTMGs. One way is to specify them manually or fine-tune them by experts~\cite{rovida18btmg}. Another way is to determine those parameters through reasoning. However this requires the existence of such a reasoner for the task at hand, which can not always be assumed. Finally, BTMG parameters can be learned through reinforcement learning (RL)~\cite{mayr2022combining, mayr2022skill, mayr22priors}. However, learned BTMG parameters are in many cases scenario-specific and changes in the setup may require relearning them. 

Setting BTMG parameters using these methods can limit the usage of BTMGs in scenarios that require quick adaptability. 
For example, tasks such as pushing an object to different locations, picking an object from various locations, or even picking objects with various shapes would require updating the parameters of the respective BTMGs. 
This problem is also present in the original formalization of dynamic motion primitives (DMPs)~\cite{976259,ijspeert2002learning} and was later addressed in~\cite{ijspeert2013dynamical}.

In this paper, we propose an extension to the BTMG formulation that enables quick adaptation to different task variations by incorporating a model that combines a Gaussian process (GP) and a weighted support vector machine (SVM) classifier. 
Our model uses a GP to learn a function that predicts the performance measure of a policy using task variations and BTMG parameters as inputs. Furthermore, the model also trains a weighted SVM classifier that predicts the feasibility of a  policy. 
For example, in a push task, the performance measure of a policy can be given by its overall reward, which depends on the error between the actual and target position of the pushed object. In this task, a policy can be feasible when this error is below a user-defined threshold. Once the model is trained, we optimize the BTMG parameters over the resulting surrogate reward function for a given new task variation.

The following are our main contributions:
\begin{itemize}
    \item We extend BTMG policy representation that enables it to quickly adapt to task variations.
    \item We propose a model that combines a GP and a weighted SVM classifier to predict the performance measure and feasibility of a BTMG policy for a new task variation, and subsequently optimize the output of the model to obtain resulting BTMG parameters.
    \item We evaluate the performance of the proposed method in simulation and on a real \textit{KUKA iiwa} robot for two tasks and compare its performance with four baselines.
\end{itemize}
\section{Related Work}
Movement primitives, based on motor primitives theory~\cite{mussa1999modular,flash2005motor}, are mathematical formulations of dynamic systems that generate motions.  Two well-known movement primitives used in robotics are Dynamic Movement Primitives (DMPs)~\cite{976259,ijspeert2002learning} and Probabilistic Movement Primitives (ProMPs)\cite{paraschos2013probabilistic}.  Movement primitives can be generalized and have proven successful in various robotics applications,  such as dynamic motion primitives ~\cite{976259,ijspeert2002learning}. Similar to our BTMGs, DMPs intially lacked the capacity to generalize to different task parameters. This was resolved later by introducing a small change in the transformation system~\cite{ijspeert2013dynamical}.

While both DMPs and BTMGs are capable of generating motions through attractor landscapes, the parameters for DMPs are learned implicitly from a set of demonstrations, whereas parameters for BTMGs can be explicitly specified manually, inferred through a reasoner, or learned using RL. Nevertheless, a comprehensive comparison of the two approaches would require further investigation and is outside the scope of this paper.

DMPs have been extended with intermediate via points~\cite{ning2011accurate,ning2012novel, weitschat2018safe, zhou2019learning}, and can generalize to new goals by interpolating weights of neighboring DMPs~\cite{weitschat2013dynamic} or by using Gaussian Process Regression (GPR) to generate new parameters~\cite{forte11ras}. Furthermore, GPs~\cite{williams2006gaussian}  have been used to generalize DMPs to external task variations, arbitrary movements, and adapting trajectories to new situations online in~\cite{alizadeh2016learning, fanger2016gaussian, forte11ras}, respectively. In~\cite{lee18joss}, Gaussian mixture models are used to learn the mapping of task parameters and the forcing term of DMPs. 

The mixture of movement primitives (MoMP) algorithm introduced in~\cite{muelling2010learning, mulling2013learning}, can also be used to generalize the basis movements stored in the library. The MoMP algorithm captures the robot's position and velocity as parameters for the expected hitting position and velocity. A new motion is generated by a weighted sum of DMPs, assigning a probability to a DMP based on the sensed state. MoMPs and ProMPs have been applied successfully in various applications, including learning striking movements for table tennis robots~\cite{Muelling_ICHR_2012, Gomez-Gonzalez_PICHR_2016} and solving Human-Robot collaborative tasks~\cite{gjm_2016_AURO_c} using ProMPs.

We draw inspiration from prior work on DMPs to extend BTMG's formulation by incorporating generalization to different task variations using GP, as seen in~\cite{forte11ras, alizadeh2016learning, fanger2016gaussian}.  These studies employed GPs to directly map task variations to DMP parameters, which we refer to as the \textit{direct} model in this paper. However, our approach differs significantly in how we use GPs. Instead of using the \textit{direct} model, we propose a model that combines GP with a weighted SVM classifier to predict the performance of tasks and the feasibility of a policy, using task variations and BTMG parameters as inputs. Since our model predicts both performance measure and feasibility, we refer to it as the \textit{PerF} model, short for performance and feasibility.
\section{BTMG and Task Variations}
We define BTMG as a parametric policy representation, $\text{BTMG}(\bds{\theta})$ where $\bds{\theta} \in \mathbb{R}^N$. The parameters $\bds{\theta}$ can range from determining the structure of the behavior tree (BT) to specifying the controller stiffness values of the motion generator (MG). These parameters are further subdivided into \textbf{intrinsic} parameters $\bds{\theta}_i$ and \textbf{extrinsic} parameters $\bds{\theta}_e$~\cite{ahmad2022generalizing}.

Intrinsic parameters $\bds{\theta}_i$ determine the structure of the behavior tree, the number of control nodes, the type of motion generator, etc. 
For example, consider a policy $T_p$ for a push task, which has intrinsic parameters $\bds{\theta}_i$. These parameters are fixed and independent of the task instance, meaning that $T_p$ uses the same $\bds{\theta}_i$ values regardless of the starting position, or the target position of the object. In other words, $\bds{\theta}_i$ is situation-invariant. Within the scope of this paper, these parameters are assumed to be known a priori.

Extrinsic parameters $\bds{\theta}_e$ are situation dependent e.g. to determine the applied force, offsets, and the velocity of the end effector. Again, $\bds{\theta}_e$ can be specified manually~\cite{rovida18btmg, rovida16icaps}, inferred through a reasoning framework, or learned using RL. We have already demonstrated how RL can be used to obtain BTMG parameters~\cite{mayr21iros} and used it in simulation and on a real robot to solve multi-objective tasks~\cite{mayr2022skill, mayr22priors}.
%\textcolor{red}{In this paper, we categorize the structure of BT as intrinsic parameters, as the tasks under consideration do not necessitate any modifications to the BT structure.[vok: I think this comment adds only confusion, and it is not even 100% right...]}

In addition to $\bds{\theta}$, we also consider task variations $\bds{v} \in \mathbb{R}^M$. 
% Every entry of $v$ denoted by $v_k$ represents a task variation. 
Task variations refer to different possible alterations of a given task, such as different start and goal positions of an object. For example, a task variation $\bds{v}$ in the case of a push task would be a $4\text{D}$ vector consisting of the values of the start and goal positions of the object along the horizontal and vertical axes. 

Note that the task variation parameters are different from the extrinsic BTMG parameters (Figure~\ref{fig:push-parameters}). We take two task variations $\bds{v}_1 = (v_{s_{x}}, v_{s_{y}}, v_{g1_{x}}, v_{g1_{y})}$ and $\bds{v}_2= (v_{s_{x}}, v_{s_{y}}, v_{g2_{x}}, v_{g2_{y})}$ that define the start and goal positions of the object. For variations $\bds{v}_1$ and $\bds{v}_2$, we have corresponding $\bds{\theta}_{e1} = (\theta_{e1_{s_{x}}}, \theta_{e1_{s_{y}}}, \theta_{e1_{g_{x}}}, \theta_{e1_{g_{y}}})$ and $\bds{\theta}_{e2} = (\theta_{e2_{s_{x}}}, \theta_{e2_{s_{y}}}, \theta_{e2_{g_{x}}}, \theta_{e2_{g_{y}}})$ that collectively define the start and the goal locations for the pushing action.

\begin{figure}[tpb!]
	{
        \vspace{0.1cm}
		\setlength{\fboxrule}{0pt}
		\framebox{\parbox{3in}{
  		\includegraphics[width=1\columnwidth]{fig/var_vs_btmg_v2.png}
        }
	}
 }
	\caption{An illustration of two simplified task variations $\bds{v}_1$ and $\bds{v}_2$ in the pushing task that only vary the goal location. The orange and blue vectors are set by the respective learned extrinsic parameters $\bds{\theta}_{e1}$ and $\bds{\theta}_{e2}$, so that they define the resulting green and red push vectors that should successfully push the object. } 
	\label{fig:push-parameters}
	%\vspace{-2em}
\end{figure}
As $\bds{\theta}_i$ has no impact on adapting BTMGs to different variations, our objective in this paper is to establish a relationship between $\bds{\theta}_e$ and $\bds{v}$ that would enable the adaptation of BTMGs to new variations.
\section{Approach}
\label{sec:approach}
In this section, we explain how we adapt BTMG parameters for a new task variation by using the \textit{PerF} model.
Figure~\ref{fig:flowchart} shows how the \textit{PerF} model works in comparison with a \textit{direct} model. The overall approach is divided into the training (Sec.~\ref{sec:trainingphase}) and query phase (Sec.~\ref{sec:Query Phase}). In the training phase, we pass each task variation $\bds{v}_k \in \mathbb{V}_{train}$, into an extended RL pipeline similar to~\cite{mayr2022skill}.
For each learning process for different task variations, we utilize three sets of outputs from the RL pipeline to train the \textit{direct} and the \textit{PerF} models:
\begin{enumerate}
\item \textit{Best policies}: For every task variation we get the best performing policy:\\
$\mathbb{T}=\{(\bds{v}_k, \bds{\theta}^*_{e,\bds{v}_k})|k=1, \ldots, n\}$

\item \textit{All evaluated configurations and their rewards:}\\
$\mathbb{K}= \{(\bds{v}_k, \bds{\theta}_{ei,\bds{v}_k}, r_{\bds{\theta}_{ei,\bds{v}_k}})|k=1, \ldots, n \text{ and } i=1, \ldots, t \leq t_\text{max}\}$

\item \textit{All evaluated configurations and their feasibility:}\\
$\mathbb{E}= \{(\bds{v}_k, \bds{\theta}_{ei,\bds{v}_k}, f_{\bds{\theta}_{ei,\bds{v}_k}})|k=1, \ldots, n \text{ and } i=1, \ldots, t \leq t_\text{max}\}$
\end{enumerate}

The \textit{direct} model $M$ is trained with the set $\mathbb{T}$ and, as a result, learns to predict $\hat{\bds{\theta}}_e$ given $\bds{v}$. On the other hand, the \textit{PerF} model is trained with the sets $\mathbb{K}$ and $\mathbb{E}$ and as a result it learns to predict the reward $\hat{r}$ and feasibility $\hat{f}$ of a policy with parameters $\bds{\theta}_e$. The model further uses $\hat{r}$ and $\hat{f}$ to generate a surrogate reward function that obtains $\hat{\bds{\theta}}_e$ given $\bds{v}$. For more details on how we obtain set $\mathbb{T}$, we direct the reader to~\cite{mayr2022skill}. To obtain sets $\mathbb{K}$ and $\mathbb{E}$, we follow the same procedure as in~\cite{mayr2022skill}, retaining all configurations along with their respective rewards and feasibilities for a given task variation.

The intuition behind using the \textit{PerF} model together with an optimizer is to guide the combination of GP and weighted SVM towards predicting policy parameters $\bds{\theta}_e$ that prioritize performance measure and feasibility. In contrast, the \textit{direct} model does not take into account the performance measure and feasibility. In the following subsections, we explain our approach in more depth.
\subsection{Training Phase}
\label{sec:trainingphase}
We frame the mapping of the task variations $\bds{v}$ to the extrinsic BTMG parameters $\bds{\theta}_e$ as a supervised learning problem. The training phase aims to learn two functions: $\hat{J}$ that predicts the reward achieved by a policy and $\hat{F}$ that predicts if a policy is feasible, see Figure~\ref{fig:flowchart}.  We propose to use GP and weighted SVM to learn $\hat{J}:(\bds{\theta}_e,\bds{v})\mapsto\hat{r}\in\mathbb{R}$ and $\hat{F}:(\bds{\theta_e},\bds{v})\mapsto\hat{f}\in\{0,1\}$. $\hat{J}$ and $\hat{F}$ are trained by data points in sets $\mathbb{K}$ and $\mathbb{E}$, provided by the RL pipeline introduced in \cite{mayr21iros}. 

For each task variation, $\bds{v}_k \in \mathbb{V}_{train}$, similar to ~\cite{mayr21iros,mayr2022skill}, we define $J_{\bds{v}_k}(\bds{\theta}_{e})$ as the expected sum of individual rewards over time, given a sequence of extrinsic parameters $\bds{\theta}_{e1}, \bds{\theta}_{e2}, \ldots, \bds{\theta}_{et} \in \bds{\theta}_e$. 

In ~\cite{mayr21iros,mayr2022skill}, we use Bayesian optimization (BO) as a black-box optimization method to obtain the optimal policy parameters $\bds{\theta}^*_{e}$ and the best reward $J_{\bds{v}_k}(\bds{\theta}^*_{e})$. In this paper, however, we use BO to obtain $J_{\bds{v}_k}(\bds{\theta}_{e})$ by computing $J_{\bds{v}_k}(\bds{\theta}_{e1}), J_{\bds{v}_k} (\bds{\theta}_{e2}), \ldots, J_{\bds{v}_k}(\bds{\theta}_{et})$ over the sequence $\bds{\theta}_{e1}, \bds{\theta}_{e2}, \ldots, \bds{\theta}_{et}$. This allows us to not only have the optimal policy parameters $\bds{\theta}^*_{e}$ and the corresponding best reward $J_{\bds{v}_k}(\bds{\theta}^*_{e})$ but it also provides us with intermediate $\bds{\theta}_{et}$ and $J_{\bds{v}_k}(\bds{\theta}_{et})$. Overall, this provides us with large amount of data to train the $\hat{J}$ function and allows us to capture the overall reward landscape better. 

In addition to learning the reward function $\hat{J}$, we also learn the feasibility function $\hat{F}$. The motivation behind learning $\hat{F}$ is twofolds: First, it provides a user-defined metric to evaluate the feasibility of a policy and second, it complements the reward formulation of a task by addressing the potential shortcomings of inaccurate reward formulations. In principle, we do not need to optimize feasibility if the reward formulation covers all aspects of the task. However, in practice, reward formulation is challenging, so feasibility addresses these shortcomings effectively. It ensures learned policies align with the task's requirements, despite imperfect reward formulations.
% \textcolor{red}{Solely optimizing for performance would result in policies that achieve high rewards but may compromise feasibility. For instance, in an obstacle avoidance task, policies could be developed that enable the end-effector to move directly towards the goal location in a straight line, thus minimizing the distance traveled and achieving a high reward. However, such a policy could potentially result in a collision with the obstacle, posing a safety risk in both simulation and reality. Therefore, such a policy would be considered infeasible.}
 
For a given task variation $\bds{v}_k$, we define the feasibility function $F_{\bds{v}_k}(\bds{\theta}_{e})$ 
as a binary function that maps to 1 or 0 depending on whether the policy achieves a user-defined metric of feasibility or not. Similar to $J_{\bds{v}_k}(\bds{\theta}_{e})$, we obtain $F_{\bds{v}_k}(\bds{\theta}_{e})$ by computing $F_{\bds{v}_k}(\bds{\theta}_{e1}), F_{\bds{v}_k}(\bds{\theta}_{e2}), \ldots, F_{\bds{v}_k}(\bds{\theta}_{et})$ for the sequence of evaluations $\bds{\theta}_{e1}, \bds{\theta}_{e2}, \ldots, \bds{\theta}_{et}$. 
For more details about the pipeline, we refer the reader to the policy optimization section in~\cite{mayr21iros,mayr2022skill}. 

To model $\hat{J}$ and $\hat{F}$, we obtain a sequence of BTMG parameter vectors, ${\bds{\theta}_{e1}, \bds{\theta}_{e2}, \ldots, \bds{\theta}_{et}}$, along with their corresponding reward values $J_{\bds{v}_k}(\bds{\theta}_{e1}), J_{\bds{v}_k}(\bds{\theta}_{e2}), \ldots, J_{\bds{v}_k}(\bds{\theta}_{et})$ and feasibility values $F_{\bds{v}_k}(\bds{\theta}_{e1}), F_{\bds{v}_k}(\bds{\theta}_{e2}), \ldots, F_{\bds{v}_k}(\bds{\theta}_{et})$ for task variations. We then use these data points to train a GP and a weighted SVM classifier. This enables us to effectively model the underlying $J$ and $F$.
\begin{figure}[tpb]
	{
        \vspace{0.1cm}
		\setlength{\fboxrule}{0pt}
            \begin{center}
                
		\framebox{\parbox{3in}{
  		\includegraphics[width=0.9\columnwidth]{fig/flowchart_model.drawio.png}
            }
            }
            \end{center}
        }
	\caption{The pipeline of our approach and the \textit{direct} model baseline. For every task variation $\mathbf{v}$, an RL problem is solved and the respective results are provided to the GP models. When querying for a new task variation $\mathbf{v}_{p}$ both models are queried for a set of extrinsic parameters $\bds{\hat{\theta}}_{e}$.}
	\label{fig:flowchart}
	%\vspace{-2em}
\end{figure}
\subsection{Query Phase}
\label{sec:Query Phase}
The goal of this phase is to query the trained model with a new task variation $\bds{v}_p \in \mathbb{V}_{test}$ and obtain a $\hat{\bds{\theta}}_e$ by optimizing $\hat{J}(\bds{\theta}_{et}|\bds{v}_p)$ under the feasibility constraint $\hat{F}(\bds{\theta}_{et}|{\bds{v}_p})$ (Figure~\ref{fig:flowchart}). For this purpose, we use the $\hat{J}$ and $\hat{F}$ obtained in the training phase. We solve this as an optimization problem over a sequence of $\bds{\theta}_{e}$ for a new $\bds{v}_p$.

We begin the optimization process by specifying the optimizer type, the bounds for $\bds{\theta}_e$, and the maximum number of iterations $t_\text{max}$. In our experiments, we used the Limited-memory Broyden–Fletcher–Goldfarb–Shanno (L-BFGS)~\cite{byrd1995limited,zhu1997algorithm} algorithm, which refines an initial estimate of $\bds{\theta}_{e1}$ to iteratively obtain improved evaluation points $\bds{\theta}_{et}$, where $t \leq t_\text{max}$, using the derivative as the driving function. For each new task variation $\bds{v}_k$, we run the optimizer to obtain a sequence of evaluation points $\bds{\theta}_{et}$.

Using $\hat{J}$ and $\hat{F}$, we define a surrogate reward \mbox{$r_{\bds{v}_{p}} = \hat{r}_{\bds{\theta}_{et},\bds{v}_{p}} - (1-\hat{f}_{\bds{\theta}_{et},\bds{v}_{p}})) * \mu$}. Here, the first term corresponds to the output reward value computed by $\hat{J}$, while the second term penalizes the reward if $\hat{f}_{\bds{\theta}_{et},\bds{v}_{p}}$ maps to 0. We penalize the reward $\hat{r}_{\bds{\theta}_{ei},\bds{v}_{p}}$ by a small factor $\mu$. 
We query the surrogate reward $r_{\bds{v}_{p}}$ for defined number of iterations or until the optimizer converges. 

After the optimization phase, we select the $\bds{\theta}_{et}$ that maximizes both $\hat{J}(\bds{\theta}_{et}|\bds{v}_p)$ and is feasible $\hat{F}(\bds{\theta}_{et}|\bds{v}_p)$. 
\section{Experiments}
\label{sec:experiments}
We evaluated the efficacy of our approach in simulation and also by transferring of the simulation results to a real \textit{KUKA iiwa} manipulator for two tasks: an obstacle avoidance task and a pushing task, each having its own challenges. For simulation, we utilized the \textit{DART} simulation toolkit \cite{lee18joss} and in both simulation and reality, the robot arm was controlled using a Cartesian impedance controller~\cite{mayr2022c++}, which helps reduce the disparities between simulation  and reality. Additionally, for the push task, we further reduce the sim-to-real gap by adjusting the friction coefficient appropriately. For more detailed information on bridging the sim-to-real gap, please refer to~\cite{mayr21iros}.

To train our model, we considered 20 task variations that are learned for the same amount of iterations each. Using the method detailed in Sec.~\ref{sec:trainingphase}, we train the GP and the weighted SVM classifier with the resulting BTMG parameters, the feasibility, and the reward values. The weights of the SVM classifier are adjusted automatically to adjust bias induced by an unequal number of feasible and non-feasible policies. We then tested our approach on 20 unknown task variations. This experiment is repeated five times for both tasks to show the robustness of the approach. 
%\textcolor{red}{In both experiments, we carefully select the task variation space, taking into account the physical limitations of the robot. We are confident that the chosen task variation space adequately showcases the effectiveness of our approach.} 

We compare the performance of our approach with four baselines: 
\begin{figure}[tpb]
	{
        \vspace{0.1cm}
		\setlength{\fboxrule}{0pt}
            \begin{center}
		\framebox{\parbox{3in}{
		      \includegraphics[width=0.9\columnwidth]{fig/obstacle-experiment2.pdf}
            }
	       }
            \end{center}
    }
	\caption{The obstacle task with some of the variations of the object location, width, and height. For each object configuration, valid example trajectories are shown in the same color. For the red trajectory, the intermediate goal points ($\mathbf{g}_1$ and $\mathbf{g}_2$) and two motion switching thresholds ($p_1$ and $p_2$) are shown.}
	\label{fig:obstacle-task}
\end{figure}

\begin{enumerate}
    \item \textit{Learned}: This baseline uses the RL pipeline described in~\cite{mayr2022skill} to learn the BTMG parameters directly for the test variations. It shows which performance could be achieved if a new variation is learned from scratch instead of querying the model. Notably, our training data is generated in this way.
    \item \textit{Direct}: This model takes the best parameters for the training variations ($\mathbb{T}$) and learns a direct mapping from task variations to BTMG parameters without explicitly learning the reward.
    \item \textit{Nearest Neighbor}: For each test variation, we select the closest task variation in the training set and choose the corresponding BTMG parameters.
    \item \textit{Single Policy}: The learned BTMG parameters of a single training variation are used for all test variations. This baseline shows how well and how often the learned parameters for one task variation can be utilized in a different one without any changes.
\end{enumerate}
Although our baselines may seem simplistic, they are deliberately selected to provide insights into the functionality and performance of our approach.
% The "Learned" baseline showcases policies acquired through RL, which are expected to perform well given their learned nature. The "Direct" baseline demonstrates that training the model solely with the best policies is insufficient; the model must also explore the task space. The "Nearest Neighbor" baseline explores the possibility of interpolating a policy for a new task variation by selecting an existing policy based on its closest distance. Lastly, the "Single Policy" baseline investigates the existence of a global policy that can effectively address all task variations.
Each of these baselines serves a specific purpose in understanding the capabilities and limitations of our approach.

% \textcolor{red}{Table~\ref{tab:learned} and Table~\ref{tab:optimization} provide statistics on policy computation for both tasks using the RL-pipeline and surrogate function optimization, respectively. These results are derived from the same 100 test variations employed in generating Figure~\ref{fig:results-obstacle},~\ref{fig:results-push}. 
%The tables display information such as the minimum and maximum computation times for policies, as well as the overall mean, median, and standard deviation. 
% Notably, optimizing with the surrogate function (Table~\ref{tab:optimization}) is significantly faster than generating a policy from scratch (Table~\ref{tab:learned}). 
%The variation in computation times across different task variations, as indicated by the standard deviation, can be attributed to the stochastic nature of the process and the optimization of varying numbers of parameters within different search spaces.
% } 
We consider task-specific reward functions for both tasks. The rewards and feasibility measures for the tasks are defined separately in their respective sections. 

\subsection{Obstacle Avoidance Task}
\label{sec:exp-obstacle}
The objective of the obstacle avoidance task is to move the robot's end effector from the start to the goal location while avoiding an obstacle in the workspace. As shown in Fig.~\ref{fig:obstacle-task}, the obstacle can vary in size and position. The goal is to find policies that navigate the robot around the obstacle while completing the task as quickly as possible, without violating the safety constraints that require the end effector to maintain a safe distance from the obstacle.

We consider three task variations: 1) obstacle height, 2) obstacle width, and 3) obstacle position in a horizontal direction (left-right in Fig.~\ref{fig:obstacle-task}). The obstacle varies in height from \SI{0.049}{\meter}
 to \SI{0.331}{\meter} and in width from \SI{0.09}{\meter} to \SI{0.331}{\meter}. The horizontal position ranges from \SI{0.274}{\meter} to \SI{0.311}{\meter} with respect to the origin. We use Latin hypercube sampling to ensure a more even sample distribution and obtain 20 task variations from the specified ranges. We learn each variation for 120 iterations.

This learning problem formulation has three rewards: 1) a fixed success reward, 2) a goal distance reward, and 3) an obstacle avoidance reward. The fixed success reward assigns a fixed reward if the BT finishes successfully. The positive goal distance reward increases, the closer the end effector gets to the goal. The obstacle avoidance reward is a negative function that penalizes end-effector states that are close to the obstacle. These reward functions are combined to encourage fast execution while discouraging getting too close to the obstacle.
A policy is considered feasible if it satisfies two conditions: First, the end effector does not come closer to the obstacle than \SI{40}{\milli\meter}. Second, the policy must successfully complete the BT by bringing the end effector to the goal position.

\begin{figure}[tpb!!]
	{
		\setlength{\fboxrule}{0pt}
		\framebox{\parbox{3in}{
            \begin{center}
		      \includegraphics[width=0.95\columnwidth]{fig/obs_sim-success-rates.pdf}
                \includegraphics[width=0.85\columnwidth]{fig/obs_real-success-rates.pdf}
            \end{center}
            }
	       }
    }
	\caption{The total reward (a, c) and the execution time (b, d) of the obstacle task in simulation (a, b) and on the real system (c, d). The box plots show the median (black line) and interquartile range ($25^{th}$ and $75^{th}$ percentile); the lines extend to the most extreme data points not considered outliers, and outliers are plotted individually. The success percentages are shown below the method names.}
	\label{fig:results-obstacle}
\end{figure}

The policy for this task has six learnable parameters consisting of two coordinates of the intermediate goal points and two thresholds to transition between goal points. A more detailed description of the task is provided in~\cite{mayr21iros, mayr2022skill}. Notably, the structure of this policy with its thresholds allows for different movement strategies. For example, for flat obstacles, the goal can be reached with only a single intermediate point, while larger obstacles require both intermediate points, as shown in Fig.~\ref{fig:obstacle-task}.
\subsubsection*{Results and Discussion}
For the evaluation, we randomly sample 20 new task variations ($\mathbb{V}_{test}$) that are not included in the training set, and compare the performance of our proposed model and the baseline methods. Specifically, we assess the execution time and the reward achieved by each parameter configuration in the new task variation. The reward value is chosen as a performance metric as it reflects how well a policy balances between the goal-reaching and obstacle-avoidance objectives expressed in the reward functions.

 The simulation results are shown in Fig.~\ref{fig:results-obstacle}a) and b) and Table~\ref{tab:results}. They show that the policies obtained by optimizing the output of our \textit{PerF} model performs similarly to the policies that are explicitly learned. Our model achieves a success percentage of 87\percent compared to the 89\percent of the learned ones and a total reward in a similar range. In contrast to that, the nearest neighbor baseline succeeds only in 71\percent of the variations. The \textit{direct} model also only achieves a success percentage of 67\percent and has significantly more outliers in the reward. Further investigation indicates that the reason for the low performance is that an interpolation between policies is often not valid. This is especially the case between motion configurations that use a single or both intermediate points.

Based on these results from simulation we also evaluated the learned policies, our model outputs and the nearest neighbor policies on the real robot system. Although this includes a transfer from simulation to the real system, the results shown in Fig.~\ref{fig:results-obstacle}c) and d) have only minor variations from the simulation results. This also demonstrates the robustness of this policy formulation as a whole.

\subsection{Push task}
\label{sec:exp-push}
The goal of this task is to push an object from a varying start location to a varying goal location. The object is shown in Fig.~\ref{fig:exp-setup} and has a skewed weight distribution with respect to its bounds.
 
We consider two types of task variations: 1) the starting position of the object in both horizontal directions and 2) the goal position of the object in both horizontal directions.
For the starting position, we consider samples from a circle with a diameter of \SI{0.16}{\meter} around a center point. For the goal position, a triangular-shaped region is used. Fig.~\ref{fig:exp-setup} shows the start and goal positions for a single repetition.

The learning formulation has two rewards: 1) the object position reward, which is a function of the difference between the actual and desired goal position, and 2) the object orientation reward, which is based on the difference between the actual and desired goal orientation. For our experiment, we prioritize the object position reward, which is weighted 10 times more heavily than the orientation reward.

Similarly to previous work~\cite{mayr2022combining, mayr2022skill}, the push task has four BTMG parameters that are learned. They are depicted in Fig.~\ref{fig:push-parameters}. These parameters control additional start and goal offsets in the horizontal directions $(x,y)$, determining the shape of the push vector that is indicated in Fig.~\ref{fig:push-parameters}. The start and goal orientation of the object for this task are fixed.

The object being pushed is an right-angled triangular object with dimensions \SI{0.3}{\meter} x \SI{0.15}{\meter} x \SI{0.07}{\meter}, and a weight of \SI{2.5}{\kilogram}. 
The tool on the end effector is a cubic peg with side lengths of \SI{45}{\milli\meter}
and therefore covers less than 15\percent of the side length of the object. 
In this task, the error between the desired goal position and orientation and the achieved one serves as direct performance measures for the policy.

\begin{figure}[tpb!!]
	{
		\setlength{\fboxrule}{0pt}
		\framebox{\parbox{3in}{
            \begin{center}
		      \includegraphics[width=0.95\columnwidth]{fig/push_sim-success-rates.pdf}
                \includegraphics[width=0.85\columnwidth]{fig/push_real-success-rates.pdf}
            \end{center}
            }
            }
	}
	\caption{The final position error (a, c) and orientation error (b, d) of the push task in simulation (a, b) and on the real system (c, d).  The box plots show the median (black line) and interquartile range ($25^{th}$ and $75^{th}$ percentile); the lines extend to the most extreme data points not considered outliers, and outliers are plotted individually. The success percentages are shown below the method names.}
	\label{fig:results-push}
	%\vspace{-2em}
\end{figure}

WASP-80 	 & 	20.1141	 & 	49.71	 & 	XMM	 & 	0744940101	 & 	PN	 & 	-11.65	 & 	29.83	 & 	This work  \\
HAT-P-18	 & 	6.1863	 & 	161.6	 & 	XMM	 & 	Slew      	 & 	PN	 & 	$\leq-11$	 & 	$\leq31.49$	 & 	Undetected  \\
55 Cnc   	 & 	79.4482	 & 	12.59	 & 	XMM	 & 	0551020801	 & 	PN	 & 	-13.22	 & 	27.05	 & 	This work  \\
HD63433 	 & 	44.6848	 & 	22.38	 & 	XMM	 & 	0882870101	 & 	EPIC	 & 	-11.9	 & 	28.88	 & 	Zhang+2022, 0.124-2.48 keV band  \\
WASP-12 	 & 	2.4213	 & 	413	 & 	XMM	 & 	0853380101	 & 	M2	 & 	$\leq-13.91$	 & 	$\leq29.4$	 & 	Undetected  \\
WASP-76 	 & 	5.2899	 & 	189	 & 	XMM	 & 	0853380501	 & 	M1	 & 	$\leq-13.7$	 & 	$\leq28.93$	 & 	Undetected  \\
HAT-P-32	 & 	3.4938	 & 	286.2	 & 	XMM	 & 	0853381001	 & 	PN	 & 	-13.24	 & 	28.75	 & 	 This work  \\
Trappist-1	 & 	80.2123	 & 	12.47	 & 	XMM	 & 	ALL-XMM     	 & 	EPIC	 & 	-14.04	 & 	26.23	 & 	This work  \\
WASP-52 	 & 	5.7262	 & 	174.6	 & 	Chandra	 & 	     15728	 & 	ACIS	 & 	-13.11	 & 	29.45	 & 	This work$^(a)$  \\
WASP-177	 & 	5.8129	 & 	172	 & 	XMM	 & 	Slew	 & 	PN	 & 	$\leq-12.1$	 & 	$\leq30.45$	 & 	Undetected  \\
HAT-P-26	 & 	6.9995	 & 	142	 & 	XMM	 & 	0804790101	 & 	PN	 & 	$\leq-14.63$	 & 	$\leq27.76$	 & 	Undetected  \\
NGTS-5  	 & 	3.2114	 & 	311.4	 & 	XMM	 & 	Slew	 & 	PN	 & 	$\leq-12.36$	 & 	$\leq30.7$	 & 	Undetected  \\

% \begin{table}[t]
\caption{We compare our featuremetric refinement method using the proposed \textbf{NeFeS} network with photometric-based refinement baselines on \textit{Cambridge Hospital}.}
\label{supp:table:photometric-refinement}
\centering
% \resizebox{\linewidth}{!}{
\begin{tabular}{lc}
\toprule
Methods & Hospital \\
\midrule
DFNet                                                   & 2.00m/2.98$\degree$\\
DFNet + Sparse NeRF photometric$_{50}$                         & 1.19m/1.52$\degree$ \\
DFNet + Dense NeRF photometric$_{50}$                         & 0.80m/1.12$\degree$ \\
\textbf{DFNet + }$\textbf{NeFeS}_{\textbf{50}}$         & \boldred{0.55m}/\boldred{0.90$\degree$} \\
\bottomrule
\end{tabular}
% }
\end{table}

% 
\begin{table*}[h]
\tabcolsep7.5pt
\caption{Binding energy (BE) of carbon-chain species}
\label{tab:Ebind}
\begin{center}
\begin{tabular}{cc|cc|cc}
\hline
Species & BE (K) & Species& BE (K) & Species& BE (K) \\
\hline
%HC$_3$N &\\
%HC$_3$N &\\
C$_2$& 10000& C$_8$N&7200&C$_3$O&2750 \\
%C$_{2}$&10000$^{b}$\\
C$_3$&2500&C$_9$N&8000&C$_5$O&4350 \\
%C$_{3}$&2500$^{b}$&&\\
C$_{4}$&3200&C$_{10}$N&8800 & C$_7$O&5950  \\ 
C$_5$&4000&C$_2$H$_2$&2587 &C$_9$O&7550 \\
C$_6$&4800&C$_2$H$_4$&2500 & HC$_2$O&2400 \\
C$_7$&5600&C$_2$H$_5$&3100 &SiC$_2$&4300 \\
%C$_{5}$&4000$^{b}$&&\\
%C$_{6}$&4800$^{b}$&&\\
%C$_{7}$&5600$^{b}$&&\\
C$_{8}$&6400&C$_2$H$_6$&1600 &SiC$_3$&5100 \\
C$_{9}$&7200&C$_4$H$_2$&4187& SiC$_4$&5900 \\
C$_{10}$&8000&C$_5$H$_2$&4987 \\
C$_{11}$&9600&C$_6$H$_2$&5787\\
C$_2$H&3000&C$_7$H$_2$&6587\\
$l$-C$_3$H&4000&C$_2$P&4300 & \\
$c$-C$_3$H&5200&C$_3$P&5900\\
C$_4$H&3737&C$_4$P&7500\\
C$_5$H&4537&C$_2$S&2700\\
C$_6$H&5337&C$_3$S&3500\\
C$_7$H&6137&C$_4$S&4300\\
C$_8$H&6937&HC$_3$N&4580\\
$c$-$\rm{C_3H_2}$&5900&HC$_4$N&5380\\
C$_2$N&2400&HC$_5$N&6180\\
C$_3$N&3200&HC$_6$N&7780\\
C$_4$N&4000&HC$_7$N&7780\\
C$_5$N&4800&HC$_8$N&9380\\
C$_6$N&5600&HC$_9$N&9380\\
C$_7$N&6400&C$_2$O&1950\\

%HC$_3$O&3111$^{b}$

%C$_7$N&6400$^{a}$&H$_2$C$_3$N&3133$^{b}$\\
%C$_8$N&7200$^{b}$&\\
%C$_9$N&8000$^{b}$&\\
%C$_2$O&1950$^{b}$&&\\
%C$_3$O&4208$^{a}$&&\\
%C$_5$O&4350$^{b}$&&\\
%C$_7$O&5950$^{b}$&&\\
%C$_9$O&7550$^{b}$&&\\
% &&C$_2$S&2943$^{a}$\\
% &&C$_3$S&3500$^{b}$\\
% &&C$_4$S&4300$^{b}$\\
% &&HC$_3$N&3475$^{a}$\\
% &&HC$_4$N&5380$^{b}$\\
% &&HC$_5$N&6180$^{b}$\\
% &&HC$_6$N&7780$^{b}$\\
% &&HC$_7$N&7780(it should change)$^{b}$\\
% &&HC$_8$N&9380$^{b}$\\
% &&HC$_9$N&9380(it should change)$^{b}$\\
% &&HC$_2$O&2400$^{b}$\\
% &&HC$_3$O&3111$^{a}$\\
% &&SiC$_2$&4300$^{b}$\\
% &&SiC$_3$&5100$^{b}$\\
% &&SiC$_4$&5900$^{b}$\\
% %&&\\
\hline
\end{tabular}
\end{center}
%\begin{tabnote}
{Taken from the KIDA (\url{https://kida.astrochem-tools.org/}), and also see \citet{wake17}, \citet{pent17}, \citet{das18}.}
% higher-order-chain values are estimated with lower order chain plus one carbon atom's BE (KIDA has used 800 K for these estimations since updated binding energy of C atom is 1300 K, thus need to update those values. HC$_{2n+1}$ show different binding energy, however, we think the BE of HC$_4$N, HC$_6$N, and HC$_8$N needs to revisit. Especially for C$_n$P group, C$_n$+P is the rule for the estimation of binding energy. Also, include BE values of HC$_n$O ($n>4$) following a similar carbon addition method. Check some estimation if possible for recently detected higher-order carbon chains.
%\end{tabnote}
\end{table*}

\subsubsection*{Results and Discussion}
The results for the simulation are shown in Figure \ref{fig:results-push}a) and b).  We consider a policy feasible if the position error between the goal location of the object and the desired goal location is less than \SI{11}{\milli\meter} and the orientation error is less than \SI{30}{\deg}.
The high success percentage of 97\percent for the learned policies shows that it is generally possible to solve this task. Our proposed model solves 86\percent of the configuration and outperforms all baselines that do not require explicit learning. The gap to the \textit{direct} model, which achieved a success rate of 65\percent, is significant. The nearest neighbor and the single policy approach only achieved 52\percent and 38\percent, which shows not only the difficulty of the task but also excludes them as practical solutions.

Similar to the obstacle task, we also executed the learned policies on the real robot system. To account for the differences of such a contact-rich task to the simulation, we increase the allowed final position error by \SI{4}{\milli\meter} but keep the same angular maximum.

The results for the evaluation on the real system are in Fig.~\ref{fig:results-push}c) and d) as well as in  Table~\ref{tab:results}. As intuitively expected, the success percentages generally drop as not all policies transfer to the real system. Similar to the evaluation in simulation, the nearest neighbor baseline performs poorly. However, it is notable that our model now outperforms the explicitly learned policies in both the success rate and the final error. A possible explanation for this is that our model needed to generalize, whereas an explicitly learned policy is able to exploit the simulation to the maximum extent possible. During the experiments, we also observed that policies from our model generally kept a larger distance from the object when approaching it and also had fewer collisions with it.

To determine the time efficiency of our approach, we compute time required to compute BTMG parameters for 60 new task variations. This analysis compares learning BTMG parameters from scratch using the RL-pipeline and obtaining BTMG parameters using our approach.
Starting from scratch with the RL-pipeline, median completion times were 770.315 seconds for the obstacle task and 1232.625 seconds for the push task. In contrast, the optimization phase of our approach achieved median completion times of 1.27 seconds for the obstacle task and 5.189 seconds for the push task. Additionally, obtaining a trained PERF model took an average of 66.628 seconds for the obstacle task and 317.025 seconds for the push task.
During optimization, we observed some outliers, likely stemming from the stochastic nature of the process. The analysis was performed on a laptop equipped with an Intel(R) Core(TM) i7-10870H CPU running at 2.20GHz with 8 physical cores and hyper-threading, along with 64GB of RAM.
\section{Conclusion and Future Work}
Agile robotics requires that a system adapts quickly to changing conditions. In this work, we introduced an extension to BTMGs, a motion representation based on behavior trees and motion generators, which addresses this challenge. Our approach enables the use of learned policies in previously unseen variations of a task, allowing for fast adaption of robot behavior to changes in the task or environment. 

The experimental evaluation demonstrates that our approach effectively learns a model capable of adapting to new task variations. Our method exhibits comparable performance to explicitly trained policies and consistently outperforms all other baseline models. Furthermore, experiments conducted on the real robotic system demonstrate the successful transferability of our approach from simulation to reality, even in a contact-rich task. Notably, our proposed method can even outperform explicitly learned policies in the same contact-rich task, indicating superior generalization capabilities.

In future work, it is worth exploring whether the uncertainty modeled by the GP can be leveraged to make more accurate predictions about successful execution. This uncertainty measure could also be used for out-of-distribution detection.
Another promising direction is to use the learned model to return policy parameters for task parameters, such as friction, for which the values are not known a priori. 
In this case, we could jointly optimize over both policy and task parameters to identify a compatible set of learned parameters. %\textcolor{red}{This method holds particular promise in peg-in-hole tasks, where our approach could be leveraged to determine the appropriate friction value for a specific experimental setup. By finding the right balance between policy and task parameters, we can enhance the effectiveness and adaptability of our approach in practical applications.}

\bibliography{2023-IROS}
\bibliographystyle{bib/IEEEtran}


\end{document}