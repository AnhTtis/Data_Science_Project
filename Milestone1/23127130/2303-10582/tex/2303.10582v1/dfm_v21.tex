%\documentclass[aps,prl,preprint,groupedaddress]{revtex4}
%\documentclass[aps,prl,reprint,superscriptaddress]{revtex4-2}
\documentclass[aps,prl,reprint,groupedaddress]{revtex4}

\usepackage{bbm}
\usepackage{graphicx}% Include figure files
\usepackage{subfigure}
\usepackage{amsmath}
\begin{document}

\title{Nonthermal entanglement dynamics in a dipole-facilitated glassy model with disconnected subspaces}

\author{Guanhua Chen$^{1}$ and Yao Yao$^{1,2}$\footnote{Electronic address:~\url{yaoyao2016@scut.edu.cn}}}

\address{$^1$ Department of Physics, South China University of Technology, Guangzhou 510640, China\\
$^2$ State Key Laboratory of Luminescent Materials and Devices, South China University of Technology, Guangzhou 510640, China}

\date{\today}

\begin{abstract}
We study a quasi-one-dimensional spin-1/2 chain with frustrated triangular form, in which the interaction is set to realize a glassy system with dipole-facilitated kinetic constraints. A notable feature of this concise model turns out to be the three disconnected subspaces, allowing us to comprehend entanglement dynamics with blocked structure of Hilbert space. Initially from the Bell state and the Greenberger-Horne-Zeilinger state, it is found that the quantum entanglement, quantified by concurrence, fidelity and 2-R\'{e}nyi entropy, exhibits a nonthermal dynamic behavior, i.e. the diffused entanglement can be spontaneously recovered which is absent in other spin models. This appealing periodic dynamics of entanglement manifests high fidelity against random flipping noise. Our work offers a sound way of ergodicity breaking and fault-tolerant quantum computations.
\end{abstract}

\maketitle

In a dynamic scenario, glassy system stems from certain kinetic constraints for the motion of individual objects \cite{1,2}. For example, in a triangular antiferromagnetic lattice, the dynamics of a local spin is frustrated due to the constraint from neighboring spins \cite{3,4}. In the spin glass phase, therefore, no long-range orders are present and the featured timescale of thermalization becomes ultraslow. There is then interesting local dynamics which can be described by kinetically constrained models (KCMs), such as the Fredrickson-Andersen (FA) model and the East model \cite{5}. Based on the disorder-free quantum East model, it has been proven that a large number of nonthermal states can be constructed to manifest area-law entanglement entropy \cite{6,7}. This greatly inspired us to apply the glassy system to the quantum information processing.

Following the rapid developing progress of numerical methods and experimental techniques, it has got promising prototype to realize ergodicity breaking, that is, the quantum thermalization, or eigenstate thermalization hypothesis (ETH) can be fairly violated \cite{8,9}. For example, the quantum many-body scars stemming from Rydberg atom simulators emerge as typical source of weak ergodicity breaking \cite{10,11}. The Hilbert space fragmentation, describing the Hilbert space splits into exponentially many dynamically disconnected blocks, is also a remarkable theory \cite{12,13}. While studying these intriguing nonthermal dynamics, quantum entanglement always serves as the preferred measure \cite{14}.

A benchmark multipartite entangled state, the Greenberger-Horne-Zeilinger (GHZ) state as a two-component cat state has been manifested to play great roles  in a number of applications such as quantum communication \cite{15,16}, quantum metrology \cite{17} and quantum codes \cite{18}. In recent years, plenty of theoretical researches and practical experiments have been carried out for this state using Rydberg atoms \cite{19,20,21}, ultrafast pulsed laser \cite{22,23}, and superconducting circuits \cite{24,25,26}. However, preparing and further controlling highly entangled states in larger spin systems still seriously challenge \cite{27,28}. In this Letter, we then introduce a toy model with dipole-facilitated kinetics, which manifests unusual ergodicity breaking by disconnected subspaces. On the basis of the Bell state and the GHZ state, we discuss dynamics of both bipartite and multipartite entanglement with this particular structure of Hilbert space.

\begin{figure}[htbp]
	\includegraphics[scale=1]{fig/fig1.eps}
	\caption{\label{fig1} (a) Schematic of a 8-site GHZ state with the circle representing that site is in spin-down state and the ball being spin-up. Red and blue zone denote the relevant sites are changeable. (b) A thermal subspace (T) and two nontrivial subspaces (L and R) are figured out in disconnected block representation. The upper and lower halves in each square (solid: $\uparrow\uparrow$, white: $\downarrow\downarrow$, gradient: $\uparrow\downarrow$ or $\downarrow\uparrow$) represent even and odd sites, respectively.  }
\end{figure}

We start from cutting a triangular spin lattice into a quasi-one-dimensional chain as sketched in Fig.~\ref{fig1}(a) \cite{29,30}. Odd and even sites are separated into lower and upper sides labeled by different colors. We propose a kinetic constraint for realizing frustration which is described as follows. First, two neighboring spins favor antiparallel configuration due to antiferromagnetic interaction. Second, a third spin in the triangle formed with these two spins is allowed to flip only if its neighboring spin is in down state. Of course we can parallel consider the constraint of up state but that is surely equivalent. The basic idea is, we have to constrain the spin flip by some specific neighboring state to produce nontrivial dynamic effects. This is very similar with the idea of CNOT gate in quantum computation but the control circuit is instead composed of two spins or a dipole. Subsequently, this newly-proposed KCM can be named as dipole-facilitated model (DFM). The model Hamiltonian is written as
\begin{equation}\label{eq1}
	\begin{aligned}
		H_{\rm df}=\sum_{i=1}^{L} (Q_i P_{i+1}X_{i+2}+X_i P_{i+1}Q_{i+2}),
	\end{aligned}
\end{equation}
where $X_i$ is the x-Pauli operator on $i$-th site and $Q_i=|\uparrow\rangle \langle \uparrow|_i$, $P_i=|\downarrow\rangle \langle \downarrow|_i$ are projectors of spin-up and spin-down, respectively. Throughout this work, the periodic boundary condition (PBC) is adopted for this model.

On the potential experimental realization of the model, we first notice that the Rydberg atoms, an ideal experimental platform for KCMs, have been used to study nonequilibrium and slow dynamics \cite{10,31,32}. The excited state and ground state of atom can be mapped as spin up and down. Under some controllable experimental conditions, the nearest-neighbour Rydberg blockade effect can be well described by the so-called PXP model \cite{11,33}, which has got very similar constraint with the present DFM. In this context, we think that the DFM should not be difficult to be realized in a Rydberg lattice.

Let us first discuss the Hilbert space of DFM in block representation \cite{34}. We consider four sites as instance. Two successive sites are grouped into one ``group-site" and denote ($\downarrow\downarrow$), ($\uparrow\downarrow$), ($\downarrow$$\uparrow$) and ($\uparrow\uparrow$) as $\circ$, $\triangleright$, $\triangleleft$ and $\bullet$, respectively. Under the action of $H_{\rm df}$, we notice that these configurations ($\circ\circ$), ($\triangleleft\triangleright$), ($\triangleleft\bullet$), ($\bullet\triangleright$) and ($\bullet\bullet$) are annihilated and the rest can be categorized into three subspaces. The transformation rules are
\begin{equation}\label{eq2}
	\begin{aligned}
		&T:\circ\bullet\longleftrightarrow \triangleleft\bullet \longleftrightarrow \triangleleft\triangleright \longleftrightarrow \bullet\triangleright \longleftrightarrow \bullet\circ, \\
        L:\circ\triangleright&\longleftrightarrow \triangleright\triangleright\longleftrightarrow \triangleright\circ, \quad R:\circ\triangleleft\longleftrightarrow \triangleleft\triangleleft\longleftrightarrow\triangleleft\circ. \\
	\end{aligned}
\end{equation}
These three reaction paths can be straightforwardly extended to any longer chains and subsequently construct a trivial thermal subspace labeled by T and two nontrivial subspaces labeled by L (all even sites are spin-down) and R (all odd sites are spin-down), as sketched in Fig.\ref{fig1}(b).

Most interestingly, these three disconnected subspaces totally decide the active spatial range of systems. For example, if the initial state of a system without perturbation is $|\uparrow\downarrow\dots\uparrow\downarrow\rangle$, even sites will persistently stay in spin-down state, implying that the time evolution of the system is confined to the subspace L. The reverse holds true as well, namely states initiated from $|\downarrow\uparrow\dots\downarrow\uparrow\rangle$ stays in the R subspace. Now, if we have an initial state $|0\rangle=|\downarrow\downarrow\dots\downarrow\downarrow\rangle$, it will evolve into completely different subspace and stay there depending on how we add spin-up into the lattice. Adding a single spin-up to even (odd) site activates R (L) subspace, respectively, and two successive spin-up's break the confinement of above two nontrivial subspaces leading to the T subspace.

One may ask if these disconnected subspaces possess similar feature with that in the Hilbert space fragmentation. Breaking the Hilbert space into disconnected sectors is the nontrivial feature of generic KCM, but in most cases the number of Krylov subspaces is exponentially dependent of the system size \cite{32}. In our DFM the number of subspaces is fixed no matter how large the system is, which is essential to produce robust periodic dynamics of entanglement, so it should be regarded as different with the fragmentation.

\begin{figure*}[htbp]
	\includegraphics[scale=1]{fig/fig2.eps}
	\caption{\label{fig2} Time evolution of $\langle Z_i \rangle$ on a 24-site lattice. (a) The initial state is $|\downarrow\uparrow\dots\downarrow\uparrow\rangle$. The odd sites keep in spin-down and even sites behave periodic oscillation. (b) The initial state is $|\downarrow\dots\downarrow\uparrow\downarrow\dots\downarrow\rangle$, i.e. a single spin-up is located at $i=12$, which diffuses to the ends of the chain but odd sites keep in spin-down persistently. (c) The initial state is $|\downarrow\dots\downarrow\uparrow\uparrow\downarrow\dots\downarrow\rangle$, i.e. the $i=12$ and $13$ sites are in spin-up. Different from the first two cases, spin-up states will cover all sites after sufficiently long time.}
\end{figure*}

In the following, we use time-evolving block decimation (TEBD) to compute the dynamics on the chain \cite{35,36}. We first show the time evolution of expectation value of z-Pauli operator $\langle Z_i\rangle$ with different initial states for $L=24$. Fig.~\ref{fig2}(a) and (b) show the results from the initial states $|\downarrow\uparrow\dots\downarrow\uparrow\rangle$ and $|\downarrow\dots\downarrow\uparrow\downarrow\dots\downarrow\rangle$, respectively. In two cases, the spins are continuously flipped on the even sites, but all odd sites keep spin-down without any flipping. That is, the system stays in the R subspace. We can also parallel consider the L subspace which is not shown. For a comparison, we calculate the initial state with two successive spin-up sites in the middle of the chain, namely $|\downarrow\dots\downarrow\uparrow\uparrow\downarrow\dots\downarrow\rangle$ as shown in Fig.\ref{fig2}(c). This will transform the subspace L and R to T and the system becomes ergodic, as all sites are being flipped in the time evolution.

It is thus intuitive to relate this phenomenon to quantum computations. Let us now categorize all sites into two. The sites always being spin-down are called idle sites and the others as work sites. As long as the system is sufficiently large, the probability that all work sites lose their memory and flip to spin-down is extremely low. Therefore, we can denote all idle sites to be 0, all work sites together as 1, to form a large and robust logical qubit. In practice, once we detect some spin-up, no matter how many, we can say it is 1 (as in Fig.\ref{fig2}(b)), otherwise it is 0. For one-dimensional superconducting quantum circuit, quantum non-demolition parity measurements can be realized by assigning some qubits as measurement or control nodes, which detect errors of adjacent data qubits \cite{37}. Analogously, herein, staggered work sites of DFM have also got similar auxiliary effects. That is, if we introduce a spin-up at a left end work site ($i=1$ or $i=2$), the right end work site will tell us the initial odevity, and errors caused by noise can be detected by the appearance of successive spin-up's. A single logical qubit can thus be in a superposition state of bases in the L and R subspace, labeled by $|L\rangle$ or $|R\rangle$ respectively, which is encoded with the overall system and unexpected flip noise on work sites do not interfere the result. This one-order code significantly reduces half flip error, which paves a novel way for fault tolerance.

\begin{figure*}[htbp]
	\includegraphics[scale=1.1]{fig/fig3.eps}
\caption{\label{fig3} Concurrence between middle two sites (i.e. $i=4$ and 5) in 8-site systems. All initial states are set as the Bell state. (a) Time evolution with DFM shows recoverable periodic oscillations. (b) Adding a random flip noise with $T_X=1$, the bipartite entanglement gradually decreases during oscillation. This calculation result is averaged over 1000 samples for random $X$ operators. Results of time evolution without noise under (c) The East model at Rokhsar-Kivelson point, (d) the PXP model from Rydberg blockade and (e) the antiferromagnetic Heisenberg model do not show periodic behavior. The East model is in open boundary condition and others are in PBC.}
\end{figure*}

Intuitively, above results provide a promising direction of studying various entangled states. By a two-qubit gate, it is easy to obtain a local Bell state in many-body system ($L=8$), i.e.,
\begin{equation}\label{eq3}
|\rm Bell\rangle=\frac{1}{\sqrt{2}}(|\downarrow\downarrow\downarrow\downarrow\uparrow\downarrow\downarrow\downarrow\rangle+|\downarrow\downarrow\downarrow\uparrow\downarrow\downarrow\downarrow\downarrow\rangle),
\end{equation}
which will be evolving into $|L\rangle+|R\rangle$. Then, we focus on the entanglement between middle two sites to see the influence of the disconnected subspaces. To this end, we calculate the concurrence to quantize the bipartite entanglement between them \cite{38,39}, which is defined as
\begin{equation}\label{eq4}
	C(\rho)=\rm max\{\lambda_1-\lambda_2-\lambda_3-\lambda_4,0\},
\end{equation}
where $\lambda_k (k=1,2,3,4)$ are the eigenvalues in descending order of the Hermitian matrix $r=\sqrt{\sqrt{\rho}\tilde{\rho}\sqrt{\rho}}$ and $\tilde{\rho}=(\sigma_{\rm y} \otimes \sigma_{\rm y})\rho^*(\sigma_{\rm y} \otimes \sigma_{\rm y})$. $\sigma_{\rm y}$ and $\rho$ are y-Pauli matrix and reduced density matrix of the two middle sites by partially tracing others \cite{40}.

As displayed in Fig.~\ref{fig3}(a), we observe a surprising periodicity of the entanglement, especially its maximum, which is very like a Newton's cradle. This novel nonthermal dynamic behavior stems from that two superposed bases of $|\rm Bell\rangle$ are restricted in two disconnected nontrivial subspaces, and the time evolution in these two subspaces are completely symmetric allowing the entanglement to be spontaneously recovered after diffusion. To be comparisons, three other typical models of spin chain are also calculated, namely the East model, the PXP model and the antiferromagnetic Heisenberg model as shown in Fig.~\ref{fig3}(c), (d) and (e). There are no conserved quantities in the PXP model which was first introduced in the quantum many-body scars and also in the classical East model for describing spin glasses \cite{7}. The Heisenberg model merely conserves the total spin. Remarkably, concurrences in these three models do not perform any visible cradle-like periodicity but just irregular oscillations.

We then mimic the inevitable noise which can be described as bit flip (off-diagonal) noise and the projection (diagonal) noise, respectively. In the first case if an idle site is flipped from spin-down to spin-up by off-diagonal noise, one of subspaces L or R will be transferred into T. As a result, the entangled state is destroyed irreversibly. To be more explicit, the state changes as
\begin{equation}\label{eq5}
	\begin{aligned}	
		(|L\rangle+|R\rangle)\stackrel{\rm noise}{\longrightarrow}&(|T\rangle+|R\rangle) \ \rm or \ (|L\rangle+|T\rangle),
	\end{aligned}
\end{equation}
which will finally be thermalized in the T subspace and the transformation of subspaces will directly influence the bipartite entanglement. We now add a stochastic projector into the Hamiltonian as a noisy source, following the form:
\begin{equation}\label{eq6}
	H_{\rm X}(t)=H_{\rm df}+\sum_{n} X_j \delta(t-nT_{\rm X})
\end{equation}
where $j$ is a random site to be disturbed and the noisy term occurs every $t=nT_{\rm X}$. In Fig.~\ref{fig3}(b), we observe that the concurrence decays gradually under the noise $T_{\rm X}=1$ and will not recover, manifesting the irreversible breaking of periodicity. For diagonal noise, we can also set a noisy Hamiltonian similar to Eq.~\ref{eq6} with projectors $Q_i$ or $P_i$. Compared with the off-diagonal noise which transforms subspaces L and R into T, these diagonal projectors do not hybridize the subspaces unless all work sites are simultaneously set to be 0 which is almost impossible, so their influence is much easier to be eliminated and is not considered here.

\begin{figure}[htbp]
	\includegraphics[scale=1]{fig/fig4.eps}
	\caption{\label{fig4} Fidelity and reduced 2-R\'{e}nyi entropy evolving with $H_{\rm X}$. (a) The fidelity $F=|\langle{\rm GHZ}| \phi(t)\rangle| $ with $T_{\rm X}=0.5$ (red), $T_{\rm X}=1$ (blue), $T_{\rm X}=2$ (orange), $T_{\rm X}=4$ (green), $T_{\rm X}=8$ (cyan) and $T_{\rm X}=\infty$ (black) is obtained after quench from 8-site $|\rm GHZ\rangle$. The system behaves high-frequency oscillations and the envelope lines of higher bound are bolded. The $T_{\rm X}=\infty$ case is equal to evolving under $H_{\rm df}$, which shows periodicity without decay. (b) The reduced 2-R\'{e}nyi entropy $S(\rho)$ between odd and even sites for different $T_{\rm X}$. At $T_{\rm X}=\infty$, the entropy is constant in time. Above calculation results are all averaged over 500 runs for $X$ operators on random sites.}
\end{figure}

In order to more comprehend the flip noise due to the special subspace structure, we further consider the 8-site GHZ state
\begin{equation}\label{eq7}
|\rm GHZ\rangle=\frac{1}{\sqrt{2}}(|\uparrow\downarrow\uparrow\downarrow\uparrow\downarrow\uparrow\downarrow\rangle+|\downarrow\uparrow\downarrow\uparrow\downarrow\uparrow\downarrow\uparrow\rangle),
\end{equation}
as initial state which also belongs to $|L\rangle+|R\rangle$ and has similar properties with the Bell state. We calculate the fidelity $F=|\langle{\rm GHZ}| \phi(t)\rangle|$ between evolving and GHZ state to characterize the influence of wrong actions under noise, which actually quantifies the fluctuation of initial state. As shown in Fig.~\ref{fig4}(a), for the system without random projectors, fidelity is able to come back to $1$ periodically, implying perfect many-body revivals. By adding a noisy term, the fidelity is gradually decreasing over time, i.e. the distance between evolving state and $|{\rm GHZ}\rangle$ is increasing. It can be rationally predicted that, the decay rate of fidelity is inversely proportional to the random flip period $T_{\rm X}$. If setting the time axis in the unit of nanosecond as for usual case, it is estimated that even in a very noisy environment with frequency being about $125$ MHz, the fidelity of $|\rm GHZ\rangle$ can keep above $0.8$ after $400$ ns, suggesting a fairly good robustness for a long-time evolution \cite{42}.

We can also calculate the entanglement entropy, but the measure of bipartite entanglement can not be directly generalized to multipartite entanglement. We thus have to employ other measures of entanglement between idle sites and work sites. We notice that, the state $|L\rangle+|R\rangle$ can be described as
\begin{equation}\label{eq8}
	|0\rangle_{\rm o} |1\rangle_{\rm e} +|1\rangle_{\rm o} |0\rangle_{\rm e},
\end{equation}
where o (e) denote all odd (even) sites. This form of logical qubits is analogous to a two-site Bell state, holding maximum entanglement. Dividing the system into odd and even groups, therefore, we can calculate the reduced 2-R\'{e}nyi entropy between them with a normalized base of logarithm as
\begin{equation}\label{eq9}
	S(\rho)=-\log_{16} \rm Tr(\rho^2),
\end{equation}
which also characterizes the localization of wave function \cite{41}. As depicted in Fig.~\ref{fig4}(b), without noise the entanglement between odd and even sites remains a fixed value $S_{\rm min}=\log_{16}2=0.25$, corresponding to two nonzero diagonal elements in the reduced density matrix. This is because the two disconnected nontrivial subspaces impede the delocalization and entropy growth. In other cases, the R\'{e}nyi entropy keeps growing before saturated, since due to the X noise, the localization and barriers of different subspaces are broken down, and more states evolve into thermal and delocalized T subspace. For $T_{\rm X}=0.5$, the maximum entanglement appears around $t=200$, implying thermalization takes place.

Before ending, we make a simple analogy between our model and the disorder-free localization model which is composed of spinless fermions and spin-1/2 on the bond \cite{43}. As a fermionic matter field, the fermion tunneling is related to coupled spins. We take one triangle on the lattice as instance. It is easy to derive that
\begin{equation}\label{eq10}
	\begin{aligned}
		H_{\rm df}^2&=[Q_1P_2(\sigma^+_3+\sigma^-_3)+(\sigma^+_1+\sigma^-_1)P_2Q_3]^2 \\
		&=Q_1P_2+P_2 Q_3+\sigma^-_1 P_2\sigma^+_3+\sigma^+_1 P_2\sigma^-_3,\end{aligned}
\end{equation}
where $\sigma^+_i=|\uparrow\rangle\langle\downarrow|$ and $\sigma^-_i=|\downarrow\rangle\langle\uparrow|$ are spin upper and lower operators. One can see that, the last two terms are nothing but the constrained hopping term in the disorder-free localization model, namely the hopping is determined by spin on bond. As a result, our model $H_{\rm df}$ just decomposes the localization into two subspaces and thus enables the nontrivial fault-tolerant processes.

In summary, originated from triangular frustrated kinetic constraints, the DFM manifests exotic disconnected subspace structure. We focused on entangled states from these two nontrivial subspaces and show how to quantify and affect the entanglement. This toy model exhibit appealing cradle-like nonthermal dynamic behavior for both bipartite and multipartite entanglement. During the past several years, Rydberg atoms have provided promising experimental platform to be quantum simulators \cite{26,10,44}. For instance, under some special parameters, we can get the PXP model in 3-site interaction \cite{11}, facilitated spin clusters \cite{32}, etc. These experimental progresses enhance the feasibility and meaning of DFM.

\section*{Acknowledgment}

The authors gratefully acknowledge support from the Special Project for Research and Development in Key Areas of
Guangdong Province (Grant No.~2020B0303300001), and National Natural Science Foundation of China (Grant Nos.~11974118).

\bibliography{dfm_v21}
\end{document}

