% CVPR 2023 Paper Template
% based on the CVPR template provided by Ming-Ming Cheng (https://github.com/MCG-NKU/CVPR_Template)
% modified and extended by Stefan Roth (stefan.roth@NOSPAMtu-darmstadt.de)

\documentclass[10pt,twocolumn,letterpaper]{article}

%%%%%%%%% PAPER TYPE  - PLEASE UPDATE FOR FINAL VERSION
%\usepackage[review]{cvpr}      % To produce the REVIEW version
%\usepackage{cvpr}              % To produce the CAMERA-READY version
\usepackage[pagenumbers]{cvpr} % To force page numbers, e.g. for an arXiv version
\usepackage[accsupp]{axessibility}  % Improves PDF readability for those with disabilities.

\makeatletter
\@namedef{ver@everyshi.sty}{}
\makeatother
\usepackage{tikz}
% Include other packages here, before hyperref.
\usepackage{graphicx}
\usepackage{amsmath}
\usepackage{amssymb}
\usepackage{booktabs}

\usepackage{tikz}
\usepackage{comment}
\usepackage{color}
\usepackage{svg}
\usepackage{subcaption}
\usepackage{algorithm}
\usepackage{algpseudocode}

% It is strongly recommended to use hyperref, especially for the review version.
% hyperref with option pagebackref eases the reviewers' job.
% Please disable hyperref *only* if you encounter grave issues, e.g. with the
% file validation for the camera-ready version.
%
% If you comment hyperref and then uncomment it, you should delete
% ReviewTempalte.aux before re-running LaTeX.
% (Or just hit 'q' on the first LaTeX run, let it finish, and you
%  should be clear).
\usepackage[pagebackref,breaklinks,colorlinks]{hyperref}

% Support for easy cross-referencing
\usepackage[capitalize]{cleveref}
\crefname{section}{Sec.}{Secs.}
\Crefname{section}{Section}{Sections}
\Crefname{table}{Table}{Tables}
\crefname{table}{Tab.}{Tabs.}


%%%%%%%%% PAPER ID  - PLEASE UPDATE
\def\cvprPaperID{7583} % *** Enter the CVPR Paper ID here
\def\confName{CVPR}
\def\confYear{2023}

\graphicspath{{Figures/}}
%%%%% GENERAL MATH COMMANDS
% Reals
\newcommand{\R}{{\mathbb R}}
% Integers
\newcommand{\Z}{{\mathbb Z}}
% Naturals
\newcommand{\N}{{\mathbb N}}
% Expectation
\DeclareMathOperator*{\E}{\mathbb{E}}
% ^th notation
\newcommand{\tth}{^{\text{th}}}
% Small dots for integer range [a .. b]
\newcommand{\sdots}{\,..\,}
% Vectorized version of matrix
\newcommand{\matvec}{\mbox{vec}}

% := sign
\newcommand{\defeq}{\vcentcolon=}
% Zero function
\newcommand{\zf}{\mathbf{0}}
% Vector of ones
\newcommand{\ones}{\mathbf{1}}

% Argmin and argmax definitions
\DeclareMathOperator*{\argmax}{arg\,max}
\DeclareMathOperator*{\argmin}{arg\,min}


%%%%% PROBLEM STATEMENT NOTATION 
% \newcommandtwoopt{\St}[2][t][]{{S_{#1}^{#2}}} % State
\newcommand{\task}[1][i]{{\mathcal{T}_{#1}}} % Task, optionally takes index
\newcommand{\tasks}{\{ \task \}_{i=1}^N}
\newcommand{\losst}[1][i]{{l_{#1}}}
\newcommand{\lossv}[1][i]{{l_{#1}^{\textrm{val}}}}
\newcommand{\tasktarget}{{\mathcal{T}_{\textrm{target}}}}
\newcommand{\lossttarget}{l_{\textrm{target}}}
\newcommand{\lossvtarget}{l_{\textrm{target}}^{\textrm{val}}}
\newcommand{\lossttargetit}{l_{\textrm{target}}^{(k)}}
\newcommand{\losstotal}{l^{\textrm{total}}}
\newcommand{\lossopt}{l^*}

\newcommand{\thetait}[2]{\theta_{#1}^{(#2)}}
\newcommand{\phit}[1]{\phi^{(#1)}}
\newcommand{\hist}[2]{S_{#1}^{(#2)}}
\newcommand{\grad}[2]{G_{#1}^{(#2)}}

\newcommand{\Alg}{\textup{\textbf{Opt}}}
\newcommand{\MetaAlg}{\textup{\textbf{MetaOpt}}}

%%%%% Theorems
\newtheoremstyle{mytheoremstyle} % name
    {\topsep}                    % Space above
    {\topsep}                    % Space below
    {\itshape}                   % Body font
    {}                           % Indent amount
    {\scshape}                   % Theorem head font
    {.}                          % Punctuation after theorem head
    {.5em}                       % Space after theorem head
    {}  % Theorem head spec (can be left empty, meaning ‘normal’)
\theoremstyle{mytheoremstyle}
\theoremstyle{plain}
\newtheorem{theorem}{Theorem}
\newtheorem{proposition}{Proposition}
\newtheorem{assumption}{Assumption}
\newtheorem{definition}{Definition}
\newtheorem{lemma}{Lemma}
\theoremstyle{remark}
\newtheorem{remark}{Remark}


\begin{document}

%%%%%%%%% TITLE - PLEASE UPDATE
\title{Full or Weak annotations? \\An adaptive strategy for budget-constrained annotation campaigns}

% \author{First Author\\
% Institution1\\
% Institution1 address\\
% {\tt\small firstauthor@i1.org}
% % For a paper whose authors are all at the same institution,
% % omit the following lines up until the closing ``}''.
% % Additional authors and addresses can be added with ``\and'',
% % just like the second author.
% % To save space, use either the email address or home page, not % both
% \and
% Second Author\\
% Institution2\\
% First line of institution2 address\\
% {\tt\small secondauthor@i2.org}
% }

\author{Javier Gamazo Tejero$^1$, Martin S. Zinkernagel$^2$, Sebastian Wolf$^2$\\
Raphael Sznitman$^1$, Pablo Márquez Neila$^1$\\
{\small     $^1$University of Bern, $^2$Inselspital Bern, Switzerland}\\
{\tt\small \{javier.gamazo-tejero, raphael.sznitman, pablo.marquez\}@unibe.ch}\\
{\tt\small \{martin.zinkernagel, sebastian.wolf\}@insel.ch}
}


\maketitle

%%%%%%%%% ABSTRACT
\begin{abstract}
Annotating new datasets for machine learning tasks is tedious, time-consuming, and costly. For segmentation applications, the burden is particularly high as manual delineations of relevant image content are often extremely expensive or can only be done by experts with domain-specific knowledge. Thanks to developments in transfer learning and training with weak supervision, segmentation models can now also greatly benefit from annotations of different kinds. However, for any new domain application looking to use weak supervision, the dataset builder still needs to define a strategy to distribute full segmentation and other weak annotations. Doing so is challenging, however, as it is a priori unknown how to distribute an annotation budget for a given new dataset. To this end, we propose a novel approach to determine annotation strategies for segmentation datasets, whereby estimating what proportion of segmentation and classification annotations should be collected given a fixed budget. To do so, our method sequentially determines proportions of segmentation and classification annotations to collect for budget-fractions by modeling the expected improvement of the final segmentation model. We show in our experiments that our approach yields annotations that perform very close to the optimal for a number of different annotation budgets and datasets.
% that allow high-performing segmentation performances

\end{abstract}

%%%%%%%%% BODY TEXT

\section{Introduction}

The ability to reason about plans is critical for performing long-horizon tasks \citep{erol1996hierarchical, sohn2018hierarchical, sharma-etal-2022-skill}, compositional generalization \citep{corona-etal-2021-modular} and generalization to unseen tasks and environments \citep{shridhar2020alfred}.
Consider a simple long-horizon planning scenario where a robot is tasked with preparing a meal and serving it on the table. 
This presents a non-trivial planning problem since the agent needs to understand the sequence of operations required to perform the task and search for the relevant objects in the unfamiliar environment by interacting with various objects. %



Large language models have been recently shown to possess commonsense knowledge about the world such as object affordances and physical dynamics \citep{ouyang2022training,chowdhery2022palm}.
Early approaches considered text based environments and fine-tuned PLMs to predict actions given the history of past observations and actions \citep{jansen-2020-visually,micheli-fleuret-2021-language,yao-etal-2020-keep}.
Recent work has used this ability to reason about plans from text instructions in simulated household environments with simplifying assumptions such as text-only environment observations or feedback \citep{huang2022language,ahn2022can,li2022pre,logeswaran-etal-2022-shot}.


We focus on \emph{visually grounded planning} with PLMs --- the ability to adapt plans based on interaction and visual feedback from the environment.
While PLMs have strong planning commonsense priors, predictions from a PLM may not be directly realizable in the environment since the observation and action spaces are unknown.
This requires \emph{grounding} the PLM in the environment and adapting it to observe visual feedback, which is highly non-trivial.
Some prior works assume the availability of a pre-trained affordance function \citep{ahn2022can} or a success detector \citep{mirchandani2021ella}.
Notably, SayCan \citep{ahn2022can} completely decouples the PLM from observation information by selecting actions that have both high affordability (through a pre-trained affordance model) and high PLM likelihood.
Although this partially addresses the grounding problem, the use of visual feedback for action affordance alone is limited.
Often an agent must choose one of many affordable actions using information from observations.
For example, a driving agent should re-navigate and possibly turn around when encountering a ``road closed'' sign, but both turning around and driving forward are indistinguishable to SayCan because they are both affordable and the PLM is blind to observations.

Another workaround explored in prior work is translating the information in the visual observations to text using a pre-trained captioning system \citep{shridhar2021alfworld,huang2022language}.
However, it can be difficult to faithfully describe an image in words and information is lost in this inherently noisy process, which limits the information available to the planner.



Recent work shows that PLMs can be adapted for various natural language tasks by inserting tunable embeddings or soft prompts at the input of the PLM (also called prompt tuning or prefix tuning)~\citep{li-liang-2021-prefix,lester-etal-2021-power}.
This approach also extends to multi-modal understanding tasks such as image captioning \citep{mokady2021clipcap} and VQA \citep{tsimpoukelli2021multimodal} where images are encoded as soft prompts and finetuned for the target task.
Transformer based architectures have also been successfully applied to offline Reinforcement Learning in recent work \citep{chen2021decision,janner2021offline,li2022pre,reid2022can}.

Taking inspiration from these works, we propose the simple approach of embedding visual observations (`visual prompts') and \textit{directly inserting them as PLM input embeddings}.
The visual encoder and PLM are jointly trained for the target task, an approach we call \textbf{\oursfull}~(\ours).
By teaching the PLM to use observations for planning in an end to end manner, we remove the dependency on external data such as captions and affordability information that was used in prior work.
We show that this simple approach performs better than prior PLM-based planning approaches on two embodied planning benchmarks based on ALFWorld~\citep{shridhar2021alfworld} and Virtualhome~\cite{puig2018virtualhome}.



  While  submodular optimization problems are generally NP-hard, the celebrated greedy algorithm \cite{nemhauser1978analysis} attains a $(1-1/e)$ approximation ratio for  submodular maximization subject to uniform matroids and a $1/2$ approximation ratio for general matroid constraints. As discussed in the introduction, the  continuous greedy algorithm \cite{calinescu2011maximizing} restores the $(1-1/e)$ approximation ratio by lifting the discrete problem to the continuous domain via the multilinear relaxation. %It is worth to mention here that the multilinear relaxation is a DR-submodular function, a.k.a. a continuous function with the diminishing returns property.

Stochastic submodular maximization, in which the objective is expressed as an expectation, has gained a lot of interest in the recent years \cite{asadpour2008stochastic, zhang2022stochastic, chen2018online}. Karimi et al. \cite{karimi2017stochastic} use a concave relaxation method that achieves the $(1-1/e)$ approximation guarantee, but only  for the class of submodular coverage functions. Hassani et al.~\cite{hassani2017gradient} provide projected gradients methods for the general case of stochastic submodular problems that achieve $1/2$ approximation guarantee.  Mokhtari et al. \cite{mokhtari2020stochastic} propose stochastic  conditional gradient methods for solving both minimization and maximization  stochastic submodular optimization problems. Their method for maximization, Stochastic Continous Greedy (SCG) can be interpreted as a stochastic variant of the continuous greedy algorithm \cite{vondrak2008optimal, calinescu2011maximizing} and achieves a tight $(1-1/e)$ approximation guarantee for monotone and submodular functions. %However, all these methods suffer from two sources of randomness (one comes from sampling the objective function and the other comes from estimating the multilinear relaxation via sampling its inputs).

Our work builds upon and relies on the approach by  \"{O}zcan et al.~\cite{ozcan2021submodular}, who studied ways of accelerating the computation of gradients via a polynomial estimator. Extending on the work of Mahdian et al.~\cite{mahdian2020kelly},  \"{O}zcan et al. show that submodular functions that can be written as compositions of (a) an analytic function and (b) a multilinear function can be arbitrarily well approximated via Taylor polynomials; in turn, this gives rise to a method for approximating their multilinear relaxation in a closed form, without sampling. We leverage this method in the context of stochastic submodular optimization, showing that it can also be applied in combination with SCG of Mokhtari et al.~\cite{mokhtari2020stochastic}: this eliminates one of the two sources of randomness, thereby reducing variance at the expense of added bias. From a technical standpoint, this requires controlling the error introduced by the bias of the polynomial estimator, while simultaneously accounting for the variance inherent in SCG, due to sampling instances.   %: this eliminates the latter source of randomness by utilizing the properties of deep submodular models that result from composition over multiple layers. In order to do so, we combine the stochastic continuous greedy algorithm proposed by Mokthari et al. \cite{mokhtari2020stochastic} with the deterministic estimator proposed by
\begin{figure*}[t!]
\includegraphics[width=1.0\linewidth, trim={0 0.3cm 0 0.1cm}, clip]{figures/architecture/architecture.pdf}
\vspace{-15pt}
\caption{
\textbf{Point2Vec pre-training.}
Our model divides the input point cloud into %
point patches using farthest point sampling (FPS) and $k$-NN aggregation.
We obtain patch embeddings by applying a mini-PointNet\,\colorsquare{m_pointnet} to each point patch (\emph{right}).
The teacher Transformer encoder\,\colorsquare{m_green} infers a contextualized %
representation for all patch embeddings which, after normalization and averaging over the last $K$ Transformer layers, serve as training targets.
The student's input is a masked view on the input data, \ie we randomly mask out a ratio of patch embeddings and only pass the remaining embeddings into the student Transformer encoder\,\colorsquare{m_blue}.
After applying a shallow decoder\,\colorsquare{m_red} on the outputs of the student, padded with learned mask embeddings\,\protect\maskembedding{}, we train the student and decoder to predict the latent teacher representation of the patch embeddings.
\vspace{-10pt}
}
\label{fig:model}
\end{figure*}
\section{Method}

The aim of this work is to unlock the full potential of data2vec-like\,\cite{baevski2022data2vec} pre-training on point clouds by addressing point cloud specific challenges.
To achieve this, we first summarize the technical concepts of data2vec (\refsec{method_d2v}) and show how to learn rich representations on point clouds using data2vec pre-training (\refsec{method_d2v_pcl}).
Finally, we propose \name{}, which accounts for the point cloud specific limitations of data2vec (\refsec{method_p2v}).

\subsection{Data2vec}\label{sec:method_d2v}
Data2vec\,\cite{baevski2022data2vec} is designed to pre-train Transformer-based models, which involve a feature encoder that maps the input data to a sequence of embeddings.
These embeddings are subsequently passed to a standard Transformer encoder to generate the final latent representations.
During pre-training, two versions of the Transformer encoder are kept: a \emph{student} and a \emph{teacher}.
The teacher is a momentum encoder, \ie its parameters $\Delta$ track the student's parameters $\theta$ by being updated after each training step according to an exponential moving average (EMA) rule\,\cite{caron2021dino, baevski2022data2vec, grill2020BYOL, he2020moco}: $\Delta \leftarrow \tau \Delta + (1-\tau)\theta$,
where $\tau \in [0,1]$ is the EMA decay rate.
The teacher provides the training targets, which the student predicts given a corrupted version of the same input.

In a first step, the teacher encodes the uncorrupted input sequence.
The training targets are then constructed by averaging the outputs of the last $K$ blocks of the teacher, which are normalized beforehand to prevent a single block from dominating the sum.
Due to the self-attention layers, these targets are \emph{contextualized}, \ie they incorporate global information from the whole input sequence.
This is an important difference to other masked-prediction methods such as BERT\,\cite{devlin2018bert} and MAE\,\cite{he2022mae}, where the targets only comprise local information, \eg a word or an image patch. %

The student is given a masked version of the same input, where some of the embeddings in the input sequence are substituted by a special learned \emph{mask embedding}. %
The student's task is to predict the targets corresponding to the masked parts of the input.
The model is trained by optimizing a Smooth L1 loss on the regressed targets. %







\subsection{Data2vec for Point Clouds}\label{sec:method_d2v_pcl}

To apply data2vec to point clouds, we utilize the same underlying model as Point-BERT\,\cite{yu2021pointbert} and Point-MAE\,\cite{pang2022pointmae}.
This model is well suited for data2vec pre-training: it extracts a sequence of patch embeddings from the input point cloud and feeds it to a standard Transformer encoder.
For downstream tasks, we append a task-specific head to the Transformer encoder (\refsec{experiments}).
Next, we describe the point cloud embedding and the Transformer in detail and conclude with a summary of data2vec for point clouds.


\parag{Point Cloud Embedding.}
First, we sample $n$ center points from the input point cloud using farthest point sampling (FPS)\,\cite{qi2017pointnetplusplus}.
Grouping the center points' $k$-nearest neighbors ($k$-NN) in the point cloud yields $n$ contiguous \emph{point patches}, \ie sub-clouds of $k$ elements.
Next, we normalize the point patches by subtracting the corresponding center point from the patch's points.
This untangles the positional and the structural information.
To account for the permutation-invariant property of point clouds, we employ a mini-PointNet\,\cite{qi2016pointnet} (\reffig{model}, \emph{right}) that maps each normalized point patch to a \emph{patch embedding}.

The mini-PointNet involves the following steps:
First, we map each point of a patch to a feature vector using a shared MLP.
Then, we concatenate max-pooled features to each feature vector.
The resulting feature vectors are then passed through a second shared MLP and a final max-pooling layer to obtain the patch embedding.

\paragraph{Transformer Encoder.}
The central component of the model is a standard Transformer encoder.
The patch embeddings form the input sequence to the Transformer encoder.
Since the point patches are normalized, the patch embeddings carry no positional information;
therefore, a two-layer MLP maps each center point to a position embedding, which is then added to the corresponding patch embedding.
Due to the special importance of positional information in point clouds, the position embeddings are added again before each subsequent Transformer block to ensure that the positional information is incorporated at every step of the encoding process.

\paragraph{\emakefirstuc{\datavec{}}.}

To establish a baseline, we apply the unmodified data2vec approach to the previously described underlying model of Point-BERT and Point-MAE.
Going forward, we will refer to this approach as \datavec{}.


\subsection{\emakefirstuc{\name{}}}\label{sec:method_p2v}
In \reffig{model}, we present the complete pipeline of our \name{} model.
Directly applying data2vec to point cloud data without modifications is not optimal, as the position embeddings are also added to the mask embeddings, revealing the overall shape of the point cloud to the student.
As positions are the only features for point clouds, this makes the masking far less effective, as noted by Pang \etal \cite{pang2022pointmae} in the context of masked autoencoders.

To solve this issue, we adopt an approach inspired by MAE\,\cite{he2022mae}, where we only feed the non-masked embeddings to the student\,\colorsquare{m_blue}.
A separate decoder\,\colorsquare{m_red}, implemented as a shallow Transformer encoder, takes the output of the student and the previously held-back masked embeddings\,\maskembedding{} as input and predicts the training targets.
In contrast to \datavec{}, this approach does not suffer from leaking positional information from the masked-out point patches to the student.
Moreover, utilizing an MAE-inspired setup provides additional benefits:
First, the student is more computationally efficient, as it only needs to process the non-masked embeddings.
Second, the model's inputs during fine-tuning are more similar to those during pre-training because the inputs during pre-training are no longer dominated by masked embeddings which are absent during fine-tuning.
This likely makes the learned representations more transferable to downstream tasks.

\section{Experiments}
\label{sec:experiments}

\subsection{Setup}
\textbf{Datasets.} We evaluate RFFR with four challenging datasets specifically designed for deepfake detection. We adopt the high quality (HQ) version of Faceforensics++ (FF)~\cite{ff} for training our deepfake detector. Faceforensics++ includes videos of real faces as well as four subsets of fake faces, each manipulated with a different algorithm, namely Deepfakes (DF), Face2Face (F2F), FaceSwap (FSW) and NeuralTextures (NT). We also utilize the test set of Celeb-DF~\cite{celeb-df} and DFDC~\cite{dfdc} for evaluating the cross-dataset performance of our model. Finally, in addition to real faces of Faceforensics++, we adopt the real face images from ForgeryNet (FN)~\cite{forgerynet} for learning RFFR, which helps improve representation learning with additional data.

\textbf{Implementation Details.} We extract the frames from all video datasets and use RetinaFace~\cite{retinaface} to detect and align the faces. All images are scaled to the size of $224 \times 224$. For our RFFR model, we adopt a base version of Masked Autoencoder (MAE)~\cite{mae} and train it on real faces with a batch size of $128$. Following MAE, we set the learning rate at $7.5 \times 10^{-5}$ and adjust it with a schedule with warmup and cosine decay. By default, we train this model with the real faces from both FF~\cite{ff} and FN~\cite{forgerynet}. 

For training the deepfake detector, we divide each image with $k = 4$ (Refer to Appendix for the motivation of choosing $k$). Each block enters the classifier with a probability of $p = 0.25$, and the residual images are amplified by $\alpha=4$. No data augmentation is applied to the images. We initialize both branches of Vision Transformer with ImageNet-pretrained weights and train them with a learning rate of $2 \times 10^{-5}$. During testing, we iteratively mask and restore all blocks to obtain a full residual image for the detector to process. We evaluate the testing results with AUC (Area Under Curve). 

\subsection{Cross-domain performance evaluation}
In this section, we test the performance of our RFFR-based deepfake detector with cross-manipulation and cross-dataset evaluations. 

\textbf{Cross-manipulation evaluations.} We train our deepfake detector on each subset of Faceforensics++ and test on all four subsets to demonstrate our model's ability to identify different manipulations, including those not seen during training. \emph{We adopt the HQ version of FF for both training and testing, and only use one frame every video for testing.} We compare our results with state-of-the-art image-based methods Multi-Attention~\cite{multiatt}, DCL~\cite{dcl}, RECCE~\cite{recce} and UIA-ViT~\cite{uia}. We ran the public code of RECCE and UIA-ViT to produce results under the same setting.

In~\cref{tab:cross-manipulation}, we show that our method outperforms the state-of-the-art methods under most settings, with a maximum improvement of $10.25\%$ (F2F $\rightarrow$FSW). Meanwhile, our model remains effective under the four intra-domain settings, which are shown in gray. The method tends to slightly underperform when trained on NeuralTextures, likely because its manipulation patterns only exist in certain small regions, and may be neglected during our block sampling. Nevertheless, compared to existing methods, our deepfake detector yields much better overall performances. 

\begin{table}[t]
\setlength\tabcolsep{4.5pt} 
\caption{Cross-manipulation performances in terms of AUC(\%) compared with previous methods. Classifiers are trained on one subset of FF and tested on all four subsets. Intra-domain results are marked in gray. We ran the public code of methods marked with "*" to produce results under identical settings \emph{(HQ for training and single frames for testing).}}
\vspace{-1.5em}
\label{tab:cross-manipulation}
\begin{center}  
\scalebox{0.80}{
\begin{tabular}{c|l|cccc|c}
\toprule
Training &\multirow{2}*{Method} & \multicolumn{4}{c|}{Test data} & \multirow{2}*{Avg} \\
\cmidrule(lr){3-6}
     data  &            ~                   & DF    & F2F   & FSW   & NT    & ~   \\
     
\midrule
\multirow{5}*{DF}
& MultiAtt~\cite{multiatt} & \cellcolor{Gray}99.92 & 75.23 & 40.61 & 71.08 & 71.71                \\ 
& DCL~\cite{dcl}       & \cellcolor{Gray}\textbf{99.98} & \textbf{77.13} & 61.01 & 75.01 & 78.28              \\
& RECCE*~\cite{recce}     & \cellcolor{Gray}99.19 & 74.39 & 57.42 & \textbf{85.04} & 79.01                \\ 
& UIA-ViT*~\cite{uia}  & \cellcolor{Gray}99.39      &   74.44    &   53.89    &   70.92    & 74.66 \\ 
& Ours  & \cellcolor{Gray}99.19 & 76.61 & \textbf{68.96} & 74.83 & \textbf{79.90}            \\ 
       
\midrule
\multirow{5}*{F2F}
        & MultiAtt~\cite{multiatt}       & 86.15 & \cellcolor{Gray}99.13 & 60.14 & 64.59 & 77.50 \\
        & DCL~\cite{dcl}       & 91.91 & \cellcolor{Gray}99.21 & 59.58 & 66.67 & 79.34 \\
       & RECCE*~\cite{recce}       & 88.04 & \cellcolor{Gray}98.93 & 67.35 & 74.16 & 82.12 \\
       & UIA-ViT*~\cite{uia}       & 83.39 & \cellcolor{Gray}98.32 & 68.37 & 67.17 & 79.31 \\
       & Ours                                  & \textbf{93.75} & \cellcolor{Gray}\textbf{99.61} & \textbf{78.62} & \textbf{79.56} & \textbf{87.81} \\

\midrule
\multirow{5}*{FSW}
& MultiAtt~\cite{multiatt} & 64.13 & 66.39 & \cellcolor{Gray}99.67 & 50.10 & 70.07              \\
& DCL~\cite{dcl}           & 74.80 & 69.75 & \cellcolor{Gray}99.90 & 52.60 & 74.26              \\
& RECCE*~\cite{recce}       & 66.66 & 73.66 & \cellcolor{Gray}\textbf{99.76} & \textbf{57.46} & 74.39               \\

& UIA-ViT*~\cite{uia}       &   81.02    &   66.30    & \cellcolor{Gray}99.04      &   49.26    & 73.91 \\ 
& Ours                                           & \textbf{87.46} & \textbf{75.96} & \cellcolor{Gray}99.42 & 55.87 & \textbf{79.68}            \\ 

\midrule
\multirow{5}*{NT}
& MultiAtt~\cite{multiatt} & 87.23 & 75.33 & 48.22 & \cellcolor{Gray}98.66 & 77.36                \\
& DCL~\cite{dcl}      & 91.23 & 79.31 & 52.13 & \cellcolor{Gray}\textbf{98.97} & 80.41                \\
& RECCE*~\cite{recce}    & \textbf{90.20}  & 76.65 & \textbf{58.06} & \cellcolor{Gray}97.17 & \textbf{80.52}                \\
 & UIA-ViT*~\cite{uia}  &    79.37   &   67.98    &   45.94    &\cellcolor{Gray}94.59       & 71.97 \\
 & Ours     & 84.31 & \textbf{81.04} & 54.67 & \cellcolor{Gray}96.19 & 79.05          \\
       
\bottomrule
\end{tabular}}
\vspace{-2em}
\end{center}
\end{table}

\textbf{Cross-dataset evaluations.} We train our model on the Faceforensics++ dataset and evaluate its performance on the test sets of Celeb-DF\cite{celeb-df} and DFDC~\cite{dfdc}. Specifically, following the previous practice in~\cite{lip}, we validate the model on Celeb-DF and use the selected model to test on DFDC.  \emph{We adopt the HQ version of FF for training, and only use one frame every video for testing.} Under the same setting, we ran the public code of RECCE~\cite{recce}, UIA-ViT~\cite{uia} and SBI~\cite{sbi} to produce corresponding results. In Table~\ref{tab:cross-dataset}, we show a competitive performance with existing image-based methods, signaling satisfying adaptability of RFFR to different datasets, especially high quality datasets like Celeb-DF. 
  
SBI~\cite{sbi} is a recent powerful deepfake detection method. By utilizing a hand-crafted blending algorithm to produce diverse fake samples, it achieves highly competitive performances on datasets including Celeb-DF. We show that by training on fake samples generated by SBI, our approach can further improve upon their state-of-the-art result. 

\begin{table}[]
\setlength\tabcolsep{4.5pt} 
\caption{Cross-dataset performances in terms of AUC(\%) compared with previous methods. Classifiers are trained on FF and tested on Celeb-DF and DFDC. We ran the public code of methods marked with "*" to produce results under identical settings \emph{(HQ for training and single frames for testing).}}
\vspace{-1em}
\label{tab:cross-dataset}
\begin{center}  
\scalebox{0.90}{
\begin{tabular}{l|cc}
\toprule
\multirow{2}*{Method} & \multicolumn{2}{c}{Test data}\\
\cmidrule{2-3}
        ~                           &     Celeb-DF         &  DFDC \\
\midrule
      Xception~\cite{xception}  &     65.30       &    -  \\
      Face X-ray~\cite{xray}          &     74.20       &     70.00 \\
      MultiAtt~\cite{multiatt}        &     67.44       &     67.34 \\
      SPSL~\cite{SPSL}                &     76.88        &   -  \\
      SOLA~\cite{sola}                &       76.02         &  -    \\
      SLADD~\cite{sladd}              &    79.70       &  -  \\
      RECCE*~\cite{recce}             &     68.94       &   68.34   \\
      UIA-ViT*~\cite{uia}             &     80.31      &   67.93   \\
      SBI*~\cite{sbi}                       &       86.46     &   66.60     \\
\midrule
 	Ours                                      &   81.97  & \textbf{72.08}  \\
    Ours + SBI~\cite{sbi}                  &  \textbf{88.98}           &    67.84   \\
\bottomrule
\end{tabular}}
\vspace{-2.5em}
\end{center}
\end{table}

\subsection{Ablation Study}
\label{ablation}

In this section, we analyze the effect of our implementations for RFFR learning and deepfake detection. 

\textbf{Effect of the training data for RFFR.} The effectiveness of deepfake detection with RFFR depends on the quality of representation learning, where the real faces plays an important role. In this experiment, we examine the effect of scaling the real face dataset for representation learning. As a baseline, we learn RFFR with only real faces from Faceforensics++ (FF), the same data we use for the downstream classification tasks. Meanwhile, another model is supplemented with real faces from both FF and ForgeryNet (FN), a significantly larger and more diverse dataset. We train deepfake detectors on the F2F subset of FF with residual images produced by these two models. In Table~\ref{tab:data}, we demonstrate that including the extra dataset of ForgeryNet for learning RFFR consistently improves the performances of the deepfake detector in all tests, creating a maximum performance gain of $9.57\%$  in terms of AUC (F2F $\rightarrow$ NT).

We note that learning RFFR with FF already allows our deepfake detector to outperform the state-of-the-arts. Nevertheless, learning with extra data enhances the efficacy of our real face foundation representations, and further improves the downstream task of deepfake detection. Therefore, refining the representation learning of real faces, especially with large-scale datasets, could be a viable path for further improving generalized deepfake detection. 

In addition, we examine the scalability of RECCE under the same setting, considering that RECCE~\cite{recce} also involves learning to reconstruct real samples for deepfake detection. However, their performance gain is less significant than ours. Although the reconstruction branch of RECCE~\cite{recce} is able to highlight forgery cues with residual images, they tend to involve more background noise caused by imperfect reconstructions, as depicted in~\cref{fig:unet_comparison},. This undermines the ability of residual images to expose artifacts for deepfake detection. 

\begin{table}[t]
\setlength\tabcolsep{4.5pt} 
\caption{Deepfake detection performances of RECCE~\cite{recce} and our method with different real face dataset, namely the real faces from Faceforensics++ (FF) alone, and FF combined with ForgeryNet (FF + FN). Classifiers are trained on F2F and tested on four subsets of FF. We present the results in AUC (\%).  }
\vspace{-1.5em}
\label{tab:data}
\begin{center}  
\scalebox{0.90}{
\begin{tabular}{c|c|cccc|c}
\toprule
\multirow{2}*{Method} & Real face  & \multicolumn{4}{c|}{Test data} & \multirow{2}*{Avg} \\
\cmidrule(lr){3-6}
&dataset  &      DF    & F2F   & FSW   & NT    & ~   \\
    \midrule
\multirow{2}*{RECCE~\cite{recce}}&FF           & 88.04          & 98.93          & 67.35          & 74.16          & 82.12          \\
&FN + FF &  90.12       & 99.24       & 69.89    & 79.59     & 84.71		\\
    \midrule
\multirow{2}*{Ours}&FF           & 90.16          & 98.56          & 74.10          & 69.99          & 83.20          \\
&FN + FF & \textbf{93.44}       & \textbf{99.61}        & \textbf{78.62}       & \textbf{79.56}        & \textbf{87.81}		\\
\bottomrule
\end{tabular}}
\vspace{-1em}
\end{center}
\end{table}

\textbf{Effect of masked image modeling for RFFR.} We analyze the effect of using MIM-based residual images for deepfake detection. We train a UNet-based autoencoder (AE) to learn the reconstruction of real faces and obtain residual images. Our MIM-trained inpainting model and the AE are compared on the quality of reconstruction in~\cref{fig:unet_comparison}. Note that despite being trained with real faces, the AE "generalizes" well to fake images, preserving delicate details, including the artifacts caused by manipulations. Such generalization leaves the residual images empty with little information. 

\begin{figure}
\centering
  \includegraphics[width=0.9\columnwidth]{figs/compare_ICCV_Final.pdf}
  \vspace{-1em}
   \caption{Reconstruction results and residual images of the autoencoder (AE), RECCE~\cite{recce} and our inpainting model. AE reconstructs both images perfectly, leaving no information in residual images. RECCE~\cite{recce} suffers from insufficient training. Our model successfully highlights potential artifacts in the residual image of only the fake face, and therefore can best facilitate deepfake detection. }
\vspace{-1em}
\label{fig:unet_comparison}
\end{figure}

Masked image modeling enables our model to learn better real face representations and inpaint fake faces with real textures instead of artifacts. In the downstream task of deepfake detection,  our classifier generalizes significantly better than the AE-based classifier, which performs only marginally better than learning with no residuals (detailed in Appendix). Both the reconstruction results and the downstream performance confirm the validity of our choice to learn RFFR with MIM instead of direct reconstruction. 


\textbf{Effect of classifier backbone.} In Table~\ref{tab:backbone}, we present the deepfake detection results of vanilla Xception~\cite{xception} and Vision Transformer (ViT)~\cite{vit}, both trained with full original images. The models are trained with the F2F subset of FF and tested on all four subsets. While a larger backbone increases a deepfake detector's generalization performance in some cases, it is not the primary factor of our performance improvement. Instead, it is the residual input aided by RFFR that leads the performance gain.

\begin{table}[t]
\setlength\tabcolsep{4.5pt} 
\caption{Comparing ours results with vanilla backbones. We present the results in AUC (\%).  }
\label{tab:backbone}
\vspace{-1.5em}
\begin{center}  
\scalebox{0.90}{
\begin{tabular}{c|c|cccc|c}
\toprule
Training  &  \multirow{2}*{Method}    &   \multicolumn{4}{c|}{Test Data} & \multirow{2}*{Avg} \\
\cmidrule(lr){3-6}
 data  &   ~  &   DF    & F2F   & FSW   & NT    & ~   \\
    \midrule
\multirow{3}*{F2F} & Xception~\cite{xception} & 84.94          & 99.26          & 58.82          & 71.19          & 78.55          \\
                                   & ViT~\cite{vit}      & 84.25          & 97.89          & 65.53          & 65.18          & 78.21          \\
                                   & Ours     & \textbf{93.44} & \textbf{99.61} & \textbf{78.62} & \textbf{79.56} & \textbf{87.81} \\
\bottomrule
\end{tabular}}
\vspace{-1.5em}
\end{center}
\end{table}

\textbf{Effect of classifier design.} We compare different variants of our classifier design. Specifically, we analyze the performance gains brought by the introduction of two branches and the random input mechanism. We test six variants of our classifier by training them with the F2F subset of FF and testing with the FSW subset. The settings of these variants are specified by the input data they accept, as shown in~\cref{tab:classifier}. 

\begin{table}[t]
\caption{Deepfake detection performances with classifiers of different inputs in terms of AUC (\%). We train the classifiers on F2F and test on FSW.}
\label{tab:classifier}
\vspace{-1.5em}
\begin{center}
\begin{tabular}{c|c|c|c|c}
\toprule
\multicolumn{2}{c|}{Original Image} & \multicolumn{2}{c|}{Residual Image} & \multirow{2}*{AUC (\%)} \\
\cline{1-4}
               Full        &             Random           &          Full          &          Random          &   ~\\
 \hline
\checkmark        &                                       &                            &                                   &  65.53\\
% \hline
                              &                                      &   \checkmark    &                                   &  66.30  \\
 %\hline
\checkmark        &                                      &   \checkmark    &                                   &  71.48  \\
 %\hline
                             &       \checkmark          &                             &                                   &  70.76  \\
%\hline
                             &                                       &                             &      \checkmark       &  68.10  \\
 %\hline
                             &        \checkmark         &                             &      \checkmark       &  \textbf{78.62}  \\
\bottomrule
\end{tabular}
\vspace{-2em}
\end{center}
\end{table}

\begin{table*}[t]
\setlength\tabcolsep{4.5pt} 
\caption{Deepfake detection performances of validated and non-validated models. Classifiers are trained on F2F and tested on four subsets of FF. We present the results and the performance gaps in AUC (\%). Second best results are underlined. }
\label{tab:validation}
\vspace{-1em}
\begin{center}  
\scalebox{0.90}{
\begin{tabular}{c|c|llll|l}
\toprule
\multirow{2}*{Method}  & \multirow{2}*{Validated} & \multicolumn{4}{c|}{Test Data} & \multirow{2}*{Avg} \\
\cmidrule(lr){3-6}
~                   &                      ~                   &      DF               & F2F                    & FSW                 & NT                    & ~   \\
    \midrule
\multirow{2}*{Xception\cite{xception}} &   \checkmark    & 84.94                 & 99.26                & 58.82                 & 71.19                & 78.55            \\
~ &                                             -                              & 83.08   (- 1.86) & 99.12   (- 0.14) & 46.63   (- 12.19) & 64.93   (- 6.26)  & 73.44   (- 5.11)  \\
 \hline
 \multirow{2}*{RECCE\cite{recce}} &\checkmark               & 88.04                & 98.93                 & 67.35                & 74.16                & 82.12            \\
 ~&                                                -                  & 74.51   (- 8.57) & 99.22   (+ 0.29)  & 50.17   (- 17.18) & 59.46   (- 14.70)  & 70.84   (- 11.28) \\
 \hline
\multirow{2}*{Ours} &    \checkmark  & \textbf{93.44}            & \textbf{99.61}            & \textbf{78.62}            & \textbf{79.56}            & \textbf{87.81}            \\
 ~&  - & \underline{91.56} (- 1.88) & \underline{99.39}   (- 0.22) & \underline{76.00}   (- 2.62)  & \underline{76.41} (- 3.15) & \underline{85.84}   ( - 1.97)    \\
\hline
\end{tabular}}
\vspace{-2em}
\end{center}
\end{table*}

We treat the vanilla ViT with full original image input as a baseline, which achieves an AUC of $65.53\%$. By switching to accept the full residual images, we obtain a $0.77\%$ performance gain. Combining the two modalities to form a dual-branch classifier further increases our result to $71.48\%$. This demonstrates that the artifacts are better exploited when both the original and the residual images enter the classifier, and are used as references to each other. Therefore, both modalities should be considered for classification. 

In addition, we improve on the test by merely modifying the baseline ViT to accept randomly selected original image blocks. This results in a $5.23\%$ increase in performance. Similarly, changing full residual input to random residual blocks also results in a $1.8\%$ improvement. These observations confirm our hypothesis in \cref{sec:method_deepfake_detection} that models benefit from learning with random inputs, which prevents the model from only focusing on the most prominent features in an image, and forces it to learn from subtle artifacts. 

Finally, bringing in the random input mechanism for the dual-branch classifier completes our full implementation, which maximally exploits the artifacts exposed by RFFR and achieves the best performance of $78.62\%$. 



\subsection{Validation-free Model Selection}
\label{sec:validation-free}

\begin{figure}
\centering
  \includegraphics[width=0.5\textwidth]{figs/validation-free_ICCV_Final.png}
  \vspace{-1.5em}
   \caption{Comparing the validation curves of RFFR-based deepfake detector and previous methods. Detectors are trained on the F2F subset of FF for $15k$ iterations and validated on four different subsets. (a) to (d) correspond to experiments on DF, F2F, FSW and NT.  Results are reported in AUC (\%). All three methods perform well when validated on F2F. However, under cross-manipulation settings, only our method avoids overfitting during training. The curves are smoothed for better visibility.}
\label{fig:validation-free}
\vspace{-1em}
\end{figure}

Models expected to generalize to other domains benefit from target domain validations~\cite{domainbed}. By frequently performing model validation, we can select the model  that best suits the detection of target manipulation, resulting in high performance on the test set. While using such an \textit{oracle} could be acceptable for the early development of cross-domain algorithms~\cite{domainbed}, it is not ideal for applications, as labeled data of unseen manipulation is usually not available. 

In this section, we demonstrate the potential of our deepfake detector to circumvent this practice and therefore avoid the need for extra validation data. As shown in \cref{tab:validation}, we train our classifier on F2F for 15k iterations and directly use the final model for testing. Simultaneously, we employ four validation sets to select the models with the best validation performances on target data. All validated and non-validated models are tested under the same conditions. We report all results on the target test sets in Table~\ref{tab:validation}. The performance gaps between validated and non-validated models are reported along with the test results. Although our non-validated models are not performing as well as those selected with a validation set, we show that our model remains effective on target data, with a maximum performance drop of $3.15\%$ and an average drop of $1.97\%$. However, previous methods~\cite{xception, recce} suffer from significantly larger performance drops when evaluated under the same procedure. 

To take a closer look at how the cross-manipulation performances vary during training, we train the deepfake detectors again with F2F. We test the AUC performances on all target subsets every 50 iterations to produce validation curves in \cref{fig:validation-free}. Our RFFR-based deepfake detector consistently maintains a high performance long after its peaks without serious overfitting. On the contrary, both previous methods compared here overfit quickly after reaching their highest target domain performances. In addition, compared methods exhibit large fluctuations across different evaluations, while our model remains stable. This suggests that with RFFR, our model focuses exclusively on generalizable features which fall outside the distribution of RFFR. Such resistance to overfitting guarantees our model a satisfying performance even when labeled validation sets are not available, which is generally expected in practice. We present more results on validation-free evaluations in Appendix.
We begin by briefly comparing the performance of the three predictors for $\hat{\bm{x}}_i$ (FE, RK4, NN) before testing \cref{algo:simulator}.

\subsection{Predictor comparison}\label{subsec:results_predictors}
The approximator $\hat{\bm{x}}_i$ should have two properties: being fast and being accurate for large time-steps $\Delta t$. \Cref{fig:predictor_characteristics} shows the two properties. The left panel displays the error in the differential variable $\Delta \omega$ across 200 points of random initial conditions $\bm{x}_{0}$ and voltage parameterizations $\yparams{}_i$. In terms of accuracy, the \gls{NN} performs well and only for smaller time-steps ($\Delta t < 0.05 \si{\second}$), the RK4 approximation becomes more accurate. The RK-schemes exhibit the expected dependency of the time-step - the local truncation error should follow $\mathcal{O}(\Delta t)$ and $\mathcal{O}(\Delta t^4)$ for FE and RK4. The error of the \gls{NN} in contrast is near independent of $\Delta t$. At the same time, the \gls{NN} is the fastest approximator. While the FE approximator is similarly fast, its poor accuracy makes it undesirable and while more accurate, the RK4 scheme has the drawback of comparably long run-time.     

\begin{figure}[!th]
    \centering
    \includegraphics[width=0.95\linewidth]{figures_pdf/predictor_error.pdf}
    \caption{Predictor characteristics: (left) prediction error of $\Delta \omega$ for 200 predictions with random $\bm{x}_0$ and \yparams{}, (right) run-time per point. These results show that \gls{NN} constitute an accurate and fast predictor.}
    \label{fig:predictor_characteristics}
\end{figure}

\subsection{Simulator results}\label{subsec:results_simulators}

We now focus on the performance of the simulator in \cref{algo:simulator}. \Cref{fig:ieee9_prediction} displays the prediction for $\Delta t = \SI{0.05}{\second}$ using \gls{NN}-based approximators $\hat{\bm{x}}_i^{NN}$. 
\begin{figure}[!ht]
    \centering
    \includegraphics[width=0.95\linewidth]{figures_pdf/simulation_results.pdf}
    \caption{Prediction of a state trajectory ($\Delta \omega_i$) and an algebraic variable $V_i$ for time-step size $\Delta t = \SI{0.05}{\second}$ with a \gls{NN}-based simulator. The predictions (dashed lines) coincide with the ground truth (gray lines).}
    \label{fig:ieee9_prediction}
\end{figure}
The results correspond to the first row in \cref{tbl:simulator_results} where we report the maximum absolute errors of $V$ and $\Delta \omega$ along the trajectory and the run-time. \Cref{tbl:simulator_results} shows further results for the \gls{RK4}-based simulator, for time-steps of $\Delta t = \SI{0.1}{\second}$ and $\Delta t = \SI{0.15}{\second}$ and for collocation points at $\bm{T}=[0.3, 0.7]\Delta t$ and $\bm{T}=[0.1, 0.3, 0.5, 0.7, 0.9]\Delta t$, i.e., $s=2$ and $s=5$. The \gls{NN}-based simulator is consistently faster and more accurate than the \gls{RK}-based simulator, except for the case of $\Delta t = \SI{0.05}{\second}$. These results confirm the predictor characteristics observed in \cref{subsec:results_predictors}. The simulation run-time scales approximately inversely proportional with $\Delta t$. Deviations can arise due to varying numbers of iteration in \cref{algo:parameter_update}, however, in the reported cases, we observe usually 5-7 iterations. Increasing the number of collocation points $s$ results in a small increase in run-time in all cases. In terms of accuracy, we observe that more collocation points can lead to better accuracy, when the overall solution quality is good. However, for too large time-steps, here $\Delta t = \SI{0.15}{\second}$, the effect might reverse. The choice of the location of the collocation points, i.e., $\bm{T}$, also matters.%\bz{Can we bold the best numbers in a box? This would help to highlight what people should look at. There are a lot of numbers.}
\begin{table}[!ht]
\renewcommand{\arraystretch}{1.2}
\caption{Comparison of simulators with different predictor schemes}
\label{tbl:simulator_results}
\centering
\begin{tabular}{ccc|ccc}
\toprule
$\Delta t$ & Predictor & s & run-time & $\max |V_i - \hat{V}_i|$ & $\max | \omega_i - \hat{\omega}_i|$ \\
$[\si{\second}]$ & & & [\si{\second}] & $\times 10^{-2} [\si{\pu}]$ & $\times 10^{-3} [\si{\pu}]$ \\ \midrule
\multirow{2}{*}{$0.05$} & NN & $5$ & $\bm{1.85}$ & $0.82$ & $0.35$\\
 & RK4 & $5$ & $3.88$ & $\bm{0.31}$ & $\bm{0.29}$\\ \midrule
\multirow{4}{*}{$0.10$} & \multirow{2}{*}{NN} & $2$ & $\bm{0.98}$ & $2.71$ & $1.29$ \\
& & $5$ & $1.05$ & $\bm{1.32}$ & $\bm{0.62}$ \\
 & \multirow{2}{*}{RK4} & $2$ & $2.21$ & $5.07$ & $2.40$ \\
& & $5$ & $2.27$ & $3.88$ & $2.01$ \\ \midrule
\multirow{4}{*}{$0.15$} & \multirow{2}{*}{NN} & $2$ & $\bm{0.60}$ & $\bm{6.28}$ & $2.93$ \\
& & $5$ & $0.77$ & $6.36$ & $\bm{2.90}$ \\
 & \multirow{2}{*}{RK4} & $2$ & $1.19$ & $11.8$ & $5.06$ \\
& & $5$ & $1.46$ & $19.0$ & $7.75$ \\
% $\approx$ 0.020 & BDF & - & 0.83 & 0.07 & 0.26\\
% $\approx$ 0.045 & BDF & - & 0.67 & 7.73 & 3.36\\
\bottomrule
\end{tabular}%
\end{table}



%The last two rows stem from the \gls{DAE}-solver in Assimulo based on a variable order backward differentiation formula (BDF) with different tolerance levels. We observe that the required time-step size is significantly smaller than for the proposed simulator. The resulting run-times are comparable, however, neither method was optimized for run-tme in this study. \bz{I would delete the the last two rows and skip this discussion. Not sure how this helps the paper.}
This paper presented a comprehensive analysis of the use of \acrfull{PINN} for power system dynamic simulations. We show that \glspl{PINN} (i) are 10 to 1'000 times faster than conventional solvers, (ii) do not face issues of numerical instability unlike conventional solvers, and, (iii) achieve a decoupling between the power system size and the required solution time. However, \glspl{PINN} are less flexible (i.e. they do not easily handle parameter changes), and require an up-front training cost. Overall, this makes \gls{PINN}-based solutions well-suited for repetitive tasks as well as task where run-time speed is crucial, such as for screening.

Besides the comparison between conventional and \gls{NN}-based methods, this paper conducts a deeper analysis on the parameters that affect the performance of the \gls{NN} solutions. In that respect, we introduce a new \gls{NN} regularisation, called dtNN, as a intermediate step between \glspl{NN} and \glspl{PINN}. We show that \glspl{PINN} achieve overall higher levels of accuracy, and more balanced error distributions thanks to the evaluation of the collocation points.

\newpage


%%%%%%%%% REFERENCES
{\small
\bibliographystyle{ieee_fullname}
\bibliography{00_main}
}

%%%%%%%%% SUPPLEMENTARY
\newpage
\onecolumn
\appendix
\input{01_supp_implementation}
\section{Sensitivity to $\alpha_s$}
\label{sec:sensitivity}

\cref{fig:alpha_results_supp} shows aditional experiments on the sensitivity of our method to~$\alpha_s$ in the considered datasets.

\begin{figure}[h]
\centering
\includegraphics[width=\linewidth]{alphas_notshared_supp.pdf}
\caption{Performance our method with $\alpha_s = \{5, 12, 25, 50\}$ on four datasets. One-sigma error bars were computed from three seeds. Blue marks show the performance of fixed strategies, with labels indicating the percentage of the budget allocated to segmentation annotations.
}
\label{fig:alpha_results_supp}
\end{figure}

\input{05_supp_average_performance}
\input{03_supp_surfaces}
\input{04_supp_gt}

\end{document}
