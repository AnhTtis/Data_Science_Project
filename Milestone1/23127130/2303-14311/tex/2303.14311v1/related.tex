\noindent
We contextualize our work with respect to prior works modelling geometry and also among works that aim to make object detection more accurate and efficient.

\noindent
\textbf{Vision Meets Geometry:} Geometry has played a crucial role in multiple vision tasks like detection~\cite{hoiem2008putting, sudowe2011efficient, wang2019pseudo, chen20153d}, segmentation~\cite{sturgess2009combining, li2017foveanet}, recognition~\cite{Geiger2014PAMI, su2015multi} and reconstruction~\cite{schoenberger2016sfm, 3DRCNN_CVPR18, Narapureddy-2018-105893}. Perspective Geometric constraints have been used to remove distortion~\cite{zhao2019learning}, improve depth prediction and semantic segmentation~\cite{ladicky2014pulling} and feature matching~\cite{toft2020single}. However, in most previous works~\cite{wang2019pseudo, schoenberger2016sfm, 3DRCNN_CVPR18} exploiting these geometric constraints have mainly been concentrated around improving 3D understanding. This can be attributed to a direct correlation between the constraints and the accuracy of reconstruction. Another advantage is the availability of large RGB-D and 3D datasets~\cite{chang2015shapenet, Geiger2012CVPR, reizenstein2021common} to learn and exploit 3D constraints. Such constraints have been under-explored for learning based vision tasks like detection and segmentation. 
A new line of work interpreting classical geometric constraints and algorithms as neural layers~\cite{rockwell20228, chen2022epro} have shown considerable promise in merging geometry with deep learning.  

\noindent
\textbf{Learning Based Detection:} Object detection has mostly been addressed as an learning problem. Even classical-vision based approaches~\cite{dalal2005histograms, viola2001rapid} extract image features and learn to classify them into detection scores. With deep learning, learnable architectures have been proposed following this paradigm~\cite{ren2015faster, redmon2016you, carion2020end, liu2021swin}, occasionally incorporating classical-vision ideas such as feature pyramids for improving scale invariance~\cite{lin2017feature}. While learning has shown large improvements in accuracy over the years they still perform poorly while detecting small objects due to lack of geometric scene understanding. To alleviate this problem, we guide the input image with geometry constraints, and our approach complements these architectural improvements.  

\noindent
\textbf{Efficient Detection with Priors:} Employing priors with learning paradigms achieves improvements with little additional human labelling effort. Object detection has traditionally been tackled as a learning problem and geometric constraints were sparsely used for such tasks, constraints like ground plane~\cite{hoiem2008putting, sudowe2011efficient} were used.  

\noindent
Temporality~\cite{ehteshami2022salisa, thavamani2021fovea, yang2022streamyolo} has been exploited for improving detection efficiently. Some of these methods~\cite{ehteshami2022salisa, thavamani2021fovea} deform the input image using approach that exploit temporality to obtain saliency. This approach handles cases where object size decreases with time (object moving away from the camera in scene), but cannot handle new incoming objects. None of these methods explicitly utilize geometry to guide detection, which handles both these cases. Our two-plane prior deforms the image while taking perspective into account without biasing towards previous detections.

\noindent
Another complementary line of works automatically learn metaparameters (like image scale)~\cite{ghosh2021adaptive, sela2022context, chin2019adascale} from image features. However, as they do not employ adaptive sampling accounting for image-specific considerations, performance improvements are limited. Methods not optimized for online perception like AdaScale~\cite{chin2019adascale} for video object detection do not perform well in real-time situations.