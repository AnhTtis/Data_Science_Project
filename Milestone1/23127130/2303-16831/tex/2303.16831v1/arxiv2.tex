\documentclass[11pt]{article}
\linespread{1.18}

\usepackage{amsfonts,amsmath}
\usepackage{latexsym}
\usepackage{amsthm}
\usepackage{amscd}
\usepackage{commath}
\usepackage{epsfig}
\usepackage{amssymb}
\usepackage{caption}
\usepackage{enumitem}
\usepackage{mathtools}
\usepackage{booktabs}
\usepackage[table]{xcolor}
\usepackage[toc,page]{appendix}
\setlength{\textwidth}{500pt}
\setlength{\textheight}{50\baselineskip}
\setlength{\topmargin}{-50pt}
\setlength{\oddsidemargin}{-15pt}

%
%
\usepackage[english]{babel}

%
%
\usepackage[letterpaper,top=2cm,bottom=2cm,left=2.5cm,right=2.5cm,marginparwidth=1.5cm]{geometry}


\usepackage[colorlinks=true, allcolors=blue,bookmarksnumbered]{hyperref}
\usepackage{hyperref}
\hypersetup{
	citecolor = red!80!black
}
\usepackage{color,soul}
\usepackage{float,subcaption}
\usepackage{graphicx}
\usepackage{stackengine}
\usepackage{tikz}
\usetikzlibrary{intersections}
\usepackage[export]{adjustbox}
\usepackage{array}

\usepackage[capitalise]{cleveref}
\usepackage{microtype}

\usepackage{lineno}
%

\newtheorem{defn}{Definition}[section]
\newtheorem{thm}[defn]{Theorem}
\newtheorem{theorem}[defn]{Theorem}
\newtheorem{prop}[defn]{Proposition}
\newtheorem{lem}[defn]{Lemma}
\newtheorem{cor}[defn]{Corollary}
\theoremstyle{remark}
\newtheorem{remark}[defn]{Remark}

\numberwithin{equation}{section}
\numberwithin{figure}{section}

\newcommand{\bb}{\begin{equation}}
\newcommand{\ee}{\end{equation}}
\newcommand{\xx}{X_v ^{(\lambda)}}
\newcommand{\x}{x_i ^{(\lambda)}}
\newcommand{\xa}{x_1 ^{(\lambda)}}
\newcommand{\xj}{x_j ^{(\lambda)}}
\newcommand{\di}{\delta_i ^{(\lambda)}(\rho)}
\renewcommand{\dj}{\delta_j ^{(\lambda)}(\rho)}
\newcommand{\half}{\frac{1}{2}}

	

%
\newcommand{\todo}[1]{{\color{red}To Be Done: #1 }}
\DeclareMathOperator{\arcsinh}{arcsinh}
 \newcommand{\new}[1]{{\color{green} #1 }}
\newcommand{\Hyp}[1]{{\mathbb{H}_{#1} }}
\newcommand{\Hypp}[1]{{\mathcal{H}^{#1} }}
\newcommand*\pFq[2]{{}_{#1}F_{#2}\genfrac[]{0pt}{}}
\newcommand{\indep}{\perp \!\!\! \perp}
\newcommand{\harpoon}{\overset{\rightharpoonup}}
\newcommand{\origin}{\mathbf{o}}
\newcommand{\bigzero}{\mbox{\normalfont\Large\bfseries 0}}
\newcommand{\rvline}{\hspace*{-\arraycolsep}\vline\hspace*{-\arraycolsep}}
\newcolumntype{?}{!{\vrule width 1.5pt }}
\newlength\savedwidth
\newcommand\whline{\arrayrulecolor{black!30}\noalign{\global\savedwidth\arrayrulewidth\global\arrayrulewidth 1.5pt}%
\hline
\noalign{\global\arrayrulewidth\savedwidth}}
\newcommand{\tightoverset}[2]{%
  \mathop{#2}\limits^{\vbox to -.5ex{\kern-0.75ex\hbox{$#1$}\vss}}}
  
\newcommand{\harp}[1]{#1}

\newcommand\sut{\,;\ }
\newcommand{\dfn}[1]{\textbf{\textit{#1}}}
\newcommand{\RL}[1]{{\color{red!50!black} Russ: #1}}
\newcommand\mob{\textnormal{\textsf{M\"ob}}}  %
\newcommand\HH{\mathbb{H}} %
\newcommand\Unif{\mathrm{Unif}}  %
\newcommand\corona{\widetilde{ \partial \HH_{d} }}
\newcommand\haar{\mathrm{Unif}}
\newcommand\Leb{\mathrm{Leb}}
\newcommand\sph{\partial \HH_d}
\newcommand\hm{\nu}  %
\newcommand{\R}{\mathbb{R}}
\newcommand{\ud}{\mathrm{d}}
\newcommand{\ue}{\mathrm{e}}
\newcommand\cells{\mathcal{V}}
\newcommand\ST{\,;\;}

\def\rlabel #1 #2{\begin{equation} \label{#1} #2 \end{equation}}

\def\rproof{\begin{proof}}

\def\Qed{\end{proof}}

\def\eqaln#1{\begin{align*} #1 \end{align*}}
\def\eqalign#1{\begin{align*} #1 \end{align*}}

\def\rcases#1{\begin{cases} #1 \end{cases}}


%
\makeatletter 
\tikzset{ 
reuse path/.code={\pgfsyssoftpath@setcurrentpath{#1}} 
} 
\tikzset{even odd clip/.code={\pgfseteorule}, 
protect/.code={ 
\clip[overlay,even odd clip,reuse path=#1] 
(current bounding box.south west) rectangle (current bounding box.north east)
%
; 
}} 
\makeatother 
\usetikzlibrary{3d,perspective,quotes,angles}
%
%
%
%

\title{\textsc{Ideal Poisson--Voronoi tessellations\\ on hyperbolic spaces, I}}
\author{
Matteo \textsc{D'Achille}\thanks{Universit\'e Paris-Saclay.\hfill  \href{mailto:nicolas.curien@gmail.com}{\texttt{matteo.dachille@universite-paris-saclay.fr}}},
Nicolas \textsc{Curien}\thanks{Universit\'e Paris-Saclay.\hfill  \href{mailto:nicolas.curien@gmail.com}{\texttt{nicolas.curien@gmail.com}}},
Nathana\"el \textsc{Enriquez}\thanks{Universit\'e Paris-Saclay.\hfill  \href{mailto:nathanael.enriquez@universite-paris-saclay.fr}{\texttt{nathanael.enriquez@universite-paris-saclay.fr}}},
Russell \textsc{Lyons}\thanks{Indiana University.\hfill  \href{mailto:rdlyons@indiana.edu}{\texttt{rdlyons@indiana.edu}}},
Meltem \textsc{{\"U}nel}\thanks{Universit\'e Paris-Saclay.\hfill  \href{mailto:nicolas.curien@gmail.com}{\texttt{meltem.unel@universite-paris-saclay.fr}}} }
\date{}

\begin{document}
\maketitle

\centerline{\texttt{PRELIMINARY VERSION}}\vspace{1cm}
%

\vspace{-0.5cm}

\begin{abstract} We study the limit in low intensity of Poisson--Voronoi tessellations in hyperbolic spaces $ \mathbb{H}_{d}$ for $d \geq 2$. In contrast to the Euclidean setting, a limiting non-trivial ideal tessellation $ \mathcal{V}_{d}$ appears as the intensity tends to $0$. The tessellation $  \mathcal{V}_{d}$ is  a natural isometry-invariant (a.k.a.\ M{\"o}bius-invariant) decomposition of $ \mathbb{H}_{d}$ into countably many infinite convex polytopes, each with a unique end.  We study its basic properties, in particular the geometric features of its cells.\end{abstract}

%
%
%
%
%
%
%
%
%
%
%





\section{Introduction}
Poisson--Voronoi tessellations are ubiquitous objects in stochastic geometry~{\cite[Chapter 4]{Moller1994}}. They have been used to model real-world networks~{\cite{baccelli2010stochastic}} and have also been studied for their purely theoretical properties and their intrinsic beauty. Motivated by the recent works \cite{bhupatiraju,BudzinskiCurienPetri} {and by related work on Poisson--Voronoi percolation in the hyperbolic plane~\cite{hansen2021poisson}}, we study Poisson--Voronoi tessellations in the limit when their intensity tends to $0$ as a limiting tessellation. 

\paragraph{Ideal Poisson--Voronoi tessellations.} Let $(E,d,\mathbf{o},\mu)$ be an abstract locally compact metric space equipped with an origin point $\origin$ and a Radon infinite measure $\mu$. For $\lambda >0$, we consider a Poisson cloud of points $  \mathbf{X}^{(\lambda)}=\{X^{{(\lambda)}}_1, X^{{(\lambda)}}_2, \ldots \}$ with intensity $\lambda \cdot \mu$ (the points being ranked {by} their increasing distances to the origin of $E$). This point process enables us to define the Voronoi tessellation $$ \mathrm{Vor}(\mathbf{X}^{(\lambda)})\coloneqq\{C_1,C_2,\ldots\}$$ relative to $\mathbf{X}^{(\lambda)}$, which is a {tiling} of $E$ where the tile $C_{i}$ is made of the points of $E$ that are closer to $X_{i}^{(\lambda)}$ than to any other $X_{j}^{(\lambda)}$. When the underlying space has polynomial growth (e.g., $\mathbb{R}^d$ equipped with Lebesgue measure), the tessellations $ \mathrm{Vor}(\mathbf{X}^{(\lambda)})$ usually degenerate towards the trivial tiling $\{E\}$ when the intensity $\lambda$ tends to $0$.  However, if the underlying space has hyperbolic features, for example on $d$-ary trees with $d \geq 3$ equipped with the length measure, or on $d$-dimensional hyperbolic spaces $d \geq 2$ equipped with their volume measure, then $ \mathrm{Vor}(\mathbf{X}^{(\lambda)})$ converges in distribution as $\lambda \to 0$ to a nontrivial random tiling, which we name the ideal Poisson--Voronoi tessellation. We refer to Section~\ref{sec:conv} for details of the convergence.

\paragraph{Ideal Poisson--Voronoi tessellations on hyperbolic spaces $\mathbb{H}_d$.} In particular, the above recipe applies when $(E,d,\origin, \mu)$ is the hyperbolic space $ \mathbb{H}_d$ equipped with its volume measure. The limiting tessellation %
$  \mathcal{V}_{d}$ can be constructed from a Poisson process of points on the boundary $\partial \mathbb{H}_d $ carrying real numbers which we call \dfn{delays}. This enables us to study the stochastic properties of $ \mathcal{V}_{d}$ directly. In particular, we prove that $\mathcal{V}_{d}$ is a natural cell decomposition of $ \Hyp{d}$ in the following sense:
\begin{thm}[\textsc{Tiles have one end}] \label{thm:decomposition} Almost surely, the tiles of $ \mathcal{V}_d$ have disjoint interiors,\begin{itemize}
\item each tile of  $ \mathcal{V}_{d}$ is an unbounded, convex hyperbolic polytope with an infinite number of facets and with a \emph{unique end},
\item at most $d+1$ tiles can share a common point,
\item $\cells_d$ is locally finite,
\item the law of $ \mathcal{V}_{d}$ is invariant under every isometry of $ \mathbb{H}_d$.
\end{itemize}
In particular,  in dimension {$2$}, the union $ \partial \mathcal{V}_{2}$ of the boundaries of the tiles  is a random infinite embedding of the 3-regular tree in $ \mathbb{H}_{2}$ with geodesic edges whose law is invariant under isometries of the hyperbolic plane.
\end{thm}

\begin{figure}[!h]
 \begin{center}
 \includegraphics[height=5cm]{poisson1}
  \includegraphics[height=5cm]{poisson2}
   \includegraphics[height=5cm]{pointlessBW}
 \caption{ From  \cite{BudzinskiCurienPetri}. Left to right: Poisson--Voronoi tessellations of the hyperbolic plane (in the unit disk model) with decreasing intensity. Their limit (on the right) is $ \mathcal{V}_2$, the  \emph{ideal Poisson--Voronoi tessellation} of the hyperbolic plane.\label{fig:voronoiH}}
 \end{center}
 \end{figure}

The last point of Theorem~\ref{thm:decomposition} follows from the M{\"o}bius invariance of the hyperbolic measure, hence of the Poisson cloud $ \mathbf{X}^{{(\lambda)}}$ for each $\lambda >0$. The second point is also classical in stochastic geometry and is in particular a well-known fact for standard Poisson--Voronoi tessellations. {We denote by $\mathcal{C}_d$ the cell of $\mathcal{V}_{d}$ containing the origin of $ \mathbb{H}_d$, which, by the M{\"o}bius invariance of $\mathcal{V}_{d}$, is considered to be {\it typical}. Through this, we investigate the fine properties of the tiles of $\mathcal{V}_{d}$.}
\begin{figure}[!hbtp]
\centering
\begin{minipage}{.53\textwidth}
     \includegraphics[width=\linewidth]{C1_dm_o.pdf}
     \end{minipage}
     \hspace{11pt}
     \begin{minipage}{.42\textwidth}
     \includegraphics[width=\linewidth]{C1_BM_jewel_mathematica_o_2}
     \end{minipage}
    \caption{Simulations of the cell of $ \mathcal{V}_d$ containing the origin in the Poincar\'e ball model of $\mathbb{H}_d$ for $d=2$ (left) and $d=3$ (right), obtained by sampling from \ref{eq.intm} conditionally on $2 \cdot \mathcal{E}_{d}=1$.}
    \label{fig.cell_B}
\end{figure}
As we saw in Theorem \ref{thm:decomposition}, the cell $ \mathcal{C}_d$ almost surely has a unique end $\in \partial \mathbb{H}_{d}$, and once we view $ \mathcal{C}_d$ in the upper-half plane model of $ \mathbb{H}_{d}$ with its unique end sent to $\infty$ and the origin sent to $(0, 0, \ldots, 0, 1)$, its law can be described in a surprisingly simple and appealing way using a \dfn{deposition model}: Let $ \mathcal{E}_{d}$ be a random variable with law ${\rm{Exp}}\big(\frac{c_{d}}{d-1}\big)$, where $c_d\coloneqq2^{2-d}\frac{\pi^{d/2}}{\Gamma(d/2)}$ is the volume of the unit $(d-1)$-sphere up to a factor of $2^{1-d}$. Conditionally on $ \mathcal{E}_{d}$, let  $\Pi_d$ be a Poisson cloud of half-spheres in $ \mathbb{R}^{d-1} \times \mathbb{R}_+$ with centers $x \in \mathbb{R}^{d-1}$ and {radii} $\rho >0$ with intensity 
\begin{equation}\label{eq.intm}
 2 \cdot \mathcal{E}_{d} \cdot \ud x \ \rho^{1-2d}\mathrm{d}\rho \, \mathbf{1}_{ \rho \leq \sqrt{1+x^{2}}}\; .
 \end{equation}
\begin{thm}[\textsc{Description of $ \mathcal{C}_d$}]\label{thm.superposition} The law of $ \mathcal{C}_d$ is given by the complement of all open half-balls whose centers and radii are given by $ \Pi_d$.
\end{thm}
In particular, if one forgets about the condition $\mathbf{1}_{ \rho \leq \sqrt{1+x^{2}}}$ in the intensity \eqref{eq.intm}, one sees that $ \mathcal{C}_d$ has almost the same law as a random dilation of the complement of balls whose centers and radii have intensity $\rho^{1-2d}\mathrm{d}\rho \,\mathrm{d}x$. This complement is the epigraph of a random continuous field over $ \mathbb{R}^{d-1}$ whose marginal law is made explicit in Proposition \ref{prop.hscpl}.
%

%
%

%
%
\begin{figure}[!h]
 \begin{center}
\includegraphics[width=0.8\linewidth]{C1_BM_foam_mathematica_o_2bis}

\vspace{30pt}

\includegraphics[width=\linewidth]{C1_hp_o_2000}
 \caption{Representation of $ \mathcal{C}_d$ in the upper half-plane models in dimensions $2$ and $3$ (conditional as in \cref{fig.cell_B}).}
 \end{center}
 \end{figure}

%
%
%
%




The above Poissonian construction of the typical cell $ \mathcal{C}_d$ is the main tool to study various distributional quantities such as: 
\begin{itemize}
\item the hole probability (Proposition \ref{prop.holeprob}) for $ \mathcal{C}_d$,
\item the stationary distribution of the height  and \dfn{azimuth} field describing the deposition model (Proposition~\ref{prop.hscpl}), shedding light on results of  Isokawa \cite{isokawa2000H2} in the $\lambda \to 0$ limit,
\item the asymptotic vertex intensity of $\mathcal{C}_d$ in the Euclidean stationary model (Proposition \ref{prop.vertint}).
\end{itemize}
Various other quantities could be studied on $ \mathcal{V}_{d}$, such as those related to the associated Delaunay simplicial complex. We shall address those questions in a forthcoming work. The paper is structured as follows: In Section \ref{sec:conv} we set up an abstract framework for convergence of Voronoi tessellations whose points converge to the ideal boundary of a space. We then apply those results to the particular case of trees and hyperbolic spaces. Section~\ref{sec.idpvths} establishes the main properties of the ideal Poisson--Voronoi tessellations in $d$-dimensional hyperbolic spaces. Theorem \ref{thm.superposition} on the Poissonian description of $ \mathcal{C}_d$ is proven in the last section, followed by precise calculations of various distributions related to $ \mathcal{C}_d$.  \medskip 


\noindent \textbf{Acknowledgements}. We thank Thomas Budzinski and Bram Petri for enlightening discussions around and during the conception of \cite{BudzinskiCurienPetri}. The work of the first two authors was supported by ANR RanTanPlan. N.E.\ was partially supported by the CNRS grant RT 3477, Geométrie Stochastique. R.L.\ was partially supported by NSF grant DMS-1954086. M.\"U.\ was supported by grants from the Fondation Mathématique Jacques Hadamard (FMJH).


\section{Abstract convergence results}\label{sec:conv}
In this section, we give an abstract point of view on the convergence of Voronoi tessellations based on the concept of Gromov boundary of a metric space $(E,d)$. The main idea is that building a Voronoi tessellation requires us to compare only \emph{differences of distances} rather than actual distances. In particular, the coming Lemma \ref{lem.convvor} is purely deterministic and could be applied to many different spaces. 

\subsection{Gromov boundary and ideal Voronoi tessellations}
Let $(E,d)$ be a locally and boundedly compact metric space. The set $C(E)$ of real-valued continuous functions on $E$ is endowed with the topology of uniform convergence on every compact set of $E$. One defines an equivalence relation on $C(E)$ by declaring two functions equal if they differ by an additive constant and  the associated quotient space endowed with the quotient topology is denoted by $C(E)/\mathbb{R}$. Following \cite{Gro81}, one can embed the original space $E$  in $C(E)/\mathbb{R}$ using the injection
$$ i : \begin{array}{ccccc} 
 E &\longrightarrow&{C}(E)& \longrightarrow& {C}(E)/\mathbb{R}\\ x&\longmapsto & d_{x}\coloneqq d(x,\cdot) & \longmapsto & \overline{d_{x}}\, .
\end{array}
$$
The {\em Gromov compactification} of $E$ is then the closure of $i(E)$ in $C(E)/\mathbb{R}$. The \dfn{Gromov boundary}%
~$\partial E$ of $E$ is composed of the points in the closure of $i(E)$ in $C(E)/ \mathbb{R}$ that are not already in $i(E)$. The points in $\partial E$ are called \dfn{horofunctions} or \dfn{Busemann functions}; see~\cite{Gro81}. They are obtained as limits of shifted distance functions $d(x_n,\cdot)-d(x_n,\origin)$ for some sequence of points $x_n \to \infty$. %
Let us denote a point on the Gromov boundary by $\theta$ and fix the associated representative function $ \mathrm{d}_{\theta}(\cdot)$ so that $ \mathrm{d}_{\theta}(\origin)=0$ for all $\theta \in \partial E$. In particular, if $\theta, \theta' \in \partial E$ and $x \in E$, one can make sense of the ``difference of distances''
 \begin{eqnarray} \mathrm{d}_{\theta}(x) - \mathrm{d}_{\theta'}(x) \in \mathbb{R}. \label{eq:difference}  \end{eqnarray}  We shall then enhance the Gromov boundary by the addition of real numbers that we call the \textit{delays}. Formally, we write $ \widehat{\partial E} = \partial E \times \mathbb{R}$, which we call the \dfn{extended ideal boundary}.  Extending the preceding display, if $x \in E$ and $(\theta,\delta), (\theta',\delta') \in \widehat{\partial E}$ are two extended ideal points, we can compare the \emph{difference} of distances to $x$ and say that $x$ is closer to $(\theta,\delta)$ than to $(\theta',\delta')$ if 
 $$ \mathrm{d}_{\theta}(x) - \mathrm{d}_{\theta'}(x) \leq \delta'- \delta,$$ and we shall write this simply as $ \mathrm{d}(x, (\theta, \delta)) \leq \mathrm{d}(x, (\theta',\delta'))$. 


%
%
 %
  %
%

%
%

%
%
%

\paragraph{(Ideal) Voronoi tessellations.} Let us first recall the basic definition of Voronoi tessellations. Fix a sequence of points $ (x_i \ST i \geq 1) \in E$, called the \dfn{nuclei}. We shall always suppose  that the nuclei $x_i$ are ranked by their increasing distances to the origin of $E$ and that $ d( \origin, x_{i}) \to \infty$. The  \dfn{Voronoi tessellation} $ \mathrm{Vor}(x_{i}: i \geq 1) = \{ C_{i} \ST i \geq 1\}$ is defined via its \dfn{tiles} $C_{i}$ associated to $ x_i $ via
$$
C_{i} \coloneqq \left\lbrace w \in E \ST d(w,x_i)\leq d(w,x_j), ~ \forall j \geq 1 \right\rbrace  .
$$

In particular, $C_i$ is a closed subset of $E$ and the union of all $C_i$ covers $E$. In degenerate cases, the tiles may not have disjoint interiors; see Remark \ref{remark:chiant}. To cope with this problem, we also define the ``open" version of the tiles denoted by $C_i^\circ$ when the weak inequality is replaced by a strict one. Notice that  the $C_i^\circ$ are  disjoint; since we imposed that $d( \origin, x_{j}) \to \infty$ as $j \to \infty$, it follows that $C_i^\circ$ are indeed open sets. We say that the tessellation is \dfn{nondegenerate} if $\overline{C_i^\circ} = C_i$ for every $i \geq 1$. \\

\medskip 



 The framework above enables us to define tessellations using nuclei \emph{that are not points of $E$} but extended ideal points of $ \widehat{\partial E}$. More precisely, if $ \boldsymbol{\theta} \coloneqq \{\theta_{i} \ST i \geq 1\}$ is a sequence of boundary points on $\partial E$ and $ \boldsymbol{\delta}\coloneqq \{\delta_{i} \in \mathbb{R} \ST i \geq 1\}$ is an increasing sequence of reals  tending to $\infty$, we define the (ideal) Voronoi tessellation associated to $  \boldsymbol{\theta},  \boldsymbol{\delta}$ as follows:
 \begin{defn}[\textsc{Ideal Voronoi tessellations}] The Voronoi tessellation $ \mathrm{Vor}( (\theta_{i}, \delta_{i}) \ST {i \geq 1})$ is given by its tiles $\{C_{i} \ST i \geq 1\}$, where 
 $$ C_{i} \coloneqq \{ x \in {E} \ST \forall j \geq 1\ \mathrm{d}(x,(\theta_{i}, \delta_{i})) \leq \mathrm{d}(x, (\theta_{j}, \delta_{j}))  \},$$ and similarly with $C_i^\circ$ defined via a strict inequality.
%
 \end{defn}
 Notice that since we impose that $\delta_{i} \to \infty$, every compact $K \subset E$ intersects only a finite number of tiles of the tessellation, and the tiles $C_i^\circ$ are again open. Also, the tessellation is invariant under a shift of all delays $(\delta_{i} \ST i \geq 0)$ because it depends only on the differences of distances. In the coming section, we will have a rather ``canonical'' way of defining the delays $\delta_{i}$.
 \subsection{Convergence of tessellations}
 As we shall see, the concept of Gromov boundary and ideal tessellations is well suited for studying \emph{convergence} of standard Voronoi tessellations.
\begin{defn}[\textsc{Convergence of tessellations}] We shall say that the sequence of tessellations $\{C_i^{(\lambda)} \ST i \geq 1\}$ converges to $\{C_i \ST i \geq 1\}$ as $ \lambda \downarrow 0$ if for each $i \geq 1$, the closed subset $C_i^{(\lambda)}$ converges to the closed subset $C_i$ locally in the Hausdorff sense. This means that for every $r$, if $B_r(\origin)$ is the closed (hence compact) ball of radius $r$ around the origin, then $$C_i^{(\lambda)} \cap B_r(\origin) \xrightarrow[\lambda \to 0]{} C_i \cap B_r(\origin),$$ for the Hausdorff distance on closed subsets of a compact space.
\end{defn}
The following lemma, roughly speaking, entails the continuity of the Voronoi tessellation mapping with respect to the convergence of nuclei towards the extended Gromov boundary $ \widehat{\partial E}$.
\begin{lem}\label{lem.convvor} Suppose a sequence of points $(x_i^{ (\lambda)} \ST i \geq 1)$ ranked by increasing distances to $\origin$ on a space $(E,d)$ satisfy the two conditions:
\begin{enumerate}
\item \textsc{(Convergence to the boundary)} for all $i\geq 1$, we have the convergence in the Gromov sense   $$x_i^{ ( \lambda)}  \xrightarrow[    \lambda \to 0]{} \theta_i \in  \partial E.$$
\item \textsc{(Convergence of delays)} for all $i\geq 1$, we have  $$d( \origin,x_i^{ ( \lambda)}) - d(\origin,x_1^{ ( \lambda)})  \xrightarrow[\lambda \to 0]{} \delta_{i}$$ as $ \lambda \downarrow 0$. Furthermore, $\delta_{i} \to \infty$ as $i \to \infty$.
\end{enumerate}
Suppose furthermore that $\mathrm{Vor}( (\theta_{i}, \delta_{i}) \ST {i \geq 1})$ is nondegenerate. Then the Voronoi tessellation $ \mathrm{Vor}(  x_{i} ^{(\lambda)} \ST i \geq 1)$ converges (locally in the Hausdorff sense) as $ \lambda \downarrow 0$ towards the ideal tessellation $ \mathrm{Vor}( (\theta_{i}, \delta_{i}) :{i \geq 1})$.
\end{lem}
\begin{remark}[Degenerate cases] \label{remark:chiant}The third condition in the lemma is to avoid ``degenerate'' cases where the limiting Voronoi tessellation is pathological. Imagine the case when $\partial E$ is made of a single point $\{\theta\}$. Then  $x_{i}^{(\lambda)} \to \theta$ as soon as $  d(x_{i}^{(\lambda)}, \origin) \to \infty$ as $\lambda \to 0$. If the limiting delays satisfy $0=\delta_{1} = \delta_{2} <  \delta_{3} \leq  \cdots$. Then the tiles $C_{i}$ of the limiting ideal tessellation
$\mathrm{Vor}( (\theta, \delta_{i}) \ST i \geq 1)$ are then $ C_{1} = C_{2} = E$ and $C_{i} = \varnothing$ for $i \geq 3$. But we may very well have $\mathrm{Vor}(  x_{i} ^{\lambda} \ST i \geq 1) \to  \varnothing$  as $\lambda \to 0$. \end{remark}


\begin{proof} Denote by $C_{i}^{(\lambda)}$ (resp., $C_i^{\circ,(\lambda)}$) the closed (resp., open) tiles of $\mathrm{Vor}(  x_{i} ^{(\lambda)} \ST i \geq 1)$ and by $C_{i}$ (resp., $C_i^\circ$) those of $\mathrm{Vor}( (\theta_{i}, \delta_{i}) \ST {i \geq 1})$. Fix $r \geq 0$ and focus on points $z \in B_{r}( \origin) \subset E$. Up to passing to a subsequence\footnote{Recall that the set of closed subsets of a compact space endowed with Hausdorff distance is a compact space.}, by compactness we may suppose that $C_i^{(\lambda)} \cap B_r( \origin) \to \mathfrak{C}_i  \cap B_r( \origin)$  for some closed subset $  \mathfrak{C}_i$, and our goal is to show that $ \mathfrak{C}_i = C_i$. We define the proto-horofunction 
$$
 \mathrm{d}_{\x} (z) \coloneqq d(\x, z) - d (\x,\origin) \, ,
$$
as well as the delays $ \delta_i^{(\lambda)} \coloneqq d( \origin,x_i^{ ( \lambda)}) - d(\origin,x_1^{ ( \lambda)})$. A given point $z \in B_{r}( \origin)$ thus belongs to $C_{i}^{(\lambda)}$ iff for all $j \ne i$, we have the inequality 

\bb\label{diffij}
d(\x,z)- d(\xj,z) =  \mathrm{d}_{\x} (z) -  \mathrm{d}_{\xj} (z) + \delta_i^{(\lambda)} - \delta_j^{(\lambda)} \leq  0 \, ,
\ee and similarly for $C_i^{\circ, (\lambda)}$ with a strict inequality. %
By our first two assumptions, the function in the last display converges uniformly  over $B_{r}( \origin)$ towards  
$$  \mathrm{d}_{\theta_{i}}(\cdot) - \mathrm{d}_{\theta_{j}}(\cdot) +\delta_{i}- \delta_{j}.$$
Furthermore, by the triangle inequality and the fact that $\delta_{i} \to \infty$, we know that asymptotically, it is sufficient to restrict our attention to a finite set of indices $j$ to check  \eqref{diffij}. Passing those inequalities to the limit, we deduce that for any compact $K \subset C_i^\circ \cap B_r(\origin)$, we have eventually $K \subset C_i^{\circ, (\lambda)}$, so that $ C_i^\circ \subset \mathfrak{C}_i$. Since the tiling is nondegenerate and $\mathfrak{C}_i$ is closed, we have $ \overline{C_i^\circ} = C_i$, so that $C_i \subset \mathfrak{C}_i$. On the other hand, we also have $ \mathfrak{C}_i = \lim_{\lambda \to 0} C_i^{(\lambda)} \subset C_i$: Indeed, if $x \in \lim_{\lambda \to 0} C_{i}^{{(\lambda)}}$ and $y \in C_j$ for $j \ne i$, then there are sequences $x_{i}^{(\lambda)} \in C_{i}^{(\lambda)} \to x$ and $x_{j}^{(\lambda)} \in C_{j}^{(\lambda)} \to y$, so that passing \eqref{diffij} to the limit (notice that all functions involved are Lipschitz), we deduce that $x \in C_{i}$. %
%
%
%
%
%
%
%
%
%
%
%
%
%
%
%
%
%
%
\end{proof}

\section{Ideal Poisson--Voronoi tessellations on hyperbolic spaces}\label{sec.idpvths}
As mentioned in the introduction, the limit in low intensity of Poisson--Voronoi tessellations in $ \mathbb{R}^{d}$ is trivial. This comes from the fact that, although the Gromov boundary of $ \mathbb{R}^{d}$ is non-trivial (it is homeomorphic to $ \mathbb{S}_{d-1}$), the polynomial growth of $ \mathbb{R}^{d}$ imposes that the difference of distances to $\origin$ (called the delays below) of the first two {closest} points in a PPP with intensity $\lambda$ tends to $\infty$ in probability as $\lambda \to 0$. Indeed, it is easy to convince oneself that one requires ``exponential growth" to get tight delays as $\lambda \to 0$. The most natural choice of such  space is obviously the $d$-dimensional hyperbolic space $ \mathbb{H}_{d}$.


\iffalse
\subsection{Background on hyperbolic spaces}
\label{sec.basicfacts}
For the hyperbolic space $ \mathbb{H}_d$ with dimension $d \geq 2$, we shall use the model of the open unit ball $ \mathbb{B}_d = \{ x \in \mathbb{R}^{d}: \|x\|_{2} < 1\}$ equipped with the  $$ \mbox{distance} \quad (x,y) \mapsto 2 \arcsinh{\frac{\|x-y\|_{2}}{\sqrt{(1-\|x\|_{2}^2)(1-\|y\|_{2}^2)}}} \quad \mbox{and measure} \quad \left(\frac{2}{1-\|x\|_{2}^{2}} \right)^{2}\mathbf{1}_{\mathbb{B}_d (x)} \cdot \mathrm{Leb} $$ and the model of the upper-half space $ \mathbb{U}_{d}= \mathbb{R}^{d-1} \times \mathbb{R}_{>0}$ equipped with the  $$\mbox{distance} \quad (x,y) \mapsto 2 \arcsinh{\frac{\|x-y\|_{2}}{2 \sqrt{x_{d} \,  y_{d}}}}\quad \mbox{and measure} \quad \frac{1}{x_{d}^{2}}\mathbf{1}_{x_{d}>0}\cdot \mathrm{Leb}; $$ see~{\cite[Exercise 3.3.1 and afterwards]{beardon2012geometry}}. If the underlying model is not specified, we shall write respectively $ \mathrm{Vol}_{ \mathbb{H}_d}$ and $\mathrm{d}_{ \mathbb{H}_d}$ for the hyperbolic measure and distances. The origin of $\mathbb{H}_d$ will be denoted by $\origin$, meaning the center of the ball or $(0, 0, \ldots, 1)$ in $\mathbb{U}_d$. \medskip 

The isometries of $\mathbb{H}_d$ form a group $\mob_d$, also known as the group of  M\"obius transformations of $\HH_{d}$. They leave both  $ \mathrm{Vol}_{ \mathbb{H}_d}$ and $\mathrm{d}_{ \mathbb{H}_d}$ invariant. M\"obius transformations  are best understood in the hyperboloid model  $$
\{(u_1,\dots,u_d , v) \ST \| u \| ^2 - v^2 = -1 \}
$$
where $(u_1, \dots, u_d, v)$ are coordinate functions in $\mathbb{R}^{d+1}$. Then the orientation-preserving isometries of $\mathbb{H}_d$ form the special orthogonal group $\text{SO}(d,1)$, which is the group of linear transformations of $\mathbb{R}^{d+1}$ leaving the Lorentz form invariant \cite[Page 41]{lee2006riemannian}. For $\mathbb{B}_d$, this corresponds to M\"obius transformations in $\mathbb{R}^d$ mapping the unit ball onto itself \cite[Theorem 3.4.1]{beardon2012geometry}.\\

The mapping from $\mathbb{B}_d$ onto $\mathbb{U}_d$ is given by the generalized Cayley transform, which is the diffeomorphism $\kappa\colon \mathbb{B}_d \rightarrow \mathbb{U}_d$

\begin{align*}
\kappa(X) &= x = \frac{1}{X_1 ^2 + \dots + X_{d-1} ^2 + (X_d-1)^2}\Big( 2X_1, \dots, 2X_{d-1}, 1-\|X\|^2 \Big) \, , \\
\kappa^{-1} (x) &= X = \frac{1}{x_1 ^2 + \dots + x_{d-1} ^2 + (x_d+1)^2}\Big( 2x_1, \dots, 2x_{d-1}, \|x\|^2 -1 \Big) 
\end{align*}

\noindent for $X \in \mathbb{B}_d$ and $x \in \mathbb{U}_d$ \cite[Proposition 3.5]{lee2006riemannian}. For instance, in {dimension} $d=2$, let $z= -X_2+i X_1 \in \mathbb{B}_2$ and $\phi(z) = \frac{2X_1}{X_1 ^2+(X_2-1)^2} + i \frac{1-X_1^2 - X_2 ^2}{X_1 ^2+(X_2-1)^2}$. This can be written as
\begin{equation}\label{eq.transphi}
	\phi(z)= \frac{i \left(1 + z \right)}{1-z} \; , \; z \in \{z \in \mathbb{C} \, \ST |z|<1 \}\; ,
\end{equation}
which is an instance of $\kappa$ in complex coordinates.

\fi

\subsection{Convergence towards the extended boundary}
Let $\mathbf{X}^{(\lambda)}=(X_i^{(\lambda)} \, \ST \, i\geq 1 )$ be a homogeneous Poisson Point Process (PPP) of intensity\footnote{The normalization of the intensity as $\lambda^{d-1}$ is here to ensure that the {closest} point to $\origin$ in $ \mathbf{X}^{(\lambda)}$ is roughly at distance $\log(1/\lambda)$ as $\lambda \to 0$ for every $d \geq 2$.} $\lambda^{d-1} \cdot \mathrm{Vol}_{ \mathbb{H}_d}$ on the hyperbolic space $\Hyp{d}$, %
where as usual the points are ranked by increasing hyperbolic distance to the origin $\origin$. In particular, almost surely $\mathrm{d}_\Hyp{d}(\origin,X_i^{(\lambda)})$ is strictly increasing in $i \geq 1$. Our goal is to prove convergence of those tessellations towards a limiting ideal tessellation as $\lambda \to 0$. To apply Lemma \ref{lem.convvor} one needs to check convergence of points towards the ideal boundary and convergence of delays; both turn out to be very easy:

\paragraph{Convergence of delays.} By the mapping theorem for Poisson processes, it is simple to see that as $\lambda \to 0$, we have 
$$ \frac{\mathrm{d}_\Hyp{d}(\origin,X_1^{(\lambda)})}{|\!\log \lambda|} \xrightarrow[\lambda \downarrow 0]{(\mathbb{P})} 1,$$ and so it is natural to introduce the shifted distances, i.e., the \dfn{delays}, as
\begin{eqnarray}\label{def.delays}D_i^{\left(\lambda\right)} \coloneqq \mathrm{d}_\Hyp{d}(\origin,X_1^{(\lambda)})- \log(1/\lambda), \quad  i\geq 1. \end{eqnarray}
%

The mapping theorem for Poisson process readily entails:
\begin{prop}[\textsc{Convergence of delays}]\label{prop.convdelays}
As $\lambda \downarrow 0$, the set of increasing delays $( D_i^{(\lambda)})_{ i\geq 1}$ converges in law to the increasing points  $(D_{i})_{i\geq 1}$ of a Poisson point process on $\mathbb{R}$ with intensity measure $$c_d \,  \mathrm{e}^{(d-1)s} \mathrm{d}s, \quad \mbox{ where} \quad c_d=2^{2-d}\frac{\pi^{\frac{d}{2}}}{\Gamma\big(\frac{d}{2}\big)}.$$
Equivalently, the set $(  \frac{c_{d}}{d-1}\mathrm{e}^{(d-1)D_i})_{ i\geq 1 }$ is a standard homogeneous Poisson point process on $ \mathbb{R}_+$.\label{cor.expdelays}
\end{prop}

\begin{proof} Recall that the volume growth function for the hyperbolic space is given by $$f_{d}(r) = \mathrm{Vol}_{\Hyp{d}}\big(B_\Hyp{d}(\origin,r)\big)=\Omega_d \int_0^{r} \left(\sinh{\rho}\right)^{d-1} \;  \mathrm{d}\rho, $$ (where $\Omega_d= 2\frac{\pi^{d/2}}{\Gamma(\frac{d}{2})}$ is the volume of $\mathbb{S}_{d-1}$). A straightforward calculation shows that for any $x,y \in \mathbb{R}$ we have
%
$$
\lim_{\lambda \downarrow 0} \lambda^{d-1} \left ( f_{d}\left(x+\log{\frac{1}{\lambda}}\right)-f_{d}\left(y+\log{\frac{1}{\lambda}}\right) \right) = \frac{2^{2-d}}{d-1}\frac{\pi^{\frac{d}{2}}}{\Gamma\big(\frac{d}{2}\big)} \left( \mathrm{e}^{(d-1)x}- \mathrm{e}^{(d-1)y} \right),$$ and both claims follow from the mapping theorem for Poisson process. 
\end{proof}



%

%
%
%
%
%
%
%
%
%
%

\paragraph{Convergence to ideal points.} Let us come back to the Poisson process of nuclei $\{X_i^{(\lambda)}:\, i\geq 1  \}$ and adopt the ball model $ \mathbb{B}_{d}$ for the hyperbolic space. By rotational symmetry, it is clear that conditionally on the distance process $\{\mathrm{d}_\Hyp{d}(\origin,X_i^{(\lambda)}) \ST i \geq 1\}$, or equivalently conditionally on the delays, the angles $\{\Theta_{i}^{(\lambda)} \ST i \geq 1\}$ of the points $X_{i}^{(\lambda)}$ in the ball model $ \mathbb{B}_{d}$ are i.i.d.\ uniform over $ \mathbb{S}_{d-1}$. It is well-known that the Gromov boundary of $ \mathbb{B}_{d}$ is $\mathbb{S}_{d-1}$, or equivalently that a divergent sequence of points $x_{i} \in \mathbb{B}_{d}$ with angles $ \theta_{i}$ converges to $ \theta \in \partial \mathbb{B}_{d} = \mathbb{S}_{d-1}$ if and only if $ x_{i} \to \infty$ and $\theta_{i} \to \theta$~(see
 {\cite{BridsonHaefliger}}, Example 8.11 pag.~265).  Passing to the upper-half space model $ \mathbb{U}_{d}$, the image of the uniform measure over $ \partial \mathbb{B}_{d}$ onto $\partial{U}_{d} = \mathbb{R}^{d-1}$ is given by:
\begin{lem}\label{lem.stereo} The image of the uniform measure on $ \mathbb{S}_{d-1}$ by the stereographical projection from the north pole onto the plane containing the equator $\mathrm{Ste}: \mathbb{S}_{d-1} \to  \mathbb{R}^{d-1}$ is the $(d-1)$-dimensional Cauchy or stereographical law given by 
$$\frac{1}{{c_d}}\frac{1}{\left(1+\sum_{i=1}^{d-1} x_i^2 \right)^{(d-1)}} \mathrm{d}x_1 \ldots  \mathrm{d}x_{d-1},  \quad \mbox{ where } c_d \mbox{ is as in Proposition \ref{prop.convdelays}}.$$
\end{lem}
\begin{proof} This can be proved by a tedious but elementary calculation or by the explicit expression of the metric tensor on the sphere $ \mathbb{S}_{d-1}$ in stereographical coordinates in~\cite[Ch.~3, Eq 3.10]{lee2006riemannian}.
\end{proof}

\subsection{M\"obius action on the corona}
 The final piece needed to construct ideal Voronoi tessellations on hyperbolic spaces is a concrete way of computing distances from extended ideal points. For this, we will see that it is more practical to first perform a change of variable and consider the exponential of the delays. 
 
 Recall the definition of an extended ideal point $(\theta, \delta) \in \widehat{\partial \mathbb{H}_d} = \partial \mathbb{H}_d\times \mathbb{R}$ where $\theta$ is a boundary point and $\delta \in \mathbb{R}$ a delay. It will be convenient in the rest of the manuscript to consider the following equivalent description after taking the exponential of the delays: introduce $ \widetilde{ \partial \mathbb{H}_d } = \partial\mathbb{H}_d\times \mathbb{R}_{+}$, which we call the \dfn{corona}, and consider the image of the extended ideal points 
\[
(\theta, \delta) \in \widehat{\partial \mathbb{H}_d} \mapsto \left( \theta, \frac{c_{d}}{d-1}\mathrm{e}^{(d-1) \delta} \right) \in  \widetilde{ \partial \mathbb{H}_d }.
\]
We call the first coordinate of a point in the corona its \dfn{angle} and the second coordinate its \dfn{radius}. We shall also use the letter ``r" for radii $r, r_i, R, R_i$ and the letter ``$\delta$'' or ``D" for delays. Corollary~\ref{cor.thetaD} can then be recast into the fact that $ \mathcal{V}_{d}$ is the ideal Voronoi tessellation associated to the logarithm of a Poisson process of points in the corona $\widetilde{ \partial \mathbb{H}_d}$ (ranked in increasing order of their radii) with intensity 
  \begin{eqnarray}  \label{eq:defmud}
\mu_{d} =  \mathrm{Unif} \otimes \mathrm{Leb}.  
\end{eqnarray}
The points of the Poisson point process in the corona will be called \dfn{ideal nuclei}. See \cref{f.corona} for an example of the Poisson point process on the corona and its associated Voronoi and Delaunay tessellations.

\paragraph{Distances to nuclei via the Poisson kernel.} Remember that the ideal boundary of $\mathbb{B}_d$ is identified to the $(d-1)$-dimensional sphere $ \mathbb{S}_{d-1}$. In this model we recall the expression of the (hyperbolic)  \dfn{Poisson kernel} that gives the density of the harmonic measure on $\mathbb{S}_{d-1}$ seen from a point $z \in \mathbb{B}_d$: For $z \in  \mathbb{B}_d$ and $\theta\in \mathbb{S}_{d-1}$, write
\[
K(z, \theta)
\coloneqq
\left(\frac{1 - |z|^2}{|z - \theta|^2}\right)^{d-1}
\]
see \cite[Definition 5.1.1]{Stoll}. This function naturally occurs when computing distances from nuclei as shown by the following lemma. The purpose of the following lemma is to characterize the extension of the M\"obius map $\phi \colon \mathbb{B}_d \to \mathbb{B}_d$, which is an isometry of the hyperbolic space, to the corona $\mathbb{S}_{d-1}\times \mathbb{R}_{+}$: 
\begin{lem}[\textsc{M\"obius action extended to the corona}] Let $(x_i^{(\lambda)} \ST i \geq 0)$ be points of $ \mathbb{B}_d$ converging to extended ideal points $(( \theta_i, \delta_i) \ST i \geq 1)$ or equivalently to the nuclei $((\theta_i, r_i) \ST i \geq 1)$ of the corona with $ r_i = \frac{c_{d}}{d-1} \mathrm{e}^{(d-1) \delta_i}$. %
If $ \phi \colon \mathbb{B}_d \to  \mathbb{B}_d$ is an isometry of the hyperbolic space, then $(\phi(x_i^{(\lambda)}) \ST i \geq 1)$ converge as $\lambda \to 0$ to the nuclei of the corona defined by 
\rlabel e.action
{\phi(\theta_i, r_i) \coloneqq \left(\phi(\theta_i), \frac{r_i}{K\bigl(\phi^{-1}(\origin), \theta_i\bigr)}\right).
}
\label{lem:action}
\end{lem}

Recall that convergence to extended ideal points is unchanged under additive shift of delays, and so the radii of the nuclei of the corona are unique up to multiplicative shift. Before proving the lemma, let us deduce a few straightforward consequences: First of all, if $ (\theta_1, r_1),(\theta_2, r_2) $ are two points of the corona, then a given point $z \in \mathbb{B}_d$ is closer\footnote{In the sense of the preceding section.} to $(\theta_1, r_1)$ than to $(\theta_2, r_2)$ iff 
 \begin{eqnarray} \label{eq:closernuclei} \frac{r_1}{K(z, \theta_1)} \leq \frac{r_2}{ K(z, \theta_2)},  \end{eqnarray} and this indeed only depends on ratio of radii. Second, the extension of the action of M\"obius map to the corona satisfies:
 \begin{cor} \label{cor:mobius} The action of the M\"obius group $ \mob_d$ on the corona $ \widetilde{ \partial \mathbb{B}}_d$ defined by \eqref{e.action} is a transitive group action that leaves the measure $\mu_d $ of \eqref{eq:defmud} invariant.
 \end{cor}
 \begin{proof} This can be checked directly on the corona using properties of the Poisson kernel, but let us prove it using  finite intensity approximation: 
 Let $ \mathbf{X}^{(\lambda)}$ be a PPP on $ \mathbb{B}_d$ with intensity $\lambda$. It follows from the last section that $ \mathbf{X}^{(\lambda)}$ converge towards a PPP on the corona with intensity $\mu_d$. On the other hand, the law of $  \mathbf{X}^{(\lambda)}$ is invariant under mapping by a M\"obius map. It follows that $ \mob_d$ is a group action that leaves the measure $\mu_d $ of \eqref{eq:defmud} invariant. It is easy to check that this action is indeed transitive. \end{proof}
 
The measure $\mu_d$ is therefore the Haar measure on $\corona$ for the action of $ \mob_d$. Also, the subgroup of $\mob_d$ that stabilizes $(\theta, x)$ is the subgroup that fixes the horosphere at $\theta$ that passes through the origin.
 %
%
%
%
\begin{figure}[htp] 
\includegraphics[width=\textwidth]{corona-500nuc-seed12345}
\caption{Portions of the ideal Voronoi $ \mathcal{V}_d$ (in black) and Delaunay tessellations (in light blue, where ideal nuclei are joined if the corresponding tiles are adjacent in $ \mathcal{V}_d$), with the corona showing the first 500 ideal nuclei. The radii of the nuclei are scaled linearly to $[1.02, 2.02]$ for graphical reasons. Each point $(\theta, x)$ in the corona is joined to $\theta$ in the ideal boundary.%
}
		\label{f.corona}
\end{figure}


\begin{proof}[Proof of Lemma \ref{lem:action}] Let points $x_{i}^{(\lambda)}  \in \mathbb{H}_{d}$ converging to the extended ideal points $(\theta_{i}, \delta_{i})$ for $i \geq 1$. This means, within the ball model, that $x_{i}^{(\lambda)}  \to \theta_{i}$ in the Euclidean ball $ \overline{\mathbb{B}_{d}}$ and that, up to normalization we can write 
$$ \mathrm{d}_{{ \mathbb{H}_{d}}}( \origin, x_{i}^{(\lambda)}) = \delta_{i}^{(\lambda)} + v_{\lambda},$$ where $\delta_{i}^{{(\lambda)}} \to \delta_{i}$ and $v_{\lambda}  \to \infty$ as $\lambda \to 0$. Let us now pick $\phi \in \mob_d$ and look at the images of those points under $\phi$. Since $\phi$ can be continuously extended to the ideal boundary it follows readily that $\phi( x_{i}^{{(\lambda)}})  \to \phi( \theta_{i})$ as $\lambda \to 0$ for each $i \geq 1$ in the Euclidean ball $ \overline{\mathbb{B}}_{d}$. Now notice that, inside $ \mathbb{B}_{d}$ we can interpret $\delta_{i}^{(\lambda)} + v_{\lambda}$ via the Euclidean distance $\Delta_{i}^{(\lambda)}$ of $x_{i}^{{(\lambda)}}$ to $  \mathbb{S}_{d-1}$, indeed for fixed $i \geq 1$ we have as $\lambda \to 0$ 
$$  \frac{2}{\Delta_{i}^{(\lambda)}} \sim  \mathrm{e}^{\delta_{i}^{{(\lambda)}}} \mathrm{e}^{{v_{\lambda}}}.$$ 
The proof is thus reduced to show that the Euclidean distance $\tilde{\Delta}_{i}^{(\lambda)}$ from $\phi( x_{i}^{{(\lambda)}})$ to $ \mathbb{S}_{d-1}$ is asymptotically equivalent to $\Delta_{i}^{(\lambda)}$ multiplied by $K( \phi^{{-1}}(\origin), \theta_{i})^{ \frac{1}{d-1}}$. This could be checked bare hands on the generators of the group, but let us present a soft proof using properties of hyperbolic isometries. Recall first that a M\"obius map is conformal, that is, it transforms infinitesimal circles into circles: in particular,  at a given angle $\theta$, the norm of the  derivative of $\phi$ in the direction orthonormal to $ \mathbb{S}_{d-1}$ is the same as in all directions in the tangent hyperplane to $ \mathbb{S}_{d-1}$. In particular,  the region $R_{i}^{(\lambda)}$ in $ \mathbb{S}_{d-1}$ around $\theta_{i}$ of radius $\Delta_{i}^{(\lambda)}$ is transformed into a region $\phi(R_{i}^{{(\lambda)}})$ around $\phi(\theta_{i})$ of radius  roughly $\tilde{\Delta}_{i}^{(\lambda)}$. Now, since M\"obius maps preserve the harmonic measure, the harmonic measure of $R_{i}^{(\lambda)}$ seen from $\phi^{-1}(\origin)$ is the same as the harmonic measure of $\phi( R_{i}^{(\lambda)})$ seen from $\origin=0$, see Figure \ref{fig:harmonic}. Expressing this in terms of the Poisson kernel we deduce that  as $\lambda \to 0$ we have 
$$ (\Delta_{i}^{(\lambda)}) ^{d-1} \cdot K(\phi^{-1}(0), \theta) \sim  (\tilde{\Delta}_{i}^{(\lambda)}) ^{d-1} \cdot K(0, \phi(\theta)) = (\tilde{\Delta}_{i}^{(\lambda)}) ^{d-1},$$
which proves our claim.

\begin{figure}[!h]
 \begin{center}
 \includegraphics[width=14cm]{harmonic}
 \caption{Expression of the asymptotic ratio $\Delta_{i}^{{(\lambda)}}/\tilde{\Delta}_{i}^{{(\lambda)}}$ using properties of harmonic measure.}\label{fig:harmonic}
 \end{center}
 \end{figure}
\end{proof}

 
 %

 
%
%
\paragraph{Geometry of perpendicular bisectors of extended ideal points in the upper half-plane.} Before moving to the proof of Theorem \ref{thm:decomposition} let us use the above lemmas to describe the perpendicular bisectors between extended ideal points in the upper half plane model. The hyperplane separating points in $ \mathbb{H}_{d}$ closer to $( \theta_{0}, \delta_{0})$ than $ (\theta, \delta)$ is very simple to describe:
\begin{prop} \label{prop:bissec} In the upper-half plane model~$\mathbb{U}_{d}$, where $\theta_{1}$ is put at $\infty$ and $\theta$ writes $(C,0,\ldots,0)$, the $d-1$-hyperplane bisecting the two nuclei $( \theta_{1}, r_{1})$ and $( \theta, r)$ is an Euclidean half-sphere centered at $(C,0,\ldots,0)$ with radius 
$$ \rho =  \sqrt{1 + C^{2}} \,  \left(\frac{r_{1}}{r}\right)^{\frac{1}{2(d-1)}}.$$
\end{prop}
\begin{proof} The proposition follows by expressing the bisector in $ \mathbb{B}_{d}$ using \eqref{eq:closernuclei} and mapping the resulting locus of $\mathbb{H}_{d}$ using the Cayley transform which maps $\mathbb{B}_{d}$ to the upper half-space model $\mathbb{U}_{d}$. %
%
%
%
%
%
%
%
%
%
%
%
%
%
%
%
%
%
%
%
%
%
%
%
%
%
%
%
%
%
%
%
%
%
%
%
%
%
%
%
%
%
%
%
%
%
%
%
%
%
%
%
%
%
%
%
%
%
\end{proof}

\subsection{The ideal tessellation $ \mathcal{V}_d$}


 
 \begin{cor}\label{cor.thetaD} The Poisson--Voronoi tessellation of $\{ X_{i}^{(\lambda)} \ST i \geq 1\}$ converges in distribution as $\lambda \downarrow 0$ towards the ideal Voronoi tessellation $  \mathrm{Vor}( \boldsymbol{\Theta},  \mathbf{D})$ (equivalently $  \mathrm{Vor}( \boldsymbol{\Theta},  \mathbf{R})$) where $\boldsymbol{\Theta} = (\Theta_{1}, \ldots )$ are iid uniform angles {over} $\mathbb{S}_{d-1} = \partial \mathbb{B}_{d}$ and $  \mathbf{D} = ( D_{i} \ST i \geq 1)$ (resp.~$  \mathbf{R} = ( R_{i} \ST i \geq 1)$) is such that $ (\frac{c_{d}}{d-1}\mathrm{e}^{(d-1)D_{i}})_{i\geq 1}$ (resp. $( R_{i} \ST i \geq 1)$) is a homogenous PPP on $ \mathbb{R}_{+}$ of unit intensity. We call $\mathrm{Vor}( \boldsymbol{\Theta},  \mathbf{D})$, equivalently $\mathrm{Vor}( \boldsymbol{\Theta},  \mathbf{R})$, the ideal Voronoi tessellation of $ \mathbb{H}_d$ and denote it by $ \mathcal{V}_{d}$.
 \end{cor}
 \begin{proof} By Skorokhod embedding theorem, we can couple on the same probability space, all the Poisson processes $\{ X_{i}^{(\lambda)} \ST i \geq 1\}$ for $\lambda >0$ in such a way that 
for every $i \geq 1$ the delays and the angles converge almost surely $D_{i}^{(\lambda)} \to D_{i}$ and $\Theta_{i}^{(\lambda)} \to \Theta_{i}$ where $D_{i}$ is the PPP described above and independent of $\Theta_{i}$ which are iid uniform over $\mathbb{S}_{d-1}$. In particular $ D_{i} \to \infty$ almost surely and the convergence of angles implies the convergence towards the Gromov boundary. It is an easy matter to check using \eqref{eq:closernuclei} that the limiting ideal tessellations are a.s.\ nondegenerate. We can then apply Lemma \ref{lem.convvor} and get the desired convergence. \end{proof}

For nonideal Voronoi tessellations, the distance $d(z, x_i)$ to the nucleus of a cell $C_i$ is an interesting quantity for $z \in C_i$. In the ideal case, however, the analogue is a normalized limit of such distances and, moreover, is not a distance. Instead, 
given $z \in \HH_d$ and $(\theta, r_1) \in \corona$, we call $ r_1/K(z, \theta)$ the \dfn{separation} between $z$ and $(\theta,  r_1)$. 
A point $z$ belongs to the cell of an ideal nucleus $(\theta,  r_1)$ iff $ r_1/K(z, \theta) \le r_2/K(z, \psi)$ for all ideal nuclei, $(\psi, r_2)$, as we saw in \eqref{eq:closernuclei}. In particular, the \dfn{separation field}, whose value at $z$ is the minimum separation between $z$ and all ideal nuclei, determines the ideal Voronoi and Delaunay tessellations. See \cref{f.field} for an example. 
A measurable map that intertwines two actions of a given group $\Gamma$ is called $\Gamma$-equivariant. When the actions preserves probability measures, the second action is then called a $\Gamma$-equivariant factor of the first. When the map is invertible with measurable inverse, then the two actions are called $\Gamma$-equivariantly isomorphic.

%

\begin{theorem}  \label{t.factor}
The separation field is $\mob_d$-equivariantly isomorphic to the point process of ideal nuclei on the corona. Hence, the ideal Voronoi and Delaunay tessellations are $\mob_d$-factors of the point process of ideal nuclei.
\end{theorem}

\rproof
For a locally finite, simple counting measure $N$ on $\corona$, let $f(N)$ be the separation field of $N$, i.e., $f(N)(z) \coloneqq \min \{ r/K(z, \theta) \sut N(\theta, r) = 1\}$. By \eqref{e.action}, $f(N)(z) = f\bigl(\phi_*(N)\bigr)(\origin)$ whenever $\phi\in \mob_d$ satisfies $\phi^{-1}(\origin) = z$. We claim that $f\bigl(\phi_*(N)\bigr) = \phi\bigl(f(N)\bigr)$, i.e., for all $z \in \HH^d$, we have $f\bigl(\phi_*(N)\bigr)(z) = f(N)\bigl(\phi^{-1}(z)\bigr)$.
Indeed, let $\psi\in \mob_d$ satisfy $\psi^{-1}(\origin) = z$. Then $f\bigl(\phi_*(N)\bigr)(z) = f\bigl(\psi_*\phi_*(N)\bigr)(\origin)$ and $f(N)\bigl(\phi^{-1}(z)\bigr) = f\bigl(\eta_*(N)\bigr)(\origin)$ when $\eta^{-1}(\origin) = \phi^{-1}(z)$. For example, we may take $\eta \coloneqq \bigl(\phi^{-1}\psi^{-1})^{-1} = \psi\phi$. This proves our claim and shows that the separation field is a factor of the ideal nuclei.

If the minimum separation from $\origin$ is achieved by the ideal nucleus $(\theta, r)$, which is unique if $z$ lies in the interior of a Voronoi cell, then it is easy to see that the gradient of the separation field at $\origin$ points along the geodesic away from $\theta$, whence the separation field determines the geodesic from $\origin$ to $\theta$. It follows that for every $z \in \HH_d$ not on the boundary of a Voronoi cell, the separation field similarly determines the ideal nucleus of the cell of $z$. Therefore, the factor map $f$ is a.s.\ invertible with measurable inverse, whence is an isomorphism. Also, this shows that the Voronoi cells are determined from the separation field, as are the Delaunay simplices.
%
\Qed


%
%
%
%

\begin{figure}[htp] 
    \centering
\includegraphics[width=.7\textwidth]{{plotVor-Seed12349-Nuclei100-Radius0.99-Scale3-legend}.pdf}
\caption{Part of the separation field.} %
		\label{f.field}
\end{figure}


%
%
%
%
%
%
%

%
%
%

%

\medskip


Let us establish the first few properties of $ \mathcal{V}_d$. First, it follows from Corollary \ref{cor:mobius}  that the law of $ \mathcal{V}_d$ is invariant under isometry. The faces of the ideal Voronoi cells are totally geodesic. This, of course, follows from the equivalent fact for the Poisson--Voronoi tessellations with positive intensity. To see it using Theorem~\ref{t.factor}, it suffices to show that the points in $\HH_d$ at equal separation from two given points in the corona is a totally geodesic subspace. One way to see this is to map the two points in the corona via a M\"obius transformation so that their angles are antipodal points in the ideal boundary, and then use another M\"obius transformation that fixes those two angles but makes the ideal radii equal. 
%
Our ideal Voronoi tessellations share the same a.s.\ local properties as standard Poisson--Voronoi tessellations in dimension $d$:
\begin{prop} $ \mathcal{V}_{d}$ is locally finite and at most $(d+1)$ tiles can share a common point.
\end{prop}

\begin{proof}[Sketch of proof.] The argument is almost the same as for standard Poisson--Voronoi tessellations: 
Given a fixed ball in $\mathbb{B}_d$, the inequality \eqref{eq:closernuclei} combined with the fact that the ideal radii tend to infinity a.s.\ shows that there are only finitely many ideal nuclei that could have smaller separation from any point in that ball than the separation from the ideal nucleus with smallest radius. Therefore, only finitely many ideal Voronoi cells intersect that ball. Since the cells are convex, so is each intersection with the ball, as well as all finite intersections among the cells. As a consequence, the tessellation has only finitely many faces in the ball, i.e., is locally finite.
%

Given $d+1$ ideal nuclei, there is at most one point in $\HH_d$ at equal separation from these ideal nuclei. Label the ideal nuclei in order of increasing radius as $Y_1, Y_2, \ldots$. For each $n$ and $1 \le k_1 < k_2< \cdots < k_{d+1} \le n$, the probability is $0$ that $Y_{k_1}, \ldots, Y_{k_{d+1}}, Y_{n+1}$ have the same separation from any point in $\HH_d$. Therefore, no set of $d+2$ nuclei is at equal separation from any point in $\HH_d$, which proves the second claim. 
%
%
%
\end{proof}





%


%
 Gathering the pieces, we have proved all item of Theorem \ref{thm:decomposition} except the fact that a.s.\ all tiles are one-ended. This will be proved in the next section using a more practical description of the cell containing the origin. Before that, let us quickly apply our machinery to the case of trees:

\subsection{Trees} 
In this section, let us focus on the $k$-regular tree denoted by $ \mathbb{T}_{k}$ with origin vertex $\origin$. It will be convenient in what follows to see $ \mathbb{T}_{k}$ as a real tree by imagining that each edge is a unit real segment (and they are glued according to the combinatorics of the tree), so that it carries a natural length measure $\mu$ which is the Lebesgue measure on its edges, and a geodesic distance $ \mathrm{d}_{ \mathbb{T}_{k}}$. As described above one can then consider a Poisson process of points $ \mathbf{X}^{(\lambda)} = \{X_{1}^{(\lambda)}, \ldots \}$ which are ranked according to their increasing distances to $\origin$. In contrast to the case of hyperbolic spaces, here the asymptotic law of delays will depend upon the fractional part of {$\log_{k-1}(\lambda)$} as $ \lambda \to 0$. More precisely, for $\lambda  \in (0,1) $ define 
$$ \ell_{\lambda} =  - \lfloor \log_{k-1} (k \lambda) \rfloor, \quad \mbox{ and put   }\quad \xi_{\lambda} \coloneqq  (k-1)^{\ell_{\lambda}} \cdot k \cdot \lambda \in [1,k-1).$$
Let us introduce the delay process
$$  \mathbb{D}_{i}^{(\lambda)} = \mathrm{d}_{ \mathbb{T}_{k}}( \origin, X_{i}^{(\lambda)}) - \ell_{\lambda}.$$
\begin{prop} 
As $\lambda \to 0$ with \emph{$\xi_{\lambda} \in [1,k-1)$ fixed}, the delay process $( \mathbb{D}_{i}^{(\lambda)} \ST i \geq 0)$ converges in law towards a Poisson process on $ \mathbb{R}$ with intensity 
$$ \xi_{\lambda} \cdot (k-1)^{m}  \quad \mbox{ over the interval} \quad [m, m+1), \quad \mbox{ for } m \in \mathbb{Z}.$$
\end{prop}
\begin{proof} For $\lambda >0$ fixed, notice that the total intensity of the points falling in edges at distance $p\geq 0$ (i.e., whose closest point is at distance $p$) from the origin is equal to $ k \lambda (k-1)^{p}$. Write then $ p= \ell_{\lambda} + m$ to see that for $m$ fixed, as $\lambda \to 0$ with $\xi_{\lambda}$ fixed, this intensity converges to  $\xi_\lambda (k-1)^m$, which concludes the proof. \end{proof}

\begin{figure}[!h]
 \begin{center}
 \includegraphics[width=17cm]{nucleitreecrop}
 \caption{Illustration of Voronoi tessellations on the $3$-regular tree.}
 \end{center}
 \end{figure}

The convergence to points on the boundary is trivial in this case. Recall that the Gromov boundary $ \partial \mathbb{T}_{k}$ of  the $k$-regular tree can just be identified with the space of all infinite rays starting from the origin equipped with the natural local topology. It has a natural uniform measure. Given this, and the obvious fact that conditionally on their distances to the origin, the points of $ \mathbf{X}^{(\lambda)}$ are iid on the spheres prescribed by their distances, it follows that they converge towards iid  uniform points on $ \partial \mathbb{T}_{k}$.  As in the preceding section, one can check that the ideal tessellations are a.s. nondegenerate (since the delays are a.s.\ distinct) and  we deduce the convergence of the Voronoi tessellations when $\lambda \to 0$ with $\xi_{\lambda}$ fixed. In other words, we have a one-parameter family of ideal tessellations $ \mathcal{I}_{\xi}$ on $ \mathbb{T}_{k}$ parametrized by $\xi \in [1,k-1)$ obtained as limit of Poisson--Voronoi tessellations on $ \mathbb{T}_{k}$. Although those ideals tessellations are indeed pairwise different, here is a surprising fact:
\begin{theorem}[\cite{bhupatiraju}]
\label{t.treeuniq}
The restriction of $ \mathcal{I}_{\xi}$ to the vertices of $ \mathbb{T}_k$ has the same law for all $\xi \in [1,k-1)$.
\end{theorem}
In fact, \cite{bhupatiraju} proved the existence of the restriction to the vertices of the limit of low-intensity Poisson--Voronoi tessellations on regular trees by explicitly calculating all finite-dimensional mar\-gin\-als. For example, \cite[Lemma 2.5]{bhupatiraju} shows that the degree of the root equals $j \in [1, k]$ with probability 
\[
%
\frac1{(k-2)(j-1) + 1}\cdot\frac1{\prod_{i=j}^{k-1} (1+\frac1{i(k-2)})}. %
\]
It is unclear whether there are any Cayley graphs other than trees where such a uniqueness result as in \cref{t.treeuniq} holds, except when the limit is trivial; see \cite{bhupatiraju}.


%
%
%
%
%
%
%
%
%
%
%
%
%
%
%
%
%
%
%
%
%
%
%
%
%
%
%
%
%
%
%
%
%
%
%
%
%
%
%
%
%
%
%
%
%
%
%
%
%
%
%
%
%
%
%
%
%
%
%
%
%
%
%
%
%
%
%
%
%
%
%
%
%
%
%
%
%
%
%
%
%
%
%
%
%
%
%
%
%
%
%
%
%
%
%
%
%
%
%
%
%
%
%
%
%
%
%
%
%
%
%
%
%
%
%
%
%
%
%
%
%
%
%
%
%
%
%
%
%
%
%

%



\section{On the tile of the origin}
In this section we shall describe the law of the cell containing the origin in the upper-half plane model $ \mathbb{U}_{d}$ of the hyperbolic space, once the closest nuclei (as it turns, the only end of this tile) is sent to $\infty$  (Theorem \ref{thm.superposition}). %

\subsection{A half-plane model for the tile of the origin}
Recall the construction of $ \mathcal{V}_{d} = \mathrm{Vor}( (\theta_{i}, R_{i})_{i\geq 1})$ given in the preceding section. We shall focus here on the cell $ \mathcal{C}_{d}$ containing the origin $ \origin \in \mathbb{H}_{d}$. Recall that by our change of variable $$R_i = \frac{c_{d}}{d-1} \mathrm{exp}((d-1)D_i)$$ are the radii of the nuclei on the corona. Clearly, the origin is closer to the nuclei $(\theta_{1}, R_{1})$ than to any other ideal nuclei. We shall then consider the tile $ \mathcal{C}_{d}$ in the upper-half plane model and where the ideal nuclei $(\theta_{1},R_1)$ has been sent to $\infty$ in $ \mathbb{U}_{d}$. Now we can apply the mapping theorem for Poisson process to give an appealing description of the cell as described in Theorem~\ref{thm.superposition}: \medskip 

\noindent \emph{Proof of Theorem~\ref{thm.superposition}}.~Recall from Corollary~\ref{cor.thetaD} that the set of ideal nuclei $(\theta_{i}, R_i)$ is a Poisson point process with intensity $\mu_{d} = \mathrm{Unif}\otimes \mathrm{Leb}_{\mathbb{R}_{+}}$. \emph{Let us condition} on the value of the corona point with smallest radius $( \theta_{1},R_{1})$. By standard properties of Poisson process, the remaining points $( \theta_{i}, R_i)$ form a PPP in $\widetilde{ \partial \mathbb{H}_{d}}$ with intensity 
$$ \mathrm{Unif} \otimes  \mathrm{d}x \mathbf{1}_{x > R_1}.$$
 We then consider all the bisectors between $( \theta_{1}, R_{1})$ and the remaining points $(\theta_{i}, R_{i})_{i\geq 2}$ once mapped into the upper-half plane model. 
 
By Proposition~\ref{prop:bissec}, inside the upper-half plane model~$\mathbb{U}_{d}$, where $\theta_{1}$ is put at $\infty$, for all $i \geq 2$ these bisectors are centered at $ \mathrm{Ste}(\theta_{i})$ and have radius $\sqrt{1+ \|\mathrm{Ste}(\theta_{i})\|^{2}} (R_1/R_i)^{ \frac{1}{2(d-1)}}$, where $\mathrm{Ste}(\theta_{i}) \in \mathbb{R}^{d-1}$ denotes the stereographical projection of $\theta_{i}$ as .

By Lemma~\ref{lem.stereo}, the point process $\left(\mathrm{Ste}(\theta_{i}),R_i-R_1\right)_{ i \geq 2 }$ forms a PPP with intensity $\mu_{d}$ which writes as follows
$$
\frac{1}{c_d}\frac{1}{\left(1+\sum_{i=1}^{d-1} x_i^2 \right)^{d-1}} \mathrm{d}x_1 \ldots  \mathrm{d}x_{d-1} \otimes   \mathrm{d}t \mathbf{1}_{t >0} 
$$
in $ \mathbb{R}^{d-1} \times \mathbb{R}_{+}$.
In a first step, conditional on $R_1=s$, we express the point process of centers and radii in terms of this latter PPP, as follows:

\begin{equation}
\begin{split}
 &\left(\mathrm{Ste}(\theta_{i}),\sqrt{1+ \|\mathrm{Ste}(\theta_{i})\|^{2}}  (R_1/R_i)^{ \frac{1}{2(d-1)}}\right) = \\
 & = \left(\mathrm{Ste}(\theta_{i}),\sqrt{1+ \|\mathrm{Ste}(\theta_{i})\|^{2}} \; \left(\frac{R_1}{R_1+\left(R_i-R_1\right)}\right)^{\frac{1}{2(d-1)}}\right) .
\end{split}
\end{equation}

In a second step, conditioned on $ R_1=s$, we apply the mapping theorem for Poisson processes to get that the point process of centers and radii is Poisson with an intensity
we compute by the following change of variables $\rho = \sqrt{1+|x|^{2}} \left(\frac{s}{s+t}\right)^{\frac{1}{2(d-1)}}$, for a test function $u$,
\begin{equation}
\begin{split}
&   \frac{1}{c_d} \int_{\mathbb{R}^{d-1}\times \mathbb{R}_{+}} u\left(x,\sqrt{1+|x|^{2}} \left(\frac{s}{s+t}\right)^{\frac{1}{2(d-1)}}\right) \frac{1}{\left(1+|x|^{2} \right)^{d-1}}\,   \mathrm{d}x_1 \ldots  \mathrm{d}x_{d-1} \mathrm{d}t\\
& =  \frac{d-1}{c_{d}} \int_{\mathbb{R}^{d-1}\times \mathbb{R}_{+}} u\left(x,\rho \right) \frac{2s}{\rho^{2d-1}} \mathbf{1}_{1+|x|^{2}\geq \rho^{2}} \mathrm{d}x_1 \ldots  \mathrm{d}x_{d-1}   \mathrm{d}\rho.
\end{split}
\end{equation}
Recalling that $R_{1}$ is an ${\rm Exp}(1)$ random variable and noting that $\frac{d-1}{c_{d}}{\rm Exp}(1)={\rm Exp}(\frac{c_{d}}{d-1})$ concludes the proof.
 \hfill \qed
 \vspace{5pt}
 
As a first application of Theorem \ref{thm.superposition}, let us prove:
\begin{lem} For every $ d \geq 2$, the cell $ \mathcal{C}_{d}$ is  almost surely one-ended. A fortiori, almost surely all cells of $ \mathcal{V}_{d}$ are one-ended.\end{lem}
\proof We use the representation of $ \mathcal{C}_{d}$ given in Theorem \ref{thm.superposition}. We first condition on the value of $ s \coloneqq 2 \, \mathcal{E}_{d} >0$ so that $ \mathcal{C}_{d}$ is the complement of the Poisson cloud of half-spheres with intensity 
$$ s \cdot  \mathrm{d}x \ \rho^{1-2d}\mathrm{d}\rho \, \mathbf{1}_{ \rho \leq \sqrt{1+x^{2}}}.$$ 
Our goal is thus to prove that this cloud of half-spheres covers the whole of $ \mathbb{R}^{d-1}$ a.s.  We claim that it is sufficient to prove it 
 the indicator $\mathbf{1}_{ \rho \leq \sqrt{1+x^{2}}}$ in the intensity, since there is a positive probability that under the intensity $s \cdot  \mathrm{d}x \ \rho^{1-2d}\mathrm{d}\rho$ no half-sphere covers the point $ ( \mathbf{0}_{ \mathbb{R}^{d-1}},1)$. Let then be $ \Pi$ a Poisson cloud of half-spheres with radii $\rho$ and centers $x$ with intensity $s \cdot  \mathrm{d}x \ \rho^{1-2d}\mathrm{d}\rho$.  Now fix a ball $B_{ \varepsilon}$ of radius $ \varepsilon>0$ and center $z$ in $  \mathbb{R}^{d-1}$ and notice that, for this ball not to be covered by $\Pi$, no points of $\Pi$ must fall in the region 
$$ \{ (x, \rho) \ST |x-z| \leq \varepsilon, \rho \geq 3 \varepsilon\},$$ which has intensity bounded below by
$$ \mathrm{cst} \cdot \varepsilon^{d-1} \int_{ 3\varepsilon}^{\infty} \frac{\mathrm{d}\rho}{\rho^{2d-1}} \geq \mathrm{cst} \cdot  \varepsilon^{-(d-1)},$$ for some constant $ \mathrm{cst}>0$.
In particular, the probability that $B_{ \varepsilon}$ is not covered is at most $ \exp( - \mathrm{cst}  \cdot \varepsilon^{-(d-1)})$. Using a covering argument and a crude union bound, we deduce that the probability that a given box $[-A,A]^{{d-1}} \subset \mathbb{R}^{{d-1}}$ is not covered by the half-spheres of $\Pi$ is at most 
$$ \mathrm{Cst} \cdot ( A/\varepsilon)^{ d-1} \cdot \exp( - \mathrm{cst}  \cdot \varepsilon^{-(d-1)}),$$ and this probability tends to $0$ as $ \varepsilon \to 0$. For all $A>0$, we deduce that $[-A,A]^{{d-1}}$ is a.s.\ covered by the half-spheres of $\Pi$, as a consequence $ \mathbb{R}^{d-1}$ is also a.s. covered by $\Pi$ and finally $ \mathcal{C}_{d}$ is almost surely one-ended. The same conclusion holds for the cell containing a fixed point $x \in \mathbb{H}_{d}$ by M\" obius invariance (Corollary \ref{cor:mobius}) and hence for all cells simultaneously since all cells have non-empty interior.
\endproof 



\subsubsection{Local property around the origin:  the ``hole probability''}
In this section, we use the preceding construction of the typical cell of $\mathcal{V}_{d}$ to compute  the probability for $\mathcal{C}_d$ to contain a ball centered at the origin: this provides the law of the distance of $\origin$ to $\partial \mathcal{V}_{d}$.


\begin{prop}[\textsc{Hole probability}]\label{prop.holeprob}
The hole probability, i.e., the probability that the ball $B_r(\origin)$ centered at $\origin$ with hyperbolic radius $r$, is contained in $\mathcal{C}_d$ is given by, respectively

\begin{enumerate}[label=(\roman*)]
\item Conditional on $R_{1}=s$, 
$$
\mathbb{P}\left(B_r(\origin) \subset \mathcal{C}_d \ \big |\ R_{1}=s \right) = {\rm exp}\left(-s I_{d}(r) \right)\, ;
$$ 

\item Averaging on the value of $D_{1}$,
$$
\mathbb{P}(B_r(\origin) \subset \mathcal{C}_d)  = \frac{1}{1+I_{d}(r)}  \, ,
$$ 
\end{enumerate}
where
\begin{equation}\label{eq.idr}
\begin{split}
1+I_{d}(r)=  \frac{1}{c_{d}} \int_{\mathbb{R}^{d-1}} \frac{\mathrm{d}x}{\left(\sqrt{\cosh^{2}{r}+|x|^{2}}-\sinh{r}\right)^{2d-2}}
%
%
%
\end{split}
\end{equation}
which can also be written as
$$
\frac{1}{(\cosh{r})^{d-1}}\left( \pFq{2}{1}{\frac{d-1}{2},d-\frac{1}{2}}{\frac{1}{2}}(\tanh^{2}{r})+\frac{2^{d-1} (d-1) \tanh^{2}{r} \Gamma \left(\frac{d}{2}\right)^2 \, \pFq{2}{1}{\frac{d}{2},d}{\frac{3}{2}}(\tanh^{2}{r})}{\sqrt{\pi } \Gamma \left(d-\frac{1}{2}\right)}\right)\; .
$$
\end{prop}
\begin{proof} First, we condition on $R_{1}=s$, so that the intensity measure of the Poisson point process $(x,\rho)$ is given by $2s \frac{d-1}{c_{d}} \mathrm{d}x \otimes \frac{ \mathrm{d}\rho}{\rho^{2d-1}}\mathbf{1}_{ \rho \leq \sqrt{1+x^{2}}}$. Second, we parametrize $B_r(\origin)$ in the upper half-space model $\mathbb{U}_{d}$: here $B_r(\origin)$ is represented by an Euclidean ball $B(C_{e},R_{e})$ of Euclidean center $C_{e}=\left(0,\ldots,0,\cosh{r}\right)$ and Euclidean radius $R_{e}=\sinh{r}$. Hence the event $B_r(\origin) \subset \mathcal{C}_{d}$ corresponds to the event that this point process has no point in the region $\lbrace  (x,\rho) \in \Pi \, \ST \,  \left(\rho+R_{E} \right)^{2} \geq |x|^{2}+|C_{e}|^{2}   \rbrace$ (remark that this region is never empty since $C_{e} \geq 1$ and $R_{e}\geq 0$ if $r \geq 0$). Therefore, by Theorem 1.2, the hole probability is 
$$
\mathbb{P}(B_r(\origin) \subset \mathcal{C}_d) = \text{exp}\left(-s \, I_{d}(r) \right)\; , \;
$$
where 
\begin{equation}\label{eq.idr2}
\begin{split}
 I_{d}(r) &= \frac{2(d-1)}{c_{d}} \int_{\mathbb{R}^{d-1}}\mathrm{d}x \int_{\mathbb{R}_{+}} \frac{\mathrm{d}\rho}{\rho^{2d-1}} \mathbf{1}_{(\rho + R_{e})^{2}\geq |x|^{2}+C_{e}^{2}}\times \mathbf{1}_{\rho^2\leq 1+|x|^{2}}\\
 &=\frac{1}{c_{d}} \int_{\mathbb{R}^{d-1}}\mathrm{d}x \frac{1}{\left(\sqrt{|C_{e}|^2+|x|^{2}}-R_e\right)^{2d-2}}-\frac{1}{c_{d}} \int_{\mathbb{R}^{d-1}}\mathrm{d}x \frac{1}{\left(1+|x|^{2}\right)^{d-1}}  \; .\\
  %
 \end{split}
\end{equation}
The second integral is equal to 1 thus giving \eqref{eq.idr}; the second expression in the statement follows from straightforward substitutions and quadratic transformations for the Gaussian (ordinary) hypergeometric function $\pFq{2}{1}{a,b}{c}(x)$ (see~\href{https://dlmf.nist.gov/15.8.iii}{https://dlmf.nist.gov/15.8.iii}).



%
%
%
%
%
%
%
%
%
%
%
%
%
%
\end{proof}
\noindent
\begin{remark} For $d=2$ we have $I_{2}(r)=\frac{1}{\pi}\left(4 \left(\arctan{\mathrm{e}^{r}}\right)\cosh^2{r}+2\sinh{r}\right)-1$, which gives, for the integrated hole probability,
\[
  \label{e.edgehole}
  \mathbb{P}(B_r(\origin) \subset \mathcal{C}_2) = \frac{\pi}{4 \left(\arctan{\mathrm{e}^{r}}\right)\cosh^2{r}+2\sinh{r}} \, ,
\]
a result first obtained in \cite{bhupatiraju} (Theorem 3.3) by computing the hole probability in finite  Poisson--Voronoi tessellations with intensity $\lambda$ on $ \mathbb{H}_d$ and then taking the limit. %
It is easily seen, using the transformation formula~\cite[9.131]{gradshteyn2014table}, that for odd $d>2$, the (integrated) hole probability reduces to a rational function of $\mathrm{e}^{r}$. For example, in $d=3$ we get
$$
  \mathbb{P}(B_r(\origin) \subset \mathcal{C}_3) =  \frac{3 \mathrm{e}^{-2r}}{2+\mathrm{e}^{2r}}\; .
$$ 
In dimension $2$ differentiating the above expression \eqref{e.edgehole} yields the density for that distance 
\[
r \mapsto \frac{\pi  \cosh r \left(2 (\arctan \ue^r) \sinh r+1\right)} {\left(\sinh r+2 (\arctan \ue^r) \cosh ^2r\right)^2}\,,
\]
whose value at $0$ is $4/\pi$. The tail probability is asymptotic to $2\ue^{-2r}$ as $r \to\infty$. The mean distance is $0.66137^+$, and the median distance is $0.50264^-$. The tail probabilities are plotted in \cref{f.edgehole}. These statistics are reflected in the portrait of the cell given by 60 (pseudoindependent, pseudorandom) samples in \cref{f.2Dportrait}.

\begin{figure}[htp] 
\centering
\includegraphics[width=.6\textwidth]{edgehole}
\caption{The tail probability to be farther than hyperbolic distance $r$ from the ideal Voronoi edges in dimension 2.}
		\label{f.edgehole}
\end{figure}

\begin{figure}[htp] 
\centering
\includegraphics[height=.9\textheight]{2Dportrait}
\caption{A portrait of the cell of the origin in two dimensions given by 60 samples. Each ideal nucleus is at the top.}
		\label{f.2Dportrait}
\end{figure}
\end{remark}

\noindent
\begin{remark}[Average Isoperimetric constant: On a result of Isokawa \cite{isokawa2000H2}]
\label{r.isokawa}
Fix $d \geq 2$ and for  $\lambda,r>0$  consider the  probability that the origin is within distance less than $r$ from the boundary of the Poisson--Voronoi tessellation with intensity $\lambda$ on $ \mathbb{H}_d$:
$$ h(r,d,\lambda) \coloneqq  \mathbb{P}\left( B_r( \origin) \cap \partial\mathrm{Vor}( \mathbf{X}^{(\lambda)}) \ne \varnothing \right) \quad \mbox{ in } \mathbb{H}_d.$$ In particular  $ h(r,d,\lambda)$ can be interpreted as the mean volume (per unit of volume of $ \mathbb{H}_d$) of the region within distance $r$ from  $\partial\mathrm{Vor}( \mathbf{X}^{(\lambda)})$. We then consider the following limits:
  \begin{eqnarray} \label{eq:limitisokawa}\lim_{\lambda \to 0} \lim_{r \to 0} \frac{h(r,d,\lambda)}{r}  \end{eqnarray} and the one in which we interchanged order of limits which can be evaluated explicitly thanks to our results   \begin{eqnarray*} \lim_{r \to 0} \lim_{\lambda \to 0} \frac{h(r,d,\lambda)}{r} &\underset{  \mathrm{Thm.} \ref{thm:decomposition}}{=} & \lim_{r\to 0} \frac{ \mathbb{P}( B_r(\origin) \cap \partial \mathcal{V}_d \ne \varnothing)}{r}\\
  & {=} &  - \left. \frac{ \mathrm{d}}{ \mathrm{d} r} \mathbb{P}_{d}(B_r(\origin) \subset \mathcal{C}_d) \right|_{r=0}\\ &\underset{ \mathrm{Prop.} \ref{prop.holeprob}}{=}&  \frac{2^{d-1} (d-1) \big(\Gamma (\frac{d}{2})\big)^2}{\sqrt{\pi } \Gamma (d-\frac{1}{2})}.  \end{eqnarray*}
 For $d=2$ the right-hand side equals $\frac{4}{\pi}$ and for large $d$ is asymptotically equal to $  \sqrt{2}d  -\frac{9}{4\sqrt{2}}+O(\frac{1}{d})$. For $\lambda$ fixed, the first limit $\lim_{r \to 0} \frac{h(r,d,\lambda)}{r} $ can be interpreted as the mean length (in general $d-1$-volume) of $ \partial \mathrm{Vor}( \mathbf{X}^{(\lambda)})$ per unit area ($d$-volume in general) of $ \mathbb{H}^d$. This quantity has been computed in Isokawa~\cite{isokawa2000H2} for $d=2$ and converges to $\frac{4}{\pi}$ when $\lambda \to 0$. Strikingly, this limit is not degenerate and it turns out to be the same as the one we obtained above. This quantity can also be can interpreted as the ``average Cheeger constant" of $ \mathcal{V}_d$ which was a key input in the work \cite{BudzinskiCurienPetri}. Proving that those limits coincide in general would require to prove concentration for the volume and surface of typical cells of $ \mathrm{Vor}( \mathbf{X}^{(\lambda)})$, a goal that we do not pursue here. 
\end{remark}

%
%
%
%
%
%
%
%
%
%
%
%
%
%
%


\subsection{Asymptotic properties of $\mathcal{C}_d$}
In this section we give the basic properties of the underlying ``stationary'' model defining the law of $ \mathcal{C}_d$ far away from the origin. It is obtained from the original deposition model of Theorem~\ref{thm.superposition} by removing the indicator function in the intensity~\ref{eq.intm}, which ensured that no ball would contain the origin.

More precisely, following the proof of Theorem~\ref{thm.superposition} just above, conditioning on the value of the smallest delay $D_{1}$, we denote by $ \mathcal{M}_{d}$ the random closed subset obtained by removing the balls whose centers and radii $(x,\rho)$ are distributed according to a Poisson point process with intensity 
\begin{equation}\label{eq.intmes}
 2 \frac{d-1}{c_{d}}R_{1} \mathrm{d}x \otimes \frac{ \mathrm{d}\rho}{\rho^{2d-1}}.
 \end{equation}
 
In the following proposition, we describe the law of the boundary of $ \mathcal{M}_{d}$ above a given point  $x_{0} \in \mathbb{R}^{d-1}$ using the height $\mathcal{H}(x_{0})$ of the point of the boundary above $x_{0}$ and the angle $\Theta(x_{0})$ of the tangent hyperplane of the boundary above $x_{0}$ with the  vertical direction. 

The law of $(\mathcal{H}(x_{0}),\Theta(x_{0}))$ is independent of $x_{0}$ and we denote it for short $(\mathcal{H}, \Theta)$.

\begin{prop}[\textsc{height and angle of the boundary}]\label{prop.hscpl} 

For $d\geq2$, the law of $(\mathcal{H}, \Theta)$ is 
\begin{enumerate}[label=(\roman*)]
\item Conditional on $R_{1}=s$, 
$$
\left(\frac{1}{ \mathcal{H} ^{d-1}}, \sin^2(\Theta) \right) \sim  {\rm{Exp}} \left( s \right)\otimes\, {\rm Beta}\left(\frac{d+1}{2},\frac{d-1}{2}\right).
$$ 

\item Averaging on the value of $R_{1}$,
$$
\left( \mathcal{H}, \sin^2(\Theta) \right) \sim  \left( \frac{1-U}{U} \right) ^{\frac{1}{d-1}}\otimes\, {\rm Beta}\left(\frac{d+1}{2},\frac{d-1}{2}\right),
$$ 

where $U$ is a uniform random variable over $[0,1]$.
\end{enumerate}
\end{prop}
 \emph{Proof of (i)} Without loss of generality, we consider the variables $(\mathcal{H}(0),\Theta(0))$.
Let us first condition on $R_{1}=s$. In a preliminary step, we study the first marginal of this couple, namely the law of $\mathcal{H}(0)$. The intensity measure of the Poisson point process $(x,\rho)$ is given by $2s \frac{d-1}{c_{d}} \mathrm{d}x \otimes \frac{ \mathrm{d}\rho}{\rho^{2d-1}}$ and the event $\mathcal{H}(0)\leq h$ corresponds to the event where this point process has no point inside the region $\lbrace (x,\rho) \, \ST \, \rho^{2} \geq |x|^{2} + h^{2}\rbrace$. Hence, 
\begin{equation}\label{eq.phq}
\begin{split}
\mathbb{P}(\mathcal{H}(0) \leq h ) &= \text{exp}\left[-2s \frac{d-1}{c_{d}}\, \int_{\mathbb{R}^{d-1}\times \mathbb{R_{+}}} \mathrm{d}x \frac{ \mathrm{d}\rho}{\rho^{2d-1}} \mathbf{1}_{\rho^{2}\geq h^{2} + |x|^{2}}\right] \\
&= \text{exp}\left[-\frac{s}{c_{d}} \int_{\mathbb{R}^{d-1}}\mathrm{d}x \, \frac{1}{\left(h^{2}+|x|^{2}\right)^{d-1}}\right] \\
& = \text{exp}\left[-\frac{s}{c_{d}} \frac{h^{d-1}}{h^{2(d-1)}} \int_{\mathbb{R}^{d-1}}\mathrm{d}y \, \frac{1}{\left(1+|y|^{2}\right)^{d-1}}\right] \\
&= \exp\left[- \frac{s}{h^{d-1}} \right] \; .
\end{split}
\end{equation}

Considering a test function $g$, and using the Poisson property of the process $(x,\rho)$, we can write

$$
\mathbb{E}\left[g(\mathcal{H},\Theta) \right]  = 2s \frac{d-1}{c_{d}} \int_{\mathbb{R}^{d-1}\times \mathbb{R_{+}}}  \frac{\mathrm{d}x \mathrm{d}\rho}{\rho^{2d-1}} g\left(\sqrt{\rho^{2}-|x|^{2}},\arccos{\frac{|x|}{\rho}}\right)  \mathbf{1}_{|x|\leq\rho}\, \mathbb{P}(\mathcal{H} \leq \sqrt{\rho^{2}-|x|^{2}} ) $$

\begin{equation}\label{eq.condexpg}
\begin{split}
&= 2s \frac{d-1}{c_{d}}  \ \Omega_{d-1}  \int_{\mathbb{R}_{+}\times \mathbb{R}_{+}}\mathrm{d}r \, r^{d-2}  \frac{ \mathrm{d}\rho}{\rho^{2d-1}}  g\left(\sqrt{\rho^{2}-r^{2}},\arccos{\frac{r}{\rho}}\right)  \mathbf{1}_{r\leq\rho}\, \mathbb{P}(\mathcal{H} \leq h )\\
&=    \int_{\mathbb{R}_{+}\times [0,2\pi)}  \left((d-1)\frac{s}{h^{d}} \ \exp\left[-\frac{s}{h^{d-1}}\right] \mathbf{1}_{h >0}\right) \left( 2^{d}\frac{ \Gamma(\frac{d}{2}) }{\sqrt{\pi}\Gamma(\frac{d-1}{2})}  \sin^{d}{\theta} \cos^{d-2}{\theta} \,  \mathbf{1}_{0\leq \theta < \frac{\pi}{2}}\right)  \\
\end{split}
\end{equation}

$_{}$ \hfill $g(h,\theta)\,\mathrm{d}h\mathrm{d}\theta$,
\\ 

which implies (\emph{i}). 

\emph{Proof of (ii)} Reminding that $R_{1} \sim \rm{Exp}(1)$, we average Eq.~\ref{eq.condexpg} and get, for any test function $g$,
\begin{equation}\label{eq.expg}
\begin{split}
&\mathbb{E}\left[g(H,\Theta) \right]  =\\
 &\int_{\mathbb{R}_{+}\times [0,2\pi)} \left((d-1)\frac{h^{d-2}}{(1+h^{d-1})^{2}} \mathbf{1}_{h >0}  \right) \cdot \left( 2^{d}\frac{ \Gamma(\frac{d}{2}) }{\sqrt{\pi}\Gamma(\frac{d-1}{2})}   \sin^{d}{\theta} \cos^{d-2}{\theta}\  \mathbf{1}_{0\leq \theta < \frac{\pi}{2}}\right) g(h,\theta) \, \mathrm{d}h \mathrm{d}\theta \; .
 \end{split}
\end{equation}
The rest is straightforward computation.

\noindent 

\qed


In the following we focus on the intensity of vertices in the stationary model:

\begin{prop}[\textsc{Vertex intensity}]\label{prop.vertint}
In the stationary model ${\mathcal M}_{d}$, the process of vertices has the following intensity:
\begin{enumerate}[label=(\roman*)]
\item Conditional on $R_{1}=s$, the intensity at a point $(\harp{x},z)\in {\mathbb R}^{d-1}\times{\mathbb R}_{+}$ is equal to
$$
 \frac{(\mathbf{c} s)^{d}}{d!}\nu_{d}\frac{\mathrm{e}^{-\frac{s}{z^{d-1}}}}{z^{d^{2}}} \mathrm{d}\harp{x} \mathrm{d} z \, ,
$$
where $\mathbf{c}=\frac{2(d-1)}{c_{d}}=\frac{2^{d-1}(d-1)\Gamma{(\frac{d}{2})}}{\pi^{d/2}}$  and
$$
\nu_{d}= \frac{1}{2^{d}}\int_{[0,1]^{d}\times \left(\mathbb{S}_{d-2}\right)^{d}} \prod_{i=1}^{d}\left[v_{i} \left(1-v_{i} \right)\right]^{\frac{d}{2}-1}
{\rm{Vol}}_{d-1}\left(\sqrt{\frac{v_i}{1-v_i}} u_{i}\right)_{1\leq i\leq d} \mathrm{d}v_{i} \mathrm{d}\harp{u_{i}}\; ,
$$
where ${\rm{Vol}}_{d-1}$ denotes the Euclidean  volume of a $d-1$ dimensional simplex. In particular $\nu_{2}=\frac{3\pi}{4}$ and $\nu_{3}=2.783(1)$.


\item Averaging on the value of $R_{1}$, the intensity at a point $(\harp{x},z)\in {\mathbb R}^{d-1}\times{\mathbb R}_{+}$ is equal to

$$
\mathbf{c}^{d}\nu_{d}\,\frac{1}{z (1+z^{d-1})^{d+1}}\mathrm{d}\harp{x} \mathrm{d} z .
$$


\end{enumerate}


\end{prop}
\begin{proof}
Let $f$ be a test function defined on $\mathbb{H}_{d}$. By the classical Mecke--Slivnyak formula 
\begin{equation}\label{eq.msf}
\mathbb{E}\left[\sum_{\substack{v\, \ST \, \text{vertex}\\ \text{of}\;  \mathcal{C}_d}}f(v) \right] = \frac{(\mathbf{c} s)^{d}}{d!} \int_{\left(\mathbb{R}^{d-1}\right)^{d}\times \left(\mathbb{R}_{+}\right)^{d}} f\left(\displaystyle{\cap_{i=1}^{d}} \partial B(\harp{x_{i}},\rho_{i}) \right)\mathrm{e}^{-\frac{s}{z^{d-1}}} \mathrm{d}\harp{x_{i}} \frac{d\rho_{i}}{\rho_{i}^{2d-1}},
\end{equation}
where $z$ is the vertical coordinate of $\cap_{i=1}^{d} \partial B(\harp{x_{i}},\rho_{i})$. If we denote by $\harp{x}$ the horizontal projection of the point $\cap_{i=1}^{d} \partial B(\harp{x_{i}},\rho_{i})$, and by $\harp{u_i}$ the unit vector $\frac{\harp{x}-\harp{x_i}}{|\harp{x}-\harp{x_i}|}$, we want to perform in the above integral the change of variable $(\harp{x_i},\rho_{i}) \rightarrow (\harp{x},z,\harp{u_i},\rho_{i})$. We use the relation $\harp{x_i}=\harp{x}+\sqrt{\rho_{i}^2-z^{2}}\harp{u_i}$ and get the following Jacobian matrix
\begin{equation}
J=\left(
\begin{array}{c | c | c | c | c | c | c}
I_{d-1} & -z \frac{\harp{u_1}}{\sqrt{\rho_{1}^{2}-z^{2}}}  & \tilde{u}_{1}^{(1)} \cdots   \tilde{u}_{1}^{(d-2)}& 0_{d-1,d-2} & \cdots  &  0_{d-1,d-2} & 0_{d-1,d}\\ \hline
I_{d-1}  & -z \frac{\harp{u_2}}{\sqrt{\rho_{2}^{2}-z^{2}}}  & 0_{d-1,d-2} & \tilde{u}_{2}^{(1)} \cdots   \tilde{u}_{2}^{(d-2)}  &\cdots  &  0_{d-1,d-2} & 0_{d-1,d}\\ \hline
\vdots &  \vdots  & \vdots &\ddots  &  \ddots & \vdots  & \vdots\\ \hline
I_{d-1} & -z \frac{\harp{u_d}}{\sqrt{\rho_{d}^{2}-z^{2}}}  &  0_{d-1,d-2} & \cdots & 0_{d-1,d-2}  &  \tilde{u}_{d}^{(1)} \cdots   \tilde{u}_{d}^{(d-2)} & 0_{d-1,d}\\ \hline
0_{d,d}  &  0_{d,1} & 0_{d,d-2} & \cdots  &  \cdots & 0_{d,d-2} & I_{d}\\ 
\end{array}\right)\, ,\end{equation}
where the blocks correspond to the partial derivatives successively of $\harp{x_1}, \ldots, \harp{x_d},\rho_{i}\text{'s}$ with respect to successively $\harp{x},z,\harp{u_1},\ldots, \harp{u_d},\rho_{i}\text{'s}$ and where for all $1\leq i \leq d$, the family of vectors 
$\left(  \tilde{u}_{i}^{(1)} \cdots   \tilde{u}_{i}^{(d-2)}\right)$
is equal to 
$\sqrt{\rho_{i}^{2}-z^{2}}\left(u_{i}^{(1)} \cdots   u_{i}^{(d-2)}\right) \, ,
$
where the family $(u_{i}^{(j)})_{1\leq j\leq d-2}$ is an orthonormal basis of the tangent space at $u_{i}$ to the unit sphere $\mathbb{S}_{d-2}$ which we shall denote by $u_{i}^\perp$. Once $\text{det}(J)$ is factorized by $z\prod_{i=1}^{d}\left(\rho_{i}^{2}-z^{2} \right)^{\frac{d-2}{2}}$, 
using the linearity of determinant wrt to the second block column and removing the last block line and block column, we deduce that 
$$
\text{det}(J) = - \left(z\prod_{i=1}^{d}\left(\rho_{i}^{2}-z^{2} \right)^{\frac{d-2}{2}}\right) \sum_{i=1}^{d} \frac{\text{det}(J_{i})}{\sqrt{\rho_{i}^{2}-z^{2}}}\, ,
$$
where
\begin{equation}
J_{i}=\left(
\begin{array}{c | c | c | c | c | c | c }
I_{d-1} & 0 & u^{\perp}_{1} & 0 & \cdots  &  \cdots & 0 \\ \hline
I_{d-1}  & 0  & 0 &  \ddots &\cdots  &  0 & 0\\ \hline
\vdots & \vdots  & \ddots &  \ddots &\cdots  &  0 & 0\\ \hline
\vdots &  \harp{u_i}  & \vdots &\ddots  &  u_{i}^{\perp} & \vdots  & \vdots\\ \hline
\vdots & \vdots  &  \vdots & \vdots & \ddots  &  \ddots & 0\\ \hline
I_{d-1} & 0  &  0& \cdots & \cdots  &  0& u_{d}^\perp\\ 
\end{array}\right)\; .\end{equation}

Dispatching the second column block to the left of the column block containing $u_{i}^\perp$, and taking into account that the matrix $\left(\harp{u_i},u_{i}^\perp\right) \in \text{SO}(d-1)$, we get that $\text{det}(J_{i})=(-1)^{i}\epsilon(d)\, \text{det}(K_{i})$, where $\epsilon(d)$ is a sign depending only on $d$ and 
\begin{equation}
K_{i}=\left(
\begin{array}{c | c | c | c | c | c }
I_{d-1} &  u^\perp_{1} & \cdots & \cdots  &  \cdots & 0 \\ \hline
I_{d-1}  &  0 &  \ddots &\cdots  &  0 & 0\\ \hline
\vdots &    \vdots &\ddots  &  \widehat{u}_{i}^\perp & \vdots  & \vdots\\ \hline
\vdots &   \vdots & \cdots & \ddots  &  \ddots & 0\\ \hline
I_{d-1} &   0& \cdots & \cdots  &  0& u_{d}^\perp\\ 
\end{array}\right)\, ,\end{equation}
in which the notation $\widehat{u}_{i}^\perp$ means that the block  $u_{i}^\perp$ has been skipped. 

For all $1\leq k \leq d-1$, we dispatch the $k$-th column of the first column block to the left of the $k+1$-th column block of type $u^\perp_{l}$, and get that the determinant $\text{det}(K_{i})=\epsilon'(d)\text{det}(L_{i})$, where
\begin{equation}
L_{i}=\left(
\bgroup
\arrayrulecolor{black!50}\arrayrulewidth 1.7pt
\begin{array}{ c | c  | c | c | c  | c  c }
 e_{1}  \bigg|  \; \; u^\perp_{1}  \; _{}  &  e_{2}  \hspace{1.1pt} \bigg|  \; \; \;  0 \, \,  \; \; _{}  &  e_{3} \; \; \; \bigg|  \; \; 0  \, \,  \; \; _{} & \cdots & e_{i}  \hspace{1.1pt} \bigg|  \; \; \;   0 \, \,  \; \; _{}   & \cdots\\ \hline %
  e_{1}  \hspace{1.1pt} \bigg|  \; \; \;   0 \, \,  \; \; _{}  &  e_{2}  \bigg|  \; \;   u^\perp_{2}  \; _{}&  e_{3} \; \; \; \bigg|  \; \; 0  \, \,  \; \; _{} & \vdots & \cdots & \cdots\\  \hline
    e_{1}  \hspace{1.1pt} \bigg|  \; \; \;  0 \, \,  \; \; _{}  &   e_{2}  \hspace{1.1pt} \bigg|  \; \;  \;  0 \, \,  \; \; _{}&  \ddots & \vdots &\\  \hline
\vdots &  \vdots &  &  e_{i-1} \; \bigg|  \; \; u^\perp_{i-1}  \; _{}    &\\  \hline
e_{1}  \hspace{1.1pt} \bigg|  \; \; \;   0 \, \,  \; \; _{}   &  \vdots &  &  \vdots  &  e_{i} \; \bigg|  \; \; u^\perp_{i+1}  \; _{}\\ \hline
\vdots &  \vdots &  &  \vdots  & & \ddots\\ 
\end{array}
\egroup
\right)\, ,\end{equation}
in which $(e_{i})_{1\leq i \leq d-1}$ denotes the canonical basis of $\mathbb{R}^{d-1}$. We now perform a change of basis on each horizontal block line, namely we use the bases $(\harp{u_{k}},u_{k}^\perp)$ in the block line in which $u_{k}^\perp$ appears and obtain a matrix $L'_{i}$, which has the same determinant as $L_{i}$ and which starts as follows
\begin{equation}
\left(
\bgroup
\arrayrulecolor{black!50}\arrayrulewidth 1.7pt
\begin{array}{ c | c  | c   }
 \langle e_{1},\harp{u_1} \rangle   \hspace{1.1pt}  \; \; \;  0 \, \,  \; \; _{}  &  \langle e_{2},\harp{u_1} \rangle   \hspace{1.1pt}  \; \; \;  0 \, \,  \; \; _{} & \cdots  \\ 
 (*)  \hspace{1.1pt}   \; \; \; \, \, I_{d-2}     &  (*)  \hspace{1.1pt}   \; \; \; \, \, 0_{d-2}     & \cdots \\ \hline  %
  \langle e_{1},\harp{u_2} \rangle   \hspace{1.1pt}  \; \; \;  0 \, \,  \; \; _{}  &    \langle e_{2},\harp{u_2} \rangle   \hspace{1.1pt}  \; \; \;  0 \, \,  \; \; _{}&  \cdots \\  
  (*)  \hspace{1.1pt}   \; \; \; \, \, 0_{d-2}      &  (*)  \hspace{1.1pt}   \; \; \; \, \, I_{d-2}    &  \cdots \\  \hline
  \vdots &   \vdots &  \ddots \\
\end{array}
\egroup
\right)\, .\end{equation}

Hence $\text{det}\left(L_{i}\right)=\text{det}\left(\left(\langle e_{k},\harp{u_l} \rangle\right)_{\substack{1\leq k \leq d-1 \\ 
1\leq l \leq d \\ l \neq i}}\right)$, which is nothing but the determinant of the family $\left(\harp{u_1}, \ldots,\widehat{\overset{}{\harp{u_i}}}\ldots, \harp{u_d}\right)$. Performing the change of variables  in Eq.~\ref{eq.msf} we obtain that 
$$
\mathbb{E}\left[\sum_{\substack{v\, \ST \, \text{vertex}\\ \text{of}\;  \mathcal{C}_d}}f(v) \right]  =  \frac{(\mathbf{c} s)^{d}}{d!} \int_{\left(\mathbb{R}^{d-1}\right)^{d}\times \left(\mathbb{R}_{+}\right)^{d}} 
f\left(\harp{x},z\right)\mathrm{e}^{-\frac{s}{z^{d-1}}} z \prod_{i=1}^{d} \left(\frac{\mathbf{1}_{\rho_{i}>z}}{\rho_{i}^{2d-1}} \left(\rho_{i}^{2}-z^{2} \right)^{\frac{d}{2}-1} \right)
$$

$_{}$ \hfill $ _{}\displaystyle \hfill\left|\sum_{i=1}^{d}(-1)^i\frac{{\rm det}\left(\harp{u_1}, \ldots,\widehat{\overset{}{\harp{u_i}}}\ldots, \harp{u_d}\right)}{\sqrt{\rho_{i}^{2}-z^{2}}} \right| \mathrm{d}\harp{x} \mathrm{d}z \mathrm{d}\harp{u_{i}} \mathrm{d}\rho_{i}\, $.\\

\noindent
We finally observe that the sum $\left|\sum_{i=1}^{d}(-1)^i\frac{{\rm det}\left(\harp{u_1}, \ldots,\widehat{\overset{}{\harp{u_i}}}\ldots, \harp{u_d}\right)}{\sqrt{\rho_{i}^{2}-z^{2}}}\right|$ is nothing but the volume of the $d-1$ dimensional Euclidean simplex generated by 
$\left(\frac{ u_{i}}{\sqrt{\rho_{i}^{2}-z^{2}}} \right)_{1\leq i\leq d}$.
Performing the change of variables $v_{i}=\frac{z^{2}}{\rho^{2}_{i}}$ gives (\emph{i}). Averaging wrt to $\mathrm{e}^{-s}\mathbf{1}_{s\geq 0}\,  \mathrm{d}s$ gives (\emph{ii}).

\end{proof}

As a corollary of the two previous propositions we get the following results about the mean length of edges in dimension $2$ and mean area of faces in dimension $3$ in the stationary model $\mathcal{M}_{d}$.

\begin{cor}[\textsc{Mean length in $d=2$ and mean area in $d=3$}]$_{}$\\\label{cor.mlma}
 For all $s >0$, conditional on $R_{1}=s$:
\begin{enumerate}[label=(\roman*)]
\item when $d=2$, the mean length of an edge is equal to $\frac{4}{3}$;
\item when $d=3$ the mean area of a face is equal to $\frac{\pi^{3}}{12 \nu_{3}}=0.928(3)$.  
\end{enumerate}
\end{cor}

 \emph{Proof of (i)}. By ergodicity, the length of the boundary between two vertical lines $x=0$ and $x=L$ is a.s.\ asymptotically, as $L\to \infty$, equal to $L  \times \mathbb{E}\left[\frac{1}{\mathcal{H} \sin{\Theta}} \right]$, which, by Proposition \ref{prop.hscpl}, is
 $$
L \; \mathbb{E}\left[\frac{1}{\mathcal{H} \sin{\Theta}} \right] = L \; \mathbb{E}\left[\frac{1}{\mathcal{H}} \right] \mathbb{E}\left[\frac{1}{\sin{\Theta}} \right] = L \; \frac{4}{\pi s} \; .
 $$
 By Propostion~\ref{prop.vertint}, the vertex intensity in dimension $2$ is equal to $\frac{s^{2}}{2\pi} \frac{\mathrm{e}^{-\frac{s}{z}}}{z^{4}} \mathrm{d}x \, \mathrm{d}z$, so that the number of vertices between these two lines is a.s.\ asymptotically equal to 
 $$
 L \int_{0}^{\infty}\frac{s^{2}}{2\pi} \frac{\mathrm{e}^{-\frac{s}{z}}}{z^{4}}  \, \mathrm{d}z = L \, \frac{3}{\pi s}\; .
 $$
 Taking the ratio of these two quantities gives \emph{(i)}.

 \emph{Proof of (ii)}. By ergodicity, the total area of the boundary inside the cylinder $[0,L]^2\times{\mathbb R}_{+}$ is a.s.\ asymptotically, as $L\to \infty$, equal to $L^{2}  \times \mathbb{E}\left[\frac{1}{\mathcal{H}^{2} \sin{\Theta}} \right]$, which, by Proposition \ref{prop.hscpl}, is
 \begin{equation}\label{eq.area3}
L^{2} \; \mathbb{E}\left[\frac{1}{\mathcal{H} ^{2}\sin{\Theta}} \right] = L ^{2}\; \mathbb{E}\left[\frac{1}{\mathcal{H}^{2}} \right] \mathbb{E}\left[\frac{1}{\sin{\Theta}} \right] = L^{2} \; \frac{4}{3}\cdot \frac{1}{s} .
 \end{equation}
  By Proposition~\ref{prop.vertint}, the vertex intensity in $d=3$ is equal to $\frac{32\, \nu_{3}}{3\pi^{3}}\, s^{3}\,  \frac{\mathrm{e}^{-\frac{s}{z^{2}}}}{z^{9}} \mathrm{d}\harp{x}  \,\mathrm{d}z$, so that the number of vertices inside this cylinder is a.s.\ asymptotically equal to
  \begin{equation}\label{eq.nv3}
  L^{2}\,  \frac{32\, \nu_{3} }{3\pi^{3}} \, s^{3}\,\int_{0}^{\infty}\frac{\mathrm{e}^{-\frac{s}{z^{2}}}}{z^{9}}  \,\mathrm{d}z = L^{2}\, \frac{32 \nu_{3}}{\pi^{3}}\cdot \frac{1}{s}\; .
 \end{equation}
  Finally, using the Euler formula for the skeleton of edges of the boundary inside the cylinder $[0,L]^2\times{\mathbb R}_{+}$, and denoting by $F_{L}, E_{L}$ and $V_{L}$ the number of faces, edges and vertices of this skeleton, we obtain
  $$
  F_{L}-E_{L}+V_{L}= o\left(L^{2}\right)\; .
  $$
  \noindent
  Moreover, since the skeleton is $3$-regular (except possibly near the boundary), we also have 
  $$
  E_{L}=\frac{3}{2} V_{L} +o\left(L^{2}\right) \; .
  $$
  Hence we get from the two previous relations that 
  $$
  F_{L}= \frac{1}{2} V_{L}+o\left(L^{2}\right)\; .
  $$
  Therefore, using \ref{eq.nv3}, we get $F_{L}=L^{2}\, \frac{16 \nu_{3}}{\pi^{3}}\cdot \frac{1}{s}+o\left(L^{2}\right)$.
Dividing \ref{eq.area3} by this quantity, we get \emph{(ii)}. \hfill \qed

\noindent
\begin{remark}[Isokawa, again] 
 Extending the proof of the previous proposition we see that the area above the cylinder $[0,L]\times{\mathbb R}_{+}$ in dimension $2$ scales as  $L \cdot \mathbb{E}\left[ \frac{1}{ \mathcal{H}} \ST  R_{1}=s\right]= \frac{L}{s}$ whereas the length of the boundary inside the same cylinder scales as $\frac{4}{\pi} \frac{L}{s} \; $ as we see in the proof of Corollary~\ref{cor.mlma}. Therefore the ratio length/area in large parts of the cell of the origin converges to $ \frac{4}{\pi}$, hence recovering, again, the results of Isokawa~\cite{isokawa2000H2} in the low intensity limit for $d=2$. \end{remark}
\bibliographystyle{alpha}

\bibliography{biblio}

\end{document}
