%\documentclass[10pt,journal,compsoc]{IEEEtran}
\documentclass[journal, compsoc]{IEEEtran}

%\ifCLASSOPTIONcompsoc
%  \usepackage[nocompress]{cite}
%\else
%  \usepackage{cite}
%\fi

% *** GRAPHICS RELATED PACKAGES ***
%\ifCLASSINFOpdf
%\else
%\fi

% Recommended, but optional, packages for figures and better typesetting:
%\usepackage{times}
%\usepackage[utf8]{inputenc} % allow utf-8 input
%\usepackage[T1]{fontenc}    % use 8-bit T1 fonts
\usepackage{url}            % simple URL typesetting
%\usepackage{hyperref}
\usepackage{amsmath,amsfonts,amssymb}       % blackboard math 
%\usepackage{nicefrac}       % compact symbols for 1/2, etc.
\usepackage{multirow}
\usepackage{algorithm,algorithmic}
\usepackage{xcolor}
\usepackage[normalem]{ulem}
\usepackage{arydshln}
\usepackage{caption}
\usepackage{enumitem}
\usepackage{wrapfig}
\usepackage{microtype}
\usepackage{graphicx}
\usepackage{subfigure}
\usepackage{booktabs} % for professional tables
\usepackage{cutwin}
\usepackage{bbm}
\usepackage[pangram]{blindtext}
\usepackage[calc]{adjustbox}

\usepackage{tabularx}
\usepackage{color, colortbl}
\usepackage{kotex}
\usepackage{lipsum}

\usepackage{enumitem} % leftmargin for items

% USER DEFINED
\makeatletter
\def\hlinewd#1{%
	\noalign{\ifnum0=`}\fi\hrule \@height #1 %
	\futurelet\reserved@a\@xhline}
\makeatother

\def\AlgoSize{\small}
\renewcommand{\algorithmiccomment}[1]{\bgroup\hfill$\triangleright$~#1\egroup}

\definecolor{crimson}{rgb}{0.86, 0.08, 0.24}
\definecolor{orange-red}{rgb}{1.0, 0.27, 0.0}
\newcommand{\highlight}[1]{{\color{crimson}{#1}}}
\newcommand{\modified}[1]{{\color{orange-red}{#1}}}
\newcommand{\bsy}{\boldsymbol}

\DeclareMathOperator*{\argmax}{argmax}
\DeclareMathOperator*{\argmin}{argmin}

\newcommand{\newor}{%
  \mathbin{%
    {\vee}\mspace{-2.9mu}
  }%
}
\usepackage{bm}
\newcommand{\bs}[1] {\bm{#1}}

\usepackage{svg}
\DeclareMathOperator*{\minimize}{minimize}
\DeclareMathOperator*{\logicalor}{\vee}
\DeclareMathOperator*{\threshold}{thresh}

\usepackage{gensymb}
\usepackage{marvosym}

% define lines
\usepackage{mathtools,tikz}
\DeclareRobustCommand\sampleline[1]{%
  \tikz\draw[#1] (0,0) (0,\the\dimexpr\fontdimen22\textfont2\relax)
  -- (2em,\the\dimexpr\fontdimen22\textfont2\relax);%
}

\definecolor{Red}{rgb}{1,0,0}
\definecolor{Blue}{rgb}{0,0,0.8}
\definecolor{Green}{rgb}{0,0.7,0.2}
\definecolor{airforceblue}{rgb}{0.36, 0.54, 0.66}
\definecolor{ao(english)}{rgb}{0.0, 0.5, 0.0}
\definecolor{azure(colorwheel)}{rgb}{0.0, 0.5, 1.0}
\definecolor{crimson}{rgb}{0.86, 0.08, 0.24}
\definecolor{darkcerulean}{rgb}{0.03, 0.27, 0.49}
\definecolor{cobalt}{rgb}{0.0, 0.28, 0.67}
\definecolor{rosegold}{rgb}{0.72, 0.43, 0.47}
\definecolor{orange-red}{rgb}{1.0, 0.27, 0.0}
\definecolor{mountainmeadow}{rgb}{0.19, 0.73, 0.56}
\definecolor{malachite}{rgb}{0.04, 0.85, 0.32}
\definecolor{darkblue}{rgb}{0.0, 0.0, 0.55}

\definecolor{customblue}{rgb}{0.2, 0.35, 0.8}
\definecolor{gg}{gray}{0.9}

%\usepackage{hyperref}
%\hypersetup{colorlinks=true}
%\hypersetup{linktoc=all}
%\hypersetup{citecolor=customblue}
%\hypersetup{linkcolor=crimson}
%\hypersetup{urlcolor=MidnightBlue}
%\usepackage[all]{hypcap}

%\usepackage[nameinlink]{cleveref}
%\creflabelformat{equation}{#2\textup{#1}#3}  
%\crefname{assumption}{assumption}{assumptions}
%\usepackage[pagebackref,breaklinks,colorlinks]{hyperref}

% Support for easy cross-referencing
\usepackage[capitalize]{cleveref}
\crefname{section}{Sec.}{Secs.}
\Crefname{section}{Section}{Sections}
\Crefname{table}{Table}{Tables}
\crefname{table}{Tab.}{Tabs.}


\usepackage{makecell}
\newcommand{\tobemorified}[1]{{\color{crimson}{#1}}}
\newcommand{\hibf}[1]{{\color{crimson}{\textbf{#1}}}}
\newcommand{\jss}[1]{{\color{azure(colorwheel)}{\sout{#1}}}}
\newcommand{\js}[1]{{\color{azure(colorwheel)}{#1}}} %ao(english)
\newcommand{\eat}[1]{{}}
\creflabelformat{equation}{#2\textup{#1}#3}  

% correct bad hyphenation here
%\hyphenation{op-tical net-works semi-conduc-tor}

%\usepackage{hyperref}
%\usepackage{url}

\begin{document}
%
% paper title
% Titles are generally capitalized except for words such as a, an, and, as,
% at, but, by, for, in, nor, of, on, or, the, to and up, which are usually
% not capitalized unless they are the first or last word of the title.
% Linebreaks \\ can be used within to get better formatting as desired.
% Do not put math or special symbols in the title.
%\title{Bare Demo of IEEEtran.cls for\\ IEEE Computer Society Journals}
\title{Forget-free Continual Learning with Soft-Winning SubNetworks}

\author{Haeyong~Kang\thanks{Email: haeyong.kang@kaist.ac.kr},
        Jaehong~Yoon, %~\IEEEmembership{Member,~IEEE,}
        Sultan Rizky Madjid, 
        Sung Ju Hwang,
        and~Chang~D.~Yoo$^{\ast}$\thanks{$^\ast$ Corresponding Author.} %~\IEEEmembership{Life~Fellow,~IEEE}% <-this % stops a space
%\IEEEcompsocitemizethanks{\IEEEcompsocthanksitem Haeyong Kang was with the School of Electrical Engineering, KAIST, Daejeon, Korea. \protect\\
% note need leading \protect in front of \\ to get a newline within \thanks as
% \\ is fragile and will error, could use \hfil\break instead.
%E-mail: haeyong.kang@kaist.ac.kr
%\IEEEcompsocthanksitem J. Doe and J. Doe are with Anonymous University.}% <-this % stops an unwanted space
%\thanks{Manuscript received April 19, 2023; revised August 26, 2023.}
}

% The paper headers
%\markboth{Journal of \LaTeX\ Class Files,~Vol.~**, No.~*, March~2023}%
\markboth{Preprint, March~2023}
{Shell \MakeLowercase{\textit{et al.}}: Bare Demo of IEEEtran.cls for Computer Society Journals}

% in the abstract or keywords.
\IEEEtitleabstractindextext{%
%%The level of delegation to be granted to AI systems will in particular heavily depend on how methodological research replies to questions of robustness. This naturally brings us back to the development of statistical learning techniques that are reliable even in presence of partly contaminated data, due to biases in measurements or the deliberate intention to impair the operation of the automated system. Preference data, observed in the form of (complete) rankings in the simplest situations, are no exception of course and the demand for appropriate concepts and tools is all the more pressing given that technologies fed by or producing this type of data (\textit{e.g.} search engines, recommending systems) are now massively deployed. The lack of vector space structure for the set of rankings (\textit{i.e.} the symmetric group $\mathfrak{S}_n$) and the very complex nature of statistics usually considered in ranking data analysis make the formulation of robustness objectives in this domain extremely challenging. In this paper, we introduce notions of robustness, together with dedicated statistical methods, for \textit{Consensus Ranking}, the flagship problem in ranking data analysis, aiming at summarizing a probability distribution on $\mathfrak{S}_n$ by a \textit{median} ranking. Precisely, we propose specific extensions of the popular concept of \textit{breakdown point}, tailored to consensus ranking, and address the related computational issues. Beyond the theoretical contributions, the relevance of the approach proposed is supported by a detailed experimental study.

As the issue of robustness in AI systems becomes vital, statistical learning techniques that are reliable even in presence of partly contaminated data have to be developed. Preference data, in the form of (complete) rankings in the simplest situations, are no exception and the demand for appropriate concepts and tools is all the more pressing given that technologies fed by or producing this type of data (\textit{e.g.} search engines, recommending systems) are now massively deployed. However, the lack of vector space structure for the set of rankings (\textit{i.e.} the symmetric group $\mathfrak{S}_n$) and the complex nature of statistics considered in ranking data analysis make the formulation of robustness objectives in this domain challenging. In this paper, we introduce notions of robustness, together with dedicated statistical methods, for \textit{Consensus Ranking} the flagship problem in ranking data analysis, aiming at summarizing a probability distribution on $\mathfrak{S}_n$ by a \textit{median} ranking. Precisely, we propose specific extensions of the popular concept of \textit{breakdown point}, tailored to consensus ranking, and address the related computational issues. Beyond the theoretical contributions, the relevance of the approach proposed is supported by an experimental study.

\begin{abstract}%v 
Inspired by \emph{Regularized Lottery Ticket Hypothesis (RLTH)}, which states that competitive smooth (non-binary) subnetworks exist within a dense network in continual learning tasks, we investigate two proposed architecture-based continual learning methods which sequentially learn and select adaptive binary- (WSN) and non-binary Soft-Subnetworks (SoftNet) for each task. WSN and SoftNet jointly learn the regularized model weights and task-adaptive non-binary masks of subnetworks associated with each task whilst attempting to select a small set of weights to be activated (winning ticket) by reusing weights of the prior subnetworks. Our proposed WSN and SoftNet are inherently immune to catastrophic forgetting as each selected subnetwork model does not infringe upon other subnetworks in Task Incremental Learning (TIL). In TIL, binary masks spawned per winning ticket are encoded into one N-bit binary digit mask, then compressed using Huffman coding for a sub-linear increase in network capacity to the number of tasks. Surprisingly, in the inference step, SoftNet generated by injecting small noises to the backgrounds of acquired WSN (holding the foregrounds of WSN) provides excellent forward transfer power for future tasks in TIL. SoftNet shows its effectiveness over WSN in regularizing parameters to tackle the overfitting, to a few examples in Few-shot Class Incremental Learning (FSCIL). 
\end{abstract}

% Note that keywords are not normally used for peerreview papers.
\begin{IEEEkeywords}
Continual Learning (CL), Task Incremental Learning (TIL), Few-shot Class Incremental Learning (FSCIL), Regularized Lottery Ticket Hypothesis (RLTH), Wining SubNetworks (WSN), Soft-Subnetwork (SoftNet)
\end{IEEEkeywords}}

% make the title area
\maketitle

\IEEEdisplaynontitleabstractindextext
% \IEEEdisplaynontitleabstractindextext has no effect when using
% compsoc or transmag under a non-conference mode.

% For peerreview papers, this IEEEtran command inserts a page break and
% creates the second title. It will be ignored for other modes.
\IEEEpeerreviewmaketitle

\section{Introduction}
Creating high-quality 3D content has traditionally been a time-consuming process, requiring specialized skills and knowledge. However, recent advances in text-to-3D synthesis~\cite{poole2022dreamfusion,lin2022magic3d,metzer2022latent, wang2022score} are revolutionizing the generation of 3D scenes. These methods use pretrained text-to-image diffusion models to optimize a Neural Radiance Field (NeRF) and generate 3D objects that match a given text prompt.

In spite of these exciting advancements, current solutions still lack the capability to create a specific envisioned scene. That is because controlling the generation of text prompts alone is extremely challenging, especially if one wants to describe specific objects with defined dimensions and geometry, and locate them at specific positions. 
Moreover, once generated, modifying specific aspects of the scene, while leaving the others untouched, can prove to be challenging. Various aspects of a scene may need to be modified after generation, such as the position or orientation of certain objects, or the texture or geometry of individual components. 
One may also want to export a single object to be used in another scene. 
However, current methods represent the scene as a whole, and objects are interdependent in their representation, making it impossible to edit a specific scene component or use objects in other scenes.


\begin{figure}
    \centering
    \setlength{\tabcolsep}{0pt}
    {\small
    \begin{tabular}{c@{\hskip 0.1cm}c c c}
    \raisebox{0.07\textwidth}{\rotatebox[origin=t]{90}{Global-Local}} & 
        \includegraphics[width=0.31\linewidth]{figures/color/images/living_source.jpg} & 
        \includegraphics[width=0.31\linewidth]{figures/color/images/living_red_sofas.jpg} & 
        \includegraphics[width=0.31\linewidth]{figures/color/images/living_green.jpg} \\ 
        \raisebox{0.07\textwidth}{\rotatebox[origin=t]{90}{Local Only}} & 
        \includegraphics[width=0.31\linewidth]{figures/color/images/living_source.jpg} &
        \includegraphics[width=0.31\linewidth]{figures/color/images/living_red_no_shared.jpg} & 
        \includegraphics[width=0.31\linewidth]{figures/color/images/living_green_no_shared.jpg} \\ 
        & Input Scene &  Edited Sofas  &  Edited Sofas+Table \\

    \end{tabular}}
    \caption{{\bf Color editing.} Using a dedicated albedo head, we can fine-tune the NeRFs to change the color scheme while keeping the geometry and other objects unchanged. For each edited object the object text prompt was changed according to the desired color, for example the prompt ``a baroque sofa`' was changed to '``a red baroque sofa''.}
    \label{fig:edit_color}
\end{figure} 


 Given that the generation process of scenes using current text-to-3D methods takes a considerable amount of time, the need for interactive editing capabilities has become even more apparent.
One may potentially save time and gain control over individual objects by generating them separately and then just rendering them together at inference time. However, such an approach has several limitations: it cannot generate objects that interact with each other; it cannot ensure consistency in style between objects; and it cannot model the interaction among objects, like shadows and other global shading effects, see Figure~\ref{fig:ablation_teaser} (A). 

In this paper, we introduce a novel framework for synthesizing a controllable scene using text and object proxies, using a \textbf{Global-Local} approach. The key idea is to represent the scene as a composition of multiple object NeRFs, each built around an object proxy. The models are jointly optimized to ``locally'' represent the required object and ``globally'' to be part of the larger scene. Both the local and the global optimization propagate gradients into the same models, creating a harmonious scene composed of disentangled objects, see Figure~\ref{fig:ablation_teaser} (B).
For optimizing our objects and scenes, we follow~\cite{poole2022dreamfusion} and use the score distillation loss. Our method leverages the composability of our representation and iteratively alternates between localized training of individual objects and optimizing the scene as a whole, where objects are dependent on their representation. When optimizing a single NeRF, we simply render it on its own from a random viewpoint and apply score distillation based on
a text prompt describing the object.  For scene-level optimization, we shift the rays using a rigid transformation to match the desired placement defined by each object proxy and apply score distillation with a ``scene text prompt''. 


In many scenarios it is desirable to not only define the placement of an object, but also its dimensions and coarse geometry. Therefore, we also optionally apply a shape loss~\cite{metzer2022latent} on each object proxy to guide it towards a specific shape. 
In addition, we demonstrate that our approach enables the definition of multiple object proxies that can be linked to a single object NeRF. This permits the specification of replicated objects that are intended to be located in multiple positions throughout the scene (e.g., chairs around a table), while aggregating the score distillation from the different placements to optimize a single NeRF

Our proposed Global-Local approach not only provides more control during the training process, but also allows for better editing and fine-tuning of generated scenes. Specifically, using object proxies, we can easily control the placement of an object without the need for further refinement and even remove or duplicate the object as desired. Additionally, we can selectively fine-tune only parts of the scene by defining the set of proxies that are trained and fine-tuning the respective NeRF with modified %
text prompts. Furthermore, we demonstrate that the object proxy can be used to define geometry edits on the coarse shape, which are then applied during the fine-tuning process.

The contributions of our paper are threefold: (i) we propose to represent each object in the scene as a separate NeRF around a proxy, which allows getting a disentangled model for each object, (ii) we introduce a new optimization strategy that interleaves between single-object optimization and scene optimization, resulting in self-contained objects that can be combined to create a plausible scene, and (iii) our strategy provides control over the generated scene, both before and after its creation.



\section{Related Work}
\label{section:related_work}

\import{tables/}{sota_comparison}

There have been many work focusing on fault injection for \titlecaseabbreviationpl{dnn}, focusing on the tools required for fault injection as well as the different sampling models. We show a comparison in Table \ref{table:sota_comparison}.
enpheeph \cite{colucciEnpheephFaultInjection2022a}, TensorFI \cite{liTensorFIConfigurableFault2018} and LLTFI \cite{agarwalLLTFIFrameworkAgnostic2022} focus on providing innovative fault injection frameworks to speed up the overhead of running fault injection compared to normal \titlecaseabbreviation{dnn} execution. However, they do not employ any specific algorithm for sampling, resorting to random uniform sampling. At the same time, they are able to cover the whole search space, as they do not filter the possible faults.
On the other hand, BinFI \cite{chenBinFIEfficientFault2019} and AVFI \cite{jhaMLBasedFaultInjection2019} provide improved fault models, not focusing on how the fault injection is executed as the previous works, but how to choose the faults using external knowledge, in their case human knowledge. However, they still employ random uniform sampling, but as they decrease the extension of the fault search space, they reach higher precision than the previous tools.
\emphasizedworkname{} employs a novel sampling method based on importance sampling, hence it is capable of achieving similar precision while still requiring no human knowledge and without reducing the fault search space.

%\section{APPROACH}
\section{Approach}
\label{sec:approach}

The proposed SC-VAE model aims to encode an image into a series of
latent vector representations and then to utilize sparse coding to generate sparse code vectors for these representations. These sparse code vectors can be subsequently decoded back to the reconstructed image with a fixed dictionary and a decoder network. The diagram of the proposed model is shown in Figure \ref{figure:3}. We discuss the model formulation in Section \ref{Model Formulation} and the loss functions in Section \ref{Loss Functions}.

\subsection{Model Formulation} \label{Model Formulation}
 
Formally, the input of SC-VAE is an image $x \in \mathbb{R}^{H \times W \times C}$, where $H$, $W$ and $C$ denote the height, width and the number of image channels, respectively. The image
$x$ goes first through an encoder $E$ to obtain 
 latent representations $E(x) \in \mathbb{R}^{h \times w \times n}$.
Here, the values of $h$ and $w$ depend on the number of downsampling blocks $d$ in the encoder. Accordingly, these are defined as follows $h=\frac{H}{2^d}$ and $w = \frac{W}{2^d}$. $n$ denotes the number of dimensions of each latent representation $E_{ij}(x)$, where $i \in [1, h]$ and $j\in [1, w]$. $E_{ij}^{\top}(x) \in \mathbb{R}^{n \times 1}$ is then given as an input to a Learnable ISTA network. The Learnable ISTA produces the sparse code vector $Z_{ij}\in \mathbb{R}^{1\times K}$ for each $E_{ij}(x)$ by using the learnable parameters $W_e$, $S$ and $\theta$. 
Here, $K$ denotes the number of atoms in the  predetermined orthogonal dictionary $\mathbf{D}$. Each reconstructed latent representation $\tilde{E}_{ij}(x)$ can be calculated by the multiplication of $Z_{ij}$ and $\mathbf{D}^{\top}$, which is then used to reconstruct the original image by going through a decoder neural network $G$.
We denote the output of SC-VAE as $G(\tilde{E}(x))$.
%The design of our convolutional encoder $E$ and decoder $G$ follows the architecture in \cite{esser2021taming}. 

\subsection{Loss Functions} \label{Loss Functions}
We need
to define loss functions at two levels: the image level and the latent representation level. The loss in the image level should encourage our model to provide a good reconstruction for the input image. The loss in the latent space should  allow us not only to obtain good latent representation reconstruction, but also to learn sparse codes. 
%which can reconstruct the latent representations well.


\noindent\textbf{Image reconstruction.} The most common image reconstruction term utilized in VAE models is the $L2$ loss. The $L2$ loss is defined as
\begin{align} \label{eq:4}
    \mathcal{L}_{rec}  &= ||G(\tilde{E}(x))-x||_2^2.
\end{align}


\noindent
\textbf{Latent representation reconstruction.} 
We aim to learn how to reconstruct each latent representation $E_{ij}(x)$ based on a linear combination of atoms in the fixed orthogonal dictionary $\mathbf{D}$.
Accordingly, the loss function for the latent representation reconstruction is given by:
\begin{align} \label{eq:4}
    \mathcal{L}_{latent}  =\sum_{ij}(||E_{ij}^{\top}(x) -\mathbf{D}Z_{ij}^{\top}||_2^2  + \alpha||Z_{ij}^{\top}||_1).
\end{align}
Similar to Eq. (\ref{eq:1}), this loss consists of two terms. The first term is a $L2$ norm to penalize differences between the latent representations of input images and their latent representation reconstructions. The second term imposes sparsity to each latent sparse code vector $Z_{ij}$. $\alpha$ controls the sparseness of the learned sparse code vectors $Z$.

\noindent\textbf{Total loss.} An intuitive way to build the total loss function would be to simply add $\mathcal{L}_{rec}$ and $\mathcal{L}_{latent}$.
However, this loss function will not allow us to learn a good image reconstruction due to the summation term in $\mathcal{L}_{latent}$.
This is because each input image corresponds to $h\times w$ latent representations. As a consequence, the model will focus on learning good sparse code vectors for these latent representations and pay less attention on  adequately optimizing $\mathcal{L}_{rec}$. 
To account for this factor, we introduce coefficients $\frac{1}{hw}$ to each of the latent representation $E_{ij}(x)$, which allows for appropriately balancing the two terms.
Thus, the total loss for our model is the following:
\begin{align} \label{eq:6}
    \mathcal{L}_{SC-VAE}(E,G,W_e, S, \theta)  =\mathcal{L}_{rec} +  \frac{1}{hw}\mathcal{L}_{latent}.
\end{align}

\section{Experiments}
\label{sec:exp}
% logic:
% 1, Experiment Settings 

% 2, Benchmark results 
    % 1, VIP-Seg
    % 2, VSPW
    % 3, KITTI-STEP

% 3, Ablation studies and analysis. 
    % 1, improvements on baseline 
    % 2, design choices of temporal contarstive loss 
    % 3, design choices of label assigin stragety
    % 4, Effect of tube frames choices for CS loss 
    % 5, Effect of large window size / overlap inference. 
    % 6, Comparison with the different tracking choices. 
    % 7, increased GFLops/Parameters analysis. 
    % 8. FPS/Window Cruves.

% 4, visualization results. 
    % 1, comparison with strong baseline. 
    % 2, attention mask arcoss differnt tube. 
    

% Due to the unavailability of the test set, we report the results on the \textit{validation set}. 

% The former mainly focuses on mask proposal level as PQ~\cite{kirillov2019panoptic} with different window sizes, while the latter emphasizes pixel-level segmentation and tracking without any thresholds.
% KITTI-STEP has 21 and 29 sequences for training and testing, respectively. The training sequences are split into a training set (12 sequences) and a validation set (9 sequences).

\subsection{Experimental Settings}
\noindent
\textbf{Dataset.} We conduct experiments on five video datasets: VIPSeg~\cite{miao2022large}, VSPW~\cite{miao2021vspw}, KITTI-STEP~\cite{STEP}, and YouTube-VIS-19/21~\cite{vis_dataset}. We mainly conduct experiments on VIPSeg due to its scene diversity and long-length clips. The training, validation, and test sets of VIPSeg contain 2,806/343/387 videos with 66,767/8,255/9,728 frames, respectively. Although VSPW and VIPSeg share the same video clips, the training details are different since they are different tasks. Please refer to the \textit{supplementary material} for other datasets.


\noindent
\textbf{Evaluation Metrics.} For the VPS task, we adopt two metrics: $VPQ$~\cite{kim2020vps} and $STQ$~\cite{STEP}. The metric $STQ$ contains geometric mean of two items: Segmentation Quality ($SQ$) and Association Quality ($AQ$), where $ STQ = (SQ \times AQ)^{\frac{1}{2}}$. The former evaluates the pixel-level tracking, while the latter evaluates the pixel-level segmentation results in a video clip. For the VSS task, the Mean Intersection over Union (\textit{mIoU}) and mean Video Consistency ($mVC$)~\cite{miao2021vspw} are used for reference. For the VIS task, \textit{mAP} is adopted.


\noindent
\textbf{Implementation Details and Baselines.} We implement our models in PyTorch~\cite{pytorch_paper} with the MMDetection toolbox~\cite{chen2019mmdetection}. We use the distributed training framework with 16 V100 GPUs. Each mini-batch has one image per GPU. Following previous work, we use the image baseline pre-trained on COCO dataset~\cite{coco_dataset}. ResNet~\cite{resnet}, STDC~\cite{STDCNet}, and Swin Transformer~\cite{liu2021swin} are adopted as the backbone networks, which are pre-trained on ImageNet, and the remaining layers adopt the Xavier initialization~\cite{xavier_init}. 
For the detailed settings of other datasets, pretraining, and fine-tuning, please refer to the \textit{supplementary material}. To further verify the effectiveness of our approach, we build a stronger baseline by unifying Video K-Net with Mask2Former, where we replace the image encoder with Mask2Former. We term it Video K-Net+. We denote the extended Mask2Former-VIS for VPS as Mask2Former-VIS+.


%%%%%%%% VIP-SEG %%%%%%%%%%%
\begin{table}[!t]
	\centering
	\caption{\small \textbf{Results on VIPSeg-VPS~\cite{miao2022large} validation dataset.} We report VPQ and STQ for reference. Following Miao~\etal~\cite{miao2022large}, we report VPQ scores at different window sizes (1, 2, 4, 6). We report the results obtained from either an efficient or a strong backbone for comparison.}
	\label{tab:vipseg_results}
  \scalebox{0.65}{
    \begin{tabular}{ r|c|cccccc}
    \toprule[0.15em]
     Method& backbone & $VPQ^{1}$ & $VPQ^{2}$ & $VPQ^{4}$ & $VPQ^{6}$ & VPQ & STQ \\
    \toprule[0.15em]
    VIP-DeepLab~\cite{ViPDeepLab} & ResNet50 & 18.4 & 16.9 & 14.8 & 13.7 & 16.0 & 22.0 \\
    VPSNet~\cite{kim2020vps} & ResNet50 & 19.9 & 18.1 & 15.8 & 14.5 & 17.0 & 20.8 \\
    SiamTrack~\cite{woo2021learning_associate_vps} & ResNet50 & 20.0 & 18.3 & 16.0 & 14.7 & 17.2 & 21.1 \\
    Clip-PanoFCN~\cite{miao2022large} & ResNet50 & 24.3 & 23.5 & 22.4 & 21.6 & 22.9 & 31.5 \\
    Video K-Net~\cite{li2022videoknet} & ResNet50 & 29.5 & 26.5 & 24.5 & 23.7 & 26.1 & 33.1 \\
    Video K-Net+~\cite{cheng2021mask2former,li2022videoknet} & ResNet50 & 32.1 & 30.5 & 28.5 & 26.7 & 29.1 & 36.6  \\
    Video K-Net~\cite{li2022videoknet} & Swin-base & 43.3 & 40.5 & 38.3 & 37.2 & 39.8 & 46.3 \\
    \hline
    Tube-Link & STDCv1 & 32.1 & 31.3 & 30.1 & 29.1 & 30.6 & 32.0 \\
    Tube-Link & STDCv2 & 33.2  & 31.8 & 30.6 & 29.6  &  31.4 & 32.8 \\
    \hline
    Tube-Link & ResNet50 & 41.2 & 39.5  & 38.0 & 37.0 &  39.2 & 39.5 \\
    Tube-Link & Swin-base & 54.5 & 51.4 & 48.6 & 47.1 & 50.4 & 49.4 \\
    % Tube-Link & Swin-large &  \lxt{wait results} \\
    \bottomrule[0.2em]
    \end{tabular}
}
\end{table}


%%%%%% VIS-Youtube %%%%%%%%%
\begin{table}[t]
  \centering
   \caption{\small \textbf{Results on the YouTube-VIS datasets.} We report the mAP metric. \textdagger~adopt COCO video pseudo labels. Axial means using the extra Axial Attention~\cite{axialDeeplab}. Our method does not apply these techniques for simplicity.}
  \label{tab:ytvis}
  \scalebox{0.68}{
  \begin{tabular}{l c | c  | c }
    \toprule[0.2em]
    Method & Backbone  & YTVIS-2019 & YTVIS-2021 \\
    \toprule[0.2em]
VISTR~\cite{VIS_TR} & ResNet50 & 36.2 & -  \\
TubeFormer~\cite{kim2022tubeformer} & ResNet50 + Aixal & 47.5  & 41.2  \\
IFC~\cite{hwang2021video} & ResNet50 & 42.8 & 36.6 \\
SeqFormer~\cite{seqformer} & ResNet50 & 47.4 & 40.5  \\
Mask2Former-VIS~\cite{cheng2021mask2former_vis}& ResNet50 & 46.4 & 40.6 \\
IDOL~\cite{IDOL} & ResNet50 & 46.4 & 43.9\\
IDOL~\cite{IDOL} \textdagger & ResNet50 & 49.5 & -\\
VITA~\cite{heo2022vita} \textdagger & ResNet50 & 49.8 & 45.7  \\
Min-VIS~\cite{huang2022minvis} &ResNet50& 47.4 & 44.2 \\
% GenVIS~\cite{heo2022generalized} & ResNet50 & 51.3 & 46.3 \\
\hline
Tube-Link & ResNet50 & 52.8 & 47.9  \\% & - \\
\hline
SeqFormer~\cite{seqformer} & Swin-large  & 59.3 & 51.8 \\% & - \\
Mask2Former-VIS~\cite{cheng2021mask2former_vis} & Swin-large &  60.4 & 52.6 \\
IDOL~\cite{IDOL}  & Swin-large  & 61.5 & 56.1 \\ %& 42.6\\
IDOL~\cite{IDOL}  & Swin-large \textdagger  & 64.3 & -\\
VITA~\cite{heo2022vita} \textdagger & Swin-large & 63.0 & 57.5 \\ 
Min-VIS~\cite{huang2022minvis} & Swin-large & 61.6 & 55.3 \\
\hline
Tube-Link & Swin-large  & 64.6 & 58.4  \\
    \bottomrule[0.2em]
  \end{tabular}
}
\end{table}



%%%%%% VSPW and VIP-Seg VSS%%%%%%%%%
\begin{table}[t]
  \centering
    \caption{\small \textbf{Results on VSPW-VSS validation set}. $mVC_{c}$ means that a clip with $c$ frames is used.}
    \label{tab:vspw}
  \scalebox{0.68}{
  \begin{tabular}{l c c c c c }
    \toprule[0.2em]
    \textbf{VPSW} & Backbone & mIoU & $mVC_{8}$ &$mVC_{16}$  \\
    \toprule[0.2em]
    DeepLabv3+~\cite{deeplabv3plus} & ResNet101 & 35.7 & 83.5 & 78.4 \\
    TCB(PSPNet)~\cite{miao2021vspw,zhao2017pyramid} & ResNet101 & 37.5 & 86.9 & 82.1  \\
    Video K-Net (Deeplabv3+)~\cite{li2022videoknet,deeplabv3plus} & ResNet101  & 37.9 & 87.0 & 82.1 \\
    Video K-Net (PSPNet)~\cite{li2022videoknet,zhao2017pyramid} & ResNet101  & 38.0 & 87.2  & 82.3 \\
    MRCFA~\cite{sun2022mining} & MiT-B5 & 49.9 & 90.9  &  87.4  \\
    CFFM~\cite{sun2022vss} & MiT-B5 & 49.3 & 90.8 & 87.1 \\
    TubeFormer~\cite{kim2022tubeformer} & Axial-ResNet50x64  &  63.2 &  92.1 & 88.0 \\
    \hline
    Tube-Link & ResNet50 & 42.3 & 86.8 & 83.2 \\
    Tube-Link & Swin-large & 59.7 & 90.3 & 88.4 \\
    \bottomrule[0.2em]
  \end{tabular}
  }

\end{table}


\begin{table}[t]
  \centering
    \caption{\small \textbf{Results on VIP-Seg-VSS validation set}. $mVC_{c}$ means that a clip with $c$ frames is used.}
    \label{tab:vipseg_vss}
  \scalebox{0.68}{
  \begin{tabular}{l c c c c c }
    \toprule[0.2em]
    \textbf{VPSW} & Backbone & mIoU & $mVC_{8}$ &$mVC_{16}$  \\
    \toprule[0.2em]
    Video K-Net (Deeplabv3+)~\cite{li2022videoknet,deeplabv3plus} & ResNet101  & 38.3 & 88.0 & 83.1 \\
    Video K-Net (PSPNet)~\cite{li2022videoknet,zhao2017pyramid} & ResNet101  & 39.0 & 88.2  & 84.2 \\
    Mask2Former~\cite{cheng2021mask2former} &  ResNet50 & 38.4 & 87.5 & 82.5 \\
    Video K-Net+~\cite{cheng2021mask2former,li2022videoknet} &  Swin-base & 57.2 & 90.1 & 87.8  \\
    \hline
    Tube-Link & ResNet50 & 43.4 & 89.2 & 85.4 \\
    Tube-Link & Swin-base & 62.3 & 91.4 & 89.3 \\
    Tube-Link & Swin-large & 64.9 & 92.4 & 89.9 \\
    \bottomrule[0.2em]
  \end{tabular}
  }

\end{table}


\subsection{Benchmark Results}


%%%%%% KITTI-STEP %%%%%%
\begin{table}[t]
  \centering
   \caption{\small \textbf{Results on the KITTI val set.} OF refers to an optical flow network~\cite{teed2020raft}.}
  \label{tab:kitti_step}
  \scalebox{0.68}{
  \begin{tabular}{l c c || c c c c }
    \toprule[0.2em]
    \textbf{KITTI-STEP} & Backbone & OF & STQ & AQ & SQ & VPQ \\
    \toprule[0.2em]
    P + Mask Propagation & ResNet50 & \checkmark & 0.67 & 0.63 & 0.71 & 0.44 \\
    Motion-Deeplab~\cite{STEP}& ResNet50 &  & 0.58 & 0.51 & 0.67 & 0.40  \\
    VPSNet~\cite{kim2020vps}& ResNet50  & \checkmark & 0.56 & 0.52 & 0.61 & {0.43}  \\
    TubeFormer-DeepLab~\cite{kim2022tubeformer} & ResNet-50 + Axial &  & 0.70 & 0.64 &  0.76 & 0.51 \\
    Video K-Net~\cite{li2022videoknet} & ResNet50 &  & 0.71 & 0.70  & 0.71  &  0.46 \\
    Video K-Net~\cite{li2022videoknet} & Swin-base &  & 0.73 & 0.72 & 0.73 & 0.53 \\
    \hline
    Tube-Link & ResNet50 &  & 0.68 & 0.67 & 0.69 & 0.51 \\
    Tube-Link & Swin-base &  & 0.72 & 0.69 & 0.74 & 0.56 \\
    \bottomrule[0.2em]
  \end{tabular}
  }
  \vspace{-4mm}
\end{table}

% \lxt{will be changed by test set Figure Results Further. This figure will be merged into it as subfigure.}
\begin{figure}[t]
  \centering
   \includegraphics[width=0.80\linewidth]{./figs/teaser_trade_off.pdf}
   \caption{\small Tube-Link also achieves the best accuracy and speed trade-off on VIP-Seg dataset. FPS is measured on RTX GPU.}
   \label{fig:curve_trade_off_vipseg}
\end{figure}

\noindent
\textbf{[VPS] Results on VIPSeg.} 
We present the results of our Tube-Link method compared to previous works on the VIPSeg dataset in Tab.~\ref{tab:vipseg_results}. Our approach outperforms Video K-Net\cite{li2022videoknet} (under the same backbone) with 12\%-15\% VPQ and 7\%-10\% STQ improvements, respectively. Notably, our method with Swin-base~\cite{liu2021swin} backbone achieves new state-of-the-art results. 
%
We also evaluate our method using a lightweight backbone~\cite{STDCNet} for more efficient inference on video clips, and it achieves even better results than all previous methods with a larger ResNet50 backbone. 
%
These results demonstrate the effectiveness of our approach in exploiting temporal information.  Benefiting from the joint inference of subclips, our method achieves a much faster inference speed, as shown in Fig.~\ref{fig:curve_trade_off_vipseg}. 



\begin{table*}[h!]
    \footnotesize
	\centering
	\caption{\small \textbf{Ablation studies and comparative analysis on VIPSeg validation set with the ResNet50 backbone.} 
	}
    \subfloat[Ablation Study on Each Component.]{
    \label{tab:ablation_a}
	    \begin{tabularx}{0.43\textwidth}{c c c c c} 
		        				\toprule[0.15em]
    	baseline  & TCL & CTL & $\mathrm{VPQ_{th}}$ & VPQ \\
        \toprule[0.15em]
            Mask2Former-VIS+ (F) & - & - & 29.4 & 32.4 \\
            \hline
            Mask2Former-VIS+ (T) & - & - & 31.0 & 34.5\\
             & \checkmark & - & 34.6  & 36.8  \\  
          \rowcolor{gray!15}  & \checkmark & \checkmark & 35.1 & 37.5 \\  
        \bottomrule[0.1em]
	    \end{tabularx}
    } \hfill
    \subfloat[Design Choices of TCL.]{
    \label{tab:ablation_b}
		\begin{tabularx}{0.28\textwidth}{c c c} 
			\toprule[0.15em]
			Method & VPQ & STQ \\
			\midrule[0.15em]
            Dense Query~\cite{qdtrack} & 30.2  & 30.1  \\
            Sparse Query~\cite{li2022videoknet} & 34.5  & 35.1 \\
            \rowcolor{gray!15} Global Query(Ours) &  37.5  & 36.5 \\
			\bottomrule[0.1em]
		\end{tabularx}
    } \hfill
    \subfloat[Association Target Assign.]{
    \label{tab:ablation_c}
		\begin{tabularx}{0.24\textwidth}{c c c} 
			\toprule[0.15em]
			Method & VPQ & STQ  \\
			\midrule[0.15em]
			All-Masks~\cite{qdtrack} & 30.1 & 29.2 \\
			GT-Mask~\cite{li2022videoknet} & 35.6 & 35.9 \\
			\rowcolor{gray!15} Tube-Mask & 37.5 & 36.5 \\
			\bottomrule[0.1em]
		\end{tabularx}
    } \hfill
    \vspace{2mm}
    \subfloat[Input Sub-clip Size with Tube Window Size of 2 as Input.]{
     \label{tab:ablation_d}
	    \begin{tabularx}{0.30\textwidth}{c c c c} 
		        				\toprule[0.15em]
    		 Clip Size & STQ & VPQ & $\mathrm{VPQ_{th}}$  \\
    		\toprule[0.15em]
    	    T=1 & 34.5 & 35.6 & 30.2 \\
    	    \rowcolor{gray!15} T=2 & 36.5 & 37.5 & 35.1 \\
    	    T=2(ovl) & 35.9 & 37.3 & 35.0 \\
    	    T=3 &  36.4 & 37.0 & 35.3 \\
        	\bottomrule[0.1em]
	    \end{tabularx}
    } \hfill
    \subfloat[Tube-Window for Inference with Input Sub-clip Size 2 for Training.]{
     \label{tab:ablation_e}
	    \begin{tabularx}{0.30\textwidth}{c  c c c} 
		        				\toprule[0.15em]
    		 Window Size & STQ & VPQ  & $\mathrm{VPQ_{th}}$ \\
    		\toprule[0.15em]
    	    W=2 &  36.5 & 37.5 & 35.1 \\
    	    W=4 &  39.2 & 39.0 & 38.2 \\
    	   \rowcolor{gray!15} W=6 &  39.5 & 39.2 & 38.9 \\
    	    W=8 &  38.3 & 38.5 & 37.3 \\
        	\bottomrule[0.1em]
	    \end{tabularx}
    } \hfill
    \subfloat[Tracking Choices with the Default Setting of Tab.(d). ]{
     \label{tab:ablation_f}
	    \begin{tabularx}{0.35\textwidth}{c c c c} 
		        				\toprule[0.15em]
    		 Settings  &  STQ & VPQ & $\mathrm{VPQ_{th}}$ \\
    		 \toprule[0.15em]
    		  Extra Tracker~\cite{wangUnitrack,deepsort}& 33.9 & 36.6 & 34.1 \\
    		  RoI Features~\cite{qdtrack} & 34.5 & 35.9 & 34.5 \\
    		  Query Embedding~\cite{li2022videoknet}  & 33.1  & 36.0  & 33.0 \\
    	     \rowcolor{gray!15} Our Tube embedding & 36.5 & 37.5 & 35.1\\
        	\bottomrule[0.1em]
	    \end{tabularx}
    } \hfill
\end{table*}


\noindent
\textbf{[VIS] Results on YouTube-VIS-2019/2021.} In Tab.~\ref{tab:ytvis}, we compare our method with state-of-the-art VIS methods on the YouTube-VIS 2019 and 2021 datasets. Our method achieves a 3.0\% and 2.2\% mAP gain over VITA~\cite{heo2022vita} when using the ResNet50 backbone. Furthermore, compared with the Mask2Former-VIS baseline~\cite{cheng2021mask2former_vis}, our method achieves 4-5\% mAP gains on the two datasets with different backbones. Our method also outperforms the previous near-online method TubeFormer~\cite{kim2022tubeformer} by 5-6\% in terms of mAP on the two VIS datasets.


\noindent
\textbf{[VSS] Results on VSPW and VIP-Seg.} We further conduct experiments on VSPW dataset~\cite{miao2021vspw} for VSS to demonstrate the generalization of Tube-Link. As shown in Tab.~\ref{tab:vspw}, our method achieves over 4\% mIoU improvement compared to the Mask2Former baseline. Under the same ResNet101 backbone, our method achieves the best results. Using the Swin base backbone, our method achieves about 3.7\% mIoU gains over Video K-Net+ with consistent improvements on $mVC$. Our method with a lightweight backbone achieves comparable results to DeepLabv3+ with ResNet101, but with about four times faster inference speed (shown in Fig.~\ref{fig:curve}). Without using any additional techniques, our method also outperforms recent methods specifically designed for VSS~\cite{sun2022vss,sun2022mining}. In Tab.~\ref{tab:vipseg_vss}, we also compare the video semantic segmentation methods in recent VIPSeg datasets with higher-resolution images. Compared with previous state-of-the-art methods, our approaches also achieve state-of-the-art results.

% Moreover, compared with the previous state-of-the-art Tubeformer~\cite{kim2022tubeformer}, our method achieves a better 1.7\% mIoU.


\noindent
\textbf{[VPS] Results on KITTI STEP.} 
We further validate our method on KITTI STEP~\cite{STEP} and report the results in Tab.~\ref{tab:kitti_step}. Our method achieves 0.51 VPQ with the ResNet50 backbone, setting a new state-of-the-art result \textit{without} using temporal attention or optical flow warping. When using a strong Swin-base~\cite{liu2021swin} backbone, our method still achieves better results than Video K-Net~\cite{li2022videoknet} by 3\% VPQ and comparable results on STQ. It is worth noting that one can further improve the performance of Tube-Link by employing a better tracker design.

\subsection{Ablation Study and Visual Analysis}
\label{sec:ablation}
% 1, improvements on baseline 
% 2, design choices of temporal contarstive loss 
% 3, design choices of label assigin stragety
% 4, Effect of tube frames choices for CS loss 
% 5, Effect of large window size / overlap inference. 
% 6, Comparison with the different tracking choices. 
% 7, increased GFLops/Parameters analysis. 
% 8. FPS/Window Cruves.

% In this section, we present some \textit{\textbf{key}} ablations on component design and analysis using VIPSeg dataset with ResNet50 backbone. 
%The default setting used in our model is indicated in gray.
%More results are provided in the supplementary material. 

% \cavan{image part? or do you mean feature extractor or encoder}
% \cite{li2022videoknet} as the baseline by replacing its encoder with Mask2Former~\cite{cheng2021mask2former}
\noindent
\textbf{Improvements over Strong VPS Baseline.} 
In Tab.~\ref{tab:ablation_a}, we demonstrate the effectiveness of each component proposed in Sec.~\ref{sec:tb_framework}. 
The first row shows the results of the frame matching baseline. After adopting the tube matching, we obtain a gain of 1.6\% $\mathrm{VPQ_{th}}$ and 2.1\% on VPQ, even without any specific tracking design, which results in the same observation as shown in Tab.~\ref{tab:toy_exp}. Thus, we use Mask2Former-VIS+ (T, T=2) as our baseline by default, which achieves a strong starting point of 34.5 VPQ. $\mathrm{VPQ_{th}}$ refers to the VPQ for the thing class. This result shows the effectiveness of the na\"{i}ve framework. The addition of TCL further boosts performance, with a gain of 3.5\% on $\mathrm{VPQ_{th}}$ and 1.7\% on VPQ. Furthermore, adding CTL, which makes the association more consistent, improves $\mathrm{VPQ_{th}}$ by 1.5\%.


\noindent
\textbf{Ablation on Temporal Contrastive Loss.} We also compare our TCL design with previous works that use dense queries~\cite{qdtrack} or sparse queries~\cite{li2022videoknet} for matching. Both settings use only one frame, while our subclip size is two. As shown in Tab.~\ref{tab:ablation_b}, our method achieves the best results since tube matching encodes more temporal information. In particular, we observe 3.0\% VPQ improvements compared to the strong Video K-Net baseline.


\begin{figure}[t!]
	\centering
	\includegraphics[width=1.0\linewidth]{./figs/tube_link_vis_results_1st.pdf}
	\caption{\small Comparison results on VIP-Seg and YuoTube-VIS. Our method achieves consistent segmentation (shown in orange boxes) and better tracking results (shown in red boxes).}
	\label{fig:visulize}
\end{figure}

% \gl{R-50 and R50 should be consistent. The same as R-101 and R101.}
\begin{figure}[t]
  \centering
   \includegraphics[width=1.\linewidth]{./figs/both.pdf}
   \caption{\small Efficiency Analysis of Tube-Link. Left: Segmentation results (mIoU) of VSPW with different subclip sizes. Right: Inference speed (FPS) with different subclip sizes.}
   \label{fig:curve}
\end{figure}



\noindent
\textbf{Ablation on Association Target Assignment.} 
In Table \ref{tab:ablation_c}, we show the results of the ablation study on building association targets. We find that using a tube-level mask achieves the best results. Using the mask from one of the input subclips leads to inferior results. This is because the ground truth masks of a single frame are not aligned with the input global queries, where the global queries are learned from multiple frames using Equation \eqref{equ:sp_attention}.

\noindent
\textbf{Effect of Sub-clip Size for Training.} 
In Tab.~\ref{tab:ablation_d}, we investigate the impact of subclip size on training. Tube-Link becomes an online method when the subclip size is 1. As shown in the table, enlarging the subclip size improves the performance. We also examine overlapping during sampling, denoted as ovl, where two input subclips overlap at one frame. As shown in Tab.~\ref{tab:ablation_d}, enlarging the subclip size to 2 achieves significant improvement. However, we find that either frame overlapping or using a larger subclip size ($T=3$) does not bring extra gains. Adding more frames does not benefit temporal association learning, since most instances are similar within a subclip. Moreover, using more frames is not memory-friendly during training. Thus, the subclip size is set to 2. We can enlarge the size for inference, as shown in Tab.~\ref{tab:ablation_e}.
  
% \cavan{to what value and why? During training?}
% \cavan{you mean we can set the subcip size to 2 during training and expand it during inference? This point is not clearly articulated here.}
%  \cavan{where? In future work?}
% \cavan{not sure why we use `Moreover', it doesn't connect well to the previous sentence.}
% % Hence, we can enlarge the subclip size for more efficient inference and global consistency within each tube. 

\noindent
\textbf{Effect of Sub-clip Size for Inference.} 
\if 0
The global queries for each tube learn to perform temporal association via cross-attention within each subclip. Despite the subclip size is limited during the training due to the memory issues, we can expand it during the inference.
For example, the subclip size is 2 during training and is set to 6 for inference. As shown in Tab.~\ref{tab:ablation_e}, we prove that enlarging subclip size for inference improves the performance by a significant margin for all three metrics: STQ, VPQ and $\mathrm{VPQ_{th}}$. When the size is 8, the performance drops. This is because the global queries cannot handle larger subclips as the offline method. Besides the effectiveness, increasing subclip size can also lead to faster speed for each clip input due to full utilization of GPU memory, as shown in Fig.~\ref{fig:curve}.
\fi
%
During training, the subclip size is limited due to memory constraints, but we can expand it during inference to improve the performance. For instance, we use a subclip size of 2 during training and increase it to 6 during inference. Tab.~\ref{tab:ablation_e} shows that enlarging the subclip size for inference improves the performance considerably for all three metrics: STQ, VPQ, and $\mathrm{VPQ_{th}}$. However, when the subclip size is further increased to 8, the performance drops because the global queries are not designed to handle larger subclips. Increasing the subclip size can also speed up the inference process by utilizing the full GPU memory, as demonstrated in Fig.~\ref{fig:curve}.

% \cavan{`lead to a higher number of frames leads to faster speed'? Rephrase this sentence.}

\noindent
\textbf{Different Tracking Choices.} 
\if 0
In Tab.~\ref{tab:ablation_f}, we compare different tracking approaches that were used in previous studies~\cite{qdtrack,li2022videoknet,deepsort}. The default Tube Embedding works best in our framework. It does not require any association embedding head or the RoI crop operation on the VIPSeg dataset. Our Tube-Link only uses the learned tube-level embedding for the association.
\fi
In Tab.~\ref{tab:ablation_f}, we compare different tracking approaches used in previous studies~\cite{qdtrack,li2022videoknet,deepsort} with our Tube-Link. Our Tube-Link only uses the learned tube-level embedding for the association. We find that the default tube embedding works best in our framework, without requiring any association embedding head or RoI crop operation on the VIPSeg dataset.  

\subsection{Visualization and More Analysis}
\label{sec:vis_analysis}

\noindent
\textbf{GFLops and Parameter Analysis.} Compared with Mask2Former baseline, we only add one $\mathrm{Emb}$ head and one self-attention layer, introducing only 2.2\% GFLops and 1.4\% extra parameters with $720 \times 1280$ input. 

%\cavan{the font size is too small to be visible. You can use a common legend for both plots and place it underneath the plots}



% \subsection{Visualization and Analysis}
% \cavan{Do we really need this section? The `Speed and Accuracy with different Input Subclip Size' can be merged with `Effect of Sub-clip Size For Inference.' in the ablation study. `Visual Improvements on Baseline' can be merged with `Improvements over Strong VPS Baseline'.}

\noindent
\textbf{Speed and Accuracy with Different Input Subclip Size.} 
As shown in Table \ref{tab:ablation_e}, adding more frames improves the VPS results. To further analyze the speed-accuracy trade-off, we present a detailed comparison of different methods on the VSPW dataset in Fig.~\ref{fig:curve}. The left plot shows that enlarging the subclip size also improves the VSS results. The right plot illustrates that increasing the subclip size improves the single-frame baseline by 1.25-1.5\% for various backbones. Both performance and speed reach a plateau when the size increases to 6. The experiment justifies our choice of using an input subclip size of 6 for inference.

\noindent
\textbf{Visual Improvements on Baselines.} In Fig.~\ref{fig:visulize}, we present the visual comparison with several strong baselines (Video K-Net+ and Mask2Fomer-VIS) in VPS and VIS settings. The results are randomly sampled from a long clip. We achieve better results on both segmentation and tracking. More visual examples can be found in the supplementary material. 


 \section{Conclusion}
 In this paper, we have presented a tactile manipulation system that is able to rotate different objects without vision. We showed an end-to-end reinforcement learning framework to learn tactile dexterity over the proposed system. We carried out experiments both in simulation and real to demonstrate its effectiveness. Our work demonstrated that we are able to achieve tactile dexterity as humans in real for the first time. In the future, there are many promising future directions to investigate, such as exploring the use of a more dense contact sensor array and scaling up the system to solve more diverse tasks. We hope that our work can pave the way for more intelligent robot hands.

%\noindent 
%\textbf{Acknowledgement}.This work was supported by Institute for Information \& communications Technology Promotion (IITP) grant funded by the Korea government (MSIT) (No. 2021-0-01381, Development of Causal AI through Video Understanding and Reinforcement Learning, and Its Applications to Real Environments) and partly supported by Institute of Information \& communications Technology Planning \& Evaluation (IITP) grant funded by the Korea government (MSIT) (No. 2022-0-00184, Development and Study of AI Technologies to Inexpensively Conform to Evolving Policy on Ethics).

\bibliographystyle{plain}
\bibliography{reference}

\vspace{-0.16in}

\begin{IEEEbiography}[{\includegraphics[width=1in,height=1.25in,clip,keepaspectratio]{./biography/haeyong.pdf}}]{Haeyong Kang}
(S'05) received the M.S. degree in Systems and Information Engineering from University of Tsukuba in 2007. From April 2007 to October 2010, he worked as an associate research engineer at LG Electronics. With working experiences at Korea Institute of Science and Technology (KIST) and the University of Tokyo, He is currently pursuing the Ph.D at School of Electrical Engineering, KAIST. His current research interests include unbiased machine learning and continual learning.
\end{IEEEbiography}

%\vspace{-20mm}

\begin{IEEEbiography}[{\includegraphics[width=1in,height=1.25in,clip,keepaspectratio]{./biography/jaehong.pdf}}]{Jaehong Yoon}
He received the B.S. and M.S. degrees in Computer Science from Ulsan National Institute of Science and Technology (UNIST), and received the Ph.D. degree in the School of Computing from Korea Advanced Institute of Science and Technology (KAIST). He is currently working as a postdoctoral research fellow at KAIST. His current research interests include efficient deep learning, on-device learning, and learning with real-world data.
\end{IEEEbiography}

%\vspace{-20mm}

\begin{IEEEbiography}[{\includegraphics[width=1in,height=1.25in,clip,keepaspectratio]{./biography/sultan.pdf}}]{Sultan Rizky Madjid}
received a B.S. degree in Electrical Engineering with a double major in Mechanical Engineering from KAIST in 2021 and an M.S. degree in Electrical Engineering from KAIST in 2023. His research interests include model compression, sparse representations in deep learning, and continual learning.
\end{IEEEbiography}

%\vspace{-20mm}

\begin{IEEEbiography}[{\includegraphics[width=1in,height=1.25in,clip,keepaspectratio]{./biography/sungjuhwang.pdf}}]{Sung Ju Hwang}
He received the B.S. degree in Computer Science and Engineering from Seoul National University. He received the M.S. and Ph.D. degrees in Computer Science from The University of Texas at Austin. From September 2013 to August 2014, he was a postdoctoral research associate at Disney Research. From September 2013 to December 2017, he was an assistant professor in the School of Electric and Computer Engineering at UNIST. Since 2017, he has been on the faculty at the Korea Advanced Institute of Science and Technology (KAIST), where he is currently a KAIST Endowed Chair Professor in the Kim Jaechul School of Artificial Intelligence and School of Computing at KAIST. 
\end{IEEEbiography}

%\vspace{-20mm}

\begin{IEEEbiography}[{\includegraphics[width=1in,height=1.25in,clip,keepaspectratio]{./biography/cdyoo.pdf}}]{Chang D. Yoo}
(Senior Member, IEEE) He received the B.S. degree in Engineering and Applied Science from the California Institute of Technology, the M.S. degree in Electrical Engineering from Cornell University, and the Ph.D. degree in Electrical Engineering from the Massachusetts Institute of Technology. From January 1997 to March 1999, he was Senior Researcher at Korea Telecom (KT). Since 1999, he has been on the faculty at the Korea Advanced Institute of Science and Technology (KAIST), where he is currently a Full Professor with tenure in the School of Electrical Engineering and an Adjunct Professor in the Department of Computer Science. He also served as Dean of the Office of Special Projects and Dean of the Office of International Relations.
\end{IEEEbiography}

\appendices
Next, we present the Supplementary Materials for the paper ``Re-ReND: Real-time Rendering of NeRFs across Devices''.
Specifically, in addition to the results reported in the paper, we report results of \methodname w.r.t. Image Quality~(Section~\ref{sec:im_qual}) and (Section~\ref{sec:quali}), Rendering Speed~(Section~\ref{sec:fps}), Mesh Size~(Section~\ref{sec:mesh_size} and Section~\ref{sec:meshi}), Disk Space~(Section~\ref{sec:disk_space}), validation of view-dependent effects (Section~\ref{sec:val}),  sensitivity to geometry variations (Section~\ref{sec:geo}) and Photo-metric quality w.r.t. embedding dimensionality $D$ (Section~\ref{sec:dim}).
Furthermore, we encourage the reviewers to watch the \textbf{associated video}, \texttt{Re-ReND.mp4}, demonstrating \methodname's capabilities of real-time rendering across devices.
% In particular, please refer to .
This video demonstrates how \methodname can render, in real time, a scene composed of tens (\Figure{composit}) or even thousands (\Figure{many_objects}) of objects. % , respectively. %  , or even with thousands of . %  in an AR headset.
\Figure{composit} illustrates such a scene, composed of moving chairs, hotdogs, the drumset, and a microphone.


% Finally, we also provide the PyTorch~\cite{NEURIPS2019_9015} and GLSL implementations of our method inside the folders called \texttt{Re-ReND\_Pytorch\_code} and \texttt{Re-ReND\_GLSL\_code}.

% \thispagestyle{empty}
% \appendix

%%%%%%%%% BODY TEXT - ENTER YOUR RESPONSE BELOW
% \section{The PyTorch code and GLSL code}

%  \begin{itemize}
%     \item Clean and README.md
%     \item Should I upload only pur method or MipNeRF and NeRF++?
%     \item Should I upload the generated data and the meshes in a google drive? What happens with anonymity?
% \end{itemize}

% \section{A video showing how we were measuring the FPS}
% \section{A video showing real scenes in comparison with MobileNeRF and SNeRG}
% \section{Qualitative Results}

%  \begin{itemize}
%     \item all objects visualizations 
% \end{itemize}

%-------------------------------------------------------------------------


\begin{figure}
    \centering
    \includegraphics[width=\linewidth]{pics/quantitative.pdf}
    \caption{Box plots of quantitative benchmarks MIG, FactorVAE, Disentanglement, and reconstruction error on dSprites and Shapes3D.}\label{fig:quantitative}
\end{figure}


% that's all folks
\end{document}


