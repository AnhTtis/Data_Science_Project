\begin{figure}[!t]
    \newcommand{\ww}{0.44}
    \centering
    \newcommand{\spacing}{3.3cm}
    \subfloat{\includegraphics[width=\spacing]{latex/supple/figures/limitation/37.png}}
    \hspace{4mm}
    \subfloat{\includegraphics[width=\spacing]{latex/supple/figures/limitation/37_mesh.png}}
    \addtocounter{subfigure}{-2}
    \\
    \vspace{1mm}
    % \subfloat[Input \text{(low-resolution blur)}][Input \\ \text{(low-resolution blur)}]{\includegraphics[width=\spacing]{latex/supple/figures/limitation/28998.png}}
    \subfloat[Input image]{\includegraphics[width=\spacing]{latex/supple/figures/limitation/28998.png}}
    \hspace{4mm}
    \subfloat[BlurHandNet (Ours)]{\includegraphics[width=\spacing]{latex/supple/figures/limitation/28998_mesh.png}}
    \vspace{-2mm}
    \captionsetup[subfigure]{labelformat=empty}
    \begin{minipage}{1.0\linewidth}
    \vspace{1.5mm}
    \subcaption{(blurry and low-resolution) \textcolor{white}{AAAAAAAAAAAAAAAAAA}}
    \end{minipage}
    \caption{\textbf{Failure cases.}
    % When degradations other than blur, low-resolution and low illumination, are included in the input images, our BlurHand often fails to produce accurate 3D hand meshes.
    Our BlurHandNet produces less accurate results when inputs contain multiple complex degradations.
    %
    }
    \label{fig:failure_case}
\end{figure}