\section{Statistics on the BlurHand dataset}
\label{sec:detail_blurhand}
% 

In Table~\ref{table:blurhand_detail}, we report the detailed number of training samples.
We note that the right and left hands are evenly distributed in the BlurHand.
% 
In \figref{motion_blur_strength_statistics}, we further report additional measurements, namely joint motion magnitude, to present the statistics on the blur strength of our BlurHand.
% 
% Specifically, we first measure 2D joint displacements between neighboring frames from five sharp frames, which constructs the blurry frame.
In detail, we first prepare five sequential sharp frames, which construct a single blurry frame in our BlurHand.
% 
Then we calculate the 2D joint distance between two adjacent sharp frames using the GT joint positions.
% 
Finally, we add all distances for each joint, which we denote as joint motion magnitude.
% 
We note that the large joint motion magnitude means a strong blur exists in hand.
% 
Our BlurHand contains samples with various joint motion magnitude in both the train and test sets.
% 

\section{Results from various deblurring methods}
\label{sec_supp:deblurring}
% In Tables~\textcolor{red}{2} and \textcolor{red}{3} in our main manuscript, we compared our BlurHandNet with the combination of state-of-the-art 3D hand mesh estimation methods~\cite{moon2022accurate, lin2021end, moon2020i2l} and off-the-shelf deblurring method~\cite{chen2022simple}.
In Tables~\textcolor{red}{2} and \textcolor{red}{3} in our main manuscript, we compared our BlurHandNet with the combination of state-of-the-art 3D hand mesh estimation methods~[\fakeref{16}, \fakeref{21}, \fakeref{22}] and off-the-shelf deblurring method~[\fakeref{3}].
% 
% In Table~\ref{sup_table:deblurring}, we additionally compare the results from another widely used deblurring method, DeepDeblur~\cite{nah2017deep}, as the final mesh estimation results might be dependent on the performance of deblurring methods.
% In Table~\ref{sup_table:deblurring}, we additionally compare the results from another widely used deblurring method, DeepDeblur~\cite{nah2017deep}, as the final mesh estimation results might be dependent on the performance of deblurring methods.
In Table~\ref{sup_table:deblurring}, we additionally compare the results from another widely used deblurring method, DeepDeblur~[\fakeref{25}], as the final mesh estimation results might be dependent on the performance of deblurring methods.
% Please note that NAFNet~\cite{chen2022simple} is a deblurring method that we used in the main manuscript.
Please note that NAFNet~[\fakeref{3}] is a deblurring method that we used in the main manuscript.
% 
%As shown in Table~\ref{sup_table:deblurring},
% Our BlurHandNet still outperforms the case when we use DeepDeblur~\cite{nah2017deep} as the deblurring method.
Our BlurHandNet still outperforms the case when we use DeepDeblur~[\fakeref{25}] as the deblurring method.
% 
The results again demonstrate that utilizing temporal information is useful rather than simply adopting deblurring methods.
% 


\begin{table}[h]
\small
\centering
\caption{
Statistics of \NAME v1.0.
}
% \begin{tabular}{P{1.7cm}P{1.2cm}P{1.2cm}P{1.0cm}P{1.2cm}P{1.2cm}P{1.2cm}P{1.1cm}P{1.1cm}P{1.1cm}}
\begin{tabular}{lr}
\toprule[1.2pt]
Statistics \\
\midrule
\# of prompts & 4,081 \\
$\ $ - \# of COCO captions & 2,000 \\
$\ $ - \# of DrawBench, PartiPrompt, PaintSkill prompts & 2,081 \\
\midrule
\# of questions & 25,829 \\
$\ $ - \# of binary questions & 17,226 \\
$\ $ - \# of multiple-choice questions & 8,603 \\
\midrule
avg. \# of questions per prompt & 6.3 \\
avg. \# of words per prompt & 10.5 \\
avg. \# of elements per prompt & 4.3 \\
\bottomrule[1.2pt]
\end{tabular}
\label{tab:statistics}
% \bottomrule
\vspace{-3mm}
\end{table}
\begin{figure}[t]
    \newcommand{\ww}{0.50}
    \centering
    \captionsetup[subfigure]{labelfont=scriptsize, textfont=scriptsize}
    \subfloat[Train set \label{motion_blur_a}]{\includegraphics[width=\ww \linewidth]{latex/supple/figures/statistics/train_v0_ylim.png}}
    % \hfill
    \subfloat[Test set \label{motion_blur_b}]{\includegraphics[width=\ww \linewidth]{latex/supple/figures/statistics/test_v0_ylim.png}}
    \\
    \caption{\textbf{Statistics on blur strength of the presented BlurHand.}
    %
    On average, the joint motion magnitude from the train and test set are 16.9 and 17.8, respectively.
    }
    \label{fig:motion_blur_strength_statistics}
\end{figure}
\begin{table}[t]
\small
\centering
\setlength\tabcolsep{1.0pt}
\def\arraystretch{1.1}
\resizebox{0.95\linewidth}{!}{
\begin{tabular}{C{2.2cm}|C{1.4cm}|C{1.4cm}|C{1.4cm}|C{1.1cm}|C{1.1cm}|C{1.1cm}}
\specialrule{.1em}{.05em}{.05em}
\hline
Deblurring & \multirow{2}{*}{Deblur} & \multirow{2}{*}{Unfolder} & \multirow{2}{*}{KTFormer} & \multicolumn{3}{c}{MPJPE} \\
\cline{5-7}
methods & & &  & initial & middle & final  \\
\hline
% \multirow{2}{*}{DeepDeblur~\cite{nah2017deep}} & \cmark & \xmark & \xmark & - & 18.03 & - \\ 
\multirow{2}{*}{DeepDeblur~[\fakeref{25}]} & \cmark & \xmark & \xmark & - & 18.03 & - \\ 
& \cmark & \cmark & \cmark  & 19.24 & 18.04 & 19.27\\
\hline
\multirow{2}{*}{NAFNet~[\fakeref{3}]} & \cmark & \xmark & \xmark & - & 17.28 & - \\ 
& \cmark & \cmark & \cmark & 18.95 & 17.28 & 19.10 \\
\hline
None & \xmark & \cmark & \cmark & \textbf{18.08} & \textbf{16.80} & \textbf{18.21} \\
\hline
\specialrule{.1em}{.05em}{.05em}
\end{tabular}}
\vspace{-3mm}
\caption{
    % \textbf{Comparison results on various deblurring methods~\cite{chen2022simple,nah2017deep}}. Instead of adopting deblurring methods, our BlurHandNet~(last row) utilizes temporal information from a blurry image.
    \textbf{Comparison results on various deblurring methods~[\fakeref{3}, \fakeref{25}]}. Instead of adopting deblurring methods, the proposed BlurHandNet (last row) directly utilizes temporal information from a single blurry image.
}
\label{sup_table:deblurring}
\end{table}
