\section{Discussion}
\label{sec:discussion}
%\subsection{Limitations} \label{limitation}
%\noindent\textbf{Limitations.}
% While recent studies show significant benefits for occlusion-robust 3D mesh recovery~\cite{park2022handoccnet, liu2021semi,hasson2019learning,hasson2020leveraging}, we designed our BlurHandNet with no such consideration in mind.
% As occlusion is a phenomenon that frequently occurs in hands alongside blur, we believe that additional designs to make networks robust to occlusion can be considered in future works.

\noindent\textbf{Limitations and future works.}
% While various types of degradation are prevalent in real world hand images, \eg, low-resolution, noise, and low illumination, we focus on tackling a specific factor into our consideration, the blur.
While various types of image degradations are prevalent in real-world hand images, \eg, low-resolution, noise, and low illumination, we especially focus on hands with blur artifacts.
% 
% Figure~\ref{fig:failure_case} shows that BlurHandNet produces not robust results when the input image contains multiple complex since our method does not consider other degradations, which should be considered in our future works.
Figure~\ref{fig:failure_case} shows that BlurHandNet produces not robust results when the input image is low-resolution with blur, which should be considered in our future works.


% Although we considered blurry hand, which is prevalent in practice, another type of artifact still exists in in-the-wild images, \eg, occlusions~\cite{park2022handoccnet} and hand-object interaction~\cite{liu2021semi, hasson2019learning, hasson2020leveraging}.
% 
% Since our methods do not consider such additional artifacts, our methods produce some failure cases when the input image contains multiple complex degradations, as shown in \figref{}, which should be considered in our future works.
% 
% Moreover, we are planning to extend our work to handle interacting blurry hands.
% 

% ~\cite{park2022handoccnet, liu2021semi,hasson2019learning,hasson2020leveraging}

% While recent studies show significant benefits for occlusion-robust 3D mesh recovery~\cite{park2022handoccnet, liu2021semi,hasson2019learning,hasson2020leveraging}, we designed our BlurHandNet with no such consideration in mind.
% As occlusion is a phenomenon that frequently occurs in hands alongside blur, we believe that additional designs to make networks robust to occlusion can be considered in future works.
% Moreover, we are planning to extend our work to handle interacting blurry hands.

% \jaeha{failure case visualization}
% In \figref{}, we visualize some failure cases from in-the-wild images.
% Despite our BlurHandNet estimates of more accurate 3D mesh from blurry hand compared to the previous methods, there exists still challenging samples caused by extreme blur and generalization problems.
%


% \noindent\textbf{Future works.}
% %
% To handle the failure cases, developing more robust algorithm, such as domain adaptation~\cite{pei2018multi,taigman2016unsupervised}, can be considered as our future work.
% %
% Moreover, we are planning to extend our work to handle interacting blurry hands.



%\subsection{Societal impacts} \label{social_impact}
\noindent\textbf{Societal impacts.}
Our BlurHand and BlurHandNet suggest a new and necessary research direction toward real-world applications, the robust 3D hand mesh estimation from blurry images.
In particular, our method can be useful for AR/VR as people often move hands fast, which causes motion blur of hands.

\begin{figure}[!t]
    \newcommand{\ww}{0.44}
    \centering
    \newcommand{\spacing}{3.3cm}
    \subfloat{\includegraphics[width=\spacing]{latex/supple/figures/limitation/37.png}}
    \hspace{4mm}
    \subfloat{\includegraphics[width=\spacing]{latex/supple/figures/limitation/37_mesh.png}}
    \addtocounter{subfigure}{-2}
    \\
    \vspace{1mm}
    % \subfloat[Input \text{(low-resolution blur)}][Input \\ \text{(low-resolution blur)}]{\includegraphics[width=\spacing]{latex/supple/figures/limitation/28998.png}}
    \subfloat[Input image]{\includegraphics[width=\spacing]{latex/supple/figures/limitation/28998.png}}
    \hspace{4mm}
    \subfloat[BlurHandNet (Ours)]{\includegraphics[width=\spacing]{latex/supple/figures/limitation/28998_mesh.png}}
    \vspace{-2mm}
    \captionsetup[subfigure]{labelformat=empty}
    \begin{minipage}{1.0\linewidth}
    \vspace{1.5mm}
    \subcaption{(blurry and low-resolution) \textcolor{white}{AAAAAAAAAAAAAAAAAA}}
    \end{minipage}
    \caption{\textbf{Failure cases.}
    % When degradations other than blur, low-resolution and low illumination, are included in the input images, our BlurHand often fails to produce accurate 3D hand meshes.
    Our BlurHandNet produces less accurate results when inputs contain multiple complex degradations.
    %
    }
    \label{fig:failure_case}
\end{figure}
