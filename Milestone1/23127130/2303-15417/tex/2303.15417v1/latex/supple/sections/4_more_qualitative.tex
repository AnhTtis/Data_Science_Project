\section{Additional qualitative results}
\label{sec:more_results}
% We show more qualitative comparisons in Figure~\ref{} and \ref{} with the recovered meshes on BlurHand dataset and in-the-wild images, respectively.
% These results clearly demonstrate that our proposed BlurHandNet produces more accurate outputs for various blurry hand images, including real blurry hand images, than previous state-of-the-art methods.

\noindent\textbf{Effectiveness of the BlurHand.}
% In Figure~\ref{fig:supple_comparison_yt3d0}, we provide additional qualitative results on YT-3D~\cite{kulon2020weakly}.
In Figure~\ref{fig:supple_comparison_yt3d0}, we provide additional qualitative results on YT-3D~[\fakeref{13}].
% 
We note that training the model on our BlurHand~(column (e) in the \figref{supple_comparison_yt3d0}) is significantly helpful in dealing with the in-the-wild blurry hand images compared to the cases using sharp images and deblurred images~(column (c) and (d) in the \figref{supple_comparison_yt3d0}).
The results justify the necessity of our BlurHand when handling the blurry hand.
% 

\noindent\textbf{Visual comparison on BlurHand.}
\label{ssec:limitations}
In \figref{comparison_BlurHand}, we present additional comparison results on BlurHand.
% 
% Compared to the previous state-of-the-art methods~\cite{lin2021end,moon2022accurate}, our BlurHandNet reconstructs more accurate 3D hand meshes by exploiting temporal information.
Compared to the previous state-of-the-art methods~[\fakeref{16}, \fakeref{21}], our BlurHandNet reconstructs more accurate 3D hand meshes by exploiting temporal information.
% 


% These results clearly demonstrate that our proposed BlurHandNet produces more accurate outputs for various blurry hand images, including real blurry hand images, than previous state-of-the-art methods.

