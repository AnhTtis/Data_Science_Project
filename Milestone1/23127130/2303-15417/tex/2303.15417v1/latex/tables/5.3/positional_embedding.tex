% \begin{table}
% \small
% \centering
% \setlength\tabcolsep{1.0pt}
% \def\arraystretch{1.1}
% \begin{tabular}{C{1.0cm}|C{1.4cm}|C{1.4cm}|C{1.1cm}|C{1.1cm}|C{1.1cm}}
% \specialrule{.1em}{.05em}{.05em}
% \hline
% & \multirow{2}{*}{Spatial} & \multirow{2}{*}{Temporal} & \multicolumn{3}{c}{MPJPE~(mm)} \\
% \cline{4-6}
% & & & past & current & future \\
% \hline
% \multirow{4}{*}{\textbf{Ours}}&\xmark & \cmark & 18.79 & 17.41 & 18.92\\
% % \cline{2-6}
% & \cmark & \xmark & 18.28 & 16.92 & 18.34 \\
% % \cline{2-6}
% & \multicolumn{2}{c|}{\cmark} & 18.27 & 16.99 & 18.37 \\
% % \cline{2-6}
% & \cmark & \cmark & \textbf{18.08} & \textbf{16.80} & \textbf{18.21} \\
% \hline
% \specialrule{.1em}{.05em}{.05em}
% \end{tabular}
% \caption{
%     \textbf{Ablation study on positional embedding.} The third row applies separate positional embedding to the whole joints. Therefore, different from our final method in the fourth row, which uses spatial~$\mathbb{R}^{J \times c}$ and temporal embedding~$\mathbb{R}^{3 \times c}$, the third row utilizes embedding with dimension $\mathbb{R}^{3J \times c}$.
% }
% \label{table:positional_embedding}
% \end{table}

\begin{table}
\small
\centering
\setlength\tabcolsep{1.0pt}
\def\arraystretch{1.1}
\resizebox{0.65\linewidth}{!}{
\begin{tabular}{C{1.4cm}|C{1.4cm}|C{1.1cm}|C{1.1cm}|C{1.1cm}}
\specialrule{.1em}{.05em}{.05em}
\hline
\multirow{2}{*}{Kinematic} & \multirow{2}{*}{Temporal} & \multicolumn{3}{c}{MPJPE} \\
\cline{3-5}
& & initial & middle & final \\
\hline
\xmark & \xmark & 18.99 & 17.79 & 19.06 \\
\cmark & \xmark & 18.28 & 16.92 & 18.34 \\
\xmark & \cmark & 18.79 & 17.41 & 18.92\\
% \cline{2-6}
% \cline{2-6}
% \multicolumn{2}{c|}{\cmark} & 18.27 & 16.99 & 18.37 \\
% \cline{2-6}
\cmark & \cmark & \textbf{18.08} & \textbf{16.80} & \textbf{18.21} \\
\hline
\specialrule{.1em}{.05em}{.05em}
\end{tabular}}
\vspace{-3mm}
\caption{\textbf{Ablation study on the kinematic and temporal positional embeddings.} 
    % \textbf{Ablation study on positional embedding.} The third row applies separate positional embedding to the whole joints. Therefore, different from our final method in the fourth row, which uses spatial~$\mathbb{R}^{J \times c}$ and temporal embedding~$\mathbb{R}^{3 \times c}$, the third row utilizes embedding with dimension $\mathbb{R}^{3J \times c}$.
}
\vspace{-3mm}
\label{table:positional_embedding}
\end{table}