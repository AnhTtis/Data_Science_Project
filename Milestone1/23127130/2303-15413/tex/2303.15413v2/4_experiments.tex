\section{Comparison with Baseline}

For the experiments, we use the highest-performing public codebase of SJC~\cite{wang2022score}, using the same optimization hyperparameters for both for a fair comparison.

As shown in the qualitative results in Fig.~\ref{fig:qualitative}, our methods reduce view inconsistencies in the 3D objects and mitigate the so-called Janus problem. This improvement come with little overhead compared to the baseline.

We propose a new metric (Adjacent LPIPS; A-LPIPS) to quantitatively measure the view consistency of 3D fields. It computes the average LPIPS~\cite{zhang2018unreasonable} between adjacent images rendered from evenly spaced azimuths, using two different backbones~\cite{simonyan2014very,deng2009imagenet}. The intuition behind this metric is that two images from adjacent viewpoints are perceptually similar if the 3D field is consistent. Our method produces more consistent 3D objects than the baseline, as demonstrated in Table~\ref{tab:colmap} based on 70 prompts. Note that removing contradictions in prompts leads to better results.

We present ablation results in Fig.~\ref{fig:ablation}, where we sequentially added prompt debiasing and score debiasing on top of the baseline. This demonstrates that they gradually improve the view consistency and reduce artifacts as intended.

Overall, the experiments corroborate that our debiasing methods improve the realism and alleviate the Janus problem of generated 3D objects, without requiring any 3D guide~\cite{seo2023let} or introducing significant overhead to the zero-shot text-to-3D setting.

\begin{table}[t]
\footnotesize  
\centering
\begin{tabular}{l c c} 
\toprule
 Method & $\text{A-LPIPS}_\text{VGG} \downarrow$ & $\text{A-LPIPS}_\text{Alex} \downarrow$ \\
\midrule
Baseline~\cite{wang2022score} & 0.2054 & 0.1526 \\
Debiased (Preserved) & \underline{0.1963} & \underline{0.1450} \\
Debiased (Ours) & \textbf{0.1940} & \textbf{0.1445} \\
\bottomrule
\end{tabular}
\vspace{-5pt}
\caption{\textbf{Quantitative evaluation.} The best values are in bold, and the second best are underlined. \emph{Preserved} means user prompts are preserved, \ie, $P(u)=1$ for all $u$.}
\vspace{-10pt}
\label{tab:colmap}
\end{table}

\begin{figure}[t]
\centering
\includegraphics[width=0.95\linewidth]{figures/ablation.pdf}\vspace{-5pt}
\caption{\textbf{Improvement of view consistency through prompt and score debiasing.} The baseline is original SJC~\cite{wang2022score}, and \textit{Prompt} and \textit{Score} denote prompt and score debiasing, respectively. The given user prompt is \textit{``a smiling cat,"} and the images are rendered from arbitrary viewpoints.}
\label{fig:ablation}\vspace{-15pt}
\end{figure}
