\section{Conclusion}

In conclusion, we have addressed the critical issue of view inconsistency in zero-shot text-to-3D generation, particularly focusing on the Janus problem. By dissecting the formulation of score-distilling text-to-3D generation and pinpointing the primary causes of the problem, we have proposed a dynamic score debiasing method that mitigates the impact of erroneous bias in the estimated score. This method significantly reduces artifacts and improves the 3D consistency of generated objects. Additionally, our prompt debiasing approach refines the use of user and view prompts to create more realistic and view-consistent 3D objects. Our work, D-SDS, presents a major step forward in the development of more robust and reliable zero-shot text-to-3D generation techniques, paving the way for further advancements in the field.

\section*{Acknowledgements}
This research was supported by the MSIT, Korea (IITP-2022-2020-0-01819, ICT Creative Consilience program, RS-2023-00227592, Development of 3D Object Identification Technology Robust to Viewpoint Changes, No. 2021-0-00155, Context and Activity Analysis-based Solution for Safe Childcare), and National Research Foundation of Korea (NRF-2021R1C1C1006897).