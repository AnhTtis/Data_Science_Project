\begin{abstract}
The view inconsistency problem in score-distilling text-to-3D generation, also known as the Janus problem, arises from the intrinsic bias of 2D diffusion models, which leads to the unrealistic generation of 3D objects. In this work, we explore score-distilling text-to-3D generation and identify the main causes of the Janus problem. Based on these findings, we propose two approaches to debias the score-distillation frameworks for robust text-to-3D generation. Our first approach, called score debiasing, involves gradually increasing the truncation value for the score estimated by 2D diffusion models throughout the optimization process. Our second approach, called prompt debiasing, identifies conflicting words between user prompts and view prompts utilizing a language model and adjusts the discrepancy between view prompts and object-space camera poses. Our experimental results show that our methods improve realism by significantly reducing artifacts and achieve a good trade-off between faithfulness to the 2D diffusion models and 3D consistency with little overhead.
\end{abstract}