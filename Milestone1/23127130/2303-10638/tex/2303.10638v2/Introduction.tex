\section{Introduction}

Let $G$ be any group, and write $S(G)$ for the group of permutations on $G$, where maps are composed from left to right.  Let
\[ \rho : G\rightarrow S(G);\,\ x^{\rho(y)} = xy\]
denote the right regular representation of $G$. The  \emph{holomorph} of $G$ may be defined as the subgroup
\[ \Hol(G) = \Aut(G)\rho(G) = \Aut(G)\ltimes \rho(G)\]
of $S(G)$, or equivalently, the normalizer of $\rho(G)$ in $S(G)$. The \emph{multiple holomorph} of $G$, which is denoted as $\mathrm{NHol}(G)$, is in turn defined to be the normalizer of $\Hol(G)$ in $S(G)$. It is easy to see that the quotient
\[ T(G) = \NHol(G)/\Hol(G)\]
acts regularly, via conjugation, on the regular subgroups of $S(G)$  that are isomorphic to $G$ and have normalizer equal to $\Hol(G)$. These subgroups are exactly the normal regular subgroups of $\Hol(G)$ when $G$ is finite, which is the case of interest of this paper.

The group $T(G)$ was first studied in \cite{Miller} and has attracted more attention recently since \cite{Kohl} was published. The structure of $T(G)$ has been computed for certain families of groups $G$, such as 
\begin{itemize}
\item finitely generately abelian groups (\cite{Mills},\cite{fg}),
\item dihedral and generalized quaternion groups (\cite{Kohl}),
\item finite groups of squarefree order (\cite{squarefree}).
\end{itemize}
In this paper, we shall consider finite $p$-groups $G$ of class two, where $p$ always denotes an odd prime. In this case, by \cite[Proposition 3.1]{class2}, we know that  
$T(G)$ has a cyclic subgroup of order $p^{r-1}(p-1)$, where $p^r$ is the exponent of the quotient of $G$ by its center. We shall say that the order of $T(G)$ is \emph{minimal} when it is exactly $p^{r-1}(p-1)$.

\begin{notationstar}For any group $G$, we employ the following notation:
\begin{itemize}
\item $G'=\mbox{the commutator subgroup of $G$}$.
\item $Z(G)=\mbox{the center of $G$}$.
\item $\mathrm{Frat}(G)=\mbox{the Frattini subgroup of $G$}$.
\item $\Aut_c(G)=\mbox{the kernel of the natural homomorphism}$
\[\Aut(G)\rightarrow \Aut(G/Z(G)).\]
Also write $\Aut^c(G)$ for the image of this homomorphism, which we shall sometimes identify as $\Aut(G)/\Aut_c(G)$.
\item $\Aut_z(G)=\mbox{the kernel of the natural homomorphism}$
\[  \Aut(G)\rightarrow \Aut(Z(G)).\]
\item In the case that $G'=Z(G)$, we have $\Aut_c(G)\subseteq \Aut_z(G)$, and this allows us to define a natural homomorphism
\[ \Aut(G/G')\rightarrow \Aut(G');\,\ \alpha\mapsto \hat{\alpha}.\]
Here $\hat{\alpha}$ is the element induced by any lift of $\alpha$ in $\Aut(G)$.
\end{itemize}
For any $x,y\in G$, we shall also write $x^y = y^{-1}xy$ and $[x,y] = x^{-1}x^y$.
\end{notationstar}

Let us now return to finite $p$-groups $G$ of class two. In this case, by \cite[Section 2]{class2}, we may study certain elements of $T(G)$ via the use of bilinear forms. In our previous work \cite[Section 4]{LMH}, we exploited this approach and showed that
\begin{equation}\label{T(G) pre} T(G)= \mathcal{S}\rtimes \mathcal{S}'\mbox{ when $G' = Z(G)$ and $\Aut^c(G)=1$}.\end{equation}
Here, loosely speaking, the subgroups $\mathcal{S}$ and $\mathcal{S}'$ consist of the elements corresponding to symmetric and anti-symmetric forms, respectively. In Section \ref{bilinear form sec}, we shall generalize this fact and show that (\ref{T(G) pre}) is still valid even when $\Aut^c(G)\neq 1$, provided that there is no non-trivial $\Aut^c(G)$-equivariant homomorphism from $G/G'$ to $\Aut^c(G)$ (Assumption \ref{assumption}).

In \cite{LMH}, we considered a family of $p$-groups $G$ of class two which may be constructed from linear maps $\pi : V\rightarrow\Lambda^2V$. Here $V$ is a finite dimensional vector space over $\mathbb{F}_p$ and $\Lambda^2V$ denotes its exterior square. The construction is from \cite{CarIJM}, to be reviewed in Section \ref{group section}. Let us note that the constructed $p$-groups $G$ have order 
\[ |G| = p^{n + {n\choose 2}}\mbox{ when }n = \dim_{\mathbb{F}_p}(V),\]
and are \emph{special} in the sense that
\[G' = Z(G) = \mathrm{Frat}(G).\]
By \cite{simple-constr}, the condition $\Aut^c(G)=1$ may be realized by certain $\pi$ of full rank for any $n\geq 4$. Using these special $p$-groups $G$ and (\ref{T(G) pre}), we proved in \cite{LMH}  that $T(G)$ can be made arbitrarily large.

In the present paper, we consider the opposite direction and search for examples of $G$ such that $T(G)$ is small (or even of minimal order). To get such examples, we naturally want $\Aut^c(G)$ to be large (see the discussion after Theorem \ref{pre thm}), and this is more likely to happen when $\pi$ has small rank (see the discussion after (\ref{iden Aut})). 

The extreme case is when $\pi$ is the trivial linear map (when the rank is zero). The associated group $G$ as defined in (\ref{Gpi}) is the free $p$-group of class two and exponent $p$. We have that
\[ \Aut^c(G) = \Aut(G/G') \]
is as large as possible, and it is already known by \cite[Theorem 5.2]{class2} that $T(G)$ has minimal order $p-1$.

The next natural case to consider is when the rank of $\pi$ is one. For $n=2$, the associated $G$ has order $p^3$, and we already know by \cite[Proposition 5.1]{class2} that  $T(G)$ has minimal order $p-1$. For $n=3,4$, as we shall show in Propositions \ref{rank one prop'} and \ref{rank one prop}, respectively, up to a choice of basis there are only two and three possibilities of $\pi$ of rank one.  It turns out that $T(G)$ has minimal order $p-1$ in four of the cases, and $T(G)$ has order $(p-1)^2$ in the remaining case. 
Our main result is the following:

\begin{theorem}\label{thm1} The following holds.
\begin{enumerate}[label = $(\alph*)$]
\item Let $p\geq 5$ be any prime. For the group
\begin{align*}
G =\Bigg\langle x_1,x_2,x_3: &\, [[x_i,x_j],x_k] = 1\mbox{ for }1\leq i,j,k\leq 3,\\[-10pt]
&x_2^p = x_3^p= 1,\, x_1^p = [x_1,x_2]\Bigg\rangle
\end{align*}
of order $p^6$, we have $T(G)\simeq \mathbb{F}_p^\times$.
\item Let $p\geq 3$ be any prime. For the group
\begin{align*}
G =\Bigg\langle x_1,x_2,x_3: &\, [[x_i,x_j],x_k] = 1\mbox{ for }1\leq i,j,k\leq 3,\\[-10pt]
&x_2^p = x_3^p= 1,\, x_1^p = [x_2,x_3]\Bigg\rangle
\end{align*}
of order $p^6$, we have $T(G)\simeq \mathbb{F}_p^\times$.
\item Let $p\geq 5$ be any prime. For the group
\begin{align*}
G =\Bigg\langle x_1,x_2,x_3,x_4 : &\, [[x_i,x_j],x_k] = 1\mbox{ for }1\leq i,j,k\leq 4,\\[-10pt]
&x_2^p = x_3^p=x_4^p = 1,\, x_1^p = [x_1,x_2]\Bigg\rangle
\end{align*}
of order $p^{10}$, we have $T(G)\simeq \mathbb{F}_p^\times$.
\item Let $p\geq 3$ be any prime. For the group
\begin{align*}
G =\Bigg\langle x_1,x_2,x_3,x_4 : &\, [[x_i,x_j],x_k] = 1\mbox{ for }1\leq i,j,k\leq 4,\\[-10pt]
&x_2^p = x_3^p=x_4^p = 1,\, x_1^p = [x_3,x_4]\Bigg\rangle
\end{align*}
of order $p^{10}$, we have $T(G)\simeq \mathbb{F}_p^\times$.
\item Let $p\geq 5$ be any prime. For the group
\begin{align*}
G =\Bigg\langle x_1,x_2,x_3,x_4 : &\, [[x_i,x_j],x_k] = 1\mbox{ for }1\leq i,j,k\leq 4,\\[-10pt]
&x_2^p = x_3^p=x_4^p = 1,\, x_1^p = [x_1,x_2][x_3,x_4]\Bigg\rangle
\end{align*}
of order $p^{10}$, we have $T(G)\simeq\mathbb{F}_p^\times \times \mathbb{F}_p^\times$.
\end{enumerate}
\end{theorem}

%\begin{remark}Still assume that $\dim_{\mathbb{F}_p}(V) =4$. The next natural step is to consider $\pi :V\rightarrow\Lambda^2V$ of higher rank. However, even up to a choice of basis, there are many possibilities of $\pi$ and we do not have a simple way of dealing with them. It seems that there are choices of $\pi$ of rank two for which $T(G)$ has minimal order for the corresponding group $G$. But we did not find any such examples of $\pi$ of rank three or four, and it might be that no such $\pi$ exists.
%\end{remark}



