\section{Multiple holomorph via bilinear forms}\label{bilinear form sec}

Let $p$ be an odd prime and let $G$ be a finite $p$-group of class two for which $G'=Z(G)$. Notice that then $\Aut_c(G)\subseteq \Aut_z(G)$. In this case, by \cite{LMH}, the structure of $T(G)$ may be studied using certain bilinear forms, as follows.

First, clearly any regular subgroup of $\Hol(G)$ must take the form
\[ N_\gamma = \{\gamma(x)\rho(x): x\in G\},\mbox{ where }\gamma :G \rightarrow\Aut(G)\]
is some suitable map. As shown in \cite[Theorem 5.2]{perfect}, the set $N_\gamma$ is a normal regular subgroup of $\Hol(G)$ exactly when
\[ \begin{cases}
 \gamma(xy) =\gamma(y)\gamma(x)\\
 \gamma(x^\beta) = \gamma(x)^\beta
 \end{cases}
 \mbox{for all }x,y\in G\mbox{ and } \beta\in \Aut(G),\]
namely when $\gamma$ is an $\Aut(G)$-equivariant anti-homomorphism. In our setting of $G$, it follows from \cite[Proposition 2.2]{class2} that such $\gamma$ with image lying inside $\Aut_c(G)$ may be parametrized by $G'$-valued bilinear forms on $G/G'$ which are also $\Aut(G)$-equivariant.

\begin{prop}%%\label{gamma prop}
The following data are equivalent.
\begin{enumerate}[label = $(\arabic*)$]
\item A normal regular subgroup $N$ of $\Hol(G)$ such that its projection onto $\Aut(G)$ along $\rho(G)$ is contained in $\Aut_c(G)$.
\item An anti-homomorphism $\gamma : G\rightarrow\Aut_c(G)$ such that
\[\gamma(x^\beta) = \gamma(x)^\beta\]
for all $x\in G$ and $\beta\in \Aut(G)$.
\item A bilinear form $\Delta : G/G'\times G/G'\rightarrow G'$ such that
\begin{equation}\label{Delta1} \Delta(u^\beta,v^\beta) = \Delta(u,v)^\beta \end{equation}
for all $u,v\in G/G'$ and $\beta\in \Aut(G)$.
%Here $\Aut(G)$ acts naturally on $G/G'$ and $G'$.
%For simplicity, we are writing $\Delta(x,y)$ for $\Delta(xG',yG')$.
\end{enumerate}
The data $(1),(2)$ are related via $N = N_\gamma$, and $(2),(3)$ are related via  
\begin{equation}\label{gamma-Delta}x^{\gamma(y)} = x\Delta(xG',yG')\end{equation}
for all $x,y\in G$.
\end{prop}

Not every $\Aut(G)$-equivariant anti-homomorphism  $\gamma : G\rightarrow\Aut(G)$ would have image lying inside $\Aut_c(G)$ in general. But for $x\in G'$, we know from \cite[Lemma 1.7]{class2} that $\gamma(x)$ is always an inner automorphism. Since $G$ is of class two, all inner automorphisms are central. It follows that $\gamma$ induces a well-defined map
\[ \overline{\gamma}: G/G'\rightarrow \Aut^c(G);\,\ \overline{\gamma}(xG') = \gamma(x)\Aut_c(G).\]
This map is clearly an anti-homomorphism, or simply homomorphism because the domain $G/G'$ is abelian, satisfying
\[ \overline{\gamma}(v^\alpha) = \overline{\gamma}(v)^\alpha\]
for all $v\in G/G'$ and $\alpha\in\Aut^c(G)$, namely it is an $\Aut^c(G)$-equivariant homomorphism. We note that the image of $\gamma$ lies inside $\Aut_c(G)$ if and only if the induced map $\overline{\gamma}$ is trivial.

In our setting of $G$, we have the inclusion $\Aut_c(G)\subseteq \Aut_z(G)$, and so the condition (\ref{Delta1}) is vacuous when $\beta\in \Aut_c(G)$. This means that we only need to consider action of $\Aut^c(G)$ here.

For simplicity, henceforth, let us further assume the following:

\begin{assume}\label{assumption}There is no non-trivial $\Aut^c(G)$-equivariant homomorphism from $G/G'$ to $\Aut^c(G)$.\end{assume}

From the above discussion, we then deduce that:

\begin{prop}\label{gamma prop new}
The following data are equivalent.
\begin{enumerate}[label = $(\arabic*)$]
\item A normal regular subgroup $N$ of $\Hol(G)$.
%\item An anti-homomorphism $\gamma : G\rightarrow\Aut(G)$ such that
%\[\gamma(x^\beta) = \gamma(x)^\beta\]
%for all $x\in G$ and $\beta\in \Aut(G)$.
\item A bilinear form $\Delta : G/G'\times G/G'\rightarrow G'$ such that
\begin{equation}\label{Delta2} \Delta(u^\alpha,v^\alpha) = \Delta(u,v)^{\hat{\alpha}} \end{equation}
for all $u,v\in G/G'$ and $\alpha\in \Aut^c(G)$. %Here $\hat{\alpha}$ is the automorphism on $G'$ induced by any lift of $\alpha$ to $\Aut(G)$.
%Here $\Aut^c(G)$ acts naturally on $G/G'$ and $G'$.
%The correspondence is given by
%\[ \Delta(x,y) = x^{-1}x^{\gamma(y)}\mbox{ for all }x,y\in G.\]
%For simplicity, we are writing $\Delta(x,y)$ for $\Delta(xG',yG')$.
\end{enumerate}
\end{prop}

Before proceeding, let us set up some notation. Define
\[ B =  \{\mbox{bilinear forms $\Delta :G/G'\times G/G'\rightarrow G'$ satisfying (\ref{Delta2})}\},\]
which is an abelian group under pointwise multiplication in $G'$. Let $S$ and $S'$, respectively, be the subgroups of $B$ consisting of the bilinear forms which are symmetric and anti-symmetric, namely
\begin{align*}
S & = \{\Delta\in B \mid \Delta(u,v) = \Delta(v,u) \mbox{ for all }u,v\in G/G'\},\\
S' & = \{\Delta\in B\mid \Delta(u,v) = \Delta(v,u)^{-1}\mbox{ for all }u,v\in G/G'\}.
\end{align*}Every $\Delta\in B$ may be decomposed as
\[ \Delta(u,v) = (\Delta(u,v)\Delta(v,u))^{1/2} \cdot (\Delta(u,v)\Delta(v,u)^{-1})^{1/2}.\]
We then deduce that $B = SS'$ with $S\cap S'=\{\Delta_0\}$ is an inner direct product of $S$ and $S'$, where $\Delta_0$ denotes the trivial bilinear form which sends everything to the identity element.

Each bilinear form $\Delta\in B$ gives rise to an $\Aut(G)$-equivariant anti-homomorphism $\gamma : G\rightarrow \Aut(G)$ via (\ref{gamma-Delta}). Let us write
 $N_\Delta = N_\gamma$ for the associated normal regular subgroup of $\Hol(G)$, which corresponds to an element of $T(G)$ if and only if $N_\Delta$ is isomorphic to $G$. We know by \cite[Proposition 3.1]{LMH} that the isomorphism class of $N_\Delta$ depends only on the anti-symmetric component of $\Delta$. Since $N_{\Delta_0}$ is simply $\rho(G)$, as shown in \cite[Corollary 3.2]{LMH}, we have $N_\Delta\simeq G$ for all symmetric forms $\Delta \in S$. For anti-symmetric forms $\Delta\in S'$, things are much more complicated and it seems to us that not much can be said in general, except the few facts proven in \cite[Proposition 3.3]{LMH} (also see Remark \ref{remark}). But for a special family (\ref{Gpi}) of groups $G$ that we shall consider in Section \ref{group section}, there is a linear algebra method to determine whether $N_\Delta$ is isomorphic to $G$ for anti-symmetric forms $\Delta \in S'$ (see Proposition \ref{criterion} for the criterion).
 
Given any $\theta\Hol(G) \in T(G)$, consider its corresponding normal regular subgroup $\rho(G)^\theta$ of $\Hol(G)$. By Proposition \ref{gamma prop new}, there is a unique bilinear form $\Delta_\theta\in B$ for which $\rho(G)^\theta = N_{\Delta_\theta}$. Let us put
\begin{align*}
\mathcal{S} & = \{\theta\Hol(G)\in T(G)\mid\Delta_\theta\in S\},\\
\mathcal{S}' & = \{\theta\Hol(G) \in T(G) \mid \Delta_\theta\in S'\}.
\end{align*}
It was shown in \cite[Section 4]{LMH} that
\[ T(G)= \mathcal{S}\rtimes \mathcal{S}'\mbox{ whenever }\Aut^c(G) =1.\]
But the condition $\Aut^c(G) =1$ was imposed there only for the sake of simplicity, and essentially the same argument shows that this in fact holds more generally. Some modifications do need to be made, so let us explain this carefully.

Given any $\theta\Hol(G)\in T(G)$ with $1^\theta = 1$, the exact same argument as in \cite{LMH} shows that $\theta$ induces, via restriction, automorphisms
\begin{equation}\label{res def} \res_c(\theta) : G/G' \rightarrow G/G'\mbox{ and }\res_z(\theta): G'\rightarrow G'. \end{equation}
Notice that the coset representative $\theta$ is unique up to $\Aut(G)$. In the case that $\Aut^c(G)=1$, both of the above are then independent of the choice of $\theta$, so as in \cite{LMH} one gets a well-defined homomorphism
\begin{align}\label{old res}
\res: T(G) &\rightarrow \Aut(G/G')\times \Aut(G')\\\notag
 \theta\Hol(G) &\mapsto (\res_c(\theta),\res_z(\theta)).
 \end{align}
 However, this is no longer well-defined when $\Aut^c(G)\neq 1$ and we need to change the range of $\res$ to avoid this problem. 

Let $\Gamma(G)$ denote the image of the  
 natural homomorphism
\begin{align*}
\Aut(G)&\rightarrow \Aut(G/G')\times \Aut(G')\\
\beta&\mapsto (\res_c(\beta),\res_z(\beta))
\end{align*}
induced by restrictions. Equivalently, in the current setting
\[ \Gamma(G) = \{(\alpha,\hat{\alpha}) : \alpha \in \Aut^c(G)\}.\]
Let $\mathrm{Norm}(\Gamma(G))$ denote its normalizer in $\Aut(G/G')\times \Aut(G')$.
 
\begin{lemma}\label{res image}For any $\theta\Hol(G)\in T(G)$ with $1^\theta = 1$, we have
\[ (\res_c(\theta),\res_z(\theta))\in \mathrm{Norm}(\Gamma(G)).\]
\end{lemma}

\begin{proof}Note that $\theta$ is an element of $\NHol(G)$. This means that $\theta$ normalizes $\Hol(G)$, but it fixes the identity, so it also normalizes $\Aut(G)$. For any $\beta\in \Aut(G)$, we then have $\theta^{-1}\beta\theta\in \Aut(G)$ and so
\begin{align*}
&(\res_c(\theta),\res_z(\theta))^{-1}(\res_c(\beta),\res_z(\beta))(\res_c(\theta),\res_z(\theta)) \\
&\hspace{5cm}=
(\res_c(\theta^{-1}\beta\theta),\res_z(\theta^{-1}\beta\theta))
\end{align*}
is again an element of $\Gamma(G)$.
\end{proof}

Returning to the discussion in (\ref{res def}), since the $\theta$ there is unique up to an element of $\Aut(G)$, the coset
\[ (\res_c(\theta) ,\res_z(\theta))\Gamma(G)\]
does not depend on the choice of $\theta$. Thus, by Lemma \ref{res image}, we obtain a well-defined homomorphism
\begin{align}\label{new res}
 \res : T(G) &\rightarrow \mathrm{Norm}(\Gamma(G))/\Gamma(G)\\\notag
 \theta\Hol(G)&\mapsto (\res_c(\theta),\res_z(\theta)) \Gamma(G).
 \end{align}
 In the case that $\Aut^c(G)=1$, the subgroup $\Gamma(G)$ is trivial and we get back the map (\ref{old res}) that we had in \cite{LMH}.

\begin{lemma}\label{action lemma}
For any $\Delta\in B$ and $(\alpha_c,\alpha_z)\in \mathrm{Norm}(\Gamma(G))$, define
\[ \Delta^{(\alpha_c,\alpha_z)}(u,v) = \Delta(u^{\alpha_c^{-1}},v^{\alpha_c^{-1}})^{\alpha_z}\]
for all $u,v\in G/G'$. Then $\Delta^{(\alpha_c,\alpha_z)}$ also belongs to $B$.
\end{lemma}

\begin{proof}Clearly $\Delta^{(\alpha_c,\alpha_z)}$ is a $G'$-valued bilinear form on $G/G'$, and the issue here is whether $\Delta^{(\alpha_c,\alpha_z)}$ also satisfies (\ref{Delta2}), or equivalently (\ref{Delta1}). Let $\beta\in \Aut(G)$, and since $(\alpha_c,\alpha_z)\in \mathrm{Norm}(\Gamma(G))$, we have
\[ (\alpha_c\res_c(\beta)\alpha_c^{-1},\alpha_z\res_z(\beta)\alpha_z^{-1}) = (\res_c(\beta_0),\res_z(\beta_0))
\]
for some $\beta_0\in \Aut(G)$. For any $u,v\in G/G'$, we then have
\begin{align*}
\Delta^{(\alpha_c,\alpha_z)}(u^\beta,v^\beta) 
& = \Delta(u^{\beta\alpha_c^{-1}},v^{\beta\alpha_c^{-1}})^{\alpha_z}\\
& = \Delta(u^{\alpha_c^{-1}\cdot \alpha_c\beta\alpha_c^{-1}}, v^{\alpha_c^{-1}\cdot \alpha_c\beta\alpha_c^{-1}})^{\alpha_z}\\
& = \Delta(u^{\alpha_c^{-1}\beta_0},v^{\alpha_c^{-1}\beta_0})^{\alpha_z}\\
& = \Delta(u^{\alpha_c^{-1}},v^{\alpha_c^{-1}})^{\beta_0\alpha_z} \\
&= \Delta(u^{\alpha_c^{-1}},v^{\alpha_c^{-1}})^{\alpha_z\beta}\\
& = \Delta^{(\alpha_c,\alpha_z)}(u,v)^\beta,
\end{align*}
where in the third last equality, we used the fact that $\Delta\in B$ satisfies (\ref{Delta1}). This shows that indeed $\Delta^{(\alpha_c,\alpha_z)}$ is an element of $B$.
\end{proof}

Lemma \ref{action lemma} shows that we have a right action of $\mathrm{Norm}(\Gamma(G))$ on the abelian group $B$. Notice that both $S$ and $S'$ are invariant under this action. Moreover, since every $\Delta\in B$ satisfies (\ref{Delta2}), this action factors through $\Gamma(G)$. Multiplication of elements in $T(G)$, when translated to the corresponding bilinear forms, may then be expressed as follows.

\begin{prop}\label{multiplication prop} For any $\theta_1\Hol(G),\theta_2\Hol(G)\in T(G)$, we have
\[ \Delta_{\theta_1\theta_2} = \Delta_{\theta_1}^{\res(\theta_2)}\Delta_{\theta_2}.\]
\end{prop}

\begin{proof}
The proof of \cite[Proposition 4.1]{LMH} verbatim.
\end{proof}

\begin{prop}\label{kernel prop}We have $\mathcal{S} = \ker(\res)$.
\end{prop}
\begin{proof}The proof of \cite[Proposition 4.2]{LMH} verbatim, except that for our generalized $\res$, when $\theta\Hol(G)\in \ker(\res)$ we only know that there exists $\beta\in \Aut(G)$ such that
\[ (\res_c(\theta),\res_z(\theta) ) = (\res_c(\beta),\res_z(\beta)).\]
But we simply replace the equation in \cite[Proposition 4.2]{LMH} by
\[ [x^\beta,y^\beta]=[x,y]^\beta = [x,y]^\theta = [x^\theta,y^\theta]_\circ = [x^\beta,y^\beta]_\circ\mbox{ for all }x,y\in G\]
and the same argument shows that $\Delta_{\theta}$ is symmetric.
 \end{proof}

\begin{prop}\label{sd prop} We have $T(G) = \mathcal{S}\rtimes \mathcal{S}'$.
\end{prop}

\begin{proof}
We have $\mathcal{S}\cap\mathcal{S}' =1$ because $S\cap S' = 1$ and the trivial bilinear form corresponds to the identity element of $T(G)$.
%The proof of \cite[Proposition 4.3]{LMH} verbatim. Here let us just explain the idea briefly.
Since $\mathcal{S}$ is a normal subgroup of $T(G)$ by Proposition \ref{kernel prop}, it remains to verify that every element $\theta\Hol(G)\in T(G)$ belongs to the product $\mathcal{S}\mathcal{S}'$.

 To do so, we first decompose $\Delta_\theta = \Delta\Delta_2$ with $\Delta\in S$ and $\Delta_2\in S'$. We have $N_{\Delta_\theta}\simeq G$ because $\theta\Hol(G)\in T(G)$. Then $N_{\Delta_2}\simeq G$ holds by \cite[Proposition 3.1]{LMH}, which means that 
\[ \Delta_ 2 = \Delta_{\theta_2}\mbox{ for some }\theta_2\Hol(G)\in T(G).\]
Consider $\Delta_1 = \Delta^{\res(\theta_2)^{-1}}$, which is an element of $S$. By \cite[Corollary 3.2]{LMH}, we have  $N_{\Delta_1}\simeq G$ and so
\[ \Delta_1 = \Delta_{\theta_1}\mbox{ for some }\theta_1\Hol(G)\in T(G).\]
From Proposition \ref{multiplication prop}, we then deduce that
\[ \Delta_{\theta_1\theta_2} = \Delta_{\theta_1}^{\res(\theta_2)}\Delta_{\theta_2} =\Delta\Delta_2 =\Delta_\theta.\]
This shows that $\theta\Hol(G) = \theta_1\Hol(G)\cdot \theta_2\Hol(G)$, which is an element of $\mathcal{S}\mathcal{S}'$ because $\Delta_{\theta_1}\in S$ and $\Delta_{\theta_2}\in S'$.
 \end{proof}

Finally, we also have:

\begin{theorem}\label{pre thm}We have $T(G)\simeq S\rtimes \res(\mathcal{S}')$.
\end{theorem}

\begin{proof}The proof of \cite[Theorem 4.4]{LMH} verbatim.
\end{proof}

%Thus, the study of $T(G)$ reduces to that of $G'$-valued bilinear forms on $G/G'$. For the groups $G$ in Theorem \ref{thm1} as well as those in \cite{LMH}, both $G/G'$ and $G'$ are elementary $p$-abelian, so then the group $B$ may be computed using linear algebra techniques.

Regarding the structure of $T(G)$, for the symmetric part, we simply have to compute $S$ and this gives us a normal subgroup of $T(G)$. For the anti-symmetric part, however, the regular subgroup arising from a $\Delta\in S'$ need not be isomorphic to $G$. The multiplication in $S'$ does not correspond to that in $T(G)$ either, so in addition to computing $S'$, we also need to understand the structure of $\res(\mathcal{S}')$.

\begin{remark}\label{remark} As observed in \cite[Example 2.6]{LMH}, each $\sigma\in \End(G')$ yields an anti-symmetric bilinear form
\[ \Delta_{[\sigma]}: G/G' \times G/G' \rightarrow G';\,\ \Delta_{[\sigma]}(xG',yG') = [x,y]^\sigma.\]
In our setting, this is an element of $S'$ if and only if 
\begin{equation}\label{sigma commute} \sigma \hat{\alpha} = \hat{\alpha}\sigma\mbox{ for all }\alpha\in \Aut^c(G).\end{equation}
For each $\lambda\in \mathbb{Z}$, we in particular have
\[ \Delta_{[\lambda]} : G/G'\times G/G'\rightarrow G';\;\ \Delta_{[\lambda]}(xG',yG') = [x,y]^\lambda,\]
which is clearly an element of $S'$. These bilinear forms were first considered in \cite[Section 3]{class2}, where it was shown that
\[  N_{\Delta_{[\lambda]}} \simeq G\iff \lambda\not\equiv -\tfrac{1}{2}\hspace{-3mm}\pmod{p}.\]
In this case, the corresponding element of $T(G)$ is the power map
\[ \theta_{\kappa} : G\rightarrow G;\,\ x^{\theta_{\kappa}} = x^{\kappa},\]
where $\kappa$ is the multiplicative inverse of $1+2\lambda$ modulo the exponent $p^r$ of $G/Z(G)$. These power maps give rise to the cyclic subgroup
\begin{equation}\label{theta}
\{ \theta_\kappa : \kappa\in \mathbb{Z}\mbox{ coprime to }p\} \simeq (\mathbb{Z}/p^r\mathbb{Z})^\times\end{equation}
of $T(G)$, and this is exactly the content of \cite[Proposition 3.1]{class2}. For an arbitrary $\sigma\in \End(G')$, however, it appears that there is no simple way to determine whether $N_{\Delta_{[\sigma]}} \simeq G$ in general, except the necessary condition $1+2\sigma\in \Aut(G')$ as shown in \cite[Example 3.4]{LMH}. But as we previously noted, for a special family (\ref{Gpi}) of groups $G$, we have a criterion (to be stated in Proposition \ref{criterion}) which can be checked using linear algebra techniques.\end{remark}

In the case that $\Aut^c(G) =1$, the condition (\ref{Delta2}) is vacuous, and $B$ is the set of all $G'$-valued bilinear forms on $G/G'$. Then $T(G)$ is likely to be large, and this is how we found examples of $G$ with large $T(G)$ in \cite{LMH}. To get examples of $G$ with a small $T(G)$, we shall consider the case when $\Aut^c(G)$ is large so that (\ref{Delta2}) yields many relations. In fact, for the groups $G$ in Theorem \ref{thm1}, there are enough relations from (\ref{Delta2}) that $S=1$ is forced to be trivial, while $\mathcal{S}'\simeq \res(\mathcal{S}')$ is forced to be the subgroup (\ref{theta})  in (a),(b) and is only slightly bigger in (c).
