\section{A special family of finite $p$-groups of class two}\label{group section}

Let $p$ be an odd prime and let $n\geq 2$ be an integer. Let us review a construction, via linear algebra as described in \cite{CarIJM}, of $p$-groups of order $p^{n+{n\choose 2}}$ and class two.

Let $V$ be an $n$-dimensional vector space over $\mathbb{F}_p$ and let $\Lambda^2 V$ denote its exterior square. We shall write their additions multiplicatively and their scalar multiplications exponentially.

Now, let $\pi : V\rightarrow \Lambda^2V$ be any linear map. By fixing an $\mathbb{F}_p$-basis 
\[v_1,v_2,\dots,v_n\]
for $V$, in which case we have that
\[v_j \wedge v_k,\,\ 1\leq j < k \leq n\]
is an $\mathbb{F}_p$-basis for $\Lambda^2 V$, we may write $\pi$ in matrix form
\[ (\pi_{i,(j,k)}),\mbox{ where }1\leq i\leq n\mbox{ and }1\leq j<k\leq n,\]
with entries in $\mathbb{F}_p$. More precisely, for each $1\leq i\leq n$, we have
\[v_i^\pi =  \prod_{j<k} (v_j\wedge v_k)^{\pi_{i,(j,k)}}.\]
We can then construct a $p$-group $G = G_\pi$ via the presentation
\begin{align}\label{Gpi}\notag
G =\Bigg\langle x_1,x_2,\dots,x_n : &\, [[x_i,x_j],x_k] = 1\mbox{ for }1\leq i,j,k\leq n,\\[-10pt]
&\,  x_i^p = \prod_{j<k} [x_j,x_k]^{\pi_{i,(j,k)}} \mbox{ for }1\leq i\leq n\Bigg\rangle.
\end{align}
This group has order $p^{n+{n\choose 2}}$ and is special, namely
\[ G' = Z(G) = \mathrm{Frat}(G).\]
Moreover, we may identify
\begin{equation}\label{iden}
 G/G' = V\mbox{ and } G' = \Lambda^2 V
 \end{equation}
by associating each $x_iG'$ with $v_i$ and each $[x_j,x_k]$ with $v_j\wedge v_k$. Then the linear map $\pi$ corresponds to the $p$th power map $G/G'\rightarrow G'$ of $G$. For each $\alpha\in \GL(V)$, let us write $ \hat{\alpha}\in \GL(\Lambda^2 V)$ for the natural linear map induced by $\alpha$, given by
\[(v_j\wedge v_k)^{\hat{\alpha}} = v_j^\alpha \wedge v_k^\alpha\mbox{ for all }1\leq j<k\leq n.\]
As shown in \cite[Section 3]{CarIJM}, by setting
\[ \Aut^c(\pi) = \{ \alpha\in \GL(V) \mid \pi\hat{\alpha} = \alpha\pi\},\]
we have a natural isomorphism
\begin{equation}\label{iden Aut}
\Aut^c(G) \simeq\Aut^c(\pi).
\end{equation}
In particular, for any $\beta\in \Aut(G)$ such that $\beta\Aut_c(G)$ corresponds to $\alpha\in \Aut^c(\pi)$, its actions on $G/G'$ and $G'$, respectively, are those given by $\alpha$ and $\hat{\alpha}$ via the identifications (\ref{iden}). Notice that $\pi\hat{\alpha} = \alpha\pi$ is likely to yield more relations on $\alpha$ when $\pi$ has high rank. To find examples with large $\Aut^c(\pi)$, naturally we want to consider $\pi$ of small rank, and this is why we restrict to $\pi$ of rank one in this paper.

The structure of $T(G)$ for such groups $G$ was considered in \cite{LMH}. As noted in \cite[Proposition 5.1]{LMH}, the universal property of the exterior square and (\ref{iden}) imply that the $G'$-valued anti-symmetric bilinear forms on $G/G'$ are precisely the $\Delta_{[\sigma]}$ in Remark \ref{remark}.

\begin{prop}\label{criterion}Let $\sigma\in \End(G')$ be such that $1+2\sigma\in \Aut(G')$ and that (\ref{sigma commute}) holds. For any $\eta\in \Aut(G/G')$, the following are equivalent:
\begin{enumerate}[label=$(\alph*)$]
\item There exists $\theta\Hol(G)\in T(G)$ with $1^\theta =1$ such that $\Delta_\theta = \Delta_{[\sigma]}$ and $\res_c(\theta) = \eta$.
\item The equality $\pi\hat{\eta}(1+2\sigma) = \eta\pi$ is satisfied.
\end{enumerate}
Moreover, in this case, we have
\[ \res(\theta \Hol(G)) = (\eta,\hat{\eta}(1+2\sigma))\Gamma(G).\]
\end{prop}

\begin{proof}
Basically the same proof as \cite[Proposition 5.2]{LMH}. The same argument is valid as long as we impose the condition (\ref{sigma commute}) and replace the map $\res$ there with our generalization (\ref{new res}).

Let us note that $(\eta,\hat{\eta}(1+2\sigma))$ is indeed an element of $\mathrm{Norm}(\Gamma(G))$. This is because  by the condition (\ref{sigma commute}), we have
\begin{align*}
& \hspace{5mm}(\eta,\hat{\eta}(1+2\sigma))(\alpha,\hat{\alpha})(\eta,\hat{\eta}(1+2\sigma))^{-1} \\
& = (\eta\alpha\eta^{-1}, \hat{\eta}(1+2\sigma) \hat{\alpha}(1+2\sigma)^{-1}\hat{\eta}^{-1})\\
&=(\eta\alpha\eta^{-1},\hat{\eta}\hat{\alpha}\hat{\eta}^{-1})
\end{align*}
for any $(\alpha,\hat{\alpha}) \in \Gamma(G)$ with $\alpha\in \Aut^c(G)$.

Let us also note that whether the equality in (b) holds depends only on the coset $\eta\Aut^c(G)$. Indeed, if $\pi\hat{\eta}(1+2\sigma) = \eta\pi$ is satisfied, then by the condition (\ref{sigma commute}) and the identification (\ref{iden Aut}), we have
\begin{align*}
\pi\widehat{\eta\alpha}(1+2\sigma) & = \pi \hat{\eta}\hat{\alpha}(1+2\sigma)\\
& = \pi\hat{\eta} (1+2\sigma) \hat{\alpha}\\
& = \eta\pi \hat{\alpha}\\
& = \eta\alpha\pi
\end{align*}
for any $\alpha\in \Aut^c(G)$. This corresponds to the fact that 
\[ \theta\Hol(G) = \theta\beta\Hol(G)\]
for any $\beta\in \Aut(G)$, and so the existence of $\theta\Hol(G)$ in (a) also only depends on the coset $\eta\Aut^c(G)$.
\end{proof}

Therefore, under Assumption \ref{assumption}, we deduce that
\begin{align*}
\res(\mathcal{S}') & = \{(\eta,\hat{\eta}(1+2\sigma))\Gamma(G) : \eta\in \Aut(G/G'),\, \sigma\in \End(G')\\
&\hspace{3.25cm}\mbox{satisfying $1+2\sigma\in \Aut(G')$ and (\ref{sigma commute})},\\
&\hspace{3.25cm}\mbox{the equation $\pi\hat{\eta}(1+2\sigma) =\eta\pi$ holds}\},
\end{align*}
or by making the change of variables $\tau = 1+2\sigma$, that
\begin{align}\notag
\res(\mathcal{S}') & = \{(\eta,\hat{\eta}\tau)\Gamma(G) : \eta\in \Aut(G/G'),\, \tau\in \Aut(G')\\\label{S'}
&\hspace{3.25cm}\mbox{satisfying (\ref{sigma commute}) and $\pi\hat{\eta}\tau =\eta\pi$}\},
\end{align}
where $\mathcal{S}'$ is the subgroup of $T(G)$ defined as in Section \ref{bilinear form sec}. This implies that the structure of $\res(\mathcal{S}')$ may be computed by solving $\pi\hat{\eta}\tau =\eta\pi$ for $\eta\in \Aut(G/G')$ for each of the $\tau\in\Aut(G')$ satisfying (\ref{sigma commute}), and this is essentially a problem in linear algebra.

In the next subsections, we shall restrict to $n=3,4$ and determine the linear maps $\pi: V\rightarrow \Lambda^2V$ of rank one, up to a change of basis. We shall use the following remark in the proof.

\begin{remark}It is a well-known fact that every element of $\Lambda^2V$ may be written as a sum (or product in our notation) of at most $\lfloor \frac{n}{2}\rfloor$ decomposable $2$-vectors (wedge products $u\wedge v$), and there are elements that attain this bound. This follows immediately from the natural identification of elements of $\Lambda^2V$ as $n\times n$ anti-symmetric matrices, and from the usual canonical form for the latter.

Let us also notice that non-zero wedge products $u\wedge v$ correspond to the $2$-dimensional subspaces $\langle u,v\rangle$ of $V$, where a change of basis of  the latter results in scaling $u\wedge v$ by the determinant of the corresponding invertible matrix. \end{remark}
 
%Since we want $T(G)$ to have minimal order, as we noted in the discussion after Theorem \ref{pre thm}, naturally we want $\Aut^c(\pi)$ to be large. The extreme case is when $\pi = \pi_0$ is the trivial linear map, namely when  $G$ is the free $p$-group of class two and exponent $p$. Then $\Aut^c(\pi_0)$ equals the full group $\GL(V)$, and $T(G)$ has order $p-1$ by \cite[Theorem 5.2]{class2}. It seems natural to consider the case when $\pi$ has rank one next.
%
%For $n=2$, we have

\subsection{$3$-dimensional case}

In this subsection, we take $n=3$. Then
\[ \dim_{\mathbb{F}_p}(V) = 3,\,\ \dim_{\mathbb{F}_p}(\Lambda^2V) ={3\choose 2} = 3,\]
and the groups (\ref{Gpi}) that we construct are of order $p^{6}$. For $\pi$ of rank one, there are only two possibilities up to a change of basis.

\begin{prop}\label{rank one prop'} Let $\pi : V\rightarrow\Lambda^2V$ be a linear map of rank one.  

There exists a basis $v_1,v_2,v_3$ of $V$ such that
\[ v_2^\pi = v_3^\pi  =1\mbox{ and } v_1^\pi=\begin{cases}
(v_1\wedge v_2),\\
(v_2\wedge v_3).\\
\end{cases}\]
\end{prop}

\begin{proof}The hypothesis means that 
\[ \dim_{\mathbb{F}_p}(\ker(\pi))=2 \mbox{ and } \dim_{\mathbb{F}_p}(V^\pi)=1.\]
Let $\omega$ be a generator of $V^\pi$. We know that $\omega$ is expressible as a single wedge product because $n=3$. This implies that $\omega\in U\wedge U$ for some $2$-dimensional subspace $U$ of $V$. It is not hard to see that there exists a basis $v_1,v_2,v_3$ of $V$ such that
\[ v_1^\pi = (v_1\wedge v_2)\mbox{ with }U = \langle v_1,v_2\rangle \mbox{ and }\ker(\pi) =\langle v_2,v_3\rangle\]
when $U\cap \ker(\pi) = \langle v_2\rangle$ has dimension $1$, and
\[ v_1^\pi = (v_2\wedge v_3)\mbox{ with }U = \ker(\pi)= \langle v_2,v_3\rangle\]
when $U = \ker(\pi)=\langle v_2,v_3\rangle$ has dimension $2$.
\end{proof}

Next, let us compute $\Aut^c(\pi)$ for the two linear maps $\pi : V\rightarrow \Lambda^2V$ of rank one in Proposition \ref{rank one prop'}. We shall fix a basis $v_1,v_2,v_3$ of $V$ and write elements of $\GL(V)$ in matrix form
\[ \alpha = \begin{bmatrix}
a_{11} & a_{12} & a_{13} \\
a_{21} & a_{22} & a_{23} \\
a_{31} & a_{32} & a_{33} 
\end{bmatrix}\iff \begin{array}{c}
v_1^\alpha = v_1^{a_{11}} v_2^{a_{12}} v_3^{a_{13}},\\
v_2^\alpha=v_1^{a_{21}} v_2^{a_{22}} v_3^{a_{23}},\\
v_3^\alpha = v_1^{a_{31}} v_2^{a_{32}} v_3^{a_{33}}.
\end{array}\]
For the $\alpha\in \GL(V)$ such that $\pi\hat{\alpha} = \alpha\pi$, clearly $\ker(\pi)= \langle v_2,v_3\rangle$ is an invariant subspace of $\alpha$, which means that
\begin{equation}\label{alpha'}
 \alpha = \begin{bmatrix}
a_{11} & a_{12} & a_{13} \\
0 & a_{22} & a_{23} \\
0 & a_{32} & a_{33} 
\end{bmatrix}.
 \end{equation}
Note that then $a_{11}\neq 0$ since $\alpha$ is invertible, and we have $\alpha \in \Aut^c(\pi)$ exactly when $v_1^{\pi\hat{\alpha}}= v_1^{\alpha\pi}$. Moreover, in order for $v_1^{\pi\hat{\alpha}}= v_1^{\alpha\pi}$ to hold, the image $\langle v_1^\pi\rangle$ of $\pi$ must be invariant under $\hat{\alpha}$.

\begin{prop}\label{auto1'}Let $\pi :V \rightarrow\Lambda^2 V$ be the linear map defined by
\[ v_2^\pi = v_3^\pi=1\mbox{ and }v_1^\pi = (v_1\wedge v_2).\]
Then $\Aut^c(\pi) = P\rtimes Q$ is a semidirect product of the subgroups
\begin{align*}
P & = \left\{ 
\begin{bmatrix}
1 & b & 0 \\
0 & 1 & 0\\
0 & c & 1 
\end{bmatrix} : b,c,\in \mathbb{F}_p
\right\},\\
Q & = \left\{ 
\begin{bmatrix}
s & 0 & 0\\
0 & 1 & 0\\
0 & 0 & t
\end{bmatrix}: s,t\in\mathbb{F}_p^\times
\right\}.
\end{align*}
\end{prop}

\begin{proof}Let $\alpha\in \GL(V)$ be as in (\ref{alpha'}). For $\langle v_1^\pi\rangle$ to be invariant under $\hat{\alpha}$, necessarily $\langle v_1,v_2\rangle$ is invariant under $\alpha$. This means that
\[a_{13} = a_{23} =0,\]
and we see that
\begin{align*}
 v_1^{\pi\hat{\alpha}}= v_1^{\alpha\pi}&\iff v_1^{a_{11}}v_2^{a_{12}} \wedge v_2^{a_{22}} = (v_1\wedge v_2)^{a_{11}}\\& \iff a_{22} = 1.
 \end{align*}
It follows that $ \alpha \in \Aut^c(\pi)$ if and only if $\alpha$ has the stated form below. It is not hard to check that we have a surjective homomorphism
\begin{align*}
\Aut^c(\pi) & \twoheadrightarrow Q\\
\begin{bmatrix}
a_{11} & a_{12} & 0 \\
0 & 1& 0 \\
0 & a_{32} & a_{33} 
\end{bmatrix}&\mapsto  \begin{bmatrix}
a_{11} & 0 & 0\\
0 & 1 & 0 \\
0 &  0&a_{33}
\end{bmatrix}
\end{align*}
with kernel $P$, and we deduce that $\Aut^c(\pi) = P\rtimes Q$.
\end{proof}

\begin{prop}\label{auto2'}Let $\pi :V \rightarrow\Lambda^2 V$ be the linear map defined by
\[ v_2^\pi = v_3^\pi  =1\mbox{ and }v_1^\pi = (v_2\wedge v_3).\]
 Then $\Aut^c(\pi) = P\rtimes Q$ is a semidirect product of the subgroups
\begin{align*}
P & = \left\{ 
\begin{bmatrix}
1 & b & c \\
0 & 1 &0\\
 0 & 0 & 1
\end{bmatrix} : b,c\in \mathbb{F}_p
\right\},\\
Q & =\left\{ 
\begin{bmatrix}
|A| &  \begin{matrix} 0 & 0 \end{matrix}\\
 \begin{matrix} 0 \\ 0 \end{matrix} & A
\end{bmatrix}:  A \in \GL_2(\mathbb{F}_p)
\right\}.
\end{align*}
\end{prop}
\begin{proof}
Let $\alpha\in \GL(V)$ be as in (\ref{alpha'}). We have
\begin{align*}
 v_1^{\pi\hat{\alpha}}= v_1^{\alpha\pi} &\iff v_2^{a_{22}}v_3^{a_{23}}  \wedge v_2^{a_{32}}v_3^{a_{33}} = (v_2\wedge v_3)^{a_{11}}\\
 & \iff \begin{vmatrix}a_{22} &a_{23}\\ a_{32} & a_{33}\end{vmatrix} =a_{11}.\end{align*}
 It follows that $ \alpha \in \Aut^c(\pi)$ if and only if $\alpha$ has the stated form below. It is not hard to check that we have a surjective homomorphism
\begin{align*}
\Aut^c(\pi) & \twoheadrightarrow Q\\
\begin{bmatrix}
\lvert\begin{smallmatrix}
a_{22} & a_{23}\\a_{32} & a_{33}
\end{smallmatrix}\rvert&a_{12}& a_{13}\\
0 & a_{22} & a_{23}\\
0 & a_{32} & a_{33}
\end{bmatrix}&\mapsto \begin{bmatrix}
\lvert\begin{smallmatrix}
a_{22} & a_{23}\\a_{32} & a_{33}
\end{smallmatrix}\rvert& 0 & 0\\
0 & a_{22} & a_{23}\\
0 & a_{32} & a_{33}
\end{bmatrix}\end{align*}
with kernel $P$, and we deduce that $\Aut^c(\pi) = P\rtimes Q$.
\end{proof}

\subsection{4-dimensional case}
In this subsection, we take $n=4$. Then
\[ \dim_{\mathbb{F}_p}(V) = 4,\,\ \dim_{\mathbb{F}_p}(\Lambda^2V) ={4\choose 2} = 6,\]
and the groups (\ref{Gpi}) that we construct are of order $p^{10}$. For $\pi$ of rank one, there are only three possibilities up to a change of basis.

\begin{prop}\label{rank one prop}Let $\pi : V\rightarrow\Lambda^2V$ be a linear map of rank one. 

There exists a basis $v_1,v_2,v_3,v_4$ of $V$ such that
\[ v_2^\pi = v_3^\pi = v_4^\pi =1\mbox{ and } v_1^\pi=\begin{cases}
(v_1\wedge v_2),\\
(v_3\wedge v_4),\\
(v_1\wedge v_2)(v_3\wedge v_4).
\end{cases}\]
%and $v_1^\pi$ is given by one of the following:
%\begin{enumerate}[label=$(\arabic*)$]
%\item $v_1^\pi= (v_1\wedge v_2)$,
%\item $v_1^\pi = (v_3\wedge v_4)$,
%\item $v_1^\pi= (v_1\wedge v_2)(v_3\wedge v_4)$.
%\end{enumerate}
\end{prop}

\begin{proof}The hypothesis means that 
\[ \dim_{\mathbb{F}_p}(\ker(\pi))=3 \mbox{ and } \dim_{\mathbb{F}_p}(V^\pi)=1.\]
Let $\omega$ be a generator of $V^\pi$. We know that either $\omega$ is a single wedge product or a product of two wedge products because $n=4$.

Assume first that $\omega$ is a single wedge product. Then $\omega\in U\wedge U$ for some $2$-dimensional subspace $U$ of $V$. It is not hard to see that there exists a basis $v_1,v_2,v_3,v_4$ of $V$ such that
\[ v_1^\pi = (v_1\wedge v_2)\mbox{ with }U = \langle v_1,v_2\rangle \mbox{ and }\ker(\pi) =\langle v_2,v_3,v_4\rangle\]
when $U\cap \ker(\pi) = \langle v_2\rangle$ has dimension $1$, and
\[ v_1^\pi = (v_3\wedge v_4)\mbox{ with }U = \langle v_3,v_4\rangle\mbox{ and } \ker(\pi)= \langle v_2,v_3,v_4\rangle\]
when $U\cap \ker(\pi) = \langle v_3,v_4\rangle$ has dimension $2$.

Assume next that $\omega = \omega_1\omega_2$ is a product of two wedge products $\omega_1$ and $\omega_2$, where $\omega_1\in U_1\wedge U_1$ and $\omega_2\in U_2\wedge U_2$ for some $2$-dimensional subspaces $U_1$ and $U_2$ of $V$. We may assume that $U_1\cap U_2$ is trivial, for otherwise $\omega$ is expressible as a single wedge product. In the case that
\[ \dim_{\mathbb{F}_p}(U_1\cap \ker(\pi)) =  \dim_{\mathbb{F}_p}(U_2\cap \ker(\pi)) = 1,\]
we can find a basis $u_1,u_2$ of $U_1$ and a basis $u_3,u_4$ of $U_2$ such that
\[ \omega_1 = (u_1\wedge u_2),\,\ \omega_2 = (u_3\wedge u_4), \,\ \ker(\pi) = \langle u_2,u_4,u_1^{-1}u_3\rangle.\]
But notice that we can rewrite
\[ \omega = \omega_1\omega_2 = (u_1\wedge u_2u_4)(u_1^{-1}u_3\wedge u_4).\]
Thus, by changing the choice of $\omega_1,\omega_2$, we may assume that
\[ \dim_{\mathbb{F}_p}(U_1\cap \ker(\pi)) = 1\mbox{ and } \dim_{\mathbb{F}_p}(U_2\cap \ker(\pi)) = 2.\]
We then deduce that 
\[ v_1^\pi = (v_1\wedge v_2)(v_3\wedge v_4)\mbox{ with } 
\begin{cases}U_1 = \langle v_1,v_2\rangle\\
U_2 = \langle v_3,v_4\rangle
\end{cases}\hspace{-1em}
\mbox{ and } \ker(\pi) =\langle v_2,v_3,v_4\rangle
 \]
 for a suitable basis $v_1,v_2,v_3,v_4$ of $V$.
\end{proof}

Next, we compute $\Aut^c(\pi)$ for the three linear maps $\pi : V\rightarrow \Lambda^2V$ of rank one in Proposition \ref{rank one prop}. We shall fix a basis $v_1,v_2,v_3,v_4$ of $V$ and write elements of $\GL(V)$ in matrix form
\[ \alpha = \begin{bmatrix}
a_{11} & a_{12} & a_{13} & a_{14}\\
a_{21} & a_{22} & a_{23} & a_{24}\\
a_{31} & a_{32} & a_{33} & a_{34}\\
a_{41} & a_{42} & a_{43} & a_{44}
\end{bmatrix}\iff  
\begin{array}{c}
v_1^\alpha = v_1^{a_{11}} v_2^{a_{12}} v_3^{a_{13}} v_4^{a_{14}}, \\
v_2^\alpha  = v_1^{a_{21}} v_2^{a_{22}} v_3^{a_{23}} v_4^{a_{24}},\\
v_3^\alpha  = v_1^{a_{31}} v_2^{a_{32}} v_3^{a_{33}} v_4^{a_{34}},\\
v_4^\alpha  = v_1^{a_{41}} v_2^{a_{42}} v_3^{a_{43}} v_4^{a_{44}}.
\end{array}\]
For the $\alpha\in \GL(V)$ such that $\pi\hat{\alpha} = \alpha\pi$, clearly $\ker(\pi)= \langle v_2,v_3,v_4\rangle$ is an invariant subspace of $\alpha$, which means that
\begin{equation}\label{alpha}
 \alpha = \begin{bmatrix}
a_{11} & a_{12} & a_{13} &a_{14}\\
0 & a_{22} & a_{23} &a_{24}\\
0 & a_{32} & a_{33}  &a_{34}\\
0 & a_{42} & a_{43} & a_{44}
\end{bmatrix}.
 \end{equation}
 Note that then $a_{11}\neq 0$ since $\alpha$ is invertible, and we have $\alpha \in \Aut^c(\pi)$ exactly when $v_1^{\pi\hat{\alpha}}= v_1^{\alpha\pi}$. Moreover, in order for  $v_1^{\pi\hat{\alpha}}= v_1^{\alpha\pi}$ to hold, the image $\langle v_1^\pi\rangle$ of $\pi$ must be invariant under $\hat{\alpha}$.
 
%The proofs of the following propositions involve matrix calculations. We shall omit some of the details as they can easily be verified, using symbolic matrix calculators say, such as \url{https://matrixcalc.org/}.

\begin{prop}\label{auto1}Let $\pi :V \rightarrow\Lambda^2 V$ be the linear map defined by
\[ v_2^\pi = v_3^\pi=v_4^\pi =1\mbox{ and }v_1^\pi = (v_1\wedge v_2).\]
Then $\Aut^c(\pi) = P\rtimes Q$ is a semidirect product of the subgroups
\begin{align*}
P & = \left\{ 
\begin{bmatrix}
1 & b & 0 & 0 \\
0 & 1 & 0 & 0\\
0 & c & 1 & 0\\
0 & d & 0 & 1
\end{bmatrix} : b,c,d\in \mathbb{F}_p
\right\},\\
Q & = \left\{ 
\begin{bmatrix}
s & 0 & \begin{matrix} 0 & 0 \end{matrix}\\
0 & 1 & \begin{matrix} 0 & 0 \end{matrix}\\
\begin{matrix} 0 \\ 0 \end{matrix} & \begin{matrix} 0 \\ 0 \end{matrix} & A
\end{bmatrix}: s\in\mathbb{F}_p^\times,\, A \in \GL_2(\mathbb{F}_p)
\right\}.
\end{align*}
\end{prop}
\begin{proof}Let $\alpha\in \GL(V)$ be as in (\ref{alpha}). For $\langle v_1^\pi\rangle$ to be invariant under $\hat{\alpha}$, necessarily $\langle v_1,v_2\rangle$ is invariant under $\alpha$. This means that 
\[a_{13} = a_{14} = a_{23} = a_{24} =0,\]
and we see that
\begin{align*}
 v_1^{\pi\hat{\alpha}}= v_1^{\alpha\pi}&\iff v_1^{a_{11}}v_2^{a_{12}} \wedge v_2^{a_{22}} = (v_1\wedge v_2)^{a_{11}}\\& \iff a_{22} = 1.
 \end{align*}
 It follows that $ \alpha \in \Aut^c(\pi)$ if and only if $\alpha$ has the stated form below. It is not hard to check that we have a surjective homomorphism
\begin{align*}
\Aut^c(\pi) & \twoheadrightarrow Q\\
\begin{bmatrix}
a_{11} & a_{12} & 0 & 0\\
0 & 1 & 0 & 0 \\
0 & a_{32} & a_{33} & a_{34} \\
0 & a_{42} & a_{43} & a_{44}
\end{bmatrix}&\mapsto  \begin{bmatrix}
a_{11} & 0 & 0 & 0\\
0 & 1 & 0 & 0 \\
0 & 0& a_{33} & a_{34} \\
0 & 0 & a_{43} & a_{44}
\end{bmatrix}
\end{align*}
with kernel $P$, and we deduce that $\Aut^c(\pi) = P\rtimes Q$.
\end{proof}

\begin{prop}\label{auto2}Let $\pi :V \rightarrow\Lambda^2 V$ be the linear map defined by
\[ v_2^\pi = v_3^\pi=v_4^\pi =1\mbox{ and }v_1^\pi = (v_3\wedge v_4).\]
 Then $\Aut^c(\pi) = P\rtimes Q$ is a semidirect product of the subgroups
\begin{align*}
P & = \left\{ 
\begin{bmatrix}
1 & b & c & e \\
0 & 1 &d & f\\
0 & 0 & 1 & 0\\
0 & 0 & 0 & 1
\end{bmatrix} : b,c,d,e,f\in \mathbb{F}_p
\right\},\\
Q & =\left\{ 
\begin{bmatrix}
|A| & 0 & \begin{matrix} 0 & 0 \end{matrix}\\
0 & s & \begin{matrix} 0 & 0 \end{matrix}\\
\begin{matrix} 0 \\ 0 \end{matrix} & \begin{matrix} 0 \\ 0 \end{matrix} & A
\end{bmatrix}: s\in \mathbb{F}_p^\times,\, A \in \GL_2(\mathbb{F}_p)
\right\}.
\end{align*}
\end{prop}
\begin{proof}Let $\alpha\in \GL(V)$ be as in (\ref{alpha}). We have
\begin{align*}
v_1^{\pi\hat{\alpha}} =v_1^{\alpha\pi} & \iff   v_2^{a_{32}} v_3^{a_{33}} v_4^{a_{34}}\wedge v_2^{a_{42}} v_3^{a_{43}} v_4^{a_{44}} = (v_3\wedge v_4)^{a_{11}}\\
& \iff  \begin{vmatrix}
a_{32} & a_{33}\\
a_{42} & a_{43}
\end{vmatrix} =\begin{vmatrix} 
a_{32} & a_{34}\\
a_{42} & a_{44}
\end{vmatrix} =0,\, 
 a_{11}=\begin{vmatrix}
a_{33} & a_{34}\\
a_{43} & a_{44}
\end{vmatrix} .
\end{align*}
Observe that then $(a_{32},a_{42}) = (0,0)$ must hold, for otherwise $(a_{33},a_{43})$ and $(a_{34},a_{44})$ would be scalar multiples of each other by the first relation, and so $a_{11} = 0$ by the second relation. It follows that $\alpha\in \Aut^c(\pi)$ if and only if $\alpha$ has the stated form below. It is not hard to check that we have a surjective homomorphism
\begin{align*}
\Aut^c(\pi) & \twoheadrightarrow Q\\
\begin{bmatrix}
\lvert\begin{smallmatrix}
a_{33} & a_{34}\\
a_{43} & a_{44}
\end{smallmatrix}\rvert & a_{12} & a_{13} & a_{14}\\
0 & a_{22} & a_{23} & a_{24} \\
0 & 0 & a_{33} & a_{34} \\
0 & 0 & a_{43} & a_{44}
\end{bmatrix}&\mapsto \begin{bmatrix}
\lvert\begin{smallmatrix}
a_{33} & a_{34}\\
a_{43} & a_{44}
\end{smallmatrix}\rvert & 0 & 0& 0\\
0 & a_{22} & 0& 0 \\
0 & 0 & a_{33} & a_{34} \\
0 & 0 & a_{43} & a_{44}
\end{bmatrix}
\end{align*}
with kernel $P$, and we deduce that $\Aut^c(\pi) = P\rtimes Q$.
\end{proof}

\begin{prop}\label{auto3}Let $\pi :V \rightarrow\Lambda^2 V$ be the linear map defined by
\[ v_2^\pi = v_3^\pi = v_4^\pi = 1 \mbox{ and }v_1^\pi = (v_1\wedge v_2)(v_3\wedge v_4).\]
Then $\Aut^c(\pi) = P\rtimes Q$ is a semidirect product of the subgroups
\begin{align*}
P & = \left\{ 
\begin{bmatrix}
1 & b & -d & c\\
0 & 1 & 0 & 0\\
0 & c & 1 & 0\\
0 & d & 0 & 1
\end{bmatrix} : b,c,d\in \mathbb{F}_p
\right\},\\
Q & =\left\{ 
\begin{bmatrix}
|A| & 0 & \begin{matrix} 0 & 0 \end{matrix}\\
0 & 1 & \begin{matrix} 0 & 0 \end{matrix}\\
\begin{matrix} 0 \\ 0 \end{matrix} & \begin{matrix} 0 \\ 0 \end{matrix} & A
\end{bmatrix}: A \in \GL_2(\mathbb{F}_p)
\right\}.
\end{align*}
\end{prop}
\begin{proof}Let $\alpha\in \GL(V)$ be as in (\ref{alpha}). For $\langle v_1^\pi\rangle$ to be invariant under $\hat{\alpha}$, necessarily $\langle v_2\rangle$ is invariant under $\alpha$. This means that 
\[a_{23} = a_{24} =0,\]
and so we have
\begin{align*}
v_1^{\pi\hat{\alpha}}&= (v_1^{a_{11}} v_2^{a_{12}} v_3^{a_{13}} v_4^{a_{14}}\wedge  v_2^{a_{22}})(v_2^{a_{32}} v_3^{a_{33}} v_4^{a_{34}}\wedge v_2^{a_{42}} v_3^{a_{43}} v_4^{a_{44}} ),\\
v_1^{\alpha\pi}&=(v_1\wedge v_2)^{a_{11}}(v_3\wedge v_4)^{a_{11}}.
\end{align*}
We then see that $v_1^{\pi\hat{\alpha}} = v_1^{\alpha\pi}$ holds if and only if
\[a_{22} = 1,\, -a_{13}+\begin{vmatrix}
a_{32} & a_{33}\\
a_{42} & a_{43}
\end{vmatrix} = -a_{14} +\begin{vmatrix}
a_{32} & a_{34}\\
a_{42} & a_{44}
\end{vmatrix} =0,\,
 \begin{vmatrix}
a_{33} & a_{34}\\
a_{43} & a_{44}
\end{vmatrix} = a_{11}.\]
It follows that $\alpha\in \Aut^c(\pi)$ if and only if $\alpha$ has the stated form below. It is not hard to check that we have a surjective homomorphism
\begin{align*}
\Aut^c(\pi) & \twoheadrightarrow Q\\
\begin{bmatrix}
\lvert\begin{smallmatrix}
a_{33} & a_{34}\\
a_{43} & a_{44}
\end{smallmatrix}\rvert& a_{12} & \lvert\begin{smallmatrix}
a_{32} & a_{33}\\
a_{42} & a_{43}
\end{smallmatrix}\rvert & \lvert\begin{smallmatrix}
a_{32} & a_{34}\\
a_{42} & a_{44}
\end{smallmatrix}\rvert\\
0 & 1 & 0 & 0 \\
0 & a_{32} & a_{33} & a_{34} \\
0 & a_{42} & a_{43} & a_{44}
\end{bmatrix}&\mapsto \begin{bmatrix}
\lvert\begin{smallmatrix}
a_{33} & a_{34}\\
a_{43} & a_{44}
\end{smallmatrix}\rvert&0 &0 &0\\
0 & 1 & 0 & 0 \\
0 & 0 & a_{33} & a_{34} \\
0 & 0 & a_{43} & a_{44}
\end{bmatrix}\end{align*}
with kernel $P$, and we deduce that $\Aut^c(\pi) = P\rtimes Q$.
\end{proof}