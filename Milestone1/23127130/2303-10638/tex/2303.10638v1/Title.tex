\date{19 March 2023, 07:00 CET --- Version 1.05%
}

\title[$p$-groups of class two with a small multiple holomorph]
      {Finite $p$-groups of class two with\\ a small multiple holomorph}
      
\author{A.~Caranti}%${}^{\dag}$}

 \address[A.~Caranti]%
  {Dipartimento di Matematica\\
   Universit\`a degli Studi di Trento\\
   via Sommarive 14\\
   I-38123 Trento\\
   Italy\\\endgraf
   ORCiD: 0000-0002-5746-9294} 

 \email{andrea.caranti@unitn.it} 
 \urladdr{https://caranti.maths.unitn.it/}

\author{Cindy (Sin Yi) Tsang}

 \address[C.~Tsang]%
  {Department of Mathematics\\
   Ochanomizu University\\
   2-1-1 Otsuka\\
   Bunkyo-ku, Tokyo\\
   Japan\\\endgraf
   ORCiD: 0000-0003-1240-8102} 

 \email{tsang.sin.yi@ocha.ac.jp} 
 \urladdr{https://sites.google.com/site/cindysinyitsang/}

\subjclass[2020]{20B35 20D15 20D45}

 \keywords{holomorph, multiple holomorph, regular subgroups, finite $p$-groups of class two, bilinear forms}

\begin{abstract}We consider the quotient group $T(G)$ of the multiple holomorph by the holomorph of a finite $p$-group $G$ of class two for an odd prime $p$. By work of the first-named author, we know that $T(G)$ contains a cyclic subgroup of order $p^{r-1}(p-1)$, where $p^r$ is the exponent of the quotient of $G$ by its center. In this paper, we shall exhibit examples of $G$ (with $r = 1$) such that $T(G)$ has order exactly $p-1$, which is as small as possible.
\end{abstract}

 \thanks{%$\dag$ Corresponding author.  \endgraf
  The first-named author is a member of GNSAGA--INdAM, Italy. 
\\The second-named author acknowledges that this research was supported by JSPS KAKENHI Grant Number 21K20319.}

