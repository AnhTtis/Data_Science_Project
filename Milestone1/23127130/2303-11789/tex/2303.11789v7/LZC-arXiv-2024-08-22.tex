
%% bare_jrnl.tex
%% V1.4b
%% 2015/08/26
%% by Michael Shell
%% see http://www.michaelshell.org/
%% for current contact information.
%%
%% This is a skeleton file demonstrating the use of IEEEtran.cls
%% (requires IEEEtran.cls version 1.8b or later) with an IEEE
%% journal paper.
%%
%% Support sites:
%% http://www.michaelshell.org/tex/ieeetran/
%% http://www.ctan.org/pkg/ieeetran
%% and
%% http://www.ieee.org/

%%*************************************************************************
%% Legal Notice:
%% This code is offered as-is without any warranty either expressed or
%% implied; without even the implied warranty of MERCHANTABILITY or
%% FITNESS FOR A PARTICULAR PURPOSE!
%% User assumes all risk.
%% In no event shall the IEEE or any contributor to this code be liable for
%% any damages or losses, including, but not limited to, incidental,
%% consequential, or any other damages, resulting from the use or misuse
%% of any information contained here.
%%
%% All comments are the opinions of their respective authors and are not
%% necessarily endorsed by the IEEE.
%%
%% This work is distributed under the LaTeX Project Public License (LPPL)
%% ( http://www.latex-project.org/ ) version 1.3, and may be freely used,
%% distributed and modified. A copy of the LPPL, version 1.3, is included
%% in the base LaTeX documentation of all distributions of LaTeX released
%% 2003/12/01 or later.
%% Retain all contribution notices and credits.
%% ** Modified files should be clearly indicated as such, including  **
%% ** renaming them and changing author support contact information. **
%%*************************************************************************


% *** Authors should verify (and, if needed, correct) their LaTeX system  ***
% *** with the testflow diagnostic prior to trusting their LaTeX platform ***
% *** with production work. The IEEE's font choices and paper sizes can   ***
% *** trigger bugs that do not appear when using other class files.       ***                          ***
% The testflow support page is at:
% http://www.michaelshell.org/tex/testflow/



\documentclass[12pt,draftcls,onecolumn]{IEEEtran}
\usepackage{CJK}
%\usepackage{cite}
\usepackage{amssymb,amsmath,amsopn,amsfonts,graphicx,color,amsthm,mathrsfs,color}
\usepackage[sort&compress, numbers]{natbib}
\usepackage{algorithmic}
\usepackage{graphicx}
\usepackage{mathtools}
%\usepackage{subfig}
\usepackage{float}
\usepackage{multirow}
\usepackage{textcomp}
%\numberwithin{equation}{section}
\usepackage{epstopdf}
\usepackage[justification=centering]{caption}
\usepackage{subfigure}
\usepackage{array}
\usepackage{longtable}

%\usepackage{square, comma, sort&compress, numbers}
%\usepackage{colorlinks,citecolor=blue,urlcolor=blue}
%%% Packages
%\RequirePackage{amsthm,amsmath,amsfonts,amssymb,amsopn,amsfonts,graphicx,color,mathrsfs}
%%\RequirePackage[numbers]{natbib}
%\RequirePackage[square, comma, sort&compress, numbers]{natbib}
%%\RequirePackage[authoryear]{natbib}%% uncomment this for author-year citations
%\RequirePackage[colorlinks,citecolor=blue,urlcolor=blue]{hyperref}%% uncomment this for coloring bibliography citations and linked URLs
%%\RequirePackage{graphicx}%% uncomment this for including figures
%\crefname{figure}{Fig.}{Fig.}

\newtheorem{remark}{Remark}[section]
\newtheorem{theorem}{Theorem}[section]
\newtheorem{lemma}{Lemma}[section]
\newtheorem{proposition}{Proposition}[section]
\newtheorem{condition}{Condition}[section]
\newtheorem{assumption}{Assumption}[section]
\newtheorem{definition}{Definition}[section]
\newtheorem{corollary}{Corollary}[section]
%\newtheorem{theorem}{Theorem}[section]
%\newtheorem{definition}[theorem]{Definition}
%\newtheorem{lemma}[theorem]{Lemma}
%\newtheorem{remark}[theorem]{Remark}
%\newtheorem{proposition}[theorem]{Proposition}


\renewcommand\thesection{\arabic{section}}
\renewcommand\thesubsectiondis{\thesection.\arabic{subsection}}

\allowdisplaybreaks
\def\1{\boldsymbol{1}}
\def\A{\mathcal A}
\def\B{\mathscr B}
\def\({\Big(}
\def\){\Big)}
\def\<{\langle}
\def\>{\rangle}
\def\D{\Delta}
\def\dd{\,\text{d}}
\def\G{\mathcal G}
\def\L{\mathcal L}
\def\E{\mathbb E}
\def\G{\mathcal G}
\def\LL{\mathscr L}
\def\Gg{\mathscr G}
\def\P{\mathbb{P}}
\def\H{\mathcal H}
\def\HH{\mathscr H}
\def\F{\mathcal{F}}
\def\Ff{\mathscr F}
\def\N{\mathcal N}
\def\V{\textbf{Vec}}
\def\a{\alpha}
\def\b{\beta}
\def\g{\gamma}
\def\ba{\begin{array}}
\def\ea{\end{array}}
\def\ban{\begin{eqnarray*}}
\def\ean{\end{eqnarray*}}
\def\bann{\begin{eqnarray*}}
\def\eann{\end{eqnarray*}}
\def\bnaa{\begin{eqnarray}}
\def\enaa{\end{eqnarray}}
\def\bd{\begin{description}}
\def\ed{\end{description}}
\def\be{\begin{equation}}
\def\ee{\end{equation}}
\def\bna{\begin{eqnarray}}
\def\ena{\end{eqnarray}}
\def\bphi{{\overline \varphi}}
\def\oa{{\overline \alpha}}
\def\oF{{\overline F}}
\def\oP{{\overline \Phi}}
\def\oPs{{\overline \Psi}}
\def\d{\delta}
\def\dfrac{\displaystyle\frac}
\def\dinf{\displaystyle\inf}
\def\dint{\displaystyle\int}
\def\dlim{\displaystyle\lim}
\def\dliminf{\displaystyle\liminf}
\def\dlimsup{\displaystyle\limsup}
\def\dmax{\displaystyle\max}
\def\dmin{\displaystyle\min}
\def\dref#1{(\ref{#1})}
\def\dsum{\displaystyle\sum}
\def\dsup{\displaystyle\sup}
\def\dt{\Delta T}
\def\e{\varepsilon}
\def\eq{\stackrel{\triangle}{=}}
\def\F{{\cal F}}
\def\f{\phi}
\def\X{\mathscr X}
\def\Y{\mathscr Y}
\def\Z{\mathscr Z}
\def\hx{{\widehat x}}
\def\istack#1{\displaystyle\mathop{-\mkern-10mu-\mkern-10mu-
              \mkern-11mu\longrightarrow}_{\mbox{\scriptsize$#1$}} }
\def\l{\lambda}
\def\lmin{\lambda_{\mbox{min}}}
\def\lmax{\lambda_{\mbox{max}}}
\def\ln{\mbox{ln}}
\def\lref#1{[\ref{#1}]}
\def\m{\mu}
\def\mf#1#2#3#4{\left[\begin{array}{cc} #1 & #2 \\ #3 & #4 \end{array}
                \right]}
\def\mt#1#2{\left(\begin{array}{c} #1 \\ #2 \end{array}\right)}
\def\n{\nu}
\def\nn{\nonumber}
\def\O{\Omega}
\def\oa{{\overline \alpha}}
\def\ovf{{\overline \varphi}}
\def\oG{{\overline \Gamma}}
\def\ok{ }
\def\ox{{\overline {\cal X}}}
\def\pd#1#2{\frac{\partial #1}{\partial #2}}
\def\NI{\mbox{I}\!\mbox{N}}
\def\R{I\!\!R}
\def\s{\theta}
\def\t{\tau}
\def\tr{\mbox{\textup{Tr}}}
\def\tx{{\widetilde x}}
\def\ve{\varepsilon}
\def\vf{\varphi}
\def\x{\xi}
\def\N{{\cal N}}
\def\z{\zeta}
\def\p{\phi}

\def\nstack#1{\displaystyle\mathop{-\mkern-10mu-\mkern-10mu-
              \mkern-11mu\longrightarrow}_{\mbox{\scriptsize $N\to\infty$}
              }^{\mbox{\scriptsize $#1$}}}

% If IEEEtran.cls has not been installed into the LaTeX system files,
% manually specify the path to it like:
% \documentclass[journal]{../sty/IEEEtran}





% Some very useful LaTeX packages include:
% (uncomment the ones you want to load)


% *** MISC UTILITY PACKAGES ***
%
%\usepackage{ifpdf}
% Heiko Oberdiek's ifpdf.sty is very useful if you need conditional
% compilation based on whether the output is pdf or dvi.
% usage:
% \ifpdf
%   % pdf code
% \else
%   % dvi code
% \fi
% The latest version of ifpdf.sty can be obtained from:
% http://www.ctan.org/pkg/ifpdf
% Also, note that IEEEtran.cls V1.7 and later provides a builtin
% \ifCLASSINFOpdf conditional that works the same way.
% When switching from latex to pdflatex and vice-versa, the compiler may
% have to be run twice to clear warning/error messages.






% *** CITATION PACKAGES ***
%
%\usepackage{cite}
% cite.sty was written by Donald Arseneau
% V1.6 and later of IEEEtran pre-defines the format of the cite.sty package
% \cite{} output to follow that of the IEEE. Loading the cite package will
% result in citation numbers being automatically sorted and properly
% "compressed/ranged". e.g., [1], [9], [2], [7], [5], [6] without using
% cite.sty will become [1], [2], [5]--[7], [9] using cite.sty. cite.sty's
% \cite will automatically add leading space, if needed. Use cite.sty's
% noadjust option (cite.sty V3.8 and later) if you want to turn this off
% such as if a citation ever needs to be enclosed in parenthesis.
% cite.sty is already installed on most LaTeX systems. Be sure and use
% version 5.0 (2009-03-20) and later if using hyperref.sty.
% The latest version can be obtained at:
% http://www.ctan.org/pkg/cite
% The documentation is contained in the cite.sty file itself.






% *** GRAPHICS RELATED PACKAGES ***
%
\ifCLASSINFOpdf
  % \usepackage[pdftex]{graphicx}
  % declare the path(s) where your graphic files are
  % \graphicspath{{../pdf/}{../jpeg/}}
  % and their extensions so you won't have to specify these with
  % every instance of \includegraphics
  % \DeclareGraphicsExtensions{.pdf,.jpeg,.png}
\else
  % or other class option (dvipsone, dvipdf, if not using dvips). graphicx
  % will default to the driver specified in the system graphics.cfg if no
  % driver is specified.
  % \usepackage[dvips]{graphicx}
  % declare the path(s) where your graphic files are
  % \graphicspath{{../eps/}}
  % and their extensions so you won't have to specify these with
  % every instance of \includegraphics
  % \DeclareGraphicsExtensions{.eps}
\fi
% graphicx was written by David Carlisle and Sebastian Rahtz. It is
% required if you want graphics, photos, etc. graphicx.sty is already
% installed on most LaTeX systems. The latest version and documentation
% can be obtained at:
% http://www.ctan.org/pkg/graphicx
% Another good source of documentation is "Using Imported Graphics in
% LaTeX2e" by Keith Reckdahl which can be found at:
% http://www.ctan.org/pkg/epslatex
%
% latex, and pdflatex in dvi mode, support graphics in encapsulated
% postscript (.eps) format. pdflatex in pdf mode supports graphics
% in .pdf, .jpeg, .png and .mps (metapost) formats. Users should ensure
% that all non-photo figures use a vector format (.eps, .pdf, .mps) and
% not a bitmapped formats (.jpeg, .png). The IEEE frowns on bitmapped formats
% which can result in "jaggedy"/blurry rendering of lines and letters as
% well as large increases in file sizes.
%
% You can find documentation about the pdfTeX application at:
% http://www.tug.org/applications/pdftex





% *** MATH PACKAGES ***
%
%\usepackage{amsmath}
% A popular package from the American Mathematical Society that provides
% many useful and powerful commands for dealing with mathematics.
%
% Note that the amsmath package sets \interdisplaylinepenalty to 10000
% thus preventing page breaks from occurring within multiline equations. Use:
%\interdisplaylinepenalty=2500
% after loading amsmath to restore such page breaks as IEEEtran.cls normally
% does. amsmath.sty is already installed on most LaTeX systems. The latest
% version and documentation can be obtained at:
% http://www.ctan.org/pkg/amsmath





% *** SPECIALIZED LIST PACKAGES ***
%
%\usepackage{algorithmic}
% algorithmic.sty was written by Peter Williams and Rogerio Brito.
% This package provides an algorithmic environment fo describing algorithms.
% You can use the algorithmic environment in-text or within a figure
% environment to provide for a floating algorithm. Do NOT use the algorithm
% floating environment provided by algorithm.sty (by the same authors) or
% algorithm2e.sty (by Christophe Fiorio) as the IEEE does not use dedicated
% algorithm float types and packages that provide these will not provide
% correct IEEE style captions. The latest version and documentation of
% algorithmic.sty can be obtained at:
% http://www.ctan.org/pkg/algorithms
% Also of interest may be the (relatively newer and more customizable)
% algorithmicx.sty package by Szasz Janos:
% http://www.ctan.org/pkg/algorithmicx




% *** ALIGNMENT PACKAGES ***
%
%\usepackage{array}
% Frank Mittelbach's and David Carlisle's array.sty patches and improves
% the standard LaTeX2e array and tabular environments to provide better
% appearance and additional user controls. As the default LaTeX2e table
% generation code is lacking to the point of almost being broken with
% respect to the quality of the end results, all users are strongly
% advised to use an enhanced (at the very least that provided by array.sty)
% set of table tools. array.sty is already installed on most systems. The
% latest version and documentation can be obtained at:
% http://www.ctan.org/pkg/array


% IEEEtran contains the IEEEeqnarray family of commands that can be used to
% generate multiline equations as well as matrices, tables, etc., of high
% quality.




% *** SUBFIGURE PACKAGES ***
%\ifCLASSOPTIONcompsoc
%  \usepackage[caption=false,font=normalsize,labelfont=sf,textfont=sf]{subfig}
%\else
%  \usepackage[caption=false,font=footnotesize]{subfig}
%\fi
% subfig.sty, written by Steven Douglas Cochran, is the modern replacement
% for subfigure.sty, the latter of which is no longer maintained and is
% incompatible with some LaTeX packages including fixltx2e. However,
% subfig.sty requires and automatically loads Axel Sommerfeldt's caption.sty
% which will override IEEEtran.cls' handling of captions and this will result
% in non-IEEE style figure/table captions. To prevent this problem, be sure
% and invoke subfig.sty's "caption=false" package option (available since
% subfig.sty version 1.3, 2005/06/28) as this is will preserve IEEEtran.cls
% handling of captions.
% Note that the Computer Society format requires a larger sans serif font
% than the serif footnote size font used in traditional IEEE formatting
% and thus the need to invoke different subfig.sty package options depending
% on whether compsoc mode has been enabled.
%
% The latest version and documentation of subfig.sty can be obtained at:
% http://www.ctan.org/pkg/subfig




% *** FLOAT PACKAGES ***
%
%\usepackage{fixltx2e}
% fixltx2e, the successor to the earlier fix2col.sty, was written by
% Frank Mittelbach and David Carlisle. This package corrects a few problems
% in the LaTeX2e kernel, the most notable of which is that in current
% LaTeX2e releases, the ordering of single and double column floats is not
% guaranteed to be preserved. Thus, an unpatched LaTeX2e can allow a
% single column figure to be placed prior to an earlier double column
% figure.
% Be aware that LaTeX2e kernels dated 2015 and later have fixltx2e.sty's
% corrections already built into the system in which case a warning will
% be issued if an attempt is made to load fixltx2e.sty as it is no longer
% needed.
% The latest version and documentation can be found at:
% http://www.ctan.org/pkg/fixltx2e


%\usepackage{stfloats}
% stfloats.sty was written by Sigitas Tolusis. This package gives LaTeX2e
% the ability to do double column floats at the bottom of the page as well
% as the top. (e.g., "\begin{figure*}[!b]" is not normally possible in
% LaTeX2e). It also provides a command:
%\fnbelowfloat
% to enable the placement of footnotes below bottom floats (the standard
% LaTeX2e kernel puts them above bottom floats). This is an invasive package
% which rewrites many portions of the LaTeX2e float routines. It may not work
% with other packages that modify the LaTeX2e float routines. The latest
% version and documentation can be obtained at:
% http://www.ctan.org/pkg/stfloats
% Do not use the stfloats baselinefloat ability as the IEEE does not allow
% \baselineskip to stretch. Authors submitting work to the IEEE should note
% that the IEEE rarely uses double column equations and that authors should try
% to avoid such use. Do not be tempted to use the cuted.sty or midfloat.sty
% packages (also by Sigitas Tolusis) as the IEEE does not format its papers in
% such ways.
% Do not attempt to use stfloats with fixltx2e as they are incompatible.
% Instead, use Morten Hogholm'a dblfloatfix which combines the features
% of both fixltx2e and stfloats:
%
% \usepackage{dblfloatfix}
% The latest version can be found at:
% http://www.ctan.org/pkg/dblfloatfix




%\ifCLASSOPTIONcaptionsoff
%  \usepackage[nomarkers]{endfloat}
% \let\MYoriglatexcaption\caption
% \renewcommand{\caption}[2][\relax]{\MYoriglatexcaption[#2]{#2}}
%\fi
% endfloat.sty was written by James Darrell McCauley, Jeff Goldberg and
% Axel Sommerfeldt. This package may be useful when used in conjunction with
% IEEEtran.cls'  captionsoff option. Some IEEE journals/societies require that
% submissions have lists of figures/tables at the end of the paper and that
% figures/tables without any captions are placed on a page by themselves at
% the end of the document. If needed, the draftcls IEEEtran class option or
% \CLASSINPUTbaselinestretch interface can be used to increase the line
% spacing as well. Be sure and use the nomarkers option of endfloat to
% prevent endfloat from "marking" where the figures would have been placed
% in the text. The two hack lines of code above are a slight modification of
% that suggested by in the endfloat docs (section 8.4.1) to ensure that
% the full captions always appear in the list of figures/tables - even if
% the user used the short optional argument of \caption[]{}.
% IEEE papers do not typically make use of \caption[]'s optional argument,
% so this should not be an issue. A similar trick can be used to disable
% captions of packages such as subfig.sty that lack options to turn off
% the subcaptions:
% For subfig.sty:
% \let\MYorigsubfloat\subfloat
% \renewcommand{\subfloat}[2][\relax]{\MYorigsubfloat[]{#2}}
% However, the above trick will not work if both optional arguments of
% the \subfloat command are used. Furthermore, there needs to be a
% description of each subfigure *somewhere* and endfloat does not add
% subfigure captions to its list of figures. Thus, the best approach is to
% avoid the use of subfigure captions (many IEEE journals avoid them anyway)
% and instead reference/explain all the subfigures within the main caption.
% The latest version of endfloat.sty and its documentation can obtained at:
% http://www.ctan.org/pkg/endfloat
%
% The IEEEtran \ifCLASSOPTIONcaptionsoff conditional can also be used
% later in the document, say, to conditionally put the References on a
% page by themselves.




% *** PDF, URL AND HYPERLINK PACKAGES ***
%
%\usepackage{url}
% url.sty was written by Donald Arseneau. It provides better support for
% handling and breaking URLs. url.sty is already installed on most LaTeX
% systems. The latest version and documentation can be obtained at:
% http://www.ctan.org/pkg/url
% Basically, \url{my_url_here}.




% *** Do not adjust lengths that control margins, column widths, etc. ***
% *** Do not use packages that alter fonts (such as pslatex).         ***
% There should be no need to do such things with IEEEtran.cls V1.6 and later.
% (Unless specifically asked to do so by the journal or conference you plan
% to submit to, of course. )


% correct bad hyphenation here
\hyphenation{op-tical net-works semi-conduc-tor}


\begin{document}
\begin{CJK}{GBK}{song}

% paper title
% Titles are generally capitalized except for words such as a, an, and, as,
% at, but, by, for, in, nor, of, on, or, the, to and up, which are usually
% not capitalized unless they are the first or last word of the title.
% Linebreaks \\ can be used within to get better formatting as desired.
% Do not put math or special symbols in the title.
\title{Decentralized Online Learning for Random Inverse Problems Over Graphs}
%
%
% author names and IEEE memberships
% note positions of commas and nonbreaking spaces ( ~ ) LaTeX will not break
% a structure at a ~ so this keeps an author's name from being broken across
% two lines.
% use \thanks{} to gain access to the first footnote area
% a separate \thanks must be used for each paragraph as LaTeX2e's \thanks
% was not built to handle multiple paragraphs
%

\author{Tao Li, Xiwei Zhang and Yan Chen
\thanks{This work was supported  by the National Natural Science Foundation
of China under Grant 62261136550. Corresponding author: Tao Li (Email: tli@math.ecnu.edu.cn). }
\thanks{The authors are with the School of Mathematical Sciences, East China Normal University, Shanghai 200241, China.}
%\thanks{Xiwei Zhang and Yan Chen are with the School of Mathematical Sciences, East China Normal University, Shanghai 200241, China.}
}

% note the % following the last \IEEEmembership and also \thanks -
% these prevent an unwanted space from occurring between the last author name
% and the end of the author line. i.e., if you had this:
%
% \author{....lastname \thanks{...} \thanks{...} }
%                     ^------------^------------^----Do not want these spaces!
%
% a space would be appended to the last name and could cause every name on that
% line to be shifted left slightly. This is one of those "LaTeX things". For
% instance, "\textbf{A} \textbf{B}" will typeset as "A B" not "AB". To get
% "AB" then you have to do: "\textbf{A}\textbf{B}"
% \thanks is no different in this regard, so shield the last } of each \thanks
% that ends a line with a % and do not let a space in before the next \thanks.
% Spaces after \IEEEmembership other than the last one are OK (and needed) as
% you are supposed to have spaces between the names. For what it is worth,
% this is a minor point as most people would not even notice if the said evil
% space somehow managed to creep in.



% The paper headers
\markboth{Journal of \LaTeX\ Class Files, June~2023}%
{Shell \MakeLowercase{\textit{et al.}}: Bare Demo of IEEEtran.cls for IEEE Journals}
% The only time the second header will appear is for the odd numbered pages
% after the title page when using the twoside option.
%
% *** Note that you probably will NOT want to include the author's ***
% *** name in the headers of peer review papers.                   ***
% You can use \ifCLASSOPTIONpeerreview for conditional compilation here if
% you desire.




% If you want to put a publisher's ID mark on the page you can do it like
% this:
%\IEEEpubid{0000--0000/00\$00.00~\copyright~2015 IEEE}
% Remember, if you use this you must call \IEEEpubidadjcol in the second
% column for its text to clear the IEEEpubid mark.



% use for special paper notices
%\IEEEspecialpapernotice{(Invited Paper)}




% make the title area
\maketitle

% As a general rule, do not put math, special symbols or citations
% in the abstract or keywords.
\begin{abstract}
We propose a decentralized online learning algorithm for distributed random inverse problems over network graphs with online measurements, and  unifies the distributed parameter estimation in Hilbert spaces and the least mean square problem in reproducing kernel Hilbert spaces (RKHS-LMS). We transform the convergence of the algorithm into the asymptotic stability of a class of inhomogeneous random difference equations in Hilbert spaces with $L_{2}$-bounded martingale difference terms and develop the $L_2$-asymptotic stability theory in Hilbert spaces. We show that if the network graph is connected and the sequence of forward operators satisfies the \emph{infinite-dimensional spatio-temporal persistence of excitation} condition, then the estimates of all nodes are mean square and almost surely strongly consistent. Moreover, we propose a decentralized online learning algorithm in RKHS based on non-stationary online data streams, and prove that the algorithm is mean square and almost surely strongly consistent if the operators induced by the random input data satisfy the \emph{infinite-dimensional spatio-temporal persistence of excitation} condition.
\end{abstract}

\begin{IEEEkeywords}
Decentralized online learning, random inverse problem, reproducing kernel Hilbert space,
randomly time-varying difference equation, persistence of excitation.
\end{IEEEkeywords}

\section{Introduction}
\label{sec:introduction}
\IEEEPARstart{I}{nverse}  problems have wide applications such as medical imaging, geophysics and oil exploration (\cite{bertero}-\cite{colton}). An inverse problem is to determine the system input (cause) from the  system output (result).
%e.g., the principle of X-ray tomography in biomedicine%: the image of the internal cross-section of an object is produced by the energy decay measured by the rays on a given cross-section, which is essentially the problem of solving the linear integral equation of the first type associated with the Radon transform (\cite{Kirsch111}).
In reality, measurements are usually affected by external disturbances, and the  inverse problems with noisy measurements have been widely studied including the cases with deterministic noises (\cite{Engl123}-\cite{Tikhonov123}) and those with Gaussian white noises  (\cite{Bissantz}-\cite{Hohage}).
%In recent years, with the development of data mining based on online data streams (\cite{Silva, Jiang}), 3DVAR and Kalman filtering methods were %also introduced to statistical inverse problems based on dynamical systems (\cite{Iglesias1, Lu1, Jonesfg}).
It is of practical significance to consider inverse problems with both randomly time-varying forward operators and random measurement noises. For example, consider the online learning problem in RKHS. Let $\X\subseteq \mathbb R^n$ be the input space and $(\HH_K,\langle \cdot,\cdot \rangle _K)$ be the Hilbert space with Mercer kernel $K:\X\times \X \to \mathbb R$. At time instant $k$, the random (with unknown distribution) input data $x(k)\in \X$ and the output data $y(k)\in \mathbb R$ satisfy the measurement equation $y(k)=f_0(x(k))+v(k),\ k\ge 0$, where $f_0\in \HH_K$ is the unknown function, and $v(k)\in \mathbb R$ is the random measurement noise. The online learning problem in $\HH_K$ is to reconstruct $f_0$ based on the online data stream $\{(x(k),y(k))\}_{k=0}^{\infty}$. By the reproducing property of RKHS, the above measurement equation can be written as
\begin{equation}\label{0}
y(k)=H(k)f_0+v(k),~k\ge 0,
\end{equation}
where $H(k)$ is the randomly time-varying forward operator induced by the input data $x(k)$,
satisfying $H(k)f:=\langle f,K(x(k),\cdot)\rangle _K$, $\forall\ f\in \HH_K$.
Thus, the online learning problem in RKHS  comes down to an inverse problem associated with the measurement model (\ref{0}). Most of the existing  works on statistical inverse problems assumed the forward operators to be deterministic and time-invariant, which can not cover the measurement model (\ref{0})  (\cite{Bissantz}-\cite{JinB}).


In addition to online data streams, many practical problems are required to be solved in a decentralized or distributed information structure. The overall large amounts of data are usually divided into several data sets, and the learning process are performed with multiple parallel processors (\cite{Rosenblatt}).
Decentralized online learning algorithms
for finite-dimensional parameter estimation have been widely studied. Pioneering works on the decentralized online parameter estimation in finite-dimensional spaces were achieved in \cite{Lopes}-\cite{kar2011} and fruitful results were obtained in \cite{Lopes}-\cite{Ishihara}.
%Pioneering works on the decentralized online parameter estimation in finite-dimensional spaces were achieved in \cite{Lopes}-\cite{Ishihara}.
Specifically, the decentralized online learning algorithms with randomly spatio-temporal independent observation matrices were proposed in \cite{Lopes}-\cite{Abdolee} via the collaborative strategy of diffusion. Kar and Moura   established a distributed observability condition in  \cite{kar2011}.
%(global observability plus mean connectedness) with temporally independent observation sequences, under which the distributed estimates are consistent and asymptotically normal.
%Kar \emph{et al}. (\cite{kar2012}-\cite{kar20132}) proposed the decentralized parameter estimation algorithms based on consensus plus innovation with measurement and communication noises. It is required that the expectations of the regression matrices be known, and the graphs, regression matrices, measurement and communication noises be spatially i.i.d.
%Piggott \emph{et al}. (\cite{Piggott}-\cite{Piggott1}) proposed decentralized algorithms over fixed communication graphs  with time-dependent observation matrices. Ishihara and Alghunaim (\cite{Ishihara}) analyzed the convergence of decentralized online learning algorithms for the case with spatial independent observation matrices.

Infinite-dimensional supervised online learning in RKHS is another important topic of random inverse problems (\cite{Lyaqini}). Based on the systematic study of batch learning in \cite{Poggio}, rich results of online learning algorithms based on i.i.d. online data streams were obtained in \cite{Ying}-\cite{Deng}.
Ying and Pontil (\cite{Ying}) considered the least-square online gradient descent
algorithm in RKHS and presented a novel capacity independent approach to derive error bounds and convergence results for this algorithm.
Tarr$\grave{\text{e}}$s and Yao (\cite{Tarres}) showed that the online regularized algorithm can achieve the strong convergence rate of batch learning, and the weak convergence rate is optimal in the sense that it reaches the minimax and individual lower rates.
Dieuleveut and Bach (\cite{Dieuleveut}) considered the random-design least-squares regression problem within the RKHS framework, and showed that the averaged unregularized LMS algorithm with a given sufficient large step-size can attain optimal rates of convergence for a variety of regimes for the smoothnesses of the optimal prediction function in RKHS.
Shin \emph{et al}. (\cite{Shin}) proposed decentralized adaptive learning algorithms over graphs in RKHS in a deterministic framework. Deng \emph{et al}. (\cite{Deng}) used the multiplicative operator in the saddle point problem to carve out the communication structure of decentralized networks and proposed a distributed consensus-based online learning algorithm with  i.i.d. measurements. The nonlinear online learning problems in RKHS with spatio-temporal independent measurements were studied in  \cite{Mitra}-\cite{SmaleYao}.



Up to now,
the existing theory of inverse problems in statistical and stochastic frameworks are far from mature and the existing related results may be divided into three categories: (i) statistical inverse problems based on deterministic time-invariant compact forward operators in Hilbert spaces (\cite{Bissantz}-\cite{JinB}); (ii) distributed parameter estimation in finite-dimensional spaces (\cite{Lopes}-\cite{Ishihara}); (iii) decentralized learning based on stationary e.g., i.i.d. measurements in RKHS (\cite{Shin}-\cite{Bouboulis}). Some basic problems are still open, such as
\begin{itemize}
\item inverse problems with randomly time-varying forward operators;
\item to establish a unified framework for random inverse problems in infinite-dimensional Hilbert spaces, distributed parameter estimation problems in finite-dimensional spaces and online learning problems in reproducing kernel Hilbert spaces;
\item to develop a decentralized RKHS learning theory based on non-stationary data streams.
\end{itemize}


Motivated by the above problems, we consider a class of random inverse problems over graphs, establish a unified framework to deal with the above three types of problems, and propose a decentralized online learning algorithm in Hilbert spaces. The learning algorithm of each node in the network takes the form of consensus plus innovation as in \cite{kar20132}. The innovation term is to update the node' estimate by using the node's own measurement data, and the consensus term is a weighted sum of its own estimate and the estimates of its neighboring nodes. The forward operator of the measurement of each node is randomly time-varying and is not required to satisfy special statistical properties, such as temporal independence (the forward operator of each node over the graph is independent with respect to time), spatial independence (the forward operators of different nodes are independent of each other at each moment) and stationarity, and the random measurement noise is no longer restricted to  Gaussian white noises.

We weaken the constraints on the forward operators compared with the most of the existing studies on inverse problems (\cite{Bissantz}-\cite{JinB}). We consider the cases with general bounded linear forward operators instead of compact ones. Besides, we allow the forward operators to be randomly time-varying, which are not restricted to be deterministic and time-invariant. These general settings bring essential difficulties to the convergence  analysis of the algorithm.
Tarr$\grave{\text{e}}$s and Yao (\cite{Tarres}), Smale and Yao (\cite{SmaleYao}) transformed the online learning problem with i.i.d. data streams in RKHS into the inverse problem with the deterministic time-invariant Hilbert-Schmidt forward operator, and then obtained the convergence result by using the singular value decomposition (SVD) of the linear compact operators in the Hilbert space.
By using the i.i.d. properties of the data, Dieuleveut and Bach (\cite{Dieuleveut}) transformed the estimation error equation into the random difference equation equivalently, where the homogeneous part is deterministically time-invariant and the inhomogeneous part is the martingale difference sequence in the Hilbert space, from which the mean-square convergence of the algorithm were obtained by means of the spectral decomposition property of the compact operator. Note that the SVD of linear compact operators in Hilbert spaces is only applicable for the inverse problems with deterministic and time-invariant linear compact forward operators. The existing  methodologies in  \cite{Math1}-\cite{Mathe123}, \cite{Tarres}-\cite{Dieuleveut} and \cite{SmaleYao} are no longer applicable for our problems.
%Specifically, Smale and Yao \cite{SmaleYao}, and Tarres and Yao \cite{Tarres} transformed the online learning problem in reproducing kernel Hilbert space into an inverse problem with Hilbert-Schmidt operator based on independent and identically distributed measurements.
%Mathe and Pereverzev \cite{Math1, Mathe123} used discrete projection regularization methods to effectively estimate the solution of the statistical inverse problem; 3DVAR and Kalman filtering methods were applied to solve the statistical inverse problem (\cite{Iglesias1, Lu1, Jonesfg}), where Iglesias \emph{et al}. \cite{ Iglesias1} gave the Tikhonov iterative regularization method based on single and real-time measurements, respectively, and the upper bound on the mean squared error of the asymptotic regularization method (ARM) was given by Kalman-Bucy filtering and 3DVAR in the continuous timescale by Lu \emph{et al}. \cite{Lu1}, Jones and Simpson \cite{Jonesfg}; Lu and Mathe \cite{Lu100} presented the batch stochastic gradient descent (BSGD) algorithm by iteratively selecting the Nystr$\ddot{\text{o}}$m estimate of the forward operator and gives an upper bound on the mean squared error of the algorithm.
%Smale and Yao \cite{SmaleYao}, and Tarres and Yao \cite{Tarres} transformed the online learning problem in reproducing kernel Hilbert space into an inverse problem with Hilbert-Schmidt operator based on independent identically distributed measurements. The above mentioned research methodology is based on the singular value decomposition (SVD) of linear compact operators in Hilbert space, and this approach can effectively handle the inverse problem with deterministic time-invariant linear compact forward operators, however, it is not applicable to the case with randomly time-varying linear bounded operators.

Note that for the decentralized online learning in finite-dimensional spaces over random graphs based on non-stationary data, we  have established the stochastic spatio-temporal persistence of excitation (SSTPE) condition to ensure convergence of the algorithm in \cite{WLZ}.  However, for this kind of finite-dimensional excitation conditions, the information matrices are all required to be positive definite, i.e., the eigenvalues of the matrix have strictly positive lower bounds. Obviously, the SSTPE  condition  is not applicable for inverse problems in infinite-dimensional Hilbert spaces. It is known that even for a strictly positive compact operator,  the infimum of its eigenvalues is zero, and then the SSTPE  condition can not hold.

%In this paper, we give a unified framework for integrating the random inverse problem, the distributed parameter estimation problem in finite-dimensional spaces, and the least mean square problem in RKHS.
To this end, by means of measurability and integration theory of mappings with values in Banach spaces, spectral decomposition theory of bounded self-adjoint operators, and martingale convergence methods, we investigate the $L_p^q$-stability condition on the sequence of operator-valued random elements.
%We establish the theory of a class of $L_2$-asymptotic stability of infinite-dimensional inhomogeneous random difference equations with $L_2$-bounded martingale difference terms.
We transform the convergence analysis of the algorithm into the $L_2$-asymptotic stability of time-varying random difference equations in Hilbert spaces. Since the forward operators of inverse problems in infinite-dimensional Hilbert spaces usually do not have bounded inverses, the existing asymptotic stability theory on infinite-dimensional random difference equations with compressive operators in Hilbert spaces  cannot be applied to inverse problems (\cite{Ungureanu1}-\cite{zwzwxcbs}). To this end, we propose the $L_p^q$-stability condition on the sequence of the products of operator-valued random elements, and establish the $L_2$-asymptotic stability theory for a class of inhomogeneous  random difference equations in Hilbert spaces with $L_2$-bounded martingale difference terms. We give sufficient conditions on the stability of a class of operator-valued random sequences composed of forward operators and Laplacian matrices of graphs. We prove that if the graph is connected, and the sequence of forward operators satisfy the \emph{infinite-dimensional spatio-temporal persistence of excitation} condition, then all nodes' estimates are all mean square and almost surely strongly consistent with the unknown function of the inverse problem.

We develop a theory of decentralized online learning in RKHS. Almost the existing literature on online supervised learning in RKHS  are based on i.i.d. data (e.g.,  \cite{GUO}-\cite{Lin}),  while we propose a decentralized online learning algorithm based on non-stationary and non-independent data streams in RKHS. We establish the convergence condition by equivalently transforming the distributed learning problem in RKHS into the random inverse problems over graphs. Especially, if the graph has only one node and the random input data is i.i.d., then our algorithm degenerates to the centralized online learning algorithm without regularization parameters in \cite{Ying} and \cite{Dieuleveut}.
%Then under this theoretical framework, we give a sufficient condition for a class of operator-valued random element sequences, which are composed of random forward operators and Laplacian matrices of  graphs,  to be $L_2^2$-stable, i.e., the \emph{infinite-dimensional spatio-temporal persistence of excitation} condition, and establish the convergence condition for  decentralized algorithms of random inverse problems over network graphs.

%We develop a theory of decentralized learning in RKHS with non-stationary and non-independent data streams. Compared with the existing literature, our contributions are summarized as follows.


%The proposed algorithm involves the sequences of random forward operators induced by random input data. To develop a completely self-contained  theoretical framework,
%we regard  both vector-valued mappings and operator-valued mappings as random elements with values in topological spaces. Based on the analytical theory of Banach spaces,
%we further explore the probability structure of random elements with values in different topological spaces, including (i) the measurable structure of random elements taking values in a strong operator topological space; (ii) the properties of expectations and conditional expectations of random elements with values in a uniform operator topological space or a strong operator topological space; (iii) the properties of independence and conditional independence of  random elements with values in a topological space.

%\item

%\item To overcome the difficulties associated with the lack of information in solving the inverse problem from a single observation, existing regularization methods all require some knowledge of the a priori information of the solution of the inverse problem. For example, the Tikhonov, Landweber, and Gauss-Newton iterative regularization methods in \cite{Hanke3, Bakushinskii111, Kindermann} usually need to select the regularization parameters according to the smoothness of the solution, so as to achieve the regularization effect by stopping the iterations early. However, it is intrinsically difficult to obtain prior information of the solution to the inverse problem. Therefore, we propose a decentralized online learning strategy based on real-time observation data, which effectively avoids the dependence on the prior information of the solution to the inverse problem.

The rest of the paper is organized as follows. In Section II, the online random inverse problems over   graphs in  Hilbert spaces are formulated and the decentralized online learning algorithm is proposed. In Section III, the convergence of the algorithm is proved. In Section IV, the decentralized online learning problems in   reproducing kernel Hilbert spaces  are studied. In Sections V and VI, simulation results  and  conclusions are given, respectively.



The following notations will be used throughout the paper.  Denote $f\in \F_0$ if $f$ is measurable with respect to the $\sigma$-algebra $\F_0$. For any given sets $\{A_i,i\in \mathscr I\}$,
where $\mathscr I$ is a set of indices, denote $\sigma\left(\bigcup_{i\in \mathscr I}A_i\right)$ by $\bigvee_{i\in \mathscr I}A_i$. Let $\tau_{\text{N}}(\X)$ be the topology induced by the norm $\|\cdot\|_{\X}$ in a Banach space $\X$.  Let $L^p(\Omega;\X)$ be the Bochner space composed of all mappings %$f:\Omega\to \X$
which are strongly measurable with respect to $\tau_{\text{N}}(\X)$. Define
$\|f\|_{L^p(\Omega;\X)}:=\left(\int_{\Omega}\|f\|_{\X}^p\dd\P\right)^{\frac{1}{p}}<\infty$, $1\leq p<\infty$.
Denote $L^p(\Omega):=L^p(\Omega;\mathbb R)$. Let $\mathscr L(\mathscr Y,\mathscr Z)$ be the linear space of all bounded linear operators from the Banach space $\mathscr Y$ to the Banach space $\mathscr Z$, in particular, $\mathscr L(\mathscr Z):=\mathscr L( \mathscr Z,\mathscr Z)$. Let $\tau_{\text{S}}(\mathscr L(\mathscr Y, \mathscr Z))$ be the strong operator topology of $\mathscr L(\mathscr Y,\mathscr Z)$. %Given a compact metric space $(X,d)$ and  $0 \leq s \leq 1$, denote the whole continuous functions defined  on $X$  by $C( X )$ and  $\|f\|_{C^s( X)}=\|f\|_{\infty}+|f|_{C^s( X)}$, $\forall\ f \in C( X)$, where $\|f\|_{\infty}=\sup _{x \in  X}|f(x)|$, and
%$
%|f|_{C^s( X)}=\sup _{x \neq y} \frac{|f(x)-f(y)|}{(d(x, y))^s} .
%$
%Denote the H\"{o}lder space by $C^s( X)=\{f \in C( X):\|f\|_{C^s( X)}<\infty\}$ and $(C^s( X))^{*}$ is the set of bounded linear functionals on $C^s( X)$.

Let $(\X_i,\langle \cdot,\cdot\rangle _{\X_i})$ be a Hilbert space, where the norm induced by the inner product is defined by $\|x_i\|_{\X_i}:=\sqrt{\langle x_i,x_i\rangle _{\X_i}}$, $x_i\in \X_ i$, $i=1,\cdots,n$. The Hilbert direct sum space is denoted by $[\bigoplus_{i=1}^n\X_i=\left\{x=(x_1,\cdots,x_n):x_i\in \X_i,1\leq i\leq n\right\}$,
%\[\bigoplus_{i=1}^n\X_i=\left\{x=(x_1,\cdots,x_n):x_i\in \X_i,1\leq i\leq n\right\},\]
where the inner product is defined by $\langle x, y\rangle _{\bigoplus_{i=1}^n\X_i}:=\sum_{i=1}^n\langle x_i,y_i\rangle _{\X_i}$,$\forall\ x=(x_1,\cdots,x_n),y=(y_1,\cdots,y_n)\in \bigoplus_{i=1}^n\X_i$. Denote $\bigoplus_{i=1}^n\X:=\X^n$. Let $\X$ and $\mathscr Y$ be Hilbert spaces.  Denote the Kronecker product of the vector $\1_n\in \mathbb R^{n}$ and $\f\in \X$ by $\1_n\otimes f:=(f,\cdots,f)\in \X^n$ and the Kronecker product of the matrix $A\in \mathbb R^{n\times m}$ and $B\in \LL(\X,\Y)$ by
\ban
A\otimes B:=\begin{pmatrix}
a_{11}B & \cdots & a_{1m}B \\
\vdots & \ddots & \vdots \\
a_{n1}B & \cdots & a_{nm}B
\end{pmatrix}
\in \mathscr L(\X^m,\mathscr Y^n).
\ean
The operations and properties of operator matrices in Hilbert direct sum spaces can be found in \cite{sl}. Let $\X$ be a Hilbert space and $T\in \LL(\X)$. If $T^*=T$, i.e., $\langle Tx,y \rangle _{\X}=\langle x, Ty \rangle _{\X}$, $\forall\ x,y\in \X$, then $T$ is called the self-adjoint operator. If a self-adjoint operator $T\in \LL(\X)$ satisfies $\langle Tx,x \rangle _{\X}\ge 0$, $\forall\ x\in \X$, then $T$ is  positive self-adjoint, denoted by $T\ge 0$. Especially if $\langle Tx,x \rangle _{\X}> 0$ for any given $x\neq 0$ in $\X$, then $T$ is strictly positive self-adjoint, denoted by $T>0$.


%Then, we know that if the sample space of the probability measure $\rho$ is $\X$, then $\rho \in \left(C^s(\X)\right)^{*}$.

Let $\mathcal{G}=\{\mathcal{V},\mathcal{E}_{\mathcal{G}},\mathcal{A}_{\mathcal{G}}\}$ denote a weighted graph, where $\mathcal{V}=\{1,...,N\}$ is the set of nodes and $\mathcal{E}_{\mathcal{G}}$ is the set of edges. The unordered pair $(j,i)\in\mathcal{E}_\mathcal G$ if and only if there exists an edge between nodes $j$ and $i$. Denote the set of neighboring nodes of node $i$ by $\mathcal{N}_i=\{j\in \mathcal{V}:(j,i)\in
\mathcal{E}_{\mathcal{G}}\}$. The matrix $\mathcal{A}_{\mathcal{G}}=[a_{ij}]\in \mathbb{R}^{N\times N}$ is called the weighted adjacency matrix of $\G$, and for any given  $i,j\in \mathcal V$, $a_{ii}=0$ and $a_{ij}=a_{ji}>0$ if and only if $j\in \mathcal{N}_i$.
%The weight $a_{ij}$ of the graph $\G$ reflect the topological structure and connection strength between the nodes.
The Laplacian matrix of $\G$ is defined by $\mathcal{L}_{\mathcal{G}}=\mathcal{D}_{\mathcal{G}}-\mathcal{A}_{\mathcal{G}}$, where the degree matrix $\mathcal{D}_{\mathcal{G}}=\text{diag}\{\sum_{j=1}^Na_{1j},\sum_{j=1}^Na_{2j},\cdots,\sum_{j=1}^Na_{Nj}\}$.



\section {Online Learning for Random Inverse Problems over Graphs}


\subsection{Online Random Inverse Problems over Graphs}
Consider a distributed communication network modeled by a weighted  graph $\mathcal{G}=\{\mathcal{V},\mathcal{E}_{\mathcal{G}},\mathcal{A}_{\mathcal{G}}\}$ consisting of $N$ nodes. The measurement $y_i(k)$ of node $i$ at instant $k$ is given by
\begin{equation}\label{measuramentmodel}
y_i(k)=H_i(k)f_0+v_i(k),\ k=0,1,2,...\ i=1,\cdots,N,
\end{equation}
where $f_0\in \X$ to be estimated is an unknown element in the Hilbert space $\X$, the random forward operator  $H_i(k):\Omega\to \mathscr L(\X,\Y_i)$ is an operator-valued random element with values in $(\LL(\X,\Y_i),\tau_{\text{S}}(\LL(\X,\Y_i)))$, and the measurement noise $v_i(k):\Omega\to \Y_i$ is a random element with values in the Hilbert space $(\Y_i,\tau_{\text{N}}(\Y_i))$.

\begin{remark}
\label{remark1adda11}
For the finite-dimensional Euclidean space $\X=\mathbb R^n$, the measurement model (\ref{measuramentmodel}) has been widely studied in \cite{Lopes}-\cite{Ishihara}, where the forward operator $H_i(k)$ degenerates to the random observation matrix and $f_0\in \mathbb R^n$ degenerates to the unknown finite-dimensional parameter vector.  Kar \emph{et al}. (\cite{kar2012}) investigated the decentralized estimation algorithms with nonlinear measurement models, where they introduced separably estimable measurement models that generalize the observability condition in linear centralized estimation to nonlinear decentralized estimation. It is worth pointing out that even for the nonlinear measurement model in \cite{kar2012}, the unknown quantity to be estimated is a parameter vector in a finite-dimensional space, different from which, the unknown quantity $f_0$ in the measurement model (\ref{measuramentmodel}) can be a nonlinear function, which is an element in the infinite-dimensional Hilbert space $\X$. The measurement model (\ref{measuramentmodel}) is essentially different from that in \cite{kar2012}. This will be further clarified in Remark \ref{remarkaddadd6}.
\end{remark}

Denote $y(k)=(y_1(k),\cdots,y_N(k)):\Omega\to \bigoplus_{i=1}^N\Y_i$, $ v(k)=(v_1(k),\cdots,v_N(k)):\Omega\to \bigoplus_{i=1}^N\Y_i$ and $ H(k) = (H_1(k),\cdots,H_N(k)):\Omega\to \mathscr L(\X,\bigoplus_{i=1}^N\Y_i)$.
We can write (\ref{measuramentmodel}) in the compact form
\begin{equation}\label{compactform}
y(k)=H(k)f_0+v(k),\ k\ge 0.
\end{equation}
The online random inverse problem means reconstructing $f_0$ by real-time random measurements $\{y(k), k\geq0\}$.

The measurement equation (\ref{compactform}), which covers the existing models of inverse problems, is more general in the sense that the forward operator can be randomly time-varying. Besides, different from  the existing literature, which has a centralized information structure, here, the reconstruction of $f_0$ is constrained by the information structure of the graph, i.e., there is no centralized fusion center collecting the overall measurements $y(k)$, and at each moment $k$, node $i$ can only use its own observation $y_i(k)$ and its neighbors' estimates $f_j(k), j\in \mathcal N_i$ to give its next estimate $f_i(k+1):\Omega\to \X$ for $f_0$, i.e.,
$$f_i(k)\in \left(\bigvee_{s=0,1,2,...,k-1}\sigma(y_i(s);\tau_{\text{N}}(\Y_i))\right)\bigvee\left(\bigvee_{j\in \mathcal N_i\cup\{i\}}\sigma\left(f_j(k-1);\tau_{\text{N}}(\X)\right)\right),~i\in \mathcal V.$$

Here,  the problem of cooperatively estimating $f_0$ by the nodes over the graph based on each node's local measurements, its own and neighbors' estimates is called the \textbf{\emph{random inverse problem over the graph}}.

\begin{remark}
Most of the existing literature on inverse problems assume the forward operator to be a deterministic and time-invariant linear compact operator $H$, associated with the measurement equation %$y=Hf_0+v.$
\begin{equation}\label{nklowokkrkr}
y=Hf_0+v.
\end{equation}
Here, different from the existing literature, the forward operator in the measurement equation (\ref{compactform}) is allowed to be random and time-varying. In the classical inverse problem, the noise $v$ in the measurement equation (\ref{nklowokkrkr}) is modeled as a deterministic perturbation  (\cite{Benning}). In the statistical inverse problem, it is modeled as Gaussian white noise (\cite{Lu}-\cite{Math1}, \cite{Lu2}).
Based on the stochastic gradient descent (SGD) algorithm, Lu and Mathe (\cite{Lu100}), Jahn and Jin (\cite{Jahn}) and Jin and Lu (\cite{JinB}) obtained the centralized learning strategy by the random discrete sampling of the forward operator and minimizing the loss functional
%$\widetilde{J}:\X\to \mathbb R$, where
\bna\label{sunshi}
\widetilde{J}(x)=\frac{1}{2}\E\left[\|y-Hx\|^2\right],\ \forall\ x\in \X.
\ena
Specifically, Jahn and Jin (\cite{Jahn}) and Jin and Lu (\cite{JinB}) investigated the regularization property of SGD with a priori and a posteriori stopping rules, and Lu and Mathe (\cite{Lu100}) gave an upper bound on the estimation error of SGD with discrete  level. Recently, Iglesias \emph{et al.} (\cite{Iglesias1}) and Lu \emph{et al.}  (\cite{Lu1}) solved the statistical inverse problem based on real-time measurements $\{y(k):y(k)=Hf_0+v(k),k\ge 0\}$, where $\{v(k),k\ge 0\}$ is an i.i.d. Gaussian white noise sequence. The statistical inverse problems in \cite{Iglesias1}-\cite{Lu1} are special cases of the online random inverse problem with random forward operators.
\end{remark}


\subsection{Decentralized Online Learning Algorithm}\label{online}
Denote $\mathcal H(k)=\text{diag}\{H_1(k),\cdots,H_N(k)\}$.  Based on the loss functional (\ref{sunshi}), we consider minimizing the loss functional $J:\X^N \to \mathbb R$ with the Laplacian regularization term given by
\bna
\label{costfunctionLaplacianregularization}
J(f)=\frac{1}{2}\left(\E\Big[\|y(k)-\mathcal H(k)f\|_{\bigoplus_{i=1}^N\Y_i}^2\Big]+\left\langle \left(\L_{\G}\otimes I_{\X}\right )f,f\right\rangle _{\X^N}\right),~\forall\ f\in \X^N,
\ena
where $I_{\X}$ is the identical operator on the Hilbert space $\X$. The loss functional  $J(f)$ consists of two terms: the mean-square estimation error term $\E[\|y(k)-\mathcal H(k)f\|_{\bigoplus_{i=1}^N\Y_i}^2]$ and the Laplacian regularization term $\langle (\L_{\G}\otimes I_{\X})f,f\rangle _{\X^N}=\frac{1}{2}\sum_{i=1}^N\sum_{j=1}^Na_{ij}\|f_i-f_j\|^2_{\X}$, where $f_i\in \X$ is the $i$-th component of $f$.

Suppose $\H(k)\in L^2(\Omega;\mathscr L(\X^N,\bigoplus_{i=1}^N\Y_i))$ and $v(k)\in L^2(\Omega;\bigoplus_{i=1}^N\Y_i)$.
If the sequences $\{\H(k),k\ge 0\}$ and $\{v(k), k\ge 0\}$ are both i.i.d, then it follows from Definition \ref{fenbudingyi} and Proposition \ref{nlllwwieiie}.(b) that $\{y(k)-\H(k)f,k\ge 0\}$ is a sequence of i.i.d. random elements with values in the Hilbert space $(\bigoplus_{i=1}^N\Y_i,\tau_{\text{N}}(\bigoplus_{i=1}^N\Y_i))$.
Noting that $\H^*(k)\H(k)\in L^1(\Omega;\LL(\X^N))$, by Proposition \ref{vnwlssfweewwfew}, the gradient operator $\text{grad}~J:\X^N\to \X^N$ is given by
%\begin{align*}
% &\text{grad}~J(f)\\
% =& \frac{1}{2}\text{grad}~\left(\E\left[\|y(k)-\mathcal H(k)f\|_{\bigoplus_{i=1}^N\Y_i}^2\right]+\left\langle \left(\L_{\G}\otimes I_{\X}\right)f,f\right\rangle _{\X^N}\right) \\
% =&\frac{1}{2}\text{grad}~\E\left[\langle \H^*(k)\H(k)f,f\rangle _{\X^N}\right]-\text{grad}~\E\left[\langle \H(k)f,y(k) \rangle _{\bigoplus_{i=1}^N\Y_i}\right] \\&+\frac{1}{2}
%\text{grad}~\left\langle \left(\L_{\G}\otimes I_{\X}\right)f,f\right\rangle _{\X^N}\\
% =&\frac{1}{2}\text{grad}~\langle \E\left[\H^*(k)\H(k)\right]f,f\rangle _{\X^N}-\text{grad}~\E\left[\langle \H(k)f,y(k) \rangle _{\bigoplus_{i=1}^N\Y_i}\right]\\ & +\frac{1}{2}
%\text{grad}~\left\langle \left(\L_{\G}\otimes I_{\X}\right)f,f\right\rangle _{\X^N}.
%\end{align*}
\ban
\text{grad}~J(f)&=&\frac{1}{2}\text{grad}~\left(\E\Big[\|y(k)-\mathcal H(k)f\|_{\bigoplus_{i=1}^N\Y_i}^2\Big]+\left\langle \left(\L_{\G}\otimes I_{\X}\right)f,f\right\rangle _{\X^N}\right)\\
&=&\frac{1}{2}\text{grad}~\E\left[\langle \H^*(k)\H(k)f,f\rangle _{\X^N}\right]-\text{grad}~\E\Big[\langle \H(k)f,y(k) \rangle _{\bigoplus_{i=1}^N\Y_i}\Big]\\ &&+\frac{1}{2}
\text{grad}~\left\langle \left(\L_{\G}\otimes I_{\X}\right)f,f\right\rangle _{\X^N}\\
&=&\frac{1}{2}\text{grad}~\langle \E\left[\H^*(k)\H(k)\right]f,f\rangle _{\X^N}-\text{grad}~\E\left[\langle \H(k)f,y(k) \rangle _{\bigoplus_{i=1}^N\Y_i}\right]\\
&&+\frac{1}{2}
\text{grad}~\left\langle \left(\L_{\G}\otimes I_{\X}\right)f,f\right\rangle _{\X^N}.
\ean
Similarly, we have
\ban
\langle \E[\H^*(k)\H(k)]x,y\rangle _{\X^N}&=&\E[\langle \H^*(k)\H(k)x,y\rangle _{\X^N}]\\ &=&\E[\langle x,\H^*(k)\H(k)y\rangle _{\X^N}]\\
&=&\langle x, \E[\H^*(k)\H(k)]y\rangle _{\X^N},\ \forall\ x, y\in \X^N.
\ean
Therefore, $\E[\H^*(k)\H(k)]:\X^N\to \X^N$ is a self-adjoint operator. Noting that the Laplacian matrix $\L_{\G}$ is positive semi-definite, it follows that $\L_{\G}\otimes I_{\X}$ is a self-adjoint operator. Then
$$
\text{grad}~\langle \E[\H^*(k)\H(k)]f,f\rangle _{\X^N}=2\E[\H^*(k)\H(k)]f,~\forall\ f\in \X^N,
$$
and $$\text{grad}~\langle (\L_{\G}\otimes I_{\X})f,f\rangle _{\X^N}=2(\L_{\G}\otimes I_{\X})f,~\forall\ f\in \X^N.$$
%\begin{align}
%\text{grad}~\langle (\L_{\G}\otimes I_{\X})f,f\rangle _{\X^N}=2(\L_{\G}\otimes I_{\X})f,~\forall\ f\in \X^N.\notag
%\end{align}
Noting that $\H^*(k)y(k)\in L^1(\Omega;\X^N)$, it follows that
\begin{align}
&\lim_{t\to 0}\frac{1}{t}\left(\E\left[\langle \H(k)(f+tg),y(k) \rangle _{\bigoplus_{i=1}^N\Y_i}\right]-\E\left[\langle \H(k)f,y(k) \rangle _{\bigoplus_{i=1}^N\Y_i}\right]\right)\cr
=&\E[\langle \H^*(k)y(k),g\rangle _{\X^N}]\cr  = & \langle \E[\H^*(k)y(k)],g\rangle _{\X^N},~\forall\ g\in \X^N.\notag
\end{align}
It follows that $\text{grad}~\E[\langle \H(k)f,y(k) \rangle _{\bigoplus_{i=1}^N\Y_i}]=\E[\H^*(k)y(k)]$. Thus, we have
$$
\text{grad}~J(f)=-\E[\H^*(k)(y(k)-\H(k)f)]+(\L_{\G}\otimes I_{\X})f,~\forall\ f \in \X^N.
$$
Then we have the stochastic gradient descent (SGD) algorithm in the Hilbert space
\bna\label{algorithm1}
f(k+1)=f(k)+a(k)\H^*(k)(y(k)-\H(k)f(k))-b(k)(\L_{\G}\otimes I_{\X})f(k),~k\ge 0.
\ena
Let $f(k)=(f_1(k),\cdots,f_N(k))$. From (\ref{algorithm1}), we obtain the decentralized online learning algorithm
\bna\label{algorithm}
f_i(k+1)&=&f_i(k)+a(k)H_i^*(k)(y_i(k)-H_i(k)f_i(k)) \cr
&&+b(k)\sum_{j\in \N_i}a_{ij}(f_j(k)-f_i(k)),~k\ge 0,~i\in \mathcal V.
\ena


\begin{remark}
The algorithm (\ref{algorithm}) takes a form of ``consensus+innovations''. This kind of decentralized estimation strategies was first proposed in \cite{kar2012}-\cite{kar20132} for estimating the parameters in finite-dimensional spaces. Here,
 for the measurement model (\ref{measuramentmodel}), the learning strategy (\ref{algorithm}), which is a stochastic gradient decent algorithm for the Laplacian regularized loss functional (\ref{costfunctionLaplacianregularization}), is exactly the ``consensus+innovations'' type. The algorithm (\ref{algorithm}) can be regarded as the extention of the ``consensus+innovations'' type algorithm to the case of non-parametric or infinite-dimensional estimation.
%Intuitively, node $i$ obtains the innovation term $y_i(k)-H_i(k)f_i(k)$ and the consensus term $\sum_{j\in \N_i}a_{ij}(f_j(k)- f_i(k))$, based on which the estimation $f_i(k)$ is updated for the next instant.
%where the gain $a(k)$ is used to adjust the node's estimation of the unknown element $f_0$ at the next moment, and the gain $b(k)$ denotes the %consensus strength, which drives the consistency of the estimations among the nodes.
\end{remark}






\section{Convergence Analysis}

Although the algorithm (\ref{algorithm1}) is designed by assuming that $\{\H(k)\in L^2(\Omega;\mathscr L(\X^N,\bigoplus_{i=1}^N\Y_i)),\\k\ge 0\}$ and $\{v(k)\in L^2(\Omega;\bigoplus_{i=1}^N\Y_i), k\ge 0\}$ are both i.i.d.
In fact, in this section, we will show that even for the non-independence and non-stationarity sequence of operator-valued random elements $\{\H(k),k\ge 0\}$ and the noise sequence $\{v(k),k\ge 0\}$, the algorithm (\ref{algorithm}) still converge under mild conditions.

Denote the global estimation error by $e(k)=f(k)-\1_N\otimes f_0$. Note that $(\L_{\G}\otimes I_{\X})(\1_N\otimes f_0)=0$ and $\H(k)(\1_N\otimes f_0)=H(k)f_0$. Subtracting $\1_N\otimes f_0$ on both sides of equation (\ref{algorithm1}) yields
\begin{align}\label{error}
&e(k+1)\notag\\
=&(I_{\X^N}-a(k)\H^*(k)\H(k)-b(k)\L_{\G}\otimes I_{\X})f(k)+a(k)\H^*(k)y(k)-\1_N\otimes f_0 \notag\\
 =&(I_{\X^N}-a(k)\H^*(k)\H(k)-b(k)\L_{\G}\otimes I_{\X})(f(k)-\1_N\otimes f_0+\1_N\otimes f_0) \notag\\
&+a(k)\H^*(k)y(k)-\1_N\otimes f_0\notag\\
%&&=(I_{\X^N}-a(k)\H^*(k)\H(k)-b(k)\L_{\G}\otimes I_{\X})(e(k)+\1_N\otimes f_0)+a(k)\H^*(k)y(k)-\1_N\otimes f_0\cr
=&(I_{\X^N}-a(k)\H^*(k)\H(k)-b(k)\L_{\G}\otimes I_{\X})e(k)-(a(k)\H^*(k)\H(k) \notag\\
&+b(k)\L_{\G}\otimes I_{\X})(\1_N\otimes f_0)+a(k)\H^*(k)y(k) \notag\\
=&(I_{\X^N}-a(k)\H^*(k)\H(k)-b(k)\L_{\G}\otimes I_{\X})e(k) +a(k)\H^*(k)(y(k)-\H(k)(\1_N\otimes f_0))\notag\\
=&(I_{\X^N}-a(k)\H^*(k)\H(k)-b(k)\L_{\G}\otimes I_{\X})e(k)+a(k)\H^*(k)v(k).
\end{align}
The estimation error equation (\ref{error}) belongs to the following family of randomly time-varying difference equations:
\bna \label{chafen}
x(k+1)=(I_{\X_1}-F(k))x(k)+G(k)u(k),~k\ge 0,
\ena
where $u(k)$ is a random element with values in the Hilbert space $(\X_2,\tau_{\text{N}}(\X_2))$, $F(k):\Omega\to\mathscr L(\X_1)$ and $G(k):\Omega\to\mathscr L(\X_2,\X_1)$ are random elements with values in the topological spaces $(\LL(\X_1),\tau_{\text{S}}(\LL(\X_1)))$ and $(\LL(\X_2,\X_1),\tau_{\text{S}}(\LL(\X_2,\X_1)))$, respectively. To analyze the convergence of the estimation error equation (\ref{error}), we will first develop an asymptotic stability theory of the randomly time-varying difference equation (\ref{chafen}).

\subsection{Asymptotic Stability of Random Difference Equations in Hilbert Spaces}
To rigorously study the asymptotic stability of the random difference equations in the Hilbert space $(\X,\tau_{\text{N}}(\X))$, we introduce the following definitions.

 \vspace{-3mm}
\begin{definition}
If the sequence of random elements $\{x(k),k\ge 0\}$ with values in the Hilbert space $(\X,\tau_{\text{N}}(\X))$ satisfies $\sup_{k\ge 0}\E\left[\|x(k)\|_{\X}^p\right]<\infty$, where $p>0$, then $\{x(k),k\ge 0\}$ is said to be $L_p$-bounded.
\end{definition}
 \vspace{-6mm}
\begin{definition}
If the sequence of random elements $\{x(k),k\ge 0\}$ with values in the Hilbert space $(\X,\tau_{\text{N}}(\X))$ satisfies $\lim_{k\to \infty}\E\left[\|x(k)\|_{\X}^p\right]=0$, where $p>0$, then $\{x(k),k\ge 0\}$ is said to be $L_p$-asymptotically stable.
\end{definition}
\vspace{-6mm}

\begin{definition} \label{dingyi}
Let $\{A(k),k\ge 0\}$ be a sequence of operator-valued random elements with values in $(\LL(\X),\tau_{\text{S}}(\LL(\X)))$ and $\{\mathcal F(k), k\ge 0\}$ be a filter in $(\Omega,\F,\P)$. If for any given $L_q$-bounded adaptive sequence $\{x(k),\F(k), k\ge 0\}$ with values in the Hilbert space $\X$,
$$
\lim_{m\to \infty}\E\left[\left\|\prod_{k=n+1}^mA(k)x(n)\right\|_{\X}^p\right]=0,\ \forall\ n\geq0,\ \text{where}\ p,\ q>0,
$$
then $\{A(k), k\geq 0\}$ is said to be $L_p^q$-stable with respect to the filter $\{\mathcal F(k), k\geq0\}$.
\end{definition}

The strong convergence of the sequence of products of deterministic non-expansive operators has attracted the attentions of many scholars. By assuming strong convergence of operator products, the convergence results on infinite-dimensional deterministic time-varying difference equations in a general metric space were obtained (\cite{Reich3}-\cite{Pustylnik4}). Reich and Zaslavski (\cite{Reich3}) studied deterministic time-varying compressive operators in general metric spaces and obtained strong convergence results on the sequence of operator products; Pustylnik \emph{et al.}  (\cite{Pustylnik2}-\cite{Pustylnik4}) studied the strong convergence of the sequence of operator products consisting of finite number of projection operators.

Noting that a sequence of deterministic operator products converging strongly to $0$ can be regarded as a $L_p^q$-stable sequence of operators w.r.t. the trivial filter $\{\F(k)=\{\emptyset,\Omega\},k\in \mathbb Z\}$, the concept of strong convergence for the sequence of operator products in the above literature can be regarded as a special case of Definition \ref{dingyi}. Besides, for the case in finite-dimensional spaces, Guo (\cite{Guo1994}) proposed the concept of $L_p$-exponentially stable random matrix sequence $\{I-B(k)\in \mathbb R^{N\times N},k\geq0\}$, i.e. there exist constants $M>0$ and $\lambda \in (0,1)$ such that
$$\E\left[\left\|\prod_{k=n+1}^m(I-B(k))\right\|_{\LL(\mathbb R^{N})}^p\right]\leq M\lambda^{m-n},\ \forall\ m> n\geq0.$$
%Here, it can be seen that if $\{I-B(k),k\in \mathbb Z\}$ is $L_p$-exponentially stable,
From Definition \ref{dingyi}, we know that
 $\{I-B(k)\in \mathbb R^{N\times N},k\geq0\}$ is $L^{s}_{r}$-stable  w.r.t. any given filter $\{\mathcal F(k), k\geq0\}$, where $r=pa^{-1}$, $s=pba^{-1}$ and $a,b$ are positive real numbers with $a^{ -1}+b^{-1}=1$.

\vskip 0.2cm

Denote
$\F(k)=\bigvee_{i=0}^k(\sigma(F(i);\tau_{\text{S}}(\LL(\X_1)))\bigvee \sigma(G(i); \tau_{\text{S}}(\LL(\X_2,\X_1)))\bigvee \sigma(u(i);\tau_{\text{N}}(\X_2)))$, $k\\ \ge 0$ and $\F(-1)=\{\emptyset,\Omega\}$.
\vskip 0.2cm

For the $L_2$-asymptotic stability of the solution sequence of the random difference equation (\ref{chafen}), we have the following lemma. The proofs of all lemmas in this section can be found in Appendix B.
\vskip 1mm

\begin{lemma}\label{wendingxing}
For the random difference equation (\ref{chafen}), let $\{u(k),\F(k),k\ge 0\}$ be a $L_2$-bounded sequence of martingale differences, and the sequence $\{u(k),k\ge 0\}$ be independent of  $\{F(k),k\ge 0\}$ and $\{G(k),k\ge 0\}$. If (i) $\{I_{\X_1}-F(k),k\ge 0\}$ is $L_2^2$-stable  w.r.t. $\{\F(k),k\ge 0\}$, (ii) there exists a sequence of nonnegative real numbers $\{\gamma(k),k\ge 0\}$ such that
\begin{align}\label{qqqqq}
\E\left[\|I_{\X_1}-F(k)\|_{\LL(\X_1)}^4 \Big|\mathcal F(k-1)\right]\leq 1+\gamma(k)~\text{a.s.},
\end{align}
$\sum_{k=0}^{\infty}\gamma(k) <\infty$, and (iii)
\begin{align}
&\sum_{k=0}^{\infty}\E\left[\|G(k)\|_{\LL(\X_2,\X_1)}^2
\right]<\infty,\label{ssafe}\\
&\sup_{k\ge 0}\E\left[\|G(k)\|_{\LL(\X_2,\X_1)}^4\right]<\infty,\label{ssafe1}
\end{align}
then, the  solution $\{x(k),k\ge 0\}$ of  (\ref{chafen}) is $L_2$-asymptotically stable.
\end{lemma}




\begin{remark}
Systematic results on the stability of randomly time-varying difference equations in finite-dimensional spaces were achieved in \cite{Guo1994}-\cite{GUO444}, while the results on randomly time-varying difference equations in infinite-dimensional spaces remain fragmented. Kubrusly (\cite{Kubrusly}), Vajjha \emph{et al.} (\cite{Vajjha}) and the references therein transformed the analysis of the mean square convergence of stochastic approximation algorithms in Hilbert spaces into the $L_2$-asymptotic stability analysis of randomly time-varying difference equation. Ungureanu \emph{et al.} (\cite{Ungureanu1}), Ungureanu (\cite{Ungureanu2}-\cite{Ungureanu3}) and Zhang \emph{et al.} (\cite{zwzwxcbs}) investigated the $L_2$-asymptotic stability of the solution sequence of the randomly time-varying difference equation $x(k+1)=A(k)x(k)+b(k)$ in Hilbert space. The above literature assumed $A(k):\Omega\to \LL(\X)$ to be mean square exponentially stable, i.e., there exist constants $M>0$ and $\lambda\in (0,1)$ such that
\bna\label{zhishuwending}
\E\left[\left\|\left(\prod_{k=n+1}^mA(k)\right)x\right\|_{\X}^2\right]\leq M\lambda^{m-n}\|x\|^2,\ \forall\ m>n\ge 0,~\forall\ x\in \X.
\ena
For the randomly time-varying difference equation (\ref{chafen}), even if $F(k)\equiv F$, where $F$ is a deterministic self-adjoint compact operator with $\|F\|_{\LL(\X)}\leq 1$, the operator $A(k)\equiv I_{\X}-F$ does not satisfy (\ref{zhishuwending}). In fact, if (\ref{zhishuwending}) holds, then
\bna\label{zhihuhhh}
\left\|\prod_{k=n+1}^{n+2^l}A(k)\right\|_{\LL(\X)}\leq \sqrt{M}\lambda^{2^{l-1}},\ \forall\ l, n\ge 0.
\ena
Noting that $\|I_{\X}-F\|_{\LL(\X)}=\sup_{\|x\|_{\X}=1}|\langle (I_{\X}-F)x,x\rangle_{\X} |=1-\inf_{\|x\|_{\X}=1}\langle Fx,x\rangle_{\X} =1$, thus we have
$\Big\|\prod_{k=n+1}^{n+2^l}A(k)\Big\|_{\LL(\X)}^2=\left\|(I_{\X}-F)^{2^l}\right\|_{\LL(\X)}=\left\|I_{\X}-F\right\|_{\LL(\X)} ^{2^l}=1,\ \forall\ l, n\ge 0,$
which is in contradiction to (\ref{zhihuhhh}). Thereby, the existing results and methods on the stability of infinite-dimensional random difference equations are not applicable to random inverse problems.
\end{remark}

\subsection{Convergence of the Decentralized Algorithm}
Let
$$\F(k)=\bigvee_{i=0}^k\left(\sigma\left(\H(i);\tau_{\text{S}}\left(\LL\left(\X^N,\bigoplus_{j=1}^N\Y_j\right)\right)\right)\bigvee \sigma\left(v(i);\tau_{\text{N}}\left(\bigoplus_{j=1}^N\Y_j\right)\right)\right),~k\ge 0,$$
and $\F(-1)=\{\emptyset,\Omega\}$, where $\H(i)$ and $v(i)$ are given in Section \ref{online}. We need the following assumptions.


\begin{assumption}\label{assumption1}
  The noises $\{v(k),k\ge 0\}$ with values in $(\bigoplus_{i=1}^N\Y_i,\tau_{\text{N}}(\bigoplus_{i=1}^N\Y_i))$ and the random forward operator sequence $\{H(k),k\ge 0\}$ with values in $(\mathscr L( \X,\bigoplus_{i=1}^N$ $\Y_i),\tau_{\text{S}}(\mathscr L(\X$, $\bigoplus_{i=1}^N\Y_i)))$ are mutually independent.
\end{assumption}
%\textbf{A1.} The noise $\{v(k),k\ge 0\}$ with values in Hilbert space $(\bigoplus_{i=1}^N\Y_i,\tau_{\text{N}}(\bigoplus_{i=1}^N\Y_i))$ and the random forward operator sequence $\{H(k),k\ge 0\}$ with values in topological space $(\mathscr L( \X,\bigoplus_{i=1}^N$ $\Y_i),\tau_{\text{S}}(\mathscr L(\X,\bigoplus_{i=1}^N\Y_i)))$ are mutually independent.
\begin{assumption}\label{assumption2}
The noises $\{v(k),\F(k),k\ge 0\}$  are martingale differences   and there exists a constant $\b_v>0$ such that $$\sup_{k\ge 0}\E\left.\left[\left\|v(k)\right\|_{\bigoplus_{i=1}^N\Y_i}^2\right|\F(k-1)\right]\leq \b_v~\text{a.s.}$$
\end{assumption}
%\textbf{A2.} The noise $\{v(k),\F(k),k\ge 0\}$ is a martingale sequence and there exists a constant $\b_v>0$ such that $$\sup_{k\ge 0}\E\left.\left[\left\|v(k)\right\|_{\bigoplus_{i=1}^N\Y_i}^2\right|\F(k-1)\right]\leq \b_v~\text{a.s.}$$


For the gains of the algorithm (\ref{algorithm}), we may need the following conditions.


\begin{condition}\label{condition1}
The algorithm gains $\{a(k),k\ge 0\}$ and $\{b(k),k\ge 0\}$ are both monotonically decreasing sequences of positive real numbers.
\end{condition}
%\textbf{C1.} The algorithm gains $\{a(k),k\ge 0\}$ and $\{b(k),k\ge 0\}$ are both monotonically decreasing sequences of positive real numbers.


\begin{condition}\label{condition2}
$\sum_{k=0}^{\infty}a^2(k)<\infty$ and $\sum_{k=0}^{\infty}b^2(k)<\infty$.
\end{condition}
%\textbf{C2.} $\sum_{k=0}^{\infty}a^2(k)<\infty$ and $\sum_{k=0}^{\infty}b^2(k)<\infty$.


\begin{condition}\label{condition3}
$\sum_{k=0}^{\infty}a(k)=\infty$ and $\max\{a(k)-a(k+1),b(k)-a(k)\}=\mathcal O(a^2(k)+b^2(k))$.
\end{condition}
%\textbf{C3.} $\sum_{k=0}^{\infty}a(k)=\infty$ and $\max\{a(k)-a(k+1),b(k)-a(k)\}=\mathcal O(a^2(k)+b^2(k))$.


We analyze the convergence of the algorithm (\ref{algorithm}) now.
%by studying the $L_2$-asymptotic stability of the solution sequence of the randomly time-varying difference equation (\ref{error}) in the Hilbert %space $(\X^N,\tau_{\text{N}}(\X^N))$.
Firstly, by Lemma \ref{wendingxing}, we have the following key theorem.

\begin{theorem}\label{dingliyi1}
For the algorithm (\ref{algorithm}), suppose that Assumptions \ref{assumption1}, \ref{assumption2} and Condition \ref{condition2} hold, there exists a sequence of nonnegative real numbers $\{\gamma(k),k\ge 0\}$ with $\sum_{k=0}^{\infty}\gamma(k)< \infty$, such that
\bna\label{qafgs}
&&~~~\left.\E\left[\|I_{\X^N}-\left(a(k)\H^*(k)\H(k)+b(k)\L_{\G}\otimes I_{\X}\right)\|^4_{\LL\left(\X^N\right)}\right|\F(k-1)\right]\cr
&&\leq 1+\gamma(k)~\text{a.s.},
\ena
and  the sequence of operator-valued random elements $\{I_{\X^N}-(a(k)\H^*(k)\H(k)+b(k)\L_{\G}\otimes I_{\X}),k\ge 0\}$ is $L_2^2$-stable  w.r.t. $\{\F(k),k\ge 0\}$.\\
I. If
$$\sup_{k\ge 0}\E\left[\|\H(k)\|_{\mathscr L\left(\X^N,\bigoplus_{i=1}^N\Y_i\right)}^2\right]<\infty,$$
then the algorithm (\ref{algorithm}) is mean square consistent, i.e., $\lim_{k\to\infty}\E[\|f_i(k)-f_0\|_{\X}^2]=0,~i\in \mathcal V$.\\
II. If there exists a constant $\rho_0>0$ such that
$$\E\left.\left[\|\H(k)\|_{\mathscr L\left(\X^N,\bigoplus_{i=1}^N\Y_i\right)}^2\right|\F(k-1)\right]\leq \rho_0~\text{a.s.},$$
then the algorithm (\ref{algorithm}) is almost surely strongly consistent, i.e., $\lim_{k\to\infty}\|f(k)-f_0\|_{\X}=0~\text{a.s.},~i\in \mathcal V$.
\end{theorem}
\begin{proof}
Denote $F(k)=a(k)\H^*(k)\H(k)+b(k)\L_{\G}\otimes I_{\X}$ and $G(k)=a(k)\mathcal H^*(k)$, respectively. Notice that $F(k)\ge 0$ and the estimation error equation (\ref{error}) can be rewritten as the following random difference equation
\ban
e(k+1)=\left(I_{\X^N}-F(k)\right)e(k)+G(k)v(k).
\ean
Given the initial value $e(0)\in \X^N$, by Proposition \ref{nlllwwieiie}.(a)-(c), we know that $\{e(k),k\ge 0\}$ is a random sequence with values in the Hilbert space $(\X^N,\tau_{\text{N}}(\X^N))$. On the one hand, it follows from Assumptions \ref{assumption1} and \ref{assumption2} that $\{v(k),k\ge 0\}$ is independent of $\{F(k),G(k),k\ge 0\}$, and $\sup_{k\ge 0}\E[\|v(k)\|^2]\leq \b_v$. On the other hand, by the condition (\ref{qafgs}), we get $\E[\|I_{\X^N}-F(k)\|^4|\F(k-1)]\leq 1+\gamma(k)~\text{a.s.}$, which gives
\bna\label{xlms}
&&~~~~\E\left[\|a(k)\H^*(k)\H(k)+b(k)\L_{\G}\otimes I_{\X}\|^2\right]\cr
&&\leq \E\left[\|I_{\X^N}-(I_{\X^N}-F(k))\|^2\right]\cr
&&\leq 2\left(1+\E\left[\|I_{\X^N}-F(k)\|^2\right]\right)\cr
&&\leq 2\left(1+\E\left[\|I_{\X^N}-F(k)\|^4\right]^{\frac{1}{2}}\right)\cr
&&\leq 2\left(1+\sqrt{1+\gamma(k)}\right).
\ena
Noting that $\L_{\G}$ is positive semi-definite, we have $a(k)\H^*(k)\H(k)+b(k)\L_{\G}\otimes I_{\X}\ge a(k)\H^*(k)\\ \H(k)\ge 0~\text{a.s.}$, which leads to $\|a(k)\H^*(k)\H(k)\|^2\leq \|a(k)\H^*(k)\H(k)+b(k)\L_{\G}\otimes I_{\X}\|^2$. By (\ref{xlms}), we obtain
\bna\label{xfwee}
&&~~~~\sup_{k\ge 0}\E\left[\|G(k)\|^4\right]\cr &&=\sup_{k\ge 0}\left\{a^2(k)\E\left[\left\|a(k)\H^*(k)\H(k)\right\|^2\right]\right\}\cr
&&\leq \sup_{k\ge 0}\left\{a^2(k)\E\left[\|a(k)\H^*(k)\H(k)+b(k)\L_{\G}\otimes I_{\X}\|^2\right]\right\}\cr
&&\leq 2\sup_{k\ge 0}\left\{a^2(k)\left(1+\sqrt{1+\gamma(k)}\right)\right\}\cr
&&<\infty,
\ena
where the last inequality is obtained from Condition \ref{condition2} and $\sum_{k=0}^{\infty}\gamma(k)<\infty$.\\
(I) If $\sup_{k\ge 0}\E[\|\H(k)\|^2]<\infty$, it follows from Condition \ref{condition2} that
\ban
\sum_{k=0}^{\infty}\E\left[\|G(k)\|^2\right]\leq \left\{\sup_{k\ge 0}\E\left[\|\H(k)\|^2\right]\right\}\sum_{k=0}^{\infty}a^2(k)<\infty.
\ean
By Lemma \ref{wendingxing}, the sequence of solutions to the estimation error equation (\ref{error}) is $L_2$-asymptotically stable, i.e., $\lim_{k\to\infty}\E[\|f_i(k)-f_0\|^2]=0,~i\in \mathcal V$. \\
(II) If $\E[\|\H(k)\|^2|\F(k-1)]\leq \rho_0~\text{a.s.}$, on the one hand, noting that $\sup_{k\ge 0}\E[\|\H(k)\|^2]\leq \rho_0<\infty$, by above conclusion (I), we have $\lim_{k\to\infty}\E[\|e(k)\|^2]=0$. On the other hand, it follows from the condition (\ref{qafgs}) and (\ref{xfwee}) that
\ban
\sup_{k\ge 0}\E\left[\|(I_{\X^N}-F(k))G(k)\|^2\right]\leq \sup_{k\ge 0}\E\left[\|I_{\X^N}-F(k)\|^4+\|G(k)\|^4\right]<\infty.
\ean
Thus, by the estimation error equation (\ref{error}), Assumptions \ref{assumption1} and \ref{assumption2}, Proposition 2.6.13 in \cite{hy} and Proposition \ref{lemmaA6}, we get
\ban
&&~~~~\E\left.\left[\|e(k+1)\|^2\right|\F(k-1)\right]\cr
&&=\E\left.\left[\|\left(I_{\X^N}-F(k)\right)e(k)+G(k)v(k)\|^2\right|\F(k-1)\right]\cr
&&=\E\left.\left[\|(I_{\X^N}-F(k))e(k)\|^2\right|\F(k-1)\right]+\E\left.\left[\|G(k)v(k)\|^2\right|\F(k-1)\right]\cr
&&~~~+2\E\left.\left[\langle e(k),(I_{\X^N}-F(k))G(k)v(k)\rangle \right|\F(k-1)\right]\cr
&&=\E\left.\left[\|(I_{\X^N}-F(k))e(k)\|^2\right|\F(k-1)\right]+\E\left.\left[\|G(k)v(k)\|^2\right|\F(k-1)\right]\cr
&&~~~+2\langle e(k),\E\left.\left[(I_{\X^N}-F(k))G(k)v(k)\right|\F(k-1)\right]\rangle\cr
&&=\E\left.\left[\|(I_{\X^N}-F(k))e(k)\|^2\right|\F(k-1)\right]+\E\left.\left[\|G(k)v(k)\|^2\right|\F(k-1)\right]\cr
&&~~~+2\langle e(k),\E\left.\left[(I_{\X^N}-F(k))G(k)\E[v(k)|\F(k-1)]\right|\F(k-1)\right]\rangle\cr
&&=\E\left.\left[\|(I_{\X^N}-F(k))e(k)\|^2\right|\F(k-1)\right]+\E\left.\left[\|G(k)v(k)\|^2\right|\F(k-1)\right]\cr
&&\leq \E\left.\left[\|(I_{\X^N}-F(k))\|^2\right|\F(k-1)\right]\|e(k)\|^2+\E\left.\left[\|G(k)v(k)\|^2\right|\F(k-1)\right]\cr
&&\leq \E\left.\left[\|(I_{\X^N}-F(k))\|^4\right|\F(k-1)\right]^{\frac{1}{2}}\|e(k)\|^2+\E\left.\left[\|G(k)v(k)\|^2\right|\F(k-1)\right]\cr
&&\leq \left(1+\gamma(k)\right)^{\frac{1}{2}}\|e(k)\|^2+\b_v\E\left.\left[\|G(k)\|^2\right|\F(k-1)\right]\cr
&&\leq \left(1+\frac{1}{2}\gamma(k)\right)\|e(k)\|^2+\rho_0\b_va^2(k)~\text{a.s.}
\ean
Noting that $\sum_{k=0}^{\infty}\gamma(k)<\infty$ and $\sum_{k=0}^{\infty}a^2(k)<\infty$, it follows from Lemma \ref{lemmaA3} that $\|e(k)\|^2$ converges almost surely, which together with $\lim_{k\to\infty}\E[\|e(k)\|^2]=0$ gives $\lim_{k\to\infty}e(k)=0~\text{a.s.}$
\end{proof}



%\begin{remark}\label{nvllwwnnvvv}
%The boundedness condition (\ref{qafgs}) in Lemma \ref{dingliyi1} guarantees the existence of the expectation and conditional expectation of the estimate $f(k)$ in the algorithm (\ref{algorithm}).
%%i.e., $f(k)$ is Bochner integrable.
%It is not difficult to verify that (\ref{qafgs}) holds if the norm of the random forward operator has a uniform upper bound independent of the sample path.
%%The existence of the Bochner integral in the infinite-dimensional Banach space needs to be guaranteed by the integrability.
%%The condition (\ref{qafgs}) can be weakened for the algorithm in finite-dimensional Hilbert spaces. Wang \emph{et al.} \cite{WLZ} considered the distributed parameter estimation algorithm, which is equivalent to the algorithm (\ref{algorithm}) in an Euclidean  space, and since the expectation of the random vector is defined by the Lebesgue integral, Wang \emph{et al.} \cite{WLZ} proposed a weaker boundedness condition than (\ref{qafgs}).
%\end{remark}





 We will next give intuitive sufficient conditions on the mean square and almost sure strong consistency of the algorithm (\ref{algorithm}).  At first, we need the following fundamental lemma.



%If the graph $\G$ is connected, then by Lemma \ref{yibanxingdejieguo}, we can further obtain a more intuitive sufficient condition, under which the operator-valued random sequence $\{I_{\X^N}-a(k)\H^*(k)\H(k)-b(k)\L_{\G}\otimes I_{\X},k\ge 0\}$ is $L_2^2$-stable w.r.t.   $\{\F (k),k\ge 0\}$.

\begin{lemma}\label{jihubiranshoulian}
For the algorithm (\ref{algorithm}), let $\G$ be connected and assume that Assumptions \ref{assumption1}, \ref{assumption2} and Conditions \ref{condition1}-\ref{condition3} hold. If there exist positive self-adjoint operators $\HH_i\in \mathscr L(\mathscr X)$, $i=1,\cdots,N$ satisfying $\sum_{i=1}^N\HH_i>0$, and an integer $h>0$, such that
\bna\label{yinlitiaojian1}
\sum_{j=1}^N\sum_{k=0}^{\infty}\E\left[\left\|\HH_jx(k)-\sum_{i=kh}^{(k+1)h-1}\E\left.\left[H_j^*(i)H_j(i)x(k)\right|\F(kh-1)\right] \right\|_{\X}^2\right]<\infty,
\ena
for any $L_2$-bounded adaptive sequence $\{x(k), \F(kh-1),k\ge 0\}$ with values in the Hilbert space $\X$, and there exists a constant $\rho_0>0$ such that
\bna\label{yinlitiaojian2}
\sup_{k\ge 0}\left(\E\left.\left[\|\H^*(k)\H(k)\|_{\LL\left(\X^N\right)}^{2^{\max\{h,2\}}}\right|\F(k-1)\right]\right)^{\frac{1}{2^ {\max\{h,2\}}}}\leq \rho_0~\text{a.s.},
\ena
then $\{I_{\X^N}-a(k)\H^*(k)\H(k)-b(k)\L_{\G}\otimes I_{\X},k\ge 0\}$ is $L_2^2$-stable  w.r.t. $\{\F(k),k\ge 0\}$.
\end{lemma}


%Through the $L_2^2$-stability analysis of $\{I_{\X^N}-a(k)\H^*(k)\H(k)\H(k)-b(k)\L_{\G}\otimes I_{\X},k\ge 0\}$,

%%Through the $L_2^2$-stability analysis of $\{I_{\X^N}-a(k)\H^*(k)\H(k)\H(k)-b(k)\L_{\G}\otimes I_{\X},k\ge 0\}$,
%Combining Theorem  \ref{dingliyi1} and Lemma  \ref{jihubiranshoulian}, we will next give intuitive sufficient conditions on the mean square and almost sure strong consistency of the algorithm (\ref{algorithm}).

Combining Theorem  \ref{dingliyi1} and Lemma  \ref{jihubiranshoulian}, we have the following theorem.

\begin{theorem}\label{vnknoklfl}
For the algorithm (\ref{algorithm}),
suppose that all the conditions in Lemma \ref{jihubiranshoulian} hold. If there exists a sequence of nonnegative real numbers $\{\Gamma(k),k\ge 0\}$ with $\sum_{k=0}^{\infty}\Gamma(k)<\infty$, such that
\bna\label{dinglitiaojian}
  \E\left[\|I_{\X^N}-4\left(a(k)\H^*(k)\H(k)+b(k)\L_{\G}\otimes I_{\X}\right)\|_{\LL\left(\X^N\right)}\big|\F(k-1)\right]
 \leq 1+\Gamma(k)~\text{a.s.},
\ena
then the algorithm (\ref{algorithm}) is both mean square and almost surely strongly consistent.
\end{theorem}
\begin{proof}
By the conditions (\ref{yinlitiaojian1})-(\ref{yinlitiaojian2}) and Lemma \ref{jihubiranshoulian}, it is known that $\{I_{\X^N}-a(k)\H^*(k)\H(k)-b(k)\L_{\G}\otimes I_{\X},k\ge 0\}$ is $L_2^2$-stable w.r.t. $\{\F(k),k\ge 0\}$. Denote $D(k)=a(k)\H^*(k)\H(k)+b(k)\L_{\G}\otimes I_{\X}$. By the condition (\ref{yinlitiaojian2}), we know that
\begin{align}\label{cmllemfnn}
&\E\left[\|D(k)\|^r|\F(k-1)\right]\notag\\
 \leq & \Big(\E\big[\|a(k)\H^*(k)\H(k)+b(k)(\L_{\G}\otimes I_{\X})\|^{2^h}\big|\F(k-1)\big]\Big)^{\frac{r}{2^h}}\notag\\
 \leq & \max\{a(k),b(k)\}^r\Big(2^{2^h-1}\E\left.\left[\|\H^*(k)\H(k)\|^{2^h}\right|\F(k-1)
 \right]+2^{2^h-1}\|\L_{\G}\otimes I_{\X}\|^{2^h}\Big)^{\frac{r}{2^h}}\notag\\
 \leq & 2^r\left(a^r(k)+b^r(k)\right)\bigg(\left(\E\left.\left[\|\H^*(k)\H(k)\|^{2^h}\right|
 \F(k-1)\right]\right)^{\frac{r}{2^h}}+\|\L_{\G}\otimes I_{\X}\|^{r}\bigg)\notag\\
 \leq& 2^r (a^r(k)+b^r(k))(\rho_0^r+\|\L_{\G}\otimes I_{\X}\|^r)\notag\\
 \leq & 2^r(a^r(k)+b^r(k))\rho_1^r~\text{a.s.},~\forall \ 1\leq r\leq 4,
\end{align}
where $\rho_1=\rho_0+\|\L_{\G}\otimes I_{\X}\|$. By the condition (\ref{dinglitiaojian}) and (\ref{cmllemfnn}), we get
 %$\E\Big[\|I_{\X^N}- (a(k)\H^*(k)\H(k)+b(k)\L_{\G}\otimes I_{\X} )\|^4|\F(k-1)\Big]
% =  \E\Big[\big\|  (I_{\X^N}- (a(k)\H^*(k)\H(k)
%  +b(k)\L_{\G}\otimes I_{\X} ) )^4\big\||\F(k-1)\Big]
% = \E\Big[\big\|I_{\X^N}-4D(k)+6D^2(k)-4D^3(k)
%  +D^4(k)\big\|
%  |\F(k-1)\Big]
% \leq   \E\Big[\|I_{\X^N}-4D(k)\|+6\|D(k)\|^2+4\|D(k)\|^3
%  +\|D(k)\|^4 |\F(k-1)\Big]
% \leq   1+\gamma(k)~\text{a.s.},$
\begin{align}
&\E\left[\|I_{\X^N}-\left(a(k)\H^*(k)\H(k)+b(k)\L_{\G}\otimes I_{\X}\right)\|^4|\F(k-1)\right]\notag\\
 =& \E\Big[\big\|  (I_{\X^N}- (a(k)\H^*(k)\H(k)
 +b(k)\L_{\G}\otimes I_{\X} ) )^4\big\||\F(k-1)\Big]\notag\\
 =& \E\Big[\big\|I_{\X^N}-4D(k)+6D^2(k)-4D^3(k) +D^4(k)\big\|
  |\F(k-1)\Big]\notag\\
 \leq & \E\Big[\|I_{\X^N}-4D(k)\|+6\|D(k)\|^2+4\|D(k)\|^3
  +\|D(k)\|^4 |\F(k-1)\Big]
 \leq   1+\gamma(k)~\text{a.s.},\notag
\end{align}
where $\gamma(k)=\Gamma(k)+6(a^2(k)+b^2(k))\rho_1^2+4(a^3(k)+b^3(k))\rho_1^3+(a^4(k)+b^4(k))\rho_1^4$. From Condition \ref{condition2} and the condition (\ref{dinglitiaojian}), we know that $\sum_{k=0}^{\infty}\gamma(k)<\infty$, which together with Theorem \ref{dingliyi1} implies that algorithm (\ref{algorithm}) is both mean square and almost surely strongly consistent.
\end{proof}

Especially, if $\{\H(k),k\ge 0\}$ with values in $(\mathscr L(\X^N,\bigoplus_{i=1}^N\Y_i),\tau_{\text{S}}(\mathscr L(\X^N,\bigoplus_{i=1}^N\Y_i)))$ and the random sequence $\{v(k),k\ge 0\}$ with values in $(\bigoplus_{i=1}^N\Y_i,\tau_{\text{N}}(\bigoplus_{i=1}^N\Y_i))$ are both i.i.d. and they are mutually independent, then the following corollary follows from Theorem \ref{vnknoklfl}.



\begin{corollary}\label{xiaosirendetuilun}
For the algorithm (\ref{algorithm}), assume that $\G$ is connected, Assumptions \ref{assumption1}, \ref{assumption2} and Conditions \ref{condition1}-\ref{condition3} hold, and $\{\H(k),k\ge 0\}$ and $\{v(k),k\ge 0\}$ are both i.i.d. sequences and they are mutually independent. If there exists a constant $\rho_0>0$ such that $\|\H(0)\|\leq \rho_0~\text{a.s.}$ and
\bna\label{tuiluntiaojian} \sum_{j=1}^N\E\left[\|H_j(0)x\|_{\Y_j}^2\right]>0,\ \forall\ x\in \X\setminus\{0\},
\ena
then the algorithm (\ref{algorithm}) is both mean square and almost surely strongly consistent.
\end{corollary}
\begin{proof}
It follows from $\|\H(0)\|\leq \rho_0~\text{a.s.}$ and Proposition \ref{nlllwwieiie}.(a) that $H^*_j(0)  H_j(0)x\in L^1(\Omega;\X)$, $x\in \X$. For the integer $h>0$ and $j\in \mathcal V$, we define the operator $\HH_j:\X\to\X$ by
\bna\label{wpkfpkpew}
\HH_j(x)=h\E\left[H^*_j(0)H_j(0)x\right],~x\in \X,~j\in \mathcal V.
\ena
For any given $x_1,x_2\in\X$ and $c_1,c_2\in \mathbb R$, we have
\bna\label{jcknvknw}
\HH_j(c_1x_1+c_2x_2)&=&c_1h\E\left[H^*_j(0)H_j(0)x_1\right]+c_2h\E\left[H^*_j(0)H_j(0)x_2\right]\cr
&=&c_1\HH_j(x_1)+c_2\HH_j(x_2).
\ena
Noting that $H^*_j(0)H_j(0)x\in L^1(\Omega;\X)$, by Proposition 2.6.13 in \cite{hy}, we get
\bna
\left\langle \HH_j(x_1), x_2\right\rangle&=&h\E\left[\left\langle H^*_j(0)H_j(0)x_1,x_2 \right\rangle \right]\cr
&=&h\E\left[\left\langle x_1,H^*_j(0)H_j(0)x_2 \right\rangle \right]\cr
&=&\left\langle x_1, \HH_j(x_2)\right\rangle,~j\in \mathcal V.\notag
\ena
This together with   (\ref{jcknvknw}) shows that $\HH_j$ is a linear self-adjoint operator, which gives
\ban
\|\HH_j(x)\|\leq h\E\left[\left\|H^*_j(0)H_j(0)\right\|\|x\|\right]\leq h\rho_0^2\|x\|,~\forall\ x\in \X,~j\in \mathcal V,
\ean
thus, the self-adjoint operator $\HH_j\in \mathscr L(\X)$. Denote $\HH_j(x):=\HH_jx$, $\forall\ x\in \X$. Noting that $H^*_j(0)H_j(0)x\in L^1(\Omega;\X)$, it follows from Proposition 2.6.13 in \cite{hy} that
\bna\label{owwwww2}
\left\langle \HH_jx,x\right\rangle&=&h\left\langle \E\left[H^*_j(0)H_j(0)x\right],x \right\rangle\cr
&=&h\E\left[\left\langle H^*_j(0)H_j(0)x,x \right\rangle \right]\cr
&=&h\E\left[\left\|H_j(0)x\right\|^2\right]\ge 0,
\ena
from which we know that the operator $\HH_j$ defined in (\ref{wpkfpkpew}) is positive bounded linear self-adjoint, $j\in \mathcal V$. Noting that $\{\H(k),k\ge 0\}$ is an i.i.d. sequence with values in the topology space $(\mathscr L(\X^N,\bigoplus_{i=1}^N\Y_i),\tau_{\text{S}}(\mathscr L(\X^N,\bigoplus_{i=1}^N\Y_i)))$, it follows from Definition \ref{dulixing} that $\{\H^*(k)\H(k)x,\ x$ $\in\X^N,\ k\ge 0\}$ is an i.i.d. sequence with values in the Banach space $(\X^N,\tau_{\text{N}}(\X^N))$. By Proposition E.1.10 in \cite{hy2}, we know that $\{\|\H^*(k)\H(k)\|,k\ge 0\}$ and $\{\|I_{\X^N}-(a(k)\H^*(k)\H(k)+b(k)\L_{\G}\otimes I_{\X})\|,k\ge 0\}$ are both independent random sequences. By Condition \ref{condition2}, there exists an integer $s_0>0$, such that $a(k)+b(k)\leq (4\rho^2_0+4\|\L_{\G}\otimes I_{\X}\|)^{-1}$, $\forall\ k\ge s_0$. We define the nonnegative real sequence $\{\Gamma(k),k\ge 0\}$ by
\bna%\label{owwwww0}
\Gamma(k)=\begin{cases}
4(a(k)+b(k))\left(\rho^2_0+\|\L_{\G}\otimes I_{\X}\|\right),& 0\leq k<s_0;\\
0, & k\ge s_0,\notag
\end{cases}
\ena
which shows that $\sum_{k=0}^{\infty}\Gamma(k)<\infty$. Noting that $\{\H(k),k\ge 0\}$ are i.i.d. and
 $P\{\|\H(0)\|\leq \rho_0\}=1$, it can be verified that $P\{\|\H(k)\|\leq \rho_0\}=1$, $k=0,1,...$, and
$\\\|I_{\X^N}-4(a(k)\H^*(k)\H(k)+b(k)\L_{\G}\otimes I_{\X})\|\leq 1+4(a(k)+b(k))(\rho^2_0+\|\L_{\G}\otimes I_{\X}\|)~\text{a.s.}$, $\forall\ k\ge 0$. Then, we obtain
\bna%\label{owwwww1}
&&~~~\left.\E\left[\left\|I_{\X^N}-4\left(a(k)\H^*(k)\H(k)+b(k)\L_{\G}\otimes I_{\X}\right)\right\|\right|\F(k-1)\right]\cr
&&=\E\left[\left\|I_{\X^N}-4\left(a(k)\H^*(k)\H(k)+b(k)\L_{\G}\otimes I_{\X}\right)\right\|\right]\cr
&&=\E\bigg[\sup_{\|x\|=1}\left|\left\langle I_{\X^N}-4\left(a(k)\H^*(k)\H(k)+b(k)\L_{\G}\otimes I_{\X}\right)x,x\right\rangle \right|\bigg]\cr
&&=\E\bigg[\sup_{\|x\|=1}\left|1-4\left\langle\left(a(k)\H^*(k)\H(k)+b(k)\L_{\G}\otimes I_{\X}\right)x,x\right\rangle \right|\bigg]\cr
&&=\E\bigg[1-4\inf_{\|x\|=1}\left\langle\left(a(k)\H^*(k)\H(k)+b(k)\L_{\G}\otimes I_{\X}\right)x,x\right\rangle \bigg]\cr
%&&=\E\left[\left\|I_{\X^N}-\left(a(k)\H^*(0)\H(0)+b(k)\L_{\G}\otimes I_{\X}\right)\right\|^4\right]\cr
 &&\leq 1~\text{a.s.},~\forall \ k\ge s_0.\notag
\ena
Then, we get
\ban
\left.\E\left[\left\|I_{\X^N}-4\left(a(k)\H^*(k)\H(k)+b(k)\L_{\G}\otimes I_{\X}\right)\right\|\right|\F(k-1)\right]\leq 1+\Gamma(k)~\text{a.s.},~\forall\ k\ge 0.
\ean
It can be verified that
\bna\label{vnkwenvefklw}
&&~~~\sup_{k\ge 0}\left(\E\left.\left[\|\H^*(k)\H(k)\|^{2^{\max\{h,2\}}}\right|\F(k-1)\right]\right)^{\frac{1}{2^{\max\{h,2\}}}}\cr
&&=\sup_{k\ge 0}\left(\E\left[\|\H^*(k)\H(k)\|^{2^{\max\{h,2\}}}\right]\right)^{\frac{1}{2^{\max\{h,2\}}}}\cr
&&=\sup_{k\ge 0}\left(\E\left[\|\H^*(0)\H(0)\|^{2^{\max\{h,2\}}}\right]\right)^{\frac{1}
{2^{\max\{h,2\}}}}
 \leq \rho^2_0~\text{a.s.}
\ena
For any given integers $k\ge 0$, $h>0$ and $x\in \X$, if $A\in \F(kh-1)$, then it follows from Proposition \ref{nlllwwieiie} and (\ref{vnkwenvefklw}) that $H^*_j(i)H_j(i)(\1_A\otimes x)\in L^1(\Omega;\X)$, $i\ge kh$, which together with Lemma \ref{nvkvpeoeo} implies that $\E[H^*_j(i)H_j(i)(\1_A\otimes x)|\F(kh-1)]$ uniquely exists. Since $\{\H(k),k\ge 0\}$ and $\{v(k),k\ge 0\}$ are both i.i.d sequences and they are mutually independent, it follows from Definition \ref{dulixing} that $H^*_j(i)H_j(i)x$ is independent of $\1_{F\cap A}$ with $F\in \F(kh-1)$. Thus,
\ban
&&~~~\int_F\E\left.\left[\sum_{i=kh}^{(k+1)h-1}H^*_j(i)H_j(i)(\1_A\otimes x)\right|\F(kh-1)\right]\dd\P\cr
&&=\int_F\sum_{i=kh}^{(k+1)h-1}H^*_j(i)H_j(i)(\1_A\otimes x)\dd\P\cr
&&=\int_{\Omega}\left(\sum_{i=kh}^{(k+1)h-1}H^*_j(i)H_j(i)x\right)\1_{F\cap A}\dd\P\cr
&&=\int_{\Omega}\sum_{i=kh}^{(k+1)h-1}H^*_j(i)H_j(i)x\dd\P\int_{\Omega}\1_{F\cap A}\dd\P\cr
&&=\E\left[\sum_{i=kh}^{(k+1)h-1}H^*_j(i)H_j(i)x\right]\P(F\cap A)\cr
&&=h\E\left[H^*_j(0)H_j(0)x\right]\P(F\cap A)\cr
&&=h\int_{F}\E\left[H^*_j(0)H_j(0)x\right]\1_A\dd\P,~\forall \ F\in \F(kh-1),~j\in \mathcal V,
\ean
which gives
\ban
\E\left.\left[\sum_{i=kh}^{(k+1)h-1}H^*_j(i)H_j(i)(\1_A\otimes x)\right|\F(kh-1)\right]=\HH_j(\1_A\otimes x)~\text{a.s.},~j\in \mathcal V.
\ean
 This together with the properties of the conditional expectation, the operator $\HH_j$ and the linearity of Bochner integral leads to
\bna\label{vnkwoejvnvvn}
\HH_jy=\E\Bigg[\sum_{i=kh}^{(k+1)h-1}H^*_j(i)H_j(i)y\Bigg|\F(kh-1)\Bigg]~\text{a.s.},~j\in \mathcal V,
\ena
where $y\in L^0(\Omega,\F(kh-1);\X)$ is a simple function. For $f\in L^2(\Omega,\F(kh-1);\X)$, by Pettis measurability theorem, we know that there exists a sequence of simple functions $\{f_n \in L^0(\Omega,\F(kh-1);\X),n\ge 0\}$ satisfying $\|f_n\|\leq \|f\|\text{a.s.}$ and $\lim_{n\to \infty}f_n=f~\text{a.s.}$, which together with (\ref{vnkwenvefklw}) and Cauchy inequality gives $H^*_j(i)H_j(i)f\in L^1(\Omega;\X)$. Thus, from Lemma \ref{nvkvpeoeo}, it is known that $\E[H^*_j(i)H_j(i)f|\F(kh-1)]$ uniquely exists. Noting that $\HH_j\in \LL(\X)$, it follows from (\ref{vnkwenvefklw})-(\ref{vnkwoejvnvvn}) and the dominated convergence theorem of conditional expectation that
\bna\label{vnowlelkel}
&&~~~\HH_jf\cr
&&=\lim_{n\to\infty}\HH_jf_n\cr &&=\lim_{n\to\infty}\E\left.\left[\sum_{i=kh}^{(k+1)h-1}H^*_j(i)H_j(i)f_n\right|\F(kh-1)\right]\cr
&&=\E\left.\left[\sum_{i=kh}^{(k+1)h-1}H^*_j(i)H_j(i)\lim_{n\to\infty}f_n\right|\F(kh-1)\right]\cr
&&=\E\left.\left[\sum_{i=kh}^{(k+1)h-1}H^*_j(i)H_j(i)f\right|\F(kh-1)\right]
~\text{a.s.},~j\in \mathcal V.
\ena
Let $\{x(k),\F(kh-1),k\ge 0\}$ be a $L_2$-bounded adaptive sequence with values in the Hilbert space $\X$, by (\ref{vnowlelkel}), we get
\ban
\HH_jx(k)=\E\left.\left[\sum_{i=kh}^{(k+1)h-1}H^*_j(i)H_j(i)x(k)\right|\F(kh-1)\right]~\text{a.s.},~j\in\mathcal V.
\ean
For any non-zero element $x$ in Hilbert space $x$, from Proposition 2.6.13 in \cite{hy}, the condition (\ref{tuiluntiaojian}) and (\ref{wpkfpkpew}), we have
\bna%\label{owwwww3}
\left\langle \sum_{j=1}^N\HH_jx,x\right\rangle=h\left\langle\E\left[\sum_{j=1}^N
H^*_j(0)H_j(0)x\right],x\right\rangle=h\sum_{j=1}^N\E\left[\|H_j(0)x\|^2\right]>0.\notag
\ena
Hence, by (\ref{owwwww2})-(\ref{vnkwenvefklw}), the above equality and Theorem \ref{vnknoklfl}, it is known that the algorithm (\ref{algorithm}) is both mean square and almost surely strongly consistent.
\end{proof}

The graph $\G$ describes the communication topology among nodes, and its connectivity ensures the nodes to collaboratively reconstruct the unknown function $f_0$ successfully. The condition (\ref{yinlitiaojian1}) in Lemma \ref{jihubiranshoulian} plays an important role in the convergence analysis of the decentralized algorithm, which we call the \textbf{\emph{infinite-dimensional spatio-temporal persistence of excitation}} condition.
%It is well known that the operator orbit $x\mapsto H_j(i)x$ can fully reflect the nature of the operator $H_j(i)$ itself. Intuitively, there exist deterministic time-invariant operators $\HH_j$, $j=1,\cdots,N$ such that the aggregated information of $H_j^*(i)H_j(i)$ over fixed-length time periods $[kh,(k+1)h-1]$ and the information provided by $\HH_j$ trend to be consistent over time.
Note that
\ban
&&~~~\left|\sum_{j=1}^N\sum_{i=kh}^{(k+1)h-1}\E\left[\|H_j(i)x\|_{\Y_j}^2\right]-\left\langle \sum_{j=1}^N\HH_jx,x\right\rangle _{\X}\right|\cr
&&=\left|\sum_{j=1}^N\left\langle \E\left[\sum_{i=kh}^{(k+1)h-1}H^*_j(i)H_j(i)x-\HH_jx\right],x\right\rangle _{\X}\right|\cr &&=\left|\sum_{j=1}^N\left\langle \E\left[\sum_{i=kh}^{(k+1)h-1}\E\left.\left[H^*_j(i)H_j(i)x\right|\F(kh-1)\right]-\HH_jx\right],x\right\rangle _{\X}\right|\cr
&&\leq \sum_{j=1}^N\left[\E\left[\left\|\HH_jx-\sum_{i=kh}^{(k+1)h-1}\E\left.\left[H_j^*(i)H_j(i)x\right|\F(kh-1)\right]\right\|_{\X}^2\right]\right]^{\frac{1}{2}}\|x\|_{\X},~x\in \X,~k\ge 0.
\ean
Therefore, the infinite-dimensional spatio-temporal persistence of excitation condition is equivalent to the combination of the following  two conditions.
\begin{itemize}
\item For any $x\in \X\setminus\{0\}$, there exists an integer $K(x)>0$ such that
\bna\label{vnkwwoelekel}
\sum_{j=1}^N\sum_{i=kh}^{(k+1)h-1}\E\left[\left\|H_j(i)x\right\|_{\Y_j}^2\right]>0,~k\ge K(x);
\ena
\item there exist deterministic time-invariant operators $\HH_j$, $j=1,\cdots,N$ such that
$$
\sum_{j=1}^N\sum_{k=0}^{\infty}\E\left[\left\|\HH_jx(k)-\sum_{i=kh}^{(k+1)h-1}\E\left.\left[H_j^*(i)H_j(i)x(k)\right|\F(kh-1)\right] \right\|_{\X}^2\right]<\infty,
$$
\end{itemize}
for any $L_2$-bounded adaptive sequence $\{x(k),\F(kh-1),k\ge 0\}$ with values in the Hilbert space $\X$.
%To reconstruct the unknown element $f_0$ under valid measurement information, the random forward operators are required to have

The \textbf{\emph{spatio-temporal persistence of excitation}} implies that
the non-zero orbits of the random forward operators of all nodes are non-degenerate in the mean square sense for a fixed length time period,  where \textbf{\emph{spatio-temporal}} refers specifically to the temporal and spatial states of the operator orbit $x\mapsto H_j(i)x$ of the random forward operator. By (\ref{vnkwwoelekel}) we know that we neither need the temporal orbit of the forward operator of each node over the graph to be non-degenerate, i.e., $$\sum_{i=kh}^{(k+1)h-1}\E \left[\|H_j(i)x\|_{\Y_j}^2\right]>0,\ \forall\ j\in \mathcal V,$$ nor need the spatial orbit of the forward operator of all nodes to be non-degenerate at each instant, i.e., $$\sum_{j=1}^N\E\left[\|H_j(i)x\|_{\Y_j}^2\right]>0,\ \forall\ i\ge 0.$$
Notably, if the sequences of random forward operator  and noises are both i.i.d. and they are mutually independent, then the \textbf{\emph{infinite-dimensional spatio-temporal persistence of excitation}} condition degenerates to the case that the spatial orbits of the forward operators of all nodes are non-degenerate at the initial moment $k=0$, i.e., the condition (i) in the Corollary \ref{xiaosirendetuilun}.

In the past decades, to solve the problems of finite-dimensional parameter estimation and signal tracking with non-stationary and non-independent data, many scholars have proposed excitation conditions based on the conditional expectation of the observation/regression matrix. The stochastic persistence of excitation condition was first proposed by Guo (\cite{Guo1990}) in the analysis of centralized Kalman filtering algorithms.  Xie and Guo (\cite{Xieguo}) proposed the cooperative information condition based on the conditional expectations of the observation matrices for the distributed adaptive filtering algorithm over connected graphs. Wang \emph{et al.} (\cite{WLZ}) proposed the stochastic spatio-temporal persistence of excitation condition for the decentralized online estimation algorithm over randomly time-varying graphs.
%and Zhang \emph{et al.} \cite{ZLF} further proposed a more general excitation condition: the sample path spatio-temporal persistence of excitation condition in the analysis of the decentralized online regularized regression algorithm over randomly time-varying graphs.
Zhang \emph{et al.} (\cite{ZLF}) proved that if the graph is connected and the randomly time-varying regression matrix satisfies the uniformly conditionally spatio-temporally joint observability condition, i.e., there exists an integer $h>0$ and a constant $\theta>0$, respectively, such that
\ban
\inf_{k\ge 0}\lambda_{\text{min}}\bigg(\sum_{j=1}^N\sum_{i=kh}^{(k+1)h-1}
\E\left[H_j^T(i)H_j(i)\big|\F(kh-1)\right]\bigg)\ge \theta \ \text{a.s.},
\ean
then the algorithm achieves mean square and almost sure convergence. For this case, it is not difficult to verify that the random matrix sequence $\{I_{Nn}-a(k)\H^*(k)\H(k)-b(k)\L_{\G}\otimes I_{n},k\ge 0\}$ satisfies the $L_2^2$-stability condition w.r.t. $\{\F(k),k\ge 0\}$, and thus the excitation conditions proposed in \cite{WLZ} and \cite{ZLF} are all special cases of the $L_2^2$-stability condition in Theorem \ref{dingliyi1}.

The persistence of excitation conditions proposed for finite-dimensional systems all require that the conditional expectation of the information matrix consisting of the observation (regression) matrices is positive definite, i.e., the information matrix has strictly positive minimum eigenvalues; however, inverse problems in infinite-dimensional Hilbert spaces are usually ill-posed. Even for a strictly positive linear compact operator, the excitation condition similar to the persistence excitation conditions for finite-dimensional systems can not hold any more since the infimum of the eigenvalues of the compact operator is always $0$.

\section{Decentralized Online Learning in Reproducing Kernel Hilbert Space}
We will discuss in this section a special class of online random inverse problems: decentralized online learning problems in reproducing kernel Hilbert spaces (RKHS). Let $\X$ be a non-empty subset of $\mathbb R^n$, $K:\X\times \X\to \mathbb R$ be a Mercer kernel, and $(\mathscr H_K,\langle \cdot,\cdot \rangle _K)$ be a reproducing kernel Hilbert space with kernel $K$, which is consisted of functions with domain $\X$.
%Suppose $f_0:\X\to \mathbb R$ is an unknown function in $\HH_K$. The nodes cooperatively estimate $f_0$ by information exchanging among them.
The observation data $y_i(k)$ of the $i$-th node at instant $k$ is given by
\bna\label{xuexi}
y_i(k)=f_0(x_i(k))+v_i(k),\ k\ge 0,\ i\in \mathcal V,
\ena
where $x_i(k):\Omega\to \X$ is a random vector with values in the Hilbert space $(\X,\tau_{\text{N}}(\X))$ at instant $k$, called the random input data, and the observation noise $v_i(k):\Omega\to \mathbb R$ is a random vector with values in the Hilbert space $(\mathbb R,\tau_{\text{N}}(\mathbb R))$,  and  $f_0:\X\to \mathbb R$ is an unknown function in $\HH_K$. The nodes cooperatively estimate $f_0$ by information exchanging among them.

\begin{remark}
\label{remarkaddadd6}
In \cite{kar2012}, the measurement of each node is given by $y_i(k)=f_i(\theta^*)+v_i(k)$, $i=1,2,...,N$, in which the nonlinear mappings $f_i(\cdot)$, $i=1,2,...,N$ are completely known in prior and it is the parameter vector $\theta^*$ in a finite-dimensional space, who is unknown and to be estimated. While for the measurement model (\ref{xuexi}), it is the mapping $f_0$ in the infinite-dimensional space $\HH_K$, who is unknown and to be estimated. %by using the information of the input data $x_i(k)$.
\end{remark}

For any given $x\in \X$, the function $K_x:\X\to \HH_K$ induced by the Mercer kernel is given by $K_x(y)=K(x,y),\ \forall\ y\in\X.$
Define the random forward operator $H_i(k)$:
$$
H_i(k)(f):=f(x_i(k)),~f\in \HH_K,~k\ge 0,~i\in \mathcal V,
$$
then the measurement model (\ref{xuexi}) can be represented as the random inverse problem with the measurement equation (\ref{measuramentmodel}). Based on the algorithm (\ref{algorithm}), the decentralized online learning strategy in $\HH_K$ is given by
\bna\label{rkhs}
&&\hspace{-0.8cm}f_i(k+1)=f_i(k)+a(k)(y_i(k)-f_i(k)(x_i(k)))K_{x_i(k)}+b(k)\sum_{j\in \N_i}a_{ij}(f_j(k)-f_i(k)),\cr
&&~~~~~~~~~~~~~~~~~~~~~~~~~~~~~~~~~~~~~~~~~~~~~~~~~~~~~~~~~~~~~~~~~~~~~~~~~~~~k\ge 0,~i\in \mathcal V.
\ena
Given $\phi,\psi\in \HH_K$, denote the rank 1 tensor product operator $\phi\otimes \psi:\HH_K\to \HH_K$ by
$$
\left(\phi\otimes \psi\right)(f):=\langle f,\psi\rangle _{K}\phi,~f\in \HH_K.
$$
Let
\[\F(k)=\bigvee_{s=0}^k\bigvee_{i=1}^N\left(\bigvee_{f\in \HH_K}\sigma\left(f(x_i(s));\tau_{\text{N}}(\mathbb R)\right)\bigvee \sigma \left(v_i(s);\tau_{\text{N}}(\mathbb R)\right)\right),~k\ge 0,\]
and $\F(-1)=\{\emptyset,\Omega\}$. For the algorithm (\ref{rkhs}), we need the following assumptions.



\begin{assumption}\label{assumption3}
The sequence $\{x_i(k),i\in \mathcal V,k\ge 0\}$ of random vectors with values in the Hilbert space $(\X,\tau_{\text{N}}(\X))$ and the sequence $\{v_i(k),i\in \mathcal V,k\ge 0\}$ of random variables with values in the Hilbert space $(\mathbb R,\tau_{\text{N}}(\mathbb R))$ are mutually independent.
\end{assumption}
%\textbf{B1.} The sequence $\{x_i(k),i\in \mathcal V,k\ge 0\}$ of random vectors with values in the Hilbert space $(\X,\tau_{\text{N}}(\X))$ and the sequence $\{v_i(k),i\in \mathcal V,k\ge 0\}$ of random variables with values in the Hilbert space $(\mathbb R,\tau_{\text{N}}(\mathbb R))$ are mutually independent.



\begin{assumption}\label{assumption4}
The noises $\{v_i(k),\F(k),k\ge 0\}$, $i\in \mathcal V$, are martingale difference sequences and there exists a constant $\b>0$, such that  $ \max_{i\in\mathcal V}\sup_{k\ge 0}\E\left.\left[\|v_i(k)\|_{\mathbb R}^2\right|\F(k-1)\right]\leq \b~\text{a.s.}$
\end{assumption}
%\textbf{B2.} The noises $\{v_i(k),\F(k),k\ge 0\}$, $i\in \mathcal V$, are martingale sequences and there exists a constant $\b>0$, such that $$\max_{i\in\mathcal V}\sup_{k\ge 0}\E\left.\left[\|v_i(k)\|_{\mathbb R}^2\right|\F(k-1)\right]\leq \b~\text{a.s.}$$



\begin{assumption}\label{assumption5}
$\sup_{x\in \X}K(x,x)<\infty$.
\end{assumption}
%\textbf{B3.} $\sup_{x\in \X}K(x,x)<\infty$.



\begin{remark}
\rm{The existing works on RKHS online learning (\cite{Ying}-\cite{SmaleYao}) all require the random input data to be i.i.d. Notice that in Assumption  \ref{assumption3}, the sequence of random forward operators with values in $(\mathscr L(\X,\bigoplus_{i=1}^N\Y_i),\tau_{\text{S}}(\mathscr L(\X,\bigoplus_{i=1}^N\Y_i)))$ is not required to satisfy special statistical properties such as independence, stationarity, etc.}
\end{remark}


%\begin{remark}
%\rm{Assumption \ref{assumption5} is often used in the learning theory in RKHS (\cite{Tarres}-\cite{Dieuleveut}, \cite{SmaleYao},  \cite{ctfg}). There are many kernel functions satisfying Assumption \ref{assumption5}, such as the Gaussian kernel $K:\mathbb R^n\times \mathbb R^n\to \mathbb R$, $K(x,y)=\text{e}^{-\frac{\|x-y\|_{\mathbb R^n}^2}{c^2}}$ and the homogeneous polynomial kernel $K:\mathbb S^{n-1}\times \mathbb S^{n-1}\to \mathbb R$ on the unit sphere in $\mathbb R^n$, $K(x,y)=\langle x,y\rangle_{\mathbb R^n}^d$, where the integer $d>0$. In addition to these common kernel functions, note that the Mercer kernel is continuous, thus Assumption  \ref{assumption5} holds for arbitrary compact set $\X\subseteq \mathbb R^n$.}
%\end{remark}

Noting the continuity of the Mercer kernel function, $K_x$ is therefore a continuous function. For any $f\in \HH_K$ and random input data $x_i(k)$, $i\in \mathcal V$, $k\ge 0$, it follows from the reproducing properties of RKHS that $f(x_i(k))=\langle f,K_{x_i(k) }\rangle _{K}$ (\cite{Poggio}), thus $f(x_i(k)):\Omega\to \mathbb R$ is a random variable with values in $(\mathbb R,\tau_{\text{N}}(\mathbb R))$. It follows from Assumption \ref{assumption5} that $H_i(k)$ is the random element with values in $(\LL(\HH_K,\mathbb R),\tau_{\text{S}}(\LL(\HH_K,\mathbb R)))$, which is induced by the random input data $x_i(k)$.

Based on the convergence results of the algorithm (\ref{algorithm}), we can obtain the following convergence results on the decentralized online learning algorithm (\ref{rkhs}) in RKHS.
%The proofs of theorems and corollaries of this section are in Appendix  \ref{appendixa1} and Appendix \ref{appendixc}.


\begin{theorem}\label{rkhsdingli}
For the algorithm (\ref{rkhs}), suppose that $\G$ is connected, Assumptions \ref{assumption3}-\ref{assumption5} and Conditions \ref{condition1}-\ref{condition3} hold. If there exist positive self-adjoint operators $N_i\in \mathscr L(\HH_K)$, $i=1,\cdots,N$ satisfying $\sum_{i=1}^NN_i>0$ and there exists an integer $h>0$, such that
  \bna\label{vnknknldldklsd}
\sum_{j=1}^N\sum_{k=0}^{\infty}\E\left[\Bigg\|\Bigg(N_j-\sum_{i=kh}^{(k+1)h-1}\E\left.\left[K_{x_j(i)}\otimes K_{x_j(i)}\right|\F( kh-1)\right]\Bigg)g(k)\Bigg\|_K^2\right]<\infty,
  \ena
 for arbitrary $L_2$-bounded adaptive sequence $\{g(k),\F(kh-1),k \ge 0\}$, then the algorithm (\ref{rkhs}) is both mean square and almost surely strongly consistent.
\end{theorem}
\begin{proof}
Let $H_i(k)$ be the mapping induced by random input data $x_i(k)$, where
$
H_i(k)(f)=f(x_i(k)),~f\in \HH_K,~k\ge 0,~i\in \mathcal V.
$
For any given $f_1,f_2\in \HH_K$ and $c_1,c_2\in \mathbb R$, it follows from the reproducing property of $\HH_K$ that
\begin{align}\label{vlwlmmfff}
&H_i(k)(c_1f_1+c_2f_2)\notag\\
=&\langle c_1f_1+c_2f_2,K_{x_i(k)}\rangle _K\cr
=&c_1H_i(k)(f_1)+c_2H_i(k)(f_2),~k\ge 0,~i\in \mathcal V,
\end{align}
thus, $H_i(k)$ is a linear operator. Noting the continuity of Mercer kernel $K:\X\times \X\to \mathbb R$, we know that the function $K_x:\X\to \HH_K$ induced by the Mercer kernel $K$ is also continuous. It is known from  $\HH_K=\overline{\textbf{span}\{K_x,x\in\X\}}$ that $f\in \HH_K$ is a Borel measurable function, which implies that  $H_i(k)(f)=f(x_i(k))$ is a random variable with values in the Hilbert space $(\mathbb R,\tau_{\text{N}}(\mathbb R))$. By Assumption \ref{assumption5} and the reproducing property of $\HH_K$, we have
\bna\label{vnklooeoeeee}
\|K_x\|_K=\sqrt{K(x,x)}\leq \sup_{x\in\X}\sqrt{K(x,x)}<\infty.
\ena
For any given sample path $\omega\in\Omega$, by (\ref{vnklooeoeeee}), we get $|H_i(k)(\omega)(f)|
= \left|\left\langle f,K_{x_i(k)(\omega)}\right\rangle _K\right|
 \leq  \sup\limits_{x\in\X} \sqrt{K(x,x)}  \|f\|_K,   \ \forall\ f\in \HH_K,~k\ge 0,~i\in \mathcal V,$
%\begin{align}\label{vnlwkfeemmff}
%&|H_i(k)(\omega)(f)|\cr
%=&\left|\left\langle f,K_{x_i(k)(\omega)}\right\rangle _K\right|
% \leq  \sup_{x\in\X}\sqrt{K(x,x)}\|f\|_K,\cr
% &\ \ \ \  \ \ \ \ \ \ \ \  \ \ \ \ \ \ \ \ \ ~\forall\ f\in \HH_K,~k\ge 0,~i\in \mathcal V,
%\end{align}
then $\|H_i(k)\|_{\LL(\HH_K,\mathbb R)}\leq \sup\limits_{x\in\X}\sqrt{K(x,x)}\   \text{a.s.}$ By (\ref{vlwlmmfff})  and Proposition \ref{nlllwwieiie}, we know that $H_i(k):\Omega\to\LL(\HH_K,\mathbb R)$ is a random element with values in the topological space $(\LL(\HH_K,\mathbb R), \tau_{\text{S}}(\LL(\HH_K,\mathbb R)))$. Denote $\H(k):=\text{diag}\{H_1(k),\cdots,H_N(k)\}$ and $v(k):=(v_1(k), \cdots,v_N(k))$, it follows from Definition \ref{tuopukongjian} that $\H(k)$ is the random element with values in the  space $(\LL(\HH_K^N,\mathbb R^N),\tau_{\text{S}}(\LL(\HH_K^N,\mathbb R^N)))$, and $v(k)$ is a random vector with values in the Hilbert space $(\mathbb R^N,\tau_{\text{N}}(\mathbb R^N))$. Thus, by Proposition \ref{nlllwwieiie}, we have
\begin{align}\label{nvwjoijjff}
\F(k)= \bigvee_{s=0}^k\bigg(\sigma\left(\H(s);\tau_{\text{S}}\left(\LL\left(\HH_K^N,\mathbb R^N\right)\right)\right)
 \bigvee \sigma\left(v(s);\tau_{\text{N}}\left(\mathbb R^N\right)\right)\bigg),~k\ge 0.
\end{align}
It follows from Assumption \ref{assumption3} and (\ref{nvwjoijjff}) that Assumption \ref{assumption1} holds. By Assumption \ref{assumption4} and (\ref{nvwjoijjff}), it is known that $\{v(k),\F(k),k\ge 0\}$ is the martingale difference sequence and there exists a constant $\b_v:=N\b>0$, such that $ \sup\limits_{k\ge 0}\E\big[\|v(k)\|^2|\F(k-1)\big]
 \leq   N\max_{i\in \mathcal V}\sup\limits_{k\ge 0}   \E \big[\|v_i(k)\|^2 |\F(k-1)\big]\leq \b_v~\text{a.s.},$
%\ban
%&& \sup_{k\ge 0}\E\left.\left[\|v(k)\|^2\right|\F(k-1)\right]\cr
%&\leq & N\max_{i\in \mathcal V}\sup_{k\ge 0}\E\left.\left[\|v_i(k)\|^2\right|\F(k-1)\right]\leq \b_v~\text{a.s.},
%\ean
which implies that Assumption \ref{assumption2} holds.
Let $x\in \X$, for any given $f_1,f_2\in \HH_K$ and $c_1,c_2\in \mathbb R$, we get
\bna\label{vmkwmefkeml1}
&&\left(K_x\otimes K_x\right)(c_1f_1+c_2f_2)\cr
&=&c_1f_1(x)K_x+c_2f_2(x)K_x\cr
&=&c_1\left(K_x\otimes K_x\right)f_1+c_2\left(K_x\otimes K_x\right)f_2,
\ena
from (\ref{vnklooeoeeee}), Assumption \ref{assumption5} and the reproducing property of $\HH_K$, we know that $\left\|\left(K_x\otimes K_x\right)f\right\|_{K}\\ \leq \|K_x\|_K^2\|f\|_K
\leq   \sup_{x\in\X}K(x,x)\|f\|_K,~\forall\ f\in \HH_K,~\forall\ x\in \X.$
%\begin{align}\label{vmkwmefkeml2}
%&\left\|\left(K_x\otimes K_x\right)f\right\|_{K}\leq \|K_x\|_K^2\|f\|_K\cr
%\leq & \sup_{x\in\X}K(x,x)\|f\|_K,~\forall\ f\in \HH_K,~\forall\ x\in \X.
%\end{align}
Thus, it follows from (\ref{vmkwmefkeml1}) that $K_x\otimes K_x\in \LL(\HH_K)$. Let $x=\sum_{i=1}^n\1_{A_i}\otimes x_i$ be a random vector with values in the Hilbert space $(\X,\tau_{\text{N}}(\X))$, where $A_i\cap A_j=\emptyset$, $1\leq i\neq j\leq n$. Noting that $K_{x}=\sum_{i=1}^n\1_{A_i}\otimes K_{x_i}$, we have $K_{x}\otimes K_{x}= \left(\sum_{i=1}^n\1_{A_i}\otimes K_{x_i}\right)\otimes \left(\sum_{i=1}^n\1_{A_i}\otimes K_{x_i}\right)
= \sum_{i=1}^n\1_{A_i}\otimes \left(K_{x_i}\otimes K_{x_i}\right),$
%\begin{align}
%K_{x}\otimes K_{x}=&\left(\sum_{i=1}^n\1_{A_i}\otimes K_{x_i}\right)\otimes \left(\sum_{i=1}^n\1_{A_i}\otimes K_{x_i}\right)\notag\\
%=&\sum_{i=1}^n\1_{A_i}\otimes \left(K_{x_i}\otimes K_{x_i}\right),\notag
%\end{align}
thus, $K_{x}\otimes K_{x}$ is a simple function with values in the Banach space $\LL(\HH_K)$. For any given random vector $x$ with values in the Hilbert space $(\X,\tau_{\text{N}}(\X))$, it is known that there exists a simple function sequence $\{x_n,n\ge 0\}$ with values in $\X$, such that $\lim_{n\to\infty}\|x-x_n\|=0~\text{a.s.}$. This together with the reproducing property of $\HH_K$ and the symmetry of Mercer kernel $K$ gives
\begin{align}\label{cnkwmekmk}
&\left\|K_x-K_{x_n}\right\|^2_K = \left\langle K_x-K_{x_n},K_x-K_{x_n} \right\rangle_K \notag\\
  = & K(x,x)-2K(x,x_n)+K(x_n,x_n).
\end{align}
Noting the continuity of Mercer kernel $K$ and Assumption \ref{assumption5}, by (\ref{cnkwmekmk}), we get
\bna\label{nncknkwnkwc1}
\lim_{n\to\infty}\left\|K_x-K_{x_n}\right\|_K=0~\text{a.s.}
\ena
It follows from (\ref{vnklooeoeeee}) and the reproducing property of $\HH_K$ that
 %$ \|K_{x}\otimes K_{x}-K_{x_n}\otimes K_{x_n} \|_{\LL(\HH_K)}
%\leq    \|(K_x -K_{x_n}) \otimes K_x \|_{\LL(\HH_K)}
% + \|K_{x_n}\otimes (K_x-K_{x_n}) \|_{\LL(\HH_K)}
% = \sup_{\|f\|_K=1} \| ((K_x-K_{x_n})\otimes K_x )f \|_K
%  +\sup_{\|f\|_K=1} \| (K_{x_n}\otimes (K_x-K_{x_n}) )f \|_K
% =  \sup_{\|f\|_K=1} \|f(x)(K_x-K_{x_n}) \|_K
% +\sup_{\|f\|_K=1}  \|(f(x)-f(x_n))K_{x_n} \|_K
% \leq  \|K_x\|_K\|K_x-K_{x_n}\|_K+\|K_{x_n}\|_K \|K_x-K_{x_n}\|_K
% \leq   2\sup_{x\in \X}\sqrt{K(x,x)}  \|K_x-K_{x_n}\|_K~\text{a.s.}$
\begin{align}
&\left\|K_{x}\otimes K_{x}-K_{x_n}\otimes K_{x_n}\right\|_{\LL(\HH_K)}\cr
\leq & \left\|(K_x-K_{x_n})\otimes K_x\right\|_{\LL(\HH_K)} +\left\|K_{x_n}\otimes (K_x-K_{x_n})\right\|_{\LL(\HH_K)}\cr
 =&\sup_{\|f\|_K=1}\left\|\left((K_x-K_{x_n})\otimes K_x\right)f\right\|_K +\sup_{\|f\|_K=1}\left\|\left(K_{x_n}\otimes (K_x-K_{x_n})\right)f\right\|_K\cr
 =& \sup_{\|f\|_K=1}\left\|f(x)(K_x-K_{x_n})\right\|_K +\sup_{\|f\|_K=1}\left\|(f(x)-f(x_n))K_{x_n}\right\|_K\cr
 \leq & \|K_x\|_K\|K_x-K_{x_n}\|_K+\|K_{x_n}\|_K\|K_x-K_{x_n}\|_K
 \leq   2\sup_{x\in \X}\sqrt{K(x,x)}\|K_x-K_{x_n}\|_K~\text{a.s.}\notag
\end{align}
Then, by Assumption \ref{assumption5} and (\ref{nncknkwnkwc1}), we have $\lim_{n\to\infty}\left\|K_{x}\otimes K_{x}-K_{x_n}\otimes K_{x_n}\right\|_{\LL(\HH_K)}  =0~\text{a.s.}$
%\ban
%\lim_{n\to\infty}\left\|K_{x}\otimes K_{x}-K_{x_n}\otimes K_{x_n}\right\|_{\LL(\HH_K)}=0~\text{a.s.}
%\ean
Noting that $K_{x_n}\otimes K_{x_n}$ is the simple function with values in Banach space $\LL(\HH_K)$, by Definition \ref{vnwkelel}, it is known that $K_{x}\otimes K_{x}$ is strongly measurable with respect to the topology $\tau_{\text{N}}(\LL(\HH_K))$, which together with Pettis measurability theorem shows that $K_{x}\otimes K_{x}$ is the random element with values in Banach space $(\LL(\HH_K),\tau_{\text{N}}(\LL(\HH_K)))$. Thus, by Assumption \ref{assumption5}, we get $K_{x_j(i)}\otimes K_{x_j(i)}\in L^1(\Omega;\mathscr L(\HH_K))$, which together with Lemma \ref{nvkvpeoeo} gives the fact that $\E[K_{x_j(i)}\otimes K_{x_j(i)}|\F(kh-1)]$ uniquely exists. Let $\{g(k),\F(kh-1),k\ge 0\}$ be the $L_2$-bounded adaptive sequence with values in $\HH_K$, by Assumption \ref{assumption5}, Proposition \ref{tiaojianqiwangxingzhi} and the condition (\ref{vnknknldldklsd}), we obtain
\begin{align}\label{vnkwenkfffnkfmkweklf}
&\sum_{j=1}^N\sum_{k=0}^{\infty}\E\Bigg[\bigg\|
\sum_{i=kh}^{(k+1)h-1}\E\left.\left[H_j^*(i)H_j(i)g(k)\right|
\F(kh-1)\right] -N_jg(k)\bigg\|_K^2\Bigg]\cr
 =&\sum_{j=1}^N\sum_{k=0}^{\infty}\E\Bigg[\bigg\|
 \sum_{i=kh}^{(k+1)h-1}\E\left.\left[K_{x_j(i)}\otimes K_{x_j(i)}g(k)\right|\F(kh-1)\right] -N_jg(k)\bigg\|_K^2\Bigg]\cr
 =& \sum_{j=1}^N\sum_{k=0}^{\infty}\E\Bigg[\bigg\|\Big(\sum_{i=kh}^{(k+1)h-1}\E\left.\left[K_{x_j(i)}\otimes K_{x_j(i)}\right|\F(kh-1)\right] -N_j\Big)g(k)\bigg\|_K^2\Bigg]
 <\infty.
\end{align}
Denote $\rho_0=N\sup_{x\in\X}K(x,x)$. Given the integer $h>0$, by Assumption \ref{assumption5} and (\ref{vnklooeoeeee}), we have
\begin{align}
& \E\left.\left[\|\H^*(k)\H(k)\|_{\mathscr L\left(\HH_K^N\right)}^{2^{\max\{h,2\}}}\right|\F(k-1)\right] \cr
\leq &
% N  \E\left.\left[\sup_{i\in \mathcal V}\left\|H_i^*(k)H_i(k)\right\|_{\LL(\HH_K)}^{2^{\max\{h,2\}}}\right|\F(k-1)\right] \cr
% =&
 N \E\left.\left[\sup_{i\in \mathcal V}\left\|K_{x_i(k)}\otimes K_{x_i(k)}\right\|_{\LL(\HH_K)}^{2^{\max\{h,2\}}}\right|\F(k-1)\right] \cr
\leq&
%N \E\left.\left[\sup_{i\in \mathcal V}\sup_{\|f\|_K=1}\left\|\left(K_{x_i(k)}\otimes K_{x_i(k)}\right)f\right\|_K^{2^{\max\{h,2\}}}\right|\F(k-1)\right] \cr
% =&
 N \E\left.\left[\sup_{i\in \mathcal V}\sup_{\|f\|_K=1}\left\|f(x_i(k))K_{x_i(k)}\right\|_K^{2^{\max\{h,2\}}}
 \right|\F(k-1)\right] \cr
 \leq &
N \E \Bigg[\sup_{i\in \mathcal V}\sup_{\|f\|_K=1}|f(x_i(k))|^{2^{\max\{h,2\}}}  \left\|K_{x_i(k)}
\right\|_K^{2^{\max\{h,2\}}}\Bigg|\F(k-1)\Bigg] \cr
 \leq & N \E\Bigg[\sup_{i\in \mathcal V}\sup_{\|f\|_K=1}\left\|f\right\|^{2^{\max\{h,2\}}}_K  \left(
\sup_{x\in\X}K(x,x)\right)^{2^{\max\{h,2\}}}\Bigg|\F(k-1)\Bigg]\cr
 \leq & N\sup_{x\in\X} K(x,x)
 =\rho_0~\text{a.s.},\notag
\end{align}
which gives
 %$\sup_{k\ge 0}\left(\E\left.\left[\|\H^*(k)\H(k)\|_{\mathscr L\left(\HH_K^N\right)}^{2^{\max\{h,2\}}}\right|\F(k-1)\right]\right)^{\frac{1}{2^{
%\max\{h,2\}}}}\leq \rho_0~\text{a.s.}$
\begin{align}
 \sup_{k\ge 0}\left(\E\left.\left[\|\H^*(k)\H(k)\|_{\mathscr L\left(\HH_K^N\right)}^{2^{\max\{h,2\}}}\right|\F(k-1)\right]\right)^{\frac{1}{2^{
\max\{h,2\}}}}
%=& N \sup_{k\ge 0}\left(\E\left.\left[\sup_{i\in \mathcal V}\left\|H_i^*(k)H_i(k)\right\|_{\LL(\HH_K)}^{2^{\max\{h,2\}}}\right|\F(k-1)\right]\right)^{\frac{1}{2^{\max\{h,2\}}}}\cr
% =&N\sup_{k\ge 0}\left(\E\left.\left[\sup_{i\in \mathcal V}\left\|K_{x_i(k)}\otimes K_{x_i(k)}\right\|_{\LL(\HH_K)}^{2^{\max\{h,2\}}}\right|\F(k-1)\right]\right)^{\frac{1}{2^{\max\{h,2\}}}}\cr
%\leq& N\sup_{k\ge 0}\left(\E\left.\left[\sup_{i\in \mathcal V}\sup_{\|f\|_K=1}\left\|\left(K_{x_i(k)}\otimes K_{x_i(k)}\right)f\right\|_K^{2^{\max\{h,2\}}}\right|\F(k-1)\right]\right)^{\frac{1}{2^{\max\{h,2\}}}}\cr
% =&N\sup_{k\ge 0}\left(\E\left.\left[\sup_{i\in \mathcal V}\sup_{\|f\|_K=1}\left\|f(x_i(k))K_{x_i(k)}\right\|_K^{2^{\max\{h,2\}}}\right|\F(k-1)\right]\right)^{\frac{1}{2^{\max\{h,2\}}}}\cr
% \leq &
%N\sup_{k\ge 0}\left(\E\left.\left[\sup_{i\in \mathcal V}\sup_{\|f\|_K=1}|f(x_i(k))|^{2^{\max\{h,2\}}}\left\|K_{x_i(k)}\right\|_K^{2^{\max\{h,2\}}}\right|\F(k-1)\right]\right)^{\frac{1}{2^{\max\{h,2\}}}}\cr
% \leq & N\sup_{k\ge 0}\left(\E\left.\left[\sup_{i\in \mathcal V}\sup_{\|f\|_K=1}\left\|f\right\|^{2^{\max\{h,2\}}}_K\left(
%\sup_{x\in\X}K(x,x)\right)^{2^{\max\{h,2\}}}\right|\F(k-1)\right]
%\right)^{\frac{1}{2^{\max\{h,2\}}}}\cr
% \leq & N\sup_{x\in\X} K(x,x)
 \leq  \rho_0~\text{a.s.} \notag
\end{align}
It follows from Condition \ref{condition2} that there exists a constant $j_0>0$ and an integer $t_0>0$, such that $\sup_{k\ge 0}(4\rho_0a(k)+4\|\L_{\G}\|b(k))\leq j_0$ and $\sup_{k\ge t_0}(a(k) +b(k)) (4\rho_0+4\|\L_{\G}\|)\leq 1$. Noting that $ \|4a(k)\H^*(k)\H(k)+4b(k)  \L_{\G}\otimes I_{\HH_K} \|_{\LL\left(\HH_K^N\right)}\leq j_0$ if $k<t_0$, and $ \|4a(k)\H^*(k)\H(k)+4b(k)\L_{\G}\otimes I_{\HH_K} \|_{\LL\left(\HH_K^N\right)}\leq 1$, otherwise.
%\begin{align}\label{vkllekkeek}
%&\left\|4a(k)\H^*(k)\H(k)+4b(k)\L_{\G}\otimes I_{\HH_K}\right\|_{\LL\left(\HH_K^N\right)}\notag\\
%\leq &
%\begin{cases}
%j_0, & k<t_0;\\
%1, & k\ge t_0
%\end{cases}
%~\text{a.s.}
%\end{align}
Then, we get
\begin{align}\label{vnksdkjdkdddd}
&\big\|I_{\HH_K^N}-4\left(a(k)\H^*(k)\H(k)+b(k)\L_{\G}\otimes I_{\HH_K}\right)\big\|_{\LL\left(\HH_K^N\right)}
\leq
1+\Gamma(k)
~\text{a.s.},
\end{align}
where
% $\Gamma(k)=j_0$, if $k<t_0$, and  $\Gamma(k)=0$, otherwise,
\ban
\Gamma(k)=
\begin{cases}
j_0, & k<t_0,\\
0, & k\ge t_0,
\end{cases}
\ean
and satisfies $\sum_{k=0}^{\infty}\Gamma(k)<\infty$. It follows from (\ref{vnksdkjdkdddd}) that
\begin{align}
 &\E\Big[\big\|I_{\HH_K^N}-4(a(k)\H^*(k)\H(k)
  +b(k)\L_{\G}\otimes I_{\HH_K})\Big\|_{\LL\left(\HH_K^N\right)}\big|\F(kh-1)\Big]
 \leq 1+\Gamma(k)~\text{a.s.}\notag
\end{align}
Hence, combining (\ref{vnkwenkfffnkfmkweklf})-(\ref{vnksdkjdkdddd}), the above inequality and Theorem \ref{vnknoklfl} gives the fact that the algorithm (\ref{rkhs}) is mean square and almost surely strongly consistent.
\end{proof}
\vskip 1mm

We now give some corollaries.

\vskip 1mm

\begin{corollary} \label{vnlllleleeemmem}
For the algorithm (\ref{rkhs}), suppose that $\G$ is connected,  Assumptions \ref{assumption3}-\ref{assumption5} and Conditions \ref{condition1}-\ref{condition3} hold. If there exist positive self-adjoint operators $N_i\in \mathscr L(\HH_K)$, $i=1,\cdots,N$ satisfying $\sum_{i=1}^NN_i>0$, and there exists an integer $h>0$, a constant $\mu_0>0$ and a nonnegative real sequence $\{\tau(k),k\ge 0\}$, respectively, such that
\bna\label{vnkmeeeemefffff}
\max_{j\in \mathcal V}\Bigg\|N_j-\sum_{i=kh}^{(k+1)h-1}\E\left.\left[K_{x_j(i)}\otimes K_{x_j(i)}\right|\F(kh-1)\right]\Bigg\|_{\LL( \HH_K)}^2\leq \mu_0\tau(k)~\text{a.s.},
\ena
where $\sum_{k=0}^{\infty}\tau(k)<\infty$, then the algorithm (\ref{rkhs}) is both mean square and almost surely strongly consistent. Besides, the algorithm (\ref{rkhs}) is pointwisely almost surely strongly consistent, that is, $\lim\limits_{k\to \infty} f_{i}(k)(x)=f_{0}(x) $   \text{ a.s.},  $\forall \ x \in \mathscr{X}, \ i \in \mathcal{V}$.
%\begin{align}
%\lim_{k\to \infty} f_{i}(k)(x)=f_{0}(x), \ \text{a.s.}, \ \forall \ x \in \mathscr{X}, \ i \in \mathcal{V};\label{almostpointcon}\\
%\lim_{k\to \infty}\E\left[\left\| f_{i}(k)(x)-f_{0}(x)\right\|^{2}\right]=0, \ \text{a.s.}, \ \forall \ x \in \mathscr{X}, \ i \in \mathcal{V};\label{almostpointcon}
%\end{align}.
\end{corollary}
\begin{proof}
It follows from Assumption \ref{assumption5} and Theorem \ref{rkhsdingli} that $K_{x_j(i)}\otimes K_{x_j(i)}\in L^1(\Omega;\mathscr L(\HH_K))$, which together with Lemma \ref{nvkvpeoeo} implies that $\E[K_{x_j(i)}\otimes $ $K_{x_j(i)}|\F(kh-1)]$ uniquely exists. Let $\{g(k),\F(kh-1),k\ge 0\}$ be a $L_2$-bounded adaptive sequence with values in $\HH_K$, by the condition (\ref{vnkmeeeemefffff}), we get
\ban
&&~~~~\sum_{j=1}^N\sum_{k=0}^{\infty}\E\left[\Bigg\|\Bigg(N_j-\sum_{i=kh}^{(k+1)h-1}\E\left.\left[K_{x_j(i)}\otimes K_{x_j(i)}\right|\F(kh-1)\right]\Bigg)g(k)\Bigg\|_K^2\right]\cr
&&\leq \sum_{j=1}^N\sum_{k=0}^{\infty}\E\left[\Bigg\|N_j-\sum_{i=kh}^{(k+1)h-1}\E\left.\left[K_{x_j(i)}\otimes K_{x_j(i)}\right|\F(kh-1)\right]\Bigg\|_{\LL(\HH_K)}^2\|g(k)\|_K^2\right]\cr
&&\leq \sum_{j=1}^N\sum_{k=0}^{\infty}\E\left[\max_{j\in\mathcal V}\Bigg\|N_j-\sum_{i=kh}^{(k+1)h-1}\E\left.\left[K_{x_j(i)}\otimes K_{x_j(i)}\right|\F(kh-1)\right]\Bigg\|_{\LL(\HH_K)}^2\|g(k)\|_K^2\right]\cr
&&\leq N\mu_0\sup_{k\ge 0}\E\left[\|g(k)\|_K^2\right]\sum_{k=0}^{\infty}\tau(k)
  <\infty,
\ean
where the last inequality is obtained from $\sum_{k=0}^{\infty}\tau(k)<\infty$. This together with Theorem \ref{rkhsdingli} implies that the algorithm (\ref{rkhs}) is both mean square and almost surely strongly consistent.

By Cauchy-Schwarz inequality and the reproducing property of $\HH_K$, we have
\begin{align}
|f_{i}(k)(x)-f_{0}(x)|=|\langle f_{i}(k)-f_{0}, K_{x}\rangle| \leq \left\|f_{i}(k)-f_{0}\right\|_{K}  \left\|K_{x}\right\|_{K}, \ \forall \ x \in \mathscr{X}, \ i \in \mathcal{V}.\notag
\end{align}
  Noting that algorithm (\ref{rkhs}) is   almost surely strongly consistent and by the above inequality, we have  $\lim\limits_{k\to \infty} f_{i}(k)(x)=f_{0}(x) $   \text{ a.s.},  $\forall \ x \in \mathscr{X}, \ i \in \mathcal{V}$.
\end{proof}


\begin{remark}
Noting that Assumption \ref{assumption5} implies that $K_{x_j(i)}\otimes K_{x_j(i)}$ is a Bochner integrable random element with values in the Banach space $(\mathscr L(\HH_K),\tau_{\text{N}}(\mathscr L(\HH_K)))$, then the conditional expectations $\E[K_{x_j(i)}\otimes K_{x_j(i)}|\F(kh-1)]$, $i,k\ge 0$, $j\in \mathcal V$ uniquely exist by Lemma \ref{nvkvpeoeo}.
\end{remark}



Especially, if the input data $\{(x_1(k),\cdots,x_N(k)),k\ge 0\}$ are i.i.d, then we have the following corollary.



\begin{corollary}\label{rkhsdinglijjjjj}
For the algorithm (\ref{rkhs}), suppose that $\{(x_1(k),\cdots,x_N(k)),k\ge 0\}$ and $\{(v_1(k),\cdots,\\v_N(k)),k\ge 0\}$ are i.i.d. sequences and they are mutually independent, and $\G$ is connected. If Assumptions \ref{assumption3}-\ref{assumption5} and Conditions \ref{condition1}-\ref{condition3} hold, and
  \bna\label{nklnkle}
\E\Bigg[\sum_{j=1}^NK_{x_j(0)}\otimes K_{x_j(0)}\Bigg]>0,
  \ena
then the algorithm (\ref{rkhs}) is both mean square and almost surely strongly consistent.
\end{corollary}
\begin{proof}
Noting that $\{(x_1(k),\cdots,x_N(k)),k\ge 0\}$ and $\{(v_1(k),\cdots ,v_N(k)),k\ge 0\}$ are both i.i.d. sequences and they are mutually independent, we have
\begin{align*}
&\sum_{i=kh}^{(k+1)h-1}\E\left.\left[K_{x_j(i)}\otimes K_{x_j(i)}\right|\F(kh-1)\right]  =  \sum_{i=kh}^{(k+1)h-1}\E\left[K_{x_j(i)}\otimes K_{x_j(i)}\right]
 =   h\E\left[K_{x_j(0)}\otimes K_{x_j(0)}\right].
\end{align*}
Denote $N_j:\HH_K\to \HH_K$ by
$
N_j=h\E\left[K_{x_j(0)}\otimes K_{x_j(0)}\right],~j\in \mathcal V.
$
It follows from Proposition 2.6.13 in \cite{hy} that $N_j\in \mathscr L(\HH_K)$ is a positive self-adjoint operator. Noting that
\ban
\bigg\|N_j-\sum_{i=kh}^{(k+1)h-1}\E\left.\left[K_{x_j(i)}\otimes K_{x_j(i)}\right|\F(kh-1)\right]\bigg\|_{\LL(\HH_K)}^2=0~\text{a.s.},
\ean
by the condition (\ref{nklnkle}), we get
\ban
\left\langle \sum_{j=1}^NN_jf,f\right\rangle _K =h\left\langle \E\left[\sum_{j=1}^NK_{x_j(0)}\otimes K_{x_j(0)}\right]f,f\right\rangle _K >0,
\ean
where $f$ is an arbitrary non-zero function in $\HH_K$. Hence, it follows from Corollary \ref{vnlllleleeemmem} that the algorithm (\ref{rkhs}) is both mean square and almost surely strongly consistent.
\end{proof}

\begin{remark}
For the centralized online learning with input data drawn independently from the probability measure $\rho_{\X}$ on $\X$, Tarres and Yao (\cite{Tarres}) defined the covariance operator of the probability measure $\rho_{\X}$ in $\HH_K$ as $L_K:\HH_K\to \HH_K$,
$$
L_K(f)(y)=\int_{\Omega}K(x,y)f(x)\dd\rho_{\X},~\forall\ f\in \HH_K.
$$
By the reproducing property and Assumption \ref{assumption5}, it follows that $L_K=\E[K_x\otimes K_x]$. This implies that for the centralized online learning problem in RKHS with i.i.d. data, the condition (\ref{nklnkle}) in Corollary \ref{rkhsdinglijjjjj} just degenerates to that in \cite{Tarres}: the covariance operator $L_K>0$.
\end{remark}

\section{Numerical Simulation}
We consider a undirected connected  graph with node set $\mathcal{V}= \{1,2,\ldots, 10\}$  and its  weighted adjacency  matrix is given by $A=[a_{i,j}]$, where  $a_{1,2}= a_{2,1}=0.2,$ $a_{1,4}=a_{4,1}=0.4$, $a_{2,3}=a_{3,2}=0.1$, $a_{2,4}=a_{4,2}=0.3$, $a_{3,5}=a_{5,3}=0.5$, $a_{4,5}=a_{5,4}=0.6$, $a_{4,6}=a_{6,4}=0.8$, $a_{5,6}=a_{5,6}=0.7,$ $a_{6,7}=a_{7,6}=0.3$, $a_{7,8}=a_{8,7}=0.2$, $a_{8,9}=a_{9,8}=0.9$, $a_{9,10}=a_{10,9}=0.1$ and for the remaining positions, $a_{i,j}=0.$
	%\begin{align*}
%		A=[a_{ij}]
%		=
%		\left(
%		\begin{array}{cccccccccc}
%			0.1 &   0.2 &  0 &  0.4 &  0 &  0 &  0 &  0 &  0 &  0\\
% 0.2 &   0 &  0.1  &  0.3 &  0 &  0 &  0 &  0 &  0 &  0\\
% 0 &  0.1 &  0 &  0 &  0.5 &  0 &  0 &  0  & 0  & 0 \\
%0.4 &  0.3  &  0 &  0  &  0.6 &  0.8 &  0  & 0 &  0 &  0\\
% 0 &  0 &  0.5 &   0.6 &  0  & 0.7 &  0 &  0  & 0 &  0\\
% 0  & 0  &  0 &  0.8 &  0.7 &  0  & 0.3  & 0 &  0 &  0\\
% 0  & 0 &  0  & 0 &  0 &  0.3 &  0 &  0.2 &  0 &  0\\
% 0 &  0 &  0 &  0 &  0 &  0  & 0.2 &  0 &  0.9 &  0\\
% 0  & 0  & 0 &  0  & 0 &  0 &  0  & 0.9 &  0 &  0.1\\
% 0  & 0  & 0 &  0 &  0 &  0 &  0  & 0 &  0.1  & 0
%		\end{array}
%		\right ).
%	\end{align*}



For $i\in \mathcal{V}$, the observation data of node $i$ at instant $k$ is $(x_{i}(k),y_{i}(k))$, where
 $y_{i}(k) = f^*(x_{i}(k)) + v_{i}(k)$,  $f^*(x) =e^{-(x-1)^2},\  \forall \ x \in \mathscr{X} = [-2,4]$ is the unknown true function to be estimated, the input data $ x_{i}(k),\ i = 1,2,\ldots, 10,\ k\in \mathbb{N}$ are  independent random variables sampling according to the following rules. For  $k\in \mathbb{N}$,   $ x_{i}(2k)$ and $ x_{i}(2k+1)$ are   with uniform distributions on $\big[-2,4- \frac{3}{k+1}\big]$ and   $\big[\frac{3}{ k+1}-2,4\big]$, respectively.  The measurement noises $v_{i}(k),\ i\in \mathcal{V},\  k\in \mathbb{N}$ are independent random variables with the same normal distribution $N(0, 0.1)$ and independent of the input data. Take the kernel function as $K(x,y) = e^{-(x-y)^2},\  \forall \ x,\ y \in \left[-2,4\right]$ and  we know that the unknown true function $f^* = K(x,1) \in \mathscr{H}_K$, where $\mathscr{H}_K$ is the reproducing kernel Hilbert space  with kernel $K$.
 The algorithm   gains are
	$ a(k) =   (k+1)^{-0.6},\ b(k) = (k+1)^{-1},\forall\ k\geq 1. $
	
It can be verified that all conditions in Corollary \ref{vnlllleleeemmem}  are satisfied.
%Denote  the distribution of  $x_{i}(k)$ as $\rho_{i,k}$ and
%the probability density function of $\rho_{i,k}$ as $m_{i,k}(x)$.
%Denote $\rho_{i}$ as the probability measure induced by a random variable with the uniform distribution on $[-2,4]$ and $m_{i}(x) = \frac{1}{6}, \, x \in [-2,4],$  as its probability density function, $ \forall \
% i\in \mathcal{V}$.
%Denote $N_i=2\int_{\mathscr{X}} K_x \otimes K_x \mathrm{~d} \rho_i, \forall\ i\in \mathcal{V}$. It can be verified that for $i\in \mathcal{V}$, $N_i \in \mathscr{L}\left(\mathscr{H}_K\right)$  is positive self-joint  and  $\sum_{i=1}^N N_i>0$.
%
% By the independence of $x_{i}(k)$	and similar to the proof of Theorem 2 in \cite{xiASCC}, we know that there exists a positive constant  $\tau_{K}$ related to the kernel $K$ such that
%\begin{align}
%& \left\|N_i-\sum_{l=2k}^{2k+1}E \left[K_{x_i(l)}\otimes K_{x_i(l)} |\mathcal{F}(2k-1)\right]  \right\|_{\LL(\HH_K)}\notag\\
%  =&  \left\|N_i-\sum_{l=2k}^{2k+1} E \left[K_{x_i(l)}\otimes K_{x_i(l)} \right]\right\|_{\LL( \HH_K)}\notag\\
%\leq & 2\tau_{K}   {\left \Vert \int_{\mathscr{X}} d\left( \rho_{i} - \frac{1}{2}\sum_{l=2k}^{2k+1}  \rho_{i,l }  \right) \right \Vert}
% \leq   6\tau_{K}\frac{1}{ k+1  }.\notag
%\end{align}
%
%Then, we have
%$  \big\|N_i-\sum_{l=2k}^{2k+1} E  [K_{x_i(l)}  \otimes K_{x_i(l)} |\mathcal{F}(2k-1) ]   \big\|_{\LL(\HH_K)}^{2} = O \big( (k+1)^{-2}\big).$
%	Similarly,  we have $ \big\|N_i -\sum_{l=2k+1}^{2k+2}E  [K_{x_i(l)}\otimes K_{x_i(l)} |\mathcal{F}(2k) ]   \big\|_{\LL(\HH_K)}^{2} = O \big( (k+1)^{-2}\big).$ Thus, we have $ \|N_i-\sum_{l=k}^{k+1}E[K_{x_i(l)}\otimes K_{x_i(l)} |\mathcal{F}(k-1)]  \|_{\LL(\HH_K)}^{2} =  O \big( (k+1)^{-2}\big),$
%which together with $\sum_{k=0}^{\infty} (k+1)^{-2}  <\infty$
% shows that  (\ref{vnkmeeeemefffff}) in Corollary \ref{vnlllleleeemmem} holds. Hence, all the conditions of  Corollary \ref{vnlllleleeemmem} hold.
%	
%Now, we will use the algorithm (\ref{rkhs})to estimate $f^*$.
We sample $1000$ points $\{z_{l},\ l=1,\ldots,1000\}$ with $z_{l}=-2+\frac{6(l-1)}{1000},\  l=1,\ldots,1000.$ We use the algorithm (\ref{rkhs})  to iterate the values of $f_{i}(k)$ at the sampled points, that is,
\begin{align}
&f_i(k+1)(z_{l})\notag\\=&f_i(k)(z_{l})+a(k)(y_i(k)-f_i(k)(x_i(k)))  K(x_i(k), z_{l} )
+b(k)\sum_{j\in \N_i}a_{ij}(f_j(k)(z_{l})-f_i(k)(z_{l})),\notag
\end{align}
where $\ l=1,\ldots,1000,\ k\ge 0,~i\in \mathcal V, \ f_{i}(0)=0,\ i\in \mathcal V$.
If $x_i(k)\notin \{z_{l},\ l=1,\ldots,1000\}$, the we get the approximation of $f_i(k)(x_i(k))$ by the cubic spline interpolation method.
%and  the values of $f_i(k)$ at sampling points $\{z_{l},\ l=1,\ldots,1000\}$.

Fig. 1(a) and Fig. 1(b)  show  the curves of all $f_i(k), \ i\in \mathcal V$, i.e.
the values of $f_i(k), \ i\in \mathcal V$  at all sampled points in $[-2,4]$ at $k=1000$  and $k=100000$ iterations, respectively. It can be seen that the  the estimation of all nodes: $f_i(k), \ i\in \mathcal V$ converge to the unknown true function $f^{*}$ pointwisely almost surely $k$ increases and the numerical result  is consistent with  Corollary \ref{vnlllleleeemmem}.

\begin{figure}[htbp]
    \centering
    \subfigure{\includegraphics[width=0.5\linewidth, height=6cm]{k1000.eps}}
      \hspace{-5mm}
  \subfigure{\includegraphics[width=0.5\linewidth, height=6cm]{k100000.eps}}
   \caption{ (a)  estimates  of nodes $f_{i},\ i=1,\cdots,10$ for $k=1000$; (b)   estimates  of nodes $f_{i},\ i=1,\cdots,10$   for $k=100000$.}
    \label{fig:mainfig}
\end{figure}






	
	



\section{Conclusions}
We have established a framework of random inverse problems with online measurements over graphs, and present a decentralized online learning algorithm with online data streams, which unifies the distributed parameter estimation in Hilbert spaces and the least mean square problem in reproducing kernel Hilbert spaces (RKHS-LMS). It is not required that the random forward operators satisfy special statistical assumptions such as mutual independence, spatio-temporal independence or stationarity. Each node updates its estimate at the next instant by using its new observation and a weighted sum of its own and neighbors' estimates. Firstly, by  exploiting the probabilistic properties of random elements with values in different topological spaces in a stochastic framework, we proposed the $L_p^q$-stability condition on the sequence of operator-valued random elements, and established the $L_2$-asymptotic stability theory of a class of   inhomogeneous random difference equations in Hilbert spaces with $L_2$-bounded martingale difference terms. Subsequently, we transform the asymptotic stability of these kinds of infinite-dimensional random difference equations into the $L_2^2$-stability condition on the operator-valued random elements. We have obtained an intuitive sufficient condition on the convergence of decentralized online learning algorithms for random inverse problems over graphs which is imposed on the random forward operators and the Laplacian matrix of the graph, i.e., the \emph{infinite-dimensional spatio-temporal persistence of excitation} condition.  We have proved that if the forward operators over connected graphs satisfy the \emph{infinite-dimensional spatio-temporal persistence of excitation} condition, then all nodes' estimates are mean square and almost surely strongly consistent. Finally, by equivalently transforming the distributed learning problem in RKHS to the random inverse problem over graphs, we propose a decentralized online learning algorithm in RKHS with non-stationary online data streams, and prove that the algorithm is mean square and almost surely strongly consistent if the operators induced by the random input data satisfy the \emph{infinite-dimensional spatio-temporal persistence of excitation} condition.



%In our measurement model (\ref{0}), the unknown function is assumed to be time-invariant, while in the  manufacturing industry (\cite{HU2017}), the estimated  manufacturing systems are often changing from time to time in different environment or with different input data. To track the model variations of the systems, it's necessary to estimate the time-varying unknown model in the future work. Besides, for the numerical simulations of the proposed algorithms, it is also worth studying the influence of difference interpolation methods on the simulation results.












\begin{appendices}


%\section{Proofs of Theorems in Sections 3-4}\label{appendixa1}
% \setcounter{equation}{0}
%\renewcommand{\theequation}{A.\arabic{equation}}
%\emph{Proof of Theorem \ref{wendingxing}.}
%
%$\hfill\blacksquare$
%
%\vskip 2mm
%\emph{Proof of Theorem \ref{vnknoklfl}.}
%
%$\hfill\blacksquare$
%\vskip 1mm

%\emph{Proof of Corollary \ref{xiaosirendetuilun}}
%%\textbf{\emph{Proof of Corollary \ref{xiaosirendetuilun}:}}
%It follows from $\|\H(0)\|\leq \rho_0~\text{a.s.}$ and Proposition \ref{nlllwwieiie}.(a) that $H^*_j(0) H_j(0)x\in L^1(\Omega;\X)$, $x\in \X$. For the integer $h>0$ and $j\in \mathcal V$, we define the operator $\HH_j:\X\to\X$ by
%\bna\label{wpkfpkpew}
%\HH_j(x)=h\E\left[H^*_j(0)H_j(0)x\right],~x\in \X,~j\in \mathcal V.
%\ena
%For any given $x_1,x_2\in\X$ and $c_1,c_2\in \mathbb R$, we have
%\begin{align}\label{jcknvknw}
%&\HH_j(c_1x_1+c_2x_2)\notag\\
%=&c_1h\E\left[H^*_j(0)H_j(0)x_1\right]+c_2h\E\left[H^*_j(0)H_j(0)x_2\right]\notag\\
% =&c_1\HH_j(x_1)+c_2\HH_j(x_2).
%\end{align}
%Noting that $H^*_j(0)H_j(0)x\in L^1(\Omega;\X)$, by Proposition 2.6.13 in \cite{hy}, we get
%\begin{align}\label{jcknvknw1}
%&\left\langle \HH_j(x_1), x_2\right\rangle\notag\\
%=&h\E\left[\left\langle H^*_j(0)H_j(0)x_1,x_2 \right\rangle \right]\notag\\
% =&h\E\left[\left\langle x_1,H^*_j(0)H_j(0)x_2 \right\rangle \right]
% =  \left\langle x_1, \HH_j(x_2)\right\rangle,~j\in \mathcal V.
%\end{align}
%From (\ref{jcknvknw})-(\ref{jcknvknw1}), it is known that $\HH_j$ is a linear self-adjoint operator, which gives
%\begin{align}
%\|\HH_j(x)\|\leq &h\E\left[\left\|H^*_j(0)H_j(0)\right\|\|x\|\right]\notag\\
%\leq & h\rho_0^2\|x\|,~\forall\ x\in \X,~j\in \mathcal V,
%\end{align}
%thus, the self-adjoint operator $\HH_j\in \mathscr L(\X)$. Denote $\HH_j(x):=\HH_jx$, $\forall\ x\in \X$. Noting that $H^*_j(0)H_j(0)x\in L^1(\Omega;\X)$, it follows from Proposition 2.6.13 in \cite{hy} that
%\begin{align}\label{owwwww2}
%&\left\langle \HH_jx,x\right\rangle\notag\\
%=&h\left\langle \E\left[H^*_j(0)H_j(0)x\right],x \right\rangle\notag\\
%=&h\E\left[\left\langle H^*_j(0)H_j(0)x,x \right\rangle \right]
%= h\E\left[\left\|H_j(0)x\right\|^2\right]\ge 0,
%\end{align}
%from which we know that the operator $\HH_j$ defined in (\ref{wpkfpkpew}) is positive bounded linear self-adjoint, $j\in \mathcal V$. Noting that $\{\H(k),k\ge 0\}$ is an i.i.d. sequence with values in the topology space $(\mathscr L(\X^N,\bigoplus_{i=1}^N\Y_i),\tau_{\text{S}}(\mathscr L(\X^N,\bigoplus_{i=1}^N\Y_i)))$, it follows from Definition \ref{dulixing} that $\{\H^*(k)\H(k)x,\ x$ $\in\X^N,\ k\ge 0\}$ is an i.i.d. sequence with values in the Banach space $(\X^N,\tau_{\text{N}}(\X^N))$. By Proposition E.1.10 in \cite{hy2}, we know that $\{\|\H^*(k)\H(k)\|,k\ge 0\}$ and $\{\|I_{\X^N}-(a(k)\H^*(k)\H(k)+b(k)\L_{\G}\otimes I_{\X})\|,k\ge 0\}$ are both independent random sequences. By Condition \ref{condition2}, there exists an integer $s_0>0$, such that $a(k)+b(k)\leq (4\rho^2_0+4\|\L_{\G}\otimes I_{\X}\|)^{-1}$, $\forall\ k\ge s_0$. We define the nonnegative real sequence $\{\Gamma(k),k\ge 0\}$ by
%\begin{align}\label{owwwww0}
%\Gamma(k)=\begin{cases}
%4(a(k)+b(k))\left(\rho^2_0+\|\L_{\G}\otimes I_{\X}\|\right),& 0\leq k<s_0;\\
%0, & k\ge s_0,
%\end{cases}
%\end{align}
%which shows that $\sum_{k=0}^{\infty}\Gamma(k)<\infty$. Noting that $\{\H(k),k\ge 0\}$ are i.i.d. and
% $P\{\|\H(0)\|\leq \rho_0\}=1$, it can be verified that $P\{\|\H(k)\|\leq \rho_0\}=1$, $k=0,1,...$, and
%$ \|I_{\X^N}-4(a(k)\H^*(k)\H(k)+b(k)\L_{\G}\otimes I_{\X})\|\leq 1+4(a(k)+b(k))(\rho^2_0+\|\L_{\G}\otimes I_{\X}\|)~\text{a.s.}$, $\forall\ k\ge 0$, we obtain
%\begin{align}\label{owwwww1}
%& \E\big[\|I_{\X^N}-4 (a(k)\H^*(k)\H(k)\notag\\
%&+b(k)\L_{\G}\otimes I_{\X} ) \| |\F(k-1)\big]\notag\\
%=&\E\big[\|I_{\X^N}-4 (a(k)\H^*(k)\H(k)\notag\\
%&+b(k)\L_{\G}\otimes I_{\X} ) \|\big]\notag\\
% =& \E\bigg[\sup_{\|x\|=1} | \langle I_{\X^N}-4 (a(k)\H^*(k)\H(k)\notag\\
% &+b(k)\L_{\G}\otimes I_{\X} )x,x \rangle |\bigg]\notag\\
% =& \E\bigg[\sup_{\|x\|=1} |1-4 \langle (a(k)\H^*(k)\H(k)\notag\\
% &+b(k)\L_{\G}\otimes I_{\X} t)x,x \rangle  |\bigg]\cr
% =& \E\bigg[1-4\inf_{\|x\|=1}\left\langle\left(a(k)\H^*(k)\H(k)+b(k)\L_{\G}\otimes I_{\X}\right)x,x\right\rangle \bigg]\cr
%%&&=\E\left[\left\|I_{\X^N}-\left(a(k)\H^*(0)\H(0)+b(k)\L_{\G}\otimes I_{\X}\right)\right\|^4\right]\cr
%  \leq & 1~\text{a.s.},~\forall \ k\ge s_0.
%\end{align}
%Then, by (\ref{owwwww0})-(\ref{owwwww1}), we get
%\begin{align}
%& \E\left[\left\|I_{\X^N}-4\left(a(k)\H^*(k)\H(k)+b(k)\L_{\G}\otimes I_{\X}\right)\right\| |\F(k-1)\right]\cr
% \leq & 1+\Gamma(k)~\text{a.s.},~\forall\ k\ge 0.\notag
%\end{align}
%It can be verified that
%\bna\label{vnkwenvefklw}
%&&~~~\sup_{k\ge 0}\left(\E\left.\left[\|\H^*(k)\H(k)\|^{2^{\max\{h,2\}}}\right|\F(k-1)\right]\right)^{\frac{1}{2^{\max\{h,2\}}}}\cr
%&&=\sup_{k\ge 0}\left(\E\left[\|\H^*(k)\H(k)\|^{2^{\max\{h,2\}}}\right]\right)^{\frac{1}{2^{\max\{h,2\}}}}\cr
%&&=\sup_{k\ge 0}\left(\E\left[\|\H^*(0)\H(0)\|^{2^{\max\{h,2\}}}\right]\right)^{\frac{1}
%{2^{\max\{h,2\}}}}\cr
%&&\leq \rho^2_0~\text{a.s.}
%\ena
%For any given integers $k\ge 0$, $h>0$ and $x\in \X$, if $A\in \F(kh-1)$, then it follows from Proposition \ref{nlllwwieiie} and (\ref{vnkwenvefklw}) that $H^*_j(i)H_j(i)(\1_A\otimes x)\in L^1(\Omega;\X)$, $i\ge kh$, which together with Lemma \ref{nvkvpeoeo} implies that $\E[H^*_j(i)H_j(i)(\1_A\otimes x)|\F(kh-1)]$ uniquely exists. Since $\{\H(k),k\ge 0\}$ and $\{v(k),k\ge 0\}$ are both i.i.d sequences and they are mutually independent, it follows from Definition \ref{dulixing} that $H^*_j(i)H_j(i)x$ is independent of $\1_{F\cap A}$ with $F\in \F(kh-1)$. Thus,
%\begin{align}
%&\int_F\E\left.\left[\sum_{i=kh}^{(k+1)h-1}H^*_j(i)H_j(i)(\1_A\otimes x)\right|\F(kh-1)\right]\dd\P\cr
% =&\int_F\sum_{i=kh}^{(k+1)h-1}H^*_j(i)H_j(i)(\1_A\otimes x)\dd\P\cr
% =&\int_{\Omega}\left(\sum_{i=kh}^{(k+1)h-1}H^*_j(i)H_j(i)x\right)\1_{F\cap A}\dd\P\cr
%=&\int_{\Omega}\sum_{i=kh}^{(k+1)h-1}H^*_j(i)H_j(i)x\dd\P\int_{\Omega}\1_{F\cap A}\dd\P\cr
%=&\E\left[\sum_{i=kh}^{(k+1)h-1}H^*_j(i)H_j(i)x\right]\P(F\cap A)\cr
%=&h\E\left[H^*_j(0)H_j(0)x\right]\P(F\cap A)\cr
%=&h\int_{F}\E\left[H^*_j(0)H_j(0)x\right]\1_A\dd\P,~\forall \ F\in \F(kh-1),~j\in \mathcal V,\notag
%\end{align}
%which gives
%\begin{align}
%&\E\left.\left[\sum_{i=kh}^{(k+1)h-1}H^*_j(i)H_j(i)(\1_A\otimes x)\right|\F(kh-1)\right]\notag\\
%=&\HH_j(\1_A\otimes x)~\text{a.s.},~j\in \mathcal V,\notag
%\end{align}
%which together with the properties of the conditional expectation, the operator $\HH_j$ and the linearity of Bochner integral leads to
%\begin{align}\label{vnkwoejvnvvn}
%&\HH_jy\notag\\
%=&\E\Bigg[\sum_{i=kh}^{(k+1)h-1}H^*_j(i)H_j(i)y\Bigg|\F(kh-1)\Bigg]~\text{a.s.},~j\in \mathcal V,
%\end{align}
%where $y\in L^0(\Omega,\F(kh-1);\X)$ is a simple function. For $f\in L^2(\Omega,\F(kh-1);\X)$, by Pettis measurability theorem, we know that there exists a sequence of simple functions $\{f_n \in L^0(\Omega,\F(kh-1);\X),n\ge 0\}$ satisfying $\|f_n\|\leq \|f\|\text{a.s.}$ and $\lim_{n\to \infty}f_n=f~\text{a.s.}$, which together with (\ref{vnkwenvefklw}) and Cauchy inequality gives $H^*_j(i)H_j(i)f\in L^1(\Omega;\X)$. Thus, from Lemma \ref{nvkvpeoeo}, it is known that $\E[H^*_j(i)H_j(i)f|\F(kh-1)]$ uniquely exists. Noting that $\HH_j\in \LL(\X)$, it follows from (\ref{vnkwenvefklw})-(\ref{vnkwoejvnvvn}) and the dominated convergence theorem of conditional expectation that, for any $j\in \mathcal V$,
%\begin{align}\label{vnowlelkel}
% &\HH_jf\notag\\
% =&\lim_{n\to\infty}\HH_jf_n\cr =&\lim_{n\to\infty}\E\left.\left[\sum_{i=kh}^{(k+1)h-1}H^*_j(i)H_j(i)f_n\right|\F(kh-1)\right]\cr
% =&\E\left.\left[\sum_{i=kh}^{(k+1)h-1}H^*_j(i)H_j(i)\lim_{n\to\infty}f_n\right|\F(kh-1)\right]\cr
% =&\E\left.\left[\sum_{i=kh}^{(k+1)h-1}H^*_j(i)H_j(i)f\right|\F(kh-1)\right]~\text{a.s.}
%\end{align}
%Let $\{x(k),\F(kh-1),k\ge 0\}$ be a $L_2$-bounded adaptive sequence with values in the Hilbert space $\X$, by (\ref{vnowlelkel}), we get,  for any $j\in \mathcal V$,
%\ban
%\HH_jx(k)=\E\left.\left[\sum_{i=kh}^{(k+1)h-1}H^*_j(i)H_j(i)x(k)\right|\F(kh-1)\right]
%~\text{a.s.}
%\ean
%For any non-zero element $x$ in Hilbert space $x$, from Proposition 2.6.13 in \cite{hy}, the condition (\ref{tuiluntiaojian}) and (\ref{wpkfpkpew}), we have
%\begin{align}\label{owwwww3}
%&\left\langle \sum_{j=1}^N\HH_jx,x\right\rangle\notag\\
%=&h\left\langle\E\left[\sum_{j=1}^NH^*_j(0)H_j(0)x\right],x\right\rangle\notag\\
%=&h\sum_{j=1}^N\E\left[\|H_j(0)x\|^2\right]>0.
%\end{align}
%Hence, by (\ref{owwwww2})-(\ref{vnkwenvefklw}), (\ref{owwwww3}) and Theorem \ref{vnknoklfl}, it is known that the algorithm (\ref{algorithm}) is both mean square and almost surely strongly consistent.
%$\hfill\blacksquare$
%\vskip 1mm
%
%\emph{Proof of Theorem \ref{rkhsdingli}.}
%
%$\hfill\blacksquare$
%\vskip 1mm

%\emph{Proof of Corollary \ref{vnlllleleeemmem}}
%%\textbf{\emph{Proof of Corollary \ref{vnlllleleeemmem}:}}
%It follows from Assumption \ref{assumption5} and Theorem \ref{rkhsdingli} that $K_{x_j(i)}\otimes K_{x_j(i)}\in L^1(\Omega;\mathscr L(\HH_K))$, which together with Lemma \ref{nvkvpeoeo} implies that $\E[K_{x_j(i)}\otimes $ $K_{x_j(i)}|\F(kh-1)]$ uniquely exists. Let $\{g(k),\F(kh-1),k\ge 0\}$ be a $L_2$-bounded adaptive sequence with values in $\HH_K$, by the condition (\ref{vnkmeeeemefffff}), we get
%\begin{align}
%&\sum_{j=1}^N\sum_{k=0}^{\infty}\E\Bigg[\bigg\|\bigg(\sum_{i=kh}^{(k+1)h-1}\E\left.\left[K_{x_j(i)}\otimes K_{x_j(i)}\right|\F(kh-1)\right]\notag\\
%&-N_j\bigg)g(k)\bigg\|_K^2\Bigg]\cr
% \leq & \sum_{j=1}^N\sum_{k=0}^{\infty}\E\Bigg[\bigg\|\sum_{i=kh}^{(k+1)h-1}
% \E\left.\left[K_{x_j(i)}\otimes K_{x_j(i)}\right|\F(kh-1)\right]\notag\\
% &-N_j\bigg\|_{\LL(\HH_K)}^2\|g(k)\|_K^2\Bigg]\cr
% \leq & \sum_{j=1}^N\sum_{k=0}^{\infty}\E\Bigg[\max_{j\in\mathcal V}\bigg\|\sum_{i=kh}^{(k+1)h-1}\E\left.\left[K_{x_j(i)}\otimes K_{x_j(i)}\right|\F(kh-1)\right]\notag\\
% &-N_j\bigg\|_{\LL(\HH_K)}^2\|g(k)\|_K^2\Bigg]\cr
% \leq & N\mu_0\sup_{k\ge 0}\E\left[\|g(k)\|_K^2\right]\sum_{k=0}^{\infty}\tau(k)
%  <\infty,
%\end{align}
%where the last inequality is obtained from $\sum_{k=0}^{\infty}\tau(k)<\infty$. This together with Theorem \ref{rkhsdingli} implies that the algorithm (\ref{rkhs}) is both mean square and almost surely strongly consistent. This together with the continuity of the inner product $\langle\cdot, \cdot\rangle_K$ show that $\lim\limits_{k\to \infty} f_{i}(k)(x)=f_{0}(x) $ and $\lim\limits_{k\to \infty}\E\left[\left\| f_{i}(k)(x)-f_{0}(x)\right\|^{2}\right]=0, \ \text{a.s.}, \ \forall \ x \in \mathscr{X}, \ i \in \mathcal{V}$.
%
%
%$\hfill\blacksquare$
%
%\emph{Proof of Corollary \ref{rkhsdinglijjjjj}}
%%\textbf{\emph{Proof of Corollary \ref{rkhsdinglijjjjj}:}}
%Noting that $\{(x_1(k),\cdots,x_N(k)),\\ k\ge 0\}$  and $\{(v_1(k),\cdots ,v_N(k)),k\ge 0\}$ are both i.i.d. sequences and they are mutually independent, we have
%\begin{align*}
%&\sum_{i=kh}^{(k+1)h-1}\E\left.\left[K_{x_j(i)}\otimes K_{x_j(i)}\right|\F(kh-1)\right]\cr = &\sum_{i=kh}^{(k+1)h-1}\E\left[K_{x_j(i)}\otimes K_{x_j(i)}\right]
% =   h\E\left[K_{x_j(0)}\otimes K_{x_j(0)}\right].
%\end{align*}
%Denote $N_j:\HH_K\to \HH_K$ by
%$
%N_j=h\E [K_{x_j(0)}\otimes K_{x_j(0)} ],~j\in \mathcal V.
%$
%It follows from  Proposition 2.6.13 in \cite{hy} that $N_j\in \mathscr L(\HH_K)$ is a positive self-adjoint operator. Noting that
%\begin{align}
%&\bigg\|N_j-\sum_{i=kh}^{(k+1)h-1}\E\left.\left[K_{x_j(i)}\otimes K_{x_j(i)}\right|\F(kh-1)\right]\bigg\|_{\LL(\HH_K)}^2\notag\\
%=&0~\text{a.s.},
%\end{align}
%by the condition (\ref{nklnkle}), we get
%\begin{align}
%&\left\langle \sum_{j=1}^NN_jf,f\right\rangle _K\notag\\
% =& h\left\langle \E\left[\sum_{j=1}^NK_{x_j(0)}\otimes K_{x_j(0)}\right]f,f\right\rangle _K >0,
%\end{align}
%where $f$ is an arbitrary non-zero function in $\HH_K$. Hence, it follows from Corollary \ref{vnlllleleeemmem} that the algorithm (\ref{rkhs}) is both mean square and almost surely strongly consistent.
%$\hfill\blacksquare$




\section{Theoretical Framework of Random Elements with Values in a Topological Space}
\label{appendixb}
 \setcounter{equation}{0}
\renewcommand{\theequation}{A.\arabic{equation}}
The proposed algorithm involves the sequences of random forward operators induced by random input data. Conventionally, a random element with values in a Banach space is required to be strongly measurable, which is almost separably valued with respect to the topology induced by the norm in the Banach space (\cite{hy2}). It is known that operator-valued mappings may not be strongly measurable since the Banach space of bounded linear operators is generally non-separable with respect to the uniform operator topology (the topology induced by the operator norm) (\cite{hy}). In this section, we develop a self-contained  theoretical framework of random elements with values in a topological space.

\subsection{Random Elements with Values in a Topological Space}\label{sectionrandonelement}
%\subsection{Preliminary analytical theory of Banach spaces}\label{appendixa}
%\setcounter{lemma}{0}
%\def\thelemma{A.\arabic{lemma}}
%\setcounter{definition}{0}
%\def\thedefinition{A.\arabic{definition}}
%\setcounter{equation}{0}
%\def\theequation{A.\arabic{equation}}

\begin{definition}\label{jihukefenzhi}
Let $(\Omega,\F,\P)$ be a complete probability space, and $(\mathscr U,\tau)$ a topological space. Given the mapping $f:\Omega\to \mathscr U$, if there exists a separable closed subset $\mathscr U_0$ of $\mathscr U$ and a subset $\Omega_0$ of $\Omega$ with probability measure $1$, such that $$f(\Omega_0):=\{f(x):x\in \Omega_ 0\}\subseteq \mathscr U_0,$$ then $f$ is called  almost separably valued with respect to $\tau$.
\end{definition}

%\begin{definition}[\cite{hy}]
%Let $(S,\mathscr A)$ and $(T,\mathscr B)$ be measurable spaces, if the mapping $f:S\to T$ satisfies $$f^{-1}(B):=\{x\in S:f(x)\in B\}\in \mathscr A,~\forall\ B\in \mathscr B,$$ then $f$ is said to be $\mathscr A/\mathscr B$-measurable.
%\end{definition}

\begin{definition}\label{tuopukongjian}
Let $(\Omega,\F,\P)$ be a complete probability space and $(\mathscr U,\B(\mathscr U;\tau))$ be a measurable space, where $\tau$ is the topology on $\mathscr U$, and $\B(\mathscr U;\tau)$ is the Borel $\sigma$-algebra of the topological space $(\mathscr U,\tau)$, i.e., the smallest $\sigma$-algebra containing all open sets in $\mathscr U$. A mapping $f:\Omega\to \mathscr U$ is said to be a random element with values in the topological space $(\mathscr U,\tau)$ if it is $\F/\B(\mathscr U;\tau)$-measurable and almost separably valued with respect to $\tau$.
\end{definition}

\begin{definition}\label{fenbudingyi}
If $f$ is a random element with values in the topological space $(\mathscr U,\tau)$, then the distribution of $f$ is defined by the Borel probability measure $\mu_{f}(B):=\P(f^{-1}(B))$ on $(\mathscr U,\tau )$, $\forall\ B\in \B(\mathscr U;\tau)$.
\end{definition}



%\begin{definition}[\cite{hy}]
%The function $f=\sum_{i=1}^n\boldsymbol{1}_{A_i}\otimes x_i$ is called a simple function with values in the Banach space $\X$, where $A_i\in \F$, $x_i\in \X$, $i=1,\cdots,n$, $\boldsymbol{1}_{A}$ is the indicator function of the set $A$ and $(\1_A\otimes x)(s):=\1_A(s)x$, $\forall\ x\in \X$, $s\in \Omega$.
%\end{definition}

\begin{definition}[\cite{hy}]\label{vnwkelel}
Given a mapping $f:\Omega \to \X$ with values in Banach space $(\X,\tau_{\text{N}}(\X))$, if there exists a sequence of simple functions $\{f_n,n\ge 1 \}$ almost everywhere converging to $f$ in the topology $\tau_{\text{N}}(\X)$, then $f$ is said to be strongly measurable with respect to $\tau_{\text{N}}(\X)$.
\end{definition}


\begin{lemma}[\cite{hp}]\label{mse}
The mapping $f:\Omega\to \X$ is strongly measurable with respect to the topology $\tau_{\text{N}}(\X)$ if and only if $f$ is a random element with values in the Banach space $(\X,\tau_{\text{N}}(\X))$.
\end{lemma}

It follows from Lemma \ref{mse} that the mapping $f:\Omega\to \X$ is strongly measurable with respect to $\tau_{\text{N}}(\X)$ if and only if $f$ is a random element with values in the Banach space $(\X,\tau_{\text{N}}(\X))$. Especially, if $\X$ is separable with respect to $\tau_{\text{N}}(\X)$, then any $\F/\B(\X;\tau_{\text{N}}(\X))$-measurable mapping $f:\Omega\to \X$ is a random element with values in $(\X,\tau_{\text{N}}(\X))$.


%\vskip 0.2cm
Combining the Definition \ref{tuopukongjian} and Corollary 1.4.7, Proposition 1.1.28 and Corollary 1.1.29 in \cite{hy}, we directly obtain the following properties of operator-valued random elements.

\begin{proposition}\label{nlllwwieiie}
%\begin{longlist}
Let $f_0:\Omega\to \mathscr L(\mathscr Y,\mathscr Z)$, $f_1:\Omega\to \X$, $f_2:\Omega\to \LL(\X,\Y)$, $g_1:\Omega\to \LL(\X,\Y)$ and $g_2:\Omega\to \LL(\Y,\mathscr Z)$.\\
\indent  (a). The mapping $f_0$ is a random element with values in $(\mathscr L(\mathscr Y,\mathscr Z),\tau_{\text{S}}(\mathscr L(\mathscr Y,\mathscr Z)))$ if and only if the mapping $f_0y:\omega\mapsto f_0(\omega)y$ is a random element with values in $(\mathscr Z,\tau_{\text{N}}(\Z))$ for arbitrary $y\in \Y$.\\
\indent  (b). If $f_1$ is strongly measurable with respect to $\tau_{\text{N}}(\X)$ and $g_1x:\Omega\to \mathscr Y$ is strongly measurable with respect to $\tau_{\text{N}}(\Y)$, $\forall\ x\in \X$, then $g_1f_1:\Omega\to \mathscr Y$ is strongly measurable with respect to $\tau_{\text{N}}(\Y)$.\\
\indent  (c). For arbitrary $x\in\X$ and $y\in \Y$, if the mapping $f_2x:\Omega\to \Y$ is strongly measurable with respect to $\tau_{\text{N}}(\Y)$ and the mapping $g_2y:\Omega\to \mathscr Z$ is strongly measurable with respect to $\tau_{\text{N}}(\mathscr Z)$, then the mapping $(g_2f_2)x':\Omega\to \mathscr Z$ is strongly measurable with respect to $\tau_{\text{N}}(\mathscr Z)$ for arbitrary $x'\in \X$.
%\end{longlist}
\end{proposition}

If the operator-valued mapping $f:\Omega\to \LL(\mathscr Y,\mathscr Z)$ is strongly measurable with respect to the uniform operator topology $\tau_{\text{N}}(\mathscr L(\mathscr Y,\mathscr Z))$, then for arbitrary $y\in \mathscr Y$, $fy:\Omega\to \mathscr Z$ is strongly measurable with respect to $\tau_{\text{N}}(\mathscr Z)$. By Proposition \ref{nlllwwieiie}. (a) and Pettis measurability theorem (\cite{Blasco111}), it follows that a random element with values in $(\mathscr L(\mathscr Y,\mathscr Z),\tau_{\text{N}}(\mathscr L(\mathscr Y,\mathscr Z)))$ is a random element with values in
$(\mathscr L(\mathscr Y,\mathscr Z),\tau_{\text{S}}(\mathscr L(\mathscr Y,\mathscr Z)))$. For the measurability of mappings with values in the space of operators with different topologies, one may refer to \cite{hy} and  \cite{hp}-\cite{Blasco111}.









%


%Throughout this paper, $(\Omega,\F,\P)$ is assumed to be a complete probability space.
 It follows from Lemma \ref{mse} that the mapping $f:\Omega\to \X$ is strongly measurable with respect to $\tau_{\text{N}}(\X)$ if and only if $f$ is a random element with values in the Banach space $(\X,\tau_{\text{N}}(\X))$. Especially, if $\X$ is separable with respect to $\tau_{\text{N}}(\X)$, then any $\F/\B(\X;\tau_{\text{N}}(\X))$-measurable mapping $f:\Omega\to \X$ is a random element with values in $(\X,\tau_{\text{N}}(\X))$.


%It follows from  \cite{hy} that $O\in \tau_{\text{S}}(\mathscr L(\mathscr Y,\mathscr Z))$ if and only if there exists an integer $k>0$, $x_1,\cdots,x_k\in \Y$ and a constant $\varepsilon>0$ such that
%\[\bigcap_{j=1}^k\left\{T\in \mathscr L(\mathscr Y,\mathscr Z): \left\|Sx_j-Tx_j\right\|_{\mathscr Z}<\varepsilon \right\}\subseteq O,~\forall\ S\in O.\]



%\begin{remark}

%\end{remark}

For a random element $f:\Omega\to \X$ with values in $(\X,\tau(\X))$, we denote the $\sigma$-algebra generated by $f$ in the sense of the topology $\tau(\X)$ by
$$\sigma(f;\tau(\X)):=\left\{f^{-1}(B):B\in \B(\X;\tau(\X))\right\}.$$
Based on the above definitions, we have the following proposition.

\begin{proposition}\label{wenknknkn}
If $f:\Omega\to \X$ is a random element with values in the Banach space $(\X,\tau_{\text{N}}(\X))$, and $T:\Omega\to \mathscr L(\mathscr X,\mathscr Y)$ is a random element with values in the topological space $(\mathscr L(\mathscr X,\mathscr Y), \tau_{\text{S}}(\mathscr L(\mathscr X,\mathscr Y)))$, then $Tf:\omega\mapsto T(\omega)f(\omega)$ satisfies
\[Tf \in \left(\bigvee_{x\in\X}\sigma(Tx;\tau_{\text{N}}(\Y))\right)\bigvee \sigma(f;\tau_{\text{N}}(\X)). \]
\end{proposition}
\begin{proof}
This proposition is proved by the \textbf{two steps} as follows.

%\emph{Step 1:} If $f=\1_{A}\otimes y$ with $A\in \F$ and $y\in \X$. Noting that $\sigma(\1_{A}\otimes y;\tau_{\text{N}}(\X))=\sigma(\1_{A};\tau_{\text{N}}(\X))=\{\emptyset,\Omega,A,A^{\complement}\}$, we get
%\ban
%(T(\1_{A}\otimes y))^{-1}(B)=
%\begin{cases}
%(Ty)^{-1}(B)\cap A\cup A^{\complement}, & 0\in B;\\
%(Ty)^{-1}(B)\cap A, & 0\notin B,
%\end{cases}
%\quad
%\forall B\in \B(\Y;\tau_{\text{N}}(\Y)),
%\ean
%which shows that $T(\1_{A}\otimes y)\in (\bigvee_{x\in \X}\sigma(Tx;\tau_{\text{N}}(\Y)))\bigvee \sigma(f;\tau_{\text{N}}(\X))$.
%
\emph{Step 1:} If $f$ is a simple function with values in the Banach space $\X$,  without loss of generality, let $f=\sum_{i=1}^n\1_{A_i}\otimes y_i$, where $A_i\in \F$, $y_i\in \X$, $A_i\cap A_j=\emptyset$, $1\leq i\neq j \leq n$. Note that $\sigma(\1_{A_i}\otimes y_i;\tau_{\text{N}}(\X))=\sigma(\1_{A_i};\tau_{\text{N}}(\X))=\{\emptyset,\Omega,A_i,A_i^{\complement}\}$. For any given $B\in \B(\Y;\tau_{\text{N}}(\Y))$, we have
\ban
&&~~~\left(T\left(\sum_{i=1}^n\1_{A_i}\otimes y_i\right)\right)^{-1}(B)\cr
&&=
\begin{cases}
\displaystyle \left(\bigcup_{i=1}^n(Ty_i)^{-1}(B)\right)\bigcap \left(\bigcup_{i=1}^n A_i\right)\bigcup \left(\bigcup_{i=1}^nA_i\right)^{\complement}, & 0\in B;\\
\displaystyle \left(\bigcup_{i=1}^n(Ty_i)^{-1}(B)\right)\bigcap \left(\bigcup_{i=1}^n A_i\right), & 0\notin B,
\end{cases}
\ean
which gives $T(\sum_{i=1}^n\1_{A_i}\otimes y_i)\in (\bigvee_{x\in \X}\sigma(Tx;\tau_{\text{N}}(\Y)))\bigvee \sigma(f;\tau_{\text{N}}(\X))$.

\emph{Step 2:} For the random element $f$ with values in the Banach space $(\X,\tau_{\text{N}}(\X))$, by Lemma \ref{mse} and Definition \ref{vnwkelel}, we know that there exists a sequence of simple functions $\{f_n,n\ge 1\}$ with values in $\X$ such that $\lim_{n\to\infty}\|f_n-f\|_{\X}=0~\text{a.s.}$ It follows from Proposition \ref{nlllwwieiie}.(b) that $Tf:\Omega\to \mathscr Y$ is strongly measurable with respect to the topology $\tau_{\text{N}}(\Y)$, i.e., there exists $\Omega_0\subseteq \Omega$, $\P(\Omega_0)=1$ and a separable closed subspace $ \Y_0\subseteq \mathscr Y$ such that $Tf(\Omega_0)\subseteq \mathscr Y_0$. Let $\mathscr Y^*$ be the dual of the Banach space $\mathscr Y$, and by the Hahn-Banach extension theorem we obtain
\ban
\|y\|=\sup_{\|y^*\|_{\mathscr Y^*}\leq 1}|y^*(y)|,~\forall\ y\in \mathscr Y.
\ean
Thus, for the closed ball $B_{\Y_0}(y_0,r):=\{y\in\mathscr Y_0:\|y-y_0\|\leq r\}$ in the Banach space $\mathscr Y_0$, it follows from Proposition B.1.10 in \cite{hy} that there exists a unit sequence $\{y^* _n,n\ge 1\}$ such that
\bna\label{nfwkjngjl}
&&~~~~(Tf)^{-1}(B(y_0,r))\cr
&&=\{\omega\in \Omega:\|T(\omega)f(\omega)-y_0\|\leq r\}\cr
&&=\bigcap_{n=1}^{\infty}\{\omega\in \Omega:|y^*_n(T(\omega)f(\omega)-y_0)|\leq r\}\cr
&&=\bigcap_{n=1}^{\infty}\left\{\omega\in \Omega:\lim_{m\to\infty}|y^*_n(T(\omega)f_m(\omega)-y_0)|\leq r\right\}.
\ena
From the analysis in \emph{Step 1}, we know that $Tf_n\in (\bigvee_{x\in \X}\sigma(Tx;\tau_{\text{N}}(\Y)))\bigvee \sigma(f;\tau_{\text{N}}(\X))$. Noting that $y^*_n$ is a linear measurable functional mapping from the Banach space $\mathscr Y$ to $\mathbb R$, we have
\bna\label{nvnejrfgiowjj}
&&~~~~\left\{\omega\in \Omega:\lim_{m\to\infty}|y^*_n(T(\omega)f_m(\omega)-y_0)|\leq r\right\}\cr
&&\in \left(\bigvee_{x\in \X}\sigma(Tx;\tau_{\text{N}}(\Y))\right)\bigvee \sigma(f;\tau_{\text{N}}(\X)).
\ena
Combining (\ref{nfwkjngjl})-(\ref{nvnejrfgiowjj}) yields $(Tf)^{-1}(B(y_0,r))\in (\bigvee_{x\in \X}\sigma(Tx;\tau_{\text{N}}(\Y)))\bigvee \sigma(f; \tau_{\text{N}}(\X))$, and since the open set of the Banach space $(\mathscr Y_0,$ $\tau_{\text{N}}(\Y_0))$ is generated by a countable number of closed spheres, we have $$Tf|_{\mathscr Y_0}\in\left(\bigvee_{x \in \X}\sigma(Tx;\tau_{\text{N}}(\Y))\right)\bigvee \sigma(f;\tau_{\text{N}}(\X)).$$ Given $B\in \B(\Y;\tau_{\text{N}}(\Y))$, noting that $B_0=B\cap \mathscr Y_0$, it is known that $B_0\in \B(\Y_0;\tau_{\text{N}}(\Y_0))$ and
\ban
(Tf)^{-1}(B)=\{\omega\in \Omega:T(\omega)f(\omega)\in B\}=\{\omega\in \Omega:T(\omega)f(\omega)\in B_0\}=(Tf)^{-1}(B_0),
\ean
which leads to
$Tf\in \left(\bigvee_{x\in \X}\sigma(Tx;\tau_{\text{N}}(\Y))\right)\bigvee \sigma(f;\tau_{\text{N}}(\X))$.
\end{proof}


%\textbf{\emph{Proof of Proposition \ref{vnwlssfweewwfew}:}}



\subsection{Conditional Expectation}\label{sectionconditionexpectation}

%As for mathematical expectations and conditional expectations of the random elements with values in the Banach space $(\X,\tau_{\text{N}}(\X))$, we introduce the following notations.
\begin{definition}[\cite{hy2}]\label{vnsllp}
\rm{If $f\in L^1(\Omega;\X)$, then the mathematical expectation of $f$ is defined as the Bochner integral $$ \E[f]=(\text{B})\int_{\Omega}f\dd\P.$$
}
\end{definition}

%\begin{remark}\label{fnklwmemmemee}
Without raising ambiguity, the Bochner integral in this paper will omit the capital letter (B) in front of the integral symbol.
%\end{remark}
Let $L^0(\Omega;\X)$ be a linear space composed of all mappings which take values in $(\X,\tau_{\text{N}}(\X))$ and are strongly measurable with respect to $\tau_{\text{N}}(\X)$.
Let $L^0(\Omega,\Gg;\X)$ be the linear space composed of all $\Gg/\B(\X;\tau_{\text{N}}(\X))$-measurable functions in $L^0(\Omega;\X)$, where $\Gg$ is a sub-$\sigma$-algebra of $\F$ and $\B(\X;\tau_{\text{N}}(\X))$ is the Borel $\sigma$-algebra of the topological space $(\X,\tau_{\text{N}}(\X))$.

\begin{definition}[\cite{hy}]
\rm{Let $\Gg\subseteq \F$ be a sub-$\sigma$-algebra, $f\in L^0(\Omega;\X)$ and $g\in L^0(\Omega,\Gg;\X)$. If
\ban
\int_Fg\dd\P=\int_Ff\dd\P,~\forall\ F\in \Gg_f\cap \Gg_g,
\ean
where $\Gg_{\phi}:=\{F\in \Gg:\1_F\phi\in L^1(\Omega;\X)\}$, $\phi\in L^0(\Omega;\X)$, then $g$ is called the conditional expectation of $f$ with respect to $\Gg$.
}
\end{definition}



\begin{definition}[\cite{hy}]
Let $\{\F_k, k\geq0\}$ be a family of sub-$\sigma$-algebras of $\F$ and $\{f_k,k\geq0\}$ be a family of random elements with values in the Hilbert space $(\X,\tau_{\text{N}}(\X))$.
\begin{itemize}
\item[1.] If $\F_n\subseteq \F_m$, $\forall\ m\geq n\geq 0$, then $\{\F_k,k\geq0\}$ is called a filter in $(\Omega,\F,\P)$.
\item[2.] If $\{\F_k, k\geq0\}$ is a filter in $(\Omega,\F,\P)$, $f_k\in L^0(\Omega,\F_k;\X)$, $\forall\ k\geq0$, then $\{f_k,\F_k,k\geq0\}$ is called an adaptive sequence.
\item[3.] If $\{f_k,\F_k, k\geq0\}$ is an adaptive sequence, $f_k$ is Bochner integrable on $\F_{k-1}$ and
\[\E[f_k|\F_{k-1}]=0,\ \forall\ k\geq0,\]
then $\{f_k,\F_k,k\geq0\}$ is called a sequence of martingale differences.
\end{itemize}
\end{definition}


\begin{lemma}[\cite{hy}]\label{nvkvpeoeo}
If $\Gg\subseteq \F$ is a sub-$\sigma$-algebra and $f\in L^1(\Omega;\X)$, then there exists a unique conditional expectation $\E[f|\Gg]\in L^0(\Omega,\Gg;\X)\cap L^1(\Omega;\X)$ satisfying
\[\int_A\E[f|\Gg]\dd\P=\int_Af\dd\P,~\forall\ A\in \Gg.\]
\end{lemma}

%\begin{lemma}[\cite{hy}]\label{lemmaA1}
%Let $\X_1,\X_2$ and $\Y$ be Banach spaces, $\beta:\X_1\times \X_2 \to \Y$ be a bounded bilinear mapping and $\Gg$ be a sub-$\sigma$-algebra of $\mathcal F$. If $g\in L^0(\Omega,\Gg;\X_1)$ and $f\in L^0(\Omega;\X_2)$ is Bochner integrable on $\Gg$, then $\beta(g,f)\in L^0(\Omega;\Y)$ is Bochner integrable on $\Gg$ and
%$\mathbb E[\beta(g,f)|\Gg]=\beta(g,\mathbb E[f|\Gg])$ a.s.
%\end{lemma}



For a Banach space $(\X,\tau_{\text{N}}(\X))$,
by Lemma \ref{nvkvpeoeo}, it follows that there exists a unique conditional expectation $\E[f|\Gg]\in L^0(\Omega,\Gg; \X)$ of the Bochner integrable random element $f$ with values in $(\X,\tau_{\text{N}}(\X))$, and $\E[f|\Gg]$ is a random element with values in $(\X,\tau_{\text{N}}(\X))$.
%The reader can further refer to \cite{hy} for the property of conditional expectation.
We have the following propositions of conditional expectations.

\begin{proposition}\label{vnwlssfweewwfew}
If $f\in L^1(\Omega;\mathscr L(\mathscr Y,\mathscr Z))$ is a random element with values in Banach space $(\mathscr L(\mathscr Y,\mathscr Z),\tau_{\text{N}}(\mathscr L(\mathscr Y,\mathscr Z))$, then $fy\in L^1(\Omega;\mathscr Z)$ is the random element with values in Banach space $(\mathscr Z,\tau_{\text{N}}(\mathscr Z))$, and $\E[fy]=\E[f]y$, $\forall\ y\in \mathscr Y$.
\end{proposition}
\begin{proof}
Since the random elements with values in the Banach space $(\LL(\Y,\Z)$, $\tau_{\text{N}}(\LL(\Y,\Z)))$ are certainly the random elements with values in $(\LL(\Y,\Z)$, $\tau_{\text{S}}(\LL(\Y,\Z)))$, it follows that $f\in L^1(\Omega;\LL(\Y,\Z))$ implies $fy\in L^1(\Omega;\Z)$, $\forall\ y\in \Y$ by Proposition (\ref{nlllwwieiie}).(a). Considering the simple function $\sum_{i=1}^n\1_{A_i}\otimes T_i$ with values in Banach space $\LL(\Y,\Z)$, where $A_i\in \F$, $T_i\in \LL(\Y,\Z)$, $A_i\cap A_j=\emptyset$, $1\leq i\neq j \leq n$, we have
\ban
\E\left[\left(\1_{A_i}\otimes T_i\right)y\right]=\P(A)(T_iy)=(\P(A_i)T_i)y=\E\left[\1_{A_i}\otimes T_i\right]y,~1\leq i\leq n,
\ean
which leads to
\bna\label{vnkwkekeneknek}
\E\left[\left(\sum_{i=1}^n\1_{A_i}\otimes T_i\right)y\right]&=&\sum_{i=1}^n\E\left[(\1_{A_i}\otimes T_i)y\right]\cr
&=&\sum_{i=1}^n\E\left[\ 1_{A_i}\otimes T_i\right]y\cr
&=&\E\left[\sum_{i=1}^n\1_{A_i}\otimes T_i\right]y.
\ena
Since $f\in L^1(\Omega;\LL(\Y,\Z))$ is the random element with values in the Banach space $(\LL(\Y,\Z),\\\tau_{\text{N}}(\LL(\Y,\Z)))$, it follows from Lemma \ref{mse} and Definition \ref{vnwkelel} that there exists a sequence of simple functions $\{T_n,n \ge 1\}$ such that $\lim_{n\to \infty}\|f-T_n\|=0~\text{a.s.}$ and $\|T_n\|\leq \|f\|~\text{a.s.}$ Noting that $\|f-T_n\|\leq 2\|f\|\in L^1(\Omega)$ and $\lim_{n\to\infty}\|fy-T_ny\|=0~\text{a.s.}$, $\forall\ y\in \Y$, by the dominated convergence theorem, it follows that $\lim_{n\to\infty}\E[(f-T_n)y]=0$, $\forall\ y\in \Y$, and $\lim_{n\to\infty}\E[T_n]=\E[f]$. Thus, it follows from (\ref{vnkwkekeneknek}) that
\ban
\E[fy]=\lim_{n\to\infty}\E[(f-T_n)y]+\lim_{n\to\infty}\E[T_ny]=\lim_{n\to\infty}\E[T_n]y=\E[f]y,~\forall\ y\in \Y.
\ean
\end{proof}
\begin{proposition}\label{tiaojianqiwangxingzhi}
If $f\in L^2(\Omega;\mathscr L(\mathscr Y,\mathscr Z))$ is a random element with values in the Banach space $(\mathscr L(\mathscr Y,\mathscr Z),\tau_{\text{N}}(\mathscr L(\mathscr Y,\mathscr Z)))$, and $y\in L^2(\Omega,\Gg;\Y)$ is a random element with values in the Banach space $(\Y,\tau_{\text{N}}(\Y))$, where $\Gg$ is a sub-$\sigma$-algebra of $\F$, then $fy\in L ^1(\Omega;\mathscr Z)$ is a random element with values in the Banach space $(\mathscr Z,\tau_{\text{N}}(\mathscr Z))$ and $\E[fy|\Gg]=\E[f|\Gg]y~\text{a.s.}$
\end{proposition}
\begin{proof}
Since the random elements with values in the Banach space $(\LL(\Y,\Z)$, $\tau_{\text{N}}(\LL(\Y,\Z)))$ are  the random elements with values in the topological space $(\LL(\Y,\Z)$, $\tau_{\text{S}}(\LL(\Y,\Z)))$, it follows from Proposition \ref{nlllwwieiie}.(b) that $fy$ is the random element with values in the Banach space $(\Z,\tau_{\text{N}}(\Z))$, $f\in L^2(\Omega;\LL(\Y,\Z))$, and $y\in L^2(\Omega;\Y)$ implies $fy\in L^1(\Omega ;\Z)$. Consider the simple function $\sum_{i=1}^n\1_{A_i}\otimes y_i$ with values in the Banach space $\Y$, where $A_i\in \Gg$, $y_i\in \Y$, $A_i\cap A_j=\emptyset$, $1\leq i\neq j\leq n$. For any given $F\in \Gg$ and $1\leq i\leq n$, by Lemma \ref{nvkvpeoeo}, we have
\bna\label{fcwlmlmcc}
\int_F\E[f(\1_{A_i}\otimes y_i)|\Gg]\dd\P=\int_Ff(\1_{A_i}\otimes y_i)\dd\P=\int_{F\cap A_i}fy_i\dd\P=\E[(\1_{F\cap A_i}f)y_i].
\ena
Noting that $F\cap A_i\in \Gg$, by Lemma \ref{nvkvpeoeo} and Proposition \ref{vnwlssfweewwfew}, we get
\ban
\E[(\1_{F\cap A_i}f)y_i]=\E[\1_{F\cap A_i}f]y_i=\left(\int_{F\cap A_i}f\dd\P \right)y_i=\left(\int_{F\cap A_i}\E[f|\Gg]\dd\P \right) y_i.
\ean
Noting that $\1_{F\cap A_i}\E[f|\Gg]\in L^2(\Omega,\Gg;\LL(\Y,\Z))$, it follows from Lemma \ref{nvkvpeoeo} and Proposition \ref{vnwlssfweewwfew} that
\bna\label{vmkweknfffff}
\left(\int_{F\cap A_i}\E[f|\Gg]\dd\P\right) y_i=\E\left[\1_{F\cap A_i}\E[f|\Gg]y_i\right]=\int_F\E[f|\Gg](\1_{A_i}\otimes y_i)\dd\P.
\ena
Noting that $\E[f|\Gg](\1_{A_i}\otimes y_i)\in L^0(\Omega,\Gg;\Z)$, it follows from (\ref{fcwlmlmcc})-(\ref{vmkweknfffff}) and Lemma \ref{nvkvpeoeo} that
\bna\label{vcmklwleklekle}
\E\left.\left[f\left(\sum_{i=1}^n\1_{A_i}\otimes y_i\right)\right|\Gg\right]=\E[f|\Gg]\left(\sum_{i=1}^n\1_{A_i}\otimes y_i\right)~\text{a.s.}
\ena
Given the random element $y\in L^2(\Omega,\Gg;\Y)$ with values in the Banach space $(\Y,\tau_{\text{N}}(\Y))$, there exists a sequence of simple functions $\{y_n\in L^0(\Omega,\Gg;\Y),n\ge 1\}$ with values in the Banach space $\Y$ such that $\lim_{n\to\infty}\|y-y_n\|=0~\text{a.s.}$ and $\|y_n\|\leq \|y\|~\text{a.s.}$ Noting that $\|fy_n\|\leq \|f\|^2+\|y\|^2\in L^1(\Omega)$ and $\E[f|\G]\in L^2(\Omega,\Gg;\LL(\Y,\Z))$, by the dominated convergence theorem of conditional expectation and (\ref{vcmklwleklekle}), we have
\ban
\E[fy|\Gg]=\E\left.\left[f\left(\lim_{n\to\infty}y_n\right)\right|\Gg\right]=\lim_{n\to\infty}\E\left.\left[f y_n\right|\Gg\right]= \E\left.\left[f\right|\Gg\right]\left(\lim_{n\to\infty}y_n\right)=\E\left.\left[f\right|\Gg\right]y~\text{a.s.}
\ean
\end{proof}
\subsection{Independence and Conditional Independence}\label{sectionconinde}

In terms of the independence and conditional independence of random elements with values in a topological space, we have the following definitions.

\begin{definition}\label{dulixing}
Let $\mathscr I$ be a set of indices, $f_j$, $j\in \mathscr I$ be the random elements with values in $(\X_j,\tau(\X_j))$. If
$\P\left(f_{\a_1}\in B_1,\cdots,f_{\a_n}\in B_n\right)=\prod_{j=1}^n\P\left(f_{\a_j}\in B_j\right)$, for arbitrarily different indices $\a_1,\cdots,\a_n\in \mathscr I$ and arbitrary $B_1\in \B(\X_1;\tau(\X_{\a_1})),\cdots,B_n\in \B(\X_n;\tau(\X_{\a_n}))$,
then $f_j$, $j\in \mathscr I$ are said to be mutually independent.
\end{definition}

\begin{definition} \label{tiaojiandulixing}
%Let $\mathscr I$ be a set of indices.
Let $\mathcal F_{i}$, $i \in \mathscr I$ and $\Gg$ be sub-$\sigma$-algebra of $\F$. If $\P\left.\left(\bigcap_{j=1}^nA_j\right|\Gg\right)=\prod\limits_{j=1}^n\P\left(A_j|\Gg\right)$ a.s., for arbitrarily different indices $\a_1,\cdots,\a_n\in \mathscr I$ and arbitrary $A_1\in \F_{\a_1},\cdots,A_n\in \F_{\a_n}$,
then it is said that $\mathcal F_{\alpha}$ is conditionally independent  w.r.t. $\Gg$, $\alpha \in \mathscr I$. If $\sigma(f_{i};\tau(\X_i))$ is conditionally independent  w.r.t. $\Gg$, $i \in \mathscr I$, then we say that the random elements $f_{i}$, $i \in \mathscr I$ with values in $(\X_i,\tau(\X_i))$ are conditionally independent  w.r.t. $\Gg$.
\end{definition}

%\begin{remark}\label{guidaoduli}
%If $f$ is a random element with values in $(\mathscr L(\mathscr Y,\mathscr Z),\tau_{\text{S}}(\mathscr L(\mathscr Y,\mathscr Z)))$.
%It follows from \cite{hy} that the Borel $\sigma$-algebra $\B(\mathscr L(\mathscr Y,\mathscr Z);\tau_{\text{S}}(\mathscr L(\mathscr Y,\mathscr Z)))$ is generated by the following open set
%\[V_{f_0,y_0,\varepsilon_0}:=\left\{f\in \mathscr L(\mathscr Y,\mathscr Z):\|(f-f_0)y_0\|_{\mathscr Z}<\varepsilon_0\right\},\]
%where $f_0\in \mathscr L(\mathscr Y,\mathscr Z),y_0\in \mathscr Y$ and $\varepsilon_0>0$. Notice that
%\[f^{-1}(V_{f_0,y_0,\varepsilon_0})=\{\omega\in \Omega:\|f(\omega)y_0-f_0y_0\|_{\mathscr Z}<\varepsilon_0\}=(fy_0)^{-1}(B_{\mathscr Z }^{\circ}(f_0y_0,\varepsilon_0)),\]
%where $B_{\mathscr Z}^{\circ}(x,r):=\{y\in \mathscr Z:\|y-x\|_{\mathscr Z}<r\}$ is the open ball in the Banach space $\mathscr Z$ centered at $x$ and with $r$ as the radius. Since a open set in the Banach space $(\mathscr Z,\tau_{\text{N}}(\mathscr Z))$ is the union of countable open balls, the random element $g$ with values in the Banach space $\mathscr Z$ is independent (conditionally independent with respect to a sub-$\sigma$-algebra $\G$ of $\mathcal F$) of the random element $f$ with values in $(\mathscr L(\mathscr Y,\mathscr Z),\tau_{\text{S}}(\mathscr L(\mathscr Y,\mathscr Z)))$ if and only if $g$ is independent (conditionally independent with respect to a sub-$\sigma$-algebra $\G$ of $\mathcal F$) of the random element $fy$ with values in $(\mathscr Z ,\tau_{\text{N}}(\Z))$, $\forall y\in \mathscr Y$.
%\end{remark}

%\vskip 0.2cm
By Definitions \ref{dulixing} and \ref{tiaojiandulixing}, we have the following propositions.

\begin{proposition}\label{lemmaA4}
If the family of random elements $\{f_k,k\ge 1\}$ with values in $(\X_1,\tau(\X_1))$ and the family of random elements $\{g_k,k\ge 1\}$ with values in $(\X_2,\tau(\X_2))$ are independent, then $\bigvee_{i=k}^{\infty}\sigma(f_i;\tau(\X_1))$ is conditionally independent of  $\bigvee_{i=k}^{\infty}\sigma(g_i;\tau(\X_2))$ with respect to $\bigvee_{i=1}^{k-1}(\sigma(f_i;\tau(\X_1))\bigvee \sigma(g_i;\tau(\X_2)))$, $\forall\ k\ge 2$.
\end{proposition}
\begin{proof}
Given an integer $k\ge 2$, denote $\Ff^k=\bigvee_{i=k}^{\infty}\sigma(f_i;\tau(\X_1))$, $\Gg^k=\bigvee_{i=k}^{\infty}\sigma(g_i;\tau(\X_2))$,  $\Ff_k=\bigvee_{i=1}^{k-1}\sigma(f_i;\tau(\X_1))$, $\Gg_k=\bigvee_{i=1}^{k-1}\sigma(g_i;\tau(\X_2))$ and $\Ff(k)=\bigvee_{i=1}^{k-1}(\sigma(f_i;\tau(\X_1))\bigvee \sigma(g_i;\tau(\X_2))).$
Let $E\in \Ff^k$, $F\in \Gg^k$, $A\in \Ff_k$ and $B\in \Gg_k$. On the one hand, noting that $\{f_k,k\ge 1\}$ and $\{g_k,k\ge 1\}$ are mutually independent, we know that $\bigvee_{i=1}^{\infty}\sigma(f_i;\tau(\X_1))$ and $\bigvee_{i=1}^{\infty}\sigma(g_i;\tau(\X_2))$ are also mutually independent. Since $A\cap E\in \bigvee_{i=1}^{\infty}\sigma(f_i;\tau(\X_1))$ and $B\cap F\in \bigvee_{i=1}^{\infty}\sigma(g_i;\tau(\X_2))$, we get $\P(E\cap F\cap A\cap B)=\P(E\cap A)\P(F\cap B)$, which together with $A\cap B \in \Ff(k)$ gives
\bna\label{apendix1}
\int_{A\cap B}\P(E\cap F|\Ff(k))\dd\P=\P(E\cap F\cap A\cap B)=\P(E\cap A)\P(F\cap B).
\ena
On the other hand, noting that $\1_F\in \Gg^k$, $\Ff(k)=\Ff_k\bigvee\Gg_k$ and $\Gg^k\bigvee\Gg_k=\bigvee_{i=1}^{\infty}\sigma(g_i;\tau(\X_2))$ is independent of $\Ff_k$, by Corollary $7.3.3$ in \cite{chow}, we obtain $\E[\1_F|\Ff(k)]=\E[\1_F|\Gg_k]$, which further implies
$$
\E[\1_E|\Ff(k)]\E[\1_F|\Ff(k)]=\E[\1_E\E[\1_F|\Ff(k)]|\Ff(k)]=\E[\1_E\E[\1_F|\Gg_k]|\Ff(k)],
$$
from which we get
\begin{align}\label{apendix3}
\int_{A\cap B}\P(E|\Ff(k))\P(F|\Ff(k))\dd\P = &\int_{A\cap B}\E[\1_E\E[\1_F|\Gg_k]|\Ff(k)]\dd\P \notag\\
 =& \int_{A\cap B}\1_E\E[\1_F|\Gg_k]\dd\P.
\end{align}
Noting that $\1_B\in \Gg_k$, $\1_{A\cap E}$ and $\E[\1_{B\cap F}|\Gg_k]$ are mutually independent, we have
\bna\label{apendix4}
\int_{A\cap B}\1_E\E[\1_F|\Gg_k]\dd\P&=&\E[\1_{A\cap E}\E[\1_{B\cap F}|\Gg_k]]\cr &=&\E[\1_{A\cap E}]\E[\1_{B\cap F}]\cr
&=&\P(A\cap E)\P(B\cap F).
\ena
It follows from (\ref{apendix3})-(\ref{apendix4}) that
\bna\label{apendix5}
\int_{A\cap B}\P(E|\Ff(k))\P(F|\Ff(k))\dd\P=\P(A\cap E)\P(B\cap F).
\ena
Combining (\ref{apendix1}) and (\ref{apendix5}) gives
\bna\label{abcd}
\int_{A\cap B}\P(E\cap F|\Ff(k))\dd\P=\int_{A\cap B}\P(E|\Ff(k))\P(F|\Ff(k))\dd\P.
\ena
Denote
\[\Pi=\left.\left\{\bigcup_{i=1}^n(A_i\cap B_i)\right|A_i \in \Ff_k, B_i\in \Gg_k, A_i\cap B_i\cap A_j\cap B_j=\emptyset, 1\leq i\neq j\leq n\right\}.\]
If $\bigcup_{i=1}^n(A_i\cap B_i)\in \Pi$ and $\bigcup_{i=1}^m(C_i\cap D_i)\in \Pi$, then we have
\bna\label{apendix6}
&&~~~\left(\bigcup_{i=1}^n(A_i\cap B_i)\right)\bigcap \left(\bigcup_{i=1}^m(C_i\cap D_i)\right)\cr
&&=\bigcup_{j=1}^m\left(\left(\bigcup_{i=1}^n(A_i\cap B_i)\right)\cap C_j\cap D_j\right) =\bigcup_{j=1}^m\bigcup_{i=1}^n\left(A_i\cap B_i\cap C_j\cap D_j\right).
\ena
Noting that $A_i\cap B_i\cap A_j\cap B_j=\emptyset, 1\leq i\neq j\leq n$, and $C_s\cap D_s\cap C_t\cap D_t=\emptyset, 1\leq s\neq t\leq m$, we know that $A_i\cap B_i\cap A_j\cap B_j\cap C_s\cap D_s\cap C_t\cap D_t=\emptyset$, $(i,j)\neq (s,t)$, which together with (\ref{apendix6}) gives
$$
\left(\bigcup_{i=1}^n(A_i\cap B_i)\right)\bigcap \left(\bigcup_{i=1}^m(C_i\cap D_i)\right)\in \Pi,
$$
thus, we conclude that $\Pi$ is a $\pi$-class, i.e., $\Pi$ is closed under the intersection operation of the set. Denote
\[\Lambda=\left\{ M\in \Ff(k)\bigg|\int_{M}\P(E\cap F|\Ff(k))\dd\P=\int_{M}\P(E|\Ff(k))\P(F|\Ff(k))\dd\P\right\}.\]
Noting that $\Pi$ is composed of finite disjoint union of elements in $\Ff_k\cap \Gg_k$, it follows from  (\ref{abcd}) that $\Omega \in \Lambda$. If $\bigcup_{i=1}^n(A_i\cap B_i)\in \Pi$, then we have
$$
\int_{\bigcup_{i=1}^n(A_i\cap B_i)}\P(E\cap F|\Ff(k))\dd\P=\int_{\bigcup_{i=1}^n(A_i\cap B_i)}\P(E|\Ff(k))\P(F|\Ff(k))\dd\P,
$$
which shows that $\bigcup_{i=1}^n(A_i\cap B_i)\in \Lambda$ and further gives $\Pi \subseteq \Lambda$. If $M_i\in \Lambda$ and $M_i \subseteq M_{i+1}$, then we obtain
\ban
\int_{\Omega}\1_{M_i}\P(E\cap F|\Ff(k))\dd\P=\int_{\Omega}\1_{M_i}\P(E|\Ff(k))\P(F|\Ff(k))\dd\P,~\forall
\ i \ge 1.
\ean
Noting that $M_i \subseteq M_{i+1}$ implies $\lim_{i\to\infty}\1_{M_i}=\1_{\bigcup_{i=1}^{\infty}M_i}$, by Lebesgue dominated convergence theorem, we get
\ban
\int_{\Omega}\1_{\bigcup_{i=1}^{\infty}M_i}\P(E\cap F|\Ff(k))\dd\P=\int_{\Omega}\1_{\bigcup_{i=1}^{\infty}M_i}
\P(E|\Ff(k))\P(F|\Ff(k))\dd\P.
\ean
Thus, we have $\bigcup_{i=1}^{\infty}M_i\in \Lambda$. If $M_1,M_2\in \Lambda$ and $M_1\subseteq M_2$, then $M_2=M_1\cup (M_2 \setminus M_1)$ implies
\ban
\left(\int_{M_1}+\int_{M_2\setminus M_1}\right)\P(E\cap F|\Ff(k))\dd\P=\left(\int_{M_1}+\int_{M_2\setminus M_1}\right)\P(E|\Ff(k))\P(F|\Ff(k))\dd\P.
\ean
It follows from $M_1\in \Lambda$ that
\ban
\int_{M_2\setminus M_1}\P(E\cap F|\Ff(k))\dd\P=\int_{M_2\setminus M_1}\P(E|\Ff(k))\P(F|\Ff(k))\dd\P,
\ean
which shows that $M_2\setminus M_1\in \Lambda$. From the above analysis, we know that $\Lambda$ is a $\lambda$-class, by Theorem 1.3.2 in \cite{chow}, we get $\sigma(\Pi)\subseteq \Lambda$. On the one hand, it follows from $\Pi \subseteq \Ff(k)=\Ff_k\bigvee\Gg_k$ that $\sigma(\Pi)\subseteq \Ff(k)$. On the other hand, noting that $\Ff_k\subseteq \Pi$ and $\Gg_k\subseteq \Pi$, we have $\Ff_k\cup\Gg_k\subseteq \sigma(\Pi)$, which leads to $\Ff(k)=\Ff_k\bigvee\Gg_k \subseteq \sigma(\Pi)$ and further gives $\Ff(k)=\sigma(\Pi)\subseteq \Lambda$. Noting that $E\in \Ff^k$ and $F\in \Gg^k$, we obtain
\ban
\int_{M}\P(E\cap F|\Ff(k))\dd\P=\int_{M}\P(E|\Ff(k))\P(F|\Ff(k))\dd\P,~\forall \ M\in \Ff(k),
\ean
which together with $\P(E|\Ff(k))\P(F|\Ff(k))\in \Ff(k)$ gives
\ban
\P(E\cap F|\Ff(k))=\P(E|\Ff(k))\P(F|\Ff(k))~\text{a.s.},
\ean
thus, we conclude that $\Ff^k$ is conditionally independent of $\Gg^k$ with respect to $\Ff(k)$.
\end{proof}

\begin{proposition} \label{lemmaA6}
Let $\Gg$ be the sub-$\sigma$-algebra of $\F$. The random element $T:\Omega\to \mathscr L (\X,\mathscr Y)$ with values in $(\mathscr L(\X,\mathscr Y),\tau_{\text{S}}(\mathscr L(\X,\mathscr Y)))$ satisfies $\E[\|T\|_{\mathscr L(\X,\mathscr Y)}^2]<\infty$ and $f\in L^2(\Omega;\X)$ is a random element with values in the Banach space $(\X,\tau_{\text{N}}(\X))$. If  $T$  is conditionally independent of  $f$   w.r.t. $\Gg$, then
$\E[Tf|\Gg]=\E[T\E[f|\Gg]|\Gg]$ a.s.
\end{proposition}

Before proving Proposition \ref{lemmaA6}, we introduce the following lemma.
\begin{lemma}\label{lemmaA5}
\rm{
Let $\Gg$ be a sub-$\sigma$-algebra of $\F$, $A\in \F$ and $f\in L^1(\Omega;\X)$. If $f$ is conditionally  independent of $\1_A$ with respect to $\Gg$, then
$\E[f\1_A|\Gg]=\E[f|\Gg]\E[\1_A|\Gg]~\text{a.s.}$}
\end{lemma}
\begin{proof}
Let $f=\1_B\otimes x$, where $B\in \F$ and $x\in \X$. Noting that
\bna\label{appendix3}
\sigma(\1_B\otimes x;\tau_{\text{N}}(\X))=\left\{(\1_B\otimes x)^{-1}(E):E\in \mathscr B(\X;\tau_{\text{N}}(\X))\right\},
\ena
it follows that
\bna\label{appendix4}
(\1_B\otimes x)^{-1}(E)=
\begin{cases}
\Omega, & \text{If $0\in E$, $x\in E$};\\
B, &  \text{If $0\in E$, $x\notin E$};\\
B^{\complement}, & \text{If $0\notin E$, $x\in E$};\\
\emptyset, & \text{If $0\notin E$, $x\notin E$}.
\end{cases}
,~
\forall \ E\in \mathscr B(\X;\tau_{\text{N}}(\X)),
\ena
which leads to $\sigma(\1_B\otimes x;\tau_{\text{N}}(\X))=\sigma(\1_B;\tau_{\text{N}}(\X))$. Since $\1_B\otimes x$ is conditionally independent of $\1_A$ with respect to $\Gg$, it follows from Definition \ref{tiaojiandulixing} that $\sigma(\1_B\otimes x;\tau_{\text{N}}(\X))$ is conditionally independent of  $\sigma(\1_A;\tau_{\text{N}}(\X))$ with respect to $\Gg$, from which we can further conclude that $\sigma(B)$ is conditionally independent of $\sigma(A)$ with respect to $\Gg$. Given $F\in \Gg$, on the one hand, we have
\bna\label{appendix1111}
\int_F\E[f\1_A|\Gg]\dd\P&=&x\int_F\E[\1_{A\cap B}|\Gg]\dd\P\cr
&=&x\int_F\P(A\cap B|\Gg)\dd\P\cr
&=&x\int_F\P(A|\Gg)\P(B|\Gg)\dd\P.
\ena
On the other hand, noting that $\E[\1_B\otimes x|\Gg]=\E[\1_B|\Gg]\otimes x~\text{a.s.}$, we get
\bna\label{appendix2222}
\int_F\E[f|\Gg]\E[\1_A|\Gg]\dd\P&=&\int_F\E[\1_B\otimes x|\Gg]\E[\1_A|\Gg]\dd\P
 = x\int_F\P(A|\Gg)\P(B|\Gg)\dd\P.
\ena
Then, by (\ref{appendix1111})-(\ref{appendix2222}), we obtain
\bna\label{appendix5}
\E[(\1_B\otimes x)\1_A|\Gg]=\E[\1_B\otimes x|\Gg]\E[\1_A|\Gg]~\text{a.s.}
\ena
Let $f=\sum_{i=1}^n\1_{B_i}\otimes x_i$, where $B_i \in \F$, $B_i\cap B_j=\emptyset$, $x_i\in \X$, $1\leq i\neq j\leq n$. Following the same way as (\ref{appendix3})-(\ref{appendix4}), we have
\ban
\sigma\left(\sum_{i=1}^n\1_{B_i}\otimes x_i;\tau_{\text{N}}(\X)\right)=\sigma\left(\bigcup_{i=1}^n B_i\right).
\ean
Thus, $f$ is conditionally independent of $\1_{A}$ with respect to $\Gg$, which implies that $\sigma(\bigcup_{i=1}^n B_i)$ is conditionally independent of $\sigma(A)$ with respect to $\Gg$, from which we know that $B_i$ is conditionally independent of $A$ with respect to $\Gg$, $1\leq i\leq n$. It follows from (\ref{appendix5}) that
\bna\label{appendix8}
\int_F\E\left.\left[\sum_{i=1}^n(\1_{B_i}\otimes x_i)\1_A\right|\Gg\right]\dd\P=\sum_{i=1}^n\int_F\E\left.\left[\1_{B_i}\otimes x\right|\Gg\right]\E[\1_A|\Gg]\dd\P,
\ena
which leads to
\ban
\E\left.\left[\sum_{i=1}^n(\1_{B_i}\otimes x_i)\1_A\right|\Gg\right]=\E\left.\left[\sum_{i=1}^n\1_{B_i}\otimes x\right|\Gg\right]\E[\1_A|\Gg]~\text{a.s.}
\ean
For the random element $f$ with values in Banach space $(\X,\tau_{\text{N}}(\X))$, there exists a sequence of simple functions with values in $\X$ such that $\lim_{n\to\infty}f_n=f~\text{a.s.}$ and $\|f_n\|\leq \|f\|~\text{a.s.}$ Noting that $\|f_n\1_A\|\leq \|f\|~\text{a.s.}$ together with $f\in L^1(\Omega;\X)$ implies that $\E[\|f\|]<\infty$, by the dominated convergence theorem, we get
\bna \label{appendix6}
\int_F\E\left.\left[\lim_{n\to\infty}f_n\1_A\right|\Gg\right]\dd\P=\int_F\lim_{n\to\infty}\E[f_n\1_A|\Gg]\dd\P,
\ena
and
\ban
\int_F\E\left.\left[\lim_{n\to\infty}f_n\right|\Gg\right]\E[\1_A|\Gg]\dd\P=\int_F\lim_{n\to\infty}\E[f_n|\Gg]\E[\1_A|\Gg]\dd\P.
\ean
It follows from (\ref{appendix8}) that
\bna\label{appendix9}
\int_F\E[f_n\1_A\Gg]\dd\P=\int_F\E[f_n|\Gg]\E[\1_A|\Gg]\dd\P.
\ena
Thus, by (\ref{appendix6})-(\ref{appendix9}), we have
\ban
\int_F\E[f\1_A|\Gg]\dd\P=\int_F\E[f|\Gg]\E[\1_A|\Gg]\dd\P,~\forall F\in \Gg.
\ean
\end{proof}


%\textbf{\emph{Proof of Proposition \ref{lemmaA6}:}}
\begin{proof}[Proof of Proposition \ref{lemmaA6}]
We first consider the case with $f=\sum_{i=1}^n\1_{A_i}\otimes x_i$, where $A_i\in \F$, $A_i\cap A_j=\emptyset$, $x_i\in \X$, $1\leq i\neq j\leq n$. Since $T$ is conditionally independent of $f$ with respect to $\Gg$, it follows that $\sigma(Tx_i;\tau_{\text{N}}(\Y))$ is conditionally independent of $\sigma(\sum_{i=1}^n\1_{A_i}\otimes x_i;\tau_{\text{N}}(\X))$ with respect to $\Gg$, $1\leq i\leq n$, from which we know that $\sigma(Tx_i;\tau_{\text{N}}(\Y))$ is conditionally independent of $\sigma(A_i)$ with respect to $\Gg$, thus, $Tx_i$ is conditionally independent of $\1_{A_i}$ with respect to $\Gg$, then by Lemma \ref{lemmaA5}, we have
\bna\label{appendix11}
\E[T(\1_{A_i}\otimes x_i)|\Gg]&=&\E[(Tx_i)\1_{A_i}|\Gg]\cr &=&\E[Tx_i|\Gg]\E[\1_{A_i}|\Gg]\cr
&=&\E[(Tx_i)\E[\1_{A_i}|\Gg]|\Gg]~\text{a.s.}
\ena
Noting that
\bna\label{appendix12}
\E[(Tx_i)\E[\1_{A_i}|\Gg]|\Gg]=\E[T(\E[\1_{A_i}|\Gg]\otimes x_i)|\Gg]=\E[T\E[\1_{A_i}\otimes x_i|\Gg]|\Gg]~\text{a.s.}
\ena
By (\ref{appendix11})-(\ref{appendix12}), we have
\bna\label{appendix14}
\E[Tf|\Gg]&=&\sum_{i=1}^n\E[T(\1_{A_i}\otimes x_i)|\Gg]\cr &=&\sum_{i=1}^n\E[T\E[\1_{A_i}\otimes x_i|\Gg]|\Gg]\cr
&=&\E[T\E[f|\Gg]|\Gg]~\text{a.s.}
\ena
Given the random element $f\in L^2(\Omega;\X)$ with values in Banach space $(\X,\tau_{\text{N}}(\X))$, by Lemma \ref{mse} and Definition \ref{vnwkelel}, we know that there exists a sequence $\{f_n,n\ge 1\}$ of simple functions, such that $\lim_{n\to\infty}f_n=f~\text{a.s.}$ and $\|f_n\|\leq \|f\|~\text{a.s.}$ If $\E[\|T\|^2]<\infty$ and $f\in L^2(\Omega;\X)$, noting that $T(\omega)\in \mathscr L(\X,\mathscr Y)$, we get $\|(Tf_n)(\omega)\|=\|T(\omega)f_n(\omega)\|\leq \|T(\omega)\|\|f_n(\omega)\|\leq \|T(\omega)\|^2+\|f(\omega)\|^2\in L^2(\Omega)$, which together with conditional dominated convergence theorem gives
\bna\label{appendix13}
\E[Tf|\Gg]=\E\left.\left[T\left(\lim_{n\to\infty}f_n\right)
\right|\Gg\right]=\E\left.\left[\lim_{n\to\infty}Tf_n\right|\Gg\right]
=\lim_{n\to\infty}\E[Tf_n|\Gg]~\text{a.s.}
\ena
By (\ref{appendix14}), we obtain
\ban
\E[Tf_n|\Gg]=\E[T\E[f_n|\Gg]|\Gg]~\text{a.s.}
\ean
It follows from $\|T\E[f_n|\Gg]\|\leq \|T\|\|\E[f_n|\Gg]\|\leq \|T\|^2+\E[\|f_n\|^2|\Gg]\leq \|T\|^2+\E[\|f\|^2|\Gg]  \in L^2(\Omega)$ and conditional dominated convergence theorem that
\bna\label{appendix16}
\lim_{n\to\infty}\E[T\E[f_n|\Gg]|\Gg]=\E\left.\left[\lim_{n\to\infty}T\E[f_n|\Gg]\right|\Gg\right]=\E\left.\left[T\E\left.\left[\lim_{n\to\infty}f_n\right|\Gg\right]\right|\Gg\right]~\text{a.s.}
\ena
Hence, by (\ref{appendix13})-(\ref{appendix16}), we get
$\E[Tf|\Gg]=\E[T\E[f|\Gg]|\Gg]$ a.s.
\end{proof}




%Proposition \ref{wenknknkn} and Propositions \ref{vnwlssfweewwfew}-\ref{lemmaA6}}
%\setcounter{lemma}{0}
%\def\thelemma{B.\arabic{lemma}}
%\setcounter{definition}{0}
%\def\thedefinition{B.\arabic{definition}}
%\setcounter{equation}{0}
%\def\theequation{B.\arabic{equation}}


%\subsection{Proofs of Appendices A.2-A.4}




%\textbf{\emph{Proof of Proposition \ref{tiaojianqiwangxingzhi}:}}



%\textbf{\emph{Proof of Proposition \ref{lemmaA4}:}}





\section{Proofs of Lemmas in Section 3}\label{appendixc}
%\setcounter{lemma}{0}
%\def\thelemma{C.\arabic{lemma}}
%\setcounter{definition}{0}
%\def\thedefinition{C.\arabic{definition}}
%\setcounter{equation}{0}
%\def\theequation{C.\arabic{equation}}
 \setcounter{equation}{0}
\renewcommand{\theequation}{B.\arabic{equation}}

For the convenience of notational writing without giving rise to ambiguity, the subscripts of the parametrization in Banach spaces and the subscripts of the inner product in Hilbert spaces will be omitted in the sequel.

%Before proving the main result, we need to introduce some key lemmas below, which are proved in Appendix \ref{appendixd}.




%\textbf{\emph{Proof of Theorem \ref{wendingxing}:}}
%\begin{proof}[Proof of Theorem \ref{wendingxing}]
%Given the initial value $x(0)\in \X_1$, by Proposition \ref{nlllwwieiie}.(a)-(c), we know that $x(k)$ is a random element with values in the Hilbert space $(\X_1,\tau_{\text{N}}(\X_1))$. It follows from the random difference equation (\ref{chafen}) that
%\begin{align}\label{okjfs}
%x(k+1) = \left(\prod_{i=0}^k(I_{\X_1}-F(i))\right)x(0)  +\sum_{i=0}^k\left(\prod_{j=i+1}^k(I_{\X_1}-F(j))\right)G(i)u(i),~k\ge 0,
%\end{align}
%by (\ref{okjfs}) and Cauchy inequality, we have
%\bna\label{ijfwg}
%\E[\|x(k+1)\|^2]&\leq& 2\E\left[\left\|\left(\prod_{i=0}^k(I_{\X_1}-F(i))\right)x(0)\right\|^2\right]\cr &&+2\E\left[\left\|\sum_{i=0}^k\left(\prod_{j=i+1}^k(I_{\X_1}-F(j))\right)G(i)u(i)\right\|^2\right].
%\ena
%It follows from Definition \ref{tiaojiandulixing} that $\sigma(u(k);\tau_{\text{N}}(\X_2))$ is independent of $\sigma\left(F(k);\tau_{\text{S}}(\LL(\X_1))\right)\bigvee\\ \sigma\left(G(k);\tau_{\text{S}}(\LL(\X_2,\X_1))\right),$ which leads to
%\bna\label{xxll}
%&~&\E\left[\left\|G(t)u(t)\right\|^2\right]\cr&\leq& \E\left[\|G(t)\|^2\right]\E\left[\|u(t)\|^2\right]\cr
%&\leq& \sup_{k\ge 0}\E\left[\|u(k)\|^2\right]\E\left[\|G(t)\|^2\right],~t\ge 0.
%\ena
%By the condition (\ref{ssafe}), we get $$\sup_{k\ge 0}\E[\|G(k)\|^2]<\infty,$$ thus, (\ref{xxll}) implies $\sup_{k\ge 0}\E[\|G(k)u(k)\|^2]<\infty$. For $0\leq t<k$, by the condition (\ref{qqqqq}), we obtain
%\bna\label{xxoopp}
%&&~~~~\E\left.\left[\prod_{j=t+1}^k\left\|I_{\X_1}-F(j)\right\|^4\right|\F(t)\right]\cr
%&&=\E\left.\left[\E\left.\left[\prod_{j=t+1}^k\left\|I_{\X_1}-F(j)\right\|^4\right|\mathcal F(k-1)\right]\right|\F(t)\right]\cr
%&&=\E\left.\left[\E\left.\left[\left\|I_{\X_1}-F(k)\right\|^4\right|\mathcal F(k-1)\right]\prod_{j=t+1}^{k-1}\|I_{\X_1}-F(j)\|^4\right|\F(t)\right]\cr
%&&\leq (1+\gamma(k))\E\left.\left[\prod_{j=t+1}^{k-1}\|I_{\X_1}-F(j)\|^4\right|\F(t)\right]\cr
%&&\vdots \cr
%&&\leq \prod_{j=t+1}^k(1+\gamma(j))~\text{a.s.}
%\ena
%It follows from $\sup_{k\ge 0}\E[\|G(k)\|^4]<\infty$ and (\ref{xxoopp}) that
%\bna\label{oopp}
%&&~~~~\E\left[\left\|\left(\prod_{j=s+1}^k(I_{\X_1}-F(j))\right)^*\left(\prod_{j=t+1}^k(I_{\X_1}-F(j))\right)G(t)\right\|^2\right]\cr
%&&\leq \E\left[\left(\prod_{j=t+1}^k\|I_{\X_1}-F(j)\|^4\right)\left(\prod_{j=s+1}^t\|I_{\X_1}-F(j)\|^2\right)\|G(t)\|^2\right]\cr
%&&=\E\left[\E\left.\left[\prod_{j=t+1}^k\|I_{\X_1}-F(j)\|^4\right|\F(t)\right]\left(\prod_{j=s+1}^t\|I_{\X_1}-F(j)\|^2\right)\|G(t)\|^2\right]\cr
%&&\leq \left(\prod_{j=t+1}^k(1+\gamma(j))\right)\E\left[\left(\prod_{j=s+1}^t\|I_{\X_1}-F(j)\|^2\right)\|G(t)\|^2\right]\cr
%&&\leq \left(\prod_{j=t+1}^k(1+\gamma(j))\right)\left(\E\left[\prod_{j=s+1}^t\|I_{\X_1}-F(j)\|^4\right]+\E\left[\|G(t)\|^4\right]\right)\cr
%&&\leq \left(\prod_{j=t+1}^k(1+\gamma(j))\right)\left(\prod_{j=s+1}^t(1+\gamma(j))+\sup_{k\ge 0}\E\left[\|G(k)\|^4\right]\right)\cr
%&&\leq \left(\prod_{k=0}^{\infty}(1+\gamma(k))\right)\left(1+\sup_{k\ge 0}\E\left[\|G(k)\|^4\right]\right)\cr
%&&<\infty,~0\leq s<t\leq k,
%\ena
%where the last inequality is due to $\sum_{k=0}^{\infty}\gamma(k)<\infty$. By $\sup_{k\ge 0}\E[\|u(k)\|^2]<\infty$,  Proposition \ref{nlllwwieiie} and (\ref{oopp}), we have
%\bna\label{vnokllllll}
% \left(\prod_{j=s+1}^k(I_{\X_1}-F(j))\right)^*\left(
%\prod_{j=t+1}^k(I_{\X_1}-F(j))\right)G(t)u(t)
% \in L^1(\Omega;\X_1),~0\leq s<t\leq k.
%\ena
%Noting that $\E[u(t)|\F(s)]=\E[\E[u(t)|\mathcal F(t-1)]|\F(s)]=0$, $0\leq s<t$, and Proposition \ref{wenknknkn} implies $G(s)u(s)\in L^0(\Omega,\F(s);\X^N)$, it follows from (\ref{oopp})-(\ref{vnokllllll}), Proposition \ref{nlllwwieiie}.(a)-(c), Lemma \ref{lemmaA1} and Propositions \ref{lemmaA4}-\ref{lemmaA6} that
%\begin{align}\label{fwwe}
%&\E\left[\left\langle \left(\prod_{j=s+1}^k(I_{\X_1}-F(j))\right)G(s)u(s), \left(\prod_{j=t+1}^k(I_{\X_1}-F(j))\right)G(t)u(t)\right\rangle \right]\notag\\
%=& \E\left[\left\langle G(s)u(s), \left(\prod_{j=s+1}^k(I_{\X_1}-F(j))\right)^*\left(\prod_{j=t+1}^k(I_{\X_1}-F(j))\right)G(t)u(t)\right\rangle \right]\notag\\
%=& \E\left[\E\left[\left\langle G(s)u(s), \left(\prod_{j=s+1}^k(I_{\X_1}-F(j))\right)^* \left(\prod_{j=t+1}^k(I_{\X_1}-F(j))\right)G(t)u(t)\right\rangle\bigg|\F(s) \right]\right]\cr
% =& \E\left[\left\langle G(s)u(s), \E\left[\left(\prod_{j=s+1}^k(I_{\X_1}-F(j))\right)^* \left(\prod_{j=t+1}^k(I_{\X_1}-F(j))\right)G(t)u(t)\bigg|\F(s) \right]\right\rangle\right]\cr
% =&\E\Bigg[\bigg\langle G(s)u(s), \E\bigg[\left(\prod_{j=s+1}^k(I_{\X_1}-F(j))\right)^*\notag\\
% &\times\left(\prod_{j=t+1}^k(I_{\X_1}-F(j))\right)G(t)\E[u(t)|\F(s)]\bigg|\F(s) \bigg]\bigg\rangle\Bigg]\notag\\
% =& 0.
%\end{align}
%%\bna\label{fwwe}
%%&&~~~~\E\left[\left\langle \left(\prod_{j=s+1}^k(I_{\X_1}-F(j))\right)G(s)u(s), \left(\prod_{j=t+1}^k(I_{\X_1}-F(j))\right)G(t)u(t)\right\rangle \right]\cr
%%&&=\E\left[\left\langle G(s)u(s), \left(\prod_{j=s+1}^k(I_{\X_1}-F(j))\right)^*\left(\prod_{j=t+1}^k(I_{\X_1}-F(j))\right)G(t)u(t)\right\rangle \right]\cr
%%&&=\E\left[\E\left[\left\langle G(s)u(s), \left(\prod_{j=s+1}^k(I_{\X_1}-F(j))\right)^*\right.\right.\right.\cr &&~~~~\times\left.\left.\left.\left.\left(\prod_{j=t+1}^k(I_{\X_1}-F(j))\right)G(t)u(t)\right\rangle\right|\F(s) \right]\right]\cr
%%&&=\E\left[\left\langle G(s)u(s), \E\left[\left(\prod_{j=s+1}^k(I_{\X_1}-F(j))\right)^*\right.\right.\right.\cr &&~~~~\times\left.\left.\left.\left.\left(\prod_{j=t+1}^k(I_{\X_1}-F(j))\right)G(t)u(t)\right|\F(s) \right]\right\rangle\right]\cr
%%&&=\E\left[\left\langle G(s)u(s), \E\left[\left(\prod_{j=s+1}^k(I_{\X_1}-F(j))\right)^*\right.\right.\right.\cr &&~~~~\times\left.\left.\left.\left.\left(\prod_{j=t+1}^k(I_{\X_1}-F(j))\right)G(t)\E[u(t)|\F(s)]\right|\F(s) \right]\right\rangle\right]
%% =0.
%%\ena
% Denote $\Lambda=\{i\in \mathbb N:\E[\|G(i)u(i)\|^2]>0\}$. By (\ref{ijfwg}), (\ref{xxll}) and (\ref{fwwe}), we obtain
%\bna\label{wiiie}
%&&~~~~\E\left[\|x(k+1)\|^2\right]\cr
%&&=\E\left[\left\|\left(\prod_{i=0}^k(I_{\X_1}-F(i))\right)x(0)\right\|^2\right]+\sum_{i=0}^k\E\left[\left\|\left(\prod_{j=i+1}^k(I_{\X_1}-F(j))\right)G(i)u(i)\right\|^2\right]\cr
%&&\leq \E\left[\left\|\left(\prod_{i=0}^k(I_{\X_1}-F(i))\right)x(0)\right\|^2\right]+\sum_{i=0}^{\infty}\E\left[\left\|\left(\prod_{j=i+1}^k(I_{\X_1}-F(j))\right)G(i)u(i)\right\|^2\right]\cr
%&&=\E\left[\left\|\left(\prod_{i=0}^k(I_{\X_1}-F(i))\right)x(0)\right\|^2\right]\cr &&~~~+\sum_{i\in \Lambda}\E\left[\|G(i)u(i)\|^2\right]\E\left[\left\|\left(\prod_{j=i+1}^k(I_{\X_1}-F(j))\right)\frac{G(i)u(i)}{\left(\E\left[\|G(i)u(i)\|^2\right]\right)^{\frac{1}{2}}}\right\|^2\right]\cr
%&&\leq \E\left[\left\|\left(\prod_{i=0}^k(I_{\X_1}-F(i))\right)x(0)\right\|^2\right]+\sup_{k\ge 0}\E\left[\|u(k)\|^2\right]\sum_{i\in \Lambda}\E\left[\|G(i)\|^2\right]\cr &&~~~~\times\E\left[\left\|\left(\prod_{j=i+1}^k(I_{\X_1}-F(j))\right)\frac{G(i)u(i)}{\left(\E\left[\|G(i)u(i)\|^2\right]\right)^{\frac{1}{2}}}\right\|^2\right].
%\ena
%Noting that the operator-valued random sequence $\{I_{\X_1}-F(k),k\ge 0\}$ is $L_2^2$-stable with respect to the filter $\{\F(k),k\ge 0\}$, we get
%\bna\label{oowf}
%\lim_{k\to \infty}\E\left[\left\|\left(\prod_{i=0}^k(I_{\X_1}-F(i))\right)x(0)\right\|^2\right]=0.
%\ena
%By Proposition \ref{wenknknkn}, we know that $G(i)u(i)\in L^0(\Omega,\mathcal F(i);\X_1)$ and
%\ban
%\E\left[\left\|\frac{G(i)u(i)}{\left(\E\left[\|G(i)u(i)\|^2\right]\right)^{\frac{1}{2}}}\right\|^2\right]=1,~i\in \Lambda,
%\ean
%which leads to
%\bna\label{wwml}
%\lim_{k\to\infty}\E\left[\left\|\left(\prod_{j=i+1}^k(I_{\X_1}-F(j))\right)\frac{G(i)u(i)}{\left(\E\left[\|G(i)u(i)\|^2\right]\right)^{\frac{1}{2}}}\right\|^2\right]=0,~ i\in \Lambda.
%\ena
%It follows from $G(i)u(i)\in L^0(\Omega,\mathcal F(i);\X_1)$ and (\ref{xxoopp}) that
%\bna\label{llnnv}
%&&~~~\sup_{k\ge 0\atop i\in \Lambda}\E\left[\left\|\left(\prod_{j=i+1}^k(I_{\X_1}-F(j))\right)\frac{G(i)u(i)}{\left(\E\left[\|G(i)u(i)\|^2\right]\right)^{\frac{1}{2}}}\right\|^2\right]\cr
%&&=\sup_{k\ge 0\atop i\in \Lambda}\E\left.\left[\E\left[\left\|\left(\prod_{j=i+1}^k(I_{\X_1}-F(j))\right)\frac{G(i)u(i)}{\left(\E\left[\|G(i)u(i)\|^2\right]\right)^{\frac{1}{2}}}\right\|^2\right|\mathcal F(i)\right]\right]\cr
%&&\leq \sup_{k\ge 0\atop i\in \Lambda}\E\left.\left[\E\left[\left\|\prod_{j=i+1}^k(I_{\X_1}-F(j))\right\|^2\right|\mathcal F(i)\right]\left\|\frac{G(i)u(i)}{\left(\E\left[\|G(i)u(i)\|^2\right]\right)^{\frac{1}{2}}}\right\|^2\right]\cr
%&&\leq \sup_{k\ge 0\atop i\in \Lambda}\E\left.\left[\E\left[\left\|\prod_{j=i+1}^k(I_{\X_1}-F(j))\right\|^4\right|\mathcal F(i)\right]^{\frac{1}{2}}\left\|\frac{G(i)u(i)}{\left(\E\left[\|G(i)u(i)\|^2\right]\right)^{\frac{1}{2}}}\right\|^2\right]\cr
%&&\leq \sup_{k\ge 0\atop i\in \Lambda}\E\left[\sqrt{\prod_{j=i+1}^k(1+\gamma(j))}\left\|\frac{G(i)u(i)}{\left(\E\left[\|G(i)u(i)\|^2\right]\right)^{\frac{1}{2}}}\right\|^2\right]\cr
%&&\leq \sqrt{\prod_{k=0}^{\infty}(1+\gamma(k))}\sup_{i\in \Lambda}\E\left[\left\|\frac{G(i)u(i)}{\left(\E\left[\|G(i)u(i)\|^2\right]\right)^{\frac{1}{2}}}\right\|^2\right]\cr
%&&=\sqrt{\prod_{k=0}^{\infty}(1+\gamma(k))}<\infty.
%\ena
%By the condition (\ref{ssafe}), (\ref{wwml})-(\ref{llnnv}) and Lemma \ref{lemma6}, we obtain
%\bna\label{llks}
%~~~\lim_{k\to \infty}\sum_{i\in \Lambda}\E\left[\|G(i)\|^2\right]\E\left[\left\|\left(\prod_{j=i+1}^k(I_{\X_1}-F(j))\right)\frac{G(i)u(i)}{\left(\E\left[\|G(i)u(i)\|^2\right]\right)^{\frac{1}{2}}}\right\|^2\right]=0.
%\ena
%Thus, substituting (\ref{oowf}) and (\ref{llks}) into (\ref{wiiie}) gives $\lim_{k\to\infty}\E[\|x(k)\|^2]=0$.
%\end{proof}


%\textbf{\emph{Proof of Lemma \ref{dingliyi1}:}}
\begin{proof}[Proof of Lemma \ref{wendingxing}]
Given the initial value $x(0)\in \X_1$, by  Proposition \ref{nlllwwieiie}.(a)-(c), we know that $x(k)$ is a random element with values in the Hilbert space $(\X_1,\tau_{\text{N}}(\X_1))$. It follows from the random difference equation (\ref{chafen}) that
$x(k+1) = \big(\prod_{i=0}^k(I_{\X_1}-F(i))\big)x(0) +\sum_{i=0}^k\big(\prod_{j=i+1}^k(I_{\X_1}-F(j))\big)G(i)u(i),~k\ge 0.$
%\begin{align}\label{okjfs}
%x(k+1) = &\left(\prod_{i=0}^k(I_{\X_1}-F(i))\right)x(0) \notag\\ &+\sum_{i=0}^k\left(\prod_{j=i+1}^k(I_{\X_1}-F(j))\right)G(i)u(i),~k\ge 0,
%\end{align}
Then, by   Cauchy inequality, we have
\begin{align}\label{ijfwg}
&\E[\|x(k+1)\|^2]\notag\\
\leq& 2\E\left[\left\| \prod_{i=0}^k(I_{\X_1}-F(i)) x(0)
\right\|^2\right] +2\E\left[\left\|\sum_{i=0}^k\left(\prod_{j=i+1}^k(I_{\X_1}-F(j))\right)G(i)u(i)\right\|^2\right].
\end{align}
It follows from Definition \ref{tiaojiandulixing} that $\sigma(u(k);\tau_{\text{N}}(\X_2))$ is independent of $\sigma\left(F(k);\tau_{\text{S}}(\LL(\X_1))\right)\bigvee\\ \sigma\left(G(k);\tau_{\text{S}}(\LL(\X_2,\X_1))\right),$ which leads to
\begin{align}\label{xxll}
 &\E\left[\left\|G(t)u(t)\right\|^2\right]\notag\\
\leq& \E\left[\|G(t)\|^2\right]\E\left[\|u(t)\|^2\right]\notag\\
\leq& \sup_{k\ge 0}\E\left[\|u(k)\|^2\right]\E\left[\|G(t)\|^2\right],~t\ge 0.
\end{align}
By the condition (\ref{ssafe}), we get $\sup_{k\ge 0}\E[\|G(k)\|^2]<\infty,$ thus, (\ref{xxll}) implies $\sup\limits_{k\ge 0}\E[\|G(k)u(k)\|^2]<\infty$. For $0\leq t<k$, by the condition (\ref{qqqqq}), we obtain
\begin{align}\label{xxoopp}
&\E\left.\left[\prod_{j=t+1}^k\left\|I_{\X_1}-F(j)\right\|^4\right|\F(t)\right]\cr
=&\E\left.\left[\E\left.\left[\prod_{j=t+1}^k\left\|I_{\X_1}-F(j)\right\|^4\right|\mathcal F(k-1)\right]\right|\F(t)\right]\cr
=&\E\bigg[\E\left[\left\|I_{\X_1}-F(k)\right\|^4\big|\mathcal F(k-1)\right]  \prod_{j=t+1}^{k-1}\|I_{\X_1}-F(j)\|^4\Big|\F(t)\bigg]\cr
\leq & (1+\gamma(k))\E\left.\left[\prod_{j=t+1}^{k-1}\|I_{\X_1}-F(j)\|^4\right|\F(t)\right]\cr
\leq & \prod_{j=t+1}^k(1+\gamma(j))~\text{a.s.}
\end{align}
It follows from $\sup_{k\ge 0}\E[\|G(k)\|^4]<\infty$ and (\ref{xxoopp}) that
\begin{align}\label{oopp}
&\E\Bigg[\bigg\|\bigg(\prod_{j=s+1}^k(I_{\X_1}-F(j))\bigg)^*
\bigg(\prod_{j=t+1}^k(I_{\X_1}-F(j))\bigg)  G(t)\bigg\|^2\Bigg]\notag\\
 \leq & \E\Bigg[\bigg(\prod_{j=t+1}^k\|I_{\X_1}-F(j)\|^4\bigg)
 \bigg(\prod_{j=s+1}^t\|I_{\X_1}-F(j)\|^2\bigg)  \|G(t)\|^2\Bigg]\notag\\
 %=&\E\Bigg[\E\left.\left[\prod_{j=t+1}^k\|I_{\X_1}-F(j)\|^4\right|\F(t)\right]\notag\\
% &\times
% \left(\prod_{j=s+1}^t\|I_{\X_1}-F(j)\|^2\right)\|G(t)\|^2\Bigg]\notag\\
 \leq & \left(\prod_{j=t+1}^k(1+\gamma(j))\right)
 \E\Bigg[\bigg(\prod_{j=s+1}^t\|I_{\X_1}-F(j)\|^2\bigg)
 \|G(t)\|^2\Bigg]\cr
 \leq & \left(\prod_{j=t+1}^k(1+\gamma(j))\right)
 \Bigg(\E\Bigg[\prod_{j=s+1}^t\|I_{\X_1}-F(j)\|^4\Bigg]
 +\E\left[\|G(t)\|^4\right]\Bigg)\cr
 \leq & \left(\prod_{j=t+1}^k(1+\gamma(j))\right) \left(\prod_{j=s+1}^t(1+\gamma(j))+\sup_{k\ge 0}\E\left[\|G(k)\|^4\right]\right)\cr
 \leq &\left(\prod_{k=0}^{\infty}(1+\gamma(k))\right) \left(1+\sup_{k\ge 0}\E\left[\|G(k)\|^4\right]\right)
 <\infty, \  ~0\leq s<t\leq k,
\end{align}
where the last inequality is due to $\sum_{k=0}^{\infty}\gamma(k)<\infty$. By $\sup_{k\ge 0}\E[\|u(k)\|^2]<\infty$,   Proposition \ref{nlllwwieiie} and (\ref{oopp}), we have
\begin{align}\label{vnokllllll}
  \left(\prod_{j=s+1}^k(I_{\X_1}-F(j))\right)^*\left(
\prod_{j=t+1}^k(I_{\X_1}-F(j))\right)G(t)u(t)
  \in L^1(\Omega;\X_1),~0\leq s<t\leq k.
\end{align}
Noting that $\E[u(t)|\F(s)]=\E[\E[u(t)|\mathcal F(t-1)]|\F(s)]=0$, $0\leq s<t$, and Proposition \ref{wenknknkn}  implies $G(s)u(s)\in L^0(\Omega,\F(s);\X^N)$, it follows from (\ref{oopp})-(\ref{vnokllllll}), Proposition \ref{nlllwwieiie}.(a)-(c),  Proposition 2.6.13 in \cite{hy}  and Propositions \ref{lemmaA4}-\ref{lemmaA6} that
\begin{align}\label{fwwe}
&\E\Bigg[\Bigg\langle  \prod_{j=s+1}^k(I_{\X_1}-F(j)) G(s)u(s), \prod_{j=t+1}^k(I_{\X_1}-F(j)) G(t)u(t)\Bigg\rangle \Bigg]\notag\\
=& \E\Bigg[\Bigg\langle G(s)u(s), \left(\prod_{j=s+1}^k(I_{\X_1}-F(j))\right)^*\left(\prod_{j=t+1}^k(I_{\X_1}-F(j))
\right)G(t)u(t)\Bigg\rangle\Bigg]\notag\\
=& \E\Bigg[\E\Bigg[\Bigg\langle G(s)u(s), \left(\prod_{j=s+1}^k(I_{\X_1}-F(j))\right)^* \left(\prod_{j=t+1}^k(I_{\X_1}-F(j))\right)G(t)u(t)\Bigg\rangle\bigg|\F(s) \Bigg]\Bigg]\cr
 =& \E\Bigg[\Bigg\langle G(s)u(s), \E\Bigg[\left(\prod_{j=s+1}^k(I_{\X_1}-F(j))\right)^*  \left(\prod_{j=t+1}^k(I_{\X_1}-F(j))\right)G(t)u(t)\bigg|\F(s) \Bigg]\Bigg\rangle\Bigg]\cr
 =&\E\Bigg[\bigg\langle G(s)u(s), \E\bigg[\left(\prod_{j=s+1}^k(I_{\X_1}-F(j))\right)^*\notag\\
 &\times\left(\prod_{j=t+1}^k(I_{\X_1}-F(j))\right)G(t)\E[u(t)|\F(s)]\bigg|\F(s) \bigg]\bigg\rangle\Bigg]\notag\\
 =& 0.
\end{align}
%\bna\label{fwwe}
%&&~~~~\E\left[\left\langle \left(\prod_{j=s+1}^k(I_{\X_1}-F(j))\right)G(s)u(s), \left(\prod_{j=t+1}^k(I_{\X_1}-F(j))\right)G(t)u(t)\right\rangle \right]\cr
%&&=\E\left[\left\langle G(s)u(s), \left(\prod_{j=s+1}^k(I_{\X_1}-F(j))\right)^*\left(\prod_{j=t+1}^k(I_{\X_1}-F(j))\right)G(t)u(t)\right\rangle \right]\cr
%&&=\E\left[\E\left[\left\langle G(s)u(s), \left(\prod_{j=s+1}^k(I_{\X_1}-F(j))\right)^*\right.\right.\right.\cr &&~~~~\times\left.\left.\left.\left.\left(\prod_{j=t+1}^k(I_{\X_1}-F(j))\right)G(t)u(t)\right\rangle\right|\F(s) \right]\right]\cr
%&&=\E\left[\left\langle G(s)u(s), \E\left[\left(\prod_{j=s+1}^k(I_{\X_1}-F(j))\right)^*\right.\right.\right.\cr &&~~~~\times\left.\left.\left.\left.\left(\prod_{j=t+1}^k(I_{\X_1}-F(j))\right)G(t)u(t)\right|\F(s) \right]\right\rangle\right]\cr
%&&=\E\left[\left\langle G(s)u(s), \E\left[\left(\prod_{j=s+1}^k(I_{\X_1}-F(j))\right)^*\right.\right.\right.\cr &&~~~~\times\left.\left.\left.\left.\left(\prod_{j=t+1}^k(I_{\X_1}-F(j))\right)G(t)\E[u(t)|\F(s)]\right|\F(s) \right]\right\rangle\right]
% =0.
%\ena
 Denote $\Lambda=\{i\in \mathbb N:\E[\|G(i)u(i)\|^2]>0\}$. By (\ref{ijfwg}), (\ref{xxll}) and (\ref{fwwe}), we obtain
\begin{align}\label{wiiie}
&\E\left[\|x(k+1)\|^2\right]\cr
=&\E\left[\left\|\left(\prod_{i=0}^k(I_{\X_1}-F(i))\right)x(0)\right\|^2\right]+\sum_{i=0}^k\E\left[\left\|\left(\prod_{j=i+1}^k(I_{\X_1}-F(j))\right)G(i)u(i)\right\|^2\right]\cr
 \leq & \E\left[\left\|\left(\prod_{i=0}^k(I_{\X_1}-F(i))\right)x(0)\right\|^2\right]+\sum_{i=0}^{\infty}\E\left[\left\|\left(\prod_{j=i+1}^k(I_{\X_1}-F(j))\right)G(i)u(i)\right\|^2\right]\cr
 =&\E\left[\left\|\left(\prod_{i=0}^k(I_{\X_1}-F(i))
 \right)x(0)\right\|^2\right]\notag\\
 &+\sum_{i\in \Lambda}\E\left[\|G(i)u(i)\|^2\right] \E\left[\left\|\left(\prod_{j=i+1}^k(I_{\X_1}-F(j))\right)\frac{G(i)u(i)}{\left(\E\left[\|G(i)u(i)\|^2\right]\right)^{\frac{1}{2}}}\right\|^2\right]\cr
 \leq & \E\left[\left\|\left(\prod_{i=0}^k(I_{\X_1}-F(i))\right)x(0)\right\|^2\right]\notag\\
 &+\sup_{k\ge 0}\E\left[\|u(k)\|^2\right]\sum_{i\in \Lambda}\E\left[\|G(i)\|^2\right]\E\Bigg[\bigg\| \prod_{j=i+1}^k(I_{\X_1}-F(j))
 \frac{G(i)u(i)}{\left(\E\left[\|G(i)u(i)\|^2\right]\right)^{\frac{1}{2}}}
 \bigg\|^2\Bigg].
\end{align}
Noting that the operator-valued random sequence $\{I_{\X_1}-F(k),k\ge 0\}$ is $L_2^2$-stable with respect to the filter $\{\F(k),k\ge 0\}$, we get
\bna\label{oowf}
\lim_{k\to \infty}\E\left[\left\|\left(\prod_{i=0}^k(I_{\X_1}-F(i))\right)x(0)\right\|^2\right]=0.
\ena
By Proposition \ref{wenknknkn}, we know that $G(i)u(i)\in L^0(\Omega,\mathcal F(i);  \X_1)$ and $\E\left[\left\|\frac{G(i)u(i)}{\left(\E\left[\|G(i)u(i)\|^2\right]\right)^{\frac{1}{2}}}\right\|^2\right]=1,~i\in \Lambda,$
%\ban
%\E\left[\left\|\frac{G(i)u(i)}{\left(\E\left[\|G(i)u(i)\|^2\right]\right)^{\frac{1}{2}}}\right\|^2\right]=1,~i\in \Lambda,
%\ean
which leads to
\begin{align}\label{wwml}
&\lim_{k\to\infty}\E\left[\left\|\left(\prod_{j=i+1}^k(I_{\X_1}-F(j))\right)
\frac{G(i)u(i)}{\left(\E\left[\|G(i)u(i)\|^2\right]
\right)^{\frac{1}{2}}}\right\|^2\right]\notag\\
=&0,~ i\in \Lambda.
\end{align}
It follows from $G(i)u(i)\in L^0(\Omega,\mathcal F(i);\X_1)$ and (\ref{xxoopp}) that
\begin{align}\label{llnnv}
&\sup_{k\ge 0\atop i\in \Lambda}\E\left[\left\|\left(\prod_{j=i+1}^k(I_{\X_1}-F(j))\right)\frac{G(i)u(i)}{\left(\E\left[\|G(i)u(i)\|^2\right]\right)^{\frac{1}{2}}}\right\|^2\right]\cr
 =&\sup_{k\ge 0\atop i\in \Lambda}\E\Bigg[\E\Bigg[\bigg\|
 \bigg(\prod_{j=i+1}^k(I_{\X_1}-F(j))\bigg)\frac{G(i)u(i)}{\left(\E\left[\|G(i)u(i)\|^2\right]\right)^{\frac{1}{2}}}
 \bigg\|^2\bigg|\mathcal F(i)\Bigg]\Bigg]\cr
 \leq &\sup_{k\ge 0\atop i\in \Lambda}\E\Bigg[\E\Bigg[\bigg\|\prod_{j=i+1}^k(I_{\X_1}
 -F(j))\bigg\|^2
 \bigg|\mathcal F(i)\Bigg]\left\|\frac{G(i)u(i)}{\left(\E\left[\|G(i)u(i)\|^2\right]
 \right)^{\frac{1}{2}}}\right\|^2\Bigg]\cr
 \leq & \sup_{k\ge 0\atop i\in \Lambda}\E\Bigg[\E\Bigg[\bigg\|\prod_{j=i+1}^k(I_{\X_1}
 -F(j))\bigg\|^4\bigg|\mathcal F(i)\Bigg]^{\frac{1}{2}}\left\|\frac{G(i)u(i)}{\left(\E\left[\|G(i)u(i)
 \|^2\right]\right)^{\frac{1}{2}}}\right\|^2\Bigg]\cr
 \leq & \sup_{k\ge 0\atop i\in \Lambda}\E\left[\sqrt{\prod_{j=i+1}^k(1+\gamma(j))}\left\|\frac{G(i)u(i)}{\left(\E\left[\|G(i)u(i)\|^2\right]\right)^{\frac{1}{2}}}\right\|^2\right]\cr
 \leq & \sqrt{\prod_{k=0}^{\infty}(1+\gamma(k))}\sup_{i\in \Lambda}\E\left[\left\|\frac{G(i)u(i)}{\left(\E\left[\|G(i)u(i)\|^2\right]\right)^{\frac{1}{2}}}\right\|^2\right]\cr
 =&\sqrt{\prod_{k=0}^{\infty}(1+\gamma(k))}<\infty.
\end{align}
By the condition (\ref{ssafe}-\ref{ssafe1}), (\ref{wwml})-(\ref{llnnv}) and Lemma \ref{lemma6}, we obtain
\begin{align}\label{llks}
 \lim_{k\to \infty}\sum_{i\in \Lambda}& \E\left[\|G(i)\|^2\right] \E\Bigg[\bigg\| \prod_{j=i+1}^k(I_{\X_1}-F(j))\frac{G(i)u(i)}{\left(\E\left[\|G(i)u(i)\|^2\right]\right)^{\frac{1}{2}}}
 \bigg\|^2\Bigg]=0.
\end{align}
Thus, substituting (\ref{oowf}) and (\ref{llks}) into (\ref{wiiie}) gives $\lim_{k\to\infty}\E[\|x(k)\|^2]=0$.
\end{proof}

To prove Lemma \ref{jihubiranshoulian}, we need the following lemma.
\begin{lemma}\label{yibanxingdejieguo}
For the algorithm (\ref{algorithm}), suppose that Assumptions \ref{assumption1}, \ref{assumption2}, Conditions \ref{condition1} and \ref{condition2} hold. If there exists an integer $h>0$ and a constant $\rho_0>0$ such that\\
%\begin{longlist}
\indent (i) $$ \left\{I_{\X^N}-\sum_{i=kh}^{(k+1)h-1}(a(i)\H^*(i)\H(i)+b(i)\L_{\G}\otimes I_{\X}),k\ge 0\right\}$$
is $L_2^2$-stable  w.r.t.  $\{\F((k+1)h-1), k\ge 0\}$;\\
\indent (ii) $\displaystyle \sup_{k\ge 0}\left(\E\left.\left[\|\H^*(k)\H(k)\|_{\LL\left(\X^N\right)}^{2^{\max\{h,2\}}}\right|\F(k-1)\right]\right)^{\frac{1}{2^{\max\{h,2\}}}}\leq \rho_0~\text{a.s.,}$\\
%\end{longlist}
then $\{I_{\X^N}-a(k)\H^*(k)\H(k)-b(k)\L_{\G}\otimes I_{\X},k\ge 0\}$ is $L_2^2$-stable w.r.t. $\{\F(k),k\ge 0\}$,
\end{lemma}
%\textbf{\emph{Proof of Lemma \ref{yibanxingdejieguo}:}}
\begin{proof}
Given the $L_2$-bounded adaptive sequence $\{x(k),\F(kh-1),k\ge 0\}$ with values in the Hilbert space $\X^N$ and the nonnegative integer $m$, we define a new sequence $\{u(k),k\ge 0\}$ by
\bna\label{diedaishi}
u(k+1)=\Phi_P((k+1)h-1,kh)u(k),~k\ge m,
\ena
where $u(m)=x(m),u(i)=0,i=0,\cdots,m-1$. It follows from Proposition \ref{nlllwwieiie}.(a)-(c) that $\{u(k),k\ge 0\}$ is a random sequence with values in the Hilbert space $(\X^N,\tau_{\text{N}}(\X^N))$. On one hand, from (\ref{diedaishi}), by iterative calculations, we get
\bna\label{wffe}
u(k+1) =& \left(\prod_{i=m}^k\Phi_P((i+1)h-1,ih)\right)u(m)\notag\\
 =& \Phi_P((k+1)h-1,mh)x(m),~k\ge m.
\ena
Noting that $x(m)\in L^0(\Omega,\F(mh-1);\X^N)$, it is known from Lemma \ref{lemma1} that
\ban
\E\left[\|u(k+1)\|^2\right]=
\E\left.\left[\E\left[\|\Phi_P((k+1)h-1,mh)x(m)\|^2\right|\F(mh-1)\right]\right]\leq d_1\E\left[\|x(m)\|^2\right],
\ean
thus, $\sup_{k\ge 0}\E[\|x(k)\|^2]<\infty$ implies $\sup_{k\ge 0}\E[\|u(k)\|^2]<\infty$. On the other hand, we can rewrite (\ref{diedaishi}) as
\ban
&&u(i+1)=\left(I_{\X^N}-\sum_{s=ih}^{(i+1)h-1}(a(s)\H^*(s)\H(s)+b(s)\L_{\G}\otimes I_{\X})\right)u(i)\cr
&&+\left(\Phi_P((i+1)h-1,ih)-\left(I_{\X^N}-\sum_{s=ih}^{(i+1)h-1}\left(a(s)\H^*(s)\H(s)+b(s)\L_{\G}\otimes I_{\X}\right)\right)\right)u(i),
\ean
which leads to
\bna\label{fwwii}
u(k+1)&=&\left(\prod_{i=m}^k\left(I_{\X^N}-\sum_{s=ih}^{(i+1)h-1}(a(s)\H^*(s)\H(s)+b(s)\L_{\G}\otimes I_{\X})\right)\right)x(m)\cr
&&+\sum_{i=m}^k\left(\prod_{j=i+1}^k\left(I_{\X^N}-\sum_{s=jh}^{(j+1)h-1}(a(s)\H^*(s)\H(s)+b(s)\L_{\G}\otimes I_{\X})\right)\right)\cr
&&\times \Bigg(\Phi_P((i+1)h-1,ih)-\Bigg(I_{\X^N}-\sum_{s=ih}^{(i+1)h-1}(a(s)\H^*(s)\H(s)\cr
&&+b(s)\L_{\G}\otimes I_{\X})\Bigg)\Bigg)u(i).
\ena
Denote the $s$-th order term in the binomial expansion of $\Phi_P((i+1)h-1,ih)$ by $M_s(i)$, $s=2,\cdots,h$. By (\ref{wffe})-(\ref{fwwii}), we have
\ban
&&~~~~\Phi_P((k+1)h-1,mh)x(m)\cr
&&=\left(\prod_{i=m}^k\left(I_{\X^N}-\sum_{s=ih}^{(i+1)h-1}(a(s)\H^*(s)\H(s)+b(s)\L_{\G}\otimes I_{\X})\right)\right)x(m)\cr
&&+\sum_{i=m}^k\left(\prod_{j=i+1}^k\left(I_{\X^N}-\sum_{s=jh}^{(j+1)h-1}(a(s)\H^*(s)\H(s)+b(s)\L_{\G}\otimes I_{\X})\right)\right)\left(\sum_{s=2}^hM_s(i)\right)u(i),
\ean
from which we get
\bna\label{ikddw}
&&~~~~\E\left[\left\|\Phi_P((k+1)h-1,mh)x(m)\right\|^2\right]\cr
&&\leq 2\E\left[\left\|\left(\prod_{i=m}^k\left(I_{\X^N}-\sum_{s=ih}^{(i+1)h-1}(a(s)\H^*(s)\H(s)+b(s)\L_{\G}\otimes I_{\X})\right)\right)x(m)\right\|^2\right]\cr &&~~~~+2\E\left[\left\|\sum_{i=m}^k\left(\prod_{j=i+1}^k\left(I_{\X^N}-\sum_{s=jh}^{(j+1)h-1}(a(s)\H^*(s)\H(s)+b(s)\L_{\G}\otimes I_{\X})\right)\right)\right.\right.\cr &&~~~~\times\left.\left.\left(\sum_{s=2}^hM_s(i)\right)u(i)\right\|^2\right].\cr
&&\,
\ena
Noting that $x(m)\in L^0(\Omega,\F(mh-1);\X^N)$ and $\{I_{\X^N}-\sum_{i=kh}^{(k+1)h-1}(a(i)\H^*(i)\H(i)+b(i)\L_{\G}\otimes I_{\X}),k\ge 0\}$ is $L_2^2$-stable with respect to the filter $\{\F((k+1)h-1),k\ge 0\}$, we obtain
\ban
\lim_{k\to\infty}\E\left[\left\|\left(\prod_{i=m}^k\left(I_{\X^N}-\sum_{s=ih}^{(i+1)h-1}(a(s)\H^*(s)\H(s)+b(s)\L_{\G}\otimes I_{\X})\right)\right)x(m)\right\|^2\right]=0.
\ean
Hereafter, we will analyze the second term on the right-hand side of the inequality in (\ref{ikddw}). Denote
$
D(s)=a(s)\H^*(s)\H(s)+b(s)(\L_{\G}\otimes I_{\X})$. On one hand, for $2\leq r\leq 2^h,ih\leq s\leq (i+1)h-1$, by Cr-inequality and the conditional Lyapunov inequality, we have
\bna\label{jssk}
&&~~~~\E\left.\left[\|D(s)\|^r\right|\F(ih-1)\right]\cr
&&\leq \left(\E\left.\left[\|a(s)\H^*(s)\H(s)+b(s)(\L_{\G}\otimes I_{\X})\|^{2^h}\right|\F(ih-1)\right]\right)^{\frac{r}{2^h}}\cr
&&\leq \max\{a(s),b(s)\}^r\left(2^{2^h-1}\E\left.\left[\|\H^*(s)\H(s)\|^{2^h}\right|\F(ih-1)\right]\right.\cr
&&\hspace{0.5cm}+\left. 2^{2^h-1}\|\L_{\G}\otimes I_{\X}\|^{2^h}\right)^{\frac{r}{2^h}}\cr
&&\leq2^r\left(a^r(s)+b^r(s)\right)\left(\left(\E\left.\left[\|\H^*(s)\H(s)\|^{2^h}\right|\F(ih-1)\right]\right)^{\frac{r}{2^h}}+\|\L_{\G}\otimes I_{\X}\|^{r}\right)\cr
&&\leq 2^r(a^r(s)+b^r(s))(\rho_0^r+\|\L_{\G}\otimes I_{\X}\|^r)
 \leq 2^r(a(s)+b(s))^r(\rho_0+\|\L_{\G}\otimes I_{\X}\|)^r~\text{a.s.}
\ena
Denote $\rho_1=\rho_0+\|\L_{\G}\otimes I_{\X}\|$. For $ih\leq n_1<\cdots< n_r\leq (i+1)h-1$, by the conditional H\"{o}lder inequality, the conditional Lyapunov inequality and (\ref{jssk}), we get
\bna\label{ijjw}
&&~~~~\E\left.\left[\left\|\prod_{j=1}^rD(n_j)\right\|^2\right|\F(ih-1)\right]\cr
&&\leq \left(\E\left.\left[\left\|\prod_{j=1}^{r-1}D(n_j)\right\|^4\right|\F(ih-1)\right]\right)^{\frac{1}{2}}\left(\E\left.\left[\left\|D(n_r)\right\|^4\right|\F(ih-1)\right]\right)^{\frac{1}{2}}\cr
&&\leq \rho_1^2(a(n_r)+b(n_r))^2\left(\E\left.\left[\left\|\prod_{j=1}^{r-1}D(n_j)\right\|^4\right|\F(ih-1)\right]\right)^{\frac{1}{2}}\cr
&&\leq \rho_1^{2r}\prod_{j=1}^r(a(n_j)+b(n_j))^2~\text{a.s.}
\ena
On the other hand, for $2\leq s\leq h$, it follows from Condition \ref{condition1} and (\ref{ijjw}) that
\bna\label{wwffw}
&&~~~~\E\left.\left[\|M_s(i)\|^2\right|\F(ih-1)\right]\cr
&&=\E\left.\left[\left\|\sum_{ih\leq n_1< \cdots< n_s\leq (i+1)h-1}\prod_{j=1}^sD(n_j)\right\|^2\right|\F(ih-1)\right]\cr
&&\leq \mathbb{C}_h^s\sum_{ih\leq n_1< \cdots< n_s\leq (i+1)h-1}\E\left.\left[\left\|\prod_{j=1}^sD(n_j)\right\|^2\right|\F(ih-1)\right]\cr
&&\leq \mathbb{C}_h^s\sum_{ih\leq n_1< \cdots< n_s\leq (i+1)h-1}\rho_1^{2s}\prod_{j=1}^s(a(n_j)+b(n_j))^2\cr
&&\leq \mathbb{C}_h^s\sum_{ih\leq n_1< \cdots< n_s\leq (i+1)h-1}\rho_1^{2s}\prod_{j=1}^s(a(i)+b(i))^2\cr
&&= \left(\mathbb{C}_h^s\right)^2\rho_1^{2s}(a(i)+b(i))^{2s}\cr
&&\leq \mathbb{C}_h^s\rho_1^{2s}(a(i)+b(i))^{2s}h!~\text{a.s.},
\ena
where the last inequality is obtained from $\mathbb{C}_h^s\leq h!$. Noting that $\{a(k),k\ge 0\}$ and $\{b(k),k\ge 0\}$ both monotonically vanish, it follows that there exists a constant $c_0>0$, such that $\sup_{k\ge 0}(a(k)+b(k))\leq c_0$. Then, by (\ref{wwffw}), we get
\bna\label{wwkks}
\sum_{s=2}^h\E\left.\left[\|M_s(i)\|^2\right|\F(ih-1)\right]&\leq& 4h!c_0^{-4}\left(a^2(i)+b^2(i)\right)^2\sum_{s=2}^h\mathbb{C}_h^s\rho_1^{2s}c_0^{2s}\cr &=&\rho_2\left(a^2(i)+b^2(i)\right)^2~\text{a.s.},
\ena
where $\rho_2=4h!c_0^{-2}((\rho_1^2c_0^2+1)^h-1-h\rho_1^2c_0^2)$. It follows from (\ref{wwkks}) that
\ban
&&~~~~\E\left[\left\|\left(\sum_{s=2}^hM_s(i)\right)u(i)\right\|^2\right]\cr
&&\leq h\E\left[\sum_{s=2}^h \E\left.\left[\|M_s(i)\|^2\right|\F(ih-1)\right]\|u(i)\|^2\right]\cr
&&\leq h\rho_2\left(a^2(i)+b^2(i)\right)^2\E\left[\|u(i)\|^2\right],
\ean
which together with $\sup_{i\ge 0}\E[\|u(i)\|^2]<\infty$ leads to
\begin{align}\label{wwooo}
&\sup_{i\ge 0}\E\left[\left\|R(i)\right\|^2\right]\cr
\leq & h\rho_2\sup_{i\ge 0}\E\left[\|u(i)\|^2\right]
<    \infty,
\end{align}
where
\ban
R(i)=\frac{1}{a^2(i)+b^2(i)}\left(\sum_{s=2}^hM_s(i)\right)u(i),~i\ge m.
\ean
By the Minkowski inequality, we obtain
\bna\label{foow}
&&\hspace{-0.2cm}\E\left[\left\|\sum_{i=m}^k\left(\prod_{j=i+1}^k\left(I_{\X^N}-\sum_{s=jh}^{(j+1)h-1}D(s)\right)\right)\left(\sum_{s=2}^hM_s(i)\right)u(i)\right\|^2\right]\cr
&&\hspace{-0.8cm}=\E\left[\left\|\sum_{i=m}^k\left(a^2(i)+b^2(i)\right)\left(\prod_{j=i+1}^k\left(I_{\X^N}-\sum_{s=jh}^{(j+1)h-1}D(s)\right)\right)R(i)\right\|^2\right]\cr
&&\hspace{-0.8cm}\leq \Bigg(\sum_{i=m}^k\left(a^2(i)+b^2(i)\right)\Bigg(\E\Bigg[\Bigg\|\Bigg(\prod_{j=i+1}^k\Bigg(I_{\X^N}-\sum_{s=jh}^{(j+1)h-1}D(s)\Bigg)\Bigg) R(i)\Bigg\|^2\Bigg]\Bigg)^{\frac{1}{2}}\Bigg)^2.
\ena
From Proposition \ref{wenknknkn}, we know that $R(i)\in L^0(\Omega,\F((i+1)h-1);\X^N)$, thus, by (\ref{wwooo}) and Lemma \ref{hhhlemma}, we know that there exists a constant $d_3>0$ such that
\bna\label{oxcnv}
&&~~~~\sup_{k\ge 0\atop i\ge 0 }\E\left[\left\|\left(\prod_{j=i+1}^k\left(I_{\X^N}-\sum_{s=jh}^{(j+1)h-1}D(s)\right)\right)R(i)\right\|^2\right]\cr
&&\leq d_3\sup_{i\ge 0}\E\left[\|R(i)\|^2\right]
 <\infty,
\ena
which together with that the operator-valued random sequence $$\left\{I_{\X^N}-\sum_{i=kh}^{(k+1)h-1}(a(i)\H^*(i)\H(i)+b(i)\L_{\G}\otimes I_{\X}),k\ge 0\right\}$$ is $L_2^2$-stable w.r.t. $\{\F((k+1)h-1),k\ge 0\}$ gives
\bna\label{xawq}
\lim_{k\to \infty}\E\left[\left\|\left(\prod_{j=i+1}^k\left(I_{\X^N}-\sum_{s=jh}^{(j+1)h-1}D(s)\right)\right)R(i)\right\|^2\right]=0,~\forall\ i\ge 0.
\ena
By Condition \ref{condition2}, (\ref{oxcnv})-(\ref{xawq}) and Lemma \ref{lemma6}, we have
\ban
\lim_{k\to\infty}\sum_{i=m}^k\left(a^2(i)+b^2(i)\right)\left(\E\left[\left\|\left(\prod_{j=i+1}^k\left(I_{\X^N}-\sum_{s=jh}^{(j+1)h-1}D(s)\right)\right)R(i)\right\|^2\right]\right)^{\frac{1}{2}}=0.
\ean
Given the $L_2$-bounded adaptive sequence $\{x(k),\F(kh-1),k\ge 0\}$ with values in the Hilbert space $\X^N$, from (\ref{ikddw}) and (\ref{foow}), we know that
\bna\label{rscsc}
\lim_{k\to \infty}\E\left[\left\|\Phi_P((k+1)h-1,mh)x(m)\right\|^2\right]=0,~\forall\ m\ge 0.
\ena
For $j\in \mathbb N$, denote $m_j=\lfloor \frac{j}{h} \rfloor,\widetilde{m}_j=\lceil \frac{j}{h} \rceil$. Let $\{y(k),\F(k),k\ge 0\}$ be a $L_2$-bounded adaptive sequence with values in the Hilbert space $\X^N$. For $0\leq i<k-3h$, noting that $0\leq k-m_kh<h$, from Propositions \ref{nlllwwieiie}-\ref{wenknknkn}, it is known that $\Phi_P(m_kh-1,\widetilde{m}_{i+1}h)\Phi_P(\widetilde{m}_{i+1}h-1,i+1)y(i)\in L^0(\Omega,\F(m_kh-1);\X^N)$, by Lemma \ref{lemma1}, we know that there exists a constant $d_2>0$ such that
\bna\label{jjkkl}
&&~~~~\mathbb E\left[\|\Phi_P(k,i+1)y(i)\|^2\right]\cr
&&=\mathbb E\left[\|\Phi_P(k,m_kh)\Phi_P(m_kh-1,\widetilde{m}_{i+1}h)\Phi_P(\widetilde{m}_{i+1}h-1,i+1)y(i)\|^2\right]\cr
&&=\mathbb E\Big[\mathbb E\Big[\Big\|\Phi_P(k,m_kh)\Phi_P(m_kh-1,\widetilde{m}_{i+1}h)\cr
&&~~~~\times \Phi_P(\widetilde{m}_{i+1}h-1,i+1)y(i)\Big\|^2\Big|\mathcal F(m_kh-1)\Big]\Big]\cr
&&\leq d_2\mathbb E\left[\|\Phi_P(m_kh-1,\widetilde{m}_{i+1}h)\Phi_P(\widetilde{m}_{i+1}h-1,i+1)y(i)\|^2\right],~0\leq i<k-3h.
\ena
Noting that $0\leq \widetilde{m}_{i+1}h-(i+1)<h$ and $y(i)\in \F(i)$, it follows from Lemma \ref{lemma1} that
\bna\label{cllll}
&&~~~\sup_{i\ge 0}\mathbb E\left[\|\Phi_P(\widetilde{m}_{i+1}h-1,i+1)y(i)\|^2\right]\cr
&&=\sup_{i\ge 0}\E\left[\mathbb E\left.\left[\|\Phi_P(\widetilde{m}_{i+1}h-1,i+1)y(i)\|^2\right|\mathcal F(i)\right]\right]\cr
&&\leq d_2\sup_{i\ge 0}\E\left[\|y(i)\|^2\right]
 <\infty.
\ena
By Propositions \ref{nlllwwieiie}-\ref{wenknknkn}, we get $\Phi_P(\widetilde{m}_{i+1}h-1,i+1)y(i)\in L^0(\Omega,\mathcal F(\widetilde{m}_{i+1}h-1);\X^N)$. Substituting (\ref{rscsc}) and (\ref{cllll}) into (\ref{jjkkl}) gives
\ban
\lim_{k\to\infty}\mathbb E\left[\|\Phi_P(k,i+1)y(i)\|^2\right]=0,~\forall\ i\ge 0,
\ean
which implies that the operator-valued random sequence $\{I_{\X^N}-a(k)\H^*(k)\H(k)-b(k)\L_{\G}\otimes I_{\X},k\ge 0\}$ is $L_2^2$-stable w.r.t. $\{\F(k),k\ge 0\}$.
\end{proof}


%\textbf{\emph{Proof of Lemma \ref{jihubiranshoulian}:}}
\begin{proof}[Proof of Lemma \ref{jihubiranshoulian}]
It follows from Condition \ref{condition3} that there exists a constant $C_1>0$, such that $|b(k)-a(k)|\leq C_1(a^2(k)+b^2(k))$, which gives
\bna\label{xmkmslff}
\sum_{i=0}^ka(i)&\leq& \sum_{i=0}^k(|a(i)-b(i)|+b(i))\cr
&\leq& C_1\sum_{i=0}^k\left(a^2(i)+b^2(i)\right)+\sum_{i=0}^kb(i),~k\ge 0.
\ena
For the integer $h>0$, denote
\bna\label{xmlwf}
c(k)=\sum_{s=kh}^{(k+1)h-1}b(s).
\ena
By (\ref{xmkmslff}) and Condition \ref{condition2}, we get
\ban
\sum_{k=0}^{\infty}c(k)=\sum_{k=0}^{\infty}\sum_{s=kh}^{(k+1)h-1}b(s)=\sum_{k=0}^{\infty}b(k)=\infty.
\ean
Denote
\bna\label{mvnjwe}
\HH=\text{diag}\left\{\frac{1}{h}\HH_1,\cdots,\frac{1}{h}\HH_N\right\}+\L_{\G}\otimes I_{\X}.
\ena
Noting that $\L_{\G}$ is positive semi-definite and $\sum_{i=1}^{N}\HH_i>0$, by Lemma \ref{lemmaA10}, we know that $\HH\in \mathscr L(\X^N)$ is a strictly positive self-adjoint operator. Let $\{x(k),\F(kh-1),k\ge 0\}$ be a $L_2$-bounded adaptive sequence with values in the Hilbert space $\X^N$, then we can write  $x(k)=(x_1(k),\cdots,x_N(k))$, where $x_i(k):\Omega\to \X$, $i=1,\cdots,N$ are the random elements with values in the Hilbert space $(\X,\tau_{\text{N}}(\X))$. Denote
\bna\label{cnlweee}
\mu(i) = c(i)\HH x(i)-\sum_{s=ih}^{(i+1)h-1}(a(s)\E[\H^*(s)\H(s)x(i)|\F(ih-1)] +b(s)(\L_{\G}\otimes I_{\X})x(i)).
\ena
By (\ref{xmlwf})-(\ref{cnlweee}), we get
\ban
&&~~~~\mu(i)\cr
&&=\sum_{s=ih}^{(i+1)h-1}b(s)\text{diag}\left\{\frac{1}{h}\HH_1,\cdots,\frac{1}{h}\HH_N\right\}x(i)-\sum_{s=ih}^{(i+1)h-1}a(s)\E[\H^*(s)\H(s)x(i)|\F(ih-1)]\cr
&&=\text{diag}\left\{\frac{1}{h}\sum_{s=ih}^{(i+1)h-1}b(s)\HH_1x_1(i)-\sum_{s=ih}^{(i+1)h-1}a(s)\E[H_1^*(s)H_1(s)x_1(i)|\F(ih-1)],\right.\cr
&&~~~~~~\left.\cdots,\frac{1}{h}\sum_{s=ih}^{(i+1)h-1}b(s)\HH_Nx_N(i)-\sum_{s=ih}^{(i+1)h-1}a(s)\E[H_N^*(s)H_N(s)x_N(i)|\F(ih-1)]\right\}.
\ean
It follows from Lemma \ref{lemmaA11} that there exists a constant $C_2>0$ such that
\bna\label{vklwmlmfm}
\max_{ih\leq s\leq (i+1)h-1}\Bigg(\frac{1}{h}\Bigg(\sum_{s=ih}^{(i+1)h-1}b(s)\Bigg)-a(s)\Bigg)^2\leq C_2\left(a^4(i)+b^4(i)\right).
\ena
Therefore, by Conditions \ref{condition1}-\ref{condition3}, the condition (\ref{yinlitiaojian2}) and (\ref{vklwmlmfm}), we have
\begin{align}\label{yuzhouwudichang}
 &\E\left[\|\mu(i)\|^2\right]\cr
 =& \sum_{j=1}^N\E\Bigg[\Bigg\|\frac{1}{h}\sum_{s=ih}^{(i+1)h-1}b(s)\HH_jx_j(i)
 -\sum_{s=ih}^{(i+1)h-1}a(s)\E[H_j^*(s)H_j(s)x_j(i)|\F(ih-1)]\Bigg\|^2\Bigg]\cr
 =&\sum_{j=1}^N\E\Bigg[\Bigg\|\Bigg(\frac{1}{h}\sum_{s=ih}^{(i+1)h-1}b(s)\Bigg)
 \Bigg(\HH_j x_j(i)-\sum_{s=ih}^{(i+1)h-1}\E\left[H_j^*(s)H_j(s)x_j(i)|\F(ih-1)\right]\Bigg)\cr & +\sum_{s=ih}^{(i+1)h-1}\Bigg(\frac{1}{h}\Bigg(\sum_{s=ih}^{(i+1)h-1}b(s)\Bigg)-a(s)\Bigg)\E\left[H_j^*(s)H_j(s)x_j(i)|\F(ih-1)\right]\Bigg\|^2\Bigg]\cr
%&&\leq \sum_{j=1}^N\left(2\E\left[\left\|\left(\frac{1}{h}\sum_{s=ih}^{(i+1)h-1}b(s)\right)\left(H_jx_j(i)-\sum_{s=ih}^{(i+1)h-1}\E\left.\left[H_j^*(s)H_j(s)x_j(i)\right|\F(ih-1)\right]\right)\right\|^2\right]\right.\cr &&~~+\left.2\E\left[\left\|\sum_{s=ih}^{(i+1)h-1}\left(\frac{1}{h}\left(\sum_{s=ih}^{(i+1)h-1}b(s)\right)-a(s)\right)\E\left.\left[H_j^*(s)H_j(s)x_j(i)\right|\F(ih-1)\right]\right\|^2\right]\right)\cr
 \leq & \sum_{j=1}^N\Bigg(2b^2(i)\E\Bigg[\Bigg\|\HH_jx_j(i)-\sum_{s=ih}^{(i+1)h-1}\E\left.\left[H_j^*(s)H_j(s)x_j(i)\right|\F(ih-1)\right]\Bigg\|^2\Bigg]\cr
& +2h\E\Bigg[\sum_{s=ih}^{(i+1)h-1}
\Bigg(\frac{1}{h}\Bigg(\sum_{s=ih}^{(i+1)h-1}b(s)\Bigg)-a(s)\Bigg)^2 \left\|\E\left[H_j^*(s)H_j(s)x_j(i)|\F(ih-1)\right]\right\|^2\Bigg]\Bigg)\cr
 \leq & \sum_{j=1}^N\left(2b^2(i)\E\left[\left\|\HH_jx_j(i)-\sum_{s=ih}^{(i+1)h-1}\E\left.\left[H_j^*(s)H_j(s)x_j(i)\right|\F(ih-1)\right]\right\|^2\right]\right.\cr
& +2h\E\Bigg[\sum_{s=ih}^{(i+1)h-1}\Bigg(\frac{1}{h}\Bigg(\sum_{s=ih}^{(i+1)h-1}b(s)\Bigg)-a(s)\Bigg)^2\cr & \times\E\left[\left\|H_j^*(s)H_j(s)\right\|^2|\F(ih-1)\right]\left\|x_j(i)\right\|^2\Bigg]\Bigg)\cr
 \leq & \sum_{j=1}^N\left(2b^2(i)\E\left[\left\|\HH_jx_j(i)-\sum_{s=ih}^{(i+1)h-1}\E\left.\left[H_j^*(s)H_j(s)x_j(i)\right|\F(ih-1)\right]\right\|^2\right]\right.\cr
& +\left.2h\rho^2_0\sup_{k\ge 0}\E\left[\|x(k)\|^2\right]\sum_{s=ih}^{(i+1)h-1}\left(\frac{1}{h}\left(\sum_{s=ih}^{(i+1)h-1}b(s)\right)-a(s)\right)^2\right)\cr
 \leq & \sum_{j=1}^N\left(2b^2(i)\E\left[\left\|\HH_jx_j(i)-\sum_{s=ih}^{(i+1)h-1}\E\left.\left[H_j^*(s)H_j(s)x_j(i)\right|\F(ih-1)\right]\right\|^2\right]\right.\cr
& +\left.2h^2\rho^2_0\sup_{k\ge 0}\E\left[\|x(k)\|^2\right]\max_{ih\leq s\leq (i+1)h-1}\left(\frac{1}{h}\left(\sum_{s=ih}^{(i+1)h-1}b(s)\right)-a(s)\right)^2\right)\cr
 \leq & \sum_{j=1}^N2b^2(i)\E\left[\left\|\HH_jx_j(i)-\sum_{s=ih}^{(i+1)h-1}\E\left.\left[H_j^*(s)H_j(s)x_j(i)\right|\F(ih-1)\right]\right\|^2\right]\cr
& +2C_2Nh^2\rho^2_0\sup_{k\ge 0}\E\left[\|x(k)\|^2\right]\left(a^4(i)+b^4(i)\right).
\end{align}
Noting that $\{x_j(k),\F(kh-1),k\ge 0\},j=1,\cdots,N$ are the $L_2$-bounded adaptive sequences with values in the Hilbert space $\X$, by (\ref{yinlitiaojian1}), we obtain
\bna\label{vnkwjofjeofe}
&&\sum_{i=0}^{\infty}\E\left[\left\|\HH_jx_j(i)-\sum_{s=ih}^{(i+1)h-1}
\E\left.\left[H_j^*(s)H_j(s)x_j(i)\right|\F(ih-1)\right]\right\|^2\right]\cr
&<&\infty,~j=1,\cdots,N.
\ena
By Condition \ref{condition2}, (\ref{yuzhouwudichang})-(\ref{vnkwjofjeofe}) and Cauchy inequality, we get
\ban
&&~~~~\sum_{i=0}^{\infty}\E\left[\|\mu(i)\|^2\right]^{\frac{1}{2}}\cr
&&\leq \sqrt{2}\sum_{i=0}^{\infty}b(i)\sum_{j=1}^N\left(\E\left[\left\|\HH_jx_j(i)-\sum_{s=ih}^{(i+1)h-1}\E\left.\left[H_j^*(s)H_j(s)x_j(i)\right|\F(ih-1)\right]\right\|^2\right]\right)^{\frac{1}{2}}\cr
&&~~~~+h\rho_0\sup_{k\ge 0}\E\left[\|x(k)\|^2\right]^{\frac{1}{2}}\sqrt{2C_2N}\sum_{i=0}^{\infty}\left(a^2(i)+b^2(i)\right)\cr
%&&\leq \sqrt{2}C_3\left(\sum_{i=0}^{\infty}\left(\sum_{j=1}^N\left(\E\left[\left\|H_jx_j(i)-\sum_{s=ih}^{(i+1)h-1}\E\left.\left[H_j^*(s)H_j(s)x_j(i)\right|\F(ih-1)\right]\right\|^2\right]\right)^{\frac{1}{2}}\right)^2\right)\cr
%&&+h\rho_0\sup_{k\ge 0}\E\left[\|x(k)\|^2\right]^{\frac{1}{2}}\sqrt{2C_2N}\sum_{i=0}^{\infty}\left(a^2(i)+b^2(i)\right)\cr
&&\leq \sqrt{2}NC_3\left(\sum_{i=0}^{\infty}\sum_{j=1}^N\E\left[\left\|\HH_jx_j(i)-\sum_{s=ih}^{(i+1)h-1}\E\left.\left[H_j^*(s)H_j(s)x_j(i)\right|\F(ih-1)\right]\right\|^2\right]\right)\cr
&&~~~~+h\rho_0\sup_{k\ge 0}\E\left[\|x(k)\|^2\right]^{\frac{1}{2}}\sqrt{2C_2N}\sum_{i=0}^{\infty}\left(a^2(i)+b^2(i)\right)\cr
&&= \sqrt{2}NC_3\left(\sum_{j=1}^N\sum_{i=0}^{\infty}\E\left[\left\|\HH_jx_j(i)-\sum_{s=ih}^{(i+1)h-1}\E\left.\left[H_j^*(s)H_j(s)x_j(i)\right|\F(ih-1)\right]\right\|^2\right]\right)\cr
&&~~~~+h\rho_0\sup_{k\ge 0}\E\left[\|x(k)\|^2\right]^{\frac{1}{2}}\sqrt{2C_2N}\sum_{i=0}^{\infty}
\left(a^2(i)+b^2(i)\right)
  <\infty,
\ean
where $C_3=\sum_{i=0}^{\infty}b^2(i)$. For any given integer $m>0$, denote $\Gamma_{m}=\{i\ge m:\E[\|\mu(i)\|^2]  >0,i\in \mathbb N\}$. Noting that $\E[\|\mu(i)\|^2]=0$ implies that $\mu(i)=0~\text{a.s.}$, we obtain
\bna\label{final2}
&&~~~~\sum_{i=m}^{k}\left(\E\left[\left\|\left(\prod_{j=i+1}^k(I_{\X^N}-c(j)\HH)\right)\mu(i)\right\|^2\right]\right)^{\frac{1}{2}}\cr
&&\leq \sum_{i=m}^{\infty}\left(\E\left[\left\|\left(\prod_{j=i+1}^k(I_{\X^N}-c(j)\HH)\right)\mu(i)\right\|^2\right]\right)^{\frac{1}{2}}\cr
&&=\sum_{i\in\Gamma_m}\left(\E\left[\left\|\left(\prod_{j=i+1}^k(I_{\X^N}-c(j)\HH)\right)\mu(i)\right\|^2\right]\right)^{\frac{1}{2}}\cr
&&=\sum_{i\in\Gamma_m}\E\left[\|\mu(i)\|^2\right]^{\frac{1}{2}}\left(\E\left[\left\|\left(\prod_{j=i+1}^k(I_{\X^N}-c(j)\HH)\right)\eta(i)\right\|^2\right]\right)^{\frac{1}{2}},
\ena
where $\eta(i)=\mu(i)\E[\|\mu(i)\|^2]^{-\frac{1}{2}},i\in \Gamma_m$. Noting that $\E[\|\eta(i)\|^2]=1$, it follows from Lemma \ref{lemmaA7} that there exist constants $M,d>0$, such that
\bna\label{final3}
&&~~~~\sup_{k\ge 0\atop i\in \Gamma_m}\left(\E\left[\left\|\left(\prod_{j=i+1}^k(I_{\X^N}-c(j)\HH)\right)\eta(i)\right\|^2\right]\right)^{\frac{1}{2}}\cr
&&\leq M^d\sup_{i\in \Gamma_m}\E\left[\|\eta(i)\|^2\right]^{\frac{1}{2}}
   =M^d.
\ena
By Lemma \ref{lemmaA7} and Lebesgue dominated convergence theorem, we get
\bna\label{final4}
\lim_{k\to \infty}\left(\E\left[\left\|\left(\prod_{j=i+1}^k(I_{\X^N}-c(j)\HH)\right)\eta(i)\right\|^2\right]\right)^{\frac{1}{2}}=0,~\forall\ i\ge 0.
\ena
Therefore, combining (\ref{final2})-(\ref{final4}) and Lemma \ref{lemma6} leads to
\ban
\lim_{k\to\infty}\sum_{i=m}^{k}\left(\E\left[\left\|\left(\prod_{j=i+1}^k(I_{\X^N}-c(j)\HH)\right)\mu(i)\right\|^2\right]\right)^{\frac{1}{2}}=0,~\forall\ m\ge 0.
\ean
It follows from Lemma \ref{henandelemma} that the operator-valued random sequence $$ \left\{I_{\X^N}-\sum_{i=kh}^{(k+1)h-1}(a(i)\H^*(i)\H(i)+b(i)\L_{\G}\otimes I_{\X}),k\ge 0\right\}$$ is $L_2^2$-stable w.r.t. $\{\F((k+1)h-1),k\ge 0\}$. Hence, from Lemma \ref{yibanxingdejieguo}, it is known that the operator-valued random sequence $\{I_{\X^N}-a(k)\H^*(k)\H(k)-b(k)\L_{\G}\otimes I_{\X},k\ge 0\}$ is $L_2^2$-stable w.r.t. $\{\F(k),k\ge 0\}$.
\end{proof}


%\textbf{\emph{Proof of Theorem \ref{vnknoklfl}:}}
%\begin{proof}[Proof of Theorem \ref{vnknoklfl}]
%By the conditions (\ref{yinlitiaojian1})-(\ref{yinlitiaojian2}) and Lemma \ref{jihubiranshoulian}, it is known that $\{I_{\X^N}-a(k)\H^*(k)\H(k)-b(k)\L_{\G}\otimes I_{\X},k\ge 0\}$ is $L_2^2$-stable w.r.t. $\{\F(k),k\ge 0\}$. Denote $D(k)=a(k)\H^*(k)\H(k)+b(k)\L_{\G}\otimes I_{\X}$. By the condition (\ref{yinlitiaojian2}) that
%\bna\label{cmllemfnn}
%&&~~~~\E\left.\left[\|D(k)\|^r\right|\F(k-1)\right]\cr
%&&\leq \left(\E\left.\left[\|a(k)\H^*(k)\H(k)+b(k)(\L_{\G}\otimes I_{\X})\|^{2^h}\right|\F(k-1)\right]\right)^{\frac{r}{2^h}}\cr
%&&\leq \max\{a(k),b(k)\}^r\Big(2^{2^h-1}\E\left.\left[\|\H^*(k)\H(k)\|^{2^h}\right|\F(k-1)\right]\cr
%&&~~+2^{2^h-1}\|\L_{\G}\otimes I_{\X}\|^{2^h}\Big)^{\frac{r}{2^h}}\cr
%&&\leq 2^r\left(a^r(k)+b^r(k)\right)\left(\left(\E\left.\left[\|\H^*(k)\H(k)\|^{2^h}\right|\F(k-1)\right]\right)^{\frac{r}{2^h}}+\|\L_{\G}\otimes I_{\X}\|^{r}\right)\cr
%&&\leq2^r (a^r(k)+b^r(k))(\rho_0^r+\|\L_{\G}\otimes I_{\X}\|^r)\cr
%&&\leq 2^r(a^r(k)+b^r(k))\rho_1^r~\text{a.s.},~\forall 1\leq r\leq 4,
%\ena
%where $\rho_1=\rho_0+\|\L_{\G}\otimes I_{\X}\|$. By the condition (\ref{dinglitiaojian}) and (\ref{cmllemfnn}), we get
%\bna
%&&~~~\left.\E\left[\|I_{\X^N}-\left(a(k)\H^*(k)\H(k)+b(k)\L_{\G}\otimes I_{\X}\right)\|^4\right|\F(k-1)\right]\cr
%&&=\left.\E\left[\left\|\left(I_{\X^N}-\left(a(k)\H^*(k)\H(k)+b(k)\L_{\G}\otimes I_{\X}\right)\right)^4\right\|\right|\F(k-1)\right]\cr
%&&=\left.\E\left[\left\|I_{\X^N}-4D(k)+6D^2(k)-4D^3(k)+D^4(k)\right\|\right|\F(k-1)\right]\cr
%&&\leq \left.\E\left[\|I_{\X^N}-4D(k)\|+6\|D(k)\|^2+4\|D(k)\|^3+\|D(k)\|^4\right|\F(k-1)\right]\cr
%&&\leq 1+\gamma(k)~\text{a.s.},
%\ena
%where $\gamma(k)=\Gamma(k)+6(a^2(k)+b^2(k))\rho_1^2+4(a^3(k)+b^3(k))\rho_1^3+(a^4(k)+b^4(k))\rho_1^4$. From Condition \ref{condition2} and the condition (\ref{dinglitiaojian}), we know that $\sum_{k=0}^{\infty}\gamma(k)<\infty$, which together with Lemma \ref{dingliyi1} implies that algorithm (\ref{algorithm}) is both mean square and almost surely strongly consistent.
%\end{proof}


%\textbf{\emph{Proof of Corollary \ref{xiaosirendetuilun}:}}


%\textbf{\emph{Proof of Theorem \ref{rkhsdingli}:}}
%\begin{proof}[Proof of Theorem \ref{rkhsdingli}]
%Let $H_i(k)$ be the mapping induced by random input data $x_i(k)$, where
%\ban
%H_i(k)(f)=f(x_i(k)),~f\in \HH_K,~k\ge 0,~i\in \mathcal V.
%\ean
%For any given $f_1,f_2\in \HH_K$ and $c_1,c_2\in \mathbb R$, it follows from the reproducing property of $\HH_K$ that
%\bna\label{vlwlmmfff}
%H_i(k)(c_1f_1+c_2f_2)&=&\langle c_1f_1+c_2f_2,K_{x_i(k)}\rangle _K\cr &=&c_1H_i(k)(f_1)+c_2H_i(k)(f_2),~k\ge 0,~i\in \mathcal V,
%\ena
%thus, $H_i(k)$ is a linear operator. Noting the continuity of Mercer kernel $K:\X\times \X\to \mathbb R$, we know that the function $K_x:\X\to \HH_K$ induced by the Mercer kernel $K$ is also continuous. It is known from  $\HH_K=\overline{\textbf{span}\{K_x,x\in\X\}}$ that $f\in \HH_K$ is a Borel measurable function, which implies that  $H_i(k)(f)=f(x_i(k))$ is a random variable with values in the Hilbert space $(\mathbb R,\tau_{\text{N}}(\mathbb R))$. By Assumption \ref{assumption5} and the reproducing property of $\HH_K$, we have
%\bna\label{vnklooeoeeee}
%\|K_x\|_K=\sqrt{\langle K_x,K_x\rangle _K}=\sqrt{K(x,x)}\leq \sup_{x\in\X}\sqrt{K(x,x)}<\infty.
%\ena
%For any given sample path $\omega\in\Omega$, by (\ref{vnklooeoeeee}), we get
%\bna\label{vnlwkfeemmff}
%|H_i(k)(\omega)(f)|&=&\left|\left\langle f,K_{x_i(k)(\omega)}\right\rangle _K\right|\cr
%&\leq& \sup_{x\in\X}\sqrt{K(x,x)}\|f\|_K,~\forall\ f\in \HH_K,~k\ge 0,~i\in \mathcal V,
%\ena
%then $\|H_i(k)\|_{\LL(\HH_K,\mathbb R)}\leq \sup_{x\in\X}\sqrt{K(x,x)}~\text{a.s.}$. By (\ref{vlwlmmfff}), (\ref{vnlwkfeemmff}) and Proposition \ref{nlllwwieiie}, we know that $H_i(k):\Omega\to\LL(\HH_K,\mathbb R)$ is a random element with values in the topological space $(\LL(\HH_K,\mathbb R),\tau_{\text{S}}(\LL(\HH_K,\mathbb R)))$. Denote $\H(k):=\text{diag}\{H_1(k),\cdots,H_N(k)\}$ and $v(k):=(v_1(k),\\\cdots,v_N(k))$, it follows from Definition \ref{tuopukongjian} that $\H(k)$ is the random element with values in the topological space $(\LL(\HH_K^N,\mathbb R^N),\tau_{\text{S}}(\LL(\HH_K^N,\mathbb R^N)))$, and $v(k)$ is a random vector with values in the Hilbert space $(\mathbb R^N,\tau_{\text{N}}(\mathbb R^N))$. Thus, by Proposition \ref{nlllwwieiie}, we have
%\bna\label{nvwjoijjff}
%\F(k)=\bigvee_{s=0}^k\left(\sigma\left(\H(s);\tau_{\text{S}}\left(\LL\left(\HH_K^N,\mathbb R^N\right)\right)\right)\bigvee \sigma\left(v(s);\tau_{\text{N}}\left(\mathbb R^N\right)\right)\right),~k\ge 0.
%\ena
%It follows from Assumption \ref{assumption3} and (\ref{nvwjoijjff}) that Assumption \ref{assumption1} holds. By Assumption \ref{assumption4} and (\ref{nvwjoijjff}), it is known that $\{v(k),\F(k),k\ge 0\}$ is the martingale difference sequence and there exists a constant $\b_v:=N\b>0$, such that
%\ban
%\sup_{k\ge 0}\E\left.\left[\|v(k)\|^2\right|\F(k-1)\right]\leq N\max_{i\in \mathcal V}\sup_{k\ge 0}\E\left.\left[\|v_i(k)\|^2\right|\F(k-1)\right]\leq \b_v~\text{a.s.},
%\ean
%which implies that Assumption \ref{assumption2} holds.
%Let $x\in \X$, for any given $f_1,f_2\in \HH_K$ and $c_1,c_2\in \mathbb R$, we get
%\bna\label{vmkwmefkeml1}
%\left(K_x\otimes K_x\right)(c_1f_1+c_2f_2)&=&c_1f_1(x)K_x+c_2f_2(x)K_x\cr
%&=&c_1\left(K_x\otimes K_x\right)f_1+c_2\left(K_x\otimes K_x\right)f_2,
%\ena
%from (\ref{vnklooeoeeee}), Assumption \ref{assumption5} and the reproducing property of $\HH_K$, it is known that
%\bna\label{vmkwmefkeml2}
%~~\left\|\left(K_x\otimes K_x\right)f\right\|_{K}\leq \|K_x\|_K^2\|f\|_K\leq \sup_{x\in\X}K(x,x)\|f\|_K,~\forall\ f\in \HH_K,~\forall\ x\in \X.
%\ena
%Thus, it follows from (\ref{vmkwmefkeml1})-(\ref{vmkwmefkeml2}) that $K_x\otimes K_x\in \LL(\HH_K)$. Let $x=\sum_{i=1}^n\1_{A_i}\otimes x_i$ be a random vector with values in the Hilbert space $(\X,\tau_{\text{N}}(\X))$, where $A_i\cap A_j=\emptyset$, $1\leq i\neq j\leq n$. Noting that $K_{x}=\sum_{i=1}^n\1_{A_i}\otimes K_{x_i}$, we have
%\ban
%K_{x}\otimes K_{x}=\left(\sum_{i=1}^n\1_{A_i}\otimes K_{x_i}\right)\otimes \left(\sum_{i=1}^n\1_{A_i}\otimes K_{x_i}\right)=\sum_{i=1}^n\1_{A_i}\otimes \left(K_{x_i}\otimes K_{x_i}\right),
%\ean
%thus, $K_{x}\otimes K_{x}$ is a simple function with values in the Banach space $\LL(\HH_K)$. For any given random vector $x$ with values in the Hilbert space $(\X,\tau_{\text{N}}(\X))$, it is known that there exists a simple function sequence $\{x_n,n\ge 0\}$ with values in $\X$, such that $\lim_{n\to\infty}\|x-x_n\|=0~\text{a.s.}$. This together with the reproducing property of $\HH_K$ and the symmetry of Mercer kernel $K$ gives
%\bna\label{cnkwmekmk}
%\left\|K_x-K_{x_n}\right\|^2_K = \left\langle K_x-K_{x_n},K_x-K_{x_n} \right\rangle_K   = K(x,x)-2K(x,x_n)+K(x_n,x_n).
%\ena
%Noting the continuity of Mercer kernel $K$ and Assumption \ref{assumption5}, by (\ref{cnkwmekmk}), we get
%\bna\label{nncknkwnkwc1}
%\lim_{n\to\infty}\left\|K_x-K_{x_n}\right\|_K=0~\text{a.s.}
%\ena
%It follows from (\ref{vnklooeoeeee}) and the reproducing property of $\HH_K$ that
%\bna\label{nncknkwnkwc2}
%&&~~~\left\|K_{x}\otimes K_{x}-K_{x_n}\otimes K_{x_n}\right\|_{\LL(\HH_K)}\cr
%&&\leq \left\|(K_x-K_{x_n})\otimes K_x\right\|_{\LL(\HH_K)}+\left\|K_{x_n}\otimes (K_x-K_{x_n})\right\|_{\LL(\HH_K)}\cr
%&&=\sup_{\|f\|_K=1}\left\|\left((K_x-K_{x_n})\otimes K_x\right)f\right\|_K+\sup_{\|f\|_K=1}\left\|\left(K_{x_n}\otimes (K_x-K_{x_n})\right)f\right\|_K\cr
%&&=\sup_{\|f\|_K=1}\left\|f(x)(K_x-K_{x_n})\right\|_K+\sup_{\|f\|_K=1}\left\|(f(x)-f(x_n))K_{x_n}\right\|_K\cr
%&&\leq \|K_x\|_K\|K_x-K_{x_n}\|_K+\|K_{x_n}\|_K\|K_x-K_{x_n}\|_K\cr
%&&\leq 2\sup_{x\in \X}\sqrt{K(x,x)}\|K_x-K_{x_n}\|_K~\text{a.s.}
%\ena
%By Assumption \ref{assumption5} and (\ref{nncknkwnkwc1})-(\ref{nncknkwnkwc2}), we have
%\ban
%\lim_{n\to\infty}\left\|K_{x}\otimes K_{x}-K_{x_n}\otimes K_{x_n}\right\|_{\LL(\HH_K)}=0~\text{a.s.}
%\ean
%Noting that $K_{x_n}\otimes K_{x_n}$ is the simple function with values in Banach space $\LL(\HH_K)$, by Definition \ref{vnwkelel}, it is known that $K_{x}\otimes K_{x}$ is strongly measurable with respect to the topology $\tau_{\text{N}}(\LL(\HH_K))$, which together with Pettis measurability theorem shows that $K_{x}\otimes K_{x}$ is the random element with values in Banach space $(\LL(\HH_K),\tau_{\text{N}}(\LL(\HH_K)))$. Thus, by Assumption \ref{assumption5}, we get $K_{x_j(i)}\otimes K_{x_j(i)}\in L^1(\Omega;\mathscr L(\HH_K))$, which together with Lemma \ref{nvkvpeoeo} gives the fact that $\E[K_{x_j(i)}\otimes K_{x_j(i)}|\F(kh-1)]$ uniquely exists. Let $\{g(k),\F(kh-1),k\ge 0\}$ be the $L_2$-bounded adaptive sequence with values in $\HH_K$, by Assumption \ref{assumption5}, Proposition \ref{tiaojianqiwangxingzhi} and the condition (\ref{vnknknldldklsd}), we obtain
%\begin{align}\label{vnkwenkfffnkfmkweklf}
%&\sum_{j=1}^N\sum_{k=0}^{\infty}\E\left[\Bigg\|N_jg(k)
%-\sum_{i=kh}^{(k+1)h-1}\E\left.\left[H_j^*(i)H_j(i)g(k)\right|
%\F(kh-1)\right]\Bigg\|_K^2\right]\cr
% =&\sum_{j=1}^N\sum_{k=0}^{\infty}\E\left[\Bigg\|N_jg(k)-\sum_{i=kh}^{(k+1)h-1}\E\left.\left[K_{x_j(i)}\otimes K_{x_j(i)}g(k)\right|\F(kh-1)\right]\Bigg\|_K^2\right]\cr
% =& \sum_{j=1}^N\sum_{k=0}^{\infty}\E\left[\Bigg\|\Bigg(N_j-\sum_{i=kh}^{(k+1)h-1}\E\left.\left[K_{x_j(i)}\otimes K_{x_j(i)}\right|\F(kh-1)\right]\Bigg)g(k)\Bigg\|_K^2\right]
% <\infty.
%\end{align}
%Denote $\rho_0=N\sup_{x\in\X}K(x,x)$. Given the integer $h>0$, by Assumption \ref{assumption5} and (\ref{vnklooeoeeee}), we have
%\ban
%&&~~~~\sup_{k\ge 0}\left(\E\left.\left[\|\H^*(k)\H(k)\|_{\mathscr L\left(\HH_K^N\right)}^{2^{\max\{h,2\}}}\right|\F(k-1)\right]\right)^{\frac{1}{2^{\max\{h,2\}}}}\cr
%&&= N\sup_{k\ge 0}\left(\E\left.\left[\sup_{i\in \mathcal V}\left\|H_i^*(k)H_i(k)\right\|_{\LL(\HH_K)}^{2^{\max\{h,2\}}}\right|\F(k-1)\right]\right)^{\frac{1}{2^{\max\{h,2\}}}}\cr
%&&=N\sup_{k\ge 0}\left(\E\left.\left[\sup_{i\in \mathcal V}\left\|K_{x_i(k)}\otimes K_{x_i(k)}\right\|_{\LL(\HH_K)}^{2^{\max\{h,2\}}}\right|\F(k-1)\right]\right)^{\frac{1}{2^{\max\{h,2\}}}}\cr
%&&\leq N\sup_{k\ge 0}\left(\E\left.\left[\sup_{i\in \mathcal V}\sup_{\|f\|_K=1}\left\|\left(K_{x_i(k)}\otimes K_{x_i(k)}\right)f\right\|_K^{2^{\max\{h,2\}}}\right|\F(k-1)\right]\right)^{\frac{1}{2^{\max\{h,2\}}}}\cr
%&&=N\sup_{k\ge 0}\left(\E\left.\left[\sup_{i\in \mathcal V}\sup_{\|f\|_K=1}\left\|f(x_i(k))K_{x_i(k)}\right\|_K^{2^{\max\{h,2\}}}\right|\F(k-1)\right]\right)^{\frac{1}{2^{\max\{h,2\}}}}\cr
%&&\leq
%N\sup_{k\ge 0}\left(\E\left.\left[\sup_{i\in \mathcal V}\sup_{\|f\|_K=1}|f(x_i(k))|^{2^{\max\{h,2\}}}\left\|K_{x_i(k)}\right\|_K^{2^{\max\{h,2\}}}\right|\F(k-1)\right]\right)^{\frac{1}{2^{\max\{h,2\}}}}\cr
%&&\leq N\sup_{k\ge 0}\left(\E\left.\left[\sup_{i\in \mathcal V}\sup_{\|f\|_K=1}\left\|f\right\|^{2^{\max\{h,2\}}}_K\left(
%\sup_{x\in\X}K(x,x)\right)^{2^{\max\{h,2\}}}\right|\F(k-1)\right]
%\right)^{\frac{1}{2^{\max\{h,2\}}}}\cr
%&&\leq N\sup_{x\in\X} K(x,x)
% =\rho_0~\text{a.s.}
%\ean
%It follows from Condition \ref{condition2} that there exists a constant $j_0>0$ and an integer $t_0>0$, such that $\sup_{k\ge 0}(4\rho_0a(k)+4\|\L_{\G}\|b(k))\leq j_0$ and $\sup_{k\ge t_0}(a(k)+b(k))(4\rho_0+4\|\L_{\G}\|)\leq 1$. Noting that
%\bna\label{vkllekkeek}
%\left\|4a(k)\H^*(k)\H(k)+4b(k)\L_{\G}\otimes I_{\HH_K}\right\|_{\LL\left(\HH_K^N\right)}\leq
%\begin{cases}
%j_0, & k<t_0;\\
%1, & k\ge t_0
%\end{cases}
%~\text{a.s.}
%\ena
%By (\ref{vkllekkeek}), we get
%\bna\label{vnksdkjdkdddd}
%\big\|I_{\HH_K^N}-4\left(a(k)\H^*(k)\H(k)+b(k)\L_{\G}\otimes I_{\HH_K}\right)\big\|_{\LL\left(\HH_K^N\right)}\leq
%1+\Gamma(k)
%~\text{a.s.},
%\ena
%where
%\ban
%\Gamma(k)=
%\begin{cases}
%j_0, & k<t_0,\\
%0, & k\ge t_0,
%\end{cases}
%\ean
%and satisfies $\sum_{k=0}^{\infty}\Gamma(k)<\infty$. It follows from (\ref{vnksdkjdkdddd}) that
%\begin{align}\label{vnkkwkwkw}
% \E\Big[\big\|I_{\HH_K^N}-4\left(a(k)\H^*(k)\H(k)+b(k)\L_{\G}\otimes I_{\HH_K}\right)\Big\|_{\LL\left(\HH_K^N\right)}\big|\F(kh-1)\Big]
% \leq 1+\Gamma(k)~\text{a.s.}
%\end{align}
%Hence, combining (\ref{vnkwenkfffnkfmkweklf})-(\ref{vnkkwkwkw}) and Theorem \ref{vnknoklfl} gives the fact that the algorithm (\ref{rkhs}) is mean square and almost surely strongly consistent.
%\end{proof}


%\textbf{\emph{Proof of Corollary \ref{vnlllleleeemmem}:}}
%\begin{proof}[Proof of Corollary \ref{vnlllleleeemmem}]
%It follows from Assumption \ref{assumption5} and Theorem \ref{rkhsdingli} that $K_{x_j(i)}\otimes K_{x_j(i)}\in L^1(\Omega;\mathscr L(\HH_K))$, which together with Lemma \ref{nvkvpeoeo} implies that $\E[K_{x_j(i)}\otimes $ $K_{x_j(i)}|\F(kh-1)]$ uniquely exists. Let $\{g(k),\F(kh-1),k\ge 0\}$ be a $L_2$-bounded adaptive sequence with values in $\HH_K$, by the condition (\ref{vnkmeeeemefffff}), we get
%\ban
%&&~~~~\sum_{j=1}^N\sum_{k=0}^{\infty}\E\left[\Bigg\|\Bigg(N_j-\sum_{i=kh}^{(k+1)h-1}\E\left.\left[K_{x_j(i)}\otimes K_{x_j(i)}\right|\F(kh-1)\right]\Bigg)g(k)\Bigg\|_K^2\right]\cr
%&&\leq \sum_{j=1}^N\sum_{k=0}^{\infty}\E\left[\Bigg\|N_j-\sum_{i=kh}^{(k+1)h-1}\E\left.\left[K_{x_j(i)}\otimes K_{x_j(i)}\right|\F(kh-1)\right]\Bigg\|_{\LL(\HH_K)}^2\|g(k)\|_K^2\right]\cr
%&&\leq \sum_{j=1}^N\sum_{k=0}^{\infty}\E\left[\max_{j\in\mathcal V}\Bigg\|N_j-\sum_{i=kh}^{(k+1)h-1}\E\left.\left[K_{x_j(i)}\otimes K_{x_j(i)}\right|\F(kh-1)\right]\Bigg\|_{\LL(\HH_K)}^2\|g(k)\|_K^2\right]\cr
%&&\leq N\mu_0\sup_{k\ge 0}\E\left[\|g(k)\|_K^2\right]\sum_{k=0}^{\infty}\tau(k)
%  <\infty,
%\ean
%where the last inequality is obtained from $\sum_{k=0}^{\infty}\tau(k)<\infty$. This together with Theorem \ref{rkhsdingli} implies that the algorithm (\ref{rkhs}) is both mean square and almost surely strongly consistent.
%
%By Cauchy-Schwarz inequality and the reproducing property of $\HH_K$, we have
%\begin{align}
%|f_{i}(k)(x)-f_{0}(x)|=|\langle f_{i}(k)-f_{0}, K_{x}\rangle| \leq \left\|f_{i}(k)-f_{0}\right\|_{K}  \left\|K_{x}\right\|_{K}, \ \forall \ x \in \mathscr{X}, \ i \in \mathcal{V}.\notag
%\end{align}
%  Noting that algorithm (\ref{rkhs}) is   almost surely strongly consistent and by the above inequality, we have  $\lim\limits_{k\to \infty} f_{i}(k)(x)=f_{0}(x) $   \text{ a.s.},  $\forall \ x \in \mathscr{X}, \ i \in \mathcal{V}$.
%%Besides, and (\ref{pointlyconverge}), we have $|f_{i}(k)(x)-f_{0}(x)|  \leq \left\|f_{i}(k)-f_{0}\right\|_{K}  \left\|K_{x}\right\|_{K} \leq \left\|f_{i}(k)-f_{0}\right\|_{K}  \sqrt{\langle K_{x}, K_{x} \rangle}\leq  \left\|f_{i}(k)-f_{0}\right\|_{K}  \sqrt{\sup_{x\in \X}K(x,x)}$.
%%This together with Assumption \ref{assumption5} and the almost surely strong consistence of $f_{i}(k)$ and $f_{0}$ show that for any $i \in \mathcal{V}$, $\lim\limits_{k\to \infty} f_{i}(k)(x)=f_{0}(x)$ uniformly on $\X$.
%
%
%
%\end{proof}


%\textbf{\emph{Proof of Corollary \ref{rkhsdinglijjjjj}:}}
%\begin{proof}[Proof of Corollary \ref{rkhsdinglijjjjj}]
%Noting that $\{(x_1(k),\cdots,x_N(k)),k\ge 0\}$ and $\{(v_1(k),\cdots ,v_N(k)),k\ge 0\}$ are both i.i.d. sequences and they are mutually independent, we have
%\begin{align*}
%&\sum_{i=kh}^{(k+1)h-1}\E\left.\left[K_{x_j(i)}\otimes K_{x_j(i)}\right|\F(kh-1)\right]  =  \sum_{i=kh}^{(k+1)h-1}\E\left[K_{x_j(i)}\otimes K_{x_j(i)}\right]
% =   h\E\left[K_{x_j(0)}\otimes K_{x_j(0)}\right].
%\end{align*}
%Denote $N_j:\HH_K\to \HH_K$ by
%$
%N_j=h\E\left[K_{x_j(0)}\otimes K_{x_j(0)}\right],~j\in \mathcal V.
%$
%It follows from Proposition 2.6.13 in \cite{hy} that $N_j\in \mathscr L(\HH_K)$ is a positive self-adjoint operator. Noting that
%\ban
%\bigg\|N_j-\sum_{i=kh}^{(k+1)h-1}\E\left.\left[K_{x_j(i)}\otimes K_{x_j(i)}\right|\F(kh-1)\right]\bigg\|_{\LL(\HH_K)}^2=0~\text{a.s.},
%\ean
%by the condition (\ref{nklnkle}), we get
%\ban
%\left\langle \sum_{j=1}^NN_jf,f\right\rangle _K =h\left\langle \E\left[\sum_{j=1}^NK_{x_j(0)}\otimes K_{x_j(0)}\right]f,f\right\rangle _K >0,
%\ean
%where $f$ is an arbitrary non-zero function in $\HH_K$. Hence, it follows from Corollary \ref{vnlllleleeemmem} that the algorithm (\ref{rkhs}) is both mean square and almost surely strongly consistent.
%\end{proof}

\section{Key Lemmas}\label{appendixee}
%\setcounter{lemma}{0}
%\def\thelemma{D.\arabic{lemma}}
%\setcounter{definition}{0}
%\def\thedefinition{D.\arabic{definition}}
%\setcounter{equation}{0}
%\def\theequation{D.\arabic{equation}}
 \setcounter{equation}{0}
\renewcommand{\theequation}{C.\arabic{equation}}
\begin{lemma}[\cite{rb}]\label{lemmaA3}
Let $\{x(k), \mathcal F(k)\}$, $\{\a(k),\mathcal F(k)\}$, $\{\b(k),\mathcal F(k)\}$ and $\{\gamma(k), \mathcal F(k)\}$ be nonnegative  adaptive sequences satisfying
$$
\E[x(k+1)|\mathcal F(k)]\le(1+\a(k))x(k)-\beta(k)+\gamma(k),~k\ge 0~\text{a.s.},
$$
and $\sum_{k=0}^\infty(\a(k)+\gamma(k))<\infty~\text{a.s.}$ Then $x(k)$ converges to a finite random variable a.s., and $\sum_{k=0}^\infty\b(k)<\infty$ a.s.
\end{lemma}


\begin{lemma}\label{lemma6}
Let $\{a(i),i\in \Lambda\}$ be a nonnegative real sequence, where $\Lambda \subseteq \mathbb N$ and $\sum_{i\in \Lambda}a(i)<\infty$, $\{b(k,i),k\in \mathbb N,i\in \Lambda\}$ be a double index real sequence. If $\lim_{k\to \infty}b(k,i)=0$, $\forall\ i\in  \Lambda$, and there exists a constant $c>0$ such that $|b(k,i)|\leq c$, $\forall\ k\in \mathbb N$, $\forall\ i\in \Lambda$, then
\bna\label{mmxxxx}
\lim_{k\to \infty}\sum_{i\in \Lambda}a(i)b(k,i)=0.
\ena
\end{lemma}

\begin{proof}
For any given $\varepsilon>0$, it follows from $\sum_{i\in \Lambda}a(i)<\infty$ that there exists an integer $N>0$, such that $\sum_{i\in \Lambda_N}a(i)<\varepsilon$, where $\Lambda_N=\{i\in \Lambda:i\ge N+1\}$. We obtain
$
\Bigg|\sum_{i\in \Lambda_N}a(i)b(k,i)\Bigg|\leq c\varepsilon,~\forall\ k\ge 0.
$
On the other hand, noting that $\lim_{k\to \infty}b(k,i)=0$, $\forall i\in \Lambda$, it follows that there exists a constant $M_i>0$, such that $|b(k,i)|\leq \varepsilon $, $\forall k\ge M_i$. Denote $M=\max\{M_i:i\in \{1,\cdots,N\}\cap \Lambda\}$. It follows from $|b(k,i)|\leq \varepsilon,~\forall\ k\ge M$ that
\ban
\Bigg|\sum_{i\in \Lambda}a(i)b(k,i)\Bigg|\leq \Bigg|\sum_{i\in \{1,\cdots,N\}\cap \Lambda}a(i)b(k,i)\Bigg|+\Bigg|\sum_{i\in \Lambda_N}a(i)b(k,i)\Bigg|\leq \varepsilon \sum_{i\in \Lambda}a(i)+c\varepsilon,~k\ge M,
\ean
which gives (\ref{mmxxxx}).
\end{proof}


\begin{lemma}\label{lemmaA7}
Let $\X$ be a Hilbert space, $H\in \mathscr L(\X)$ be a strictly positive self-adjoint operator, and $\{\mu(k),k\ge 0\}$ be a real sequence monotonically decreasing to $0$ with $\sum_{k=0}^{\infty}\mu(k)=\infty$. Then there exist positive constants $M$ and $d$, such that
\bna
\label{lemmaaddaddadd1}
\sup_{s\ge 0}\left\|\left(\prod_{j=t}^s(I_{\X}-\mu(j)H)\right)x\right\|\leq M^d\|x\|,~\forall\ x\in \X,~\forall\ t\ge 0,
\ena
and
\bna
\label{lemmaaddaddadd2}
\lim_{k\to \infty}\left(\prod_{j=0}^k(I_{\X}-\mu(j)H)\right)x=0,~\forall\ x \in \X.
\ena
\end{lemma}

\begin{proof}
Noting that $H\in \mathscr L(\X)$ is the bounded self-adjoint operator, it is shown that $H$ has the following spectral decomposition
$$
H=\int_{-\infty}^{+\infty}\lambda \dd p_{\lambda}=\int_{\sigma (H)}\lambda \dd p_{\lambda}.
$$
From the property of the self-adjoint operator, we have
\bna\label{nvmsacc}
\prod_{j=0}^{k
}(I_{\X}-\mu(j)H)^2=\int_{\sigma(H)}\prod_{j=0}^{k}(1-\mu(j)\lambda)^2\dd p_{\lambda},
\ena
which together with (\ref{nvmsacc}) gives
\begin{align}\label{mmcc}
\Bigg\|\prod_{j=0}^{k
}(I_{\X}-\mu(j)H)x\Bigg\|^2 =& \Bigg\langle\prod_{j=0}^{k
}(I_{\X}-\mu(j)H)^2x,x\Bigg\rangle \notag\\
 =& \int_{\sigma(H)}\prod_{j=0}^{k}(1-\mu(j) \lambda)^2\dd\langle p_{\lambda}x,x\rangle,
 ~\forall\ x\in \X.
\end{align}
It follows from $H>0$ that $\sigma(H)\subset [0,\|H\|]$, which leads to
\ban
(1-\mu(j)\lambda)^2\leq \max\left\{(1-\mu(j)\|H\|)^2,1\right\},~\forall\ \lambda \in \sigma(H).
\ean
Noting that $\mu(k)\to 0,k\to \infty$, it follows that there exists an integer $d>0$ such that $\mu(j)\leq \|H\|^{-1}$, $\forall\ j>2d$, which further shows that $(1-b(j)\lambda)^2\leq 1,\forall\ j> 2d$. Denote $M=\max\{(1-\mu(j)\|H\|)^2,1:0\leq j\leq 2d\}$, we have
$$
\prod_{j=t}^s(1-\mu(j)\lambda)^2\leq M^{2d},~\forall\ \lambda \in \sigma(H),~\forall\ t\ge 0,
$$
and
\ban
%\label{hhkds}
\left\|\prod_{j=t}^{s
}(I_{\X}-\mu(j)H)x\right\|^2\leq M^{2d}\int_{\sigma(H)}\dd\langle p_{\lambda}x,x\rangle =M^{2d}\|x\|^2,~\forall\ x\in \X,~\forall\ t\ge 0.
\ean
This gives (\ref{lemmaaddaddadd1}).
%%By (\ref{hhkds}), we get
%\bna\label{nslcs}
%\left\|\prod_{j=t}^{s
%}(I_{\X}-\mu(j)H)x\right\|\leq M^{d}\|x\|,~\forall\ x\in \X,~\forall\ t\ge 0.
%\ena
Noting the inequality $1-a\leq \text{e}^{-a}$, $\forall\ a\ge 0$, we obtain
$
\prod_{j=0}^{k}(1-\mu(j)\lambda)^2\leq M^{2d}\text{e}^{-2\lambda\sum_{j=2d+1}^{k}\mu(j)}.
$
It follows from $\sum_{j=0}^{\infty}\mu(k)=\infty$ that
\bna\label{ttkktt}
\lim_{k\to \infty}\prod_{j=0}^{k}(1-\mu(j)\lambda)^2=0,~\forall\ \lambda\in \sigma(H)\cap \mathbb R^+.
\ena
Since $H$ is the strictly positive self-adjoint operator, it follows that $0$ is not in the point spectrum of the operator $H$, which gives
\ban
\int_{\sigma(H)}\prod_{j=0}^{k}(1-\mu(j)\lambda)^2\dd\langle p_{\lambda}x,x\rangle =\int_{\sigma(H)\cap \mathbb R^+}\prod_{j=0}^{k}(1-\mu(j)\lambda)^2\dd\langle p_{\lambda}x,x\rangle,~\forall\ x\in \X.
\ean
By (\ref{mmcc}), (\ref{lemmaaddaddadd1}), (\ref{ttkktt}) and the dominated convergence theorem, we get
\ban
\lim_{k\to \infty}\left\|\prod_{j=0}^{k
}(I_{\X}-\mu(j)H)x\right\|^2=\int_{\sigma(H)\cap \mathbb R^+}\lim_{k\to\infty}\prod_{j=0}^{k}(1-\mu(j)\lambda)^2\dd\langle p_{\lambda}x,x\rangle=0,~\forall\ x\in \X,
\ean
which gives (\ref{lemmaaddaddadd2}).
\end{proof}


\begin{lemma}\label{lemmaA8}
Let $\X$ be a Hilbert space, $H\in \mathscr L(\X)$ be a strictly positive self-adjoint operator, and $\{x(k),k\ge 0\}$ be a $L_2$-bounded random sequence with values in Hilbert space. If $\{\mu(k),k\ge 0\}$ and $\{\gamma(k),k\ge 0\}$ are both real sequences monotonically decreasing to $0$ with $\sum_{k=0}^{\infty}\mu(k)=\infty$ and $\sum_{k=0}^{\infty}\gamma(k)<\infty$, then
$$
\lim_{k\to\infty}\sum_{i=0}^k\gamma(i)\left(\E\left[\left\|\left(\prod_{j=i+1}^k(I_{\X}-\mu(j)H)\right)x(i)\right\|^2\right]\right)^{\frac{1}{2}}=0.
$$
\end{lemma}

\begin{proof}
Denote
\ban
b(k,i)=\left(\E\left[\left\|\prod_{j=i+1}^k(I_{\X}-\mu(j)H)x(i)\right\|^2\right]\right)^{\frac{1}{2}},~\forall\ k,i\ge 0,
\ean
it follows from Lemma \ref{lemmaA7} that
\bna\label{rrkkrr}
0\leq b(k,i)\leq M^d\sup_{i\ge 0}\left(\E\left[\|x(i)\|^2\right]\right)^{\frac{1}{2}}<\infty,~\forall\ k,i\ge 0,
\ena
and
\ban
\left\|\prod_{j=i+1}^k(I_{\X}-\mu(j)H)x(i)\right\|\leq M^d\|x(i)\|~\text{a.s.},~\forall\ i\ge 0.
\ean
Thus, by the Lebesgue dominated convergence theorem and Lemma \ref{lemmaA7}, we get
\bna\label{rrkkll}
\lim_{k\to\infty}\E\left[\left\|\prod_{j=i+1}^k(I_{\X}-\mu(j)H)x(i)\right\|^2\right]=0,~\forall\ i\ge 0,
\ena
which further gives $\lim_{k\to\infty}b(k,i)=0$, $\forall\ i\ge 0$. Noting that $\sum_{k=0}^{\infty}\gamma(k)<\infty$, by (\ref{rrkkrr})-(\ref{rrkkll}) and Lemma \ref{lemma6}, we obtain
$
\lim_{k\to\infty}\sum_{i=0}^k\gamma(i)\left(\E\left[\left\|\left(\prod_{j=i+1}^k(I_{\X}-\mu(j)H)\right)x(i)\right\|^2\right]\right)^{\frac{1}{2}}=0.
$
\end{proof}


\begin{lemma}\label{lemmaA10}
Let $\mathcal A=[a_{ij}]\in \mathbb R^{N\times N}$ be the adjacency matrix of an undirected connected graph $\G$, $\L_{\G}$ be the Laplacian matrix, $H_i\in \mathscr L(\X)$, $i=1,\cdots,N$ are positive self-adjoint operators. If
\bna\label{lcms}
\sum_{i=1}^NH_i>0,
\ena
then $\text{diag}\{H_1,\cdots,H_N\}+\L_{\G}\otimes I_{\X}>0$.
\end{lemma}

\begin{proof}
Given the non-zero element $x=(x_1,\cdots,x_N)$ with values in Hilbert space $\X^N$, where $x_i\in \X$, $i=1,\cdots,N$. Here, we will prove $\langle (\text{diag}\{H_1,\cdots,H_N\}+\L_{\G}\otimes I_{\X})x,x\rangle _{\X^N}>0$ in two steps.
\begin{itemize}
\item If there exists a non-zero element $a\in \X$ with $x_1=\cdots=x_N=a$, then $x=\textbf{1}_N\otimes a$. Noting that $\L_{\G}\textbf{1}_N=0$, it follows from (\ref{lcms}) that
\bna\label{ofuncc}
&&~~~\left\langle \left(\text{diag}\{H_1,\cdots,H_N\}+\L_{\G}\otimes I_{\X}\right)x,x\right\rangle _{\X^N}\cr
&&=\left\langle \left(\text{diag}\{H_1,\cdots,H_N\}+\L_{\G}\otimes I_{\X}\right)\left(\textbf{1}_N\otimes a\right),\left(\textbf{1}_N\otimes a\right)\right\rangle _{\X^N}\cr
&&=\langle \text{diag}\{H_1,\cdots,H_N\}\left(\textbf{1}_N\otimes a\right),\left(\textbf{1}_N\otimes a\right)\rangle _{\X^N}\cr
&&+\langle (\L_{\G}\otimes I_{\X})\left(\textbf{1}_N\otimes a\right),\left(\textbf{1}_N\otimes a\right)\rangle _{\X^N}\cr
&&=\left\langle \left(\sum_{i=1}^NH_i\right)a,a\right\rangle _{\X}+\langle (\L_{\G}\textbf{1}_N)\otimes a,\left(\textbf{1}_N\otimes a\right) \rangle _{\X^N}\cr
&&=\left\langle \left(\sum_{i=1}^NH_i\right)a,a\right\rangle _{\X}
 >0.
\ena
\item If there exist $1\leq i_0\neq j_0\leq N$, such that $x_{i_0}\neq x_{j_0}$. It follows from $H_i\ge 0$, $i=1,\cdots,N$ that $\text{diag}\{H_1,\cdots,H_N\}\ge 0$. Noting that the graph is undirected and connected, it is shown that $a_{ij}=a_{ji}>0$, $1\leq i\neq j\leq N$. Thus, we get
\bna\label{ofunc}
&&~~~\left\langle \left(\text{diag}\{H_1,\cdots,H_N\}+\L_{\G}\otimes I_{\X}\right)x,x\right\rangle _{\X^N}\cr
&&\ge \langle (\L_{\G}\otimes I_{\X})x,x\rangle _{\X^N}\cr
&&=\frac{1}{2}\sum_{i=1}^N\sum_{j=1}^Na_{ij}\|x_i-x_j\|^2_{\X}
 \ge \frac{1}{2}a_{i_0j_0}\|x_{i_0}-x_{j_0}\|_{\X}^2
 >0.
\ena
\end{itemize}
Combining (\ref{ofuncc})-(\ref{ofunc}) gives that  $\text{diag}\{H_1,\cdots,H_N\}+\L_{\G}\otimes I_{\X}\in \mathscr L(\X^N)$ is strictly positive.
\end{proof}


\begin{lemma}\label{lemmaA11}
Let $\{a(k),k\ge 0\}$ and $\{b(k),k\ge 0\}$ be monotonically decreasing sequences of positive real numbers. If
\bna\label{tiaojian}
\max\left\{a(k)-a(k+1),b(k)-a(k)\right\}=\mathcal O\left(a^2(k)+b^2(k)\right),
\ena
then
\bna\label{fulu11}
\max_{ih\leq s\leq (i+1)h-1}\left(\frac{1}{h}\left(\sum_{s=ih}^{(i+1)h-1}b(s)\right)-a(s)\right)^2=\mathcal O \left(a^4(i)+b^4(i)\right),~\forall\ h=1,2,...
\ena
\end{lemma}

\begin{proof}
Let $h$ be any given positive integer. Noting that $\{a(k),k\ge 0\}$ and $\{b(k),k\ge 0\}$ are both monotonically decreasing sequences, it follows that
$$\frac{1}{h}\left(\sum_{s=ih}^{(i+1)h-1}b(s)\right)\in [b((i+1)h-1),b(ih)],$$
and $$a(s)\in [a((i+1)h-1),a(ih)],~ih\leq s\leq (i+1)h-1,$$
%\ban
%\begin{cases}
%\displaystyle \frac{1}{h}\left(\sum_{s=ih}^{(i+1)h-1}b(s)\right)\in [b((i+1)h-1),b(ih)],\\
%\displaystyle a(s)\in [a((i+1)h-1),a(ih)],~ih\leq s\leq (i+1)h-1,
%\end{cases}
%\ean
from which we obtain
\bna\label{fulu0}
&&~~~\max_{ih\leq s\leq (i+1)h-1}\left(\frac{1}{h}\left(\sum_{s=ih}^{(i+1)h-1}b(s)\right)-a(s)\right)^2\cr
&&\leq \max_{ih\leq s\leq (i+1)h-1}\left\{(b((i+1)h-1)-a(s))^2,(b(ih)-a(s))^2\right\}\cr
&&\leq \max\big\{(b((i+1)h-1)-a(ih))^2,(b((i+1)h-1)-a((i+1)h-1))^2,\cr
&&~~~(b(ih)-a(ih))^2,(b(ih)-a((i+1)h-1))^2\big\}.
\ena
It follows from (\ref{tiaojian}) that there exists a constant $C>0$, such that
\bna\label{fnlwlmv}
\max\left\{(a(k)-a(k+1))^2,(b(k)-a(k))^2\right\}\leq C\left(a^4(k)+b^4(k)\right),
\ena
which gives
\bna\label{fulu1}
\max\left\{(b((i+1)h-1)-a((i+1)h-1))^2,(b(ih)-a(ih))^2\right\}
\leq C\left(a^4(i)+b^4(i)\right).
\ena
Combining (\ref{fnlwlmv}) and the monotonicity of sequences $\{a(k),k\ge 0\}$ and $\{b(k),k\ge 0\}$ leads to
\begin{align}\label{fulu2}
 &\left(b((i+1)h-1)-a(ih)\right)^2\cr
 =& \left(b((i+1)h-1)-a((i+1)h-1)+a((i+1)h-1)-a(ih)\right)^2\cr
 =&\Bigg(b((i+1)h-1)-a((i+1)h-1)+\sum_{s=ih}^{(i+1)h-2}(a(s+1)-a(s))\Bigg)^2\cr
 \leq& h\Bigg((b((i+1)h-1)-a((i+1)h-1))^2+\sum_{s=ih}^{(i+1)h-2}(a(s)-a(s+1))^2\Bigg)\cr
\leq& h\Bigg(C\left(a^4((i+1)h-1)+b^4((i+1)h-1)
\right)+C\sum_{s=ih}^{(i+1)h-2}\left(a^4(s)+b^4(s)\right)\Bigg)\cr
\leq& Ch^2\left(a^4(i)+b^4(i)\right).
\end{align}
Similarly, we obtain
\bna\label{fulu3}
&&~~~~\left(a((i+1)h-1)-b(ih)\right)^2\cr
&&=\left(b(ih)-a((i+1)h-1)\right)^2\cr
&&=\left(b(ih)-a(ih)+a(ih)-a((i+1)h-1)\right)^2\cr
&&=\left(b(ih)-a(ih)+\sum_{s=ih}^{(i+1)h-2}(a(s)-a(s+1))\right)^2\cr
&&\leq h\left((b(ih)-a(ih))^2+\sum_{s=ih}^{(i+1)h-2}(a(s)-a(s+1))^2\right)\cr
&&\leq h\left(C\left(a^4(ih)+b^4(ih)\right)+C\sum_{s=ih}^{(i+1)h-2}
\left(a^4(s)+b^4(s)\right)\right) \cr
 &&\leq Ch^2\left(a^4(i)+b^4(i)\right).
\ena
Combining (\ref{fulu0}) and (\ref{fulu1})-(\ref{fulu3}) gives (\ref{fulu11}).
\end{proof}


\begin{lemma}\label{lemma1}
For the algorithm (\ref{algorithm}), suppose that Conditions \ref{condition1} and \ref{condition2} hold, and there exists an integer $h>0$ and a constant $\rho>0$, such that
\bna\label{nxsl}
\sup_{k\ge 0}\E\left.\left[\|\H^*(k)\H(k)\|^{2^{\max\{h,2\}}}\right|\F(k-1)\right]^{\frac{1}{2^{\max\{h,2\}}}}\leq \rho~\text{a.s.}
\ena
Then
(I) for any given integer $n\ge 0$ and $x\in L^0(\Omega,\F(nh-1);\X^N)$, there exists a constant $d_1>0$ such that
\bna\label{yinliwudianyi1}
\sup_{m\ge 0}\mathbb E\left.\left[\|\Phi_P(mh-1,nh)x\|^2\right|\mathcal F(nh-1)\right]\leq d_1\|x\|^2~\text{a.s.};
\ena
(II) for any given integer $n\ge 0$ and $x\in L^0(\Omega,\F(n-1);\X^N)$, then there exists a constant $d_2>0$ such that
\bna\label{yinliwudianyi2}
\displaystyle \sup_{0\leq m\leq n+h}\mathbb E\left[\|\Phi_P(m,n)x\|^2|\mathcal F(n-1)\right]\leq d_2\|x\|^2~\text{a.s.}
\ena
\end{lemma}

\begin{proof}%[Proof of Lemma \ref{lemma1}]
Denote $D(k)=a(k)\H^*(k)\H(k)+b(k)\L_{\G}\otimes I_{\X}$ and  $P(k)=I_{\X^N}-D(k)$. Noting that $\L_{\G}\otimes I_{\X}\ge 0$, we have $D(k)\ge 0~\text{a.s.}$ Let $x$ be a random element with values in Hilbert space $(\X^N,\tau_{\text{N}}(\X^N))$, we get
\bna\label{kkdsa}
&&~~~\mathbb \|\Phi_P(m,n)x\|^2\cr
&&=\langle\Phi_P(m,n)x,\Phi_P(m,n)x\rangle\cr
&&=\left\langle x,\Phi^*_P(m,n)\Phi_P(m,n)x\right\rangle\cr
&&=\left\langle x,(I_{\X^N}-D^*(n))\cdots(I_{\X^N}-D^*(m))(I_{\X^N}-D(m))\cdots (I_{\X^N}-D(n))x\right\rangle\cr
&&=\left\langle x,x-2\sum_{k=n}^mD(k)x+\sum_{k=2}^{2(m-n+1)}M_kx\right\rangle\cr
&&=\|x\|^2-2\sum_{k=n}^m\langle x,D(k)x\rangle+\left\langle x,\sum_{k=2}^{2(m-n+1)}M_kx\right\rangle\cr
&&\leq \|x\|^2+\left\langle x,\sum_{k=2}^{2(m-n+1)}M_kx\right\rangle
 \leq \Bigg(1+\sum_{k=2}^{2(m-n+1)}\|M_k\| \Bigg)\|x\|^2~\text{a.s.},~0\leq n \leq m,
\ena
where $M_k,k=2,\cdots,2(m-n+1)$ denote the $k$-th order terms in the binomial expansion of $\Phi_P^*(m,n)\Phi_P(m,n)$. For $0\leq m-n\leq h$, by the conditional Lyapunov inequality and (\ref{nxsl}), we obtain
\bna\label{jjkkss}
&&\hspace{-1.4cm}\sup_{k\ge 0}\mathbb E\left.\left[\|D(k)\|^i\right|\mathcal F(k-1)\right]\leq \sup_{k\ge 0}\mathbb E\left.\left[\|D(k)\|^{2^h}\right|\mathcal F(k-1) \right]^{\frac{i}{2h}}\leq \rho_0^i(a(k)+b(k))^i~\text{a.s.},\cr
&&~~~~~~~~~~~~~~~~~~~~~~~~~~~~~~~~~~~~~~~~~~~~~~~~~~~~~~~~~~~~~~~~~~~~~~~~~~~~~2\leq i\leq 2^h,
\ena
where $\rho_0:=\rho+\|\L_{\G}\|$. Note that
\bna\label{nkvvssk}
&&~~~~\mathbb E\left.\left[\|D(k)\|^i\right|\mathcal F(n-1) \right]\cr
&&=\mathbb E\left.\left.\left[\mathbb E\left[\|D(k)\|^i\right|\mathcal F(k-1)\right]\right|\mathcal F(n-1)\right],~2\leq i\leq 2^h,~k\ge n.
\ena
Since the real sequences $\{a(k),k\ge 0\}$ and $\{b(k),k\ge 0\}$ are both monotonically decreasing to $0$, it follows that there exists a constant $c_0>0$ such that $\sup_{k\ge 0}(a(k)+b(k))\leq c_0$. For $0\leq m-n\leq h$, from the definition of $M_k$ and (\ref{jjkkss})-(\ref{nkvvssk}), by termwise multiplication and using the H\"{o}lder inequality repeatedly, we have
\bna\label{pplds}
&&\hspace{-0.8cm}\mathbb E[\|M_k\||\mathcal F(n-1)]\leq \mathbb C_{2(m-n+1)}^k\rho_0^k(a(n)+b(n))^k\leq 2c_0^{-2}\left(a^2(n)+b^2(n)\right)\mathbb C_{2(m-n+1)}^k\rho_0^kc_0^k~\text{a.s.},\cr &&~~~~~~~~~~~~~~~~~~~~~~~~~~~~~~~~~~~~~~~~~~~~~~~~~~~~~~~~~~k=2,\cdots,2(m-n+1).
\ena
Denote $c_k=2c_0^{-2}(1+\rho_0c_0)^{2k}$, by (\ref{pplds}), we get
\bna\label{wwddsaf}
\sum_{k=2}^{2(m-n+1)}\mathbb E[\|M_k\||\mathcal F(n-1)]\leq c_{m-n+1}\left(a^2(n)+b^2(n)\right)~\text{a.s.}
\ena
If $x\in L^0(\Omega,\F(n-1);\X^N)$, substituting (\ref{wwddsaf}) into (\ref{kkdsa}) gives
\bna\label{kkxnms}
\mathbb E\left.\left[\|\Phi_P(m,n)x\|^2\right|\mathcal F(n-1)\right]\leq \left(1+c_{h+1}a^2(n)+c_{h+1}b^2(n)\right)\|x\|^2~\text{a.s.}
\ena
If $x\in L^0(\Omega,\F(ih-1);\X^N)$, it follows from (\ref{kkxnms}) that
\bna\label{eedd}
&&~~~~\mathbb E\left.\left[\|\Phi_P((i+1)h-1,ih)x\|^2\right|\mathcal F(ih-1)\right]\cr
&&\leq \left(1+c_ha^2(i)+c_hb^2(i)\right)\|x\|^2~\text{a.s.},~i\ge 0.
\ena
(I) For the given integer $n$ and $x\in L^0(\Omega,\F(nh-1);\X^N)$, by (\ref{eedd}), we obtain
\bna\label{vnklmef}
&&~~~~\mathbb E\left.\left[\|\Phi_P(mh-1,nh)x\|^2\right|\mathcal F(nh-1)\right]\cr
&&=\mathbb E\left.\left[\|\Phi_P(mh-1,(m-1)h)\Phi_P((m-1)h-1,nh)x\|^2\right|\mathcal F(nh-1)\right]\cr
&&=\mathbb E\big[\mathbb E\big[\|\Phi_P(mh-1,(m-1)h)\cr
&&~~~\times\Phi_P((m-1)h-1,nh)x\|^2|\mathcal F((m-1)h-1)\big]|\mathcal F(nh-1)\big]\cr
&&\leq \left(1+c_ha^2(m-1)+c_hb^2(m-1)\right)\mathbb E\left.\left[\|\Phi_P((m-1)h-1,nh)x\|^2\right|\mathcal F(nh-1)\right]\cr
&&\leq \prod_{k=n}^{m-1}\left(1+c_ha^2(k)+c_hb^2(k)\right)\|x\|^2~\text{a.s.}
\ena
Denote $d_1=\prod_{k=0}^{\infty}(1+c_ha^2(k)+c_hb^2(k))$, from (\ref{vnklmef}) and Condition \ref{condition2}, we get (\ref{yinliwudianyi1}).

(II) For the given integer $n$ and $x\in L^0(\Omega,\F(n-1);\X^N)$, it follows from Condition \ref{condition1} that there exists a constant $d_2>0$, such that $\sup_{k\ge 0}(1+c_{h+1}a^2(k)+c_{h+1}b^2(k))\leq d_2$, which together with (\ref{kkxnms}) gives (\ref{yinliwudianyi2}).
\end{proof}

\begin{lemma}\label{hhhlemma}
For the algorithm (\ref{algorithm}), if Conditions \ref{condition1} and \ref{condition2} hold, and there exists an integer $h>0$ and a constant $\rho>0$, such that
\ban
\sup_{k\ge 0}\E\left.\left[\|\H^*(k)\H(k)\|^{2^{\max\{h,2\}}}\right|\F(k-1)\right]^{\frac{1}{2^{\max\{h,2\}}}}\leq \rho~\text{a.s.},
\ean
then there exits a constant $d_3>0$, such that
\ban
&&~~~~\sup_{k\ge 0}\E\left[\left\|\left(\prod_{j=i+1}^k\left(I_{\X^N}-\sum_{s=jh}^{(j+1)h-1}(a(s)\H^*(s)\H(s)+b(s)(\L_{\G}\otimes I_{\X}))\right)\right)y\right\|^2\right]\cr
&&\leq d_3\E\left[\|y\|^2\right],~\forall\ i\ge 0,
\ean
for any given $y\in L^0(\Omega,\F((i+1)h-1);\X^N)$.
\end{lemma}

\begin{proof}%[Proof of Lemma \ref{hhhlemma}]
Denote $D'(k)=\sum_{s=kh}^{(k+1)h-1}(a(s)\H^*(s)\H(s)+b(s)(\L_{\G}\otimes I_{\X}))$ and $P'(k)=I_{\X^N}-D'(k)$. Noting that $\L_{\G}\otimes I_{\X}\ge 0$, we have $D'(k)\ge 0~\text{a.s.}$ Let $x\in L^0(\Omega,\F(nh^2-1);\X^N)$ be a random element with values in the Hilbert space $\X^N$. Then we get
\ban
&&\mathbb E\left.\left[\|\Phi_{P'}((n+1)h-1,nh)x\|^2\right|\mathcal F(nh^2-1)\right]
 \leq \left(1+\sum_{k=2}^{2h}\left.\mathbb E\left[\|M_k'\|\right|\mathcal F(nh^2-1)\right]\right)\|x\|^2,
\ean
where $M_k',k=2,\cdots,2h$ denote the $k$-th order terms in the binomial expansion of $\Phi_{P'}^*((n+1)h-1,nh)\Phi_{P'}((n+1)h-1,nh)$. It follows from the conditional Lyapunov inequality and Condtion \ref{condition1} that
\bna\label{jjkkssssfffw}
&&\hspace{-1.6cm}\sup_{k\ge 0}\mathbb E\left.\left[\|D'(k)\|^i\right|\mathcal F(k-1)\right]\leq  h\rho_0^i\sum_{s=kh}^{(k+1)h-1}(a(s)+b(s))^i\leq h^2\rho_0^i(a(n)+b(n))^i~\text{a.s.},\cr
&&~~~~~~~~~~~~~~~~~~~~~~~~~~~~~~~~~~~~~~~~~~~~~~~~~~~~~~~~~~~~~~~~~~~~~~~~~2\leq i\leq 2^h,
\ena
where $\rho_0=\rho+\|\L_{\G}\|$. Since $\{a(k),k\ge 0\}$ and $\{b(k),k\ge 0\}$ are both monotonically decreasing to  $0$, there exists a constant $c_0>0$ such that $\sup_{k\ge 0}(a(k)+b(k))\leq c_0$. From the definition of $M_k'$, by termwise multiplication and using the H\"{o}lder inequality of the conditional expectation, we have
\bna\label{ppldssss}
\mathbb E\left.\left[\left\|M_k'\right\|\right|\mathcal F(nh^2-1)\right]&\leq& h^2\mathbb C_{2h}^k\rho_0^k(a(n)+b(n))^k\cr
&\leq& 2h^2c_0^{-2}\left(a^2(n)+b^2(n)\right)\mathbb C_{2h}^k\rho_0^kc_0^k~\text{a.s.},~k=2,\cdots,2h.
\ena
Noting that (\ref{ppldssss}) and $x\in L^0(\Omega,\F(nh^2-1);\X^N)$, we get
\bna\label{eefffdddssd}
&&~~~~\mathbb E\left.\left[\|\Phi_{P'}((n+1)h-1,nh)x\|^2\right|\mathcal F(nh^2-1)\right]\cr
&&\leq \left(1+c'a^2(n)+c'b^2(n)\right)\|x\|^2~\text{a.s.},
\ena
where $c'=2h^2c_0^{-2}(1+\rho_0c_0)^{2h}$. By (\ref{eefffdddssd}), we obtain
\ban
&&~~~~\mathbb E\left.\left[\|\Phi_{P'}(mh-1,nh)x\|^2\right|\mathcal F(nh^2-1)\right]\cr
&&=\mathbb E\left.\left[\|\Phi_{P'}(mh-1,(m-1)h)\Phi_{P'}((m-1)h-1,nh)x\|^2\right|\mathcal F(nh^2-1)\right]\cr
&&=\mathbb E\big[\mathbb E\big[\|\Phi_{P'}(mh-1,(m-1)h)\cr
&&~~~\times\Phi_{P'}((m-1)h-1,nh)x\|^2|\mathcal F((m-1)h^2-1)\big]|\mathcal F(nh^2-1)\big]\cr
&&\leq \left(1+c'a^2(m-1)+c'b^2(m-1)\right)\mathbb E\left.\left[\|\Phi_{P'}((m-1)h-1,nh)x\|^2\right|\mathcal F(nh^2-1)\right]\cr
&&\leq \prod_{k=n}^{m-1}\left(1+c'a^2(k)+c'b^2(k)\right)\|x\|^2~\text{a.s.},~m>n\ge 0.
\ean
Denote $q_1=\prod_{k=0}^{\infty}(1+c'a^2(k)+c'b^2(k))$, it follows from Condition \ref{condition2} that
\ban
\sup_{m\ge 0}\mathbb E\left.\left[\|\Phi_{P'}(mh-1,nh)x\|^2\right|\mathcal F(nh^2-1)\right]\leq q_1\|x\|^2~\text{a.s.}
\ean
Following the same way as the proof of Lemma \ref{lemma1}, it shows that there exists a constant $q_2>0$, such that
\bna\label{leisileisi}
\sup_{0\leq m\leq n+h}\mathbb E\left.\left[\|\Phi_{P'}(m,n)x\|^2\right|\mathcal F(nh-1)\right]\leq q_2\|x\|^2~\text{a.s.}
\ena
For any given positive integer $j$, denote $m_j=\lfloor \frac{j}{h} \rfloor,\widetilde{m}_j=\lceil \frac{j}{h} \rceil$. Firstly, if $0\leq i<k-3h$, then $m_kh>\widetilde{m}_{i+1}h$. Let $y\in L^0(\Omega,\F((i+1)h-1);\X^N)$ be the random element with values in the Hilbert space $\X^N$. Noting that $0\leq k-m_kh<h$, $0\leq \widetilde{m}_{i+1}h-(i+1)<h$, $\Phi_{P'}(m_kh-1,\widetilde{m}_{i+1}h)\Phi_{P'}(\widetilde{m}_{i+1}h-1,i+1)y\in \F(m_kh^2-1)$ and $\Phi_{P'}(\widetilde{m}_{i+1}h-1,i+1)y\in \F(\widetilde{m}_{i+1}h-1)$, by (\ref{eefffdddssd})-(\ref{leisileisi}), we get
\bna\label{jjkklkkklll}
&&~~~~~~\mathbb E\left[\left\|\Phi_{P'}(k,i+1)y\right\|^2\right]\cr
&&~~=\mathbb E\left[\|\Phi_{P'}(k,m_kh)\Phi_{P'}(m_kh-1,\widetilde{m}_{i+1}h)\Phi_{P'}(\widetilde{m}_{i+1}h-1,i+1)y\|^2\right]\cr
%&&=\mathbb E\left[\mathbb E\left.\left[\|\Phi_{P'}(k,m_kh)\Phi_{P'}(m_kh-1,\widetilde{m}_{i+1}h)\Phi_{P'}(\widetilde{m}_{i+1}h-1,i+1)y\|^2\right|\mathcal F(m_kh^2-1)\right]\right]\cr
&&~~\leq q_2\mathbb E\left[\|\Phi_{P'}(m_kh-1,\widetilde{m}_{i+1}h)\Phi_{P'}(\widetilde{m}_{i+1}h-1,i+1)y\|^2\right]\cr
&&~~=q_2\mathbb E\left[\E\left.\left[\|\Phi_{P'}(m_kh-1,\widetilde{m}_{i+1}h)\Phi_{P'}(\widetilde{m}_{i+1}h-1,i+1)y\|^2\right|\F(\widetilde{m}_{i+1}h-1)\right]\right]\cr
&&~~\leq q_1q_2\E\left[\|\Phi_{P'}(\widetilde{m}_{i+1}h-1,i+1)y\|^2\right]\cr
&&~~=q_1q_2\E\left[\E\left.\left[\|\Phi_{P'}(\widetilde{m}_{i+1}h-1,i+1)y\|^2\right|\F((i+1)h-1)\right]\right]\cr
&&~~\leq q_1q_2^2\E\left[\|y\|^2\right],~0\leq i<k-3h.
\ena
Secondly, it follows from (\ref{leisileisi}) that
\bna\label{kjifw}
&&~~~~\mathbb E\left[\|\Phi_{P'}(k,i+1)y\|^2\right]\cr
&&=\mathbb E\left[\mathbb E\left.\left[\|\Phi_{P'}(k,i+1)y\|^2\right|\mathcal F((i+1)h-1)\right]\right]
 \leq q_2\E\left[\|y\|^2\right],~k-h\leq i<k.
\ena
From (\ref{leisileisi}) and (\ref{kjifw}), it is known that
\bna\label{ikdjdf}
&&~~~~\mathbb E\left[\|\Phi_{P'}(k,i+1)y\|^2\right]\cr
&&=\mathbb E\left[\mathbb E\left.\left[\|\Phi_{P'}(k,i+1)y\|^2\right|\mathcal F((i+1)h-1)\right]\right]\cr
&&=\mathbb E\left[\mathbb E\left.\left[\|\Phi_{P'}(k,k-h+1)\Phi_{P'}(k-h,i+1)y\|^2\right|\mathcal F((i+1)h-1)\right]\right]\cr
%&&=\mathbb E\left[\mathbb E\left.\left[\mathbb E\left.\left[\|\Phi_{P'}(k,k-h+1)\Phi_{P'}(k-h,i+1)y\|^2\right|\mathcal F((k-h+1)h-1)\right]\right|\mathcal F((i+1)h-1)\right]\right]\cr
&&\leq q_2\mathbb E\left[\mathbb E\left.\left[\|\Phi_{P'}(k-h,i+1)y\|^2\right|\mathcal F((i+1)h-1)\right]\right]\cr
&&\leq q_2^2\E\left[\|y\|^2\right],~k-2h\leq i<k-h.
\ena
Finally, it follows from (\ref{leisileisi}) and (\ref{ikdjdf}) that
\bna\label{wdkkdd}
&&~~~~\mathbb E\left[\|\Phi_{P'}(k,i+1)y\|^2\right]\cr
&&=\mathbb E\left[\mathbb E\left.\left[\|\Phi_{P'}(k,i+1)y\|^2\right|\mathcal F((i+1)h-1)\right]\right]\cr
&&=\mathbb E\left[\mathbb E\left.\left[\|\Phi_{P'}(k,k-h+1)\Phi_{P'}(k-h,i+1)y\|^2\right|\mathcal F((i+1)h-1)\right]\right]\cr
%&&=\mathbb E\left[\mathbb E\left.\left[\mathbb E\left.\left[\|\Phi_{P'}(k,k-h+1)\Phi_{P'}(k-h,i+1)y\|^2\right|\mathcal F((k-h+1)h-1)\right]\right|\mathcal F((i+1)h-1)\right]\right]\cr
&&\leq q_2\mathbb E\left[\mathbb E\left.\left[\|\Phi_{P'}(k-h,i+1)y\|^2\right|\mathcal F((i+1)h-1)\right]\right]\cr
&&\leq q_2^3\E\left[\|y\|^2\right],~k-3h\leq i<k-2h.
\ena
Combining (\ref{kjifw})-(\ref{wdkkdd}), we get
\bna\label{llcck}
\mathbb E\left[\|\Phi_{P'}(k,i+1)y\|^2\right]\leq \max\left\{q_2,q_2^2,q_2^3\right\}\E\left[\|y\|^2\right],~0<k-i\leq 3h.
\ena
Denote $d_3=\max\{q_1q_2^2,q_2,q_2^2,q_2^3\}$. By (\ref{jjkklkkklll}) and (\ref{llcck}), we have
$
\sup_{k\ge 0}\mathbb E\left[\|\Phi_{P'}(k,i+1)y\|^2\right]\leq d_3\E\left[\|y\|^2\right],~\forall\ i\ge 0.
$
\end{proof}



\begin{lemma}\label{henandelemma}
For the algorithm (\ref{algorithm}), suppose that  Assumptions \ref{assumption1}, \ref{assumption2}, Conditions \ref{condition1} and \ref{condition2} hold, there exists an integer $h>0$, a constant $\rho_0>0$, a strictly positive self-adjoint operator $\HH\in \mathscr L(\mathscr X^N)$ and a nonnegative real sequence $\{c(k),k\ge 0\}$, respectively, satisfying the following conditions: \\
(i) for any given $L_2$-bounded adaptive sequence $\{x(k),\F(kh-1),k\ge 0\}$ with values in the Hilbert space $\X^N$,
\ban \lim_{k\to\infty}\sum_{i=m}^{k}\left(\E\left[\left\|\left(\prod_{j=i+1}^k(I_{\X^N}-c(j)\HH)\right)\mu(i)\right\|^2\right]\right)^{\frac{1}{2}}=0,~\forall\ m\in \mathbb N,\quad \sum_{k=0}^{\infty}c(k)=\infty,
\ean
where $\displaystyle\mu(i):=c(i) \mathscr{H} x(i)-\sum_{s=i h}^{(i+1) h-1}\left(a(s) \mathbb{E}\left[\mathcal{H}^{*}(s) \mathcal{H}(s) x(i) \mid \mathcal{F}(i h-1)\right]+b(s)\left(\mathcal{L}_{\mathcal{G}} \otimes I_{\mathscr{X}}\right) x(i)\right)$;\\
(ii) $\displaystyle \sup_{k\ge 0}\left(\E\left.\left[\|\H^*(k)\H(k)\|^{2^{\max\{h,2\}}}\right|\F(k-1)\right]\right)^{\frac{1}{2^{\max\{h,2\}}}}\leq \rho_0~\text{a.s.}
$\\
Then the sequence of operator-valued random elements $$ \left\{I_{\X^N}-\sum_{i=kh}^{(k+1)h-1}(a(i)\H^*(i)\H(i)+b(i)\L_{\G}\otimes I_{\X}),k\ge 0\right\}$$ is $L_2^2$-stable w.r.t.  $\{\F((k+1)h-1),k\ge 0\}$.
\end{lemma}

\begin{proof}%[Proof of Lemma \ref{henandelemma}]
Let $\{x(k),\F(kh-1),k\ge 0\}$ be a $L_2$-bounded adaptive sequence with values in the Hilbert space $\X^N$. For any given integer $m\geq0$, let
\bna\label{diedaishii}
&&~~~~w(k+1)\cr
&&=\left(I_{\X^N}-\sum_{s=kh}^{(k+1)h-1}\left(a(s)\H^*(s)\H(s)+b(s)\L_{\G}\otimes I_{\X}\right)\right)w(k),~k\ge m,
\ena
where $w(m)=x(m),w(i)=0,i=0,\cdots,m-1$. It follows from Proposition \ref{nlllwwieiie}.(a)-(c) that $\{w(k),k\ge 0\}$ is a random sequence with values in the Hilbert space $(\X^N,\tau_{\text{N}}(\X^N))$. On one hand, from the definition of $w(k)$, it follows that
\bna\label{wffee}
&&~~~~w(k+1)\cr
&&=\left(\prod_{i=m}^k\left(I_{\X^N}-\sum_{s=ih}^{(i+1)h-1}(a(s)\H^*(s)\H(s)+b(s)\L_{\G}\otimes I_{\X})\right)\right)x(m),~k\ge m.
\ena
Noting that $x(m)\in L^0(\Omega,\F(mh-1);\X^N)$, it follows from Lemma \ref{hhhlemma} that there exists a constant $d_3>0$, such that
\bna\label{lsaew}
\sup_{k\ge 0}\E\left[\|w(k+1)\|^2\right]
%&&=\sup_{k\ge 0}\E\left[\left\|\left(\prod_{i=m}^k\left(I_{\X^N}-\sum_{s=ih}^{(i+1)h-1}(a(s)\H^*(s)\H(s)+b(s)\L_{\G}\otimes I_{\X})\right)\right)x(m)\right\|^2\right]\cr
\leq d_3\E\left[\|x(m)\|^2\right]
\leq d_3\sup_{k\ge 0}\E\left[\|x(k)\|^2\right]<\infty.
\ena
Moreover, (\ref{diedaishii}) can be rewritten as
\ban
w(i+1)&=&(I_{\X^N}-c(i)\HH)w(i)\cr
&&+\left(c(i)\HH-\sum_{s=ih}^{(i+1)h-1}(a(s)\H^*(s)\H(s)+b(s)\L_{\G}\otimes I_{\X})\right)w(i),~i\ge m,
\ean
which gives
\bna\label{fwwiii}
&&~~~~w(k+1)\cr
&&=\left(\prod_{i=m}^k(I_{\X^N}-c(i)\HH)\right)x(m)+\sum_{i=m}^k\left(\prod_{j=i+1}^k(I_{\X^N}-c(j)\HH)\right)\cr
&&~~~\times\left(c(i)\HH-\sum_{s=ih}^{(i+1)h-1}(a(s)\H^*(s)\H(s)+b(s)\L_{\G}\otimes I_{\X})\right)w(i),~k\ge m.
\ena
From (\ref{lsaew}) and the condition (ii), it is known that
\ban
\sup_{s\ge 0,\ i\ge 0}\E\left[\|\H^*(s)\H(s)w(i)\|\right]\leq \sup_{s\ge 0}\E\left[\|\H^*(s)\H(s)\|^2\right]+\sup_{i\ge 0}\E\left[\|w(i)\|^2\right]<\infty,
\ean
which together Proposition \ref{wenknknkn} leads to $\H^*(s)\H(s)w(i)\in L^1(\Omega,\F(s);\X^N)$, $s\ge ih$. Thus, by Lemma \ref{nvkvpeoeo}, we know that there exists a unique conditional expectation $\E[\H^*(s)\H(s)w(i)|\F(ih\\-1)]$ of $\H^*(s)\H(s)w(i)$ w.r.t. $\F(ih-1)$. Then we have
\bna\label{fewdee}
~~\Bigg(c(i)\HH-\sum_{s=ih}^{(i+1)h-1}(a(s)\H^*(s)\H(s)+b(s)\L_{\G}\otimes I_{\X})\Bigg)w(i)
%&&=c(i)\H w(i)-\sum_{s=ih}^{(i+1)h-1}(a(s)\E[\H^*(s)\H(s)w(i)|\F(ih-1)]+b(s)(\L_{\G}\otimes I_{\X})w(i))\cr
%&&~~~~+\sum_{s=ih}^{(i+1)h-1}a(s)(\E[\H^*(s)\H(s)w(i)|\F(ih-1)]-\H^*(s)\H(s)w(i))\cr
=w_1(i)+w_2(i),
\ena
where
\bna\label{mlwer}
\begin{cases}
  w_1(i)=\displaystyle c(i)\HH w(i)-\sum_{s=ih}^{(i+1)h-1}(a(s)\E[\H^*(s)\H(s)w(i)|
  \F(ih-1)] +b(s)(\L_{\G}\otimes I_{\X})w(i)),\\
  w_2(i)=\displaystyle \sum_{s=ih}^{(i+1)h-1}a(s)(\E[\H^*(s)\H(s)w(i)|\F(ih-1)]-\H^*(s)\H(s)w(i)).
\end{cases}
\ena
By (\ref{wffee}) and (\ref{fwwiii})-(\ref{fewdee}), we get
\ban
&&~~~\left(\prod_{i=m}^k\left(I_{\X^N}-\sum_{s=ih}^{(i+1)h-1}(a(s)\H^*(s)\H(s)+b(s)\L_{\G}\otimes I_{\X})\right)\right)x(m)\cr &&=\left(\prod_{i=m}^k(I_{\X^N}-c(i)\HH)\right)x(m)+\sum_{i=m}^k\left(\prod_{j=i+1}^k(I_{\X^N}-c(j)\HH)\right)(w_1(i)+w_2(i)),
\ean
which together with Cauchy inequality leads to
\bna\label{ikddww}
&&~~~~\E\left[\Bigg\|\Bigg(\prod_{i=m}^k\Bigg(I_{\X^N}-\sum_{s=ih}^{(i+1)h-1}(a(s)\H^*(s)\H(s)+b(s)\L_{\G}\otimes I_{\X})\Bigg)\Bigg)x(m)\Bigg\|^2\right]\cr
&&\leq 2\E\left[\Bigg\|\Bigg(\prod_{i=m}^k(I_{\X^N}-c(i)\HH)\Bigg)x(m)\Bigg\|^2\right]\cr &&~~~+2\E\left[\left\|\sum_{i=m}^k\left(\prod_{j=i+1}^k(I_{\X^N}-c(j)\HH)\right)(w_1(i)+w_2(i))\right\|^2\right],~k\ge m.
\ena
By Lemma \ref{lemmaA7}, it is known that there exist constants $M,d>0$, such that
\ban
\left\|\left(\prod_{i=m}^k(I_{\X^N}-c(i)\HH)\right)x(m)\right\|^2\leq M^{2d}\|x(m)\|^2~\text{a.s.}
\ean
Noting that $\sup_{k\ge 0}\E[\|x(k)\|^2]<\infty$, it follows from Lebesgue dominated convergence theorem and Lemma \ref{lemmaA7} that
\bna\label{oo1}
\lim_{k\to\infty}\E\left[\left\|\left(\prod_{i=m}^k(I_{\X^N}-c(i)\HH)\right)x(m)\right\|^2\right]=0,~\forall\ m\ge 0.
\ena
By using Cauchy inequality again, we obtain
\bna\label{oooff}
&&~~~~\E\left[\left\|\sum_{i=m}^k\left(\prod_{j=i+1}^k(I_{\X^N}-c(j)\HH)\right)(w_1(i)+w_2(i))\right\|^2\right]\cr
&&\leq 2\E\left[\left\|\sum_{i=m}^k\left(\prod_{j=i+1}^k(I_{\X^N}-c(j)\HH)\right)w_1(i)\right\|^2\right]\cr &&~~~+2\E\left[\left\|\sum_{i=m}^k\left(\prod_{j=i+1}^k(I_{\X^N}-c(j)\HH)\right)w_2(i)\right\|^2\right].
\ena
We now consider the right-hand side of (\ref{oooff}) term by term. Firstly, by Minkowski inequality, we get
\bna\label{oolkf}
&&~~~~\E\left[\left\|\sum_{i=m}^k\left(\prod_{j=i+1}^k(I_{\X^N}-c(j)\HH)\right)w_1(i)\right\|^2\right]\cr
&&\leq \left(\sum_{i=m}^k\left(\E\left[\left\|\left(\prod_{j=i+1}^k(I_{\X^N}-c(j)\HH)\right)w_1(i)\right\|^2\right]\right)^{\frac{1}{2}}\right)^2.
\ena
It follows from (\ref{diedaishii}) and (\ref{lsaew}) that $\{w(k),\F(kh-1),k\ge 0\}$ is an adaptive sequence with $\sup_{k\ge 0}\E[\|w(k)\|^2]<\infty$, which together with (\ref{mlwer}), the condition (i) and (\ref{oolkf}) gives
\bna\label{vspppw}
\lim_{k\to \infty}\E\left[\left\|\sum_{i=m}^k\left(\prod_{j=i+1}^k(I_{\X^N}-c(j)\HH)\right)w_1(i)\right\|^2\right]=0.
\ena
Secondly, noting that $w_2(i)\in \F((i+1)h-1)$, $w(i)\in \F(ih-1)$, by Condition \ref{condition1}, it follows from the condition (ii) and (\ref{lsaew}) that
\bna\label{fuckd}
&&~~~~\E\left[\|w_2(i)\|^2\right]\cr
&&= \E\left[\left\|\sum_{s=ih}^{(i+1)h-1}a(s)(\E[\H^*(s)\H(s)w(i)|\F(ih-1)]-\H^*(s)\H(s)w(i))\right\|^2\right]\cr
&&\leq ha^2(i)\E\Bigg[\sum_{s=ih}^{(i+1)h-1}\|\E[\H^*(s)\H(s)w(i)|\F(ih-1)]
-\H^*(s)\H(s)w(i)\|^2\Bigg]\cr
&&\leq 2ha^2(i)\E\Bigg[\sum_{s=ih}^{(i+1)
h-1}\big(\E\big[\|\H^*(s)\H(s)w(i)\|^2|\F(ih-1)\big]
 +\|\H^*(s)\H(s)w(i)\|^2\big)\Bigg]\cr
&&\leq 2ha^2(i)\E\Bigg[\sum_{s=ih}^{(i+1)h-1}\left(\rho_0^2\|w(i)\|^2
+\|\H^*(s)\H(s)w(i)\|^2\right)\Bigg]\cr
&&\leq 2ha^2(i)\E\Bigg[\|w(i)\|^2\sum_{s=ih}^{(i+1)h-1}\left(\rho_0^2
+\|\H^*(s)\H(s)\|^2\right)\Bigg]\cr
&&= 2ha^2(i)\E\Bigg[\E\Bigg[\|w(i)\|^2\sum_{s=ih}^{(i+1)h-1}
\left(\rho_0^2+\|\H^*(s)\H(s)\|^2\right) \Bigg|\F(ih-1)\Bigg]\Bigg]\cr
&&= 2ha^2(i)\E\Bigg[\E\Bigg[\sum_{s=ih}^{(i+1)h-1}\left(\rho_0^2
+\|\H^*(s)\H(s)\|^2\right)\Bigg|\F(ih-1)\Bigg]\|w(i)\|^2\Bigg]\cr
&&\leq 4h^2\rho_0^2a^2(i)\E\left[\|w(i)\|^2\right],
\ena
which together with Condition \ref{condition2} gives
\bna\label{vlllls}
\sup_{i\ge 0}\E\left[\|w_2(i)\|^2\right]<\infty.
\ena
Thus, it follows from Lemma \ref{nvkvpeoeo} that $\E[w_2(i)|\F(ih-1)]$ exists and
\bna\label{iwsf}
&&~~~~\E[w_2(i)|\F(ih-1)]\cr
&&=\E\Bigg[\sum_{s=ih}^{(i+1)h-1}a(s)(\E[\H^*(s)\H(s)w(i)|\F(ih-1)] -\H^*(s)\H(s)w(i))\bigg|\F(ih-1)\Bigg]\cr
&&=\sum_{s=ih}^{(i+1)h-1}a(s)\E[\E[\H^*(s)\H(s)w(i)|\F(ih-1)]-\H^*(s)
\H(s)w(i)|\F(ih-1)] \cr
&& =0.
\ena
Meanwhile, from Lemma \ref{lemmaA7}, it is known that there exist constants $M,d>0$ such that
\begin{align}\label{nvweefff}
 &\sup_{\|x\|=1\atop x\in \X^N}\inf\left\{r\ge 0:\P\left(\Bigg\|\Bigg(\prod_{j=t+1}^k(I_{\X^N}-c(j)\HH)\Bigg)x\Bigg\|
 <r\right)=1\right\}\notag\\
 \leq &  \sup_{\|x\|=1\atop x\in \X^N}M^d\|x\|<\infty.
\end{align}
For $m\leq s<t\leq k$, it follows from Proposition \ref{wenknknkn}, Proposition 2.6.13 in \cite{hy}, Lemma 3.5.2 in \cite{hy}, Proposition \ref{lemmaA6} and (\ref{vlllls})-(\ref{nvweefff}) that
\bna\label{ffkk}
~~~~\E\left[\left\langle \left(\prod_{j=s+1}^k(I_{\X^N}-c(j)\HH)\right)w_2(s),\left(\prod_{j=t+1}^k(I_{\X^N}-c(j)\HH)\right)w_2(t)\right\rangle\right]
%&&=\E\left[\E\left.\left[\left\langle \left(\prod_{j=s+1}^k(I_{\X^N}-c(j)\H)\right)w_2(s),\left(\prod_{j=t+1}^k(I_{\X^N}-c(j)\H)\right)w_2(t)\right\rangle\right|\F(th-1)\right]\right]\cr
%&&=\E\left[\left\langle \left(\prod_{j=s+1}^k(I_{\X^N}-c(j)\H)\right)w_2(s),\E\left.\left[\left(\prod_{j=t+1}^k(I_{\X^N}-c(j)\H)\right)w_2(t)\right|\F(th-1)\right]\right\rangle\right]\cr
%&&=\E\left[\left\langle \left(\prod_{j=s+1}^k(I_{\X^N}-c(j)\H)\right)w_2(s),\left(\prod_{j=t+1}^k(I_{\X^N}-c(j)\H)\right)\E[w_2(t)|\F(th-1)]\right\rangle\right]\cr
=0.
\ena
On one hand, by Lemma \ref{lemmaA7}, it is known that there exist positive constants $M$ and $d$, such that
\bna\label{wiiiwww}
\left\|\left(\prod_{j=i+1}^k(I_{\X^N}-c(j)\HH)\right)\frac{1}{a(i)}w_2(i)\right\|\leq M^d\left\|\frac{1}{a(i)}w_2(i)\right\|~\text{a.s.}
\ena
Thus, by Lemma \ref{lemmaA7} and (\ref{ffkk})-(\ref{wiiiwww}), we get
\begin{align}\label{oolll}
&\E\left[\left\|\sum_{i=m}^k\left(\prod_{j=i+1}^k(I_{\X^N}-c(j)\HH)\right)w_2(i)\right\|^2\right]\cr
 =&\sum_{i=m}^k\E\left[\left\|\left(\prod_{j=i+1}^k(I_{\X^N}-c(j)\HH)\right)
 w_2(i)\right\|^2\right]\cr & +2\sum_{m\leq s<t\leq k}\E\Bigg[\Bigg\langle \left(\prod_{j=s+1}^k(I_{\X^N}-c(j)\HH)\right)w_2(s), \left(\prod_{j=t+1}^k(I_{\X^N}-c(j)\HH)\right)w_2(t)\Bigg\rangle\Bigg]\cr
 =&\sum_{i=m}^k\E\left[\left\|\left(\prod_{j=i+1}^k(I_{\X^N}-c(j)\HH)\right)w_2(i)\right\|^2\right]\cr
=&\sum_{i=m}^ka^2(i)\E\left[\left\|\left(\prod_{j=i+1}^k(I_{\X^N}-c(j)\HH)\right)\frac{1}{a(i)}w_2(i)\right\|^2\right]\cr
 \leq& M^d\sum_{i=m}^ka^2(i)
\left[\E\left[\left\|\left(\prod_{j=i+1}^k(I_{\X^N}-c(j)\HH)\right)
\frac{1}{a(i)}w_2(i)\right\|^2\right]\right]^{\frac{1}{2}} \cr & \times\left[\E\left[\left\|\frac{1}{a(i)}w_2(i)\right\|^2\right]\right]^{\frac{1}{2}}.
\end{align}
%\bna\label{oolll}
%&&~~~~\E\left[\left\|\sum_{i=m}^k\left(\prod_{j=i+1}^k(I_{\X^N}-c(j)\HH)\right)w_2(i)\right\|^2\right]\cr
%&&=\sum_{i=m}^k\E\left[\left\|\left(\prod_{j=i+1}^k(I_{\X^N}-c(j)\HH)\right)w_2(i)\right\|^2\right]\cr &&~~~~+2\sum_{m\leq s<t\leq k}\E\Bigg[\Bigg\langle \left(\prod_{j=s+1}^k(I_{\X^N}-c(j)\HH)\right)w_2(s),\cr &&~~~~\left(\prod_{j=t+1}^k(I_{\X^N}-c(j)\HH)\right)w_2(t)\Bigg\rangle\Bigg]\cr
%&&=\sum_{i=m}^k\E\left[\left\|\left(\prod_{j=i+1}^k(I_{\X^N}-c(j)\HH)\right)w_2(i)\right\|^2\right]\cr
%&&=\sum_{i=m}^ka^2(i)\E\left[\left\|\left(\prod_{j=i+1}^k(I_{\X^N}-c(j)\HH)\right)\frac{1}{a(i)}w_2(i)\right\|^2\right]\cr
%&&\leq M^d\sum_{i=m}^ka^2(i)
%\left[\E\left[\left\|\left(\prod_{j=i+1}^k(I_{\X^N}-c(j)\HH)\right)
%\frac{1}{a(i)}w_2(i)\right\|^2\right]\right]^{\frac{1}{2}}\cr &&~~~~\times\left[\E\left[\left\|\frac{1}{a(i)}w_2(i)\right\|^2\right]\right]^{\frac{1}{2}}.
%\ena
On the other hand, by (\ref{lsaew}) and (\ref{fuckd}), we have
\bna\label{ofskj}
&&\sup_{i\ge 0}\mathbb E\left[\left\|\frac{1}{a(i)}w_2(i)\right\|^2\right]\cr
&\leq & 4h^2\rho_0^2\sup_{i\ge 0}\E\left[\|w(i)\|^2\right]<\infty.
\ena
Substituting (\ref{ofskj}) into (\ref{oolll}) gives
\ban
&&~~~~\E\left[\left\|\sum_{i=m}^k\left(\prod_{j=i+1}^k(I_{\X^N}-c(j)\HH)\right)w_2(i)\right\|^2\right]\cr
&&\leq M^d\sup_{i\ge 0}\left[\E\left[\left\|\frac{1}{a(i)}w_2(i)\right\|^2\right]\right]^{\frac{1}{2}}\cr
&&~~~~\times\sum_{i=m}^ka^2(i)\left(\E\left[\left\|\left(\prod_{j=i+1}^k(I_{\X^N}-c(j)\HH)\right)\frac{1}{a(i)}w_2(i)\right\|^2\right]\right)^{\frac{1}{2}},
\ean
which together with Condition \ref{condition2}, (\ref{ofskj}) and Lemma \ref{lemmaA8} leads to
\bna\label{iikdd}
\lim_{k\to\infty}\E\left[\left\|\sum_{i=m}^k\left(\prod_{j=i+1}^k(I_{\X^N}-c(j)\HH)\right)w_2(i)\right\|^2\right]=0,~\forall\ m\ge 0.
\ena
Hence, substituting (\ref{oo1})-(\ref{oooff}), (\ref{vspppw}) and (\ref{iikdd}) into (\ref{ikddww}) completes the proof of Lemma \ref{henandelemma}.
\end{proof}


\end{appendices}


\begin{thebibliography}{99}
\bibitem{bertero}
 M. Bertero and  P. Boccacci, {\it Introduction to Inverse Problems in Imaging}. Boca Raton: CRC Press, 1998.

%\bibitem{cakoni}
%G. Cakoni and D. Colton, ``Open problems in the qualitative approach to inverse electromagnetic scattering theory,'' {\it Eur. J. Appl. Math.}, vol. 16, pp. 411--425, 2005.

\bibitem{colton}
D. Colton and  R. Kress, {\it Inverse Acoustic and Electromagnetic Scattering Theory}. Berlin: Springer Science and Business Media, 2012.
%\bibitem{groetsch}
%C. W. Groetsch, \emph{Inverse problems in the mathematical sciences}, Vieweg Mathematics for Scientists and Engineers. Friedr. Vieweg and Sohn, Braunschweig, 1993.
%\bibitem{isakov}
%V. Isakov, ``On inverse problems in secondary oil recovery,'' {\it Eur. J. Appl. Math.}, vol. 19, pp. 459--478, 2008.

%\bibitem{tarantola}
%A. Tarantola and B. Valette, \emph{Inverse problems=quest for information}, J. geophys, 50 (1982), pp. 150–170.
%\bibitem{engl}
% H. W. Engl,  A. K. Louis   and A. K. Rundell, ``Inverse problems in medical imaging and nondestructive testing,'' in {\it Proc.  Conference in Oberwolfach}, Oberwolfach,  Federal Republic of Germany,  Feb. 4--10, 1996.

%\bibitem{Kirsch111}
%A. Kirsch, {\it An Introduction to the Mathematical Theory of Inverse Problems}. Berlin: Springer, 1996.

\bibitem{Engl123}
 H. W. Engl,  M. Hanke and A. Neubauer, {\it Regularization of Inverse Problems}. Berlin: Springer Science and Business Media, 1996.

\bibitem{Morozov123}
V. A. Morozov, {\it Methods for Solving Incorrectly Posed Problems}. Berlin: Springer Science and Business Media, 2012.

\bibitem{Werschulz}
A. G. Werschulz, {\it The Computational Complexity of Differential and Integral Equations: An Information-Based Approach}. Oxford: Oxford University Press, 1991.

\bibitem{Tikhonov123}
A. N. Tikhonov, ``Regularization of incorrectly posed problems,'' {\it Soviet Mathematics Doklady}, vol. 4, pp.  1624--1627, 1963.

\bibitem{Bissantz}
 N. Bissantz, T. Hohage and A. Munk, ``Consistency and rates of convergence of nonlinear Tikhonov regularization with random noise,'' {\it  Inverse Probl.}, vol. 20, no. 6,  pp. 1773, 2004.

%\bibitem{Bissantz1}
%N. Bissantz, T. Hohage, A. Munk, and F. Ruymgaart, \emph{Convergence rates of general regularization methods for statistical inverse problems and applications}, SIAM Journal on Numerical Analysis, 45 (2007), pp. 2610–2636.

\bibitem{Cavalier}
L. Cavalier, ``Nonparametric statistical inverse problems,'' {\it Inverse Probl.}, vol. 24, no. 3, pp.   034004.
\bibitem{Kekkonen}
H. Kekkonen, M. Lassas and S. Siltanen, ``Analysis of regularized inversion of data corrupted by white Gaussian noise,'' {\it Inverse Probl.}, vol. 30, no. 4, pp. 045009, 2014.

\bibitem{Gine}
E. Gine and  R. Nickl, {\it Mathematical Foundations of Infinite-Dimensional Statistical Models}. Cambridge: Cambridge University Press, 2015.

\bibitem{Hohage}
T. Hohage and  F. Werner, ``Inverse problems with Poisson data: Statistical regularization theory, applications and algorithms,'' {\it Inverse Probl.}, vol. 32, no. 9, pp. 093001, 2016.

%\bibitem{Janz}
%D. Janz,  D. Burt,   and   J.  Gonzalez, ``Bandit optimisation of functions in the Matern kernel RKHS,'' in {\it Proc.   23rd AISTATS}, Online, Aug. 26-28, 2020, pp. 2486--2495.
%
%\bibitem{Takemori}
% S. Takemori   and M. Sato, ``Approximation theory based methods for RKHS bandits,'' in {\it Proc. 38th ICML}, Online, July. 18-24, 2021, pp. 10076-10085.



\bibitem{Lu}
S. Lu and   P. Mathe, ``Discrepancy based model selection in statistical inverse problems,'' {\it J. Complex.}, vol. 30, no. 3, pp. 290--308, 2014.

\bibitem{Math1}
P. Mathe and S. Pereverzev, ``Regularization of some linear ill-posed problems with discretized random noisy data,'' {\it Math. Comput.}, vol. 75, no. 256, pp. 1913--1929, 2006.




\bibitem{Iglesias1}
M. A. Iglesias, K. Lin, S. Lu and A. M. Stuart, ``Filter based methods for statistical linear inverse problems,'' {\it Commun. Math. Sci.},
vol. 15, no. 7, pp. 1867--1896, 2017.


\bibitem{Lu1}
 S. Lu, P. Niu and F. Werner, ``On the asymptotical regularization for linear inverse problems in presence of white noise,'' { \it SIAM-ASA J. Uncertain. Quantif.}, vol. 9, no. 1, pp. 1--28, 2021.




\bibitem{Lu100}
S. Lu and P. Mathe, ``Stochastic gradient descent for linear inverse problems in Hilbert spaces,'' {\it Math. Comput.}, vol. 91, no. 336, pp. 1763--1788, 2017.
\bibitem{Lu2}
S. Lu,  P. Mathe and S. V. Pereverzev,, ``Randomized matrix approximation to enhance regularized projection schemes in inverse problems,'' {\it Inverse Probl.}, vol. 36, no. 8, pp. 085013, 2020.
\bibitem{Jonesfg}
F. G. Jones  and G. Simpson,  ``Iterate averaging, the Kalman filter, and 3DVAR for linear inverse problems,'' {\it Numer. Algorithms}, vol. 92, no. 2, pp.  1--21, 2022.


\bibitem{Mathe123}
 P. Mathe  and S. V. Pereverzev, ``Optimal discretization of inverse problems in Hilbert scales. Regularization and self-regularization of projection methods,'' {\it SIAM J. Numer. Anal.}, vol. 38, no. 6 , pp. 1999--2021, 2001.
 \bibitem{Jahn}
T. Jahn and B. Jin, ``On the discrepancy principle for stochastic gradient descent,''
{\it  Inverse Probl.}, vol. 36, no. 9, pp. 095009, 2020.
\bibitem{JinB}
 B. Jin and X. Lu, ``On the regularizing property of stochastic gradient descent,'' {\it  Inverse Probl.}, vol. 35, no. 1, pp. 015004, 2018.
 \bibitem{Rosenblatt}
J. D. Rosenblatt and B. Nadler, ``On the optimality of averaging in distributed statistical learning,'' {\it Inf. Inference}, vol. 5, no. 4, pp. 379--404, 2016.
%\bibitem{Ippel}
%L. Ippel,  M. Kaptein   and J. Vermunt,  ``Dealing with data streams: An online, row-by-row, estimation tutorial,'' {\it Methodology (Gott.)}, vol. 12, no. 4, pp. 124, 2016.


\bibitem{Lopes}
 C. G. Lopes  and  A. H. Sayed, ``Diffusion least-mean squares over adaptive networks: Formulation and performance analysis,'' {\it IEEE Trans. Signal Process.},  vol. 56, no. 7, pp. 3122--3136, 2008.
 \bibitem{kar2011}
S. Kar and J. M. F. Moura,  ``Convergence rate analysis of distributed gossip (linear parameter) estimation: Fundamental limits and tradeoff,'' {\it IEEE J. Sel. Top. Signal Process.}, vol. 5, no. 4, pp.  674--690, 2011.
\bibitem{kar2011}
S. Kar and J. M. F. Moura,  ``Convergence rate analysis of distributed gossip (linear parameter) estimation: Fundamental limits and tradeoff,'' {\it IEEE J. Sel. Top. Signal Process.}, vol. 5, no. 4, pp.  674--690, 2011.
\bibitem{Cattivelli}
 F. S. Cattivelli and A. H. Sayed,  ``Diffusion strategies for distributed Kalman filtering and smoothing,'' {\it IEEE Trans. Autom. Control}, vol. 55, no. 9, pp. 2069--2084, 2010.
\bibitem{Sayed}
S. Al-Sayed,    A. M. Zoubir  and A. H. Sayed,  ``Robust distributed estimation by networked agents,'' {\it IEEE Trans. Signal Process.}, vol. 65, no. 15, pp.  3909--3921, 2017.
\bibitem{Gholami}
 M. R. Gholami, M. Jansson,  E. G. Str\"{o}m and  A. H. Sayed, ``Diffusion estimation over cooperative multi-agent networks with missing data,'' {\it IEEE Trans. Signal Inf. Proc. Netw.}, vol. 2, no. 3, pp. 276--289, 2016.
\bibitem{Abdolee}
R. Abdolee,   B. Champagne   and  A. H. Sayed, ``Diffusion adaptation over multi-agent networks with wireless link impairments,'' {\it IEEE. Trans. Mob. Comput.}, vol. 15, no. 6, pp. 1362--1376, 2016.
%\bibitem{kar20112}
%S. Kar   and J. M. F. Moura,  ``Gossip and distributed Kalman filtering: Weak consensus under weak detectability,'' {\it IEEE Trans. Signal Process.}, vol. 59, no. 4, pp. 1766--1786, 2011.
\bibitem{kar2012}
S. Kar,  J. M. F. Moura and K. Ramanan,  ``Distributed parameter estimation in sensor networks: Nonlinear observation models and imperfect communication,'' {\it   IEEE Trans. Inf. Theory}, vol. 58, no. 6, pp. 3575--3605, 2012.
\bibitem{kar20132}
S. Kar    and J. M. F. Moura,  ``Consensus+innovations distributed inference over networks: Cooperation and sensing in  networked systems,'' {\it  IEEE Signal Process. Mag.}, vol. 30, no. 3, pp. 99--109, 2013.
%\bibitem{sahu2016}
%\textsc{Sahu, A. K.}, \textsc{Kar, S.}, \textsc{Moura, J. M. F.} and \textsc{Poor, H. V.} (2016). Distributed constrained recursive nonlinear least-squares estimation: algorithms and asymptotics. \textit{IEEE Trans. Signal and Information Processing over Networks.} \textbf{2} 426--441.
\bibitem{Piggott}
M. J. Piggott and V. Solo,  ``Diffusion LMS with correlated regressors I: Realization-wise stability,'' {\it IEEE Trans. Signal Process.}, vol. 64, no. 21, pp.  5473--5484, 2016.
\bibitem{Piggott1}
M. J. Piggott and   V. Solo, ``Diffusion LMS with correlated regressors II: Performance,'' {\it IEEE Trans. Signal Process.}, vol. 65, no. 15,  pp. 3934-3947, 2017.
\bibitem{WLZ}
J. Wang,  T. Li  and X. Zhang,  ``Decentralized cooperative online estimation with random observation matrices, communication graphs and time delays,'' {\it IEEE Trans. Inf. Theory}, vol. 67, no. 6, pp. 4035--4059, 2021.
\bibitem{Ishihara}
 J. Y. Ishihara and S. C. Alghunaim, ``Diffusion LMS filter for distributed estimation of systems with stochastic state transition and observation matrices,'' in {\it Proc. 2017 Am. Control Conf.}, Seattle, USA,  May. 24-26, 2017, pp. 5199--5204.

 \bibitem{Lyaqini}
S. Lyaqini, M. Quafafou,   M. Nachaoui  and  A. Chakib, ``Supervised learning as an inverse problem based on non-smooth loss function,'' {\it Knowl. Inf. Syst.}, vol. 62, no. 8, pp.  3039--3058, 2020.
\bibitem{Poggio}
S. Smale, F. Cucker, ``On the mathematical foundations of learning,'' {\it Bull. Amer. Math. Soc.}, vol. 39, no.1 , pp. 1--49, 2002.
\bibitem{Ying}
Y. Ying and  M. Pontil, ``Online gradient descent learning algorithms,'' {\it Found. Comput. Math.}, vol. 8, pp. 561--596, 2008.
\bibitem{Tarres}
P. Tarres  and  Y. Yao, ``Online learning as stochastic approximation of regularization paths: Optimality and almost-sure convergence,'' {\it IEEE Trans. Inf. Theory}, vol. 60, no. 9, pp. 5716--5735, 2014.

\bibitem{Dieuleveut}
A. Dieuleveut and F. Bach, ``Nonparametric stochastic approximation with large step-sizes,'' {\it Ann. Stat.},  vo. 44, no. 4, pp. 1363--1399, 2016.
\bibitem{Shin}
B. Shin,   M. Yukawa,   R. L. G. Cavalcante  and A. Dekorsy, ``Distributed adaptive learning with multiple kernels in diffusion networks,'' {\it IEEE Trans. Signal Process.}, vol. 66, no. 21,  pp. 5505--5519,  2018.
\bibitem{Deng}
 Z. Deng, J. Gregory  and A. Kurdila, ``Learning theory with consensus in reproducing kernel Hilbert spaces,'' in {\it Proc. 2012 Am. Control Conf.}, Montreal,  Canada, Jun. 27-29, 2012, pp. 1400--1405.


%\bibitem{Mathé}
%Mathé, P., \& Pereverzev, S. V. (2001). \emph{Optimal discretization of inverse problems in Hilbert scales. Regularization and self-regularization of projection methods}. SIAM Journal on Numerical Analysis, 38(6), 1999-2021.









\bibitem{Mitra}
 R. Mitra and V. Bhatia, ``The diffusion-KLMS algorithm,'' {\it Proc. 2014  ICCIT}, Bhubaneswar, India, Dec. 22-24, 2014, pp. 256--259.

\bibitem{Chouvardas}
R. Chouvardas  and  M. Draief, ``A diffusion kernel LMS algorithm for nonlinear adaptive networks,'' in {\it Proc. 2016 IEEE Int. Conf. Acoust. Speech Signal Process.}, Shanghai, China,  Mar. 20-25, 2016, pp. 4164--4168.
\bibitem{Bouboulis}
  P. Bouboulis, S. Chouvardas and  S. Theodoridis,  ``Online distributed learning over networks in RKHS spaces using random Fourier features,'' {\it IEEE Trans. Signal Process.},   vol. 66, no. 7, pp. 1920--1932, 2018.
\bibitem{ctfg}
X. Chen,   B. Tang, J. Fan and  X. Guo, ``Online gradient descent algorithms for functional data learning,'' {\it J. Complex.}, vol. 70, pp. 101635, 2022.

\bibitem{Lei}
Y. Lei, L. Shi and Z. C. Guo, ``Convergence of unregularized online learning algorithms,'' {\it J. Mach. Learn. Res.}, vo. 18, no. 171, pp. 1-33, 2018.
\bibitem{GUO}
Z. C. Guo,  S. B. Lin    and L. Shi, ``Distributed learning with multi-penalty regularization,'' {\it Appl. Comput. Harmon. Anal.}, vol. 46, no. 3, pp.  478--499, 2019.


\bibitem{GUO1}
 Z. C. Guo, S. B. Lin    and D. X. Zhou, ``Learning theory of distributed spectral algorithms,'' {\it Inverse Probl.}, vol. 33, no. 7, pp. 074009, 2017.

\bibitem{Lin}
J. Lin  and V. Cevher,
``Optimal convergence for distributed learning with stochastic gradient methods and spectral algorithms,'' {\it  J. Mach. Learn. Res.}, vol. 21, no. 147, pp. 1--63,   2020.
\bibitem{SmaleYao}
S. Smale  and Y. Yao, ``Online learning algorithms,'' {\it Found. Comput. Math.}, vol. 6, pp. 145--170, 2006.

%\bibitem{ZLG}
% X. Zhang, T. Li  and Y. Gu, ``Consensus+ innovations distributed estimation with random network graphs, observation matrices and noises,'' in {\it Proc. 59th IEEE Conf. Decis. Control}, Online, Dec. 14-18, 2020, pp. 4318--4323.
\bibitem{Ungureanu1}
 V. M. Ungureanu and  S. S. Cheng, ``Mean stability of a stochastic difference equation,'' {\it Ann. Pol. Math.}, vol. 1, no. 93, pp. 33--52, 2008.
%Institute of Mathematics Polish Academy of Sciences.
\bibitem{Kubrusly}
S. C. Kubrusly, ``Applied stochastic approximation algorithms in Hilbert space,'' {\it Int. J. Control}, vol. 28, no. 1, pp. 23--31, 1978.

\bibitem{Vajjha}
K. Vajjha,   B. Trager,  A. Shinnar and V. Pestun,  ``Formalization of a stochastic approximation theorem,''   arXiv:2202.05959, 2022.

\bibitem{Ungureanu2}
 V. M. Ungureanu, ``Stability, stabilizability and detectability for Markov jump discrete-time linear systems with multiplicative noise in Hilbert spaces,'' {\it Optimization}, vol. 63, no. 11, pp. 1689--1712, 2014.


\bibitem{Ungureanu3}
V. M. Ungureanu, ``Optimal control for linear discrete-time systems with Markov perturbations in Hilbert spaces,'' {\it IMA J. Math. Control Inf.}, vol. 26, no. 1, pp. 105--127, 2009.


\bibitem{zwzwxcbs}
W. Zhang,   W. X. Zheng  and B. S. Chen, ``Detectability, observability and Lyapunov-type theorems of linear discrete time-varying stochastic systems with multiplicative noise,'' {\it Int. J. Control}, vol. 90, no. 11, pp. 2490--2507, 2017.


\bibitem{sl}
U. Schmitt and  A. K. Louis, ``Efficient algorithms for the regularization of dynamic inverse problems: I. Theory,'' {\it Inverse Probl.}, vol. 18, no. 3, pp. 6, 2002.





\bibitem{Benning}
M. Benning  and  M. Burger, ``Modern regularization methods for inverse problems,'' {\it Acta Numer.}, vol. 27, pp. 1--111, 2018.


\bibitem{Reich3}
S. Reich  and  A. J. Zaslavski, ``Convergence of generic infinite products of affine operators,'' {\it Abstract Appl. Anal.}, vol. 4, no. 1, pp. 1--19, 1999.


\bibitem{Pustylnik2}
E. Pustylnik    and S. Reich, ``Infinite products of arbitrary operators and intersections of subspaces in Hilbert space,'' {\it J. Approx. Theory}, vol. 178, pp. 91--102,
 2014.

\bibitem{Pustylnik3}
E. Pustylnik,  S. Reich    and  A. J. Zaslavski, ``Convergence of non-cyclic infinite products of operators,'' {\it J. Math. Anal. Appl.}, vol. 380, no. 2, pp. 759--767, 2011.

\bibitem{Pustylnik4}
E. Pustylnik,  S. Reich   and  A. J. Zaslavski, ``Convergence of non-periodic infinite products of orthogonal projections and nonexpansive operators in Hilbert space,'' {\it J. Approx. Theory}, vol. 164, no. 5, pp. 611--624, 2012.


\bibitem{Guo1994}
L. Guo, ``Stability of recursive stochastic tracking algorithms,'' {\it SIAM J. Control Optim.}, vol. 32, no. 5, pp. 1195--1225,
 1994.

\bibitem{Guo1990}
L. Guo, ``Estimating time-varying parameters by the Kalman filter based algorithm: stability and convergence,'' {\it IEEE Trans. Autom. Control}, vol. 35, no. 2,  pp. 141--147, 1990.

\bibitem{GUO111}
L. Guo    and  L. Ljung, ``Exponential stability of general tracking algorithms,'' {\it IEEE Trans. Autom. Control}, vol. 40, no. 8, pp. 1376--1387, 1995.

\bibitem{GUO222}
L. Guo    and  L. Ljung, ``Performance analysis of general tracking algorithms,'' {\it IEEE Trans. Autom. Control},  vol. 40, no. 8,  pp. 1388--1402, 1995.


\bibitem{GUO333}
 L. Guo,  L. Ljung and  P. Priouret, ``Performance analysis of forgetting factor RLS algorithms,'' {\it Int. J. Adapt. Control Signal Process.}, vol. 7, no. 6, pp. 525--537,
 1993.

\bibitem{GUO444}
L. Guo,  L. Ljung   and G. Wang, ``Necessary and sufficient conditions for stability of LMS,'' {\it IEEE Trans. Autom. Control}, vol. 42, no. 6, pp. 761--770, 1997.
\bibitem{hy}
T. Hyt\"{o}nen, J. Van Neerven,  M. Veraar  and L. Weis,  {\it Analysis in Banach spaces}. Berlin: Springer, 2016.
\bibitem{hy2}
T. Hyt\"{o}nen, J. Van Neerven,  M. Veraar  and L. Weis, {\it Analysis in Banach Spaces: Volume II: Probabilistic Methods and Operator Theory}. Berlin: Springer, 2018.
\bibitem{Xieguo}
S. Xie  and L. Guo,  ``Analysis of normalized least mean squares-based consensus adaptive filters under a general information condition,'' {\it SIAM J. Control Optim.}, vol. 56, no. 5, pp.  3404--3431, 2018.

\bibitem{ZLF}
X. Zhang,  T. Li   and  X. Fu,  ``Decentralized online regularized learning over random time-varying graphs, arXiv: 2206.03861, 2024.

%\bibitem{Theodoridis}
%S.  Theodoridis,  {\it Machine Learning: A Bayesian and Optimization Perspective}. Cambridge:  Academic Press, 2015.



%\bibitem{hp}
%E. Hille    and  R. C. Phillips, {\it Functional Analysis and Semi-groups}. Providence: American Mathematical Society, 1996.
%
%\bibitem{Blasco111}
%O. Blasco  and   I. Garcia-Bayona, ``Remarks on measurability of operator-valued functions,'' {\it Mediterr. J. Math.}, no. 13, pp. 5147--5162, 2016.

%\bibitem{Silva}
%textsc\{Silva, J. A.}, \textsc{Faria, E. R.}, \textsc{Barros, R. C.}, \textsc{Hruschka, E. R.}, \textsc{Carvalho, A. C. D.} and \textsc{Gama, J.} (2013). Data stream clustering: A survey. \textit{ACM Computing Surveys (CSUR).} \textbf{46} 1--31.
%\bibitem{Jiang}
%\textsc{Jiang, N.} and \textsc{Gruenwald, L.} (2006). Research issues in data stream association rule mining. \textit{ACM Sigmod Record.} \textbf{35} 14--19.
%\bibitem{chow}
%Y. S.  Chow  and H. Teicher, {\it Probability Theory: Independence, Tnterchangeability, Martingales}. Beilin: Springer Science and Business Media, 2012.
%
%
%
%\bibitem{rb}
% H. Robbins and D. Siegmund, ``A convergence theorem for non-negative almost supermartingales and some applications,'' {\it Optimizing Methods in Statistics}, pp.  233--257, 1971.

%\bibitem{xiASCC}
%X. Zhang  and T. Li, ``Decentralized online learning in RKHS with non-stationary
%data streams: Non-regularized algorithm,'' in {\it Proc. 14th ASCC}, Dalian, China, July 5-8, 2024, pp. 94-99.

%\bibitem{Theodoridis2015}
%\textsc{Theodoridis, S.} (2015). \textit{Machine learning: A Bayesian and optimization perspective.} Elsevier.

%\bibitem{hy2}
%T. Hyt\"{o}nen, J. Van Neerven,  M. Veraar  and L. Weis, {\it Analysis in Banach Spaces: Volume II: Probabilistic Methods and Operator Theory}. Berlin: Springer, 2018.

%\bibitem{hy2}
%T. Hyt\"{o}nen, J. Van Neerven,  M. Veraar  and L. Weis, {\it Analysis in Banach Spaces: Volume II: Probabilistic Methods and Operator Theory}. Berlin: Springer, 2018.
%\bibitem{HU2017}
% J. Hu, M. Zhou, X.  Li and  Z. Xu, ``Online model regression for nonlinear time-varying manufacturing systems,''  {\it Automatica}, vol. 100, no. 78, pp.  163-173, 2017.
%\bibitem{hp}
%E. Hille    and  R. C. Phillips, {\it Functional Analysis and Semi-groups}. Providence: American Mathematical Society, 1996.
%
%\bibitem{Blasco111}
%O. Blasco  and   I. Garcia-Bayona, ``Remarks on measurability of operator-valued functions,'' {\it Mediterr. J. Math.}, no. 13, pp. 5147--5162, 2016.

%\bibitem{Silva}
%textsc\{Silva, J. A.}, \textsc{Faria, E. R.}, \textsc{Barros, R. C.}, \textsc{Hruschka, E. R.}, \textsc{Carvalho, A. C. D.} and \textsc{Gama, J.} (2013). Data stream clustering: A survey. \textit{ACM Computing Surveys (CSUR).} \textbf{46} 1--31.
%\bibitem{Jiang}
%\textsc{Jiang, N.} and \textsc{Gruenwald, L.} (2006). Research issues in data stream association rule mining. \textit{ACM Sigmod Record.} \textbf{35} 14--19.
%\bibitem{chow}
%Y. S.  Chow  and H. Teicher, {\it Probability Theory: Independence, Tnterchangeability, Martingales}. Beilin: Springer Science and Business Media, 2012.
%
%
%
%\bibitem{rb}
% H. Robbins and D. Siegmund, ``A convergence theorem for non-negative almost supermartingales and some applications,'' {\it Optimizing Methods in Statistics}, pp.  233--257, 1971.

%\bibitem{xiASCC}
%X. Zhang  and T. Li, ``Decentralized online learning in RKHS with non-stationary
%data streams: Non-regularized algorithm,'' in {\it Proc. 14th ASCC}, Dalian, China, July 5-8, 2024, pp. 94-99.
%\bibitem{hy}
%T. Hyt\"{o}nen, J. Van Neerven,  M. Veraar  and L. Weis,  {\it Analysis in Banach spaces}. Berlin: Springer, 2016.
%\bibitem{Theodoridis2015}
%\textsc{Theodoridis, S.} (2015). \textit{Machine learning: A Bayesian and optimization perspective.} Elsevier.

%\bibitem{hy2}
%T. Hyt\"{o}nen, J. Van Neerven,  M. Veraar  and L. Weis, {\it Analysis in Banach Spaces: Volume II: Probabilistic Methods and Operator Theory}. Berlin: Springer, 2018.

%\bibitem{hy2}
%T. Hyt\"{o}nen, J. Van Neerven,  M. Veraar  and L. Weis, {\it Analysis in Banach Spaces: Volume II: Probabilistic Methods and Operator Theory}. Berlin: Springer, 2018.
%\bibitem{HU2017}
% J. Hu, M. Zhou, X.  Li and  Z. Xu, ``Online model regression for nonlinear time-varying manufacturing systems,''  {\it Automatica}, vol. 100, no. 78, pp.  163-173, 2017.

\bibitem{hp}
E. Hille    and  R. C. Phillips, {\it Functional Analysis and Semi-groups}. Providence: American Mathematical Society, 1996.
\bibitem{Blasco111}
O. Blasco  and   I. Garcia-Bayona, ``Remarks on measurability of operator-valued functions,'' {\it Mediterr. J. Math.}, no. 13, pp. 5147--5162, 2016.
%\bibitem{hy}
%T. Hyt\"{o}nen, J. Van Neerven,  M. Veraar  and L. Weis,  {\it Analysis in Banach spaces}. Berlin: Springer, 2016.
\bibitem{chow}
Y. S.  Chow  and H. Teicher, {\it Probability Theory: Independence, Tnterchangeability, Martingales}. Berlin: Springer Science and Business Media, 2012.

\bibitem{rb}
 H. Robbins and D. Siegmund, ``A convergence theorem for non-negative almost supermartingales and some applications,'' {\it Optimizing Methods in Statistics}, pp.  233--257, 1971.
\end{thebibliography}
\end{CJK}
\end{document}
