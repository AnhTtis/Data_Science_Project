
\appendix

%=============================================================================================================

\section{Cryostat pre-cooler}
\label{sec:pre-cooler}

The inner vessel of the cryostat requires an initial cooling period prior to the filling of LXe. A liquid-nitrogen (\LN)~based pre-cooling system made of four stainless steel coolers was designed to reduce the initial cool-down time and to speed up the filling process by increasing the cooling power available to liquefy the xenon. The coolers are made of two machined stainless steel layers each, welded at their perimeter. Parallel channels are machined inside the plates to help distribute the \LN~over the whole surface. Shown in figure~\ref{fig:pre-cooler} are four segments connected in series and attached to one another around the highest straight section of the inner vessel, making direct contact with the outside wall of the inner cryostat vessel. The assembly is compressed on the outside of the inner vessel, applying a concentric force to the cooler to ensure optimal thermal contact.

\begin{figure}[h!]
\centering
\includegraphics[width=\linewidth]{figures/pre-cooler-real.pdf}
\caption[Pre-cooler installed at the top of the inner vessel]{Pre-cooler installed at the top of the inner cryostat vessel. \LN~is flushed from the bottom of the outer vessel through a DN40CF liquid feedthrough up to the four-quarter circle plates, connected in series. The boil-off of the \LN~inside the plates provides up to \SI{0.57\pm0.03}{\kilo\watt} of cooling power.}
\label{fig:pre-cooler}
\end{figure}

\mbox{Apiezon\textsuperscript{\textregistered} N Grease}~\cite{apiezon}, a cryogenic thermal paste, was uniformly applied between the pre-cooler plates and the inner vessel of the cryostat to maximise the heat transfer, avoiding the trapping of air pockets which could produce virtual leaks~\cite{edwards_influence_1979}. Two 1/\SI{4}{\inch} braided stainless steel hoses connect the coolers to a DN40CF liquid feedthrough located at the bottom of the outer vessel where they exit the vacuum. \LN~is flushed into the coolers from a self-pressurising storage Dewar. The \LN~is pushed upwards and boils off in the cooling plates, exiting in gas form by the outlet line terminated by a brass phase separator. 

The maximum cooling power of the pre-cooler was calculated from the maximum achievable xenon filling speed for the purity monitor run. At all times during the fill, gaseous xenon is recirculated to and from the cryostat with a gas compressor between \SI{25}{} and \SI{40}{slpm} to ensure sufficient gas convection inside the inner vessel, helping the uniform cooldown of the pressure vessel. The filling speed is measured as 
the difference between the readings of the mass flow meter and the mass flow controller, respectively downstream and upstream of the gaseous xenon inlet~\cite{Baudis:2021ipf}. The measured mass flow difference solely using the full cooling power of the PTR was measured to be $\mathrm{\Delta} \dot{m}_{\mathrm{{fill}_{PTR}}} = $~\SI{17.2(15)}{~slpm}, while the mass flow using both the PTR and the pre-cooler was $\mathrm{\Delta} \dot{m}_{\mathrm{{fill}_{PTR+PC}}} = $~\SI{70.9\pm 1.5}{~slpm}. Therefore, the increase in filling speed from the addition of the pre-cooler is $\mathrm{\Delta} \dot{m}_{\mathrm{{fill}_{PC}}} = $~\SI{53.7(22)}{~slpm}, neglecting any difference in the cooling rate of the inner vessel between the two filling methods as the temperature of the inner vessel is assumed constant.

The cooling power of the pre-cooler can be defined as the sum of the power required to cool the xenon down to the liquefaction point and the power needed to liquefy it:
%
\begin{align}
    \centering
P_{\mathrm{PC}} &= \mathrm{\Delta}\dot{m}_{\mathrm{fill_{PC}}} \cdot \rho_{\mathrm{GXe}} \cdot \left( C_\mathrm{p} \cdot \mathrm{\Delta}T + \mathrm{\Delta}H_{\mathrm{vap}}\right)\\
&=\SI{0.57(3)}{\kilo\watt}~,
\end{align}
%
\noindent where $\rho_{\mathrm{GXe}} =$~\SI{5.4885}{\gram\per\liter} is the density of GXe at \SI{1}{\bar}, $C_\mathrm{p} = $~\SI{0.16067}{\joule\per\gram\per\kelvin} is the heat capacity of GXe, $\mathrm{\Delta}T = $~\SI{119}{K} is the temperature difference between room temperature and LXe temperature and $\mathrm{\Delta}H_{\mathrm{vap}} =$~\SI{95.587}{\joule\per\gram} is its latent heat of vaporisation~\cite{noauthor_gas_2018}.

With the addition of the pre-cooler, we completed the filling of \SI{343}{kg} of xenon in \SI{20.5}{\hour}, discontinuously over a period of three days, for an average filling speed of \SI{16.77}{kg/h}. With the average filling speed during the commissioning run of Xenoscope, reported in Ref.~\cite{Baudis:2021ipf}, estimated at \SI{3.95}{kg/h}, we estimate that the addition of pre-cooler reduces the filling time by a factor 4.25.


\section{Ball of Xenon}
\label{sec:BoX}

The xenon recovery and storage system consists of two separate units: a gas recuperation system exploiting cryogenic pumping comprised of a gas bottle array and a newly-developed gravity-assisted liquid recuperation system. The former consists of an array of up to ten \SI{40}{L} aluminium gas cylinders of which four hang inside interconnected Dewar flasks that can be filled with~\LN. The vacuum created inside the cooled cylinders allows for the recovery of the xenon from the cryostat in gaseous form. The array can store a maximum of $\SI{470}{kg}$ xenon. 

The speed of the recovery process with the gas recuperation system is limited by the evaporation rate of LXe inside the cryostat. This motivated the installation of a second unit, a liquid phase recuperation system, which allows for efficient xenon recovery and shorter downtimes between runs of Xenoscope~\cite{Thieme:2022dze}. Its main component is the spherical pressure vessel Box of Xenon~(BoX) which was designed and manufactured by \textit{KASAG~Swiss~AG}~\cite{kasag} compliant with the European Pressure Equipment Directive PED 2014/68/EU. The stainless steel sphere has a wall thickness of \SI{15}{mm} and an inner radius of \SI{450}{mm}. With a maximum allowable pressure of \SI{90}{bar}, BoX can store up to \SI{450}{kg} of xenon at room temperature. \LN~can be flushed through a copper cooling block located beneath the spherical vessel to cool it down before liquid recuperation is performed. Good thermal contact between the sphere and the copper cooler is ensured by a layer of Apiezon\textsuperscript{\textregistered}~N grease~\cite{apiezon} mixed with $\SIrange{0.5}{1.0}{\micro m}$ silver powder. BoX is thermally insulated with multiple layers of double-sided aluminised bubble wrap. 

BoX is connected between the bottom of the inner vessel and the high-pressure side of the gas system. Unlike the gas cylinder array, BoX can thus be used to drain xenon in liquid form directly from the cryostat, assisted by gravity.

\begin{figure}[h!]
\centering
\includegraphics[width=\linewidth]{figures/BoX-no-logo.pdf}
\caption{Liquid recuperation and storage system Ball of Xenon (BoX). The inner vessel of Xenoscope is isolated from BoX by two high-pressure valves, rated for cryogenic operation. During liquid recovery, LXe flows through a vacuum-insulated transfer line, visible on the right side, to the high-pressure vessel. At its top, BoX is connected to the high-pressure side of the gas system, allowing the boil-off gas to return to the cryostat. The pressure vessel is equipped with an over-pressure safety relief valve, an analogue pressure gauge and a pressure transducer which allows for pressure monitoring via the slow control system. The \LN~cooler system is attached to the underside of the sphere. The vessel is thermally insulated with aluminised bubble wrap.}
\label{fig:Box}
\end{figure}

At room temperature, BoX would have a pressure of \SI{63}{bar} if filled with \SI{450}{kg} of xenon. Past its critical point (above \SI{289.7}{K} and \SI{58.4}{bar}~\cite{NIST}), xenon becomes supercritical and the pressure increases rapidly with temperature. The maximum allowable pressure of \SI{90}{bar} will however not be exceeded below \SI{40}{\celsius}. BoX is instrumented with both an analogue pressure gauge and an electronic pressure transducer. Pressure values from the latter are sent to the slow control system for monitoring. An overpressure-relief valve to air with a set pressure of \SI{89}{bar} relative to atmospheric pressure is installed as an ultimate safety system.

Prior to liquid recuperation, BoX must be pre-cooled to enhance the performance of the process. \LN~is therefore flushed for $\mathcal{O}(24\mathrm{h})$ through the copper cooler. With the pressure in BoX at vacuum level, the recuperation process is started by opening the two valves connecting BoX to the cryostat. A fraction of the LXe coming from the cryostat evaporates after it first contacts the walls of the pressure vessel. This gas is returned to the top of the inner vessel of the cryostat to equalise the pressures of BoX and the cryostat, the gas flow is measured with the mass flow meter. As the process goes on, the evaporation rate stabilises, indicating the halting of recuperation. Xenon remaining in the inner vessel is finally cryo-pumped into the gas cylinder array.

BoX was used successfully to perform gravity-assisted LXe recuperation at the end of three operational runs. With \SI{62.3}{bar} of xenon in BoX at \SI{295.5}{K} after the recuperation of the purity monitor run, we estimate that approximately \SI{294}{kg} of LXe were recuperated into BoX in $\sim$~\SI{11.5}{\hour}, corresponding to an average recovery speed of $\SI{25.4}{\kilogram/\hour}$. This is an increase in recovery speed by factor 8 compared to gas recuperation. The addition of a liquid-height measuring device for future runs will enable the live monitoring of the liquid level during the gravity-assisted recuperation.

%=============================================================================================================
