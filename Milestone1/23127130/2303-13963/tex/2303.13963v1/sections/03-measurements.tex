\section{The purity monitor and measurements}
\label{sec:purity_monitor}

Common impurities in commercially available xenon consist of parts-per-million (ppm) levels of O$_{2}$, N$_{2}$, H$_{2}$O, CO, as well as organic molecules~\cite{Hasterok:2017ehi}. Additionally, detector and subsystem materials introduce impurities by outgassing. The purification of xenon prevents electron losses via their attachment to electronegative impurities and allows to achieve high light and charge yields.

Most purity monitors measure the charge deficit of an initially known population of electrons after their drift through the liquid. By comparing the number of electrons before, $N_0$, and after the drift, $N(t_\mathrm{d})$, an indirect measurement of the impurity concentration in LXe can be achieved. The deficit can be modelled as a decaying exponential:
%
\begin{align}    
    N(t_{\text{d}}) = N_0~\mathrm{e}^{-t_{\text{d}}/\tau}~,
    \label{eq:electron_loss}
\end{align}
where $\tau$ is the electron drift lifetime. It relates to the concentration of electronegative impurities as:
%
\begin{align}
    \tau = \frac{1}{\sum_{i} k_{i} n_{i}}~,
    \label{eq:attachement}
\end{align}
\noindent where $k_i$ is the attachment rate specific to the impurity type in units of volume per mol per time, usually given in $\si{\mathrm{\liter/(\mol}\cdot\mathrm{\second})}$, $n_i$ is the impurity concentration given in \si{\mol/\liter}, and the sum extends over the different electronegative species $i$ in the LXe. The attachment rate coefficient depends on the electric field strength.

A schematic of the working principle of the purity monitor is shown in figure~\ref{fig:purity_monitor_concept}, left. An optical fibre transmits the light from a xenon flash lamp to the centre of a photocathode. The incident photons produce electrons via the photoelectric effect. The electrons are drifted via extraction, drift, and collection electric fields, generated by four biased electrodes. The first drift region (1) is located between the cathode (with the photocathode in the centre) and the cathode grid; the second region (2) extends up to the anode grid; the third region (3) extends from the anode grid to the anode. The charges induce a current signal in the cathode as they drift towards the cathode grid. The screening grids prevent current induction in the cathode and anode when the electrons are drifting along the second region. Once the electrons reach the third drift region, a second signal is generated at the anode, until the electrons are fully collected. Two electronic circuits amplify and convert the induced currents to voltage signals. 

The data is acquired and digitised, triggered by the pulse generator which also starts the discharge in the xenon flash lamp, with a window of \SI{100}{\micro \second} for the anode and cathode waveforms. Once digitised, the voltage signals are integrated to obtain the charges, i.e., the number of extracted and surviving electrons. With the induced charges and the time between the two signals, which corresponds to the drift time for the applied electric field, the electron drift lifetime can be inferred by solving numerically the equation:

\begin{align}
\frac{Q_A}{Q_C} = \frac{t_1}{t_3} \mathrm{e}^{-(t_1+t_2+t_3)/\tau} \frac{(\mathrm{e}^{t_3/\tau} - 1)}{(\mathrm{e}^{-t_1/\tau} - 1)}~.
\label{eq:lifetime}
\end{align}

\noindent Here, $Q_A$ and $Q_C$ are the charges from the integrated signals measured in the anode and cathode, respectively, $t_{1}$ is the rise time of the first signal, $t_{2}$ is the time between the minimum of the signal in the cathode and the rise time of the signal in the anode, with $t_{3}$ the time from $t_{2}$ up to the maximum of the anode signal. Given the motion of the charges, the signal in the cathode has negative polarity, while in the anode the polarity is positive. 
Figure~\ref{fig:signals_inLXe} shows an example of signals acquired in LXe from the cathode and anode at \SI{40}{slpm} along with the three drift times.

The design of the Xenoscope purity monitor is described in detail in Refs.~\cite{Baudis:2021ipf,Biondi:2022T}, and the assembled module is shown in figure~\ref{fig:purity_monitor_concept}, right. It features a field cage built with high conductivity, oxygen-free copper rings, supported by six polyamide-imide pillars. The rings are connected by a resistor chain of $\SI{5}{G\Omega}$ impedance each, and enclose a cylindrical drift region of \SI{15}{\cm}\,\diameter\,$\times$\,\SI{53.1}{\cm}. The cathode and anode grids consist of hexagonally-patterned, etched stainless steel meshes with high optical transparency ($\sim 93\%$), while the cathode and anode are solid stainless steel disks.


\begin{figure}[h!]
\centering
\includegraphics[width=0.24\textwidth]{figures/pm_schematic.jpg}
\includegraphics[width=0.15\textwidth]{figures/PM-high-res.jpg}
\caption{(Left): Schematic of the purity monitor. A pulse generator triggers a flash from the xenon lamp and the light is transmitted through an optical fibre to the photocathode, where photoelectrons are produced. The electrons are extracted, transported and collected by three electric fields, defined by the cathode ($\mathrm{G_C}$) and cathode grid ($\mathrm{G_1}$), the anode grid ($\mathrm{G_2}$) and the anode ($\mathrm{G_A}$). In the longest region (2), the field shaping rings~(FSR) maintain the uniformity of the drift field $\vec{E_d}$ in the vertical direction. (Right): Assembled purity monitor in Xenoscope.}
\label{fig:purity_monitor_concept}
\end{figure}

\begin{figure}[h!]
\centering
\includegraphics[width=\columnwidth]{figures/signals.pdf}
\caption[Signal acquired at \SI{40}{slpm} xenon recirculation speed.]{Signals acquired at \SI{40}{slpm} xenon recirculation The rise time of the cathode signal (blue) is taken as $t_{1}$, the time interval between the minimum of the cathode signal and the start of the anode signal (red) is taken as $t_{2}$, with half of the charge cloud completely collected at $t_{3}$. These values are later used to calculate the electron drift lifetime.}
\label{fig:signals_inLXe}
\end{figure}
%=============================================================================
\subsection{Optical components and photocathode}

The utilised lamp is a \SI{60}{\watt} xenon flash lamp with a \mbox{built-in} reflective mirror (model number L7685) from \textit{Hamamatsu}~\cite{hamamatsu}. The window is a single sapphire crystal allowing short wavelength light ($\sim$~\SI{190}{\nano\meter}) to reach the output of the lamp, with a spectral emission from \SI{190}{\nano\meter} to \SI{2000}{\nano\meter}. The lamp generates a discharge which excites the gas producing scintillation, with reflective mirrors directing photons from all directions towards the output. The xenon lamp can be triggered internally, or externally via a pulse generator. The intensity of the light emission is adjusted by setting the voltage for the discharge in the lamp between \SI{600}{\volt} and \SI{1}{\kilo\volt}. The selection of the latter maximises the number of produced electrons.

 Measures were adopted to mitigate the electronic noise produced in the signal waveforms by the external trigger: the xenon flash lamp was rehoused in a stray electromagnetic interference box, and galvanic insulation and ferrite filters were added to the trigger line. The box was customised by adding a potentiometer to manually change the voltage of the discharge. A trigger circuit controlled by the pulse generator was added as well. The lamp is placed outside the cryostat, with an optical fibre carrying the light from its output to the surface of the photocathode. A UV grade sapphire lens produced by \textit{Hamamatsu} is placed at the output of the lamp to collimate the light to the optical fibre. The selected fibre is $\SI{600}{\micro\meter}$ in diameter, ultra-high vacuum rated with a polyimide buffer from \textit{LewVac}~\cite{LewVac}. It is resistant to UV solarisation, i.e.~the degradation in the fibre material due to the exposure to light of wavelength lower than \SI{300}{\nano\meter}. The fibre feedthrough, produced by \textit{Thorlabs}~\cite{feedthrough}, consists of a $\SI{600}{\micro m}$ multimode fibre in an SMA connector welded on a CF40 flange. The feedthrough requires the fibres to be terminated, hence these were prepared and polished in-house with a set of 8~lapping sheets, made of aluminium oxide, silicon carbide, and calcined alumina, from grits of $\SI{30}{\micro\meter}$ to $\SI{0.02}{\micro\meter}$. 
 
One of the critical parts of a purity monitor is the photocathode, for it directly impacts the size of the initial signal. It consists of a thin layer of a low work function metal, deposited on a quartz substrate that has low absorption of UV photons~\cite{Valentini2002}. The photocathode was produced in-house using a turbomolecular pumped coater Q150T Plus from \textit{Quorum}~\cite{coater}. The desired thickness of the layer was monitored with a quartz crystal microbalance. Different materials for the thin layer were tested, including gold and silver, and the coater was used to produce photocathodes of \SI{50}{\nano\meter} thickness on a \SI{2}{\mm} thick quartz substrate, with a diameter of \SI{30.00(5)}{\mm}. The deposition of a \SI{5}{\nano\meter} thick layer of titanium was required in the case of gold for adhesion to the substrate. The choice of thickness was based on the effective probe depth of gold layers, and previous works~\cite{Manenti2020}. Additional technical details can be found in Ref.~\cite{Biondi:2022T}.

The photocathodes were tested in a vacuum setup, where the xenon lamp was flashed onto the photocathode material and the induced current was measured. Both gold and silver showed high yields, with gold reaching a stable state in fewer hours of exposure to the UV signal. Silver and gold photocathodes showed an increasing quantum efficiency with time when exposed to light, and this increased yield did not revert back in subsequent tests. The increase in quantum efficiency of the photocathode with UV-light exposure was also observed in Ref.~\cite{Manenti2020}. The gold photocathode was selected due to its stability over time and higher quantum efficiency than the silver photocathode, requiring smaller electric fields to produce a higher charge signal. 

%===============================================================================
\subsection{Current readout and signal processing}

The readout electronics amplify the induced currents from the cathode and anode and are placed inside the cryostat to avoid signal losses along the \SI{9}{\meter} signal cables. The circuits were designed together with the Electronics Workshop at the University of Zurich. The circuit consists of an AC-coupling component, a transimpedance amplifier, and a final voltage amplifier with a $\SI{50}{\ohm}$ impedance termination to match the one from the data acquisition. The transimpedance and voltage amplifiers are implemented with two low-cost operational amplifiers, model AD8066 from \textit{Analog Devices}~\cite{analogdev}. An AC-coupling filter in the circuit board removes high-frequency noise, which enhances the signal quality, and the AC-coupling removes the DC component of the HV applied to the electrodes. The usage of a transimpedance amplifier, in contrast to a charge amplifier, allows for more precise timing and signal spread analyses due to its small resistive-capacitive constant (RC) and short rise time of \SI{0.14}{\micro\second}. However, due to its fast response, a low-pass filter for frequencies below \SI{800}{\kilo\hertz} is applied to the signals to decrease the electronic noise induced by, e.g., the pulse generator that triggers the lamp, two temperature sensors, and the uninterruptible power supply. The preamplifier operates in current mode, as the capacitance discharges rapidly, resulting in an output voltage proportional to the instantaneous current. The frequency response of the readout electronics was benchmarked, with a negligible effect on the signal shape due to the $\SI{100}{MHz}$ bandwidth. 

The performance of the readout electronics was tested in a climate chamber in steps of \SI{10}{\kelvin} from room temperature down to \SI{190}{\kelvin}. The calibration showed a charge amplification of $\SI{0.18}{\femto\coulomb/(\milli\volt\cdot\micro\second)}$, with good thermal stability. Additionally, the RC decay constant of the circuit, which could be a source of systematic error for time measurements, was estimated at $\sim$~\SI{150}{\nano\second} by feeding a \SI{2}{\micro\second} wide square pulse to the circuit.

An oscilloscope, \textit{Teledyne LeCroy} model Waverunner 6104A~\cite{teledyne}, and an analog-to-digital converter from \textit{CAEN}, model v1724~\cite{CAEN}, acquired the waveforms produced by the cathode and anode readout. Each acquisition consisted of the average of \SI{1000}{} waveforms acquired over \SI{16.7}{minutes} to minimise the baseline noise. The signals were then processed by fitting the expected signal shape with a Gaussian distribution. In some waveforms, a noise introduced by external electronic devices could be discerned as part of the background noise, and the fit included a sine function to account for this effect, with an inferred model uncertainty of $5\%$ for the ones where the sine fit to the noise did not converge. The current-equivalent voltage signals in the cathode and anode were integrated to obtain a charge-proportional value. The residuals of the fits were used as weights for the charge values obtained in the averaged data shown in the next section. The uncertainties in charges and times obtained in the fits were propagated to obtain the uncertainty of the electron drift lifetime value. An example of the raw anode signal at \SI{53}{\volt/\cm} drift field in region 2 is shown in figure~\ref{fig:FWHM_signals}, together with the post-processing signal with a low-pass filter. The calculated baseline and Gaussian fit of the signals are also shown.

\begin{figure}[h!]
\centering
\includegraphics[width=0.49\textwidth]{figures/signal_example_fit.pdf}
\caption[Anode signals at \SI{53}{\volt/\cm}]{Anode signal at \SI{53}{\volt/\cm} prior to (blue) and after (orange) the low-pass filter. The calculated baseline (red) and a Gaussian fit of the signal (green) are also shown. The signal is an average over \SI{1000}{} recorded waveforms.}
\label{fig:FWHM_signals}
\end{figure}

\subsection{Measurements}

Once installed in the cryostat, the purity monitor was first operated in vacuum ($\sim \SI{1e-5}{\milli\bar}$). Data was acquired to investigate the signal shape and response in this configuration with negligible charge losses due to residual gas. The measurement additionally provided the delay time of the electronics chain, from the pulse generator for the xenon lamp to the signal amplification and readout of $\SI{18}{\micro\second}$.

After the calibration of the purity monitor in vacuum, gaseous xenon was flushed inside the detector and purified through recirculation in the gas system. The LXe run started with the filling of \SI{343}{\kilogram} of xenon. As the xenon recirculates through the getter, electronegative impurities are removed, and the electron drift lifetime is expected to increase in two steps: an initial exponentially increasing phase where the bulk impurities are rapidly removed, and a second phase where the change is dominated by the materials outgassing, and where the electron drift lifetime slowly increases over time. At different recirculation speeds, the electron drift lifetime reaches increasingly higher values in the second phase.

The recirculation speed was set with flows of \SI{30}{standard\ litres \ per\ minute\ (slpm)}, \SI{35}{slpm} and \SI{40}{slpm}, with the xenon lamp illuminating the photocathode with a frequency of \SI{1}{\hertz}. In the cryostat, the temperature and pressure were maintained around \SI{177.6(1)}{\kelvin} and  \SI{2.05(1)}{\bar}, respectively. Following the commissioning run, the displacement of the GXe compressor was reduced to increase its lifespan. This constrained the maximum purification speed to \SI{40}{slpm}, compared to the \SI{80}{slpm} reported in Ref.~\cite{Baudis:2021ipf}. The initial impurity level in the xenon gas impacts the number of days before a signal can be seen in the purity monitor: the  first waveforms in the cathode and anode were observed after $\sim$26.5 days.

During data taking, the cathode and cathode grid were biased at \SI{-2710}{\volt} and \SI{-2650}{\volt}, respectively. The anode grid was kept at ground while the anode was biased at \SI{500}{\volt}. The values were selected based on COMSOL~\cite{multiphysicscomsol} simulations which yielded nearly $100\%$ extraction efficiency of the electrons produced in the centre of the photocathode. Table~\ref{table:drift_times} shows the summary of the distances, times electric fields for the  extraction (1), drift (2) and collection (3) regions.
\begin{table}[h!]
\centering
\caption[Distance and drift fields]{Electric fields, distances and times $t_{i}$ measured for the three regions in the PM, with voltages \SI{-2710}{\volt}, \SI{-2650}{\volt}, \SI{0}{\volt}, and \SI{500}{\volt} for the cathode, cathode grid, anode grid and anode, respectively, for a purification speed of \SI{40}{slpm}.}
\begin{tabular}{clll} 
\hline 
Drift region $i$ & Distance [mm] & Field [V/cm] & $t_{i}$ $[\SI{}{\micro \second}]$\\
\hline
 1 &  $18 \pm 1$  & $33 \pm 1 $ & $12.8 \pm 0.8$\\
 2 &  $503 \pm 5$ &  $53 \pm 1 $ & $433.5\pm0.7$ \\
 3 &  $10 \pm 1$  & $500 \pm 5$ & $7.6 \pm 0.7$
\label{table:drift_times}
\end{tabular}
\end{table}

Figure~\ref{fig:charge_anode_cathode} shows the anode and cathode signals with their integral, where the integrated signals show a step-like feature after the charges move entirely to the next drift region, or are collected in the anode. The integration corresponds to the total area of the Gaussian fit. The charge measured in the cathode corresponds to $N_{\text{e}^{-}} \cong 10^{6}$ electrons extracted from the photocathode at each pulse.

\begin{figure}[b]
\centering
\includegraphics[width=\columnwidth]{figures/charge_cathode.pdf}
\centering
\includegraphics[width=\columnwidth]{figures/charge_anode.pdf}
\caption[Signal readout at the cathode with the integrated charge signal]{Signal readout (blue) at the cathode (top) and anode (bottom)  with their respective Gaussian fits (orange) and integrated charge signals (red).}
\label{fig:charge_anode_cathode}
\end{figure}


