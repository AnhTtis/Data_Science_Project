\section{The Xenoscope facility}
\label{sec:experimental_setup}

Xenoscope can house up to \SI{400}{kg} of LXe in a double-walled stainless steel cryostat. The facility, its subsystems, and the outcome from the first commissioning run are described in Ref.~\cite{Baudis:2021ipf}. Xenoscope was first equipped with a purity monitor (section \ref{sec:purity_monitor}) fully submerged in LXe, while the cryostat aspect ratio was chosen to allow for the operation, in the next phase of the project, of a \SI{2.6}{m} two-phase TPC, with the primary goal of demonstrating the drift of electrons in LXe over this distance for the first time. A computer-aided design (CAD) rendering of the cryostat with the purity monitor is shown in figure~\ref{fig:phases}.

The facility includes a gas purification system with a series of filters and a commercial zirconium alloy getter. The LXe is extracted at the top of the liquid column, where the impurity concentration is higher. It is evaporated in the heat exchanger system and circulated through the purification system at a fixed flow. The purified xenon is recondensed in the heat exchanger and reintroduced in the cooling tower, which comprises a pulse tube refrigerator (PTR) connected to a cold head mounted atop the cooling chamber. The xenon is then directed to the bottom of the cryostat. A slow control system built from open-source software oversees and sends alarms on relevant parameters.

Two system upgrades were performed prior to the installation of the purity monitor. First, a pre-cooler was manufactured and installed at the top of the inner cryostat vessel to provide additional peak cooling power, and thus to reduce the system cooldown and xenon liquefaction time during filling by a factor of 4.25. The design of the pre-cooler and details of its commissioning are presented in~\ref{sec:pre-cooler}. Furthermore, a gravity-assisted recuperation and storage system for LXe, Ball of Xenon (BoX), was deployed to allow for the storage of up to \SI{450}{\kilogram} of xenon at room temperature, as well as for recuperation in liquid phase. The latter enhances the speed of the recuperation process by a factor $\sim$ 8 compared to gaseous recuperation to a bottle array via cryogenic pumping. More details of its design and performance are presented in~\ref{sec:BoX}.

\begin{figure}[ht!]
\centering
\includegraphics[width=\columnwidth]{figures/render-PM3.pdf}
\caption[The purity monitor in the Xenoscope cryostat]{The purity monitor in the Xenoscope cryostat. Legend: (1)~top flange; (2)~outer vessel; (3)~inner vessel; (4)~pre-cooler; (5)~purity monitor; (6)~BoX recuperation line; (7)~anode; (8)~anode grid; (9)~field-shaping rings and resistor chain; (10)~support pillars; (11)~cathode grid; (12)~cathode disk; (13)~photocathode and optical fibre.}
\label{fig:phases}
\end{figure}

