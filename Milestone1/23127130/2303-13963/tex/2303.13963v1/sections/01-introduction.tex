\section{Introduction}
\label{sec:intoduction}

The next-generation liquid xenon (LXe) detector DARWIN~\cite{Baudis:2012bc,Aalbers:2016jon} will be sensitive to a multitude of astroparticle physics signals. With 40 tonnes of instrumented xenon in a \mbox{dual-phase} time projection chamber (TPC), it will probe the parameter space for weakly interacting massive particle dark matter down to the neutrino fog~\cite{Schumann:2015cpa, Aalbers:2016jon, OHare:2021utq}. With the large target mass and projected low background level, competitive searches for neutrinoless double-beta decay of $^{136}\text{Xe}$~\cite{Baudis:2014naa, Agostini:2020adk}, solar axions, bosonic dark matter and dark photons are viable. Solar $^8\text{B}$ and supernova neutrinos can be detected via coherent neutrino-nucleus scattering~\cite{Lang:2016zhv, Raj:2019sci}, while solar $pp$-neutrinos can be observed with high statistics at low energies via their scattering off atomic electrons~\cite{Aalbers:2020gsn}. 

At its core, DARWIN will operate a cylindrical \mbox{dual-phase} TPC, almost entirely filled with LXe, with a thin gaseous xenon (GXe) layer at the top. Energy depositions generate scintillation light and ionization electrons whose signals, denoted as S1 and S2, respectively, are measured with light sensors covering the top and bottom faces of the cylindrical detector. The dual signal readout enables energy and position reconstruction, as well as background discrimination. A $\mathcal{O}(100)$~V/cm electric drift field is applied throughout the LXe volume between a cathode and a gate electrode, suppressing electron recombination while moving them towards the gate below the liquid-gas interface. A stronger $\mathcal{O}(10)$~kV/cm extraction field between the gate and the anode accelerates the electrons into the GXe, where they produce an electroluminescence signal proportional to the number of electrons reaching the liquid-gas interface.

 Electron losses mostly occur due to attachment to impurities during the drift. A depth-dependent widening of the S2 arises from electron cloud diffusion during the drift. Longitudinal diffusion (parallel to the electron cloud propagation) affects the ability to distinguish multiple S2s occurring at different depths ($z$) within a single event. Additionally, lateral diffusion can impact position reconstruction in the $x$-$y$ plane. Since such signatures are expected in background events with multiple interactions of neutrons or $\gamma$-rays, diffusion impacts the background rejection capability of a detector.

With its \SI{2.6}{\m} in drift length and \SI{2.6}{\m} diameter, the cylindrical DARWIN TPC will be significantly larger than currently operative LZ~\cite{LZ:2021xov}, PandaX-4T~\cite{Zhang:2018xdp} and XENONnT~\cite{Aprile:2020vtw}, with dimensions up to \SI{1.5}{m} drift and diameter. To study several challenges related to the increased dimensions, such as high-voltage delivery, electron loss, diffusion of the electron cloud and light attenuation over large distances, the Xenoscope facility was designed and built as a vertical scale technical demonstrator~\cite{Baudis:2021ipf}. Section~\ref{sec:experimental_setup} presents the Xenoscope facility during its first science phase. The facility was equipped with a \SI{53}{cm} tall purity monitor, described in section~\ref{sec:purity_monitor}, which was targeted at measuring electron drift and quantifying the xenon purification capabilities along  with the electron transport measurements. Section~\ref{sec:results_discussion} discusses the results from the electron drift lifetime measurement for different purification regimes and presents the fit of an electron drift lifetime model to the data. It also reports on measurements of the electron drift velocity and of the longitudinal diffusion coefficient of electron clouds at different drift fields. Conclusions and an outlook are given in section~\ref{sec:conclusions}. 

