\section{Results and discussion}
\label{sec:results_discussion}

\begin{figure*}[t!]
\centering
\includegraphics[width=2\columnwidth]{figures/Elife_model.pdf}
\caption[Purification flow-dependent electron drift lifetime measured in the Xenoscope.]{Purification flow-dependent electron drift lifetime measured in Xenoscope. The data was averaged in \SI{6}{\hour} time bins. The dashed lines indicate a change in flow, while the dash-dotted lines indicate short-term irregularities in the pressure and flow conditions (see text). The red line shows the best-fit model from equations \ref{eq:e-lifetime-gas} and \ref{eq:e-lifetime-liquid}, while the black points show the residuals.}
\label{fig:electron_lifetime_datasets}
\end{figure*}

 The electron drift lifetime measurement campaign with the purity monitor lasted a total of \SI{88}{days}. The purification was performed at \SI{30}{slpm} for \SI{46.6}{days}, at \SI{35}{slpm} for \SI{20.0}{days}, and at \SI{40}{slpm} for \SI{21.2}{days}. After the electron drift lifetime measurements in LXe, signals for drift fields from \SI{25}{V/cm} to \SI{75}{V/cm} were acquired to study field-dependent electron transport properties, such as the drift velocity and longitudinal electron cloud diffusion.

\subsection{Electron drift lifetime}

Figure~\ref{fig:electron_lifetime_datasets} shows the electron drift lifetime calculated with the charge signals acquired at the cathode and anode over the entire acquisition period. When the recirculation speed changes, the electron drift lifetime drops, most likely due to a change in the height of the liquid level, resulting in the release of trapped impurities in the high-surface tension region at the LXe \enquote{collar} (LXe/GXe/inner vessel interface). A drop in electron drift lifetime was also observed at \SI{59.8}{\day}, as expected, when the GXe compressor was stopped for a period of approximately \SI{15}{minutes} due to a communication error with the slow control software. Again, the change in liquid level most likely resulted in the sudden release of impurities from the collar. Shortly following these events, the electron drift lifetime increased exponentially to return to the outgassing-limited values.

A review of the slow control data allowed for the identification of three irregularities in the pressure and flow conditions, at \SI{38.9}{\day}, \SI{52.6}{\day}, and \SI{80.7}{\day}. The first was a brief moment of excess flow downstream of the GXe compressor with a slight pressure increase of approximately $\SI{10}{\milli bar}$, suggesting the release of trapped gas in the xenon handling system from vibration, or the unlikely development of a micro-leak. The second irregularity was a quick fluctuation in the purification flow, both upstream and downstream of the GXe compressor, resulting in a momentary increase in pressure both downstream of the compressor and in the inner vessel, also leading to a change in the liquid level. The last irregularity was again a marginal increase in the flow downstream of the compressor. In this case, however, the event lasted approximately \SI{3}{\hour}.
After \SI{88.0}{\day} of almost continuous purification, the electron drift lifetime reached a value of $\SI{664(23)}{\micro\second}$. This is consistent with the value reached in other LXe experiments, such as XENON1T~\cite{XENON:2021nad} and LUX~\cite{LUX:2020vbj}, with \SI{660}{\micro\second} and \SI{750}{\micro\second}, respectively. 

While the purity level demonstrated in Xenoscope could be sufficient to drift electrons over \SI{2.6}{\meter} in LXe, it can be further improved by increasing the purification speed. To investigate this, a simple model of the electron drift lifetime, assuming O$_2$-like impurities, was adapted from Refs.~\cite{greeneXENON1TSpinIndependentWIMP2018,Plante:2022khm} and fitted to the electron drift lifetime data shown in figure~\ref{fig:electron_lifetime_datasets}. Two coupled differential equations describe changes in impurity levels $M \dv{I_{}}{t}$ after a time $\dd t$, where $M$ is the mass of xenon, and $I$ is the impurity concentration. The gas and liquid phases (denoted by the subscripts $g$ and $l$, respectively) are evaluated separately to determine the impurity concentration at time $t+\dd t$:
%
\begin{multline}
M_{\mathrm{g}} \dv{I_{\mathrm{g}}}{t}^{(j)} = -F_{\mathrm{g}} \rho I_{\mathrm{g}} + \left(\frac{\Lambda_{\mathrm{g},0}}{1+\frac{t-\Delta t^{(j)}_g}{T_{1/2,g}}} + C_\mathrm{g}\right)\\ + \frac{\epsilon_1 P_{\mathrm{C}} I_{\mathrm{l}}}{h} - \frac{\epsilon_2 P I_{\mathrm{g}}}{h} + M_{\mathrm{g}} \Delta I_{\mathrm{g}}^{(j)} \int \delta\left(t-t^{(j)}\right) \dd t
\label{eq:e-lifetime-gas}
\end{multline}
%
\begin{multline}
M_{\mathrm{l}} \dv{I_{\mathrm{l}}}{t}^{(j)} = -F_{\mathrm{l}} \rho I_{\mathrm{l}} + \left(\frac{\Lambda_{\mathrm{l},0}}{1+\frac{t-\Delta t^{(j)}_l}{T_{1/2,l}}} + C_\mathrm{l}\right)\\- \frac{\epsilon_1 P_{\mathrm{C}} I_{\mathrm{l}}}{h} + \frac{\epsilon_2 P I_{\mathrm{g}}}{h} + M_{\mathrm{l}} \Delta I_{\mathrm{l}}^{(j)} \int \delta\left(t-t^{(j)}\right) \dd t~.
\label{eq:e-lifetime-liquid}
\end{multline}
%
\noindent The index $j = \{0, 1,...,7\}$ indicates the regions between discontinuities, marked by the dashed lines in figure~\ref{fig:electron_lifetime_datasets}. Each equation consists of five terms. The first accounts for the purification rate, with $F$ being the purification flow, $\rho$ the density of xenon at \SI{1}{bar}, \SI{0}{\celsius}, and $I$ the concentration in impurities. The second term accounts for the time-dependent average outgassing rate from detector materials, where $\Lambda_0$ is the outgassing at time $t=0$, $T_{1/2}$ is the decay-time of the outgassing rate and $C$ is a constant outgassing term which brings the system to an equilibrium point when $t \gg T_{1/2}$. After the sudden increase of impurity concentration in the xenon during flow changes, described by the delta function in the fifth term, the outgassing rate modelled by the second terms of the differential equations is expected to revert back to a high value, as impurities can be adsorbed in outgassed materials~\cite{Plante:2022khm}. This is accounted for with the time-offset parameters $\Delta t^{(j)} = \sum_1^j \Delta t ^{(i)}$, which are cumulative since the dataset is fitted as a whole, and $\Delta t^{(0)} = 0$. Finally, the third and fourth terms describe the exchange of impurities between the gas and the liquid phases, and thus the signs are inverted between the two equations. The $\epsilon$ parameters are efficiencies of the exchange process, $P_{\mathrm{C}}$ is the cooling power of the system in the absence of purification, proportional to the evaporation rate of the xenon, $P$ is the cooling power deployed by the cryogenics, proportional to the condensation rate, and $h$ is the latent heat of xenon.

The electron drift lifetime is then calculated by solving the system of differential equations for equation~\ref{eq:attachement} and minimising with the iMinuit (Python) package~\cite{James:1975dr} simultaneously in all $j$-regions, using a least-square method. The best-fit model is displayed as a solid red curve in figure~\ref{fig:electron_lifetime_datasets}. Assuming the same initial conditions obtained from the fit of the model to the electron drift lifetime data, we obtain the purification flow-dependent electron drift lifetime predictions shown in figure~\ref{fig:prediction}.
%
\begin{figure}[h!]
\centering
\includegraphics[width=0.49\textwidth]{figures/prediction.pdf}
\caption{Purification flow-dependent electron drift lifetime prediction. Assuming the same operating conditions as in the measurement campaign, an increased flow (solid lines) could significantly reduce the purification time needed before the electron drift lifetime begins its exponential rise. Furthermore, the addition of a \SI{2}{slpm} gas phase extraction to the purification loop can also improve the electron drift lifetime (dashed lines).}
\label{fig:prediction}
\end{figure}
%

As expected, an increased purification speed would yield longer electron drift lifetimes, attained in a shorter purification time. The addition of a purification flow of the gas phase ($F_\mathrm{g}= \SI{2}{slpm}$) suggests an expected increase in electron drift lifetime of up to \SI{15}{\%}. Therefore, prior to the start of the next phase of Xenoscope, a parallel gas extraction line inspired by the gas purification system reported in Ref.~\cite{Plante:2022khm} was added to the gas handling system with a second flow controller, allowing for the purification of both the liquid and gas phases in the same purification loop. 
 
\subsection{Electron transport}

The purity monitor allows for a dedicated measurement of the arrival time of the electron cloud at the anode. The drift velocity $v_{\mathrm{d}}$ of the cloud given a drift field $E_{\mathrm{d}}$ is:
\begin{equation}
\label{velocity}
v_{\text{d}} =d_{2}/t_{2}.
\end{equation}
Considering that the accuracy of the time measurement between the extrema of the cathode and anode signals is higher, and region 3 has a fast detection, it is convenient to calculate instead:
\begin{equation}
\label{velocity}
v_{\text{d}} = (d_{2} + d_{3}) / (t_{2} + t_{3}).
\end{equation}
This approach introduces an additional term, $(d_{2}t_{3} - d_{3}t_{2})/(t_{2} (t_{2}+t_{3}))$, which induces a negligible $0.1\%$ bias in the final values given the ratios between $t_{3}/t_{2}$ and $d_{3}/d_{2}$. The drift velocity can also be expressed in terms of the drift field $E_{\mathrm{d}}$:
\begin{equation}
v_{\mathrm{d}} = \mu E_{\mathrm{d}}\:,
\end{equation}
where $\mu$ is the electron mobility. The mobility is related to the average time, for a given temperature, density, and energy of the electrons, between elastic collisions with the xenon and electronegative impurities, and potential inelastic collisions with impurities. Thus, the acquisition of waveforms used to derive the drift velocity was performed at a constant electron drift lifetime to avoid systematic uncertainties. Benchmark regimes can be used to visualise these dependencies for electron mobility, such as \enquote{cold electrons} and \enquote{hot electrons}~\cite{Schmidt:1984zz}. In the cold electrons regime, the energies of the electrons are mostly due to the thermal bath in the xenon fluid (around \SI{0.015}{eV} at \SI{177}{\kelvin}~\cite{Boyle:2016wpy}), and they rapidly acquire energy with increasing electric fields, resulting in a linear gain in velocity. For cold electrons, the mobility can be expressed as:
%
\begin{equation}
    \mu=\frac{2}{3}\left(\frac{2}{\pi\, m_e \,k_B \,T}\right)^{\frac{1}{2}} e \,\frac{\lambda}{v}\:,
    \label{eq:cold_electrons}
\end{equation}
where $k_{B}$ is the Boltzmann constant, $T$ is the temperature, $e$ is the charge of the electrons, m$_e$ is their mass, $v$ is the magnitude of the velocity in all directions, and $\lambda$ is the mean free path of the electrons in their collisions with atoms, inversely proportional to the number density and cross section. In contrast to cold electrons, hot electrons have gained most of their energy through acceleration by the drift field, and experience increased energy losses on their drift path due to collisions with xenon atoms, which results in a slower rate of change in their velocity. For this case, the electron mobility can be expressed as:
\begin{equation}
\mu = \frac{4\, e \, \lambda}{3 v  \pi^{1 / 2} m_e^*}\:,
\label{eq:hot_electrons}
\end{equation}
where $m_e^*$ is the effective mass of the electrons in the medium. While these equations are not used in this work to infer properties such as electron mobility, or electron cloud diffusion, they are useful to scrutinise the results of our measured drift velocity and discuss potential systematic effects, discussed later in this section. 

The data for this measurement was acquired at day 89, after the electron drift lifetime entered a region of slow change, and in a time interval of half an hour, to minimise systematic effects. Drift fields of \SI{25}{\volt/\cm} to \SI{75}{\volt/\cm} were scanned in steps of \SI{5}{\volt/\cm}, where the former was the threshold for a discernible signal above noise level in the cathode. The extraction field was changed to maintain the ratio between the extraction and drift field used in the electron drift lifetime data, while the collection field was fixed at \SI{500}{\volt/\centi\meter}. Figure~\ref{fig:drift_time_field} shows the drift velocity at different fields, calculated with equation~\ref{velocity}, with good agreement with previous measurements in LXe~\cite{gushchin1982electron, Albert:2016bhh,Baudis:2017xov,Thieme:2022dze,Jorg:2021hzu,Baur:2022sel}. The prediction from NEST (Noble Element Simulation Technique)~\cite{szydagis_m_2018_1314669}, based on data-driven empirical models, is also shown. The curves for cold and hot electrons are derived by using equations~\ref{eq:cold_electrons} and~\ref{eq:hot_electrons}, respectively, to fit the data from Ref.~\cite{gushchin1982electron}, for it covers both regimes with a high density of points.

\begin{figure}[h!]
\centering
\includegraphics[width=0.49\textwidth]{figures/drift_velocity_withXenoscope.pdf}
\caption{Measured drift velocity with electric field values from \SI{25}{} to \SI{75}{\volt/\cm}, in steps of \SI{5}{\volt/\cm}. The results are compared to literature values from Gushchin (\SI{165}{K})~\cite{gushchin1982electron}, EXO-200 (\SI{167}{\kelvin})~\cite{Albert:2016bhh}, Xurich II (2018, \SI{184}{\kelvin})~\cite{Baudis:2017xov}, Xurich II (2021, \SI{177}{\kelvin})~\cite{Thieme:2022dze}, HeXe (\SI{174}{K})~\cite{Jorg:2021hzu} and XeBRA (\SI{173}{\kelvin})~\cite{Baur:2022sel}. The prediction from NEST v2.3.8~\cite{szydagis_m_2018_1314669} is shown as a solid blue curve.}
\label{fig:drift_time_field}
\end{figure}

The spread of the electron cloud in time was studied by analysing the anode signal at the previously mentioned drift fields. With this purity monitor, only the longitudinal diffusion can be observed, for there is no information on the $x-y$ charge distribution. The standard deviation ($\sigma$) of the Gaussian fit of the signal in the anode is used to calculate the longitudinal diffusion coefficient. For the case of the initial and final distributions of the electron cloud following a Gaussian distribution, the width of the anode signal is related to the longitudinal diffusion~\cite{Albert:2016bhh,Li:2015rqa,Rolandi2008} as:
%
\begin{equation}
D_{L}=\frac{d^{2} \sigma_{L}^{2}}{2 t^{3}}\:,
\label{eq:diffusion1}
\end{equation}
\noindent
where $D_L$ is the diffusion coefficient of the electron population due to the random walk of the electrons in the longitudinal direction, and:
%
\begin{equation}
\label{eq:diffusion}
\sigma_{L}^{2}=\sigma^{2}-\sigma_{0}^{2}\:,
\end{equation}
\noindent
 is the width at one $\sigma$ considering diffusion effects only, which is the value to extract. The widths $\sigma_{0}$ and $\sigma$ belong to the initial signal at the cathode and the final signal measured in the anode, respectively, and $d$ is the drift length ($d_{2}+d_{3}$) at drift time $t$ ($t_{2}+t_{3}$). By rewriting equation~\ref{eq:diffusion}, we obtain the width of the anode signal:
%
\begin{equation}
    \sigma^2 = \frac{D_L 2 t^3}{d^2} + \sigma_0^2\:,
\label{eq:diffusion2}
\end{equation}
The diffusion of the electron cloud is related to the previously introduced electron mobility. At low drift fields, it follows the Einstein-Smoluchowski relation~\cite{einstein,smoluchowski,Schmidt:1984zz}:
\begin{equation}
D_L=\frac{k_{B} T}{e} \mu = \frac{\epsilon_{T}}{e} \mu\:,
\label{eq:diffusion_einstein}
\end{equation}
with thermal energy $\epsilon_{T}$.
The diffusion is thus affected by the cross section of elastic and inelastic interactions with the medium species (xenon and impurities), analogously to the drift velocity. 

 Since the electrons have energies above the thermal bath and are not in equilibrium, a characteristic energy, $\epsilon_k$, is defined as:
\begin{equation}
\epsilon_k = \frac{e D_L}{\mu}\:,
\label{eq:characteristic_energy}
\end{equation}
representing the energy associated with the longitudinal diffusion, where now $D_L$ has a contribution beyond the thermal energy of electrons:
\begin{equation}
D_L= \frac{(\epsilon_T + \epsilon )}{e}\mu\:,
\label{eq:characteristic_energy_longitudinal_diffusion}
\end{equation}
with $\epsilon =  \epsilon_k - \epsilon_{T}$.

In previous studies, it was reported that impurities diffused in the medium, from water vapour to organic molecules, can provide a more effective energy loss mechanism for electrons, with the consequence of higher mobility and decreased diffusion~\cite{Yoshino:1976zz,Pack:1961jag} (see equations~\ref{eq:cold_electrons}  and~\ref{eq:hot_electrons}). This effect could explain the mechanism behind the higher drift velocities in the early measurements of Guschin et. al. at \SI{164}{\kelvin}~\cite{gushchin1982electron}, where no information about xenon purification methods was given.

Additional effects can play a role in the detected spread of the charge distribution in our detector and must be corrected to obtain a longitudinal diffusion coefficient that is independent of energy, electron source or detector response. From the original number of extracted photoelectrons to the measured signal in the anode, the following effects can impact the width:
\begin{itemize}
    \item Duration of the pulse of the lamp, which introduces an initial signal width at one sigma of $\SI{2.4(2)}{\micro\second}$. In this work, the initial signal width was derived from the data acquired in vacuum and in LXe, and the signal in the cathode is deconvolved with the detector and electronics responses. 
    \item The detection response of the screening region~\cite{Shockley:1938itm}. The weighting potential between the solid electrodes and the hexagonal meshes is taken into account in the measured signal to yield the original electron cloud spread in the $z$-direction. The effect of the hexagonal screening meshes was simulated by modelling the 3D geometry of the meshes and detector and performing electrostatic simulations with COMSOL. The method to derive the weighting potential is adopted from Ref.~\cite{gook_application_2012}. The weighting potential is obtained by averaging the potential over different electron drift paths to smooth out local effects. 
    \item Coulomb repulsion between electrons, where each electron is affected by the electric field induced by other electrons, can increase the size of the electron cloud. The Coulomb repulsion calculation follows the empirical approach in Ref.~\cite{Njoya:2019ldm}, which considers the ellipsoid explosion model from Ref.~\cite{PhysRevE.84.056404}, where a set of differential equations is solved to obtain the final width of the signal after the drift. The repulsive forces are stronger immediately after the charge creation and become smaller as the electron cloud spreads along the drift path. After the electrons have drifted and reached the anode grid, an additional width value of 5\% compared to the no repulsion forces case is inferred from the empirical approach, and taken as an uncertainty in the charge distribution after their extraction from the photocathode.
    \item Electron attachment to electronegative impurities which can potentially change the distribution of the electron cloud. This could in principle affect the diffusion of electrons in LXe, and the method to estimate this impact is taken from Ref.~\cite{Li:2015rqa}. The longitudinal diffusion coefficient and drift velocity formulae are expanded to include higher-order terms containing the attachment rate. The effects of electron attachment in the diffusion can be neglected according to $\frac{D_L}{(v_\mathrm{d} \cdot d)} \ll 1$, which is the case for this study.
    \item Readout response times of the pre-amplifiers. The circuit response was estimated when benchmarking the electronics and has a negligible effect on the signal shape.
\end{itemize}

Combining all the effects introduced above, a response function is obtained to deconvolve the observed signal. Table~\ref{systematics} compiles the systematic effects treatment for the calculation of the longitudinal diffusion coefficient. The results of this deconvolution, for each measurement at different drift fields, are shown in figure~\ref{fig:longitudinal_diffusion}, together with literature values~\cite{Njoya:2019ldm,Hogenbirk2018}. The longitudinal diffusion coefficient was measured at relatively low drift fields (i.e.~$<\SI{100}{\volt/\cm}$). By using the values derived for the mobility for cold electrons and hot electrons included in  figure~\ref{fig:drift_time_field}, of $\SI{0.29}{\milli\meter^{2}/(\micro \second \cdot \volt)}$ and $\SI{0.01}{\milli\meter^{2}/(\micro \second \cdot \volt)}$, respectively, together with the thermal energy of electrons, reference coefficients for the diffusion can be obtained from equation~\ref{eq:characteristic_energy}. These are included in figure~\ref{fig:longitudinal_diffusion}. By comparing the experimental values with the benchmark equations, the difference between these is contained in terms of the characteristic energy of electrons and their mobility, given equation~\ref{eq:characteristic_energy_longitudinal_diffusion}.

\begin{table}
\centering
\caption{Systematic effects and impact on the uncertainty for the inferred longitudinal diffusion coefficient $D_L$.}
\label{systematics}
% @{\extracolsep{\fill}}
\begin{tabularx}{\columnwidth}{lX}
\hline\noalign{\smallskip}
Systematic effect & Treatment and uncertainty \\
\noalign{\smallskip}\hline\noalign{\smallskip}
\bf{Measured}        & \\
Anode signal width, $\sigma$        & Gaussian plus sine fit, $2-3\%$.  \\
Drift time,   $t_2 + t_{3}$     &  Time interval between extrema of the cathode and anode signal fits. \\
Initial signal width       & Introduced by the lamp pulse. Measurements in vacuum and in LXe,  $\SI{2.4(2)}{\micro\second}$.    \\
Electronics  & RC time constant calculation from a square pulse,  $\SI{0.2}{\micro\second}$.    \\
Drift length, $d_{2} + d_{3}$ & Drift distance of the electron cloud, taken as $\SI{513(7)}{\milli\meter}$ when accounting for the potential contraction of the components at \SI{177}{\kelvin} with an assumed 1\% thermal contraction, and the position of the centre of the cloud distribution in drift regions with respect the extrema of the signals.\\  
Filtering and processing & Maximum 4\% of anode signal width.\\
\noalign{\smallskip}\hline\noalign{\smallskip}
\bf{Simulated}        & \\
Detector response   & COMSOL 3D model of the detector to derive the weighting potential, 10\% uncertainty in the response.\\
\noalign{\smallskip}\hline\noalign{\smallskip}
\bf{Assumed}        & \\
Coulomb repulsion   & Calculated with empirical model from~\cite{Njoya:2019ldm}, assumption of additional 5\% uncertainty in the initial signal spread.\\
Electron attachment   &  Neglected, 4th-order correction: $\frac{D_L}{(v_d \cdot d)} \ll 1$. \\
\hline
\end{tabularx}
\end{table}

From the listed sources of uncertainty for the diffusion coefficient $D_L$, the largest impact originates (in percentages of the total uncertainty from \SI{25}{} to \SI{75}{V/cm}) from the final width measured at the anode (95\% to 80\%), the initial signal width (20\% to 50\%), the Coulomb repulsion (1\% to 5\%), the uncertainty in the drift distance (20 to 5\%) and the drift time (4 to 1\%). 

\begin{figure}[h!]
\centering
\includegraphics[width=0.5\textwidth]{figures/longitudinal_diff.pdf}
\caption[Longitudinal diffusion calculated in this work]{Longitudinal diffusion coefficient calculated in this work, with an electron drift lifetime of $\tau = \SI{649(23)}{\micro\second}$ compared to the results from a purity monitor by Njoya et. al. ($\tau \sim$ $\SIrange{1}{35}{\micro s}$)~\cite{Njoya:2019ldm}, and a TPC from Hogenbirk ($\tau \sim $ $\SI{430}{\micro s}$)~\cite{Hogenbirk2018} and NEST~\cite{szydagis_m_2018_1314669}. The model by NEST version 2.3.7 (solid grey curve) predicted diffusion values approaching zero for lower drift fields. In version 2.3.8, a fix for this behaviour was introduced, as shown in the solid blue curve.}
\label{fig:longitudinal_diffusion}
\end{figure}

NEST version 2.3.7 implemented an empirical model based on previous measurements for the longitudinal diffusion coefficient prediction. At the time of this analysis, NEST lacked data at low drift fields (below $\sim \SI{100}{V/cm}$), and the model predicted a considerably lower longitudinal diffusion coefficient with lower drift fields, in conflict  with the measured values in this and other works. An alternative model is included in NEST, based on differential cross sections derived from Dirac-Fock solutions, combined with Maxwell-Boltzmann distributions~\cite{Boyle:2016wpy}, which do not predict the existing experimental values at all drift fields. The subsequent NEST version 2.3.8 includes a correction on the diffusion modelling, also shown in figure~\ref{fig:longitudinal_diffusion}, resulting from an exchange with the developers. Our analysis aimed to not only infer the values of longitudinal diffusion coefficient at low drift fields for liquid xenon, but also to understand their origin, related to the temperature and purity of the xenon. 