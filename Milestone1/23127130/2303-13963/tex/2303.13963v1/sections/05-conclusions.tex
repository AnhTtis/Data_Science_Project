\section{Conclusions and outlook}
\label{sec:conclusions}

The construction of a next-generation liquid xenon detector at the \SI{50}{t} scale and beyond will face several technological challenges. To address some of these, the Xenoscope facility was designed and built to house a \SI{2.6}{m} tall two-phase TPC at the University of Zurich, with a total LXe mass of $\sim$~\SI{400}{\kg}. After a commissioning run described in Ref.~\cite{Baudis:2021ipf}, we presented here the first results from a run with a \SI{53}{cm} tall purity monitor.

The electron drift lifetime was monitored for 88 days with varying xenon recirculation speeds. For a speed of \SI{40}{slpm}, the highest achieved lifetime was \SI{664(23)}{\micro \s}. A parametric model of the effect of the purification rate, the time-dependent outgassing rate, the liquid-gas impurity diffusion, and the injection of impurities due to operational changes, was fitted to the data. The resulting model was used to predict the electron drift lifetime evolution for different purification conditions and, therefore, inform future design and operation choices.

The electron drift velocity and the longitudinal diffusion coefficient of the electron cloud in liquid xenon were calculated based on data acquired at drift fields between \SI{25}{} and \SI{75}{V/cm}. With the increasing size of LXe TPCs, diffusion strongly affects the position reconstruction of events and the ability to discriminate between single and multiple interactions. Thus, accurate measurements of drift and diffusion properties, combined with an improved understanding of the systematic effect of impurity concentrations on these properties on large scales, are crucial. Our results are in agreement with previous studies both for drift velocity~\cite{gushchin1982electron,Albert:2016bhh,Baudis:2017xov,Thieme:2022dze} and longitudinal diffusion~\cite{Njoya:2019ldm, Hogenbirk2018}. They also triggered an update of NEST~\cite{szydagis_m_2018_1314669}, a simulation package largely used in the community, regarding the modelling of longitudinal diffusion of electron clouds in LXe.

For the next stage of the Xenoscope project, a two-phase xenon TPC was recently built and installed, and will be operated to observe electron drift over distances up to \SI{2.6}{\metre}. The upgrade includes liquid-level control and monitoring, high-voltage supply up to \SI{50}{kV} via a ceramic feedthrough, and an array of silicon photomultipliers for light-readout in the gas phase located just above the gas/liquid interface~\cite{sipm_proceedings}. The latter replaces the charge readout used at the anode of the purity monitor, detecting instead the proportional scintillation produced in the xenon gas region of the TPC. The TPC equipped with the SiPM array will be used to study electron cloud diffusion in both longitudinal and transverse directions. Transverse diffusion is another critical parameter for more accurate modelling of electron transport in xenon-based detectors, for which measurements in the literature are scarce~\cite{Albert:2016bhh, Aprile:2009dv}. Another goal of the upgrade is to study optical properties of liquid xenon at large scales, as well as new types of photosensors under the operating conditions of DARWIN. In addition, the facility will be available to the collaboration for various R\&D projects related to the realisation of a large-scale xenon TPC.


