\pdfoutput=1 
\RequirePackage{fix-cm}
\documentclass[twocolumn,epjc3]{svjour3}  
\RequirePackage{graphicx}
\usepackage{tikz}
\RequirePackage{mathptmx}
\usepackage[english]{babel}
\usepackage{amsmath}
\usepackage{mathtools}
\usepackage[separate-uncertainty=true]{siunitx}
\DeclareSIUnit\bar{bar}
\sisetup{range-phrase = -, range-units=single}
\usepackage{comment}
\usepackage{tabularx}
\usepackage[version=4]{mhchem}
\usepackage[font={small},labelfont=bf,labelsep=quad]{caption}
\usepackage[font={normal}]{subcaption}
\usepackage{cite}
\usepackage{wasysym}
\usepackage{textcomp}
\usepackage{physics}
\usepackage{csquotes}
\RequirePackage[colorlinks,citecolor=blue,urlcolor=blue,linkcolor=blue]{hyperref}

\defineshorthand{"~}{\babelhyphen{nobreak}}
\useshorthands{"}
\newcommand{\LN}{LN$_2$}
\DeclareSIUnit[number-unit-product = {}]{\inch}{"}
%\sisetup{}
\sisetup{per-mode = power}
\sisetup{exponent-product = \cdot}
\sisetup{group-separator = {\,}}
\sisetup{group-minimum-digits = 3}
\DeclareSIUnit\litre{L}
\DeclareSIUnit\liter{L}

% \usepackage[mathlines]{lineno}
% \let\oldequation\equation
% \let\oldendequation\endequation

% \renewenvironment{equation}
%   {\linenomathNonumbers\oldequation}
%   {\oldendequation\endlinenomath}
% \linenumbers

\journalname{Eur. Phys. J. C}
\begin{document}
\sloppy

\title{Electron transport measurements in liquid xenon with Xenoscope, a large-scale DARWIN demonstrator}

\author{L.~Baudis
        \and
        Y.~Biondi\thanksref{e1}\thanksref{e4}
        \and
        A.~Bismark
        \and
        A.~P.~Cimental Ch\'avez
        \and
        J.~J.~Cuenca-Garc\'ia
        \and
        J.~Franchi
        \and
        M.~Galloway
        \and
        F.~Girard\thanksref{e2}
       \and
       R.~Peres\thanksref{e3}
       \and
       D.~Ram\'irez~Garc\'ia
       \and
       P.~Sanchez-Lucas\thanksref{e5}
       \and
       K.~Thieme\thanksref{e6}
       \and
       C.~Wittweg
}
\thankstext{e1}{e-mail: \url{yanina.biondi@physik.uzh.ch}}
\thankstext{e2}{e-mail: \url{frederic.girard@physik.uzh.ch}}
\thankstext{e3}{e-mail: \url{ricardo.peres@physik.uzh.ch}} 
\thankstext{e4}{Now at Karlsruhe Institute of Technology}
\thankstext{e5}{Now at University of Granada}
\thankstext{e6}{Now at University of Hawai\textquoteleft{i} at M\={a}noa}
\institute{\normalsize{Department of Physics, University of Zurich, Winterthurerstrasse 190, 8057 Zurich, Switzerland}
\label{addr1}}

\date{Received: date / Accepted: date}

\maketitle

\begin{abstract}
There is a compelling physics case for a large, xenon-based underground detector devoted to dark matter and other rare-event searches. A two-phase time projection chamber as inner detector allows for a good energy resolution, a three-dimensional position determination of the interaction site and particle discrimination. To study challenges related to the construction and operation of a \mbox{multi-tonne} scale detector, we have designed and constructed a vertical, full-scale demonstrator for the DARWIN experiment at the University of Zurich. Here we present first results from a several-months run with {\SI{343}{kg}} of xenon and electron drift lifetime and transport measurements with a \SI{53}{cm} tall purity monitor immersed in the cryogenic liquid. After \SI{88}{\day} of continuous purification, the electron lifetime reached a value of $\SI{664(23)}{\micro\second}$. We measured the drift velocity of electrons for electric fields in the range (25--75)\,V/cm, and found values consistent with previous measurements. We also calculated the longitudinal diffusion constant of the electron cloud in the same field range, and compared with previous data, as well as with predictions from an empirical model.
\end{abstract}

\section{introduction}

% 1. importance of TKGs and reasoning on TKGs. 
% 2. low resource languages, main main idea.
% 3. relations and limitations of current works.
% 4. summarize our solutions and contributions.

Temporal Knowledge Graphs (TKGs)~\cite{YAGO,ICEWS18,WIKI,acekg} characterize temporally evolving events, where each event, represented as ({\em subject}, {\em relation}, {\em object}), is associated with temporal information ({\em time}), e.g., ({\em Macron}, {\em reelected}, {\em French president}, {\em 2022}). TKGs has facilitated various knowledge-intensive Web applications with timeliness, such as question answering~\cite{KBQA}, product recommendation~\cite{RippleNet,TKG4Rec,TKG4Rec2,RETE}, and social event forecasting~\cite{KG4Social,DyDiff-VAE,andgan,belief,misinfo,polarization}. 

As new events are continually emerging, modern TKGs are still far from being complete. Conventionally, the TKG construction process relies primarily on information extraction from unstructured corpus~\cite{WIKI,YAGO, EventKG}, which necessitates extensive manual annotations to keep up with changing events. For instance, the recent transition from Trump to Biden as the President of the United States has not been reflected in many TKGs, highlighting the need for timely updates. This spurs research on temporal knowledge graph reasoning to automate evolving events prediction over time~\cite{TA-DistMult,Know-Evolve,Renet,RE-GCN}. Unfortunately, the problem of TKG incompleteness is particularly pronounced in low-resource languages, where it is unable to collect enough corpus and annotations to support robust TKG construction. This results in suboptimal reasoning performance and distinctly unsatisfying accuracy in predicting recent and future events.

% whose performance can degrade significantly in low-resource language TKGs that suffer from severe incompleteness over time. 
% \jingfeng{why don't people  study cross-lingual TKG previously, (i.e. use language alignment to improve TKG). Is it really helpful intuitively to use high resource language to help TKGC? For instance, is it enough to use static langauge-alignment to help KGC, ignoring the temporal information? Are those langauge-alignment changing across time?}



\begin{figure}
    \centering
    \includegraphics[width = 1.0\linewidth]{fig/task.pdf}
    \caption{An illustrative example of cross-lingual reasoning on TKGs. 1) We aim to transfer knowledge from English TKG to Japanese TKG, where the English version provides more complete information; 2) Cross-lingual alignments only cover a small ratio of entities, e.g., Apple Inc; 3) Cross-lingual alignments can be noisy and misleading, e.g., A city called Ventura is linked to new macOS Ventura at $t_2$, introducing noise for reasoning in Japanese.}
    \label{fig:illustration}
    %\vspace{-6mm}
\end{figure}

Inspired by the incompleteness issue facing low-resource languages in constructing TKGs, we introduce a novel task named Cross-Lingual Temporal Knowledge Graph Reasoning (as shown in Figure~\ref{fig:illustration}). This task aims to alleviate the reliance on supervision for TKGs in low-resource languages (referred to as the target language) by transferring temporal knowledge from high-resource languages (referred to as the source language)~\footnote{In this paper, for the sake of brevity, we interchangeably use the terms high-resource/low-resource and source/target.}. In contrast, all the existing efforts are either limited to reasoning in monolingual TKGs (usually high-resource languages, e.g., English)~\cite{TA-DistMult,Know-Evolve,Renet,RE-GCN}, or multilingual static KGs~\cite{KEnS,AlignKGC,SS-AGA}. To the best of our knowledge, cross-lingual TKG reasoning that transfers temporal knowledge between TKGs has not been investigated. 

%Motivated by this, we study a new task named {\em cross-lingual temporal knowledge graph reasoning} as shown in Figure~\ref{fig:illustration}, to alleviate the heavy dependence on supervision for any resource-poor language TKGs by distilling the temporal knowledge from resource-rich ones. Differently, all the existing efforts are either limited to reasoning in monolingual (usually high-resource languages, e.g., English) temporal KGs~\cite{TA-DistMult,Know-Evolve,Renet,RE-GCN}, or multilingual static KG~\cite{KEnS,AlignKGC,SS-AGA}, but neglecting the reasoning in a both temporal and cross-lingual manner that highly requires capturing time-evolving patterns and language discrepancy. To the best of our knowledge, this problem, regarding how to transfer cross-lingual knowledge between TKGs, has still not been formally investigated. 

% Unlike conventional TKG reasoning, 
The fulfillment of this task poses tremendous challenges in two aspects: 1) \textbf{Scarcity of cross-lingual alignment}: as the informative bridge of two separate TKGs, cross-lingual alignment is imperative for cross-lingual knowledge transfer~\cite{AlignKGC,KEnS,SS-AGA}. However, obtaining alignments between languages is a time-consuming and resource-intensive process that heavily relies on human annotations. The transfer of knowledge through a limited number of alignments is often insufficient to fully enhance the TKG in the target language. 2) \textbf{Temporal knowledge discrepancy}: the information associated with two aligned entities is not necessarily identical, especially with regards to temporal patterns. Utilizing a rough approach to equate the aligned entities at all times can result in the transfer of misleading knowledge and negatively impact performance. This becomes more pronounced when the alignments are noisy and unreliable. For example, at the time step $t_2$, a new event about operating system ``{\it Ventura}'' from Apple company occurs in the source English TKG, and meanwhile there is a noisy aligned entity ``{\it Ventura city}'' in the target Japanese TKG. Directly pulling those two entities at this point, can inevitably introduce  noise and fail to predict a set of related events in the target TKG. Therefore, it is crucial to dynamically regulate the alignment strength of each local graph structure over time in order to maximize the effectiveness of cross-lingual knowledge distillation.

% Pulling those entities together cannot augment information in target languages. Small alignment strength is beneficial in the unreliable alignment cases, otherwise the misleading knowledge transferring can even hurt the performance.

% Moreover, in a case that the alignments are not fully reliable, directly pulling the two aligned entities together 


% optimally dynamic alignment strength
% {\em Optimal alignment strength to maximize the benefits of knowledge distillation is difficult to obtain, especially in the temporal manner.} 
% In practical, although the aligned entities can share similar information, they may still differ in other perspectives, including but not limited to frequency, interactions, and temporal patterns. How to adjust the alignment strength (i.e., the distance constrains of the aligned entities in the uni-space) accordingly for different entities at different time is unclear. \zheng{Ruijie TODO: add Ventura case}Moreover, in a case that the alignments are not fully reliable, directly pulling the two aligned entities together can even hurt the performance.



% scarcity of hinders the efficient
% knowledge transfer across languages. 
% {\em Transferring knowledge through a small set of alignments is hard to augment information for all entities.} 

% Aligning the same entities across languages rely heavily on manual labeling or rule-based inference~\cite{EA1,EA2,EA3,selfKG}, which is too time-consuming and impractical to obtain the alignments covering most of the entities in target language. 

% In this paper, we study how to boost the TKG reasoning performance in low-resource languages by explicitly increasing the completeness of those TKGs in history. Instead of improving the underlying information extraction techniques in low-data regime, we propose a new task called {\em Cross-lingual Temporal Knowledge Graph Reasoning}, motivated by the facts that there exists common or complementary knowledge shared by the TKGs in different languages under similar topics. The new task aims to facilitate TKG reasoning in low-resource languages (target languages) by distilling knowledge from a corresponding TKG in high-resource language (source language)  through a small set of entity alignments as bridges~\footnote{In this paper, we interchangeably use the terminology high-resource/low-resource and source/target for briety.}. Figure~\ref{fig:illustration} provides an illustrative example of the proposed task.


% Unfortunately, recent breakthroughs in temporal knowledge graph reasoning model~\cite{TA-DistMult,Know-Evolve,Renet,RE-GCN} highly rely on the completeness of the TKGs, especially for the most recent events. 

% However, the completeness of TKGs varies a lot across different languages, even under similar topics. Conventionally, the TKG construction process relies primarily on information extraction techniques built on the unstructured corpus~\cite{WIKI,YAGO, EventKG}. Therefore, the amount of corpus and human annotations in different languages significantly influence the quality of the corresponding TKGs . 
% Therefore, automatically completing/updating TKGs has been attracting enormous interests in recently years, which aims to predict recent/future events on TKGs based on historical events~\cite{TA-DistMult,Know-Evolve,Renet,RE-GCN}, namely temporal knowledge graph reasoning~\footnote{Broadly speaking, TKG reasoning includes interpolation to predict historical events and extrapolation to predict future events. In this paper, we refer to extrapolation task as TKG reasoning, since it is more vital for time-sensitive downstream tasks.}.


% For languages with large-scale and carefully labeled corpus (we refer to as high-resource languages, e.g., English), the constructed TKGs are more comprehensive than TKGs in other languages that lack the high-quality corpus (we refer to as low-resource languages, e.g., Spanish, Slovene, Danish, etc). Such completeness discrepancy leads to distinctly uneven TKG reasoning performances in different languages, which in turn affects the quality of service of the downstream applications. 


% Compared with the traditional TKG reasoning task, the new task imposes non-trivial challenges. An intuitive solution is to construct a unified graph including two TKGs in both source and target languages, and the knowledge distillation can be fulfilled by pulling the aligned entities from two languages close to each other in the uni-space~\cite{AlignKGC,KEnS}. However, there are still two challenges to be addressed. 

% \zheng{Ruijie TODO, Place this part to related works.}
% Existing works in related areas fail to address the aforementioned challenges. Monolingual reasoning methods on static/temporal knowledge graphs~\cite{TransE,TranR,ComplEX,RotatE,TA-DistMult,Know-Evolve,Renet,RE-GCN} is incapable of the desired knowledge transferring due to the insufficient alignment modeling. Although they can be extended on the cross-lingual scenario by viewing the alignments as a new relation on the merged TKGs, the limited amount of alignments prevent them from augmenting information for most of the entities. Entity alignment methods on KGs~\cite{EA1,EA2,EA3,EA4,EA5,selfKG} can automatically enlarge the alignments by  predicting the correspondence between the two TGs. But most of them, if not all, require the relatively even completeness of two TGs to capture the structural similarities, which can not be satisfied in our case, as target TKGs are far from complete. Some recent works start to study the multilingual TK reasoning on static graphs~\cite{AlignKGC,KEnS,SS-AGA}, which similarly aim to extract knowledge from several source KGs to boost the reasoning performance in the target KG, while they still require a sufficient amount of cross-lingual alignments and totally ignore the temporal perspective in our task.

% to facilitate temporal knowledge graph reasoning in low-resource languages. 
% increase the TKG connection and target TKG capacity
% In light of the mutual benefits, we iteratively generate pseudo alignment pairs and pseudo temporal events to address the co-existing scarcity issue in both cross-lingual alignment and target TKGs. 


In this paper, we propose a novel Mutually-paced Knowledge Distillation (\model) framework, where a teacher network learns more enriched temporal knowledge and reasoning skills from the source TKG to facilitate the learning of a student network in the low-data target one. The knowledge transfer is enabled via an alignment module, which estimates entity correspondence across languages based on temporal patterns. Firstly, to alleviate the limited language alignments (\textbf{Challenge \#1}), such a knowledge distillation process is mutually paced over time. This means, on one hand, we encourage the mutually interactive learning between the teacher and student. Concretely, the alignment module between the teacher and the student learns to generate pseudo alignment between TKGs to maximally expand the upper bound of knowledge transfer. And subsequently, it empowers the student to encode more informative knowledge in target TKG, which can in turn boost the alignment module to explore more reasonable alignments as the bridge across TKGs. One the other hand, inspired by self-paced learning~\cite{spl-1,spl-2}, we make the generations as a progressively easy-to-hard process over time. We start from generating reliable pseudo data with high confidence. As time goes by, we then gradually increase the generation amount by relieving the restriction over time. Secondly, to inhibit the temporal knowledge mismatch (\textbf{Challenge \#2}), the attention module can estimate the graph alignment strength distribution over time. This is achieved by a temporal cross-lingual attention in terms of the local graph structure and temporal-evolving patterns of aligned entities. As such, it can dynamically control the negative effect and suppress noise  propagation from the source TKG. Moreover, we provide a theoretical convergence guarantee for the training objective on both initial ground-truth data and pseudo data. To evaluate \model, we conduct extensive experiments of 12 cross-lingual TKG transfer tasks in multilingual EventKG dataset~\cite{EventKG}. Our empirical results show that the \model method outperforms state-of-the-art baselines in both with and without alignment noise settings, where only $20\%$ of temporal events in the target KG and $10\%$ of cross-lingual alignments are preserved.

% To validate the effectiveness of \model, we conduct extensive experiments of 12 cross-lingual TKG transfer tasks in multilingual EventKG benchmark dataset~\cite{EventKG} . Our experimental results empirically demonstrate the superiority of the \model method over state-of-the-art baselines, ranging from static KG embedding~\cite{TransE,TransR,DistMult,RotatE}, temporal KG reasoning~\cite{TA-DistMult,Renet,RE-GCN} to multilingual KG completion~\cite{KEnS,AlignKGC,SS-AGA}, in both with and without alignment noise settings. We further conduct comprehensive ablation and hyperparameter studies to validate the effectiveness of each design choices. Moreover, we provide theoretical analysis of convergence guarantee for the training objective on both initial groundtruth data and pseudo generative data.



To sum up, our contributions are three-fold:

\begin{itemize}[leftmargin = 15pt]
    \item \textbf{Problem formulation}: We propose the cross-lingual temporal knowledge graph reasoning task, to boost the temporal reasoning performance in target TKG by transferring knowledge from source TKG;
    \item \textbf{Novel framework}: We propose a novel \model framework, which enables the mutually-paced learning between the teacher and student networks, to promote both pseudo alignments and knowledge transfer reliability. Besides, \model involves a dynamic alignment estimation across TKGs that inhibits the influence of temporal knowledge discrepancy.
    \item \textbf{Extensive evaluations}: Empirically, extensive experiments on 12 cross-lingual TKG transfer tasks in multilingual EventKG benchmark dataset demonstrate the effectiveness of \model.
\end{itemize}
% pseudo data generation technique to progressively enhance the training data. The generated pseudo alignments can help the training of the representation modules by the knowledge distillation, and in turn adding pseudo events in the target TKG can improves alignment module by providing high-quality representations. 




% interactively
% TKGs in a source language and a target language are represented by a teacher representation module and a student one into a uni-space, respectively. 
% The knowledge distillation is enabled by a cross-lingual alignment module which pulls the aligned entities close to each other and push other entities far away. 
% To address the challenge caused by the scarcity of cross-lingual alignment, 



\section{The Xenoscope facility}
\label{sec:experimental_setup}

Xenoscope can house up to \SI{400}{kg} of LXe in a double-walled stainless steel cryostat. The facility, its subsystems, and the outcome from the first commissioning run are described in Ref.~\cite{Baudis:2021ipf}. Xenoscope was first equipped with a purity monitor (section \ref{sec:purity_monitor}) fully submerged in LXe, while the cryostat aspect ratio was chosen to allow for the operation, in the next phase of the project, of a \SI{2.6}{m} two-phase TPC, with the primary goal of demonstrating the drift of electrons in LXe over this distance for the first time. A computer-aided design (CAD) rendering of the cryostat with the purity monitor is shown in figure~\ref{fig:phases}.

The facility includes a gas purification system with a series of filters and a commercial zirconium alloy getter. The LXe is extracted at the top of the liquid column, where the impurity concentration is higher. It is evaporated in the heat exchanger system and circulated through the purification system at a fixed flow. The purified xenon is recondensed in the heat exchanger and reintroduced in the cooling tower, which comprises a pulse tube refrigerator (PTR) connected to a cold head mounted atop the cooling chamber. The xenon is then directed to the bottom of the cryostat. A slow control system built from open-source software oversees and sends alarms on relevant parameters.

Two system upgrades were performed prior to the installation of the purity monitor. First, a pre-cooler was manufactured and installed at the top of the inner cryostat vessel to provide additional peak cooling power, and thus to reduce the system cooldown and xenon liquefaction time during filling by a factor of 4.25. The design of the pre-cooler and details of its commissioning are presented in~\ref{sec:pre-cooler}. Furthermore, a gravity-assisted recuperation and storage system for LXe, Ball of Xenon (BoX), was deployed to allow for the storage of up to \SI{450}{\kilogram} of xenon at room temperature, as well as for recuperation in liquid phase. The latter enhances the speed of the recuperation process by a factor $\sim$ 8 compared to gaseous recuperation to a bottle array via cryogenic pumping. More details of its design and performance are presented in~\ref{sec:BoX}.

\begin{figure}[ht!]
\centering
\includegraphics[width=\columnwidth]{figures/render-PM3.pdf}
\caption[The purity monitor in the Xenoscope cryostat]{The purity monitor in the Xenoscope cryostat. Legend: (1)~top flange; (2)~outer vessel; (3)~inner vessel; (4)~pre-cooler; (5)~purity monitor; (6)~BoX recuperation line; (7)~anode; (8)~anode grid; (9)~field-shaping rings and resistor chain; (10)~support pillars; (11)~cathode grid; (12)~cathode disk; (13)~photocathode and optical fibre.}
\label{fig:phases}
\end{figure}



\section{The purity monitor and measurements}
\label{sec:purity_monitor}

Common impurities in commercially available xenon consist of parts-per-million (ppm) levels of O$_{2}$, N$_{2}$, H$_{2}$O, CO, as well as organic molecules~\cite{Hasterok:2017ehi}. Additionally, detector and subsystem materials introduce impurities by outgassing. The purification of xenon prevents electron losses via their attachment to electronegative impurities and allows to achieve high light and charge yields.

Most purity monitors measure the charge deficit of an initially known population of electrons after their drift through the liquid. By comparing the number of electrons before, $N_0$, and after the drift, $N(t_\mathrm{d})$, an indirect measurement of the impurity concentration in LXe can be achieved. The deficit can be modelled as a decaying exponential:
%
\begin{align}    
    N(t_{\text{d}}) = N_0~\mathrm{e}^{-t_{\text{d}}/\tau}~,
    \label{eq:electron_loss}
\end{align}
where $\tau$ is the electron drift lifetime. It relates to the concentration of electronegative impurities as:
%
\begin{align}
    \tau = \frac{1}{\sum_{i} k_{i} n_{i}}~,
    \label{eq:attachement}
\end{align}
\noindent where $k_i$ is the attachment rate specific to the impurity type in units of volume per mol per time, usually given in $\si{\mathrm{\liter/(\mol}\cdot\mathrm{\second})}$, $n_i$ is the impurity concentration given in \si{\mol/\liter}, and the sum extends over the different electronegative species $i$ in the LXe. The attachment rate coefficient depends on the electric field strength.

A schematic of the working principle of the purity monitor is shown in figure~\ref{fig:purity_monitor_concept}, left. An optical fibre transmits the light from a xenon flash lamp to the centre of a photocathode. The incident photons produce electrons via the photoelectric effect. The electrons are drifted via extraction, drift, and collection electric fields, generated by four biased electrodes. The first drift region (1) is located between the cathode (with the photocathode in the centre) and the cathode grid; the second region (2) extends up to the anode grid; the third region (3) extends from the anode grid to the anode. The charges induce a current signal in the cathode as they drift towards the cathode grid. The screening grids prevent current induction in the cathode and anode when the electrons are drifting along the second region. Once the electrons reach the third drift region, a second signal is generated at the anode, until the electrons are fully collected. Two electronic circuits amplify and convert the induced currents to voltage signals. 

The data is acquired and digitised, triggered by the pulse generator which also starts the discharge in the xenon flash lamp, with a window of \SI{100}{\micro \second} for the anode and cathode waveforms. Once digitised, the voltage signals are integrated to obtain the charges, i.e., the number of extracted and surviving electrons. With the induced charges and the time between the two signals, which corresponds to the drift time for the applied electric field, the electron drift lifetime can be inferred by solving numerically the equation:

\begin{align}
\frac{Q_A}{Q_C} = \frac{t_1}{t_3} \mathrm{e}^{-(t_1+t_2+t_3)/\tau} \frac{(\mathrm{e}^{t_3/\tau} - 1)}{(\mathrm{e}^{-t_1/\tau} - 1)}~.
\label{eq:lifetime}
\end{align}

\noindent Here, $Q_A$ and $Q_C$ are the charges from the integrated signals measured in the anode and cathode, respectively, $t_{1}$ is the rise time of the first signal, $t_{2}$ is the time between the minimum of the signal in the cathode and the rise time of the signal in the anode, with $t_{3}$ the time from $t_{2}$ up to the maximum of the anode signal. Given the motion of the charges, the signal in the cathode has negative polarity, while in the anode the polarity is positive. 
Figure~\ref{fig:signals_inLXe} shows an example of signals acquired in LXe from the cathode and anode at \SI{40}{slpm} along with the three drift times.

The design of the Xenoscope purity monitor is described in detail in Refs.~\cite{Baudis:2021ipf,Biondi:2022T}, and the assembled module is shown in figure~\ref{fig:purity_monitor_concept}, right. It features a field cage built with high conductivity, oxygen-free copper rings, supported by six polyamide-imide pillars. The rings are connected by a resistor chain of $\SI{5}{G\Omega}$ impedance each, and enclose a cylindrical drift region of \SI{15}{\cm}\,\diameter\,$\times$\,\SI{53.1}{\cm}. The cathode and anode grids consist of hexagonally-patterned, etched stainless steel meshes with high optical transparency ($\sim 93\%$), while the cathode and anode are solid stainless steel disks.


\begin{figure}[h!]
\centering
\includegraphics[width=0.24\textwidth]{figures/pm_schematic.jpg}
\includegraphics[width=0.15\textwidth]{figures/PM-high-res.jpg}
\caption{(Left): Schematic of the purity monitor. A pulse generator triggers a flash from the xenon lamp and the light is transmitted through an optical fibre to the photocathode, where photoelectrons are produced. The electrons are extracted, transported and collected by three electric fields, defined by the cathode ($\mathrm{G_C}$) and cathode grid ($\mathrm{G_1}$), the anode grid ($\mathrm{G_2}$) and the anode ($\mathrm{G_A}$). In the longest region (2), the field shaping rings~(FSR) maintain the uniformity of the drift field $\vec{E_d}$ in the vertical direction. (Right): Assembled purity monitor in Xenoscope.}
\label{fig:purity_monitor_concept}
\end{figure}

\begin{figure}[h!]
\centering
\includegraphics[width=\columnwidth]{figures/signals.pdf}
\caption[Signal acquired at \SI{40}{slpm} xenon recirculation speed.]{Signals acquired at \SI{40}{slpm} xenon recirculation The rise time of the cathode signal (blue) is taken as $t_{1}$, the time interval between the minimum of the cathode signal and the start of the anode signal (red) is taken as $t_{2}$, with half of the charge cloud completely collected at $t_{3}$. These values are later used to calculate the electron drift lifetime.}
\label{fig:signals_inLXe}
\end{figure}
%=============================================================================
\subsection{Optical components and photocathode}

The utilised lamp is a \SI{60}{\watt} xenon flash lamp with a \mbox{built-in} reflective mirror (model number L7685) from \textit{Hamamatsu}~\cite{hamamatsu}. The window is a single sapphire crystal allowing short wavelength light ($\sim$~\SI{190}{\nano\meter}) to reach the output of the lamp, with a spectral emission from \SI{190}{\nano\meter} to \SI{2000}{\nano\meter}. The lamp generates a discharge which excites the gas producing scintillation, with reflective mirrors directing photons from all directions towards the output. The xenon lamp can be triggered internally, or externally via a pulse generator. The intensity of the light emission is adjusted by setting the voltage for the discharge in the lamp between \SI{600}{\volt} and \SI{1}{\kilo\volt}. The selection of the latter maximises the number of produced electrons.

 Measures were adopted to mitigate the electronic noise produced in the signal waveforms by the external trigger: the xenon flash lamp was rehoused in a stray electromagnetic interference box, and galvanic insulation and ferrite filters were added to the trigger line. The box was customised by adding a potentiometer to manually change the voltage of the discharge. A trigger circuit controlled by the pulse generator was added as well. The lamp is placed outside the cryostat, with an optical fibre carrying the light from its output to the surface of the photocathode. A UV grade sapphire lens produced by \textit{Hamamatsu} is placed at the output of the lamp to collimate the light to the optical fibre. The selected fibre is $\SI{600}{\micro\meter}$ in diameter, ultra-high vacuum rated with a polyimide buffer from \textit{LewVac}~\cite{LewVac}. It is resistant to UV solarisation, i.e.~the degradation in the fibre material due to the exposure to light of wavelength lower than \SI{300}{\nano\meter}. The fibre feedthrough, produced by \textit{Thorlabs}~\cite{feedthrough}, consists of a $\SI{600}{\micro m}$ multimode fibre in an SMA connector welded on a CF40 flange. The feedthrough requires the fibres to be terminated, hence these were prepared and polished in-house with a set of 8~lapping sheets, made of aluminium oxide, silicon carbide, and calcined alumina, from grits of $\SI{30}{\micro\meter}$ to $\SI{0.02}{\micro\meter}$. 
 
One of the critical parts of a purity monitor is the photocathode, for it directly impacts the size of the initial signal. It consists of a thin layer of a low work function metal, deposited on a quartz substrate that has low absorption of UV photons~\cite{Valentini2002}. The photocathode was produced in-house using a turbomolecular pumped coater Q150T Plus from \textit{Quorum}~\cite{coater}. The desired thickness of the layer was monitored with a quartz crystal microbalance. Different materials for the thin layer were tested, including gold and silver, and the coater was used to produce photocathodes of \SI{50}{\nano\meter} thickness on a \SI{2}{\mm} thick quartz substrate, with a diameter of \SI{30.00(5)}{\mm}. The deposition of a \SI{5}{\nano\meter} thick layer of titanium was required in the case of gold for adhesion to the substrate. The choice of thickness was based on the effective probe depth of gold layers, and previous works~\cite{Manenti2020}. Additional technical details can be found in Ref.~\cite{Biondi:2022T}.

The photocathodes were tested in a vacuum setup, where the xenon lamp was flashed onto the photocathode material and the induced current was measured. Both gold and silver showed high yields, with gold reaching a stable state in fewer hours of exposure to the UV signal. Silver and gold photocathodes showed an increasing quantum efficiency with time when exposed to light, and this increased yield did not revert back in subsequent tests. The increase in quantum efficiency of the photocathode with UV-light exposure was also observed in Ref.~\cite{Manenti2020}. The gold photocathode was selected due to its stability over time and higher quantum efficiency than the silver photocathode, requiring smaller electric fields to produce a higher charge signal. 

%===============================================================================
\subsection{Current readout and signal processing}

The readout electronics amplify the induced currents from the cathode and anode and are placed inside the cryostat to avoid signal losses along the \SI{9}{\meter} signal cables. The circuits were designed together with the Electronics Workshop at the University of Zurich. The circuit consists of an AC-coupling component, a transimpedance amplifier, and a final voltage amplifier with a $\SI{50}{\ohm}$ impedance termination to match the one from the data acquisition. The transimpedance and voltage amplifiers are implemented with two low-cost operational amplifiers, model AD8066 from \textit{Analog Devices}~\cite{analogdev}. An AC-coupling filter in the circuit board removes high-frequency noise, which enhances the signal quality, and the AC-coupling removes the DC component of the HV applied to the electrodes. The usage of a transimpedance amplifier, in contrast to a charge amplifier, allows for more precise timing and signal spread analyses due to its small resistive-capacitive constant (RC) and short rise time of \SI{0.14}{\micro\second}. However, due to its fast response, a low-pass filter for frequencies below \SI{800}{\kilo\hertz} is applied to the signals to decrease the electronic noise induced by, e.g., the pulse generator that triggers the lamp, two temperature sensors, and the uninterruptible power supply. The preamplifier operates in current mode, as the capacitance discharges rapidly, resulting in an output voltage proportional to the instantaneous current. The frequency response of the readout electronics was benchmarked, with a negligible effect on the signal shape due to the $\SI{100}{MHz}$ bandwidth. 

The performance of the readout electronics was tested in a climate chamber in steps of \SI{10}{\kelvin} from room temperature down to \SI{190}{\kelvin}. The calibration showed a charge amplification of $\SI{0.18}{\femto\coulomb/(\milli\volt\cdot\micro\second)}$, with good thermal stability. Additionally, the RC decay constant of the circuit, which could be a source of systematic error for time measurements, was estimated at $\sim$~\SI{150}{\nano\second} by feeding a \SI{2}{\micro\second} wide square pulse to the circuit.

An oscilloscope, \textit{Teledyne LeCroy} model Waverunner 6104A~\cite{teledyne}, and an analog-to-digital converter from \textit{CAEN}, model v1724~\cite{CAEN}, acquired the waveforms produced by the cathode and anode readout. Each acquisition consisted of the average of \SI{1000}{} waveforms acquired over \SI{16.7}{minutes} to minimise the baseline noise. The signals were then processed by fitting the expected signal shape with a Gaussian distribution. In some waveforms, a noise introduced by external electronic devices could be discerned as part of the background noise, and the fit included a sine function to account for this effect, with an inferred model uncertainty of $5\%$ for the ones where the sine fit to the noise did not converge. The current-equivalent voltage signals in the cathode and anode were integrated to obtain a charge-proportional value. The residuals of the fits were used as weights for the charge values obtained in the averaged data shown in the next section. The uncertainties in charges and times obtained in the fits were propagated to obtain the uncertainty of the electron drift lifetime value. An example of the raw anode signal at \SI{53}{\volt/\cm} drift field in region 2 is shown in figure~\ref{fig:FWHM_signals}, together with the post-processing signal with a low-pass filter. The calculated baseline and Gaussian fit of the signals are also shown.

\begin{figure}[h!]
\centering
\includegraphics[width=0.49\textwidth]{figures/signal_example_fit.pdf}
\caption[Anode signals at \SI{53}{\volt/\cm}]{Anode signal at \SI{53}{\volt/\cm} prior to (blue) and after (orange) the low-pass filter. The calculated baseline (red) and a Gaussian fit of the signal (green) are also shown. The signal is an average over \SI{1000}{} recorded waveforms.}
\label{fig:FWHM_signals}
\end{figure}

\subsection{Measurements}

Once installed in the cryostat, the purity monitor was first operated in vacuum ($\sim \SI{1e-5}{\milli\bar}$). Data was acquired to investigate the signal shape and response in this configuration with negligible charge losses due to residual gas. The measurement additionally provided the delay time of the electronics chain, from the pulse generator for the xenon lamp to the signal amplification and readout of $\SI{18}{\micro\second}$.

After the calibration of the purity monitor in vacuum, gaseous xenon was flushed inside the detector and purified through recirculation in the gas system. The LXe run started with the filling of \SI{343}{\kilogram} of xenon. As the xenon recirculates through the getter, electronegative impurities are removed, and the electron drift lifetime is expected to increase in two steps: an initial exponentially increasing phase where the bulk impurities are rapidly removed, and a second phase where the change is dominated by the materials outgassing, and where the electron drift lifetime slowly increases over time. At different recirculation speeds, the electron drift lifetime reaches increasingly higher values in the second phase.

The recirculation speed was set with flows of \SI{30}{standard\ litres \ per\ minute\ (slpm)}, \SI{35}{slpm} and \SI{40}{slpm}, with the xenon lamp illuminating the photocathode with a frequency of \SI{1}{\hertz}. In the cryostat, the temperature and pressure were maintained around \SI{177.6(1)}{\kelvin} and  \SI{2.05(1)}{\bar}, respectively. Following the commissioning run, the displacement of the GXe compressor was reduced to increase its lifespan. This constrained the maximum purification speed to \SI{40}{slpm}, compared to the \SI{80}{slpm} reported in Ref.~\cite{Baudis:2021ipf}. The initial impurity level in the xenon gas impacts the number of days before a signal can be seen in the purity monitor: the  first waveforms in the cathode and anode were observed after $\sim$26.5 days.

During data taking, the cathode and cathode grid were biased at \SI{-2710}{\volt} and \SI{-2650}{\volt}, respectively. The anode grid was kept at ground while the anode was biased at \SI{500}{\volt}. The values were selected based on COMSOL~\cite{multiphysicscomsol} simulations which yielded nearly $100\%$ extraction efficiency of the electrons produced in the centre of the photocathode. Table~\ref{table:drift_times} shows the summary of the distances, times electric fields for the  extraction (1), drift (2) and collection (3) regions.
\begin{table}[h!]
\centering
\caption[Distance and drift fields]{Electric fields, distances and times $t_{i}$ measured for the three regions in the PM, with voltages \SI{-2710}{\volt}, \SI{-2650}{\volt}, \SI{0}{\volt}, and \SI{500}{\volt} for the cathode, cathode grid, anode grid and anode, respectively, for a purification speed of \SI{40}{slpm}.}
\begin{tabular}{clll} 
\hline 
Drift region $i$ & Distance [mm] & Field [V/cm] & $t_{i}$ $[\SI{}{\micro \second}]$\\
\hline
 1 &  $18 \pm 1$  & $33 \pm 1 $ & $12.8 \pm 0.8$\\
 2 &  $503 \pm 5$ &  $53 \pm 1 $ & $433.5\pm0.7$ \\
 3 &  $10 \pm 1$  & $500 \pm 5$ & $7.6 \pm 0.7$
\label{table:drift_times}
\end{tabular}
\end{table}

Figure~\ref{fig:charge_anode_cathode} shows the anode and cathode signals with their integral, where the integrated signals show a step-like feature after the charges move entirely to the next drift region, or are collected in the anode. The integration corresponds to the total area of the Gaussian fit. The charge measured in the cathode corresponds to $N_{\text{e}^{-}} \cong 10^{6}$ electrons extracted from the photocathode at each pulse.

\begin{figure}[b]
\centering
\includegraphics[width=\columnwidth]{figures/charge_cathode.pdf}
\centering
\includegraphics[width=\columnwidth]{figures/charge_anode.pdf}
\caption[Signal readout at the cathode with the integrated charge signal]{Signal readout (blue) at the cathode (top) and anode (bottom)  with their respective Gaussian fits (orange) and integrated charge signals (red).}
\label{fig:charge_anode_cathode}
\end{figure}




\section{Results and discussion}
\label{sec:results_discussion}

\begin{figure*}[t!]
\centering
\includegraphics[width=2\columnwidth]{figures/Elife_model.pdf}
\caption[Purification flow-dependent electron drift lifetime measured in the Xenoscope.]{Purification flow-dependent electron drift lifetime measured in Xenoscope. The data was averaged in \SI{6}{\hour} time bins. The dashed lines indicate a change in flow, while the dash-dotted lines indicate short-term irregularities in the pressure and flow conditions (see text). The red line shows the best-fit model from equations \ref{eq:e-lifetime-gas} and \ref{eq:e-lifetime-liquid}, while the black points show the residuals.}
\label{fig:electron_lifetime_datasets}
\end{figure*}

 The electron drift lifetime measurement campaign with the purity monitor lasted a total of \SI{88}{days}. The purification was performed at \SI{30}{slpm} for \SI{46.6}{days}, at \SI{35}{slpm} for \SI{20.0}{days}, and at \SI{40}{slpm} for \SI{21.2}{days}. After the electron drift lifetime measurements in LXe, signals for drift fields from \SI{25}{V/cm} to \SI{75}{V/cm} were acquired to study field-dependent electron transport properties, such as the drift velocity and longitudinal electron cloud diffusion.

\subsection{Electron drift lifetime}

Figure~\ref{fig:electron_lifetime_datasets} shows the electron drift lifetime calculated with the charge signals acquired at the cathode and anode over the entire acquisition period. When the recirculation speed changes, the electron drift lifetime drops, most likely due to a change in the height of the liquid level, resulting in the release of trapped impurities in the high-surface tension region at the LXe \enquote{collar} (LXe/GXe/inner vessel interface). A drop in electron drift lifetime was also observed at \SI{59.8}{\day}, as expected, when the GXe compressor was stopped for a period of approximately \SI{15}{minutes} due to a communication error with the slow control software. Again, the change in liquid level most likely resulted in the sudden release of impurities from the collar. Shortly following these events, the electron drift lifetime increased exponentially to return to the outgassing-limited values.

A review of the slow control data allowed for the identification of three irregularities in the pressure and flow conditions, at \SI{38.9}{\day}, \SI{52.6}{\day}, and \SI{80.7}{\day}. The first was a brief moment of excess flow downstream of the GXe compressor with a slight pressure increase of approximately $\SI{10}{\milli bar}$, suggesting the release of trapped gas in the xenon handling system from vibration, or the unlikely development of a micro-leak. The second irregularity was a quick fluctuation in the purification flow, both upstream and downstream of the GXe compressor, resulting in a momentary increase in pressure both downstream of the compressor and in the inner vessel, also leading to a change in the liquid level. The last irregularity was again a marginal increase in the flow downstream of the compressor. In this case, however, the event lasted approximately \SI{3}{\hour}.
After \SI{88.0}{\day} of almost continuous purification, the electron drift lifetime reached a value of $\SI{664(23)}{\micro\second}$. This is consistent with the value reached in other LXe experiments, such as XENON1T~\cite{XENON:2021nad} and LUX~\cite{LUX:2020vbj}, with \SI{660}{\micro\second} and \SI{750}{\micro\second}, respectively. 

While the purity level demonstrated in Xenoscope could be sufficient to drift electrons over \SI{2.6}{\meter} in LXe, it can be further improved by increasing the purification speed. To investigate this, a simple model of the electron drift lifetime, assuming O$_2$-like impurities, was adapted from Refs.~\cite{greeneXENON1TSpinIndependentWIMP2018,Plante:2022khm} and fitted to the electron drift lifetime data shown in figure~\ref{fig:electron_lifetime_datasets}. Two coupled differential equations describe changes in impurity levels $M \dv{I_{}}{t}$ after a time $\dd t$, where $M$ is the mass of xenon, and $I$ is the impurity concentration. The gas and liquid phases (denoted by the subscripts $g$ and $l$, respectively) are evaluated separately to determine the impurity concentration at time $t+\dd t$:
%
\begin{multline}
M_{\mathrm{g}} \dv{I_{\mathrm{g}}}{t}^{(j)} = -F_{\mathrm{g}} \rho I_{\mathrm{g}} + \left(\frac{\Lambda_{\mathrm{g},0}}{1+\frac{t-\Delta t^{(j)}_g}{T_{1/2,g}}} + C_\mathrm{g}\right)\\ + \frac{\epsilon_1 P_{\mathrm{C}} I_{\mathrm{l}}}{h} - \frac{\epsilon_2 P I_{\mathrm{g}}}{h} + M_{\mathrm{g}} \Delta I_{\mathrm{g}}^{(j)} \int \delta\left(t-t^{(j)}\right) \dd t
\label{eq:e-lifetime-gas}
\end{multline}
%
\begin{multline}
M_{\mathrm{l}} \dv{I_{\mathrm{l}}}{t}^{(j)} = -F_{\mathrm{l}} \rho I_{\mathrm{l}} + \left(\frac{\Lambda_{\mathrm{l},0}}{1+\frac{t-\Delta t^{(j)}_l}{T_{1/2,l}}} + C_\mathrm{l}\right)\\- \frac{\epsilon_1 P_{\mathrm{C}} I_{\mathrm{l}}}{h} + \frac{\epsilon_2 P I_{\mathrm{g}}}{h} + M_{\mathrm{l}} \Delta I_{\mathrm{l}}^{(j)} \int \delta\left(t-t^{(j)}\right) \dd t~.
\label{eq:e-lifetime-liquid}
\end{multline}
%
\noindent The index $j = \{0, 1,...,7\}$ indicates the regions between discontinuities, marked by the dashed lines in figure~\ref{fig:electron_lifetime_datasets}. Each equation consists of five terms. The first accounts for the purification rate, with $F$ being the purification flow, $\rho$ the density of xenon at \SI{1}{bar}, \SI{0}{\celsius}, and $I$ the concentration in impurities. The second term accounts for the time-dependent average outgassing rate from detector materials, where $\Lambda_0$ is the outgassing at time $t=0$, $T_{1/2}$ is the decay-time of the outgassing rate and $C$ is a constant outgassing term which brings the system to an equilibrium point when $t \gg T_{1/2}$. After the sudden increase of impurity concentration in the xenon during flow changes, described by the delta function in the fifth term, the outgassing rate modelled by the second terms of the differential equations is expected to revert back to a high value, as impurities can be adsorbed in outgassed materials~\cite{Plante:2022khm}. This is accounted for with the time-offset parameters $\Delta t^{(j)} = \sum_1^j \Delta t ^{(i)}$, which are cumulative since the dataset is fitted as a whole, and $\Delta t^{(0)} = 0$. Finally, the third and fourth terms describe the exchange of impurities between the gas and the liquid phases, and thus the signs are inverted between the two equations. The $\epsilon$ parameters are efficiencies of the exchange process, $P_{\mathrm{C}}$ is the cooling power of the system in the absence of purification, proportional to the evaporation rate of the xenon, $P$ is the cooling power deployed by the cryogenics, proportional to the condensation rate, and $h$ is the latent heat of xenon.

The electron drift lifetime is then calculated by solving the system of differential equations for equation~\ref{eq:attachement} and minimising with the iMinuit (Python) package~\cite{James:1975dr} simultaneously in all $j$-regions, using a least-square method. The best-fit model is displayed as a solid red curve in figure~\ref{fig:electron_lifetime_datasets}. Assuming the same initial conditions obtained from the fit of the model to the electron drift lifetime data, we obtain the purification flow-dependent electron drift lifetime predictions shown in figure~\ref{fig:prediction}.
%
\begin{figure}[h!]
\centering
\includegraphics[width=0.49\textwidth]{figures/prediction.pdf}
\caption{Purification flow-dependent electron drift lifetime prediction. Assuming the same operating conditions as in the measurement campaign, an increased flow (solid lines) could significantly reduce the purification time needed before the electron drift lifetime begins its exponential rise. Furthermore, the addition of a \SI{2}{slpm} gas phase extraction to the purification loop can also improve the electron drift lifetime (dashed lines).}
\label{fig:prediction}
\end{figure}
%

As expected, an increased purification speed would yield longer electron drift lifetimes, attained in a shorter purification time. The addition of a purification flow of the gas phase ($F_\mathrm{g}= \SI{2}{slpm}$) suggests an expected increase in electron drift lifetime of up to \SI{15}{\%}. Therefore, prior to the start of the next phase of Xenoscope, a parallel gas extraction line inspired by the gas purification system reported in Ref.~\cite{Plante:2022khm} was added to the gas handling system with a second flow controller, allowing for the purification of both the liquid and gas phases in the same purification loop. 
 
\subsection{Electron transport}

The purity monitor allows for a dedicated measurement of the arrival time of the electron cloud at the anode. The drift velocity $v_{\mathrm{d}}$ of the cloud given a drift field $E_{\mathrm{d}}$ is:
\begin{equation}
\label{velocity}
v_{\text{d}} =d_{2}/t_{2}.
\end{equation}
Considering that the accuracy of the time measurement between the extrema of the cathode and anode signals is higher, and region 3 has a fast detection, it is convenient to calculate instead:
\begin{equation}
\label{velocity}
v_{\text{d}} = (d_{2} + d_{3}) / (t_{2} + t_{3}).
\end{equation}
This approach introduces an additional term, $(d_{2}t_{3} - d_{3}t_{2})/(t_{2} (t_{2}+t_{3}))$, which induces a negligible $0.1\%$ bias in the final values given the ratios between $t_{3}/t_{2}$ and $d_{3}/d_{2}$. The drift velocity can also be expressed in terms of the drift field $E_{\mathrm{d}}$:
\begin{equation}
v_{\mathrm{d}} = \mu E_{\mathrm{d}}\:,
\end{equation}
where $\mu$ is the electron mobility. The mobility is related to the average time, for a given temperature, density, and energy of the electrons, between elastic collisions with the xenon and electronegative impurities, and potential inelastic collisions with impurities. Thus, the acquisition of waveforms used to derive the drift velocity was performed at a constant electron drift lifetime to avoid systematic uncertainties. Benchmark regimes can be used to visualise these dependencies for electron mobility, such as \enquote{cold electrons} and \enquote{hot electrons}~\cite{Schmidt:1984zz}. In the cold electrons regime, the energies of the electrons are mostly due to the thermal bath in the xenon fluid (around \SI{0.015}{eV} at \SI{177}{\kelvin}~\cite{Boyle:2016wpy}), and they rapidly acquire energy with increasing electric fields, resulting in a linear gain in velocity. For cold electrons, the mobility can be expressed as:
%
\begin{equation}
    \mu=\frac{2}{3}\left(\frac{2}{\pi\, m_e \,k_B \,T}\right)^{\frac{1}{2}} e \,\frac{\lambda}{v}\:,
    \label{eq:cold_electrons}
\end{equation}
where $k_{B}$ is the Boltzmann constant, $T$ is the temperature, $e$ is the charge of the electrons, m$_e$ is their mass, $v$ is the magnitude of the velocity in all directions, and $\lambda$ is the mean free path of the electrons in their collisions with atoms, inversely proportional to the number density and cross section. In contrast to cold electrons, hot electrons have gained most of their energy through acceleration by the drift field, and experience increased energy losses on their drift path due to collisions with xenon atoms, which results in a slower rate of change in their velocity. For this case, the electron mobility can be expressed as:
\begin{equation}
\mu = \frac{4\, e \, \lambda}{3 v  \pi^{1 / 2} m_e^*}\:,
\label{eq:hot_electrons}
\end{equation}
where $m_e^*$ is the effective mass of the electrons in the medium. While these equations are not used in this work to infer properties such as electron mobility, or electron cloud diffusion, they are useful to scrutinise the results of our measured drift velocity and discuss potential systematic effects, discussed later in this section. 

The data for this measurement was acquired at day 89, after the electron drift lifetime entered a region of slow change, and in a time interval of half an hour, to minimise systematic effects. Drift fields of \SI{25}{\volt/\cm} to \SI{75}{\volt/\cm} were scanned in steps of \SI{5}{\volt/\cm}, where the former was the threshold for a discernible signal above noise level in the cathode. The extraction field was changed to maintain the ratio between the extraction and drift field used in the electron drift lifetime data, while the collection field was fixed at \SI{500}{\volt/\centi\meter}. Figure~\ref{fig:drift_time_field} shows the drift velocity at different fields, calculated with equation~\ref{velocity}, with good agreement with previous measurements in LXe~\cite{gushchin1982electron, Albert:2016bhh,Baudis:2017xov,Thieme:2022dze,Jorg:2021hzu,Baur:2022sel}. The prediction from NEST (Noble Element Simulation Technique)~\cite{szydagis_m_2018_1314669}, based on data-driven empirical models, is also shown. The curves for cold and hot electrons are derived by using equations~\ref{eq:cold_electrons} and~\ref{eq:hot_electrons}, respectively, to fit the data from Ref.~\cite{gushchin1982electron}, for it covers both regimes with a high density of points.

\begin{figure}[h!]
\centering
\includegraphics[width=0.49\textwidth]{figures/drift_velocity_withXenoscope.pdf}
\caption{Measured drift velocity with electric field values from \SI{25}{} to \SI{75}{\volt/\cm}, in steps of \SI{5}{\volt/\cm}. The results are compared to literature values from Gushchin (\SI{165}{K})~\cite{gushchin1982electron}, EXO-200 (\SI{167}{\kelvin})~\cite{Albert:2016bhh}, Xurich II (2018, \SI{184}{\kelvin})~\cite{Baudis:2017xov}, Xurich II (2021, \SI{177}{\kelvin})~\cite{Thieme:2022dze}, HeXe (\SI{174}{K})~\cite{Jorg:2021hzu} and XeBRA (\SI{173}{\kelvin})~\cite{Baur:2022sel}. The prediction from NEST v2.3.8~\cite{szydagis_m_2018_1314669} is shown as a solid blue curve.}
\label{fig:drift_time_field}
\end{figure}

The spread of the electron cloud in time was studied by analysing the anode signal at the previously mentioned drift fields. With this purity monitor, only the longitudinal diffusion can be observed, for there is no information on the $x-y$ charge distribution. The standard deviation ($\sigma$) of the Gaussian fit of the signal in the anode is used to calculate the longitudinal diffusion coefficient. For the case of the initial and final distributions of the electron cloud following a Gaussian distribution, the width of the anode signal is related to the longitudinal diffusion~\cite{Albert:2016bhh,Li:2015rqa,Rolandi2008} as:
%
\begin{equation}
D_{L}=\frac{d^{2} \sigma_{L}^{2}}{2 t^{3}}\:,
\label{eq:diffusion1}
\end{equation}
\noindent
where $D_L$ is the diffusion coefficient of the electron population due to the random walk of the electrons in the longitudinal direction, and:
%
\begin{equation}
\label{eq:diffusion}
\sigma_{L}^{2}=\sigma^{2}-\sigma_{0}^{2}\:,
\end{equation}
\noindent
 is the width at one $\sigma$ considering diffusion effects only, which is the value to extract. The widths $\sigma_{0}$ and $\sigma$ belong to the initial signal at the cathode and the final signal measured in the anode, respectively, and $d$ is the drift length ($d_{2}+d_{3}$) at drift time $t$ ($t_{2}+t_{3}$). By rewriting equation~\ref{eq:diffusion}, we obtain the width of the anode signal:
%
\begin{equation}
    \sigma^2 = \frac{D_L 2 t^3}{d^2} + \sigma_0^2\:,
\label{eq:diffusion2}
\end{equation}
The diffusion of the electron cloud is related to the previously introduced electron mobility. At low drift fields, it follows the Einstein-Smoluchowski relation~\cite{einstein,smoluchowski,Schmidt:1984zz}:
\begin{equation}
D_L=\frac{k_{B} T}{e} \mu = \frac{\epsilon_{T}}{e} \mu\:,
\label{eq:diffusion_einstein}
\end{equation}
with thermal energy $\epsilon_{T}$.
The diffusion is thus affected by the cross section of elastic and inelastic interactions with the medium species (xenon and impurities), analogously to the drift velocity. 

 Since the electrons have energies above the thermal bath and are not in equilibrium, a characteristic energy, $\epsilon_k$, is defined as:
\begin{equation}
\epsilon_k = \frac{e D_L}{\mu}\:,
\label{eq:characteristic_energy}
\end{equation}
representing the energy associated with the longitudinal diffusion, where now $D_L$ has a contribution beyond the thermal energy of electrons:
\begin{equation}
D_L= \frac{(\epsilon_T + \epsilon )}{e}\mu\:,
\label{eq:characteristic_energy_longitudinal_diffusion}
\end{equation}
with $\epsilon =  \epsilon_k - \epsilon_{T}$.

In previous studies, it was reported that impurities diffused in the medium, from water vapour to organic molecules, can provide a more effective energy loss mechanism for electrons, with the consequence of higher mobility and decreased diffusion~\cite{Yoshino:1976zz,Pack:1961jag} (see equations~\ref{eq:cold_electrons}  and~\ref{eq:hot_electrons}). This effect could explain the mechanism behind the higher drift velocities in the early measurements of Guschin et. al. at \SI{164}{\kelvin}~\cite{gushchin1982electron}, where no information about xenon purification methods was given.

Additional effects can play a role in the detected spread of the charge distribution in our detector and must be corrected to obtain a longitudinal diffusion coefficient that is independent of energy, electron source or detector response. From the original number of extracted photoelectrons to the measured signal in the anode, the following effects can impact the width:
\begin{itemize}
    \item Duration of the pulse of the lamp, which introduces an initial signal width at one sigma of $\SI{2.4(2)}{\micro\second}$. In this work, the initial signal width was derived from the data acquired in vacuum and in LXe, and the signal in the cathode is deconvolved with the detector and electronics responses. 
    \item The detection response of the screening region~\cite{Shockley:1938itm}. The weighting potential between the solid electrodes and the hexagonal meshes is taken into account in the measured signal to yield the original electron cloud spread in the $z$-direction. The effect of the hexagonal screening meshes was simulated by modelling the 3D geometry of the meshes and detector and performing electrostatic simulations with COMSOL. The method to derive the weighting potential is adopted from Ref.~\cite{gook_application_2012}. The weighting potential is obtained by averaging the potential over different electron drift paths to smooth out local effects. 
    \item Coulomb repulsion between electrons, where each electron is affected by the electric field induced by other electrons, can increase the size of the electron cloud. The Coulomb repulsion calculation follows the empirical approach in Ref.~\cite{Njoya:2019ldm}, which considers the ellipsoid explosion model from Ref.~\cite{PhysRevE.84.056404}, where a set of differential equations is solved to obtain the final width of the signal after the drift. The repulsive forces are stronger immediately after the charge creation and become smaller as the electron cloud spreads along the drift path. After the electrons have drifted and reached the anode grid, an additional width value of 5\% compared to the no repulsion forces case is inferred from the empirical approach, and taken as an uncertainty in the charge distribution after their extraction from the photocathode.
    \item Electron attachment to electronegative impurities which can potentially change the distribution of the electron cloud. This could in principle affect the diffusion of electrons in LXe, and the method to estimate this impact is taken from Ref.~\cite{Li:2015rqa}. The longitudinal diffusion coefficient and drift velocity formulae are expanded to include higher-order terms containing the attachment rate. The effects of electron attachment in the diffusion can be neglected according to $\frac{D_L}{(v_\mathrm{d} \cdot d)} \ll 1$, which is the case for this study.
    \item Readout response times of the pre-amplifiers. The circuit response was estimated when benchmarking the electronics and has a negligible effect on the signal shape.
\end{itemize}

Combining all the effects introduced above, a response function is obtained to deconvolve the observed signal. Table~\ref{systematics} compiles the systematic effects treatment for the calculation of the longitudinal diffusion coefficient. The results of this deconvolution, for each measurement at different drift fields, are shown in figure~\ref{fig:longitudinal_diffusion}, together with literature values~\cite{Njoya:2019ldm,Hogenbirk2018}. The longitudinal diffusion coefficient was measured at relatively low drift fields (i.e.~$<\SI{100}{\volt/\cm}$). By using the values derived for the mobility for cold electrons and hot electrons included in  figure~\ref{fig:drift_time_field}, of $\SI{0.29}{\milli\meter^{2}/(\micro \second \cdot \volt)}$ and $\SI{0.01}{\milli\meter^{2}/(\micro \second \cdot \volt)}$, respectively, together with the thermal energy of electrons, reference coefficients for the diffusion can be obtained from equation~\ref{eq:characteristic_energy}. These are included in figure~\ref{fig:longitudinal_diffusion}. By comparing the experimental values with the benchmark equations, the difference between these is contained in terms of the characteristic energy of electrons and their mobility, given equation~\ref{eq:characteristic_energy_longitudinal_diffusion}.

\begin{table}
\centering
\caption{Systematic effects and impact on the uncertainty for the inferred longitudinal diffusion coefficient $D_L$.}
\label{systematics}
% @{\extracolsep{\fill}}
\begin{tabularx}{\columnwidth}{lX}
\hline\noalign{\smallskip}
Systematic effect & Treatment and uncertainty \\
\noalign{\smallskip}\hline\noalign{\smallskip}
\bf{Measured}        & \\
Anode signal width, $\sigma$        & Gaussian plus sine fit, $2-3\%$.  \\
Drift time,   $t_2 + t_{3}$     &  Time interval between extrema of the cathode and anode signal fits. \\
Initial signal width       & Introduced by the lamp pulse. Measurements in vacuum and in LXe,  $\SI{2.4(2)}{\micro\second}$.    \\
Electronics  & RC time constant calculation from a square pulse,  $\SI{0.2}{\micro\second}$.    \\
Drift length, $d_{2} + d_{3}$ & Drift distance of the electron cloud, taken as $\SI{513(7)}{\milli\meter}$ when accounting for the potential contraction of the components at \SI{177}{\kelvin} with an assumed 1\% thermal contraction, and the position of the centre of the cloud distribution in drift regions with respect the extrema of the signals.\\  
Filtering and processing & Maximum 4\% of anode signal width.\\
\noalign{\smallskip}\hline\noalign{\smallskip}
\bf{Simulated}        & \\
Detector response   & COMSOL 3D model of the detector to derive the weighting potential, 10\% uncertainty in the response.\\
\noalign{\smallskip}\hline\noalign{\smallskip}
\bf{Assumed}        & \\
Coulomb repulsion   & Calculated with empirical model from~\cite{Njoya:2019ldm}, assumption of additional 5\% uncertainty in the initial signal spread.\\
Electron attachment   &  Neglected, 4th-order correction: $\frac{D_L}{(v_d \cdot d)} \ll 1$. \\
\hline
\end{tabularx}
\end{table}

From the listed sources of uncertainty for the diffusion coefficient $D_L$, the largest impact originates (in percentages of the total uncertainty from \SI{25}{} to \SI{75}{V/cm}) from the final width measured at the anode (95\% to 80\%), the initial signal width (20\% to 50\%), the Coulomb repulsion (1\% to 5\%), the uncertainty in the drift distance (20 to 5\%) and the drift time (4 to 1\%). 

\begin{figure}[h!]
\centering
\includegraphics[width=0.5\textwidth]{figures/longitudinal_diff.pdf}
\caption[Longitudinal diffusion calculated in this work]{Longitudinal diffusion coefficient calculated in this work, with an electron drift lifetime of $\tau = \SI{649(23)}{\micro\second}$ compared to the results from a purity monitor by Njoya et. al. ($\tau \sim$ $\SIrange{1}{35}{\micro s}$)~\cite{Njoya:2019ldm}, and a TPC from Hogenbirk ($\tau \sim $ $\SI{430}{\micro s}$)~\cite{Hogenbirk2018} and NEST~\cite{szydagis_m_2018_1314669}. The model by NEST version 2.3.7 (solid grey curve) predicted diffusion values approaching zero for lower drift fields. In version 2.3.8, a fix for this behaviour was introduced, as shown in the solid blue curve.}
\label{fig:longitudinal_diffusion}
\end{figure}

NEST version 2.3.7 implemented an empirical model based on previous measurements for the longitudinal diffusion coefficient prediction. At the time of this analysis, NEST lacked data at low drift fields (below $\sim \SI{100}{V/cm}$), and the model predicted a considerably lower longitudinal diffusion coefficient with lower drift fields, in conflict  with the measured values in this and other works. An alternative model is included in NEST, based on differential cross sections derived from Dirac-Fock solutions, combined with Maxwell-Boltzmann distributions~\cite{Boyle:2016wpy}, which do not predict the existing experimental values at all drift fields. The subsequent NEST version 2.3.8 includes a correction on the diffusion modelling, also shown in figure~\ref{fig:longitudinal_diffusion}, resulting from an exchange with the developers. Our analysis aimed to not only infer the values of longitudinal diffusion coefficient at low drift fields for liquid xenon, but also to understand their origin, related to the temperature and purity of the xenon. 

\section{Conclusions and outlook}
\label{sec:conclusions}

The construction of a next-generation liquid xenon detector at the \SI{50}{t} scale and beyond will face several technological challenges. To address some of these, the Xenoscope facility was designed and built to house a \SI{2.6}{m} tall two-phase TPC at the University of Zurich, with a total LXe mass of $\sim$~\SI{400}{\kg}. After a commissioning run described in Ref.~\cite{Baudis:2021ipf}, we presented here the first results from a run with a \SI{53}{cm} tall purity monitor.

The electron drift lifetime was monitored for 88 days with varying xenon recirculation speeds. For a speed of \SI{40}{slpm}, the highest achieved lifetime was \SI{664(23)}{\micro \s}. A parametric model of the effect of the purification rate, the time-dependent outgassing rate, the liquid-gas impurity diffusion, and the injection of impurities due to operational changes, was fitted to the data. The resulting model was used to predict the electron drift lifetime evolution for different purification conditions and, therefore, inform future design and operation choices.

The electron drift velocity and the longitudinal diffusion coefficient of the electron cloud in liquid xenon were calculated based on data acquired at drift fields between \SI{25}{} and \SI{75}{V/cm}. With the increasing size of LXe TPCs, diffusion strongly affects the position reconstruction of events and the ability to discriminate between single and multiple interactions. Thus, accurate measurements of drift and diffusion properties, combined with an improved understanding of the systematic effect of impurity concentrations on these properties on large scales, are crucial. Our results are in agreement with previous studies both for drift velocity~\cite{gushchin1982electron,Albert:2016bhh,Baudis:2017xov,Thieme:2022dze} and longitudinal diffusion~\cite{Njoya:2019ldm, Hogenbirk2018}. They also triggered an update of NEST~\cite{szydagis_m_2018_1314669}, a simulation package largely used in the community, regarding the modelling of longitudinal diffusion of electron clouds in LXe.

For the next stage of the Xenoscope project, a two-phase xenon TPC was recently built and installed, and will be operated to observe electron drift over distances up to \SI{2.6}{\metre}. The upgrade includes liquid-level control and monitoring, high-voltage supply up to \SI{50}{kV} via a ceramic feedthrough, and an array of silicon photomultipliers for light-readout in the gas phase located just above the gas/liquid interface~\cite{sipm_proceedings}. The latter replaces the charge readout used at the anode of the purity monitor, detecting instead the proportional scintillation produced in the xenon gas region of the TPC. The TPC equipped with the SiPM array will be used to study electron cloud diffusion in both longitudinal and transverse directions. Transverse diffusion is another critical parameter for more accurate modelling of electron transport in xenon-based detectors, for which measurements in the literature are scarce~\cite{Albert:2016bhh, Aprile:2009dv}. Another goal of the upgrade is to study optical properties of liquid xenon at large scales, as well as new types of photosensors under the operating conditions of DARWIN. In addition, the facility will be available to the collaboration for various R\&D projects related to the realisation of a large-scale xenon TPC.




\begin{acknowledgements}
This work was supported by the European Research Council (ERC) under the European Union's Horizon 2020 research and innovation programme, grant agreement No. 742789 ({\sl Xenoscope}), by the SNF grant 20FL20-201437, as well as by the European Union’s Horizon 2020 research and innovation programme under the Marie Skłodowska -Curie grant agreement No 860881-HIDDeN. We thank the electronics and mechanical workshops in the UZH Physics Department for their continuous support. We thank Laura Manenti for insightful discussions about purity monitors.
\end{acknowledgements}

\appendix
\section{Additional Experimental Results}

\subsection{Implementation Details of Baselines}
We compare three baselines in our paper. For Tune-A-Video, to ensure a fair comparison, we use the pre-trained weight of Stable Diffusion-v1.5\footnote{https://github.com/CompVis/stable-diffusion} (same as our model) to initialize the UNet and we fine-tune the model with an image resolution of $256 \times 256$ on the training sets of Something Something-V2 (SSv2) and Bridgedata for 200k training steps. For MCVD, we train the model with an image resolution of $256 \times 256$ for 300k training steps. For TATS, we fine-tune the pre-trained UCF-101 model with an image resolution of $128 \times 128$ on the training sets of SSv2 and Bridgedata for 300k training steps.



\subsection{Evaluation Details and Results of UCF-101}~\label{appendix:sec:ucf101}
Most prior text-conditioned video generation methods~\cite{hong2023cogvideo,vdm,makeavideo,magicvideo} evaluate their performance on the UCF-101~\cite{ucf101} benchmark. However, since our proposed method, Seer, is designed for text-conditioned video prediction (TVP) on task-level video datasets, the UCF-101 benchmark, which evaluates class-conditioned video prediction on random short-horizon video clips, is not an ideal evaluation benchmark for TVP. Nonetheless, in order to fairly compare these baselines, we still evaluate the class-conditioned video prediction performance of Seer on UCF-101. 

\paragraph{Settings}Specifically, we fine-tune our model with a video resolution of $16\times256\times256$ on UCF-101. Following the evaluation protocols of ~\cite{hong2023cogvideo}, Seer predicts the videos conditioned on 5 reference frames during fine-tuning and inference stage. We report FVD and Inception score (IS) metrics on the UCF-101 dataset~\cite{ucf101}. The IS is calculated by a C3D model\cite{c3d} that is pre-trained on the Sports-1M dataset~\cite{sports} and fine-tuned on UCF101. We follow the evaluation code of TGAN-v2~\cite{tganv2} to calculate IS metric. Following ~\cite{hong2023cogvideo,vdm,makeavideo}, we evaluate the FVD metric with 2,048 samples and IS metric with 100k samples in the validation set of UCF-101.

\paragraph{Results} Table~\ref{table:tvp:ucf} presents the class-conditioned video prediction results on UCF-101, demonstrating that Seer outperforms CogVideo~\cite{hong2023cogvideo} and MagicVideo~\cite{magicvideo}, but falls short of Make-A-Video~\cite{makeavideo}. Make-A-Video employs unlabelled video pre-training on temporal layers and achieves the best performance among all other methods. While Make-A-Video shows superior performance on FVD and IS, Seer has the potential to further improve its generation performance by addressing the following two limitations. First, Seer has not been pre-trained on video data. Second, Seer obtains latent vectors via a pre-trained 2D VAE, which has not been fine-tuned on UCF-101 and limits the video generation quality of Seer (with 259.4 FVD and 68.16 IS reconstruction quality). However, as we focus on the text-conditioned video prediction task, addressing the above limitations on UCF-101 is out of the scope of this paper.

\begin{table*}
\begin{threeparttable}
\centering\small
\tablestyle{2pt}{1.1}
\setlength{\tabcolsep}{5pt}
\caption{\textbf{ Class-conditioned video prediction performance on UCF-101} we evaluate the Seer on the UCF-101 with 16-frames-long videos. Ex.data indicates that the model has been pre-trained or fine-tuned on extra datasets.
}
\label{table:tvp:ucf}
\begin{tabular}{cccc|cc}
\specialrule{.1em}{.05em}{.05em} 
 Method & Ex.data & Cond. & Resolution & FVD$\downarrow$ & IS$\uparrow$\\
 \hline
MoCoGAN-HD~\cite{mocogan} & No & Class.  & $256\times 256$ & 700\tiny{$\pm$24} & 33.95\tiny{$\pm$0.25}\\
VideoGPT~\cite{videogpt} & No & No  & $128\times 128$ & - & 24.69\tiny{$\pm$0.30}\\
RaMViD~\cite{ramvid} & No & No  & $128\times 128$ & - & 21.71\tiny{$\pm$0.21}\\
StyleGAN-V~\cite{styleganv} & No & No  & $128\times 128$ & - & 23.94\tiny{$\pm$0.73}\\
DIGAN~\cite{digan} & No & No  & & 577\tiny{$\pm$22} & 32.70\tiny{$\pm$0.35}\\
TGANv2~\cite{tganv2} & No & Class.  & $128\times 128$ & 1431.0 & 26.60\tiny{$\pm$0.47}\\
VDM~\cite{vdm} & No & No  & $64\times 64$ & - & 57.80\tiny{$\pm$1.3}\\
TATS-base~\cite{tats} & No & Class.  & $128\times 128$ & 278\tiny{$\pm$11} & 79.28\tiny{$\pm$0.38}\\
MCVD~\cite{mcvd} & No & No  & $64\times 64$ & 1143.0 & -\\
LVDM~\cite{lvdm} & No & No  & $256\times 256$ & 372\tiny{$\pm$11} & 27\tiny{$\pm$1}\\
MAGVIT-B~\cite{magvit} & No & Class.  & $128\times 128$ & 159\tiny{$\pm$2} & 83.55\tiny{$\pm$0.14}\\
 \hline
CogVideo~\cite{hong2023cogvideo} & txt-img \& txt-video & Class.  & $160\times 160$ & 626 & 50.46\\
Make-A-Video~\cite{makeavideo} & txt-img \& video & Class.  & $256\times 256$ & 81.25 & 82.55\\
MagicVideo~\cite{magicvideo} & txt-img \& txt-video & Class.  & & 699 & -\\
\textbf{Seer(Ours)} & txt-img & Class. & $256\times 256$ & 287.8 & 57.74\\
\specialrule{.1em}{.05em}{.05em} 
\textbf{pre-trained VAE}\tnote{*} & - & - & $256\times 256$ & 259.4 & 68.16\\
\specialrule{.1em}{.05em}{.05em} 
\end{tabular}
\begin{tablenotes}\footnotesize
    \item [*] we evaluate the reconstruction quality of pre-trained 2D VAE in this table, the pre-trained 2D VAE is initialized with the pre-trained weight from Stable Diffusion-v1.5 without extra fine-tuning.
\end{tablenotes}
\end{threeparttable}
\end{table*}



\subsection{Evaluation Results of Sampling Steps}
To further investigate the generation effects of sampling steps during evaluation, we conduct a comparison between Seer and Tune-A-Video. We apply a series of DDIM sampling steps (10, 20, 30, 40, 50 DDIM steps), as shown in Figure~\ref{fig:ddimstep}. Seer consistently outperformed Tune-A-Video in terms of both FVD and KVD, with improvements observed from 20 DDIM steps to 50 DDIM steps. Particularly noteworthy is Seer's advantage in video quality (280.7 FVD and 0.73 KVD) compared to Tune-A-Video (419.3 FVD and 1.5 KVD) when using only 10 DDIM steps, demonstrating Seer's ability to sample high-fidelity videos efficiently with minimal denoising steps.
\begin{figure}
\centering
\includegraphics[width=1.0\linewidth]{fig_appendix/ddim_curve.pdf}
\caption{Evaluation results of Seer and Tune-A-Video with DDIM sampling steps ranging from 10 to 50 on the Something-Something V2 dataset.}
\vspace{-8pt}
\label{fig:ddimstep}
\end{figure}



\subsection{Additional Ablation Results}~\label{appendix:sec:fstext}
\paragraph{FSText layer depth}In this section, we additionally investigate the impact of FSText Decomposer's layer depth in Table~\ref{table:ablation:layer}. Our default setting (8-layer FSText Decomposer) outperforms shallower models (2-layer and 4-layer) in terms of FVD. Though the 4-layer model shows a marginal advantage over the 8-layer model in terms of KVD, our experiments indicate that the 8-layer FSText Decomposer shows a remarkable advantage on FVD metrics and exhibits robustness in text-video alignment. Therefore, we adopt the 8-layer FSText Decomposer as the default setting for Seer.

\begin{table}
\centering\small
\tablestyle{2pt}{1.0}
\setlength{\tabcolsep}{5pt}
\caption{\textbf{Layer depth in FSText Decomposer}.}
\label{table:ablation:layer}
\vspace*{-3mm}
\begin{tabular}{c|cc}
num. layers.& FVD$\downarrow$ & KVD$\downarrow$\\
 \hline
 2 & 238.6 & 0.51\\
 4 & 229.7 & 0.23\\
8(Ours) &200.1 & 0.30\\
\end{tabular}
\end{table}

\paragraph{Qualitative results of fine-tuning ablation} We conduct a qualitative analysis of various fine-tune settings. We provide additional visualizations of Fine-tune Setting ablation in Section 5.5 of the main paper. Figure~\ref{fig:ablate:finetune} shows the results of different settings. Among these settings, our default setting \textit{``temp+FSText"} stands out as it preserves a higher-level temporal consistency in video prediction starting from reference frames and also delivers superior text-based video motion compared to the other fine-tune settings. 
\begin{figure}
\centering
\includegraphics[width=1.0\linewidth]{fig_appendix/ablation_visualization.pdf}
\caption{Additional qualitative results of fine-tuning ablation. \textit{“temp+FSText.”} is our default setting.}
\vspace{-8pt}
\label{fig:ablate:finetune}
\end{figure}


\begin{table}
\centering\small
\tablestyle{2pt}{1.1}
\caption{Hyperparameters and details of Fine Tuning/Inference}
\label{table:hyperparam:finetune}
\begin{tabular}{c|cc}
\hline
param. & value\\
\hline
optim. & AdamW\\
Adam-$\beta_1$ &  0.9\\
Adam-$\beta_2$ &  0.99\\
Adam-$\epsilon$ &  $1e^{-8}$\\
weight decay &  $1e^{-2}$\\
lr &  $1.28e^{-5}$\\
end lr & 0.0\\
lr sche. & cosine\\
noise sche. & cosine\\
train batch size& 1/GPU\\
grad. acc.& 2\\
warmup steps& 10k\\
resolution& $256 \times 256$\\
train. steps & 200k\\
train. hardware & 4 RTX 3090\\
val. batch size& 2/GPU\\
sampler& DDIM\\
sampling steps & 30\\
guidance scale & 7.5\\
\hline
\end{tabular}
\end{table}

\begin{table}
\centering\small
\tablestyle{2pt}{1.1}
\caption{Hyperparameters of 3D U-Net}
\label{table:3dunet}
\begin{tabular}{c|cc}
\hline
hyperparam. & value\\
\hline
input/output channels &  4\\
Base channels & 320\\
Channel multipliers&  1,2,4,4\\
3D Downsample blocks &  4\\
3D Upsample blocks &  4\\
Number of layers (per block) &  2\\
\hline
Modules of layer & 3D ResnetBlock\\
 & Spatial-cross Atten.\\
 & ATS Atten.\\
 & Down./Up. 3D ResnetBlock\\
\hline
Dimension of atten. heads &  8\\
activation function &  SiLU\\
Dimension of cross-atten. &  768\\
\hline
\end{tabular}
\end{table}

\begin{table}
\centering\small
\tablestyle{2pt}{1.1}
\caption{Hyperparameters of FSText Decomposer}
\label{table:hyperparam:fstext}
\begin{tabular}{c|cc}
\hline
hyperparam. & value\\
\hline
learnable tokens channels &  768\\
output channels &  768\\
Base channels & 768\\
Number of layers &  8\\
\hline
Modules of layer & Seq-cross Atten.\\
 & Feedforward\\
 & Directed temporal Atten.\\
 & Feedforward\\
\hline
Number of atten. heads &  8\\
Dimension of cross-atten. &  768\\
\hline
\end{tabular}
\end{table}




\section{Implementation Details}~\label{appendix:sec:impl}

\subsection{Fine-tuning and Sampling}\label{sec:finetuneparam}
 In this section, we list the hyperparameters, fine-tuning details, sampling details, and hardware information of our model in Table~\ref{table:hyperparam:finetune}.
 
\subsection{Architecture information}\label{sec:arch}
In this section, we list the hyperparameters of 3D U-Net in Table~\ref{table:3dunet} and hyperparameters of FSText Decomposer in Table~\ref{table:hyperparam:fstext}.



\section{Visualization} 

\subsection{Additional qualitative results} 
We provide additional visualization on Something-Something v2 (SSv2) of our text-conditioned video prediction in Figure~\ref{fig:ssv2pred}, and text-conditioned video prediction/manipulation results in Figure~\ref{fig:ssv2mani}. Additionally, we provide the visualization on BridgeData of text-conditioned video prediction in Figure~\ref{fig:bridgepred} and text-conditioned video prediction/manipulation in Figure~\ref{fig:bridgemani}.
\section{Human Evaluation Details}~\label{appendix:sec:humaneval} 
To evaluate the quality of video predictions according to human preferences, we conducted a human evaluation with 99 video clips on the validation set of the Something-Something V2 dataset (SSv2), the evaluation process involved 54 anonymous evaluators. To eliminate biases towards specific baselines, we randomly selected 20 questions for each evaluator. Each single-choice question consisted of a ground-truth video as a reference, a manually modified text instruction, and two video prediction results generated by Seer and another baseline method. The evaluators were required to choose the video clip that is more consistent with the text instruction and has higher fidelity from the two options.
To ensure the clarity of the questions, we provided an example to explain the options in each questionnaire. Moreover, we recommended that evaluators prioritize video predictions with strong text-based motions as their first preference and the fidelity of the generated video as their second preference. For reference, Figure~\ref{fig:humanevalexp} provides a screenshot of an example questionnaire.

In total, we collected 342 responses for the Seer vs. TATS comparison, 363 responses for the Seer vs. Tune-A-Video comparison, and 357 responses for the Seer vs. MCVD comparison. And the results in the main paper Figure 7 are calculated based on the collected questionnaires.
\clearpage
\begin{figure}
\centering
\includegraphics[width=1.0\linewidth]{fig_appendix/sth_predict.pdf}
\caption{Text-conditioned video prediction of Seer on SSv2.}
\vspace{-8pt}
\label{fig:ssv2pred}
\end{figure}
\begin{figure}
\centering
\includegraphics[width=1.0\linewidth]{fig_appendix/sth_manipulate.pdf}
\caption{Text-conditioned video prediction/manipulation of Seer on SSv2, where ``pred." refers to prediction, ``mani." refers to manipulation.}
\vspace{-8pt}
\label{fig:ssv2mani}
\end{figure}
\begin{figure*}
\centering
\includegraphics[width=0.9\linewidth]{fig_appendix/bridge_pred.pdf}
\caption{Text-conditioned video prediction of Seer on BridgeData.}
\vspace{-8pt}
\label{fig:bridgepred}
\end{figure*}
\begin{figure*}
\centering
\includegraphics[width=1.0\linewidth]{fig_appendix/bridge_manipulate.pdf}
\caption{Text-conditioned video prediction/manipulation of Seer on BridgeData, where ``pred." refers to prediction, ``mani." refers to manipulation.}
\vspace{-8pt}
\label{fig:bridgemani}
\end{figure*}


\begin{figure}
\centering
\includegraphics[width=1.0\linewidth]{fig_appendix/screenshot.PNG}
\caption{Screenshot of a questionnaire example shown to human evaluators.}
\vspace{-8pt}
\label{fig:humanevalexp}
\end{figure}

%\bibliographystyle{JHEP}
%\bibliography{demo-firstresults}  
\pdfoutput=1 
\RequirePackage{fix-cm}
\documentclass[twocolumn,epjc3]{svjour3}  
\RequirePackage{graphicx}
\usepackage{tikz}
\RequirePackage{mathptmx}
\usepackage[english]{babel}
\usepackage{amsmath}
\usepackage{mathtools}
\usepackage[separate-uncertainty=true]{siunitx}
\DeclareSIUnit\bar{bar}
\sisetup{range-phrase = -, range-units=single}
\usepackage{comment}
\usepackage{tabularx}
\usepackage[version=4]{mhchem}
\usepackage[font={small},labelfont=bf,labelsep=quad]{caption}
\usepackage[font={normal}]{subcaption}
\usepackage{cite}
\usepackage{wasysym}
\usepackage{textcomp}
\usepackage{physics}
\usepackage{csquotes}
\RequirePackage[colorlinks,citecolor=blue,urlcolor=blue,linkcolor=blue]{hyperref}

\defineshorthand{"~}{\babelhyphen{nobreak}}
\useshorthands{"}
\newcommand{\LN}{LN$_2$}
\DeclareSIUnit[number-unit-product = {}]{\inch}{"}
%\sisetup{}
\sisetup{per-mode = power}
\sisetup{exponent-product = \cdot}
\sisetup{group-separator = {\,}}
\sisetup{group-minimum-digits = 3}
\DeclareSIUnit\litre{L}
\DeclareSIUnit\liter{L}

% \usepackage[mathlines]{lineno}
% \let\oldequation\equation
% \let\oldendequation\endequation

% \renewenvironment{equation}
%   {\linenomathNonumbers\oldequation}
%   {\oldendequation\endlinenomath}
% \linenumbers

\journalname{Eur. Phys. J. C}
\begin{document}
\sloppy

\title{Electron transport measurements in liquid xenon with Xenoscope, a large-scale DARWIN demonstrator}

\author{L.~Baudis
        \and
        Y.~Biondi\thanksref{e1}\thanksref{e4}
        \and
        A.~Bismark
        \and
        A.~P.~Cimental Ch\'avez
        \and
        J.~J.~Cuenca-Garc\'ia
        \and
        J.~Franchi
        \and
        M.~Galloway
        \and
        F.~Girard\thanksref{e2}
       \and
       R.~Peres\thanksref{e3}
       \and
       D.~Ram\'irez~Garc\'ia
       \and
       P.~Sanchez-Lucas\thanksref{e5}
       \and
       K.~Thieme\thanksref{e6}
       \and
       C.~Wittweg
}
\thankstext{e1}{e-mail: \url{yanina.biondi@physik.uzh.ch}}
\thankstext{e2}{e-mail: \url{frederic.girard@physik.uzh.ch}}
\thankstext{e3}{e-mail: \url{ricardo.peres@physik.uzh.ch}} 
\thankstext{e4}{Now at Karlsruhe Institute of Technology}
\thankstext{e5}{Now at University of Granada}
\thankstext{e6}{Now at University of Hawai\textquoteleft{i} at M\={a}noa}
\institute{\normalsize{Department of Physics, University of Zurich, Winterthurerstrasse 190, 8057 Zurich, Switzerland}
\label{addr1}}

\date{Received: date / Accepted: date}

\maketitle

\begin{abstract}
There is a compelling physics case for a large, xenon-based underground detector devoted to dark matter and other rare-event searches. A two-phase time projection chamber as inner detector allows for a good energy resolution, a three-dimensional position determination of the interaction site and particle discrimination. To study challenges related to the construction and operation of a \mbox{multi-tonne} scale detector, we have designed and constructed a vertical, full-scale demonstrator for the DARWIN experiment at the University of Zurich. Here we present first results from a several-months run with {\SI{343}{kg}} of xenon and electron drift lifetime and transport measurements with a \SI{53}{cm} tall purity monitor immersed in the cryogenic liquid. After \SI{88}{\day} of continuous purification, the electron lifetime reached a value of $\SI{664(23)}{\micro\second}$. We measured the drift velocity of electrons for electric fields in the range (25--75)\,V/cm, and found values consistent with previous measurements. We also calculated the longitudinal diffusion constant of the electron cloud in the same field range, and compared with previous data, as well as with predictions from an empirical model.
\end{abstract}

\section{introduction}

% 1. importance of TKGs and reasoning on TKGs. 
% 2. low resource languages, main main idea.
% 3. relations and limitations of current works.
% 4. summarize our solutions and contributions.

Temporal Knowledge Graphs (TKGs)~\cite{YAGO,ICEWS18,WIKI,acekg} characterize temporally evolving events, where each event, represented as ({\em subject}, {\em relation}, {\em object}), is associated with temporal information ({\em time}), e.g., ({\em Macron}, {\em reelected}, {\em French president}, {\em 2022}). TKGs has facilitated various knowledge-intensive Web applications with timeliness, such as question answering~\cite{KBQA}, product recommendation~\cite{RippleNet,TKG4Rec,TKG4Rec2,RETE}, and social event forecasting~\cite{KG4Social,DyDiff-VAE,andgan,belief,misinfo,polarization}. 

As new events are continually emerging, modern TKGs are still far from being complete. Conventionally, the TKG construction process relies primarily on information extraction from unstructured corpus~\cite{WIKI,YAGO, EventKG}, which necessitates extensive manual annotations to keep up with changing events. For instance, the recent transition from Trump to Biden as the President of the United States has not been reflected in many TKGs, highlighting the need for timely updates. This spurs research on temporal knowledge graph reasoning to automate evolving events prediction over time~\cite{TA-DistMult,Know-Evolve,Renet,RE-GCN}. Unfortunately, the problem of TKG incompleteness is particularly pronounced in low-resource languages, where it is unable to collect enough corpus and annotations to support robust TKG construction. This results in suboptimal reasoning performance and distinctly unsatisfying accuracy in predicting recent and future events.

% whose performance can degrade significantly in low-resource language TKGs that suffer from severe incompleteness over time. 
% \jingfeng{why don't people  study cross-lingual TKG previously, (i.e. use language alignment to improve TKG). Is it really helpful intuitively to use high resource language to help TKGC? For instance, is it enough to use static langauge-alignment to help KGC, ignoring the temporal information? Are those langauge-alignment changing across time?}



\begin{figure}
    \centering
    \includegraphics[width = 1.0\linewidth]{fig/task.pdf}
    \caption{An illustrative example of cross-lingual reasoning on TKGs. 1) We aim to transfer knowledge from English TKG to Japanese TKG, where the English version provides more complete information; 2) Cross-lingual alignments only cover a small ratio of entities, e.g., Apple Inc; 3) Cross-lingual alignments can be noisy and misleading, e.g., A city called Ventura is linked to new macOS Ventura at $t_2$, introducing noise for reasoning in Japanese.}
    \label{fig:illustration}
    %\vspace{-6mm}
\end{figure}

Inspired by the incompleteness issue facing low-resource languages in constructing TKGs, we introduce a novel task named Cross-Lingual Temporal Knowledge Graph Reasoning (as shown in Figure~\ref{fig:illustration}). This task aims to alleviate the reliance on supervision for TKGs in low-resource languages (referred to as the target language) by transferring temporal knowledge from high-resource languages (referred to as the source language)~\footnote{In this paper, for the sake of brevity, we interchangeably use the terms high-resource/low-resource and source/target.}. In contrast, all the existing efforts are either limited to reasoning in monolingual TKGs (usually high-resource languages, e.g., English)~\cite{TA-DistMult,Know-Evolve,Renet,RE-GCN}, or multilingual static KGs~\cite{KEnS,AlignKGC,SS-AGA}. To the best of our knowledge, cross-lingual TKG reasoning that transfers temporal knowledge between TKGs has not been investigated. 

%Motivated by this, we study a new task named {\em cross-lingual temporal knowledge graph reasoning} as shown in Figure~\ref{fig:illustration}, to alleviate the heavy dependence on supervision for any resource-poor language TKGs by distilling the temporal knowledge from resource-rich ones. Differently, all the existing efforts are either limited to reasoning in monolingual (usually high-resource languages, e.g., English) temporal KGs~\cite{TA-DistMult,Know-Evolve,Renet,RE-GCN}, or multilingual static KG~\cite{KEnS,AlignKGC,SS-AGA}, but neglecting the reasoning in a both temporal and cross-lingual manner that highly requires capturing time-evolving patterns and language discrepancy. To the best of our knowledge, this problem, regarding how to transfer cross-lingual knowledge between TKGs, has still not been formally investigated. 

% Unlike conventional TKG reasoning, 
The fulfillment of this task poses tremendous challenges in two aspects: 1) \textbf{Scarcity of cross-lingual alignment}: as the informative bridge of two separate TKGs, cross-lingual alignment is imperative for cross-lingual knowledge transfer~\cite{AlignKGC,KEnS,SS-AGA}. However, obtaining alignments between languages is a time-consuming and resource-intensive process that heavily relies on human annotations. The transfer of knowledge through a limited number of alignments is often insufficient to fully enhance the TKG in the target language. 2) \textbf{Temporal knowledge discrepancy}: the information associated with two aligned entities is not necessarily identical, especially with regards to temporal patterns. Utilizing a rough approach to equate the aligned entities at all times can result in the transfer of misleading knowledge and negatively impact performance. This becomes more pronounced when the alignments are noisy and unreliable. For example, at the time step $t_2$, a new event about operating system ``{\it Ventura}'' from Apple company occurs in the source English TKG, and meanwhile there is a noisy aligned entity ``{\it Ventura city}'' in the target Japanese TKG. Directly pulling those two entities at this point, can inevitably introduce  noise and fail to predict a set of related events in the target TKG. Therefore, it is crucial to dynamically regulate the alignment strength of each local graph structure over time in order to maximize the effectiveness of cross-lingual knowledge distillation.

% Pulling those entities together cannot augment information in target languages. Small alignment strength is beneficial in the unreliable alignment cases, otherwise the misleading knowledge transferring can even hurt the performance.

% Moreover, in a case that the alignments are not fully reliable, directly pulling the two aligned entities together 


% optimally dynamic alignment strength
% {\em Optimal alignment strength to maximize the benefits of knowledge distillation is difficult to obtain, especially in the temporal manner.} 
% In practical, although the aligned entities can share similar information, they may still differ in other perspectives, including but not limited to frequency, interactions, and temporal patterns. How to adjust the alignment strength (i.e., the distance constrains of the aligned entities in the uni-space) accordingly for different entities at different time is unclear. \zheng{Ruijie TODO: add Ventura case}Moreover, in a case that the alignments are not fully reliable, directly pulling the two aligned entities together can even hurt the performance.



% scarcity of hinders the efficient
% knowledge transfer across languages. 
% {\em Transferring knowledge through a small set of alignments is hard to augment information for all entities.} 

% Aligning the same entities across languages rely heavily on manual labeling or rule-based inference~\cite{EA1,EA2,EA3,selfKG}, which is too time-consuming and impractical to obtain the alignments covering most of the entities in target language. 

% In this paper, we study how to boost the TKG reasoning performance in low-resource languages by explicitly increasing the completeness of those TKGs in history. Instead of improving the underlying information extraction techniques in low-data regime, we propose a new task called {\em Cross-lingual Temporal Knowledge Graph Reasoning}, motivated by the facts that there exists common or complementary knowledge shared by the TKGs in different languages under similar topics. The new task aims to facilitate TKG reasoning in low-resource languages (target languages) by distilling knowledge from a corresponding TKG in high-resource language (source language)  through a small set of entity alignments as bridges~\footnote{In this paper, we interchangeably use the terminology high-resource/low-resource and source/target for briety.}. Figure~\ref{fig:illustration} provides an illustrative example of the proposed task.


% Unfortunately, recent breakthroughs in temporal knowledge graph reasoning model~\cite{TA-DistMult,Know-Evolve,Renet,RE-GCN} highly rely on the completeness of the TKGs, especially for the most recent events. 

% However, the completeness of TKGs varies a lot across different languages, even under similar topics. Conventionally, the TKG construction process relies primarily on information extraction techniques built on the unstructured corpus~\cite{WIKI,YAGO, EventKG}. Therefore, the amount of corpus and human annotations in different languages significantly influence the quality of the corresponding TKGs . 
% Therefore, automatically completing/updating TKGs has been attracting enormous interests in recently years, which aims to predict recent/future events on TKGs based on historical events~\cite{TA-DistMult,Know-Evolve,Renet,RE-GCN}, namely temporal knowledge graph reasoning~\footnote{Broadly speaking, TKG reasoning includes interpolation to predict historical events and extrapolation to predict future events. In this paper, we refer to extrapolation task as TKG reasoning, since it is more vital for time-sensitive downstream tasks.}.


% For languages with large-scale and carefully labeled corpus (we refer to as high-resource languages, e.g., English), the constructed TKGs are more comprehensive than TKGs in other languages that lack the high-quality corpus (we refer to as low-resource languages, e.g., Spanish, Slovene, Danish, etc). Such completeness discrepancy leads to distinctly uneven TKG reasoning performances in different languages, which in turn affects the quality of service of the downstream applications. 


% Compared with the traditional TKG reasoning task, the new task imposes non-trivial challenges. An intuitive solution is to construct a unified graph including two TKGs in both source and target languages, and the knowledge distillation can be fulfilled by pulling the aligned entities from two languages close to each other in the uni-space~\cite{AlignKGC,KEnS}. However, there are still two challenges to be addressed. 

% \zheng{Ruijie TODO, Place this part to related works.}
% Existing works in related areas fail to address the aforementioned challenges. Monolingual reasoning methods on static/temporal knowledge graphs~\cite{TransE,TranR,ComplEX,RotatE,TA-DistMult,Know-Evolve,Renet,RE-GCN} is incapable of the desired knowledge transferring due to the insufficient alignment modeling. Although they can be extended on the cross-lingual scenario by viewing the alignments as a new relation on the merged TKGs, the limited amount of alignments prevent them from augmenting information for most of the entities. Entity alignment methods on KGs~\cite{EA1,EA2,EA3,EA4,EA5,selfKG} can automatically enlarge the alignments by  predicting the correspondence between the two TGs. But most of them, if not all, require the relatively even completeness of two TGs to capture the structural similarities, which can not be satisfied in our case, as target TKGs are far from complete. Some recent works start to study the multilingual TK reasoning on static graphs~\cite{AlignKGC,KEnS,SS-AGA}, which similarly aim to extract knowledge from several source KGs to boost the reasoning performance in the target KG, while they still require a sufficient amount of cross-lingual alignments and totally ignore the temporal perspective in our task.

% to facilitate temporal knowledge graph reasoning in low-resource languages. 
% increase the TKG connection and target TKG capacity
% In light of the mutual benefits, we iteratively generate pseudo alignment pairs and pseudo temporal events to address the co-existing scarcity issue in both cross-lingual alignment and target TKGs. 


In this paper, we propose a novel Mutually-paced Knowledge Distillation (\model) framework, where a teacher network learns more enriched temporal knowledge and reasoning skills from the source TKG to facilitate the learning of a student network in the low-data target one. The knowledge transfer is enabled via an alignment module, which estimates entity correspondence across languages based on temporal patterns. Firstly, to alleviate the limited language alignments (\textbf{Challenge \#1}), such a knowledge distillation process is mutually paced over time. This means, on one hand, we encourage the mutually interactive learning between the teacher and student. Concretely, the alignment module between the teacher and the student learns to generate pseudo alignment between TKGs to maximally expand the upper bound of knowledge transfer. And subsequently, it empowers the student to encode more informative knowledge in target TKG, which can in turn boost the alignment module to explore more reasonable alignments as the bridge across TKGs. One the other hand, inspired by self-paced learning~\cite{spl-1,spl-2}, we make the generations as a progressively easy-to-hard process over time. We start from generating reliable pseudo data with high confidence. As time goes by, we then gradually increase the generation amount by relieving the restriction over time. Secondly, to inhibit the temporal knowledge mismatch (\textbf{Challenge \#2}), the attention module can estimate the graph alignment strength distribution over time. This is achieved by a temporal cross-lingual attention in terms of the local graph structure and temporal-evolving patterns of aligned entities. As such, it can dynamically control the negative effect and suppress noise  propagation from the source TKG. Moreover, we provide a theoretical convergence guarantee for the training objective on both initial ground-truth data and pseudo data. To evaluate \model, we conduct extensive experiments of 12 cross-lingual TKG transfer tasks in multilingual EventKG dataset~\cite{EventKG}. Our empirical results show that the \model method outperforms state-of-the-art baselines in both with and without alignment noise settings, where only $20\%$ of temporal events in the target KG and $10\%$ of cross-lingual alignments are preserved.

% To validate the effectiveness of \model, we conduct extensive experiments of 12 cross-lingual TKG transfer tasks in multilingual EventKG benchmark dataset~\cite{EventKG} . Our experimental results empirically demonstrate the superiority of the \model method over state-of-the-art baselines, ranging from static KG embedding~\cite{TransE,TransR,DistMult,RotatE}, temporal KG reasoning~\cite{TA-DistMult,Renet,RE-GCN} to multilingual KG completion~\cite{KEnS,AlignKGC,SS-AGA}, in both with and without alignment noise settings. We further conduct comprehensive ablation and hyperparameter studies to validate the effectiveness of each design choices. Moreover, we provide theoretical analysis of convergence guarantee for the training objective on both initial groundtruth data and pseudo generative data.



To sum up, our contributions are three-fold:

\begin{itemize}[leftmargin = 15pt]
    \item \textbf{Problem formulation}: We propose the cross-lingual temporal knowledge graph reasoning task, to boost the temporal reasoning performance in target TKG by transferring knowledge from source TKG;
    \item \textbf{Novel framework}: We propose a novel \model framework, which enables the mutually-paced learning between the teacher and student networks, to promote both pseudo alignments and knowledge transfer reliability. Besides, \model involves a dynamic alignment estimation across TKGs that inhibits the influence of temporal knowledge discrepancy.
    \item \textbf{Extensive evaluations}: Empirically, extensive experiments on 12 cross-lingual TKG transfer tasks in multilingual EventKG benchmark dataset demonstrate the effectiveness of \model.
\end{itemize}
% pseudo data generation technique to progressively enhance the training data. The generated pseudo alignments can help the training of the representation modules by the knowledge distillation, and in turn adding pseudo events in the target TKG can improves alignment module by providing high-quality representations. 




% interactively
% TKGs in a source language and a target language are represented by a teacher representation module and a student one into a uni-space, respectively. 
% The knowledge distillation is enabled by a cross-lingual alignment module which pulls the aligned entities close to each other and push other entities far away. 
% To address the challenge caused by the scarcity of cross-lingual alignment, 



\section{The Xenoscope facility}
\label{sec:experimental_setup}

Xenoscope can house up to \SI{400}{kg} of LXe in a double-walled stainless steel cryostat. The facility, its subsystems, and the outcome from the first commissioning run are described in Ref.~\cite{Baudis:2021ipf}. Xenoscope was first equipped with a purity monitor (section \ref{sec:purity_monitor}) fully submerged in LXe, while the cryostat aspect ratio was chosen to allow for the operation, in the next phase of the project, of a \SI{2.6}{m} two-phase TPC, with the primary goal of demonstrating the drift of electrons in LXe over this distance for the first time. A computer-aided design (CAD) rendering of the cryostat with the purity monitor is shown in figure~\ref{fig:phases}.

The facility includes a gas purification system with a series of filters and a commercial zirconium alloy getter. The LXe is extracted at the top of the liquid column, where the impurity concentration is higher. It is evaporated in the heat exchanger system and circulated through the purification system at a fixed flow. The purified xenon is recondensed in the heat exchanger and reintroduced in the cooling tower, which comprises a pulse tube refrigerator (PTR) connected to a cold head mounted atop the cooling chamber. The xenon is then directed to the bottom of the cryostat. A slow control system built from open-source software oversees and sends alarms on relevant parameters.

Two system upgrades were performed prior to the installation of the purity monitor. First, a pre-cooler was manufactured and installed at the top of the inner cryostat vessel to provide additional peak cooling power, and thus to reduce the system cooldown and xenon liquefaction time during filling by a factor of 4.25. The design of the pre-cooler and details of its commissioning are presented in~\ref{sec:pre-cooler}. Furthermore, a gravity-assisted recuperation and storage system for LXe, Ball of Xenon (BoX), was deployed to allow for the storage of up to \SI{450}{\kilogram} of xenon at room temperature, as well as for recuperation in liquid phase. The latter enhances the speed of the recuperation process by a factor $\sim$ 8 compared to gaseous recuperation to a bottle array via cryogenic pumping. More details of its design and performance are presented in~\ref{sec:BoX}.

\begin{figure}[ht!]
\centering
\includegraphics[width=\columnwidth]{figures/render-PM3.pdf}
\caption[The purity monitor in the Xenoscope cryostat]{The purity monitor in the Xenoscope cryostat. Legend: (1)~top flange; (2)~outer vessel; (3)~inner vessel; (4)~pre-cooler; (5)~purity monitor; (6)~BoX recuperation line; (7)~anode; (8)~anode grid; (9)~field-shaping rings and resistor chain; (10)~support pillars; (11)~cathode grid; (12)~cathode disk; (13)~photocathode and optical fibre.}
\label{fig:phases}
\end{figure}



\section{The purity monitor and measurements}
\label{sec:purity_monitor}

Common impurities in commercially available xenon consist of parts-per-million (ppm) levels of O$_{2}$, N$_{2}$, H$_{2}$O, CO, as well as organic molecules~\cite{Hasterok:2017ehi}. Additionally, detector and subsystem materials introduce impurities by outgassing. The purification of xenon prevents electron losses via their attachment to electronegative impurities and allows to achieve high light and charge yields.

Most purity monitors measure the charge deficit of an initially known population of electrons after their drift through the liquid. By comparing the number of electrons before, $N_0$, and after the drift, $N(t_\mathrm{d})$, an indirect measurement of the impurity concentration in LXe can be achieved. The deficit can be modelled as a decaying exponential:
%
\begin{align}    
    N(t_{\text{d}}) = N_0~\mathrm{e}^{-t_{\text{d}}/\tau}~,
    \label{eq:electron_loss}
\end{align}
where $\tau$ is the electron drift lifetime. It relates to the concentration of electronegative impurities as:
%
\begin{align}
    \tau = \frac{1}{\sum_{i} k_{i} n_{i}}~,
    \label{eq:attachement}
\end{align}
\noindent where $k_i$ is the attachment rate specific to the impurity type in units of volume per mol per time, usually given in $\si{\mathrm{\liter/(\mol}\cdot\mathrm{\second})}$, $n_i$ is the impurity concentration given in \si{\mol/\liter}, and the sum extends over the different electronegative species $i$ in the LXe. The attachment rate coefficient depends on the electric field strength.

A schematic of the working principle of the purity monitor is shown in figure~\ref{fig:purity_monitor_concept}, left. An optical fibre transmits the light from a xenon flash lamp to the centre of a photocathode. The incident photons produce electrons via the photoelectric effect. The electrons are drifted via extraction, drift, and collection electric fields, generated by four biased electrodes. The first drift region (1) is located between the cathode (with the photocathode in the centre) and the cathode grid; the second region (2) extends up to the anode grid; the third region (3) extends from the anode grid to the anode. The charges induce a current signal in the cathode as they drift towards the cathode grid. The screening grids prevent current induction in the cathode and anode when the electrons are drifting along the second region. Once the electrons reach the third drift region, a second signal is generated at the anode, until the electrons are fully collected. Two electronic circuits amplify and convert the induced currents to voltage signals. 

The data is acquired and digitised, triggered by the pulse generator which also starts the discharge in the xenon flash lamp, with a window of \SI{100}{\micro \second} for the anode and cathode waveforms. Once digitised, the voltage signals are integrated to obtain the charges, i.e., the number of extracted and surviving electrons. With the induced charges and the time between the two signals, which corresponds to the drift time for the applied electric field, the electron drift lifetime can be inferred by solving numerically the equation:

\begin{align}
\frac{Q_A}{Q_C} = \frac{t_1}{t_3} \mathrm{e}^{-(t_1+t_2+t_3)/\tau} \frac{(\mathrm{e}^{t_3/\tau} - 1)}{(\mathrm{e}^{-t_1/\tau} - 1)}~.
\label{eq:lifetime}
\end{align}

\noindent Here, $Q_A$ and $Q_C$ are the charges from the integrated signals measured in the anode and cathode, respectively, $t_{1}$ is the rise time of the first signal, $t_{2}$ is the time between the minimum of the signal in the cathode and the rise time of the signal in the anode, with $t_{3}$ the time from $t_{2}$ up to the maximum of the anode signal. Given the motion of the charges, the signal in the cathode has negative polarity, while in the anode the polarity is positive. 
Figure~\ref{fig:signals_inLXe} shows an example of signals acquired in LXe from the cathode and anode at \SI{40}{slpm} along with the three drift times.

The design of the Xenoscope purity monitor is described in detail in Refs.~\cite{Baudis:2021ipf,Biondi:2022T}, and the assembled module is shown in figure~\ref{fig:purity_monitor_concept}, right. It features a field cage built with high conductivity, oxygen-free copper rings, supported by six polyamide-imide pillars. The rings are connected by a resistor chain of $\SI{5}{G\Omega}$ impedance each, and enclose a cylindrical drift region of \SI{15}{\cm}\,\diameter\,$\times$\,\SI{53.1}{\cm}. The cathode and anode grids consist of hexagonally-patterned, etched stainless steel meshes with high optical transparency ($\sim 93\%$), while the cathode and anode are solid stainless steel disks.


\begin{figure}[h!]
\centering
\includegraphics[width=0.24\textwidth]{figures/pm_schematic.jpg}
\includegraphics[width=0.15\textwidth]{figures/PM-high-res.jpg}
\caption{(Left): Schematic of the purity monitor. A pulse generator triggers a flash from the xenon lamp and the light is transmitted through an optical fibre to the photocathode, where photoelectrons are produced. The electrons are extracted, transported and collected by three electric fields, defined by the cathode ($\mathrm{G_C}$) and cathode grid ($\mathrm{G_1}$), the anode grid ($\mathrm{G_2}$) and the anode ($\mathrm{G_A}$). In the longest region (2), the field shaping rings~(FSR) maintain the uniformity of the drift field $\vec{E_d}$ in the vertical direction. (Right): Assembled purity monitor in Xenoscope.}
\label{fig:purity_monitor_concept}
\end{figure}

\begin{figure}[h!]
\centering
\includegraphics[width=\columnwidth]{figures/signals.pdf}
\caption[Signal acquired at \SI{40}{slpm} xenon recirculation speed.]{Signals acquired at \SI{40}{slpm} xenon recirculation The rise time of the cathode signal (blue) is taken as $t_{1}$, the time interval between the minimum of the cathode signal and the start of the anode signal (red) is taken as $t_{2}$, with half of the charge cloud completely collected at $t_{3}$. These values are later used to calculate the electron drift lifetime.}
\label{fig:signals_inLXe}
\end{figure}
%=============================================================================
\subsection{Optical components and photocathode}

The utilised lamp is a \SI{60}{\watt} xenon flash lamp with a \mbox{built-in} reflective mirror (model number L7685) from \textit{Hamamatsu}~\cite{hamamatsu}. The window is a single sapphire crystal allowing short wavelength light ($\sim$~\SI{190}{\nano\meter}) to reach the output of the lamp, with a spectral emission from \SI{190}{\nano\meter} to \SI{2000}{\nano\meter}. The lamp generates a discharge which excites the gas producing scintillation, with reflective mirrors directing photons from all directions towards the output. The xenon lamp can be triggered internally, or externally via a pulse generator. The intensity of the light emission is adjusted by setting the voltage for the discharge in the lamp between \SI{600}{\volt} and \SI{1}{\kilo\volt}. The selection of the latter maximises the number of produced electrons.

 Measures were adopted to mitigate the electronic noise produced in the signal waveforms by the external trigger: the xenon flash lamp was rehoused in a stray electromagnetic interference box, and galvanic insulation and ferrite filters were added to the trigger line. The box was customised by adding a potentiometer to manually change the voltage of the discharge. A trigger circuit controlled by the pulse generator was added as well. The lamp is placed outside the cryostat, with an optical fibre carrying the light from its output to the surface of the photocathode. A UV grade sapphire lens produced by \textit{Hamamatsu} is placed at the output of the lamp to collimate the light to the optical fibre. The selected fibre is $\SI{600}{\micro\meter}$ in diameter, ultra-high vacuum rated with a polyimide buffer from \textit{LewVac}~\cite{LewVac}. It is resistant to UV solarisation, i.e.~the degradation in the fibre material due to the exposure to light of wavelength lower than \SI{300}{\nano\meter}. The fibre feedthrough, produced by \textit{Thorlabs}~\cite{feedthrough}, consists of a $\SI{600}{\micro m}$ multimode fibre in an SMA connector welded on a CF40 flange. The feedthrough requires the fibres to be terminated, hence these were prepared and polished in-house with a set of 8~lapping sheets, made of aluminium oxide, silicon carbide, and calcined alumina, from grits of $\SI{30}{\micro\meter}$ to $\SI{0.02}{\micro\meter}$. 
 
One of the critical parts of a purity monitor is the photocathode, for it directly impacts the size of the initial signal. It consists of a thin layer of a low work function metal, deposited on a quartz substrate that has low absorption of UV photons~\cite{Valentini2002}. The photocathode was produced in-house using a turbomolecular pumped coater Q150T Plus from \textit{Quorum}~\cite{coater}. The desired thickness of the layer was monitored with a quartz crystal microbalance. Different materials for the thin layer were tested, including gold and silver, and the coater was used to produce photocathodes of \SI{50}{\nano\meter} thickness on a \SI{2}{\mm} thick quartz substrate, with a diameter of \SI{30.00(5)}{\mm}. The deposition of a \SI{5}{\nano\meter} thick layer of titanium was required in the case of gold for adhesion to the substrate. The choice of thickness was based on the effective probe depth of gold layers, and previous works~\cite{Manenti2020}. Additional technical details can be found in Ref.~\cite{Biondi:2022T}.

The photocathodes were tested in a vacuum setup, where the xenon lamp was flashed onto the photocathode material and the induced current was measured. Both gold and silver showed high yields, with gold reaching a stable state in fewer hours of exposure to the UV signal. Silver and gold photocathodes showed an increasing quantum efficiency with time when exposed to light, and this increased yield did not revert back in subsequent tests. The increase in quantum efficiency of the photocathode with UV-light exposure was also observed in Ref.~\cite{Manenti2020}. The gold photocathode was selected due to its stability over time and higher quantum efficiency than the silver photocathode, requiring smaller electric fields to produce a higher charge signal. 

%===============================================================================
\subsection{Current readout and signal processing}

The readout electronics amplify the induced currents from the cathode and anode and are placed inside the cryostat to avoid signal losses along the \SI{9}{\meter} signal cables. The circuits were designed together with the Electronics Workshop at the University of Zurich. The circuit consists of an AC-coupling component, a transimpedance amplifier, and a final voltage amplifier with a $\SI{50}{\ohm}$ impedance termination to match the one from the data acquisition. The transimpedance and voltage amplifiers are implemented with two low-cost operational amplifiers, model AD8066 from \textit{Analog Devices}~\cite{analogdev}. An AC-coupling filter in the circuit board removes high-frequency noise, which enhances the signal quality, and the AC-coupling removes the DC component of the HV applied to the electrodes. The usage of a transimpedance amplifier, in contrast to a charge amplifier, allows for more precise timing and signal spread analyses due to its small resistive-capacitive constant (RC) and short rise time of \SI{0.14}{\micro\second}. However, due to its fast response, a low-pass filter for frequencies below \SI{800}{\kilo\hertz} is applied to the signals to decrease the electronic noise induced by, e.g., the pulse generator that triggers the lamp, two temperature sensors, and the uninterruptible power supply. The preamplifier operates in current mode, as the capacitance discharges rapidly, resulting in an output voltage proportional to the instantaneous current. The frequency response of the readout electronics was benchmarked, with a negligible effect on the signal shape due to the $\SI{100}{MHz}$ bandwidth. 

The performance of the readout electronics was tested in a climate chamber in steps of \SI{10}{\kelvin} from room temperature down to \SI{190}{\kelvin}. The calibration showed a charge amplification of $\SI{0.18}{\femto\coulomb/(\milli\volt\cdot\micro\second)}$, with good thermal stability. Additionally, the RC decay constant of the circuit, which could be a source of systematic error for time measurements, was estimated at $\sim$~\SI{150}{\nano\second} by feeding a \SI{2}{\micro\second} wide square pulse to the circuit.

An oscilloscope, \textit{Teledyne LeCroy} model Waverunner 6104A~\cite{teledyne}, and an analog-to-digital converter from \textit{CAEN}, model v1724~\cite{CAEN}, acquired the waveforms produced by the cathode and anode readout. Each acquisition consisted of the average of \SI{1000}{} waveforms acquired over \SI{16.7}{minutes} to minimise the baseline noise. The signals were then processed by fitting the expected signal shape with a Gaussian distribution. In some waveforms, a noise introduced by external electronic devices could be discerned as part of the background noise, and the fit included a sine function to account for this effect, with an inferred model uncertainty of $5\%$ for the ones where the sine fit to the noise did not converge. The current-equivalent voltage signals in the cathode and anode were integrated to obtain a charge-proportional value. The residuals of the fits were used as weights for the charge values obtained in the averaged data shown in the next section. The uncertainties in charges and times obtained in the fits were propagated to obtain the uncertainty of the electron drift lifetime value. An example of the raw anode signal at \SI{53}{\volt/\cm} drift field in region 2 is shown in figure~\ref{fig:FWHM_signals}, together with the post-processing signal with a low-pass filter. The calculated baseline and Gaussian fit of the signals are also shown.

\begin{figure}[h!]
\centering
\includegraphics[width=0.49\textwidth]{figures/signal_example_fit.pdf}
\caption[Anode signals at \SI{53}{\volt/\cm}]{Anode signal at \SI{53}{\volt/\cm} prior to (blue) and after (orange) the low-pass filter. The calculated baseline (red) and a Gaussian fit of the signal (green) are also shown. The signal is an average over \SI{1000}{} recorded waveforms.}
\label{fig:FWHM_signals}
\end{figure}

\subsection{Measurements}

Once installed in the cryostat, the purity monitor was first operated in vacuum ($\sim \SI{1e-5}{\milli\bar}$). Data was acquired to investigate the signal shape and response in this configuration with negligible charge losses due to residual gas. The measurement additionally provided the delay time of the electronics chain, from the pulse generator for the xenon lamp to the signal amplification and readout of $\SI{18}{\micro\second}$.

After the calibration of the purity monitor in vacuum, gaseous xenon was flushed inside the detector and purified through recirculation in the gas system. The LXe run started with the filling of \SI{343}{\kilogram} of xenon. As the xenon recirculates through the getter, electronegative impurities are removed, and the electron drift lifetime is expected to increase in two steps: an initial exponentially increasing phase where the bulk impurities are rapidly removed, and a second phase where the change is dominated by the materials outgassing, and where the electron drift lifetime slowly increases over time. At different recirculation speeds, the electron drift lifetime reaches increasingly higher values in the second phase.

The recirculation speed was set with flows of \SI{30}{standard\ litres \ per\ minute\ (slpm)}, \SI{35}{slpm} and \SI{40}{slpm}, with the xenon lamp illuminating the photocathode with a frequency of \SI{1}{\hertz}. In the cryostat, the temperature and pressure were maintained around \SI{177.6(1)}{\kelvin} and  \SI{2.05(1)}{\bar}, respectively. Following the commissioning run, the displacement of the GXe compressor was reduced to increase its lifespan. This constrained the maximum purification speed to \SI{40}{slpm}, compared to the \SI{80}{slpm} reported in Ref.~\cite{Baudis:2021ipf}. The initial impurity level in the xenon gas impacts the number of days before a signal can be seen in the purity monitor: the  first waveforms in the cathode and anode were observed after $\sim$26.5 days.

During data taking, the cathode and cathode grid were biased at \SI{-2710}{\volt} and \SI{-2650}{\volt}, respectively. The anode grid was kept at ground while the anode was biased at \SI{500}{\volt}. The values were selected based on COMSOL~\cite{multiphysicscomsol} simulations which yielded nearly $100\%$ extraction efficiency of the electrons produced in the centre of the photocathode. Table~\ref{table:drift_times} shows the summary of the distances, times electric fields for the  extraction (1), drift (2) and collection (3) regions.
\begin{table}[h!]
\centering
\caption[Distance and drift fields]{Electric fields, distances and times $t_{i}$ measured for the three regions in the PM, with voltages \SI{-2710}{\volt}, \SI{-2650}{\volt}, \SI{0}{\volt}, and \SI{500}{\volt} for the cathode, cathode grid, anode grid and anode, respectively, for a purification speed of \SI{40}{slpm}.}
\begin{tabular}{clll} 
\hline 
Drift region $i$ & Distance [mm] & Field [V/cm] & $t_{i}$ $[\SI{}{\micro \second}]$\\
\hline
 1 &  $18 \pm 1$  & $33 \pm 1 $ & $12.8 \pm 0.8$\\
 2 &  $503 \pm 5$ &  $53 \pm 1 $ & $433.5\pm0.7$ \\
 3 &  $10 \pm 1$  & $500 \pm 5$ & $7.6 \pm 0.7$
\label{table:drift_times}
\end{tabular}
\end{table}

Figure~\ref{fig:charge_anode_cathode} shows the anode and cathode signals with their integral, where the integrated signals show a step-like feature after the charges move entirely to the next drift region, or are collected in the anode. The integration corresponds to the total area of the Gaussian fit. The charge measured in the cathode corresponds to $N_{\text{e}^{-}} \cong 10^{6}$ electrons extracted from the photocathode at each pulse.

\begin{figure}[b]
\centering
\includegraphics[width=\columnwidth]{figures/charge_cathode.pdf}
\centering
\includegraphics[width=\columnwidth]{figures/charge_anode.pdf}
\caption[Signal readout at the cathode with the integrated charge signal]{Signal readout (blue) at the cathode (top) and anode (bottom)  with their respective Gaussian fits (orange) and integrated charge signals (red).}
\label{fig:charge_anode_cathode}
\end{figure}




\section{Results and discussion}
\label{sec:results_discussion}

\begin{figure*}[t!]
\centering
\includegraphics[width=2\columnwidth]{figures/Elife_model.pdf}
\caption[Purification flow-dependent electron drift lifetime measured in the Xenoscope.]{Purification flow-dependent electron drift lifetime measured in Xenoscope. The data was averaged in \SI{6}{\hour} time bins. The dashed lines indicate a change in flow, while the dash-dotted lines indicate short-term irregularities in the pressure and flow conditions (see text). The red line shows the best-fit model from equations \ref{eq:e-lifetime-gas} and \ref{eq:e-lifetime-liquid}, while the black points show the residuals.}
\label{fig:electron_lifetime_datasets}
\end{figure*}

 The electron drift lifetime measurement campaign with the purity monitor lasted a total of \SI{88}{days}. The purification was performed at \SI{30}{slpm} for \SI{46.6}{days}, at \SI{35}{slpm} for \SI{20.0}{days}, and at \SI{40}{slpm} for \SI{21.2}{days}. After the electron drift lifetime measurements in LXe, signals for drift fields from \SI{25}{V/cm} to \SI{75}{V/cm} were acquired to study field-dependent electron transport properties, such as the drift velocity and longitudinal electron cloud diffusion.

\subsection{Electron drift lifetime}

Figure~\ref{fig:electron_lifetime_datasets} shows the electron drift lifetime calculated with the charge signals acquired at the cathode and anode over the entire acquisition period. When the recirculation speed changes, the electron drift lifetime drops, most likely due to a change in the height of the liquid level, resulting in the release of trapped impurities in the high-surface tension region at the LXe \enquote{collar} (LXe/GXe/inner vessel interface). A drop in electron drift lifetime was also observed at \SI{59.8}{\day}, as expected, when the GXe compressor was stopped for a period of approximately \SI{15}{minutes} due to a communication error with the slow control software. Again, the change in liquid level most likely resulted in the sudden release of impurities from the collar. Shortly following these events, the electron drift lifetime increased exponentially to return to the outgassing-limited values.

A review of the slow control data allowed for the identification of three irregularities in the pressure and flow conditions, at \SI{38.9}{\day}, \SI{52.6}{\day}, and \SI{80.7}{\day}. The first was a brief moment of excess flow downstream of the GXe compressor with a slight pressure increase of approximately $\SI{10}{\milli bar}$, suggesting the release of trapped gas in the xenon handling system from vibration, or the unlikely development of a micro-leak. The second irregularity was a quick fluctuation in the purification flow, both upstream and downstream of the GXe compressor, resulting in a momentary increase in pressure both downstream of the compressor and in the inner vessel, also leading to a change in the liquid level. The last irregularity was again a marginal increase in the flow downstream of the compressor. In this case, however, the event lasted approximately \SI{3}{\hour}.
After \SI{88.0}{\day} of almost continuous purification, the electron drift lifetime reached a value of $\SI{664(23)}{\micro\second}$. This is consistent with the value reached in other LXe experiments, such as XENON1T~\cite{XENON:2021nad} and LUX~\cite{LUX:2020vbj}, with \SI{660}{\micro\second} and \SI{750}{\micro\second}, respectively. 

While the purity level demonstrated in Xenoscope could be sufficient to drift electrons over \SI{2.6}{\meter} in LXe, it can be further improved by increasing the purification speed. To investigate this, a simple model of the electron drift lifetime, assuming O$_2$-like impurities, was adapted from Refs.~\cite{greeneXENON1TSpinIndependentWIMP2018,Plante:2022khm} and fitted to the electron drift lifetime data shown in figure~\ref{fig:electron_lifetime_datasets}. Two coupled differential equations describe changes in impurity levels $M \dv{I_{}}{t}$ after a time $\dd t$, where $M$ is the mass of xenon, and $I$ is the impurity concentration. The gas and liquid phases (denoted by the subscripts $g$ and $l$, respectively) are evaluated separately to determine the impurity concentration at time $t+\dd t$:
%
\begin{multline}
M_{\mathrm{g}} \dv{I_{\mathrm{g}}}{t}^{(j)} = -F_{\mathrm{g}} \rho I_{\mathrm{g}} + \left(\frac{\Lambda_{\mathrm{g},0}}{1+\frac{t-\Delta t^{(j)}_g}{T_{1/2,g}}} + C_\mathrm{g}\right)\\ + \frac{\epsilon_1 P_{\mathrm{C}} I_{\mathrm{l}}}{h} - \frac{\epsilon_2 P I_{\mathrm{g}}}{h} + M_{\mathrm{g}} \Delta I_{\mathrm{g}}^{(j)} \int \delta\left(t-t^{(j)}\right) \dd t
\label{eq:e-lifetime-gas}
\end{multline}
%
\begin{multline}
M_{\mathrm{l}} \dv{I_{\mathrm{l}}}{t}^{(j)} = -F_{\mathrm{l}} \rho I_{\mathrm{l}} + \left(\frac{\Lambda_{\mathrm{l},0}}{1+\frac{t-\Delta t^{(j)}_l}{T_{1/2,l}}} + C_\mathrm{l}\right)\\- \frac{\epsilon_1 P_{\mathrm{C}} I_{\mathrm{l}}}{h} + \frac{\epsilon_2 P I_{\mathrm{g}}}{h} + M_{\mathrm{l}} \Delta I_{\mathrm{l}}^{(j)} \int \delta\left(t-t^{(j)}\right) \dd t~.
\label{eq:e-lifetime-liquid}
\end{multline}
%
\noindent The index $j = \{0, 1,...,7\}$ indicates the regions between discontinuities, marked by the dashed lines in figure~\ref{fig:electron_lifetime_datasets}. Each equation consists of five terms. The first accounts for the purification rate, with $F$ being the purification flow, $\rho$ the density of xenon at \SI{1}{bar}, \SI{0}{\celsius}, and $I$ the concentration in impurities. The second term accounts for the time-dependent average outgassing rate from detector materials, where $\Lambda_0$ is the outgassing at time $t=0$, $T_{1/2}$ is the decay-time of the outgassing rate and $C$ is a constant outgassing term which brings the system to an equilibrium point when $t \gg T_{1/2}$. After the sudden increase of impurity concentration in the xenon during flow changes, described by the delta function in the fifth term, the outgassing rate modelled by the second terms of the differential equations is expected to revert back to a high value, as impurities can be adsorbed in outgassed materials~\cite{Plante:2022khm}. This is accounted for with the time-offset parameters $\Delta t^{(j)} = \sum_1^j \Delta t ^{(i)}$, which are cumulative since the dataset is fitted as a whole, and $\Delta t^{(0)} = 0$. Finally, the third and fourth terms describe the exchange of impurities between the gas and the liquid phases, and thus the signs are inverted between the two equations. The $\epsilon$ parameters are efficiencies of the exchange process, $P_{\mathrm{C}}$ is the cooling power of the system in the absence of purification, proportional to the evaporation rate of the xenon, $P$ is the cooling power deployed by the cryogenics, proportional to the condensation rate, and $h$ is the latent heat of xenon.

The electron drift lifetime is then calculated by solving the system of differential equations for equation~\ref{eq:attachement} and minimising with the iMinuit (Python) package~\cite{James:1975dr} simultaneously in all $j$-regions, using a least-square method. The best-fit model is displayed as a solid red curve in figure~\ref{fig:electron_lifetime_datasets}. Assuming the same initial conditions obtained from the fit of the model to the electron drift lifetime data, we obtain the purification flow-dependent electron drift lifetime predictions shown in figure~\ref{fig:prediction}.
%
\begin{figure}[h!]
\centering
\includegraphics[width=0.49\textwidth]{figures/prediction.pdf}
\caption{Purification flow-dependent electron drift lifetime prediction. Assuming the same operating conditions as in the measurement campaign, an increased flow (solid lines) could significantly reduce the purification time needed before the electron drift lifetime begins its exponential rise. Furthermore, the addition of a \SI{2}{slpm} gas phase extraction to the purification loop can also improve the electron drift lifetime (dashed lines).}
\label{fig:prediction}
\end{figure}
%

As expected, an increased purification speed would yield longer electron drift lifetimes, attained in a shorter purification time. The addition of a purification flow of the gas phase ($F_\mathrm{g}= \SI{2}{slpm}$) suggests an expected increase in electron drift lifetime of up to \SI{15}{\%}. Therefore, prior to the start of the next phase of Xenoscope, a parallel gas extraction line inspired by the gas purification system reported in Ref.~\cite{Plante:2022khm} was added to the gas handling system with a second flow controller, allowing for the purification of both the liquid and gas phases in the same purification loop. 
 
\subsection{Electron transport}

The purity monitor allows for a dedicated measurement of the arrival time of the electron cloud at the anode. The drift velocity $v_{\mathrm{d}}$ of the cloud given a drift field $E_{\mathrm{d}}$ is:
\begin{equation}
\label{velocity}
v_{\text{d}} =d_{2}/t_{2}.
\end{equation}
Considering that the accuracy of the time measurement between the extrema of the cathode and anode signals is higher, and region 3 has a fast detection, it is convenient to calculate instead:
\begin{equation}
\label{velocity}
v_{\text{d}} = (d_{2} + d_{3}) / (t_{2} + t_{3}).
\end{equation}
This approach introduces an additional term, $(d_{2}t_{3} - d_{3}t_{2})/(t_{2} (t_{2}+t_{3}))$, which induces a negligible $0.1\%$ bias in the final values given the ratios between $t_{3}/t_{2}$ and $d_{3}/d_{2}$. The drift velocity can also be expressed in terms of the drift field $E_{\mathrm{d}}$:
\begin{equation}
v_{\mathrm{d}} = \mu E_{\mathrm{d}}\:,
\end{equation}
where $\mu$ is the electron mobility. The mobility is related to the average time, for a given temperature, density, and energy of the electrons, between elastic collisions with the xenon and electronegative impurities, and potential inelastic collisions with impurities. Thus, the acquisition of waveforms used to derive the drift velocity was performed at a constant electron drift lifetime to avoid systematic uncertainties. Benchmark regimes can be used to visualise these dependencies for electron mobility, such as \enquote{cold electrons} and \enquote{hot electrons}~\cite{Schmidt:1984zz}. In the cold electrons regime, the energies of the electrons are mostly due to the thermal bath in the xenon fluid (around \SI{0.015}{eV} at \SI{177}{\kelvin}~\cite{Boyle:2016wpy}), and they rapidly acquire energy with increasing electric fields, resulting in a linear gain in velocity. For cold electrons, the mobility can be expressed as:
%
\begin{equation}
    \mu=\frac{2}{3}\left(\frac{2}{\pi\, m_e \,k_B \,T}\right)^{\frac{1}{2}} e \,\frac{\lambda}{v}\:,
    \label{eq:cold_electrons}
\end{equation}
where $k_{B}$ is the Boltzmann constant, $T$ is the temperature, $e$ is the charge of the electrons, m$_e$ is their mass, $v$ is the magnitude of the velocity in all directions, and $\lambda$ is the mean free path of the electrons in their collisions with atoms, inversely proportional to the number density and cross section. In contrast to cold electrons, hot electrons have gained most of their energy through acceleration by the drift field, and experience increased energy losses on their drift path due to collisions with xenon atoms, which results in a slower rate of change in their velocity. For this case, the electron mobility can be expressed as:
\begin{equation}
\mu = \frac{4\, e \, \lambda}{3 v  \pi^{1 / 2} m_e^*}\:,
\label{eq:hot_electrons}
\end{equation}
where $m_e^*$ is the effective mass of the electrons in the medium. While these equations are not used in this work to infer properties such as electron mobility, or electron cloud diffusion, they are useful to scrutinise the results of our measured drift velocity and discuss potential systematic effects, discussed later in this section. 

The data for this measurement was acquired at day 89, after the electron drift lifetime entered a region of slow change, and in a time interval of half an hour, to minimise systematic effects. Drift fields of \SI{25}{\volt/\cm} to \SI{75}{\volt/\cm} were scanned in steps of \SI{5}{\volt/\cm}, where the former was the threshold for a discernible signal above noise level in the cathode. The extraction field was changed to maintain the ratio between the extraction and drift field used in the electron drift lifetime data, while the collection field was fixed at \SI{500}{\volt/\centi\meter}. Figure~\ref{fig:drift_time_field} shows the drift velocity at different fields, calculated with equation~\ref{velocity}, with good agreement with previous measurements in LXe~\cite{gushchin1982electron, Albert:2016bhh,Baudis:2017xov,Thieme:2022dze,Jorg:2021hzu,Baur:2022sel}. The prediction from NEST (Noble Element Simulation Technique)~\cite{szydagis_m_2018_1314669}, based on data-driven empirical models, is also shown. The curves for cold and hot electrons are derived by using equations~\ref{eq:cold_electrons} and~\ref{eq:hot_electrons}, respectively, to fit the data from Ref.~\cite{gushchin1982electron}, for it covers both regimes with a high density of points.

\begin{figure}[h!]
\centering
\includegraphics[width=0.49\textwidth]{figures/drift_velocity_withXenoscope.pdf}
\caption{Measured drift velocity with electric field values from \SI{25}{} to \SI{75}{\volt/\cm}, in steps of \SI{5}{\volt/\cm}. The results are compared to literature values from Gushchin (\SI{165}{K})~\cite{gushchin1982electron}, EXO-200 (\SI{167}{\kelvin})~\cite{Albert:2016bhh}, Xurich II (2018, \SI{184}{\kelvin})~\cite{Baudis:2017xov}, Xurich II (2021, \SI{177}{\kelvin})~\cite{Thieme:2022dze}, HeXe (\SI{174}{K})~\cite{Jorg:2021hzu} and XeBRA (\SI{173}{\kelvin})~\cite{Baur:2022sel}. The prediction from NEST v2.3.8~\cite{szydagis_m_2018_1314669} is shown as a solid blue curve.}
\label{fig:drift_time_field}
\end{figure}

The spread of the electron cloud in time was studied by analysing the anode signal at the previously mentioned drift fields. With this purity monitor, only the longitudinal diffusion can be observed, for there is no information on the $x-y$ charge distribution. The standard deviation ($\sigma$) of the Gaussian fit of the signal in the anode is used to calculate the longitudinal diffusion coefficient. For the case of the initial and final distributions of the electron cloud following a Gaussian distribution, the width of the anode signal is related to the longitudinal diffusion~\cite{Albert:2016bhh,Li:2015rqa,Rolandi2008} as:
%
\begin{equation}
D_{L}=\frac{d^{2} \sigma_{L}^{2}}{2 t^{3}}\:,
\label{eq:diffusion1}
\end{equation}
\noindent
where $D_L$ is the diffusion coefficient of the electron population due to the random walk of the electrons in the longitudinal direction, and:
%
\begin{equation}
\label{eq:diffusion}
\sigma_{L}^{2}=\sigma^{2}-\sigma_{0}^{2}\:,
\end{equation}
\noindent
 is the width at one $\sigma$ considering diffusion effects only, which is the value to extract. The widths $\sigma_{0}$ and $\sigma$ belong to the initial signal at the cathode and the final signal measured in the anode, respectively, and $d$ is the drift length ($d_{2}+d_{3}$) at drift time $t$ ($t_{2}+t_{3}$). By rewriting equation~\ref{eq:diffusion}, we obtain the width of the anode signal:
%
\begin{equation}
    \sigma^2 = \frac{D_L 2 t^3}{d^2} + \sigma_0^2\:,
\label{eq:diffusion2}
\end{equation}
The diffusion of the electron cloud is related to the previously introduced electron mobility. At low drift fields, it follows the Einstein-Smoluchowski relation~\cite{einstein,smoluchowski,Schmidt:1984zz}:
\begin{equation}
D_L=\frac{k_{B} T}{e} \mu = \frac{\epsilon_{T}}{e} \mu\:,
\label{eq:diffusion_einstein}
\end{equation}
with thermal energy $\epsilon_{T}$.
The diffusion is thus affected by the cross section of elastic and inelastic interactions with the medium species (xenon and impurities), analogously to the drift velocity. 

 Since the electrons have energies above the thermal bath and are not in equilibrium, a characteristic energy, $\epsilon_k$, is defined as:
\begin{equation}
\epsilon_k = \frac{e D_L}{\mu}\:,
\label{eq:characteristic_energy}
\end{equation}
representing the energy associated with the longitudinal diffusion, where now $D_L$ has a contribution beyond the thermal energy of electrons:
\begin{equation}
D_L= \frac{(\epsilon_T + \epsilon )}{e}\mu\:,
\label{eq:characteristic_energy_longitudinal_diffusion}
\end{equation}
with $\epsilon =  \epsilon_k - \epsilon_{T}$.

In previous studies, it was reported that impurities diffused in the medium, from water vapour to organic molecules, can provide a more effective energy loss mechanism for electrons, with the consequence of higher mobility and decreased diffusion~\cite{Yoshino:1976zz,Pack:1961jag} (see equations~\ref{eq:cold_electrons}  and~\ref{eq:hot_electrons}). This effect could explain the mechanism behind the higher drift velocities in the early measurements of Guschin et. al. at \SI{164}{\kelvin}~\cite{gushchin1982electron}, where no information about xenon purification methods was given.

Additional effects can play a role in the detected spread of the charge distribution in our detector and must be corrected to obtain a longitudinal diffusion coefficient that is independent of energy, electron source or detector response. From the original number of extracted photoelectrons to the measured signal in the anode, the following effects can impact the width:
\begin{itemize}
    \item Duration of the pulse of the lamp, which introduces an initial signal width at one sigma of $\SI{2.4(2)}{\micro\second}$. In this work, the initial signal width was derived from the data acquired in vacuum and in LXe, and the signal in the cathode is deconvolved with the detector and electronics responses. 
    \item The detection response of the screening region~\cite{Shockley:1938itm}. The weighting potential between the solid electrodes and the hexagonal meshes is taken into account in the measured signal to yield the original electron cloud spread in the $z$-direction. The effect of the hexagonal screening meshes was simulated by modelling the 3D geometry of the meshes and detector and performing electrostatic simulations with COMSOL. The method to derive the weighting potential is adopted from Ref.~\cite{gook_application_2012}. The weighting potential is obtained by averaging the potential over different electron drift paths to smooth out local effects. 
    \item Coulomb repulsion between electrons, where each electron is affected by the electric field induced by other electrons, can increase the size of the electron cloud. The Coulomb repulsion calculation follows the empirical approach in Ref.~\cite{Njoya:2019ldm}, which considers the ellipsoid explosion model from Ref.~\cite{PhysRevE.84.056404}, where a set of differential equations is solved to obtain the final width of the signal after the drift. The repulsive forces are stronger immediately after the charge creation and become smaller as the electron cloud spreads along the drift path. After the electrons have drifted and reached the anode grid, an additional width value of 5\% compared to the no repulsion forces case is inferred from the empirical approach, and taken as an uncertainty in the charge distribution after their extraction from the photocathode.
    \item Electron attachment to electronegative impurities which can potentially change the distribution of the electron cloud. This could in principle affect the diffusion of electrons in LXe, and the method to estimate this impact is taken from Ref.~\cite{Li:2015rqa}. The longitudinal diffusion coefficient and drift velocity formulae are expanded to include higher-order terms containing the attachment rate. The effects of electron attachment in the diffusion can be neglected according to $\frac{D_L}{(v_\mathrm{d} \cdot d)} \ll 1$, which is the case for this study.
    \item Readout response times of the pre-amplifiers. The circuit response was estimated when benchmarking the electronics and has a negligible effect on the signal shape.
\end{itemize}

Combining all the effects introduced above, a response function is obtained to deconvolve the observed signal. Table~\ref{systematics} compiles the systematic effects treatment for the calculation of the longitudinal diffusion coefficient. The results of this deconvolution, for each measurement at different drift fields, are shown in figure~\ref{fig:longitudinal_diffusion}, together with literature values~\cite{Njoya:2019ldm,Hogenbirk2018}. The longitudinal diffusion coefficient was measured at relatively low drift fields (i.e.~$<\SI{100}{\volt/\cm}$). By using the values derived for the mobility for cold electrons and hot electrons included in  figure~\ref{fig:drift_time_field}, of $\SI{0.29}{\milli\meter^{2}/(\micro \second \cdot \volt)}$ and $\SI{0.01}{\milli\meter^{2}/(\micro \second \cdot \volt)}$, respectively, together with the thermal energy of electrons, reference coefficients for the diffusion can be obtained from equation~\ref{eq:characteristic_energy}. These are included in figure~\ref{fig:longitudinal_diffusion}. By comparing the experimental values with the benchmark equations, the difference between these is contained in terms of the characteristic energy of electrons and their mobility, given equation~\ref{eq:characteristic_energy_longitudinal_diffusion}.

\begin{table}
\centering
\caption{Systematic effects and impact on the uncertainty for the inferred longitudinal diffusion coefficient $D_L$.}
\label{systematics}
% @{\extracolsep{\fill}}
\begin{tabularx}{\columnwidth}{lX}
\hline\noalign{\smallskip}
Systematic effect & Treatment and uncertainty \\
\noalign{\smallskip}\hline\noalign{\smallskip}
\bf{Measured}        & \\
Anode signal width, $\sigma$        & Gaussian plus sine fit, $2-3\%$.  \\
Drift time,   $t_2 + t_{3}$     &  Time interval between extrema of the cathode and anode signal fits. \\
Initial signal width       & Introduced by the lamp pulse. Measurements in vacuum and in LXe,  $\SI{2.4(2)}{\micro\second}$.    \\
Electronics  & RC time constant calculation from a square pulse,  $\SI{0.2}{\micro\second}$.    \\
Drift length, $d_{2} + d_{3}$ & Drift distance of the electron cloud, taken as $\SI{513(7)}{\milli\meter}$ when accounting for the potential contraction of the components at \SI{177}{\kelvin} with an assumed 1\% thermal contraction, and the position of the centre of the cloud distribution in drift regions with respect the extrema of the signals.\\  
Filtering and processing & Maximum 4\% of anode signal width.\\
\noalign{\smallskip}\hline\noalign{\smallskip}
\bf{Simulated}        & \\
Detector response   & COMSOL 3D model of the detector to derive the weighting potential, 10\% uncertainty in the response.\\
\noalign{\smallskip}\hline\noalign{\smallskip}
\bf{Assumed}        & \\
Coulomb repulsion   & Calculated with empirical model from~\cite{Njoya:2019ldm}, assumption of additional 5\% uncertainty in the initial signal spread.\\
Electron attachment   &  Neglected, 4th-order correction: $\frac{D_L}{(v_d \cdot d)} \ll 1$. \\
\hline
\end{tabularx}
\end{table}

From the listed sources of uncertainty for the diffusion coefficient $D_L$, the largest impact originates (in percentages of the total uncertainty from \SI{25}{} to \SI{75}{V/cm}) from the final width measured at the anode (95\% to 80\%), the initial signal width (20\% to 50\%), the Coulomb repulsion (1\% to 5\%), the uncertainty in the drift distance (20 to 5\%) and the drift time (4 to 1\%). 

\begin{figure}[h!]
\centering
\includegraphics[width=0.5\textwidth]{figures/longitudinal_diff.pdf}
\caption[Longitudinal diffusion calculated in this work]{Longitudinal diffusion coefficient calculated in this work, with an electron drift lifetime of $\tau = \SI{649(23)}{\micro\second}$ compared to the results from a purity monitor by Njoya et. al. ($\tau \sim$ $\SIrange{1}{35}{\micro s}$)~\cite{Njoya:2019ldm}, and a TPC from Hogenbirk ($\tau \sim $ $\SI{430}{\micro s}$)~\cite{Hogenbirk2018} and NEST~\cite{szydagis_m_2018_1314669}. The model by NEST version 2.3.7 (solid grey curve) predicted diffusion values approaching zero for lower drift fields. In version 2.3.8, a fix for this behaviour was introduced, as shown in the solid blue curve.}
\label{fig:longitudinal_diffusion}
\end{figure}

NEST version 2.3.7 implemented an empirical model based on previous measurements for the longitudinal diffusion coefficient prediction. At the time of this analysis, NEST lacked data at low drift fields (below $\sim \SI{100}{V/cm}$), and the model predicted a considerably lower longitudinal diffusion coefficient with lower drift fields, in conflict  with the measured values in this and other works. An alternative model is included in NEST, based on differential cross sections derived from Dirac-Fock solutions, combined with Maxwell-Boltzmann distributions~\cite{Boyle:2016wpy}, which do not predict the existing experimental values at all drift fields. The subsequent NEST version 2.3.8 includes a correction on the diffusion modelling, also shown in figure~\ref{fig:longitudinal_diffusion}, resulting from an exchange with the developers. Our analysis aimed to not only infer the values of longitudinal diffusion coefficient at low drift fields for liquid xenon, but also to understand their origin, related to the temperature and purity of the xenon. 

\section{Conclusions and outlook}
\label{sec:conclusions}

The construction of a next-generation liquid xenon detector at the \SI{50}{t} scale and beyond will face several technological challenges. To address some of these, the Xenoscope facility was designed and built to house a \SI{2.6}{m} tall two-phase TPC at the University of Zurich, with a total LXe mass of $\sim$~\SI{400}{\kg}. After a commissioning run described in Ref.~\cite{Baudis:2021ipf}, we presented here the first results from a run with a \SI{53}{cm} tall purity monitor.

The electron drift lifetime was monitored for 88 days with varying xenon recirculation speeds. For a speed of \SI{40}{slpm}, the highest achieved lifetime was \SI{664(23)}{\micro \s}. A parametric model of the effect of the purification rate, the time-dependent outgassing rate, the liquid-gas impurity diffusion, and the injection of impurities due to operational changes, was fitted to the data. The resulting model was used to predict the electron drift lifetime evolution for different purification conditions and, therefore, inform future design and operation choices.

The electron drift velocity and the longitudinal diffusion coefficient of the electron cloud in liquid xenon were calculated based on data acquired at drift fields between \SI{25}{} and \SI{75}{V/cm}. With the increasing size of LXe TPCs, diffusion strongly affects the position reconstruction of events and the ability to discriminate between single and multiple interactions. Thus, accurate measurements of drift and diffusion properties, combined with an improved understanding of the systematic effect of impurity concentrations on these properties on large scales, are crucial. Our results are in agreement with previous studies both for drift velocity~\cite{gushchin1982electron,Albert:2016bhh,Baudis:2017xov,Thieme:2022dze} and longitudinal diffusion~\cite{Njoya:2019ldm, Hogenbirk2018}. They also triggered an update of NEST~\cite{szydagis_m_2018_1314669}, a simulation package largely used in the community, regarding the modelling of longitudinal diffusion of electron clouds in LXe.

For the next stage of the Xenoscope project, a two-phase xenon TPC was recently built and installed, and will be operated to observe electron drift over distances up to \SI{2.6}{\metre}. The upgrade includes liquid-level control and monitoring, high-voltage supply up to \SI{50}{kV} via a ceramic feedthrough, and an array of silicon photomultipliers for light-readout in the gas phase located just above the gas/liquid interface~\cite{sipm_proceedings}. The latter replaces the charge readout used at the anode of the purity monitor, detecting instead the proportional scintillation produced in the xenon gas region of the TPC. The TPC equipped with the SiPM array will be used to study electron cloud diffusion in both longitudinal and transverse directions. Transverse diffusion is another critical parameter for more accurate modelling of electron transport in xenon-based detectors, for which measurements in the literature are scarce~\cite{Albert:2016bhh, Aprile:2009dv}. Another goal of the upgrade is to study optical properties of liquid xenon at large scales, as well as new types of photosensors under the operating conditions of DARWIN. In addition, the facility will be available to the collaboration for various R\&D projects related to the realisation of a large-scale xenon TPC.




\begin{acknowledgements}
This work was supported by the European Research Council (ERC) under the European Union's Horizon 2020 research and innovation programme, grant agreement No. 742789 ({\sl Xenoscope}), by the SNF grant 20FL20-201437, as well as by the European Union’s Horizon 2020 research and innovation programme under the Marie Skłodowska -Curie grant agreement No 860881-HIDDeN. We thank the electronics and mechanical workshops in the UZH Physics Department for their continuous support. We thank Laura Manenti for insightful discussions about purity monitors.
\end{acknowledgements}

\appendix
\section{Additional Experimental Results}

\subsection{Implementation Details of Baselines}
We compare three baselines in our paper. For Tune-A-Video, to ensure a fair comparison, we use the pre-trained weight of Stable Diffusion-v1.5\footnote{https://github.com/CompVis/stable-diffusion} (same as our model) to initialize the UNet and we fine-tune the model with an image resolution of $256 \times 256$ on the training sets of Something Something-V2 (SSv2) and Bridgedata for 200k training steps. For MCVD, we train the model with an image resolution of $256 \times 256$ for 300k training steps. For TATS, we fine-tune the pre-trained UCF-101 model with an image resolution of $128 \times 128$ on the training sets of SSv2 and Bridgedata for 300k training steps.



\subsection{Evaluation Details and Results of UCF-101}~\label{appendix:sec:ucf101}
Most prior text-conditioned video generation methods~\cite{hong2023cogvideo,vdm,makeavideo,magicvideo} evaluate their performance on the UCF-101~\cite{ucf101} benchmark. However, since our proposed method, Seer, is designed for text-conditioned video prediction (TVP) on task-level video datasets, the UCF-101 benchmark, which evaluates class-conditioned video prediction on random short-horizon video clips, is not an ideal evaluation benchmark for TVP. Nonetheless, in order to fairly compare these baselines, we still evaluate the class-conditioned video prediction performance of Seer on UCF-101. 

\paragraph{Settings}Specifically, we fine-tune our model with a video resolution of $16\times256\times256$ on UCF-101. Following the evaluation protocols of ~\cite{hong2023cogvideo}, Seer predicts the videos conditioned on 5 reference frames during fine-tuning and inference stage. We report FVD and Inception score (IS) metrics on the UCF-101 dataset~\cite{ucf101}. The IS is calculated by a C3D model\cite{c3d} that is pre-trained on the Sports-1M dataset~\cite{sports} and fine-tuned on UCF101. We follow the evaluation code of TGAN-v2~\cite{tganv2} to calculate IS metric. Following ~\cite{hong2023cogvideo,vdm,makeavideo}, we evaluate the FVD metric with 2,048 samples and IS metric with 100k samples in the validation set of UCF-101.

\paragraph{Results} Table~\ref{table:tvp:ucf} presents the class-conditioned video prediction results on UCF-101, demonstrating that Seer outperforms CogVideo~\cite{hong2023cogvideo} and MagicVideo~\cite{magicvideo}, but falls short of Make-A-Video~\cite{makeavideo}. Make-A-Video employs unlabelled video pre-training on temporal layers and achieves the best performance among all other methods. While Make-A-Video shows superior performance on FVD and IS, Seer has the potential to further improve its generation performance by addressing the following two limitations. First, Seer has not been pre-trained on video data. Second, Seer obtains latent vectors via a pre-trained 2D VAE, which has not been fine-tuned on UCF-101 and limits the video generation quality of Seer (with 259.4 FVD and 68.16 IS reconstruction quality). However, as we focus on the text-conditioned video prediction task, addressing the above limitations on UCF-101 is out of the scope of this paper.

\begin{table*}
\begin{threeparttable}
\centering\small
\tablestyle{2pt}{1.1}
\setlength{\tabcolsep}{5pt}
\caption{\textbf{ Class-conditioned video prediction performance on UCF-101} we evaluate the Seer on the UCF-101 with 16-frames-long videos. Ex.data indicates that the model has been pre-trained or fine-tuned on extra datasets.
}
\label{table:tvp:ucf}
\begin{tabular}{cccc|cc}
\specialrule{.1em}{.05em}{.05em} 
 Method & Ex.data & Cond. & Resolution & FVD$\downarrow$ & IS$\uparrow$\\
 \hline
MoCoGAN-HD~\cite{mocogan} & No & Class.  & $256\times 256$ & 700\tiny{$\pm$24} & 33.95\tiny{$\pm$0.25}\\
VideoGPT~\cite{videogpt} & No & No  & $128\times 128$ & - & 24.69\tiny{$\pm$0.30}\\
RaMViD~\cite{ramvid} & No & No  & $128\times 128$ & - & 21.71\tiny{$\pm$0.21}\\
StyleGAN-V~\cite{styleganv} & No & No  & $128\times 128$ & - & 23.94\tiny{$\pm$0.73}\\
DIGAN~\cite{digan} & No & No  & & 577\tiny{$\pm$22} & 32.70\tiny{$\pm$0.35}\\
TGANv2~\cite{tganv2} & No & Class.  & $128\times 128$ & 1431.0 & 26.60\tiny{$\pm$0.47}\\
VDM~\cite{vdm} & No & No  & $64\times 64$ & - & 57.80\tiny{$\pm$1.3}\\
TATS-base~\cite{tats} & No & Class.  & $128\times 128$ & 278\tiny{$\pm$11} & 79.28\tiny{$\pm$0.38}\\
MCVD~\cite{mcvd} & No & No  & $64\times 64$ & 1143.0 & -\\
LVDM~\cite{lvdm} & No & No  & $256\times 256$ & 372\tiny{$\pm$11} & 27\tiny{$\pm$1}\\
MAGVIT-B~\cite{magvit} & No & Class.  & $128\times 128$ & 159\tiny{$\pm$2} & 83.55\tiny{$\pm$0.14}\\
 \hline
CogVideo~\cite{hong2023cogvideo} & txt-img \& txt-video & Class.  & $160\times 160$ & 626 & 50.46\\
Make-A-Video~\cite{makeavideo} & txt-img \& video & Class.  & $256\times 256$ & 81.25 & 82.55\\
MagicVideo~\cite{magicvideo} & txt-img \& txt-video & Class.  & & 699 & -\\
\textbf{Seer(Ours)} & txt-img & Class. & $256\times 256$ & 287.8 & 57.74\\
\specialrule{.1em}{.05em}{.05em} 
\textbf{pre-trained VAE}\tnote{*} & - & - & $256\times 256$ & 259.4 & 68.16\\
\specialrule{.1em}{.05em}{.05em} 
\end{tabular}
\begin{tablenotes}\footnotesize
    \item [*] we evaluate the reconstruction quality of pre-trained 2D VAE in this table, the pre-trained 2D VAE is initialized with the pre-trained weight from Stable Diffusion-v1.5 without extra fine-tuning.
\end{tablenotes}
\end{threeparttable}
\end{table*}



\subsection{Evaluation Results of Sampling Steps}
To further investigate the generation effects of sampling steps during evaluation, we conduct a comparison between Seer and Tune-A-Video. We apply a series of DDIM sampling steps (10, 20, 30, 40, 50 DDIM steps), as shown in Figure~\ref{fig:ddimstep}. Seer consistently outperformed Tune-A-Video in terms of both FVD and KVD, with improvements observed from 20 DDIM steps to 50 DDIM steps. Particularly noteworthy is Seer's advantage in video quality (280.7 FVD and 0.73 KVD) compared to Tune-A-Video (419.3 FVD and 1.5 KVD) when using only 10 DDIM steps, demonstrating Seer's ability to sample high-fidelity videos efficiently with minimal denoising steps.
\begin{figure}
\centering
\includegraphics[width=1.0\linewidth]{fig_appendix/ddim_curve.pdf}
\caption{Evaluation results of Seer and Tune-A-Video with DDIM sampling steps ranging from 10 to 50 on the Something-Something V2 dataset.}
\vspace{-8pt}
\label{fig:ddimstep}
\end{figure}



\subsection{Additional Ablation Results}~\label{appendix:sec:fstext}
\paragraph{FSText layer depth}In this section, we additionally investigate the impact of FSText Decomposer's layer depth in Table~\ref{table:ablation:layer}. Our default setting (8-layer FSText Decomposer) outperforms shallower models (2-layer and 4-layer) in terms of FVD. Though the 4-layer model shows a marginal advantage over the 8-layer model in terms of KVD, our experiments indicate that the 8-layer FSText Decomposer shows a remarkable advantage on FVD metrics and exhibits robustness in text-video alignment. Therefore, we adopt the 8-layer FSText Decomposer as the default setting for Seer.

\begin{table}
\centering\small
\tablestyle{2pt}{1.0}
\setlength{\tabcolsep}{5pt}
\caption{\textbf{Layer depth in FSText Decomposer}.}
\label{table:ablation:layer}
\vspace*{-3mm}
\begin{tabular}{c|cc}
num. layers.& FVD$\downarrow$ & KVD$\downarrow$\\
 \hline
 2 & 238.6 & 0.51\\
 4 & 229.7 & 0.23\\
8(Ours) &200.1 & 0.30\\
\end{tabular}
\end{table}

\paragraph{Qualitative results of fine-tuning ablation} We conduct a qualitative analysis of various fine-tune settings. We provide additional visualizations of Fine-tune Setting ablation in Section 5.5 of the main paper. Figure~\ref{fig:ablate:finetune} shows the results of different settings. Among these settings, our default setting \textit{``temp+FSText"} stands out as it preserves a higher-level temporal consistency in video prediction starting from reference frames and also delivers superior text-based video motion compared to the other fine-tune settings. 
\begin{figure}
\centering
\includegraphics[width=1.0\linewidth]{fig_appendix/ablation_visualization.pdf}
\caption{Additional qualitative results of fine-tuning ablation. \textit{“temp+FSText.”} is our default setting.}
\vspace{-8pt}
\label{fig:ablate:finetune}
\end{figure}


\begin{table}
\centering\small
\tablestyle{2pt}{1.1}
\caption{Hyperparameters and details of Fine Tuning/Inference}
\label{table:hyperparam:finetune}
\begin{tabular}{c|cc}
\hline
param. & value\\
\hline
optim. & AdamW\\
Adam-$\beta_1$ &  0.9\\
Adam-$\beta_2$ &  0.99\\
Adam-$\epsilon$ &  $1e^{-8}$\\
weight decay &  $1e^{-2}$\\
lr &  $1.28e^{-5}$\\
end lr & 0.0\\
lr sche. & cosine\\
noise sche. & cosine\\
train batch size& 1/GPU\\
grad. acc.& 2\\
warmup steps& 10k\\
resolution& $256 \times 256$\\
train. steps & 200k\\
train. hardware & 4 RTX 3090\\
val. batch size& 2/GPU\\
sampler& DDIM\\
sampling steps & 30\\
guidance scale & 7.5\\
\hline
\end{tabular}
\end{table}

\begin{table}
\centering\small
\tablestyle{2pt}{1.1}
\caption{Hyperparameters of 3D U-Net}
\label{table:3dunet}
\begin{tabular}{c|cc}
\hline
hyperparam. & value\\
\hline
input/output channels &  4\\
Base channels & 320\\
Channel multipliers&  1,2,4,4\\
3D Downsample blocks &  4\\
3D Upsample blocks &  4\\
Number of layers (per block) &  2\\
\hline
Modules of layer & 3D ResnetBlock\\
 & Spatial-cross Atten.\\
 & ATS Atten.\\
 & Down./Up. 3D ResnetBlock\\
\hline
Dimension of atten. heads &  8\\
activation function &  SiLU\\
Dimension of cross-atten. &  768\\
\hline
\end{tabular}
\end{table}

\begin{table}
\centering\small
\tablestyle{2pt}{1.1}
\caption{Hyperparameters of FSText Decomposer}
\label{table:hyperparam:fstext}
\begin{tabular}{c|cc}
\hline
hyperparam. & value\\
\hline
learnable tokens channels &  768\\
output channels &  768\\
Base channels & 768\\
Number of layers &  8\\
\hline
Modules of layer & Seq-cross Atten.\\
 & Feedforward\\
 & Directed temporal Atten.\\
 & Feedforward\\
\hline
Number of atten. heads &  8\\
Dimension of cross-atten. &  768\\
\hline
\end{tabular}
\end{table}




\section{Implementation Details}~\label{appendix:sec:impl}

\subsection{Fine-tuning and Sampling}\label{sec:finetuneparam}
 In this section, we list the hyperparameters, fine-tuning details, sampling details, and hardware information of our model in Table~\ref{table:hyperparam:finetune}.
 
\subsection{Architecture information}\label{sec:arch}
In this section, we list the hyperparameters of 3D U-Net in Table~\ref{table:3dunet} and hyperparameters of FSText Decomposer in Table~\ref{table:hyperparam:fstext}.



\section{Visualization} 

\subsection{Additional qualitative results} 
We provide additional visualization on Something-Something v2 (SSv2) of our text-conditioned video prediction in Figure~\ref{fig:ssv2pred}, and text-conditioned video prediction/manipulation results in Figure~\ref{fig:ssv2mani}. Additionally, we provide the visualization on BridgeData of text-conditioned video prediction in Figure~\ref{fig:bridgepred} and text-conditioned video prediction/manipulation in Figure~\ref{fig:bridgemani}.
\section{Human Evaluation Details}~\label{appendix:sec:humaneval} 
To evaluate the quality of video predictions according to human preferences, we conducted a human evaluation with 99 video clips on the validation set of the Something-Something V2 dataset (SSv2), the evaluation process involved 54 anonymous evaluators. To eliminate biases towards specific baselines, we randomly selected 20 questions for each evaluator. Each single-choice question consisted of a ground-truth video as a reference, a manually modified text instruction, and two video prediction results generated by Seer and another baseline method. The evaluators were required to choose the video clip that is more consistent with the text instruction and has higher fidelity from the two options.
To ensure the clarity of the questions, we provided an example to explain the options in each questionnaire. Moreover, we recommended that evaluators prioritize video predictions with strong text-based motions as their first preference and the fidelity of the generated video as their second preference. For reference, Figure~\ref{fig:humanevalexp} provides a screenshot of an example questionnaire.

In total, we collected 342 responses for the Seer vs. TATS comparison, 363 responses for the Seer vs. Tune-A-Video comparison, and 357 responses for the Seer vs. MCVD comparison. And the results in the main paper Figure 7 are calculated based on the collected questionnaires.
\clearpage
\begin{figure}
\centering
\includegraphics[width=1.0\linewidth]{fig_appendix/sth_predict.pdf}
\caption{Text-conditioned video prediction of Seer on SSv2.}
\vspace{-8pt}
\label{fig:ssv2pred}
\end{figure}
\begin{figure}
\centering
\includegraphics[width=1.0\linewidth]{fig_appendix/sth_manipulate.pdf}
\caption{Text-conditioned video prediction/manipulation of Seer on SSv2, where ``pred." refers to prediction, ``mani." refers to manipulation.}
\vspace{-8pt}
\label{fig:ssv2mani}
\end{figure}
\begin{figure*}
\centering
\includegraphics[width=0.9\linewidth]{fig_appendix/bridge_pred.pdf}
\caption{Text-conditioned video prediction of Seer on BridgeData.}
\vspace{-8pt}
\label{fig:bridgepred}
\end{figure*}
\begin{figure*}
\centering
\includegraphics[width=1.0\linewidth]{fig_appendix/bridge_manipulate.pdf}
\caption{Text-conditioned video prediction/manipulation of Seer on BridgeData, where ``pred." refers to prediction, ``mani." refers to manipulation.}
\vspace{-8pt}
\label{fig:bridgemani}
\end{figure*}


\begin{figure}
\centering
\includegraphics[width=1.0\linewidth]{fig_appendix/screenshot.PNG}
\caption{Screenshot of a questionnaire example shown to human evaluators.}
\vspace{-8pt}
\label{fig:humanevalexp}
\end{figure}

%\bibliographystyle{JHEP}
%\bibliography{demo-firstresults}  
\pdfoutput=1 
\RequirePackage{fix-cm}
\documentclass[twocolumn,epjc3]{svjour3}  
\RequirePackage{graphicx}
\usepackage{tikz}
\RequirePackage{mathptmx}
\usepackage[english]{babel}
\usepackage{amsmath}
\usepackage{mathtools}
\usepackage[separate-uncertainty=true]{siunitx}
\DeclareSIUnit\bar{bar}
\sisetup{range-phrase = -, range-units=single}
\usepackage{comment}
\usepackage{tabularx}
\usepackage[version=4]{mhchem}
\usepackage[font={small},labelfont=bf,labelsep=quad]{caption}
\usepackage[font={normal}]{subcaption}
\usepackage{cite}
\usepackage{wasysym}
\usepackage{textcomp}
\usepackage{physics}
\usepackage{csquotes}
\RequirePackage[colorlinks,citecolor=blue,urlcolor=blue,linkcolor=blue]{hyperref}

\defineshorthand{"~}{\babelhyphen{nobreak}}
\useshorthands{"}
\newcommand{\LN}{LN$_2$}
\DeclareSIUnit[number-unit-product = {}]{\inch}{"}
%\sisetup{}
\sisetup{per-mode = power}
\sisetup{exponent-product = \cdot}
\sisetup{group-separator = {\,}}
\sisetup{group-minimum-digits = 3}
\DeclareSIUnit\litre{L}
\DeclareSIUnit\liter{L}

% \usepackage[mathlines]{lineno}
% \let\oldequation\equation
% \let\oldendequation\endequation

% \renewenvironment{equation}
%   {\linenomathNonumbers\oldequation}
%   {\oldendequation\endlinenomath}
% \linenumbers

\journalname{Eur. Phys. J. C}
\begin{document}
\sloppy

\title{Electron transport measurements in liquid xenon with Xenoscope, a large-scale DARWIN demonstrator}

\author{L.~Baudis
        \and
        Y.~Biondi\thanksref{e1}\thanksref{e4}
        \and
        A.~Bismark
        \and
        A.~P.~Cimental Ch\'avez
        \and
        J.~J.~Cuenca-Garc\'ia
        \and
        J.~Franchi
        \and
        M.~Galloway
        \and
        F.~Girard\thanksref{e2}
       \and
       R.~Peres\thanksref{e3}
       \and
       D.~Ram\'irez~Garc\'ia
       \and
       P.~Sanchez-Lucas\thanksref{e5}
       \and
       K.~Thieme\thanksref{e6}
       \and
       C.~Wittweg
}
\thankstext{e1}{e-mail: \url{yanina.biondi@physik.uzh.ch}}
\thankstext{e2}{e-mail: \url{frederic.girard@physik.uzh.ch}}
\thankstext{e3}{e-mail: \url{ricardo.peres@physik.uzh.ch}} 
\thankstext{e4}{Now at Karlsruhe Institute of Technology}
\thankstext{e5}{Now at University of Granada}
\thankstext{e6}{Now at University of Hawai\textquoteleft{i} at M\={a}noa}
\institute{\normalsize{Department of Physics, University of Zurich, Winterthurerstrasse 190, 8057 Zurich, Switzerland}
\label{addr1}}

\date{Received: date / Accepted: date}

\maketitle

\begin{abstract}
There is a compelling physics case for a large, xenon-based underground detector devoted to dark matter and other rare-event searches. A two-phase time projection chamber as inner detector allows for a good energy resolution, a three-dimensional position determination of the interaction site and particle discrimination. To study challenges related to the construction and operation of a \mbox{multi-tonne} scale detector, we have designed and constructed a vertical, full-scale demonstrator for the DARWIN experiment at the University of Zurich. Here we present first results from a several-months run with {\SI{343}{kg}} of xenon and electron drift lifetime and transport measurements with a \SI{53}{cm} tall purity monitor immersed in the cryogenic liquid. After \SI{88}{\day} of continuous purification, the electron lifetime reached a value of $\SI{664(23)}{\micro\second}$. We measured the drift velocity of electrons for electric fields in the range (25--75)\,V/cm, and found values consistent with previous measurements. We also calculated the longitudinal diffusion constant of the electron cloud in the same field range, and compared with previous data, as well as with predictions from an empirical model.
\end{abstract}

\section{introduction}

% 1. importance of TKGs and reasoning on TKGs. 
% 2. low resource languages, main main idea.
% 3. relations and limitations of current works.
% 4. summarize our solutions and contributions.

Temporal Knowledge Graphs (TKGs)~\cite{YAGO,ICEWS18,WIKI,acekg} characterize temporally evolving events, where each event, represented as ({\em subject}, {\em relation}, {\em object}), is associated with temporal information ({\em time}), e.g., ({\em Macron}, {\em reelected}, {\em French president}, {\em 2022}). TKGs has facilitated various knowledge-intensive Web applications with timeliness, such as question answering~\cite{KBQA}, product recommendation~\cite{RippleNet,TKG4Rec,TKG4Rec2,RETE}, and social event forecasting~\cite{KG4Social,DyDiff-VAE,andgan,belief,misinfo,polarization}. 

As new events are continually emerging, modern TKGs are still far from being complete. Conventionally, the TKG construction process relies primarily on information extraction from unstructured corpus~\cite{WIKI,YAGO, EventKG}, which necessitates extensive manual annotations to keep up with changing events. For instance, the recent transition from Trump to Biden as the President of the United States has not been reflected in many TKGs, highlighting the need for timely updates. This spurs research on temporal knowledge graph reasoning to automate evolving events prediction over time~\cite{TA-DistMult,Know-Evolve,Renet,RE-GCN}. Unfortunately, the problem of TKG incompleteness is particularly pronounced in low-resource languages, where it is unable to collect enough corpus and annotations to support robust TKG construction. This results in suboptimal reasoning performance and distinctly unsatisfying accuracy in predicting recent and future events.

% whose performance can degrade significantly in low-resource language TKGs that suffer from severe incompleteness over time. 
% \jingfeng{why don't people  study cross-lingual TKG previously, (i.e. use language alignment to improve TKG). Is it really helpful intuitively to use high resource language to help TKGC? For instance, is it enough to use static langauge-alignment to help KGC, ignoring the temporal information? Are those langauge-alignment changing across time?}



\begin{figure}
    \centering
    \includegraphics[width = 1.0\linewidth]{fig/task.pdf}
    \caption{An illustrative example of cross-lingual reasoning on TKGs. 1) We aim to transfer knowledge from English TKG to Japanese TKG, where the English version provides more complete information; 2) Cross-lingual alignments only cover a small ratio of entities, e.g., Apple Inc; 3) Cross-lingual alignments can be noisy and misleading, e.g., A city called Ventura is linked to new macOS Ventura at $t_2$, introducing noise for reasoning in Japanese.}
    \label{fig:illustration}
    %\vspace{-6mm}
\end{figure}

Inspired by the incompleteness issue facing low-resource languages in constructing TKGs, we introduce a novel task named Cross-Lingual Temporal Knowledge Graph Reasoning (as shown in Figure~\ref{fig:illustration}). This task aims to alleviate the reliance on supervision for TKGs in low-resource languages (referred to as the target language) by transferring temporal knowledge from high-resource languages (referred to as the source language)~\footnote{In this paper, for the sake of brevity, we interchangeably use the terms high-resource/low-resource and source/target.}. In contrast, all the existing efforts are either limited to reasoning in monolingual TKGs (usually high-resource languages, e.g., English)~\cite{TA-DistMult,Know-Evolve,Renet,RE-GCN}, or multilingual static KGs~\cite{KEnS,AlignKGC,SS-AGA}. To the best of our knowledge, cross-lingual TKG reasoning that transfers temporal knowledge between TKGs has not been investigated. 

%Motivated by this, we study a new task named {\em cross-lingual temporal knowledge graph reasoning} as shown in Figure~\ref{fig:illustration}, to alleviate the heavy dependence on supervision for any resource-poor language TKGs by distilling the temporal knowledge from resource-rich ones. Differently, all the existing efforts are either limited to reasoning in monolingual (usually high-resource languages, e.g., English) temporal KGs~\cite{TA-DistMult,Know-Evolve,Renet,RE-GCN}, or multilingual static KG~\cite{KEnS,AlignKGC,SS-AGA}, but neglecting the reasoning in a both temporal and cross-lingual manner that highly requires capturing time-evolving patterns and language discrepancy. To the best of our knowledge, this problem, regarding how to transfer cross-lingual knowledge between TKGs, has still not been formally investigated. 

% Unlike conventional TKG reasoning, 
The fulfillment of this task poses tremendous challenges in two aspects: 1) \textbf{Scarcity of cross-lingual alignment}: as the informative bridge of two separate TKGs, cross-lingual alignment is imperative for cross-lingual knowledge transfer~\cite{AlignKGC,KEnS,SS-AGA}. However, obtaining alignments between languages is a time-consuming and resource-intensive process that heavily relies on human annotations. The transfer of knowledge through a limited number of alignments is often insufficient to fully enhance the TKG in the target language. 2) \textbf{Temporal knowledge discrepancy}: the information associated with two aligned entities is not necessarily identical, especially with regards to temporal patterns. Utilizing a rough approach to equate the aligned entities at all times can result in the transfer of misleading knowledge and negatively impact performance. This becomes more pronounced when the alignments are noisy and unreliable. For example, at the time step $t_2$, a new event about operating system ``{\it Ventura}'' from Apple company occurs in the source English TKG, and meanwhile there is a noisy aligned entity ``{\it Ventura city}'' in the target Japanese TKG. Directly pulling those two entities at this point, can inevitably introduce  noise and fail to predict a set of related events in the target TKG. Therefore, it is crucial to dynamically regulate the alignment strength of each local graph structure over time in order to maximize the effectiveness of cross-lingual knowledge distillation.

% Pulling those entities together cannot augment information in target languages. Small alignment strength is beneficial in the unreliable alignment cases, otherwise the misleading knowledge transferring can even hurt the performance.

% Moreover, in a case that the alignments are not fully reliable, directly pulling the two aligned entities together 


% optimally dynamic alignment strength
% {\em Optimal alignment strength to maximize the benefits of knowledge distillation is difficult to obtain, especially in the temporal manner.} 
% In practical, although the aligned entities can share similar information, they may still differ in other perspectives, including but not limited to frequency, interactions, and temporal patterns. How to adjust the alignment strength (i.e., the distance constrains of the aligned entities in the uni-space) accordingly for different entities at different time is unclear. \zheng{Ruijie TODO: add Ventura case}Moreover, in a case that the alignments are not fully reliable, directly pulling the two aligned entities together can even hurt the performance.



% scarcity of hinders the efficient
% knowledge transfer across languages. 
% {\em Transferring knowledge through a small set of alignments is hard to augment information for all entities.} 

% Aligning the same entities across languages rely heavily on manual labeling or rule-based inference~\cite{EA1,EA2,EA3,selfKG}, which is too time-consuming and impractical to obtain the alignments covering most of the entities in target language. 

% In this paper, we study how to boost the TKG reasoning performance in low-resource languages by explicitly increasing the completeness of those TKGs in history. Instead of improving the underlying information extraction techniques in low-data regime, we propose a new task called {\em Cross-lingual Temporal Knowledge Graph Reasoning}, motivated by the facts that there exists common or complementary knowledge shared by the TKGs in different languages under similar topics. The new task aims to facilitate TKG reasoning in low-resource languages (target languages) by distilling knowledge from a corresponding TKG in high-resource language (source language)  through a small set of entity alignments as bridges~\footnote{In this paper, we interchangeably use the terminology high-resource/low-resource and source/target for briety.}. Figure~\ref{fig:illustration} provides an illustrative example of the proposed task.


% Unfortunately, recent breakthroughs in temporal knowledge graph reasoning model~\cite{TA-DistMult,Know-Evolve,Renet,RE-GCN} highly rely on the completeness of the TKGs, especially for the most recent events. 

% However, the completeness of TKGs varies a lot across different languages, even under similar topics. Conventionally, the TKG construction process relies primarily on information extraction techniques built on the unstructured corpus~\cite{WIKI,YAGO, EventKG}. Therefore, the amount of corpus and human annotations in different languages significantly influence the quality of the corresponding TKGs . 
% Therefore, automatically completing/updating TKGs has been attracting enormous interests in recently years, which aims to predict recent/future events on TKGs based on historical events~\cite{TA-DistMult,Know-Evolve,Renet,RE-GCN}, namely temporal knowledge graph reasoning~\footnote{Broadly speaking, TKG reasoning includes interpolation to predict historical events and extrapolation to predict future events. In this paper, we refer to extrapolation task as TKG reasoning, since it is more vital for time-sensitive downstream tasks.}.


% For languages with large-scale and carefully labeled corpus (we refer to as high-resource languages, e.g., English), the constructed TKGs are more comprehensive than TKGs in other languages that lack the high-quality corpus (we refer to as low-resource languages, e.g., Spanish, Slovene, Danish, etc). Such completeness discrepancy leads to distinctly uneven TKG reasoning performances in different languages, which in turn affects the quality of service of the downstream applications. 


% Compared with the traditional TKG reasoning task, the new task imposes non-trivial challenges. An intuitive solution is to construct a unified graph including two TKGs in both source and target languages, and the knowledge distillation can be fulfilled by pulling the aligned entities from two languages close to each other in the uni-space~\cite{AlignKGC,KEnS}. However, there are still two challenges to be addressed. 

% \zheng{Ruijie TODO, Place this part to related works.}
% Existing works in related areas fail to address the aforementioned challenges. Monolingual reasoning methods on static/temporal knowledge graphs~\cite{TransE,TranR,ComplEX,RotatE,TA-DistMult,Know-Evolve,Renet,RE-GCN} is incapable of the desired knowledge transferring due to the insufficient alignment modeling. Although they can be extended on the cross-lingual scenario by viewing the alignments as a new relation on the merged TKGs, the limited amount of alignments prevent them from augmenting information for most of the entities. Entity alignment methods on KGs~\cite{EA1,EA2,EA3,EA4,EA5,selfKG} can automatically enlarge the alignments by  predicting the correspondence between the two TGs. But most of them, if not all, require the relatively even completeness of two TGs to capture the structural similarities, which can not be satisfied in our case, as target TKGs are far from complete. Some recent works start to study the multilingual TK reasoning on static graphs~\cite{AlignKGC,KEnS,SS-AGA}, which similarly aim to extract knowledge from several source KGs to boost the reasoning performance in the target KG, while they still require a sufficient amount of cross-lingual alignments and totally ignore the temporal perspective in our task.

% to facilitate temporal knowledge graph reasoning in low-resource languages. 
% increase the TKG connection and target TKG capacity
% In light of the mutual benefits, we iteratively generate pseudo alignment pairs and pseudo temporal events to address the co-existing scarcity issue in both cross-lingual alignment and target TKGs. 


In this paper, we propose a novel Mutually-paced Knowledge Distillation (\model) framework, where a teacher network learns more enriched temporal knowledge and reasoning skills from the source TKG to facilitate the learning of a student network in the low-data target one. The knowledge transfer is enabled via an alignment module, which estimates entity correspondence across languages based on temporal patterns. Firstly, to alleviate the limited language alignments (\textbf{Challenge \#1}), such a knowledge distillation process is mutually paced over time. This means, on one hand, we encourage the mutually interactive learning between the teacher and student. Concretely, the alignment module between the teacher and the student learns to generate pseudo alignment between TKGs to maximally expand the upper bound of knowledge transfer. And subsequently, it empowers the student to encode more informative knowledge in target TKG, which can in turn boost the alignment module to explore more reasonable alignments as the bridge across TKGs. One the other hand, inspired by self-paced learning~\cite{spl-1,spl-2}, we make the generations as a progressively easy-to-hard process over time. We start from generating reliable pseudo data with high confidence. As time goes by, we then gradually increase the generation amount by relieving the restriction over time. Secondly, to inhibit the temporal knowledge mismatch (\textbf{Challenge \#2}), the attention module can estimate the graph alignment strength distribution over time. This is achieved by a temporal cross-lingual attention in terms of the local graph structure and temporal-evolving patterns of aligned entities. As such, it can dynamically control the negative effect and suppress noise  propagation from the source TKG. Moreover, we provide a theoretical convergence guarantee for the training objective on both initial ground-truth data and pseudo data. To evaluate \model, we conduct extensive experiments of 12 cross-lingual TKG transfer tasks in multilingual EventKG dataset~\cite{EventKG}. Our empirical results show that the \model method outperforms state-of-the-art baselines in both with and without alignment noise settings, where only $20\%$ of temporal events in the target KG and $10\%$ of cross-lingual alignments are preserved.

% To validate the effectiveness of \model, we conduct extensive experiments of 12 cross-lingual TKG transfer tasks in multilingual EventKG benchmark dataset~\cite{EventKG} . Our experimental results empirically demonstrate the superiority of the \model method over state-of-the-art baselines, ranging from static KG embedding~\cite{TransE,TransR,DistMult,RotatE}, temporal KG reasoning~\cite{TA-DistMult,Renet,RE-GCN} to multilingual KG completion~\cite{KEnS,AlignKGC,SS-AGA}, in both with and without alignment noise settings. We further conduct comprehensive ablation and hyperparameter studies to validate the effectiveness of each design choices. Moreover, we provide theoretical analysis of convergence guarantee for the training objective on both initial groundtruth data and pseudo generative data.



To sum up, our contributions are three-fold:

\begin{itemize}[leftmargin = 15pt]
    \item \textbf{Problem formulation}: We propose the cross-lingual temporal knowledge graph reasoning task, to boost the temporal reasoning performance in target TKG by transferring knowledge from source TKG;
    \item \textbf{Novel framework}: We propose a novel \model framework, which enables the mutually-paced learning between the teacher and student networks, to promote both pseudo alignments and knowledge transfer reliability. Besides, \model involves a dynamic alignment estimation across TKGs that inhibits the influence of temporal knowledge discrepancy.
    \item \textbf{Extensive evaluations}: Empirically, extensive experiments on 12 cross-lingual TKG transfer tasks in multilingual EventKG benchmark dataset demonstrate the effectiveness of \model.
\end{itemize}
% pseudo data generation technique to progressively enhance the training data. The generated pseudo alignments can help the training of the representation modules by the knowledge distillation, and in turn adding pseudo events in the target TKG can improves alignment module by providing high-quality representations. 




% interactively
% TKGs in a source language and a target language are represented by a teacher representation module and a student one into a uni-space, respectively. 
% The knowledge distillation is enabled by a cross-lingual alignment module which pulls the aligned entities close to each other and push other entities far away. 
% To address the challenge caused by the scarcity of cross-lingual alignment, 



\section{The Xenoscope facility}
\label{sec:experimental_setup}

Xenoscope can house up to \SI{400}{kg} of LXe in a double-walled stainless steel cryostat. The facility, its subsystems, and the outcome from the first commissioning run are described in Ref.~\cite{Baudis:2021ipf}. Xenoscope was first equipped with a purity monitor (section \ref{sec:purity_monitor}) fully submerged in LXe, while the cryostat aspect ratio was chosen to allow for the operation, in the next phase of the project, of a \SI{2.6}{m} two-phase TPC, with the primary goal of demonstrating the drift of electrons in LXe over this distance for the first time. A computer-aided design (CAD) rendering of the cryostat with the purity monitor is shown in figure~\ref{fig:phases}.

The facility includes a gas purification system with a series of filters and a commercial zirconium alloy getter. The LXe is extracted at the top of the liquid column, where the impurity concentration is higher. It is evaporated in the heat exchanger system and circulated through the purification system at a fixed flow. The purified xenon is recondensed in the heat exchanger and reintroduced in the cooling tower, which comprises a pulse tube refrigerator (PTR) connected to a cold head mounted atop the cooling chamber. The xenon is then directed to the bottom of the cryostat. A slow control system built from open-source software oversees and sends alarms on relevant parameters.

Two system upgrades were performed prior to the installation of the purity monitor. First, a pre-cooler was manufactured and installed at the top of the inner cryostat vessel to provide additional peak cooling power, and thus to reduce the system cooldown and xenon liquefaction time during filling by a factor of 4.25. The design of the pre-cooler and details of its commissioning are presented in~\ref{sec:pre-cooler}. Furthermore, a gravity-assisted recuperation and storage system for LXe, Ball of Xenon (BoX), was deployed to allow for the storage of up to \SI{450}{\kilogram} of xenon at room temperature, as well as for recuperation in liquid phase. The latter enhances the speed of the recuperation process by a factor $\sim$ 8 compared to gaseous recuperation to a bottle array via cryogenic pumping. More details of its design and performance are presented in~\ref{sec:BoX}.

\begin{figure}[ht!]
\centering
\includegraphics[width=\columnwidth]{figures/render-PM3.pdf}
\caption[The purity monitor in the Xenoscope cryostat]{The purity monitor in the Xenoscope cryostat. Legend: (1)~top flange; (2)~outer vessel; (3)~inner vessel; (4)~pre-cooler; (5)~purity monitor; (6)~BoX recuperation line; (7)~anode; (8)~anode grid; (9)~field-shaping rings and resistor chain; (10)~support pillars; (11)~cathode grid; (12)~cathode disk; (13)~photocathode and optical fibre.}
\label{fig:phases}
\end{figure}



\section{The purity monitor and measurements}
\label{sec:purity_monitor}

Common impurities in commercially available xenon consist of parts-per-million (ppm) levels of O$_{2}$, N$_{2}$, H$_{2}$O, CO, as well as organic molecules~\cite{Hasterok:2017ehi}. Additionally, detector and subsystem materials introduce impurities by outgassing. The purification of xenon prevents electron losses via their attachment to electronegative impurities and allows to achieve high light and charge yields.

Most purity monitors measure the charge deficit of an initially known population of electrons after their drift through the liquid. By comparing the number of electrons before, $N_0$, and after the drift, $N(t_\mathrm{d})$, an indirect measurement of the impurity concentration in LXe can be achieved. The deficit can be modelled as a decaying exponential:
%
\begin{align}    
    N(t_{\text{d}}) = N_0~\mathrm{e}^{-t_{\text{d}}/\tau}~,
    \label{eq:electron_loss}
\end{align}
where $\tau$ is the electron drift lifetime. It relates to the concentration of electronegative impurities as:
%
\begin{align}
    \tau = \frac{1}{\sum_{i} k_{i} n_{i}}~,
    \label{eq:attachement}
\end{align}
\noindent where $k_i$ is the attachment rate specific to the impurity type in units of volume per mol per time, usually given in $\si{\mathrm{\liter/(\mol}\cdot\mathrm{\second})}$, $n_i$ is the impurity concentration given in \si{\mol/\liter}, and the sum extends over the different electronegative species $i$ in the LXe. The attachment rate coefficient depends on the electric field strength.

A schematic of the working principle of the purity monitor is shown in figure~\ref{fig:purity_monitor_concept}, left. An optical fibre transmits the light from a xenon flash lamp to the centre of a photocathode. The incident photons produce electrons via the photoelectric effect. The electrons are drifted via extraction, drift, and collection electric fields, generated by four biased electrodes. The first drift region (1) is located between the cathode (with the photocathode in the centre) and the cathode grid; the second region (2) extends up to the anode grid; the third region (3) extends from the anode grid to the anode. The charges induce a current signal in the cathode as they drift towards the cathode grid. The screening grids prevent current induction in the cathode and anode when the electrons are drifting along the second region. Once the electrons reach the third drift region, a second signal is generated at the anode, until the electrons are fully collected. Two electronic circuits amplify and convert the induced currents to voltage signals. 

The data is acquired and digitised, triggered by the pulse generator which also starts the discharge in the xenon flash lamp, with a window of \SI{100}{\micro \second} for the anode and cathode waveforms. Once digitised, the voltage signals are integrated to obtain the charges, i.e., the number of extracted and surviving electrons. With the induced charges and the time between the two signals, which corresponds to the drift time for the applied electric field, the electron drift lifetime can be inferred by solving numerically the equation:

\begin{align}
\frac{Q_A}{Q_C} = \frac{t_1}{t_3} \mathrm{e}^{-(t_1+t_2+t_3)/\tau} \frac{(\mathrm{e}^{t_3/\tau} - 1)}{(\mathrm{e}^{-t_1/\tau} - 1)}~.
\label{eq:lifetime}
\end{align}

\noindent Here, $Q_A$ and $Q_C$ are the charges from the integrated signals measured in the anode and cathode, respectively, $t_{1}$ is the rise time of the first signal, $t_{2}$ is the time between the minimum of the signal in the cathode and the rise time of the signal in the anode, with $t_{3}$ the time from $t_{2}$ up to the maximum of the anode signal. Given the motion of the charges, the signal in the cathode has negative polarity, while in the anode the polarity is positive. 
Figure~\ref{fig:signals_inLXe} shows an example of signals acquired in LXe from the cathode and anode at \SI{40}{slpm} along with the three drift times.

The design of the Xenoscope purity monitor is described in detail in Refs.~\cite{Baudis:2021ipf,Biondi:2022T}, and the assembled module is shown in figure~\ref{fig:purity_monitor_concept}, right. It features a field cage built with high conductivity, oxygen-free copper rings, supported by six polyamide-imide pillars. The rings are connected by a resistor chain of $\SI{5}{G\Omega}$ impedance each, and enclose a cylindrical drift region of \SI{15}{\cm}\,\diameter\,$\times$\,\SI{53.1}{\cm}. The cathode and anode grids consist of hexagonally-patterned, etched stainless steel meshes with high optical transparency ($\sim 93\%$), while the cathode and anode are solid stainless steel disks.


\begin{figure}[h!]
\centering
\includegraphics[width=0.24\textwidth]{figures/pm_schematic.jpg}
\includegraphics[width=0.15\textwidth]{figures/PM-high-res.jpg}
\caption{(Left): Schematic of the purity monitor. A pulse generator triggers a flash from the xenon lamp and the light is transmitted through an optical fibre to the photocathode, where photoelectrons are produced. The electrons are extracted, transported and collected by three electric fields, defined by the cathode ($\mathrm{G_C}$) and cathode grid ($\mathrm{G_1}$), the anode grid ($\mathrm{G_2}$) and the anode ($\mathrm{G_A}$). In the longest region (2), the field shaping rings~(FSR) maintain the uniformity of the drift field $\vec{E_d}$ in the vertical direction. (Right): Assembled purity monitor in Xenoscope.}
\label{fig:purity_monitor_concept}
\end{figure}

\begin{figure}[h!]
\centering
\includegraphics[width=\columnwidth]{figures/signals.pdf}
\caption[Signal acquired at \SI{40}{slpm} xenon recirculation speed.]{Signals acquired at \SI{40}{slpm} xenon recirculation The rise time of the cathode signal (blue) is taken as $t_{1}$, the time interval between the minimum of the cathode signal and the start of the anode signal (red) is taken as $t_{2}$, with half of the charge cloud completely collected at $t_{3}$. These values are later used to calculate the electron drift lifetime.}
\label{fig:signals_inLXe}
\end{figure}
%=============================================================================
\subsection{Optical components and photocathode}

The utilised lamp is a \SI{60}{\watt} xenon flash lamp with a \mbox{built-in} reflective mirror (model number L7685) from \textit{Hamamatsu}~\cite{hamamatsu}. The window is a single sapphire crystal allowing short wavelength light ($\sim$~\SI{190}{\nano\meter}) to reach the output of the lamp, with a spectral emission from \SI{190}{\nano\meter} to \SI{2000}{\nano\meter}. The lamp generates a discharge which excites the gas producing scintillation, with reflective mirrors directing photons from all directions towards the output. The xenon lamp can be triggered internally, or externally via a pulse generator. The intensity of the light emission is adjusted by setting the voltage for the discharge in the lamp between \SI{600}{\volt} and \SI{1}{\kilo\volt}. The selection of the latter maximises the number of produced electrons.

 Measures were adopted to mitigate the electronic noise produced in the signal waveforms by the external trigger: the xenon flash lamp was rehoused in a stray electromagnetic interference box, and galvanic insulation and ferrite filters were added to the trigger line. The box was customised by adding a potentiometer to manually change the voltage of the discharge. A trigger circuit controlled by the pulse generator was added as well. The lamp is placed outside the cryostat, with an optical fibre carrying the light from its output to the surface of the photocathode. A UV grade sapphire lens produced by \textit{Hamamatsu} is placed at the output of the lamp to collimate the light to the optical fibre. The selected fibre is $\SI{600}{\micro\meter}$ in diameter, ultra-high vacuum rated with a polyimide buffer from \textit{LewVac}~\cite{LewVac}. It is resistant to UV solarisation, i.e.~the degradation in the fibre material due to the exposure to light of wavelength lower than \SI{300}{\nano\meter}. The fibre feedthrough, produced by \textit{Thorlabs}~\cite{feedthrough}, consists of a $\SI{600}{\micro m}$ multimode fibre in an SMA connector welded on a CF40 flange. The feedthrough requires the fibres to be terminated, hence these were prepared and polished in-house with a set of 8~lapping sheets, made of aluminium oxide, silicon carbide, and calcined alumina, from grits of $\SI{30}{\micro\meter}$ to $\SI{0.02}{\micro\meter}$. 
 
One of the critical parts of a purity monitor is the photocathode, for it directly impacts the size of the initial signal. It consists of a thin layer of a low work function metal, deposited on a quartz substrate that has low absorption of UV photons~\cite{Valentini2002}. The photocathode was produced in-house using a turbomolecular pumped coater Q150T Plus from \textit{Quorum}~\cite{coater}. The desired thickness of the layer was monitored with a quartz crystal microbalance. Different materials for the thin layer were tested, including gold and silver, and the coater was used to produce photocathodes of \SI{50}{\nano\meter} thickness on a \SI{2}{\mm} thick quartz substrate, with a diameter of \SI{30.00(5)}{\mm}. The deposition of a \SI{5}{\nano\meter} thick layer of titanium was required in the case of gold for adhesion to the substrate. The choice of thickness was based on the effective probe depth of gold layers, and previous works~\cite{Manenti2020}. Additional technical details can be found in Ref.~\cite{Biondi:2022T}.

The photocathodes were tested in a vacuum setup, where the xenon lamp was flashed onto the photocathode material and the induced current was measured. Both gold and silver showed high yields, with gold reaching a stable state in fewer hours of exposure to the UV signal. Silver and gold photocathodes showed an increasing quantum efficiency with time when exposed to light, and this increased yield did not revert back in subsequent tests. The increase in quantum efficiency of the photocathode with UV-light exposure was also observed in Ref.~\cite{Manenti2020}. The gold photocathode was selected due to its stability over time and higher quantum efficiency than the silver photocathode, requiring smaller electric fields to produce a higher charge signal. 

%===============================================================================
\subsection{Current readout and signal processing}

The readout electronics amplify the induced currents from the cathode and anode and are placed inside the cryostat to avoid signal losses along the \SI{9}{\meter} signal cables. The circuits were designed together with the Electronics Workshop at the University of Zurich. The circuit consists of an AC-coupling component, a transimpedance amplifier, and a final voltage amplifier with a $\SI{50}{\ohm}$ impedance termination to match the one from the data acquisition. The transimpedance and voltage amplifiers are implemented with two low-cost operational amplifiers, model AD8066 from \textit{Analog Devices}~\cite{analogdev}. An AC-coupling filter in the circuit board removes high-frequency noise, which enhances the signal quality, and the AC-coupling removes the DC component of the HV applied to the electrodes. The usage of a transimpedance amplifier, in contrast to a charge amplifier, allows for more precise timing and signal spread analyses due to its small resistive-capacitive constant (RC) and short rise time of \SI{0.14}{\micro\second}. However, due to its fast response, a low-pass filter for frequencies below \SI{800}{\kilo\hertz} is applied to the signals to decrease the electronic noise induced by, e.g., the pulse generator that triggers the lamp, two temperature sensors, and the uninterruptible power supply. The preamplifier operates in current mode, as the capacitance discharges rapidly, resulting in an output voltage proportional to the instantaneous current. The frequency response of the readout electronics was benchmarked, with a negligible effect on the signal shape due to the $\SI{100}{MHz}$ bandwidth. 

The performance of the readout electronics was tested in a climate chamber in steps of \SI{10}{\kelvin} from room temperature down to \SI{190}{\kelvin}. The calibration showed a charge amplification of $\SI{0.18}{\femto\coulomb/(\milli\volt\cdot\micro\second)}$, with good thermal stability. Additionally, the RC decay constant of the circuit, which could be a source of systematic error for time measurements, was estimated at $\sim$~\SI{150}{\nano\second} by feeding a \SI{2}{\micro\second} wide square pulse to the circuit.

An oscilloscope, \textit{Teledyne LeCroy} model Waverunner 6104A~\cite{teledyne}, and an analog-to-digital converter from \textit{CAEN}, model v1724~\cite{CAEN}, acquired the waveforms produced by the cathode and anode readout. Each acquisition consisted of the average of \SI{1000}{} waveforms acquired over \SI{16.7}{minutes} to minimise the baseline noise. The signals were then processed by fitting the expected signal shape with a Gaussian distribution. In some waveforms, a noise introduced by external electronic devices could be discerned as part of the background noise, and the fit included a sine function to account for this effect, with an inferred model uncertainty of $5\%$ for the ones where the sine fit to the noise did not converge. The current-equivalent voltage signals in the cathode and anode were integrated to obtain a charge-proportional value. The residuals of the fits were used as weights for the charge values obtained in the averaged data shown in the next section. The uncertainties in charges and times obtained in the fits were propagated to obtain the uncertainty of the electron drift lifetime value. An example of the raw anode signal at \SI{53}{\volt/\cm} drift field in region 2 is shown in figure~\ref{fig:FWHM_signals}, together with the post-processing signal with a low-pass filter. The calculated baseline and Gaussian fit of the signals are also shown.

\begin{figure}[h!]
\centering
\includegraphics[width=0.49\textwidth]{figures/signal_example_fit.pdf}
\caption[Anode signals at \SI{53}{\volt/\cm}]{Anode signal at \SI{53}{\volt/\cm} prior to (blue) and after (orange) the low-pass filter. The calculated baseline (red) and a Gaussian fit of the signal (green) are also shown. The signal is an average over \SI{1000}{} recorded waveforms.}
\label{fig:FWHM_signals}
\end{figure}

\subsection{Measurements}

Once installed in the cryostat, the purity monitor was first operated in vacuum ($\sim \SI{1e-5}{\milli\bar}$). Data was acquired to investigate the signal shape and response in this configuration with negligible charge losses due to residual gas. The measurement additionally provided the delay time of the electronics chain, from the pulse generator for the xenon lamp to the signal amplification and readout of $\SI{18}{\micro\second}$.

After the calibration of the purity monitor in vacuum, gaseous xenon was flushed inside the detector and purified through recirculation in the gas system. The LXe run started with the filling of \SI{343}{\kilogram} of xenon. As the xenon recirculates through the getter, electronegative impurities are removed, and the electron drift lifetime is expected to increase in two steps: an initial exponentially increasing phase where the bulk impurities are rapidly removed, and a second phase where the change is dominated by the materials outgassing, and where the electron drift lifetime slowly increases over time. At different recirculation speeds, the electron drift lifetime reaches increasingly higher values in the second phase.

The recirculation speed was set with flows of \SI{30}{standard\ litres \ per\ minute\ (slpm)}, \SI{35}{slpm} and \SI{40}{slpm}, with the xenon lamp illuminating the photocathode with a frequency of \SI{1}{\hertz}. In the cryostat, the temperature and pressure were maintained around \SI{177.6(1)}{\kelvin} and  \SI{2.05(1)}{\bar}, respectively. Following the commissioning run, the displacement of the GXe compressor was reduced to increase its lifespan. This constrained the maximum purification speed to \SI{40}{slpm}, compared to the \SI{80}{slpm} reported in Ref.~\cite{Baudis:2021ipf}. The initial impurity level in the xenon gas impacts the number of days before a signal can be seen in the purity monitor: the  first waveforms in the cathode and anode were observed after $\sim$26.5 days.

During data taking, the cathode and cathode grid were biased at \SI{-2710}{\volt} and \SI{-2650}{\volt}, respectively. The anode grid was kept at ground while the anode was biased at \SI{500}{\volt}. The values were selected based on COMSOL~\cite{multiphysicscomsol} simulations which yielded nearly $100\%$ extraction efficiency of the electrons produced in the centre of the photocathode. Table~\ref{table:drift_times} shows the summary of the distances, times electric fields for the  extraction (1), drift (2) and collection (3) regions.
\begin{table}[h!]
\centering
\caption[Distance and drift fields]{Electric fields, distances and times $t_{i}$ measured for the three regions in the PM, with voltages \SI{-2710}{\volt}, \SI{-2650}{\volt}, \SI{0}{\volt}, and \SI{500}{\volt} for the cathode, cathode grid, anode grid and anode, respectively, for a purification speed of \SI{40}{slpm}.}
\begin{tabular}{clll} 
\hline 
Drift region $i$ & Distance [mm] & Field [V/cm] & $t_{i}$ $[\SI{}{\micro \second}]$\\
\hline
 1 &  $18 \pm 1$  & $33 \pm 1 $ & $12.8 \pm 0.8$\\
 2 &  $503 \pm 5$ &  $53 \pm 1 $ & $433.5\pm0.7$ \\
 3 &  $10 \pm 1$  & $500 \pm 5$ & $7.6 \pm 0.7$
\label{table:drift_times}
\end{tabular}
\end{table}

Figure~\ref{fig:charge_anode_cathode} shows the anode and cathode signals with their integral, where the integrated signals show a step-like feature after the charges move entirely to the next drift region, or are collected in the anode. The integration corresponds to the total area of the Gaussian fit. The charge measured in the cathode corresponds to $N_{\text{e}^{-}} \cong 10^{6}$ electrons extracted from the photocathode at each pulse.

\begin{figure}[b]
\centering
\includegraphics[width=\columnwidth]{figures/charge_cathode.pdf}
\centering
\includegraphics[width=\columnwidth]{figures/charge_anode.pdf}
\caption[Signal readout at the cathode with the integrated charge signal]{Signal readout (blue) at the cathode (top) and anode (bottom)  with their respective Gaussian fits (orange) and integrated charge signals (red).}
\label{fig:charge_anode_cathode}
\end{figure}




\section{Results and discussion}
\label{sec:results_discussion}

\begin{figure*}[t!]
\centering
\includegraphics[width=2\columnwidth]{figures/Elife_model.pdf}
\caption[Purification flow-dependent electron drift lifetime measured in the Xenoscope.]{Purification flow-dependent electron drift lifetime measured in Xenoscope. The data was averaged in \SI{6}{\hour} time bins. The dashed lines indicate a change in flow, while the dash-dotted lines indicate short-term irregularities in the pressure and flow conditions (see text). The red line shows the best-fit model from equations \ref{eq:e-lifetime-gas} and \ref{eq:e-lifetime-liquid}, while the black points show the residuals.}
\label{fig:electron_lifetime_datasets}
\end{figure*}

 The electron drift lifetime measurement campaign with the purity monitor lasted a total of \SI{88}{days}. The purification was performed at \SI{30}{slpm} for \SI{46.6}{days}, at \SI{35}{slpm} for \SI{20.0}{days}, and at \SI{40}{slpm} for \SI{21.2}{days}. After the electron drift lifetime measurements in LXe, signals for drift fields from \SI{25}{V/cm} to \SI{75}{V/cm} were acquired to study field-dependent electron transport properties, such as the drift velocity and longitudinal electron cloud diffusion.

\subsection{Electron drift lifetime}

Figure~\ref{fig:electron_lifetime_datasets} shows the electron drift lifetime calculated with the charge signals acquired at the cathode and anode over the entire acquisition period. When the recirculation speed changes, the electron drift lifetime drops, most likely due to a change in the height of the liquid level, resulting in the release of trapped impurities in the high-surface tension region at the LXe \enquote{collar} (LXe/GXe/inner vessel interface). A drop in electron drift lifetime was also observed at \SI{59.8}{\day}, as expected, when the GXe compressor was stopped for a period of approximately \SI{15}{minutes} due to a communication error with the slow control software. Again, the change in liquid level most likely resulted in the sudden release of impurities from the collar. Shortly following these events, the electron drift lifetime increased exponentially to return to the outgassing-limited values.

A review of the slow control data allowed for the identification of three irregularities in the pressure and flow conditions, at \SI{38.9}{\day}, \SI{52.6}{\day}, and \SI{80.7}{\day}. The first was a brief moment of excess flow downstream of the GXe compressor with a slight pressure increase of approximately $\SI{10}{\milli bar}$, suggesting the release of trapped gas in the xenon handling system from vibration, or the unlikely development of a micro-leak. The second irregularity was a quick fluctuation in the purification flow, both upstream and downstream of the GXe compressor, resulting in a momentary increase in pressure both downstream of the compressor and in the inner vessel, also leading to a change in the liquid level. The last irregularity was again a marginal increase in the flow downstream of the compressor. In this case, however, the event lasted approximately \SI{3}{\hour}.
After \SI{88.0}{\day} of almost continuous purification, the electron drift lifetime reached a value of $\SI{664(23)}{\micro\second}$. This is consistent with the value reached in other LXe experiments, such as XENON1T~\cite{XENON:2021nad} and LUX~\cite{LUX:2020vbj}, with \SI{660}{\micro\second} and \SI{750}{\micro\second}, respectively. 

While the purity level demonstrated in Xenoscope could be sufficient to drift electrons over \SI{2.6}{\meter} in LXe, it can be further improved by increasing the purification speed. To investigate this, a simple model of the electron drift lifetime, assuming O$_2$-like impurities, was adapted from Refs.~\cite{greeneXENON1TSpinIndependentWIMP2018,Plante:2022khm} and fitted to the electron drift lifetime data shown in figure~\ref{fig:electron_lifetime_datasets}. Two coupled differential equations describe changes in impurity levels $M \dv{I_{}}{t}$ after a time $\dd t$, where $M$ is the mass of xenon, and $I$ is the impurity concentration. The gas and liquid phases (denoted by the subscripts $g$ and $l$, respectively) are evaluated separately to determine the impurity concentration at time $t+\dd t$:
%
\begin{multline}
M_{\mathrm{g}} \dv{I_{\mathrm{g}}}{t}^{(j)} = -F_{\mathrm{g}} \rho I_{\mathrm{g}} + \left(\frac{\Lambda_{\mathrm{g},0}}{1+\frac{t-\Delta t^{(j)}_g}{T_{1/2,g}}} + C_\mathrm{g}\right)\\ + \frac{\epsilon_1 P_{\mathrm{C}} I_{\mathrm{l}}}{h} - \frac{\epsilon_2 P I_{\mathrm{g}}}{h} + M_{\mathrm{g}} \Delta I_{\mathrm{g}}^{(j)} \int \delta\left(t-t^{(j)}\right) \dd t
\label{eq:e-lifetime-gas}
\end{multline}
%
\begin{multline}
M_{\mathrm{l}} \dv{I_{\mathrm{l}}}{t}^{(j)} = -F_{\mathrm{l}} \rho I_{\mathrm{l}} + \left(\frac{\Lambda_{\mathrm{l},0}}{1+\frac{t-\Delta t^{(j)}_l}{T_{1/2,l}}} + C_\mathrm{l}\right)\\- \frac{\epsilon_1 P_{\mathrm{C}} I_{\mathrm{l}}}{h} + \frac{\epsilon_2 P I_{\mathrm{g}}}{h} + M_{\mathrm{l}} \Delta I_{\mathrm{l}}^{(j)} \int \delta\left(t-t^{(j)}\right) \dd t~.
\label{eq:e-lifetime-liquid}
\end{multline}
%
\noindent The index $j = \{0, 1,...,7\}$ indicates the regions between discontinuities, marked by the dashed lines in figure~\ref{fig:electron_lifetime_datasets}. Each equation consists of five terms. The first accounts for the purification rate, with $F$ being the purification flow, $\rho$ the density of xenon at \SI{1}{bar}, \SI{0}{\celsius}, and $I$ the concentration in impurities. The second term accounts for the time-dependent average outgassing rate from detector materials, where $\Lambda_0$ is the outgassing at time $t=0$, $T_{1/2}$ is the decay-time of the outgassing rate and $C$ is a constant outgassing term which brings the system to an equilibrium point when $t \gg T_{1/2}$. After the sudden increase of impurity concentration in the xenon during flow changes, described by the delta function in the fifth term, the outgassing rate modelled by the second terms of the differential equations is expected to revert back to a high value, as impurities can be adsorbed in outgassed materials~\cite{Plante:2022khm}. This is accounted for with the time-offset parameters $\Delta t^{(j)} = \sum_1^j \Delta t ^{(i)}$, which are cumulative since the dataset is fitted as a whole, and $\Delta t^{(0)} = 0$. Finally, the third and fourth terms describe the exchange of impurities between the gas and the liquid phases, and thus the signs are inverted between the two equations. The $\epsilon$ parameters are efficiencies of the exchange process, $P_{\mathrm{C}}$ is the cooling power of the system in the absence of purification, proportional to the evaporation rate of the xenon, $P$ is the cooling power deployed by the cryogenics, proportional to the condensation rate, and $h$ is the latent heat of xenon.

The electron drift lifetime is then calculated by solving the system of differential equations for equation~\ref{eq:attachement} and minimising with the iMinuit (Python) package~\cite{James:1975dr} simultaneously in all $j$-regions, using a least-square method. The best-fit model is displayed as a solid red curve in figure~\ref{fig:electron_lifetime_datasets}. Assuming the same initial conditions obtained from the fit of the model to the electron drift lifetime data, we obtain the purification flow-dependent electron drift lifetime predictions shown in figure~\ref{fig:prediction}.
%
\begin{figure}[h!]
\centering
\includegraphics[width=0.49\textwidth]{figures/prediction.pdf}
\caption{Purification flow-dependent electron drift lifetime prediction. Assuming the same operating conditions as in the measurement campaign, an increased flow (solid lines) could significantly reduce the purification time needed before the electron drift lifetime begins its exponential rise. Furthermore, the addition of a \SI{2}{slpm} gas phase extraction to the purification loop can also improve the electron drift lifetime (dashed lines).}
\label{fig:prediction}
\end{figure}
%

As expected, an increased purification speed would yield longer electron drift lifetimes, attained in a shorter purification time. The addition of a purification flow of the gas phase ($F_\mathrm{g}= \SI{2}{slpm}$) suggests an expected increase in electron drift lifetime of up to \SI{15}{\%}. Therefore, prior to the start of the next phase of Xenoscope, a parallel gas extraction line inspired by the gas purification system reported in Ref.~\cite{Plante:2022khm} was added to the gas handling system with a second flow controller, allowing for the purification of both the liquid and gas phases in the same purification loop. 
 
\subsection{Electron transport}

The purity monitor allows for a dedicated measurement of the arrival time of the electron cloud at the anode. The drift velocity $v_{\mathrm{d}}$ of the cloud given a drift field $E_{\mathrm{d}}$ is:
\begin{equation}
\label{velocity}
v_{\text{d}} =d_{2}/t_{2}.
\end{equation}
Considering that the accuracy of the time measurement between the extrema of the cathode and anode signals is higher, and region 3 has a fast detection, it is convenient to calculate instead:
\begin{equation}
\label{velocity}
v_{\text{d}} = (d_{2} + d_{3}) / (t_{2} + t_{3}).
\end{equation}
This approach introduces an additional term, $(d_{2}t_{3} - d_{3}t_{2})/(t_{2} (t_{2}+t_{3}))$, which induces a negligible $0.1\%$ bias in the final values given the ratios between $t_{3}/t_{2}$ and $d_{3}/d_{2}$. The drift velocity can also be expressed in terms of the drift field $E_{\mathrm{d}}$:
\begin{equation}
v_{\mathrm{d}} = \mu E_{\mathrm{d}}\:,
\end{equation}
where $\mu$ is the electron mobility. The mobility is related to the average time, for a given temperature, density, and energy of the electrons, between elastic collisions with the xenon and electronegative impurities, and potential inelastic collisions with impurities. Thus, the acquisition of waveforms used to derive the drift velocity was performed at a constant electron drift lifetime to avoid systematic uncertainties. Benchmark regimes can be used to visualise these dependencies for electron mobility, such as \enquote{cold electrons} and \enquote{hot electrons}~\cite{Schmidt:1984zz}. In the cold electrons regime, the energies of the electrons are mostly due to the thermal bath in the xenon fluid (around \SI{0.015}{eV} at \SI{177}{\kelvin}~\cite{Boyle:2016wpy}), and they rapidly acquire energy with increasing electric fields, resulting in a linear gain in velocity. For cold electrons, the mobility can be expressed as:
%
\begin{equation}
    \mu=\frac{2}{3}\left(\frac{2}{\pi\, m_e \,k_B \,T}\right)^{\frac{1}{2}} e \,\frac{\lambda}{v}\:,
    \label{eq:cold_electrons}
\end{equation}
where $k_{B}$ is the Boltzmann constant, $T$ is the temperature, $e$ is the charge of the electrons, m$_e$ is their mass, $v$ is the magnitude of the velocity in all directions, and $\lambda$ is the mean free path of the electrons in their collisions with atoms, inversely proportional to the number density and cross section. In contrast to cold electrons, hot electrons have gained most of their energy through acceleration by the drift field, and experience increased energy losses on their drift path due to collisions with xenon atoms, which results in a slower rate of change in their velocity. For this case, the electron mobility can be expressed as:
\begin{equation}
\mu = \frac{4\, e \, \lambda}{3 v  \pi^{1 / 2} m_e^*}\:,
\label{eq:hot_electrons}
\end{equation}
where $m_e^*$ is the effective mass of the electrons in the medium. While these equations are not used in this work to infer properties such as electron mobility, or electron cloud diffusion, they are useful to scrutinise the results of our measured drift velocity and discuss potential systematic effects, discussed later in this section. 

The data for this measurement was acquired at day 89, after the electron drift lifetime entered a region of slow change, and in a time interval of half an hour, to minimise systematic effects. Drift fields of \SI{25}{\volt/\cm} to \SI{75}{\volt/\cm} were scanned in steps of \SI{5}{\volt/\cm}, where the former was the threshold for a discernible signal above noise level in the cathode. The extraction field was changed to maintain the ratio between the extraction and drift field used in the electron drift lifetime data, while the collection field was fixed at \SI{500}{\volt/\centi\meter}. Figure~\ref{fig:drift_time_field} shows the drift velocity at different fields, calculated with equation~\ref{velocity}, with good agreement with previous measurements in LXe~\cite{gushchin1982electron, Albert:2016bhh,Baudis:2017xov,Thieme:2022dze,Jorg:2021hzu,Baur:2022sel}. The prediction from NEST (Noble Element Simulation Technique)~\cite{szydagis_m_2018_1314669}, based on data-driven empirical models, is also shown. The curves for cold and hot electrons are derived by using equations~\ref{eq:cold_electrons} and~\ref{eq:hot_electrons}, respectively, to fit the data from Ref.~\cite{gushchin1982electron}, for it covers both regimes with a high density of points.

\begin{figure}[h!]
\centering
\includegraphics[width=0.49\textwidth]{figures/drift_velocity_withXenoscope.pdf}
\caption{Measured drift velocity with electric field values from \SI{25}{} to \SI{75}{\volt/\cm}, in steps of \SI{5}{\volt/\cm}. The results are compared to literature values from Gushchin (\SI{165}{K})~\cite{gushchin1982electron}, EXO-200 (\SI{167}{\kelvin})~\cite{Albert:2016bhh}, Xurich II (2018, \SI{184}{\kelvin})~\cite{Baudis:2017xov}, Xurich II (2021, \SI{177}{\kelvin})~\cite{Thieme:2022dze}, HeXe (\SI{174}{K})~\cite{Jorg:2021hzu} and XeBRA (\SI{173}{\kelvin})~\cite{Baur:2022sel}. The prediction from NEST v2.3.8~\cite{szydagis_m_2018_1314669} is shown as a solid blue curve.}
\label{fig:drift_time_field}
\end{figure}

The spread of the electron cloud in time was studied by analysing the anode signal at the previously mentioned drift fields. With this purity monitor, only the longitudinal diffusion can be observed, for there is no information on the $x-y$ charge distribution. The standard deviation ($\sigma$) of the Gaussian fit of the signal in the anode is used to calculate the longitudinal diffusion coefficient. For the case of the initial and final distributions of the electron cloud following a Gaussian distribution, the width of the anode signal is related to the longitudinal diffusion~\cite{Albert:2016bhh,Li:2015rqa,Rolandi2008} as:
%
\begin{equation}
D_{L}=\frac{d^{2} \sigma_{L}^{2}}{2 t^{3}}\:,
\label{eq:diffusion1}
\end{equation}
\noindent
where $D_L$ is the diffusion coefficient of the electron population due to the random walk of the electrons in the longitudinal direction, and:
%
\begin{equation}
\label{eq:diffusion}
\sigma_{L}^{2}=\sigma^{2}-\sigma_{0}^{2}\:,
\end{equation}
\noindent
 is the width at one $\sigma$ considering diffusion effects only, which is the value to extract. The widths $\sigma_{0}$ and $\sigma$ belong to the initial signal at the cathode and the final signal measured in the anode, respectively, and $d$ is the drift length ($d_{2}+d_{3}$) at drift time $t$ ($t_{2}+t_{3}$). By rewriting equation~\ref{eq:diffusion}, we obtain the width of the anode signal:
%
\begin{equation}
    \sigma^2 = \frac{D_L 2 t^3}{d^2} + \sigma_0^2\:,
\label{eq:diffusion2}
\end{equation}
The diffusion of the electron cloud is related to the previously introduced electron mobility. At low drift fields, it follows the Einstein-Smoluchowski relation~\cite{einstein,smoluchowski,Schmidt:1984zz}:
\begin{equation}
D_L=\frac{k_{B} T}{e} \mu = \frac{\epsilon_{T}}{e} \mu\:,
\label{eq:diffusion_einstein}
\end{equation}
with thermal energy $\epsilon_{T}$.
The diffusion is thus affected by the cross section of elastic and inelastic interactions with the medium species (xenon and impurities), analogously to the drift velocity. 

 Since the electrons have energies above the thermal bath and are not in equilibrium, a characteristic energy, $\epsilon_k$, is defined as:
\begin{equation}
\epsilon_k = \frac{e D_L}{\mu}\:,
\label{eq:characteristic_energy}
\end{equation}
representing the energy associated with the longitudinal diffusion, where now $D_L$ has a contribution beyond the thermal energy of electrons:
\begin{equation}
D_L= \frac{(\epsilon_T + \epsilon )}{e}\mu\:,
\label{eq:characteristic_energy_longitudinal_diffusion}
\end{equation}
with $\epsilon =  \epsilon_k - \epsilon_{T}$.

In previous studies, it was reported that impurities diffused in the medium, from water vapour to organic molecules, can provide a more effective energy loss mechanism for electrons, with the consequence of higher mobility and decreased diffusion~\cite{Yoshino:1976zz,Pack:1961jag} (see equations~\ref{eq:cold_electrons}  and~\ref{eq:hot_electrons}). This effect could explain the mechanism behind the higher drift velocities in the early measurements of Guschin et. al. at \SI{164}{\kelvin}~\cite{gushchin1982electron}, where no information about xenon purification methods was given.

Additional effects can play a role in the detected spread of the charge distribution in our detector and must be corrected to obtain a longitudinal diffusion coefficient that is independent of energy, electron source or detector response. From the original number of extracted photoelectrons to the measured signal in the anode, the following effects can impact the width:
\begin{itemize}
    \item Duration of the pulse of the lamp, which introduces an initial signal width at one sigma of $\SI{2.4(2)}{\micro\second}$. In this work, the initial signal width was derived from the data acquired in vacuum and in LXe, and the signal in the cathode is deconvolved with the detector and electronics responses. 
    \item The detection response of the screening region~\cite{Shockley:1938itm}. The weighting potential between the solid electrodes and the hexagonal meshes is taken into account in the measured signal to yield the original electron cloud spread in the $z$-direction. The effect of the hexagonal screening meshes was simulated by modelling the 3D geometry of the meshes and detector and performing electrostatic simulations with COMSOL. The method to derive the weighting potential is adopted from Ref.~\cite{gook_application_2012}. The weighting potential is obtained by averaging the potential over different electron drift paths to smooth out local effects. 
    \item Coulomb repulsion between electrons, where each electron is affected by the electric field induced by other electrons, can increase the size of the electron cloud. The Coulomb repulsion calculation follows the empirical approach in Ref.~\cite{Njoya:2019ldm}, which considers the ellipsoid explosion model from Ref.~\cite{PhysRevE.84.056404}, where a set of differential equations is solved to obtain the final width of the signal after the drift. The repulsive forces are stronger immediately after the charge creation and become smaller as the electron cloud spreads along the drift path. After the electrons have drifted and reached the anode grid, an additional width value of 5\% compared to the no repulsion forces case is inferred from the empirical approach, and taken as an uncertainty in the charge distribution after their extraction from the photocathode.
    \item Electron attachment to electronegative impurities which can potentially change the distribution of the electron cloud. This could in principle affect the diffusion of electrons in LXe, and the method to estimate this impact is taken from Ref.~\cite{Li:2015rqa}. The longitudinal diffusion coefficient and drift velocity formulae are expanded to include higher-order terms containing the attachment rate. The effects of electron attachment in the diffusion can be neglected according to $\frac{D_L}{(v_\mathrm{d} \cdot d)} \ll 1$, which is the case for this study.
    \item Readout response times of the pre-amplifiers. The circuit response was estimated when benchmarking the electronics and has a negligible effect on the signal shape.
\end{itemize}

Combining all the effects introduced above, a response function is obtained to deconvolve the observed signal. Table~\ref{systematics} compiles the systematic effects treatment for the calculation of the longitudinal diffusion coefficient. The results of this deconvolution, for each measurement at different drift fields, are shown in figure~\ref{fig:longitudinal_diffusion}, together with literature values~\cite{Njoya:2019ldm,Hogenbirk2018}. The longitudinal diffusion coefficient was measured at relatively low drift fields (i.e.~$<\SI{100}{\volt/\cm}$). By using the values derived for the mobility for cold electrons and hot electrons included in  figure~\ref{fig:drift_time_field}, of $\SI{0.29}{\milli\meter^{2}/(\micro \second \cdot \volt)}$ and $\SI{0.01}{\milli\meter^{2}/(\micro \second \cdot \volt)}$, respectively, together with the thermal energy of electrons, reference coefficients for the diffusion can be obtained from equation~\ref{eq:characteristic_energy}. These are included in figure~\ref{fig:longitudinal_diffusion}. By comparing the experimental values with the benchmark equations, the difference between these is contained in terms of the characteristic energy of electrons and their mobility, given equation~\ref{eq:characteristic_energy_longitudinal_diffusion}.

\begin{table}
\centering
\caption{Systematic effects and impact on the uncertainty for the inferred longitudinal diffusion coefficient $D_L$.}
\label{systematics}
% @{\extracolsep{\fill}}
\begin{tabularx}{\columnwidth}{lX}
\hline\noalign{\smallskip}
Systematic effect & Treatment and uncertainty \\
\noalign{\smallskip}\hline\noalign{\smallskip}
\bf{Measured}        & \\
Anode signal width, $\sigma$        & Gaussian plus sine fit, $2-3\%$.  \\
Drift time,   $t_2 + t_{3}$     &  Time interval between extrema of the cathode and anode signal fits. \\
Initial signal width       & Introduced by the lamp pulse. Measurements in vacuum and in LXe,  $\SI{2.4(2)}{\micro\second}$.    \\
Electronics  & RC time constant calculation from a square pulse,  $\SI{0.2}{\micro\second}$.    \\
Drift length, $d_{2} + d_{3}$ & Drift distance of the electron cloud, taken as $\SI{513(7)}{\milli\meter}$ when accounting for the potential contraction of the components at \SI{177}{\kelvin} with an assumed 1\% thermal contraction, and the position of the centre of the cloud distribution in drift regions with respect the extrema of the signals.\\  
Filtering and processing & Maximum 4\% of anode signal width.\\
\noalign{\smallskip}\hline\noalign{\smallskip}
\bf{Simulated}        & \\
Detector response   & COMSOL 3D model of the detector to derive the weighting potential, 10\% uncertainty in the response.\\
\noalign{\smallskip}\hline\noalign{\smallskip}
\bf{Assumed}        & \\
Coulomb repulsion   & Calculated with empirical model from~\cite{Njoya:2019ldm}, assumption of additional 5\% uncertainty in the initial signal spread.\\
Electron attachment   &  Neglected, 4th-order correction: $\frac{D_L}{(v_d \cdot d)} \ll 1$. \\
\hline
\end{tabularx}
\end{table}

From the listed sources of uncertainty for the diffusion coefficient $D_L$, the largest impact originates (in percentages of the total uncertainty from \SI{25}{} to \SI{75}{V/cm}) from the final width measured at the anode (95\% to 80\%), the initial signal width (20\% to 50\%), the Coulomb repulsion (1\% to 5\%), the uncertainty in the drift distance (20 to 5\%) and the drift time (4 to 1\%). 

\begin{figure}[h!]
\centering
\includegraphics[width=0.5\textwidth]{figures/longitudinal_diff.pdf}
\caption[Longitudinal diffusion calculated in this work]{Longitudinal diffusion coefficient calculated in this work, with an electron drift lifetime of $\tau = \SI{649(23)}{\micro\second}$ compared to the results from a purity monitor by Njoya et. al. ($\tau \sim$ $\SIrange{1}{35}{\micro s}$)~\cite{Njoya:2019ldm}, and a TPC from Hogenbirk ($\tau \sim $ $\SI{430}{\micro s}$)~\cite{Hogenbirk2018} and NEST~\cite{szydagis_m_2018_1314669}. The model by NEST version 2.3.7 (solid grey curve) predicted diffusion values approaching zero for lower drift fields. In version 2.3.8, a fix for this behaviour was introduced, as shown in the solid blue curve.}
\label{fig:longitudinal_diffusion}
\end{figure}

NEST version 2.3.7 implemented an empirical model based on previous measurements for the longitudinal diffusion coefficient prediction. At the time of this analysis, NEST lacked data at low drift fields (below $\sim \SI{100}{V/cm}$), and the model predicted a considerably lower longitudinal diffusion coefficient with lower drift fields, in conflict  with the measured values in this and other works. An alternative model is included in NEST, based on differential cross sections derived from Dirac-Fock solutions, combined with Maxwell-Boltzmann distributions~\cite{Boyle:2016wpy}, which do not predict the existing experimental values at all drift fields. The subsequent NEST version 2.3.8 includes a correction on the diffusion modelling, also shown in figure~\ref{fig:longitudinal_diffusion}, resulting from an exchange with the developers. Our analysis aimed to not only infer the values of longitudinal diffusion coefficient at low drift fields for liquid xenon, but also to understand their origin, related to the temperature and purity of the xenon. 

\section{Conclusions and outlook}
\label{sec:conclusions}

The construction of a next-generation liquid xenon detector at the \SI{50}{t} scale and beyond will face several technological challenges. To address some of these, the Xenoscope facility was designed and built to house a \SI{2.6}{m} tall two-phase TPC at the University of Zurich, with a total LXe mass of $\sim$~\SI{400}{\kg}. After a commissioning run described in Ref.~\cite{Baudis:2021ipf}, we presented here the first results from a run with a \SI{53}{cm} tall purity monitor.

The electron drift lifetime was monitored for 88 days with varying xenon recirculation speeds. For a speed of \SI{40}{slpm}, the highest achieved lifetime was \SI{664(23)}{\micro \s}. A parametric model of the effect of the purification rate, the time-dependent outgassing rate, the liquid-gas impurity diffusion, and the injection of impurities due to operational changes, was fitted to the data. The resulting model was used to predict the electron drift lifetime evolution for different purification conditions and, therefore, inform future design and operation choices.

The electron drift velocity and the longitudinal diffusion coefficient of the electron cloud in liquid xenon were calculated based on data acquired at drift fields between \SI{25}{} and \SI{75}{V/cm}. With the increasing size of LXe TPCs, diffusion strongly affects the position reconstruction of events and the ability to discriminate between single and multiple interactions. Thus, accurate measurements of drift and diffusion properties, combined with an improved understanding of the systematic effect of impurity concentrations on these properties on large scales, are crucial. Our results are in agreement with previous studies both for drift velocity~\cite{gushchin1982electron,Albert:2016bhh,Baudis:2017xov,Thieme:2022dze} and longitudinal diffusion~\cite{Njoya:2019ldm, Hogenbirk2018}. They also triggered an update of NEST~\cite{szydagis_m_2018_1314669}, a simulation package largely used in the community, regarding the modelling of longitudinal diffusion of electron clouds in LXe.

For the next stage of the Xenoscope project, a two-phase xenon TPC was recently built and installed, and will be operated to observe electron drift over distances up to \SI{2.6}{\metre}. The upgrade includes liquid-level control and monitoring, high-voltage supply up to \SI{50}{kV} via a ceramic feedthrough, and an array of silicon photomultipliers for light-readout in the gas phase located just above the gas/liquid interface~\cite{sipm_proceedings}. The latter replaces the charge readout used at the anode of the purity monitor, detecting instead the proportional scintillation produced in the xenon gas region of the TPC. The TPC equipped with the SiPM array will be used to study electron cloud diffusion in both longitudinal and transverse directions. Transverse diffusion is another critical parameter for more accurate modelling of electron transport in xenon-based detectors, for which measurements in the literature are scarce~\cite{Albert:2016bhh, Aprile:2009dv}. Another goal of the upgrade is to study optical properties of liquid xenon at large scales, as well as new types of photosensors under the operating conditions of DARWIN. In addition, the facility will be available to the collaboration for various R\&D projects related to the realisation of a large-scale xenon TPC.




\begin{acknowledgements}
This work was supported by the European Research Council (ERC) under the European Union's Horizon 2020 research and innovation programme, grant agreement No. 742789 ({\sl Xenoscope}), by the SNF grant 20FL20-201437, as well as by the European Union’s Horizon 2020 research and innovation programme under the Marie Skłodowska -Curie grant agreement No 860881-HIDDeN. We thank the electronics and mechanical workshops in the UZH Physics Department for their continuous support. We thank Laura Manenti for insightful discussions about purity monitors.
\end{acknowledgements}

\appendix
\section{Additional Experimental Results}

\subsection{Implementation Details of Baselines}
We compare three baselines in our paper. For Tune-A-Video, to ensure a fair comparison, we use the pre-trained weight of Stable Diffusion-v1.5\footnote{https://github.com/CompVis/stable-diffusion} (same as our model) to initialize the UNet and we fine-tune the model with an image resolution of $256 \times 256$ on the training sets of Something Something-V2 (SSv2) and Bridgedata for 200k training steps. For MCVD, we train the model with an image resolution of $256 \times 256$ for 300k training steps. For TATS, we fine-tune the pre-trained UCF-101 model with an image resolution of $128 \times 128$ on the training sets of SSv2 and Bridgedata for 300k training steps.



\subsection{Evaluation Details and Results of UCF-101}~\label{appendix:sec:ucf101}
Most prior text-conditioned video generation methods~\cite{hong2023cogvideo,vdm,makeavideo,magicvideo} evaluate their performance on the UCF-101~\cite{ucf101} benchmark. However, since our proposed method, Seer, is designed for text-conditioned video prediction (TVP) on task-level video datasets, the UCF-101 benchmark, which evaluates class-conditioned video prediction on random short-horizon video clips, is not an ideal evaluation benchmark for TVP. Nonetheless, in order to fairly compare these baselines, we still evaluate the class-conditioned video prediction performance of Seer on UCF-101. 

\paragraph{Settings}Specifically, we fine-tune our model with a video resolution of $16\times256\times256$ on UCF-101. Following the evaluation protocols of ~\cite{hong2023cogvideo}, Seer predicts the videos conditioned on 5 reference frames during fine-tuning and inference stage. We report FVD and Inception score (IS) metrics on the UCF-101 dataset~\cite{ucf101}. The IS is calculated by a C3D model\cite{c3d} that is pre-trained on the Sports-1M dataset~\cite{sports} and fine-tuned on UCF101. We follow the evaluation code of TGAN-v2~\cite{tganv2} to calculate IS metric. Following ~\cite{hong2023cogvideo,vdm,makeavideo}, we evaluate the FVD metric with 2,048 samples and IS metric with 100k samples in the validation set of UCF-101.

\paragraph{Results} Table~\ref{table:tvp:ucf} presents the class-conditioned video prediction results on UCF-101, demonstrating that Seer outperforms CogVideo~\cite{hong2023cogvideo} and MagicVideo~\cite{magicvideo}, but falls short of Make-A-Video~\cite{makeavideo}. Make-A-Video employs unlabelled video pre-training on temporal layers and achieves the best performance among all other methods. While Make-A-Video shows superior performance on FVD and IS, Seer has the potential to further improve its generation performance by addressing the following two limitations. First, Seer has not been pre-trained on video data. Second, Seer obtains latent vectors via a pre-trained 2D VAE, which has not been fine-tuned on UCF-101 and limits the video generation quality of Seer (with 259.4 FVD and 68.16 IS reconstruction quality). However, as we focus on the text-conditioned video prediction task, addressing the above limitations on UCF-101 is out of the scope of this paper.

\begin{table*}
\begin{threeparttable}
\centering\small
\tablestyle{2pt}{1.1}
\setlength{\tabcolsep}{5pt}
\caption{\textbf{ Class-conditioned video prediction performance on UCF-101} we evaluate the Seer on the UCF-101 with 16-frames-long videos. Ex.data indicates that the model has been pre-trained or fine-tuned on extra datasets.
}
\label{table:tvp:ucf}
\begin{tabular}{cccc|cc}
\specialrule{.1em}{.05em}{.05em} 
 Method & Ex.data & Cond. & Resolution & FVD$\downarrow$ & IS$\uparrow$\\
 \hline
MoCoGAN-HD~\cite{mocogan} & No & Class.  & $256\times 256$ & 700\tiny{$\pm$24} & 33.95\tiny{$\pm$0.25}\\
VideoGPT~\cite{videogpt} & No & No  & $128\times 128$ & - & 24.69\tiny{$\pm$0.30}\\
RaMViD~\cite{ramvid} & No & No  & $128\times 128$ & - & 21.71\tiny{$\pm$0.21}\\
StyleGAN-V~\cite{styleganv} & No & No  & $128\times 128$ & - & 23.94\tiny{$\pm$0.73}\\
DIGAN~\cite{digan} & No & No  & & 577\tiny{$\pm$22} & 32.70\tiny{$\pm$0.35}\\
TGANv2~\cite{tganv2} & No & Class.  & $128\times 128$ & 1431.0 & 26.60\tiny{$\pm$0.47}\\
VDM~\cite{vdm} & No & No  & $64\times 64$ & - & 57.80\tiny{$\pm$1.3}\\
TATS-base~\cite{tats} & No & Class.  & $128\times 128$ & 278\tiny{$\pm$11} & 79.28\tiny{$\pm$0.38}\\
MCVD~\cite{mcvd} & No & No  & $64\times 64$ & 1143.0 & -\\
LVDM~\cite{lvdm} & No & No  & $256\times 256$ & 372\tiny{$\pm$11} & 27\tiny{$\pm$1}\\
MAGVIT-B~\cite{magvit} & No & Class.  & $128\times 128$ & 159\tiny{$\pm$2} & 83.55\tiny{$\pm$0.14}\\
 \hline
CogVideo~\cite{hong2023cogvideo} & txt-img \& txt-video & Class.  & $160\times 160$ & 626 & 50.46\\
Make-A-Video~\cite{makeavideo} & txt-img \& video & Class.  & $256\times 256$ & 81.25 & 82.55\\
MagicVideo~\cite{magicvideo} & txt-img \& txt-video & Class.  & & 699 & -\\
\textbf{Seer(Ours)} & txt-img & Class. & $256\times 256$ & 287.8 & 57.74\\
\specialrule{.1em}{.05em}{.05em} 
\textbf{pre-trained VAE}\tnote{*} & - & - & $256\times 256$ & 259.4 & 68.16\\
\specialrule{.1em}{.05em}{.05em} 
\end{tabular}
\begin{tablenotes}\footnotesize
    \item [*] we evaluate the reconstruction quality of pre-trained 2D VAE in this table, the pre-trained 2D VAE is initialized with the pre-trained weight from Stable Diffusion-v1.5 without extra fine-tuning.
\end{tablenotes}
\end{threeparttable}
\end{table*}



\subsection{Evaluation Results of Sampling Steps}
To further investigate the generation effects of sampling steps during evaluation, we conduct a comparison between Seer and Tune-A-Video. We apply a series of DDIM sampling steps (10, 20, 30, 40, 50 DDIM steps), as shown in Figure~\ref{fig:ddimstep}. Seer consistently outperformed Tune-A-Video in terms of both FVD and KVD, with improvements observed from 20 DDIM steps to 50 DDIM steps. Particularly noteworthy is Seer's advantage in video quality (280.7 FVD and 0.73 KVD) compared to Tune-A-Video (419.3 FVD and 1.5 KVD) when using only 10 DDIM steps, demonstrating Seer's ability to sample high-fidelity videos efficiently with minimal denoising steps.
\begin{figure}
\centering
\includegraphics[width=1.0\linewidth]{fig_appendix/ddim_curve.pdf}
\caption{Evaluation results of Seer and Tune-A-Video with DDIM sampling steps ranging from 10 to 50 on the Something-Something V2 dataset.}
\vspace{-8pt}
\label{fig:ddimstep}
\end{figure}



\subsection{Additional Ablation Results}~\label{appendix:sec:fstext}
\paragraph{FSText layer depth}In this section, we additionally investigate the impact of FSText Decomposer's layer depth in Table~\ref{table:ablation:layer}. Our default setting (8-layer FSText Decomposer) outperforms shallower models (2-layer and 4-layer) in terms of FVD. Though the 4-layer model shows a marginal advantage over the 8-layer model in terms of KVD, our experiments indicate that the 8-layer FSText Decomposer shows a remarkable advantage on FVD metrics and exhibits robustness in text-video alignment. Therefore, we adopt the 8-layer FSText Decomposer as the default setting for Seer.

\begin{table}
\centering\small
\tablestyle{2pt}{1.0}
\setlength{\tabcolsep}{5pt}
\caption{\textbf{Layer depth in FSText Decomposer}.}
\label{table:ablation:layer}
\vspace*{-3mm}
\begin{tabular}{c|cc}
num. layers.& FVD$\downarrow$ & KVD$\downarrow$\\
 \hline
 2 & 238.6 & 0.51\\
 4 & 229.7 & 0.23\\
8(Ours) &200.1 & 0.30\\
\end{tabular}
\end{table}

\paragraph{Qualitative results of fine-tuning ablation} We conduct a qualitative analysis of various fine-tune settings. We provide additional visualizations of Fine-tune Setting ablation in Section 5.5 of the main paper. Figure~\ref{fig:ablate:finetune} shows the results of different settings. Among these settings, our default setting \textit{``temp+FSText"} stands out as it preserves a higher-level temporal consistency in video prediction starting from reference frames and also delivers superior text-based video motion compared to the other fine-tune settings. 
\begin{figure}
\centering
\includegraphics[width=1.0\linewidth]{fig_appendix/ablation_visualization.pdf}
\caption{Additional qualitative results of fine-tuning ablation. \textit{“temp+FSText.”} is our default setting.}
\vspace{-8pt}
\label{fig:ablate:finetune}
\end{figure}


\begin{table}
\centering\small
\tablestyle{2pt}{1.1}
\caption{Hyperparameters and details of Fine Tuning/Inference}
\label{table:hyperparam:finetune}
\begin{tabular}{c|cc}
\hline
param. & value\\
\hline
optim. & AdamW\\
Adam-$\beta_1$ &  0.9\\
Adam-$\beta_2$ &  0.99\\
Adam-$\epsilon$ &  $1e^{-8}$\\
weight decay &  $1e^{-2}$\\
lr &  $1.28e^{-5}$\\
end lr & 0.0\\
lr sche. & cosine\\
noise sche. & cosine\\
train batch size& 1/GPU\\
grad. acc.& 2\\
warmup steps& 10k\\
resolution& $256 \times 256$\\
train. steps & 200k\\
train. hardware & 4 RTX 3090\\
val. batch size& 2/GPU\\
sampler& DDIM\\
sampling steps & 30\\
guidance scale & 7.5\\
\hline
\end{tabular}
\end{table}

\begin{table}
\centering\small
\tablestyle{2pt}{1.1}
\caption{Hyperparameters of 3D U-Net}
\label{table:3dunet}
\begin{tabular}{c|cc}
\hline
hyperparam. & value\\
\hline
input/output channels &  4\\
Base channels & 320\\
Channel multipliers&  1,2,4,4\\
3D Downsample blocks &  4\\
3D Upsample blocks &  4\\
Number of layers (per block) &  2\\
\hline
Modules of layer & 3D ResnetBlock\\
 & Spatial-cross Atten.\\
 & ATS Atten.\\
 & Down./Up. 3D ResnetBlock\\
\hline
Dimension of atten. heads &  8\\
activation function &  SiLU\\
Dimension of cross-atten. &  768\\
\hline
\end{tabular}
\end{table}

\begin{table}
\centering\small
\tablestyle{2pt}{1.1}
\caption{Hyperparameters of FSText Decomposer}
\label{table:hyperparam:fstext}
\begin{tabular}{c|cc}
\hline
hyperparam. & value\\
\hline
learnable tokens channels &  768\\
output channels &  768\\
Base channels & 768\\
Number of layers &  8\\
\hline
Modules of layer & Seq-cross Atten.\\
 & Feedforward\\
 & Directed temporal Atten.\\
 & Feedforward\\
\hline
Number of atten. heads &  8\\
Dimension of cross-atten. &  768\\
\hline
\end{tabular}
\end{table}




\section{Implementation Details}~\label{appendix:sec:impl}

\subsection{Fine-tuning and Sampling}\label{sec:finetuneparam}
 In this section, we list the hyperparameters, fine-tuning details, sampling details, and hardware information of our model in Table~\ref{table:hyperparam:finetune}.
 
\subsection{Architecture information}\label{sec:arch}
In this section, we list the hyperparameters of 3D U-Net in Table~\ref{table:3dunet} and hyperparameters of FSText Decomposer in Table~\ref{table:hyperparam:fstext}.



\section{Visualization} 

\subsection{Additional qualitative results} 
We provide additional visualization on Something-Something v2 (SSv2) of our text-conditioned video prediction in Figure~\ref{fig:ssv2pred}, and text-conditioned video prediction/manipulation results in Figure~\ref{fig:ssv2mani}. Additionally, we provide the visualization on BridgeData of text-conditioned video prediction in Figure~\ref{fig:bridgepred} and text-conditioned video prediction/manipulation in Figure~\ref{fig:bridgemani}.
\section{Human Evaluation Details}~\label{appendix:sec:humaneval} 
To evaluate the quality of video predictions according to human preferences, we conducted a human evaluation with 99 video clips on the validation set of the Something-Something V2 dataset (SSv2), the evaluation process involved 54 anonymous evaluators. To eliminate biases towards specific baselines, we randomly selected 20 questions for each evaluator. Each single-choice question consisted of a ground-truth video as a reference, a manually modified text instruction, and two video prediction results generated by Seer and another baseline method. The evaluators were required to choose the video clip that is more consistent with the text instruction and has higher fidelity from the two options.
To ensure the clarity of the questions, we provided an example to explain the options in each questionnaire. Moreover, we recommended that evaluators prioritize video predictions with strong text-based motions as their first preference and the fidelity of the generated video as their second preference. For reference, Figure~\ref{fig:humanevalexp} provides a screenshot of an example questionnaire.

In total, we collected 342 responses for the Seer vs. TATS comparison, 363 responses for the Seer vs. Tune-A-Video comparison, and 357 responses for the Seer vs. MCVD comparison. And the results in the main paper Figure 7 are calculated based on the collected questionnaires.
\clearpage
\begin{figure}
\centering
\includegraphics[width=1.0\linewidth]{fig_appendix/sth_predict.pdf}
\caption{Text-conditioned video prediction of Seer on SSv2.}
\vspace{-8pt}
\label{fig:ssv2pred}
\end{figure}
\begin{figure}
\centering
\includegraphics[width=1.0\linewidth]{fig_appendix/sth_manipulate.pdf}
\caption{Text-conditioned video prediction/manipulation of Seer on SSv2, where ``pred." refers to prediction, ``mani." refers to manipulation.}
\vspace{-8pt}
\label{fig:ssv2mani}
\end{figure}
\begin{figure*}
\centering
\includegraphics[width=0.9\linewidth]{fig_appendix/bridge_pred.pdf}
\caption{Text-conditioned video prediction of Seer on BridgeData.}
\vspace{-8pt}
\label{fig:bridgepred}
\end{figure*}
\begin{figure*}
\centering
\includegraphics[width=1.0\linewidth]{fig_appendix/bridge_manipulate.pdf}
\caption{Text-conditioned video prediction/manipulation of Seer on BridgeData, where ``pred." refers to prediction, ``mani." refers to manipulation.}
\vspace{-8pt}
\label{fig:bridgemani}
\end{figure*}


\begin{figure}
\centering
\includegraphics[width=1.0\linewidth]{fig_appendix/screenshot.PNG}
\caption{Screenshot of a questionnaire example shown to human evaluators.}
\vspace{-8pt}
\label{fig:humanevalexp}
\end{figure}

%\bibliographystyle{JHEP}
%\bibliography{demo-firstresults}  
\pdfoutput=1 
\RequirePackage{fix-cm}
\documentclass[twocolumn,epjc3]{svjour3}  
\RequirePackage{graphicx}
\usepackage{tikz}
\RequirePackage{mathptmx}
\usepackage[english]{babel}
\usepackage{amsmath}
\usepackage{mathtools}
\usepackage[separate-uncertainty=true]{siunitx}
\DeclareSIUnit\bar{bar}
\sisetup{range-phrase = -, range-units=single}
\usepackage{comment}
\usepackage{tabularx}
\usepackage[version=4]{mhchem}
\usepackage[font={small},labelfont=bf,labelsep=quad]{caption}
\usepackage[font={normal}]{subcaption}
\usepackage{cite}
\usepackage{wasysym}
\usepackage{textcomp}
\usepackage{physics}
\usepackage{csquotes}
\RequirePackage[colorlinks,citecolor=blue,urlcolor=blue,linkcolor=blue]{hyperref}

\defineshorthand{"~}{\babelhyphen{nobreak}}
\useshorthands{"}
\newcommand{\LN}{LN$_2$}
\DeclareSIUnit[number-unit-product = {}]{\inch}{"}
%\sisetup{}
\sisetup{per-mode = power}
\sisetup{exponent-product = \cdot}
\sisetup{group-separator = {\,}}
\sisetup{group-minimum-digits = 3}
\DeclareSIUnit\litre{L}
\DeclareSIUnit\liter{L}

% \usepackage[mathlines]{lineno}
% \let\oldequation\equation
% \let\oldendequation\endequation

% \renewenvironment{equation}
%   {\linenomathNonumbers\oldequation}
%   {\oldendequation\endlinenomath}
% \linenumbers

\journalname{Eur. Phys. J. C}
\begin{document}
\sloppy

\title{Electron transport measurements in liquid xenon with Xenoscope, a large-scale DARWIN demonstrator}

\author{L.~Baudis
        \and
        Y.~Biondi\thanksref{e1}\thanksref{e4}
        \and
        A.~Bismark
        \and
        A.~P.~Cimental Ch\'avez
        \and
        J.~J.~Cuenca-Garc\'ia
        \and
        J.~Franchi
        \and
        M.~Galloway
        \and
        F.~Girard\thanksref{e2}
       \and
       R.~Peres\thanksref{e3}
       \and
       D.~Ram\'irez~Garc\'ia
       \and
       P.~Sanchez-Lucas\thanksref{e5}
       \and
       K.~Thieme\thanksref{e6}
       \and
       C.~Wittweg
}
\thankstext{e1}{e-mail: \url{yanina.biondi@physik.uzh.ch}}
\thankstext{e2}{e-mail: \url{frederic.girard@physik.uzh.ch}}
\thankstext{e3}{e-mail: \url{ricardo.peres@physik.uzh.ch}} 
\thankstext{e4}{Now at Karlsruhe Institute of Technology}
\thankstext{e5}{Now at University of Granada}
\thankstext{e6}{Now at University of Hawai\textquoteleft{i} at M\={a}noa}
\institute{\normalsize{Department of Physics, University of Zurich, Winterthurerstrasse 190, 8057 Zurich, Switzerland}
\label{addr1}}

\date{Received: date / Accepted: date}

\maketitle

\begin{abstract}
There is a compelling physics case for a large, xenon-based underground detector devoted to dark matter and other rare-event searches. A two-phase time projection chamber as inner detector allows for a good energy resolution, a three-dimensional position determination of the interaction site and particle discrimination. To study challenges related to the construction and operation of a \mbox{multi-tonne} scale detector, we have designed and constructed a vertical, full-scale demonstrator for the DARWIN experiment at the University of Zurich. Here we present first results from a several-months run with {\SI{343}{kg}} of xenon and electron drift lifetime and transport measurements with a \SI{53}{cm} tall purity monitor immersed in the cryogenic liquid. After \SI{88}{\day} of continuous purification, the electron lifetime reached a value of $\SI{664(23)}{\micro\second}$. We measured the drift velocity of electrons for electric fields in the range (25--75)\,V/cm, and found values consistent with previous measurements. We also calculated the longitudinal diffusion constant of the electron cloud in the same field range, and compared with previous data, as well as with predictions from an empirical model.
\end{abstract}

\input{sections/01-introduction}

\input{sections/02-experimental_setup}

\input{sections/03-measurements}

\input{sections/04-results_discussion}

\input{sections/05-conclusions}

\begin{acknowledgements}
This work was supported by the European Research Council (ERC) under the European Union's Horizon 2020 research and innovation programme, grant agreement No. 742789 ({\sl Xenoscope}), by the SNF grant 20FL20-201437, as well as by the European Union’s Horizon 2020 research and innovation programme under the Marie Skłodowska -Curie grant agreement No 860881-HIDDeN. We thank the electronics and mechanical workshops in the UZH Physics Department for their continuous support. We thank Laura Manenti for insightful discussions about purity monitors.
\end{acknowledgements}

\input{sections/06-appendix}

%\bibliographystyle{JHEP}
%\bibliography{demo-firstresults}  
\input{demo-firstresults.bbl} 

\end{document} 

\end{document} 

\end{document} 

\end{document}