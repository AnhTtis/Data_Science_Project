\section{Properties}
\label{sec:Properties}

In this section we will present a number of interesting properties of the Love symmetries. First, we will show how the near-zone $\SL$ symmetries acquire a geometric interpretation as isometries of effective geometries within the framework of subtracted geometries~\cite{Cvetic:2011dn,Cvetic:2011hp}. Then, we will demonstrate how both near-zone symmetries can be realized as subalgebras of a larger, infinite-dimensional $\SL\ltimes \hat{U}\left(1\right)^2_{\mathcal{V}}$ extension, which is interpreted as the $5$-d version of the $\SL\ltimes \hat{U}\left(1\right)_{\mathcal{V}}$ infinite extension of the Love symmetry for Kerr-Newman black holes~\cite{Charalambous:2021kcz,Charalambous:2022rre}. We will also present another interesting generalization of the near-zone symmetries which will exhaust all the possible near-zone truncations of the equations of motion that are equipped with an enhanced $\SL$ symmetry and acquire a subtracted geometry interpretation. We will close this section by proposing a physical interpretation of the states in the highest-weight Love multiplet we saw in the previous section. In particular, by employing a partial wave analysis of the black hole scattering problem, we will argue that states with vanishing Love numbers can be interpreted as total transmission modes~\cite{Hod:2013fea,Cook:2016fge,Cook:2016ngj,Ivanov:2022qqt}.

\subsection{Near-zone symmetries as isometries of subtracted geometries}

To introduce the notion of subtracted geometries, we begin by writing the geometry of the $5$-d Myers-Perry black hole as a fibration over a $4$-d base space~\cite{Chong:2006zx,Cvetic:2011hp},
\be
	\begin{gathered}
		ds^2 = -\Delta_0^{-2/3}G\left(dt+\mathcal{A}\right)^2 + \Delta_0^{1/3}ds_4^2 \,, \\
		ds_4^2 = \frac{d\rho^2}{4X} + \,d\theta^2 + \frac{1}{4}\sum_{i,j=1}^{2}\gamma_{ij}d\phi^{i}d\phi^{j} \,,
	\end{gathered}
\ee
where we use small Latin indices from the middle of alphabet to label the azimuthal angles, with $\phi^1\equiv\phi$ and $\phi^2\equiv\psi$. In the notation more conventionally used to write down the line element of the geometry (see Appendix \ref{sec:Ap5dMPGeometry}),
\be
	\begin{gathered}
		X = \Delta \,,\quad \Delta_0=\Sigma^3 \,,\quad G=\Sigma\left(\Sigma-\rho_{s}\right) \,, \\
		\mathcal{A} = \frac{\rho_{s}\Sigma}{G}\left(a\sin^2\theta\,d\phi + b\cos^2\theta\,d\psi\right) \,, \\
		\frac{1}{4}\sum_{i,j}\gamma_{ij}d\phi^{i}d\phi^{j} = \frac{\rho_{s}}{G}\left(a\sin^2\theta\,d\phi+b\cos^2\theta\,d\psi\right)^2 + \frac{\rho+a^2}{\Sigma}d\phi^2+\frac{\rho+b^2}{\Sigma}d\psi^2 \,.
	\end{gathered}
\ee

Let us forget for a moment what the explicit expressions for $\Delta_0$, $G$, $\mathcal{A}$, $X$ and $\gamma_{ij}$ are. The generic effective geometry then describes a stationary, axisymmetric black hole whose event horizon is the larger root of the function $X$. The function $G$ captures the static limit at the surface $G=0$, setting the boundaries of the ergosphere. The thermodynamic properties of the black hole are completely independent of the warp-factor $\Delta_0$, while they only depend on the near-horizon behavior of the function $G$, the angular potential $\mathcal{A}$ and the induced metric $\gamma_{ij}$~\cite{Cvetic:2011hp}. The warp-factor $\Delta_0$ can therefore be interpreted as to encode information about the environment surrounding the black hole, rather than its internal structure. A subtracted geometry is then a geometry obtained by a modification of the warp-factor, while keeping all the other metric functions fixed~\cite{Cvetic:2011hp,Cvetic:2011dn}.

To reveal a connection to the Love symmetries, it is more convenient to look at the inverse metric,
\be
	\begin{gathered}
		g^{\mu\nu}\partial_{\mu}\partial_{\nu} = \frac{4}{\Delta_0^{1/3}}\left\{X\partial_{\rho}^2 + \frac{1}{4}\,\partial_{\theta}^2 - \frac{\Delta_0}{4G}\partial_{t}^2 + \sum_{i,j}\gamma^{ij}D_{i}D_{j}\right\} \,,\quad D_{i}\equiv \partial_{i} - \mathcal{A}_{i}\partial_{t} \,,
	\end{gathered}
\ee
with $\gamma^{ij}$ the components of the inverse of the induced azimuthal metric $\gamma_{ij}$. The Love symmetry $\SL_{\left(\sigma\right)}$ can then be realized as an isometry of the effective geometry with inverse metric
\be
	\tilde{g}^{\mu\nu}_{\left(\sigma\right)}\partial_{\mu}\partial_{\nu} = \frac{4}{\tilde{\Delta}_{0}^{\left(\sigma\right)1/3}}\left\{X\partial_{\rho}^2 + \frac{1}{4}\,\partial_{\theta}^2 - \frac{\tilde{\Delta}_{0}^{\left(\sigma\right)}}{4G}\partial_{t}^2 + \sum_{i,j}\gamma^{ij}\tilde{D}_{i}^{\left(\sigma\right)}\tilde{D}_{j}^{\left(\sigma\right)}\right\} \,,\quad \tilde{D}_{i}^{\left(\sigma\right)} = \partial_{i} - \tilde{\mathcal{A}}_{i}^{\left(\sigma\right)}\partial_{t} \,,
\ee
where
\be
	\begin{gathered}
		\tilde{\Delta}_{0}^{\left(\sigma\right)} = 16\rho_{+}\rho_{s}^2\left[1+\beta^2\left(\Omega_{\phi}-\sigma\Omega_{\psi}\right)^2\right] \,, \\
		\tilde{\mathcal{A}}^{\left(\sigma\right)} = -\frac{\rho_{+}\rho_{s}^2}{4\Delta}\left(\Omega_{\phi}\,\partial_{\phi}+\Omega_{\psi}\,\partial_{\psi}\right) - \frac{\rho_{+}-\rho_{-}}{\rho-\rho_{-}}\frac{\beta^2}{4}\left(\Omega_{\phi}-\sigma\Omega_{\psi}\right)\left(\partial_{\phi}-\sigma\partial_{\psi}\right) \,.
	\end{gathered}
\ee
More importantly, this effective geometry has exactly the same $4$-d base space $ds_4^2$ and preserves the entire form of the function $G$ which captures information about the static limit. The only difference relevant to the original definition of subtracted geometries~\cite{Cvetic:2011hp,Cvetic:2011dn} is that the angular potential is itself modified, but in such a way that the thermodynamic properties of the black hole remain unaltered.

As a side note, we remark here that it might be possible to realize subtracted geometries by a scaling limit of the full geometry, see e.g.~\cite{Cvetic:2012tr}. It would be interesting to investigate whether the effective geometries associated with the Love symmetries can be manifested as similar scaling limits of the full $5$-d Myers-Perry black hole geometry. We leave such an analysis for future work.


\subsection{Infinite-dimensional extension}
In the previous section, we presented how two different near-zone truncations of the massless Klein-Gordon equation admit two different $\SL$ symmetries, $\SL_{\left(\sigma\right)}$, $\sigma=\pm$. Both of these $\SL$ algebras can be realized as subalgebras of the semi-direct product $\SL_{\left[0,0\right]}\ltimes U\left(1\right)^2_{\mathcal{V}}$, where $\SL_{\left[0,0\right]}$ is generated by the following vector fields
\be
	\begin{gathered}
		L_0^{\left[0,0\right]} = -\beta\,\partial_{t} \,, \\
		L_{\pm1}^{\left[0,0\right]} = e^{\pm t/\beta}\left[\mp\sqrt{\Delta}\,\partial_{\rho} + \partial_{\rho}\left(\sqrt{\Delta}\right)\beta\,\partial_{t} + \frac{\rho_{+}-\rho_{-}}{2\sqrt{\Delta}}\beta\left(\Omega_{+}\,\partial_{+}+\Omega_{-}\,\partial_{-}\right)\right] \,.
	\end{gathered}
\ee
Each of the $U\left(1\right)_{\mathcal{V}}$ factors is generated by vector fields of the form $\upsilon\,\beta\Omega_{\sigma}\,\partial_{\sigma}$ with $\upsilon$ belonging to a representation $\mathcal{V}$ of $\SL_{\left[0,0\right]}$, for each sign $\sigma=+,-$ respectively.

Let us construct one such representation $\mathcal{V}=\left\{\upsilon_{0,k},k\in\mathbb{Z}\right\}$. We first specify $\upsilon_{0,0}=-1$, which belongs to the singleton representation of $\SL_{\left[0,0\right]}$, satisfying $L_{\pm1}^{\left[0,0\right]}\upsilon_{0,0}=0$, and has vanishing azimuthal numbers, therefore further satisfying $L_0^{\left[0,0\right]}\upsilon_{0,0}=0$. We then built the $\upsilon_{0,\pm1}$ states, under the conditions that they can reach the singleton state via the action of $L_{\mp1}^{\left[0,0\right]}$ and that they have weights $h=\mp1$,
\be
	L_0^{\left[0,0\right]}\upsilon_{0,\pm1} = \mp \upsilon_{0,\pm1} \,,\quad L_{\mp1}^{\left[0,0\right]}\upsilon_{0,\pm1} = \mp\upsilon_{0,0} \,.
\ee
Solving these, we arrive at the following basic states of the representation $\mathcal{V}$
\be
	\upsilon_{0,0}=-1 \,,\quad \upsilon_{0,\pm1}=e^{\pm t/\beta}\sqrt{\frac{\rho-\rho_{+}}{\rho-\rho_{-}}} \,,
\ee
which are automatically regular at both the future and the past event horizons. The rest of the representation $\mathcal{V}$ can then be constructed by climbing up or down the ladder,
\be
	\upsilon_{0,\pm n} = \left[L_{\pm1}^{\left[0,0\right]}\right]^{n-1}\upsilon_{0,\pm1} = \left(\pm1\right)^{n-1}\left(n-1\right)!\,e^{\pm nt/\beta}\left(\frac{\rho-\rho_{+}}{\rho-\rho_{-}}\right)^{n/2} \,,
\ee
with integer $n\ge1$. This construction is depicted in Figure \ref{fig:VSL2R}.

\begin{figure}
	\centering
	\begin{tikzpicture}
		\node at (0,0) (um3) {$\upsilon_{0,-3}$};
		\node at (0,1) (um2) {$\upsilon_{0,-2}$};
		\node at (0,2) (um1) {$\upsilon_{0,-1}$};
		\node at (0,3) (u0) {$\upsilon_{0,0}$};
		\node at (0,4) (up1) {$\upsilon_{0,+1}$};
		\node at (0,5) (up2) {$\upsilon_{0,+2}$};
		\node at (0,6) (up3) {$\upsilon_{0,+3}$};
		
		\node at (3,-0.4) (um) {$\vdots$};
		\draw (1,0) -- (5,0);
		\draw (1,1) -- (5,1);
		\draw (1,2) -- (5,2);
		\draw [snake=zigzag] (1,2.9) -- (5,2.9);
		\draw (1,3) -- (5,3);
		\draw [snake=zigzag] (5,3.1) -- (1,3.1);
		%		\draw [snake=zigzag] (1,3.1) -- (5,3.1);
		\draw (1,4) -- (5,4);
		\draw (1,5) -- (5,5);
		\draw (1,6) -- (5,6);
		\node at (3,6.4) (up) {$\vdots$};
		
		\draw[blue] [->] (2,0) -- node[left] {$L_{+1}^{\left[0,0\right]}$} (2,1);
		\draw[blue] [->] (2.5,1) -- node[left] {$L_{+1}^{\left[0,0\right]}$} (2.5,2);
		\draw[blue] [->] (3,2) -- node[left] {$L_{+1}^{\left[0,0\right]}$} (3,3);
		\draw[blue] [->] (2.5,4) -- node[left] {$L_{+1}^{\left[0,0\right]}$} (2.5,5);
		\draw[blue] [->] (2,5) -- node[left] {$L_{+1}^{\left[0,0\right]}$} (2,6);
		\draw[red] [<-] (4,0) -- node[right] {$L_{-1}^{\left[0,0\right]}$} (4,1);
		\draw[red] [<-] (3.5,1) -- node[right] {$L_{-1}^{\left[0,0\right]}$} (3.5,2);
		\draw[red] [<-] (3,3) -- node[right] {$L_{-1}^{\left[0,0\right]}$} (3,4);
		\draw[red] [<-] (3.5,4) -- node[right] {$L_{-1}^{\left[0,0\right]}$} (3.5,5);
		\draw[red] [<-] (4,5) -- node[right] {$L_{-1}^{\left[0,0\right]}$} (4,6);
	\end{tikzpicture}
	\caption{A representation $\mathcal{V}$ of $\SL_{\left[0,0\right]}$ used to construct the $\SL_{\left[0,0\right]}\ltimes\hat{U}\left(1\right)^2_{\mathcal{V}}$ extension.}
	\label{fig:VSL2R}
\end{figure}

Consequently, we can extend $\SL_{\left[0,0\right]}$ into $\SL_{\left[0,0\right]}\ltimes\hat{U}\left(1\right)^2_{\mathcal{V}}$, via the $U\left(1\right)^2_{\mathcal{V}}$ elements
\be
	\upsilon = \sum_{k\in\mathbb{Z}}\upsilon_{0,k}\left(\alpha_{k,+}\,\beta\Omega_{+}\,\partial_{+} + \alpha_{k,-}\,\beta\Omega_{-}\,\partial_{-}\right) \,.
\ee

Within this infinite extension lies a particular $2$-parameter family of $\SL$ subalgebras,
\be
	\SL_{\left[\alpha_{+},\alpha_{-}\right]}\subset \SL_{\left[0,0\right]}\ltimes U\left(1\right)^2_{\mathcal{V}} \,,
\ee
generated by the vector fields
\be
	L_{m}^{\left[\alpha_{+},\alpha_{-}\right]} = L_{m}^{\left[0,0\right]} + \upsilon_{0,m}\left(\alpha_{+}\,\beta\Omega_{+}\,\partial_{+} + \alpha_{-}\,\beta\Omega_{-}\,\partial_{-}\right) \,,\quad m=0,\pm1 \,.
\ee
The corresponding Casimir operator is given by
\be\ba
	{}&\mathcal{C}_2^{\left[\alpha_{+},\alpha_{-}\right]} = \partial_{\rho}\,\Delta\,\partial_{\rho}- \frac{\rho_{s}^2\rho_{+}}{4\Delta}\left(\partial_{t} + \Omega_{+}\,\partial_{+} + \Omega_{-}\,\partial_{-}\right)^2 \\
	&+ \frac{\rho_{+}-\rho_{-}}{\rho-\rho_{-}}\beta^2\left(\partial_{t} + \alpha_{+}\,\Omega_{+}\,\partial_{+} + \alpha_{-}\,\Omega_{-}\,\partial_{-}\right)\left[\left(\alpha_{+}-1\right)\Omega_{+}\,\partial_{+} + \left(\alpha_{-}-1\right)\Omega_{-}\,\partial_{-}\right] \,.
\ea\ee
Note that for an arbitrary pair $\left(\alpha_{+},\alpha_{-}\right)$, these Casimirs do not correspond to any
consistent physical near-zone truncation of the Klein-Gordon equation in the background of the $5$-d Myers-Perry black hole, except in two cases;
\be
	\alpha^{\text{NZ}}_{\sigma} = 1 \,\quad \text{AND} \quad\, \alpha^{\text{NZ}}_{-\sigma} = 0 \,,
\ee
for $\sigma=+$ or $\sigma=-$. These are precisely our two $\SL$ symmetries of the near-zone truncations \eqref{eq:NZRadial1}-\eqref{eq:NZRadial2} which are now realized as subalgebras of the same larger structure. In the current notation,
\be
	\SL_{\left(+\right)} = \SL_{\left[1,0\right]} \,,\quad \SL_{\left(-\right)} = \SL_{\left[0,1\right]} \,.
\ee
However, the Casimirs with generic $\alpha_{+}$ and $\alpha_{-}$ have another remarkable property: they are the most general globally defined and time-reversal symmetric truncations of the equations of motion which preserve the characteristic exponents in the vicinity of the event horizon (see Appendix \ref{sec:ApSL2RGenerators}).
%\textcolor{blue}{\textbf{In fact, all of these $\SL$ subalgebras can be realized as isometries of subtracted geometries. The corresponding subtracted geometry for the pair $\left(\alpha_{+},\alpha_{-}\right)$ is given by,}}
%\todo[inline]{Write down metric of subtracted geometry}

\subsection{Infinite zones of Love from local time translations}

Beyond the infinite extension described above involving subtracted geometry truncations of the radial Klein-Gordon operator equipped with an enhanced $\SL$ symmetry, there is another, different type of generalization that gives rise to all the possible near-zone truncations of the equations of motion such that an $\SL$ symmetry emerges.
The corresponding generators make up two towers of near-zone $\SL$'s and are given by
\be
	\begin{gathered}
		L_0^{\left(\sigma\right)}\left[g\left(\rho\right)\right] = L_0^{\left(\sigma\right)}\left[0\right] \,, \\
		L_{\pm1}^{\left(\sigma\right)}\left[g\left(\rho\right)\right] = e^{\pm g\left(\rho\right)/\beta}L_{\pm1}^{\left(\sigma\right)}\left[0\right] \pm e^{\pm \left(t+g\left(\rho\right)\right)/\beta}\sqrt{\Delta}\left(\partial_{\rho}g\left(\rho\right)\right)\partial_{t} \,,
	\end{gathered}
\ee
where $L_{m}^{\left(\sigma\right)}\left[0\right]$ are the Love symmetries generators \eqref{eq:LoveGen} and $g\left(\rho\right)$ is an arbitrary radial function which is regular and non-vanishing at the event horizon. All of these $\SL$ algebras, however, can be realized as cousins of the Love symmetries, corresponding to local $\rho$-dependent temporal translations,
\be
	\tilde{t} = t + g\left(\rho\right) \,.
\ee
Indeed, when writing the generators using this time coordinate, they acquire the same form as the Love symmetries generators,
\be
	\begin{gathered}
		L_0^{\left(\sigma\right)}\left[g\left(\rho\right)\right] = -\beta\left(\partial_{\tilde{t}} + \Omega_{\sigma}\,\partial_{\sigma}\right) \,, \\
		L_{\pm1}^{\left(\sigma\right)}\left[g\left(\rho\right)\right] = e^{\pm \tilde{t}/\beta}\left[\mp\sqrt{\Delta}\,\partial_{\rho} + \partial_{\rho}\left(\sqrt{\Delta}\right)\,\beta\left(\partial_{\tilde{t}} + \Omega_{\sigma}\,\partial_{\sigma}\right) + \frac{\rho_{+}-\rho_{-}}{2\sqrt{\Delta}} \beta\Omega_{-\sigma}\,\partial_{-\sigma}\right] \,.
	\end{gathered}
\ee
The associated Casimir is then given by \eqref{eq:LoveCasimir} with $t$ replaced $\tilde{t}$. Furthermore, the argument of vanishing static Love numbers remains unaltered even when using these generalized Love symmetries. Namely, when the conditions for vanishing static Love numbers are satisfied, the corresponding regular at the future event horizon static solution is a descendant in the highest-weight representation of these generalized near-zone $\SL$'s and the highest-weight property dictates the (quasi-)polynomial form of the solution.

\subsection{Vanishing Love numbers as Total Transmission Modes}
In Section~\ref{sec:SL2R} we have demonstrated that highest-weight multiplets are spanned by near-zone solutions with vanishing Love numbers. We will argue here that another interpretation of these states is that they are total transmission modes of the effective near-zone geometry. To show this, we consider the full radial operator in \eqref{eq:FullEOM} and treat it as a scattering problem. The field redefinition
\be
	\Phi = A\left(r\right)\, \Psi \,,\quad A\left(r\right)\equiv \sqrt{\frac{r}{\left(r^2+a^2\right)\left(r^2+b^2\right)}}
\ee
brings the radial problem to its canonical form,
\be
	\left[\partial_{r_{\ast}}^2 - \mathcal{K}^2 - \frac{r^2\Delta}{\left(r^2+a^2\right)^2\left(r^2+b^2\right)^2}\mathcal{V}\right]\Psi = 0 \,,
\ee
where $\mathcal{K}=\partial_{t} + \frac{a}{r^2+a^2}\partial_{\phi} + \frac{b}{r^2+b^2}\partial_{\psi}$ and with the reduced scalar potential given by
\be
	\mathcal{V} = 4\mathbb{P}_{\text{full}} - \frac{a^2b^2}{r^2}\left(\partial_{t}+\frac{1}{a}\,\partial_{\phi}+\frac{1}{b}\,\partial_{\psi}\right)^2 - 2\left(a\,\partial_{\phi}+b\,\partial_{\psi}\right)\partial_{t} - \frac{1}{rA\left(r\right)}\partial_{r}\left(\frac{\Delta}{r}A^{\prime}\left(r\right)\right) \,.
\ee

The setup then consists of an incident scalar wave of frequency $\omega$ coming from infinity and scattered by the black hole. After separating the variables
\be
	\Psi_{\omega\ell mj} = e^{-i\omega t}e^{im\phi}e^{ij\psi}P_{\omega\ell mj}\left(r\right)S_{\omega\ell mj}\left(\theta\right) \,,
\ee
the asymptotic radial wavefunction $P^{\infty}_{\omega\ell mj}$ satisfies the following differential equation in the far-zone region $r\gg r_{+}$,
\be
	\left[\frac{d^2}{dr^2} + \omega^2 - \frac{\ell\left(\ell+2\right)-\frac{3}{4}}{r^2}\right]P^{\infty}_{\omega\ell mj} = 0 \,,
\ee
with $\ell$ an effective orbital number in terms of which the angular eigenvalues are $\ell\left(\ell+2\right)$, but which is non-integer for $\omega\ne0$. The asymptotic solution can then be found in terms of Hankel functions to be
\be\label{eq:FZRadialSolution}
	P^{\infty}_{\omega\ell mj} = \sqrt{\frac{\pi\omega r}{2}}e^{-i\frac{\pi}{2}\left(\ell+\frac{3}{2}\right)}\mathcal{I}_{\ell mj} \left[ H_{\ell+1}^{\left(2\right)}\left(\omega r\right) + \mathcal{R}_{\ell mj}\left(\omega\right)e^{i\pi\left(\ell+\frac{3}{2}\right)}H_{\ell+1}^{\left(1\right)}\left(\omega r\right) \right] \,.
\ee
The integration constants were fixed such that
\be
	P^{\infty}_{\omega\ell mj} \xrightarrow{r\rightarrow\infty} \mathcal{I}_{\ell mj}\left[e^{-i\omega r} + \mathcal{R}_{\ell mj}\left(\omega\right) e^{+i\omega r}\right] \,,
\ee
that is, $\left|\mathcal{I}_{\ell mj}\right|^2$ is the incoming flux and $\mathcal{R}_{\ell mj}\left(\omega\right)$ is the reflection amplitude.

Let us now look at what happens in the near-zone region, $\omega r\ll 1$. For the asymptotic wavefunction solution \eqref{eq:FZRadialSolution} to be supported in the near-zone we must of course also have $\omega r_{s}\ll1$ such that the intermediate region $r_{+}\ll r \ll \omega^{-1}$ is non-empty. From the asymptotic behaviors of the Hankel functions for small arguments, we see that we get an expression of the form
\be
	P^{\infty}_{\omega\ell mj} \xrightarrow{\omega r\ll1} \tilde{\mathcal{I}}_{\ell mj}\left(\omega\right) r^{\ell+\frac{3}{2}}\left[1+k_{\ell mj}\left(\omega\right) \left(\frac{r_{s}}{r}\right)^{2\ell+2}\right] \,,
\ee
from which we can match the response coefficients onto the reflection amplitude. This is in turn related to the real-valued (conservative) phase-shifts $\delta_{\ell}\left(\omega\right)$ and (dissipative) transmission factors $\eta_{\ell}\left(\omega\right)$ that enter a partial wave analysis of the scattering problem~\cite{Matzner1978ApJS,Futterman:1988ni,Dolan:2008kf}. Namely, to linear order in the response coefficients, we have~\cite{Ivanov:2022qqt}
\be
	\eta_{\ell}\left(\omega\right)e^{2i\delta_{\ell}\left(\omega\right)} = e^{i\pi\left(\ell+\frac{3}{2}\right)}\mathcal{R}_{\ell00}\left(\omega\right) = 1 +i\frac{2\pi\sin^2\pi\left(\ell+1\right)}{\Gamma\left(\ell+1\right)\Gamma\left(\ell+2\right)}\left(\frac{\omega r_{s}}{2}\right)^{2\ell+2} k_{\ell 00}\left(\omega\right)
\ee
and, therefore, we can extract a matching condition between the real and imaginary parts of the response coefficients and the phase-shifts and transmission factors respectively~\cite{Ivanov:2022qqt},
\be\ba
	\eta_{\ell}\left(\omega\right) &= 1 -\frac{2\pi\sin^2\pi\left(\ell+1\right)}{\Gamma\left(\ell+1\right)\Gamma\left(\ell+2\right)}\left(\frac{\omega r_{s}}{2}\right)^{2\ell+2} \text{Im}\left\{k_{\ell00}\left(\omega\right)\right\} \,, \\
	\delta_{\ell}\left(\omega\right) &= \frac{\pi\sin^2\pi\left(\ell+1\right)}{\Gamma\left(\ell+1\right)\Gamma\left(\ell+2\right)}\left(\frac{\omega r_{s}}{2}\right)^{2\ell+2} \text{Re}\left\{k_{\ell00}\left(\omega\right)\right\} \,,
\ea\ee
which is an alternative way to see that the Love numbers enter only in the conservative dynamics. From these expressions, one can then compute elastic and absorption cross-sections. In $d=1+4$ spacetime dimensions a partial wave analysis of the scattering problem results in
\be\ba
	\sigma_{\text{elastic}} &= \frac{16\pi}{\omega^3}\sum_{\ell=0}^{\infty}\left(\ell+1\right)^2\left(\ell+2\right)\sin^2\delta_{\ell}\left(\omega\right) \,, \\
	\sigma_{\text{absorption}} &= \frac{4\pi}{\omega^3}\sum_{\ell=0}^{\infty}\left(\ell+1\right)^2\left(\ell+2\right)\left[1-\eta_{\ell}^2\left(\omega\right)\right] \,.
\ea\ee
Consequently, vanishing (dynamical) Love numbers
for certain frequencies 
imply a vanishing partial elastic cross-section,
\be
	\sigma_{\text{elastic},\ell} = 0 \quad \text{ if } \quad k_{\ell00}^{\text{Love}}\left(\omega\right) = 0 \,,
\ee
while the corresponding partial absorption cross-section is maximized. Vanishing Love numbers are, thus, interpreted as reflectionless, total transmission modes~\cite{Hod:2013fea,Cook:2016fge,Cook:2016ngj,Ivanov:2022qqt}. We note, however, that strictly speaking, 
the connection of the Love symmetries highest-weight multiplet states and total transmission modes is not $1$-to-$1$. For instance, the analysis of~\cite{Cook:2016fge} indicates that, for the $d=4$ Kerr black hole, the corresponding Love symmetry highest-weight multiplet~\cite{Charalambous:2021kcz,Charalambous:2022rre} does appear to capture these algebraically special modes but is also over-counting them, in the sense that it also contains states that are not true algebraically special quasinormal modes of the full geometry that 
includes the asymptotically flat region. 
This feature is quite surprising because the Love symmetries manifest themselves only in the near-zone region and, therefore, their highest-weight multiplet states are expected to be accurate only in the regime of low perturbation frequencies. We leave a better understanding of this connection for future work.