\section{Response coefficients and Love numbers}
\label{sec:TLNsDefinition}

In this section, we present the formulation of the response problem for general compact bodies in terms of response tensors in any number of spatial dimensions with the eventual goal to define the Love numbers in $5$ spacetime dimensions.


\subsection{Newtonian definition}
Let us start with the standard formulation of the tidal response problem for a compact body in Newtonian gravity~\cite{PoissonWill2014}. This consists of solving the Poisson equation\footnote{Newton's gravitational constant $G$ is identified as the coupling constant appearing in the Einstein-Hilbert action $S=\frac{1}{16\pi G}\int d^{d}x\,\sqrt{-g}\,R$ such that the Einstein field equations $G_{\mu\nu}=8\pi G T_{\mu\nu}$ preserve their form in any number of spacetime dimensions.} for the Newtonian gravitation potential $\Phi_{\text{N}}$,
\be
	\nabla^2\Phi_{\text{N}} = \frac{d-3}{d-2}8\pi G\rho \,,
\ee
with $\rho=\rho\left(t,\mathbf{x}\right)$ the mass density of the mass configuration. In practice, this is accompanied by two field equations; the continuity equation and Euler's equations. Once supplemented with an equation of state, the problem is well posed.

The setup for introducing the tidal response coefficients begins with an unperturbed mass configuration at equilibrium, practically being hydrostatic equilibrium, which is perturbed by a weak distant mass configuration sourcing tidal forces parameterized by its tidal moments $\bar{\mathcal{E}}_{L}\left(t\right)$. In response, the mass distribution of the body rearranges until a new equilibrium state is reached. The response is encoded in induced mass multipole moments $\delta Q_{L}\left(t\right)$~\cite{PoissonWill2014}. The perturbation in the Newtonian gravitational potential in the exterior of the body is then given by, in frequency space,
\be
	\delta\Phi_{\text{N}}\left(\omega,\mathbf{x}\right) = \sum_{\ell=2}^{\infty}\frac{\left(\ell-2\right)!}{\ell!}\left[\bar{\mathcal{E}}_{L}\left(\omega\right) - N_{\ell}G\frac{\delta Q_{L}\left(\omega\right)}{r^{2\ell+d-3}}\right]x^{L} \,,
\ee
where we have defined the dimensionless constants
\be\label{eq:NnormDimLess}
	N_{\ell} \equiv  \frac{8\pi}{\left(d-2\right)\Omega_{d-2}}\frac{\left(2\ell+d-5\right)!!}{\left(\ell-2\right)!\left(d-5\right)!!} \,,
\ee
with $\Omega_{d-2}=2\pi^{\left(d-1\right)/2}/\Gamma\left(\frac{d-1}{2}\right)$ the surface area of the unit $\left(d-2\right)$-sphere $\mathbb{S}^{d-2}$. Both the tidal moments $\bar{\mathcal{E}}_{L}$ and the induced mass multipole moments $\delta Q_{L}$ are rank-$\ell$ STF spatial tensors by virtue of the Laplace equation in the exterior\footnote{If $V\left(\mathbf{x}\right)$ is a harmonic function, then $\partial_{L}V$ are rank-$\ell$ STF spatial tensors. In the same way, $\partial_{L}\frac{1}{r^{d-3}}$, which are involved in the multipole expansion defining the mass multipole moments, are STF tensors of rank-$\ell$, since $\left|\mathbf{x}-\mathbf{x}^{\prime}\right|^{-\left(d-3\right)}$ is a harmonic function of $\mathbf{x}$ in the exterior.}. We also remark here that we are working in the body-centered frame where the induced dipole moment vanishes identically and all the sums start from $\ell=2$.

Assuming that the body exhibits no gravitational hysteresis, i.e. that it develops no new permanent multipole moments after the tidal source is switched off, we can apply linear response theory and define the dimensionful tidal response tensor $\lambda_{LL^{\prime}}$ in frequency space as the corresponding retarded Green's function\footnote{Lower and upper spatial indices in Newtonian gravity are raised and lowered with the flat space metric.},
\be
	\delta Q_{L}\left(\omega\right) = -\sum_{\ell^{\prime}=2}^{\infty}\lambda_{LL^{\prime}}\left(\omega\right)\bar{\mathcal{E}}^{L^{\prime}}\left(\omega\right) \,,
\ee
where the mixing of different $\ell$-modes is a necessary implementation for rotating and non-spherically symmetric bodies and we are suppressing non-linear corrections. The tidal response tensor $\lambda_{LL^{\prime}}$ is STF with respect to the first multi-index $L$, while only $\lambda_{L\left\langle L^{\prime} \right\rangle}$ is physically relevant. Although we have only displayed an $\omega$-dependence, $\lambda_{LL^{\prime}}$ will strongly depend on other properties associated with the internal structure of the body, e.g. its background multipole moments, including its mass and angular momentum, as well as other parameters entering its equation of state. As a last conventional step, we introduce the computationally favorable dimensionless tidal response tensor $k_{LL^{\prime}}\left(\omega\right)$, defined according to
\be\label{eq:DimensionlessResponse}
	\lambda_{LL^{\prime}}\left(\omega\right) \equiv k_{LL^{\prime}}\left(\omega\right)\frac{\mathcal{R}^{2\ell+d-3}}{N_{\ell}G} \,,
\ee
with $\mathcal{R}$ a scale associated to the unperturbed body's size, e.g. its radius if it is spherically symmetric, such that the frequency space gravitational potential perturbation in the exterior takes the form
\be
	\delta\Phi_{\text{N}}\left(\omega,\mathbf{x}\right) = \sum_{\ell,\ell^{\prime}=2}^{\infty}\frac{\left(\ell-2\right)!}{\ell!}\left[\delta_{L,L^{\prime}} + k_{LL^{\prime}}\left(\omega\right)\left(\frac{\mathcal{R}}{r}\right)^{2\ell+d-3}\right]\bar{\mathcal{E}}^{L^{\prime}}\left(\omega\right)x^{L} \,.
\ee
Although we have presented an analysis for the tidal deformation of a self-gravitating body, it can be extended to compact bodies supported by other types of long-range forces as well. In particular, it can be extended to systems responding to spin-$1$ and spin-$0$ forces and define the corresponding response tensors. Then, the $\ell$-sums will start from $\ell=1$ or $\ell=0$ respectively, but the general prescription outlined above can be carried away unaffected, up to the conventional overall $\left(\ell-s\right)!/\ell!$ factor for a spin-$s$ force system, and define the spin-$s$ response tensors $k_{LL^{\prime}}^{\left(s\right)}$,
\be\label{eq:NewtonianResponseTensros}
	\delta\Phi^{\left(s\right)}\left(\omega,\mathbf{x}\right) = \sum_{\ell,\ell^{\prime}=s}^{\infty}\frac{\left(\ell-s\right)!}{\ell!}\left[\delta_{L,L^{\prime}} + k_{LL^{\prime}}^{\left(s\right)}\left(\omega\right)\left(\frac{\mathcal{R}}{r}\right)^{2\ell+d-3}\right]\bar{\mathcal{E}}^{\left(s\right)L^{\prime}}\left(\omega\right)x^{L} \,.
\ee
In this language, the above analysis of the tidal response of a gravitational system corresponds to the Newtonian definition of the $k_{LL^{\prime}}^{\left(2\right)}$ response tensor. For $s=1$, $\Phi^{\left(1\right)}$ is the electrostatic potential and $k_{LL^{\prime}}^{\left(1\right)}$ defines the electric susceptibility tensor of the body, while there is an analogous definition of the magnetic susceptibility tensor associated with induced electric current multipole moments in the vector potential profile which we do not write down here. Last, for $s=0$, $\Phi^{\left(0\right)}$ is a scalar field for the potential of a system interacting via scalar forces and $k_{LL}^{\left(0\right)}$ defines the scalar susceptibility tensor of the body.


\subsection{General relativistic EFT definition and Newtonian matching}
When relativistic effects are taken into account, the definition of the associated response tensors is more subtle. To begin with, the response tensors should be gauge invariant under diffeomorphisms. In addition, the growing mode in the profiles of the potentials, the ``source'' part of the field, acquires relativistic corrections and results in an overlapping with the decaying mode, the ``response'' part of the field, thus, raising concerns for a source/response ambiguity~\cite{Gralla:2017djj}.

These concerns are all addressed within the framework of the worldline EFT~\cite{Goldberger:2004jt,Porto:2005ac} whose starting point is the universal point-particle appearance of compact bodies from large distances. We will only briefly review the EFT definition of Love numbers here. A more complete review fitted to the tidal response problem can be found in~\cite{Charalambous:2021mea,Ivanov:2022hlo}, while comprehensive reviews of the worldline EFT formalism can be found in~\cite{Porto:2016pyg,Levi:2018nxp}.
One is effectively integrating out the short-scale modes associated with the internal structure of the body, leaving an effective action whose degrees of freedom are the worldline position $x_{\text{cm}}\left(\lambda\right)$ along which the center of the mass of the body propagates and parameterized by an affine parameter $\lambda$, a set of vielbein vectors $e_{a}^{\mu}\left(\lambda\right)$ localized on the worldline and capturing rotational degrees of freedom in the case the body is spinning\footnote{If the body is spinning, the notion of a ``center of mass'' is not invariant and one needs to supplement with a spin gauge symmetry~\cite{Porto:2005ac} to compensate for this. The center of mass is then fixed via a ``Spin Supplementary Condition'' (SSC) corresponding to fixing the time-like vector of the worldline vielbein $e_{a=0}^{\mu}$~\cite{Porto:2016pyg,Levi:2018nxp}.}, and the long distance metric perturbations with respect to the Minkowski background $h_{\mu\nu}=g_{\mu\nu}-\eta_{\mu\nu}$. These are supplemented with other types of bulk fields and symmetries in the case of non-pure gravity, e.g. with a $U\left(1\right)$ gauge field $A_{\mu}$ for the Einstein-Maxwell theory or a real scalar field $\Phi$ for systems interacting via scalar forces. The effective action then contains a ``minimal'' point-particle action, while finite-size effects are captured by non-minimal couplings of the worldline with higher-derivative operators,
\be
	S_{\text{EFT}}\left[x_{\text{cm}},e,h,A,\Phi\right] = S_{\text{bulk}}\left[\eta+h,A,\Phi\right] + S_{\text{pp}}\left[x_{\text{cm}},e,h\right] + S_{\text{finite-size}}\left[x_{\text{cm}},e,h,A,\Phi\right] \,.
\ee

Love numbers are defined as particular Wilson coefficients in front of quadratic couplings of the worldline with field strength tensors. For the simplest case of a spherically symmetric, non-rotating body, for example, the static Love numbers are defined from
\be
	S_{\text{finite-size}} \supset S_{\text{Love}} = \sum_{s=0}^{2}\sum_{\ell=s}^{\infty}\frac{C_{\text{el},\ell}^{\left(s\right)}}{2\ell!}\int d\tau\,\mathcal{E}_{L}^{\left(s\right)}\left(x_{\text{cm}}\left(\tau\right)\right)\mathcal{E}^{\left(s\right)L}\left(x_{\text{cm}}\left(\tau\right)\right) + \left(\mathcal{E}\leftrightarrow \mathcal{B},\mathcal{T}\right) \,,
\ee
where we have chosen the affine parameter to be equal to the proper time along the worldline and $\mathcal{E}_{L}^{\left(s\right)}\equiv \mathcal{E}_{a_1\dots a_{\ell}}^{\left(s\right)}$ are the multipole moments of the electric-type field strength tensors projected onto spatial slices orthogonal to the $d$-velocity $u^{\mu}=\frac{dx_{\text{cm}}^{\mu}}{d\tau}$ of the body defined via a set of local vielbein vectors $e_{a}^{\mu}$ satisfying $u_{\mu}e_{a}^{\mu}=0$. For $s=0,1,2$,
\be\ba
	\mathcal{E}_{L}^{\left(s=0\right)} &= e_{a_1}^{\mu_1}\dots e_{a_{\ell}}^{\mu_{\ell}}\nabla_{\langle \mu_1}\dots \nabla_{\mu_{\ell}\rangle}\Phi \,, \\
	\mathcal{E}_{L}^{\left(s=1\right)} &= e_{a_1}^{\mu_1}\dots e_{a_{\ell}}^{\mu_{\ell}}\nabla_{\langle \mu_1}\dots \nabla_{\mu_{\ell-1}}E_{\mu_{\ell}\rangle} \,,\quad E_{\mu} = u^{\nu}F_{\mu\nu} \,, \\
	\mathcal{E}_{L}^{\left(s=2\right)} &= e_{a_1}^{\mu_1}\dots e_{a_{\ell}}^{\mu_{\ell}}\nabla_{\langle \mu_1}\dots \nabla_{\mu_{\ell-2}}E_{\mu_{\ell-1}\mu_{\ell}\rangle} \,,\quad E_{\mu\nu} = u^{\rho}u^{\sigma}C_{\mu\rho\nu\sigma} \,,
\ea\ee
and they are by construction rank-$\ell$ STF spatial tensors. The Wilson coefficients $C_{\text{el},\ell}^{\left(s\right)}$ define the spin-$s$ electric-type static Love numbers. Although not written here explicitly, there is also a magnetic version of this interaction term defining the magnetic-type static Love numbers, while, for $s=2$ in $d>4$, one should furthermore take into account tensor-type gravitational perturbations which define the tensor-type tidal Love numbers, see e.g.~\cite{Hui:2020xxx}. In the rest of this section, we will focus to the electric-type responses for economy but all the analysis below can be straightforwardly applied for magnetic-type and tensor-type responses as well.

For a generic compact body, the Wilson coefficients $C_{\text{el},\ell}^{\left(s\right)}$ become ``Wilson tensors'' $C_{\text{el},LL^{\prime}}^{\left(s\right)}$ defining the static Love tensors\footnote{We are also assuming parity invariance of the geometry of the unperturbed body here, i.e. we are omitting mixing between electric and magnetic components which would otherwise be allowed.},
\be
	S_{\text{Love}}^{\text{electric}} = \sum_{s=0}^{2}\sum_{\ell,\ell^{\prime}=s}^{\infty}\int d\tau\,\frac{C_{\text{el},LL^{\prime}}^{\left(s\right)}}{2\ell!}\,\mathcal{E}^{\left(s\right)L}\left(x_{\text{cm}}\left(\tau\right)\right)\mathcal{E}^{\left(s\right)L^{\prime}}\left(x_{\text{cm}}\left(\tau\right)\right) \,.
\ee
The time dependent conservative responses can 
be captured by operators like $D\mathcal{E}D\mathcal{E}$
on the world line, where $D\equiv (d x^{\mu}_{\rm cm}/d\tau)\partial_\mu$. It is more practical, however, 
to switch to frequency space in this case, 
and consider the ``dynamical'' Love number
action of the form, 
\be
	S_{\text{dynamical Love}}^{\text{electric}} = \sum_{s=0}^{2}\sum_{\ell,\ell^{\prime}=s}^{\infty}\int\frac{d\omega}{2\pi}\,\frac{C_{\text{el},LL^{\prime}}^{\left(s\right)}\left(\omega\right)}{2\ell!}\,\mathcal{E}^{\left(s\right)L}\left(-\omega\right)\mathcal{E}^{\left(s\right)L^{\prime}}\left(\omega\right) \,.
\ee

To compute the Love tensors, one matches onto observables of the full theory. We will employ here the ``Newtonian matching'' condition consisting of inserting a pure $2^{\ell}$-pole background Newtonian source at large distances,
\be
	\Phi^{\left(s\right)}\left(\omega,\mathbf{x}\right) = \bar{\Phi}^{\left(s\right)}\left(\omega,\mathbf{x}\right) + \delta\Phi^{\left(s\right)}\left(\omega,\mathbf{x}\right) \,,\quad \bar{\Phi}^{\left(s\right)}\left(\omega,\mathbf{x}\right) = \frac{\left(\ell-s\right)!}{\ell!}\bar{\mathcal{E}}_{L}^{\left(s\right)}\left(\omega\right)x^{L} \,,
\ee
and matching EFT $1$-point functions onto microscopic computations of perturbation theory~\cite{Kol:2011vg,Charalambous:2021mea,Ivanov:2022hlo}. Diagrammatically\footnote{We are employing the dimensional regularization scheme here which sets all the unphysical power-law divergences to zero.},
\be\label{eq:1ptNewtonianMatching}
	\vev{\delta\Phi^{\left(s\right)}\left(\omega,\mathbf{x}\right)} = \vcenter{\hbox{\begin{tikzpicture}
				\begin{feynman}
					\vertex (a0);
					\vertex[right=0.6cm of a0] (gblobaux);
					\vertex[left=0.00cm of gblobaux, blob] (gblob){};
					\vertex[below=1cm of a0] (p1);
					\vertex[above=1cm of a0] (p2);
					\vertex[right=1cm of p1] (a1);
					\vertex[right=1cm of p2] (a2);%{$\times$};
					\vertex[right=0.69cm of p2] (a22){$\times$};
					\diagram*{
						(p1) -- [double,double distance=0.5ex] (p2),
						(a1) -- (gblob) -- (a2),
					};
				\end{feynman}
	\end{tikzpicture}}} =
	\underbrace{\vcenter{\hbox{\begin{tikzpicture}
					\begin{feynman}
						\vertex[dot] (a0);
						\vertex[below=1cm of a0] (p1);
						\vertex[above=1cm of a0] (p2);
						\vertex[right=0.4cm of a0, blob] (gblob){};							
						\vertex[right=1.5cm of p1] (b1);
						\vertex[right=1.5cm of p2] (b2);
						\vertex[right=1.19cm of p2] (b22){$\times$};
						\vertex[above=0.7cm of a0] (g1);
						\vertex[above=0.4cm of a0] (g2);
						\vertex[right=0.05cm of a0] (gdtos){$\vdots$};
						\vertex[below=0.7cm of a0] (gN);
						\diagram*{
							(p1) -- [double,double distance=0.5ex] (p2),
							(g1) -- [photon] (gblob),
							(g2) -- [photon] (gblob),
							(gN) -- [photon] (gblob),
							(b1) -- (gblob) -- (b2),
						};
					\end{feynman}
	\end{tikzpicture}}}}_{\text{``source''}} + 
	\underbrace{\vcenter{\hbox{\begin{tikzpicture}
					\begin{feynman}
						\vertex[dot] (a0);
						\vertex[left=0.00cm of a0] (lambda){$C_{\text{el},LL^{\prime}}^{\left(s\right)}\left(\omega\right)$};
						\vertex[below=1.6cm of a0] (p1);
						\vertex[above=0.4cm of a0] (p2);
						\vertex[right=1.5cm of p1] (b1);
						\vertex[right=1.5cm of p2] (b2);
						\vertex[right=1.19cm of p2] (b22){$\times$};
						\vertex[below=0.5cm of a0] (g1);
						\vertex[below=0.3cm of g1] (g2);
						\vertex[below=0.15cm of g2] (gdotsaux);
						\vertex[right=0.00cm of gdotsaux] (gdtos){$\vdots$};
						\vertex[below=0.5cm of gdotsaux] (gN);
						\vertex[below=0.7cm of a0] (gblobaux);
						\vertex[right=0.4cm of gblobaux, blob] (gblob){};
						\diagram*{
							(p1) -- [double,double distance=0.5ex] (p2),
							(b2) -- (a0) -- (gblob) -- (b1),
							(g1) -- [photon] (gblob),
							(g2) -- [photon] (gblob),
							(gN) -- [photon] (gblob),
						};
					\end{feynman}
	\end{tikzpicture}}}}_{\text{``response''}} \,,
\ee
where the double line represents the worldline, straight lines indicate propagators of the fields $\delta\Phi^{\left(s\right)}$, a ``$\times$'' represents a $\bar{\Phi}^{\left(s\right)}$ insertion and wavy lines correspond to interactions of the worldline with the graviton arising from the minimal point-particle action. We note here that we are not including dissipative effects which will be addressed shortly. In the above diagrammatic representation we have also demonstrated how the worldline EFT definition allows to unambiguously separate relativistic corrections in the ``source'' part of the field profile from tidal effects~\cite{Charalambous:2021mea,Ivanov:2022hlo}. This splitting is in fact equivalent to the method of analytically continuing the spacetime dimensionality $d$~\cite{Kol:2011vg} or the multipolar order $\ell$~\cite{LeTiec:2020bos,Charalambous:2021mea,Ivanov:2022hlo,Creci:2021rkz} as the ``source'' and ``response'' diagrams have indicial powers $r^{\alpha}$ with $\alpha=\ell$ and $\alpha=-\left(\ell+d-3\right)$ respectively. These receive PN corrections from the interaction of the graviton with the worldline which have the form $r^{\alpha-n}$ with positive integer $n$.

In the Newtonian limit, in a gauge where the fields $\delta\Phi^{\left(s\right)}$ are canonical variables up to an overall normalization constant $N^{\left(s\right)}_{\text{prop}}$ in momentum space,
\be
	\vev{\delta\Phi^{\left(s\right)}\delta\Phi^{\left(s\right)}}\left(p\right) = N^{\left(s\right)}_{\text{prop}}\frac{-i}{p^2} \,,
\ee
and in the body centered frame where $x_{\text{cm}}=\left(t,\mathbf{0}\right)$ and $u^{\mu}=\left(1,\mathbf{0}\right)$, this gives
\be\ba
	{}&\vev{\delta\Phi^{\left(s\right)}\left(\omega,\mathbf{x}\right)} \rightarrow
		\vcenter{\hbox{\begin{tikzpicture}
						\begin{feynman}
							\vertex[dot] (a0);
							\vertex[below=1cm of a0] (p1);
							\vertex[above=1cm of a0] (p2);						
							\vertex[right=1cm of p1] (b1);
							\vertex[right=1cm of p2] (b2);
							\vertex[right=0.69cm of p2] (b22){$\times$};
							\diagram*{
								(p1) -- [double,double distance=0.5ex] (p2),
								(b1) -- (b2),
							};
						\end{feynman}
		\end{tikzpicture}}} + 
		\vcenter{\hbox{\begin{tikzpicture}
						\begin{feynman}
							\vertex[dot] (a0);
							\vertex[left=0.00cm of a0] (lambda){$C_{\text{el},LL^{\prime}}^{\left(s\right)}\left(\omega\right)$};
							\vertex[below=1cm of a0] (p1);
							\vertex[above=1cm of a0] (p2);
							\vertex[right=1.2cm of p1] (b1);
							\vertex[right=1.2cm of p2] (b2);
							\vertex[right=0.89cm of p2] (b22){$\times$};
							\diagram*{
								(p1) -- [double,double distance=0.5ex] (p2),
								(b2) -- (a0) -- (b1),
							};
						\end{feynman}
		\end{tikzpicture}}} \\
		&= \frac{\left(\ell-s\right)!}{\ell!} \sum_{\ell^{\prime}=s}^{\infty}\left[\delta_{L,L^{\prime}} + \frac{2^{\ell-2}\Gamma\left(\ell+\frac{d-3}{2}\right)}{\pi^{\left(d-1\right)/2}}N^{\left(s\right)}_{\text{prop}}\frac{\left[C_{\text{el},LL^{\prime}}^{\left(s\right)}\left(\omega\right)\right]_{\text{TRS}}}{r^{2\ell+d-3}}\right]\bar{\mathcal{E}}^{\left(s\right)L^{\prime}}\left(\omega\right)x^{L} \,,
\ea\ee
where,
\be
	\left[C_{\text{el},LL^{\prime}}^{\left(s\right)}\left(\omega\right)\right]_{\text{TRS}}\equiv\frac{1}{2}\left(C_{\text{el},LL^{\prime}}^{\left(s\right)}\left(\omega\right)+C_{\text{el},L^{\prime}L}^{\left(s\right)}\left(-\omega\right)\right) \,,
\ee
with ``TRS'' standing for time-reversal symmetric. From this, we identify the explicit correspondence between the electric-type Love tensor and the Wilson tensor for a compact body of size $\mathcal{R}$,
\be\label{eq:WTesnorConsResp}
	k_{LL^{\prime}}^{\left(s\right)\text{Love}}\left(\omega\right) = \frac{2^{\ell-2}\Gamma\left(\ell+\frac{d-3}{2}\right)}{\pi^{\left(d-1\right)/2}}N^{\left(s\right)}_{\text{prop}}\frac{\left[C_{\text{el},LL^{\prime}}^{\left(s\right)}\left(\omega\right)\right]_{\text{TRS}}}{\mathcal{R}^{2\ell+d-3}} \,.
\ee
We see therefore that the Love tensor is defined from the \textit{conservative} response, i.e. the part of the response tensor invariant under the time-reversal transformations which corresponds to simultaneously flipping the sign of the frequency, $\omega\rightarrow-\omega$, and the exchange $L\leftrightarrow L^{\prime}$. This is implicit by the definition at the level of the action and the use of the in-out formalism since
\be\ba
	\sum_{\ell,\ell^{\prime}}\int\frac{d\omega}{2\pi}&\,\frac{\left[C_{\text{el},LL^{\prime}}^{\left(s\right)}\left(\omega\right)\right]_{\text{TRS}}}{2\ell!}\,\mathcal{E}^{\left(s\right)L}\left(-\omega\right)\mathcal{E}^{\left(s\right)L^{\prime}}\left(\omega\right) = \\
	&\sum_{\ell,\ell^{\prime}}\int\frac{d\omega}{2\pi}\,\frac{C_{\text{el},LL^{\prime}}^{\left(s\right)}\left(\omega\right)}{2\ell!}\,\mathcal{E}^{\left(s\right)L}\left(-\omega\right)\mathcal{E}^{\left(s\right)L^{\prime}}\left(\omega\right) \,.
\ea\ee


\subsubsection{Dissipation in EFT}
As we just saw, only $\left[C_{\text{el},LL^{\prime}}^{\left(s\right)}\left(\omega\right)\right]_{\text{TRS}}$ is relevant when computing $1$-point functions via the standard in-out formalism, i.e. local operators in the worldline EFT action capture only conservative effects. Dissipative effects are incorporated by introducing gapless internal degrees of freedom $X$. One then considers composite operators $Q_{L}^{\left(s\right)}\left(X\right)$ corresponding to the full multipole moments, including the dissipative multipole moments due to the internal degrees of freedom $X$, but whose exact dependence on $X$ is not known. These are then coupled to the field moments~\cite{Goldberger:2005cd,Goldberger:2019sya,Goldberger:2020fot,Goldberger:2020wbx},
\be
	S_{\text{diss}} = \sum_{s=0}^{2}\sum_{\ell=s}^{\infty}\frac{1}{\ell!}\int d\tau\,Q_{L}^{\left(s\right),\mathcal{E}}\left(X\right)\mathcal{E}^{\left(s\right)L}\left(x_{\text{cm}}\left(\tau\right)\right) + \left(\mathcal{E}\leftrightarrow\mathcal{B},\mathcal{T}\right) \,.
\ee
In order to account for dissipative effects at the level of the $1$-point function, one then employs the in-in (Schwinger-Keldysh) formalism~\cite{Schwinger:1960qe,Keldysh:1964ud,Goldberger:2005cd,Goldberger:2019sya,Goldberger:2020fot,Goldberger:2020wbx}. Within this framework~\cite{Ivanov:2022hlo},
\be\ba\label{eq:1ptInInResponse}
    {}&\vev{\delta\Phi^{\left(s\right)}\left(\omega,\mathbf{x}\right)} \supset
	\vcenter{\hbox{\begin{tikzpicture}
			\begin{feynman}
				\vertex[] (a0){};
				\vertex[left=0.2cm of a0] (a00);
				\vertex[above=0.4cm of a00] (a0t){};
				\vertex[below=0.4cm of a00] (a0b){};
				\vertex[left=0.4cm of a0] (p0);
				\vertex[above=0.4cm of p0] (p0t){$Q^{\left(s\right)}$};
				\vertex[below=0.4cm of p0] (p0b){$Q^{\left(s\right)}$};
				\vertex[below=1.5cm of a0] (p1);
				\vertex[above=1.5cm of a0] (p2);
				\vertex[below=0.6cm of a0] (p11);
				\vertex[above=0.6cm of a0] (p22);
				\vertex[below right=2cm of a0] (a1);
				\vertex[above right=2cm of a0] (a2);%{$\times$};
				\vertex[above right=2cm of a0] (a22){$\times$};
				\diagram*{
					(p1) -- [double,double distance=0.5ex] (p2),
					(a1) -- (a0b) , (a0t) -- (a2),
					(p11) -- [ghost](p22)
				};
			\end{feynman}
	\end{tikzpicture}}} \\
	&= \frac{\left(\ell-s\right)!}{\ell!}\sum_{\ell^{\prime}=s}^{\infty}\left[\frac{2^{\ell-2}\Gamma\left(\ell+\frac{d-3}{2}\right)}{\pi^{\left(d-1\right)/2}}\frac{\vev{Q_{L}^{\left(s\right),\mathcal{E}}Q_{L^{\prime}}^{\left(s\right),\mathcal{E}}}\left(-\omega\right)}{r^{2\ell+d-3}}\right]\bar{\mathcal{E}}^{\left(s\right)L^{\prime}}\left(\omega\right)x^{L}
	\,,
\ea\ee
and the full electric-type response tensors $k_{LL^{\prime}}^{\left(s\right)}\left(\omega\right)$ are matched onto the (retarded) $2$-point function~\cite{Ivanov:2022hlo},
\be\label{eq:WTesnorFullResp}
	k_{LL^{\prime}}^{\left(s\right)}\left(\omega\right) = \frac{2^{\ell-2}\Gamma\left(\ell+\frac{d-3}{2}\right)}{\pi^{\left(d-1\right)/2}}\frac{\vev{Q_{L}^{\left(s\right),\mathcal{E}}Q_{L^{\prime}}^{\left(s\right),\mathcal{E}}}\left(-\omega\right)}{\mathcal{R}^{2\ell+d-3}} \,.
\ee
Note that the real part of $k_{LL^{\prime}}^{\left(s\right)}$
is indistinguishable from $k_{LL^{\prime}}^{\left(s\right)\text{Love}}$
and therefore can be ignored. Its imaginary part 
though cannot be reproduced from a local 
world line action and thus it encodes 
non-conservative effects such as horizon absorption.


\subsection{Love numbers of 5-d rotating bodies}
So far we have formulated the response problem in terms of the response tensors $k_{LL^{\prime}}^{\left(s\right)}$. In practice, one is interested in the harmonic response coefficients arising after performing a harmonic expansion thanks to the $1$-to-$1$ correspondence between spatial STF tensors and spherical harmonics. From these, the Love numbers are identified as the conservative harmonic response coefficients~\cite{Charalambous:2021mea,Ivanov:2022hlo}. Isolating the conservative part of the harmonic response coefficients is in general non-trivial, but for some particular configurations, e.g. the remarkably integrable black hole perturbations, this decomposition allows one to identify the Love numbers as the real part of the harmonic response coefficients, while the imaginary part captures dissipative effects.

While this can be done in $d=1+3$ spacetime dimensions by performing an expansion into spherical harmonics over $\mathbb{S}^2$~\cite{LeTiec:2020bos}, it fails to work for a general rotating body in higher spacetime dimensions. For $d>4$, an expansion into spherical harmonics on $\mathbb{S}^{d-2}$ allows us to extract a simple isolating prescription of the conservative part of the response coefficients only for spherically symmetric and non-rotating bodies. For axisymmetric distributions, one should instead perform a modified harmonic expansion over $\left[\mathbb{S}^1\right]^{N}\subset \mathbb{S}^{d-2}$, with $N=\left\lfloor\frac{d-1}{2}\right\rfloor$ factors of $\mathbb{S}^1$, appropriate for the isometry subgroup $\left[U\left(1\right)\right]^{N}\subset SO\left(d-1\right)$ of such configurations. In $d=5$ spacetime dimensions, this is a modified harmonic expansion over the $\mathbb{S}^1\times\mathbb{S}^1$ part of $\mathbb{S}^3$ in accordance with the $U\left(1\right)\times U\left(1\right)$ azimuthal symmetries. This modified spherical harmonics basis for $d=5$ is introduced and analyzed in Appendix \ref{sec:ApSphericalHarmonics}.

To this end, we begin by expanding the $4$-dimensional spatial STF tensors $\bar{\mathcal{E}}_{L}^{\left(s\right)}$ into modified spherical harmonics of orbital number $\ell\in\mathbb{N}$,
\be
	\bar{\mathcal{E}}^{\left(s\right)L} = \sum_{m,j}\bar{\mathcal{E}}_{\ell m j}^{\left(s\right)}\mathcal{Y}_{\ell m j}^{L\ast} \,,
\ee
where the constant STF tensors $\mathcal{Y}_{\ell m j}^{L}$ are given by
\be
	\mathcal{Y}_{\ell m j}^{L} = \frac{\left(2\ell+2\right)!!}{4\pi^2\ell!}\oint_{\mathbb{S}^3}d\Omega_3\,\Omega^{\left\langle L \right\rangle}\tilde{Y}_{\ell m j}^{\ast}\left(\mathbf{\Omega}\right) \,,
\ee
with $\tilde{Y}_{\ell m j}\left(\mathbf{\Omega}\right)\equiv \tilde{Y}_{\ell m j}\left(\theta,\phi,\psi\right)$ the modified spherical harmonics on $\mathbb{S}^3$, $\Omega^{i}\equiv x^{i}/r$ and asterisks indicate complex conjugation. For future reference, the explicit limits of the sums over the azimuthal numbers $m$ and $j$ are
\be
	\sum_{m,j}\left(\dots\right)\equiv \sum_{m=-\ell}^{\ell}\left(\sum_{j=-\left(\ell-\left|m\right|\right),2}^{\ell-\left|m\right|}\left(\dots\right)\right) \,.
\ee
We note that the $j$-sum is being performed with a step $2$. This is merely a convention chosen such that the azimuthal number $m$ resembles the usual azimuthal number of scalar spherical harmonics on $\mathbb{S}^2$.

Then, the response coefficients $k_{\ell m j;\ell^{\prime} m^{\prime} j^{\prime}}^{\left(s\right)}\left(\omega\right)$ are related to the response tensor $k_{LL^{\prime}}^{\left(s\right)}\left(\omega\right)$ according to
\be\label{eq:RNsToRTs}
	k_{\ell m j;\ell^{\prime} m^{\prime} j^{\prime}}^{\left(s\right)}\left(\omega\right) = \frac{4\pi^2\ell!}{\left(2\ell+2\right)!!}k_{LL^{\prime}}^{\left(s\right)}\left(\omega\right)\mathcal{Y}_{\ell m j}^{L}\mathcal{Y}_{\ell^{\prime} m^{\prime} j^{\prime}}^{L^{\prime}\ast} \,.
\ee
Using the fact that the induced multipole moments $\delta Q_{L}^{\left(s\right)}\left(t\right)$ and source multipole moments $\mathcal{E}_{L}^{\left(s\right)}\left(t\right)$ are real in position space as well as the assumption that the response tensors $k_{LL^{\prime}}^{\left(s\right)}\left(\omega\right)$ are analytic in $\omega$, i.e. that
\be
	k_{LL^{\prime}}^{\left(s\right)}\left(\omega\right) = \sum_{n=0}^{\infty}k_{LL^{\prime};n}^{\left(s\right)}\left(i\omega\right)^{n} \,,
\ee
with real-valued modes $k_{LL^{\prime};n}^{\left(s\right)}$, we see that,
\be
	k_{LL^{\prime}}^{\left(s\right)\ast}\left(\omega\right)=k_{LL^{\prime}}^{\left(s\right)}\left(-\omega\right) \,.
\ee
From the complex conjugacy relation of the modified spherical harmonics, $\tilde{Y}_{\ell m j}^{\ast}=\tilde{Y}_{\ell,-m,-j}$, we then deduce the following complex conjugacy relation for the response coefficients
\be\label{eq:klmCC}
	k_{\ell m j;\ell^{\prime} m^{\prime} j^{\prime}}^{\left(s\right)\ast}\left(\omega\right) = k_{\ell,-m,-j;\ell^{\prime},-m^{\prime},-j^{\prime}}^{\left(s\right)}\left(-\omega\right) \,.
\ee

We can now translate the conservative/dissipative decomposition of the response tensor (see \eqref{eq:WTesnorConsResp},\eqref{eq:WTesnorFullResp}),
\be\ba
	k_{LL^{\prime}}^{\left(s\right)\text{Love}}\left(\omega\right) &= \frac{1}{2}\left(k_{LL^{\prime}}^{\left(s\right)}\left(\omega\right) + k_{L^{\prime}L}^{\left(s\right)}\left(-\omega\right)\right) \,, \\
	k_{LL^{\prime}}^{\left(s\right)\text{diss}}\left(\omega\right) &= \frac{1}{2}\left(k_{LL^{\prime}}^{\left(s\right)}\left(\omega\right) - k_{L^{\prime}L}^{\left(s\right)}\left(-\omega\right)\right) \,,
\ea\ee
at the level of the response coefficients $k_{\ell mj;\ell^{\prime}m^{\prime}j^{\prime}}^{\left(s\right)}\left(\omega\right)$. The definition \eqref{eq:RNsToRTs} and the complex conjugacy relation \eqref{eq:klmCC} immediately imply
\be\ba\label{eq:RCsConsDissGen}
	k_{\ell mj;\ell^{\prime}m^{\prime}j^{\prime}}^{\left(s\right)\text{Love}}\left(\omega\right) &= \frac{1}{2}\left(k_{\ell mj;\ell^{\prime}m^{\prime}j^{\prime}}^{\left(s\right)}\left(\omega\right)+k_{\ell^{\prime}m^{\prime}j^{\prime};\ell mj}^{\left(s\right)\ast}\left(\omega\right)\right) \,, \\
	k_{\ell mj;\ell^{\prime}m^{\prime}j^{\prime}}^{\left(s\right)\text{diss}}\left(\omega\right) &= \frac{1}{2i}\left(k_{\ell mj;\ell^{\prime}m^{\prime}j^{\prime}}^{\left(s\right)}\left(\omega\right)-k_{\ell^{\prime}m^{\prime}j^{\prime};\ell mj}^{\left(s\right)\ast}\left(\omega\right)\right) \,,
\ea\ee
such that
\be
	k_{\ell mj;\ell^{\prime}m^{\prime}j^{\prime}}^{\left(s\right)}\left(\omega\right) = k_{\ell mj;\ell^{\prime}m^{\prime}j^{\prime}}^{\left(s\right)\text{Love}}\left(\omega\right) + ik_{\ell mj;\ell^{\prime}m^{\prime}j^{\prime}}^{\left(s\right)\text{diss}}\left(\omega\right) \,.
\ee
We note, however, that $k_{\ell mj;\ell^{\prime}m^{\prime}j^{\prime}}^{\left(s\right)\text{Love}}\left(\omega\right)$ and $k_{\ell mj;\ell^{\prime}m^{\prime}j^{\prime}}^{\left(s\right)\text{diss}}\left(\omega\right)$ are in general complex numbers.

We now focus to axisymmetric configurations. The axisymmetry of the background implies the decoupling of $m$-modes and $j$-modes, while we further specialize here to the particular case where there is no $\ell$-mode mixing either, a case relevant for Myers-Perry black holes. Then,
\be
	k_{\ell m j;\ell^{\prime} m^{\prime} j^{\prime}}^{\left(s\right)}\left(\omega\right) = k_{\ell m j}^{\left(s\right)}\left(\omega\right)\delta_{\ell\ell^{\prime}}\delta_{mm^{\prime}}\delta_{jj^{\prime}} \,,
\ee
and the frequency space potential perturbation harmonic modes in the Newtonian limit simplify to
\be
	\delta\Phi^{\left(s\right)}_{\ell m j}\left(\omega,r\right) = \frac{\left(\ell-s\right)!}{\ell!}\left[1 + k_{\ell m j}^{\left(s\right)}\left(\omega\right)\left(\frac{\mathcal{R}}{r}\right)^{2\ell+2}\right]r^{\ell}\bar{\mathcal{E}}_{\ell m j}^{\left(s\right)}\left(\omega\right) \,.
\ee

These response coefficients $k_{\ell m j}^{\left(s\right)}\left(\omega\right)$ will in general be analytic functions in the angular momenta of the rotating body as well as the frequency $\omega$ with respect to an inertial observer. The complex conjugacy relation, which now reads $k_{\ell m j}^{\left(s\right)\ast}\left(\omega\right) = k_{\ell,-m,-j}^{\left(s\right)}\left(-\omega\right)$, then allows to explicitly separate the $m$- and $j$-dependencies of the response coefficients as
\be\label{eq:TLNs5d_mjExpansion}
	k_{\ell m j}^{\left(s\right)}\left(\omega\right) = k_{\ell}^{\left(0\right)}\left(\omega\right) + \chi\sum_{n_{\phi}=1}^{\infty}\sum_{n_{\psi}=1}^{\infty}k_{\ell}^{(n_{\phi},n_{\psi})}\left(\omega,\chi_{\phi},\chi_{\psi}\right) \left(im\right)^{n_{\phi}}\left(ij\right)^{n_{\psi}} \,,
\ee
with $\chi_{\phi}$ and $\chi_{\psi}$ the dimensionless spin parameters associated with the $J_{\phi}$ and $J_{\psi}$ angular momenta and the overall formal $\chi$ is to separate the non-spinning part $k_{\ell}^{\left(0\right)}\left(\omega\right)$. All $k_{\ell}^{(n_{\phi},n_{\psi})}\left(\omega,\chi_{\phi},\chi_{\psi}\right)$ are smooth functions of $\chi_{\phi}$ and $\chi_{\psi}$, satisfying the complex conjugation relation $k_{\ell}^{(n_{\phi},n_{\psi})\ast}\left(\omega,\chi_{\phi},\chi_{\psi}\right)=k_{\ell}^{(n_{\phi},n_{\psi})}\left(-\omega,\chi_{\phi},\chi_{\psi}\right)$.

Let us now extract a necessary condition for such a decoupling to occur. This analysis is the $d=5$ version of~\cite{LeTiec:2020bos}. Starting from the physically relevant part of the response tensor,
\be
	k_{L\left\langle L^{\prime} \right\rangle}^{\left(s\right)}\left(\omega\right) = \frac{4\pi^2\ell!}{\left(2\ell+2\right)!!}\sum_{m,j}k_{\ell m j}^{\left(s\right)}\left(\omega\right)\mathcal{Y}^{\ell m j\ast}_{L}\mathcal{Y}^{\ell m j}_{L^{\prime}} \,,
\ee
with $\ell^{\prime}=\ell$ understood, the expansion \eqref{eq:TLNs5d_mjExpansion} implies
\be\ba
	{}&k_{L\left\langle L^{\prime} \right\rangle}^{\left(s\right)}\left(\omega\right) = k_{\ell}^{\left(0\right)}\left(\omega\right)\delta_{L,L^{\prime}} + \chi\sum_{n_{\phi}=1}^{\infty}\sum_{n_{\psi}=1}^{\infty} \left(-1\right)^{n_{\phi}+n_{\psi}} \\
	&\times\bigg[k_{\ell}^{(2n_{\phi},2n_{\psi})}\left(\omega,\chi_{\phi},\chi_{\psi}\right)R_{LL^{\prime}}^{(2n_{\phi},2n_{\psi})} - k_{\ell}^{(2n_{\phi}-1,2n_{\psi}-1)}\left(\omega,\chi_{\phi},\chi_{\psi}\right)R_{LL^{\prime}}^{(2n_{\phi}-1,2n_{\psi}-1)} \\
	&+ k_{\ell}^{(2n_{\phi}-1,2n_{\psi})}\left(\omega,\chi_{\phi},\chi_{\psi}\right)I_{LL^{\prime}}^{(2n_{\phi}-1,2n_{\psi})} + k_{\ell}^{(2n_{\phi},2n_{\psi}-1)}\left(\omega,\chi_{\phi},\chi_{\psi}\right)I_{LL^{\prime}}^{(2n_{\phi},2n_{\psi}-1)}\bigg] \,,
\ea\ee
and the tensorial structure of $k_{L\left\langle L^{\prime} \right\rangle}^{\left(s\right)}$ is completely determined by two real-valued symmetric and two real-valued antisymmetric STF tensors,
\be\ba
	R_{LL^{\prime}}^{(2n_{\phi},2n_{\psi})} &\equiv \frac{8\pi^2\ell!}{\left(2\ell+2\right)!!}\sum_{m=1}^{\ell}\left(\sum_{j=-\left(\ell-m\right),2}^{\ell-m}m^{2n_{\phi}}j^{2n_{\psi}}\text{Re}\left\{\mathcal{Y}^{\ell mj\ast}_{L}\mathcal{Y}^{\ell mj}_{L^{\prime}}\right\}\right) \,, \\
	R_{LL^{\prime}}^{(2n_{\phi}-1,2n_{\psi}-1)} &\equiv \frac{8\pi^2\ell!}{\left(2\ell+2\right)!!}\sum_{m=1}^{\ell}\left(\sum_{j=-\left(\ell-m\right),2}^{\ell-m}m^{2n_{\phi}-1}j^{2n_{\psi}-1}\text{Re}\left\{\mathcal{Y}^{\ell mj\ast}_{L}\mathcal{Y}^{\ell mj}_{L^{\prime}}\right\}\right) \,, \\
	I_{LL^{\prime}}^{(2n_{\phi}-1,2n_{\psi})} &\equiv \frac{8\pi^2\ell!}{\left(2\ell+2\right)!!}\sum_{m=1}^{\ell}\left(\sum_{j=-\left(\ell-m\right),2}^{\ell-m}m^{2n_{\phi}-1}j^{2n_{\psi}}\text{Im}\left\{\mathcal{Y}^{\ell mj\ast}_{L}\mathcal{Y}^{\ell mj}_{L^{\prime}}\right\}\right) \,, \\
	I_{LL^{\prime}}^{(2n_{\phi},2n_{\psi}-1)} &\equiv \frac{8\pi^2\ell!}{\left(2\ell+2\right)!!}\sum_{m=1}^{\ell}\left(\sum_{j=-\left(\ell-m\right),2}^{\ell-m}m^{2n_{\phi}}j^{2n_{\psi}-1}\text{Im}\left\{\mathcal{Y}^{\ell mj\ast}_{L}\mathcal{Y}^{\ell mj}_{L^{\prime}}\right\}\right) \,,
\ea\ee
\be\ba
	R_{LL^{\prime}}^{(2n_{\phi},2n_{\psi})} = +R_{L^{\prime}L}^{(2n_{\phi},2n_{\psi})} \,,\quad R_{LL^{\prime}}^{(2n_{\phi}-1,2n_{\psi}-1)} = +R_{L^{\prime}L}^{(2n_{\phi}-1,2n_{\psi}-1)} \,, \\
	I_{LL^{\prime}}^{(2n_{\phi}-1,2n_{\psi})} = -I_{L^{\prime}L}^{(2n_{\phi}-1,2n_{\psi})} \,, \quad I_{LL^{\prime}}^{(2n_{\phi},2n_{\psi}-1)} = -I_{L^{\prime}L}^{(2n_{\phi},2n_{\psi}-1)} \,.
\ea\ee

Finally, let us write the conservative/dissipative decomposition of the response coefficients \eqref{eq:RCsConsDissGen} for the current special configuration, which is also the main result of interest of this analysis,
\be\ba\label{eq:RCsConsDiss}
	k_{\ell mj}^{\left(s\right)\text{Love}}\left(\omega\right) &= \text{Re}\left\{ k_{\ell mj}^{\left(s\right)}\left(\omega\right) \right\} \,, \\
	k_{\ell mj}^{\left(s\right)\text{diss}}\left(\omega\right) &= \text{Im}\left\{ k_{\ell mj}^{\left(s\right)}\left(\omega\right) \right\} \,.
\ea\ee
The Love numbers are therefore just the real part of the response coefficients, while the imaginary part encodes all the dissipative effects. We remark here that dissipative effects can survive even in the static limit due to frame dragging~\cite{Chia:2020yla,Charalambous:2021mea,Ivanov:2022hlo}.
