\section{Modified Spherical Harmonics basis}
\label{sec:ApSphericalHarmonics}

In this appendix, we present the basis used for the modified harmonic expansion, naturally associated with the computation of Love numbers for axisymmetric distributions in $d=5$ spacetime dimensions. For the sake of this, we need to perform a harmonic expansion over the $\mathbb{S}^1\times\mathbb{S}^1$ subpart of $\mathbb{S}^3$, appropriate for the general isometry group factor $U\left(1\right)\times U\left(1\right)$ of such configurations. The basis we are looking for is extracted by solving the eigenvalue problem for the Laplace-Beltrami operator on $\mathbb{S}^3$ expressed in the direction cosine angular coordinates appearing in the Myers-Perry black hole line element,
\be\label{eq:LaplaceBeltramiEigenvalue}
	\left[\frac{1}{\sin\theta\cos\theta}\partial_{\theta}\left(\sin\theta\cos\theta\,\partial_{\theta}\right) + \frac{1}{\sin^2\theta}\,\partial_{\phi}^2 + \frac{1}{\cos^2\theta}\,\partial_{\psi}^2\right] \tilde{Y}_{\lambda}\left(\theta,\phi,\psi\right) = \lambda \tilde{Y}_{\lambda}\left(\theta,\phi,\psi\right) \,.
\ee

Let us present first how this is indeed the Laplace-Beltrami operator on $\mathbb{S}^3$. In usual spherical coordinates $\left(r,\vartheta_1,\vartheta_2,\varphi\right)$, the $4$-dimensional spatial position vector with components $\left(x^1,x^2,x^3,x^4\right)$ in Cartesian coordinates is written as
\be\ba
	x^1 &= r\cos\vartheta_1 \,, \\
	x^2 &= r\sin\vartheta_1\cos\vartheta_2 \,, \\
	x^3 &= r\sin\vartheta_1\sin\vartheta_2\cos\varphi \,, \\
	x^4 &= r\sin\vartheta_1\sin\vartheta_2\sin\varphi \,,
\ea\ee
where $x^1$ plays the role of the $3$-dimensional $z$-axis. In these coordinates, $\vartheta_1\in\left[0,\pi\right]$ and $\vartheta_2\in\left[0,\pi\right]$ are two polar angles, while $\varphi\in\left[0,2\pi\right)$ is a periodically identified azimuthal angle. The Laplacian operator acting on scalar functions then reads
\be
	\triangle_{4} = \partial_{1}^2 + \partial_{2}^2 + \partial_{3}^2 + \partial_{4}^2 = \frac{1}{r^3}\partial_{r}\left(r^3\partial_{r}\right) + \frac{1}{r^2}\triangle_{\mathbb{S}^3}^{\left(0\right)} \,,
\ee
with $\triangle_{\mathbb{S}^3}^{\left(0\right)}$ the Laplace-Beltrami operator on $\mathbb{S}^3$ acting on scalar ($s=0$) functions, which in spherical coordinates is given explicitly by
\be
	\triangle_{\mathbb{S}^3}^{\left(0\right)} = \frac{1}{\sin^2\vartheta_1}\left[\partial_{\vartheta_1}\left(\sin^2\vartheta_1\,\partial_{\vartheta_1}\right) + \frac{1}{\sin\vartheta_2}\partial_{\vartheta_2}\left(\sin\vartheta_2\,\partial_{\vartheta_2}\right) + \frac{1}{\sin^2\vartheta_2}\,\partial_{\varphi}^2\right] \,.
\ee
In order to transit to direction cosine coordinates $\left(r,\theta,\phi,\psi\right)$, we split the four Cartesian coordinates $x^{i}$ into two pairs of two and project onto these two planes of rotation,
\be\ba
	x^1 &= r\cos\theta\cos\psi \,, \\
	x^2 &= r\cos\theta\sin\psi \,, \\
	\\
	x^3 &= r\sin\theta\cos\phi \,, \\
	x^4 &= r\sin\theta\sin\phi \,,
\ea\ee
with $\theta\in\left[0,\frac{\pi}{2}\right]$ a direction cosine angle, while $\phi\in\left[0,2\pi\right)$ and $\psi\in\left[0,2\pi\right)$ are two periodically identified azimuthal angles. Then, the Laplace operator on scalar functions has the same form as in spherical coordinates, but with the Laplace-Beltrami operator now identified as
\be
	\triangle_{\mathbb{S}^3}^{\left(0\right)} = \frac{1}{\sin\theta\cos\theta}\partial_{\theta}\left(\sin\theta\cos\theta\,\partial_{\theta}\right) + \frac{1}{\sin^2\theta}\,\partial_{\phi}^2 + \frac{1}{\cos^2\theta}\,\partial_{\psi}^2 \,.
\ee
In particular, the transformation rule between spherical coordinates $\left(\vartheta_1,\vartheta_2,\varphi\right)$ and direction cosine coordinates $\left(\theta,\phi,\psi\right)$ allows to identify $\phi=\varphi$, while $\psi$ and $\theta$ are related to $\vartheta_1$ and $\vartheta_2$ according to
\be
	\sin\theta = \sin\vartheta_1\sin\vartheta_2\,\,\,,\,\,\,\tan\psi = \tan\vartheta_1\cos\vartheta_2 \,.
\ee
The generators of the algebra $\mathfrak{so}\left(4\right)\simeq\mathfrak{so}\left(3\right)\oplus\mathfrak{so}\left(3\right)$ in direction cosine coordinates are more transparent by introducing the sum/difference azimuthal angles $\psi_{\pm}=\psi\pm\phi$ and they are organized in the two commuting $\mathfrak{so}\left(3\right)$'s, labeled by a sign $\sigma=+$ or $\sigma=-$,
\be
	\begin{gathered}
		J_0^{\left(\sigma\right)} = -i\partial_{\sigma} \,, \\
		J_{\pm1}^{\left(\sigma\right)} = e^{\pm i\psi_{\sigma}}\left[\partial_{2\theta} \pm i\cot2\theta\,\partial_{\sigma} \mp\frac{i}{\sin2\theta}\,\partial_{-\sigma}\right] \,, \\
		\left[J_{\pm1}^{\sigma},J_0^{\sigma^{\prime}}\right] = \mp J_{\pm1}^{\left(\sigma\right)}\delta_{\sigma,\sigma^{\prime}} \,\,\,,\,\,\, \left[J_{\pm1}^{\sigma},J_{\mp1}^{\sigma^{\prime}}\right] = \mp 2J_{0}^{\left(\sigma\right)}\delta_{\sigma,\sigma^{\prime}} \,\,\,,\,\,\, \left[J_{\pm1}^{\sigma},J_{\pm1}^{\sigma^{\prime}}\right] = 0 \,.
	\end{gathered}
\ee

Returning to the eigenvalue problem \eqref{eq:LaplaceBeltramiEigenvalue}, this is reduced to a one-dimensional problem after separating the azimuthal angles as
\be
	\tilde{Y}_{\ell m j}\left(\theta,\phi,\psi\right) = S_{\ell m j}\left(\theta\right)\frac{e^{im\phi}}{\sqrt{2\pi}}\frac{e^{ij\psi}}{\sqrt{2\pi}} \,,
\ee
with the azimuthal numbers $m$ and $j$ being integers by virtue of the periodicity of the angles $\phi$ and $\psi$ with period $2\pi$, and $S_{\ell m j}\left(\theta\right)$ satisfying the first order ordinary differential equation
\be
	\left[\frac{1}{\sin\theta\cos\theta}\frac{d}{d\theta}\left(\sin\theta\cos\theta\,\frac{d}{d\theta}\right) - \frac{m^2}{\sin^2\theta} - \frac{j^2}{\cos^2\theta}\right]S_{\ell m j}\left(\theta\right) = -\ell\left(\ell+2\right)S_{\ell m j}\left(\theta\right) \,.
\ee
We remark here that we have set the eigenvalues to $\lambda \equiv -\ell\left(\ell+2\right)$ such that $\ell$ resembles the corresponding orbital quantum number appearing in scalar spherical harmonics on $\mathbb{S}^3$. However, at this point, $\ell$ is not restricted to be a whole number.

This differential equation must now be solved in parallel with the boundary condition of regularity of $S_{\ell m j}\left(\theta\right)$ along the full domain of the direction cosine angle, $\theta\in\left[0,\frac{\pi}{2}\right]$. The unique solution is not hard to extract,
\be
	S_{\ell m j}\left(\theta\right) = N_{\ell m j}\sin^{\left|m\right|}\theta\,\cos^{\left|j\right|}\theta\,{}_2F_1\left(-\frac{\ell-\left|m\right|-\left|j\right|}{2},\frac{\ell+\left|m\right|+\left|j\right|}{2}+1;1+\left|j\right|;\cos^2\theta\right) \,,
\ee
with $N_{\ell m j}$ a normalization constant, while regularity at $\theta=0$ imposes the discretization condition
\be
	\frac{\ell-\left|m\right|-\left|j\right|}{2} \in \mathbb{N}_0 = \left\{0,1,2,\dots\right\} \,.
\ee
We can, thus, assign whole numbers to $\ell$ as in the usual scalar spherical harmonics of $\mathbb{S}^3$, and restrict the domain of the azimuthal numbers $m$ and $j$ such that the above condition is satisfied. This can be achieved, for example, by letting $\left|m\right|\le\ell$ to resemble the corresponding azimuthal number of the usual scalar spherical harmonics, but restricting $j$ to take integer values according to
\be
	j = -\left(\ell-\left|m\right|\right)\,,\,\, -\left(\ell-\left|m\right|\right) + 2\,,\,\, \dots\,,\,\, \left(\ell-\left|m\right|\right) - 2\,,\,\, \left(\ell-\left|m\right|\right) \,,
\ee
where we emphasize the step $2$ in the successively allowed values of $j$.

The eigenfunctions $\tilde{Y}_{\ell m j}$ can be seen to be orthogonal on $\mathbb{S}^3$ in these coordinates using properties of the Jacobi polynomials, while choosing the normalization constant to be
\be
	N_{\ell m j} = \frac{1}{\left(\left|j\right|\right)!}\sqrt{\left(2\ell+2\right)\frac{\left(\frac{\ell+\left|m\right|+\left|j\right|}{2}\right)!\left(\frac{\ell-\left|m\right|+\left|j\right|}{2}\right)!}{\left(\frac{\ell-\left|m\right|-\left|j\right|}{2}\right)!\left(\frac{\ell+\left|m\right|-\left|j\right|}{2}\right)!}} \,,
\ee
ensures the orthonormality identity
\be
	\oint_{\mathbb{S}^3} d\Omega_3\,\tilde{Y}_{\ell m j}^{\ast}\tilde{Y}_{\ell^{\prime}m^{\prime}j^{\prime}} = \delta_{\ell\ell^{\prime}}\delta_{mm^{\prime}}\delta_{jj^{\prime}} \,,
\ee
where asterisks indicate complex conjugation and the $3$-sphere integration measure in the direction cosine coordinates reads
\be
	\oint_{\mathbb{S}^3} d\Omega_3 = \int_{0}^{\frac{\pi}{2}}d\theta\int_{0}^{2\pi}d\phi\int_{0}^{2\pi}d\psi\,\sin\theta\cos\theta \,.
\ee

Two useful properties of these modified spherical harmonic functions are their transformation under complex conjugation, which simply reverses the sign of the azimuthal numbers,
\be
	\tilde{Y}_{\ell m j}^{\ast} = \tilde{Y}_{\ell,-m,-j} \,,
\ee
and under a parity transformation\footnote{In spherical coordinates, parity acts as $\left(\vartheta_1,\vartheta_2,\varphi\right)\rightarrow\left(\pi-\vartheta_1,\pi-\vartheta_2,\pi+\varphi\right)$ which is translated into directed cosine coordinates to $\left(\theta,\phi,\psi\right)\rightarrow\left(\theta,\pi+\phi,\pi+\psi\right)$.} $\left(\theta,\phi,\psi\right)\rightarrow\left(\theta,\pi+\phi,\pi+\psi\right)$, which only adds an overall phase,
\be
	\tilde{Y}_{\ell m j}\left(\theta,\pi+\phi,\pi+\psi\right) = \left(-1\right)^{m+j}\tilde{Y}_{\ell m j}\left(\theta,\psi,\phi\right) \,.
\ee

Let us count how many such basis states exist for a given value of the orbital number $\ell$. This is given by
\be
	\tilde{d}_{\ell} = \sum_{m=-\ell}^{\ell}\left(\sum_{j=-\left(\ell-\left|m\right|\right),2}^{\ell-\left|m\right|}\right) = \sum_{m=-\ell}^{\ell}\left(\ell-\left|m\right|+1\right) = \left(\ell+1\right)^2
\ee
and is exactly the same as the degeneracy of the scalar spherical harmonics\footnote{The scalar spherical harmonics of degree $\ell$ on $\mathbb{S}^{n}$ have a degeneracy,
\be
	d_{\ell}\left(n\right) = \frac{\left(2\ell+n-1\right)\left(\ell+n-2\right)!}{\ell!\left(n-1\right)!} \,.
\ee
For $n=3$, this gives $d_{\ell}\left(3\right)=\left(\ell+1\right)^2$.} on $\mathbb{S}^3$. Subsequently, the modified spherical harmonics $\tilde{Y}_{\ell m j}$ are equivalent to the usual scalar spherical harmonics on $\mathbb{S}^3$. In particular, the scalar spherical harmonics $Y_{\ell,\ell_2,\mu}$ on $\mathbb{S}^3$, with $\left|\mu\right|\le\ell_2\le\ell$ can always be written as a linear combination of the modified spherical harmonics,
\be
	Y_{\ell,\ell_2,\mu}\left(\vartheta_1,\vartheta_2,\varphi\right) = \sum_{m,j}c_{\ell,\ell_2,\mu;m,j}\tilde{Y}_{\ell m j}\left(\theta,\phi,\psi\right) \,,
\ee
and, thus, they form a complete set of $\tilde{d}_{\ell}$ linearly independent and orthonormal unit vectors. We remark here that the azimuthal numbers $\mu$ and $m$ are \textit{not} the same as they appear in $Y_{\ell,\ell_2,\mu}$ and $Y_{\ell m j}$; while $\mu$ and $m$ have the same range in both spherical harmonics bases, their multiplicities do not match. Nevertheless, in the above expansion it is straightforward to see that $c_{\ell,\ell_2,\mu;m,j}=c_{\ell,\ell_2,\mu;j}\delta_{m,\mu}$. In addition, the complex conjugacy relation $Y_{\ell,\ell_2,\mu}^{\ast} = \left(-1\right)^{\mu}Y_{\ell,\ell_2,-\mu}$ further implies
\be
	c_{\ell,\ell_2,\mu;j}^{\ast} = \left(-1\right)^{\mu} c_{\ell,\ell_2,-\mu;-j}\,.
\ee

One useful consequence of this completeness is that the two spherical harmonics bases obey the same addition theorem, expressed in terms of the Gegenabuer polynomials $C_{\ell}^{\left(1\right)}\left(x\right)$ as
\be
	\sum_{m,j}\tilde{Y}_{\ell m j}\left(\mathbf{\Omega}\right)\tilde{Y}_{\ell m j}^{\ast}\left(\mathbf{\Omega}^{\prime}\right) = \sum_{\ell_2,\mu}Y_{\ell,\ell_2,\mu}\left(\mathbf{\Omega}\right)Y_{\ell,\ell_2,\mu}^{\ast}\left(\mathbf{\Omega}^{\prime}\right) = \frac{\ell+1}{2\pi^2}C_{\ell}^{\left(1\right)}\left(\mathbf{\Omega}\cdot\mathbf{\Omega}^{\prime}\right) \,.
\ee

%One useful consequence of this completeness is that the two spherical harmonics bases obey the same addition theorem, that is,
%\be
%	\sum_{\ell_2,\mu}Y_{\ell,\ell_2,\mu}\left(\mathbf{\Omega}\right)Y_{\ell,\ell_2,\mu}^{\ast}\left(\mathbf{\Omega}^{\prime}\right) = \sum_{m,j}\tilde{Y}_{\ell m j}\left(\mathbf{\Omega}\right)\tilde{Y}_{\ell m j}^{\ast}\left(\mathbf{\Omega}^{\prime}\right) \,.
%\ee
%To see this more explicitly, we consider the quantity,
%\be
%	\tilde{Y}_{\ell}\left(\mathbf{\Omega},\mathbf{\Omega}^{\prime}\right) \equiv \sum_{m,j} \tilde{Y}_{\ell m j}\left(\mathbf{\Omega}\right)\tilde{Y}_{\ell m j}^{\ast}\left(\mathbf{\Omega}^{\prime}\right) \,.
%\ee
%Under a rotation $R\in SO\left(4\right)$, the transformed modified spherical harmonic can be written as a linear combination of the unrotated modified spherical harmonics by virtue of their completeness,
%\be
%	\tilde{Y}_{\ell m j}\left(R\mathbf{\Omega}\right) = \sum_{m^{\prime},j^{\prime}}C_{mj;m^{\prime}j^{\prime}}^{\left(\ell\right)} \tilde{Y}_{\ell m^{\prime}j^{\prime}}\left(\mathbf{\Omega}\right) \,.
%\ee
%From the orthonormality of $\tilde{Y}_{\ell m j}$ and the invariance of the integration measure $d\Omega_3$ under such rotations, it is immediately seen that the coefficients $C_{mj;m^{\prime}j^{\prime}}^{\left(\ell\right)}$ satisfy the unitarity relation
%\be
%	\sum_{m^{\prime\prime}j^{\prime\prime}} C_{mj;m^{\prime\prime}j^{\prime\prime}}^{\left(\ell\right)}C_{m^{\prime}j^{\prime};m^{\prime\prime}j^{\prime\prime}}^{\left(\ell\right)\ast} = \delta_{mm^{\prime}}\delta_{jj^{\prime}} \,.
%\ee
%As a result, $\tilde{Y}_{\ell}\left(\mathbf{\Omega},\mathbf{\Omega}^{\prime}\right)$ is invariant under rotations,
%\be\ba
%	\tilde{Y}_{\ell}\left(R\mathbf{\Omega},R\mathbf{\Omega}^{\prime}\right) &= \sum_{m,j} \tilde{Y}_{\ell m j}\left(R\mathbf{\Omega}\right)\tilde{Y}_{\ell m j}^{\ast}\left(R\mathbf{\Omega}^{\prime}\right) \\
%	&= \sum_{m^{\prime},j^{\prime}}\sum_{m^{\prime\prime},j^{\prime\prime}} \left(\sum_{m,j} C_{mj;m^{\prime},j^{\prime}}^{\left(\ell\right)}C_{mj;m^{\prime\prime},j^{\prime\prime}}^{\left(\ell\right)\ast}\right) \tilde{Y}_{\ell m^{\prime}j^{\prime}}\left(\mathbf{\Omega}\right) \tilde{Y}_{\ell m^{\prime\prime}j^{\prime\prime}}\left(\mathbf{\Omega}^{\prime}\right) \\
%	&= \sum_{m^{\prime},j^{\prime}} \tilde{Y}_{\ell m^{\prime} j^{\prime}}\left(\mathbf{\Omega}\right)\tilde{Y}_{\ell m^{\prime} j^{\prime}}^{\ast}\left(\mathbf{\Omega}^{\prime}\right) \\
%	&= \tilde{Y}_{\ell}\left(\mathbf{\Omega},\mathbf{\Omega}^{\prime}\right) \,,
%\ea\ee
%where we also used the fact that
%\be
%	C_{mj;m^{\prime}j^{\prime}}^{\left(\ell\right)} = \left[C_{m^{\prime}j^{\prime};mj}^{\left(\ell\right)\ast}\right]^{-1} \,,
%\ee
%with the inverse operation above referring to the inverse rotation $R^{-1}$. Consequently, $\tilde{Y}_{\ell}\left(\mathbf{\Omega},\mathbf{\Omega}^{\prime}\right)$ is a zone spherical harmonic on $\mathbb{S}^3$ of order $\ell$ when regarded as a function of $\mathbf{\Omega}$ or $\mathbf{\Omega}^{\prime}$ only. But this is precisely what the Gegenbauer polynomial $C_{\ell}^{\left(1\right)}\left(\mathbf{\Omega}\cdot\mathbf{\Omega}^{\prime}\right)$ is, therefore,
%\be
%	C_{\ell}^{\left(1\right)}\left(\mathbf{\Omega}\cdot\mathbf{\Omega}^{\prime}\right) = B_{\ell} \tilde{Y}_{\ell}\left(\mathbf{\Omega},\mathbf{\Omega}^{\prime}\right) \,,
%\ee
%with the proportionality constant fixed by computing the $\mathbf{\Omega}^{\prime}=\mathbf{\Omega}$ case,
%\be
%	B_{\ell} = \frac{C_{\ell}^{\left(1\right)}\left(1\right)}{\tilde{Y}_{\ell}\left(\mathbf{\Omega},\mathbf{\Omega}\right)} \,.
%\ee
%The constant $\tilde{Y}_{\ell}\left(\mathbf{\Omega},\mathbf{\Omega}\right)$ can be computed by integrating its expansion into products of modified spherical harmonics over the sphere,
%\be
%	\tilde{Y}_{\ell}\left(\mathbf{\Omega},\mathbf{\Omega}\right) = \frac{1}{\Omega_3}\sum_{m,j}\oint_{\mathbb{S}^3} d\Omega_3 \left|\tilde{Y}_{\ell m j}\left(\mathbf{\Omega}\right)\right|^2 = \frac{\tilde{d}_{\ell}}{\Omega_3} = \frac{\left(\ell+1\right)^2}{2\pi^2} \,,
%\ee
%while, using the known result $C_{\ell}^{\left(1\right)}\left(1\right)=\ell+1$, we conclude that
%\be
%	B_{\ell} = \frac{2\pi^2}{\ell+1} \,.
%\ee

\subsection{Correspondence with STF tensors}
We will now present the $1$-to-$1$ correspondence between $4$-dimensional spatial STF tensors of rank-$\ell$ and the modified spherical harmonics $\tilde{Y}_{\ell m j}$ with the same orbital number $\ell$. There are two key observations that ensure this correspondence. First, the basic STF tensor of rank-$\ell$ $\Omega^{\left\langle L\right\rangle}$, where $\Omega^{i}=\frac{x^{i}}{r}$ are the projectors along the $i$-th spatial direction $x^{i}$, has eigenvalue $-\ell\left(\ell+2\right)$ under the action of the $4$-dimensional flat space Laplace operator, i.e. the same eigenvalue as all $\tilde{Y}_{\ell m j}$ with the same $\ell$.

Second, the number of independent components of a rank-$\ell$ STF tensor in  $n+1$ spatial dimensions is the number of degrees of freedom of a rank-$\ell$ symmetric tensor minus the number of traces that need to be removed,
\be
	d_{\ell}^{\text{STF}}\left(n\right) =
	\begin{pmatrix}
		n+\ell \\
		\ell
	\end{pmatrix} -
	\begin{pmatrix}
		n+\ell-2 \\
		\ell-2
	\end{pmatrix}
	= \frac{\left(2\ell+n-1\right)\left(\ell+n-2\right)!}{\ell!\left(n-1\right)!} \,.
\ee
For $n=3$, this gives that the number of independent components of a $4$-dimensional spatial STF tensor of rank-$\ell$ is equal to $\left(\ell+1\right)^2$, i.e. the same as the number $\tilde{d}_{\ell}$ of basis function $\tilde{Y}_{\ell m j}$ with orbital number $\ell$. Consequently, any $4$-dimensional spatial STF tensor of rank-$\ell$ can be written as a linear combination of $\tilde{Y}_{\ell m j}$ for all the possible values of the azimuthal numbers $m$ and $j$. In particular, for the basic STF tensor $\Omega^{\left\langle L \right\rangle}$,
\be
	\Omega^{\left\langle L \right\rangle} = A_{\ell}\sum_{m,j}\mathcal{Y}^{L}_{\ell m j}\tilde{Y}_{\ell m j}\left(\mathbf{\Omega}\right) \,,
\ee
where $\mathbf{\Omega}$ is a shorthand for the direction cosine coordinates $\left(\theta,\phi,\psi\right)$ associated with the position vector $x^{i}$ from which $\Omega^{i}$ is constructed, the constant STF tensors $\mathcal{Y}^{L}_{\ell m j}$ are given by
\be
	\mathcal{Y}^{L}_{\ell m j} = \frac{1}{A_{\ell}}\oint_{\mathbb{S}^3} d\Omega_{3}\,\Omega^{\left\langle L \right\rangle}\tilde{Y}_{\ell m j}^{\ast}\left(\mathbf{\Omega}\right) \,,
\ee
and $A_{\ell}$ is a real normalization constant chosen such that\footnote{Using the addition theorem of scalar spherical harmonics, which is also an addition theorem for the modified spherical harmonics, this normalization constant can be found to be,
\be
	A_{\ell} = \frac{4\pi^2\ell!}{\left(2\ell+2\right)!!}
\ee}
\be
	\tilde{Y}_{\ell m j}\left(\mathbf{\Omega}\right) = \mathcal{Y}^{L\ast}_{\ell m j}\Omega_{\left\langle L \right\rangle} \,.
\ee

A general $4$-dimensional spatial STF tensor $\mathcal{E}^{L}=\mathcal{E}^{\left\langle L \right\rangle}$ can then be expanded into its modified multipole moments $\mathcal{E}_{\ell m j}$ according to
\be
	\mathcal{E}^{L} = \sum_{m,j}\mathcal{E}_{\ell m j}\mathcal{Y}^{L\ast}_{\ell m j} \,,
\ee
with
\be
	\mathcal{E}_{\ell m j} = A_{\ell} \mathcal{E}_{L}\mathcal{Y}^{L}_{\ell m j} = \mathcal{E}_{L}\oint_{\mathbb{S}^3} d\Omega_{3}\,\Omega^{\left\langle L \right\rangle}\tilde{Y}_{\ell m j}^{\ast}\left(\mathbf{\Omega}\right) \,,
\ee
such that
\be
	\mathcal{E}_{L}\Omega^{L} = \sum_{m,j}\mathcal{E}_{\ell m j}\tilde{Y}_{\ell m j} \,.
\ee

Last, from the complex conjugacy relation of the modified spherical harmonics, we can see that
\be
	\mathcal{Y}^{L\ast}_{\ell m j} = \mathcal{Y}^{L}_{\ell, -m, -j} \,,\quad \mathcal{E}_{\ell m j}^{\ast} = \mathcal{E}_{\ell, -m, -j} \,.
\ee