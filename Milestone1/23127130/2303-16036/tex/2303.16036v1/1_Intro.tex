\section{Introduction}
\label{sec:Intro}

The engulfment of gravitational wave astronomy into the area of precision astronomy~\cite{LIGOScientific:2016aoc} has attracted growing attention towards the tidal response problem of a compact body. During the early stages of the inspiraling phase of a binary system, the tidal perturbations of the bodies are parameterized by the static Love numbers~\cite{Love1912,PoissonWill2014}. More importantly, Love numbers are able to probe the equation of state of the involved relativistic configurations~\cite{Flanagan:2007ix,Yagi:2013bca,LIGOScientific:2017vwq,Raithel:2018ncd,Chatziioannou:2020pqz}, while their measurement has been proposed as a testing arena for strong-field gravity~\cite{Cardoso:2017cfl}.

Love numbers' imprint on gravitational wave signals can be calculated by means of the worldline Effective Field Theory (EFT), 
a toolkit for the construction of gravitational waveform templates
during the inspiral phase~\cite{Goldberger:2004jt,Porto:2005ac,Porto:2016pyg,Levi:2015msa,Levi:2018nxp,Goldberger:2022ebt}. Within the worldline EFT, a compact body is approximated by its large-distance universal appearance as a point-particle evolving along a worldline. Finite size effects are systematically captured by non-minimal couplings of the worldline with various curvature operators. The Love numbers, in particular, appear as Wilson coefficients for operators quadratic in the curvature and their computation reduces to a matching condition. %\cite{Kol:2011vg,Charalambous:2021mea,Ivanov:2022hlo,Ivanov:2022qqt}.

The static Love numbers for general-relativistic black holes have recently been gaining a growing theoretical interest. In four spacetime dimensions, both static black holes~\cite{Fang:2005qq,Damour:2009vw,Binnington:2009bb,Gurlebeck:2015xpa}, as well as spinning black holes~\cite{Poisson:2014gka,Landry:2015zfa,Pani:2015hfa,LeTiec:2020spy,LeTiec:2020bos,Chia:2020yla,Charalambous:2021mea,Ivanov:2022hlo,Ivanov:2022qqt} have been shown to posses vanishing static Love numbers. This fact raises naturalness concerns from the EFT point of view, calling upon the existence of an enhanced symmetry structure~\cite{Porto:2016zng,tHooft:1979rat}.

Black holes have been dubbed ``the hydrogen atom of the 21st century''~\cite{tHooft:2016sdu,EHT2019}. This statement has been given some rigor for asymptotically flat black holes in General Relativity~\cite{Bertini:2011ga,Kim:2012mh,Charalambous:2021kcz,Charalambous:2022rre}. It is ought to persisting hidden conformal structures of asymptotically flat black holes, with the classic paradigms being the extremal~\cite{Bardeen:1999px,Guica:2008mu,Lu:2008jk} and the non-extremal~\cite{Castro:2010fd,Krishnan:2010pv} Kerr/CFT conjectures, while more efforts have recently been put forward to constructing holographic-like dictionaries between black hole geometries and conformal field theories~\cite{Bonelli:2021uvf,Consoli:2022eey}. It has also been suggested that conformal structures associated with black holes can leave distinct signatures on polarimteric observations, see e.g.~\cite{Johnson:2019ljv,Himwich:2020msm}.

We have recently proposed that one new form of such conformal structures of black holes can be used to explain the vanishing of static Love numbers for general-relativistic black holes in four spacetime dimensions~\cite{Charalambous:2021kcz,Charalambous:2022rre}, see also~\cite{Cvetic:2021vxa} of its presence in SUGRA black holes. This ``Love'' symmetry is an $\SL$ symmetry manifesting itself in the near-zone region, where perturbations have small frequencies compared to the inverse distance from the black hole. The Love symmetry outputs the vanishing of the static Love numbers as a selection rule following from the fact that the relevant perturbation solution belongs to a highest-weight representation. Geometrically, it can be realized as an approximate isometry of the black hole geometry, in the sense that it is an exact isometry of an effective black hole geometry which preserves the thermodynamic properties of the black hole but approximates (''subtracts'') its environment. Such geometries have been introduced in~\cite{Cvetic:2011hp,Cvetic:2011dn} and go by the name ``subtracted geometries''. Interestingly, the Love symmetry appears to be a cousin of another well-known $\SL$ symmetry associated with degenerate black holes: the enhanced isometry of the near-horizon geometry for extremal black holes~\cite{Bardeen:1999px,Kunduri:2007vf}. This hints at the interpretation of the Love symmetry as a remnant of this enhanced isometry for extremal black holes. It should be remarked here that there have been other attempts of explaining the vanishing of black holes Love numbers via symmetry arguments, notably, the ``ladder symmetries'' proposal~\cite{Hui:2021vcv,Hui:2022vbh} (see also~\cite{BenAchour:2022uqo,Katagiri:2022vyz}), which bares resemblance to the earlier notion of ``mass ladder operators'' for spacetimes admitting closed conformal Killing vectors~\cite{Cardoso:2017qmj}.

The static Love numbers have also been studied for higher-dimensional general-relativistic black holes in~\cite{Kol:2011vg,Hui:2020xxx,Pereniguez:2021xcj}, all of which have been spherically symmetric. In these examples, the static Love numbers are in general non-zero and exhibit no running. However, there exist some resonant conditions between the multipolar order $\ell$ and the spacetime dimensionality $d$, for which the static Love numbers vanish again, which hints again on a symmetry explanation.
Specifically, for the case of massless scalar perturbations,
this happens when
the generalized 
angular momentum 
$\hat{\ell}=\ell/\left(d-3\right)$ is an integer. 
Despite this more intricate structure of the black hole static Love numbers, the Love symmetry exists for any multipolar order and in any spacetime dimension, and explains these puzzling results~\cite{Charalambous:2021kcz,Charalambous:2022rre}. Here, we extend this analysis to higher-dimensional asymptotically flat, axisymmetric spinning (Myers-Perry) black holes~\cite{Myers:1986un}. We will devote the present work to setting up the arena for this investigation by focusing to the $d=5$ scalar Love numbers.

The structure of this paper is as follows. We begin by reviewing the definition of the response coefficients in Newtonian gravity~\cite{PoissonWill2014} and the subsequent generalization to relativistic configurations within the worldline EFT formalism~\cite{Goldberger:2004jt,Porto:2005ac} in Section~\ref{sec:TLNsDefinition}. We also describe how it is natural to distinguish between the conservative and dissipative parts of the response and we identify the former as the ``Love'' part of the response~\cite{Chia:2020yla,Charalambous:2021mea,Ivanov:2022hlo}. This prescription is then applied to axisymmetric spinning bodies in $5$ spacetime dimensions whose background symmetries do not allow the mixing between different multipolar orders, a case relevant for the Myers-Perry black hole, to extract a simple matching formula for the extraction of the Love numbers from the microscopic computation.

In Section \ref{sec:TLNsComputation}, we treat massless scalar perturbations of the $5$-d Myers-Perry black hole with the scope of extracting the static scalar Love numbers. We introduce a near-zone approximation that allows us to analytically solve the equations of motion in terms of hypergeometric functions and extract the scalar response coefficients that are exact in the static limit. We find some qualitatively new features compared to the case of higher-dimensional non-spinning black holes, namely, that the static scalar Love numbers are always non-zero and exhibit running for generic black hole spin and angular modes of perturbations. We also explore the resonant conditions for which the Love numbers extracted from the near-zone approximation acquire some seemingly fine-tuned properties in the form of ``magic zeroes.''

These resonant conditions are addressed in Section~\ref{sec:SL2R} by introducing two Love symmetries for the $5$-d Myers-Perry black hole, that is, by two enhanced $\SL$ symmetries associated with the two near-zone truncations of the equations of motion that had been employed. Similar to the Love symmetry argument for $4$-dimensional black holes~\cite{Charalambous:2021kcz,Charalambous:2022rre}, these resonant conditions are in one-to-one correspondence with the states spanning the highest-weight representations of the Love symmetries. The highest-weight property then outputs the vanishing of Love numbers, while the absence of running is also realized from algebraic local criteria.

In Section~\ref{sec:Properties}, we begin investigating the properties of the Love symmetries. We start by their geometric interpretation as exact isometries of effective black hole geometries which are identified as relatives to subtracted geometries~\cite{Cvetic:2011dn,Cvetic:2011hp}. We then reveal an infinite-dimensional extension of the symmetry structure into \mbox{$\SL\ltimes\hat{U}\left(1\right)_{\mathcal{V}}^2$} that contains both Love symmetries as particular subalgebras, similar to the infinite extension for $4$-dimensional spinning black holes~\cite{Charalambous:2021kcz,Charalambous:2022rre}.

This larger symmetry also contains a family of $\SL$ subalgebras which are closely related to the enhanced isometries of the near-horizon geometries in the extremal limit~\cite{Bardeen:1999px,Kunduri:2007vf}. This is explicitly studied in Section~\ref{sec:NHE} where we show that appropriately taking the extremal limit of these $\SL$ subalgebras of the infinite extension precisely recovers the Killing vectors of the near-horizon $\text{AdS}_2$ throat. This further hints towards the interpretation of the Love symmetries as remnants of this enhanced isometry associated with the extremal black holes.

We close with a summary and discussion in Section \ref{sec:Discussion}. We also supplement with a number of appendices. In Appendix \ref{sec:Ap5dMPGeometry}, we review the geometry of the $5$-d Myers-Perry black hole. In Appendix \ref{sec:ApSphericalHarmonics}, we introduce a modified spherical harmonics basis which proves more natural to employ for the study of axisymmetric bodies in $5$ spacetime dimensions. In Appendix \ref{sec:ApSchwarzschildLimit}, we collect the formulas for the source/response split of the black hole perturbations as dictated by the matching with the worldline EFT. We also compare our
findings with known results for Schwarzschild black holes~\cite{Kol:2011vg,Hui:2020xxx} by taking the spinless limit of the relevant expressions extracted in the current work. Last, in Appendix \ref{sec:ApSL2RGenerators}, we sketch the derivation of the $\SL$ symmetries associated with truncations of the equations of motion that preserve the characteristic exponents of the black hole near the event horizon, a subset of which contains near-zone truncations and the associated Love symmetries.

\textit{Notation and conventions}: We will employ geometrized units with $c=1$ and work with the mostly-positive metric Lorentzian signature, $\left(\eta_{\mu\nu}\right) = \text{diag}\left(-1,+1,+1,\dots\right)$. Small Latin letters from the beginning of the alphabet will denote spatial indices running from $1$ to $d-1$ for a $d$-dimensional spacetime, while small Greek letters will denote spacetime indices running from $0$ to $d-1$, with $x^0$ the temporal coordinate, and repeated indices will be summed over. We will also be employing the multi-index notation $a_1\dots a_{\ell}\equiv L$, within which $x^{a_1}\dots x^{a_{\ell}}\equiv x^{L}$ and $\partial_{a_1}\dots\partial_{a_{\ell}}\equiv\partial_{L}$. Last, we will denote the symmetric trace-free (STF) part of a tensor with respect to a set of indices $\left\{a_1,a_2,\dots\right\}$ by enclosing the indices within angular brackets (``$\left\langle a_1a_2 \dots \right\rangle$'').