\section{Source/response split and Schwarzschild limit}
\label{sec:ApSchwarzschildLimit}

In this appendix we demonstrate explicitly the source/response split of the scalar field \eqref{eq:NZRadialSolution} and how the spinless limit recovers the corresponding solution in the $5$-d Schwarzschild black hole background~\cite{Kol:2011vg}. To extract the source and response parts of the solution, we need to identify those components that solve the linearized Klein-Gordon equation and asymptotically behave as
\be
	\begin{gathered}
		R_{\omega\ell m j}\left(\rho\right) = \bar{\mathcal{E}}_{\ell m j}\left(\omega\right)\,\rho^{\hat{\ell}}\left[Z_{\omega\ell m j}^{\text{source}}\left(\rho\right) + k_{\ell m j}\left(\omega\right)\,\left(\frac{\rho_{s}}{\rho}\right)^{2\hat{\ell}+1} Z_{\omega\ell m j}^{\text{response}}\left(\rho\right)\right] \,, \\
		Z_{\omega\ell m j}^{\text{source/response}}\left(\rho\right) \xrightarrow{\rho\rightarrow\infty} 1 \,,
	\end{gathered}
\ee
where $k_{\ell m j}$ are the scalar response coefficients. The source/response split ambiguity that is encountered for the physical values $\ell\in\mathbb{N}$ is bypassed by treating the orbital number $\ell$ as a generic real-valued number $\ell\in\mathbb{R}$. The above asymptotic behaviors then allow to uniquely distinguish between the source and response components of the solution without worrying about ambiguities arising from overlapping series.

The hypergeometric function involved in the near-zone scalar field solution that is regular at the future event horizon \eqref{eq:NZRadialSolution} is expressed in terms of the variable $x=\frac{\rho-\rho_{+}}{\rho_{+}-\rho_{-}}$. Due to a branch cut at $\left|x\right|=1$, this series needs to be analytically continued at large distances according to
\be\ba
	{}_2F_1\left(a,b;c;z\right) &= \frac{\Gamma\left(c\right)\Gamma\left(b-a\right)}{\Gamma\left(b\right)\Gamma\left(c-a\right)}\left(-z\right)^{-a}{}_2F_1\left(a,a-c+1;a-b+1;\frac{1}{z}\right) \\
	&+ \frac{\Gamma\left(c\right)\Gamma\left(a-b\right)}{\Gamma\left(a\right)\Gamma\left(c-b\right)}\left(-z\right)^{-b}{}_2F_1\left(b,b-c+1;b-a+1;\frac{1}{z}\right) \,,
\ea\ee
which is valid for $\left|\text{Arg}\left(-z\right)\right|<\pi$. Applying this prescription for \eqref{eq:NZRadialSolution}, we match the normalization constants $\bar{R}_{\ell m j}\left(\omega\right)$ to
\be
	\bar{R}_{\ell m j}\left(\omega\right) = \bar{\mathcal{E}}_{\ell m j}\left(\omega\right)\left(\rho_{+}-\rho_{-}\right)^{\hat{\ell}}\frac{\Gamma\left(\hat{\ell}+1+i\Gamma^{\left(\sigma\right)}_{+\sigma}\right)\Gamma\left(\hat{\ell}+1+i\Gamma^{\left(\sigma\right)}_{-\sigma}\right)}{\Gamma\left(2\hat{\ell}+1\right)\Gamma\left(1+2iZ_{+}\left(\omega\right)\right)} \,,
\ee
and, then, identify the full source and response components to be\footnote{We remark here the analytic properties of the hypergeometric function ${}_2F_1\left(a,b;c;z\right)$ with respect to its arguments and how this is related to the near-zone scalar field solution \eqref{eq:NZRadialSolution}. In general, ${}_2F_1\left(a,b;c;z\right)$ does not exist when $c$ is a non-positive integer and the analytic continuation formula for large $z$ becomes problematic. Nevertheless, the \textit{regularized} hypergeometric function,
\be
	\mathbf{F}\left(a,b;c,z\right) \equiv \frac{{}_2F_1\left(a,b;c;z\right)}{\Gamma\left(c\right)} \,,
\ee
is well-defined for all values of $c\in\mathbb{C}$ and can be analytically continued at $\left|z\right|>1$ without any issues. This is also the situation with our near-zone scalar field solution \eqref{eq:NZRadialSolution} as the normalization constants $\bar{R}_{\ell m j}\left(\omega\right)$ themselves contain precisely an inverse $\Gamma\left(c\right)$ factor when matched to the field strengths $\bar{\mathcal{E}}_{\ell m j}\left(\omega\right)$. In other words, the solution \eqref{eq:NZRadialSolution} has all the nice analytic properties with respect to all the arguments of the hypergeometric function.}
\be
	\begin{gathered}
		\ba
			Z_{\omega\ell m j}^{\text{source}}\left(\rho\right) &= \left(1-\frac{\rho_{+}}{\rho}\right)^{\hat{\ell}} \left(\frac{\rho-\rho_{-}}{\rho-\rho_{+}}\right)^{\mp i\sigma\tilde{Z}_{-}^{\left(\sigma\right)}\left(\omega\right)} \\
			&\times\,{}_2F_1\left(-\hat{\ell}+i\Gamma^{\left(\sigma\right)}_{\mp\sigma}\left(\omega\right),-\hat{\ell}-i\Gamma^{\left(\sigma\right)}_{\pm\sigma}\left(\omega\right);-2\hat{\ell};\frac{\rho_{+}-\rho_{-}}{\rho_{+}-\rho}\right) \,,
		\ea \\
		\ba
			Z_{\omega\ell m j}^{\text{response}}\left(\rho\right) &= \left(1-\frac{\rho_{+}}{\rho}\right)^{-\hat{\ell}-1} \left(\frac{\rho-\rho_{-}}{\rho-\rho_{+}}\right)^{\mp i\sigma\tilde{Z}_{-}^{\left(\sigma\right)}\left(\omega\right)} \\
			&\times\,{}_2F_1\left(\hat{\ell}+1+i\Gamma^{\left(\sigma\right)}_{\mp\sigma}\left(\omega\right),\hat{\ell}+1-i\Gamma^{\left(\sigma\right)}_{\pm\sigma}\left(\omega\right);2\hat{\ell}+2;\frac{\rho_{+}-\rho_{-}}{\rho_{+}-\rho}\right) \,,
		\ea
	\end{gathered}
\ee
and the scalar response coefficients $k_{\ell m j}\left(\omega\right)$ are given explicitly in \eqref{eq:NZSLNs}.

The two radial functions $\rho^{\hat{\ell}}Z_{\omega\ell m j}^{\text{source}}$ and $\rho^{-\hat{\ell}-1}Z_{\omega\ell m j}^{\text{response}}$ are linearly independent solutions of the near-zone radial Klein-Gordon equation for generic orbital number $\ell$ and have the nice property that they transform into each other under the discrete symmetry transformation $\hat{\ell} \leftrightarrow -\hat{\ell}-1$ of the near-zone Klein-Gordon equation,
\be
	\rho^{\hat{\ell}}Z_{\omega\ell m j}^{\text{source}} \xleftrightarrow{\hat{\ell} \leftrightarrow -\hat{\ell}-1} \rho^{-\hat{\ell}-1}Z_{\omega\ell m j}^{\text{response}} \,.
\ee
%The linear independence can be directly seen from the Wronskian,
%\be
%	\mathscr{W}\left\{\rho^{\hat{\ell}}Z_{\ell m j}^{\text{source}},\rho^{-\hat{\ell}-1}Z_{\ell m j}^{\text{response}},\rho\right\} = -\frac{2\hat{\ell}+1}{\left(\rho-\rho_{+}\right)\left(\rho-\rho_{-}\right)}\,e^{2\pi\Gamma_{-}}
%\ee
%which never vanishes for generic $\hat{\ell}$. However, when $2\hat{\ell}$ is an integer, the above two solutions become degenerate and need to be treated more carefully. This is closely related to the source/response split ambiguity issue that gets lifted when we analytically continue the orbital number to range over the field of real numbers.
%\todo[inline]{Used equation 15.10.7 in NIST}
%
%We point out here that, instead of finding the solution regular at the black hole future event horizon and analytically continuing to large distances, we could instead use a generic linear combination of the source and response solutions above, with arbitrary response coefficients $k_{\ell m j}\left(\omega\right)$, which are eventually fixed by imposing the ingoing boundary condition at the event horizon.

We now consider the spinless limit of our solution and compare it with the already known results obtained in~\cite{Kol:2011vg} in the static limit. To do this, we need to extract the corresponding behavior of the parameters $Z_{+}\left(\omega\right)$ and $\tilde{Z}^{\left(\sigma\right)}_{-}\left(\omega\right)$, which are given in explicitly in \eqref{eq:NHbc} and \eqref{eq:NZZetaMinus}. First of all, as the spin parameters $a$ and $b$ are sent to zero, the inner and outer horizons have the asymptotic behaviors
\be
	\rho_{+} \xrightarrow{a,b\rightarrow0} \rho_{s}\left[1 + \mathcal{O}\left(a^2,b^2\right)\right] \,\,\,,\,\,\,\rho_{-} \xrightarrow{a,b\rightarrow0} \frac{a^2b^2}{\rho_{s}}\left[1+\mathcal{O}\left(a^2,b^2\right)\right] \,.
\ee
As such,
\be
	Z_{+}\left(\omega\right) \xrightarrow{a,b\rightarrow0} \frac{\omega r_{s}}{2} + \mathcal{O}\left(a,b\right) \,,\quad \tilde{Z}^{\left(\sigma\right)}_{-}\left(\omega\right) \xrightarrow{a,b\rightarrow0} \sigma\frac{\omega r_{s}}{2} + \mathcal{O}\left(a,b\right) \,.
\ee
We see, thus, that both $Z_{+}\left(\omega\right)$ and $\tilde{Z}^{\left(\sigma\right)}_{-}\left(\omega\right)$ become independent of the azimuthal numbers $m$ and $j$ in the spinless limit, as is appropriate for the spherically symmetric $5$-d Schwarzschild black hole background geometry. Furthermore, $\Gamma^{\left(\sigma\right)}_{\pm\sigma}\left(\omega\right)=\omega r_{s} \left(1\mp1\right)/2 + \mathcal{O}\left(a,b\right)$ and the near-zone solution in the spinless limit becomes
\be\ba
	R_{\omega\ell m j} &\xrightarrow{a,b\rightarrow0} \bar{\mathcal{E}}_{\ell m j}\left(\omega\right)\,\rho_{s}^{\hat{\ell}}\,\frac{\Gamma\left(\hat{\ell}+1\right)\Gamma\left(\hat{\ell}+1+i\omega r_{s}\right)}{\Gamma\left(2\hat{\ell}+1\right)\Gamma\left(1+i\omega r_{s}\right)} \\
	&\times \left(1-\frac{\rho_{s}}{\rho}\right)^{i\omega r_{s}/2}{}_2F_1\left(\hat{\ell}+1,-\hat{\ell};1+2i\omega r_{s};1-\frac{\rho}{\rho_{s}}\right) \,.
\ea\ee
Using the Pfaff transformation,
\be
	{}_2F_1\left(a,b;c;z\right) = \left(1-z\right)^{-b}{}_2F_1\left(c-a,b;c;\frac{z}{z-1}\right) \,,
\ee
which is valid for $\left|\text{Arg}\left(1-z\right)\right|<\pi$, this can be rewritten as
\be\ba
	R_{\omega\ell m j} &\xrightarrow{a,b\rightarrow0} \bar{\mathcal{E}}_{\ell m j}\left(\omega\right)\,\rho_{s}^{\hat{\ell}}\,\frac{\Gamma\left(\hat{\ell}+1\right)\Gamma\left(\hat{\ell}+1+i\omega r_{s}\right)}{\Gamma\left(2\hat{\ell}+1\right)\Gamma\left(1+i\omega r_{s}\right)} \\
	&\times \left(1-\frac{\rho_{s}}{\rho}\right)^{i\omega r_{s}/2}\left(\frac{\rho_{s}}{\rho}\right)^{-\hat{\ell}}{}_2F_1\left(-\hat{\ell},-\hat{\ell}+2i\omega r_{s};1+2i\omega r_{s};1-\frac{\rho_{s}}{\rho}\right) \,.
\ea\ee
In the $\omega\rightarrow0$ limit, this is in complete agreement, up to an overall conventional constant factor, with the corresponding result in equation (4.6) of~\cite{Kol:2011vg}. The scalar response coefficients \eqref{eq:NZSLNs} also have a smooth spinless limit that can be easily shown to agree with equation (4.7) of~\cite{Kol:2011vg} when $\omega=0$ after employing the Legendre duplication formula for the Gamma function,
\be
	\Gamma\left(z\right)\Gamma\left(z+\frac{1}{2}\right) = 2^{1-2z}\sqrt{\pi}\,\Gamma\left(2z\right) \,.
\ee