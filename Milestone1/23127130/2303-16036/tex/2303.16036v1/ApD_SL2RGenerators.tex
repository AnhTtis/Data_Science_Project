\section{Derivation of $\SL$ generators}
\label{sec:ApSL2RGenerators}

In Section \ref{sec:SL2R} we presented the existence of near-zone truncations of the massless Klein-Gordon equation in the background of the $5$-d Myers-Perry black hole equipped with an $\SL$ symmetry structure. In this appendix, we will sketch the derivation of the vector fields generating these symmetries. We will do this by starting with a generic ansatz for the form of the $\SL$ generators and require that the associated Casimir operators yield operators that preserve the characteristic exponents of the full equations of motion in the vicinity of the black hole event horizon. We will end up with an infinite number of $\SL$ algebras, most of which are not consistent near-zone truncations and also not globally defined.

The upshot of using this approach is that finding the most general truncation that preserves the near-horizon characteristic exponent also ensures that we will find all the possible near-zone truncations admitting $\SL$ symmetries as a subset. As we will see, there will be two towers of near-zone $\SL$ symmetries controlled by an arbitrary parameter which spontaneously breaks the $\SL$ symmetry down to $U\left(1\right)$. The only possible globally defined near-zone $\SL$ symmetries will then correspond to setting this symmetry breaking parameter to zero and will precisely correspond to the Love symmetries presented in Section \ref{sec:SL2R}. We will also investigate the situations where the $\SL$ symmetry of the truncations preserving the near-horizon characteristic exponents can be enhanced to full $2$-d conformal structure $\SL\times\SL$.

\subsection{Truncated radial operators preserving the near-horizon characteristic exponents}
The full massless Klein-Gordon operator in the background of the $5$-d Myers-Perry black hole has been introduced in Section \ref{sec:TLNsComputation} and the corresponding radial and angular operators are given in \eqref{eq:FullEOM}. We wish to explore the possibility of truncating the radial operator such that we preserve the characteristic exponents as we approach the event horizon at $\rho=\rho_{+}$. The most general such truncation has the form
\be
	\mathbb{O}_{\text{trunc}} =\partial_{\rho}\,\Delta\,\partial_{\rho} - \frac{\left(\rho_{+}-\rho_{-}\right)^2}{4\Delta}\beta^2\left(\partial_{t}+\Omega_{+}\,\partial_{+}+\Omega_{-}\,\partial_{-}\right)^2 + \delta\gamma^{\mu\nu}\partial_{\mu}\partial_{\nu} + \delta\gamma^{\mu}\,\partial_{\mu} \,,
\ee
where $\delta\gamma^{\mu\nu}$ and $\delta\gamma^{\mu}$ are terms that are subleading in the vicinity of the event horizon, $\Delta=\left(\rho-\rho_{+}\right)\left(\rho-\rho_{-}\right)$ is the discriminant function for the $5$-d Myers-Perry black hole and $\beta=\rho_{s}r_{+}/\left(\rho_{+}-\rho_{-}\right)$ is its inverse Hawking temperature. We note here that we are working in sum/difference azimuthal angles, $\psi_{\pm}=\psi\pm\phi$, with $\Omega_{\pm}=\Omega_{\psi}\pm\Omega_{\phi}$ the angular velocities along these two directions, and we are using the notation $\partial_{\pm}\equiv\partial_{\psi_{\pm}} = \left(\partial_{\psi}\pm\partial_{\phi}\right)/2$. The subleading terms $\delta\gamma^{\mu\nu}$ and $\delta\gamma^{\mu}$ that preserve the background $\mathbb{R}_{t}\times U\left(1\right)_{\phi}\times U\left(1\right)_{\psi}$ symmetries will then have the generic form
\be
	\begin{gathered}
		\delta\gamma^{tt} = f_{tt}\left(\rho\right) \,,\quad \delta\gamma^{t\psi_{\pm}} = \Omega_{\pm}f_{t\psi_{\pm}}\left(\rho\right) \,, \\
		\delta\gamma^{\psi_{\pm}\psi_{\pm}} = \Omega_{\pm}^2f_{\psi_{\pm}\psi_{\pm}}\left(\rho\right) \,,\quad \delta\gamma^{\psi_{+}\psi_{-}} = \Omega_{+}\Omega_{-}f_{\psi_{+}\psi_{-}}\left(\rho\right) \,, \\
		\delta\gamma^{t\rho} = \Delta f_{t\rho}\left(\rho\right) \,,\quad \delta\gamma^{\psi_{\pm}\rho} = \Omega_{\pm}\Delta f_{\psi_{\pm}\rho	}\left(\rho\right) \,,\quad \delta\gamma^{\rho\rho} = \Delta^2 f_{\rho\rho}\left(\rho\right) \,,
	\end{gathered}
\ee
\be
	\begin{gathered}
		\delta\gamma^{t} = f_{t}\left(\rho\right) \,,\quad \delta\gamma^{\psi_{\pm}} = \Omega_{\pm}f_{\psi_{\pm}}\left(\rho\right) \,,\quad \delta\gamma^{\rho} = \Delta f_{\rho}\left(\rho\right) \,,
	\end{gathered}
\ee
with all $f_{\mu\nu}$ and $f_{\mu}$ being radial functions that are regular as $\rho\rightarrow\rho_{+}$. We remark here that we allow for time-reversal violating terms. To simplify the problem, we can perform $\rho$-dependent coordinate transformations to eliminate as many as possible of the above terms. In particular, introducing coordinates $(\tilde{t},\tilde{\rho},\tilde{\psi}_{+},\tilde{\psi}_{-})$, related to $\left(t,\rho,\psi_{+},\psi_{-}\right)$ according to
\be
	\begin{gathered}
		\frac{d\tilde{\rho}}{\sqrt{\Delta\left(\tilde{\rho}\right)}} = \frac{d\rho}{\sqrt{\Delta\left(\rho\right)+\delta\gamma^{\rho\rho}\left(\rho\right)}} \,, \\
		d\tilde{t} = dt - \frac{\delta\gamma^{t\rho}\left(\rho\right)}{\Delta\left(\rho\right)+\delta\gamma^{\rho\rho}\left(\rho\right)}d\rho \,,\quad d\tilde{\psi}_{\pm} = d\psi_{\pm} - \frac{\delta\gamma^{\psi_{\pm}\rho}\left(\rho\right)}{\Delta\left(\rho\right)+\delta\gamma^{\rho\rho}\left(\rho\right)}d\rho \,,
	\end{gathered}
\ee
we can set
\be
	\delta\tilde{\gamma}^{\tilde{\rho}\tilde{\rho}} = 0 \,,\quad \delta\tilde{\gamma}^{\tilde{t}\tilde{\rho}} = 0 \,,\quad \delta\tilde{\gamma}^{\tilde{\psi}_{\pm}\tilde{\rho}} = 0 \,.
\ee
We will adopt these coordinates henceforth and drop the tildes to ease our notation.

\subsection{Solving the $\SL$ constraints}
With this setup for the generic truncation of the radial operator, we now explore the existence of three operators, $L_0$, $L_{+1}$ and $L_{-1}$, generating the $\SL$ algebra,
\be\label{eq:AlgebraConstraints}
	\left[L_{m},L_{n}\right] = \left(m-n\right)L_{m+n} \,,\quad m,n=0,\pm1 \,,
\ee
and whose Casimir precisely recovers a truncation of the radial operator of the type we just described,
\be\label{eq:CasimirConstraints}
	\mathcal{C}_2 = L_0^2 - \frac{1}{2}\left(L_{+1}L_{-1}+L_{-1}L_{+1}\right) \equiv \mathbb{O}_{\text{trunc}} \,.
\ee
Representations of $\SL$ are labeled by $\mathcal{C}_2$ and $L_0$. Using the $\mathbb{R}_{t}\times U\left(1\right)_{\phi}\times U\left(1\right)_{\psi}$ isometry of the background, we therefore make the following generic ansatz for the $\SL$ generators
\be
	\begin{gathered}
		L_0 = -\left(\beta_{t}\,\partial_{t} + \beta_{+}\Omega_{+}\,\partial_{+} + \beta_{-}\Omega_{-}\,\partial_{-}\right) \,, \\
		L_{\pm1} = \tilde{G}_{\pm}\left(x\right)\partial_{\rho} + \tilde{K}_{\pm}\left(x\right)\partial_{t} + \tilde{H}_{\pm}^{\left(+\right)}\left(x\right)\Omega_{+}\,\partial_{+} + \tilde{H}_{\pm}^{\left(-\right)}\left(x\right)\Omega_{-}\,\partial_{-} \,,
	\end{gathered}
\ee
where all components of the $L_0$ vector field are constants and all components of the $L_{\pm1}$ vector fields are spacetime functions $\tilde{X}_{\pm}\left(x\right)=\tilde{X}_{\pm}\left(t,\rho,\psi_{+},\psi_{-}\right)$. The algebra constraints \eqref{eq:AlgebraConstraints} and the Casimir constraints \eqref{eq:CasimirConstraints} will now be solved to fix the exponents $\beta_{\pm}$ and $\beta_{t}$, the functions $\tilde{X}_{\pm}\left(x\right)$ and the subleading terms $\delta\gamma^{\mu\nu}\left(\rho\right)$ and $\delta\gamma^{\mu}\left(\rho\right)$ appearing in the truncation of the radial operator.

We start with the algebra constraint $\left[L_{\pm1},L_0\right]=\pm L_{\pm1}$,
\be
	\left(\beta_{t}\,\partial_{t} + \beta_{+}\Omega_{+}\,\partial_{+} + \beta_{-}\Omega_{-}\,\partial_{-}\right)\tilde{X}_{\pm}\left(t,\rho,\psi_{+},\psi_{-}\right) = \pm \tilde{X}_{\pm}\left(t,\rho,\psi_{+},\psi_{-}\right) \,.
\ee
This can be used to eliminate the explicit $t$-dependence by introducing
\be
	\hat{\psi}_{\pm} = \psi_{\pm} - \frac{\beta_{\pm}}{\beta_{t}}\Omega_{\pm}t \quad \Rightarrow \quad \tilde{X}_{\pm}\left(t,\rho,\psi_{+},\psi_{-}\right) = e^{\pm t/\beta_{t}}X_{\pm}(\rho,\hat{\psi}_{+},\hat{\psi}_{-}) \,.
\ee
It is therefore favorable to work in the $(t,\rho,\hat{\psi}_{+},\hat{\psi}_{-})$ coordinates instead of $\left(t,\rho,\psi_{+},\psi_{-}\right)$ to solve the constraints. The generators in these coordinates are given by
\be
	\begin{gathered}
		L_0 = -\beta_{t}\,\partial_{t} \,, \\
		\ba
			L_{\pm1} &= e^{\pm t/\beta_{t}}\bigg[G_{\pm}(\rho,\hat{\psi}_{+},\hat{\psi}_{-})\partial_{\rho} + K_{\pm}(\rho,\hat{\psi}_{+},\hat{\psi}_{-})\partial_{t} \\
			&+ \hat{H}_{\pm}^{\left(+\right)}(\rho,\hat{\psi}_{+},\hat{\psi}_{-})\Omega_{+}\,\hat{\partial}_{+} + \hat{H}_{\pm}^{\left(-\right)}(\rho,\hat{\psi}_{+},\hat{\psi}_{-})\Omega_{-}\,\hat{\partial}_{-} \bigg] \,,
		\ea
	\end{gathered}
\ee
with $\hat{H}_{\pm}^{\left(i\right)} = H_{\pm}^{\left(i\right)} - \frac{\beta_{i}}{\beta_{t}}K_{\pm}$.

Moving forward, the Casimir constraints \eqref{eq:CasimirConstraints} imply that products of components of the $L_{\pm1}$ vector fields are independent of the azimuthal angles. We, thus, write $X_{\pm}(\rho,\hat{\psi}_{+},\hat{\psi}_{-}) = e^{\pm A(\hat{\psi}_{+},\hat{\psi}_{-})}\chi_{\pm}\left(\rho\right)$. In fact, taking derivatives with respect to $\hat{\psi}_{\pm}$ of the various Casimir and algebra constraints reveals that the dependence on the azimuthal angles in the exponential must be linear,
\be
	X_{\pm}(\rho,\hat{\psi}_{+},\hat{\psi}_{-}) = e^{\pm (\tau_{+}\hat{\psi}_{+} + \tau_{-}\hat{\psi}_{-})}\chi_{\pm}\left(\rho\right) \,.
\ee
The $\rho\rho$-component of the Casimir constraints \eqref{eq:CasimirConstraints} and the $\rho$-component of the last algebra constraint $\left[L_{+1},L_{-1}\right]=2L_0$ then completely fix the $\rho$-components of $L_{\pm1}$ to be, up to automorphisms,
\be
	g_{\pm}\left(\rho\right) = \mp\sqrt{\Delta} \,.
\ee
The $t\rho$- and $\psi_{\pm}\rho$-components of the Casimir constraints then imply
\be
	k_{+}\left(\rho\right) = k_{-}\left(\rho\right) \equiv k\left(\rho\right) \,,\quad \hat{h}_{+}^{\left(i\right)}\left(\rho\right) = \hat{h}_{-}^{\left(i\right)}\left(\rho\right) \equiv \hat{h}^{\left(i\right)}\left(\rho\right) \,.
\ee
The remaining algebra constraints from $\left[L_{+1},L_{-1}\right]=2L_0$ become
\begin{subequations}
	\be\label{eq:AC1}
		\sqrt{\Delta}\,k^{\prime} + \left(\frac{k}{\beta_{t}}+\sum_{i=+,-}\tau_{i}\Omega_{i}\hat{h}^{\left(i\right)}\right)k = \beta_{t} \,,
	\ee
	\be\label{eq:AC2}
		\sqrt{\Delta}\,\hat{h}^{\left(i\right)\prime} + \left(k+\sum_{j=+,-}\tau_{j}\Omega_{j}\hat{h}^{\left(j\right)}\right)\hat{h}^{\left(i\right)} = 0 \,,
	\ee
\end{subequations}
while the remaining Casimir constraints read
\begin{subequations}
	\begin{align}
		\label{eq:CC1} k^2 -1 &= \frac{\left(\rho_{+}-\rho_{-}\right)^2}{4\Delta}\beta^2 - f_{tt}\left(\rho\right) \,, \\
		\label{eq:CC2} k\hat{h}^{\left(\pm\right)} &= \frac{\left(\rho_{+}-\rho_{-}\right)^2}{4\Delta}\beta^2\left(1-\frac{\beta_{\pm}}{\beta_{t}}\right) - \left(f_{t\psi_{\pm}} - \frac{\beta_{\pm}}{\beta_{t}}f_{tt}\right) \,, \\
		\label{eq:CC3} \left[\hat{h}^{\left(\pm\right)}\right]^2 &= \frac{\left(\rho_{+}-\rho_{-}\right)^2}{4\Delta}\beta^2\left(1-\frac{\beta_{\pm}}{\beta_{t}}\right)^2 - \left(f_{\psi_{\pm}\psi_{\pm}} - 2\frac{\beta_{\pm}}{\beta_{t}}f_{t\psi_{\pm}} + \frac{\beta_{\pm}^2}{\beta_{t}^2}f_{tt}\right) \,, \\
		\begin{split}\label{eq:CC4}
			\hat{h}^{\left(+\right)}\hat{h}^{\left(-\right)} &= \frac{\left(\rho_{+}-\rho_{-}\right)^2}{4\Delta}\beta^2\left(1-\frac{\beta_{+}}{\beta_{t}}\right)\left(1-\frac{\beta_{-}}{\beta_{t}}\right) \\
			&\quad - \left(f_{\psi_{+}\psi_{-}} - \frac{\beta_{+}}{\beta_{t}}f_{t\psi_{+}}-\frac{\beta_{-}}{\beta_{t}}f_{t\psi_{-}} + \frac{\beta_{+}\beta_{-}}{\beta_{t}^2}f_{tt}\right) \,.
		\end{split}
	\end{align}
\end{subequations}

Let us sketch how to solve these. The algebra constraints can be solved for the functions $k\left(\rho\right)$ and $\hat{h}^{\left(\pm\right)}\left(\rho\right)$. The integration constants associated with the differential equations \eqref{eq:AC1}-\eqref{eq:AC2} are then fixed by the near-horizon behaviors of these functions as dictated by the Casimir constraints \eqref{eq:CC1}-\eqref{eq:CC4}. These near-horizon behaviors also result in a relation between the constants $\beta_{t}$, $\beta_{\pm}$ and $\tau_{\pm}$,
\be\label{eq:SL2Rbetas}
	\beta_{t}\left(1-\beta\sum_{i=+,-}\tau_{i}\Omega_{i}\right) = \beta\left(1-\sum_{i=+,-}\tau_{i}\beta_{i}\Omega_{i}\right) \,.
\ee
The final expressions of the generators $L_0$, $L_{\pm1}$ after translating back into $\left(t,\rho,\psi_{+},\psi_{-}\right)$ coordinates are
\be\label{eq:SL2RGenericGenerators}
	\begin{gathered}
		L_0 = -\left(\beta_{t}\,\partial_{t} + \beta_{+}\Omega_{+}\,\partial_{+} + \beta_{-}\Omega_{-}\,\partial_{-}\right) \,, \\
		\ba
			L_{\pm1} &= \exp\left\{\pm\left[t/\beta + \tau_{+}\left(\psi_{+}-\Omega_{+}t\right) + \tau_{-}\left(\psi_{-}-\Omega_{-}t\right)\right]\right\} \\
			&\quad\times\left[\mp\sqrt{\Delta}\,\partial_{\rho} - \partial_{\rho}\left(\sqrt{\Delta}\right)L_0 + \frac{\rho_{+}-\rho_{-}}{2\sqrt{\Delta}}\left(\beta K+L_0\right) \right] \,,
		\ea
	\end{gathered}
\ee
with the associated Casimir given by
\be\label{eq:SL2RGenericCasimir}
	\mathcal{C}_2 = \partial_{\rho}\,\Delta\,\partial_{\rho} - \frac{\left(\rho_{+}-\rho_{-}\right)^2}{4\Delta}\beta^2K^2 + \frac{\rho_{+}-\rho_{-}}{\rho-\rho_{-}}L_0\left(L_0+\beta K\right) \,,
\ee
where $K=\partial_{t} + \Omega_{+}\,\partial_{+} + \Omega_{-}\,\partial_{-}$ is the Killing vector relative to which the event horizon is a Killing horizon. Interestingly, by working in the advanced (retarded) null coordinates \eqref{eq:NullCoordinates}, one can then check that the above vectors fields are automatically regular at the future (past) event horizon. We also remind here that one has the freedom of performing the coordinate transformations
\be
	\rho\rightarrow\rho+\Delta^2g_{\rho}\left(\rho\right) \,,\quad t\rightarrow t+g_{t}\left(\rho\right) \,,\quad \psi_{\pm}\rightarrow\psi_{\pm}+g_{\psi_{\pm}}\left(\rho\right) \,,
\ee
for arbitrary radial functions $g_{\rho}\left(\rho\right)$, $g_{t}\left(\rho\right)$, $g_{\psi_{+}}\left(\rho\right)$ and $g_{\psi_{-}}\left(\rho\right)$ that are regular at the event horizon; these give rise to non-zero subleading contributions $\delta\gamma^{\rho\rho}$, $\delta\gamma^{t\rho}$ and $\delta\gamma^{\psi_{\pm}\rho}$ respectively.

Let us now start discussing the properties of these truncations of the radial operators that are equipped with $\SL$ symmetries as demonstrated above. First of all, we notice that the $\SL$ symmetry generated by \eqref{eq:SL2RGenericGenerators} is in general spontaneously broken down to $U\left(1\right)$, generated by $L_0$, from the periodic identification of the azimuthal coordinates when $\tau_{\pm}\ne0$. An interesting remark here is that the $\tau_{\pm}\ne0$ generators can be obtained from the globally defined ones, with $\tau_{\pm}=0$, via temporal translations involving the co-rotating azimuthal angles,
\be
	L_{m}^{\left(\tau_{\pm}=0\right)} \xrightarrow{t\rightarrow t + \beta\sum_{i=+,-}\tau_{i}\left(\psi_{i}-\Omega_{i}t\right)} L_{m}^{\left(\tau_{\pm}\ne0\right)} \,.
\ee

Furthermore, the Casimir operator \eqref{eq:SL2RGenericCasimir} has the property of preserving the characteristic exponents near the event horizon by construction. However, it does not in general capture any near-zone truncation of the radial operator. For this to happen, the Casimir operator must exactly reproduce all the static terms in the radial operator as well. This additional requirement gives two infinite towers of near-zone $\SL$'s controlled by the parameters $\tau_{\pm}$. The two towers can be labeled by a sign $\sigma=+,-$ and correspond to fixing the parameters $\beta_{\pm}$ to
\be
	\beta_{\sigma} = \beta \,,\quad \beta_{-\sigma} = 0 \,\quad \text{for $\sigma=+$ OR $\sigma=-$} \,.
\ee
These near-zone $\SL$'s are generated by the vector fields
\be\label{eq:SL2RNZGenerators}
	\begin{gathered}
		L_0^{\left(\sigma\right)} = -\left(\beta_{t}^{\left(\sigma\right)}\,\partial_{t} + \beta\Omega_{\sigma}\,\partial_{\sigma}\right) \,, \\
		\ba
			L_{\pm1}^{\left(\sigma\right)} &= \exp\left\{\pm\left[t/\beta + \tau_{+}\left(\psi_{+}-\Omega_{+}t\right)+\tau_{-}\left(\psi_{-}-\Omega_{-}t\right)\right]\right\} \\
			&\quad\times\bigg[\mp\sqrt{\Delta}\,\partial_{\rho} + \partial_{\rho}\left(\sqrt{\Delta}\right)\left(\beta_{t}^{\left(\sigma\right)}\,\partial_{t} + \beta\Omega_{\sigma}\,\partial_{\sigma}\right) \\
			&\quad\quad + \frac{\rho_{+}-\rho_{-}}{2\sqrt{\Delta}}\left[\left(\beta-\beta_{t}^{\left(\sigma\right)}\right)\partial_{t}+ \beta\Omega_{-\sigma}\,\partial_{-\sigma}\right] \bigg] \,,
		\ea
	\end{gathered}
\ee
and the associated Casimir operator is given by
\be\ba\label{eq:SL2RNZCasimir}
	\mathcal{C}_2^{\left(\sigma\right)} &= \partial_{\rho}\,\Delta\,\partial_{\rho} - \frac{\left(\rho_{+}-\rho_{-}\right)^2}{4\Delta}\beta^2\left(\partial_{t}+\Omega_{+}\,\partial_{+}+\Omega_{-}\,\partial_{-}\right)^2 \\
	&\quad- \frac{\rho_{+}-\rho_{-}}{\rho-\rho_{-}}\left(\beta_{t}^{\left(\sigma\right)}\,\partial_{t} + \beta\Omega_{\sigma}\,\partial_{\sigma}\right)\left[\left(\beta-\beta_{t}^{\left(\sigma\right)}\right)\partial_{t}+\beta\Omega_{-\sigma}\partial_{-\sigma}\right] \,,
\ea\ee
where $\beta_{t}^{\left(\sigma\right)}$, $\tau_{+}^{\left(\sigma\right)}$ and $\tau_{-}^{\left(\sigma\right)}$ are related according to
\be
	\beta_{t}^{\left(\sigma\right)}\left(1-\beta\tau_{+}^{\left(\sigma\right)}\Omega_{+}-\beta\tau_{-}^{\left(\sigma\right)}\Omega_{-}\right) = \beta\left(1-\beta\tau_{\sigma}^{\left(\sigma\right)}\Omega_{\sigma}\right) \,.
\ee
Supplemented with the near-zone-approximation-preserving temporal translations $t\rightarrow t + g_{t}\left(\rho\right)$, this exhausts all the possible near-zone $\SL$ symmetries.

If we want these near-zone $\SL$ symmetries to be globally defined, one must further impose $\tau_{\pm}=0$, in which case we must have $\beta_{t}^{\left(\sigma\right)}=\beta$ and we end up with the fact that the most general globally defined near-zone $\SL$ symmetries are, up to $\rho$-dependent temporal translations, the Love symmetries presented in Section \ref{sec:SL2R}.

\subsection{Extension to $\SL\times\SL$}
We will finish with a short investigation on the possibility of extending the above-found $\SL$ symmetries, for which the radial operator truncations preserve the characteristic exponents near the event horizon, into the full $2$-d global conformal structure $\SL\times\SL$.

Consider, therefore, two general such $\SL$ symmetries generated by vector fields $L_{m}$ and $\bar{L}_{m}$ of the form \eqref{eq:SL2RGenericGenerators}. They are characterized by parameters $\{\beta_{t},\beta_{\pm},\tau_{\pm}\}$ and $\{\bar{\beta}_{t},\bar{\beta}_{\pm},\bar{\tau}_{\pm}\}$ respectively, with each set of parameters being subject to the relation \eqref{eq:SL2Rbetas}. By working out the requirement that
\be
	\left[L_{m},\bar{L}_{n}\right] = 0 \,,\quad m,n=0,\pm1 \,,
\ee
we extract the following summarizing condition
\be
	L_0 + \bar{L}_0 = -\beta K \,,
\ee
that is, $\bar{\beta}_{t} = \beta - \beta_{t}$ and $\bar{\beta}_{\pm} = \beta - \beta_{\pm}$. The associated Casimirs turn out to be exactly the same and can be written in the suggestive form
\be
	\mathcal{C}_2 = \bar{\mathcal{C}}_2 = \partial_{\rho}\,\Delta\,\partial_{\rho} - \frac{\rho_{+}-\rho_{-}}{\rho-\rho_{+}}\left(\frac{L_0+\bar{L}_0}{2}\right)^2 + \frac{\rho_{+}-\rho_{-}}{\rho-\rho_{-}}\left(\frac{L_0-\bar{L}_0}{2}\right)^2 \,.
\ee
In a $\text{CFT}_2$ interpretation, this shows that the characteristic exponents near the outer and inner horizons are the (squares of half of the) scaling dimension and spin-weight of the spacetime scalar field under the action of the $\text{CFT}_2$ dilaton and Lorentz generators respectively. In this language, the above truncations of the radial operator seek to preserve the scaling dimension but allow to approximate the $\text{CFT}_2$ spin-weight. The remaining generators can similarly be written as
\be\ba
	L_{\pm1} &= e^{\pm\left[t/\beta + \tau_{+}\left(\psi_{+}-\Omega_{+}t\right) + \tau_{-}\left(\psi_{-}-\Omega_{-}t\right)\right]} \\
	&\quad\times\left[\mp\sqrt{\Delta}\,\partial_{\rho} - \sqrt{\frac{\rho-\rho_{-}}{\rho-\rho_{+}}}\frac{L_0+\bar{L}_0}{2} - \sqrt{\frac{\rho-\rho_{+}}{\rho-\rho_{-}}}\frac{L_0-\bar{L}_0}{2}\right] \,, \\
	\bar{L}_{\pm1} &= e^{\pm\left[t/\beta + \bar{\tau}_{+}\left(\psi_{+}-\Omega_{+}t\right) + \bar{\tau}_{-}\left(\psi_{-}-\Omega_{-}t\right)\right]} \\
	&\quad\times\left[\mp\sqrt{\Delta}\,\partial_{\rho} - \sqrt{\frac{\rho-\rho_{-}}{\rho-\rho_{+}}}\frac{L_0+\bar{L}_0}{2} + \sqrt{\frac{\rho-\rho_{+}}{\rho-\rho_{-}}}\frac{L_0-\bar{L}_0}{2}\right] \,.
\ea\ee

Last, for the case of near-zone $\SL\times\SL$'s, there are two towers of such enhancements labeled by a sign $\sigma=+,-$. They correspond to $\left(\beta_{\sigma},\beta_{-\sigma}\right)=\left(\beta,0\right)$ and, thus, $\left(\bar{\beta}_{\sigma},\bar{\beta}_{-\sigma}\right)=\left(0,\beta\right)$. One of the outcomes of the current analysis is then that near-zone $\SL\times\SL$'s can \textit{never} be globally defined. The best one can do is to have near-zone $\SL\times\SL$ symmetries spontaneously broken down to $\SL\times U\left(1\right)$ from the periodic identification of the azimuthal angles. These are precisely the enhancements of the Love symmetries presented in Section \ref{sec:Properties}. We remark here that the construction of near-zone $\SL\times\SL$ symmetries described above also contains the Kerr/CFT proposal for $5$-dimensional rotating black holes in~\cite{Krishnan:2010pv} as a special case, corresponding to a different near-zone truncation with $\beta_{t}=\beta\frac{r_{+}-r_{-}}{2r_{+}}$ and $\bar{\beta}_{t}=\beta\frac{r_{+}+r_{-}}{2r_{+}}$ and which has the unique property of preserving the characteristic exponent at the inner horizon as well.