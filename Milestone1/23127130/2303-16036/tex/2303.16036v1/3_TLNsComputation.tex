\section{Scalar Love numbers of 5-d Myers-Perry black hole}
\label{sec:TLNsComputation}

We will now apply the tools presented in the previous section to compute the static scalar susceptibilities of $5$-dimensional Myers-Perry black holes and identify the corresponding static scalar ($s=0$) Love numbers. We will begin with a description of the equations of motion to be solved. These will be supplemented with boundary conditions as dictated by the $1$-body worldline EFT which motivates the use of the near-zone approximation. Then, this microscopic computation will be matched onto the worldline EFT $1$-point function definition to extract the conservative and dissipative contributions of the static scalar susceptibilities. We will finally analyze various cases of interest associated with the vanishing or the non-vanishing/RG flow of the static scalar Love numbers.

\subsection{Equations of motion and boundary conditions}
The $5$-dimensional Myers-Perry black hole geometry is presented in Appendix \ref{sec:Ap5dMPGeometry}. The massless Klein-Gordon equation in this background is separable,
\be
	\nabla^2 \Phi = \frac{4}{\Sigma}\left[\mathbb{O}_{\text{full}} - \mathbb{P}_{\text{full}}\right]\Phi = 0 \,,
\ee
with the radial and angular operators given by, after introducing the variable $\rho=r^2$,
\be\ba\label{eq:FullEOM}
	\mathbb{O}_{\text{full}} &\equiv \partial_{\rho}\,\Delta\,\partial_{\rho} - \frac{\left(a^2-b^2\right)}{4}\left(\frac{1}{\rho+a^2}\,\partial_{\phi}^2 - \frac{1}{\rho+b^2}\,\partial_{\psi}^2\right)  - \frac{1}{4}\left(\rho+a^2+b^2\right)\partial_{t}^2	\\
	&\quad- \frac{\left(\rho+a^2\right)\left(\rho+b^2\right)r_{s}^2}{4\Delta}\left(\partial_{t} + \frac{a}{\rho+a^2}\,\partial_{\phi} + \frac{b}{\rho+b^2}\,\partial_{\psi}\right)^2 \,, \\
	\mathbb{P}_{\text{full}} &\equiv -\frac{1}{4}\left[ \triangle_{\mathbb{S}^3}^{\left(0\right)} + \left(a^2\sin^2\theta+b^2\cos^2\theta\right)\partial_{t}^2 \right] \,.
\ea\ee
In the above expressions, the discriminant $\Delta$ is a quadratic polynomial in $\rho$,
\be
	\Delta = \left(\rho-\rho_{+}\right)\left(\rho-\rho_{-}\right) \,,
\ee
with $\rho_{\pm}$ the locations of the outer (``$+$'') and and inner (``$-$'') horizons and the warp factor $\Sigma$ for the $5$-d Myers-Perry black hole is given by
\be
	\Sigma = r^2 + a^2\cos^2\theta + b^2\sin^2\theta \,.
\ee
We have also identified the scalar ($s=0$) Laplace-Beltrami operator on $\mathbb{S}^3$ in the current direction cosine angular coordinates,
\be
	\triangle_{\mathbb{S}^3}^{\left(0\right)} \equiv \frac{1}{\sin\theta\cos\theta}\partial_{\theta}\left(\sin\theta\cos\theta\,\partial_{\theta}\right) + \frac{1}{\sin^2\theta}\,\partial_{\phi}^2 + \frac{1}{\cos^2\theta}\,\partial_{\psi}^2 \,.
\ee
which is used to define a modified spherical harmonics expansion in the static case from which the Love numbers are more naturally extracted. This expansion is illustrated in detail in Appendix \ref{sec:ApSphericalHarmonics}.

After separating the variables,
\be\label{eq:PhiSepVar}
	\Phi_{\omega\ell m j}\left(t,\rho,\theta,\phi,\psi\right) = e^{-i\omega t}e^{im\phi}e^{ij\psi}R_{\omega\ell m j}\left(\rho\right) S_{\omega\ell m j}\left(\theta\right) \,,
\ee
this $5$-d scalar Teukolsky equation decomposes to
\be\ba
	\mathbb{O}_{\text{full}}\Phi_{\omega\ell m j} &= \hat{\ell}(\hat{\ell}+1)\Phi_{\omega\ell m j} \,, \\
	\mathbb{P}_{\text{full}}\Phi_{\omega\ell m j} &= \hat{\ell}(\hat{\ell}+1)\Phi_{\omega\ell m j} \,,
\ea\ee
with
\be
	\hat{\ell} \equiv \frac{\ell}{d-3} = \frac{\ell}{2} \,,
\ee
and $\ell$ an effective orbital number which is in general non-integer for $\omega\ne0$.


\subsubsection{Near-zone approximation}
In order to match observables onto the $1$-body worldline EFT according to \eqref{eq:1ptNewtonianMatching}, we should solve the massless Klein-Gordon equation in the appropriate regime. The physical setup consists of a binary system of compact bodies during the early stages of their inspiraling phase where a Post-Newtonian expansion is accurate. Centering the body of interest at the origin, the companion sources perturbations with frequency equal to the orbital frequency of the system $\omega=\omega_{\text{orb}}$. The system loses energy by emitting radiation with frequency $\omega_{\text{rad}}\propto \omega_{\text{orb}}$ which is then detected by an observer located at infinity through, for example, an interferometer.

The worldline EFT arises after integrating out the short scale internal degrees of freedom of the centered compact body, i.e. it is valid for low frequency perturbations. Furthermore, the $1$-body worldline EFT ignores the dynamics of the companion body sourcing the perturbations and a second condition for its validity is that the frequency of the perturbations is low with respect to the inverse separation of the two bodies. This combination of conditions defines the \textit{near-zone region}. For the current configuration of a black hole of outer horizon $r_{+}$, the near-zone approximation consists of working in the regime~\cite{Chia:2020yla,Charalambous:2021kcz,Starobinsky:1973aij,Starobinskil:1974nkd,Castro:2010fd,Maldacena:1997ih}
\be\label{eq:NZapprox}
	\omega \left(r-r_{+}\right) \ll 1 \,,\quad \omega r_{+} \ll 1\,.
\ee
In the near-zone region, one imposes the asymptotic boundary condition
\be\label{eq:Asympbc}
	R_{\omega\ell m j} \xrightarrow{r\rightarrow\infty} \bar{\mathcal{E}}_{\ell m j}\left(\omega\right)\,r^{\ell} = \bar{\mathcal{E}}_{\ell m j}\left(\omega\right)\,\rho^{\hat{\ell}} \,,
\ee
indicating the presence of a source at large distances with multipole moments $\bar{\mathcal{E}}_{\ell m j}\left(\omega\right)$, along with the ingoing boundary condition at the event horizon. Ingoing boundary condition at the future/past event horizon is imposed by requiring ingoing waves at the horizon in advanced ($+$)/retarded ($-$) coordinates,
\be
	\Phi_{\omega\ell m j} \xrightarrow{r\rightarrow r_{+}} T_{\ell m j}^{\left(\pm\right)}\left(\omega\right)\,e^{-i\omega t_{\pm}}e^{im\varphi_{\pm}}e^{ijy_{\pm}}S_{\omega\ell m j}\left(\theta\right) \,,
\ee
with $T_{\ell m j}^{\left(\pm\right)}\left(\omega\right)$ the transmission amplitudes. The relation between advanced/retarded coordinates $\left(t_{\pm},r,\theta,\varphi_{\pm},y_{\pm}\right)$ and Boyer-Lindquist coordinates  $\left(t,r,\theta,\phi,\psi\right)$ is given in Appendix \ref{sec:Ap5dMPGeometry} and implies
\be\label{eq:NHbc}
	\begin{gathered}
		R_{\omega\ell m j} \sim \left(\rho-\rho_{+}\right)^{\pm iZ_{+}\left(\omega\right)}\left(\rho-\rho_{-}\right)^{\pm iZ_{-}\left(\omega\right)}\,\,\,\,\,\,\, \text{as $\rho\rightarrow \rho_{+}$} \,, \\
		Z_{\pm}\left(\omega\right) \equiv \pm\frac{r_{\pm}}{2}\frac{\rho_{s}}{\rho_{+}-\rho_{-}}\left(m\Omega_{\phi}^{\left(\pm\right)}+j\Omega_{\psi}^{\left(\pm\right)}-\omega\right) \,,
	\end{gathered}
\ee
where $\Omega_{\phi}^{\left(\pm\right)}=\frac{a}{\rho_{\pm}+a^2}$ and $\Omega_{\psi}^{\left(\pm\right)}=\frac{b}{\rho_{\pm}+b^2}$. Physically, of course, we are only interested in regularity at the future event horizon and $\Omega_{\phi}^{\left(+\right)}\equiv\Omega_{\phi}$ and $\Omega_{\psi}^{\left(+\right)}\equiv\Omega_{\psi}$ are identified as the black hole angular velocities with respect to the two rotation planes.

An important remark here is that the near-zone approximation extends beyond the near-horizon or the low frequency regimes. In particular, not only does it preserve the near-horizon dynamics in the radial operator for any frequency $\omega$, but it also overlaps with the asymptotically flat far-zone region $r\gg r_{+}$ where outgoing boundary conditions are imposed. The overlapping intermediate region $r_{+}\ll r \ll \omega^{-1}$ then serves as a matching region that probes the response of the centered body in the outgoing waves that are detected at infinity.

It should be noted that the near-zone approximation is not unique as there are infinitely many ways to truncate the equations of motion as long as they differ by subleading terms. In practice, the truncation is done such that the equations of motion are exactly solvable in terms of elementary functions. There are two particular near-zone truncations of interest in the current work controlled by a sign $\sigma=\pm $. We split the radial operator as
\be\label{eq:NZRadial1}
	\mathbb{O}_{\text{full}} = \partial_{\rho}\,\Delta\,\partial_{\rho} + V_0^{\left(\sigma\right)} + \epsilon V_1^{\left(\sigma\right)} \,,
\ee
with $\epsilon$ a formal parameter which is equal to unity for the full equations of motion and equal to zero for the near-zone approximation and
\be\ba\label{eq:NZRadial2}
	V_0^{\left(\sigma\right)} = &- \frac{\rho_{s}^2\rho_{+}}{4\Delta}\left(\partial_{t}+\Omega_{\phi}\,\partial_{\phi}+\Omega_{\psi}\,\partial_{\psi}\right)^2 - \frac{a^2-b^2}{4\left(\rho-\rho_{-}\right)}\left(\partial_{\phi}^2-\partial_{\psi}^2\right) \\
	&- \frac{\rho_{s}^2\rho_{+}}{2\left(\rho_{+}-\rho_{-}\right)\left(\rho-\rho_{-}\right)} \left(\Omega_{\phi}-\sigma\,\Omega_{\psi}\right)\partial_{t}\left(\partial_{\phi}-\sigma\,\partial_{\psi}\right) \,,
\ea\ee
\be\ba\label{eq:NZRadial3}
	V_1^{\left(\sigma\right)} &= \frac{\rho_{s}\left(a-\sigma b\right)\left[\rho_{s}-\left(a+\sigma b\right)^2\right]}{4\left(\rho_{+}-\rho_{-}\right)\left(\rho-\rho_{-}\right)}\partial_{t}\left(\partial_{\phi}-\sigma\,\partial_{\psi}\right) \\
	&- \frac{\rho_{s}}{4\left(\rho-\rho_{-}\right)}\left[\rho_{s}\,\partial_{t}+\left(a+\sigma b\right)\left(\partial_{\phi}+\sigma\,\partial_{\psi}\right)\right]\partial_{t} - \frac{1}{4}\left(\rho+a^2+b^2+\rho_{s}\right)\partial_{t}^2 \,.
\ea\ee

For the angular operator, we use the splitting
\be\label{eq:NZAngular}
	\mathbb{P}_{\text{full}} = -\frac{1}{4}\left[ \triangle_{\mathbb{S}^3}^{\left(0\right)} + \epsilon\left(a^2\sin^2\theta+b^2\cos^2\theta\right)\partial_{t}^2 \right] \,,
\ee
that is, we are near-zone approximating it with the static angular operator. In the static limit, the angular problem is solved by $S_{\ell m j}\left(\theta\right)$, given in Appendix \ref{sec:ApSphericalHarmonics}, from which $\ell$ is set to be an orbital number ranging in the set of whole numbers, $\ell\in\mathbb{N}$, $m$ is identified as a spherical harmonics integer azimuthal number $\left|m\right|\le\ell$ and $j$ is a second integer azimuthal number ranging from $-\left(\ell-\left|m\right|\right)$ up to $\ell-\left|m\right|$, but with step $2$.


\subsection{Near-zone solution and scalar Love numbers}
After separating the variables as in \eqref{eq:PhiSepVar} and introducing
\be
	x \equiv \frac{r^2-r_{+}^2}{r_{+}^2-r_{-}^2} = \frac{\rho-\rho_{+}}{\rho_{+}-\rho_{-}} \,,
\ee
the near-zone equation of motion for the radial wavefunction can be massaged into
\be
	\left[ \frac{d}{dx}\,x\left(1+x\right)\frac{d}{dx} + \frac{Z_{+}^2\left(\omega\right)}{x}-\frac{\tilde{Z}_{-}^{\left(\sigma\right)2}\left(\omega\right)}{1+x} \right] R_{\omega\ell m j} = \hat{\ell}(\hat{\ell}+1)R_{\omega\ell m j} \,,
\ee
with $Z_{+}\left(\omega\right)$ given in \eqref{eq:NHbc} and dictating the near-horizon behavior of the solution and
\be\label{eq:NZZetaMinus}
	\tilde{Z}_{-}^{\left(\sigma\right)}\left(\omega\right) \equiv -\frac{r_{+}}{2}\frac{\rho_{s}}{\rho_{+}-\rho_{-}}\left(m\Omega_{\psi}+j\Omega_{\phi}-\sigma\omega\right) \,.
\ee
We note in particular that $\tilde{Z}_{-}^{\left(\sigma\right)}\left(\omega=0\right)=Z_{-}\left(\omega=0\right)$ in \eqref{eq:NHbc} reflecting how the near-zone approximation becomes exact in the static limit. The above differential equation has three regular singular points at $x=0$, $x=-1$ and $x\rightarrow\infty$ with the characteristic exponents given in the following table:
\begin{table}[h]
	\centering
	\begin{tabular}{c||c|c|c}
		singular point & $x=0$ & $x=-1$ & $x\rightarrow\infty$ \\
		\hline
		characteristic exponent 1 & $+iZ_{+}\left(\omega\right)$ & $+i\tilde{Z}_{-}^{\left(\sigma\right)}\left(\omega\right)$ & $\hat{\ell}$ \\
		\hline
		characteristic exponent 2 & $-iZ_{+}\left(\omega\right)$ & $-i\tilde{Z}_{-}^{\left(\sigma\right)}\left(\omega\right)$ & $-\hat{\ell}-1$
	\end{tabular}
\end{table}

The differential equation can be solved analytically in terms of Euler's hypergeometric functions. For future convenience, we introduce the parameters
\be\ba\label{eq:GammaPlusMinus}
	\Gamma^{\left(\sigma\right)}_{\pm\sigma}\left(\omega\right) &\equiv Z_{+}\left(\omega\right) \mp \sigma\, \tilde{Z}_{-}^{\left(\sigma\right)}\left(\omega\right) \\
	&= \frac{r_{+}}{2}\frac{\rho_{s}}{\rho_{+}-\rho_{-}}\left[\left(m\pm \sigma j\right)\left(\Omega_{\phi}\pm\sigma\Omega_{\psi}\right)-\omega\left(1\pm1\right)\right] \,.
\ea\ee
The solution that is regular at the future event horizon then reads
\be\ba\label{eq:NZRadialSolution}
	{}&R_{\omega\ell m j} = \bar{R}_{\ell m j}\left(\omega\right)\,\left(\frac{x}{1+x}\right)^{iZ_{+}\left(\omega\right)} \\
	&\left(1+x\right)^{i\Gamma_{\pm\sigma}^{\left(\sigma\right)}\left(\omega\right)}\,{}_2F_1\left(\hat{\ell}+1+i\Gamma^{\left(\sigma\right)}_{\pm\sigma}\left(\omega\right),-\hat{\ell}+i\Gamma^{\left(\sigma\right)}_{\pm\sigma}\left(\omega\right);1+2iZ_{+}\left(\omega\right);-x\right) \,,
\ea\ee
where $\bar{R}_{\ell m j}\left(\omega\right)$ is fixed from the asymptotic boundary condition \eqref{eq:Asympbc} to be proportional to the source moments harmonic modes $\bar{\mathcal{E}}_{\ell mj}\left(\omega\right)$ according to (see Appendix~\ref{sec:ApSchwarzschildLimit} for more details)
\be
	\bar{R}_{\ell mj}\left(\omega\right) = \bar{\mathcal{E}}_{\ell mj}\left(\omega\right) \left(\rho_{+}-\rho_{-}\right)^{\hat{\ell}}\frac{\Gamma\left(\hat{\ell}+1+i\Gamma^{\left(\sigma\right)}_{+\sigma}\left(\omega\right)\right)\Gamma\left(\hat{\ell}+1+i\Gamma^{\left(\sigma\right)}_{-\sigma}\right)}{\Gamma\left(2\hat{\ell}+1\right)\Gamma\left(1+2iZ_{+}\left(\omega\right)\right)} \,.
\ee

Up to this point in solving the static problem, we only assumed that the orbital number $\ell$ is a generic real number. 
This prescription is known
as ``analytic continuation.''
It is often used for the
Newtonian matching within the worldline EFT from which the Love numbers are defined~\cite{Kol:2011vg,Chia:2020yla,Charalambous:2021mea,Ivanov:2022hlo}, 
see also~\cite{Starobinsky:1973aij,Starobinskil:1974nkd} for the first practical 
use of this approach. Namely, in order to extract the response coefficients from the above microscopic computation, one should first analytically continue $\ell$ to be a real number, then expand around large $r$ to read the coefficient in front of the $r^{-\ell-2}$ term and then send $\ell$ to its physical integer values at the end. Doing this, we find, before sending $\ell$ to take its physical values,
\be\label{eq:NZSLNs}
	k_{\ell m j}\left(\omega\right) = \frac{\Gamma\left(-2\hat{\ell}-1\right)\Gamma\left(\hat{\ell}+1+i\Gamma^{\left(\sigma\right)}_{+\sigma}\left(\omega\right)\right)\Gamma\left(\hat{\ell}+1+i\Gamma^{\left(\sigma\right)}_{-\sigma}\right)}{\Gamma\left(2\hat{\ell}+1\right)\Gamma\left(-\hat{\ell}+i\Gamma^{\left(\sigma\right)}_{+\sigma}\left(\omega\right)\right)\Gamma\left(-\hat{\ell}+i\Gamma^{\left(\sigma\right)}_{-\sigma}\right)}\left(\frac{\rho_{+}-\rho_{-}}{\rho_{s}}\right)^{2\hat{\ell}+1} \,,
\ee
which can be massaged using the mirror formula for the $\Gamma$-functions into the more transparent result
\be\ba
	k_{\ell m j}&\left(\omega\right) = A_{\ell m j}\left(\omega\right)\times\bigg\{- i\sinh2\pi Z_{+}\left(\omega\right) \\	&+\tan\pi\hat{\ell}\cosh\pi\Gamma^{\left(\sigma\right)}_{+\sigma}\left(\omega\right)\cosh\pi\Gamma^{\left(\sigma\right)}_{-\sigma} - \cot\pi\hat{\ell}\sinh\pi\Gamma^{\left(\sigma\right)}_{+\sigma}\left(\omega\right)\sinh\pi\Gamma^{\left(\sigma\right)}_{-\sigma} \bigg\} \,,
\ea\ee
where $A_{\ell m j}\left(\omega\right)$ is a real constant given by
\be
	A_{\ell m j}\left(\omega\right) \equiv \frac{\left|\Gamma\left(\hat{\ell}+1+i\Gamma^{\left(\sigma\right)}_{+\sigma}\left(\omega\right)\right)\right|^2\left|\Gamma\left(\hat{\ell}+1+i\Gamma^{\left(\sigma\right)}_{-\sigma}\right)\right|^2}{2\pi\,\Gamma\left(2\hat{\ell}+1\right)\Gamma\left(2\hat{\ell}+2\right)}\left(\frac{\rho_{+}-\rho_{-}}{\rho_{s}}\right)^{2\hat{\ell}+1} \,.
\ee
The conservative/dissipative decomposition \eqref{eq:RCsConsDiss} then implies
\be\label{eq:5dMPRCsDiss}
	k_{\ell mj}^{\text{diss}}\left(\omega\right) = -A_{\ell mj}\left(\omega\right)\sinh2\pi Z_{+}\left(\omega\right) \,,
\ee
\be\ba\label{eq:5dMPLove}
	k_{\ell mj}^{\text{Love}}\left(\omega\right) = A_{\ell mj}\left(\omega\right)\times&\bigg\{\tan\pi\hat{\ell}\cosh\pi\Gamma^{\left(\sigma\right)}_{+\sigma}\left(\omega\right)\cosh\pi\Gamma^{\left(\sigma\right)}_{-\sigma} \\
	&- \cot\pi\hat{\ell}\sinh\pi\Gamma^{\left(\sigma\right)}_{+\sigma}\left(\omega\right)\sinh\pi\Gamma^{\left(\sigma\right)}_{-\sigma} \bigg\} \,.
\ea\ee
At this point, we would like stress out that the above results should be trusted only for small values of $\omega$. Indeed, the near-zone approximation is accurate only for low frequencies. In particular, in the near-zone split \eqref{eq:NZRadial1}-\eqref{eq:NZAngular}, we are already approximating at order $\mathcal{O}\left(\omega a,\omega b,\omega^2\right)$~\cite{Charalambous:2022rre}. Nevertheless, the above result is accurate in the static limit which is also the case of interest. In the rest of this work, we will be using the above $\omega$-dependent expressions but it should always be kept in mind that they are accurate only in the static limit for non-zero spin parameters.
The Love numbers behaviors following from Eq.~\eqref{eq:5dMPRCsDiss}
are listed in table~\ref{tbl:StaticSLNs}
and in table~\ref{tbl:WSLNs}
for static and time-dependent cases, 
respectively. Let us discuss the 
Love number phenomenology in detail. 


\begin{table}[!t]
	\centering
	\begin{tabular}{|l|l||l|}
		\hline
		\multicolumn{2}{|c||}{Range of parameters} & Behavior of $k^{\text{Love}}_{\ell mj}\left(\omega=0\right)$ \\
		\hline\hline
		\multicolumn{2}{|c||}{$\ell\in2\mathbb{N}+1$} & \multicolumn{1}{c|}{Running} \\
		\hline
		\multirow{2}{*}{$\ell\in2\mathbb{N}$} & \multicolumn{1}{c||}{$\left|a\right|=\left|b\right|$ OR $\left|m\right|=\left|j\right|$} & \multicolumn{1}{c|}{Vanishing} \\
		\cline{2-3}
		& \multicolumn{1}{c||}{$\left|a\right|\ne\left|b\right|$ AND $\left|m\right|\ne\left|j\right|$} & \multicolumn{1}{c|}{Running} \\
		\hline
	\end{tabular}
	\caption{Behavior of static scalar Love numbers as a function of the $5$-d Myers-Perry black hole angular momenta $a$ and $b$ and the scalar field perturbation orbital number $\ell$ and azimuthal numbers $m$ and $j$. For generic angular momenta and azimuthal and orbital numbers, the static scalar Love numbers exhibit a classical RG flow. The only exception is when the orbital number is even ($\hat{\ell}$ is integer) and the angular momenta or the azimuthal numbers are equal in magnitude in which case the static Love numbers turn out to vanish.
	}
	\label{tbl:StaticSLNs}
\end{table}

\begin{table}[!t]
	\centering
	\begin{tabular}{|l|l||l|}
		\hline
		\multicolumn{2}{|c||}{Range of parameters} & Behavior of $k^{\text{Love}}_{\ell mj}\left(\omega\ne0\right)$ \\
		\hline\hline
		\multicolumn{2}{|c||}{\Gape[5pt]{$i\Gamma^{\left(\sigma\right)}_{\pm}\left(\omega\right)-\hat{\ell} \notin \mathbb{Z}$}} & \multicolumn{1}{c|}{Running} \\
		\hline
		\multirow{2}{*}{$\begin{matrix} i\Gamma^{\left(\sigma\right)}_{+}\left(\omega\right)-\hat{\ell} = k\in\mathbb{Z} \\ \text{OR} \\ i\Gamma^{\left(\sigma\right)}_{-}\left(\omega\right)-\hat{\ell} = k\in\mathbb{Z} \end{matrix}$} & \multicolumn{1}{c||}{\Gape[9pt]{$-\ell \le k \le 0$}} & \multicolumn{1}{c|}{Vanishing} \\
		\cline{2-3}
		& \multicolumn{1}{c||}{\Gape[10pt]{$k>0$ OR $k<-\ell$}} & \multicolumn{1}{c|}{Non-running and non-vanishing} \\
		\hline
	\end{tabular}
	\caption{Behavior of $\omega$-dependent near-zone scalar Love numbers as a function of the parameters $\Gamma^{\left(\sigma\right)}_{\pm}\left(\omega\right)$, given in \eqref{eq:GammaPlusMinus}, and the generalized orbital number $\hat{\ell}=\frac{\ell}{2}$ of the perturbation. For generic angular momenta and azimuthal and orbital numbers, the static scalar Love numbers exhibit a classical RG flow. However, there is a discrete series of imaginary-valued $\Gamma^{\left(\sigma\right)}_{\pm}$'s for which the near-zone Love numbers do not run. For the $\Gamma^{\left(\sigma\right)}_{-\sigma}$ branch, these are unphysical, accompanied by conical singularities in the scalar field profile. For the $\Gamma^{\left(\sigma\right)}_{+\sigma}\left(\omega\right)$ branch, the $-\ell\le k<0$ modes acquire the interpretation of Total Transmission Modes.
 }
	\label{tbl:WSLNs}
\end{table}

\subsection{Running Love}
We begin analyzing our above result for the scalar Love numbers by first addressing the divergent behavior
associated with RG running. 
For general non-zero spin parameters $a$ and $b$ and general azimuthal numbers $m$ and $j$ and frequency $\omega$, i.e. for general non-zero $\Gamma^{\left(\sigma\right)}_{\pm\sigma}\left(\omega\right)$, the scalar Love numbers \textit{always} diverge, either as $\cot\pi\hat{\ell}$ for integer $\hat{\ell}$ (even $\ell$) or as $\tan\pi\hat{\ell}$ for half-integer $\hat{\ell}$ (odd $\ell$). More specifically, in the limit $\varepsilon\rightarrow0$ where $2\hat{\ell}=n-\varepsilon$ approaches a whole number $n\in\mathbb{N}$, the response coefficients \eqref{eq:NZSLNs} develop a simple pole due to the diverging $\Gamma(-2\hat{\ell}-1)$. From the residue of the $\Gamma$-function near negative integers, $\Gamma\left(-n+\varepsilon\right) = \frac{\left(-1\right)^{n}}{n!\,\varepsilon} + \mathcal{O}\left(\varepsilon^0\right)$, the developed pole can be worked out to be
\be\label{eq:kRes}
	k_{\ell m j}\left(\omega\right) = -\frac{\left(-1\right)^{n}}{n!\,\varepsilon}\frac{\Gamma\left(\frac{n}{2}+1+i\Gamma^{\left(\sigma\right)}_{+\sigma}\left(\omega\right)\right)\Gamma\left(\frac{n}{2}+1+i\Gamma^{\left(\sigma\right)}_{-\sigma}\right)}{\Gamma\left(-\frac{n}{2}+i\Gamma^{\left(\sigma\right)}_{+\sigma}\left(\omega\right)\right)\Gamma\left(-\frac{n}{2}+i\Gamma^{\left(\sigma\right)}_{-\sigma}\right)\left(n+1\right)!}\left(\frac{\rho_{+}-\rho_{-}}{\rho_{s}}\right)^{n+1} + \mathcal{O}\left(\varepsilon^{0}\right) \,.
\ee

The full solution, however, is regular due to a compensating divergence in the ``source'' part of the scalar field profile. More specifically, as is illustrated in detail in Appendix \ref{sec:ApSchwarzschildLimit}, the source/response split is performed prior to sending the orbital number to take its physical values with the end result
\be\ba\label{eq:SourceReponseSplit}
	R_{\omega\ell m j}\left(\rho\right) &= \bar{\mathcal{E}}_{\ell m j}\left(\omega\right)\,\rho^{\hat{\ell}}\left[Z_{\omega\ell m j}^{\text{source}}\left(\rho\right) + k_{\ell m j}\left(\omega\right)\,\left(\frac{\rho_{s}}{\rho}\right)^{2\hat{\ell}+1} Z_{\omega\ell m j}^{\text{response}}\left(\rho\right)\right] \,, \\
	Z_{\omega\ell m j}^{\text{source}}\left(\rho\right) &= \left(1-\frac{\rho_{+}}{\rho}\right)^{\hat{\ell}} \left(\frac{\rho-\rho_{-}}{\rho-\rho_{+}}\right)^{\mp i\sigma\tilde{Z}_{-}^{\left(\sigma\right)}\left(\omega\right)} \\
	&\times\,{}_2F_1\left(-\hat{\ell}+i\Gamma^{\left(\sigma\right)}_{\mp\sigma}\left(\omega\right),-\hat{\ell}-i\Gamma^{\left(\sigma\right)}_{\pm\sigma}\left(\omega\right);-2\hat{\ell};\frac{\rho_{+}-\rho_{-}}{\rho_{+}-\rho}\right) \,, \\
	Z_{\omega\ell m j}^{\text{response}}\left(\rho\right) &= \left(1-\frac{\rho_{+}}{\rho}\right)^{-\hat{\ell}-1} \left(\frac{\rho-\rho_{-}}{\rho-\rho_{+}}\right)^{\mp i\sigma\tilde{Z}_{-}^{\left(\sigma\right)}\left(\omega\right)} \\
	&\times\,{}_2F_1\left(\hat{\ell}+1+i\Gamma^{\left(\sigma\right)}_{\mp\sigma}\left(\omega\right),\hat{\ell}+1-i\Gamma^{\left(\sigma\right)}_{\pm\sigma}\left(\omega\right);2\hat{\ell}+2;\frac{\rho_{+}-\rho_{-}}{\rho_{+}-\rho}\right) \,.
\ea\ee
In the limit $\varepsilon\rightarrow0$ where $2\hat{\ell}=n-\varepsilon$ approaches a whole number $n\in\mathbb{N}$, $\rho^{\hat{\ell}}Z_{\omega\ell m j}^{\text{source}}$ also develops a simple pole. The diverging component of the ``source'' part of the solution can be obtained from the following residue formula for the hypergeometric function
\be\ba
	{}_2F_1\left(a,b;-n+\varepsilon;z\right) &= \Gamma\left(-n+\varepsilon\right)\frac{\Gamma\left(a+n+1\right)\Gamma\left(b+n+1\right)}{\Gamma\left(a\right)\Gamma\left(b\right)\left(n+1\right)!} \\
	&\times z^{n+1}{}_2F_1\left(a+n+1,b+n+1;n+2;z\right) + \mathcal{O}\left(\varepsilon^0\right) \,,
\ea\ee
and, consequently,
\be\ba
	Z_{\omega\ell m j}^{\text{source}} &= -\frac{1}{n!\,\varepsilon}\frac{\Gamma\left(\frac{n}{2}+1+i\Gamma^{\left(\sigma\right)}_{+\sigma}\left(\omega\right)\right)\Gamma\left(\frac{n}{2}+1-i\Gamma^{\left(\sigma\right)}_{-\sigma}\right)}{\Gamma\left(-\frac{n}{2}+i\Gamma^{\left(\sigma\right)}_{+\sigma}\left(\omega\right)\right)\Gamma\left(-\frac{n}{2}-i\Gamma^{\left(\sigma\right)}_{-\sigma}\right)\left(n+1\right)!}\left(\frac{\rho_{+}-\rho_{-}}{\rho_{s}}\right)^{n+1} \\
	&\times \,\left(\frac{\rho_{s}}{\rho}\right)^{n+1}Z_{\omega\ell m j}^{\text{response}} + \mathcal{O}\left(\varepsilon^0\right) \,,
\ea\ee
which exactly cancels with the divergence in the scalar Love numbers whenever $n$ is an integer.

From the EFT point of view, this diverging behavior of the scalar Love numbers is interpreted as a classical RG flow. More specifically, power counting arguments reveal that whenever $2\hat{\ell}\in\mathbb{N}$, the Wilson coefficients defining the Love numbers get renormalized from an overlapping with the ``source'' part of the $1$-point function, namely, from the following type of diagrams~\cite{Kol:2011vg,Charalambous:2022rre,Ivanov:2022hlo}
\be
	\Phi \supset \vcenter{\hbox{\begin{tikzpicture}
			\begin{feynman}
				\vertex[dot] (a0);
				\vertex[below=1cm of a0] (p1);
				\vertex[above=1cm of a0] (p2);
				\vertex[right=0.4cm of a0, blob] (gblob){};							
				\vertex[right=1.5cm of p1] (b1);
				\vertex[right=1.5cm of p2] (b2);
				\vertex[right=1.19cm of p2] (b22){$\times$};
				\vertex[above=0.7cm of a0] (g1);
				\vertex[above=0.4cm of a0] (g2);
				\vertex[right=0.05cm of a0] (gdtos){$\vdots$};
				\vertex[below=0.7cm of a0] (gN);
				\vertex[left=0.2cm of gN] (gN1);
				\vertex[left=0.2cm of g1] (g11);
				\diagram*{
					(p1) -- [double,double distance=0.5ex] (p2),
					(g1) -- [photon] (gblob),
					(g2) -- [photon] (gblob),
					(gN) -- [photon] (gblob),
					(b1) -- (gblob) -- (b2),
				};
				\draw [decoration={brace}, decorate] (gN1.south west) -- (g11.north west)
				node [pos=0.55, left] {\(2\hat{\ell}+1\)};
			\end{feynman}
	\end{tikzpicture}}} \,.
\ee
From the theory of differential equations point of view, whenever $2\hat{\ell}\in\mathbb{N}$, the series solutions of the radial differential equation fall into the degenerate case where the characteristic exponents near $x\rightarrow\infty$ differ by an integer number and, thus, according to Fuchs' theorem, only one of the independent solutions can be written as a Frobenius series around there, while the second independent solution will unavoidably contain logarithms. More explicitly, the solution regular at the future event horizon is still given by \eqref{eq:NZRadialSolution}, but its analytic continuation at large distances must be taken as a limiting case with the end result being
\be\ba
	{}&R_{\omega\ell m j} = \bar{\mathcal{E}}_{\ell m j}\left(\omega\right) \rho_{s}^{\hat{\ell}}\left(\frac{\rho-\rho_{-}}{\rho-\rho_{+}}\right)^{i\tilde{Z}_{-}^{\left(\sigma\right)}\left(\omega\right)} \\
	&\times\bigg\{ \left(\frac{\rho-\rho_{+}}{\rho_{s}}\right)^{\hat{\ell}}\sum_{k=0}^{\ell}\frac{\left(-\hat{\ell}+i\Gamma^{\left(\sigma\right)}_{+\sigma}\left(\omega\right)\right)_{k}}{\left(-k+\hat{\ell}+1+i\Gamma^{\left(\sigma\right)}_{-\sigma}\right)_{k}}\frac{\left(\ell-k\right)!}{\ell!\,k!}\left(-x\right)^{-k} \\
	&\,\,\,+\underset{\varepsilon\rightarrow0}{\text{Res}}\left\{k_{\ell-\varepsilon, m j}\left(\omega\right)\right\} \left(\frac{\rho-\rho_{+}}{\rho_{s}}\right)^{-\hat{\ell}-1}\sum_{k=0}^{\infty}\frac{\left(\hat{\ell}+1+i\Gamma^{\left(\sigma\right)}_{+\sigma}\left(\omega\right)\right)_{k}}{\left(-k-\hat{\ell}+i\Gamma^{\left(\sigma\right)}_{-\sigma}\right)_{k}}\frac{\left(\ell+1\right)!}{\left(k+\ell+1\right)!\,k!}\left(+x\right)^{-k} \\
	{}&\,\,\,\times\bigg[\log x + \psi\left(k+1\right) + \psi\left(k+\ell+2\right) - \psi\left(k+\hat{\ell}+1+i\Gamma^{\left(\sigma\right)}_{+\sigma}\left(\omega\right)\right) - \psi\left(-k-\hat{\ell}+i\Gamma^{\left(\sigma\right)}_{-\sigma}\right)\bigg] \bigg\} \,,
\ea\ee
where we have identified the coefficient multiplying the second term as the residue \eqref{eq:kRes} of the response coefficients as $\ell$ approaches a whole number. From the EFT point of view, this residue is precisely the $\beta$-function dictating the classical RG flow of the Love numbers~\cite{Kol:2011vg,Ivanov:2022hlo,Charalambous:2022rre},
\be
	L\frac{d k_{\ell mj}}{dL} = -\frac{\left(-1\right)^{\ell}}{\ell!}\frac{\Gamma\left(\frac{\ell}{2}+1+i\Gamma^{\left(\sigma\right)}_{+\sigma}\left(\omega\right)\right)\Gamma\left(\frac{\ell}{2}+1+i\Gamma^{\left(\sigma\right)}_{-\sigma}\right)}{\Gamma\left(-\frac{\ell}{2}+i\Gamma^{\left(\sigma\right)}_{+\sigma}\left(\omega\right)\right)\Gamma\left(-\frac{\ell}{2}+i\Gamma^{\left(\sigma\right)}_{-\sigma}\right)\left(\ell+1\right)!}\left(\frac{\rho_{+}-\rho_{-}}{\rho_{s}}\right)^{\ell+1} \,.
\ee
This $\beta$-function is evidently real and is therefore entirely associated with the running of the Love numbers, while the dissipative response exhibits no RG flow.

\subsection{Vanishing static Love}
We now turn to the possible resonant conditions for which the $\beta$-function associated with the near-zone Love numbers is zero. Let us start with the case of static scalar Love numbers, which is after all the only regime within which the near-zone approximation is accurate for generic spin parameters. The static scalar Love numbers vanish only if $\hat{\ell}\in\mathbb{N}$ ($\ell$ is an even integer) and
\be\label{eq:VanishingLove_IntL_Static}
	\begin{gathered}
		\Gamma^{\left(\sigma\right)}_{+\sigma}\left(\omega=0\right) = 0 \Rightarrow \frac{j+\sigma m}{2}\left(\Omega_{\psi}+\sigma\Omega_{\phi}\right) = 0 \,, \\
		\text{OR} \\
		\Gamma^{\left(\sigma\right)}_{-\sigma} = 0 \Rightarrow \frac{j-\sigma m}{2}\left(\Omega_{\psi}-\sigma\Omega_{\phi}\right) = 0 \,.
	\end{gathered}
\ee
In other words, these conditions 
are satisfied if either $\left|a\right|=\left|b\right|$ or $\left|m\right|=\left|j\right|$. The first case describes an equi-rotating Myers-Perry black hole in $5$ spacetime dimensions, which has the property of being equipped with an enhanced isometry group $U\left(1\right) \times U\left(1\right) \rightarrow U\left(2\right)$, while the second case can be regarded as a higher-dimensional generalization of ``axisymmetric'' perturbations~\cite{Pani:2015hfa,Gurlebeck:2015xpa,LeTiec:2020bos} that actually includes
here non-axisymmetric cases ($m,j\ne0$). This situation is also similar to the higher-dimensional Schwarzschild black holes~\cite{Kol:2011vg}: for integer $\hat{\ell}$, the static scalar Love numbers vanish, while for half-integer $\hat{\ell}$ they are non-zero and exhibit a classical RG flow discussed above. The corresponding static scalar field profile for $\hat{\ell}\in\mathbb{N}$ is given by
\be\ba\label{eq:StaticRadialVanishingLove}
	R_{\omega=0,\ell m j}\bigg|_{\Gamma^{\left(\sigma\right)}_{\pm\sigma}=0} &= \bar{R}_{\ell m j}\left(\omega=0\right)\,\left(\frac{x}{1+x}\right)^{i\Gamma^{\left(\sigma\right)}_{\mp\sigma}/2} \,{}_2F_1\left(\hat{\ell}+1,-\hat{\ell};1+i\Gamma^{\left(\sigma\right)}_{\mp\sigma};-x\right) \\
	&= \bar{R}_{\ell m j}\left(\omega=0\right)\,\left(\frac{x}{1+x}\right)^{i\Gamma^{\left(\sigma\right)}_{\mp\sigma}/2} \sum_{n=0}^{\hat{\ell}} \frac{\left(\hat{\ell}+n\right)!}{\left(\hat{\ell}-n\right)!} \frac{1}{\left(1+i\Gamma^{\left(\sigma\right)}_{\mp\sigma}\right)_{n}}\frac{x^{n}}{n!} \,,
\ea\ee
where we have used the polynomial form of the hypergeometric function whose one of first two parameters is a negative integer and $\left(a\right)_{n}$ is the Pochhammer symbol.

The vanishing of static Love numbers raises naturalness concerns from the point of view of the worldline EFT~\cite{tHooft:1979rat,Porto:2016zng}. In the absence of any selection rules imposed by an enhanced symmetry structure, the Love numbers are expected to be $\mathcal{O}\left(1\right)$ numbers and exhibit running. In contrast, we find here situations where the static Love numbers vanish at all scales and call upon a symmetry explanation to be presented in the next section. Related to this, for generic real values of the orbital number, the ``source'' and ``response'' parts are given by two infinite series in inverse powers of $\rho$, see Appendix \ref{sec:ApSchwarzschildLimit}. These two series overlap when $\ell$ takes its physical integer values. When the static Love numbers vanish, the final result after summing these two series is the quasi-polynomial form shown above. However, the ``source'' and ``response'' parts are still given by two infinite series which now conspire to give the quasi-polynomial radial solution. In particular, it is these infinite cancellations resulting in the quasi-polynomial form that need to be addressed by a symmetry argument.

\subsection{Non-running Love}
It is also interesting to investigate other situations where $k_{\ell mj}^{\text{Love}}\left(\omega\right)$ 
are fine tuned. 
For even $\ell$ (integer $\hat{\ell}$), we find that the scalar Love numbers \eqref{eq:NZSLNs} exhibit no RG flow under the conditions
\be\label{eq:VanishingLove_IntL}
	\begin{gathered}
		i\Gamma^{\left(\sigma\right)}_{+\sigma}\left(\omega\right) \in \mathbb{Z} \Rightarrow \omega_{k} = \frac{j+\sigma m}{2}\left(\Omega_{\psi}+\sigma\Omega_{\phi}\right) + \frac{i}{\beta}k \,, \\
		\text{OR} \\
		i\Gamma^{\left(\sigma\right)}_{-\sigma} \in \mathbb{Z} \Rightarrow \frac{j-\sigma m}{2}\left(\Omega_{\psi}-\sigma\Omega_{\phi}\right) = -\frac{i}{\beta}k \,,
	\end{gathered}
\ee
where $k\in\mathbb{Z}$ and we have introduced the inverse Hawking temperature of the $5$-dimensional Myers-Perry black hole,
\be\label{eq:betaMP}
	\beta = \frac{1}{2\pi T_{H}} = \frac{\rho_{s}r_{+}}{\rho_{+}-\rho_{-}} \,.
\ee
In particular, for $-\hat{\ell}\le k\le\hat{\ell}$ the scalar Love numbers vanish identically, while for $k\ge\hat{\ell}+1$ or $k\le -\hat{\ell}-1$, they are non-zero but still exhibit no running.

Similarly, for odd $\ell$ (half-integer $\hat{\ell}$), the conditions read
\be\label{eq:VanishingLove_HIntL}
	\begin{gathered}
		i\Gamma^{\left(\sigma\right)}_{+\sigma}\left(\omega\right) \in \mathbb{Z} + \frac{1}{2} \Rightarrow \omega_{k} = \frac{j+\sigma m}{2}\left(\Omega_{\psi}+\sigma\Omega_{\phi}\right) + \frac{i}{\beta}\left(k+\frac{1}{2}\right) \,, \\
		\text{OR} \\
		i\Gamma^{\left(\sigma\right)}_{-\sigma} \in \mathbb{Z} + \frac{1}{2} \Rightarrow \frac{j-\sigma m}{2}\left(\Omega_{\psi}-\sigma\Omega_{\phi}\right) = -\frac{i}{\beta}\left(k+\frac{1}{2}\right) \,,
	\end{gathered}
\ee
with vanishing Love numbers whenever $-\hat{\ell}\le k+\frac{1}{2}\le \hat{\ell}$.

Regarding the conditions on $\Gamma^{\left(\sigma\right)}_{-\sigma}$, these are in general accompanied by conical deficits in the scalar field profile because they imply imaginary azimuthal numbers which break the periodicity of the scalar field with respect to azimuthal rotations. The only situation where this does not happen is when $\Gamma^{\left(\sigma\right)}_{-\sigma}=0$ for $\hat{\ell}\in\mathbb{N}$.

Despite their unphysical
nature in certain cases, the vanishing/non-running of scalar Love numbers beyond the static limit and for all the situations in \eqref{eq:VanishingLove_IntL}-\eqref{eq:VanishingLove_HIntL} is still an interesting result. The corresponding near-zone radial wavefunction for integer $\hat{\ell}$ takes the form
\be\ba\label{eq:NZRadialVanishingLove}
	R_{\omega\ell m j}\bigg|_{i\Gamma^{\left(\sigma\right)}_{\pm\sigma}\left(\omega\right)=k} &= \bar{R}_{\ell m j}\left(\omega\right)\,\left(\frac{x}{1+x}\right)^{i\Gamma^{\left(\sigma\right)}_{\mp\sigma}\left(\omega\right)} \\
	&\left[x\left(1+x\right)\right]^{k/2}\,{}_2F_1\left(\hat{\ell}+1+k,-\hat{\ell}+k;1+k+i\Gamma^{\left(\sigma\right)}_{\mp\sigma}\left(\omega\right);-x\right) \,.
\ea\ee
For $k\le\hat{\ell}$, this again takes a particular quasi-polynomial form to be addressed in the next section via symmetry arguments as well. As we will see, it is the highest-weight property that dictates this quasi-polynomial form. Nevertheless, for $k<-\hat{\ell}$ the polynomial starts developing $\rho^{-\hat{\ell}-1}$ terms. The absence of logarithms indicates that this is not due to an overlapping with PN corrections to the Newtonian source and therefore these are interpreted as non-vanishing and non-running Love numbers.

As we will see in the next section, the highest-weight representation relevant for the properties of the near-zone Love numbers is actually an indecomposable $\SL$ representation of type ``$\circ[\circ[\circ$''. The states ``sandwiched'' between the two highest-weight modes are the ones for which the near-zone Love numbers vanish, while the states spanning the irreducible (lower) highest-weight representation have non-vanishing and non-running near-zone Love numbers. As for the vectors spanning the upper ladder above the reducible (higher) highest-weight representation, these will be spanned by the states characterized by the resonant conditions with $k\ge\hat{\ell}+1$. For half-integer $\hat{\ell}$ for which $i\Gamma^{\left(\sigma\right)}_{+\sigma}\left(\omega\right)$ or $i\Gamma^{\left(\sigma\right)}_{-\sigma}$ is a half-integer, the same conclusions are drawn after replacing $k\rightarrow k+\frac{1}{2}$.

%\subsection{Vanishing Love}
%We begin analyzing our above result for the scalar Love numbers by investigating under what conditions they vanish. The divergent behavior will be examined more carefully shortly.
%
%Let us start with the case of static scalar Love numbers, which is after all the only regime within which the near-zone approximation is accurate for generic spin parameters. The static scalar Love numbers vanish only if  $\hat{\ell}\in\mathbb{N}$ ($\ell$ is an even integer) and
%\be\label{eq:VanishingLove_IntL_Static}
%	\begin{gathered}
%		\Gamma^{\left(\sigma\right)}_{+\sigma}\left(\omega=0\right) = 0 \Rightarrow \frac{j+\sigma m}{2}\left(\Omega_{\psi}+\sigma\Omega_{\phi}\right) = 0 \,, \\
%		\text{OR} \\
%		\Gamma^{\left(\sigma\right)}_{-\sigma} = 0 \Rightarrow \frac{j-\sigma m}{2}\left(\Omega_{\psi}-\sigma\Omega_{\phi}\right) = 0 \,.
%	\end{gathered}
%\ee
%In other words, either $\left|a\right|=\left|b\right|$ or $\left|m\right|=\left|j\right|$. The first case describes an equi-rotating Myers-Perry black hole, which has the property of being equipped with an enhanced isometry group $U\left(1\right) \times U\left(1\right) \rightarrow U\left(2\right)$, while the second case can be regarded as the higher-dimensional analogue of ``axisymmetric'' perturbations~\cite{Pani:2015hfa,Gurlebeck:2015xpa,LeTiec:2020bos} that also includes non-axisymmetric ($m,j\ne0$) perturbations. This situation has a similar behavior as with the higher-dimensional Schwarzschild black hole~\cite{Kol:2011vg}, that is, for integer $\hat{\ell}$, the static scalar Love numbers vanish, while for half-integer $\hat{\ell}$ they are non-zero and exhibit a classical RG flow to be addressed later. The corresponding static scalar field profile for $\hat{\ell}\in\mathbb{N}$ is given by
%\be\ba\label{eq:StaticRadialVanishingLove}
%	R_{\omega=0,\ell m j}\bigg|_{\Gamma^{\left(\sigma\right)}_{\pm\sigma}=0} &= \bar{R}_{\ell m j}\left(\omega=0\right)\,\left(\frac{x}{1+x}\right)^{i\Gamma^{\left(\sigma\right)}_{\mp\sigma}/2} \,{}_2F_1\left(\hat{\ell}+1,-\hat{\ell};1+i\Gamma^{\left(\sigma\right)}_{\mp\sigma};-x\right) \\
%	&= \bar{R}_{\ell m j}\left(\omega=0\right)\,\left(\frac{x}{1+x}\right)^{i\Gamma^{\left(\sigma\right)}_{\mp\sigma}/2} \sum_{n=0}^{\hat{\ell}} \frac{\left(\hat{\ell}+n\right)!}{\left(\hat{\ell}-n\right)!} \frac{1}{\left(1+i\Gamma^{\left(\sigma\right)}_{\mp\sigma}\right)_{n}}\frac{x^{n}}{n!} \,,
%\ea\ee
%where we have used the polynomial form of the hypergeometric function whose one of first two parameters is a negative integer and $\left(a\right)_{n}$ is the Pochhammer symbol.

%The vanishing of static Love numbers raises naturalness concerns from the point of view of the worldline EFT~\cite{tHooft:1979rat,Porto:2016zng}. In the absence of any selection rules imposed by an enhanced symmetry structure, the Love numbers are expected to be $\mathcal{O}\left(1\right)$ numbers and exhibit running. In contrast, we find here situations where the Love numbers vanish at all scales and call upon a symmetry explanation to be presented in the next section. Related to this, for generic real values of the orbital number, the ``source'' and ``response'' parts are given by two infinite series in inverse powers of $\rho$, see Appendix \ref{sec:ApSchwarzschildLimit}. These two series overlap when $\ell$ takes its physical integer values. When the static Love numbers vanish, the final result after summing these two series is the quasi-polynomial form shown above. However, the ``source'' and ``response'' parts are still given by two infinite series which now conspire to give the quasi-polynomial radial solution. In particular, it is these infinite cancellations resulting in the quasi-polynomial form that need to be addressed by a symmetry argument.

%It is also interesting to investigate other situations where $k_{\ell mj}^{\text{Love}}\left(\omega\right)$ acquire seemingly fine-tuned values. For even $\ell$ (integer $\hat{\ell}$), we find that the scalar Love numbers \eqref{eq:NZSLNs} exhibit no RG flow under the conditions
%\be\label{eq:VanishingLove_IntL}
%	\begin{gathered}
%		i\Gamma^{\left(\sigma\right)}_{+\sigma}\left(\omega\right) \in \mathbb{Z} \Rightarrow \omega_{k} = \frac{j+\sigma m}{2}\left(\Omega_{\psi}+\sigma\Omega_{\phi}\right) + \frac{i}{\beta}k \,, \\
%		\text{OR} \\
%		i\Gamma^{\left(\sigma\right)}_{-\sigma} \in \mathbb{Z} \Rightarrow \frac{j-\sigma m}{2}\left(\Omega_{\psi}-\sigma\Omega_{\phi}\right) = -\frac{i}{\beta}k \,,
%	\end{gathered}
%\ee
%where $k\in\mathbb{Z}$ and we have introduced the inverse Hawking temperature of the $5$-dimensional Myers-Perry black hole,
%\be\label{eq:betaMP}
%	\beta = \frac{1}{2\pi T_{H}} = \frac{\rho_{s}r_{+}}{\rho_{+}-\rho_{-}} \,.
%\ee
%In particular, for $k\le\hat{\ell}$ the scalar Love numbers vanish identically, while for $k\ge\hat{\ell}+1$, they are non-zero but still exhibit no running.
%
%Similarly, for odd $\ell$ (half-integer $\hat{\ell}$), the conditions read
%\be\label{eq:VanishingLove_HIntL}
%	\begin{gathered}
%		i\Gamma^{\left(\sigma\right)}_{+\sigma}\left(\omega\right) \in \mathbb{Z} + \frac{1}{2} \Rightarrow \omega_{k} = \frac{j+\sigma m}{2}\left(\Omega_{\psi}+\sigma\Omega_{\phi}\right) + \frac{i}{\beta}\left(k+\frac{1}{2}\right) \,, \\
%		\text{OR} \\
%		i\Gamma^{\left(\sigma\right)}_{-\sigma} \in \mathbb{Z} + \frac{1}{2} \Rightarrow \frac{j-\sigma m}{2}\left(\Omega_{\psi}-\sigma\Omega_{\phi}\right) = -\frac{i}{\beta}\left(k+\frac{1}{2}\right) \,,
%	\end{gathered}
%\ee
%with vanishing Love numbers whenever $k\le \hat{\ell}-\frac{1}{2}$.
%
%Regarding the conditions on $\Gamma^{\left(\sigma\right)}_{-\sigma}$, these are in general accompanied by conical singularities in the scalar field profile because they imply imaginary azimuthal numbers which break the periodicity of the scalar field with respect to azimuthal rotations. The only situation where this does not happen is when $\Gamma^{\left(\sigma\right)}_{-\sigma}=0$ for $\hat{\ell}\in\mathbb{N}$.
%
%Despite their unphysical
%\textcolor{blue}{approximate?}
%nature, the vanishing/non-running of scalar Love numbers beyond the static limit and for all the situations in \eqref{eq:VanishingLove_IntL}-\eqref{eq:VanishingLove_HIntL} is still an interesting result. The corresponding near-zone radial wavefunction for integer $\hat{\ell}$ takes the form
%\be\ba\label{eq:NZRadialVanishingLove}
%	R_{\omega\ell m j}\bigg|_{i\Gamma^{\left(\sigma\right)}_{\pm\sigma}\left(\omega\right)=k} &= \bar{R}_{\ell m j}\left(\omega\right)\,\left(\frac{x}{1+x}\right)^{i\Gamma^{\left(\sigma\right)}_{\mp\sigma}\left(\omega\right)} \\
%	&\left[x\left(1+x\right)\right]^{k/2}\,{}_2F_1\left(\hat{\ell}+1+k,-\hat{\ell}+k;1+k+i\Gamma^{\left(\sigma\right)}_{\mp\sigma}\left(\omega\right);-x\right) \,.
%\ea\ee
%For $k\le\hat{\ell}$, this again takes a particular quasi-polynomial form to be addressed in the next section via symmetry arguments as well. As we will see, it is the highest-weight property that dictates this quasi-polynomial form. Even though for $k\ge\hat{\ell}+1$ the near-zone solutions no longer have any seemingly conspiring polynomial pieces, they will still acquire an elegant representation theory interpretation. In particular, they will supplement in furnishing the \text{entire} type ``$\circ[\circ$'' representation, spanning the ladder above the highest-weight state. For half-integer $\hat{\ell}$ for which $i\Gamma^{\left(\sigma\right)}_{+\left(\sigma\right)}\left(\omega\right)$ or $i\Gamma^{\left(\sigma\right)}_{-\left(\sigma\right)}\left(\omega\right)$ is a half-integer, the same conclusions are drawn after replacing $k\rightarrow k+\frac{1}{2}$.
%
%
%\subsection{Diverging Love}
%For general non-zero spin parameters $a$ and $b$ and general azimuthal numbers $m$ and $j$ and frequency $\omega$, i.e. for general non-zero $\Gamma^{\left(\sigma\right)}_{\pm\sigma}\left(\omega\right)$, the scalar Love numbers \textit{always} diverge, either as $\cot\pi\hat{\ell}$ for integer $\hat{\ell}$ (even $\ell$) or as $\tan\pi\hat{\ell}$ for half-integer $\hat{\ell}$ (odd $\ell$). More specifically, in the limit $\varepsilon\rightarrow0$ where $2\hat{\ell}=n-\varepsilon$ approaches a whole number $n\in\mathbb{N}$, the response coefficients \eqref{eq:NZSLNs} develop a simple pole due to the diverging $\Gamma\left(-2\hat{\ell}-1\right)$. From the residue of the $\Gamma$-function near negative integers, $\Gamma\left(-n+\varepsilon\right) = \frac{\left(-1\right)^{n}}{n!\,\varepsilon} + \mathcal{O}\left(\epsilon^0\right)$, the developed pole can be worked out to be
%\be\label{eq:kRes}
%	k_{\ell m j}\left(\omega\right) = -\frac{\left(-1\right)^{n}}{n!\,\varepsilon}\frac{\Gamma\left(\frac{n}{2}+1+i\Gamma^{\left(\sigma\right)}_{+\sigma}\left(\omega\right)\right)\Gamma\left(\frac{n}{2}+1+i\Gamma^{\left(\sigma\right)}_{-\sigma}\right)}{\Gamma\left(-\frac{n}{2}+i\Gamma^{\left(\sigma\right)}_{+\sigma}\left(\omega\right)\right)\Gamma\left(-\frac{n}{2}+i\Gamma^{\left(\sigma\right)}_{-\sigma}\right)\left(n+1\right)!}\left(\frac{\rho_{+}-\rho_{-}}{\rho_{s}}\right)^{n+1} + \mathcal{O}\left(\varepsilon^{0}\right) \,.
%\ee
%
%The full solution, however, is regular due to a compensating divergence in the ``source'' part of the scalar field profile. More specifically, as is illustrated in detail in Appendix \ref{sec:ApSchwarzschildLimit}, the source/response split is performed prior to sending the orbital number to take its physical values with the end result
%\be\ba\label{eq:SourceReponseSplit}
%	R_{\omega\ell m j}\left(\rho\right) &= \bar{\mathcal{E}}_{\ell m j}\left(\omega\right)\,\rho^{\hat{\ell}}\left[Z_{\omega\ell m j}^{\text{source}}\left(\rho\right) + k_{\ell m j}\left(\omega\right)\,\left(\frac{\rho_{s}}{\rho}\right)^{2\hat{\ell}+1} Z_{\omega\ell m j}^{\text{response}}\left(\rho\right)\right] \,, \\
%	Z_{\omega\ell m j}^{\text{source}}\left(\rho\right) &= \left(1-\frac{\rho_{+}}{\rho}\right)^{\hat{\ell}} \left(\frac{\rho-\rho_{-}}{\rho-\rho_{+}}\right)^{\mp i\sigma\tilde{Z}_{-}^{\left(\sigma\right)}\left(\omega\right)} \\
%	&\times\,{}_2F_1\left(-\hat{\ell}+i\Gamma^{\left(\sigma\right)}_{\mp\sigma}\left(\omega\right),-\hat{\ell}-i\Gamma^{\left(\sigma\right)}_{\pm\sigma}\left(\omega\right);-2\hat{\ell};\frac{\rho_{+}-\rho_{-}}{\rho_{+}-\rho}\right) \,, \\
%	Z_{\omega\ell m j}^{\text{response}}\left(\rho\right) &= \left(1-\frac{\rho_{+}}{\rho}\right)^{-\hat{\ell}-1} \left(\frac{\rho-\rho_{-}}{\rho-\rho_{+}}\right)^{\mp i\sigma\tilde{Z}_{-}^{\left(\sigma\right)}\left(\omega\right)} \\
%	&\times\,{}_2F_1\left(\hat{\ell}+1+i\Gamma^{\left(\sigma\right)}_{\mp\sigma}\left(\omega\right),\hat{\ell}+1-i\Gamma^{\left(\sigma\right)}_{\pm\sigma}\left(\omega\right);2\hat{\ell}+2;\frac{\rho_{+}-\rho_{-}}{\rho_{+}-\rho}\right) \,.
%\ea\ee
%In the limit $\varepsilon\rightarrow0$ where $2\hat{\ell}=n-\varepsilon$ approaches a whole number $n\in\mathbb{N}$, $\rho^{\hat{\ell}}Z_{\omega\ell m j}^{\text{source}}$ also develops a simple pole. The diverging component of the ``source'' part of the solution can be obtained from the following residue formula for the hypergeometric function
%\be\ba
%	{}_2F_1\left(a,b;-n+\varepsilon;z\right) &= \Gamma\left(-n+\varepsilon\right)\frac{\Gamma\left(a+n+1\right)\Gamma\left(b+n+1\right)}{\Gamma\left(a\right)\Gamma\left(b\right)\left(n+1\right)!} \\
%	&\times z^{n+1}{}_2F_1\left(a+n+1,b+n+1;n+2;z\right) + \mathcal{O}\left(\varepsilon^0\right) \,,
%\ea\ee
%and, consequently,
%\be\ba
%	Z_{\omega\ell m j}^{\text{source}} &= -\frac{1}{n!\,\varepsilon}\frac{\Gamma\left(\frac{n}{2}+1+i\Gamma^{\left(\sigma\right)}_{+\sigma}\left(\omega\right)\right)\Gamma\left(\frac{n}{2}+1-i\Gamma^{\left(\sigma\right)}_{-\sigma}\right)}{\Gamma\left(-\frac{n}{2}+i\Gamma^{\left(\sigma\right)}_{+\sigma}\left(\omega\right)\right)\Gamma\left(-\frac{n}{2}-i\Gamma^{\left(\sigma\right)}_{-\sigma}\right)\left(n+1\right)!}\left(\frac{\rho_{+}-\rho_{-}}{\rho_{s}}\right)^{n+1} \\
%	&\times \,\left(\frac{\rho_{s}}{\rho}\right)^{n+1}Z_{\omega\ell m j}^{\text{response}} + \mathcal{O}\left(\varepsilon^0\right) \,,
%\ea\ee
%which exactly cancels with the divergence in the scalar Love numbers whenever $n$ is an integer.
%
%From the EFT point of view, this diverging behavior of the scalar Love numbers is interpreted as a classical RG flow. More specifically, when $2\hat{\ell}\in\mathbb{N}$, the Wilson coefficients defining the Love numbers get renormalized from an overlapping with the ``source'' part of the $1$-point function, namely, from the following type of diagrams~\cite{Kol:2011vg,Charalambous:2022rre,Ivanov:2022hlo}
%\be
%	\Phi \supset \vcenter{\hbox{\begin{tikzpicture}
%				\begin{feynman}
%					\vertex[dot] (a0);
%					\vertex[below=1cm of a0] (p1);
%					\vertex[above=1cm of a0] (p2);
%					\vertex[right=0.4cm of a0, blob] (gblob){};							
%					\vertex[right=1.5cm of p1] (b1);
%					\vertex[right=1.5cm of p2] (b2);
%					\vertex[right=1.19cm of p2] (b22){$\times$};
%					\vertex[above=0.7cm of a0] (g1);
%					\vertex[above=0.4cm of a0] (g2);
%					\vertex[right=0.05cm of a0] (gdtos){$\vdots$};
%					\vertex[below=0.7cm of a0] (gN);
%					\vertex[left=0.2cm of gN] (gN1);
%					\vertex[left=0.2cm of g1] (g11);
%					\diagram*{
%						(p1) -- [double,double distance=0.5ex] (p2),
%						(g1) -- [photon] (gblob),
%						(g2) -- [photon] (gblob),
%						(gN) -- [photon] (gblob),
%						(b1) -- (gblob) -- (b2),
%					};
%				\draw [decoration={brace}, decorate] (gN1.south west) -- (g11.north west)
%				node [pos=0.55, left] {\(2\hat{\ell}+1\)};
%				\end{feynman}
%	\end{tikzpicture}}} \,.
%\ee
%From the theory of differential equations point of view, whenever $2\hat{\ell}\in\mathbb{N}$, the series solutions of the radial differential equation fall into the degenerate case where the characteristic exponents near $x\rightarrow\infty$ differ by an integer number and, thus, according to Fuchs' theorem, only one of the independent solutions can be written as a Frobenius series around there, while the second independent solution will unavoidably contain logarithms. More explicitly, the solution regular at the future event horizon is still given by \eqref{eq:NZRadialSolution}, but its analytic continuation at large distances must be taken as a limiting case with the end result being
%\be\ba
%	{}&R_{\omega\ell m j} = \bar{\mathcal{E}}_{\ell m j}\left(\omega\right) \rho_{s}^{\hat{\ell}}\left(\frac{\rho-\rho_{-}}{\rho-\rho_{+}}\right)^{i\tilde{Z}_{-}^{\left(\sigma\right)}\left(\omega\right)} \\
%	&\times\bigg\{ \left(\frac{\rho-\rho_{+}}{\rho_{s}}\right)^{\hat{\ell}}\sum_{k=0}^{\ell}\frac{\left(-\hat{\ell}+i\Gamma^{\left(\sigma\right)}_{+\sigma}\left(\omega\right)\right)_{k}}{\left(-k+\hat{\ell}+1+i\Gamma^{\left(\sigma\right)}_{-\sigma}\right)_{k}}\frac{\left(\ell-k\right)!}{\ell!\,k!}\left(-x\right)^{-k} \\
%	&\,\,\,+\underset{\varepsilon\rightarrow0}{\text{Res}}\left\{k_{\ell-\varepsilon, m j}\left(\omega\right)\right\} \left(\frac{\rho-\rho_{+}}{\rho_{s}}\right)^{-\hat{\ell}-1}\sum_{k=0}^{\infty}\frac{\left(\hat{\ell}+1+i\Gamma^{\left(\sigma\right)}_{+\sigma}\left(\omega\right)\right)_{k}}{\left(-k-\hat{\ell}+i\Gamma^{\left(\sigma\right)}_{-\sigma}\right)_{k}}\frac{\left(\ell+1\right)!}{\left(k+\ell+1\right)!\,k!}\left(+x\right)^{-k} \\
%	{}&\,\,\,\times\bigg[\log x + \psi\left(k+1\right) + \psi\left(k+\ell+2\right) - \psi\left(k+\hat{\ell}+1+i\Gamma^{\left(\sigma\right)}_{+\sigma}\left(\omega\right)\right) - \psi\left(-k-\hat{\ell}+i\Gamma^{\left(\sigma\right)}_{-\sigma}\right)\bigg] \bigg\} \,,
%\ea\ee
%where we have identified the coefficient multiplying the second term as the residue \eqref{eq:kRes} of the response coefficients as $\ell$ approaches an integer number. From the EFT point of view, this residue is precisely the $\beta$-function dictating the classical RG flow of the Love numbers~\cite{Kol:2011vg,Ivanov:2022hlo,Charalambous:2022rre},
%\be
%	L\frac{d k_{\ell mj}}{dL} = -\frac{\left(-1\right)^{\ell}}{\ell!}\frac{\Gamma\left(\frac{\ell}{2}+1+i\Gamma^{\left(\sigma\right)}_{+\sigma}\left(\omega\right)\right)\Gamma\left(\frac{\ell}{2}+1+i\Gamma^{\left(\sigma\right)}_{-\sigma}\right)}{\Gamma\left(-\frac{\ell}{2}+i\Gamma^{\left(\sigma\right)}_{+\sigma}\left(\omega\right)\right)\Gamma\left(-\frac{\ell}{2}+i\Gamma^{\left(\sigma\right)}_{-\sigma}\right)\left(\ell+1\right)!}\left(\frac{\rho_{+}-\rho_{-}}{\rho_{s}}\right)^{\ell+1} \,.
%\ee