\section{Love symmetries}
\label{sec:SL2R}

In~\cite{Charalambous:2021kcz,Charalambous:2022rre}, we have demonstrated the emergence of an enhanced $\SL$ symmetry, dubbed ``Love symmetry'', of the near-zone equations of motion for black hole perturbations. 
In the context of this symmetry, the vanishing of Love numbers appears  
as a constraint imposed 
by 
the highest-weight 
Love symmetry representation 
structure, to which the relevant 
 black hole perturbation solutions belong.
In particular, the highest-weight property dictates a (quasi-)polynomial form of the regular radial wavefunctions which is the behavior indicative of the vanishing of the Love numbers. The intricate structure of the scalar Love numbers for $5$-d Myers-Perry black holes extracted in the previous section sets a good example of examining this hypothesis. In this section, we will demonstrate the existence of $\SL$ structures of the near-zone Klein-Gordon equation for all values of the spin parameters and azimuthal numbers. As we will see, it is only when the scalar Love numbers vanish that the corresponding regular scalar field solution belongs to a highest-weight representation of the corresponding $\SL$ and the vanishing of the scalar Love numbers will be immediately implied through the highest-weight property.

To demonstrate the enhanced symmetry structure of the near-zone equations of motion, it is convenient to introduce the following sum/difference azimuthal angles,
\be
	\psi_{\pm}=\psi\pm\phi \,. 
\ee
The corresponding angular velocities and azimuthal numbers with respect to these two directions are then
\be
	\Omega_{\pm} = \Omega_{\psi} \pm \Omega_{\phi} \,,\quad m_{\pm}= \frac{j\pm m}{2} \,.
\ee

The two near-zone splits \eqref{eq:NZRadial1}-\eqref{eq:NZRadial2} have the property of each being equipped with an $\SL$ structure. Indeed, for each sign $\sigma=\pm$, we can find a set of three vector fields satisfying the $\SL$ algebra,
\be
	\left[L_{m}^{\left(\sigma\right)},L_{n}^{\left(\sigma\right)}\right] = \left(m-n\right)L_{m+n}^{\left(\sigma\right)} \,,\quad m,n=0,\pm1 \,.
\ee
These generators are given by the vector fields
\be\label{eq:LoveGen}
	\begin{gathered}
		L_0^{\left(\sigma\right)} = -\beta\left(\partial_{t} + \Omega_{\sigma}\,\partial_{\sigma}\right) \,, \\
		L_{\pm1}^{\left(\sigma\right)} = e^{\pm t/\beta}\left[\mp\sqrt{\Delta}\,\partial_{\rho} + \partial_{\rho}\left(\sqrt{\Delta}\right)\,\beta\left(\partial_{t} + \Omega_{\sigma}\,\partial_{\sigma}\right) + \frac{\rho_{+}-\rho_{-}}{2\sqrt{\Delta}} \beta\Omega_{-\sigma}\,\partial_{-\sigma}\right] \,,
	\end{gathered}
\ee
with $\partial_{\pm}\equiv\partial_{\psi_{\pm}}=\left(\partial_{\psi}\pm\partial_{\phi}\right)/2$ and $\beta$ the inverse Hawking temperature \eqref{eq:betaMP} of the $5$-d Myers-Perry black hole. The corresponding Casimirs,
\be\ba\label{eq:LoveCasimir}
	{}&\mathcal{C}_2^{\left(\sigma\right)} = L_0^{\left(\sigma\right)2} - \frac{1}{2}\left(L_{+1}^{\left(\sigma\right)}L_{-1}^{\left(\sigma\right)}+L_{-1}^{\left(\sigma\right)}L_{+1}^{\left(\sigma\right)}\right) \\
	&= \partial_{\rho}\,\Delta\,\partial_{\rho} - \frac{\rho_{s}^2\rho_{+}}{4\Delta}\left(\partial_{t}+\Omega_{+}\,\partial_{+}+\Omega_{-}\,\partial_{-}\right)^2 - \frac{\rho_{+}-\rho_{-}}{\rho-\rho_{-}}\beta^2\,\left(\partial_{t}+\Omega_{\sigma}\,\partial_{\sigma}\right)\Omega_{-\sigma}\,\,\partial_{-\sigma} \,,
\ea\ee
are precisely equal to the two near-zone truncations of the radial operator \eqref{eq:NZRadial1}-\eqref{eq:NZRadial2} when working in the initial Boyer-Lindquist azimuthal coordinates $\left(\phi,\psi\right)$. 
We will denote the individual algebras generated for each sign $\sigma$ as $\SL_{\left(\sigma\right)}$. A crucial property of these generators is that they are regular at both the future and the past event horizons as can be seen by switching to the advanced and retarded null coordinates respectively (see Appendix \ref{sec:ApSL2RGenerators}).

Solutions of the near-zone equations of motion then form representations of the corresponding $\SL_{\left(\sigma\right)}$ symmetry, labeled by their Casimir eigenvalues, which are equal to the angular eigenvalues $\hat{\ell}(\hat{\ell}+1)$, and the $L_0$-weights~\cite{Miller1968,Howe1992},
\be
	\mathcal{C}_2^{\left(\sigma\right)}\Phi_{\omega\ell mj} = \hat{\ell}(\hat{\ell}+1)\Phi_{\omega\ell mj} \,,\quad L_0^{\left(\sigma\right)}\Phi_{\omega\ell mj} = h^{\left(\sigma\right)}\Phi_{\omega\ell mj} \,.
\ee
We remark, in particular, that
\be
	h^{\left(\sigma\right)}=i\beta\left(\omega-m_{\sigma}\Omega_{\sigma}\right) = -i\Gamma^{\left(\sigma\right)}_{+\sigma}\left(\omega\right) \,.
\ee

\subsection{Highest-weight banishes static Love}
We saw in the previous section that the static scalar Love numbers vanish whenever $m_{\sigma}\Omega_{\sigma}=0$ for $\sigma=+$ or $\sigma=-$ and only for integer $\hat{\ell}$ (see Eq. \eqref{eq:VanishingLove_IntL_Static}). We will show here that the vanishing of the static Love numbers follows from the fact that the relevant solution of the near-zone equations of motion belongs to a highest-weight representation of $\SL_{\left(\sigma\right)}$.

Let us construct the highest-weight representation of $\SL_{\left(\sigma\right)}$ with highest-weight $h^{\left(\sigma\right)}_{-\hat{\ell},0}=-\hat{\ell}$~\cite{Miller1968,Howe1992,Charalambous:2021kcz,Charalambous:2022rre}. The primary state $\upsilon_{-\hat{\ell},0}^{\left(\sigma\right)}$ satisfies,
\be
	L_0^{\left(\sigma\right)}\upsilon_{-\hat{\ell},0}^{\left(\sigma\right)}=-\hat{\ell}\,\upsilon_{-\hat{\ell},0}^{\left(\sigma\right)} \,,\quad L_{+1}^{\left(\sigma\right)}\upsilon_{-\hat{\ell},0}^{\left(\sigma\right)} = 0 \,.
\ee
Supplementing with the condition of definite azimuthal numbers,
\be
	J_0^{\left(\pm\right)}\upsilon_{-\hat{\ell},0}^{\left(\sigma\right)} = m_{\pm }\upsilon_{-\hat{\ell},0}^{\left(\sigma\right)} \,,
\ee
where $J_0^{\left(\pm\right)}=-i\partial_{\pm}$ are the two $\mathfrak{so}\left(3\right)$ $J_0$-generators of the rotation group algebra $\mathfrak{so}\left(4\right)\simeq \mathfrak{so}\left(3\right)\oplus\mathfrak{so}\left(3\right)$ (see Appendix \ref{sec:ApSphericalHarmonics}), we find, up to an overall normalization constant,
\be\label{eq:SL2RHighestL}
	\upsilon_{-\hat{\ell},0}^{\left(\sigma\right)} = \mathcal{F}_{\sigma}\left(\rho\right) e^{im_{\sigma}\left(\psi_{\sigma}-\Omega_{\sigma}t\right)}e^{im_{-\sigma}\psi_{-\sigma}} \left[e^{t/\beta}\sqrt{\Delta}\right]^{\hat{\ell}} \,,
\ee
with the form factor given by
\be
	\mathcal{F}_{\sigma}\left(\rho\right) \equiv \left(\frac{\rho-\rho_{+}}{\rho-\rho_{-}}\right)^{im_{-\sigma}\beta\Omega_{-\sigma}/2} = \left(\frac{\rho-\rho_{+}}{\rho-\rho_{-}}\right)^{i\Gamma^{\left(\sigma\right)}_{-\sigma}/2} \,.
\ee
This highest-weight vector is regular at the future event horizon but singular at the past event horizon. From the regularity of the generators, all the descendants will also be regular at the future event horizon and singular at the past one. For generic parameters, the highest-weight representation is an infinite-dimensional Verma module and is spanned by the vectors~\cite{Howe1992,Miller1968,Miller1970}
\be
	\upsilon_{-\hat{\ell},n}^{\left(\sigma\right)} = \left[L_{-1}^{\left(\sigma\right)}\right]^{n}\upsilon_{-\hat{\ell},0}^{\left(\sigma\right)} \,,\quad n\ge0\,,
\ee
whose charge under $L_0^{\left(\sigma\right)}$ is
\be
	h_{-\hat{\ell},n}^{\left(\sigma\right)} = n-\hat{\ell} \,.
\ee

Let us compare with the properties of the regular static solution with $m_{\sigma}\Omega_{\sigma}=0$ and $\hat{\ell}\in\mathbb{N}$ which corresponds to vanishing static Love numbers. This is a null state of $L_0^{\left(\sigma\right)}$ regular at the future event horizon and is therefore identified as the $n=\hat{\ell}$ descendant in the highest-weight multiplet,
\be
	\Phi_{\omega=0,\ell mj}\bigg|_{m_{\sigma}\Omega_{\sigma}=0} \propto \upsilon_{-\hat{\ell},\hat{\ell}}^{\left(\sigma\right)} = \left[L_{-1}^{\left(\sigma\right)}\right]^{\hat{\ell}}\upsilon_{-\hat{\ell},0}^{\left(\sigma\right)} \,.
\ee
Noticing that, for generic $\upsilon\left(\rho\right)$,
\be\ba\label{eq:Lp1n}
	\left[L_{+1}^{\left(\sigma\right)}\right]^{n}&\left(\mathcal{F}_{\sigma}\left(\rho\right)e^{im_{\sigma}\left(\psi_{\sigma}-\Omega_{\sigma}t\right)}e^{im_{-\sigma}\psi_{-\sigma}}\upsilon\left(\rho\right)\right) = \\
	&\mathcal{F}_{\sigma}\left(\rho\right)e^{im_{\sigma}\left(\psi_{\sigma}-\Omega_{\sigma}t\right)}e^{im_{-\sigma}\psi_{-\sigma}}\left[-e^{t/\beta}\sqrt{\Delta}\right]^{n}\frac{d^{n}}{d\rho^{n}}\upsilon\left(\rho\right) \,,
\ea\ee
we see that the highest-property,
\be
	\left[L_{+1}^{\left(\sigma\right)}\right]^{\hat{\ell}+1}\Phi_{\omega=0,\ell mj}\bigg|_{m_{\sigma}\Omega_{\sigma}=0} = 0 \,,
\ee
dictates a quasi-polynomial form of the near-zone radial wavefunction,
\be
	R_{\omega=0,\ell mj}\bigg|_{m_{\sigma}\Omega_{\sigma}=0} = \mathcal{F}_{\sigma}\left(\rho\right) \sum_{n=0}^{\hat{\ell}}c_{n}\rho^{n} \,.
\ee
This is precisely the quasi-polynomial form \eqref{eq:StaticRadialVanishingLove} that we wanted to address. In conclusion, we have seen how the vanishing of static scalar Love numbers of the $5$-dimensional Myers-Perry black hole is automatically outputted as a selection rule following from the fact that corresponding solution belongs to a highest-weight representation of the near-zone $\SL_{\left(\sigma\right)}$. On the opposite route, the conditions for the regular at the future event horizon static solution to belong to a highest-weight representation are that $\hat{\ell}\in\mathbb{N}$ and that $m_{\sigma}\Omega_{\sigma}=0$, which are precisely the conditions of vanishing static scalar Love numbers.

Let us briefly comment on the structure of the lowest-weight representation of $\SL_{\left(\sigma\right)}$, spanned by ascendants $\bar{\upsilon}_{+\hat{\ell},n}^{\left(\sigma\right)}$ of a lowest-weight state $\bar{\upsilon}_{+\hat{\ell},0}^{\left(\sigma\right)}$,
\be
	\bar{\upsilon}_{+\hat{\ell},n}^{\left(\sigma\right)} = \left[-L_{+1}^{\left(\sigma\right)}\right]^{n}\bar{\upsilon}_{+\hat{\ell},0}^{\left(\sigma\right)} \,,
\ee
satisfying
\be
	L_0^{\left(\sigma\right)}\bar{\upsilon}_{+\hat{\ell},0}^{\left(\sigma\right)}=+\hat{\ell}\,\bar{\upsilon}_{+\hat{\ell},0}^{\left(\sigma\right)} \,,\quad L_{-1}^{\left(\sigma\right)}\bar{\upsilon}_{+\hat{\ell},0}^{\left(\sigma\right)} = 0 \,,
\ee
and having definite azimuthal numbers. We find
\be\label{eq:SL2RLowestL}
	\bar{\upsilon}_{+\hat{\ell},0}^{\left(\sigma\right)} = \bar{\mathcal{F}}_{\sigma}\left(\rho\right) e^{im_{\sigma}\left(\psi_{\sigma}-\Omega_{\sigma}t\right)}e^{im_{-\sigma}\psi_{-\sigma}} \left[e^{-t/\beta}\sqrt{\Delta}\right]^{\hat{\ell}} \,,\quad \bar{\mathcal{F}}_{\sigma}\left(\rho\right) \equiv \left(\frac{\rho-\rho_{+}}{\rho-\rho_{-}}\right)^{-i\Gamma^{\left(\sigma\right)}_{+\sigma}/2} \,.
\ee
This state is a solution of the near-zone equations of motion that is regular at the past event horizon, but singular at the future event horizon. As a result, from the regularity of the generators, all the ascendants will also be regular at the past event horizon and singular at the future one, with $L_0$-charges $\bar{h}^{\left(\sigma\right)}_{\hat{\ell},n}=\hat{\ell}-n$. In particular, when $\hat{\ell}$ is an integer, the $n=\hat{\ell}$ ascendant $\bar{\upsilon}_{+\hat{\ell},\hat{\ell}}^{\left(\sigma\right)}$ will be the static solution of the Klein-Gordon equation with $m_{\sigma}\Omega_{\sigma}=0$ that is singular at the future event horizon. We have just demonstrated that the static solutions regular and singular at the future event horizon belong to different, locally distinguishable, representations of $\SL_{\left(\sigma\right)}$; the highest-weight representation and lowest-weight representation respectively. This is the algebraic manifestation of the absence of RG flow for the static scalar Love numbers in the particular case of $\hat{\ell}\in\mathbb{N}$ and $m_{\sigma}\Omega_{\sigma}=0$. The construction of these highest-weight and lowest-weight representations is demonstrated graphically in Figure \ref{fig:SL2R_HW_LW}.


\begin{figure}
	\centering
	\begin{subfigure}[b]{0.49\textwidth}
		\centering
		\begin{tikzpicture}
			\node at (0,1) (uml3) {$\upsilon^{\left(\sigma\right)}_{-\hat{\ell},\hat{\ell}}$};
			\node at (0,2) (uml2) {$\upsilon^{\left(\sigma\right)}_{-\hat{\ell},2}$};
			\node at (0,3) (uml1) {$\upsilon^{\left(\sigma\right)}_{-\hat{\ell},1}$};
			\node at (0,4) (uml0) {$\upsilon^{\left(\sigma\right)}_{-\hat{\ell},0}$};
			
			\draw (1,1) -- (5,1);
			\node at (3,1.5) (up) {$\vdots$};
			\node at (3,0.5) (um) {$\vdots$};
			\draw (1,2) -- (5,2);
			\draw (1,3) -- (5,3);
			\draw (1,4) -- (5,4);
			\draw [snake=zigzag] (1,4.1) -- (5,4.1);
			
			\draw[red] [<-] (2.5,2) -- node[left] {$L_{-1}^{\left(\sigma\right)}$} (2.5,3);
			\draw[red] [<-] (2,3) -- node[left] {$L_{-1}^{\left(\sigma\right)}$} (2,4);
			\draw[blue] [->] (4,3) -- node[right] {$L_{+1}^{\left(\sigma\right)}$} (4,4);
			\draw[blue] [->] (3.5,2) -- node[right] {$L_{+1}^{\left(\sigma\right)}$} (3.5,3);
		\end{tikzpicture}
		\caption{The highest-weight representation spanned by states $\{\upsilon^{\left(\sigma\right)}_{-\hat{\ell},n}|n\in\mathbb{N}\}$ which are regular (singular) at the future (past) event horizon and have weights $h^{\left(\sigma\right)}_{-\hat{\ell},n}=n-\hat{\ell}$.}
	\end{subfigure}
	\hfill
	\begin{subfigure}[b]{0.49\textwidth}
		\centering
		\begin{tikzpicture}
			\node at (0,3) (upll) {$\bar{\upsilon}^{\left(\sigma\right)}_{+\hat{\ell},\hat{\ell}}$};
			\node at (0,2) (upl2) {$\bar{\upsilon}^{\left(\sigma\right)}_{+\hat{\ell},2}$};
			\node at (0,1) (upl1) {$\bar{\upsilon}^{\left(\sigma\right)}_{+\hat{\ell},1}$};
			\node at (0,0) (upl0) {$\bar{\upsilon}^{\left(\sigma\right)}_{+\hat{\ell},0}$};
			
			\draw [snake=zigzag] (1,-0.1) -- (5,-0.1);
			\draw (1,0) -- (5,0);
			\draw (1,1) -- (5,1);
			\draw (1,2) -- (5,2);
			\node at (3,2.5) (up) {$\vdots$};
			\draw (1,3) -- (5,3);
			\node at (3,3.5) (um) {$\vdots$};
			
			\draw[blue] [->] (2.5,0) -- node[left] {$L_{+1}^{\left(\sigma\right)}$} (2.5,1);
			\draw[blue] [->] (2,1) -- node[left] {$L_{+1}^{\left(\sigma\right)}$} (2,2);
			\draw[red] [<-] (4,1) -- node[right] {$L_{-1}^{\left(\sigma\right)}$} (4,2);
			\draw[red] [<-] (3.5,0) -- node[right] {$L_{-1}^{\left(\sigma\right)}$} (3.5,1);
		\end{tikzpicture}
		\caption{The lowest-weight representation spanned by states $\{\bar{\upsilon}^{\left(\sigma\right)}_{+\hat{\ell},n}|n\in\mathbb{N}\}$ which are singular (regular) at the future (past) event horizon and have weights $\bar{h}^{\left(\sigma\right)}_{+\hat{\ell},n}=\hat{\ell}-n$.}
	\end{subfigure}
	\caption{Infinite-dimensional highest-weight and lowest-weight representations of $\SL_{\left(\sigma\right)}$ containing near-zone solutions for a massless scalar field in the $5$-d Myers-Perry black hole background with multipolar index $\ell$. Whenever $\hat{\ell}=\frac{\ell}{2}\in\mathbb{N}$ and $m_{\sigma}\Omega_{\sigma}=0$, the solution regular (singular) at the future event horizon is the $\hat{\ell}$'th descendant (ascendant) in the highest-weight (lowest-weight) representation.}
	\label{fig:SL2R_HW_LW}
\end{figure}


\subsection{Highest-weight banishes near-zone Love}
Interestingly, the near-zone symmetries can address all the situations in Eq.~\eqref{eq:VanishingLove_IntL} and Eq.~\eqref{eq:VanishingLove_HIntL}, even the unphysical ones.
%Recall that in the near-zone truncations \eqref{eq:NZRadial1}-\eqref{eq:NZRadial2}, scalar Love numbers may vanish whenever $i\Gamma^{\left(\sigma\right)}_{+\sigma}\left(\omega\right)\in\mathbb{Z}$ or $i\Gamma^{\left(\sigma\right)}_{-\sigma}\in\mathbb{Z}$ if $\hat{\ell}$ is an integer, or whenever $i\Gamma^{\left(\sigma\right)}_{+\sigma}\left(\omega\right)\in\mathbb{Z}+\frac{1}{2}$ or $i\Gamma^{\left(\sigma\right)}_{-\sigma}\in\mathbb{Z}+\frac{1}{2}$ if $\hat{\ell}$ is a half-integer (see \eqref{eq:VanishingLove_IntL}-\eqref{eq:VanishingLove_HIntL}).

If $\hat{\ell}$ is an integer and $i\Gamma^{\left(\sigma\right)}_{+\sigma}\left(\omega\right)=k\in\mathbb{Z}$, then the corresponding near-zone solution has the form
\be
	\Phi_{\omega\ell mj}\bigg|_{i\Gamma^{\left(\sigma\right)}_{+\sigma}\left(\omega\right)=k} \propto e^{kt/\beta}e^{im_{\sigma}\left(\psi_{\sigma}-\Omega_{\sigma}t\right)}e^{im_{-\sigma}\psi_{-\sigma}}R_{\omega\ell mj}\left(\rho\right)\bigg|_{i\Gamma^{\left(\sigma\right)}_{+\sigma}\left(\omega\right)=k} \,,
\ee
where we are suppressing the $\theta$-dependence, and therefore satisfies
\be
	L_0^{\left(\sigma\right)}\Phi_{\omega\ell mj}\bigg|_{i\Gamma^{\left(\sigma\right)}_{+\sigma}\left(\omega\right)=k} = -k\,\Phi_{\omega\ell mj}\bigg|_{i\Gamma^{\left(\sigma\right)}_{+\sigma}\left(\omega\right)=k} \,.
\ee
If $k\le\hat{\ell}$, the regular at the future event horizon solution is then recognized to be the $n=\hat{\ell}-k$ descendant,
\be
	\Phi_{\omega\ell mj}\bigg|_{i\Gamma^{\left(\sigma\right)}_{+\sigma}\left(\omega\right)=k\le\hat{\ell}} \propto \upsilon_{-\hat{\ell},\hat{\ell}-k}^{\left(\sigma\right)} = \left[L_{-1}^{\left(\sigma\right)}\right]^{\hat{\ell}-k}\upsilon_{-\hat{\ell},0}^{\left(\sigma\right)} \,.
\ee
The highest-property,
\be
	\left[L_{+1}^{\left(\sigma\right)}\right]^{\hat{\ell}-k+1}\Phi_{\omega\ell mj}\bigg|_{i\Gamma^{\left(\sigma\right)}_{+\sigma}\left(\omega\right)=k\le\hat{\ell}} = 0 \,,
\ee
does not directly imply any useful quasi-polynomial form itself. Rather, it is the fact that
\be
	\upsilon_{-\hat{\ell},\hat{\ell}-k}^{\left(\sigma\right)} =
	\begin{cases}
	    \left(-1\right)^{k}\left[L_{+1}^{\left(\sigma\right)}\right]^{k}\upsilon_{-\hat{\ell},\hat{\ell}}^{\left(\sigma\right)} \quad \text{for }0\le k \le \hat{\ell} \\
	    \left[L_{-1}^{\left(\sigma\right)}\right]^{-k}\upsilon_{-\hat{\ell},\hat{\ell}}^{\left(\sigma\right)} \quad\quad\quad\, \text{for }k<0
    \end{cases}
\ee
and the quasi-polynomial form of the static element $\upsilon_{-\hat{\ell},\hat{\ell}}^{\left(\sigma\right)}$ that give a useful expression for the relevant near-zone solution. From \eqref{eq:Lp1n}, and an analogous relation for the action of $L_{-1}^{\left(\sigma\right)}$ we conclude that
\be
	R_{\omega\ell mj}\left(\rho\right)\bigg|_{i\Gamma^{\left(\sigma\right)}_{+\sigma}\left(\omega\right)=k\le\hat{\ell}} = \mathcal{F}_{\sigma}\left(\rho\right) \Delta^{k/2} \sum_{n=0}^{\hat{\ell}-k}c_{n}\rho^{n} \,,
\ee
which has the exact quasi-polynomial form of \eqref{eq:NZRadialVanishingLove} with $k\le\hat{\ell}$.

Another remark here is there are in general two possible highest-weight representations of $\SL$; one has $h^{\left(\sigma\right)}=-\hat{\ell}$, which is the one we saw, and the other has $h^{\left(\sigma\right)}=+\hat{\ell}+1$. Both highest-weight representations contain solutions that are regular at the future event horizon. Since the weights differ by the integer amount $2\hat{\ell}+1$, we see that the primary state with $h^{\left(\sigma\right)}=+\hat{\ell}+1$ is actually a descendant of the primary state with $h^{\left(\sigma\right)}=-\hat{\ell}$,
\be
    \upsilon_{+\hat{\ell}+1,0} = \left[L_{-1}^{\left(\sigma\right)}\right]^{2\hat{\ell}+1}\upsilon_{-\hat{\ell},0} \,.
\ee
Therefore, the highest-weight Verma module of $\SL_{\left(\sigma\right)}$ we have been working with so far is actually of type ``$[\circ[\circ$''. With this fact in hand, we see an interesting algebraic interpretation of the elements of this representation: The irreducible (lower) highest-weight is spanned by states with non-vanishing but also non-running near-zone scalar Love numbers, while the quotient representation sandwiched between the two primary states is spanned by all the possible near-zone solutions regular at the future event horizon that have vanishing scalar Love numbers.

Let us now supplement with the case where $k\ge\hat{\ell}+1$ for which the scalar Love numbers are not zero but still exhibit no running. For the sake of this, we need to look into states $\{\upsilon^{\left(\sigma\right)}_{-\hat{\ell},-\left(n+1\right)}|n\in\mathbb{N}\}$ that span the ladder above this highest-weight representation. These states are ascendants of the state $\upsilon^{\left(\sigma\right)}_{-\hat{\ell},-1}$,
\be
	\upsilon^{\left(\sigma\right)}_{-\hat{\ell},-\left(n+1\right)} = \frac{1}{n!(2\hat{\ell}+2)_{n}}\left[L_{+1}^{\left(\sigma\right)}\right]^{n}\upsilon^{\left(\sigma\right)}_{-\hat{\ell},-1} \,,
\ee
which satisfies
\be
	L_0^{\left(\sigma\right)}\upsilon^{\left(\sigma\right)}_{-\hat{\ell},-1} = -(\hat{\ell}+1)\,\upsilon^{\left(\sigma\right)}_{-\hat{\ell},-1} \,,\quad L_{-1}^{\left(\sigma\right)}\upsilon^{\left(\sigma\right)}_{-\hat{\ell},-1} = \upsilon^{\left(\sigma\right)}_{-\hat{\ell},0} \,.
\ee
The resulting first-order inhomogeneous differential equation can be solved to get
\be\ba
	\upsilon^{\left(\sigma\right)}_{-\hat{\ell},-1} &= \frac{\mathcal{F}_{\sigma}\left(\rho\right)e^{im_{\sigma}\left(\psi_{\sigma}-\Omega_{\sigma}t\right)}e^{im_{-\sigma}\Omega_{-\sigma}}}{\left(\rho_{+}-\rho_{-}\right)\left(\hat{\ell}+1+i\Gamma^{\left(\sigma\right)}_{-\sigma}\right)}\left(e^{t/\beta}\sqrt{\Delta}\right)^{\hat{\ell}+1} \\
	&\quad\quad\times{}_2F_1\left(1,2\hat{\ell}+2;2+\hat{\ell}+i\Gamma^{\left(\sigma\right)}_{-\sigma};-\frac{\rho-\rho_{+}}{\rho_{+}-\rho_{-}}\right) \,,
\ea\ee
which is regular at the future event horizon and singular at the past one. Consequently, all the ascendants $\upsilon^{\left(\sigma\right)}_{-\hat{\ell},-\left(n+1\right)}$, with $n\in\mathbb{N}$ will also be regular at the future event horizon near-zone solutions, with $L_0$-eigenvalues $h^{\left(\sigma\right)}_{-\hat{\ell},-\left(n+1\right)}=-(n+\hat{\ell}+1)$. From this, we therefore identify
\be
	\Phi_{\omega\ell mj}\bigg|_{i\Gamma^{\left(\sigma\right)}_{+\sigma}\left(\omega\right)=k\ge\hat{\ell}+1} \propto \upsilon_{-\hat{\ell},-\left(k-\hat{\ell}\right)}^{\left(\sigma\right)} \propto \left[L_{+1}^{\left(\sigma\right)}\right]^{k-\hat{\ell}-1}\upsilon_{-\hat{\ell},-1}^{\left(\sigma\right)} \,.
\ee
As already discussed around \eqref{eq:NZRadialVanishingLove}, we do not expect to find any conspiring quasi-polynomial solution for $k\ge\hat{\ell}+1$. However, we have supplemented with the algebraic property of these remaining states to span the \textit{entire} type ``$\circ[\circ[\circ$'' representation of $\SL_{\left(\sigma\right)}$ for which the highest-weight state has weight $h^{\left(\sigma\right)}_{-\hat{\ell},0}=-\hat{\ell}$ (see~\cite{Howe1992} for our notation).

As for the near-zone solutions singular at the future event horizon, these can be similarly worked out to span the entire type ``$\circ]\circ]\circ$'' representation of $\SL_{\left(\sigma\right)}$ for which the lowest-weight state has weight $\bar{h}^{\left(\sigma\right)}_{+\hat{\ell},0}=+\hat{\ell}$, thus providing us with an algebraic argument of the absence of RG flow of the Love numbers. These constructions are shown graphically in Figure \ref{fig:SL2R_oHW_oLW}.


\begin{figure}
	\centering
	\begin{subfigure}[b]{0.49\textwidth}
		\centering
		\begin{tikzpicture}
		    \node at (0,-1) (uml4) {$\upsilon^{\left(\sigma\right)}_{-\hat{\ell},2\hat{\ell}+2}$};
		    \node at (0,0) (uml3) {$\upsilon^{\left(\sigma\right)}_{-\hat{\ell},2\hat{\ell}+1}$};
			\node at (0,1) (uml2) {$\upsilon^{\left(\sigma\right)}_{-\hat{\ell},2\hat{\ell}}$};
			\node at (0,2) (uml1) {$\upsilon^{\left(\sigma\right)}_{-\hat{\ell},1}$};
			\node at (0,3) (uml0) {$\upsilon^{\left(\sigma\right)}_{-\hat{\ell},0}$};
			\node at (0,4) (umlm1) {$\upsilon^{\left(\sigma\right)}_{-\hat{\ell},-1}$};
			\node at (0,5) (umlm2) {$\upsilon^{\left(\sigma\right)}_{-\hat{\ell},-2}$};
			
			\node at (3,-1.5) (um) {$\vdots$};
			\draw (1,-1) -- (5,-1);
			\draw [snake=zigzag] (1,0.1) -- (5,0.1);
			\draw (1,0) -- (5,0);
			\draw (1,1) -- (5,1);
			\node at (3,1.5) (up) {$\vdots$};
			\draw (1,2) -- (5,2);
			\draw (1,3) -- (5,3);
			\draw (1,4) -- (5,4);
			\draw [snake=zigzag] (1,3.1) -- (5,3.1);
			\draw (1,5) -- (5,5);
			\node at (3,5.5) (up) {$\vdots$};
			
			\draw[red] [<-] (3,3) -- node[left] {$L_{-1}^{\left(\sigma\right)}$} (3,4);
			\draw[red] [<-] (2,2) -- node[left] {$L_{-1}^{\left(\sigma\right)}$} (2,3);
			\draw[red] [<-] (3,0) -- node[left] {$L_{-1}^{\left(\sigma\right)}$} (3,1);
			\draw[red] [<-] (2,-1) -- node[left] {$L_{-1}^{\left(\sigma\right)}$} (2,0);
			\draw[blue] [->] (4,-1) -- node[right] {$L_{+1}^{\left(\sigma\right)}$} (4,0);
			\draw[blue] [->] (4,2) -- node[right] {$L_{+1}^{\left(\sigma\right)}$} (4,3);
			\draw[red] [<-] (2,4) -- node[left] {$L_{-1}^{\left(\sigma\right)}$} (2,5);
			\draw[blue] [->] (4,4) -- node[right] {$L_{+1}^{\left(\sigma\right)}$} (4,5);
		\end{tikzpicture}
		\caption{The type ``$\circ[\circ[\circ$'' representation spanned by states $\{\upsilon^{\left(\sigma\right)}_{-\hat{\ell},j}|j\in\mathbb{Z}\}$ which are regular (singular) at the future (past) event horizon and have weights $h^{\left(\sigma\right)}_{-\hat{\ell},j}=j-\hat{\ell}$.}
	\end{subfigure}
	\hfill
	\begin{subfigure}[b]{0.49\textwidth}
		\centering
		\begin{tikzpicture}
		    \node at (0,4) (upl4) {$\bar{\upsilon}^{\left(\sigma\right)}_{+\hat{\ell},2\hat{\ell}+2}$};
		    \node at (0,3) (upl3) {$\bar{\upsilon}^{\left(\sigma\right)}_{+\hat{\ell},2\hat{\ell}+1}$};
			\node at (0,2) (upl2) {$\bar{\upsilon}^{\left(\sigma\right)}_{+\hat{\ell},2\hat{\ell}}$};
			\node at (0,1) (upl1) {$\bar{\upsilon}^{\left(\sigma\right)}_{+\hat{\ell},1}$};
			\node at (0,0) (upl0) {$\bar{\upsilon}^{\left(\sigma\right)}_{+\hat{\ell},0}$};
			\node at (0,-1) (uplm1) {$\bar{\upsilon}^{\left(\sigma\right)}_{+\hat{\ell},-1}$};
			\node at (0,-2) (uplm2) {$\bar{\upsilon}^{\left(\sigma\right)}_{+\hat{\ell},-2}$};
			
			\draw [snake=zigzag] (1,-0.1) -- (5,-0.1);
			\draw (1,0) -- (5,0);
			\draw (1,1) -- (5,1);
			\node at (3,1.5) (up) {$\vdots$};
			\draw (1,2) -- (5,2);
			\node at (3,4.5) (um) {$\vdots$};
			\draw [snake=zigzag] (1,2.9) -- (5,2.9);
			\draw (1,3) -- (5,3);
			\draw (1,4) -- (5,4);
			\draw (1,-1) -- (5,-1);
			\draw (1,-2) -- (5,-2);
			\node at (3,-2.5) (up) {$\vdots$};
			
			\draw[blue] [->] (3,-1) -- node[left] {$L_{+1}^{\left(\sigma\right)}$} (3,0);
			\draw[blue] [->] (2,0) -- node[left] {$L_{+1}^{\left(\sigma\right)}$} (2,1);
			\draw[blue] [->] (3,2) -- node[left] {$L_{+1}^{\left(\sigma\right)}$} (3,3);
			\draw[blue] [->] (2,3) -- node[left] {$L_{+1}^{\left(\sigma\right)}$} (2,4);
			\draw[red] [<-] (4,3) -- node[right] {$L_{-1}^{\left(\sigma\right)}$} (4,4);
			\draw[red] [<-] (4,0) -- node[right] {$L_{-1}^{\left(\sigma\right)}$} (4,1);
			\draw[blue] [->] (2,-1) -- node[left] {$L_{+1}^{\left(\sigma\right)}$} (2,-2);
			\draw[red] [<-] (4,-1) -- node[right] {$L_{-1}^{\left(\sigma\right)}$} (4,-2);
		\end{tikzpicture}
		\caption{The type ``$\circ]\circ]\circ$'' representation spanned by states $\{\bar{\upsilon}^{\left(\sigma\right)}_{+\hat{\ell},j}|j\in\mathbb{Z}\}$ which are singular (regular) at the future (past) event horizon and have weights $\bar{h}^{\left(\sigma\right)}_{+\hat{\ell},j}=\hat{\ell}-j$.}
	\end{subfigure}
	\caption{The type ``$\circ[\circ[\circ$'' and type ``$\circ]\circ]\circ$'' representations of $\SL_{\left(\sigma\right)}$ containing all the near-zone solutions for a massless scalar field in the $5$-d Myers-Perry black hole background that have vanishing/non-running Love numbers regarding the conditions on $\Gamma^{\left(\sigma\right)}_{+\sigma}$ (see \eqref{eq:VanishingLove_IntL}-\eqref{eq:VanishingLove_HIntL}). For integer $\hat{\ell}$, the relevant condition is $i\Gamma^{\left(\sigma\right)}_{+\sigma}=k\in\mathbb{Z}$ and is depicted above. For half-integer $\hat{\ell}$, the condition becomes $i\Gamma^{\left(\sigma\right)}_{+\sigma}=k+\frac{1}{2}$, $k\in\mathbb{Z}$, and the structure of the representations is as in the above figures after replacing $k\rightarrow k+\frac{1}{2}$.}
	\label{fig:SL2R_oHW_oLW}
\end{figure}

Last, if $\hat{\ell}$ is a half-integer and $i\Gamma^{\left(\sigma\right)}_{+\sigma}\left(\omega\right)=k+\frac{1}{2}$, with $k\in\mathbb{Z}$, then we are looking at near-zone solutions of the form
\be
	\Phi_{\omega\ell mj}\bigg|_{i\Gamma^{\left(\sigma\right)}_{+\sigma}\left(\omega\right)=k+\frac{1}{2}} \propto e^{\left(k+1/2\right)t/\beta}e^{im_{\sigma}\left(\psi_{\sigma}-\Omega_{\sigma}t\right)}e^{im_{-\sigma}\psi_{-\sigma}}R_{\omega\ell mj}\left(\rho\right)\bigg|_{i\Gamma^{\left(\sigma\right)}_{+\sigma}\left(\omega\right)=k+\frac{1}{2}} \,,
\ee
which satisfy
\be
	L_0^{\left(\sigma\right)}\Phi_{\omega\ell mj}\bigg|_{i\Gamma^{\left(\sigma\right)}_{+\sigma}\left(\omega\right)=k+\frac{1}{2}} = -\left(k+\frac{1}{2}\right)\Phi_{\omega\ell mj}\bigg|_{i\Gamma^{\left(\sigma\right)}_{+\sigma}\left(\omega\right)=k+\frac{1}{2}} \,.
\ee
The above analysis is then carried away identically, after replacing $k\rightarrow k+\frac{1}{2}$.

\subsection{Local near-zone $\SL\times\SL$}
Somewhat surprisingly, we can address the remaining situations where $i\Gamma^{\left(\sigma\right)}_{-\sigma}\in\mathbb{Z}$ for integer $\hat{\ell}$ or $i\Gamma^{\left(\sigma\right)}_{-\sigma}\in\mathbb{Z}+\frac{1}{2}$ for half-integer $\hat{\ell}$, even for $\omega\ne0$. This is ought to the observation that the particular near-zone truncations are equipped with a larger $\SL_{\left(\sigma\right),\text{L}}\times\SL_{\left(\sigma\right),\text{R}}$ structure. The first $\SL$ factor is the Love symmetry $\SL$,
\be
	\SL_{\left(\sigma\right),\text{L}}=\SL_{\left(\sigma\right)} \,,
\ee
generated by the globally defined vector fields \eqref{eq:LoveGen}, $L_{m}^{\left(\sigma\right),\text{L}}=L_{m}^{\left(\sigma\right)}$, $m=0,\pm1$. The second, $\SL_{\left(\sigma\right),\text{R}}$, factor is generated by the following vector fields
\be\label{eq:LoveGenSpBroken}
	\begin{gathered}
		L_0^{\left(\sigma\right),\text{R}} = -\beta\Omega_{-\sigma}\,\partial_{-\sigma} \,, \\
		L_{\pm1}^{\left(\sigma\right),\text{R}} = e^{\pm\psi_{-\sigma}/\left(\beta\Omega_{-\sigma}\right)}\left[\mp\sqrt{\Delta}\partial_{\rho}+\partial_{\rho}\left(\sqrt{\Delta}\right)\beta\Omega_{-\sigma}\,\partial_{-\sigma}+\frac{\rho_{+}-\rho_{-}}{2\sqrt{\Delta}}\beta\left(\partial_{t}+\Omega_{\sigma}\,\partial_{\sigma}\right)\right] \,.
	\end{gathered}
\ee
The Casimirs of the two commuting $\SL$'s are exactly the same,
\be
	\mathcal{C}_2^{\left(\sigma\right),\text{R}} = \mathcal{C}_2^{\left(\sigma\right),\text{L}} = \mathcal{C}_2^{\left(\sigma\right)} \,.
\ee
In addition, the sets of vector fields generating the two $\SL$'s are regular at both the future and the past event horizons with respect to the radial variable, i.e. they do not develop poles as $\rho\rightarrow \rho_{+}$. However, $\SL_{\left(\sigma\right),\text{R}}$ is spontaneously broken down to $U\left(1\right)_{\left(\sigma\right),\text{R}}$ by the periodic identification of the azimuthal angles, $\psi_{\pm}\sim\psi_{\pm}+2\pi$, under which $L_{\pm1}^{\left(\sigma\right),\text{R}}$ develop conical deficits.

Despite this breaking of $\SL_{\left(\sigma\right),\text{R}}$ by the periodic identification of the azimuthal angles, it can still be used to explain the vanishing/non-running corresponding to the situations where $i\Gamma^{\left(\sigma\right)}_{-\sigma}\in\mathbb{Z}$ for integer $\hat{\ell}$ or $i\Gamma^{\left(\sigma\right)}_{-\sigma}\in\mathbb{Z}+\frac{1}{2}$ for half-integer $\hat{\ell}$ in a similar fashion as in the previous subsection. Solutions of the near-zone equations of motion will furnish representations labeled by the Casimir and $L_0$-eigenvalues,
\be
	\mathcal{C}_2^{\left(\sigma\right),\text{R}}\Phi_{\omega\ell mj}=\hat{\ell}(\hat{\ell}+1)\Phi_{\omega\ell mj} \,,\quad L_0^{\left(\sigma\right),\text{R}}\Phi_{\omega\ell mj} = h^{\left(\sigma\right),\text{R}}\Phi_{\omega\ell mj} \,,
\ee
with
\be
	h^{\left(\sigma\right),\text{R}} = -i\beta m_{-\sigma}\Omega_{-\sigma} = -i\Gamma^{\left(\sigma\right)}_{-\sigma} \,.
\ee
Let us construct the analogous highest-weight representation of $\SL_{\left(\sigma\right),\text{R}}$. The primary state with highest-weight $h_{-\hat{\ell},0}^{\left(\sigma\right),\text{R}}=-\hat{\ell}$, satisfying
\be
	L_0^{\left(\sigma\right),\text{R}}\upsilon_{-\hat{\ell},0}^{\left(\sigma\right),\text{R}} = -\hat{\ell}\,\upsilon_{-\hat{\ell},0}^{\left(\sigma\right),\text{R}} \,,\quad 	L_{+1}^{\left(\sigma\right),\text{R}}\upsilon_{-\hat{\ell},0}^{\left(\sigma\right),\text{R}} = 0 \,,
\ee
and having definite azimuthal numbers and frequency is given by, up to an overall normalization constant,
\be
	\upsilon_{-\hat{\ell},0}^{\left(\sigma\right),\text{R}} = \mathcal{F}_{\sigma}^{\text{R}}\left(\rho\right)e^{-i\omega t}e^{im_{\sigma}\psi_{\sigma}}e^{\hat{\ell}\psi_{-\sigma}/\left(\beta\Omega_{-\sigma}\right)}\Delta^{\hat{\ell}/2} \,, \quad \mathcal{F}_{\sigma}^{\text{R}}\left(\rho\right) \equiv \left(\frac{\rho-\rho_{+}}{\rho-\rho_{-}}\right)^{i\Gamma^{\left(\sigma\right)}_{+\sigma}\left(\omega\right)/2} \,.
\ee
This state is singular at the past event horizon and develops conical deficits as we go around the azimuthal circles, but develops no pole at the future event horizon with respect to the radial variable. The descendants,
\be
	\upsilon_{-\hat{\ell},n}^{\left(\sigma\right),\text{R}} = \left[L_{+1}^{\left(\sigma\right),\text{R}}\right]^{n}\upsilon_{-\hat{\ell},0}^{\left(\sigma\right),\text{R}} \,,
\ee
share the same boundary conditions, with the conical deficit measured by their charge under $L_0^{\left(\sigma\right),\text{R}}$,
\be
	h_{-\hat{\ell},n}^{\left(\sigma\right),\text{R}} = n-\hat{\ell} \,.
\ee

If $\hat{\ell}\in\mathbb{N}$, there exists a particular descendant that develops no conical deficit and is therefore truly regular at the future event horizon. This is the $n=\hat{\ell}$ descendant which is a null state under $L_0^{\left(\sigma\right),\text{R}}$ and corresponds to the regular near-zone solution with $\Gamma^{\left(\sigma\right)}_{+\sigma}=0$. Noticing that
\be\ba
	\left[L_{+1}^{\left(\sigma\right)}\right]^{n}&\left(\mathcal{F}_{\sigma}^{\text{R}}\left(\rho\right)e^{i\omega t}e^{im_{\sigma}\psi_{\sigma}}\upsilon\left(\rho\right)\right) = \\
	&\mathcal{F}_{\sigma}^{\text{R}}\left(\rho\right)e^{i\omega t}e^{im_{\sigma}\psi_{\sigma}}\left[-e^{\psi_{-\sigma}/\left(\beta\Omega_{-\sigma}\right)}\sqrt{\Delta}\right]^{n}\frac{d^{n}}{d\rho^{n}}\upsilon\left(\rho\right) \,,
\ea\ee
for generic $\upsilon\left(\rho\right)$, we see that the highest-weight property,
\be
	\left[L_{+1}^{\left(\sigma\right),\text{R}}\right]^{\hat{\ell}+1}\Phi_{\omega\ell mj}\bigg|_{\Gamma^{\left(\sigma\right)}_{-\sigma}=0} = 0 \,,
\ee
implies the following quasi-polynomial form for the radial wavefunction
\be
	R_{\omega\ell mj}\bigg|_{\Gamma^{\left(\sigma\right)}_{-\sigma}=0} = \mathcal{F}_{\sigma}^{\text{R}}\left(\rho\right)\sum_{n=0}^{\hat{\ell}}c_{n}\rho^{n} \,,
\ee
which exactly matches the one from \eqref{eq:StaticRadialVanishingLove} we wanted to address. The absence of RG flow is also encoded in the representation theory analysis of $\SL_{\left(\sigma\right),\text{R}}$, with the solution singular at the future event horizon (and regular at the past event horizon) being the $n=\hat{\ell}$ ascendant of the locally distinguishable lowest-weight representation of $\SL_{\left(\sigma\right),\text{R}}$ with lowest-weight $\bar{h}_{+\hat{\ell},0}^{\left(\sigma\right),\text{R}}=+\hat{\ell}$.

For the other situations of vanishing/non-running scalar Love numbers that regard the conditions on $\Gamma^{\left(\sigma\right)}_{-\sigma}$ for which the relevant near-zone solutions develop conical deficits, we can follow the same procedure as in the previous subsection and show that all the relevant regular (singular) at the future event horizon near-zone solutions span the entire representation of type ``$\circ[\circ[\circ$'' (type ``$\circ]\circ]\circ$'').