\section{Summary and Discussion}
\label{sec:Discussion}

In this work, we have extended the proposal of the Love symmetry resolution of the seemingly unnatural values of the black hole Love numbers~\cite{Charalambous:2021kcz,Charalambous:2022rre} to higher-dimensional rotating black holes in General Relativity. Namely, we have explored in full the case of static scalar responses of the $5$-dimensional Myers-Perry black hole.

Compared to the examples of Kerr-Newman black holes in $d=4$ spacetime dimensions~\cite{Charalambous:2021mea} and Schwarzschild black holes in $d=5$ spacetime dimensions~\cite{Kol:2011vg}, we find some interesting exact results. To start with, static scalar Love numbers do not in general vanish in $d=5$ for generic spin parameters, not even when $\hat{\ell}\in\mathbb{N}$, in contrast to the Schwarzschild case~\cite{Kol:2011vg}. Beyond vanishing for ``axisymmetric'' perturbations~\cite{Landry:2015zfa,Pani:2015hfa,Gurlebeck:2015xpa}, we also find that the static Love numbers vanish
for equi-rotating black holes,
which does not have a counterpart in $d=4$. 
We remark here that the current results can be straightforwardly extended to include the case of $5$-d electrically charged Myers-Perry black holes, mainly due to the fact that the discriminant function remains a quadratic polynomial in $\rho$~\cite{Chong:2005hr,Chong:2006zx}. Scalar Love numbers for $5$-d charged Myers-Perry black holes were also considered in~\cite{Consoli:2022eey}, who however focused on their slowly-rotating limits thus missing the classical RG flow feature, which we study in detail here.

It appears that the vanishing of static Love numbers for rotating black holes in $d=4$ is an exception rather than the norm. Indeed, as we have demonstrated in this work, Love numbers for rotating black holes in $d=5$ are in general non-zero and exhibit running, in agreement with
Wilsonian 
naturalness arguments. Regardless, we were still able to find near-zone truncations acquiring $\SL$ Love symmetries just like in $d=4$ Kerr-Newman black holes and $d\ge4$ Reissner-Nordstr\"{o}m black holes~\cite{Bertini:2011ga,Kim:2012mh,Charalambous:2021kcz,Charalambous:2022rre}. In the special situations where Love numbers do vanish, however, it is the highest-property of the corresponding Love symmetry that outputs this vanishing as a selection rule. We see therefore that the existence of near-zone Love symmetries appears to be routed in black holes in General Relativity, rather than only with background geometries and perturbations with vanishing Love numbers.

At the same time, we have demonstrated here that the highest-weight representation of the near-zone $\SL$'s, along with its full extension into the representation of type ``$\circ[\circ[\circ$'', plays a special role in the scalar response problem: it is entirely spanned by near-zone solutions with vanishing/non-running Love numbers. These properties are in fact \'{a} posteriori seen to be shared with the Love symmetry presented in~\cite{Charalambous:2021kcz,Charalambous:2022rre} for the $d=4$ Kerr-Newman black hole. These two features, the existence of near-zone $\SL$ symmetries and the vanishing/non-running of Love numbers, appear therefore to be mutually inclusive, with the solutions of vanishing Love numbers furnishing a quotient representation of the highest-weight Verma module of the near-zone $\SL$. We remind here, though, that only the static results can be trusted within the near-zone regime.

On that account, it is interesting to further study this hypothesis. On the one hand, it is instructive to extend the analysis to other general-relativistic black holes. The obvious next candidate to analyze is the higher-dimensional Myers-Perry black holes whose scalar field perturbations are still separable~\cite{Frolov:2006pe,Frolov:2008jr}. A technical obstacle in this approach, however, is the fact that the angular eigenvalues in $d>5$ are not known in closed form, but can be obtained as an expansion in spin parameters ratios, see e.g.~\cite{Cho:2011yp}. It would be interesting, in particular, to analyze the fate of scalar Love numbers for equi-rotating Myers-Perry black holes in odd spacetime dimensions which have the enhanced isometry subgroup $U\left(1\right)^{N}\rightarrow U\left(N\right)$. Moreover, it only deems appropriate to extend to higher-spin fields, namely, electromagnetic and gravitational perturbations. At least for spin-$1$ perturbations, this should be very similar to the work done here thanks to the separability of electromagnetic perturbations in the background of Myers-Perry black holes~\cite{Lunin:2017drx}.

On the other hand, it is still an open question whether Love symmetry exists in theories of gravity beyond General Relativity. A preliminary analysis around this was done in~\cite{Charalambous:2022rre}, where a sufficient geometric condition was extracted for spherically symmetric black holes. It would be interesting to supplement that analysis with sufficient \text{and} necessary constraints, investigate what type of theories of gravity support such geometries and whether the corresponding Love symmetries live up to their names, i.e. whether they can address the potential vanishing of Love numbers. As a counterexample, it was shown in~\cite{Charalambous:2022rre} that Love symmetry does not exist for the case of Riemann-cubed modifications of general relativity, see also~\cite{Cai:2019npx,Cardoso:2018ptl}. This nicely fitted with the corresponding computation of static scalar Love numbers which were found to be non-zero and exhibit the expected RG flow.

The nature of the Love symmetry is still not fully understood. 
From arguments stated here and in~\cite{Charalambous:2022rre}, 
the approximate Love symmetry for non-extremal black holes can be 
interpreted to be a remnant of enhanced isometries of extremal black holes. 
This could be further supported by studying the perturbations of black objects with non-spherical horizons, namely, black $p$-branes~\cite{Duff:1993ye}. One would then attempt to identify near-zone truncations admitting an $SO\left(p+1,2\right)$ symmetry. We leave this for future work. If such an analysis turns out to yield affirmative results, then the Love symmetry may shine more light on the potential holographic descriptions of asymptotically flat general-relativistic black holes~\cite{Bardeen:1999px,Guica:2008mu,Castro:2010fd,Lu:2008jk}.

There are some unconventional features of the Love symmetries regarding their property to offer IR selection rules. In particular, they have the feature of mixing IR and UV modes as can be seen from the fact that representations of the near-zone $\SL$ have non-zero frequencies and, thus, they are not directly manifested at the level of the worldline EFT. For the highest-weight representation, the corresponding frequencies have the same form as the ``near-horizon'' modes presented in~\cite{Zimmerman:2011dx}. More interestingly, the purely imaginary spacing is precisely equal to the universal QNMs level spacing as extracted from Padmanabhan's argument~\cite{Padmanabhan:2003fx}. Another possible connection to the QNM spectrum has been suggested in~\cite{Charalambous:2022rre}, where the complex frequencies of highest-weight elements were contrasted to highly dumped QNMs and total transmission modes~\cite{Cook:2016fge,Cook:2016ngj}. A more direct way to reveal such ``beyond-near-zone'' connections would be to identify symmetry-breaking parameters that depart from Love symmetric near-zone configurations and apply a spurion analysis to extract relations similar to the Gell-Mann-Okubo mass formulas~\cite{Gell-Mann:1961omu,Okubo:1961jc}.
A potential approach along these lines would be to identify the effective black hole geometries, for which Love symmetries are isometries, as scaling limits of the full asymptotically flat Myers-Perry black hole solution, see e.g.~\cite{Cvetic:2012tr}.

\paragraph{Acknowledgments}
We are grateful to Sergei Dubovsky for many 
insightful comments on the draft of this paper. 
We also thank Barak Kol 
and Zihan Zhou for useful discussions. PC is partially supported by the
2022-2023 Dean's Dissertation Fellowship.