\documentclass[twoside,11pt]{article}

\usepackage{blindtext}

% Any additional packages needed should be included after jmlr2e.
% Note that jmlr2e.sty includes epsfig, amssymb, natbib and graphicx,
% and defines many common macros, such as 'proof' and 'example'.
%
% It also sets the bibliographystyle to plainnat; for more information on
% natbib citation styles, see the natbib documentation, a copy of which
% is archived at http://www.jmlr.org/format/natbib.pdf

% Available options for package jmlr2e are:
%
%   - abbrvbib : use abbrvnat for the bibliography style
%   - nohyperref : do not load the hyperref package
%   - preprint : remove JMLR specific information from the template,
%         useful for example for posting to preprint servers.
%
% Example of using the package with custom options:
%
\usepackage[abbrvbib,preprint]{jmlr2e}

%\usepackage{jmlr2e}
\usepackage{algorithm}
\usepackage{amsmath,amssymb,amsfonts,mathrsfs}
\usepackage{breakcites}
\usepackage{graphicx}
\usepackage{soul}
\usepackage{pdfpages}
\usepackage{bm}
\usepackage{array}
\usepackage{multirow}
\usepackage{booktabs}  
\usepackage{lipsum}
\usepackage{makecell}
\usepackage{rotating}
\usepackage{tablefootnote}
%\usepackage{cleveref}

%\displaybreak
\allowdisplaybreaks

\usepackage{threeparttable}

% Definitions of handy macros can go here

%%%%% GENERAL MATH COMMANDS
% Reals
\newcommand{\R}{{\mathbb R}}
% Integers
\newcommand{\Z}{{\mathbb Z}}
% Naturals
\newcommand{\N}{{\mathbb N}}
% Expectation
\DeclareMathOperator*{\E}{\mathbb{E}}
% ^th notation
\newcommand{\tth}{^{\text{th}}}
% Small dots for integer range [a .. b]
\newcommand{\sdots}{\,..\,}
% Vectorized version of matrix
\newcommand{\matvec}{\mbox{vec}}

% := sign
\newcommand{\defeq}{\vcentcolon=}
% Zero function
\newcommand{\zf}{\mathbf{0}}
% Vector of ones
\newcommand{\ones}{\mathbf{1}}

% Argmin and argmax definitions
\DeclareMathOperator*{\argmax}{arg\,max}
\DeclareMathOperator*{\argmin}{arg\,min}


%%%%% PROBLEM STATEMENT NOTATION 
% \newcommandtwoopt{\St}[2][t][]{{S_{#1}^{#2}}} % State
\newcommand{\task}[1][i]{{\mathcal{T}_{#1}}} % Task, optionally takes index
\newcommand{\tasks}{\{ \task \}_{i=1}^N}
\newcommand{\losst}[1][i]{{l_{#1}}}
\newcommand{\lossv}[1][i]{{l_{#1}^{\textrm{val}}}}
\newcommand{\tasktarget}{{\mathcal{T}_{\textrm{target}}}}
\newcommand{\lossttarget}{l_{\textrm{target}}}
\newcommand{\lossvtarget}{l_{\textrm{target}}^{\textrm{val}}}
\newcommand{\lossttargetit}{l_{\textrm{target}}^{(k)}}
\newcommand{\losstotal}{l^{\textrm{total}}}
\newcommand{\lossopt}{l^*}

\newcommand{\thetait}[2]{\theta_{#1}^{(#2)}}
\newcommand{\phit}[1]{\phi^{(#1)}}
\newcommand{\hist}[2]{S_{#1}^{(#2)}}
\newcommand{\grad}[2]{G_{#1}^{(#2)}}

\newcommand{\Alg}{\textup{\textbf{Opt}}}
\newcommand{\MetaAlg}{\textup{\textbf{MetaOpt}}}

%%%%% Theorems
\newtheoremstyle{mytheoremstyle} % name
    {\topsep}                    % Space above
    {\topsep}                    % Space below
    {\itshape}                   % Body font
    {}                           % Indent amount
    {\scshape}                   % Theorem head font
    {.}                          % Punctuation after theorem head
    {.5em}                       % Space after theorem head
    {}  % Theorem head spec (can be left empty, meaning ‘normal’)
\theoremstyle{mytheoremstyle}
\theoremstyle{plain}
\newtheorem{theorem}{Theorem}
\newtheorem{proposition}{Proposition}
\newtheorem{assumption}{Assumption}
\newtheorem{definition}{Definition}
\newtheorem{lemma}{Lemma}
\theoremstyle{remark}
\newtheorem{remark}{Remark}


\newcolumntype{M}[1]{>{\centering\arraybackslash}m{#1}}
%\newcommand{\dataset}{{\cal D}}
%\newcommand{\fracpartial}[2]{\frac{\partial #1}{\partial  #2}}

\newcommand{\peiyuan}[1]{{\color{blue} Peiyuan's comment: #1}}
\newcommand{\jingzhao}[1]{{\color{red} Jingzhao's comment: #1}}
\newcommand{\jiaye}[1]{{\color{magenta} Jiaye's comment: #1}}


% Heading arguments are {volume}{year}{pages}{date submitted}{date published}{paper id}{author-full-names}
\jmlrheading{24}{2023}{1-\pageref{LastPage}}{04/03; Revised XX/XX}{XX/XX}{XX-XXXX}{Zhang, Teng and Zhang}
\usepackage{lastpage}
%\jmlrheading{23}{2022}{1-\pageref{LastPage}}{1/21; Revised 5/22}{9/22}{21-0000}{Peiyuan Zhang, Jiaye Teng and Jingzhao Zhang}

% Short headings should be running head and authors last names

\ShortHeadings{Lower Generalization Bound in Smooth SCO}{Zhang, Teng and Zhang}
\firstpageno{1}

\begin{document}

\title{Lower Generalization Bounds for GD and SGD \\
in Smooth Stochastic Convex Optimization}

\author{\name Peiyuan Zhang%\thanks{Partly done when the author was an intern at Shanghai Qizhi Institute.} 
       \email peiyuan.zhang@yale.edu \\
       \addr Yale University%\\
       %New Haven, CT
       \AND
       \name Jiaye Teng  \email tjy20@mail.tsinghua.edu.cn \\
       \addr Tsinghua University%\\
       %Beijing, China
       \AND
       \name Jingzhao Zhang\thanks{Corresponding author.} \email jingzhaoz@mail.tsinghua.edu.cn \\
       \addr Tsinghua University \& Shanghai Qizhi Institute
       %Beijing, China
       }

\editor{My editor}

\maketitle

\begin{abstract}%   <- trailing '%' for backward compatibility of .sty file
This work studies the generalization error of gradient methods. More specifically, we focus on how training steps $T$ and step-size $\eta$ might affect generalization in \emph{smooth} stochastic convex optimization (SCO) problems. We first provide tight excess risk lower bounds for Gradient Descent (GD) and Stochastic Gradient Descent (SGD) under the general \emph{non-realizable} smooth SCO setting, suggesting that existing stability analyses are tight in step-size and iteration dependence, and that overfitting provably happens. 
Next, we study the case when the loss is \emph{realizable}, i.e. an optimal
solution minimizes all the data points. Recent works show better rates can be attained but the improvement is reduced when training time is long. Our paper examines this observation by providing excess risk lower bounds for GD and SGD in two \emph{realizable} settings: 1)  $\eta T = \bigO{n}$, and (2)  $\eta T = \bigOmega{n}$, where $n$ is the size of dataset. In the first  case $\eta T = \bigOmega{n}$, our lower bounds tightly match and certify the respective upper bounds. However, for the case $\eta T = \bigOmega{n}$, our analysis indicates a gap between the lower and upper bounds. A conjecture is proposed that the gap can be closed by improving upper bounds, supported by analyses in two special scenarios. 
%\lipsum[1]
\end{abstract}

\begin{keywords}
  generalization, gradient methods, stochastic convex optimization, realizable setting, lower bounds
\end{keywords}

\section{Introduction}

\begin{figure}[t]
  \centering
     \includegraphics[width=\linewidth]{figure/teaser_v2.pdf}
   \caption{\textbf{Illustration.} UrbanGIRAFFE generates a photorealistic image given a sampled panoptic prior in the form of a semantic voxel grid and object layout. Our method enables diverse controllability regarding camera pose, instance, and stuff.}
   \label{fig:overview}
   \vspace{-0.2cm}
\end{figure}

Generating photorealistic urban scenes has many applications in simulation, gaming and virtual reality. 
Unfortunately, designing diverse urban scenes with novel 3D visual content is typically expensive and time-consuming as it requires the expertise of professional artists. 

Recent advances in generative models have demonstrated a promising direction to reduce the cost via learning to generate images from data.
Ideally, the generated scenes should be controllable in terms of camera pose and 3D content. For example, the camera should be able to move freely in the scene with six degrees of freedom. The poses of instantiated objects (\eg, cars) should be able to be manipulated independently. Furthermore, the layout of the scene should be controllable.

There are many attempts to generate photorealistic urban images. 
Several methods study semantic image synthesis to transfer a 2D semantic segmentation map to an RGB urban scene image~\cite{Isola2017CVPR,Park2019CVPRa,Schonfeld2021ICLR}. However, when changing the camera poses, the generated images across multiple frames may not be consistent using such 2D generative models.
Recently, 3D-aware generative models have witnessed a rapid progress by lifting the generation process to the 3D space. Despite achieving multi-view consistency, most existing 3D-aware generative models are limited to object-centric images, e.g., faces and cars~\cite{Schwarz2020NIPS,Chan2021CVPR,Chan2022CVPR}. 
There are a few attempts to generate scene images in a compositonal manner~\cite{Liao2020CVPRa, Niemeyer2021CVPR,Nguyen-Phuoc2020NEURIPS,Xu2022ARXIV}. However, all these methods struggle to learn a good geometry of the background and hence do not support large camera movement, e.g., moving the camera along the road. Another line of work enables camera movement but ignores the compositional nature of the scene, thus lacking controllability of the 3D content~\cite{DeVries2021ICCV,Bautista2022NEURIPS}.





In this paper, we propose UrbanGIRAFFE to address the challenging task of compositional and controllable 3D-aware image synthesis of urban scenes, see \figref{fig:overview}. Our key idea is to leverage scene-level but coarse 3D panoptic prior, simplifying the task of learning complex geometry through 2D supervision and incorporating semantic information for scene editing. The panoptic prior, including semantic voxel grids of uncountable stuff and bounding boxes of countable objects, can be obtained from existing datasets~\cite{Liao2022PAMI} or inferred from pre-trained models~\cite{Cao2022CVPR}.
Specifically, our model represents the scene as compositional neural feature fields consisting of stuff, objects, and sky.  %
We propose a semantic voxel-conditioned stuff generator, effectively preserving the semantic and geometry information provided by the prior. In terms of objects, we follow GIRAFFE~\cite{Niemeyer2021CVPR} to generate objects in canonical space by leveraging the object layout prior.  We further model the sky and far regions using a sky generator.
With all three generators, we render a composited feature map via volume rendering and upsample it to the target image using a neural renderer. 
For the complicated urban scenes, we observe that training with an adversarial loss on the full image alone is insufficient. We additionally employ an adversarial loss applied to objects and a reconstruction loss to the stuff image regions to improve the image fidelity.

Our contributions are as follows. i) We propose to study the challenging task of 3d-aware generative models for urban scenes with diverse controllability in terms of large camera movement, objects manipulation and stuff editing.  ii) We leverage coarse 3D panoptic prior to address this challenging task and design compositional generative radiance fields that leverages the prior information effectively.
iii) Our method demonstrates state-of-the-art performance compared to existing methods on both synthetic and real-world datasets, including the challenging KITTI-360 dataset. 

\section{Problem Setup and Background} \label{sec: setting}
We focus on the standard setting of stochastic convex optimization (SCO) problem. Consider an element $z$ in a sample space $\cZ$. We receive a dataset of finite samples $S = \{z_1, \dots, z_n\}$, where each $z_i$ is i.i.d. drawn from an unknown distribution $D$ and $n$ is the size of dataset.  Our goal is to find a model parameterized by $w \in \cW \subseteq \bbR^d$ that minimizes the \emph{population} (or expected) risk over $D$, defined as:
\begin{equation}
    F(w) = \bbE_{z \sim D}[f(w,z)]
\end{equation}
where $f(w, z): \bbR^d \times \cZ \to \bbR$ is the loss function evaluated on a single example $z \in \cZ$.

Since the population loss $F$ is typically inaccessible, we instead employ an averaged substitute on sample $S$, known as the \emph{empirical} risk:
\begin{equation}
    F_S(w) = \frac{1}{n} \sum_{i=1}^n f(w,z), \qquad S \sim D^n.
\end{equation}

Given dataset $S$ and any (stochastic) algorithm $\cA$, we denote $\cA[S]$ as the output of running $\cA$ on the training sample $S$. In this paper, we are interested in bounding the \emph{excess population risk} of $\cA[S]$:
\begin{equation*}
    \bbE_{S,\cA}[F(\cA[S])] - \min_{w \in \cW} F(w),
\end{equation*}
where the expectation is taken over the randomness of sample $S$ and algorithm $\cA$.

\subsection{Gradient methods}
In this work, we focus on understanding the excess risk for two simplified algorithms: Gradient Descent (GD) and Stochastic Gradient Descent (SGD). Gradient descent is one of the most well-known optimization methods. At iteration $t$, GD employs the following recurrence:
\begin{equation} \label{eq: gd}
    w_{t+1} = w_t - \eta \nabla F_S(w_t),  
\end{equation}
where $\eta > 0$ is the step-size and $\nabla F_S(w)$ is the average stochastic gradient on sample set $S$. We usually employ the time average $w_{\gd} = \hat{w}_T = \frac{1}{T} \sum_{t=1}^T w_t$ as the output of GD.  

In practice, many practitioners in learning applications favor the Stochastic Gradient Descent (SGD) method over GD due to its computational efficiency. More precisely, the variant people usually refer as SGD is the the \emph{with-replacement} version, i.e., in iteration $t \in [T]$,
\begin{equation} \label{eq: sgd}
    w_{t+1} = w_t - \eta \nabla f(w_t,z_{i_t}), 
\end{equation}
where $z_{i_t}$ is uniformly sampled from $S$ with replacement as $i_t \sim \text{Unif}([n])$. We will keep the convention and use SGD to denote the with-replacement version throughout the paper. The output for SGD is the average $w_{\sgd} = \frac{1}{T} \sum_{t=1}^T w_t$.  

\subsection{Smooth stochastic convex  optimization}
In order to derive non-vacuous bounds on the excess risk for SCO, we put standard assumptions on the properties of $f(w, z)$. We assume the access to the value $f(w,z)$ and the unbiased stochastic gradient estimator $\nabla f(w, z)$ for any $w \in \cW$ and $z \in \cZ$. When the function is nonsmooth, this problem has been extensively studied \citep{bassily2020stability}, and known rates were proven to be optimal \citep{amir2021sgd,sekhari2021sgd,nemirovskij1983problem}.

However, less is known when the function is differentiable and smooth. Indeed, while upper bounds have been well-established in literature \citep{hardt2016train,lei2020fine,nikolakakis2022beyond}, the optimality of these results need to be certified by corresponding lower bound constructions. In this work, we focus on the smooth SCO setting and make the following assumptions.
%\begin{definition} \label{def: lipschitz}
%$f(w, z)$ has $G$-Lipschitz gradient if for any $w$, it holds $\| \nabla f(w, z) \| \leq G$ and $z \in \cZ$.
%\end{definition}

\begin{definition} \label{def: smooth}
$f(w, z)$ is $L$-smooth if it satisfies $\| \nabla f(w_1, z) - \nabla f(w_2, z) \| \leq L \| w_1 - w_2 \|$ for any $w_1, w_2$ and $z \in \cZ$.
\end{definition}

%\begin{definition} \label{def: bounded}
%Each $f(w, z)$ is $L$-smooth, i.e. $\| \nabla f(w_1, z) - \nabla f(w_2, z) \| \leq L \| w_1 - w_2 \|$ for any $w_1, w_2$.
%\end{definition}

\begin{definition} \label{def: convex}
$f(w, z)$ is convex if it satisfies $f(w_1, z) \geq f(w_2, z) + \langle w_1 - w_2, \nabla f(w_2, z)\rangle$ for any $w_1, w_2$ and $z \in \cZ$.
\end{definition}


\begin{table}[t]
\centering
\begin{threeparttable}
\begin{tabular}{ M{0.9cm} M{2.5cm}  M{3.6cm}  M{3.6cm} M{2.3cm}  }
\toprule
& & & & \\[-0.3cm]
& & GD & SGD & Best Sample Complexity \\[0.1cm] 
 \midrule 
& & & & \\[-0.2cm]
\multirow{4}{*}{\begin{sideways}Non-realizable\end{sideways}}
%\multirow{4}{*}{\begin{sideways}\ \end{sideways}}
\multirow{4}{*}{\begin{sideways}(Any $T$) $\quad$ \end{sideways}}
& Upper bound & \thead{{\normalsize	 $\bigO{\frac{1}{\eta T} + \frac{\eta T}{n} }$} \\{\footnotesize\citep{hardt2016train}}}  & \thead{{\normalsize	 $\bigO{\frac{1}{\eta T} + \frac{\eta T}{n} }$} \\{\footnotesize\citep{hardt2016train}}} & $\bigO{1/\sqrt{n}}$ \\[0.3cm]
& & & & \\[-0.1cm]
& Lower bound & \thead{{\normalsize	 $\bigOmega{\frac{1}{\eta T} + \frac{\eta T}{n} }$} \\ {(\footnotesize Theorem.~\ref{thm: lb-sco-gd})}} & \thead{{\normalsize	 $\bigOmega{\frac{1}{\eta T} + \frac{\eta T}{n} }$} \\{(\footnotesize Theorem.~\ref{thm: lb-sco-sgd})}} & $\bigOmega{1/\sqrt{n}}$  \\[0.4cm]
\midrule 

& & & & \\[-0.2cm]
\multirow{4}{*}{\begin{sideways}Realizable \end{sideways}}
\multirow{4}{*}{\begin{sideways}\ \end{sideways}}
\multirow{4}{*}{\begin{sideways}$\quad $($T = \bigO{n}$) \end{sideways}}
& Upper bound  & $\bigO{\frac{1}{\eta T} + \frac{1}{n} + \frac{\eta T}{n^2} }$\tnote{$\dag$}  {\footnotesize \citep{nikolakakis2022beyond}} & $\bigO{\frac{1}{\eta T} + \frac{\eta}{n} + \frac{\eta T}{n^2} }$ {\footnotesize	 \citep{lei2020fine}} & $\bigO{1/n}$  \\[0.3cm]
& & & & \\[-0.1cm]
& Lower bound  & $\bigOmega{\frac{1}{\eta T} + \frac{1}{n} + \frac{\eta T}{n^2}}$ {(\footnotesize Theorem.~\ref{thm: lb-1})} & $\bigOmega{\frac{1}{\eta T} + \frac{1}{n} + \frac{\eta T}{n^2}}$ {(\footnotesize Theorem.~\ref{thm: lb-1})} & $\bigOmega{1/n}$ \\[0.4cm] %\cmidrule{2-5}
\midrule 

& & & & \\[-0.2cm]
\multirow{4}{*}{\begin{sideways}Realizable \end{sideways}}
\multirow{4}{*}{\begin{sideways}\ \end{sideways}}
\multirow{4}{*}{\begin{sideways}$\ $($T = \bigOmega{n}$) \end{sideways}}
% & & & & \\[-0.2cm]
& Upper bound   & $\bigO{\frac{1}{\eta T} + \frac{\sqrt{\eta T}}{n}}$ {\footnotesize	 \citep{schliserman2022stability}} & $\bigO{\frac{1}{\eta T} + \frac{\sqrt{\eta T}}{n}}$ {\footnotesize	 \citep{schliserman2022stability}} & $\bigO{1/n^{2/3}}$  \\[0.3cm] 
 & & & & \\[-0.1cm]
& Lower bound  & $\bigOmega{\frac{1}{\eta T} + \frac{1}{n} + \frac{\eta T}{n^2}}$ {(\footnotesize Theorem.~\ref{thm: lb-2})} & $\bigOmega{\frac{1}{\eta T} + \frac{1}{n} + \frac{\eta T}{n^2}}$ {(\footnotesize Theorem.~\ref{thm: lb-2})} & $\bigOmega{1/n}$ \\[0.4cm] 
\bottomrule
\end{tabular}
\caption{\small Summary of our results. We present our lower bounds and compare with existing upper bounds. In particular, we split to three setting: non-realizable, realizable under $T = \bigO{n}$ and realizable under $T = \bigOmega{n}$. For each setting, we provide lower bounds for the excess risk of GD, SGD, and corresponding best possible sample complexity bounds.}
\label{tab: summary}
\begin{tablenotes}
\item[$\dag$] The bound in $\eta, T$ and $n$ is not explicitly stated in \citet{nikolakakis2022beyond}. For the expression, please refer to a derivation in Appendix~\ref{appendix: gd-ub}.
\end{tablenotes}
\end{threeparttable}
%\hspace{-0.3cm}
\end{table}

The smooth SCO problem can be divided into two cases depending whether the optimal solution minimizes all data points simultaneously. 

\paragraph{Realizable smooth SCO.}
A SCO problem is said to be realizable if there exists a solution that simultaneously minimizes over all the instance loss function $f(w,z)$ for any $z \in \cZ$. We give the formal definition below.
\begin{definition} \label{def: realizable}
We say that $f(w, z)$, $z \in \cZ$ formalizes a realizable setting if for any $z \in \cZ$ $$f(w^*, z) = \min_{w} f(w, z), \quad \text{where} \quad w^* = \argmin_{w}F(w).$$
\end{definition}
Combined with the above definitions, we assume $f(w,z)$ to be smooth, convex, and also realizable. We will refer to the setting as the \emph{realizable} smooth SCO. The realizable condition implies immediately the property called \emph{weak growth condition}, which is stated in the following lemma.
\begin{lemma} \label{lemma: growth-condition} If $f(w, z), z \in Z$ is realizable and $L$-smooth, then for any $w, z \in \cZ$ it holds 
    \begin{equation*}
        \| \nabla f(w, z) \|^2 \leq 2L \left( f(w, z)  - f(w^*, z)\right).
    \end{equation*}
\end{lemma}
The growth condition connects the rates at which the stochastic gradients shrink relative to its value. It is widely employed in stochastic optimization literature to improve the convergence rate of SGD and GD under overparameterized or realizable setting \citep{vaswani2019fast}. Recent papers \citep{lei2020fine,schliserman2022stability,nikolakakis2022beyond} focused on the generalization bound under realizable smooth SCO also suggest that such an assumption improves the sample complexity upper bounds beyond the rate of $\bigO{1/\sqrt{n}}$, which is the well-known result for GD and SGD without the realizable assumption \citep{hardt2016train}.

\paragraph{Non-realizable smooth SCO.} 
We say a convex learning problem is non-realizable if it does not satisfy the condition in Definition~\ref{def: realizable}. In this setting, known upper bounds for sample complexity actually yield a slower convergence rate at $\bigO{1/\sqrt{n}}$. 

\paragraph{Different complexities in small and large time horizon.}
Although upper bounds for non-realizable and realizable problems differ, they share one thing in common, that the excess risk first decreases and then grows with the number of iterations. Based on this, we classify the bounds into two categories according to their assumptions on the number of iterations $T$.

We highlight the difference between the small time horizon regime $T = \bigO{n}$ and large time horizon regime $T = \bigOmega{n}$. Existing results suggest a gap between the two regimes for the realizable setting. In particular, we will use SGD as an example to shed some light on the gap: the two regimes are known as the \emph{single-pass} and \emph{multi-pass} SGD. In the setting $T = \bigO{n}$, \citet{lei2020fine} established the first excess risk upper bound for SGD under realizable smooth SCO:
\begin{equation} \label{eq: ub-sgd-tn}
     \bbE[F(w_{\sgd})] - \min_{w \in \cW} F(w) = \bigO{\frac{1}{\eta T} + \frac{\eta}{n} + \frac{\eta T}{n^2}}.
\end{equation}
By choosing $T = n$ and $\eta = 1$, we obtain a sample complexity upper bound of $\bigO{1/n}$. However, the bound in Eq.~\eqref{eq: ub-sgd-tn} suffers from the weakness that the optimal bound of $\bigO{1/n}$ only holds when $T = \bigO{n}$ and becomes vacuous under large time horizon $T = \bigOmega{n}$. To overcome this issue, \citet{schliserman2022stability} provided an alternative bound for excess population risk:
\begin{equation} 
 \label{eq: ub-sgd-t}
     \bbE[F(w_{\sgd})] - \min_{w \in \cW} F(w) = \bigO{\frac{1}{\eta T} + \frac{\sqrt{\eta T}}{n}}.
\end{equation}
This result gives non-vacuous sample complexity lower bound under the large or infinite time regime $T = \bigOmega{n}$: for any such $T$, we can set $\eta = n^{2/3}/T$ and obtain sample complexity bound of $\bigO{1/n^{2/3}}$. Compared with the optimal $\bigO{1/n}$ from Eq.~\eqref{eq: ub-sgd-tn} under regime $T = \bigO{n}$, this is \emph{suboptimal} in the rate of $n$, but holds for a more general choice of $T$. Similar results hold for GD under realizable smooth SCO \citep{nikolakakis2022beyond,schliserman2022stability} and are summarized in Table~\ref{tab: summary}. 

In the non-realizable setting, the upper bound for SGD is characterized in the seminal paper of \citet{hardt2016train}:
\begin{align*}
    \label{eq: ub-sgd-sco}
     \bbE[F(w_{\sgd})] - \min_{w \in \cW} F(w) = \bigO{\frac{1}{\eta T} + \frac{\eta T}{n}}.
\end{align*}
This implies a sample complexity bound of $\bigO{1/\sqrt{n}}$ when we set $\eta = \sqrt{n}/T$ for any $T$. Nevertheless, the best known result for lower bound $\bigOmega{1/\sqrt{n}}$ is established \emph{only} in the single-pass regime $T = n$ \citep{nemirovskij1983problem}. For GD, there is currently no lower bound result for \emph{either} small or large time horizons.

Although the above upper bounds for both realizable and non-realizable problems exhibit an increase for a large time horizon, in practice, we often observe that training for longer periods does not lead to a larger generalization gap. In the next section, we provide tight lower bounds to show that realizability can be a key factor in understanding when models trained for longer periods will overfit.


%  is obtained from Eq.~\eqref{eq: ub-gd-t} for any $\eta,T$ satisfying $\eta T = n^{2/3}$, which is suboptimal compared with the optimal $\bigO{1/n}$ from Eq.~\eqref{eq: ub-gd-tn}. %Nevertheless, the bound in Eq.~\eqref{eq: ub-gd-t} has an advantage since it achieves $\bigO{1/n^{2/3}}$ for any large time horizon $T = \bigOmega{n}$ when choosing small step-size $\eta = n^{2/3}/T$.


%\paragraph{Algorithm stability and generalization.}
%Excess population risk
%\begin{equation}
%    \bbE_{S,A}[F(w_{A})] \leq F(w^*) + \epsilon
%\end{equation}

% \paragraph{Gradient descent.}
% % \jiaye{I do not think it is wise to mention SCO above and introduce GD here}
% The most well-known optimization method is perhaps the \emph{Gradient Descent} (GD) method. At iteration $t$, GD employs the following recurrence:
% \begin{equation} \label{eq: gd}
%     w_{t+1} = w_t - \eta \nabla F_S(w_t),  
% \end{equation}
% where $\eta > 0$ is the step-size and $\nabla F_S(w)$ is the average stochastic gradient on sample set $S$. We usually employ the time average $w_{\gd} = \hat{w}_T = \frac{1}{T} \sum_{t=1}^T w_t$ as the output of SGD. \peiyuan{Comment on last-iterate.} 

% \citet{nikolakakis2022beyond} established the first generalization bound for GD under the realizable smooth SCO setting: \jiaye{I check the bound and believe that it is 1/n. I put the bound in the appendix. }
% \begin{equation} \label{eq: ub-gd-tn}
%     \bbE[F(w_{\gd})] - \min_{w \in \cW} F(w) = \bigO{\frac{1}{\eta T} + \frac{\eta}{n} + \frac{\eta T}{n^2}}.
% \end{equation}
% In particular, choosing $T = n$ and $\eta = 1$, we obtain an sample complexity upper bound of $\bigO{1/n}$. However, the bound in Eq.~\eqref{eq: ub-gd-tn} suffers from the weakness that the optimal bound of $\bigO{1/n}$ only holds when $T = \bigO{n}$ and becomes vacuous under infinite time horizon $T = \bigOmega{n}$. To overcome this issue, \citet{schliserman2022stability}  provided an alternative bound for excess population risk:
% \begin{equation} 
% \label{eq: ub-gd-t}
%     \bbE[F(w_{\gd})] - \min_{w \in \cW} F(w) = \bigO{\frac{1}{\eta T} + \frac{\sqrt{\eta T}}{n}}.
% \end{equation}
% This result gives non-vacuous sample complexity lower bound under the large or infinite time regime $T = \bigOmega{n}$: for any such $T$, we can set $\eta = n^{2/3}/T$ and obtain sample complexity bound of $\bigO{1/n^{2/3}}$. Compared with the optimal $\bigO{1/n}$ from Eq.~\eqref{eq: ub-gd-tn} under regime $T = \bigO{n}$, this is \emph{suboptimal} in the rate of $n$, but holds for a more general choice of $T$.%  is obtained from Eq.~\eqref{eq: ub-gd-t} for any $\eta,T$ satisfying $\eta T = n^{2/3}$, which is suboptimal compared with the optimal $\bigO{1/n}$ from Eq.~\eqref{eq: ub-gd-tn}. %Nevertheless, the bound in Eq.~\eqref{eq: ub-gd-t} has an advantage since it achieves $\bigO{1/n^{2/3}}$ for any large time horizon $T = \bigOmega{n}$ when choosing small step-size $\eta = n^{2/3}/T$.
% \jiaye{The logic here is a little bit strange, I am not sure about the role of Schliserman et al. here. It also cannot generalize to infinite cases. It might be better if we can reorganize it as GD-SGD-O(n) bound - Omega(n) bound}

% \paragraph{Stochastic gradient descent.}
% Many practitioners in learning applications favor the \emph{Stochastic Gradient Descent} (SGD) method over GD due to its computational efficiency. More precisely, the variant people usually refer as SGD is the the \emph{with-replacement} version, i.e., in iteration $t \in [T]$,
% \begin{equation} \label{eq: sgd}
%     w_{t+1} = w_t - \eta \nabla f(w_t,z_{i_t}), 
% \end{equation}
% where $i_t \sim \text{Unif}([n])$ is uniformly sampled from $S$ with replacement. %This is in contrast with the without-replacement version stated below. 
% We will keep the convention and use SGD to denote the with-replacement version throughout the paper. The output for SGD is the average $w_{\sgd} = \frac{1}{T} \sum_{t=1}^T w_t$.  

% Similar to GD, the generalization upper bound for SGD under realizable smooth SCO is also \emph{bipartite} with respect to the condition between $T$ and $n$. In particular, \cite{lei2020fine} provided a bound\footnote{The Theorem in \citet{lei2020fine} requires an additional condition $f(w,z)$ is nonnegative to enforce the self-boundedness of $\nabla f(w,z)$. In our setting, this can be replaced by considering realizable condition and Lemma~\ref{lemma: growth-condition}. We make a remark to avoid the incompatibility of assumptions. } of 
% \begin{equation} \label{eq: ub-sgd-tn}
%      \bbE[F(w_{\sgd})] - \min_{w \in \cW} F(w) = \bigO{\frac{1}{\eta T} + \frac{\eta}{n} + \frac{\eta T}{n^2}},
% \end{equation}
% which leads to an optimal $\bigO{1/n}$ when we set $\eta = 1$, $T = n$ albeit the bound holds only when $T = \bigOmega{n}$. Concurrently, the alternative bound 
% \begin{equation} \label{eq: ub-sgd-t}
%      \bbE[F(w_{\sgd})] - \min_{w \in \cW} F(w) = \bigO{\frac{1}{\eta T} + \frac{\sqrt{\eta T}}{n}},
% \end{equation}
% established by \citet{schliserman2022stability}, holds for large and infinite time horizon case $T = \bigOmega{n}$ but only yields an suboptimal $\bigO{1/n^{2/3}}$ when we set $\eta = n^{2/3}/T$.

% % \paragraph{Without-replacement SGD.} In contrast to the with-replacement SGD, the without-replacement variant has become an essential topic in optimization community. In this work we focus on the Single Shuffling version of without-replacement methods.  comes with a fixed permutation of stochastic loss function from the sample set $S$. Without the loss of generality we assume permutation $\vpi(i) = i$ for any $i \in [n]$ to simplify the analysis. Therefore, we can formulate the iterations of single-shuffling SGD as
% % \begin{equation} \label{eq: ss}
% %     w^k_{t+1} = w^k_t - \eta \nabla f(w^k_t,z_t),  
% % \end{equation}
% % where.

% We remark that we will use compact notation $w_{\gd}$, $w_{\sgd}$ and $w_{\sgdss}$ to denote the output of respective methods, i.e., the value $w_T$ at final iteration $T$.


\section{Main Result: Lower Bounds in Smooth SCO} \label{sec: lb}
In this section we proceed to state our main results on lower bounds for the smooth SCO.
We aim at filling the gaps mentioned in Section~\ref{sec: setting} for all the above settings. To this end, we will split our discussion to three parts: (1) non-realizable, (2) realizable under $T = \bigO{n}$ and (3) realizable under $T = \bigOmega{n}$.

\subsection{Non-realizable setting} \label{sec: nonrealizable}
We first discuss the non-realizable setting and provide a novel lower bound for the excess risk of GD in the following theorem.
\begin{theorem} \label{thm: lb-sco-gd}
    For any $\eta > 0$, $T \geq 1$ with $\bigOmega{1/T} = \eta = \bigO{1}$\footnote{This is a regular condition in optimization and generalization. This is because (1) step-size cannot exceed $\bigO{1}$, in order to make the optimization method converge for $\bigO{1}$-smooth function and (2) $T$ is arbitrarily large to ensure $\eta T = \bigOmega{1}$. We will assume this holds everywhere and do not repeat the regularity in the statement of rest theorems and lemmas.}, there exists a convex, $1$-smooth $f(w,z): \bbR \to \bbR$ for every $z \in \cZ$, and a distribution $D$ such that, with probability $\Theta(1)$, the output $w_{\gd}$ for GD satisfies
    \begin{align*}
        \bbE[F(w_T)] - F(w^*) = \bigOmega{\frac{1}{\eta T} + \frac{\eta T}{n}}.
    \end{align*}
\end{theorem}
The lower bound in Theorem~\ref{thm: lb-sco-gd} tightly matches the corresponding upper bound established in \cite{hardt2016train} (see Table~\ref{tab: summary}). It can be translated to a lower bound of sample complexity: for any $T \geq 1$, by setting $\eta = \sqrt{n}/T$, we derive a $\bigOmega{1/\sqrt{n}}$ bound which certifies the optimality of existing upper bound of $\bigO{1/\sqrt{n}}$. To the best of our knowledge, this is the first such result for GD. A recent work provides lower bound for the uniform stability of (S)GD \citep{zhang2022stability}, but it does not trivially imply a bound in excess risk.

The crucial step in the proof of Theorem~\ref{thm: lb-sco-gd} is to find a hard instance that gives an overfitting lower bound $\bigOmega{\eta T/n}$. To this end, we employ a technique inspired by Theorem 3 and Lemma 7 in \citet{sekhari2021sgd}: in non-realizable setting, the stochastic gradient does not necessarily scale down with the value of $f(w,z)$. As a result, by utilizing an \emph{anti-concentration} argument, we show that with non-vanishing probability $\bigOmega{1}$, the absolute value of $w_t$ increases by a rate of $\bigOmega{\eta/\sqrt{n}}$ in each step. Subsequently, calculation suggests a $\bigOmega{\eta T/n}$ bound for the function value. The details can be found in Appendix~\ref{appendix: lb-sco-gd}. In the meanwhile, the term $\bigOmega{1/\eta T}$ reflects the optimization error and the proof is provided in Lemma~\ref{lemma: lb-suboptimality}, Appendix~\ref{appendix: lb-suboptimality}. 

Subsequently, we proceed to a lower bound for SGD that is similar to GD. We state it in the following theorem.
\begin{theorem} \label{thm: lb-sco-sgd}
    For any $\eta > 0$, $T \geq 1$, there exists a convex, $1$-smooth $f(w,z): \bbR \to \bbR$ for every $z \in \cZ$, and a distribution $D$ such that, %with probability $\Theta(1)$, 
    the output $w_{\sgd}$ for SGD satisfies
      \begin{align*}
        \bbE[F(w_{\sgd})] - F(w^*) = \bigOmega{\frac{1}{\eta T} + \frac{\eta T}{n}}.
    \end{align*}
\end{theorem}
This also matches the SGD upper bound in \cite{hardt2016train} and implies a sample complexity bound $\bigOmega{1/\sqrt{n}}$ if we set $\eta = \sqrt{n}/T$ for any $T$.  

We emphasize the bound for SGD is novel compared with existing works: it is a forklore that in \citet{nemirovskij1983problem}, single-pass SGD achieves a sample complexity lower bounds for Lipschitz convex functions (where a smooth function within a bounded domain is automatically Lipschitz). Nevertheless, our result is first to provide such results in the multi-pass or large time horizon $T = \bigOmega{n}$ setting and reflects a hardship between $T$ and $n$ since it suggests overfitting and an increase of error when the iteration becomes large.

\subsection{Realizable: case \texorpdfstring{$T = O(n)$}{T=O(n)}} \label{sec: t_equal_n}
In the subsection we provide our lower bounds for the realizable setting when condition $T = \bigO{n}$ is satisfied. The next theorem  characterizes the lower bounds for GD and SGD.
\begin{theorem} \label{thm: lb-1}
For every $\eta > 0$, $T > 1$, if condition $T = \cO(n)$ holds, then there exists a convex, $1$-smooth and realizable $f(w, z): \bbR^d \to \bbR$ for every $z \in \cZ$, and a distribution $D$ such that, with a bounded initialization $\| w_0 - w^* \| = \bigO{1}$, the output $w_{\gd}$ for GD satisfies
\begin{align*}
    \bbE[F(w_{\gd})] - F(w^*) = \bigOmega{\frac{1}{\eta T} + \frac{1}{n} + \frac{\eta T}{n^2}}.
\end{align*}
Similarly, the output $w_{\sgd}$ for SGD satisfies
\begin{align*}
    \bbE[F(w_{\sgd})] - F(w^*) = \bigOmega{\frac{1}{\eta T} + \frac{1}{n} + \frac{\eta T}{n^2}}.
\end{align*}
\end{theorem}
Theorem~\ref{thm: lb-1} suggests that there is an hard instance for both GD and SGD under the realizable smooth SCO. It is worth noting that we assume bounded initialization $\| w_0 - w^* \| = \bigO{1}$. This is standard and necessary in the generalization literature: the bound will be vacuous and arbitrarily bad if initial point is away from the optimal point with infinite distance.

Under regime $T = \bigO{n}$, the lower bound for GD tightly matches the upper bound in \cite{nikolakakis2022beyond}, and the lower bound for SGD almost tightly matches the lower bound in \cite{lei2020fine} up to a $\eta$ factor in the second term. Please refer to Table~\ref{tab: summary} for a comparison. We will combine the discussion for GD and SGD due to the similarity. Both upper and lower bounds are non-vacuous only when $T \leq n$ for $\eta = \Theta(1)$. As a result, to obtain the optimal sample complexity $\bigOmega{1/n}$ from Theorem~\ref{thm: lb-1}, we need to restrict the number of steps to $T = \Theta(n)$ and set $\eta = \Theta(1)$. In this way, it matches the sample complexity upper bound $\bigO{1/n}$ under the regime of $T = \bigO{n}$.
% \jiaye{I am confused, does the lower bound hold for $\cO(n)$ or $\Theta(n)$?} \peiyuan{Jingzhao suggested using $\bigO{n}$ instead of $\Theta(n)$.}

The major difficulty in lower bound construction and our major novelty is the proof of term $\bigOmega{\eta T/n^2}$. Similar to the non-realizable setting, the term $\bigOmega{1/(\eta T)}$ reflects the optimization error. In the meanwhile, the term $\bigOmega{1/n}$ comes from a universal hard instance that holds for any deterministic or stochastic gradient methods. Notice that the term $\bigOmega{1/n}$ does not suggest the rest two terms are vacuous since they are hard in the sense of characterizing the relationship between $\eta$, $T$ and $n$. Therefore, we emphasize that our lower bound construction does not only provide the worst instance on sample complexity but also reflects possibly worst growth of generalization error along time. 

%Theorem~\ref{thm: lb-sgd-1} shows that under realizable smooth SCO, SGD enjoys the same hardship with GD. The lower bound matches the upper bound of Eq.~\ref{eq: ub-sgd-tn} in \cite{lei2020fine} up to a factor of $\eta$ in the second term. It implies a sample complexity lower bound when $T = \bigO{n}$: when we select $T = n$, $\eta = 1$, it achieves the optimal sample complexity $\bigOmega{1/n}$. This also matches the corresponding upper bound.

\subsection{Realizable: case \texorpdfstring{$T = \bigOmega{n}$}{T=Omega(n)}} \label{sec: t_larger_than_n}
In this subsection we focus on the case that allows large or infinite time horizon $T = \bigOmega{n}$. We provide lower bounds for different algorithms and discuss their relationship with upper bounds. This is stated in the following theorem.
\begin{theorem} \label{thm: lb-2}
For every $\eta > 0$, $T > 1$, if condition $T = \bigOmega{n}$ holds, then there exists a convex, $1$-smooth and realizable $f(w, z): \bbR^d \to \bbR$ for every $z \in \cZ$, and a distribution $D$ such that, with a bounded initialization $\| w_0 - w^* \| = \bigO{1}$, the output $w_{\gd}$ for GD satisfies
\begin{align*}
    \bbE[F(w_{\gd})] - F(w^*) = \bigOmega{\frac{1}{\eta T} + \frac{1}{n}}.
\end{align*}
Similarly, the output $w_{\sgd}$ for SGD satisfies
\begin{align*}
    \bbE[F(w_{\sgd})] - F(w^*) = \bigOmega{\frac{1}{\eta T} + \frac{1}{n}}.
\end{align*}
\end{theorem}
Again, we combine the discussion of GD and SGD because they have the same upper and lower bounds. Theorem~\ref{thm: lb-2} indicates that, different from the case $T = \bigO{n}$, our lower bound for both GD and SGD does not match the corresponding upper bounds in \cite{schliserman2022stability} (see Table~\ref{tab: summary}). And a gap exists between the sample complexity upper and lower bounds: for lower bound, we may achieve $\bigOmega{1/n}$ for any given $T$ when we select $\eta = n/T$; in contrast, we only obtain the rate $\bigO{1/n^{2/3}}$ from the upper bound  by setting $\eta = n^{2/3}/T$. 

To obtain the lower bound in Theorem~\ref{thm: lb-2} we employ a strategy similar to the proof of Theorem~\ref{thm: lb-1}. Term $\bigOmega{1/n}$ comes from the universal sample hardness for any algorithm, and term $\bigOmega{1/(\eta T)}$ is obtained from the construction used to prove $\bigOmega{\eta T/n}$ in Theorem~\ref{thm: lb-1}. The details are postponed to Appendix~\ref{appendix: lb-gd-t}.

We conjecture the sample complexity bound under a large or infinite time horizon can be closed by proving upper bound $\bigO{1/n}$ is achievable for GD. We will discuss the conjecture and provide several evidences in Section~\ref{sec: infinite}.% In the next theorem, we proceed to the lower bound for SGD.
%\begin{theorem} \label{thm: lb-sgd-2}
%For every $\eta > 0$, $T > 1$, there exists a convex, $1$-smooth and realizable $f(w, z): \bbR^d \to \bbR$ for every $z \in \cZ$, and a distribution $D$ such that, with a bounded initialization $\| w_0 - w^* \| = \bigO{1}$, the output $w_{\sgd}$ for SGD satisfies
%\begin{align*}
%    \bbE_{S\sim D^n}[F(w_{\sgd})] - F(w^*) = \bigOmega{\frac{1}{\eta T} + \frac{1}{n}}.
%\end{align*}
%Similarly, the output $w_{\sgd}$ for SGD satisfies
%\begin{align*}
%   \bbE_{S\sim D^n}[F(w_{\sgd})] - F(w^*) = \bigOmega{\frac{1}{\eta T} + \frac{1}{n}}.
%\end{align*}
%\end{theorem}
%Similar to GD, generalization error for SGD has a sample complexity lower bound of $\bigOmega{1/n}$ when we set $\eta = 1$, for any $T = \bigOmega{n}$. This does not match the upper bound $\bigO{1/n^{2/3}}$ from Eq.~\eqref{eq: ub-sgd-t}. We also conjecture the gap can be closed by proving upper bound $\bigO{1/n}$ for SGD, which we will discuss in next section. The proof to the theorem can be found in Appendix~\ref{appendix: t_larger_than_n}.


\section{Infinite Time Horizon Instances} \label{sec: infinite}
In Section~\ref{sec: lb}, we provide lower bounds for both realizable and non-realizable cases. For non-realizable losses, both the upper and lower bounds are consistent within different time horizon. The result differs for realizable cases: for small time horizon $T = \bigO{n}$, our lower bounds match the upper bound in sample complexity $\Theta(1/n)$; nevertheless, for large time horizon $T = \bigOmega{n}$, there exists a gap between the sample complexity lower bounds $\bigO{1/n}$ and upper bounds $\bigOmega{1/n}$. It is natural to ask
\begin{center}
    \it Can we close the gap between upper and lower bounds\\ for realizable SCO when $T$ goes to infinity?
\end{center}
We conjecture that the above problem can be tackled by proving GD and SGD can achieve $\bigO{1/n}$ even for large time horizon $T = \bigOmega{n}$. In the section, we provide evidences to support the conjecture: we consider the example of one-dimensional function and linear regression. On both examples, $\Theta(1/n)$ is achieved for GD and SGD when $T = \bigOmega{n}$.
\subsection{One-dimensional feasibility}
 We showcase the reasoning behind our conjecture by providing a first evidence in dimension one: under $d = 1$, we close the gap between upper and lower bound by establishing $\Theta(1/n)$ sample complexity in the rest part of the subsection.
\paragraph{Upper bounds.} We begin by presenting Lemma~\ref{lemma: dim-1-ub-sgd}, which establishes an upper bound for SGD based on the result of \citet{lei2020fine}.
\begin{lemma} \label{lemma: dim-1-ub-sgd}
In dimension one, if $f(w,z)$ is convex, $1$-smooth and realizable with $z \sim D$, then for every $\eta = \Theta(1)$, there exists $T_0 = \Theta(n)$ such that for $T \geq T_0$, the output $w_{\sgd}$ of SGD satisfies
\begin{equation*}
    \bbE[ F(w_{\sgd})] - F(w^*) = \bigO{\frac{1}{n}}.
\end{equation*} 
\end{lemma}
\begin{proof}
From Theorem~4 in \citet{lei2020fine}, it holds that for realizable cases (we rescale it to $f(w^*, z) = 0$ for each $z$) with step size $\eta =  \Theta(1)$, it holds that
\begin{equation}
    \bbE[F(w_{\sgd})] = \bigO{\frac{1}{T_0} + \frac{1 + T_0/n}{n}}.
\end{equation}
Therefore, for $T_0= \Theta(n)$, it holds that
\begin{equation*}
    \bbE[F(w_{\sgd})] = \bigO{\nfrac{1}{n}}.
\end{equation*}
For SGD, the iteration formulates the iterate 
\begin{equation*}
    w_{t+1} = w_t - \eta \nabla f(w_t, z_i).
\end{equation*}
Under the realizable and convex assumption, for any $z_i \in \cZ$, the iteration becomes
\begin{equation*}
    w_{t+1} - w^* = (1 - \eta \nabla^2 f(\xi, z_i)) (w_t - w^*),
\end{equation*}
using mean value theorem, where $\xi$ is a point between $w_t$ and $w^*$. This indicates that the distance $w_t - w^*$ shrinks in each step for any $z_i \in \cZ$. Due to the convexity of $F$, it holds that 
$F(w_{t+1}) \leq F(w_t)$. In summary, for any $T \geq T_0$, it holds that 
\begin{equation*}
    \bbE[F(w_{\sgd})] - F(w^*) = \bigO{\nfrac{1}{n}}.
\end{equation*} 
\end{proof}
A similar result can be established for GD, as in the next lemma. Its proof is postponed to Appendix~\ref{appendix: infinite}.
\begin{lemma} \label{lemma: dim-1-ub-gd}
In dimension one, if $f(w,z)$ is convex, $1$-smooth and realizable with $z \sim D$, then for every $\eta = \Theta(1)$, there exists $T_0 = \Theta(n)$ such that for $T \geq T_0$, the output $w_{\gd}$ of GD satisfies
\begin{equation*}
    \bbE[ F(w_{\gd})] - F(w^*) = \bigO{\frac{1}{n}}.
\end{equation*} 
\end{lemma}
\paragraph{Lower bound.}
Besides upper bound, we present a hard instance in dimension one case, which leads to sample complexity lower bound of $\bigOmega{1/n}$.
\begin{lemma} \label{lemma: dim-1-lb}
    In dimension one, for every $\eta > 0$, $T \geq 1$, there exists a convex, $1$-smooth and realizable $f(w, z): \bbR \to \bbR$ for every $z \in \cZ$, and a distribution $D$ such that, %with high probability $\Theta(1)$,
    it holds that for the output of any gradient-based algorithm $\cA[S]$
    \begin{align*}
        \bbE[F(\cA[S])] - F(w^*) = \bigOmega{\nfrac{1}{n}}.  
    \end{align*}  
\end{lemma}
\begin{proof}
    We consider the following instance
    \begin{equation*}
        f(w, z) = \begin{cases} \frac{1}{2}w^2, \qquad z = 1, \\0, \qquad\quad\ z = 0, \end{cases}
    \end{equation*}
    where $z \sim \text{Bern}(1/n)$. The population risk is
    \begin{align*}
        F(w) = \bbE_{z\sim\text{Bern}(1/n)}[f(w,z)] = \frac{1}{2n} w^2,
    \end{align*}
    which achieves minimum at $w^* = 0$. It is easy to reckon that $f(w,z)$ is smooth, convex and realizable. Now consider any dataset $S$ of $n$ samples. With probability
    $ \left(1 - \nfrac{1}{n}\right)^n = \Theta(1)$, dataset $S$ of size $n$ does not observe stochastic function $f(w,z=1) = w^2/2$. So with initialization $w_0 = 1$, $w_t$ remains unchanged for any gradient-base algorithm and any $t \in [T]$. Then we have the following lower bound:
    \begin{align*}
        \bbE[F(\cA[S])] - F(w^*) =  \frac{1}{2n} = \bigOmega{\frac{1}{n}},
    \end{align*}
    which is the desired result.
\end{proof}
The lemma holds for any gradient-based algorithm and provides a sample complexity lower bound of $\bigOmega{1/n}$ for any $T \geq 1$. This matches the upper bound of GD and SGD in Lemma~\ref{lemma: dim-1-ub-gd} and Lemma~\ref{lemma: dim-1-ub-sgd} under large time horizon regime $T = \bigOmega{1/n}$. Unfortunately, we cannot employ the same technique to extend the result to high dimensional case. However, we show that the gap can be closed for a special case in the high-dimensional regime in next subsection.

\subsection{Linear Regression}
In this subsection, we demonstrate that when $T = \bigOmega{n}$, $\Theta(1/n)$ can be achieved on \emph{linear regression} problem, whatever underparameterized ($ d < n$) or overparameterized ($ d \geq n$). 
In realizable linear regression problems, the $i$-th sample $z_i = (x_i, y_i)$ in dataset $S= \{z_1, \dots, z_n \}$ satisfies that $y_i = x_i^\top w^*$ and $x_i$ is i.i.d. drawn from an unknown distribution. Under the linear predictor $x_i^\top w$, the loss term is defined as $f(w,z) = (y_i - x_i^\top w)^2$. Under this regime, a bounded feature $\|x\| \leq 1$ suffices to guarantee that $f(w, z)$ is convex, $1$-smooth, and realizable. 
In this case, the upper bound would be $\bigO{1/n}$ and the lower bound would be ${\cO}(\log^3n/n)$, which is optimal up to a $\log$-factor. 

\paragraph{Upper bounds.}
We begin by presenting Lemma~\ref{lemma: ub-regression}, which establishes an upper bound using local Rademacher Complexity.
\begin{lemma}[From \citet{srebro2010optimistic}]
\label{lemma: ub-regression}
    In the realizable linear regression cases, for every $\eta > 0$ and $T\geq 1$, if the feature $x_i$ is bounded, it holds that for the output of SGD
\begin{align*}
    & \bbE[F(w_{\sgd})] - F(w^*) = \bigO{\frac{1}{\eta T} + \frac{\log^3 n}{n} }, 
    \end{align*}
    and also the output for GD
    \begin{align*}
    & \bbE[F(w_{\gd})] - F(w^*) = \bigO{\frac{1}{\eta T} + \frac{\log^3 n}{n}}.
\end{align*}
\end{lemma}
\begin{proof}
    One could directly apply Theorem~1 in \citep{srebro2010optimistic}.
    Specifically, we plug in the realizability assumption and the Rademacher complexity of linear function class, which is in order $\bigO{1/n}$ in bounded norm cases. 
\end{proof}
%One can prove similar results for GD in Lemma~\ref{lem: upper bound for linear regression GD}.\jiaye{Do we need to merge the two lemmas?}
%\begin{lemma}[From \citet{srebro2010optimistic}]
%\label{lem: upper bound for linear regression GD}
%    For realizable linear regression cases, if the feature $x$ is bounded, then for every $\eta$, it holds that 
%\begin{equation*}
%\end{equation*}
%\end{lemma}
\paragraph{Lower bounds.}
We now proceed to a corresponding lower bound for regressional problems.
\begin{lemma}[Lower bound in linear regression.]
\label{lemma: lb-regression}
    For any $\eta > 0$, $T \geq 1$, there exists a linear regressional instance such that, if the feature $x_i \sim D$ is bounded, it holds that for the output of any gradient-based algorithm $\cA[S]$
\begin{equation*}
    \bbE[F(\cA[S])] - F(w^*) = \bigOmega{\frac{1}{n}}.
\end{equation*}
\end{lemma}
\begin{proof}
    We consider the following regression problem: we generate $x_i = e_i \in \bbR^{2n}$, where $i \sim \text{Unif}([2n])$. The ground truth weight is set to $w^* = 0$ and $y_i = x_i^\top w^*$. Hence, the loss function $f(w, z): \bbR^{2n} \times \cZ \to \bbR$ is 
    \begin{align*}
        f(w, z = i) = w(i)^2, \qquad z \sim \text{Unif}([2n]).
    \end{align*}
    From Lemma~\ref{lemma: lb-sample} in Appendix~\ref{appendix: lb-sample}, we know that for any gradient-based algorithm, the lower bound is $\bigOmega{1/n}$.
\end{proof}
Lemma~\ref{lemma: ub-regression} and Lemma~\ref{lemma: lb-regression} establish a sample complexity rate of $\Theta(1/n)$ for linear regression when $T$ grows large. Our evidence on both dimension one case and regression suggests the gap in the regime $T = \bigOmega{n}$ might be closed by improving the upper bounds of excess risk or sample complexity. We hope our analysis can motivate future exploration into the topic.
\section{Proof Overviews} \label{sec: proof}
In this section we provide a brief overview regarding our technique used in the proofs of theorems for realizable cases in Section~\ref{sec: lb}. In particular, we will focus on the lower bound construction for the output of GD when $\eta T = \bigO{n}$, i.e. first part in Theorem~\ref{thm: lb-1}, to showcase the major intuition and idea behind our constructions. 

As discussed in the above section, the main technical difficulty in the GD part of Theorem~\ref{thm: lb-1} lies in proving
\begin{equation} \label{eq: lb-major}
    \bbE[F(w_{\gd})] - F(w^*) = \bigOmega{\frac{\eta T}{n^2}}.
\end{equation}
The proof of rest terms is based on easier constructions and we recommend referring to Lemma~\ref{lemma: lb-sample} and Lemma~\ref{lemma: lb-suboptimality} in Appendix~\ref{appendix: minor}. These lemmas are general and hold for any deterministic or stochastic gradient methods. Here we focus on the proof of \eqref{eq: lb-major}. Our technique is novel and inspired by the work of \citet{amir2021sgd, sekhari2021sgd}. However, their construction critically relies on nonsmoothness and nonrealizability. 

We start by considering running GD on the following 2-dimensional quadratic function
\begin{equation*}
    h(x, y) = \frac{\alpha x^2}{2} + \frac{y^2}{2} - 2\sqrt{\alpha} xy = \frac{1}{2} \left| \sqrt{\alpha} x -  y \right|^2
\end{equation*}
with step-size $\eta$ and initialization $x_1 = 1$ and $y_1 = 0$. We choose a small enough $\alpha = \nfrac{1}{\eta T} \ll 1$. In every iteration, since $\alpha$ is small, $x$ is pulled back to zero slowly: it is easy to lower bound the value since 
\begin{equation*}
    x_{t+1} \geq (1 - \alpha\eta) x_{t} \geq e^{-\alpha \eta t} x_1  \geq e^{-t/T} x_1 \geq 1/e = \Theta(1).
\end{equation*}
Hence $x_t = \bigOmega{1}$ for any $t \in [T]$. Meanwhile, 
coordinate $y$ is simultaneously (1) pushed away from zero by $x$ on the scale of $\bigOmega{\eta\sqrt{\alpha}}$ and (2) pulled back towards zero by itself. As a result, despite the pulling influence, we can still guarantee that $y_t$ is bounded away from zero for all $t \in [T]$.

We now want to improve over the naive two-dimensional quadratic example to make sure that multiple coordinates are bounded away from zero. This intuitively might provide a hard instance for the GD algorithm. Also, we hope stochasticity plays a role in the hard instance such that we can introduce the factor of $n$.  We then devise the following instance $g(w, z): \bbR^{n+1} \times \cZ \to \bbR$ belonging to the realizable smooth SCO setting:  
\begin{equation}
    g(w, z = i) = \frac{\alpha}{2} x^2 + \frac{1}{2} \big(y(i)\big)^2  - \sqrt{\alpha} x \cdot y(i) = \frac{1}{2} \left|\sqrt{\alpha}x - y(i)\right|^2
\end{equation}
where $w = (x,y)$, $x \in \bbR$, $y \in \bbR^n$ and $z \sim \text{Unif}([n])$.  We still set parameter $\alpha$ to be $1/(\eta T)$. We are given a dataset $S$ of $n$ examples i.i.d. from the distribution. This leads to population loss
\begin{equation}
    G(w) = \bbE_{z\sim\text{Unif}([n])}[g(w,z)] =  \frac{1}{2n} \left\| y - \sqrt{\alpha} x \right\|^2.
\end{equation}

We generalize the idea from the two-dimensional case to $n+1$ dimension. To this end, we need every example $z_i \in S$ corresponds to one coordinate $y(i)$. This is, however, an improbable event that occurs with probability $\Theta(\sqrt{n} \cdot e^{-n})$. We use the intuition from \citet{amir2021sgd,sekhari2021sgd}: if we consider multiple independent copies of $g(w,z)$, then with probability $\Theta(1)$, there exists at least one copy that satisfies the condition.

We focus on the particular copy only. Our calculations shows that under assumption $\eta T = \bigO{n}$, it holds that, for any $t \in [T]$, (1) $x_t = \Theta(1)$ and (2) $y_t(i) = \bigOmega{\sqrt{\eta t}/n}$ for any coordinates $i \in [n]$. With a slight abuse of the notation $w$, we put everything together and guarantee that 
\begin{align*}
    F(w_{\gd}) - F(w^*) = \bigOmega{\frac{1}{2n} \cdot \| y_T \|^2} = \bigOmega{\frac{\eta T}{n^2}}.
\end{align*}

The details in the proof of Theorem~\ref{thm: lb-1} can be found in Appendix~\ref{appendix: lb-gd-tn}. The idea behind the proof for the case $\eta T = \bigOmega{n}$ (Theorem~\ref{thm: lb-2}) differs only in calculations and hence we omit the repetition. Proof for SGD is also similar. The details of proof for other theorems can be found in Appendix~\ref{appendix: t_equal_n}.
\section{Related Work}
Convex optimization has a rich history and its generalization error has been extensively explored in the literature~\citep{boyd2004convex,shalev2014understanding}, with one-pass SGD~\citep{pillaud2018exponential}, multi-pass SGD~\citep{pillaud2018statistical,sekhari2021sgd,lei2021generalization}, DP-SGD~\citep{bassily2019private,ma2022dimension}, ERM solution~\citep{feldman2016generalization,aubin2020generalization} and so on. 
One of the most famous results is that one-pass SGD can achieve an optimal error rate of $\cO(1/\sqrt{n})$ in convex optimization, even in the presence of non-smooth loss functions~\citep{nemirovskij1983problem}.

However, in the case of convex regimes with realizability constraints, existing analyses typically focus only on upper bounds and lack corresponding lower bounds~\citep{lei2020fine,nikolakakis2022beyond,schliserman2022stability,taheri2023generalization}. 
Realizability is closely related to label noise, which can have a substantial impact on generalization performance~\citep{song2019does,harutyunyan2020improving,DBLP:conf/iclr/TengMY22,wen2022realistic}. 

From a perspective of lower bounds,
\cite{amir2021sgd} show that no less than $\bigOmega{1/\epsilon^4}$ steps is needed to generalize for GD. \cite{sekhari2021sgd} further indicate that GD suffers from a $\bigOmega{1/n^{5/12}}$ sample complexity, which is slower than the well-established bound $\bigOmega{1/\sqrt{n}}$ for SGD \citep{nemirovskij1983problem}.
Besides the upper/lower bound mentioned above, a line of lower bounds in generalization analysis typically focuses on the failure of techniques. 
For instance, despite the optimal rate of $\cO(1/\sqrt{n})$ in convex optimization, uniform convergence only returns a lower bound of $\Omega(\sqrt{d/n})$~\citep{shalev2010learnability,feldman2016generalization}, leading to a constant lower bound in overparameterized regimes. 
A line of works further illustrate the inherent weakness of uniform convergence~\citep{nagarajan2019uniform,glasgow2022max}.
Regarding stability-based bounds, \citet{bassily2020stability} presents a lower bound under non-smooth convex losses. 


To bridge the gap between lower and upper bound, a fast rate upper bound in order $O(1/n)$ is required. 
One of the most well-known fast-rate bound is local Rademacher complexity, which works well under low-noise regimes~\citep{bartlett2005local}. 
However, it typically relies on a specific function class and may not be directly applied into the general convex optimization regimes~\citep{steinwart2007fast,srebro2010optimistic,zhou2021optimistic}. 
Alternatively, stability-based analyses have shown promise and work well in convex optimization regimes, which provide fast-rate generelization bound~\citep{bousquet2002stability,hardt2016train,feldman2019high,zhang2022stability}.
In addition to these bounds, one can also derive fast rate bound for finite-dimensional cases~\citep{lee1996importance,bousquet2002concentration}, aggregation~\citep{tsybakov2004optimal,chesneau2009adapting,dalalyan2018exponentially}, PAC-Bayesian and information-based analysis~\citep{yang2019fast,grunwald2021pac}.



% To close the gap between upper and lower bound in realizable regimes, a fast rate upper bound is required.
% Local Rademacher complexity is among the most famous fast-rate bound, which \citep{}. 
% However, the local Rademacher complexity usually relies on 
% Recently, stability stands out due to 

% It is known that one-pass SGD can attain the optimal error $\cO(1/\sqrt{n})$ in convex optimization,  with smoothness~\citep{nemirovskij1983problem}. 
% Researchers also focus on the multi-pass SGD regimes using 
% Under convex optimization problems, \citet argues that $O(1/\sqrt{n})$ convergence rate is optimal for one-pass SGD \jiaye{I did not check the details of this paper because it is hard to download, it is in paper \url{https://proceedings.neurips.cc/paper/2020/file/2e2c4bf7ceaa4712a72dd5ee136dc9a8-Paper.pdf}}. 
% \jiaye{It might be in non-smooth convex regimes. }
% For multi-pass SGD, Hardit et al, any new results?



% \textbf{Other types of lower bounds.}
% A line of works makes progress on the generalization lower bounds under different regimes. 
% For uniform-convergence bound, a lower bound of $\Omega(\sqrt{d/n})$ is derived in  SCO settings~\citep{DBLP:journals/jmlr/Shalev-ShwartzSSS10,feldman2016generalization}, leading to a constant lower bound in overparameterized regimes.
% Besides, \citet{} further show the inherent weakness of uniform convergence. 
% For stability-based bounds, \citet{bassily2020stability} presents the lower bound under non-smooth convex losses. 
% Different from these approaches, we derive problem-intrinsic lower bounds which do not rely on any techniques. 

% \textbf{Other types of lower bounds.}
% Analyzing lower bounds helps understand generalization. 
% In convex settings, stability and uniform convergence both fails.
% In general settings, there are still lower bounds.

% \textbf{Fast-rate upper bounds.}
% Since our ultimate goal is to close the gap between upper and lower bound, a fast rate upper bound is required.
% Local Rademacher complexity is among the most famous fast-rate bound, which \citep{}. 
% However, the local Rademacher complexity usually relies on some specific function class and therefore cannot directly apply into the SCO regimes. 
% Recently, stability stands out due to 

% Under convex training regimes, the stability-based bound usually 

% \textbf{Realizable settings.}
% Despite the equivalence of realizable and agnostic learnability under the PAC-learning framework~\citep{vapnik1974theory,DBLP:journals/jacm/BlumerEHW89,DBLP:journals/iandc/Haussler92,DBLP:conf/colt/HopkinsKLM22},
% researchers find that the realistic generalization performances can be pretty different where the label noise matters~\citep{}.
% The key to the gap comes from 


\section{Conclusion}
In this work, we focus on generalization bounds under the smooth SCO setting. In particular, we investigate the relationship between sample complexity and iteration $T$ under three settings: (1) non-realizable, (2) realizable with $T = \bigO{n}$, and (3) realizable with $T =  \bigOmega{n}$. We provide novel excess population risk lower bounds for Gradient Descent and Stochastic Gradient Descent methods in all the three cases. For the first two cases, our lower bounds match the corresponding upper bounds and certificate the optimal sample complexity. Nevertheless, under the realizable case with $T = \bigO{n}$, we observe a gap between existing sample complexity upper bounds $\bigO{1/n^{2/3}}$ and lower bounds $\bigOmega{1/n}$ derived from our results. We conjecture that this gap can be closed by improving the upper bound under the long time horizon regime and provide evidence in the particular one-dimensional and regressional setting to support our hypothesis.

%\acks{All acknowledgements go at the end of the paper before appendices and references. Moreover, you are required to declare funding (financial activities supporting the submitted work) and competing interests (related financial activities outside the submitted work). More information about this disclosure can be found on the JMLR website.}

% Manual newpage inserted to improve layout of sample file - not
% needed in general before appendices/bibliography.


%\input{texts/appendix_4(upper bound 1-dim)}

% Note: in this sample, the section number is hard-coded in. Following
% proper LaTeX conventions, it should properly be coded as a reference:

%In this appendix we prove the following theorem from
%Section~\ref{sec:textree-generalization}:

\newpage

\appendix

\section{Missing Proofs from Section~\ref{sec: nonrealizable}} \label{appendix: nonrealizable}
\subsection{Proof of Theorem~\ref{thm: lb-sco-gd}} \label{appendix: lb-sco-gd}
The theorem provides an excess risk lower bound $\bigOmega{\nfrac{1}{\eta T} + \nfrac{\eta T}{n}}$ for GD under the \emph{non-realizable} smooth SCO scenario. The result is obtained by combining a $\bigOmega{1/{\eta T}}$ bound in Lemma~\ref{lemma: lb-suboptimality} and a $\bigOmega{\eta T/n}$ bound in Lemma~\ref{lemma: lb-gd-overfit} stated below. The first bound reflects an optimization error and is postponed to Appendix~\ref{appendix: lb-suboptimality}. In the rest part, we present the proof of the latter lemma.
\begin{lemma} \label{lemma: lb-gd-overfit}
    For any $\eta > 0$, $T > 1$, there exists a convex, $1$-smooth $f(w,z): \bbR \to \bbR$ for every $z \in \cZ$, and a distribution $D$ such that, with probability $\Theta(1)$, the output $w_{\gd}$ for GD satisfies
    \begin{align*}
        F(w_\gd) - F(w^*) = \bigOmega{\frac{\eta T}{n}}.
    \end{align*}
\end{lemma}

\begin{proof}
We define loss function $f: \bbR \times \cZ \to \bbR$ as
\begin{align*}
    f(w, z) = \frac{w^2}{2\eta T}  + zw
\end{align*}
where $z \sim \text{Unif}(\{ \pm 1\})$. It is obvious that $f(w,z)$ is $1$-smooth and convex since $\eta T \geq 1$. The population risk is computed as
\begin{align*}
    F(w) = \bbE_{z\sim\text{Unif}(\{\pm 1\})}[f(w, z)] = \frac{w^2}{2\eta T}.
\end{align*}
The minimizer is then $w^* = 0$. GD formulates the following recurrence on dataset $S$ with initialization $w_1=0$:
\begin{align*}
    w_{t+1} = w_t - \frac{\eta}{n} \sum_{i=1}^n\left( \frac{w_t}{\eta T} + z_i \right) = \left(1-\frac{1}{T} \right) w_t - \frac{\eta}{n}\sum_{i=1}^nz_i,
\end{align*}
where each $z_i \sim \text{Unif}(\{\pm 1\})$ for $i \in [n]$. We want to use an anti-concentration result to lower bound the recurrence: from Lemma 7 in \citet{sekhari2021sgd}, with probability $\bigOmega{1}$, it holds that
\begin{align*}
    \sum_{i=1}^n z_i \leq - \frac{\sqrt{n}}{2}.
\end{align*}
We get lower bound
\begin{align*}
    w_{t+1} \geq  \left(1-\frac{1}{T} \right) w_t + \frac{\eta}{2\sqrt{n}}.
\end{align*}
Then we have for any $t \in [T]$
\begin{align*}
    w_t & \geq \frac{\eta}{2\sqrt{n}} \left( 1 + \left( 1 - \frac{1}{T}\right) + \cdots + \left( 1 - \frac{1}{T}\right)^{t-1} \right) \\
    & \geq \frac{\eta t}{8\sqrt{n}}
\end{align*}
where the second inequality is due to the fact 
\begin{align*}
    1 > 1 - \frac{1}{T} > \cdots  \left( 1 - \frac{1}{T}\right)^t > \cdots > \left( 1 - \frac{1}{T}\right)^{T} \geq \frac{1}{4}
\end{align*}
for any $t \in [T]$ and $T \geq 2$. Then the average is lower bounded as
\begin{align*}
    \bar{w}_T = \frac{1}{T} \sum_{t=1}^T w_t \geq \sum_{t=1}^T \frac{\eta t}{8\sqrt{n}} = \frac{\eta (T-1)}{16\sqrt{n}}. 
\end{align*}
As a result, we have
\begin{align*}
    F(w_{\gd}) - F(w^*) = \frac{w_{\gd}^2}{2\eta T}  = \bigOmega{\frac{\eta T}{n}},
\end{align*}
which is the desired result.
\end{proof}
%Combined with the suboptimality lower bound in Lemma~\ref{lemma: lb-suboptimality}, we have lower bound $$ \bbE[F(w_{\gd})] - F(w^*) = \bigOmega{\frac{1}{\eta T} + \frac{\eta T}{n}}.$$

\subsection{Proof of Theorem~\ref{thm: lb-sco-sgd}}
% We consider this to be a forklore knowledge. Nevertheless, we fail to find any reference to such a result and present a short proof here. We first give a lower bound $\bigOmega{\eta T/n}$ by the following construction.
Similar to the proof of Theorem~\ref{thm: lb-sco-gd}, we prove the excess risk lower bound for SGD by combining Lemma~\ref{lemma: lb-suboptimality} and the following lemma.
\begin{lemma} \label{lemma: lb-sgd-overfit}
    For any $\eta > 0$, $T > 1$,  there exists a convex, $1$-smooth $f(w,z): \bbR \to \bbR$ for every $z \in \cZ$, and a distribution $D$ such that, with probability $\Theta(1)$, the output $w_{\sgd}$ for SGD satisfies
    \begin{align*}
        \bbE[F(w_{\sgd})] - F(w^*) = \bigOmega{\frac{\eta T}{n}}.
    \end{align*}
\end{lemma}

\begin{proof}
We use the same construction in Lemma~\ref{lemma: lb-gd-overfit}. Consider dataset $S=\{z_1,\dots,z_n\}$ where $z_i \sim \text{Bern}(\{\pm 1\})$. Given $x_t$, SGD formulates the following recurrence on dataset $S$ with initialization $w_1=0$:
\begin{align*}
    \bbE[w_{t+1}] = w_t - \frac{\eta}{n} \sum_{i=1}^n\left( \frac{w_t}{\eta T} + z_i \right) = \left(1-\frac{1}{T} \right) w_t - \frac{\eta}{n}\sum_{i=1}^nz_i,
\end{align*}
where $z_i \sim \text{Unif}(\{\pm 1\})$. From Lemma 7 in \citet{sekhari2021sgd}, with probability $\bigOmega{1}$, it holds that
\begin{align*}
    \sum_{i=1}^n z_i \leq - \frac{\sqrt{n}}{2}.
\end{align*}
Then we get lower bound
\begin{align*}
    \bbE[w_{t+1}] \geq  \left(1-\frac{1}{T} \right) w_t + \frac{\eta}{2\sqrt{n}}.
\end{align*}
Similar to the proof of Lemma~\ref{lemma: lb-gd-overfit}, we have for any $t \in [T]$
\begin{align*}
    \bbE[w_t] & \geq \frac{\eta}{2\sqrt{n}} \left( 1 + \left( 1 - \frac{1}{T}\right) + \cdots + \left( 1 - \frac{1}{T}\right)^{t-1} \right) \\
    & \geq  \frac{\eta t}{8\sqrt{n}}.
\end{align*}
Then the average is lower bounded as
\begin{align*}
    \bbE[\bar{w}_T] = \frac{1}{T} \sum_{t=1}^T \bbE[w_t] \geq \sum_{t=1}^T \frac{\eta t}{8\sqrt{n}} = \frac{\eta (T-1)}{16\sqrt{n}}. 
\end{align*}
As a result, we have
\begin{align*}
    \bbE[F(w_{\sgd})] - F(w^*) \geq F(\bbE[w_{\sgd}]) - F(w^*) = \frac{(\bbE[w_{\sgd}])^2}{2\eta T}  = \bigOmega{\frac{\eta T}{n}},
\end{align*}
by Jensen's inequality.
\end{proof}



\section{Missing proofs from Section~\ref{sec: t_equal_n} and Section~\ref{sec: t_larger_than_n} } \label{appendix: t_equal_n}

\subsection{Proof of Theorem~\ref{thm: lb-1}} \label{appendix: lb-gd-tn}
The proof of GD is immediate from combining the lower bound constructions in Lemma~\ref{lemma: lb-sample}, Lemma~\ref{lemma: lb-suboptimality}, and most importantly, Lemma~\ref{lemma: lb-gd-tn} to be stated below. In precise, Lemma~\ref{lemma: lb-gd-tn} is the core part of our result and gives a lower bound of $\bigOmega{\eta T/n}$ when $T = \bigO{n}$. We postpone its proof to Appenidx~\ref{appendix: lb-gd}. The proof of the rest two lemmas can be found in Appendix~\ref{appendix: lb-sample}, \ref{appendix: lb-suboptimality}.
\begin{lemma} \label{lemma: lb-gd-tn}
For every $\eta > 0$, $T \geq 1$, if $T =  \bigO{n}$, then there exists a convex, $1$-smooth and realizable $f(w, z): \bbR^d \to \bbR$ for every $z \in \cZ$, and a distribution $D$ such that, with initialization $\| w_0 - w^* \| = \bigO{1}$, the output $w_{\gd}$ for GD satisfies
\begin{align*}
 \bbE[F(w_{\gd})] - F(w^*) = \bigOmega{\frac{\eta T}{n^2}}.  
\end{align*}
Similarly, if $T =  \bigOmega{n}$, then 
the output $w_{\gd}$ for GD satisfies
\begin{align*}
 \bbE[F(w_{\gd})] - F(w^*) = \bigOmega{\frac{1}{\eta T}}.  
\end{align*}
\end{lemma}
Similar to the proof of GD, the result on SGD is also obtained by combining lower bound construction in following lemma and Lemma~\ref{lemma: lb-sample}, \ref{lemma: lb-suboptimality} in Appendix~\ref{appendix: lb-sample}, \ref{appendix: lb-suboptimality}. Lemma~\ref{lemma: lb-sgd-tn} establishes a lower bound of $\bigOmega{\eta T/n}$ when $T = \bigO{n}$. Its proof can be found in Appendix~\ref{appendix: lb-sgd}.
\begin{lemma} \label{lemma: lb-sgd-tn}
For every $\eta > 0$, $T \geq 1$, if $T =  \bigO{n}$, then there exists a convex, $2$-smooth and realizable $f(w, z): \bbR^d \to \bbR$ for every $z \in \cZ$, and a distribution $D$ such that, with initialization $\| w_0 - w^* \| = \bigO{1}$, the output $w_{\sgd}$ for SGD satisfies
\begin{align*}
 \bbE[F(w_{\sgd})] - F(w^*) = \bigOmega{\frac{\eta T}{n^2}}.  
\end{align*}
Similarly, if $T =  \bigOmega{n}$, then 
the output $w_{\sgd}$ for SGD satisfies
\begin{align*}
 \bbE[F(w_{\sgd})] - F(w^*) = \bigOmega{\frac{1}{\eta T}}.  
\end{align*}
\end{lemma}


\subsection{Proof of Theorem~\ref{thm: lb-2}} \label{appendix: lb-gd-t}
Similar to the proof of Theorem~\ref{thm: lb-1}, the proof of GD is immediate from combining the lower bound constructions in Lemma~\ref{lemma: lb-sample}, Lemma~\ref{lemma: lb-suboptimality}, and Lemma~\ref{lemma: lb-gd-tn}. In particular, Lemma~\ref{lemma: lb-gd-tn} gives a lower bound of $\bigOmega{1/{\eta T}}$ when $T = \bigOmega{n}$. 

Concurrently, the SGD proof is obtained from combining the lower bounds in Lemma~\ref{lemma: lb-sample}, Lemma~\ref{lemma: lb-suboptimality}, and Lemma~\ref{lemma: lb-sgd-tn}. In particular, Lemma~\ref{lemma: lb-sgd-tn} gives a lower bound of $\bigOmega{1/{\eta T}}$ when $T = \bigOmega{n}$. 

\subsection{Proof of Lemma~\ref{lemma: lb-gd-tn}} \label{appendix: lb-gd}
\begin{lemma}[Restated Lemma~\ref{lemma: lb-gd-tn}]
For every $\eta > 0$, $T \geq 1$, if $T =  \bigO{n}$, then there exists a convex, $1$-smooth and realizable $f(w, z): \bbR^d \to \bbR$ for every $z \in \cZ$, and a distribution $D$ such that, with initialization $\| w_0 - w^* \| = \bigO{1}$, the output $w_{\gd}$ for GD satisfies
\begin{align*}
 \bbE[F(w_{\gd})] - F(w^*) = \bigOmega{\frac{\eta T}{n^2}}.  
\end{align*}
Similarly, if $T =  \bigOmega{n}$, then 
the output $w_{\gd}$ for GD satisfies
\begin{align*}
 \bbE[F(w_{\gd})] - F(w^*) = \bigOmega{\frac{1}{\eta T}}.  
\end{align*}
\end{lemma}

\begin{proof}
    The hard instance $f(w,z)$ employed in the proof is constructed by \emph{glueing} parallelly in dimension multiple copies of basic instance $g(w,z): \bbR^{n+1} \times \cZ \to \bbR$. So we start with the construction of $g(w, z)$ and slightly abuse the notation of $w, z$ to denote the parameter and random variable of $g$. $w$ is defined as a tuple of two variables, i.e. $w = (x, y)$, where $x \in \bbR$ and $y \in \bbR^n$. Instance $g: \bbR^{w+1} \times \cZ \to \bbR$ is defined as
    \begin{equation} \label{eq: lb-instance}
        g(w, z = i) = \frac{\alpha}{2} x^2 + \frac{1}{2} \big(y(i)\big)^2  - \sqrt{\alpha} x \cdot y(i) = \frac{1}{2} \Big(\sqrt{\alpha}x - y(i)\Big)^2
    \end{equation}
    where $y(i)$ is the $i$-th coordinate of $y$ and $z \sim \text{Unif}([n])$.  Parameter $\alpha$ is set to $C/(\eta T)$ where $C$ is a constant. When condition $\eta T = \Omega(1)$ holds, it is easy to check that $g(w,z)$ is $\bigO{1}$-smooth and convex. The population risk $G$ is 
    \begin{align*}
        G(w) = \bbE_{z\sim\text{Unif}([n])}[g(w,z)] = \frac{\alpha}{2} x^2 + \frac{1}{2n} \|y\|^2 - \frac{\sqrt{\alpha}}{n}x \cdot \vone^\top y = \frac{1}{2n} \left\| y - \sqrt{\alpha} x \cdot \vone \right\|^2,
    \end{align*}
    which attains minimum at $(x^*, y^*) = (0, 0)$. So $g(w,z)$ satisfies the realizable condition.
    
    Before we proceed to describe the method of obtaining $f(w,z)$, we define the following probability event: given dataset $S = \{ z_1, \dots, z_n\}$, for one copy of $g(w, z)$, event $\cE$ is defined as 
    \begin{equation*}
        \cE = \left\{ z_i = i \text{ for any } \vpi(i) \in [n] \right\}
    \end{equation*}
    where $\vpi:[n] \to [n]$ is any permutation on $[n]$. Intuitively, when $\cE$ happens, in dataset $S$, each coordinates of $y$ is selected only for once. The probability of $\cE$ is calculated from the without-replacement sampling: 
    \begin{align*}
        p = \pr[\text{Event } \cE_1 \text{ happens}] = 1 \cdot \frac{n-1}{n} \cdots \cdot  \frac{1}{n} = \frac{n!}{n^n} = \Theta(\sqrt{n}\cdot e^{-n})
    \end{align*}
    where the last step is from Stirling approximation.

    Clearly for dataset $S$ and single copy $g(w,z)$, the probability of  events $\cE$ is small. Nevertheless, if we consider $m = \Theta(1/p)$ copies of such $g(w,z)$, then with high probability there exists at least one copy such event happen. Denote $R$ to be the random variable counting such copies out of total $m$ copies. Using second moment method, we upper bound the following probability: 
    \begin{equation}
        \pr[R > 0] \geq \frac{(\bbE[R])^2}{\bbE[R^2]} = \frac{m^2p^2}{mp(1-p)} \geq \frac{mp}{2} = \frac{1}{2}.
    \end{equation}

    This allows us to define $f: \bbR^{(n+1) \times m} \times \cZ^m \to \bbR$ as the sum of $m$ independent copies of $g(w,z)$:
    \begin{equation}
        f(w, z) = \sum_{i=1}^m g(w^{(i)}, z^{(i)}),
    \end{equation}
    where $w = ( w^{(1)}, \cdots, w^{(m)})$, $z = ( z^{(1)}, \cdots, z^{(m)})$, and each $w^{(i)} \in \bbR^{n+1}$, $z^{(j)} \sim \text{Unif}([n])$. This is to say, we define $f$ as the summation over $m$ copies of $g(w^{(j)},z^{(j)})$ and the variable $w^{(j)}$ are put parallel in coordinates to form a large vector $w$. The total dimension of $f$ is then $\Theta(\sqrt{n} \cdot e^{n} )$ and it is easy to check $f$ is still $1$-smooth, convex and realizable. Then with probability at least $\Omega\left(\nfrac{1}{2}\right)$, there exists $j \in [m]$ such that $\cE$ happens for $j$-copy. Without the loss of generalization, we set $\vpi(i) = i$ for any $i \in [n]$. This suggests in dataset $S$, we have
    \begin{equation}
        g(w^{(j)}, z^{(j)}_i) = \frac{\alpha}{2} (x^{(j)})^2 + \frac{1}{2} \|y^{(j)}\|^2 - \frac{x^{(j)}\sqrt{\alpha}}{n} y^{(j)}(i), \qquad \forall i \in [n].
    \end{equation}

    Next, we consider GD trajectory on the instance $f$ and dataset $S$, with the following initialization 
    \begin{align*}
        x_0^{(j)} = 1, \quad x_0^{(i)} = 0, \quad \forall i \ne j \in [m], \quad \text{and} \quad y_0^{(i)} = 0, \quad \forall i \in [m]. 
    \end{align*}
    The initialization satisfies $\| w_0 - w^* \| = \bigO{1}$. This allows us to focus on the $j$-th copy, i.e. $w^{(j)}$ and $z^{(j)}$. For simplicity, we suppress the upscript of $j$. From the above construction, GD formulates the following update on $j$-th copy:
    \begin{align*}
        w_{t+1} = w_t - \frac{\eta}{n} \sum_{i=1}^n \nabla_w g(w_t, z_i)
    \end{align*}
    with initialization $x_0 = 1$, $y_0 = 0$. The stochastic gradient is computed as 
    \begin{align}
        \nabla_x g(w, z_i) = \alpha x - \sqrt{\alpha}y(i), \qquad \nabla_{y} g(w, z_i) = (y(i) - \sqrt{\alpha}x) \cdot \ve_i.
    \end{align}
    Since all coordinates in $y$ are equivalent in the construction, we suppress the index of $i$ and write $y_t = y_t(i)$ for any $i \in [n]$, $t \in [T]$. Then it formulates
    \begin{align*}
        & x_{t+1} = x_t - \eta \alpha x_t + \frac{\eta \sqrt{\alpha}}{n}\sum_{i=1}^ny_t(i) = (1-\alpha \eta)x_t + \eta \sqrt{\alpha}y_t, \\
        & y_{t+1} = y_t - \frac{\eta}{n}y_t + \frac{\eta\sqrt{\alpha}}{n}x_t = \left(1 - \frac{\eta}{n} \right)y_t + \frac{\eta\sqrt{\alpha}}{n}x_t.
    \end{align*}
    We give an upper bound for $x_t$ and $y_t$ by the following induction. If condition
    \begin{equation} \label{eq: induction-GD-2}
        x_{t} \leq 1, \qquad y_t \leq \sqrt{\alpha}
    \end{equation} 
    holds for $t$, then the above condition also holds for $t+1$:
    \begin{align*}
        x_{t+1} & \leq (1 - \alpha \eta) + \eta\sqrt{\alpha} \cdot \sqrt{\alpha} = 1 - \frac{\eta}{\eta T} + \frac{\eta }{\eta T}  \leq 1, \\
        y_{t+1} & \leq \left(1 - \frac{\eta}{n} \right) \sqrt{\alpha} + \eta \frac{\sqrt{\alpha}}{n} = \sqrt{\alpha}.
    \end{align*}
    Then by induction we conclude that \eqref{eq: induction-GD-2} is true.
    For any $t \in [T]$, the lower bound for $x_t$ is much simpler to compute under our choice of parameter $\alpha = 1/(\eta T)$:
    \begin{align*}
        x_{t} & \geq (1 - \alpha \eta)  x_{t-1} \geq (1 - \alpha \eta )^tx_0 = e^{-Ct/T} \geq e^{-C}.
    \end{align*}
    Hence $\hat{x}_T = \frac{1}{T} \sum_{t=1}^T x_t = \Theta(1)$. This then allows us to lower bound $y$ at iteration $t \in [T]$:
    \begin{align*}
        y_t & \geq \left(1-\frac{\eta}{n}\right) y_{t-1} + \frac{\eta\sqrt{\alpha}}{e^Cn} \\
        & \geq \frac{\eta\sqrt{\alpha}}{e^Cn} \cdot \left( 1 + (1-\eta/n) + \cdots (1-\eta/n)^{t-1} \right) \\
        & \geq \frac{\eta\sqrt{\alpha}}{e^Cn} \cdot \frac{1 - (1-\eta/n)^t}{1 - (1 - \eta/n)} .
     \end{align*}
    Now, we discuss two cases: $T = \bigO{n}$ and $T = \bigOmega{n}$.
    \paragraph{Case $T = \bigO{n}$.} We decompose $t = n \cdot \tfrac{t}{n}$ and obtain
    \begin{align*}
        y_t & \geq \frac{\eta\sqrt{\alpha}}{e^Cn} \cdot \frac{1 - (1-\eta/n)^t}{1 - (1 - \eta/n)} = \frac{\eta\sqrt{\alpha}}{e^Cn} \cdot \frac{1 - (1-\eta/n)^{\tfrac{t}{n} \cdot n}}{1 - (1 - \eta/n)} \\
        & \overset{\text{(A)}}{\geq} \frac{\eta \sqrt{\alpha}}{e^C}\left( \frac{t}{n} - \frac{\eta t^2}{2n^2}\right)  \overset{\text{(B)}}{=} \frac{\eta t \sqrt{\alpha}}{2e^Cn} = \sqrt{\frac{\eta}{CT}} \cdot \frac{t}{2e^Cn}%= \Omega\left( \frac{\sqrt{\eta t}}{n}\right).
    \end{align*}
    where $\text{(A)}$ is due to Taylor expansion, $\text{(B)}$ is due to the condition $\eta t \leq \eta T \leq T = \bigO{n}$ and $\alpha = C/(\eta T)$. We then calculate the average output
    \begin{align*}
        \hat{y}_T = \frac{1}{T} \sum_{t=1}^T y_t = \frac{1}{T} \sum_{t=1}^T \sqrt{\frac{\eta}{CT}} \cdot \frac{t}{2e^Cn} \geq \frac{1}{4e^C} \cdot \sqrt{\frac{\eta T}{C}}.
    \end{align*}
    We return to the original $f(w,z)$ by inserting the above analysis on the $j$-th copy:
    \begin{align*}
        \bbE[F(w_{\gd})] \geq \bbE[G(w^{(j)}_{\gd})] \geq \frac{1}{n} \sum_{i=1}^n \left(\frac{x^{(j)}_{\gd}}{\sqrt{\eta T}} - y^{(j)}_{\gd}(i)\right)^2 \geq \bigOmega{\max\left\{ \frac{1}{\eta T,} \ \frac{\eta T}{n^2 }\right\}} = \bigOmega{\frac{\eta T}{n^2 }}
    \end{align*}
    where the last inequality is due to the fact $e^{-C} \leq x_{\gd} \leq 1$. We can always choose a proper $C$ such that the difference is non-vanishing.
    \paragraph{Case $T = \bigOmega{n}$.} We can directly lower bound $y_t$ as
    \begin{align*}
        y_t \geq \frac{\eta \sqrt{\alpha}}{e^C n} \cdot \frac{n}{2\eta} = \frac{1}{2e^C\sqrt{C \eta T}} = \bigOmega{\frac{1}{\sqrt{\eta T}}}
    \end{align*}
    since $(1 - \eta/n)^t \leq 1/2$ when $\eta = \bigO{1}$ and $T = \bigOmega{n}$. We then calculate the average output
    \begin{align*}
        \hat{y}_T = \frac{1}{T} \sum_{t=1}^T y_t = \frac{1}{T} \sum_{t=1}^T \bigOmega{\frac{1}{\sqrt{\eta T}}} \geq \bigOmega{\frac{1}{\sqrt{\eta T}}}.
    \end{align*}
    Similarly, we return to the original $f(w,z)$ by inserting the above analysis on the $j$-th copy and obtain a non-vanishing lower bound by choosing proper $C$:
    \begin{align*}
        \bbE[F(w_{\gd})] \geq \bbE[G(w^{(j)}_{\gd})] \geq \frac{1}{n} \sum_{i=1}^n \left(\frac{x^{(j)}_{\gd}}{\sqrt{\eta T}} - y^{(j)}_{\gd}(i)\right)^2 \geq \bigOmega{\frac{1}{\eta T}}.
    \end{align*}
    This completes our proof.
\end{proof}

\subsection{Proof of Lemma~\ref{lemma: lb-sgd-tn}} \label{appendix: lb-sgd}
\begin{lemma}[Restated Lemma~\ref{lemma: lb-sgd-tn}]
For every $\eta > 0$, $T \geq 1$, if $T =  \bigO{n}$, then there exists a convex, $1$-smooth and realizable $f(w, z): \bbR^d \to \bbR$ for every $z \in \cZ$, and a distribution $D$ such that, with initialization $\| w_0 - w^* \| = \bigO{1}$, the output $w_{\sgd}$ for SGD satisfies
\begin{align*}
 \bbE[F(w_{\sgd})] - F(w^*) = \bigOmega{\frac{\eta T}{n^2}}.  
\end{align*}
Similarly, if $T =  \bigOmega{n}$, then 
the output $w_{\sgd}$ for SGD satisfies
\begin{align*}
 \bbE[F(w_{\sgd})] - F(w^*) = \bigOmega{\frac{1}{\eta T}}.  
\end{align*}
\end{lemma}

\begin{proof}
    We utilize the similar strategy employed in Lemma~\ref{lemma: lb-gd-tn}: we glue multiple copies of basic construction $g: \bbR^{n+1} \times \cZ \to \bbR$ defined in Eq.~\eqref{eq: lb-instance} with $z \sim \text{Unif}([n])$. With probability $\bigOmega{1}$, event $\cE$ happens for at least one copy, i.e. each coordinates We label as the $j$-copy and consider the trajectory of $j$-th copy only. Without the loss of generalization, we set $\vpi(i) = i$ for any $i \in [n]$. This suggests in dataset $S = \{z_1, \dots, z_n\}$, we have
    \begin{equation}
        g(w^{(j)}, z^{(j)}_i) = \frac{\alpha}{2} (x^{(j)})^2 + \frac{1}{2} \|y^{(j)}\|^2 - \frac{x^{(j)}\sqrt{\alpha}}{n} y^{(j)}(i), \qquad \forall i \in [n].
    \end{equation}
    Again, for simplicity, we suppress the upscript of $j$. SGD formulates the update:
    \begin{align*}
        w_{t+1} = w_{t+1} - \eta g(w_t, z_{i_t}) 
    \end{align*}
    where $z_{i_t} \sim \text{Unif}([n])$. The initialization is $x_0 = 1$, $y_0 = 0$. Based on the current value of $w_t$, under expectation, we have
    \begin{align*}
        \bbE[w_{t+1}] = w_t - \frac{\eta}{n} \sum_{i=1}^n \nabla_w g(w_t, z_i).
    \end{align*}
    We write the update of $\bbE[x_t]$ and $\bbE[y_t]$ by plugging stochastic gradients: it easy to see all coordinates in $\bbE[y_t]$ are equivalent, we suppress the index of $i$ and write $y_t = y_t(i)$ for any $i \in [n]$, $t \in [T]$. Then it formulates
    \begin{align*}
        & \bbE[x_{t+1}] = x_t - \eta \alpha x_t + \frac{\eta \sqrt{\alpha}}{n}\sum_{i=1}^ny_t(i) = (1-\alpha \eta)x_t + \eta \sqrt{\alpha}y_t, \\
        & \bbE[y_{t+1}] = y_t - \frac{\eta}{n}y_t + \frac{\eta\sqrt{\alpha}}{n}x_t = \left(1 - \frac{\eta}{n} \right)y_t + \frac{\eta\sqrt{\alpha}}{n}x_t.
    \end{align*}
    We give an upper bound for $\bbE[x_t]$ and $\bbE[y_t]$ by the following induction. If condition
    \begin{equation} \label{eq: induction-SGD-2}
        \bbE[x_{t}] \leq 1, \qquad \bbE[y_t] \leq \sqrt{\alpha}
    \end{equation} 
    holds for $t$, then the above condition also holds for $t+1$:
    \begin{align*}
        \bbE[x_{t+1}] & \leq (1 - \alpha \eta) + \eta\sqrt{\alpha} \cdot \sqrt{\alpha} = 1 - \frac{\eta}{\eta T} + \frac{\eta }{\eta T}  \leq 1, \\
        \bbE[y_{t+1}] & \leq \left(1 - \frac{\eta}{n} \right) \sqrt{\alpha} + \eta \frac{\sqrt{\alpha}}{n} = \sqrt{\alpha}.
    \end{align*}
    Then by induction we conclude that \eqref{eq: induction-SGD-2} is true. For any $t \in [T]$, the lower bound for $x_t$ is much simpler to compute under our choice of parameter $\alpha = C/(\eta T)$:
    \begin{align*}
        \bbE[x_{t+1}] & \geq (1 - \alpha \eta)  x_{t} \geq (1 - \alpha \eta )^tx_0 = e^{-t/T} \geq e^{-C}.
    \end{align*}
    Hence $\bbE[\hat{x}_T] = \frac{x_1}{T} + \frac{1}{T} \sum_{t=2}^T \bbE[x_t|w_{t-1}] = \Theta(1)$. This then allows us to lower bound $y$ at iteration $t \in [T]$:
    \begin{align*}
        \bbE[y_{t+1}] & \geq \left(1-\frac{\eta}{n}\right) y_{t} + \frac{\eta\sqrt{\alpha}}{e^Cn} \\
        & \geq \frac{\eta\sqrt{\alpha}}{e^Cn} \cdot \left( 1 + (1-\eta/n) + \cdots (1-\eta/n)^{t} \right) \\
        & \geq \frac{\eta\sqrt{\alpha}}{e^Cn} \cdot \frac{1 - (1-\eta/n)^{t+1}}{1 - (1 - \eta/n)} .
     \end{align*}
     Now, we discuss two cases: $T = \bigO{n}$ and $T = \bigOmega{n}$.
    \paragraph{Case $T = \bigO{n}$.} We decompose $t = n \cdot \tfrac{t}{n}$ and obtain
    \begin{align*}
        \bbE[y_t] & \geq \frac{\eta\sqrt{\alpha}}{e^Cn} \cdot \frac{1 - (1-\eta/n)^t}{1 - (1 - \eta/n)} = \frac{\eta\sqrt{\alpha}}{e^Cn} \cdot \frac{1 - (1-\eta/n)^{\tfrac{t}{n} \cdot n}}{1 - (1 - \eta/n)} \\
        & \overset{\text{(A)}}{\geq} \frac{\eta \sqrt{\alpha}}{e^C}\left( \frac{t}{n} - \frac{\eta t^2}{2n^2}\right)  \overset{\text{(B)}}{=} \frac{\eta t \sqrt{\alpha}}{2e^Cn} = \sqrt{\frac{\eta}{CT}} \cdot \frac{t}{2e^Cn}%= \Omega\left( \frac{\sqrt{\eta t}}{n}\right).
    \end{align*}
    where $\text{(A)}$ is due to Taylor expansion, $\text{(B)}$ is due to the condition $\eta t \leq \eta T \leq T = \bigO{n}$ and $\alpha = C/(\eta T)$. We then calculate the average output
    \begin{align*}
        \bbE[\hat{y}_T] = \frac{1}{T} \sum_{t=1}^T \bbE[y_t] = \frac{1}{T} \sum_{t=1}^T \sqrt{\frac{\eta}{CT}} \cdot \frac{t}{2e^Cn} \geq \frac{1}{4e^C} \cdot \sqrt{\frac{\eta T}{C}}.
    \end{align*}
    We return to the original $f(w,z)$ by inserting the above analysis on the $j$-th copy:
    \begin{align*}
        \bbE[F(w_{\sgd})] \geq \bbE[G(w^{(j)}_{\sgd})] \geq \frac{1}{n} \sum_{i=1}^n \left(\frac{x^{(j)}_{\sgd}}{\sqrt{\eta T}} - y^{(j)}_{\sgd}(i)\right)^2 \geq \bigOmega{\max\left\{ \frac{1}{\eta T,} \ \frac{\eta T}{n^2 }\right\}} = \bigOmega{\frac{\eta T}{n^2 }}
    \end{align*}
    where the last inequality is due to the fact $e^{-C} \leq x_{\gd} \leq 1$. We can always choose a proper $C$ such that the difference is non-vanishing.
    \paragraph{Case $T = \bigOmega{n}$.} We can directly lower bound $\bbE[y_t]$ as
    \begin{align*}
        \bbE[y_t] \geq \frac{\eta \sqrt{\alpha}}{e^C n} \cdot \frac{n}{2\eta} = \frac{1}{2e^C\sqrt{C \eta T}} = \bigOmega{\frac{1}{\sqrt{\eta T}}}
    \end{align*}
    since $(1 - \eta/n)^t \leq 1/2$ when $\eta = \bigO{1}$ and $T = \bigOmega{n}$. We then calculate the average output
    \begin{align*}
        \bbE[\hat{y}_T] = \frac{1}{T} \sum_{t=1}^T \bbE[y_t] = \frac{1}{T} \sum_{t=1}^T \bigOmega{\frac{1}{\sqrt{\eta T}}} \geq \bigOmega{\frac{1}{\sqrt{\eta T}}}.
    \end{align*}
    Similarly, we return to the original $f(w,z)$ by inserting the above analysis on the $j$-th copy and obtain a non-vanishing lower bound by choosing proper $C$:
    \begin{align*}
        \bbE[F(w_{\sgd})] \geq \bbE[G(w^{(j)}_{\sgd})] \geq \frac{1}{n} \sum_{i=1}^n \left(\frac{x^{(j)}_{\sgd}}{\sqrt{\eta T}} - y^{(j)}_{\sgd}(i)\right)^2 \geq \bigOmega{\frac{1}{\eta T}}.
    \end{align*}
    This completes our proof.
\end{proof}



\section{Missing proof in Section~\ref{sec: infinite}} \label{appendix: infinite}
Here we provide Lemma~\ref{lemma: dim-1-ub-gd}, the GD version of Lemma~\ref{lemma: dim-1-ub-sgd}: in dimension one, GD is also able to achieve $\bigO{1/n}$ bound under the regime $T = \bigOmega{n}$.
\begin{lemma} [Restated Lemma~\ref{lemma: dim-1-ub-gd}]
In dimension one, if $f(w,z)$ is convex, $1$-smooth and realizable with $z \sim D$, then for every $\eta = \Theta(1)$, there exists $T_0 = \Theta(n)$ such that for $T \geq T_0$, the output $w_{\gd}$ of GD satisfies
\begin{equation*}
    \bbE[ F(w_{\gd})] - F(w^*) = \bigO{\frac{1}{n}}.
\end{equation*} 
\end{lemma}
\begin{proof}
From Theorem~10 in \citet{nikolakakis2022beyond}, it holds that for realizable cases (we rescale it to $f(w^*, z) = 0$ for each $z$) with step size $\eta =  \Theta(1)$, it holds that
\begin{equation}
    \bbE[F(w_{\gd})] = \bigO{\frac{1}{T_0} + \frac{1 + T_0/n}{n}}.
\end{equation}
Therefore, for $T_0= \Theta(n)$, it holds that
\begin{equation*}
    \bbE[F(w_{\gd})] = \bigO{\nfrac{1}{n}}.
\end{equation*}
For SGD, the iteration formulates the iterate 
\begin{equation*}
    w_{t+1} = w_t - \eta \nabla F_S(w_t).
\end{equation*}
Under the realizable and convex assumption, the iteration becomes
\begin{equation*}
    w_{t+1} - w^* = (1 - \eta \nabla^2 F_S(\xi)) (w_t - w^*),
\end{equation*}
using mean value theorem, where $\xi$ is a point between $w_t$ and $w^*$. This indicates that the distance $w_t - w^*$ shrinks in each step. Due to the convexity of $F$, it holds that 
$F(w_{t+1}) \leq F(w_t)$. In summary, for any $T \geq T_0$, it holds that 
\begin{equation*}
    \bbE[F(w_{\gd})] - F(w^*) = \bigO{\nfrac{1}{n}}.
\end{equation*} 
which is the desired result.
\end{proof}

\section{Minor proofs} \label{appendix: minor}
\subsection{Lower bound of term \texorpdfstring{$1/n$}{1/n}} \label{appendix: lb-sample}
\begin{lemma} \label{lemma: lb-sample}
    For every $\eta > 0$, $T \geq 1$, there exists a convex, $1$-smooth and realizable $f(w, z): \bbR^{2n} \to \bbR$ for every $z \in \cZ$, and a distribution $D$ such that, it holds for the output of any gradient-based algorithm $\cA[S]$
    \begin{align*}
        \bbE[F(\cA[S])] - F(w^*) = \bigOmega{\nfrac{1}{n}}.  
    \end{align*}
\end{lemma}
\begin{proof}
    We consider the following instance
    \begin{equation*}
        f(w, z = i) = \frac{1}{2} w(i)^2, \qquad z \sim \textrm{Uniform}([2n]),
    \end{equation*}
    where $w(i)$ denotes the $i$-th coordinate of $w$. Then the population risk is
    \begin{align*}
        F(w) = \bbE_{z\sim\text{Uniform}([2n])}[f(w,z)] = \frac{1}{4n} \|w\|^2,
    \end{align*}
    which achieves minimum at $w^* = 0$. It is easy to reckon that $f(w,z)$ is smooth, convex and realizable. Now consider any dataset $S$ of $n$ samples. Since  $z \sim \textrm{Uniform}(2n)$, with probability $\Omega(1)$,  $\Theta(n)$ coordinates are not observed. For any gradient-based algorithm with initialization $w_0 = \frac{1}{\sqrt{2n}} \cdot \bm{1}_d$, the unobserved $\Theta(n)$ coordinates will remain unchanged for any step-size $\eta$ and $T$. Then we have the following lower bound:
    \begin{align*}
        \bbE[F(\cA[S])] - F(w^*) = F(w_{\gd}) - F(w^*) = \Omega\left(\frac{1}{4n} \cdot \frac{n}{2n}\right) = \Omega\left(\frac{1}{n}\right),
    \end{align*}
    which is the desired result.
\end{proof}


\subsection{Lower bound of term \texorpdfstring{$1/\eta T$}{1/eta T}}  \label{appendix: lb-suboptimality}
\begin{lemma} \label{lemma: lb-suboptimality}
For every $\eta > 0$, $T \geq 2$, there exists a convex, $\bigO{1}$-smooth and realizable $f(w, z): \bbR^2 \to \bbR$ for every $z \in \cZ$, and a distribution $D$ such that, the output $w_{\gd}$ for GD satisfies
\begin{align*}
 \bbE[F(w_{\gd})] - F(w^*) = \Omega\left(\frac{1}{\eta T}\right). 
\end{align*}
The same result also holds for SGD.
\end{lemma}
\begin{proof}
    We define the deterministic convex and $\bigO{1}$-smooth function as
    \begin{align}
        f(w) = \frac{1}{2} w^2(1) + \frac{\lambda}{2} w^2(2)
    \end{align}
    with $0 < \lambda < 1$, $w(1)$ and $w(2)$ are the value of first and second coordinate of $w$. Then GD formulates the iteration
    \begin{align*}
        w_{t+1} = w_t - \eta \nabla f(w_t),
    \end{align*}
    with initialization $w_0 = (1,1)$. This is then precisely:
    \begin{align*}
        w_{t+1}(1) & = (1 -\eta) w_{t+1}(1), \qquad w_{t+1}(2) = (1 -\lambda\eta) w_{t+1}(2).
    \end{align*}
    At iteration $t \in [T]$, we can upper bound 
    \begin{align*}
        w_{t}(1) \geq \frac{1}{4} e^{-\eta t} \cdot x_0(1), \qquad w_{t}(2) \geq \frac{1}{4} e^{- \lambda \eta t} \cdot x_0(2) = \frac{e^{-t/T}}{4}.
    \end{align*}
    The averaged output is then
    \begin{align*}
        \hat{w}_T(2) = \sum_{t=1}^T \hat{w}_t(2) \geq \sum_{t=1}^T \frac{e^{-t/T}}{4T} \geq \frac{1-e^{-1}}{4} > \frac{1}{8}.
    \end{align*}
    With $\lambda = \tfrac{1}{\eta T}$, the suboptimality is then
    \begin{equation}
        f(w_{\gd}) - f(w^*) \geq \frac{\lambda}{2} |w_{\gd}(2)|^2 \geq \frac{1}{128\eta T}.
    \end{equation}
    Then the following result holds:
    \begin{equation*}
        \bbE[F(w_{\gd})] - F(w^*) = f(w_T) - f(w^*) \geq \Omega\left(\frac{1}{\eta T}\right)
    \end{equation*}
    because $f(w)$ is a deterministic function. Since the instance is deterministic, then the suboptimality lower bound $\bigOmega{\frac{1}{\eta T}}$ also holds for SGD.
\end{proof}


\subsection{GD upper bound for realizable smooth SCO} \label{appendix: gd-ub}
Here we derive the upper bound for GD under realizable smooth SCO, as in Table~\ref{tab: summary}. The derivation is based on Theorem~10 in \citet{nikolakakis2022beyond}.
In the realizable cases, it holds that (see \citet{nikolakakis2022beyond} for the notations):
\begin{equation*}
    \begin{split}
        &\epsilon_{\text{opt}} = \frac{\bbE \| w_1 - w_S^*\|^2}{\eta T} \\
        &\epsilon_{\text{path}} = \beta (\bbE \| w_1 - w_S^*\|^2 + \epsilon_{{\boldsymbol{c}}} \eta T) \\
        &\epsilon_{{\boldsymbol{c}}} = 0 
    \end{split}
\end{equation*}
Plug them in Theorem~10, it holds that for some constant c,
\begin{equation*}
    |\epsilon_{\text{gen}}| \leq c \frac{\beta \bbE \| w_1 - w_S^*\|^2}{n} + \frac{\beta^2 \eta T \bbE \| w_1 - w_S^*\|^2}{n^2}.
\end{equation*}
Combined with the optimization upper bound $\bigO{1/{\eta T}}$, we obtain the upper bound
\begin{equation*}
    \bbE[F(w_{\gd})] - F(w^*) = \bigO{\frac{1}{\eta T} + \frac{1}{n} + \frac{\eta T}{n^2}}.
\end{equation*}
%\input{texts/appendix_5(ub-gd)}

\vskip 0.2in
\bibliography{bib}

%\documentclass{scrartcl}
\usepackage{tikz,pgfplots}
\usepackage{filecontents}
\begin{document}
\pgfkeys{/pgf/number format/.cd,1000 sep={\,}}

\begin{filecontents}{data.csv}
year,count
2016,998
2015,1000
2014,900
2013,837
2012,826
2011,784
2010,801
2009,731
2008,703
2007,632
2006,629
2005,516
2004,512
2003,476
2002,444
2001,497
2000,478
1999,400
1998,393
1997,399
1996,387
\end{filecontents}

\begin{tikzpicture}
\begin{axis}[ xlabel=Count, ylabel=Year]


\addplot[color=blue,mark=*] table[x=year, y=count, col sep=comma]{data.csv};

 \end{axis} 
 \end{tikzpicture}
\end{document}


\end{document}