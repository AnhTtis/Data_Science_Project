%packages
\usepackage{amsmath}
\usepackage{amssymb}
\usepackage{amsthm}
\usepackage{mathtools}
\usepackage{letterswitharrows}
%enumerating
\usepackage{enumitem}
\setenumerate{label={\normalfont (\roman*)}}
\renewcommand{\labelitemi}{$\diamond$}
\renewcommand{\labelitemii}{$\cdot$}
\renewcommand{\labelitemiii}{$\diamond$}
\renewcommand{\labelitemiv}{$\ast$}
%input
\usepackage[utf8]{inputenc}
\usepackage[T1]{fontenc}
\usepackage{lmodern}
\usepackage[babel]{microtype}
\usepackage[english]{babel}
\usepackage{relsize}


%graphics
\usepackage{graphicx}
\usepackage{subcaption}

%pagesetup
\linespread{1.19}
\usepackage{geometry}
\geometry{left=26.5mm,right=26.5mm, top=32mm, bottom=32mm, marginparwidth=20mm}
\lineskiplimit=-4pt

%references
\usepackage{xcolor} 	
\usepackage{hyperref}
\hypersetup{
	colorlinks,
	linkcolor={red!60!black},
	citecolor={green!60!black},
	urlcolor={blue!60!black},
}
\usepackage[abbrev, msc-links]{amsrefs}
\usepackage[nameinlink, capitalise, noabbrev]{cleveref}
\crefformat{enumi}{#2#1#3}
\crefformat{equation}{#2(#1)#3}
\crefname{mainresult}{Theorem}{Theorems}
\usepackage{doi}
\renewcommand{\doitext}{DOI\,}
\renewcommand{\PrintDOI}[1]{\doi{#1}}
\renewcommand{\eprint}[1]{\href{http://arxiv.org/abs/#1}{arXiv:#1}}
\let\setminus=\smallsetminus

\renewcommand{\subset}{\subseteq}
\renewcommand{\supset}{\supseteq}
\renewcommand{\leq}{\leqslant}
\renewcommand{\geq}{\geqslant}
\renewcommand{\ge}{\geq}
\renewcommand{\le}{\leq}

\DeclareMathOperator{\medcup}{\mathsmaller{\bigcup}}
\DeclareMathOperator{\medsqcup}{\mathsmaller{\bigsqcup}}
\DeclareMathOperator{\medcap}{\mathsmaller{\bigcap}}

% theorem enviroments
% Ordinary theorems that are no main theorems that are numbered with respect to the section
\newtheorem{theorem}{Theorem}[section] 
\newtheorem{proposition}[theorem]{Proposition}
\newtheorem{corollary}[theorem]{Corollary}
\newtheorem{lemma}[theorem]{Lemma}
\newtheorem{lemmaDefinition}[theorem]{Lemma and Definition}
\newtheorem{observation}[theorem]{Observation}
\newtheorem{conjecture}[theorem]{Conjecture}
\newtheorem{problem}[theorem]{Problem}
% Mainresults
\newtheorem{mainresult}{Theorem} 
\newtheorem{maincorollary}[mainresult]{Corollary}
% The follwowing is for Theorems that have a custom number, which is particularly of help if you restate Theorems (just take the ref of the theorem that you want to restate as parameter)  
\newtheorem{innercustomthm}{}
\newtheorem{innercustomprop}{}
\newtheorem{innercustomlem}{}
\newtheorem{innercustomcor}{}
\newenvironment{customthm}[1]
  {\renewcommand\theinnercustomthm{#1}\innercustomthm}
  {\endinnercustomthm}
\newenvironment{customprop}[1]
  {\renewcommand\theinnercustomprop{#1}\innercustomprop}
  {\endinnercustomprop}
\newenvironment{customlem}[1]
  {\renewcommand\theinnercustomlem{#1}\innercustomlem}
  {\endinnercustomlem}
\newenvironment{customcor}[1]
  {\renewcommand\theinnercustomcor{#1}\innercustomcor}
  {\endinnercustomcor}
% Not italic 
\theoremstyle{definition}
\newtheorem{example}[theorem]{Example}
\newtheorem{fact}[theorem]{Fact}
\newtheorem{definition}[theorem]{Definition}
\newtheorem{construction}[theorem]{Construction}
% Theorem styles without a number 
\theoremstyle{remark}
\newtheorem*{notation}{Notation}
\newtheorem*{convention}{Convention}
\newtheorem*{acknowledgment}{Acknowledgement}
\newtheorem*{claim*}{Claim}

%claims numbered within proofs
\newtheorem{claim}{Claim}
\crefname{claim}{Claim}{Claims}
%\AtEndEnvironment{proof}{\setcounter{claim}{0}}

\newenvironment{claimproof}{\noindent\textit{Proof of the claim.}}{\hfill\ensuremath{\blacksquare}\medskip}
\usepackage{etoolbox}

\newenvironment{customclaimproof}[1]{\medskip\noindent\textit{#1.}}{\hfill\ensuremath{\blacksquare}\medskip}

  
%colors for commenting
\newcommand{\magenta}[1]{{\color{magenta}{#1}}}
\newcommand{\todo}[1]{{\color{orange}{#1}}}
\newcommand{\purple}[1]{{\color{purple}{#1}}}
\newcommand{\brown}[1]{{\color{brown}{#1}}}
\newcommand{\red}[1]{{\color{red}{#1}}}
\newcommand{\blue}[1]{{\color{blue}{#1}}}
\newcommand{\dblue}[1]{{\color{darkishBlue}{#1}}}
\newcommand{\dgreen}[1]{{\color{darkishGreen}{#1}}}
\newcommand{\green}[1]{{\color{green}{#1}}}
\newcommand{\violet}[1]{{\color{violet}{#1}}}
\newcommand{\paul}[1]{\todo{(Paul: {#1})}}
\newcommand{\raphael}[1]{\todo{(Raphael: {#1})}}
\newcommand{\COMMENT}[1]{{}}

%letters
%for better greeks
\let\eps=\varepsilon
\let\theta=\vartheta
\let\rho=\varrho
\let\phi=\varphi
%shortcuts
%sets of numbers
\def\N{\mathbb N}
\def\R{\mathbb R}
\def\Z{\mathbb Z}
\def\Q{\mathbb Q}

% \fraka für \mathfrak{a}, \cC für \mathcal{C} etc
\makeatletter

\def\calCommandfactory#1{%
  \expandafter\def\csname c#1\endcsname{\mathcal{#1}}}
\def\frakCommandfactory#1{%
  \expandafter\def\csname frak#1\endcsname{\mathfrak{#1}}}
   
\newcounter{ctr}
\loop
  \stepcounter{ctr}
  \edef\X{\@Alph\c@ctr}
  \expandafter\calCommandfactory\X
  \expandafter\frakCommandfactory\X
  %\edef\Y{\@alph\c@ctr}
  %\expandafter\frakCommandfactory\Y
\ifnum\thectr<26
\repeat

\renewcommand{\cC}{\mathscr{C}}
%\renewcommand{\cD}{\mathscr{D}}
\renewcommand{\cK}{\mathscr{K}}
%\renewcommand{\cP}{\mathscr{P}}
% \renewcommand{\cR}{\mathscr{R}} -- R: I changed this to get curly R back.
%\renewcommand{\cF}{\mathscr{F}}


%commands