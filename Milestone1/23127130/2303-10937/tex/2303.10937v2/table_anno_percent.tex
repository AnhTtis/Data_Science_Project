%\begin{table*}[t]
%\centering
%\resizebox{2.05\columnwidth}{!}{
%\begin{tabular}{ lcc|cc|cc|cc|cc|cc|cc|cc|cc|cc}
%\toprule
 %& \multicolumn{3}{c}{Avg. Precision, IoU}  & \multicolumn{3}{c}{Avg. Precision, Area} \\
 %\cmidrule{2-4} \cmidrule{5-7}

 %& \multicolumn{3}{c}{Avg. Precision, IoU}\\
 %\cmidrule{2-4}
%\% of GT anno. & \multicolumn{2}{c}{person}&\multicolumn{2}{c}{car}&\multicolumn{2}{c}{motorcycle}&\multicolumn{2}{c}{handbag}&\multicolumn{2}{c}{bowl}&\multicolumn{2}{c}{hair drier}&\multicolumn{2}{c}{toaster}&\multicolumn{2}{c}{scissors}&\multicolumn{2}{c}{parking meter}&\multicolumn{2}{c}{tennis racket}  \\

%\midrule 

%\textsc{1\%}  &0.23&0.49&0.26&0.53&0.17&0.43&0.13&0.41&0.13&0.43&0.39&0.52&0.32&0.42&0.08&0.42&0.11&0.39&0.23&0.48\\

%\textsc{5\%} 
%&0.23&0.49&0.28&0.54&0.15&0.41&0.14&0.41&0.13&0.43&0.22&0.46&0.27&0.55&0.15&0.45&0.12&0.35&0.16&0.44\\

%\textsc{10\%}
%&0.23&0.49&0.28&0.54&0.16&0.41&0.14&0.42&0.13&0.43&0.27&0.47&0.26&0.54&0.15&0.46&0.16&0.44&0.16&0.43\\

%\textsc{50\%}
%&0.23&0.49&0.27&0.54&0.16&0.42&0.15&0.42&0.14&0.44&0.24&0.50&0.26&0.54&0.16&0.45&0.15&0.44&0.17&0.44\\
%\midrule 
%\textsc{Average}
%&0.230&0.490&0.272&0.537&0.160&0.417&0.140&0.415&0.132&0.432&0.280&0.487&0.277&0.512&0.135&0.445&0.135&0.405&0.180&0.447\\
%\textsc{Stdev}
%&0&0&0.009&0.005&0.008&0.009&0.008&0.005&0.005&0.005&0.070&0.027&0.028&0.061&0.036&0.017&0.023&0.043&0.033&0.022\\

%\bottomrule
%\hline \\
%\end{tabular}}
%\caption{The table displays the lower and upper depth range values, denoted as $r_c = [\text{lower}, \text{upper}]$, for each object class $c$, which are extracted using varying percentages of ground truth bounding box annotations.
%, such as $1\%$, $5\%$, $10\%$, and $50\%$. 
%Among the ten object classes considered, the first 5 %, namely person, car, motorcycle, handbag, and bowl, 
%exhibit the lowest standard deviations across the different percentages, and the last 5 %, including hair drier, toaster, scissors, parking meters, and tennis racket, 
%display the highest deviations out of the 80 COCO classes.} % for visual methods.} 
%\label{Table:4}
%\end{table*}

\begin{table*}[t]
\centering
\resizebox{2.05\columnwidth}{!}{
\begin{tabular}{ cc|cc|cc|cc|cc|cc|cc|cc|cc|cc}
\toprule
 \multicolumn{2}{c}{person}&\multicolumn{2}{c}{car}&\multicolumn{2}{c}{motorcycle}&\multicolumn{2}{c}{handbag}&\multicolumn{2}{c}{bowl}&\multicolumn{2}{c}{hair drier}&\multicolumn{2}{c}{toaster}&\multicolumn{2}{c}{scissors}&\multicolumn{2}{c}{parking meter}&\multicolumn{2}{c}{tennis racket}  \\

\midrule 
0.23&0.49&0.26&0.53&0.17&0.43&0.13&0.41&0.13&0.43&0.39&0.52&0.32&0.42&0.08&0.42&0.11&0.39&0.23&0.48\\


\bottomrule
\hline \\
\end{tabular}}
\caption{The table displays the lower and upper depth range values per object, denoted as $r_c = [\text{lower}, \text{upper}]$. While the first 5 objects have a smaller depth value, the last 5 objects have larger depth values.} % for visual methods.} 
\label{Table:4}
\end{table*}

