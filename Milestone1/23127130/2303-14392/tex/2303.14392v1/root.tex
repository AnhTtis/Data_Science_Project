%%%%%%%%%%%%%%%%%%%%%%%%%%%%%%%%%%%%%%%%%%%%%%%%%%%%%%%%%%%%%%%%%%%%%%%%%%%%%%%%
%2345678901234567890123456789012345678901234567890123456789012345678901234567890
%        1         2         3         4         5         6         7         8

\documentclass[letterpaper, 10 pt, conference]{ieeeconf}  % Comment this line out if you need a4paper
\let\labelindent\relax
%\documentclass[a4paper, 10pt, conference]{ieeeconf}      % Use this line for a4 paper

\IEEEoverridecommandlockouts                              % This command is only needed if 
                                                          % you want to use the \thanks command

%\overrideIEEEmargins                                      % Needed to meet printer requirements.

%In case you encounter the following error:
%Error 1010 The PDF file may be corrupt (unable to open PDF file) OR
%Error 1000 An error occurred while parsing a contents stream. Unable to analyze the PDF file.
%This is a known problem with pdfLaTeX conversion filter. The file cannot be opened with acrobat reader
%Please use one of the alternatives below to circumvent this error by uncommenting one or the other
%\pdfobjcompresslevel=0
%\pdfminorversion=4

% See the \addtolength command later in the file to balance the column lengths
% on the last page of the document

% The following packages can be found on http:\\www.ctan.org
\usepackage{graphics} % for pdf, bitmapped graphics files

\usepackage{xcolor}
\usepackage{framed,enumitem} 
\usepackage{epsfig} % for postscript graphics files
%\usepackage{mathptmx} % assumes new font selection scheme installed
\usepackage{times} % assumes new font selection scheme installed
\usepackage{amsmath, bm}
\usepackage{amssymb}  % assumes amsmath package installed
\usepackage{arydshln}
\usepackage{amsfonts}
%\usepackage{newtxtext, newtxmath}
\usepackage{algorithm}
\usepackage[noend]{algpseudocode}
\usepackage[noadjust]{cite}
\usepackage{stfloats}
\usepackage{bm}

%\usepackage[sorting = none, backend = bibtex, style=numeric-comp]{biblatex}

\title{\LARGE \bf
On improved commutation for moving-magnet planar actuators*
}


\author{Yorick Broens, Hans Butler and Roland T\' oth% <-this % stops a space
\thanks{*This work has received funding from the ECSEL Joint Undertaking (JU) under grant agreement No 875999 and from the European Union within the framework of the National Laboratory for Autonomous Systems (RRF-2.3.1-21.2022-00002).}% <-this % stops a space
\thanks{Y.Broens, H.Butler and R.T\'oth are with the Department of Electrical Engineering, Eindhoven University of Technology, Eindhoven, The Netherlands. H.Butler is also affiliated with ASML, Veldhoven, The Netherlands. R.T\'oth is also affiliated with the Systems and Control Laboratory, Institute for Computer Science and Control, Hungary,  ({\tt  email: Y.L.C.Broens@tue.nl}). }
}


\begin{document}



\maketitle
\thispagestyle{empty}
\pagestyle{empty}


%%%%%%%%%%%%%%%%%%%%%%%%%%%%%%%%%%%%%%%%%%%%%%%%%%%%%%%%%%%%%%%%%%%%%%%%%%%%%%%%
\begin{abstract}
The demand for high-precision and high-throughput motion control systems has increased significantly in recent years.
The use of moving-magnet planar actuators (MMPAs) is gaining popularity due to their advantageous characteristics, such as complete environmental decoupling and reduction of stage mass. Nonetheless, model-based commutation techniques for MMPAs are compromised by misalignment between the mover and coil array and mismatch between the ideal electromagnetic model and the physical system, often leading to decreased system performance. To address this issue, a novel improved commutation approach is proposed in this paper by means of dynamic regulation of the position dependence of the ideal model-based commutation algorithm,
%two novel electromagnetic calibration approaches are proposed in this paper. First, a static calibration approach is presented which utilizes gradient descent optimization-based strategies to align the moving-magnet with the coil arrays. Secondly, a novel dynamic regulation of the scheduling dependence of the model-based commutation is presented, 
which allows for attenuation of magnetic misalignment, manufacturing inaccuracies and other unmodelled phenomena. The effectiveness of the proposed approach is validated through experiments using a state-of-the-art moving-magnet planar actuator prototype.
%In recent years, there has been a significant increase in demand for motion control systems that are capable of high precision and high throughput. Moving-magnet planar actuators (MMPAs) have gained popularity due to their favorable characteristics such as complete environmental decoupling and a reduction in stage mass. Nonetheless, model-based commutation approaches for MMPAs face challenges due to the necessity for a precise characterization of the highly nonlinear behavior of the actuator. To address this issue, this paper proposes two novel electromagnetic calibration approaches which are able to account for misalignment of the moving-magnet with respect to the coil arrays. Additionally, a secondary approach is proposed which can account for manufacturing inaccuracies and unforeseen physical phenomena, therefore allowing for a significant performance improvement. The proposed approaches are validated using a state-of-the-art MMPA prototype.

\end{abstract}


%%%%%%%%%%%%%%%%%%%%%%%%%%%%%%%%%%%%%%%%%%%%%%%%%%%%%%%%%%%%%%%%%%%%%%%%%%%%%%%%
\section{Introduction}
\label{Section:Introduction}

In recent years, the demand for high-precision and high-throughput motion control systems has seen a significant increase across various fields, such as microelectronics, biotechnology, and nanotechnology, see \cite{zhang2016development,qu2016motion,Butler}. To achieve highly accurate positioning of the mover while allowing for increased throughput, new electromagnetic actuator configurations with improved performance have been developed, particularly in the form of planar motors utilizing a moving-magnet configuration, see \cite{Proimadis-phd,Rovers-phd,Lierop-phd}. The use of \emph{moving-magnet planar actuators} (MMPAs) is gaining popularity due to their advantageous characteristics, such as complete environmental decoupling of the mover and reduction of the stage mass, over their alternatives. However, these benefits come at the cost of introducing an additional surge of complexity from a motion control perspective due to the presence of complex nonlinear multi-physical effects that can only be approximately modeled based on first-principles knowledge, see \cite{6557499}. Additional complexity arises from position dependent effects which are introduced by relative position measurements and actuation of the moving-body. To address these, coordinate frame transformations are required which accurately connect the actuation forces (stator frame) and the position measurements (metrology frame) to the specific point of control on the moving-body (translation frame), see \cite{Steinbuch2013}.

Typically, motion control design for MMPAs is simplified through the use of model-based commutation approaches, see \cite{Proimadis-phd,Rovers-phd,6557499,Lierop-phd}. These approaches rely on a first-principles based model of the inverse \emph{electromagnetic} (EM) behavior of the motor, with the goal of compensating the nonlinear electromagnetic interactions between the coil array and the magnet array, thereby allowing for independent control of the mechanical \emph{degrees of freedom} (DoFs) of the mover. However, despite the benefits of these approaches, they still require a precise characterization of the highly nonlinear behavior of the actuator, which can be difficult to obtain with sufficient accuracy due to the presence of complex nonlinear multi-physical phenomena combined  with unknown manufacturing inaccuracies.
Additionally, model-based commutation techniques utilized in magnetically levitated movers encounter a challenge of misalignment during system initialization due to position uncertainty of the magnetic-mover with respect to the coil array, i.e., the stator, leading to a decrease in system performance. These properties necessitate the adoption of sophisticated calibration strategies for accurate alignment of the moving-magnets with the stator to allow for high-precision positioning of the moving-body.

%This paper introduces a novel control approach for improving the initial model-based commutation while simultaneously attenuating for misalignment of the magnetic-mover and manufacturing inaccuracies. Moreover, 
To address these issues, this paper presents a novel approach that allows for improving the ideal model-based commutation by calibration and active regulation of the position dependence. First, a static gradient-descent based optimization approach is investigated which provides automatic calibration of the commutation frame in terms of aligning the magnetic-mover with the coil array without the use of additional sensors.
Secondly, an active regulatory approach is presented that can adapt the commutation frame to local variations of the EM relations. The latter method incorporates a secondary feedback control loop that includes a learning-based feedforward controller and a feedback controller. This loop can adapt the commutation to variations of the EM relationship due to misalignment, coil pitch, eddy currents or other manufacturing imperfections, allowing for high-precision motion performance of the mover.

The main contributions of this paper are:
\begin{itemize}
\item[(C1)] The development of a novel electromagnetic calibration approach for MMPAs by means of a gradient-descent based optimization strategy. The proposed approach allows for compensation of static misalignment of the moving-magnets with respect to the coil array.
% of an innovative approach for electromagnetic calibration that involves dynamically regulating the scheduling dependence of the initial model-based commutation algorithm. The suggested method incorporates a secondary feedback control loop that includes a learning-based feedforward controller and a feedback controller to address issues such as misalignment, manufacturing imperfections, and higher harmonics that were not considered in the original model-based commutation algorithm, thereby allowing for nanometer accurate positioning of the magnetically levitated mover.
  \item[(C2)] The development of a novel improved commutation approach which  dynamically regulates the commutation frame, therefore attenuating for effects of misalignment, manufacturing inaccuracies and other remnant effects.
%   Check this method on NAPAS setup for 6 DoF control with a stop criterium. Not sure if that's the problem with respect to proper calibration as we only use x,y, for now?
\item [(C3)] The development of a position dependent learning-based feedforward 
%, based on local measurements obtained by (C1) and (C2) 
with the aim of improving the dynamic regulation of the commutation frame.
\end{itemize}

This paper is organized as follows. First, the problem formulation is presented in Section \ref{Section_Problem_formulation}. Next, Section \ref{Section_GRADDESC} presents the proposed gradient descent based static calibration of the commutation frame, i.e., the misalignment between the moving-magnets and the coil array. Section \ref{Section_Dynamic_Regulation} introduces
the proposed dynamic regulation of the commutation frame. In Section \ref{Section:CommutationFeedforward}, the design of a learning-based commutation feedforward is proposed for increasing the performance of the dynamic regulation. Section \ref{Section_Experimental_Validation} presents experimental results of the proposed approaches on a state-of-the-art MMPA prototype. Finally, Section \ref{Section_Conclusions} presents the overal conclusion on the presented work.




%   Basic introduction of MMPA systems. Note the advantages and disadvantages (-> complex commutation and posdep actuation). Build up to misalignment issues and how they limit achievable position tracking performance. 

%   List contributions
%   List organization of paper
%%%%%%%%%%%%%%%%%%%%%%%%%%%%%%%%%%%%%%%%%%%%%%%%%

\vspace*{-.1cm}
\section{Problem formulation}
\label{Section_Problem_formulation}
\subsection{Background}

%The problem at hand concerns the dynamic behavior of a moving-magnet planar actuator (MMPA) and its modeling as a complex multiple-input multiple-output (MIMO) system. The MMPA system is governed by a combination of electromagnetic and mechanical phenomena that exhibit position dependent effects due to the relative actuation and sensing of the moving body. To accurately capture the MMPA system's behavior, it is often represented in linear-parameter-varying (LPV) form. The equations of motion of the MMPA system are given by

The dynamic behavior of an MMPA system is governed by a combination of electromagnetic and mechanical phenomena, see Figure \ref{Fig:ContolInterconnection}, resulting in a complex \emph{multiple-input multiple-output} (MIMO) system, which exhibits position dependent effects due to the relative displacement of the mover with respect to the measurement (metrology) and actuation (stator) frames.
%To accurately capture the MMPA system's behavior, it is often represented in \emph{linear-parameter-varying} (LPV) form, see \cite{5530659}, where the position dependency is captured in a scheduling variable. 
The equations of motion, denoted by $P$, are given by:

\vspace*{-.25cm}
\begin{equation}
M\ddot{q}_\mathcal{T}^{\mathcal{M}}(t) + D\dot{q}_\mathcal{T}^{\mathcal{M}}(t)+Kq_\mathcal{T}^{\mathcal{M}}(t) =H F_m(t)
    \label{eq:MechanicalModel_mover}
\end{equation}

\vspace*{-.15cm}	
\noindent where $M$,$D$ and $K$ are the symmetric mass, damping and stiffness matrices of the mover and $q_\mathcal{T}^{\mathcal{M}}(t) \in \mathbb{R}^{n_q}$ corresponds to the position vector of the mover in the metrology coordinate frame.
$H\in \mathbb{R}^{n_q \times n_{F_m}}$ represents the mapping of the forces acting on the magnet plate to its center of gravity.

Assuming a \emph{rigid-body} (RB) mover, the EM interaction, which relates the currents in the stator frame coils  to forces that are exerted on the magnet plate,
%in the metrology coordinate frame, 
is given by (see \cite{Rovers-phd}):
\vspace*{-.2cm}

\begin{equation}
F_m(t) =\Omega \left(q_{\mathcal{T}}^{\mathcal{S}}(t) \right)i(t),
     \label{eq:EMInteraction}
\end{equation}

\vspace*{-.2cm}
\noindent where $q_\mathcal{T}^{\mathcal{S}}(t) \in \mathbb{R}^{n_q}$ corresponds to the position vector of the mover in the stator coordinate frame. 
%In case of misalignment between the coordinate frames, e.g. $q_{\mathcal{T}}^{\mathcal{S}} = q_{\mathcal{T}}^{\mathcal{M}} + \Delta(q_{\mathcal{T}}^{\mathcal{M}})$, where $\Delta(q_{\mathcal{T}}^{\mathcal{M}})$ expresses the misalignment between the stator frame and the metrology frame. 
In case of misalignment between the two coordinate frames, i.e. $q_{\mathcal{T}}^{\mathcal{S}} = q_{\mathcal{T}}^{\mathcal{M}} + \Delta(q_{\mathcal{T}}^{\mathcal{M}})$, 
the EM interaction can be reformulated as:
%If there is a misalignment between the coordinate frames, which may occur due to positional uncertainty of the mover with respect to the stator during system initialization, it can be expressed as $q_{\mathcal{T}}^{\mathcal{S}} = q_{\mathcal{T}}^{\mathcal{M}} + \Delta(q_{\mathcal{T}}^{\mathcal{M}})$, where $\Delta(q_{\mathcal{T}}^{\mathcal{M}})$ denotes the extent of the misalignment between the stator frame and the metrology frame.
\vspace*{-.2cm}

\begin{equation}
F_m(t) = \Omega \left(q_{\mathcal{T}}^{\mathcal{M}}(t) + \Delta \big(q_{\mathcal{T}}^{\mathcal{M}}(t)\big) \right)i(t),
     \label{eq:EMInteraction}
\end{equation}

\vspace*{-.1cm}
%\noindent where the EM interaction depends on the relative position of the mover.

%Note that the EM interaction is dependent on the relative position of the mover in the metrology coordinate frame. 
%Moreover, in case of misalignment between the metrology frame and the stator frame, 

%In case of misalignment of the stator with respect to the 



%$\Omega \big(\cdot \big) \in \mathbb{R}^{n^{F_m}\times n_{i}}$ corresponds to the electromagnetic interaction, relating the currents in the coil arrays to  forces acting on the moving-magnet.  $H\in \mathbb{R}^{n_q \times n_{F_m}}$ represents the mapping of the forces acting on the magnet plate to the center of gravity of the mover, resulting in a displacement of the magnetically levitated mover. 
%The symbol $\Delta(t)$ represents the lack of alignment between the moving-magnet and the coil arrays, manufacturing imperfections and remnant effects. 
%\noindent Note that due to the time dependency of $\Delta(q_{\mathcal{T}}^{\mathcal{M}}(t))$, effects of manufacturing inaccuracies and eddy currents can be described using this parameter as well. 
\noindent where $\Delta(q_{\mathcal{T}}^{\mathcal{M}}(t))$ represents the variations of the EM relationship due to misalignment, coil pitch, eddy currents and other manufacturing imperfections.
Furthermore, by combining (\ref{eq:MechanicalModel_mover}) and (\ref{eq:EMInteraction}), the MMPA system can be represented in state-space form, denoted by $\mathcal{P}$, as:

\vspace*{-.5cm}
\begin{equation}
\resizebox{.91\hsize}{!}{$
\mathcal{P} = 
 \left[ \begin{array}{cc|c}
    0 & I & 0\\
    -M^{-1}K & -M^{-1}D & M^{-1}H \Omega \left(q_{\mathcal{T}}^{\mathcal{M}}(t) + \Delta \big(q_{\mathcal{T}}^{\mathcal{M}}(t)\big) \right)
    \\
    \hline 
    I &0 & 0
 \end{array}\right]
 \label{eq:PlantDynamics}$}
\end{equation}
\vspace*{-.2cm}

%where $C(q(t))\in \mathbb{R}^{n_s \times n_q}$ corresponds to the mapping of the displacement of the moving-magnet $q(t)$ to the position measurements and $n_s$ is the number of sensors. 
\noindent Note that if the position vector $q_\mathcal{T}^{\mathcal{M}}(t)$ is assumed to be fixed, (\ref{eq:PlantDynamics}) becomes an LTI system, which is often referred to as the \emph{local dynamics} for a particular \emph{frozen position} across the operating envelope of the system and is denoted by $\mathcal{P}(\tt q)$. 

%In the industry, it is typically desired to independently the mechanical DoFs of a MMPA. This is achieved through rigid body decoupling strategies, see \cite{STEINBUCH1998278}, where sensor decoupling is achieved by utilizing the \emph{Moore-Penrose inverse} of the rigid body relation of the output mapping $C(q(t))$ and the actuator decoupling is achieved through a model-based commutation algorithm $\Omega^\dagger \big(q(t) \big)$, see \cite{Proimadis-phd,Rovers-phd,Lierop-phd}, which aims to cancel out the nonlinear electromagnetic interactions between the coil arrays and the magnet array of the mover.

It is a common practice to aim for independent control of the mechanical DoFs of an MMPA. To achieve this, rigid body decoupling methods as described in \cite{STEINBUCH1998278} are generally 

\newpage
\begin{figure}[h]
\vspace{7pt}

    \centering
    \includegraphics[trim={.5cm 0.5cm 2.2cm 0cm}
    ,width=.95\linewidth]{Figures/CDC_Figure_1_Correct.pdf}
    \caption{Schematic representation of a control structure of an MMPA system, where $\Delta \big(q_{\mathcal{T}}^{\mathcal{M}}(t)\big )$ describes the variations of the EM relationship due to misalignment and remnant effects of the system and ${\eta}(t)$ represents an additional control parameter which allows for active commutation control.}
    \label{Fig:ContolInterconnection}
    \vspace*{-.2cm}
\end{figure}

\noindent 
utilized. % by applying the Moore-Penrose inverse of the output mapping's rigid body relation. 
%For MMPA systems, actuator decoupling is achieved through a model-based commutation algorithm $\hat{\Omega}^\dagger \in \mathbb{R}^{n_i \times n_{\tilde{F}_c}}$, as discussed in \cite{Proimadis-phd,Rovers-phd,Lierop-phd}, which intends to eliminate the system's input non-linearity based on the inverse of the ideal EM relationship, e.g. $ \hat{\Omega}^\dagger \big(q_{\mathcal{T}}^{\mathcal{M}}(t) \big )\Omega \big (q_{\mathcal{T}}^{\mathcal{S}}(t) \big ) = I$. Nonetheless, despite the implementation of a model-based commutation algorithm, the heavily nonlinear behavior of a motor may not be perfectly mitigated, particularly in scenarios involving misalignment of the mover in relation to the coil arrays, manufacturing imperfections, presence of eddy currents. Consequently, a mismatch between the control forces and the physical forces acting on the mover arises, which limits the  system performance.
Actuator decoupling in MMPA systems is achieved through a model-based commutation algorithm, which is discussed in \cite{Proimadis-phd,Rovers-phd,Lierop-phd}. This algorithm aims to eliminate the input non-linearity of the system based on the inverse of the ideal EM relationship, i.e.\begin{small} $\Omega \big(q_{\mathcal{T}}^{\mathcal{M}}(t) \big)$$\hat{\Omega}^\dagger  \big(q_{\mathcal{T}}^{\mathcal{M}}(t) \big) = I$ \end{small}. However, due to the presence of misalignment of the mover in relation to the coil array, manufacturing imperfections and eddy currents, the highly nonlinear behavior of the motor may not be perfectly mitigated, resulting in a mismatch between the control forces and the physical forces acting on the mover. 

We propose an approach that involves an extension of the conventional rigid body feedback control interconnection, as depicted in Figure \ref{Fig:ContolInterconnection}, through the incorporation of a new controllable parameter ${\eta}(t)$. Through regulation of $\eta(t)$, we aim to minimize the following objective function: 
%Two distinct strategies are introduced to implement this approach. The first strategy involves the static calibration of $\tilde{\eta}(t)$ to account for any misalignment of the magnetically levitated mover with respect to the coil arrays. The second strategy involves a dynamic regulation of $\tilde{\eta}(t)$ through the use of a learning-based feedforward controller in combination with a feedback controller.
\vspace*{-.5cm}

\begin{equation}
\resizebox{.91\hsize}{!}{$
\min_{\eta(t)}
\left 
\| \Omega\Big(q_{\mathcal{T}}^{\mathcal{M}}(t)+\Delta(q_{\mathcal{T}}^{\mathcal{M}}(t))\Big) \Omega^\dagger \Big (q_{\mathcal{T}}^{\mathcal{M}}(t)+\eta(t)\Big) - I \right \|,
\label{costfunctionpar2}$}
\end{equation}

\vspace*{-.1cm}

 \noindent such that effects originating from misalignment, eddy currents or other manufacturing imperfections are attenuated for, allowing high-precision motion control of the mover.

\vspace*{-.1cm}
% -> indicate that commutation offset forces can be seen as static function of the position
%   Introduce the control scheme (FIG 1) and its components as well as a MMPA system.

\vspace*{-.1cm}
\subsection{Problem statement}

\label{Subsection_problem_statement}
\vspace*{-.1cm}


The problem that is being addressed in this paper is to develop an enhanced commutation approach which aims at eliminating the effects of $\Delta(q_{\mathcal{T}}^{\mathcal{M}} (t))$ through regulation of $\eta(t)$, such that the cost function expressed by (\ref{costfunctionpar2}) is minimized. We aim at developing an improved commutation approach, such that the following requirements are satisfied:
%Moreover, the objective of this paper is to construct improved commutation approaches, such that the following requirements are satisfied: 
%The problem that is being addressed in this paper is to
%develop electromagnetic calibration approaches which are able to effectively deal with misalignment of the magnetic mover with respect to the coil arrays, manufacturing imperfections and unforeseen physical phenomena. Moreover, the objective of this paper is to construct a electromagnetic calibration approaches, such that the following requirements are satisfied:
\begin{itemize}
  %\item[(R1)] The electromagnetic calibration approach must be able to mitigate the effects of misalignment between the magnetic mover and the coil arrays, thereby improving closed-loop performance of the system. 
  %The closed-loop system is stable for all $\tt q \in \mathbb{Q}$.
  \item[(R1)] The approach is able to establish an EM calibration of the commutation frame using local measurements, thus avoiding 
  %the necessity of 
  extensive remodelling of the EM relations.
  \item[(R2)] The approach is capable to attenuate for EM discrepancies, originating from coil pitch, eddy currents or other manufacturing imperfections.
   \item[(R3)] The closed-loop system is locally stable \emph{for all positions} during electromagnetic calibration.
\end{itemize}
%   State the limitation of misalignment of the translator w.r.t. stator regarding postracking performance (why do we require calibration approaches).
%   Also list the contributions in the requirements we need to fulfill in order to resolve the limitations introduced by misalignment. 

\section{Static calibration of the commutation}
\label{Section_GRADDESC}
%This Section presents a static EM calibration approach of the commutation frame in terms of aligning the magnetic-mover with the coil array without the use of additional sensors. Moreover, by considering a static misalignment of the commutation frame, under the assumption that $\Delta$ is invariant of $q_{\mathcal{T}}^{\mathcal{M}}$, the objective function expressed by (\ref{costfunctionpar2}) is reformulated as:

In this section, a static EM calibration method for the commutation frame is introduced, which involves aligning the magnetic-mover with the coil array without relying on supplementary sensors. The proposed approach employs a static optimization technique to address commutation frame misalignment under the assumption that $\Delta$ is invariant with respect to $q_{\mathcal{T}}^{\mathcal{M}}$. 
%The cost function represented by (\ref{costfunctionpar2}) is reformulated accordingly:
%\begin{equation}
%\min_{\eta}
%\left 
%\| \Omega\Big(q_{\mathcal{T}}^{\mathcal{M}}(t)+\Delta\Big) \Omega^\dagger \Big (q_{\mathcal{T}}^{\mathcal{M}}(t)+\eta\Big) - I \right \|
%\label{costfunctionpar3}
%\end{equation}

\begin{algorithm}[b]
\caption{Gradient-descent based static calibration}\label{euclid}
\begin{algorithmic}[1]
\State Set initial step size $\lambda_{k}$
\For {$k=1:n_{\mathrm{iterations}}$}
\For {$i=1:n_{\tt {p}}$} 
\State  Measure $\|F_c(\tt q_i+{\eta}_k)\|_2$ 
\For {$j=1:n_{{\eta}}$}
%\If {$j\leq n_\tilde{\eta}$}
\State Measure $\|F_c(\tt q_i+{\eta}_k +\xi_j )\|_2$ 
\EndFor
\State \textbf{end}
\State Construct $\nabla \|F_c(\tt q_i+{\eta}_k)\|_2$  by (\ref{Gradientestimation})
\EndFor
\State \textbf{end}
\State Construct $\eta_{k+1}$ by (\ref{UpdateLaw})
\State update $\lambda_{k+1}$
\EndFor
\State \textbf{end}
\end{algorithmic}
\end{algorithm} 
As the model-based commutation relies on the inverse of the ideal EM behavior of the motor, a static misalignment of the commutation frame results in static disturbance forces, which appear in a position dependent manner due to the position dependent nature of the motor. These disturbance forces can be indirectly measured from the steady-state response of the rigid body feedback control forces as the integral action of the controller attenuates for these effects. In this context, the static attenuation forces can be viewed as a function of the position, i.e. $F_c(q_{\mathcal{T}}^{\mathcal{M}}+\eta)$, which can be manipulated through regulation of ${\eta}$. Moreover,
%therefore allowing for alignment of the commutation frame. 
%the aim is to determine the optimal commutation calibration parameter ${\eta}$ that minimizes the $2$-norm of the static compensation forces for all local positions of the MMPA system. 
%This accounts for commutation frame misalignment. 
The objective is to find the optimal calibration parameter ${\eta}$ for commutation frame alignment, which minimizes the 2-norm of the static compensation forces across all local positions of the MMPA system.
Mathematically, this is described as:
%the steady-state feedback control force vector's $2$-norm minimization across multiple local positions is formulated as: %Moreover, the objective is to find the optimal commutation calibration parameter ${\eta}$ which minimizes the $2$-norm of the static compensation forces over all local positions of the MMPA system, thus accounting for misalignment of the commutation frame. The $2$-norm minimization of the steady-state feedback control force vector over multiple local positions is mathematically formulated as:
\vspace*{-.3cm}

\begin{equation}
\min_{{\eta}}\sum_{i=1}^{n_{\mathrm{p}}} \|  F_c \left ({\tt q_i}+\eta\right )   \|_2,\label{GD_ObjectiveFunction}
\end{equation}

\vspace*{-.2cm}

%The model-based commutation algorithm, $\Omega^\dagger\big(q(t)+\tilde{\eta}(t)\big)$, is based on a first-principles based model of the inverse electrostatic behavior of the motor. Mismatch between the control commutation and the physical relationship between the currents in the coils and the forces acting on the mover results in position dependent disturbance forces for which the integral action of the feedback controller $K_{\mathrm{RB}}$ will will attenuate.
%This section presents a novel approach for static electromagnetic calibration of MMPAs, which involves static optimization of the control parameter $\tilde{\eta}$, see Figure \ref{Fig:ContolInterconnection}, using a gradient-descent based approach. Due to the nature of the model-based commutation algorithm, which relies on a first-principles based model of the inverse electrostatic behavior of the motor, mismatch between the control commutation and the physical relationship between the currents in the coils and the forces acting on the mover will result in position dependent disturbance forces for which the integral action of the rigid body feedback controller $K_{\mathrm{RB}}$ must attenuate.
%This allows for reformulation of the problem as a gradient-descent optimization, see \cite{ruder2016overview}. 
 
%Therefore, we can see the position dependent static attenuation forces as a function of the scheduling position, which, in case of the control interconnection proposed by Figure \ref{Fig:ContolInterconnection}, is dependent on our introduced control parameter $\tilde{\eta}(t)$, therefore allowing for formulation of this problem in a gradient-descent setting, see \cite{ruder2016overview}.
%The objective is to find an optimal commutation calibration parameter $\tilde{\eta}^*$ which minimizes the $\mathcal{L}_2$-norm of the static feedback control forces $F_c$ over all local positions of the operating envelope of the MMPA system, thereby accounting for possible misalignment of the moving-magnet with respect to the coil arrays. The $\mathcal{L}_2$-minimization of the static feedback control force vector $F_c$ is mathematically formulated as:
%In this section, we propose a novel method for the static electromagnetic calibration of MMPA systems. The approach involves optimizing the control parameter $\tilde{\eta}$, see Figure \ref{Fig:ContolInterconnection}, using a gradient-descent based approach. The model-based commutation algorithm utilized in this method is based on a first-principles model of the inverse electrostatic behavior of the motor. Mismatch between the control commutation and the physical relationship between the currents in the coils and the forces acting on the mover may result in position dependent disturbance forces. However, the integral action of the rigid body feedback controller $K_{\mathrm{RB}}$ will attenuate these forces.
%In this context, the position dependent static attenuation forces can be viewed as a function of the scheduling position, which is dependent on the control parameter $\tilde{\eta}(t)$ introduced in Figure \ref{Fig:ContolInterconnection}. This allows us to formulate the problem as a gradient-descent optimization, as described in \cite{ruder2016overview}. The objective is to find the optimal commutation calibration parameter $\tilde{\eta}^*$ that minimizes the $\mathcal{L}_2$-norm of the static feedback control forces $F_c$ over all local positions of the operating envelope of the MMPA system, accounting for possible misalignment of the moving-magnet with respect to the coil arrays. This optimization problem is mathematically formulated as minimizing the $\mathcal{L}_2$-norm of the static feedback control force vector $F_c$.

\noindent where $n_{\mathrm{p}}$ corresponds to the number of local positions. The objective function given by (\ref{GD_ObjectiveFunction}) is reformulated into a gradient-descent (GD) based setting, see \cite{ruder2016overview,berglund2022novel}:
%In order to achieve the objective function defined by (\ref{GD_ObjectiveFunction}), a least-squares type setting is augmented in the gradient-descent algorithm, such that the calibration vector $\tilde{\eta}(t)$ is obtained while considering multiple local positions of the system. Moreover, the augmented gradient-descent algorithm is given by:
%\begin{equation}
%\resizebox{.91\hsize}{!}%{$
    %\Theta{\eta}(k+1) = %\Theta{\eta}(k) - %\lambda(k) %\begin{pmatrix}
      %\nabla \left| \left %|F_c({\tt q_1}+%{\eta}(k)) \right |% \right |_2^\top  %\ \hdots \ \nabla %\left| \left %|F_c({\tt q_{n_{\tt% p}}}+{\eta}(k)) %\right | \right %|_2^\top 
    %\end{pmatrix}^\top %,$}
%\end{equation}
\vspace*{-.3cm}
\begin{equation}
\resizebox{.91\hsize}{!}{$
    \Theta{\eta}_{k+1} = \Theta{\eta}_k - \lambda_k \begin{pmatrix}
      \nabla \left| \left |F_c({\tt q_1}+{\eta}_k) \right | \right |_2^\top  \ \hdots \ \nabla \left| \left |F_c({\tt q_{n_{\tt p}}}+{\eta}_k) \right | \right |_2^\top 
    \end{pmatrix}^\top ,$}
\end{equation}

\vspace*{-.1cm}

\noindent where $k \in \mathbb{N}$ corresponds to the optimization step, $\lambda_k$ is the learning rate and \begin{small}$ \Theta  = ( I_{{n_{\mathrm{p}}}\times 1} \otimes I_{n_{{\eta}} \times n_{{\eta}}})$\end{small} denotes a projection matrix. The corresponding GD optimizer is given by:
\vspace*{-.2cm}
\begin{equation}
\resizebox{.91\hsize}{!}{$
    {\eta}_{k+1} = {\eta}_k - \lambda_k\Theta^\dagger \begin{pmatrix}
      \nabla \left| \left |F_c({\tt q_1}+{\eta}_k) \right | \right |_2^\top  \ \hdots \ \nabla \left| \left |F_c({\tt q_{n_{\tt p}}}+{\eta}_k) \right | \right |_2^\top 
    \end{pmatrix}^\top$} 
    \label{UpdateLaw}
\end{equation}

\vspace*{-.1cm}
%Since in the real system, the gradient is unknown, gradient approximation approaches must be performed, see \cite{berglund2022novel}. Moreover, this is achieved by introducing a perturbation parameter $\xi_j \in [1 \ n_{\eta} ]$, resulting in the following gradient approximation: 

%In practise, it is not possible to directly measure gradients. As a result, an approximation technique is employed which estimates the corresponding gradients using the available feedback control forces. This is realized by by introducing a small perturbation parameter $\xi_j$ with $j\in [1 \ n_{\eta} ]$, which allows for approximation of the gradients as:
%The direct measurement of gradients is infeasible in practical settings, thus necessitating for an approximation approach by leveraging the available feedback control forces. The approximation approach involves introducing a small perturbation parameter $\xi_j$, with $j \in [1 \ n_{\eta}]$, which allows for gradient estimation using multiple measurements of the corresponding feedback control forces. Moreover, each $i$-th gradient is approximated as:

In practical settings, the direct measurement of gradients is infeasible, hence an online gradient approximation method is required which capitalizes on the accessible static feedback control forces. This is achieved through $(n_\eta+1)$ measurements per local position by introducing a perturbation parameter $\xi_j$ with $j \in [1 \  n_{\eta}]$, thus yielding the following gradient approximation:
\vspace*{-.2cm}
\begin{equation}
\resizebox{.91\hsize}{!}{$
    \nabla \|F_c ({\tt q_i}+\eta_k) \|_2 \approx 
    \left( \begin{array}{c}
  \xi_1^{-1}\left(\left | \left | F_c({\tt q_i}+{\eta}_k+\xi_1 )\right| \right|_2-  \|  F_c({\tt q_i}+{\eta}_k ) \|_2 \right)
  \\
  \vdots \\
  \xi_{n_{\eta}}^{-1}\left(\left | \left | F_c({\tt q_i}+{\eta}_k+\xi_{n_\eta} )\right| \right|_2-  \|  F_c({\tt q_i}+{\eta}_k ) \|_2 \right)
   \end{array}\right)
    $}
    \label{Gradientestimation}
\end{equation}

%\noindent where online approximation of the corresponding gradients is achieved by introducing a small perturbation parameter $\xi_j$ with $j \in [1 \ \ n_{\tilde{\eta}}]$, resulting in the following approximation of the gradients:
%\begin{equation}
%\resizebox{.90\hsize}{!}{$
%    \nabla \left | \left | F_c^i(\tilde{\eta} (k))\right| \right|_2 \approx \left(
 %       \frac{\left | \left | F_c^i(\tilde{\eta}(k)+\xi_j )\right| \right|_2-\left | \left | F_c^i(\tilde{\eta}(k) )\right| \right|_2}{\xi_j} \right), \ \forall j
 %       $}
%\end{equation}


%The pseudo-code of the proposed gradient-descent based static electromagnetic calibration is illustrated in Algorithm \ref{euclid}, where the main idea is to measure the averaged steady-state control forces $\left | \left | F_c^i(\tilde{\eta}(k) )\right| \right|_2$ and the averaged perturbed control forces $\left | \left | F_c^i(\tilde{\eta}(k)+\epsilon_j )\right| \right|_2$, such that the gradient can be reconstructed to allow for the update of the calibration parameter $\tilde{\eta}(k+1)$.
The resulting approach is described by Algorithm \ref{euclid}. The key concept is to obtain measurements of two types of control forces: (i) the averaged steady-state control forces and (ii) the averaged perturbed static control forces for all $n_{\tt p}$ local positions. These measurements are then used to reconstruct the local gradients, providing the calibration parameter update process ${\eta}_{k+1}$ by (\ref{UpdateLaw}). To speed up the optimization process, the step size $\lambda_{k+1}$ can be updated in an iterative manner by using line search approaches, see \cite{berglund2022novel}, or by 
considering a Gauss-Newton type approaches, e.g. using a similar strategy for Hessian approximation as (\ref{Gradientestimation}).



%the learning rate $\lambda_k$ is updated each iteration according to a backtracking inexact line search based approach, where $\alpha \in (0 \ 1)$.%Furthermore, the learning rate $\lambda(k)$ is empirically chosen and updated each iteration according to a scalar $0< \alpha < 1$ to prevent the optimization from not converging to a minimum.


%%%%%%%%%%%%%%%%%%%%%%%%%%%%%%%%%%%%%%%%%%%%%%%%%%%%%%%%%%%%%%%%%%%%%%
%   This part belongs to section II

%Consider the control interconnection illustrated by Figure \ref{fig:DefaultCTRLScheme}. In case of misalignment between the magnetically levitated mover and the coil arrays, i.e. $\Tilde{\eta} \neq \Delta$, the first-principles model-based control commutation will not be able to perfectly cancel out the electrostatic behaviour of the planar motor, resulting in nonlinear static disturbances for which the feedback controller must attenuate.
%%%%%%%%%%%%%%%%%%%%%%%%%%%%%%%%%%%%%%%%%%%%%%%%%%%%%%%%%%%%%%%%%%%%%%

\vspace*{-.2cm}

\section{Dynamic regulation of the commutation}
\label{Section_Dynamic_Regulation}
\vspace*{-.1cm}
\begin{figure}[t]
\vspace{7pt}
    \centering
        \includegraphics[trim={.5cm 0.5cm 2.2cm 0cm}
    ,width=.95\linewidth]{Figures/CDC_Figure_2_Correct.pdf}
    \caption{Proposed feedback control interconnection for dynamic regulation of the commutation frame.}  \label{Fig:ContolInterconnection1}
        \vspace*{-.5cm}
\end{figure}

%This section introduces a novel and enhanced commutation approach that incorporates a dynamic regulation of the commutation frame. This regulation enables the adaptation of the commutation frame to account for local variations in the EM interaction. The proposed dynamic regulation, illustrated in Figure \ref{Fig:ContolInterconnection1}, employs a commutation controller $K_c$ which actively regulates the low-frequent content of the rigid body feedback control forces to zero, such that the objective function, expressed by (\ref{costfunctionpar2}), is satisfied. To design $K_c$ the control interconnection shown in Figure \ref{Fig:ContolInterconnection1} is transformed into a \emph{linear fractional representation} (LFR) as depicted in Figure \ref{fig:LFR_Representation}a. This LFR separates the controllers $K_{\mathrm{RB}}$ and $K_c$ into a lower connection, while $\Omega$ and $\hat{\Omega}^\dagger$ are lifted to an upper connection denoted by $\tilde{\Delta}$.
\begin{figure}[b]
    \centering
\includegraphics[trim={1.9cm 0cm 2.2cm .7cm}    ,height=4cm,width=\linewidth]{Figures/CDCTEST5.pdf}  
\vspace*{-.8cm}
\caption{Left Figure illustrates the LFR representation of the control interconnection illustrated by Figure \ref{Fig:ContolInterconnection1}. Right Figure illustrates the nonlinear plant as seen by the commutation controller $K_c$.}
\label{fig:LFR_Representation}
\end{figure}

In this Section, a novel and enhanced commutation approach is introduced that incorporates a dynamic regulation of the commutation frame. This regulation enables the adaptation of the commutation frame to account for local variations of the EM interaction. Figure \ref{Fig:ContolInterconnection1} illustrates the proposed dynamic regulation, in which the commutation controller $K_c$ is designed to regulate the low-frequent content of the rigid-body feedback control forces to zero, such that the objective function, expressed by (\ref{costfunctionpar2}), is satisfied. To allow for design of $K_c$, the control interconnection illustrated by Figure \ref{Fig:ContolInterconnection1} is reformulated as an LFR, see Figure \ref{fig:LFR_Representation}(a), where the controllers $K_{\mathrm{RB}}$ and $K_c$ are extracted into a lower connection while $\Omega$ and $\Omega^\dagger$ are lifted out as an upper connection, forming a diagonal block nonlinearity, which is denoted by $\tilde{\Delta}$. The resulting disconnected plant $\Tilde{P}$ is assumed to be a generalized plant. Moreover, under the assumption that $K_{\mathrm{RB}}$ robustly internally stabilizes the plant in the conventional rigid-body control case, e.g., under $K_c = 0$, see Figure \ref{fig:LFR_Representation}(b), the resulting open loop transfer $r \rightarrow q_{\mathcal{T}}^\mathcal{M}$ is stable with respect to the block diagonal nonlinearity $\Tilde{\Delta}$ for the full operating envelope of the system. In this context, a set of local dynamics, denoted by $\lbrace  \tilde{\Delta} \star \tilde{P} \star K_{\mathrm{RB}} \rbrace_{i=1}^{n_{\tt p}}$, can be obtained by performing open-loop identification strategies for various forced equilibria, i.e. frozen positions, of the nonlinear system. Consequently, this set of local dynamics can be utilized for robust control design of the commutation controller $K_c$ by means of robust sequential loop closing strategies, see \cite{Oomen,skogestad2007multivariable}, such that closed-loop performance is improved by regulation of the commutation frame while simultaneously maintaining internal stability.

%From experimental data, obtained from a state-of-the-art moving-magnet planar actuator prototype, several conclusions can be drawn. First, it is observed that all local models are decoupled for low frequencies due to the application of rigid body decoupling strategies in the position control loop. Secondly, it is observed that the diagonal elements of the local models exhibit a 0 $\frac{\mathrm{dB}}{\mathrm{dec}}$ slope for low frequencies, which can be explained by the static relation between the compensatory control forces of the feedback controller and the position of the mover. 
Analysis of experimental data collected from a cutting-edge MMPA prototype leads to several noteworthy observations. First, it is evident that, owing to the utilization of rigid body decoupling techniques in the position control loop, all local models are decoupled at low frequencies. Secondly, the diagonal elements of the local models demonstrate a 0 $\frac{\mathrm{dB}}{\mathrm{dec}}$ slope at low frequencies, which can be attributed to the static correlation between the compensatory control forces of the rigid body feedback controller and the relative position of the commutation frame. Therefore, the diagonal elements of the commutation controller are structured as PI-type of controllers, such that the steady-state error of the rigid body feedback control forces is actively regulated to zero by the integral action of the commutation controller.
%it makes sense to structure $K_c$ as a PI type of controller, such that the integrator actively steers the steady-state error of $F_c(t)$ to zero by manipulating $\tilde{\eta}(t)$. 
Hence, the commutation feedback controller $K_c$ can be chosen of the form:

\vspace*{-.45cm}
\begin{equation}
    K_c = \text{diag}\left(\frac{2\pi f_{bw}^i}{s} \right), \quad i \in [1 \ \ n_{{\eta}}],
    \label{PICOTROLLER}
\end{equation}

\vspace*{-.15cm}

\noindent where $s$ is the complex frequency and $f_{bw}^i$ denotes the intended commutation feedback control bandwidth, which is intentionally chosen to be 100 times smaller than \emph{bandwidth of the position controller} to avoid excessive interaction between the two loops at high frequencies.


%In order to prevent high-frequenct interaction between the commutation loop and the position control loop, the commutation bandwidth is deliberately set at a value which is at least 100 times smaller than the position control bandwidth.

%To avoid high-frequent interaction between the commutation loop and the position control loop, the commutation bandwidth is chosen to be 100 times smaller than the position control bandwidth.
%To achieve this, the proposed approach introduces a commutation feedback controller $K_c$ 
%and a learning-based commutation feedforward controller $K_c^{\mathrm{FF}}$ 
%to regulate the control parameter $\tilde{\eta}(t)$. As shown in Figure \ref{Fig:ContolInterconnection1}, $K_c$ actively steers the feedback control forces $F_c(t)$ to zero by actively manipulating the initial model-based commutation algorithm, while $K_c^{\mathrm{FF}}$ aims to alleviate the burden on the commutation feedback controller.

%This section presents a new electromagnetic calibration approach for MMPAs using a dynamic regulation strategy of the scheduling dependence of the initial model-based commutation algorithm. 
%The proposed control approach is capable of addressing various sources of error, such as misalignment between the moving-magnet and the coil arrays, manufacturing inaccuracies, and unaccounted remnant effects.
% To achieve this, the proposed approach introduces a commutation feedback controller $K_c$ and a learning-based commutation feedforward controller $K_c^{\mathrm{FF}}$ to regulate the control parameter $\tilde{\eta}(t)$. As shown in Figure \ref{Fig:ContolInterconnection1}, $K_c$ actively steers the low-frequency feedback control force content of $F_c(t)$ to zero by manipulating the initial model-based commutation algorithm, while $K_c^{\mathrm{FF}}$ aims to alleviate the burden on the commutation feedback controller. 
 %Incorporation of integral action in $K_c$ enables it to address issues such as misalignment of the moving magnet with respect to the coil arrays and manufacturing inaccuracies, in addition to unforeseen physical phenomena. 
%This section presents a novel electromagnetic calibration approach for MMPAs using a dynamic regulation strategy of the scheduling dependence of the initial model-based commutation algorithm. The proposed control approach is depicted in Figure \ref{Fig:ContolInterconnection1}, where the control parameter $\tilde{\eta}(t)$ is regulated through a commutation feedback controller $K_c$ and a learning-based commutation feedforward controller $K_c^{\mathrm{FF}}$. The task of $K_c$ is to actively steer the low-frequent feedback control force content of $F_c(t)$, which is introduced by the aforementioned phenomena,  to zero by actively manipulating the initial model-based commutation algorithm, while the learning-based commutation feedforward $K_c^{\mathrm{FF}}$ aims to alleviate the burden on the commutation feedback controller. The proposed approach is capable of addressing issues such as misalignment of the moving magnet with respect to the coil arrays, manufacturing inaccuracies, and unforeseen physical phenomena through incorporation of integral action in $K_c$.



%approach for dynamic regulation of the scheduling dependence of the initial model-based commutation algorithm through modulation of $\tilde{\eta}(t)$ using a combination of a learning-based feedforward controller $K_c^{\mathrm{FF}}$ and a commutation feedback controller $K_c$ as illustrated in Figure Fig:ContolInterconnection. The proposed approach allows for attenuation of misalignment of the moving-magnet with respect to the coil arrays, manufacturing inaccuracies and unforeseen physical phenomena that were not considered in the initial model-based commutation model.

%\subsection{Feedback commutation control design}

%Misalignment of the mover with respect to the coil arrays, e.g. $\Delta(t) \neq 0$, results in static disturbance forces to be attenuated for by the integral action of the rigid body feedback controller $K_{\mathrm{RB}}$. In order to construct a control law  that regulates the initial model-based commutation, the closed-loop transfer of the position control loop, i.e. $\Tilde{\eta}(t) \rightarrow F_c(t)$, can be seen as an equivalent plant for which the aim is to design a commutation feedback controller $K_c$, such that the low-frequent content of $F_c(t)$ is actively steered to zero. Under the assumption that the position control loop is locally stable for all $\tt q \in \mathbb{Q}$, obtaining a model of the transfer between $\tilde{\eta}(t)$ and $F_c(t)$ can be trivially achieved through open-loop system identification strategies. Moreover, the proposed approach allows for tuning of a commutation feedback controller using \emph{nonparametric models}, e.g. frequency response functions (FRFs), resulting in a set of \emph{local dynamics} of the position dependent system, denoted by $\lbrace \mathcal{G}_{\tt q}\rbrace_{i=1}^{n_{\tt p}}$, where $n_p$ corresponds to the number of local positions, considered for the measurements. 

%Next, the goal is to design a feedback controller $K_c$, which locally stabilizes the MIMO transfer between $F_c(t)$ and $\tilde{\eta}(t)$. Application of the model-based commutation proposed in \cite{Proimadis-phd,Rovers-phd,6557499,Lierop-phd} results in control decoupling of the position control loop. Moreover, the closed-loop interconnection  $\lbrace \mathcal{G}_{\tt q}\rbrace_{i=1}^{n_{\tt p}}$ is automatically decoupled as well, which can be validated through the \emph{relative gain array}, see \cite{skogestad2007multivariable}, thus allowing for well-understood SISO control design approaches, such as loop-shaping based approaches, see \cite{Oomen} , for each independent commutation loop. Furthermore, since effects of misalignment result in low-frequent position dependent feedback control forces $F_c(t)$, $K_c$ can be designed to be of form:
%\begin{equation}
%K_c =  \text{diag}\left( \frac{\tau_i}{s} \right), \qquad i \in [1 \ \ n_{\tilde{\eta}}],
 %   \label{K_c}
%\end{equation}

%\noindent where $s$ corresponds to the complex frequency and $\tau_i$ is a proportional gain that is utilized to achieve a desired commutation feedback control bandwidth.
%\subsection{Feedforward commutation control design}

%To alleviate the burden of the commutation feedback controller $K_c$, a learning-based feedforward controller $K_c^{\mathrm{FF}}$ is introduced.
%Due to the static nonlinear behavior of the commutation algorithm, the feedforward controller $K_c^{\mathrm{FF}}$ is modeled using the \emph{Gaussian Process} (GP) framework, which is advantageous in this scenario as it is capable of handling complex nonlinear mapping while providing uncertainty bounds, see \cite{rasmussen2006gaussian}. Moreover, $K_c^{\mathrm{FF}}$ is modeled as:

 \vspace*{-.1cm}
\section{Feedforward enhanced dynamic regulation}
\label{Section:CommutationFeedforward}
%\vspace{-.1cm}

\begin{figure}[b]
\vspace*{-.5cm}
    \centering
\includegraphics[trim={.5cm 0.5cm 2.2cm 0cm}
    ,width=\linewidth]{Figures/CDCFIG3}
    \caption{Proposed dynamic regulation of the commutation frame using commutation feedback controller $K_c$ and feedforward controller $K_c^{\mathrm{FF}}$.}
    \label{fig:fullcontrolllopp}
\end{figure}

%To alleviate the burden of the commutation feedback controller $K_c$, the control interconnection is extended by a commutation feedforward controller $K_c^{\mathrm{FF}}$, as illustrated by Figure \ref{fig:fullcontrolllopp}, which corresponds to a position dependent feedforward selection of the initial model-based commutation algorithm. To construct the position dependent feedforward policy, local experiments have been performed on a state-of-the-art moving-magnet planar actuator prototype to obtain a set of locally optimized parameter vectors $\lbrace \Tilde{\eta}^* \rbrace_{i=1}^{n_{\tt p}}$. 
 
 %The optimal parameter vector set  $\lbrace \Tilde{\eta}^* \rbrace_{i=1}^{n_{\tt p}}$ can either be obtained by performing local gradient descent strategies, e.g. see Section \ref{Section_GRADDESC}, or by considering the output of the dynamic regulation at a grid of local positions, which is viable due to the integral action of the commutation feedback controller, which actively regulates the parameter $\Tilde{\eta}$, such that the rigid body feedback control forces are steered to zero.

%In order to achieve better performance of the dynamic regulation of the commutation frame with a more aggressive controller $K_c$, a commutation feedforward controller $K_c^{\mathrm{FF}}$ is introduced to the control interconnection as illustrated by Figure \ref{fig:fullcontrolllopp}. For design of the commutation feedforward, several considerations can be made. 
%For the design of the commutation feedforward, the static solution of the gradient descent based approach explained in Section \ref{Section_GRADDESC} can be considered. Alternatively, the spatial behavior of $\Delta(q_{\mathcal{T}}^{\mathcal{M}})$ can be taken into consideration, thus accounting for local discrepancies in the EM relationship that might limit performance of the MMPA system.
%In order to improve the performance of the dynamic regulation of the commutation frame using a more aggressive controller, a commutation feedforward controller $K_c^{\mathrm{FF}}$ is incorporated into the control interconnection, see Figure \ref{fig:fullcontrolllopp}. For commutation feedforward design, several approaches can be considered. First, the solution of the gradient descent based approach, i.e. the average solution of $\eta$ over the operating envelope, can be employed to account for static misalignment of the commutation frame, such that the commutation feedback only has to attenuate for local variations of the EM interaction. Alternatively, one can consider a position-dependent commutation feedforward which accounts for local discrepancies of $\Delta(q_{\mathcal{T}}^{\mathcal{M}})$ through the feedforward controller.

In order to enhance the performance of the dynamic regulation of the commutation frame with a more aggressive controller, a commutation feedforward controller $K_c^{\mathrm{FF}}$ is introduced to the control interconnection, as depicted in Figure \ref{fig:fullcontrolllopp}. The design of the commutation feedforward can be approached in various ways. First, the solution of the gradient descent-based method, explained in section \ref{Section_GRADDESC}, can be employed, thus accounting for static misalignment of the commutation frame. 
%This ensures that the commutation feedback only needs to compensate for local deviations in the EM interaction. 
Alternatively, one can aim at constructing a position dependent feedforward policy that attempts to capture the spatial characteristics of  $\Delta(q_{\mathcal{T}}^{\mathcal{M}}(t))$ in terms of learning the mapping of $q_{\mathcal{T}}^{\mathcal{M}} \rightarrow \eta$. To construct such a feedforward,
%controller can be considered,hich accounts for both misalignment of the commutation frame and local discrepancies in $\Delta(q_{\mathcal{T}}^{\mathcal{M}})$, introduced by coil pitch, eddy currents or other manufacturing imperfections.
%, or (ii) the local outputs of the commutation controller can be considered, which is viable due to the integral action of the commutation controller, thus taking the local EM discrepancies of $\Delta(q_{\mathcal{T}}^{\mathcal{M}})$ into consideration.  
%, as illustrated in Figure \ref{fig:fullcontrolllopp}. Several factors can be taken into account when designing the commutation feedforward. One approach is to consider the static solution of the gradient descent-based method explained in Section \ref{Section_GRADDESC}. Alternatively, the spatial characteristics of $\Delta(q_{\mathcal{T}}^{\mathcal{M}})$ can be considered to address local deviations in the EM relationship, which can limit the performance of the MMPA system.
%First, a static feedforward, obtained from the gradient descent approach explained in Section \ref{Section_GRADDESC} can be employed. Nonetheless, one can also learn the full spatial behavior of the 
%where $K_c^{\mathrm{FF}}$ corresponds to a position dependent feedforward selection of the scheduling dependence of the initial model-based commutation algorithm.
$n$ experiments are conducted with the aim of constructing a dataset $\mathcal{D}_n$, which consists of locally optimized parameter vectors as a function of the position, i.e.  $\mathcal{D}_n = \{ \eta^*(i),x_{\mathcal{T}}^{\mathcal{M}} (i),y_{\mathcal{T}}^{\mathcal{M}}(i)\}_{i=1}^n$. Moreover, this dataset is obtained by assessing the output of the dynamic regulator $K_c$ at $n$ local positions, which is facilitated by the integral action of the commutation controller. 
%Moreover, to obtain $\lbrace {\eta}_t^* \rbrace_{i=1}^N$, 
%two approaches can be considered. First, local gradient descent strategies can be applied for $N$ local positions using a similar methodology as elaborated on in Section \ref{Section_GRADDESC}. Secondly, 
%the output of the dynamic regulation $K_c$ can be evaluated for $N$ local positions, which is viable due to the integral action of the commutation feedback controller. The experimental results are illustrated in Figure \ref{fig:OptimalEta} from which several observations can be made. First, it is observed that a static misalignment of the commutation frame is present. This is concluded from the fact that the mean of the optimized parameters is not equivalent to zero. Additionally, it is observed that the spatial behavior of $\lbrace {\eta}_t^* \rbrace_{i=1}^N$ is dominated by position dependent nonlinear effects, originating from coil pitch, eddy currents and other manufacturing inaccuracies. 
%Furthermore, the optimized parameter vector sets $\lbrace {\eta}_{t}^*(i) \rbrace_{i=1}^{\mathrm{ N}}$ are obtained by assessing the output of the dynamic regulator $K_c$ at $\mathrm{N}$ local positions. This approach is feasible due to the integral action of the commutation feedback controller. 
Figure \ref{fig:OptimalEta} illustrates the experimental findings in terms of its $\eta_x^*$ and $\eta_y^*$ components, revealing various noteworthy observations. Primarily, a static misalignment of the commutation frame is evident, as inferred by the non-zero mean of the optimized parameters. Secondly, it is observed that the spatial behavior is dominated by nonlinear effects, which can be attributed to coil pitch, eddy currents, and other unmodelled EM relations.

%In order to estimate $\Delta(q_{\mathcal{T}}^{\mathcal{M}})$ from these local experiments, the Gaussian Process (GP) framework is employed,
%for the  construction of a commutation feedforward controller $K_c^{\mathrm{FF}}$. 
%which exhibits several practical advantages over alternative approaches, such as implementation of look-up tables and artificial neural networks. 
\begin{figure}[t]
\vspace{7pt}
    \centering  \includegraphics[trim={1.5cm 0.5cm 2.5cm 1.5cm}
    ,height=3.5cm,width=.9\linewidth]{Figures/untitled}
    \caption{Optimized $\lbrace {\eta}_{t}^*(i) \rbrace_{i=1}^\mathrm{N}$ in terms of its $x$ and $y$ components, obtained from experiments using a state-of-the-art MMPA prototype.}
    \label{fig:OptimalEta}
    \vspace*{-.6cm}
\end{figure}
%Alternatively, the commutation feedforward $K_c^{\mathrm{FF}}$ can be modeled using the Gaussian Process (GP) framework, which exhibits several practical advantages over alternative approaches. 
%First, the GP framework provides confidence bounds on the posterior estimate, which guarantees reliability of the corresponding learning-based estimator. Secondly, the GP framework is capable to efficiently capture nonlinear mappings by means of appropriate kernel selection which has proven to be effective in practise, see \cite{proimadis2021learning, poot2022gaussian, VANMEER2022302}. 
To estimate the EM discrepancies $\Delta(q_{\mathcal{T}}^{\mathcal{M}})$ based on the outcomes of these local experiments, we adopt the \emph{Gaussian Process} (GP) framework, which offers a number of practical benefits compared to other techniques, including lookup tables and artificial neural networks. First, the GP framework provides confidence bounds on the posterior estimate, ensuring the accuracy of the corresponding learning-based estimator. Additionally, the GP framework is able to effectively model nonlinear mappings through appropriate kernel selection, which has been demonstrated to be effective in practical settings, see  \cite{proimadis2021learning, poot2022gaussian, VANMEER2022302}. Using the GP framework, the local variations of the EM relations are modelled as:
%The aim of the GP framework is to estimate the local variations of the commutation frame using the training data set $\mathcal{D}_n = \{\eta_t^*(i) ,w(i)\}_{i=1}^{\mathrm{N}}$, where the GP input is defined as $w(i) = [x_{\mathcal{T}}^{\mathcal{M}}(i), y_{\mathcal{T}}^{\mathcal{M}}(i)]^\top$ observations which are assumed to be contaminated with noise, i.e.:
\vspace*{-.2cm}
\begin{equation}
    \eta_m^*(i) = \Delta_m(w(i)) + \epsilon(i), \quad \text{with} \ \epsilon\sim \mathcal{N}(0,\sigma_\epsilon^2),
\end{equation}

\vspace*{-.2cm}

\noindent where $i$ corresponds to the time index, $\eta_m^*$ denotes the $m$-th element of the $\eta^*$ vector and $\epsilon$ is assumed to be independent and identically distributed (i.i.d.) white Gaussian noise with variance $\sigma_\epsilon^2$, see \cite{dekking2005modern}.  %The local commutation frame discrepancies $\Delta_t$, which are parameterized with respect to input vector $w(i)=[x_{\mathcal{T}}^{\mathcal{M}}(i) \ y_{\mathcal{T}}^{\mathcal{M}}(i)]^\top $, are modelled as a GP, see \cite{williams2006gaussian}:
The local commutation frame discrepancies $\Delta_m$ are parameterized by an input vector $w(i)$, where $w(i)=[x_{\mathcal{T}}^{\mathcal{M}}(i) \ y_{\mathcal{T}}^{\mathcal{M}}(i)]^\top$, and are modelled as a GP following the approach presented in \cite{williams2006gaussian} as:
\vspace*{-.2cm}
\begin{equation}
     \Delta_m(w) \sim \mathcal{GP} (0,\kappa_m(w,w'))
\end{equation}

\vspace*{-.2cm}

\noindent where the covariance is fully characterized by the kernel functions $k_m(w,w')$. Moreover, the joint Gaussian distribution of the observed target values $\{\eta_m^* (i)\}_{i=1}^n$ and the commutation frame discrepancies for a given test input $w^* \in \mathbb{R}^2$ is given by:
\vspace*{-.1cm}
\begin{equation}
\resizebox{.89\hsize}{!}{$
    \begin{bmatrix}
        \lbrace{\eta}^*_m (i)\rbrace_{i=1}^n \\ \Delta_m(w^*) 
    \end{bmatrix} \sim \mathcal{N}\left(0,\left [\begin{array}{cc}
        \kappa_m(W,W) + \sigma_\epsilon^2 I & \kappa_m(W,w^*) \\
        \kappa_m^\top(W,w^*) & \kappa_m(w^*,w^*)
    \end{array}\right] \right),$}
    \label{jointdis}
\end{equation}

\vspace*{-.1cm}

\noindent where $W \in {\mathbb{R}^{2n}}$ corresponds to the observed input data points of $\mathcal{D}_n$. Based on $\mathcal{D}_n$, the predictive a posteriori distribution computed from (\ref{jointdis}) is given by:
\vspace*{-.1cm}
\begin{equation}
\resizebox{.85\hsize}{!}{$
    \begin{split}
    \hat{\Delta}_m(w^*)&\triangleq  {{}\mathbb{E}}\{\Delta_m(w^*) | \mathcal{D}_n,w^* \}=\Psi \lbrace{\eta}^*_m(i) \rbrace_{i=1}^n \\
       \mathrm{cov}\{ \Delta_m(w^*) | \mathcal{D}_n,w^*\} &= \kappa_m(w^*,w^*)-\Psi \kappa_m(W,w^*)
    \end{split},
    \label{GPposterior}$}
\end{equation}

\vspace*{-.1cm}

\noindent where:
\vspace*{-.25cm}
\begin{equation}
    \Psi = \kappa_m^\top(W,w^*)(\kappa_m(W,W)+\sigma_\epsilon^2 I)^{-1} 
\end{equation}

\vspace*{-.15cm}

%Moreover, the feedforward controller $K_{c}^{\mathrm{FF}}$ is constructed from the mean of the posterior estimates, resulting the a diagonal GP-based feedforward:
Moreover, the construction of the feedforward controller $K_{c}^{\mathrm{FF}}$ involves using the mean of the \emph{posterior GP distribution}, leading to a diagonal Gaussian process-based feedforward control strategy:
\vspace*{-.1cm}
\begin{equation}
    K_c^{\mathrm{FF}} = \mathrm{diag}(\{ \hat{\Delta}_m (w^*) \}_{m=1}^{n_\eta})
\end{equation}

\vspace*{-.2cm}

%\noindent where due to the high precision positioning of MMPA systems, the test input $w^* = [x_{\mathcal{T}}^{\mathcal{M}} \ y_{\mathcal{T}}^{\mathcal{M}} ]^\top \approx [r_x \ r_y]^\top$, allowing for implementation of the GP-based estimator as feedforward controller as illustrated in Figure \ref{fig:fullcontrolllopp}.

\noindent where it is reasonable to assume that the test input is $w^* \approx [r_x \ r_y]^\top$. Based on this assumption, the Gaussian process-based estimator can be employed as a feedforward controller, as depicted in Figure \ref{fig:fullcontrolllopp}.

Achieving a precise and reliable estimator $\hat{\Delta}_m$ hinges on the careful selection of the kernels $k_m(w,w')$, which determine the \emph{Hilbert space} within which the mean of (\ref{GPposterior}) is searched for as a true estimate of $\Delta(q_{\mathcal{T}}^{\mathcal{M}})$. In light of the experimental findings presented in Figure \ref{fig:OptimalEta}, we propose the adoption of the following kernel selection as a baseline approach:
%To establish an accurate estimator $\hat{\Delta}_t$, selection of the kernels $k_t(w,w')$ is of great importance as it defines the \emph{Hilbert space} in which the mean of (\ref{GPposterior}) as a true estimate of $\Delta(q_{\mathcal{T}}^{\mathcal{M}})$ is searched for. Based on the experimental results shown in Figure \ref{fig:OptimalEta}, the following baseline kernel selection is proposed:
\vspace*{-.2cm}
\begin{equation}
\resizebox{.88\hsize}{!}{$
    k_m(w,w') = \sigma_{m_1}^2\text{exp} \Bigg(-\sum_{v=1}^2 \frac{||w_v-w_v'||_2^2}{\sigma^2_{m_{v+1}}}- \sum_{v=1}^2 
    \frac{2\text{sin}^2\left(\frac{w_v-w_v'}{2} \right)}{\sigma^2_{m_{v+3}}     } \Bigg)
  %  \frac{\mathrm{sin}^2\left(\frac{\pi (w_v-w_v')}{p_{\mathrm{sin}}} \right)}{\lambda_v^{\mathrm{sin}}} \Bigg)
  $},
    \label{kernel}
\end{equation}

\vspace*{-.1cm}

%Using the GP framework, the optimal parameter vectors are modelled as:
%\begin{equation}
%     \eta_t^*(w) \sim \mathcal{GP} (0,\kappa_t(w,w')), \quad \forall t \in [1 \  n_{{\eta}}],
%\end{equation}

%\noindent where the covariance of $\eta_t^*(w)$ is fully characterized by the kernel functions $\kappa_t(w,w')$, which are parameterized with respect to the input vector $\{w_i\}_{i=1}^{\mathrm{N}}$, where 
%$w = \begin{small}[x_{\mathcal{T}}^{\mathcal{M}}  \ y_{\mathcal{T}}^{\mathcal{M}}]^\top  \end{small}$, see Figure \ref{fig:OptimalEta}. Moreover, by assuming a zero-mean prior distribution on the estimator $\hat{\eta}_t^*(w)$, see \cite{williams2006gaussian}, the joint Gaussian distribution between the observed target values $\lbrace{\eta}_t^* \rbrace_{i=1}^{n_{\tt p}}$ and the estimator $\hat{\eta}_t^*(w^*)$ at a given  test input $w^* \in \mathbb{R}^{2}$ is of form: 
%\begin{equation}
%\resizebox{.89\hsize}{!}{$
%    \begin{bmatrix}
%        \lbrace{\eta}^*_t \rbrace_{i=1}^N \\ \hat{\eta}^*_t(w^*) 
%    \end{bmatrix} \sim \mathcal{N}\left(0,\left [\begin{array}{cc}
 %       \kappa_t(W,W) + \sigma_e^2 I & \kappa_t(W,w^*) \\
  %      \kappa_t^\top(W,w^*) & \kappa_t(w^*,w^*)
  %  \end{array}\right] \right),$}
   % \label{jointdis}
%\end{equation}

%\noindent where $W\in \mathbb{R}^{n_{\tt p}\times 2}$ is the input data vector and $\sigma_e^2$ denotes the variance of the noise acting on the observations. 
%Note that the assumption of a zero-mean prior is valid if the static misalignment, e.g. the mean of $\lbrace\tilde{\eta}_t^* \rbrace_{i=1}^{n_{\tt p}}$, is feedforwarded independently of the GP. Therefore, the GP solely captures the variance, e.g. the position dependent effects. 
%The resulting posterior distribution corresponds to:
%\begin{equation}
%    \begin{split}
%       \mathop{{}\mathbb{E}}[\hat{\eta}^*_t(w^*) ]&=\Psi \lbrace{\eta}^*_t \rbrace_{i=1}^N \\
 %      \mathrm{cov}[\hat{\eta}^*_t(w^*) ] &= \kappa_t(w^*,w^*)-\Psi \kappa_t(W,w^*)
 %   \end{split},
 %   \label{GPposterior}
%\end{equation}

%\noindent where:
%\begin{equation}
%    \Psi = \kappa_t^\top(W,w^*)(\kappa_t(W,W)+\sigma_e^2 I)^{-1} 
%\end{equation}

%Moreover, the GP model is statistically characterized by the posterior distribution expressed by (\ref{GPposterior}). 
%Based on the discussion so far, there is one question yet to be answered. Namely, how to construct the kernel functions $\kappa_t(w,w')$.
%Based on the experimental results illustrated by Figure \ref{fig:OptimalEta}, the kernels are defined as a product of a periodic kernel, see \cite{mackay1998introduction}, which captures the effects of manufacturing inaccuracies and higher harmonics, and a radial basis function kernel, see \cite{neal2012bayesian}, which captures the nonlinear remnant effects that were not considered in the construction of the model-based commutation. Moreover, the kernel functions $k_t(w,w')$ are given by:
%\begin{table}[b]
%\begin{center}
%\caption{Validation of the estimated GP-based commutation feedforward controller.}
%\label{BFRResultstable}
%\scalebox{.95}{
%\begin{tabular}{|l|l|l|}
%\hline
%    & \textbf{Training set} & \textbf{Validation set} \\ \hline
%\textbf{BFR} $  \hat{\Delta}_x$ [\%]&            89.80  &   85.18             \\ \hline
%\textbf{BFR} $ \hat{\Delta}_y$  [\%]&            84.84  &      83.34          \\ \hline
%\end{tabular}}
%\end{center}
%\end{table}
\noindent 
where $w_v$ corresponds to the $v_{\mathrm{th}}$ element of the input vector, i.e. $[x_{\mathcal{T}}^\mathcal{M}$ \ $y_{\mathcal{T}}^\mathcal{M}]^\top$. The proposed baseline kernel is composed of a combination of a periodic kernel \cite{mackay1998introduction} and a radial basis function kernel \cite{neal2012bayesian}. The former kernel accounts for the manufacturing inaccuracies and higher harmonics, while the latter captures the non-linear residual EM effects that were not incorporated in the ideal model-based commutation.

The proposed kernel functions, given by (\ref{kernel}), are characterized by 6 hyper-parameters, namely $[\sigma_{m_1}^2 \ \hdots \ \sigma_{m_5}^2 \ \sigma_\epsilon^2]^\top$. These parameters are tuned to suit the specific estimation problem by adjusting the Hilbert space accordingly. Hence, the variance $\sigma_\epsilon^2$ is jointly optimized with the kernel hyperparameters. The computation of these hyperparameters is carried out via the maximization of the marginalized likelihood, using the available training set $\mathcal{D}_n$, as outlined in \cite{rasmussen2006gaussian}. For validation of the GP-based estimators, the {Best Fit Ratio} (BFR) is used, see \cite{simpkins2012system}, which is given by:
\vspace*{-.1cm}
\begin{equation}
\resizebox{.85\hsize}{!}{$
    \mathrm{BFR} = 100 \% \cdot \mathrm{max}\left(1 - \frac{\left | \left | \lbrace  {{\eta}_m^*(i)} \rbrace_{i=1}^{n} - \lbrace \hat{\Delta}_m (w(i)) \rbrace_{i=1}^{n} \right | \right |_2}{\left |\left |\lbrace {{\eta}_m^*(i)}  \rbrace_{i=1}^{n} - { \Bar{{\eta}}_m^*} \right |\right |_2} ,0\right)$},
\end{equation}

\vspace*{-.1cm}

\noindent where \begin{small}${\Bar{{\eta}}_m^*}$\end{small} corresponds to the sample mean of \begin{small}$\lbrace {{{\eta}}_m^*(i)} \rbrace_{i=1}^{n}$\end{small}. 
Table \ref{table:GPResults} presents the BFR results of the GP-based feedforward in terms of its $x$ and $y$ components, relative to both its training dataset $\mathcal{D}_n$ and a novel validation dataset. The findings indicate that the proposed baseline kernels, namely the combination of a periodic kernel and a radial basis function kernel, enable the GP-based estimator to accurately capture the local discrepancies of the EM relationships, thus rendering it a reliable commutation feedforward controller.

%The BFR results of the GP-based feedforward in terms of its $x,y$ components, both in comparison to its training data set and a fresh validation data set, are presented in Table \ref{table:GPResults}. From Table \ref{table:GPResults} it is observed that the GP-based estimator is well capable of capturing the local EM discrepancies of the MMPA system using the proposed baseline kernels, comprised of a periodic kernel and a radial basis function kernel. Therefore, allowing for implementation of the GP-based estimator as an accurate commutation feedforward controller.

\vspace*{-.2cm}
\begin{table}[h]
\begin{center}
\caption{Validation of estimated GP-based feedforward on a validation set. }
\label{table:GPResults}
\scalebox{.9}{
\begin{tabular}{|c|c|c|}
\hline
             & \textbf{Training data set} & \textbf{Validation data set} \\ \hline
\textbf{BFR} $\hat{\Delta}_x (w^*)$ [\%] & 89.80                 & 85.18                   \\ \hline
\textbf{BFR} $\hat{\Delta}_y(w^*)$[\%]  & 84.84                & 83.34                   \\ \hline
\end{tabular}}
\end{center}
\end{table}



%Moreover, the BFR results of the GP-based feedforward, compared to the training set and a fresh validation set, are presented in Table \ref{BFRResultstable}.

%The proposed kernel functions, expressed by (\ref{kernel}), are parameterized using 6 hyper-parameters, e.g. \begin{small}$[\sigma_{t_1}^2 \ \hdots \ \sigma_{t_5}^2 \ \sigma_\epsilon^2]^\top$\end{small}, to adjust the Hilbert space to the estimation problem. Therefore, the variance $\sigma_\epsilon^2$ is optimized together with the kernel hyperparameters. The computation of these hyperparameters is realized through maximization of the marginalized likelihood using the collected training set, see \cite{rasmussen2006gaussian}.

%From Table \ref{BFRResultstable} it is concluded that the BFR performance deteriorates a significant amount when considering a fresh dataset. This can be explained by the fact that the metrology frame, e.g. the frame on which the position sensors are mounted, exhibits drift as it floats on top of air mounts, see \cite{Proimadis-phd}. For improvement of the GP-based feedforward, these coordinate frame deviations should be considered as additional GP inputs. Nonetheless, mismatch between the GP-based feedforward and the optimized parameter vector set will be attenuated for by the commutation feedback controller $K_c$.  

%Based on the findings presented in Table \ref{BFRResultstable}, it can be inferred that the performance of the GP-based commutation feedforward experiences a considerable decline when compared to a fresh validation dataset. This outcome can be attributed to the drift of the measurement frame, which is mounted on top of air mounts to suppress the effects of floor disturbances, see \cite{Proimadis-phd}. To enhance the accuracy of the GP-based feedforward, it is recommended that the coordinate frame discrepancies, which are measurable, are taken into consideration as supplementary inputs for the GP model. Nonetheless, any mismatch between the GP-based feedforward and the actual optimized parameter vector set will be mitigated by the integral action of the commutation feedback controller. 


\vspace*{-.6cm}
\section{Experimental validation}
\label{Section_Experimental_Validation}

\subsection{MMPA prototype system}

The MMPA system, depicted in Figure \ref{fig:MMPA}, consists of three main components: a stator base on which a double layer coil array is mounted, a lightweight translator on which a Hallbach array, comprised of 281 permanent magnets, is mounted, and a metrology frame on which 9 laser interferometers are mounted to measure the relative displacement of the translator with respect to the metrology frame. The double layer coil array of the stator base consists of 160 coils of which 40 coils are simultaneously activated at every time instant using 40 power amplifiers, allowing for levitation and propulsion of the magnet plate in 6 DoF. 
%The considered MMPA system, which is depicted in Figure \ref{fig:MMPA}, is comprised of three main components: (i) the stator base, (ii) the translator and (iii) the metrology frame. The stator base is a double layer coil array, consisting of 160 coils of which 40 coils are simultaneously activated at every time instant using 40 power amplifiers, depending on the relative position of the translator. Proper actuation of the coils offers both the means of levitation and propulsion of the magnet plate in 6 DoF. The translator, comprised of 281 permanent magnets structured in a Hallbach array, is designed to be lightweight to facilitate high accelerations. The metrology frame, which rests on air mounts to suppress the effects of floor disturbances, is used as a global reference frame to assess position tracking performance of the moving-magnet.
%On the metrology frame, 9 laser interferometers are mounted to measure the relative 
\begin{figure}[t]
\vspace{7pt}
    \centering
    \includegraphics[height=4cm,width=\linewidth]{Figures/NAPAS.pdf}
            \vspace*{-.6cm}
    \caption{Photograph of a moving-magnet planar actuator system prototype.}
    \label{fig:MMPA}
        \vspace*{-.6cm}
\end{figure}
%displacement of the translator with respect to the metrology frame.
%Additionally, two sets of eddy current sensors (ECs) are mounted on the metrology frame as auxiliary measurement systems, where one set of ECs is used for system initialization, while the second set of ECs captures displacements between the stator and the metrology frame. 
For a more comprehensive description of the MMPA prototype, readers can refer to \cite{proimadis2021active}.

\begin{figure}[b]
\vspace*{-.8cm}
    \centering
    \includegraphics[trim={2cm 1cm 3.5cm 5cm}
    ,height=2.3cm,width=.9\linewidth]{Figures/Reference.eps}
    \vspace*{-.2cm}
    \caption{Normalized motion profiles considered for experimental validation, where $a_{\mathrm{max}}=5\frac{m}{s^2}$, $v_{\mathrm{max}}=0.1\frac{m}{s}$ and $x_{\mathrm{max}}=0.05{m}$.}
    \label{fig:motionprofile}
\end{figure}

\vspace*{-.1cm}

\subsection{Experimental results}
%\begin{table}[h]
%\begin{center}
%\caption{Validation of the estimated GP-based commutation feedforward controller.}
%\label{BFRResultstable}
%\begin{tabular}{|l|l|l|}
%\hline
%    & \textbf{Training set} & \textbf{Validation set} \\ \hline
%\textbf{BFR} $ K_{c_1}^{\mathrm{FF}}$ [\%]&            89.80  &   85.18             \\ \hline
%\textbf{BFR}  $ K_{c_2}^{\mathrm{FF}}$ [\%]&            84.84  &      83.34          \\ \hline
%\end{tabular}
%\end{center}
%\end{table}

To illustrate the functionality of the proposed approaches, i.e. static calibration of the commutation frame (Section \ref{Section_GRADDESC}) and GP-feedforward enhanced dynamic regulation of the commutation frame (Section \ref{Section_Dynamic_Regulation}-\ref{Section:CommutationFeedforward}), time-domain experiments have been performed on the MMPA prototype. For implementation of the proposed approaches on the experimental prototype, the static EM calibration approach, presented in Section \ref{Section_GRADDESC}, was employed based on 36 local positions, leading to an optimal static commutation frame alignment parameter. For implementation of the GP-feedforward enhanced dynamic regulation of the commutation frame, the GP-based feedforward, discussed in Section \ref{Section:CommutationFeedforward}, is implemented in conjunction with the dynamic regulation proposed in Section \ref{Section_Dynamic_Regulation}. The commutation feedback controllers are designed based on (\ref{PICOTROLLER}), where the bandwidth of the commutation feedback controllers is set to be 100 times smaller than the bandwidth of the corresponding controllers of the position loop.

To assess performance in lithographic applications, the Moving-Average (MA) performance metric is introduced, see \cite{Butler}, which allows to analyse the low-frequent spectral content of the position tracking error during the lithographic exposure process, which takes place in the constant velocity interval of the motion profile. The MA performance metric is given by:
\vspace*{-.3cm}

\begin{equation}\begin{small}
        \mathrm{MA}(t) = \frac{1}{T}\int_{t-\frac{T}{2}}^{t+\frac{T}{2}} e(\tau)d\tau,
\end{small}
        \label{MAFILTER}
\end{equation}

\newpage 

\noindent where $e$ corresponds to the raw position tracking error and $T=0.0144s$ is the exposure time of a single point on the wafer. For experimental validation of the proposed approaches, a lithographic scanning motion is performed in the $y_{\mathcal{T}}^{\mathcal{M}}$ direction using the normalized motion profiles presented in Figure \ref{fig:motionprofile}, where $a_{\mathrm{max}}=5 \frac{m}{s^2}$, $v_{\mathrm{max}}=0.1\frac{m}{s}$, and $x_{\mathrm{max}}=0.05{m}$. The experimental results, e.g. the MA filtered position tracking errors in $y_{\mathcal{T}}^{\mathcal{M}}$-direction during the constant velocity interval of the motion profile, are illustrated in Figure \ref{fig:EXPResults}. Specifically, the blue graph depicts the position tracking error in  of the conventional rigid body control structure, the red graph shows the position tracking error using the proposed static EM alignment approach and the green graph corresponds to the position tracking error of the proposed GP-feedforward enhanced dynamic regulation of the commutation frame. Based on Figure \ref{fig:EXPResults}, several observations can be made. First, the static calibration approach proposed in Section \ref{Section_GRADDESC} is not sufficient at improving position tracking performance. This can be explained by the behavior of $\Delta(q_{\mathcal{T}}^{\mathcal{M}})$, which is dominated by nonlinear effects that can be attributed to coil pitch,  eddy currents and other manufacturing imperfections as illustrated in Figure \ref{fig:OptimalEta}. Secondly, it is observed that implementation of the GP-feedforward enhanced dynamic regulation of the commutation frame improves the peak MA error from $\approx 18.7$ nm to $\approx 6.6$ nm, resulting in a relative reduction of the peak MA error of 64.71\%, thus illustrating the potential of the proposed approach in terms of improving the EM relations of the ideal model-based commutation algorithm. 


%Moreover, the experiments involved executing a lithographic scanning motion in the $y_{\mathcal{T}}^{\mathcal{M}}$-direction using the normalized motion profiles illustrated in Figure \ref{fig:motionprofile}, where $a_{\mathrm{max}}=5 \frac{m}{s^2}$, $v_{\mathrm{max}}=0.1\frac{m}{s}$, and $x_{\mathrm{max}}=0.05{m}$. The corresponding experimental results are illustrated in Figure \ref{fig:EXPResults}, where the blue graph corresponds to the MA filtered position tracking error in $y_{\mathcal{T}}^{\mathcal{M}}$-direction of the conventional rigid body control structure, the red graph denotes the MA filtered position tracking error of the MMPA using the proposed static EM alignment approach and the green graph corresponds to the MA filtered position tracking error of the GP-feedforward enhanced dynamic regulation of the commutation frame. 


%The experimental results, i.e. the MA filtered position tracking error during the constant velocity interval of the lithographic scan in $y$-direction, is illustrated in Figure \ref{fig:EXPResults}. From Figure \ref{fig:EXPResults} various observations can be made.





%During the scanning motion, it is of high importance that the position tracking error in the constant velocity interval is minimized as this is were the lithographic exposure process takes place.


%is assessed with using the \emph{Moving-Average} (MA) performance metric, which corresponds to:
%\vspace*{-.3cm}

%\begin{equation}
%        \mathrm{MA}(t) = \frac{1}{T}\int_{t-\frac{T}{2}}^{t+\frac{T}{2}} e(\tau)d\tau,
%\end{equation}

%\vspace*{-.2cm}



%To compare the performance of the proposed techniques, the static EM calibration from Section \ref{Section_GRADDESC} was carried out using 36 local positions, i.e., $n_{\tt p} = 36$. For the GP-feedforward enhanced dynamic regulation of the commutation frame, the GP-based feedforward was implemented in conjunction with the dynamic regulation proposed in Section \ref{Section_Dynamic_Regulation}. Here, the commutation feedback controller's bandwidth was selected to be 100 times lower than the corresponding position controllers' bandwidth. 
% To evaluate the improvement in closed-loop performance with respect to the commutation error's contribution to the position tracking error, the Moving-Average (MA) performance metric was employed, as defined in \cite{Butler}.
%Furthermore, to evaluate the closed-loop performance improvement with respect to the commutation error's contribution to the position tracking error, the Moving-Average (MA) performance metric is introduced, as described in \cite{Butler}, where the MA performance metric is defined as:
%by performing a scanning motion in $y_{\mathcal{T}}^{\mathcal{M}}$-direction using the normalized motion profiles illustrated by Figure \ref{fig:motionprofile}, where $a_{\mathrm{max}}=5\frac{m}{s^2}$, $v_{\mathrm{max}}=0.1\frac{m}{s}$ and $x_{\mathrm{max}}=0.05{m}$. To compare the performance of the proposed approaches, the static EM calibration proposed in Section \ref{Section_GRADDESC} is performed using 36 local positions, e.g. $n_{\tt p} = 36$. For the GP-feedforward enhanced dynamic regulation of the commutation frame, the GP-based feedforward is implemented on the system in combination with the dynamic regulation proposed in Section \ref{Section_Dynamic_Regulation}, where the bandwidth of the commutation feedback controller was chosen to be 100 times smaller than the bandwidth of the corresponding position controllers. Moreover, to assess the increase in closed-loop performance with respect to the contribution of the commutation error towards the position tracking error, the \emph{Moving-Average} (MA) performance measure is introduced, see \cite{Butler}, which is given by:
%To compare the performance of the proposed approaches with respect to the conventional rigid body control interconnection, a performance measure, which is used to assess the accuracy of the lithographic exposure process in the semiconductor industry, is introduced.  
%For the static calibration of the commutation frame, 25 local positions were considered for the gradient descent based optimization.
%For the static regulation of the commutation frame, 25 local positions are considered for the gradient-descent based algorithm, resulting in a static commutation feedforward compensating for the static misalignment of the commutation frame. For dynamic regulation of the commutation frame, commutation feedback controllers with a bandwidth of 0.1 $Hz$ are considered in combination with the proposed GP-based feedforward. The BFR performance of the corresponding GP-based estimator on a fresh validation set is illustrated in Table \ref{BFRResultstable} 
%effectiveness of the approaches, experiments have been performed on a MMPA prototype, see Figure \ref{fig:MMPA}.
%To validate the approaches proposed in Section \ref{Section_GRADDESC} and Section \ref{Section_Dynamic_Regulation}-\ref{Section:CommutationFeedforward}, experiments have been performed on a state-of-the-art MMPA prototype, see Figure \ref{fig:MMPA}. In the semiconductor industry, accuracy of the lithographic exposure process, which takes place during constant velocity of the motion profile, is assessed by two performance measures: Moving Average (MA) and Moving Standard Deviation (MSD), see \cite{Butler}. 
%To validate the improvements in commutation, e.g. low-frequent position tracking error performance, only MA performance is considered in this paper. The MA performance measure is given by:


%\noindent where $T$ corresponds to the exposure window. Moreover, to evaluate the performance of the proposed approaches, a scan is performed in $y$-direction using the $4^{\mathrm{th}}$ order motion profile  illustrated by Figure \ref{fig:motionprofile}, which is generated based on the methodology proposed by \cite{208Lambrechts}. The experimental results are illustrated in Figure \ref{fig:EXPResults}, where the top Figure illustrates the MA filtered position tracking error in $y$-direction and bottom Figure illustrates the cumulative power spectral density of the raw position tracking error in $y$-direction. From Figure \ref{fig:EXPResults} several observations can be made. First it is observed that the approach presented by Section \ref{Section_GRADDESC} is not sufficient to improve position tracking performance. This can be explained by dominant nonlinear behavior of $\Tilde{\eta}^*$, see Figure \ref{fig:OptimalEta}, which is not accounted for by the GD-based solution. Secondly, it is observed that implementation of the dynamic regulation, which is designed to have a commutation feedback control bandwidth of 0.1Hz, in combination with the learning-based commutation feedforward drastically improves low-frequent position tracking performance, thus illustrating the effectiveness of the proposed approach.

\begin{figure}[t]
\vspace{7pt}
    \centering
    \includegraphics[trim={3cm 0cm 3cm 0cm}
    ,height=4cm,width=.95\linewidth]{Figures/TIMERESULT}
    \vspace*{-.4cm}
    \caption{MA filtered position tracking error in $y$-direction during constant velocity interval of the motion profile with: (\Large\textcolor{blue}{-}\footnotesize) No commutation control, (\Large\textcolor{red}{-}\footnotesize) Static regulation of the commutation frame (Section \ref{Section_GRADDESC}) (\Large\textcolor{green}{-}\footnotesize)  Dynamic regulation of the commutation frame (Section \ref{Section_Dynamic_Regulation}) combined with a learning-based commutation feedforward (Section \ref{Section:CommutationFeedforward}).}
    \label{fig:EXPResults}
    \vspace*{-0.5cm}
\end{figure}



\section{Conclusions}
\label{Section_Conclusions}

This paper introduces a methodology to enhance the ideal model-based commutation by regulating the commutation frame both statically and dynamically, thereby aiming at minimization of the cost function expressed by (\ref{costfunctionpar2}). It is found that the static optimization of the commutation frame is not sufficient in terms of improving the position tracking performance of the MMPA prototype, as it does not account for the highly nonlinear EM behavior induced by coil pitch, eddy currents, and other unmodelled EM effects. Nonetheless, the GP-feedforward enhanced dynamic regulation of the commutation frame demonstrates great potential as it reduces the peak MA error by 64.71\% compared to conventional rigid body control interconnection. This is achieved by the GP-based feedforward's ability to effectively capture the nonlinear behavior of the EM relations, in combination with the integral action of the dynamic regulation, which effectively attenuates any possible EM misalignment, thus enabling high-precision motion control of the mover.

%This paper presents an approach for static and dynamic regulation of the commutation frame with the aim of improving the ideal model-based commutation, such that the cost function expressed by (\ref{costfunctionpar2}) is minimized. Experimental results illustrate that considering a static EM misalignment is not sufficient in terms of improving position tracking performance of the MMPA prototype. This can be attributed highly nonlinear EM behavior, introduced by coil pitch, eddy currents and other unmmodelled EM effects, which is not accounted for in the gradient-descent based static optimization of the commutation frame.
%Nonetheless, the GP-feedforward enhanced dynamic regulation of the commutation frame proves to be a promising approach as it is able to reduce the peak MA error by 64.71\% compared to the conventional control interconnection. This can be attributed to the combination of the GP-based feedforward, which is able to effectively capture the nonlinear behavior of the EM relations in combination with the integral action of the dynamic regulation, which actively attenuates for any possible EM misalignment, thus allowing for high-precision motion control of the mover.
%two strategies for improved commutation of moving-magnet planar actuator systems. First, a gradient-descent based strategy is presented which allows for compensation of static misalignment between the moving-magnet and the coil arrays. Nonetheless, due to the presence of manufacturing inaccuracies and remnant effects, this approach is not sufficient for improving overall position tracking performance of the MMPA system. Secondly, a novel dynamic regulation approach is presented, consisting of a commutation feedback controller and a learning-based commutation feedforward, which allows for active regulation of the scheduling parameter of the model-based commutation algorithm. Moreover, the dynamic regulation approach has proven to be efficient in practise through experimental validation to attenuate for effects of misalignment between the moving-magnet and the coil arrays, manufacturing inaccuracies and remnant effects that were not taken into account in the model-based commutation algorithm, allowing for increased closed-loop performance of MMPA systems respectively.
 
\addtolength{\textheight}{-12cm}   % This command serves to balance the column lengths
                                  % on the last page of the document manually. It shortens
                                  % the textheight of the last page by a suitable amount.
                                  % This command does not take effect until the next page
                                  % so it should come on the page before the last. Make
                                  % sure that you do not shorten the textheight too much.

%%%%%%%%%%%%%%%%%%%%%%%%%%%%%%%%%%%%%%%%%%%%%%%%%%%%%%%%%%%%%%%%%%%%%%%%%%%%%%%%



%%%%%%%%%%%%%%%%%%%%%%%%%%%%%%%%%%%%%%%%%%%%%%%%%%%%%%%%%%%%%%%%%%%%%%%%%%%%%%%%



%%%%%%%%%%%%%%%%%%%%%%%%%%%%%%%%%%%%%%%%%%%%%%%%%%%%%%%%%%%%%%%%%%%%%%%%%%%%%%%%
%\section*{APPENDIX}
\bibliographystyle{ieeetr}       
\bibliography{MyBib} 




\end{document}
