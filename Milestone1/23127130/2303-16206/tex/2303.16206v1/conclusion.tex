\section{Conclusion}
\label{sec:conclusion}

We propose a novel iterative encoder-based method, \method{}, for image steganography. The LISO encoder is designed to learn the update rule of a general gradient-based optimization algorithm for steganography, and is trained simultaneously with a corresponding decoder. 
It learns a more efficient dynamic update rule for steganography when compared to PGD or L-BFGS, and the learned decoder is particularly suitable for further L-BFGS optimization.
\method{} also implicitly learns the manifold of images and therefore produces high-quality steganographic images. 
\method{} achieves state-of-the-art results for steganography, while being fairly fast. 
It is also flexible and can incorporate additional constraints, like producing JPEG-resistant steganographic images or making steganographic images undetectable by specific steganalysis systems. In the future, we plan to investigate ways to increase image quality while maintaining low error for JPEG-resistant steganography. 

\subsubsection*{Acknowledgements}
This research is supported by grants from DARPA AIE program, Geometries of Learning (HR00112290078), DARPA Techniques for Machine Vision Disruption grant (HR00112090091), the National Science Foundation NSF (IIS-2107161, III1526012, IIS-1149882, and IIS-1724282), and the Cornell Center for Materials Research with funding from the NSF MRSEC program (DMR-1719875). We would like to thank Oliver Richardson and all the reviewers for their feedback.