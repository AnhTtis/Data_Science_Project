\section*{Appendix B - Choice of $K$}
\label{sec:appB}
\setcounter{table}{0}
\renewcommand{\thetable}{B\arabic{table}}
\setcounter{figure}{0}
\renewcommand{\thefigure}{B\arabic{figure}}

The number of prototypes $K$ under supervision exactly corresponds to the number of ground truth classes. Without ground truth labels, the number of prototypes is selected arbitrarily and should hopefully be higher than the number of expected true classes. In Figure~\ref{fig:num_proto_used} we report the MA of our method with and without offset for different numbers of learned prototypes on TS2C dataset. Being entirely unsupervised, there is no restriction on how prototypes relate to classes: complex classes can be represented by several prototypes and others only by a single prototypical time series as shown in Figure~\ref{fig:visu_unsup}. The value $K=32$ prototypes appears to provide a favorable balance between classification accuracy and the number of learned parameters. As a result, we conducted all our unsupervised experiments using this chosen value.

\begin{figure}[t]
    \centering
    \includegraphics[trim={0 0 0 0}, clip, width=0.55\linewidth]{figures/ts2c_nb_proto.pdf}
    \caption{\textbf{Number of learned prototypes.} Mean accuracy of our method on TS2C depending on the number of learned prototypes. We show results averaged over 5 runs.}
    \label{fig:num_proto_used}
\end{figure}
