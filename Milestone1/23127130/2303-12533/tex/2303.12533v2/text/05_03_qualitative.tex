\subsection{Qualitative evaluation} \label{sec:qualitative}

\subsubsection{Land cover maps}

We provide in Figure~\ref{fig:comparative_visu_super} a visualization of the land cover maps obtained by our method and competing supervised approaches on PASTIS dataset. We show 4 randomly selected image time series from Fold 2 test set. One can see how our method improves over NCC by allowing pixels of the same field to be classified similarly. We highlight with black circles ({\color{black}$\mathbf{\bigcirc}$}) examples of areas where NCC gives different labels to central and border pixels whereas our method use the same deformable prototype to reconstruct all pixels of the field. OS-CNN and TapNet fail to classify properly the sea as background which we highlight with a yellow circle ({\color{yellow}$\mathbf{\bigcirc}$}). TapNet land cover maps are the most noisy, with a salt-and-pepper effect that is particularly noticeable on the third row. MLP+LTAE is the best at maintaining spatial consistency within crop classes and at accurately delineating boundaries.
\begin{figure}[t]
    \centering
    \includegraphics[width=1\linewidth]{images/qualitative_results/legend.pdf}
    \resizebox{\linewidth}{!}{
    \scriptsize
    \begin{tabular}{ccccccc}
    \qualitative{7}
    \qualitative{39}
    \qualitative{45}
    \qualitative{51}
    (a) Input & (b) MLP+LTAE  & (c) OS-CNN & (d) TapNet & (e) NCC & (f) \modelname & (g) GT
    \end{tabular}
    }
    \caption{
    \textbf{Qualitative comparisons of supervised methods.} We show predicted segmentation maps for best-performing supervised methods (b-d), NCC (e) and our method (f) for randomly selected SITS from Fold 2 test set of PASTIS (a). Dark grey segments correspond to the \textit{void} class and are ignored by all methods. The legend above is used for all other semantic segmentation visualizations of this paper.}
    \label{fig:comparative_visu_super}
\end{figure}
Similarly, we visually compare in Figure~\ref{fig:comparative_visu_unsup} land cover maps obtained with K-means and our method. We highlight with black circles {\color{black}($\mathbf{\bigcirc}$}) areas where our approach distinguish more faithfully agricultural parcels from the background class than K-means. Since cluster labeling is performed through majority voting, most clusters get assign to the majority background class on PASTIS: it is the case for 47\% of K-means clusters on Fold 2. However, our deformable prototypes can represent the same class with less clusters, hence only 41\% of them account for the background class.
\begin{figure}[t]
    \centering
\resizebox{0.57\linewidth}{!}{
\begin{tabular}{@{}cccc@{}}
\qualitativeunsup{39}
\qualitativeunsup{45}
\qualitativeunsup{55}
(a) Input & (b) K-means & (c) \modelname & (d) GT
\end{tabular}
}
    \caption{
    \textbf{Qualitative comparisons of unsupervised methods.} We show predicted segmentation maps for K-means (b) and our method (c) for randomly selected SITS from Fold 2 test set of PASTIS (a). Dark grey segments correspond to the \textit{void} class and are ignored by all methods.}
    \label{fig:comparative_visu_unsup}
\end{figure}
All pixel-wise SITS semantic segmentation methods can benefit from a post-processing step taking into account spatial information, \textit{e.g.}, aggregating predictions in nearby pixels, or on each field. While such post-processing is not the focus of our paper, we demonstrate in Appendix~\hyperref[sec:appC]{C} the benefits of several such post-processing methods.

\subsubsection{Visualizing prototypes}

We show in Figure~\ref{fig:visu_super} our prototypes and how they are deformed to reconstruct a given input. For each class of the SA dataset, we show an input time series that has been correctly assigned to its corresponding prototype by our model trained with supervision but without $\mathcal{L}_\text{cont}$. We see that the inputs are best reconstructed by a prototype of their class. Looking at any of the columns, we see that prototypes of other classes can also be deformed to reconstruct a given input, but only to a certain extent. This confirms that the transformations considered are simple enough so that the reconstruction power of each prototype is limited, but powerful enough to allow the prototypes to adapt to their input. 

Figure~\ref{fig:visu_unsup} shows the 32 prototypes learned by our unsupervised model on SA, grouped by assigned label. For each prototype, we show an example input sample whose best reconstruction is obtained using this particular prototype and the obtained corresponding reconstruction. We see that prototypes are not equally assigned to classes, with class \textit{Canola} having 14 prototypes when class \textit{Small Grain Gazing} only has 1. This is due to the high imbalance of the classes in the datasets and different intra-class variabilities. Inside a class, different prototypes account for intra-class variability beyond what our deformations can model. 
\begin{table}[t]
    \centering    
    \caption{\textbf{Detailed evaluation of our method.}~We show the impact of the increasing complexity of our modeling for reconstruction and accuracy on all datasets in both the supervised and unsupervised settings. Note that learning raw prototypes boils down to the NCC method~\citep{cover1967nearest} in the supervised setting and to the K-means algorithm~\citep{macqueen1967classification}  in the unsupervised setting.\\}
    \resizebox{\linewidth}{!}{
    \begin{tabular}{llcccccccccccc}
        \toprule
        & & \multicolumn{6}{c}{Supervised} & \multicolumn{6}{c}{Unsupervised}\\
        \cmidrule(lr){3-8}\cmidrule(lr){9-14}
        & & \multicolumn{3}{c}{Val} & \multicolumn{3}{c}{Test}  & \multicolumn{3}{c}{Train}  & \multicolumn{3}{c}{Test}\\ \cmidrule(lr){3-5}\cmidrule(lr){6-8}\cmidrule(lr){9-11}\cmidrule(lr){12-14}
        & & {\color{ACC}OA$\uparrow$} & MA$\uparrow$ & {\color{ACC}$\lrec\downarrow$} & {\color{ACC}OA$\uparrow$} & MA$\uparrow$ & {\color{ACC}$\lrec\downarrow$} & {\color{ACC}OA$\uparrow$} & MA$\uparrow$ & {\color{ACC}$\lrec\downarrow$} & {\color{ACC}OA$\uparrow$} & MA$\uparrow$ & {\color{ACC}$\lrec\downarrow$} \\
        \midrule
        \parbox[t]{2.3mm}{\multirow{4}{*}{\rotatebox[origin=c]{90}{PASTIS}}} & Raw prototypes
        & {\color{ACC}$57.3$} & $50.0$ & {\color{ACC}$4.43$} & {\color{ACC}$56.5$} & $48.4$ & {\color{ACC}$4.46$} & {\color{ACC}$69.1$} & $29.8$ & {\color{ACC}$2.77$} & {\color{ACC}$69.0$} & $29.8$ & {\color{ACC}$2.78$}\\
        & ~~$+$~time warping    
        & {\color{ACC}$56.8$} & $53.7$ & {\color{ACC}$4.00$} & {\color{ACC}$56.2$} & $51.4$ & {\color{ACC}$4.04$} & {\color{ACC}$\mathbf{69.2}$} & $\mathbf{30.4}$ & {\color{ACC}$2.53$} & {\color{ACC}$\mathbf{69.1}$} & $\mathbf{30.4}$ & {\color{ACC}$2.53$}\\
        & ~~~~$+$~offset    
        & {\color{ACC}$55.0$} & $55.7$ & {\color{ACC}$\mathbf{2.57}$} & {\color{ACC}$53.5$} & $53.8$ & {\color{ACC}$\mathbf{2.65}$} & {\color{ACC}$67.8$} & $28.5$ & {\color{ACC}$\mathbf{1.91}$} & {\color{ACC}$67.7$} & $28.6$ & {\color{ACC}$\mathbf{1.91}$}\\
	    & ~~~~~~$+$~$\mathcal{L}_\text{cont}$  
	    & {\color{ACC}$\mathbf{74.8}$} & $\mathbf{61.3}$ & {\color{ACC}$2.90$} & {\color{ACC}$\mathbf{73.7}$} & $\mathbf{59.1}$ & {\color{ACC}$3.00$} & {\color{ACC}---} & --- & {\color{ACC}---} & {\color{ACC}---} & --- & {\color{ACC}---}\\
	    \midrule
        \parbox[t]{2.3mm}{\multirow{4}{*}{\rotatebox[origin=c]{90}{TS2C}}} & Raw prototypes 
        & {\color{ACC}$57.4$} & $51.2$ & {\color{ACC}$4.89$} & {\color{ACC}$57.4$} & $49.5$ & {\color{ACC}$4.36$} & {\color{ACC}$56.2$} & $34.2$ & {\color{ACC}$3.52$} & {\color{ACC}$49.5$} & $32.5$ & {\color{ACC}$3.56$}\\
        & ~~$+$~time warping    
        & {\color{ACC}$56.0$} & $51.2$ & {\color{ACC}$4.64$} & {\color{ACC}$59.9$} & $52.3$ & {\color{ACC}$4.15$} & {\color{ACC}$59.1$} & $38.6$ & {\color{ACC}$3.04$} & {\color{ACC}$\mathbf{52.3}$} & $\mathbf{36.0}$ & {\color{ACC}$3.09$}\\
        & ~~~~$+$~offset    
        & {\color{ACC}$56.9$} & $51.8$ & {\color{ACC}$3.50$} & {\color{ACC}$57.3$} & $55.0$ & {\color{ACC}$3.49$} & {\color{ACC}$\mathbf{60.0}$} & $\mathbf{39.3}$ & {\color{ACC}$\mathbf{2.40}$} & {\color{ACC}$52.0$} & $35.5$ & {\color{ACC}$\mathbf{2.53}$}\\
	    & ~~~~~~$+$~$\mathcal{L}_\text{cont}$  
	    & {\color{ACC}$\mathbf{74.5}$} & $\mathbf{64.4}$ & {\color{ACC}$\mathbf{3.46}$} & {\color{ACC}$\mathbf{78.5}$} & $\mathbf{70.5}$ & {\color{ACC}$\mathbf{3.46}$} & {\color{ACC}---} & --- & {\color{ACC}---} & {\color{ACC}---} & --- & {\color{ACC}---}\\
	    \midrule
        \parbox[t]{2.3mm}{\multirow{4}{*}{\rotatebox[origin=c]{90}{SA}}} & Raw prototypes
        \tableABLline{54.8}{50.0}{3.43}{51.3}{46.4}{4.62}\tableABLline{60.9}{50.9}{1.43}{61.9}{47.8}{1.85}\\
        & ~~$+$~time warping    
        \tableABLline{57.5}{53.9}{2.93}{54.5}{49.7}{4.13}\tableABLline{62.2}{53.1}{1.03}{\mathbf{64.1}}{\mathbf{51.7}}{1.46}\\
        & ~~~~$+$~offset    
        \tableABLline{63.5}{58.0}{\mathbf{1.34}}{60.6}{50.0}{\mathbf{2.01}}\tableABLline{\mathbf{63.7}}{\mathbf{54.5}}{\mathbf{0.67}}{63.6}{50.4}{\mathbf{0.91}}\\
	    & ~~~~~~$+$~$\mathcal{L}_\text{cont}$  
	    \tableABLline{\mathbf{71.0}}{\mathbf{64.7}}{1.89}{\mathbf{62.3}}{\mathbf{54.9}}{2.66} & {\color{ACC}---} & --- & {\color{ACC}---} & {\color{ACC}---} & --- & {\color{ACC}---}\\
        \midrule
        \midrule
        \parbox[t]{2.3mm}{\multirow{4}{*}{\rotatebox[origin=c]{90}{DENETH.}}} & Raw prototypes
        & {\color{ACC}$68.3$} & $58.0$ & {\color{ACC}$3.89$} & {\color{ACC}$61.3$} & $55.5$ & {\color{ACC}$4.58$} & {\color{ACC}$63.8$} & $52.8$ & {\color{ACC}$2.67$} & {\color{ACC}$57.2$} & $48.5$ & {\color{ACC}$2.41$}\\
        & ~~$+$~time warping 
        \tableABLline{70.1}{59.5}{3.52}{\mathbf{62.4}}{56.4}{4.21}\tableABLline{64.8}{54.0}{2.23}{57.6}{51.1}{2.01}\\
        & ~~~~$+$~offset    
        \tableABLline{77.3}{64.9}{\mathbf{2.39}}{59.8}{\mathbf{62.9}}{\mathbf{3.55}}\tableABLline{\mathbf{66.2}}{\mathbf{56.3}}{\mathbf{1.70}}{\mathbf{58.5}}{\mathbf{52.6}}{\mathbf{1.56}}\\
	    & ~~~~~~$+$~$\mathcal{L}_\text{cont}$  
	    \tableABLline{\mathbf{85.1}}{\mathbf{75.5}}{3.00}{56.5}{54.2}{4.35} & {\color{ACC}---} & --- & {\color{ACC}---} & {\color{ACC}---} & --- & {\color{ACC}---}\\
        \bottomrule
    \end{tabular}
    }
    \label{tab:super_ablation}
\end{table}
\begin{figure}[t]
    \centering
    \resizebox{\linewidth}{!}{
    \begin{tabular}{cc|ccccc}
        & & \multicolumn{5}{c}{\bf Inputs} \\
        & & Wheat & Barley & Lucerne/Medics & Canola & Small Grain Gazing\\
        & & \showinput{0} &
        \showinput{1} &
        \showinput{2} &
        \showinput{3} &
        \showinput{4}
        \\\midrule
        \multirow{10}{*}{\rotatebox{90}{\bf Prototypes~~~~~~~~~~~~}}
        & \showproto{0} &
        \boxspect{images/recons/recons_input_0_with_proto_0.pdf}{Kselect} &
        \boxspect{images/recons/recons_input_1_with_proto_0.pdf}{Knotselect} &
        \boxspect{images/recons/recons_input_2_with_proto_0.pdf}{Knotselect} &
        \boxspect{images/recons/recons_input_3_with_proto_0.pdf}{Knotselect} &
        \boxspect{images/recons/recons_input_4_with_proto_0.pdf}{Knotselect}
        \\
        & Wheat & 
        $\lrec=0.134$ &
        $\lrec=0.494$ &
        $\lrec=0.609$ &
        $\lrec=0.575$ &
        $\lrec=0.542$
        \\
        & \showproto{1} &
        \boxspect{images/recons/recons_input_0_with_proto_1.pdf}{Knotselect} &
        \boxspect{images/recons/recons_input_1_with_proto_1.pdf}{Kselect} &
        \boxspect{images/recons/recons_input_2_with_proto_1.pdf}{Knotselect} &
        \boxspect{images/recons/recons_input_3_with_proto_1.pdf}{Knotselect} &
        \boxspect{images/recons/recons_input_4_with_proto_1.pdf}{Knotselect}
        \\
        & Barley & 
        $\lrec=0.145$ &
        $\lrec=0.456$ &
        $\lrec=0.568$ &
        $\lrec=0.685$ &
        $\lrec=0.589$
        \\
        & \showproto{2} &
        \boxspect{images/recons/recons_input_0_with_proto_2.pdf}{Knotselect} &
        \boxspect{images/recons/recons_input_1_with_proto_2.pdf}{Knotselect} &
        \boxspect{images/recons/recons_input_2_with_proto_2.pdf}{Kselect} &
        \boxspect{images/recons/recons_input_3_with_proto_2.pdf}{Knotselect} &
        \boxspect{images/recons/recons_input_4_with_proto_2.pdf}{Knotselect}
        \\
        & Lucerne/Medics & 
        $\lrec=0.388$ &
        $\lrec=0.578$ &
        $\lrec=0.501$ &
        $\lrec=0.666$ &
        $\lrec=0.704$
        \\
        & \showproto{3} &
        \boxspect{images/recons/recons_input_0_with_proto_3.pdf}{Knotselect} &
        \boxspect{images/recons/recons_input_1_with_proto_3.pdf}{Knotselect} &
        \boxspect{images/recons/recons_input_2_with_proto_3.pdf}{Knotselect} &
        \boxspect{images/recons/recons_input_3_with_proto_3.pdf}{Kselect} &
        \boxspect{images/recons/recons_input_4_with_proto_3.pdf}{Knotselect}
        \\
        & Canola & 
        $\lrec=0.302$ &
        $\lrec=0.960$ &
        $\lrec=0.822$ &
        $\lrec=0.249$ &
        $\lrec=0.852$
        \\
        & \showproto{4} &
        \boxspect{images/recons/recons_input_0_with_proto_4.pdf}{Knotselect} &
        \boxspect{images/recons/recons_input_1_with_proto_4.pdf}{Knotselect} &
        \boxspect{images/recons/recons_input_2_with_proto_4.pdf}{Knotselect} &
        \boxspect{images/recons/recons_input_3_with_proto_4.pdf}{Knotselect} &
        \boxspect{images/recons/recons_input_4_with_proto_4.pdf}{Kselect}
        \\
        & Small Grain Gazing & 
        $\lrec=0.153$ &
        $\lrec=0.510$ &
        $\lrec=0.563$ &
        $\lrec=0.478$ &
        $\lrec=0.537$
    \end{tabular}
    }
    \caption{
    \textbf{Reconstructions from different prototypes.}~{We show the reconstructions of input samples (columns) from SA~\citep{kondmann2022early} by learned prototypes (rows) in the supervised setting without $\mathcal{L}_\text{cont}$. Selected prototypes (frames) correspond to the lowest reconstruction error.}}
    \label{fig:visu_super}
\end{figure}