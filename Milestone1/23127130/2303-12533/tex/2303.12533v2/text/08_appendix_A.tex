\setcounter{table}{0}
\renewcommand{\thetable}{A\arabic{table}}
\setcounter{equation}{0}
\renewcommand{\theequation}{A\arabic{equation}}
\setcounter{figure}{0}
\renewcommand{\thefigure}{A\arabic{figure}}
\section*{Appendix A - Prototype initialization}
\label{sec:appA}
NCC or K-means centroids are used to initialize our prototypes, thus impacting the performance of our method. Plus, it is not obvious what is the best way to run NCC or K-means when faced with time series with missing data. The centroids can be computed by only giving weight to existing data points or after a gap filling operation. In this section, we focus on the supervised case and investigate other simple gap filling methods and the respective performance of both NCC and our method. Following the notations of Section~\ref{sec:missing_data}, we define the following gap filling methods:
\paragraph{None} No gap filling is done and the data processed by the method correspond to the raw input data: $\inputseq=\inputseq_\text{raw}$ and $\mask=\mask_\text{raw}$.
\paragraph{Previous} Missing time stamps take the value of the closest previous data point in the time series:
    \begin{equation}
        \inputseq[t] = \inputseq_\text{raw}\Big[\underset{t'\leq t}{\max}\ \ \mask_\text{raw}[t']=1\Big],
        \label{eq:previous}
    \end{equation}
    and
    \begin{equation} 
        \mask[t] = \mathbb{1}_{\{t'<t|\mask_\text{raw}[t']=1\}\neq \emptyset}[t].
        \label{eq:previous_mask}
    \end{equation}
\paragraph{Moving average} The value of each time stamps is set as the non-weighted average of the data points inside of a centered time window:
    \begin{equation}
        \inputseq[t] = \frac{1}{\mask[t]}\sum_{t'=t-\sigma}^{t+\sigma} \frac{\inputseq_\text{raw}[t']}{2\sigma+1},
        \label{eq:mov_avg}
    \end{equation}
    where $\sigma$ is a hyperparameter set to 7 days in our experiments - same as in Equation~\ref{eq:gaussian_filter}. We also define the associated filtered mask $\mask$ for $t\in [1, T]$ by:
    \begin{equation}
        \mask[t] = \sum_{t'=t-\sigma}^{t+\sigma} \frac{\mask_\text{raw}[t']}{2\sigma+1},
        \label{eq:mov_avg_mask}
    \end{equation}
    for $t\in [1, T]$ and with the same hyperparameter $\sigma$.
\paragraph{Gaussian filter} We can consider the filtering of Equations~\ref{eq:gaussian_filtering} to~\ref{eq:gaussian_filtering_mask} as a gap filling method.

The NCC centroid corresponding to class $k$ is then given by:
\begin{equation}
    \mathbf{C}_k[t] = \frac{1}{N_kC} \underset{y_i=k}{\sum_{i=1}^N} \frac{\mask_i[t]}{\sum_{t'=1}^T \mask_i[t']} \inputseq_i.
    \label{eq:ncc_centroids}
\end{equation}
In Table~\ref{tab:ts2c_filtering}, we report the MA of both NCC and our method on TS2C dataset, using these different gap filling settings to compute NCC centroids. For our method, the only difference between experiments is the initialization of the prototypes. Filling missing data with Gaussian filtering improves over no gap filling by almost +5pt of MA. We also see that the ranking of the different gap filling methods is preserved with our method which confirms the importance of the initialization of the prototypes when training our model. Qualitatively, Figure~\ref{fig:ncc_vs_ours} illustrates that our approach does not effectively address the inadequate quality of NCC centroids when gap filling is not employed. In contrast, all three gap filling strategies yield similar learned potato prototypes, even when initialized with NCC centroids that display substantial differences. 

In Section~\ref{sec:missing_data}, we also present a filtering scheme of input data to prevent learning from potential outliers. Note that this can also be done during the assignment step when running NCC. Table~\ref{tab:ts2c_filtering} also shows how this input filtering is necessary for both NCC and our method to reach their best performance.

\begin{table}[!t]
  \centering
  \caption{\textbf{Comparison of various initialization of NCC centroids} and resulting performance with our method. We also show the effect of applying a Gaussian filter on the input data following Equation~\ref{eq:gaussian_filtering}.\\}
  \resizebox{0.6\linewidth}{!}{
  \begin{tabular}{lccccc}
  \toprule
    \multirow{2}{*}{Gap filling} & Input & \multicolumn{2}{c}{NCC} & \multicolumn{2}{c}{\modelname}\\
    & filtering & OA & MA & OA & MA \\
    \midrule
    None            & \xmark & 53.1 & 45.3 & 69.2 & 58.5 \\
    Previous        & \xmark & 56.8 & 48.1 & 72.8 & 63.1 \\
    Moving average  & \xmark & 56.6 & 49.1 & 73.2 & 63.9 \\
    \multirow{2}{*}{Gaussian filter} & \xmark & 57.1 & \textbf{49.9} & 76.4 & 68.2 \\    
    \               & \cmark & \textbf{57.7} & \textbf{49.9} & \textbf{78.5} & \textbf{70.5} \\    
  \bottomrule
  \end{tabular}
  }
  \label{tab:ts2c_filtering}
\end{table}

\begin{figure}[t]
    \centering
    \begin{tabular}{cc}
        \includegraphics[width=0.27\linewidth]{figures/potato/potato_none_ncc.pdf}
        &
        \includegraphics[width=0.27\linewidth]{figures/potato/potato_none_ours.pdf}\\
        (a) NCC+None
        &
        (b) \modelname+None\\
        \includegraphics[width=0.27\linewidth]{figures/potato/potato_previous_ncc.pdf}
        &
        \includegraphics[width=0.27\linewidth]{figures/potato/potato_previous_ours.pdf}\\
        (c) NCC+Previous
        &
        (d) \modelname+Previous\\
        \includegraphics[width=0.27\linewidth]{figures/potato/potato_avg_ncc.pdf}
        &
        \includegraphics[width=0.27\linewidth]{figures/potato/potato_avg_ours.pdf}\\
        (e) NCC+Moving average
        &
        (f) \modelname+Moving average\\
        \includegraphics[width=0.27\linewidth]{figures/potato/potato_gaussian_ncc.pdf}
        &
        \includegraphics[width=0.27\linewidth]{figures/potato/potato_gaussian_ours.pdf}\\
        (g) NCC+Gaussian filter
        &
        (h) \modelname+Gaussian filter\\
    \end{tabular}
    \caption{\textbf{NCC centroids and learned prototypes.} \textbf{\color{red}{R}\color{green}{G}\color{blue}{B}} and \textbf{IR} spectral bands of centroids and prototypes obtained with NCC and our method respectively using different gap filling methods and corresponding to TS2C potato class.}
    \label{fig:ncc_vs_ours}
\end{figure}