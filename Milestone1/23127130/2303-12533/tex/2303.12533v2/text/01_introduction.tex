\section{Introduction}

\begin{figure}[t]
    \centering
    \begin{tabular}{ccc}
        \multicolumn{3}{c}{\includegraphics[trim={0.3cm 0 0.3cm 0}, clip,width=0.87\linewidth]{figures/teaser_sits.pdf}}\\
        \multicolumn{3}{c}{(a) Satellite image time series (SITS)}\\
        \includegraphics[trim={0.6cm 0 2.9cm 0}, clip, 
        width=0.28\linewidth]{figures/teaser_input.pdf} & \includegraphics[trim={0.6cm 0 2.9cm 0}, clip, width=0.28\linewidth]{figures/teaser_proto.pdf} & \includegraphics[trim={0.6cm 0 2.9cm 0}, clip, width=0.28\linewidth]{figures/teaser_recons.pdf}\\
        (b) Input & (c) Prototype & (d) Reconstruction
    \end{tabular}
    \caption{\textbf{Reconstructing pixel sequences from satellite image time series (SITS) through learned prototypes and transformations.}~Given a SITS (a), we reconstruct pixel-wise multi-spectral sequences using learned prototypes and transformations. Here, we show the \textbf{\color{red}{R}\color{green}{G}\color{blue}{B}} and \textbf{IR} spectral intensities over time for a corn (\raisebox{-1.2pt}{\includegraphics[width=9px]{images/cereals/corn.pdf}}) and a wheat (\raisebox{-1pt}{\includegraphics[width=8px]{images/cereals/wheat.pdf}}) pixel sequence (b), along with their corresponding prototype before (c) and after (d) transformation.}
  \label{fig:teaser}
\end{figure}
With risks of food supply disruptions, constantly increasing energy needs, population growth and climate change, the threats faced by global agriculture production are plenty~\citep{PROSEKOV201873, mb01000b}. Monitoring crop yield production, controlling plant health and growth, and optimizing crop rotations are among the essential tasks to be carried out at both national and global scales. Because regular ground-based surveys are challenging, remote sensing has very early on appeared as the most practical tool~\citep{justice2007report}. 

Thanks to public and commercial satellite launches such as ESA's Sentinel constellation~\citep{drusch2012sentinel, Aschbacher2017}, NASA's Landsat~\citep{Woodcock2008} or Planet's PlanetScope constellation~\citep{boshuizen2014results, team2017planet}, Earth observation is now possible at both high temporal frequency and moderate spatial resolution, typically in the range of 10m/pixel. Sensed data can thus be processed to form satellite image time series (SITS) for further analysis either at the image or pixel level. In particular, several recent agricultural SITS datasets~\citep{Kondmann2021denethor, kondmann2022early, Weikmann2021, Garnot2021, breizhcrops2020} make such data available to the machine learning community, mainly for improving crop type classification. 

In this paper, we focus on methods approaching SITS segmentation as multivariate time series classification (MTSC) by considering multi-spectral pixel sequences as the data to classify. While this excludes whole series-based methods like those of \citet{Garnot2021} or \citet{tarasiou2023vits} which explicitly leverage the extent of individual parcels, it enables us to extensively evaluate more general MTSC methods that have not yet been applied to agricultural SITS classification. We give particular attention to unsupervised methods as well as interpretability, which we believe would be appealing for extending results beyond well annotated geographical areas.

Our contributions are twofold. First, we benchmark MTSC approaches on four recent SITS datasets \citep{Kondmann2021denethor, kondmann2022early, Weikmann2021, Garnot2021} (Sections~\ref{sec:dataset} and~\ref{sec:baselines}). State-of-the-art supervised methods~\citep{garnot2020lightweight, tang2020rethinking, zhang2020tapnet} are typically complex and require vast amounts of labeled data, i.e., time series with accurate crop labels. We show that, while they provide strong accuracy boosts over more traditional methods like Random Forest or Support Vector Machine classifiers on datasets with limited domain gap between train and test data, they do not improve over the simple nearest centroid classification baseline on the more challenging DENETHOR~\citep{Kondmann2021denethor} dataset (Section~\ref{sec:supclass}). In the unsupervised setting, K-means clustering~\citep{macqueen1967classification} and its variant~\citep{Petitjean2011, zhang2014modis} using dynamic time warping (DTW) measure - instead of Euclidean distance - are the strongest baselines~\citep{rivera2020preliminary} (Sections~\ref{sec:unsupclus}).

Second, we design a transformation module corresponding to time warping which enables to adapt deep transfor\-mation-invariant (DTI) clustering~\citep{monnier2020deep} to SITS classification and improve nearest centroid classifier~\citep{cover1967nearest}. We refer to our method as \modelname. While deep unsupervised methods for SITS classification typically rely either on representation learning or pseudo-labeling, our method learns deformable prototypical sequences (Figure~\ref{fig:teaser}) by optimizing a reconstruction loss (Section~\ref{sec:method}). Our prototypes are learned multivariate time series, typically representing a type of crop, and they can be deformed to model intra-class variabilities. DTI-TS can be trained with or without supervision. In the unsupervised case, we achieve best scores on all studied datasets by adding spectro-temporal invariance to K-means clustering~\citep{macqueen1967classification}. In the supervised case, our model can be seen as an extension of the nearest centroid classifier~\citep{cover1967nearest}. In the low data regime, \textit{i.e.} with few labeled image time series, or when there is a temporal domain shift between train and test data, we outperform all competing methods.