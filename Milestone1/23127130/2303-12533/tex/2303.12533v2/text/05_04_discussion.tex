\subsection{Discussion} \label{sec:discussion}

\begin{table}[!t]
  \centering
  \caption{\textbf{Reconstruction loss of correct and wrong predictions.} We report the average reconstruction loss of correct and wrong predictions on all datasets for our method in the supervised case. We highlight in bold the lowest reconstruction loss for each row: time series that we tend to misclassify are also more difficult to reconstruct in general.\\}
  \resizebox{0.6\linewidth}{!}{
  \begin{tabular}{lcc}
  \toprule
    & \cmark\ Correct & \xmark\ Wrong\\
    & predictions & predictions\\
    \midrule
    PASTIS~\citep{Garnot2021}   & \textbf{2.52} & 2.72 \\
    TS2C~\citep{Weikmann2021}     & \textbf{3.44} & 3.79 \\
    SA~\citep{kondmann2022early}       & \textbf{1.84} & 2.26 \\
    DENETHOR~\citep{Kondmann2021denethor} & \textbf{3.54} & 3.57 \\
  \bottomrule
  \end{tabular}
  }
  \label{tab:loss_correct_wrong}
\end{table}

\begin{figure}[!t]
  \centering
  \resizebox{\linewidth}{!}{
  \scriptsize
  \begin{tabular}{cc}
    \includegraphics[width=0.4\linewidth]{images/failure_cases/classif_barley_sa.pdf}&\includegraphics[width=0.4\linewidth]{images/failure_cases/classif_canola_sa.pdf}\\
    (a) Barley (SA) & (b) Canola (SA) \\
    \includegraphics[width=0.4\linewidth]{images/failure_cases/classif_corn_denethor.pdf}&\includegraphics[width=0.4\linewidth]{images/failure_cases/classif_rootcrops_denethor.pdf}\\
    (c) Corn (DENETHOR) & (d) Root crops (DENETHOR) \\
  \end{tabular}
  }
  \caption{\textbf{Visual representation of failure cases.} We show the normalized red band of randomly selected time series from two classes of SA~\citep{kondmann2022early} and DENETHOR~\citep{Kondmann2021denethor}. We distinguish between time series  correctly classified by our model (in blue) and time series misclassified (in red). We also display the red band of the corresponding learned prototype in each case.}
  \label{fig:visu_correct_wrong}
\end{figure}

DTI-TS fails at classifying an input pixel time series when the prototype of a wrong crop type is able to better reconstruct it than the prototype of the true class. This may happen in three cases that we detail below: (i) because both classes are very similar, (ii) because our deformations are powerful enough to align semantically different prototypes to the same input sequence or (iii) simply because the input time series is a difficult sample to reconstruct.

\paragraph{Similar classes.} Example of similar classes can be seen in Figure~\ref{fig:visu_ndvi} where the mean NDVI over time of the Winter wheat and the Rye classes on TS2C as well as the Barley and Rye classes of DENETHOR are very close. Our transformations may align indifferently both class prototypes to an input sequence, discarding small differencies that would have helped classify it.

\paragraph{Transformation design.} While our deformations are simple, they may not be constrained enough for the task of crop classification. The time warping stretches or squeezes temporally a time series using uniformly spaced control points. In Figure~\ref{fig:visu_ndvi}, looking at the Wheat class for SA, note how this time warping is able to align train (in blue) and test (in red) curves, despite a clear temporal shift. Even though these deformations are limited to 7 days in each direction, they do not focus on a specific period in the year. Our offset transformation assumes that intra-class spectral distortions are time-independent. Though we show empirically that we can reconstruct better time series when using this transformation, this comes at the price of reduced classification performance. We believe performances could be further improved by the design of physics-based transformations that could account for actual meteorologic events.

\paragraph{Reconstruction performance on misclassified samples.} Misclassified samples by our method tend to be the most difficult to reconstruct. This statement is supported quantitatively in Table~\ref{tab:loss_correct_wrong} where we show that the reconstruction loss is higher on average for misclassified time series on all datasets. In Figure~\ref{fig:visu_correct_wrong}, we can see that wrongly classified time series (in red) often show clear differences from the learned prototype (in bold blue). Again, better suited deformations for this task should help prototypes reconstruct more accurately diverse time series of the same class and meanwhile not let them fit times series of other classes. We believe this to be a challenge that should be addressed in future work.
