\section{Introduction}

\begin{figure}[t]
    \centering
\setlength{\tabcolsep}{2pt}
\renewcommand{\arraystretch}{1}
\begin{tabular}{@{}ccc@{}}
\multicolumn{3}{c}{\includegraphics[trim={0.3cm 0 0.3cm 0}, clip,width=0.98\linewidth]{figures/teaser_sits.pdf}}\\
\multicolumn{3}{c}{(a) Satellite image time series (SITS)}\\
\includegraphics[trim={0.6cm 0 2.9cm 0}, clip, 
width=0.32\linewidth]{figures/teaser_input_2.pdf} & \includegraphics[trim={0.6cm 0 2.9cm 0}, clip, width=0.32\linewidth]{figures/teaser_proto_2.pdf} & \includegraphics[trim={0.6cm 0 2.9cm 0}, clip, width=0.32\linewidth]{figures/teaser_recons_2.pdf}\\
(b) Input & (c) Prototypes & (d) Recons.
\end{tabular}
\vspace{-1em}
    \caption{
    \textbf{Reconstructing pixel sequences from satellite image time series (SITS) through learned prototypes and transformations.}~Given a SITS (a), we reconstruct pixel-wise multi-spectral sequences using learned prototypes and transformations. Here, we show the \textbf{\color{red}{R}\color{green}{G}\color{blue}{B}} and \textbf{IR} spectral intensities over time for a corn (\raisebox{-1.2pt}{\includegraphics[width=9px]{images/cereals/corn.pdf}}) and a wheat (\raisebox{-1pt}{\includegraphics[width=8px]{images/cereals/wheat.pdf}}) pixel sequence (b), along with their  corresponding prototype before (c) and after (d) transformation.
    }
  \label{fig:teaser}
\end{figure}



With risks of food supply disruptions, constantly increasing energy needs, population growth and climate change, the threats faced by global agriculture production are plenty~\cite{PROSEKOV201873, mb01000b}. Monitoring crop yield production, controlling vegetal health and growth, and optimizing crop rotations are among the essential tasks to be carried out at both national and global scales. Because regular ground-based surveys are challenging, remote sensing has very early on appeared as the most practical tool~\cite{justice2007report}. 

Thanks to public and commercial satellite launches such as ESA's Sentinel constellation~\cite{drusch2012sentinel, Aschbacher2017}, NASA's Landsat~\cite{Woodcock2008} or Planet's PlanetScope constellation~\cite{boshuizen2014results, team2017planet}, Earth observation is now possible at both high temporal frequency and moderate spatial resolution, typically in the range of 10m/pixel. Sensed data can thus be processed as satellite image time series (SITS) at either the image or pixel level. In particular, several recent agricultural SITS datasets~\cite{Kondmann2021denethor, kondmann2022early, Weikmann2021, Garnot2021, breizhcrops2020} make such data available to the machine learning community, mainly for improving crop type classification. 

In this paper, we focus on methods approaching SITS segmentation as multivariate time series classification (MTSC) by considering multi-spectral pixel sequences as the data to classify. While this excludes some methods, such as~\cite{Garnot2021} which explicitly leverages the extent of individual parcels, it enables us to extensively evaluate more general MTSC methods that have not yet been applied to agricultural SITS classification. We give particular attention to unsupervised methods as well as interpretability, which we believe would be appealing for extending results beyond well annotated geographical areas.

Our contributions are twofold. First, we benchmark approaches on four recent SITS datasets~\cite{Kondmann2021denethor, kondmann2022early, Weikmann2021, Garnot2021} in both the supervised and unsupervised settings. State-of-the-art supervised methods~\cite{garnot2020lightweight, tang2020rethinking, zhang2020tapnet} are typically complex and require vast amounts of labeled data, i.e., time series with accurate crop labels. We show that, while they provide strong accuracy boosts on datasets with limited domain gap between train and test data, they do not improve over the simple nearest centroid classification baseline on the more challenging DENETHOR~\cite{Kondmann2021denethor} dataset. K-means clustering~\cite{macqueen1967classification} and its variant~\cite{Petitjean2011, zhang2014modis} using dynamic time warping (DTW) metric - instead of euclidean distance -  are the strongest baselines~\cite{rivera2020preliminary} in the unsupervised setting.

Second, we adapt the deep transformation-invariant (DTI) clustering~\cite{monnier2020deep} to SITS classification by designing a transformation module corresponding to time warping. While deep unsupervised methods rely either on representation learning or pseudo-labeling, our method learns deformable prototypical sequences (Figure~\ref{fig:teaser}) by optimizing a reconstruction loss. Prototypes are multivariate time series, typically representing a type of crop, and that can be deformed to model intra-class variabilities. Following~\cite{loiseau22amodelyoucanhear}, we present results with prototypes learned with and without supervision, as extensions of the nearest centroid classifier~\cite{cover1967nearest} or the K-means clustering~\cite{macqueen1967classification}, depending on the case.
