\documentclass[12pt]{article}
\usepackage[pass]{geometry}

%\usepackage{natbib} %Literatur
%\usepackage{chicago} %Literatur
\usepackage{graphicx} %Graphen
\usepackage{subfigure} %Abbildungen
\usepackage{lscape}	%Gedrehte Darstellungen
%\usepackage[latin1]{inputenc} %Umlaute
\usepackage{pst-plot, pstricks} %Graphen
\usepackage{fancybox,amssymb,color} %Fancy stuff (z.B. Umrahmungen), Formeln, Farben
%\usepackage[british,UKenglish,ngerman]{babel} %Sprache
\usepackage[english]{babel}
%\usepackage{bibgerm} %Deutsche Abkürzungen
\usepackage{setspace} %Zeilenabstand
\usepackage{epstopdf}
\usepackage[capposition=top]{floatrow} 
\usepackage{float}
%\usepackage{hyperref} 
\usepackage{enumitem}
\usepackage{longtable}
\usepackage{tabularx,calc}
\usepackage{multirow}
\usepackage{booktabs}
\usepackage{acro}
\usepackage{setspace}
\usepackage{lscape}
\usepackage{mathtools }
 \usepackage{rotating}

\usepackage[authoryear]{natbib}
\bibliographystyle{apalike} %apalike unsrt u

%\usepackage[demo]{graphicx}
\usepackage{caption}
%\usepackage{subcaption}

%\usepackage{amsmath} %Formeln
%\usepackage{wasysym}
%\usepackage{amssymb}

\usepackage{amsmath}
%\usepackage{wasysym}
\usepackage{amsthm} 
\usepackage{amssymb}
\usepackage{amsbsy}
\usepackage{bm}

% Einstellung zum Seitenlayout: 
\pagestyle{plain} %Seitenkopf leer, Fußzeile mit zentrierter Seitennummer 
%\setlength{\parskip}{1ex} %Abstand zwischen Absätzen
%\topmargin-20pt \oddsidemargin0pt \evensidemargin0pt %\textheight665pt
%\textwidth420pt \marginparwidth0pt %Seitenabstände
%\linespread {1.5}

\addtolength{\oddsidemargin}{-.5in}%
\addtolength{\evensidemargin}{-1in}%
\addtolength{\textwidth}{1in}%
\addtolength{\textheight}{1.7in}%
\addtolength{\topmargin}{-1in}%

\doublespacing

 

\newenvironment{mybsmallmatrix}
{\hbox\bgroup
	\delimiterfactor=1000
	\delimitershortfall=0pt
	$\left[\smallmatrix}
{\endsmallmatrix\right]$\egroup}

\DeclareMathOperator*{\argmin}{arg\,min}


\usepackage{changes}%added
\definechangesauthor[name={Sascha}, color=green]{Sascha}

\newtheorem{proposition}{Proposition} 
\newtheorem{lemma}{Lemma} 
\newtheorem{corollary}{Corollary} 
\newtheorem{assumption}{Assumption} 
\newtheorem{definition}{Definition} 
 \setlength{\parindent}{0em} 


\title{Uncertain Prior Economic Knowledge and Statistically Identified Structural Vector Autoregressions}


 \author{Sascha  A. Keweloh  \\
	Department of Economics,  TU Dortmund }
%\vspace{1em}} 
\date{\today}

\begin{document} 
\clearpage\maketitle
\thispagestyle{empty}
\begin{abstract}

\noindent This study proposes an estimator that combines statistical identification with economically motivated restrictions on the interactions. The estimator is identified by (mean) independent non-Gaussian shocks and allows for incorporation of uncertain prior economic knowledge through an adaptive ridge penalty. The estimator shrinks towards economically motivated restrictions when the data is consistent with them and stops shrinkage when the data provides evidence against the restriction.
The estimator is applied to analyze the interaction between the stock  and  oil market. The results suggest that what is usually identified as oil-specific demand shocks can actually be attributed to information shocks extracted from the stock market, which  explain about $30-40$\% of the oil price variation.
 
 
\vspace{0.75cm}
\end{abstract}
 

\clearpage
\pagenumbering{arabic} 
 


\section{Introduction}
\label{sec: Introduction} 
Traditional identification methods of structural vector autoregressions (SVAR) rely on   imposing prior economic knowledge on the interaction, such as short- and long-run restrictions, sign restrictions, or proxy variables.   
More recently, data-driven methods have been used to ensure identification by imposing structure on the stochastic properties of the shocks, such as time-varying volatility   or non-Gaussian and independent shocks.
Statistical identification approaches do not rely on prior economic knowledge on the interaction to ensure identification. However, prior economic knowledge is still required to attach a structural interpretation to the shocks, i.e., label the identified shocks. Therefore, prior economic knowledge remains necessary, even if it is not required for identification. Consequently, the question is not if we have prior economic knowledge, but how   we want to use it? 

The question relates to the critique of the "all-or-nothing approach" w.r.t. prior economic knowledge raised by   \cite{baumeister2019structural} and \cite{baumeister2022structural}. That is, traditional methods usually  impose prior  knowledge as uncontestable truth, e.g. by  restricting   a given response to zero and only zero without the ability to ever update the restriction by the data, and sometimes completely ignore any prior  knowledge. In this line of thought, estimators based on statistical identification approaches ignoring any prior economic knowledge on the interaction represent the extreme case of the  "nothing approach".



This study proposes  to incorporate uncertain prior economic knowledge into the estimation of  SVAR models identified by stochastic properties.  The  proposed   Ridge SVAR-GMM estimator (RGMM)   uses higher-order moment conditions to ensure identification by non-Gaussian and (mean) independent shocks and adds a ridge penalty to penalize deviations from restrictions implied by prior economic knowledge. 
The intuition of the shrinkage estimator is that before seeing the data, the researcher believes that the restrictions are correct and thus shrinks towards the restrictions.  However,   the restrictions representing prior knowledge are not required for identification and therefore, the estimator can update the prior knowledge and shrinks less towards the restriction  if the data provide  evidence against a given restriction.


The main contribution of this study is to incorporate uncertain prior economic knowledge  using  a ridge penalty with adaptive weights (see, e.g., \cite{zou2006adaptive}).  This approach allows   to shrink towards economically motivated restrictions while accounting for uncertainty in the prior knowledge.
The adaptive weights have an important feature: it is cheap to deviate from false restrictions and costly to deviate from correct restrictions. Therefore, the weights determine the importance of a given restriction in a data-driven way.
The simulation results demonstrate that correctly imposed restrictions increase the accuracy of the estimation, highlighting the value of prior economic knowledge beyond identification. Additionally, the approach is able to detect false restrictions and reduce their impact, as shown in the simulation. Furthermore, the RGMM estimator does not require   restrictions   for identification. Therefore,   researchers can include an arbitrary number of theoretically well-founded restrictions and is not forced to include additional  controversial ones. 


This approach is only possible if the restrictions are not required to ensure identification. In this study, identification follows from non-Gaussian and   (mean) independent shocks. 
The assumption of independent shocks has been criticized in the literature with the common objection that shocks driven by the same volatility process are not independent, see, e.g. \cite{montiel2022svar}. 
However, the RGMM estimator addresses this limitation by relaxing the independence assumption. Specifically, the estimator can identify the SVAR even if the shocks are driven by the same volatility process, as long as the shocks are sufficiently skewed. Only if all shocks are symmetric, identification requires independent shocks with at most one shock exhibiting zero excess kurtosis.
To achieve this, the    RGMM estimator uses a  modified version of the fast SVAR-GMM estimator proposed by \cite{keweloh2020generalized} which corresponds to a  computationally cheap version of a   generalized method of moments (GMM) estimator minimizing all covariance, coskewness, and cokurtosis moment conditions implied by   mean independent  structural shocks. 



Bayesian approaches offer a natural way to incorporate economic knowledge using the prior distribution of the parameters.   
However, traditional Bayesian SVAR models rely on exact prior information for identification, whereas recent advances  by \cite{baumeister2019structural} and \cite{baumeister2022structural}     propose using imperfect prior information  for identification. 
One limitation of using prior information for identification is that it cannot be updated. Alternatively, Bayesian models identified by independence and non-Gaussianity in principle allow for imposing and updating   economically motivated priors. However, these estimators are not widely used in the literature, with only a few exceptions such as  \cite{lanne2020identification}, \cite{anttonen2021statistically},  \cite{braun2021importance}, and \cite{keweloh2023estimating}.
The  estimator proposed in this study offers a frequentist  alternative to Bayesian methods with the  advantage that the proposed estimator does not require the oftentimes criticized assumption of mutually independent structural shocks.



The application analyzes the interaction of the oil and stock market.
\cite{kilian2009impact} propose  recursive   restrictions  to identify and estimate the effects of different oil     and  stock market shocks. The proposed restrictions are widely used  to analyze the impact of oil market shocks on the stock market, see, e.g., \cite{apergis2009structural}, \cite{abhyankar2013oil}, \cite{kang2013oil},  \cite{sim2015oil}, \cite{ahmadi2016global}, \cite{lambertides2017effects}, \cite{mokni2020time},  \cite{arampatzidis2021oil}, \cite{kwon2022impacts}. However, the impact and importance of stock market shocks for the oil price   is usually not analyzed. The application in this study fills this gap.  I present evidence that   oil   and   stock prices     cannot be ordered recursively. Allowing both variables to interact simultaneously, I find that information shocks derived from the stock market do contain important information on the oil price.  
The non-recursive inclusion of stock market information reveals that a significant portion of the shocks that were previously classified as oil-specific demand shocks can actually be attributed to information shocks. These findings highlights the crucial role of stock market information in explaining oil price fluctuations.





The remainder of the paper is organized as follows: 
Section \ref{sec: Overview: SVAR} contains a brief overview on SVAR models. 
Section \ref{sec: Estimator}  introduces the RGMM estimator.
Section \ref{sec: Finite Sample Performance}  uses a Monte Carlo simulation to illustrate the ability of the  estimator to exploit correctly   and discard falsely imposed restrictions.
Section \ref{sec: Application}  applies the  estimator to an oil and stock market SVAR.
Section \ref{sec: Conclusion} concludes.


\section{Overview: SVAR}
\label{sec: Overview: SVAR}  
Consider the SVAR with $n$ variables
\begin{align}
\label{eq: SVAR}
y_t &=  \nu +  A_1 y_{t-1} + ... + A_p y_{t-p}  + u_t 
\\
\label{eq: simult SVAR}
u_t&=   B_0  \varepsilon_{t} ,
\end{align}
with 
parameter matrices $A_1,...,A_p \in \mathbb{R}^{n \times n}$ which satisfy  $det(I-A_1c-...-A_pc^p)\neq0$ for $|c|\leq 1$,
an intercept term $\nu\in \mathbb{R}^n $, 
an invertible matrix $B_0 \in \mathbb{B} := \{B \in \mathbb{R}^{n \times n} | det(B)\neq 0 \}$, 
an $n$-dimensional vector of  time series  $y_t=[y_{1t} ,...,y_{nt} ]'$,  
an $n$-dimensional vector of reduced form shocks $u_t=[u_{1t} ,...,u_{nt} ]'$,
and an $n$-dimensional vector of  structural shocks $\varepsilon_t=[\varepsilon_{1t},...,\varepsilon_{nt}]'$ with mean zero and unit variance.
The parameter matrices $A_1,...,A_p$ and the intercept term can consistently be estimated to obtain the reduced form shocks $u_t = y_t - \nu - A_1 y_{t-1} - ... - A_p y_{t-p}$. To simplify, I treat the reduced form shocks   as observable random variables  and focus on the simultaneous interaction  in Equation (\ref{eq: simult SVAR}).


Define the  innovations 
\begin{align}
\label{eq: define unmixed innovations}
e(B)_t := B^{-1} u_t, 
\end{align}
equal to the innovations obtained by unmixing the reduced form shocks with a matrix $B\in \mathbb{B}$.
For $B=B_0$, the  innovations are equal to the structural shocks, i.e.,  $e(B_0)_t = \varepsilon_t$. For estimates  $\hat{B}$   of $B_0$, I refer to $e(\hat{B})_t$ as the estimated structural shocks.
In the remainder of this section I show how imposed structure on $B_0$ and $\epsilon_t$ can be used to identify the SVAR, i.e., ensure   that $B=B_0$ and $e(B_0)_t = \varepsilon_t$.

The imposed structure includes assumptions on the mutual (in)dependencies of the structural shocks. Therefore, I first state the definitions of uncorrelated, mean independent, and independent structural shocks. 
Two  shocks are uncorrelated if $E\left[\varepsilon_{it } \varepsilon_{jt }\right]  = E\left[\varepsilon_{it }\right] E\left[\varepsilon_{jt }\right]  $ for $i \neq j$ which has no implications on the higher-order dependencies of the shocks.
The $i$-th structural shock is mean independent of the $j$-th shock if $E\left[ \varepsilon_{it}  g(\varepsilon_{jt })\right]  = E\left[ \varepsilon_{it } \right] E\left[g(\varepsilon_{jt })\right]  $  for $i \neq j$ with a bounded, measurable function $g(\cdot)$, meaning the $j$-th shock contains no information on the mean of the $i$-th shock.
Two structural shocks are independent if   $E\left[h(\varepsilon_{it }) g(\varepsilon_{jt })\right]  = E\left[h(\varepsilon_{it })\right] E\left[g(\varepsilon_{jt })\right]  $ for $i \neq j$  and any  bounded, measurable functions $g(\cdot)$ and $h(\cdot)$, meaning a given shock contains no information on  other shocks.


Almost all identification approaches assume uncorrelated structural shocks. Therefore, the matrix $B$ should  generate uncorrelated  innovations with unit variance, which yields $(n+1)n/2$ moment conditions. However, the matrix $B$ has $n^2$   coefficients. Therefore, infinitely many matrices $B \in \mathbb{B}$ generate uncorrelated  innovations with unit variance, meaning the assumption of uncorrelated structural shocks is not sufficient to identify the SVAR. 
Traditional identification methods solve the identification problem by  imposing   structure on the interaction of the shocks (e.g. short-run restrictions in \cite{sims1980macroeconomics}, long-run restrictions in \cite{blanchard1989dynamic}, sign restrictions  in \cite{uhlig2005effects}, or proxy variables in \cite{mertens2013dynamic}). The probably most frequently imposed structure are   short-run restrictions, meaning restrictions on coefficients of the $B$ matrix which reduce the number of free coefficients to $(n+1)n/2$ such that the remaining unrestricted coefficients are identified by the $(n+1)n/2$ moment conditions implied by uncorrelated shocks with unit variance.   
Note that identification requires at least  $(n-1)n/2$ restrictions to ensure identification. 
Moreover, incorrect  restrictions lead to inconsistent estimates. 
Additionally, with  $(n-1)n/2$ restrictions, the SVAR is just identified. Therefore, even when the sample size goes to infinity, we are not be able to detect incorrect restrictions. 



More recently, identification approaches based on additional  structure imposed on the stochastic properties of the structural shocks have been put forward in the literature. These approaches use properties like time-varying volatility (see, e.g.,  \cite{rigobon2003identification}, \cite{lanne2010structural},   \cite{lutkepohl2017structural}, \cite{lewis2021identifying}, or \cite{bertsche2022identification}) or  the non-Gaussianity and independence of the shocks (see, e.g., \cite{matteson2017independent}, \cite{herwartz2016macroeconomic},  \cite{gourieroux2017statistical}, \cite{lanne2017identification}, \cite{lanne2021gmm}, \cite{keweloh2020generalized},   \cite{guay2021identification}, or \cite{lanne2022identifying}) to ensure identification without any imposed structure on the interaction of the shocks.
In this study, I use information in   third and fourth moments of the shocks to identify a non-Gaussian SVAR. 

The assumption of   mutually independent shocks  can be used to derive   higher-order moment conditions which allow to identify
the $n^2$   coefficients of $B_0$ up to sign and permutation without any restrictions on $B_0$, see, e.g. \cite{lanne2021gmm}, \cite{keweloh2020generalized},   \cite{guay2021identification}, or \cite{lanne2022identifying}. 
For example, if the structural shocks are mutually independent, it holds that the coskewness $E[\epsilon_{it}\epsilon_{jt}\epsilon_{kt}]$ is equal to zero for any combination of structural shocks $i,j,k\in {1,...,n}$ except for $i=j=k$. Analogously to the traditional  approach sketched above,  the matrix $B$ should  generate    innovations which   satisfy the conditions $E[e(B)_{it}e(B)e(B)_{kt}]=0$. The same logic can be applied to derive cokurtosis conditions. Therefore, imposing more structure on the (in)dependencies of the structural shocks allows to derive additional moment conditions, such that the number of moment conditions exceeds the number of unknown parameters of $B_0$. \cite{lanne2021gmm}, \cite{keweloh2020generalized},   \cite{guay2021identification}, and \cite{lanne2022identifying} then derive explicit conditions, including conditions on the non-Gaussianity of the shocks, required to ensure that the higher-order moment conditions identify the SVAR.





\section{Ridge  SVAR-GMM  estimator}
\label{sec: Estimator} 
This section proposes to incorporate short-run restrictions using a ridge penalty with adaptive weights  into the data-driven SVAR estimation approach based on non-Gaussianity and (mean) independent shocks. The restrictions reflect prior economic knowledge and, in contrast to traditional SVAR estimators based on restrictions, the restrictions are not required to ensure identification. Therefore, the researcher is not forced to include a fixed number of restrictions and false restrictions can be detected and neglected.



Let  $\mathcal{R} $ be the set of all tuples $(i,j)$ with $i,j \in \{1,...,n\}$ corresponding to restricted elements $B_{ij}$ and let $R_{ij} \in \mathbb{R}$ be the restrictions corresponding to the element $B_{ij}$. For example, imposing a recursive order implies that all elements in the upper-triangular of $B_0$ are equal to zero, which corresponds to   $R_{ij}=0$ for $(i,j) \in \mathcal{R}=\{(i,j) \in \{1,...,n\}^2 | j>i \}$.  Importantly,  various other identification approaches including long-run restrictions and proxy variables can be written as short-run restrictions and thus can be implemented using the ridge estimator proposed in this section. Moreover, the same approach can be applied to restrictions in an $A$-Type SVAR model referring   to the system $A_0 u_t = \varepsilon_t$ instead of $u_t = B_0 \varepsilon_t$.

In general, a penalized data-driven SVAR estimator can be written as
\begin{align}
\label{eq: Ridge in general}
\hat{B} = \argmin \limits_{B \in \mathbb{B} } \text{ }
&Q(B,u) + \lambda \sum_{(i,j)\in \mathcal{R}}^ {} v_{ij} p(B_{ij},R_{ij}) 
\\ \nonumber
&\text{s.t. } J(B,u) = 0,
\end{align}
where
$Q(B,u): \mathbb{B},\mathbb{R}^{T \times n}  \rightarrow \mathbb{R}$ is a loss  function,  $p(B_{ij},R_{ij}): \mathbb{R},\mathbb{R} \rightarrow \mathbb{R}^{\geq 0 }$ is a penalty function which penalizes deviations of $B_{ij}$ from $R_{ij}$,  
$v_{ij} $ are positive data-dependent weights for the penalty of the element  $B_{ij}$,  
$\lambda$ is a non-negative tuning parameter,  and $J(B,u): \mathbb{B},\mathbb{R}^{T \times n} \rightarrow \mathbb{R}^s$ contains $s$ further constraints, e.g.  constraints on values of the $B$ matrix   or constraints on the combination of $B$ and $u$ like for instance narrative sign restrictions or the constraint that $B$ unmixes $u$ into innovations with unit variance.
Importantly, the combination of the loss function $Q(B,u)$ and constraints $J(B,u)$ need to ensure identification of the SVAR.



In this study, I use a loss function  and constraints  which correspond to a  modified version  of the fast SVAR-GMM estimator in \cite{keweloh2020generalized}. The loss and constraints lead to a computationally cheap estimator which is identified under the assumption of skewed mutually mean independent shocks  or independent shocks with non-zero excess kurtosis, see Proposition \ref{proposition: 1} below.
In particular, this study uses the loss 
\begin{align}
\label{eq: Q}
Q(B,u) :=& 
- \sum_{i=1}^{n} \left( 1/ T  \sum_{t=1}^{T} e(B)_{it}^3 \right)^2 
- \sum_{i=1}^{n}  \left( 1/ T  \sum_{t=1}^{T} e(B)_{it}^4 -3  \right)^2 
\\ \nonumber
&- 6  \sum_{i=1}^{n}  \sum_{i=j+1}^{n}\left( 1/ T  \sum_{t=1}^{T} e(B)_{it}^2 e(B)_{jt}^2 -1 \right)^2  
\end{align}
and the   constraint   
\begin{align}
\label{eq: J}
J(B,u) :=& 1/ T  \sum_{t=1}^{T} e(B)_{t} e(B)_{t}' - I,
\end{align}  
which  ensure  that the estimated structural shocks are uncorrelated with unit variance.
For any $B$ matrix satisfying the constraint $J(B,u)$,  the first term of the loss in  Equation (\ref{eq: Q}) is equal to a weighted sum of all squared coskewness and cokurtosis conditions implied by mutually independent shocks, see \cite{keweloh2020generalized}. I propose to add  the second term   to the loss function in Equation (\ref{eq: Q}), such that the loss   $Q(B,u)$ under the   constraint $J(B,u)$ is equal to a weighted sum of all squared coskewness and cokurtosis conditions implied by mutually mean independent shocks.
In particular,  the   loss $Q(B,u)$ under the constraint $J(B,u)$    can be written as
\begin{align}
\nonumber
Q(B,u) = \omega 
&+ 
3\sum_{i =1 }^{n } 
\sum_{j  =1 ; j \neq i}^{n} 
\left( 1/ T  \sum_{t=1}^{T} e(B)_{it}^2 e(B)_{jt} \right)^2 
\\ \nonumber
& 
+  6
\sum_{i =1 }^{n } 
\sum_{j =i+1 }^{n } 
\sum_{k =j+1 }^{n } 
\left( 1/ T  \sum_{t=1}^{T} e(B)_{it}  e(B)_{jt} e(B)_{kt} \right)^2 
\\ 
\label{eq: Q1 equiv}
& 
+ 4
\sum_{i =1 }^{n } 
\sum_{j  =1 ; j\neq i}^{n} 
\left( 1/ T  \sum_{t=1}^{T} e(B)_{it}^3  e(B)_{jt} \right)^2 
\\ \nonumber
& 
+ 12
\sum_{i =1 }^{n } 
\sum_{j  =1;  j\neq i}^{n} 
\sum_{k = j+1;  k\neq i }^{n } 
\left( 1/ T  \sum_{t=1}^{T} e(B)_{it}^2  e(B)_{jt} e(B)_{kt} \right)^2 
\\ \nonumber
& 
+ 24
\sum_{i =1 }^{n } 
\sum_{j  =i+1}^{n} 
\sum_{k = j+1 }^{n } 
\sum_{l = k+1 }^{n } 
\left( 1/ T  \sum_{t=1}^{T} e(B)_{it}   e(B)_{jt} e(B)_{kt} e(B)_{lt} \right)^2, 
\end{align}
where $\omega$ is a scalar invariant with respect to $B$ satisfying $J(B,u)$ in Equation (\ref{eq: J}), compare \cite{keweloh2020generalized}. 
Importantly, the number of coskewness and cokurtosis  conditions in Equation (\ref{eq: Q1 equiv})  increases quickly with the dimension of the SVAR and thus leads to an increase of the  computational complexity.  The advantage of the  loss $Q(B,u)$  in Equation (\ref{eq: Q}) is that  it remains computationally cheap to evaluate even when the dimension of the SVAR increases.


The following proposition provides the conditions under which the loss in Equation (\ref{eq: Q}) and the constraint  in  Equation (\ref{eq: J}) identify the SVAR.
\begin{proposition}
	\label{proposition: 1}
	Consider the SVAR $u_t = B_0 \epsilon_t$ with   $B_0\in  \mathbb{B} $ and structural shocks with  zero mean, unit variance, and
	finite third and fourth moments.  
	\begin{enumerate} 
		\item If the components of    the structural shocks are mutually mean independent and at   most one component of the structural shocks has zero skewness,  global identification of $B_0$ up to sign and permutation follows from \cite{bonhomme2009consistent}.
		\item If the components of    the structural shocks are mutually   independent  and at   most one component of the structural shocks has zero excess kurtosis,  local identification of $B_0$   follows from \cite{lanne2021gmm}.
	\end{enumerate}  
\end{proposition}
\begin{proof}
	See the appendix. 
\end{proof}
The proposition shows that if the shocks are sufficiently skewed, identification only requires mean independent shocks and thus allows that multiple shocks are driven by the same volatility process.
Only if the shocks are not sufficiently skewed, identification relies on independent shocks with sufficient excess kurtosis.

To be precise, the proof of the first statement in Proposition \ref{proposition: 1} technically only requires that coskewness conditions implied by  mutually independent shocks hold, which follows from mutually mean independent shocks. 
As for the second identification statement, it relies on the local identification result   in \cite{lanne2021gmm}, which is based on asymmetric cokurtosis conditions. Nonetheless, the proof of the proposition technically assumes that all cokurtosis conditions (not just the asymmetric ones) implied by mutually independent shocks hold.  Therefore, the statement does not necessarily require independent shocks, but it necessitates that all cokurtosis conditions resulting from independent shocks hold. However, the conditions do not follow from mean independent shocks and finding an economically plausible process other than independent shocks that would yield shocks satisfying all such conditions is not straightforward.





Note that the fast SVAR-GMM estimator is not necessarily asymptotically efficient, meaning  an efficiently weighted SVAR-GMM estimator  can have a smaller asymptotic variance. However, with higher-order moment conditions  the asymptotically efficient weighting matrix is difficult to estimate in sample sizes and in this case would require finite moments up to order eight.
\cite{keweloh2021Afeasible} shows that standard approaches to estimate the asymptotically efficient weighting matrix lead to  estimators with poor small sample performance. 
Instead, the goal of this study is to  use structure from economic theory included via the penalty function to increase the precision of   estimation in small samples.





In particular, I use a $\mathit{L}_2$ (ridge) penalty 
\begin{align}
\label{eq: p}
p(B_{ij},R_{ij}) = %w_{ij} 
(B_{ij}-R_{ij})^2 
\end{align} 
%$w_{ij}\in  $ .... set by the researcher ...., 
and adaptive weights $v_{ij}= 1/(\hat{B}_{ij(0)}-R_{ij})^2$ where $\hat{B}_{ij(0)}$ is an initial consistent estimator of the corresponding $B_0$ element, i.e. obtained by a data-driven estimator without penalties. Due to the adaptive weights, it becomes more costly to deviate from the restriction for elements where the initial estimate is close to the restriction and less costly if the initial estimate  is further away from the restriction, compare \cite{zou2006adaptive} for adaptive Lasso and \cite{dai2018broken} for adaptive Ridge estimators.\footnote{
	The weights are implemented using the formula provided by \cite{frommlet2016adaptive} to avoid numerical instabilities.
}   



Proposition \ref{proposition: 1} only ensures   identification up to sign and permutation. 
The indeterminacy implies that it is not possible to statistically discriminate between different orders of the structural shock, e.g. for any sign-permutation matrix $P$ the models  $u_t = B_0  \epsilon_t$ and $u_t = \tilde{B}_0 \tilde{\epsilon}_t$ with $\tilde{B}_0= B_0 P^{-1}$ and $\tilde{\epsilon}_t= P \epsilon_t$ have the same dependency structure and hence it is not possible to use the assumption of independent shocks to discriminate between different ordering of the shocks.  In contrast to traditional restriction based estimators where the restrictions imply the labels a priori to the estimation, the shocks estimated using a data-driven estimator are typically labeled a posterior to the estimation. However, if the position of a structural shock of interest is unknown a priori, it is not possible to impose a priori restrictions or shrinkage on the impact of the structural shock. 
Therefore, similar to \cite{keweloh2023estimating}, I  use a constrained set  of admissible mixing matrices   containing unique sign-permutation representatives centered at an initial labeled estimator or guess, $\bar{B}$ of $B_0$, i.e.,
\begin{align}
\bar{\mathbb{B}}_{\bar{B}} := \{B \in \mathbb{B}|& |C_{kk}|>|C_{kl}| \text{ for } l=k+1,...,n \text{ and } C:= \bar{B}^{-1} B D \\ \nonumber
&\text{ and a scaling matrix $D$ scaling each column of $C$ to Euclidean norm one}\}.
\end{align}
For $\bar{B}=I$ the  set $\bar{\mathbb{B}}_{\bar{B}=I} $ is equal to the set of unique sign-permutation representatives proposed by \cite{lanne2017identification}. Without centering the set near $B_0$, the set can contain estimators corresponding to different orders of the shocks.	 For example, consider the bivariate SVAR with
\begin{align}
\nonumber
B_0 = \begin{bmatrix}
\sqrt{0.5} & -\sqrt{0.499}  \\
\sqrt{0.5}  &  \sqrt{0.501}  
\end{bmatrix}
\quad
\text{ and }
u_t = B_0 \begin{bmatrix}
\epsilon_{1t}\\ \epsilon_{2t}
\end{bmatrix}
\end{align}
with $B_0 \in \bar{\mathbb{B}}_{\bar{B}=I}$. Consider an estimator $\hat{B}$ and its sign permutation $\tilde{B}$ with
\begin{align}
\hat{B}= \begin{bmatrix}
\sqrt{0.5} & - \sqrt{0.501}  \\
\sqrt{0.5}  &  \sqrt{0.499}  
\end{bmatrix}
\quad \text{ and } \quad
\tilde{B}= \begin{bmatrix}
\sqrt{0.501} & \sqrt{0.5}  \\
-\sqrt{0.499}  &  \sqrt{0.5}  
\end{bmatrix}
\end{align}
which both solve the optimization problem in Equation (\ref{eq: Ridge  SVAR GMM}). Obviously, $\hat{B}$ corresponds to the same order of the shocks  while $\tilde{B}$ corresponds to the reverse order   compared to the  order in $B_0$. However, even with $B_0 \in \bar{\mathbb{B}}_{\bar{B}=I}$ the estimator $\hat{B}$ close to $B_0$ is not contained in $ \bar{\mathbb{B}}_{\bar{B}=I}$ while the estimator $\tilde{B}$ corresponding to the reverse order is contained in $ \bar{\mathbb{B}}_{\bar{B}=I}$. This problem of different orders contained in $ \bar{\mathbb{B}}_{\bar{B}}$ occurs if $B_0$ is located close to the boundary of  $ \bar{\mathbb{B}}_{\bar{B}}$. Centering the set close to $B_0$ mitigates the problem, i.e., $\hat{B}$ corresponding to the correct order is contained in $ \bar{\mathbb{B}}_{\bar{B}=B_0}$ and $\tilde{B}$ corresponding to the reverse order is not.
The initial  labeled estimator or guess $\bar{B}$ determines the order of the structural shocks a priori to the estimation of the penalized data-driven estimator  which allows to apply the shrinkage penalty on the impact of structural shocks in the set $ \bar{\mathbb{B}}_{\bar{B}}$. 



Put together, the  Ridge  SVAR-GMM (RGMM) estimator   is given by
\begin{align}
\nonumber
\hat{B}  = \argmin \limits_{B \in \bar{\mathbb{B}}_{\bar{B}} } \text{ }
&- \sum_{i=1}^{n} \left( 1/ T  \sum_{t=1}^{T} e(B)_{it}^3 \right)^2 
- \sum_{i=1}^{n}  \left( 1/ T  \sum_{t=1}^{T} e(B)_{it}^4 -3  \right)^2 
\\  \nonumber
&- 6  \sum_{i=1}^{n}  \sum_{i=j+1}^{n}\left( 1/ T  \sum_{t=1}^{T} e(B)_{it}^2 e(B)_{jt}^2 -1 \right)^2 
\\ 
\label{eq: Ridge  SVAR GMM}
&+ \lambda \sum_{(i,j)\in \mathcal{R}}^{ }  %w_{ij} 
\frac{(B_{ij}-R_{ij})^2}{(\hat{B}_{ij(0)}-R_{ij})^2}  
\\
\nonumber
&\text{s.t. } 1/ T  \sum_{t=1}^{T} e(B)_{t} e(B)_{t}' - I.
\end{align} 
The RGMM estimator has several appealing features. 
First,  identification  only requires that the coskewness and cokurtosis conditions implied by mutually mean independent shocks  hold. In particular, all moment conditions remain valid if multiple shocks are driven by the same variance process.
Second, the estimator allows to include restrictions motivated by prior knowledge from economic theory in a non-invasive manner, meaning restrictions which fit the data are costly to discard whereas restrictions which do not fit the data get less penalized and deviations from these restrictions are less costly.  
Third, due to the loss and the  $\mathit{L}_2$ penalty, the estimator is computationally cheap.


With  the $\mathit{L}_2$ penalty, the RGMM estimator never shrinks   $\hat{B}_{ij}$ to $R_{ij}$   and thus does not allow to select valid restrictions.
Using a $\mathit{L}_1$ (Lasso) penalty would allow to shrink $\hat{B}_{ij}$ exactly to $R_{ij}$ and thus allows  to select valid restrictions. However,  the loss function $Q(B,u)$ is non-linear and      solving the optimization problem is  computationally demanding with the   $\mathit{L}_1$ penalty. \cite{dai2018broken} propose the broken adaptive ridge estimator which iteratively updates the adaptive weights of the $\mathit{L}_2$ penalty and show that the estimator approximates an $\mathit{L}_0$ penalty and allows to select variables.
%  For the RGMM estimator and $k=1,2,....$ let $\hat{B}_{ij(k)} \rightarrow \tilde{B}_{ij}$  
%   such that the iterative penalty is equal to
%  \begin{align}
%  	\lambda \sum_{(i,j)\in \mathcal{R}}^{ }  %w_{ij} 
%  	\frac{(\hat{B}_{ij(k )}-R_{ij})^2}{(\hat{B}_{ij(k-1)}-R_{ij})^2}
%  	\overset{k\rightarrow\infty}{\rightarrow}
%  	\lambda \sum_{(i,j)\in \mathcal{R}}^{ }  %w_{ij} 
%  	|\tilde{B}_{ij}-R_{ij}|_0,
%  \end{align}
%  where $|\tilde{B}_{ij}-R_{ij}|_0$   is the $L_0$ penalty.
If  selecting restrictions is desired, an analogous iterative procedure can be applied to the RGMM estimator.





\section{Finite sample performance}
\label{sec: Finite Sample Performance} 
The following Monte Carlo study illustrates the benefits of correctly imposed restrictions via the penalty term of the RGMM estimator  and   sheds light on its ability to distinguish between correct and incorrect restrictions. Correctly imposed restrictions via the penalty term  lead to an increase in performance in terms of bias and MSE  and the impact of false restrictions decreases with increasing sample size.


I simulate an   SVAR     with four variables  
\begin{align}
\label{eq: MC} 
\begin{bmatrix}
u_{1t} \\
u_{2t} \\
u_{3t} \\
u_{4t} \\
\end{bmatrix} =
\begin{bmatrix}
10 & 0  & 0 & 0 \\
5 & 10 & 0 & 0\\
5 & 5 & 10 & 5\\
5 & 5 & 5 & 10
\end{bmatrix}
\begin{bmatrix}
\varepsilon_{1t} \\
\varepsilon_{2t} \\
\varepsilon_{3t} \\
\varepsilon_{4t} \\
\end{bmatrix}  ,
\end{align}  
where the structural shocks are independently and identically     drawn from   the  two-component mixture
$
\epsilon_{it} \sim  0.79\; \mathcal{N}(-0.2,0.7^2) + 0.21 \;  \mathcal{N}(0.75,1.5^2),
$
where  $\mathcal{N}(\mu, \sigma^2)$ indicates a normal distribution with mean $\mu$ and standard deviation $\sigma$. The shocks have skewness $0.9$ and excess kurtosis $2.4$.  


The estimators use  two different sets of restrictions for the penalty
\begin{align}
R_1 =  
\begin{bmatrix}
. & 0  & 0 & 0 \\
. & . & 0 & 0 \\
. & . & .  & .  \\
. & . & .  & . 
\end{bmatrix} 
\text{ and } 
R_2 =  
\begin{bmatrix}
. & 0  & 0 & 0 \\
. & . & 0 & 0 \\
. & . & .  & 0  \\
. & . & .  & . 
\end{bmatrix}.  
\end{align}
The first set of restrictions $R_1$ contains the correct zero restrictions.
The second set   $R_2$ imposes a recursive structure and thus contains one incorrect restrictions. 
In each simulation, I calculate three estimators:
The first estimator, denoted by fGMM,  is   the modified fast SVAR GMM estimator and uses no restrictions. 
The second estimator, denoted by RGMM($R_1$), is the RGMM estimator with a penalty based on the restrictions $R_1$.
The third estimator, denoted by RGMM($R_2$), is the RGMM estimator with a penalty based on the restrictions $R_2$. The adaptive weights of the  RGMM estimators are calculated based on the unrestricted fGMM estimator.
The tuning parameter $\lambda$ of the RGMM estimators is chosen using a repeated cross-validation  with  two folds, $50$ repetitions, and a sequence of $40$  potential $\lambda$ values.\footnote{
	The loss in the let-out fold in the cross-validation is calculated as the weighted GMM loss with the variance and covariance conditions from the constraint   $J(B,u)$ and the coskewness and cokurtosis conditions implied by mutually mean independent shock and each moment is weighted by the inverse of the variance of the corresponding moment under the assumption of independent and normal  distributed shocks. 
}
The tuning parameter is chosen as the maximum of the parameters minimizing the median, $40$\% and $60$\% quantiles of the loss in the let-out fold. 

Table \ref{Table: Finite sample performance (MoG) - recursive SVAR} shows the average and MSE of each estimated element for the three estimators.  
\begin{table}
	\caption{Finite sample performance.}
	\label{Table: Finite sample performance (MoG) - recursive SVAR}
	\begin{tabular}{ c  | c    c   c      }
		
		&$fGMM$&$ RGMM(R_1)$&$ RGMM(R_2)$\\ \hline 
		
		\begin{turn}{90}\,$T = 100$ \end{turn} &
		
		$\begin{bmatrix}\underset{(4.64)}{8.94} & \underset{(5.63)}{-0.06} & \underset{(5.64)}{-0.03} & \underset{(5.26)}{-0.11} \\ \underset{(6.74)}{4.52} & \underset{(6.51)}{8.93} & \underset{(7.09)}{0.01} & \underset{(6.88)}{-0.01} \\ \underset{(9.58)}{4.59} & \underset{(10.14)}{4.42} & \underset{(7.82)}{8.99} & \underset{(9.19)}{4.38} \\ \underset{(9.51)}{4.66} & \underset{(10.25)}{4.38} & \underset{(8.73)}{4.53} & \underset{(9.02)}{8.87} \\ \end{bmatrix}$ 
		
		&
		
		$\begin{bmatrix}\underset{(3.6)}{9.65} & \underset{(1.29)}{-0.02} & \textcolor{red}{\underset{(1.06)}{-0.02}} & \textcolor{red}{\underset{(1.09)}{-0.06}} \\ \underset{(3.18)}{4.87} & \underset{(4.24)}{9.6} & \textcolor{red}{\underset{(1.55)}{-0.01}} & \textcolor{red}{\underset{(1.31)}{-0.0}} \\ \underset{(3.78)}{4.92} & \underset{(4.24)}{4.74} & \underset{(5.74)}{9.34} & \underset{(7.61)}{4.57} \\ \underset{(3.85)}{4.97} & \underset{(3.95)}{4.71} & \underset{(7.32)}{4.72} & \underset{(6.49)}{9.24} \\ \end{bmatrix}$ 
		
		&
		
		$\begin{bmatrix}\underset{(3.66)}{9.57} & \textcolor{red}{\underset{(1.75)}{-0.01}} & \textcolor{red}{\underset{(1.37)}{-0.02}} & \textcolor{red}{\underset{(1.73)}{-0.17}} \\ \underset{(3.4)}{4.84} & \underset{(4.47)}{9.52} & \textcolor{red}{\underset{(2.05)}{-0.04}} & \textcolor{red}{\underset{(2.32)}{-0.18}} \\ \underset{(4.22)}{4.92} & \underset{(4.85)}{4.8} & \underset{(5.17)}{9.97} & \textcolor{red}{\underset{(13.2)}{2.65}} \\ \underset{(4.48)}{5.04} & \underset{(4.86)}{4.84} & \underset{(9.65)}{6.27} & \underset{(9.56)}{8.03} \\ \end{bmatrix}$  
		
		\\	& & & \\
		\\  \begin{turn}{90}\,$T = 250$ \end{turn} &
		
		$\begin{bmatrix}\underset{(1.36)}{9.53} & \underset{(2.98)}{-0.04} & \underset{(2.55)}{-0.05} & \underset{(2.59)}{-0.03} \\ \underset{(3.37)}{4.8} & \underset{(2.5)}{9.47} & \underset{(3.29)}{0.04} & \underset{(3.46)}{-0.02} \\ \underset{(4.3)}{4.85} & \underset{(4.82)}{4.68} & \underset{(3.47)}{9.53} & \underset{(4.74)}{4.72} \\ \underset{(4.25)}{4.85} & \underset{(4.6)}{4.69} & \underset{(4.49)}{4.78} & \underset{(3.78)}{9.51} \\ \end{bmatrix}$ 
		
		&
		
		$\begin{bmatrix}\underset{(0.85)}{9.89} & \underset{(0.52)}{0.0} & \textcolor{red}{\underset{(0.4)}{-0.02}} & \textcolor{red}{\underset{(0.39)}{0.01}} \\ \underset{(1.14)}{4.96} & \underset{(1.42)}{9.84} & \textcolor{red}{\underset{(0.46)}{0.02}} & \textcolor{red}{\underset{(0.65)}{-0.01}} \\ \underset{(1.3)}{4.97} & \underset{(1.54)}{4.91} & \underset{(2.03)}{9.74} & \underset{(3.58)}{4.83} \\ \underset{(1.29)}{4.96} & \underset{(1.56)}{4.92} & \underset{(3.38)}{4.9} & \underset{(2.16)}{9.72} \\ \end{bmatrix}$ 
		
		&
		
		$\begin{bmatrix}\underset{(0.96)}{9.83} & \textcolor{red}{\underset{(1.02)}{0.01}} & \textcolor{red}{\underset{(0.64)}{-0.03}} & \textcolor{red}{\underset{(0.83)}{-0.07}} \\ \underset{(1.65)}{4.91} & \underset{(1.73)}{9.77} & \textcolor{red}{\underset{(0.9)}{0.02}} & \textcolor{red}{\underset{(1.18)}{-0.12}} \\ \underset{(1.75)}{4.98} & \underset{(2.45)}{4.9} & \underset{(1.81)}{10.2} & \textcolor{red}{\underset{(8.16)}{3.34}} \\ \underset{(1.87)}{5.01} & \underset{(2.44)}{4.94} & \underset{(5.46)}{6.1} & \underset{(4.81)}{8.78} \\ \end{bmatrix}$ 
		
		
		\\	& & & \\
		\\ \begin{turn}{90}\,$T = 500$ \end{turn} &
		
		
		$\begin{bmatrix}\underset{(0.5)}{9.79} & \underset{(1.16)}{-0.03} & \underset{(1.32)}{-0.01} & \underset{(1.26)}{0.0} \\ \underset{(1.24)}{4.92} & \underset{(0.81)}{9.78} & \underset{(1.6)}{-0.05} & \underset{(1.36)}{-0.01} \\ \underset{(2.12)}{4.92} & \underset{(1.99)}{4.93} & \underset{(1.53)}{9.75} & \underset{(1.96)}{4.9} \\ \underset{(2.11)}{4.9} & \underset{(1.77)}{4.92} & \underset{(2.1)}{4.86} & \underset{(1.37)}{9.79} \\ \end{bmatrix}$ 
		
		&
		
		$\begin{bmatrix}\underset{(0.43)}{9.96} & \underset{(0.1)}{-0.0} & \textcolor{red}{\underset{(0.15)}{0.0}} & \textcolor{red}{\underset{(0.16)}{-0.0}} \\ \underset{(0.36)}{4.97} & \underset{(0.42)}{9.94} & \textcolor{red}{\underset{(0.16)}{-0.0}} & \textcolor{red}{\underset{(0.18)}{0.01}} \\ \underset{(0.53)}{4.98} & \underset{(0.5)}{4.98} & \underset{(0.73)}{9.88} & \underset{(1.48)}{4.95} \\ \underset{(0.53)}{4.98} & \underset{(0.51)}{4.97} & \underset{(1.47)}{4.94} & \underset{(0.72)}{9.9} \\ \end{bmatrix}$ 
		
		&
		
		$\begin{bmatrix}\underset{(0.48)}{9.93} & \textcolor{red}{\underset{(0.29)}{-0.02}} & \textcolor{red}{\underset{(0.31)}{-0.0}} & \textcolor{red}{\underset{(0.36)}{-0.04}} \\ \underset{(0.5)}{4.98} & \underset{(0.6)}{9.9} & \textcolor{red}{\underset{(0.4)}{-0.03}} & \textcolor{red}{\underset{(0.37)}{-0.08}} \\ \underset{(0.73)}{5.0} & \underset{(0.78)}{5.0} & \underset{(0.79)}{10.19} & \textcolor{red}{\underset{(3.51)}{4.05}} \\ \underset{(0.74)}{5.01} & \underset{(0.74)}{5.02} & \underset{(2.55)}{5.69} & \underset{(1.8)}{9.35} \\ \end{bmatrix}$ 
		
		
		\\	& & & \\
		\\  \begin{turn}{90}\,$T = 1000$ \end{turn} &
		
		$\begin{bmatrix}\underset{(0.15)}{9.92} & \underset{(0.46)}{0.01} & \underset{(0.46)}{0.0} & \underset{(0.45)}{0.02} \\ \underset{(0.5)}{4.95} & \underset{(0.25)}{9.93} & \underset{(0.57)}{0.0} & \underset{(0.6)}{0.02} \\ \underset{(0.74)}{4.96} & \underset{(0.72)}{4.96} & \underset{(0.39)}{9.94} & \underset{(0.72)}{4.98} \\ \underset{(0.74)}{4.96} & \underset{(0.72)}{4.96} & \underset{(0.65)}{4.98} & \underset{(0.5)}{9.94} \\ \end{bmatrix}$ 
		
		&
		
		$\begin{bmatrix}\underset{(0.11)}{9.98} & \underset{(0.04)}{0.01} & \textcolor{red}{\underset{(0.03)}{0.0}} & \textcolor{red}{\underset{(0.03)}{-0.0}} \\ \underset{(0.14)}{4.99} & \underset{(0.11)}{9.99} & \textcolor{red}{\underset{(0.04)}{-0.0}} & \textcolor{red}{\underset{(0.04)}{0.0}} \\ \underset{(0.18)}{5.01} & \underset{(0.17)}{5.0} & \underset{(0.21)}{9.97} & \underset{(0.51)}{4.98} \\ \underset{(0.18)}{5.01} & \underset{(0.17)}{5.0} & \underset{(0.46)}{4.99} & \underset{(0.26)}{9.97} \\ \end{bmatrix}$ 
		
		&
		
		$\begin{bmatrix}\underset{(0.11)}{9.98} & \textcolor{red}{\underset{(0.09)}{0.01}} & \textcolor{red}{\underset{(0.06)}{-0.0}} & \textcolor{red}{\underset{(0.07)}{-0.02}} \\ \underset{(0.18)}{4.99} & \underset{(0.13)}{9.98} & \textcolor{red}{\underset{(0.09)}{-0.0}} & \textcolor{red}{\underset{(0.16)}{-0.04}} \\ \underset{(0.23)}{5.02} & \underset{(0.23)}{5.0} & \underset{(0.25)}{10.17} & \textcolor{red}{\underset{(1.0)}{4.51}} \\ \underset{(0.24)}{5.03} & \underset{(0.27)}{5.03} & \underset{(0.76)}{5.41} & \underset{(0.53)}{9.7} \\ \end{bmatrix}$ 
		
		
	\end{tabular}
	\begin{minipage}{1\textwidth} %
		{   \footnotesize  
			\textit{Note:}	
			Monte Carlo simulation with $M=1000$ replications for the  SVAR in Equation (\ref{eq: MC}).
			The table shows  the average,  $ 1/M \sum_{m=1}^{M} \hat{b}^m_{ij}  $  and   in parentheses the  mean squared error   $    1/M \sum_{m=1}^{M}    \left(  \hat{b}^m_{ij} - b_{ij} \right)^2$ of each estimated element $\hat{b}^m_{ij}$ in simulation $m$.
			Penalized elements are highlighted in red.  
			\par}
	\end{minipage}
\end{table}    
Imposing structure with the correct $R_1$   penalty substantially reduces the average bias and MSE compared to the unpenalized  fGMM estimator. The greatest performance improvements are observed in the penalized elements, where the MSE of the RGMM($R_1$) estimator is approximately $70-80$\% smaller than that of the fGMM estimator. Importantly, the penalty also enhances the performance of the unpenalized elements, where the MSE of the RGMM($R_1$) estimator is approximately  $50-70$\% smaller than that of the fGMM estimator.

In contrast to the $R_1$ penalty, the $R_2$ penalty   contains a false restriction that shrinks $B_{34}$ to zero, even though the true value of the parameter is five.  The simulation shows that the estimator is able to detect and dismiss false restrictions. Specifically, the  RGMM($R_2$) estimator deviates from the incorrect restriction and does not force the estimated $\hat{B}_{34}$  element to zero. Nevertheless, the $R_2$ penalty leads to an increase of the bias and MSE of the $B_{34}$ element,    however,  the bias and MSE decreases with the sample size. At the same time, the correct restrictions of the $R_2$ penalty lead to a performance increase of the correctly penalized and also unpenalized estimated elements. Overall, the positive impact of the correct restrictions outweighs  the impact of the incorrect restrictions, and the overall performance of the  RGMM($R_2$) estimator is superior to the performance of unrestricted fGMM estimator.


These results demonstrate that the RGMM estimator can effectively use correctly imposed restrictions to achieve more precise estimates. Moreover, the estimator can detect and ignore incorrect restrictions, especially as more data and information become available against a given restriction.






\section{Application}
\label{sec: Application} 
This section  analyzes the effects of different approaches to incorporate   prior knowledge on the oil and stock market interaction. 
The results using the traditional recursive approach suggest that the stock market provides no additional information about  the oil price. 
In contrast, when prior knowledge is incorporated through the proposed ridge estimator, it becomes apparent that information shocks originating from stock prices are significant drivers of oil price fluctuations.  Specifically, including stock market information in a non-recursive manner reveals  that a considerable portion of what is usually classified as oil-specific demand shocks can actually be attributed to information shocks. These findings highlight the importance of  stock market information when examining oil price fluctuations, as it provides valuable insights into the underlying forces driving the oil market.


\subsection{Specification and estimators}
The  SVAR  uses  monthly data from  February $1974$ to September $2022$ with 
\begin{align} 
\label{eq: application svar}   
\begin{bmatrix}
q_t
\\
y_t 
\\
p_t
\\
s_t 
\end{bmatrix}
=
\alpha  
+
\sum_{i=1}^{12}
A_i
\begin{bmatrix}
q_{t-i}
\\
y_{t-i} 
\\
p_{t-i}
\\
s_{t-i} 
\end{bmatrix}
+
\begin{bmatrix}
b_{11} & b_{12}  &  b_{13}  & b_{14} \\
b_{21} & b_{22} & b_{23}   & b_{24}   \\
b_{31}  & b_{32} & b_{33}    &  b_{34}    \\ 
b_{41} & b_{42} & b_{43}   & b_{44} 
\end{bmatrix}
\begin{bmatrix}
\varepsilon_{S,t} 
\\
\varepsilon_{Y,t}  
\\
\varepsilon_{D,t} 
\\
\varepsilon_{SM,t} 
\end{bmatrix},
\end{align} 
where $q_t$ is $100$ times the log   of world crude oil production, $y_t$ is $100$ times the log   of  global industrial production, $p_t$ is  $100$ times the log   of the real oil price,   and $s_t$ is $100$ times the log of a monthly U.S. stock price index.\footnote{
	Global oil production is given by the global crude oil including lease condensate production obtained from the U.S. EIA. Global industrial production is given by the  monthly industrial production index in the OECD and six major other
	countries obtained from \cite{baumeister2019structural}. The real oil price is equal to the refiner's acquisition cost of imported crude oil  from the U.S. EIA deflated by the U.S. CPI. Real stock prices correspond to the aggregate   U.S. stock index constructed by the OECD deflated by the U.S. CPI.
}


\cite{kilian2009impact} estimate a similar four variable oil and stock market model  and propose to identify four shocks using a recursive order.\footnote{
	The model analyzed  in \cite{kilian2009impact} uses a slightly different specification. Specifically, the authors use an economic activity index   based on   shipping costs.  However, as noted by  \cite{baumeister2022energy},  the shipping index may not always be a reliable indicator of changes in global economic activity. Therefore,   I follow the approach taken by \cite{baumeister2019structural} and   use a conventional measure of economic activity based on industrial production.
	Despite the potential limitations of the shipping index, \cite{baumeister2022energy} note that it can serve as a forward-looking indicator. As such, I include an alternative specification in the appendix where the shipping index replaces the stock price. The results indicate that information shocks based on the shipping index have a similar impact on the oil price compared to information shocks based on stock prices analyzed in this section.
}
Specifically, in the recursive model  economic activity shocks $\varepsilon_{Y,t} $ cannot simultaneously affect oil supply, oil-specific demand shocks $\varepsilon_{D,t} $ cannot simultaneously affect oil supply and economic activity, and stock market information shocks $\varepsilon_{SM,t} $ cannot simultaneously affect oil supply, economic activity, and the oil price. 

The recursive restrictions have two major limitations. 
Firstly, the reduced form oil supply shocks are, by construction, identified as  structural oil supply shocks which  implies that   oil supply cannot respond simultaneously   to demand shocks.
Secondly, the reduced form  oil price shocks that cannot be explained by supply and economic activity shocks are, by construction, identified as oil-specific demand shocks. However, if the oil price responds immediately  to information shocks extracted from stock prices, these information shocks would end up in the oil-specific demand shock of the recursive model. 
The former issue regarding the response of oil supply to non-supply shocks received a lot of attention in the literature, see, e.g. \cite{kilian2012agnostic}, \cite{kilian2014role}, \cite{baumeister2019structural}, \cite{caldara2019oil}, and \cite{braun2021importance}, while the latter issue on the impact of stock market information shocks on the oil price received little attention.



In contrast to the recursive estimator, which relies on a fixed number of restrictions to ensure identification, the proposed RGMM estimator    does not use restrictions to ensure identification. As a result, it does not require to impose the two assumptions in question. 
The simulations  in the previous section show  that the ridge estimator can handle false restrictions, however, including them can lead to a small sample bias and a performance decrease. Therefore, the  ridge estimator  in this section does not impose the two restrictions in question.  
Instead, the  estimator  uses the   set of  restrictions: 
\begin{align}
\label{eq: application R}
R = \begin{bmatrix}
. & 0  & . & 0 \\
0 & . & 0 & 0 \\
. & . & .  & .  \\
. & . & .  & . 
\end{bmatrix}.
\end{align}
This allows oil supply to   simultaneously respond to demand shocks $\varepsilon_{D,t}$ and it   allows the oil price to simultaneously respond to stock market information shocks $\varepsilon_{SM,t}$. Furthermore, the restriction in Equation (\ref{eq: application R}) additionally incorporates the assumption that economic activity behaves sluggishly and does not simultaneously respond to oil supply shocks, which is the same argument motivating the zero response of economic activity to oil-specific demand and stock market information shocks.







I use three different estimators to analyze the impact of incorporating prior economic knowledge on the interaction between the oil and stock market  in the SVAR model:
\begin{enumerate}
	\item Recursive: Recursive SVAR  estimated using a Cholesky decomposition.
	
	\item Ridge:   RGMM estimator   penalizing deviations from   $R$ in Equation (\ref{eq: application R}).
	
	\item Unrestricted:   RGMM estimator without a penalty.  
\end{enumerate}   
The recursive estimator represents the traditional  approach to include prior economic knowledge using  restrictions. The estimator requires a fixed number of restrictions, which are used to ensure identification and cannot be updated by the data. 
The ridge estimator represents the proposed approach to include prior knowledge. It shrinks towards the economically motivated restrictions in Equation (\ref{eq: application R}), but does not rely on them to ensure identification. As a result, the estimator can deviate from the restrictions if they are not consistent with the data.
The unrestricted estimator represents a data-driven estimation approach, which ignores any prior knowledge on the interaction.


The labeling of all estimators    is determined by the solution to the recursive  model. Specifically, I estimate the recursive model using the Cholesky decomposition, and  use the resulting estimated simultaneous interaction to construct a set of unique-sign permutation representatives centered at the recursive solution, see Section \ref{sec: Estimator}. This set restricts the admissible $B$ matrices in all estimations and determines the labeling: within the set and in line with the recursive labeling, the first shock represents an oil supply shock, the second shock is an economic activity shock, the third shock is an oil-specific demand shock, and the last shock is the stock market shock.
In addition, the weights of the ridge estimators are constructed based on the unrestricted estimator.
Furthermore, the tuning parameter $\lambda$ required for the ridge estimator  is determined in a similar fashion to the previous section, using a repeated cross-validation with two folds and $100$ repetitions. The results are shown in the appendix.
Lastly, the non-Gaussianity measured by the skewness, excess kurtosis, and Jarque-Bera test of the reduced  form and estimated structural  form shocks from all estimators are presented in the appendix. The results indicate that all shocks are left skewed with heavy tails.





\subsection{Empirical results}


Figures \ref{fig: IRF 02} and \ref{fig: IRF 20} display the estimated response of the oil price to oil supply shocks and the estimated response of oil supply to oil-specific demand shocks, respectively. 
The recursive estimator suggests that a one standard deviation  oil supply shock leads to an immediate oil price increase of around one percent, while the unrestricted and ridge estimator both find a  larger response to supply shocks.
The smaller oil price response  in the recursive SVAR can be attributed to the  response of oil supply to oil-specific demand shocks.
Specifically, in the recursive model, oil supply cannot contemporaneously respond to oil-specific demand shocks. Therefore, the correlation of oil supply and the oil price is  determined by the response of the oil price to oil-specific demand shocks.
In contrast, the ridge estimator and the unrestricted estimator both find a positive response of oil supply to oil-specific demand shocks.   Thus, the response of the oil price to a (negative) supply shock needs to increase in order to explain the correlation of oil supply and the oil price. 

\begin{figure}[h] 
	\centering
	\caption{Oil price response to a one standard deviation oil supply shock.} 
	\includegraphics[width=0.80\textwidth]{img/02.pdf}
	\label{fig: IRF 02} 
	\begin{minipage}{1\textwidth} %
		{   \footnotesize  
			\textit{Note:} Symmetric $68$\% bootstrap confidence bands based on $1000$ replications.
			\par}
	\end{minipage}
\end{figure}   

\begin{figure}[h] 
	\centering
	\caption{Oil supply response to a one standard deviation oil-specific demand shock.}
	\includegraphics[width=0.80\textwidth]{img/20.pdf}
	\label{fig: IRF 20} 
	\begin{minipage}{1\textwidth} %
		{   \footnotesize  
			\textit{Note:} Symmetric $68$\% bootstrap confidence bands based on $1000$ replications.
			\par}
	\end{minipage}
\end{figure}   









Figure  \ref{fig: IRF 32}   presents  the estimated response  of the oil price to stock market information shocks. 
All models suggest that the stock market information shock leads to an immediate increase in the stock price of around three percent and a medium-run increase in oil supply and economic activity, as documented in the  appendix. This suggests that the stock market information shock contains news about future economic activity in all models. The ridge estimator and the unrestricted estimator allow to disentangle  the simultaneous movement of residual oil and stock prices into oil-specific demand and stock market information shocks, which can affect both variables simultaneously. Both   estimators find that the oil price immediately increases in response to the stock market information shock.
In contrast, in the recursive model, the response of the oil price to stock market information shocks is restricted to zero on impact and the recursive model suggests that the information shock has no significant impact on the oil price in the medium and long run. In the recursive model, a shock that  simultaneously affects the residual oil price unexplained by supply and economic activity shocks is by construction identified as an oil-specific demand shock.  Therefore, information shocks which affect  the oil price will be contained in the oil-specific demand shock of the recursive model, while the stock market information shock in the recursive model contains a mixture of demand and information shocks with no immediate impact on the oil price.



%and Figure \ref{fig: IRF 23}  shows the estimated stock price response to oil-specific demand shocks. 
\begin{figure}[h] 
	\centering
	\caption{Oil price response to a one standard deviation stock market news shock.}
	\includegraphics[width=0.80\textwidth]{img/32.pdf}
	\label{fig: IRF 32} 
	\begin{minipage}{1\textwidth} %
		{   \footnotesize  
			\textit{Note:} Symmetric $68$\% bootstrap confidence bands based on $1000$ replications.
			\par}
	\end{minipage}
\end{figure}   

Table \ref{table: fevd} shows  the effect of the recursiveness restrictions on the forecast error variance decomposition of the oil price.
According to the recursive model,  oil-specific demand shocks are the primary driver, explaining more than $80$\% of the variation, while supply shocks   account for less than $6$\% and stock market information shocks even less. 
In contrast, the ridge estimator suggests  that oil supply and oil-specific demand both explain around $20-30$\% of the oil price variation each. Therefore, the importance of supply shocks for the oil price variation  is in line with the results in \cite{baumeister2019structural}. 
Moreover,  the ridge estimator allows to disentangle oil-specific demand and stock market information shocks which simultaneously affect stock and oil prices. Allowing both shocks to affect the oil price simultaneously results in a larger importance of stock market information shocks, which now explain around $30-40$\% of the oil price variation. 


\begin{table}[h] 
	\centering
	\caption{Forecast error variance decomposition of the real price of oil. } 
	\begin{tabular}{ c|  c c c  c c   c|  c  c c c       }
		\multicolumn{5}{c}{Recursive estimator} 
		&  $\quad$&
		\multicolumn{5}{c}{Ridge estimator}  
		\\ 
		horizon &  $\varepsilon_{S}$ & $\varepsilon_{Y}$  & $\varepsilon_{D}$   & $\varepsilon_{SM}$   
		&  $\quad$&
		horizon &  $\varepsilon_{S}$ & $\varepsilon_{Y}$  & $\varepsilon_{D}$   & $\varepsilon_{SM}$   
		\\ \cline{1-5} \cline{7-11}
		
		
		$4$& $\underset{0.02/0.08}{0.04}$ & $\underset{0.02/0.07}{0.04}$ & $\underset{0.86/0.93}{0.91}$ & $\underset{0.0/0.02}{0.01}$
		&&
		$4$&  $\underset{0.17/0.54}{0.25}$ & $\underset{0.03/0.09}{0.06}$ & $\underset{0.1/0.34}{0.27}$ & $\underset{0.18/0.54}{0.42}$
		\\ 
		$12$&  $\underset{0.03/0.11}{0.06}$ & $\underset{0.05/0.16}{0.1}$ & $\underset{0.74/0.87}{0.84}$ & $\underset{0.01/0.03}{0.0}$
		&&
		$12$&    $\underset{0.18/0.53}{0.26}$ & $\underset{0.06/0.18}{0.12}$ & $\underset{0.07/0.32}{0.24}$ & $\underset{0.15/0.49}{0.38}$
		\\ 
		$24$&$\underset{0.03/0.1}{0.04}$ & $\underset{0.05/0.15}{0.09}$ & $\underset{0.73/0.87}{0.86}$ & $\underset{0.01/0.06}{0.01}$
		&&
		$24$&  $\underset{0.14/0.51}{0.23}$ & $\underset{0.05/0.18}{0.12}$ & $\underset{0.1/0.4}{0.32}$ & $\underset{0.13/0.46}{0.34}$
		
		
	\end{tabular}     
	\label{table: fevd}
	\begin{minipage}{1\textwidth} %
		{   \footnotesize  
			\textit{Note:} The table shows the estimated contribution of each shock to the forecast error variance decomposition of the real price of oil at $4$, $12$, and $24$ month horizon together with $68$\% bootstrap confidence bands.
			\par}
	\end{minipage}
\end{table}     


Figure \ref{fig: ImpactOnrealoilprice} presents the historical decomposition of the oil price and sheds  light on the importance of supply, demand, and information shocks in different periods.  
In the recursive SVAR, the oil price   is predominately driven by oil-specific demand shocks, as seen in the oil price movements during significant events such as  the collapse of OPEC in $1985$,  the Persian Gulf War in $1990$,   the oil price run up in  $2007-2008$, the oil price decline and recovery following the collapse of Lehman Brothers in $2008$,  the oil price decline in $2014-2016$, the oil price collapse and recovery at the beginning of the COVID-$19$ pandemic, and   the recent oil price increase in $2022$ following the invasion of Ukraine.


On the other hand, the ridge estimator provides are more nuanced picture. 
Firstly, it suggests that supply shocks played a larger role  during    the collapse of OPEC in $1985$, the Persian Gulf War in $1990$, and the oil price run up in  $2007-2008$.
Secondly, the ridge estimator suggests that information shocks extracted from stock prices contributed largely to the oil price increase in  $2007-2008$, the oil price decrease following the collapse of Lehman Brothers in $2008$, the oil price decrease in  $2014-2016$, and also to the oil price decrease   and recovery at the beginning of the COVID-$19$ pandemic.


\begin{figure}[h] 
	\centering
	\caption{Historical decomposition of the real oil price.   }
	\includegraphics[width=0.99\textwidth]{img/histshocks.pdf}
	\label{fig: ImpactOnrealoilprice}
	\begin{minipage}{1\textwidth} %
		{   \footnotesize  
			\textit{Note:}  Red [blue] shows the decomposition for the recursive [ridge] estimator and grey shows the historical real oil price.
			The vertical bars indicate the following events:  
			Iran Iraq War ($1980:9$), 
			collapse of OPEC ($1985:12$), 
			Persian Gulf War ($1990:8$),  
			the collapse of  Lehman Brothers ($2008:9$), 
			the oil price  decline in mid $2014$,
			the beginning of the COVID-19 pandemic ($2019:3$),
			and the invasion of the Ukraine ($2022:3$). 
			\par}
	\end{minipage} 
\end{figure}  





Overall, the analysis suggests that the oil price    responds immediately to  information shocks extracted from stock prices and that these information shocks contributed largely to the oil price variation. Furthermore, including stock market information  non-recursively  reveals  that a considerable portion of what is typically classified as oil-specific demand shocks can actually be attributed to information shocks. This finding underscores the importance of considering stock market information when examining oil price fluctuations, as it provides valuable insights into the underlying factors driving the market.

















\section{Conclusion}
\label{sec: Conclusion}
This study demonstrates the value of prior economic knowledge in combination with statistical identification approaches for SVAR models. 
The estimator proposed in this study uses  a data-driven   approach which   only requires mean independent shocks to ensure identification and adds a  ridge  penalty to include restrictions representing uncertain prior economic knowledge.
The simulations show that incorporating uncertain prior economic knowledge can enhance the accuracy of estimates and that the estimator can detect and reduce the impact of incorrect restrictions as sample size increases. 
The application illustrates the usefulness of the proposed estimator and highlights the   role of stock market information in explaining oil price fluctuations.





\bibliographystyle{Chicago} 
\bibliography{literatur}



\appendix


\section{Appendix - Proof}
\label{Appendix Proof}

\begin{proof}[Proof of Proposition 1]
	$ $ \\
	If $B\in  \mathbb{B}$  satisfies the constraint $J(B,u)$
	\begin{align}
	\label{eq: appendix J}
	E \left[ e(B)_t e(B)'_t \right] = I,
	\end{align} 
	the objective function 
	\begin{align}
	\nonumber
	Q(B,u) :=& 
	- \sum_{i=1}^{n} \left(E [e(B)_{it}^3] \right)^2 
	- \sum_{i=1}^{n}  \left(E [e(B)_{it}^4 -3]  \right)^2 
	\\    
	&- 6  \sum_{i=1}^{n}  \sum_{i=j+1}^{n}\left(E[ e(B)_{it}^2 e(B)_{jt}^2 -1] \right)^2  
	\end{align}
	is equal to
	\begin{align} 
	Q(B,u) &= 
	\omega   \\ \nonumber 
	&+
	3\sum_{i =1 }^{n } 
	\sum_{j  =1 ;j\neq i}^{n} 
	E [e(B)_{it}^2 e(B)_{jt} ]^2 
	\\ \nonumber
	& 
	+  6
	\sum_{i =1 }^{n } 
	\sum_{j =i+1 }^{n } 
	\sum_{k =j+1 }^{n } 
	E [e(B)_{it}  e(B)_{jt} e(B)_{kt} ]^2 
	\\   \nonumber
	& 
	+ 4
	\sum_{i =1 }^{n } 
	\sum_{j  =1 ;j\neq i}^{n} 
	E [e(B)_{it}^3  e(B)_{jt} ]^2 
	\\ \nonumber
	& 
	+ 12
	\sum_{i =1 }^{n } 
	\sum_{j  =1 ;j\neq i}^{n} 
	\sum_{k = j+1;k\neq i }^{n } 
	E [e(B)_{it}^2  e(B)_{jt} e(B)_{kt}] ^2 
	\\ \nonumber
	& 
	+ 24
	\sum_{i =1 }^{n } 
	\sum_{j  =i+1}^{n} 
	\sum_{k = j+1 }^{n } 
	\sum_{l = k+1 }^{n } 
	E [e(B)_{it}   e(B)_{jt} e(B)_{kt} e(B)_{lt} ]^2 ,
	\end{align}
	where $\omega$ is a scalar invariant with respect to $B$ satisfying Equation (\ref{eq: appendix J}),  see \cite{comon1994independent}.
	
	If $B$ minimizes $	Q(B,u)$ under the constraint $J(B,u)$, the $B$ matrix satisfies the moment conditions
	\begin{align}
	\label{eq: proof cos1}
	E [e(B)_{it}^2 e(B)_{jt} ] &= 0, & \text{for } i&\neq j \\
	\label{eq: proof cos2}
	E [e(B)_{it}  e(B)_{jt} e(B)_{kt} ] &= 0, & \text{for } i&<j<k  \\
	\label{eq: proof cokurtosis}
	E [e(B)_{it}^3  e(B)_{jt} ] &= 0, & \text{for } i&\neq j \\
	E [e(B)_{it}^2  e(B)_{jt} e(B)_{kt}]  &= 0, & \text{for } i&\neq j, i \neq k, j<k \\
	\label{eq: proof cokurtosis 3}
	E [e(B)_{it}   e(B)_{jt} e(B)_{kt} e(B)_{lt} ]&=0, & \text{for } i&< j<k<l.	   	
	\end{align}
	The conditions hold if the shocks are mean independent or independent.
	
	% 	If at most one component of the structural shocks has zero excess kurtosis,  local identification follows from  \cite{lanne2021gmm} with  the constraint $J(B,u)$ and Equation (\ref{eq: proof cokurtosis}).
	
	If $B$ satisfies the constraint $J(B,u)$ and Equations (\ref{eq: proof cos1}) - (\ref{eq: proof cos2}), it holds that
	\begin{align}
	E[u_{it}  u_{jt} u_{kt}] = \sum_{l=1}^{n} b_{il} b_{jl} b_{kl} E[e(B)_{lt}^3]   & \text{, for } i,j,k = 1,...,n.\
	\end{align} 	
	Therefore, if at most one component of the structural shocks has zero skewness,   Theorem $4$ $(ii)$ in  \cite{bonhomme2009consistent} implies global identification up to sign and permutation. 
	
	Moreover, for independent shocks with at most shock exhibiting zero excess kurtosis, local identification is ensured by the asymmetric cokurtosis conditions in Equation (\ref{eq: proof cokurtosis}), see \cite{lanne2021gmm}. However, note that even though it is not stated in the proposition in \cite{lanne2021gmm}, the proof of the proposition in \cite{lanne2021gmm} uses the assumption that all cokurtosis conditions implied by independent shocks hold, including symmetric conditions of the type $E [e(B)_{it}^2  e(B)_{jt}^2 ] = 1$. Therefore, the proposition in \cite{lanne2021gmm} does not ensure identification under the weaker assumption of mean independent shocks.
\end{proof}













\section{Appendix - Application}
\label{Appendix Application} 
This section contains supplementary material  for the application including several robustness checks.


Table \ref{Table: Non-Gaussianity reduced form} - \ref{Table: Non-Gaussianity structural shocks unrestricted} show the skewness, kurtosis,   and the p-value of the Jarque-Bera test of the estimated reduced form shocks, the estimated structural shocks from the recursive estimator, the estimated structural shocks from the ridge estimator, and the estimated structural form shocks from the unrestricted estimator. The tables show that all estimated shocks are non-Gaussian, i.e. the shocks are left skewed with heavy tails.
\begin{table}[h]
	\caption{Skewness, kurtosis, and the p-value of the Jarque-Bera test of the estimated reduced form shocks}
	\label{Table: Non-Gaussianity reduced form}
	\begin{center} 
		\begin{tabular}{ c|  c c c c         }
			&$u_{q,t}$  &$u_{y,t}$  &$u_{p,t}$ & $u_{s,t}$    
			\\ \hline
			Skewness &$ -1.249  $& $ -1.596 $ &$ -0.229 $ &$ -1.183  $
			\\
			Kurtosis &    $10.108 $ & $ 19.084 $ & $ 6.6 $ & $  8.757 $
			\\
			JB-Test &  $0 $   &  $0  $   &  $0 $ & $ 0 $
		\end{tabular} 
	\end{center}
\end{table} 
\begin{table}[h]
	\caption{Skewness, kurtosis and the p-value of the Jarque-Bera test of the estimated structural shocks (recursive estimator)}
	\label{Table: Non-Gaussianity structural shocks restricted}
	\begin{center} 
		\begin{tabular}{ c|  c c c c         }
			&$\varepsilon_{S,t} $  &$\varepsilon_{Y,t} $  &$\varepsilon_{D,t}  $ & $\varepsilon_{SM,t} $    
			\\ \hline
			Skewness &$ -1.249 $& $  -1.612 $ & $- 0.38 $& $-0.966$
			\\
			Kurtosis  &$ 10.108 $ &$   19.091$& $ 5.794 $ &$  7.365 $
			\\
			JB-Test  &   $0  $   &  $0$ &  $0 $    &$  0$
		\end{tabular} 
	\end{center}
\end{table}    
\begin{table}[h]
	\caption{Skewness, kurtosis and the p-value of the Jarque-Bera test of the estimated structural shocks (ridge estimator)  }
	\label{Table: Non-Gaussianity structural shocks lambda}
	\begin{center} 
		\begin{tabular}{ c|  c c c c      }
			&$\varepsilon_{S,t} $  &$\varepsilon_{Y,t} $  &$\varepsilon_{D,t}  $ & $\varepsilon_{SM,t} $      
			\\ \hline
			Skewness &    $-1.323$  & $-1.649$ & $-0.136$ & $-1.52$
			\\
			Kurtosis  &  $9.671$ & $ 19.482$ &  $ 3.538$ & $11.976$
			\\
			JB-Test  &  $0  $   &  $0$ &  $0 $ & $ 0 $  
		\end{tabular} 
	\end{center}
\end{table}  
\begin{table}[h]
	\caption{Skewness, kurtosis and the p-value of the Jarque-Bera test of the estimated structural shocks (unrestricted estimator)}
	\label{Table: Non-Gaussianity structural shocks unrestricted}
	\begin{center} 
		\begin{tabular}{ c|  c c c c        }
			&$\varepsilon_{S,t} $  &$\varepsilon_{Y,t} $  &$\varepsilon_{D,t}  $ & $\varepsilon_{SM,t} $   
			\\ \hline
			Skewness & $-1.36$ & $-1.692$ &$-0.081$ & $-1.5$
			\\
			Kurtosis  & $10.349 $ &$ 19.763$  &  $ 3.954$ & $12.173$
			\\
			JB-Test  &   $0  $   & $ 0  $ &  $0$ &  $0 $
		\end{tabular} 
	\end{center}
\end{table}   


Figure \ref{fig: Loss CV 1}   shows  the median, $40$\%, and $60$\% quantiles of the loss in the let-out fold of the cross-validation for the ridge estimator. The tuning parameter  is chosen analogous to the cross-validation used in the Monte Carlo simulation. 
\begin{figure}[h]
	\centering
	\caption{ Median, $40$\%, and $60$\% quantiles of the loss in the let-out fold in a repeated  cross-validation with $100$ repetitions and two folds (ridge estimator)% pos 16 lam 5.74
		.}
	\includegraphics[width=0.6\textwidth]{img/cv_65576180.pdf}
	\label{fig: Loss CV 1}    
\end{figure} 



Figure \ref{fig: IRF rec}, \ref{fig: IRF ridge }, \ref{fig: IRF unrest} show the estimated impulse response for the recursive, ridge, and unrestricted estimator respectively.
%Figure \ref{fig: IRF ridge recursive } additionally shows \textcolor{red}{the impulse response function for the ridge estimator penalizing deviations from the recursive oil and stock market SVAR and thus also shrinking the response of   oil supply to oil-specific demand shocks and the response of the oil price to stock market information shocks to zero. However, the estimator does not shrink the simultaneous response of the oil price to stock market information shocks to zero.  }.
\begin{figure}[h] 
	\centering
	\caption{Impulse responses  to one standard deviation shocks (recursive estimator).}
	\includegraphics[width=0.99\textwidth]{img/estimator_WFGMMRest_irf.pdf}
	\label{fig: IRF rec}
	\begin{minipage}{1\textwidth} %
		{   \footnotesize  
			\textit{Note:}  Symmetric $68$\% bootstrap confidence bands based on $1000$ replications.
			\par}
	\end{minipage}
\end{figure}     
\begin{figure}[h] 
	\centering
	\caption{Impulse responses  to one standard deviation shocks (ridge estimator).}
	\includegraphics[width=0.99\textwidth]{img/estimator_WFGMMRidge2_irf.pdf}
	\label{fig: IRF ridge }
	\begin{minipage}{1\textwidth} %
		{   \footnotesize  
			\textit{Note:}  Symmetric $68$\% bootstrap confidence bands based on $1000$ replications.
			\par}
	\end{minipage}
\end{figure}     
\begin{figure}[h] 
	\centering
	\caption{Impulse responses  to one standard deviation shocks (unrestricted estimator).}
	\includegraphics[width=0.99\textwidth]{img/estimator_WFGMM_irf.pdf}
	\label{fig: IRF unrest}
	\begin{minipage}{1\textwidth} %
		{   \footnotesize  
			\textit{Note:}  Symmetric $68$\% bootstrap confidence bands based on $1000$ replications.
			\par}
	\end{minipage}
\end{figure}   
%\begin{figure}[h] 
%	\centering
%	\caption{Impulse responses  to one standard deviation shocks (ridge  estimator with recursive penalty).}
%	\includegraphics[width=0.99\textwidth]{img/estimator_WFGMMRidge_irf.pdf}
%	\label{fig: IRF ridge recursive }
%	\begin{minipage}{1\textwidth} %
%		{   \footnotesize  
%			\textit{Note:} Symmetric $68$\% bootstrap confidence bands based on $1000$ replications.
%			\par}
%	\end{minipage}
%\end{figure}  



Figure \ref{fig: IRF v2}  -   \ref{fig: IRF v5}  and Table \ref{table: fevd v2} - \ref{table: fevd v5} show that the main findings from Section \ref{sec: Application} are robust to various modifications of the baseline specification. Specifically, all specifications find an immediate positive response of the oil price to information shocks and information shocks explain a sizable fraction of the oil price variation.
Figure \ref{fig: IRF v2} and Table \ref{table: fevd v2} show the results for a specification where the log real stock price is replaced by monthly real stock returns.
Figure \ref{fig: IRF v3} and Table \ref{table: fevd v3} show the results for a specification with $24$ lags.
Figure \ref{fig: IRF v4} and Table \ref{table: fevd v4} show the results for a specification where log industrial production is replaced by the deviation of log industrial production from a linear time trend.
Figure \ref{fig: IRF v5} and Table \ref{table: fevd v5} show the results for a specification where the log real stock price is replaced by the economic activity index based on shipping cost that was used in \cite{kilian2009impact}. In this specification, the information shock is derived from the shipping index instead of the stock market. 
\begin{figure}[h] 
	\centering
	\caption{Oil price response to i) a one standard deviation oil supply shock (first row) and  ii) to a one standard deviation stock market information shock (second row) - Specification with monthly stock returns.} 
	\includegraphics[width=0.80\textwidth]{img/Application_OilBRidge_M0_Bcenter_v2_2_02.pdf}
	\includegraphics[width=0.80\textwidth]{img/Application_OilBRidge_M0_Bcenter_v2_2_32.pdf}
	\label{fig: IRF v2} 
	\begin{minipage}{1\textwidth} %
		{   \footnotesize  
			\textit{Note:} Symmetric $68$\% bootstrap confidence bands based on $1000$ replications.
			This specification includes  monthly real stock returns instead of log real stock prices.
			\par}
	\end{minipage}
\end{figure}   


\begin{figure}[h] 
	\centering
	\caption{Oil price response to i) a one standard deviation oil supply shock (first row) and  ii) to a one standard deviation stock market information shock (second row) - Specification with $24$ lags.} 
	\includegraphics[width=0.80\textwidth]{img/Application_OilBRidge_M0_Bcenter_v2_3_02.pdf}
	\includegraphics[width=0.80\textwidth]{img/Application_OilBRidge_M0_Bcenter_v2_3_32.pdf}
	\label{fig: IRF v3} 
	\begin{minipage}{1\textwidth} %
		{   \footnotesize  
			\textit{Note:} Symmetric $68$\% bootstrap confidence bands based on $1000$ replications.
			This specification includes $24$ lags.
			\par}
	\end{minipage}
\end{figure}   

\begin{figure}[h] 
	\centering
	\caption{Oil price response to i) a one standard deviation oil supply shock (first row) and  ii) to a one standard deviation stock market information shock (second row) - Specification with output gap.} 
	\includegraphics[width=0.80\textwidth]{img/Application_OilBRidge_M0_Bcenter_v2_4_02.pdf}
	\includegraphics[width=0.80\textwidth]{img/Application_OilBRidge_M0_Bcenter_v2_4_32.pdf}
	\label{fig: IRF v4} 
	\begin{minipage}{1\textwidth} %
		{   \footnotesize  
			\textit{Note:} Symmetric $68$\% bootstrap confidence bands based on $1000$ replications.
			This specification includes the deviation of log industrial production from a linear time trend instead of 	log industrial production.
			\par}
	\end{minipage}
\end{figure}

\begin{figure}[h] 
	\centering
	\caption{Oil price response to i) a one standard deviation oil supply shock (first row) and  ii) to a one standard deviation dry cargo shipping information shock (second row) - Specification with the dry cargo shipping index.} 
	\includegraphics[width=0.80\textwidth]{img/Application_OilBRidge_M0_Bcenter_v2_5_02.pdf}
	\includegraphics[width=0.80\textwidth]{img/Application_OilBRidge_M0_Bcenter_v2_5_32.pdf}
	\label{fig: IRF v5} 
	\begin{minipage}{1\textwidth} %
		{   \footnotesize  
			\textit{Note:} Symmetric $68$\% bootstrap confidence bands based on $1000$ replications.
			This specification includes the economic activity index based on shipping cost  used in \cite{kilian2009impact} instead of 	  log real stock prices.
			\par}
	\end{minipage}
\end{figure} 



\begin{table}[h] 
	\centering
	\caption{FEVD of the real price of oil - Specification with monthly stock returns. } 
	\begin{tabular}{ c|  c c c  c c   c|  c  c c c       }
		\multicolumn{5}{c}{Recursive estimator} 
		&  $\quad$&
		\multicolumn{5}{c}{Ridge estimator}  
		\\ 
		horizon &  $\varepsilon_{S}$ & $\varepsilon_{Y}$  & $\varepsilon_{D}$   & $\varepsilon_{SM}$   
		&  $\quad$&
		horizon &  $\varepsilon_{S}$ & $\varepsilon_{Y}$  & $\varepsilon_{D}$   & $\varepsilon_{SM}$   
		\\ \cline{1-5} \cline{7-11}
		
		
		$4$& $\underset{0.03/0.08}{0.05}$ & $\underset{0.02/0.07}{0.05}$ & $\underset{0.84/0.93}{0.89}$ & $\underset{0.0/0.02}{0.01}$
		&&
		$4$&  $\underset{0.12/0.49}{0.2}$ & $\underset{0.03/0.09}{0.06}$ & $\underset{0.08/0.34}{0.27}$ & $\underset{0.19/0.6}{0.47}$
		\\ 
		
		$12$&   $\underset{0.03/0.12}{0.07}$ & $\underset{0.06/0.16}{0.1}$ & $\underset{0.73/0.87}{0.82}$ & $\underset{0.0/0.03}{0.01}$
		&&
		$12$&   $\underset{0.13/0.49}{0.21}$ & $\underset{0.06/0.17}{0.12}$ & $\underset{0.06/0.31}{0.22}$ & $\underset{0.17/0.57}{0.45}$
		\\ 
		
		$24$& $\underset{0.03/0.11}{0.05}$ & $\underset{0.04/0.15}{0.09}$ & $\underset{0.74/0.88}{0.85}$ & $\underset{0.01/0.04}{0.01}$
		&&
		$24$&   $\underset{0.1/0.48}{0.18}$ & $\underset{0.05/0.17}{0.11}$ & $\underset{0.08/0.37}{0.27}$ & $\underset{0.16/0.56}{0.45}$
		
		
	\end{tabular}     
	\label{table: fevd v2}
	\begin{minipage}{1\textwidth} %
		{   \footnotesize  
			\textit{Note:} The table shows the estimated contribution of each shock to the forecast error variance decomposition of the real price of oil at $4$, $12$, and $24$ month horizon together with $68$\% bootstrap confidence bands.
			This specification includes  monthly real stock returns instead of log real stock prices.
			\par}
	\end{minipage}
\end{table}     

\begin{table}[h] 
	\centering
	\caption{FEVD of the real price of oil - Specification with $24$ lags.} 
	\begin{tabular}{ c|  c c c  c c   c|  c  c c c       }
		\multicolumn{5}{c}{Recursive estimator} 
		&  $\quad$&
		\multicolumn{5}{c}{Ridge estimator}  
		\\ 
		horizon &  $\varepsilon_{S}$ & $\varepsilon_{Y}$  & $\varepsilon_{D}$   & $\varepsilon_{SM}$   
		&  $\quad$&
		horizon &  $\varepsilon_{S}$ & $\varepsilon_{Y}$  & $\varepsilon_{D}$   & $\varepsilon_{SM}$   
		\\ \cline{1-5} \cline{7-11}
		
		
		$4$&  $\underset{0.02/0.08}{0.05}$ & $\underset{0.03/0.1}{0.06}$ & $\underset{0.84/0.92}{0.89}$ & $\underset{0.0/0.01}{0.0}$
		&&
		$4$& $\underset{0.22/0.63}{0.3}$ & $\underset{0.04/0.1}{0.06}$ & $\underset{0.07/0.28}{0.19}$ & $\underset{0.08/0.55}{0.45}$
		\\ 
		
		$12$&    $\underset{0.03/0.12}{0.06}$ & $\underset{0.08/0.22}{0.14}$ & $\underset{0.68/0.85}{0.8}$ & $\underset{0.0/0.03}{0.0}$
		&&
		$12$&   $\underset{0.22/0.61}{0.31}$ & $\underset{0.08/0.22}{0.14}$ & $\underset{0.05/0.25}{0.16}$ & $\underset{0.07/0.49}{0.39}$
		\\ 
		
		$24$&   $\underset{0.03/0.13}{0.05}$ & $\underset{0.08/0.24}{0.16}$ & $\underset{0.63/0.82}{0.79}$ & $\underset{0.01/0.05}{0.0}$
		&&
		$24$&   $\underset{0.2/0.59}{0.3}$ & $\underset{0.08/0.25}{0.16}$ & $\underset{0.06/0.29}{0.18}$ & $\underset{0.07/0.47}{0.36}$
		
		
	\end{tabular}     
	\label{table: fevd v3}
	\begin{minipage}{1\textwidth} %
		{   \footnotesize  
			\textit{Note:} The table shows the estimated contribution of each shock to the forecast error variance decomposition of the real price of oil at $4$, $12$, and $24$ month horizon together with $68$\% bootstrap confidence bands.
			This specification includes $24$ lags.
			\par}
	\end{minipage}
\end{table}     

\begin{table}[h] 
	\centering
	\caption{FEVD of the real price of oil - Specification with output gap.} 
	\begin{tabular}{ c|  c c c  c c   c|  c  c c c       }
		\multicolumn{5}{c}{Recursive estimator} 
		&  $\quad$&
		\multicolumn{5}{c}{Ridge estimator}  
		\\ 
		horizon &  $\varepsilon_{S}$ & $\varepsilon_{Y}$  & $\varepsilon_{D}$   & $\varepsilon_{SM}$   
		&  $\quad$&
		horizon &  $\varepsilon_{S}$ & $\varepsilon_{Y}$  & $\varepsilon_{D}$   & $\varepsilon_{SM}$   
		\\ \cline{1-5} \cline{7-11}
		
		
		$4$& $\underset{0.02/0.08}{0.05}$ & $\underset{0.02/0.07}{0.05}$ & $\underset{0.85/0.93}{0.9}$ & $\underset{0.0/0.02}{0.01}$
		&&
		$4$&  $\underset{0.19/0.52}{0.24}$ & $\underset{0.03/0.09}{0.06}$ & $\underset{0.11/0.34}{0.27}$ & $\underset{0.17/0.51}{0.42}$
		\\ 
		
		$12$&   $\underset{0.03/0.11}{0.06}$ & $\underset{0.06/0.17}{0.12}$ & $\underset{0.72/0.87}{0.82}$ & $\underset{0.01/0.03}{0.01}$
		&&
		$12$&   $\underset{0.19/0.52}{0.25}$ & $\underset{0.07/0.19}{0.14}$ & $\underset{0.08/0.31}{0.23}$ & $\underset{0.15/0.47}{0.38}$
		\\ 
		
		$24$&  $\underset{0.03/0.1}{0.04}$ & $\underset{0.06/0.2}{0.12}$ & $\underset{0.69/0.86}{0.83}$ & $\underset{0.01/0.05}{0.01}$
		&&
		$24$&   $\underset{0.15/0.48}{0.22}$ & $\underset{0.07/0.23}{0.15}$ & $\underset{0.09/0.37}{0.29}$ & $\underset{0.12/0.45}{0.34}$
		
		
	\end{tabular}     
	\label{table: fevd v4}
	\begin{minipage}{1\textwidth} %
		{   \footnotesize  
			\textit{Note:} The table shows the estimated contribution of each shock to the forecast error variance decomposition of the real price of oil at $4$, $12$, and $24$ month horizon together with $68$\% bootstrap confidence bands.
			This specification includes the deviation of log industrial production from a linear time trend instead of 	log industrial production.
			\par}
	\end{minipage}
\end{table}     

\begin{table}[h] 
	\centering
	\caption{FEVD of the real price of oil - Specification with the dry cargo shipping index.} 
	\begin{tabular}{ c|  c c c  c c   c|  c  c c c       }
		\multicolumn{5}{c}{Recursive estimator} 
		&  $\quad$&
		\multicolumn{5}{c}{Ridge estimator}  
		\\ 
		horizon &  $\varepsilon_{S}$ & $\varepsilon_{Y}$  & $\varepsilon_{D}$   & $\varepsilon_{SM}$   
		&  $\quad$&
		horizon &  $\varepsilon_{S}$ & $\varepsilon_{Y}$  & $\varepsilon_{D}$   & $\varepsilon_{SM}$   
		\\ \cline{1-5} \cline{7-11}
		
		
		$4$&  $\underset{0.02/0.07}{0.04}$ & $\underset{0.03/0.08}{0.05}$ & $\underset{0.83/0.91}{0.88}$ & $\underset{0.02/0.05}{0.03}$
		&&
		$4$&  $\underset{0.04/0.6}{0.15}$ & $\underset{0.03/0.1}{0.07}$ & $\underset{0.21/0.63}{0.58}$ & $\underset{0.04/0.3}{0.2}$
		\\ 
		
		$12$&    $\underset{0.03/0.11}{0.06}$ & $\underset{0.06/0.17}{0.11}$ & $\underset{0.68/0.82}{0.78}$ & $\underset{0.03/0.1}{0.06}$
		&&
		$12$&  $\underset{0.06/0.58}{0.16}$ & $\underset{0.07/0.19}{0.14}$ & $\underset{0.15/0.52}{0.46}$ & $\underset{0.06/0.35}{0.25}$
		\\ 
		
		$24$& $\underset{0.03/0.1}{0.04}$ & $\underset{0.05/0.15}{0.09}$ & $\underset{0.64/0.81}{0.78}$ & $\underset{0.04/0.18}{0.09}$
		&&
		$24$&   $\underset{0.05/0.53}{0.12}$ & $\underset{0.05/0.18}{0.12}$ & $\underset{0.15/0.51}{0.45}$ & $\underset{0.08/0.43}{0.31}$
		
		
	\end{tabular}     
	\label{table: fevd v5}
	\begin{minipage}{1\textwidth} %
		{   \footnotesize  
			\textit{Note:} The table shows the estimated contribution of each shock to the forecast error variance decomposition of the real price of oil at $4$, $12$, and $24$ month horizon together with $68$\% bootstrap confidence bands.
			This specification includes the economic activity index based on shipping cost  used in \cite{kilian2009impact} instead of 	  log real stock prices.
			\par}
	\end{minipage}
\end{table}     



\end{document}
