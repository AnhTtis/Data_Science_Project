\usepackage{xifthen}
\usepackage{xspace}

%\usepackage[style=authoryear-comp, natbib=true, backend=biber, doi=false, isbn=false, maxcitenames=2, uniquelist=false, dashed=false, firstinits=true]{biblatex}
\usepackage{natbib}

% Komma bei citet, Semikolon bei citep
\usepackage{etoolbox}
\makeatletter
\newcommand\bibstyle@comma{\bibpunct(),a,,}
\newcommand\bibstyle@semicolon{\bibpunct();a,,}
\makeatother
\pretocmd\citet{\citestyle{comma}}\relax\relax
\pretocmd\Citet{\citestyle{comma}}\relax\relax
\pretocmd\citep{\citestyle{semicolon}}\relax\relax
\pretocmd\Citep{\citestyle{semicolon}}\relax\relax

\bibliographystyle{apa}

\newtheorem{mythm}{Theorem}
\numberwithin{mythm}{section}
\newtheorem{mylem}[mythm]{Lemma}
\newtheorem{myprop}[mythm]{Proposition}
\newtheorem{mykor}[mythm]{Corollary}
\theoremstyle{definition}
\newtheorem{mydef}[mythm]{Definition}
\newtheorem{mybsp}[mythm]{Example}
\newtheorem{myann}[mythm]{Assumption}
\newtheorem{mynot}[mythm]{Notation}
\theoremstyle{remark}
\newtheorem{mybem}[mythm]{Remark}
\newtheorem{myerin}[mythm]{Recall}
\newtheorem{myfolg}[mythm]{Conclusion}

\numberwithin{figure}{section}

\crefname{mythm}{theorem}{theorems}
\crefname{mylem}{lemma}{lemmas}
\crefname{myprop}{proposition}{propositions}


\newcommand{\iid}{\text{i.i.d.}}
\newcommand{\ie}{\enm{\text{i.e.\@}}}
\newcommand{\vgl}{\enm{\text{cf.\@}}}
\newcommand{\seite}{\enm{\text{p.\@}}}
\newcommand{\seiten}{\enm{\text{pp.\@}}}
\newcommand{\eg}{\enm{\text{e.g.\@}}}
\newcommand{\wenn}{\text{, if }}
\newcommand{\sonst}{\text{, else}}
\newcommand{\fs}{\enm{\text{a.s.\@}}}
\newcommand{\enm}[1]{\ensuremath{#1}\xspace}
\newcommand{\limn}{\enm{\lim_{n\to\infty}}}
\newcommand{\diff}{\enm{\,\mathrm{d}}}
\newcommand{\R}{\enm{\mathbb{R}}}
\newcommand{\N}{\enm{\mathbb{N}}}
\newcommand{\eps}{\enm{\varepsilon}}
\newcommand{\lb}{\enm{\lambda}}
\newcommand{\lbn}{\enm{\lb_n}}
\renewcommand{\P}{\enm{\textnormal{P}}}
\newcommand{\Px}{\enm{\P_X}}
\newcommand{\Pbed}[2][\cdot]{\enm{\P(#1\,|\,#2)}} 
\newcommand{\Q}{\enm{\textnormal{Q}}}
\newcommand{\D}{\enm{D}} % Datensatz
\newcommand{\DVert}{\enm{\textnormal{D}}}
\newcommand{\Dn}{\enm{\D_n}} 
\newcommand{\DVertn}{\enm{\DVert_n}}
\newcommand{\ew}[2][]{\enm{\mathbb{E}_{#1}\left[#2\right]}}
\newcommand{\X}{\enm{\mathcal{X}}}
\newcommand{\Y}{\enm{\mathcal{Y}}}
\newcommand{\XX}{\enm{\X\times\X}}
\newcommand{\XY}{\enm{\X\times\Y}}
\newcommand{\XYR}{\enm{\X\times\Y\times\R}}
\newcommand{\YR}{\enm{\Y\times\R}}
\newcommand{\MXY}{\enm{\mathcal{M}_1(\XY)}}
\newcommand{\BX}{\enm{\mathcal{B}_{\X}}}
\newcommand{\BY}{\enm{\mathcal{B}_{\Y}}}
\newcommand{\BXY}{\enm{\mathcal{B}_{\XY}}}
\renewcommand{\L}[2][]{\enm{L_{#2}\ifthenelse{\isempty{#1}}{}{(#1)}}}
\newcommand{\Lp}{\L{p}}
\newcommand{\La}{\L{1}}
\newcommand{\Lppx}{\enm{\Lp(\Px)}} % Lp(P^X)
\newcommand{\Lapx}{\enm{\La(\Px)}} % L1(P^X)
\newcommand{\sobolevX}[1][m,q]{\enm{W^{#1}(\X)}}
\newcommand{\norm}[2]{\enm{\left|\left|#2\right|\right|_{#1}}}
\newcommand{\normSup}[1]{\norm{\infty}{#1}}
\newcommand{\normH}[1]{\norm{\H}{#1}}
\newcommand{\normLppx}[1]{\norm{\Lppx}{#1}} % Lp(P^X)
\newcommand{\normLapx}[1]{\norm{\Lapx}{#1}} % L1(P^X)
\renewcommand{\H}{\enm{H}} 
\renewcommand{\k}{\enm{k}}
\newcommand{\loss}{\enm{L}}
\newcommand{\lossshift}{\enm{\loss^\star}}
\newcommand{\losspin}{\enm{\loss_{\tau\text{-pin}}}}
\newcommand{\losspinshift}{\enm{\losspin^\star}}
\newcommand{\risk}[1][\loss,\P]{\enm{\mathcal{R}_{#1}}} 
\newcommand{\riskempn}[1][\loss,\DVertn]{\enm{\mathcal{R}_{#1}}} % R_LDn
\newcommand{\riskshift}[1][\lossshift,\P]{\enm{\mathcal{R}_{#1}}} % R_L*P
\newcommand{\riskbayes}[1][\loss,\P]{\enm{\mathcal{R}_{#1}^*}}
\newcommand{\riskshiftbayes}[1][\lossshift,\P]{\enm{\mathcal{R}_{#1}^*}} % R_L*P^*
\newcommand{\riskoptH}[1][\loss,\P,\H]{\enm{\mathcal{R}_{#1}^*}} % R_LPH^*
\newcommand{\innerrisk}[1][\loss,\Q]{\enm{\mathcal{C}_{#1}}} % C_LQ, Indizes anpassbar
\newcommand{\innerriskshiftbed}[1][x]{\innerrisk[\lossshift,\Pbed{#1}]} % Inneres Risiko bzgl. L* und P(.|x)
\newcommand{\fbayes}{\enm{f_{\loss,\P}^*}}
\newcommand{\fshiftbayes}{\enm{f_{\lossshift,\P}^*}}
\newcommand{\ftaubayes}{\enm{f_{\tau,\P}^*}} % f_tP^*
\newcommand{\ftaubayeswithshiftloss}{\enm{f_{\losspinshift,\P}^*}} % f_Lt*P^*
\newcommand{\fn}{\enm{f_n}}
\newcommand{\fstar}{\enm{f^*}}
\newcommand{\ftilde}[1][]{\ifthenelse{\isempty{#1}}{\enm{\tilde{f}}}{\enm{\tilde{f}_{#1}}}}
\newcommand{\ftheo}{\enm{f_{\loss,\P,\lb}}} % Standard
\newcommand{\ftheon}{\enm{f_{\loss,\P,\lbn}}} % lambda_n statt lambda
\newcommand{\fempn}{\enm{f_{\loss,\DVert_n,\lb}}} % Dn statt D
\newcommand{\fempnn}{\enm{f_{\loss,\DVert_n,\lbn}}} % Dn statt D und lambda_n statt lambda
\newcommand{\fshifttheo}{\enm{f_{\lossshift,\P,\lb}}} % Shifted loss
\newcommand{\ftaushiftempn}{\enm{f_{\losspinshift,\DVertn,\lbn}}} % Emp. SVM bzgl. Shifted Pinball-Loss mit Dn
\newcommand{\unif}[1][\enm{0,1}]{\enm{\mathcal{U}(#1)}}
\newcommand{\dirac}[1][x]{\enm{\delta_{#1}}}
\newcommand{\bigcupdot}{\charfusion[\mathop]{\bigcup}{\cdot}}
\makeatletter
\def\moverlay{\mathpalette\mov@rlay}
\def\mov@rlay#1#2{\leavevmode\vtop{%
		\baselineskip\z@skip \lineskiplimit-\maxdimen
		\ialign{\hfil$\m@th#1##$\hfil\cr#2\crcr}}}
\newcommand{\charfusion}[3][\mathord]{
	#1{\ifx#1\mathop\vphantom{#2}\fi
		\mathpalette\mov@rlay{#2\cr#3}
	}
	\ifx#1\mathop\expandafter\displaylimits\fi}
\makeatother
