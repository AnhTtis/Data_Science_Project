\appendix

%%%%%%%%%%%%%%%%%%%
% Re-count the Figure/Algorithm/Tables after this point. 
%%%%%%%%%%%%%%%%%%%
\counterwithin{figure}{section}
\counterwithin{table}{section}


\section{Detailed Setting Used in~\sref{sec:evaluation_of_existing_methods}}
\label{sec:hyperparameter_used_in_exisiting_method_evaluation}
Same as~\cite{frantar2022gptq}, for all methods, we use C4 dataset to randomly select 128 sentences for training and each of them has 2048 tokens.

For \gptq, we check its main hyperparameter, i.e., the dampening factor, and find out the method is not sensitive to it.
As such, we use the hyparameter suggested by the author for all of our experiments. 

For \zqglobal and \zqlocal, as mentioned the in main text, the hyperparameters suggested by the original work~\cite{yao2022zeroquant} is suboptimal. 
We find that a linear decay learning rate schedule is very helpful in our initial test.
As such, we add this as our default setting.
Meanwhile, we extensively test a wide range (1e-3 to 5e-8) of learning rate for different models until we find the best learning rate (i.e., larger or smaller learning rate leads to worse accuracy performance).
We use Adam optimizer and a default batch size 1.

For all three methods, we run them on a single GPU (either V100-32GB or A100-80GB). 
For the largest model tested in the paper, i.e., \bloom-176B, the cost of all methods is lower than 1 GPU-day on A100-80G.

\section{Best \ptq Methods with Per-row Quantization}
Table~\ref{tab:opt-quantization-method} and~\ref{tab:bloom-quantization-method} summarize the best \ptq methods with per-row optimization.

\begin{table}[t]
\caption{
Best optimization method of \opt family in~\sref{sec:evaluation_of_existing_methods}.
}\centering
\label{tab:opt-quantization-method}
\begin{adjustbox}{width=0.9\linewidth}
\centering
\begin{tabular}{lcccccccccccccc }
\toprule
Precision     & 125m	& 350m 	& 1.3b	& 2.7b	& 6.7b	& 13b &30b	& 66b \\
\midrule
Weight Only (INT4) &\zqglobal &\zqglobal &\gptq &\gptq &\gptq &\gptq &\gptq &\gptq\\
\midrule
Weight \& Activation (W4A8) &\zqglobal &\zqglobal &\zqglobal &\gptq &\zqglobal &\zqglobal &\zqglobal &\zqlocal\\
\bottomrule
\end{tabular}
\end{adjustbox}
\end{table}

\begin{table}[t]
\caption{
Best optimization method of \bloom family in~\sref{sec:evaluation_of_existing_methods}.
}\centering
\label{tab:bloom-quantization-method}
\begin{adjustbox}{width=0.9\linewidth}
\centering
\begin{tabular}{lcccccccccccccc}
\toprule
Precision     & 560m   &1.1b   & 1.7b  & 3b & 7.1b & 176b \\
\midrule
Weight Only (INT4) &\gptq &\zqglobal &\zqglobal &\zqglobal/\gptq &\gptq &\gptq\\
\midrule
Weight \& Activation (W4A8) &\zqglobal &\zqglobal &\zqglobal &\zqglobal &\zqglobal &\zqlocal \\
\bottomrule
\end{tabular}
\end{adjustbox}
\end{table}


\section{3-bit Weight-only Quantization}
\label{sec:3bit_weightonly_quantization}
\begin{table}[t]
\caption{
Results of W3\asym-A16 quantization on \opt with various block-size out of the best result from optimization-based methods. 
See~\tref{tab:opt-3bit-blocksize-full} for full results including \rtn.
N/A means that the block size is not divisible by the hidden size.
}\centering
\label{tab:opt-3bit-blocksize}
\begin{adjustbox}{width=0.9\linewidth}
\centering
\begin{tabular}{lcccccccccccccc }
\toprule
Block-size     & 125m	& 350m 	& 1.3b	& 2.7b	& 6.7b	& 13b &30b	& 66b \\
\midrule 
W16-A16 &28.27 &22.93 &15.44 &13.58 &11.90 &11.22 &10.70 &10.33\\

\midrule
Per-row &46.82 &30.30 &22.36 &17.06 &14.18 &12.43 &11.28 &17.77 \\
1024 &N/A &29.62 &20.16 &N/A &12.90 &11.74 &11.03 &12.95 \\
512 &N/A &28.65 &18.94 &15.47 &12.82 &11.67 &10.97 &12.33 \\
256 &38.85 &27.92 &17.95 &15.10 &12.79 &11.63 &10.90 &11.34 \\
128 &36.80 &26.97 &17.61 &15.05 &12.69 &11.59 &10.91 &11.27 \\
64 &35.48 &26.76 &17.40 &14.85 &12.58 &11.62 &10.92 &10.97\\
32 &33.75 &26.38 &17.11 &14.73 &12.64 &11.70 &10.99 &10.95\\
\bottomrule
\end{tabular}
\end{adjustbox}
\end{table}

\begin{table}[t]
\caption{
Results of W3\asym-A16 quantization on \bloom with various block-size out of the best result from optimization-based methods. 
See~\tref{tab:bloom-3bit-blocksize-full} for full results including \rtn.
N/A means that the block size is not divisible by the hidden size.
}\centering
\label{tab:bloom-3bit-blocksize}
\begin{adjustbox}{width=0.9\linewidth}
\centering
\begin{tabular}{lcccccccccccccc }
\toprule
Block-size     & 560m   &1.1b   & 1.7b  & 3b & 7.1b & 176b \\
\midrule 
W16-A16  &29.35 &28.32 &20.43 &17.58 &14.96 &10.90 \\
\midrule
Per-row &43.37 &54.48 &25.59 &24.10 &271.31 &49.46\\
1024    &38.10 &N/A   &24.24 &N/A   &16.68 &11.15\\
512     &35.20 &33.75 &23.58 &19.58 &16.21 &11.15\\
256     &34.43 &32.46 &23.08 &19.31 &16.15 &11.13\\
128     &33.49 &31.95 &22.62 &18.98 &15.96 &11.10\\
64      &33.26 &31.51 &22.41 &18.91 &15.86 &11.10\\
32      &32.93 &31.34 &22.15 &18.95 &15.85 &11.12\\
\bottomrule
\end{tabular}
\end{adjustbox}
\end{table}

We report W3A16 results of \opt and \bloom in~\ref{tab:opt-3bit-blocksize} and \ref{tab:bloom-3bit-blocksize} with various quantization block sizes, respectively.
Similar to 4-bit quantization, smaller block size brings better accuracy. 
However, none of the models can achieve \classone quantization error, and more importantly, 3-bit with block size 32, which has similar actually bits as 4-bit per-row quantization (since block size 32 has one FP16 scaling factor and one FP16 zeropoint), has worse performance than 4-bit per-row quantization, which demonstrates that fine-grained quantization might be able to close the gap from the reduction of bits.


\section{Full results of Our Evaluation}
\label{sec:full_tables_results}
We put the full results of our evaluations in this section.
% \subsection{Full results of \sref{sec:ptq_challenge}}
\begin{table}[t]
\caption{
\opt ppl on wikitext/ptb/c4 (full results of~\tref{tab:ptq_challenge_opt_average_in_maintext}).
}\centering
\label{tab:ptq_challenge_opt_full_in_appendix}
\begin{adjustbox}{width=0.9\linewidth}
\centering
\begin{tabular}{lcccccccccccccc }
\toprule
Precision     & 125m	& 350m 	& 1.3b	& 2.7b	& 6.7b	& 13b &30b	& 66b \\
\midrule
W16-A16 &27.65/32.55/24.61 &22.00/26.08/20.71 &14.62/16.97/14.72 &12.47/15.11/13.17 &10.86/13.09/11.74 &10.13/12.34/11.20 &9.56/11.84/10.69 &9.34/11.36/10.28\\
\midrule
W8A8\sym-A16 &27.64/32.53/24.65 &22.06/26.10/20.72 &14.63/16.98/14.73 &12.48/15.13/13.17 &10.85/13.11/11.75 &10.12/12.34/11.20 &9.55/11.85/10.70 &9.34/11.36/10.29 \\
W8\asym-A16  &27.71/32.58/24.64 &22.04/26.12/20.73 &14.67/16.99/14.73 &12.50/15.14/13.17 &10.86/13.11/11.75 &10.11/12.34/11.20 &9.55/11.84/10.69 &9.35/11.36/10.29\\
W4\sym-A16 &45.89/53.68/36.68 &25.95/31.11/23.94 &19.85/23.61/18.90 &22.86/30.01/22.29 &12.41/17.05/13.62 &11.06/14.90/12.23 &10.18/13.26/11.86 &57.73/134.91/98.51 \\
W4\asym-A16  &36.71/44.76/30.92 &25.51/30.90/23.86 &19.38/21.95/17.93 &17.92/22.48/18.32 &11.91/15.39/13.01 &10.67/13.53/12.07 &10.10/13.13/11.33 &20.24/48.45/25.86 \\
\midrule
W16-A8\sym &27.96/32.57/24.69 &22.06/26.42/20.95 &15.21/18.18/15.81 &12.98/16.01/13.89 &20.99/25.94/31.18 &3341.50/2618.38/3554.59 &1681.48/2221.62/2241.53 &2696.91/2647.41/2569.94 \\
W16-A8\asym &27.84/32.60/24.66 &22.04/26.22/20.81 &15.14/17.65/15.39 &12.51/15.38/13.38 &11.24/14.17/12.45 &11.83/18.87/15.39 &14.08/31.54/25.09 &442.66/524.57/716.83  \\
\bottomrule
\end{tabular}
\end{adjustbox}
\end{table}

\begin{table}[t]
\caption{
\bloom ppl on wikitext/ptb/c4 (full results of~\tref{tab:ptq_challenge_opt_average_in_maintext}).
}\centering
\label{tab:ptq_challenge_bloom_full_in_appendix}
\begin{adjustbox}{width=0.9\linewidth}
\centering
\begin{tabular}{lcccccccccccccc }
\toprule
Precision     & 560m   &1.1b   & 1.7b  & 3b & 7.1b & 176b \\
\midrule
W16-A16  &22.43/41.25/24.38 &17.69/46.98/20.29 &15.39/27.93/17.97 &13.48/23.12/16.14 &11.37/19.40/14.13  &8.11/13.62/10.97 \\
\midrule
W8\sym-A16 &22.44/41.28/24.39 &17.70/47.01/20.29 &15.40/27.91/17.98 &13.49/23.14/16.14 &11.37/19.40/14.13 &8.11/13.63/10.98 \\
W8\asym-A16 &22.43/41.24/24.40 &17.69/47.00/20.29 &15.40/27.96/17.97 &13.48/23.14/16.14 &11.37/19.40/14.13 &8.10/13.62/10.98 \\
W4\sym-A16 &26.49/49.73/27.98 &20.27/56.64/22.81 &17.47/32.20/19.88 &14.96/25.59/17.51 &12.38/21.36/15.06 &8.40/14.15/11.30 \\
W4\asym-A16 &25.31/46.79/27.10 &23.90/68.31/25.99 &16.93/31.02/19.47 &14.65/25.12/17.26 &12.06/20.83/14.83 &8.34/14.03/11.23 \\
\midrule
W16-A8\sym &22.50/41.58/24.46 &17.78/47.28/20.38 &15.57/28.36/18.13 &13.57/23.38/16.25 &11.58/19.92/14.35 &8.75/14.94/12.61 \\
W16-A8\asym &22.45/41.37/24.42 &17.71/47.05/20.32 &15.45/28.09/18.02 &13.52/23.24/16.19 &11.47/19.71/14.25 &8.41/14.52/11.93 \\
\bottomrule
\end{tabular}
\end{adjustbox}
\end{table}


% \subsection{Full Results of~\sref{sec:weight_only_quantization_existing_method}}
\begin{table}[t]
\caption{
\opt ppl on wikitext/opt/c4 with W4\asym-A16 (full table of~\tref{tab:weight_only_quantization_opt_existing_method_average_in_main_text}).
See~\tref{tab:weight_only_quantization_opt_existing_method_full_zqlocal} for all learning rate results of \zqlocal and~\tref{tab:weight_only_quantization_opt_existing_method_full_zqglobal} of \zqglobal.
}\centering
\label{tab:weight_only_quantization_opt_existing_method_full_in_appendix}
\begin{adjustbox}{width=0.9\linewidth}
\centering
\begin{tabular}{lcccccccccccccc }
\toprule
Precision     & 125m	& 350m 	& 1.3b	& 2.7b	& 6.7b	& 13b &30b	& 66b \\
\midrule
\rtn &36.71/44.76/30.92 &25.51/30.90/23.86 &19.38/21.95/17.93 &17.92/22.48/18.32 &11.91/15.39/13.01 &10.67/13.53/12.07 &10.10/13.13/11.33 &20.24/48.45/25.86 \\
\gptq &32.52/40.25/27.78 &23.50/29.14/22.41 &15.52/18.16/15.56 &13.02/15.84/13.73 &11.16/13.59/12.08 &10.29/12.61/11.35 &9.61/11.95/10.79 &9.54/11.67/10.52\\
\zqlocalstar &33.05/39.34/28.11 &24.40/29.22/22.82 &15.81/18.66/15.76 &13.22/16.19/13.96 &11.32/13.79/12.26 &10.42/12.90/11.60 &9.97/12.32/11.03 &9.91/11.87/10.59 \\
\zqglobalstar &31.44/36.66/27.21 &23.32/28.05/21.98 &15.46/18.31/15.67 &13.03/16.04/13.83 &11.30/13.69/12.17 &10.38/12.85/11.62 &9.90/12.24/10.99 &9.62/11.81/10.61 \\
\bottomrule
\end{tabular}
\end{adjustbox}
\end{table}

\begin{table}[t]
\caption{
\opt ppl on wikitext/opt/c4 with W4\asym-A16 and \zqlocal. 
}\centering
\label{tab:weight_only_quantization_opt_existing_method_full_zqlocal}
\begin{adjustbox}{width=0.9\linewidth}
\centering
\begin{tabular}{lcccccccccccccc }
\toprule
LR (W4\asym-A16)     & 125m	& 350m 	& 1.3b	& 2.7b	& 6.7b	& 13b &30b	& 66b \\
\midrule
0.001 &33.67/39.45/29.11 &26.33/31.94/24.49 &16.27/19.91/16.46 &14.34/17.76/14.93 &11.87/15.04/13.06 &13.68/18.89/14.46 &171.35/151.55/46.14 &814.22/601.74/308.53  \\
0.0005 &32.76/39.51/28.64 &25.88/30.95/23.96 &16.29/19.82/16.27 &14.16/17.65/14.79 &11.92/15.23/12.95 &10.93/13.82/12.03 &10.23/13.46/11.44 &10.10/12.27/10.81  \\
0.0001 &33.86/40.01/28.29 &24.64/30.26/23.33 &16.07/19.25/15.93 &14.36/17.38/14.41 &11.85/14.64/12.74 &10.93/13.48/11.88 &10.18/12.67/11.13 &10.12/12.01/10.67  \\
5e-05 &33.05/39.34/28.11 &25.42/29.65/23.22 &15.79/19.16/15.88 &13.70/16.80/14.16 &11.71/14.32/12.41 &10.75/13.38/11.77 &9.95/12.54/11.09 &10.02/11.89/10.64  \\
1e-05 &33.78/40.41/28.84 &24.40/29.22/22.82 &15.81/18.66/15.76 &13.55/16.46/13.96 &11.32/13.79/12.26 &10.54/13.05/11.61 &9.98/12.22/10.99 &9.91/11.87/10.59  \\
5e-06 &34.47/41.04/29.02 &24.50/29.27/23.00 &16.01/18.73/15.91 &13.22/16.19/13.96 &11.33/13.86/12.29 &10.42/12.90/11.60 &9.86/12.33/10.97 &9.97/11.86/10.60  \\
1e-06 &35.88/43.69/30.35 &24.54/29.87/23.17 &16.77/19.45/16.47 &13.60/17.02/14.46 &11.41/14.10/12.41 &10.53/13.01/11.70 &9.97/12.33/11.04 &10.01/11.93/10.66  \\
\bottomrule
\end{tabular}
\end{adjustbox}
\end{table}


\begin{table}[t]
\caption{
\opt ppl on wikitext/opt/c4 with W4\asym-A16 and \zqglobal. 
NaN here means the PPL is larger than 1e6.
}\centering
\label{tab:weight_only_quantization_opt_existing_method_full_zqglobal}
\begin{adjustbox}{width=0.9\linewidth}
\centering
\begin{tabular}{lcccccccccccccc }
\toprule
LR (W4\asym-A16)     & 125m	& 350m 	& 1.3b	& 2.7b	& 6.7b	& 13b &30b	& 66b \\
\midrule
0.001 &4057.13/2718.91/1247.78 &5071.35/5229.93/687.35 &12105.25/10154.73/7893.43 &18965.76/17112.60/16316.31 &60014.66/56041.86/78085.84 &232421.09/98305.32/119762.73 &93917.09/70170.34/51124.06 & NaN  \\
0.0005 &31.94/38.61/27.17 &27.11/33.91/24.07 &10900.84/8322.65/8425.10 &14412.30/8676.76/10154.55 &18527.46/13530.12/13029.95 &109006.53/62584.41/125349.50 &303235.75/230599.62/430480.03 &36439.32/30554.19/33756.93  \\
0.0001 &31.44/36.66/27.21 &24.08/29.08/22.27 &15.91/20.08/16.35 &118.38/53.47/54.08 &7604.92/5339.10/5161.49 &12638.86/7639.95/8243.63 &16276.68/9890.26/6176.27 &8367.31/4728.13/5533.59  \\
5e-05 &31.97/36.93/27.12 &23.55/28.06/22.02 &15.82/18.65/15.65 &13.40/16.44/13.97 &26.54/25.67/17.60 &909.99/316.82/370.84 &6238.21/3291.04/3743.01 &9296.98/6687.44/5363.29  \\
1e-05 &32.31/37.93/27.38 &23.32/28.05/21.98 &15.60/18.42/15.64 &13.09/16.05/13.78 &11.41/13.82/12.20 &10.80/13.16/11.66 &10.06/12.44/11.07 &9.73/12.09/10.98  \\
5e-06 &32.69/38.91/27.76 &23.26/28.33/22.05 &15.46/18.31/15.67 &13.03/16.04/13.83 &11.30/13.69/12.17 &10.50/12.89/11.58 &9.95/12.28/11.01 &9.62/11.81/10.61  \\
1e-06 &34.63/41.75/29.43 &23.82/28.96/22.48 &16.12/19.46/16.27 &13.03/16.27/14.04 &11.29/13.88/12.27 &10.38/12.85/11.62 &9.90/12.24/10.99 &9.58/12.17/10.78  \\
5e-07 & NaN & NaN & NaN & NaN & NaN &10.51/12.96/11.70 &9.89/12.41/11.04 &9.90/12.45/11.00  \\
1e-07 & NaN & NaN & NaN & NaN & NaN &10.63/13.29/11.89 &10.02/12.82/11.18 &11.03/13.91/11.73  \\
5e-08 & NaN & NaN & NaN & NaN & NaN &10.66/13.42/11.97 &10.05/13.00/11.24 &12.41/17.45/13.02  \\
\bottomrule
\end{tabular}
\end{adjustbox}
\end{table}


\begin{table}[t]
\caption{
\bloom ppl on wikitext/opt/c4 with W4\asym-A16 (full table of~\tref{tab:weight_only_quantization_opt_existing_method_average_in_main_text}).
See~\tref{tab:weight_only_quantization_opt_existing_method_full_zqlocal} for all learning rate results of \zqlocal and~\tref{tab:weight_only_quantization_opt_existing_method_full_zqglobal} of \zqglobal.}\centering
\label{tab:weight_only_quantization_bloom_existing_method_full_in_appendix}
\begin{adjustbox}{width=0.9\linewidth}
\centering
\begin{tabular}{lcccccccccccccc}
\toprule
Precision     & 560m   &1.1b   & 1.7b  & 3b & 7.1b & 176b \\
\midrule
\rtn &25.31/46.79/27.10 &23.90/68.31/25.99 &16.93/31.02/19.47 &14.65/25.12/17.26 &12.06/20.83/14.83 &8.34/14.03/11.23 \\
\gptq &23.90/43.76/25.59 &24.34/68.10/26.58 &16.36/29.58/18.79 &14.10/24.23/16.66 &11.80/20.23/14.47 &8.22/13.78/11.07 \\
\zqlocalstar &24.23/44.94/26.05 &19.22/52.36/21.59 &16.37/29.89/18.86 &14.23/24.41/16.86 &11.80/20.28/14.56 &8.27/13.91/11.16 \\
\zqglobalstar &23.84/44.17/25.60 &19.50/51.33/21.72 &16.19/29.28/18.66 &14.14/24.16/16.69 &11.77/20.27/14.52 &8.24/13.82/11.10 \\
\bottomrule
\end{tabular}
\end{adjustbox}
\end{table}

\begin{table}[t]
\caption{
\bloom ppl on wikitext/opt/c4 with W4\asym-A16 and \zqlocal. 
}\centering
\label{tab:weight_only_quantization_bloom_existing_method_full_zqlocal}
\begin{adjustbox}{width=0.9\linewidth}
\centering
\begin{tabular}{lcccccccccccccc}
\toprule
LR (W4\asym-A16)     & 560m   &1.1b   & 1.7b  & 3b & 7.1b & 176b \\
\midrule
0.001 &25.37/47.36/27.03 &19.89/53.86/22.11 &16.70/31.19/19.30 &14.45/25.28/17.16 &12.22/21.34/15.04 &8.82/15.77/11.98  \\
0.0005 &25.17/46.83/26.87 &19.57/53.66/21.92 &16.58/30.27/19.15 &14.43/25.47/17.07 &11.94/20.54/14.67 &8.35/14.01/11.20  \\
0.0001 &24.59/46.11/26.32 &19.22/52.36/21.59 &16.41/30.29/18.90 &14.35/24.81/16.87 &11.83/20.34/14.58 &8.28/13.92/11.14  \\
5e-05 &24.44/46.04/26.16 &23.28/65.68/25.42 &16.39/30.01/18.86 &14.34/24.43/16.83 &11.80/20.28/14.56 &8.27/13.93/11.15  \\
1e-05 &24.23/44.94/26.05 &23.45/66.29/25.52 &16.37/29.89/18.86 &14.23/24.41/16.86 &11.84/20.39/14.58 &8.27/13.91/11.16  \\
5e-06 &24.21/45.21/26.10 &23.26/65.72/25.42 &16.42/30.09/18.94 &14.25/24.55/16.87 &11.87/20.50/14.61 &8.29/13.98/11.16  \\
1e-06 &24.71/45.86/26.50 &23.45/66.28/25.56 &16.64/30.52/19.15 &14.46/24.76/17.04 &11.94/20.55/14.70 &8.29/13.97/11.18  \\
\bottomrule
\end{tabular}
\end{adjustbox}
\end{table}

\begin{table}[t]
\caption{
\bloom ppl on wikitext/opt/c4 with W4\asym-A16 and \zqglobal. 
}\centering
\label{tab:weight_only_quantization_bloom_existing_method_full_zqglobal}
\begin{adjustbox}{width=0.9\linewidth}
\centering
\begin{tabular}{lcccccccccccccc}
\toprule
LR (W4\asym-A16)     & 560m   &1.1b   & 1.7b  & 3b & 7.1b & 176b \\
\midrule
0.001 &6853935.00/30441738.00/3222857.25 &528072.88/828428.62/356031.97 &597410.50/973155.88/1280478.12 &878460.69/2175974.25/441401.94 &nan/nan/nan & NaN  \\
0.0005 &29671.52/1795030.88/4653.35 &28112.96/87515.64/1826.82 &141110.14/204295.86/40146.11 &265457.25/741326.38/99882.45 &944784.19/774538.25/395960.03 & NaN  \\
0.0001 &23.92/45.68/25.72 &19.34/52.78/21.63 &16.35/29.22/18.76 &14.27/24.46/16.80 &12.17/22.16/14.80 & NaN  \\
5e-05 &23.84/44.17/25.60 &19.50/51.33/21.72 &16.19/29.28/18.66 &14.14/24.16/16.69 &11.81/20.41/14.50 & NaN  \\
1e-05 &23.85/44.20/25.65 &22.64/56.79/23.41 &16.23/29.73/18.73 &14.14/24.31/16.74 &11.77/20.27/14.52 &8.24/13.82/11.10  \\
5e-06 &24.02/44.62/25.79 &23.46/63.27/24.88 &16.28/29.83/18.81 &14.19/24.38/16.80 &11.77/20.33/14.54 &8.24/13.82/11.10  \\
1e-06 &24.46/45.41/26.20 &24.62/70.16/26.64 &16.48/30.15/19.02 &14.35/24.56/16.95 &11.89/20.54/14.67 &8.23/13.82/11.12  \\
5e-07 & NaN & NaN & NaN & NaN & NaN &8.26/13.86/11.13  \\
\bottomrule
\end{tabular}
\end{adjustbox}
\end{table}

% \subsection{Full Results of~\sref{sec:weightactivation_quantization_existing_method}}
\begin{table}[t]
\caption{
\opt ppl on wikitext/opt/c4 with W4\asym-A8\sym/A8\asym (full table of~\tref{tab:weightactivation_quantization_opt_existing_method_average_in_main_text}).
See~\tref{tab:weightactivation_quantization_opt_existing_method_full_zqlocal} for all learning rate results of \zqlocal and~\tref{tab:weightactivation_quantization_opt_existing_method_full_zqglobal} of \zqglobal.
}\centering
\label{tab:weightactivation_quantization_opt_existing_method_full_in_appendix}
\begin{adjustbox}{width=0.9\linewidth}
\centering
\begin{tabular}{lcccccccccccccc }
\toprule
Precision     & 125m	& 350m 	& 1.3b	& 2.7b	& 6.7b	& 13b &30b	& 66b \\
\midrule
W4\asym-A8\sym Block\\
\rtn &36.69/44.34/30.60 &26.59/32.13/24.81 &25.31/26.89/22.01 &30.84/35.73/29.01 &164.51/110.85/162.94 &4460.61/3145.51/4255.84 &3216.45/2929.40/3570.19 &3038.22/2930.92/3001.82 \\
\gptq &32.20/38.49/27.47 &24.35/29.82/23.24 &16.28/19.64/16.73 &13.86/17.51/15.00 &46.22/53.98/55.13 &3611.71/2796.71/3820.57 &1738.44/1810.08/2119.82 &5992.87/4115.01/4360.16 \\ 
\zqlocalstar &32.88/38.23/28.20 &25.18/30.06/23.62 &16.78/20.25/17.09 &14.82/18.77/15.61 &16.08/21.15/18.77 &2680.33/1876.48/3052.51 &1884.90/1603.23/1348.08 &575.20/499.42/437.94 \\
\zqglobalstar &32.04/37.48/27.23 &24.01/28.81/22.57 &16.12/19.15/16.23 &13.98/17.70/14.87 &38.27/39.77/52.26 &117.83/141.63/96.83 &253.71/700.40/337.15 &1715.98/1546.50/1799.35\\ 
\midrule 
W4\asym-A8\asym Block \\
\rtn &36.61/44.48/30.64 &25.79/31.28/24.13 &21.23/23.54/19.19 &23.82/29.77/22.60 &13.18/17.04/14.19 &19.87/32.93/26.28 &36.07/136.88/85.84 &627.15/880.79/937.08 \\ 
\gptq &32.22/38.83/27.43 &23.90/29.29/22.63 &15.75/18.74/15.93 &13.23/16.31/14.03 &12.50/15.86/13.29 &12.79/21.99/17.05 &12.96/25.03/24.14 &495.70/681.68/768.69 \\ 
\zqlocalstar &33.60/38.57/28.02 &24.57/29.27/22.98 &15.98/19.13/16.20 &13.44/16.81/14.26 &11.76/14.97/13.00 &11.69/16.98/14.01 &12.38/24.25/18.96 &12.19/23.31/13.47 \\
\zqglobalstar &31.61/37.00/27.10 &23.66/28.56/22.21 &15.77/18.61/15.83 &13.09/16.56/14.00 &12.03/14.60/12.86 &11.80/15.01/12.41 &12.94/17.61/13.41 &31.51/58.00/23.95\\
\midrule
\bottomrule
\end{tabular}
\end{adjustbox}
\end{table}

\begin{table}[t]
\caption{
\opt ppl on wikitext/opt/c4 with W4\asym-A8\sym/A8\asym and \zqlocal. 
}\centering
\label{tab:weightactivation_quantization_opt_existing_method_full_zqlocal}
\begin{adjustbox}{width=0.9\linewidth}
\centering
\begin{tabular}{lcccccccccccccc }
\toprule
Precision     & 125m	& 350m 	& 1.3b	& 2.7b	& 6.7b	& 13b &30b	& 66b \\
\midrule
W4\asym-A8\sym Block\\
0.001 &34.91/40.43/29.37 &26.82/32.68/25.24 &17.68/21.72/18.11 &19.40/27.59/20.05 &36.70/59.32/45.17 &7240.89/5506.67/4889.34 &8229.32/5068.14/5005.13 & Diverge  \\
0.0005 &34.16/39.00/28.58 &26.75/32.05/24.60 &17.19/21.42/17.55 &19.43/25.54/19.41 &29.33/48.38/43.28 &56836.57/36810.64/31073.67 &5448.96/3826.63/3196.49 &575.20/499.42/437.94  \\
0.0001 &32.88/38.23/28.20 &25.31/31.60/23.98 &16.93/20.77/17.36 &17.05/21.50/17.42 &25.24/31.66/26.82 &6125.07/3817.01/4121.70 &1884.90/1603.23/1348.08 &5427.12/3449.58/3289.01  \\
5e-05 &32.86/39.17/27.91 &25.91/31.24/24.07 &16.99/20.02/17.23 &15.07/19.00/15.54 &16.08/21.15/18.77 &6037.51/3617.64/3819.63 &3266.46/2533.64/2463.21 &11631.78/10489.81/7880.43  \\
1e-05 &34.00/39.76/28.62 &25.40/30.60/23.75 &16.87/20.26/17.11 &14.82/18.77/15.61 &26.60/32.09/28.76 &5346.85/3788.29/4903.31 &3364.70/2372.71/3370.97 &5793.44/3544.90/3925.34  \\
5e-06 &34.37/41.46/28.71 &25.18/30.06/23.62 &16.78/20.25/17.09 &14.87/19.42/15.86 &34.53/39.98/38.22 &2680.33/1876.48/3052.51 &3566.45/2532.54/3678.75 &4916.96/3783.69/3716.49  \\
1e-06 &36.05/43.46/30.00 &25.73/30.69/24.05 &19.58/22.57/19.04 &18.66/24.19/19.98 &77.99/62.27/83.19 &3893.00/2672.11/3849.59 &3233.72/2944.44/3732.18 &4238.57/3621.09/3541.33  \\
\midrule 
W4\asym-A8\asym Block \\
0.001 &33.57/40.84/29.00 &27.29/32.48/24.68 &17.41/20.70/17.07 &15.98/20.45/16.23 &12.63/17.21/14.25 &9889.96/7605.54/6328.91 &2009.66/1637.69/2011.15 &5070.07/3124.56/2683.19  \\
0.0005 &34.58/40.45/28.69 &25.81/31.56/24.09 &16.89/20.66/16.93 &15.00/19.47/15.61 &12.55/17.00/14.29 &13.18/19.65/15.18 &36.51/75.89/60.58 &3249.10/63.17/119.55  \\
0.0001 &33.91/38.39/28.12 &25.37/31.24/23.66 &16.78/20.09/16.72 &14.26/18.49/14.90 &12.13/15.97/13.48 &13.48/20.42/16.68 &110.20/117.28/257.96 &12.19/23.31/13.47  \\
5e-05 &33.60/38.57/28.02 &24.67/29.60/23.34 &16.31/19.56/16.42 &13.90/19.16/15.05 &12.30/15.95/13.56 &12.05/18.00/15.77 &37.68/59.83/124.75 &29.72/95.99/69.60  \\
1e-05 &33.80/40.21/28.56 &24.57/29.27/22.98 &15.98/19.13/16.20 &13.44/16.81/14.26 &11.76/14.97/13.00 &11.69/16.98/14.01 &14.39/31.47/24.45 &217.93/313.13/298.24  \\
5e-06 &34.62/41.07/28.93 &24.68/29.46/23.12 &16.26/19.23/16.27 &13.44/17.00/14.36 &11.96/14.86/13.10 &12.31/18.55/15.16 &12.38/24.25/18.96 &85.96/185.07/180.88  \\
1e-06 &35.94/43.35/30.00 &24.92/30.18/23.45 &17.98/20.89/17.45 &14.79/18.90/15.52 &12.10/15.47/13.35 &15.48/22.00/17.84 &14.86/31.16/26.21 &411.89/620.52/652.55  \\
\bottomrule
\end{tabular}
\end{adjustbox}
\end{table}

\begin{table}[t]
\caption{
\opt ppl on wikitext/opt/c4 with W4\asym-A8\sym/A8\asym and \zqglobal. 
}\centering
\label{tab:weightactivation_quantization_opt_existing_method_full_zqglobal}
\begin{adjustbox}{width=0.9\linewidth}
\centering
\begin{tabular}{lcccccccccccccc }
\toprule
Precision     & 125m	& 350m 	& 1.3b	& 2.7b	& 6.7b	& 13b &30b	& 66b \\
\midrule
W4\asym-A8\sym Block\\
0.001 &34.90/44.82/28.27 &8988.08/5862.33/384.69 &nan/nan/nan &18290.16/9784.37/12099.01 &16014.50/8655.69/12304.55 &248961.98/84832.78/104880.55 &56675.05/23709.03/33007.17 &29782.43/20410.10/23559.66  \\
0.0005 &31.78/38.56/27.20 &39.24/54.15/29.76 &10610.96/9438.99/6752.84 &12499.29/8411.26/10677.01 &nan/nan/nan &74731.13/44494.68/29286.49 &51871.73/28548.95/23056.78 &18717.63/11744.97/12903.33  \\
0.0001 &32.04/37.48/27.23 &24.14/29.21/22.47 &17.04/23.64/17.13 &175.67/165.81/162.24 &12305.50/11472.90/10223.89 &16303.04/10731.12/10669.52 &22548.81/12474.28/7405.46 &7926.43/4377.36/4805.98  \\
5e-05 &32.16/37.54/27.27 &24.15/28.87/22.46 &16.02/19.61/16.59 &13.88/20.27/14.79 &5241.10/3284.47/2187.15 &13297.25/7781.85/7467.30 &9542.44/4543.45/5373.00 & NaN  \\
1e-05 &32.57/38.43/27.53 &24.01/28.81/22.57 &16.12/19.15/16.23 &13.98/17.70/14.87 &99.27/118.19/88.74 &529.82/361.44/256.46 &1936.12/1388.68/947.45 &10077.70/9208.34/11462.28  \\
5e-06 &32.83/38.37/27.71 &24.13/29.30/22.68 &16.45/19.64/16.57 &14.42/18.01/15.27 &70.26/62.28/54.47 &373.82/494.33/170.40 &820.90/847.19/543.59 &1867.57/1878.76/4117.49  \\
1e-06 &34.79/41.79/29.30 &24.68/30.01/23.23 &17.90/21.94/18.01 &14.83/18.63/15.70 &38.27/39.77/52.26 &117.83/141.63/96.83 &261.19/844.40/272.04 &1500.51/1275.54/1649.50  \\
5e-07 & NaN & NaN & NaN & NaN & NaN & NaN &253.71/700.40/337.15 &1715.98/1546.50/1799.35  \\
1e-07 & NaN & NaN & NaN & NaN & NaN & NaN &913.95/1117.58/1065.87 &2012.91/1917.48/1817.92  \\
\midrule 
W4\asym-A8\asym Block \\
0.001 &37.89/47.68/30.43 &9023.01/4309.50/1186.96 &12638.86/nan/9164.64 &11285.86/6477.19/nan &12222.01/6933.34/8989.30 &132962.69/73768.05/59268.76 &328993.91/187752.97/163157.59 &48298.52/30548.89/42797.96  \\
0.0005 &32.65/39.86/27.20 &28.46/36.94/24.68 &nan/nan/nan &nan/nan/nan &23287.96/15508.32/16243.28 &22052.30/10852.90/11588.02 &63084.59/39919.41/42499.90 & NaN  \\
0.0001 &31.61/37.00/27.10 &24.64/29.13/22.28 &16.31/19.71/16.44 &43.76/29.11/33.35 &22024.01/13962.04/14130.94 &10171.49/7200.78/7954.12 &18603.08/11639.42/10798.26 &nan/nan/nan  \\
5e-05 &32.21/37.46/27.18 &23.66/28.56/22.21 &16.02/19.02/15.92 &13.48/17.57/14.24 &839.48/213.76/286.05 &1035.13/nan/1472.08 &8085.92/3545.21/4893.07 &nan/nan/nan  \\
1e-05 &32.35/38.21/27.38 &23.59/28.66/22.24 &15.77/18.61/15.83 &13.09/16.56/14.00 &12.09/14.69/12.90 &11.80/15.01/12.41 &13.76/22.87/15.72 &974.58/1557.95/1039.65  \\
5e-06 &32.59/38.49/27.68 &23.62/28.63/22.33 &15.78/18.80/15.95 &13.23/16.65/14.12 &12.03/14.60/12.86 &12.72/16.31/13.20 &12.94/17.61/13.41 &83.35/137.83/128.11  \\
1e-06 &34.68/41.56/29.26 &24.08/29.21/22.68 &16.66/20.03/16.69 &13.30/16.74/14.33 &12.43/15.52/13.36 &12.28/16.13/13.19 &16.00/19.60/14.88 &31.51/58.00/23.95  \\
5e-07 & NaN & NaN & NaN & NaN & NaN & NaN & NaN &31.09/73.23/24.44  \\
1e-07 & NaN & NaN & NaN & NaN & NaN & NaN & NaN &241.81/544.81/505.58  \\
\bottomrule
\end{tabular}
\end{adjustbox}
\end{table}





%%%%%%%%%%%%%%%%

\begin{table}[t]
\caption{
\bloom ppl on wikitext/opt/c4 with W4\asym-A8\sym/A8\asym (full table of~\tref{tab:weightactivation_quantization_bloom_existing_method_average_in_main_text}).
See~\tref{tab:weightactivation_quantization_bloom_existing_method_full_zqlocal} for all learning rate results of \zqlocal and~\tref{tab:weightactivation_quantization_bloom_existing_method_full_zqglobal} of \zqglobal.
}\centering
\label{tab:weightactivation_quantization_bloom_existing_method_full_in_appendix}
\begin{adjustbox}{width=0.9\linewidth}
\centering
\begin{tabular}{lcccccccccccccc }
\toprule
Precision     & 560m   &1.1b   & 1.7b  & 3b & 7.1b & 176b \\
\midrule
W4\asym-A8\sym Block\\
\rtn  &25.56/47.53/27.31 &24.80/70.99/26.71 &17.36/31.95/19.89 &14.82/25.63/17.47 &12.33/21.62/15.13 &9.12/15.58/14.04 \\
\gptq &24.13/44.79/25.86 &25.69/68.65/27.08 &16.63/30.54/19.12 &14.18/24.42/16.82 &12.04/21.07/14.75 &8.92/15.16/13.56 \\ 
\zqlocalstar &24.45/45.73/26.22 &19.50/52.67/21.73 &16.71/30.23/19.09 &14.37/24.72/16.99 &12.00/20.79/14.78 &8.52/14.29/11.41 \\
\zqglobalstar &23.93/44.31/25.68 &19.71/51.98/21.85 &16.34/29.36/18.82 &14.13/24.34/16.76 &11.84/20.58/14.59 &8.76/14.60/11.68\\ 
\midrule 
W4\asym-A8\asym Block \\
\rtn &25.37/46.99/27.16 &24.08/68.95/26.17 &17.12/31.46/19.67 &14.74/25.38/17.37 &12.22/21.36/15.00 &8.73/15.10/12.83 \\ 
\gptq &24.09/44.29/25.66 &24.50/67.37/26.62 &16.39/29.83/18.91 &14.13/24.47/16.73 &11.91/20.72/14.62 &8.55/14.74/12.31 \\ 
\zqlocalstar &24.29/45.19/26.10 &19.13/52.89/21.63 &16.54/30.11/18.92 &14.32/24.73/16.94 &11.94/20.63/14.68 &8.33/14.01/11.22\\
\zqglobalstar &23.86/44.16/25.62 &19.54/51.72/21.79 &16.23/29.40/18.68 &14.15/24.29/16.72 &11.80/20.37/14.56 &8.62/14.40/11.49\\
\midrule
\bottomrule
\end{tabular}
\end{adjustbox}
\end{table}

\begin{table}[t]
\caption{
\bloom ppl on wikitext/opt/c4 with W4\asym-A8\sym/A8\asym and \zqlocal. 
}\centering
\label{tab:weightactivation_quantization_bloom_existing_method_full_zqlocal}
\begin{adjustbox}{width=0.9\linewidth}
\centering
\begin{tabular}{lcccccccccccccc }
\toprule
Precision     & 560m   &1.1b   & 1.7b  & 3b & 7.1b & 176b \\
\midrule
W4\asym-A8\sym Block\\
0.001 &25.51/47.89/27.15 &19.73/54.63/22.18 &16.96/31.47/19.44 &14.59/25.69/17.32 &12.51/21.85/15.34 &8.62/14.42/11.50  \\
0.0005 &25.18/47.35/26.95 &19.62/53.64/22.03 &16.98/31.75/19.47 &14.52/25.22/17.18 &12.03/21.01/14.82 &8.59/14.38/11.45  \\
0.0001 &24.79/46.37/26.44 &19.50/52.67/21.73 &16.68/30.51/19.18 &14.44/25.12/17.05 &12.00/20.79/14.78 &8.52/14.29/11.41  \\
5e-05 &24.56/46.29/26.34 &23.93/69.17/26.19 &16.71/30.23/19.09 &14.37/24.72/16.99 &12.05/20.92/14.82 &8.55/14.34/11.44  \\
1e-05 &24.45/45.73/26.22 &23.65/66.73/25.80 &16.66/30.69/19.16 &14.40/24.94/17.02 &12.12/21.14/14.86 &8.65/14.97/12.01  \\
5e-06 &24.48/45.66/26.33 &23.87/67.26/25.84 &16.78/30.72/19.23 &14.44/24.91/17.07 &12.15/21.23/14.88 &8.70/15.04/12.37  \\
1e-06 &24.91/46.35/26.72 &24.09/68.13/26.05 &17.03/31.28/19.52 &14.60/25.18/17.24 &12.22/21.31/14.99 &8.91/15.25/13.35  \\
\midrule 
W4\asym-A8\asym Block \\
0.001 &25.26/46.43/26.98 &19.69/54.26/22.14 &16.88/32.16/19.40 &15.15/26.58/17.76 &12.40/22.29/15.28 &8.40/14.06/11.26  \\
0.0005 &24.89/47.99/26.82 &19.54/53.57/21.98 &16.73/31.02/19.29 &14.50/25.52/17.11 &11.94/20.70/14.76 &8.33/14.01/11.22  \\
0.0001 &24.60/45.75/26.44 &19.13/52.89/21.63 &16.54/30.36/19.10 &14.37/24.91/16.93 &11.94/20.63/14.68 &8.35/14.04/11.24  \\
5e-05 &24.41/45.08/26.23 &23.59/67.14/25.79 &16.54/30.11/18.92 &14.29/24.83/16.92 &11.95/20.71/14.71 &8.36/14.10/11.25  \\
1e-05 &24.29/45.19/26.10 &23.35/65.26/25.38 &16.51/30.20/19.00 &14.32/24.73/16.94 &11.97/20.93/14.74 &8.44/14.30/11.45  \\
5e-06 &24.31/45.25/26.15 &23.41/66.18/25.48 &16.63/30.37/19.09 &14.33/24.74/16.96 &12.03/20.95/14.78 &8.52/14.66/11.86  \\
1e-06 &24.76/45.92/26.62 &23.52/66.38/25.66 &16.81/30.71/19.30 &14.53/24.92/17.14 &12.10/21.07/14.87 &8.62/14.92/12.41  \\
\bottomrule
\end{tabular}
\end{adjustbox}
\end{table}

\begin{table}[t]
\caption{
\bloom ppl on wikitext/opt/c4 with W4\asym-A8\sym/A8\asym and \zqglobal. 
}\centering
\label{tab:weightactivation_quantization_bloom_existing_method_full_zqglobal}
\begin{adjustbox}{width=0.9\linewidth}
\centering
\begin{tabular}{lcccccccccccccc }
\toprule
Precision    & 560m   &1.1b   & 1.7b  & 3b & 7.1b & 176b \\
\midrule
W4\asym-A8\sym Block\\
0.001 &174250016.00/201477664.00/1348168.88 &423532.56/906908.06/322995.69 &573201.81/1089364.38/498071.91 &544376.56/696942.56/540949.06 &nan/nan/nan & NaN  \\
0.0005 &70978.52/29214230.00/1151.72 &2880.81/15732.60/309.13 &505479.44/629035.56/29283.36 &140595.53/181082.25/33785.79 &378033.53/789890.00/191543.91 & NaN  \\
0.0001 &24.04/45.38/25.83 &19.44/52.38/21.77 &16.34/29.36/18.82 &14.32/24.74/16.88 &12.12/22.00/14.80 &249.47/26690.76/26.96  \\
5e-05 &23.93/44.31/25.68 &19.71/51.98/21.85 &16.18/29.71/18.71 &14.13/24.34/16.76 &11.84/20.58/14.59 &9.00/15.57/11.61  \\
1e-05 &23.99/44.44/25.77 &22.75/58.31/23.63 &16.28/29.96/18.81 &14.29/24.53/16.87 &11.87/20.57/14.64 &8.76/14.60/11.68  \\
5e-06 &24.14/44.77/25.90 &23.90/64.81/25.29 &16.36/30.03/18.91 &14.32/24.68/16.95 &11.91/20.60/14.71 &9.07/15.12/11.98  \\
1e-06 &24.62/45.70/26.33 &25.55/71.49/27.44 &16.61/30.47/19.17 &14.51/24.91/17.11 &12.06/20.93/14.86 &11.25/19.93/15.76  \\
\midrule 
W4\asym-A8\asym Block \\
0.001 &9059092.00/2932002.50/131873960.00 &499829.19/393190.53/346682.47 &1260531.12/2019747.88/460627.16 &1022130.19/872164.88/679662.62 &nan/nan/nan & NaN  \\
0.0005 &7633.14/378055.53/1032.16 &4271.83/85847.50/1555.66 &87087.04/217513.30/37000.13 &575008.56/814032.50/230285.80 &1212241.00/2389840.25/1504266.50 & NaN  \\
0.0001 &23.96/45.36/25.80 &19.37/52.25/21.88 &16.29/29.36/18.81 &14.32/24.66/16.86 &12.05/22.30/14.77 &1400.84/11880.12/392.79  \\
5e-05 &23.86/44.16/25.62 &19.54/51.72/21.79 &16.23/29.40/18.68 &14.15/24.29/16.72 &11.82/20.44/14.54 &8.73/20.30/11.41  \\
1e-05 &23.96/44.24/25.72 &22.55/58.10/23.49 &16.27/29.82/18.78 &14.16/24.35/16.80 &11.80/20.37/14.56 &8.62/14.40/11.49  \\
5e-06 &24.01/44.68/25.83 &23.67/64.20/25.08 &16.30/29.96/18.85 &14.24/24.49/16.86 &11.81/20.50/14.60 &8.69/14.56/11.58  \\
1e-06 &24.53/45.60/26.26 &24.82/71.17/26.84 &16.55/30.35/19.10 &14.40/24.76/17.01 &11.97/20.83/14.77 &9.14/16.63/17.69  \\
\bottomrule
\end{tabular}
\end{adjustbox}
\end{table}

% \subsection{Full Resuts of~\sref{sec:main_result_weightonly_quantization}}
\begin{table}[t]
\caption{
\opt full results of \tref{tab:opt-4bit-blocksize}.
}\centering
\label{tab:opt-4bit-blocksize-full}
\begin{adjustbox}{width=0.9\linewidth}
\centering
\begin{tabular}{lcccccccccccccc }
\toprule
Method     & 125m	& 350m 	& 1.3b	& 2.7b	& 6.7b	& 13b &30b	& 66b \\
\midrule
BS=1024 \\
\rtn  &N/A &25.42/30.62/23.61 &16.90/19.78/16.59 &N/A &11.63/14.41/12.65 &10.47/13.09/11.75 &9.97/12.40/11.09 &9.83/12.31/10.77 \\
&N/A &26.55 &17.76 &N/A &12.90 &11.77 &11.15 &10.97 \\
\gptq &N/A &23.65/29.09/22.43 &15.16/18.00/15.34 &N/A &11.10/13.40/11.99 &10.28/12.49/11.29 &9.58/11.91/10.75 &9.56/11.61/10.44\\
&N/A &25.05 &16.17 &N/A &12.16 &11.36 &10.75 &10.54\\
\zqglobalstar &N/A &23.27/27.97/21.93 &12.93/15.90/13.64 &N/A &10.98/13.60/12.04 &10.33/12.69/11.50 &9.78/12.16/10.90 &9.52/11.58/10.46\\
&N/A &24.39 &16.18 &N/A &12.21 &11.50 &10.95 &10.52\\
\midrule
BS=512 \\
\rtn &N/A &25.05/29.74/23.21 &15.71/19.05/16.09 &13.67/16.93/14.23 &11.32/14.22/12.50 &10.45/12.99/11.68 &10.03/12.27/11.03 &9.83/12.15/10.67 \\
&N/A &26.00 &16.95 &14.94 &12.68 &11.71 &11.11 &10.89 \\
\gptq &N/A &23.33/28.48/22.13 &15.15/17.95/15.26 &12.65/15.61/13.53 &10.94/13.37/11.94 &10.18/12.49/11.29 &9.58/11.87/10.75 &9.53/11.59/10.43\\
&N/A &24.65 &16.12 &13.93 &12.08 &11.32 &10.73 &10.52\\
\zqglobalstar &N/A &23.41/27.67/21.92 &14.91/17.73/15.25 &12.92/15.59/13.55 &11.08/13.51/11.99 &10.29/12.68/11.46 &9.79/12.16/10.87 &9.51/11.65/10.44\\
&N/A &24.34 &15.97 &14.02 &12.19 &11.48 &10.94 &10.53\\
\midrule
BS=256 \\
\rtn  &31.62/38.19/27.62 &24.76/29.44/22.96 &15.54/18.96/15.90 &13.56/16.62/14.02 &11.19/14.12/12.40 &10.39/12.93/11.61 &9.95/12.24/10.98 &9.70/12.09/10.62\\
&32.48 &25.72 &16.80 &14.73 &12.57 &11.64 &11.06 &10.80\\
\gptq &30.56/37.20/26.68 &23.37/28.33/21.97 &14.95/17.63/15.16 &12.59/15.60/13.49 &10.93/13.29/11.92 &10.15/12.43/11.27 &9.58/11.91/10.74 &9.49/11.60/10.40\\
&31.48 &24.56 &15.91 &13.89 &12.05 &11.28 &10.74 &10.50\\
\zqglobalstar &30.45/35.35/26.24 &23.06/27.72/21.74 &14.93/17.45/15.15 &12.99/15.47/13.50 &10.96/13.45/12.00 &10.25/12.61/11.43 &9.73/12.14/10.89 &9.49/11.58/10.42\\
&30.68 &24.17 &15.84 &13.99 &12.14 &11.43 &10.92 &10.50\\
\midrule
BS=128 \\
\rtn  &30.62/36.67/27.10 &24.12/29.34/22.70 &15.35/18.52/15.66 &13.19/16.24/13.88 &11.11/13.94/12.28 &10.31/12.82/11.54 &9.93/12.12/10.93 &9.56/11.85/10.56\\
&31.47 &25.39 &16.51 &14.43 &12.44 &11.56 &11.00 &10.65\\
\gptq &30.76/36.13/26.52 &23.29/27.94/21.98 &14.93/17.51/15.10 &12.49/15.59/13.46 &10.87/13.34/11.90 &10.11/12.47/11.27 &9.60/11.88/10.73 &9.44/11.53/10.40\\
&31.14 &24.40 &15.85 &13.85 &12.03 &11.28 &10.74 &10.45\\
\zqglobalstar &29.52/34.63/25.98 &22.78/27.56/21.65 &15.02/17.50/15.07 &12.67/15.37/13.45 &10.92/13.42/11.96 &10.16/12.61/11.41 &9.74/12.01/10.82 &9.43/11.49/10.40\\
&30.04 &23.99 &15.86 &13.83 &12.10 &11.39 &10.86 &10.44\\
\midrule
BS=64 \\
\rtn  &30.74/36.68/26.87 &24.28/28.95/22.59 &15.21/18.15/15.47 &13.20/16.13/13.75 &11.01/13.71/12.17 &10.27/12.79/11.49 &9.82/12.05/10.89 &9.46/11.70/10.49\\
&31.43 &25.27 &16.28 &14.36 &12.30 &11.52 &10.92 &10.55\\
\gptq &30.25/35.72/26.43 &23.39/27.55/21.75 &14.81/17.40/15.06 &12.54/15.54/13.44 &10.87/13.29/11.89 &10.09/12.44/11.27 &9.55/11.89/10.72 &9.33/11.49/10.38\\
&30.80 &24.23 &15.76 &13.84 &12.02 &11.27 &10.72 &10.40\\
\zqglobalstar &29.69/34.24/25.72 &22.94/27.49/21.54 &14.90/17.43/15.01 &12.80/15.47/13.44 &10.92/13.33/11.93 &10.21/12.58/11.38 &9.69/12.01/10.81 &9.41/11.49/10.39\\
&29.88 &23.99 &15.78 &13.90 &12.06 &11.39 &10.84 &10.43\\
\midrule
BS=32 \\
\rtn &30.48/36.32/26.64 &23.88/28.66/22.36 &14.99/17.87/15.32 &12.89/16.00/13.67 &10.89/13.70/12.13 &10.32/12.73/11.45 &9.76/12.00/10.85 &9.56/11.55/10.44\\
&31.14 &24.97 &16.06 &14.18 &12.24 &11.50 &10.87 &10.52 \\
\gptq &29.13/34.89/25.90 &23.09/27.59/21.65 &14.80/17.41/15.04 &12.45/15.55/13.42 &10.89/13.32/11.89 &10.08/12.48/11.27 &9.51/11.92/10.73 &Diverge\\
&29.97 &24.11 &15.75 &13.81 &12.03 &11.28 &10.72 &Diverge\\
\zqglobalstar &28.93/34.29/25.63 &22.85/27.23/21.50 &14.80/17.34/14.99 &12.74/15.32/13.40 &10.82/13.36/11.91 &10.23/12.61/11.37 &9.68/11.95/10.80 &9.37/11.47/10.38 \\
&29.62 &23.86 &15.71 &13.82 &12.03 &11.41 &10.81 &10.41\\
\bottomrule
\end{tabular}
\end{adjustbox}
\end{table}

\begin{table}[t]
\caption{
\bloom W4\asym-A16 with various block-size out of the best result from \gptq and \zqglobal.
See~\tref{tab:bloom-4bit-blocksize}.
}\centering
\label{tab:bloom-4bit-blocksize-full}
\begin{adjustbox}{width=0.9\linewidth}
\centering
\begin{tabular}{lcccccccccccccc }
\toprule
Method     & 560m   &1.1b   & 1.7b  & 3b & 7.1b & 176b \\
\midrule
BS=1024 \\
\rtn &24.90/46.37/26.68 &N/A &16.57/30.14/19.00 &N/A &1019.51/1351.45/601.35 &53.41/160.05/43.64\\
&32.65 &N/A &21.90 &N/A &990.77 &85.70\\
\gptq &23.90/43.99/25.47 &N/A &16.12/29.13/18.61 &N/A &11.57/19.82/14.33 &8.16/13.70/11.02\\
&31.12 &N/A &21.29 &N/A &15.24 &10.96\\
\zqglobal &23.62/43.90/25.41 &N/A &15.98/28.67/18.44 &N/A &11.91/20.84/14.58 &8.23/13.94/11.09 \\
&30.98 &N/A &21.03 &N/A &15.78 &11.09\\
\midrule
BS=512 \\
\rtn &24.78/46.07/26.45 &19.41/53.64/21.85 &16.47/29.84/18.88 &14.29/24.84/17.05 &142.38/314.10/100.09 &33.88/103.57/31.02\\
&32.44 &31.63 &21.73 &18.73 &185.52 &56.16\\
\gptq &23.63/43.96/25.36 &18.52/49.73/20.91 &16.07/29.87/18.50 &13.79/23.77/16.41 &11.54/19.75/14.30 &8.14/13.70/11.02\\
&30.98 &29.72 &21.48 &17.99 &15.20 &10.95\\
\zqglobal &23.50/43.53/25.23 &18.31/49.06/20.82 &15.93/28.47/18.38 &13.82/23.92/16.47 &11.85/20.17/14.42 &8.20/13.86/11.07\\
&30.75 &29.40 &20.93 &18.07 &15.48 &11.04\\
\midrule
BS=256 \\
\rtn &24.09/45.13/26.02 &18.87/52.29/21.44 &16.27/29.72/18.76 &14.16/24.42/16.90 &121.09/281.67/88.59 &12.55/27.29/15.60\\
&31.75 &30.87 &21.58 &18.49 &163.78 &18.48\\
\gptq &23.31/43.43/25.12 &18.36/49.13/20.79 &16.07/29.10/18.46 &13.76/23.61/16.38 &11.55/19.72/14.29 &8.14/13.70/11.01\\
&30.62 &29.42 &21.21 &17.92 &15.18 &10.95\\
\zqglobal &23.17/43.16/25.13 &18.24/48.78/20.75 &15.81/28.71/18.32 &13.79/23.69/16.42 &11.59/19.92/14.36 &8.17/13.80/11.06\\
&30.49 &29.26 &20.95 &17.97 &15.29 &11.01\\
\midrule
BS=128 \\
\rtn &23.82/44.78/25.75 &18.62/51.31/21.17 &16.13/29.89/18.66 &14.00/24.19/16.71 &23.90/49.80/24.15 &8.84/15.62/11.70\\
&31.45 &30.37 &21.56 &18.30 &32.62 &12.06\\
\gptq &23.27/43.10/24.99 &18.14/48.72/20.73 &16.03/28.96/18.41 &13.72/23.65/16.34 &11.52/19.73/14.26 &8.14/13.67/11.01\\
&30.45 &29.20 &21.13 &17.90 &15.17 &10.94\\
\zqglobal &23.14/42.95/24.97 &18.17/48.53/20.70 &15.75/28.71/18.29 &13.73/23.65/16.37 &11.56/19.77/14.32 &8.17/13.78/11.03\\
&30.35 &29.13 &20.92 &17.92 &15.22 &10.99\\
\midrule
BS=64 \\
\rtn &23.65/44.04/25.51 &18.53/50.02/21.03 &16.06/29.57/18.60 &13.93/23.95/16.60 &11.85/20.51/14.65 &8.31/14.14/11.18\\
&31.07 &29.86 &21.41 &18.16 &15.67 &11.21\\
\gptq &23.11/42.95/24.94 &18.14/48.87/20.65 &16.00/28.91/18.38 &13.72/23.68/16.33 &11.51/19.70/14.27 &8.14/13.69/11.00\\
&30.33 &29.22 &21.10 &17.91 &15.16 &10.94\\
\zqglobal &23.00/42.80/24.91 &18.10/48.30/20.64 &15.68/28.55/18.25 &13.70/23.63/16.36 &11.53/19.67/14.27 &8.17/13.72/11.02\\
&30.24 &29.01 &20.82 &17.90 &15.16 &10.97\\
\midrule
BS=32 \\
\rtn &23.60/43.91/25.50 &18.63/50.13/21.04 &15.98/29.56/18.56 &13.92/23.90/16.53 &11.65/20.01/14.43 &8.20/13.86/11.07\\
&31.00 &29.93 &21.37 &18.12 &15.36 &11.04\\
\gptq &23.10/43.19/24.91 &18.17/48.35/20.66 &15.95/28.95/18.36 &13.76/23.60/16.33 &11.53/19.71/14.27 &8.14/13.70/11.00\\
&30.40 &29.06 &21.08 &17.89 &15.17 &10.95\\
\zqglobal &23.07/42.63/24.82 &18.07/48.07/20.59 &15.66/28.58/18.21 &13.72/23.59/16.33 &11.52/19.71/14.26 &8.16/13.69/11.01\\
&30.18 &28.91 &20.82 &17.88 &15.16 &10.95\\
\bottomrule
\end{tabular}
\end{adjustbox}
\end{table}


% 3bit
\begin{table}[t]
\caption{
\opt full results of \tref{tab:opt-3bit-blocksize}.
}\centering
\label{tab:opt-3bit-blocksize-full}
\begin{adjustbox}{width=0.9\linewidth}
\centering
\begin{tabular}{lcccccccccccccc }
\toprule
Method     & 125m	& 350m 	& 1.3b	& 2.7b	& 6.7b	& 13b &30b	& 66b \\
\midrule
Full Row \\
\rtn  &2095.20/1848.83/1222.00 &47.43/53.38/36.93 &4399.18/4400.98/3551.88 &8326.78/4208.57/4895.83 &878.00/735.86/910.10 &1953.43/1953.60/1669.76 &439.39/691.94/437.96 &1465.06/1564.59/1282.58\\
&1722.01 &45.91 &4117.35 &5810.40 &841.32 &1858.93 &523.09 &1437.41\\
\gptq &845.81/599.71/496.14 &30.65/34.09/26.15 &20.23/27.39/19.45 &15.91/19.26/16.01 &12.69/15.90/13.96 &11.36/13.71/12.21 &10.10/12.54/11.20 &16.77/21.16/15.39\\
&647.22 &30.30 &22.36 &17.06 &14.18 &12.43 &11.28 &17.77\\
\zqglobalstar &46.47/58.55/35.45 &29.64/36.51/25.55 &32.48/94.57/28.97 &60.91/116.22/36.45 &23.87/29.75/23.88 &44.70/60.78/46.18 &13.16/20.49/13.48 &28.93/75.91/27.28\\
&46.82 &30.57 &52.01 &71.19 &25.83 &50.55 &15.71 &44.04\\
\midrule
BS=1024 \\
\rtn &N/A &44.57/49.58/35.09 &1950.00/2317.55/1913.55 &3810.79/2563.06/3054.91 &50.01/70.17/99.21 &265.62/417.03/261.93 &362.47/252.33/364.45 &523.81/846.60/1021.17\\
&N/A &43.08 &2060.37 &3142.92 &73.13 &314.86 &326.42 &797.20\\
\gptq &N/A &29.78/33.76/25.66 &19.03/23.32/18.14 &N/A &11.69/14.31/12.70 &10.56/12.96/11.70 &9.89/12.19/11.02 &12.84/16.17/13.02\\
&N/A &29.73 &20.16 &N/A &12.90 &11.74 &11.03 &14.01\\
\zqglobalstar &N/A &29.19/34.57/25.11 &19.83/29.77/19.79 &N/A &13.99/18.82/14.76 &13.43/19.28/13.76 &11.10/14.46/11.94 &11.87/14.86/12.13\\
&N/A &29.62 &23.13 &N/A &15.86 &15.49 &12.50 &12.95\\
\midrule
BS=512 \\
\rtn &N/A &37.74/45.10/31.85 &1777.53/1304.55/852.03 &1604.07/1407.49/1487.78 &25.13/40.56/40.08 &130.75/175.33/135.67 &620.53/340.68/416.28 &198.01/457.78/426.15\\
&N/A &38.23 &1311.37 &1499.78 &35.26 &147.25 &459.16 &360.65\\
\gptq &N/A &28.46/32.54/25.14 &18.02/21.35/17.46 &14.38/17.24/14.79 &11.57/14.33/12.57 &10.41/12.97/11.64 &9.77/12.18/10.97 &11.89/14.48/12.40\\
&N/A &28.71 &18.94 &15.47 &12.82 &11.67 &10.97 &12.92\\
\zqglobalstar &N/A &27.81/33.57/24.55 &18.31/23.54/17.99 &18.10/29.47/17.15 &12.54/16.60/13.62 &11.82/15.98/12.81 &10.48/13.36/11.66 &11.26/13.95/11.79\\
&N/A &28.65 &19.95 &21.57 &14.25 &13.54 &11.83 &12.33\\
\midrule
BS=256 \\
\rtn &4349.14/2907.61/2510.75 &35.36/42.07/30.81 &127.17/358.19/142.49 &670.51/550.66/531.80 &19.10/32.39/27.26 &42.52/56.35/43.32 &32.84/60.38/33.48 &210.01/478.13/413.00\\
&3255.84 &36.08 &209.28 &584.32 &26.25 &47.40 &42.23 &367.05\\
\gptq &41.81/49.95/32.48 &27.60/33.73/24.88 &16.97/20.19/16.70 &13.69/17.06/14.54 &11.65/14.24/12.48 &10.35/12.93/11.61 &9.66/12.10/10.93 &11.60/13.98/11.92\\
&41.41 &28.74 &17.95 &15.10 &12.79 &11.63 &10.90 &12.50\\
\zqglobalstar &38.60/46.57/31.36 &26.88/32.79/24.08 &16.82/21.21/17.05 &14.86/19.63/15.37 &11.86/15.87/13.10 &11.33/14.95/12.48 &10.41/12.95/11.41 &10.26/12.66/11.08\\
&38.85 &27.92 &18.36 &16.62 &13.61 &12.92 &11.59 &11.34\\
\midrule
BS=128 \\
\rtn &3446.89/2156.26/1484.15 &33.13/41.23/29.51 &49.40/88.45/45.07 &153.68/155.21/113.98 &16.34/26.86/21.98 &17.80/25.95/18.28 &45.83/43.91/57.50 &106.84/241.02/212.94\\
&2362.43 &34.62 &60.97 &140.96 &21.72 &20.67 &49.08 &186.93 \\
\gptq &40.00/45.73/31.15 &27.68/34.04/25.18 &16.47/19.90/16.47 &13.81/16.96/14.37 &11.57/14.10/12.41 &10.35/12.84/11.58 &9.73/12.08/10.91 &10.96/13.27/11.45\\
&38.96 &28.97 &17.61 &15.05 &12.69 &11.59 &10.91 &11.90\\
\zqglobalstar &36.57/43.88/29.94 &25.75/31.59/23.57 &16.28/20.20/16.67 &14.27/18.41/14.90 &11.70/15.05/12.68 &11.13/15.07/12.17 &10.31/12.99/11.32 &10.12/12.66/11.01\\
&36.80 &26.97 &17.72 &15.86 &13.14 &12.79 &11.54 &11.27\\
\midrule
BS=64 \\
\rtn &708.02/477.13/287.03 &32.61/42.14/29.09 &25.43/38.84/24.63 &72.84/69.27/48.07 &14.11/21.71/16.56 &14.13/20.08/15.25 &20.55/32.74/24.49 &30.66/70.73/65.57\\
&490.73 &34.61 &29.63 &63.39 &17.46 &16.48 &25.93 &55.65\\
\gptq &37.15/42.59/30.07 &27.68/33.55/25.12 &16.25/19.80/16.32 &13.66/16.69/14.37 &11.42/13.98/12.37 &10.37/12.90/11.58 &9.68/12.17/10.92 &10.39/12.65/11.15\\
&36.60 &28.78 &17.46 &14.91 &12.59 &11.62 &10.92 &11.40\\
\zqglobalstar &35.82/40.98/29.65 &25.31/31.60/23.38 &16.05/19.77/16.39 &13.33/16.92/14.31 &11.56/14.70/12.59 &10.88/13.64/12.04 &10.04/12.70/11.27 &10.04/12.06/10.81\\
&35.48 &26.76 &17.40 &14.85 &12.95 &12.19 &11.34 &10.97\\
\midrule
BS=32 \\
\rtn &72.83/88.62/54.25 &32.36/40.76/29.06 &20.22/27.31/19.81 &31.12/42.01/26.83 &13.38/18.56/15.44 &13.06/18.35/14.38 &11.12/15.05/12.35 &19.29/43.61/34.10\\
&71.90 &34.06 &22.44 &33.32 &15.79 &15.26 &12.84 &32.33\\
\gptq &38.26/45.01/30.92 &27.16/33.65/24.97 &16.13/19.83/16.45 &13.66/17.06/14.50 &11.43/14.08/12.42 &10.48/12.96/11.65 &9.78/12.24/10.96 &Diverge\\
&38.06 &28.59 &17.47 &15.07 &12.64 &11.70 &10.99 &Diverge\\
\zqglobalstar &33.44/39.48/28.33 &25.19/30.73/23.22 &15.62/19.52/16.20 &13.35/16.64/14.18 &11.56/14.38/12.61 &10.86/13.64/12.03 &10.25/12.86/11.28 &9.99/12.05/10.81\\
&33.75 &26.38 &17.11 &14.73 &12.85 &12.17 &11.46 &10.95\\
\bottomrule
\end{tabular}
\end{adjustbox}
\end{table}


\begin{table}[t]
\caption{
\bloom W3\asym-A16 with various block-size out of the best result from \gptq and \zqglobal.
See~\tref{tab:bloom-3bit-blocksize}.
}\centering
\label{tab:bloom-3bit-blocksize-full}
\begin{adjustbox}{width=0.9\linewidth}
\centering
\begin{tabular}{lcccccccccccccc }
\toprule
Method     & 560m   &1.1b   & 1.7b  & 3b & 7.1b & 176b \\
Full row \\
\rtn &68.45/132.83/59.22 &118.61/317.41/99.65 &31.15/67.23/34.02 &31.07/59.03/32.17 &66140.72/78568.16/44504.19 &100371.84/166012.19/137892.34 \\
&86.83 &178.56 &44.14 &40.76 &63071.02 &134758.79\\
\gptq &46.92/84.69/39.50 &49.78/142.95/43.84 &19.70/41.35/21.74 &22.84/46.49/22.90 &52966.59/52979.88/37115.48 &Diverge\\
&57.04 &78.85 &27.59 &30.74 &47687.32 &Diverge\\
\zqglobal &33.20/64.61/32.30 &34.16/100.05/29.22 &19.22/36.30/21.25 &18.41/33.10/20.79 &273.55/439.59/100.79 &27.19/75.74/45.45\\
&43.37 &54.48 &25.59 &24.10 &271.31 &49.46\\
\midrule
\midrule
BS=1024 \\
\rtn &47.00/86.57/43.37 &70.81/230.74/70.78 &35.41/65.75/33.54 &22.12/40.65/24.55 &25654.77/25531.66/15868.46 &141324.41/183583.73/200436.33\\
&58.98 &124.11 &44.90 &29.11 &22351.63 &175114.82\\
\gptq &31.25/58.80/30.94 &N/A &19.11/37.07/20.90 &N/A &12.59/21.95/15.21 &8.31/13.96/11.17 \\
&40.33 &N/A &25.69 &N/A &16.58 &11.15 \\
\zqglobal &28.91/55.81/29.59 &N/A &18.20/34.13/20.40 &N/A &30.94/119.98/21.39 &15.98/32.85/19.85\\
&38.10 &N/A &24.24 &N/A &57.44 &22.89\\
\midrule
BS=512 \\
\rtn &41.58/79.83/39.41 &33.83/116.88/37.34 &25.95/49.65/26.77 &19.94/38.58/22.58 &9777.49/8000.29/5407.46 &202051.34/273707.81/279776.97\\
&53.61 &62.68 &34.12 &27.03 &7728.41 &251845.38\\
\gptq &28.08/53.15/29.05 &21.20/61.42/23.33 &18.41/34.47/20.43 &15.08/26.14/17.53 &12.32/21.29/15.01 &8.30/13.98/11.16\\
&36.76 &35.32 &24.44 &19.58 &16.21 &11.15\\
\zqglobal &26.80/50.49/28.31 &20.77/57.57/22.89 &17.64/33.19/19.91 &15.16/26.51/17.57 &16.35/28.75/15.76 &11.38/20.36/14.66\\
&35.20 &33.75 &23.58 &19.75 &20.29 &15.47\\
\midrule
BS=256 \\
\rtn &36.13/70.37/36.29 &28.65/95.72/31.80 &21.67/42.59/23.80 &17.64/32.82/20.69 &1322.61/1864.55/946.92 &166006.80/187829.98/198052.83\\
&47.60 &52.06 &29.35 &23.72 &1378.02 &183963.20\\
\gptq &27.10/51.11/28.24 &20.60/56.57/22.77 &17.97/33.28/20.04 &14.82/25.79/17.31 &12.27/21.24/14.93 &8.27/13.99/11.14\\
&35.48 &33.31 &23.76 &19.31 &16.15 &11.13\\
\zqglobal &25.96/49.75/27.59 &20.21/54.83/22.33 &17.43/32.14/19.67 &14.85/25.79/17.33 &12.85/22.00/15.04 &9.07/15.88/11.88\\
&34.43 &32.46 &23.08 &19.32 &16.63 &12.28\\
\midrule
BS=128 \\
\rtn &34.71/66.56/35.27 &24.43/73.77/26.90 &19.59/37.22/21.98 &16.11/28.81/18.89 &108.32/252.15/74.42 &111057.84/101926.99/105339.26\\
&45.51 &41.70 &26.26 &21.27 &144.96 &106108.03\\
\gptq &26.29/49.86/27.54 &20.26/55.76/22.42 &17.77/32.65/19.92 &14.58/25.25/17.11 &12.18/21.06/14.86 &8.26/13.92/11.12\\
&34.56 &32.81 &23.45 &18.98 &16.03 &11.10\\
\zqglobal &25.28/48.24/26.96 &19.79/54.04/22.03 &17.12/31.42/19.31 &14.62/25.73/17.17 &12.04/21.02/14.82 &8.43/14.44/11.29\\
&33.49 &31.95 &22.62 &19.17 &15.96 &11.39\\
\midrule
BS=64 \\
\rtn &30.88/59.01/32.08 &23.04/67.93/25.49 &19.35/37.67/21.80 &15.64/27.56/18.39 &37.15/65.22/33.22 &198.66/488.11/128.62\\
&40.66 &38.82 &26.27 &20.53 &45.20 &271.80\\
\gptq &26.31/49.91/27.17 &20.11/55.06/22.23 &17.94/32.42/19.76 &14.62/25.39/17.07 &12.13/21.07/14.83 &8.26/13.93/11.11\\
&34.46 &32.47 &23.37 &19.02 &16.01 &11.10\\
\zqglobal &25.17/48.01/26.59 &19.51/53.27/21.75 &16.88/31.14/19.22 &14.51/25.18/17.05 &12.00/20.85/14.74 &8.35/14.06/11.20\\
&33.26 &31.51 &22.41 &18.91 &15.86 &11.21\\
\midrule
BS=32 \\
\rtn &30.15/57.55/31.51 &23.49/70.15/25.56 &18.96/36.54/21.42 &15.56/27.48/18.32 &13.06/23.77/16.05 &10.28/18.90/13.27\\
&39.74 &39.73 &25.64 &20.46 &17.62 &14.15\\
\gptq &25.96/49.99/27.06 &19.97/54.79/22.16 &17.60/32.24/19.76 &14.55/25.76/17.06 &12.20/21.01/14.85 &8.28/13.95/11.13\\
&34.33 &32.31 &23.20 &19.12 &16.02 &11.12\\
\zqglobal &25.09/47.36/26.34 &19.43/52.95/21.64 &16.86/30.49/19.11 &14.50/25.36/16.99 &12.00/20.84/14.72 &8.35/14.04/11.20\\
&32.93 &31.34 &22.15 &18.95 &15.85 &11.20\\
\bottomrule
\end{tabular}
\end{adjustbox}
\end{table}


\begin{table}[t]
\caption{
Full results of \bloom-176B with different quantization bits
}\centering
\label{tab:bloom-176-different-bits-full}
\begin{adjustbox}{width=0.9\linewidth}
\centering
\begin{tabular}{lcccccccccccccc }
\toprule
Bits     & 3   &4    &5 &6 &7 &8 \\
\midrule
Per-row &27.19/75.74/45.45 &8.16/13.70/11.02 &8.13/13.67/10.99 &8.11/13.63/10.98 &8.11/13.62/10.97 &8.10/13.62/10.98\\
1024    &8.31/13.96/11.17  &8.14/13.70/11.02 &8.11/13.62/10.97  &8.11/13.62/10.97 &8.11/13.63/10.97 &N/A\\
64      &8.26/13.93/11.11  &8.14/13.69/11.00 &8.11/13.62/10.96 &N/A &N/A &N/A\\
\bottomrule
\end{tabular}
\end{adjustbox}
\end{table}
\begin{table}[t]
\caption{
\opt full results of \tref{tab:opt-4bit8bit-blocksize}.
}\centering
\label{tab:opt-4bit8bit-blocksize-full}
\begin{adjustbox}{width=0.9\linewidth}
\centering
\begin{tabular}{lcccccccccccccc }
\toprule
Method     & 125m	& 350m 	& 1.3b	& 2.7b	& 6.7b	& 13b &30b	& 66b \\
\midrule
W4\asym full row and A8\sym 128\\
\rtn &36.64/44.84/30.90 &25.58/31.06/23.99 &19.96/22.31/18.20 &18.42/23.01/18.56 &12.04/15.92/13.20 &10.79/13.65/12.11 &10.10/13.17/11.37 &20.50/45.58/25.37\\
&37.46 &26.88 &20.16 &20.00 &13.72 &12.18 &11.54 &30.48\\
\gptq &31.82/38.82/27.54 &23.78/28.96/22.61 &15.56/18.27/15.62 &13.02/15.88/13.76 &11.22/13.59/12.11 &10.25/12.65/11.37 &9.56/11.94/10.79 &9.62/11.72/10.54\\
&32.73 &25.12 &16.48 &14.22 &12.31 &11.42 &10.76 &10.63\\
\zqlocal &&&&& &&&9.79/11.94/10.65\\
&&&&&&&&10.79\\
\zqglobal &31.69/36.66/27.19 &23.47/28.18/22.03 &15.53/18.35/15.73 &13.02/16.11/13.82 &11.29/13.70/12.19 &10.43/12.91/11.64 &9.86/12.28/11.00 &9.62/11.84/10.63\\
&31.85 &24.56 &16.54 &14.32 &12.39 &11.66 &11.05 &10.70\\
\midrule
W4\asym 128 and A8\sym 128\\
\rtn &30.61/36.57/27.08 &24.14/29.47/22.80 &15.46/18.68/15.77 &13.24/16.36/13.95 &11.16/14.08/12.35 &10.35/12.89/11.57 &9.95/12.15/10.95 &9.58/11.90/10.58\\
&31.42 &25.47 &16.64 &14.52 &12.53 &11.60 &11.02 &10.69\\
\gptq &30.47/36.45/26.45 &23.43/28.12/22.06 &14.90/17.62/15.17 &12.51/15.63/13.48 &10.88/13.35/11.93 &10.17/12.48/11.28 &9.58/11.86/10.74 &9.35/11.54/10.40\\
&31.12 &24.54 &15.90 &13.87 &12.05 &11.31 &10.73 &10.43\\
\zqlocal &&&&&&&&9.40/11.63/10.51\\
&&&&&&&&10.51\\
\zqglobal &29.59/34.68/25.91 &22.59/27.93/21.68 &14.87/17.55/15.11 &12.65/15.45/13.48 &10.88/13.40/11.94 &10.20/12.67/11.43 &9.74/12.03/10.83 &9.40/11.51/10.42\\
&30.06 &24.07 &15.84 &13.86 &12.08 &11.43 &10.87 &10.44\\
\midrule
W4\asym full row and A8\asym 128\\
\rtn &36.61/44.71/30.85 &25.50/30.93/23.88 &19.58/22.08/18.01 &19.53/24.38/19.68 &11.91/15.35/13.01 &10.68/13.50/12.02 &10.13/13.21/11.37 &17.90/32.15/20.02\\
&37.39 &26.77 &19.89 &21.20 &13.42 &12.07 &11.57 &23.36\\
\gptq &32.15/39.58/27.65 &23.48/28.92/22.46 &15.43/18.24/15.55 &12.92/15.94/13.74 &11.17/13.59/12.09 &10.35/12.63/11.36 &9.65/11.95/10.79 &9.58/11.71/10.55\\
&33.13 &24.95 &16.40 &14.20 &12.29 &11.45 &10.80 &10.61\\
\zqlocal &&&&&&& &10.05/11.91/10.61\\
&&&&&&&&10.86\\
\zqglobal &31.55/37.49/27.25 &23.34/28.33/22.08 &15.52/18.55/15.61 &13.07/16.09/13.82 &11.32/13.65/12.16 &10.42/12.86/11.63 &9.86/12.30/11.00 &9.67/12.22/10.86\\
&32.10 &24.58 &16.56 &14.33 &12.37 &11.64 &11.05 &10.91\\
\midrule
W4\asym 128 and A8\asym 128\\
\rtn &30.59/36.56/27.07 &24.11/29.43/22.74 &15.38/18.57/15.69 &13.22/16.32/13.91 &11.13/13.97/12.30 &10.34/12.82/11.55 &9.98/12.15/10.96 &9.57/11.86/10.58\\
&31.41 &25.43 &16.55 &14.49 &12.47 &11.57 &11.03 &10.67\\
\gptq &30.47/36.19/26.40 &23.35/27.96/21.94 &14.92/17.57/15.12 &12.48/15.60/13.46 &10.87/13.34/11.91 &10.20/12.45/11.28 &9.62/11.88/10.74 &9.39/11.55/10.41\\
&31.02 &24.42 &15.87 &13.85 &12.04 &11.31 &10.75 &10.45\\
\zqlocal &&&&&&& &9.37/11.70/10.49\\
&&&&&&&&10.52\\
\zqglobal &29.85/34.52/26.10 &22.70/27.72/21.64 &14.96/17.55/15.09 &12.64/15.40/13.47 &10.93/13.43/11.95 &10.18/12.68/11.42 &9.74/12.02/10.83 &9.39/11.53/10.42\\
&30.16 &24.02 &15.86 &13.84 &12.10 &11.42 &10.86 &10.45\\
\bottomrule
\end{tabular}
\end{adjustbox}
\end{table}


\begin{table}[t]
\caption{
\bloom full results of \tref{tab:bloom-176-different-blocks}.
}\centering
\label{tab:bloom-4bit8bit-blocksize-full}
\begin{adjustbox}{width=0.9\linewidth}
\centering
\begin{tabular}{lcccccccccccccc }
\toprule
Method     & 560m   &1.1b   & 1.7b  & 3b & 7.1b & 176b \\
\midrule
W4\asym full row and A8\sym 128\\
\rtn &25.32/46.98/27.12 &23.87/68.29/25.97 &16.99/31.15/19.51 &14.69/25.22/17.30 &12.07/20.86/14.84 &8.34/14.05/11.24\\
&33.14 &39.38 &22.55 &19.07 &15.92 &11.21\\
\gptq &24.00/44.47/25.66 &24.14/66.95/26.17 &16.38/29.64/18.79 &14.10/24.19/16.67 &11.77/20.22/14.48 &8.20/13.82/11.07\\
&31.37 &39.09 &21.61 &18.32 &15.49 &11.03\\
\zqlocal &&&&&&8.30/14.01/11.20 \\
 &&&&&&11.17\\
\zqglobal &23.92/44.23/25.69 &22.53/57.71/23.51 &16.25/29.72/18.74 &14.12/24.26/16.74 &11.78/20.30/14.53 &8.24/13.82/11.10\\
&31.28 &34.58 &21.57 &18.38 &15.53 &11.05\\
\midrule
W4\asym 128 and A8\sym 128\\
\rtn &23.84/44.94/25.79 &18.65/51.54/21.21 &16.18/30.03/18.70 &14.04/24.32/16.77 &23.05/48.33/23.69 &8.87/15.68/11.72\\
&31.53 &30.46 &21.64 &18.38 &31.69 &12.09\\
\gptq &23.22/43.24/25.01 &18.25/48.89/20.74 &16.00/29.44/18.41 &13.77/23.68/16.35 &11.54/19.76/14.27 &8.13/13.69/11.01\\
&30.49 &29.29 &21.29 &17.93 &15.19 &10.95\\
\zqlocal &&&&&&8.20/13.87/11.08\\
&&&&& &11.05\\
\zqglobal &23.12/43.22/25.03 &18.19/48.96/20.72 
&15.75/28.81/18.30 &13.73/23.65/16.39 &11.57/19.85/14.32 &8.17/13.76/11.03\\
&30.45 &29.29 &20.95 &17.92 &15.25 &10.99\\
\midrule
W4\asym full row and A8\asym 128\\
\rtn &25.30/46.87/27.10 &23.90/68.31/25.98 &16.96/31.09/19.48 &14.68/25.19/17.28 &12.07/20.86/14.84 &8.34/14.06/11.24\\
&33.09 &39.39 &22.51 &19.05 &15.92 &11.21\\
\gptq &23.97/44.15/25.62 &24.61/68.19/26.53 &16.36/29.77/18.81 &14.10/24.17/16.66 &11.78/20.32/14.49 &8.20/13.82/11.07\\
&31.24 &39.78 &21.65 &18.31 &15.53 &11.03\\
\zqlocal &&&&& &8.32/13.97/11.20\\
&&&&& &11.16\\
\zqglobal &23.88/44.40/25.68 &22.63/57.91/23.39 &16.25/29.77/18.74 &14.17/24.24/16.74 &11.77/20.28/14.52 &8.25/13.82/11.10\\
&31.32 &34.64 &21.59 &18.38 &15.52 &11.06\\
\midrule
W4\asym 128 and A8\asym 128\\
\rtn &23.83/44.89/25.77 &18.63/51.46/21.19 &16.16/29.95/18.68 &14.03/24.27/16.75 &23.51/49.07/23.96 &8.85/15.65/11.72\\
&31.50 &30.43 &21.60 &18.35 &32.18 &12.08\\
\gptq &23.26/43.24/25.00 &18.18/48.84/20.73 &16.05/29.34/18.42 &13.69/23.56/16.34 &11.54/19.75/14.28 &8.14/13.71/11.02\\
&30.50 &29.25 &21.27 &17.86 &15.19 &10.96\\
\zqlocal &&&&&&8.19/13.90/11.07\\
&&&&&&11.06\\
\zqglobal &23.12/43.14/25.01 &18.18/48.99/20.73 &15.71/28.73/18.30 &13.74/23.68/16.39 &11.56/19.85/14.31 &8.17/13.78/11.04\\
&30.42 &29.30 &20.91 &17.94 &15.24 &11.00\\
\bottomrule
\end{tabular}
\end{adjustbox}
\end{table}

\begin{table}[t]
\caption{
Full results of~\tref{tab:bloom-176-different-blocks}.
}\centering
\label{tab:bloom-176-different-blocks-full}
\begin{adjustbox}{width=0.9\linewidth}
\centering
\begin{tabular}{lcccccccccccccc }
\toprule
Block SIze     & 1024   &512    &256 &128 &64 &32 \\
\midrule
\ppl &8.16/13.75/11.04 &8.15/13.75/11.02 &8.15/13.70/11.01 &8.13/13.69/11.01 &8.14/13.69/11.01 &8.14/13.69/11.01\\
\bottomrule
\end{tabular}
\end{adjustbox}
\end{table}


\begin{table}[t]
\caption{
Results of applying \lorc on top of \zqglobal for INT8 Activation.
}\centering
\label{tab:LORC-int8}
\begin{adjustbox}{width=0.9\linewidth}
\centering
\begin{tabular}{lcc|ccccc|ccccccc }
\toprule
                      &                    &            & \multicolumn{5}{c|}{Learning Rate}                    &        \\
model-size            & precision         & LoRC-dim & 0.0005   & 0.0001   & 5.00E-05 & 1.00E-05 & 5.00E-06 & Best   \\\midrule
\multirow{3}{*}{125m} & \multirow{3}{*}{W4A8} & 0          & 4482.1   & 31.15    & 30.40     & 30.55    & 30.72    & 30.40   \\
                      &                    & 8          & 5996.14  & 30.96    & 30.24    & 30.37    & 30.61    & 30.24  \\
                      &                    & 16         & 3577.12  & 31.02    & 30.26    & 30.2     & 30.37    & 30.20   \\\midrule
 \multirow{3}{*}{125m}& \multirow{3}{*}{W3A8} & 0          & 4283.28  & 41.03    & 40.93    & 55.74    & 86.34    & 40.93  \\
                      &                    & 8          & 2396.92  & 37.25    & 36.65    & 37.85    & 39.06    & 36.65  \\
                      &                    & 16         & 1787.74  & 36.66    & 36.55    & 37.46    & 38.21    & 36.55  \\\midrule
\multirow{3}{*}{125m}& \multirow{3}{*}{W2A8} & 0          & 3473.18  & 583.72   & 996.76   & 2480.69  & 3203.11  & 583.72 \\
                      &                    & 8          & 3815.37  & 144.85   & 160.71   & 362.17   & 466.98   & 144.85 \\
                      &                    & 16         & 3324.23  & 135.25   & 156.28   & 295.78   & 372.7    & 135.25 \\\toprule
                                         &                    &            & \multicolumn{5}{c}{Learning Rate}                    &        \\
                      &                    & LoRC-dim & 5.00E-05 & 1.00E-05 & 5.00E-06 & 1.00E-06 & 5.00E-07 & best   \\\midrule
\multirow{3}{*}{350m} & \multirow{3}{*}{W4A8} & 0          & 25.65    & 24.38    & 24.34    & 24.55    & 24.75    & 24.34  \\
                      &                    & 8          & 25.56    & 24.3     & 24.24    & 24.45    & 24.66    & 24.24  \\
                      &                    & 16         & 25.45    & 24.39    & 24.21    & 24.39    & 24.63    & 24.21  \\\midrule
  \multirow{3}{*}{350m}                     & \multirow{3}{*}{W3A8} & 0          & 30.59    & 28.45    & 28.94    & 31.51    & 32.39    & 28.45  \\
                      &                    & 8          & 30.1     & 28.22    & 28.71    & 30.81    & 32.09    & 28.22  \\
                      &                    & 16         & 30.64    & 28.02    & 28.50     & 30.62    & 31.69    & 28.02  \\\midrule
  \multirow{3}{*}{350m}                     & \multirow{3}{*}{W2A8} & 0          & 97.40     & 177.43   & 257.61   & 668.19   & 722.19   & 97.4   \\
                      &                    & 8          & 95.79    & 139.68   & 194.36   & 437.18   & 459.92   & 95.79  \\
                      &                    & 16         & 106.51   & 137.81   & 172.93   & 400.91   & 421.59   & 106.51\\
\bottomrule
\end{tabular}
\end{adjustbox}
\end{table}


% \begin{figure}[H]
% \centering
% \includegraphics[width=0.8\textwidth]{figures/eigenvalues.png}
%  \captionof{figure}{ \small Eigenvalues of the Error matrix $E$ for W4A16}\label{fig:lorc-despription}
% \end{figure}


