\begin{abstract}
Allocation of scarce healthcare resources under limited logistical and infrastructural facilities is a major issue in the modern society. We consider the problem of allocation of healthcare resources like vaccines to people or hospital beds to patients in an ongoing manner. Our model takes into account the arrival of resources on a day-to-day basis, different categories of agents, the possible unavailability of agents on certain days, and the utility associated with each allotment as well as its variation over time. 

We propose a model where priorities for various categories are modelled in terms of utilities of agents.
We give online and offline algorithms to compute an allocation that respects eligibility of agents into different categories, and incentivizes agents not to hide their eligibility for some category. The offline algorithm gives an optimal allocation while the online algorithm gives an approximation to the optimal allocation in terms of total utility. Our algorithms are efficient, and maintain fairness among different categories of agents. Our models have applications in other areas like refugee settlement and visa allocation. We evaluate the performance of our algorithms on real and synthetic datasets. The experimental results show that the online algorithm is fast and performs better than the given theoretical bound in terms of total utility. Moreover, the experimental results confirm that our utility-based model correctly captures the priorities of categories.
\end{abstract}