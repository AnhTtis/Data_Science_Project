\section{Conclusion}\label{sec:conclusion}
We investigate the problem of dynamically allocating perishable healthcare goods to agents arriving over a period of time. We capture various constraints while allocating a scarce resource to a large population, like production constraint on the resource, infrastructure and constraints. 
While we give an offline optimal algorithm for Model $1$, getting one for Model $2$ or showing NP-hardness remains open.
%We proposed an offline algorithm that maximizes welfare that requires agents to declare their availability for the entire schedule upfront. 
We also propose an online algorithm approximating welfare that elicits information every day and makes an immediate decision. The online algorithm does not require a foresight and hence has a practical appeal.% We further run various experiments to compare the average-case performance of the proposed online algorithm against the offline algorithm. We observed 
Our experiments show that the online algorithm generates a utility roughly equal to the utility of the offline algorithm while achieving very little to no wastage. % Further, we observed that a high fraction of the perishable good is utilized and is not discarded at the end of the day in both the online and the offline algorithm.