\section{Algorithms for Model~1}\label{sec:model1}
We give a flow based polynomial-time optimal offline algorithm for Model~1 in Appendix. Here, we give an online algorithm for the same which achieves a competitive ratio of \(1+\delta\), where \(\delta\) is the discounting factor of the agents.

\subsection{Online Algorithm for Model~1}
We present an online algorithm which greedily maximizes utility on each day. We show that this algorithm indeed achieves a competitive ratio of \(1+\delta\).\\

\emph{Outline of the Algorithm:} On each day \(d_i\), starting from day \(d_1\), we construct a bipartite graph \({H}_i=({A}_i\cup C, E_i, w_i)\) where \({A}_i\) is the set of agents who are available on day \(d_i\) and are not vaccinated  earlier than day $d_i$. 
%An edge $(a_j, c_k) \in E_i$ if $a_i$ is available on day $d_i$ and belongs to the category $c_k$.
Let the weight of the edge \((a_j, c_k) \in E_i\) be \(w_i(a_j, c_k) = \alpha_j.\delta^{i-1}\). We define capacity of the category $c _k \in C$ as  \(b'_{i,k}\). In this graph, our algorithm finds a maximum weighted b-matching of size not more than the daily supply value \(s_i\). 
% This can be found in polynomial time \cite{lawler2001combinatorial}. (Line 15 can be done in polytime)

\begin{algorithm}
  \caption{Online Algorithm for Vaccine Allocation}
  \label{alg:online-greedy-m1}
\textbf{Input:} An instance \(I\) of Model~1 \\
\textbf{Output:} A matching $M:A\to(C\times D)\cup\{\varnothing\}$ \\
\begin{algorithmic}[1] %[1] enables line numbers
    \STATE Let \(D,A,C\) be the set of Days, Agents and Categories respectively.
    \STATE \(M(a_j)\gets \varnothing\) for each $a_j \in A$
    \FOR {day \(d_i\) in \(D\)}
    \STATE \({A}_i \gets \{ a_j \in A \mid a_j\) is available on \(d_i\) and \(a_j\) is not vaccinated\}
    \STATE \(E_i\gets\{(a_j,c_k) \in {A}_i \times C \mid a_j\)is eligible to be vaccinated under category $c_k$ \}
    \FOR  {\((a_j,c_k)\) in \(E_i\)}
    \STATE Let \(w_i(a_j,c_k) \gets \alpha_j\delta^{i-1}\) 
    \ENDFOR
    \STATE Construct weighted bipartite graph \({H}_i=({A}_i\cup C, E_i,w_i)\).
    \FOR {\(c_k\) in \(C\)}
    \STATE \(b'_{i,k} \gets q_{ik}\) \COMMENT {Where \(q_{ik}\) is the daily quota}
    \ENDFOR
    \STATE Find maximum weight b-matching \(M_i\)  in \(H_i\) of size at most \(s_i\). \COMMENT{Where \(s_i\) is the daily supply}
    \FOR {each edge \((a_j,c_k)\) in \(M_i\)}
    \STATE \(M(a_j) \gets (c_k,d_i)\) \COMMENT{Mark \(a_j\) as vaccinated on day \(d_i\) under category \(c_k\)}
    \ENDFOR
    \ENDFOR\\
    \RETURN \(M\)
\end{algorithmic}
\end{algorithm}




The following lemma shows that the maximum weight b-matching computed in Algorithm~\ref{alg:online-greedy-m1} is also a maximum size b-matching of size at most \(s_i\). 

\begin{lemma}\label{lemma:max-size-max-wt}
  The maximum weight b-matching in \(H_i\) of size at most \(s_i\) is also a maximum size b-matching of size at most \(s_i\). 
\end{lemma}

\begin{proof}
  We prove that applying an augmenting path in \(H_i\) increases the weight of the matching. Consider a matching $M_i$ in $H_i$ such that \(M_i\) is not of maximum size and \(|M_i|<s_i\). Let $\rho = (a_1,c_1,a_2,c_2,\cdots,a_k,c_k)$ be an $M_i$-augmenting path in $H_i$. We know that every edge incident to an agent has the same weight in $H_i$. If we apply the augmenting path $\rho$, the weight of the matching increases by the weight of the edge \((a_1,c_1)\). This proves that a maximum weight matching in \(H_i\) of size at most $s_i$ is also a maximum size b-matching of size at most $s_i$.
\end{proof}



\subsection{Charging scheme}\label{sec:analysis-model-1}
We compare the solution obtained by Algorithm~\ref{alg:online-greedy-m1} with the optimal offline solution to get the worst-case competitive ratio for Algorithm~\ref{alg:online-greedy-m1}. Let $M$ be the output of Algorithm~\ref{alg:online-greedy-m1} and $N$ be an optimal offline solution. To compare $M$ and $N$, we devise a {\em charging scheme} by which, each agent $a_p$ matched in $N$ {\em charges} a unique agent $a_q$ matched in $M$. The amount charged, referred to as the {\em charging factor} here is the ratio of utilities obtained by matching $a_p$ and $a_q$ in $M$ and $N$ respectively.

Properties of the charging scheme:
\begin{enumerate}
\item Each agent matched in $N$ charges exactly one agent matched in $M$,
\item Each agent \(a_q\) matched in $M$ is charged by at most two agents matched in $N$, with charging factors at most $1$ and $\delta$. This implies that the utility of $N$ is at most $(1+\delta)$ times the utility of $M$.
\end{enumerate}

We divide the agents matched in $N$ into two types. Type $1$ agents are those which are matched in $M$ on an earlier day compared to that in $N$. Thus $a_p \in A$ is a Type $1$ agent if $a_p$ is matched on day $d_i$ in $M$ and on day $d_j$ in $N$, such that $i<j$. The remaining agents are called Type $2$ agents.
Our charging scheme is as follows:

\begin{enumerate}
\item Each Type~$1$ agent \(a_p\) charges themselves with a charging factor $\delta$, since the utility associated with them in $N$ is at most $\delta$ times that in $M$. 
  
\item Here onwards, we consider only Type $2$ agents and discuss the charging scheme associated with them.

  Let $X_i$ be the set of Type $2$ agents matched on day $d_i$ in $N$, and let $Y_i$ be the set of agents matched on day $d_i$ in $M$. Since Algorithm~\ref{alg:online-greedy-m1} greedily finds a maximum size b-matching of size at most \(s_i\), and as each edge in the b-matching corresponds to a unique agent, we show the following lemma holds:
\end{enumerate}
  \begin{lemma} For each \(d_i \in D\),  the set \(|X_i| \le |Y_i|\).\end{lemma} 

\begin{proof}
  Since $X_i$ contains only Type $2$ agents matched in $N_i$, the agents in $X_i$ are not matched by $M$ until day $i-1$. Therefore $X_i \subseteq A_i$, where $A_i$ is defined in Algorithm~\ref{alg:online-greedy-m1}. 
  The daily quota and the daily supply available for computation of $N_i$ and $M_i$ is the same i.e. $q_{i,k}$, and $s_i$ respectively.
  %The  of categories  We know that the capacity of each category in Model~1 is precisely her daily quota. Therefore $N_i$ is a matching in $H_i$ of size at most $s_i$. Since 
  By construction, $M_i$ is a matching that matches maximum number of agents in $A_i$, up to an upper limit of $s_i$, $|X_i| \leq |Y_i|$. 
\end{proof}
  To obtain the desired competitive ratio we design an injective mapping according to which, each agent \(a_p\) in \(X_i\) can uniquely charge an agent \(a_q\) in \(Y_i\) such that \(\alpha_p \le \alpha_q\). The following lemma shows that such an injective mapping always exists.

  \begin{lemma}\label{lemma:injection-exists}
    There exists an injective mapping \(f:X_i\to Y_i\) such that if \(f(a_p) = a_q\), then \(\alpha_p \le \alpha_q\). 
  \end{lemma}

\begin{proof}
    Let $N_i$ and $M_i$ respectively be the restrictions of $N$ and $M$ to day $d_i$. We construct an auxiliary bipartite graph $G_i$ where $X_i\cup Y_i$ form one bipartition and categories form another bipartition. The edge set is $N_i\cup M_i$. Then we set the capacity of $c_k$ in $G_i$ to be $b_{i,k}=q_{i,k}$. 

    The charging scheme is as follows. Consider the symmetric difference $M_i\oplus N_i$. It is known that  $M_i\oplus N_i$ can be decomposed into edge disjoint paths and even cycles \cite{NasreR17}.
    
    Consider a component \(C\) which is an even cycle as shown in Fig~\ref{fig:cycle-charging}. Since each agent \(a_p\) in \(C\) has both \(M_i\) edge and \(N_i\) edge indecent on it, agent \(a_p\) in \(X_i\) charges her own image in \(Y_i\) with a charging factor of \(1\).

    \begin{figure}[h]
    \begin{subfigure}[b]{0.49\textwidth}
      \centering
      \scalebox{0.5}{
      \begin{tikzpicture}[scale=1]
        
        %% vertex labels
        \node at (0,-0.4) {\(a_1\)};
        \node at (1,-0.4) {\(c_1\)};
        \node at (2.4,1) {\(a_2\)};
        \node at (2.4,2) {\(c_2\)};
        \node at (1,3.4) {\(a_3\)};
        \node at (0,3.4) {\(c_3\)};
        \node at (-1.4,2) {\(a_4\)};
        \node at (-1.4,1) {\(c_4\)};
        %%% edges
        \draw[very thick,red] (0,0) -- (1,0);
        \draw[very thick,red] (2,1) -- (2,2);
        \draw[very thick,red] (1,3) -- (0,3);
        \draw[very thick,red] (-1,2) -- (-1,1);

        \draw[very thick,blue] (1,0) -- (2,1);
        \draw[very thick,blue] (2,2) -- (1,3);
        \draw[very thick,blue] (0,3) -- (-1,2);
        \draw[very thick,blue] (-1,1) -- (0,0);

        \path[->,>={Stealth[scale=1.2]}, every loop/.style={min distance=2cm, looseness=40}, thin] (0,0)  edge  [in=-140,out=-40,loop] node [label=below:{1}] {} (0,0);
        
        \path[->,>={Stealth[scale=1.2]}, every loop/.style={min distance=2cm, looseness=40}, thin] (2,1)  edge  [in=-60,out=40,loop] node [label=below:{1}] {} (2,1);
        
        \path[->,>={Stealth[scale=1.2]}, every loop/.style={min distance=2cm, looseness=40}, thin] (1,3)  edge  [in=120,out=40,loop] node [label=above:{1}] {} (1,3);
        
        \path[->,>={Stealth[scale=1.2]}, every loop/.style={min distance=2cm, looseness=40}, thin] (-1,2)  edge  [in=-140,out=130,loop] node [label=below left:{1}] {} (-1,2);

        %% vertices
        \draw[fill=black] (0,0) circle (2pt);
        \draw[fill=black] (1,0) circle (2pt);
        \draw[fill=black] (2,1) circle (2pt);
        \draw[fill=black] (2,2) circle (2pt);
        \draw[fill=black] (1,3) circle (2pt);
        \draw[fill=black] (0,3) circle (2pt);
        \draw[fill=black] (-1,2) circle (2pt);
        \draw[fill=black] (-1,1) circle (2pt);

        
        % \end{scope}
      \end{tikzpicture}}
      \caption{Agents in cycles charge themselves with a charging factor or \(1\)}
      \label{fig:cycle-charging}
    \end{subfigure}
    \begin{subfigure}[b]{0.49\textwidth}
      \centering
      \scalebox{0.5}{
      \begin{tikzpicture}
        
        %% vertex labels
        \node at (-1.3,1) {\(a_1\)};
        \node at (0.3,0) {\(c_1\)};
        \node at (1.3,1) {\(a_2\)};
        \node at (2.3,0) {\(c_2\)};
        \node at (3.3,1) {\(a_3\)};
        \node at (4.3,0) {\(c_3\)};

        \node at (6.5,0) {\(c_{k-2}\)};
        \node at (7.6,1) {\(a_{k-1}\)};
        \node at (8.6,0) {\(c_{k-1}\)};
        \node at (9.3,1) {\(a_k\)};
        %%% edges
        \draw[very thick,red] (0,0) -- (1,1);
        \draw[very thick,red] (2,0) -- (3,1);
        \draw[very thick,red] (6,0) -- (7,1);
        \draw[very thick,red] (8,0) -- (9,1);
        
        \draw[very thick,blue] (1,1) -- (2,0);
        \draw[very thick,blue] (0,0) -- (-1,1);
        \draw[very thick,blue] (3,1) -- (4,0);
        \draw[very thick,blue] (7,1) -- (8,0);

        \path[->,>={Stealth[scale=1.2]}, every loop/.style={min distance=2cm, looseness=40}, thin] (1,1)  edge  [in=135,out=45,loop] node [label=below:{1}] {} (1,1);
        \path[->,>={Stealth[scale=1.2]}, every loop/.style={min distance=2cm, looseness=40}, thin] (3,1)  edge  [in=135,out=45,loop] node [label=below:{1}] {} (3,1);
        \path[->,>={Stealth[scale=1.2]}, every loop/.style={min distance=2cm, looseness=40}, thin] (7,1)  edge  [in=135,out=45,loop] node [label=below:{1}] {} (7,1);

        \path[->,>={Stealth[scale=1.3]}, thin] (9,1)  edge  [in=45,out=135] node [label=above right:{\({\alpha_k}/{\alpha_1}\)}] {} (-1,1);

        %% vertices
        \draw[fill=black] (-1,1) circle (2pt);
        \draw[fill=black] (0,0) circle (2pt);
        \draw[fill=black] (1,1) circle (2pt);
        \draw[fill=black] (2,0) circle (2pt);
        \draw[fill=black] (3,1) circle (2pt);
        \draw[fill=black] (4,0) circle (2pt);
        \draw[fill=black] (4.5,0.5) circle (1pt);
        \draw[fill=black] (5,0.5) circle (1pt);
        \draw[fill=black] (5.5,0.5) circle (1pt);
        \draw[fill=black] (6,0) circle (2pt);
        \draw[fill=black] (7,1) circle (2pt);
        \draw[fill=black] (8,0) circle (2pt);
        \draw[fill=black] (9,1) circle (2pt);
        
      \end{tikzpicture}}
      \caption{Agents who are matched in both \(N_i\) and \(M_i\) charge themselves. Agent~\(a_k\) charges \(a_1\) with a factor or \({\alpha_k}/{\alpha_1}\). Red edges represent \(N_i\) and blue edges represent \(M_i\)}
      \label{fig:path-charging}
    
    \end{subfigure}
    \caption{Charging schemes}
    \end{figure}

    Now, Consider a component which is a path \(\rho\). There are two cases.
    
    \begin{enumerate}
    \item {\em Case 1: The path $\rho$ has an even length: }   If $\rho$ starts and ends at a category node, then each agent along the path is matched  in both \(N_i\) and \(M_i\). Hence, all such agents can charge themselves with a charging factor of \(1\).
    Suppose $\rho$ starts and ends at an agent as shown in Fig~\ref{fig:path-charging} i.e. \(\rho=(a_1,c_1,a_2,c_2,\cdots,a_{k-1},c_{k-1},a_k)\). Let \(a_1\) be matched in \(M_i\) and \(a_k\) is matched in \(N_i\). Then, \(\alpha_1\) must be greater than or equal to \(\alpha_k\). Otherwise from Lemma~\ref{lemma:max-size-max-wt}, \(M_i \oplus \rho\) is a matching of higher weight - which contradicts the fact that \(M_i\) is the maximum weight matching. Now, every agent in \(\rho\) except \(a_1\) and \(a_k\) charge themselves with a charging factor of \(1\) and \(a_k\) charges \(a_1\) with a charging factor of \({\alpha_k}/{\alpha_1}\).
    \item {\em Case 2: The path $\rho$ has an odd length:} Then either $\rho$ begins and ends with an $M_i$ edge or with an $N_i$ edge. If $\rho$ starts and ends with an $M_i$ edge, then every agent along the path who is matched in \(N_i\) is also matched in \(M_i\). Therefore all the agents on $\rho$ charge themselves.

Consider the case when $\rho$ starts with an \(N_i\) edge. Since $c_k$ is an end-point of $\rho$ with an $N_i$-edge, $c_k$ must have more agents matched to it in $N_i$ than that in $M_i$. So $c_k$ cannot be saturated in $M_i$.


%Now, depending on the saturation of $c_k$ in $M_i$ we have two subcases:
    
   % \emph{Subcase 1:} Suppose $c_k$ is saturated in $M_i$. That is, the number of \(M_i\) edges incident on \(c_k\) is equal to its capacity \(b_{i,k}\). But we know that the number of \(N_i\) edges incident on \(c_k\) is at most \(b_{i,k}\). This contradicts the fact that $c_k$ is an end-point of a path in $M_i \oplus N_i$. Therefore \emph{subcase 1} is not possible.
    As \(M_i\) is a maximum size matching [\ref{lemma:max-size-max-wt}], we cannot augment $M_i$ to $M_i\oplus \rho$ in $G_i$ even though both endpoints are unsaturated. This can happen only because the daily supply is met. That is $|M_i| = s_i$.  As $a_1$ is vaccinated in category $c_1$ in $N_i$, we claim that the weight $w(a_1, c_1)$ is less than every other edge in $M_i$. This is because if there exists an edge $e \in M_i$ such that $w(e) < w(a_1, c_1)$, we can remove the edge $e$ from $M_i$ and apply the augmenting path $\rho$ to get a matching with a higher weight, which is a contradiction. Therefore, as $w(a_1, c_1)$ is less than every other edge in \(M_i\), agent \(a_1\) can safely charge any agent \(a_q\) who is matched in \(M_i\). Since \(|M_i| \ge |N_i|\), we are guaranteed to have sufficient agents in \(N_i\) for charging.  
    \end{enumerate}
  \end{proof}
   

  % \item {\em Case 2: $|X_i|=|Y_i|+z, z>0$: } Let $N_i$ and $M_i$ respectively be the restrictions of $N$ and $M$ to day $d_i$. We construct an auxiliary bipartite graph $G_i$ where $X_i\cup Y_i$ form one bipartition and categories form another bipartition. The edge set is $N_i\cup M_i$. For a category $c_k$, let $n_{j,k}$ and $m_{j,k}$ be the number of agents matched in $N$ and $M$ respectively, under category $c_k$ on day $d_j$. Then we set the quota of $c_k$ in $G_i$ to be $b_{i,k}=\min\{q_{i,k}, \max\{q_k-\sum_{j=1}^{i-1}n_{j,k}, q_k-\sum_{j=1}^{i-1} m_{j,k}\}\}$. This is the maximum of the quotas of $c_k$ that were available for computation of $N_i$ and $M_i$ respectively.
  
  %   The charging scheme is given by the following. Consider the symmetric difference $M_i\oplus N_i$. Since $|N_i|=|M_i|+z$, there are exactly $z$ edge-disjoint alternating paths in $M_i\oplus N_i$ that start and end with an edge of $N$ \cite{NasreR17}. Let $\rho=\langle a_1,c_1,a_2,\ldots, a_k, c_k\rangle$ be one such path. Then $a_2,\ldots,a_{k-1}$ are matched in both $M_i$ and $N_i$, so they charge themselves with a charging factor of $1$. The agent $a_1$ charges $a_k$ with charging factor of $1$. It remains to decide whom $a_k$ charges.
  
  %   Since $\rho$ terminates at $c_k$ with an $N_i$-edge, the number of agents matched to $c_k$ in $N_i$ is more than those matched to $c_k$ in $M_i$. In Lemma~\ref{lem:saturated}, we show that this can happen only because of exhaustion of $q_k$ in Algorithm~\ref{alg:online-greedy-m2} on or before day $d_i$. %since $|M_i|, |N_i|\leq s_i$ and $M_i$ is a maximum matching on day $d_i$. 
  %   So agent $a_k$ can charge one of the agents matched to $c_k$ in $M$ on an earlier day, with charging factor $\delta$.


% \begin{lemma}\label{lem:saturated}
%   If node $c_k$ is an end-point of a path $\rho$ in $G_i$, then $q_k$ is exhausted in Algorithm \ref{alg:online-greedy-m2} on or before day $d_i$. 
% \end{lemma}
% \begin{proof}
%   Suppose $c_k$ be an endpoint of $\rho$ in $G_i$. The number of agents matched to $c_k$ in $N_i$ is more than those matched to $c_k$ in $M_i$. We know that the daily supply $s_i$ of the day $d_i$ is an upperbound for both $|M_i|$ and $|N_i|$. Since $|N_i| = |M_i| + z$, we have $|M_i| < s_i$. From Algorithm \ref{alg:online-greedy-m2} we know that $M_i$ is a maximum-size b-matching in $H_i$ of size at most \(s_i\). If the capacity of $c_k$ is not saturated in $H_i$, then we can augment the path $\rho$ contradicting the maximality of $M_i$.  Since $c_k$ has more edges of $M_i$ than $N_i$ incident to it, from the definition of $b_{i,k}$, category $c_k$ must have exhausted the overall quota $q_k$ in Algorithm \ref{alg:online-greedy-m2} on or before day $d_i$. 
% \end{proof}


% To show that Algorithm~\ref{alg:online-greedy-m1} achieves the desired competitive ratio, we introduce a \emph{charging scheme} - a mapping technique to compare the computed allocation with an optimal allocation. We say that an agent \(a_j\) who is vaccinated by an optimal allocation \emph{charges} an agent \(a_{j'}\) with a \emph{charging factor}. To be precise, we show that the set \(S_{OPT}\) of agents who are vaccinated by an optimal allocation can be partitioned into two parts \(H_1 \text{ and } H_2\) such that any agent \(a_{j'}\) who is vaccinated by the online allocation (computed by Algorithm~\ref{alg:online-greedy-m1}) is charged by at most one agent from each part. Furthermore, we show that each agent in \(H_1\) charges exactly one agent, with a charging factor of \(\delta_i\). And each agent \(a_j\) in \(H_2\) charges exactly one agent with a charging factor less than or equal to \(1\).

% Let \(ON\) be an allocation computed by Algorithm~\ref{alg:online-greedy-m1} on an instance \(I\), and let \(OPT\) be an optimal allocation on the same instance \(I\). Recall that an allocation \(M\) is of the form \(M:A\to(C\times D)\cup\{\varnothing\}\). Let \(S_{OPT}\) and \(S_{ON}\) be the sets of agents who got vaccinated by \(OPT\) and \(ON\) respectively. Now, \(H_1 \subseteq S_{OPT} \) can be described as the set of agents who are vaccinated by \(ON\) at least one day before they are vaccinated by \(OPT\) That is,
% \begin{align*}
%   H_1&=\{a\in S_{OPT} \mid OPT(a)=(c_i,d_j),\  ON(a)=(c_k,d_\ell), d_\ell<d_j\}
% \end{align*}

% For every agent \(a_i\in H_1\), \(a_i\) charges herself i.e. \(a_i \in S_{ON}\). Therefore, by definition, no two agents in \(H_1\) charge the same agent, and every agent in \(H_1\) charges exactly one agent in \(S_{ON}\). Since the utility associated with vaccinating $a_i$ in \(OPT\)is at most $\delta_i$ times that in \(ON\),the charging factor is $\delta_i$ in this case.

% Let \(S'_{OPT} = S_{OPT}\setminus H_1\). %be the remaining agents from \(S_{OPT}\) who are not in \(H_1\).

% To construct \(H_2\) and to introduce a charging scheme for \(H_2\), we first partition the agents in \(S'_{OPT}\) based on the day on which they got vaccinated by \(OPT\). That is, \[ S' = X_1\uplus X_2 \uplus \cdots \uplus X_{r=|D|} \]where,
% \begin{align*}
%   X_i &= \{a\in S' \mid a \text{ is vaccinated on day } d_i \text{ by } OPT  \} &&\forall {i\in\{1,2,\cdots,|D|\}}\\
% \end{align*}
% Similarly, let\[ S'_{ON} = Y_1\uplus Y_2 \uplus \cdots \uplus Y_{r=|D|} \] where,
% \begin{align*}
%   Y_i &= \{a\in S'_{ON} \mid a \text{ is vaccinated on day } d_i \text{ by } ON  \} &&\forall {i\in\{1,2,\cdots,|D|\}}\\
% \end{align*}

% Though Algorithm~\ref{alg:online-greedy-m1} finds maximum weight b-matching, the following lemma shows that the computed matching is also a maximum size matching.

% \begin{lemma}\label{lemma:max-size-max-wt}
%   The maximum weight b-matching in \((G_i,b^i_{ON})\) of size at most \(s_i\) is also a maximum size b-matching of size at most \(s_i\). 
% \end{lemma}
% \begin{proof}
%   We prove that applying a augmenting path increases the weight of that matching in \((G_i,b^i_{ON})\). Suppose $M_i$ be a matching of $(G_i, b^i_{ON})$ and $p$ be an $M_i$-augmenting path between the vertices $a$ and $c$ in $(G_i, b^i_{ON})$. We know that every edge incident to an agent has the same weight in $(G_i, b^i_{ON})$. When we apply the augmenting path $p$ the weight of the matching increases by the weight of the edge incident to $a$ in $p$. This proves that a maximum weight matching of \((G_i,b^i_{ON})\) of size at most $s_i$ is also a maximum size matching of \((G_i,b^i_{ON})\) of size at most $s_i$.
% \end{proof}

% Since Algorithm~\ref{alg:online-greedy-m1} greedily finds a maximum size b-matching of size at most \(s_i\), and since each edge in the b-matching corresponds to a unique agent, we note the following property
% \begin{remark} For each \(d_i \in D\),  the set \(|X_i| \le |Y_i|\).
% \end{remark}

% Now, we propose a charging scheme for the agents in the set \(H_2\). For each \(X_i\) and \(Y_i\), we give an injective mapping \(f\) from \(X_i\) to \(Y_i\) such that if \(f(a_i)=a_j\) then \(\delta_i\le \delta_j\).


% Now, it is apparent that an allocation by \(ON\) for \(A_i\) on day \(d_i\) is a valid matching in \((G_i,b^{ON}_i)\). Similarly, an allocation by \(OPT\) can be viewed as a matching in \((G_i,b^{OPT}_i)\). Let us call these matchings as \(M_i^{ON}\) and \(M_i^{OPT}\) respectively.
% \begin{align*}
%   M_i^{ON} &= \{(a_j,c_k) \mid a_j\in A_i \text{ is vaccinated on day } d_i \text{ under } c_k \text{ by } ON\}\\
%   M_i^{OPT} &= \{(a_j,c_k) \mid a_j\in A_i \text{ is vaccinated on day } d_i \text{ under } c_k \text{ by } OPT\}\\
% \end{align*}

% It is clear that \(|M_i^{OPT}| = |X_i|\) and \(|M_i^{ON}| = |Y_i|\).

% Let \(b_i:C \to \mathbb{Z}_{\ge 0}\) be defined as
% \begin{align*}
%   b_i(c_k) &= \max(b^{ON}_i(c_k), b^{OPT}_i(c_k))
% \end{align*}

% Clearly, both \(M_i^{OPT}\) and \(M_i^{ON}\) are valid b-matchings in \((G_i,b_i)\).

% Next we reduce the instance of a b-Matching problem to 1-matching \cite{gabow1983efficient}\cite{marsh1979matching} problem. Let \(G_i^{*}\) be the instance of $1$-matching problem corresponding to \((G_i,b_i)\). Both $N_i$ and $M_i$ are matchings in $G^{*}_i$ corresponding the matchings $M_i^{OPT}$ and $M_i^{ON}$ in $(G_i, b_i)$.



% % {\color{red} To be filled later - define both b-matchings. Decompose it to normal matchings. Say things about them....}

% \begin{lemma}
%   Bla!
% \end{lemma}
% \begin{proof}
%   Bhoom!
% \end{proof}

\emph{Order of charging among Type~2 agents:} First, every agent who has both $M_i$ and $N_i$ edges indecent on it, charges herself. Next every agent who is an end-point of an even-length path charges the agent represented by the other end-point. The rest of the agents are end-points of an odd-length path matched in $N_i$. We proved that the edges incident on these agents have a weight smaller than every edge in $M_i$. They can charge any agent of $M_i$ who has not been charged yet by any agent of $N_i$, as stated above.
%Finally the agents who are part of an odd-length path in $N_i \oplus M_i$, 
%\textbf{charges any agent who is vaccinated by $M_i$ and is not charged by anyone yet.}\textcolor{red}{This last step is not very concrete. Why should such an agent exist?} 


\begin{proof}[of Theorem~\ref{thm:gen-on-same-utility}~(i)]
  Let $a_q$ be an agent who is vaccinated by the online matching \(M\) on day $i$. Then $a_q$ can be charged by at most two agents matched in \(N\). Suppose $a_q$ is vaccinated by the optimal matching \(N\) on some day $i' >i$. Assume that the agent $a_p$ of type 2 who also charges $a_q$. If the priority factor of $a_q$ and $a_p$ are $\alpha_q$ and $\alpha_p$ respectively, then   
  \begin{align*}
    &\frac{\alpha_p.\delta^i + \alpha_q.\delta^{i'}}{\alpha_q.\delta^i}\  =\  \left(\frac{\alpha_p}{\alpha_q}\right)^i + {\delta^{i'-i}} \leq\  {1 + \delta}. 
  \end{align*}
The last inequality follows as \(0 < \alpha_p \leq \alpha_q < 1,  \text{ and }i' > i\).  Therefore the utility obtained by $a_p$ and $a_q$ in $M_i$ is atmost $1 + \delta$ times the the utility of $a_q$ in $M_i$. Therefore the competitive ratio of Algorithm~\ref{alg:online-greedy-m1} is at most ${1 + \delta}$.  
\end{proof}

In the Appendix, we show a tight example which achieves this compititve ratio.


Since the daily supply of day \(d_1\) is \(1\), vaccinating \(a_1\) maximizes the utility gained on the first day. Hence there exists a run of Algorithm~\ref{alg:online-greedy-m1} where \(a_1\) is  vaccinated under category \(c_1\) on day \(d_1\). In this run, agent \(a_2\) cannot be vaccinated on day \(d_2\) as she is unavailable on that day. Hence, total utility gained by the online allocation is \(\alpha_1\). Whereas in a optimal allocation scheme all the agents can be vaccinated. We vaccinate agent \(a_2\) on day \(d_1\) under category \(c_2\), agent \(a_1\) on day \(d_2\) under category \(c_1\). This sums to a total utility of \(\alpha_1+\alpha_1\delta\). Therefore the competitive ratio is $\frac{\alpha_1+\alpha_1\delta}{\alpha_1} = 1 + \delta$. 

% [Done][Check Once] Before: max size max-wt

% Next  find an inj mapping b/w Xi and Yi with the following property:
% [Done] Prperty: If a_i \in X_i maps to aj in Yi then delta_i must be smaller than delta_j

% Following lemmas : We can always find such a mapping.

% lemma~1: b-matching is normal matching and vice versa

% lemma~2: charging scheme:
% proof : symmetric difference bla... bla... case analysis

% main thm: 1+max delta aprox
% proof.

% Tight example.

\subsection{Tight example for the Online Algorithm}
\begin{figure}[!h]
  \centering
  \scalebox{1}{
  \begin{tikzpicture}[
    on_node/.style={circle, draw=red!30, fill=red!20, ultra thin, minimum size=4mm},
    opt_node/.style={circle, draw=green!30, fill=green!20, ultra thin, minimum size=4mm},
    null_node/.style={circle, draw=white!10, fill=white!10, ultra thin, minimum size=4mm},
    node distance=0.2cm,->,>=stealth',
    ]
    %Nodes
    \node (a1) {\(a_1\)};

    \node[on_node] (a1d1) [right=of a1] {$\alpha_1$};
    \node[opt_node] (a1d2) [right=1cm of a1d1] {\(\alpha_1\delta\)};

    \node[opt_node] (a3d1) [below=of a1d1] {$\alpha_1$}; 
    \node[null_node] (a3d2) [below=of a1d2] {-};

    \node (a3) [left=of a3d1] {\(a_2\)};

    \node (d1) [above=of a1d1] {\(d_1\)};
    \node (d2) [above=of a1d2] {\(d_2\)};

    % Edges
    \draw[->] (a1d2.west) -- (a1d1.east);
    \draw[->] (a3d1) to [out=30,in=310] (a1d1);
  \end{tikzpicture}}
  \caption[Tight Example]{A tight example with competitive ratio \(1 + \delta \). Online allocation indicated in red, Optimal allocation indicated in green and arrows indicate charging} 
  \label{fig:tight-model-1}
\end{figure}

The following example shows that the competitive ratio of Algorithm~\ref{alg:online-greedy-m1} is tight. Let the set of agents \(A=\{a_1, a_2\}\) and categories \(C=\{c_1, c_2\}\). Agent \(a_1\) is eligible under \(\{c_1,c_2\}\) and agent \(a_2\) is eligible only under \(\{c_2\}\).  The daily~supply: \(s_{1}=1 \text{ and } s_{2}=1\). The daily quota of each category on each day is set to \(1\). The  priority factor for both the agents  is~\(\alpha_1\). Assume that \(a_1\) is available on both the days whereas the agent \(a_2\) is available only on the first day.
Figure~\ref{fig:tight-model-1} depicts this example. 
