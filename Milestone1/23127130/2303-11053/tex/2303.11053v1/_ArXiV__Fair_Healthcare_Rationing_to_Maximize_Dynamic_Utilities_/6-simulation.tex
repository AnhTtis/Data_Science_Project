\section{Experimental Evaluation}\label{sec:simulation}
In Section~\ref{sec:model1} we prove worst-case guarantees for the online algorithm. We also give a tight example instance achieving a competitive ratio of \(1+2\delta\). Here, we experimentally evaluate the  performance of the online algorithm and compare it with the worst-case guarantees on a real-life dataset. For finding the optimal allocation that maximizes utility, we solve the  networkflow linear program  with the additional constraint for overall quota $\sum_{i\in A, k\in D} x_{ijk}\leq q_j\quad \forall c_j\in C$. This LP is described in the Appendix. The code and datasets for the experiments can be found at~\cite{exp-github}


\subsection{Methodology}
All experiments run on a 64-bit Ubuntu 20.04 desktop of 2.10GHz * 4 Intel Core i3 CPU with 8GB memory. 

The proposed online approximation algorithm runs in polynomial time. In contrast, the optimal offline algorithm solves an integer linear program which might take exponential time depending on the integrality of the polytope. We relax the integrality constraints to achieve an upper-bound on the optimal allocation. For comparing the performance of the online Algorithm~\ref{alg:online-greedy-m1} and the offline Algorithm, we use vaccination data of 24 hospitals in Chennai, India for the month of May 2022. We use small data-sets with varying instance sizes for evaluating the running times of the algorithms. We use large data-sets of smaller instance sizes for evaluating competitive ratios.      

All the programs used for the simulation are written in Python language. For solving LP, ILP, and LPR, we use the general mathematical programming solver COIN-OR Branch and Cut solver MILP (Version:~2.10.3)\cite{CBC} on PuLP (Version~2.6) framework\cite{pulp}. When measuring the running time, we consider the time taken to solve the LP. 

\subsection{Datasets}
Our dataset can be divided into two parts.

\textit{Supply:} We consider vaccination data of twenty four hospitals of Chennai, India for the month of May 2022. This data is obtained from the official COVID portal of India using the API's provided. The data-set consists of details such as daily vaccination availability, type of vaccines, age limit, hospital ID, hospital zip code, etc. for each hospital. 

\textit{Demand:} Using the Google Maps API \cite{google-maps-api}, we consider the road network for these 24 hospitals in our data-set. From this data we construct a complete graph with hospitals as vertices and edge weights as the shortest distance between any two hospitals. For each hospital \(h\in H\), we consider the cluster \(C(h)\) as the set of hospitals which are at most five kilo meters away from \(h\). We consider these clusters as our categories. Now, we consider 10000 agents who are to be vaccinated. For each agent \(a\), we pick a hospital \(h\) uniformly at random. The agent \(a\) belongs to every hospital in the cluster \(C(h)\). Each agent's availability over 30 days is independently sampled from the uniform distribution. Now, we consider the age wise population distribution of the city. For each agent we assign an age sampled from this distribution. Now, we partition the set of agents as agents of age 18-45years, 45-60years and 60+. We assign \(\alpha\)-values \(0.96,0.97 \text{ and } 0.99\) respectively. We also consider the same dataset with \(\alpha\)-values \(0.1,0.5 \text{ and } 0.9\) respectively. We set the discounting factor \(\delta\) to be 0.95.

For analyzing the running time of our algorithms, we use synthetically generated datasets with varying number of instance sizes ranging from 100 agents to 20000 agents. Each agent's availability and categories are chosen randomly from a uniform distribution.  


\subsection{Results and Discussions}

We show that the online algorithm runs significantly faster than the offline algorithm while achieving almost similar results. We give a detailed emperical evaluation of the running times in the Appendix. 

\ \\
To compare the performance of the online Algorithm~\ref{alg:online-greedy-m1} against the offline algorithm we define a notion of \textit{remaining fraction of un-vaccinated agents}. That is, on a given day \(d_i\), we take the set of agents \(P_{d_i}\) who satisfy both of the following conditions:
\begin{enumerate}
\item Agent \(a\) is available on some day \(d_j\) on or before day \(d_i\).
\item Agent \(a\) belongs to some hospital \(h\) and \(h\) has non-zero capacity on day \(d_j\)
\end{enumerate}

\(P_{d_i}\) is the set of agents who could have been vaccinated without violating any constraints. Let \(\gamma_{i} = \lvert P_{d_i} \rvert\).

Let \(V_{d_i}\) be the set of agents who are vaccinated by the algorithm on or before day \(d_i\).  Let \(\eta_i = \lvert V_{d_i} \rvert\). Now, \(1-\eta_i / \gamma_i \) represents the fraction of unvaccinated agents. In Figure~\ref{fig:final_compare} we compare the age-wise  \(1-\eta_i / \gamma_i \) of both of our online and offline algorithms. We note that the vaccination priorities given to vulnerable groups by the online approximation algorithm is very close to that of the offline optimal algorithm. In both the algorithms, By the end of day 2, 50\% of \(1-\eta_i / \gamma_i \) was achieved for agents of 60+ age group. By the end of day 8, only 10\% of the most vulnerable group remained unvaccinated. 


\begin{figure}
    \centering
\includegraphics[width=0.9\textwidth]{final_compare}
    \caption{The \(1-\eta_i / \gamma_i \) value achieved by the online algorithm is very similar to that of the offline algorithm across age groups. Both algorithm vaccinate achieves vaccinate 90\% of the most vulnerable group within 8 days. }
    \label{fig:final_compare}
  \end{figure}













\subsection{Running Time Analysis}
In Table~\ref{table:final-table} we compare the performance of the online algorithm and the offline algorithm against the same dataset. We consider alpha values \((0.96,0.97,0.99)\) and \((0.1,0.5,0.9)\). In both the cases, the online algorithm vaccinates almost the same number of agents as that of the offline while algorithm achieving similar total utility. The competitive ratio is \(0.99\). The online algorithm runs significantly faster than the offline algorithm. 

\begin{minipage}{\textwidth}
\begin{minipage}{0.48\textwidth}
\scalebox{0.7}{
\begin{tabular}{|c|ll|ll|}
\hline
                                                                            & \multicolumn{2}{c|}{\begin{tabular}[c]{@{}c@{}}Online\\ Algorithm\end{tabular}} & \multicolumn{2}{c|}{\begin{tabular}[c]{@{}c@{}}Offline\\ Algorithm\end{tabular}} \\ \hline
\(\alpha\) value                                                            & \multicolumn{1}{l|}{\(\vec{\alpha_1}\)}           & \(\vec{\alpha_2}\)          & \multicolumn{1}{l|}{\(\vec{\alpha_1}\)}           & \(\vec{\alpha_2}\)           \\ \hline
\(\delta\)                                                                  & \multicolumn{1}{l|}{0.95}                         & 0.95                        & \multicolumn{1}{l|}{0.95}                         & 0.95                         \\ \hline
\begin{tabular}[c]{@{}c@{}}Running time \\ (in sec)\end{tabular}            & \multicolumn{1}{l|}{319.04}                       & 336.55                      & \multicolumn{1}{l|}{888.90}                       & 806.65                       \\ \hline
\begin{tabular}[c]{@{}c@{}}Total \\ no. of agents\\ vaccinated\end{tabular} & \multicolumn{1}{l|}{7154}                         & 7145                        & \multicolumn{1}{l|}{7192}                         & 7192                         \\ \hline
Total Utility                                                               & \multicolumn{1}{l|}{3567.95}                      & 1550.23                     & \multicolumn{1}{l|}{3580.68}                      & 1573.95                      \\ \hline
\end{tabular}}
\captionof{table}{ The vector \(\vec{\alpha_1} = \)  \((0.96, 0.97, 0.99)\) and vector \(\vec{\alpha_2} = \)  \((0.1, 0.5, 0.9)\) represent the \(alpha\) values for the three age groups . The average competitive ratio is \(0.99\). The average running time of the online and the offline algorithms are 327.79 seconds and 847.77 seconds respectively.  }
\label{table:final-table}
\end{minipage}\hfill
\begin{minipage}[t]{0.48\textwidth}
  \pgfplotstableread[row sep=\\,col sep=&]{
    size   & Off                & On                 \\
    100    & 0.019134283065796  & 0.019458115100861  \\
    500    & 0.138595390319824  & 0.116458749771118  \\
    1000   & 0.228429079055786  & 0.135910940170288  \\
    2000   & 0.592561912536621  & 0.356951093673706  \\
    5000   & 1.8075364112854    & 0.909572839736939  \\
    10000  & 4.43315873146057   & 1.69299368858337   \\
    20000  & 13.7637612819672   &  5.67742729187012  \\
    }\mydata
 \scalebox{0.7}{
	\begin{tikzpicture}
		\begin{axis}[
		            % title = Running time comparision,
		            enlarge x limits=0,
		            enlarge y limits=0.1,
		            xbar,
		            bar width=6pt,
		            symbolic y coords={100,500,1000,2000,5000,10000,20000},
		            ytick=data,
		            nodes near coords,
		            every node near coord/.append style={font=\scriptsize},
		            xmax=16,
		            legend style={at={(0.5,1.05), font=\scriptsize},
		            anchor=south,legend columns=-1},
		            ylabel={Size of Instance (number of agents)},
		            xlabel={Running time (in 50 sec)},
		            label style={font=\small},
		            tick label style={font=\small},
		        ]
		        \addplot[fill=blue] table[x=On,y=size]{\mydata};
		        \addplot[fill=red] table[x=Off,y=size]{\mydata};
		        \legend{Online Algorithm, Offline Algorithm}
		    \end{axis}  
	\end{tikzpicture}}
	\captionof{figure}{Time taken by offline and online algorithms (on synthetic datasets) vs instance size}
	\label{fig:time_graph}
\end{minipage}
\end{minipage}

  Comparing the running time of Algorithm~\ref{alg:online-greedy-m1} and the offline algorithm,  Figure~\ref{fig:time_graph} shows that the online algorithm runs significantly faster than the offline algorithm for all input sizes.

\subsection{Performance Analysis}
In Figure~\ref{fig:age_0_compare_alphas}, we plot the number of agents of age group 18-45 getting vaccinated by the online algorithm~\ref{alg:online-greedy-m1} on each day for \(alpha\) values \(0.96\)  and \(0.1\). It is clear that the vaccination follows almost identical pattern as long as the order of \(alpha\) values remain the same. Figure~\ref{fig:age_0_compare_alphas_offline} shows similar results for the optimal offline algorithm. The independence on cardinal values shows that the algorithm is practically useful as ordering the vulnerable groups is much more feasible than assigning a particular value.  Similar plots for other age groups are given in the appendix. 

  
  \begin{figure}
    \centering
    \includegraphics[width=1\textwidth]{age_2_compare_alphas}
    \caption{Number of agents in the 60+ age group vaccinated by the online algorithm for \(alpha\)-values \(0.96\) and \(0.1\) respectively.}
    \label{fig:age_0_compare_alphas}
  \end{figure}

  \begin{figure}
    \centering
    \includegraphics[width=1\textwidth]{age_2_compare_alphas_offline}
    \caption{Number of agents in the 60+ age group vaccinated by the offline algorithm for \(alpha\)-values \(0.96\) and \(0.1\) respectively.}
    \label{fig:age_0_compare_alphas_offline}
  \end{figure}

In Figure~\ref{fig:age_1_compare_alphas}, we plot the number of agents of age group 45-60 getting vaccinated by the online algorithm 1 on each day for alpha values 0.97 and 0.5. It is clear that the vaccination follows almost identical pattern as long as the order of alpha values remain the same. Figure~\ref{fig:age_1_compare_alphas_offline} shows similar results for the optimal offline algorithm. Figure~\ref{fig:age_2_compare_alphas} and Figure~\ref{fig:age_2_compare_alphas_offline} plot similar results for the 60+ age group population. We note that in both online and the offline algorithm,  allocations of vaccines for the age group 60+ are higher in the initial days and decreases with days. Most of the agents from this group are vaccinated by the end of 10th day. 



  \begin{figure}
    \centering
    \includegraphics[width=1\textwidth]{age_1_compare_alphas}
    \caption{Number of agents in the 45-60 age group vaccinated by the online algorithm for \(alpha\)-values \(0.97\) and \(0.5\) respectively.}
    \label{fig:age_1_compare_alphas}
  \end{figure}

  \begin{figure}
    \centering
    \includegraphics[width=1\textwidth]{age_1_compare_alphas_offline}
    \caption{Number of agents in the 45-60 age group vaccinated by the offline algorithm for \(alpha\)-values \(0.97\) and \(0.5\) respectively.}
    \label{fig:age_1_compare_alphas_offline}
  \end{figure}
  
    \begin{figure}
    \centering
    \includegraphics[width=1\textwidth]{age_2_compare_alphas}
    \caption{Number of agents in the 60+ age group vaccinated by the online algorithm for \(alpha\)-values \(0.99\) and \(0.9\) respectively.}
    \label{fig:age_2_compare_alphas}
  \end{figure}

  \begin{figure}
    \centering
    \includegraphics[width=1\textwidth]{age_2_compare_alphas_offline}
    \caption{Number of agents in the 60+ age group vaccinated by the offline algorithm for \(alpha\)-values \(0.99\) and \(0.9\) respectively.}
    \label{fig:age_2_compare_alphas_offline}
  \end{figure}
