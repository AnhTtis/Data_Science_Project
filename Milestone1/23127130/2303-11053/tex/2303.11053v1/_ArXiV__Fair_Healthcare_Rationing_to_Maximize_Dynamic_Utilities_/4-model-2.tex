\section{Online Algorithm for Model~$2$}\label{sec:onlineapprox}
We present an online algorithm which greedily maximizes utility on each day. We assume that the discounting factor of the agents is \(\delta\). Moreover each agent $a_k$ has a priority factor $\alpha_k$. Let $\alpha_{\max} = \max_i\{\alpha_i \mid \alpha_i \text{ is the priority factor of agent $a_i$}\}$ and $\alpha_{\min} = \min_i\{\alpha_i \mid \alpha_i \text{ is the priority factor of agent $a_i$}\}$. We show that this algorithm indeed achieves a competitive ratio of \(1+\delta + \frac{\alpha_{\max}}{\alpha_{\min}}\delta\). 

\emph{Outline of the Algorithm:} On each day \(d_i\), starting from day \(d_1\), we construct a bipartite graph \({H}_i=({A}_i\cup C, E_i, w_i)\) where set \({A}_i\) is the set of agents who are available on day \(d_i\) and are not vaccinated earlier than day $d_i$.  Let the weight of the
edge $(a_j , c_k) \in E_i$ be $w_i(a_j , c_k) = \alpha_j .\delta^{i-1}$. Let \(b'_{i,k}\) represent the capacity of \(c_k\in C\) in \(H_i\). In this graph, our algorithm finds a maximum weighted b-matching of size not more than the daily supply value \(s_i\). This can be found in polynomial time \cite{lawler2001combinatorial}. Lemma~\ref{lemma:max-size-max-wt} proves that the maximum weight b-matching is also a maximum cardinality b-matching of $H_i$.  

\begin{algorithm}
\caption{Online Algorithm for Vaccine Allocation}
\label{alg:online-greedy-m2}
\hspace*{\algorithmicindent} \textbf{Input:} An instance \(I\) of Model~2 \\
\hspace*{\algorithmicindent} \textbf{Output:} An allocation \(M:A\to(C\times D)\cup\{\varnothing\}\)

\begin{algorithmic}[1]
  \STATE Let \(D,A,C\) be the set of Days, Agents and Categories respectively.
  \STATE \(M(a_j)\gets \varnothing\) for each agent $a_j \in A$
  \STATE \(r_k \gets q_k\) for each category $c_k \in C$
  \FOR {day \(d_i\) in \(D\)}
  \STATE ${A}_i \gets \{ a_j \in A \mid a_j$ is available on $d_i$ and $a_j$ is not vaccinated\}
  \STATE $E_i=\{(a_j,c_k) \in {A}_i \times C \mid a_j$ is eligible to be vaccinated under category $c_k \}$
  \STATE Construct bipartite graph \({H}_i=({A}_i\cup C, E_i)\).
  \FOR {\(c_k\) in \(C\)}
  \STATE \(b'_{i,k} \gets min(q_{ik}, r_k)\) \COMMENT{Capacity for each \(c_k\) in \(H_i\)}
  \ENDFOR
    \STATE Find maximum weight b-matching \(N_i\)  in \(H_i\) of size at most \(s_i\).
  \FOR {each edge \((a_j,c_k)\) in \(M_i\)}
  \STATE \(M(a_j) \gets (c_k,d_i)\) \COMMENT{Mark \(a_j\) as vaccinated on day \(d_i\) under category \(c_k\)}
  \STATE \(r_k \gets r_k -1 \) \COMMENT {Update remaining overall quota}
  \ENDFOR
  \ENDFOR\\
  \RETURN \(M\)
\end{algorithmic}
\end{algorithm}

\subsection{Outline of the charging scheme}
We compare the solution obtained by Algorithm~\ref{alg:online-greedy-m2} with the optimal offline solution to get the worst-case competitive ratio for Algorithm~\ref{alg:online-greedy-m2}. Let $M$ be the output of Algorithm~\ref{alg:online-greedy-m2} and $N$ be an optimal offline solution. To compare $M$ and $N$, we devise a {\em charging scheme} similar to that in Section~\ref{sec:analysis-model-1}, by which each agent $a$ matched in $N$ {\em charges} a unique agent $a'$ matched in $M$. The amount charged, referred to as the {\em charging factor} here is the ratio of utilities obtained by matching $a$ and $a'$ in $M$ and $N$ respectively.

Properties of the charging scheme:
\begin{enumerate}
    \item Each agent matched in $N$ charges exactly one agent matched in $M$,
    \item Each agent matched in $M$ is charged by at most three agents matched in $N$, with charging factors at most $1,\delta$ and $\frac{\alpha_{\max}}{\alpha_{\min}}\delta$. This implies that the utility of $N$ is at most $(1+\delta+\frac{\alpha_{\max}}{\alpha_{\min}}\delta)$ times the utility of $M$.
\end{enumerate}

We divide the agents matched in $N$ into two types. Type $1$ agents are those which are matched in $M$ on an earlier day compared to that in $N$. Thus $a\in A$ is a Type $1$ agent if $a$ is matched on day $d_i$ in $M$ and on day $d_j$ in $N$, such that $i<j$. The remaining agents are called Type $2$ agents.
Our charging scheme is as follows:
    
\begin{enumerate}
    \item Type $1$ agents charge themselves with a charging factor $\delta$, since the utility associated with them in $N$ is at most $\delta$ times that in $M$. 
    
    \item Here onwards, we consider only Type $2$ agents and discuss the charging scheme associated with them.

Let $X_i$ be the set of Type $2$ agents matched on day $d_i$ in $N$, and let $Y_i$ be the set of agents matched on day $d_i$ in $M$. 

\begin{enumerate}
    \item {\em Case 1: $|X_i|\leq |Y_i|$: } From Lemma \ref{lemma:injection-exists} we claim that each agent $a_p \in X_i$ charges an agent in $a_q \in Y_i$ with $\alpha_p \leq \alpha_q$. Therefore the agents in $X_i$ charge the agents in $Y_i$ with a charging factor of $1$. 
    

\item {\em Case 2: $|X_i|=|Y_i|+z, z>0$: } Let $N_i$ and $M_i$ respectively be the restrictions of $N$ and $M$ to day $d_i$. We construct an auxiliary bipartite graph $G_i$ where $X_i\cup Y_i$ form one bipartition and categories form another bipartition. The edge set is $N_i\cup M_i$. For a category $c_k$, let $n_{j,k}$ and $m_{j,k}$ be the number of agents matched in $N$ and $M$ respectively, under category $c_k$ on day $d_j$. Then we set the quota of $c_k$ in $G_i$ to be $b_{i,k}=\min\{q_{i,k}, \max\{q_k-\sum_{j=1}^{i-1}n_{j,k}, q_k-\sum_{j=1}^{i-1} m_{j,k}\}\}$. This is the maximum of the quotas of $c_k$ that were available for computation of $N_i$ and $M_i$ respectively.
        
The charging scheme is given by the following. Consider the symmetric difference $M_i\oplus N_i$. Since $|N_i|=|M_i|+z$, there are exactly $z$ edge-disjoint alternating paths in $M_i\oplus N_i$ that start and end with an edge of $N$ \cite{NasreR17}. Let $\rho=\langle a_1,c_1,a_2,\ldots, a_k, c_k\rangle$ be one such path. Then $a_2,\ldots,a_{k-1}$ are matched in both $M_i$ and $N_i$, so they charge themselves with a charging factor of $1$. From Lemma \ref{lemma:injection-exists}, the agent $a_1$ charges $a_k$ with charging factor of at most $1$. It remains to decide whom $a_k$ charges.
        
Since $\rho$ terminates at $c_k$ with an $N_i$-edge, the number of agents matched to $c_k$ in $N_i$ is more than those matched to $c_k$ in $M_i$. In Lemma~\ref{lem:saturated}, we show that this can happen only because of exhaustion of $q_k$ in Algorithm~\ref{alg:online-greedy-m2} on or before day $d_i$. %since $|M_i|, |N_i|\leq s_i$ and $M_i$ is a maximum matching on day $d_i$. 
So agent $a_k$ can charge some agent $a_l$ matched to $c_k$ in $M$ on an earlier day, with charging factor $\frac{\alpha_{k}}{\alpha_{l}}\delta \leq \frac{\alpha_{\max}}{\alpha_{\min}}\delta$.
\end{enumerate}
\end{enumerate}

\begin{lemma}\label{lem:saturated}
If node $c_k$ is an end-point of a path $\rho$ in $G_i$, then $q_k$ is exhausted in Algorithm \ref{alg:online-greedy-m2} on or before day $d_i$. 
\end{lemma}
\begin{proof}
Suppose $c_k$ be an endpoint of $\rho$ in $G_i$. The number of agents matched to $c_k$ in $N_i$ is more than those matched to $c_k$ in $M_i$. We know that the daily supply $s_i$ of the day $d_i$ is an upperbound for both $|M_i|$ and $|N_i|$. Since $|N_i| = |M_i| + z$, we have $|M_i| < s_i$. From Algorithm \ref{alg:online-greedy-m2} we know that $M_i$ is a maximum-size b-matching in $H_i$ of size at most \(s_i\). If the capacity of $c_k$ is not saturated in $H_i$, then we can augment the path $\rho$ contradicting the maximality of $M_i$.  Since $c_k$ has more edges of $M_i$ than $N_i$ incident to it, from the definition of $b_{i,k}$, category $c_k$ must have exhausted the overall quota $q_k$ in Algorithm \ref{alg:online-greedy-m2} on or before day $d_i$. 
\end{proof}



%\begin{enumerate}
%    \item An agent who gets matched on day $d_i$ in the online solution and on day $d_j$ in the optimal solution, where $i<j$ charges itself. This forms the set $H_1$.
%    \item Along an augmenting path $\langle a_1,c_1,a_2,\ldots, a_k, c_k\rangle$ in $G_i$, $a_1,\ldots, a_{k-1}$ charge $a_2,\ldots,a_k$. Agent $a_k$ charges someone previously vaccinated in $ON$ in the same category $c_k$.
%    \item Note that $OPT_i$ matches more agents in $c_k$ than $ON_i$, and $|OPT_i|>|ON_i|$. So why did $ON_i$ not match more agents to $c_k$? The daily supply $s_i\geq %|OPT_i|>|ON_i|$. So $ON_i$ must have exhausted the overall %quota of $c_k$. Therefore, corresponding to $a_k$, there %must be an agent previously vaccinated by $ON$ in category %$c_k$. That agent is charged for $a_k$ with charging factor %$\delta$.
%\end{enumerate}

% \subsection{Analysis of the Algorithm}\label{sec:analysis-model-generic}
% To show that Algorithm~\ref{alg:online-greedy} achieves the desired competitive ratio, we introduce a \emph{charging scheme} - a mapping technique to compare the computed allocation with an optimal allocation. We say that an agent \(a\) who is vaccinated by an optimal allocation \emph{charges} an agent \(b\) with a \emph{charging factor}. To be precise, we show that the set \(S_{OPT}\) of agents who are vaccinated by an optimal allocation can be partitioned into three parts \(H_1, H_2 \text{ and } H_3\) such that any agent \(b\) who is vaccinated by the online allocation (computed by Algorithm~\ref{alg:online-greedy}) is charged by at most one agent from each part. Furthermore, each agent in \(H_1\cup H_3\) charges $b$ with a charging factor of \(\delta\), and each agent in \(H_2\) charges at most one agent with a charging factor \(1\).

% Let \(ON\) be an allocation computed by Algorithm~\ref{alg:online-greedy} on an instance \(I\), and let \(OPT\) be an optimal allocation on the same instance \(I\). Recall that an allocation \(M\) is of the form \(M:A\to(C\times D)\cup\{\varnothing\}\). Let \(S_{OPT}\) and \(S_{ON}\) be the sets of agents who got vaccinated by \(OPT\) and \(ON\) respectively. Now, \(H_1 \subseteq S_{OPT} \) can be described as the set of agents who are vaccinated by \(ON\) at least one day before they are vaccinated by \(OPT\) That is,

% \begin{align*}
%   H_1&=\{a\in S_{OPT} \mid OPT(a)=(c_i,d_j),\  ON(a)=(c_k,d_\ell), d_\ell<d_j\}
% \end{align*}

% For every agent \(a\in H_1\), \(a\) charges itself i.e. \(a \in S_{ON}\). Therefore, by definition, no two agents in \(H_1\) charge the same agent, and every agent in \(H_1\) charges exactly one agent in \(S_{ON}\). Since the utility associated with vaccinating $a$ in \(OPT\)is at most $\delta$ times that in \(ON\),the charging factor is $\delta$ in this case.

% Let \(S'_{OPT} = S_{OPT}\setminus H_1\). %be the remaining agents from \(S_{OPT}\) who are not in \(H_1\).

% To construct the parts \(H_2\) and \(H_3\), we first partition the agents in \(S'_{OPT}\) based on the day on which they got vaccinated by \(OPT\). That is, \[ S' = X_1\uplus X_2 \uplus \cdots \uplus X_{r=|D|} \]where,
% \begin{align*}
%   X_i &= \{a\in S' \mid a \text{ is vaccinated on day } d_i \text{ by } OPT  \} &&\forall {i\in\{1,2,\cdots,|D|\}}\\
% \end{align*}
% Similarly, let\[ S'_{ON} = Y_1\uplus Y_2 \uplus \cdots \uplus Y_{r=|D|} \] where,
% \begin{align*}
%   Y_i &= \{a\in S'_{ON} \mid a \text{ is vaccinated on day } d_i \text{ by } ON  \} &&\forall {i\in\{1,2,\cdots,|D|\}}\\
% \end{align*}

% Now, to construct the sets \(H_2\) and \(H_3\), we go over each set \(X_i\) and decide whether to include either the whole set \(X_i\) or a subset of it into \(H_2\). For each set \(X_i\), if \(|X_i|\le|Y_i|\), then we add \(X_i\) to \(H_2\).  This means the agents in \(X_i\) charge the agents in \(Y_i\), and the charging factor is \(1\) as the utility associated with vaccinating an agent in $X_i$ and its counterpart in $Y_i$ is the same. Since  \(|X_i|\le|Y_i|\), we can choose any injective mapping from \(X_i\) to \(Y_i\) for this charging scheme. On the other hand, if \(|X_i|=|Y_i|+m,\ m>0\), then we carefully pick \(m\) agents from \(X_i\) and put them in \(H_3\) and the remaining agents to \(H_2\). In the following sections, we describe how to pick these agents and how these agents charge agents in \(S_{ON}\).

% \begin{algorithm}
% \caption{Algorithm to construct \(H_2\) and \(H_3\)}
% \label{alg:partition-h2-h3}
% \begin{algorithmic}[1]
%   \State Initialize \(H_2,H_3 \gets \varnothing\)
%   \For {\(i = 1, 2, \ldots, r=|D|\)}
%   \If{\(|X_i| \le |Y_i|\)}
%   \State \(H_2 \gets H_2 \cup X_i\)
%   \ElsIf{\(|X_i| = |Y_i| + m, m>0\)}
%   \State \(( \mathcal{U}_{d_i}, \mathcal{V}_{d_i}) \gets\textsc{Careful-partition}(X_i)\)
%   \State \(H_2\gets H_2 \cup  \mathcal{U}_{d_i}\)
%   \State \(H_3 \gets H_3 \cup  \mathcal{V}_{d_i}\) \Comment{\(| \mathcal{V}_{d_i}|=m\)}
%   \EndIf
%   \EndFor
%   \Procedure{Careful-partition}{$X_i$}
%   \State Let  \(A_i \gets \{a\in A \mid a \text{ is available on day } d_i, ON(a)=(c_k,d_j)\ s.t\  d_j\ge d_i\}\)
%   \State Let \(M_i^{ON} \gets \{(a_j,c_k) \mid a_j\in A_i \text{ is vaccinated on day } d_i \text{ under } c_k \text{ by } ON\}\)
%   \State Let \(M_i^{OPT} \gets \{(a_j,c_k) \mid a_j\in A_i \text{ is vaccinated on day } d_i \text{ under } c_k \text{ by } OPT\}\)
%   \State \(M \gets M_i^{ON} \triangle M_i^{OPT}\) \Comment{Symmetric difference}
%   \State Compute \(P_i\) the set of \(m\) edge disjoint \(M_i^{ON}\)-augmenting paths in \(M\).
%   \State \( \mathcal{V}_{d_i} \gets \{a \in A_i \mid \text{\(a\) is a penultimate agent in one of the paths in \(P_i\)} \}\)
%   \State \( \mathcal{U}_{d_i} \gets X_i\setminus  \mathcal{V}_{d_i}\)\\
%   \Return \(( \mathcal{U}_{d_i}, \mathcal{V}_{d_i})\)
%   \EndProcedure
% \end{algorithmic}
% \end{algorithm}


% Consider the case when \(|X_i| = |Y_i|+m\) such that \(m>0\) for some \(i\). Now, to partition \(X_i\) as \(X_i = \mathcal{U}_{d_i}\uplus  \mathcal{V}_{d_i}\), we construct an auxiliary bipartite graph \(G_i = (A_i\cup C, E_i)\), where \(A_i\) is the set of agents, who are available on day \(d_i\) and are not vaccinated until day \(d_{i-1}\) by \(ON\) i.e. 
% $A_i = \{a\in A \mid \ a \text{ is available on day } d_i \text{ and } ON(a)\ne(c_k,d_j)\ \forall  d_j< d_i \}$.


% There is an edge \((a_i,c_k)\) if \(a_i\) is eligible under category \(c_k\). To view allocations as \emph{b-matchings}, we introduce capacity functions \(b_{ON}^i:C\to \mathbb{Z}_{\ge 0}\) and \(b_{OPT}^i:C\to \mathbb{Z}_{\ge 0}\), which is the minimum of daily quota \(q_{ik}\) and total \emph{remaining} quota from the overall quota \(q_k\). Let \(n_{(i,k)}^{ON}\) and \(n_{(i,k)}^{OPT}\) represent total number of agents who got vaccinated under category \(c_k\) on or before day \(d_i\) by \(ON\) and \(OPT\) respectively. Then,
% \begin{align*}
%   b^{ON}_i(c_k) &= \min(q_{ik}, q_k-n_{(i-1,k)}^{ON})\\
%   b^{OPT}_i(c_k) &= \min(q_{ik}, q_k-n_{(i-1,k)}^{OPT})\\
% \end{align*}

% %b^i(c_k) &= max( b_{ON}^i(c_k),  b_{OPT}^i(c_k))
% Now, it is apparent that an allocation by \(ON\) for \(A_i\) on day \(d_i\) is a valid matching in \((G_i,b^{ON}_i)\). Similarly, an allocation by \(OPT\) can be viewed as a matching in \((G_i,b^{OPT}_i)\). Let us call these matchings as \(M_i^{ON}\) and \(M_i^{OPT}\) respectively.
% \begin{align*}
%   M_i^{ON} &= \{(a_j,c_k) \mid a_j\in A_i \text{ is vaccinated on day } d_i \text{ under } c_k \text{ by } ON\}\\
%   M_i^{OPT} &= \{(a_j,c_k) \mid a_j\in A_i \text{ is vaccinated on day } d_i \text{ under } c_k \text{ by } OPT\}\\
% \end{align*}

% It is clear that \(|M_i^{OPT}| = |X_i|\) and \(|M_i^{ON}| = |Y_i|\). %The set \(X_i\) is a set of agents whereas \(M_i^{OPT}\) is a set of edges - one edge per each agent in \(X_i\).\\

% Let \(b_i:C \to \mathbb{Z}_{\ge 0}\) be defined as
% \begin{align*}
%   b_i(c_k) &= \max(b^{ON}_i(c_k), b^{OPT}_i(c_k))
% \end{align*}

% Clearly, both \(M_i^{OPT}\) and \(M_i^{ON}\) are valid b-matchings in \((G_i,b_i)\). We use the following lemma to get \(m\) edge disjoint \(M_i^{ON}\)-augmenting paths \(P_i\) in \((G_i,b_i)\).
% \begin{lemma}
%   If \(|M_i^{OPT}| - |M_i^{ON}| = m > 0\), then there exists \(m\) edge disjoint \(M_i^{ON}\)-augmenting paths in \((G_i,b_i)\).
% \end{lemma}
% \begin{proof}
%   We prove this by principle of mathematical induction.
%   We are given that \(|M_i^{OPT}| - |M_i^{ON}| = m > 0\).
%   \paragraph{Base Case:} Let \(m=1\)
%   As \(|M_i^{OPT}| > |M_i^{ON}|\), there exists an \(M_i^{ON}\)-augmenting path in \((G_i,b_i)\) whose unmatched edges are from \(M_i^{OPT}\).
%   \paragraph{Induction Hypothesis: } Suppose the claim hold for \(m=k\) for some \(k\). That is, there exists \(k\) edge disjoint  \(M_i^{ON}\)-augmenting paths in \((G_i,b_i)\).

%   Now, let \(|M_i^{OPT}|-|M_i^{ON}|= k+1\). Since, \(|M_i^{OPT}|>|M_i^{ON}|\), there exists at least one  \(M_i^{ON}\)-augmenting path \(p\) in \((G_i,b_i)\) whose unmatched edges are from \(M_i^{OPT}\). Since \(p\) contains one more edge from \(M_i^{OPT}\) than \(M_i^{ON}\), \(|M_i^{OPT}\setminus~p|-|M_i^{ON}\setminus~p|=k>0\). By induction hypothesis, there exists \(k\) edge disjoint \(M_i^{ON}\)-augmenting paths in \((G_i,b_i)\). These \(k\) paths along with \(p\), give \(k+1\) many edge disjoint \(M_i^{ON}\)-augmenting paths in \((G_i,b_i)\).
% \end{proof}

% Let \(C_i\) be the set of categories which are end points of paths in \(P_i\). We consider the set \( \mathcal{V}_{d_i}\) of agents who are penultimate vertices of these paths. The sets can be formally defined as:  

% \begin{align*}
%              C_i = \{c_k \in C \mid\  &c_k \text{ is an end point of a path } p \in P_i\}\\
%    \mathcal{V}_{d_i} = \{a \in A_i \mid\  &(a,c_k) \in G_i \text{ for some } c_k\in C_i,\\
%                                     &a \text{ is a penultimate vertex of the paths in } P_i\}
% \end{align*}

% Let \(\mathcal{V}\) be the union of \(\mathcal{V}_{d_i}\) over all days.
% \begin{align*}
%   \mathcal{V} \defeq \bigcup_{d_i \in D} \mathcal{V}_{d_i}
% \end{align*}

% We repartition \(\mathcal{V}\) in terms of categories as follows
% \begin{align*}
%   \mathcal{V} &= \mathcal{V}_{c_1} \uplus \mathcal{V}_{c_2} \uplus \cdots \uplus \mathcal{V}_{c_k} &&\text{where,}\\
%   \mathcal{V}_{c_j} &= \{a\in \mathcal{V} \mid a \text{ is vaccinated under category } c_j \text{ by } OPT\}\\
% \end{align*}

% To show that \(\mathcal{V}_{d_i} \uplus \mathcal{U}_{d_i} = X_i\), \marginpar{\textcolor{red}{Is $\mathcal{U}_{d_i}$ defined at this stage?}} we first show that \(\mathcal{V}_{d_i} \subseteq X_i\).  
% \begin{lemma}
%   \( \mathcal{V}_{d_i}\subseteq X_i\)
% \end{lemma}
% \begin{proof}
%   For any agent \(a\in \mathcal{V}_{d_i}\), consider the \(M_i^{ON}\)-augmenting path \(p\) which has \((a,c_k)\) as one of its end edges. Since \(p\) is a  \(M_i^{ON}\)-augmenting path, \((a,c_k)\in M_i^{OPT}\). Thus \(a\) is vaccinated by \(OPT\) on day \(d_i\).

%   As \(a\in A_i\), \(a\) is not vaccinated by \(ON\) on or before day \(d_{i-1}\). And \(a\) got vaccinated by \(OPT\). Therefore, \(a \notin H_1\).  Which implies \(a\in S'\).

%   Hence, \(a\in X_i\).
% \end{proof}


% \begin{remark}\label{remark:m-agents}
%   If \(|P_i|=m\), then \(|\mathcal{V}_{d_i}|=m\).
% \end{remark}
% Remark~\ref{remark:m-agents} holds as the paths in \(P_i\) are edge disjoint, and \(\mathcal{V}_{d_i}\) is the set of penultimate vertices of these paths.

% Let \( \mathcal{U}_{d_i}=X_i\setminus  \mathcal{V}_{d_i}\). Now, while constructing \(H_2\) and \(H_3\), we add \( \mathcal{U}_{d_i}\) to \(H_2\) and \( \mathcal{V}_{d_i}\) to \(H_3\). Since \(|Y_i| = | \mathcal{U}_{d_i}|\), and by the construction of \(H_2\), the agents in \(Y_i\) are not yet charged by anyone from \(H_2\), the agents in \( \mathcal{U}_{d_i}\) charge the agents in \(Y_i\) with a charging factor of \(1\). Any arbitrary injective mapping works for this charging scheme.

% All that is left to prove now, is that each agent \(a\) in \(H_3\) who got vaccinated by \(OPT\) on some day \(d_j\), can uniquely charge an agent who got vaccinated by \(ON\) on some day \(d_k<d_j\). We prove this with the help of following lemmas.



% For all \(c_k \in C_i\), the set \( \mathcal{V}_{c_k} \cap \mathcal{V}_{d_i} \subseteq \mathcal{V}\) is the set of agents from \(\mathcal{V}\) who are penultimate vertices of paths in \(P_i\) ending at category \(c_k\).  These agents are vaccinated by \(OPT\) on day \(d_i\) under category \(c_k\). Suppose \(| \mathcal{V}_{c_k} \cap \mathcal{V}_{d_i}| =l\), then we have the following lemma.

% \begin{lemma}
%   \label{lemma:end-point}
%   If \(c_k\in C_i\) is the end point of \(l\) many paths from \(P_i\), then\\
%   \(b^i_{OPT}(c_k)~\ge~b^i_{ON}(c_k)~+~l\).
% \end{lemma}
% \begin{proof}
% We know that \(|M_i^{OPT}| = |M_i^{ON}| + l \le s_i\). Therefore the size of \(M_i^{ON}\) is strictly less than the daily supply \(s_i\), i.e,  \(|M_i^{ON}| < s_i\). Furthermore, the matching \(M_i^{ON}\) is a maximum size b-matching on \((G_i,b^i_{ON})\). If the capacity of \(c_k\) is not saturated in \((G_i,b^{ON}_i)\), then there exists an augmenting path \(p\in P_i\). This contradicts the maximality of \(M_i^{ON}\) as \(|M_i^{ON} \triangle\ p|>|M_i^{ON}|\). Therefore, the number of agents vaccinated on day \(d_i\) under category \(c_k\) is equal to \(b^i_{ON}(c_k)\).

%   From our assumption, there are \(l\) many \(M_i^{ON}\)-augmenting paths ending at \(c_k\) in \((G_i,b_i)\). From Remark~\ref{remark:m-agents}, we know that each augmenting path has a unique \(M_i^{OPT}\) edge to \(c_k\). If we apply every \(M_i^{ON}\)-augmenting path from \(P_i\), then the number of \(M_i^{ON}\) edges incident to \(c_k\) increases by \(l\). As \(M_i^{ON}\) saturates \(c_k\) in \((G_i,b_i^{ON})\), the capacity \(b_i(c_k) \ge b_i^{ON}(c_k) + l\). From the fact that \(b_i(c_k) = \max(b_i^{OPT}(c_k), b_i^{ON}(c_k))\), we can conclude that \(b_i(c_k) = b_i^{OPT}(c_k)\). Thus, \(b^i_{OPT}(c_k)~\ge~b^i_{ON}(c_k)~+~l\).
% \end{proof}

% \begin{corollary}\label{corollary:strict-less}
%   For each \(c_k \in C_i\), the online capacity \(b^i_{ON}(c_k) = q_k-n_{(i-1,k)}^{ON}\)
% \end{corollary}
% \begin{proof}
%   By definition  \(b_{ON}^i(c_k) = min(q_{ik}, q_k-n_{(i-1,k)}^{ON})\), From the above lemma, we know \(b^i_{OPT}(c_k)>b^i_{ON}(c_k)\). Moreover, \(q_{ik} \ge b^i_{OPT}\). Therefore, \(q_{ik}>b^i_{ON}\). Hence, \(b^i_{ON}(c_k) = q_k-n_{(i-1,k)}^{ON}\).
% \end{proof}

% % {\color{red}Now, using Corollary~\ref{corollary:strict-less} and Lemma~\ref{lemma:end-point}, the following theorem shows that every agent in \( \mathcal{V}_k\), for all \(c_k\in C_i\), can uniquely charge some agent who got vaccinated by \(ON\) under category \(c_k\) at least one day earlier. }

% \begin{corollary}\label{corollary:category-saturation}
%   For each \(c_k \in C_i\), the total number of agents vaccinated until (including) day \(d_i\) is \(q_k\)
% \end{corollary}
% \begin{proof}
%   % Without loss of generality, assume \(i<j\). From Corollary~\ref{corollary:strict-less}, we know that the capacity of category \(c_k\) on day \(d_i\) is \(b^i_{ON}  = q_k-n_{(i-1,k)}^{ON}\). As \(M_{i}^{ON}\) is a maximum size b-matching of size at most \(s_i\), and as \(|M_{i}^{OPT}|\le s_i\), It is clear that number of agents vaccinated by \(ON\) on day \(d_i\) under category \(c_k\)is exactly equal to it's capacity \(c_k\). Because otherwise, the size of \(M_i^{ON}\) can be increased by augmenting along one of the \(M_{i}^{ON}\)-augmenting path ending at \(c_k\). Therefore the total number of agents vaccinated by \(ON\) under category \(c_k\) on or before day \(d_i\) is
% We know that the number of agents vaccinated until day \(d_{i-1}\) under category \(c_k\) is \(n^{ON}_{(i-1,k)} \). From Corollary~\ref{corollary:strict-less} the number of agents vaccinated under category \(c_k\) on day \(d_i\) is \(q_k-n_{(i-1,k)}^{ON}\). Therefore, the total number of agents vaccinated under category \(c_k\) until day \(d_i\) is:
%   \begin{align*}
%     &= n^{ON}_{(i-1,k)} + q_k-n_{(i-1,k)}^{ON}\\
%     &= q_k\\
%   \end{align*}
% \end{proof}

% From Corollary~\ref{corollary:category-saturation}, we learn that the overall~quota \(q_k\) of the category is saturated by day \(d_i\).

% Now, we compare the number of agents vaccinated by \(OPT\) and \(ON\) until day \(d_{i-1}\) under category \(c_k\).


% \begin{lemma}\label{lemma:end-point-2}
%   If \(c_k \in C_i\) is the end point of \(l\) many paths from \(P_i\), then \( n_{(i-1,k)}^{OPT} \le n_{(i-1,k)}^{ON} - l\).   
% \end{lemma}
% \begin{proof}
%   From Corollary~\ref{corollary:strict-less}, Capacity \(b^i_{ON}(c_k) = q_k-n_{(i-1,k)}^{ON}\). From Lemma~\ref{lemma:end-point}, we know that  \(b^i_{OPT}(c_k)~\ge~b^i_{ON}(c_k)~+~l\). Therefore,
%   \begin{align*}
%     b^i_{OPT}(c_k)~\ge~ q_k-n_{(i-1,k)}^{ON}~+~l
%   \end{align*}
%   We know that \ \  \(b^i_{OPT}(c_k) = min(q_{ik}, q_k-n_{(i-1,k)}^{OPT}) \). Therefore,
%   \begin{align*}
%     &q_k-n_{(i-1,k)}^{OPT} \ge q_k-n_{(i-1,k)}^{ON} + l\\[10pt]
%       \implies& n_{(i-1,k)}^{OPT} \le n_{(i-1,k)}^{ON} - l
%   \end{align*}
%   % That is, the number of agents who got vaccinated by \(ON\) until day \(d_{i-1}\) under category \(c_k\) is at least \(l\) more than the number of agents who got vaccinated by \(OPT\) until day \(d_{i-1}\) under category \(c_k\). Therefore, there are sufficient number of agents for these \(l\) agents to charge to. Thus, the \(l\) penultimate agents of category \(c_k\) uniquely charge some \(l\) agents who were vaccinated by \(ON\) on or before day \(d_{i-1}\) under category \(c_k\) with a charging factor of \(\delta\).  
% \end{proof}


% \begin{theorem}\label{thm:main}
%   Every agent \(a \in \mathcal{V}_{c_k} \cap \mathcal{V}_{d_i}\) can uniquely charge an agent who is vaccinated by \(ON\) under the same category \(c_k\) on some day \(d_j\), where \(d_j < d_i\).
% \end{theorem}
% \begin{proof}
%   If \(\mathcal{V}_{c_k} = \varnothing\), then the theorem vacuously holds. Suppose \(\mathcal{V}_{c_k} \neq \varnothing\). Then by Corollary~\ref{corollary:category-saturation}, The overall quota of category \(c_k\) is saturated by \(ON\) for the first time on some day \(d_q\).

%   Now, we claim that \(\mathcal{V}_{c_k}\) does not contain any agent who is vaccinated on or before day \(d_{q-1}\). Because, if \(a \in \mathcal{V}_{c_k} \cap \mathcal{V}_{d_p}\) with \(d_p < d_q \), then from Corollary~\ref{corollary:category-saturation}, the overall quota of \(c_k\) is saturated by \(ON\) on day \(d_p\), which is a contradiction as we assumed \(d_q\) to be the first such day.

%   Suppose \(|\mathcal{V}_{c_k} \cap \mathcal{V}_{d_q}| = l\), then from Lemma \ref{lemma:end-point-2}, the number of agents who got vaccinated by \(ON\) until day \(d_{q-1}\) under category \(c_k\) is at least \(l\) more than the number of agents who got vaccinated by \(OPT\) until day \(d_{q-1}\) under category \(c_k\). Therefore the agents in \(|\mathcal{V}_{c_k} \cap \mathcal{V}_{d_q} |\) can uniquely charge \(l\) agents  who are vaccinated by \(ON\) on or before day \(q-1\).
  
%   Clearly \(|\mathcal{V}_{c_k}| \leq q_k \). Also, the number of agents vaccinated until day \(d_q\) by \(ON\) under category \(c_k\) is \(q_k\). Therefore the agents in \(\mathcal{V}_{c_k}\) who are vaccinated after day \(d_q\) can also uniquely charge some agent who got vaccinated by on \(ON\) under category \(c_k\) on or before day \(d_q\).

%   Therefore, every agent \(a \in \mathcal{V}_{c_k} \cap \mathcal{V}_{d_i}\) can uniquely charge an agent who is vaccinated by \(ON\) under the same category \(c_k\) on some day \(d_j\) before \(d_i\).
% \end{proof}

% It is clear that \(\bigcup_{d_i \in D}\mathcal{V}_{d_i} = \mathcal{V} = \bigcup_{c_k \in C}\mathcal{V}_{c_k}\) and \(\mathcal{V} = H_3\). As Theorem~\ref{thm:main} holds for every \(\mathcal{V}_{c_k}\), every agent in \(H_3\) can uniquely charge some agent with a charging factor of \(\delta\)

% \begin{proof}[of Theorem~\ref{thm:gen-on-same-utility}~(ii)]
%   Let $a_q$ be an agent who is vaccinated by the online matching \(M\) on day $i$. Then $a_q$ can be charged by at most three agents matched in \(N\). Suppose $a_q$ is vaccinated by the optimal matching \(N\) on some day $i' >i$. Assume that the agent $a_p$ and  of type 2 who also charges $a_p$. If the discounting factor of $a_q$ and $a_p$ are $\delta_q$ and $\delta_p$ respectively, then    
%   \begin{align*}
%     \frac{\delta_p^i + \delta_q^{i'}}{\delta_q^i}\  =\  \left(\frac{\delta_p}{\delta_q}\right)^i + {\delta_q^{i'-i}} \  \leq\  1 + \delta_q 
%   \end{align*}
% The last inequality follows as \(\delta_p \leq \delta_q \text{ and }i' > i\).  Therefore the utility obtained by $a_p$ and $a_q$ in $M_i$ is atmost $1 + \delta_q$ times the the utility of $a_q$ in $M_i$. Therefore the competitive ratio of Algorithm~\ref{alg:online-greedy-m1} is at most $1 + \max_i\{1 + \delta_i\}$.  
% \end{proof}


\subsection {Tight Example}
The following example shows that the competitive ratio of Algorithm~\ref{alg:online-greedy-m2} is tight. Let set of agents \(A=\{a_1, a_2, a_3\}\) and categories \(C=\{c_1, c_2\}\). Agent \(a_1\) is eligible under \(\{c_1,c_2\}\). Agent \(a_2\) is eligible only under \(\{c_1\}\) and agent \(a_3\) is eligible only under \(\{c_2\}\).  The daily~supply: \(s_{1}=1 \text{ and } s_{2}=2\). Overall~quotas: \(q_1=1 \text{ and } q_2=2\). The daily quota of each category on each day is set to \(1\). The utility discounting factor for each agent is~\(\delta\).  The priority factor of the agent $a_i$ is $\alpha_i$ for $i = 1,2,3$. We assume that $0 \leq \alpha_1 = \alpha_3 < \alpha_2 \leq 1$. Agent \(a_1\) is available on both the days.  Agent \(a_3\) is available only on the first day, whereas agent $a_2$ is available only on the second day.  
Figure~\ref{fig:tight-general} depicts this example. 
\begin{figure}
  \centering
  \begin{tikzpicture}[
    on_node/.style={circle, draw=red!30, fill=red!20, ultra thin, minimum size=4mm},
    opt_node/.style={circle, draw=green!30, fill=green!20, ultra thin, minimum size=4mm},
    null_node/.style={circle, draw=white!10, fill=white!10, ultra thin, minimum size=4mm},
    node distance=0.2cm,->,>=stealth',
    ]
    %Nodes
    \node (a1) {\(a_1\)};
    \node (a2) [below=1cm of a1] {\(a_2\)};
    \node (a3) [below=1cm of a2] {\(a_3\)};

    \node[on_node] (a1d1) [right=of a1] {$\alpha_1$};
    \node[opt_node] (a1d2) [right=1cm of a1d1] {\(\alpha_1\delta\)};
    
    \node[null_node] (a2d1) [right=of a2] {-};
    \node[opt_node] (a2d2) [right=1.35cm of a2d1] {\(\alpha_2\delta\)};
    
    \node[opt_node] (a3d1) [right= of a3] {$\alpha_3$}; 
    \node[null_node] (a3d2) [right=1.35cm of a3d1] {-};

    

    \node (d1) [above=of a1d1] {\(d_1\)};
    \node (d2) [above=of a1d2] {\(d_2\)};

    % Edges
    \draw[->] (a1d2.west) -- (a1d1.east);
    \draw[->] (a2d2) -- (a1d1);
    \draw[->] (a3d1) to [out=30,in=310] (a1d1);
  \end{tikzpicture}
  \caption[Tight Example]{A tight example with competitive ratio \(1+\delta+\frac{\alpha_2}{\alpha_1}\delta\). Online allocation indicated in red, Optimal allocation indicated in green and arrows indicate charging} 
  \label{fig:tight-general}
\end{figure}

Since the daily supply of day \(d_1\) is \(1\), vaccinating \(a_1\) maximizes the utility gained on the first day. Hence there exists a run of Algorithm~\ref{alg:online-greedy-m2} where \(a_1\) is  vaccinated under category \(c_1\) on day \(d_1\). In this run, agent \(a_2\) cannot be vaccinated on day \(d_2\) as she is eligible only under category \(c_1\) and overall quota of category \(c_1\) is exhausted. Hence, total utility gained by the online allocation is \(\alpha_1\). Whereas in a optimal allocation scheme all the agents can be vaccinated. Vaccinate agent \(a_3\) on day \(d_1\) under category \(c_2\), agent \(a_1\) and \(a_2\) on day \(d_2\) under categories \(c_2,c_1\) respectively. This sums to a total utility of \(\alpha_3+\alpha_1\delta+{\alpha_2}\delta\). Therefore the competitive ratio of the online algorithm is $\frac{\alpha_3+\alpha_1\delta+{\alpha_2}\delta}{\alpha_1} = \frac{\alpha_1+\alpha_1\delta+{\alpha_2}\delta}{\alpha_1} = 1+\delta+\frac{\alpha_{\max}}{\alpha_{\min}}\delta$. The first equality holds as $\alpha_1 = \alpha_3$. The second equality holds as $\alpha_{\max} = \alpha_2$ and $\alpha_{\min} = \alpha_1$.

