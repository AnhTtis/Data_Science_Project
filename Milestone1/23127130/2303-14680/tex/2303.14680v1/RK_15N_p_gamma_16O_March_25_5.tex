%\usepackage{amsmath}
%\usepackage{amssymb}
%\usepackage{graphicx}


\documentclass[preprintnumbers]{revtex4}
%%%%%%%%%%%%%%%%%%%%%%%%%%%%%%%%%%%%%%%%%%%%%%%%%%%%%%%%%%%%%%%%%%%%%%%%%%%%%%%%%%%%%%%%%%%%%%%%%%%%%%%%%%%%%%%%%%%%%%%%%%%%%%%%%%%%%%%%%%%%%%%%%%%%%%%%%%%%%%%%%%%%%%%%%%%%%%%%%%%%%%%%%%%%%%%%%%%%%%%%%%%%%%%%%%%%%%%%%%%%%%%%%%%%%%%%%%%%%%%%%%%%%%%%%%%%
\usepackage{color}
\usepackage{graphicx}
\usepackage{eurosym}
\usepackage{graphicx}
\usepackage{amsmath,amssymb}
\usepackage{color}
\usepackage[outdir=./]{epstopdf}
\usepackage{amsmath}
\usepackage{amsfonts}
\usepackage{amssymb}

\setcounter{MaxMatrixCols}{10}
%TCIDATA{OutputFilter=LATEX.DLL}
%TCIDATA{Version=5.50.0.2960}
%TCIDATA{<META NAME="SaveForMode" CONTENT="1">}
%TCIDATA{BibliographyScheme=Manual}
%TCIDATA{LastRevised=Saturday, March 25, 2023 20:05:47}
%TCIDATA{<META NAME="GraphicsSave" CONTENT="32">}
%TCIDATA{Language=American English}

\flushbottom \footnotesep = 0pt
\def\topfraction{1}
\def\bottomfraction{1}
\def\textfraction{0.05}
\def\floatpagefraction{0.95}
\textfloatsep = 0.5cm \floatsep = 0.0cm
\setcounter{topnumber}{2}
\setcounter{bottomnumber}{2}
%\input{tcilatex}
\begin{document}

\date{\today }
\title{The astrophysical $S-$factor and reaction rate for $^{15}$N($p,\gamma
$)$^{16}$O within the modified potential cluster model}
\author{S. B. Dubovichenko$^{1}$, N. A. Burkova$^{1,2}$, R. Ya. Kezerashvili$%
^{3,4,5}$, A. S. Tkachenko$^{1}$, and B. M. Yeleusheva$^{1,2}$}
\affiliation{\mbox{$^{1}$V.G. Fesenkov Astrophysical Institute, Almaty, Kazakhstan} \\
$^{2}$al-Farabi Kazakh National University, Almaty, Kazakhstan\\
$^{3}$New York City College of Technology, The City University of New York,
Brooklyn, USA\\
$^{4}$The Graduate School and University Center, The City University of New
York, New York, USA\\
$^{5}$Long Island University, New York, USA}

\begin{abstract}
We study a radiative $p^{15}$N capture on the ground state of $^{16}$O at
stellar energies within the framework of a modified potential cluster model
(MPCM) with forbidden states, including low lying resonances. The
investigation of the $^{15}$N($p,\gamma $)$^{16}$O reaction includes the
consideration of $^{3}S_{1}$ resonances due to $E1$ transitions and
contribution of $^{3}P_{1}$ scattering wave in $p$ + $^{15}$N channel due to
$^{3}P_{1}\longrightarrow $ $^{3}P_{0}$ $M1$ transition. We calculate the
astrophysical low-energy $S-$factor and extrapolated $S(0)$ turned out to be
within $34.7-40.4$ keV$\cdot $b. It is elucidated the important role of the
asymptotic constant (AC) for the $^{15}$N($p,\gamma $)$^{16}$O process with
interfering $^{3}S_{1}$(312) and $^{3}S_{1}$(962) resonances. A comparison
of our calculation for $S-$factor with existing experimental and theoretical
data is addressed and the reasonable agreement is found.

The reaction rate is calculated and compared with the existing rates. It has
negligible dependence on the variation of AC, but shows strong impact of the
interference of $^{3}S_{1}$(312) and $^{3}S_{1}$(962) resonances, especially
at $T_{9}$ referring to the CNO Gamow windows.
%The reaction rate shows minor sensitivity to the variation of ANC, but illustrates strong dependence on the interfering effects, especially at $T_{9}$ referring to the CNO Gamow windows.
We present a stellar temperature dependence on the Gamow energy and a
comparison of rates for radiative proton capture reactions for CNO cycle on
nitrogen isotopes obtained in the framework of the MPCM and give temperature
windows, prevalence, and significance of each process.
\end{abstract}

\maketitle

\section{Introduction}

Stars burning depend on the star's initial mass and can proceed either
through the $p-p$ chain or through the Carbon-Nitrogen-Oxygen (CNO) cycle,
fusing hydrogen to helium through a chain fusion processes, $-$ sequence of
thermonuclear reactions that provides most of the energy radiated by the hot
stars \cite{Barnes1982,Arnould2020}. Unlike the $p-p$ chain, the CNO cycle
is a catalytic cycle, that converts 4 protons into one helium nucleus but
does so via reactions on the preexistent seed isotopes of carbon, nitrogen
and oxygen nuclei. The carbon, nitrogen and oxygen isotopes act just as
catalysts in the CNO cycle. However, the second branch of CNO cycle involves
the seed isotopes of oxygen and fluorine and are largely transformed into $%
^{14}$N. The fluorine produced in this branch is merely an intermediate
product and at steady state, it does not accumulate in the star. The CNO\
bi-cycle involves the following chains of nuclear reactions:

\begin{equation}
^{\text{12}}\text{C(}p,\gamma \text{)}^{\text{13}}\text{N(}\beta ^{+}\nu
\text{)}^{\text{13}}\text{C(}p,\gamma \text{)}^{\text{14}}\text{N(}p,\gamma
\text{ )}^{\text{15}}\text{O(}\beta ^{+}\nu \text{)}^{\text{15}}\text{N}%
\begin{tabular}{c}
\ \ \ \ \ \ $\longrightarrow \text{ }^{\text{15}}\text{N(}p,\alpha \text{)}^{%
\text{12}}\text{C \ \ \ \ \ \ \ \ \ \ \ \ \ \ \ \ \ \ \ \ \ \ \ \ \ \ \ \ \
\ \ \ \ \ \ \ \ \ \ \ \ \ }$ \\
$\longrightarrow \text{ \ }^{\text{15}}\text{N(}p,\gamma \text{)}^{\text{16}}%
\text{O(}p,\gamma \text{)}^{\text{17}}\text{F(}\beta ^{+}\nu \text{)}^{\text{%
17}}\text{O(}p,\alpha \text{)}^{\text{14}}\text{N}$.%
\end{tabular}
\label{Limit}
\end{equation}%
Therefore, the CNO bi-cycle produces three electron neutrinos from beta
decays of $^{\text{13}}$N, $^{\text{15}}$O and $^{\text{17}}$F and is also
referred to as the \textquotedblleft cold\textquotedblright\ CNO cycle \cite%
{Wiescher2010}. The CN cycle contains no stable $^{\text{13}}$N and $^{\text{%
15}}$O isotopes of nitrogen and oxygen, that decay to the stable isotopes $^{%
\text{13}}$C and $^{\text{15}}$N, respectively. \ The catalic nuclei are
lost from the process via the leak reaction $^{\text{15}}$N($p,\gamma $)$^{%
\text{16}}$O and the subsequent reactions in (\ref{Limit}) restore the
catalic material, generating $^{\text{16}}$O and heavier isotopes leading to
the accumulation of the\ $^{\text{4}}$He and $^{\text{14}}$N nuclei. This
second branch produces $^{\text{17}}$F, which decays beta with the emission
of the 1.74 MeV electron neutrinos. Thus, $^{\text{15}}$N($p,\gamma $)$^{%
\text{16}}$O process represents a breakout reaction linking the alternative
NO channel of the CNO cycle that produces the stable oxygen isotopes.
Therefore, in the CNO cycle, the proton capture reaction on $^{15}$N allows
two possible channels: the branch of the cycle\ $^{15}$N($p,^{4}$He)$^{12}$C
and the branch of the cycle $^{15}$N($p,\gamma $)$^{16}$O reactions and they
intersect at the $^{\text{15}}$N nucleus.

The rate of the CN with respect to the NO cycle depends on the branching
ratio of the $^{\text{15}}$N($p,\gamma $)$^{\text{16}}$O and $^{\text{15}}$N(%
$p,\alpha $)$^{\text{12}}$C reaction cross sections. The probability for the
$^{\text{15}}$N($p,\gamma $)$^{\text{16}}$O process to occur is about one
for every thousand of the second \cite{Caciolli2011}, thus the contribution
to the overall nuclear energy production is negligible, while the
consequences on the nucleosynthesis are critical \cite{Boeltzig2016}.
Therefore, in the case of an active NO cycle, the correct evaluation of the $%
^{\text{15}}$N($p,\gamma $)$^{\text{16}}$O reaction is crucial to properly
predict the abundances of all the stable $^{\text{16}}$O, $^{\text{17}}$O
and $^{\text{18}}$O isotopes and their relative ratios \cite%
{Caughlan1962,Caughlan1983,Caughlan1988}. The reaction rates ratio
determines on how much nucleosynthesis of $^{\text{16}}$O, $^{\text{17}}$O
and $^{\text{18}}$O takes place during CNO burning \cite{Caughlan1962}.

Since the first experimental study of $^{15}$N($p,\gamma $)$^{16}$O reaction
in 1952 \cite{Schardt1952} experimental data \cite%
{Hebbard,Brochard1973,Rolf1974,Bemmerer2009,LeBlanc2010,Imbriani2012,Caciolli2011}
for total sections of the radiative $p^{15}$N capture in the energy region
from 80 keV to 2.5 MeV were collected \cite{Angulo1999,Xu2013}. Analysis of
exciting experimental measurements of the low-energy $^{\text{15}}$N($%
p,\gamma $)$^{\text{16}}$O reaction shows that cross-section data differ
substantially at lower energies.

In the past, variety of theoretical approaches from potential cluster models
to multilevel $R-$matrix formalisms \cite%
{Barker2008,Mukhamedzhanov2008,LeBlanc2010,Mukhamedzhanov2011,deBoer2013,Dubovichenko2014}
were used to describe the $^{\text{15}}$N($p,\gamma $)$^{\text{16}}$O
reaction cross-section at the stellar energies and astrophysical $S-$factor
that is the main characteristic of this process at low energies. In the
framework of the selective resonant tunneling model \cite{Tian2000} $^{\text{%
15}}$N($p,\gamma $)$^{\text{16}}$O cross-section and $S-$factor have studied
\cite{Khan2022}. Most recently, the astrophysical $S-$factor for the
radiative proton capture process on the $^{15}$N nucleus at stellar energies
are studied within the framework of the cluster effective field theory \cite%
{SonNP2022,Son2022}. The authors perform the single channel calculations
where only the first resonance was considered \cite{SonNP2022} and then
reported the results by including two low-energy resonances \cite{Son2022}.

In this paper we are continuing the study of the reactions of radiative
capture of protons on light atomic nuclei \cite{PRCDTKB2020,PRC2022DTKB} and
consider the radiative proton capture on $^{15}$N at astrophysical energies
in the framework of a modified potential cluster model (MPCM). Within the
MPCM over thirty reactions of radiative capture of protons, neutrons and
other charged particles on light atomic nuclei were considered and results
are summarized in \cite{DubBook2015,DubBook2019}. References \cite%
{PRCDTKB2020,PRC2022DTKB} provide the basic theoretical framework of the
MPCM approach for the description of a charged-particle-induced radiative
capture reactions. Calculation methods based on the MPCM of light nuclei
with forbidden states (FS) are used \cite{Neudatchin1992}. The presence of
allowed states (AS) and FS are determined on the basis of the classification
of the orbital states of clusters according to Young diagrams \cite%
{Kukulin1983}. In this approach, the potentials of intercluster interactions
for scattering processes are constructed based on the reproduction of
elastic scattering phase shifts, taking into account their resonance
behavior or the spectra of the levels of the final nucleus. For the bound
states (BSs) or ground state (GS) of nuclei in cluster channels,
intercluster potentials are built on the basis of a description of the
binding energy and some of the basic characteristics of such states, e.g.
asymptotic constant (AC) and mean square radius \cite%
{DubBook2019,PRCDTKB2020,PRC2022DTKB}.

This paper is organized as follows. Section II presents a structure of
resonance states and construction of interaction potentials based on the
scattering phase shifts, mean square radius, asymptotic constant and bound
states or ground state of $^{16}$O nucleus. Results of calculations of the
astrophysical $S-$factor and reaction rate for the proton radiative capture
on $^{15}$N are presented in Secs. III and IV, respectively. In the same
sections we discuss a comparison of our calculation for $S-$factor and
reaction rate with existing experimental and theoretical data. Moreover, in
Sec. IV comparison of rates for proton capture reactions on nitrogen
isotopes that have the same Coulomb barrier and are involved into the CNO
cycle is presented. Conclusions follow in Sec. V.

\section{Interaction potentials and structure of resonance states}

The $E1$ transitions from resonant $^{3}S_{1}-$scattering states are the
main contributions to the total cross-section of the radiative proton
capture on $^{15}$N to the ground state of $^{16}$O \cite{deBoer2013}. In
the channel of $p$ + $^{15}$N in continuum there are two $^{3}S_{1}$
resonances:

1.\qquad The first resonance is at an energy of 335(4) keV with a width of
110(4) keV in the laboratory frame and has a quantum numbers $J^{\pi }$, $%
T=1^{-}$, $0$ (see Table 16.22 in \cite{Ajzenberg1993}). The latter can be
due to the triplet $^{3}S_{1}$ scattering state and leads to $E1$ transition
to the GS. This resonance is at an energy of 312(2) keV with a width of
91(6) keV in the center-of-mass (c.m.) frame and corresponds to the resonant
state of the $^{16}$O at an excitation energy of $E_{\text{x}}=12.440(2)$
MeV (see Table 16.13 in \cite{Ajzenberg1993}). However, in the new data base
\cite{Sukhoruchkin2016}, for this resonance, the excitation energy of $E_{%
\text{x}}=12.445(2)$ MeV and the width of $\Gamma =101(10)$ keV in the c.m.
are reported.

2.\qquad The second resonance is at an energy of 1028(10) keV with a width
of 140(10) keV in laboratory frame and has a quantum numbers $J^{\pi }=1^{-}$
and $T=1$ \cite{Ajzenberg1993}. This is also due to the triplet $^{3}S_{1}$
scattering and leads to $E1$ transition to the GS of $^{16}$O. The resonance
emerges at an energy of 962(2) keV with a width of $\Gamma =139(2)$ keV in
the c.m. and corresponds to the excitation energy of $E_{\text{x}}=13.090(2)$
MeV of $^{16}$O in a new data base \cite{Sukhoruchkin2016}. In the data base
\cite{Ajzenberg1993} for this resonance, the excitation energy of $E_{\text{x%
}}=13.090(8)$ MeV and width of $\Gamma =130(5)$ keV in the c.m. are
reported.
\begin{table}[b]
\caption{Data on the $^{3}S_{1}$ resonance states in $p$ + $^{15}$N channel.
$E_{\text{x}}$ is the excitation energy, $E_{\text{res}}$ and $\Gamma _{%
\text{res}}$ are the experimental resonance energy and the width,
respectively. $E_{\text{theory}}$ and $\Gamma _{\text{theory}}$ are the
resonance energy and the width, respectively, obtained in the present
calculations.}
\label{tab:Table1}
\begin{center}
\begin{tabular}{cccccc}
\hline\hline
$^{2S+1}L_{J}$ & $E_{\text{x}}$, MeV & $E_{\text{res}}$, keV & $\Gamma _{%
\text{res}}$, keV & $E_{\text{theory}}$, keV & $\Gamma _{\text{theory}}$, keV
\\ \hline
$^{3}S_{1}$(312) &
\begin{tabular}{c}
12.440(2) \cite{Ajzenberg1993} \\
12.445(2)\cite{Sukhoruchkin2016}%
\end{tabular}
&
\begin{tabular}{c}
312(2) \cite{Ajzenberg1993} \\
317(2) \cite{Sukhoruchkin2016}%
\end{tabular}
&
\begin{tabular}{c}
91(6) \cite{Ajzenberg1993} \\
101.5(10) \cite{Sukhoruchkin2016}%
\end{tabular}
& 312 & $125-141$ \\
$^{3}S_{1}$(962) &
\begin{tabular}{c}
13.090(8) \cite{Ajzenberg1993} \\
13.090(2) \cite{Sukhoruchkin2016}%
\end{tabular}
&
\begin{tabular}{c}
962(8) \cite{Ajzenberg1993} \\
962(2) \cite{Sukhoruchkin2016}%
\end{tabular}
&
\begin{tabular}{c}
130(5) \cite{Ajzenberg1993} \\
139(2) \cite{Sukhoruchkin2016}%
\end{tabular}
& 962 & $131$ \\ \hline\hline
\end{tabular}%
\end{center}
\end{table}
The compilation of experimental data on the $^{3}S_{1}$ resonances are
presented in Table \ref{tab:Table1}.

In data bases \cite{Ajzenberg1993,Sukhoruchkin2016} are reported the other
resonances as well. The third resonance has an energy of 1640(3) keV with a
width of 68(3) keV in laboratory frame and quantum numbers $J^{\pi }$,$%
T=1^{+},0$. This resonance can be due to the triplet $^{3}P_{1}$ scattering
and leads to $M1$ transition to the GS. The resonance is at an energy of
1536(3) keV with a width of 64(3) keV in the c.m. that corresponds to the
excitation energy 13.664(3) MeV of the $^{16}$O \cite{Ajzenberg1993} and
\cite{Sukhoruchkin2016} reported the excitation energy of 13.665(3) MeV and
the width of 72(6) keV in the c.m. However, this resonance was observed only
in measurements \cite{Rolf1974}, and in the later measurements \cite%
{LeBlanc2010,Imbriani2012} the resonance is absent. Therefore, we will not
consider it in our calculations. The next resonance is excited at the energy
of 16.20(90) MeV ($J^{\pi }$,$T=1^{-},0$) has a larger width of 580(60) keV
in the c.m. and its contribution to the reaction rate will be small. In
addition, in the spectra of $^{16}$O \cite{Ajzenberg1993}, another resonance
is observed at an excitation energy of 16.209 (2) MeV ($J^{\pi }$,$T=1^{+},1$%
) with a width of 19 (3) keV in the c.m. However, the resonance energy is
too large and its width too small to make a noticeable contribution to the
reaction rates. Therefore, in calculations we are considering only the above
two $^{3}S_{1}$ resonance transitions and non-resonance $^{3}P_{1}$
scattering for the $M1$\ transition to the $^{16}$O GS. Let us notice that
the multipole-channel analysis of ground-state-cascade transitions for the
reaction $^{15}$N($p,\gamma$)$^{16}$O \cite{deBoer2013} demonstrated that
contributions to the total $S-$factor the transition to the ground state is
a dominant.

For calculations of the total radiative capture cross-sections, the nuclear
part of the $p^{15}$N interaction potential is contracted using Gaussian
form \cite{DubBook2019,PRCDTKB2020,PRC2022DTKB}:
\begin{equation}
V(r,JLS,\{f\})=-V_{0}(JLS,\{f\})exp\left( -\alpha (JLS,\{f\})r^{2}\right).
\label{potential}
\end{equation}
The parameters $\alpha$ and $V_{0}$ in Eq. (\ref{potential}) are the
interaction range and the strength of the potential, respectively.

\begin{table}[b]
\caption{Parameters of interaction potentials $V_0$ in MeV and $\protect%
\alpha $ in fm$^{-2}$ for the GS and continuum states. The $C_W$ is
dimensionless constant. The theoretical widths, $\Gamma _{\text{theory}}$ in
keV, for the resonance $^3S_1$(312) and $^3S_1$(962) are calculated using
the corresponding parameters of the potentials.}
\label{tab:Table1_1}
\begin{center}
\begin{tabular}{c|ccc|ccc|ccc|cc}
\hline\hline
& \multicolumn{3}{|c|}{$^{3}P_{0}$, GS} & \multicolumn{3}{|c|}{$^{3}S_{1}$%
(312), $E1$} & \multicolumn{3}{|c|}{$^{3}S_{1}$(962), $E1$} &
\multicolumn{2}{|c}{$^{3}P_{1}$, $M1$} \\ \cline{2-12}
Set & $V_{0}$ & $\alpha $ & $C_{W}$ & $V_{0}$ & $\alpha $ & $\Gamma _{\text{%
theory}}$ & $V_{0}$ & $\alpha $ & $\Gamma _{\text{theory}}$ & $V_{0}$ & $%
\alpha $ \\ \hline
I & 976.85193 & 1.1 & 2.05 & 1.0193 & 0.0028 & 141 &  &  &  &  &  \\
II & 1057.9947 & 1.2 & 1.94 & 1.0552 & 0.0029 & 131 & 105.0675 & 1.0 & 131 &
14.4 & 0.025 \\
III & 1179.3299 & 1.35 & 1.8 & 1.0902 & 0.003 & 125 &  &  &  &  &  \\
\hline\hline
\end{tabular}%
\end{center}
\end{table}

The strength and the interaction range of the potential (\ref{potential})
depends on the total and angular momenta, the spin, $JLS$, and Young diagrams%
$\{f\}$ \cite{DubBook2015,DubBook2019}. For description of the $^{3}S_{1}$
scattering states we use the corresponding experimental energies and widths {%
from Table \ref{tab:Table1}. For the second $^{3}S_{1}$(962) resonance the
found parameters of interaction potential allow to reproduce the resonance
energy $E_{\text{res}} = 962(1)$ keV and width $\Gamma_{\text{res}} = 131$
keV given in Table \ref{tab:Table1_1}. }

A construction of the potentials that give the energies and widths of $%
^{3}S_{1}$(312) and $^{3}S_{1}$(962) resonances reported in the literature
is a challenging task. % due to the interference of these resonance.
One has to find the optimal parameters of the potentials for the description
of $E1$ transitions that leads to the fitting of the experimental resonance
energies and the widths of both interfering resonances. For the second $%
^{3}S_{1}$(962) resonance the found optimal parameters of the interaction
potential that allow {to reproduce the resonance energy $E_{\text{res}%
}=962(1)$ keV and width $\Gamma _{\text{res}}=131$ keV are }reported in {%
Table \ref{tab:Table1_1}.} The situation is more complicated with the first $%
^{3}S_{1}$(312) resonance.%
% due to the interference of two $^{3}S_{1}$ resonances.
While it is possible to reproduce rather accurately the position and the
width of the $^{3}S_{1}$(312) resonance, the consideration of the
interference of $^{3}S_{1}$ resonances gives different sets of optimal
parameters for the potential. We found three sets I $-$ III of the optimal
values for $V_{0}$ and $\alpha $ parameters reproducing exactly the energy
of the first resonance $E_{\text{theory}}=312(1)$ keV, but the width $\Gamma
_{\text{theory}}$ is varying in the range $125-141$ keV.

The dependence of the elastic $p^{15}$N scattering phases shifts for the $E1$
transitions on the energy are shown in Fig. \ref{Fig1}. The result of the
calculation of the $^{3}S_{1}$ phase shift with the parameters for the $S$
scattering potential without FS from Table \ref{tab:Table1_1} leads to the
value of 90$^{\circ }$(1) at the energies 312(1) and 962(1) keV,
respectively. In case of the $^{3}S_{1}$ phase shift providing the first
resonance our calculations with sets I $-$ III show very close energy
dependence in the whole energy range up to 5 MeV at fixed resonance
position.
\begin{figure}[h!]
\centering
\includegraphics[width=8.0cm]{graph1.png}
\caption{(Color online) The dependence of the elastic $p^{15}$N scattering
phase shifts on the energy. Calculations are performed by using the
potentials with parameters from Table \protect\ref{tab:Table1_1}. The phase
shift for $^{3}S_{1}$(312) resonance is calculated using the Set I and is
shown by the dash-dotted curve. The phase shifts for $^{3}S_{1}$(962)
resonance and $^3P_1$ are presented by the dotted and dashed curves,
respectively.}
\label{Fig1}
\end{figure}

The dependence of the elastic $p^{15}$N scattering phase shifts for the $E1$
transitions on the energy are shown in Fig. \ref{Fig1}. The result of the
calculation of the $^{3}S_{1}$ phase shift with the parameters for the $S$
scattering potential without FS from Table \ref{tab:Table1_1} leads to the
value of 90$^{\circ }$(1) at the energies 312(1) and 962(1) keV,
respectively. In the case of the $^{3}S_{1}$(312) phase shift reproducing
the first resonance our calculations with sets I $-$ III shows very close
energy dependence in the whole energy range up to 5 MeV at the fixed
resonance position.

In elastic $p^{15}$N scattering spectra at energies up to 5 MeV, there are
no resonance levels with $J^{\pi }=0^{+},1^{+}$,$2$ except for the mentioned
above, and widths greater than 10 keV \cite{Ajzenberg1993}. Therefore, for
potentials of non-resonance $^{3}P-$waves with one bound FS parameters can
be determined based on the assumption that in the energy region under
consideration their phase shifts are practically zero or have a gradually
declining character \cite{PRC2022DTKB}. For such potential the optimal
parameters are: $V_{p}=14.4$ MeV$,$ $\alpha _{p}=0.025$ fm$^{\text{-2}}$.
The result of calculation of $P-$phase shift with such a potential at an
energy of up to 5 MeV is shown in Fig. \ref{Fig1}. To determine the values
of phase shifts at zero energy, we use the generalized Levinson theorem \cite%
{Neudatchin1992}, so the phase shifts of the potential with one bound FS
should begin from 180$^{\circ }$. In the energy region $E_{\text{cm}}<5$ MeV
the $^{3}P_{1}$ phase shift has very weak energy dependence and it is almost
constant up to $E_{\text{cm}}\lesssim 2.2$ MeV.

We construct the potential for $^{16}$O in GS with \textit{J}$^{\pi }$%
\textit{,T} = 0$^{+}$,0 in $p^{15}$N-channel based on the following
characteristics: the binding energy of 12.1276 MeV, the experimental values
of 2.710(15) fm and 2.612(9) fm \cite{Ajzenberg1993} for the root mean
square radii of $^{16}$O and $^{15}$N of \cite{Ajzenberg1991}, respectively,
and a charge and matter radius of a proton 0.8775(51) fm \cite{Website1}.
The potential also should reproduce the AC. The corresponding potential
includes the FS and refers as the $^{3}P_{0}$ state.

Usually for a proton radiative capture reaction of astrophysical interest
one assumes that it is peripheral, occurring at the surface of the nucleus.
If the nuclear process is purely peripheral, then the final bound-state wave
function can be replaced by its asymptotic form, so the capture occurs
through the tail of the nuclear overlap function in the corresponding
two-body channel. The shape of this tail is determined by the Coulomb
interaction and is proportional to the asymptotic normalization coefficient
(ANC). The role of the ANC in nuclear astrophysics was first discussed by
Mukhamedzhanov and Timofeyuk \cite{Timofeyuk1990} and in Ref. \cite{Xu1994}.
These works paved the way for using the ANC approach as an indirect
technique in nuclear astrophysics. See Refs. \cite%
{Timofeyuk1995,Mukhamedzhanov2001,Timofeyuk2003,Mukhamedzhanov2003,Timofeyuk2009,Timofeyuk2013,Mukhamedzhanov2014,Blokhintsev2021,Blokhintsev2022}
and citation herein and the most recent review \cite{Mukhamedzhanov2023}.

We construct a potential with the FS $^{3}P_{0}$ state using the
experimental ANC given in Ref. \cite{Mukhamedzhanov2008} that relates to the
asymptotics of radial wave function as $\chi _{L}(R)=CW_{-\eta
L+1/2}(2k_{0}R)$. The dimensional constant $C$ is linked with the ANC via
the spectroscopic factor $S_{F}$. In our calculations we exploited the
dimensionless constant \cite{Plattner1981} $C_{W}$, which is defined in \cite%
{PRC2022DTKB} {as $C_{W}=C/\sqrt{2k_{0}}$, where $k_{0}$ is wave number
related to the binding energy}. In Ref. \cite{Mukhamedzhanov2008} the values$%
\ 192(26)$ fm$^{-1}$ and $2.1$ were reported for the ANC\ and spectroscopic
factor, respectively. In Ref. \cite{Mukhamedzhanov2011} the dimensional ANC
includes the antisymmetrization factor $N$ into the radial overlap function
as it was clarified by authors in \cite{Mukhamedzhanov2014}. The factor $N$
is defined as $N=\left(
\begin{matrix}
A \\
x%
\end{matrix}%
\right) ^{1/2}=\sqrt{\dfrac{A!}{(A-x)!x!}},$ where ${x}$ and ${A}$ are the
atomic numbers of constituent nucleus from ${x}$ and ${A-x}$ nucleons,
respectively \cite{Blokhintsev1977}. If $x=1$, then $N=\sqrt{A}$ and for the
reaction $^{15}$N($p,\gamma $)$^{16}$O $N=4$. Finally, we obtained the
interval for dimensionless AC used in our calculations $C_{W}=1.82-2.09$,
which corresponds to the ANC $192(26)$ fm$^{-1}$ from \cite%
{Mukhamedzhanov2008,Mukhamedzhanov2011}.

The $^{15}$N($p,\gamma $)$^{16}$O is the astrophysical radiative capture
process, in which the role of the ANC is elucidated \cite{Mukhamedzhanov2023}%
. In Table \ref{tab:Table1_1} are listed three sets of parameters for the $%
^{3}P_{0}$ GS potential and AC $C_{W}$. The asymptotic constant $C_{W}$ is
calculated over averaging at the interval $5-10$ fm. Each set leads to the
binding energy of 12.12760 MeV, the root mean square charge radius of 2.54
fm and the matter radius of 2.58 fm, but the sets of $C_{W}$ leads to the
different widths of the $^{3}S_{1}$(312) resonance.

Note, that there is one important benchmark for the choice of optimal sets
for the parameters of interaction potentials for the first $E1$(312)
resonance. There are the experimental values of the total cross-section $%
\sigma_{\text{exp}}(312)=6.0 \pm 0.6$ $\mu$b \cite{Caciolli2011} and $6.5
\pm 0.6$ $\mu$b \cite{LeBlanc2010}, which are in excellent agreement with
earlier data $6.3$ $\mu$b \cite{Brochard1973} and $6.5 \pm 0.7$ $\mu$b \cite%
{Hebbard}. Simultaneous variation of $C_W$ for the GS and parameters $V_0$
and $\alpha$ for the $^3S_1$(312) was implemented in order to keep the value
of the cross-section $\sigma_{\text{theory}}$(312)$=5.8-5.9$ $\mu$b matching
the experimental data. The result of this optimization is presented in Table %
\ref{tab:Table1_1} as sets I$-$III.

In the present calculations, we use for the proton mass $m_{p}=1.00727646677$
amu \cite{Website1}, $^{15}$N mass 15.000108 amu \cite{Website2}, and the
constant $\hbar ^{2}/m_{0}=41.4686$ MeV\textperiodcentered fm$^{2}$, where $%
m_{0}=931.494$ MeV is the atomic mass unit (amu).\

Table \ref{tab:Table1_1} summarizes the potential parameters used in the
case where the MPCM works reasonably well for a radiative proton capture in
the $^{\text{15}}$N($p,\gamma $)$^{\text{16}}$O reaction.

\section{Astrophysical $S-$Factor}

The astrophysical $S-$factor is the main characteristic of any thermonuclear
reaction at low energies. The present analysis focuses primarily on
extrapolating the low-energy $S-$factor of the reaction $^{15}$N($p,\gamma $)%
$^{16}$O into the stellar energy range. Since the first experimental study
of $^{15}$N($p,\gamma $)$^{16}$O reaction in 1960 \cite{Hebbard},
experimental data \cite%
{Rolf1974,LeBlanc2010,Imbriani2012,Caciolli2011,Xu2013} for total sections
of the radiative $p^{15}$N capture in the energy region from 80 keV to 2.5
MeV have been collected. These experimental studies verified and confirmed
that the radiative $p^{15}$N capture is dominated by the first two
interfering resonances at 312 keV and 962 keV with the quantum numbers $%
J^{\pi }$, $T=1^{-}$,$0$ and $J^{\pi }$, $T=1^{-}$,$1$, respectively.

\subsection{$E1$ transitions}

The $E1$ transitions are the main input parts of the radiative capture
amplitude for $^{15}$N($p,\gamma $)$^{16}$O reaction. Therefore, it is
required to determine the resonance capture cross-sections for these
transitions accurately to avoid one the main source of uncertainty. The
radiative capture resonance to the bound states is reviewed in Ref. \cite%
{Mukhamedzhanov2023}. Following Ref. \cite{PRC2022DTKB} after algebraic
calculations using quantum numbers related to the $^{15}$N($p,\gamma $)$%
^{16} $O reaction, one can write the cross-section for the radiative capture
$p^{15}$N to the ground state of $^{16}$O as

\begin{equation}
\sigma _{E1}(E_{cm})=\frac{4\pi e^{2}}{9\hbar ^{2}}\left( \frac{K }{k}%
\right) ^{3}\left( \frac{1}{m_{p}}-\frac{7}{m_{^{15}\text{N}}}\right)
^{2}\left\vert I(k;E1)\right\vert ^{2}.  \label{CrossSec}
\end{equation}%
In Eq. (\ref{CrossSec}) $\mu $ is the reduced mass of the proton and $^{15}$%
N nucleus, $K=E_{\gamma }/\hbar c$ is the wave number of the emitted photon
with energy $E_{\gamma }$, $k$ is relative motion wave number and

\begin{equation}
\left\vert I(k;E1)\right\vert ^{2}=\left\vert e^{-i\delta
_{^{3}S_{1}(312)}}I_{1}+e^{-i\delta _{^{3}S_{1}(962)}}I_{2}\right\vert
^{2}=\left\vert I_{1}\right\vert ^{2}+\left\vert I_{2}\right\vert
^{2}+2\cos\left( \delta _{^{3}S_{1}(312)}-\delta _{^{3}S_{1}(962)}\right)
I_{1}I_{2},  \label{Interference}
\end{equation}%
where the overlapping integral between the initial $\chi _{i}$ and final $%
\chi _{f}$\ states radial wave functions is $I=\left\langle \chi
_{f}\right\vert r\left\vert \chi _{i}\right\rangle $. As it is follows from
Eq. (\ref{Interference}) $E1$ resonance$\rightarrow $ground state
transitions\ is given by the interference of $^{3}S_{1}(312)$ and $%
^{3}S_{1}(962)$\ resonances contributions into the cross-section. The
interference is determined by the difference of the $\delta
_{^{3}S_{1}(312)} $ and $\delta _{^{3}S_{1}(962)}$\ phase shifts via the
factor $\cos\left( \delta _{^{3}S_{1}(312)}-\delta _{^{3}S_{1}(962)}\right) $%
. We depict the behavior of this factor as a function of energy in Fig. \ref%
{Interfer} using the phases shifts shown in Fig. \ref{Fig1}. One can
conclude that the contribution of the interfering term into the $E1$
transitions cross-section is very significant at the energies up to 2.5 MeV.

\begin{figure}[h]
\centering
\includegraphics[width=8.0cm]{Interfer1.PNG} \centering
\caption{(Color online) The energy dependence of the factor $\cos\left(
\protect\delta _{^{3}S_{1}(312)}-\protect\delta _{^{3}S_{1}(962)}\right) $.}
\label{Interfer}
\end{figure}

\begin{figure}[b]
\centering
\includegraphics[width=8.0cm]{graph2.PNG} \centering
\caption{(Color online) The astrophysical $S-$factor of radiative $p^{15}$N
capture on the ground state of $^{16}$O. The $E1$ transition $%
^{3}S_{1}\longrightarrow $ $^{3}P_{0}$ for the resonance at 312(2) keV (the
shaded area bounded by the upper and lower dash-dotted curves, which are
obtained using set I and set III parameters, respectively, from Table
\protect\ref{tab:Table1_1}). The $E1$ transition $^{3}S_{1}\longrightarrow $
$^{3}P_{0}$ for the resonance at 962(8) keV (the shaded area bounded by two
curves) and $M1$ transition $^{3}P_{1}\longrightarrow $ $^{3}P_{0}$ (the
shaded area bounded by two dashed curves). The lower and upper bounded
curves for $E1$(962) resonance and $M1$ transitions are obtained for $%
C_{W}=1.8$ and $C_{W}=2.05$, respectively. The contribution of all
transitions to the astrophysical $S-$factor is shown by the shaded area
bounded by two solid curves.}
\label{Fig2}
\end{figure}

\subsection{\protect\bigskip Analysis of $S-$factor}

Results of calculations for the astrophysical $S-$factor basing on the
potential parameters given in Table \ref{tab:Table1_1} along with the
compilation of experimental data \cite%
{Rolf1974,LeBlanc2010,Imbriani2012,Caciolli2011,Xu2013} are presented in
Fig. \ref{Fig2}. Notice that in Ref. \cite{deBoer2013} using $R-$matrix
approach was considered the contribution of 2$^{+}$ level at the excitation
energy 12.97 MeV. The contribution of this transition to the GS is much
smaller than non-resonance $M1$ $^{3}P_{1}\longrightarrow $ $^{3}P_{0}$
transition. The interference of two resonant contributions are dominant and
gives the total capture cross-section for the $^{15}$N($p,\gamma $)$^{16}$O
process. The interference of $^{3}S_{1}(312)$ and $^{3}S_{1}(962)$\
resonances lead to the significant increase of $S-$factor at the energies up
to 300 keV. Consideration of all transitions leads to increase of the $S-$%
factor at energies less than 300 keV and larger than 1500 keV. Moreover, one
can see the discrepancies between the experimental data and theoretical
calculation at energies where the minimum of the $S-$factor is observed.
This is related to the distractive interference of $^{3}S_{1}(312)$ and $%
^{3}S_{1}(962)$\ resonances at this energy due to the factor $\cos \left(
\delta _{^{3}S_{1}(312)}-\delta _{^{3}S_{1}(962)}\right) $. This factor has
the minimum at about 500 keV as it is depicted in Fig. \ref{Interfer}. The
similar results are obtained within the $R-$matrix approach \cite{deBoer2013}
when the authors are considering the different reaction components
contributions in the fitting of the $^{15}$N($p,\gamma $)$^{16}$O reaction
data (see Fig. 49, \cite{deBoer2013}). Only by using the set of fitting
parameters can be described the region of 0.5 MeV between resonances for the
$S-$factor \cite{deBoer2013}. Let us mention that in the MPCM one has only
the input parameters that are asymptotic constants and binding energies for
the construction of BSs potentials, and spectra of the final nucleus or the
scattering phase shifts of the particles of the entry channel for the
construction of potentials of the scattering processes.

The shaded areas in Fig. \ref{Fig2} correspond to the contribution
of the $E1$ and $M1$ transitions and the sum of all transitions into the $%
S(E)$ factor. The shaded areas show the range of $S(E)$ changes for
different values of the AC. Thus, the values transitions amplitudes are
governed by the AC. At an energy of 30--60 keV, the $S-$factor is
practically constant %and equal to 29.1(6) keV$\cdot $b and this
and the corresponding value can be considered as the $S-$factor at zero
energy. Thus, the theoretical calculation predicts very smooth behavior of $%
S(E)$ at very low energies that converges to $S(0)=35.2(5)$ keV$\cdot $b for
$C_{W}=1.8$. The increase of the AC leads to the increase of $S(0)$. The
variation of the AC within the experimental uncertainties, leads to the
increase of the $S-$factor up to $S(0)=39.6(8)$ keV$\cdot $b for $C_{W}=2.05$%
. Therefore, depending on the value of the AC, $S(0)$ varies in the range of
$34.7-40.4$ keV$\cdot $b. Our predictions overlaps with the value of the $%
S(0)-$factor reported in Ref. \cite{Mukhamedzhanov2008}. Note that in Ref.
\cite{Son2022} the value of 29.8(1.1) keV$\cdot $b was obtained in the
framework of the effective field theory. However, \cite{Son2022} describes
the experimental $S-$factor with the resonances energies $\approx $ 360--370
keV and widths $\approx $ 250--350 keV for the first resonance and the
resonances energies $\approx $ 960--970 keV and widths $\approx $ 155--160
keV for the second resonance, respectively, which are very far off from the
experimental data.
\begin{table}[h]
\caption{Experimental data on the astrophysical $S-$factor of the $^{15}$N($%
p,\protect\gamma $)$^{16}$O reaction. The values of $S(E_{\min })$ listed in
rows 1 and 2 are taken from Fig. 8 in Ref. \protect\cite{LeBlanc2010}. $%
S(E_{\min })$ are given with the precision of one tenth of keV$\cdot $b.}
\label{tab:TableSEx}
\begin{center}
\begin{tabular}{ccccc}
\hline\hline
& Reference &
\begin{tabular}{c}
$E_{cm}$, keV \\
Experimental range%
\end{tabular}
& $E_{\min }$, keV & $S(E_{\min })$, keV$\cdot $b \\ \hline
1 & Hebbard et al., 1960 \cite{Hebbard} & $206-656$ & 230 & $138.6\pm 15.2$
\\
2 & Brochard et al., 1973 \cite{Brochard1973} & $234-1219$ & 256 & $215.1\pm
27.3$ \\
3 & Rolfs \& Rodney 1974 \cite{Rolf1974} & $139-2344$ & 139 & $124.2\pm 52.6$
\\
4 & Bemmerer et al., 2009 \cite{Bemmerer2009} & $90-230$ & 90 & $38.4\pm 5.4$
\\
5 & LeBlanc et al., 2010 \cite{LeBlanc2010} & $123-1687$ & 123 & $53\pm 7.1$
\\
6 & Caciolli et al., 2011 \cite{Caciolli2011} & $70-370$ & 70 & $52\pm 4$ \\
7 & Imbriani et al., 2012 \cite{Imbriani2012} & $131-1687$ & 131 & $48.4\pm
4.8$ \\ \hline\hline
\end{tabular}%
\end{center}
\end{table}

In Table \ref{tab:TableSEx} the experimental data for the GS astrophysical $%
S-$factor in the measured energy ranges are given. The experimental range of
energy is dramatically different which leads to the different values of $%
S(E_{\min }).$ In Ref. \cite{Rolf1974} the cross-section is measured for the
highest energy, while in Ref. \cite{Caciolli2011} are reported the
cross-section for the lowest energy $E_{\text{cm}}=70$ keV, which is near of
the Gamow range. It is obvious that extrapolation of the $S(E)$ to the $S(0)$
using each listed experimental energy range will give the different values
of the $S(0)$, sometimes dramatically different.

The determination of $S(0)$ relies on the dual approach of experimental
measurement of the cross-section complemented by theoretical interpretation
and extrapolation from the experimental range of energy to the zero energy.
In Table \ref{tab:TableSTher} are listed the estimates of the astrophysical $%
S-$factor at zero energy $S(0)$ obtained using the $R-$matrix fits of the
different sets of experimental data, different model calculations and
extrapolation of the experimental data. By varying the fitting method,
authors obtained different values of $S(0)$, see for example Ref. \cite%
{Mukhamedzhanov2011}. Depending on the data used for the fit, the values of $%
S(0)$ are scattered from 21 keV$\cdot $b to 75 keV$\cdot $b, as it is seen
in Table \ref{tab:TableSEx}. Theoretical evaluation of astrophysical $S(E)$
and its extrapolation to $S(0)$ are also model dependent, consequently the
uncertainties in the computed $S-$factor can be significant \cite%
{Yakovlev2010}. The extrapolation is of insufficient accuracy because of the
difficulties in taking full account of the complexities of the reaction
mechanisms \cite{Wiescher2012} as well.
\begin{table}[h]
\caption{Values of the astrophysical $S(0)$ factor of the $^{15}$N($p,%
\protect\gamma $)$^{16}$O reaction. The estimations for values of the $S(0)$
are obtained based of experimental data from references listed in the
parentheses.}
\label{tab:TableSTher}
\begin{center}
\begin{tabular}{cc}
\hline\hline
Reference & $S(0)$, keV$\cdot $b \\ \hline
Rolfs \& Rodney, 1974 \cite{Rolf1974} &
\begin{tabular}{c}
32 (\cite{Hebbard}) \\
$64\pm 6$ (\cite{Rolf1974})%
\end{tabular}
\\
Barker, 2008 \cite{Barker2008} & $%
\begin{tabular}{c}
$\approx $ 50 -- 55 (\cite{Rolf1974}) \\
$\approx $ 35 (\cite{Hebbard})%
\end{tabular}%
$ \\
Mukhamedzhanov et al., 2008 \cite{Mukhamedzhanov2008} & 36.0 $\pm $ 6 \\
LeBlanc et al., 2010 \cite{LeBlanc2010} & $39.6\pm 2.6$ \\
Huang et al.,2010 \ \cite{Huang2010} & 21.1 \\
Mukhamedzhanov et al., 2011 \cite{Mukhamedzhanov2011} & $33.1-40.1$ \\
Xu et al., 2013 \cite{Xu2013} & 45$_{-7}^{+9}$ \\
deBoer et al., 2013 \cite{deBoer2013} & 40 $\pm \ $3 \\
Dubovichenko et al., 2014 \cite{Dubovichenko2014} & $39.5-43.35$ \\
Son et al., 2022 \cite{SonNP2022} & 30.4 (\cite{Caciolli2011}) \\
Son et al., 2022 \cite{Son2022} &
\begin{tabular}{c}
$75.3\pm 12.1$ (\cite{Rolf1974}) \\
$34.1\pm 0.9$ (\cite{LeBlanc2010}) \\
$29.8\pm 1.1$ (\cite{Imbriani2012})%
\end{tabular}
\\
Present work & $34.7-40.4$ \\ \cline{1-1}
\multicolumn{2}{c}{Results for $S(0)$ 35.2 $\pm $ 0.5$^{a}$ and 39.6 $\pm $
0.8$^{b}$ are obtained for AC $C_{W}$= 1.8$^{a}$ and $C_{W}$= 2.05$^{b}$.}
\\ \hline\hline
\end{tabular}%
\end{center}
\end{table}

At ultra-low energies, the energy dependence of the $S-$factor can be
modified by "a screening effect". The screening effects in plasma in
laboratory as well as astrophysical conditions are discussed in detail in
Refs. \cite{Bertulani2016,Famiano2020,Bertulani2020,Casey2023}. Despite
various theoretical studies conducted over the past two decades, a theory
has not yet been found that can explain the cause of the exceedingly high
values of the screening potential needed to explain the data \cite%
{Spitaleri2016}. In our study \cite{Dub2019nuc} of the radiative $^{3}$He($%
^{2}$H,$\gamma $)$^{5}$Li capture at astrophysical energies we evaluated
this phenomenon numerically using the parameters for the screening potential
from the data on D$^{3}$He plasma. It is demonstrated that the consideration
of the screening effects leads to increase of the $S-$factor when the energy
decreases. Our expectation is that consideration of the screening will
increase $S(0)$.
%{\color{red}The parameters for the screening potential for the $p$ + $^{14}$N channel was reported in \cite{Assenbaum1987}. Following the estimation of \cite{Assenbaum1987}, the enhancement of $S-$factor at the
%energies $\sim$ 70 keV corresponding to the LUNA lower data consists near 11\%. At the same time for the energies from 40 keV to 10 keV $S-$factor is $% 1.25 - 5.80$ reaches 119.73 at 5 keV. }
The lack of parameters for the screening potential in the $p^{15}$N medium
do not allow us to estimate a role of the screening effect in the $^{15}$N($%
p,\gamma $)$^{16}$O reaction. However, if one considers the estimation of
\cite{Assenbaum1987}, the enhancement of $S-$factor at the energies $\sim $
70 keV corresponding to the LUNA lower data consists near 11\%.

%The parameters for the screening potential for
%the $p$ + $^{14}$N channel was reported in \cite{Assenbaum1987}. Following
%the estimation of \cite{Assenbaum1987}, the enhancement of $S-$factor at the
%energies $\sim$ 70 keV corresponding to the LUNA lower data consists near
%11\%. At the same time for the energies from 40 keV to 10 keV $S-$factor is $%
%1.25 - 5.80$ reaches 119.73 at 5 keV. } %potential
%the lack of the parameters for the screening
%potential in the $p^{15}$N medium do not allow us to estimate a role of the
%screening effect in the $^{15}$N($p,\gamma $)$^{16}$O reaction.

\section{Reaction Rate}

The reaction rates for nuclear fusion are the critical component for
stellar-burning processes and study of stellar evolution \cite{Wiescher2010}%
. In stellar interiors, where the interacting particles follow a
Maxwell-Boltzmann distribution, the reaction rate describes the probability
of nuclear interaction between two particles with an energy-dependent
reaction cross section $\sigma (E)$. For charged particle--induced
interactions the reaction rate per particle pair can be written as \cite%
{Wiescher1999,Iliadis2015}.

\begin{equation}
N_{A}\left\langle \sigma \nu \right\rangle =N_{A}\left( \frac{8}{\pi \mu }%
\right) ^{1/2}\left( k_{B}T\right) ^{-3/2}\int \sigma (E)E\exp \left( -\frac{%
E}{k_{B}T}\right) dE=N_{A}\left( \frac{8}{\pi \mu }\right) ^{1/2}\left(
k_{B}T\right) ^{-3/2}\int S(E)e^{-2\pi \eta }\exp \left( -\frac{E}{k_{B}T}%
\right) dE.  \label{Rate1}
\end{equation}%
In Eq. (\ref{Rate1}) $N_{A}$ is the Avogadro number, $\mu $ is the reduced
mass of two interacting particles, $k_{B}$ is the Boltzmann constant, $T$ is
the temperature of the stellar environment, and the factor $e^{-2\pi \eta }$
approximates the permeability of the Coulomb barrier between two point-like
particles with charge $eZ_{1}$ and $eZ_{2}$.

\subsection{$^{15}$N($p,\protect\gamma $)$^{16}$O reaction rate}

The reaction rate can be numerically obtained in the framework of the
standard formalism outline in Ref. \cite{Angulo1999} based on the $S-$factor
that includes the contributions of all transitions shown in Fig. \ref{Fig2}
as well as fractional contributions of $E1$ transitions and $%
^{3}P_{1}\longrightarrow $ $^{3}P_{0}$ $M1$ transition to the total $^{15}$N(%
$p,\gamma $)$^{16}$O reaction rate. In Fig. \ref{Fig3} are presented the
reaction rate and fractional contributions of each transition to the
reaction rate. The insert in Fig. \ref{Fig3} shows the contribution of each
resonance and $^{3}P_{1}\longrightarrow $ $^{3}P_{0}$ transition with
respect to the total reaction rate as a function of astrophysical
\begin{figure}[h]
\centering
\includegraphics[width=10.0cm]{graph3.PNG} %
%\includegraphics[width=8.0cm]{graph3b2.png}
\caption{(Color online) ($a$) The dependence of the reaction rate of the $%
^{15}$N($p,\protect\gamma $)$^{16}$O radiative capture on astrophysical
temperature. The solid curve presents our calculations for the sum of $E1$
and $M1$ transitions performed for the potentials with the sets of
parameters from Table \protect\ref{tab:Table1_1}. The inset shows the
fractional contributions of the reaction rates from the $^{3}S_{1}$
resonances at 312 keV and 962 keV, respectively, and non-resonance
transition $^{3}P_{1}\longrightarrow $ $^{3}P_{0}$ with respect to the
reaction rate of $^{15}$N($p,\protect\gamma $)$^{16}$O, as a function of
astrophysical temperature. The resonances are identified with the c.m.
energy in keV.}
\label{Fig3}
\end{figure}
\begin{figure}[h]
\centering
%\includegraphics[width=10.0cm]{Fig3.png} %
\includegraphics[width=8.0cm]{graph4.png}
\caption{(Color online) The dependence of the ratio of the proton radiative
capture on $^{15}$N reaction rate from NACRE \protect\cite{Angulo1999}
(curve 1), \protect\cite{LeBlanc2010} (curve 2), NACRE II \protect\cite%
{Xu2013} (curve 3) and the present calculation on astrophysical temperature
in the range of 0.01$T_{9}$--10$T_{9}$. The shaded areas within the dashed
curves represent the uncertainties from NACRE and NACRE II. }
\label{Fig4}
\end{figure}
temperature. Such presentation is useful to quickly understand the relevance
of each transition at a given temperature. At 0.01$T_{9}$ the fractional
contribution from the 312 keV resonance is 71\%, while the fractional
contributions of the 962 keV resonance and non-resonance transition $%
^{3}P_{1}\longrightarrow $ $^{3}P_{0}$ are 16\% and 13\%, respectively.
However, in contrast, at temperature 10$T_{9}$ the fractional contribution
from the 962 keV resonance is 89\% and contributions of 312 keV resonance
and $^{3}P_{1}\longrightarrow $ $^{3}P_{0}$ transition are commensurate: 6\%
and 5\%, respectively. The $E1$ transitions from 312 keV and 962 keV
resonances have a maximal fractional contributions 95\% and 93\% at
temperature 0.4$T_{9}$ and 4.1$T_{9}$, respectively. The fractional
contribution from non-resonance transition $^{3}P_{1}\longrightarrow $ $%
^{3}P_{0}$ increases with the decrease of the energy. Thus, at the
temperatures range between $0.01T_{9}-10T_{9}$ the $^{15}$N($p,\gamma $)$%
^{16}$O reaction rate is dominated by the tails of the resonances at energy
312 keV and 962 keV in the c.m.

The interference of the $^{3}S_{1}$(312) and $^{3}S_{1}$(962) resonances
requires the special consideration. The solid line in the insert in Fig. \ref%
{Fig3} shows the fractional contribution of the interference term between
the $^{3}S_{1}$(312) and $^{3}S_{1}$(962) resonances to the total reaction
rate. In the Gamow CNO window $T_{9}=0.01-0.03$, which is shown in Fig. \ref%
{Fig6}, the contribution of the interference term into the total reaction
rate is 41\% up to 0.1$T_{9}$. In the temperature interval $T_{9}=0.01-0.1$
the contribution of the interference term to the total reaction rate is $%
\sim $40\%, while at $T_{9}=3-10$ the contribution of this term does not
exceed $\sim $11\%. For the stellar CNO temperature range $T_{9}=0.1-0.5$
which is shown in Fig. \ref{Fig7}, the contribution of the interference term
is dropping from $\sim $40\% to $\sim $12.5 \%. The destructive interference
is observed in the range from 0.89$T_{9}$ to 1.36$T_{9}$, but its role is on
the level of $\sim $2\%. Starting from the end of this interval up to 10$%
T_{9}$ the moderate increasing of the constructive interference is observed
from zero up to 11\%.

The dependence of the reaction rate of the $^{15}$N($p,\gamma $)$^{16}$O
radiative capture as a function of temperature for astrophysical temperature
between 0.01$T_{9}$ and 10$T_{9}$ is shown in Fig. \ref{Fig3}. Results of
the reaction rate calculations using the parameters' sets I $-$ III for the
potentials from Table \ref{tab:Table1_1} show that the differences of
reaction rates are negligible. Thus, while the astrophysical factor is
sensitive to parameters of the potential, the latter do not affect the
reaction rate. Results for the reaction rates for sets I $-$ III are
coincident and are shown by a single solid curve in Fig. \ref{Fig4}. The
reaction rates of $^{15}$N($p,\gamma $)$^{16}$O were reported earlier in
Refs. \cite{Angulo1999,LeBlanc2010,Xu2013}. We normalized the reaction rate
obtained within the $R-$matrix approach \cite{LeBlanc2010}, NACRE \cite%
{Angulo1999} and NACRE II \cite{Xu2013} reaction rates by dividing the
corresponding data on the reaction rate obtained in the present calculation.
In Fig. \ref{Fig4} are shown the dependencies of these ratios as the
function of astrophysical temperature. One can see the agreement between the
reaction rate \cite{LeBlanc2010} and our calculation. It should be noted the
agreement is also observed for the astrophysical factor: the range of our
results for $S(0)$ $34.7\leq S(0)\leq 40.4$ keV$\cdot $b and from $37\leq
S(0)\leq 42.2$ keV$\cdot $b from \cite{LeBlanc2010} are overlaping. In
calculations of $S(0)$ in \cite{LeBlanc2010} are used the ANC about 3 times
larger than the experimental value \cite{Mukhamedzhanov2008}. One can
conclude that the reaction rate is weakly responsive to the value of $S(0)$.

One can see the significant discrepancies in the reaction rates,
particularly, for NACRE \cite{Angulo1999} and NACRE II \cite{Xu2013} data at
temperatures 0.1$T_{9}$ and $T_{9}$, respectively, where the ratios are
reaching the maximums. Moreover, there is a significant disagreement between
the $^{15}$N($p,\gamma $)$^{16}$O reaction rate obtained in our calculations
and NACRE \cite{Angulo1999} and NACRE II \cite{Xu2013} data. However, it
should be noted that NACRE and NACRE II data for the $^{15}$N($p,\gamma $)$%
^{16}$O reaction rate are obtained taking into account cascade transitions,
which consideration are out of the scope of present paper.

The results of $R-$matrix calculations of the reaction rate \cite%
{LeBlanc2010} was parameterized in the form
\begin{equation}
N_{A}\left\langle \sigma \nu \right\rangle =\frac{a_{1}10^{9}}{T_{9}^{2/3}}%
\exp \left[ a_{2}/T_{9}^{1/3}-\left( T_{9}/a_{3}\right) ^{2}\right] \left[
1.0+a_{4}T_{9}+a_{5}T_{9}^{2}\right] +\frac{a_{6}10^{3}}{T_{9}^{3/2}}\exp
\left( a_{7}/T_{9}\right) +\frac{a_{8}10^{6}}{T_{9}^{3/2}}\exp \left[
a_{9}/T_{9}\right]  \label{analytical}
\end{equation}%
and calculations with the parameters from \cite{LeBlanc2010} brought us to $%
\chi ^{2}=20.8$. However, by varying the parameters, we get a much smaller $%
\chi ^{2}=0.4$. The corresponding parameters are given in the first column
in Table \ref{tab:Table2} in Appendix. Parametrization coefficients of the
reaction rate obtained in framework of MPCM for the analytical expression (%
\ref{analytical}) with the parameters from Table \ref{tab:Table1_1} are
presented in Appendix and are leading to $\chi ^{2}=0.084$, $\chi ^{2}=0.086$%
, and $\chi ^{2}=0.09$ for the sets I, II and III, respectively.
%column in Table \ref{tab:Table2} is leading to $\chi ^{2}=0.1$.
%An approximation of MPCM reaction rate, made by a curve of the same kind
%with parameters from the third column of Table. \ref{tab:Table2} and leading t%

\subsection{Comparison of rates for proton capture reactions on nitrogen
isotopes}

There are two stable nitrogen isotopes $^{14}$N and $^{15}$N and all other
radioisotopes are short-lived. Among short-lived isotopes, longest-lived are
$^{12}$N and $^{13}$N with a half-life of about 11 ms and 9.965 min,
respectively, and they are of nuclear astrophysics interest. A radiative
proton capture on nitrogen isotopes in the reactions $^{12}$N($p,\gamma $)$%
^{13}$O, $^{13}$N($p,\gamma $)$^{14}$O, $^{14}$N($p,\gamma $)$^{15}$O and $%
^{15}$N($p,\gamma $)$^{16}$O produces the short-lived $^{13}$O, $^{14}$O $%
^{15}$O isotope with a half-life of $\sim $ 6 ms, $\sim $ 71 s and $\sim $
122 s, respectively, and a stable $^{16}$O nucleus. These radiative capture
reactions caused by the electromagnetic interaction are significantly slower
than reactions induced by the strong interactions. Therefore, these slow
reactions control the rate and time of cycles of oxygen isotopes
nucleosynthesis at particular astrophysical temperatures.

\begin{figure}[h]
\centering
\includegraphics[width=8.0cm]{graph5insert.png}
\caption{(Color online) The reaction rates of the radiative proton capture
on nitrogen isotopes leading to the production of oxygen isotopes as a
function of astrophysical temperature. The insert shows the fractional
contributions from $^{12}$N($p,\protect\gamma $)$^{13}$O, $^{13}$N($p,%
\protect\gamma $)$^{14}$O, $^{14}$N($p,\protect\gamma $)$^{15}$O with
respect to the $^{15}$N($p,\protect\gamma $)$^{16}$O reaction rate as a
function of astrophysical temperature.}
\label{Fig5}
\end{figure}

The authors of Ref. \cite{Weischer1989} suggested and discussed three
alternative paths of rapid processes of the CNO cycle leading to the
formation of $^{14}$O through the breakout reactions $^{9}$C($\alpha ,p$)$%
^{12}$N and $^{11}$C($p,\gamma $)$^{12}$N. In these three branches of
subsequent reaction sequences involve $^{12}$N($p,\gamma $)$^{13}$O, $^{13}$%
N($p,\gamma $)$^{14}$O reactions. Thus, these processes are of particular
interest for nuclear astrophysics. In the framework of the MPCM\ the
radiative proton capture on nitrogen isotopes $^{12}$N, $^{13}$N, and $^{14}$%
N were investigated \cite{DubN12,DubN14,PRCDTKB2020}. Because the reactions $%
^{12}$N($p,\gamma $)$^{13}$O, $^{13}$N($p,\gamma $)$^{14}$O, and $^{14}$N($%
p,\gamma $)$^{15}$O and present study of $^{15}$N($p,\gamma $)$^{16}$O are
considered on the same footing within the MPCM, it is useful to compare the
reaction rates to understand the relevance of each process at a given
astrophysical temperature.

The radiative proton $^{12}$N($p,\gamma $)$^{13}$O, $^{13}$N($p,\gamma $)$%
^{14}$O, $^{14}$N($p,\gamma $)$^{15}$O, $^{15}$N($p,\gamma $)$^{16}$O
processes have the same Coulomb barrier and, as follows from Eq. (\ref{Rate1}%
), the reaction rates will differ only due to the different values of the $%
S(E)$ and reduced mass $\mu $ of interacting particles in the entrance
channel. The reduced masses of the pairs $p^{12}$N, $p^{13}$N, $p^{14}$N,
and $p^{15}$N are always less than the proton mass and are within the range $%
0.9294$ amu $\leq \mu \leq 0.9439$ amu Therefore, the influence of the
reduced mass on the reaction rates of the proton capture on nitrogen
isotopes is negligible and can be omitted. Therefore, the rates of these
processes completely depends on the reaction $S-$factor. Figure \ref{Fig5}
gives an overview of the reactions rates for typical CNO temperature and
explosive hydrogen burning scenarios. The $^{15}$N($p,\gamma $)$^{16}$O
reaction is the fastest one with the biggest rate up to $T_{9}\sim 0.175$
and $p^{14}$N is the slowest process up to $T_{9}\sim 0.1$ and it controls
the rate and time of nucleosynthesis cycles. One should notice that $^{15}$N(%
$p,\gamma $)$^{16}$O rate becomes the dominate one at temperature explosive
hydrogen burning scenarios in stars. The analysis of the result presented in
the insert in Fig. \ref{Fig5} leads to the conclusion that only in the
temperature windows $0.18\lesssim T_{9}\lesssim 1.14$ and $0.66\lesssim
T_{9}\lesssim 3$ the reaction $^{15}$N($p,\gamma $)$^{16}$O is slower than $%
^{13}$N($p,\gamma $)$^{14}$O and $^{14}$N($p,\gamma $)$^{15}$O reactions,
respectively. Hence this slow reaction control the rate and time of cycles
of nucleosynthesis.

It is useful to show reaction rates of proton radiative capture. The
radiative hydrogen burning induced nucleosynthesis at specific temperatures
has the Gamow peak energy \cite{Fowler1975,Iliadis2015}
\begin{equation}
E_{0}=\left[ \frac{\pi ^{2}}{\hbar ^{2}}\left( Z_{1}Z_{2}e^{2}\right) ^{2}%
\frac{\mu }{2}\left( k_{B}T\right) ^{2}\right] ^{\frac{1}{3}}
\label{GamovE0}
\end{equation}%
which is defined by the condition $\frac{d}{dE}f_{G}(E,T)=0$, where $%
f_{G}(E,T)=e^{-2\pi \eta }\exp \left( -\frac{E}{k_{B}T}\right) $ is a Gamow
function. In the case for the proton and nitrogen isotopes in the entrance
channel $Z_{1}=1$ and $Z_{2}=7$ for (\ref{GamovE0}) in keV for temperature $%
T_{9}$ one obtains
\begin{equation}
E_{0}=466.4353\left[ \mu T_{9}^{2}\right] ^{\frac{1}{3}},  \label{GamovE}
\end{equation}%
and the effective energy range determined by the Gamow range $\Delta E_{G}$
(in keV) around the Gamow energy $E_{0}$ is
\begin{equation}
\Delta E_{G}=452.9821\left[ \mu T_{9}^{5}\right] ^{\frac{1}{6}}.
\label{GamovD}
\end{equation}%
\begin{figure}[h]
\centering
\includegraphics[width=12.0cm]{Fig5.png}
%\includegraphics[width=7.0cm]{graph4c.png}
\caption{(Color online) ($a$) Dependencies of reaction rates of the
radiative proton capture on nitrogen isotopes on astrophysical temperature
in the range of 0.01$T_{9}-0.03T_{9}$ ($b$)\ The stellar temperatures\ as a
function of the Gamow energy for CNO cycle $^{12}$N($p,\protect\gamma $)$%
^{13}$O, $^{13}$N($p,\protect\gamma $)$^{14}$O, $^{14}$N($p,\protect\gamma $)%
$^{15}$O and $^{15}$N($p,\protect\gamma $)$^{16}$O reactions. }
\label{Fig6}
\end{figure}

\begin{figure}[h]
\centering
\includegraphics[width=12.0cm]{Fig6.png}
%\includegraphics[width=7.0cm]{graph4e.png}
\caption{(Color online) ($a$) Dependencies of reaction rates of the
radiative proton capture on nitrogen isotopes on astrophysical temperature
in the range of 0.1$T_{9}-0.5T_{9}$ ($b$)\ The stellar temperatures\ as a
function of the Gamow energy for CNO cycle $^{12}$N($p,\protect\gamma $)$%
^{13}$O, $^{13}$N($p,\protect\gamma $)$^{14}$O, $^{14}$N($p,\protect\gamma $)%
$^{15}$O and $^{15}$N($p,\protect\gamma $)$^{16}$O reactions. }
\label{Fig7}
\end{figure}
Thermonuclear reactions occur mainly over the Gamow energy window from $%
E_{0}-\Delta E_{G}/2$ to $E_{0}+\Delta E_{G}/2$ except in the case of narrow
resonances \cite{Iliadis2015}. From Eqs. (\ref{GamovE}) and (\ref{GamovD})
is clear that the Gamow's peak energies and ranges for $^{12}$N($p,\gamma $)$%
^{13}$O, $^{13}$N($p,\gamma $)$^{14}$O, $^{14}$N($p,\gamma $)$^{15}$O, $%
^{15} $N($p,\gamma $)$^{16}$O reactions are completely determined by the
astrophysical temperature. The variation of the reduced mass within $0.9294$
amu $\leq \mu \leq 0.9439$ amu change the Gamow's peak energy and the energy
range only within 0.5\% and 0.3\%, respectively.

It is useful to present the reaction rate for a particular temperature range
along with the Gamow window of CNO reactions for the radiative proton
capture on nitrogen isotopes. The corresponding results of calculations are
shown in Figs. \ref{Fig6} and \ref{Fig7}. It should be noticed that the main
difficulty in determining reliable reaction rates of $^{12}$N($p,\gamma $)$%
^{13}$O, $^{13}$N($p,\gamma $)$^{14}$O, $^{14}$N($p,\gamma $)$^{15}$O, $%
^{15} $N($p,\gamma $)$^{16}$O reactions for the CNO cycles is the
uncertainty in the very low cross-sections at the Gamow range. Developments
within the low-energy underground accelerator facility LUNA in the Gran
Sasso laboratory \cite{Costantini2009} and recent improvements in the
detection setup \cite{Skowronski2023} makes taking direct measurements of
nuclear reactions near the Gamow range feasible. This advantage has been
demonstrated in the $^{14}$N($p,\gamma $)$^{15}$O reaction, which was
successfully measured down to energies of 70 keV at LUNA \cite{Lemut2006}.

\section{Conclusion}

We present the results of calculations and analysis of the astrophysical $S-$%
factor and reaction rate for the $^{15}$N($p,\gamma $)$^{16}$O reaction in
the framework of MPCM with forbidden states, including low lying $^{3}S_{1}$
resonances and $^{3}P_{1}\longrightarrow ^{3}P_{0}$ $M1$ transition. The
intercluster potentials of the bound state, constructed on the basis of
quite obvious requirements for the description of the binding energy and AC
in $p^{15}$N channel of the GS and the scattering potentials describing the
resonances, make it possible to reproduce the available experimental data
for the total cross-section of radiative proton capture on $^{15}$N nucleus
at astrophysical energies.

The interference of $^{3}S_{1}$(312) and $^{3}S_{1}$(962) resonances leads
to the %oscillatory behavior and the
significant increase of $S-$factor at the energies up to 300 keV. The
consideration of interfering $^{3}S_{1}$ resonances
%at energies 312 keV and 962 keV in the c.m. due to $E1$ transitions
and contribution of $^{3}P_{1}$ scattering wave in $p+^{\text{15}}$N channel
due to $^{3}P_{1}\longrightarrow $ $^{3}P_{0}$ $M1$ transition leads to
increase of the $S-$factor at energies up to 1200 keV. The extrapolation of
the $S-$factor at the low energy leads to 35.2 $\pm $ 0.5 keV$\cdot $b and
39.6 $\pm $ 0.8 keV$\cdot $b, depending on the value of the asymptotic
constant, which turned out to be within $34.7-40.4$ keV$\cdot $b. It is
elucidated the important role of the asymptotic constant for the $^{15}$N($%
p,\gamma $)$^{16}$O process, where the interfering $^{3}S_{1}$(312) and $%
^{3}S_{1}$(962) resonances give the main contribution to the cross-section.
A comparison of our calculation for $S-$factor with existing experimental
and theoretical data shows a reasonable agreement with experimental
measurements. Interestingly, the values of $S(0)$ are consistent with each
other, regardless of utilizing various $R-$matrix method approaches  \cite%
{Barker2008,Mukhamedzhanov2008,LeBlanc2010,Mukhamedzhanov2011,Xu2013,deBoer2013}
and present MPCM calculations.
%Interesting enough that $S(0)$ evaluated in different approaches employing $% R-$matrix method \cite{Barker2008,Mukhamedzhanov2008,LeBlanc2010,Mukhamedzhanov2011,Xu2013,deBoer2013}
%and present MPCM calculations, are consistent with each other.
The deviation for $S(0)$ in different approaches are within an accuracy the
main sources of uncertainties.

The reaction rate is calculated and parameterized by the analytical
expression at temperatures ranging from 0.01$T_{9}$ to 10$T_{9}$ and
compared with the existing rates. The reaction rate has negligible
dependence on the variation of AC, but shows strong impact of the
interference of $^{3}S_{1}$(312) and $^{3}S_{1}$(962) resonances, especially
at $T_{9}$ referring to the CNO Gamow windows. It was found the significant
discrepancies between the $^{15}$N($p,\gamma $)$^{16}$O reaction rate
presented in the NACRE \cite{Angulo1999} and NACRE II \cite{Xu2013} data
bases and our calculations.

We compare the reaction rates for$\ ^{12}$N($p,\gamma $)$^{13}$O, $^{13}$N($%
p,\gamma $)$^{14}$O, $^{14}$N($p,\gamma $)$^{15}$O, $^{15}$N($p,\gamma $)$%
^{16}$O reactions involved into different branches of the CNO\ cycle
obtained in the framework of the same model, MPCM, and determine temperature
windows, prevalence, and significance of each process. The comparison of the
reactions rate indicates which slow reactions control the rate and time of
cycles of oxygen isotopes nucleosynthesis at particular astrophysical
temperature. The have present temperature range along with the Gamow window
for CNO reactions for the radiative proton capture on nitrogen isotopes.\
This is useful as a guideline for experimental measurements involving
reactions of the radiative proton capture on nitrogen isotopes to obtain a
reliable extrapolation of $S(E)$ to astrophysical $S-$factor at zero energy.

%\begin{acknowledgement}
\textbf{Acknowledgments.} This research was supported by the Ministry of
Science and Higher Education of the Republic of Kazakhstan under the grant
AP09259174.
% "Study of the processes of thermonuclear combustion of hydrogen in the CNO cycle on the Sun and in stars" . %\end{acknowledgement}
\newpage \appendix

\section{Parameters for analytical parameterization}

Parameterization coefficients for the analytical expression (\ref{analytical}%
) of the $^{15}$N($p,\gamma $)$^{16}$O reaction rate date in Ref. \cite%
{LeBlanc2010} and obtained within the framework of MPCM are presented in
Table \ref{tab:Table2}. The sets of parameters I, II, and III from Table \ref%
{tab:Table1_1} lead to the three sets of parametrization coefficients for
Eq. (\ref{analytical}).
%The appendix fragment is used only once. Subsequent appendices can be created using the Section Section/Body Tag.
\begin{table}[h]
\caption{Parameters of analytical parametrization of the reaction rate $%
p^{15}$N capture. The parameters for the reaction rate presented in Ref.
\protect\cite{LeBlanc2010} with $\protect\chi ^{2}=0.4$ and results obtained
in the present MPCM calculation.}
\label{tab:Table2}
\begin{center}
\begin{tabular}{ccccc}
\hline\hline
& Parametrization for \cite{LeBlanc2010} & \multicolumn{1}{|c}{
Parametrization for MPCM} & \multicolumn{1}{|c}{Parametrization for MPCM} &
\multicolumn{1}{|c}{Parametrization for MPCM} \\
& $\chi ^{2}=0.4$ & \multicolumn{1}{|c}{Set I, $\chi ^{2}=0.084$} &
\multicolumn{1}{|c}{Set II, $\chi ^{2}=0.086$} & \multicolumn{1}{|c}{Set
III, $\chi ^{2}=0.09$} \\ \hline
$i$ & $a_{i}$ & \multicolumn{1}{|c}{$a_{i}$} & \multicolumn{1}{|c}{$a_{i}$}
& \multicolumn{1}{|c}{$a_{i}$} \\ \hline
1 & $0.4874952$ & $1.0375$ & $1.00436$ & $0.92832$ \\
2 & $-15.22289$ & $-15.41934$ & $-15.4231$ & $-15.42226$ \\
3 & $0.8597972$ & $2.19708$ & $2.17155$ & $2.16317$ \\
4 & $6.734083$ & $0.10981$ & $0.10166$ & $0.11569$ \\
5 & $-2.462556$ & $-0.01995$ & $-0.01655$ & $-0.01654$ \\
6 & $0.7971639$ & $1.67272$ & $1.72304$ & $1.68868$ \\
7 & $-2.930568$ & $-3.0594$ & $-3.06727$ & $-3.07264$ \\
8 & $3.224569$ & $4.38681$ & $4.20809$ & $3.97887$ \\
9 & $-11.00680$ & $-12.10183$ & $-12.09517$ & $-12.10662$ \\ \hline\hline
\end{tabular}%
\end{center}
\end{table}

\begin{thebibliography}{99}
\bibitem{Barnes1982} C. A. Barnes, D. D. Clayton, D. N. Schramm, Essays in
Nuclear Astrophysics. Presented to William A. Fowler. UK, Cambridge:
Cambridge University Press. 1982. 562p.

\bibitem{Arnould2020} M. Arnould and S. Goriely, Astronuclear Physics: A
tale of the atomic nuclei in the skies. Prog. Part. Nucl. Phys. \textbf{112}%
, 103766 (2020).

\bibitem{Wiescher2010} M. Wiescher, J. G\"{o}rres, E. Uberseder, G.
Imbriani, and M. Pignatari, The cold and hot CNO cycles. Annu. Rev. Nucl.
Part. Sci. \textbf{60}, 381 (2010).

\bibitem{Caciolli2011} A. Caciolli, et al., Revision of the $^{15}$N($%
p,\gamma $)$^{16}$O reaction rate and oxygen abundance in H-burning zones.
A\&A A\textbf{66}, 533 (2011).

\bibitem{Boeltzig2016} A. Boeltzig, et al., Shell and explosive hydrogen
burning. Nuclear reaction rates for hydrogen burning in RGB, AGB and Novae.
Eur. Phys. J. A \textbf{52,} 75 (2016).

\bibitem{Caughlan1962} G. R. Caughlan and W. A. Fowler, Mean lifetimes of
carbon, nitrogen, and oxygen nuclei in CNO bi-cycle. Astrophys. J. \textbf{%
136}, 453 (1962).

\bibitem{Caughlan1983} M. J. Harris, W. A. Fowler, G. R. Caughlan, and B. A.
Zimmerman, Thermonuclear reaction rates, III. Ann. Rev. Astron. Astrophys.
\textbf{21}, 165 (1983).

\bibitem{Caughlan1988} G. R. Caughlan and W. A. Fowler, Thermonuclear
reaction rates V. Atomic Data and Nucl. Data Tables \textbf{40}, 283 (1988).

\bibitem{Schardt1952} A. Schardt, W. A. Fowler, and C. C. Lauritsen, The
disintegration of N$^{15}$ by protons, Phys. Rev. \textbf{86}, 527 (1952).

\bibitem{Hebbard} D. F. Hebbard, Proton capture by $^{15}$N. Nucl. Phys.
\textbf{15}, 289 (1960).

\bibitem{Rolf1974} C. Rolfs and W. S. Rodney, Proton capture by $^{15}$N at
stellar energies. Nucl. Phys. A \textbf{235,} 450 (1974).

\bibitem{Brochard1973} F. Brochard, P. Chevallier, D. Disdier, and F.
Scheibling, \'{E}tude des d\'{e}sexcitations \'{e}lectromagn\'{e}tiques des
niveaux 1- situ\'{e}s \'{a} 12.44 et 13.09 MeV dans le noyau $^{16}$O. J.
Phys. \textbf{34}, 363 (1973).

\bibitem{Bemmerer2009} D. Bemmerer, et al., Direct measurement of the $^{15}$%
N($p,\gamma $)$^{16}$O total cross section at novae energies. J. Phys. G:
Nucl. Part. Phys. \textbf{36}, 045202 (2009).

\bibitem{LeBlanc2010} P. J. LeBlanc, et al., Constraining the $S-$factor of $%
^{15}$N($p,\gamma $)$^{16}$O at astrophysical energies. Phys. Rev. C \textbf{%
82}, 055804 (2010); Phys. Rev. C \textbf{84}, 019902 (2011).

\bibitem{Imbriani2012} G. Imbriani, et al., Measurement of $\gamma $ rays
from $^{15}$N($p,\gamma $)$^{16}$O cascade and $^{15}$N($p,\alpha _{1}\gamma
$)$^{12}$C. Phys. Rev. C \textbf{85}, 065810 (2012).

\bibitem{Angulo1999} C. Angulo et al., A compilation of charged particle
induced thermonuclear reaction rates. Nucl. Phys. A \textbf{656}, 3 (1999).

\bibitem{Xu2013} Y. Xu, et al. NACRE II: an update of the NACRE compilation
of charged-particle-induced thermonuclear reaction rates for nuclei with
mass number A$<16$. Nucl. Phys. \textbf{918}, 61 (2013).

\bibitem{Barker2008} F. C. Barker, $^{15}$N($p,\gamma _{0}$)$^{16}$O $S-$%
factor. Phys. Rev. C \textbf{78}, 044612 (2008).

\bibitem{Mukhamedzhanov2008} A. M. Mukhamedzhanov et al., New astrophysical
S factor for the $^{15}$N($p,\gamma $)$^{16}$O reaction via the asymptotic
normalization coefficient (ANC) method. Phys. Rev. C \textbf{78}, 015804
(2008).

\bibitem{deBoer2013} R. J. deBoer, J. G\"{o}rres, G. Imbriani, P. J.
LeBlanc, E. Uberseder, and M. Wiescher, $R-$matrix analysis of $^{16}$O
compound nucleus reactions, Phys. Rev. C \textbf{87}, 015802 (2013).

\bibitem{Dubovichenko2014} S. B. Dubovichenko and A. V.
Dzhazairov-Kakhramanov, Study of the neutron and proton capture reactions $%
^{10,11}$B($n,\gamma $), $^{11}$B($p,\gamma $), $^{14}$C($n,\gamma $), and $%
^{15}$N($p,\gamma $) at thermal and astrophysical energies. Int. J. Mod.
Phys. E \textbf{23}, 1430012 (2014).

\bibitem{Mukhamedzhanov2011} A. M. Mukhamedzhanov, M. L. Cognata, and V.
Kroha, Astrophysical $S-$factor for the $^{15}$N($p,\gamma $)$^{16}$O
reaction. Phys. Rev. C \textbf{83}, 044604 (2011).

\bibitem{Tian2000} X. Z. Li, J. Tian, M. Y. Mei, and C. X. Li, Sub-barrier
fusion and selective resonant tunneling. Phys. Rev. C \textbf{61}, 024610
(2000).

\bibitem{Khan2022} S. H. Mondal and Md. A. Khan, Study of fusion
cross-section and astrophysical S-factor for $p+$ $^{15}$N and $\alpha +$ $%
^{12}C$ at sub-barrier energy. Int. J. Mod. Phys. E \textbf{31}, 2250045
(2022).

\bibitem{SonNP2022} S. Son, S.-I. Ando, and Y. Oh, Determination of
astrophysical $S-$factor for $^{15}$N($p,\gamma $)$^{16}$O at low-energies
within effective field theory. New Physics: Sae Mulli \textbf{72}, 291
(2022).

\bibitem{Son2022} S. Son, S.-I. Ando, and Y. Oh, Radiative proton capture on
$^{15}$N within effective field theory. Phys. Rev. C \textbf{106}, 055807
(2022).

\bibitem{PRCDTKB2020} S. B. Dubovichenko, R. Ya. Kezerashvili , N. A.
Burkova, and A. V. Dzhazairov-Kakhramanov, and B. Beisenov, Reanalysis of
the $^{13}$N($p,\gamma $)$^{14}$O\ reaction and its role in the stellar CNO
cycle. Phys. Rev. C \textbf{102}, 045805 (2020).

\bibitem{PRC2022DTKB} S. B. Dubovichenko, A. S. Tkachenko, R. Ya.
Kezerashvili, N. A. Burkova, and A. V. Dzhazairov-Kakhramanov, $^{6}$Li($%
p,\gamma $)$^{7}$Be reaction rate in the light of the new data of the
Laboratory for Underground Nuclear Astrophysics. Phys. Rev. C \textbf{105},
065806 (2022).

\bibitem{DubBook2015} S. B. Dubovichenko, Thermonuclear Processes in Stars
and Universe. Second English edition, expanded and corrected. Germany,
Saarbrucken: Scholar's Press. 2015. 332p.

\bibitem{DubBook2019} S. B. Dubovichenko, Radiative Neutron Capture.
Primordial Nucleosynthesis of the Universe. Berlin/Munich/Boston. Walter de
Gruyter GmbH. 2019. 310p.

\bibitem{Neudatchin1992} V. G. Neudatchin, et al., Generalized potential
model description of mutual scattering of the lightest $p^{2}$H, $^{2}$H, $%
^{3}$He nuclei and the corresponding photonuclear reactions. Phys. Rev. C
\textbf{45,} 1512 (1992).

\bibitem{Kukulin1983} V. I. Kukulin, V. G. Neudatchin, I. T. Obukhovsky, and
Yu. F. Smirnov, in \textit{Clustering Phenomena in Nuclei. Clusters as
subsystems in light nuclei. }Ed. by K. Wildermuth and P. Kramer. Springer,
Brawnschweig, Vol. 3, 4--155, 1983.

\bibitem{Ajzenberg1993} D. R. Tilley, H. R. Weller, C. M. Cheves, Energy
levels of light nuclei $A=16-17$. Nucl. Phys. A \textbf{564}, 1 (1993).

\bibitem{Sukhoruchkin2016} S. I. Sukhoruchkin and Z. N. Soroko, Excited
nuclear states. Sub. G. Suppl. I/25 A-F. Springer. 2016.

\bibitem{Ajzenberg1991} F. Ajzenberg-Selove, Energy level of light nuclei
A=13,14,15. Nucl. Phys. A \textbf{523}, 1 (1991).

\bibitem{Website1} http://physics.nist.gov/cgi-bin/cuu/Value?rp

\bibitem{Timofeyuk1990} A. M. Mukhamedzhanov and N. K. Timofeyuk,
Microscopic calculations of nucleon-separation vertex constants for 1$p-$%
nuclei. Sov. J. Nucl. Phys. \textbf{51,} 431 (1990).

\bibitem{Xu1994} H. M. Xu, et al., Overall normalization of the
astrophysical $S-$factor and the nuclear vertex constant for $^{7}$Be($%
p,\gamma $)$^{8}$B reactions. Phys. Rev. Lett. \textbf{73,} 2027 (1994).

\bibitem{Timofeyuk1995} A. M. Mukhamedzhanov, R. E. Tribble, and N. K.
Timofeyuk, Possibility to determine the astrophysical $S$ factor for the $%
^{7}$Be($p$,$\gamma $)$^{8}$B radiative capture from analysis of the $^{7}$%
Be($^{3}$He,$d$)$^{8}$B reaction. Phys. Rev. C \textbf{51}, 3472 (1995).

\bibitem{Mukhamedzhanov2001} A. M. Mukhamedzhanov, C. A. Gagliardi, and R.
E. Tribble, Asymptotic normalization coefficients, spectroscopic factors,
and direct radiative capture rates. Phys. Rev. C \textbf{63}, 024612 (2001).

\bibitem{Timofeyuk2003} N. K. Timofeyuk, R. C. Johnson, and A. M.
Mukhamedzhanov, Relation between proton and neutron asymptotic normalization
coefficients for light mirror nuclei and its relevance to nuclear
astrophysics. Phys. Rev. Lett. \textbf{91}, 232501 (2003).

\bibitem{Mukhamedzhanov2003} A. M. Mukhamedzhanov, et al., Asymptotic
normalization coefficients for $^{14}$N+$p\rightarrow $ $^{15}$O and the
astrophysical $S$ factor for $^{14}$N($p$,$\gamma $)$^{15}$O. Phys. Rev. C
\textbf{67}, 065804 (2003).

\bibitem{Timofeyuk2009} N. K. Timofeyuk, New insight into the observation of
spectroscopic strength reduction in atomic nuclei: implication for the
physical meaning of spectroscopic factors. Phys. Rev. Lett. \textbf{103},
242501 (2009).

\bibitem{Timofeyuk2013} N. K. Timofeyuk, Spectroscopic factors and
asymptotic normalization coefficients for 0 p-shell nuclei: Recent updates.
Phys. Rev. C \textbf{88}, 044315 (2013).

\bibitem{Mukhamedzhanov2014} R. E. Tribble, C. A. Bertulani, M. La Cognata,
A. M. Mukhamedzhanov, and C. Spitaleri, Indirect techniques in nuclear
astrophysics: a review. Rep. Prog. Phys. \textbf{77}, 106901 (2014).

\bibitem{Blokhintsev2021} L. D. Blokhintsev and D. A. Savin, Study of the
influence of different methods of taking into account the Coulomb
interaction on determining asymptotic normalization coefficients within the
framework of exactly solvable model. Phys. Atom. Nucl. \textbf{84}, 401
(2021).

\bibitem{Blokhintsev2022} A. M. Mukhamedzhanov and L. D. Blokhintsev,
Asymptotic normalization coefficients in nuclear reactions and nuclear
astrophysics. Eur. Phys. J. A \textbf{58}, 29 (2022).

\bibitem{Mukhamedzhanov2023} A. M. Mukhamedzhanov, Resonances in low-energy
nuclear processes and nuclear astrophysics and asymptotic normalization
coefficients: a review. Eur. Phys. J. A \textbf{59}, 43 (2023).

\bibitem{Plattner1981} G. R. Plattner and R. D. Viollier, Coupling constants
of commonly used nuclear probes. Nucl. Phys. A \textbf{365}, 8 (1981).

\bibitem{Blokhintsev1977} L. D. Blokhintsev, I. Borbey, E. I. Dolinskii,
Nuclear vertex constants. Phys. Part. Nucl. \textbf{8}, 1189 (1977).

\bibitem{Website2} http://cdfe.sinp.msu.ru/services/pnisearch.html

\bibitem{Yakovlev2010} D. G. Yakovlev, M. Beard, L. R. Gasques, and M.
Wiescher,\ Simple analytic model for astrophysical $S$ factors. Phys. Rev. C
\textbf{82}, 044609 (2010).

\bibitem{Wiescher2012} M. Wiescher, F. K\"{a}ppeler, and K. Langanke,
Critical reactions in contemporary nuclear astrophysics. Annu. Rev. Astron.
Astrophys. \textbf{50}, 165 (2012).

\bibitem{Bertulani2016} C. A. Bertulani and T. Kajino, Frontiers in Nuclear
Astrophysics. Prog. Part. Nucl. Phys. \textbf{89}, 56 (2016).

\bibitem{Famiano2020} M. Famiano, A. B. Balantekin, T. Kajino, M. Kusakabe,
K. Mori, and Y. Luo, Nuclear Reaction Screening, Weak interactions, and
r-process nucleosynthesis in high magnetic fields. Astrophys. J. \textbf{898}%
, 163 (2020).

\bibitem{Bertulani2020} T. Aumann and C. A. Bertulani, Indirect methods in
nuclear astrophysics with relativistic radioactive beams. Prog. Part. Nucl
Phys. \textbf{112}, 103753 (2020).

\bibitem{Casey2023} D. T. Casey, et al., Towards the first plasma-electron
screening experiment. Front. Phys. \textbf{10,} 1057603 (2023).

\bibitem{Spitaleri2016} C. Spitaleri, C. A. Bertulani, L. Fortunato, and A.
Vitturi, The electron screening puzzle and nuclear clustering. Phys. Lett. B
\textbf{755}, 275 (2016).

\bibitem{Dub2019nuc} S. B. Dubovichenko, N. A. Burkova, A. V.
Dzhazairov-Kakhramanova, R. Ya. Kezerashvili, Ch. T. Omarov, A. S.
Tkachenko, and D. M. Zazulin, Radiative $^{3}$He($^{2}$H,$\gamma $)$^{5}$Li
capture at astrophysical energy and its role in accumulation of $^{6}$Li at
the BBN. Nucl. Phys. A \textbf{987}, 46 (2019).

\bibitem{Assenbaum1987} H. J. Assenbaum, K. Langanke, and C. Rolfs, Effects
of electron screening on low-energy Fusion cross sections. Z. Phys. A Atomic
Nuclei, \textbf{327}, 451 (1987).

\bibitem{Huang2010} T. Huang, C. A. Bertulani, and V. Guimar\~{a}es,
Radiative capture of nucleons at astrophysical energies with single-particle
states. At. Data Nucl. Data Tables \textbf{96}, 824 (2010).

\bibitem{Wiescher1999} M. Wiescher, J. G\"{o}rres, and H. Schatz, Break-out
reactions from the CNO cycles. J. Phys. G: Nucl. Part. Phys. \textbf{25},
R133 (1999).

\bibitem{Iliadis2015} C. Iliadis, Nuclear Physics of Stars, 2nd ed.,
Wiley-VCH, Weinheim, 2015, 672p.

\bibitem{Weischer1989} M. Weischer, J. G\"{o}rres, S. Graff, L. Buchmann,
and F. -K. Thielemann, Hot $pp$ chains in low metallicity objects,
Astrophys. J. \textbf{343}, 352 (1989).

\bibitem{DubN12} S. B. Dubovichenko, N. A. Burkova, D. M. Zazulin, Reaction
rate of radiative $p^{12}$N capture. Nucl. Phys. A \textbf{1028}, 122543
(2022).

\bibitem{DubN14} S. Dubovichenko, N. Burkova, A. Dzhazairov-Kakhramanov, and
B. Beysenov, Reaction rate of $p^{14}$N$\longrightarrow $ $^{15}$O$\gamma $
capture to all bound states in potential cluster model. Int. J. Mod. Phys. E
\textbf{29}, 1930007 (2020).

\bibitem{Fowler1975} W. A. Fowler, G. R. Caughlan, and B. A. Zimmerman,
Thermonuclear reaction rates. II. Annu. Rev. Astron. Astrophys. \textbf{13},
69 (1975).

\bibitem{Costantini2009} H. Costantini, et al. LUNA: a laboratory for
underground nuclear astrophysics. Rep. Prog. Phys. \textbf{72}, 086301
(2009).

\bibitem{Skowronski2023} J. Skowronski, et al., Advances in radiative
capture studies at LUNA with a segmented BGO detector. J. Physi. G: Nucl.
Part. Phys. (2023). %DOI 10.1088/1361-6471/acb961

\bibitem{Lemut2006} LUNA Collaboration: Lemut A, et al., First measurement
of the $^{14}$N($p,\gamma $)$^{15}$O cross section down to 70 keV. Phys.
Lett. B \textbf{634}, 483 (2006).

%\bibitem{Lane1958} A.M. Lane, R.G. Thomas, $R-$matrix theory of nuclear reactions. Rev. Mod. Phys. \textbf{30}, 257 (1958).
\end{thebibliography}

\end{document}
