\documentclass[10pt,twocolumn,letterpaper]{article}
\usepackage{arxiv}
\usepackage{graphicx}
\usepackage{placeins}
\usepackage{amsmath}
\usepackage{amssymb}
\usepackage{booktabs}
\usepackage{array}
\usepackage{bm}
\usepackage[pagebackref,breaklinks,colorlinks]{hyperref}
\usepackage[capitalize]{cleveref}
\usepackage[title]{appendix}


\newcolumntype{x}[1]{>{\centering\arraybackslash\hspace{0pt}}p{#1}}

\DeclareMathOperator*{\argmax}{argmax}
\DeclareMathOperator*{\argmin}{argmin}

\crefname{section}{Sec.}{Secs.}
\Crefname{section}{Section}{Sections}
\Crefname{table}{Table}{Tables}
\crefname{table}{Tab.}{Tabs.}
\Crefname{equation}{Equation}{Equations}
\crefname{equation}{eq.}{eqs.}

\newcommand{\bx}{\bm{x}}
\newcommand{\bv}{\bm{v}}

\makeatletter
\renewcommand{\paragraph}{%
  \@startsection{paragraph}{4}%
  {\z@}{0.25em}{-1em}%
  {\normalfont\normalsize\bfseries}%
}
\makeatother

\begin{document}

\title{Real-time volumetric rendering of dynamic humans}

\author{Ignacio Rocco\quad\quad Iurii Makarov\quad\quad Filippos Kokkinos\quad\quad David Novotny\\ Benjamin Graham\quad\quad Natalia Neverova\quad\quad Andrea Vedaldi\\\,\\Meta AI}
\maketitle

\begin{abstract}
We present a method for fast 3D reconstruction and real-time rendering of dynamic humans from monocular videos with accompanying parametric body fits.
Our method can reconstruct a dynamic human in less than 3h using a single GPU, compared to recent state-of-the-art alternatives that take up to 72h.
These speedups are obtained by using a lightweight deformation model solely based on linear blend skinning, and an efficient factorized volumetric representation for modeling the shape and color of the person in canonical pose.
Moreover, we propose a novel local ray marching rendering which, by exploiting standard GPU hardware and without any baking or conversion of the radiance field, allows visualizing the neural human on a mobile VR device at 40 frames per second with minimal loss of visual quality.
Our experimental evaluation shows superior or competitive results with state-of-the art methods while obtaining large training speedup, using a simple model, and achieving real-time rendering. 
\end{abstract}

\section{Introduction}%
\label{sec:intro}

\begin{center}
\small
\setlength{\fboxrule}{1pt}
\setlength{\tabcolsep}{2pt}
\begin{tabular}{ccc}
\centeredtab{\begin{tikzpicture}
 \foreach \X [count=\Z]in {fig/teaser/input/000000.jpg,fig/teaser/input/000194.jpg,fig/teaser/input/000395.jpg,fig/teaser/input/000600.jpg,fig/teaser/input/000799.jpg,fig/teaser/input/000989.jpg,fig/teaser/input/001081.jpg}
 {\node[opacity=1] at (\Z/5,-\Z/2.5,0) {\fbox{\includegraphics[width=3cm]{\X}}};}
\end{tikzpicture} \\
Input: casually captured long video \\
\\ \\ \\
\includegraphics[width=5cm]{fig/teaser/teaser_plot3.png} \\
Output: jointly estimated camera poses \\ and local radiance fields
} &
\setlength{\tabcolsep}{1pt}
\renewcommand{\arraystretch}{0.8}
\begin{tabular}{cccc}
{\includegraphics[width=3.9cm]{fig/teaser/000020.jpg}} & 
{\includegraphics[width=3.9cm]{fig/teaser/000530.jpg}} & 
{\includegraphics[width=3.9cm]{fig/teaser/001190.jpg}} \\
\multicolumn{3}{c}{LocalRF (ours): high-quality novel view synthesis} \\
\rule{0pt}{2.35cm}{\includegraphics[width=3.9cm]{fig/teaser/rgb_2.jpg}} & 
{\includegraphics[width=3.9cm]{fig/teaser/rgb_53.jpg}} & 
{\includegraphics[width=3.9cm]{fig/teaser/rgb_119.jpg}} \\
\multicolumn{3}{c}{BARF~\cite{lin2021barf}: the estimated poses often fall into local minima for long sequences} \\
\rule{0pt}{2.35cm}{\includegraphics[width=3.9cm]{fig/teaser/color_002.jpg}} & 
{\includegraphics[width=3.9cm]{fig/teaser/color_053.jpg}} & 
{\includegraphics[width=3.9cm]{fig/teaser/color_119.jpg}} \\
\multicolumn{3}{c}{Mip-NeRF360~\cite{barron2022mipnerf360}: the spatial resolution is often limited throughout the video}
\end{tabular}
\end{tabular}%
\captionof{figure}{
\label{fig:teaser}
\textbf{High-quality novel view synthesis from a long casually captured video.} 
We jointly optimize camera poses and a scene representation using a progressive scheme that dynamically allocates local radiance fields (blue boxes).
Our method robustly handles casual hand-held captures, scales to processing arbitrarily long videos with limited memory usage, and maintains high resolution throughout the entire video.
}
\end{center}

Emerging technologies such as virtual and mixed reality make photorealistic 3D reconstruction increasingly relevant and impactful.
We can envisage a future in which, instead of taking pictures or movies with smartphone cameras, anyone will be able to capture and experience full $360^\circ$ holograms with equal quality, simplicity, and generality.
However, despite recent progress in methods such as neural rendering, we remain quite far from this objective.

Many neural rendering approaches assume static scenes and the availability of dozens if not hundreds of input views for each reconstructed scene~\cite{nerf,mipnerf,mipnerf360,FVS,SVS,neus}.
Many dynamic extensions of neural rendering~\cite{nerfies,hypernerf,dnerf,nsff,nguyen2021human} handle small deformations.
The most interesting content, however, is highly dynamic and often involves people.
This has motivated the development of specialised models that can account for the highly-deformable structure of such objects, but most of these still assume that scenes are captured from multiple cameras~\cite{neuralactor,neuralbody,tava, humannerf-shanghai}, which is incompatible with consumer applications.
When data is captured by a smartphone or pair of AR glasses, only a single slowly-varying viewpoint is available for 3D reconstruction.

A few methods such as \mbox{HumanNeRF-UW}~\cite{humannerf-uw}, \mbox{A-NeRF}~\cite{anerf} and NeuMan~\cite{neuman} can obtain photorealistic reconstructions of articulated humans from such monocular videos.
These methods use articulated human models such as SMPL~\cite{smpl} to define a \emph{scaffold} to model the object deformation, using a pre-trained predictor to obtain an initial estimate of the 3D body shape and pose.
Then, they capture the shape of the subject over time as a deformation of a \emph{canonical} radiance field, which is time-invariant.
Unfortunately, jointly learning a radiance and deformation fields with a neural network requires several days of optimization to reconstruct a single monocular video.
Furthermore, 3D video playback using the standard emission-absorption method is also slow due to the necessity of evaluating the neural network hundreds of times for each rendered pixel.

In this paper, we reconsider these architectures and we improve them significantly in terms of training and rendering speed, achieving fast training and real-time rendering.
This is achieved by three design decisions, discussed next.

First, instead of representing the radiance field with a neural network, we use a tensor-based decomposition of the field, inspired by~\cite{tensorf,pifu,eggan}.
This representation leads to faster convergence than when using an MLP\@.

Second, we directly leverage linear blend skinning (LBS) in SMPL to  model motion during training and rendering.
This differ from previous approaches that trained a module to invert SMPL~\cite{humannerf-uw,tava} or learned a 3D scene flow field using a separate MLP~\cite{humannerf-uw,neuralactor,neuman}, which is costly.
Instead, we map each 3D point to canonical space \emph{solely} by approximating the inverse LBS function using the inverse transformation of the closest SMPL face.
This approximation is good where it matters, namely when the queried 3D point is already in the vicinity of the body surface.

Third, we propose to extract a personalized human mesh template from the reconstructed human in the factorized radiance field.
During training, the exact shape of the person is not known, and the continuous volumetric representation allows to learn it though back-propagation.
However, neural rendering requires sampling the full estimated object bounding box, which is too slow for real-time rendering.
After the training is completed, we thus extract a mesh that is close to the surface of the reconstructed person.
This mesh can then be used as a tight scaffold to implement neural rendering efficiently on device.
Specifically, we describe a new local ray marching algorithm that carries out, using a custom shader, ray marching and emission absorption on a short distance from the mesh triangles as these are processed by the GPU rasteriser.
Compared to other recent hardware-accelerated neural rendering techniques, our technique avoids baking, \ie, sampling the radiance field values and storing them in a different specialized data structure, which is usually expensive and lossy.
Instead, it displays the \emph{unmodified} radiance field directly.

To summarize, our \textbf{contributions} are the following:
%
\textbf{1.} We show that adopting a factorized radiance field representation and a simple LBS-based deformation model allows for fast reconstruction ($24$ times faster than HumanNeRF-UW) with comparable or better rendering quality on the ZJU-Mocap~\cite{neuralbody} scenes.
%
\textbf{2.} We show that it is possible to extract a canonical mesh from the learned radiance field, together with a rig derived from the LBS model in SMPL, which approximates well each individual.
%
\textbf{3.} We show that, given these design choices, it is possible to use GPU hardware to renderer the deformable radiance field model preserving quality while achieving real-time playback (at 40 FPS on the mobile GPU of a consumer VR device), which is three orders of magnitude faster than the HumanNeRF-UW approach (0.05 FPS)\@.

\section{Related work}%
\label{sec:related_work}

\paragraph{Reconstructing humans from single images.}

Performing 3D reconstruction of humans from single images is ill-posed and requires leveraging prior information on human geometry and poses.
Several methods~\cite{smplify,romp,spin} use the SMPL model~\cite{smpl} as prior, but they result in only approximate reconstructions which are insufficient for VR/AR applications.
PIFu~\cite{pifu,pifuhd} directly predicts a volumetric representation of a human from a single RGB image using an MLP\@, but generalizes poorly beyond the training data distribution.
ARCH~\cite{arch,archpp} combines parametric models and implicit function representations to obtain higher fidelity reconstructions, which are also rigged and animatable, but achieve limited overall quality due to the fact that they use a \emph{single} image and cannot take advantage of video data.

\paragraph{Reconstructing humans from multiple videos.}

In order to obtain higher quality reconstructions, several methods~\cite{neuralactor, neuralbody, tava, humannerf-shanghai} leverage multiple synchronized videos.
Neural Actor~\cite{neuralactor} learns neural radiance fields~\cite{nerf} to model a human in canonical pose, and uses the inverse linear blend skinning (LBS) transform from SMPL to map between posed and canonical spaces.
TAVA~\cite{tava} uses a similar approach, but proposes to learn the inverse LBS weights and use Mip-NeRF~\cite{mipnerf} instead of vanilla NeRF\@.
Neural Body~\cite{neuralbody} attaches neural codes to the SMPL vertices, poses them, and converts them to a full volumetric representation of the radiance field using a 3D sparse CNN~\cite{sparseconvnet, minkowskiengine}.
HumanNeRF-ShanghaiTech~\cite{humannerf-shanghai} combines NeRF and the inverse LBS mapping (as in Neural Actor), but conditions the NeRF model on features sampled from neighbouring input views similar to IBRNet~\cite{ibrnet} for rigid scenes.
However, these methods require multiple views, and are thus inapplicable when only a monocular sensor is available.

\paragraph{Reconstructing humans from monocular videos.}

Recent methods considered 3D human reconstruction from \emph{monocular videos}.
Vid2Actor~\cite{vid2actor} extracts from SMPL a motion basis, uses it for inverse LBS, and trains two 3D CNNs to regress the inverse LBS weights in posed space and the density and color of the canonical human.
HumanNeRF-UW~\cite{humannerf-uw} extends Vid2Actor by replacing their voxel-grid representation with a coordinate MLP and further refine deformations using SMPL pose refinement and a non-rigid scene-flow MLP\@.
A-NeRF~\cite{anerf} computes codes of the posed 3D points that are relative to the SMPL skeleton and use those to construct a pose-dependent NeRF model (somewhat analogously to the approach used in Neural Body).
NeuMan~\cite{neuman} decomposes the scene into a static background component and the foreground dynamic human, and learns a different NeRF model for each component separately.
They use inverse LBS for warping points from posed to canonical space and a standard NeRF MLP for modelling the canonical human.

While powerful, these models are typically very slow to train (sometimes in the order of days on a single GPU) and to render, making them unsuitable for playback in mobile devices such as in VR headsets.

\paragraph{Real-time neural and volumetric rendering.}

Most of the works discussed above represent radiance fields using MLPs and use raymarching for rendering.
Hence, the MLPs need to be evaluated hundreds of time for each pixel, which requires seconds for each rendered image.
For real-time rendering, several methods have proposed to ``bake'' or cache the NeRF MLP~\cite{snerg, plenoctrees, fastnerf}.
However, such representations are memory intensive, do not support dynamic content, and require a high-end GPU for achieving real-time rendering.

Closer to ours, MobileNeRF~\cite{mobilenerf} leverages the fast triangle rasterization hardware in modern GPUs for real-time rendering of NeRF, but with substantial differences.
They also need to ``bake'' the radiance field into an especially-crafted mesh to avoid emission-absorption rendering.
Because they change the rendering model, they require a conversion step that involves further learning and approximations, which also does not support dynamic content.
In contrast, our technique renders in real time the \emph{original} dynamic radiance field, accelerating emission absorption.
Because we load the original model into the GPU, we do not require conversion, and we can preserve the shape and appearance details in the source radiance field with only minor approximations.
As far as we are aware of, our method is the first one to be able to render dynamic radiance fields in real-time on mobile hardware without resorting to baking.

\begin{figure*}
\centering
\includegraphics[width=\textwidth]{figures/model_blurred.pdf}
\caption{\textbf{Overview of the proposed method.}
During training, we shoot rays from the training cameras onto the scene and sample points withing the bounding box of the parametric human body mesh.
Points which are close to the mesh are warped to canonical space via inverse linear-blend-skinning (LBS), where the factorized volumetric representation is sampled for density and color, and whose values are used to perform raymarching.
After training, a canonical mesh is extracted from the learnt factorized volumetric representation. For real-time rendering, the canonical mesh is posed, rasterized, and used to guide the inverse skinning and local raymarching.}%
\label{fig:model}
\end{figure*}

\section{Fast reconstruction of dynamic humans}%
\label{sec:proposed_method}

Our method leverages recent progress in neural fields and monocular 3D reconstruction in order to allow for fast 3D reconstruction of humans from monocular videos and corresponding parametric SMPL fits.
During reconstruction, our method employs the following modules:
(i) a factorized volumetric representation that learns the shape and color of the human in canonical pose,
(ii) a deformation module which performs inverse linear blend skinning to map 3D points from posed space to canonical space.
In addition, we propose
(iii) a rasterization-based method for rendering the trained model in real-time.
Our full model is illustrated in \cref{fig:model}.
Each of these components is presented next in more detail.

\subsection{Factorized volumetric radiance fields}%
\label{sec:radiance_fields}

In this work, we leverage recent progress on neural radiance fields~\cite{nerf,tensorf} as a continuous and differentiable representation for modeling shape and color. Next, we review these models and present the factorized formulation that we adopt in this work.  

\paragraph{Radiance fields.}

An image $I : \Omega \rightarrow \mathbb{R}^3$ is a map from pixels $u\in\Omega = [0,\dots,W] \times [0,\dots,H]$ to corresponding colors $I(u) \in \mathbb{R}^3$.
The camera projection function $\pi : \mathbb{R}^3 \rightarrow \mathbb{R}^2$ is a map from 3D points $\bx \in \mathbb{R}^3$ in the world space to corresponding pixels (image points) $u \in \mathbb{R}^2$.
A \emph{radiance field} is a pair of functions mapping each 3D point $\bx$ to a corresponding density $\sigma(\bx) \in \mathbb{R}_+$ and corresponding color $c(\bx) \in \mathbb{R}^3$.

The color of a pixel $u$ is obtained by \emph{marching} along the ray $r_u$ originating from the camera center and propagating in the direction of pixel $u$ (interpreted as a 3D point).
Let $(\bx_i)_0^{N-1}$ be a sequence of $N$ samples along ray $r_u$ separated by steps $\Delta$.
The color of pixel $u$ is extracted from the radiance field via the following emission-absorption rendering equation:
\begin{equation}\label{eq:rend}
  I(u)
  =
  \sum_{i=0}^{N-1}
  (T_i - T_{i+1}) c(\bx_i),
  ~~~
  T_i = e^{-\Delta \sum_{j=0}^{i-1} \sigma(\bx_i)}.
\end{equation}
where $T_i$ is the accumulated transmittance up to $\bx_i$, namely the probability that a photon is transmitted from point $\bx_i$ back to the camera center without being absorbed.

\paragraph{Deformable radiance fields.}

The radiance field of~\cref{eq:rend} could be extended to represent the different poses of an articulated object by adding the object's pose parameters $\theta$ as an additional parameter of the density $\sigma(\bx;\theta)$ and color $c(\bx;\theta)$ functions.
Modelling such functions directly, however, would be statistically inefficient because it would not account for the fact that the different poses are not arbitrary, but related by geometric deformations.

Therefore, in this work we adopt more efficient approach with  considers \emph{canonical} density $\bar \sigma(\bar{\bx})$ and color $\bar c(\bar{\bx})$ fields that are pose-invariant and, separately, a \emph{posing} function $\bx = h(\bar{\bx}; \theta)$ mapping points $\bar{\bx}$ from the canonical space to their posed locations $\bx$.
The pose-dependent fields are the composition of the pose-invariant fields and of the \emph{inverse} of the posing function:
\begin{equation}\label{eq:def-rf}
  \sigma(\bx;\theta) = \bar \sigma(h^{-1}(\bx;\theta)),
  ~~~
  c(\bx;\theta) =  \bar c(h^{-1}(\bx;\theta)).
\end{equation}

\paragraph{Tensorial fields.}

We leverage the recent factorized volumetric neural field formulation from TensoRF~\cite{tensorf} to represent the shape and color of the \emph{canonical} human.

In order to model the field $(\sigma,c)$, we do not use the standard approach of adopting an MLP and positional encoding~\cite{nerf}, but use instead the TensoRF parameterization.
Specifically, we consider a \emph{voxel grid} of resolution $D\times H\times W$ and further decompose it as the sum of three matrix-vector products:
\begin{equation}\label{eq:tensorf-sigma}
\bar \sigma(\bar \bx)
\!=\!
\rho \left(
  \sum_{r=1}^{R_\sigma}\!
  M^{YX}_{r,\bar y, \bar x} v^{Z}_{r,\bar z}\!+\!  M^{YZ}_{r,\bar y, \bar z} v^{X}_{r,\bar x}\!+\!
  M^{XZ}_{r,\bar x, \bar z} v^{Y}_{r,\bar y}\!
\right)\!.
\end{equation}
Here $\bar{\bx} = (\bar x, \bar y, \bar z)$, $r$ is the channel index, and the sub-indices indicate the tensor sampling position.\footnote{In order to map coordinates $\bar{\bx} = (\bar x, \bar y, \bar z)$ defined in canonical space to indices in the respective tensors and matrices above, we use bilinear interpolation after remapping the nominal bounding box of the object to the corresponding grid dimensions.} In addition, $M^{YX}$ is a $R_\sigma \times H\times W$ tensor, $v^{Z}$ a ${R_\sigma}\times D$ matrix, and the other terms follow a similar pattern, so in total there are $R_\sigma(HW+HD+WD+H+W+D)$ parameters only, far less than $HWD$ as long as the number of components $R_\sigma \ll (HWD)^\frac{1}{3}$.
The activation function $\rho(a) = \log(1 + \exp(G a))$ is the \textit{softplus} operator with a fixed gain $G \approx 1/\Delta$, where $\Delta$ is the step between ray samples.

The color field $\bar c$ is defined in a similar manner, for each of the three RGB components, and uses the sigmoid as activation function.

The adoption of these factorized representation not only allows for faster training compared to the MLP-based models, but is \emph{particularly suitable} for fast real-time rendering by storing the different tensor factors as textures that are naturally interpolated by the  graphic shaders of GPUs (cf.~Sec.~\ref{sec:rt}).

\subsection{Skinning-based deformation module\label{sec:deformation}}

Several previous works use MLP-based deformation models~\cite{nerfies, hypernerf, humannerf-uw, neuralactor,neuman}, alone, or in conjunction with articulated deformations~\cite{neuralactor,humannerf-uw,neuman}.
Because these models are slow to evaluate, we propose to use only a linear blend skinning (LBS) articulated deformation model, and show that this simple deformation model can still achieve competitive results while being much simpler and allowing for real-time rendering.
We next review LBS and present our proposed approach.

\paragraph{Posing via blend skinning.}

Given that humans are articulated objects, we leverage the parametric SMPL model~\cite{smpl} and the linear blend skinning formulation which allows to easily build the posing function $h$, given the SMPL template mesh model and the pose parameters.
In particular, the SMPL model provides a template mesh $\mathcal{M}=(\mathcal{V},\mathcal{F})$ with points ${\bar{\bv}} \in \mathcal{V} \subset \mathbb{R}^3$ and triangular faces $f \in \mathcal{F}$.
In addition, the template mesh is associated with a \emph{skeleton}, which is a collection of bones $b\in \{ 1,\dots,B \}$ whose angles define the pose parameters.
Each template vertex $\bar{\bv}$ is~\emph{softly} attached to the different bones by its \emph{skinning weights} $w(\bar{\bv})\in\mathbb{R}_+^B$.
Then, the posed vertices can be obtained by an affine map
$
\bv = h(\bar{\bv}; \theta) = A(\bar{\bv} ; \theta) \bar{\bv}\
$,
where:
\begin{equation}\label{eq:fwd}
  A(\bar{\bv}; \theta)
=
\sum_{b=1}^B
w_b(\bar{\bv})
A_b(\theta),
\end{equation}
and $A_b(\theta)$ is the affine transform associated to bone $b$ for the pose $\theta$.
More details are provided in Appendix~\ref{appendix:lbs}.

\paragraph{Volumetric and inverse blend skinning.}

Skinned models like SMPL only define the skinning weights $w(\bar {\bv})$ for the vertices of the template mesh, which approximates the real object surface. %$S_0$.
However, because the radiance field is a volumetric representation, we need transformations to be defined \emph{around} the object surface, and not just \emph{on} the surface, and thus the skinning weights need to be extended to nearby 3D points.
Furthermore, in order to perform ray marching and compute the pose-invariant fields from \cref{eq:def-rf}, we need to map points $\bx$ from the posed space \emph{back} to canonical space, and therefore knowledge of the posing function $h(\bar{\bx}; \theta)$ is not enough; instead, we require its \emph{inverse}:
\begin{equation}
  \bar{\bx} \!=\! h^{-1}(\bx;\theta)
  \!=\!
  A(\bar{\bx} ; \theta)^{-1} \!\bx
  \!=\!\!
\left[
  \sum_{b=1}^B
w_b(\bar{\bx})~
\!A_b(\theta)
\right]^{-1}%
\!\!\!\!\!\!\!\cdot \bx.%
\label{eq:inv_lbs}
\end{equation}
Because $\bar{\bx}$ appears on the l.h.s.~and r.h.s.~of this expression, this defines $\bar{\bx}$ as the solution to an equation which cannot be solved in closed form.
Prior works~\cite{deng20nasa,chen21snarf:} addressed these issues by \emph{learning} the extended skinning weights and/or the inverse posing function, but these approaches increase the complexity of the model and therefore the training and rendering time.

\begin{figure}
\includegraphics*[width=0.9\columnwidth]{figures/rt}
\caption{\textbf{Real-time radiance fields via rasterization.} The posed face $f$ is used to determine the portion of the radiance field that is likely to affect given pixel $u$ and to map computations to the rasterization unit of the GPU.}\label{fig:rt}
\end{figure}

In contrast, we employ the approach introduced in~\cite{neuralactor}, where the inverse transformation of a point $\bx$ is approximated to that of the closest mesh vertex $\bar{\bv}_{i^*}$:
\begin{align}
    \bar{\bx} = h^{-1}(\bx;\theta) \approx A(\bar{\bv}_{i^*};\theta)^{-1} \bx,\nonumber\\
    \text{s.t.}~i^*=\operatornamewithlimits{argmin}_{i} \| \bx - \bv_i \|_2.
\end{align}
This approximation is only valid locally, and therefore only applied to ray points $\bx$ where the distance to the closest vertex is below a threshold $\tau$.
Otherwise, points are discarded and not used for raymarching, which is equivalent to setting
$
\sigma(\bx;\theta)
= \mathbf{1}_{[d(\bx, {\bv}_{i^*}) \leq \tau]} ~ \bar \sigma(h^{-1}(\bx;\theta)).
$

Contrary to prior work, we do not combine this deformation with any additional trainable model, in order to allow for real-time evaluation of the deformation model in mobile hardware. Our experimental results show that our method performs similarly to prior state-of-the-art despite of this simplified deformation model.
\FloatBarrier

\section{Real-time dynamic radiance fields}%
\label{sec:rt}

The factorized radiance field and deformation model presented in Sec.~\ref{sec:radiance_fields} and Sec.~\ref{sec:deformation} were chosen specifically to allow for real-time rendering, through a customized \emph{GPU shader} program that implements an efficient local emission-absortion raymarching.
Our approach \emph{avoids baking or conversion steps} and allows to replay the original radiance field with only minor approximations. 

\paragraph{Customized human mesh.}

Our proposed local raymarching is guided by an initial rasterization step of the posed mesh. In order to handle occlusions and object boundaries properly, we replace the SMPL mesh with one that more accurately approximates the body shape at rest, and transfer to the SMPL joints and blendshapes so that it is fully-rigged and amenable for posing.

We extract this mesh from the radiance field by:
(i) rendering frames with masks and depth-maps from the canonical model,
(ii) unprojecting these depth maps to form a dense point cloud and,
(iii) converting this point cloud onto a mesh. An example of the extracted mesh is shown on \cref{fig:model}.
Please refer to the Appendix~\ref{appendix:mesh} for further details.

\paragraph{Rasterization of the posed mesh.}

Without any guidance, volumetric rendering using~\cref{eq:rend} requries to sample hundreds of ray points for each generated pixel.
For an opaque object, this is extermely inefficient, as only a tiny fraction of those samples land close enough to the surface to make a significant contribution to the final color.
Instead, we resort to rasterization of the customized posed mesh as an initial step of our rendering pipeline, which provides guidance for performing a local volumetric rendering. 

In this case, the rasterizer iterates over the triangles $f\in\mathcal{F}$ of the mesh, quickly finding the pixels $u$ that are contained in them (\cref{fig:rt}).
The rasterizer also quickly computes the \emph{barycentric coordinates} $\alpha_i$ of the 3D face point $\bv_{fu} \in f$ that projects onto pixel $u$, defined as:
\begin{equation}\label{eq:bary}
  \bv_{fu} = \sum_{i=1}^3\alpha_i \bv_{fi},
  ~~~
  \alpha_i \geq 0,
  ~~~
  \sum_{i=1}^3 \alpha_i = 1,
\end{equation}
where $\bv_{fi}$ are the three mesh vertices.
Then, the color $I(u)$ of the pixel is obtained by computing the color $c(\bv_{fu})$ of point $\bv_{fu}$.
This calculation is done in parallel for all pixels using a \emph{shader program} running in the GPU. 

\paragraph{Local emission-absorption raymarching.} 
Instead of computing the color $c(\bv_{fu})$ by using an UV-texture mapping, as it is usually done for coloring meshes, we compute $c(\bv_{fu})$ by performing a local emission-absorption raymarching in the vicinity of point $\bv_{fu}$, as illustrated in \cref{fig:rt}. 

In practice, we first map the point $\bv_{fu}$ from the posed mesh to its counterpart in canonical space $\bar{\bv}_{fu}$ by applying the transformation $A(\bar{\bv}_{fu})^{-1}$. For this, $A(\bar{\bv}_{fu})^{-1}$ is efficiently and automatically approximated by the \emph{vertex shader} through barycentric interpolation, by storing the inverse transformation $A(\bar{\bv})^{-1}$ of each vertex $\bar{\bv}\in \mathcal{V}$ as a vertex property.

Once $\bar{\bv}_{fu}$ is computed, we can consider a small number of samples $\bar{\bx}_i$ (as illustrated in the right side of \cref{fig:rt}) to perform local emission-absorption raymarching. For this, the evaluation of the factors $M$ and $v$ of the density and color for fields can be done very efficiently on the GPU by mapping these 2D and 1D tensors to 2D and 1D \emph{textures maps} and using the native GPU texture sampling functions. 

Once the densities and colors are computed for all samples $\bar{\bx}_i$ in the local ray segment, \cref{eq:rend} is used to obtain the final pixel color $I(u)$.

\begin{figure*}
\centering
{\small
\setlength{\tabcolsep}{0pt}
\begin{tabular}{c}
 \includegraphics[width=\textwidth]{figures/zju_377_direct_new.png} \\
 \includegraphics[width=\textwidth]{figures/zju_393_direct_new.png} \\
 \includegraphics[width=\textwidth]{figures/zju_386_direct_new.png} \\
 \includegraphics[width=\textwidth]{figures/zju_394_direct_new.png} \\
\end{tabular}
{\setlength{\tabcolsep}{0pt}
\begin{tabular}{p{3.5cm}p{3.5cm}p{3.5cm}p{3.5cm}p{3.5cm}}
\centering (a) Ground-truth & 
\centering (b) NeuralBody~\cite{neuralbody} &
\centering (c) HumanNeRF-UW~\cite{humannerf-uw}  & 
\centering (d) Ours - reconstruction & 
\centering (e) Ours - real-time
\end{tabular}
}
}
\caption{\textbf{Qualitative results on ZJU-Mocap.}
We show a qualitative comparison between our proposed method and other state-of-the-art methods on test images.
Our method produces sharp images, perceptually similar to those of HumanNeRF-UW, despite presenting a simpler deformation model. Faces have been blurred for privacy reasons.}%
\label{fig:zju_qual}
\end{figure*}




\begin{table*}
\resizebox{\textwidth}{!}{
{\footnotesize
\setlength{\tabcolsep}{2pt}
\begin{tabular}{@{}lcccccccccccccccccc@{}}
\toprule
& \multicolumn{6}{c}{LPIPS*1000 $\downarrow$} & \multicolumn{6}{c}{PSNR $\uparrow$} & \multicolumn{6}{c}{SSIM $\uparrow$} \\
\cmidrule(lr){2-7} \cmidrule(lr){8-13} \cmidrule(lr){14-19} 
& \multicolumn{6}{c}{Sequence ID} & \multicolumn{6}{c}{Sequence ID} & \multicolumn{6}{c}{Sequence ID} \\
Method & 377   & 386   & 387   & 392   & 393   & 394   & 377   & 386   & 387   & 392   & 393   & 394   & 377 & 386 & 387 & 392 & 393 & 394 \\
\cmidrule(r){1-1} \cmidrule(lr){2-7} \cmidrule(lr){8-13} \cmidrule(lr){14-19}
Neural Body (14h)& 43.08 & 48.08 & 57.34 & 50.39 & 57.09 & 54.36 & 28.87 & 30.12 & 26.76 & 29.84 & 27.80 & 28.90 & 0.9609 & 0.9588 & 0.9462 & 0.9575 & 0.9472 & 0.9497 \\[1mm]
HumanNeRF-UW (72h)    & \underline{30.28} & \textbf{34.16} & \textbf{41.10} & \underline{36.24} & \textbf{40.30} & \underline{38.12} & \textbf{30.19} & 32.83 & 28.06 & 30.91 & 28.44 & \textbf{30.47} & \underline{0.9642} & \underline{0.9669} & \textbf{0.9551} & \underline{0.9641} & \textbf{0.9544} & \textbf{0.9567} \\
Ours - reconstruction (2-3h) & \textbf{27.72} & \underline{34.42} & \underline{43.44} & \textbf{35.68} & \underline{41.28} & \textbf{38.00} & \underline{30.00} & \textbf{32.90} & \textbf{28.08} & \textbf{31.08} & \textbf{28.51} & \underline{30.28} & \textbf{0.9710} & \textbf{0.9687} & \underline{0.9545} & \textbf{0.9645} & \underline{0.9535} & \underline{0.9564} \\ \midrule
Ours - real-time (40FPS)  & 32.76 & 35.01 & 42.87 & 37.79 & 41.86 & 40.53 & 28.42 & 32.19 & 27.76 & 30.35 & 27.91 & 29.46 & 0.9671 & 0.9677 & 0.9540 & 0.9635 & 0.9531 & 0.9555 \\
\bottomrule
\end{tabular}
}
}
\vspace{-0.5em}
\caption{\textbf{Results on the ZJU-mocap benchmark.}
Our method obtains the best reconstruction results on most of the scenes on the PSNR metric, and on half of the scenes, with respect to the LPIPS and SSIM metrics, while being significantly faster to train (2--3h vs 14h or 72h). Our real-time variant introduces a small performance drop but mantains similar visual quality (cf. Fig.~\ref{fig:zju_qual}).
}
\label{tab:zju}
\end{table*}

\paragraph{Limitations.}

The proposed local emission-absorption raymarching relies on the rasterization of the posed customized human mesh to define the local ray segments where ray marching is performed. This can cause small issues around occlusion boundaries, if the estimated mesh is either too small or too large. If the occluding part of the object is too small, the render will show a slight ``loss of mass" when rendered. If it is too large, this additional occluding part of the object will produce "black pixels", which should in reality show the background part of the object. 
This issue is minimized by making the scaffold as tight as possible.

\section{Experimental evaluation}%
\label{sec:experimental_evaluation}

In order to demonstrate the performance of our proposed method, we evaluate results on two different benchmarks, the ZJU-Mocap scenes and the NeuMan scenes.
In addition, we show the results of the proposed on-device real-time renderer and ablations on the color model employed by the factorized volumetric representation.


\begin{figure}[ht]
\centering
\includegraphics[width=\columnwidth]{figures/neuman_citron_direct_squeezed_new.png}
\includegraphics[width=\columnwidth]{figures/neuman_seattle_direct_squeezed_new.png}
\includegraphics[width=\columnwidth]{figures/neuman_lab_direct_squeezed_new.png}
{\small
{\footnotesize \setlength{\tabcolsep}{0pt}
\begin{tabular}{p{2.1cm}p{2.1cm}p{2.1cm}p{2.1cm}p{2.1cm}}
\centering Ground-truth & 
\centering NeuMan &
\centering Ours - recon. & 
\centering Ours - real-time
\end{tabular}
}
\vspace*{-4mm}
\caption{\textbf{Qualitative results on NeuMan scenes.} 
Our method obtains state-of-the-art visual quality, comparable to that of NeuMan, while being fast to train and render. Faces have been blurred for privacy reasons.}
\label{fig:qual_neuman}\vspace*{-3mm}
}
\end{figure}


\begin{table*}[ht]
\resizebox{1.01\textwidth}{!}{
{\footnotesize
\setlength{\tabcolsep}{1.55pt}
\begin{tabular}{@{}lcccccccccccccccccc@{}}
\toprule
& \multicolumn{6}{c}{LPIPS*1000 $\downarrow$} & \multicolumn{6}{c}{PSNR $\uparrow$} & \multicolumn{6}{c}{SSIM $\uparrow$} \\
\cmidrule(lr){2-7} \cmidrule(lr){8-13} \cmidrule(lr){14-19} 
& \multicolumn{6}{c}{Sequence name} & \multicolumn{6}{c}{Sequence name} & \multicolumn{6}{c}{Sequence name} \\
Method & {\scriptsize bike} & {\scriptsize citron} & {\scriptsize jogging} & {\scriptsize lab} & {\scriptsize parkinglot} & {\scriptsize seattle} & {\scriptsize bike} & {\scriptsize citron} & {\scriptsize jogging} & {\scriptsize lab} & {\scriptsize parkinglot} & {\scriptsize seattle} & {\scriptsize bike} & {\scriptsize citron} & {\scriptsize jogging} & {\scriptsize lab} & {\scriptsize parkinglot} & {\scriptsize seattle} \\
\cmidrule(r){1-1} \cmidrule(lr){2-7} \cmidrule(lr){8-13} \cmidrule(lr){14-19}
NeuMan & \textbf{44.65} & \textbf{28.00} & \textbf{41.53} & 43.38 & \textbf{44.23} & 24.23 & \textbf{26.73} & \textbf{27.88} & \textbf{26.31} & \textbf{28.37} & \textbf{27.43} & 27.80 & \textbf{0.9521} & \textbf{0.9633} & \textbf{0.9496} & \textbf{0.9593} & \textbf{0.9581} & 0.9687 \\
Ours - reconstruction & 49.63 & 29.60 & 42.00 & \textbf{42.18} & 49.24 & \textbf{19.76} & 26.37 & 26.60 & 25.88 & 28.29 & 24.12 & \textbf{28.43} & 0.9465 & 0.9588 & 0.9461 & 0.9574 & 0.9524& 0.\textbf{9722} \\\midrule
Ours - real-time & 48.04 & 31.43 & 41.86 & 42.71 & 56.54 & 21.14 & 25.86 & 26.12 & 25.13 & 27.39 & 23.12 & 27.86 & 0.9457 & 0.9580 & 0.9451 & 0.9552 & 0.9438 & 0.9706 \\
\bottomrule
\end{tabular}
}
}
\vspace{-0.5em}
\caption{\textbf{Quantitative results on the NeuMan benchmark.}
Our method obtains comparable reconstruction performance to NeuMan~\cite{neuman} for most sequences, with superior performance for \emph{seattle}, while having real-time rendering capabilities without significant loss in visual quality (cf. Fig.~\ref{fig:qual_neuman}). 
}%
\label{tab:neuman}
\end{table*}


\paragraph{Implementation details.}
We implement our model using PyTorch~\cite{pytorch}. In particular, we use 8 channels for modeling density and color ($R_\sigma {=} R_c {=} 8$) in the tensorial model. In addition, we use a coarse-to-fine training approach, increasing the voxel grid resolution $D\times H\times W$ several times during training, following~\cite{tensorf}. We begin training with a resolution $HWD=10^6$ and end with $HWD=4.096 \times 10^6$, scaled accordingly to the person's bounding box. At each training iteration, a batch of training rays $r_u$ is constructed by sampling six $32\times 32$ image patches, each centered around a foreground point. The models are trained for 30 epochs of 1000 iterations each, regardless of the number of training images. Before posing the SMPL template with the provided $\theta_{\text{data}}$ parameters from each dataset, we compute a small refinement $\theta_{\text{ref}}$ using a 4-layer MLP with 256 hidden units, following~\cite{humannerf-uw}. The final body parameters are thus $\theta = \theta_{\text{ref}} \circ \theta_{\text{data}}$. The factorized neural field and the pose correction MLP are trained jointly. For training, we employ a photometric loss, a sparsity regularization loss, as well as a perceptual loss (LPIPS). The whole training procedure takes between 2 and 3 hours on a single modern GPU. 

\paragraph{ZJU-Mocap benchmark.}

We first evaluate our method on the ZJU-Mocap dataset introduced in~\cite{neuralbody}.
We follow the evaluation protocol from~\cite{neuralbody} and, for each sequence, use images from 19 unseen cameras for testing, sampled at a 30 frames interval for the first 300 frames, giving 190 test images per sequence.
Following~\cite{humannerf-uw}, we evaluate on 6 sequences, identified by their sequence IDs (377, 386, 387, 392, 393, and 394). In \cref{tab:zju} and Fig.~\ref{fig:zju_qual} we present the quantitative and qualitative results of our method compared to NeuralBody~\cite{neuralbody} and HumanNeRF-UW~\cite{humannerf-uw}.
We present results both for the reconstruction phase (Sec.~\ref{sec:proposed_method}) using standard emission-absorption raymarching and for the proposed real-time rendering (Sec.~\ref{sec:rt}) that can run at 40FPS on mobile hardware. Our reconstructions obtain the best PSNR results for most of the scenes, and best LPIPS and SSIM results for half of the scenes, while being significantly faster to train than the baselines (2--3h for our method vs. 72h for HumanNeRF-UW and 15h for Neural Body). In addition, we observe that the rasterization-guided raymarching introduces a small performance loss, while allowing for real-time rendering. As shown in Fig.~\ref{fig:zju_qual}, the visual quality of our reconstuction and real-time variants is on-par or superior to previous methods.

\paragraph{NeuMan benchmark.}

In addition, we evaluate our method on the scenes from NeuMan~\cite{neuman}. The NeuMan method uses a NeRF~\cite{nerf} module to model the background scene, which is used for segmenting the person.
As our model does not explicitly model the background, we use XMeM~\cite{xmem} to segment the foreground person by scribbling the first frame of each sequence and propagating results automatically.
The results in \cref{tab:neuman} show that our reconstructions outperform NeuMan on two scenes according to LPIPS, and are slightly inferior for the other four scenes.
Nevertheless, qualitative results from \cref{fig:qual_neuman} show that the perceptual quality of our method is similar to that of NeuMan, even for our real-time renders.

\begin{table}[t]
    \centering
    \small
    \setlength{\tabcolsep}{4pt}
        \begin{tabular}{lcccc}
        \toprule
        {Method}     & {Params}  &  {GFLOPs}  &  {SSv2}   & {Views}\\
        \midrule
        \multicolumn{5}{l}{\hspace{-5pt}\textit{Effectiveness of Each Path}} \\ 
        \method w/o TA   &  5M  &  1016 & 53.7  &  16$\times$3$\times$1   \\
        \method w/o SA  &  8M  &   134  &   55.1   &    16$\times$3$\times$1    \\
        \midrule
        \multicolumn{4}{l}{\hspace{-5pt}\textit{Effectiveness of $T_S$}}    \\
        $T_S=8$   &  13M  &  642 & 69.3  &  16$\times$3$\times$1   \\
        $T_S=12$   &  13M  &  896 & 69.6  &  16$\times$3$\times$1  \\
        $T_S=16$   &  13M  &  1150 & 69.8  &  16$\times$3$\times$1   \\  \midrule
        \multicolumn{4}{l}{\hspace{-5pt}\textit{Effectiveness of scaling factors}}    \\
        $w$=$h$=$2 \;(T_G$=$16)$   &  13M  &  1752 & 66.4  &  64$\times$3$\times$1   \\
        $w$=$h$=$4 \;(T_G$=$4)$   &  13M  &  864 & 71.8  &  64$\times$3$\times$1  \\
        $w$=$h$=$8 \;(T_G$=$1)$   &  13M  &  642 & 61.5  &  64$\times$3$\times$1   \\  \midrule
        \rowcolor{Light}
        \textbf{\method}      &  13M &    642   &  69.3  &  16$\times$3$\times$1      \\
        \bottomrule
        \end{tabular}\vspace{-7pt}
        % }
    \caption{Performance with respect to variants of the components.
    }\vspace{-5pt}\label{tab:ablation}
    \end{table}

\paragraph{Model ablations.}

We also evaluate the effect of modifying the color model of the factorized volumetric representation.
In \cref{sec:radiance_fields} we proposed to model color directly, \ie outputting RGB values directly from the factorized volumetric representation.
Alternatively, we can have the factorized representation output spherical harmonic~\emph{coefficients} or color \emph{descriptors} which are then used to and compute the final RGB values via spherical harmonic functions or a small color MLP, respectively.
The results of using these different color models are presented in \cref{tab:ablation}.
We observe that the choice of the color model does not affect performance significantly.
We therefore propose to use direct color prediction (RGB values), for its simplicity compared to using spherical harmonics or MLP for color prediction, allowing for faster real-time rendering.
Finally, we observe that disabling the body pose correction also diminishes performance to a small extent.

\section{Conclusion}%
\label{sec:conclusions}

We have introduced a method capable of learning neural radiance fields of articulated humans from monocular videos.
By adopting a factorized radiance field model and a simple deformation model, we were able to cut down the training time by an order of magnitude or more compared to prior work, while matching these methods in visual quality.
Furthermore, we have introduced a novel approach to render these dynamic reconstructions in real-time on a mobile GPU. This is done through a rasterization-guided local raymarching which leverages a refined customized human mesh for rendering the neural radiance field without resorting to baking, and with minimal loss of quality. 
As far as we know, this is the first work to perform realtime rendering of dynamic humans in mobile hardware using radiance fields, and without resorting to baking. We hope this work will inspire other future work in this area.

{\small\bibliographystyle{ieee_fullname}\bibliography{shortstrings, bibliography}}
\clearpage

% %\newpage
\section{Alternative Definitions}\label{sec:other-definitions-short}
In this section, we discuss other potential definitions of Leximin approximation that might be considered intuitive.
\eden{removed ack. for anonymous submission}
% \footnote{We thank Sylvain Bouveret for suggesting definitions \ref{altDef:5} and \ref{altDef:6}.}.
For each alternative, we provide an example that illustrates why we believe it is inappropriate and a conclusion based on that example.
It should be noted that in order to avoid confusion, the error parameter $\gamma \in (0,1)$ is used in the alternative definitions (instead of $\beta$), to emphasize that these are only alternatives we do not use.


\begin{potentialDefinition}\label{altDef:2}
    A solution $x$ is a $(1-\gamma)$-approximately optimal if given a Leximin-optimal solution, $x^*$, there exists an integer $k \in [n]$ such that: 
    \begin{align*}
    \forall j < k: & \valBy{j}{x} \geq (1-\gamma) \cdot \valBy{j}{x^*}\\
    & \valBy{k}{x} > \valBy{k}{x^*}
    \end{align*}
\end{potentialDefinition}

\paragraph{Bad example def. \ref{altDef:2}:} Consider the following example with three objectives:
\begin{align*}
    \max \quad &\{f_i(x) = x_i \mid \forall 1 \leq i \leq 3 \} \\ \tag{E1}\label{eq:alt-def-eaxmple-1}
    s.t. \quad  & 99 x_1 + x_2 \leq 100\\
    &  x_3 \leq 100\\
    & x \in \mathbb{R}^3_{+}
\end{align*}
The Leximin optimal solution $x^*$ is $(1,1, 100)$ and therefore, by taking $k$ to be $2$, we get that any solution that its minimum objective value is at least $(1-\gamma)$ and its second-smallest objective value is more than $1$ is considered $(1-\gamma)$-approximately optimal Leximin solution.
For instance, consider $\gamma = 0.1$, the solution $(0.9, 1.1, 1.1)$ should be considered a $0.9$-approximately optimal according to this definition.
However, it is easy to see that this solution is quite bad for $f_2$ who can achieve $10.9$ (higher by a factor $> 9$) and very bad for $f_3$ who can achieve $100$ (higher by a factor $>90$).
And so, it seem reasonable to require that a good definition will consider as many objectives as possible.
% \erel{What objective values exactly? Do you mean: as many objectives as possible?}
% \eden{yes. To myself: this comment might be relevant to other places..}

\paragraph{Conclusion def. \ref{altDef:2}:} An appropriate definition should take into account as many objectives as possible.

\begin{potentialDefinition}\label{altDef:1}
    A solution $x$ is a $(1-\gamma)$-approximately optimal if for a Leximin-optimal solution, $x^*$, and for each $j = 1, \dots, n$ the following holds: 
    \begin{align*}
        \valBy{j}{x} \geq (1-\gamma) \cdot \valBy{j}{x^*} 
    \end{align*}
\end{potentialDefinition}

\paragraph{Bad example def. \ref{altDef:1}:} 
% An error in the first objective value might cause the other values to increase significantly.
Consider example \eqref{eq:alt-def-eaxmple-1} again.
Here, as the optimal solution is $(1,1, 100)$, any solution that yields at least $(1-\gamma,1-\gamma, (1-\gamma)\cdot 100)$.
However, considering $\gamma = 0.1$, $f_2$ can again achieve $9.1$ which is higher by a factor $> 100$ than the value it got $0.1$.

\paragraph{Conclusion def. \ref{altDef:1}:} An appropriate definition should consider the fact that an error in one objective might change the optimal value of other objectives.
As a consequence, another conclusion is that an appropriate definition should not consider the optimal solution at all.



\begin{potentialDefinition}\label{altDef:3}
    A solution $x$ is a $(1-\gamma)$-approximately optimal 
    if it satisfies the following requirements:
    \begin{enumerate}
        \item The objective-function with the smallest objective value achieves at least its maximum value times $(1-\gamma)$:
        \begin{align*}
            \valBy{1}{x} \geq (1-\gamma) \cdot \valBy{1}{x^*} 
        \end{align*}
        
        \item Given all the solutions that satisfies the first condition, let $m_2$ be the highest second-smallest objective value.
        The objective-function with the second-smallest objective value achieves at least the $m_2$ times $(1-\gamma)$.
        
        \item Given all the solutions that satisfies the former conditions, let $m_3$ be the highest third-smallest objective value.
        The objective-function with the third-smallest objective value achieves at least the $m_3$ times $(1-\gamma)$.
        
        \item and so on.
    \end{enumerate}
\end{potentialDefinition}

\paragraph{Bad example def. \ref{altDef:3}:}
Consider the following example with only two objectives:
\begin{align*}
    \max \quad &\{f_1(x) = x_1, f_2(x)=x_2\} \\
    s.t. \quad  & 99 x_1 + x_2 \leq 100\\
    & x \in \mathbb{R}^2_{+}
\end{align*}
The Leximin-optimal solution is $(1,1)$. Consider $\gamma = 0.1$, according to part (1) of this definition, all solutions in which the smallest objective value is at least $(1-\gamma)=0.9$ should be considered in order to determine $m_2$.
So, in this case, $m_2$ is determined to be $100 - 0.9 \cdot 99 = 10.9$.
Then, according to part (2), in order to be considered a $0.9$-approximately optimal, the second value must be at least $0.9 \cdot 10.9 = 9.81$.
But, even the exact Leximin optimal solution does not satisfy this requirement, so this cannot be considered an approximation to Leximin optimal.

In general, this definition has the disadvantage of favoring solutions that give the lowest bounds to the objective functions considered in the earlier steps,  since this may enable to increase the values of the higher objectives.
According to the Leximin nature, the most important thing is to make the worst-off player as happy as possible (and then the second worst-off and so on), therefore, we emphasize the importance of this characteristic also in the definition of the approximated version.

\paragraph{Conclusion def. \ref{altDef:3}:} An appropriate definition should also capture the Leximin optimal solutions, and maintain the Leximin nature whenever possible.

% \eden{I think this definition is actually equivalent to our current... need to think about it again}
% \begin{potentialDefinition}\label{altDef:4}
%     A solution $x$ is a $\gamma$-approximately-optimal Leximin solution if it can be viewed as the result of this process:
%     \begin{enumerate}
%         \item Choose a solution in which the objective-function with the smallest objective value achieves at least the maximum value minus $\gamma$:
%         \begin{align*}
%             \valBy{1}{x} \geq \valBy{1}{x^*} - \gamma
%         \end{align*}
%         Let $z_1$ be the value it achieves (i.e., $\valBy{1}{x})$.
        
%         \item Consider all the solutions in which the objective-function with the smallest objective value achieves at least $z_1$ and let $m_2$ be the highest second-smallest objective value.
%         Then, choose a solution in which the objective-function with the second-smallest objective value achieves at least the $m_2$ minus $\gamma$.
%         Let $z_2$ be the value it achieves.
%         \item Consider all the solutions in which the objective-function, the smallest objective value achieves at least $z_1$ and the second-smallest objective value achieves at least $z_2$, and let $m_3$ be the highest third-smallest objective value.
%         Then, choose a solution in which the objective-function with the third-smallest objective value achieves at least the $m_3$ minus $\gamma$.
        
%         \item and so on...
%     \end{enumerate}
% \end{potentialDefinition}

% \paragraph{Bad example def. \ref{altDef:3}:} Although in this definition, the Leximin optimal solution is also approximately-optimal as we wanted, another issue arises.

% \begin{itemize}
%     \item \textbf{Bad example:} two solutions that meet this definition, but one of them is strictly better (by more than $\gamma$) than the other from some point.
%     \item \textbf{Conclusion:} an appropriate (good?) definition should determine between two solutions if possible. 
% \end{itemize}

% -----------------------------------
% \subsection{others}
% (From the correspondence of Erel with Lemaitre and Bouveret)

%----------------------------------
% need to think about a corresponding def for mult...
\begin{potentialDefinition}\label{altDef:5}
    A solution $x$ is a $(1-\gamma)$-approximately optimal if for a Leximin-optimal solution, $x^*$, and for each $j = 1, \dots, n$: 
    % in the addive version was $$$
    \begin{align*}
        % |\valBy{j}{x} - \valBy{j}{x^*}| \leq \gamma\\
         \max\{\valBy{j}{y},\valBy{j}{x}\}  \leq \frac{1}{1-\gamma} \cdot \min\{\valBy{j}{y},\valBy{j}{x}\}
    \end{align*}
\end{potentialDefinition}

\paragraph{Bad example and conclusion def. \ref{altDef:5}:}
This definition is close to definition \ref{altDef:1} but weaker, still the same example and conclusion apply.

\begin{potentialDefinition}\label{altDef:6}
    A solution $x$ is a $(1-\gamma)$-approximately optimal if given a Leximin-optimal solution, $x^*$, there exists an integer $k \in [n]$ such that: 
    \begin{align*}
    \forall j < k: & \valBy{j}{x} = \valBy{j}{x^*}\\
    & \valBy{k}{x} > (1-\gamma) \cdot \valBy{k}{x^*}
    \end{align*}
\end{potentialDefinition}
% \eden{I'm not sure it is well defined, since by decreasing $\gamma$ (for example) the second value might become smaller than the first.}

\paragraph{Bad example def. \ref{altDef:6}:} As in the case of definition \ref{altDef:2}, by taking a small $k$, we cannot distinguish between two solutions that satisfy this definition, but one of them should be definitely preferred.
Consider again the following example with three objectives, where:
\begin{align*}
    \max \quad &\{f_i(x) = x_i \mid \forall 1 \leq i \leq 3 \} \\
    s.t. \quad  & 9 x_1 + x_2 \leq 10\\
    &  x_3 \leq 100\\
    & x \in \mathbb{R}^3_{+}
\end{align*}
The Leximin optimal solution $x^*$ is $(1,1, 100)$ and therefore, by taking $k$ to be $2$, we get that any solution that its minimum value is $1$ and its second-smallest objective value is more than $(1-\gamma)$ is considered $(1-\gamma)$-approximately optimal.
As an example, the solution $(1, 1, 1)$ is considered $(1-\gamma)$-approximately-optimal Leximin solution (as $(1-\gamma) < 1$).
But it is easy to see that this solution is quite bad for $f_3$ (who can achieve $100$).

\paragraph{Conclusion def. \ref{altDef:6}:} Same as for def. \ref{altDef:2}, an appropriate definition should take into account as many objectives as possible.

% \begin{potentialDefinition}
%     OWA.
% \end{potentialDefinition}

%--------------------------

\begin{potentialDefinition}\label{altDef:7}
    A solution $x$ is a $(1-\gamma)$-approximately optimal if there is no other solution $y$ that is $(1-\gamma)$-Leximin preferred over it, where this relation is defined as follows: $y$ is preferred over $x$ if  there exists an integer $k \in [n]$ such that:
    \begin{align*}
        \forall j < k \colon \quad &   \max\{\valBy{j}{y},\valBy{j}{x}\}  \leq \frac{1}{(1-\gamma)} \cdot \min\{\valBy{j}{y},\valBy{j}{x}\}\\
        & \valBy{k}{y} > \frac{1}{(1-\gamma)} \cdot 
\valBy{k}{x}
    \end{align*}
     [This relation is related to a one suggested in \cite{kalai_lexicographic_2012}, it is described in more detail in the Related work Section]. 
\end{potentialDefinition}

\paragraph{Bad example and conclusion def. \ref{altDef:7}:} As with definition \ref{altDef:3}, here also, the Leximin optimal solution is not optimal according to this relation and it might favor solutions with lower smallest objective values. 
Consider again the following example:
\begin{align*}
    \max \quad &\{f_1(x) = x_1, f_2(x)=x_2\} \\
    s.t. \quad  & 99 x_1 + x_2 \leq 100\\
    & x \in \mathbb{R}^2_{+}
\end{align*}
Assume that $\gamma = 0.1$, the Leximin-optimal solution is $(1,1)$, but the solution $(0.9,10.9)$ is preferred over it according to this relation (since for $k=2$ we get that $\max\{0.9,1\} \leq \frac{1}{0.9}\cdot\min\{0.9,1\}$ and $10.9 > \frac{1}{0.9} \cdot 1$) and therefore, it is not approximately-optimal.



\section{The Approximate Leximin Order}\label{sec:approx-order-is-strict-partial}

Unlike the leximin order, $\leximinPreferred$, which is a strict \textbf{total} order, the approximate leximin order, $\alphaBetaPreferred$ for $\DEFmultApprox\in (0,1]$ and $\DEFadditiveApprox \geq 0$ is a strict \textbf{partial} order.
The difference is that in partial orders, not all vectors are comparable.
Consider for example the sorted vectors $(1,2)$ and $(1, 3)$. 
According to the leximin order, $(1,3)$ is clearly preferred (as $3>2$), but according to many approximate leximin orders neither one is preferred over the other, for example according to the orders $\alphaBetaPreferredParams{0.6}{0}$,$ \alphaBetaPreferredParams{1}{1}$ or $\alphaBetaPreferredParams{0.8}{0.5}$.
% (irreflexive, asymmetric and transitive).

An order is a strict partial order if it is irreflexive, transitive and asymmetric.
Lemma \ref{lemma:order-is-irreflexive} proves that the order is irreflexive, Lemma \ref{lemma:order-is-transitive} proves it is transitive, and Lemma \ref{lemma:order-is-asymmetric} proves that it is asymmetric.
% \erel{It would be good to show an example why this is not a total order.}

% need to prove irreflexive, asymmetric (we have already proved that it is transitive).

% ***[I thought it would be better to prove it on vectors (rather than "solutions") to make it as general as possible]\\

Let $\DEFmultApprox\in (0,1]$ and $\DEFadditiveApprox \geq 0$. 

\begin{lemma}\label{lemma:order-is-irreflexive}
    The approximate leximin order $\alphaBetaPreferred$ is irreflexive.
\end{lemma}

\begin{proof}
    % \eden{I used $x$ only to remind the reader what irreflexive is, maybe it should simply be in the lemma description}
    Let $x$ be a solution. We will show that $x \nAlphaBetaPreferred x$.
    As the definition requires that one component be \emph{strictly greater} than the other, it is trivial.
\end{proof}

\begin{lemma}\label{lemma:order-is-transitive}
    The approximate leximin order $\alphaBetaPreferred$ is transitive.
\end{lemma}

\begin{proof}
    Let $x,y$ and $z$ be solutions such that $x \alphaBetaPreferred y$ and $y \alphaBetaPreferred z$.
    We will prove that $x \alphaBetaPreferred z$.

    
    Since $x \alphaBetaPreferred y$, there exists an integer $ k_1 \in [n]$ such that:
    \begin{align*}
        \forall j<k_1 \colon &  \valBy{j}{x} \geq \valBy{j}{y}\\
            & \valBy{k_1}{x} > \frac{1}{\DEFmultApprox} \left( \valBy{k_1}{y} + \DEFadditiveApprox \right)
    \end{align*}
    And since $y \alphaBetaPreferred z$, there exists an integer $k_2 \in [n]$ such that:
    \begin{align*}
        \forall j<k_2 \colon &  \valBy{j}{y} \geq \valBy{j}{z}\\
            & \valBy{k_2}{y} > \frac{1}{\DEFmultApprox} \left( \valBy{k_2}{z} + \DEFadditiveApprox \right) 
    \end{align*}

    As $\DEFmultApprox \in (0,1]$ and $\DEFadditiveApprox \geq 0$, it follows that:
    \begin{align}\label{eq:trans-k-s}
        \valBy{k_1}{x} > \valBy{k_1}{y}, \Hquad \valBy{k_2}{y} >  \valBy{k_2}{z}
    \end{align}

    % Accordingly, if $k_1=k_2$, then this integer, denoted by $k$, allows us to conclude that $x \alphaBetaPreferred z$. 
    % By the definitions of $k_1$ and $k_2$, for any $j<k_1=k_2$ the required holds as $\valBy{j}{x} \geq \valBy{j}{y} \geq \valBy{j}{z}$.
    % In addition, $\valBy{k_1}{x}> \valBy{k_1}{y}$ by equation \ref{eq:trans-k-s}
    % $ > \frac{1}{\DEFmultApprox} \left( \valBy{k_1}{y} + \DEFadditiveApprox \right)$ and nd  and 
    
    Let $k = \min\{k_1,k_2\}$.
    
    If $k = k_1$, by the definition of $k_1$, $\valBy{k}{x} > \frac{1}{\DEFmultApprox} \left( \valBy{k}{y} + \DEFadditiveApprox \right)$.
    However, $\valBy{k}{y} \geq \valBy{k}{z}$, by definition if $k<k_2$ and by equation \ref{eq:trans-k-s} if $k=k_2$. \ref{eq:transitive-k}
    Therefore, $\valBy{k}{x} > \frac{1}{\DEFmultApprox} \left( \valBy{k}{z} + \DEFadditiveApprox \right)$.
    
    Otherwise, if $k=k_2$, by the definition of $k_2$, $\valBy{k}{y} > \frac{1}{\DEFmultApprox} \left( \valBy{k}{z} + \DEFadditiveApprox \right)$. But, $\valBy{k}{x} \geq \valBy{k}{y}$, by definition if $k<k_1$ and by equation \ref{eq:trans-k-s} if $k=k_1$. Again, we can conclude that $\valBy{k}{x} > \frac{1}{\DEFmultApprox} \left( \valBy{k}{z} + \DEFadditiveApprox \right)$.

     In addition, for each $j<k$, since $j< k_1$ and $j < k_2$, by definition the following holds:
    \begin{align}\label{eq:transitive-k}
        \valBy{j}{x} \geq \valBy{j}{y} \geq \valBy{j}{z}
    \end{align}
    So, $k$ is an integer that satisfy all the requirements, and so, $x \alphaBetaPreferred z$.
    \end{proof}
    

    
    \begin{lemma}\label{lemma:order-is-asymmetric}
        The approximate leximin order $\alphaBetaPreferred$ is asymmetric.
    \end{lemma}
    
    \begin{proof}
        Let $x$ and $y$ be solutions such that $x \alphaBetaPreferred y$. We will show that $y \nAlphaBetaPreferred x$. 
        Assume by contradiction that $y \alphaBetaPreferred x$. 
        From Lemma \ref{lemma:order-is-transitive}, this relation is transitive. Therefore, since $x \alphaBetaPreferred y$ and $y \alphaBetaPreferred x$, also $x \alphaBetaPreferred x$.
        But, from Lemma \ref{lemma:order-is-irreflexive}, this relation is irreflexive --- a contradiction.
    \end{proof}
\section{Proof of Theorem \ref{th:main}}\label{sec:algo-sec-proofs}
\eden{should probably change the title}

This section is dedicated to proving Theorem \ref{th:main}.
To this end, we use another equivalent representation of \eqref{eq:sums-OP}, which was also introduced by \cite{Ogryczak_2006} 
(we provide the proof of equivalence in Appendix \ref{sec:equivalent-proofs}). 
\erel{Can't we just use it directly instead of P2?}
% (here also, the variables are $\ztVar{x}$ and $x$, and $z_1, \ldots z_{t-1}$ are constants)
\begin{align*}
    \max \quad &z_t \tag{P2-compact}\label{eq:compact-OP} \;\;
        s.t. &\quad  & (1) \quad x \in S\\
                    &&& (\Tilde{2}) \quad \sum_{i=1}^{\ell} \valBy{i}{x} \geq \sum_{i=1}^{\ell}  z_i && \ell = 1,\ldots, t-1 \nonumber\\
                    &&& (\Tilde{3}) \quad \sum_{i=1}^{t} \valBy{i}{x} \geq \sum_{i=1}^{t}  z_i
\end{align*}
In this problem, constraints $(\hat{2})$ and $(\hat{3})$ are replaced by  $(\Tilde{2})$ and $(\Tilde{3})$, respectively.  
The difference is that
$(\hat{2})$ gives, for each $\ell$, a lower bound on the sum for \emph{any} set of $\ell$ objective functions; whereas $(\Tilde{2})$ only considers the sum of the $\ell$ \emph{smallest} such values.  
% However, since the constraints set the same lower bound on this sum, the constraints are equivalent.  
Similarly for $(\hat{3})$ and $(\Tilde{3})$. 
Since  the problems are equivalent, a solver, either exact or approximate, for one can be used as a solver, with the same level of accuracy, for the other (Lemma \ref{lemma:solver-equivalent-prob}). 
Therefore, as \eqref{eq:compact-OP} is equivalent to \eqref{eq:sums-OP}, which, in turn, is equivalent to \eqref{eq:vsums-OP}, in proving the theorem we may assume that \textsf{OP} is an approximation procedure for \eqref{eq:compact-OP}.  
This will simplify the proofs. \eden{I added the line from the comment back, isn't it important to explain why we need this representation?}

% \erel{*** I do not understand. We say that P1 and P2 are equivalent with an exact solver, but not with an approximate solver. Here, we claim that P3 and P2-compact are equivalent, but this is true only with an exact solver. Don't we have to prove that they are equivalent also with an approximate solver? ***}

We denote $\retSol := x_n$ = the solution $x$ attained at the last iteration ($t=n$) of the algorithm. 

Following are some observations regarding the set of feasible solutions in each iteration, their objective values, and the solution $\retSol$ that will be useful later on.

% For any constants $z_1,\ldots, z_{t-1}$,
% any vector $x \in S$ that satisfies constraint $(\Tilde{2})$ of \eqref{eq:compact-OP} 
% is feasible to this problem.
% This is because any solution $x \in S$ can satisfy constraint $(\Tilde{3})$ with a small enough assignment to the variable $z_t$. \eden{I'm not sure how to explain it....}
\begin{observation}\label{obs:feasi-and-constraint2}
For any constants $z_1,\ldots, z_{t-1}$,
any vector $x \in S$ that satisfies constraint $(\Tilde{2})$ of \eqref{eq:compact-OP} 
can be a part of a feasible solution $(x,z_t)$ for any $z_t \leq \sum_{i=1}^{t} \valBy{i}{x} - \sum_{i=1}^{t-1} z_i$.
\end{observation}

Since $\retSol$ is a feasible solution of \eqref{eq:compact-OP} in iteration $n$, and as each
iteration only adds new constraints to $(\Tilde{2})$, it follows that $\retSol$ is also a feasible solution of \eqref{eq:compact-OP} in any iteration $1 \leq t\leq n$. 
\begin{observation}\label{obs:retSol-solves-any-t}
$\retSol$ is a feasible solution of \eqref{eq:compact-OP} in any iteration $1 \leq t\leq n$.
\end{observation}

Now, consider the problem \eqref{eq:compact-OP} that was solved in iteration $t$.
Here, $z_t$ is a \emph{variable} and $z_1, \ldots z_{t-1}$ are constants.
The objective of this problem is $\max z_t$, and the only constraint that includes the variable $z_t$ is  $(\Tilde{3})$.
Therefore, rearranging it to $\sum_{i=1}^{t} \valBy{i}{x} - \sum_{i=1}^{t-1}  z_i\geq z_t$, allows us to conclude that the objective value is determined by the left side of this inequality (as $z_t$ is maximized when the inequality turns to equality).
\begin{observation}\label{obs:obj-value}
The objective value obtained by a feasible solution $x$ to the problem \eqref{eq:compact-OP} that was solved in iteration $t$ is $\sum_{i=1}^{t} \valBy{i}{x} - \sum_{i=1}^{t-1}  z_i$.
\end{observation}

Lastly, as the value obtained as a $(\multApprox, \additiveApprox)$-approximation for this problem is the \emph{constant} $z_t$, the optimal value is at most $\frac{1}{\multApprox} (z_t+\additiveError)$. 
Consequently, the objective value of any feasible solution is at most this value.
Since $\retSol$ is feasible for any iteration $t$ (Observation \ref{obs:retSol-solves-any-t}) and since its objective is $\sum_{i=1}^t \valBy{i}{\retSol} - \sum_{i=1}^{t-1} z_i$ (Observation \ref{obs:obj-value}), we can conclude:

\begin{observation}\label{obs:obj-xt-to-zt}
    The objective value obtained by $\retSol$ to the problem \eqref{eq:compact-OP} that was solved in iteration $t$ is at most $\frac{1}{\multApprox} (z_t+\additiveError)$. That is:
    \begin{align*}
        \sum_{i=1}^t \valBy{i}{\retSol} - \sum_{i=1}^{t-1} z_i \leq \frac{1}{\multApprox} \left(z_t+\additiveError \right).
    \end{align*}
\end{observation}

% This conclusion also implies that for any $1 \leq t \leq n$, the solution $(x_t, z_t)$ that that was outputted for \eqref{eq:compact-OP} in iteration $t$, satisfies constraint $(\Tilde{3})$ as equality. That is:
% \begin{observation}\label{obs:equality-xt-zt}
% For any $1 \leq t \leq n$,  $\sum_{i=1}^{t} \valBy{i}{x_t} = \sum_{i=1}^{t}  z_i$.
% \end{observation}



%%%
% OVERALL EXPLANATION 
We start with Lemmas \ref{lemma:beta-vk}-\ref{lemma:fk-to-all}, which establish a relationship between the $k$-th least objective value obtained by $\retSol$ 
% ($\valBy{k}{\retSol}$) 
and the difference between the sum of the $(k-1)$ least objective values obtained by $\retSol$ and the sum of the $(k-1)$ first $z_i$ values.
% ($\sum_{i=1}^{k-1}\valBy{k}{\retSol} - \sum_{i=1}^{k-1}z_i$). 
Theorem \ref{th:main} then uses this relation to prove that the existence of another solution that would be $\left(\frac{\multApprox^2}{1-\multApprox + \multApprox^2}, \frac{\multApprox(2-\multApprox)\additiveApprox}{1-\multApprox +\multApprox^2}\right)$-preferred over $\retSol$ would lead to a contradiction.

For clarity, throughout the proofs, we denote the multiplicative error factor by $\multError = 1-\multApprox$.

% LEMMAS.
% BLAH BLAH.

\begin{lemma}\label{lemma:beta-vk}
    For all $k\in[n]$, 
    \begin{align*}
        \multError \valBy{k}{\retSol} \geq \left(\sum_{i=1}^k \valBy{i}{\retSol} - \sum_{i=1}^k z_i\right) -\multError \left(\sum_{i=1}^{k-1} \valBy{i}{\retSol} - \sum_{i=1}^{k-1} z_i\right) -\additiveError
    \end{align*}
\end{lemma}

\begin{proof}
By Observation \ref{obs:obj-xt-to-zt},
    \begin{align*}
         &\sum_{i=1}^k \valBy{i}{\retSol} - \sum_{i=1}^{k-1} z_i \leq \frac{1}{\multApprox} \left(z_k + \additiveError \right) = \frac{1}{1-\multError} \left(z_k + \additiveError \right)\\
         &\Rightarrow z_k +\additiveError \geq (1-\multError) \left(\sum_{i=1}^{k} \valBy{i}{\retSol} - \sum_{i=1}^{k-1}  z_i\right)\\
        &\Rightarrow z_k +\additiveError\geq \left(\sum_{i=1}^{k} \valBy{i}{\retSol} - \sum_{i=1}^{k-1}  z_i\right) - \multError \left(\sum_{i=1}^{k} \valBy{i}{\retSol} - \sum_{i=1}^{k-1}  z_i\right)\\
        &\Rightarrow \multError \valBy{k}{\retSol} \geq \left(\sum_{i=1}^k \valBy{i}{\retSol} - \sum_{i=1}^k z_i\right) -\multError \left(\sum_{i=1}^{k-1} \valBy{i}{\retSol} - \sum_{i=1}^{k-1} z_i\right) -\additiveError.
        \qedhere
    \end{align*}
\end{proof}


\begin{lemma}\label{lemma:beta-sums-to-diff}
    For all $k\in[n]$, 
    \begin{align*}
        \sum_{i=1}^k \multError^{i} \valBy{k-i+1}{\retSol} \geq \sum_{i=1}^k \valBy{i}{\retSol} - \sum_{i=1}^{k} z_i -\additiveError
    \end{align*}
\end{lemma}

\begin{proof}
    The proof is by induction on $k$.
    For $k=1$ the claim follows directly from Lemma \ref{lemma:beta-vk}.
    Assuming the claim is true for $1,\ldots k-1$, we show it is true for $k$:
    \begin{align*}
        &\sum_{i=1}^k \multError^{i} \valBy{k-i+1}{\retSol} = \multError \valBy{k}{\retSol} + \sum_{i=2}^k \multError^{i} \valBy{k-i+1}{\retSol}\\
        &= \multError \valBy{k}{\retSol} + \sum_{i=1}^{k-1} \multError^{i+1} \valBy{k-(i+1)+1}{\retSol} \\
        &= \multError \valBy{k}{\retSol} + \multError \sum_{i=1}^{k-1} \multError^{i} \valBy{(k-1) -i+1}{\retSol}\\
        &= \multError \valBy{k}{\retSol} + \multError \left(\sum_{i=1}^{k-1} \valBy{i}{\retSol} - \sum_{i=1}^{k-1} z_i\right) && \text{(by induction assumption)}\\
        &\geq \left(\sum_{i=1}^k \valBy{i}{\retSol} - \sum_{i=1}^k z_i\right) -\multError \left(\sum_{i=1}^{k-1} \valBy{i}{\retSol} - \sum_{i=1}^{k-1} z_i\right)-\additiveError  \\
        & \quad +  \multError \left(\sum_{i=1}^{k-1} \valBy{i}{\retSol} - \sum_{i=1}^{k-1} z_i\right) && \text{(by Lemma \ref{lemma:beta-vk})} \\
        &= \sum_{i=1}^k \valBy{i}{\retSol} - \sum_{i=1}^{k} z_i -\additiveError.
        \qed
    \end{align*}
\end{proof}


\begin{lemma}\label{lemma:fk-to-all}
    For all $1<k \leq n$, 
    \begin{align*}
        \frac{\multError}{1-\multError} \valBy{k}{\retSol} \geq \sum_{i=1}^{k-1}\valBy{i}{\retSol} - \sum_{i=1}^{k-1}z_i - \additiveError
    \end{align*}
\end{lemma}

\begin{proof}
    First, notice that since $k \geq (k-1)-i+1$ for any $1\leq i \leq k$ and as the function $\valBy{i}$ represents the $i$-th smallest objective value, also:
    \begin{align}\label{eq:increase-by-obj-size}
        \forall 1\leq i \leq k \colon \quad \valBy{k}{\retSol} \geq \valBy{(k-1)-i+1}{\retSol}
    \end{align}
    In addition, consider the geometric series with a first element $1$, a ratio $\multError$, and a length $(k-1)$. 
    As $\multError < 1$, its sum can be bounded in the following way:
    \begin{align}\label{eq:geometric-series-beta}
        \sum_{i=1}^{k-1} \multError^{i-1} = \frac{1-\multError^{k-1}}{1-\multError} < \lim_{k \to \infty}\frac{1-\multError^{k-1}}{1-\multError} = \frac{1}{1-\multError}
    \end{align}
    
    Now, the claim can be concluded as follows:
    \begin{align*}
        & \frac{\multError}{1-\multError}\valBy{k}{\retSol} = \multError \left(\frac{1}{1-\multError} \valBy{k}{\retSol} \right)\\
        & > \multError \left(\sum_{i=1}^{k-1} \multError^{i-1} \valBy{k}{\retSol} \right) && \text{(by Equation \eqref{eq:geometric-series-beta})}\\
        & \geq  \multError \left(\sum_{i=1}^{k-1} \multError^{i-1} \valBy{(k-1)-i+1}{\retSol} \right) && \text{(by Equation \eqref{eq:increase-by-obj-size})}\\
        &= \sum_{i=1}^{k-1} \multError^{i} \valBy{(k-1)-i+1}{\retSol} \\
        &\geq \sum_{i=1}^{k-1}\valBy{i}{\retSol} - \sum_{i=1}^{k-1}z_i - \additiveError && \text{(by Lemma \ref{lemma:beta-sums-to-diff})}
\end{align*}
\erel{Formally, Lemma \ref{lemma:beta-sums-to-diff} is for $k\geq 1$, and we apply it for $k-1$, which might be $0$.}\eden{I tried to fixed it, is it better?}
\end{proof}



%------
% thm.

We are now ready to prove the Theorem \ref{th:main}.
\begin{proof}[Proof of Theorem \ref{th:main}]
% \eden{I'm not sure if we should write again about the claim with $\multApprox$}
Recall that the claim is that $\retSol$ is a $\left(\frac{\multApprox^2}{1-\multApprox + \multApprox^2}, \frac{\multApprox(2-\multApprox)\additiveApprox}{1-\multApprox +\multApprox^2}\right)$-approximation.

For brevity, we define the following constants:
\begin{align*}
    \Delta^{mult} = \frac{\multApprox}{1-\multApprox + \multApprox^2}, \quad  \Delta^{add} = \frac{\multApprox(2-\multApprox)}{1-\multApprox +\multApprox^2}
\end{align*}
Accordingly, we need to prove that $\retSol$ is a $\left(\Delta^{mult} \cdot \multApprox, \Delta^{add}\cdot\additiveApprox\right)$-approximation.

We prove the following equation, that will be helpful later:
\begin{align}\label{equ:mu}
\frac{1}{\Delta^{mult} \cdot \multApprox} = \frac{1-\multError +\multError^2}{(1-\multError)^2}
\end{align}
This is true because
\begin{align*}
    &\Delta^{mult} \cdot \multApprox =   \frac{\multApprox^2}{1-\multApprox + \multApprox^2} && \text{(Definition of $\Delta^{mult}$)} \\
    &= \frac{(1-\multError)^2}{\multError +(1-\multError)^2} = \frac{(1-\multError)^2}{1-\multError +\multError^2} &&\text{(since $\multApprox = 1-\multError$)}\\
    & \Rightarrow \frac{1}{\Delta^{mult} \cdot \multApprox} = \frac{1-\multError +\multError^2}{(1-\multError)^2}
    \end{align*}
    Another equation that will be useful later is:
    \begin{align}\label{eq:additive-error}
        \frac{\Delta^{add}}{\Delta^{mult}\cdot \multApprox}  = \frac{1+\multError}{1-\multError}.
    \end{align}
    The reason for this is that
    \begin{align*}
        &\frac{\Delta^{add}}{\Delta^{mult}\cdot \multApprox} =\frac{1-\multApprox + \multApprox^2}{\multApprox^2} \cdot \frac{\multApprox(2-\multApprox)}{1-\multApprox +\multApprox^2}&& \text{(Definitions of $\Delta^{mult}$ and $\Delta^{add}$)}\\
        &=\frac{\multApprox(2-\multApprox)}{\multApprox^2} = \frac{(1-\multError)(1 + \multError)}{(1-\multError)^2} =\frac{1+\multError}{1-\multError}  &&\text{(since $\multApprox = 1-\multError$)}
    \end{align*}

    Now, suppose by contradiction that $\retSol$ is \emph{not} $\left(\Delta^{mult} \cdot \multApprox, \Delta^{add}\cdot\additiveApprox\right)$-approximately-optimal.
    By definition, this means there exists a solution $y \in S$  that is $\left(\Delta^{mult} \cdot \multApprox, \Delta^{add}\cdot\additiveApprox\right)$-preferred over it.
    That is, there exists an integer $1 \leq k \leq n$ such that:
    \begin{align*}
        \forall j < k \colon &\valBy{j}{y} \geq \valBy{j}{\retSol};\\
        & \valBy{k}{y} > \frac{1}{\Delta^{mult} \cdot\multApprox} \left(\valBy{k}{\retSol} + \Delta^{add} \cdot\additiveError \right).
    \end{align*}

    Since $\retSol$ was obtained in \eqref{eq:compact-OP} that was solved in the last iteration $n$, it is clear that $\sum_{i=1}^k \valBy{i}{\retSol} \geq \sum_{i=1}^{k} z_i$ (by constraint $(\Tilde{2})$ if $k<n$ and $(\Tilde{3})$ otherwise).
    Which implies:
    \begin{align}\label{eq:fk-to-zk}
        \sum_{i=1}^k \valBy{i}{\retSol} - \sum_{i=1}^{k-1} z_i \geq z_k
    \end{align}

    Now, consider \eqref{eq:compact-OP} that was solved in iteration $k$.
    By Observation \ref{obs:retSol-solves-any-t}, $\retSol$ is feasible to this problem.
    As the $(k-1)$ smallest objective values of $y$ are at least as high as those of $\retSol$, it is easy to conclude that $y$ also satisfies constraints $(\Tilde{2})$ of this problem; since, for any $\ell < k$:
    \begin{align*}
        \sum_{i=1}^{\ell} \valBy{i}{y} \geq\sum_{i=1}^{\ell} \valBy{i}{\retSol} \geq \sum_{i=1}^{\ell} z_i
    \end{align*}
    Therefore, by Observation \ref{obs:feasi-and-constraint2}, $y$ is also feasible to this problem. 

    If $k=1$, the objective value $y$ in this problem is $\valBy{1}{y}$ (Observation \ref{obs:obj-value}).
    In addition, $\valBy{1}{\retSol} \geq z_1$ by equation \ref{eq:fk-to-zk}. As $\Delta^{mult}\geq 0$ and $\Delta^{add}\geq 0$, it follows that:
    \begin{align*}
        \valBy{1}{y}> \frac{1}{\Delta^{mult} \cdot\multApprox} \left(\valBy{1}{\retSol} + \Delta^{add} \cdot\additiveError \right)\geq \frac{1}{\multApprox} \left(z_1 + \additiveError \right)
    \end{align*}
    But, $z_1$ was obtained as an approximation for this problem, therefore the optimal value is at most $\frac{1}{\multApprox}\left(z_1 + \additiveError \right)$ --- a contradiction.

    
    Otherwise, $k>1$, we shall now see that in this case $y$ also satisfies the following:
    \begin{align}\label{eq:yk-to-sum}
        \valBy{k}{y} > \frac{1}{1-\multError} \valBy{k}{\retSol} + \frac{\multError}{1-\multError}\sum_{i=1}^{k-1}\valBy{i}{\retSol} - \frac{\multError}{1-\multError} \sum_{i=1}^{k-1}z_i  +\frac{1}{1-\multError}\cdot\additiveError
    \end{align}
    this is true because
    \begin{align*}
        &\valBy{k}{y} > \frac{1}{ \Delta^{mult} \cdot\multApprox} \left(\valBy{k}{\retSol} + \Delta^{add}\cdot \additiveError \right) && \text{(Definition of $y$ for $k$)}\\
        &= \frac{1-\multError +\multError^2}{(1-\multError)^2} \valBy{k}{\retSol}+ \frac{\Delta^{add}}{\Delta^{mult} \multApprox}\cdot\additiveError && \text{(by Equation \ref{equ:mu})}\\
        &= \frac{1-\multError +\multError^2}{(1-\multError)^2} \valBy{k}{\retSol}+ \frac{1+\multError}{1-\multError}\cdot\additiveError && \text{(by Equation \ref{eq:additive-error})} \erel{???}\\
        &\geq\frac{1}{1-\multError} \valBy{k}{\retSol} + \frac{\multError}{1-\multError}\left(\sum_{i=1}^{k-1}\valBy{i}{\retSol} - \sum_{i=1}^{k-1}z_i-\additiveError\right) +\frac{1+\multError}{1-\multError}\cdot\additiveError && \text{(by Lemma \ref{lemma:fk-to-all} for $k>1$)}\\
        & = \frac{1}{1-\multError} \valBy{k}{\retSol} +\frac{\multError}{1-\multError}\sum_{i=1}^{k-1}\valBy{i}{\retSol} - \frac{\multError}{1-\multError} \sum_{i=1}^{k-1}z_i +\frac{1}{1-\multError}\cdot\additiveError &&\erel{???}\text{\eden{is it more clear?}}
    \end{align*}    
    
    We compute the objective value of $y$, which is $\sum_{i=1}^k \valBy{i}{y} - \sum_{i=1}^{k-1} z_i$ (by Observation \ref{obs:obj-value}):  
    \begin{align*}
        &\sum_{i=1}^k \valBy{i}{y} - \sum_{i=1}^{k-1} z_i=\sum_{i=1}^{k-1} \valBy{i}{y} - \sum_{i=1}^{k-1} z_i + \valBy{k}{y}\\
        &\geq \sum_{i=1}^{k-1} \valBy{i}{\retSol} - \sum_{i=1}^{k-1} z_i + \valBy{k}{y} && \text{(Definition of $y$ for $j<k$)}\\
        &> \sum_{i=1}^{k-1} \valBy{i}{\retSol} - \sum_{i=1}^{k-1} z_i + \frac{1}{1-\multError} \valBy{k}{\retSol} \\
        & \quad + \frac{\multError}{1-\multError}\sum_{i=1}^{k-1}\valBy{i}{\retSol} - \frac{\multError}{1-\multError}\sum_{i=1}^{k-1}z_i +\frac{1}{1-\multError}\cdot\additiveError && \text{(by Equation \ref{eq:yk-to-sum})}\\
        & = \frac{1}{1-\multError} \left(\sum_{i=1}^k \valBy{k}{\retSol} - \sum_{i=1}^{k-1}z_i + \additiveError\right) &&\text{(since  $1+\frac{\multError}{1-\multError} = \frac{1}{1-\multError}$)}\erel{???}\text{\eden{is it more clear?}}
        \\
        &\geq \frac{1}{1-\multError} \left(z_k +\additiveError\right) && \text{(by Equation \ref{eq:fk-to-zk}) }
    \end{align*}
    \eden{I'm not sure why to comment the lines, shouldn't we explain why it is a contradiction?how is the following?}
    % \emark{However, the approximately-optimal solution obtained for this problem during the algorithm run is $z_k$, so the optimal value is at most $\frac{1}{(1-\multError)}\left(z_k+\additiveError\right)$.
    % But, as we shall see, the objective $y$ yields in this problem, $\sum_{i=1}^k \valBy{i}{y} - \sum_{i=1}^{k-1} z_i$ (by Observation \ref{obs:obj-value}), is higher than this value, which is of course a contradiction:}
    However, the approximately-optimal value obtained for this problem during the algorithm run is $z_k$, so the optimal value is at most $\frac{1}{(1-\multError)}\left(z_k+\additiveError\right)$, which is, again, a contradiction.
    
\end{proof}

\section{Proof of Theorem \ref{th:app-main}}\label{sec:app-sec-proofs}
% \eden{should probably change the title}

% Agents are assumed to care only about their own share (allowing us to use the following abuse of notation in which $u_j$ takes a bundle $b$ of items), their utilities are assumed to be normalized ($u_j(\emptyset) = 0$), monotone ($u_j(b_1) \leq u_j(b_2)$ if $b_1 \subseteq b_2$), and submodular ($u_j(b_1) + u_j(b_2) \geq u_j(b_1 \cup b_2) + u_j(b_1 \cap b_2)$ for any bundles $b_1,b_2$).
% It is assumed that each agent assigns a positive utility to the set of all items.
% The utilities $(u_i)_{i=1}^n$ are assumed to be given in the \emph{value oracle model}, meaning that we do not have a direct access to them, but only to an oracle that indicates the value of an agent from a given simple allocation.
% % \eden{z1 > 0}

This section proves Theorem \ref{th:app-main}:
suppose we are given a randomized algorithm that returns a simple allocation that approximates the utilitarian welfare with multiplicative error $\multError$ (with success probability $p$).
Then, Algorithm \ref{alg:basic-ordered-Outcomes} can be used to obtain a stochastic allocation that approximates leximin with a multiplicative error of at most $\frac{\multError}{1-\multError +\multError^2}$ (with the same probability).

% title: the specific problem as P3
As we saw in Section \ref{sec:algo-short}, an approximation to leximin can be obtained by providing a procedure \textsf{OP} to approximate \eqref{eq:vsums-OP}  (Theorem \ref{th:main}), which, under these particular settings, becomes:
% \erel{Why do you call it "configuration LP"? I think this term refers to something else: \url{https://en.wikipedia.org/wiki/Configuration_linear_program}}
\begin{align}
&\max \quad z_t \quad s.t. \tag{\progAppFirst}\label{eq:app-vsums-OP}\\
& (\text{\progAppFirst.1.1}) \Hquad \sum_{A \in \mathcal{A}} p_d(A) = 1 \nonumber\\
& (\text{\progAppFirst.1.2}) \Hquad p_d(A) \geq 0  && \forall A \in \mathcal{A} \nonumber\\
& (\text{\progAppFirst.2}) \Hquad \ell y_{\ell} - \sum_{j=1}^n m_{\ell,j}\geq \sum_{i=1}^{\ell}  z_i && \forXinY{\ell}{t-1} \nonumber \\
& (\text{\progAppFirst.3}) \Hquad t y_t - \sum_{j=1}^{n} m_{t,j} \geq \sum_{i=1}^{t}  z_i \nonumber \\
& (\text{\progAppFirst.4}) \Hquad m_{\ell,j} \geq y_{\ell} - \sum_{A \in \mathcal{A}}p_d(A) \cdot u_j(A)  && \forXinY{\ell}{t},\Hquad \forXinY{j}{n} \nonumber \\
& (\text{\progAppFirst.5}) \Hquad m_{\ell,j} \geq 0  && \forXinY{\ell}{t},\Hquad \forXinY{j}{n} \nonumber
\end{align}
Here the variables are $p_d(A)$ for any simple allocation $A \in \mathcal{A}$, $\ztVar{}$, and $y_{\ell}$ and $m_{\ell,j}$ for all $\ell \in [t]$ and $ j\in [n]$; and the values $z_1, \ldots z_{t-1}$ are constants.
Notice that it is a \emph{linear program} that has a polynomial number of constraints thanks to \eqref{eq:vsums-OP} representation, but an exponential number of variables (since there is a variable $p_d(A)$ for each simple allocation).
So, it is unclear how to approach it directly in polynomial time.
% \eden{here?}
In addition, it means that the output size is exponential in $n$.
To deal with this issue, the solutions are considered in \emph{sparse form} --- a list of the variables with positive values, along with their values.
Accordingly, if a solution has only a polynomial number of variables with positive values it can be represented by a polynomial size.
We will later see that the procedure described in this section returns such a solution in polynomial time.
% \eden{should write something about the output size, as \cite{kawase_max-min_2020}}

% title: baseline
% \erel{I would move the following paragraph upwards}
With $t=1$, \eqref{eq:app-vsums-OP} can be viewed as the problem of egalitarian welfare maximization, indeed, Kawase and Sumita \cite{kawase_max-min_2020} who studied this problem, considered a slightly simpler representation. 
% After proving that approximating the optimal value to a factor better than $(1-\frac{1}{e})$ is NP-hard, they present a dual-based algorithm that achieves this accuracy \er{w.h.p (?)}.
We now show how their dual-based technique can be applied to approximate \eqref{eq:app-vsums-OP} for any $t\geq 1$ while maintaining the same approximation factor.


To begin, consider the following program \eqref{eq:app-ver2-vsums-OP}, which is the result of modifying \eqref{eq:app-vsums-OP} in three ways. 
First, changing the objective-function to $\min 1/z_t$ instead of $\max z_t$. 
Second, replacing all the original variables and constants, except $z_t$, with new ones that are smaller by a factor $z_t$ (that is, $p'_A = p_d(A)/z_t$ for all $A \in \mathcal{A}$, $,y'_{\ell} = y_{\ell}/z_t,m'_{\ell,j} = m_{\ell,j}/z_t$ for $\ell \in [t]$ and $ j\in [n]$,  and $z'_i = z_i/z_t$ for $i \in [t-1]$).
And third, dividing all the constraints by $z_t$ ($z_t > 0$ since $z_t \geq z_1$ for any $t \geq 1$ and  $z_1 >0$).
\eden{to myself: maybe to explain why $z_1>0$}

\begin{align}
& \min \quad 1/z_t \quad s.t. \tag{\progAppSecond}\label{eq:app-ver2-vsums-OP}\\
& (\text{\progAppSecond.1.1}) \Hquad \sum_{A \in \mathcal{A}} p'_A = 1/z_t \nonumber\\
& (\text{\progAppSecond.1.2}) \Hquad p'_A \geq 0  && \forall A \in \mathcal{A} \nonumber\\
& (\text{\progAppSecond.2}) \Hquad \ell y'_{\ell} - \sum_{j=1}^n m'_{\ell,j}\geq \sum_{i=1}^{\ell}  z'_i && \forXinY{\ell}{t-1} \nonumber \\
& (\text{\progAppSecond.3}) \Hquad t y'_t - \sum_{j=1}^{n} m'_{t,j} \geq \sum_{i=1}^{t-1}  z'_i + 1 \nonumber \\
& (\text{\progAppSecond.4}) \Hquad m'_{\ell,j} \geq y'_{\ell} - \sum_{A \in \mathcal{A}}p'_A \cdot u_j(A)  && \forXinY{\ell}{t},\Hquad \forXinY{j}{n} \nonumber \\
& (\text{\progAppSecond.5}) \Hquad m'_{\ell,j} \geq 0  && \forXinY{\ell}{t},\Hquad \forXinY{j}{n} \nonumber
\end{align}
The programs \eqref{eq:app-vsums-OP} and \eqref{eq:app-ver2-vsums-OP} are related in the following way:
% \erel{I would make this a lemma:}
\begin{lemma}\label{lemma:bijection}
There exists a bijection mapping each solution of 
\eqref{eq:app-vsums-OP} with objective value $V$ to a unique solution of 
\eqref{eq:app-ver2-vsums-OP} with objective value $1/V$.
\end{lemma}
\begin{proof}
Let $p_d(A)$ for $A \in \mathcal{A}$, $\ztVar{}$, and $y_{\ell}$ and $m_{\ell,j}$ for all $\ell \in [t]$ and $ j\in [n]$ be a feasible solution to the program \eqref{eq:app-vsums-OP} with objective value $V$.
It can be easily verified that $p'_A = p_d(A)/z_t$ for $A \in \mathcal{A}$, $z_t$, and $y'_{\ell} = y_{\ell}/z_t$ and $m'_{\ell,j} = m_{\ell,j}/z_t$ for all $\ell \in [t]$ and $ j\in [n]$ is a feasible solution to the program \eqref{eq:app-ver2-vsums-OP} with objective value $1/V$.
\end{proof}
% \eden{maybe to write something about why it is a bijection (or to write that it is straightforward)}

Denote this bijection by $\Psi$, this also implies the following:
\begin{lemma}\label{lemma:approx-acc-by-bijection}
    If a solution approximates the program \eqref{eq:app-ver2-vsums-OP} with a multiplicative error of $\frac{\multError}{1-\multError}$. Then the corresponding solution to \eqref{eq:app-vsums-OP} according to the bijection $\Psi$ approximates this program with a multiplicative error of $\multError$.
\end{lemma}

\begin{proof}
    Let $V^*$ be the optimal objective value of \eqref{eq:app-vsums-OP}. 
    By Lemma \ref{lemma:bijection}, there exists a solution to \eqref{eq:app-ver2-vsums-OP} with value $1/V^{*}$.
    This solution yields the optimal value for \eqref{eq:app-ver2-vsums-OP} --- if there was a solution that had a value \emph{lower} than $1/V^{*}$ (\eqref{eq:app-ver2-vsums-OP} is a minimization problem), then the corresponding solution to \eqref{eq:app-vsums-OP} (by the bijection $\Psi$) would have a value higher than the optimal value $V^*$.
    Now, let the value of the solution that approximates the program \eqref{eq:app-ver2-vsums-OP} with a multiplicative error of $\frac{\multError}{1-\multError}$ be $1/V$. 
    Since \eqref{eq:app-ver2-vsums-OP} is a minimization problem, assuming that $1/V$ approximates $1/V^*$ with a multiplicative error of $\frac{\multError}{1-\multError}$ means that:
    \begin{align*}
        \frac{1}{V} \leq \left(1+\frac{\multError}{1-\multError}\right)\frac{1}{V^*},
    \end{align*}
 which implies that $V \geq (1-\multError)V^*$.
    As \eqref{eq:app-vsums-OP} is a maximization problem, this means that $V$ approximates this problem with multiplicative error $\multError$.
    By Lemma \ref{lemma:bijection}, $V$ is the value of the corresponding solution to \eqref{eq:app-vsums-OP} by the bijection $\Psi$.
\end{proof}

Notice that the only constraint of \eqref{eq:app-ver2-vsums-OP} that includes the variable $z_t$, (\progAppSecond.1.1), says that $\sum_{A \in \mathcal{A}}p'_A = 1/z_t$, and also that its objective function is $\min 1/z_t$.
As a result, we can reduce the need for the variable $z_t$ by removing constraint (\progAppSecond.1.1) and changing the objective function to $\min \sum_{A \in \mathcal{A}}p'_A$.
This change makes \eqref{eq:app-ver2-vsums-OP} a \emph{linear} program.
This will allow us to approximate it using its dual, as we will see.

The following observation will be useful later:
\begin{observation}\label{obs:c2-to-c1-in-poly-time}
    If a solution to \eqref{eq:app-ver2-vsums-OP} is given in a sparse form --- a list of the variables with nonzero value and their values, then the corresponding solution to \eqref{eq:app-vsums-OP} in a sparse form can be computed in time polynomial to the number of nonzero variables.
\end{observation}
\noindent For completeness, we briefly outline the process. 
When given a list of variables with nonzero values, we first iterate the list and sum all variables of the form $p'_A$, and then set $z_t$ to be $1$ divided by this sum. 
After, for each variable $\nu'$ in the list, we set the corresponding variable, $\nu$, to $z_t \cdot \nu'$.


% title: dual 
Now, let us consider the dual program of \eqref{eq:app-ver2-vsums-OP}, which can be described as follows:
% \erel{When you present an LP, it can help the reader if you mention what exactly the variables of the LP are.}
\begin{align}
    \max &&& \sum_{\ell=1}^{t-1} q_{\ell} \sum_{i=1}^{\ell} z_i + q_t (\sum_{i=1}^{t-1} z_i +1) \tag{\progAppDual}\label{eq:app-dual}\\
        s.t. &&& (\text{\progAppDual.1}) \Hquad \sum_{j=1}^n u_j(A) \sum_{\ell=1}^t v_{\ell,j} \leq 1  && \forall A \in \mathcal{A} \nonumber\\
                    &&& (\text{\progAppDual.2}) \Hquad \ell q_{\ell} - \sum_{j=1}^n v_{\ell,j} = 0 && \forXinY{\ell}{t} \nonumber \\
                    &&& (\text{\progAppDual.3}) \Hquad q_{\ell} - v_{\ell,j} \leq 0  && \forXinY{\ell}{t},\Hquad \forXinY{j}{n} \nonumber \\
                    &&& (\text{\progAppDual.4}) \Hquad v_{\ell,j} \geq 0  && \forXinY{\ell}{t},\Hquad \forXinY{j}{n} \nonumber \\
                    &&& (\text{\progAppDual.5}) \Hquad q_{\ell} \geq 0  && \forXinY{\ell}{t} \nonumber
\end{align}
Here, the variables are $q_{\ell}$ and $v_{\ell,j}$ for any $\ell \in [t]$ and $j \in [n]$; and the constants are (as before) $z_i$ for $i \in [t-1]$.
Recall that $u_j(A)$ is the utility that agent $j$ assigns to simple allocation $A$, as given by the value oracle.
% title: ellipsoid variant
This problem has an exponential number of constraints --- a constraint for each allocation (in line (\progAppDual.1)) but only a polynomial number of variables.
Using the ellipsoid method \cite{grotschel_ellipsoid_1981}, it could be solved in polynomial time 
if we had a \emph{separation oracle} ---
an oracle that given a vector $\upsilon$ either determines that $\upsilon$ is infeasible and returns a violated constraint, or asserts that $\upsilon$ is feasible.
Unfortunately, as we shall now see, it is NP-hard to compute a separation oracle to this problem.
\begin{lemma}
    Computing a separation oracle to \eqref{eq:app-dual} is NP-hard.
\end{lemma}

% very similar to what they did in yonatan's paper..
\begin{proof}
We prove that a separation oracle for \eqref{eq:app-dual} would allow us to compute a leximin optimal stochastic allocation.
    As discussed previously, computing such an allocation is NP-hard, so the same applies for computing a separation oracle for \eqref{eq:app-dual}.

    First, we prove that such a separation oracle can be used to extract an optimal solution to \eqref{eq:app-ver2-vsums-OP}.
    Assume that the ellipsoid method was operated with the given oracle to solve \eqref{eq:app-dual}.
    Let $C$ be the set of constraints that the oracle determined as being violated.
    Since the ellipsoid method operates in polynomial time, the size of the set $C$ is also polynomial.
    Let $V_C$ be the set of variables of \eqref{eq:app-ver2-vsums-OP} associated with the constraints in $C$.
    By complementary slackness, the variables in $V_C$ are the only ones that may get a \emph{positive} value in the corresponding optimal solution to \eqref{eq:app-ver2-vsums-OP}.
    Therefore, the program \eqref{eq:app-ver2-vsums-OP} with only the variables in $V_C$ (and the other variables equal to zero) has a polynomial size, and therefore can be solved exactly.


    But, by Observation \ref{obs:c2-to-c1-in-poly-time}, this would allow us to find the corresponding optimal solution to \eqref{eq:app-vsums-OP} in polynomial time.
    % \erel{Did we say that $\psi$ can be computed in polynomial time?}\eden{in the way it is written now is not, it iterate over each variable of \eqref{eq:app-vsums-OP} and there are exponential number of them. I need to think how to write it appropriately. maybe "that can be computed in time equals to the number of positive variables"?}
    % \erel{If it is not polynomial, then the reduction is not polynomial, so it does not imply NP-hardness}
    This means the described process can be used as an approximation procedure to \eqref{eq:vsums-OP} (that became \eqref{eq:app-vsums-OP} under the settings of this problem) with $\multError = \additiveError = 0$.
    Therefore, by Theorem \ref{th:main}, this means we can use Algorithm \ref{alg:basic-ordered-Outcomes} to obtain a leximin optimal solution\footnote{Actually, Theorem \ref{th:main} says that Algorithm \ref{alg:basic-ordered-Outcomes} will output a $(1,0)$-leximin-approximation; But Lemma \ref{lemma:absence-of-errors} says that such a solution is, indeed, a leximin optimal solution.}.
\end{proof}


% --- it would allow us to compute a leximin optimal stochastic allocation, which is, as discussed previously, NP-hard.

In Appendix \ref{sec:mult-variant-ellipsoid}, we present another variant of the ellipsoid method, which allows us to approximate the program \eqref{eq:app-ver2-vsums-OP} given a \emph{half-randomized approximate separation oracle} to \eqref{eq:app-dual}.
That is, an oracle that, given a multiplicative error $\multError$, a success probability $p$, and a vector $\upsilon$, either determines that $\upsilon$ is infeasible and returns a violated constraint; or determines that $\upsilon$ is $\multError$-\textit{approximately-feasible}, which means that for any constraint $a \cdot x \leq b$, the vector $\upsilon$ satisfies $a \cdot \upsilon \leq (1+\multError)\cdot b$.
When the oracle says that $\upsilon$ is $\multError$-approximately-feasible, it is correct with probability at least $p$.
Given such an oracle for the dual program, the ellipsoid method variant can be used to output a solution to the primal, that approximates it to the same factor with probability at least $p^I$, where $I$ is an upper bound on the number of iterations in any execution of the ellipsoid method variant on the dual (if it is given a deterministic oracle).
We can therefore conclude the following result:
\begin{lemma}\label{lemma:approx-sep-oracle-to-goal}
    Given a half-randomized approximate separation oracle to the problem \eqref{eq:app-dual}, with a multiplicative error of $\frac{\beta}{1-\beta}$ and a success probability $p$, a stochastic allocation that approximates leximin to a multiplicative error $\frac{\multError}{1-\multError+\multError^2}$ can be obtained with probability $p^{nI}$.
\end{lemma}

\begin{proof}
    % To begin, assume that we are given a deterministic approximate separation oracle (i.e., with failure probability $p=0$).
    As described above, we can use the ellipsoid method variant of Appendix \ref{sec:mult-variant-ellipsoid} with the given oracle to \eqref{eq:app-dual} to obtain a solution to \eqref{eq:app-ver2-vsums-OP},  that approximates it with a multiplicative error of $\frac{\multError}{1-\multError}$ with probability $p^I$.
    Then, by Observation \ref{obs:c2-to-c1-in-poly-time}, this would allow us to find the corresponding solution to \eqref{eq:app-vsums-OP}, that, with probability $p^I$, approximates it with a multiplicative error of $\multError$.
    That is, the described process can be used as a randomized approximation procedure to \eqref{eq:vsums-OP} (that became \eqref{eq:app-vsums-OP} under the settings of this problem).
    % with $\multError = \additiveError = 0$.
    Therefore, by Theorem \ref{th:main}, Algorithm \ref{alg:basic-ordered-Outcomes} can be used to obtain a leximin approximation to the original problem with only a multiplicative error of $\frac{\multError}{1-\multError+\multError^2}$ with probability $p^{nI}$ (Corollary \ref{corollary:main-with-probability}).
\end{proof}

Now, we show that such an oracle can be designed given a randomized approximation algorithm for computing a simple allocation that approximates the utilitarian welfare. Specifically, 

\begin{lemma}\label{lemma:alg-for-utilitarian-to-sep-oracle}
    Given a randomized approximation algorithm for computing a simple allocation that approximates the utilitarian welfare with multiplicative error $\multError$ and a success probability $p$, a half-randomized approximate separation oracle to \eqref{eq:app-dual} can be designed with a multiplicative error of $\frac{\beta}{1-\beta}$ and a success probability at least $\left(1-\frac{1}{nI}(1-p)\right)$.
\end{lemma}

% \eden{should say somewhere that the oracle is polynomial time and therefore everything is?...}
% FROM HERE: https://tex.stackexchange.com/a/675333/20929
\algdef{SE}[REPEATN]{REPEATN}{ENDREP}[1]{\algorithmicrepeat\ #1 \textbf{times}}{\algorithmicend\ \algorithmicrepeat}
\begin{algorithm}[!tbp]
\caption{A Half-Randomized Approximate Separation Oracle to \eqref{eq:app-dual}}
\label{alg:sep-oracle}
INPUT: variables $q_{\ell}$ and $v_{\ell,j}$ for any $\ell \in [t]$ and $j \in [n]$, an $\multApprox$-approximation algorithm for the utilitarian welfare problem (\eqref{eq:utilitarian}) with success probability $p$.
\begin{algorithmic}[1] %[1] enables line numbers
\STATE Iterate over constraints (\progAppDual.2)-(\progAppDual.5). If one of them is  violated, stop and return it.
\STATE \textbf{If} $p=1$ then set $T:=1$; \textbf{else} set $T := 1 + \lceil-\log_{(1-p)}(nI)\rceil$.

\REPEATN{$T$}
    \STATE Operate the algorithm for the utilitarian welfare problem on $n,m,(u'_j)_{j=1}^n$ to obtain an allocation $\Tilde{A}$ with value $\nu$.
    \IF{$\nu > 1$}  
        \STATE Return the corresponding violated constraint $\sum_{j=1}^n u_j(\Tilde{A}) \sum_{\ell=1}^t v_{\ell,j} > 1$
    \ENDIF
\ENDREP
\STATE Return "the assignment is approximately-feasible".

\end{algorithmic}
\end{algorithm}


Algorithm \ref{alg:sep-oracle} describes the oracle.
It accepts as input an assignment to the variables of \eqref{eq:app-dual}, that is, $q_{\ell}$ and $v_{\ell,j}$ for any $\ell \in [t]$ and $j \in [n]$, and an algorithm for approximating the maximum utilitarian welfare.
It starts by verifying constraints (\progAppDual.2)-(\progAppDual.5) one by one (this is possible as their number is polynomial in $n$ and $m$). 
If a violated constraint was found, the oracle simply returns it. Otherwise, it proceeds to check constraints (\progAppDual.1).
Although the number of constraints described by (\progAppDual.1) is exponential in $n$, they can be treated collectively in polynomial time (as in \cite{kawase_max-min_2020}).
% \eden{here maybe to say something about the randomness}.\erel{Maybe mention that \textcite{kawase_max-min_2020} ignored this issue.}
First, notice that in order to determine whether the expression $\sum_{j=1}^n u_j(A) \sum_{\ell=1}^t v_{\ell,j}$ is at most $1$ for all simple allocations ($A \in \mathcal{A}$), it is sufficient to check the allocation that maximizes this expression and compare it to $1$.
Define new utility functions for all $j \in [n]$ and $A \in \mathcal{A}$, 
\begin{align*}
u'_j(A) := \sum_{\ell=1}^t v_{\ell,j} \cdot u_j(A) 
\end{align*}
The above expression can be simplified to $\sum_{j=1}^n u'_j(A)$. An allocation that maximizes this expression is an allocation that maximizes the utilitarian welfare (i.e., the sum of utilities) when the same sets of agents and items is considered but with different utilities%
\footnote{Notice that the utilities $u'_j$ are  normalized, monotone, submodular, and can be computed using $t\leq n$ calls to the value oracle of $u_j$}
($u'_j$ instead of $u_j$ for $j \in [n]$).
Such an allocation cannot be found in polynomial time since approximating the utilitarian welfare up to a factor better than $(1-\frac{1}{e})$ in the case of submodular utilities is known to be NP-hard \cite{khot_inapproximability_2008}.
However, the oracle is given an approximation algorithm to the utilitarian welfare problem as input.
Therefore, an allocation $\Tilde{A}$ with utilitarian value at least $(1-\multError)$ of the optimal can be obtained with probability $p$.
We shall now see that it is enough.


\begin{proof}[Proof of Lemma \ref{lemma:alg-for-utilitarian-to-sep-oracle}]
First, observe that when Algorithm \ref{alg:sep-oracle} returns a violated constraint, it is always correct.
This is obvious for constraints described by (\progAppDual.2)-(\progAppDual.5), since these constraints have been verified directly.
For constraints described by (\progAppDual.1), it means that the algorithm found an allocation $\Tilde{A}$ that satisfies $\sum_{j=1}^n u'_j(\Tilde{A}) > 1$.
    By the definition of $u'$, the constraint corresponding to this allocation is, indeed, violated:
    \begin{align*}
         \sum_{j=1}^n u_j(\Tilde{A}) \sum_{\ell=1}^t v_{\ell,j} = \sum_{j=1}^n u'_j(\Tilde{A}) > 1.
    \end{align*}
Let us assume that the given algorithm for the utilitarian welfare problem is deterministic (i.e., $p=1$) and then revisit the case $p<1$.
    Assume that the oracle said that the assignment is approximately-feasible.
    This means that the algorithm for the utilitarian welfare problem found an allocation $\Tilde{A}$ with value at most $1$.
    Since $\Tilde{A}$ is approximately-optimal, the optimal utilitarian value is at most $1/(1-\multError)\cdot 1$.
    As this is an upper bound of the utilitarian value of any allocation, it follows that all the constraints described bu (\progAppDual.1) are $\frac{\multError}{1-\multError}$-approximately maintained --- that is, for any allocation $A \in \mathcal{A}$ the following holds:
    \begin{align*}
        \sum_{j=1}^n u'_j(A) = \sum_{j=1}^n u_j(A) \sum_{\ell=1}^t v_{\ell,j} \leq \frac{1}{1-\multError}\cdot 1 = \left(1+\frac{\multError}{1-\multError}\right)\cdot1
    \end{align*}
    We get that, in this case, the oracle is also deterministic, and that the success probability is at least $\left(1-\frac{1}{nI}(1-p)\right) = 1$ for $p=1$.

    Assume now that $p<1$. Then, the oracle may be incorrect when it says the assignment is approximately feasible, but only if the algorithm for the utilitarian welfare problem did not return an appropriate approximation in all $T = \lceil-\log_{(1-p)}(nI)\rceil + 1$ operations, that is, with probability at most $(1-p)^T$.
    % as each operation of the oracle is independent
    Notice that $T>1$ since $\log_{(1-p)}(nI) < 0$\footnote{
    % The fact that  $\log_{(1-p)}(nI) < 0$ can be easily concluded 
    Since $(1-p)\in(0,1)$ and $nI>1$ by change of base: $\log_{(1-p)}(nI) = \log(nI)/\log(1-p)$, the numerator is positive and the denominator is negative.}.
    Now, as $T \geq -\log_{(1-p)}(nI) + 1$ and $(1-p)<1$ we get that:
    \begin{align*}
        &(1-p)^T \leq (1-p)\cdot(1-p)^{-\log_{(1-p)}(nI)} = (1-p)(nI)^{-1}
    \end{align*}
    So, the success probability is at least $\left(1-\frac{1}{nI}(1-p)\right)$.
\end{proof}

We can now prove Theorem \ref{th:app-main}.

\begin{proof}[Proof of Theorem \ref{th:app-main}]
    Assume we are given an algorithm that returns a simple allocation that approximates the utilitarian welfare with multiplicative error $\multError$ with success probability $p$.
    By Lemma \ref{lemma:alg-for-utilitarian-to-sep-oracle} this algorithm can be used to obtain an half-randomized approximate separation oracle to \eqref{eq:app-dual} with a multiplicative error $\frac{\multError}{1-\multError}$ with success probability $\left(1-\frac{1}{nI}(1-p)\right)$.
    By Lemma \ref{lemma:approx-sep-oracle-to-goal}, with such an oracle a stochastic allocation that approximates leximin to a multiplicative error of $\frac{\multError}{1-\multError+\multError^2}$ can be obtained with probability $\left(1-\frac{1}{nI}(1-p)\right)^{nI}$.
    If $p=1$ then the success probability is $1$ too (at least $\left(1-\frac{1}{nI}(1-p)\right)^{nI}= 1$).
    However, if $p<1$, then $\frac{1}{nI}(1-p) \in (0,1)$ and therefore the success probability is at least $p$\footnote{For any $\epsilon \in (0,1)$ and $k \in \mathbb{Z}_{+} \colon \Hquad (1 - \epsilon)^k \geq 1 - k \cdot \epsilon$}:
    \begin{align*}
        \left(1-\frac{1}{nI}(1-p)\right)^{nI} \geq \left(1-nI\cdot\frac{1}{nI}(1-p)\right) = p.   \end{align*}
\end{proof}

\section{Equivalent Single-objective Optimization Problems in the Presence of Errors}\label{sec:equivalent-proofs}

Many times, when referring to two optimization problems\footnote{In this section,  we consider only single-objective optimization problems.} as equivalent, one means that they have the same optimal value.
When two problems satisfy this relation, it is clear that in order to obtain an optimal \emph{value}, a solver\footnote{It is assumed that a solver (either approximate or exact) for a single-objective optimization problem returns a solution and its objective value.\eden{maybe to explain it better..}} for one can be used as a solver for the other. 
However, if we are interested in an optimal \emph{solution} that yields this value, a solver that returns an optimal solution for another problem with the same optimal value is not enough\eden{reduction to feasibility problem}.
Moreover, when it comes to approximation, even if we are only concerned about the objective value, an approximate solver for one can no longer be used for the other.
To illustrate, consider the following problems:
\begin{align*}
    (E1) \Hquad &\max\quad x                         &&& (E2)\Hquad &  \max\quad x\\
    &\Hquad s.t.\quad  x \in \{0.9,1\}       &&&& \Hquad s.t.\quad x \in \{0.95,1\} 
\end{align*}    
Both problems have the same optimal objective value $1$.
Now, assume that a multiplicative error of $0.1$ is acceptable.
An approximate solver for the problem $(E1)$ may return the objective value $0.9$, which is not a possible value of $(E2)$; similarly, an approximate solver for the problem $(E2)$ may return the objective value $0.95$, which is not a possible value of $(E2)$.
% Thus, although both problems have the same optimal value, an approximate solver for one problem \emph{cannot} be used as an approximate solver for the other.

In this appendix, we present a new definition of equivalent optimization problems, which requires a stronger relationship.
We prove that, according to our definition, when two optimization problems are equivalent, a solver for one, either exact or approximate, can also be used for the other.

\paragraph{Equivalent problems definition} We say that two (single-objective) optimization problems, $OP1 = (S_1,f_1)$ and $OP2 = (S_2,f_2)$, are \emph{equivalent} if they are from the same type --- either both are maximization problems or both are minimization problems; and there exists a bijection, $B \colon S_1 \to S_2$, mapping each solution of $OP1$, $x \in S_1$, to a unique solution of $OP2$, $B(x) \in S_2$, and they have the same objective value $f_1(x) = f_2(B(x))$.

% It is easy to conclude that this relation is symmetric, reflexive and transitive and therefore it is, indeed, an equivalence relation.
The following observation can be easily concluded by the definition:
\begin{observation}
    The equivalent relation between problems is transitive, reflexive and symmetric.
\end{observation}
% \begin{observation}
%     The equivalent relation between problems is transitive.
% \erel{maybe also argue that it is reflexive and symmetric, so it is an equivalence relation.}
% \end{observation}
% \eden{to explain }

If we are only concerned with the objective value, then the following lemma ensures that an approximate solver for one problem can be applied, as is, to the other (it is not necessary to know what the bijection is):

\begin{lemma}\label{lemma:approx-value-equivalent-prob}
    Let $OP1 = (S_1,f_1)$ and $OP2 = (S_2,f_2)$ be equivalent optimization problems, and let $v_1 \in \mathbb{R}$ be an $(\multApprox,\additiveApprox)$-approximation of the optimal objective value of $OP1$.
    Then, $v_1$ is also an $(\multApprox,\additiveApprox)$-approximation of the optimal objective value of $OP2$.
% \erel{Add that the approximation ratios ($\alpha$,$\epsilon$) is the same}
\end{lemma}

\begin{proof}
    For brevity, we prove the claim only for maximization problems, the proof for minimization problems is similar.
    
    Let $x^*\in S_1$ and $y^*\in S_2$ be optimal solutions of the problems $OP1$ and $OP2$ respectively.
    In order to prove that $v_1$ is an $(\multApprox,\additiveApprox)$-approximation of the optimal objective value of $OP2$, we will show that there is a solution $y \in S_2$ with objective value $v_1$, and also that $v_1 \geq \multApprox f_2(y^*) - \additiveApprox$.

    First, since $v_1$ is an $(\multApprox,\additiveApprox)$-approximation of the optimal objective value of $OP1$, there exists a solution $x \in S_1$ such that $f_1(x) = v_1$ and also $v_1 \geq \multApprox f_1(x^*) - \additiveApprox$.
    By definition of equivalent problems, the corresponding solution to $OP2$ by the bijection, $B(X) \in S_2$, has the same objective value $f_2(B(x)) = v_1$.
    
    In addition, we shall now see that both problems have the same optimal objective value.
    Let $B: S_1\to S_2$ be a bijection as described in the definition of equivalent problems.    
    So $f_1(x^*)=f_2(B(x^*))$, and $f_2(B(x^*))\leq f_2(y^*)$ by optimality of $y^*$, so $f_1(x^*)\leq f_2(y^*)$. By analogous arguments $f_2(y^*)\leq f_1(x^*)$, so in fact $f_1(x^*) = f_2(y^*)$.
    
    Therefore, we can conclude that:
    \begin{align*}
        f_2(B(x)) = v_1 \geq \multApprox f_1(x^*) - \additiveApprox =  \multApprox f_2(y^*) - \additiveApprox
    \end{align*}
    as required.
\end{proof}

% \begin{lemma}\label{lemma:solver-equivalent-prob}
%     Let $OP1 = (S_1,f_1)$ and $OP2 = (S_2,f_2)$ be equivalent optimization problems. Then, in order to approximate the optimal value, an $(\multApprox,\additiveApprox)$-approximate solver for one can be used as an $(\multApprox,\additiveApprox)$-approximate solver for the other.
% % \erel{Add that the approximation ratios ($\alpha$,$\epsilon$) is the same}
% \end{lemma}

% \begin{proof}
%     For brevity, we prove the claim only for maximization problems, the proof for minimization problems is similar.
%     Let $x^*\in S_1$ and $y^*\in S_2$ be optimal solutions of the problems $OP1$ and $OP2$ respectively.
%     Let $B: S_1\to S_2$ be a bijection as described in the definition of equivalent problems.    
%     So $f_1(x^*)=f_2(B(x^*))$, and $f_2(B(x^*))\leq f_2(y^*)$ by optimality of $y^*$, so $f_1(x^*)\leq f_2(y^*)$. By analogous arguments $f_2(y^*)\leq f_1(x^*)$, so in fact $f_1(x^*) = f_2(y^*)$.
%     % , since otherwise one of them is higher, and therefore the bijection can be used to obtain a solution to the second problem with value higher than optimal. \eden{need to rewrite it..}
%     Now, 
%     % without loss of generality, 
%     assume that we have an $(\multApprox, \additiveApprox)$-approximate solver for $OP1$, for some $\multApprox\in(0,1]$ and $\additiveError\geq 0$.
%     That is, the solver returns a solution $x \in S_1$ such that $f_1(x) \geq \multApprox \cdot f_1(x^*) - \additiveError$. 
%     Consider the corresponding solution to $OP2$ by the bijection, $B(X) \in S_2$, we know that $f_1(x_1) = f_2(B(x_1))$.
%     It follows that $B(x)$ is an $(\multApprox, \additiveApprox)$-approximation to $OP2$:
%     \begin{align*}
%         f_2(B(x)) = f_1(x) \geq \multApprox \cdot f_1(x^*) - \additiveError = \multApprox \cdot f_2(y^*) - \additiveError
%     \end{align*}
% \end{proof}

Notice that the approximation value is obtained by the corresponding solution ($B(X)$), and therefore, we can also conclude the following result:
% Therefore, if we also have access to procedures to calculate the bijection and its inverse, then we can use a solver for one problem to find the solution to the other, that is:
\begin{corollary}\label{corollary:solver-equivalent-prob}
    Let $OP1 = (S_1,f_1)$ and $OP2 = (S_2,f_2)$ be equivalent optimization problems, and let $P_{1\to 2}$ be a procedure that, given a solution to $OP1$, returns the corresponding solution to $OP2$.
    Then, an $(\multApprox, \additiveApprox)$-approximate solver for $OP1$ can be used to obtain a \emph{solution} that is an $(\multApprox, \additiveApprox)$-approximation for $OP2$.
\end{corollary}

If the procedure from $OP1$ to $OP2$ operates in polynomial time we say that $OP1$ is \emph{polynomial-time equivalent} to $OP2$. 

\eden{how is the name "polynomial-time equivalent"?}

% \eden{If we will have time: "Further, if the bijection is given and can be calculated in polynomial time, then ....}



\subsection{Relationships Between Single-Objective Problems for Leximin Optimization}
\eden{I'm not sure which title to give}

For clarity, descriptions of all the problems are provided here as well (table \ref{table:prob-des}).

\begin{table}[h!]
\begin{tabular}{l}
\hline
\\
$\begin{aligned}
     \text{(P1)}\Hquad \max \quad &\ztVar{x}  \;\;
        s.t. &\quad  & (1) \quad x \in S \\
              &     & & (2) \quad \valBy{\ell}{x}\geq z_{\ell} & \ell = 1,\ldots,t-1\nonumber \\
               &    & & (3) \quad \valBy{t}{x} \geq \ztVar{x} \nonumber  \\\\
    \text{(P2)}\Hquad\max \quad &\ztVar{x}  \;\;
        s.t. &\quad  & (1) \quad x \in S  \\
        &&& (\hat{2}) \quad \sum_{i \in F'} f_i(x) \geq \sum_{i=1}^{|F'|}  z_i & \forall F' \subseteq [n], |F'| < t \\
        &&& (\hat{3}) \quad \sum_{i \in F'} f_i(x) \geq \sum_{i=1}^{t}  z_i  & \forall F' \subseteq [n], |F'| = t\\\\
     \text{(P3)}\Hquad \max \quad &\ztVar{x}  \;\;
        s.t. &\quad  & (1) \quad x \in S  \\
                    &&& (2) \quad \ell y_{\ell} - \sum_{j=1}^n m_{\ell,j}\geq \sum_{i=1}^{\ell}  z_i & \ell = 1, \ldots,t-1 \nonumber \\
                    &&& (3) \quad t y_t - \sum_{j=1}^{n} m_{t,j} \geq \sum_{i=1}^{t}  z_i  \nonumber \\
                    &&& (4) \quad m_{\ell,j} \geq y_{\ell} - f_j(x)  & \ell = 1, \ldots,t,\Hquad j = 1, \ldots,n \nonumber \\
                    &&& (5) \quad m_{\ell,j} \geq 0  & \ell = 1, \ldots,t,\Hquad j = 1, \ldots,n \nonumber\\\\
    \text{(P2-compact)}& \\
    \max \quad &z_t  \;\;
        s.t. &\quad  & (1) \quad x \in S \\
                    &&& (\Tilde{2}) \quad \sum_{i=1}^{\ell} \valBy{i}{x} \geq \sum_{i=1}^{\ell}  z_i & \ell = 1,\ldots, t-1 \nonumber\\
                    &&& (\Tilde{3}) \quad \sum_{i=1}^{t} \valBy{i}{x} \geq \sum_{i=1}^{t}  z_i\\
\end{aligned}$\\
\\
\hline
\end{tabular}
\caption{Summary description of the problems.}
\label{table:prob-des}
\end{table}


\subsubsection{Equivalence of The Problems \eqref{eq:sums-OP} and \eqref{eq:compact-OP}}\label{sec:prob-sums-and-comp}
we prove that the \emph{identity function} is an appropriate bijection between \eqref{eq:sums-OP} and \eqref{eq:compact-OP}. Therefore, they are polynomial-time equivalent to each other. 

We start by proving the following lemma:
\begin{lemma}\label{lemma:sums-to-comp-constrants}
    For any $x \in S$, any $\ell \in [n]$ and a constant $c \in \mathbb{R}$ the following two conditions are equivalent:
    \begin{align}\label{eq:sums-to-comp-constrants}
         \forall F' \subseteq [n], |F'| = \ell \colon \sum_{i \in F'} f_i(x) &\geq c 
         \\
         \sum_{i=1}^{\ell} \valBy{i}{x}&\geq c 
    \end{align}
\end{lemma}

\begin{proof}
    For the first direction, recall that the values $ \valBy{1}{x}, \dots,  \valBy{\ell}{x}$ were obtained from $\ell$ objective functions (those who yield the smallest value).
    By the assumption, the sum of any set of function with size $\ell$ is at least $c$; therefore, it is true in particular for the functions corresponding to the values $ (\valBy{1}{x})_{i=1}^{\ell}$.
    For the second direction, assume that $\sum_{i=1}^{\ell} \valBy{i}{x}\geq c$.
    Since $ \valBy{1}{x}, \dots,  \valBy{\ell}{x}$ are the $\ell$ smallest values in $\allValues{x}$, we get that:
    \begin{align*}
       \forall F' \subseteq [n],\Hquad |F'| = \ell \colon \quad \sum_{i \in F'}f_i(x) \geq \sum_{i=1}^s \valBy{i}{x}\geq c.
    \end{align*}
\end{proof}

    Now, let $(x,z_t)$ be a solution to \eqref{eq:sums-OP}. 
    As $x$ satisfies constraint (1) of \eqref{eq:sums-OP}), it is also satisfies constraint (1) of \eqref{eq:compact-OP} (as both constraints are the same, $x \in S$).
    In addition, as $x$ satisfies constraint $(\hat{2})$ of \eqref{eq:sums-OP}, for any $\ell \in [t-1]$, 
    \begin{align*}
        \forall F' \subseteq [n], |F'| = \ell \colon \sum_{i \in F'} f_i(x) \geq \sum_{i=1}^{\ell} z_i
    \end{align*}
    by Lemma \ref{lemma:sums-to-comp-constrants}, also $\sum_{i=1}^{\ell} \valBy{i}{x} \geq \sum_{i=1}^{\ell} z_i$. Therefore, $x$ satisfies constraint $(\Tilde{2})$ of \eqref{eq:compact-OP}.
    Lastly, as $x$ and $z_t$ satisfy constraint $(\hat{3})$ of \eqref{eq:sums-OP}, 
    \begin{align*}
        \forall F' \subseteq [n], |F'| = t \colon \sum_{i \in F'} f_i(x) \geq \sum_{i=1}^{t} z_i
    \end{align*}
    again, by Lemma \ref{lemma:sums-to-comp-constrants} also $\sum_{i=1}^{t} \valBy{i}{x} \geq \sum_{i=1}^{t} z_i$.
    So, $x$ ans $z_t$   satisfy constraint $(\Tilde{3})$ of \eqref{eq:compact-OP}.
    Since we saw that $x$ ans $z_t$ satisfy all the constraints of \eqref{eq:compact-OP}, it is feasible to this problem.

    As in both problems the objective value is determined by $z_t$, it is clear that $(x,z_t)$ obtains the same objective value from both \eqref{eq:sums-OP} and \eqref{eq:compact-OP}.

    Therefore, the identity function (i.e., $B((x,z_t)) = (x,z_t)$) is an appropriate bijection and so, the problems are equivalent.



%--------------------------------------------------
\subsubsection{Equivalence of The  problems \eqref{eq:compact-OP} and \eqref{eq:vsums-OP}} We prove that these problems are equivalent by describing an appropriate bijection.
We will also see that this bijection and its inverse can be calculated in polynomial time and therefore, each problem is polynomial-time equivalent to the other.

We start with the following lemma:
\begin{lemma}\label{lemma:comp-to-p3-m-sums}
    For any $x \in S$ and any constant $c \in C$,
    \begin{align*}
        \sum_{j=1}^n \max(0, c - f_j(x) ) = \sum_{j=1}^n \max(0, c - \valBy{j}{x} )
    \end{align*}
\end{lemma}
\begin{proof}
     Let $(\pi_1, \ldots, \pi_n)$ be a permutation of $\{1,\ldots,n\}$ such that $f_{\pi_i}(x) = \valBy{i}{x}$ for any $i \in [n]$ (notice that such permutation exists by the definition of $\valBy{}{}$).
     That is, the value that $f_{\pi_i}$ obtains is the ${\pi_i}$-th smallest one in the multiset of all values $\allValues{x}$.
    Since each element in the sum $\sum_{j=1}^n \max(0, c - f_j(x))$ is affected by $j$ only through $f_j(x)$, the permutation $\pi$ allows us to conclude the following:
    \erel{This argument is not clear}\eden{better?}
    \begin{align*}
        \sum_{j=1}^n \max(0, c -f_j(x)) &= \sum_{j=\pi_1}^{\pi_n} \max(0,c -f_j(x)\\ 
        &= \sum_{j=1}^n \max(0,c -f_{\pi_i}(x)  = \sum_{j=1}^{n} \max(0,c -\valBy{j}{x})
    \end{align*}
\end{proof}



Following is Lemma \ref{lemma:comp-to-p3-mapping}, which describes a function $B$ and proves that it is a mapping from the feasible region of the problem \eqref{eq:compact-OP} to the feasible region of the problem \eqref{eq:vsums-OP}.
Then, Lemma \ref{lemma:comp-to-p3-is-bij} proves that this mapping is a bijection.
Lastly, Lemma \ref{lemma:comp-to-p3-obj} shows that the same objective value is obtained.

\begin{lemma}\label{lemma:comp-to-p3-mapping}
    Let $(x,z_t)$ be a feasible solution  to \eqref{eq:compact-OP}. Then $B((x,z_t)) = (x, z_t, (y_1,\ldots,y_n), (m_{1,1},\ldots,m_{n,n}))$ is a feasible solution to \eqref{eq:vsums-OP}, where
    \begin{align*}
        \quad y_{\ell} &:= \valBy{\ell}{x} \Hquad\forall \ell \in [n], 
        \\
        m_{\ell,j} &:= \max(0,y_{\ell} -f_j(x)) \Hquad \forall \ell \in [n], \Hquad \forall 1 \leq j \leq n 
    \end{align*}
\end{lemma}

\begin{proof}
    First, since $x$ satisfies constraint (1) of \eqref{eq:compact-OP}, it is also satisfies constraint (1) of \eqref{eq:vsums-OP} (as both constraints are the same).
    Also, as $m_{\ell,j} \geq 0$ and $m_{\ell,j} \geq y_{\ell} - f_j(x)$ for any $\ell \in [n]$ and $j \in [n]$, this assignment satisfies constraints (4) and (5) of \eqref{eq:vsums-OP}.
    
    To show that this assignment also satisfies constraints (2) and (3) of problem \eqref{eq:vsums-OP}, we first prove that for any $\ell \in [n]$ and any constant $c \in \mathbb{R}$ this assignment satisfies the following:
    \begin{align}\label{eq:comp-to-p3}
        \sum_{i=1}^{\ell} \valBy{i}{x}\geq c \Hquad \Longrightarrow \Hquad \ell y_{\ell} - \sum_{j=1}^n m_{\ell,j}\geq c
    \end{align}
    As $y_{\ell} = \valBy{\ell}{x}$, also  $m_{\ell,j} = \max(0,\valBy{\ell}{x} -f_j(x))$.
    % in this way it is easy to see that $j$ affects $m$ only through $f_j(x)$.
    And so, by Lemma \ref{lemma:comp-to-p3-m-sums}, it can also be described as $\sum_{j=1}^{n} \max(0,\valBy{\ell}{x} -\valBy{j}{x})$.
    Since $\valBy{\ell}{x}$ is the $\ell$-th smallest objective, it is clear that $\valBy{\ell}{x} - \valBy{j}{x} \leq 0$ for any $j > \ell$, and $\valBy{\ell}{x} - \valBy{j}{x} \geq 0$ for any $j \leq \ell$.
    We can now conclude that $\ell y_{\ell} - \sum_{j=1}^n m_{\ell,j}\geq c$:
    \begin{align*}
        &\ell y_{\ell} - \sum_{j=1}^n m_{\ell,j} = \ell \cdot \valBy{\ell}{x} - \sum_{j=1}^n \max(0,\valBy{\ell}{x} -\valBy{j}{x}) \\
        &= \ell \cdot \valBy{\ell}{x} - \sum_{j=1}^{\ell} \max(0,\valBy{\ell}{x} -\valBy{j}{x}) - \sum_{j=\ell+1}^n \max(0,\valBy{\ell}{x} -\valBy{j}{x}) \\
        &= \ell \cdot \valBy{\ell}{x} - \sum_{j=1}^{\ell} \left(\valBy{\ell}{x} -\valBy{j}{x}\right) - \sum_{j=\ell+1}^n 0 = \ell \cdot \valBy{\ell}{x} - \ell \cdot\valBy{\ell}{x} + \sum_{j=1}^{\ell} \valBy{j}{x}\\
        &= \sum_{j=1}^{\ell} \valBy{j}{x} \geq  c \text{~~~by assumption.}
    \end{align*}

    Now, since $x$ satisfies constraint $(\Tilde{2})$ of \eqref{eq:compact-OP}, for any $\ell \in [t-1]$, $\sum_{i=1}^{\ell} \valBy{i}{x} \geq \sum_{i=1}^{\ell} z_i$ and so by equation \ref{eq:comp-to-p3}, $\ell y_{\ell} - \sum_{j=1}^n m_{\ell,j}\geq  \sum_{i=1}^{\ell} z_i$
    Therefore, this assignment constraint (2) of problem \eqref{eq:vsums-OP}.
    In addition, as $x$ and $z_t$ satisfy constraint $(\Tilde{3})$ of \eqref{eq:compact-OP}, $\sum_{i=1}^{t} \valBy{i}{x} \geq \sum_{i=1}^{t} z_i$ and so by equation \ref{eq:comp-to-p3}, \ref{eq:comp-to-p3}, $t y_{t} - \sum_{j=1}^n m_{t,j}\geq  \sum_{i=1}^{t} z_i$.
    This means that also satisfies constraints (3) of problem \eqref{eq:vsums-OP}.
\end{proof}

\begin{lemma}\label{lemma:comp-to-p3-is-bij}
    The mapping $B$ is a bijection.
\end{lemma}

\begin{proof}
    Injective ($B(a) = B(b) \Rightarrow a = b$) is trivial since $x$ and $z_t$ are part of the solution.
    
    To prove that the mapping is surjective, we will show that for any feasible solution to \eqref{eq:vsums-OP}, that is,
    \begin{align*}
        (x \in S, z_t, y_1, \ldots, y_t, m_{1,1}, \ldots, m_{1,n}, m_{2,1}, \ldots, m_{2,n},\ldots, m_{t,1}, \ldots, m_{t,n})
    \end{align*}
    there is a feasible solution to \eqref{eq:compact-OP} that is  mapped to this solution.
    In fact, we prove that $(x,z_t)$ does.

    It is easy to see that since $x$ satisfies constraint (1) of \eqref{eq:vsums-OP}, it is also satisfies constraint (1) of \eqref{eq:compact-OP} (as both are the same).
    To show that it also satisfies constraints $(\Tilde{2})$ and $(\Tilde{3})$ of \eqref{eq:compact-OP}, we start by proving that for any $\ell \in [n]$ and any constant $c \in \mathbb{R}$:
    \begin{align}\label{eq:p3-to-comp}
         \ell y_{\ell} - \sum_{j=1}^n m_{\ell,j}\geq c
         \Hquad \Longrightarrow \Hquad \sum_{i=1}^{\ell} \valBy{i}{x}\geq c
    \end{align}
    Notice that, for any $j\in [n]$ and any $\ell \in [n]$, $m_{\ell,j} \geq y_{\ell} - f_j(x)$ by constraint (4) of \eqref{eq:vsums-OP}, and also $m_{\ell,j} \geq 0$ by constraint (5) of \eqref{eq:vsums-OP}.
    Therefore, $m_{\ell,j} \geq \max(0,y_{\ell} -f_j(x))$.
    And so, by Lemma \ref{lemma:comp-to-p3-m-sums}:
    \begin{align}\label{eq:p3-to-conp-m-sum}
        \sum_{j=1}^n m_{\ell,j} \geq  \sum_{j=1}^n \max(0,y_{\ell} -f_j(x)) = \sum_{j=1}^n \max(0,y_{\ell} -\valBy{j}{x})
    \end{align}
    Now, suppose by contradiction that $\ell y_{\ell} - \sum_{j=1}^n m_{\ell,j}\geq c$ but at the same time $\sum_{i=1}^{\ell} \valBy{i}{x}< c$ (equation \ref{eq:p3-to-comp}).
    Since $\ell y_{\ell} - \sum_{j=1}^n m_{\ell,j}\geq c$, by equation \ref{eq:p3-to-conp-m-sum} also:
    \begin{align*}
        c \leq \ell y_{\ell} - \sum_{j=1}^n m_{\ell,j} \leq \ell y_{\ell} - \sum_{j=1}^n \max(0,y_{\ell} -\valBy{j}{x})
    \end{align*}
    But, as $\sum_{i=1}^{\ell} \valBy{i}{x}< c$ this lead to contradiction:
\begin{align*}
       &\sum_{i=1}^{\ell} \valBy{i}{x} < c \leq \ell y_{\ell} - \sum_{j=1}^n \max(0,y_{\ell} -\valBy{j}{x})\\
       \Rightarrow \Hquad & \ell y_{\ell} - \sum_{i=1}^{\ell} \valBy{i}{x} - \sum_{j=1}^n \max(0,y_{\ell} -\valBy{j} {x}) > 0\\
       \Rightarrow \Hquad & \sum_{i=1}^{\ell} y_{\ell} - \sum_{i=1}^{\ell} \valBy{i}{x} - \sum_{j=1}^n \max(0,y_{\ell} -\valBy{j} {x}) > 0\\
       \Rightarrow \Hquad & \sum_{i=1}^{\ell}\left( y_{\ell} - \valBy{i}{x} \right) - \sum_{j=1}^{\ell} \max(0,y_{\ell} -\valBy{j} {x}) - \sum_{j=\ell+1}^n \max(0,y_{\ell} -\valBy{j} {x}) > 0\\
        \Rightarrow \Hquad &  \sum_{j=1}^{\ell} \underbrace{\left((y_{\ell} - \valBy{j}{x}) - \max(0,y_{\ell} -\valBy{j}{x})\right)}_{\text{each element } \leq 0} - \sum_{j=\ell+1}^n \underbrace{\max(0,y_{\ell} -\valBy{j}{x})}_{\text{each element } \geq 0} >  0\\
     \Rightarrow \Hquad & 0 > 0
   \end{align*}

    Now, as constraint (2) of problem \eqref{eq:vsums-OP} is satisfied, for any $\ell \in [t-1]$,  $\ell y_{\ell} - \sum_{j=1}^n m_{\ell,j}\geq  \sum_{i=1}^{\ell} z_i$, and so by equation \ref{eq:p3-to-comp}, also $\sum_{i=1}^{\ell} \valBy{i}{x} \geq \sum_{i=1}^{\ell} z_i$.
    This implies that $x$ satisfies constraint $(\Tilde{2})$ of \eqref{eq:compact-OP}.
    Similarly, as constraint (3) of problem \eqref{eq:vsums-OP} is satisfied,  $t y_{t} - \sum_{j=1}^n m_{t,j}\geq  \sum_{i=1}^{t} z_i$, and so by equation \ref{eq:p3-to-comp}, also $\sum_{i=1}^{t} \valBy{i}{x} \geq \sum_{i=1}^{t} z_i$.
    This implies that $x$ and $z_t$ satisfy constraint $(\Tilde{3})$ of \eqref{eq:compact-OP}.
\end{proof}


\begin{lemma}\label{lemma:comp-to-p3-obj}
    $(x,z_t)$ and $B((x,z_t))$ obtain the same objective value from the problems \eqref{eq:compact-OP} and \eqref{eq:vsums-OP} respectively.
\end{lemma}

\begin{proof}
    As in both problems the objective value is determined by $z_t$, by the definition of $B$ (the variable $z_t$ is mapped to itself), it is clear that $(x,z_t)$ and $B((x,z_t))$ obtains the same objective value from \eqref{eq:compact-OP} and \eqref{eq:vsums-OP} respectively.
\end{proof}


%--------------------------------------------------
\subsubsection{Relationship Between the Problems \eqref{eq:basic-OP} and \eqref{eq:sums-OP}} 
% Both problems are depended on a set of constants $z_1, \ldots, z_{t-1}$, 
We shall now prove Lemma \ref{lemma:alg-1-can-use-sums-exact} (Section \ref{sec:algo-short}), which says that in Algorithm \ref{alg:basic-ordered-Outcomes}, a solver for \eqref{eq:sums-OP} can be used (instead of for \eqref{eq:basic-OP}), and the algorithm will still output a leximin optimal solution.

\begin{proof}[Proof of Lemma \ref{lemma:alg-1-can-use-sums-exact}]
    Contrariwise, suppose that the returned solution, $x^*$, is not leximin optimal.
    This means that there exists a solution, $y \in S$, that leximin-preferred over it.
    That is, there exists an integer $k \in [n]$ such that:
    \begin{align*}
        \forall j < k \colon &\valBy{j}{y} = \valBy{j}{\retSol};\\
        & \valBy{k}{y} > \valBy{k}{\retSol}.
    \end{align*}
    In addition, since $x^*$ is the returned solution, it is the solution of \eqref{eq:sums-OP} that was solved in the last iteration and therefore $\sum_{i=1}^{s} \valBy{i}{s} \geq \sum_{i=1}^{s} z_i$ for any $s \in [n]$ (by constraint  $(\hat{2})$ for $s<n$ and constraint  $(\hat{3})$ for $s=n$).
    Now, consider \eqref{eq:sums-OP} that was solved in iteration $t$.
    Since $y$ is a solution ($y \in S$) it satisfies constraint (1).
    It is also easy to see that $y$ satisfies constraint $(\hat{2})$ --- for any $s \in [k-1]$:
    \begin{align*}
        &\sum_{i=1}^s \valBy{i}{y} = \sum_{i=1}^s \valBy{i}{\retSol}
        && \text{since } i\leq s<k \text{ and $y$'s def.}\\
        & \geq \sum_{i=1}^s z_i
    \end{align*}
    Moreover, since $z_t$ is a variable in this problem, it satisfies constraint $(\hat{3})$ with any $z_t \geq \sum_{i=1}^t \valBy{i}{y} - \sum_{i=1}^{t-1} z_i$.
    Therefore, it is feasible to this problem. 
    But, the objective value obtained by $y$ is higher than the optimal value, $z_t$, which is a contradiction:
    \begin{align*}
        \sum_{i=1}^t \valBy{i}{y} - \sum_{i=1}^{t-1} z_i > \sum_{i=1}^t \valBy{i}{x^*} - \sum_{i=1}^{t-1} z_i \geq \sum_{i=1}^t z_t - \sum_{i=1}^{t-1} z_i = z_t
    \end{align*}
\end{proof}
% We start by proving that, for $t \in [n]$, when the constants $z_1, \ldots, z_{t-1}$ represent the optimal values of \eqref{eq:basic-OP} in iterations $1, \ldots t$ respectively, the programs \eqref{eq:basic-OP} and \eqref{eq:sums-OP} are equivalent.

\eden{alternative: is this better?
\begin{proof}
    In Section \ref{sec:algo-sec-proofs}, it was proven that if Algorithm \ref{alg:basic-ordered-Outcomes} uses an $(\multApprox, \additiveApprox)$-approximate solver for \eqref{eq:compact-OP} as \textsf{OP}, then the returned solution is an $(\multApprox, \additiveApprox)$-approximation to leximin. 
    This means that, given an exact solver to \eqref{eq:compact-OP}, the algorithm will output a leximin optimal solution.
    However, we saw that \eqref{eq:sums-OP} and \eqref{eq:compact-OP} are equivalent and that the identity function is an appropriate bijection (Section \ref{sec:prob-sums-and-comp}).
    Therefore, in each iteration, a solver for \eqref{eq:sums-OP} will output the same solution and the same result will be obtained.
\end{proof}
}

\eden{in the next version we can also prove it in a maybe more interesting way.. that when the constants $z_1, \ldots, z_{t-1}$ represent the optimal values of \eqref{eq:basic-OP} the programs are equivalent}

\section{Ellipsoid Method Variant for Approximation}\label{sec:mult-variant-ellipsoid}
This Appendix describes a variant of the ellipsoid method that can be used to approximate  LPs that cannot be solved directly due to a large number of variables.
% It requires an approximate separation oracle for the dual program.
The method combines techniques presented in \cite{grotschel_geometric_1993,grotschel_ellipsoid_1981,karmarkar_efficient_1982}.

\subsection{Using Approximate Separation Oracles (multiple error)}
Our goal is to solve the following linear program (the primal):
\begin{align}
\tag{P}
\begin{split}
\min \quad &c^T \cdot x \\
s.t. \quad &A \cdot x \geq b, \quad x\geq 0;
\end{split}
\end{align}
We assume that (P) has a small number of constraints, but may have a huge number of variables, so we cannot solve (P) directly. We consider its \emph{dual}:
\begin{align}
\tag{D}
\begin{split}
\max \quad & b^T \cdot y \\
s.t. \quad &A^T \cdot y \leq c,\quad y\geq 0.
\end{split}
\end{align}
Assume that both problems have optimal solutions and denote the optimal solutions of (P) and (D) by $x^{*}$ and $y^{*}$ respectively. By the strong duality theorem:
\begin{align}
    c^T \cdot x^{*} = b^T \cdot y^{*}
\end{align}

While (D) has a small number of variables, it has a huge number of constraints, so
we cannot solve it directly either. 
In this Appendix, we show that it can be approximately using the following tool:

\begin{definition}
An \emph{approximate separation oracle} with multiplicative error (MASO) for the dual LP is an efficient function parameterized by a constant $\multError \geq 0$.
Given a vector $y$  it returns one of the following two answers:
\begin{enumerate}
\item "$y$ is infeasible". In this case, is returns a violated constraint, that is, a row $a_i^T \in A^T$ such that $a_i^T  y > c_i$.
\item "$y$ is \emph{approximately feasible}". 
That means that $A^T y \leq (1+\multError) \cdot c$
\end{enumerate}

\end{definition}
Given the MASO, we apply the ellipsoid method as follows (this is just a sketch
to illustrate the way we use the MASO; it omits some technical details):
\begin{itemize}
    \item Let $E_0$ be a large ellipsoid, that contains the entire feasible region, that is, all $y \geq 0$ for which $A^T y \leq c$.

    \item For $k = 0,1,\dots, K$ (where $K$ is a fixed constant, as will be explained later):
    \begin{itemize}
        \item Let $y_k$ be the centroid of ellipsoid $E_k$.
        
        \item Run the MASO on $y_k$.
        
        \item If the MASO returns "$y_k$ is infeasible" and a violated constraint $a_i^T$, then make a \emph{feasibility cut} --- keep in $E_{k+1}$ only those $y \in E_k$ for which $a_i^T y \leq c_i$.
        
        \item If the MASO returns "$y$ is approximately feasible", then make an \emph{optimality cut} --- keep in $E_{k+1}$ only those $y \in E_k$ for which $b^T y \geq b^T y_k$.
    \end{itemize}
    
    \item From the set $y_0, y_1, \dots, y_K$, choose the point with the highest $b^T \cdot y_k$ among all the approximately-feasible points.
\end{itemize}
Since both cuts are through the center of the ellipsoid, the ellipsoid dilates by a factor of at least $\frac{1}{r}$ at each iteration, where $r > 1$ is some constant (see \cite{grotschel_ellipsoid_1981} for computation of $r$). Therefore, by choosing $K := \log_2 r \cdot L$, where $L$ is the
number of bits in the binary representation of the input, the last ellipsoid $E_K$ is so small that all points in it can be considered equal (up to the accuracy of the binary representation).


The solution $y'$ returned by the above algorithm satisfies the following two conditions:
\begin{equation} \label{mult:y-star-is-approximetly-feasible}
     A^T y' \leq (1+\multError)\cdot c
\end{equation}
\begin{equation} \label{mult:y-star-obj-geq-opt}
     b^T y' \geq b^T y^{*}
\end{equation}
Inequality \ref{mult:y-star-is-approximetly-feasible} holds since, by definition, $y'$ is approximately-feasible.

To prove \ref{mult:y-star-obj-geq-opt}, suppose by contradiction that $b^T y^{*} > b^T y'$. 
Since $y^{*}$ is feasible for (D), it is in the initial ellipsoid. 
It remains in the ellipsoid throughout the algorithm: it is removed neither by a feasibility cut (since it is
feasible), nor by an optimality cut (since its value is at least as large as all values used for optimality cuts).
Therefore, it remains in the final ellipsoid, and it is chosen as the highest-valued feasible point rather than $y'$ --- a contradiction.

Now, we construct a reduced version of (D), where there are only at most $K$ constraints --- only the constraints used to make feasibility cuts.
Denote the reduced constraints by $A_{red}^T \cdot y \leq c_{red}$, where $A_{red}^T$ is a matrix containing a subset of at most $K$ rows of of $A^T$, and $c_{red}$ is a vector containing the corresponding subset of the elements of $c$. The reduced-dual LP is:
\begin{equation}
\tag{RD}
\begin{split}
\max  \quad & b^T y \\
s.t. \quad & A_{red}^T \cdot y \leq c_{red}, \quad y\geq 0
\end{split}
\end{equation}
Notice that it has the same number of variables as the program (D). Further, if we had run this ellipsoid method variant on (RD) (instead of (D)), then the result would have been exactly the same --- $y'$.
Therefore, (\ref{mult:y-star-obj-geq-opt}) holds for the (RD) too:
\begin{equation} \label{mult:y-star-to-y-redopt}
    b^T y' \geq b^T y^{*}_{red}
\end{equation}
where $y^{*}_{red}$ is the optimal value of (RD).


As $A_{red}^T$ contains a subset of at most $K$ rows of $A^T$, the matrix $A_{red}$ contains a subset of \emph{columns} of $A$.
Therefore, the dual of (RD) has only at most $K$ variables, which are those who correspond to the remaining columns of $A$:
\begin{equation}
	\tag{RP}
    \begin{split}
     \min \quad &c_{red}^T \cdot x_{red} \\
            s.t. \quad &A_{red} \cdot x_{red} \geq b, \quad x_{red}\geq 0
    \end{split}
\end{equation}
%  reduced-primal
%\er{Note that $A_{red}$ is a matrix with the same number of rows as $A$, but only at most $K$ columns.}
Since (RP) has a polynomial number of variables  and constraints, it can be solved exactly by any LP solver (not necessarily the ellipsoid method).
Denote the optimal solution by $x^{*}_{red}$. 

Let $x'$ be a vector which describes an assignment to the variables of (P), in which all variables that exist in (RP) have the same value as in $x^{*}_{red}$, and all other variables are set to $0$.
It follows that $A \cdot x' = A_{red} \cdot x^{*}_{red}$, therefore, since $x^{*}_{red}$ is feasible to RD, also $x'$ is a feasible solution to (P).
\erel{
In second reading, I think this should be made more formal.
Let $x'$ be a solution to (P), in which all variables that exist in (RP) have the same value as in $x^{*}_{red}$, and all other variables are set to 0.
We have to prove that 
(1) $x'$ is feasible for (P);
(2) $c^T x' \leq (1+\epsilon)\cdot c^T\cdot x^{*}$.
}
\eden{better?}
Similarly, $c^T \cdot x' = c^T_{red} \cdot x^{*}_{red}$.
We shall now see that this implies that the objective obtained by $x'$ approximates the objective obtained by $x^{*}$:
\begin{align*} 
&c^T \cdot x' = c^T_{red} \cdot x^{*}_{red} \\
&=  b^T \cdot y^{*}_{red} & \text{(by strong duality for the reduced LPs)} \\
                     &\leq  b^T\cdot y' & \text{(By (\ref{mult:y-star-to-y-redopt}))}\\
                     &\leq  (A \cdot x^{*})^T y' & \text{(definition of (P))} \\
                     &=  (x^{*})^T (A^T\cdot y') & \text{(properties of transpose and associativity of multiplication)} \\
                     &\leq  (x^{*})^T ((1+\multError)\cdot c) & \text{(by \ref{mult:y-star-is-approximetly-feasible})} \\
                     & = (1+\multError) \cdot (c^T x^{*}) & \text{(properties of transpose)}
\end{align*}
So, $x'$ ($x^{*}_{red}$ with all missing variables set to $0$) is an approximate solution to the primal LP (P) --- as required.

\subsection{Using Half-Randomized Approximate Separation Oracles}
Here, we allow the oracle to also be \emph{half-randomized}, that is, when it says that a solution is infeasible, it is always correct; however, when it says that a solution is approximately feasible, it is only correct with some probability $p \in [0,1]$.

Since the ellipsoid method variant is iterative, and since the oracle calls are independent, there is a probability $p^T$ that the oracle answers correctly in each iteration, and so, the overall process performs as before. 
We shall now explain why, using a half-randomized oracle, this ellipsoid method variant \emph{always} returns a feasible solution to the primal (even if the oracle was incorrect).

First, notice that the oracle is always correct when it determines that a solution is infeasible.
In addition, the construction of RD is only depended by these set of constraints.
Therefore, by the same arguments, $x'$ would still be a feasible solution to P (but not necessarily with an approximately-optimal objective value).

This means that given a half-randomized approximate separation oracle for the dual with error $\multError$ and success probability $p$, this ellipsoid method variant can be used as a randomized approximation algorithm for the primal with the same error and success probability $p^I$ (where $I$ is an upper bound on the number of iteration of the method on the given input). 
% \section{Saturation Algorithm}\label{sec:saturation-algorithm}
The following algorithm was independently proposed by different researchers for different problems \cite{willson,airiau_portioning_2019,nace_max-min_2008}.
% --- by Willson \cite{willson} for the  problem of fair allocation of divisible items, Airiau et al. \cite{airiau_portioning_2019} for the problem of portioning with ordinal preferences, Bei at el. \cite{bei_truthful_2022} for a variant of cake cutting that they called cake sharing and Nace and Pioro \cite{nace_max-min_2008} for multi-commodity flow problem 
But it can be generalized to capture the following case:
\begin{enumerate}
    \item The feasible region $S$ is \textit{convex}: for any two solutions $x, y \in S$ and for any $\lambda \in [0,1]$, the convex combination of $x$ and $y$ in relation to $\lambda$ is also a solution:
    \begin{align*}
        \forall x, y \in S, \quad \forall \lambda \in [0,1] \colon \quad  
        \bigl(\lambda x + (1-\lambda)y\bigl)\in S
    \end{align*}

    \item The size of the feasible region $S$ is polynomial with respect to $n$.
    % \eden{To myself: to check if this is accurate: i.e., it can be described with a number of variables and constraints that is polynomial to $n$.}

    \item The objective-functions are \textit{additive}: let $x,y,z \in S$ be solutions for which $\alpha,\beta \in \mathbb{R}$ exist such that $z = \alpha x + \beta y$, then for each objective-function $f_i \in \allObjFunc$:
    \begin{align*}
        f_i(z) &= f_i(\alpha x + \beta y) =\\
        &= \alpha f_i(x) + \beta f_i(y)
    \end{align*}
    \erel{ 
    For this condition, we must say that the solutions are vectors (we did not say this so far). Otherwise there is no meaning to adding or multiplying by scalars.
    }

    \item The objective-functions are \textit{concave}: for any objective-function $f_i \in \allObjFunc$ the set $\{f(x) \mid x \in S\}$ is concave (equivalently, the set $\{-f(x) \mid x \in S\}$ is convex). 

    \item There is a black-box for finding  the \textit{next maximin} value (denote by $OP1$): given a subset of objective-functions ($\mathcal{A}\subset \allObjFunc$) for which lower bounds have been set ($\forall f_i \in \mathcal{A} \colon z_i \in \mathbb{R}$), finds the highest value that all other objective functions can achieve simultaneously:
    \begin{align*}
        \max \quad &z\\
        s.t. \quad  & x \in S\\
                    & f_i(x) = z_i   & f_i \in \mathcal{A}\\
                    & f_i(x) \geq z   & f_i \notin \mathcal{A}
    \end{align*}

    \item There is a black-box for solving a saturation test (denote by $OP2$):
    % \eden{I think we should name this process, but I'm not sure if it is the best name...}: 
    For each objective-function $f_k \in \allObjFunc$, a single-objective optimization version of the problem with lower bounds on the values of the other objectives ($\forall f_i \in \mathcal{A} \colon z_i \in \mathbb{R}$ and $z \in \mathbb{R}$):
    \begin{align*}
    \max \quad &f_i(x)\\
    s.t. \quad  & x \in S\\
                    & f_i(x) = z_i   & f_i \in \mathcal{A}\\
                    & f_i(x) \geq z   & f_i \notin \mathcal{A}
    \end{align*}
\end{enumerate}
The algorithm is described in detail (in our terms and notations) in Algorithm \ref{alg:willson-leximin}. 


\begin{algorithm}[!htbp]
\caption{Saturation Algorithm--- for finding the Leximin optimal solution}
\label{alg:willson-leximin}
% \textbf{Input}: A black-box for OP1 and a black-box for OP2\\
% \textbf{Output}: The Lexical optimal solution
\begin{algorithmic}[1] %[1] enables line numbers 
\STATE Initialize the set of \textit{saturated} objective-functions $\mathcal{A} = \{\}$ and initialize $t=0$ (a step counter).

\STATE increase $t$ ($t = t+1$).

\STATE Use the black-box for $OP1$ to solve the following  problem, where the variables are $x$ (a vector) and $v$ (a scalar): 
\begin{align*}
\max \quad &v\\
        s.t. \quad  & x \in S\\
                    & f_i(x) \geq z_i   & f_i \in \mathcal{A}\\
                    & f_i(x) \geq v   & f_i \notin \mathcal{A}
\end{align*}
Let $x_t$ and $v_t$ be the optimal solution. 
    
\FOR{$f_k \notin \mathcal{A}$}
    \STATE Use the black-box for $OP2$ to solve the following problem, where the variables are $x$ (a vector) and $v$ (a scalar):
    \begin{align*}
    \max \quad & v\\
            s.t. \quad  & x \in S\\
                        & f_i(x) \geq z_i   & f_i \in \mathcal{A}\\
                        & f_i(x) \geq v_t   & f_i \notin \mathcal{A}\\
                        & f_k(x) \geq v
    \end{align*}
    Let $x_t^k$ and $v_t^k$ be the optimal solution. 

    \STATE \textbf{if} $v_t^k = v_t$ \textbf{then} set $f_k$ as \textit{saturated}: add it to $\mathcal{A}$ ($\mathcal{A} = \mathcal{A} \cup \{f_k\}$) and set its value to $v_t$ ($z_k = v_t$).
    % \IF{$z_{max}^k = z_{max}$}
        % \STATE Set $f_k$ as saturated: add it to $\mathcal{A}$ ($\mathcal{A} = \mathcal{A} \cup \{f_k\}$) and set its value to $z_{max}$ ($z_k = z_{max}$).
    % \ENDIF
\ENDFOR
\STATE \textbf{if} $|\mathcal{A}| = n$ \textbf{then} return $x_t$ \textbf{else} Goto line 2.
% \IF{$|\mathcal{A}| = n$}
    % \STATE return $x$ \eden{To myself: the return part of all algorithms should be explain better}
% \ENDIF
\end{algorithmic}
\end{algorithm}

The algorithm keeps a set of objective-functions that are saturated ($\mathcal{A}$) and lower bounds on their values ($\forall f_i \in \mathcal{A} \colon z_i \in \mathbb{R}$). 
The set is initially empty. 
At each iteration, at least one function becomes saturated and its lower bound is set.
When all functions become saturated, the algorithm terminates.
Each iteration of the algorithm can be divided into two parts.
In the first part, the first black-box is used to find the \textit{next max-min} value, which is the maximum value that all functions outside of $\mathcal{A}$ can achieve at the same time, given that all functions within $\mathcal{A}$ achieve their lower bounds.
In the second part, \textit{a saturation test} is made.
% \eden{I think we should name this process, but I'm not sure if it is the best name...}
For every function not in $\mathcal{A}$, the second black-box is used to find the maximum value of this function when all saturated functions ($f_i \in \mathcal{A}$) achieve their lower bounds and all other functions (outside of $\mathcal{A}$) achieve at least the max-min value from the first part.
This value is used to determine if this objective function is saturated, that is, if its maximal value from the saturation test is equal to the max-min value obtained in the first fart.
If so, we add it to the set of saturated objective-functions ($\mathcal{A}$) and set its lower bound to this value.

% \section{Additive Variant}\label{sec:additive}

\begin{theorem}\label{thm:leximin-approx-alg-leximin-opt}
    Let $\epsilon \in [0,1]$ and \textsf{OP} be a procedure that outputs a $\epsilon$ \emph{additive} approximation to \eqref{eq:vsums-OP}. Then Algorithm \ref{alg:basic-ordered-Outcomes} outputs a $\epsilon$ additive-approximate Leximin solution.  
\end{theorem}

\begin{proof}
    Contrariwise, suppose that $\retSol$ is not an $\epsilon$-approximately Leximin-optimal solution.
    This means that there exists a solution $y$ that is $\epsilon$-preferred over $\retSol$.
    That is, there exists an integer $k \in [n]$ such that:
    \begin{align*}
        \forall j < k \colon &\valBy{j}{y} \geq \valBy{j}{\retSol};\\
        & \valBy{k}{y} > \valBy{k}{\retSol} + \epsilon.
    \end{align*}
    We get that for all $s \in [k-1]$:
    \begin{align*}
         &\sum_{i=1}^s \valBy{i}{y} \geq \sum_{i=1}^s \valBy{i}{\retSol}
        && \text{since } i\leq s<k \text{ and $y$'s def.}\\
        & \geq \sum_{i=1}^s z_i && \text{constraint (2) for $t=n$.}
    \end{align*}
    Therefore, $y$ is a solution to the OP that was solved when $t = k$.
    \erel{You proved that $y$ satisfies constraint (2), but what about constraint (3)?}
    \eden{I'm not sure how to explain that constraint (3) is not a \textbf{standard} constraint. It determines the objective value $z$, so although it is not always optimal, it is always valid.}
    
    In addition, either $k<n$ or $k=n$. 
    If $k<n$ then constraint (2) for $t=n$ says that:
    \begin{align}\label{equ:approx-sum-k-geq-z-1}
        \sum_{i=1}^k \valBy{i}{\retSol} \geq \sum_{i=1}^k z_i
    \end{align}
    If $k=n$ then since $z=z_n$ constraint (3) says it.
    In both cases, we know that equation \ref{equ:approx-sum-k-geq-z-1} holds.
    
    And so, we get that:
    \begin{align*}
         \sum_{i=1}^k \valBy{i}{y} &= \sum_{i=1}^{k-1} \valBy{i}{y} + \valBy{k}{y}\\
         &  \geq\sum_{i=1}^{k-1}\valBy{i}{\retSol} + \valBy{k}{y} &&  \text{since } i \leq k-1 < k \text{ and $y$'s def.}\\
        & > \sum_{i=1}^{k-1}\valBy{i}{\retSol} + \valBy{k}{\retSol} + \epsilon &&  \text{$y$'s def. for } k
        \\
        & = \sum_{i=1}^{k}\valBy{i}{\retSol} + \epsilon \\
        & \geq \sum_{i=1}^{k} z_i + \epsilon &&  \text{equation } \ref{equ:approx-sum-k-geq-z-1}
    \end{align*}
    Which simply means that:
    \begin{align}\label{equ:sum-y-geq-sum-z-plus-eps}
         \sum_{i=1}^k \valBy{i}{y} > \sum_{i=1}^{k} z_i +\epsilon
    \end{align}
    That means that the $z$ achieved by the solution $y$ in the OP that was solved when $t = k$ is strictly more than the value we achieved $z_k$ plus $\epsilon$:
    \begin{align*}
        &\sum_{i=1}^k \valBy{i}{y} - \sum_{i=1}^{k-1} z_i && \text{insulated } z\\
        &> \sum_{i=1}^{k} z_i + \epsilon - \sum_{i=1}^{k-1} z_i  && \text{equation } \ref{equ:sum-y-geq-sum-z-plus-eps} \\
        &= z_k + \epsilon
    \end{align*}
    But we know that the error in this OP is at most $\epsilon$ --- a contradiction.
\end{proof}

\end{document}