\documentclass[letterpaper, 10 pt, conference]{ieeeconf}  
%\documentclass[article]{IEEEtran}
\IEEEoverridecommandlockouts
% The preceding line is only needed to identify funding in the first footnote. If that is unneeded, please comment it out.
\usepackage{cite}
\usepackage{amsmath,amssymb,amsfonts}
\usepackage{algorithmic}
\usepackage{graphicx}
\graphicspath{{Figures/}}
\usepackage{textcomp}
\usepackage{subfigure}
\usepackage{verbatim}
\usepackage{xcolor}
\usepackage{multirow}
\usepackage{adjustbox}
\usepackage{textcomp}
\usepackage{epsfig}
\usepackage{import}
\usepackage{subfiles} 
 \usepackage{comment}
 %\usepackage[retainorgcmds]{IEEEtrantools}
\usepackage{transparent}

\makeatletter
\newcommand\makebig[2]{%
	\@xp\newcommand\@xp*\csname#1\endcsname{\bBigg@{#2}}%
	\@xp\newcommand\@xp*\csname#1l\endcsname{\@xp\mathopen\csname#1\endcsname}%
	\@xp\newcommand\@xp*\csname#1r\endcsname{\@xp\mathclose\csname#1\endcsname}%
}
\makeatother

\makebig{biggg} {3.0}
\makebig{Biggg} {3.5}
\makebig{bigggg}{4.0}
\makebig{Bigggg}{4.5}

\def\BibTeX{{\rm B\kern-.05em{\sc i\kern-.025em b}\kern-.08em
    T\kern-.1667em\lower.7ex\hbox{E}\kern-.125emX}}
\begin{document}

\renewcommand{\arraystretch}{1.5}

\title{Six-degree-of-freedom Localization Under Multiple Permanent Magnets Actuation\\
%\thanks{Identify applicable funding agency here. If none, delete this.}
}

\author{Tom\'as da Veiga, Giovanni Pittiglio, \textit{Member, IEEE}, Michael Brockdorff, James H. Chandler, \textit{Member, IEEE},\\ and Pietro Valdastri, \textit{Fellow, IEEE}% <-this % stops a space
	\thanks{Research reported in this article was supported by the Engineering and Physical Sciences Research Council (EPSRC) under grants number EP/R045291/1 and EP/V009818/1, and by the European Research Council (ERC) under the European Union’s Horizon 2020 research and innovation programme (grant agreement No 818045). Any opinions, findings and conclusions, or recommendations expressed in this article are those of the authors and do not necessarily reflect the views of the EPSRC or the ERC.}
	\thanks{~Tom\'as~da~Veiga,
		Michael~Brockdorff,
		~James~H.~Chandler,
		~and~Pietro~Valdastri are with the STORM Lab, Institute of Autonomous Systems and Sensing (IRASS), School of Electronic and Electrical Engineering, University of Leeds, Leeds, UK.
		Email: \{\tt{eltgdv, elmbr, j.h.chandler, p.valdastri}\}@leeds.ac.uk}% <-this % stops a space
	\thanks{Giovanni Pittiglio is with the Department of Cardiovascular Surgery, Boston Children’s Hospital, Harvard Medical School, Boston, MA 02115, USA. Email:
	{\tt giovanni.pittiglio@childrens.harvard.edu}}
}

\maketitle

\begin{abstract}
Localization of magnetically actuated medical robots is essential for accurate actuation, closed loop control and delivery of functionality. Despite extensive progress in the use of magnetic field and inertial measurements for pose estimation, these have been either under single external permanent magnet actuation or coil systems. With the advent of new magnetic actuation systems comprised of multiple external permanent magnets for increased control and manipulability, new localization techniques are necessary to account for and leverage the additional magnetic field sources. In this letter, we introduce a novel magnetic localization technique in the Special Euclidean Group SE(3) for multiple external permanent magnetic field actuation and control systems. The method relies on a milli-meter scale three-dimensional accelerometer and a three-dimensional magnetic field sensor and is able to estimate the full 6 degree-of-freedom pose without any prior pose information. We demonstrated the localization system with two external permanent magnets and achieved localization errors of 8.5 $\pm$ 2.4~mm in position norm and 3.7 $\pm$ 3.6$^{\circ}$ in orientation, across a cubic workspace with \textbf{20~cm} length.
\end{abstract}

\begin{keywords}
	Medical Robots and Systems, Localization, Magnetic Actuation, State Estimation, Kalman Filter
\end{keywords}

\section{Introduction} \label{sec:intro}
\subfile{subfiles/introduction_v2}

\section{Localization Strategy} \label{sec:loc-strategy}
\subfile{subfiles/loc_strategy}

\section{Simulation} \label{sec:simulation}
\subfile{subfiles/simulation}

\vspace{-0.3cm}

\section{Experimental Setup} \label{sec:exp-setup}
\subfile{subfiles/setup}

\section{Results} \label{sec:results}
\subfile{subfiles/results}

\section{Conclusions} \label{sec:conclusions}
\subfile{subfiles/conclusions}



%\section{Contents}
%\subfile{subfiles/contents}

%\section*{Acknowledgement}

%Research reported in this article was supported by the Royal Society, by the Engineering and Physical Sciences Research Council (EPSRC) under grant numbers EP/R045291/1 and EP/V009818/1, and by the European Research Council (ERC) under the European Union’s Horizon 2020 research and innovation programme (grant agreement No 818045). Any opinions, findings and conclusions, or recommendations expressed in this article are those of the authors and do not necessarily reflect the views of the Royal Society, EPSRC, or the ERC.

\bibliographystyle{IEEEtr}
\bibliography{references}
%\bibliographystyle{IEEEtran}

\end{document}