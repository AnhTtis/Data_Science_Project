\documentclass[../main.tex]{subfiles}
\begin{document}

To evaluate the proposed localization system performance, a sensing platform was developed and tested with a 2-EPM system. 

The sensor board was composed by a 3D IMU (LSM6DS3, STMicroelectronics, Switzerland. Accelerometer sensing range $\pm2$g, Sensitivity $0.061$mg/LSB$_{16}$, Footprint $2.5\times3\times0.83$~mm) and a 3D HE (MLX90395, Melexis, Belgium. Sensing range $\pm 50$~mT; Sensitivity $2.5~\mu$T/LSB$_{16}$, Footprint $3\times3\times0.9$~mm). \textcolor{black}{The sensors used were chosen due to their dimensions, sensitivity and sensing range, allowing their use in embedded devices of the millimeter scale under high magnetic fields.} The sensors were interfaced with a Raspberry Pi 4B through I$^2$C protocol. The HE sensor was calibrated by placing it in the center of a 1D Helmholtz coil (DXHC10-200, Dexing Magnet Tech. Co., Ltd, Xiamen, China) under known magnetic field vectors.

The dual EPM platform (dEPM) was used~\cite{pittiglio2022collaborative,pittiglio2022patient}, consisting on two KUKA LBR iiwa14 robots (KUKA, Germany), each manipulating one EPM (cylindrical permanent magnet with diameter and lenght of 101.6~mm and axial magnetization of 970.1~Am$^2$ (Grade N52)) (see Fig.~\ref{fig:setup}).

To fully assess the localization performance across the dEPM large workspace, a 3D printed plate (20-by-20~cm) was placed in between the two robots, delimiting the localization workspace in two dimensions. The sensor board was attached to 3D printed holders of various heights and orientations, which were in turn attached to the plate, allowing full variation of position and orientation.

Additionally, ground truth data was collected via a 4-camera optical tracking system (OptiTrack, Prime 13, NaturalPoint, Inc., USA, with submilimeter accuracy). With optical markers attached to the end-effectors of both robots, to the workspace plate and to the sensor board, the relative pose of each robotic arm base and the sensor board with respect to \{$\mathcal{W}$\} was found before each experiment (see Fig.~\ref{fig:setup}). While the EPMs were in motion, their poses were determined by reading the robotic arms joints and computing the inverse kinematics. This ensures a more accurate tracking of the motion of the EPMs since the markers may be blocked from the field of view during the motion. 

Finally, the Raspberry Pi, the robotic arms, and the optical tracking system were all connected using ROS. Data from the robotic arms encoders and sensors was collected at a rate of 50Hz. Given the inclusion of 20 EPM configurations in the measurement model, the EKF was ran at $50/20 = 2.5$~Hz. The EKF parameters used are shown in Table~\ref{tb:EKF_param}. These were determined by the simulation step in Section~\ref{sec:sim_obser} and the sensors used. \textcolor{black}{Additionally, the state was initialized at the origin of the workspace, $T_0 = I$.}
\begin{table}[h!]
	\caption{EKF Covariance Matrices} \label{tb:EKF_param}
	\centering
	\begin{tabular}{c|c|}
		\cline{2-2}
		\textbf{} & \textbf{EKF} \\ \hline
		\multicolumn{1}{|l|}{\textbf{State}} & $P_0 = \text{diag}(10^{-4}, 10^{-4})$ \\ \hline
		\multicolumn{1}{|l|}{\textbf{Input}} & $Q_n = \text{diag}(10^{-5}, 10^{-3})$ \\ \hline
		\multicolumn{1}{|l|}{\textbf{Measurement $\|\textbf{B}\|$}} & $R_{n_{\|B\|}} = 10^{-8}$ \\ \hline
		\multicolumn{1}{|l|}{\textbf{Measurement $\textbf{B}$}} & $R_{n_B} = \text{diag}(10^{-8},10^{-8},20^{-8})$ \\ \hline
		\multicolumn{1}{|l|}{\textbf{Measurement $\textbf{G}$}} & $R_{n_G} = 10^{-6}I$ \\ \hline
	\end{tabular}
\end{table}

\begin{figure}[!ht]
	\centering
	 \resizebox{.8\linewidth}{!}{\input{./Figures/setup-inkscape_bitmap.pdf_tex}}
	\caption{Experimental setup, comprised of two robotic arms with EPM at the end-effectors, Optical Tracking system, and sensor board.}
	\label{fig:setup}
\end{figure}

\end{document}
