\documentclass[../main.tex]{subfiles}
\begin{document}
\vspace{-0.15cm}
To infer the stability and performance of the observer, first, an observability analysis on the system was done to assess the \textcolor{black}{minimum number of magnetic field measurements for observability. Second, the impact the number of EPMs $m$ and the maximum number of magnetic field measurements $n$ in the measurement model (see eq.~(\ref{eq:mag_model}) and~(\ref{eq:meas_model})) have on the stability of the observer was analyzed. Lastly, the observer was run within a simulated environment to infer the EKF's performance and expected convergence time. EKF covariance matrices $P_0$, $Q_n$ and $R_n$ were tuned in this step.}

The number of EPMs was varied between one and six. \textcolor{black}{Given that EPMs are used for actuation, localization should not rely on a specific EPM motion. Therefore, random motion paths were generated for each EPM.} Additionally, each EPM was constrained to a plane 15~cm away from the workspace edge, as seen in Fig.~\ref{fig:EPMs_paths}.
\begin{figure}[!ht]
	\centering
	\includegraphics[scale=0.2]{./Figures/EPMs_edited_2.png}
	\caption{Planes covered by the generated EPM paths. Each EPM is constrained to a plane 15~cm from the workspace edge.}
	\label{fig:EPMs_paths}
\end{figure}
\vspace{-0.4cm}
\subsection{Observability Analysis} \label{sec:observability}
\textcolor{black}{To assess the minimum number of magnetic field measurements $n$ needed for observability}, an observability analysis was performed for system in eq.~(\ref{eq:SE3-system}) with measurement model in eq.~(\ref{eq:meas_model}). Local weak observability of a non-linear system is defined by the following codistribution being full rank, i.e $\text{rank}(\nabla_T \mathcal{O}) = 6$.
\begin{equation}
	\nabla_T \mathcal{O} = \text{span}(\{\nabla_T \mathcal{L}_{\dot{T}}^i\textbf{ h}, i \in \mathcal{N}^+ \cup 0\})
\end{equation}
where $\mathcal{L}_{\dot{T}}^i \textbf{h}$ defines the $i$th-order Lie derivative of $\textbf{h}$ with respect to the state $T$. \textcolor{black}{Further details on the notation and derivation of an observability analysis can be found in~\cite{pittiglio2020observability}}. In this work, we consider the first order derivative only, and so, this codistribution can be expanded as
\begin{equation}
\nabla_T \mathcal{O} = \left[  \nabla_p \mathcal{O} \quad \nabla_R \mathcal{O} \right ] =  \begin{bmatrix}
\nabla_p \mathcal{O}_{\|\textbf{B}_1\|} & \nabla_R \mathcal{O}_{\|\textbf{B}_1\|} \\
\vdots & \vdots\\
\nabla_p \mathcal{O}_{\|\textbf{B}_n\|} & \nabla_R \mathcal{O}_{\|\textbf{B}_n\|} \\
\nabla_p \mathcal{O}_{\textbf{B}_1} & \nabla_R \mathcal{O}_{\textbf{B}_1} \\
\vdots & \vdots\\
\nabla_p \mathcal{O}_{\textbf{B}_n} & \nabla_R \mathcal{O}_{\textbf{B}_n} \\
\nabla_p \mathcal{O}_{\textbf{G}} & \nabla_R \mathcal{O}_{\textbf{G}}\\
\end{bmatrix}
\end{equation}
making explicit the two components of the state, position and orientation, and the different types of measurement.

As shown in~\cite{pittiglio2020observability}, $\nabla_R \mathcal{O}$ represents the Lie derivative with respect to the orientation. Since the norm of the magnetic field has no orientation information, $\nabla_R \mathcal{O}_{\|\textbf{B}_i\|}$ is equal to zero. 
\begin{equation}
\nabla_R \mathcal{O}_{\|\textbf{B}_i\|} = 0_{1 \times 3}
\end{equation}
\begin{equation}
\nabla_R \mathcal{O}_{\textbf{B}_i} = \begin{bmatrix}
0 & -R_{:,3} \cdot \textbf{B}_i & R_{:,2} \cdot \textbf{B}_i \\
R_{:,3} \cdot \textbf{B}_i & 0 & -R_{:,1} \cdot \textbf{B}_i \\
- R_{:,2} \cdot \textbf{B}_i & R_{:.1} \cdot \textbf{B}_i & 0 \\
\end{bmatrix}
\end{equation}
\begin{equation}
\nabla_R \mathcal{O}_{G} = \begin{bmatrix}
0 & R_{33} & -R_{32} \\
-R_{33} & 0 & R_{31} \\
R_{32} & -R_{31} & 0
\end{bmatrix}
\end{equation}

$\nabla_p \mathcal{O}$ represents the Lie derivative with respect to the position, and can be expressed as follows. Given that IMU measurements only contain information regarding orientation, $\nabla_p \mathcal{O}_{G}$ is equal to zero. 
\begin{equation}
\nabla_p \mathcal{O}_{\|\textbf{B}_i\|} = \begin{bmatrix}
\frac{\partial \|\textbf{B}_i\|}{\partial x} & \frac{\partial \|\textbf{B}_i\|}{\partial y} & \frac{\partial \|\textbf{B}_i\|}{\partial z}
\end{bmatrix}
\end{equation}
\begin{equation}
\nabla_p \mathcal{O}_{\textbf{B}_n} = \begin{bmatrix}
\frac{\partial \textbf{B}_{i_x}}{\partial x} & \frac{\partial \textbf{B}_{i_x}}{\partial y} & \frac{\partial \textbf{B}_{i_x}}{\partial z} \\
\frac{\partial \textbf{B}_{i_y}}{\partial x} & \frac{\partial \textbf{\textbf{B}}_{i_y}}{\partial y} & \frac{\partial \textbf{\textbf{B}}_{i_y}}{\partial z} \\
\frac{\partial \textbf{B}_{i_z}}{\partial x} & \frac{\partial \textbf{B}_{i_z}}{\partial y} & \frac{\partial \textbf{B}_{i_z}}{\partial z} \\
\end{bmatrix}
\end{equation}
\begin{equation}
\nabla_p \mathcal{O}_{G} = 0_{3 \times 3}
\end{equation}

Looking at the full observability matrix  $\nabla_T \mathcal{O}$, we see that for when $n=1$, $\text{rank}(\nabla_T \mathcal{O}) = 5$ making the system not observable.  In fact, a single configuration of the EPMs and its respective magnetic field $\textbf{B}_i$ together with its norm and $\textbf{G}$ are not enough to solve the full 6-DOF pose. This can intuitively be inferred as the gravity vector measurement is able to provide 2-modes of the orientation, with the rotation around its own axis, i.e. rotation around gravity, missing~\cite{pittiglio2020observability}. Since the magnetic field vector and its norm are not linearly independent, only three of the remaining 4 modes of the pose dynamics can be solved for. Therefore, without any prior pose information, a minimum of 2 measurements of magnetic field are necessary in order to make the system observable and estimate the full 6-DOF pose. Additional measurements of the magnetic field can be taken for different EPM configurations.

\subsection{\textcolor{black}{Magnetic Analysis}}
\textcolor{black}{Having shown that a minimum of two magnetic field measurements for different EPM configurations are needed for observability, the effect this number ($ 2  \leqslant n  \leqslant 100$) has on the stability of the observer is further inferred. Additionally, the effect the number of EPMs ($ 1  \leqslant m  \leqslant 6$) in the workspace has on the stability was also analyzed. This was done by taking the condition number $N_c$ across multiple planes of the workspace for the different cases.} The condition number is defined as the ratio between the maximum and minimum singular values of $\nabla_T \mathcal{O}$, and, as such, lower values indicate a better conditioned system. 
\begin{figure}[!h]
	\centering
	\includegraphics[scale=0.8]{./Figures/cond_num.png}
	\caption{System's condition number $N_c$ for different numbers of EPMs $m$ and different number of EPM configurations $n$ in the model. (a) Shows the condition number $N_c$ across the XZ plane of the workspace for one, two, four and six EPMs, when $n = 100$. (b) Plot showing how the condition number $N_c$ changes with higher number of EPM configurations in the measurement model for each number of EPMs.}
	\label{fig:cond_num}
\end{figure}
Fig.~\ref{fig:cond_num}(a) shows $N_c$ across the XZ plane ($y = 0$) for $n=100$ and for one, two, four, and six EPMs in the workspace, respectively. Fig.~\ref{fig:cond_num}(b) plots how $N_c$ changes when multiple EPM configurations $n$ are added to the measurement model, for each number of EPMs. $N_c$ was computed at three planes of the workspace XY ($z=0$), XZ ($y=0$, represented in (a)), and YZ ($x=0$). As we can see, there is a significant difference between a single EPM $m=1$ and multiple EPMs $ m \geqslant 2$, with $ m \geqslant 2$ having significantly lower  $N_c$ for any number of EPM configurations $n$. This is due to the fact that when multiple EPMs are present in the workspace, the resulting magnetic field becomes considerably less trivial, reducing the number of possible solutions for a specific measured magnetic field. However, there is no significant difference for when $m$ increases beyond two. Additionally, $N_c$ lowers as more EPM configurations $n$ are added to the measurement model. However, a plateau is reached at around $n = 20$, as more EPM configurations do not lower $N_c$. 

\subsection{Simulated Observer} \label{sec:sim_obser}
To further predict the performance of the EKF, the observer was ran with the MAMR fixed at 100 different randomly generated poses across the workspace. Convergence was deemed achieved once position error was below 5~mm in all axis, and the trace of the orientation error under 0.1, both for over 150 consecutive time-steps. Since the number of EPM configurations $n$ in the measurement model affects the EKF's frequency due to robot movement and data acquisition time, rather than assessing speed through EKF iterations $k$, speed was assessed by the total number of different EPM configurations needed until convergence was reached, $ n \cdot k$. The results were averaged across all 100 tested MAMR poses for each number of EPMs and EPM configurations.

Fig.~\ref{fig:sim_results} plots the results obtained. As expected from the previous condition number analysis, there is a clear distinction between a single EPM and multiple EPMs. Multiple EPMs lead to a much faster convergence needing a significantly lower total number of EPM configurations. However, the difference between two and six EPMs is marginal. Additionally, the higher the number of EPM configurations $n$ in the measurement model the faster the convergence for a single EPM, as the associated $N_c$ gets lower. However, with multiple EPMs this effect is not as noticeable, with $n$ around $20$ leading to a faster convergence. 
\begin{figure}[!ht]
	\centering
	\includegraphics[scale = 0.8]{./Figures/num_meas_vs_epms.png}
	\caption{Effect that multiple EPMs and the number of EPM configurations in the measurement model $n$ have on convergence speed $n \cdot k$. Convergence was achieved once errors in position were below 5~mm across all axis, and the trace of the orientation error $e_R$ below 0.1, for over 150 consecutive time-steps.}
	\label{fig:sim_results}
\end{figure}

Given these results, we consider from this point forward the case for which $m = 2$ and $n= 20$, i.e. there are two EPMs in the workspace, and the measurement model is comprised of 20 different EPM configurations. To further assess the localization performance for these conditions, a simulation was ran for 10,000 different random MAMR poses across the workspace. Fig.~\ref{fig:sim_results_10k}(a),(c) shows the error in position e$_p$ and orientation e$_R$ over time for all tested poses. As we can see, the observer converged for all tested poses, with 95.0\% of tested poses with norm position errors below 1~mm at finish. Additionally, as the histograms show, convergence in orientation is achieved faster than position, with 100\% of the poses having converged fully in orientation before 1000 iterations (see Fig.~\ref{fig:sim_results_10k}(d)).
\begin{figure}[!ht]
	\centering
	\includegraphics[scale=0.8]{./Figures/sim_results.png}
	\caption{Simulation errors for 10,000 random poses across the workspace over EKF iterations, with 2 EPMs and 20 EPM configurations in the measurement model. (a) Norm of the position error, (b) Histogram showing the distribution of convergence in position, (c) Error in orientation, (d) Histogram showing the distribution of convergence in orientation.}
	\label{fig:sim_results_10k}
\end{figure}
	
\end{document}
