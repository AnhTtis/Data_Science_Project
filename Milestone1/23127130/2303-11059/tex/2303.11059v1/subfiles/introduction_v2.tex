\documentclass[../main.tex]{subfiles}
\begin{document}

Magnetically actuated medical robots (MAMR) have seen significant focus and development in recent decades due to their potential for miniaturization~\cite{hu2018small}, tether-less actuation~\cite{popek2016six} and high number of controllable degrees-of-freedom (DOFs)~\cite{salmanipour2018eight,pittiglio2022collaborative}. In fact, magnetically guided catheters have been used to treat cardiac arrhythmias since 2003~\cite{nelson2022magnetically,carpi2009stereotaxis}.

A key aspect in their actuation is pose estimation~\cite{bianchi2019localization,barducci2019adaptive}, enabling closed loop control and delivery of functionality~\cite{norton2019intelligent}. Imaging techniques have long been used for this purpose but are generally tied to limited resolution, harmful radiation exposure and need for additional hospital equipment~\cite{aziz2020medical,pane2022ultrasound,daguerre2022localization}. As such, methods based on magnetic field measurements have received significant attention, with magnetic tracking systems being widely available on the market. These, however, are not compatible with magnetic actuation systems due to distortions on the localization magnetic fields.

To address this issue, significant research on magnetic localization coupled with magnetic actuation systems has been done~\cite{khalil2019magnetic,popek2016six,son2018simultaneous,shao2019novel,taddese2018enhanced}. Several works have been based on magnetic field sensing arrays external to MAMR~\cite{micheal20222d,son2018simultaneous}. While advantageous from a miniaturization and internal power consumption point of view, these systems require calibration of large sensor arrays and have limited localization workspace dimensions. Internal sensing to the MAMR, on the other hand, does not suffer from workspace dimension restrictions. It requires, however, on-board power and heterogeneous localization magnetic fields, with 6-DOF localization having been shown for systems with a single external permanent magnet (EPM). \textcolor{black}{Internal sensing methods have been shown for endoscopic capsules, as well as magnetically guided catheters}~\cite{popek2016six,sperry2022six,taddese2018enhanced,fischer2022using}.

Over recent years the need for enhanced control and manipulability of MAMRs has led to the advent of actuation platforms based on multiple magnetic field sources (MMFS) such as multiple electromagnetic coils and multiple permanent magnets~\cite{kummer2010octomag,hoang2019independent,hong2020magnetic,pittiglio2022collaborative,ryan2017magnetic,stereotaxis_patent}. Some of these platforms have been cleared for human use such as Stereotaxis Genesis RMN\textsuperscript{\tiny\textregistered} based on two permanent magnets, and Magnetecs and Aeon Scientific based on multiple electromagnetic coils. 

Despite this progress, magnetic localization for such systems is lagging behind, with fluoroscopic imaging being currently used~\cite{nelson2022magnetically}. Unlike single magnetic field source systems where the singularity regions and localization limitations have been thoroughly investigated and solved for~\cite{taddese2018enhanced}, magnetic localization for MMFS systems suffers from additional challenges due to the superposition of the magnetic fields leading to configuration-specific singularity regions. Only recently, a 3D position localization system with internal magnetic field sensing was demonstrated for a multi-coil system, \textcolor{black}{for a 3~mm catheter}~\cite{fischer2022using}.

Furthermore, a common conundrum in 6-DOF magnetic localization with internal sensing is finding the rotation around gravity, due to the absence of the Earth's magnetic field measurement~\cite{mahony2008nonlinear}. This has been solved in the past by accurately initializing this missing rotation angle and tracking it with a gyroscope~\cite{pittiglio2020observability, di2016jacobian}. However, this is prone to errors over time, especially for slow moving systems where gyroscope data is not as sensitive. Additionally, if communication to the MAMR is lost, a new accurate initialization is needed, proving impossible mid medical intervention. More recently, Taddese et al.~\cite{taddese2018enhanced} fitted an auxiliary coil around a single EPM providing a second set of magnetic field measurements. This solves the missing rotation angle and is also able to eliminate the localization singularity plane when it comes to localization with respect to a single EPM. However, when MMFS are present in the workspace, that singularity plane ceases to exist due to the superposition of magnetic fields, and instead singularity regions are present depending on the relative pose of each EPM.

This paper introduces, for the first time, a 6-DOF magnetic localization method for systems with multiple EPMs without the need for any prior pose information. The method relies on measurements from an accelerometer and a single 3D magnetic field Hall effect sensor (HE), both internal to the MAMR. We analyze the effect that the number of EPMs in the workspace has on the full pose estimation; and demonstrate its performance in a two EPM magnetic actuation platform. Since adding an orthogonal coil is not able to solve for the singularity regions, in this work we do not consider it and instead solve for the missing rotation angle by using multiple magnetic field measurements at different EPM configurations. This works for static or quasi-static systems, with maximum MAMR velocity highly dependent on the actuation system and the magnetic field generated. This is the case for non-actuated parts of a larger system, such as the deployment point at the tip of an endoscope, or for MAMRs while the generated magnetic fields are sufficiently weak to induce actuation. Additionally, unlike common works in literature which parameterize the rotation matrix, in this work the full 6-DOF pose is estimated directly in the special euclidean group $SE(3)$. This avoids any singularities or non-unique representations of the orientation when using Euler angles or quaternions~\cite{mathavaraj2021se,mayhew2011quaternion,taddese2018enhanced}.

\end{document}
