\documentclass[../main.tex]{subfiles}
\begin{document}

In this letter, a 6-DOF localization strategy without any prior pose information for actuation systems under multiple EPMs was presented. \textcolor{black}{The method relies on the measurements from a 3D accelerometer and a 3D HE sensor. These sensors are low-cost and widely available. Additionally, their small footprint makes them easily embedded in small-scale medical robots. In fact, magnetic localization based on these sensors has long been in use in medical robots, ranging from catheters~\cite{fischer2022using} to endoscopic capsules~\cite{taddese2018enhanced}. However, as new platforms based on multiple EPMs emerge for the control and actuation of magnetically actuated continuum robots for endoluminal procedures, localization techniques that take into account multiple magnetic field sources are needed. The internal placement of the sensors to the MAMR should be carefully designed to better offset any internal magnetic field measurements from the sensor. This will ensure accurate external magnetic field readings}.

\textcolor{black}{Unlike previous work that shows localization with respect to a single EPM, in this work we developed a localization technique under multiple EPM control}. We showed that, when compared to a single EPM, multiple EPMs lead to faster convergence speeds. The method was tested across a $8000~\text{cm}^3$ workspace, with average errors of 8.5 $\pm$ 2.4 mm in position norm and $0.032 \pm 0.027$ in orientation trace error. \textcolor{black}{This localization technique can thus be applied to endoscopic capsules, or magnetically guided catheters, which are under MMFS control or in close proximity to additional magnetic field sources.}

\textcolor{black}{In this work, the EPMs were moved randomly around the workspace, as their movement should be mainly optimized for actuation}. However, this was shown to lead to localization singularity regions and varying results when it comes to convergence speed and error. \textcolor{black}{Optimizing the EPM paths for both actuation and localization for active sensing will allow for reliable simultaneous localization and actuation under multiple EPM control. This could be achieved by analysing each specific EPM configuration required for actuation, and finding an alternative whenever such a configuration leads to non-observability. Additionally, this would allow a reduction in the number of measurements needed per time step, increasing the state estimation update rate and convergence speed.}

Lastly, \textcolor{black}{the speed at which the EPMs are moving is crucial for the convergence speed of the observer, as well as, the MAMR's speed}. Due to multiple instances of the EPM configurations present in the measurement model, a significant change in magnetic field should be captured across different EPM configurations. With the robotic arms moving at 30\%\ of their full speed and 20 EPM configurations per iteration, the observer was running at 2.5~Hz allowing MAMR's speeds of up to 0.2~mm/s. \textcolor{black}{The robotic arms speed was constrained for safety reasons due to the random motion travelled. It is expected that in a realistic operative scenario, the robotic arms would be travelling well-defined trajectories allowing for faster safe speeds. This would allow faster update rates and MAMR's speeds.}
\end{document}
