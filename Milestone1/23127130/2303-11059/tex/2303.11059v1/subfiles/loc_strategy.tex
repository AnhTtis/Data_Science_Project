\documentclass[../main.tex]{subfiles}
\begin{document}

\subsection{Problem Formulation}
We consider finding the pose of a MAMR, with frame \{$\mathcal{A}$\} within our workspace  \{$\mathcal{W}$\} (see Fig.~\ref{fig:loc-alg}). Its position is denoted as $ \textbf{p} \in \mathbb{R}^3$ in \{$\mathcal{W}$\} and attitude as rotation matrix $ R \in SO(3)$ of the MAMR frame \{$\mathcal{A}$\} relative to \{$\mathcal{W}$\}. Additionally, the MAMR's linear velocity is denoted by $\textbf{V} \in \mathbb{R}^3$ expressed in \{$\mathcal{W}$\}. The MAMR's angular velocity expressed in \{$\mathcal{W}$\} relative to \{$\mathcal{A}$\} is represented by $\boldsymbol{\Omega} \in \mathbb{R}^3$.

We describe our state in the special euclidean group $SE(3)$, i.e. the group of homogenous transformations with entries in $\mathbb{R}^3$ associated with the Lie algebra, $\mathfrak{se}(3)$ of dimension 6. The main goal is to estimate the homogenous transformation matrix from the MAMR reference frame \{$\mathcal{A}$\} to the global frame \{$\mathcal{W}$\} (see Fig.~\ref{fig:loc-alg}).
\begin{equation*}
	T = ^\mathcal{W}T_\mathcal{A} : \{A\} \rightarrow \{W\}
\end{equation*}

Therefore, the dynamics model can be represented as
\begin{equation} \label{eq:SE3-system}
\dot{T} = T \begin{bmatrix}
(\boldsymbol{\Omega} + \textbf{b} + \boldsymbol{\delta})_\times & \textbf{V} \\
0 & 1
\end{bmatrix}
\end{equation}
with $(\boldsymbol{\Omega} + \textbf{b} + \boldsymbol{\delta})$ the measured angular velocity from the gyroscope including its bias $\textbf{b}$ and noise $\boldsymbol{\delta}$. Additionally, $(\cdot)_\times$ denotes the skew-symmetric matrix associated with the cross product by itself. 
\begin{figure}[h!]
	\centering
	\def\svgwidth{\columnwidth}
	\resizebox{.8\linewidth}{!}{\input{EPMs_diagram.pdf_tex}}
	\caption{Representation of the world reference frame \{$\mathcal{W}$\} and MAMR reference frame \{$\mathcal{A}$\}, together with gravity vector $G$ in green, and magnetic field measurements $B_i$ in orange for \textit{m} EPMs. In purple is the state to estimate.}
	\label{fig:loc-alg}
\end{figure}

\subsection{Measurement Model}
We consider our MAMR to be under $m$ EPMs actuation, and to be fitted with an accelerometer and a 3D HE sensor, providing two types of measurements: acceleration, and magnetic field vector.

Considering that gravitational acceleration ($\textbf{g}$) dominates over linear accelerations as per common approach in literature~\cite{mahony2008nonlinear,pittiglio2020observability}, the accelerometer measurement can be represented as (see Fig.~\ref{fig:loc-alg} in green)
\begin{equation}
	\textbf{G} = R^T\textbf{g}
\end{equation}
\textcolor{black}{where $R^T$ denotes the transpose of the MAMR's rotation matrix.}

The magnetic field vector generated by an EPM$_j$ (with $j = 1, ..., m$) can be modeled as a dipole
\begin{equation} \label{eq:dipole_model}
\textbf{B}_{j} :=  \textbf{B}(\mu_{j},\textbf{r}_{j}) = \frac{\mu_0 |\mu_{j}|}{4 \pi |\textbf{r}_{j}|^3} \left(3 \hat{\textbf{r}}_{j} \hat{\textbf{r}}_{j}^T - I \right)\hat \mu_{j}
\end{equation}
with $\textbf{r}_{j}$ the distance between \{$\mathcal{A}$\} and EPM$_j$, and $\mu_{j}$ the EPM's magnetic moment in \{$\mathcal{A}$\}. This assumption is valid for far-enough distances from the EPMs and is commonly employed in other magnetic localization works~\cite{petruska2012optimal,taddese2018enhanced}. Assuming that there are no metal objects in the workspace, the measured magnetic field $\textbf{B}$ equals the sum of the magnetic fields generated by each EPM.
\begin{equation} \label{eq:mag_model}
	\textbf{B} = \sum_{j = 1}^{m} \textbf{B}_{j}
\end{equation}

Given the absence of the Earth's magnetic field measurement, a minimum of two magnetic field measurements for different configurations of the $m$ EPMs are needed for observability (see Fig.~\ref{fig:loc-alg} in orange, and Section~\ref{sec:observability}). This is a valid assumption for systems where the magnetic field changes much quicker than the MAMR's pose, such as static or quasi-static systems. This being so, assuming null mean Gaussian measurement noises~\cite{mahony2008nonlinear}, the measurement model can be expressed as follows. \textcolor{black}{In addition to $n$ measurements of the magnetic field, their norm $\|\textbf{B}_i\|$ was also included. Unlike the full magnetic field measurement vector, which contains information on both position and orientation, the magnetic field norm is dependent only on the MAMR's position. When multiple measurements are present, the addition of the magnetic field norm increased convergence speed.}
\begin{equation} \label{eq:meas_model}
	\textbf{h} = \begin{bmatrix}
	\|\textbf{B}_1\| \\
	\vdots \\
	\| \textbf{B}_i \| \\
	\textbf{B}_1 \\
	\vdots \\
	\textbf{B}_i \\
	\textbf{G}
	\end{bmatrix}, \qquad i = 2, \ldots, n
\end{equation}

\subsection{Extended Kalman Filter}
Extended Kalman Filters (EKF) in $SO(3)$ and $SE(3)$ have been widely used and proved effective~\cite{pittiglio2020observability,mathavaraj2021se}. \textcolor{black}{For the sake of summary, only the EKF equations are explicitly described here. Further detail on the formulation of EKF can be found in~\cite{pittiglio2020observability} and~\cite{barfoot2017state}}.

The discrete dynamics of the estimated state can be described as 
\begin{align} \label{eq:dynamics}
	\hat{T}_{k+1} = \hat{T}_k \text{exp}(\text{K}_k\tilde{\textbf{y}}_kt) \\
	\tilde{\textbf{y}}_k = \textbf{y}_k - \textcolor{black}{\textbf{h}}(\hat{T}_k)
\end{align}
with time-step $k = 0, t, 2t, ...$, $K_k$ the gain defined by the standard EKF prediction and update steps below, $\text{exp}(\cdot)$ the exponential map of $SE(3)$, \textcolor{black}{$\textbf{h}$ the measurement model defined in eq.(\ref{eq:meas_model}), and $\textbf{y}_k$ the sensors' outputs in the measurement model format, i.e. the norm of the magnetic field, followed by the magnetic field and gravity.}

\subsubsection{Prediction} \textcolor{black}{This step sees the propagation of the state covariance matrix $P_k \in \mathbb{R}^{6 \times 6}$ as}
\begin{equation*}
P_k = F_k\overline{P}_{k-1}F_k^T + G_kQ_nG_k^T
\end{equation*}	
\textcolor{black}{with $P_k = \text{diag}(P_{k_{p}}, P_{k_{R}})$, where $P_{k_{p}}$ and $P_{k_{R}}$ denote the state covariance matrix of the position and orientation respectively. Additionally, input noise is considered as a null-mean Gaussian distribution with constant covariance $Q_n \in \mathbb{R}^{6\times6}$. Lastly, $F_k = \text{exp}(A_k t)$ and $G_k = T_k \frac{\partial}{\partial A_k}\text{exp}(A_k t)$, with $A_k$ defined by the Lie algebra as matrix $A_k = [\Omega_\times \quad V; 0 \quad 0]$}.

\subsubsection{Update} The second step sees the computation of the gain $K_k$ used in the update of the state as shown in eq.~(\ref{eq:dynamics}) through
\begin{align*}
	S_k & = H_kP_kH_k^T + R_n \\
	K_k & = P_kH_k^TS_k^{-1} \\
	\overline{P}_k & = P_k - K_kS_kK_k^T \\
\end{align*}
\vspace{0.5cm}
where $H_k = \frac{\partial \textcolor{black}{\textbf{h}_k}}{\partial T_k} $. Additionally, measurement noise is considered as a null-mean Gaussian distribution with constant covariance matrix $R_n \in \mathbb{R}^{m\times m}$ - $\textcolor{black}{\textbf{h}} \in \mathbb{R}^m$.

\subsection{Error metrics: } The observer's performance was assessed through two different error metrics: one for the estimation of the MAMR's position and one for the MAMR's attitude.

\begin{IEEEeqnarray*}{rCl}
	\text{e}_p = \|\textbf{p} - \hat{\textbf{p}}\|\\
	\text{e}_R = \text{tr}(I - \hat{R}^TR) \IEEEyesnumber
\end{IEEEeqnarray*}

\end{document}
