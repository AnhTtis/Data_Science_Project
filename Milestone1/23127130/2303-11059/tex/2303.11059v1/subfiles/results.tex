\documentclass[../main.tex]{subfiles}
\begin{document}
The localization algorithm was tested for eight different poses across the workspace (see Fig.~\ref{fig:tested_poses}). Each pose was tested twice, with the EPMs doing a different random motion each time composed of 200 different poses.
\begin{figure}[!ht]
	\centering
	\includegraphics[scale=0.8]{./Figures/tested_poses.png}
	\caption{Tested poses across the workspace.}
	\label{fig:tested_poses}
\end{figure}

Fig.~\ref{fig:pos_results} and~\ref{fig:ori_results} depict the error in position e$_p$ and orientation e$_R$ respectively, for each tested pose and repeat. The observer converged to the right solution for all tested poses with an average error of 8.5 $\pm$ 2.4 mm in position norm - with 4.14 $\pm$ 3.0 mm along the X axis, 4.13 $\pm$ 3.0 mm on the Y axis, and 3.44 $\pm$ 2.5 mm along the Z axis - and $0.032 \pm 0.027$ in orientation trace error, i.e. 3.7 $\pm$ 3.6$^{\circ}$.
\begin{figure*}[!ht]
	\vspace{0.5cm}
	\centering
	\includegraphics[scale=0.8]{./Figures/pos_results.png}
	\caption{Error in position estimation for the ten tested poses across the workspace. Two repeats for each pose were performed.}
	\label{fig:pos_results}
\end{figure*}
\begin{figure*}[!ht]
	\centering
	\includegraphics[scale=0.8]{./Figures/ori_results.png}
	\caption{Error in orientation estimation for the ten tested poses across the workspace. Two repeats for each pose were performed.}
	\label{fig:ori_results}
\end{figure*}

However, as Fig.~\ref{fig:pos_results} and~\ref{fig:ori_results} show, there is significant variation in convergence speed and stability of the solution across repeats for the same pose. Given that the only difference between repeats is the EPMs motion, and therefore, the magnetic field measured by the sensors, the path each EPM takes and their combination have a big impact on the algorithm performance. This seems to be more significant for the estimation of the position than for the orientation, given that position estimation relies exclusively on magnetic field measurements. Fig.~\ref{fig:pos_results}(c) illustrates this effect very clearly, where for repeat 1 the algorithm converged to the right solution only to start diverging towards the end, and repeat 2 took longer to converge than all other cases. Unlike localization with a single EPM where the localization singularity plane is well defined and known, when multiple EPMs are present in the workspace, their relative pose dictates whether there are singularity regions and where they are. \textcolor{black}{Since the EPMs are travelling random paths}, it is possible that at times the sensors were located in a singularity region. Given the presence of multiple EPM configurations at each iteration of the observer, this does not seem to impact convergence but rather convergence speed. If the measurement model contained only a single configuration of EPMs, ideal for fast moving MAMR, these singularity conditions would need to be well defined and avoided.

To test the observer's behavior in non-static conditions two different scenarios were tested. First, to address periodic motions such as breathing, linear and angular velocities were given to the world reference frame \{$\mathcal{W}$\} in the previous set of experiments as to mimic MAMR motion. The obtained results are shown in Fig.~\ref{fig:speed_results}. Linear velocities of up to 0.1~mm/s and angular velocities up to 2 $^\circ$/s produced marginal differences when compared to the static cases. Velocities above these values had significant impact on the results. Second, the observer's robustness for occasional spike movements such as coughing was tested. Spikes of 5~cm of up to 4~seconds, and spikes of 10~cm of up to 2~seconds did not produce significant changes in results. Longer spike times made the results unreliable. These values, however, are highly dependent on the platform. In this case, the robotic arms were operating at 30\% of their full speed for safety reasons. Increasing this speed, and/or including less EPM configurations in the measurement model, would allow faster MAMR speeds and longer spike motions.
\begin{figure}[!ht]
	\centering
	\includegraphics[scale=0.8]{./Figures/speed_plot.png}
	\caption{Error in (a) position and (b) orientation estimation for different linear and angular MAMR velocities.}
	\vspace{-0.5cm}
	\label{fig:speed_results}
	
\end{figure}

\end{document}
