\documentclass[10pt,twocolumn,letterpaper]{article}

\usepackage{iccv}
\usepackage{times}
\usepackage{epsfig}
\usepackage{graphicx}
\usepackage{amsmath}
\usepackage{amssymb}
\usepackage{bm}

% Include other packages here, before hyperref.

% If you comment hyperref and then uncomment it, you should delete
% egpaper.aux before re-running latex.  (Or just hit 'q' on the first latex
% run, let it finish, and you should be clear).
\usepackage[pagebackref=true,breaklinks=true,letterpaper=true,colorlinks,bookmarks=false]{hyperref}

\iccvfinalcopy % *** Uncomment this line for the final submission

\def\iccvPaperID{6125} % *** Enter the ICCV Paper ID here
\def\httilde{\mbox{\tt\raisebox{-.5ex}{\symbol{126}}}}

% Pages are numbered in submission mode, and unnumbered in camera-ready
\ificcvfinal\pagestyle{empty}\fi

% \usepackage[latin1]{inputenc}
\usepackage[british]{babel}
\usepackage[all]{xy}
\usepackage{amscd}
\usepackage{amssymb}
\usepackage{amsthm}
\usepackage{enumitem}
\usepackage{mathrsfs,bbm}
\usepackage{xcolor,graphicx}
\usepackage{graphics}
\usepackage{soul}
\usepackage{comment}
\usepackage[all]{xy}
\usepackage{amscd}
\usepackage{amssymb,amsmath,latexsym}
\usepackage{amsthm}
\usepackage{enumitem}
\usepackage{mathrsfs,bbm}
\usepackage{dsfont}
\usepackage{tikz-cd}
\usepackage[T1]{fontenc}
\usepackage[utf8]{inputenc}  
 %
%%%%%%%%%%%%%%%%%%%%%%%%%%%%%%%%%%
%pagestyle
%%%%%%%%%%%%%%%%%%%%%%%%%%%%%%%%%%
%\pagestyle{plain}
\textwidth=430pt
\headsep=.7cm
\evensidemargin=15pt
\oddsidemargin=15pt
\leftmargin=0cm
\rightmargin=0cm
%%
%%%%%%%%%%%%%%%%%%%%%%%
\newcommand*\fixitem {\item[]%
  \refstepcounter{enumi}\hskip-\leftmargin\labelenumi\hskip\labelsep}
\newtheorem*{mainthm}{Main Theorem}
\newtheorem*{mainthm1}{Theorem}
\newtheorem*{maincor}{Corollary}
\usepackage[colorlinks=true]{hyperref}
\DeclareMathOperator{\Forall}{\forall}
\DeclareMathOperator{\Exists}{\exists}
\DeclareMathOperator{\ord}{ord}
\newcommand{\phiD}{\varphi_D}
\newcommand{\phiDI}{\varphi_{\mathbf{D}_I}}
\newcommand{\phiDIj}{\varphi_{\mathbf{D}_I (j)}}
\newcommand{\phiH}{\varphi_H}
\newcommand{\phiTimes}{\phiD \otimes \phiH}
\newcommand{\phiTimesDI}{\varphi_{\mathbf{D}_I} \otimes \phiH}
\newcommand{\R}{\mathscr{A}}
\newcommand{\X}{\mathscr{X}}
\newcommand{\Xf}{\mathscr{X}_{(k_0 ,i)}[r_0]}
\newcommand{\Xfr}{\mathscr{X}_{(k_0,i)}[r]}
\newcommand{\hotimes}{\widehat{\otimes}}
\newcommand{\C}{\mathbb{C}_p}
\newcommand{\V}{\mathscr{V}}
\newcommand{\B}{\mathscr{B}}
\newcommand{\dualD}{\mathfrak{D}}
\newcommand{\Dg}{\mathbf{D}}
\newcommand{\DD}{\mathcal{D}^0}
\newcommand{\DDg}{\mathcal{D}}
\newcommand{\DV}{\mathcal{D}}
\newcommand{\W}{\mathscr{W}_N}
\newcommand{\Ao}{\mathbf{A}^\circ}
\newcommand{\AoK}{\mathbf{A}^\circ_{\K}}
\newcommand{\AK}{\mathbf{A}_{/\K}}
\newcommand{\OOO}{\mathscr{A}^\circ}
\newcommand{\K}{\mathcal{K}} 
\newcommand{\OK}{\mathcal{O}_{\K}}
\newcommand{\varprojlog}[1]{\underleftarrow{\log\!^{#1}}}
\newcommand{\T}{\mathscr{T}}
\newcommand{\TT}{\mathbf{T}}
\newcommand{\VV}{\mathbf{V}}
\newcommand{\HH}{\mathcal{H}}
\newcommand{\hh}{\mathcal{H}^+}
\newcommand{\HG}[2]{\mathcal{H}_{#1}(#2)}
\newcommand{\hhl}{\mathcal{H}^{+,[l]}}
\newcommand{\hhj}{\mathcal{H}^{+,[j]}}
\newcommand{\hhjj}{\mathcal{H}^{+,[l,l']}}
\newcommand{\GS}{G_{\mathbb{Q},S}}
\newcommand{\Rf}{R_{(k_0 ,i)}[r_0]}
\newcommand{\Rfr}{R_{(k_0 ,i)}[r]}
\newcommand{\parT}{\langle T\rangle}
\newcommand{\Zf}{Z_{(k_0 ,i)}[r_0]}
\newcommand{\Zfr}{\mathscr{Z}_{(k_0 ,i)}[r]}
\newcommand{\ZFf}{\mathscr{Z}_{(k_0 ,i)}[r_0]}
\newcommand{\ZFfr}{\mathscr{Z}_{(k_0 ,i)}[r]}
\newcommand{\ZF}{\mathscr{Z}}

\begin{document}

%%%%%%%%% TITLE
% \title{NeRF-Surf: Implicit Surface Optimization from Trained NeRFs for Delicate and Semi-Transparent Surface Reconstruction}
% \title{\name{}: Implicit Surface Reconstruction for Semi-Transparent and Thin Objects}

\title{\name{}: Implicit Surface Reconstruction for Semi-Transparent and Thin Objects with Decoupled Geometry and Opacity}

\author{Tianhao Wu\\
University of Cambridge\\
% For a paper whose authors are all at the same institution,
% omit the following lines up until the closing ``}''.
% Additional authors and addresses can be added with ``\and'',
% just like the second author.
% To save space, use either the email address or home page, not both
\and
Hanxue Liang\\
University of Cambridge\\
\and
Fangcheng Zhong\\
University of Cambridge\\
\and
Gernot Riegler\\
Unity\\
\and
Shimon Vainer\\
Unity\\
\and
Cengiz Oztireli\\
Google Research\\
University of Cambridge\\
}

\maketitle
% Remove page # from the first page of camera-ready.
\ificcvfinal\thispagestyle{empty}\fi


%%%%%%%%% ABSTRACT
\begin{abstract}
% \FZ{I am strongly inclined not to mention NeRF in the title}\WW{I agree, let's discuss about the new title}
% \CO{We have to make the distinction between representation vs. diff. rendering method. All the methods we consider, including what we propose, works with diff. vol. rendering.}

% \CO{We have to also clearly state the task: surface from images.}

% \CO{As we do 1) voxels and interpolation, 2) decouple geometry and appearance, this looks a lot more like plenoxels-type than nerf-type.}

% \CO{Contribution: novel surface representation, 1. supports multiple surface patches for explicit surface modeling without SDF-related constraints, 2. with a closed-form ray-surface intersections for fast rendering, 3. can initialiezd from pre-trained vol. density based models.}

% \FZ{sketched ideas for a new abstract}
% \fz{
% \begin{itemize}
%     \item differentiable rendering and implicit surface representation gain success in recovering surfaces from images with scenes containing complex geometry and topology.
%     \item an open challenge is to recover implicit surface for semi-transparent objects
%     \item the same issue applies to the reconstruction of extremely thin surfaces due to the same cause.
%     \item differentiable SDF rendering fails to achieve this as their rendering only considers the nearest ray-surface intersection.
%     \item differentiable volume rendering fails to achieve this as density couples geometry with material properties
%     \item we propose a novel approach to resolve these issues
%     \item our results are amazing
% \end{itemize}
% }



Implicit surface representations such as the signed distance function (SDF) have emerged as a promising approach for image-based surface reconstruction. 
However, existing optimization methods assume solid surfaces 
and are therefore unable to properly reconstruct semi-transparent surfaces and thin structures, which also exhibit low opacity due to the blending effect with the background.
While neural radiance field (NeRF) based methods can model semi-transparency and achieve photo-realistic quality in synthesized novel views, their volumetric geometry representation tightly couples geometry and opacity, and therefore cannot be easily converted into surfaces without introducing artifacts.  
We present \name{}, a novel surface representation with decoupled geometry and opacity for the reconstruction of semi-transparent and thin surfaces where the colors mix. Ray-surface intersections on our representation can be found in closed-form via analytical solutions of cubic polynomials, avoiding Monte-Carlo sampling and is fully differentiable by construction.
Our qualitative and quantitative evaluations show that our approach can accurately reconstruct surfaces with semi-transparent and thin parts with fewer artifacts, achieving better reconstruction quality than state-of-the-art SDF and NeRF methods.\footnote{Preprint. Website: \url{https://alphasurf.netlify.app/}}
\end{abstract}



%%%%%%%%% BODY TEXT
% Importance and appeal of children's drawings
Children's depictions of the human figure are highly expressive and varied.
As one of the very first subjects children attempt to draw, the representation begins as an almost unintelligible cloud of scribbles. 
As the child grows, their representation of the human figure becomes more developed and is extended to graphically represent many different types of characters: people, animals, and even personified objects (see Figure 1).

Who among us has not wished, either as a child or as an adult, to see such figures come to life and move around on the page?
Sadly, while it is relatively fast to produce a single drawing, creating the sequence of images necessary for animation is a much more tedious endeavor, requiring discipline, skill, patience, and sometimes complicated software.
As a result, most of these figures remain static upon the page.

% We built a system to animate them.
Inspired by the importance and appeal of the drawn human figure, we design and build a system to automatically animate it given an in-the-wild photograph of a child's drawing. 
Our system is fast, intuitive, and robust to much of the variation present in these types of drawings, making it well-suited to allow our target audience--children--to see their own characters coming to life.
The system is comprised of four stages: figure detection, segmentation masking, pose estimation/rigging, and animation. 
We describe each stage and identify common causes of failure in each. 
For object detection and pose estimation, we make use of existing computer vision models designed to detect human figures and joints in photographs; we fine-tune these models for use with children's drawings.
For segmentation, we present a straightforward, image processing-based method that, for animation purposes, is more useful and accurate than segmentation masks obtained from a fine-tuned object detection model.
During the animation step, we take advantage of the \textit{twisted perspective} commonly seen in children’s drawings to retarget motion capture data onto the character in a novel and appealing way.

% We use existing machine learning models. However, given the wide domain gap it's not clear how much fine-tuning data was needed. So we ran some experiments to find out and report it.
While our system leverages existing models and techniques, most are not directly applicable to the task due to the many differences between photographic images and simple pen and paper representations. 
To this end, we couple the presentation of our system with a set of experiments exploring the relationship between fine-tuning training set size and success rates.
We also include a perceptual study validating viewer preference for incorporating \textit{twisted perspective} into the motion retargeting step.

We validate the desirability and appeal of our system by building and publicly releasing a version of it as the \AD Demo \,\cite{animateddrawings}.
Launched in December 2021, this demo has been used by millions of people around the world to animate their children's drawings.
Inspired by this reception, our second contribution is The Amateur Drawings Dataset: \hjs{180,000 drawings and user-accepted annotations collected, with consent, through the demo. See Section \ref{sec:UI} for a description of how the annotations were generated.}
We believe this dataset will be a resource to researchers from various fields seeking to better understand the space of amateur drawings, evaluate new algorithms in this domain, or develop new drawing-based tools in general.

To summarize, our contributions are as follows:
\begin{enumerate}
    \item 
    We explore the problem of automatic sketch-to-animation for children's drawings of human figures and present a framework that achieves this effect. We also present a set of experiments determining the amount of training data necessary to achieve high levels of success and a perceptual study validating the usefulness of our motion retargeting technique.
    \item To encourage additional research in the domain of amateur drawings, we present a first-of-its-kind dataset of 180,000 user-submitted amateur drawings, along with user-accepted bounding box, segmentation mask, and joint location annotations.
\end{enumerate}

Upon acceptance of this paper, we plan to publicly release the Amateur Drawings Dataset, project code, and fine-tuned model weights.


\section{Related Works}
% \FZ{can rw be shortened a bit?}\WW{removed a few sentences, will come back and remove more if needed}

\paragraph{Neural Radiance Fields} 
NeRF~\cite{nerf} is a 5D plenoptic function that models volume density and view-dependent appearance. Its differentiable volume rendering allows robust image-based 3D reconstruction and motivated an explosion of related works in areas including high-quality novel view synthesis~\cite{mipnerf360, refnerf, hdrnerf, nerf++}, 3D asset synthesis~\cite{gram, graf}, and efficient reconstruction and rendering~\cite{instantNGP, kilonerf, plenoxels, directvoxgo, plenoctrees}. Despite their outstanding novel view synthesis performance, many analyses suggest their geometry tends to produce artifacts such as sparse density floaters and inner volume~\cite{mipnerf360, refnerf, nerf++}. 
% Marching cube~\cite{marching_cube} 
Direct surface extraction on the density field hence suffers from those artifacts, whereas the depth extraction method does not guarantee watertight surfaces and requires additional surface reconstruction \cite{tsdf, alpha_shape, poisson_recon}
% such as TSDF fusion~\cite{tsdf}, poison surface reconstruction~\cite{poisson_recon} or alpha shapes~\cite{alpha_shape}
. UNISURF~\cite{UNISURF} attempts to mitigate this issue by gradually transiting from volume rendering to surface rendering, but its surface is tightly coupled as a fixed level set on the opacity field and it still only works on solid objects. Our approach deviates from this by using a separate implicit surface field with decoupled opacity to model semi-transparent and thin surfaces.

\vspace{-5pt}
\paragraph{SDF For Multi-view 3D Reconstruction} 
% To reconstruct well-defined surfaces from multi-view images, SDF has been widely utilized with various differentiable rendering techniques, which map 3D representations into 2D images in a differentiable way and therefore allow optimization via photometric loss.
SDF has been extensively employed with differentiable rendering methods to reconstruct surfaces from multi-view images. 
SDFDiff~\cite{sdfdiff} uses a voxel grid and trilinear interpolation to represent the SDF and develops differentiable sphere tracing to find an estimated intersection. 
Our approach deviates from this by using a closed-form solution to solve for \textit{exact} intersections with a more generalized implicit surface field, while also finding more than just the nearest intersections.
% supporting differentiable multi-surface rendering through alpha compositing. 
IDR~\cite{idr} proposes a differentiable sphere-tracing algorithm and optimizes the surface together with a volumetric BRDF. 
% However, this rendering procedure only has gradients defined at first intersections, causing the optimization to be difficult at depth discontinuities. 
VolSDF~\cite{volsdf} maps SDF to volume density via Laplacian CDF and optimizes the SDF via volume rendering.
% within a derived error bound on the approximated transparency function. 
NeuS~\cite{neus} similarly maps an SDF to unbiased weights in the volume rendering equation via a logistic sigmoid function. 
% With a trainable parameter that controls the spread of the weights around the zero-level set, it starts as a coarse surface with "spread-out" effects and converges to a concrete surface in the end. 
% This means their optimization still assumes surface solidity, and hence struggles on semi-transparent or thin surfaces. 
% Although VolSDF and NeuS both incorporate volume rendering in the optimization of SDF, they still assume surface solidity and encourage the rendering weights to become $1$ on the surface to fully occlude the ray. They hence still struggle on semi-transparent or thin surfaces. 
NeuS has motivated several further applications in different areas, such as sparse view surface reconstruction~\cite{sparse_neus}, fast reconstruction~\cite{neus2, voxel_surf, hash_sdf}, and finer details modeling such as HFS~\cite{HFS}, which applies a mapping from SDF to transparency and incorporates an additional displacement field to improve the reconstruction of fine details. 
A crucial and common limitation in existing SDF optimization methods is the assumption of surface solidity. Even the methods that utilize volume rendering in surface reconstruction, such as VolSDF and NeuS, still enforce convergence to solid surfaces in the end. Hence, they cannot properly reconstruct semi-transparent surfaces and suffer from thin structures with strong blending effects. 
% In comparison, our approach does not assume solid surfaces and can model both semi-transparent and thin objects.






\section{Testing for anisotropy}\label{sec:TestingAnisotropy}
The specific hypothesis to be tested is whether, above some energy threshold, $E_{\rm th}$, the mean composition of UHECRs coming from directions near to the galactic plane is significantly higher in mass than those arriving further from it. This is to be tested using \xmax{} as a mass sensitive parameter. Typically, \xmax{} based composition analyses leverage the first two moments of \xmax{} distributions binned in energy, to comment on primary mass. This approach, however, does not lend itself well to quantifying the significance of a result testing the above statement. Instead, a test statistic, $TS$, which quantifies the degree of dissimilarity between the \xmax{} distributions in the two regions in a single value is preferred. For this, the returned value from the Anderson-Darling two-sample homogeneity test \cite{andersondarling}, \textit{AD-test}, has been selected as it scales with the dissimilarity of the tested distributions. The AD-test has good sensitivity to the full width of a distribution \cite{scholz1987k}, and has more power than the Kolmorogov-Smirnov test while remaining robust against false positives \cite{engmann2011comparing}.

To use the AD-test and \xmax{} for this purpose, two modifications are required. First, a single $TS$ comparing all events in each region above $E_{\rm th}$ is desired. So, all events with $E\geq E_{\rm th}$ in the on- and off-plane samples separately need to be collected into a common on-plane distribution and a common off-plane distribution. To do this, the natural evolution of \xmax{} with energy needs to be removed so that spectral features in the flux do not influence the result. Therefore, we define an energy-normalized \xmax{} value
\begin{equation}\label{eq:XmaxNorm}
X_{\text{max}}^{'} =  X_{\text{max}} -  \underbrace{\left(649 + 63.1 \, Z_{18} + 1.97 \, Z_{18}^{2}\right)}_\text{EPOS-LHC elongation rate for iron},
\end{equation}
where $Z_{18}=\log_{10}\left(E_\text{rec}/\,\text{EeV}\right)$. The last term in \autoref{eq:XmaxNorm} is the natural energy evolution of mean \xmax{} for iron primaries as predicted by EPOS-LHC~\cite{Pierog:2013ria}\footnote{Choice of hadronic interaction model varies result by $\sim0.02$\,\gcm{}.}. Second, the \xmaxnorm{} distribution of an on-plane sample populated with primaries which are on average heavier than those in the off-plane sample will display a lower mean and a narrower width than that of the off-plane \xmaxnorm{} distribution. Since the null hypothesis is that there is either no composition difference or a heavier off-plane sample, a $TS$ sensitive to the ordering of the \xmaxnorm{} distributions is required\footnote{Modifying the test to also require $\sigma( X_{\text{max}}^\prime)^{\rm on} < \sigma( X_{\text{max}}^\prime)^{\rm off}$ would be more restrictive, but conservatively has not been applied.}. The AD-test is insensitive to ordering, so it is modified to
\begin{equation}
TS =
\begin{cases}
    AD: \langle X_{\text{max}}^\prime \rangle^{\rm on} < \langle X_{\text{max}}^\prime \rangle^{\rm off} \\
    -3\hspace{1mm}: \text{else}
\end{cases},
\end{equation}
where $AD$ is the result of the AD-test comparing the on- and off-plane distributions, and $-3$ is selected as it is well below the minimum of the AD-test.

\vspace{-.1cm}
\myparagraph{Scan for energy and galactic latitude thresholds}
\vspace{-.1cm}
A scan has been used to select the optimal on/off splitting latitude, $b_{\rm split}$, and minimum energy, $E_{\rm th}$, as uncertainties in GMF models and source distributions make other approaches impractical. In this scan, each trial [$E_{\rm th}$, $b_{\rm split}$] pair is used to form on- and off-plane subsets and the $TS$ is extracted. To preserve the statistical strength of the sparse FD data set, a coarse scan of $5^\circ$ steps in $\abs{\,b\,}$ from $20^\circ$ to $35^\circ$ and 0.1\,\lge{} steps in energy from $18.4$ to $19.4$\,\lge{} is used. The scan is performed on the data set from~\cite{Aab:2014kda}, which includes events through Dec 31\textsuperscript{st} 2012. At the time of writing, this \textit{scan data set} represents $54\,\%$ of the analyzed events. The remaining $46\,\%$ of events, the \textit{post-scan data set}, is reserved as blind. 

\begin{figure}[!htb]
    % \vspace{-.4cm}
    \centering
    \includegraphics[width=0.45\textwidth]{Figures/ScanResults.pdf}
    \vspace{-2mm}
    \caption{Parameter scan over 54\% of the data.}\label{fig:PRDScan}
    % \vspace{-7mm}
\end{figure}

Interestingly, as shown in \autoref{fig:PRDScan}, all tested pairs result in $\langle X_{\text{max}}^\prime \rangle^{\rm on} < \langle X_{\text{max}}^\prime \rangle^{\rm off}$. An optimal [$E_{\rm th}$, $b_{\rm split}$] of [$10^{18.7}$\,eV,$30^\circ$] was found with a $TS = 8.4$. The selected [$E_{\rm th}$, $b_{\rm split}$] is applied as a prescription to the post-scan data set, which independently confirms the result with a $TS = 12.6$, for a total $TS=21.0$ for the full data set. 

\vspace{-.1cm}
\myparagraph{Statistical significance}
\vspace{-.1cm}
The chance probability of the observed TS occurring with in an isotropic sky is tested using Monte Carlo methods on randomized skies derived from the real data. To form each randomized sky, the arrival direction is first decoupled from the energy and \xmaxnorm{} values of each event. These are then randomly re-paired to create a new sky which maintains the real \xmax{}, energy, and sky exposure distributions, but has a scrambled arrival direction/composition pairing. The above analysis is then used to extract a $TS$ from each sky which is compared to the result in data. Skies which display more extreme on-/off-plane differences than those observed in data are tallied and used to calculate the probability of an isotropic sky generating the observed $TS$. The results of this procedure are shown in  \autoref{fig:TStoSignificanceConversionNew}.

\begin{figure}[!htb]
    \centering
    \includegraphics[width=.8\columnwidth]{Figures/MCADSig.pdf}
    % \vspace{-3mm}
    \caption{The Monte Carlo determination of the post-scan (red) and all-data (blue) significance with 1 and 10 billion randomized skies, respectively.}\label{fig:TStoSignificanceConversionNew}
    % \vspace{-3mm}
\end{figure}

For the blind, post-scan data set, the prescribed [$E_{\rm th},b_{\rm split}$] pair is used to split each randomized sky into on- and off-plane samples and a $TS$ is extracted. In one billion random skies, only 5865 resulted in a more extreme $TS$ than the 12.6 observed in data. This indicates a chance probability of $5.87\times10^{-6}$ which corresponds to 4.4\,$\sigma$. 

To calculate the significance of the result when the scan- and post-scan data sets are combined, the entire analysis chain, including the scan, is duplicated. In each random sky, 54\,\% of the data is used to scan for the [$E_{\rm th},b_{\rm split}$] pair which results in the most extreme result, fully penalizing for the scan. These values are then used to split all data in the random sky into on- and off-plane subsamples and the $TS$ for the sky is extracted. From 10 billion random skies, only 5964 resulted in a more extreme $TS$ than the 21.0 observed in data. This indicates to a chance probability of $5.96\times10^{-7}$ which corresponds to 4.9\,$\sigma$. The strong penalization of the scanned data is evident as the additional 54\,\% of the data (with \Dxmaxmunorm{} $= 8.5$\,\gcm{}) only resulted in an 11\,\% increase of the significance of the observation. 

\myparagraph{\xmax{} moments and trends}

To illustrate the difference in composition on and off the plane, the first two moments of the \xmax{} distribution in each 0.1\,\lge{} energy bin has been plotted in \autoref{fig:CompositionPlots} for both regions. Above $10^{18.7}$\,eV there is a clear separation in \xmaxmu{} for all energy bins. Most energy bins also display a separation in \xmaxsigma{}. Heavier primaries are expected to, on average, have a shallower \xmax{} and lower shower-to-shower fluctuations. Therefore the correlated difference seen here indicates that, for this data sample, primaries from the on-plane region have a higher mean mass than that of the off-plane region above $10^{18.7}$\,eV.

To evaluate the degree to which fluctuation plays a role in the observed result, the growth of the $TS$ over time has been plotted in \autoref{fig:TimeEvolution}. The time evolution of the signal is consistent with linear growth at a rate of 1.3\,$TS$\,yr$^{-1}$. This behavior is in line with expectations for a real difference in mean mass between the subsamples. The shaded region of \autoref{fig:TimeEvolution} shows preliminary data from 2019. These reconstructions were not subject to a validated reconstruction chain and may change. Still, when added, a 3.7/4.4\,$\sigma$ (post-scan/all data) statistical significance is expected. The best fit rate of growth of 1.3\,$TS$\,yr$^{-1}$ remains unchanged.

\begin{figure*}[!hbt]
\centering
    \begin{minipage}{.63\textwidth}
      \centering
      \captionsetup{width=.9\linewidth}
      \includegraphics[width=.49\textwidth,valign=t]{Figures/Mean-crop.pdf}
      \includegraphics[width=.49\textwidth,valign=t]{Figures/Sigma-crop.pdf}
      \vspace{-1mm}
      \caption{The first (left) and second (right) moments of the \xmax{} distributions from on- and off-plane regions.}
      \label{fig:CompositionPlots} 
    \end{minipage}%
    \hfill
    \begin{minipage}{.35\textwidth}
      \centering
      \vspace{-1mm}
      \includegraphics[width=\textwidth,valign=t]{Figures/TimePredict.pdf}
      \vspace{1.5mm}
      \captionof{figure}{The time evolution of the TS with significance indicated on the right. The shaded region is preliminary data.}
      \label{fig:TimeEvolution}
    \end{minipage}
\end{figure*}

\section{Evaluation}\label{sec:evaluation}

We evaluate our observer-based runtime verification approach by investigating the following research questions: 
\begin{itemize}
    \item [RQ1:] How efficient is the observer-based runtime verification in terms of time for processing monitored events with observers and memory needed to represent observers?
    \item [RQ2:] How accurate is the observer-based runtime verification in detecting violations of properties? 
    \item [RQ3:] How fast is a dynamic adaptation of an observer at runtime compared to a redeployment of the observer? 
    \item [RQ4:] Can the observer-based runtime verification with adapting observers increase the trustworthiness of SAS?
\end{itemize}

To perform our evaluation, we implemented our runtime verification approach with its  observers and adaptations following the PSP and PAP with C++ on Arduino.\footnote{The PSP/PAP catalog, implementation, and replication package for the evaluation are available at: \url{https://www.github.com/HUB-SE/PAP/}} 
We deployed the code on multiple Arduino Mega\footnote{Arduino Mega 2560 Rev3, 8 KB SRAM, 16MHz clock speed.}. 

\noindent
\textbf{RQ1 } 
This research question addresses the efficiency of our observer-based runtime verification. To perform verification online, the runtime verifier is desired to process monitored events faster than the managed system and environment emit them. Thus, the verifier can provide fast results without the possibility of an overflowing event queue. 

To determine how much time it takes for our observer-based verification technique to process events, we generated ten artificial traces containing events of five different types. The traces have a length of $50,000$ events. We measured the time that our technique took with an artificial observer to process these traces. This artificial observer contains five states. Each state has five outgoing transitions, each labeled with one of the five event types.\footnote{We omit evaluating the costs of managing timers due to guarded transitions in the observer because they are similar to processing an event requiring in both cases to check all outgoing transitions of the observer's current state.} Thus, with each processed event of the trace, one transition will be enabled regardless of the current state of the observer.
For each processed event, any outgoing transition of the current state has to be checked until the transition with the matching label is found. Overall, the artificial observer has 25 transitions, which  in our experience is a realistic upper bound for an observer~\cite{vogel2023property}. 

We execute our runtime verification technique with the artificial observer on Arduino Mega against the ten artificial traces. On average, our technique took $6571.1ms$ to process a trace with a standard deviation of $\mu$\,$\leq$\,$7.2ms$.
Thus, on average it takes $0.13ms$ ($6571.1ms$/$50,000$ events) for our technique to process a single event with an observer. 
Prominent benchmarks for runtime verification such as Timescales~\cite{ulus2019timescales} provide traces that contain one event per \textit{ms}. Thus, we conclude that our observer-based runtime verification technique is sufficiently efficient concerning the execution time. This especially holds since we check only monitored events representing changes of the managed system or environment, where we consider a rate of one change per $ms$ as extraordinarily high with respect to our experience with the BSN. 

We also investigate the memory needed to represent an observer in a data structure implemented on Arduino in terms of SRAM usage. For this purpose, we use observers for nine properties with different combinations of patterns and scopes as well as the artificial observer discussed previously.
As shown in Table~\ref{tab:memory}, the observers consume between \textit{355} and \textit{1,136 bytes} of memory. For each observer, we list the number of states and transitions to illustrate the size of the observer. Such sizes are representative of properties following the PSP. 
We conclude that observers can be efficiently represented given their size in terms of states and transitions and multiple observers each representing a property can be deployed to one Arduino Mega that has \textit{8KB} of SRAM. 
\begin{boxA}
    Our runtime verification is efficient as it just requires $0.13ms$ on average to process a monitored event and between $355$ and $1,136$ \textit{bytes} of memory to represent an observer. 
\end{boxA}

\begin{table}[tbp]
    \begin{center}
        \caption{Size of observers for properties in terms of numbers of states (\#S) and transitions (\#T), and memory usage in bytes.}
        \label{tab:memory}
        \vspace{-.5em}
        \begin{tabular}{c c c c}
            \toprule
            \textbf{Pattern + Scope} & \textbf{\#S} & \textbf{\#T} & \textbf{Memory (bytes)}\\
            \midrule
            Absence After & 5 & 4 & 614 \\
            Absence Before & 5 & 4 & 558 \\
            Absence Between & 6 & 8 & 866 \\
            Recurrence Globally & 2 & 2 & 355 \\
            Recurrence Between & 4 & 5 & 605 \\
            Response Globally & 3 & 3 & 458 \\
            Response Between & 4 & 6 & 652 \\
            Response Chain Between, 2 responses & 6 & 11 & 940 \\
            Response Chain Between, 3 responses & 7 & 14 & 1,136 \\ \midrule
            Artificial observer & 5 & 25 & 1,047 \\
            \bottomrule
            \end{tabular}
    \end{center}
    \vspace{-2.25em}
\end{table}

\noindent
\textbf{RQ2 } 
In this research question, we investigate the correctness of our runtime verification. To this end, we implemented a trace generator according to the grammar of Timescales~\cite{ulus2019timescales}, which is a runtime verification benchmark. Each trace targets a property based on the PSP catalog and can be generated to either satisfy or violate the property. Thus, the generator provides the ground truth of whether a generated trace violates or satisfies the property.
We considered nine properties that follow the patterns and scopes shown in Table~\ref{tab:memory}.
For each property, we generated 20 different traces, each consisting of about 60 events. Ten of them satisfy and ten violate the property. 
Afterward, we instantiated the observer template of our PSP catalog for the property. We deployed the resulting observer and evaluated, whether it reaches an \textit{error} state when processing the trace. 
We found that our observers classified each of the 20 traces correctly for each of the nine properties. 
\begin{boxA}
    We can report 100\% accuracy in detecting property violations since our observer-based runtime verification has provided correct results for all 180 runs of the experiment (20 traces for each of the nine properties).
\end{boxA}

\begin{table*}[tbp]
    \begin{center}
        \caption{Requirements changes for the BSN. Added/updated parts of the property along the changes are highlighted in blue.}
        \label{tab:scenarios}
        \begin{tabular*}{\textwidth}{c L{0.12\textwidth} L{0.13\textwidth} L{0.67\textwidth}}
            \toprule
            \textbf{\#} & \textbf{Req. Change} & \textbf{PAP} & \textbf{MTL Property} \\
            \midrule
            0 & Initial situation (cf. Eq.~\ref{eq:req1}) & -- &$\square (( \text{cycle\_starting} \wedge \lozenge \text{cycle\_ending} ) \rightarrow (\text{request} \rightarrow (\neg \text{cycle\_ending } \text{ }\mathcal{U}^{[0,2]}\text{ } (\text{thermometer\_reply} \wedge \linebreak \neg \text{cycle\_ending} \wedge (\lozenge ^{[0, 2]} (\text{pulse\_reply} ))  ))) \text{ }\mathcal{U}\text{ } \text{cycle\_ending})$\\ \midrule
            1 & Add a glucometer & Adding a Response to the Chain & $\square (( \text{cycle\_starting} \wedge \lozenge \text{cycle\_ending} ) \rightarrow (\text{request} \rightarrow  (\neg \text{cycle\_ending} \text{ }\mathcal{U}^{[0,2]}\text{ }  (\text{thermometer\_reply} \wedge \linebreak \neg \text{cycle\_ending} \wedge (\lozenge ^{[0, 2]} (\text{pulse\_reply} )) \textcolor{blue}{\wedge \neg \text{cycle\_ending} \wedge (\lozenge ^{[0, 2]} (\text{glucose\_reply} ))} ))) \text{ }\mathcal{U} \text{ }\text{cycle\_ending})$\\ \midrule
            2 & Update time guard & Updating a Time Guard & $\square (( \text{cycle\_starting} \wedge \lozenge \text{cycle\_ending} ) \rightarrow (\text{request} \rightarrow (\neg \text{cycle\_ending} \text{ }\mathcal{U}^{[0,\textcolor{blue}{3}]}\text{ } (\text{thermometer\_reply} \wedge \neg \text{cycle\_ending} \wedge  (\lozenge ^{[0, \textcolor{blue}{3}]} (\text{pulse\_reply} )) \wedge \neg \text{cycle\_ending} \wedge (\lozenge ^{[0, \textcolor{blue}{3}]} (\text{glucose\_reply} )) ))) \text{ }\mathcal{U}\text{ } \text{cycle\_ending})$\\ \midrule
            3 & Remove the thermometer & Rem. a Response from the Chain & $\square (( \text{cycle\_starting} \wedge \lozenge \text{cycle\_ending} ) \rightarrow (\text{request} \rightarrow  \neg \text{cycle\_ending} \wedge  (\lozenge ^{[0, 3]} (\text{pulse\_reply} )) \wedge \neg \text{cycle\_ending} \wedge (\lozenge ^{[0, 3]} (\text{glucose\_reply} )) ) \text{ }\mathcal{U}\text{ } \text{cycle\_ending})$\\ \midrule
            4 & Scheduler requests data & Updating an Event & $\square (( \text{cycle\_starting} \wedge \lozenge \text{cycle\_ending} ) \rightarrow (\text{\textcolor{blue}{s\_request}} \rightarrow  (\neg \text{cycle\_ending} \text{ }\mathcal{U}^{[0,3]}\text{ } (\text{pulse\_reply} \wedge \neg \text{cycle\_ending} \wedge (\lozenge ^{[0, 3]} (\text{glucose\_reply} ))  ))) \text{ }\mathcal{U} \text{ }\text{cycle\_ending})$\\ \midrule     
            5 & Neglect order of sensors & Splitting the Response Chain & $\square (( \text{cycle\_starting} \wedge \lozenge \text{cycle\_ending} ) \rightarrow (\text{s\_request} \rightarrow  (\neg \text{cycle\_ending} \text{ }\mathcal{U}^{[0,3]}\text{ } (\text{pulse\_reply}))) \text{ }\mathcal{U}\text{ } \text{cycle\_ending})$ -- \textit{and a similar property for {glucose\_reply}}\\

            \bottomrule
            \end{tabular*}
    \end{center}
    \vspace{-2em}

\end{table*}

\noindent
\textbf{RQ3 }
This research question addresses the performance of a dynamic adaptation of a property at runtime. Thus, we compare the runtime efficiency of a \textit{dynamic adaptation} based on our PAP and a \textit{redeployment} of an observer. A redeployment comprises invoking the destructor to free up the memory consumed by the observer and the constructor to instantiate and represent the new observer in freshly allocated memory. 
For this experiment, we use the Response Chain property shown in Eq.~\ref{eq:req1}. For requirements changes, we alternate between adding and removing responses from the chain as well as updating response events in the chain. Such changes can easily be repeated multiple times on an observer to achieve reliable time measurements. For one run, we alternate between the changes until each of them is performed $1,000$ times resulting in a total of $3,000$ changes that are either realized by $3,000$ dynamic adaptations or $3,000$ redeployment of the observer. We repeat both runs ten times. The runs are all executed on Arduino.

On average across the ten runs, the $3,000$ dynamic adaptations of the observer took in total $3.034s$ (stdev $\mu_1$\,$\leq$\,$0.49ms$). 
Thus, one dynamic adaptation of an observer takes on average \textit{1.01ms}. 
In contrast, the $3,000$ redeployments of the observer took on average $15.330s$ (stdev $\mu_2$\,$\leq$\,$0.85ms$), that is, on average \textit{5.11ms} for one redeployment of an observer. 

\begin{boxA}
    A dynamic adaptation of an observer ($1.01ms$) is more than five times faster than a redeployment ($5.11ms$). 
\end{boxA}

\noindent
\textbf{RQ4 }
For the last research question, we investigate how our approach of dynamically adapting observers to reflect requirements changes can increase the trustworthiness of a SAS. To this end, we use a port of the BSN artifact~\cite{BSN} we implemented for the Arduino platform.
Starting with an initial situation of the BSN described by the requirement discussed in Section~\ref{subsec:initialzingRV} and formalized by the MTL property in Eq.~\ref{eq:req1}, we consider a sequence of five requirements changes shown in Table~\ref{tab:scenarios}. For each requirements change, the table shows the PAP to specify the adaptation of the property/observer and the MTL property after adaptation. We execute the BSN alongside our runtime verification approach and use the PAP to specify and perform the dynamic adaptations of the observer to reflect sequentially these five requirements changes in the verification.

With this demonstration of our approach, we show that using PAP allows us to specify adaptations of observers with precise semantics as shown by the corresponding MTL properties before and after an adaptation (cf.~Table~\ref{tab:scenarios}). Such adaptations of observers are dynamically and safely performed so that our approach preserves the knowledge---in terms of intermediate verification results as progress made in the observer---without compromising the integrity of the observer. 
For the given property that is adapted (Table~\ref{tab:scenarios}), the knowledge preserved in the observer comprises whether a scheduler cycle has already started and if so, which of the sensors already have and which still need to send data to the BodyHub. 
Thus, our approach achieves an incremental, continuous verification of the currently executing scheduler cycle against the adapted property.
Without preserving this knowledge (e.g., by a redeployment), the currently executing scheduler cycle remains unverified against the adapted property as the observer is reset to its initial state where it expects a novel cycle to start (cf.~\textit{cycle\_starting} event). In such a situation, there is no verification evidence about the safety of the BSN. 
\begin{boxA}
    Applying the PAP enables a continuous, incremental verification of the BSN that increases the trustworthiness of the BSN when requirements changes occur. 
\end{boxA}

\noindent
\textbf{Discussion}
In our evaluation, we have shown the efficiency of our observer-based runtime verification ($0.13ms$ to process a monitored event and at most $1.136$ bytes to represent a monitor) and adaptation of properties based on PAP ($1.01ms$ to dynamically adapt an observer). Thus, our approach can efficiently be used on microcontrollers such as Arduino. 

Moreover, we have shown empirically the correctness of our observer-based runtime verification using a benchmark based on Timescales~\cite{ulus2019timescales} as ground truth. Since there is no ground truth for verifying a running system against adapted properties, we cannot validate the correctness of our verification approach under changing requirements. Thus, we demonstrated qualitatively the benefits of continuous, incremental runtime verification on the trustworthiness of the~BSN~\cite{BSN}. 

\noindent
\textbf{Threats to Validity}
Threats to the validity of our study are as follows. 
\textit{Construct:}
Potential errors in our implementation of the observers and the Timescales grammar cause a threat to the validity of our reported results on correctness. We address this threat by having reviewed the observers and making the implementation and replication package publicly available. 
\textit{Internal:}
Threats of this category concern the experiments and measurements we conducted. To mitigate measurement errors and obtain reliable results, we repeated experiments and performed them on a SEAMS artifact ported to Arduino and on a benchmark based on Timescales that is used by runtime verification research community. 
Moreover, requirements formalized with the Structured English Grammar~\cite{AutiliGLPT15} might not match the stakeholders' intentions, which is also true for the properties/observers and eventually for the verification results. In this context, we rely on our expertise on the BSN~\cite{BSN,Solano+2019} and property specification patterns~\cite{AutiliGLPT15,vogel2023property}.  
\textit{External:}
We considered only the BSN with one requirement that changes in five ways in our study. Thus, our results may not generalize to other SAS, requirements, and changes.
Finally, our PSP/PAP catalog currently supports four PSP with different scopes and five PAP, two of which can be applied to all four PSP and three only to the Response Chain pattern. Thus, we cannot generalize our catalog to other patterns collected in~\cite{AutiliGLPT15}.


%\section{}
%\label{sec:resDir}


\section{Conclusion}
\label{sec:conclusion}
% <>
Since its advent in 1931, Koopman operator theory \cite{koopman:1931} has only recently been actively utilized for solving practical problems, thanks to the introduction of the DMD algorithm in 2008 \cite{schmid:2008}. Since then, a multitude of DMD algorithm variations have risen to prominence and found utility across various fields. A notable feature of our survey paper was reviewing and categorizing the results of over 100 research papers based on both application and algorithm type in smart mobility and vehicle engineering  (see Table~\ref{tab1} and Section~\ref{sec:vehicApp}).  Additionally, this survey paper identified potential research gaps in smart mobility and vehicular engineering applications (Remarks~\ref{remGap1}--\ref{remGap6}). Finally, this review paper discussed theoretical aspects of Koopman operator theory that have been largely neglected by the smart mobility and vehicle engineering community and yet have large potential for contributing to solving open problems in these areas (see Section~\ref{subsec:theorIssue}).

\noindent{\textbf{Future Research Directions.}}	Given the emergence of cyber-threats against connected and autonomous vehicles as well as robotic systems (see, e.g.,~\cite{nekouei2021randomized,mohammadi2022generation}), a future research direction might include utilizing Koopman operator-based algorithms for designing cyber-resilient vehicular and smart mobility applications (see, e.g.,~\cite{taheri2022data} for a related line of research). Another potential research direction is using Koopman operator-based algorithms for predicting the motion of vulnerable road users (VRUs), e.g., pedestrians and cyclists (see, e.g.,~\cite{pool2019context,scholler2020constant}). Finally, rehabilitation robotics and robotic exoskeletons can be the benefactors of the predictive capabilities of Koopman operator-based algorithms for detecting tripping events and/or system  identification in various modes of locomotion (see, e.g.,~\cite{kumar2019extremum,aprigliano2019pre}).



%Fig. 1 depicts the accumulation of such algorithms since 2014, which are particular to vehicle engineering and smart mobility, i.e., the focus of this review. Table 1 summarizes the varieties of relevant algorithms developed in those studies. Furthermore, we have highlighted theoretical issues, whose expansion will have potential applications to the wide research area of smart mobility and vehicle engineering.  

%Although fairly comprehensive, we have found several gaps in this research area. In particular, we could not find any studies related to elevators, robots/vehicles employing crawling, slithering, hopping or peristaltic locomotion, arctic or special-terrain vehicles such as those employing screws or tracks, hovercraft and other amphibious vehicles or subsystems which tolerate flexible environments, classification or guidance systems related to vehicles for drilling or agriculture, or for current-ripple, power-split, battery health monitoring, nuclear propulsion, exoskeletons/prosthetics, personal mobility, motorsports, specialized rovers or similar open problems in emerging areas.  These examples are, of course, not exhaustive.  
%
%The purely data-driven nature of Koopman operators holds the promise of capturing unknown and complex dynamics for reduced-order model generation and system identification, through which the rich machinery of linear control techniques can be utilized. The emergent nature of the smart mobility and vehicular-related applications, where  the Koopman operator  in each particular application needs to be approximated, implies that the development of various Koopman operator approximation  algorithms is expected to grow along with the vehicular problems they aim to solve.  Given the ongoing development of this research area and the many existing open problems in the fields of smart mobility and vehicle engineering, a survey of techniques and open challenges of applying Koopman operator theory to this vibrant area is warranted.  To the best of our knowledge, this survey paper is the \emph{first of its kind} reviewing the applications of Koopman operator theory within a focused research area, namely, smart mobility and vehicle engineering applications. A \emph{notable feature} of our survey paper is reviewing and categorizing the results of over 100 research papers based on both application and algorithm type  (see Tables~\ref{tab1}--~\ref{tab4} and Section~\ref{sec:vehicApp}) that are concerned with the applications of Koopman operator theory to the field of smart mobility and vehicular engineering. Such a \emph{comprehensive and  detailed categorization} will be beneficial to the research practitioners working in the field.  Furthermore, this review paper discusses theoretical aspects of Koopman operator theory that have been largely neglected by the smart mobility and vehicle engineering community and yet have large potential for contributing to solving open problems in these areas. Additionally, our survey paper seeks to \emph{identify gaps} in the smart mobility and vehicle engineering research where new and existing Koopman operator-based methods have the potential to further develop and address unsolved problems  potentially benefiting from the perspectives of nonlinear system identification, control, global linearization, and the predictive powers that Koopman operator theory has to offer (see, e.g., Remarks~\ref{remGap1}--\ref{remGap6}). 



{\small
\bibliographystyle{ieee_fullname}
\bibliography{egbib}
}


% ****** Start of file apssamp.tex ******
%
%   This file is part of the APS files in the REVTeX 4.1 distribution.
%   Version 4.1r of REVTeX, August 2010
%
%   Copyright (c) 2009, 2010 The American Physical Society.
%
%   See the REVTeX 4 README file for restrictions and more information.
%
% TeX'ing this file requires that you have AMS-LaTeX 2.0 installed
% as well as the rest of the prerequisites for REVTeX 4.1
%
% See the REVTeX 4 README file
% It also requires running BibTeX. The commands are as follows:
%
%  1)  latex apssamp.tex
%  2)  bibtex apssamp
%  3)  latex apssamp.tex
%  4)  latex apssamp.tex
%
\documentclass[%
preprint,
superscriptaddress,
%groupedaddress,
%unsortedaddress,
%runinaddress,
%frontmatterverbose, 
%preprint,
%showpacs,preprintnumbers,
%nofootinbib,
%nobibnotes,
%bibnotes,
 amsmath,amssymb,
 aps,
%pra,
%prb,
%rmp,
%prstab,
%prstper,
%floatfix,
]{revtex4-1}

\usepackage{graphicx}% Include figure files
\usepackage{dcolumn}% Align table columns on decimal point
\usepackage{bm}% bold math
%\usepackage{hyperref}% add hypertext capabilities
%\usepackage[mathlines]{lineno}% Enable numbering of text and display math
%\linenumbers\relax % Commence numbering lines

\usepackage{todonotes}
\usepackage{xcolor}
\usepackage[normalem]{ulem}
\usepackage{braket}


\usepackage[pagebackref=false]{hyperref}
\renewcommand{\thefigure}{S\arabic{figure}}

\begin{document}
\author{L.~Banszerus}
\thanks{These two authors contributed equally.}
\author{S.~M\"oller}
\thanks{These two authors contributed equally.}
\affiliation{JARA-FIT and 2nd Institute of Physics, RWTH Aachen University, 52074 Aachen, Germany,~EU}%
\affiliation{Peter Gr\"unberg Institute  (PGI-9), Forschungszentrum J\"ulich, 52425 J\"ulich,~Germany,~EU}
\author{K.~Hecker}
\author{E.~Icking}
\affiliation{JARA-FIT and 2nd Institute of Physics, RWTH Aachen University, 52074 Aachen, Germany,~EU}%
\affiliation{Peter Gr\"unberg Institute  (PGI-9), Forschungszentrum J\"ulich, 52425 J\"ulich,~Germany,~EU}
\author{K.~Watanabe}
\affiliation{Research Center for Functional Materials, 
National Institute for Materials Science, 1-1 Namiki, Tsukuba 305-0044, Japan}
\author{T.~Taniguchi}
\affiliation{International Center for Materials Nanoarchitectonics, 
National Institute for Materials Science,  1-1 Namiki, Tsukuba 305-0044, Japan}%
\author{F.~Hassler}
\affiliation{JARA-Institute for Quantum Information, RWTH Aachen University, 52056 Aachen, Germany, EU}
\author{C.~Volk}
\author{C.~Stampfer}
\email{stampfer@physik.rwth-aachen.de}
\affiliation{JARA-FIT and 2nd Institute of Physics, RWTH Aachen University, 52074 Aachen, Germany,~EU}%
\affiliation{Peter Gr\"unberg Institute  (PGI-9), Forschungszentrum J\"ulich, 52425 J\"ulich,~Germany,~EU}%


\title{Supporting Information:\\ Particle-hole symmetry protects spin-valley blockade in graphene quantum dots}


\date{\today}

\maketitle
\tableofcontents

\clearpage
\subsection{Charge stability diagrams for opposite bias voltages in DQD \#1}

\begin{figure}[!thb]
\centering
\includegraphics[draft=false,keepaspectratio=true,clip,width=\linewidth]{FigS1A.pdf}
\caption[Fig01]{Charge stability diagrams of DQD \#1 (as in Fig.~1d of the main text) measured at a bias voltage of \textbf{a} $V_\mathrm{SD} = 1$~mV and \textbf{b} $V_\mathrm{SD} = -1$mV (T=10mK).
The dashed circles mark the formation of single electron -- single hole DQDs using the hole QD and an electron QD to the left (red) or right (black) of the hole QD. \textbf{c-d} Schematics of the valence and conduction band edge profiles along the p-type
channel. An electron-hole double quantum dot is formed using the hole QD and the electron QD underneath the left (right) FG (see red (black) circles in Fig.~S1a,b).
 }
\label{S1}
\end{figure}

Fig.~\ref{S1} compares charge stability diagrams measured at positive and negative bias voltage in DQD \#1 (c.f. Figs. 1, 2 and 3 in the main text). 
The dashed lines indicate the charge transitions of the electron (black) and hole (red) QDs. 
Electron-hole (e-h) DQDs are formed at the intersections of these charging lines. 
For the left electron-hole DQD ($(0h,0e)\leftrightarrow(1h,1e)$ transition, see red circle), transport is blocked at positive bias, while for the right electron-hole DQD ($(0h,0e)\leftrightarrow(1h,1e)$ transition, see black circle), transport is blocked at negative bias. The data in the main text has been obtained in the latter regime.

\clearpage
\subsection{Extracting $\Delta_\text{SO}$ from measurements on a single-electron DQD in the same device}
To compare the measured value for $\Delta_\mathrm{SO}$ in the electron-hole DQD and to demonstrate that the magnitude of the SO gap is symmetric for electrons and holes, we present measurements of  $\Delta_\mathrm{SO}$ in an electron-electron DQD. Fig.~\ref{S5} shows a close-up of the first triple point of an electron-electron DQD formed in the same device (c.f. Fig.~\ref{S1}). Transport via a ground state and an excited state can be observed. 
We extract their energy splitting by fitting two Lorentzian peaks to a linecut through the triple point (see Fig.~\ref{S5}b). The determined value of $\Delta_\mathrm{SO} = 68 \pm 7~\mu$eV is in good agreement with the ones observed in the electron-hole DQD regime. A detailed discussion of $\Delta_\mathrm{SO}$ and the single particle spectrum in the electron DQD in this device is given in Ref.~\cite{Banszerus2021Sep}.


\begin{figure}[!thb]
\centering
\includegraphics[draft=false,keepaspectratio=true,clip,width=0.9\linewidth]{FigS2A.pdf}
\caption[Fig01]{
\textbf{a} Charge stability diagrams of the $(1e,0e)\leftrightarrow(0e,1e)$ transition of an electron-electron DQD measured at $V_\mathrm{SD} = 1~$mV and $B_\perp = 0~$T (T=10mK).
A ground state and an excited state transition are visible (see black arrows). 
\textbf{b} Cut along the yellow dashed line in a. Two Lorentzian peaks (dashed lines) are fitted to the data.
Inset: Schematic energy diagrams of an electron-electron DQD in the finite bias regime for different interdot detuning
energies $\varepsilon$, illustrating resonant transport from the left (L) to the right (R) QD through the ground state of each QD (transition (i)) and resonant transport at $\varepsilon = \Delta_\mathrm{SO}$ (transition (ii)).
}
\label{S5}
\end{figure}

\clearpage
\subsection{Additional data set for another e-h double quantum dot (DQD \#2) in the same device }

\begin{figure}[!thb]
\centering
\includegraphics[draft=false,keepaspectratio=true,clip,width=0.85\linewidth]{FigS3A.pdf}
\caption[Fig03]{ \textbf{a} and \textbf{b} Gate configurations used to form DQD \#1 and DQD \#2 in the device, respectively. \textbf{c}
 Charge stability diagram of an e-h DQD formed with the second set of gate fingers (DQD \#2, see panel b). The dashed circle marks the $(0h,0e)\rightarrow(1h,1e)$ transition. $V_\mathrm{SD} = 1~$mV (T=10mK).
}
\label{S3}
\end{figure}

\begin{figure}[!thb]
\centering
\includegraphics[draft=false,keepaspectratio=true,clip,width=1\linewidth]{FigS4A.pdf}
\caption[Fig03]{  
\textbf{a, b} Close-ups of the $(0h,0e)\rightarrow(1h,1e)$ triple point at $V_\mathrm{SD} = 0.5~$mV and $V_\mathrm{SD} = 1.5~$mV, respectively. Transport only occurs via the $\alpha$ and $\beta$ transition.
\textbf{c} Charge stability diagram as in c measured at $B_\perp = 0.6$~T. 
\textbf{d} Charge stability diagram as in b at $B_\parallel = 0.7$~T.
\textbf{e, f} Charge stability diagrams as in b and c at $V_\mathrm{SD} = -0.5$~mV and $V_\mathrm{SD} = -1.5$~mV. Transport is strongly suppressed, only co-tunneling can be observed. 
\textbf{g, h}  Charge stability diagrams as in g measured at $B_\perp = 0.6$~T and $B_\parallel = 0.7$~T.
}
\label{S32}
\end{figure}

A second e-h DQD has been studied, formed with a different set of gate fingers on the same gated bilayer graphene device as presented in the main text (DQD \#2 depicted in Fig.~S2b). 
The single electron -- single hole transition, $(0h,0e)\rightarrow(1h,1e)$, is highlighted by the dashed circle in the charge stability diagram (see Fig.~\ref{S3}c). 

Measurements of that bias triangle are shown in Figs.~\ref{S32} for different $V_\mathrm{SD}$ and magnetic fields, showing good agreement with the data presented for DQD \#1 in Fig.~2.
In contrast to the data presented in the main manuscript, co-tunneling is more pronounced due to a strong coupling of the hole QD to the reservoir. 


\begin{figure}[!thb]
\centering
\includegraphics[draft=false,keepaspectratio=true,clip,width=\linewidth]{FigS5A.pdf}
\caption[Fig01]{
\textbf{a} Energy dispersion of single-particle states in the first orbital for electrons and holes as a function of in-plane ($B_\parallel$, left) and out-of-plane ($B_\perp$, right) magnetic fields. States and transitions are labelled as in Fig.~3a of the main text.
\textbf{b} Current through DQD \# 2 as a function of the detuning energy $\widetilde \varepsilon$ (see yellow dashed line in Fig.~\ref{S32}b) and $B_\perp$ at $V_\mathrm{SD} = 1.5~$mV. The white dashed line marks the onset of the bias transport window.
\textbf{c} Current through the device as a function of $\widetilde \varepsilon$ and $B_\parallel$ at $V_\mathrm{SD} = 1.5~$mV. \textbf{d, e} Data acquired in the blockade regime ($V_\mathrm{SD} = -1.5~$mV). The current has been measured as a function of $B_\perp$ and $B_\parallel$, respectively. Data has been symmetrized around $B = 0$.
}
\label{S4}
\end{figure}




The magnetic field dependent spectrum of the first electron and the first hole states is depicted in Fig.~\ref{S4}a (c.f. Fig.~3a of the main text).
Figs.~\ref{S4}b and c show measurements complementary to the one presented in Fig.~3 of the main text, recorded for DQD \#2 shown in Fig.~\ref{S3}b. 
The measurements show that the difference in detuning energy between  $\alpha$ and $\beta$ is independent of $B_\perp$, the energy splitting measures $\Delta \varepsilon = 150 \pm 10 \,\mu$eV, which corresponds to $2\Delta_\mathrm{SO}$.
The background current originates from co-tunneling in the bias transport window (its onset is highlighted by the white dashed line), which shifts in energy with increasing $|B_\perp|$. This is due to the fact that the bias window is defined by the (forbidden) ground state transition $\ket{K'\uparrow}_e \leftrightarrow  \ket{K'\uparrow}_h$, which requires less detuning for increasing $|B_\perp|$.
The same measurement for parallel magnetic fields shows the effect of the spins being continuously canted into the BLG plane. 
The difference in detuning of the transitions $\alpha$ and $\beta$ increases while a third resonance, $\gamma$, emerges.
The data is in good qualitative and quantitative agreement with the data presented in Fig.~3 of the main text. 
Figs.~\ref{S4}d and \ref{S4}e show magneto-transport data in the single-particle blockade regime. The spin-valley blockade is not lifted under the influence of both in-plane and out-of-plane magnetic fields. 
Transport via co-tunneling is suppressed at increasing $B_\perp$ as (also in this case) the tunneling barriers turn more opaque due to magnetic confinement.

\clearpage
\subsection{Simulation of magnetotransport through an e-h DQD}
We simulate transport within the DQD bias triangles and along the detuning cuts by solving the rate equations for the electron and hole QD states presented in Fig.~3a following the approach used in Ref.~\cite{Knothe2022Apr}. The energy of the respective electron and hole states is given by 


\begin{align}
    \mathrm{H_{e}} &= \frac{1}{2} \Delta_\text{SO} \tau_z s_z + \frac{1}{2} g_\text{s} \mu_\text{B} \mathbf{B} \cdot \mathbf{s} + \frac{1}{2} g_\text{v} \mu_\text{B} \mathrm{B}_z \tau_z  \\
    \mathrm{H_h} &=  - \mathrm{H_{e}},
\end{align}
with the spin and valley g-factors $g_\text{s} = 2 $ and $g_\text{v} = 15$,
the Bohr magneton $\mu_\text{B}$, the proximity enhanced (intrinsic) Kane-Mele spin-orbit coupling $\Delta_\text{SO} = 70\, \mu$eV and the Pauli matrices $s_i$ and $\tau_i$ which act on spin and valley, respectively. We approximate the effect of the right (R) and left (L) finger gate on the charging energy of the system by 
\begin{equation}
    E_\text{c} (N_\text{R}, N_\text{L}) = e N_\text{R}  V_\text{R} + e N_\text{L}  V_\text{L} \,,
\end{equation}
with the absolute value of the elementary charge, $e$, the QD occupation number $N_\text{L} = -1$ (1h),  $N_\text{R} = 1$ (1e) and the gate voltages $V_\text{R}$ and $V_\text{L}$. For describing transport through the e-h DQD, we focus on the $(0,0) \rightarrow (-1,1) \rightarrow (-1, 0) \rightarrow (0,0)$ charge cycle and only consider sequential tunneling. There are in total 25 possible states of the system $\chi$ = (hole QD state, electron QD state) with
\begin{align}
    \chi &= (\overline{\phi_\text{h}}\, , \overline{\psi_\text{e}}) \\
    \overline{\phi_\text{h}}\, , \overline{\psi_\text{e}}  &\in \{0,K{\uparrow},K{\downarrow},K'{\uparrow},K'{\downarrow} \}.  
\end{align}
%
Here, $ \overline{\phi_\text{h}}\, , \overline{\psi_\text{e}}$ describe the state of the left and right QD, which includes the four single particle states, as well as the QD being empty.


We assume no mixing between lead and QD states and equal tunnel probabilities to and from the leads for all states, $\gamma^\text{L,R} = 1.7$ GHz. Thus, we obtain the transition rates between QD states involving tunneling processes from the leads (L,R) by computing
\begin{align}
W^\text{L,R}_{\chi \leftarrow \chi^{\prime}} &= \gamma^\text{L,R} \,  f(E_{\chi} - E_{\chi^{\prime}} - \mu^\text{L,R}),
\end{align}
with the Fermi-function, $f$, at $T = 0.1\,$K, and the electron and hole QD states $\phi_\text{h}, \psi_\text{e}$. Note that hole states only tunnel to the left lead and electron states only tunnel to the right lead. 

For interdot transitions, we assume no mixing of electron and hole states due to the small interdot tunnel coupling. For simplicity, relaxation is neglected. We obtain the rates of the interdot transition by computing 
\begin{align}
W^\text{inter}_{(0,0) \leftarrow (\phi_\text{h}, \psi_\text{e})} &=  W^\text{inter}_{(\phi_\text{h}, \psi_\text{e}) \leftarrow (0,0) }= G^\text{inter} \, \braket{\phi_\text{h} | \psi_\text{e}}  \frac{1}{\sqrt{2 \pi \sigma}} \mathrm{exp}\left(  {-\frac{(E_{(0,0)} - E_{(\phi_\text{h}, \psi_\text{e})})^2}{4 \sigma^2}} \right),
\label{align:interdot}
\end{align} 
 with the interdot tunnel rate $\gamma^\text{inter} = 6$ kHz and $\phi_\text{h}, \psi_\text{e} \in \{ K{\uparrow},K{\downarrow},K'{\uparrow},K'{\downarrow} \}$. The Gaussian energy smearing models the experimentally observed peaks with an estimated width of the resonances $\Gamma = 40 \,\mu$eV. We expect that this smearing originates from voltage fluctuations of the finger gates. The overlap between electron and hole states is given by $\braket{\phi_\text{h} | \psi_\text{e}} = (\sigma_y \tau_x s_y)_{\phi_\text{h}, \psi_\text{e}}$ in order to assure that only electrons and holes with opposite quantum numbers are created (or annihilated). With equation (\ref{align:interdot}) we implicitly assume that the states in the left and right QD have no coherent phase relation. 

We solve the master equation of the probabilities, $\mathrm{P}_{\chi}$, for the system to be in state $\chi$,
\begin{equation}
\dot{\mathrm{P}}_{\chi} =\sum_{\chi'} (W_{ \chi \leftarrow \chi'} \, \dot{\mathrm{P}}_{\chi'}  - W_{ \chi' \leftarrow \chi} \, \dot{\mathrm{P}}_{\chi} ) ,
\label{eqn:rateeqn}
\end{equation}

in the stationary limit, $\dot{\mathrm{P}}_{\chi} = 0$, normalizing the probabilities to $\sum_{\chi} \mathrm{P}_{\chi} = 1$.
In the stationary limit, we can compute the current through the double QD by computing the current flow from the right QD to lead R:
\begin{equation} 
 I^\text{R}= e \sum_{\phi_\text{h}, \psi_\text{e} } \left( W^\text{R}_{(\phi_\text{h}, 0) \leftarrow (\phi_\text{h}, \psi_\text{e})} \dot{\mathrm{P}}_{(\phi_\text{h}, \psi_\text{e})}
-  W^\text{R}_{(\phi_\text{h}, \psi_\text{e}) \leftarrow (\phi_\text{h}, 0) } \dot{\mathrm{P}}_{(\phi_\text{h}, 0)} \right).
 \label{eqn:Iseq}
\end{equation}

We follow this procedure for different magnetic fields and different gate voltage combinations $V_\text{L}$, $V_\text{R}$. The result is shown in Fig.~\ref{S6}, where we are able to reproduce the experimental data of Figs.~2b-d and Figs.~2f-h. Additionally, we simulate the current along the detuning axis of the $(0h,0e)\rightarrow(1h,1e)$ triple point as a function of parallel magnetic field, which is presented in Fig.~3e of the main manuscript.



\begin{figure}[!thb]
\centering
\includegraphics[draft=false,keepaspectratio=true,clip,width=\linewidth]{FigS6A.pdf}
\caption[S6]{  
Charge stability diagrams of the first triple point simulated by solving the rate equation. 
\textbf{a - c} depict the forward bias direction ($V_\mathrm{SD}=1$~mV) for different magnetic fields, showing the same features as the experimental data presented in Fig.~2. 
\textbf{d - f} show the blocked bias direction ($V_\mathrm{SD}=-1$~mV) for the same magnetic fields. For zero magnetic field, the blockade is lifted at the corners of the bias triangle, where back and forth tunneling to source (or drain) allows lifting the blockade. The effect is even larger at finite parallel magnetic fields, where the spins are tilted into the plane of the BLG.}
\label{S6}
\end{figure}



\clearpage
\subsection{Electron-hole symmetry breaking due to Rashba spin-orbit coupling}
Since we are explicitly breaking the inversion symmetry of BLG with a perpendicular electric field, extrinsic (Rashba) spin-orbit coupling poses an additional mechanism to break the electron-hole symmetry in our DQD system. The corresponding full spin-orbit Hamiltonian acting on the low energy bands is then given by [6]
\begin{equation*}
\begin{split}
H_\mathrm{SO} =  \Psi^\dag \Big( & \frac{1}{4} [(\Delta^\text{t}_\mathrm{SO} + \Delta^\text{b}_\mathrm{SO}) \sigma_z - (\Delta^\text{t}_\mathrm{SO} -  \Delta^\text{b}_\mathrm{SO}) \sigma_0 ] \tau_z s_z  
 + \frac{1}{2} \lambda_\text{ex} (\sigma_y s_x + i \tau_z \sigma_x s_y ) \Big)  \Psi ,   
\end{split}
\end{equation*}
with the Pauli matrices $\tau, \sigma, s$ as defined in the main text, the extrinsic (Rashba) SO coupling  $\lambda_\text{ex}$, which scales linearly with the applied electric displacement field, and the proximity enhanced intrinsic (Kane-Mele) spin-orbit coupling energies $\Delta^\text{t}_\mathrm{SO}$ and $\Delta^\text{b}_\mathrm{SO}$ for the top and bottom layer of the BLG \footnote{$\Delta^\text{t}_\mathrm{SO}$, $\Delta^b_\mathrm{SO}$ and $\lambda_\text{ex}$ correspond to $\lambda_{I1}$ and $\lambda'_{I1}$ and $\lambda_3$ in Ref. \cite{Konschuh2012Mar}} \cite{Konschuh2012Mar}.
%
The influence of the proximity enhanced Kane-Mele spin-orbit coupling on electron-hole symmetry is discussed in the main text.

For understanding the influence of the extrinsic (Rashba) term, we note that for Fermi energies close to the band edge, the sublattice space is equivalent to the layer space and therefore to conduction and valence band. 
%
This is caused by the fact that excess charge is strongly layer polarized, only leading to a small admixture of the sublattices~\cite{McCann2013Apr, Banszerus2020May}.
%
The extrinsic SO term couples the two sublattices via $\sigma_{x,y}$ and therefore to the two layers, which experience a potential difference due to the electric displacement field.
%
As a consequence, the extrinsic spin-orbit term is suppressed to first order by~$\lambda_\mathrm{ex}^2/E_\mathrm{g}^2$.
%
Theoretical predictions of $\lambda_\text{ex}$ are at least three orders of magnitude smaller than the band gap ($E_\mathrm{g}$), rendering extrinsic spin-orbit coupling irrelevant for our system ~\cite{Konschuh2012Mar, Banszerus2021Sep}.


\clearpage
\subsection{Electron-hole symmetry breaking due to different valley g-factors in the electron and hole QDs}

We investigate how asymmetric valley g-factors would affect the transition spectrum of the e-h DQD. In Fig.~\ref{S7}a-d we simulate the current through the device as a function of the detuning energy $\widetilde \varepsilon$ and perpendicular magnetic field, $B_\perp$, for different combinations of valley g-factors in the hole and electron QD, respectively. 
As clearly visible in Figs.~S7a,b, both the $\alpha$ and $\beta$ transition split due to the difference in valley g-factors (see colored lines in Fig.~\ref{S7}a) by $\Delta E = \frac{1}{2} \mu_\text{B} |g^\text{e}_\text{v}-g^\text{h}_\text{v}| B_\perp$. For equal valley g-factors, $\alpha$ and $\beta$ do not show any $B_\perp$-dependence, as shown in Fig.~S7c. A tiny asymmetry in valley g-factors is allowed without significantly changing the observed features for magnetic fields below 1T, as shown in Fig.~S7d, where a g-factor asymmetry of 0.1 is assumed.


\begin{figure}[!thb]
\centering
\includegraphics[draft=false,keepaspectratio=true,clip,width=0.8\linewidth]{FigS7A.pdf}
\caption[Fig07]{Calculation of the current through the device as a function of the detuning energy $\widetilde \varepsilon$ (see arrow in Fig.~2c of the main text) and perpendicular magnetic field at a finite bias of $V_\mathrm{SD} = 1~$mV.
In \textbf{a}, the valley g-factors of the two QDs are chosen asymmetrically ($g^\text{e}_\text{v}=15$ for the electron QD and $g^\text{h}_\text{v}=20$ for the hole QD), resulting in a splitting of both, the $\alpha$ and $\beta$ transition, which scales with the difference in the valley g-factors. In \textbf{b}, the valley g-factors of the two QDs are chosen less asymmetrically ($g^\text{e}_\text{v}=15$ for the electron QD and $g^\text{h}_\text{v}=17$ for the hole QD), resulting in a smaller splitting of both, the $\alpha$ and $\beta$ transition, which scales with the difference in the valley g-factors. In \textbf{c} the valley g-factors are chosen symmetrically ($g_\text{v}=15$), and no dependence on $B_\perp$ is observed. In \textbf{d}, the experimentally observed g-factor difference of $g^\text{e}_\text{v}=15$ and $g^\text{h}_\text{v}=15.1$ is used for the simulation.
}
\label{S7}
\end{figure}


To quantitatively estimate the valley g-factor asymmetry, we fit Gaussian peaks with width, $\Gamma$, to the detuning cuts presented in Fig.~3b in the main manuscript, allowing for a constant background and assuming equal width for both peaks, i.e. the $\alpha$ and $\beta$ peak. Such a fit is exemplarily shown in Fig.~S8a. The fitted width of the two peaks increases slightly for increasing $B_\perp$, as shown in Fig~\ref{S8}b. Attributing this effect entirely to a difference of the electron and hole g-factors, we obtain a maximum g-factor difference of $g_\text{v} \approx 0.1$ (c.f. with Fig.~S7d). 


\begin{figure}[!thb]
\centering
\includegraphics[draft=false,keepaspectratio=true,clip,width=0.85\linewidth]{FigS8A.pdf}
\caption[Fig08]{\textbf{a} Exemplary line trace of the tunneling current as a function of the detuning. The sum of two Gauss curves with width $\Gamma$ is fitted to the data (see dashed line). \textbf{b}  $\Gamma$ extracted from the line fits as shown in a, as a function of $B_\perp$. Attributing the linear broadening of $\alpha$ and $\beta$ to an asymmetry of valley g-factors between electron and hole QD yields $\Delta g \approx 0.11$ .
}
\label{S8}
\end{figure}

% \bibliography{Literature}

%merlin.mbs apsrev4-1.bst 2010-07-25 4.21a (PWD, AO, DPC) hacked
%Control: key (0)
%Control: author (8) initials jnrlst
%Control: editor formatted (1) identically to author
%Control: production of article title (-1) disabled
%Control: page (0) single
%Control: year (1) truncated
%Control: production of eprint (0) enabled
\begin{thebibliography}{6}%
\makeatletter
\providecommand \@ifxundefined [1]{%
 \@ifx{#1\undefined}
}%
\providecommand \@ifnum [1]{%
 \ifnum #1\expandafter \@firstoftwo
 \else \expandafter \@secondoftwo
 \fi
}%
\providecommand \@ifx [1]{%
 \ifx #1\expandafter \@firstoftwo
 \else \expandafter \@secondoftwo
 \fi
}%
\providecommand \natexlab [1]{#1}%
\providecommand \enquote  [1]{``#1''}%
\providecommand \bibnamefont  [1]{#1}%
\providecommand \bibfnamefont [1]{#1}%
\providecommand \citenamefont [1]{#1}%
\providecommand \href@noop [0]{\@secondoftwo}%
\providecommand \href [0]{\begingroup \@sanitize@url \@href}%
\providecommand \@href[1]{\@@startlink{#1}\@@href}%
\providecommand \@@href[1]{\endgroup#1\@@endlink}%
\providecommand \@sanitize@url [0]{\catcode `\\12\catcode `\$12\catcode
  `\&12\catcode `\#12\catcode `\^12\catcode `\_12\catcode `\%12\relax}%
\providecommand \@@startlink[1]{}%
\providecommand \@@endlink[0]{}%
\providecommand \url  [0]{\begingroup\@sanitize@url \@url }%
\providecommand \@url [1]{\endgroup\@href {#1}{\urlprefix }}%
\providecommand \urlprefix  [0]{URL }%
\providecommand \Eprint [0]{\href }%
\providecommand \doibase [0]{http://dx.doi.org/}%
\providecommand \selectlanguage [0]{\@gobble}%
\providecommand \bibinfo  [0]{\@secondoftwo}%
\providecommand \bibfield  [0]{\@secondoftwo}%
\providecommand \translation [1]{[#1]}%
\providecommand \BibitemOpen [0]{}%
\providecommand \bibitemStop [0]{}%
\providecommand \bibitemNoStop [0]{.\EOS\space}%
\providecommand \EOS [0]{\spacefactor3000\relax}%
\providecommand \BibitemShut  [1]{\csname bibitem#1\endcsname}%
\let\auto@bib@innerbib\@empty
%</preamble>
\bibitem [{\citenamefont {Banszerus}\ \emph {et~al.}(2021)\citenamefont
  {Banszerus}, \citenamefont {M{\ifmmode\ddot{o}\else\"{o}\fi}ller},
  \citenamefont {Steiner}, \citenamefont {Icking}, \citenamefont {Trellenkamp},
  \citenamefont {Lentz}, \citenamefont {Watanabe}, \citenamefont {Taniguchi},
  \citenamefont {Volk},\ and\ \citenamefont {Stampfer}}]{Banszerus2021Sep}%
  \BibitemOpen
  \bibfield  {author} {\bibinfo {author} {\bibfnamefont {L.}~\bibnamefont
  {Banszerus}}, \bibinfo {author} {\bibfnamefont {S.}~\bibnamefont
  {M{\ifmmode\ddot{o}\else\"{o}\fi}ller}}, \bibinfo {author} {\bibfnamefont
  {C.}~\bibnamefont {Steiner}}, \bibinfo {author} {\bibfnamefont
  {E.}~\bibnamefont {Icking}}, \bibinfo {author} {\bibfnamefont
  {S.}~\bibnamefont {Trellenkamp}}, \bibinfo {author} {\bibfnamefont
  {F.}~\bibnamefont {Lentz}}, \bibinfo {author} {\bibfnamefont
  {K.}~\bibnamefont {Watanabe}}, \bibinfo {author} {\bibfnamefont
  {T.}~\bibnamefont {Taniguchi}}, \bibinfo {author} {\bibfnamefont
  {C.}~\bibnamefont {Volk}}, \ and\ \bibinfo {author} {\bibfnamefont
  {C.}~\bibnamefont {Stampfer}},\ }\href {\doibase 10.1038/s41467-021-25498-3}
  {\bibfield  {journal} {\bibinfo  {journal} {Nat. Commun.}\ }\textbf {\bibinfo
  {volume} {12}},\ \bibinfo {pages} {5250} (\bibinfo {year}
  {2021})}\BibitemShut {NoStop}%
\bibitem [{\citenamefont {Knothe}\ \emph {et~al.}(2022)\citenamefont {Knothe},
  \citenamefont {Glazman},\ and\ \citenamefont {Fal{'}ko}}]{Knothe2022Apr}%
  \BibitemOpen
  \bibfield  {author} {\bibinfo {author} {\bibfnamefont {A.}~\bibnamefont
  {Knothe}}, \bibinfo {author} {\bibfnamefont {L.~I.}\ \bibnamefont {Glazman}},
  \ and\ \bibinfo {author} {\bibfnamefont {V.~I.}\ \bibnamefont {Fal{'}ko}},\
  }\href {\doibase 10.1088/1367-2630/ac5d00} {\bibfield  {journal} {\bibinfo
  {journal} {New J. Phys.}\ }\textbf {\bibinfo {volume} {24}},\ \bibinfo
  {pages} {043003} (\bibinfo {year} {2022})}\BibitemShut {NoStop}%
\bibitem [{Note1()}]{Note1}%
  \BibitemOpen
  \bibinfo {note} {$\Delta ^\protect \text {t}_\protect \mathrm {SO}$, $\Delta
  ^b_\protect \mathrm {SO}$ and $\lambda _\protect \text {ex}$ correspond to
  $\lambda _{I1}$ and $\lambda '_{I1}$ and $\lambda _3$ in Ref. \cite
  {Konschuh2012Mar}}\BibitemShut {NoStop}%
\bibitem [{\citenamefont {Konschuh}\ \emph {et~al.}(2012)\citenamefont
  {Konschuh}, \citenamefont {Gmitra}, \citenamefont {Kochan},\ and\
  \citenamefont {Fabian}}]{Konschuh2012Mar}%
  \BibitemOpen
  \bibfield  {author} {\bibinfo {author} {\bibfnamefont {S.}~\bibnamefont
  {Konschuh}}, \bibinfo {author} {\bibfnamefont {M.}~\bibnamefont {Gmitra}},
  \bibinfo {author} {\bibfnamefont {D.}~\bibnamefont {Kochan}}, \ and\ \bibinfo
  {author} {\bibfnamefont {J.}~\bibnamefont {Fabian}},\ }\href {\doibase
  10.1103/PhysRevB.85.115423} {\bibfield  {journal} {\bibinfo  {journal} {Phys.
  Rev. B}\ }\textbf {\bibinfo {volume} {85}},\ \bibinfo {pages} {115423}
  (\bibinfo {year} {2012})}\BibitemShut {NoStop}%
\bibitem [{\citenamefont {McCann}\ and\ \citenamefont
  {Koshino}(2013)}]{McCann2013Apr}%
  \BibitemOpen
  \bibfield  {author} {\bibinfo {author} {\bibfnamefont {E.}~\bibnamefont
  {McCann}}\ and\ \bibinfo {author} {\bibfnamefont {M.}~\bibnamefont
  {Koshino}},\ }\href {\doibase 10.1088/0034-4885/76/5/056503} {\bibfield
  {journal} {\bibinfo  {journal} {Rep. Prog. Phys.}\ }\textbf {\bibinfo
  {volume} {76}},\ \bibinfo {pages} {056503} (\bibinfo {year}
  {2013})}\BibitemShut {NoStop}%
\bibitem [{\citenamefont {Banszerus}\ \emph {et~al.}(2020)\citenamefont
  {Banszerus}, \citenamefont {Frohn}, \citenamefont {Fabian}, \citenamefont
  {Somanchi}, \citenamefont {Epping}, \citenamefont
  {M{\ifmmode\ddot{u}\else\"{u}\fi}ller}, \citenamefont {Neumaier},
  \citenamefont {Watanabe}, \citenamefont {Taniguchi}, \citenamefont {Libisch},
  \citenamefont {Beschoten}, \citenamefont {Hassler},\ and\ \citenamefont
  {Stampfer}}]{Banszerus2020May}%
  \BibitemOpen
  \bibfield  {author} {\bibinfo {author} {\bibfnamefont {L.}~\bibnamefont
  {Banszerus}}, \bibinfo {author} {\bibfnamefont {B.}~\bibnamefont {Frohn}},
  \bibinfo {author} {\bibfnamefont {T.}~\bibnamefont {Fabian}}, \bibinfo
  {author} {\bibfnamefont {S.}~\bibnamefont {Somanchi}}, \bibinfo {author}
  {\bibfnamefont {A.}~\bibnamefont {Epping}}, \bibinfo {author} {\bibfnamefont
  {M.}~\bibnamefont {M{\ifmmode\ddot{u}\else\"{u}\fi}ller}}, \bibinfo {author}
  {\bibfnamefont {D.}~\bibnamefont {Neumaier}}, \bibinfo {author}
  {\bibfnamefont {K.}~\bibnamefont {Watanabe}}, \bibinfo {author}
  {\bibfnamefont {T.}~\bibnamefont {Taniguchi}}, \bibinfo {author}
  {\bibfnamefont {F.}~\bibnamefont {Libisch}}, \bibinfo {author} {\bibfnamefont
  {B.}~\bibnamefont {Beschoten}}, \bibinfo {author} {\bibfnamefont
  {F.}~\bibnamefont {Hassler}}, \ and\ \bibinfo {author} {\bibfnamefont
  {C.}~\bibnamefont {Stampfer}},\ }\href {\doibase
  10.1103/PhysRevLett.124.177701} {\bibfield  {journal} {\bibinfo  {journal}
  {Phys. Rev. Lett.}\ }\textbf {\bibinfo {volume} {124}},\ \bibinfo {pages}
  {177701} (\bibinfo {year} {2020})}\BibitemShut {NoStop}%
\end{thebibliography}%


\end{document}



\end{document}