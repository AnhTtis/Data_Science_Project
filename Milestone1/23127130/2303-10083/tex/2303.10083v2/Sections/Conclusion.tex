


\section{Conclusion}

We present \name, a grid-based surface representation with decoupled geometry, opacity, and appearance. We develop closed-form intersection finding and differentiable alpha compositing to optimize the surface via photometric loss. 
Our representation utilizes initialization from efficient Plenoxels \cite{plenoxels}, and incorporates truncated rendering and additional surface regularizations to reconstruct high-quality surfaces for translucent objects and thin structures with heavy blending effects.
% We believe this surface representation can also play a role in efficient rendering and robust few-shot scene reconstruction, and hope our work delights a path toward more research in the future.
% \GR{re-write last sentence without believe and hope. End with a smile.} \WW{Should we just remove the last sentence?}
% \GR{Maybe combine the limitations paragraph and the conclusion, then the conclusion can end with the outlook from the limitations}\WW{I'll try, the limitation is a bit long, but I'll see what's best}

% Our representation can fully utilize the information in a pre-trained NeRF model by initializing surface geometry, opacity, and appearance from it. Further optimization with respect to the photometric loss and additional surface regularization allow our approach to reconstruct high-quality surfaces for semi-transparent objects and thin structures with heavy blending effects. We believe this surface representation can also play a role in efficient rendering and robust few-shot scene reconstruction, and hope our work delights a path toward more research in the future.

% \newpage

\noindent
\textbf{Limitation}: 
% We only evaluated our method by initializing from Plenoxels, which provide a less noisy density field compared to MLP-based NeRFs such as MipNeRF360. Different hyperparameters might be needed to properly utilize initialization from different NeRF methods due to varying levels of density magnitude and noises. Moreover, 
Compared to MLP-based SDF methods such as NeuS~\cite{neus} and neuralangelo~\cite{neuralangelo}, our reconstructed surface tends to be less smooth due to the lack of spatial smoothness encoded in the MLP; see Figure~\ref{fig:main_syn}. It presents a trade-off: stronger surface regularization can certainly give smoother surfaces, but can also destroy delicate thin structures in the reconstruction. 
\ww{In addition, we focus only on decoupling geometry and material ambiguity in existing volumetric representation, and do not specifically handle strong reflections as in Ref-NeRF~\cite{refnerf}. 
% Real-world reconstruction of highly specular translucent surfaces such as the ``yellow cup" scene therefore still presents some artifacts. The quality can be further improved by incorporating reflection rendering similar to~\cite{refnerf}.
}

