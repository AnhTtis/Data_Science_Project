\documentclass[11pt]{article}

%\usepackage{setspace}
%\doublespacing
\usepackage{cite}
\usepackage{url}
\usepackage{graphicx}
\usepackage{hyperref}
\usepackage{psfrag}
\usepackage{stfloats}
\usepackage{amsfonts,amssymb,amsmath,bm,paralist,theorem,cite,ifthen,color,wrapfig}
%\usepackage{amsthm}

\usepackage{array}
\usepackage{caption}
\usepackage{calc}
% \usepackage{subfigure}
\usepackage{subcaption}
\usepackage{fullpage}
\captionsetup[figure]{labelfont={small,rm,bf},labelsep=period,font={small,rm}}
\captionsetup[table]{labelfont={rm,bf},labelsep=period,belowskip=3pt}
\theorembodyfont{\rmfamily}
%\newcommand{\figurename}{\textbf{Figure}}
%

\usepackage{pifont} 
\newcommand{\cmark}{\ding{51}} 
\newcommand{\xmark}{\ding{55}} 

\newenvironment{Ventry}[1]%
{\begin{list}{}{
    \renewcommand{\makelabel}[1]{\mbox{\textnormal{##1}}\hfil}%
    \settowidth{\labelwidth}{\mbox{\textnormal{#1}}}%
    %\setlength{\itemsep}{-.4\baselineskip}%
    \setlength{\leftmargin}{\labelwidth+\labelsep}}}%
    %\setlength
{\end{list}}

\newtheorem{Lemma}{Lemma}
\newtheorem{Prop}{Proposition}
\newtheorem{Theorem}{Theorem}
\newtheorem{Def}{Definition}
\newtheorem{Corollary}{Corollary}
\newtheorem{Property}{Property}
\theorembodyfont{\rmfamily}
\newtheorem{Exa}{Example}
\newtheorem{Rmk}{Remark}
\newtheorem{assumption}{Assumption}
\newtheorem{Example}{Example}


\begin{document}

\bibliographystyle{IEEEtran}
%-------------
\begin{center}
{\large {\bf 
Learning An Active Inference Model of Driver Perception and Control:
Application to Vehicle Car-Following}} \\
\medskip by \\

\medskip
by Ran Wei,
Alfredo Garcia,
Anthony McDonald,
Gustav Markkula,
Johan Engstrom,
Matthew O'Kelly
\\
\medskip
{\em Second Revision}, \today

\end{center}


We thank the reviewers and the editorial team for their careful reading of our manuscript and
for the valuable comments and suggestions. 

\medskip
In response to reviewer 1's comments:
\begin{enumerate}
    \item We have added a paragraph to clarify the relationship between Time-to-Collision and $1/\tau$.
    \item Added desired speed in the description of IDM model.
    \item We estimated the IDM parameters by initializing the desired speed to 10 m/s. The new fitted parameters are very close to the previously reported ones, except for the minimum distance $d_{0}$ and maximum acceleration $a_{max}$. This does not significantly alter the largely poor performance of the IDM model. 
\end{enumerate}
\newpage



\section{Response to Reviewer 1's Comments}

\medskip
{\bf General Comments}
\medskip
\begin{enumerate}
\item { \color{blue} I would especially like to see the authors double-check the IDM implementation and fitting as there may be a bug in its initialization (see below). I recommend minor revisions, but this is contingent on the IDM fitting being verified.

\begin{quote}
{\em 
``20. Page 26, line 46: $\tau$ is invariant to the width of the vehicle (i.e. it has the same value regardless
of the lead vehicle width).
Response: As explained in text, $\tau$ is not time-to-collision but a quantity referred to as
“looming” by a recent line of car following models (e.g., [9]) which computes the rate of
change of the visual angle divided by the visual angle itself. Since the visual angle depends
on the width of the lead vehicle, it is not invariant to it. However, lead vehicle width does
have little effect on this quantity compared to time-to-collision. As explained in the response
to comment 19, this feature was chosen mainly to be consistent with recent related work on
car following models."}
\end{quote}
The original point of the tau theory (or the tau braking model) is that tau is not dependent on the object size (or any other allocentric variable), and with the small angle approximation (which applies here well) the angular extent divided by the angular expansion gives the time-to-collision (see equation 11 in \href{https://journals.sagepub.com/doi/10.1068/p050437}{``A Theory of Visual Control of Braking Based on Information about Time-to-Collision"}. This is also mentioned in [9] in section 2.4.

Misconceptions about tau vs time-to-contact is sometimes made in the literature but I'd like to see them not proliferating so the (practical) equivalence of tau and time-to-contact should be mentioned.

Technically adding tau (or usually 1/tau) is perfectly reasonable as it is a good feature for controlling speed in car following scenarios and behaves quite linearly with reasonable acceleration outputs. I would emphasis this rationale for justification of including it in the model input.}

{\bf Response:} Thank you for highlighting this. We agree with your comment and have updated section B.1 correspondingly (see text highlighted in {\color{blue} blue}). 

\item {\color{blue} Comment 23:
\begin{quote}
{\em ``There was previously an implementation mistake in sampling from the fitted Gaussian policy. We have corrected the mistake by discarding the variance during testing and only use the mean prediction so the resulting control rule is deterministic."}
\end{quote}

Is this fix included in the Github source code? It's latest commit is from two years ago.}

{\bf Response:} We have not updated the Github source code yet. We will update it after the review process with all incurred modifications. 


\item {\color{blue} 
I also noticed now that IDM parameterization is described as:``The IDM has the following parameters: $a_{max}$ the maximum acceleration rate which can be implemented by the driver, $d_0$ the minimum allowable distance headway, $\tau$ the desired headway time, and $b$ the maximum deceleration rate."

However, IDM has five parameters (the desired speed $v_0$ and the minimum distance $d_0$) and there are rightly five parameters ($v_0$, $\tau$, s, a, b) in the source code, and if I understand correctly these are all picked up by the BehaviorCloning class, and initialized from the default values given in the IDM Agent class? Table II also gives IDM (correctly) 6 parameters.}

{\bf Response:} Thank you for highlighting this. We missed $\Tilde{v}$ the desired speed and have added it to the description.


\item {\color{blue} The initial value for $v_0$ is 120. The documentation says this is in km/h, but I don't see a conversion from $km/h$ to $m/s$ required by the computation in {\tt compute\_action\_dist}, so it seems like it is actually treated as m/s, which would make the initial value for the speed huge at 431 km/h. A high desired speed makes the 
I suspect such initial value may lead to a bad local optimum in the fit. What were the typical fitted parameter values for the IDM and do they change if the initial value is something reasonable (e.g. the speed limit of the road the data is from)?}

{\bf Response:} Thank you for highlighting this issue. We first examined the learned parameters from the previous experiments (see table below). The learned desired speed was on average 12.2 m/s (44 km/h) with standard deviation $0.2$ m/s. The standard deviations of all other parameters are also very small. 

We repeated the IDM fitting process as you suggested by initializing the desired speed to 10 m/s, which is close to the mean dataset speed ($\approx$ 7.8 m/s). Due to some corruptions in the file from 2 years ago, we could only fit the model to a subset of the data (the first one of the 3 processed data files) and we did so for 1 seed. The new fitted parameters are shown in the table below. We see that the parameter values are very close, except for the minimum distance $d_{0}$ and maximum acceleration $a_{max}$. This does not significantly alter the largely poor performance of the IDM model. 

\begin{table}[!htb]
\caption{Learned IDM parameters. The first row is the mean and standard deviations of the parameters from the previous experiments. The second row is fitted parameters of the new experiments where we initialize the desired speed to 10 m/s.}\label{eq:hyperparams}
\centering
\begin{tabular}{ccccccc}
\hline
& $\Tilde{v}$ & $\tau$ & $d_{0}$ & $a_{max}$ & $b$ & $\sigma$ \\
\hline
Old & 12.2 $\pm$ 0.2 & 0.83 $\pm$ 0.03 & 1.07 $\pm$ 0.07 & 0.21 $\pm$ 0.006 & 2.68 $\pm$ 0.19 & 0.46 $\pm$ 0.004 \\
New & 12.31 & 0.76 & 6.38 & 0.05 & 1.83 & 0.47 \\
\hline
\end{tabular}
\end{table}

\end{enumerate}

\newpage

%\bibliography{PaperBIB,ref,main,example}
\end{document}