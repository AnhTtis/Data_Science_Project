

\section{D{\'e}tails sur le changement de variables symplectique pour l'Oscillateur Harmonique ({\`a} retirer de l'article apr{\`e}s relecture)}
\label{sect: appendix HO symplectic change of coordinates}


On a 
\begin{equation*}
    \frac{\textd }{\textd t} \begin{pmatrix}
        k\\B
    \end{pmatrix} = 
    \begin{pmatrix}
        0 & -I \\ I & 0
    \end{pmatrix} \nabla_{(k, B)} \mathcal{E} = J^{-1} \nabla_{(k, B)} \mathcal{E}
,\end{equation*}
%
avec la matrice symplectique usuelle 
\begin{equation*}
    J = \begin{pmatrix}
        0 & I \\ -I & 0
    \end{pmatrix}    
.\end{equation*}
%
Dans les notations de \cite{hairerGeometricNumericalIntegration2006}, on aurait
\begin{equation*}
    \frac{\textd }{\textd t} \begin{pmatrix}
        p\\q
    \end{pmatrix} = J^{-1} \nabla_{(p, q)} \mathcal{E}
,\end{equation*}
%
avec \( p \equiv k \) et \( q \equiv B \). Les {\'e}quations que \cite{hairerGeometricNumericalIntegration2006} donne pour avoir un changement de variables \( (p, q)\mapsto (P, Q) \) symplectique sont les suivantes :
\begin{equation*}
    P = \frac{\partial S}{\partial Q}(q, Q), \qquad p = - \frac{\partial S}{\partial q}(q, Q) 
.\end{equation*}
%
Cela voudrait dire qu'on peut exprimer \( k \) en fonction de \( (B, Q)  \), avec \( Q \) une variable qu'il reste {\`a} d{\'e}terminer. Bien que cela soit a priori faisable, ce n'est pas le plus simple et on voudrait plutôt exprimer \( B \) en fonction de \( k \) et d'une nouvelle variable. Autrement dit, on voudrait plutôt avoir les {\'e}quations
\begin{equation*}
    Q = \frac{\partial S}{\partial P}(p, P), \qquad q = - \frac{\partial S}{\partial p}(p, P) 
.\end{equation*}
%

Pour avoir ces {\'e}quations, on a besoin de reprendre \cite[Sect.~VI.5,~Eqn~(5.1)]{hairerGeometricNumericalIntegration2006} et d'avoir le r{\'e}sultat suivant : le changement de coordonn{\'e}es \( (p,q) \overset{\Phi}{\mapsto} (P, Q) \) est symplectique ssi il existe une fonction \( S(p, q) \) telle que
\begin{equation}
    \label{eqn: appendix change of coords HO -- 5.1bis}
    Q^T dP - q^T dp = dS
.\end{equation}
%

C'est sûrement visible si on est quelqu'un de malin, ce n'est pas mon cas donc je vais reprendre la preuve de \cite{hairerGeometricNumericalIntegration2006} et l'adapter.

\begin{proof}
    On {\'e}crit le jacobien de \( \Phi \):
    \begin{equation*}
       \frac{\partial (P, Q)}{\partial (p, q)} = 
       \begin{pmatrix}
        P_p & P_q \\ Q_p & Q_q
       \end{pmatrix}  
    .\end{equation*}
    %
    On rappelle la condition de symplecticit{\'e} pour \( \Phi \):
    \begin{align*}
        J &= \left( \frac{\partial \Phi}{\partial (p, q)}  \right)^T J \left( \frac{\partial \Phi}{\partial (p, q)}  \right) \\
        &= \begin{pmatrix}
                P_p & P_q \\ Q_p & Q_q
            \end{pmatrix}^T J \begin{pmatrix}
                P_p & P_q \\ Q_p & Q_q
            \end{pmatrix}  \\
        &= \begin{pmatrix}
                P_p^T & Q_p^T \\ P_q^T & Q_q^T
            \end{pmatrix}^T J \begin{pmatrix}
                P_p & P_q \\ Q_p & Q_q
            \end{pmatrix}  \\
        \begin{pmatrix}
        0 & I \\ - I & 0
        \end{pmatrix}
        &= \begin{pmatrix}
            P_p^T Q_p - Q_p^T P_p & P_p^T Q_q - Q_p^T P_q \\
            P_q^T Q_p - Q_q^T P_p & P_q^T Q_q - Q_q^T P_q
        \end{pmatrix}
    .\end{align*}
    %
    On obtient alors les conditions suivantes 
    \begin{align*}
        P_p^T Q_p - Q_p^T P_p &= 0 \\
        P_p^T Q_q - Q_p^T P_q &= I \\
        P_q^T Q_p - Q_q^T P_p &= -I \\
        P_q^T Q_q - Q_q^T P_q &= 0
    \end{align*}
    %
    Les deuxi{\`e}me et troisi{\`e}me lignes expriment la m{\^e}me condition, donc au final on a les conditions suivantes pour que \( \Phi \) soit un changement de coordonn{\'e}es symplectique :
    \begin{equation}
        \label{eqn: appendix change of coords HO -- conditions on Phi to be symplectic}
        P_p^T Q_p - Q_p^T P_p = 0, \quad P_p^T Q_q - Q_p^T P_q = I, \quad P_q^T Q_q - Q_q^T P_q = 0
    \end{equation}
    %
    On utilise maintenant l'{\'e}galit{\'e} 
    \begin{equation}
        dP = P_p dp + P_q dq = \begin{pmatrix}
            P_p \\ P_q
        \end{pmatrix}^T \begin{pmatrix}
            dp \\ dq
        \end{pmatrix}
    \end{equation}
    %
    dans le LHS de \eqref{eqn: appendix change of coords HO -- 5.1bis}:
    \begin{align*}
        Q^T dP - q^T dp
        &= \begin{pmatrix}
            Q^T P_p, & Q^T P_q
        \end{pmatrix} \begin{pmatrix}
            dp \\ dq
        \end{pmatrix} - q^T dp  \\
        &= \begin{pmatrix}
            Q^T P_p - q^T, & Q^T P_q
        \end{pmatrix} \begin{pmatrix}
            dp & dq
        \end{pmatrix} \\
        &= \begin{pmatrix}
            P_p^T Q - q \\ P_q^T Q
        \end{pmatrix}^T \begin{pmatrix}
            dp \\ dq
        \end{pmatrix}
    \end{align*}
    %
    On souhaite appliquer le lemme d'int{\'e}grabilit{\'e}, et pour cela on a besoin de v{\'e}rifier la sym{\'e}trie du jacobien de la matrice coefficient:
    \begin{align*}
        \frac{\partial \begin{pmatrix} P_p^T Q - q \\ P_q^T Q \end{pmatrix}}{\partial (p, q)} 
        &= \begin{pmatrix}
            \frac{\partial}{\partial p} \begin{pmatrix} P_p^T Q - q \\ P_q^T Q \end{pmatrix} &
            \frac{\partial}{\partial q} \begin{pmatrix} P_p^T Q - q \\ P_q^T Q \end{pmatrix}
        \end{pmatrix} \\
        &= \begin{pmatrix}
            P_p^T Q_p + Q^T P_{pp}, & P_p^T Q_q - I + Q^T P_{pq} \\
            P_q^T Q_p + Q^T P_{pq}, & P_q^T Q_q + Q^T P_{qq}
        \end{pmatrix} \\
        &= \begin{pmatrix}
            P_p^T Q_p, & P_p^T Q_q - I \\
            P_q^T Q_p, & P_q^T Q_q
        \end{pmatrix} + Q^T \frac{\partial^2 P}{\partial^2 (p, q)} 
    \end{align*}
    %
    La Hessienne de \( P \) est sym{\'e}trique si on suppose \( \Phi \in C^2( \mathbb{R}^2, \mathbb{R}^2) \), et de plus la matrice qui apparaît dans le RHS est sym{\'e}trique grâce aux conditions \eqref{eqn: appendix change of coords HO -- conditions on Phi to be symplectic}.
    Alors par le lemme d'int{\'e}grabilit{\'e} on peut trouver une fonction \( S(p, q) \) telle que 
    \begin{align*}
        dS &= \begin{pmatrix}
            P_p^T Q - q \\ P_q^T Q
        \end{pmatrix}^T 
        \begin{pmatrix}
            dp \\ dq
        \end{pmatrix}\\
        %
        &= (Q^T P_p - q^T) dp + Q^T P_q dq \\
        &= Q^T \left( P_p dp + P_q dq \right) - q^T dp \\
        &= Q^T dP - q^T dp
    \end{align*}
    %
\end{proof}

L'{\'e}quation \eqref{eqn: appendix change of coords HO -- 5.1bis} sugg{\`e}re de prendre \( (p, P) \) comme nouvelles variables au lieu de \( (p, q) \). C'est un changement de coordonn{\'e}es bien d{\'e}fini si \( q \) peut {\^e}tre exprim{\'e} en terme des nouvelles coordonn{\'e}es \( (p, P) \), ce qui est possible par le th{\'e}or{\`e}me des fonctions implicites si \( \frac{\partial P}{\partial q}  \) est inversible. Notons \( S(p, P) \equiv S(\Phi(p, P)) \). En comparant les coefficients de \( dS = \frac{\partial S}{\partial p} dp + \frac{\partial S}{\partial P} dP  \) avec ceux de \eqref{eqn: appendix change of coords HO -- 5.1bis}, on obtient 
\begin{equation*}
    q = - \frac{\partial S}{\partial p}(p, P), \qquad Q = \frac{\partial S}{\partial P}(p, P) 
.\end{equation*}
%

En reprenant le parall{\`e}le \( (p, q) \leftrightarrow (k, B) \), cela revient {\`a} chercher les variables \( (P, Q) \) telles que 
\begin{equation*}
    B = - \frac{\partial S}{\partial k}(k, P), \qquad Q = \frac{\partial S}{\partial P}(k, P) 
.\end{equation*}
%
Puisqu'on peut facilement exprimer \( B \) en fonction de \( k \) et du Hamiltonien \( \mathcal{E} \), on prend \( P = \mathcal{E} \). Alors 
\begin{equation*}
    Q = \frac{\partial S}{\partial \mathcal{E}}(k, \mathcal{E}) 
.\end{equation*}
%


\begin{remark}
    Cependant, en prenant ces conventions de signe on n'obtient pas les bonnes variables symplectiques dans notre cas, il y a un probl{\`e}me de signe ! On va donc utiliser ce que \cite{hairerGeometricNumericalIntegration2006} appelle les \emph{mixed-variable generating functions}. Cela consiste {\`a} {\'e}crire \( d(Q^T P) = Q^T dP + P^T dQ \) et {\`a} l'injecter dans \eqref{eqn: appendix change of coords HO -- 5.1bis}:
    \begin{equation*}
        - P^T dQ - q^T dp = d(S - Q^T P) \iff P^T dQ + q^T dp = d(Q^T P - S)
    .\end{equation*}
    %
    
    Ensuite, en disant \( dS = S_Q dQ + S_p dp \), on obtient alors les {\'e}quations suivantes :
    \begin{equation*}
        P = \frac{\partial S}{\partial Q}(Q, p), \quad q = \frac{\partial S}{\partial p}(Q, p)
    .\end{equation*}
    %
    Le parall{\`e}le \( (p,q) \leftrightarrow (k, B) \) donne alors :
    \begin{align*}
        P &= \frac{\partial S}{\partial Q}(Q, k), \quad B = \frac{\partial S}{\partial k}(Q, k) \\
        \implies \phi &= \frac{\partial S}{\partial \mathcal{E}}(\mathcal{E}, k), \quad B = \frac{\partial S}{\partial k}(\mathcal{E}, k)
    ,\end{align*}
    %
    ce qui corrige le probl{\`e}me de signe !
\end{remark}
    