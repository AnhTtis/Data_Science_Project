

\section{The Dirac-Frenkel-MacLachlan principle}
\label{sect: DFMP}

In this section we consider the Schrödinger equation \eqref{eqn: cNLS with psi}.
As it has been explained before, the equation consists in two parts: the linear part \eqref{eqn: cNLS with psi -- linear part}, and the nonlinear part \eqref{eqn: cNLS with psi -- nonlinear part}.
Section \ref{sect: The Harmonic Oscillator} was dedicated to solving the Harmonic oscillator, namely the linear part.
We are interested now in solving the nonlinear part, and as it is usually done for numerical simulations, we will use a splitting method.
This will allow us to solve \eqref{eqn: cNLS with psi} by solving separately \eqref{eqn: cNLS with psi -- linear part} and \eqref{eqn: cNLS with psi -- nonlinear part}, one after the other. 
By doing so, a splitting error is made, which depends on the timestep \( \Delta t \), and the order of the error depends on the specific splitting method.
It is also possible to apply high-order splitting methods.


\vspace{1em}


We focus on approximating numerically the solution to \eqref{eqn: cNLS with psi -- nonlinear part}:
\begin{equation*}
    i \partial_{t} \psi = \psi |\psi|^2
.\end{equation*}
%
We are free to use any method we want, but one has to keep in mind that Algorithm \ref{algo: HO -- exact solve} solves \eqref{eqn: cNLS with psi -- linear part} exactly when \( \psi \) is expressed under the form \eqref{eqn: generic discretization of psi -- ansatz}, i.e. as a sum of bubbles. Therefore we would like the approximate solution to \eqref{eqn: cNLS with psi -- nonlinear part} to keep this particular form. This naturally calls for the use of the Dirac-Frenkel-MacLachlan principle (abbreviated DFMP). For more details, see \cite[Sect.~3]{lasserComputingQuantumDynamics2020}. 
Let \( \mathcal{M} \) be a manifold of complex-valued Gaussian functions:
\begin{equation}
    \label{eqn: manifold for the dirac-frenkel-maclachlan principle}
    \mathcal{M} := \left\{
        u \in \mathbb{L}^2( \mathbb{R}^d ) 
        \left| \begin{aligned}
            u(x) = \sum_{j=1}^N \frac{A_j}{L_j} e^{i\gamma_j + i\beta_j \cdot (x - X_j) - \frac{2+iB_j}{4L_j^2} \left| x - X_j \right|^2} &,\\
            A_j, B_j, \gamma_j \in \mathbb{R},\, L_j \in \mathbb{R}_+^*, \, X_j, \beta_j \in \mathbb{R}^d &
        \end{aligned} \right.
    \right\}
.\end{equation}
%
We look for a function \( u \in \mathcal{M} \) that solves \eqref{eqn: cNLS with psi -- nonlinear part} on \( \mathcal{M} \). More precisely, \( u \) is defined such that its time derivative lies in the tangent space of \( \mathcal{M} \) at \( u \), \( \mathcal{T}_{u(t)} \mathcal{M} \), and such that the residual of equation \eqref{eqn: cNLS with psi -- nonlinear part} is orthogonal to the tangent space. That is, 
\begin{equation}
    \label{eqn: definition of u(t) in DFMP}
    \begin{aligned}
        &\partial_{t} u(t) \in \mathcal{T}_{u(t)} \mathcal{M}, \quad \text{ such that } \\
        &\qquad \langle f, i \partial_{t} u(t) - u(t) |u(t)|^2 \rangle = 0, \, \forall f \in \mathcal{T}_{u(t)} \mathcal{M}.
    \end{aligned}
\end{equation}

\begin{remark}
    The definition of \( \partial_{t} u(t) \) via \eqref{eqn: definition of u(t) in DFMP}, initially proposed by Dirac and Frenkel \cite{diracNoteExchangePhenomena1930,frenkelWaveMechanicsAdvanced1934}, has been later criticized by MacLachlan \cite{mclachlanVariationalSolutionTimedependent1964}. 
    He proposed an alternative approach, which would consist in minimizing the quantity
    \begin{equation*}
        \left|\left| i \partial_{t} u(t) - |u(t)|^2 u(t) \right|\right|^2
    .\end{equation*}
    %
    However, the two formulations are equivalent if the tangent space \( \mathcal{T}_{u(t)} \mathcal{M} \) is \( \mathbb{C} \)-linear \cite{broeckhoveEquivalenceTimedependentVariational1988}. 
    This is the case here because multiplying by the complex unit \( i \) simply amounts to \( \gamma_j \mapsto \gamma_j + \frac{\pi}{2}  \). 
    Therefore, the approaches by Dirac-Frenkel and MacLachlan are equivalent.
\end{remark}
%
\bigskip

Let \( B_{u(t)} \) be a basis of \( \mathcal{T}_{u(t)} \mathcal{M} \), then \eqref{eqn: definition of u(t) in DFMP} is equivalent to
\begin{equation}
    \label{eqn: definition of u(t) in DFMP -- with general basis of Tu M}
    \begin{aligned}
        &\partial_{t} u(t) \in \mathcal{T}_{u(t)} \mathcal{M}, \quad \text{ such that } \\
        &\qquad \langle f, i \partial_{t} u(t) \rangle = \langle f, u(t) |u(t)|^2 \rangle = 0, \, \forall f \in B_{u(t)}.
    \end{aligned}
\end{equation}
%
A family (which may happen to be linearly dependent) spanning the tangent space \( \mathcal{T}_{u(t)} \mathcal{M} \) is given by 
\begin{equation}
    \label{eqn: base of Tu M in DFMP}
    \begin{aligned}
        B_{u(t)} 
        &= \left\{ e^{i\Gamma_j - \frac{| y |^2}{2}},
                    (y_j)_1 e^{i\Gamma_j - \frac{| y_j |^2}{2}},
                    \dots,
                    (y_j)_d e^{i\Gamma_j - \frac{| y_j |^2}{2}},
                    |y_j|^2 e^{i\Gamma_j - \frac{| y_j |^2}{2}}
            : j=1, \dots, N   \right\}, \\
        &=: \left\{ b_{(d+2)(j-1)+1}, b_{(d+2)(j-1)+2}, \dots, b_{(d+2)(j-1)+d+1}, b_{(d+2)j} : j=1, \dots, N \right\},
    \end{aligned}
\end{equation}
%
where we defined
\begin{equation*}
    \Gamma_j(y_j) := \gamma_j + L_j \beta_j \cdot y_j - \frac{B_j}{4} |y_j|^2
.\end{equation*}
%


Thus, \eqref{eqn: definition of u(t) in DFMP -- with general basis of Tu M} is equivalent to 
\begin{equation}
    \label{eqn: definition of u(t) in DFMP -- with particular basis of Tu M}
    \langle i \partial_{t} u(t), b_{(d+2)(j-1)+l} \rangle = \langle u |u|^2, b_{(d+2)(j-1)+l}  \rangle
    , \quad \forall j = 1, \dots, N,\quad l=0, \dots, d+1
.\end{equation}
%

The next step consists in expressing \eqref{eqn: definition of u(t) in DFMP -- with particular basis of Tu M} as a linear system involving the parameters of the bubbles and their time derivative. We then solve the linear system to obtain ODEs on the parameters, and be able to integrate them numerically.

The main advantage of this approach is that it guarantees to keep the approximate solution of \eqref{eqn: cNLS with psi -- nonlinear part} as a sum of \( N \) bubbles, for which we know the value of the parameters.

\hspace{1em}

In order to obtain the linear system, we first have to recall the expression of \( i\partial_{t} u(t) \) \eqref{eqn: i dt uj}:
\begin{equation}
    \label{eqn: i dt u}
    \begin{aligned}
        i \partial_{t} u
        = \sum_{j=1}^N \frac{u_j}{L_j^2} 
                &\left\{ |y_j|^2 \left( i \frac{(L_j)_s}{L_j} - \frac{B_j (L_j)_s}{2L_j} + \frac{(B_j)_s}{4} \right) \right. \\
                &\quad+ y_j \cdot \left( -L_j(\beta_j)_s + i\frac{(X_j)_s}{L_j} - \frac{B_j}{2L_j} (X_j)_s  \right) \\
                &\quad\left. + i\frac{(A_j)_s}{A_j} - i\frac{(L_j)_s}{L_j} + \beta \cdot (X_j)_s - (\gamma_j)_s \right\}.
    \end{aligned}
\end{equation}
%
More concisely, we have
\begin{equation}
    \label{eqn: i dt u -- with basis elements}
    \begin{aligned}
        i \partial_{t} u
        &= \sum_{j=1}^N \frac{A_j}{L_j^3} e^{i\Gamma_j -\frac{| y |^2}{2}}
            \left\{ |y_j|^2 \left( E_j^{(5)} + i E_j^{(6)}  \right) \right. \\
            &\hspace{6em}+ y_j \cdot \left( E_j^{(3)(1, \dots, d)} + iE_j^{(4)(1, \dots, d)}   \right) \\
            &\hspace{6em}\left. + \left( E_j^{(1)} + iE_j^{(2)} \right) \right\} \\
        &= \sum_{j=1}^N \frac{A_j}{L_j^3} 
            \left\{ b_{(d+2)(j-1)+1} \left( E_j^{(1)} + iE_j^{(2)} \right) \right. \\
            &\hspace{6em}+ b_{(d+2)(j-1)+2} \left( E_j^{(3)(1)} + iE_j^{(4)(1)}   \right) \\
            &\hspace{6em}\ \, \vdots \\
            &\hspace{6em}+ b_{(d+2)(j-1)+d+1} \left( E_j^{(3)(d)} + iE_j^{(4)(d)}   \right) \\
            &\hspace{6em}+ \left. b_{(d+2)j} \left( E_j^{(5)} + i E_j^{(6)}  \right) \right\},
    \end{aligned}
\end{equation}
%
where
\begin{equation}
    \label{eqn: DFMP -- definition of the Ej}
    \begin{alignedat}{3}
    &E_j^{(1)} := \beta_j \cdot (X_j)_s - (\gamma_j)_s,        &\qquad& E_j^{(2)} := \frac{(A_j)_s}{A_j} - \frac{(L_j)_s}{L_j}, \\
    &E_j^{(3)(l)} := -L_j((\beta_j)_l)_s - \frac{B_j}{2L_j} ((X_j)_l)_s, &\qquad& E_j^{(4)(l)} := \frac{((X_j)_l)_s}{L_j}, \qquad l = 1, \dots, d, \\
    &E_j^{(5)} := \frac{(B_j)_s}{4} - \frac{B_j (L_j)_s}{2L_j}, &\qquad& E_j^{(6)} := \frac{(L_j)_s}{L_j}
    \end{alignedat}
\end{equation}
%
and where \( E_j^{(k)(1, \dots, d)} \) denotes the vector \( (E_j^{(k)(1)}, \dots,  E_j^{(k)(d)}) \). We recall the subscript \( {}_s \) denotes the derivative with respect to time \( s \).

According to \eqref{eqn: definition of u(t) in DFMP -- with particular basis of Tu M}, we then want to project \( i \partial_{t} u(t) \) against every element of \( B_{u(t)} \). We obtain the following linear system:
\begin{equation}
    \label{eqn: linear system for DFMP}
    \mathbf{A} \mathbf{E} = S
,\end{equation}
%
where
\begin{equation*}
    \mathbf{A} :=
    \begin{pmatrix}
        \langle b_1, b_1 \rangle & i\langle b_1, b_1 \rangle &  \dots & \langle b_{(d+2)N}, b_1 \rangle & i\langle b_{(d+2)N}, b_1 \rangle \\
        \vdots                   &                          &        &                                 & \vdots\\
        \langle b_1, b_{(d+2)N} \rangle& i\langle b_1, b_{(d+2)N} \rangle & \dots & \langle b_{(d+2)N}, b_{(d+2)N} \rangle & i\langle b_{(d+2)N}, b_{(d+2)N} \rangle \\
    \end{pmatrix} \in \mathbb{C}^{(d+2)N, 2(d+2)N}
,\end{equation*}
%
\begin{equation*}
    \mathbf{E} := \begin{pmatrix}
        \frac{A_1}{L_1^3} E_1^{(1)}\\
        \frac{A_1}{L_1^3} E_1^{(2)}\\
        \frac{A_1}{L_1^3} E_1^{(3)(1)} \\
        \frac{A_1}{L_1^3} E_1^{(4)(1)} \\
        \vdots \\
        \frac{A_1}{L_1^3} E_1^{(3)(d)} \\
        \frac{A_1}{L_1^3} E_1^{(4)(d)} \\
        \frac{A_1}{L_1^3} E_1^{(5)} \\
        \frac{A_1}{L_1^3} E_1^{(6)} \\
        \vdots \\
        \frac{A_N}{L_N^3} E_N^{(1)} \\
        \frac{A_N}{L_N^3} E_N^{(2)} \\
        \frac{A_N}{L_N^3} E_N^{(3)(1)} \\
        \frac{A_N}{L_N^3} E_N^{(4)(1)} \\
        \vdots \\
        \frac{A_N}{L_N^3} E_N^{(3)(d)} \\
        \frac{A_N}{L_N^3} E_N^{(4)(d)}\\
        \frac{A_N}{L_N^3} E_N^{(5)} \\
        \frac{A_N}{L_N^3} E_N^{(6)}
    \end{pmatrix} \in \mathbb{R}^{2(d+2)N},
    %
    \quad \text{ and }
    \quad
    S := \begin{pmatrix}
        \langle u|u|^2, b_1 \rangle \\
        \vdots \\
        \langle u|u|^2, b_{(d+2)N} \rangle \\
    \end{pmatrix} \in \mathbb{C}^{(d+2)N}
.\end{equation*}
%


The matrix \( \mathbf{A} \) is the Gram matrix of the family \( B_{u(t)} \) whose columns have been duplicated, which obviously depends on time. In order to solve the linear system \eqref{eqn: linear system for DFMP} we shall use the Moore-Penrose pseudo-inverse, which corresponds to the Least Squares solution if the matrix \( \mathbf{A}^* \mathbf{A} \) is invertible. The pseudo-inverse is invertible if and only if \( B_{u(t)} \) is a linearly independent family of \( \mathbb{L}^2( \mathbb{R}^d ) \).

We can already notice that if two bubbles have the same parameters then the family will be linearly dependent. 

\vspace{1em}

Once the linear system \eqref{eqn: linear system for DFMP} is solved, we obtain \( \mathbf{E} \), from which we can update the modulation parameters. Using the definitions \eqref{eqn: DFMP -- definition of the Ej} we have, for \( j = 1, \dots, N \),
\begin{align*}
    \frac{A_j}{L_j^3} \begin{pmatrix}
        E_j^{(1)} \\ 
        E_1^{(2)} \\
        E_j^{(3)(1, \dots, d)} \\
        E_j^{(4)(1, \dots, d)} \\
        E_j^{(5)} \\
        E_j^{(6)}
    \end{pmatrix}
    = \Re \begin{pmatrix}
        \mathbf{E}_{2(d+2)j+1} \\
        \vdots \\
        \mathbf{E}_{2(d+2)(j+1)}
    \end{pmatrix} 
    &=: \Re \begin{pmatrix}
        \mathbf{E}_{j, 1} \\
        \vdots \\
        \mathbf{E}_{j, 2(d+2)}
    \end{pmatrix} \\
    %
    \iff \begin{pmatrix}
        \beta_j\cdot (X_j)_s - (\gamma_j)_s \\
        \frac{(A_j)_s}{A_j} - \frac{(L_j)_s}{L_j} \\
        -L_j (\beta_j)_s - \frac{B_j}{2L_j} (X_j)_s \\
        \frac{(X_j)_s}{L_j} \\
        \frac{(B_j)_s}{4} - \frac{B_j (L_j)_s}{2L_j} \\
        \frac{(L_j)_s}{L_j}
    \end{pmatrix}
    &= \Re \begin{pmatrix}
        \mathbf{E}_{j, 1} \\
        \vdots \\
        \mathbf{E}_{j, 2(d+2)}
    \end{pmatrix}
.\end{align*}
%

Hence
\begin{equation*}
    \begin{aligned}
        \beta_j\cdot (X_j)_s - (\gamma_j)_s &= \Re \left( \mathbf{E}_{j, 1} \right), \\
        \frac{(A_j)_s}{A_j} - \frac{(L_j)_s}{L_j} &= \Re \left( \mathbf{E}_{j, 2} \right), \\
        -L_j (\beta_j)_s - \frac{B_j}{2L_j} (X_j)_s &= \Re \begin{pmatrix}
                \mathbf{E}_{j, 3} \\ \vdots \\ \mathbf{E}_{j, d+2}
            \end{pmatrix}, \\
        \frac{(X_j)_s}{L_j} &= \Re \begin{pmatrix}
                \mathbf{E}_{j, d+3} \\ \vdots \\ \mathbf{E}_{j, 2d+2}
            \end{pmatrix}, \\
        \frac{(B_j)_s}{4} - \frac{B_j (L_j)_s}{2L_j} &= \Re \left( \mathbf{E}_{j, 2d+3} \right), \\
        \frac{(L_j)_s}{L_j} &= \Re \left( \mathbf{E}_{j, 2d+4} \right).
    \end{aligned}
\end{equation*}
%
Therefore
\begin{equation*}
    \begin{aligned}
        (A_j)_s &= A_j\Re \left( \mathbf{E}_{j, 2} \right) + A_j\Re \left( \mathbf{E}_{j, 2d+4} \right), \\
        (L_j)_s &= L_j \Re \left( \mathbf{E}_{j, 2d+4} \right), \\
        (B_j)_s &= 4\Re \left( \mathbf{E}_{j, 2d+4} \right) + 2B_j \Re \left( \mathbf{E}_{j, 2d+4} \right), \\
        (X_j)_s &= L_j \Re \begin{pmatrix}
                \mathbf{E}_{j, d+3} \\ \vdots \\ \mathbf{E}_{j, 2d+2}
            \end{pmatrix}, \\
        (\beta_j)_s &= -\frac{1}{L_j} \Re \begin{pmatrix}
                \mathbf{E}_{j, 3} \\ \vdots \\ \mathbf{E}_{j, d+2}
            \end{pmatrix} - \frac{B_j}{2L_j} \Re \begin{pmatrix}
                \mathbf{E}_{j, d+3} \\ \vdots \\ \mathbf{E}_{j, 2d+2}
            \end{pmatrix}, \\        
        (\gamma_j)_s &= L_j \beta_j\cdot \Re \begin{pmatrix}
            \mathbf{E}_{j, 3} \\ \vdots \\ \mathbf{E}_{j, d+2}
        \end{pmatrix} - \Re \left( \mathbf{E}_{j, 1} \right),
    \end{aligned}
\end{equation*}
%
and with respect to time \( t \),
\begin{equation}
    \label{eqn: DFMP -- update of parameters with interactions -- wrt time t}
    \begin{aligned}
        (A_j)_s &= \frac{A_j}{L_j^2} \left( \Re \left( \mathbf{E}_{j, 2} \right) + \Re \left( \mathbf{E}_{j, 2d+4} \right) \right), \\
        (L_j)_s &= \frac{1}{L_j}  \Re \left( \mathbf{E}_{j, 2d+4} \right), \\
        (B_j)_s &= \frac{4}{L_j^2} \Re \left( \mathbf{E}_{j, 2d+4} \right) + \frac{2B_j}{L_j^2}  \Re \left( \mathbf{E}_{j, 2d+4} \right), \\
        (X_j)_s &= \frac{1}{L_j}  \Re \begin{pmatrix}
                \mathbf{E}_{j, d+3} \\ \vdots \\ \mathbf{E}_{j, 2d+2}
            \end{pmatrix}, \\
        (\beta_j)_s &= -\frac{1}{L_j^3} \Re \begin{pmatrix}
                \mathbf{E}_{j, 3} \\ \vdots \\ \mathbf{E}_{j, d+2}
            \end{pmatrix} - \frac{B_j}{2L_j^3} \Re \begin{pmatrix}
                \mathbf{E}_{j, d+3} \\ \vdots \\ \mathbf{E}_{j, 2d+2}
            \end{pmatrix}, \\        
        (\gamma_j)_s &= \frac{1}{L_j} \beta_j\cdot \Re \begin{pmatrix}
            \mathbf{E}_{j, 3} \\ \vdots \\ \mathbf{E}_{j, d+2}
        \end{pmatrix} - \frac{1}{L_j^2}  \Re \left( \mathbf{E}_{j, 1} \right).
    \end{aligned}
\end{equation}
%






\subsection{Computing coefficients of the linear system \eqref{eqn: linear system for DFMP}}


In order to be able to compute \( \mathbf{A} \) and \( S \), we give the exact expression of the inner products involved. For \( j, l = 1, \dots, N \), let 
\begin{equation}
    \left |  \begin{aligned}
        z &:= \frac{2+iB_l}{4L_l^2} + \frac{2-iB_j}{4L_j^2}, \\
        a &:= \frac{X_l}{L_l^2} + \frac{X_j}{L_j^2}, \\
        \xi &:= \frac{B_j}{2L_j^2} X_j + \beta_j - \frac{B_l}{2L_l^2} X_l - \beta_l, \\
        C &= \exp\left\{i(\gamma_l - \gamma_j) - \frac{2+iB_l}{4L_l^2}  | X_l | ^2 - \frac{2-iB_j}{4L_j^2}  | X_j | ^2 - i\beta_l \cdot X_l + i\beta_j \cdot X_j \right\}.
    \end{aligned}
    \right.
\end{equation}
%
Those quantities obviously depend on the indices \( j \) and \( l \), but for clarity we do not write explicitly these dependences since they are pretty clear. Then, for \( n, m = 1, \dots, d \),
\begin{align*}
    \langle b_{(d+2)(l-1)+1}, b_{(d+2)(j-1)+1} \rangle &= C \widehat{f}(\xi) \\
    %
    \langle b_{(d+2)(l-1)+n+1}, b_{(d+2)(j-1)+1} \rangle &= \frac{C}{L_l} \left( \widehat{x f}_n - (X_l)_n \widehat{f} \right)(\xi) \\
    %
    \langle b_{(d+2)l}, b_{(d+2)(j-1)+1} \rangle 
    &= \frac{C}{L_l^2} \left( \widehat{ | x | ^2 f} - 2X_l \cdot \widehat{x f} +  | X_l | ^2 \widehat{f} \right)(\xi) \\
    %
    \langle b_{(d+2)(l-1)+n+1}, b_{(d+2)(j-1)+m+1} \rangle
    &= \frac{C}{L_j L_l} \left[ \widehat{x_n x_m f} - (X_l)_n \widehat{x_m f} - (X_j)_m \widehat{x_n f} + (X_l)_n (X_j)_m \widehat{f} \right](\xi) \\
    %
    \langle b_{(d+2)l}, b_{(d+2)(j-1)+m+1} \rangle &=
    \frac{C}{L_l^2 L_j} \left[ \widehat{x_m | x | ^2 f} - 2X_l \cdot \widehat{x_m x f} + | X_l | ^2 \widehat{x_m f} \right.\\
    &\hspace{1cm} \left. - (X_j)_m \widehat{ | x | ^2 f} + 2(X_j)_m X_l \cdot \widehat{x f} - | X_l | ^2 (X_j)_m \widehat{f} \right](\xi) \\
    %
    \langle b_{(d+2)l}, b_{(d+2)j} \rangle  &= \frac{C}{L_l^2 L_j^2} \left[ \widehat{ | x | ^4 f} - 2X_l \cdot \widehat{ | x | ^2 x f} +  | X_l | ^2 \widehat{ | x | ^2 f} - 2X_j \cdot \widehat{x | x | ^2 f} \right. \\
    &\hspace{2cm}  + 4\sum_{n, m=1}^d (X_l)_n (X_j)_m \widehat{x_n x_m f} - 2 | X_l | ^2 X_j \cdot \widehat{x f} \\
    &\hspace{2cm} \left. + \widehat{ | x | ^2 f}  | X_j | ^2 - 2 | X_j | ^2 X_l \cdot \widehat{x f} +  | X_l | ^2  | X_j | ^2 \widehat{f} \right](\xi)
\end{align*}
%

We now compute the components of the vector \( S \). For \( j, k, l, m=1, \dots, N \), let
\begin{equation}
    \left| 
        \begin{aligned}
            C_\Im &:= \exp\left\{ i\left( \gamma_k + \gamma_l - \gamma_m - \gamma_j \right) \right\} \\
            &\hspace{1em} \times \exp\left\{ i\left(\beta_j \cdot X_j + \beta_m \cdot X_m - \beta_l\cdot X_l - \beta_k \cdot X_k \right) \right\} \\
            &\hspace{1em} \times \exp\left\{- i\left( \frac{B_k}{4L_k^2} |X_k|^2 + \frac{B_l}{4L_l^2} |X_l|^2 - \frac{B_m}{4L_m^2} |X_m|^2 - \frac{B_j}{4L_j^2} |X_j|^2  \right) \right\}, \\
            C_\Re &:= \exp\left\{-\frac{1}{2} \left( \frac{|X_k|^2}{L_k^2} + \frac{|X_l|^2}{L_l^2} + \frac{|X_m|^2}{L_m^2} + \frac{|X_j|^2}{L_j^2} \right) \right\}, \\
            C &:= \frac{A_k A_l A_m}{L_k L_l L_m} C_\Im C_\Re, \\
            \xi &:= -\left[ \beta_k + \beta_l - \beta_m - \beta_j + \frac{B_k}{2L_k^2} X_k + \frac{B_l}{2L_l^2} X_l - \frac{B_m}{2L_m^2} X_m - \frac{B_j}{2L_j^2} X_j \right], \\
            z &:= \frac{1}{2} \left( \frac{1}{L_k^2} + \frac{1}{L_l^2} + \frac{1}{L_m^2} + \frac{1}{L_j^2} \right) + i \left( \frac{B_k}{4L_k^2} + \frac{B_l}{4L_l^2} - \frac{B_m}{4L_m^2} - \frac{B_j}{4L_j^2} \right), \\
            a &:= \frac{1}{L_k^2} X_k + \frac{1}{L_l^2} X_l + \frac{1}{L_m^2} X_m + \frac{1}{L_j^2} X_j.
        \end{aligned}
    \right.
\end{equation}
%
Those quantities obviously depend on the indices \( j, k, l \) and \( m \), but for clarity we do not write explicitly these dependences since they are pretty clear. Then, for \( 1 \leq r \leq d \),
\begin{align*}
    \langle u|u|^2, b_{(d+2)(j-1)+1} \rangle &= \sum_{k,l,m} C \widehat{f}(\xi) \\
    \langle u|u|^2, b_{(d+2)(j-1)+r+1} \rangle &= \sum_{k,l,n} \frac{C}{L_j} \left( \widehat{x_r f} - (X_j)_r \widehat{f} \right) \\
    \langle u|u|^2, b_{(d+2)j} \rangle &= \sum_{k,l,m} \frac{C}{L_j^2} \left( \widehat{|x|^2 f} - 2X_j \cdot \widehat{xf} + |X_j|^2 \widehat{f} \right).
\end{align*}
%
We refer to Appendix \ref{appendix: populating the linear system for DFMP} for more details. Moreover, Appendix \ref{appendix: fourier transforms of gaussians} contains Table \ref{table: useful Fourier transforms} which sums up the useful Fourier transforms. 




\begin{remark}[Computational complexity]
    In \eqref{eqn: generic discretization of psi -- expression for uj -  vj gaussian} we chose \( v_j(s_j, y_j) = e^{-\frac{1}{2} | y_j |^2} \).
    This choice was made so that the inner products involved in the application of the DFMP are computable exactly.
    Therefore we do not rely on numerical integration to compute the coefficients of the linear system \eqref{eqn: linear system for DFMP}. 
    In particular, this shows that the computational effort required to obtain the linear system is \( \mathcal{O}(N^4d + N^2(d+2)^2)\).
    To obtain the total complexity, we have to add the cost computing the pseudo-inverse of the hermitian matrix \( \mathbf{A} \in \mathbb{C}^{(d+2)N, (d+2)N} \), which is \( O((d+2)^3 N^3)\).
    This yields the overall computational complexity: \( \mathcal{O}(N^4d + d^3 N^3)\).
\end{remark}




We obtain Algorithm \ref{algo: DFMP -- solve approximately cNLS} which can be used to obtain an approximate solution to \eqref{eqn: cNLS with psi} as a sum of bubbles, using the Strang splitting.

\begin{algorithm}
    \caption{Approximating a solution to \eqref{eqn: cNLS with psi} as a sum of bubbles.}
    \label{algo: DFMP -- solve approximately cNLS}
    \begin{algorithmic}
        \For{ Each timestep of size \( dt \) } 
            \For{ \( j = 1, \dots, N \) }
            \Comment{\(j\) denotes a bubble's index}
                \State Use Algorithm \ref{algo: HO -- exact solve} to update the bubbles over a timestep of size \( dt/2 \).
                \State Compute the coefficients of the linear system \eqref{eqn: linear system for DFMP}.
                \State Solve the linear system \eqref{eqn: linear system for DFMP} to obtain \( \mathbf{E} \).
                \State Use \eqref{eqn: DFMP -- update of parameters with interactions -- wrt time t} to update the parameters over a timestep of size \( dt \).
                \State Use Algorithm \ref{algo: HO -- exact solve} to update the bubbles over a timestep of size \( dt/2 \).
            \EndFor
        \EndFor
    \end{algorithmic}
\end{algorithm}






\subsection{Hamiltonian and norm conservation for the interactions}

When solving \eqref{eqn: cNLS with psi -- nonlinear part} via the DFMP, i.e. when solving the linear system \eqref{eqn: linear system for DFMP}, a Hamiltonian is conserved. 
\begin{lemma}
    Let \( u(t) \) be the approximation to \eqref{eqn: cNLS with psi -- nonlinear part} obtained by applying the Dirac-Frenkel-MacLachlan principle, and define
    \begin{equation*}
        H_{\textnormal{interactions}}(t) := \frac{1}{4} \langle u(t), u(t)| u(t)|^2 \rangle = \frac{1}{4} \langle u(t)^2, u(t)^2 \rangle
    .\end{equation*}
    %
    Then \( H_{\textnormal{interactions}} \) is conserved, i.e.
    \begin{equation*}
        \frac{\textnormal{d}}{\textnormal{d}t} H_{\textnormal{interactions}}(t) = 0
    ,\end{equation*}
    %
    and the \( \mathbb{L}^2 \) norm of \( u \) is also conserved.
\end{lemma}


\begin{proof}
    We have
    \begin{equation*}
        H_{\textnormal{interactions}}(t) := \frac{1}{4} \langle u(t), u(t)| u(t)|^2 \rangle = \frac{1}{4}  \langle u(t)^2, u(t)^2 \rangle
    ,\end{equation*}
    %
    by using the Hermitian property of the inner product \( \langle \cdot, \cdot \rangle \). Then,
    \begin{align*}
        \frac{\textnormal{d}}{\textnormal{d}t} H_{\textnormal{interactions}}(t)
        &= \frac{1}{4} \frac{\textnormal{d}}{\textnormal{d}t} \langle u(t)^2, u(t)^2 \rangle \\
        &= \frac{1}{4} \left\langle 2u(t) \partial_{t} u(t) , u(t)^2 \right\rangle + \left\langle u(t)^2, 2u(t) \partial_{t} u(t) \right\rangle \\
        &= \Re \left\langle u(t) \partial_{t} u(t) , u(t)^2 \right\rangle \\
        &= \Re \left\langle \partial_{t} u(t) , u(t) |u(t)|^2 \right\rangle.
    \end{align*}
    %
    By definition of \( \partial_{t} u(t) \), we have \( \partial_{t} u(t) \in \mathcal{T}_{u(t)} \mathcal{M} \), hence we can take \( f = \partial_{t} u(t) \) in \eqref{eqn: definition of u(t) in DFMP}. We obtain the following equality:
    \begin{equation*}
        \langle \partial_{t} u(t), u(t) |u(t)|^2 \rangle = \langle \partial_{t} u(t), i \partial_{t} u(t) \rangle = -i \| \partial_{t} u(t) \|^2
    .\end{equation*}
    %
    Therefore,
    \begin{equation*}
        \frac{\textnormal{d}}{\textnormal{d}t} H_{\textnormal{interactions}}(t) = \Re \left( -i \| \partial_{t} u(t) \|^2 \right) = 0
    .\end{equation*}
    %

    Using similar ideas, we can easily show the conservation of the \( \mathbb{L}^2 \) norm: we obviously have \( u(t) \in \mathcal{T}_{u(t)} \mathcal{M} \), hence
    \begin{align*}
        \frac{\textd}{\textd t} \|u(t)\|^2 
        &= 2\Re\langle u(t), \partial_{t} u(t) \rangle = 2 \Re \langle u(t), -i u(t)|u(t)|^2 \rangle \\
        &= 2 \Re \left( i \langle |u(t)|^2, |u(t)|^2 \rangle\right) = 0
    \end{align*}
    %
\end{proof}







\subsection{Recovering the Harmonic Oscillator equations}

Suppose the family \( B_{u(t)} \subset \mathbb{L}^2( \mathbb{R}^d ) \) defined by \eqref{eqn: base of Tu M in DFMP} is linearly independent, and consider the equation \eqref{eqn: cNLS with psi -- linear part}.
By summing equation \eqref{eqn: HO -- idt u - Hu -- with v} over \( j=1, \dots, N \) with \( v_j(s_j,y_j) = e^{-\frac{|y_j|^2}{2}} \), and letting this sum be equal to zero, we obtain an equation of the form
\begin{equation}
    \label{eqn: recovering the HO eqns -- write HO with generic coefficients in basis B}
    \begin{aligned}
        \sum_{j=1}^N c_{(d+2)(j-1)+1} b_{(d+2)(j-1)+1} + c_{(d+2)(j-1)+2} b_{(d+2)(j-1)+2} + \dots& \\
        \qquad + c_{(d+2)(j-1)+d+1} b_{(d+2)(j-1)+d+1} + c_{(d+2)j} b_{(d+2)j} &= 0.
    \end{aligned}
\end{equation}
%
Thanks to the assumption that \( B_{u(t)} \) is a linearly independent family, we know that we must have 
\begin{equation}
    \label{eqn: recovering HO equations from DFMP -- all coeffs equal to zero}
    c_{k} = 0,\quad k=1,\dots, (d+2)N    
.\end{equation}
%
This yields exactly the system of equations \eqref{eqn: modulation ODEs -- linear part wrt time s - with v}, with the \( \gamma \) equation replaced by \eqref{eqn: HO -- ODE on gamma}.
In other words, the DFMP approach gives the same equations as those given in Section \ref{sect: HO -- ODEs on the modulation parameters with particular vj} when \( B_{u(t)} \) is a linearly independent family. However, our approach as described in Section \ref{sect: HO -- ODEs on the modulation parameters with particular vj} allows to solve them exactly and not only numerically with some numerical time-integrator.

Finally, if the family \( B_{u(t)} \) is linearly dependent, then we cannot write equation \eqref{eqn: recovering HO equations from DFMP -- all coeffs equal to zero} anymore, hence the DFMP approach fails. Our approach allows to circumvent this issue by naturally imposing conditions \eqref{eqn: recovering HO equations from DFMP -- all coeffs equal to zero} in all cases.




\subsection{Imposing conservation of \( \mathbb{L}^2 \) norm}

Our numerical experiments have shown that the DFMP approach does not always yield a nice conservation of the \( \mathbb{L}^2 \) norm, even though it should be conserved. This section is dedicated to finding a way of imposing it explicitely.

By using \eqref{eqn: i dt u -- with basis elements}, the conservation of the \( \mathbb{L}^2 \) norm writes
\begin{align*}
    0 &= \partial_{t} \int_{ \mathbb{R}^2 } |u(t,x)|^2 = 2\Re\langle \partial_{t} u, u \rangle \\
    &= 2 \sum_{k=1}^N \frac{A_k}{L_k^3} \Re \left\langle -i \left\{ b_{(d+2)(k-1)+1} \left( E_k^{(1)} + iE_k^{(2)} \right) \right. \right. \\
    &\hspace{6em}+ b_{(d+2)(k-1)+2} \left( E_k^{(3)(1)} + iE_k^{(4)(1)}   \right) \\
    &\hspace{6em}\ \, \vdots \\
    &\hspace{6em}+ b_{(d+2)(k-1)+d+1} \left( E_k^{(3)(d)} + iE_k^{(4)(d)}   \right) \\
    &\hspace{6em}+ \left. \left. b_{(d+2)k} \left( E_k^{(5)} + i E_k^{(6)}  \right) \right\}, u \right\rangle
.\end{align*}
%
Under matrix form, this yields
\begin{equation*}
    0 = 2\Im \left[ \left( \langle b_1, u \rangle, i\langle b_1, u \rangle, \dots, \langle b_{(d+2)N}, u \rangle, i\langle b_{(d+2)N}, u \rangle \right) \mathbf{E} \right].
\end{equation*}
%
Since \( \mathbf{E} \in \mathbb{R}^{2(d+2)N} \), we obtain 
\begin{equation*}
    \left( \Im \langle b_1, u \rangle, \Re \langle b_1, u \rangle, \dots, \Im \langle b_{(d+2)N}, u \rangle, \Re \langle b_{(d+2)N}, u \rangle \right) \mathbf{E} = 0
.\end{equation*}
%
In order to impose the conservation of the \( \mathbb{L}^2 \) norm, we have to add this line to the linear system \eqref{eqn: linear system for DFMP}. Moreover, this line has already been computed: indeed, we have
\begin{equation*}
    u = \sum_{j=1}^N \frac{A_j}{L_j} b_{(d+2)(j-1)+1} 
,\end{equation*}
%
hence
\begin{equation*}
    \langle b_k, u \rangle = \sum_{j=1}^N \frac{A_j}{L_j} \langle b_k, b_{(d+2)(j-1)+1} \rangle,\quad k=1, \dots, (d+2)N
.\end{equation*}
%


The idea motivating the addition of this line to the linear system is the following: whenever the pseudo-inverse of the matrix \( \mathbf{A} \) is well-conditioned the conservation of \( \mathbb{L}^2 \) norm is automatically satisfied, so the new line doesn't change anything. When the pseudo-inverse is ill-conditioned, it may happen that the \( \mathbb{L}^2 \) norm isn't preserved anymore. Hence, at each time step we compute the solution to the initial system of size \( (d+2)N\times 2(d+2)N \), and if the variation of the \( \mathbb{L}^2 \) norm of \( u \) is too high (e.g. larger than some numerical threshold) then we solve again the augmented linear system.


