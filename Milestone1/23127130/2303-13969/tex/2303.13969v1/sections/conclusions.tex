

\section{Conclusion}


We presented in this work an approach based on recent results \cite{faouWeaklyTurbulentSolutions2020}.
It allows to solve exactly the Harmonic Oscillator \eqref{eqn: cNLS with psi -- linear part} for initial functions that can be represented as a sum of modulated functions (the \emph{bubbles}), for a certain kind of modulation.


In this context we focused on a particular subclass of such functions, modulated gaussians, which have the advantage of allowing explicit computations.
This is particularly interesting since we do not have to rely on any sort of discretization of the phase space, which is usually the main computational burden in numerical simulations.
We obtain an algorithm which yields an exact solution as soon as the initial data is a sum of modulated gaussians.
If we consider an arbitrary function, the only error this algorithm makes is the initial discretization of said function as a finite sum of modulated gaussians. 

We also extended the results from \cite{faouWeaklyTurbulentSolutions2020} by allowing polynomial interactions, at the cost of approximating the solution to \eqref{eqn: cNLS with psi -- nonlinear part} via the Dirac-Frenkel-MacLachlan principle.
Once again, considering only modulated gaussian functions allowed us to perform explicit computations and to obtain a numerical algorithm whose computational complexity writes \( \mathcal{O}(N^4d + N^3 d^3) \).
Here \( d \) is the dimension and \( N \) is the number of bubbles.
The most critical parameter is \( N \), which corresponds roughly to the precision of the discretization when considering arbitrary initial data. 
For any given function, the higher \( N \), the better we can approximate it as a sum of modulated gaussian functions.
We then have a clear trade-off between the speed of the algorithm and the precision of the discretization.

Finally, our numerical examples show that the bubbles algorithm outperforms a spectral method combined with time splitting, where each splitting step is solved exactly.
We compared the results obtained when solving the Dirac-Frenkel-MacLachlan linear system, with and without imposing the conservation of the \( \mathbb{L}^2 \) norm: as expected, imposing the conservation of the \( \mathbb{L}^2 \) norm worsens the conservation of the energy and momentum.

As a final note, we emphasize the fact that any grid-based method is inherently bound to a finite subset of \( \mathbb{R}^d \) to which we have to add boundary conditions, while the bubble approach does not have such restrictions.