\documentclass[12pt,a4paper]{article}

\pdfoutput=1

\usepackage[utf8]{inputenc}
\usepackage[T1]{fontenc}
\usepackage{listings}
\usepackage{mathtools}
\usepackage{algorithm}
\usepackage{algpseudocode}
\usepackage{amsthm}
\usepackage{amssymb}
\usepackage{amsfonts}
\usepackage[title]{appendix}%
\usepackage{authblk} % autheurs et affiliations
\usepackage{array}
\usepackage[english]{babel}
\usepackage{bbm}  % pour faire l'indicatrice avec \mathbbm{1}
%\usepackage{calrsfs} % jolies lettres calligraphiees
\usepackage{empheq}
\usepackage[shortlabels]{enumitem} % \begin{enumerate}[leftmargin=10pt, (a)]
\usepackage{graphicx}
\usepackage[hypertexnames=false]{hyperref}	%pour avoir des hyperliens dans la table des matieres
\usepackage{cleveref} 
% \usepackage{autonum} % number only referenced equations
\usepackage[top=3cm, bottom=3cm, left=2.5cm, right=2.5cm]{geometry}
\usepackage{fancyhdr}%utile pour changer header & footer
\usepackage{lastpage}%pour avoir le num de la derniere page (utile pour 
%numerotation)
\usepackage{lmodern}
\usepackage{tikz,tkz-tab}
\usetikzlibrary{babel}
% \usepackage{thmbox}
\usepackage{subfiles}
\usepackage{stmaryrd}
\usepackage{subcaption} 
\usepackage{tabularx}
% \usepackage{subeqn}
\usepackage{pgf}
\usepackage{pgfplots} 
\usepackage{wrapfig}

\DeclareUnicodeCharacter{2212}{-}
\usepgfplotslibrary{groupplots,dateplot,fillbetween}
\usetikzlibrary{patterns,shapes.arrows,intersections,backgrounds,external}
\newcommand{\customTikzInputFolder}{tikz/input}
\newcommand{\customTikzOutputFolder}{tikz/output}
\tikzexternalize[prefix=\customTikzOutputFolder]%,figure name=_plot_subsect_\thesubsubsection_nb]
\pgfplotsset{compat=newest}

%%%%% Essai
% Define the command. Note that input folder is NOT STATIC ANYMORE!
\newcommand{\inputtikz}[2]{% 
	% First argument is the width as a proportion of \textwidth
	% Second argument is the filename
    \tikzsetnextfilename{#2}%
    \ifcustomcompileusingtikzexternalize
	\resizebox{#1\textwidth}{!}{\input{\customTikzInputFolder#2.tex}}%
	\else
	\includegraphics[width=#1\textwidth]{\customTikzOutputFolder#2.pdf}
	\fi
}
%%%%%



\usepackage{float} % Mettre "H" dans param d'une figure pour empecher le texte
%de se mettre avant la figure

%\usepackage{nameref} % Pour labelliser des sections/chapitres non numérotés.
%Fonctionne en appelant \nameref{} au lieu de \ref

%\usepackage{parskip} % Ne pas mettre des indentations partout

% %%%%%%%%%%%%%%%%%%%%%%%%%%%%%%%%%%%%%%%%%%%%%%%%%%%%%%%%%%%%%%%%%%%%%%%%%%%%%%
% % use ``\ghostref{myeq}'' to reference equation labelled ``myeq'' 
% % but do nothing with it (i.e. don't show number)
% \newcommand\ghostref[1]{%
% % Do nothing!
% }
% \makeatletter
% \autonum@generatePatchedReference{ghostref}
% \makeatother


\bibliographystyle{alpha}
% \bibliographystyle{plain}

\definecolor{dkgreen}{rgb}{0,0.5,0}
\definecolor{gray}{rgb}{0.5,0.5,0.5}
\definecolor{mauve}{rgb}{0.58,0,0.82}


% Police alternative
%\usepackage{concmath}
% \usepackage{mlmodern}
%\usepackage{fouriernc}


% Modifie ce qui est affiché comme nom de section
% Ne marche pas correctement avec \section*
%\usepackage[explicit]{titlesec}
%\titleformat{\section}
%{\Large\bfseries}{}
%{0pt}{#1\quad\thesection}


\setenumerate[1]{label=\thesection.\arabic*.} % Pour avoir les numeros des 
%questions qui incluent le numero de section.


% Pour avoir les chiffres romains écrits en lettres majuscules
%\newcommand{\RomanNumeralCaps}[1]
%    {\MakeUppercase{\romannumeral #1}}



\hypersetup{
colorlinks=true, % false: boxed links; true: colored links
linkcolor=red	%pas de coloration particulière pour les liens internes
}


\fancypagestyle{plain}{
%	\fancyhf{}
	\renewcommand{\headrulewidth}{0pt}
	\fancyhead[]{}	%header à gauche
	\fancyfoot[L]{}	%footer au centre
	\fancyfoot[R]{}
	\fancyfoot[C]{\thepage{} / \pageref{LastPage}} %footer a droite
}
\pagestyle{plain}


%%%%%%%%%%%%%%
% change l'affichage de cleveref
% Second argument is singular, third is plural
\crefname{thm}{theorem}{theorems}
\Crefname{thm}{Theorem}{Theorems}

\crefname{demo}{proof}{proofs}
\Crefname{demo}{Proof}{Proofs}

\crefname{prop}{proposition}{propositions}
\Crefname{prop}{Proposition}{Propositions}

\crefname{lemma}{lemma}{lemmata}
\Crefname{lemma}{Lemma}{Lemmaata}

\crefname{definition}{definition}{definitions}
\Crefname{definition}{Definition}{Definitions}

\crefname{rmk}{remark}{remarks}
\Crefname{rmk}{Remark}{Remarks}
%%%%%%%%%%%%%%

\newcommand\crefpairconjunction{ and } % change le mot utilise entre les 
%references lorsqu'on utilise \cref{qqch1,qqch2}.



%%%%%%%%%%%%%%

\theoremstyle{plain}
\newtheorem{thm}{Theorem}[section]
%
%\newtheorem*{thm*}{Théorème} % reset theorem numbering for each chapter
%\newtheorem{thm}{Théorème}

\theoremstyle{plain}
%\newtheorem*{demo*}{Démonstration}
\newtheorem{demo}{Proof}[section]
%\newtheorem{demo}{Démonstration}

\theoremstyle{plain}
%\newtheorem*{prop*}{Proposition}
\newtheorem{prop}{Proposition}[section]
%\newtheorem{prop}{Proposition}

\theoremstyle{plain}
%\newtheorem*{lemma*}{Lemme}
\newtheorem{lemma}{Lemma}[section]
%\newtheorem{lemma}{Lemme}

\theoremstyle{definition}
%\newtheorem*{definition*}{Définition}
\newtheorem{definition}{Definition}[section]
%\newtheorem{definition}{Définition}

\theoremstyle{remark}
%\newtheorem*{rmk*}{Remarque}
\newtheorem{remark}{Remark}[section]
%\newtheorem{rmk}{Remarque}
%%%%%%%%%%%%%%


\newcommand{\numberthis}{\stepcounter{equation}\tag{\theequation}} %utiliser 
%'\numberthis' pour numéroter une équation dans un environnement align* par ex


\renewcommand*{\proofname}{Proof} % change le nom "proof" en 
%l'argument, lorsqu'on utilise l'environnement 'proof'


%%%%%%%%%%%%%%
%% Julia language is not supported :(
% \lstset{
% language=Julia,
% basicstyle=\footnotesize\ttfamily,
% numberstyle=\normalsize,
% numbersep=7pt,
% frame = L,
% breaklines = true,
% extendedchars=true,
% keywordstyle=\color{blue},      % keyword style
%    commentstyle=\color{dkgreen},   % comment style
%    stringstyle=\color{black},      % string literal style
% literate={á}{{\'a}}1 {ã}{{\~a}}1 {é}{{\'e}}1 {è}{{\`e}}1 
% {à}{{\`a}}1{ç}{{\c{c}}}1 {ê}{{\^e}}1,%cette ligne permet de bien afficher tous 
% %les accents dans les codes
% }
%%%%%%%%%%%%%%

\makeatletter 
\@ifclassloaded{report}{%
\renewcommand{\thesection}{\thechapter.\arabic{section}} %pour faire commencer les 
%sections à 1 au lieu de 0, et pour avoir le numéro du chapitre dans le numéro de section. 
% /!\ Utile seulement dans un report ou book, et pas dans un article car alors ``\thechapter'' n'existe pas !!
}{}
\makeatother

\renewcommand\theequation{\thesection.\arabic{equation}}



\numberwithin{equation}{section} % Pour faire correspondre la numérotation des 
%equations aux sections

%%%%%%%%%%%%%%%%%%%%%%%%%%%%%%%%%%%%%%%%%%%%%%%%%%%%

\newcommand{\BB}[1]{\mathbbm{#1}} % Lettres en mathbb
\newcommand{\CAL}[1]{\mathcal{#1}} % Lettres en mathcal

\newcommand{\cv}[2]{\overset{#1}{\underset{#2}{\longrightarrow}}} % ecrire 
%rapidement une convergence. Premier argument, au-dessus de la fleche de 
%limite, deuxieme argument en dessous

\newcommand{\sign}{\ \textnormal{sign}}
\newcommand{\textd}{\textnormal{d}}

\newcommand{\QED}{\hfill $\qed$}
\newcommand{\supp}{\textnormal{supp }}
\DeclareMathOperator{\arctanTwo}{arctan2}


\newcommand{\TODO}[1]{\begin{center} \textcolor{red}{\textbf{{\Huge #1}}} \end{center}}


\allowdisplaybreaks % Pour autoriser des pages breaks dans des align


%%%%%%%%%%%%%%%%%%%%%%%%%%%%%%%%%%%%%%%%%%%%%%%%%%%%%%%%%%%%%%%%%%%%%
%% D'après le template Springer  %%%%%%%%%%%%%%%%%%%%%%%%%%%%%%%%%%%%

%%%%%%%%%%%%%%%%%%%%%%%%%%%%%%%%%%%%%%%%%%%%%%%%%%%%%%%%%%%%%%%%%%%%%
%%%%%%%%%%%%%%%%%%%%%%%%%%%%%%%%%%%%%%%%%%%%%%%%%%%%%%%%%%%%%%%%%%%%%

% \renewcommand{\customTikzInputFolder}{./tikz/input/}
% \renewcommand{\customTikzOutputFolder}{./tikz/output/}
\renewcommand{\customTikzOutputFolder}{./tikz_clean/output/} % This folder only contains pdf, a ``clean'' directory
\tikzsetexternalprefix{\customTikzOutputFolder}

\input{tikz_clean/tikz_additional_preamble.tex}


% =========================================================================
% ===== Choose between including TikZ externalized figures and
% ===== including the generated pdf figure.
\newif\ifcustomcompileusingtikzexternalize
% --
% \customcompileusingtikzexternalizetrue % uncomment to compile tikz figures with TikZ externalize, 
\customcompileusingtikzexternalizefalse % uncomment to include only the pdf files (generated by TikZ)
% =========================================================================


\author[1]{Erwan Faou\footnote{\url{erwan.faou@inria.fr}}}
\author[1]{Yoann Le H{\'e}naff\footnote{\url{yoann.lehenaff@univ-rennes.fr}}}
\author[2]{Pierre Rapha{\"e}l\footnote{\url{pr463@dpmms.cam.ac.uk}}}

\affil[1]{\small Rennes University, IRMAR UMR 6625 \\INRIA centre at Rennes University (MINGuS project-team), Rennes, France}
\affil[2]{\small Department of Pure Mathematics and Mathematical Statistics, University of Cambridge, Cambridge, U.K.}


\begin{document}

\title{On a Bubble algorithm for the cubic Nonlinear Schr\"odinger equation}


\maketitle

\begin{abstract}
    Based on very recent and promising ideas, stemming from the use of \emph{bubbles}, we discuss an algorithm for the numerical simulation of the cubic nonlinear Schr\"odinger equation  with harmonic potential (cNLS) in any dimension, that could easily be extended to other polynomial nonlinearities.
    This algorithm consists in discretizing the initial function as a sum of modulated complex gaussian functions (the bubbles), each one having its own set of parameters, and then updating the parameters according to cNLS.
    Numerically, we solve exactly the linear part of the equation and use the Dirac-Frenkel-MacLachlan principle to approximate the nonlinear part. We then obtain a grid free algorithm in any dimension whose efficiency compared with spectral methods is illustrated by numerical examples. 
\end{abstract}
%



\tableofcontents

\section{Introduction}

The increasing complexity of source code poses a key challenge to the reliability of large-scale software systems. Software bugs in these systems can lead to safety issues~\cite{bug_safety} for users around the world as well as cause non-negligible financial losses~\cite{bug_loss}. As such, developers have to spend a large amount of time and effort on bug fixing. Consequently, \aprfull (\apr), designed to automatically generate patches to fix software bugs, has attracted wide attention from both academia and industry~\cite{long2016prophet, legoues2012genprog, long2015spr, lou2020can, tufano2018empstudy}. 


To achieve \apr, one popular approach is known as Generate-and-Validate (G\&V)~\cite{qi2015gv, ghanbari2019prapr, lou2020can, le2016hdrepair, legoues2012genprog, wen2018capgen, hua2018sketchfix, martinez2016astor, koyuncu2020fixminder, liu2019tbar, liu2019avatar}, which is typically based on the following pipeline: First, fault localization techniques~\cite{wong2016fl, abreu2007ochiai, zhang2013injecting, papadakis2015metallaxis, li2019deepfl, li2017transforming} are applied to determine the suspicious locations in programs where bugs are likely to exist. Then, the buggy locations are used by the \apr tools to generate a list of patches that replace buggy lines with correct lines. Afterward, each patch is validated against the original test suite to identify any \emph{plausible patches} (i.e., passing all tests in the test suite). Finally, to determine the \emph{correct patches}, developers examine the list of plausible patches to see if any of them can correctly fix the bug. 

Traditional \apr tools can mainly be categorized into heuristic-based~\cite{legoues2012genprog, le2016hdrepair, wen2018capgen}, constraint-based~\cite{mechtaev2016angelix, le2017s3, demacro2014nopol, long2015spr} and \template~\cite{ghanbari2019prapr, hua2018sketchfix, martinez2016astor, liu2019tbar, liu2019avatar}. Among these traditional tools, \template \apr tools~\cite{ghanbari2019prapr, liu2019tbar, benton2020effectiveness} have been able to achieve state-of-the-art results. \Template \apr tools typically leverage pre-defined templates (e.g., adding a nullness check) for bug fixing. However, since these fix templates are typically handcrafted, the number and types of bugs they are able to fix can be limited. 



To address the limitations of traditional \apr, researchers have proposed various \learning \apr tools~\cite{li2020dlfix, chen2018sequencer, jiang2021cure, lutellier2020coconut, zhu2021recoder, ye2022rewardrepair} based on the \nmtfull (\nmt) architecture~\cite{sutskever2014mt} where the input is the buggy code snippets and the goal is to translate the buggy code snippets into a fixed version. To accomplish this, \learning \apr tools require supervised training datasets with pairs of both buggy and fixed code snippets in order to learn how to perform this translation step. These training data are usually obtained by mining historical bug fixes using heuristics/keywords~\cite{dallmeier2007benchmark}, which can be imprecise for identifying bug-fixing commits; even the actual bug-fixing commits can include irrelevant code changes, leading to further pollution in the dataset~\cite{xia2022alpharepair}.
% 
Moreover, it can be hard for such \apr tools to generalize and fix bug types unseen during training. 



To better leverage recent advances in \plmfull{s} (\plm{s}), researchers~\cite{xia2022alpharepair, xia2023repairstudy, kolak2022patch, prenner2021codexws} have directly applied \plm{s} to generate patches without bug-fixing datasets. These \llm-based \apr tools work by either directly generating a complete code function~\cite{prenner2021codexws, xia2023repairstudy} or predict/infill the correct code snippet given its surrounding context~\cite{xia2022alpharepair, xia2023repairstudy}. By directly using \llm{s} that are pre-trained on billions of open-source code snippets, \llm-based \apr tools can achieve state-of-the-art performance on many repair datasets~\cite{xia2022alpharepair}. 


% 
%
%

Traditional \apr tools have long used the insight of the \emph{plastic surgery hypothesis}~\cite{barr2014plastic} where it states that the code ingredients to fix a bug already exist within the same project. Traditional \apr tools have manually designed pattern-~\cite{ghanbari2019prapr, saha2017elixir} or heuristic-based~\cite{jiang2018simfix, legoues2012genprog} approaches to finding and using such relevant code ingredients to generate fixes for bugs. However, the plastic surgery hypothesis has been largely ignored in \llm-based \apr. In fact, \llm provides a unique opportunity to fully automate the plastic surgery hypothesis idea via fine-tuning (learning project-specific information via model updates from the buggy project) and prompting (directly providing relevant code ingredients to the model), and make it directly applicable to different languages (since the \llm{s} are typically multi-lingual).%
Moreover, despite the intensive manual efforts involved, traditional \apr tools still cannot fully leverage project-specific information due to large search space for leveraging/composing existing code ingredients. In contrast, the project-specific information can effectively leveraged by \llm{s} due to their power in code understanding/vectorization, e.g., even partial/imprecise information may still guide \llm{s} in correct patch generation!
 To this end, we ask the question: \emph{How useful is the plastic surgery hypothesis in the era of \plm{s}}?








\mypara{Our Work.} To answer the question, we present \ourtech{\xspace} -- a \llm-based approach that automatically utilizes the plastic surgery hypothesis by systematically combining multiple fine-tuning and prompting strategies for \apr. \ourtech fine-tunes \plm{s} using two novel domain-specific training strategies: \textbf{\epfinetune} -- we fine-tune using the original buggy project by aggressively masking out a high percentage of tokens, which allows \plm to learn project-specific code tokens and programming styles; and \textbf{\rofinetune} -- which only masks out a single continuous code sequence per training sample, allowing the model to get used to the final \csapr task of predicting a single continuous code sequence. Furthermore, we directly leverage the ability for \plm{s} to understand natural language instructions and introduce a novel prompting strategy, \textbf{\idprompting}, which uses information retrieval and static analysis to obtain a list of relevant identifiers for the buggy lines. While such relevant identifiers are critical for fixing some difficult bugs, they may not be seen by the \llm during inference due to limited context window size. Through the use of prompting, we directly tell the model to use these extracted identifiers (relevant code ingredients) to generate the correct code. Finally, to perform repair, we combine all four model variants (including the base model, both fine-tuned models and the base model with prompting) for the final repair.





While our insight of leveraging the plastic surgery hypothesis for \llm-based \apr is generalizable across different types of \plm{s}, to implement \ourtech, we choose a recent \plm{\xspace}, \ctfive~\cite{wang2021codet5}, which is pre-trained on millions of open-source code snippets. \ctfive is an encoder-decoder model trained using \mspfull (\msp) objective where a percentage of tokens are masked out and each continuous masked token sequence is referred to as a masked span. Also, although we only extract relevant identifiers from the current buggy project (since this paper focuses on the plastic surgery hypothesis), our work can be easily extended to obtain other code information (such as relevant statements or functions) from other sources, such as  the massive pre-training corpora~\cite{husain2020codesearchnet} or historical bug-fixing datasets~\cite{jiang2019infer}, which can provide more coding knowledge for \llm{s}. Besides, although we mainly focus on using traditional string comparison algorithms for information retrieval in this paper, these techniques can be easily replaced by other frequency-based retrieval~\cite{robertson2009probabilistic} and neural search (or embedding-based search)~\cite{reimers2019sentence}.
  In summary, this paper makes the following contributions:


%


\begin{itemize}[noitemsep, leftmargin=*, topsep=0pt]
    \item \textbf{Dimension.} This paper is the first to revisit the important plastic surgery hypothesis in the era of \llm{s}. It opens up a new dimension for \llm-based \apr to incorporate previously neglected information from the buggy project itself to boost \apr performance. Furthermore, it demonstrates the promising future of retrieval-based prompting for modern \llm-based \apr.
    \item \textbf{Implementation.} We implement \ourtech based on the recent \ctfive model. We augment the model using two novel fine-tuning strategies: \epfinetune and \rofinetune, along with a novel prompting strategy based on information retrieval and static analysis: \idprompting. We combine the patches generated by all four models together and perform patch ranking to speed up \apr.% 
    \item \textbf{Evaluation Study.} We conduct an extensive evaluation against state-of-the-art \apr tools. On the widely studied \dfj 1.2 and 2.0 datasets~\cite{just2014dfj}, \ourtech is able to achieve the new state-of-the-art results of 89 and 44 correct bug fixes (15 and 8 more than best baseline) respectively.  Furthermore, we perform a broad ablation study to justify our design. \ourtech demonstrates for the first time that the plastic surgery hypothesis can substantially boost \llm-based \apr and advance state-of-the-art \apr, while being fully automated and general. Moreover, even partial/imprecise code ingredients may still effectively guide \llm{s} for \apr!
\end{itemize}




\section{The Harmonic Oscillator}
\label{sect: The Harmonic Oscillator}


In this section we focus onto the linear part of the cubic Non Linear Schr\"odinger equation, namely the Harmonic Oscillator \eqref{eqn: cNLS with psi -- linear part}.

\subsection{General case: \( v = v(s,y) \)}

\subsubsection{Conservation Laws}
\label{sect: HO -- Conservation Laws}

We recall classical laws for the Harmonic oscillator equation (see for instance
\cite{killipNonlinearSchrOdinger2013,taoNonlinearDispersiveEquations2006}).  
Let \( \psi \) be the solution to \eqref{eqn: cNLS with psi -- linear part}.
We will need the following result, known as the Pohozaev identity. 

\begin{lemma}[Pohozaev Identity]
    \label{lemma: pohozaev identity}
    Let \( x\in \mathbb{R}^d \), and \( f\in H^1( \mathbb{R}^d ) \) such that \( xf\in \mathbb{L}^2( \mathbb{R}^d ) \). Then    
    \begin{equation}
        \label{eqn: pohozaev identity}
        \int \Delta f \overline{\left( \frac{d}{2} f + x\cdot \nabla f \right)} dx = - \int |\nabla f|^2 dx
    .\end{equation}
\end{lemma}


\begin{proof}
    By density, we only need to prove equation \eqref{eqn: pohozaev identity} for \( f \in \mathcal{C}^\infty_c( \mathbb{R}^d ) \), where \( \mathcal{C}^\infty_c ( \mathbb{R}^d ) \) denotes the space of infinitely smooth functions with compact support in \( \mathbb{R}^d \). Let 
    \begin{equation*}
        f_\lambda(x) := \lambda^{\frac{d}{2} } f(\lambda x)
    ,\end{equation*}
    %
    then
    \begin{equation*}
        \int |\nabla f_\lambda |^2 dx = \lambda^2 \int |\nabla f|^2 dx
    .\end{equation*}
    %
    Differentiating this identity with respect to \( \lambda \) and evaluating the result at \( \lambda = 1 \) yields 
    \begin{equation*}
        \int \nabla f \cdot \overline{ \nabla \left( \frac{d}{2} f + x \cdot \nabla f \right) } dx = \int |\nabla f|^2 dx
    .\end{equation*}
    %
    We integrate by parts the LHS, and obtain \eqref{eqn: pohozaev identity}.
\end{proof}



\begin{lemma}[Conserved quantities in dimension \( d=2 \)]
    \label{lemma: conserved quantities in HO}
    We consider a two-parameter family of equations containing \eqref{eqn: cNLS with psi -- linear part} and \eqref{eqn: cNLS with psi}:
    \begin{equation*}
        i \partial_{t} \psi + \mu(\Delta \psi - |x|^2 \psi) = \lambda |\psi|^2 \psi, \quad \mu, \lambda \in \mathbb{R}
    .\end{equation*}
    %
    The (radial) conservation laws are mass \( \|\psi\|_{ \mathbb{L}^2 } \), energy
    \begin{equation*}
        E_{\mu,\lambda} = \frac{\mu}{2} \left\langle H\psi, \psi \right\rangle + \frac{\lambda}{4} \left\langle |\psi|^2 \psi, \psi \right\rangle
    ,\end{equation*}
    %
    where \( H = -\Delta + |x|^2 \) and \( \langle f,g \rangle := \int_{\mathbb{R}^d} f\bar{g} \),
    and momentum
    \begin{equation*}
        M_{\mu, \lambda} = \left( E_{\mu,\lambda} - \mu \|x\psi\|^2_{ \mathbb{L}^2 } \right)^2 + \mu^2 \left( \Im \int x\cdot \nabla \psi \bar{\psi} \right)^2
    ,\end{equation*}
    %
    and the same applied to any power \( (-H)^s \psi \). There also holds the non radial conservation law 
    \begin{equation*}
        \mathcal{P}_j = \frac{1}{4} \left( \Im \int \partial_{j} \psi \bar{\psi} \right)^2 + \mu^2 \left( \int x_j |\psi|^2 \right)^2, \quad j=1, 2
    .\end{equation*}
    %
\end{lemma}

\begin{proof}
    Mass conservation:
    \begin{align*}
        \partial_{t} \|\psi\|^2_{\mathbb{L}^2}
        &= \partial_{t} \int |\psi|^2 = 2 \Re \int \bar{\psi} \partial_{t} \psi = 2 \Re \int -i \bar{\psi} \left( -\mu\Delta \psi + \mu|x|^2 \psi + \lambda |\psi|^2 \psi \right) \\
        &= 2 \Re \int -i \left( \mu\bar{\psi} \Delta \psi + \mu|x|^2 |\psi|^2 + \lambda |\psi|^4  \right) = 2\mu\Re \int -i \bar{\psi} \Delta \psi\\
        &= 2 \mu \Re \int i |\nabla \psi|^2 = 0 
    .\end{align*}
    %
    Energy conservation:
    \begin{align*}
        \frac{\textd}{\textd t} E_{\mu,\lambda}
        &= \frac{1}{2} \frac{\textd}{\textd t} \left\langle -\mu\Delta \psi + \mu|x|^2 \psi + \frac{\lambda}{2} |\psi|^2 \psi, \psi \right\rangle \\
        % &= \frac{1}{2} \frac{\textd}{\textd t} \left( \langle \nabla \psi, \nabla \psi \rangle + \langle |x|^2 \psi, \psi \rangle + \langle \lambda |\psi|^2, |\psi|^2 \rangle \right) \\
        &= \frac{1}{2} \left( -2\mu\Re\langle \Delta \psi, \partial_{t} \psi \rangle + 2\mu\Re \langle |x|^2 \psi, \partial_{t} \psi \rangle + 4 \Re\left\langle \frac{\lambda}{2} |\psi|^2 \psi, \partial_{t} \psi \right\rangle \right) \\
        &= \Re \left(i \langle \partial_{t} \psi, \partial_{t} \psi \rangle\right) = 0
    .\end{align*}
    %
    For the momentum, we compute
    \begin{align}
        \frac{1}{2} \frac{\textd}{\textd t} \int |x|^2 |\psi|^2
        &= \frac{1}{2} \frac{\textd}{\textd t} \langle |x|^2 \psi, \psi \rangle = \Re \langle |x|^2 \psi, \partial_{t} \psi \rangle = \Re \langle |x|^2 \psi, i \mu\Delta \psi - i \mu|x|^2 \psi - i\lambda|\psi|^2 \psi \rangle \nonumber \\
        &= \mu\Im \langle |x|^2 \psi, \Delta \psi \rangle = \mu\Im \int |x|^2 \psi \Delta \bar{\psi} = -\mu\Im \int \nabla \bar{\psi} \cdot \nabla \left( |x|^2 \psi \right) \nonumber \\
        &= -\mu\Im \int \nabla \bar{\psi} \cdot 2x \psi - \mu\Im \int \nabla \bar{\psi} \cdot \nabla \psi |x|^2 = -2\mu\Im \int x\cdot \nabla \bar{\psi} \psi \nonumber \\
        &= 2\mu\Im \int x\cdot \nabla \psi \bar{\psi} \label{eqn: lemma conservation -- intermediate equation no 1}
    ,\end{align}
    %
    and
    \begin{equation*}
        \frac{1}{2} \frac{\textd}{\textd t} \Im \int x\cdot \nabla \psi \bar{\psi}
        = \frac{1}{2} \Im \int \left( x\cdot \nabla \partial_{t} \psi \bar{\psi} + x\cdot \nabla \psi \partial_{t} \bar{\psi} \right)
    .\end{equation*}
    %
    An integration by parts gives
    \begin{equation*}
        \int x\cdot \nabla \phi \psi = - \int \phi \nabla \cdot (x \psi) = -\int \phi \left( \psi d + x \cdot \nabla \psi \right)
    ,\end{equation*}
    %
    hence
    \begin{align*}
        \frac{1}{2} \frac{\textd}{\textd t} \Im \int x\cdot \nabla \psi \bar{\psi}
        &= \frac{1}{2} \Im \int \left( - \partial_{t} \psi \left( \bar{\psi} d + x \cdot \nabla \bar{\psi} \right) + x\cdot \nabla \psi \partial_{t} \bar{\psi} \right) \\
        &= \frac{1}{2} \Im \int \left( - \partial_{t} \psi \bar{\psi} d - \partial_{t} \psi x\cdot \nabla \bar{\psi} + \partial_{t} \bar{\psi} x\cdot \nabla \psi \right) \\
        &= \frac{1}{2} \Im \int \left( - \partial_{t} \psi \bar{\psi} d + 2 i \Im \left[ \partial_{t} \bar{\psi} x\cdot \nabla \psi \right] \right) \\
        &= -\frac{d}{2} \Im \int \partial_{t} \psi \bar{\psi} + \Im \int \partial_{t} \bar{\psi} x\cdot \nabla \psi
    .\end{align*}
    %
    Recall the equation satisfied by \( \psi \):
    \begin{equation*}
        \partial_{t} \psi = i\mu \Delta \psi - i\mu |x|^2 \psi - i \lambda |\psi|^2 \psi
    ,\end{equation*}
    %
    therefore
    \begin{align*}
        \frac{1}{2} \frac{\textd}{\textd t} \Im \int x\cdot \nabla \psi \bar{\psi}
        &= -\frac{d}{2} \Im \int i\left[ \mu\Delta \psi - \mu|x|^2 \psi - \lambda |\psi|^2 \psi \right] \bar{\psi} \\
            &\qquad + \Im \int i\left[ - \mu\Delta \bar{\psi} + \mu|x|^2 \bar{\psi} + \lambda |\psi|^2 \bar{\psi} \right] x\cdot \nabla \psi
    .\end{align*}
    %
    We have
    \begin{equation*}
        -\frac{d}{2} \Im \int i\left[ \mu\Delta \psi - \mu|x|^2 \psi - \lambda |\psi|^2 \psi \right] \bar{\psi}
        = \frac{d}{2} \int \left[ \mu|\nabla \psi|^2 + \mu |x|^2 |\psi|^2 + \lambda |\psi|^4 \right]
    ,\end{equation*}
    %
    and
    \begin{equation*}
        \Im \int i\left[ - \mu\Delta \bar{\psi} + \mu|x|^2 \bar{\psi} + \lambda |\psi|^2 \bar{\psi} \right] x\cdot \nabla \psi = \Re \int \left[ - \mu \Delta \bar{\psi} + \mu |x|^2 \bar{\psi} + \lambda |\psi|^2 \bar{\psi} \right] x\cdot \nabla \psi
    .\end{equation*}
    %
    Moreover,
    \begin{align*}
        \int |x|^2 \bar{\psi} x\cdot \nabla \psi
        = -\int \psi \nabla \cdot \left( x|x|^2 \bar{\psi} \right) &= - \int \psi \left( d |x|^2 \bar{\psi} + 2|x|^2 \bar{\psi} + x|x|^2 \cdot \nabla \bar{\psi} \right) \\
        \iff \int |x|^2 \bar{\psi} x\cdot \nabla \psi + \overline{\int |x|^2 \bar{\psi} x\cdot \nabla \psi} &= - \int \psi \left( d |x|^2 \bar{\psi} + 2|x|^2 \bar{\psi} \right)  \\
        \iff \Re \int |x|^2 \bar{\psi} x\cdot \nabla \psi &= - \int \psi \left( \frac{d}{2} |x|^2 \bar{\psi} + |x|^2 \bar{\psi} \right)
    .\end{align*}
    %
    Finally,
    \begin{align*}
        \frac{1}{2} \frac{\textd}{\textd t} \Im \int x\cdot \nabla \psi \bar{\psi}
        &= \frac{d}{2} \int \left[ \mu |\nabla \psi|^2 + \mu |x|^2 |\psi|^2 + \lambda |\psi|^4 \right] \\
            &\qquad + \Re \int \left[ - \mu \Delta \bar{\psi} + \lambda |\psi|^2 \bar{\psi} \right] x\cdot \nabla \psi - \mu \int \psi \left( \frac{d}{2} |x|^2 \bar{\psi} + |x|^2 \bar{\psi} \right) \\
        &= \frac{d}{2} \int \left[ \mu |\nabla \psi|^2 + \lambda |\psi|^4 \right] + \Re \int \left[ -\mu \Delta \bar{\psi} + \lambda |\psi|^2 \bar{\psi} \right] x\cdot \nabla \psi - \mu \int |x|^2 |\psi|^2 \\
        &= \frac{d}{2}\mu  \int |\nabla \psi|^2 - \mu \int |x|^2 |\psi|^2 + \frac{d}{2} \lambda \int |\psi|^4 + \Re \int \left[ - \mu \Delta \bar{\psi} + \lambda |\psi|^2 \bar{\psi} \right] x\cdot \nabla \psi
    \end{align*}
    %
    We are in the two-dimensional case \( d=2 \), hence
    \begin{align*}
        \frac{1}{2} \frac{\textd}{\textd t} \Im \int x\cdot \nabla \psi \bar{\psi}
        &= \int \mu |\nabla \psi|^2 - \mu \int |x|^2 |\psi|^2 + \lambda \int |\psi|^4 + \Re \int \left[ - \mu \Delta \bar{\psi} + \lambda |\psi|^2 \bar{\psi} \right] x\cdot \nabla \psi \\
        &= 2 E_\lambda + \frac{\lambda}{2} \int |\psi|^4  - 2\mu \int |x|^2 |\psi|^2 + \Re \int \left[ - \mu \Delta \bar{\psi} + \lambda |\psi|^2 \bar{\psi} \right] x\cdot \nabla \psi
    .\end{align*}
    %
    Moreover,
    \begin{align*}
        \int |\psi|^2 \bar{\psi} x\cdot \nabla \psi
        &= - \int \psi \nabla \cdot \left( |\psi|^2 \bar{\psi} x \right) \\
        &= - \int \psi \left( 2\Re \left( \bar{\psi} \nabla \psi \right) \cdot \bar{\psi} x + |\psi|^2 \nabla\bar{\psi}\cdot x + d|\psi|^2 \bar{\psi} \right) \\
        &= - \int \left( 2\Re \left( \bar{\psi} \nabla \psi \right) \cdot |\psi|^2 x + \psi |\psi|^2 \nabla\bar{\psi}\cdot x + 2|\psi|^4 \right) \\
        \iff 2 \Re \int |\psi|^2 \bar{\psi} x\cdot \nabla \psi &= - 2\Re \int \bar{\psi} \nabla \psi \cdot |\psi|^2 x - 2 \int |\psi|^4 \\
        \iff \Re \int |\psi|^2 \bar{\psi} x\cdot \nabla \psi &= - \frac{1}{2} \int |\psi|^4
    ,\end{align*}
    %
    Finally,
    \begin{align*}
        \frac{1}{2} \frac{\textd}{\textd t} \Im \int x\cdot \nabla \psi \bar{\psi} 
        &= 2 E_\lambda + \frac{\lambda}{2} \int |\psi|^4  - 2\mu \int |x|^2 |\psi|^2 - \mu \Re \int \Delta \bar{\psi} x\cdot \nabla \psi - \frac{\lambda}{2} \int |\psi|^4 \\
        &= 2 E_\lambda - 2\mu \int |x|^2 |\psi|^2 - \mu \Re \int \Delta \bar{\psi} x\cdot \nabla \psi
    \end{align*}
    %
    We then use the Pohozaev identity \eqref{eqn: pohozaev identity} in dimension \( d=2 \), which yields
    \begin{equation*}
        \Re\left( \int x\cdot \nabla \psi \Delta \bar{\psi} \right) = 0
    .\end{equation*}
    %
    Therefore,
    \begin{equation*}
        \frac{1}{2} \frac{\textd}{\textd t} \Im \int x\cdot \nabla \psi \bar{\psi}  = 2 E_{\mu,\lambda} - 2\mu \int |x|^2 |\psi|^2
    .\end{equation*}
    %

    From the conservation of the energy \( E_\lambda \) and equation \eqref{eqn: lemma conservation -- intermediate equation no 1},
    \begin{equation*}
        \frac{\textd^2}{\textd t^2} \Im \int x\cdot \nabla \psi \bar{\psi} = -16\mu^2 \Im \int x\cdot \nabla \psi \bar{\psi}
    .\end{equation*}
    %
    Hence, the conservation laws
    \begin{align*}
        &\frac{1}{16} \left( \frac{\textd}{\textd t} \left[ \Im \int x\cdot \nabla \psi \bar{\psi} \right] \right)^2 + \mu^2\left( \Im \int x\cdot \nabla \psi \bar{\psi}  \right)^2 \\
        &= \left( E_{\mu,\lambda} - \mu \|x\psi\|^2_{ \mathbb{L}^2 } \right)^2 + \mu^2 \left( \Im \int x\cdot \nabla \psi \bar{\psi} \right)^2
    \end{align*}
    %
    
    For the non radial conservation law:    
    \begin{equation*}
        \frac{\textd}{\textd t} \Im \int \partial_{j} \psi \bar{\psi} 
        = -2\Im \int \partial_{t} \psi \overline{ \partial_{j} \psi }
        = 2\Re \int i \partial_{t} \psi \overline{ \partial_{j} \psi } 
        = 2 \mu \int |x|^2 \Re \left(\psi \overline{ \partial_{j} \psi }\right)
        = -2\mu \int x_j |\psi|^2
    ,\end{equation*}
    %
    owing to the facts that integrations by parts yield
    \begin{equation*}
        -\Re \int \Delta \psi \partial_{j} \bar{\psi} = \Re \int \partial_{j} \psi \Delta \bar{\psi}
    ,\end{equation*}
    %
    and 
    \begin{equation*}
        2 \Re \int |\psi|^2 \psi \partial_{j} \bar{\psi}
        = \int |\psi|^2 \partial_{j} |\psi|^2 = - \int |\psi|^2 \partial_{j} |\psi|^2  \implies \Re \int |\psi|^2 \psi \partial_{j} \bar{\psi} = 0
    .\end{equation*}
    %
    We also have
    \begin{equation*}
        \frac{1}{2} \frac{\textd}{\textd t} \int x_j |\psi|^2
        = \Re \int x_j \partial_{t} \psi \bar{\psi} 
        = \Im \int x_j i \partial_{t} \psi \bar{\psi} = \mu \Im \int -\Delta \psi x_j \bar{\psi}
        = \mu \Im \int \partial_{j} \psi \bar{\psi}
    .\end{equation*}
    %
    Hence the relations
    \begin{equation*}
        \left|
        \begin{aligned}
            \frac{\textd}{\textd t} \int x_j |\psi|^2 &= 2\mu \Im \int \partial_{j} \psi \bar{\psi} \\
            \frac{\textd}{\textd t} \Im \int \partial_{j} \psi \bar{\psi} &= -2\mu \int x_j |\psi|^2,
        \end{aligned}
        \right.
    \end{equation*}
    %
    which have the conservation law
    \begin{equation*}
        \mathcal{P}_j = \frac{1}{4} \left( \Im \int \partial_{j} \psi \bar{\psi} \right)^2 + \mu^2 \left( \int x_j |\psi|^2 \right)^2
    .\end{equation*}
    %
\end{proof}



% \subsubsection{Galilean symmetry}
% \label{sect: HO -- Galilean symmetry}


% \begin{lemma}[Galilean symmetry]
%     If \( v \) solves \eqref{eqn: cNLS with psi -- linear part}, then so does 
%     \begin{equation*}
%         u(t,x) = v(t,y) e^{i\beta\cdot y + i\gamma(t)}, \quad y = x-x(t)
%     ,\end{equation*}
%     %
%     for
%     \begin{equation*}
%         \dot{x} = 2\beta,\quad \dot{\beta} = -2x,\quad \dot{\gamma} = |\beta|^2 - |x|^2
%     .\end{equation*}
%     %
% \end{lemma}

% \begin{remark}
%     Please note we consider now that \( u \) is solution to \eqref{eqn: cNLS with psi -- linear part} because we are looking at a solution under the form of \emph{bubbles}, whereas \( \psi \) has no particular form.
% \end{remark}

% \begin{proof}
%     We change variables:
%     \begin{equation*}
%         u(t,x) = v(t,y) e^{i\gamma}, \quad y = x-x(t)
%     ,\end{equation*}
%     %
%     then 
%     \begin{equation*}
%         i\partial_t v - \dot{\gamma}v - i\dot{x} \cdot \nabla v + \Delta v - \left| y - x(t) \right|^2 v + T(v,v,v) = 0
%     .\end{equation*}
%     %
%     Let
%     \begin{equation*}
%         \dot{x} = 2\beta,\quad v = e^{i\beta\cdot y} w
%     ,\end{equation*}
%     %
%     then
%     \begin{equation*}
%         i \partial_{t} w + \Delta w - |y|^2 w - \left[ \dot{\beta}+ 2x(t) \right]\cdot yw + i(2\beta - \dot{x}) \cdot \nabla w + \left( -\dot{\gamma} + x_t \cdot \beta  - \left( |\beta|^2 + |x|^2 \right)\right) w = 0
%     ,\end{equation*}
%     %
%     and hence the symmetry.
% \end{proof}



\subsubsection{Renormalized flow}
\label{sect: HO -- Renormalized flow}

Recall the expression \eqref{eqn: generic discretization of psi -- expression for uj} of a bubble:
\begin{equation}
    \label{eqn: HO -- Renormalized flow}
    u(t,x) = \frac{A}{L} e^{i\gamma + iL\beta \cdot y - i\frac{B}{4} |y|^2} v(s,y), \quad y = \frac{x-X(t)}{L(t)}, \, \frac{\textd s}{\textd t} = \frac{1}{L(t)^2} 
.\end{equation}
%

We compute, in dimension \( d \geq 1 \):
\begin{equation*}
    \Delta_x u = \frac{Ae^{i\gamma}}{L^3} \Delta_y \left[ e^{iL\beta\cdot y - i\frac{B}{4} |y|^2} v(s, y) \right]
,\end{equation*}
%
and 
\begin{equation}
    \partial_{k} \left[ e^{ iL\beta\cdot y - i \frac{B}{4} |y|^2 } v \right]
    = e^{iL\beta\cdot y - i \frac{B}{4} |y|^2} \left[ \partial_{k} v + i\left( L\beta_k - \frac{B}{2} y_k \right) v \right],\quad k=1, \dots, d
,\end{equation}
%
and
\begin{align*}
    &\partial_{k}^2 \left[ e^{iL\beta\cdot y - i \frac{B}{4} |y|^2} v \right]  = e^{iL\beta\cdot y - i\frac{B}{4} |y|^2} \\
    &\quad \times \left[ \partial_{k}^2 v + i \left( L\beta_k - \frac{B}{2} y_k \right) \partial_{k} v - i\frac{B}{2} v + i\left( L\beta_k - \frac{B}{2} y_k \right) \left[ \partial_{k} v + i \left( L\beta_k - \frac{B}{2} y_k \right) v \right] \right] \\
    &= e^{iL\beta\cdot y - i \frac{B}{4} |y|^2} \left[ \partial_{k}^2 v + i\left( 2L\beta_k - By_k \right) \partial_{k} v + \left( -i\frac{B}{2} - L^2\beta_k^2 + LB\beta_k y_k - \frac{B^2}{4} y_k^2 \right) v \right].
\end{align*}
%
Hence,
\begin{align*}
    \Delta_x u &= \frac{A}{L^3} e^{i\gamma + iL\beta\cdot y - i\frac{B}{4} |y|^2 } v \\
    &\quad \times \left[ \Delta_y v + i\left( 2L\beta - By \right) \cdot \nabla v + \left( -i\frac{B}{2}d  - L^2|\beta|^2 + LB\beta \cdot y - \frac{B^2}{4} |y|^2  \right) v \right].
\end{align*}
%
We have
\begin{align*}
    -|x|^2 u
    &= -\frac{A}{L} e^{i\gamma + iL\beta\cdot y - i \frac{B}{4} |y|^2} \left| Ly + X \right|^2 v \\
    &= \frac{A}{L^3} e^{i\gamma + iL\beta\cdot y - i \frac{B}{4} |y|^2} \left( -L^4 |y|^2 - 2L^3 X \cdot y - L^2 |X|^2 \right) v
,\end{align*}
%
thus
\begin{equation}
    \label{eqn: HO -- -Hu with general v}
    \begin{aligned}
        -Hu &= \frac{A}{L^3} e^{i\gamma + iL\beta\cdot y - i \frac{B}{4} |y|^2} \left\{ \Delta_y v - i B \left( \frac{d}{2}  v + \Lambda v \right) - L^2\left( |\beta|^2 + |X|^2 \right) v \right. \\
        & \qquad \left. +2iL\beta\cdot \nabla v + \left( LB\beta - 2L^3 X \right) \cdot yv + \left( -\frac{B^2}{4} -L^4 \right) |y|^2 v \right\},
    \end{aligned}
\end{equation}
%
where we denoted \( \Lambda v := y \cdot \nabla v \).
We now compute
\begin{align*}
    \partial_{t} u &= \partial_{t} \left( e^{i\gamma + i \beta\cdot (x-X) - i\frac{B}{4L^2} |x-X|^2} \frac{A}{L} v(s,y)  \right) \nonumber \\
    &\begin{aligned}
        &= e^{i\gamma + i\beta\cdot (x-X) - i\frac{B}{4L^2} |x-X|^2} \frac{A}{L} \left[ \partial_{t} v + \frac{A_t}{A} v - \frac{L_t}{L} (v + \Lambda v) - \frac{X_t}{L} \cdot \nabla v \right] \\
        &\quad + e^{i\gamma + i\beta\cdot (x-X) - i\frac{B}{4L^2} |x-X|^2} \frac{A}{L} iv \\
        &\qquad \times \left[ \gamma_t + \beta_t \cdot (x-X) - \beta\cdot X_t - \frac{B_t}{4L^2} |x-X|^2 \right. \\
        &\qquad\qquad \left. + \frac{2L_t B}{4L^3} |x-X(t)|^2 + \frac{2B}{4L^2} (x-X) \cdot X_t \right]
    \end{aligned}\\
    &\begin{aligned}
        &= e^{i\gamma + i\beta\cdot (x-X) - i\frac{B}{4L^2} |x-X|^2} \frac{A}{L^3} \left[ \partial_{s} v + \frac{A_s}{A} v - \frac{L_s}{L} (v+\Lambda v) - \frac{X_s}{L} \cdot \nabla v \right] \\
        &\quad + e^{i\gamma + i\beta\cdot (x-X) - i\frac{B}{4L^2} |x-X|^2} \frac{A}{L^3} iv\\
        &\qquad \times \left[\gamma_s + L\beta_s \cdot y - \beta \cdot X_s - \frac{B_s}{4} |y|^2 + \frac{2L_s B}{4L} |y|^2 + \frac{B}{2} y\cdot \frac{X_s}{L}  \right],
    \end{aligned}
\end{align*}
%
and hence 
\begin{equation}
    \label{eqn: i dt uj}
    \begin{aligned}
        i \partial_{t} u 
        &= e^{i\gamma + i\beta\cdot (x-X) - i\frac{B}{4L^2} |x-X|^2} \frac{A}{L^3} \left\{ i \partial_{s} v + (-\gamma_s + \beta\cdot X_s) v + \left( \frac{A_s}{A} - \frac{L_s}{L}  \right) iv - \frac{L_s}{L} i\Lambda v \right. \\
        &\qquad\qquad \left. - i\frac{X_s}{L} \cdot \nabla v + \left( -L\beta_s - \frac{BX_s}{2L} \right) \cdot yv + \left( \frac{B_s}{4} - \frac{B}{2} \frac{L_s}{L} \right) |y|^2 v \right\}.
    \end{aligned}
\end{equation}
%
This yields
\begin{equation}
    \label{eqn: HO -- idt u - Hu -- with v}
    \begin{aligned}
        i \partial_{t} u - Hu
        &= \frac{A}{L^3} e^{i\gamma + iL\beta\cdot y - i\frac{B}{4} |y|^2} \left\{ i \partial_{s} v + \left( -\gamma_s + \beta\cdot X_s - L^2 \left( |\beta|^2 + |X|^2 \right) \right) v \right. \\
        &\quad + \left( \frac{A_s}{A} - \frac{L_s}{L} - B \frac{d}{2}  \right) iv + \left( - \frac{L_s}{L} - B \right)i\Lambda v + i \left( 2L\beta - \frac{X_s}{L}  \right) \cdot \nabla v \\
        &\quad + \left( -2L^3 X + LB\beta - L\beta_s - \frac{B}{2} \frac{X_s}{L}  \right) \cdot yv \\
        &\quad + \left. \Delta_y v + \left[ \frac{B_s}{4} - \left( \frac{B^2}{4} + L^4 \right) - \frac{B}{2} \frac{L_s}{L} \right] |y|^2 v \right\}(s,y).
    \end{aligned}
\end{equation}
%




\subsubsection{Hamiltonian formulation for the free bubble}
\label{sect: HO -- Hamiltonian formulation for the free bubble}

The one free bubble corresponds to the modulation equations
\begin{equation*}
    \left| 
    \begin{aligned}
        &-\gamma_s + \beta\cdot X_s - L^2\left( |\beta|^2 + |X|^2 \right) = 0 \\
        &\frac{A_s}{A} - \frac{L_s}{L} - \frac{B}{2} d = 0  \\
        &- \frac{L_s}{L} - B = 0 \\
        &X_s = 2L^2 \beta \\
        &-2L^3X + LB\beta - L\beta_s - \frac{BX_s}{2L} = 0 \\
        &\frac{B_s}{4} - \left( \frac{B^2}{4} + L^4 \right) - \frac{B}{2} \frac{L_s}{L} = -1,
    \end{aligned}
    \right.
\end{equation*}
%
or equivalently
\begin{equation}
    \label{eqn: modulation ODEs -- linear part wrt time s - with v}
    \left|
        \begin{aligned}
            A_s &= \frac{AB}{2} (d-2) \\
            L_s &= -BL \\
            B_s &= -4 + 4L^4 - B^2 \\
            X_s &= 2L^2 \beta \\
            \beta_s &= -2L^2 X \\
            \gamma_s &= L^2 \left( |\beta|^2 - |X|^2 \right).
        \end{aligned}
    \right.
\end{equation}
%
If the parameters are solutions to system \eqref{eqn: modulation ODEs -- linear part wrt time s - with v}, the function \( v \) then has to solve the Harmonic Oscillator with respect to time \( s \) and space \( y \).

In time \( t \), as \( \frac{\textd}{\textd s} = L^2 \frac{\textd}{\textd t}  \), this system is 
\begin{equation}
    \label{eqn: modulation ODEs -- linear part wrt time t - with v}
    \left| 
        \begin{aligned}
            A_t &= \frac{AB}{2L^2} (d-2) \\
            L_t &= - \frac{B}{L} = -2L \partial_{B} \mathcal{E} \\
            B_t &= -\frac{4}{L^2} + 4L^2 - \frac{B^2}{L^2} = 2L \partial_{L} \mathcal{E} \\
            X_t &= 2\beta = \nabla_\beta \mathcal{R} \\
            \beta_t &= -2X = -\nabla_X \mathcal{R} \\
            \gamma_t &=  |\beta|^2 - |X|^2,
        \end{aligned}
    \right.
\end{equation}
%
with
\begin{equation*}
    \mathcal{E}(B, L) = \frac{1}{L^2} \left( 1 + \frac{B^2}{4}  \right) + L^2, \quad \text{ and } \quad 
    \mathcal{R}(X, \beta) = |X|^2 + |\beta|^2
.\end{equation*}
%
Then we set
\begin{equation*}
    k = \frac{1}{2} \log L, \quad L = e^{2k}
,\end{equation*}
%
and the system becomes
\begin{equation}
    \label{eqn: modulation ODEs -- linear part wrt time t - with hamiltonian and k}
    \left| 
        \begin{aligned}
            A_t &= \frac{AB}{2} (d-2) e^{-4k}\\
            k_t &= - \partial_{B} \mathcal{H} \\
            B_t &= \partial_{k} \mathcal{H} \\
            X_t &= \nabla_\beta \mathcal{H} \\
            \beta_t &= - \nabla_X \mathcal{H} \\
            \gamma_t &=  |\beta|^2 - |X|^2,
        \end{aligned}
    \right.
\end{equation}
%
with
\begin{equation*}
    \mathcal{H}(k, B, X, \beta) = \mathcal{E}(k, B) + \mathcal{R}(X, \beta) = e^{-4k} \left( \frac{B^2}{4} + 1 \right) + e^{4k} + |X|^2 + |\beta|^2
.\end{equation*}
%




\begin{lemma}
    \label{lemma: HO exact integration of parameters}
    There exists a symplectic change of variable \( (X, B, k, \beta) \mapsto (h, a, \xi, \theta) \in \mathbb{R}\times \mathbb{R}^d  \times [0, 2\pi] \times [0, 2\pi]^d \), such that the Hamiltonian in these variables is given by
    \begin{equation}
        E(h, a, \xi, \theta) = 4h + 2|a|^2
    ,\end{equation}
    %
    so that the flow in variable \( (h, a, \xi, \theta) \) is given by
    \begin{equation}
        \label{eqn: lemma -- HO exact integration of parameters - update of action-angle variables}
        \begin{aligned}
            a(t) &= a(0), \\
            \theta(t) &= \theta(0) + 2t, \\
            h(t) &= h(0), \\
            \xi(t) &= \xi(0) - 4t.
        \end{aligned}
    \end{equation}
    % 
    We have the explicit formulae:
    \begin{equation}
        \label{eqn: lemma -- HO exact integration of parameters - update of parameters - with v}
        \begin{aligned}
            A(t) &= A(0)  \left( \frac{L(t)}{L(0)}  \right)^{\frac{2-d}{2}}, \\
            e^{4k(t)} &= L(t)^2 = 2h(t) - \cos(\xi(t)) \sqrt{4h(t)^2 - 1}, \\
            B(t) &= 2\sin(\xi(t))\sqrt{4h(t)^2-1}, \\
            X_i(t) &= \sin(\theta_i(t)) \sqrt{2a_i(t)}, \quad i=1, \dots, d,\\
            \beta_i(t) &= \cos(\theta_i(t)) \sqrt{2a_i(t)}, \quad i=1, \dots, d,\\
            \gamma(t) &= \gamma(0) + \sum_{l=1}^d a_l(0) \left[ \sin(2\theta_l(t)) - \sin(2\theta_l(0)) \right]
        \end{aligned}
    \end{equation}
    %
\end{lemma}
%
%% Lemme complètement copié-collé depuis l'article non publié d'Erwan et Pierre (Proposition 3).

\begin{proof}
    For the \( (X, \beta) \) part, it suffices to check that 
    \begin{equation*}
        X_i = \sqrt{2a_i(0)} \sin(2t + \theta_i(0)) \quad \text{ and } \quad
        \beta_i = \sqrt{2a_i(0)} \cos(2t + \theta_i(0))\qquad i=1,\dots, d
    ,\end{equation*}
    %
    are solutions. The expression for \( \gamma(t) \) will be computed later within the proof of Lemma \ref{lemma: HO -- integration of gamma with v gaussian}.
    For the \( (k, B) \) part we use the method of generating functions, described e.g. in \cite[Sect.~VI.5]{hairerGeometricNumericalIntegration2006}.
    We can express \( B \) in terms of \( k \) and the Hamiltonian \( \mathcal{E} \), so that on the set \( \{B > 0\} \) we have:
    \begin{equation}
        \label{eqn: proof HO -- definition of B in terms of CAL E and k}
        B = 2\sqrt{e^{4k} \mathcal{E} - e^{8k} - 1}
    .\end{equation}
    %
    This equality holds for \( e^{4k} \in [e^{4k_0}, e^{4k_1}] \), where \( e^{4k_0}, e^{4k_1} \) are the reals roots of the polynomial \( -z^2 + \mathcal{E} z - 1 \),
    \begin{equation}
        \label{eqn: HO proof -- action angle variable - definition of k0 and k1}
        e^{4k_0} = \frac{1}{2} \left( \mathcal{E} - \sqrt{ \mathcal{E}^2 - 4 } \right),\quad e^{4k_1} = \frac{1}{2} \left( \mathcal{E} + \sqrt{ \mathcal{E}^2 - 4 } \right)
    .\end{equation}
    %

    In order to obtain a symplectic change of variables, we look for a function \( S(k, \mathcal{E}) \) such that 
    \begin{equation*}
        B = \frac{\partial S}{\partial k} (k, \mathcal{E})
    .\end{equation*}
    %
    We easily obtain \( S(k, \mathcal{E}) \), by integrating on \( [k_0, k] \):
    \begin{equation*}
        S(k, \mathcal{E}) = 2 \int_{k_0}^k \sqrt{e^{4z} \mathcal{E} - e^{8z} - 1} dz
    .\end{equation*}
    %
    The variable \( \phi \) which makes the mapping \( (B, k) \mapsto (\phi, \mathcal{E}) \) symplectic is defined by
    \begin{equation*}
        \phi = \frac{\partial S}{\partial \mathcal{E}}(k, \mathcal{E}) = \int_{k_0}^k \frac{e^{4z}}{\sqrt{e^{4z} \mathcal{E} - e^{8z} - 1}} dz
    .\end{equation*}
    %

    We have 
    \begin{equation*}
        \frac{\textd \phi}{\textd t} = \frac{e^{4k} k_t}{\sqrt{e^{4k} - e^{8k} - 1}} = \frac{- e^{4k} \partial_{B} \mathcal{E}}{\frac{B}{2} } =  \frac{- e^{-4k} \frac{B}{2} e^{4k}}{\frac{B}{2} } = -1
    .\end{equation*}
    %
    We now proceed to obtaining an explicit expression for \( \psi \):
    \begin{align*}
        \phi &= \int_{k_0}^k \frac{e^{4z}}{\sqrt{e^{4z} \mathcal{E} - e^{8z} - 1}} dz = \frac{1}{4} \int_{e^{4k_0}}^{e^{4k}} \frac{1}{\sqrt{ \mathcal{E}u - u^2 - 1 }} du \\
        &= \frac{1}{4 \sqrt{\frac{ \mathcal{E}^2 }{4} - 1 }}  \int_{e^{4k_0}}^{e^{4k}} \frac{1}{\sqrt{ 1 - \left( \frac{u - \frac{ \mathcal{E} }{2} }{\sqrt{ \frac{ \mathcal{E}^2 }{4} - 1  }}  \right)^2 }} du \\
        &= \frac{1}{4} \int_{ \frac{e^{4k_0} - \frac{ \mathcal{E} }{2} }{\sqrt{ \frac{ \mathcal{E}^2 }{4} - 1 }}  }^{ \frac{e^{4k} - \frac{ \mathcal{E} }{2} }{\sqrt{ \frac{ \mathcal{E}^2 }{4} - 1 }}  } \frac{1}{\sqrt{1 - u^2}} du.
    \end{align*}
    %
    Recall the definition \eqref{eqn: HO proof -- action angle variable - definition of k0 and k1} of \( k_0 \), which yields
    \begin{equation*}
        e^{4k_0} - \frac{ \mathcal{E} }{2} = - \sqrt{ \frac{\mathcal{E}^2}{4} - 1 } 
    .\end{equation*}
    %
    Therefore, 
    \begin{align*}
        \phi &= \frac{1}{4} \int_{-1}^{ \frac{e^{4k} - \frac{ \mathcal{E} }{2} }{\sqrt{ \frac{ \mathcal{E}^2 }{4} - 1 }}  } \frac{1}{\sqrt{1 - u^2}} du = \frac{1}{4} \left( \arcsin\left( { \frac{e^{4k} - \frac{ \mathcal{E} }{2} }{\sqrt{ \frac{ \mathcal{E}^2 }{4} - 1 }}  }  \right) + \frac{\pi}{2}  \right) \nonumber \\
        &= \frac{1}{4} \arcsin\left( { \frac{e^{4k} - \frac{ \mathcal{E} }{2} }{\sqrt{ \frac{ \mathcal{E}^2 }{4} - 1 }}  }  \right) + \frac{\pi}{8} \in \left[ 0, \frac{\pi}{4}  \right]
    .\end{align*}
    %
    We want the angle variable to lie in \( \left[ 0, 2\pi \right] \) so the above expression describes an eigth of a period. But we are only considering the set \( \{B>0\} \), thus the angle \( \xi \) we are looking for must lie only in \( [0, \pi] \). Hence we set \( (\xi, h) = (4\phi, \mathcal{E}/4) \) and let the Hamiltonian \( \mathcal{E}(\xi, h) = 4h \) with a slight abuse of notation. It is then clear that \( \frac{\textd h}{\textd t} = 0 \) and \( \frac{\textd \xi}{\textd t} = -4 \). Moreover,
    \begin{equation}
        \label{proof: HO proof -- definition of xi with k}
        \xi = \arcsin\left( \frac{e^{4k} - \frac{ \mathcal{E} }{2} }{\sqrt{ \frac{ \mathcal{E}^2 }{4} - 1 }} \right) + \frac{\pi}{2}
        \in \left[ 0, \pi  \right]
    ,\end{equation}
    %
    and hence
    \begin{equation*}
        \frac{e^{4k} - \frac{ \mathcal{E} }{2} }{\sqrt{ \frac{ \mathcal{E}^2 }{4} - 1 }} = \sin\left( \xi - \frac{\pi}{2} \right)
        = - \cos\left( \xi \right)
    .\end{equation*}
    %
    We obtain 
    \begin{align*}
        e^{4k} = L^2 &= \frac{ \mathcal{E}}{2} - \cos(\xi) \sqrt{ \frac{ \mathcal{E}^2 }{4} - 1 } = 2h - \cos(\xi) \sqrt{4 h^2 - 1} \\
        &= 2h \left( 1 - \cos(\xi) \sqrt{1 - \frac{1}{4h^2}} \right)
    .\end{align*}
    %
    
    With this formula, we have
    \begin{equation*}
        0 < L^2 < 4h = \mathcal{E}
    ,\end{equation*}
    %
    and \eqref{eqn: proof HO -- definition of B in terms of CAL E and k} becomes
    \begin{align*}
        B &= 2\sqrt{ \mathcal{E}e^{4k} - e^{8k} - 1} = 2\sqrt{4he^{4k} - (e^{4k})^2 - 1} = 2\sqrt{ (4h^2 - 1)\sin^2(\xi) } \\
        &= 2 \sin(\xi) \sqrt{4h^2 - 1},
    \end{align*}
    %
    where the last equality holds for \( \xi \in [0, \pi] \).

    Finally, it remains to integrate the ODE on \( A \), which is
    \begin{equation*}
        A_t = \frac{AB}{2} (d-2) e^{-4k}
    .\end{equation*}
    %
    From the expressions we just obtained we get
    \begin{equation*}
        A_t = A (d-2) \frac{\sin(\xi) \sqrt{4h^2 - 1}}{2h - \cos(\xi) \sqrt{ 4 h^2 - 1 }}
    .\end{equation*}
    %
    The solution to this equation is of the form
    \begin{equation*}
        A(t) = A(0) \exp\left\{(d-2) \int_{0}^t \frac{\sin(\xi(s)) \sqrt{4h(s)^2 - 1}}{2h(s) - \cos(\xi(s)) \sqrt{ 4 h(s)^2 - 1 }} \textd s\right\}
    .\end{equation*}
    %
    Moreover, we know that \( s\mapsto h(s) \) is constant, and that \( \xi(s) = \xi(0) - 4s \). Hence we have to solve
    \begin{equation*}
        A(t) = A(0) \exp\left\{(d-2) \int_{0}^t \frac{\sin(\xi(0) - 4s) \sqrt{4h(0)^2 - 1}}{2h(0) - \cos(\xi(0) - 4s) \sqrt{ 4 h(0)^2 - 1 }} \textd s\right\}
    .\end{equation*}
    %
    One can easily check that we have the following equality:
    \begin{align*}
        &\int_{0}^t \frac{\sin(\xi(0) - 4s) \sqrt{4h(0)^2 - 1}}{2h(0) - \cos(\xi(0) - 4s) \sqrt{ 4 h(0)^2 - 1 }} \textd s \\
        &= -\frac{1}{4} \left[ \log\left( 2h(t) - \cos(\xi(t)) \sqrt{4h(t)^2 - 1} \right) - \log\left( 2h(0) - \cos(\xi(0)) \sqrt{4h(0)^2 - 1} \right) \right].
    \end{align*}
    %
    Note that, unless \( h(0) = \frac{1}{2}  \) or \( h(t) = \frac{1}{2} \), these quantities are well-defined since \( 2h(s) > \sqrt{4h(s)^2 - 1}, s\in \{0, t\} \).
    Thus, we obtain 
    \begin{align*}
        A(t) &= A(0) e^{ \frac{2-d}{4} \left[ \log\left( 2h(t) - \cos(\xi(t)) \sqrt{4h(t)^2 - 1} \right) - \log\left( 2h(0) - \cos(\xi(0)) \sqrt{4h(0)^2 - 1} \right) \right] } \\
        &= C \left( 2h(0) - \cos(\xi(0) - 4t) \sqrt{4h(0)^2 - 1} \right)^{\frac{2-d}{4}},
    \end{align*}
    %
    where we defined \( C :=  A(0) \left( 2h(0) - \cos(\xi(0)) \sqrt{4h(0)^2 - 1} \right)^{\frac{d-2}{4}} \). We recognize here the expressions for \( L(0)^2 \) and \( L(t)^2 \), hence the claimed result.
\end{proof}
%




In pratice, one would know the parameters \( (A, L, B, X, \beta, \gamma) \) and would apply \eqref{eqn: lemma -- HO exact integration of parameters - update of parameters - with v} in order to update them. We have the following result, which allows to obtain action-angle variables from the bubble's parameters:
\begin{lemma}
    \label{lemma: HO change of variables between params and action-angle variables}
    The change of variables \( (L, B, X, \beta) \mapsto (h, a, \xi, \theta) \) is explicit, and at time \( t=0 \) we have
    \begin{equation}
        \label{eqn: lemma -- HO exact integration of parameters - change of variables}
        \begin{aligned}
            a_i(0) &= \frac{1}{2} \left( X_i(0)^2 + \beta_i(0)^2 \right), i=1, \dots, d, \\
            \theta_i(0) &= \arctan\left( \frac{X_i(0)}{\beta_i(0)}  \right), i=1, \dots, d, \\
            h(0) &= \frac{L(0)^4 + 1 + \frac{B(0)^2}{4}}{4L(0)^2}, \\
            \xi(0) &= \arctan\left( \frac{B(0)}{4h(0) - 2L(0)^2}  \right),
        \end{aligned}
    \end{equation}
    %
    whenever \( \theta(0) \) and \( \xi(0) \) are well-defined. When any one of them is ill-defined -- which happens when \( X_i(0) = \beta_i(0) = 0, i\in \{1, \dots, d\} \) or when \( L(0)=1 \text{ and } B(0)=0 \) -- any value can be taken and the time-evolution of \( A(t), L(t), B(t), X(t) \) and \( \beta(t) \) will not depend on the value.
\end{lemma}


Note that to define \( \xi(0) \) we could also use equation \eqref{proof: HO proof -- definition of xi with k}, but this is not appropriate from a computational point of view. Some more details are given in Remark \ref{rmk: HO -- numerical considerations}.


\begin{proof}
    We have \( a_i(0) = \frac{1}{2} \left( X_i(0)^2 + \beta_i(0)^2 \right), i=1, \dots, d \). If \( a_i(0) > 0 \) we can define \( \theta_i(0) \) as \( \theta_i(0) = \arctan \left( \frac{X_i(0)}{\beta_i(0)}  \right) \). Otherwise, if \( a_i(0) = 0 \), then we recall that \( a(t) = a(0) \) and hence -- whatever \( \theta(0) \) -- we have \( X_i(t) = 0 \) and \( \beta_i(t) = 0 \). Therefore, in the case \( a_i(0) = 0 \), the exact value of \( \theta_i(0) \) does not change the behavior of \( t\mapsto (X_i(t), \beta_i(t)) \).
    
    For the \( (L, B) \) part,
    \begin{equation*}
        L(0)^2 - 2h(0) = -\cos(\xi(0)) \sqrt{4h(0)^2 - 1}
    ,\end{equation*}
    %
    hence
    \begin{equation*}
        (L(0)^2 - 2h(0))^2 = L(0)^4 - 4L(0)^2 h(0) + 4h(0)^2 = \cos(\xi(0))^2 \left( 4 h(0)^2 - 1 \right)
    .\end{equation*}
    %
    We also have
    \begin{equation*}
        \left( \frac{B(0)}{2} \right)^2 = \frac{B(0)^2}{4} = \sin(\xi(0))^2 \left( 4h(0)^2 - 1 \right) 
    .\end{equation*}
    %
    Then,
    \begin{equation*}
        L(0)^4 - 4L(0)^2 h(0) + 4h(0)^2 + \frac{B(0)^2}{4} = 4h(0)^2 - 1
    ,\end{equation*}
    %
    that is
    \begin{equation*}
        4L(0)^2 h(0) = L(0)^4 + \frac{B(0)^2}{4} + 1 
    .\end{equation*}
    %
    We deduce that \( h(0), L(0) \neq 0 \), and therefore
    \begin{equation*}
        h(0) = \frac{L(0)^4 + \frac{B(0)^2}{4} + 1}{4L(0)^2} 
    .\end{equation*}
    %
    Note that \( h(0) \) is bounded from below by \( \frac{1}{2}  \). Indeed,
    \begin{align*}
        &L(0)^4 - 2L(0)^2 + 1 + \frac{B(0)^2}{4} 
        = \left( L(0)^2 - 1 \right)^2 + \frac{B(0)^2}{4} \geq 0 \\
        \iff& L(0)^4 + 1 + \frac{B(0)^2}{4} \geq 2L(0)^2\\
        \iff& h(0) \geq \frac{1}{2}.
    \end{align*}
    %
    From this we also get that \( h(0) = \frac{1}{2} \iff L(0)^2 = 1  \) and \( B(0) = 0 \).
    
    If \( h(0) > \frac{1}{2}  \), we have
    \begin{equation*}
        \left\{ \begin{aligned}
            2h(0) - L(0)^2 &= \cos(\xi(0)) \sqrt{4h(0)^2 - 1}\\
            \frac{B(0)}{2} &= \sin(\xi(0)) \sqrt{4h(0)^2 - 1},
        \end{aligned} \right.
        \implies \frac{B(0)/2}{2h(0) - L(0)^2} = \tan(\xi(0))
    ,\end{equation*}
    %
    hence
    \begin{equation*}
        \xi(0) = \arctan\left( \frac{B(0)/2}{2h(0) - L(0)^2} \right)
    .\end{equation*}
    %
    Otherwise, in the case \( h(0) = \frac{1}{2} \), the value of \( \xi(0) \) is not rigourously defined. However, as previously, the exact value of \( \xi(0) \) is not important because \( h(t) = h(0) = \frac{1}{2} \), which means that \( L(t)^2=1 \) and \( B(t) = 0 \). Therefore, in the case \( h(0) = \frac{1}{2} \), the mapping \( t\mapsto (L(t), B(t)) \) does not depend on the value of \( \xi(0) \). Finally, since the mapping \( t\mapsto L(t) \) does not depend on \( \xi(0) \) in the case \( h(0) = \frac{1}{2} \), we also have that \( t\mapsto A(t)\) does not depend on the exact value of \( \xi(0) \), thanks to the expression of \( A(t) = A(0) \left( L(t) / L(0)  \right)^{\frac{2-d}{2}}  \).
\end{proof}



%
%\subsubsection{Computation of Sobolev norms}
%
%We let
%\begin{equation*}
%    u(x) = \frac{A}{L} e^{i\gamma + i L\beta\cdot y - i \frac{B}{4} |y|^2} v(s,y)
%,\end{equation*}
%%
%and compute the conservation laws. 
%
%
%\begin{lemma}
%    \label{lemma: HO -- computation of sobolev norms}
%    Assume \( v \) is real-valued and radial, then 
%    \begin{equation}
%        \|u\|_{ \mathbb{L}^2 } = A \|v\|_{ \mathbb{L}^2 }
%    ,\end{equation}
%    %
%    and 
%    \begin{equation*}
%        E(u) = \frac{A^2 L^{d-4}}{2} \left\{ \left( |\beta|^2 + |X|^2 \right)\|v\|^2_{ \mathbb{L}^2 } +
%        \|\nabla v\|_{ \mathbb{L}^2 }^2 + \left( \frac{B^2}{4} + L^4 \right) \|yv\|^2_{ \mathbb{L}^2 } + \frac{A^2}{2}\|v\|^4_{ \mathbb{L}^4 } \right\}.
%    \end{equation*}
%    %
%    and 
%    \begin{equation*}
%        \mathcal{P} = A^4 L^{2(d-2)} \left( |\beta|^2 + |X|^2 \right) \|v\|^4_{ \mathbb{L}^2 }
%    .\end{equation*}
%    %
%\end{lemma}
%
%
%% \begin{remark}
%%     Note that the energy bound implies
%%     \begin{equation*}
%%         A^2 \mathcal{E} \lesssim 1
%%     ,\end{equation*}
%%     %
%%     and hence the typically expected blow-up regime is 
%%     \begin{equation*}
%%         A \to 0, \quad \mathcal{E} \sim \frac{1}{A^2} 
%%     .\end{equation*}
%%     %
%%     % We probably expect this with \( A \) being oscillatory again.
%% \end{remark}
%
%
%\begin{proof}
%    The \( \mathbb{L}^2 \) norm of \( u \) is trivial. We now compute the kinetic momentum:
%    \begin{align*}
%        \Im \int_{ \mathbb{R}^d } \partial_{j} u \bar{u} 
%        &= \frac{A^2}{L^2} \int_{ \mathbb{R}^d } \frac{1}{L} \left( L\beta_j - \frac{B}{2} y_j \right)|v(y)|^2 L^d dy \\
%        &= \beta_j A^2 L^{d-2} \|v\|^2_{ \mathbb{L}^2 } - \frac{A^2 B}{2} L^{d-3} \int_{ \mathbb{R}^d } y_j |v|^2,
%    \end{align*}
%    %
%    and
%    \begin{equation*}
%        \int_{ \mathbb{R}^d } x_j |u|^2 = A^2 L^{d-1} \int_{ \mathbb{R}^d } y_j |v(y)|^2 dy + A^2 X_j L^{d-2} \|v\|^2_{ \mathbb{L}^2 }
%    .\end{equation*}
%    %
%    Hence, as \( v \) is symmetric,
%    \begin{equation*}
%        \mathcal{P}_j = A^4 L^{2(d-2)} \left( |\beta_j|^2 + |X_j|^2 \right) \|v\|^4_{ \mathbb{L}^2 }
%    .\end{equation*}
%    %
%    We have by \eqref{eqn: HO -- -Hu with general v}:
%    \begin{equation*}
%        \begin{aligned}
%            -Hu &= \frac{A}{L^3} e^{i\gamma + iL\beta\cdot y - i \frac{B}{4} |y|^2} \left\{ \Delta_y v - i B \left( \frac{d}{2}  v + \Lambda v \right) - L^2\left( |\beta|^2 + |X|^2 \right) v \right. \\
%            & \qquad \left. +2iL\beta\cdot \nabla v + \left( LB\beta - 2L^3 X \right) \cdot yv + \left( -\frac{B^2}{4} -L^4 \right) |y|^2 v \right\},
%        \end{aligned}
%    \end{equation*}
%    %
%    and hence as \( v \) is real-valued and symmetric,
%    \begin{equation*}
%        \Re \int_{ \mathbb{R}^d } -Hu \bar{u} = A^2 L^{d-4} \left\{ -L^2 \left( |\beta|^2 + |X|^2 \right) \|v\|^2_{ \mathbb{L}^2 }
%        + \int_{ \mathbb{R}^d } \left[ \Delta - \left( \frac{B^2}{4} + L^4 \right) |y|^2 \right] v\bar{v} dy \right\}
%    ,\end{equation*}
%    %
%    which yields
%    \begin{align*}
%        E(u) &= \frac{1}{2} \langle Hu, u \rangle + \frac{1}{4} \int_{ \mathbb{R}^d } |u|^4 \\
%        &= \frac{A^2 L^{d-4}}{2} \left\{ \left( |\beta|^2 + |X|^2 \right)\|v\|^2_{ \mathbb{L}^2 } +
%        \|\nabla v\|_{ \mathbb{L}^2 }^2 + \left( \frac{B^2}{4} + L^4 \right) \|yv\|^2_{ \mathbb{L}^2 } + \frac{A^2}{2}\|v\|^4_{ \mathbb{L}^4 } \right\}.
%    \end{align*}
%    %
%\end{proof}
%
%
%
%
%
%
%
%







\subsection{A particular case: \( v(s,y) = e^{-\frac{1}{2} |y|^2} \)}
\label{sect: HO -- ODEs on the modulation parameters with particular vj}


When we have chosen the modulation equations \eqref{eqn: modulation ODEs -- linear part wrt time t - with v}, we chose them so that \( v \) is itself solution to the Harmonic Oscillator in time \( s \) and space \( y \). This could maybe ease the process of finding a solution, but we can do even better: if we consider a particular expression for the function \( v \) we are also able to compute its time derivative and its laplacian in \eqref{eqn: HO -- idt u - Hu -- with v}. By doing so, we can take into account those operators within new modulation equations. 

We then have to choose a particular shape for the functions \( v \). Note that the treatment of each bubble \( j \) is done separately, hence each function \( v_j \) can have its own shape different from the others. For simplicity, we take all the functions \( v_j \) to be gaussian functions. There are two main reasons for this choice: the first one is that when the width of a normalized gaussian function tend to zero, we can assimilate the function to a Dirac mass and hence view the proposed algorithm as an exact particle algorithm. The second reason is that we will have to consider the shapes of the functions \( v_j \) when we will add cubic interactions, and the gaussian functions allow us to perform most of the computations exactly, which greatly reduces the computational complexity and improves its accuracy. The gaussian shape is also motivated by the idea of Gaussian Beams.

From now on, the bubbles will have the following expression:
\begin{equation}
    \label{eqn: generic discretization of psi -- expression for uj -  vj gaussian}
    u_j(t,x) := \frac{A_j}{L_j} e^{i\gamma_j + i L_j \beta_j \cdot y_j - i \frac{B_j}{4} |y_j|^2} e^{- \frac{1}{2} |y_j|^2}, \quad \text{with } \quad y_j := \frac{x - X_j}{L_j}
.\end{equation}


Since equation \eqref{eqn: cNLS with psi -- linear part} is linear with respect to \( \psi \), if we use the ansatz \eqref{eqn: generic discretization of psi -- ansatz} then we have to solve the Harmonic Oscillator equation for each bubble \( j \):
\begin{equation}
    \label{eqn: harmonic oscillator -- bubble j}
    i \partial_{t} u_j = -\Delta_x u_j + |x|^2 u_j
\end{equation}
%
We can reuse equation \eqref{eqn: HO -- idt u - Hu -- with v} with \( v_j(s_j, y_j) = e^{-\frac{|y_j|^2}{2}}  \), and take advantage of the following equality:
\begin{equation*}
    \Delta_y v(y) = \Delta_y \left( e^{-\frac{|y|^2}{2} } \right) = \left( |y|^2 - d \right) v
\end{equation*}
%
This yields almost the same system of equations as \eqref{eqn: modulation ODEs -- linear part wrt time s - with v}: only the equation on \( \gamma_s \) changes, and we have
\begin{equation*}
    (\gamma_j)_s = L^2 \left( |\beta|^2 - |X|^2 \right) - d
.\end{equation*}
%
With respect to time \( t \), we get 
\begin{equation}
    \label{eqn: HO -- ODE on gamma}
    (\gamma_j)_t = |\beta|^2 - |X|^2 - \frac{d}{L^2} .
\end{equation}
%

If the parameters satisfy the modulation equations \eqref{eqn: modulation ODEs -- linear part wrt time t - with v} (where the equation on \( \gamma_t \) has been changed to \eqref{eqn: HO -- ODE on gamma}), we have
\begin{equation*}
    i \partial_{t} u_j - Hu_j = 0
,\end{equation*}
%
{\em i.e.} the Harmonic oscillator equation is automatically satisfied for the bubble \( j \).
Furthermore, we have the following result:
\begin{lemma} 
    \label{lemma: HO -- integration of gamma with v gaussian}
    Equation \eqref{eqn: HO -- ODE on gamma} can be integrated explicitely by using equation \eqref{eqn: lemma -- HO exact integration of parameters - update of parameters - with v}, and we have 
    \begin{equation}
        \label{eqn: HO -- explicit expression of gamma}
        \begin{aligned}
            \gamma(T) &= 
                \gamma(0) + \sum_{l=1}^d \frac{a_l(0)}{2} \left[ \sin(2\theta_l(T)) - \sin(2\theta_l(0)) \right] \\
                &\qquad + \frac{d}{2} \arctan\left( \left( 2h(0) + \sqrt{4h(0)^2 - 1} \right) \tan\left( \frac{\xi(0)}{2} - 2T \right) \right) \\
                &\qquad - \frac{d}{2} \arctan\left( \left( 2h(0) + \sqrt{4h(0)^2 - 1} \right) \tan\left( \frac{\xi(0)}{2} \right) \right) -m_T \frac{\pi d}{2} ,
        \end{aligned}
    \end{equation}
    %
    where, if \( m_0 \in \mathbb{Z} \) is such that \( \frac{\xi(0)}{2} \in m_0\pi + \left[ -\frac{\pi}{2}, \frac{\pi}{2}  \right] \), then \( m_T \in \mathbb{Z} \) is defined by \( \frac{\xi(T)}{2} \in (m_0-m_T)\pi + \left[ -\frac{\pi}{2}, \frac{\pi}{2}  \right] \). 
    
    Moreover, in the cases where \( a_i(0) = 0,\, i\in\{1, \dots, d\} \) or \( h(0) = \frac{1}{2} \), the formula \eqref{eqn: lemma -- HO exact integration of parameters - change of variables} for \( \theta_i(0),\, i\in \{1, \dots, d\} \) or \( \xi(0) \) are ill-defined, but any value can be taken as a substitution and this will not affect the behavior of the mapping \( t\mapsto \gamma(t) \).
\end{lemma}


\begin{proof}
    We now proceed to the direct integration of \( \gamma_t \).
    \begin{equation*}
        \gamma(T) - \gamma(0) = \int_0^T \dot{\gamma}(t)dt = \int_0^T \left[ |\beta(t)|^2 - |X(t)|^2 - \frac{d}{L(t)^2}  \right]dt
    .\end{equation*}
    %
    We have
    \begin{align*}
        &\int_0^T \left[ |\beta(t)|^2 - |X(t)|^2 \right] dt \\
        &= \int_0^T \left\{ \sum_{l=1}^d 2a_l \cos(\theta_l(t))^2 - \sum_{l=1}^d 2a_l\sin(\theta_l(t))^2\right\} dt\\
        &= \int_0^T 2\sum_{l=1}^d a_l \left( \cos(\theta_l(t))^2 - \sin(\theta_l(t))^2 \right) dt \\
        &= \int_0^T \sum_{l=1}^d 2a_l \cos(2\theta_l(t)) dt \\
        &= \sum_{l=1}^d \frac{a_l}{2} \left[ \sin(2\theta_l(T)) - \sin(2\theta_l(0)) \right],
    \end{align*}
    %
    where the last equality has been obtained using \eqref{eqn: lemma -- HO exact integration of parameters - update of action-angle variables}. Owing to \eqref{eqn: lemma -- HO exact integration of parameters - update of parameters - with v},
    \begin{align*}
        \int_0^T \frac{d}{L(t)^2} dt &= \int_0^T \frac{d}{\underbrace{2h(0)}_{=:c_1} - \underbrace{\sqrt{4h(0)^2-1}}_{=:c_2} \cos(\xi(0)-4t)} dt\\
        &= \int_0^T \frac{d}{c_1 - c_2 \cos(\xi(0) - 4t)} dt \\
        &= \frac{d}{4} \int^{\xi(0)}_{\xi(0)-4T} \frac{1}{c_1 - c_2 \cos(t)} dt.
    \end{align*}
    %
    Recall the following trig. identity:
    \begin{equation*}
        \cos(2\tau) = \frac{1 - \tan(\tau)^2}{1 + \tan(\tau)^2}, \quad \tau \in \mathbb{R}
    ,\end{equation*}
    %
    hence
    \begin{align*}
        &\int_0^T \frac{d}{c_1 - c_2 \cos(\xi(0) - 4t)} dt \\
        &= \frac{d}{4} \int^{\xi(0)}_{\xi(0)-4T} \frac{1}{c_1 - c_2 \frac{1 - \tan(t/2)^2}{1 + \tan(t/2)^2} } dt \\
        &= \frac{d}{4} \int^{\xi(0)}_{\xi(0)-4T} \frac{1 + \tan(t/2)^2}{c_1(1 + \tan(t/2)^2) - c_2(1 - \tan(t/2)^2) } dt \\
        &= \frac{d}{4} \int^{\xi(0)}_{\xi(0)-4T} \frac{1 + \tan(t/2)^2}{(c_1 + c_2)\tan(t/2)^2 + c_1 - c_2} dt \\
        &= \frac{d}{4(c_1 - c_2)} \int^{\xi(0)}_{\xi(0)-4T} \frac{1 + \tan(t/2)^2}{\frac{c_1 + c_2}{c_1-c_2}\tan(t/2)^2 + 1} dt \\
        &= \frac{d}{2(c_1 - c_2)} \int^{\frac{\xi(0)}{2}}_{\frac{\xi(0)}{2}-2T} \frac{1 + \tan(t)^2}{\frac{c_1 + c_2}{c_1-c_2}\tan(t)^2 + 1} dt \\
        &= \frac{d}{2(c_1 - c_2)} \int^{\frac{\xi(0)}{2}}_{\frac{\xi(0)}{2}-2T} \frac{ \frac{d}{dt} (\tan(t)) }{\frac{c_1 + c_2}{c_1-c_2}\tan(t)^2 + 1} dt \\
        &= \frac{d}{2(c_1 - c_2)} \frac{1}{\sqrt{\frac{c_1 + c_2}{c_1-c_2}}}  \int^{\frac{\xi(0)}{2}}_{\frac{\xi(0)}{2}-2T} \frac{ \frac{d}{dt} \left( \sqrt{\frac{c_1 + c_2}{c_1-c_2}} \tan(t) \right) }{\left[\sqrt{\frac{c_1 + c_2}{c_1-c_2}} \tan(t)\right]^2 + 1} dt.
    \end{align*}
    %
    Moreover, \( (c_1 - c_2)(c_1 + c_2) = c_1^2 - c_2^2 = (2h)^2 - (4h^2 - 1) = 1 \) and \( c_1 - c_2 > 0 \), thus \( \sqrt{\frac{c_1+c_2}{c_1-c_2} } = (c_1+c_2) \) and 
    \begin{equation*}
        \int_0^T \frac{d}{L(t)^2} dt
        = \frac{d}{2} \int_{\frac{\xi(0)}{2} - 2T}^{\frac{\xi(0)}{2}} \frac{\frac{d}{dt} \left( (c_1+c_2) \tan(t) \right)}{\left( (c_1+c_2) \tan(t) \right)^2 + 1} dt.
    \end{equation*}
    %
    Now let \( m_0 \in \mathbb{Z} \) such that \( \frac{\xi(0)}{2} \in m_0\pi + \left( -\frac{\pi}{2}, \frac{\pi}{2}  \right] \), and \( m_T \in \mathbb{Z} \) such that \( \frac{\xi(T)}{2} \in (m_0-m_T)\pi + \left( -\frac{\pi}{2}, \frac{\pi}{2}  \right] \). We recall that \( \xi(T) = \xi(0) - 4T \). Then
    \begin{align*}
        &\int_0^T \frac{d}{L(t)^2} dt = \frac{d}{2} \int_{\frac{\xi(0)}{2}-2T}^{\frac{\xi(0)}{2}} \underbrace{\frac{\frac{d}{dt} \left( (c_1+c_2) \tan(t) \right)}{\left( (c_1+c_2) \tan(t) \right)^2 + 1}}_{=:f(t)} dt \\
        &= \frac{d}{2} \int_{m_0\pi - \frac{\pi}{2} }^{\frac{\xi(0)}{2}} f(t) dt  
            + \frac{d}{2} \int_{(m_0-1)\pi - \frac{\pi}{2} }^{m_0\pi - \frac{\pi}{2}} f(t) dt 
            + \dots + \frac{d}{2} \int_{\frac{\xi(0)}{2}-2T}^{(m_0-m_T)\pi + \frac{\pi}{2} } f(t) dt.
    \end{align*}
    %
    For \( m\in \mathbb{Z} \), we have 
    \begin{align*}
        \int_{m\pi - \frac{\pi}{2}}^{m\pi + \frac{\pi}{2}} f(t) dt 
        &= \left[ \arctan\left( (c_1+c_2) \tan(t) \right) \right]_{m\pi - \frac{\pi}{2}}^{m\pi + \frac{\pi}{2}} \\
        &= \left[ \arctan\left( (c_1+c_2) \tan(t) \right) \right]_{- \frac{\pi}{2}}^{\frac{\pi}{2}} = \pi.
    \end{align*}
    %
    Now write \( \widetilde{\frac{\xi(0)}{2}} := \frac{\xi(0)}{2} - m_0\pi \in \left( -\frac{\pi}{2}, \frac{\pi}{2} \right] \), and \( \widetilde{\frac{\xi(t)}{2}} := \frac{\xi(t)}{2} - (m_0 - m_T)\pi \in \left( -\frac{\pi}{2}, \frac{\pi}{2} \right] \). Then,
    \begin{align*}
        &\int_0^T \frac{d}{L(t)^2} dt \\
        &= \frac{d}{2} (m_T-1)\pi
            + \frac{d}{2} \int_{m_0\pi - \frac{\pi}{2} }^{\frac{\xi(0)}{2}} f(t) dt
            + \frac{d}{2} \int_{\frac{\xi(0)}{2}-2T}^{(m_0-m_T)\pi + \frac{\pi}{2}} f(t) dt\\
        &= (m_T-1)\frac{\pi d}{2} 
            + \frac{d}{2} \int_{- \frac{\pi}{2} }^{\frac{\widetilde{\xi(0)}}{2}} f(t) dt
            + \frac{d}{2} \int_{\frac{\widetilde{\xi(t)}}{2}}^{\frac{\pi}{2}} f(t) dt\\
        &= (m_T-1)\frac{\pi d}{2}
            + \frac{d}{2} \left[ \arctan\left( (c_1+c_2) \tan(t) \right) \right]_{- \frac{\pi}{2} }^{\frac{\widetilde{\xi(0)}}{2}}
            + \frac{d}{2} \left[ \arctan\left( (c_1+c_2) \tan(t) \right) \right]_{\frac{\widetilde{\xi(t)}}{2}}^{\frac{\pi}{2}} \\
        &= (m_T-1)\frac{\pi d}{2}
            + \frac{d}{2} \arctan\left( (c_1+c_2) \tan\left( \frac{\widetilde{\xi(0)}}{2} \right) \right) + \frac{\pi d}{2} \\
        &\qquad + \frac{\pi d}{2} - \arctan\left( (c_1+c_2) \tan\left( \frac{\widetilde{\xi(t)}}{2} \right) \right) \\
        &= m_T \frac{\pi d}{2}
            + \frac{d}{2} \arctan\left( (c_1+c_2) \tan\left( \frac{\widetilde{\xi(0)}}{2} \right) \right)
            - \frac{d}{2} \arctan\left( (c_1+c_2) \tan\left( \frac{\widetilde{\xi(t)}}{2} \right) \right) \\
        &= m_T \frac{\pi d}{2}
            + \frac{d}{2} \arctan\left( (c_1+c_2) \tan\left( \frac{\xi(0)}{2} \right) \right)
            - \frac{d}{2} \arctan\left( (c_1+c_2) \tan\left( \frac{\xi(0)}{2} - 2T \right) \right)
    \end{align*}
    %
    Hence
    \begin{align*}
        -\int_0^T \frac{d}{L(t)^2} dt 
        &= \frac{d}{2} \arctan\left( (c_1+c_2) \tan\left( \frac{\xi(0)}{2} - 2T \right) \right) \\
        &\quad - \frac{d}{2} \arctan\left( (c_1+c_2) \tan\left( \frac{\xi(0)}{2} \right) \right) -m_T \frac{\pi d}{2}
    .\end{align*}
    %

    Finally, it remains to show that if \( a_i(0) = 0,\, i\in \{1, \dots, d\} \) or \( h(0) = \frac{1}{2} \), then the behavior of \( t\mapsto \gamma(t) \) does not depend on the exact value of \( \theta_i(0),\, i\in\{1, \dots, d\}  \) or \( \xi(0) \). The exact formula for \( \gamma(T) \) is:
    \begin{align*}
        \gamma(T) &= 
                \gamma(0) + a_1(0) \left[ \sin(2\theta_1(T)) - \sin(2\theta_1(0)) \right] \\
                &\qquad + a_2(0) \left[ \sin(2\theta_2(T)) - \sin(2\theta_2(0)) \right] \\
                &\qquad + \frac{d}{2} \arctan\left( \left( 2h(0) + \sqrt{4h(0)^2 - 1} \right) \tan\left( \frac{\xi(0)}{2} - 2T \right) \right) \\
                &\qquad - \frac{d}{2} \arctan\left( \left( 2h(0) + \sqrt{4h(0)^2 - 1} \right) \tan\left( \frac{\xi(0)}{2} \right) \right) -m_T \frac{\pi d}{2} .
    \end{align*}
    %
    It is clear that if \( a_i(0) = 0 \) then \( \gamma(t) \) does not depend on \( \theta_i(0) \) nor \( \theta_i(t) \), \( i\in \{1, \dots, d\} \). If \( h(0) = \frac{1}{2} \), then
    \begin{equation*}
        2h(0) + \sqrt{4h(0)^2 - 1} = 1
    ,\end{equation*}
    %
    so that 
    \begin{align*}
        &\frac{d}{2} \arctan\left( \left( 2h(0) + \sqrt{4h(0)^2 - 1} \right) \tan\left( \frac{\xi(0)}{2} - 2T \right) \right) \\
        &\qquad - \frac{d}{2} \arctan\left( \left( 2h(0) + \sqrt{4h(0)^2 - 1} \right) \tan\left( \frac{\xi(0)}{2} \right) \right) -m_T \frac{\pi d}{2} \\
        &= \frac{d}{2} \arctan\left( \tan\left(\frac{\xi(T)}{2}\right) \right) - \arctan\left( \tan\left(\frac{\xi(0)}{2}\right) \right) - m_T\frac{\pi d}{2} 
    \end{align*}
    %
    Since \( \arctan: \mathbb{R}\mapsto \left( -\frac{\pi}{2}, \frac{\pi}{2} \right] \), we have
    \begin{align*}
        &\frac{d}{2} \arctan\left( \tan\left(\frac{\xi(T)}{2}\right) \right) - \frac{d}{2} \arctan\left( \tan\left(\frac{\xi(0)}{2}\right) \right) - m_T\frac{\pi d}{2}  \\
        &= \frac{d}{2} \widetilde{\frac{\xi(T)}{2}} - \frac{d}{2} \widetilde{\frac{\xi(0)}{2}} - m_T \frac{\pi d}{2} \\
        &= \frac{d}{2} \frac{\xi(T)}{2} - (m_0 - m_T)\frac{\pi d}{2}  - \frac{d}{2} \left(\frac{\xi(0)}{2} - m_0\pi \right) - m_T \frac{\pi d}{2}  \\
        &=  \frac{d}{2} \left( \frac{\xi(T)}{2} - \frac{\xi(0)}{2}\right) = -Td.
    \end{align*}
    %
    This shows that, in the case \( h(0) = \frac{1}{2}  \), the mapping \( t\mapsto \gamma(t) \) does not depend on the value chosen for the ill-defined quantity \( \xi(0) \).
\end{proof}



Using Lemmata \ref{lemma: HO exact integration of parameters} and \ref{lemma: HO change of variables between params and action-angle variables}, we are now able to obtain an easy numerical algorithm which simulates the evolution of bubbles according to the Harmonic Oscillator on a time interval \( [0, T] \). It is described in Algorithm \ref{algo: HO -- exact solve}.
\begin{algorithm}
    \caption{Solving the Harmonic oscillator with Bubbles}
    \label{algo: HO -- exact solve}
    \begin{algorithmic}
        \For{ \( j = 1, \dots, N \) } 
            \Comment{\(j\) denotes a bubble's index}
            \State Use \eqref{eqn: lemma -- HO exact integration of parameters - change of variables} to get the action-angle variables \( (h, a, \xi, \theta) \) at time 0.
            \State Use \eqref{eqn: lemma -- HO exact integration of parameters - update of action-angle variables} to update the variables \( (h, a, \xi, \theta) \) up to time \( T \).
            \State Use \eqref{eqn: lemma -- HO exact integration of parameters - update of parameters - with v} (with the expression for \( \gamma(t) \) replaced by \eqref{eqn: HO -- explicit expression of gamma}) to get the parameters of bubble \( u_j \) at time \( T \).
        \EndFor
    \end{algorithmic}
\end{algorithm}


\begin{remark}[Numerical considerations]
    \label{rmk: HO -- numerical considerations}
    Here are a few remarks about Algorithm \ref{algo: HO -- exact solve}:
    \begin{itemize}
        \item When applying equation \eqref{eqn: lemma -- HO exact integration of parameters - change of variables} to obtain the action-angle variables from the bubbles' parameters, it is advised to use the function \( \arctanTwo(y, x) \) instead of \( \arctan(y/x) \) because it allows to obtain an angle lying in \( (-\pi, \pi] \) instead of \( (-\pi/2, \pi/2] \) by taking into account the signs of both \( x \) and \( y \). This is also the reason why we do not define \( \xi(0) \) by \eqref{proof: HO proof -- definition of xi with k}.
        Moreover most numerical implementations of \( \arctanTwo \) return a finite value for \( \arctanTwo(0, 0) \), which avoids the manual tuning of a numerical threshold to know whether \( a_i(0) \) or \( h(0) \) vanish numerically or not. We recall that in this case the exact value returned does not impact the behavior of \( t\mapsto (L(t), B(t), X(t), \beta(t), \gamma(t)) \).
        \item The algorithm yields an exact solution as soon as the initial data is a sum of bubbles. If not, then the only error committed is the discretization error when approximating the initial condition \( \psi(t=0) \) by the ansatz \eqref{eqn: generic discretization of psi -- ansatz}.
        \item This numerical algorithm does not need any discretization in time nor in space.
        \item The solution obtained is the \emph{exact} solution of the equation \eqref{eqn: cNLS with psi -- linear part} defined on the whole space \( \mathbb{R}^d \), and no numerical boundary conditions are needed.
    \end{itemize}
\end{remark}





\section{The Dirac-Frenkel-MacLachlan principle}
\label{sect: DFMP}

In this section we consider the Schrödinger equation \eqref{eqn: cNLS with psi}.
As it has been explained before, the equation consists in two parts: the linear part \eqref{eqn: cNLS with psi -- linear part}, and the nonlinear part \eqref{eqn: cNLS with psi -- nonlinear part}.
Section \ref{sect: The Harmonic Oscillator} was dedicated to solving the Harmonic oscillator, namely the linear part.
We are interested now in solving the nonlinear part, and as it is usually done for numerical simulations, we will use a splitting method.
This will allow us to solve \eqref{eqn: cNLS with psi} by solving separately \eqref{eqn: cNLS with psi -- linear part} and \eqref{eqn: cNLS with psi -- nonlinear part}, one after the other. 
By doing so, a splitting error is made, which depends on the timestep \( \Delta t \), and the order of the error depends on the specific splitting method.
It is also possible to apply high-order splitting methods.


\vspace{1em}


We focus on approximating numerically the solution to \eqref{eqn: cNLS with psi -- nonlinear part}:
\begin{equation*}
    i \partial_{t} \psi = \psi |\psi|^2
.\end{equation*}
%
We are free to use any method we want, but one has to keep in mind that Algorithm \ref{algo: HO -- exact solve} solves \eqref{eqn: cNLS with psi -- linear part} exactly when \( \psi \) is expressed under the form \eqref{eqn: generic discretization of psi -- ansatz}, i.e. as a sum of bubbles. Therefore we would like the approximate solution to \eqref{eqn: cNLS with psi -- nonlinear part} to keep this particular form. This naturally calls for the use of the Dirac-Frenkel-MacLachlan principle (abbreviated DFMP). For more details, see \cite[Sect.~3]{lasserComputingQuantumDynamics2020}. 
Let \( \mathcal{M} \) be a manifold of complex-valued Gaussian functions:
\begin{equation}
    \label{eqn: manifold for the dirac-frenkel-maclachlan principle}
    \mathcal{M} := \left\{
        u \in \mathbb{L}^2( \mathbb{R}^d ) 
        \left| \begin{aligned}
            u(x) = \sum_{j=1}^N \frac{A_j}{L_j} e^{i\gamma_j + i\beta_j \cdot (x - X_j) - \frac{2+iB_j}{4L_j^2} \left| x - X_j \right|^2} &,\\
            A_j, B_j, \gamma_j \in \mathbb{R},\, L_j \in \mathbb{R}_+^*, \, X_j, \beta_j \in \mathbb{R}^d &
        \end{aligned} \right.
    \right\}
.\end{equation}
%
We look for a function \( u \in \mathcal{M} \) that solves \eqref{eqn: cNLS with psi -- nonlinear part} on \( \mathcal{M} \). More precisely, \( u \) is defined such that its time derivative lies in the tangent space of \( \mathcal{M} \) at \( u \), \( \mathcal{T}_{u(t)} \mathcal{M} \), and such that the residual of equation \eqref{eqn: cNLS with psi -- nonlinear part} is orthogonal to the tangent space. That is, 
\begin{equation}
    \label{eqn: definition of u(t) in DFMP}
    \begin{aligned}
        &\partial_{t} u(t) \in \mathcal{T}_{u(t)} \mathcal{M}, \quad \text{ such that } \\
        &\qquad \langle f, i \partial_{t} u(t) - u(t) |u(t)|^2 \rangle = 0, \, \forall f \in \mathcal{T}_{u(t)} \mathcal{M}.
    \end{aligned}
\end{equation}

\begin{remark}
    The definition of \( \partial_{t} u(t) \) via \eqref{eqn: definition of u(t) in DFMP}, initially proposed by Dirac and Frenkel \cite{diracNoteExchangePhenomena1930,frenkelWaveMechanicsAdvanced1934}, has been later criticized by MacLachlan \cite{mclachlanVariationalSolutionTimedependent1964}. 
    He proposed an alternative approach, which would consist in minimizing the quantity
    \begin{equation*}
        \left|\left| i \partial_{t} u(t) - |u(t)|^2 u(t) \right|\right|^2
    .\end{equation*}
    %
    However, the two formulations are equivalent if the tangent space \( \mathcal{T}_{u(t)} \mathcal{M} \) is \( \mathbb{C} \)-linear \cite{broeckhoveEquivalenceTimedependentVariational1988}. 
    This is the case here because multiplying by the complex unit \( i \) simply amounts to \( \gamma_j \mapsto \gamma_j + \frac{\pi}{2}  \). 
    Therefore, the approaches by Dirac-Frenkel and MacLachlan are equivalent.
\end{remark}
%
\bigskip

Let \( B_{u(t)} \) be a basis of \( \mathcal{T}_{u(t)} \mathcal{M} \), then \eqref{eqn: definition of u(t) in DFMP} is equivalent to
\begin{equation}
    \label{eqn: definition of u(t) in DFMP -- with general basis of Tu M}
    \begin{aligned}
        &\partial_{t} u(t) \in \mathcal{T}_{u(t)} \mathcal{M}, \quad \text{ such that } \\
        &\qquad \langle f, i \partial_{t} u(t) \rangle = \langle f, u(t) |u(t)|^2 \rangle = 0, \, \forall f \in B_{u(t)}.
    \end{aligned}
\end{equation}
%
A family (which may happen to be linearly dependent) spanning the tangent space \( \mathcal{T}_{u(t)} \mathcal{M} \) is given by 
\begin{equation}
    \label{eqn: base of Tu M in DFMP}
    \begin{aligned}
        B_{u(t)} 
        &= \left\{ e^{i\Gamma_j - \frac{| y |^2}{2}},
                    (y_j)_1 e^{i\Gamma_j - \frac{| y_j |^2}{2}},
                    \dots,
                    (y_j)_d e^{i\Gamma_j - \frac{| y_j |^2}{2}},
                    |y_j|^2 e^{i\Gamma_j - \frac{| y_j |^2}{2}}
            : j=1, \dots, N   \right\}, \\
        &=: \left\{ b_{(d+2)(j-1)+1}, b_{(d+2)(j-1)+2}, \dots, b_{(d+2)(j-1)+d+1}, b_{(d+2)j} : j=1, \dots, N \right\},
    \end{aligned}
\end{equation}
%
where we defined
\begin{equation*}
    \Gamma_j(y_j) := \gamma_j + L_j \beta_j \cdot y_j - \frac{B_j}{4} |y_j|^2
.\end{equation*}
%


Thus, \eqref{eqn: definition of u(t) in DFMP -- with general basis of Tu M} is equivalent to 
\begin{equation}
    \label{eqn: definition of u(t) in DFMP -- with particular basis of Tu M}
    \langle i \partial_{t} u(t), b_{(d+2)(j-1)+l} \rangle = \langle u |u|^2, b_{(d+2)(j-1)+l}  \rangle
    , \quad \forall j = 1, \dots, N,\quad l=0, \dots, d+1
.\end{equation}
%

The next step consists in expressing \eqref{eqn: definition of u(t) in DFMP -- with particular basis of Tu M} as a linear system involving the parameters of the bubbles and their time derivative. We then solve the linear system to obtain ODEs on the parameters, and be able to integrate them numerically.

The main advantage of this approach is that it guarantees to keep the approximate solution of \eqref{eqn: cNLS with psi -- nonlinear part} as a sum of \( N \) bubbles, for which we know the value of the parameters.

\hspace{1em}

In order to obtain the linear system, we first have to recall the expression of \( i\partial_{t} u(t) \) \eqref{eqn: i dt uj}:
\begin{equation}
    \label{eqn: i dt u}
    \begin{aligned}
        i \partial_{t} u
        = \sum_{j=1}^N \frac{u_j}{L_j^2} 
                &\left\{ |y_j|^2 \left( i \frac{(L_j)_s}{L_j} - \frac{B_j (L_j)_s}{2L_j} + \frac{(B_j)_s}{4} \right) \right. \\
                &\quad+ y_j \cdot \left( -L_j(\beta_j)_s + i\frac{(X_j)_s}{L_j} - \frac{B_j}{2L_j} (X_j)_s  \right) \\
                &\quad\left. + i\frac{(A_j)_s}{A_j} - i\frac{(L_j)_s}{L_j} + \beta \cdot (X_j)_s - (\gamma_j)_s \right\}.
    \end{aligned}
\end{equation}
%
More concisely, we have
\begin{equation}
    \label{eqn: i dt u -- with basis elements}
    \begin{aligned}
        i \partial_{t} u
        &= \sum_{j=1}^N \frac{A_j}{L_j^3} e^{i\Gamma_j -\frac{| y |^2}{2}}
            \left\{ |y_j|^2 \left( E_j^{(5)} + i E_j^{(6)}  \right) \right. \\
            &\hspace{6em}+ y_j \cdot \left( E_j^{(3)(1, \dots, d)} + iE_j^{(4)(1, \dots, d)}   \right) \\
            &\hspace{6em}\left. + \left( E_j^{(1)} + iE_j^{(2)} \right) \right\} \\
        &= \sum_{j=1}^N \frac{A_j}{L_j^3} 
            \left\{ b_{(d+2)(j-1)+1} \left( E_j^{(1)} + iE_j^{(2)} \right) \right. \\
            &\hspace{6em}+ b_{(d+2)(j-1)+2} \left( E_j^{(3)(1)} + iE_j^{(4)(1)}   \right) \\
            &\hspace{6em}\ \, \vdots \\
            &\hspace{6em}+ b_{(d+2)(j-1)+d+1} \left( E_j^{(3)(d)} + iE_j^{(4)(d)}   \right) \\
            &\hspace{6em}+ \left. b_{(d+2)j} \left( E_j^{(5)} + i E_j^{(6)}  \right) \right\},
    \end{aligned}
\end{equation}
%
where
\begin{equation}
    \label{eqn: DFMP -- definition of the Ej}
    \begin{alignedat}{3}
    &E_j^{(1)} := \beta_j \cdot (X_j)_s - (\gamma_j)_s,        &\qquad& E_j^{(2)} := \frac{(A_j)_s}{A_j} - \frac{(L_j)_s}{L_j}, \\
    &E_j^{(3)(l)} := -L_j((\beta_j)_l)_s - \frac{B_j}{2L_j} ((X_j)_l)_s, &\qquad& E_j^{(4)(l)} := \frac{((X_j)_l)_s}{L_j}, \qquad l = 1, \dots, d, \\
    &E_j^{(5)} := \frac{(B_j)_s}{4} - \frac{B_j (L_j)_s}{2L_j}, &\qquad& E_j^{(6)} := \frac{(L_j)_s}{L_j}
    \end{alignedat}
\end{equation}
%
and where \( E_j^{(k)(1, \dots, d)} \) denotes the vector \( (E_j^{(k)(1)}, \dots,  E_j^{(k)(d)}) \). We recall the subscript \( {}_s \) denotes the derivative with respect to time \( s \).

According to \eqref{eqn: definition of u(t) in DFMP -- with particular basis of Tu M}, we then want to project \( i \partial_{t} u(t) \) against every element of \( B_{u(t)} \). We obtain the following linear system:
\begin{equation}
    \label{eqn: linear system for DFMP}
    \mathbf{A} \mathbf{E} = S
,\end{equation}
%
where
\begin{equation*}
    \mathbf{A} :=
    \begin{pmatrix}
        \langle b_1, b_1 \rangle & i\langle b_1, b_1 \rangle &  \dots & \langle b_{(d+2)N}, b_1 \rangle & i\langle b_{(d+2)N}, b_1 \rangle \\
        \vdots                   &                          &        &                                 & \vdots\\
        \langle b_1, b_{(d+2)N} \rangle& i\langle b_1, b_{(d+2)N} \rangle & \dots & \langle b_{(d+2)N}, b_{(d+2)N} \rangle & i\langle b_{(d+2)N}, b_{(d+2)N} \rangle \\
    \end{pmatrix} \in \mathbb{C}^{(d+2)N, 2(d+2)N}
,\end{equation*}
%
\begin{equation*}
    \mathbf{E} := \begin{pmatrix}
        \frac{A_1}{L_1^3} E_1^{(1)}\\
        \frac{A_1}{L_1^3} E_1^{(2)}\\
        \frac{A_1}{L_1^3} E_1^{(3)(1)} \\
        \frac{A_1}{L_1^3} E_1^{(4)(1)} \\
        \vdots \\
        \frac{A_1}{L_1^3} E_1^{(3)(d)} \\
        \frac{A_1}{L_1^3} E_1^{(4)(d)} \\
        \frac{A_1}{L_1^3} E_1^{(5)} \\
        \frac{A_1}{L_1^3} E_1^{(6)} \\
        \vdots \\
        \frac{A_N}{L_N^3} E_N^{(1)} \\
        \frac{A_N}{L_N^3} E_N^{(2)} \\
        \frac{A_N}{L_N^3} E_N^{(3)(1)} \\
        \frac{A_N}{L_N^3} E_N^{(4)(1)} \\
        \vdots \\
        \frac{A_N}{L_N^3} E_N^{(3)(d)} \\
        \frac{A_N}{L_N^3} E_N^{(4)(d)}\\
        \frac{A_N}{L_N^3} E_N^{(5)} \\
        \frac{A_N}{L_N^3} E_N^{(6)}
    \end{pmatrix} \in \mathbb{R}^{2(d+2)N},
    %
    \quad \text{ and }
    \quad
    S := \begin{pmatrix}
        \langle u|u|^2, b_1 \rangle \\
        \vdots \\
        \langle u|u|^2, b_{(d+2)N} \rangle \\
    \end{pmatrix} \in \mathbb{C}^{(d+2)N}
.\end{equation*}
%


The matrix \( \mathbf{A} \) is the Gram matrix of the family \( B_{u(t)} \) whose columns have been duplicated, which obviously depends on time. In order to solve the linear system \eqref{eqn: linear system for DFMP} we shall use the Moore-Penrose pseudo-inverse, which corresponds to the Least Squares solution if the matrix \( \mathbf{A}^* \mathbf{A} \) is invertible. The pseudo-inverse is invertible if and only if \( B_{u(t)} \) is a linearly independent family of \( \mathbb{L}^2( \mathbb{R}^d ) \).

We can already notice that if two bubbles have the same parameters then the family will be linearly dependent. 

\vspace{1em}

Once the linear system \eqref{eqn: linear system for DFMP} is solved, we obtain \( \mathbf{E} \), from which we can update the modulation parameters. Using the definitions \eqref{eqn: DFMP -- definition of the Ej} we have, for \( j = 1, \dots, N \),
\begin{align*}
    \frac{A_j}{L_j^3} \begin{pmatrix}
        E_j^{(1)} \\ 
        E_1^{(2)} \\
        E_j^{(3)(1, \dots, d)} \\
        E_j^{(4)(1, \dots, d)} \\
        E_j^{(5)} \\
        E_j^{(6)}
    \end{pmatrix}
    = \Re \begin{pmatrix}
        \mathbf{E}_{2(d+2)j+1} \\
        \vdots \\
        \mathbf{E}_{2(d+2)(j+1)}
    \end{pmatrix} 
    &=: \Re \begin{pmatrix}
        \mathbf{E}_{j, 1} \\
        \vdots \\
        \mathbf{E}_{j, 2(d+2)}
    \end{pmatrix} \\
    %
    \iff \begin{pmatrix}
        \beta_j\cdot (X_j)_s - (\gamma_j)_s \\
        \frac{(A_j)_s}{A_j} - \frac{(L_j)_s}{L_j} \\
        -L_j (\beta_j)_s - \frac{B_j}{2L_j} (X_j)_s \\
        \frac{(X_j)_s}{L_j} \\
        \frac{(B_j)_s}{4} - \frac{B_j (L_j)_s}{2L_j} \\
        \frac{(L_j)_s}{L_j}
    \end{pmatrix}
    &= \Re \begin{pmatrix}
        \mathbf{E}_{j, 1} \\
        \vdots \\
        \mathbf{E}_{j, 2(d+2)}
    \end{pmatrix}
.\end{align*}
%

Hence
\begin{equation*}
    \begin{aligned}
        \beta_j\cdot (X_j)_s - (\gamma_j)_s &= \Re \left( \mathbf{E}_{j, 1} \right), \\
        \frac{(A_j)_s}{A_j} - \frac{(L_j)_s}{L_j} &= \Re \left( \mathbf{E}_{j, 2} \right), \\
        -L_j (\beta_j)_s - \frac{B_j}{2L_j} (X_j)_s &= \Re \begin{pmatrix}
                \mathbf{E}_{j, 3} \\ \vdots \\ \mathbf{E}_{j, d+2}
            \end{pmatrix}, \\
        \frac{(X_j)_s}{L_j} &= \Re \begin{pmatrix}
                \mathbf{E}_{j, d+3} \\ \vdots \\ \mathbf{E}_{j, 2d+2}
            \end{pmatrix}, \\
        \frac{(B_j)_s}{4} - \frac{B_j (L_j)_s}{2L_j} &= \Re \left( \mathbf{E}_{j, 2d+3} \right), \\
        \frac{(L_j)_s}{L_j} &= \Re \left( \mathbf{E}_{j, 2d+4} \right).
    \end{aligned}
\end{equation*}
%
Therefore
\begin{equation*}
    \begin{aligned}
        (A_j)_s &= A_j\Re \left( \mathbf{E}_{j, 2} \right) + A_j\Re \left( \mathbf{E}_{j, 2d+4} \right), \\
        (L_j)_s &= L_j \Re \left( \mathbf{E}_{j, 2d+4} \right), \\
        (B_j)_s &= 4\Re \left( \mathbf{E}_{j, 2d+4} \right) + 2B_j \Re \left( \mathbf{E}_{j, 2d+4} \right), \\
        (X_j)_s &= L_j \Re \begin{pmatrix}
                \mathbf{E}_{j, d+3} \\ \vdots \\ \mathbf{E}_{j, 2d+2}
            \end{pmatrix}, \\
        (\beta_j)_s &= -\frac{1}{L_j} \Re \begin{pmatrix}
                \mathbf{E}_{j, 3} \\ \vdots \\ \mathbf{E}_{j, d+2}
            \end{pmatrix} - \frac{B_j}{2L_j} \Re \begin{pmatrix}
                \mathbf{E}_{j, d+3} \\ \vdots \\ \mathbf{E}_{j, 2d+2}
            \end{pmatrix}, \\        
        (\gamma_j)_s &= L_j \beta_j\cdot \Re \begin{pmatrix}
            \mathbf{E}_{j, 3} \\ \vdots \\ \mathbf{E}_{j, d+2}
        \end{pmatrix} - \Re \left( \mathbf{E}_{j, 1} \right),
    \end{aligned}
\end{equation*}
%
and with respect to time \( t \),
\begin{equation}
    \label{eqn: DFMP -- update of parameters with interactions -- wrt time t}
    \begin{aligned}
        (A_j)_s &= \frac{A_j}{L_j^2} \left( \Re \left( \mathbf{E}_{j, 2} \right) + \Re \left( \mathbf{E}_{j, 2d+4} \right) \right), \\
        (L_j)_s &= \frac{1}{L_j}  \Re \left( \mathbf{E}_{j, 2d+4} \right), \\
        (B_j)_s &= \frac{4}{L_j^2} \Re \left( \mathbf{E}_{j, 2d+4} \right) + \frac{2B_j}{L_j^2}  \Re \left( \mathbf{E}_{j, 2d+4} \right), \\
        (X_j)_s &= \frac{1}{L_j}  \Re \begin{pmatrix}
                \mathbf{E}_{j, d+3} \\ \vdots \\ \mathbf{E}_{j, 2d+2}
            \end{pmatrix}, \\
        (\beta_j)_s &= -\frac{1}{L_j^3} \Re \begin{pmatrix}
                \mathbf{E}_{j, 3} \\ \vdots \\ \mathbf{E}_{j, d+2}
            \end{pmatrix} - \frac{B_j}{2L_j^3} \Re \begin{pmatrix}
                \mathbf{E}_{j, d+3} \\ \vdots \\ \mathbf{E}_{j, 2d+2}
            \end{pmatrix}, \\        
        (\gamma_j)_s &= \frac{1}{L_j} \beta_j\cdot \Re \begin{pmatrix}
            \mathbf{E}_{j, 3} \\ \vdots \\ \mathbf{E}_{j, d+2}
        \end{pmatrix} - \frac{1}{L_j^2}  \Re \left( \mathbf{E}_{j, 1} \right).
    \end{aligned}
\end{equation}
%






\subsection{Computing coefficients of the linear system \eqref{eqn: linear system for DFMP}}


In order to be able to compute \( \mathbf{A} \) and \( S \), we give the exact expression of the inner products involved. For \( j, l = 1, \dots, N \), let 
\begin{equation}
    \left |  \begin{aligned}
        z &:= \frac{2+iB_l}{4L_l^2} + \frac{2-iB_j}{4L_j^2}, \\
        a &:= \frac{X_l}{L_l^2} + \frac{X_j}{L_j^2}, \\
        \xi &:= \frac{B_j}{2L_j^2} X_j + \beta_j - \frac{B_l}{2L_l^2} X_l - \beta_l, \\
        C &= \exp\left\{i(\gamma_l - \gamma_j) - \frac{2+iB_l}{4L_l^2}  | X_l | ^2 - \frac{2-iB_j}{4L_j^2}  | X_j | ^2 - i\beta_l \cdot X_l + i\beta_j \cdot X_j \right\}.
    \end{aligned}
    \right.
\end{equation}
%
Those quantities obviously depend on the indices \( j \) and \( l \), but for clarity we do not write explicitly these dependences since they are pretty clear. Then, for \( n, m = 1, \dots, d \),
\begin{align*}
    \langle b_{(d+2)(l-1)+1}, b_{(d+2)(j-1)+1} \rangle &= C \widehat{f}(\xi) \\
    %
    \langle b_{(d+2)(l-1)+n+1}, b_{(d+2)(j-1)+1} \rangle &= \frac{C}{L_l} \left( \widehat{x f}_n - (X_l)_n \widehat{f} \right)(\xi) \\
    %
    \langle b_{(d+2)l}, b_{(d+2)(j-1)+1} \rangle 
    &= \frac{C}{L_l^2} \left( \widehat{ | x | ^2 f} - 2X_l \cdot \widehat{x f} +  | X_l | ^2 \widehat{f} \right)(\xi) \\
    %
    \langle b_{(d+2)(l-1)+n+1}, b_{(d+2)(j-1)+m+1} \rangle
    &= \frac{C}{L_j L_l} \left[ \widehat{x_n x_m f} - (X_l)_n \widehat{x_m f} - (X_j)_m \widehat{x_n f} + (X_l)_n (X_j)_m \widehat{f} \right](\xi) \\
    %
    \langle b_{(d+2)l}, b_{(d+2)(j-1)+m+1} \rangle &=
    \frac{C}{L_l^2 L_j} \left[ \widehat{x_m | x | ^2 f} - 2X_l \cdot \widehat{x_m x f} + | X_l | ^2 \widehat{x_m f} \right.\\
    &\hspace{1cm} \left. - (X_j)_m \widehat{ | x | ^2 f} + 2(X_j)_m X_l \cdot \widehat{x f} - | X_l | ^2 (X_j)_m \widehat{f} \right](\xi) \\
    %
    \langle b_{(d+2)l}, b_{(d+2)j} \rangle  &= \frac{C}{L_l^2 L_j^2} \left[ \widehat{ | x | ^4 f} - 2X_l \cdot \widehat{ | x | ^2 x f} +  | X_l | ^2 \widehat{ | x | ^2 f} - 2X_j \cdot \widehat{x | x | ^2 f} \right. \\
    &\hspace{2cm}  + 4\sum_{n, m=1}^d (X_l)_n (X_j)_m \widehat{x_n x_m f} - 2 | X_l | ^2 X_j \cdot \widehat{x f} \\
    &\hspace{2cm} \left. + \widehat{ | x | ^2 f}  | X_j | ^2 - 2 | X_j | ^2 X_l \cdot \widehat{x f} +  | X_l | ^2  | X_j | ^2 \widehat{f} \right](\xi)
\end{align*}
%

We now compute the components of the vector \( S \). For \( j, k, l, m=1, \dots, N \), let
\begin{equation}
    \left| 
        \begin{aligned}
            C_\Im &:= \exp\left\{ i\left( \gamma_k + \gamma_l - \gamma_m - \gamma_j \right) \right\} \\
            &\hspace{1em} \times \exp\left\{ i\left(\beta_j \cdot X_j + \beta_m \cdot X_m - \beta_l\cdot X_l - \beta_k \cdot X_k \right) \right\} \\
            &\hspace{1em} \times \exp\left\{- i\left( \frac{B_k}{4L_k^2} |X_k|^2 + \frac{B_l}{4L_l^2} |X_l|^2 - \frac{B_m}{4L_m^2} |X_m|^2 - \frac{B_j}{4L_j^2} |X_j|^2  \right) \right\}, \\
            C_\Re &:= \exp\left\{-\frac{1}{2} \left( \frac{|X_k|^2}{L_k^2} + \frac{|X_l|^2}{L_l^2} + \frac{|X_m|^2}{L_m^2} + \frac{|X_j|^2}{L_j^2} \right) \right\}, \\
            C &:= \frac{A_k A_l A_m}{L_k L_l L_m} C_\Im C_\Re, \\
            \xi &:= -\left[ \beta_k + \beta_l - \beta_m - \beta_j + \frac{B_k}{2L_k^2} X_k + \frac{B_l}{2L_l^2} X_l - \frac{B_m}{2L_m^2} X_m - \frac{B_j}{2L_j^2} X_j \right], \\
            z &:= \frac{1}{2} \left( \frac{1}{L_k^2} + \frac{1}{L_l^2} + \frac{1}{L_m^2} + \frac{1}{L_j^2} \right) + i \left( \frac{B_k}{4L_k^2} + \frac{B_l}{4L_l^2} - \frac{B_m}{4L_m^2} - \frac{B_j}{4L_j^2} \right), \\
            a &:= \frac{1}{L_k^2} X_k + \frac{1}{L_l^2} X_l + \frac{1}{L_m^2} X_m + \frac{1}{L_j^2} X_j.
        \end{aligned}
    \right.
\end{equation}
%
Those quantities obviously depend on the indices \( j, k, l \) and \( m \), but for clarity we do not write explicitly these dependences since they are pretty clear. Then, for \( 1 \leq r \leq d \),
\begin{align*}
    \langle u|u|^2, b_{(d+2)(j-1)+1} \rangle &= \sum_{k,l,m} C \widehat{f}(\xi) \\
    \langle u|u|^2, b_{(d+2)(j-1)+r+1} \rangle &= \sum_{k,l,n} \frac{C}{L_j} \left( \widehat{x_r f} - (X_j)_r \widehat{f} \right) \\
    \langle u|u|^2, b_{(d+2)j} \rangle &= \sum_{k,l,m} \frac{C}{L_j^2} \left( \widehat{|x|^2 f} - 2X_j \cdot \widehat{xf} + |X_j|^2 \widehat{f} \right).
\end{align*}
%
We refer to Appendix \ref{appendix: populating the linear system for DFMP} for more details. Moreover, Appendix \ref{appendix: fourier transforms of gaussians} contains Table \ref{table: useful Fourier transforms} which sums up the useful Fourier transforms. 




\begin{remark}[Computational complexity]
    In \eqref{eqn: generic discretization of psi -- expression for uj -  vj gaussian} we chose \( v_j(s_j, y_j) = e^{-\frac{1}{2} | y_j |^2} \).
    This choice was made so that the inner products involved in the application of the DFMP are computable exactly.
    Therefore we do not rely on numerical integration to compute the coefficients of the linear system \eqref{eqn: linear system for DFMP}. 
    In particular, this shows that the computational effort required to obtain the linear system is \( \mathcal{O}(N^4d + N^2(d+2)^2)\).
    To obtain the total complexity, we have to add the cost computing the pseudo-inverse of the hermitian matrix \( \mathbf{A} \in \mathbb{C}^{(d+2)N, (d+2)N} \), which is \( O((d+2)^3 N^3)\).
    This yields the overall computational complexity: \( \mathcal{O}(N^4d + d^3 N^3)\).
\end{remark}




We obtain Algorithm \ref{algo: DFMP -- solve approximately cNLS} which can be used to obtain an approximate solution to \eqref{eqn: cNLS with psi} as a sum of bubbles, using the Strang splitting.

\begin{algorithm}
    \caption{Approximating a solution to \eqref{eqn: cNLS with psi} as a sum of bubbles.}
    \label{algo: DFMP -- solve approximately cNLS}
    \begin{algorithmic}
        \For{ Each timestep of size \( dt \) } 
            \For{ \( j = 1, \dots, N \) }
            \Comment{\(j\) denotes a bubble's index}
                \State Use Algorithm \ref{algo: HO -- exact solve} to update the bubbles over a timestep of size \( dt/2 \).
                \State Compute the coefficients of the linear system \eqref{eqn: linear system for DFMP}.
                \State Solve the linear system \eqref{eqn: linear system for DFMP} to obtain \( \mathbf{E} \).
                \State Use \eqref{eqn: DFMP -- update of parameters with interactions -- wrt time t} to update the parameters over a timestep of size \( dt \).
                \State Use Algorithm \ref{algo: HO -- exact solve} to update the bubbles over a timestep of size \( dt/2 \).
            \EndFor
        \EndFor
    \end{algorithmic}
\end{algorithm}






\subsection{Hamiltonian and norm conservation for the interactions}

When solving \eqref{eqn: cNLS with psi -- nonlinear part} via the DFMP, i.e. when solving the linear system \eqref{eqn: linear system for DFMP}, a Hamiltonian is conserved. 
\begin{lemma}
    Let \( u(t) \) be the approximation to \eqref{eqn: cNLS with psi -- nonlinear part} obtained by applying the Dirac-Frenkel-MacLachlan principle, and define
    \begin{equation*}
        H_{\textnormal{interactions}}(t) := \frac{1}{4} \langle u(t), u(t)| u(t)|^2 \rangle = \frac{1}{4} \langle u(t)^2, u(t)^2 \rangle
    .\end{equation*}
    %
    Then \( H_{\textnormal{interactions}} \) is conserved, i.e.
    \begin{equation*}
        \frac{\textnormal{d}}{\textnormal{d}t} H_{\textnormal{interactions}}(t) = 0
    ,\end{equation*}
    %
    and the \( \mathbb{L}^2 \) norm of \( u \) is also conserved.
\end{lemma}


\begin{proof}
    We have
    \begin{equation*}
        H_{\textnormal{interactions}}(t) := \frac{1}{4} \langle u(t), u(t)| u(t)|^2 \rangle = \frac{1}{4}  \langle u(t)^2, u(t)^2 \rangle
    ,\end{equation*}
    %
    by using the Hermitian property of the inner product \( \langle \cdot, \cdot \rangle \). Then,
    \begin{align*}
        \frac{\textnormal{d}}{\textnormal{d}t} H_{\textnormal{interactions}}(t)
        &= \frac{1}{4} \frac{\textnormal{d}}{\textnormal{d}t} \langle u(t)^2, u(t)^2 \rangle \\
        &= \frac{1}{4} \left\langle 2u(t) \partial_{t} u(t) , u(t)^2 \right\rangle + \left\langle u(t)^2, 2u(t) \partial_{t} u(t) \right\rangle \\
        &= \Re \left\langle u(t) \partial_{t} u(t) , u(t)^2 \right\rangle \\
        &= \Re \left\langle \partial_{t} u(t) , u(t) |u(t)|^2 \right\rangle.
    \end{align*}
    %
    By definition of \( \partial_{t} u(t) \), we have \( \partial_{t} u(t) \in \mathcal{T}_{u(t)} \mathcal{M} \), hence we can take \( f = \partial_{t} u(t) \) in \eqref{eqn: definition of u(t) in DFMP}. We obtain the following equality:
    \begin{equation*}
        \langle \partial_{t} u(t), u(t) |u(t)|^2 \rangle = \langle \partial_{t} u(t), i \partial_{t} u(t) \rangle = -i \| \partial_{t} u(t) \|^2
    .\end{equation*}
    %
    Therefore,
    \begin{equation*}
        \frac{\textnormal{d}}{\textnormal{d}t} H_{\textnormal{interactions}}(t) = \Re \left( -i \| \partial_{t} u(t) \|^2 \right) = 0
    .\end{equation*}
    %

    Using similar ideas, we can easily show the conservation of the \( \mathbb{L}^2 \) norm: we obviously have \( u(t) \in \mathcal{T}_{u(t)} \mathcal{M} \), hence
    \begin{align*}
        \frac{\textd}{\textd t} \|u(t)\|^2 
        &= 2\Re\langle u(t), \partial_{t} u(t) \rangle = 2 \Re \langle u(t), -i u(t)|u(t)|^2 \rangle \\
        &= 2 \Re \left( i \langle |u(t)|^2, |u(t)|^2 \rangle\right) = 0
    \end{align*}
    %
\end{proof}







\subsection{Recovering the Harmonic Oscillator equations}

Suppose the family \( B_{u(t)} \subset \mathbb{L}^2( \mathbb{R}^d ) \) defined by \eqref{eqn: base of Tu M in DFMP} is linearly independent, and consider the equation \eqref{eqn: cNLS with psi -- linear part}.
By summing equation \eqref{eqn: HO -- idt u - Hu -- with v} over \( j=1, \dots, N \) with \( v_j(s_j,y_j) = e^{-\frac{|y_j|^2}{2}} \), and letting this sum be equal to zero, we obtain an equation of the form
\begin{equation}
    \label{eqn: recovering the HO eqns -- write HO with generic coefficients in basis B}
    \begin{aligned}
        \sum_{j=1}^N c_{(d+2)(j-1)+1} b_{(d+2)(j-1)+1} + c_{(d+2)(j-1)+2} b_{(d+2)(j-1)+2} + \dots& \\
        \qquad + c_{(d+2)(j-1)+d+1} b_{(d+2)(j-1)+d+1} + c_{(d+2)j} b_{(d+2)j} &= 0.
    \end{aligned}
\end{equation}
%
Thanks to the assumption that \( B_{u(t)} \) is a linearly independent family, we know that we must have 
\begin{equation}
    \label{eqn: recovering HO equations from DFMP -- all coeffs equal to zero}
    c_{k} = 0,\quad k=1,\dots, (d+2)N    
.\end{equation}
%
This yields exactly the system of equations \eqref{eqn: modulation ODEs -- linear part wrt time s - with v}, with the \( \gamma \) equation replaced by \eqref{eqn: HO -- ODE on gamma}.
In other words, the DFMP approach gives the same equations as those given in Section \ref{sect: HO -- ODEs on the modulation parameters with particular vj} when \( B_{u(t)} \) is a linearly independent family. However, our approach as described in Section \ref{sect: HO -- ODEs on the modulation parameters with particular vj} allows to solve them exactly and not only numerically with some numerical time-integrator.

Finally, if the family \( B_{u(t)} \) is linearly dependent, then we cannot write equation \eqref{eqn: recovering HO equations from DFMP -- all coeffs equal to zero} anymore, hence the DFMP approach fails. Our approach allows to circumvent this issue by naturally imposing conditions \eqref{eqn: recovering HO equations from DFMP -- all coeffs equal to zero} in all cases.




\subsection{Imposing conservation of \( \mathbb{L}^2 \) norm}

Our numerical experiments have shown that the DFMP approach does not always yield a nice conservation of the \( \mathbb{L}^2 \) norm, even though it should be conserved. This section is dedicated to finding a way of imposing it explicitely.

By using \eqref{eqn: i dt u -- with basis elements}, the conservation of the \( \mathbb{L}^2 \) norm writes
\begin{align*}
    0 &= \partial_{t} \int_{ \mathbb{R}^2 } |u(t,x)|^2 = 2\Re\langle \partial_{t} u, u \rangle \\
    &= 2 \sum_{k=1}^N \frac{A_k}{L_k^3} \Re \left\langle -i \left\{ b_{(d+2)(k-1)+1} \left( E_k^{(1)} + iE_k^{(2)} \right) \right. \right. \\
    &\hspace{6em}+ b_{(d+2)(k-1)+2} \left( E_k^{(3)(1)} + iE_k^{(4)(1)}   \right) \\
    &\hspace{6em}\ \, \vdots \\
    &\hspace{6em}+ b_{(d+2)(k-1)+d+1} \left( E_k^{(3)(d)} + iE_k^{(4)(d)}   \right) \\
    &\hspace{6em}+ \left. \left. b_{(d+2)k} \left( E_k^{(5)} + i E_k^{(6)}  \right) \right\}, u \right\rangle
.\end{align*}
%
Under matrix form, this yields
\begin{equation*}
    0 = 2\Im \left[ \left( \langle b_1, u \rangle, i\langle b_1, u \rangle, \dots, \langle b_{(d+2)N}, u \rangle, i\langle b_{(d+2)N}, u \rangle \right) \mathbf{E} \right].
\end{equation*}
%
Since \( \mathbf{E} \in \mathbb{R}^{2(d+2)N} \), we obtain 
\begin{equation*}
    \left( \Im \langle b_1, u \rangle, \Re \langle b_1, u \rangle, \dots, \Im \langle b_{(d+2)N}, u \rangle, \Re \langle b_{(d+2)N}, u \rangle \right) \mathbf{E} = 0
.\end{equation*}
%
In order to impose the conservation of the \( \mathbb{L}^2 \) norm, we have to add this line to the linear system \eqref{eqn: linear system for DFMP}. Moreover, this line has already been computed: indeed, we have
\begin{equation*}
    u = \sum_{j=1}^N \frac{A_j}{L_j} b_{(d+2)(j-1)+1} 
,\end{equation*}
%
hence
\begin{equation*}
    \langle b_k, u \rangle = \sum_{j=1}^N \frac{A_j}{L_j} \langle b_k, b_{(d+2)(j-1)+1} \rangle,\quad k=1, \dots, (d+2)N
.\end{equation*}
%


The idea motivating the addition of this line to the linear system is the following: whenever the pseudo-inverse of the matrix \( \mathbf{A} \) is well-conditioned the conservation of \( \mathbb{L}^2 \) norm is automatically satisfied, so the new line doesn't change anything. When the pseudo-inverse is ill-conditioned, it may happen that the \( \mathbb{L}^2 \) norm isn't preserved anymore. Hence, at each time step we compute the solution to the initial system of size \( (d+2)N\times 2(d+2)N \), and if the variation of the \( \mathbb{L}^2 \) norm of \( u \) is too high (e.g. larger than some numerical threshold) then we solve again the augmented linear system.





\section{Numerical examples}
\label{sect: numerical experiments}




In this section we will assess the efficiency of the Bubbles approach with the Dirac-Frenkel-MacLachlan approach against a spectral method in the two-dimensional case.



\subsection{Spectral scheme}

We start by discussing the spectral method we shall use to compare with the results of Algorithm \ref{algo: DFMP -- solve approximately cNLS}. We refer to \cite{fornbergPracticalGuidePseudospectral1996} for a general introduction to spectral methods for the Schr{\"o}dinger equation, and to \cite{antoineComputationalMethodsDynamics2013} for grid-based schemes applied to the Gross-Pitaevskii equation. 

We now present a method which can be understood as the application of \cite{bernierExactSplittingMethods2021} to a simpler equation, namely the Harmonic Oscillator. We use a splitting method to simulate the linear part \eqref{eqn: cNLS with psi -- linear part}, and thanks to \cite{bernierExactSplittingMethods2020,alphonsePolarDecompositionSemigroups2021} we have:
\begin{align}
    e^{-it (-\Delta + |x|^2)} 
    &= e^{- \frac{1}{2} \tanh(it) |x|^2} e^{\frac{1}{2} \sinh(2it)\Delta_x} e^{- \frac{1}{2} \tanh(it) |x|^2} \nonumber\\
    &= e^{- \frac{i}{2} \tan(t) |x|^2} e^{\frac{i}{2} \sin(2t)\Delta_x} e^{- \frac{i}{2} \tan(t) |x|^2}
    \label{eqn: num -- exact time splitting of HO}
.\end{align}
%
We can cite \cite{jinMathematicalComputationalMethods2011} which also presents a spectral method based on the Fourier transform with time splitting, however our method is different in that \eqref{eqn: num -- exact time splitting of HO} is exact and hence we do not have any time-splitting error.


The first and third exponentials on the RHS are straightforward to compute on a grid. For the second one, we use a Fourier transform: \( e^{\frac{i}{2} \sin(2t)\Delta_x} \) is the propagator of the following equation:
\begin{equation*}
    \partial_{t} \psi = i\cos(2t) \Delta_x \psi
.\end{equation*}
%
By using a Fourier transform, we get
\begin{equation*}
    \partial_{t} \mathcal{F}(\psi)(\xi)
    = i\cos(2t) \mathcal{F} \left( \Delta_x \psi\right)(\xi)
    = -i\cos(2t) |\xi|^2 \mathcal{F} \left( \psi \right)(\xi)
.\end{equation*}
%
Hence,
\begin{equation*}
    \mathcal{F}(\psi(t, \cdot))(\xi) = e^{-\frac{i}{2} \sin(2t) |\xi|^2} \mathcal{F}(\psi(0, \cdot))(\xi)
.\end{equation*}
%
The RHS exponential is straightforward to compute in the Fourier space. 
Hence, an exact-time spectral approximation of the solution to \eqref{eqn: cNLS with psi -- linear part} is given by Algorithm \ref{algo: num -- spectral solver linear part}.
From this, it is easy to obtain an algorithm which simulates \eqref{eqn: cNLS with psi} with interactions.
It consists in using a Strang splitting method on \eqref{eqn: cNLS with psi}, by splitting the linear part \eqref{eqn: cNLS with psi -- linear part} and the nonlinear part \eqref{eqn: cNLS with psi -- nonlinear part}. 
The linear part is approximated via Algorithm \ref{algo: num -- spectral solver linear part}, and the computation of interactions is explicit thanks to the fact that \( |u(t, x)|^2 \) does not depend on time (see e.g. \cite[Sect.~2.2]{faouGeometricNumericalIntegration2012}).
This fully describes Algorithm \ref{algo: num -- spectral solver}.

\begin{algorithm}
    \caption{Spectral solver for \eqref{eqn: cNLS with psi -- linear part}, with an exact time resolution for each splitting step.}
    \label{algo: num -- spectral solver linear part}
    \begin{algorithmic}
        \State Discretize the initial data \( \eta \) on a \( \text{Grid} \subset \mathbb{R}^d \).
        \For{ Each timestep of size \( \Delta t \) } 
            \For{ \( x \in \textsc{Grid} \) }
                \Comment{\(x \in \mathbb{R}^d\).}
                \State Multiply \( \eta(x) \) by \( e^{- \frac{i}{2} \tan(\Delta t) |x|^2} \).
            \EndFor
            %
            \State Apply a FFT to \( \eta \).
            \Comment{FFT: Fast Fourier Transform.}
            %
            \For{ \( \xi \in \textsc{Fourier Grid} \) }
                \Comment{\( \xi \in \mathbb{R}^d\).}
                \State Multiply \( \mathcal{F}(\eta)(\xi) \) by \( e^{-\frac{i}{2} \sin(2\Delta t) |\xi|^2} \).
            \EndFor
            %
            \State Apply an inverse FFT to \( \mathcal{F}(\eta) \).
            %
            \For{ \( x \in \textsc{Grid} \) }
                \State Multiply \( \eta(x) \) by \( e^{- \frac{i}{2} \tan(\Delta t) |x|^2} \).
            \EndFor
        \EndFor
    \end{algorithmic}
\end{algorithm}



\begin{algorithm}
    \caption{Spectral solver for \eqref{eqn: cNLS with psi}, with a Strang Splitting method.}
    \label{algo: num -- spectral solver}
    \begin{algorithmic}
        \State Discretize the initial data \( \eta \) on a \( \text{Grid} \subset \mathbb{R}^d \).
        \For{ Each timestep of size \( \Delta t \) } 
            \State Use Algorithm \ref{algo: num -- spectral solver linear part} with a stepsize \( \Delta t/2 \).
            \For{ \( x \in \textsc{Grid} \) } 
                \Comment{Add interactions.}
                \State Multiply \( \eta(x) \) by \( e^{-i\,\Delta t\, |\eta(x)|^2} \).
            \EndFor
            \State Use Algorithm \ref{algo: num -- spectral solver linear part} with a stepsize \( \Delta t/2 \).
        \EndFor
    \end{algorithmic}
\end{algorithm}




Of course, in pratical applications one is not able to define a grid over \( \mathbb{R}^d \). 
Hence, Algorithms \ref{algo: num -- spectral solver linear part} and \ref{algo: num -- spectral solver} have to be modified by defining \textsc{Grid} as a discretization of a finite-volume subset of \( \mathbb{R}^d \), typically a product of intervals in each dimension.
For all of our numerical examples, this will \( [-15, 15]\times[-15, 15] \), discretized using \( N_x\times N_y \) points.
In order to have an easily computable FFT, one has to use a spatial uniform grid, which then defines the \textsc{Fourier Grid}.
Special care has to be paid when choosing the number of points: if we have Fourier frequencies larger than the \emph{Nyquist frequency}, then we will observe a phenomenon known as \emph{aliasing}. 
This may not be problematic for the Harmonic Oscillator \eqref{eqn: cNLS with psi -- linear part} depending on the initial condition, but will eventually become an issue when simulating \eqref{eqn: cNLS with psi} because it involves interactions and hence an infinite number of frequencies.
Moreover, by using a FFT-based algorithm we implicitely impose periodic boundary conditions.





\subsection{Discretization into a sum of Bubbles}

We need to decompose any arbitrary function into a finite sum of \( N \) bubbles.
A solution to this question has been proposed in \cite{qianFastGaussianWavepacket2010}, but it involves integrals over the whole phase space, which is something we want to avoid.

We could also use a nonlinear least squares approach, but our experimental results showed that it tends to yield spread out gaussians, which may present huge overlaps between them. The overlaps cause issues with the DFMP, for instance a blow-up of the conservative quantities. This has been observed during our experiments but the results are not reported in this paper.
The issue of discretizing an arbitrary function into a sum of bubble without too much overlaps is not the main concern of this paper, hence we will use a visual trial-and-error discretization. Another possible way of discretizing the initial data is outlined in \cite{adamowiczLaserinducedDynamicAlignment2022}.




\subsection{Observables}

In order to compare the bubbles scheme against the spectral method, we compare them in absence of interactions, i.e. on the Harmonic Oscillator \eqref{eqn: cNLS with psi -- linear part}, as well as in the presence of nonlinear interactions, i.e. on \eqref{eqn: cNLS with psi}.
We showed in Lemma \ref{lemma: conserved quantities in HO} the conservation of some quantities for \eqref{eqn: cNLS with psi -- linear part} and \eqref{eqn: cNLS with psi}, we will focus on mass, energy and momentum.
When computing the observables for the spectral solution, we noted that the approximation of the gradient using finite differences with periodic boundary conditions yielded very rough results while the gradient approximation using the Fast Fourier Transform gave more accurate results. We use the latter approximation in the Figures of Section \ref{subsect: num results}.
In the case of bubbles, we compute every integral by hand thanks to the assumption \( v(s,y) = e^{-\frac{|y|^2}{2}} \), some details are given in Appendix \ref{appendix: Miscellaneous computations}. When reporting the results in the following \( \log \)-plots, all values with an amplitude smaller than \( 10^{-16} \) were set to be equal to \( 10^{-16} \).


For all of the results shown, the spectral scheme is supplied with the exact initial condition and not a projection on the grid of the bubbles discretization.





\subsection{Results}
\label{subsect: num results}


We consider examples adapted from \cite{baoNumericalSolutionGross2003}.


\subsubsection{Test case 1: Zero phase initial data}

The initial condition reads 
\begin{equation}
    \psi(t=0, x) = \pi e^{-\frac{|x-\mu_1|^2}{2} } + 2e^{-\frac{|x-\mu_2|^2}{2} }, \quad x\in \mathbb{R}^2,\quad \mu_1 = (0, 2), \ \mu_2 = (1, 0)
.\end{equation}
%




\begin{figure}
    \begin{subfigure}{\textwidth}
        \centering
        \inputtikz{1}{bubblesVSspectral_conservative_quantities_dt-0.001_coeffs-1.0-0.0_testcase1_Nx128_Ny129}
        \caption{\centering Approximate solution to the Harmonic Oscillator \eqref{eqn: cNLS with psi -- linear part}.}
    \end{subfigure}
    \begin{subfigure}{\textwidth}
        \centering
        \inputtikz{1}{bubblesVSspectral_conservative_quantities_dt-0.001_coeffs-1.0-1.0_testcase1_Nx128_Ny129}
        \caption{\centering Approximate solution to the Schrödinger equation \eqref{eqn: cNLS with psi} using DFMP Algorithm.}
    \end{subfigure}
    \caption{\centering Test case 1. Relative evolution of mass, energy and momentum with bubbles and spectral methods. \( \Delta t = 10^{-3} \). Time-integrator for the nonlinear part of the splitting: Runge-Kutta of order 4. Spectral scheme with \( N_x = 128, N_y = 129 \).}
    \label{fig: num -- test case 1}
\end{figure}




% \begin{figure}
%     \centering
%     \inputtikz{0.4}{courbe_cv_Nmin=16_Nmax=2048_testcase=1_HO=false_T=1.0_dt=0.005}
%     \caption{\centering Test case 1. Evolution of the \( L^2 \) norm of the difference between spectral and bubble schemes against the number of points in each direction \( N_x = N_y \), at time \( T = 1 \) with \( dt=5\cdot 10^{-3} \), relative w.r.t. the exact \( L^2 \) norm of the discretized initial data.}
%     \label{fig: num -- test case 1 - evolution of relative l2 norm}
% \end{figure}




The results are displayed in Figure \ref{fig: num -- test case 1}.
The solution approximated with the DFMP approach globally outperforms the spectral method on both the Harmonic Oscillator and the cubic NonLinear Schrödinger equations.
On the Harmonic Oscillator, the solution obtained with the Bubbles scheme is about one order of magnitude better than the spectral scheme. When we compare them on \eqref{eqn: cNLS with psi}, i.e. when adding interactions, the \( \mathbb{L}^2 \) norm is better conserved for the spectral scheme, but the other conservative quantities are better conserved on a long time for the Bubbles scheme.

The ``jumps'' in the DFMP approach may be explained by an ill-conditioned Gram matrix, which would then yield a very rough approximation of the modulation parameters.






\subsubsection{Test case 2: Weak interactions}

The initial condition reads 
\begin{equation}
    \psi(t=0, x) = e^{-|x - \mu_3|^2} e^{i \cosh |x - \mu_3|}, \quad x\in \mathbb{R}^2, \quad \mu_3 = (1, 1)
.\end{equation}
%

The approximation of this function as a sum of bubbles is pretty straightforward: we know that for \( x \) small, \( \cosh x \approx 1 + \frac{x^2}{2}  \), hence 
\begin{equation*}
    \psi(t=0, x) \approx e^{-|x - \mu_3|^2} e^{i + i\frac{|x - \mu_3|^2}{2} }, \quad x\in \mathbb{R}^2
.\end{equation*}
%



\begin{figure}
    \begin{subfigure}{\textwidth}
        \centering
        \inputtikz{1}{bubblesVSspectral_conservative_quantities_dt-0.001_coeffs-1.0-0.0_testcase2_Nx128_Ny129}
        \caption{\centering Approximate solution to the Harmonic Oscillator \eqref{eqn: cNLS with psi -- linear part}.}
    \end{subfigure}
    \begin{subfigure}{\textwidth}
        \centering
        \inputtikz{1}{bubblesVSspectral_conservative_quantities_dt-0.001_coeffs-1.0-1.0_testcase2_Nx128_Ny129}
        \caption{\centering Approximate solution to the Schrödinger equation \eqref{eqn: cNLS with psi} using DFMP Algorithm.}
    \end{subfigure}
    \caption{\centering Test case 2. Relative evolution of mass, energy and momentum with bubbles and spectral methods. \( \Delta t = 10^{-3} \). Time-integrator for the nonlinear part of the splitting: Runge-Kutta of order 4. Spectral scheme with \( N_x = 128, N_y = 129 \).}
    \label{fig: num -- test case 2}
\end{figure}





% \begin{figure}
%     \centering
%     \inputtikz{0.4}{courbe_cv_Nmin=16_Nmax=2048_testcase=2_HO=false_T=1.0_dt=0.005}
%     \caption{\centering Test case 2. Evolution of the \( L^2 \) norm of the difference between spectral and bubble schemes against the number of points in each direction \( N_x = N_y \), at time \( T = 1 \) with \( dt=5\cdot 10^{-3} \), relative w.r.t. the exact \( L^2 \) norm of the discretized initial data.}
%     \label{fig: num -- test case 2 - evolution of relative l2 norm}
% \end{figure}


The results are displayed in Figure \ref{fig: num -- test case 2}.
This example shows the performance of the DFMP approach in its most efficient setting: it only has one bubble. This explains the very good conservation results obtained: the Bubbles scheme outperforms the spectral scheme on both \eqref{eqn: cNLS with psi -- linear part} and \eqref{eqn: cNLS with psi}, except for the energy on \eqref{eqn: cNLS with psi}. However, even in this case, the error of the DFMP method remains generally less than one order of magnitude larger than the error from the spectral method.








\subsubsection{Test case 3: Strong interactions}

The initial condition reads 
\begin{equation}
    \psi(t=0, x) = 
    \begin{cases}
        \sqrt{M^2 - |x|^2} e^{i\cosh \sqrt{x_1^2 + x_2^2}}, & |x|^2 < M^2 \\
        0 & \text{otherwise}
    \end{cases}, \quad M = 4.
\end{equation}
%

We apply the same approximation for the complex exponential as previously explained, and use a ``visual trial-and-error'' discretization of the square root. It yields a number of \( N=9 \) bubbles. 
We emphasize the fact that this discretization may be far from being the best one achievable, however the process of discretizing an arbitrary function into a sum of bubbles is not the main concern of this paper. The discretization of the initial square root is given in Figure \ref{fig: num -- approximation of sqrt with bubbles}.


\begin{figure}
    \centering
    \inputtikz{0.7}{approx_sqroot_with_bubbles}
    \caption{\centering Approximation of \( x\mapsto \sqrt{M^2 - |x|^2} \) as a sum of bubbles}
    \label{fig: num -- approximation of sqrt with bubbles}
\end{figure}



\begin{figure}
    \begin{subfigure}{\textwidth}
        \centering
        \inputtikz{1}{bubblesVSspectral_conservative_quantities_dt-0.001_coeffs-1.0-0.0_testcase3_Nx128_Ny129}
        \caption{\centering Approximate solution to the Harmonic Oscillator \eqref{eqn: cNLS with psi -- linear part}.}
    \end{subfigure}
    \begin{subfigure}{\textwidth}
        \centering
        \inputtikz{1}{bubblesVSspectral_conservative_quantities_dt-0.001_coeffs-1.0-1.0_testcase3_Nx128_Ny129}
        \caption{\centering Approximate solution to the Schrödinger equation \eqref{eqn: cNLS with psi} using DFMP Algorithm.}
    \end{subfigure}
    \caption{\centering Test case 3. Relative evolution of mass, energy and momentum with bubbles and spectral methods. \( \Delta t = 10^{-3} \). Time-integrator for the nonlinear part of the splitting: Runge-Kutta of order 4. Spectral scheme with \( N_x = 128, N_y = 129 \).}
    \label{fig: num -- test case 3}
\end{figure}



% \begin{figure}
%     \centering
%     \inputtikz{0.4}{courbe_cv_Nmin=16_Nmax=2048_testcase=3_HO=false_T=1.0_dt=0.005}
%     \caption{\centering Test case 3. Evolution of the \( L^2 \) norm of the difference between spectral and bubble schemes against the number of points in each direction \( N_x = N_y \), at time \( T = 1 \) with \( dt=5\cdot 10^{-3} \), relative w.r.t. the exact \( L^2 \) norm of the discretized initial data.}
%     \label{fig: num -- test case 3 - evolution of relative l2 norm}
% \end{figure}


The results are displayed in Figure \ref{fig: num -- test case 3}.
This example is by far the most interesting of the three test cases presented in this paper, because it shows that with the discretization given in Figure \ref{fig: num -- approximation of sqrt with bubbles} the conservation properties are pretty good for the Bubbles scheme, even when there are a lot of interactions between bubbles. 
The spectral scheme is globally outperformed by the Bubbles scheme, except for the \( \mathbb{L}^2 \) norm in the presence of interactions, which is better conserved by the spectral scheme. Even in this case, the conservation of this quantity with the DFMP is relatively correct.

The ``jumps'' in the relative evolution of conservative quantities may be explained by an ill-conditioned Gram matrix in DFMP. It also has to be noted that if the discretization presents too much overlap between the gaussian functions, then the DFMP approach fails and the conservative quantities blow up: this has been observed with other discretizations of the same initial data, and is not reported here.

\section{Conclusions}
We consider the phase-extraction problem, and we showed that, given a unitary $U = e^{i\pi H}$ and its inverse $U^{\dag}$, we could implement a block-encoding of $\phi(H)$ for some smooth function $\phi(x)$. The word `smooth' here means existence and continuity of the derivatives: the higher the number of continuous derivatives that a function has, the faster its Fourier sum (and thus the Laurent polynomial on the eigenphases) uniformly converges to that function. We are confident this can have many more applications beyond what is shown in this work. It is also worth remarking that Jackson showed that the convergence rate of a Fourier series is almost-optimal, in the sense that no trigonometric (or, equivalently, complex exponential) series can approximate the desired function faster, up to that $\log d$ factor~\cite[p.\ 21]{jacksonTheoryApproximation1930a}. Also remember that `smoothing' a function, i.e., replacing its derivative with a continuous function, does not give faster convergence for free in general, as its derivative will become steep in the points where we smooth out discontinuities, and this translates to a high Lipschitz constant: a~clear example is given by Eq.~\ref{eq:lipschitz-constant-recurrence-solution}, but in that case, fortunately, nothing depends on the size of the input $N$, and thus does not influence the asymptotic query complexity of Algorithm~\ref{alg:prop-sampling-qsp}, although the constant factor can become large even for $p = 20$. From a theoretical point of view, this work shows that, for any $\eta > 0$, there is an algorithm with query complexity 
$$\Tilde{\bigO}\left(\frac{1}{\bar{c}^{\frac{1}{2} + \eta}} \frac{1}{\epsilon^\eta} \right)$$
solving the proportional-sampling problem. This statement seems to suggest there exists an algorithm which directly solves the problem with $\eta = 0$, and an open question would be to find such algorithm.


It is also interesting to remark that Theorems~\ref{thm:haah-construction},~\ref{thm:haah-completion} indeed allow the construction for any $\phi$, even complex-valued, provided that its absolute value is reciprocal.

One could think that, in Section~\ref{sec:prop-sampling}, instead of using the linear function in the phase-extraction subroutine, we could approximate the square root and then apply the transformation directly on $e^{i \pi c(x)}$. However, in the case of proportional sampling this would be inconvenient, as the derivative of the square root function has a discontinuity with an infinite jump around 0, and we could not choose a constant $\delta$ if we had values of the oracle that are too close to $0$.



\begin{appendices}
    % 

\section{D{\'e}tails sur le changement de variables symplectique pour l'Oscillateur Harmonique ({\`a} retirer de l'article apr{\`e}s relecture)}
\label{sect: appendix HO symplectic change of coordinates}


On a 
\begin{equation*}
    \frac{\textd }{\textd t} \begin{pmatrix}
        k\\B
    \end{pmatrix} = 
    \begin{pmatrix}
        0 & -I \\ I & 0
    \end{pmatrix} \nabla_{(k, B)} \mathcal{E} = J^{-1} \nabla_{(k, B)} \mathcal{E}
,\end{equation*}
%
avec la matrice symplectique usuelle 
\begin{equation*}
    J = \begin{pmatrix}
        0 & I \\ -I & 0
    \end{pmatrix}    
.\end{equation*}
%
Dans les notations de \cite{hairerGeometricNumericalIntegration2006}, on aurait
\begin{equation*}
    \frac{\textd }{\textd t} \begin{pmatrix}
        p\\q
    \end{pmatrix} = J^{-1} \nabla_{(p, q)} \mathcal{E}
,\end{equation*}
%
avec \( p \equiv k \) et \( q \equiv B \). Les {\'e}quations que \cite{hairerGeometricNumericalIntegration2006} donne pour avoir un changement de variables \( (p, q)\mapsto (P, Q) \) symplectique sont les suivantes :
\begin{equation*}
    P = \frac{\partial S}{\partial Q}(q, Q), \qquad p = - \frac{\partial S}{\partial q}(q, Q) 
.\end{equation*}
%
Cela voudrait dire qu'on peut exprimer \( k \) en fonction de \( (B, Q)  \), avec \( Q \) une variable qu'il reste {\`a} d{\'e}terminer. Bien que cela soit a priori faisable, ce n'est pas le plus simple et on voudrait plutôt exprimer \( B \) en fonction de \( k \) et d'une nouvelle variable. Autrement dit, on voudrait plutôt avoir les {\'e}quations
\begin{equation*}
    Q = \frac{\partial S}{\partial P}(p, P), \qquad q = - \frac{\partial S}{\partial p}(p, P) 
.\end{equation*}
%

Pour avoir ces {\'e}quations, on a besoin de reprendre \cite[Sect.~VI.5,~Eqn~(5.1)]{hairerGeometricNumericalIntegration2006} et d'avoir le r{\'e}sultat suivant : le changement de coordonn{\'e}es \( (p,q) \overset{\Phi}{\mapsto} (P, Q) \) est symplectique ssi il existe une fonction \( S(p, q) \) telle que
\begin{equation}
    \label{eqn: appendix change of coords HO -- 5.1bis}
    Q^T dP - q^T dp = dS
.\end{equation}
%

C'est sûrement visible si on est quelqu'un de malin, ce n'est pas mon cas donc je vais reprendre la preuve de \cite{hairerGeometricNumericalIntegration2006} et l'adapter.

\begin{proof}
    On {\'e}crit le jacobien de \( \Phi \):
    \begin{equation*}
       \frac{\partial (P, Q)}{\partial (p, q)} = 
       \begin{pmatrix}
        P_p & P_q \\ Q_p & Q_q
       \end{pmatrix}  
    .\end{equation*}
    %
    On rappelle la condition de symplecticit{\'e} pour \( \Phi \):
    \begin{align*}
        J &= \left( \frac{\partial \Phi}{\partial (p, q)}  \right)^T J \left( \frac{\partial \Phi}{\partial (p, q)}  \right) \\
        &= \begin{pmatrix}
                P_p & P_q \\ Q_p & Q_q
            \end{pmatrix}^T J \begin{pmatrix}
                P_p & P_q \\ Q_p & Q_q
            \end{pmatrix}  \\
        &= \begin{pmatrix}
                P_p^T & Q_p^T \\ P_q^T & Q_q^T
            \end{pmatrix}^T J \begin{pmatrix}
                P_p & P_q \\ Q_p & Q_q
            \end{pmatrix}  \\
        \begin{pmatrix}
        0 & I \\ - I & 0
        \end{pmatrix}
        &= \begin{pmatrix}
            P_p^T Q_p - Q_p^T P_p & P_p^T Q_q - Q_p^T P_q \\
            P_q^T Q_p - Q_q^T P_p & P_q^T Q_q - Q_q^T P_q
        \end{pmatrix}
    .\end{align*}
    %
    On obtient alors les conditions suivantes 
    \begin{align*}
        P_p^T Q_p - Q_p^T P_p &= 0 \\
        P_p^T Q_q - Q_p^T P_q &= I \\
        P_q^T Q_p - Q_q^T P_p &= -I \\
        P_q^T Q_q - Q_q^T P_q &= 0
    \end{align*}
    %
    Les deuxi{\`e}me et troisi{\`e}me lignes expriment la m{\^e}me condition, donc au final on a les conditions suivantes pour que \( \Phi \) soit un changement de coordonn{\'e}es symplectique :
    \begin{equation}
        \label{eqn: appendix change of coords HO -- conditions on Phi to be symplectic}
        P_p^T Q_p - Q_p^T P_p = 0, \quad P_p^T Q_q - Q_p^T P_q = I, \quad P_q^T Q_q - Q_q^T P_q = 0
    \end{equation}
    %
    On utilise maintenant l'{\'e}galit{\'e} 
    \begin{equation}
        dP = P_p dp + P_q dq = \begin{pmatrix}
            P_p \\ P_q
        \end{pmatrix}^T \begin{pmatrix}
            dp \\ dq
        \end{pmatrix}
    \end{equation}
    %
    dans le LHS de \eqref{eqn: appendix change of coords HO -- 5.1bis}:
    \begin{align*}
        Q^T dP - q^T dp
        &= \begin{pmatrix}
            Q^T P_p, & Q^T P_q
        \end{pmatrix} \begin{pmatrix}
            dp \\ dq
        \end{pmatrix} - q^T dp  \\
        &= \begin{pmatrix}
            Q^T P_p - q^T, & Q^T P_q
        \end{pmatrix} \begin{pmatrix}
            dp & dq
        \end{pmatrix} \\
        &= \begin{pmatrix}
            P_p^T Q - q \\ P_q^T Q
        \end{pmatrix}^T \begin{pmatrix}
            dp \\ dq
        \end{pmatrix}
    \end{align*}
    %
    On souhaite appliquer le lemme d'int{\'e}grabilit{\'e}, et pour cela on a besoin de v{\'e}rifier la sym{\'e}trie du jacobien de la matrice coefficient:
    \begin{align*}
        \frac{\partial \begin{pmatrix} P_p^T Q - q \\ P_q^T Q \end{pmatrix}}{\partial (p, q)} 
        &= \begin{pmatrix}
            \frac{\partial}{\partial p} \begin{pmatrix} P_p^T Q - q \\ P_q^T Q \end{pmatrix} &
            \frac{\partial}{\partial q} \begin{pmatrix} P_p^T Q - q \\ P_q^T Q \end{pmatrix}
        \end{pmatrix} \\
        &= \begin{pmatrix}
            P_p^T Q_p + Q^T P_{pp}, & P_p^T Q_q - I + Q^T P_{pq} \\
            P_q^T Q_p + Q^T P_{pq}, & P_q^T Q_q + Q^T P_{qq}
        \end{pmatrix} \\
        &= \begin{pmatrix}
            P_p^T Q_p, & P_p^T Q_q - I \\
            P_q^T Q_p, & P_q^T Q_q
        \end{pmatrix} + Q^T \frac{\partial^2 P}{\partial^2 (p, q)} 
    \end{align*}
    %
    La Hessienne de \( P \) est sym{\'e}trique si on suppose \( \Phi \in C^2( \mathbb{R}^2, \mathbb{R}^2) \), et de plus la matrice qui apparaît dans le RHS est sym{\'e}trique grâce aux conditions \eqref{eqn: appendix change of coords HO -- conditions on Phi to be symplectic}.
    Alors par le lemme d'int{\'e}grabilit{\'e} on peut trouver une fonction \( S(p, q) \) telle que 
    \begin{align*}
        dS &= \begin{pmatrix}
            P_p^T Q - q \\ P_q^T Q
        \end{pmatrix}^T 
        \begin{pmatrix}
            dp \\ dq
        \end{pmatrix}\\
        %
        &= (Q^T P_p - q^T) dp + Q^T P_q dq \\
        &= Q^T \left( P_p dp + P_q dq \right) - q^T dp \\
        &= Q^T dP - q^T dp
    \end{align*}
    %
\end{proof}

L'{\'e}quation \eqref{eqn: appendix change of coords HO -- 5.1bis} sugg{\`e}re de prendre \( (p, P) \) comme nouvelles variables au lieu de \( (p, q) \). C'est un changement de coordonn{\'e}es bien d{\'e}fini si \( q \) peut {\^e}tre exprim{\'e} en terme des nouvelles coordonn{\'e}es \( (p, P) \), ce qui est possible par le th{\'e}or{\`e}me des fonctions implicites si \( \frac{\partial P}{\partial q}  \) est inversible. Notons \( S(p, P) \equiv S(\Phi(p, P)) \). En comparant les coefficients de \( dS = \frac{\partial S}{\partial p} dp + \frac{\partial S}{\partial P} dP  \) avec ceux de \eqref{eqn: appendix change of coords HO -- 5.1bis}, on obtient 
\begin{equation*}
    q = - \frac{\partial S}{\partial p}(p, P), \qquad Q = \frac{\partial S}{\partial P}(p, P) 
.\end{equation*}
%

En reprenant le parall{\`e}le \( (p, q) \leftrightarrow (k, B) \), cela revient {\`a} chercher les variables \( (P, Q) \) telles que 
\begin{equation*}
    B = - \frac{\partial S}{\partial k}(k, P), \qquad Q = \frac{\partial S}{\partial P}(k, P) 
.\end{equation*}
%
Puisqu'on peut facilement exprimer \( B \) en fonction de \( k \) et du Hamiltonien \( \mathcal{E} \), on prend \( P = \mathcal{E} \). Alors 
\begin{equation*}
    Q = \frac{\partial S}{\partial \mathcal{E}}(k, \mathcal{E}) 
.\end{equation*}
%


\begin{remark}
    Cependant, en prenant ces conventions de signe on n'obtient pas les bonnes variables symplectiques dans notre cas, il y a un probl{\`e}me de signe ! On va donc utiliser ce que \cite{hairerGeometricNumericalIntegration2006} appelle les \emph{mixed-variable generating functions}. Cela consiste {\`a} {\'e}crire \( d(Q^T P) = Q^T dP + P^T dQ \) et {\`a} l'injecter dans \eqref{eqn: appendix change of coords HO -- 5.1bis}:
    \begin{equation*}
        - P^T dQ - q^T dp = d(S - Q^T P) \iff P^T dQ + q^T dp = d(Q^T P - S)
    .\end{equation*}
    %
    
    Ensuite, en disant \( dS = S_Q dQ + S_p dp \), on obtient alors les {\'e}quations suivantes :
    \begin{equation*}
        P = \frac{\partial S}{\partial Q}(Q, p), \quad q = \frac{\partial S}{\partial p}(Q, p)
    .\end{equation*}
    %
    Le parall{\`e}le \( (p,q) \leftrightarrow (k, B) \) donne alors :
    \begin{align*}
        P &= \frac{\partial S}{\partial Q}(Q, k), \quad B = \frac{\partial S}{\partial k}(Q, k) \\
        \implies \phi &= \frac{\partial S}{\partial \mathcal{E}}(\mathcal{E}, k), \quad B = \frac{\partial S}{\partial k}(\mathcal{E}, k)
    ,\end{align*}
    %
    ce qui corrige le probl{\`e}me de signe !
\end{remark}
    
    

\section{Computing the coefficients of the linear system \eqref{eqn: linear system for DFMP}}
\label{appendix: populating the linear system for DFMP}

\subsection{Coefficients of the matrix \( \mathbf{A} \)}


\noindent {\underline{\( \langle b_{(d+2)(l-1)+1}, b_{(d+2)(j-1)+1} \rangle \).}}

\begin{align*}
    &\langle b_{(d+2)(l-1)+1}, b_{(d+2)(j-1)+1} \rangle\\
    &= e^{i\gamma_l - i\gamma_j} \int_{ \mathbb{R}^d } e^{iL_l \beta_l \cdot \frac{x - X_l}{L_l} - i \frac{B_l}{4} \left |  \frac{x-X_l}{L_l} \right | ^2} e^{- \frac{1}{2} \left |  \frac{x-X_l}{L_l} \right | ^2} \\
    &\hspace{8em} \times e^{-iL_j \beta_j \cdot \frac{x - X_j}{L_j} + i \frac{B_j}{4} \left |  \frac{x-X_j}{L_j} \right | ^2} e^{- \frac{1}{2} \left |  \frac{x-X_j}{L_j} \right | ^2} dx\\
    &= e^{i(\gamma_l - \gamma_j)} \int_{ \mathbb{R}^d } e^{iL_l \beta_l \cdot \frac{x - X_l}{L_l} - iL_j \beta_j \cdot \frac{x-X_j}{L_j} } e^{-\frac{2+iB_l}{4} \left |  \frac{x-X_l}{L_l} \right | ^2} e^{- \frac{2-iB_j}{4}  \left |  \frac{x-X_j}{L_j} \right | ^2} dx\\
    &= e^{i(\gamma_l - \gamma_j) - \frac{2+iB_l}{4L_l^2}  | X_l | ^2 - \frac{2-iB_j}{4L_j^2}  | X_j | ^2 - i\beta_l \cdot X_l + i\beta_j \cdot X_j } \\
    &\qquad \times \int_{ \mathbb{R}^d } e^{i(\beta_l - \beta_j) \cdot x} e^{-\frac{2+iB_l}{4L_l^2} ( | x | ^2 - 2x\cdot X_l)} e^{-\frac{2-iB_j}{4L_j^2} ( | x | ^2 - 2x\cdot X_j)} dx \\
    &= e^{i(\gamma_l - \gamma_j) - \frac{2+iB_l}{4L_l^2}  | X_l | ^2 - \frac{2-iB_j}{4L_j^2}  | X_j | ^2 - i\beta_l \cdot X_l + i\beta_j \cdot X_j } \\
    &\qquad \times\int_{ \mathbb{R}^d } e^{i(\beta_l - \beta_j + \frac{B_l}{2L_l^2} X_l - \frac{B_j}{2L_j^2} X_j  ) \cdot x} e^{x\cdot \left( \frac{1}{L_l^2} X_l + \frac{1}{L_j^2} X_j  \right)} e^{-\left(\frac{2-iB_j}{4L_j^2} + \frac{2+iB_l}{4L_l^2} \right)  | x | ^2} dx
\end{align*}
%
Let
\begin{equation}
    \left |  \begin{aligned}
        z &:= \frac{2+iB_l}{4L_l^2} + \frac{2-iB_j}{4L_j^2}, \\
        a &:= \frac{X_l}{L_l^2} + \frac{X_j}{L_j^2}, \\
        \xi &:= \frac{B_j}{2L_j^2} X_j + \beta_j - \frac{B_l}{2L_l^2} X_l - \beta_l, \\
        C &= \exp\left\{i(\gamma_l - \gamma_j) - \frac{2+iB_l}{4L_l^2}  | X_l | ^2 - \frac{2-iB_j}{4L_j^2}  | X_j | ^2 - i\beta_l \cdot X_l + i\beta_j \cdot X_j \right\},
    \end{aligned}
    \right.
\end{equation}
%
and \( f(x) := e^{-z | x | ^2 + a\cdot x} \). Then
\begin{equation*}
    \langle b_{(d+2)l-d-1}, b_{(d+2)j-d-1} \rangle
    = C \int_{ \mathbb{R}^d } e^{-i\xi\cdot x} f(x) dx = C \widehat{f}(\xi)
\end{equation*}
%



\noindent {\underline{\( \langle b_{(d+2)(l-1)+n+1}, b_{(d+2)(j-1)+1} \rangle \), \( 1 \leq n \leq d \).}}
%

\begin{align*}
    \langle b_{(d+2)(l-1)+1+n}, b_{(d+2)(j-1)+1} \rangle
    &= C \int_{ \mathbb{R}^d } \frac{(x-X_l)_n}{L_l} e^{-i\xi\cdot x} f(x) dx\\
    &= \frac{C}{L_l} \left( \widehat{x f}_n - (X_l)_n \widehat{f} \right)(\xi)
\end{align*}
%





\noindent {\underline{\( \langle b_{(d+2)l}, b_{(d+2)(j-1)+1} \rangle \) }}

\begin{align*}
    \langle b_{(d+2)l}, b_{(d+2)(j-1)+1} \rangle
    &= C \int_{ \mathbb{R}^d } e^{-i\xi\cdot x} f(x) \frac{ | x-X_l | ^2}{L_l^2} dx \\
    &= \frac{C}{L_l^2}  \int_{ \mathbb{R}^d } e^{-i\xi\cdot x} f(x) \left(  | x | ^2 - 2x\cdot X_l +  | X_l | ^2 \right) dx\\
    &= \frac{C}{L_l^2} \left( \widehat{ | x | ^2 f} - 2X_l \cdot \widehat{x f} +  | X_l | ^2 \widehat{f} \right)(\xi)
\end{align*}
%





\noindent {\underline{\( \langle b_{(d+2)(l-1)+n+1}, b_{(d+2)(j-1)+m+1} \rangle \), \( 1 \leq n, m \leq d \).}}

\begin{align*}
    \langle b_{(d+2)(l-1)+n+1}, b_{(d+2)(j-1)+m+1} \rangle
    &= C \int_{ \mathbb{R}^d } \frac{x_n-(X_l)_n}{L_l} \frac{x_m-(X_j)_m}{L_j}  e^{-i\xi\cdot x} f(x) dx \\
    &= \frac{C}{L_j L_l}  \int_{ \mathbb{R}^d } (x_n-(X_l)_n) (x_m-(X_j)_m)  e^{-i\xi\cdot x} f(x) dx \\
    &= \frac{C}{L_j L_l}  \int_{ \mathbb{R}^d } \left[ x_n x_m - x_n (X_j)_m - x_m (X_l)_n + (X_l)_n (X_j)_m \right] \\
        &\hspace{2cm} \times e^{-i\xi\cdot x} f(x) dx \\
    &= \frac{C}{L_j L_l} \left[ \widehat{x_n x_m f} - (X_l)_n \widehat{x_m f} - (X_j)_m \widehat{x_n f} + (X_l)_n (X_j)_m \widehat{f} \right](\xi)
.\end{align*}
%



\noindent {\underline{\( \langle b_{(d+2)l}, b_{(d+2)(j-1)+m+1} \rangle \), \( 1 \leq m \leq d\).}}


\begin{align*}
    &\langle b_{(d+2)l}, b_{(d+2)(j-1)+m+1} \rangle \\
    &= C \int_{ \mathbb{R}^d } e^{-i\xi\cdot x} e^{-z | x | ^2 + a\cdot x} \frac{ | x-X_l | ^2}{L_l^2} \frac{x_m-(X_j)_m}{L_j} dx \\
    &= \frac{C}{L_l^2 L_j} \int_{ \mathbb{R}^d } e^{-i\xi\cdot x} e^{-z | x | ^2 + a\cdot x} \left(  | x | ^2 - 2x\cdot X_l +  | X_l | ^2 \right) \left( x_m - (X_j)_m \right) dx \\
    &= \frac{C}{L_l^2 L_j} \left[ \widehat{x_m | x | ^2 f} - 2X_l \cdot \widehat{x_m x f} + | X_l | ^2 \widehat{x_m f} \right.\\
    &\hspace{1cm} \left. - (X_j)_m \widehat{ | x | ^2 f} + 2(X_j)_m X_l \cdot \widehat{x f} - | X_l | ^2 (X_j)_m \widehat{f} \right](\xi)
.\end{align*}
%




\noindent {\underline{\( \langle b_{(d+2)l}, b_{(d+2)j} \rangle \) }}


\begin{align*}
    \langle b_{(d+2)l}, b_{(d+2)j} \rangle 
    &= C \int_{ \mathbb{R}^d } e^{-i\xi\cdot x} e^{-z | x | ^2 + a\cdot x} \frac{ | x-X_l | ^2}{L_l^2} \frac{ | x-X_j | ^2}{L_j^2} dx \\
    &= \frac{C}{L_l^2 L_j^2} \int_{ \mathbb{R}^d } e^{-i\xi\cdot x} e^{-z | x | ^2 + a\cdot x} \left(  | x | ^2 - 2x\cdot X_l +  | X_l | ^2 \right) \\
    &\hspace{2cm} \times \left(  | x | ^2 -2x\cdot X_j +  | X_j | ^2 \right) dx \\
    &= \frac{C}{L_l^2 L_j^2} \int_{ \mathbb{R}^d } e^{-i\xi\cdot x} e^{-z | x | ^2 + a\cdot x} \left(  | x | ^4 - 2 | x | ^2 x\cdot X_l +  | X_l | ^2  | x | ^2 \right. \\
    &\hspace{2cm} - 2(x\cdot X_j) | x | ^2 + 4(x\cdot X_l)(x\cdot X_j) - 2(x\cdot X_j)  | X_l | ^2 \\
    &\hspace{2cm} \left. +  | x | ^2  | X_j | ^2 - 2(x\cdot X_l)  | X_j | ^2 +  | X_l | ^2  | X_j | ^2 \right) dx \\
    &= \frac{C}{L_l^2 L_j^2} \left[ \widehat{ | x | ^4 f} - 2X_l \cdot \widehat{ | x | ^2 x f} +  | X_l | ^2 \widehat{ | x | ^2 f} \right. \\
    &\hspace{2cm} - 2X_j \cdot \widehat{x | x | ^2 f} + 4\widehat{(x\cdot X_l)(x\cdot X_j) f} - 2 | X_l | ^2 X_j \cdot \widehat{x f} \\
    &\hspace{2cm} \left. + \widehat{ | x | ^2 f}  | X_j | ^2 - 2 | X_j | ^2 X_l \cdot \widehat{x f} +  | X_l | ^2  | X_j | ^2 \widehat{f} \right](\xi)
\end{align*}
%
Moreover,
\begin{align*}
    (x\cdot X_l)(x\cdot X_j) 
    &= \left( \sum_{n=1}^d x_n (X_l)_n \right) \left( \sum_{m=1}^d x_m (X_j)_m \right) \\
    &= \sum_{n,m = 1}^d (X_l)_n (X_j)_m x_n x_m,
\end{align*}
%
Hence
\begin{align*}
    \widehat{(x\cdot X_l)(x\cdot X_j) f} &= \sum_{n, m=1}^d (X_l)_n (X_j)_m \widehat{x_n x_m f}
.\end{align*}
%




\subsection{Coefficients of the vector of interactions \( S \)}



\noindent {\underline{\( \langle u|u|^2, b_{(d+2)(j-1)+1} \rangle \)}}

\begin{align*}
    \langle u|u|^2, b_{(d+2)j-d-1} \rangle 
    &= \sum_{k,l,m} \frac{A_k A_l A_m}{L_k L_l L_m} \left\langle e^{i\Gamma_k + i\Gamma_l - i\Gamma_m} e^{-\frac{|y_k|^2 + |y_l|^2 + |y_m|^2}{2}}, e^{i\Gamma_j} e^{-\frac{1}{2} |y_j|^2} \right\rangle
\end{align*}
%
We recall the previously defined notations:
\begin{equation*}
    \left| 
        \begin{aligned}
            y_k &= \frac{x-X_k}{L_k} \\
            \Gamma_k(x) &= \gamma_k + \beta_k \cdot (x-X_k) - \frac{B_k}{4L_k^2} |x-X_k|^2
        \end{aligned}
    \right.
\end{equation*}
%
Then,
\begin{align*}
    &|y_k|^2 + |y_l|^2 + |y_m|^2 + |y_j|^2
    = \frac{1}{L_k^2} |x-X_k|^2 + \frac{1}{L_l^2} |x-X_l|^2 + \frac{1}{L_m^2} |x-X_m|^2 \\
    &\hspace{12em}+ \frac{1}{L_j^2} |x-X_j|^2 \\
    &= \left( \frac{1}{L_k^2} + \frac{1}{L_l^2} + \frac{1}{L_m^2} + \frac{1}{L_j^2} \right) |x|^2 - 2\left( \frac{1}{L_k^2} X_k + \frac{1}{L_l^2} X_l + \frac{1}{L_m^2} X_m + \frac{1}{L_j^2} X_j  \right) \cdot x \\
    &\qquad + \left( \frac{|X_k|^2}{L_k^2} + \frac{|X_l|^2}{L_l^2} + \frac{|X_m|^2}{L_m^2} + \frac{|X_j|^2}{L_j^2} \right)
,\end{align*}
%
and
\begin{align*}
    &(\Gamma_k + \Gamma_l - \Gamma_m - \Gamma_j) \\
    &= \gamma_k + \beta_k \cdot (x-X_k) - \frac{B_k}{4L_k^2} |x-X_k|^2 + \gamma_l + \beta_l \cdot (x-X_l) - \frac{B_l}{4L_l^2} |x-X_l|^2 \\
    &\quad - \gamma_m - \beta_m \cdot (x-X_m) + \frac{B_m}{4L_m^2} |x-X_m|^2 - \gamma_j - \beta_j \cdot (x-X_j) + \frac{B_j}{4L_j^2} |x-X_j|^2 \\
    &= (\gamma_k + \gamma_l - \gamma_m - \gamma_j) + (\beta_j \cdot X_j + \beta_m \cdot X_m - \beta_l\cdot X_l - \beta_k \cdot X_k) \\
    &\quad - \left( \frac{B_k}{4L_k^2} |X_k|^2 + \frac{B_l}{4L_l^2} |X_l|^2 - \frac{B_m}{4L_m^2} |X_m|^2 - \frac{B_j}{4L_j^2} |X_j|^2  \right) \\
    &\quad + x \cdot \left( \beta_k + \beta_l - \beta_m - \beta_j + \frac{B_k}{2L_k^2} X_k + \frac{B_l}{2L_l^2} X_l - \frac{B_m}{2L_m^2} X_m - \frac{B_j}{2L_j^2} X_j \right)\\
    &\quad - \left( \frac{B_k}{4L_k^2} + \frac{B_l}{4L_l^2} - \frac{B_m}{4L_m^2} - \frac{B_j}{4L_j^2} \right) |x|^2
\end{align*}
%

Define
\begin{equation*}
    \left| 
        \begin{aligned}
            C_\Im &:= \exp\left\{ i\left( \gamma_k + \gamma_l - \gamma_m - \gamma_j \right) \right\} \\
            &\hspace{1em} \times \exp\left\{ i\left(\beta_j \cdot X_j + \beta_m \cdot X_m - \beta_l\cdot X_l - \beta_k \cdot X_k \right) \right\} \\
            &\hspace{1em} \times \exp\left\{- i\left( \frac{B_k}{4L_k^2} |X_k|^2 + \frac{B_l}{4L_l^2} |X_l|^2 - \frac{B_m}{4L_m^2} |X_m|^2 - \frac{B_j}{4L_j^2} |X_j|^2  \right) \right\}  \\
            C_\Re &:= \exp\left\{-\frac{1}{2} \left( \frac{|X_k|^2}{L_k^2} + \frac{|X_l|^2}{L_l^2} + \frac{|X_m|^2}{L_m^2} + \frac{|X_j|^2}{L_j^2} \right) \right\} \\
            C &:= \frac{A_k A_l A_m}{L_k L_l L_m} C_\Im C_\Re \\
            \xi &:= -\left[ \beta_k + \beta_l - \beta_m - \beta_j + \frac{B_k}{2L_k^2} X_k + \frac{B_l}{2L_l^2} X_l - \frac{B_m}{2L_m^2} X_m - \frac{B_j}{2L_j^2} X_j \right] \\
            z &:= \frac{1}{2} \left( \frac{1}{L_k^2} + \frac{1}{L_l^2} + \frac{1}{L_m^2} + \frac{1}{L_j^2} \right) + i \left( \frac{B_k}{4L_k^2} + \frac{B_l}{4L_l^2} - \frac{B_m}{4L_m^2} - \frac{B_j}{4L_j^2} \right) \\
            a &:= \frac{1}{L_k^2} X_k + \frac{1}{L_l^2} X_l + \frac{1}{L_m^2} X_m + \frac{1}{L_j^2} X_j
        \end{aligned}
    \right.
\end{equation*}
%
and \( f(x) := e^{-z | x | ^2 + a\cdot x} \). Then
\begin{equation}
    \langle u|u|^2, b_{(d+2)(j-1)+1} \rangle = \sum_{k,l,m} C \widehat{f}(\xi)
.\end{equation}
%


\noindent {\underline{\( \langle u|u|^2, b_{(d+2)(j-1)+r+1} \rangle \), \( r = 1, \dots, d \)}}

\begin{align*}
    &\langle u|u|^2, b_{(d+2)(j-1)+r+1} \rangle \\
    &= \sum_{k,l,m} \frac{A_k A_l A_m}{L_k L_l L_m} \left\langle e^{i\Gamma_k + i\Gamma_l - i\Gamma_m} e^{-\frac{|y_k|^2 + |y_l|^2 + |y_m|^2}{2}}, e^{i\Gamma_j} e^{-\frac{1}{2} |y_j|^2} \frac{x_r-(X_j)_r}{L_j} \right\rangle \\
    &= \sum_{k,l,m} \frac{C}{L_j} \left( \widehat{x_r f} - (X_j)_r \widehat{f} \right).
\end{align*}
%




\noindent {\underline{\( \langle u|u|^2, b_{(d+2)j} \rangle \)}}

\begin{align*}
    &\langle u|u|^2, b_{(d+2)j} \rangle \\
    &= \sum_{k,l,m} \frac{A_k A_l A_m}{L_k L_l L_m} \left\langle e^{i\Gamma_k + i\Gamma_l - i\Gamma_m} e^{-\frac{|y_k|^2 + |y_l|^2 + |y_m|^2}{2}}, e^{i\Gamma_j} e^{-\frac{1}{2} |y_j|^2} \left|\frac{x-X_j}{L_j}\right|^2 \right\rangle \\
    &= \sum_{k,l,m} \frac{C}{L_j^2} \left( \widehat{|x|^2 f} - 2X_j \cdot \widehat{xf} + |X_j|^2 \widehat{f} \right).
\end{align*}
%
    

\section{Fourier transforms of Gaussians}
\label{appendix: fourier transforms of gaussians}



\begin{lemma}[Fourier transform of complex Gaussians]
    Let \( z\in \mathbb{C},\ \Re(z) \geq 0 \).
    Then, 
    \begin{equation}
        \mathcal{F}\left( e^{-z|\cdot|^2} \right)(\xi) = \left( \frac{\pi}{z} \right)^{\frac{d}{2}} e^{-\frac{| \xi |^2}{4z} }, \quad \xi \in \mathbb{R}^d
    .\end{equation}
    %
    More generally, let \( z = z_1 + iz_2 \in \mathbb{C} \), \( z_1, z_2 \in \mathbb{R}, \ z_1 > 0 \), \( a\in \mathbb{R}^d \) and 
    \begin{equation}
        f:\ x \in \mathbb{R}^d \mapsto \exp\left( -z|x|^2 + a\cdot x \right) \in \mathbb{C}
    ,\end{equation}
    %
    then we have the Fourier transforms given by Table \ref{table: useful Fourier transforms}.
\end{lemma}




\begin{table}
    \begin{center}
    \begin{tabular}{cc}
        \hline
        \( h(x) \) & \( \widehat{h}(\xi) / e^{- \frac{(\xi + ia)\cdot (\xi + ia)}{4z}} \) \\[0.5em]
        \hline
        \( f \) & \( \left( \frac{\pi}{z}  \right)^{\frac{d}{2}} \)\\[1em]
        \( xf \) & \( -i \left( \frac{\pi}{z} \right)^{\frac{d}{2}} \frac{\xi + ia}{2z} \)\\[1em]
        \( x_m x_n f \) & \( - \frac{1}{4z^2} \left(\frac{\pi}{z} \right)^{\frac{d}{2}}  \left( \xi_n + ia_n \right) \left( \xi_m + ia_m \right) \) \\[1em]
        \( x_m^2 f \) & \( \frac{1}{2z} \left(\frac{\pi}{z} \right)^{\frac{d}{2}}  \left[ 1 - \frac{(\xi_m + ia_m)^2}{2z} \right] \) \\[1em]
        \( | x |^2 f \) & \( \frac{1}{2z} \left(\frac{\pi}{z} \right)^{\frac{d}{2}} \left[ d - \frac{|\xi|^2 + 2ia\cdot \xi - |a|^2}{2z} \right] \) \\[1em]
        \( x_m | x |^2 f \) & \( - \frac{i}{4z^2} \left(\frac{\pi}{z} \right)^{\frac{d}{2}} \left( \xi_m + ia_m \right) \left[ d+2 - \frac{|\xi|^2 + 2ia\cdot \xi - |a|^2}{2z} \right] \) \\[1em]
        \( x_m^2 x_n^2 f \) & \( \frac{1}{4z^2} \left( \frac{\pi}{z} \right)^{\frac{d}{2}} \left( 1 - \frac{(\xi_n + ia_n)^2}{2z} \right) \left( 1 - \frac{(\xi_m + ia_m)^2}{2z} \right) \) \\[1em]
        \( x_m^4 f \) & \( \frac{1}{4z^2} \left(\frac{\pi}{z} \right)^{\frac{d}{2}} \left[ 3 - 6 \frac{(\xi_m + ia_m)^2}{2z} + \frac{(\xi_m + ia_m)^4}{4z^2} \right] \) \\[0.5em]
        \hline
    \end{tabular}
    \caption{\centering Fourier Transform of some polynomials in \( x = (x_1, \dots, x_d) \in \mathbb{R}^d \) multiplied by \( f(x) = e^{-z|x|^2 +a\cdot x}, \, z\in \mathbb{C}, \Re(z) > 0, a \in \mathbb{R}^d \).}
    \label{table: useful Fourier transforms}
    \end{center}
\end{table}
%


\begin{proof}

For the sake of clarity, for \( \xi, a \in \mathbb{R}^d \) and \( z\in \mathbb{C} \), let
\begin{equation*}
    E(\xi, a, z) := \exp\left\{- \frac{|\xi|^2 + 2ia\cdot \xi - |a|^2}{4z} \right\} = \exp\left\{- \frac{(\xi + ia)\cdot (\xi+ia)}{4z} \right\}
.\end{equation*}
%

\noindent \underline{\( \widehat{f} \).}

We have
\begin{equation*}
    -z|x|^2 + a\cdot x = -z\left| x - \frac{a}{2z_1} \right|^2 - i \frac{z_2 a}{z_1} \cdot x + \frac{z|a|^2}{4z_1^2}
.\end{equation*}
%
Recall the following usual properties on Fourier transform:
\begin{equation*}
    \widehat{f(x-a)} = \widehat{f}(\xi)e^{-ia\cdot \xi},\quad \widehat{fe^{-ia\cdot x}} = \widehat{f}(\xi+a)
.\end{equation*}
%
Let
\begin{equation*}
    g(x) = e^{-z\left| x - \frac{a}{2z_1} \right|^2}
,\end{equation*}
%
then
\begin{equation*}
    \hat{g}(\xi) = \left( \frac{\pi}{z} \right)^{\frac{d}{2} } e^{-\frac{|\xi|^2}{4z} - \frac{ia\cdot \xi}{2z_1} } 
\end{equation*}
%
and
\begin{equation*}
    f(x) = g(x) e^{-\frac{iz_2}{z_1} a\cdot x + \frac{z|a|^2}{4z_1^2} } 
.\end{equation*}
%
Hence,
\begin{align*}
    \widehat{f}(\xi)
    &= e^{\frac{z|a|^2}{4z_1^2}} \hat{g}\left( \xi + \frac{z_2}{z_1} a \right) = \left( \frac{\pi}{z} \right)^{\frac{d}{2} } e^{\frac{z|a|^2}{4z_1^2}} e^{-\frac{1}{4z} \left| \xi + \frac{z_2}{z_1} a \right|^2 - \frac{ia}{2z_1} \cdot \left( \xi + \frac{z_2}{z_1} a \right)}\\
    &= \left( \frac{\pi}{z}  \right)^{\frac{d}{2} } e^{\frac{z|a|^2}{4z_1^2} - \frac{1}{4z} \left( |\xi|^2 + 2\frac{z_2}{z_1} a\cdot \xi + \frac{z_2^2}{z_1^2} |a|^2 \right) - \frac{ia\cdot \xi}{2z_1} - \frac{i|a|^2 z_2}{2z_1^2} } \\
    &= \left( \frac{\pi}{z}  \right)^{\frac{d}{2} } e^{-\frac{|\xi|^2}{4z} + (a\cdot \xi)\left( -\frac{z_2}{2zz_1} - \frac{i}{2z_1}  \right) + |a|^2 \left( \frac{z}{4z_1^2} - \frac{z_2^2}{4zz_1^2} - \frac{iz_2}{2z_1^2}  \right)} \\
    &= \left( \frac{\pi}{z}  \right)^{\frac{d}{2}} e^{-\frac{|\xi|^2}{4z} - \frac{a\cdot \xi}{2zz_1} \left[ z_2 + i(z_1 + iz_2) \right] + \frac{|a|^2}{4zz_1^2} \left[ (z_1+iz_2)^2 - z_2^2 - 2iz_2(z_1+iz_2) \right]} \\
    &= \left( \frac{\pi}{z}  \right)^{\frac{d}{2}} e^{-\frac{|\xi|^2}{4z} - i\frac{a\cdot \xi}{2z} + \frac{|a|^2}{4z}} \\
    &= \left( \frac{\pi}{z}  \right)^{\frac{d}{2}} E(\xi, a, z).
\end{align*}
%

\vspace{1em}



\noindent \underline{\( \widehat{x f} \).}

\begin{align*}
    \widehat{xf}(\xi) &= i\nabla_\xi \hat{f} = i \nabla_\xi \left[ \left( \frac{\pi}{z} \right)^{\frac{d}{2}}  E(\xi, a, z) \right] = i\left( \frac{\pi}{z} \right)^{\frac{d}{2}} E(\xi, a, z) \left[ -\frac{\xi}{2z} -\frac{ia}{2z}  \right] \\
    &= -i \left( \frac{\pi}{z} \right)^{\frac{d}{2}} \frac{\xi + ia}{2z} E(\xi, a, z).
\end{align*}
%
\vspace{1em}




\noindent \underline{\( \widehat{x_m^2 f} \), \( m=1, \dots, d \).}

\begin{align*}
    \widehat{x_m^2 f}(\xi) &= i\partial_{\xi_m} \left(\widehat{xf}\right)_m = i \partial_{\xi_m} \left[ -i \left( \frac{\pi}{z} \right)^{\frac{d}{2}} \frac{(\xi + ia)_m}{2z} E(\xi, a, z)   \right] \\
    &= \frac{1}{2z} \left(\frac{\pi}{z} \right)^{\frac{d}{2}} \left[ E(\xi, a, z) + \left( \xi_m + ia_m \right) E(\xi, a, z) \left( -\frac{\xi_m}{2z} -\frac{ia_m}{2z}  \right) \right] \\
    &= \frac{1}{2z} \left(\frac{\pi}{z} \right)^{\frac{d}{2}}  \left[ 1 - \frac{(\xi_m + ia_m)^2}{2z} \right] E(\xi, a, z).
\end{align*}
%
\vspace{1em}




\noindent \underline{\( \widehat{x_m x_n f} \), \( m, n=1, \dots, d, \, n\neq m \).}

\begin{align*}
    \widehat{x_m x_n f}(\xi) &= i\partial_{\xi_m} \left(\widehat{xf}\right)_n = i \partial_{\xi_m} \left[ -i \left( \frac{\pi}{z} \right)^{\frac{d}{2}} \frac{\xi_n + ia_n}{2z} E(\xi, a, z)   \right] \\
    &= \frac{1}{2z} \left(\frac{\pi}{z} \right)^{\frac{d}{2}} \left( \xi_n + ia_n \right) \left[ - \frac{\xi_m + ia_m}{2z}  \right] E(\xi, a, z) \\
    &= - \frac{1}{4z^2} \left(\frac{\pi}{z} \right)^{\frac{d}{2}}  \left( \xi_n + ia_n \right) \left( \xi_m + ia_m \right) E(\xi, a, z).
\end{align*}
%
\vspace{1em}




\noindent \underline{\( \widehat{|x|^2 f} \).}

\begin{align*}
    \widehat{|x|^2 f}(\xi) &= \widehat{x_1^2 f}(\xi) + \dots + \widehat{x_d^2 f}(\xi) \\
    &= \frac{1}{2z} \left(\frac{\pi}{z} \right)^{\frac{d}{2}} \left[ d - \frac{(\xi_1 + ia_1)^2 + \dots + (\xi_d + ia_d)^2}{2z} \right] E(\xi, a, z) \\
    &= \frac{1}{2z} \left(\frac{\pi}{z} \right)^{\frac{d}{2}} \left[ d - \frac{|\xi|^2 + 2ia\cdot \xi - |a|^2}{2z} \right] E(\xi, a, z).
\end{align*}
%
\vspace{1em}






\noindent \underline{\( \widehat{x_m|x|^2 f} \), \( m=1, \dots, d \).}

\begin{align*}
    &\widehat{x_m|x|^2 f}(\xi) \\
    &= i \partial_{\xi_m}\left[ \widehat{|x|^2 f}(\xi) \right] = i \partial_{\xi_m} \left[ \frac{1}{2z} \left(\frac{\pi}{z} \right)^{\frac{d}{2}} \left( d - \frac{|\xi|^2 + 2ia\cdot \xi - |a|^2}{2z} \right) E(\xi, a, z) \right] \\
    &= \frac{i}{2z} \left(\frac{\pi}{z} \right)^{\frac{d}{2}} \left[ -2\frac{\xi_m + ia_m}{2z} + \left( d - \frac{|\xi|^2 + 2ia\cdot \xi - |a|^2}{2z} \right) \left( - \frac{\xi_m + ia_m}{2z}  \right)  \right]  E(\xi, a, z) \\
    &= - \frac{i}{4z^2} \left(\frac{\pi}{z} \right)^{\frac{d}{2}} \left( \xi_m + ia_m \right) \left[ d+2 - \frac{|\xi|^2 + 2ia\cdot \xi - |a|^2}{2z} \right]  E(\xi, a, z)
.\end{align*}
%
\vspace{1em}





\noindent \underline{\( \widehat{x_m^3 f} \), \( m=1, \dots, d\).}

\begin{align*}
    \widehat{x_m^3 f}(\xi) 
    &= i \partial_{\xi_m} \left[ \widehat{x_m^2 f}(\xi) \right] = i \partial_{\xi_m} \left[ \frac{1}{2z} \left(\frac{\pi}{z} \right)^{\frac{d}{2}} \left( 1 - \frac{(\xi_m + ia_m)^2}{2z} \right) E(\xi, a, z) \right] \\
    &= \frac{i}{2z} \left(\frac{\pi}{z} \right)^{\frac{d}{2}} \left[ -2 \frac{\xi_m + ia_m}{2z} + \left( - \frac{\xi_m + ia_m}{2z}  \right) \left( 1 - \frac{(\xi_m + i a_m)^2}{2z}  \right) \right] E(\xi, a, z)\\
    &= -\frac{i}{4z^2} \left(\frac{\pi}{z} \right)^{\frac{d}{2}} \left(\xi_m + ia_m\right) \left[ 3 - \frac{(\xi_m + ia_m)^2}{2z} \right] E(\xi, a, z) \\
    &= -\frac{i}{4z^2} \left(\frac{\pi}{z} \right)^{\frac{d}{2}} \left[ 3(\xi_m+ia_m) - \frac{(\xi_m + ia_m)^3}{2z} \right] E(\xi, a, z).
\end{align*}
%
\vspace{1em}




\noindent \underline{\( \widehat{x_m x_n^2 f} \), \( m, n=1, \dots, d,\, n\neq m \).}

\begin{align*}
    \widehat{x_m x_n^2 f}(\xi) &= i\partial_{\xi_m} \left(\widehat{x_n^2 f}\right) = i \partial_{\xi_m} \left[ \frac{1}{2z} \left(\frac{\pi}{z} \right)^{\frac{d}{2}}  \left( 1 - \frac{(\xi_n + ia_n)^2}{2z} \right) E(\xi, a, z) \right] \\
    &= - \frac{i}{2z} \left(\frac{\pi}{z} \right)^{\frac{d}{2}}  \left( 1 - \frac{(\xi_n + ia_n)^2}{2z} \right) \frac{\xi_m + ia_m}{2z} E(\xi, a, z).
\end{align*}
%
\vspace{1em}




\noindent \underline{\( \widehat{x_m^4 f} \), \( m=1, \dots, d \).}

\begin{align*}
    &\widehat{x_m^4 f}(\xi) \\
    &= i \partial_{\xi_m} \left[ \widehat{x_m^3 f}(\xi) \right] = i \partial_{\xi_m} \left[ -\frac{i}{4z^2} \left(\frac{\pi}{z} \right)^{\frac{d}{2}} \left( 3(\xi_m+ia_m) - \frac{(\xi_m + ia_m)^3}{2z} \right) E(\xi, a, z) \right] \\
    &= \frac{1}{4z^2} \left(\frac{\pi}{z} \right)^{\frac{d}{2}} \left[ 3 - 3\frac{(\xi+ia_m)^2}{2z} + \left( 3(\xi_m+ia_m) - \frac{(\xi_m + ia_m)^3}{2z} \right) \left( - \frac{\xi_m + ia_m}{2z}  \right) \right] E(\xi, a, z) \\
    &= \frac{1}{4z^2} \left(\frac{\pi}{z} \right)^{\frac{d}{2}} \left[ 3 - 6 \frac{(\xi_m + ia_m)^2}{2z} + \frac{(\xi_m + ia_m)^4}{4z^2} \right] E(\xi, a, z).
\end{align*}
%

\vspace{1em}


\noindent \underline{\( \widehat{x_m^2 x_n^2 f} \), \( m=1, \dots, d,\, n\neq m \).}

\begin{align*}
    \widehat{x_m^2 x_n^2 f}(\xi) &= i\partial_{\xi_m} \left(\widehat{x_m x_n^2 f}\right)_n = i \partial_{\xi_m} \left[ - \frac{i}{2z} \left(\frac{\pi}{z} \right)^{\frac{d}{2}} \left( 1 - \frac{(\xi_n + ia_n)^2}{2z} \right) \frac{\xi_m + ia_m}{2z} E(\xi, a, z) \right] \\
    &= \frac{1}{4z^2} \left( \frac{\pi}{z} \right)^{\frac{d}{2}} \left( 1 - \frac{(\xi_n + ia_n)^2}{2z} \right) \partial_{\xi_m} \left[ (\xi_m + ia_m) E(\xi, a, z)  \right] \\
    &= \frac{1}{4z^2} \left( \frac{\pi}{z} \right)^{\frac{d}{2}} \left( 1 - \frac{(\xi_n + ia_n)^2}{2z} \right) \left( 1 - \frac{(\xi_m + ia_m)^2}{2z} \right) E(\xi, a, z).
\end{align*}
%


\end{proof}
    

\section{Miscellaneous computations}
\label{appendix: Miscellaneous computations}


We provide in this section some miscellaneous computations, which hold in dimension \( d=2 \) as long as \( v_j(s_j, y_j) = e^{- \frac{|y_j|^2}{2} },\ j=1, \dots, N \).



\subsection{Conservative quantities in dimension \( d=2 \)}

We give the explicit expressions for the conservative quantities involved in Lemma \ref{lemma: conserved quantities in HO}, in the two-dimensional case.

The \( \mathbb{L}^2 \) norm of a sum of \( N \) bubbles is given by
\begin{align*}
    \|u\|_{ \mathbb{L}^2 }^2 &= \sum_{k,l=1}^N \frac{A_k A_l}{L_k L_l} \langle b_{k, 1}, b_{l, 1} \rangle.
\end{align*}
%


The energy of a sum of bubbles is given by
\begin{equation*}
    E_{\mu,\lambda} = \frac{\mu}{2} \left\langle -\Delta u + |x|^2 u, u \right\rangle + \frac{\lambda}{4} \left\langle |u|^2 u, u \right\rangle
    = E_{\mu, 0} + E_{0,\lambda} = \mu E_{1, 0} + \lambda E_{0, 1}
.\end{equation*}
%
We have
\begin{equation*}
    2E_{1, 0} = \langle Hu, u \rangle = \langle -\Delta u, u \rangle + \langle |x|^2 u, u \rangle = \sum_{j,k=1}^N \langle \nabla_x u_j, \nabla_x u_k \rangle + \sum_{j,k=1}^N \langle |x|^2 u_j, u_k \rangle
.\end{equation*}
%
Furthermore,
\begin{align*}
    \langle \nabla_x u_j, \nabla_x u_k \rangle 
    &= \frac{A_j A_k}{L_j L_k}  \left\langle \left(i\beta_j - \frac{2+iB_j}{2L_j} y_j\right) b_{j, 1}, \left(i\beta_k - \frac{2+iB_k}{2L_k} y_k\right) b_{k, 1} \right\rangle \\
    &= \frac{A_j A_k}{L_j L_k} \left\{ \beta_j \cdot \beta_k  \left\langle b_{j, 1}, b_{k, 1} \right\rangle +i \frac{2+iB_j}{2L_j} \beta_k \cdot \begin{pmatrix}
            \left\langle b_{j, 2},  b_{k, 1} \right\rangle \\
            \left\langle b_{j, 3}, b_{k, 1} \right\rangle
        \end{pmatrix} \right.\\
        &\qquad - i  \frac{2-iB_k}{2L_k} \beta_j \cdot \begin{pmatrix}
             \left\langle  b_{j, 1}, b_{k, 2} \right\rangle  \\
             \left\langle  b_{j, 1}, b_{k, 3} \right\rangle  \\
        \end{pmatrix}\\
        &\qquad \left. + \frac{2+iB_j}{2L_j} \frac{2-iB_k}{2L_k}  \left(\left\langle b_{j, 2}, b_{k, 2} \right\rangle + \left\langle b_{(j, 3}, b_{k, 3} \right\rangle \right)\right\},
\end{align*}
%
and
\begin{align*}
    \langle |x|^2 u_j, u_k \rangle
    &= \frac{A_j A_k}{L_j L_k} \left\langle \left( L_j^2 |y_j|^2 + 2L_j y_j \cdot X_j + |X_j|^2 \right) b_{j, 1}, b_{k, 1} \right\rangle \\
    &= \frac{A_j A_k}{L_j L_k} \left\{ L_j^2 \langle b_{j, 4}, b_{k, 1} \rangle + 2L_j X_j \cdot \begin{pmatrix}
            \langle b_{j, 2}, b_{k, 1} \rangle \\
            \langle b_{j, 3}, b_{k, 1} \rangle
        \end{pmatrix} + |X_j|^2 \langle b_{j, 1}, b_{k, 1} \rangle \right\}.
\end{align*}
%

We also have
\begin{equation*}
    E_{0, 1} = \langle u|u|^2, u \rangle = \sum_{j=1}^N \frac{A_j}{L_j} \langle u|u|^2, b_{j, 1} \rangle
.\end{equation*}
%

We now proceed to computing the momentum, given by 
\begin{equation*}
    M_{\mu, \lambda} = \left( E_{\mu,\lambda} - \mu \|xu\|^2_{ \mathbb{L}^2 } \right)^2 + \mu^2 \left( \Im \int x\cdot \nabla u \bar{u} \right)^2
.\end{equation*}
%
We know how to compute \( E_{\mu, \lambda} \) from previously, as well as \( \|xu\|_{\mathbb{L}^2}^2 = \langle |x|^2 u, u \rangle \). It only remains to compute
\begin{align*}
    \int x\cdot \nabla u \bar{u}
    &= \sum_{j,k=1}^N \frac{A_j A_k}{L_j L_k} \left\langle (L_j y_j + X_j) \cdot \left( i \beta_j - \frac{2+iB_j}{2L_j} y_j \right) b_{j, 1}, b_{k, 1} \right\rangle \\
    &= \sum_{j,k=1}^N \frac{A_j A_k}{L_j L_k} \left\{ iL_j \beta_j \cdot \begin{pmatrix}
        \left\langle b_{j, 2}, b_{k, 1} \right\rangle \\
        \left\langle b_{j, 3}, b_{k, 1} \right\rangle
    \end{pmatrix}  - \frac{2+iB_j}{2} \left\langle b_{j, 4}, b_{k, 1} \right\rangle \right. \\
        &\qquad \left. + i \beta_j \cdot X_j \left\langle b_{j, 1}, b_{k, 1} \right\rangle - \frac{2+iB_j}{2L_j} X_j \cdot \begin{pmatrix}
            \left\langle b_{j, 2}, b_{k, 1} \right\rangle \\
            \left\langle b_{j, 3}, b_{k, 1} \right\rangle
        \end{pmatrix} \right\}
.\end{align*}
%


Note that all the inner products involved have already been computed when creating the DFMP matrix.

\end{appendices}


\bibliography{PhD}



\end{document}