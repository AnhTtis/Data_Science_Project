\usepackage[utf8]{inputenc}
\usepackage[T1]{fontenc}
\usepackage{listings}
\usepackage{mathtools}
\usepackage{algorithm}
\usepackage{algpseudocode}
\usepackage{amsthm}
\usepackage{amssymb}
\usepackage{amsfonts}
\usepackage[title]{appendix}%
\usepackage{authblk} % autheurs et affiliations
\usepackage{array}
\usepackage[english]{babel}
\usepackage{bbm}  % pour faire l'indicatrice avec \mathbbm{1}
%\usepackage{calrsfs} % jolies lettres calligraphiees
\usepackage{empheq}
\usepackage[shortlabels]{enumitem} % \begin{enumerate}[leftmargin=10pt, (a)]
\usepackage{graphicx}
\usepackage[hypertexnames=false]{hyperref}	%pour avoir des hyperliens dans la table des matieres
\usepackage{cleveref} 
% \usepackage{autonum} % number only referenced equations
\usepackage[top=3cm, bottom=3cm, left=2.5cm, right=2.5cm]{geometry}
\usepackage{fancyhdr}%utile pour changer header & footer
\usepackage{lastpage}%pour avoir le num de la derniere page (utile pour 
%numerotation)
\usepackage{lmodern}
\usepackage{tikz,tkz-tab}
\usetikzlibrary{babel}
% \usepackage{thmbox}
\usepackage{subfiles}
\usepackage{stmaryrd}
\usepackage{subcaption} 
\usepackage{tabularx}
% \usepackage{subeqn}
\usepackage{pgf}
\usepackage{pgfplots} 
\usepackage{wrapfig}

\DeclareUnicodeCharacter{2212}{-}
\usepgfplotslibrary{groupplots,dateplot,fillbetween}
\usetikzlibrary{patterns,shapes.arrows,intersections,backgrounds,external}
\newcommand{\customTikzInputFolder}{tikz/input}
\newcommand{\customTikzOutputFolder}{tikz/output}
\tikzexternalize[prefix=\customTikzOutputFolder]%,figure name=_plot_subsect_\thesubsubsection_nb]
\pgfplotsset{compat=newest}

%%%%% Essai
% Define the command. Note that input folder is NOT STATIC ANYMORE!
\newcommand{\inputtikz}[2]{% 
	% First argument is the width as a proportion of \textwidth
	% Second argument is the filename
    \tikzsetnextfilename{#2}%
    \ifcustomcompileusingtikzexternalize
	\resizebox{#1\textwidth}{!}{\input{\customTikzInputFolder#2.tex}}%
	\else
	\includegraphics[width=#1\textwidth]{\customTikzOutputFolder#2.pdf}
	\fi
}
%%%%%



\usepackage{float} % Mettre "H" dans param d'une figure pour empecher le texte
%de se mettre avant la figure

%\usepackage{nameref} % Pour labelliser des sections/chapitres non numérotés.
%Fonctionne en appelant \nameref{} au lieu de \ref

%\usepackage{parskip} % Ne pas mettre des indentations partout

% %%%%%%%%%%%%%%%%%%%%%%%%%%%%%%%%%%%%%%%%%%%%%%%%%%%%%%%%%%%%%%%%%%%%%%%%%%%%%%
% % use ``\ghostref{myeq}'' to reference equation labelled ``myeq'' 
% % but do nothing with it (i.e. don't show number)
% \newcommand\ghostref[1]{%
% % Do nothing!
% }
% \makeatletter
% \autonum@generatePatchedReference{ghostref}
% \makeatother


\bibliographystyle{alpha}
% \bibliographystyle{plain}

\definecolor{dkgreen}{rgb}{0,0.5,0}
\definecolor{gray}{rgb}{0.5,0.5,0.5}
\definecolor{mauve}{rgb}{0.58,0,0.82}


% Police alternative
%\usepackage{concmath}
% \usepackage{mlmodern}
%\usepackage{fouriernc}


% Modifie ce qui est affiché comme nom de section
% Ne marche pas correctement avec \section*
%\usepackage[explicit]{titlesec}
%\titleformat{\section}
%{\Large\bfseries}{}
%{0pt}{#1\quad\thesection}


\setenumerate[1]{label=\thesection.\arabic*.} % Pour avoir les numeros des 
%questions qui incluent le numero de section.


% Pour avoir les chiffres romains écrits en lettres majuscules
%\newcommand{\RomanNumeralCaps}[1]
%    {\MakeUppercase{\romannumeral #1}}



\hypersetup{
colorlinks=true, % false: boxed links; true: colored links
linkcolor=red	%pas de coloration particulière pour les liens internes
}


\fancypagestyle{plain}{
%	\fancyhf{}
	\renewcommand{\headrulewidth}{0pt}
	\fancyhead[]{}	%header à gauche
	\fancyfoot[L]{}	%footer au centre
	\fancyfoot[R]{}
	\fancyfoot[C]{\thepage{} / \pageref{LastPage}} %footer a droite
}
\pagestyle{plain}


%%%%%%%%%%%%%%
% change l'affichage de cleveref
% Second argument is singular, third is plural
\crefname{thm}{theorem}{theorems}
\Crefname{thm}{Theorem}{Theorems}

\crefname{demo}{proof}{proofs}
\Crefname{demo}{Proof}{Proofs}

\crefname{prop}{proposition}{propositions}
\Crefname{prop}{Proposition}{Propositions}

\crefname{lemma}{lemma}{lemmata}
\Crefname{lemma}{Lemma}{Lemmaata}

\crefname{definition}{definition}{definitions}
\Crefname{definition}{Definition}{Definitions}

\crefname{rmk}{remark}{remarks}
\Crefname{rmk}{Remark}{Remarks}
%%%%%%%%%%%%%%

\newcommand\crefpairconjunction{ and } % change le mot utilise entre les 
%references lorsqu'on utilise \cref{qqch1,qqch2}.



%%%%%%%%%%%%%%

\theoremstyle{plain}
\newtheorem{thm}{Theorem}[section]
%
%\newtheorem*{thm*}{Théorème} % reset theorem numbering for each chapter
%\newtheorem{thm}{Théorème}

\theoremstyle{plain}
%\newtheorem*{demo*}{Démonstration}
\newtheorem{demo}{Proof}[section]
%\newtheorem{demo}{Démonstration}

\theoremstyle{plain}
%\newtheorem*{prop*}{Proposition}
\newtheorem{prop}{Proposition}[section]
%\newtheorem{prop}{Proposition}

\theoremstyle{plain}
%\newtheorem*{lemma*}{Lemme}
\newtheorem{lemma}{Lemma}[section]
%\newtheorem{lemma}{Lemme}

\theoremstyle{definition}
%\newtheorem*{definition*}{Définition}
\newtheorem{definition}{Definition}[section]
%\newtheorem{definition}{Définition}

\theoremstyle{remark}
%\newtheorem*{rmk*}{Remarque}
\newtheorem{remark}{Remark}[section]
%\newtheorem{rmk}{Remarque}
%%%%%%%%%%%%%%


\newcommand{\numberthis}{\stepcounter{equation}\tag{\theequation}} %utiliser 
%'\numberthis' pour numéroter une équation dans un environnement align* par ex


\renewcommand*{\proofname}{Proof} % change le nom "proof" en 
%l'argument, lorsqu'on utilise l'environnement 'proof'


%%%%%%%%%%%%%%
%% Julia language is not supported :(
% \lstset{
% language=Julia,
% basicstyle=\footnotesize\ttfamily,
% numberstyle=\normalsize,
% numbersep=7pt,
% frame = L,
% breaklines = true,
% extendedchars=true,
% keywordstyle=\color{blue},      % keyword style
%    commentstyle=\color{dkgreen},   % comment style
%    stringstyle=\color{black},      % string literal style
% literate={á}{{\'a}}1 {ã}{{\~a}}1 {é}{{\'e}}1 {è}{{\`e}}1 
% {à}{{\`a}}1{ç}{{\c{c}}}1 {ê}{{\^e}}1,%cette ligne permet de bien afficher tous 
% %les accents dans les codes
% }
%%%%%%%%%%%%%%

\makeatletter 
\@ifclassloaded{report}{%
\renewcommand{\thesection}{\thechapter.\arabic{section}} %pour faire commencer les 
%sections à 1 au lieu de 0, et pour avoir le numéro du chapitre dans le numéro de section. 
% /!\ Utile seulement dans un report ou book, et pas dans un article car alors ``\thechapter'' n'existe pas !!
}{}
\makeatother

\renewcommand\theequation{\thesection.\arabic{equation}}



\numberwithin{equation}{section} % Pour faire correspondre la numérotation des 
%equations aux sections

%%%%%%%%%%%%%%%%%%%%%%%%%%%%%%%%%%%%%%%%%%%%%%%%%%%%

\newcommand{\BB}[1]{\mathbbm{#1}} % Lettres en mathbb
\newcommand{\CAL}[1]{\mathcal{#1}} % Lettres en mathcal

\newcommand{\cv}[2]{\overset{#1}{\underset{#2}{\longrightarrow}}} % ecrire 
%rapidement une convergence. Premier argument, au-dessus de la fleche de 
%limite, deuxieme argument en dessous

\newcommand{\sign}{\ \textnormal{sign}}
\newcommand{\textd}{\textnormal{d}}

\newcommand{\QED}{\hfill $\qed$}
\newcommand{\supp}{\textnormal{supp }}
\DeclareMathOperator{\arctanTwo}{arctan2}


\newcommand{\TODO}[1]{\begin{center} \textcolor{red}{\textbf{{\Huge #1}}} \end{center}}


\allowdisplaybreaks % Pour autoriser des pages breaks dans des align


%%%%%%%%%%%%%%%%%%%%%%%%%%%%%%%%%%%%%%%%%%%%%%%%%%%%%%%%%%%%%%%%%%%%%
%% D'après le template Springer  %%%%%%%%%%%%%%%%%%%%%%%%%%%%%%%%%%%%

%%%%%%%%%%%%%%%%%%%%%%%%%%%%%%%%%%%%%%%%%%%%%%%%%%%%%%%%%%%%%%%%%%%%%
%%%%%%%%%%%%%%%%%%%%%%%%%%%%%%%%%%%%%%%%%%%%%%%%%%%%%%%%%%%%%%%%%%%%%