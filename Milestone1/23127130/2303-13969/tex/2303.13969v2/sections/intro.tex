

\section{Introduction}

We are interested in approximating numerically the solution $\psi(t,x)$ to the cubic nonlinear Schr{\"o}dinger equation with harmonic potential,  
\begin{equation}
    i \partial_{t} \psi + \Delta_x  \psi - | x |^2 \psi = \psi |\psi|^2, \quad x\in \mathbb{R}^d
    \tag{cNLS}
    \label{eqn: cNLS with psi}
,\end{equation}
%
where \( d \geq 1 \), \( | \cdot | \) denotes the usual Euclidian norm over \( \mathbb{R}^d \), and \( \Delta_x \) denotes the Laplace operator over \( \mathbb{R}^d \): \( \Delta_x = \sum_{i=1}^d \partial^2_{x_i} \).
This equation is also sometimes called \emph{time-dependent Gross-Pitaevskii equation} \cite{cerimeleNumericalSolutionGrossPitaevskii2000,baoNumericalSolutionGross2003,wangNumericalSimulationGrossPitaevskii2017,tsedneeNumericalSolutionTimedependent2022}.
We focus on a cubic nonlinearity for the sake of clarity, but we emphasize the fact that everything we present is also applicable to other types of polynomial  nonlinearities, \emph{mutatis mudandis}.
The Schr{\"o}dinger equation \eqref{eqn: cNLS with psi} can be split into two parts, namely the linear and nonlinear parts:
\begin{equation}
    \label{eqn: cNLS with psi -- linear part}
    i \partial_{t} \psi + \Delta_x \psi - |x|^2 \psi = 0
    \tag{HO}
,\end{equation}
%
and
\begin{equation}
    \label{eqn: cNLS with psi -- nonlinear part}
    i \partial_{t} \psi = \psi |\psi|^2
    \tag{cNLS-nonLin.}
.\end{equation}
%
The linear equation \eqref{eqn: cNLS with psi -- linear part} is also called the Harmonic Oscillator. Traditional well known numerical schemes are based on this abstract decomposition and it is easy to determine high order splitting methods obtained by solving alternately the linear and nonlinear part, like Lie, Strang Splitting or triple jump composition, see for instance \cite{mclachlanSplittingMethods2002,hairerGeometricNumericalIntegration2006,casasHighorderHamiltonianSplitting2017}. However, the discretization of each of these two parts remains to be done using time and space discretization. 

Very recent works \cite{martelStronglyInteractingBlowup2018,faouWeaklyTurbulentSolutions2020} suggest to discretize the solution \( \psi \) of \eqref{eqn: cNLS with psi} as a sum of \( N \) modulated functions, which write as:
\begin{equation}
    \label{eqn: generic discretization of psi -- ansatz}
    \psi(t,x) \approx u(t,x) := \sum_{j=1}^N u_j(t,x)
,\end{equation}
%
where
\begin{equation}
    \label{eqn: generic discretization of psi -- expression for uj}
    u_j(t,x) := \frac{A_j}{L_j} e^{i\gamma_j + i L_j \beta_j \cdot y_j - i \frac{B_j}{4} |y_j|^2} v_j(s_j, y_j), \quad \text{with } \quad
    \left| \begin{aligned}
         & \frac{\textnormal{d} s_j}{\textnormal{d} t} := \frac{1}{L_j^2}, \\
         & y_j := \frac{x - X_j}{L_j},
    \end{aligned} \right.
\end{equation}
%
and \( N \in \mathbb{N}^* \). 
We shall call each modulated function \( u_j \) a \emph{bubble}.



The time dependence of the parameters \( A_j, L_j, B_j, X_j, \beta_j, \gamma_j \) has not been written in \eqref{eqn: generic discretization of psi -- expression for uj} for the sake of clarity, but it is one of the main ingredients of the approach.
More precisely, the core idea is to plug the ansatz \eqref{eqn: generic discretization of psi -- ansatz} into \eqref{eqn: cNLS with psi} in order to obtain ODEs for the parameters.
The idea of relying on time-dependent parameters to represent the solution, or an approximation, is not new and was formerly called  \emph{Variational Gaussian wave packets} by Lasser and Lubich \cite{lasserComputingQuantumDynamics2020}, where they applied the Dirac-Frenkel-MacLachlan principe (DFMP) to \eqref{eqn: cNLS with psi -- linear part}.
The Dirac-Frenkel-MacLachlan is the name given to the Dirac-Frenkel principle, or the MacLachlan principle, because in this case they are equivalent.

\vspace{1em}

This method is widely used in the field of Chemical Physics 
\cite{hellerTimeDependentVariational1976,huberGeneralizedGaussianWave1987,coalsonMultidimensionalVariationalGaussian1990,worthNovelAlgorithmNonadiabatic2004,adamowiczLaserinducedDynamicAlignment2022}.
The different methods used are variations of the same idea, and possess many names: superposition of Gaussian Wavepackets, Gaussian beams, Thawed Gaussians, Frozen Gaussian... 
All of these algorithms simply consist in applying a Dirac-Frenkel-MacLachlan principle to \eqref{eqn: cNLS with psi -- linear part}, the difference lying in the the way the parameters are updated 
and how.
For example, the Thawed Gaussian method allows the width matrix to be time-dependent while the Frozen Gaussian does not.
However, they all produce an approximate solution, even in the linear case.
In general, the chemists' setting of Gaussian beams involve correlated components, hence the width matrix. 
In the present paper we make the simplifying assumption that our gaussians are not correlated, and that their width is parametrized by a real positive scalar.

\vspace{1em}


Let us explain the main ideas underlying the full modulation \eqref{eqn: generic discretization of psi -- expression for uj}, and developed in various works, see for instance \cite{martelStronglyInteractingBlowup2018,faouWeaklyTurbulentSolutions2020} and the references therein. 

Consider for instance the case of one bubble, {\em i.e.} $N = 1$. When plugging the Ansatz \eqref{eqn: generic discretization of psi -- expression for uj} into \eqref{eqn: cNLS with psi}, we obtain an equation of the form 
\[
i \partial_{s} v + \Delta_y  v - | y |^2 v -  |v|^2 v + P(s; y,\partial_y) v = 0  
\]
where $P(s; y,\partial_y)$ is a quadratic operator in $y$ and $\partial_y$ depending on the time through the parameters $(A,L,B,X,\beta,\gamma)$ and their time derivatives with respect to $s$, 
see \eqref{eqn: HO -- idt u - Hu -- with v} for more precise detail. It is then possible to choose the parameters in such a way that $P(s; y,\partial_y) v = -\lambda v$ for some $\lambda \in \mathbb{R}$, and to take $v$ as a soliton solution of the stationary equation 
\begin{equation}
    \label{soliton}
    - \Delta_y  v + | y |^2 v + |v|^2 v = \lambda v. 
\end{equation}
This yields to a differential system to be solved by the parameters $(A,L,B,X,\beta,\gamma)$ which is given below by the equations \eqref{eqn: modulation ODEs -- linear part wrt time s - with v}. It turns out that these equations form a {\em completely integrable Poisson system}  that can be solved, and the solution for a single bubble can be thus taken as a {\em modulated soliton}. 


This kind of approach has been used successfully in various situations from a theoretical point of view, see  \cite{merleBlowupDynamicUpper2005,martelStronglyInteractingBlowup2018,faouWeaklyTurbulentSolutions2020,merleIMPLOSIONTHREEDIMENSIONAL2020} and the references therein. Typically, when several modulated solitons $N \geq 2$ are in interactions, this can produce finite time blow-up of growth of Sobolev norm phenomena, and a large part of the analysis relies on the ability of calculating nonlinear interactions between two modulated solitons. This can be done for instance in an integrable situation, e.g. the Szeg\"o equation \cite{gerardTwoSolitonTransientTurbulent2018}.  

Another byproduct of these modulation technics in 2D is to make a link between \eqref{eqn: cNLS with psi} on a finite time interval and the Schr\"odinger equation without harmonic potential 
\[
    i \partial_{s} \psi + \Delta_x  \psi  = \psi |\psi|^2, \quad x\in \mathbb{R}^d
\]
on an unbounded time interval. 
In this case, the modulation equations generate the so called {\em lens} transform, see for instance \cite{carlesSemiclassicalAnalysisNonlinear2021} (note that our algorithms could be also be applied to the latter equation but we will restrict our analysis to the Harmonic case). Let us also note that such modulation technics can also be related with the families of exact splitting introduced in \cite{bernierExactSplittingMethods2020} where the time coefficients can be seen as specific time changes $s$ in the modulation equations. 


Inspired by these sucessful theoretical works, we retain the idea of 
 approximate solutions to the equations \eqref{eqn: cNLS with psi -- linear part} and \eqref{eqn: cNLS with psi} by modulating the parameters
\( A_j, L_j, B_j, X_j, \beta_j, \gamma_j \) in such a way that $v_j(s_j,x)$ satisfies a {\em smoother in time} equation - typically a stationary soliton equation. 
From the numerical point of view, however, chosing the $v_j$ as stationary solitons would require first to  solve explicitly the nonlinear equation \eqref{soliton} and more problematically, to estimate numerically the nonlinear interactions between the modulated solitons by using the Dirac-Frenkel-MacLachlan principle. The latter consists essentially in a projection on the manifold of modulated solitons, which is in practice very difficult to evaluate numerically. 

Moreover, one is naturally interested in using a splitting strategy between the linear and nonlinear part which would typically destroy the soliton structure in the equation. 


This is why we decide here to use Gaussians for the functions $v_j$. Let us remark that in the case of small nonlinearities, ground states solitons are close to gaussians which makes the approximation relevant. 

This Gaussian choice has the main following advantages : 
First we can solve exactly the linear part by using the classical modulation equations. Second, we can calculate explicitly the nonlinear interactions obtained by applying the Dirac-Frenkel-MacLachlan principle on the manifold of modulated gaussians to polynomial interactions. In the end, we thus obtain a algorithm for modulated Gaussians that can be easily implemented numerically, grid free, and able to capture high oscillations of the solution.

The paper is organized as follows. 
Section \ref{sect: The Harmonic Oscillator} is devoted to \eqref{eqn: cNLS with psi -- linear part} and can be seen as a specialization of \cite{lasserComputingQuantumDynamics2020} for a quadratic potential.
We recall some conservation laws and growth estimates when considering \emph{bubbles} in \eqref{eqn: cNLS with psi}, as well as exhibit \emph{modulation equations} which turn to have a completely integrable Hamiltonian structure. 
Those equations allow us to obtain an analytical formula for most of the modulation parameters.
By choosing a shape for those bubbles -- in our case gaussian -- all of the modulation parameters can be obtained exactly.

The second part of this work takes into account cubic interactions, which were not accounted for in \cite{lasserComputingQuantumDynamics2020}. We focus in Section \ref{sect: DFMP} on \eqref{eqn: cNLS with psi -- nonlinear part}.
The nonlinearity that is introduced is the core of difficulties arising in the Schr{\"o}dinger equation, and it is hopeless to look for exact solutions in the general case.
The proposed approach uses the Dirac-Frenkel-MacLachlan principle to obtain modulation equations.
This allows us to take into account as many bubbles as one desires at the cost of a computational complexity of order \( \mathcal{O}(N^4 d + d^3 N^3) \). Here, \( N \) is the total number of bubbles. The fourth power of \( N \) is due to polynomial interactions of order three.
This complexity almost does not suffer from the well-known ``curse of dimensionality'' since it is at most polynomial with respect to the dimension \( d \). This relies on the exact computations of all of the integrals involved, hence we do not need to use numerical integration.

Finally, Section \ref{sect: numerical experiments} is dedicated to illustrating numerically the fine details that are obtainable with the Dirac-Frenkel-MacLachlan principle (abbreviated DFMP), as well as the long-time behavior, compared to a FFT-based spectral scheme.
Our experiments show that, if the initial data is discretized ``nicely'', the DFMP yields satisfying results.
