

\section{The Harmonic Oscillator}
\label{sect: The Harmonic Oscillator}


In this section we focus onto the linear part of the cubic Non Linear Schr\"odinger equation, namely the Harmonic Oscillator \eqref{eqn: cNLS with psi -- linear part}.

\subsection{General case: \( v = v(s,y) \)}

\subsubsection{Conservation Laws}
\label{sect: HO -- Conservation Laws}

We recall classical laws for the Harmonic oscillator equation (see for instance
\cite{killipNonlinearSchrOdinger2013,taoNonlinearDispersiveEquations2006}).  
Let \( \psi \) be the solution to \eqref{eqn: cNLS with psi -- linear part}.
We will need the following result, known as the Pohozaev identity. 

\begin{lemma}[Pohozaev Identity]
    \label{lemma: pohozaev identity}
    Let \( x\in \mathbb{R}^d \), and \( f\in H^1( \mathbb{R}^d ) \) such that \( xf\in \mathbb{L}^2( \mathbb{R}^d ) \). Then    
    \begin{equation}
        \label{eqn: pohozaev identity}
        \int \Delta f \overline{\left( \frac{d}{2} f + x\cdot \nabla f \right)} dx = - \int |\nabla f|^2 dx
    .\end{equation}
\end{lemma}


\begin{proof}
    By density, we only need to prove equation \eqref{eqn: pohozaev identity} for \( f \in \mathcal{C}^\infty_c( \mathbb{R}^d ) \), where \( \mathcal{C}^\infty_c ( \mathbb{R}^d ) \) denotes the space of infinitely smooth functions with compact support in \( \mathbb{R}^d \). Let 
    \begin{equation*}
        f_\lambda(x) := \lambda^{\frac{d}{2} } f(\lambda x)
    ,\end{equation*}
    %
    then
    \begin{equation*}
        \int |\nabla f_\lambda |^2 dx = \lambda^2 \int |\nabla f|^2 dx
    .\end{equation*}
    %
    Differentiating this identity with respect to \( \lambda \) and evaluating the result at \( \lambda = 1 \) yields 
    \begin{equation*}
        \int \nabla f \cdot \overline{ \nabla \left( \frac{d}{2} f + x \cdot \nabla f \right) } dx = \int |\nabla f|^2 dx
    .\end{equation*}
    %
    We integrate by parts the LHS, and obtain \eqref{eqn: pohozaev identity}.
\end{proof}



\begin{lemma}[Conserved quantities in dimension \( d=2 \)]
    \label{lemma: conserved quantities in HO}
    We consider a two-parameter family of equations containing \eqref{eqn: cNLS with psi -- linear part} and \eqref{eqn: cNLS with psi}:
    \begin{equation*}
        i \partial_{t} \psi + \mu(\Delta \psi - |x|^2 \psi) = \lambda |\psi|^2 \psi, \quad \mu, \lambda \in \mathbb{R}
    .\end{equation*}
    %
    The (radial) conservation laws are mass \( \|\psi\|_{ \mathbb{L}^2 } \), energy
    \begin{equation*}
        E_{\mu,\lambda} = \frac{\mu}{2} \left\langle H\psi, \psi \right\rangle + \frac{\lambda}{4} \left\langle |\psi|^2 \psi, \psi \right\rangle
    ,\end{equation*}
    %
    where \( H = -\Delta + |x|^2 \) and \( \langle f,g \rangle := \int_{\mathbb{R}^d} f\bar{g} \),
    and momentum
    \begin{equation*}
        M_{\mu, \lambda} = \left( E_{\mu,\lambda} - \mu \|x\psi\|^2_{ \mathbb{L}^2 } \right)^2 + \mu^2 \left( \Im \int x\cdot \nabla \psi \bar{\psi} \right)^2
    ,\end{equation*}
    %
    and the same applied to any power \( (-H)^s \psi \). There also holds the non radial conservation law 
    \begin{equation*}
        \mathcal{P}_j = \frac{1}{4} \left( \Im \int \partial_{j} \psi \bar{\psi} \right)^2 + \mu^2 \left( \int x_j |\psi|^2 \right)^2, \quad j=1, 2
    .\end{equation*}
    %
\end{lemma}

\begin{proof}
    Mass conservation:
    \begin{align*}
        \partial_{t} \|\psi\|^2_{\mathbb{L}^2}
        &= \partial_{t} \int |\psi|^2 = 2 \Re \int \bar{\psi} \partial_{t} \psi = 2 \Re \int -i \bar{\psi} \left( -\mu\Delta \psi + \mu|x|^2 \psi + \lambda |\psi|^2 \psi \right) \\
        &= 2 \Re \int -i \left( \mu\bar{\psi} \Delta \psi + \mu|x|^2 |\psi|^2 + \lambda |\psi|^4  \right) = 2\mu\Re \int -i \bar{\psi} \Delta \psi\\
        &= 2 \mu \Re \int i |\nabla \psi|^2 = 0 
    .\end{align*}
    %
    Energy conservation:
    \begin{align*}
        \frac{\textd}{\textd t} E_{\mu,\lambda}
        &= \frac{1}{2} \frac{\textd}{\textd t} \left\langle -\mu\Delta \psi + \mu|x|^2 \psi + \frac{\lambda}{2} |\psi|^2 \psi, \psi \right\rangle \\
        % &= \frac{1}{2} \frac{\textd}{\textd t} \left( \langle \nabla \psi, \nabla \psi \rangle + \langle |x|^2 \psi, \psi \rangle + \langle \lambda |\psi|^2, |\psi|^2 \rangle \right) \\
        &= \frac{1}{2} \left( -2\mu\Re\langle \Delta \psi, \partial_{t} \psi \rangle + 2\mu\Re \langle |x|^2 \psi, \partial_{t} \psi \rangle + 4 \Re\left\langle \frac{\lambda}{2} |\psi|^2 \psi, \partial_{t} \psi \right\rangle \right) \\
        &= \Re \left(i \langle \partial_{t} \psi, \partial_{t} \psi \rangle\right) = 0
    .\end{align*}
    %
    For the momentum, we compute
    \begin{align}
        \frac{1}{2} \frac{\textd}{\textd t} \int |x|^2 |\psi|^2
        &= \frac{1}{2} \frac{\textd}{\textd t} \langle |x|^2 \psi, \psi \rangle = \Re \langle |x|^2 \psi, \partial_{t} \psi \rangle = \Re \langle |x|^2 \psi, i \mu\Delta \psi - i \mu|x|^2 \psi - i\lambda|\psi|^2 \psi \rangle \nonumber \\
        &= \mu\Im \langle |x|^2 \psi, \Delta \psi \rangle = \mu\Im \int |x|^2 \psi \Delta \bar{\psi} = -\mu\Im \int \nabla \bar{\psi} \cdot \nabla \left( |x|^2 \psi \right) \nonumber \\
        &= -\mu\Im \int \nabla \bar{\psi} \cdot 2x \psi - \mu\Im \int \nabla \bar{\psi} \cdot \nabla \psi |x|^2 = -2\mu\Im \int x\cdot \nabla \bar{\psi} \psi \nonumber \\
        &= 2\mu\Im \int x\cdot \nabla \psi \bar{\psi} \label{eqn: lemma conservation -- intermediate equation no 1}
    ,\end{align}
    %
    and
    \begin{equation*}
        \frac{1}{2} \frac{\textd}{\textd t} \Im \int x\cdot \nabla \psi \bar{\psi}
        = \frac{1}{2} \Im \int \left( x\cdot \nabla \partial_{t} \psi \bar{\psi} + x\cdot \nabla \psi \partial_{t} \bar{\psi} \right)
    .\end{equation*}
    %
    An integration by parts gives
    \begin{equation*}
        \int x\cdot \nabla \phi \psi = - \int \phi \nabla \cdot (x \psi) = -\int \phi \left( \psi d + x \cdot \nabla \psi \right)
    ,\end{equation*}
    %
    hence
    \begin{align*}
        \frac{1}{2} \frac{\textd}{\textd t} \Im \int x\cdot \nabla \psi \bar{\psi}
        &= \frac{1}{2} \Im \int \left( - \partial_{t} \psi \left( \bar{\psi} d + x \cdot \nabla \bar{\psi} \right) + x\cdot \nabla \psi \partial_{t} \bar{\psi} \right) \\
        &= \frac{1}{2} \Im \int \left( - \partial_{t} \psi \bar{\psi} d - \partial_{t} \psi x\cdot \nabla \bar{\psi} + \partial_{t} \bar{\psi} x\cdot \nabla \psi \right) \\
        &= \frac{1}{2} \Im \int \left( - \partial_{t} \psi \bar{\psi} d + 2 i \Im \left[ \partial_{t} \bar{\psi} x\cdot \nabla \psi \right] \right) \\
        &= -\frac{d}{2} \Im \int \partial_{t} \psi \bar{\psi} + \Im \int \partial_{t} \bar{\psi} x\cdot \nabla \psi
    .\end{align*}
    %
    Recall the equation satisfied by \( \psi \):
    \begin{equation*}
        \partial_{t} \psi = i\mu \Delta \psi - i\mu |x|^2 \psi - i \lambda |\psi|^2 \psi
    ,\end{equation*}
    %
    therefore
    \begin{align*}
        \frac{1}{2} \frac{\textd}{\textd t} \Im \int x\cdot \nabla \psi \bar{\psi}
        &= -\frac{d}{2} \Im \int i\left[ \mu\Delta \psi - \mu|x|^2 \psi - \lambda |\psi|^2 \psi \right] \bar{\psi} \\
            &\qquad + \Im \int i\left[ - \mu\Delta \bar{\psi} + \mu|x|^2 \bar{\psi} + \lambda |\psi|^2 \bar{\psi} \right] x\cdot \nabla \psi
    .\end{align*}
    %
    We have
    \begin{equation*}
        -\frac{d}{2} \Im \int i\left[ \mu\Delta \psi - \mu|x|^2 \psi - \lambda |\psi|^2 \psi \right] \bar{\psi}
        = \frac{d}{2} \int \left[ \mu|\nabla \psi|^2 + \mu |x|^2 |\psi|^2 + \lambda |\psi|^4 \right]
    ,\end{equation*}
    %
    and
    \begin{equation*}
        \Im \int i\left[ - \mu\Delta \bar{\psi} + \mu|x|^2 \bar{\psi} + \lambda |\psi|^2 \bar{\psi} \right] x\cdot \nabla \psi = \Re \int \left[ - \mu \Delta \bar{\psi} + \mu |x|^2 \bar{\psi} + \lambda |\psi|^2 \bar{\psi} \right] x\cdot \nabla \psi
    .\end{equation*}
    %
    Moreover,
    \begin{align*}
        \int |x|^2 \bar{\psi} x\cdot \nabla \psi
        = -\int \psi \nabla \cdot \left( x|x|^2 \bar{\psi} \right) &= - \int \psi \left( d |x|^2 \bar{\psi} + 2|x|^2 \bar{\psi} + x|x|^2 \cdot \nabla \bar{\psi} \right) \\
        \iff \int |x|^2 \bar{\psi} x\cdot \nabla \psi + \overline{\int |x|^2 \bar{\psi} x\cdot \nabla \psi} &= - \int \psi \left( d |x|^2 \bar{\psi} + 2|x|^2 \bar{\psi} \right)  \\
        \iff \Re \int |x|^2 \bar{\psi} x\cdot \nabla \psi &= - \int \psi \left( \frac{d}{2} |x|^2 \bar{\psi} + |x|^2 \bar{\psi} \right)
    .\end{align*}
    %
    Finally,
    \begin{align*}
        \frac{1}{2} \frac{\textd}{\textd t} \Im \int x\cdot \nabla \psi \bar{\psi}
        &= \frac{d}{2} \int \left[ \mu |\nabla \psi|^2 + \mu |x|^2 |\psi|^2 + \lambda |\psi|^4 \right] \\
            &\qquad + \Re \int \left[ - \mu \Delta \bar{\psi} + \lambda |\psi|^2 \bar{\psi} \right] x\cdot \nabla \psi - \mu \int \psi \left( \frac{d}{2} |x|^2 \bar{\psi} + |x|^2 \bar{\psi} \right) \\
        &= \frac{d}{2} \int \left[ \mu |\nabla \psi|^2 + \lambda |\psi|^4 \right] + \Re \int \left[ -\mu \Delta \bar{\psi} + \lambda |\psi|^2 \bar{\psi} \right] x\cdot \nabla \psi - \mu \int |x|^2 |\psi|^2 \\
        &= \frac{d}{2}\mu  \int |\nabla \psi|^2 - \mu \int |x|^2 |\psi|^2 + \frac{d}{2} \lambda \int |\psi|^4 + \Re \int \left[ - \mu \Delta \bar{\psi} + \lambda |\psi|^2 \bar{\psi} \right] x\cdot \nabla \psi
    \end{align*}
    %
    We are in the two-dimensional case \( d=2 \), hence
    \begin{align*}
        \frac{1}{2} \frac{\textd}{\textd t} \Im \int x\cdot \nabla \psi \bar{\psi}
        &= \int \mu |\nabla \psi|^2 - \mu \int |x|^2 |\psi|^2 + \lambda \int |\psi|^4 + \Re \int \left[ - \mu \Delta \bar{\psi} + \lambda |\psi|^2 \bar{\psi} \right] x\cdot \nabla \psi \\
        &= 2 E_\lambda + \frac{\lambda}{2} \int |\psi|^4  - 2\mu \int |x|^2 |\psi|^2 + \Re \int \left[ - \mu \Delta \bar{\psi} + \lambda |\psi|^2 \bar{\psi} \right] x\cdot \nabla \psi
    .\end{align*}
    %
    Moreover,
    \begin{align*}
        \int |\psi|^2 \bar{\psi} x\cdot \nabla \psi
        &= - \int \psi \nabla \cdot \left( |\psi|^2 \bar{\psi} x \right) \\
        &= - \int \psi \left( 2\Re \left( \bar{\psi} \nabla \psi \right) \cdot \bar{\psi} x + |\psi|^2 \nabla\bar{\psi}\cdot x + d|\psi|^2 \bar{\psi} \right) \\
        &= - \int \left( 2\Re \left( \bar{\psi} \nabla \psi \right) \cdot |\psi|^2 x + \psi |\psi|^2 \nabla\bar{\psi}\cdot x + 2|\psi|^4 \right) \\
        \iff 2 \Re \int |\psi|^2 \bar{\psi} x\cdot \nabla \psi &= - 2\Re \int \bar{\psi} \nabla \psi \cdot |\psi|^2 x - 2 \int |\psi|^4 \\
        \iff \Re \int |\psi|^2 \bar{\psi} x\cdot \nabla \psi &= - \frac{1}{2} \int |\psi|^4
    ,\end{align*}
    %
    Finally,
    \begin{align*}
        \frac{1}{2} \frac{\textd}{\textd t} \Im \int x\cdot \nabla \psi \bar{\psi} 
        &= 2 E_\lambda + \frac{\lambda}{2} \int |\psi|^4  - 2\mu \int |x|^2 |\psi|^2 - \mu \Re \int \Delta \bar{\psi} x\cdot \nabla \psi - \frac{\lambda}{2} \int |\psi|^4 \\
        &= 2 E_\lambda - 2\mu \int |x|^2 |\psi|^2 - \mu \Re \int \Delta \bar{\psi} x\cdot \nabla \psi
    \end{align*}
    %
    We then use the Pohozaev identity \eqref{eqn: pohozaev identity} in dimension \( d=2 \), which yields
    \begin{equation*}
        \Re\left( \int x\cdot \nabla \psi \Delta \bar{\psi} \right) = 0
    .\end{equation*}
    %
    Therefore,
    \begin{equation*}
        \frac{1}{2} \frac{\textd}{\textd t} \Im \int x\cdot \nabla \psi \bar{\psi}  = 2 E_{\mu,\lambda} - 2\mu \int |x|^2 |\psi|^2
    .\end{equation*}
    %

    From the conservation of the energy \( E_\lambda \) and equation \eqref{eqn: lemma conservation -- intermediate equation no 1},
    \begin{equation*}
        \frac{\textd^2}{\textd t^2} \Im \int x\cdot \nabla \psi \bar{\psi} = -16\mu^2 \Im \int x\cdot \nabla \psi \bar{\psi}
    .\end{equation*}
    %
    Hence, the conservation laws
    \begin{align*}
        &\frac{1}{16} \left( \frac{\textd}{\textd t} \left[ \Im \int x\cdot \nabla \psi \bar{\psi} \right] \right)^2 + \mu^2\left( \Im \int x\cdot \nabla \psi \bar{\psi}  \right)^2 \\
        &= \left( E_{\mu,\lambda} - \mu \|x\psi\|^2_{ \mathbb{L}^2 } \right)^2 + \mu^2 \left( \Im \int x\cdot \nabla \psi \bar{\psi} \right)^2
    \end{align*}
    %
    
    For the non radial conservation law:    
    \begin{equation*}
        \frac{\textd}{\textd t} \Im \int \partial_{j} \psi \bar{\psi} 
        = -2\Im \int \partial_{t} \psi \overline{ \partial_{j} \psi }
        = 2\Re \int i \partial_{t} \psi \overline{ \partial_{j} \psi } 
        = 2 \mu \int |x|^2 \Re \left(\psi \overline{ \partial_{j} \psi }\right)
        = -2\mu \int x_j |\psi|^2
    ,\end{equation*}
    %
    owing to the facts that integrations by parts yield
    \begin{equation*}
        -\Re \int \Delta \psi \partial_{j} \bar{\psi} = \Re \int \partial_{j} \psi \Delta \bar{\psi}
    ,\end{equation*}
    %
    and 
    \begin{equation*}
        2 \Re \int |\psi|^2 \psi \partial_{j} \bar{\psi}
        = \int |\psi|^2 \partial_{j} |\psi|^2 = - \int |\psi|^2 \partial_{j} |\psi|^2  \implies \Re \int |\psi|^2 \psi \partial_{j} \bar{\psi} = 0
    .\end{equation*}
    %
    We also have
    \begin{equation*}
        \frac{1}{2} \frac{\textd}{\textd t} \int x_j |\psi|^2
        = \Re \int x_j \partial_{t} \psi \bar{\psi} 
        = \Im \int x_j i \partial_{t} \psi \bar{\psi} = \mu \Im \int -\Delta \psi x_j \bar{\psi}
        = \mu \Im \int \partial_{j} \psi \bar{\psi}
    .\end{equation*}
    %
    Hence the relations
    \begin{equation*}
        \left|
        \begin{aligned}
            \frac{\textd}{\textd t} \int x_j |\psi|^2 &= 2\mu \Im \int \partial_{j} \psi \bar{\psi} \\
            \frac{\textd}{\textd t} \Im \int \partial_{j} \psi \bar{\psi} &= -2\mu \int x_j |\psi|^2,
        \end{aligned}
        \right.
    \end{equation*}
    %
    which have the conservation law
    \begin{equation*}
        \mathcal{P}_j = \frac{1}{4} \left( \Im \int \partial_{j} \psi \bar{\psi} \right)^2 + \mu^2 \left( \int x_j |\psi|^2 \right)^2
    .\end{equation*}
    %
\end{proof}



% \subsubsection{Galilean symmetry}
% \label{sect: HO -- Galilean symmetry}


% \begin{lemma}[Galilean symmetry]
%     If \( v \) solves \eqref{eqn: cNLS with psi -- linear part}, then so does 
%     \begin{equation*}
%         u(t,x) = v(t,y) e^{i\beta\cdot y + i\gamma(t)}, \quad y = x-x(t)
%     ,\end{equation*}
%     %
%     for
%     \begin{equation*}
%         \dot{x} = 2\beta,\quad \dot{\beta} = -2x,\quad \dot{\gamma} = |\beta|^2 - |x|^2
%     .\end{equation*}
%     %
% \end{lemma}

% \begin{remark}
%     Please note we consider now that \( u \) is solution to \eqref{eqn: cNLS with psi -- linear part} because we are looking at a solution under the form of \emph{bubbles}, whereas \( \psi \) has no particular form.
% \end{remark}

% \begin{proof}
%     We change variables:
%     \begin{equation*}
%         u(t,x) = v(t,y) e^{i\gamma}, \quad y = x-x(t)
%     ,\end{equation*}
%     %
%     then 
%     \begin{equation*}
%         i\partial_t v - \dot{\gamma}v - i\dot{x} \cdot \nabla v + \Delta v - \left| y - x(t) \right|^2 v + T(v,v,v) = 0
%     .\end{equation*}
%     %
%     Let
%     \begin{equation*}
%         \dot{x} = 2\beta,\quad v = e^{i\beta\cdot y} w
%     ,\end{equation*}
%     %
%     then
%     \begin{equation*}
%         i \partial_{t} w + \Delta w - |y|^2 w - \left[ \dot{\beta}+ 2x(t) \right]\cdot yw + i(2\beta - \dot{x}) \cdot \nabla w + \left( -\dot{\gamma} + x_t \cdot \beta  - \left( |\beta|^2 + |x|^2 \right)\right) w = 0
%     ,\end{equation*}
%     %
%     and hence the symmetry.
% \end{proof}



\subsubsection{Renormalized flow}
\label{sect: HO -- Renormalized flow}

Recall the expression \eqref{eqn: generic discretization of psi -- expression for uj} of a bubble:
\begin{equation}
    \label{eqn: HO -- Renormalized flow}
    u(t,x) = \frac{A}{L} e^{i\gamma + iL\beta \cdot y - i\frac{B}{4} |y|^2} v(s,y), \quad y = \frac{x-X(t)}{L(t)}, \, \frac{\textd s}{\textd t} = \frac{1}{L(t)^2} 
.\end{equation}
%

We compute, in dimension \( d \geq 1 \):
\begin{equation*}
    \Delta_x u = \frac{Ae^{i\gamma}}{L^3} \Delta_y \left[ e^{iL\beta\cdot y - i\frac{B}{4} |y|^2} v(s, y) \right]
,\end{equation*}
%
and 
\begin{equation}
    \partial_{k} \left[ e^{ iL\beta\cdot y - i \frac{B}{4} |y|^2 } v \right]
    = e^{iL\beta\cdot y - i \frac{B}{4} |y|^2} \left[ \partial_{k} v + i\left( L\beta_k - \frac{B}{2} y_k \right) v \right],\quad k=1, \dots, d
,\end{equation}
%
and
\begin{align*}
    &\partial_{k}^2 \left[ e^{iL\beta\cdot y - i \frac{B}{4} |y|^2} v \right]  = e^{iL\beta\cdot y - i\frac{B}{4} |y|^2} \\
    &\quad \times \left[ \partial_{k}^2 v + i \left( L\beta_k - \frac{B}{2} y_k \right) \partial_{k} v - i\frac{B}{2} v + i\left( L\beta_k - \frac{B}{2} y_k \right) \left[ \partial_{k} v + i \left( L\beta_k - \frac{B}{2} y_k \right) v \right] \right] \\
    &= e^{iL\beta\cdot y - i \frac{B}{4} |y|^2} \left[ \partial_{k}^2 v + i\left( 2L\beta_k - By_k \right) \partial_{k} v + \left( -i\frac{B}{2} - L^2\beta_k^2 + LB\beta_k y_k - \frac{B^2}{4} y_k^2 \right) v \right].
\end{align*}
%
Hence,
\begin{align*}
    \Delta_x u &= \frac{A}{L^3} e^{i\gamma + iL\beta\cdot y - i\frac{B}{4} |y|^2 } v \\
    &\quad \times \left[ \Delta_y v + i\left( 2L\beta - By \right) \cdot \nabla v + \left( -i\frac{B}{2}d  - L^2|\beta|^2 + LB\beta \cdot y - \frac{B^2}{4} |y|^2  \right) v \right].
\end{align*}
%
We have
\begin{align*}
    -|x|^2 u
    &= -\frac{A}{L} e^{i\gamma + iL\beta\cdot y - i \frac{B}{4} |y|^2} \left| Ly + X \right|^2 v \\
    &= \frac{A}{L^3} e^{i\gamma + iL\beta\cdot y - i \frac{B}{4} |y|^2} \left( -L^4 |y|^2 - 2L^3 X \cdot y - L^2 |X|^2 \right) v
,\end{align*}
%
thus
\begin{equation}
    \label{eqn: HO -- -Hu with general v}
    \begin{aligned}
        -Hu &= \frac{A}{L^3} e^{i\gamma + iL\beta\cdot y - i \frac{B}{4} |y|^2} \left\{ \Delta_y v - i B \left( \frac{d}{2}  v + \Lambda v \right) - L^2\left( |\beta|^2 + |X|^2 \right) v \right. \\
        & \qquad \left. +2iL\beta\cdot \nabla v + \left( LB\beta - 2L^3 X \right) \cdot yv + \left( -\frac{B^2}{4} -L^4 \right) |y|^2 v \right\},
    \end{aligned}
\end{equation}
%
where we denoted \( \Lambda v := y \cdot \nabla v \).
We now compute
\begin{align*}
    \partial_{t} u &= \partial_{t} \left( e^{i\gamma + i \beta\cdot (x-X) - i\frac{B}{4L^2} |x-X|^2} \frac{A}{L} v(s,y)  \right) \nonumber \\
    &\begin{aligned}
        &= e^{i\gamma + i\beta\cdot (x-X) - i\frac{B}{4L^2} |x-X|^2} \frac{A}{L} \left[ \partial_{t} v + \frac{A_t}{A} v - \frac{L_t}{L} (v + \Lambda v) - \frac{X_t}{L} \cdot \nabla v \right] \\
        &\quad + e^{i\gamma + i\beta\cdot (x-X) - i\frac{B}{4L^2} |x-X|^2} \frac{A}{L} iv \\
        &\qquad \times \left[ \gamma_t + \beta_t \cdot (x-X) - \beta\cdot X_t - \frac{B_t}{4L^2} |x-X|^2 \right. \\
        &\qquad\qquad \left. + \frac{2L_t B}{4L^3} |x-X(t)|^2 + \frac{2B}{4L^2} (x-X) \cdot X_t \right]
    \end{aligned}\\
    &\begin{aligned}
        &= e^{i\gamma + i\beta\cdot (x-X) - i\frac{B}{4L^2} |x-X|^2} \frac{A}{L^3} \left[ \partial_{s} v + \frac{A_s}{A} v - \frac{L_s}{L} (v+\Lambda v) - \frac{X_s}{L} \cdot \nabla v \right] \\
        &\quad + e^{i\gamma + i\beta\cdot (x-X) - i\frac{B}{4L^2} |x-X|^2} \frac{A}{L^3} iv\\
        &\qquad \times \left[\gamma_s + L\beta_s \cdot y - \beta \cdot X_s - \frac{B_s}{4} |y|^2 + \frac{2L_s B}{4L} |y|^2 + \frac{B}{2} y\cdot \frac{X_s}{L}  \right],
    \end{aligned}
\end{align*}
%
and hence 
\begin{equation}
    \label{eqn: i dt uj}
    \begin{aligned}
        i \partial_{t} u 
        &= e^{i\gamma + i\beta\cdot (x-X) - i\frac{B}{4L^2} |x-X|^2} \frac{A}{L^3} \left\{ i \partial_{s} v + (-\gamma_s + \beta\cdot X_s) v + \left( \frac{A_s}{A} - \frac{L_s}{L}  \right) iv - \frac{L_s}{L} i\Lambda v \right. \\
        &\qquad\qquad \left. - i\frac{X_s}{L} \cdot \nabla v + \left( -L\beta_s - \frac{BX_s}{2L} \right) \cdot yv + \left( \frac{B_s}{4} - \frac{B}{2} \frac{L_s}{L} \right) |y|^2 v \right\}.
    \end{aligned}
\end{equation}
%
This yields
\begin{equation}
    \label{eqn: HO -- idt u - Hu -- with v}
    \begin{aligned}
        i \partial_{t} u - Hu
        &= \frac{A}{L^3} e^{i\gamma + iL\beta\cdot y - i\frac{B}{4} |y|^2} \left\{ i \partial_{s} v + \left( -\gamma_s + \beta\cdot X_s - L^2 \left( |\beta|^2 + |X|^2 \right) \right) v \right. \\
        &\quad + \left( \frac{A_s}{A} - \frac{L_s}{L} - B \frac{d}{2}  \right) iv + \left( - \frac{L_s}{L} - B \right)i\Lambda v + i \left( 2L\beta - \frac{X_s}{L}  \right) \cdot \nabla v \\
        &\quad + \left( -2L^3 X + LB\beta - L\beta_s - \frac{B}{2} \frac{X_s}{L}  \right) \cdot yv \\
        &\quad + \left. \Delta_y v + \left[ \frac{B_s}{4} - \left( \frac{B^2}{4} + L^4 \right) - \frac{B}{2} \frac{L_s}{L} \right] |y|^2 v \right\}(s,y).
    \end{aligned}
\end{equation}
%




\subsubsection{Hamiltonian formulation for the free bubble}
\label{sect: HO -- Hamiltonian formulation for the free bubble}

The one free bubble corresponds to the modulation equations
\begin{equation*}
    \left| 
    \begin{aligned}
        &-\gamma_s + \beta\cdot X_s - L^2\left( |\beta|^2 + |X|^2 \right) = 0 \\
        &\frac{A_s}{A} - \frac{L_s}{L} - \frac{B}{2} d = 0  \\
        &- \frac{L_s}{L} - B = 0 \\
        &X_s = 2L^2 \beta \\
        &-2L^3X + LB\beta - L\beta_s - \frac{BX_s}{2L} = 0 \\
        &\frac{B_s}{4} - \left( \frac{B^2}{4} + L^4 \right) - \frac{B}{2} \frac{L_s}{L} = -1,
    \end{aligned}
    \right.
\end{equation*}
%
or equivalently
\begin{equation}
    \label{eqn: modulation ODEs -- linear part wrt time s - with v}
    \left|
        \begin{aligned}
            A_s &= \frac{AB}{2} (d-2) \\
            L_s &= -BL \\
            B_s &= -4 + 4L^4 - B^2 \\
            X_s &= 2L^2 \beta \\
            \beta_s &= -2L^2 X \\
            \gamma_s &= L^2 \left( |\beta|^2 - |X|^2 \right).
        \end{aligned}
    \right.
\end{equation}
%
If the parameters are solutions to system \eqref{eqn: modulation ODEs -- linear part wrt time s - with v}, the function \( v \) then has to solve the Harmonic Oscillator with respect to time \( s \) and space \( y \).

In time \( t \), as \( \frac{\textd}{\textd s} = L^2 \frac{\textd}{\textd t}  \), this system is 
\begin{equation}
    \label{eqn: modulation ODEs -- linear part wrt time t - with v}
    \left| 
        \begin{aligned}
            A_t &= \frac{AB}{2L^2} (d-2) \\
            L_t &= - \frac{B}{L} = -2L \partial_{B} \mathcal{E} \\
            B_t &= -\frac{4}{L^2} + 4L^2 - \frac{B^2}{L^2} = 2L \partial_{L} \mathcal{E} \\
            X_t &= 2\beta = \nabla_\beta \mathcal{R} \\
            \beta_t &= -2X = -\nabla_X \mathcal{R} \\
            \gamma_t &=  |\beta|^2 - |X|^2,
        \end{aligned}
    \right.
\end{equation}
%
with
\begin{equation*}
    \mathcal{E}(B, L) = \frac{1}{L^2} \left( 1 + \frac{B^2}{4}  \right) + L^2, \quad \text{ and } \quad 
    \mathcal{R}(X, \beta) = |X|^2 + |\beta|^2
.\end{equation*}
%
Then we set
\begin{equation*}
    k = \frac{1}{2} \log L, \quad L = e^{2k}
,\end{equation*}
%
and the system becomes
\begin{equation}
    \label{eqn: modulation ODEs -- linear part wrt time t - with hamiltonian and k}
    \left| 
        \begin{aligned}
            A_t &= \frac{AB}{2} (d-2) e^{-4k}\\
            k_t &= - \partial_{B} \mathcal{H} \\
            B_t &= \partial_{k} \mathcal{H} \\
            X_t &= \nabla_\beta \mathcal{H} \\
            \beta_t &= - \nabla_X \mathcal{H} \\
            \gamma_t &=  |\beta|^2 - |X|^2,
        \end{aligned}
    \right.
\end{equation}
%
with
\begin{equation*}
    \mathcal{H}(k, B, X, \beta) = \mathcal{E}(k, B) + \mathcal{R}(X, \beta) = e^{-4k} \left( \frac{B^2}{4} + 1 \right) + e^{4k} + |X|^2 + |\beta|^2
.\end{equation*}
%




\begin{lemma}
    \label{lemma: HO exact integration of parameters}
    There exists a symplectic change of variable \( (X, B, k, \beta) \mapsto (h, a, \xi, \theta) \in \mathbb{R}\times \mathbb{R}^d  \times [0, 2\pi] \times [0, 2\pi]^d \), such that the Hamiltonian in these variables is given by
    \begin{equation}
        E(h, a, \xi, \theta) = 4h + 2|a|^2
    ,\end{equation}
    %
    so that the flow in variable \( (h, a, \xi, \theta) \) is given by
    \begin{equation}
        \label{eqn: lemma -- HO exact integration of parameters - update of action-angle variables}
        \begin{aligned}
            a(t) &= a(0), \\
            \theta(t) &= \theta(0) + 2t, \\
            h(t) &= h(0), \\
            \xi(t) &= \xi(0) - 4t.
        \end{aligned}
    \end{equation}
    % 
    We have the explicit formulae:
    \begin{equation}
        \label{eqn: lemma -- HO exact integration of parameters - update of parameters - with v}
        \begin{aligned}
            A(t) &= A(0)  \left( \frac{L(t)}{L(0)}  \right)^{\frac{2-d}{2}}, \\
            e^{4k(t)} &= L(t)^2 = 2h(t) - \cos(\xi(t)) \sqrt{4h(t)^2 - 1}, \\
            B(t) &= 2\sin(\xi(t))\sqrt{4h(t)^2-1}, \\
            X_i(t) &= \sin(\theta_i(t)) \sqrt{2a_i(t)}, \quad i=1, \dots, d,\\
            \beta_i(t) &= \cos(\theta_i(t)) \sqrt{2a_i(t)}, \quad i=1, \dots, d,\\
            \gamma(t) &= \gamma(0) + \sum_{l=1}^d a_l(0) \left[ \sin(2\theta_l(t)) - \sin(2\theta_l(0)) \right]
        \end{aligned}
    \end{equation}
    %
\end{lemma}
%
%% Lemme complètement copié-collé depuis l'article non publié d'Erwan et Pierre (Proposition 3).

\begin{proof}
    For the \( (X, \beta) \) part, it suffices to check that 
    \begin{equation*}
        X_i = \sqrt{2a_i(0)} \sin(2t + \theta_i(0)) \quad \text{ and } \quad
        \beta_i = \sqrt{2a_i(0)} \cos(2t + \theta_i(0))\qquad i=1,\dots, d
    ,\end{equation*}
    %
    are solutions. The expression for \( \gamma(t) \) will be computed later within the proof of Lemma \ref{lemma: HO -- integration of gamma with v gaussian}.
    For the \( (k, B) \) part we use the method of generating functions, described e.g. in \cite[Sect.~VI.5]{hairerGeometricNumericalIntegration2006}.
    We can express \( B \) in terms of \( k \) and the Hamiltonian \( \mathcal{E} \), so that on the set \( \{B > 0\} \) we have:
    \begin{equation}
        \label{eqn: proof HO -- definition of B in terms of CAL E and k}
        B = 2\sqrt{e^{4k} \mathcal{E} - e^{8k} - 1}
    .\end{equation}
    %
    This equality holds for \( e^{4k} \in [e^{4k_0}, e^{4k_1}] \), where \( e^{4k_0}, e^{4k_1} \) are the reals roots of the polynomial \( -z^2 + \mathcal{E} z - 1 \),
    \begin{equation}
        \label{eqn: HO proof -- action angle variable - definition of k0 and k1}
        e^{4k_0} = \frac{1}{2} \left( \mathcal{E} - \sqrt{ \mathcal{E}^2 - 4 } \right),\quad e^{4k_1} = \frac{1}{2} \left( \mathcal{E} + \sqrt{ \mathcal{E}^2 - 4 } \right)
    .\end{equation}
    %

    In order to obtain a symplectic change of variables, we look for a function \( S(k, \mathcal{E}) \) such that 
    \begin{equation*}
        B = \frac{\partial S}{\partial k} (k, \mathcal{E})
    .\end{equation*}
    %
    We easily obtain \( S(k, \mathcal{E}) \), by integrating on \( [k_0, k] \):
    \begin{equation*}
        S(k, \mathcal{E}) = 2 \int_{k_0}^k \sqrt{e^{4z} \mathcal{E} - e^{8z} - 1} dz
    .\end{equation*}
    %
    The variable \( \phi \) which makes the mapping \( (B, k) \mapsto (\phi, \mathcal{E}) \) symplectic is defined by
    \begin{equation*}
        \phi = \frac{\partial S}{\partial \mathcal{E}}(k, \mathcal{E}) = \int_{k_0}^k \frac{e^{4z}}{\sqrt{e^{4z} \mathcal{E} - e^{8z} - 1}} dz
    .\end{equation*}
    %

    We have 
    \begin{equation*}
        \frac{\textd \phi}{\textd t} = \frac{e^{4k} k_t}{\sqrt{e^{4k} - e^{8k} - 1}} = \frac{- e^{4k} \partial_{B} \mathcal{E}}{\frac{B}{2} } =  \frac{- e^{-4k} \frac{B}{2} e^{4k}}{\frac{B}{2} } = -1
    .\end{equation*}
    %
    We now proceed to obtaining an explicit expression for \( \psi \):
    \begin{align*}
        \phi &= \int_{k_0}^k \frac{e^{4z}}{\sqrt{e^{4z} \mathcal{E} - e^{8z} - 1}} dz = \frac{1}{4} \int_{e^{4k_0}}^{e^{4k}} \frac{1}{\sqrt{ \mathcal{E}u - u^2 - 1 }} du \\
        &= \frac{1}{4 \sqrt{\frac{ \mathcal{E}^2 }{4} - 1 }}  \int_{e^{4k_0}}^{e^{4k}} \frac{1}{\sqrt{ 1 - \left( \frac{u - \frac{ \mathcal{E} }{2} }{\sqrt{ \frac{ \mathcal{E}^2 }{4} - 1  }}  \right)^2 }} du \\
        &= \frac{1}{4} \int_{ \frac{e^{4k_0} - \frac{ \mathcal{E} }{2} }{\sqrt{ \frac{ \mathcal{E}^2 }{4} - 1 }}  }^{ \frac{e^{4k} - \frac{ \mathcal{E} }{2} }{\sqrt{ \frac{ \mathcal{E}^2 }{4} - 1 }}  } \frac{1}{\sqrt{1 - u^2}} du.
    \end{align*}
    %
    Recall the definition \eqref{eqn: HO proof -- action angle variable - definition of k0 and k1} of \( k_0 \), which yields
    \begin{equation*}
        e^{4k_0} - \frac{ \mathcal{E} }{2} = - \sqrt{ \frac{\mathcal{E}^2}{4} - 1 } 
    .\end{equation*}
    %
    Therefore, 
    \begin{align*}
        \phi &= \frac{1}{4} \int_{-1}^{ \frac{e^{4k} - \frac{ \mathcal{E} }{2} }{\sqrt{ \frac{ \mathcal{E}^2 }{4} - 1 }}  } \frac{1}{\sqrt{1 - u^2}} du = \frac{1}{4} \left( \arcsin\left( { \frac{e^{4k} - \frac{ \mathcal{E} }{2} }{\sqrt{ \frac{ \mathcal{E}^2 }{4} - 1 }}  }  \right) + \frac{\pi}{2}  \right) \nonumber \\
        &= \frac{1}{4} \arcsin\left( { \frac{e^{4k} - \frac{ \mathcal{E} }{2} }{\sqrt{ \frac{ \mathcal{E}^2 }{4} - 1 }}  }  \right) + \frac{\pi}{8} \in \left[ 0, \frac{\pi}{4}  \right]
    .\end{align*}
    %
    We want the angle variable to lie in \( \left[ 0, 2\pi \right] \) so the above expression describes an eigth of a period. But we are only considering the set \( \{B>0\} \), thus the angle \( \xi \) we are looking for must lie only in \( [0, \pi] \). Hence we set \( (\xi, h) = (4\phi, \mathcal{E}/4) \) and let the Hamiltonian \( \mathcal{E}(\xi, h) = 4h \) with a slight abuse of notation. It is then clear that \( \frac{\textd h}{\textd t} = 0 \) and \( \frac{\textd \xi}{\textd t} = -4 \). Moreover,
    \begin{equation}
        \label{proof: HO proof -- definition of xi with k}
        \xi = \arcsin\left( \frac{e^{4k} - \frac{ \mathcal{E} }{2} }{\sqrt{ \frac{ \mathcal{E}^2 }{4} - 1 }} \right) + \frac{\pi}{2}
        \in \left[ 0, \pi  \right]
    ,\end{equation}
    %
    and hence
    \begin{equation*}
        \frac{e^{4k} - \frac{ \mathcal{E} }{2} }{\sqrt{ \frac{ \mathcal{E}^2 }{4} - 1 }} = \sin\left( \xi - \frac{\pi}{2} \right)
        = - \cos\left( \xi \right)
    .\end{equation*}
    %
    We obtain 
    \begin{align*}
        e^{4k} = L^2 &= \frac{ \mathcal{E}}{2} - \cos(\xi) \sqrt{ \frac{ \mathcal{E}^2 }{4} - 1 } = 2h - \cos(\xi) \sqrt{4 h^2 - 1} \\
        &= 2h \left( 1 - \cos(\xi) \sqrt{1 - \frac{1}{4h^2}} \right)
    .\end{align*}
    %
    
    With this formula, we have
    \begin{equation*}
        0 < L^2 < 4h = \mathcal{E}
    ,\end{equation*}
    %
    and \eqref{eqn: proof HO -- definition of B in terms of CAL E and k} becomes
    \begin{align*}
        B &= 2\sqrt{ \mathcal{E}e^{4k} - e^{8k} - 1} = 2\sqrt{4he^{4k} - (e^{4k})^2 - 1} = 2\sqrt{ (4h^2 - 1)\sin^2(\xi) } \\
        &= 2 \sin(\xi) \sqrt{4h^2 - 1},
    \end{align*}
    %
    where the last equality holds for \( \xi \in [0, \pi] \).

    Finally, it remains to integrate the ODE on \( A \), which is
    \begin{equation*}
        A_t = \frac{AB}{2} (d-2) e^{-4k}
    .\end{equation*}
    %
    From the expressions we just obtained we get
    \begin{equation*}
        A_t = A (d-2) \frac{\sin(\xi) \sqrt{4h^2 - 1}}{2h - \cos(\xi) \sqrt{ 4 h^2 - 1 }}
    .\end{equation*}
    %
    The solution to this equation is of the form
    \begin{equation*}
        A(t) = A(0) \exp\left\{(d-2) \int_{0}^t \frac{\sin(\xi(s)) \sqrt{4h(s)^2 - 1}}{2h(s) - \cos(\xi(s)) \sqrt{ 4 h(s)^2 - 1 }} \textd s\right\}
    .\end{equation*}
    %
    Moreover, we know that \( s\mapsto h(s) \) is constant, and that \( \xi(s) = \xi(0) - 4s \). Hence we have to solve
    \begin{equation*}
        A(t) = A(0) \exp\left\{(d-2) \int_{0}^t \frac{\sin(\xi(0) - 4s) \sqrt{4h(0)^2 - 1}}{2h(0) - \cos(\xi(0) - 4s) \sqrt{ 4 h(0)^2 - 1 }} \textd s\right\}
    .\end{equation*}
    %
    One can easily check that we have the following equality:
    \begin{align*}
        &\int_{0}^t \frac{\sin(\xi(0) - 4s) \sqrt{4h(0)^2 - 1}}{2h(0) - \cos(\xi(0) - 4s) \sqrt{ 4 h(0)^2 - 1 }} \textd s \\
        &= -\frac{1}{4} \left[ \log\left( 2h(t) - \cos(\xi(t)) \sqrt{4h(t)^2 - 1} \right) - \log\left( 2h(0) - \cos(\xi(0)) \sqrt{4h(0)^2 - 1} \right) \right].
    \end{align*}
    %
    Note that, unless \( h(0) = \frac{1}{2}  \) or \( h(t) = \frac{1}{2} \), these quantities are well-defined since \( 2h(s) > \sqrt{4h(s)^2 - 1}, s\in \{0, t\} \).
    Thus, we obtain 
    \begin{align*}
        A(t) &= A(0) e^{ \frac{2-d}{4} \left[ \log\left( 2h(t) - \cos(\xi(t)) \sqrt{4h(t)^2 - 1} \right) - \log\left( 2h(0) - \cos(\xi(0)) \sqrt{4h(0)^2 - 1} \right) \right] } \\
        &= C \left( 2h(0) - \cos(\xi(0) - 4t) \sqrt{4h(0)^2 - 1} \right)^{\frac{2-d}{4}},
    \end{align*}
    %
    where we defined \( C :=  A(0) \left( 2h(0) - \cos(\xi(0)) \sqrt{4h(0)^2 - 1} \right)^{\frac{d-2}{4}} \). We recognize here the expressions for \( L(0)^2 \) and \( L(t)^2 \), hence the claimed result.
\end{proof}
%




In pratice, one would know the parameters \( (A, L, B, X, \beta, \gamma) \) and would apply \eqref{eqn: lemma -- HO exact integration of parameters - update of parameters - with v} in order to update them. We have the following result, which allows to obtain action-angle variables from the bubble's parameters:
\begin{lemma}
    \label{lemma: HO change of variables between params and action-angle variables}
    The change of variables \( (L, B, X, \beta) \mapsto (h, a, \xi, \theta) \) is explicit, and at time \( t=0 \) we have
    \begin{equation}
        \label{eqn: lemma -- HO exact integration of parameters - change of variables}
        \begin{aligned}
            a_i(0) &= \frac{1}{2} \left( X_i(0)^2 + \beta_i(0)^2 \right), i=1, \dots, d, \\
            \theta_i(0) &= \arctan\left( \frac{X_i(0)}{\beta_i(0)}  \right), i=1, \dots, d, \\
            h(0) &= \frac{L(0)^4 + 1 + \frac{B(0)^2}{4}}{4L(0)^2}, \\
            \xi(0) &= \arctan\left( \frac{B(0)}{4h(0) - 2L(0)^2}  \right),
        \end{aligned}
    \end{equation}
    %
    whenever \( \theta(0) \) and \( \xi(0) \) are well-defined. When any one of them is ill-defined -- which happens when \( X_i(0) = \beta_i(0) = 0, i\in \{1, \dots, d\} \) or when \( L(0)=1 \text{ and } B(0)=0 \) -- any value can be taken and the time-evolution of \( A(t), L(t), B(t), X(t) \) and \( \beta(t) \) will not depend on the value.
\end{lemma}


Note that to define \( \xi(0) \) we could also use equation \eqref{proof: HO proof -- definition of xi with k}, but this is not appropriate from a computational point of view. Some more details are given in Remark \ref{rmk: HO -- numerical considerations}.


\begin{proof}
    We have \( a_i(0) = \frac{1}{2} \left( X_i(0)^2 + \beta_i(0)^2 \right), i=1, \dots, d \). If \( a_i(0) > 0 \) we can define \( \theta_i(0) \) as \( \theta_i(0) = \arctan \left( \frac{X_i(0)}{\beta_i(0)}  \right) \). Otherwise, if \( a_i(0) = 0 \), then we recall that \( a(t) = a(0) \) and hence -- whatever \( \theta(0) \) -- we have \( X_i(t) = 0 \) and \( \beta_i(t) = 0 \). Therefore, in the case \( a_i(0) = 0 \), the exact value of \( \theta_i(0) \) does not change the behavior of \( t\mapsto (X_i(t), \beta_i(t)) \).
    
    For the \( (L, B) \) part,
    \begin{equation*}
        L(0)^2 - 2h(0) = -\cos(\xi(0)) \sqrt{4h(0)^2 - 1}
    ,\end{equation*}
    %
    hence
    \begin{equation*}
        (L(0)^2 - 2h(0))^2 = L(0)^4 - 4L(0)^2 h(0) + 4h(0)^2 = \cos(\xi(0))^2 \left( 4 h(0)^2 - 1 \right)
    .\end{equation*}
    %
    We also have
    \begin{equation*}
        \left( \frac{B(0)}{2} \right)^2 = \frac{B(0)^2}{4} = \sin(\xi(0))^2 \left( 4h(0)^2 - 1 \right) 
    .\end{equation*}
    %
    Then,
    \begin{equation*}
        L(0)^4 - 4L(0)^2 h(0) + 4h(0)^2 + \frac{B(0)^2}{4} = 4h(0)^2 - 1
    ,\end{equation*}
    %
    that is
    \begin{equation*}
        4L(0)^2 h(0) = L(0)^4 + \frac{B(0)^2}{4} + 1 
    .\end{equation*}
    %
    We deduce that \( h(0), L(0) \neq 0 \), and therefore
    \begin{equation*}
        h(0) = \frac{L(0)^4 + \frac{B(0)^2}{4} + 1}{4L(0)^2} 
    .\end{equation*}
    %
    Note that \( h(0) \) is bounded from below by \( \frac{1}{2}  \). Indeed,
    \begin{align*}
        &L(0)^4 - 2L(0)^2 + 1 + \frac{B(0)^2}{4} 
        = \left( L(0)^2 - 1 \right)^2 + \frac{B(0)^2}{4} \geq 0 \\
        \iff& L(0)^4 + 1 + \frac{B(0)^2}{4} \geq 2L(0)^2\\
        \iff& h(0) \geq \frac{1}{2}.
    \end{align*}
    %
    From this we also get that \( h(0) = \frac{1}{2} \iff L(0)^2 = 1  \) and \( B(0) = 0 \).
    
    If \( h(0) > \frac{1}{2}  \), we have
    \begin{equation*}
        \left\{ \begin{aligned}
            2h(0) - L(0)^2 &= \cos(\xi(0)) \sqrt{4h(0)^2 - 1}\\
            \frac{B(0)}{2} &= \sin(\xi(0)) \sqrt{4h(0)^2 - 1},
        \end{aligned} \right.
        \implies \frac{B(0)/2}{2h(0) - L(0)^2} = \tan(\xi(0))
    ,\end{equation*}
    %
    hence
    \begin{equation*}
        \xi(0) = \arctan\left( \frac{B(0)/2}{2h(0) - L(0)^2} \right)
    .\end{equation*}
    %
    Otherwise, in the case \( h(0) = \frac{1}{2} \), the value of \( \xi(0) \) is not rigourously defined. However, as previously, the exact value of \( \xi(0) \) is not important because \( h(t) = h(0) = \frac{1}{2} \), which means that \( L(t)^2=1 \) and \( B(t) = 0 \). Therefore, in the case \( h(0) = \frac{1}{2} \), the mapping \( t\mapsto (L(t), B(t)) \) does not depend on the value of \( \xi(0) \). Finally, since the mapping \( t\mapsto L(t) \) does not depend on \( \xi(0) \) in the case \( h(0) = \frac{1}{2} \), we also have that \( t\mapsto A(t)\) does not depend on the exact value of \( \xi(0) \), thanks to the expression of \( A(t) = A(0) \left( L(t) / L(0)  \right)^{\frac{2-d}{2}}  \).
\end{proof}



%
%\subsubsection{Computation of Sobolev norms}
%
%We let
%\begin{equation*}
%    u(x) = \frac{A}{L} e^{i\gamma + i L\beta\cdot y - i \frac{B}{4} |y|^2} v(s,y)
%,\end{equation*}
%%
%and compute the conservation laws. 
%
%
%\begin{lemma}
%    \label{lemma: HO -- computation of sobolev norms}
%    Assume \( v \) is real-valued and radial, then 
%    \begin{equation}
%        \|u\|_{ \mathbb{L}^2 } = A \|v\|_{ \mathbb{L}^2 }
%    ,\end{equation}
%    %
%    and 
%    \begin{equation*}
%        E(u) = \frac{A^2 L^{d-4}}{2} \left\{ \left( |\beta|^2 + |X|^2 \right)\|v\|^2_{ \mathbb{L}^2 } +
%        \|\nabla v\|_{ \mathbb{L}^2 }^2 + \left( \frac{B^2}{4} + L^4 \right) \|yv\|^2_{ \mathbb{L}^2 } + \frac{A^2}{2}\|v\|^4_{ \mathbb{L}^4 } \right\}.
%    \end{equation*}
%    %
%    and 
%    \begin{equation*}
%        \mathcal{P} = A^4 L^{2(d-2)} \left( |\beta|^2 + |X|^2 \right) \|v\|^4_{ \mathbb{L}^2 }
%    .\end{equation*}
%    %
%\end{lemma}
%
%
%% \begin{remark}
%%     Note that the energy bound implies
%%     \begin{equation*}
%%         A^2 \mathcal{E} \lesssim 1
%%     ,\end{equation*}
%%     %
%%     and hence the typically expected blow-up regime is 
%%     \begin{equation*}
%%         A \to 0, \quad \mathcal{E} \sim \frac{1}{A^2} 
%%     .\end{equation*}
%%     %
%%     % We probably expect this with \( A \) being oscillatory again.
%% \end{remark}
%
%
%\begin{proof}
%    The \( \mathbb{L}^2 \) norm of \( u \) is trivial. We now compute the kinetic momentum:
%    \begin{align*}
%        \Im \int_{ \mathbb{R}^d } \partial_{j} u \bar{u} 
%        &= \frac{A^2}{L^2} \int_{ \mathbb{R}^d } \frac{1}{L} \left( L\beta_j - \frac{B}{2} y_j \right)|v(y)|^2 L^d dy \\
%        &= \beta_j A^2 L^{d-2} \|v\|^2_{ \mathbb{L}^2 } - \frac{A^2 B}{2} L^{d-3} \int_{ \mathbb{R}^d } y_j |v|^2,
%    \end{align*}
%    %
%    and
%    \begin{equation*}
%        \int_{ \mathbb{R}^d } x_j |u|^2 = A^2 L^{d-1} \int_{ \mathbb{R}^d } y_j |v(y)|^2 dy + A^2 X_j L^{d-2} \|v\|^2_{ \mathbb{L}^2 }
%    .\end{equation*}
%    %
%    Hence, as \( v \) is symmetric,
%    \begin{equation*}
%        \mathcal{P}_j = A^4 L^{2(d-2)} \left( |\beta_j|^2 + |X_j|^2 \right) \|v\|^4_{ \mathbb{L}^2 }
%    .\end{equation*}
%    %
%    We have by \eqref{eqn: HO -- -Hu with general v}:
%    \begin{equation*}
%        \begin{aligned}
%            -Hu &= \frac{A}{L^3} e^{i\gamma + iL\beta\cdot y - i \frac{B}{4} |y|^2} \left\{ \Delta_y v - i B \left( \frac{d}{2}  v + \Lambda v \right) - L^2\left( |\beta|^2 + |X|^2 \right) v \right. \\
%            & \qquad \left. +2iL\beta\cdot \nabla v + \left( LB\beta - 2L^3 X \right) \cdot yv + \left( -\frac{B^2}{4} -L^4 \right) |y|^2 v \right\},
%        \end{aligned}
%    \end{equation*}
%    %
%    and hence as \( v \) is real-valued and symmetric,
%    \begin{equation*}
%        \Re \int_{ \mathbb{R}^d } -Hu \bar{u} = A^2 L^{d-4} \left\{ -L^2 \left( |\beta|^2 + |X|^2 \right) \|v\|^2_{ \mathbb{L}^2 }
%        + \int_{ \mathbb{R}^d } \left[ \Delta - \left( \frac{B^2}{4} + L^4 \right) |y|^2 \right] v\bar{v} dy \right\}
%    ,\end{equation*}
%    %
%    which yields
%    \begin{align*}
%        E(u) &= \frac{1}{2} \langle Hu, u \rangle + \frac{1}{4} \int_{ \mathbb{R}^d } |u|^4 \\
%        &= \frac{A^2 L^{d-4}}{2} \left\{ \left( |\beta|^2 + |X|^2 \right)\|v\|^2_{ \mathbb{L}^2 } +
%        \|\nabla v\|_{ \mathbb{L}^2 }^2 + \left( \frac{B^2}{4} + L^4 \right) \|yv\|^2_{ \mathbb{L}^2 } + \frac{A^2}{2}\|v\|^4_{ \mathbb{L}^4 } \right\}.
%    \end{align*}
%    %
%\end{proof}
%
%
%
%
%
%
%
%







\subsection{A particular case: \( v(s,y) = e^{-\frac{1}{2} |y|^2} \)}
\label{sect: HO -- ODEs on the modulation parameters with particular vj}


When we have chosen the modulation equations \eqref{eqn: modulation ODEs -- linear part wrt time t - with v}, we chose them so that \( v \) is itself solution to the Harmonic Oscillator in time \( s \) and space \( y \). This could maybe ease the process of finding a solution, but we can do even better: if we consider a particular expression for the function \( v \) we are also able to compute its time derivative and its laplacian in \eqref{eqn: HO -- idt u - Hu -- with v}. By doing so, we can take into account those operators within new modulation equations. 

We then have to choose a particular shape for the functions \( v \). Note that the treatment of each bubble \( j \) is done separately, hence each function \( v_j \) can have its own shape different from the others. For simplicity, we take all the functions \( v_j \) to be gaussian functions. There are two main reasons for this choice: the first one is that when the width of a normalized gaussian function tend to zero, we can assimilate the function to a Dirac mass and hence view the proposed algorithm as an exact particle algorithm. The second reason is that we will have to consider the shapes of the functions \( v_j \) when we will add cubic interactions, and the gaussian functions allow us to perform most of the computations exactly, which greatly reduces the computational complexity and improves its accuracy. The gaussian shape is also motivated by the idea of Gaussian Beams.

From now on, the bubbles will have the following expression:
\begin{equation}
    \label{eqn: generic discretization of psi -- expression for uj -  vj gaussian}
    u_j(t,x) := \frac{A_j}{L_j} e^{i\gamma_j + i L_j \beta_j \cdot y_j - i \frac{B_j}{4} |y_j|^2} e^{- \frac{1}{2} |y_j|^2}, \quad \text{with } \quad y_j := \frac{x - X_j}{L_j}
.\end{equation}


Since equation \eqref{eqn: cNLS with psi -- linear part} is linear with respect to \( \psi \), if we use the ansatz \eqref{eqn: generic discretization of psi -- ansatz} then we have to solve the Harmonic Oscillator equation for each bubble \( j \):
\begin{equation}
    \label{eqn: harmonic oscillator -- bubble j}
    i \partial_{t} u_j = -\Delta_x u_j + |x|^2 u_j
\end{equation}
%
We can reuse equation \eqref{eqn: HO -- idt u - Hu -- with v} with \( v_j(s_j, y_j) = e^{-\frac{|y_j|^2}{2}}  \), and take advantage of the following equality:
\begin{equation*}
    \Delta_y v(y) = \Delta_y \left( e^{-\frac{|y|^2}{2} } \right) = \left( |y|^2 - d \right) v
\end{equation*}
%
This yields almost the same system of equations as \eqref{eqn: modulation ODEs -- linear part wrt time s - with v}: only the equation on \( \gamma_s \) changes, and we have
\begin{equation*}
    (\gamma_j)_s = L^2 \left( |\beta|^2 - |X|^2 \right) - d
.\end{equation*}
%
With respect to time \( t \), we get 
\begin{equation}
    \label{eqn: HO -- ODE on gamma}
    (\gamma_j)_t = |\beta|^2 - |X|^2 - \frac{d}{L^2} .
\end{equation}
%

If the parameters satisfy the modulation equations \eqref{eqn: modulation ODEs -- linear part wrt time t - with v} (where the equation on \( \gamma_t \) has been changed to \eqref{eqn: HO -- ODE on gamma}), we have
\begin{equation*}
    i \partial_{t} u_j - Hu_j = 0
,\end{equation*}
%
{\em i.e.} the Harmonic oscillator equation is automatically satisfied for the bubble \( j \).
Furthermore, we have the following result:
\begin{lemma} 
    \label{lemma: HO -- integration of gamma with v gaussian}
    Equation \eqref{eqn: HO -- ODE on gamma} can be integrated explicitely by using equation \eqref{eqn: lemma -- HO exact integration of parameters - update of parameters - with v}, and we have 
    \begin{equation}
        \label{eqn: HO -- explicit expression of gamma}
        \begin{aligned}
            \gamma(T) &= 
                \gamma(0) + \sum_{l=1}^d \frac{a_l(0)}{2} \left[ \sin(2\theta_l(T)) - \sin(2\theta_l(0)) \right] \\
                &\qquad + \frac{d}{2} \arctan\left( \left( 2h(0) + \sqrt{4h(0)^2 - 1} \right) \tan\left( \frac{\xi(0)}{2} - 2T \right) \right) \\
                &\qquad - \frac{d}{2} \arctan\left( \left( 2h(0) + \sqrt{4h(0)^2 - 1} \right) \tan\left( \frac{\xi(0)}{2} \right) \right) -m_T \frac{\pi d}{2} ,
        \end{aligned}
    \end{equation}
    %
    where, if \( m_0 \in \mathbb{Z} \) is such that \( \frac{\xi(0)}{2} \in m_0\pi + \left[ -\frac{\pi}{2}, \frac{\pi}{2}  \right] \), then \( m_T \in \mathbb{Z} \) is defined by \( \frac{\xi(T)}{2} \in (m_0-m_T)\pi + \left[ -\frac{\pi}{2}, \frac{\pi}{2}  \right] \). 
    
    Moreover, in the cases where \( a_i(0) = 0,\, i\in\{1, \dots, d\} \) or \( h(0) = \frac{1}{2} \), the formula \eqref{eqn: lemma -- HO exact integration of parameters - change of variables} for \( \theta_i(0),\, i\in \{1, \dots, d\} \) or \( \xi(0) \) are ill-defined, but any value can be taken as a substitution and this will not affect the behavior of the mapping \( t\mapsto \gamma(t) \).
\end{lemma}


\begin{proof}
    We now proceed to the direct integration of \( \gamma_t \).
    \begin{equation*}
        \gamma(T) - \gamma(0) = \int_0^T \dot{\gamma}(t)dt = \int_0^T \left[ |\beta(t)|^2 - |X(t)|^2 - \frac{d}{L(t)^2}  \right]dt
    .\end{equation*}
    %
    We have
    \begin{align*}
        &\int_0^T \left[ |\beta(t)|^2 - |X(t)|^2 \right] dt \\
        &= \int_0^T \left\{ \sum_{l=1}^d 2a_l \cos(\theta_l(t))^2 - \sum_{l=1}^d 2a_l\sin(\theta_l(t))^2\right\} dt\\
        &= \int_0^T 2\sum_{l=1}^d a_l \left( \cos(\theta_l(t))^2 - \sin(\theta_l(t))^2 \right) dt \\
        &= \int_0^T \sum_{l=1}^d 2a_l \cos(2\theta_l(t)) dt \\
        &= \sum_{l=1}^d \frac{a_l}{2} \left[ \sin(2\theta_l(T)) - \sin(2\theta_l(0)) \right],
    \end{align*}
    %
    where the last equality has been obtained using \eqref{eqn: lemma -- HO exact integration of parameters - update of action-angle variables}. Owing to \eqref{eqn: lemma -- HO exact integration of parameters - update of parameters - with v},
    \begin{align*}
        \int_0^T \frac{d}{L(t)^2} dt &= \int_0^T \frac{d}{\underbrace{2h(0)}_{=:c_1} - \underbrace{\sqrt{4h(0)^2-1}}_{=:c_2} \cos(\xi(0)-4t)} dt\\
        &= \int_0^T \frac{d}{c_1 - c_2 \cos(\xi(0) - 4t)} dt \\
        &= \frac{d}{4} \int^{\xi(0)}_{\xi(0)-4T} \frac{1}{c_1 - c_2 \cos(t)} dt.
    \end{align*}
    %
    Recall the following trig. identity:
    \begin{equation*}
        \cos(2\tau) = \frac{1 - \tan(\tau)^2}{1 + \tan(\tau)^2}, \quad \tau \in \mathbb{R}
    ,\end{equation*}
    %
    hence
    \begin{align*}
        &\int_0^T \frac{d}{c_1 - c_2 \cos(\xi(0) - 4t)} dt \\
        &= \frac{d}{4} \int^{\xi(0)}_{\xi(0)-4T} \frac{1}{c_1 - c_2 \frac{1 - \tan(t/2)^2}{1 + \tan(t/2)^2} } dt \\
        &= \frac{d}{4} \int^{\xi(0)}_{\xi(0)-4T} \frac{1 + \tan(t/2)^2}{c_1(1 + \tan(t/2)^2) - c_2(1 - \tan(t/2)^2) } dt \\
        &= \frac{d}{4} \int^{\xi(0)}_{\xi(0)-4T} \frac{1 + \tan(t/2)^2}{(c_1 + c_2)\tan(t/2)^2 + c_1 - c_2} dt \\
        &= \frac{d}{4(c_1 - c_2)} \int^{\xi(0)}_{\xi(0)-4T} \frac{1 + \tan(t/2)^2}{\frac{c_1 + c_2}{c_1-c_2}\tan(t/2)^2 + 1} dt \\
        &= \frac{d}{2(c_1 - c_2)} \int^{\frac{\xi(0)}{2}}_{\frac{\xi(0)}{2}-2T} \frac{1 + \tan(t)^2}{\frac{c_1 + c_2}{c_1-c_2}\tan(t)^2 + 1} dt \\
        &= \frac{d}{2(c_1 - c_2)} \int^{\frac{\xi(0)}{2}}_{\frac{\xi(0)}{2}-2T} \frac{ \frac{d}{dt} (\tan(t)) }{\frac{c_1 + c_2}{c_1-c_2}\tan(t)^2 + 1} dt \\
        &= \frac{d}{2(c_1 - c_2)} \frac{1}{\sqrt{\frac{c_1 + c_2}{c_1-c_2}}}  \int^{\frac{\xi(0)}{2}}_{\frac{\xi(0)}{2}-2T} \frac{ \frac{d}{dt} \left( \sqrt{\frac{c_1 + c_2}{c_1-c_2}} \tan(t) \right) }{\left[\sqrt{\frac{c_1 + c_2}{c_1-c_2}} \tan(t)\right]^2 + 1} dt.
    \end{align*}
    %
    Moreover, \( (c_1 - c_2)(c_1 + c_2) = c_1^2 - c_2^2 = (2h)^2 - (4h^2 - 1) = 1 \) and \( c_1 - c_2 > 0 \), thus \( \sqrt{\frac{c_1+c_2}{c_1-c_2} } = (c_1+c_2) \) and 
    \begin{equation*}
        \int_0^T \frac{d}{L(t)^2} dt
        = \frac{d}{2} \int_{\frac{\xi(0)}{2} - 2T}^{\frac{\xi(0)}{2}} \frac{\frac{d}{dt} \left( (c_1+c_2) \tan(t) \right)}{\left( (c_1+c_2) \tan(t) \right)^2 + 1} dt.
    \end{equation*}
    %
    Now let \( m_0 \in \mathbb{Z} \) such that \( \frac{\xi(0)}{2} \in m_0\pi + \left( -\frac{\pi}{2}, \frac{\pi}{2}  \right] \), and \( m_T \in \mathbb{Z} \) such that \( \frac{\xi(T)}{2} \in (m_0-m_T)\pi + \left( -\frac{\pi}{2}, \frac{\pi}{2}  \right] \). We recall that \( \xi(T) = \xi(0) - 4T \). Then
    \begin{align*}
        &\int_0^T \frac{d}{L(t)^2} dt = \frac{d}{2} \int_{\frac{\xi(0)}{2}-2T}^{\frac{\xi(0)}{2}} \underbrace{\frac{\frac{d}{dt} \left( (c_1+c_2) \tan(t) \right)}{\left( (c_1+c_2) \tan(t) \right)^2 + 1}}_{=:f(t)} dt \\
        &= \frac{d}{2} \int_{m_0\pi - \frac{\pi}{2} }^{\frac{\xi(0)}{2}} f(t) dt  
            + \frac{d}{2} \int_{(m_0-1)\pi - \frac{\pi}{2} }^{m_0\pi - \frac{\pi}{2}} f(t) dt 
            + \dots + \frac{d}{2} \int_{\frac{\xi(0)}{2}-2T}^{(m_0-m_T)\pi + \frac{\pi}{2} } f(t) dt.
    \end{align*}
    %
    For \( m\in \mathbb{Z} \), we have 
    \begin{align*}
        \int_{m\pi - \frac{\pi}{2}}^{m\pi + \frac{\pi}{2}} f(t) dt 
        &= \left[ \arctan\left( (c_1+c_2) \tan(t) \right) \right]_{m\pi - \frac{\pi}{2}}^{m\pi + \frac{\pi}{2}} \\
        &= \left[ \arctan\left( (c_1+c_2) \tan(t) \right) \right]_{- \frac{\pi}{2}}^{\frac{\pi}{2}} = \pi.
    \end{align*}
    %
    Now write \( \widetilde{\frac{\xi(0)}{2}} := \frac{\xi(0)}{2} - m_0\pi \in \left( -\frac{\pi}{2}, \frac{\pi}{2} \right] \), and \( \widetilde{\frac{\xi(t)}{2}} := \frac{\xi(t)}{2} - (m_0 - m_T)\pi \in \left( -\frac{\pi}{2}, \frac{\pi}{2} \right] \). Then,
    \begin{align*}
        &\int_0^T \frac{d}{L(t)^2} dt \\
        &= \frac{d}{2} (m_T-1)\pi
            + \frac{d}{2} \int_{m_0\pi - \frac{\pi}{2} }^{\frac{\xi(0)}{2}} f(t) dt
            + \frac{d}{2} \int_{\frac{\xi(0)}{2}-2T}^{(m_0-m_T)\pi + \frac{\pi}{2}} f(t) dt\\
        &= (m_T-1)\frac{\pi d}{2} 
            + \frac{d}{2} \int_{- \frac{\pi}{2} }^{\frac{\widetilde{\xi(0)}}{2}} f(t) dt
            + \frac{d}{2} \int_{\frac{\widetilde{\xi(t)}}{2}}^{\frac{\pi}{2}} f(t) dt\\
        &= (m_T-1)\frac{\pi d}{2}
            + \frac{d}{2} \left[ \arctan\left( (c_1+c_2) \tan(t) \right) \right]_{- \frac{\pi}{2} }^{\frac{\widetilde{\xi(0)}}{2}}
            + \frac{d}{2} \left[ \arctan\left( (c_1+c_2) \tan(t) \right) \right]_{\frac{\widetilde{\xi(t)}}{2}}^{\frac{\pi}{2}} \\
        &= (m_T-1)\frac{\pi d}{2}
            + \frac{d}{2} \arctan\left( (c_1+c_2) \tan\left( \frac{\widetilde{\xi(0)}}{2} \right) \right) + \frac{\pi d}{2} \\
        &\qquad + \frac{\pi d}{2} - \arctan\left( (c_1+c_2) \tan\left( \frac{\widetilde{\xi(t)}}{2} \right) \right) \\
        &= m_T \frac{\pi d}{2}
            + \frac{d}{2} \arctan\left( (c_1+c_2) \tan\left( \frac{\widetilde{\xi(0)}}{2} \right) \right)
            - \frac{d}{2} \arctan\left( (c_1+c_2) \tan\left( \frac{\widetilde{\xi(t)}}{2} \right) \right) \\
        &= m_T \frac{\pi d}{2}
            + \frac{d}{2} \arctan\left( (c_1+c_2) \tan\left( \frac{\xi(0)}{2} \right) \right)
            - \frac{d}{2} \arctan\left( (c_1+c_2) \tan\left( \frac{\xi(0)}{2} - 2T \right) \right)
    \end{align*}
    %
    Hence
    \begin{align*}
        -\int_0^T \frac{d}{L(t)^2} dt 
        &= \frac{d}{2} \arctan\left( (c_1+c_2) \tan\left( \frac{\xi(0)}{2} - 2T \right) \right) \\
        &\quad - \frac{d}{2} \arctan\left( (c_1+c_2) \tan\left( \frac{\xi(0)}{2} \right) \right) -m_T \frac{\pi d}{2}
    .\end{align*}
    %

    Finally, it remains to show that if \( a_i(0) = 0,\, i\in \{1, \dots, d\} \) or \( h(0) = \frac{1}{2} \), then the behavior of \( t\mapsto \gamma(t) \) does not depend on the exact value of \( \theta_i(0),\, i\in\{1, \dots, d\}  \) or \( \xi(0) \). The exact formula for \( \gamma(T) \) is:
    \begin{align*}
        \gamma(T) &= 
                \gamma(0) + a_1(0) \left[ \sin(2\theta_1(T)) - \sin(2\theta_1(0)) \right] \\
                &\qquad + a_2(0) \left[ \sin(2\theta_2(T)) - \sin(2\theta_2(0)) \right] \\
                &\qquad + \frac{d}{2} \arctan\left( \left( 2h(0) + \sqrt{4h(0)^2 - 1} \right) \tan\left( \frac{\xi(0)}{2} - 2T \right) \right) \\
                &\qquad - \frac{d}{2} \arctan\left( \left( 2h(0) + \sqrt{4h(0)^2 - 1} \right) \tan\left( \frac{\xi(0)}{2} \right) \right) -m_T \frac{\pi d}{2} .
    \end{align*}
    %
    It is clear that if \( a_i(0) = 0 \) then \( \gamma(t) \) does not depend on \( \theta_i(0) \) nor \( \theta_i(t) \), \( i\in \{1, \dots, d\} \). If \( h(0) = \frac{1}{2} \), then
    \begin{equation*}
        2h(0) + \sqrt{4h(0)^2 - 1} = 1
    ,\end{equation*}
    %
    so that 
    \begin{align*}
        &\frac{d}{2} \arctan\left( \left( 2h(0) + \sqrt{4h(0)^2 - 1} \right) \tan\left( \frac{\xi(0)}{2} - 2T \right) \right) \\
        &\qquad - \frac{d}{2} \arctan\left( \left( 2h(0) + \sqrt{4h(0)^2 - 1} \right) \tan\left( \frac{\xi(0)}{2} \right) \right) -m_T \frac{\pi d}{2} \\
        &= \frac{d}{2} \arctan\left( \tan\left(\frac{\xi(T)}{2}\right) \right) - \arctan\left( \tan\left(\frac{\xi(0)}{2}\right) \right) - m_T\frac{\pi d}{2} 
    \end{align*}
    %
    Since \( \arctan: \mathbb{R}\mapsto \left( -\frac{\pi}{2}, \frac{\pi}{2} \right] \), we have
    \begin{align*}
        &\frac{d}{2} \arctan\left( \tan\left(\frac{\xi(T)}{2}\right) \right) - \frac{d}{2} \arctan\left( \tan\left(\frac{\xi(0)}{2}\right) \right) - m_T\frac{\pi d}{2}  \\
        &= \frac{d}{2} \widetilde{\frac{\xi(T)}{2}} - \frac{d}{2} \widetilde{\frac{\xi(0)}{2}} - m_T \frac{\pi d}{2} \\
        &= \frac{d}{2} \frac{\xi(T)}{2} - (m_0 - m_T)\frac{\pi d}{2}  - \frac{d}{2} \left(\frac{\xi(0)}{2} - m_0\pi \right) - m_T \frac{\pi d}{2}  \\
        &=  \frac{d}{2} \left( \frac{\xi(T)}{2} - \frac{\xi(0)}{2}\right) = -Td.
    \end{align*}
    %
    This shows that, in the case \( h(0) = \frac{1}{2}  \), the mapping \( t\mapsto \gamma(t) \) does not depend on the value chosen for the ill-defined quantity \( \xi(0) \).
\end{proof}



Using Lemmata \ref{lemma: HO exact integration of parameters} and \ref{lemma: HO change of variables between params and action-angle variables}, we are now able to obtain an easy numerical algorithm which simulates the evolution of bubbles according to the Harmonic Oscillator on a time interval \( [0, T] \). It is described in Algorithm \ref{algo: HO -- exact solve}.
\begin{algorithm}
    \caption{Solving the Harmonic oscillator with Bubbles}
    \label{algo: HO -- exact solve}
    \begin{algorithmic}
        \For{ \( j = 1, \dots, N \) } 
            \Comment{\(j\) denotes a bubble's index}
            \State Use \eqref{eqn: lemma -- HO exact integration of parameters - change of variables} to get the action-angle variables \( (h, a, \xi, \theta) \) at time 0.
            \State Use \eqref{eqn: lemma -- HO exact integration of parameters - update of action-angle variables} to update the variables \( (h, a, \xi, \theta) \) up to time \( T \).
            \State Use \eqref{eqn: lemma -- HO exact integration of parameters - update of parameters - with v} (with the expression for \( \gamma(t) \) replaced by \eqref{eqn: HO -- explicit expression of gamma}) to get the parameters of bubble \( u_j \) at time \( T \).
        \EndFor
    \end{algorithmic}
\end{algorithm}


\begin{remark}[Numerical considerations]
    \label{rmk: HO -- numerical considerations}
    Here are a few remarks about Algorithm \ref{algo: HO -- exact solve}:
    \begin{itemize}
        \item When applying equation \eqref{eqn: lemma -- HO exact integration of parameters - change of variables} to obtain the action-angle variables from the bubbles' parameters, it is advised to use the function \( \arctanTwo(y, x) \) instead of \( \arctan(y/x) \) because it allows to obtain an angle lying in \( (-\pi, \pi] \) instead of \( (-\pi/2, \pi/2] \) by taking into account the signs of both \( x \) and \( y \). This is also the reason why we do not define \( \xi(0) \) by \eqref{proof: HO proof -- definition of xi with k}.
        Moreover most numerical implementations of \( \arctanTwo \) return a finite value for \( \arctanTwo(0, 0) \), which avoids the manual tuning of a numerical threshold to know whether \( a_i(0) \) or \( h(0) \) vanish numerically or not. We recall that in this case the exact value returned does not impact the behavior of \( t\mapsto (L(t), B(t), X(t), \beta(t), \gamma(t)) \).
        \item The algorithm yields an exact solution as soon as the initial data is a sum of bubbles. If not, then the only error committed is the discretization error when approximating the initial condition \( \psi(t=0) \) by the ansatz \eqref{eqn: generic discretization of psi -- ansatz}.
        \item This numerical algorithm does not need any discretization in time nor in space.
        \item The solution obtained is the \emph{exact} solution of the equation \eqref{eqn: cNLS with psi -- linear part} defined on the whole space \( \mathbb{R}^d \), and no numerical boundary conditions are needed.
    \end{itemize}
\end{remark}


