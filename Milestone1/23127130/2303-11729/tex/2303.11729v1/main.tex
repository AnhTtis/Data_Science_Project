\documentclass[prl,aps,amssymb,twocolumn]{revtex4-1}
%\documentclass[prl,twocolumn,superscriptaddress]{revtex4}
\usepackage{amsmath}
\usepackage{amssymb}
\usepackage{amsthm}
\usepackage{amsfonts}
\usepackage{listings}

\usepackage{physics}

\lstloadlanguages{Matlab}
%\usepackage{algorithmic}
\usepackage{enumerate}
\usepackage{latexsym}
%\usepackage[dvips]{graphicx}
\usepackage{color}
%\usepackage{xcolor}
\usepackage{bm}
\usepackage{hyperref}
\hypersetup{
 pdfnewwindow=true, colorlinks=true,
 linkcolor=blue, anchorcolor=blue,
 citecolor=blue, filecolor=blue,
 menucolor=blue, urlcolor=blue}

\newcommand{\beginsupplement}{%
        \setcounter{table}{0}
        \renewcommand{\thetable}{S\arabic{table}}%
        \setcounter{figure}{0}
        \renewcommand{\thefigure}{S\arabic{figure}}%
     }

\usepackage{psfrag}

\usepackage{bm}
%\usepackage[pdftex]{graphics}
\usepackage{graphicx}
\usepackage{subfigure}


%DIF <
%DIF PREAMBLE EXTENSION ADDED BY LATEXDIFF
%DIF UNDERLINE PREAMBLE %DIF PREAMBLE
\RequirePackage[normalem]{ulem} %DIF PREAMBLE
\RequirePackage{color}\definecolor{RED}{rgb}{1,0,0}\definecolor{BLUE}{rgb}{0,0,1} %DIF PREAMBLE
\providecommand{\DIFadd}[1]{{\protect\color{blue}\uwave{#1}}} %DIF PREAMBLE
\providecommand{\DIFdel}[1]{{\protect\color{red}\sout{#1}}}                      %DIF PREAMBLE
%%DIF SAFE PREAMBLE %DIF PREAMBLE
%\providecommand{\DIFaddbegin}{} %DIF PREAMBLE
%\providecommand{\DIFaddend}{} %DIF PREAMBLE
%\providecommand{\DIFdelbegin}{} %DIF PREAMBLE
%\providecommand{\DIFdelend}{} %DIF PREAMBLE
%%DIF FLOATSAFE PREAMBLE %DIF PREAMBLE
%\providecommand{\DIFaddFL}[1]{\DIFadd{#1}} %DIF PREAMBLE
%\providecommand{\DIFdelFL}[1]{\DIFdel{#1}} %DIF PREAMBLE
%\providecommand{\DIFaddbeginFL}{} %DIF PREAMBLE
%\providecommand{\DIFaddendFL}{} %DIF PREAMBLE
%\providecommand{\DIFdelbeginFL}{} %DIF PREAMBLE
%\providecommand{\DIFdelendFL}{} %DIF PREAMBLE
%DIF END PREAMBLE EXTENSION ADDED BY LATEXDIFF

%\newcommand{\beq}{\begin{equation}}
%\newcommand{\eneq}{\end{equation}}
% boldsymbol (requires amsmath)
%\newcommand{\bs}[1]{\boldsymbol{#1}}
%\newcommand{\bal}{\begin{align}}
\newcommand{\imp}{\;\;\Rightarrow\;\;}


\newcommand{\sy}{\sigma_y}
\newcommand{\hatti}{\hat{T}_i}
\newcommand{\hattj}{\hat{T}_j}
\newcommand{\hattone}{\hat{T}_1}
\newcommand{\hatttwo}{\hat{T}_2}
\newcommand{\hattthree}{\hat{T}_3}


% A command for inner product and bras and kets
%\newcommand{\braket}[2]{\left\langle #1 | #2 \right\rangle}
%\newcommand{\bra}[1]{\left\langle#1\right|}
%\newcommand{\ket}[1]{\left|#1\right\rangle}
%\newcommand{\bigket}[1]{\bigl|#1\bigr\rangle}
%\newcommand{\textket}[1]{|#1\rangle}

% Various bracketing commands
\newcommand{\of}[1]{\!\left(#1\right)}
\newcommand{\sqof}[1]{\left[#1\right]}
\newcommand{\cuof}[1]{\left\{#1\right\}}

% insert text in equation
\newcommand{\ins}[1]{\;\;\text{#1}\;\;}

% commutator and anticommutator
%\newcommand{\comm}[2]{\left[#1,#2\right]}
%\newcommand{\anticomm}[2]{\left\{#1,#2\right\}}
% sum on nearest neighbor bonds
\newcommand{\bond}{\left\langle i, j \right\rangle}
%\newcommand{\bondsum}{\sum_{\left\langle i, j \right\rangle}}
\newcommand{\nbond}{\left\langle\left\langle i, j \right\rangle\right\rangle}
% 1/2
\newcommand{\half}{$\frac{1}{2}$ }
% simplifies using the up and down arrows to denote spin
\newcommand{\up}{\uparrow}
\newcommand{\dw}{\downarrow}
\newcommand{\bb}{{\bf b}}
\newcommand{\bk}{{\vb* k}}
\newcommand{\br}{{\vb* r}}
\newcommand{\bkp}{{\vb* k_{\parallel}}}
\newcommand{\bkpsq}{{\vb* k_\parallel^2}}

% Theta function
\newcommand{\tfunc}{\vartheta_1}
% notation for vacuum, an empty set inside a ket
\newcommand{\vac}{\left|\,0\,\right\rangle}
% Absolute value
%\newcommand{\abs}[1]{\left|#1\right|}

% Roman functions for real and imaginary parts
\newcommand{\re}{\mathrm{Re}}
\newcommand{\im}{\mathrm{Im}}

% Sets of up-spin and down-spin locations
\newcommand{\bket}{\left\{z_1 \cdots z_{\num}\right\}}
\newcommand{\wket}{\left\{w_1 \cdots w_{\num}\right\}}

%Expectation values
\newcommand{\expect}[1]{\left\langle#1\right\rangle}

% reference with parenthesis
\newcommand{\pref}[1]{(\ref{#1})}

%for type II article
\newcommand{\I}{\mathrm{\uppercase\expandafter{\romannumeral1}}}
\newcommand{\II}{\mathrm{\uppercase\expandafter{\romannumeral2}}}
\newcommand{\III}{\mathrm{\uppercase\expandafter{\romannumeral3}}}
\newcommand{\IV}{\mathrm{\uppercase\expandafter{\romannumeral4}}}
\newcommand{\bw}[1]{\textcolor{blue}{\uwave{#1}}}
\newcommand{\rs}[1]{\textcolor{red}{\sout{#1}}}

\def \tic{Cu$_3$SbS$_4$}
\def \wsmc{Cu$_2$ZnGeSe$_4$}
\def \ti{Case I}
\def \wsm{Case II}

\def\ie{{\it i.e.},\ }
\def\eg{{\it e.g.}\ }
\def\ea{{\it et al.}}
\input{epsf}

\begin{document}

\tolerance 10000

\newcommand{\vk}{{\bf k}}

\draft

\title{Hybrid Topological Superconductivity and Hinge Majorana Flat Band in Type-II Dirac Semimetals}

%in English titles articles and words like to, on, at etc are always spelled with small letters
%\author{Goodguys}
%\affiliation{Beijing National Laboratory for Condensed Matter Physics,
%and Institute of Physics, Chinese Academy of Sciences, Beijing 100190, China}
%\affiliation{University of Chinese Academy of Sciences, Beijing 100049, China}

\author{Yue Xie}
%\thanks{These authors contributed equally to this work.}
\affiliation{Beijing National Laboratory for Condensed Matter Physics,
and Institute of Physics, Chinese Academy of Sciences, Beijing 100190, China}
\affiliation{University of Chinese Academy of Sciences, Beijing 100049, China}

\author{Zhijun Wang}
\email{wzj@iphy.ac.cn}
\affiliation{Beijing National Laboratory for Condensed Matter Physics,
and Institute of Physics, Chinese Academy of Sciences, Beijing 100190, China}
\affiliation{University of Chinese Academy of Sciences, Beijing 100049, China}

%\pacs{03.67.Mn, 05.30.Pr, 73.43.-f}

\begin{abstract}
Type-II Dirac semimetals (DSMs) have a distinct Fermi surface topology, which allows them to host novel topological superconductivity (TSC) different from type-I DSMs. By performing a low-energy analysis on a generic DSM model, we introduce an inherited pairing through band hybridization, which describes the second-order TSC. In a type-$\II$ DSM model, it is a two-dimensional~(2D) first-order time-reversal invariant topological superconductor for the $k_z = 0/\pi$ slice, whereas for the $k_z$ slices other than 0 or $\pi$, they belong to 2D second-order topological superconductors. 
Our analysis reveals a Majorana flat band on the $z$-directed hinge of the type-II DSM, which penetrates through the whole hinge Brillouin zone. 
These higher-order hinge modes are protected by $C_{4v}$ symmetry and can even host strong stability against finite $C_{4z}$ rotation symmetry-breaking order. We suggest that experimental realization of these findings can be achieved in transition metal dichalcogenides.
\end{abstract}

\maketitle






%\section{Introduction}
\paragraph*{Introduction.---}
Majorana fermions, quasiparticle excitations in topological superconductors, are known for their non-Abelian statistics~\cite{NA1,NA2,NA3,NA4,NA5} and potential use in fault-tolerant quantum computing~\cite{FT1,FT2,FT3,FT4,FT5}. Their unique properties have generated significant interest in realizing Majoranas in various topological materials~\cite{Fu_Kane,Alicea,Qi_Wen,Qi,SrRuO,WS1,WS2}, including type-I DSMs~\cite{Sato_TCSC,Node_vortex,HOTDS}. However, type-II DSMs, potentially hosting novel Majorana states due to its disparate Fermi surface geometry from that of type-I, are lacking of investigations. As the Lorentz-violating velocity tilting the Dirac cone from type-I to type-II, the Fermi surface experiences a Lifshitz transition from a closed sphere to an open pocket~\cite{Type_II}. This transition provides distinct Fermi surface physics and thus potential for TSC. Recently, a class of transition metal dichalcogenides, $X$Te$_2$ ($X=$ Pd, Ni, Ir), has been shown to host type-II Dirac fermions near the Fermi surface~\cite{PdTe2,PtTe2,TMD,NiTe2,DingHong}. Moreover, intrinsic bulk superconductivity has been observed in PdTe$_2$, manipulated NiTe$_2$ and Pd/Pt doped IrTe$_2$~\cite{PdTe2_SC,PdTe2_SC2,NiTe2_SC,NiTe2_SC2,proximity_SC,IrTe2,IrTe2_2}. 
These facts make them promising platforms for studying TSC in type-II DSMs~\cite{PdTe2_SC3}.

Higher-order topology, which manifests as localized modes at $(d$$-$$m)$-dimensional ($m$$\geqslant$$ 2$) boundaries in a $d$-dimensional material~\cite{quadrupole,quadrupolar_semimetal,GuoZhaopeng}, is an exciting area of research that intersects with superconductivity. While numerous theoretical studies have investigated higher-order TSC~\cite{Po_1,Po_2,theory_1,theory_2,theory_3,theory_4,theory_5,theory_6,theory_7,theory_8,theory_9,theory_10,theory_11,Dirac_SC}, only a limited number of realistic models have been proposed in intrinsic materials~\cite{Das_Sarma,TRI_TSC}, by proximity effect~\cite{MKP}, with magnetic configuration~\cite{BHZ_TSC}.

In this letter, we explore the possibility of higher-order TSC in a type-II DSM through a generic model with favored $B_{1u}$ and $B_{2u}$ superconducting (SC) pairing channels. Remarkably, our analysis reveals that although with a $\bk$-independent pairing, the mixing with high-energy bands can lead to an anisotropic boundary pairing field, giving rise to higher-order TSC. Detailed calculations show this model has hybrid TSC. It is a 2D first-order time-reversal invariant topological superconductor for the $k_z = 0/\pi$ slice, whereas for the $k_z$ slices other than 0 or $\pi$, they belong to 2D second-order topological superconductors. Our numerical simulations demonstrate the presence of a hinge Majorana flat band in the type-II DSM, which extends through the entire hinge Brillouin zone (Fig.~\ref{fig1}(b)). Additionally, we investigate the SC nodal structure and the symmetry-breaking instability of the hinge Majorana modes. 
\begin{figure}[!t]
	\centering
	\includegraphics[trim=30 0 0 0, scale=0.25]{fig1_new.pdf}
	\caption{Hinge Majorana states (red lines) in type-$\I$ (a) and type-$\II$ (b) DSMs. Fermi surfaces of normal state is schematically plotted (yellow pockets). Bulk superconducting Dirac points (dots) appear in type-$\I$ ($\mathcal{Q}_t=2$), while full BdG gap appears in type-$\II$ ($\mathcal{Q}_t=0$).}
	\label{fig1}
\end{figure}






%\section{Model Hamiltonian}
\paragraph*{Model Hamiltonian.---}Considering that a DSM is a collection of 2D $k_z$-slices, our minimal model Hamiltonian describes a generic DSM on a tetragonal lattice ($D_{4h}$ symmetry) in Bogoliubov-de Gennes (BdG) redundancy:
	\begin{equation}\label{Hk}\begin{aligned}
	H(\bkp,k_z) &= \left(\begin{array}{cc}
		H_0(\bkp,k_z)-\mu & D  \\
		D^\dagger & \mu-H_0^T(-\bkp,-k_z)   \\
	\end{array}\right),\\
	H_0(\bkp,k_z)&=t_{\II}\cos{k_z}\Gamma_0-t_{b}\left( 2-\cos{k_x}-\cos{k_y} \right)\Gamma_0 \\
&+ t_{\I}\cos{k_z}\Gamma_2-t_{a}\left( 2-\cos{k_x}-\cos{k_y} \right)\Gamma_2 \\
&+ \eta\left( \sin{k_x}\Gamma_5 - \sin{k_y}\Gamma_1 \right) \\
&+ \sin{k_z} [\alpha\left( \cos{k_y}-\cos{k_x} \right)\Gamma_3 + \beta\sin{k_x}\sin{k_y}\Gamma_4],
	\end{aligned}\end{equation}
	\noindent The normal state $H_0$ describes a typical DSM on the basis of total angular momenta $J=\{3/2,\ 1/2\}$ with different parities~\cite{Wang_Na3Bi,Wang_Cd3As2}. Here, $\bkp=(k_x,k_y)^T$, $\Gamma_0$ is $4\times 4$ identity and the anticommuting matrix set $\Gamma_i$ are chosen to be
	\begin{equation}\label{gamma}\begin{aligned}
	&\Gamma_1=\sigma_y\otimes s_0, \quad \Gamma_2=\sigma_z\otimes s_0, \quad
	\Gamma_3=\sigma_x\otimes s_x, \\ &\Gamma_4=\sigma_x\otimes s_y, \quad
	\Gamma_5=\sigma_x\otimes s_z, \quad \Gamma_{ij}=\left[\Gamma_i,\Gamma_j\right]/2i.
	\end{aligned}\end{equation}
\noindent The charge-conjugate operator is $\mathcal{C}=\tau_x\mathcal{K}$. $\tau$, $\sigma$ and s are Pauli matrices describing the particle-hole, orbital and spin degrees of freedom, respectively. The generators of $D_{4h}$ are time reversal ($\mathcal{T}$), inversion ($\mathcal{P}$), mirror ($\mathcal{M}_z$) and four-fold rotation ($C_{4z}$) symmetries as below,
	\begin{equation}\begin{aligned}
	&\mathcal{T}=-is_y\mathcal{K},\quad \mathcal{P}=\sigma_z,\quad
	\mathcal{M}_z=-is_z,\\ &C_{4z}=\exp\left[i(\pi/4)(2+\sigma_z)\otimes s_z\right].
	\end{aligned}\end{equation}
In order to study a good Dirac fermion physics, we first set the chemical potential to be at the level of the Dirac points (\ie $\mu=0$). Particularly, $t_{\II}$ tilts the Dirac cones. If $\vert t_{\II}\vert \textgreater \vert t_{\I}\vert$, Dirac cones are strongly tilted to be of type-$\II$. Otherwise, they are of type-I. Meanwhile, for a fixed-$k_z$ slice, odd (even) number of $\mathcal{PT}$-pairs of FS sheets appear in type-$\II$ (type-$\I$) DSMs (Fig.~\ref{fig2}(a-b)). This FS geometry difference contributes to the TSC crucially.
\begin{table}[b]
\begin{center}
	\begin{tabular}{|c|c|c|c|c|c|c|}\hline
        Irreps. & pairing order & $\ \mathcal{P}\ $ & $C_{4z}$ & $\mathcal{M}_z$ & $\mathcal{M}_x$ & $\mathcal{M}_y$ \\ \hline
        $B_{1u}$ & $i\Delta_1\Gamma_1$ & $-$ & $-$ & $-$ & $-$ & $-$ \\
        \hline
        $B_{2u}$ & $\Delta_2\Gamma_{52}$ & $-$ & $-$ & $-$ & $+$ & $+$ \\
        \hline
    \end{tabular}
    \caption{Symmetries for $B_{1u}$ and $B_{2u}$ pairing channels.}\label{irreps}
\end{center}
\end{table}







\paragraph*{SC pairing channels.---}The orbit-momentum locking nature on the Fermi surface suggests an intra-spin inter-orbital pairing is favored in the type-$\I$ DSM~\cite{Sato_TCSC}. Here, when the Dirac cone is tilted to be of type $\II$, the orbital texture of Fermi surface contour in a fixed $k_z$ slice is shown explicitly in Fig.~\ref{fig2}(c). As is similar to the type-I~\cite{Sato_TCSC}, we take the favored inter-orbital pairing functions,
	\begin{equation}\label{paring}\begin{aligned}
	D=i\Delta_1\Gamma_1+\Delta_2\Gamma_{52}.
	\end{aligned}\end{equation}
\noindent Such pairing can originate from an dominant attractive inter-orbital interaction $H_{int}=-2Vn_1n_2$, with $n_\sigma=\sum_{s=\uparrow,\downarrow}c^\dagger_{s,\sigma}(\br)c_{s,\sigma}(\br)$ ($V>0$)~\cite{TSC_4,TSC_5,TSC_6,TSC_2,TSC_3}. $i\Delta_1\Gamma_1$ and $\Delta_2\Gamma_{52}$ are pairing channels of $B_{1u}$ and $B_{2u}$ irrepresentation of $D_{4h}$, respectively. Their symmetry properties are shown in Table~\ref{irreps}. Eq.~\ref{paring} allows us to consider both channels in a unify framwork.






\paragraph*{First-order TSC for $k_z=0/\pi$.---} 
Due to the odd-parity nature, $\mathcal{P}D\mathcal{P}^T$=$-D$, the TSC can be characterized by the number of FS sheets of the normal state~\cite{TSC_1,TSC_2,TSC_3}. In type-$\II$ DSM, the Lorentz violating kinetic term $t_\II$ integrally shifts the energy bands, making it a first-order time-reversal (TR) invariant 2D topological superconductor for $k_z=0/\pi$. An odd pairs of FS sheets that enclose TR-invariant momenta result in a nontrivial $\mathbb{Z}_2$ phase of class DIII~\cite{AZ_class}. Accordingly, surface helical Majorana states are found on (100) and (010) surfaces, as shown in Fig.~\ref{fig3}(a).
\\
\indent For other 2D $k_z$-slices ($k_z\neq 0/\pi$) without TR symmetry, gaps are introduced to the surface states and trivialize the first-order topology. Nevertheless, they hold 2D second-order TSC, because there is an anisotropic pairing field generated on the boundaries. Next, we will carry out a low-energy analysis to describe the second-order TSC.
\begin{table}[b!]
\begin{center}
	\begin{tabular}{|c|c|c|c|c|c|c|c|c|c|c|}\hline
        Parameters & $\ t_\I\ $ & $t_\II$ & $t_a$ & $t_b$ & $\eta$ & $\alpha$ & $\beta$ & $\Delta_1$ & $\Delta_2$ & $\mu$ \\ \hline
        Type-$\II$ & 1 & 1.5 & 1.5 & 1 & 0.8 & 0.5 & 0.5 & 1.2 & 0.2 & -0.3 \\
        \hline
        Type-$\I$ & 1 & 0 & 1 & 0.5 & 0.5 & 2 & 2 & 0.5 & 0 & -0.2 \\
        \hline
    \end{tabular}
    \caption{Parameter setting for numerical simulations.}\label{parameter}
\end{center}
\end{table}
\begin{figure}[!tb]
	\centering
	\includegraphics[trim=15 0 0 0, scale=0.32]{fig2.pdf}
	\caption{(a-b) Typical in-plane/out-of-plane dispersion of a type-$\I$ (a) or type-$\II$ (b) DSM, with Dirac points locating on $\pm \bk^D=\left(0,0,\pm\frac{\pi}{2}\right)$. Dashed line denotes Fermi level. Total angular momenta are labeled with $J=1/2\ or\ 3/2$. (c) Orbital texture on Fermi surface with fixed-$k_z$ slices for spin-up (left) and spin-down (right) sectors. Arrows on Fermi surfaces indicate the orbit-momentum texture. A Cooper pair between electrons with identical spin always have the antiparallel orbit configuration (red solid double-headed arrow). On the other hand, orbit configuration for inter-spin pairing is $\bk$-dependent (red dashed double-headed arrow). Such orbit-momentum locking is consistent with a constant pairing function only in the intra-spin inter-orbital channel but not conventional s-wave channel. (d) Bulk BdG dispersion without ($D=0$, left panel) and with ($D\neq 0$, right panel) SC pairing. SC DPs annihilate each other to create full BdG bulk gap, with $\mathcal{Q}(\II)=\{\mathcal{Q}_{1/2},\mathcal{Q}_{3/2}\}=(1,-1)$ and $\mathcal{Q}_t(\II)=0$.} %(a)(b) For a fix-$k_z$ slice, normal state dispersion with and without OM locking term for solid and dashed lines, respectively. (a)$\tilde{\mu}_{\I a}$ and $\tilde{\mu}_{\II a}$ for case $\I$a and $\II$a. (b)$\tilde{\mu}_{\I b}$ and $\tilde{\mu}_{\II b}$ for case $\I$b and $\II$b. Effeactive chemical potential $\tilde{\mu}_{\II a}$ and $\tilde{\mu}_{\I b}$ intersect with odd number of effective Fermi surface sheets, while $\tilde{\mu}_{\I a}$ and $\tilde{\mu}_{\II b}$ with even. (d)Yellow dots denote gapless edges, thus the mass sign transition points.}
	\label{fig2}
\end{figure}








\paragraph*{Low-energy analysis.---} In a system with helical texture, the low energy downfolding yields effective typical topological superconductor model to decribe first-order Majoranas~\cite{Fu_Kane,Alicea}. Here we will demonstrate that along with additional self-energy correction, an effective pairing field is introduced on the first-order boundaries and second-order Majorana zero modes (SOMZMs) emerge, ensured by a discrete rotation symmetry.

For simplicity, we would investigate each $k_z$-fixed slice individually. A continuous 2D-slice Hamiltonian $H_0^{k_z}(\bkp)$ is defined as $H_0^{k_z}(\bkp)\equiv  H_0(\bkp,k_z)|_{\bkp \rightarrow (0,0)}$. To $\mathcal{O}(\bkpsq)$, 
	\begin{equation}\begin{aligned}\label{continuous}
	 H_0^{k_z}(\bkp) &=\tilde{t}_{\I}\Gamma_2+\tilde{t}_{\II}\Gamma_0 - (k_x^2+k_y^2)(t_{a}\Gamma_2+t_{b}\Gamma_0)/2 \\&+ \eta(k_x\Gamma_5-k_y\Gamma_1) + V(\bkp),\\
	V(\bkp)&=\tilde{\alpha}(k_x^2-k_y^2)\Gamma_3/2+\tilde{\beta}k_xk_y\Gamma_4.
	\end{aligned}\end{equation}
\noindent Here, $\tilde{t}_{\II,\I}$=$t_{\II,\I}\cos k_z$, $\tilde{\alpha}$=$\alpha\sin{k_z}$ and $\tilde{\beta}$=$\beta\sin{k_z}$. The spin-orbit coupling (SOC) term $V(\bkp)$ explicitly breaks $\mathcal{M}_z$, $\mathcal{T}$, $\mathcal{P}$ and $\mathcal{C}$ in the 2D slice, while preserves joint symmetries $\mathcal{M}_z\mathcal{P}$ and $\mathcal{M}_z\mathcal{C}$.
\\
\indent We focus on the type-$\II$ and take $\tilde{t}_{\I,\II}>0$ without loss of generality. When orbit-momentum locking is strong enough ($\eta^2$$>$$\tilde{t}_\I\vert t_a$$-$$t_b\vert$) and assuming weak pairing assumption~\cite{TSC_1,TSC_2,TSC_3} ($k_z \simeq 0,\pm\pi$), a pair of FS sheets can be very well described by a low-energy analysis, which will tell how the effective pairing becomes $\bk$-dependent. More specifically, first apply unitary transformation $U(\bkp)\equiv(\Psi_\bk,\sigma_x\Psi^*_{-\bk})$, with
	\begin{equation}\begin{aligned}
	\Psi_\bk &= \left(\begin{array}{cccc} \varphi_\bk & -is_y\varphi^*_{-\bk} & \tau_x\varphi^*_{-\bk} & -i\tau_xs_y\varphi_{\bk} \end{array}\right),\\
	\varphi_\bk &= \left(\begin{array}{cccccccc} 
		A_\bk & 0 & B_\bk & 0 & 0 & 0 & 0 & 0 
	\end{array}\right)^T/\sqrt{\mathcal{N}_\bk}.
	\end{aligned}\end{equation}
\noindent Here, $B_\bk=2\tilde{t}_{\I}-t_{a}\bkpsq+\sqrt{(-2\tilde{t}_{\I}+t_{a}\bkpsq)^2+4\eta^2\bkpsq}$ and $A_\bk=-2\eta(k_x+ik_y)$, with the normalization factor $\mathcal{N}_\bk$. Now, the first-order perturbated Hamiltonian is
	\begin{equation}\begin{aligned}
	H_{\mathrm{eff}}(\bkp)
	= U^\dagger H^{k_z}(\bkp) U
	= \left(\begin{array}{cc}
		H_l(\bkp) & H_s(\bkp) \\
		H^\dagger_s(\bkp) & H_h(\bkp)
	\end{array}\right).
	\end{aligned}\end{equation}
\noindent Here, the effective low-energy Hamiltonian $H_l(\bkp) = ( -M + T\bkpsq )\tilde{\Gamma}_2 - (\tilde{\Delta}_1k_x+\tilde{\Delta}_2k_y)\tilde{\Gamma}_5 - (\tilde{\Delta}_1k_y-\tilde{\Delta}_2k_x)\tilde{\Gamma}_1$, with $M=\tilde{t}_{\I}-\tilde{t}_{\II}$, $T=\left(t_{a}-t_{b}-\eta^2/\tilde{t}_{\I}\right)/2$ and $\tilde{\Delta}_{1,2}=\eta\Delta_{1,2}/\tilde{t}_{\I}$. Low-energy space is spanned by $\tilde{\tau}\otimes\tilde{s}$, respectively describing pseudo-particle-hole and -spin degrees of freedom. $\tilde{\Gamma}$s are similar as in Eq.~\ref{gamma}. The low-energy physics is clearly a TR-invariant $p_x$+$ip_y$-like model~\cite{pxpy}, originating from both orbital texture and inter-orbital pairing. By Fu-Kane criterion~\cite{Fu_Kane_criterion}, edge-localized states always exist for there is one pair of FS sheets in a type-II DSM. For the TR-invaiant slice ($k_z=0/\pi$), nontrivial $\mathbb{Z}_2$ phase ensures such localized states to be helical Majorana states.
\\
\indent For $k_z\neq 0/\pi$ slices, however, the SOC term $V(\bkp)\neq0$ and it becomes essential to introduce an anisotropic pairing field to the edge states. More specifically, one can perform a higher-order perturbation by introducing scattering with high-energy bands $H_h$ through Green's function $G(\bkp,\omega)=( \omega-H_{\mathrm{eff}})^{-1}$. The self-energy correction to $H_l$ is $\Sigma(\bkp,\omega)=H_s\left(\omega-H_h\right)^{-1}H^\dagger_s$. Choose $\omega=0$ as an approximation, the corrected low-energy bands becomes,
	\begin{equation}\begin{aligned}
	\tilde{H}_l(\bkp) = H_l(\bkp)+\Sigma_0(\bkp).
	\end{aligned}\end{equation}
\noindent Here, $\Sigma_0(\bkp)\equiv\Sigma(\bkp,0)=-\left[ \gamma_1(k_x^2-k_y^2)-2\gamma_2k_xk_y \right]\tilde{\Gamma}_3$ is anti-diagonal, with $\gamma_1=\tilde{\alpha}\Delta_1/(\tilde{t}_{\I}+\tilde{t}_{\II})$ and $\gamma_2=\tilde{\beta}\Delta_2/(\tilde{t}_{\I}+\tilde{t}_{\II})$. One can see that the $\bk$-independent pairing $D$ enters into the low-energy bands mediating through SOC as an \textit{inherited pairing filed} $\Sigma_0(\bkp)$, \ie an effective pairing inherits the functional distribution of the SOC field, and becomes distributed. Interestingly, $B_{1u}$ and $B_{2u}$ pairing channels enter the low-energy bands separately through $\alpha$ and $\beta$ terms.
\begin{figure}[!t]
	\centering
	\includegraphics[scale=0.27]{fig3.pdf}
	\caption{(a) Spectrum of the $N_x=30$ slab model with open boundary in $x$ direction. Surface helical Majorana states is highlighted in blue. (b) Energy dispersion in a wire geometry along $k_z$ with open boundary along both $x$ and $y$ (20 lattice sites along each direction). The zero modes inside the surface gap gather to form the intact hinge Majorana flat bands (highlighted in red), linking the projections of surface cones. Upper-right inset: total wave function distribution of the four-degenerate SOMZMs of a fixed $k_z=\pi/4$ slice (yellow cross in (b)). The type-$\II$ parameter setting is in Table~\ref{parameter}.}%corner Majorona state in type II regime, wave function of corner Majorona state in type II regime, Surface and corner state of the 2D model. Schematic plot around kc type I, schematic plot around kc type II, Boundary states of diffenrent kinds. }
	\label{fig3}
\end{figure}






\paragraph*{Second-order TSC and Corner Majorana zero modes for $k_z\neq 0/\pi$.---}To visualize the anisotropy of the mass field on edge states, we first notice that any edge termination of a 2D silce can be described as a tangent line of the unit circle~\cite{33_from_Das,54_from_Das}, as shown in Fig.~\ref{fig4}(a). An arbitrary edge $\mathcal{L}(\theta)$ can be uniquely labeled by its unit normal vector defined as $\textbf{n}_{\mathcal{L}}=\left(\cos\theta,\sin\theta\right)^T$.  To solve edge dispersion for $\mathcal{L}(\theta)$, we apply Euler rotation around $z$ axis $R_Z(-\theta)$ to local coordinates, \ie $\bk'\equiv\left(k_1,k_2\right)^T=R_Z(-\theta)\left(k_x,k_y\right)^T$. Noticing that $k_1=\textbf{n}_{\mathcal{L}}\cdot\bkp$, now one can solve edge theory on arbitrary $\mathcal{L}(\theta)$ by imposing open boundary conditions on $\tilde{H}_l(\bk')$ along the $k_1$ direction. Then taking $k_2$ and $\Sigma_0(\bk')$ as perturbations, we extract the edge dispersion of $\mathcal{L}(\theta)$ (see details in the Supplemental Material (SM)),
	\begin{equation}\label{edge}\begin{aligned}
	&H_{\mathcal{L}}(k_2,\theta) = \vert\Delta\vert k_2\varsigma_z + m_{\mathrm{eff}}(\theta)\varsigma_y.
	\end{aligned}\end{equation}
\noindent $\varsigma$ is pseudospin localized to the edges. $\vert \Delta \vert = \sqrt{\tilde{\Delta}_1^2+\tilde{\Delta}_2^2}$. The effective mass field on edge $\mathcal{L}(\theta)$ is $m_{\mathrm{eff}}(\theta) \propto \tilde{\alpha}\Delta_1\cos{2\theta}-\tilde{\beta}\Delta_2\sin{2\theta}$ (Explicitly in the SM, Eq.~S6). This is one of our main results that the anisotropy of edge mass field originates from the inherited pairing $\Sigma_0(\bkp)$. Notably, the sign flip of the edge mass field between $\mathcal{L}(\theta)$ and $\mathcal{L}(\theta+\pi/2)$ is protected by four-fold rotation, \ie $m(\theta+\pi/2)=-m(\theta)$.
\\
\indent The angular anisotropy of $m_{\mathrm{eff}}(\theta)$ enables a nontrivial situation where it flips its sign while crossing a criticle edge $\mathcal{L}(\theta_c)$. Mathematically, when
	\begin{equation}\label{condition}\begin{aligned}
	m(\theta_n)m(\theta_{n+1})<0,
	\end{aligned}\end{equation}
\noindent with $\theta_n=n\pi/2$, $n=0,1,2,3$, the four corners of a 2D nanosheet bind four-fold degenerate SOMZMs.
\noindent When four-fold rotation persists, Eq.~\ref{condition} draws a trivial condition that $\alpha\neq0$. Numerical study of a $k_z=\pi/4$ lattice model of a type-$\II$ DSM comfirms such fourfold degenerate SOMZMs, which are exponentially localized to the corners, as shown in the inset of Fig.~\ref{fig3}(b).









\paragraph*{SC nodal structure and intact hinge Majorana flat bands.---}Next, we will discuss the SC nodal structure in a generic DSM under BdG redundancy. Around Dirac point (DP; $(0,0,k^D_z)$), we have $H_0^D(\bk)=(- t_\II k_z-\mu)\Gamma_0 - t_\I k_z\Gamma_2 + \eta( k_x\Gamma_5 - k_y\Gamma_1 )$. With $B_{1u}$ pairing channel, band crossings appear on $k_z$ axis when $(t_\II/t_\I)^2$$<$$1+(\mu/\Delta_1)^2$. This shows that band crossings always exist in type-$\I$ region, and full BdG gap appears only in type-$\II$ region when DPs are closed enough to the FS (\eg NiTe$_2$/IrTe$_2$, 20 meV above the FS~\cite{NiTe2,TMD,DingHong}).
\\
\indent Odd-$C_{4z}$ and odd-parity pairing nature (Table~\ref{irreps}) make $B_{1u}$ and $B_{2u}$ channels carry an angular momentum $\Delta J=2$ and flip the parity, so that the crossing of a $\mathcal{P}\mathcal{T}$-pair of normal-state bands with its own charge-conjugate counterparts along $\Gamma$-$Z$ will be superconducting Dirac points (SC DPs), protected by four-fold rotation $C_{4z}$. With joint symmetry $\mathcal{P}\mathcal{T}$ and weak pairing assumption presented, the number of the SC DPs with monopole charge $\mathcal{Q}_J/\abs{\mathcal{Q}_J}$ can be characterized by a $\mathbb{Z}$-value $\mathcal{Q}_J$,
	\begin{equation}\begin{aligned}\label{monopole}
	\mathcal{Q}_J=(1-J)(N_J^\Gamma-N_J^Z).
	\end{aligned}\end{equation}
\noindent $N_J^{\Gamma(Z)}$ is the number of occupied normal-state bands  at $\Gamma(Z)$ with total angular momentum $J$. Hence in our model, $\mathcal{Q}=\{\mathcal{Q}_{1/2},\mathcal{Q}_{3/2}\}$ characterizes the SC nodal structure. As shown in Fig.~\ref{fig1}(a), in type-$\I$ DSM, Fermi surfaces (yellow spheres) are gapped out by SC pairing with two SC DPs left (black dots), which is characterized by $\mathcal{Q}(\I)=\{1,1\}$. The number of total stable SC DPs equals to $\mathcal{Q}_t=\sum_{J}\mathcal{Q}_J$. Here, $\mathcal{Q}_t(\I)=2$ for type~$\I$ \cite{Sato_TCSC}. Meanwhile, the low-energy analysis shows that $k_z\simeq0/\pi$ slices are trivial SC in type-$\I$ DSM (due to even pair of FS sheets). Therefore, Majorana hinge bands appear between the projection of the bulk SC DPs (see Fig.~\ref{fig1}(a)).
\\
\indent As $t_\II$ tilting the normal DPs to be of type-$\II$, one SC DP escapes away and another opposite-charged one comes out. Two unstable monopoles present, charactered by $\mathcal{Q}(\II)=\{1,-1\}$, with $\mathcal{Q}_t(\II)=0$. Hence, they annihilate each other to form a full gap. It is shown in Fig.~\ref{fig1}(b) and specifically in Fig.~\ref{fig2}(d). Together with the knowledge of SC nodal structure and the low-energy analysis, we find in type-$\II$ DSMs that, an clean and intact hinge Majorana flat band penetrates through the whole hinge Brillouin zone and links the projection of the surface helical Majorana cones at TR-invariant points. Numerical calculation using the original TR-invariant BdG lattice Hamiltonian of Eq.~\ref{Hk} is shown in Fig.~\ref{fig3}(b). The intact hinge Majorana flat bands are clearly observed. Its topological characterization using nested Wilson loop method is discussed in the SM.
\begin{figure}[!tb]
	\centering
	\includegraphics[scale=0.37]{fig4.pdf}
	\caption{(a) Schematic plot of a circular geometry with arbitrary edge $\mathcal{L}(\theta)$ (purple tangent line). In the red (blue) region, the effective edge mass $m_{\mathrm{eff}}(\theta)$ is positive (negative). The yellow dots denote the SOMZMs. Against $C_{4z}$ instability, they will move along the circle and finally annihilate at the edges $\mathcal{L}(\theta=0,\pi)$ or $\mathcal{L}(\theta=\frac{\pi}{2},\frac{3\pi}{2})$. (b) Evolution of the $x$- and $y$-edge gaps of the $k_z=\pi/4$ slice against $C_{4z}$ instability. The lines use Eq.~S6-S7. Dots are for numerical calculations on the full lattice model. The yellow region denotes the nontrivial second-order TSC phase with the presence of corner Majorana zero modes.}
	\label{fig4}
\end{figure}








%\section{Stability of higher-order MBSs}
\paragraph*{Stability of Majorana hinge modes.---}Unconventional superconductivity is always accompanied with other symmetry-breaking orders, such as nematicity, spin or charge density wave, etc~\cite{IrTe2,IrTe2_2}. These symmetry lowering orders can stablize the SC phase by increasing the condensation energy, and always break discrete rotation symmetries. Therefore, we consider the stability of the SOMZMs against $C_{4z}$ breaking in the 2D model.
\\
\indent  We phenomenologically add $\delta\sin{k_z}\Gamma_3$ to Eq.~\ref{Hk}, which explicitly breaks $C_{4z}$ to $C_{2z}$. This results in a constant tensor $m_\delta\varsigma_y$ to the edge dispersion of Eq.~\ref{edge}, with $m_\delta\propto\delta\Delta_1\sin{k_z}$ (Explicitly in the SM, Eq.~S7). For a fixed-$k_z$ slice, $m_\delta$ endows any edge $\mathcal{L}(\theta)$ with an extra constant gap, coupling to the anisotropic one $m_{\mathrm{eff}}(\theta)$. Considering only $B_{1u}$ pairing, as $\delta$ is turned on, $C_{4z}$ is broken and mirror symmetries $\mathcal{M}_x$ and $\mathcal{M}_y$ drive two pairs of the SOMZMs (yellow dots in Fig.~\ref{fig4}(a)) to move towards and finally annihilate each other at the edges $\mathcal{L}(\theta=0,\pi)$ or $\mathcal{L}(\theta=\frac{\pi}{2},\frac{3\pi}{2})$. Eventually, Eq.~\ref{condition} no longer holds and the slice becomes trivial SC. Fig.~\ref{fig4}(b) shows energy gaps of $x$- and $y$-edges for $k_z=\pi/4$ slice as a function of $\delta$ (in the unit of $\Delta_1$). Yellow shaded block shows the higher-order nontrivial region. The match of analytical and numerical datas in the behavior of edge gaps against $C_{4z}$ instability shows the availability of our low-energy analysis for the higher-order topology. According to the expressions of $m_\mathrm{eff}(\theta)$ and $m_\delta$ (Eq.~S6,~S7 in the SM), the range of $C_{4z}$ stability linearly depends on the strength of SOC and SC pairing $\alpha\Delta_1$. A wide range of hinge Majorana stability can be manipulated by enhancing either SOC or SC pairing.









%\section{Summary and material realization}
\paragraph*{Discussion.---}With favored inter-orbital pairing, we find that a type-$\II$ DSM is a mix-order topological superconductor. There are first-order TSC in $k_z=0/\pi$ slice and second-order TSC in $k_z\neq 0/\pi$ slices. Particularly in $k_z\neq 0/\pi$ slices, an anisotropic edge mass field binds SOMZMs to the corners at which exists a mass sign flip. Meanwhile, the SC nodal structure analysis shows a full BdG bulk gap. Therefore, there are intact and very observable hinge Majorana flat bands penetrating through the whole hinge Brillouin zone.
\\
\indent Although the $X$Te$_2$ lattice is hexagonal, it is shown that the type-$\II$ DPs in $X$Te$_2$ are also described under the basis $\ket{J=3/2,1/2}$~\cite{TMD,NiTe2}. Therefore, the continuous model for the point group $D_{3d}$ is explicitly the same as the 2D Hamiltonian $H_0^{k_z}(\bkp)$ (Eq.~\ref{continuous}) with constrain $\beta=-\alpha$. Through favored inter-orbital $E_u$ pairing channel, due to the similar inherited pairing and SC nodal structure, hinge Majorana flat bands will present since they hold strong stability against rotation symmetry breaking. In the meantime, we have realized that in Refs.~\cite{s_F_p,s_p,s_p_Experiment}, it is shown that in an $s$-wave$-$ferromagnetic-sheet$-$$p$-wave tri-junction, the low $T_c$ of an odd-parity SC can even be raised up to 1.5 times. Therefore, we suggest that the intact hinge Majorana flat bands demonstrated in this letter is experimentally available in $X$Te$_2$ materials.











\ \\
\noindent \textbf{Acknowledgments}
This work was supported by the National Natural Science Foundation of China (Grants No. 11974395, No. 12188101 and No. 52188101), the Strategic Priority Research Program of Chinese Academy of Sciences (Grant No. XDB33000000), and the Center for Materials Genome.



\begin{thebibliography}{71}%
\makeatletter
\providecommand \@ifxundefined [1]{%
 \@ifx{#1\undefined}
}%
\providecommand \@ifnum [1]{%
 \ifnum #1\expandafter \@firstoftwo
 \else \expandafter \@secondoftwo
 \fi
}%
\providecommand \@ifx [1]{%
 \ifx #1\expandafter \@firstoftwo
 \else \expandafter \@secondoftwo
 \fi
}%
\providecommand \natexlab [1]{#1}%
\providecommand \enquote  [1]{``#1''}%
\providecommand \bibnamefont  [1]{#1}%
\providecommand \bibfnamefont [1]{#1}%
\providecommand \citenamefont [1]{#1}%
\providecommand \href@noop [0]{\@secondoftwo}%
\providecommand \href [0]{\begingroup \@sanitize@url \@href}%
\providecommand \@href[1]{\@@startlink{#1}\@@href}%
\providecommand \@@href[1]{\endgroup#1\@@endlink}%
\providecommand \@sanitize@url [0]{\catcode `\\12\catcode `\$12\catcode
  `\&12\catcode `\#12\catcode `\^12\catcode `\_12\catcode `\%12\relax}%
\providecommand \@@startlink[1]{}%
\providecommand \@@endlink[0]{}%
\providecommand \url  [0]{\begingroup\@sanitize@url \@url }%
\providecommand \@url [1]{\endgroup\@href {#1}{\urlprefix }}%
\providecommand \urlprefix  [0]{URL }%
\providecommand \Eprint [0]{\href }%
\providecommand \doibase [0]{http://dx.doi.org/}%
\providecommand \selectlanguage [0]{\@gobble}%
\providecommand \bibinfo  [0]{\@secondoftwo}%
\providecommand \bibfield  [0]{\@secondoftwo}%
\providecommand \translation [1]{[#1]}%
\providecommand \BibitemOpen [0]{}%
\providecommand \bibitemStop [0]{}%
\providecommand \bibitemNoStop [0]{.\EOS\space}%
\providecommand \EOS [0]{\spacefactor3000\relax}%
\providecommand \BibitemShut  [1]{\csname bibitem#1\endcsname}%
\let\auto@bib@innerbib\@empty
%</preamble>
\bibitem [{\citenamefont {Leinaas}\ and\ \citenamefont {Myrheim}(1977)}]{NA1}%
  \BibitemOpen
  \bibfield  {author} {\bibinfo {author} {\bibfnamefont {J.~M.}\ \bibnamefont
  {Leinaas}}\ and\ \bibinfo {author} {\bibfnamefont {J.}~\bibnamefont
  {Myrheim}},\ }\href@noop {} {\bibfield  {journal} {\bibinfo  {journal} {Il
  Nuovo Cimento B (1971-1996)}\ }\textbf {\bibinfo {volume} {37}},\ \bibinfo
  {pages} {1} (\bibinfo {year} {1977})}\BibitemShut {NoStop}%
\bibitem [{\citenamefont {Fredenhagen}\ \emph {et~al.}(1989)\citenamefont
  {Fredenhagen}, \citenamefont {Rehren},\ and\ \citenamefont {Schroer}}]{NA2}%
  \BibitemOpen
  \bibfield  {author} {\bibinfo {author} {\bibfnamefont {K.}~\bibnamefont
  {Fredenhagen}}, \bibinfo {author} {\bibfnamefont {K.-H.}\ \bibnamefont
  {Rehren}}, \ and\ \bibinfo {author} {\bibfnamefont {B.}~\bibnamefont
  {Schroer}},\ }\href@noop {} {\bibfield  {journal} {\bibinfo  {journal}
  {Communications in Mathematical Physics}\ }\textbf {\bibinfo {volume}
  {125}},\ \bibinfo {pages} {201} (\bibinfo {year} {1989})}\BibitemShut
  {NoStop}%
\bibitem [{\citenamefont {Fr{\"o}hlich}\ and\ \citenamefont
  {Gabbiani}(1990)}]{NA3}%
  \BibitemOpen
  \bibfield  {author} {\bibinfo {author} {\bibfnamefont {J.}~\bibnamefont
  {Fr{\"o}hlich}}\ and\ \bibinfo {author} {\bibfnamefont {F.}~\bibnamefont
  {Gabbiani}},\ }\href@noop {} {\bibfield  {journal} {\bibinfo  {journal}
  {Reviews in Mathematical Physics}\ }\textbf {\bibinfo {volume} {2}},\
  \bibinfo {pages} {251} (\bibinfo {year} {1990})}\BibitemShut {NoStop}%
\bibitem [{\citenamefont {Read}\ and\ \citenamefont {Green}(2000)}]{NA4}%
  \BibitemOpen
  \bibfield  {author} {\bibinfo {author} {\bibfnamefont {N.}~\bibnamefont
  {Read}}\ and\ \bibinfo {author} {\bibfnamefont {D.}~\bibnamefont {Green}},\
  }\href {\doibase 10.1103/PhysRevB.61.10267} {\bibfield  {journal} {\bibinfo
  {journal} {Phys. Rev. B}\ }\textbf {\bibinfo {volume} {61}},\ \bibinfo
  {pages} {10267} (\bibinfo {year} {2000})}\BibitemShut {NoStop}%
\bibitem [{\citenamefont {Ivanov}(2001)}]{NA5}%
  \BibitemOpen
  \bibfield  {author} {\bibinfo {author} {\bibfnamefont {D.~A.}\ \bibnamefont
  {Ivanov}},\ }\href {\doibase 10.1103/PhysRevLett.86.268} {\bibfield
  {journal} {\bibinfo  {journal} {Phys. Rev. Lett.}\ }\textbf {\bibinfo
  {volume} {86}},\ \bibinfo {pages} {268} (\bibinfo {year} {2001})}\BibitemShut
  {NoStop}%
\bibitem [{\citenamefont {Kitaev}(2003)}]{FT1}%
  \BibitemOpen
  \bibfield  {author} {\bibinfo {author} {\bibfnamefont {A.}~\bibnamefont
  {Kitaev}},\ }\href {\doibase https://doi.org/10.1016/S0003-4916(02)00018-0}
  {\bibfield  {journal} {\bibinfo  {journal} {Annals of Physics}\ }\textbf
  {\bibinfo {volume} {303}},\ \bibinfo {pages} {2} (\bibinfo {year}
  {2003})}\BibitemShut {NoStop}%
\bibitem [{\citenamefont {Freedman}(2002)}]{FT2}%
  \BibitemOpen
  \bibfield  {author} {\bibinfo {author} {\bibfnamefont {M.~H.}\ \bibnamefont
  {Freedman}},\ }\href {\doibase https://doi.org/10.1090/S0273-0979-02-00964-3}
  {\bibfield  {journal} {\bibinfo  {journal} {Bull. Am. Math. Soc.}\ }\textbf
  {\bibinfo {volume} {40}},\ \bibinfo {pages} {31} (\bibinfo {year}
  {2002})}\BibitemShut {NoStop}%
\bibitem [{\citenamefont {Das~Sarma}\ \emph {et~al.}(2005)\citenamefont
  {Das~Sarma}, \citenamefont {Freedman},\ and\ \citenamefont {Nayak}}]{FT3}%
  \BibitemOpen
  \bibfield  {author} {\bibinfo {author} {\bibfnamefont {S.}~\bibnamefont
  {Das~Sarma}}, \bibinfo {author} {\bibfnamefont {M.}~\bibnamefont {Freedman}},
  \ and\ \bibinfo {author} {\bibfnamefont {C.}~\bibnamefont {Nayak}},\ }\href
  {\doibase 10.1103/PhysRevLett.94.166802} {\bibfield  {journal} {\bibinfo
  {journal} {Phys. Rev. Lett.}\ }\textbf {\bibinfo {volume} {94}},\ \bibinfo
  {pages} {166802} (\bibinfo {year} {2005})}\BibitemShut {NoStop}%
\bibitem [{\citenamefont {Bonderson}\ \emph {et~al.}(2008)\citenamefont
  {Bonderson}, \citenamefont {Freedman},\ and\ \citenamefont {Nayak}}]{FT4}%
  \BibitemOpen
  \bibfield  {author} {\bibinfo {author} {\bibfnamefont {P.}~\bibnamefont
  {Bonderson}}, \bibinfo {author} {\bibfnamefont {M.}~\bibnamefont {Freedman}},
  \ and\ \bibinfo {author} {\bibfnamefont {C.}~\bibnamefont {Nayak}},\ }\href
  {\doibase 10.1103/PhysRevLett.101.010501} {\bibfield  {journal} {\bibinfo
  {journal} {Phys. Rev. Lett.}\ }\textbf {\bibinfo {volume} {101}},\ \bibinfo
  {pages} {010501} (\bibinfo {year} {2008})}\BibitemShut {NoStop}%
\bibitem [{\citenamefont {Nayak}\ \emph {et~al.}(2008)\citenamefont {Nayak},
  \citenamefont {Simon}, \citenamefont {Stern}, \citenamefont {Freedman},\ and\
  \citenamefont {Das~Sarma}}]{FT5}%
  \BibitemOpen
  \bibfield  {author} {\bibinfo {author} {\bibfnamefont {C.}~\bibnamefont
  {Nayak}}, \bibinfo {author} {\bibfnamefont {S.~H.}\ \bibnamefont {Simon}},
  \bibinfo {author} {\bibfnamefont {A.}~\bibnamefont {Stern}}, \bibinfo
  {author} {\bibfnamefont {M.}~\bibnamefont {Freedman}}, \ and\ \bibinfo
  {author} {\bibfnamefont {S.}~\bibnamefont {Das~Sarma}},\ }\href {\doibase
  10.1103/RevModPhys.80.1083} {\bibfield  {journal} {\bibinfo  {journal} {Rev.
  Mod. Phys.}\ }\textbf {\bibinfo {volume} {80}},\ \bibinfo {pages} {1083}
  (\bibinfo {year} {2008})}\BibitemShut {NoStop}%
\bibitem [{\citenamefont {Fu}\ and\ \citenamefont {Kane}(2008)}]{Fu_Kane}%
  \BibitemOpen
  \bibfield  {author} {\bibinfo {author} {\bibfnamefont {L.}~\bibnamefont
  {Fu}}\ and\ \bibinfo {author} {\bibfnamefont {C.~L.}\ \bibnamefont {Kane}},\
  }\href {\doibase 10.1103/PhysRevLett.100.096407} {\bibfield  {journal}
  {\bibinfo  {journal} {Phys. Rev. Lett.}\ }\textbf {\bibinfo {volume} {100}},\
  \bibinfo {pages} {096407} (\bibinfo {year} {2008})}\BibitemShut {NoStop}%
\bibitem [{\citenamefont {Alicea}(2011)}]{Alicea}%
  \BibitemOpen
  \bibfield  {author} {\bibinfo {author} {\bibfnamefont {J.}~\bibnamefont
  {Alicea}},\ }\href {\doibase 10.1038/nphys1915} {\bibfield  {journal}
  {\bibinfo  {journal} {Nature Physics}\ }\textbf {\bibinfo {volume} {7}},\
  \bibinfo {pages} {412–417} (\bibinfo {year} {2011})}\BibitemShut {NoStop}%
\bibitem [{\citenamefont {Qi}\ \emph {et~al.}(2010{\natexlab{a}})\citenamefont
  {Qi}, \citenamefont {Hughes},\ and\ \citenamefont {Zhang}}]{Qi_Wen}%
  \BibitemOpen
  \bibfield  {author} {\bibinfo {author} {\bibfnamefont {X.-L.}\ \bibnamefont
  {Qi}}, \bibinfo {author} {\bibfnamefont {T.~L.}\ \bibnamefont {Hughes}}, \
  and\ \bibinfo {author} {\bibfnamefont {S.-C.}\ \bibnamefont {Zhang}},\ }\href
  {\doibase 10.1103/PhysRevB.81.134508} {\bibfield  {journal} {\bibinfo
  {journal} {Phys. Rev. B}\ }\textbf {\bibinfo {volume} {81}},\ \bibinfo
  {pages} {134508} (\bibinfo {year} {2010}{\natexlab{a}})}\BibitemShut
  {NoStop}%
\bibitem [{\citenamefont {Hosur}\ \emph {et~al.}(2014)\citenamefont {Hosur},
  \citenamefont {Dai}, \citenamefont {Fang},\ and\ \citenamefont {Qi}}]{Qi}%
  \BibitemOpen
  \bibfield  {author} {\bibinfo {author} {\bibfnamefont {P.}~\bibnamefont
  {Hosur}}, \bibinfo {author} {\bibfnamefont {X.}~\bibnamefont {Dai}}, \bibinfo
  {author} {\bibfnamefont {Z.}~\bibnamefont {Fang}}, \ and\ \bibinfo {author}
  {\bibfnamefont {X.-L.}\ \bibnamefont {Qi}},\ }\href {\doibase
  10.1103/PhysRevB.90.045130} {\bibfield  {journal} {\bibinfo  {journal} {Phys.
  Rev. B}\ }\textbf {\bibinfo {volume} {90}},\ \bibinfo {pages} {045130}
  (\bibinfo {year} {2014})}\BibitemShut {NoStop}%
\bibitem [{\citenamefont {Ueno}\ \emph {et~al.}(2013)\citenamefont {Ueno},
  \citenamefont {Yamakage}, \citenamefont {Tanaka},\ and\ \citenamefont
  {Sato}}]{SrRuO}%
  \BibitemOpen
  \bibfield  {author} {\bibinfo {author} {\bibfnamefont {Y.}~\bibnamefont
  {Ueno}}, \bibinfo {author} {\bibfnamefont {A.}~\bibnamefont {Yamakage}},
  \bibinfo {author} {\bibfnamefont {Y.}~\bibnamefont {Tanaka}}, \ and\ \bibinfo
  {author} {\bibfnamefont {M.}~\bibnamefont {Sato}},\ }\href {\doibase
  10.1103/PhysRevLett.111.087002} {\bibfield  {journal} {\bibinfo  {journal}
  {Phys. Rev. Lett.}\ }\textbf {\bibinfo {volume} {111}},\ \bibinfo {pages}
  {087002} (\bibinfo {year} {2013})}\BibitemShut {NoStop}%
\bibitem [{\citenamefont {Cho}\ \emph {et~al.}(2012)\citenamefont {Cho},
  \citenamefont {Bardarson}, \citenamefont {Lu},\ and\ \citenamefont
  {Moore}}]{WS1}%
  \BibitemOpen
  \bibfield  {author} {\bibinfo {author} {\bibfnamefont {G.~Y.}\ \bibnamefont
  {Cho}}, \bibinfo {author} {\bibfnamefont {J.~H.}\ \bibnamefont {Bardarson}},
  \bibinfo {author} {\bibfnamefont {Y.-M.}\ \bibnamefont {Lu}}, \ and\ \bibinfo
  {author} {\bibfnamefont {J.~E.}\ \bibnamefont {Moore}},\ }\href {\doibase
  10.1103/PhysRevB.86.214514} {\bibfield  {journal} {\bibinfo  {journal} {Phys.
  Rev. B}\ }\textbf {\bibinfo {volume} {86}},\ \bibinfo {pages} {214514}
  (\bibinfo {year} {2012})}\BibitemShut {NoStop}%
\bibitem [{\citenamefont {Bednik}\ \emph {et~al.}(2015)\citenamefont {Bednik},
  \citenamefont {Zyuzin},\ and\ \citenamefont {Burkov}}]{WS2}%
  \BibitemOpen
  \bibfield  {author} {\bibinfo {author} {\bibfnamefont {G.}~\bibnamefont
  {Bednik}}, \bibinfo {author} {\bibfnamefont {A.~A.}\ \bibnamefont {Zyuzin}},
  \ and\ \bibinfo {author} {\bibfnamefont {A.~A.}\ \bibnamefont {Burkov}},\
  }\href {\doibase 10.1103/physrevb.92.035153} {\bibfield  {journal} {\bibinfo
  {journal} {Physical Review B}\ }\textbf {\bibinfo {volume} {92}} (\bibinfo
  {year} {2015}),\ 10.1103/physrevb.92.035153}\BibitemShut {NoStop}%
\bibitem [{\citenamefont {Kobayashi}\ and\ \citenamefont
  {Sato}(2015)}]{Sato_TCSC}%
  \BibitemOpen
  \bibfield  {author} {\bibinfo {author} {\bibfnamefont {S.}~\bibnamefont
  {Kobayashi}}\ and\ \bibinfo {author} {\bibfnamefont {M.}~\bibnamefont
  {Sato}},\ }\href {\doibase 10.1103/PhysRevLett.115.187001} {\bibfield
  {journal} {\bibinfo  {journal} {Phys. Rev. Lett.}\ }\textbf {\bibinfo
  {volume} {115}},\ \bibinfo {pages} {187001} (\bibinfo {year}
  {2015})}\BibitemShut {NoStop}%
\bibitem [{\citenamefont {Yan}\ \emph {et~al.}(2020)\citenamefont {Yan},
  \citenamefont {Wu},\ and\ \citenamefont {Huang}}]{Node_vortex}%
  \BibitemOpen
  \bibfield  {author} {\bibinfo {author} {\bibfnamefont {Z.}~\bibnamefont
  {Yan}}, \bibinfo {author} {\bibfnamefont {Z.}~\bibnamefont {Wu}}, \ and\
  \bibinfo {author} {\bibfnamefont {W.}~\bibnamefont {Huang}},\ }\href
  {\doibase 10.1103/PhysRevLett.124.257001} {\bibfield  {journal} {\bibinfo
  {journal} {Phys. Rev. Lett.}\ }\textbf {\bibinfo {volume} {124}},\ \bibinfo
  {pages} {257001} (\bibinfo {year} {2020})}\BibitemShut {NoStop}%
\bibitem [{\citenamefont {Zhang}\ \emph {et~al.}(2020)\citenamefont {Zhang},
  \citenamefont {Hsu},\ and\ \citenamefont {Das~Sarma}}]{HOTDS}%
  \BibitemOpen
  \bibfield  {author} {\bibinfo {author} {\bibfnamefont {R.-X.}\ \bibnamefont
  {Zhang}}, \bibinfo {author} {\bibfnamefont {Y.-T.}\ \bibnamefont {Hsu}}, \
  and\ \bibinfo {author} {\bibfnamefont {S.}~\bibnamefont {Das~Sarma}},\ }\href
  {\doibase 10.1103/PhysRevB.102.094503} {\bibfield  {journal} {\bibinfo
  {journal} {Phys. Rev. B}\ }\textbf {\bibinfo {volume} {102}},\ \bibinfo
  {pages} {094503} (\bibinfo {year} {2020})}\BibitemShut {NoStop}%
\bibitem [{\citenamefont {Soluyanov}(2015)}]{Type_II}%
  \BibitemOpen
  \bibfield  {author} {\bibinfo {author} {\bibfnamefont {A.~A.}\ \bibnamefont
  {Soluyanov}},\ }\href {\doibase 10.1038/nature15768} {\bibfield  {journal}
  {\bibinfo  {journal} {Nature}\ }\textbf {\bibinfo {volume} {527}},\ \bibinfo
  {pages} {495–498} (\bibinfo {year} {2015})}\BibitemShut {NoStop}%
\bibitem [{\citenamefont {Noh}\ \emph {et~al.}(2017)\citenamefont {Noh},
  \citenamefont {Jeong}, \citenamefont {Cho}, \citenamefont {Kim},
  \citenamefont {Min},\ and\ \citenamefont {Park}}]{PdTe2}%
  \BibitemOpen
  \bibfield  {author} {\bibinfo {author} {\bibfnamefont {H.-J.}\ \bibnamefont
  {Noh}}, \bibinfo {author} {\bibfnamefont {J.}~\bibnamefont {Jeong}}, \bibinfo
  {author} {\bibfnamefont {E.-J.}\ \bibnamefont {Cho}}, \bibinfo {author}
  {\bibfnamefont {K.}~\bibnamefont {Kim}}, \bibinfo {author} {\bibfnamefont
  {B.~I.}\ \bibnamefont {Min}}, \ and\ \bibinfo {author} {\bibfnamefont
  {B.-G.}\ \bibnamefont {Park}},\ }\href {\doibase
  10.1103/PhysRevLett.119.016401} {\bibfield  {journal} {\bibinfo  {journal}
  {Phys. Rev. Lett.}\ }\textbf {\bibinfo {volume} {119}},\ \bibinfo {pages}
  {016401} (\bibinfo {year} {2017})}\BibitemShut {NoStop}%
\bibitem [{\citenamefont {Yan}(2017)}]{PtTe2}%
  \BibitemOpen
  \bibfield  {author} {\bibinfo {author} {\bibfnamefont {M.}~\bibnamefont
  {Yan}},\ }\href {\doibase 10.1038/s41467-017-00280-6} {\bibfield  {journal}
  {\bibinfo  {journal} {Nature Communications}\ }\textbf {\bibinfo {volume}
  {8}},\ \bibinfo {pages} {1–6} (\bibinfo {year} {2017})}\BibitemShut
  {NoStop}%
\bibitem [{\citenamefont {Bahramy}(2018)}]{TMD}%
  \BibitemOpen
  \bibfield  {author} {\bibinfo {author} {\bibfnamefont {M.}~\bibnamefont
  {Bahramy}},\ }\href {\doibase 10.1038/nmat5031} {\bibfield  {journal}
  {\bibinfo  {journal} {Nature Materials}\ }\textbf {\bibinfo {volume} {17}},\
  \bibinfo {pages} {21–28} (\bibinfo {year} {2018})}\BibitemShut {NoStop}%
\bibitem [{\citenamefont {Xu}\ \emph {et~al.}(2018)\citenamefont {Xu},
  \citenamefont {Li}, \citenamefont {Jiao}, \citenamefont {Zhou}, \citenamefont
  {Qian}, \citenamefont {Sankar}, \citenamefont {Zhigadlo}, \citenamefont {Qi},
  \citenamefont {Qian}, \citenamefont {Chou},\ and\ \citenamefont
  {Xu}}]{NiTe2}%
  \BibitemOpen
  \bibfield  {author} {\bibinfo {author} {\bibfnamefont {C.}~\bibnamefont
  {Xu}}, \bibinfo {author} {\bibfnamefont {B.}~\bibnamefont {Li}}, \bibinfo
  {author} {\bibfnamefont {W.}~\bibnamefont {Jiao}}, \bibinfo {author}
  {\bibfnamefont {W.}~\bibnamefont {Zhou}}, \bibinfo {author} {\bibfnamefont
  {B.}~\bibnamefont {Qian}}, \bibinfo {author} {\bibfnamefont {R.}~\bibnamefont
  {Sankar}}, \bibinfo {author} {\bibfnamefont {N.~D.}\ \bibnamefont
  {Zhigadlo}}, \bibinfo {author} {\bibfnamefont {Y.}~\bibnamefont {Qi}},
  \bibinfo {author} {\bibfnamefont {D.}~\bibnamefont {Qian}}, \bibinfo {author}
  {\bibfnamefont {F.-C.}\ \bibnamefont {Chou}}, \ and\ \bibinfo {author}
  {\bibfnamefont {X.}~\bibnamefont {Xu}},\ }\href {\doibase
  10.1021/acs.chemmater.8b02132} {\bibfield  {journal} {\bibinfo  {journal}
  {Chemistry of Materials}\ }\textbf {\bibinfo {volume} {30}},\ \bibinfo
  {pages} {4823} (\bibinfo {year} {2018})}\BibitemShut {NoStop}%
\bibitem [{\citenamefont {Wan-Dong}\ \emph {et~al.}(2015)\citenamefont
  {Wan-Dong}, \citenamefont {Hu}, \citenamefont {Tian}, \citenamefont
  {Zhi-Jun}, \citenamefont {Gang}, \citenamefont {Ai-Fang}, \citenamefont
  {Yao-Bo}, \citenamefont {Peng}, \citenamefont {Xun}, \citenamefont {Zhong},
  \citenamefont {Xi}, \citenamefont {Pierre}, \citenamefont {Nan-Lin},\ and\
  \citenamefont {Hong}}]{DingHong}%
  \BibitemOpen
  \bibfield  {author} {\bibinfo {author} {\bibfnamefont {K.}~\bibnamefont
  {Wan-Dong}}, \bibinfo {author} {\bibfnamefont {M.}~\bibnamefont {Hu}},
  \bibinfo {author} {\bibfnamefont {Q.}~\bibnamefont {Tian}}, \bibinfo {author}
  {\bibfnamefont {W.}~\bibnamefont {Zhi-Jun}}, \bibinfo {author} {\bibfnamefont
  {X.}~\bibnamefont {Gang}}, \bibinfo {author} {\bibfnamefont {F.}~\bibnamefont
  {Ai-Fang}}, \bibinfo {author} {\bibfnamefont {H.}~\bibnamefont {Yao-Bo}},
  \bibinfo {author} {\bibfnamefont {Z.}~\bibnamefont {Peng}}, \bibinfo {author}
  {\bibfnamefont {S.}~\bibnamefont {Xun}}, \bibinfo {author} {\bibfnamefont
  {F.}~\bibnamefont {Zhong}}, \bibinfo {author} {\bibfnamefont
  {D.}~\bibnamefont {Xi}}, \bibinfo {author} {\bibfnamefont {R.}~\bibnamefont
  {Pierre}}, \bibinfo {author} {\bibfnamefont {W.}~\bibnamefont {Nan-Lin}}, \
  and\ \bibinfo {author} {\bibfnamefont {D.}~\bibnamefont {Hong}},\ }\href
  {\doibase 10.1088/0256-307X/32/7/077402} {\bibfield  {journal} {\bibinfo
  {journal} {Chinese Physics Letters}\ }\textbf {\bibinfo {volume} {32}},\
  \bibinfo {eid} {077402} (\bibinfo {year} {2015})}\BibitemShut {NoStop}%
\bibitem [{\citenamefont {Leng}\ \emph {et~al.}(2017)\citenamefont {Leng},
  \citenamefont {Paulsen}, \citenamefont {Huang},\ and\ \citenamefont
  {de~Visser}}]{PdTe2_SC}%
  \BibitemOpen
  \bibfield  {author} {\bibinfo {author} {\bibfnamefont {H.}~\bibnamefont
  {Leng}}, \bibinfo {author} {\bibfnamefont {C.}~\bibnamefont {Paulsen}},
  \bibinfo {author} {\bibfnamefont {Y.~K.}\ \bibnamefont {Huang}}, \ and\
  \bibinfo {author} {\bibfnamefont {A.}~\bibnamefont {de~Visser}},\ }\href
  {\doibase 10.1103/PhysRevB.96.220506} {\bibfield  {journal} {\bibinfo
  {journal} {Phys. Rev. B}\ }\textbf {\bibinfo {volume} {96}},\ \bibinfo
  {pages} {220506} (\bibinfo {year} {2017})}\BibitemShut {NoStop}%
\bibitem [{\citenamefont {Le}\ \emph {et~al.}(2019)\citenamefont {Le},
  \citenamefont {Yin}, \citenamefont {Feng}, \citenamefont {Huang},
  \citenamefont {Che}, \citenamefont {Li}, \citenamefont {Shi},\ and\
  \citenamefont {Lu}}]{PdTe2_SC2}%
  \BibitemOpen
  \bibfield  {author} {\bibinfo {author} {\bibfnamefont {T.}~\bibnamefont
  {Le}}, \bibinfo {author} {\bibfnamefont {L.}~\bibnamefont {Yin}}, \bibinfo
  {author} {\bibfnamefont {Z.}~\bibnamefont {Feng}}, \bibinfo {author}
  {\bibfnamefont {Q.}~\bibnamefont {Huang}}, \bibinfo {author} {\bibfnamefont
  {L.}~\bibnamefont {Che}}, \bibinfo {author} {\bibfnamefont {J.}~\bibnamefont
  {Li}}, \bibinfo {author} {\bibfnamefont {Y.}~\bibnamefont {Shi}}, \ and\
  \bibinfo {author} {\bibfnamefont {X.}~\bibnamefont {Lu}},\ }\href {\doibase
  10.1103/PhysRevB.99.180504} {\bibfield  {journal} {\bibinfo  {journal} {Phys.
  Rev. B}\ }\textbf {\bibinfo {volume} {99}},\ \bibinfo {pages} {180504}
  (\bibinfo {year} {2019})}\BibitemShut {NoStop}%
\bibitem [{\citenamefont {{de Lima}}\ \emph {et~al.}(2018)\citenamefont {{de
  Lima}}, \citenamefont {{de Cassia}}, \citenamefont {Santos}, \citenamefont
  {Correa}, \citenamefont {Grant}, \citenamefont {Manesco}, \citenamefont
  {Martins}, \citenamefont {Eleno}, \citenamefont {Torikachvili},\ and\
  \citenamefont {Machado}}]{NiTe2_SC}%
  \BibitemOpen
  \bibfield  {author} {\bibinfo {author} {\bibfnamefont {B.}~\bibnamefont {{de
  Lima}}}, \bibinfo {author} {\bibfnamefont {R.}~\bibnamefont {{de Cassia}}},
  \bibinfo {author} {\bibfnamefont {F.}~\bibnamefont {Santos}}, \bibinfo
  {author} {\bibfnamefont {L.}~\bibnamefont {Correa}}, \bibinfo {author}
  {\bibfnamefont {T.}~\bibnamefont {Grant}}, \bibinfo {author} {\bibfnamefont
  {A.}~\bibnamefont {Manesco}}, \bibinfo {author} {\bibfnamefont
  {G.}~\bibnamefont {Martins}}, \bibinfo {author} {\bibfnamefont
  {L.}~\bibnamefont {Eleno}}, \bibinfo {author} {\bibfnamefont
  {M.}~\bibnamefont {Torikachvili}}, \ and\ \bibinfo {author} {\bibfnamefont
  {A.}~\bibnamefont {Machado}},\ }\href {\doibase
  https://doi.org/10.1016/j.ssc.2018.08.014} {\bibfield  {journal} {\bibinfo
  {journal} {Solid State Communications}\ }\textbf {\bibinfo {volume} {283}},\
  \bibinfo {pages} {27} (\bibinfo {year} {2018})}\BibitemShut {NoStop}%
\bibitem [{\citenamefont {Feng}\ \emph {et~al.}(2021)\citenamefont {Feng},
  \citenamefont {Si}, \citenamefont {Li}, \citenamefont {Dong}, \citenamefont
  {Xu}, \citenamefont {Yang}, \citenamefont {Zhang}, \citenamefont {Wang},
  \citenamefont {Wu}, \citenamefont {Hou}, \citenamefont {Xing}, \citenamefont
  {Wan}, \citenamefont {Li}, \citenamefont {Deng}, \citenamefont {Feng},
  \citenamefont {Pal}, \citenamefont {Chen}, \citenamefont {Hu}, \citenamefont
  {Ge}, \citenamefont {Dong}, \citenamefont {Wang}, \citenamefont {Ren},
  \citenamefont {Cao}, \citenamefont {Liu}, \citenamefont {Xu}, \citenamefont
  {Zhang}, \citenamefont {Chen},\ and\ \citenamefont {Yeh}}]{NiTe2_SC2}%
  \BibitemOpen
  \bibfield  {author} {\bibinfo {author} {\bibfnamefont {Z.}~\bibnamefont
  {Feng}}, \bibinfo {author} {\bibfnamefont {J.}~\bibnamefont {Si}}, \bibinfo
  {author} {\bibfnamefont {T.}~\bibnamefont {Li}}, \bibinfo {author}
  {\bibfnamefont {H.}~\bibnamefont {Dong}}, \bibinfo {author} {\bibfnamefont
  {C.}~\bibnamefont {Xu}}, \bibinfo {author} {\bibfnamefont {J.}~\bibnamefont
  {Yang}}, \bibinfo {author} {\bibfnamefont {Z.}~\bibnamefont {Zhang}},
  \bibinfo {author} {\bibfnamefont {K.}~\bibnamefont {Wang}}, \bibinfo {author}
  {\bibfnamefont {H.}~\bibnamefont {Wu}}, \bibinfo {author} {\bibfnamefont
  {Q.}~\bibnamefont {Hou}}, \bibinfo {author} {\bibfnamefont {J.-J.}\
  \bibnamefont {Xing}}, \bibinfo {author} {\bibfnamefont {S.}~\bibnamefont
  {Wan}}, \bibinfo {author} {\bibfnamefont {S.}~\bibnamefont {Li}}, \bibinfo
  {author} {\bibfnamefont {W.}~\bibnamefont {Deng}}, \bibinfo {author}
  {\bibfnamefont {J.}~\bibnamefont {Feng}}, \bibinfo {author} {\bibfnamefont
  {A.}~\bibnamefont {Pal}}, \bibinfo {author} {\bibfnamefont {F.}~\bibnamefont
  {Chen}}, \bibinfo {author} {\bibfnamefont {S.}~\bibnamefont {Hu}}, \bibinfo
  {author} {\bibfnamefont {J.-Y.}\ \bibnamefont {Ge}}, \bibinfo {author}
  {\bibfnamefont {C.}~\bibnamefont {Dong}}, \bibinfo {author} {\bibfnamefont
  {S.}~\bibnamefont {Wang}}, \bibinfo {author} {\bibfnamefont {W.}~\bibnamefont
  {Ren}}, \bibinfo {author} {\bibfnamefont {S.}~\bibnamefont {Cao}}, \bibinfo
  {author} {\bibfnamefont {Y.}~\bibnamefont {Liu}}, \bibinfo {author}
  {\bibfnamefont {X.}~\bibnamefont {Xu}}, \bibinfo {author} {\bibfnamefont
  {J.}~\bibnamefont {Zhang}}, \bibinfo {author} {\bibfnamefont
  {B.}~\bibnamefont {Chen}}, \ and\ \bibinfo {author} {\bibfnamefont {N.-C.}\
  \bibnamefont {Yeh}},\ }\href {\doibase
  https://doi.org/10.1016/j.mtphys.2020.100339} {\bibfield  {journal} {\bibinfo
   {journal} {Materials Today Physics}\ }\textbf {\bibinfo {volume} {17}},\
  \bibinfo {pages} {100339} (\bibinfo {year} {2021})}\BibitemShut {NoStop}%
\bibitem [{\citenamefont {Huang}\ \emph {et~al.}(2018)\citenamefont {Huang},
  \citenamefont {Narayan}, \citenamefont {Zhang}, \citenamefont {Liu},
  \citenamefont {Yan}, \citenamefont {Wang}, \citenamefont {Zhang},
  \citenamefont {Wang}, \citenamefont {Zhou}, \citenamefont {Yi}, \citenamefont
  {Liu}, \citenamefont {Ling}, \citenamefont {Zhang}, \citenamefont {Liu},
  \citenamefont {Sankar}, \citenamefont {Chou}, \citenamefont {Wang},
  \citenamefont {Shi}, \citenamefont {Law}, \citenamefont {Sanvito},
  \citenamefont {Zhou}, \citenamefont {Han},\ and\ \citenamefont
  {Xiu}}]{proximity_SC}%
  \BibitemOpen
  \bibfield  {author} {\bibinfo {author} {\bibfnamefont {C.}~\bibnamefont
  {Huang}}, \bibinfo {author} {\bibfnamefont {A.}~\bibnamefont {Narayan}},
  \bibinfo {author} {\bibfnamefont {E.}~\bibnamefont {Zhang}}, \bibinfo
  {author} {\bibfnamefont {Y.}~\bibnamefont {Liu}}, \bibinfo {author}
  {\bibfnamefont {X.}~\bibnamefont {Yan}}, \bibinfo {author} {\bibfnamefont
  {J.}~\bibnamefont {Wang}}, \bibinfo {author} {\bibfnamefont {C.}~\bibnamefont
  {Zhang}}, \bibinfo {author} {\bibfnamefont {W.}~\bibnamefont {Wang}},
  \bibinfo {author} {\bibfnamefont {T.}~\bibnamefont {Zhou}}, \bibinfo {author}
  {\bibfnamefont {C.}~\bibnamefont {Yi}}, \bibinfo {author} {\bibfnamefont
  {S.}~\bibnamefont {Liu}}, \bibinfo {author} {\bibfnamefont {J.}~\bibnamefont
  {Ling}}, \bibinfo {author} {\bibfnamefont {H.}~\bibnamefont {Zhang}},
  \bibinfo {author} {\bibfnamefont {R.}~\bibnamefont {Liu}}, \bibinfo {author}
  {\bibfnamefont {R.}~\bibnamefont {Sankar}}, \bibinfo {author} {\bibfnamefont
  {F.}~\bibnamefont {Chou}}, \bibinfo {author} {\bibfnamefont {Y.}~\bibnamefont
  {Wang}}, \bibinfo {author} {\bibfnamefont {Y.}~\bibnamefont {Shi}}, \bibinfo
  {author} {\bibfnamefont {K.~T.}\ \bibnamefont {Law}}, \bibinfo {author}
  {\bibfnamefont {S.}~\bibnamefont {Sanvito}}, \bibinfo {author} {\bibfnamefont
  {P.}~\bibnamefont {Zhou}}, \bibinfo {author} {\bibfnamefont {Z.}~\bibnamefont
  {Han}}, \ and\ \bibinfo {author} {\bibfnamefont {F.}~\bibnamefont {Xiu}},\
  }\href {\doibase 10.1021/acsnano.8b03102} {\bibfield  {journal} {\bibinfo
  {journal} {ACS Nano}\ }\textbf {\bibinfo {volume} {12}},\ \bibinfo {pages}
  {7185} (\bibinfo {year} {2018})},\ \bibinfo {note} {pMID:
  29901987}\BibitemShut {NoStop}%
\bibitem [{\citenamefont {Yang}\ \emph {et~al.}(2012)\citenamefont {Yang},
  \citenamefont {Choi}, \citenamefont {Oh}, \citenamefont {Hogan},
  \citenamefont {Horibe}, \citenamefont {Kim}, \citenamefont {Min},\ and\
  \citenamefont {Cheong}}]{IrTe2}%
  \BibitemOpen
  \bibfield  {author} {\bibinfo {author} {\bibfnamefont {J.~J.}\ \bibnamefont
  {Yang}}, \bibinfo {author} {\bibfnamefont {Y.~J.}\ \bibnamefont {Choi}},
  \bibinfo {author} {\bibfnamefont {Y.~S.}\ \bibnamefont {Oh}}, \bibinfo
  {author} {\bibfnamefont {A.}~\bibnamefont {Hogan}}, \bibinfo {author}
  {\bibfnamefont {Y.}~\bibnamefont {Horibe}}, \bibinfo {author} {\bibfnamefont
  {K.}~\bibnamefont {Kim}}, \bibinfo {author} {\bibfnamefont {B.~I.}\
  \bibnamefont {Min}}, \ and\ \bibinfo {author} {\bibfnamefont {S.-W.}\
  \bibnamefont {Cheong}},\ }\href {\doibase 10.1103/PhysRevLett.108.116402}
  {\bibfield  {journal} {\bibinfo  {journal} {Phys. Rev. Lett.}\ }\textbf
  {\bibinfo {volume} {108}},\ \bibinfo {pages} {116402} (\bibinfo {year}
  {2012})}\BibitemShut {NoStop}%
\bibitem [{\citenamefont {Pyon}\ \emph {et~al.}(2012)\citenamefont {Pyon},
  \citenamefont {Kudo},\ and\ \citenamefont {Nohara}}]{IrTe2_2}%
  \BibitemOpen
  \bibfield  {author} {\bibinfo {author} {\bibfnamefont {S.}~\bibnamefont
  {Pyon}}, \bibinfo {author} {\bibfnamefont {K.}~\bibnamefont {Kudo}}, \ and\
  \bibinfo {author} {\bibfnamefont {M.}~\bibnamefont {Nohara}},\ }\href
  {\doibase 10.1143/JPSJ.81.053701} {\bibfield  {journal} {\bibinfo  {journal}
  {Journal of the Physical Society of Japan}\ }\textbf {\bibinfo {volume}
  {81}},\ \bibinfo {pages} {053701} (\bibinfo {year} {2012})}\BibitemShut
  {NoStop}%
\bibitem [{\citenamefont {Teknowijoyo}\ \emph {et~al.}(2018)\citenamefont
  {Teknowijoyo}, \citenamefont {Jo}, \citenamefont {Scheurer}, \citenamefont
  {Tanatar}, \citenamefont {Cho}, \citenamefont {Bud'ko}, \citenamefont {Orth},
  \citenamefont {Canfield},\ and\ \citenamefont {Prozorov}}]{PdTe2_SC3}%
  \BibitemOpen
  \bibfield  {author} {\bibinfo {author} {\bibfnamefont {S.}~\bibnamefont
  {Teknowijoyo}}, \bibinfo {author} {\bibfnamefont {N.~H.}\ \bibnamefont {Jo}},
  \bibinfo {author} {\bibfnamefont {M.~S.}\ \bibnamefont {Scheurer}}, \bibinfo
  {author} {\bibfnamefont {M.~A.}\ \bibnamefont {Tanatar}}, \bibinfo {author}
  {\bibfnamefont {K.}~\bibnamefont {Cho}}, \bibinfo {author} {\bibfnamefont
  {S.~L.}\ \bibnamefont {Bud'ko}}, \bibinfo {author} {\bibfnamefont {P.~P.}\
  \bibnamefont {Orth}}, \bibinfo {author} {\bibfnamefont {P.~C.}\ \bibnamefont
  {Canfield}}, \ and\ \bibinfo {author} {\bibfnamefont {R.}~\bibnamefont
  {Prozorov}},\ }\href {\doibase 10.1103/PhysRevB.98.024508} {\bibfield
  {journal} {\bibinfo  {journal} {Phys. Rev. B}\ }\textbf {\bibinfo {volume}
  {98}},\ \bibinfo {pages} {024508} (\bibinfo {year} {2018})}\BibitemShut
  {NoStop}%
\bibitem [{\citenamefont {Benalcazar}\ \emph {et~al.}(2017)\citenamefont
  {Benalcazar}, \citenamefont {Bernevig},\ and\ \citenamefont
  {Hughes}}]{quadrupole}%
  \BibitemOpen
  \bibfield  {author} {\bibinfo {author} {\bibfnamefont {W.~A.}\ \bibnamefont
  {Benalcazar}}, \bibinfo {author} {\bibfnamefont {B.~A.}\ \bibnamefont
  {Bernevig}}, \ and\ \bibinfo {author} {\bibfnamefont {T.~L.}\ \bibnamefont
  {Hughes}},\ }\href {\doibase 10.1126/science.aah6442} {\bibfield  {journal}
  {\bibinfo  {journal} {Science}\ }\textbf {\bibinfo {volume} {357}},\ \bibinfo
  {pages} {61} (\bibinfo {year} {2017})}\BibitemShut {NoStop}%
\bibitem [{\citenamefont {Lin}\ and\ \citenamefont
  {Hughes}(2018)}]{quadrupolar_semimetal}%
  \BibitemOpen
  \bibfield  {author} {\bibinfo {author} {\bibfnamefont {M.}~\bibnamefont
  {Lin}}\ and\ \bibinfo {author} {\bibfnamefont {T.~L.}\ \bibnamefont
  {Hughes}},\ }\href {\doibase 10.1103/PhysRevB.98.241103} {\bibfield
  {journal} {\bibinfo  {journal} {Phys. Rev. B}\ }\textbf {\bibinfo {volume}
  {98}},\ \bibinfo {pages} {241103} (\bibinfo {year} {2018})}\BibitemShut
  {NoStop}%
\bibitem [{\citenamefont {Guo}\ \emph {et~al.}(2022)\citenamefont {Guo},
  \citenamefont {Deng}, \citenamefont {Xie},\ and\ \citenamefont
  {Wang}}]{GuoZhaopeng}%
  \BibitemOpen
  \bibfield  {author} {\bibinfo {author} {\bibfnamefont {Z.}~\bibnamefont
  {Guo}}, \bibinfo {author} {\bibfnamefont {J.}~\bibnamefont {Deng}}, \bibinfo
  {author} {\bibfnamefont {Y.}~\bibnamefont {Xie}}, \ and\ \bibinfo {author}
  {\bibfnamefont {Z.}~\bibnamefont {Wang}},\ }\href {\doibase
  10.1038/s41535-022-00498-8} {\bibfield  {journal} {\bibinfo  {journal} {npj
  Quantum Materials}\ }\textbf {\bibinfo {volume} {7}} (\bibinfo {year}
  {2022}),\ 10.1038/s41535-022-00498-8}\BibitemShut {NoStop}%
\bibitem [{\citenamefont {Ono}\ \emph {et~al.}(2020)\citenamefont {Ono},
  \citenamefont {Po},\ and\ \citenamefont {Watanabe}}]{Po_1}%
  \BibitemOpen
  \bibfield  {author} {\bibinfo {author} {\bibfnamefont {S.}~\bibnamefont
  {Ono}}, \bibinfo {author} {\bibfnamefont {H.~C.}\ \bibnamefont {Po}}, \ and\
  \bibinfo {author} {\bibfnamefont {H.}~\bibnamefont {Watanabe}},\ }\href
  {\doibase 10.1126/sciadv.aaz8367} {\bibfield  {journal} {\bibinfo  {journal}
  {Science Advances}\ }\textbf {\bibinfo {volume} {6}},\ \bibinfo {pages}
  {eaaz8367} (\bibinfo {year} {2020})}\BibitemShut {NoStop}%
\bibitem [{\citenamefont {Ono}\ \emph {et~al.}(2021)\citenamefont {Ono},
  \citenamefont {Po},\ and\ \citenamefont {Shiozaki}}]{Po_2}%
  \BibitemOpen
  \bibfield  {author} {\bibinfo {author} {\bibfnamefont {S.}~\bibnamefont
  {Ono}}, \bibinfo {author} {\bibfnamefont {H.~C.}\ \bibnamefont {Po}}, \ and\
  \bibinfo {author} {\bibfnamefont {K.}~\bibnamefont {Shiozaki}},\ }\href
  {\doibase 10.1103/PhysRevResearch.3.023086} {\bibfield  {journal} {\bibinfo
  {journal} {Phys. Rev. Research}\ }\textbf {\bibinfo {volume} {3}},\ \bibinfo
  {pages} {023086} (\bibinfo {year} {2021})}\BibitemShut {NoStop}%
\bibitem [{\citenamefont {Ahn}\ and\ \citenamefont {Yang}(2020)}]{theory_1}%
  \BibitemOpen
  \bibfield  {author} {\bibinfo {author} {\bibfnamefont {J.}~\bibnamefont
  {Ahn}}\ and\ \bibinfo {author} {\bibfnamefont {B.-J.}\ \bibnamefont {Yang}},\
  }\href {\doibase 10.1103/PhysRevResearch.2.012060} {\bibfield  {journal}
  {\bibinfo  {journal} {Phys. Rev. Research}\ }\textbf {\bibinfo {volume}
  {2}},\ \bibinfo {pages} {012060} (\bibinfo {year} {2020})}\BibitemShut
  {NoStop}%
\bibitem [{\citenamefont {Wang}\ \emph
  {et~al.}(2018{\natexlab{a}})\citenamefont {Wang}, \citenamefont {Lin},\ and\
  \citenamefont {Hughes}}]{theory_2}%
  \BibitemOpen
  \bibfield  {author} {\bibinfo {author} {\bibfnamefont {Y.}~\bibnamefont
  {Wang}}, \bibinfo {author} {\bibfnamefont {M.}~\bibnamefont {Lin}}, \ and\
  \bibinfo {author} {\bibfnamefont {T.~L.}\ \bibnamefont {Hughes}},\ }\href
  {\doibase 10.1103/PhysRevB.98.165144} {\bibfield  {journal} {\bibinfo
  {journal} {Phys. Rev. B}\ }\textbf {\bibinfo {volume} {98}},\ \bibinfo
  {pages} {165144} (\bibinfo {year} {2018}{\natexlab{a}})}\BibitemShut
  {NoStop}%
\bibitem [{\citenamefont {Tiwari}\ \emph {et~al.}(2020)\citenamefont {Tiwari},
  \citenamefont {Jahin},\ and\ \citenamefont {Wang}}]{theory_3}%
  \BibitemOpen
  \bibfield  {author} {\bibinfo {author} {\bibfnamefont {A.}~\bibnamefont
  {Tiwari}}, \bibinfo {author} {\bibfnamefont {A.}~\bibnamefont {Jahin}}, \
  and\ \bibinfo {author} {\bibfnamefont {Y.}~\bibnamefont {Wang}},\ }\href
  {\doibase 10.1103/PhysRevResearch.2.043300} {\bibfield  {journal} {\bibinfo
  {journal} {Phys. Rev. Research}\ }\textbf {\bibinfo {volume} {2}},\ \bibinfo
  {pages} {043300} (\bibinfo {year} {2020})}\BibitemShut {NoStop}%
\bibitem [{\citenamefont {Zhu}(2019)}]{theory_4}%
  \BibitemOpen
  \bibfield  {author} {\bibinfo {author} {\bibfnamefont {X.}~\bibnamefont
  {Zhu}},\ }\href {\doibase 10.1103/PhysRevLett.122.236401} {\bibfield
  {journal} {\bibinfo  {journal} {Phys. Rev. Lett.}\ }\textbf {\bibinfo
  {volume} {122}},\ \bibinfo {pages} {236401} (\bibinfo {year}
  {2019})}\BibitemShut {NoStop}%
\bibitem [{\citenamefont {Langbehn}\ \emph {et~al.}(2017)\citenamefont
  {Langbehn}, \citenamefont {Peng}, \citenamefont {Trifunovic}, \citenamefont
  {von Oppen},\ and\ \citenamefont {Brouwer}}]{theory_5}%
  \BibitemOpen
  \bibfield  {author} {\bibinfo {author} {\bibfnamefont {J.}~\bibnamefont
  {Langbehn}}, \bibinfo {author} {\bibfnamefont {Y.}~\bibnamefont {Peng}},
  \bibinfo {author} {\bibfnamefont {L.}~\bibnamefont {Trifunovic}}, \bibinfo
  {author} {\bibfnamefont {F.}~\bibnamefont {von Oppen}}, \ and\ \bibinfo
  {author} {\bibfnamefont {P.~W.}\ \bibnamefont {Brouwer}},\ }\href {\doibase
  10.1103/PhysRevLett.119.246401} {\bibfield  {journal} {\bibinfo  {journal}
  {Phys. Rev. Lett.}\ }\textbf {\bibinfo {volume} {119}},\ \bibinfo {pages}
  {246401} (\bibinfo {year} {2017})}\BibitemShut {NoStop}%
\bibitem [{\citenamefont {Khalaf}(2018)}]{theory_6}%
  \BibitemOpen
  \bibfield  {author} {\bibinfo {author} {\bibfnamefont {E.}~\bibnamefont
  {Khalaf}},\ }\href {\doibase 10.1103/PhysRevB.97.205136} {\bibfield
  {journal} {\bibinfo  {journal} {Phys. Rev. B}\ }\textbf {\bibinfo {volume}
  {97}},\ \bibinfo {pages} {205136} (\bibinfo {year} {2018})}\BibitemShut
  {NoStop}%
\bibitem [{\citenamefont {Kheirkhah}\ \emph {et~al.}(2020)\citenamefont
  {Kheirkhah}, \citenamefont {Yan}, \citenamefont {Nagai},\ and\ \citenamefont
  {Marsiglio}}]{theory_7}%
  \BibitemOpen
  \bibfield  {author} {\bibinfo {author} {\bibfnamefont {M.}~\bibnamefont
  {Kheirkhah}}, \bibinfo {author} {\bibfnamefont {Z.}~\bibnamefont {Yan}},
  \bibinfo {author} {\bibfnamefont {Y.}~\bibnamefont {Nagai}}, \ and\ \bibinfo
  {author} {\bibfnamefont {F.}~\bibnamefont {Marsiglio}},\ }\href {\doibase
  10.1103/PhysRevLett.125.017001} {\bibfield  {journal} {\bibinfo  {journal}
  {Phys. Rev. Lett.}\ }\textbf {\bibinfo {volume} {125}},\ \bibinfo {pages}
  {017001} (\bibinfo {year} {2020})}\BibitemShut {NoStop}%
\bibitem [{\citenamefont {Yan}(2019)}]{theory_8}%
  \BibitemOpen
  \bibfield  {author} {\bibinfo {author} {\bibfnamefont {Z.}~\bibnamefont
  {Yan}},\ }\href {\doibase 10.1103/PhysRevLett.123.177001} {\bibfield
  {journal} {\bibinfo  {journal} {Phys. Rev. Lett.}\ }\textbf {\bibinfo
  {volume} {123}},\ \bibinfo {pages} {177001} (\bibinfo {year}
  {2019})}\BibitemShut {NoStop}%
\bibitem [{\citenamefont {Pan}\ \emph {et~al.}(2019)\citenamefont {Pan},
  \citenamefont {Yang}, \citenamefont {Chen}, \citenamefont {Xu}, \citenamefont
  {Liu},\ and\ \citenamefont {Liu}}]{theory_9}%
  \BibitemOpen
  \bibfield  {author} {\bibinfo {author} {\bibfnamefont {X.-H.}\ \bibnamefont
  {Pan}}, \bibinfo {author} {\bibfnamefont {K.-J.}\ \bibnamefont {Yang}},
  \bibinfo {author} {\bibfnamefont {L.}~\bibnamefont {Chen}}, \bibinfo {author}
  {\bibfnamefont {G.}~\bibnamefont {Xu}}, \bibinfo {author} {\bibfnamefont
  {C.-X.}\ \bibnamefont {Liu}}, \ and\ \bibinfo {author} {\bibfnamefont
  {X.}~\bibnamefont {Liu}},\ }\href {\doibase 10.1103/PhysRevLett.123.156801}
  {\bibfield  {journal} {\bibinfo  {journal} {Phys. Rev. Lett.}\ }\textbf
  {\bibinfo {volume} {123}},\ \bibinfo {pages} {156801} (\bibinfo {year}
  {2019})}\BibitemShut {NoStop}%
\bibitem [{\citenamefont {Vu}\ \emph {et~al.}(2020)\citenamefont {Vu},
  \citenamefont {Zhang},\ and\ \citenamefont {Das~Sarma}}]{theory_10}%
  \BibitemOpen
  \bibfield  {author} {\bibinfo {author} {\bibfnamefont {D.}~\bibnamefont
  {Vu}}, \bibinfo {author} {\bibfnamefont {R.-X.}\ \bibnamefont {Zhang}}, \
  and\ \bibinfo {author} {\bibfnamefont {S.}~\bibnamefont {Das~Sarma}},\ }\href
  {\doibase 10.1103/PhysRevResearch.2.043223} {\bibfield  {journal} {\bibinfo
  {journal} {Phys. Rev. Research}\ }\textbf {\bibinfo {volume} {2}},\ \bibinfo
  {pages} {043223} (\bibinfo {year} {2020})}\BibitemShut {NoStop}%
\bibitem [{\citenamefont {Ghosh}\ \emph {et~al.}(2021)\citenamefont {Ghosh},
  \citenamefont {Nag},\ and\ \citenamefont {Saha}}]{theory_11}%
  \BibitemOpen
  \bibfield  {author} {\bibinfo {author} {\bibfnamefont {A.~K.}\ \bibnamefont
  {Ghosh}}, \bibinfo {author} {\bibfnamefont {T.}~\bibnamefont {Nag}}, \ and\
  \bibinfo {author} {\bibfnamefont {A.}~\bibnamefont {Saha}},\ }\href {\doibase
  10.1103/PhysRevB.104.134508} {\bibfield  {journal} {\bibinfo  {journal}
  {Phys. Rev. B}\ }\textbf {\bibinfo {volume} {104}},\ \bibinfo {pages}
  {134508} (\bibinfo {year} {2021})}\BibitemShut {NoStop}%
\bibitem [{\citenamefont {Ghorashi}\ \emph {et~al.}(2019)\citenamefont
  {Ghorashi}, \citenamefont {Hu}, \citenamefont {Hughes},\ and\ \citenamefont
  {Rossi}}]{Dirac_SC}%
  \BibitemOpen
  \bibfield  {author} {\bibinfo {author} {\bibfnamefont {S.~A.~A.}\
  \bibnamefont {Ghorashi}}, \bibinfo {author} {\bibfnamefont {X.}~\bibnamefont
  {Hu}}, \bibinfo {author} {\bibfnamefont {T.~L.}\ \bibnamefont {Hughes}}, \
  and\ \bibinfo {author} {\bibfnamefont {E.}~\bibnamefont {Rossi}},\ }\href
  {\doibase 10.1103/PhysRevB.100.020509} {\bibfield  {journal} {\bibinfo
  {journal} {Phys. Rev. B}\ }\textbf {\bibinfo {volume} {100}},\ \bibinfo
  {pages} {020509} (\bibinfo {year} {2019})}\BibitemShut {NoStop}%
\bibitem [{\citenamefont {Zhang}\ \emph
  {et~al.}(2019{\natexlab{a}})\citenamefont {Zhang}, \citenamefont {Cole},\
  and\ \citenamefont {Das~Sarma}}]{Das_Sarma}%
  \BibitemOpen
  \bibfield  {author} {\bibinfo {author} {\bibfnamefont {R.-X.}\ \bibnamefont
  {Zhang}}, \bibinfo {author} {\bibfnamefont {W.~S.}\ \bibnamefont {Cole}}, \
  and\ \bibinfo {author} {\bibfnamefont {S.}~\bibnamefont {Das~Sarma}},\ }\href
  {\doibase 10.1103/PhysRevLett.122.187001} {\bibfield  {journal} {\bibinfo
  {journal} {Phys. Rev. Lett.}\ }\textbf {\bibinfo {volume} {122}},\ \bibinfo
  {pages} {187001} (\bibinfo {year} {2019}{\natexlab{a}})}\BibitemShut
  {NoStop}%
\bibitem [{\citenamefont {Zhang}\ and\ \citenamefont
  {Das~Sarma}(2021)}]{TRI_TSC}%
  \BibitemOpen
  \bibfield  {author} {\bibinfo {author} {\bibfnamefont {R.-X.}\ \bibnamefont
  {Zhang}}\ and\ \bibinfo {author} {\bibfnamefont {S.}~\bibnamefont
  {Das~Sarma}},\ }\href {\doibase 10.1103/PhysRevLett.126.137001} {\bibfield
  {journal} {\bibinfo  {journal} {Phys. Rev. Lett.}\ }\textbf {\bibinfo
  {volume} {126}},\ \bibinfo {pages} {137001} (\bibinfo {year}
  {2021})}\BibitemShut {NoStop}%
\bibitem [{\citenamefont {Wang}\ \emph
  {et~al.}(2018{\natexlab{b}})\citenamefont {Wang}, \citenamefont {Liu},
  \citenamefont {Lu},\ and\ \citenamefont {Zhang}}]{MKP}%
  \BibitemOpen
  \bibfield  {author} {\bibinfo {author} {\bibfnamefont {Q.}~\bibnamefont
  {Wang}}, \bibinfo {author} {\bibfnamefont {C.-C.}\ \bibnamefont {Liu}},
  \bibinfo {author} {\bibfnamefont {Y.-M.}\ \bibnamefont {Lu}}, \ and\ \bibinfo
  {author} {\bibfnamefont {F.}~\bibnamefont {Zhang}},\ }\href {\doibase
  10.1103/PhysRevLett.121.186801} {\bibfield  {journal} {\bibinfo  {journal}
  {Phys. Rev. Lett.}\ }\textbf {\bibinfo {volume} {121}},\ \bibinfo {pages}
  {186801} (\bibinfo {year} {2018}{\natexlab{b}})}\BibitemShut {NoStop}%
\bibitem [{\citenamefont {Zhang}\ \emph
  {et~al.}(2019{\natexlab{b}})\citenamefont {Zhang}, \citenamefont {Cole},
  \citenamefont {Wu},\ and\ \citenamefont {Das~Sarma}}]{BHZ_TSC}%
  \BibitemOpen
  \bibfield  {author} {\bibinfo {author} {\bibfnamefont {R.-X.}\ \bibnamefont
  {Zhang}}, \bibinfo {author} {\bibfnamefont {W.~S.}\ \bibnamefont {Cole}},
  \bibinfo {author} {\bibfnamefont {X.}~\bibnamefont {Wu}}, \ and\ \bibinfo
  {author} {\bibfnamefont {S.}~\bibnamefont {Das~Sarma}},\ }\href {\doibase
  10.1103/PhysRevLett.123.167001} {\bibfield  {journal} {\bibinfo  {journal}
  {Phys. Rev. Lett.}\ }\textbf {\bibinfo {volume} {123}},\ \bibinfo {pages}
  {167001} (\bibinfo {year} {2019}{\natexlab{b}})}\BibitemShut {NoStop}%
\bibitem [{\citenamefont {Wang}\ \emph {et~al.}(2012)\citenamefont {Wang},
  \citenamefont {Sun}, \citenamefont {Chen}, \citenamefont {Franchini},
  \citenamefont {Xu}, \citenamefont {Weng}, \citenamefont {Dai},\ and\
  \citenamefont {Fang}}]{Wang_Na3Bi}%
  \BibitemOpen
  \bibfield  {author} {\bibinfo {author} {\bibfnamefont {Z.}~\bibnamefont
  {Wang}}, \bibinfo {author} {\bibfnamefont {Y.}~\bibnamefont {Sun}}, \bibinfo
  {author} {\bibfnamefont {X.-Q.}\ \bibnamefont {Chen}}, \bibinfo {author}
  {\bibfnamefont {C.}~\bibnamefont {Franchini}}, \bibinfo {author}
  {\bibfnamefont {G.}~\bibnamefont {Xu}}, \bibinfo {author} {\bibfnamefont
  {H.}~\bibnamefont {Weng}}, \bibinfo {author} {\bibfnamefont {X.}~\bibnamefont
  {Dai}}, \ and\ \bibinfo {author} {\bibfnamefont {Z.}~\bibnamefont {Fang}},\
  }\href {\doibase 10.1103/PhysRevB.85.195320} {\bibfield  {journal} {\bibinfo
  {journal} {Phys. Rev. B}\ }\textbf {\bibinfo {volume} {85}},\ \bibinfo
  {pages} {195320} (\bibinfo {year} {2012})}\BibitemShut {NoStop}%
\bibitem [{\citenamefont {Wang}\ \emph {et~al.}(2013)\citenamefont {Wang},
  \citenamefont {Weng}, \citenamefont {Wu}, \citenamefont {Dai},\ and\
  \citenamefont {Fang}}]{Wang_Cd3As2}%
  \BibitemOpen
  \bibfield  {author} {\bibinfo {author} {\bibfnamefont {Z.}~\bibnamefont
  {Wang}}, \bibinfo {author} {\bibfnamefont {H.}~\bibnamefont {Weng}}, \bibinfo
  {author} {\bibfnamefont {Q.}~\bibnamefont {Wu}}, \bibinfo {author}
  {\bibfnamefont {X.}~\bibnamefont {Dai}}, \ and\ \bibinfo {author}
  {\bibfnamefont {Z.}~\bibnamefont {Fang}},\ }\href {\doibase
  10.1103/PhysRevB.88.125427} {\bibfield  {journal} {\bibinfo  {journal} {Phys.
  Rev. B}\ }\textbf {\bibinfo {volume} {88}},\ \bibinfo {pages} {125427}
  (\bibinfo {year} {2013})}\BibitemShut {NoStop}%
\bibitem [{\citenamefont {Nakosai}\ \emph {et~al.}(2012)\citenamefont
  {Nakosai}, \citenamefont {Tanaka},\ and\ \citenamefont {Nagaosa}}]{TSC_4}%
  \BibitemOpen
  \bibfield  {author} {\bibinfo {author} {\bibfnamefont {S.}~\bibnamefont
  {Nakosai}}, \bibinfo {author} {\bibfnamefont {Y.}~\bibnamefont {Tanaka}}, \
  and\ \bibinfo {author} {\bibfnamefont {N.}~\bibnamefont {Nagaosa}},\ }\href
  {\doibase 10.1103/PhysRevLett.108.147003} {\bibfield  {journal} {\bibinfo
  {journal} {Phys. Rev. Lett.}\ }\textbf {\bibinfo {volume} {108}},\ \bibinfo
  {pages} {147003} (\bibinfo {year} {2012})}\BibitemShut {NoStop}%
\bibitem [{\citenamefont {Mizushima}\ \emph {et~al.}(2014)\citenamefont
  {Mizushima}, \citenamefont {Yamakage}, \citenamefont {Sato},\ and\
  \citenamefont {Tanaka}}]{TSC_5}%
  \BibitemOpen
  \bibfield  {author} {\bibinfo {author} {\bibfnamefont {T.}~\bibnamefont
  {Mizushima}}, \bibinfo {author} {\bibfnamefont {A.}~\bibnamefont {Yamakage}},
  \bibinfo {author} {\bibfnamefont {M.}~\bibnamefont {Sato}}, \ and\ \bibinfo
  {author} {\bibfnamefont {Y.}~\bibnamefont {Tanaka}},\ }\href {\doibase
  10.1103/PhysRevB.90.184516} {\bibfield  {journal} {\bibinfo  {journal} {Phys.
  Rev. B}\ }\textbf {\bibinfo {volume} {90}},\ \bibinfo {pages} {184516}
  (\bibinfo {year} {2014})}\BibitemShut {NoStop}%
\bibitem [{\citenamefont {Brydon}\ \emph {et~al.}(2014)\citenamefont {Brydon},
  \citenamefont {Das~Sarma}, \citenamefont {Hui},\ and\ \citenamefont
  {Sau}}]{TSC_6}%
  \BibitemOpen
  \bibfield  {author} {\bibinfo {author} {\bibfnamefont {P.~M.~R.}\
  \bibnamefont {Brydon}}, \bibinfo {author} {\bibfnamefont {S.}~\bibnamefont
  {Das~Sarma}}, \bibinfo {author} {\bibfnamefont {H.-Y.}\ \bibnamefont {Hui}},
  \ and\ \bibinfo {author} {\bibfnamefont {J.~D.}\ \bibnamefont {Sau}},\ }\href
  {\doibase 10.1103/PhysRevB.90.184512} {\bibfield  {journal} {\bibinfo
  {journal} {Phys. Rev. B}\ }\textbf {\bibinfo {volume} {90}},\ \bibinfo
  {pages} {184512} (\bibinfo {year} {2014})}\BibitemShut {NoStop}%
\bibitem [{\citenamefont {Fu}\ and\ \citenamefont {Berg}(2010)}]{TSC_2}%
  \BibitemOpen
  \bibfield  {author} {\bibinfo {author} {\bibfnamefont {L.}~\bibnamefont
  {Fu}}\ and\ \bibinfo {author} {\bibfnamefont {E.}~\bibnamefont {Berg}},\
  }\href {\doibase 10.1103/PhysRevLett.105.097001} {\bibfield  {journal}
  {\bibinfo  {journal} {Phys. Rev. Lett.}\ }\textbf {\bibinfo {volume} {105}},\
  \bibinfo {pages} {097001} (\bibinfo {year} {2010})}\BibitemShut {NoStop}%
\bibitem [{\citenamefont {Sato}(2010)}]{TSC_3}%
  \BibitemOpen
  \bibfield  {author} {\bibinfo {author} {\bibfnamefont {M.}~\bibnamefont
  {Sato}},\ }\href {\doibase 10.1103/PhysRevB.81.220504} {\bibfield  {journal}
  {\bibinfo  {journal} {Phys. Rev. B}\ }\textbf {\bibinfo {volume} {81}},\
  \bibinfo {pages} {220504} (\bibinfo {year} {2010})}\BibitemShut {NoStop}%
\bibitem [{\citenamefont {Qi}\ \emph {et~al.}(2010{\natexlab{b}})\citenamefont
  {Qi}, \citenamefont {Hughes},\ and\ \citenamefont {Zhang}}]{TSC_1}%
  \BibitemOpen
  \bibfield  {author} {\bibinfo {author} {\bibfnamefont {X.-L.}\ \bibnamefont
  {Qi}}, \bibinfo {author} {\bibfnamefont {T.~L.}\ \bibnamefont {Hughes}}, \
  and\ \bibinfo {author} {\bibfnamefont {S.-C.}\ \bibnamefont {Zhang}},\ }\href
  {\doibase 10.1103/PhysRevB.81.134508} {\bibfield  {journal} {\bibinfo
  {journal} {Phys. Rev. B}\ }\textbf {\bibinfo {volume} {81}},\ \bibinfo
  {pages} {134508} (\bibinfo {year} {2010}{\natexlab{b}})}\BibitemShut
  {NoStop}%
\bibitem [{\citenamefont {Schnyder}\ \emph {et~al.}(2008)\citenamefont
  {Schnyder}, \citenamefont {Ryu}, \citenamefont {Furusaki},\ and\
  \citenamefont {Ludwig}}]{AZ_class}%
  \BibitemOpen
  \bibfield  {author} {\bibinfo {author} {\bibfnamefont {A.~P.}\ \bibnamefont
  {Schnyder}}, \bibinfo {author} {\bibfnamefont {S.}~\bibnamefont {Ryu}},
  \bibinfo {author} {\bibfnamefont {A.}~\bibnamefont {Furusaki}}, \ and\
  \bibinfo {author} {\bibfnamefont {A.~W.~W.}\ \bibnamefont {Ludwig}},\ }\href
  {\doibase 10.1103/PhysRevB.78.195125} {\bibfield  {journal} {\bibinfo
  {journal} {Phys. Rev. B}\ }\textbf {\bibinfo {volume} {78}},\ \bibinfo
  {pages} {195125} (\bibinfo {year} {2008})}\BibitemShut {NoStop}%
\bibitem [{\citenamefont {Qi}\ \emph {et~al.}(2009)\citenamefont {Qi},
  \citenamefont {Hughes}, \citenamefont {Raghu},\ and\ \citenamefont
  {Zhang}}]{pxpy}%
  \BibitemOpen
  \bibfield  {author} {\bibinfo {author} {\bibfnamefont {X.-L.}\ \bibnamefont
  {Qi}}, \bibinfo {author} {\bibfnamefont {T.~L.}\ \bibnamefont {Hughes}},
  \bibinfo {author} {\bibfnamefont {S.}~\bibnamefont {Raghu}}, \ and\ \bibinfo
  {author} {\bibfnamefont {S.-C.}\ \bibnamefont {Zhang}},\ }\href {\doibase
  10.1103/PhysRevLett.102.187001} {\bibfield  {journal} {\bibinfo  {journal}
  {Phys. Rev. Lett.}\ }\textbf {\bibinfo {volume} {102}},\ \bibinfo {pages}
  {187001} (\bibinfo {year} {2009})}\BibitemShut {NoStop}%
\bibitem [{\citenamefont {Fu}\ and\ \citenamefont
  {Kane}(2007)}]{Fu_Kane_criterion}%
  \BibitemOpen
  \bibfield  {author} {\bibinfo {author} {\bibfnamefont {L.}~\bibnamefont
  {Fu}}\ and\ \bibinfo {author} {\bibfnamefont {C.~L.}\ \bibnamefont {Kane}},\
  }\href {\doibase 10.1103/PhysRevB.76.045302} {\bibfield  {journal} {\bibinfo
  {journal} {Phys. Rev. B}\ }\textbf {\bibinfo {volume} {76}},\ \bibinfo
  {pages} {045302} (\bibinfo {year} {2007})}\BibitemShut {NoStop}%
\bibitem [{\citenamefont {Zhang}\ \emph {et~al.}(2013)\citenamefont {Zhang},
  \citenamefont {Kane},\ and\ \citenamefont {Mele}}]{33_from_Das}%
  \BibitemOpen
  \bibfield  {author} {\bibinfo {author} {\bibfnamefont {F.}~\bibnamefont
  {Zhang}}, \bibinfo {author} {\bibfnamefont {C.~L.}\ \bibnamefont {Kane}}, \
  and\ \bibinfo {author} {\bibfnamefont {E.~J.}\ \bibnamefont {Mele}},\ }\href
  {\doibase 10.1103/PhysRevLett.110.046404} {\bibfield  {journal} {\bibinfo
  {journal} {Phys. Rev. Lett.}\ }\textbf {\bibinfo {volume} {110}},\ \bibinfo
  {pages} {046404} (\bibinfo {year} {2013})}\BibitemShut {NoStop}%
\bibitem [{\citenamefont {Zhang}\ \emph {et~al.}(2012)\citenamefont {Zhang},
  \citenamefont {Kane},\ and\ \citenamefont {Mele}}]{54_from_Das}%
  \BibitemOpen
  \bibfield  {author} {\bibinfo {author} {\bibfnamefont {F.}~\bibnamefont
  {Zhang}}, \bibinfo {author} {\bibfnamefont {C.~L.}\ \bibnamefont {Kane}}, \
  and\ \bibinfo {author} {\bibfnamefont {E.~J.}\ \bibnamefont {Mele}},\ }\href
  {\doibase 10.1103/PhysRevB.86.081303} {\bibfield  {journal} {\bibinfo
  {journal} {Phys. Rev. B}\ }\textbf {\bibinfo {volume} {86}},\ \bibinfo
  {pages} {081303} (\bibinfo {year} {2012})}\BibitemShut {NoStop}%
\bibitem [{\citenamefont {Olde~Olthof}\ \emph {et~al.}(2021)\citenamefont
  {Olde~Olthof}, \citenamefont {Johnsen}, \citenamefont {Robinson},\ and\
  \citenamefont {Linder}}]{s_F_p}%
  \BibitemOpen
  \bibfield  {author} {\bibinfo {author} {\bibfnamefont {L.~A.~B.}\
  \bibnamefont {Olde~Olthof}}, \bibinfo {author} {\bibfnamefont {L.~G.}\
  \bibnamefont {Johnsen}}, \bibinfo {author} {\bibfnamefont {J.~W.~A.}\
  \bibnamefont {Robinson}}, \ and\ \bibinfo {author} {\bibfnamefont
  {J.}~\bibnamefont {Linder}},\ }\href {\doibase
  10.1103/PhysRevLett.127.267001} {\bibfield  {journal} {\bibinfo  {journal}
  {Phys. Rev. Lett.}\ }\textbf {\bibinfo {volume} {127}},\ \bibinfo {pages}
  {267001} (\bibinfo {year} {2021})}\BibitemShut {NoStop}%
\bibitem [{\citenamefont {Farrell}\ and\ \citenamefont {Bishop}(1989)}]{s_p}%
  \BibitemOpen
  \bibfield  {author} {\bibinfo {author} {\bibfnamefont {M.~E.}\ \bibnamefont
  {Farrell}}\ and\ \bibinfo {author} {\bibfnamefont {M.~F.}\ \bibnamefont
  {Bishop}},\ }\href {\doibase 10.1103/PhysRevB.40.10786} {\bibfield  {journal}
  {\bibinfo  {journal} {Phys. Rev. B}\ }\textbf {\bibinfo {volume} {40}},\
  \bibinfo {pages} {10786} (\bibinfo {year} {1989})}\BibitemShut {NoStop}%
\bibitem [{\citenamefont {Han}\ \emph {et~al.}(2021)\citenamefont {Han},
  \citenamefont {Ling}, \citenamefont {Liu}, \citenamefont {Li}, \citenamefont
  {Zhang},\ and\ \citenamefont {Wang}}]{s_p_Experiment}%
  \BibitemOpen
  \bibfield  {author} {\bibinfo {author} {\bibfnamefont {H.}~\bibnamefont
  {Han}}, \bibinfo {author} {\bibfnamefont {J.}~\bibnamefont {Ling}}, \bibinfo
  {author} {\bibfnamefont {W.}~\bibnamefont {Liu}}, \bibinfo {author}
  {\bibfnamefont {H.}~\bibnamefont {Li}}, \bibinfo {author} {\bibfnamefont
  {C.}~\bibnamefont {Zhang}}, \ and\ \bibinfo {author} {\bibfnamefont
  {J.}~\bibnamefont {Wang}},\ }\href {\doibase 10.1063/5.0051968} {\bibfield
  {journal} {\bibinfo  {journal} {Applied Physics Letters}\ }\textbf {\bibinfo
  {volume} {118}},\ \bibinfo {pages} {253101} (\bibinfo {year}
  {2021})}\BibitemShut {NoStop}%
\end{thebibliography}%





\ \\

\clearpage




\begin{widetext}
        \beginsupplement{}
        \setcounter{section}{0}
		 \setcounter{equation}{0}
        %\renewcommand{\thesubsection}{\arabic{subsection}}
        \renewcommand{\thesubsection}{\Alph{subsection}}
        \renewcommand{\thesubsubsection}{\alph{subsubsection}}
		 \renewcommand{\theequation}{S\arabic{equation}}
%\section*{Appendix}
\section*{SUPPLEMENTARY MATERIAL}




\subsection{A. Theory of edge states}
\label{edge theory}
\indent In order to solve a general theory for edge states, we have performed an Euler rotation to local coordinate of edge $\mathcal{L}(\theta)$. In principle, one can draw out the edge theory by directly solving the local Hamiltonian $\tilde{H}_\mathrm{low}(\bk')$ with open boundary conditions. However, $\tilde{H}_\mathrm{low}(\bk')$ generally has a complicated form in $\bk'$. Therefore, after the local coordinate rotation, we simultaneously apply a unitary transformation $\tilde{U}(\theta)=e^{i\omega_{\Delta}\tilde{\Gamma}_{51}/2}e^{-i\theta\tilde{\Gamma}_{51}/2}$ on $\tilde{H}_{\mathrm{low}}(\bk')$ such that $\tilde{H}'_{\mathrm{low}}(\bk',\theta)=\tilde{U}^\dagger(\theta) \tilde{H}_{\mathrm{low}}(\bk') \tilde{U}(\theta)=h_0+h_1$ has a simple form $\left(\mathrm{to}\ \mathcal{O}(k_1^2,k_2)\right)$,
	\begin{equation}\begin{aligned}
	&h_0(k_1)=
	(-M+Tk_1^2)\tilde{\Gamma}'_2
	-\vert\Delta\vert k_1\tilde{\Gamma}'_5,\\
	&h_1(\bk',\theta)=-\vert\Delta\vert k_2\tilde{\Gamma}'_1 - m(\theta)k_1^2\tilde{\Gamma}'_3.
	\end{aligned}\end{equation}
\noindent Here, $\vert\Delta\vert e^{i\omega_{\Delta}}$ $\equiv$ $\tilde{\Delta}_1+i\tilde{\Delta}_2$, $m(\theta)$=$\gamma_1\cos{2\theta}-\gamma_2\sin{2\theta}$. Now we solve for the zero mode equation of $h_0$ and consider $h_1$ as a perturbation to extract the edge state dispersion of $\mathcal{L}(\theta)$.
\\
\indent Replacing $k_1 \rightarrow -i\partial_{x_1}$ yields the zero mode equation $h_0(-i\partial_{x_1})\psi(x_1)=0$. With the boundary conditions $\psi(x_1$=$0)=\psi(x_1$=$-\infty)=0$, two exponentially localized solutions $\psi_{1,2}$ are found near $x_1$=$0$:
	\begin{equation}\begin{aligned}
	\psi_i=\mathcal{N} \sin\left(\lambda_2 x_1\right) e^{\lambda_1 x_1} e^{ik_2 x_2} \phi_i, \quad i=1,2,
	\end{aligned}\end{equation}
\noindent where the eigen constants are
	\begin{equation}\begin{aligned}
	\lambda_1=-\frac{\vert\Delta\vert}{2T}, \quad \lambda_2=\frac{\sqrt{4MT-\vert\Delta\vert^2}}{2T},
	\end{aligned}\end{equation}
\noindent with $\mathcal{N}=2\sqrt{-M\vert\Delta\vert/(4MT-\vert\Delta\vert^2)}$ the normalization factor. The spinor part $\phi_i$ are eigenstates of $\Gamma_{25}$ with eigenvalue $+1$:
	\begin{equation}\begin{aligned}
	\phi_1=(\begin{array}{cccc}0&i&0&1\end{array})^T, \quad
	\phi_2=(\begin{array}{cccc}-i&0&1&0\end{array})^T.
	\end{aligned}\end{equation}
\noindent Treating $h_1$ as a perturbation, the surperconducting edge dispersion for any boundary $\mathcal{L}(\theta)$ is described by $\left(H_{\mathcal{L}}\right)_{ij}=\int_{-\infty}^0d{x_1}\int_{-\infty}^{+\infty}dx_2\psi_i^\dagger h_1 \psi_j$, thus
	\begin{equation}\begin{aligned}
	&H_{\mathcal{L}}(k_2,\theta) = \vert\Delta\vert k_2\varsigma_z + m_{\mathrm{eff}}(\theta)\varsigma_y,
	\end{aligned}\end{equation}
	\begin{equation}\label{meff}\begin{aligned}
	&m_{\mathrm{eff}}(\theta) = \frac{2(\tilde{t}_{\II}-\tilde{t}_{\I})(\tilde{\alpha}\Delta_1\cos{2\theta}-\tilde{\beta}\Delta_2\sin{2\theta})}{(\tilde{t}_{\II}+\tilde{t}_{\I})(t_a-t_b-\eta^2/\tilde{t}_{\I})}.
	\end{aligned}\end{equation}
\noindent $\varsigma$ is in the sapce spanned by $\phi_{1,2}$. The four-fold rotation operator on the edge is $C_{4z,\mathcal{L}}(\theta)=-i\sigma_z$. As a check, it acts on the matrix part of the edge mass term as $C_{4z,\mathcal{L}}(\theta)\varsigma_y	C^\dagger_{4z,\mathcal{L}}(\theta)=-\varsigma_y$, and thus for the scalar part there is $m(\theta+\pi/2)=-m(\theta)$. Therefore, the sign flip of the effective mass field between $\mathcal{L}(\theta)$ and $\mathcal{L}(\theta+\pi/2)$ is protected by the four-fold rotation.
\\
\indent To consider the stability of second-order Majorana zero modes, the phenomenologically $C_{4z}$-breaking term $\delta\sin{k_z}\Gamma_3$ adding to $H(\bkp,k_z)$ (Eq.~1 in the main text) endows a constant tensor $m_\delta\varsigma_y$ to the edge dispersion $H_{\mathcal{L}}(k_2,\theta)$, which explicitly is that
\begin{equation}\label{mdelta}\begin{aligned}
	m_\delta=2\delta\Delta_1\frac{\eta^2(\tilde{t}_{\I}^2-\tilde{t}_{\II}^2)-2\tilde{t}_{\I}^2(t_a\tilde{t}_{\I}-t_b\tilde{t}_{\II}-\eta^2)}{\tilde{t}_{\I}^2(\tilde{t}_{\I}+\tilde{t}_{\II})^2(t_a-t_b-\eta^2/\tilde{t}_{\I})}\sin{k_z}.
\end{aligned}\end{equation}
\noindent Coupling to the anisotropic mass field $m_{\mathrm{eff}}(\theta)$, $m_\delta$ endows an extra constant gap for any edge $\mathcal{L}(\theta)$, which pushes the second-order Majorana zero modes to annihilate each other at the edges $\mathcal{L}(\theta=0,\pi)$ or $\mathcal{L}(\theta=\frac{\pi}{2},\frac{3\pi}{2})$.








\subsection{B. Topological characterization with nested Wilson loop}
\label{WL}

\begin{figure}[b!]
	\centering
	\includegraphics[scale=0.44]{figS1.pdf}
	\caption{(a,~b) Energy dispersion in a wire geometry along $k_z$ with open boundary along both $x$ and $y$. The hinge localized Majorana states are highlighted in red. (c,~d) Eigen value of Wilson loop operator of all occupied BdG bands of $H(\bkp,k_z)$ (Eq.~1 in the main text) along $k_x$. (e,~f) Dots: Eigen value of nested Wilson loop operator of all occupied wannier bands in (c,~d) along $k_y$. Lines: corner charge $Q_c$. $D_1$ and $D_2$ are the location of SC DPs. (a-c) are for type-$\I$ DSMs. (d-f) are for type-$\II$ DSMs. In the nested Wilson loop calculation, the gapless $k_z$-slices in (b,~e) have been taken out. Note that here we have set the $B_{2u}$ pairing channel to be zero, $\Delta_2=0$.}
	\label{figS1}
\end{figure}

In this section, we discuss the topological characterization of the hinge Majorana states in our model using the concept of nested Wilson loop~\cite{quadrupole}. A full BdG Hamiltonian is diagonalized,
\begin{equation}
    H = \sum_{\mu,\bk}\gamma_{\mu \bk}^\dagger E_{\mu \bk}\gamma_{\mu \bk},
\end{equation}
where $\gamma_{\mu \bk}^\dagger=\sum_\alpha u_{\mu \bk}^\alpha c_{\alpha \bk}^\dagger$, $\alpha=1,2,...,N_{\mathrm{orb}}$. $N_{\mathrm{orb}}$ is the number of orbitals. To calculate the Wilson loop along $k_x$ direction, we can construct the overlap matrix $F_x(k_y,k_z)$ using the occupied bands,
\begin{equation}
    F_{x,k_x^j}^{mn}(k_y,k_z)=\sum_\alpha u^{\alpha*}_{m,\bk^j} u^\alpha_{n,\bk^{j+1}}
\end{equation}
where $k_{x}^{j} = 2{\pi}j/N_x$ and $m, n = 1, 2, ... , N_{\rm occ}$, with $N_{\rm occ}$ is the number of occupied bands. Define $\bk'\equiv(k_y,k_z)$, hence,
\begin{equation}
    \mathcal{W}_{x}(\bk') = \prod_{j = 0}^{N - 1} F_{x,k_x^j}(\bk').
\end{equation}
Extracting the phase part, define a Wannier Hamiltonian $\mathcal{W}_{x}(\bk') = e^{iH_{\mathcal{W}_x}(\bk')}$ and diagonalize it, $H_{\mathcal{W}_x}(\bk')=2\pi \sum_{n=1}^{N_\mathrm{occ}}\xi^\dagger_{n,\bk'}\nu_x^n(\bk')\xi_{n,\bk'}$, with $\xi^\dagger_n(\bk')=\sum_{m=1}^{N_\mathrm{occ}}\tilde{u}^m_{n,\bk'}\gamma^\dagger_{m,\bk'}$. In the following Wannier band subsapces, $\xi^\dagger_n(\bk')=\sum_{\alpha=1}^{N_{\mathrm{orb}}}w^\alpha_{n\bk'}c^\dagger_{\alpha\bk'}$, with components $w^\alpha_{n\bk'}=\sum_{m=1}^{N_\mathrm{{occ}}}\tilde{u}^m_{n,\bk'}u^\alpha_{m,\bk'}$, the nested Wilson loop is similarly defined by constructed 
\begin{equation}
	\tilde{F}_{xy,k_y^j}^{mn}(k_z)=\sum_\alpha w^{\alpha*}_{m,\bk'^j} w^\alpha_{n,\bk'^{j+1}}
\end{equation}
where $k_{y}^{j} = 2{\pi}j/N_y$ and $m, n = 1, 2, ... , N_{\rm P}$. In our case, $N_{\rm P}=2$. And
\begin{equation}
    \mathcal{W}_{xy}(k_z) = \prod_{j = 0}^{N_y - 1} \tilde{F}_{xy,k_y^j}(k_z).
\end{equation}
Thus the final Berry phase of nested Wilson loop $p_{xy}$ is obtained by diagonalize $\mathcal{W}_{xy}(k_z)=\sum_{n=1}^{N_{{\rm P}}}\rho^\dagger_{n,k_z}e^{i2\pi p_{xy}^{n}(k_z)}\rho_{n,k_z}$. From nested Wilson loop, the equivalent ``corner charge" can be characterized as $Q_c=\sum_n p_{xy}^n\ {\rm mod}\ 1$.

In Fig.~\ref{figS1}, we have plotted the hinge states, eigen value of (nested) Wilson loop and corner charge for type-$\I$ (a-c) and type-$\II$ (d-f) DSMs.

In type-$\I$ DSMs, It is shown that the hinge Majorana states appear, linking the projection of two bulk SC DPs ($\bk=(0,0,D_{1,2})$). Accordingly, taking out two SC DPs, the eigen value of Wilson loop $v_x$ in $k_y$-$k_z$ plane shows a full gap, except $(k_y,k_z)=(0,0)$ point, at which the pumping indicate a set of mirror-symmetry protected Majorana surface modes discussed in Ref.~\cite{Sato_TCSC}. Meanwhile, the corner charge is characterized by the nested Wilson loop, showing the nontrivial Wannier charge center $Q_c=0.5$ between two bulk SC DPs.

In type-$\II$ DSMs, on the contrary, the bulk SC DPs are annihilating each other and creating a full bulk gap. Therefore, the intact hinge Majorana flat bands extends across the whole hinge Brillouin zone, linking the projection of the surface helical Majorana states. Correspondingly, the eigen value of Wilson loop $v_x$ shows $\mathbb{Z}_2$ pumpings at $(k_y,k_z)=(0,0/\pi)$ and is gapped out elsewhere. Meanwhile, taking out two ill-defined points ($k_z=0/\pi$), each $k_z$-slice shows an nontrivial corner charge $Q_c=0.5$, consisting to the hinge Majorana states.






\clearpage

\end{widetext}




\end{document}
