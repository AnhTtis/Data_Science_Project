\documentclass[prl,aps,amssymb,twocolumn,superscriptaddress]{revtex4-1}
%\documentclass[prl,twocolumn,superscriptaddress]{revtex4}
\usepackage{amsmath}
\usepackage{amssymb}
\usepackage{amsthm}
\usepackage{amsfonts}
\usepackage{listings}

\usepackage{physics}
\usepackage{multirow}
\usepackage{diagbox}

\lstloadlanguages{Matlab}
%\usepackage{algorithmic}
\usepackage{enumerate}
\usepackage{latexsym}
%\usepackage[dvips]{graphicx}
\usepackage{color}
%\usepackage{xcolor}
\usepackage{bm}
\usepackage{hyperref}
\hypersetup{
 pdfnewwindow=true, colorlinks=true,
 linkcolor=blue, anchorcolor=blue,
 citecolor=blue, filecolor=blue,
 menucolor=blue, urlcolor=blue}

\newcommand{\beginsupplement}{%
        \setcounter{table}{0}
        \renewcommand{\thetable}{S\arabic{table}}%
        \setcounter{figure}{0}
        \renewcommand{\thefigure}{S\arabic{figure}}%
     }

\usepackage{psfrag}

\usepackage{bm}
%\usepackage[pdftex]{graphics}
\usepackage{graphicx}
\usepackage{subfigure}


%DIF <
%DIF PREAMBLE EXTENSION ADDED BY LATEXDIFF
%DIF UNDERLINE PREAMBLE %DIF PREAMBLE
\RequirePackage[normalem]{ulem} %DIF PREAMBLE
\RequirePackage{color}\definecolor{RED}{rgb}{1,0,0}\definecolor{BLUE}{rgb}{0,0,1} %DIF PREAMBLE
\providecommand{\DIFadd}[1]{{\protect\color{blue}\uwave{#1}}} %DIF PREAMBLE
\providecommand{\DIFdel}[1]{{\protect\color{red}\sout{#1}}}                      %DIF PREAMBLE
%%DIF SAFE PREAMBLE %DIF PREAMBLE
%\providecommand{\DIFaddbegin}{} %DIF PREAMBLE
%\providecommand{\DIFaddend}{} %DIF PREAMBLE
%\providecommand{\DIFdelbegin}{} %DIF PREAMBLE
%\providecommand{\DIFdelend}{} %DIF PREAMBLE
%%DIF FLOATSAFE PREAMBLE %DIF PREAMBLE
%\providecommand{\DIFaddFL}[1]{\DIFadd{#1}} %DIF PREAMBLE
%\providecommand{\DIFdelFL}[1]{\DIFdel{#1}} %DIF PREAMBLE
%\providecommand{\DIFaddbeginFL}{} %DIF PREAMBLE
%\providecommand{\DIFaddendFL}{} %DIF PREAMBLE
%\providecommand{\DIFdelbeginFL}{} %DIF PREAMBLE
%\providecommand{\DIFdelendFL}{} %DIF PREAMBLE
%DIF END PREAMBLE EXTENSION ADDED BY LATEXDIFF

%\newcommand{\beq}{\begin{equation}}
%\newcommand{\eneq}{\end{equation}}
% boldsymbol (requires amsmath)
%\newcommand{\bs}[1]{\boldsymbol{#1}}
%\newcommand{\bal}{\begin{align}}
\newcommand{\imp}{\;\;\Rightarrow\;\;}


\newcommand{\sy}{\sigma_y}
\newcommand{\hatti}{\hat{T}_i}
\newcommand{\hattj}{\hat{T}_j}
\newcommand{\hattone}{\hat{T}_1}
\newcommand{\hatttwo}{\hat{T}_2}
\newcommand{\hattthree}{\hat{T}_3}


% A command for inner product and bras and kets
%\newcommand{\braket}[2]{\left\langle #1 | #2 \right\rangle}
%\newcommand{\bra}[1]{\left\langle#1\right|}
%\newcommand{\ket}[1]{\left|#1\right\rangle}
%\newcommand{\bigket}[1]{\bigl|#1\bigr\rangle}
%\newcommand{\textket}[1]{|#1\rangle}

% Various bracketing commands
\newcommand{\of}[1]{\!\left(#1\right)}
\newcommand{\sqof}[1]{\left[#1\right]}
\newcommand{\cuof}[1]{\left\{#1\right\}}

% insert text in equation
\newcommand{\ins}[1]{\;\;\text{#1}\;\;}

% commutator and anticommutator
%\newcommand{\comm}[2]{\left[#1,#2\right]}
%\newcommand{\anticomm}[2]{\left\{#1,#2\right\}}
% sum on nearest neighbor bonds
\newcommand{\bond}{\left\langle i, j \right\rangle}
%\newcommand{\bondsum}{\sum_{\left\langle i, j \right\rangle}}
\newcommand{\nbond}{\left\langle\left\langle i, j \right\rangle\right\rangle}
% 1/2
\newcommand{\half}{$\frac{1}{2}$ }
% simplifies using the up and down arrows to denote spin
\newcommand{\up}{\uparrow}
\newcommand{\dw}{\downarrow}
\newcommand{\bb}{{\bf b}}
\newcommand{\bk}{{\vb* k}}
\newcommand{\br}{{\vb* r}}
\newcommand{\bkp}{{\vb* k_{\parallel}}}
\newcommand{\bkpsq}{{\vb* k_\parallel^2}}

% Theta function
\newcommand{\tfunc}{\vartheta_1}
% notation for vacuum, an empty set inside a ket
\newcommand{\vac}{\left|\,0\,\right\rangle}
% Absolute value
%\newcommand{\abs}[1]{\left|#1\right|}

% Roman functions for real and imaginary parts
\newcommand{\re}{\mathrm{Re}}
\newcommand{\im}{\mathrm{Im}}

% Sets of up-spin and down-spin locations
\newcommand{\bket}{\left\{z_1 \cdots z_{\num}\right\}}
\newcommand{\wket}{\left\{w_1 \cdots w_{\num}\right\}}

%Expectation values
\newcommand{\expect}[1]{\left\langle#1\right\rangle}

% reference with parenthesis
\newcommand{\pref}[1]{(\ref{#1})}

%for type II article
\newcommand{\I}{\mathrm{\uppercase\expandafter{\romannumeral1}}}
\newcommand{\II}{\mathrm{\uppercase\expandafter{\romannumeral2}}}
\newcommand{\III}{\mathrm{\uppercase\expandafter{\romannumeral3}}}
\newcommand{\IV}{\mathrm{\uppercase\expandafter{\romannumeral4}}}
\newcommand{\bw}[1]{\textcolor{blue}{\uwave{#1}}}
\newcommand{\rs}[1]{\textcolor{red}{\sout{#1}}}

\def \tic{Cu$_3$SbS$_4$}
\def \wsmc{Cu$_2$ZnGeSe$_4$}
\def \ti{Case I}
\def \wsm{Case II}

\def\ie{{\it i.e.},\ }
\def\eg{{\it e.g.}\ }
\def\ea{{\it et al.}}
\input{epsf}

\begin{document}

\tolerance 10000

\newcommand{\vk}{{\bf k}}

\draft

\title{Hybrid Topological Superconductivity \\and Hinge Majorana Flat Band in Type-II Dirac Semimetals}

%in English titles articles and words like to, on, at etc are always spelled with small letters
%\author{Goodguys}
%\affiliation{Beijing National Laboratory for Condensed Matter Physics,
%and Institute of Physics, Chinese Academy of Sciences, Beijing 100190, China}
%\affiliation{University of Chinese Academy of Sciences, Beijing 100049, China}

\author{Yue Xie}
%\thanks{These authors contributed equally to this work.}
\affiliation{Beijing National Laboratory for Condensed Matter Physics,
and Institute of Physics, Chinese Academy of Sciences, Beijing 100190, China}
\affiliation{University of Chinese Academy of Sciences, Beijing 100049, China}

\author{Xianxin Wu}
 \affiliation{CAS Key Laboratory of Theoretical Physics, Institute of Theoretical Physics,
Chinese Academy of Sciences, Beijing 100190, China}

\author{Zhong Fang}
\affiliation{Beijing National Laboratory for Condensed Matter Physics,
and Institute of Physics, Chinese Academy of Sciences, Beijing 100190, China}
\affiliation{University of Chinese Academy of Sciences, Beijing 100049, China}

\author{Zhijun Wang}
\email{wzj@iphy.ac.cn}
\affiliation{Beijing National Laboratory for Condensed Matter Physics,
and Institute of Physics, Chinese Academy of Sciences, Beijing 100190, China}
\affiliation{University of Chinese Academy of Sciences, Beijing 100049, China}

%\pacs{03.67.Mn, 05.30.Pr, 73.43.-f}

\begin{abstract}
Type-II Dirac semimetals (DSMs) have a distinct Fermi surface topology, which allows them to host novel topological superconductivity (TSC) different from type-I DSMs. Depending on the relationship between intra- and inter-orbital electron-electron interactions, the phase diagram of superconductivity is obtained in type-II DSMs. We find that when the inter-orbital attraction is dominant, an unconventional inter-orbital intra-spin superconducting (SC) state ($B_{1u}$ and $B_{2u}$ pairing channels of $D_{4h}$ point group) is realized, yielding hybrid TSC, \ie first- and second-order TSC exists at the same time. Further analysis reveals the Majorana flat bands on the $z$-directed hinges, which penetrate through the whole hinge Brillouin zone and link the projections of the surface helical Majorana cones at time-reversal-invariant momenta. These higher-order hinge modes are symmetry-protected  and can even host strong stability against finite $C_{4z}$ rotation symmetry-breaking order. We suggest that experimental realization of these findings can be explored in transition metal dichalcogenides.
\end{abstract}

\maketitle






%\section{Introduction}
\paragraph*{Introduction.---}
Majorana fermions, quasiparticle excitations in topological superconductors, are known for their non-Abelian statistics~\cite{NA1,NA2,NA3,NA4,NA5} and potential use in fault-tolerant quantum computing~\cite{FT1,FT2,FT3,FT4,FT5}. Their unique properties have generated significant interest in realizing Majoranas in various topological materials~\cite{Fu_Kane,Alicea,Qi_Wen,Qi,SrRuO,WS1,WS2,TSC_6}, including type-I DSMs~\cite{Sato_TCSC,Node_vortex,HOTDS}. Recently, a class of superconducting~\cite{PdTe2_SC,PdTe2_SC2,PdTe2_SC3,NiTe2_SC,NiTe2_SC2,proximity_SC,IrTe2,IrTe2_2,DingHongnew} transition metal dichalcogenides, $X$Te$_2$ ($X=$ Pd, Ni, Ir), has been reported to be type-II DSMs~\cite{PdTe2,PtTe2,TMD,NiTe2,DingHong}. As the Lorentz-violating velocity tilting the Dirac cone from type I to type II, the Fermi surface experiences a Lifshitz transition from a closed sphere to an open pocket~\cite{Type_II}. This transition provides distinct Fermi surface physics and thus makes type-II DSMs have potential for realizing novel TSC different from other topological materials, which is lacking of investigations.
\begin{figure}[!t]
	\centering
	\includegraphics[trim=0 0 0 0, scale=0.44]{Figs/fig1.pdf}
	\caption{(a-b) Schematic plot of hinge Majorana flat bands (red lines) in type-$\I$ (a) and type-$\II$ (b) DSMs. Fermi surfaces of normal state is schematically plotted (yellow pockets). Bulk superconducting Dirac points (dots) appear in type-$\I$ DSMs, while full BdG gap appears in type-$\II$ DSMs. (c-d) Typical in-plane/out-of-plane dispersion of a type-$\I$ (c) or type-$\II$ (d) DSM, with Dirac points locating at $\pm \bk^D=\left(0,0,\pm\frac{\pi}{2}\right)$. Dashed line denotes Fermi level. Total angular momenta are labeled with $J=1/2\ or\ 3/2$. }
	\label{fig1}
\end{figure}

Higher-order topology~\cite{HOTI,secondorder,d-2}, which manifests as localized states (or charges) at $(d$$-$$m)$-dimensional ($m$$\geqslant$$ 2$) boundaries in a $d$-dimensional material, has expanded the classification of topological phases of matter. Currently, three mechanisms are known to characterize higher-order topological phases: (i) Wannier center configurations that give rise to corner-induced filling anomalies~\cite{d-2,quadrupole,BBHPRB,Ben,WC1,WC2,WC3,WC4,quadrupolar_semimetal,GuoZhaopeng}, (ii) the existence of boundary-localized mass domains protected by crystalline symmetries~\cite{secondorder,d-2,HOTI,CS1,CS2,CS3,CS4,2013HOTI} and (iii) higher-order multipole winding number defined in real space~\cite{NW1,WN2}. Besides, higher-order topology is an exciting area of research that intersects with superconductivity, which was first introduced in Ref.~\cite{secondorder}. Numerous theoretical studies have investigated higher-order TSC~\cite{Po_1,Po_2,theory_1,theory_2,theory_3,theory_4,theory_5,theory_6,theory_7,theory_8,theory_9,theory_10,theory_11,Dirac_SC,HOtypeI,PhysRevResearch.2.043155,Li_2022,PhysRevB.105.195149}, and some of realistic models have been proposed in intrinsic materials~\cite{Das_Sarma,TRI_TSC}, by proximity effect~\cite{MKP}, with magnetic configuration~\cite{BHZ_TSC}. However, the study for higher-order TSC in type-II DSMs is absent.

In this letter, we explore the possibility of TSC in a type-II DSM. Depending on the relationship between intra- and inter-orbital electron-electron interactions, we revealed  a novel phase diagram of superconductivity. When the inter-orbital attraction is dominant, an unconventional odd-parity superconducting pairing is realized. Explicit analysis shows that through such pairing, a type-II DSM becomes a hybrid topological superconductor, hosting first- and second-order TSC at the same time. It is a 2D first-order time-reversal invariant topological superconductor for the $k_z = 0/\pi$ slice, whereas for the $k_z$ slices other than 0 or $\pi$, they belong to 2D second-order topological superconductors. Our numerical simulations demonstrate the presence of a hinge Majorana flat band in the type-II DSM, which extends through the entire hinge Brillouin zone, as shown in Fig.~\ref{fig1}(b). Additionally, we investigate the SC nodal structure and the symmetry-breaking instability of the hinge Majorana modes. 







%\section{Model Hamiltonian}
\paragraph*{Model Hamiltonian.---}The two-orbital model Hamiltonian describes a time-reversal ($\mathcal{T}$) invariant DSM on a tetragonal lattice of the $D_{4h}$ symmetry in Bogoliubov-de Gennes (BdG) redundancy:
	\begin{equation}\label{Hk}\begin{aligned}
	H(\bk) =\ & \left(\begin{array}{cc}
		H_0(\bk)-\mu & D  \\
		D^\dagger & \mu-H_0^*(-\bk)   \\
	\end{array}\right),\\
	H_0(\bk)=\ &[t_{\II}\cos{k_z}-t_{b}\left( 2-\cos{k_x}-\cos{k_y} \right)]\sigma_0s_0 \\
\ &+ [t_{\I}\cos{k_z}-t_{a}\left( 2-\cos{k_x}-\cos{k_y} \right)]\sigma_zs_0 \\
\ &+ \eta\sin{k_x}\sigma_xs_z - \eta\sin{k_y}\sigma_ys_0\\
\ &+ \alpha\sin{k_z} \left( \cos{k_y}-\cos{k_x} \right)\sigma_xs_x\\
\ &+ \beta\sin{k_z}\sin{k_x}\sin{k_y}\sigma_xs_y.
	\end{aligned}\end{equation}
	\noindent The normal state $H_0$ describes a typical DSM on the basis of total angular momenta $J=\{3/2,\ 1/2\}$ with different parities~\cite{Wang_Na3Bi,Wang_Cd3As2}. The charge-conjugate operator is $\mathcal{C}=\tau_x\mathcal{K}$. $\tau$, $\sigma$ and s are Pauli matrices describing the particle-hole, orbital and spin degrees of freedom, respectively.

The typical in-plane/out-of-plane dispersions of the type-$\I$ and type-$\II$ DSM, with Dirac points locating on $k_z$ axis, are shown in Figs.~\ref{fig1}(c) and \ref{fig1}(d). In simplicity, we first set the chemical potential to be at the level of the Dirac points (\ie $\mu=0$). Particularly, $t_{\II}$ tilts the Dirac cones. If $\vert t_{\II}\vert \textgreater \vert t_{\I}\vert$, Dirac cones are strongly tilted to be of type-$\II$. Otherwise, they are of type-I. Meanwhile, for a fixed-$k_z$ slice with $k_z=0/\pi$, odd (even) number of $\mathcal{PT}$-pairs of FS sheets appear in type-$\II$ (type-$\I$) DSMs. This FS geometry difference is crucial to the TSC.
\begin{table}[b]
\begin{center}
	\begin{tabular}{|c|c|c|c|c|c|c|}\hline
        Pairing channel & Matrix & Irreps. & $\ \mathcal{P}\ $ & $C_{4z}$ & $\mathcal{M}_z$ & $\mathcal{M}_x$ \\ \hline
		  $c_{1\uparrow}c_{1\downarrow}+c_{2\uparrow}c_{2\downarrow}$ & $D_{1a}=\Delta_{1a}\sigma_0s_y$ & \multirow{2}{*}{$A_{1g}$} & $+$ & $+$ & $+$ & $+$ \\
		  $c_{1\uparrow}c_{1\downarrow}-c_{2\uparrow}c_{2\downarrow}$ & $D_{1b}=\Delta_{1b}\sigma_zs_y$ &  & $+$ & $+$ & $+$ & $+$ \\
        \hline
		  $c_{1\uparrow}c_{2\uparrow}+c_{1\downarrow}c_{2\downarrow}$ & $D_{2}=\Delta_{2}\sigma_ys_0$ & $B_{1u}$ & $-$ & $-$ & $-$ & $-$ \\
        \hline
        $c_{1\uparrow}c_{2\uparrow}-c_{1\downarrow}c_{2\downarrow}$ & $D_{3}=\Delta_3\sigma_ys_z$ & $B_{2u}$ & $-$ & $-$ & $-$ & $+$ \\
        \hline
		  $c_{1\uparrow}c_{2\downarrow}-c_{1\downarrow}c_{2\uparrow}$ & $D_{4a}=\Delta_{4a}\sigma_xs_y$ & \multirow{2}{*}{$E_{u}$} & $-$ & \multirow{2}{*}{\diagbox{\ }{\ }} & $+$ & $+$ \\
        $c_{1\uparrow}c_{2\downarrow}+c_{1\downarrow}c_{2\uparrow}$ & $D_{4b}=\Delta_{4b}\sigma_ys_x$ &  & $-$ &  & $+$ & $-$ \\
        \hline
    \end{tabular}
    \caption{Possible pairing potential and their symmetry properties of $D_{4h}$ point group.}\label{irreps}
\end{center}
\end{table}

\begin{table}[b!]
\begin{center}
\begin{ruledtabular}
	\begin{tabular}{ccccccccc}
        $\ t_\I\ $ & $t_\II$ & $t_a$ & $t_b$ & $\eta$ & $\alpha$ & $\beta$ & $\Delta_2$ & $\Delta_3$ \\ \hline
        1 & 1.5 & 2 & 0 & 2 & 1.2 & 1.2 & 1.2 & 0.2 \\
    \end{tabular}
\end{ruledtabular}
    \caption{Parameters of the two-orbital model.}\label{parameter}
\end{center}
\end{table}




\begin{figure}[!t]
	\centering
	\includegraphics[trim=40 0 0 0, scale=0.29]{Figs/fig2.pdf}
	\caption{(a)~Phase diagram of the superconducting pairing channel. In the blue region, the conventional $A_{1g}$ pairing, $\Delta_1$ is favored when intra-orbital interaction $U$ is dominate. In the red region, type-II DSMs favor the topological superconducting $B_{1u}$ ($B_{2u}$) pairing channels, $\Delta_2$ ($\Delta_3$) when inter-orbital interaction $V$ is sufficiently dominate. The normal phase is determined with critical temperature below $5\cross10^{-3}K$. (b)~Critical temperature of different superconducting pairing potentials versus inter-orbital interaction $V$. The strength of intra-orbital interaction is set, $U=1.1$. The critical temperature of $E_u$ pairing is much lower than that of $B_{1u}/B_{2u}$ pairing. And note that $T_c(B_{1u})\approx T_c(B_{2u})$.}
	\label{fig2}
\end{figure}




\begin{figure*}[!t]
	\centering
	\includegraphics[scale=0.55]{Figs/fig3.pdf}
	\caption{(a) Bulk BdG dispersion in weak pairing limit ($\Delta_2,\Delta_3\rightarrow0$, left panels) and with finite pairing strength (right panels). Top panels show that in a type-I DSM, two SC DPs appear with $\mathcal{Q}(\I)=\{\mathcal{Q}_{1/2},\mathcal{Q}_{3/2}\}=(-1,-1)$ and $\mathcal{Q}_t(\I)=-2$. Bottom panels show that in a type-II DSM, SC DPs annihilate each other to create full BdG bulk gap, with $\mathcal{Q}(\II)=\{\mathcal{Q}_{1/2},\mathcal{Q}_{3/2}\}=(1,-1)$ and $\mathcal{Q}_t(\II)=0$. (b) Spectrum of the $N_x=40$ slab model with open boundary in $x$ direction. Surface helical Majorana states appear at $\Gamma'$ and $Z'$ (red dots). The states highlighted in blue are localized to the surface. (c) Energy dispersion in a wire geometry along $z$ direction with open boundary in both $x$ and $y$ (40 lattice sites along each direction). The zero modes inside the surface gap gather to form the intact hinge Majorana flat bands (highlighted in red), linking the projections of surface cones. Upper-right inset: total wave function distribution of the four-degenerate SOMZMs of a fixed $k_z=\pi/4$ slice (yellow cross in (c)), which is localized to the four corners.}%corner Majorona state in type II regime, wave function of corner Majorona state in type II regime, Surface and corner state of the 2D model. Schematic plot around kc type I, schematic plot around kc type II, Boundary states of diffenrent kinds. }
	\label{fig3}
\end{figure*}




\paragraph*{SC pairing channels.---}In this section, we will discuss all possible $k$-independent pairing potentials in type-II DSM~\cite{TSC_4,dopedDSM,TSC_2}. Now considering the following short-range density-density interaction in the two-orbital model,
\begin{equation}
    H_{int}(x)=-U[n_1^2(x)+n_2^2(x)]-2Vn_1(x)n_2(x)
\end{equation}
where $U$ and $V$ describe intra- and inter-orbital interactions, respectively. In the mean-field level, there are six possible pairing potentials that satisfy the Fermi-Dirac statistics. Their irreducible representations and symmetry properties are listed in Table~\ref{irreps}. Among them, the $A_{1g}$ pairing channel is of even-parity and intra-orbital, while other pairing channels ($B_{1u}$, $B_{2u}$ and $E_u$) are all odd-parity and inter-orbital pairings.\\
\indent We numerically solve the linearized gap equation (see Section A of the Supplementary Material (SM)) to obtain critical temperatures for the possible pairing channels and the results are shown in Fig.~\ref{fig2}. As is shown in the U-V phase diagram of Fig.~\ref{fig2}(a), the conventional $A_{1g}$ pairing, $\Delta_{1a/1b}$ is favored when intra-orbital interaction $U$ is dominant (green area). Otherwise, when inter-orbital interaction $V$ is sufficiently dominant (red area), type-II DSMs favor the superconducting $B_{1u}$ ($B_{2u}$) pairing channels, $\Delta_2$ ($\Delta_3$). Additionally, the Coulomb repulsion can lead to stronger $V$ mediating by phonons~\cite{TSC_6}.\\
\indent In Fig.~\ref{fig2}(b), we plot the critical temperature of different superconducting pairing potentials versus inter-orbital interaction $V$. One can see that the critical temperature of $\Delta_2$ and $\Delta_3$ pairing channels are very close. Therefore, without loss of generality, in the later discussions about TSC, we consider both of them at the same time and take the following form of pairing function,
\begin{equation}\label{paring}\begin{aligned}
	D=i\Delta_2\sigma_ys_0+\Delta_3\sigma_ys_z.
\end{aligned}\end{equation}
This allows us to discuss the TSC of both $B_{1u}$ and $B_{2u}$ channels in a unify framework in the following.







\paragraph*{First-order TSC and helical Majorana surface states for $k_z=0/\pi$.---} 
Due to the odd-parity nature of $B_{1u}$ and $B_{2u}$ pairing, $\mathcal{P}D\mathcal{P}^T$=$-D$ ($\mathcal{P}=\sigma_z$ is the inversion operator), the TSC can be characterized by the number of FS (Kramers degenerate) pairs of the normal state~\cite{TSC_1,TSC_2,TSC_3}. Here in our model, for the $\mathcal{T}$-invariant $k_z=0/\pi$ slice, in type-II (type-I) DSM region, an odd (even) number of pairs of FS sheets that enclose $\mathcal{T}$-invariant momenta result in a nontrivial (trivial) $\mathbb{Z}_2$ phase of class DIII~\cite{AZ_class}. This makes the $k_z=0/\pi$ slice of a type-II DSM become a first-order $\mathcal{T}$-invariant 2D topological superconductor. Consequently, surface helical Majorana states are found on (100) and (010) surfaces, as shown in Fig.~\ref{fig3}(b) at $\Gamma'$ and $Z'$.





\paragraph*{Second-order TSC and hinge Majorana flat bands.---}For those 2D $k_z$-slices other than $0\ or\ \pi$, without time-reversal symmetry, gaps are introduced to their surface states and trivialize the first-order topology. Nevertheless, we find that they all holds 2D second-order TSC, and consequently, second-order Majorana zero modes (SOMZMs) emerge in a square geometry. These SOMZMs of different $k_z$ slices assemble to form intact Majorana flat bands at the $z$-directed hinges of a squared rod geometry. These hinge Majorana flat bands penetrate through the whole hinge Brillouin zone and link the projections of the surface helical Majorana cones at $\mathcal{T}$-invariant points, as shown in Fig.~\ref{fig3}(c). The topological characterization for second-order TSC using the nested Wilson loop method is discussed in the SM, section D.




\paragraph*{Low-energy analysis.---} The hybrid TSC can be understood uniformly by a low-energy analysis. We consider every $k_z$ slice as a 2D subsystem individually, and for a given $k_z$, we extract a low-energy four-band description near the Fermi level (see SM, Section B for details):
\begin{widetext}
\begin{equation}\label{massful_pxpy}\begin{aligned}
	H^{2D}_{k_z}(\bkp) &= \left(\begin{array}{cccc}
		-M(\bkp,k_z) & 0 & - \Delta_-k_- & S(\bkp,k_z) \\
		0 & -M(\bkp,k_z) & S(\bkp,k_z) &  \Delta_+k_+ \\
		- \Delta_+k_+ & S(\bkp,k_z) & M(\bkp,k_z) & 0 \\
		S(\bkp,k_z) &  \Delta_-k_- & 0 & M(\bkp,k_z) \\
		\end{array}\right),\\
\text{with}\ M(\bkp&,k_z)=(t_\I- t_{\II})\cos k_z - \frac{t_{a}-t_{b}-\eta^2}{2t_{\I}\cos k_z}\bkpsq,\ \Delta_\pm=\eta\frac{\Delta_{2}\pm i\Delta_3}{t_{\I}\cos k_z}.
\end{aligned}\end{equation}
\end{widetext}
Here, $k_\pm=k_x\pm ik_y$ and $\bkp=(k_x,k_y)$. The low-energy space is spanned by pseudo-particle-hole and pseudo-spin degrees of freedom ($\tilde{\tau}\otimes\tilde{s}$). This low-energy effective Hamiltonian is a spinful $p_x$+$ip_y$-like model~\cite{pxpy}, with an additional $\mathcal{T}$-breaking term $S(\bkp,k_z)=-\sin k_z\left[ \gamma_2(k_x^2-k_y^2)+2\gamma_3k_xk_y \right]$, with $\gamma_2=\frac{\alpha\Delta_2}{(t_\I+ t_{\II})\cos k_z}$ and $\gamma_3=\frac{\beta\Delta_3}{(t_\I+ t_{\II})\cos k_z}$.

For $\mathcal{T}$-invariant $k_z=0/\pi$ slice, $S(\bkp,k_z)=0$ and Eq.~(\ref{massful_pxpy}) gives rise to the first-order TSC. For the $k_z\neq 0/\pi$ slice, $S(\bkp,k_z)\neq0$ and the edge states of the 2D model are generally gapped. However The gaps are anisotropic and result in second-order TSC. To visualize the anisotropic gaps, we deduce the effective edge Hamiltonian of a termination labeled by general direction $\theta$ in a dish geometry (see Fig.~\ref{fig4}(a) and details in Section C of SM)~\cite{33_from_Das,54_from_Das}:
\begin{equation}\label{edge}\begin{aligned}
	&H_{\mathcal{L}}(k_2,\theta) = \vert\Delta\vert k_2\varsigma_z + m_{\mathrm{eff}}(\theta)\varsigma_y.
\end{aligned}\end{equation}
\noindent $\varsigma$ is pseudospin localized to the edge. $k_2$ is the momentum perpendicular to the edge in the local coordinate. $\vert \Delta \vert = \sqrt{\Delta_+\Delta_-}$. The effective pairing gap on edge $\mathcal{L}(\theta)$ is $m_{\mathrm{eff}}(\theta) \propto \alpha\Delta_2\cos{2\theta}+\beta\Delta_3\sin{2\theta}$. One can see that the angular anisotropy of $m_{\mathrm{eff}}(\theta)$ enables four nontrivial criticle edges $\mathcal{L}(\theta_c)$ where $m_{\mathrm{eff}}$ flips its sign while crossing $\theta_c$, resulting in four-fold degenerate SOMZMs.

Notably, the existence of the four mass-sign flipping domainwall of edges is protected by four-fold rotation symmetry. Specifically, the four-fold rotation operator on the edge is $C_{4z,\mathcal{L}}(\theta)=-i\varsigma_z$ and the generally allowing mass term is $m_1(\theta)\varsigma_x+m_2(\theta)\varsigma_y$. Since the matrix part satisfies $C_{4z,\mathcal{L}}(\theta)\varsigma_j	C^\dagger_{4z,\mathcal{L}}(\theta)=-\varsigma_j$ for either $j=x$ or $y$, the scalar part must have $m_{1,2}(\theta+\pi/2)=-m_{1,2}(\theta)$. Therefore, the four-fold rotation enforces the existence of four critical edges with vanishing mass terms, corresponding to the four-fold degenerate SOMZMs.









\paragraph*{SC nodal structure.---}Next, we will discuss the SC nodal structure in a generic DSM through pairing of Eq.~(\ref{paring}). Around the Dirac point (DP) $\bk_D$, the effective Hamiltonian reads $H_0^D(\delta\bk)=(- t_\II \delta k_z-\mu)\sigma_0s_0 - t_\I \delta k_z\sigma_zs_0 + \eta( \delta k_x\sigma_xs_z - \delta k_y\sigma_ys_0 )$, $\delta\bk=\bk-\bk_D$. With only $\Delta_2$ pairing, band crossings appear on $k_z$ axis when $(t_\II/t_\I)^2$$<$$1+(\mu/\Delta_2)^2$. This shows that band crossings always exist in type-$\I$ region, and full BdG gap appears only in type-$\II$ region when DPs are closed enough to the FS (\eg NiTe$_2$/IrTe$_2$, 20 meV above the FS~\cite{NiTe2,TMD,DingHong}).
\\
\indent Odd-$C_{4z}$ pairing nature (Table~\ref{irreps}) make $B_{1u}$ and $B_{2u}$ channels carry an angular momentum $\Delta J=2$, so that the crossing of a $\mathcal{P}\mathcal{T}$-pair of normal-state bands with its own charge-conjugate counterparts along $\Gamma$-$Z$ will be superconducting Dirac points (SC DPs), protected by four-fold rotation $C_{4z}^\mathrm{BdG}=\mathrm{diag}\left(C_{4z},-C_{4z}^*\right)$. In our model, with joint symmetry $\mathcal{P}\mathcal{T}$ and weak pairing assumption presented, an SC DP with monopole charge $\mathcal{Q}_J/\abs{\mathcal{Q}_J}$ can be characterized by,
	\begin{equation}\begin{aligned}\label{monopole}
	\mathcal{Q}_J=(1-J)(N_J^\Gamma-N_J^Z).
	\end{aligned}\end{equation}
\noindent $N_J^{\Gamma(Z)}$ is the number of occupied normal-state bands  at $\Gamma$ $(Z)$ with total angular momentum $J$. Hence in our model, $\mathcal{Q}=\{\mathcal{Q}_{1/2},\mathcal{Q}_{3/2}\}$ characterizes the SC nodal structure.

As shown in the top panels of Fig~\ref{fig3}(a), for a type-I DSM, along $\Gamma$-$Z$ line of the BdG band dispersion, there are two SC DPs, which are characterized by $\mathcal{Q}(\I)=\{-1,-1\}$. The number of total stable SC DPs equals to $\mathcal{Q}_t=\sum_{J}\vert\mathcal{Q}_J\vert$. Here, $\mathcal{Q}_t(\I)=-2$ for type-$\I$ DSM~\cite{Sato_TCSC}. Hinge Majorana flat bands are found between the projection of the bulk SC DPs (see Fig.~\ref{fig1}(a))~\cite{HOtypeI}. In the bottom panels of Fig.~\ref{fig3}(a), in type-II region, the SC DPs are characterized by $\mathcal{Q}(\II)=\{1,-1\}$, with $\mathcal{Q}_t(\II)=0$. Hence, they annihilate each other to form a full BdG gap. These annihilating SC DPs (termed as inherited nodes~\cite{inherited}) can be viewed as the origin of the hybrid TSC in type-II DSMs~\cite{Ben}.





\begin{figure}[!tb]
	\centering
	\includegraphics[trim=2 140 0 120, scale=0.29]{Figs/fig4.pdf}
	\caption{(a) Schematic plot of a circular geometry with arbitrary edge $\mathcal{L}(\theta)$ (purple tangent line). In the red (blue) region, the effective edge mass $m_{\mathrm{eff}}(\theta)$ is positive (negative). The green dots denote the SOMZMs. Against $C_{4z}$ instability, they will move along the circle and finally annihilate at the edges $\mathcal{L}(\theta=0,\pi)$ or $\mathcal{L}(\theta=\frac{\pi}{2},\frac{3\pi}{2})$, as indicated by the green arrows. (b) Evolution of the $x$- and $y$-edge gaps of the $k_z=\pi/4$ slice against $C_{4z}$ instability strength $\delta$ (in the unit of $\Delta_1$). The yellow region denotes the nontrivial second-order TSC phase with the presence of corner Majorana zero modes.}
	\label{fig4}
\end{figure}





%\section{Stability of higher-order MBSs}
\paragraph*{Stability of Majorana hinge modes.---}Unconventional superconductivity is always accompanied with other symmetry-breaking orders, such as nematicity, spin or charge density wave, etc~\cite{IrTe2,IrTe2_2}. These symmetry lowering orders can stablize the SC phase by increasing the condensation energy, and always break discrete rotation symmetries. Therefore, we consider the instability of the SOMZMs against $C_{4z}$-breaking order in the 2D model with a given $k_z$ by phenomenologically introducing $\delta\sin{k_z}\sigma_xs_x$ to Eq.~(\ref{Hk}). This results in a constant gap $m_\delta\propto\delta\Delta_2\sin{k_z}$ to the edge dispersion of Eq.~(\ref{edge}), coupling to the anisotropic one $m_{\mathrm{eff}}(\theta)$. As $\delta$ is turned on, $C_{4z}$ is broken and mirror symmetries $\mathcal{M}_x$ and $\mathcal{M}_y$ drive two pairs of the SOMZMs (green dots in Fig.~\ref{fig4}(a)) to move towards and finally annihilate with each other at the edges $\mathcal{L}(\theta=0,\pi)$ or $\mathcal{L}(\theta=\frac{\pi}{2},\frac{3\pi}{2})$. Fig.~\ref{fig4}(b) shows energy gaps of $x$- and $y$-edges for $k_z=\pi/4$ slice as a function of $\delta$ (in the unit of $\Delta_2$). Yellow shaded block shows the higher-order nontrivial region. Due to $m_\mathrm{eff}(\theta)\propto\alpha\Delta_2\cos{2\theta}+\beta\Delta_3\sin{2\theta}$, the $C_{4z}$ stability of SOMZMs  linearly depend on the spin-orbit coupling strength ($\alpha$ and $\beta$). Therefore, A wide range of hinge Majorana stability can be manipulated by enhancing the spin-orbit coupling.









%\section{Summary and material realization}
\paragraph*{Discussion.---} In summary, we find that when the inter-orbital interaction sufficiently dominates over the intra-orbital one, an unconventional inter-orbital intra-spin superconducting phase is realized in a type-II DSM, giving rise to a hybrid topological superconductor. It hosts surface helical Majorana states and hinge Majorana flat bands at the same time. In terms of the material realization, although the $X$Te$_2$ lattice is hexagonal and belongs to point group $D_{3d}$, the type-$\II$ DPs in $X$Te$_2$ are still described under the basis $\ket{J=3/2,~1/2}$~\cite{TMD,NiTe2}. Therefore, the continuous model of the point group $D_{3d}$ is exactly the same as the one of $D_{4h}$, so that the same results are expected. Additionally, Refs.~\cite{s_F_p,s_p,s_p_Experiment} show that in an superconductor junction, the low $T_c$ of an odd-parity superconductor can even be raised up to 1.5 times. In conclusion, we suggest that the surface helical Majorana states and the intact hinge Majorana flat bands demonstrated in this letter is experimentally accessible in $X$Te$_2$ materials.











\ \\
\noindent \textbf{Acknowledgments}
This work was supported by the National Natural Science Foundation of China (Grants No. 11974395, No. 12188101 and No. 52188101), the Strategic Priority Research Program of Chinese Academy of Sciences (Grant No. XDB33000000), and the Center for Materials Genome.



\bibliography{Refs}





\ \\

\clearpage




\begin{widetext}
        \beginsupplement{}
        \setcounter{section}{0}
		 \setcounter{equation}{0}
        %\renewcommand{\thesubsection}{\arabic{subsection}}
        \renewcommand{\thesubsection}{\Alph{subsection}}
        \renewcommand{\thesubsubsection}{\alph{subsubsection}}
		 \renewcommand{\theequation}{S\arabic{equation}}
%\section*{Appendix}
\section*{SUPPLEMENTARY MATERIAL}




\subsection{A. Linearized gap equation of the possible pairing channels}
In this section, we are going to demonstrate the linearized gap equation for the possible pairing channels. First recall the main Hamiltonian of type-II Dirac semimetal in the main text and define some abbreviations, 
	\begin{equation}\label{HinS}\begin{aligned}
	H_0(\bk)&=[t_{\II}\cos{k_z}-t_{b}\left( 2-\cos{k_x}-\cos{k_y} \right)]\sigma_0s_0
+ [t_{\I}\cos{k_z}-t_{a}\left( 2-\cos{k_x}-\cos{k_y} \right)]\sigma_zs_0 \\
&\quad+ \eta\left( \sin{k_x}\sigma_xs_z - \sin{k_y}\sigma_ys_0 \right)
+ \sin{k_z} [\alpha\left( \cos{k_y}-\cos{k_x} \right)\sigma_xs_x + \beta\sin{k_x}\sin{k_y}\sigma_xs_y].\\
& \equiv -\mu_\II(\bk)\Gamma_0+a(\bk)\Gamma_1+b(\bk)\Gamma_2+c(\bk)\Gamma_3+d(\bk)\Gamma_4+e(\bk)\Gamma_5
	\end{aligned}\end{equation}
$\Gamma_0$ is $4\times 4$ identity and $\Gamma_1=\sigma_z\otimes s_0$, $\Gamma_2=\sigma_x\otimes s_z$, $\Gamma_3=\sigma_y\otimes s_0$, $\Gamma_4=\sigma_x\otimes s_x$, $\Gamma_5=\sigma_x\otimes s_y$, $\Gamma_{ij}=\left[\Gamma_i,\Gamma_j\right]/2i$. $\sigma$ and s are Pauli matrices describing the orbital and spin degrees of freedom, respectively. The system has time reversal ($\mathcal{T}$), inversion ($\mathcal{P}$), mirror ($\mathcal{M}_z$) and four-fold rotation ($C_{4z}$) symmetries as below,
	\begin{equation}\begin{aligned}
	\mathcal{T}=-is_y\mathcal{K},\quad \mathcal{P}=\sigma_z,\quad
	\mathcal{M}_z=-is_z,\quad C_{4z}=\exp\left[i(\pi/4)(2+\sigma_z)\otimes s_z\right].
	\end{aligned}\end{equation}
The bands energy dispersion is $E_\pm(\bk)=\pm E_0(\bk)-\mu_\II(\bk)$ with $E_0=\sqrt{a^2+b^2+c^2+d^2+e^2}$. It is clear that the parameter associate with the identical term, $\mu_\II(\bk)=-(t_{\II}\cos{k_z}-t_{b}\left( 2-\cos{k_x}-\cos{k_y} \right))$, can be regarded as a $\bk$-dependent chemical potential.

Considering the following short-range density-density interaction,
\begin{equation}
    H_{int}(x)=-U[n_1^2(x)+n_2^2(x)]-2Vn_1(x)n_2(x)
\end{equation}
where $U$ and $V$ are intra- and inter-orbital interaction strength. $n_\sigma=n_{\sigma\uparrow}+n_{\sigma\downarrow}$ and $\sigma=1,2$ denotes orbitals. For our two-orbital model, in the mean-field level, there are six possible pairing potentials that satisfy the Fermi-Dirac statistics. Expressed in the combinations of two Pauli matrices, they are $D_{1a}=\Delta_{1a}\sigma_0s_y$, $D_{1b}=\Delta_{1b}\sigma_zs_y$, $D_{2}=\Delta_{2}\sigma_ys_0$, $D_{3}=\Delta_{3}\sigma_ys_z$, $D_{4a}=\Delta_{4a}\sigma_xs_y$, $D_{4b}=\Delta_{4b}\sigma_ys_x$. Here, $\Delta_{1a,1b}$, $\Delta_2$, $\Delta_3$ and $\Delta_{4a,4b}$ are of $A_{1g}$, $B_{1u}$, $B_{2u}$ and $E_u$ pairing channels.

Applying the linearized gap equation to possible pairing channels, we have for the $A_{1g}$ channel,
\begin{equation}\label{lge1}
    \left|\begin{array}{cc} U\chi_{1a}-1 & U\chi_{1ab} \\ U\chi_{1ab} & U\chi_{1b}-1 \end{array}\right|=0,
\end{equation}
and for other channels ($B_{1u}$, $B_{2u}$, $E_g$)
\begin{equation}\label{lge2}
    V\chi_j = 1,
\end{equation}
with $j=2,3,4a,4b$. Here, $\chi_j$'s are finite temperature superconducting susceptibilities in different pairing channels. By defining the single-particle normal-state Green function $G_0(\bk,i\omega_n)=\sum_{m=\pm}P_m(k)/\{i\omega_n-[mE_0(k)-\mu_{\II}(k)]\}$, with the projection operator $P_{+(-)}$ onto the higher (lower) energy bands,
\begin{equation}
    P_\pm(\bk)=\frac{1}{2}(1\pm\frac{H_0(\bk)+\mu_\II(\bk)}{ E_0(\bk)}),
\end{equation}
the superconducting susceptibility is written as,
\begin{equation}\label{susceptibility}\begin{aligned}
    \chi_j &=\frac{1}{2N\beta_c}\sum_k\sum_{i\omega_n}\mathrm{Tr}\{M_\Delta G_0(\bk,i\omega_n)M_{\Delta'} G_0^T(-\bk,-i\omega_n)\}\\
		&=\frac{1}{N}\sum_{m=\pm}\sum_k F_j^m(\bk)\frac{\tanh{\frac{1}{2}\beta_c[mE_0(\bk)-\mu_\II(\bk)]}}{2[mE_0(\bk)-\mu_\II(\bk)]}.
\end{aligned}\end{equation}
Here, $M_\Delta$ is the matrix part of the pairing potential and $\beta_c=1/k_BT_c$. Note that type-$\II$ Dirac semimetals are different from type-$\I$ that they have both electron and hole pockets centering around center and boundary of the Brillouin zone, saparately contributed by higher and lower energy bands. This gives rise to the summation $m=\pm$ in Eq.~(\ref{susceptibility}). $F_j^\pm(\bk)$ are form factors and is given by $F_{1a}^\pm(\bk)=1$, $-F_{1ab}^-(\bk) = F_{1ab}^+(\bk) = \frac{a(k)}{E_0(k)}$, $F_{1b}^\pm(\bk)=\frac{a^2(k)}{E_0^2(k)}$, $F_2^\pm(\bk)=\frac{b^2(k)+c^2(k)+d^2(k)}{E_0^2(k)}$, $F_3^\pm(\bk)=\frac{b^2(k)+c^2(k)+e^2(k)}{E_0^2(k)}$, $F_{4a}^\pm(\bk)=\frac{b^2(k)+d^2(k)+e^2(k)}{E_0^2(k)}$ and $F_{4b}^\pm(\bk)=\frac{c^2(k)+d^2(k)+e^2(k)}{E_0^2(k)}$. From the form factors, it is obvious that $\chi_{2,3}>\chi_{4a,4b}$ since $d^2(\bk)$ and $e^2(\bk)$ are $k^3$ terms, so that the $E_u$ channel will not dominant for any choices of U and V. Around Fermi surface, take $E_0(\bk)\approx \mu_\II(\bk)$, and in the U-V phase diagram, the phase boundary between $D_1$ and $D_{2,3}$ is approximately given by~\cite{dopedDSM},
\begin{equation}\begin{aligned}
    \frac{V}{U}=\int_{FS\pm}d\bk\frac{\mu^2_\II(\bk)+a^2(\bk)}{\mu_\II^2(\bk)-a^2(\bk)}.
\end{aligned}\end{equation}
For type-I DSM, the phase boundary is approximitely $V/U\approx 2.1$~\cite{dopedDSM}. For type-II Dirac semimetal, at the limit $t_\II\gg t_\I$ and thus $\mu^2_\II(k)\gg a^2(k)$, the $B_{1u}/B_{2u}$ region will be broaden to $V/U\approx 1$. The critical temperatures of different pairing channels can be solved numerically using Eq.~(\ref{lge1} ,\ref{lge2}).

Additionally, From the form factor, we note that $F_{2,3}^\pm$ vanishes at ($k_x=k_y=0$), which are the polar points of the Fermi surfaces. Therefore, these $\bk$ points do not participate in pairing and they will become gapless nodal points in BdG band dispersion if the pairing is weak. However, as we shall see, these nodal points have different monopole charges and thus will annihilate each other to give rise to a fullgap superconductor.







\subsection{B. Low-energy analysis}
In this section, we are going to extract a low-energy Hamiltonian around the Fermi surfaces to describe the hybrid topological superconductivity. For simplicity, we would investigate each $k_z$-fixed slice individually. With fixed $k_z$, the continuous 2D-slice Hamiltonian is, 
	\begin{equation}\begin{aligned}\label{continuous}
	&H^{\mathrm{2D}}(\bkp) = \left(\begin{array}{cc}
		H_0^{\mathrm{2D}}(\bkp)-\mu & D  \\
		D^\dagger & \mu-H_0^{\mathrm{2D}*}(-\bkp)   \\
	\end{array}\right),\\
	 &H_0^{\mathrm{2D}}(\bkp) = \tilde{t}_{\II}\Gamma_0 + \tilde{t}_{\I}\Gamma_1 - \frac{1}{2}(k_x^2+k_y^2)(t_{b}\Gamma_0+t_{a}\Gamma_1) \\&\qquad\qquad+ \eta(k_x\Gamma_2-k_y\Gamma_3) + \frac{1}{2}\tilde{\alpha}(k_x^2-k_y^2)\Gamma_4+\tilde{\beta}k_xk_y\Gamma_5,\\
	&\tilde{t}_{\II,\I}=t_{\II,\I}\cos k_z,\ \tilde{\alpha}=\alpha\sin{k_z},\ \tilde{\beta}=\beta\sin{k_z}.
	\end{aligned}\end{equation}
\noindent  Here, $\bkp=(k_x,k_y)^T$. We now take $\tilde{t}_{\I,\II}>0$ without loss of generality. In a system with helical texture, the low energy downfolding using helical basis $\psi_k=(c_{\uparrow k}+e^{i\theta_k}c_{\downarrow k})/\sqrt{2}$ yields effective typical topological superconductor model~\cite{Fu_Kane,Alicea}. Here in our model, the Fermi surfaces of the normal state have helical orbit-momentum locking, as shown in Fig.~\ref{figS1}(a). When orbit-momentum locking is strong enough ($\eta^2$$>$$\tilde{t}_\I\vert t_a$$-$$t_b\vert$) and assuming weak pairing assumption ($k_z \simeq 0,\pm\pi$), a pair of Kramers degenerate FS sheets can be very well described by a low-energy analysis by apply unitary transformation $U(\bkp)\equiv(\Psi_\bk,\sigma_x\Psi^*_{-\bk})$, with $\Psi_\bk$ for low-energy states and $\sigma_x\Psi^*_{-\bk}$ for high-energy states. However, such transformation in our model is complicated, and it reads:
	\begin{equation}\begin{aligned}
	\Psi_\bk &= \left(\begin{array}{cccc} \varphi_\bk & -is_y\varphi^*_{-\bk} & \tau_x\varphi^*_{-\bk} & -i\tau_xs_y\varphi_{\bk} \end{array}\right),\\
	\varphi_\bk &= \left(\begin{array}{cccccccc} 
		A_\bk & 0 & B_\bk & 0 & 0 & 0 & 0 & 0 
	\end{array}\right)^T/\sqrt{\mathcal{N}_\bk}.
	\end{aligned}\end{equation}
\noindent Here, $B_\bk=2\tilde{t}_{\I}-t_{a}\bkpsq+\sqrt{(-2\tilde{t}_{\I}+t_{a}\bkpsq)^2+4\eta^2\bkpsq}$ and $A_\bk=-2\eta(k_x+ik_y)$, with the normalization factor $\mathcal{N}_\bk$. $\tau$ is Pauli matrix describing the particle-hole degree of freedom. Note that $-is_y\varphi^*_{-\bk}$ is the time-reversal (TR) counterpart of $\varphi_\bk$ and $\tau_x\varphi^*_{-\bk}$ is the charge-conjugate counterpart of $\varphi_\bk$. Now, the effective Hamiltonian is
	\begin{equation}\begin{aligned}
	&H_{\mathrm{eff}}(\bkp)
	= U^\dagger H^{\mathrm{2D}}(\bkp) U
	= \left(\begin{array}{cc}
		H_l(\bkp) & H_s(\bkp) \\
		H^\dagger_s(\bkp) & H_h(\bkp)
	\end{array}\right).\\
	&H_l(\bkp) = \left(\begin{array}{cccc}
		-M + T\bkpsq & 0 & - \tilde{\Delta}_-k_- & 0 \\
		0 & -M + T\bkpsq & 0 &  \tilde{\Delta}_+k_+ \\
		- \tilde{\Delta}_+k_+ & 0 & M - T\bkpsq & 0 \\
		0 &  \tilde{\Delta}_-k_- & 0 & M - T\bkpsq \\
		\end{array}\right)
	\end{aligned}\end{equation}
\noindent Here, $H_l(\bkp)$ is the effective low-energy Hamiltonian, with $M=\tilde{t}_{\I}-\tilde{t}_{\II}$, $T=\left(t_{a}-t_{b}-\eta^2/\tilde{t}_{\I}\right)/2$, $\tilde{\Delta}_\pm=\eta(\Delta_{2}\pm i\Delta_3)/\tilde{t}_{\I}$ and $k_\pm=k_x\pm ik_y$. Low-energy space is spanned by $\tilde{\tau}\otimes\tilde{s}$, respectively describing pseudo-particle-hole and -spin degrees of freedom. The low-energy physics is clearly a TR-invariant $p_x$+$ip_y$-like model~\cite{pxpy}, originating from both orbital texture and inter-orbital pairing. By Fu-Kane criterion~\cite{Fu_Kane_criterion}, edge-localized states always exist for there is one pair of FS sheets in a type-II DSM. For the TR-invaiant slice ($k_z=0\ or \ \pi$), nontrivial $\mathbb{Z}_2$ phase ensures such localized states to be helical Majorana states.
\\
\indent For $k_z\neq 0\ or\ \pi$ slices, however, due to the lack of time reversal symmetry that square to $-1$, the SOC term will gap out the edge states. More specifically, one can perform a higher-order perturbation by introducing scattering with high-energy bands $H_h$ through Green's function $G(\bkp,\omega)=( \omega-H_{\mathrm{eff}})^{-1}$. The correction to $H_l$ is $H'_l(\bkp,\omega)=H_s\left(\omega-H_h\right)^{-1}H^\dagger_s$. Choose $\omega=0$ as an approximation, the corrected low-energy bands becomes,
	\begin{equation}\begin{aligned}\label{fullHeff}
	H_{k_z}^{2D}(\bkp) = H_l(\bkp)+H_l'(\bkp,0).
	\end{aligned}\end{equation}
\noindent $H'_l(\bkp,0)=-\left[ \gamma_1(k_x^2-k_y^2)+2\gamma_2k_xk_y \right]\Gamma_4$ is anti-diagonal, with $\gamma_1=\tilde{\alpha}\Delta_2/(\tilde{t}_{\I}+\tilde{t}_{\II})$ and $\gamma_2=\tilde{\beta}\Delta_3/(\tilde{t}_{\I}+\tilde{t}_{\II})$. One can see that the $\bk$-independent pairing $D$ enters into the low-energy bands mediating through SOC as an \textit{inherited pairing} $H'_l(\bkp,0)$, \ie an effective pairing inherits the functional distribution of the SOC term, and becomes distributed. Interestingly, $B_{1u}$ and $B_{2u}$ pairing channels enter the low-energy bands separately through $\alpha$ and $\beta$ terms. Additionally, it can be seen from Eq.~(\ref{fullHeff}) that the pairing potential vanishes on $\Gamma$-$Z$, and it will leave two nodes on the Fermi surface ungapped at weak pairing limit. However, as discussed in the main text, the nodes are of opposite monopole charges and will vanish eventually when increasing pairing strength. This is consistent with the form factors (Eq.~\ref{susceptibility}) obtained from the linearized gap equation.



\begin{figure}[t!]
	\centering
	\includegraphics[scale=0.34]{Figs/figS1.pdf}
	\caption{(a) Orbital texture on Fermi surface with fixed-$k_z$ slices for spin-up (left) and spin-down (right) sectors. Arrows on Fermi surfaces indicate the orbit-momentum texture ($\ev{\sigma_x},\ev{\sigma_y}$). (b) Schematic plot of a circular geometry with arbitrary edge $\mathcal{L}(\theta)$ (purple tangent line). In the red (blue) region, the effective edge mass $m_{\mathrm{eff}}(\theta)$ is positive (negative). The yellow dots denote the SOMZMs. Against $C_{4z}$ instability, they will move along the circle and finally annihilate at the edges $\mathcal{L}(\theta=0,\pi)$ or $\mathcal{L}(\theta=\frac{\pi}{2},\frac{3\pi}{2})$, as indicated by the yellow arrows. }
	\label{figS1}
\end{figure}






\subsection{C. Theory of edge states}
\label{edge theory}
\indent To understand the higher-order topological superconductivity, we deduce an theory of edge states to the low-energy effective Hamiltonian, Eq.~(\ref{fullHeff}). We first notice that any edge termination of a 2D silce can be described as a tangent line of the unit circle~\cite{33_from_Das,54_from_Das}, as shown in Fig.~\ref{figS1}(b). An arbitrary edge $\mathcal{L}(\theta)$ can be uniquely labeled by its unit normal vector defined as $\textbf{n}_{\mathcal{L}}=\left(\cos\theta,\sin\theta\right)^T$.  To solve edge dispersion for $\mathcal{L}(\theta)$, we apply Euler rotation around $z$ axis $R_Z(-\theta)$ to local coordinates, \ie $\bk'\equiv\left(k_1,k_2\right)^T=R_Z(-\theta)\left(k_x,k_y\right)^T$. Noticing that $k_1=\textbf{n}_{\mathcal{L}}\cdot\bkp$, now one can solve edge theory on arbitrary $\mathcal{L}(\theta)$ by imposing open boundary conditions on $H^{2D}_{k_z}(\bk')$ along the $k_1$ direction. However, $H^{2D}_{k_z}(\bk')$ generally has a complicated form in $\bk'$. Therefore, after the local coordinate rotation, we simultaneously apply a unitary transformation $\tilde{U}(\theta)=e^{-i\omega_{\Delta}\tilde{\Gamma}_{23}/2}e^{-i\theta\tilde{\Gamma}_{23}/2}$ on $H^{2D}_{k_z}(\bk')$ such that $\tilde{H}^{2D}_{k_z}(\bk',\theta)=\tilde{U}^\dagger(\theta) H^{2D}_{k_z}(\bk') \tilde{U}(\theta)=h_0+h_1$ has a simple form $\left(\mathrm{to}\ \mathcal{O}(k_1^2,k_2)\right)$,
	\begin{equation}\begin{aligned}
	&h_0(k_1)=
	(-M+Tk_1^2)\tilde{\Gamma}'_1
	-\vert\Delta\vert k_1\tilde{\Gamma}'_2,\\
	&h_1(\bk',\theta)=-\vert\Delta\vert k_2\tilde{\Gamma}'_3 - m(\theta)k_1^2\tilde{\Gamma}'_4.
	\end{aligned}\end{equation}
\noindent Here, $\vert\Delta\vert e^{i\omega_{\Delta}}$ $\equiv$ $\eta(\Delta_2+i\Delta_3)/\tilde{t_I}$, $m(\theta)$=$\gamma_1\cos{2\theta}+\gamma_2\sin{2\theta}$. Now we solve for the zero mode equation of $h_0$ and consider $h_1$ as a perturbation to extract the edge state dispersion of $\mathcal{L}(\theta)$.
\\
\indent Replacing $k_1 \rightarrow -i\partial_{x_1}$ yields the zero mode equation $h_0(-i\partial_{x_1})\psi(x_1)=0$. With the boundary conditions $\psi(x_1$=$0)=\psi(x_1$=$-\infty)=0$, two exponentially localized solutions $\psi_{1,2}$ are found near $x_1$=$0$:
	\begin{equation}\begin{aligned}
	\psi_i=\mathcal{N} \sin\left(\lambda_2 x_1\right) e^{\lambda_1 x_1} e^{ik_2 x_2} \phi_i, \quad i=1,2,
	\end{aligned}\end{equation}
\noindent where the eigen constants are
	\begin{equation}\begin{aligned}
	\lambda_1=-\frac{\vert\Delta\vert}{2T}, \quad \lambda_2=\frac{\sqrt{4MT-\vert\Delta\vert^2}}{2T},
	\end{aligned}\end{equation}
\noindent with $\mathcal{N}=2\sqrt{-M\vert\Delta\vert/(4MT-\vert\Delta\vert^2)}$ the normalization factor. The spinor part $\phi_i$ are eigenstates of $\Gamma_{12}$ with eigenvalue $+1$:
	\begin{equation}\begin{aligned}
	\phi_1=(\begin{array}{cccc}0&i&0&1\end{array})^T, \quad
	\phi_2=(\begin{array}{cccc}-i&0&1&0\end{array})^T.
	\end{aligned}\end{equation}
\noindent Treating $h_1$ as a perturbation, the surperconducting edge dispersion for any boundary $\mathcal{L}(\theta)$ is described by $\left(H_{\mathcal{L}}\right)_{ij}=\int_{-\infty}^0d{x_1}\int_{-\infty}^{+\infty}dx_2\psi_i^\dagger h_1 \psi_j$, thus
	\begin{equation}\begin{aligned}
	&H_{\mathcal{L}}(k_2,\theta) = \vert\Delta\vert k_2\varsigma_z + m_{\mathrm{eff}}(\theta)\varsigma_y,
	\end{aligned}\end{equation}
	\begin{equation}\label{meff}\begin{aligned}
	&m_{\mathrm{eff}}(\theta) = \frac{2(\tilde{t}_{\II}-\tilde{t}_{\I})(\tilde{\alpha}\Delta_2\cos{2\theta}+\tilde{\beta}\Delta_3\sin{2\theta})}{(\tilde{t}_{\II}+\tilde{t}_{\I})(t_a-t_b-\eta^2/\tilde{t}_{\I})}.
	\end{aligned}\end{equation}
\noindent $\varsigma$ is in the sapce spanned by $\phi_{1,2}$. The four-fold rotation operator on the edge is $C_{4z,\mathcal{L}}(\theta)=-i\varsigma_z$. As a check, it acts on the matrix part of the edge mass term as $C_{4z,\mathcal{L}}(\theta)\varsigma_y	C^\dagger_{4z,\mathcal{L}}(\theta)=-\varsigma_y$, and thus for the scalar part there is $m(\theta+\pi/2)=-m(\theta)$. Therefore, the sign flip of the effective pairing gap on edges between $\mathcal{L}(\theta)$ and $\mathcal{L}(\theta+\pi/2)$ is protected by the four-fold rotation.
\\
\indent To consider the stability of second-order Majorana zero modes, the phenomenologically $C_{4z}$-breaking term $\delta\sin{k_z}\Gamma_4$ adding to $H(\bk)$ (Eq.~(\ref{HinS})) endows a constant tensor $m_\delta\varsigma_y$ to the edge dispersion $H_{\mathcal{L}}(k_2,\theta)$, which explicitly is that
\begin{equation}\label{mdelta}\begin{aligned}
	m_\delta=2\delta\Delta_2\frac{\eta^2(\tilde{t}_{\I}^2-\tilde{t}_{\II}^2)-2\tilde{t}_{\I}^2(t_a\tilde{t}_{\I}-t_b\tilde{t}_{\II}-\eta^2)}{\tilde{t}_{\I}^2(\tilde{t}_{\I}+\tilde{t}_{\II})^2(t_a-t_b-\eta^2/\tilde{t}_{\I})}\sin{k_z}.
\end{aligned}\end{equation}
\noindent Coupling to the anisotropic mass term $m_{\mathrm{eff}}(\theta)$, $m_\delta$ endows an extra constant gap for any edge $\mathcal{L}(\theta)$, which pushes the second-order Majorana zero modes to annihilate each other at the edges $\mathcal{L}(\theta=0,\pi)$ or $\mathcal{L}(\theta=\frac{\pi}{2},\frac{3\pi}{2})$.








\subsection{D. Topological characterization with nested Wilson loop}
\label{WL}

In this section, we discuss the topological characterization of the hinge Majorana states in our model using the concept of nested Wilson loop~\cite{quadrupole}. A full BdG Hamiltonian is diagonalized,
\begin{equation}
    H = \sum_{\mu,\bk}\gamma_{\mu \bk}^\dagger E_{\mu \bk}\gamma_{\mu \bk},
\end{equation}
where $\gamma_{\mu \bk}^\dagger=\sum_\alpha u_{\mu \bk}^\alpha c_{\alpha \bk}^\dagger$, $\alpha=1,2,...,N_{\mathrm{orb}}$. $N_{\mathrm{orb}}$ is the number of orbitals. To calculate the Wilson loop along $k_x$ direction, we can construct the overlap matrix $F_x(k_y,k_z)$ using the occupied bands,
\begin{equation}
    F_{x,k_x^j}^{mn}(k_y,k_z)=\sum_\alpha u^{\alpha*}_{m,\bk^j} u^\alpha_{n,\bk^{j+1}}
\end{equation}
where $k_{x}^{j} = 2{\pi}j/N_x$ and $m, n = 1, 2, ... , N_{\rm occ}$, with $N_{\rm occ}$ is the number of occupied bands. Define $\bk'\equiv(k_y,k_z)$, hence,
\begin{equation}
    \mathcal{W}_{x}(\bk') = \prod_{j = 0}^{N - 1} F_{x,k_x^j}(\bk').
\end{equation}
Extracting the phase part, define a Wannier Hamiltonian $\mathcal{W}_{x}(\bk') = e^{iH_{\mathcal{W}_x}(\bk')}$ and diagonalize it, $H_{\mathcal{W}_x}(\bk')=2\pi \sum_{n=1}^{N_\mathrm{occ}}\xi^\dagger_{n,\bk'}\nu_x^n(\bk')\xi_{n,\bk'}$, with $\xi^\dagger_n(\bk')=\sum_{m=1}^{N_\mathrm{occ}}\tilde{u}^m_{n,\bk'}\gamma^\dagger_{m,\bk'}$. In the following Wannier band subsapces, $\xi^\dagger_n(\bk')=\sum_{\alpha=1}^{N_{\mathrm{orb}}}w^\alpha_{n\bk'}c^\dagger_{\alpha\bk'}$, with components $w^\alpha_{n\bk'}=\sum_{m=1}^{N_\mathrm{{occ}}}\tilde{u}^m_{n,\bk'}u^\alpha_{m,\bk'}$, the nested Wilson loop is similarly defined by constructed 
\begin{equation}
	\tilde{F}_{xy,k_y^j}^{mn}(k_z)=\sum_\alpha w^{\alpha*}_{m,\bk'^j} w^\alpha_{n,\bk'^{j+1}}
\end{equation}
where $k_{y}^{j} = 2{\pi}j/N_y$ and $m, n = 1, 2, ... , N_{\rm P}$. In our case, $N_{\rm P}=2$. And
\begin{equation}
    \mathcal{W}_{xy}(k_z) = \prod_{j = 0}^{N_y - 1} \tilde{F}_{xy,k_y^j}(k_z).
\end{equation}
Thus the final Berry phase of nested Wilson loop $p_{xy}$ is obtained by diagonalize $\mathcal{W}_{xy}(k_z)=\sum_{n=1}^{N_{{\rm P}}}\rho^\dagger_{n,k_z}e^{i2\pi p_{xy}^{n}(k_z)}\rho_{n,k_z}$. From nested Wilson loop, the equivalent ``corner charge" can be characterized as $Q_c=\sum_n p_{xy}^n\ {\rm mod}\ 1$.

In Fig.~\ref{figS2}, we have plotted the hinge states, eigen value of (nested) Wilson loop and corner charge for type-$\I$ (a-c) and type-$\II$ (d-f) DSMs.

In type-$\I$ DSMs, It is shown that the hinge Majorana states appear, linking the projection of two bulk SC DPs ($\bk=(0,0,D_{1,2})$). Accordingly, taking out two SC DPs, the eigen value of Wilson loop $v_x$ in $k_y$-$k_z$ plane shows a full gap, except $(k_y,k_z)=(0,0)$ point, at which the pumping indicate a set of mirror-symmetry protected Majorana surface modes discussed in Ref.~\cite{Sato_TCSC}. Meanwhile, the corner charge is characterized by the nested Wilson loop, showing the nontrivial Wannier charge center $Q_c=0.5$ between two bulk SC DPs.

In type-$\II$ DSMs, on the contrary, the bulk SC DPs are annihilating each other and creating a full bulk gap. Therefore, the intact hinge Majorana flat bands extends across the whole hinge Brillouin zone, linking the projection of the surface helical Majorana states. Correspondingly, the eigen value of Wilson loop $v_x$ shows $\mathbb{Z}_2$ pumpings at $(k_y,k_z)=(0,0/\pi)$ and is gapped out elsewhere. Meanwhile, taking out two ill-defined points ($k_z=0/\pi$), each $k_z$-slice shows an nontrivial corner charge $Q_c=0.5$, consisting to the hinge Majorana states.

\begin{figure}[t!]
	\centering
	\includegraphics[scale=0.48]{Figs/figS2.pdf}
	\caption{(a,~b) Energy dispersion in a wire geometry along $k_z$ with open boundary along both $x$ and $y$. The hinge localized Majorana states are highlighted in red. (c,~d) Eigen value of Wilson loop operator of all occupied BdG bands of $H(\bkp,k_z)$ (Eq.~1 in the main text) along $k_x$. (e,~f) Dots: Eigen value of nested Wilson loop operator of all occupied wannier bands in (c,~d) along $k_y$. Lines: corner charge $Q_c$. $D_1$ and $D_2$ are the location of SC DPs. (a-c) are for type-$\I$ DSMs. (d-f) are for type-$\II$ DSMs. In the nested Wilson loop calculation, the gapless $k_z$-slices in (b,~e) have been taken out. Note that here we have set the $B_{2u}$ pairing channel to be zero, $\Delta_3=0$.}
	\label{figS2}
\end{figure}





\clearpage

\end{widetext}




\end{document}
