

\section{Introduction}
\label{sec:intro}
Large language models (LLMs) have transformed many fields, including natural language processing~\cite{brown2020language,touvron2023llama}, computer vision~\cite{wu2023visual,shao2023prompting}, and reinforcement learning~\cite{du2023guiding}. These models have also made a significant impact in the field of law, where they are being increasingly utilized to automate various legal tasks, such as legal judgement prediction, legal document analysis, and legal document writing~\cite{trautmann2022legal,blair2023can,yu2022legal,choi2023chatgpt,pettinato2023chatgpt,nay2023large,macey2023chatgpt,hargreaves2023words,iu2023chatgpt,nay2022law}. However, the integration of LLMs into the legal field has also raised several legal problems, including privacy concerns, bias, and explainability~\cite{tamkin2021understanding,felkner2022towards,abid2021persistent}. In this survey, we explore the integration of LLMs into the field of law. We discuss the various applications of LLMs in legal tasks, examine the legal challenges that arise from their use, and explore the data resources that can be used to specialize LLMs in the legal domain~\cite{xiao2018cail2018,zheng2021does,ma2021lecard}. Finally, we discuss several promising directions and conclude this paper. By doing so, we hope to provide an overview of the current state of LLMs in law and highlight the potential benefits and challenges of their integration. 
\subsubsection{$\bullet$ Related Surveys}
Most of the existing survey papers, which cover intelligent legal system papers, only focus on traditional natural language technologies. Some of these surveys focus on one legal task, such as legal case prediction. On the other hand, the other surveys cover multiple legal tasks. Most of the survey papers summarize the current resources such as open-source tools and datasets for legal research. 

\begin{table}[!htp]
    \centering
    \caption{Comparison with existing surveys. For each survey, we summarize the topics covered and the main scope to survey.}
    \begin{tabular}{c|ccc|c}
    \hline
        Surveys & Topics& & & Latest Year \\ \cline{0-0} \cline{5-5}
        &LLMs & dataset source & multi-domain& \\ \cline{2-4}
        \cite{chalkidis2019deep} & no & yes & yes & 2019 \\ \hline
        \cite{cui2022survey} & no & yes & no & 2022 \\ \hline
        \cite{dias2022state} & no & yes & yes & 2022 \\ \hline
        \cite{katz2023natural} & no & yes & yes & 2023 \\ \hline
        Ours & yes & yes & yes & 2023 \\ \hline
    \end{tabular}
    \label{tab:survey_analysis}

\end{table}

As shown in Table~\autoref{tab:survey_analysis}, no survey paper exists in the literature that focuses specifically on LLMs. In the current work, we aim to fill this gap by providing a comprehensive and systematic survey on LLMs-based intelligent legal systems. The classification of the surveyed papers are shown in~\autoref{tab:paper_classification}.


\begin{table}[!htp]
  \centering
  \caption{Classification of papers}
  \label{tab:paper_classification}
  \begin{tabular}{cc}
    \toprule
    \textbf{Papers} & \textbf{Category}\\
    \midrule
    \cite{trautmann2022legal,blair2023can,yu2022legal,choi2023chatgpt,pettinato2023chatgpt,nay2023large,macey2023chatgpt,hargreaves2023words,iu2023chatgpt,nay2022law} & Applications of Large Language Models in Legal Tasks\\
    \cite{tamkin2021understanding,felkner2022towards,abid2021persistent} & Legal Problems of Large Language Models\\
    \cite{xiao2018cail2018,zheng2021does,ma2021lecard} & Data Resources for Large Language Models in Law\\
    \bottomrule
  \end{tabular}
\end{table}



\subsubsection{$\bullet$ Contributions}

In this survey, we have made several contributions to the field of law and natural language processing, including:
\begin{itemize}
    \item An overview of the applications of large language models in legal tasks, such as legal judgement prediction, legal document analysis, and legal document writing.
    \item An analysis of the legal problems raised by the use of large language models in law, including privacy concerns, bias and fairness, and explainability and transparency.
    \item A discussion of the data resources that can be used to specialize large language models in the legal domain, such as case law datasets and tools.
    \item Suggestions for future research directions to address the legal challenges posed by the use of large language models in law, such as developing methods to mitigate bias and ensure transparency.
\end{itemize}

Through our contributions, we hope to provide a comprehensive understanding of the current state of large language models in law and highlight the potential benefits and challenges of their integration. We also aim to encourage further research in this area and facilitate the responsible and ethical integration of large language models into the legal domain. The paper list is shown in~\url{https://github.com/Jeryi-Sun/LLM-and-Law}, which will be updated on time.

\section{Applications of Large Language Models in Legal Tasks}
Large language mode (LLMs), such as ChatGPT, have demonstrated considerable potential in various legal tasks, showcasing their ability to transform the legal domain. This comprehensive analysis delves into the recent applications of LLMs in legal tasks, focusing on the challenges and opportunities they present, while highlighting the relationships and differences between the relevant studies.

One notable area where LLMs have shown promise is in legal judgment prediction and statutory reasoning. The study by~\cite{trautmann2022legal} introduces legal prompt engineering (LPE) to enhance LLM performance in legal judgment prediction tasks. This method has proven effective across three multilingual datasets, highlighting the model's potential in handling the complexity of legal language and reasoning across multiple sources of information. Another study by~\cite{blair2023can} investigates GPT-3's capacity for statutory reasoning, revealing that dynamic few-shot prompting enables the model to achieve high accuracy and confidence in this task.
Advancements in prompting techniques have played a crucial role in the success of LLMs in legal reasoning tasks. The paper by~\cite{yu2022legal} introduces Chain-of-Thought (CoT) prompts, which guide LLMs in generating coherent and relevant sentences that follow a logical structure, mimicking a lawyer's analytical approach. The study demonstrates that CoT prompts outperform baseline prompts in the COLIEE entailment task based on Japanese Civil Code articles.
LLMs have also been employed to understand fiduciary obligations, as explored in~\cite{nay2023large}. This study employs natural language prompts derived from U.S. court opinions, illustrating that LLMs can capture the spirit of a directive, thus facilitating more effective communication between AI agents and humans using legal standards.
The potential of LLMs in legal education has been examined in studies such as~\cite{choi2023chatgpt} and~\cite{hargreaves2023words}. The authors of~\cite{choi2023chatgpt} task ChatGPT with writing law school exams without human assistance, revealing potential concerns and insights into LLM capabilities in legal assessment. On the other hand, the paper by~\cite{hargreaves2023words} addresses the ethical use of AI language models like ChatGPT in law school assessments, proposing ways to teach students appropriate and ethical AI usage.
The role of LLMs in supporting law professors and providing legal advice has also been investigated. The study in~\cite{pettinato2023chatgpt} suggests that LLMs can assist law professors in administrative tasks and streamline scholarly activities. Furthermore, LLMs have been explored as quasi-expert legal advice lawyers in~\cite{macey2023chatgpt}, showcasing the possibility of using AI models to support individuals seeking affordable and prompt legal advice.
The potential impact of LLMs on the legal profession has been a subject of debate, as discussed in~\cite{iu2023chatgpt}. This paper evaluates the extent to which ChatGPT can serve as a replacement for litigation lawyers by examining its drafting and research capabilities.
Finally, the study by~\cite{nay2022law} proposes a legal informatics approach to align AI with human goals and societal values. By embedding legal knowledge and reasoning in AI, the paper contributes to the research agenda of integrating AI and law more effectively.

In conclusion, LLMs have shown promising results in various legal tasks, with the advancement of prompting techniques playing a crucial role in their success. However, challenges remain in ensuring the ethical use of LLMs and addressing their potential impact on the legal profession. Future research should continue to explore the capabilities and limitations of LLMs in the legal domain while ensuring their alignment with human values and societal needs.



\section{Legal Problems of Large Language Models}

Large Language Models (LLMs) such as GPT-3 have exhibited transformative potential across various domains, including science, society, and AI~\cite{tamkin2021understanding}. However, the growing capabilities of these models have also given rise to several legal challenges. This comprehensive analysis delves into the legal problems concerning intellectual property, data privacy, and bias and discrimination in LLMs, emphasizing the need for collaboration between researchers and policymakers in addressing these issues.
Intellectual property concerns emerge with LLMs' ability to generate human-like text that may resemble existing copyrighted works or create original content.~\cite{tamkin2021understanding} highlights the uncertainty surrounding copyright ownership in such cases, whether it should be attributed to the model developer, the user, or the model itself~\cite{tamkin2021understanding}. Addressing this issue necessitates the reevaluation and clarification of existing copyright laws and the development of new legal frameworks.
Data privacy is another legal challenge associated with LLMs, as they are trained on extensive datasets that may contain personal or sensitive information. Despite anonymization efforts, LLMs might unintentionally disclose private information or reconstruct protected data, raising questions about their compliance with existing data privacy legislation, such as the General Data Protection Regulation (GDPR)~\footnote{\url{https://gdpr-info.eu/}}. To tackle this problem, the research and development of advanced data anonymization techniques and privacy-preserving training methods should be prioritized.
Moreover, LLMs have been found to perpetuate biases present in their training data, leading to discriminatory outcomes. For instance,~\cite{felkner2022towards} demonstrated the presence of anti-queer bias in models like BERT. Similarly,~\cite{abid2021persistent} revealed that GPT-3 captures persistent Muslim-violence bias. These biases may result in models that contravene anti-discrimination laws or unfairly disadvantage specific groups. Consequently, researchers and policymakers must collaborate to develop guidelines and legal frameworks to mitigate harmful biases, ensuring the responsible deployment of LLMs~\cite{tamkin2021understanding}.
As large language models continue to advance, it is crucial to address the legal challenges they pose. By thoroughly investigating intellectual property issues, data privacy concerns, and biases within LLMs, researchers and policymakers can work together to establish an environment where LLMs are responsibly developed and deployed, maximizing their benefits for society as a whole.

\section{Data Resources for Large Language Models in Law}

In recent years, there has been a growing interest in applying large language models (LLMs) to the legal domain, given the potential benefits of such models in tasks such as legal judgment prediction (LJP), case retrieval, and understanding legal holdings. However, due to the unique linguistic features and specialized domain knowledge of law, LLMs often require targeted data resources to adapt and fine-tune effectively.

One essential data resource is the CAIL2018 dataset, introduced in CAIL2018~\cite{xiao2018cail2018}. Comprising more than 2.6 million criminal cases from China, this dataset allows researchers to delve into various aspects of LJP, such as multi-label classification, multi-task learning, and explainable reasoning. The detailed annotations of applicable law articles, charges, and prison terms provide a rich source of information for LLMs to specialize in the legal domain.
Another valuable resource is the CaseHOLD dataset, presented in CaseHOLD~\cite{zheng2021does}. It contains over 53,000 multiple-choice questions covering various areas of law, including constitutional law, criminal law, contract law, and tort law. Additionally, the paper introduces two domain pretrained models, BERT-Law and BERT-CaseLaw, which are based on BERT-base but pretrained on different subsets of US legal documents. These models, along with the dataset, contribute to the specialization of LLMs in the legal domain and help address the challenges and limitations of domain pretraining for law.
Furthermore, LeCaRD offers a novel dataset for legal case retrieval based on the Chinese law system~\cite{ma2021lecard}. The Chinese Legal Case Retrieval Dataset (LeCaRD) consists of 107 query cases and over 43,000 candidate cases sourced from criminal cases published by the Supreme People's Court of China. This dataset, along with the relevance judgment criteria and query sampling strategy proposed in the paper, provides a valuable resource for specializing LLMs in the Chinese legal system and its unique terminology, logic, and structure.

The growing availability of specialized legal datasets, such as CAIL2018, CaseHOLD, and LeCaRD, enables researchers to train and fine-tune LLMs effectively in the legal domain. By utilizing these data resources, LLMs can better capture the unique linguistic features and specialized domain knowledge of law, leading to improved performance in various legal tasks such as LJP, case retrieval, and understanding legal holdings.



\section{Conclusion and Future Directions}
In conclusion, the integration of large language models into the field of law has great potential to improve the efficiency and accuracy of legal tasks. LLMs have already shown promising results in legal document analysis, contract review, and legal research. However, their use also raises legal concerns related to privacy, bias, and explainability, which must be carefully considered and addressed. Moreover, the development of specialized data resources is crucial to ensure the accuracy and reliability of LLMs in the legal domain.

Looking ahead, further research is needed to address the legal challenges posed by the use of LLMs in law. This includes developing methods to mitigate the potential biases in LLMs and ensure that they provide transparent and interpretable outputs. Additionally, the development of specialized data resources and tools is necessary to further improve the accuracy and effectiveness of LLMs in legal tasks. Finally, there is also a need to develop guidelines and standards for the use of LLMs in the legal domain to ensure that their integration is done in a responsible and ethical manner. With these efforts, the integration of LLMs into the field of law holds great promise for improving legal processes and access to justice.




\bibliographystyle{abbrv}
\bibliography{2022000149}