\documentclass[10pt,twocolumn,letterpaper]{article}

\usepackage{iccv}
\usepackage{times}
\usepackage{epsfig}
\usepackage{graphicx}
\usepackage{amsmath}
\usepackage{amssymb}
% \usepackage{tablestyles}
\DeclareMathOperator*{\argmin}{argmin}
\newcolumntype{P}[1]{>{\centering\arraybackslash}p{#1}}

\usepackage{booktabs, multirow} % for borders and merged ranges
\usepackage{soul}% for underlines
\usepackage[table]{xcolor} % for cell colors
\usepackage{changepage,threeparttable} % for wide tables

% Include other packages here, before hyperref.

% If you comment hyperref and then uncomment it, you should delete
% egpaper.aux before re-running latex.  (Or just hit 'q' on the first latex
% run, let it finish, and you should be clear).
\usepackage[breaklinks=true,bookmarks=false]{hyperref}

\iccvfinalcopy % *** Uncomment this line for the final submission

\def\iccvPaperID{****} % *** Enter the ICCV Paper ID here
\def\httilde{\mbox{\tt\raisebox{-.5ex}{\symbol{126}}}}

% Pages are numbered in submission mode, and unnumbered in camera-ready
\ificcvfinal\pagestyle{empty}\fi

\begin{document}

\title{GeoSpark: Sparking up Point Cloud Segmentation with Geometry Clue}


\author{Zhening Huang$^{1}$ \hspace{1.0cm} Xiaoyang Wu$^{2}$\hspace{1.0cm}  Hengshuang Zhao$^{2}$\hspace{1.0cm} Lei Zhu$^{3}$\hspace{1.0cm} Shujun Wang$^{1}$ \\ Georgios Hadjidemetriou$^{1}$ \hspace{1.0cm} Ioannis Brilakis$^{1}$ \\
\vspace{1.0cm}
$^{1}$University of Cambridge~~~
$^{2}$The Unversity of Hong Kong~~~
$^{3}$HKUST(GuangZhou)~~~
% $^{4}$MPI Informatics~~~
% $^{5}$MIT\\
% Institution1 address\\
% {\tt\small firstauthor@i1.org}
%\texttt{\footnotesize
%\{xinlai,lwwang,leojia\}@cse.cuhk.edu.hk\quad sliu@smartmore.com\quad hengshuang.zhao@eng.ox.ac.uk
%}
}

\maketitle
% Remove page # from the first page of camera-ready.
\ificcvfinal\thispagestyle{empty}\fi




Over the past few years, there has been a significant amount of research focused on studying the ReLU activation function, with the aim of achieving neural network convergence through over-parametrization. However, recent developments in the field of Large Language Models (LLMs) have sparked interest in the use of exponential activation functions, specifically in the attention mechanism.

Mathematically, we define the neural function $F: \R^{d \times m} \times  \mathbb{R}^d \rightarrow \mathbb{R}$ using an exponential activation function. Given a set of data points with labels $\{(x_1, y_1), (x_2, y_2), \dots, (x_n, y_n)\} \subset \mathbb{R}^d \times \mathbb{R}$ where $n$ denotes the number of the data. Here $F(W(t),x)$ can be expressed as $F(W(t),x) := \sum_{r=1}^m a_r \exp(\langle w_r, x \rangle)$, where $m$ represents the number of neurons, and $w_r(t)$ are weights at time $t$. It's standard in literature that $a_r$ are the fixed weights and it's never changed during the training. We initialize the weights $W(0) \in \mathbb{R}^{d \times m}$ with random Gaussian distributions, such that $w_r(0) \sim \mathcal{N}(0, I_d)$ and initialize $a_r$ from random sign distribution for each $r \in [m]$.

Using the gradient descent algorithm, we can find a weight $W(T)$ such that $\| F(W(T), X) - y \|_2 \leq \epsilon$ holds with probability $1-\delta$, where $\epsilon \in (0,0.1)$ and $m = \Omega(n^{2+o(1)}\log(n/\delta))$. To optimize the over-parametrization bound $m$, we employ several tight analysis techniques from previous studies [Song and Yang arXiv 2019, Munteanu, Omlor, Song and Woodruff ICML 2022]. 

 


%%%%%%%%% BODY TEXT
% Importance and appeal of children's drawings
Children's depictions of the human figure are highly expressive and varied.
As one of the very first subjects children attempt to draw, the representation begins as an almost unintelligible cloud of scribbles. 
As the child grows, their representation of the human figure becomes more developed and is extended to graphically represent many different types of characters: people, animals, and even personified objects (see Figure 1).

Who among us has not wished, either as a child or as an adult, to see such figures come to life and move around on the page?
Sadly, while it is relatively fast to produce a single drawing, creating the sequence of images necessary for animation is a much more tedious endeavor, requiring discipline, skill, patience, and sometimes complicated software.
As a result, most of these figures remain static upon the page.

% We built a system to animate them.
Inspired by the importance and appeal of the drawn human figure, we design and build a system to automatically animate it given an in-the-wild photograph of a child's drawing. 
Our system is fast, intuitive, and robust to much of the variation present in these types of drawings, making it well-suited to allow our target audience--children--to see their own characters coming to life.
The system is comprised of four stages: figure detection, segmentation masking, pose estimation/rigging, and animation. 
We describe each stage and identify common causes of failure in each. 
For object detection and pose estimation, we make use of existing computer vision models designed to detect human figures and joints in photographs; we fine-tune these models for use with children's drawings.
For segmentation, we present a straightforward, image processing-based method that, for animation purposes, is more useful and accurate than segmentation masks obtained from a fine-tuned object detection model.
During the animation step, we take advantage of the \textit{twisted perspective} commonly seen in children’s drawings to retarget motion capture data onto the character in a novel and appealing way.

% We use existing machine learning models. However, given the wide domain gap it's not clear how much fine-tuning data was needed. So we ran some experiments to find out and report it.
While our system leverages existing models and techniques, most are not directly applicable to the task due to the many differences between photographic images and simple pen and paper representations. 
To this end, we couple the presentation of our system with a set of experiments exploring the relationship between fine-tuning training set size and success rates.
We also include a perceptual study validating viewer preference for incorporating \textit{twisted perspective} into the motion retargeting step.

We validate the desirability and appeal of our system by building and publicly releasing a version of it as the \AD Demo \,\cite{animateddrawings}.
Launched in December 2021, this demo has been used by millions of people around the world to animate their children's drawings.
Inspired by this reception, our second contribution is The Amateur Drawings Dataset: \hjs{180,000 drawings and user-accepted annotations collected, with consent, through the demo. See Section \ref{sec:UI} for a description of how the annotations were generated.}
We believe this dataset will be a resource to researchers from various fields seeking to better understand the space of amateur drawings, evaluate new algorithms in this domain, or develop new drawing-based tools in general.

To summarize, our contributions are as follows:
\begin{enumerate}
    \item 
    We explore the problem of automatic sketch-to-animation for children's drawings of human figures and present a framework that achieves this effect. We also present a set of experiments determining the amount of training data necessary to achieve high levels of success and a perceptual study validating the usefulness of our motion retargeting technique.
    \item To encourage additional research in the domain of amateur drawings, we present a first-of-its-kind dataset of 180,000 user-submitted amateur drawings, along with user-accepted bounding box, segmentation mask, and joint location annotations.
\end{enumerate}

Upon acceptance of this paper, we plan to publicly release the Amateur Drawings Dataset, project code, and fine-tuned model weights.


\section{Related work} \label{sec:intro}
\paragraph{Point Cloud Segmentation.}
Deep learning methods for point cloud processing can be categorized into three types: projection-based methods \cite{Tatarchenko2018Tangent3D, Wu_2019,Wu_2018, Boulch_2018, Jaritz_2019}, voxel-based methods \cite{Choy20194DNetworks,Graham_2015, Park_2022}, and point-based methods \cite{Charles_2017,Qi2017PointNet++:Space, Zhao2019Pointweb:Processing, Hu2019RandLA-Net:Cloudsb, Jiang2018PointSIFT:Segmentation, Li2018, Boulch_2020}. Recent work on point-based methods mostly adopts an encoder-decoder paradigm while applying various sub-sampling methods to expand the receptive field for point features. A variety of local aggregation modules were developed, including convolution-based methods \cite{Thomas2019KPConv:Cloudsb, Boulch_2020}, graph-based methods \cite{Landrieu2018Large-scaleGraphs, Landrieu2019PointLearning}, and attention-based methods \cite{Hu2019RandLA-Net:Clouds, Zhao_2021}. Meanwhile, different downsampling methods \cite{Chen2022SASA:Detection, Dovrat_2019, Yan2020PointASNL:Sampling}, upsampling \cite{Qiu2021SemanticFusion, Varney_2022}, and post-processing methods \cite{Hu_2020, Lu_2021} have been proposed to enhance the network's robustness. The inherent permutation invariance of attention operations makes them well-suited for point cloud learning, leading to the application of transformer operations in point clouds \cite{Zhao_2021, Engel_2021, Xie_2020}, following their success in 2D and NLP. However, global self-attention is impractical due to massive computational costs, leading to recent work utilizing local-scale self-attention to learn point features. Point Transformer \cite{Zhao_2021} proposed a vector self-attention with k nearest neighbor points for feature aggregation and introduced the concept of learnable positional embedding, which has proved to be very powerful across various tasks. Nevertheless, local attention has a limited receptive field and cannot explicitly model long-range dependencies. Thus, Stratified Transformer \cite{Lai_2022} was developed and adopted a shifted window strategy to include long-range contexts. Similarly, Fast Point Transformer \cite{Park_2022} utilized a voxel hashing-based architecture to enhance computational efficiency. However, these data-agnostic methods only select receptive fields based on spatial information, which could fail to focus on relevant areas.
\vspace{-5mm}
\paragraph{Geometric Partition}
Geometric partitions are an over-segmentation technique for point clouds that groups point with similar geometric features. Over the years, several methods have been developed to learn hand-crafted features of point clouds and utilize clustering or graph partition methods to generate meaningful partitions \cite{Papon_2013, Guinard2017WeaklyClouds, Lin_2018}. However, the performance of these methods is limited by the handcrafted feature of point clouds. SPNet \cite{Hui_2021} introduced a superpoint center updating scheme that generates superpoints with the supervision of point cloud labels. Although very efficient, SPNet fails to produce intuitive partitions beyond grouping points with the same semantic label together. On the other hand, the supervised superpoint method uses MLP to learn point features and combines points with graph-structured deep metric learning \cite{Landrieu2018Large-scaleGraphs}. Nevertheless, Geometric partition techniques offer impressive outcomes in finding geometric shapes in point clouds. However, despite their impressive performance, there have been few attempts to utilize these techniques in the deep learning framework \cite{Landrieu2018Large-scaleGraphs, Landrieu2019PointLearning}. In this regard, we propose GeoSpark, which leverages the geometry information stored in the partition to enhance feature aggregation and downsampling.

\section{GeoSpark}
Our goal is to leverage the explicit geometry clue stored in the geometric partitions to improve the feature aggregation and downsampling process. In this section, we will elaborate on these are achieved on GeoSpark. We will begin by explaining how the geometric partition is generated and embedding, followed by technical details of our \textit{Geometry-Informed Aggregation} (GIA) module as well as \textit{Geometric Downsampling} (GD) module. 

\begin{figure*}[ht]
 \centering
 \includegraphics[width=0.9\textwidth]{section/images/architect.pdf} 
 %\includegraphics[width=\textwidth]{fig/network.pdf}
 \centering
 \caption{\textbf{The Paradigm of GeoSpark Architecture} Initially, the point input is fed into the "Geometric Partition \& Embedding" module to obtain geometric partitions and encode them into superpoints. These superpoints are then processed in the global branch to learn features at multiple scales before being added to the local branch as geometry guidance. In addition to the primary prediction loss, a superpoint loss is introduced to guide the training of the global branch. The aggregation module can be altered to various backbones.}

 \label{fig:networkTeaser}
\end{figure*}
\subsection{Geometric Partition \& Embedding}
The first step of the proposed GeoSpark network is to effectively learn the low-level features of point clouds and group similar points. This process significantly reduces the redundancy of point clouds and makes it possible to encode long-range context inexpensively in future steps. This module will process the entire input scene and produce an initial geometric partition, so it needs to be fast and effective. In our experiment, we tested different geometric partition methods, including VCCS\cite{Papon_2013}, Global Energy Model (GEM)\cite{Guinard2017WeaklyClouds}, Graph-Structured Method (GSM) \cite{Landrieu2019PointLearning} and SPnet \cite{Hui_2021}. The GEM \cite{Guinard2017WeaklyClouds} offers the best trade-off between speed and performance, so we integrate that into our \textit{Geometric Partition \& Embedding} module. Specifically, for each point, a set of geometric features, such as linearity, planarity, scattering, and verticality \(f_i \in \mathbb{R}^{c}\) are computed, characterizing local features and shapes. The geometrically homogeneous partition is defined as the constantly connected components and therefore as the solution to the following optimization problem\cite{Guinard2017WeaklyClouds}: 
\begin{equation}
 \argmin_{g \in \mathbb{R}^{c}} \sum_{i \in C} ||g_i-f_i||^2 + \lambda \sum_{(i,j)\in E_{nn}} \omega_{i, j}[g_i \neq g_j]
 \label{optimization}
 \vspace{-2mm}
\end{equation}
where \(\lambda\) is the \textit{regularization strength} that determines the coarseness of the partition, \(\omega_{i, j}\) is a weight matrix that is linearly inversely proportional to the distance between points and \([\cdot]\) is the Iverson bracket. We use the \textit{\(l_o\)-cut} pursuit algorithm\cite{LandrieuCutFunctions} to quickly find approximate solutions for Eq. \ref{optimization}, as it is impractical to find exact solutions when the number of points is really large. 
Given point cloud set \( X = (P, F)\), the geometric partition module splits the point cloud into point subsets \([\hat{X_1}, \hat{X_2}, \hat{X_3} \dots, \hat{X_m}] \), and each set contains a different number of points. 

To encode each partition into a superpoint, we adopted a lightweight MLP structure, inspired by PointNet \cite{Charles_2017}. The input point feature firstly go through MLP layers to gradually project the features of the points to higher dimension space. Points in each partition are then fused into one superpoint by applying MaxPooling in feature space \(\hat{f_i}\) and AvgPooling in coordinates space \(\hat{p_i}\). We concatenate global information of each partition \(\hat{f_{i,g}}\), such as \textit{partition diameter} into the features of the fused points, before applying other layers of MLP to reduce the feature dimension to \(c\). Formally,
\begin{equation}
\begin{aligned}
 \hat{F_i} &= T_2 \ ( \text{MaxPool}\ \{{T_1\ (\hat{f_i}) \ | \ \hat{f_i}\in \hat{X_i}}\} \ \oplus \ \hat{f_{i,g}} )\\
 \hat{P_i} &= \text{AvgPool} \ \{{\hat{p_i} \ | \ \hat{p_i}\in X_i}\} 
\end{aligned}
\end{equation}
where \(T_1 : \mathbb{R}^c \xrightarrow{} \mathbb{R}^{c_{l}}\) and \(T_2 :\mathbb{R}^{c_l} \xrightarrow{} \mathbb{R}^c\) are MLP layers that enlarge and condense point feature dimensions. 

\subsection{Geometry-Informed Aggregation}
The key idea of Geometry-Informed Aggregation (GIA) is to attend point feature both from local neighbour points and global geometric partition regions, encoded in superpoints. Therefore, GIA module takes two sets of inputs: local points \( X\ = (p_i, f_i)\) and encoded superpoints \(\hat{X} =(\hat{p_i}, \hat{f_i})\). Local point features \(f_i\) and superpoint features \(\hat{f_i}\) are first processed by separated MLP layers before being fed into the \textit{GIA} module. The following explain the proposed method using Point Transformer as backbone. However, similar strategies can be incorporated with other backbone easily.

\paragraph{Local Neighbor Aggregation}
The local neighbor aggregation follows the orginal design of backbone structure. For Point Transformer, it \cite{Zhao_2021} first uses three linear projections \(\phi, \psi, \alpha\) to obtain \textit{query}, \textit{key} and \textit{value} for local points. Following \textit{vector self-attention}, a weight encoding function \(\omega\) is introduced to encode the subtraction relations between point \textit{queries} and \textit{keys}, before processing to the \textit{softmax} operation to form the attention map. A trainable parameterized position encoding, formed by an MLP encoding function \(\theta\), is added to both the attention map and the value vectors. Formally, for query point \(x_i = (p_i, f_i)\) and reference points \(x_j= (p_j,f_j)\), the local attention map is formed as follows:
\begin{equation}
\begin{aligned}
 \delta_{i,j} & = \theta(p_i-p_j) \\
w_{i,j} &= \omega(\phi(f_{j})- \psi(f_{i})+\delta_{i,j}) \\
\end{aligned}
\end{equation}
\paragraph{Geometric Partition Aggregation}

To learn point feature from superpoint, a feature map is built between local points and global superpoints. Specifically, \textit{key} and \textit{value} is projected from superpoint features \(\hat{f_i}\) with linear operation \(\Psi\), and \(A\) respectively. Attention is formed similarly to \textit{Local Attention}. However, an independent weight encoding function \(\Omega\) is used and the positional embedding also employs a different MLP \(\Theta\) to encode the coordinate difference between the query points and the reference superpoints. 
\begin{equation}
\begin{aligned}
\Delta_{i,j} &= \Theta(p_i-\hat{p_j}) \\
W_{i,j} &= \Omega(\phi(\hat{f_{i}})- \hat{\Psi(f_{j})} +\Delta_{i,j})\\
\end{aligned}
\end{equation}
In practice, \textit{Local Neighbor Aggregation} and \textit{Geometric Partition Aggregation} are performed simultaneously and merged to form \textit{Geometry-Informed Aggregation}, formally:
\begin{equation}
\begin{aligned}
f_{i,local}' &= \sum_{f_i \in K} \text{softmax}(w_{i,j}) \odot (\alpha(f_j)+\delta_{i,j}) \\
f_{i,global}' &= \sum_{\hat{f_i} \in \hat{K}}\text{softmax}(W_{i,j}) \odot (A(\hat{f_j})+\Delta_{i,j})\\
f_i^{'} &= \xi \left[{f_{i,local}' + f_{i,global}'}\right]
\end{aligned}
\end{equation}
where \(\odot\) represents Hadamard product, \(K\) is defined as reference local points, taken from \(k_{local}\) nearest neighbour, and \(\hat{K}\) is the reference superpoints, sampled with \(k_{global}\) nearest neighbour. After merging the outputs of \textit{Local Neighbor Aggregation} and \textit{Geometric Partition Aggregation}, an MLP layer (\(\xi : \mathbb{R}^c \xrightarrow{} \mathbb{R}^{c}\)) is employed to further glue global and local features and form the final outputs. Notice that it is important to maintain a good global and local balance to ensure that both fine details and long-range dependencies are included. More details on this can be found in the ablation study.

The features of superpoints \(\hat{X} =(\hat{p_i}, \hat{f_i})\) are learned in a separate global branch. In this branch, the superpoint features are projected to different dimensions at various stages to match the dimension space of the local branch. Since the number of superpoints is typically several orders of magnitude smaller than that of the local branch, the feature learning in the global branch is very fast.




\paragraph{Loss function.} A simple yet important superpoint loss is introduced to assist feature learning in the global branch. 
Specifically, given a local point set \(X\), and its label \(E = \{e_i \in \mathbb{R}^l \ | \ i=1,\dots, n\}\), where \(e_i\) is the one-hot vector. We generate a \textit{soft pseudo label} by calculating the label distribution in each geometric partition \(W = \{w_j \in \mathbb{R}^l\ | \ j = 1, \dots, m\} \). We optimize the global branch by minimising the distance between superpoint prediction \(u\) and \textit{soft pseudo label} \(w\). Overall, our loss is constructed with per-point prediction loss and the superpoint loss while a parameter \(\beta\) is introduced to determine the weight of the superpoint loss. Given the pre-point predication as \(e'\), the loss function is
\begin{equation}
 L_{total} = \frac{1}{n}\sum_{i=1}^{n}\mathcal{L}_{loss}(e_i,e_i') + \beta \frac{1}{m}\sum_{i=1}^{n}\mathcal{L}_{loss}(w_j,u_j)
\end{equation}
where \(\mathcal{L}_{loss}\) can be various type of loss function. Cross-entropy is selected in our experiments.

\subsection{Geometric Downsampling}
\begin{figure}[b!]
 \centering
 \includegraphics[width=0.5\textwidth]{section/images/GD.pdf}
 %\includegraphics[width=\textwidth]{g}
 \centering
 \caption{\textbf{Geometric Downsampling.}
\textit{Geometric Downsampling} module fuses points in small partitions into superpoints and further split oversized partitions with a grid to control the downsampling rate. This method drops redundant points and retains points with unique features}
\label{fig:GD}
\end{figure}

One notable challenge in the downsampling process is the early loss of the under-representative points, such as points for small objects, leading to insufficient feature learning and unsatisfactory prediction results. Data agnostic sampling methods such as random sampling, and FPS, do not consider the uniqueness of points and are likely to cause this issue. To mitigate this problem, we develop the \textit{Geometric Downsampling} module with the help of geometric partitioning.
The motivation is to preserve points with unique features and drop redundancy points in the downsampling process. Given a point set \( X = \{(p_i, f_i)\ |\ i=1 \dots n\}\), instead of sampling from the whole point set, we conduct sampling from each geometric partition. In particular, we set a target diameter \(a\) for new points, and further split every partition that is larger than the predetermined size \(a\) with voxel grids to obtain subsets, e.g. \([\hat{X_1}]\) to \([\hat{X_{11}}, \hat{X_{12}}, \hat{X_{13}}, \dots]\). Once this process is done, we fuse the fine partition with pooling operations, where \textit{MaxPooling} is applied at features space and \textit{AvgPooling} is conducted on coordinate space, as illustrated in Figure \ref{fig:GD}. Similar to Point Transformer, we apply MLP layers \(D: \mathbb{R}^c \xrightarrow{} \mathbb{R}^{c}\) on feature space before pooling. Mathematically: 
\begin{equation}
\begin{aligned}
{f_{2,i}} = \text{MaxPool}\{Df_{1, i}|f_i\in X_{1, ii}\}\\
{p_{2,i}} = \text{AvgPool}\{p_{1, i}|p_i\in X_{1, ii}\} 
\end{aligned}
\end{equation}

\section{Experiments}
\label{sec:Exp}
\subsection{Datasets and Evaluation Metrics}
We conducted extensive experiments and ablations on two standard WSVAD evaluation datasets~\cite{sultani2018real,lv2021localizing}. As per standard in WSVAD, the training videos only have video-level labels, and the test videos have frame-level labels. Other details are given below:

\noindent\textbf{UCF-Crime}~\cite{sultani2018real} is a large-scale dataset that contains 1,900 untrimmed real-world outdoor and indoor surveillance videos. The total length of the videos is 128 hours, which contains 13 classes of anomalous events.
We follow the standard split: the training set contains 1,610 videos, and the test set contains 290 videos.

\noindent\textbf{TAD} dataset~\cite{lv2021localizing} contains real-world videos of traffic scenes with a total length of 25 hours and 1,075 average frames per video.
The videos contain more than 7 categories of anomalies that are common on roads.
The dataset is partitioned as a training set with 400 videos, and a test set with 100 videos.

\noindent\textbf{Evaluation Metrics}. Following previous works~\cite{sultani2018real,zhong2019graph}, we used the Area Under the Curve (AUC) of the frame-level ROC (Receiver Operating Characteristic) as the main evaluation metric for TAD and UCF-Crime. Intuitively, a larger AUC means a larger margin between the normal and abnormal snippet predictions, hence indicating a better anomaly classifier.
Inspired by Lv~\etal~\cite{lv2021localizing}, besides evaluating AUC on the overall test set with normal and abnormal videos, denoted as $\mathrm{AUC}_O$, we also computed the AUC on abnormal ones alone, denoted as $\mathrm{AUC}_A$.
The rationale is to remove normal videos where all snippets are normal (label 0), and keep only the abnormal ones with both kinds of snippets (label 0,1), which truly challenges a classifier's capability of localizing anomalies.
 
\subsection{Implementation Details}
\label{sec:ID}

\noindent\textbf{Video Sequence Partition}. Existing works partition each video into multiple coarse snippets, and use the \emph{average feature} in each one as the input to their classifiers (Figure~\ref{fig:pip} left). However, we find that the subtle anomaly feature is often diluted by averaging features over the coarse snippets (see Appendix).
This has less impact on the traditional MIL compared to our UMIL, as MIL only leverages the confident snippets with apparent anomalies.
Therefore, in UMIL training, we used fine-grained snippets with one-second lengths. In testing, to generate the prediction for a coarse snippet, we used the \emph{average predictions} over the fine snippets inside the coarse one (Figure~\ref{fig:pip} right).
\begin{figure}[t]
	\centering
	\includegraphics[width=\linewidth]{figure/pipeline.pdf}
    \vspace{-8mm}
	\caption{Average feature versus average prediction testing. $\theta,f$: the feature backbone and anomaly classifier, respectively.}
	\label{fig:pip}
    \vspace{-6mm}
\end{figure}

\noindent\textbf{Baseline}. We built a baseline to validate that the improvements of UMIL are indeed from the unbiased training scheme (Section~\ref{sec:Abla}), rather than the above testing scheme based on average predictions. Specifically, the baseline has exactly the same model design as UMIL, and we trained it with the MIL objective in Eq.~\eqref{eq:1} on fine snippets and tested it by averaging predictions. Hence the only difference between the baseline and UMIL is the training objective.

\noindent\textbf{Model Training}. We implemented the backbone $\theta$ with the X-CLIP-B/32 model~\cite{xclip} fine-tuned on Kinectics-400~\cite{carreira2017quo} to improve its capabilities in action recognition. We used the fully connected layer to implement the anomaly classifier $f$ and the cluster head $g$.
We trained our model with the AdamW optimizer~\cite{loshchilov2019decoupled} using an initial learning rate of $8$e-$6$, weight decay of $0.001$, and batch size of $8$.
We utilized the cosine annealing scheduler and warmed up the learning rate for 5 epochs.
Our UMIL model was pre-trained with MIL for 30 epochs, followed by 10 epochs of UMIL training.
We conducted all experiments on $4$ TITAN RTX GPUs.
We implement the max value scores as well as max margin scores~\cite{lv2021localizing} in $\mathcal{C}$ supervision of Eq~\ref{eq:4}.
We also incorporated entropy minimization as a standard auxiliary objective~\cite{liu2021cycle,long2018conditional}, and added the self-training loss, which leverages the learned unbiased anomaly classifier $f$ to generate accurate pseudo-labels on samples in the ambiguous set $\mathcal{A}$ for additional supervision. Details in Appendix.

\subsection{Main Results}
\label{sec:4.3}
\noindent\textbf{UCF-Crime and TAD}.
In Table~\ref{tab:ucf-crime}, we compared our UMIL with other state-of-the-art (SOTA) methods in both Unsupervised VAD (UVAD) and WSVAD. On UCF-Crime~\cite{sultani2018real}, UMIL achieves the best $\mathrm{AUC}_O$ and $\mathrm{AUC}_A$ among all the methods, with an improvement of +1.37\% and +1.3\%, respectively. UMIL also significantly outperforms all methods in TAD~\cite{lv2021localizing} by +3.3\% on $\mathrm{AUC}_O$ and +4.2\% on $\mathrm{AUC}_A$.

\begin{table}[t]
    \centering
    \scalebox{0.8}{
    % \resizebox{\linewidth}{!}{%
    \begin{tabular}{@{}c|c|c|c}
      \toprule\hline
        Category & Method         & $\mathrm{AUC}_O$ (\%) & $\mathrm{AUC}_A$ (\%)\\ 
      \hline\hline
      \multirow{6}{*}{\rotatebox{90}{UVAD}}
      & SVM Baseline   & 50.00  & 50.00     \\
      & Conv-AE~\cite{hasan2016learning}   & 50.60    & -   \\
      & Sohrab et al.~\cite{sohrab2018subspace}  & 58.50  & -  \\
      & Lu et al.~\cite{lu2013abnormal}  & 65.51  & -  \\
      & BODS~\cite{wang2019gods}           & 68.26  & -  \\
      & GODS~\cite{wang2019gods}           & 70.46  & -  \\ \hline
      \multirow{9}{*}{\rotatebox{90}{WSVAD}}
      & Sultani et al.~\cite{sultani2018real} & 75.41 &54.25    \\
      & Zhang et al.~\cite{zhang2019temporal}            & 78.66  & -    \\
      & Motion-Aware~\cite{zhu2019motion} & 79.10   & 62.18    \\
      & GCN-Anomaly~\cite{zhong2019graph} & 82.12  & 59.02    \\
      & Wu et al.~\cite{Wu2020not} & 82.44  & -    \\
      & RTFM~\cite{tian2021weakly}          & 84.30 & -   \\ 
      & WSAL~\cite{lv2021localizing}          & 85.38  & 67.38\\ 
      & \cellcolor{mygray}Baseline & \cellcolor{mygray}80.67  & \cellcolor{mygray}60.57 \\ 
      & \cellcolor{mygray}\textbf{UMIL}  & \cellcolor{mygray}\textbf{86.75}  & \cellcolor{mygray}\textbf{68.68} \\ \hline\bottomrule
    \end{tabular}%
    % }
    }
    \vspace{-2mm}
    \caption{Frame-level AUC performance on UCF-Crime. Best results in bold. $\mathrm{AUC}_O$ and $\mathrm{AUC}_A$ denote that the AUC computed on the overall test set and only abnormal test videos, respectively. ``UVAD'' and ``WSVAD'' under category denote Unsupervised VAD and Weakly-Supervised VAD, respectively.} 
    \label{tab:ucf-crime}
    \vspace{-4mm}
\end{table}

\noindent\textbf{Overall Observations}.
1) Notice that our baseline performs similarly (\eg, $\mathrm{AUC}_O$ on TAD) or even worse (\eg, 60.57\% versus 67.38\% on UCF-Crime $\mathrm{AUC}_O$) compared to existing MIL-based methods. This validates that the improvements from UMIL are not from the test scheme of averaging predictions. 
2) In particular, our improvement in $\mathrm{AUC}_A$ indicates that the superior performance of UMIL on $\mathrm{AUC}_O$ is not merely from easy normal videos, but also from improved capabilities to identify anomalous snippets in abnormal videos.
3) Moreover, on both datasets, WSVAD significantly improves over UVAD on $\mathrm{AUC}_O$, which empirically validates that detecting open-set anomalies in UVAD is ill-posed (Section~\ref{sec:intro}). However, the improvements in $\mathrm{AUC}_A$ are much smaller (\eg, 54.25\% over 50.00\% on UCF-Crime). This shows that the existing WSVAD methods are still biased toward the apparent normal/abnormal, causing many false positives and negatives on ambiguous snippets from the abnormal videos.
4) Our UMIL significantly improves the $\mathrm{AUC}_A$ over MIL (\eg, +4.2\% on TAD), which demonstrates the effectiveness of using ambiguous snippets in UMIL to learn an unbiased invariant classifier.
5) Interestingly, TAD tends to have larger $\mathrm{AUC}_O$ but lower $\mathrm{AUC}_A$, \eg, from UCF-Crime to TAD, UMIL's $\mathrm{AUC}_O$ is 6.2\% higher, but $\mathrm{AUC}_A$ is 2.8\% lower. The improved overall performance suggests that TAD has stronger context bias in the confident set, \ie, more apparent normal/abnormal snippets, and the dropped $\mathrm{AUC}_A$ indicates that it contains more subtle anomalies in the ambiguous snippets that are hard to detect and localize.
This also explains why our UMIL improves $\mathrm{AUC}_A$ more on TAD than UCF-Crime by incorporating ambiguous snippets to remove the context bias from the confident set.

\begin{table}[t]
    \centering
    \scalebox{0.95}{
    % \resizebox{\linewidth}{!}{%
    \begin{tabular}{@{}c|c|c|c}
      \toprule\hline
      Category       & Method         & $\mathrm{AUC}_O$ (\%) & $\mathrm{AUC}_A$ (\%)\\ \hline\hline
      \multirow{3}{*}{\rotatebox{90}{UVAD}}
      & SVM Baseline   & 50.00  & 50.00     \\
      & Luo~\etal~\cite{luo2017revisit}  & 57.89  & 55.84  \\
      & Liu~\etal~\cite{liu2018future}           & 69.13 & 55.38   \\ \hline
      \multirow{6}{*}{\rotatebox{90}{WSVAD}}
      & Sultani~\etal~\cite{sultani2018real} & 81.42 &55.97    \\
      & Motion-Aware~\cite{zhu2019motion} & 83.08  & 56.89    \\
      & GIG~\cite{lv2020global}          & 85.64 & 58.65   \\ 
      & WSAL~\cite{lv2021localizing}          & 89.64  & 61.66\\ 
      & \cellcolor{mygray}Baseline & \cellcolor{mygray}89.10 & \cellcolor{mygray}56.47 \\ 
      %& MIL baseline + RTFM\cite{tian2021weakly}        &  91.28 & 57.65 \\ 
      & \cellcolor{mygray}Ours & \cellcolor{mygray}{\textbf{92.93}} & \cellcolor{mygray}{\textbf{65.82}} \\ \hline\bottomrule
    \end{tabular}%
    % }
    }
    \vspace{-3mm}
    \caption{Frame-level AUC performance on TAD benchmark.} 
    \vspace{-3mm}
    \label{tab:tad}
\end{table}

\begin{table}[t]
\centering
\scalebox{0.75}{
% \resizebox{\linewidth}{!}{%
\begin{tabular}{cccc|cc}
\toprule\hline
Baseline & ST & RTFM* & UMIL &  $\mathrm{AUC}_O$ (\%) - UCF  &  $\mathrm{AUC}_O$ (\%) - TAD\\ \hline \hline
\checkmark & & & & 80.67 & 89.10 \\
\checkmark & \checkmark & & & 82.01 & 90.80\\ \hline
\checkmark & \checkmark & \checkmark & & 83.45 & 91.28 \\ 
\checkmark & & & \checkmark & 83.66 & 91.74 \\ 
\cellcolor{mygray}\checkmark & \cellcolor{mygray}\checkmark & \cellcolor{mygray} & \cellcolor{mygray}\checkmark & \cellcolor{mygray}\textbf{86.75} & \cellcolor{mygray}\textbf{92.93}  \\ \hline
\bottomrule
\end{tabular}%
}
\vspace{-3mm}
\caption{Ablation studies of the components in UMIL on UCF-Crime and TAD. *: we re-implemented RTFM with our backbone and average-prediction-based testing scheme for fair comparison.}
\vspace{-3mm}
\label{tab:ablation}
\end{table}

\begin{table}[t]
\centering
\scalebox{1.0}{
% \resizebox{\linewidth}{!}{%
\begin{tabular}{cccccc}
\toprule\hline
Threshold(\%) & 10 & \cellcolor{mygray}\textbf{30} & 50 & 70 & 90  \\ \hline \hline
$\mathrm{AUC}_O$ (\%) - UCF & 86.8 & \cellcolor{mygray}\textbf{86.8} & 85.9 & 84.3 & 83.1 \\ 
$\mathrm{AUC}_O$ (\%) - TAD & 92.7 & \cellcolor{mygray}\textbf{93.0} & 92.8 & 91.5 & 91.1 \\ \hline
\bottomrule
\end{tabular}%
}
\vspace{-3mm}
\caption{Ablation on the threshold to divide the confident/ambiguous snippet set on UCF-Crime and TAD.}
\label{tab:thre}
\vspace{-5mm}
\end{table}
\subsection{Ablations}
\label{sec:Abla}

\noindent\textbf{Components}. Our approach has 2 main components: 1) the self-training objective; 2) the UMIL objective in Eq.~\eqref{eq:4}. We validate the effectiveness of each component in Table~\ref{tab:ablation} with $\mathrm{AUC}_O$. All ablations in the table are on the equal ground---using average prediction instead of average feature for anomaly detection (\ie, Baseline). By comparing the first two lines, we observe that self-training can improve $\mathrm{AUC}_O$ from $80.67\%$ to $82.01\%$ on UCF-crime and $89.10\%$ to $90.80\%$ on TAD. To independently evaluate the effectiveness of UMIL objective, we re-implement the SOTA RTFM~\cite{tian2021weakly} using our backbone and add the self-training objective, namely RTFM*. The result is listed in line 3. Our UMIL in line 4 still significantly outperforms RTFM* (+$3.3\%$ on UCF-crime and +$1.7\%$ on TAD), hence validating the effectiveness of our unbiased learning objectives.
\begin{figure}[t]
	\centering
	\includegraphics[width=0.4\textwidth]{figure/cm.pdf}
    \vspace{-4mm}
	\caption{Ablations on the trade-off parameters.}
	\label{fig:cm}
    \vspace{-4mm}
\end{figure}

\noindent\textbf{Confident Threshold}. We then conducted experiments to analyze the effects of the variance threshold for dividing confident and ambiguous snippets as in Section~\ref{sec:step1}.
Specifically, we selected $k$ (\%) training snippets with the minimum variance on their prediction history with varying $k$ as in Table~\ref{tab:thre}. Overall the threshold is easy to determine, \ie, 10-50\% is a reasonable range with 30\% being the best.

\noindent\textbf{Trade-off Parameters}. Recall that we use $\alpha$ and $\beta$ in Eq.~\eqref{eq:4} as the trade-off for the supervision from the ambiguous set $\mathcal{A}$ and clustering, respectively. We empirically find in Figure~\ref{fig:cm} that $\alpha,\beta=0.1$ are suitable across the two datasets, hence we used this setting in the experiments by default. In general, the choice of $\alpha$ depends on the strength of the context bias in the confident set, \eg, TAD has strong bias as analyzed in Section~\ref{sec:4.3}, which cannot be overcome with a small $\alpha$ (\eg, $\alpha$=0.01 has low performance).

\begin{figure}
    \centering
    \footnotesize
    \scalebox{1.05}{
    \begin{subfigure}[t]{0.23\textwidth}
         \includegraphics[width=\textwidth]{figure/rocs_u.pdf}
         \phantomcaption
         \label{fig:roca}
    \end{subfigure}
    \begin{subfigure}[t]{0.23\textwidth} % the hidden unwanted image
         \includegraphics[width=\textwidth]{figure/rocs_t.pdf}
         \phantomcaption
         \label{fig:rocb}   
    \end{subfigure}}
    \vspace{-8mm}
    \caption{ROC curves on UCF and TAD. Note that we only show part of the curves for visual clarity, as the other part of the methods have a large overlap when the true positive rate approaches 100\%.}
    \label{fig:roc}
    \vspace{-6mm}
\end{figure}

\noindent\textbf{Class-wise AUC}. On UCF-Crime dataset, the class of anomaly in each test video is given. This allows us to plot the class-wise $\mathrm{AUC}_A$ to examine models' capabilities to detect subtle abnormal events. In Figure~\ref{fig:hist}, we compared UMIL with baseline and RTFM*, where ``Average'' shows the overall $\mathrm{AUC}_A$ and the rest shows the class-wise one.
We have the following observation:
1) Both of the two MIL-based methods perform well on human-centric anomaly classes with drastic motions, \eg, ``Assault'' and ``Burglary''. These classes correspond to apparent anomalies as the backbone expresses the human action feature well (fine-tuned on the action recognition Kinetics400 dataset\cite{carreira2017quo}).
2) However, we notice that they easily fail to distinguish anomalies with subtle motions, \eg, ``Arson'' and ``Vandalism'', as well as non-human-centric anomalies, \eg, ``Explosion''. These classes correspond to ambiguous anomalies discarded by the biased training in MIL.
3) Our UMIL performs similarly on the above apparent anomaly classes and much better on the other subtle anomalies, which largely contributes to the superior anomaly detection and localization performance.
Overall, observation 1 and 2 empirically verifies the biased prediction situation of MIL in Figure~\ref{fig:abstract} and Figure~\ref{fig:2}. In contrast, our UMIL convincingly improves the performance on ambiguous anomalies with almost no sacrifice on the confident ones, which validates the effectiveness of our approach, \ie, identifying the invariance between the two types of anomalies to remove the bias in MIL.
\begin{figure}[t]
	\centering
	\includegraphics[width=\linewidth]{figure/subtle.pdf}
    \vspace{-6mm}
	\caption{Class-wise $\mathrm{AUC}_A$ of three methods on UCF-Crime.}
	\label{fig:hist}
    \vspace{-7mm}
\end{figure}

\begin{figure*}[t!]
	\centering
    \vspace{-4mm}
	\includegraphics[width=1\textwidth]{figure/cases_new.pdf}
    \vspace{-10mm}
	\caption{Visualization cases of ground-truth and anomaly score curves of various approaches. The white and black triangles denote the location of the normal and abnormal frame displayed on the left, respectively. The green curves represent the anomaly predictions of various methods. The pink background corresponds to the ground-truth abnormal regions.}
	\label{fig:cases}
    \vspace{-4mm}
\end{figure*}

\noindent\textbf{ROC Curve}. In Figure~\ref{fig:roc}, we draw the ROC Curve on the overall test set for our baseline, the re-implemented RTFM* and UMIL, which shows the true and false positive rate for detecting anomaly on a sweeping threshold over the predictions. VAD is evaluated using the area under this curve to demonstrate the overall separation of normal and abnormal snippet predictions. However, when applying a detector for real-world usage, we need to choose a specific threshold (\eg, with a maximum tolerable false positive rate). We observe from Figure~\ref{fig:roc} that our UMIL outperforms the two MIL baselines in every inch, which further shows the strength of our proposed unbiased training.

\noindent\textbf{Qualitative Analysis}. In Figure~\ref{fig:cases}, we show the continuous predictions of anomaly probabilities from our baseline, RTFM*, and our UMIL on 4 test videos on UCF-crime. We summarize the observations:
1) For the MIL baseline (2nd column), we observe that it assigns a larger probability on the pre-explosion snippets from B1 and B2 (top two videos), \eg, workers performing maintenance and snippets with smoke, yet the actual explosion may have a lower prediction (\eg, comparing the height of the green lines on the white and black triangle locations). Similarly, on B3, the running person (white triangle) triggers a larger anomaly prediction than the actual vandalism (black triangle). This further illustrates the biased prediction problem in MIL.
2) RTFM (3rd column) uses feature magnitudes to assist anomaly detection by assuming anomalous snippets have larger magnitudes, which indeed improves over the baseline sometimes, \eg, R2 is no longer biased to smoke. However, its assumption has no guarantee to hold and hence the failure on subtle anomalies persists, \eg, false alarm in R1 white triangle location and low prediction in R3 black triangle location.
3) In contrast, our UMIL localizes the anomalies accurately in U1-U3, \eg, having consistently high scores in the pink areas, which holds its ground on the name ``unbiased''.
4) In the 4th video, however, RTFM's prediction in the pink area is more consistent than ours. By inspecting the frames on the left, we realize that the two peaks in the pink area of U4 correspond to the burning fire and the running suspect caught on fire. Hence UMIL's prediction is reasonable and sufficient for triggering the alarm on the first peak.

\noindent\textbf{Computational Efficiency}. Lastly, we investigated the speed of the proposed model.
For inference, our method processes a 5-frame clip in $0.003$ seconds on a Nvidia 2080Ti GPU.
Notably, this is almost $80 \times$ faster than the SOTA RTFM~\cite{tian2021weakly}, which spends 0.76 seconds to process a 16-frame clip on Nvidia 2080Ti.
Thanks to our unbiased training scheme, we can fine-tune the backbone to learn a WSVAD-tailored representation, which achieves even better performance than existing SOTA.
This also shows the promising future of UMIL in real-time applications.
%\section{}
%\label{sec:resDir}


\section{Conclusion}
\label{sec:conclusion}
% <>
Since its advent in 1931, Koopman operator theory \cite{koopman:1931} has only recently been actively utilized for solving practical problems, thanks to the introduction of the DMD algorithm in 2008 \cite{schmid:2008}. Since then, a multitude of DMD algorithm variations have risen to prominence and found utility across various fields. A notable feature of our survey paper was reviewing and categorizing the results of over 100 research papers based on both application and algorithm type in smart mobility and vehicle engineering  (see Table~\ref{tab1} and Section~\ref{sec:vehicApp}).  Additionally, this survey paper identified potential research gaps in smart mobility and vehicular engineering applications (Remarks~\ref{remGap1}--\ref{remGap6}). Finally, this review paper discussed theoretical aspects of Koopman operator theory that have been largely neglected by the smart mobility and vehicle engineering community and yet have large potential for contributing to solving open problems in these areas (see Section~\ref{subsec:theorIssue}).

\noindent{\textbf{Future Research Directions.}}	Given the emergence of cyber-threats against connected and autonomous vehicles as well as robotic systems (see, e.g.,~\cite{nekouei2021randomized,mohammadi2022generation}), a future research direction might include utilizing Koopman operator-based algorithms for designing cyber-resilient vehicular and smart mobility applications (see, e.g.,~\cite{taheri2022data} for a related line of research). Another potential research direction is using Koopman operator-based algorithms for predicting the motion of vulnerable road users (VRUs), e.g., pedestrians and cyclists (see, e.g.,~\cite{pool2019context,scholler2020constant}). Finally, rehabilitation robotics and robotic exoskeletons can be the benefactors of the predictive capabilities of Koopman operator-based algorithms for detecting tripping events and/or system  identification in various modes of locomotion (see, e.g.,~\cite{kumar2019extremum,aprigliano2019pre}).



%Fig. 1 depicts the accumulation of such algorithms since 2014, which are particular to vehicle engineering and smart mobility, i.e., the focus of this review. Table 1 summarizes the varieties of relevant algorithms developed in those studies. Furthermore, we have highlighted theoretical issues, whose expansion will have potential applications to the wide research area of smart mobility and vehicle engineering.  

%Although fairly comprehensive, we have found several gaps in this research area. In particular, we could not find any studies related to elevators, robots/vehicles employing crawling, slithering, hopping or peristaltic locomotion, arctic or special-terrain vehicles such as those employing screws or tracks, hovercraft and other amphibious vehicles or subsystems which tolerate flexible environments, classification or guidance systems related to vehicles for drilling or agriculture, or for current-ripple, power-split, battery health monitoring, nuclear propulsion, exoskeletons/prosthetics, personal mobility, motorsports, specialized rovers or similar open problems in emerging areas.  These examples are, of course, not exhaustive.  
%
%The purely data-driven nature of Koopman operators holds the promise of capturing unknown and complex dynamics for reduced-order model generation and system identification, through which the rich machinery of linear control techniques can be utilized. The emergent nature of the smart mobility and vehicular-related applications, where  the Koopman operator  in each particular application needs to be approximated, implies that the development of various Koopman operator approximation  algorithms is expected to grow along with the vehicular problems they aim to solve.  Given the ongoing development of this research area and the many existing open problems in the fields of smart mobility and vehicle engineering, a survey of techniques and open challenges of applying Koopman operator theory to this vibrant area is warranted.  To the best of our knowledge, this survey paper is the \emph{first of its kind} reviewing the applications of Koopman operator theory within a focused research area, namely, smart mobility and vehicle engineering applications. A \emph{notable feature} of our survey paper is reviewing and categorizing the results of over 100 research papers based on both application and algorithm type  (see Tables~\ref{tab1}--~\ref{tab4} and Section~\ref{sec:vehicApp}) that are concerned with the applications of Koopman operator theory to the field of smart mobility and vehicular engineering. Such a \emph{comprehensive and  detailed categorization} will be beneficial to the research practitioners working in the field.  Furthermore, this review paper discusses theoretical aspects of Koopman operator theory that have been largely neglected by the smart mobility and vehicle engineering community and yet have large potential for contributing to solving open problems in these areas. Additionally, our survey paper seeks to \emph{identify gaps} in the smart mobility and vehicle engineering research where new and existing Koopman operator-based methods have the potential to further develop and address unsolved problems  potentially benefiting from the perspectives of nonlinear system identification, control, global linearization, and the predictive powers that Koopman operator theory has to offer (see, e.g., Remarks~\ref{remGap1}--\ref{remGap6}). 


\appendix

\section*{Appendix 1: Runtime \& Parameters Analysis}
Table \ref{runtime} presents the training time and a number of parameters on the S3DIS dataset, based on experiments conducted using the Point Transformer backbone, which yielded top-ranking results. The inclusion of a global branch results in an increase in the number of parameters, but it has a manageable effect on training time, as shown in Table \ref{runtime}. This is because the global branch processes points that are typically 3-4 orders of magnitude smaller than the original points (approximately 700-1000 points per S3DIS scan and 900-1200 points on ScanNetV2 scans). The smaller size of global points allows faster processing by the network. Despite the fact that the Geospark+Point Transformer slightly underperforms Stratified Transformer on the S3DIS dataset, we have fewer parameters and shorter training times, demonstrating the design's greater efficiency.

The \textit{geometry partition} module can process 100k points within 2-3 seconds, and the entire S3DIS dataset can be processed in less than an hour on a modern eight-core CPU.


\begin{table}[!htp]\centering
\caption{\textbf{Run time \& No. Parameter Comparison on S3DIS}. Tested on two A100 GPUs with batch size 16. }
\scriptsize
\vspace{2mm}
\begin{tabular}
{lcc}\toprule
\label{runtime}
Methods& Train time(H)$\downarrow$ &No.Para(M) $\downarrow$ \\ \hline
PointTransformer &26 &10.2  \\ 
PointTransformer + GeoSpark &30 \textcolor{purple}{(+15\%)} &15.3 \textcolor{purple}{(+50\%)} \\
Stratified Transformer &100 \textcolor{purple}{(+233\%)} &18.8 \textcolor{purple}{(+84\%)} \\
SparseUNet &42\textcolor{purple}{(+61\%)} &38.0 \textcolor{purple}{(+272\%}\\
\bottomrule
\end{tabular}
\end{table}
\maketitle
% Remove page # from the first page of camera-ready.
\ificcvfinal\thispagestyle{empty}\fi

{\small
\bibliographystyle{ieee_fullname}
\bibliography{egbib}
}

\end{document}
