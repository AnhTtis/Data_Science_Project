\section{Introduction} 

\begin{figure}[t!]
 \centering
 %\includegraphics[width=0.6\textwidth]{fig/visual_reg_result_v1.pdf}
 \includegraphics[width=\linewidth]{section/images/Geospark.jpg}
 \caption{\textbf{Illustration of \textbf{GeoSpark} on feature aggregation and downsampling modules.} Current feature learning predominantly depends on local aggregation (shown in \textbf{a}). Introducing GeoSpark not only expands the receptive field but also concentrates on the relevant areas (shown in \textbf{b}). For instance, if the query point is located at the end of the bed, Geometry-Informed Aggregation will expand the receptive field to cover the entire area of the bed, as well as the bed mat on the floor. Moreover, \textbf{GeoSpark} also enhances sampling by using geometry clues as guidance. Compared to FPS (shown in \textbf{c}), Geometric Downsampling (shown in \textbf{d}) samples coarsely in areas of simple geometry and densely in areas with complex geometry, such as a dishwasher, a painting on the wall, and objects on the table. The joint of these two modules yields impressive segmentation results.} 
 \label{stitching_effect}
 \vspace{-4mm}
\end{figure}

Since PointNet++\cite{Qi2017PointNet++:Space}, mainstream point cloud segmentation methods \cite{Zhao2019Pointweb:Processing, Jiang2018PointSIFT:Segmentation, LiPointCNN:Points, Boulch_2020, Thomas2019KPConv:Cloudsb} conduct feature aggregation with local neighborhood, because learning global features for large-scale point clouds is infeasible. However, due to the redundancy in point clouds, local neighborhoods often contain a high percentage of similar points, limiting models' ability to learn from diverse contexts. For this reason, how to model long-range features in point cloud is a long-standing challenge in point cloud processing \cite{Lai_2022, Hu2019RandLA-Net:Clouds, Peng2023}.

Moreover, hierarchical paradigms are predominantly utilized in point cloud understanding architectures, where points undergo multiple downsampling stages to learn features at various scales. These frameworks typically employ data-agnostic sampling methods, such as FPS \cite{Qi2017PointNet++:Space}, voxelization \cite{Graham_2015}, or random sampling \cite{Hu2019RandLA-Net:Clouds}, which tend to dilute points for small objects. This is due to the severe data imbalance issues present in point clouds. The reduction in the number of points during later stages restricts the model's ability to learn expressive features for small objects, resulting in unsatisfactory performance. 



 \begin{figure*}[ht!]
 \centering
 \includegraphics[width=1.0\textwidth]{section/images/GIA.pdf}
 \centering
\vspace{-3mm}
  \caption{Illustration of \textit{Geometric Partition \& Embedding} and \textit{Geometry-Informed Aggregation} Modules. Geometric Partition \& Embedding split point clouds into simple geometric shapes, such as table surfaces, chair backs, table edges etc, which are then embedded into superpoints. The query point \(x_i\) conducts local aggregation with \(k_{local}\) local neighbour points \(x_j\), and then performs Geometry Partition aggregation with \(k_{global}\) superpoints \(\hat{x_j}\) to acquire global contexts. The \textit{Geometry-Informed Aggregation} module allows the model to learn both local fine details as well as expressive global contexts.}
 \label{attention_ill}
 \vspace{-4mm}
\end{figure*}


We argue that these challenges can be mitigated by introducing explicate geometry clue into the framework and preformed geometry-aware feature learning and downsampling. We define the geometry information as an initial geometry partition that can be obtained with conventional point cloud processing techniques \cite{Hui_2021, LandrieuCutFunctions, Landrieu2019PointLearning,Guinard2017WeaklyClouds,Lin_2018} where redundancy point clouds can be clustered into simple shapes. To this end, we introduce a Plug-in module named {\textbf{GeoSpark}} that can be integrated with various point cloud segmentation backbones to spark up semantic segmentation results by enhancing feature aggregation and the downsampling process. 

Specifically, for feature aggregation, GeoSpark offers a \textit{Geometry-Informed Aggregation} (GIA) module which attends to both local neighbor points and geometry partition entities. Unlike methods that rely solely on local neighborhoods, our approach offers two compelling advantages. Firstly, it significantly enlarges the receptive field of the feature aggregation module without dramatically increasing the number of reference points. This is achieved by encoding geometry partition into superpoints, which provide a compact yet information-rich format for the raw input. Secondly, the geometry partition entities are highly tailored to the input point cloud data. Consequently, the receptive field for each point is tailored to its specific geometry, enabling the model to learn from relevant regions rather than data-agnostic areas. This concept is akin to the methodology employed in the deformable structure \cite{DBLP:journals/corr/abs-2010-04159, Xia_2022_CVPR, Thomas2019KPConv:Clouds, DBLP:journals/corr/DaiQXLZHW17}, which learns offset values from the originally prescribed regions. Here, we achieve a similar effect without requiring additional offset design.

For the downsampling module, GeoSpark introduces \textit{Geometric Downsampling} (GD). Unlike conventional methods such as Further Point Sampling, Random Sampling, and Voxelization which drop points without considering their importance, our approach uses geometry partition as guidance and fuses points of redundancy into one while keeping points with unique features. In other words, the downsampling is the process of reducing point redundancy, ensuring that important points are retained. This data-dependent downsampling approach has been shown to be very beneficial, especially for small objects.

These two simple yet powerful module designs can be incorporated into various backbone models. We tested them on three backbone models, including Pointnet++ \cite{Qi2017PointNet++:Space}, KPConv \cite{Thomas2019KPConv:Cloudsb}, and Point Transformer \cite{Engel_2021}, and consistently observed improvements. In summary, our contribution is threefold.

\begin{itemize}
\item We demonstrate that utilizing geometric partition to enhance feature aggregation is an effective approach for performance gain. It not only expands the receptive field in a cost-efficient manner but also concentrates on relevant regions that are customized for the query point.

\item We enhance the downsampling module in the architecture by fusing redundant points and retaining points with unique features, resulting in better information retention throughout the network.

\item GeoSpark has proven to be universal and transferable to various baseline models. Notably, when incorporating it with Point Transformer \cite{Zhao_2021}, it yielded a 4.1\% improvement in the mIoU score on the ScanNetV2 \cite{Dai_2017} dataset, and achieved a score of 71.5\% on the S3DIS \cite{Armeni_2016} dataset, ranking high on both benchmarks.

\end{itemize}