\begin{table*}[t]
    \begin{minipage}{.48\textwidth}
\centering
% \tablestyle{9pt}{1.2}
\begin{tabular}{c|ccc|ccc}\toprule
\textbf{ID} &\textbf{GD} &\textbf{GIA} &\textbf{\(\mathcal{L}_{sup}\)} &{mIoU} &mAcc &OA \\\midrule
I & & & & 72.4 & 81.3 & 90.6\\
II & &\checkmark & & 72.8 & 81.3 & 90.7\\
III & &\checkmark &\checkmark &73.8 & 82.1 & 91.0\\
IV &\checkmark & & & 73.5 & 81.4 & 90.7\\
\cellcolor[HTML]{E8E8E8}VI &\cellcolor[HTML]{E8E8E8}\checkmark &\cellcolor[HTML]{E8E8E8}\checkmark &\cellcolor[HTML]{E8E8E8}\checkmark &\cellcolor[HTML]{E8E8E8}\cellcolor[HTML]{E8E8E8}{74.7} & \cellcolor[HTML]{E8E8E8}{82.7} &\cellcolor[HTML]{E8E8E8}{91.4}\\
\bottomrule
\end{tabular}
\vspace{1mm}
\caption{\textbf{Module ablation study.}. GD: \textit{Geometric Downsampling}. \textbf{GIA}: \textit{Geometry-Informed Aggregation}. \(\mathcal{L}_{sup}\): Superpoint loss. Each module demonstrated its ability to enhance the final results.}
\vspace{2mm}
\label{MODULE}
    \end{minipage}
    \hspace{8mm}
    \begin{minipage}{.45\textwidth}
% \tablestyle{10pt}{0.95}
\begin{tabular}{c|c|ccc}\toprule 
Dataset & SP. \textit{dia} & mIoU & mACC & OA \\\midrule
\multirow{3}{*}{{S3DIS}} & 0.5 & 71.1 & 76.8 & 91.2\\
& \cellcolor[HTML]{E8E8E8} 1 & \cellcolor[HTML]{E8E8E8} {71.5} & \cellcolor[HTML]{E8E8E8}77.3 &\cellcolor[HTML]{E8E8E8} 91.1 \\
&1.5 &70.8 & 76.5 & 90.7 \\ \midrule
\multirow{3}{*}{{ScanNetV2}} &2 & 74.2 & 82.4 & 91.2 \\
& 2.5 & 73.9 & 82.4& 91.2 \\
& \cellcolor[HTML]{E8E8E8}3 & \cellcolor[HTML]{E8E8E8}{74.7} & \cellcolor[HTML]{E8E8E8}{82.7} & \cellcolor[HTML]{E8E8E8} {91.4} \\
\bottomrule
\end{tabular}
\vspace{1mm}
\caption{\textbf{Ablation study of max superpoint diameter setting.} Selecting the appropriate size of superpoint optimize the performance, and the size is tailored to the dataset.}
\vspace{2mm}
\label{partition_cap}
    \end{minipage} \\
    \begin{minipage}{.45\textwidth}
    % \vspace{4mm}
\centering
% \tablestyle{10.5pt}{1.08}
\begin{tabular}{c|cc|ccc}\toprule
ID &\(k_{local}\) &\(k_{global}\) & mIoU &mAcc &OA \\\midrule
I &8 &4 &72.1 &80.6 &90.6 \\
II &8 &8 &72.5 &81.7 &90.6 \\
III &10 &5 &73.3 &82.0 &90.9 \\
IV &16 &0 &73.5 &82.1 &91.2 \\
V &16 &16 &74.2 &82.4 &91.2 \\
VI &16 &4 &74.4 &82.7 &90.7 \\
\cellcolor[HTML]{E8E8E8}VII &\cellcolor[HTML]{E8E8E8}16 &\cellcolor[HTML]{E8E8E8} 8 &\cellcolor[HTML]{E8E8E8}{74.7} & \cellcolor[HTML]{E8E8E8}{82.7} &\cellcolor[HTML]{E8E8E8} {91.4} \\
\bottomrule
\end{tabular}
\vspace{0.2mm}
\caption{\textbf{Ablation study on local-global balance.} Maintaining a balance between global and local features is crucial for optimizing feature aggregation while minimizing noise introduction.}
\label{balance}
    \end{minipage}
    \hspace{8mm}
    \begin{minipage}{.48\textwidth}
\centering
% \tablestyle{7pt}{1.08}
\begin{tabular}{c|c|ccc}
\toprule
Method &Sample ratio/size &mIoU & mACC & OA \\\midrule
\multirow{2}{*}{FPS} &\(1/4\) &73.8& 82.1& 91.2 \\
&\(1/6\) &72.3 & 81.0 & 90.1 \\ \hline
% Voxel &\([0.25, 0.50, 0.75, 1.00]\) &73.5&82.1 &91.1 \\
\multirow{2}{*}{Voxel} &\([~0.10, 0.20, 0.40, 0.80~]\) &74.2& 82.1& 91.2 \\
&\([~0.25, 0.50, 0.75, 1.00~]\) &74.3 & 81.7 & 90.2 \\ \hline
\multirow{3}{*}{GD} &\([~0.06, 0.15, 0.38, 0.90~]\) &74.1& 82.1& 91.2 \\
&\([~0.10, 0.20, 0.40, 0.80~]\) &74.3& 82.1& 91.2 \\
&\cellcolor[HTML]{E8E8E8} \([~0.25, 0.50, 0.75, 1.00~]\) 
&\cellcolor[HTML]{E8E8E8}{74.7} 
&\cellcolor[HTML]{E8E8E8}{82.7} 
&\cellcolor[HTML]{E8E8E8} {91.4}\\
\bottomrule
\end{tabular}
\vspace{0.5mm}
\caption{\textbf{Ablation Study on Downsampling Approach} Geometric Downsampling (GD) is generally superior to voxel sampling and FPS, regardless of the partition size setting. Additionally, fusing-based sampling outperforms FPS.}
\label{ab_samp}
    \end{minipage}
    \vspace{-5mm}
\end{table*}