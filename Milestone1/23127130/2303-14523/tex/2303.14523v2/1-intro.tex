
%https://github.com/OWASP/www-community


\section{Introduction}
\label{sec:intro}
%\hl{1. blockchain e smart contracts}

Smart contracts play an important role in advancing blockchain because they expand the application of the technology to various domains (e.g., education \cite{Grech2017a}, healthcare \cite{Zhang2022}, government \cite{Geneiatakis2020}). While they are essential for the consolidation and expansion of the technology, they also bring serious risks, namely those associated with the potential presence of vulnerabilities that can affect the security of the blockchain system \cite{Atzei2017}.

Just as conventional programs, smart contracts are being deployed with residual software faults (i.e., bugs or defects), including security vulnerabilities (i.e., internal faults that enable external events to harm the system) \cite{Qian2022, BasicConcepts2004}. However, the consequences of deploying a faulty contract have particular characteristics in the context of blockchain systems, such as: i) if faulty code is identified, the respective contract cannot be patched, it must be terminated, and a new one should be created \cite{8847638}; ii) once the potentially erroneous data (generated/updated by faulty contracts) has been stored in the blockchain, there is no way to change it, i.e., to undo the respective transactions (and subsequent transactions that rely on this data) \cite{Yaga2018}; and iii) if the faulty contract has been executed, the associated impact may be irreparable (e.g., reputation costs) \cite{antonopoulos_mastering_2018}.


Several initiatives have been created that ultimately aim at contributing to the development of more secure smart contracts.  Among these initiatives, we find three main types: i) New smart contract programming languages (e.g.,  Clarify  \cite{Blockstack2021}, Vyper \cite{Vyper.vs.solidity.2020}, Obsidian \cite{Coblenz2019}), which aim at increasing protection against faults; ii) New vulnerability detection tools (e.g., Mythril \cite{Mythril}, Neucheck \cite{lu_neucheck_2019}, \cite{SAILFISH_2022}, SoliDetector \cite{Hu2023}), which have the main goal of detecting vulnerabilities in smart contracts so that vulnerable contracts do not reach the deployment phase; and also, iii) and also vulnerability classifications that mostly allow knowledge regarding vulnerabilities to be identified in a standard manner and systematized.

The existence of vulnerability (or software defects, in general) classifications is quite important, as we can observe by the research and industry effort associated with well-known cases like OWASP \cite{OWASP2001}, NVD \cite{U.S.government1999}, CVE \cite{CVE1999}, CWE \cite{CWECommunity2009}, or, in the case of smart contracts most notably by SWC \cite{swc}, and DASP \cite{nccgroup_decentralized_2021}. Generally, they raise the level of awareness among developers and may allow, in a uniform manner, for development tools to assist developers regarding mistakes being placed in the code. They may also help in the design and development of vulnerability detection tools and in the assessment of their detection capabilities \cite{durieux_empirical_2020, Hu2021, di_angelo_survey_2019}. This case also holds for programming languages. It is known that languages, such as Obsidian, have benefited from the systematized knowledge of vulnerabilities. There are even studies that use taxonomies as a basis for comparing different programming languages with respect to the protection offered against certain types of vulnerabilities, e.g., \cite{Vyper.vs.solidity.2020}.

At the time of writing, vulnerability classifications for smart contracts have significant limitations. They are generally \textbf{outdated} with, at the time of writing, popular schemes like DASP or SWC not being updated since 2018. This largely differs from the state of the practice, in which we find cases of tools like Securify2 \cite{Tsankov2018} already detecting several vulnerabilities for which there is no accurate description. As in other software areas, with new vulnerabilities being continuously discovered, having a flexible way of integrating (and possibly restructuring the classification) new defects is crucial.

Vulnerability naming and classification schemes are being defined using \textbf{arbitrary nomenclatures}. This is easily visible just by analyzing a few of the most cited papers in vulnerability detection, e.g., \cite{luu_making_2016,tsankov_securify_2018,kalra_zeus_2018}. The lack of a standard nomenclature leads to verification tools mostly using arbitrary names to present their result, e.g., SmartCheck \cite{tikhomirov_smartcheck_2018} and Slither \cite{feist_slither_2019} respectively use \textit{balance equality} and \textit{incorrect-equality/locked-ether} to refer to the same vulnerability. As a result, it is very difficult to compare the effectiveness of different tools. As classifications are many times built based on multiple sources, such as different industry tools and several research papers \cite{Rameder2022}, terms easily end up being inconsistent. This is aggravated when there is no active maintenance even for known issues. Indeed, \textbf{Reduced community contribution} is known to be a problem, with the main classifications that are community-oriented (i.e., DASP, SWC) showing residual community activity, many times related to minor issues (e.g., broken links) \cite{DASP,swc}.

Many times, vulnerability classification schemes mix the characteristics of a certain vulnerability with the effect of exploiting it, how it is exploited, or its impact. This \textbf{concept inconsistency} is quite visible in current taxonomies. As an example, in \cite{Vyper.vs.solidity.2020} presents \textit{DoS with unbounded operation} as a vulnerability, but it is not possible to understand what the vulnerability is with this name (e.g., it can be a problem in a loop, it can be a malicious call that is externally triggered several times). Instead, the given name refers to the possible impact of exploiting a vulnerability, which should be a separate dimension for characterizing the defect. Similarly, this occurs in DASP \cite{DASP}, in which one of the categories is precisely \textit{Denial of Service}. Another aspect this latter example shows is that taxonomies are being built with \textbf{inadequate granularity}, often too coarse to be really helpful. For instance, the \textit{Denial of Service} category in DASP may refer to \textit{gas limit reached}, \textit{unexpected throw}, \textit{unexpected kill}, or \textit{access control breached}. Moreover, the description is sometimes so short that may become ambiguous (e.g., \textit{access control breached} may refer to a vulnerability that would simply fit in \textit{access control}, which is another DASP category).

\textbf{In this paper, we propose OpenSCV, a new hierarchical and Open taxonomy for Smart Contract Vulnerabilities} (available at \url{http://openscv.dei.uc.pt}), which is open to community contributions \cite{openscvGithub}, aims at matching the current state of the practice and is prepared to handle future modifications and evolution. To build the taxonomy, we analyzed current smart contract vulnerability classifications and discussed their gaps and limitations. We then analyzed the announced detection capabilities of 49 research works on smart contract vulnerability detection with the goal of collecting an heterogeneous set of 357 vulnerability definitions. We then mapped the vulnerabilities in existing classifications, namely DASP \cite{DASP}, SWC \cite{swc}, \cite{Rameder2022}, and CWE \cite{CWECommunity2009} and further characterized them using the Orthogonal Defect Classification (ODC) \cite{ODC,odcExtension} and with a code excerpt. Names were then consolidated and grouped in a structure that was built bottom-up. This process involved 2 Experienced Researchers and 1 Early Stage Researcher, which revised the proposed taxonomy iteratively in terms of structure, correctness, and uniformity.

We structured OpenSCV to allow it to be flexible to changes and evolution by preparing a supporting infrastructure at github. We are able to receive change requests easily and integration information from new research on vulnerability detection into the taxonomy. All OpenSCV entries are supported by a code example, with the goal of mitigating possible ambiguities in the description of each vulnerability and we also prepared an initial dataset holding vulnerable contracts (one per each of the vulnerabilities present in OpenSCV) and their respective correction. OpenSCV is live and available at \url{http://openscv.dei.uc.pt} \cite{openscvSite}, the github repository is available at \cite{openscvGithub} and linked to Zenodo which permanently hosts the dataset \cite{openscvZenodo}. It is worthwhile mentioning that the taxonomy considers mostly software vulnerabilities and a few software defects considered in the literature to be associated with high-security risks. For simplicity, \textit{we use the term vulnerability throughout the paper} to refer to both cases.

The rest of this paper is organized as follows. section 2 discusses the related work and limitations of current vulnerability classification schemes. Section 3 presents the process followed to build the taxonomy and overviews the final outcome. Section 4 presents the taxonomy structure and provides a brief description of all vulnerabilities included in the taxonomy. Section 5 characterizes and discusses the coverage of the taxonomy in perspective with the state-of-the-art. Section 6 presents the threats to the validity of this work, and finally, Section 7 concludes this paper.