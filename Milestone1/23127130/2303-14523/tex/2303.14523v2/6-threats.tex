\section{Threats to Validity}

This section discusses the main threats to the validity of this work. To minimize the chances of creating an \textit{incorrect structure or providing incorrect vulnerability information}, we formalized the taxonomy creation process, which was based on several quality criteria identified in the state of the art, and especially made use of several researchers (i.e., one Early Stage Researcher and 2 Experienced Researchers) who incrementally and iteratively built the taxonomy following a bottom-up approach. The process was enriched by establishing relations to other classifications in the blockchain context (i.e., SWC, DASP, \cite{Rameder2022}) and in a more general context (i.e., CWE). We also characterized each vulnerability using ODC and an example, which served also to minimize doubts and clear divergences among researchers. In addition, we provide the taxonomy as a live structure at \cite{openscvSite} supported by a Github repository \cite{openscvGithub} so that possible mistakes are corrected and also allow future updates, changes, and overall taxonomy evolution. 

We are aware that a classification or categorization scheme or \textit{a taxonomy may assume one of several possible forms}. We may have more or less main categories, we may have a deeper tree, the organization may or may not be hierarchical, and so on. While such diversity is acceptable (as long as the organization and individual items are correct), we opted to focus on the taxonomy creation process instead of on forcing a certain structure. For this purpose, we identified quality criteria, analyzed similar structures in the state of the art so that we could learn from possible mistakes and incorporate lessons learned by previous researchers. While the current structure is a proposal, we prepared it built to change and evolve, by opening it to the community and also by directly providing 'Request For Change' templates to facilitate changes or additions to the present form.

An important aspect is that the taxonomy creation process was guided by the research that was found during the analysis of the state of the art. Thus, we may have missed some relevant work in this context and, with time this gap may become greater. The fact that we were already aware of contributions coming from 3 areas: research on vulnerability classification, initiatives on vulnerability classification that are community-oriented, and research on vulnerability detection, allowed for a more efficient search, with which we believe captured representative research in this context. Despite this, and to mitigate possible gaps between the set of works considered to build OpenSCV and the set not captured during the collection of papers in this work, we prepared a supporting infrastructure to allow continuous update and evolution of OpenSCV. Thus, we will be able to capture and integrate new research in vulnerability detection that may bring in emerging smart contract vulnerabilities.