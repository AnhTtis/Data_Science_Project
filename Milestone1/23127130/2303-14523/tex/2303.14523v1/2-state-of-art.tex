\section{State of the Art}
\label{sec:SectionII}

This section presents the quality properties of taxonomies and then discusses the existing classification schemes for smart contract vulnerabilities. The classification schemes presented are have their origins in: a) existing research on smart contract vulnerability classification; b) community-oriented initiatives; and also c) vulnerability detection research. The section closes with a discussion of the gaps and limitations of current classification.

\subsection{Taxonomy Quality Properties}

We analyzed a set of reference works centered around the definition of taxonomies as well as critical analyses of vulnerability taxonomies \cite{Bishop1996, lindqvist_how_1997, Mann1999, Rameder2022, lough_taxonomy_2001, hansman_taxonomy_2005} to identify a set of quality properties criteria, which should be followed when designing a taxonomy that is expected to be long-lived. The following paragraphs discuss the identified properties.

A classification system may benefit from a \textbf{hierarchical organization} as it allows to show similar characteristics of related vulnerabilities, which may be helpful for vulnerability prevention \cite{Bishop1996}. A hierarchical structure may be a tree in which each node refers to a category of vulnerabilities, and each leaf corresponds to individual vulnerabilities. Thus, the granularity of the categories should generally vary from large to fine as we traverse the tree from the root to the leaves.

Nodes at a certain tree level must be as uniform as possible, i.e., ideally representing the same \textbf{level of abstraction} or a group of vulnerabilities viewed from the same perspective. Obviously, this is quite difficult to achieve because, many times, this has to be balanced with the creation of taxonomy trees that become too complex, which in the end, may make it less comprehensible or less helpful. Also, sometime the nature of the problem is simply an unbalanced or heterogeneous (in structure) one, which basically disallows this criteria. Anyway, a uniform taxonomy may contribute to fewer errors (in its use) and, as such, a higher probability of adoption by practitioners. In practice, it may contribute to a taxonomy that is \textbf{useful} and \textbf{comprehensible} \cite{lindqvist_how_1997} (i.e., understandable by security experts but also by less specialized people). 

The selection of names to be used in a classification scheme is particularly important. The name that describes a certain vulnerability must be a \textbf{unique identifier}) and \textbf{non-ambiguous} \cite{howard_analysis_1997, lindqvist_how_1997}, meaning that the name and also the associated description must allow not only for easy identification but should include enough information to distinguish it from other vulnerabilities \cite{Mann1999, Bishop1996}. Whenever possible, existing \textbf{terminology} should be used \cite{lindqvist_how_1997}. The name and characteristics of a certain defect should characterize what the issue is and not additional dimensions, such as the effect of exploiting it. While it is acceptable to understand the effect of exploiting a certain vulnerability starting from its description, the characteristics of the problem itself cannot be omitted and should be clearly identified \cite{Mann1999, Bishop1996}.

Regardless of the perspective of the individual using the taxonomy, a certain defect should be classified in the same manner by different individuals (e.g., developers, users, testers). This means that not only the names and structure should be as clear as possible, but also that the process of classifying a certain defect must be made clear (whenever the structure and nomenclature are not sufficient), i.e., there must be a \textbf{deterministic} \cite{krsul_software_1998} way of classifying a certain defect, which fosters \textbf{repeatability} \cite{howard_analysis_1997, krsul_software_1998} of using the classification.

Finally, a taxonomy should also allow for \textbf{completeness} \cite{amoroso_fundamentals_1994}, i.e., the taxonomy should provide a \textbf{good coverage} \cite{Rameder2022} of the vulnerabilities identified in state of the art or reported by vulnerability detection tools. Also, it should be \textbf{open to the community} (i.e., accept new entries from the community) and shareable (i.e., no distribution restrictions) \cite{Mann1999}. The fact that it is open is also a factor that can contribute to it being \textbf{accepted} \cite{amoroso_fundamentals_1994, howard_analysis_1997}.

%Mutually exclusive (Howard, 1997, Lindqvist and Jonsson, 1997): A mutually exclusive taxonomy will categorise each attack into, at most, one category.

\subsection{Smart Contract Vulnerability Classification Schemes}

To the best of our knowledge, the first initiative to classify smart contract vulnerabilities (for Ethereum systems) is proposed in \cite{Atzei2017}. The authors listed 12 vulnerabilities, which we overview in Table \ref{tab:atzei}, and implemented nine of the corresponding attacks.

% Please add the following required packages to your document preamble:
% \usepackage{multirow}
\begin{table}[ht]
\scriptsize
\centering
\caption{Classification proposed in \cite{Atzei2017}.}
\label{tab:atzei}
\begin{tabular}{|l|l|}
\hline
\textbf{Level}                                    & \textbf{Vulnerability} \\ \hline
{Solidity}                         & Call to the unknown     \\ \cline{2-2} 
                                                  & Gasless send           \\ \cline{2-2} 
                                                  & Exception disorders    \\ \cline{2-2} 
                                                  & Type casts             \\ \cline{2-2} 
                                                  & Reentrancy             \\ \cline{2-2} 
                                                  & Keeping secrets        \\ \hline
{EVM}                              & Immutable bugs         \\ \cline{2-2} 
                                                  & Ether lost in transfer \\ \cline{2-2} 
                                                  & Stack size limit       \\ \hline
\multicolumn{1}{|c|}{{Blockchain}} & Unpredicable state     \\ \cline{2-2} 
\multicolumn{1}{|c|}{}                            & Generating randomness  \\ \cline{2-2} 
\multicolumn{1}{|c|}{}                            & Time constraints       \\ \hline
\end{tabular}
\end{table}



This initial effort is quite relevant but holds some limitations. Some of the selected names do not really specify the nature of the vulnerabilities or are not clear about the problem being characterized (e.g., "call to the unknown"). This limitation was mitigated in \cite{Zhou2022} and \cite{arganaraz_detection_2020}, where the authors tried to make the names used more specific. In \cite{Atzei2017} three categories of issues are proposed: i) Solidity Issues (i.e., language weaknesses), ii) EVM Issues (i.e., residuals faults in byte code), and iii) Blockchain Issues (i.e., vulnerabilities from blockchain technology). Despite allowing an initial separation of the issues (which may help developers in dealing with the faults), this scheme does not benefit from the presence of a more complex hierarchy, which is a better fit for cases where we find several interrelated families of vulnerabilities. We have also identified that these three categories may generate some ambiguity as some cases could potentially fit into multiple categories. For example, \textit{Immutable Bugs} could be classified into EVM or Blockchain. Despite this, the separation between the cases referring to the programs (i.e., solidity source code or EVM binary code) and the blockchain platform is helpful. This classification is not available in a public repository and, due to its age, its coverage is relatively low, accounting for 12 evulnerabilities.

Table  \ref{tab:Vyper} overviews the vulnerability classification presented in \cite{Vyper.vs.solidity.2020},  which has the goal of allowing comparison between the security of the Solidity and Vype languages. The work presents 18 vulnerabilities, along with a detailed explanation for each one, and individual code examples for each vulnerability. Being mostly a list of vulnerabilities, there are no benefits associated with hierarchical structures. There is no open public repository associated with the proposal, and the 18 vulnerabilities are nowadays a small amount, e.g., the work in \cite{ Rameder2022} identifies a total of 54 defects.

 \begin{table}[ht]
\scriptsize
\centering
\caption{Classification proposal in \cite{Vyper.vs.solidity.2020}.}
\label{tab:Vyper}
\begin{tabular}{|l|}
\hline
\textbf{Vulnerabilities}             \\ \hline
Integer   overflow and underflow     \\ \hline
DoS with   unbounded operation       \\ \hline
Unchecked call   return value        \\ \hline
Reentrancy                           \\ \hline
Delegate call   injection            \\ \hline
Forced Ether to contract             \\ \hline
DoS with unexpected revert           \\ \hline
Erroneous visitility                 \\ \hline
Uninitialized storage pointer        \\ \hline
Upgradeable contract                 \\ \hline
Type casts                           \\ \hline
Insufficient signature   information \\ \hline
Frozen Ether                         \\ \hline
Authentication through tx.   Origin  \\ \hline
Unprotected suicide                  \\ \hline
Leaking Ether to arbitrary   address \\ \hline
Secrecy failure                      \\ \hline
Outdated compiler version            \\ \hline
\end{tabular}
\end{table}

A vulnerability classification is presented in \cite{arganaraz_detection_2020} with the goal of exposing threats and, ultimately, minimizing the presence of software bugs in smart contracts. Table \ref{tab:arganaraz} presents the proposed classification in which we find faults separated into two levels: i) security (i.e., faults that may be exploited by attacks; and ii) functional (i.e., faults that violate the program's functionality). Each fault is also associated with a criticality level, which may be useful for getting developers' attention while coding. We found that certain cases, such as \textit{Non-verified maths} and \textit{Malicious libraries}, may indicate the same vulnerability, indicating a potential need for further clarification and refinement of the classification process to address any ambiguity. Similar to the previously presented works, there is no hierarchical structure besides the two groups of vulnerabilities. The names used in the classification are quite specific (i.e.,  \textit{Use of tx.origin}), which makes it difficult to understand the problem in a more abstract manner. Only 13 faults are considered, with no possibility of expansion. Still, the idea of classifying the faults into two broad concepts of security and functionality is a vision that may be interesting for newer classifications (e.g., targeting different types of systems).


% Please add the following required packages to your document preamble:
% \usepackage{multirow}
\begin{table}[ht]
\scriptsize
\centering
\caption{Classification proposal in\cite{arganaraz_detection_2020}.}
\label{tab:arganaraz}
\begin{tabular}{|l|l|l|}
\hline
\textbf{Level}              & \textbf{Vulnerability}                              & \textbf{Impact}  \\ \hline
{Security}   & Equality on the balance                             & Average \\ \cline{2-3} 
                            & Non-verified external call                          & High    \\ \cline{2-3} 
                            & Use of send instead of transfer                     & Average \\ \cline{2-3} 
                            & Denial of a service because of an external contract & High    \\ \cline{2-3} 
                            & Re-entrancy                                         & High    \\ \cline{2-3} 
                            & Malicious libraries                                 & Low     \\ \cline{2-3} 
                            & Use of tx.origin                                    & Average \\ \cline{2-3} 
                            & Transfer of all the gas                             & High    \\ \hline
{Functional} & Integer division                                    & Low     \\ \cline{2-3} 
                            & Blocked money                                       & Average \\ \cline{2-3} 
                            & Non-verified maths                                  & Low     \\ \cline{2-3} 
                            & Dependence on the timestamp                         & Average \\ \cline{2-3} 
                            & Unsecure inference                                  & Average \\ \hline
\end{tabular}
\end{table}

A smart contract vulnerability classification is presented in \cite{Zhou2022}, based on the previous work in \cite{Atzei2017}. In this classification, summarized in Table \ref{tab:Zhou}, the groups were maintained (i.e., Solidity, EVM, Blockchain), but the vulnerability entries were modified (i.e., some names were removed, like \textit{stack size limit} and \textit{gasless send} and others names were included, like \textit{tx. origin} and \textit{default visibility}). The authors linked the proposed names to an external taxonomy, namely CWE \cite{CWECommunity2009}, which is helpful for understanding each vulnerability, verifying the correctness of the proposed classification, and also for standardization purposes. The proposed classification defines a basic separation of vulnerabilities, mostly distinguishing cases related to the programs from cases related to the platform. Again the number of vulnerabilities listed is quite small (i.e.,13 vulnerabilities), and the work could benefit from a repository open to community contributions.

% Please add the following required packages to your document preamble:
% \usepackage{multirow}
\begin{table}[ht]
\scriptsize
\centering
\caption{Vulnerability classification in \cite{Zhou2022}.}
\label{tab:Zhou}
\begin{tabular}{|c|l|l|l|}
\hline
\multicolumn{1}{|l|}{\textbf{Level}} & \textbf{Vulnerability}                                                        & \textbf{CWE} & \textbf{Real-Word Attack}                                                           \\ \hline
{Solidity}            & Re-entrancy                                                                   & CWE-841      & \begin{tabular}[c]{@{}l@{}}The   DAO\\      Attack\end{tabular}                     \\ \cline{2-4} 
                                     & Arithmetic issues                                                             & CWE-682      & \begin{tabular}[c]{@{}l@{}}PoWHcoin\\      attack\end{tabular}                      \\ \cline{2-4} 
                                     & \begin{tabular}[c]{@{}l@{}}Delegatecall to \\ insecure contracts\end{tabular} & CWE-829      & \begin{tabular}[c]{@{}l@{}}Parity\\      Wallet\\ (Second\\      Hack)\end{tabular} \\ \cline{2-4} 
                                     & Selfdestruct                                                                  & CWE-284      & \begin{tabular}[c]{@{}l@{}}Parity   Library\\      bug\end{tabular}                 \\ \cline{2-4} 
                                     & Tx.origin                                                                     & CWE-477      & -                                                                                   \\ \cline{2-4} 
                                     & Mishandled exception                                                          & CWE-252      & \begin{tabular}[c]{@{}l@{}}King of The \\ Ether attack\end{tabular}                 \\ \cline{2-4} 
                                     & Default visibility                                                            & CWE-710      & \begin{tabular}[c]{@{}l@{}}Parity\\      Wallet\\ (First\\      Hack)\end{tabular}  \\ \cline{2-4} 
                                     & External contract referencing                                                 & CWE-829      & Honey   Pot                                                                         \\ \hline
{EVM}                 & \begin{tabular}[c]{@{}l@{}}Short   address/parameter\\  issues\end{tabular}   & CWE-88       & -                                                                                   \\ \cline{2-4} 
                                     & Freezing Ether                                                                & CWE-17       & -                                                                                   \\ \hline
{Blockchain}          & Transaction order   dependence                                                & CWE-362      & \begin{tabular}[c]{@{}l@{}}Attack   on\\      Bancor\end{tabular}                   \\ \cline{2-4} 
                                     & Generating randomness                                                         & CWE-330      & \begin{tabular}[c]{@{}l@{}}PRNG\\      contract\end{tabular}                        \\ \cline{2-4} 
                                     & Timestamp dependence                                                          & CWE-829      & \begin{tabular}[c]{@{}l@{}}GovernMental\\      attack\end{tabular}                  \\ \hline
\end{tabular}
\end{table}



Table \ref{tab:Amiet} presents a vulnerability classification proposed by \cite{Amiet2021}. The classification is based in two categories: i) core blockchain vulnerabilities (i.e., vulnerabilities related to the blockchain platform); ii) smart contracts vulnerabilities (i.e., vulnerabilities related to the programs deployed in the blockchain). At the blockchain level, examples are provided (e.g., attacks on the consensus mechanism), whereas, at the contract level, pseudo-code is presented, which clarifies the security issues identified. These two broad groups are a basis for applying the classification to other types of systems. There are no further hierarchical levels present in this taxonomy, and we found vulnerability names that are unclear, such as \textit{Improper Blockchain Magic Validation}, which does not really characterize the technical details involving the vulnerability. As with previous cases, the 12 vulnerabilities represent a quite small number of currently known vulnerabilities.

% Please add the following required packages to your document preamble:
% \usepackage{multirow}
\begin{table}[ht]
\scriptsize
\centering
\caption{Classification proposed in \cite{Amiet2021}.}
\label{tab:Amiet}
\begin{tabular}{|l|l|}
\hline
\textbf{Group}                     & \textbf{Vulnerabilities}                \\ \hline
{Core   Blockchain} & Consensus Mechanism   Manipulation      \\ \cline{2-2} 
                                   & Underlying Cryptosystem Vulnerabilities \\ \cline{2-2} 
                                   & Improper Blockchain Magic Validation    \\ \cline{2-2} 
                                   & Improper Transaction Nonce Validation   \\ \cline{2-2} 
                                   & Denial of Service                       \\ \cline{2-2} 
                                   & Public-key and Address Mismatch         \\ \hline
{Smart   Contract}  & Reentrancy                              \\ \cline{2-2} 
                                   & Arithmetic Issues                       \\ \cline{2-2} 
                                   & Unprotected Selfdestruct                \\ \cline{2-2} 
                                   & Visibility Issues                       \\ \cline{2-2} 
                                   & Weak Randomness                         \\ \cline{2-2} 
                                   & Transaction Order Dependence            \\ \hline
\end{tabular}
\end{table}

A classification of 28 vulnerabilities is proposed in \cite{Staderini_2020} and was further evolved to categorize a total of 33 vulnerabilities in \cite{Staderini2022}. Table \ref{tab:Mirko} presents an overview of the authors' classification, identifying the acronym and name of the vulnerability and an associated CWE \cite{CWECommunity2009}.

\input{tables/TableX.tex}

As we can see in Table \ref{tab:Mirko}, the additional characterization by CWE is quite helpful although it is not accompanied by a blockchain-specific classification scheme, such as SWC \cite{swc} which could help in unifying knowledge. The source of information is based on a set of four references in \cite{Staderini_2020} and in five in \cite{Staderini2022}, which do not  directly map with the state of the practice (e.g., tools for vulnerability detection). Still, the process for building the classification is insightful and helpful as a way to solidify our own classification, e.g., by allowing a verification of our own mapping to CWE.


A consolidated taxonomy is presented in \cite{Rameder2022}. The authors were able to collect 54 vulnerabilities reported from different verification tools and grouped them into 10 categories. Table \ref{tab:Rameder} overviews the taxonomy created by the authors.

% Please add the following required packages to your document preamble:
% \usepackage{multirow}


\begin{table}[h!]
\caption{Vulnerability classification in \cite{Rameder2022}}
\label{tab:Rameder}
\centering
\resizebox{0.62\textwidth}{!}{
\begin{tabular}{|l|l|l|}
\hline
\textbf{Group}                                                                                                 & \textbf{Code} & \textbf{Vulnerability}                                                                                                     \\ \hline
{\begin{tabular}[c]{@{}l@{}}Malicious   Environment, \\ Transactions or Input\end{tabular}}     & 1A            & Reentrancy                                                                                                                 \\ \cline{2-3} 
                                                                                                               & 1B            & Call to the   unknown                                                                                                      \\ \cline{2-3} 
                                                                                                               & 1C            & Exact balance   dependency                                                                                                 \\ \cline{2-3} 
                                                                                                               & 1D            & Improper data   validation                                                                                                 \\ \cline{2-3} 
                                                                                                               & 1E            & Vulnerable   DELEGATECALL                                                                                                  \\ \hline
{\begin{tabular}[c]{@{}l@{}}Blockchain/Environment   \\ Dependency\end{tabular}}                & 2A            & Timestamp dependency                                                                                                       \\ \cline{2-3} 
                                                                                                               & 2B            & Transaction-ordering   dependency (TOD)                                                                                    \\ \cline{2-3} 
                                                                                                               & 2C            & Bad random number   generation                                                                                             \\ \cline{2-3} 
                                                                                                               & 2D            & Leakage of   confidential information                                                                                      \\ \cline{2-3} 
                                                                                                               & 2E            & Unpredictable state   (dynamic libraries)                                                                                  \\ \cline{2-3} 
                                                                                                               & 2F            & Blockhash   dependency                                                                                                     \\ \hline
{\begin{tabular}[c]{@{}l@{}}Exception   \& \\ Error Handling Disorders\end{tabular}}            & 3A            & \begin{tabular}[c]{@{}l@{}}Unchecked low level   \\ call/send return values\end{tabular}                                   \\ \cline{2-3} 
                                                                                                               & 3B            & Unexpected throw or   revert                                                                                               \\ \cline{2-3} 
                                                                                                               & 3C            & Mishandled out-of-gas   exception                                                                                          \\ \cline{2-3} 
                                                                                                               & 3D            & Assert, require or   revert violation                                                                                      \\ \hline
{Denial   of Service}                                                                           & 4A            & Frozen Ether                                                                                                               \\ \cline{2-3} 
                                                                                                               & 4B            & Ether lost in   transfer                                                                                                   \\ \cline{2-3} 
                                                                                                               & 4C            & DoS with block gas   limit reached                                                                                         \\ \cline{2-3} 
                                                                                                               & 4D            & DoS by exception   inside loop                                                                                             \\ \cline{2-3} 
                                                                                                               & 4E            & Insufficient gas   griefing                                                                                                \\ \hline
{\begin{tabular}[c]{@{}l@{}}Resource   Consumption\\ \& Gas Issues\end{tabular}}                & 5A            & Gas costly loops                                                                                                           \\ \cline{2-3} 
                                                                                                               & 5B            & Gas costly   pattern                                                                                                       \\ \cline{2-3} 
                                                                                                               & 5C            & \begin{tabular}[c]{@{}l@{}}High gas consumption   of \\ variable data type or declaration\end{tabular}                     \\ \cline{2-3} 
                                                                                                               & 5D            & High gas consumption   function type                                                                                       \\ \cline{2-3} 
                                                                                                               & 5E            & Under-priced   opcodes                                                                                                     \\ \hline
{\begin{tabular}[c]{@{}l@{}}Authentication   \& \\ Access Control Vulnerabilities\end{tabular}} & 6A            & Authorization via   transaction origin                                                                                     \\ \cline{2-3} 
                                                                                                               & 6B            & \begin{tabular}[c]{@{}l@{}}Unauthorized   accessibility due to \\ wrong function or state variable visibility\end{tabular} \\ \cline{2-3} 
                                                                                                               & 6C            & Unprotected   self-destruction                                                                                             \\ \cline{2-3} 
                                                                                                               & 6D            & Unauthorized Ether   withdrawal                                                                                            \\ \cline{2-3} 
                                                                                                               & 6E            & Signature based   vulnerabilities                                                                                          \\ \hline
{Arithmetic   Bugs}                                                                             & 7A            & Integer over- or   underflow                                                                                               \\ \cline{2-3} 
                                                                                                               & 7B            & Integer division                                                                                                           \\ \cline{2-3} 
                                                                                                               & 7C            & Integer bugs or   arithmetic issues                                                                                        \\ \hline
{\begin{tabular}[c]{@{}l@{}}Bad   Coding and \\ Language Specifics\end{tabular}}               & 8A            & Type cast                                                                                                                  \\ \cline{2-3} 
                                                                                                               & 8B            & Coding error                                                                                                               \\ \cline{2-3} 
                                                                                                               & 8C            & Bad coding   pattern                                                                                                       \\ \cline{2-3} 
                                                                                                               & 8D            & Deprecated source   language features                                                                                      \\ \cline{2-3} 
                                                                                                               & 8E            & Write to arbitrary   storage location                                                                                      \\ \cline{2-3} 
                                                                                                               & 8F            & Use of assembly                                                                                                            \\ \cline{2-3} 
                                                                                                               & 8G            & Incorrect inheritance   order                                                                                              \\ \cline{2-3} 
                                                                                                               & 8H            & Variable   shadowing                                                                                                       \\ \cline{2-3} 
                                                                                                               & 8I            & Misleading source   code                                                                                                   \\ \cline{2-3} 
                                                                                                               & 8J            & Missing logic,   logical errors or dead code                                                                               \\ \cline{2-3} 
                                                                                                               & 8K            & Insecure contract   upgrading                                                                                              \\ \cline{2-3} 
                                                                                                               & 8L            & \begin{tabular}[c]{@{}l@{}}Inadequate or   incorrect logging \\ or documentation\end{tabular}                              \\ \hline
{\begin{tabular}[c]{@{}l@{}}Environment   \\ Configuration Issues\end{tabular}}                 & 9A            & Short address                                                                                                              \\ \cline{2-3} 
                                                                                                               & 9B            & Outdated compiler   version                                                                                                \\ \cline{2-3} 
                                                                                                               & 9C            & Floating or no   pragma                                                                                                    \\ \cline{2-3} 
                                                                                                               & 9D            & Token API   violation                                                                                                      \\ \cline{2-3} 
                                                                                                               & 9E            & Ethereum update   incompatibility                                                                                          \\ \cline{2-3} 
                                                                                                               & 9F            & Configuration   error                                                                                                      \\ \hline
{\begin{tabular}[c]{@{}l@{}}Eliminated/Deprecated\\  Vulnerabilities\end{tabular}}              & 10A           & Callstack depth   limit                                                                                                    \\ \cline{2-3} 
                                                                                                               & 10B           & Uninitialized storage   pointer                                                                                            \\ \cline{2-3} 
                                                                                                               & 10C           & Erroneous constructor   name                                                                                               \\ \hline
\end{tabular}
}
\end{table}


 This classification is more fine-grained than the previously discussed ones. However, there are a few issues with some names given to the vulnerabilities. For instance, it is not obvious to what extent \textit{integer bugs or arithmetic issues - 7C} is different from \textit{Integer over- or underflow - 7A} or \textit{Integer division - 7B}, and there are names like \textit{Gas costly loops} and \textit{Gas costly pattern} which seem very similar. Also, names like \textit{configuration error} are quite generic and could lead to a more specific vulnerability like \textit{Environment Configuration Issues}. 
Regarding the structure itself, the taxonomy has a flat organization in which the categories do not really represent aspects at the same abstraction or conceptual level. For instance, \textit{Denial of Service} is generally considered as a type of attack or the effect of an exploited vulnerability, or \textit{Configuration Issues} is quite generic, and it does not characterize the vulnerability sufficiently. We can observe a similar issue between the names given to the categories and to the specific vulnerabilities, e.g., \textit{Bad Coding Group} versus \textit{Coding error Vulnerability}. Although the classification lists 54 vulnerabilities, it would benefit of a way for evolving and including more recent ones (e.g., via an open repository).


\subsection{Community-Based Classification Schemes}

This section discusses taxonomies or classification initiatives maintained by communities. One of the most popular ones is \textit{Smart Contract Weakness Classification} SWC \cite{swc}, a vulnerability classification scheme for smart contracts whose main goals are: i) Provide a straightforward way to classify 'weaknesses' of a smart contract; ii) Identify weaknesses that lead to vulnerabilities; iii) Define a common language to describe weaknesses in the architecture, design, and coding of smart contracts; and finally, iv) Being a way to improve the effectiveness of smart contract security analysis tools \cite{epi1470}.

In SWC, each software defect has an external relationship with another taxonomy (i.e., CWE \cite{CWECommunity2009}), and there are examples (i.e., faulty and non-faulty code) to illustrate the vulnerability and a correction. SWC is a flat list structure, where the distinction between vulnerabilities and other types of defects is many times unclear. Also, it is worthwhile mentioning that there are cases where it is difficult to distinguish whether the problem is related to the blockchain platform or to the smart contract itself (e.g., \textit{Weak Sources of Randomness from Chain Attributes}, \textit{Unencrypted Private Data On-Chain}). A positive aspect is that SWC is associated withan open repository, although, at the time of writing, the last update was made in 2018. Considering the changes and new knowledge about smart contract vulnerabilities, this means that practitioners' involvement is now impaired. For instance, the classification presented in \cite{Rameder2022} identifies several new defects that are not present in SWC.

The NCC Group initiated the Decentralized Application Security Project (DASP) in 2018, which includes a vulnerability classification scheme for smart contracts. The idea of the classification is to present the top 10 threats to smart contract security, for which a single iteration was carried precisely in 2018. Thus, it does not really reflect the whole landscape of vulnerabilities. DASP provides a short description for each class of vulnerabilities, which is accompanied by pseudo-code as a way of explaining the defects. The classification emphasizes the impact the vulnerability had in real-world scenarios (e.g., reentrancy loss estimated at 3.5M ETH ~50M USD at the time). References to real-world attacks are provided (i.e., reports, magazines, etc.), which present a historical view of vulnerability exploitation. The nomenclature is clear, although some parts of the structure are questionable. For instance, the \textit{Denial of Service} category in DASP refers to \textit{gas limit reached}, \textit{unexpected throw}, \textit{unexpected kill}, and \textit{access control breached}. The description is sometimes so short that may become ambiguous (e.g., \textit{access control breached} may refer to a vulnerability that would simply fit in \textit{Access Control}, which is another DASP category). In \cite{durieux_empirical_2020}, the authors used DASP but concluded that the categories were not sufficient to cover the vulnerabilities found.

SIGP \cite{SIGP2018} is a vulnerability classification scheme for smart contracts written in Solidity that forms the basis of 
of the work in \cite{antonopoulos_mastering_2018}. The classification considers three main elements: vulnerability, preventive technique, and a real-world example. The first element conceptually describes the reported vulnerability. It also presents the vulnerable code and explains how the attack is performed. The second element presents a solution for the problem, and the last element discusses a real-world attack in which the vulnerability was exploited. The clarity of the names used for the vulnerabilities could be improved (e.g., \textit{entropy illusion} and \textit{constructors with care} are ambiguous). There is an open repository associated, but not receiving any updated, at the time of writing. As in previous cases, there are only 16 vulnerabilities listed, which is currently far from the state of the practice.

The SMARTDEC classification \cite{SmartDec2018} originated from the experience gathered from the creation of Smartcheck \cite{tikhomirov_smartcheck_2018}. The vulnerabilities are organized into three main categories: Blockchain (i.e., vulnerabilities from  the blockchain system), Language (i.e., programming language defects), and Model (i.e., vulnerabilities caused by mistakes in the model). Each group has several entries (up to a total of 11), where each entry corresponds to a set of related vulnerabilities. The entry names are unique, although they are also quite generic, and therefore less descriptive (e.g., \textit{Trust}). The authors provide a mapping between their taxonomy and other classifications, namely DASP \cite{nccgroup_decentralized_2021}, SWC \cite{swc}, and SIGP \cite{SIGP2018}. As an example, the \textit{Arithmetic} category is related to \textit{Over/underflow} in SWC-101, DASP-3, and SP-2 and to \textit{Precision issues} in SP-15. The repository is open to contributions, although, at the time of writing, there has been no update since 2018.

\subsection{Classification Schemes used in Vulnerability Detection Research}

Research in smart contract vulnerability detection for smart contracts is generally accompanied by custom vulnerability classification schemes \cite{luu_making_2016,kalra_zeus_2018,wang_vultron_2019,ghaleb_how_2020,SMARTIAN_2021,SAILFISH_2022}. This is primarily due to lacking an appropriate and up-to-date classification standard or taxonomy. As a result, biased and limited classifications emerged, which are coupled to the context in which they were created. The next paragraphs describe the classification schemes of selected research, namely of three of the most cited vulnerability detection research works (at the time of writing and according to Google Scholar). In all of these cases, the heterogeneity is clear, as well as the divergence with other classification schemes, such as the ones previously presented in this section.


A symbolic execution tool named Oyente is proposed in \cite{luu_making_2016} with the goal of allowing practitioners to detect security vulnerabilities. Within the tool proposal the authors identify a small set of security vulnerabilities, as illustrated in Table \ref{tab:Oyente}. 

\begin{table}[h]
\centering
\caption{List of vulnerabilities in \cite{luu_making_2016}}
\label{tab:Oyente}
\begin{tabular}{|l|}
\hline
\textbf{Vulnerability name}                         \\ \hline
Mishandled   Exceptions               \\ \hline
Reentrancy                            \\ \hline
Timestamp   dependence                \\ \hline
Transaction-Ordering   Dependence     \\ \hline
\end{tabular}
\end{table}

Although the work in Oyente targets a specific set of vulnerabilities, the absence of a standard way for categorizing and naming the vulnerabilities impairs the assessment and comparison of results with other tools or approaches.


Securify \cite{tsankov_securify_2018} is a vulnerability detection tool based on symbolic execution methods, which, at the time of writing, is able to detect 37 security defects \cite{Tsankov2018}, which the tool groups by severity, as we can see in Table \ref{tab:Securify}. 


% Please add the following required packages to your document preamble:
% \usepackage{multirow}
\begin{table}[h]
\centering
\caption{Vulnerability classification in \cite{tsankov_securify_2018} and extended in \cite{Tsankov2018}}
\label{tab:Securify}
\begin{tabular}{|l|l|}
\hline
\textbf{Severity}         & \textbf{Vulnerability}          \\ \hline
\multirow{4}{*}{Critical} & TODAmount                       \\ \cline{2-2} 
                          & TODReceiver                     \\ \cline{2-2} 
                          & TODTransfer                     \\ \cline{2-2} 
                          & UnrestrictedWrite               \\ \hline
\multirow{7}{*}{High}     & RightToLeftOverride             \\ \cline{2-2} 
                          & ShadowedStateVariable           \\ \cline{2-2} 
                          & UnrestrictedSelfdestruct        \\ \cline{2-2} 
                          & UninitializedStateVariable      \\ \cline{2-2} 
                          & UninitializedStorage            \\ \cline{2-2} 
                          & UnrestrictedDelegateCall        \\ \cline{2-2} 
                          & DAO                             \\ \hline
\multirow{10}{*}{Medium}  & ERC20Interface                  \\ \cline{2-2} 
                          & ERC721Interface                 \\ \cline{2-2} 
                          & IncorrectEquality               \\ \cline{2-2} 
                          & LockedEther                     \\ \cline{2-2} 
                          & ReentrancyNoETH                 \\ \cline{2-2} 
                          & TxOrigin                        \\ \cline{2-2} 
                          & UnhandledException              \\ \cline{2-2} 
                          & UnrestrictedEtherFlow           \\ \cline{2-2} 
                          & UninitializedLocal              \\ \cline{2-2} 
                          & UnusedReturn                    \\ \hline
\multirow{6}{*}{Low}      & ShadowedBuiltin                 \\ \cline{2-2} 
                          & ShadowedLocalVariable           \\ \cline{2-2} 
                          & CallToDefaultConstructor?       \\ \cline{2-2} 
                          & CallInLoop                      \\ \cline{2-2} 
                          & ReentrancyBenign                \\ \cline{2-2} 
                          & Timestamp                       \\ \hline
\multirow{10}{*}{Info}    & AssemblyUsage                   \\ \cline{2-2} 
                          & ERC20Indexed                    \\ \cline{2-2} 
                          & LowLevelCalls                   \\ \cline{2-2} 
                          & NamingConvention                \\ \cline{2-2} 
                          & SolcVersion                     \\ \cline{2-2} 
                          & UnusedStateVariable             \\ \cline{2-2} 
                          & TooManyDigits                   \\ \cline{2-2} 
                          & ConstableStates                 \\ \cline{2-2} 
                          & ExternalFunctions               \\ \cline{2-2} 
                          & StateVariablesDefaultVisibility \\ \hline
\end{tabular}
\end{table}


Again, as with the previous tool, the groups and the names or vulnerability definition are non-standard, although there is an effort in to classify most of them according to SWC \cite{swc}.


Zeus is a tool based on abstract interpretation and symbolic execution \cite{kalra_zeus_2018}. Table \ref{tab:Zeus} shows the vulnerability classification performed by the authors and targeted by the tools.

% !TEX TS-program = pdflatex
% !TEX root = ../ArsClassica.tex

%************************************************
\chapter{Zeus}
\label{chp:zeus}
%************************************************
 
\lstset{numbers=left,
    numberstyle=\scriptsize,
    stepnumber=1,
    numbersep=8pt
}    

This chapter presents \textit{zeus} which is the main contribution introduced in the paper titled \textit{zeus: A Python implementation of Ensemble Slice Sampling for efficient Bayesian parameter inference} that was published in the journal \textit{Monthly Notices of the Royal Astronomical Society} in 2021~\parencite{karamanis2021zeus}. The content of the chapter is almost identical to that included in the aforementioned publication with the exception of minor text and figure formatting differences.

\begin{center}
\rule{0.5\textwidth}{.4pt}
\end{center}
\vspace{8pt}

We introduce \texttt{zeus}, a well-tested \texttt{Python} implementation of the Ensemble Slice Sampling (ESS) method for Bayesian parameter inference. ESS is a novel Markov chain Monte Carlo (MCMC) algorithm specifically designed to tackle the computational challenges posed by modern astronomical and cosmological analyses. In particular, the method requires only minimal hand--tuning of $1-2$ hyper-parameters that are often trivial to set; its performance is insensitive to linear correlations and it can scale up to 1000s of CPUs without any extra effort. Furthermore, its locally adaptive nature allows to sample efficiently even when strong non-linear correlations are present. Lastly, the method achieves a high performance even in strongly multimodal distributions in high dimensions. Compared to \texttt{emcee}, a popular MCMC sampler, \texttt{zeus} performs $9$ and $29$ times better in a cosmological and an exoplanet application respectively.

\section{Introduction}
\label{sec_zeus:intro}

Over the past few decades the volume of astronomical and cosmological data has increased substantially. In response to that, a variety of astrophysical models have been developed to explain the plethora of observations. Markov chain Monte Carlo (MCMC) has been established as the standard procedure of inferring the model parameters subject to the available data in a Bayesian framework. Within the Bayesian context, the object that quantifies the probability distribution of the model parameters $\theta$ given the data $D$ and model $\mathcal{M}$ is the posterior distribution $\mathcal{P}(\theta)\equiv P(\theta | D, \mathcal{M})$ which is defined using Bayes's theorem:
\begin{equation}
    \label{eq_zeus:bayes}
    \mathcal{P}(\theta) = \frac{\mathcal{L}(\theta) \pi (\theta)}{\mathcal{Z}},
\end{equation}
where $\mathcal{L}(\theta) \equiv P(D|\theta, \mathcal{M})$ is the likelihood function, $\pi (\theta) \equiv P (\theta | \mathcal{M})$ is the prior distribution of the model parameters $\theta$, and $\mathcal{Z}\equiv P(D|\mathcal{M})$ is the, so called, Bayesian model evidence or marginal likelihood and in this context can be treated as a simple normalisation constant.

MCMC does not in general require knowing the value of the model evidence and it only depends on the ability to evaluate the unnormalised posterior distribution for arbitrary values of $\theta$. MCMC methods can then be used to generate (Markov) chains of samples from the posterior distribution. Those samples can be used to calculate integrals (e.g. parameter uncertainties, marginal distributions etc.) that are paramount for modern astronomical and cosmological analyses.

The most commonly used MCMC methods are variants of the Metropolis-Hastings (MH) algorithm \parencite{metropolis1953equation, hastings1970}. MH consists of two steps. First, given the last sample in the chain, a new sample is proposed and then the Metropolis criterion determines whether or not that new sample should be accepted and thus added to the chain. The resulting chain is Markovian in the sense that each sample is proposed based only on the previous sample. The purpose of the Metropolis acceptance criterion is to bias the chain so that the time spent in a region of the parameter space would be proportional to the posterior probability in that region. In other words, the stationary distribution of the Markov chain is the target distribution i.e. the posterior distribution. For a detailed introduction to MCMC methods we direct the reader to \textcite{mackay2003information} and for an intuitive introduction to Bayesian inference to \textcite{jaynes2003probability}.

Arguably, the most difficult part of the MH algorithm is the proposal step. There are many ways of choosing a new sample and the efficiency of the method depends on this choice. By far the simplest one is the use of a normal (Gaussian) distribution, centred around the previous sample to generate the new proposed sample. The resulting method is often called Random Walk Metropolis algorithm and its performance is highly sensitive to the $n(n+1)/2$ elements that form its covariance matrix. Those elements generally need to be chosen \textit{a priori} or be adaptively tuned. More efficient methods utilise the gradient of the target distribution \parencite{2017arXiv170102434B} or an ensemble of parallel and communicating chains \parencite{gilks1994adaptive,ter2006markov, ter2008differential, goodman2010ensemble}.

Out of the methods mentioned in the previous paragraph we will focus our attention on the last one, the ensemble or population MCMC variety. The reason is simple: the Random Walk Metropolis algorithm requires a great amount of tuning (or \textit{a priori} knowledge) for it to perform efficiently and even then there is no guarantee that the proposal covariance matrix is optimal for the whole parameter space. On the other hand, gradient based methods, although very powerful, are in general unsuitable for astronomical applications in which the models that are used are almost always not differentiable.

One benefit of ensemble MCMC over its alternatives is that the ensemble of parallel chains (also known as walkers) collectively sample the posterior, thus information about their distribution can be shared and used to make better educated proposals. Other advantages include the lack of hand-tuning of hyper-parameters and their capacity for parallel implementation. For the aforementioned reasons, ensemble MCMC methods have dominated astronomical analyses. The most common ones are affine--invariant ensemble sampling (AIES) \parencite{goodman2010ensemble} and differential evolution MCMC (DEMC) \parencite{ter2006markov, ter2008differential}, both implemented in the popular \texttt{Python} package \texttt{emcee} \parencite{foreman2013emcee,foreman2019emcee}.

In this paper we introduce \texttt{zeus}, a stable and well-tested \texttt{Python} implementation of Ensemble Slice Sampling (ESS) \parencite{karamanis2021ensemble}. ESS is a method based on the ensemble MCMC paradigm, with the crucial difference being that its proposals are performed via Slice Sampling updates \parencite{neal2003slice} instead of Metropolis-Hastings ones. As we will thoroughly demonstrate in Section \ref{sec_zeus:tests}, this subtle difference leads to substantial improvements in terms of sampling efficiency and robustness. \texttt{zeus} is a user-friendly tool that does not require any hand-tuning or preliminary runs and can scale up to 1000s of CPUs without any extra effort from the user.

\texttt{zeus} has been used in various astronomical and cosmological analyses, including cosmological tests of gravity \parencite{Tamosiunas2020}, relativistic effects and primordial non-Gaussianity \parencite{Wang2020}, 21cm intensity mapping \parencite{Umeh2021}, and has been implemented as part of the \texttt{CosmoSIS} package \parencite{Zuntz2015}.

\texttt{zeus} is open source software that is publicly available at \url{https://github.com/minaskar/zeus} under the \texttt{GPL-3 Licence}. Detailed documentation and examples on how to get started are available at \url{https://zeus-mcmc.readthedocs.io}.

\section{Ensemble Slice Sampling}
\label{sec_zeus:ess}

\texttt{zeus} is a \texttt{Python} implementation of the Ensemble Slice Sampling (ESS) method presented in \textcite{karamanis2021ensemble}. 
Here we will provide a high-level description of the method and will refer to the accompanying paper for more details about the underlying algorithmic structure and mathematics.

ESS combines the ensemble MCMC paradigm with slice sampling. Since the use of slice sampling in astronomical parameter inference is rare we will start by explaining its function and how it differs from Metropolis updates. Then we will move on to discuss how it can be efficiently combined with ensemble MCMC.

\subsection{Slice sampling}

Slice sampling is based on the idea that sampling from a distribution with density $P(x)$ is equivalent to uniform sampling from the area under the plot of $f(x)\propto P(x)$. To this end, we introduce an auxiliary variable $y$, called height, such that the joint distribution $P(x,y)$ is uniform over the region $U=\lbrace (x,y): 0<y<f(x) \rbrace$. To sample from the marginal distribution $P(x)$, we first sample from $P(x,y)$ and then we marginalise by dropping the $y$ value of each sample.

In order to generate samples from $P(x,y)$ we utilise the following scheme \parencite{neal2003slice}:
\begin{enumerate}
    \item Given the current state $x_{0}$, draw $y_{0}$ uniformly from $(0,f(x_{0}))$.
    \item Find an interval $I=(L,R)$ that contains all, or at least part, of the slice $s=\lbrace x: y_{0}<f(x)\rbrace$.
    \item Draw the new sample $x_{1}$ uniformly from $I\cap S$.
\end{enumerate}

\begin{figure}[H]
    \centering
	\centerline{\includegraphics[scale=0.9]{Graphics/Zeus/slice.pdf}}
    \caption{Illustration of the univariate slice sampling update. Given the current sample $x_{0}$, a value $y_{0}$ is uniformly sampled along the vertical slice $(0,f(x_{0}))$ (dashed line) thus defining the initial point (blue). An interval $(L,R)$ is uniformly positioned horizontally around $(x_{0},y_{0})$ and it is expanded in steps of size $R-L$ until both its ends are outside the slice. The new sample is generated by  repeatedly sampling (uniformly) from the interval $(L',R')$ until a sample (green star) is found inside the slice. Samples outside of the slice (red star) are rejected and they are instead used to shrink $(L',R')$.}
    \label{fig_zeus:slice}
\end{figure}

To construct the interval $I$ (step ii), \textcite{neal2003slice} introduced the stepping-out procedure that works by randomly positioning an interval of length $\mu$ around the sample $x_{0}$ (i.e. blue dot in Figure \ref{fig_zeus:slice}) and then expanding it in steps of size $\mu$ until both its ends (i.e. $L'$ and $R'$) are outside the slice. To obtain $x_{1}$ (i.e. green star in Figure \ref{fig_zeus:slice}) we then use the shrinking procedure in which candidates are sampled uniformly from $I$ until a point inside the slice $S$ is found. Samples outside of the slice are used to shrink the interval $I$. The two procedures are shown in Figure \ref{fig_zeus:slice}.

The length scale $\mu$ is the only free hyperparameter of slice sampling and although its choice can reduce or increase the computational cost of the method it generally does not affect its mixing properties (e.g. convergence rate, autocorrelation time, etc.). \texttt{zeus} utilises a stochastic optimization algorithm similar to \textcite{tibbits2014automated} and based on the \textcite{robbins1951stochastic} optimisation scheme in order to tune $\mu$ to its optimal value (see Section 3.1 of \textcite{karamanis2021ensemble} for more details).

It is important to note here that for multimodal target distributions there is no guarantee that the approximate slice would cross any of the other modes. In particular, if the initial estimate of the length scale $\mu$ is low then the probability of missing the other peaks, assuming that they are located far away, is also low. As we will show in Section \ref{sec_zeus:tests}, unlike simple slice sampling, ESS and thus \texttt{zeus} does not suffer from this effect.

\subsection{Walkers, moves and parallelism}

\begin{figure}
    \centering
	\centerline{\includegraphics[scale=0.9]{Graphics/Zeus/move.pdf}}
    \caption{The figure illustrates the differential move in the context of Ensemble Slice Sampling. The walker $X_{k}$ to be updated is shown in red. Two walkers, $X_{l}$ and $X_{m}$, (blue) are uniformly selected from the complementary ensemble (grey). The approximate slice (dotted line) is constructed parallel to the two walkers $X_{l}$ and $X_{m}$ using the stepping-out procedure. The new position $Y$ (green) of $X_{k}$ is sampled using the shrinking procedure along the approximate slice.}
    \label{fig_zeus:move}
\end{figure}

The slice sampling update described in the previous paragraphs is a univariate update scheme. For it to be used to sample from multivariate target distributions it needs to be generalised accordingly. Perhaps the simplest such generalisation in a multivariate setting is the use of slice sampling to sample along each coordinate axis in turn (i.e. component-wise slice sampling) or to sample along randomly selected directions in parameter space \parencite{mackay2003information}. Although valid, both of these approaches are unsuitable in cases of correlated parameters in which the proper choice of direction can substantially accelerate mixing.

To address this issue, \textcite{tibbits2014automated} proposed to orthogonalise the parameter space using the sample covariance, thus getting rid of linear correlations between parameters. We will instead follow a different, perhaps more flexible, approach to construct an efficient slice sampler. Our aim is to utilise an ensemble of parallel chains/walkers that can exchange information about the covariance structure of the target distribution and thus by-pass the difficulties posed by correlations.

As hinted in the introduction, the ensemble of walkers collectively sample the target distribution and thus their positions encode information about the correlations between the parameters. One way to take advantage of this information is to use it to construct direction vectors along which slice sampling can take place. Many moves that generate direction vectors from the complementary ensemble are possible. \texttt{zeus} offers a collection of them, including some that utilise clustering algorithms and density estimation methods. As we will show in Section \ref{sec_zeus:tests}, such moves can help accelerate sampling in difficult cases such as strongly multimodal distributions. Any distribution of the complementary ensemble can be used as a valid proposal to generate such direction vectors and \texttt{zeus} offers a highly flexible interface for the user to define such a move or choose one (or a mixture) from the ones that are already implemented and tested. Here is a list of the currently implemented moves in \texttt{zeus}:

\begin{itemize}
    \item \textbf{Differential move:} This is the default move used by \texttt{zeus} and shown in Figure \ref{fig_zeus:move}. Using the differential move, Ensemble Slice Sampling updates the position of each walker in the ensemble by slice sampling along a direction defined by the difference between two uniformly selected walkers from the rest of the ensemble (i.e. the complementary ensemble).
    \item \textbf{Gaussian move:} The Gaussian move samples the direction vectors along which slice sampling is performed from a normal distribution that shares the same covariance structure as the complementary ensemble. This approach is very efficient in cases in which the target distribution is close to normal.
    \item \textbf{Global move:} The Global move utilises a Dirichlet Process Gaussian Mixture to fit the complementary ensemble and proposes directions along different peaks of the target distribution in cases of strong multi-modality.
    \item \textbf{KDE move:} The KDE move samples the direction vectors from a Gaussian Kernel Density Estimate of the complementary ensemble. This can be useful in cases of highly non-Gaussian target distributions.
    \item \textbf{Random move:} The Random move performs slice sampling along isotropic directions. This is equivalent of standard multivariate slice sampling and it is mostly offered for testing purposes as it cannot handle correlations efficiently.
\end{itemize}
For more information on how those moves work as well as a comparison of the Differential, Gaussian and Global moves we direct the interested reader to \textcite{karamanis2021ensemble}. Unless stated otherwise the Differential move will be used for the following examples.

To parallelise this process and capitalise on the availability of multiple CPUs we randomly split the ensemble into two sets of walkers (i.e. active and passive sets)~\parencite{foreman2013emcee} and choose to update the positions of the active walkers along direction vectors defined by passive walkers. Then the passive becomes active and \textit{vice versa} and the process is repeated. The ensemble splitting technique is required in order to parallelise the algorithm without violating detailed balance. Parallelisation is achieved in practice using either \texttt{multiprocessing} or \texttt{MPI} using the implemented \texttt{ChainManager} utility that can distribute both multiple ensembles and multiple chains in parallel computing environments at the same time. Heuristics to determine the number of required walkers per application are discussed in Section \ref{sec_zeus:discussion}.

\section{Empirical Evaluation}
\label{sec_zeus:tests}

For the empirical evaluation of \texttt{zeus} we use five toy examples that manifest significant aspects of real astronomical applications\footnote{For additional demonstrations on similarly common structures (e.g. the funnel) we direct the reader to the accompanying paper \parencite{karamanis2021ensemble}.} (i.e. linear and non-linear correlations, multimodality, heavy tails, hard boundaries) and two real-world astronomical examples characteristic of modern astronomical analyses.

\subsection{Toy examples}
In order to understand the behaviour of \texttt{zeus} in various sampling scenarios, it is important to study its performance in different toy examples that demonstrate different characteristics of common target distributions that arise in astronomical applications. For that reason, we chose five such toy examples. The first one is a normal (Gaussian) distribution which by definition is characterised only by the linear correlation between its parameters. The second toy problem is the ring distribution, a characteristic example of strong non-linear correlations. The third example is a Gaussian mixture with two components. While the purpose of the first two examples is to study the behaviour of the algorithm in the presence of linear and non-linear correlations respectively, the goal of the third example is to demonstrate the ability of \texttt{zeus} to sample efficiently from multimodal target distributions. The fourth toy example investigates the effect that heavy tails have on the sampling efficiency and the fifth shows the effects that hard boundaries have on sampling.

We compare \texttt{zeus} with two popular alternatives offered by \texttt{emcee}, namely affine--invariant ensemble sampling with the \textit{stretch} move (\texttt{emcee}/AIES) and the differential evolution move (\texttt{emcee}/DEMC). The main goal of this analysis is to justify our choice of slice sampling as the basis of \texttt{zeus} instead of Metropolis updates through the use of simple yet instructive toy examples.

For all three toy examples discussed below we adopt the same analysis procedure, where we initialise the walkers by sampling from a normal distribution $\mathcal{N}(\mathbf{0}, \mathbf{I})$ where $\mathbf{I}$ is the identity covariance matrix and we discarded $10^4$ iterations as burn--in. 

The main metric that we use to investigate the behaviour of the samplers in those toy examples and to compare their performance is the distribution of steps performed by the walkers. As a step, we define the distance spanned in parameter space by a single walker in a single iteration. This is a fundamental measure of the efficiency of an MCMC method and it is directly related to the expected squared jump distance (ESJD)~\parencite{pasarica2010adaptively} given by:
\begin{equation}
    \label{eq_zeus:esjd}
    \text{ESJD} = \mathbf{E}\left[|\theta_{t+1}-\theta_{t}|^{2}\right]= 2 \,( 1 - \rho_{1}) \cdot\text{Var}_{(\pi)}(\theta_{t})\,,
\end{equation}
where $\theta_t$ are the chain samples, $\rho_{1}$ is the first-order autocorrelation, and $\text{Var}_{(\pi)}(\theta_{t})$ is a function of the stationary distribution only. Assuming that the higher-order autocorrelations $\rho_{2}, \rho_{3}, \dots$ are monotonically decreasing with respect to $\rho_{1}$, then maximising the ESJD leads to minimisation of the autocorrelation between chain elements and thus maximisation of the sampling efficiency. In other words, the further away (i.e. the greater the ESJD) the walkers jump per iteration,  the higher the sampling efficiency of the method. A benefit of using ESJD instead of the autocorrelation time as a metric is that the former, as an expectation value, is more accurate when computed using short chains.

In order to account for the different computational costs (i.e. different number of model evaluations per iteration) between \texttt{zeus} and \texttt{emcee} we thinned the chains of the latter method according to the average number of model evaluations of \texttt{zeus}. This allowed us to compare the distribution of steps of the three samplers as shown in Figures \ref{fig_zeus:gaussian_sep}, \ref{fig_zeus:ring_sep}, \ref{fig_zeus:bimodal_sep}, \ref{fig_zeus:student_sep}, and \ref{fig_zeus:truncated_sep} for the five toy examples respectively.

\subsubsection{The correlated normal distribution}

%\begin{landscape}
\begin{figure*}[!htb]
    \centering
	\centerline{\includegraphics[scale=0.45]{Graphics/Zeus/gaussian.pdf}}
    \caption{The figure shows numerical results (i.e. walker trajectories/chains for the first parameter) demonstrating the performance of the three ensemble MCMC methods in the case of a normal (Gaussian) target distribution in $10, 25$ and $50$ dimensions respectively. The last column illustrates the 1-D marginal posterior corresponding to the first parameter $x_{1}$ estimated directly from the samples for the 50-dimensional case.}
    \label{fig_zeus:gaussian}
\end{figure*}
%\end{landscape}
Starting with the normal target distribution it is important to note here that all three of the methods used in the comparison are affine--invariant\footnote{Differential evolution Metropolis is only approximately affine--invariant due to the jitter that it is often added to its proposal. This however has a negligible effect.}, meaning that their performance is immune to any linear correlations between the parameters. Since the normal distribution incorporates, by construction, only linear correlations (i.e. the 2D marginal distribution contours look like ellipses), it is the perfect testing ground to assess the effect that high dimensionality has on the three methods independently of other complications. For our example, we used a zero-mean normal distribution with a covariance matrix in which the diagonal elements are set to $1$ and the off-diagonal ones are equal to $0.95$. We then proceed by sampling the aforementioned distribution in $10$, $25$ and $50$ dimensions. Based on Figure \ref{fig_zeus:gaussian} one can see that the walkers of \texttt{emcee}/AIES dissolve into an inefficient random walk characterised by low step size and high autocorrelation time as the number of parameters increases. \texttt{zeus} and \texttt{emcee}/DEMC are not so severely affected by the high number of parameters exhibiting a substantially lower autocorrelation.

\begin{figure}[ht!]
    \centering
	\centerline{\includegraphics[scale=0.65]{Graphics/Zeus/gaussian_sep.pdf}}
    \caption{This figure shows the distribution of step sizes of walkers for the three different samplers in the case of a normal (Gaussian) target distribution in $D=50$. It is important to note here that both \texttt{emcee} algorithms exhibit a peak at zero separation; \texttt{zeus} on the other hand does not due to its non-rejection nature.}
    \label{fig_zeus:gaussian_sep}
\end{figure}
Let us now try to explain this difference in behaviour by looking into the distribution of the steps of the walkers in Figure \ref{fig_zeus:gaussian_sep}. One thing to notice here is that the distribution of the steps of \texttt{zeus}'s walkers extends significantly further away than those of \texttt{emcee}/AIES and \texttt{emcee}/DEMC. This should come as no surprise since the construction of the approximate slice allows for larger steps than Metropolis updates as shown in Table \ref{tab_zeus:table1}. This is because when a proposal is rejected in slice sampling the approximate slice shrinks and another sample is proposed instead. In this way, \texttt{zeus}'s walkers always move and the chance of staying fixed is zero -- unlike MH-based updates in which frequent rejection of samples is a necessity. This aforementioned procedure leads to greater steps in parameter space. The difference between \texttt{emcee}/AIES and \texttt{emcee}/DEMC is attributed to the fact that DEMC uses a proposal scale\footnote{The proposal scale $\gamma$ is similar to $\mu$ used in ESS in the sense that its value determines the length scale of the proposed jumps in parameter space. A high value would lead to large steps that are often rejected and a low value would lead to small steps that are often accepted but do not carry the walkers far. For such methods, a balance must be found.} $\gamma = 2.38 / \sqrt{D}$ that guarantees a constant acceptance rate accounting for the number of dimensions $D$. This proposal scale is however optimal only in the case of a normal target distribution such as the one that we are studying here and there is no guarantee that it would return acceptable results in non-Gaussian distributions. For the case of \texttt{emcee}/AIES, the relevant proposal scale $\gamma$ is allowed to vary in the range between $1/\alpha$ and $\alpha$ where $\alpha = 2$ is often taken as the typical value. It is clear that in the latter case $\gamma$ does not possess the desired scaling $\gamma\propto 1/\sqrt{D}$ and thus, although the method generates proposals in the right overall direction, most of the samples do not reside in the typical set \parencite{speagle2019conceptual}. In other words, the lack of proper scaling of the proposal scale with the number of dimensions leads to \texttt{emcee}/AIES ``overshooting'' the typical set where most of the posterior mass is located.

%\begin{landscape}
\begin{figure}[!ht]
    \centering
	\centerline{\includegraphics[scale=0.45]{Graphics/Zeus/act.pdf}}
    \caption{The figure shows numerical estimates of the integrated autocorrelation time (number of steps along a chain required to obtain an independent sample; left panel), the effective sample size (percentage of effectively independent samples in a chain; middle panel), and the sampling efficiency (i.e. effective sample size per model evaluation; right panel) for a normal target distribution and varying number of dimensions. The number of walkers was set to $4\times D$ for \texttt{zeus} and $16\times D$ for \texttt{emcee}, this was the optimal choice (i.e. the one maximising the efficiency for the given dimensionality) for each sampler. \texttt{zeus} and \texttt{emcee/DEMC} exhibit linear scaling of the autocorrelation time with the number of dimensions whereas \texttt{emcee/AIES} scales exponentially.}
    \label{fig_zeus:act}
\end{figure}
%\end{landscape}
We can also draw some useful insights about the sampling efficiency of those samplers and their scaling with the number of dimensions by estimating the integrated autocorrelation time of the chains. Given the autocorrelation time, we can also estimate the effective sample size as the percentage of effectively independent samples in a chain. By dividing the effective sample size by the computational cost of each method we can then estimate the sampling efficiency. The results of such a comparison are shown in Figure \ref{fig_zeus:act}. We immediately notice here that the autocorrelation times of \texttt{zeus} and \texttt{emcee}/DEMC scale linearly with the number of dimensions, whereas the autocorrelation time of \texttt{emcee}/AIES scales exponentially. The computational cost of \texttt{zeus} per iteration per walker, although somewhat higher than that of \texttt{emcee}, does not vary with the number of dimensions. This means that in high dimensions, \texttt{zeus} dominates over \texttt{emcee}/AIES in terms of sampling efficiency.


\begin{table}[ht!]
    \centering
    \caption{The table shows a comparison of \texttt{emcee}/AIES, \texttt{emcee}\allowbreak/DEMC and \texttt{zeus} in terms of the expected squared jump distance (ESJD; higher is better) for the five toy examples i.e. $50$-$D$ normal distribution, $25$-$D$ ring distribution, $25$-$D$ Gaussian mixture, $25$-$D$ Student's $t$-distribution, and $25$-$D$ truncated normal distribution.}
    \def\arraystretch{1.1}
    \begin{tabular}{lcccc}
        \toprule[0.75pt]
         & \texttt{emcee}/AIES   & \texttt{emcee}/DEMC   & \textbf{zeus}  \\
        \midrule[0.5pt]
        Normal    &    $0.5288$    &    $1.1162$    &   $\mathbf{2.1354}$   \\
        Ring   &    $0.0043$   &    $0.0006$    & $\mathbf{0.1257}$  \\
        Mixture   &    $0.0037$    &    $0.0056$    &  $\mathbf{0.1015}$ \\
        Student   &    $12.9124$    &    $2.4137$    &  $\mathbf{23.5720}$ \\
        Truncated   &    $0.0940$    &    $0.3501$    &  $\mathbf{0.5882}$ \\
        \bottomrule[0.75pt]
        \end{tabular}
    \label{tab_zeus:table1}
\end{table}

The above discussion allows us to clearly state a crucial distinction between the three methods, which is their response to the curse of dimensionality. As the number of dimensions increases, the probability mass of a distribution is concentrated into a thin shell within the tails of the distribution (i.e. the typical set). To account for this and maintain its efficiency, a sampling method has to adjust its proposal scale -- otherwise, the proposals will not be located in the typical set and thus they will not be accepted. The three methods that we mentioned so far deal with this in different ways. \texttt{emcee}/AIES's proposal scale is not adjusted and thus its proposals become increasingly inefficient in high dimensions. \texttt{emcee}/DEMC's proposal scale is adjusted based on the theoretical expectation for the case of the normal target distribution. Although both \texttt{emcee} methods perform well in this example, their sub-optimal scaling will degrade their performance in non-Gaussian target distributions as we will demonstrate in the next toy example. Finally, \texttt{zeus}'s proposal scale is continuously adapted, as the slice expands and contracts in every iteration, thus guaranteeing optimal scaling. \textcite{huijser2017properties} found that the suboptimal scaling of \texttt{emcee}/AIES with the number of dimensions can introduce biases into the expectation values derived from the chains in high dimensions that are hard to diagnose. The locally adaptive nature of \texttt{zeus} allows it to avoid this problem by adjusting its proposals accordingly.

\begin{figure}[ht!]
    \centering
	\centerline{\includegraphics[scale=0.45]{Graphics/Zeus/convergence.pdf}}
    \caption{The figure shows the computational cost until convergence is reached in terms of the number of model evaluations for the different ensemble samplers for a highly correlated 25--dimensional normal distribution. The left panel shows the computational cost for a single walker. From this we can see that the cost for a single walker decreases as we increase the number of walkers until it reaches a plateau. The high computational cost for low numbers of walkers can be attributed to the low variety or sparsity of possible proposals; this is significantly higher for \texttt{emcee}/AIES. The right panel takes into account the linear scaling of the total computational cost as we increase the number of walkers and shows the total computational cost for the whole ensemble until it converges.}
    \label{fig_zeus:convergence}
\end{figure}
Another kind of analysis we can perform is to use the highly correlated 25--dimensional normal distribution as the target distribution and estimate the convergence rate of the three samplers. Although simple, the normal distribution is a valid approximation of many realistic astronomical posterior distributions and as such we expect the results presented in this paragraph to be applicable to a wide range of other distributions that resemble the normal distribution to some extent. We acknowledge however that the \emph{no free lunch} theorem also applies to this case, and there are bound to be cases in which the results would be qualitatively different. That being said, we initialised the walkers from a compact normal distribution (i.e. standard deviation equal to $10^{-4}$ times that of the target distribution) centred around a point along the first axis of the parameter space at a distance of $100$ standard deviations from the mode. We then measured the number of model evaluations required until the samplers have converged to the target distribution. The results for varying number of walkers are presented in Figure \ref{fig_zeus:convergence}.

In general, walkers move along directions defined by the walkers of the complementary ensemble. Thus, increasing the number of walkers offers a wider variety of available directions along which the walkers of \texttt{zeus} or \texttt{emcee} can move via slice sampling or Metropolis updates respectively. This is demonstrated in the left panel of Figure \ref{fig_zeus:convergence} in which the computational cost until convergence (i.e. number of model evaluations) for a single walker diminishes and then reaches a plateau as the number of walkers is increased. We notice however that, at the level of a single walker, the computational cost of \texttt{emcee}/AIES is significantly higher compared to that of either \texttt{zeus} or \texttt{emcee}/DEMC. This is due to the way that different samplers choose the directions along which walkers move. In particular, both \texttt{zeus} and \texttt{emcee}/DEMC define a direction vector as the difference between two walkers from the complementary ensemble, thus two walkers are required to define a direction. On the other hand, \texttt{emcee}/AIES requires only a single walker from the complementary ensemble as the direction is defined by the difference between the updated walker and the complementary one. This stark contrast between the way those samplers choose their direction vectors lies at the heart of the difference in the computational cost of \texttt{emcee}/AIES as compared to \texttt{zeus} and \texttt{emcee}/DEMC in the limit of low number of walkers. In order to dive a little deeper into this, we can compute the exact number of possible directions for all three methods. Since \texttt{emcee}/AIES requires only a single walker from the complementary ensemble the number of available directions is equal to the size of the complementary ensemble. On the other hand, \texttt{zeus}'s and \texttt{emcee}/DEMC's requirement for a pair of walkers means that the number of available directions is equal to $\binom{n}{2}$, meaning the 2--combination from a set of $n$ walkers that comprise the complementary ensemble. Clearly, as shown in Figure \ref{fig_zeus:proposals}, the latter increases faster with the size of the complementary ensemble, thus explaining the larger variety of possible directions available in the case of \texttt{zeus} and \texttt{emcee}/DEMC compared to \texttt{emcee}/AIES.
\begin{figure}[!ht]
    \centering
	\centerline{\includegraphics[scale=0.65]{Graphics/Zeus/number_of_proposals.pdf}}
    \caption{The figure shows the number of possible directions along which \texttt{zeus} and \texttt{emcee}/AIES can propose new samples as a function of the number of walkers in the complementary ensemble. \texttt{emcee}/DEMC exhibits the same number of proposals as \texttt{zeus} and it is not plotted here. \texttt{zeus} has a much higher variety of possible directions compared to \texttt{emcee}/AIES for any given number of walkers, assuming that that number is greater than $2$.}
    \label{fig_zeus:proposals}
\end{figure}

The discussion so far was about the computational cost of convergence in terms of the number of model evaluations for a single walker. Of course, the ensemble of walkers consists by definition of more than a single walker. Therefore, in order to compute the total number of model evaluations required until the ensemble converges we need to multiply the results of the single walker with the total number of walkers. Those results are presented in the right panel of Figure \ref{fig_zeus:convergence}. From this plot we can see that both \texttt{zeus} and \texttt{emcee}/DEMC converge faster when the number of walkers is close to its minimum value i.e. $2\times D$. \texttt{emcee}/AIES on the other hand prefers a higher number of walkers (i.e. $32\times D$) in order to overcome the sparsity of available directions in the limit of low number of walkers. This, however, means that even if we choose the optimal number of walkers for \texttt{emcee}/AIES it would still converge slower than either \texttt{zeus} or \texttt{emcee}/DEMC. Furthermore, we cannot know \emph{a priori} the optimal number of walkers for \texttt{emcee}/AIES unlike for \texttt{zeus} and \texttt{emcee}/DEMC in which the optimal size of the ensemble is close to $2\times D$. Finally, the faster convergence of \texttt{zeus} compared to \texttt{emcee}/DEMC can be attributed to the local adaptation that the former performs by extending the length of the slice and thus allowing larger steps in parameter space.

\subsubsection{The ring distribution}

The ring distribution defined as
\begin{equation}
\label{eq_zeus:ring}
    \ln P (x) = - \Bigg[ \frac{(x_{n}^{2} + x_{1}^{2} - a)^{2}}{b}\Bigg]^{2}  -\sum_{i=1}^{n-1} \Bigg[ \frac{(x_{i}^{2} + x_{i+1}^{2} - a)^{2}}{b}\Bigg]^{2} ,
\end{equation}
where $a=2$, $b=1$ and $n$ is the total number of parameters; this is an artificial target distribution that exhibits strong non-linear correlations between its parameters. This aspect of the ring distribution allows us to demonstrate the locally adaptive nature of \texttt{zeus}. Whereas \texttt{emcee}/AIES and \texttt{emcee}/DEMC use a single global proposal scale for all regions of the parameter space, \texttt{zeus} has the ability to adjust its proposal scale locally by expanding the slice appropriately. As expected, this will allow \texttt{zeus} to sample efficiently even in cases in which strong non-linear correlations are present. Looking at Figure \ref{fig_zeus:ring} one can see that \texttt{zeus} manages to generate multiple samples efficiently even in high dimensions. On the other hand, \texttt{emcee}/AIES and \texttt{emcee}/DEMC do not efficiently produce valid proposals: for \texttt{emcee}/AIES this leads to an inefficient random walk, characterised by small steps; for \texttt{emcee}/DEMC the acceptance rate almost vanishes beyond $D=2$. The expected squared jump distance of each method for the case of $D=25$ is shown in Table \ref{tab_zeus:table1}. It is important to note here that out of the three samplers only \texttt{zeus} manages to converge in all three cases (i.e. in 2, 10 and 25 dimensions). \texttt{emcee}/AIES and \texttt{emcee}/DEMC on the other hand converge successfully only in 2 dimensions.
%\begin{landscape}
\begin{figure}[!ht]
    \centering
	\centerline{\includegraphics[scale=0.45]{Graphics/Zeus/ring.pdf}}
    \caption{The figure shows numerical results (i.e. walker trajectories/chains for the first parameter) demonstrating the performance of the three ensemble MCMC methods in the case of the ring target distribution in $2, 10$ and $25$ dimensions respectively. The last column illustrates the 1-D marginal posterior corresponding to the first parameter $x_{1}$ estimated directly from the samples for the 25-dimensional case. One can notice here that in $10$ and $25$ dimensions both \texttt{emcee} methods mix very slowly. In the 25-dimensional case almost all of \texttt{emcee}/DEMC's walkers are unable to move and the autocorrelation time is effectively infinite.}
    \label{fig_zeus:ring}
\end{figure}
%\end{landscape}

To explain this result one only has to look at the distribution of walker steps of the different methods at Figure \ref{fig_zeus:ring_sep}. \texttt{zeus}'s steps extend to large distances in parameter space whereas most of \texttt{emcee}/AIES's and \texttt{emcee}/DEMC's steps are rejected (i.e. shown as zero in the histogram). We can see that \texttt{emcee}/DEMC manages to perform some long distance steps but those are few and there is almost nothing in between. It is clear from this and the previous toy examples that the $\gamma = 2.38/\sqrt{D}$ scaling of \texttt{emcee}/DEMC's scale factor does not generalise well beyond the Gaussian case.
\begin{figure}[ht!]
    \centering
	\centerline{\includegraphics[scale=0.65]{Graphics/Zeus/ring_sep.pdf}}
    \caption{This figure shows the distribution of step sizes of walkers for the three different samplers in the case of a ring target distribution in $D=25$. It is important to note here that both \texttt{emcee} algorithms exhibit a peak at zero separation; \texttt{zeus} on the other hand does not. The existence of the zero-peak in \texttt{emcee} is due to the high number of rejected proposals (i.e. low acceptance rate).}
    \label{fig_zeus:ring_sep}
\end{figure}

\subsubsection{The two-component Gaussian mixture distribution}

One other important aspect of astronomical posterior distributions is the fact that many of them exhibit multiple peaks. Multimodality can arise either from non-linear models or sparse and uninformative data. In either case, multimodal target distributions present a formidable challenge for most MCMC methods. Perhaps the simplest example of such a distribution is the two-component Gaussian mixture. In this example we will position the two, equal-mass, components at $\mathbf{-0.5}$ and $\mathbf{+0.5}$ respectively with standard deviation of $0.1$. Sampling from multimodal distributions requires two types of proposals, local proposals that sample different modes individually and global proposals that transfer walkers from one mode to the other. For this reason we will make use of \texttt{zeus}'s \texttt{GlobalMove} that uses a Dirichlet Process Gaussian Mixture model of the ensemble to efficiently propose between-mode and within-mode steps.

As seen in Figure \ref{fig_zeus:bimodal}, \texttt{zeus}'s walkers manage to move from one mode to the other frequently enough for mixing to be efficient even in the $D=25$ case. Out of \texttt{emcee}/AIES and \texttt{emcee}/DEMC, only the latter proposes valid steps from one mode to the other in the $D=2$ case. As for the $D=25$ case, one can see in Figure \ref{fig_zeus:bimodal_sep} that \texttt{zeus}'s walkers perform numerous jumps whereas \texttt{emcee}'s walkers are unable to do so. The ability of the walkers to jump from mode to mode is of paramount importance if we want to sample correctly from the target distribution. Lack of such proposals will lead to an improper probability mass ratio between the two modes and thus biased inference. The expected squared jump distance of each method for the case of $D=25$ is shown in Table \ref{tab_zeus:table1}.
%\begin{landscape}
\begin{figure}[!ht]
    \centering
	\centerline{\includegraphics[scale=0.45]{Graphics/Zeus/bimodal.pdf}}
    \caption{The figure shows numerical results (i.e. walker trajectories/chains for the first parameter) demonstrating the performance of the three ensemble MCMC methods in the case of a two-component  Gaussian mixture target distribution in $2, 10$ and $25$ dimensions respectively. The last column illustrates the 1-D marginal posterior corresponding to the first parameter $x_{1}$ estimated directly from the samples for the 25-dimensional case. Whereas all three samplers make valid within-mode proposals, it is only \texttt{zeus} that manages to perform between-mode jumps and thus sample correctly from the target distribution in the 10 and 25-dimensional cases. Between-mode jumps are paramount in order to distribute the probability mass correctly between different modes.}
    \label{fig_zeus:bimodal}
\end{figure}
%\end{landscape}

Clustering-based proposals have also been applied to MH-type ensemble MCMC methods but as shown in \textcite{karamanis2021ensemble}, they fail to generate valid proposals in problems with moderate number of dimensions. The reason is, as discussed in Section \ref{sec_zeus:tests}, that MH has to propose a valid point in the other mode. In other words, whereas Ensemble Slice Sampling only needs to determine the direction of the other mode relative to the chosen walker correctly, MH needs to guess both the direction and the distance, a task that rapidly becomes very hard as the number of dimensions rises.
\begin{figure}[ht!]
    \centering
	\centerline{\includegraphics[scale=0.65]{Graphics/Zeus/bimodal_sep.pdf}}
    \caption{This figure shows the distribution of step sizes of walkers for the three different samplers in the case of a two-component  Gaussian mixture target distribution in $D=25$. It is important to note here that both \texttt{emcee} algorithms exhibit a peak at zero separation; \texttt{zeus} on the other hand does not due to its non-rejection basis.}
    \label{fig_zeus:bimodal_sep}
\end{figure}

\subsubsection{The Student's $t$-distribution}

The fourth toy example tests the case in which the target distribution is characterised by heavy-tails. In order to demonstrate \texttt{zeus}'s ability to sample efficiency is such cases we chose to use the multivariate Student's $t$-distribution with $2$ degrees of freedom. The aforementioned density exhibits heavier tails than a normal distribution which means that it is more likely to produce samples that are far away from the mean. The $t$-distribution arises when estimating the mean of a normally distributed sample with unknown standard deviation and small size. The probability density function of a $p$--dimensional Student's $t$-distribution with $\nu$ degrees of freedom is given by:
\begin{equation}
    \label{eq_zeus:student}
    P(x) = \frac{\Gamma[(\nu+p)/2]}{\Gamma(\nu/2)\nu^{p/2}\pi^{p/2}|\boldsymbol{\Sigma}|^{1/2}}\exp\bigg[ 1 + \frac{1}{\nu}(\mathbf{x}-\boldsymbol{\mu})^{T}\boldsymbol{\Sigma}^{-1}(\mathbf{x}-\boldsymbol{\mu})\bigg]^{-\frac{(\nu+p)}{2}}\,,
\end{equation}
where $\boldsymbol{\Sigma}$ is the $p\times p$ positive semi-definite shape matrix and $\boldsymbol{\mu}$ is the mean vector.
%\begin{landscape}
\begin{figure}[!ht]
    \centering
	\centerline{\includegraphics[scale=0.45]{Graphics/Zeus/student.pdf}}
    \caption{The figure shows numerical results (i.e. walker trajectories/chains for the first parameter) demonstrating the performance of the three ensemble MCMC methods in the case of the Student's $t$-distribution with $2$ degrees of freedom in $2, 10$ and $25$ dimensions respectively. The last column illustrates the 1-D marginal posterior corresponding to the first parameter $x_{1}$ estimated directly from the samples for the 25-dimensional case.}
    \label{fig_zeus:student}
\end{figure}
%\end{landscape}

We sampled the above distribution using the three samplers in $2$,  $10$ and $25$ dimensions respectively as shown in Figure \ref{fig_zeus:student}. The diagonal elements of shape matrix $\boldsymbol{\Sigma}$ were set to $1$ and the off-diagonal elements to $0.95$. The mean vector $\boldsymbol{\mu}$ was set to $\mathbf{0}$. All three samplers managed to sample efficiently in $2$,  $10$ and $25$ dimensions as shown in Figure~\ref{fig_zeus:student} and Table~\ref{tab_zeus:table1}. Overall, \texttt{zeus} was the most efficient method with \texttt{emcee}/AIES being second and \texttt{emcee}/DEMC last. One can see from Figure \ref{fig_zeus:student_sep} that the distributions of steps of \texttt{zeus} and \texttt{emcee}/AIES are very similar whereas that of \texttt{emcee}/DEMC is substantially shorter. Unlike the previous toy examples in which the proposal strategy of \texttt{emcee}/AIES was causing it to overshoot the bulk of posterior mass, in the case of the heavy-tailed $t$-distribution more proposals are accepted. On the other hand, \texttt{emcee}/DEMC's proposals which are optimised for Gaussian targets are more conservative in the case of the $t$-distribution and they do not extend far away. As also demonstrated in the previous toy examples, the locally adaptive nature of \texttt{zeus} allows it to perform efficient proposals that span large distances in parameter space.
\begin{figure}[ht!]
    \centering
	\centerline{\includegraphics[scale=0.65]{Graphics/Zeus/student_sep.pdf}}
    \caption{This figure shows the distribution of step sizes of walkers for the three different samplers in the case of the Student's $t$-distribution with $2$ degrees of freedom in $D=25$. \texttt{zeus} and \texttt{emcee}/AIES exhibit similar distributions whereas \texttt{emcee}/DEMC performs shorter steps.}
    \label{fig_zeus:student_sep}
\end{figure}

\subsubsection{The truncated normal distribution}

The fifth and final toy example tests the case in which the target distribution is bounded from below or above. We chose to employ a truncated normal distribution similar to the one used in the first toy example, with the additional constraint being that $x > 0$. This effectively introduces a hard boundary along all dimensions. One of the reasons that we study this distribution is to assess the bias introduced by the presence of the hard boundary.
%\begin{landscape}
\begin{figure}[!ht]
    \centering
	\centerline{\includegraphics[scale=0.45]{Graphics/Zeus/truncated.pdf}}
    \caption{The figure shows numerical results (i.e. walker trajectories/chains for the first parameter) demonstrating the performance of the three ensemble MCMC methods in the case of the truncated normal distribution in $2, 10$ and $25$ dimensions respectively. The last column illustrates the 1-D marginal posterior corresponding to the first parameter $x_{1}$ estimated directly from the samples for the 25-dimensional case. \texttt{zeus} exhibits the least amount of bias near the hard boundary at zero compared to \texttt{emcee}/AIES and \texttt{emcee}\allowbreak/DEMC.}
    \label{fig_zeus:truncated}
\end{figure}
%\end{landscape}

We sampled the above distribution using the three samplers in $2$,  $10$ and $25$ dimensions respectively as shown in Figure \ref{fig_zeus:truncated}. The diagonal elements of the covariance matrix were set to $1$ and the off-diagonal to $0.95$. The mean vector $\boldsymbol{\mu}$ was set to $\mathbf{0}$. All three samplers managed to sample efficiently in $2$,  $10$ and $25$ dimensions as shown in Figure~\ref{fig_zeus:truncated} and Table~\ref{tab_zeus:table1}.  Overall, \texttt{zeus} was the most efficient method with \texttt{emcee}/DEMC being second and \texttt{emcee}/AIES last. One can see from Figure \ref{fig_zeus:truncated_sep} that the distributions of steps of \texttt{zeus} and \texttt{emcee}/AIES are very similar whereas that of \texttt{emcee}/AIES is slightly shorter. As shown in the right panels of Figure \ref{fig_zeus:truncated} \texttt{zeus} exhibits the least amount of bias compared to \texttt{emcee}/AIES and \texttt{emcee}/DEMC. In practical astronomical examples however, only one or two parameters would usually be bounded (e.g. the neutrino mass in galaxy clustering analyses) and thus unbiased sampling would be easier to perform by either of the three samplers.
\begin{figure}[ht!]
    \centering
	\centerline{\includegraphics[scale=0.65]{Graphics/Zeus/truncated_sep.pdf}}
    \caption{This figure shows the distribution of step sizes of walkers for the three different samplers in the case of the truncated normal distribution in $D=25$. \texttt{zeus} and \texttt{emcee}/AIES exhibit similar distribution whereas \texttt{emcee}/DEMC performs shorter steps.}
    \label{fig_zeus:truncated_sep}
\end{figure}

\subsection{Real astronomical analyses}

The previous section employs toy examples in order to exhibit various scenarios that might emerge during sampling, and shows how \texttt{zeus} is better equipped to handle them. To demonstrate the efficiency of \texttt{zeus} compared to other samplers in realistic target distributions, we chose two common astronomical inference problems as the testing ground. Those are the cases of baryon acoustic oscillation (BAO) parameter inference and exoplanet parameter estimation.

We used the same three samplers in our comparison, namely \texttt{emcee} with AIES and DEMC, and of course \texttt{zeus}. We performed three distinct tests:

\begin{itemize}
    \item The first test was to estimate the efficiency for each sampler, defined as the number of independent samples produced per log-likelihood evaluation. To this end, we ran the MCMC procedure 5 times for each sampler and computed the mean efficiency using the estimated autocorrelation time of the chains. The autocorrelation time was estimated using the method presented in \textcite{karamanis2021ensemble}.
    \item The second test relates to the convergence rate of the three algorithms. As a measure of convergence rate, we adopt the inverse of the number of iterations required until all the convergence criteria specified below are met. In order to estimate the mean convergence rate we ran the sampling procedure 40 times for each sampler initialising the walkers close to the \textit{Maximum a Posteriori} (MAP) estimate. 
    \item Finally, we tested the sensitivity of the samplers to the initial conditions by running 40 realisations with the walkers initialised from a small sphere (of radius $10^{-4}$) around a randomly chosen point in the prior volume, counting how many of those attempts led to converged chains before a predetermined number of likelihood evaluations.
\end{itemize}

To determine whether a chain has converged we used four different metrics: the Gelman-Rubin split-$R$ statistic \parencite{gelman1992inference, gelman2013bayesian} using four independent ensembles of walkers; the Geweke test \parencite{geweke1992evaluating}; a minimum length of the chain as a multiple of the integrated autocorrelation time (IAT); as well as an upper bound on the rate of change of the IAT. Only the second half of the chains was used to evaluate the aforementioned criteria. The number of walkers used in both examples was close to the minimum value of $2\times D$ as specified below. As we will discuss in Section \ref{sec_zeus:discussion} this often leads to faster convergence.

\subsubsection{Cosmological inference}

The particular inference problem that we consider here is that of the anisotropic BAO parameter inference using estimates of the galaxy power spectrum. The data we used comes from the 12th data release (DR12) of the high-redshift North Galactic Cap (NGC) sample as observed by the Sloan Digital Sky Survey (SDSS) \parencite{SDSSIII} Baryon Oscillation Spectroscopic Survey (BOSS) \parencite{BOSS}. Our analysis follows closely that of \textcite{beutler2017clustering} with the difference that we chose not to fix any parameters and fit the hexadecapole multipole of the power spectrum as well as the monopole and quadrupole. Those choices were made solely to render the problem more challenging. Indeed the inclusion of the hexadecapole does not contribute any additional constraining power for the data that we used. However, such extended models will prove useful when analysing data from larger galaxy surveys such as DESI \parencite{DESI}. In terms of Bayesian inference, the problem has 22 free parameters. The results of our analysis are consistent with those of \textcite{beutler2017clustering}. We used weakly informative flat (uniform) priors for all parameters except for the two scaling parameters, $\alpha_{\parallel}$ and $\alpha_{\bot}$ for which we used normal (Gaussian) priors. We used $50$ walkers in total.

In terms of efficiency, \texttt{zeus} generates at least 5 effectively independent samples for each one generated by \texttt{emcee}/DEMC and at least 9 for each one generated by \texttt{emcee}/AIES factoring in the different computational costs of the methods. As for the convergence rate, \texttt{zeus} converges more than 3 times faster than either \texttt{emcee} variant. Finally, we found that \texttt{zeus} is less sensitive to the initialisation than either of the other two methods. In particular, out of the 40 tests conducted with different initialisation, \texttt{zeus} converged 36 times, \texttt{emcee}/DEMC 14 times and \texttt{emcee}/AIES 7 times prior to the predetermined maximum number of likelihood evaluations (i.e. $5\times 10^6$ in this case). The aforementioned results are presented in detail in Table \ref{tab_zeus:table2}. The 1-D and 2-D marginal posterior distributions are shown in Figure \ref{fig_zeus:bao} demonstrating the agreement between the three methods\footnote{No upper limit on the number of likelihood evaluations or iterations was used for this run and convergence was diagnosed using all the metrics that we introduced.}.

\begin{table}[ht!]
    \centering
    \caption{The table shows a comparison of \texttt{emcee}/AIES, \texttt{emcee}\allowbreak/DEMC and \texttt{zeus} in terms of the inverse efficiency (i.e. reciprocal of the number of independent samples per model evaluation or the autocorrelation time estimate times the average number of model evaluations per iteration per walker), the convergence cost (i.e. number of model evaluations until convergence) and the convergence fraction (i.e. fraction of converged chains for given maximum number of model evaluations).}
    \def\arraystretch{1.1}
    \begin{tabular}{lcccc}
        \toprule[0.75pt]
         & \texttt{emcee}/AIES   & \texttt{emcee}/DEMC   & \textbf{zeus}  \\
        \midrule[0.5pt]
        \multicolumn{4}{l}{Cosmological inference} \\
        \midrule[0.5pt]
        efficiency$^{-1}$    &    12140    &    6750    &   $\mathbf{1320}$   \\
        convergence cost & $24\times10^{5}$ & $22\times10^{5}$ & $\mathbf{6.6\times10^{5}}$   \\
        convergence fraction   &    7/40   &    14/40    & $\mathbf{36/40}$  \\
        \midrule[0.5pt]
        \multicolumn{4}{l}{Exoplanet inference} \\
        \midrule[0.5pt]
        efficiency$^{-1}$    &    $1386$    &    $338$    &   $\mathbf{47}$   \\
        convergence cost & $36.0\times 10^{2}$ & $17.1\times 10^{2}$ & $\mathbf{4.8\times 10^{2}}$   \\
        convergence fraction   &    23/40    &    29/40    &  $\mathbf{38/40}$ \\
        \bottomrule[0.75pt]
        \end{tabular}
    \label{tab_zeus:table2}
\end{table}

\begin{figure}[ht!]
    \centering
	\centerline{\includegraphics[scale=0.25]{Graphics/Zeus/bao.pdf}}
    \caption{A corner plot showing the 1-D and 2-D marginalised posteriors for the 22-parameter Baryon Acoustic Oscillation model as produced by the three different ensemble MCMC methods.}
    \label{fig_zeus:bao}
\end{figure}

\subsubsection{Exoplanet inference}

Another common application of MCMC methods in astronomy is the problem of exoplanet parameter inference through modelling of Keplerian orbits and radial velocity time series data. In this section we demonstrate the performance of \texttt{zeus} using a two-planet model with $14$ free parameters and real data from the K2-24 (EPIC-203771098) extrasolar system \parencite{k224} that is known to host two exoplanets. We used the popular \texttt{Python} package \texttt{RadVel} \parencite{radvel} for the Keplerian modelling of the planetary orbits. The results of our analysis are consistent with published constraints for the aforementioned extrasolar system \parencite{k224}. We used $30$ walkers in total for sampling.

We performed the same suite of tests as in the cosmological inference case. In terms of efficiency, \texttt{zeus} generates more than $7$ independent samples per each one generated by \texttt{emcee}/DEMC and more than $29$ independent samples per each one generated by \texttt{emcee}/AIES. As for the convergence rate, \texttt{zeus} converges $7.5$ times faster than \texttt{emcee}/AIES and $3.5$ faster than \texttt{emcee}/DEMC on average. Finally, we found again that \texttt{zeus} is less sensitive to the specific initialisation of the walkers. In particular, out of the 40 tests conducted with different initialisation, \texttt{zeus} converged 38 times, \texttt{emcee}/DEMC 29 times and \texttt{emcee}/AIES 23 times prior to the predetermined maximum number of likelihood evaluations (i.e. $5\times 10^3$ in this case). Detailed results about the values of the used metrics are shown in Table \ref{tab_zeus:table2}. The 1-D and 2-D marginal posterior distributions are shown in Figure \ref{fig_zeus:exo}, demonstrating the agreement between the three methods.

\begin{figure}[ht!]
    \centering
	\centerline{\includegraphics[scale=0.40]{Graphics/Zeus/exo.pdf}}
    \caption{A corner plot showing the 1-D and 2-D marginalised posteriors for the 14-parameter radial velocity model as produced by the three different ensemble MCMC methods.}
    \label{fig_zeus:exo}
\end{figure}

\section{Discussion}
\label{sec_zeus:discussion}
Following the analysis we conducted in Section \ref{sec_zeus:tests} using the normal distribution there are two important questions that need to be answered about the initialisation of the walkers. First, how many walkers are necessary and, second, how to choose the initial positions of the walkers. Although there are many ways of answering those questions and there is no consistent solution that works for all target distributions, we will try to provide some general rules and heuristics to help ease the task of choosing the number and initial positions of the walkers for most cases.

Let us first discuss the effect of the number of the walkers on the general performance of \texttt{zeus}. Naively, one might expect that the minimum number of walkers should be $D+1$, where $D$ is the number of dimensions. However, the ensemble splitting technique, which was introduced in Section \ref{sec_zeus:tests} to render the algorithm parallelisable, requires at least $2\times D$ walkers in order to produce $2$ linearly independent samples. If a smaller number is chosen then the walkers can be trapped in a lower--dimensional hyper--plane of the parameter space, being unable to sample properly and leading to erroneous results. Although there is no upper bound on the number of walkers, we recommend to use between two to four times the number of dimensions. The reason is that increasing the number of dimensions can increase the cost of the burn-in period as we explained in detail in Section \ref{sec_zeus:tests}. Ideally, one wants to use the minimum number (or close to that) of walkers until the burn-in period is over and then increase the number of walkers to rapidly produce a great number of independent samples. It is also worth noting that in cases in which either non-linear correlations or multiple modes are present it is recommended to use more walkers (e.g. 4-8 times the number of parameters for a bimodal target distribution).

As for the initialisation of the walkers, there are many ways to choose their starting positions ranging from prior sampling to more localised initial positions. Empirical tests indicate that the latter often outperforms the former (i.e. leads to shorter burn-in periods). That is not surprising since the total probability of a prior-sampled initialisation can be very small when the number of parameters is high. In particular we found that initialising the walkers from a tight region in parameter space (i.e. normal distribution with small variance) consistently leads to good performance. For low to moderate dimensional problems initialising the walkers from a tight ball around the \textit{Maximum A Posteriori} (MAP) estimate can substantially reduce the burn-in period~\parencite{foreman2013emcee}.

Finally, while \texttt{emcee}/AIES and \texttt{emcee}/DEMC can sample acceptably from most target distributions with $D\lesssim 20$, the efficient scaling of \texttt{zeus} with the number of parameters allows us to extend this range and efficiently test more complicated models \parencite{karamanis2021ensemble}. Like most gradient-free methods, \texttt{zeus} will fail to sample efficiently in very high dimensional problems in which $D=\mathcal{O}(10^{2})$. In such cases, more sophisticated algorithms (e.g. tempering, block updating, Hamiltonian dynamics etc.) need to be used \parencite{2018arXiv180402719R}.

\section{Conclusions}
\label{sec_zeus:conclusions}

The aim of this project was to develop a tool that could facilitate Bayesian parameter inference in computationally demanding astronomical analyses and tackle the challenges posed by the complexity of the models and data that are often used by astronomers. To this end, we introduced \texttt{zeus}, a parallel, general-purpose and gradient-free \texttt{Python} implementation of Ensemble Slice Sampling.

After introducing the method in Section \ref{sec_zeus:ess}, we thoroughly demonstrated its performance compared to two popular alternatives (i.e. \texttt{emcee} with affine-invariant ensemble sampling and differential evolution Metropolis) using a variety of artificial and realistic target distributions in Section \ref{sec_zeus:tests}. The artificial toy examples helped to shed light on the general behaviour of the samplers in target distributions characterised by linear and non-linear correlations as well as multimodal densities. When compared to \texttt{emcee}/AIES and \texttt{emcee}/DEMC in the problems of Baryon Acoustic Oscillation parameter inference and exoplanet radial velocity fitting, \texttt{zeus} consistently converges faster (i.e. its burn-in is shorter by a factor of at least 3), it is less sensitive to the initialisation of the walkers and generates substantially more independent samples per likelihood evaluation (i.e. approximately $\times 9$ and  $\times 29$ speed-up compared to \texttt{emcee}/AIES in the cosmological and exoplanet examples, respectively). 

We have shown that \texttt{zeus} performs similarly or better than existing MCMC methods in a range of problems. We hope that \texttt{zeus} will prove useful to the astronomical and cosmological community by complementing existing approaches and facilitating the study of novel models and data over the coming years. \texttt{zeus} is publicly available at \url{https://github.com/minaskar/zeus} with detailed documentation and examples that can be found at \url{https://zeus-mcmc.readthedocs.io}.

As we can see in Table \ref{tab:Zeus}, the authors created several groups (e.g., incorrect contracts, unfair contracts), in which several defects are placed. Although this is obviously a partial classification of known vulnerabilities, the heterogeneity of the naming and definitions and also general classification structures is clear (when compared to other works), which again emphasizes the need for a more standard way of categorizing defects.


%\input{tables/tableIX}


\subsection{Limitations of Current Classification Schemes}

In this section, we highlight the main gaps and limitations identified during the analysis of the different vulnerability classifications previously described, as follows: 

\begin{enumerate}[-]

\item Classifications proposed in the literature tend to have simple structures, most of them simply grouping the vulnerabilities into related groups. Many times, no groups at all are used. Such structures are often ad-hoc and consequently short-lived, resulting in limited adoption. The classifications that collect more vulnerabilities are found in \cite{Rameder2022, swc, Tsankov2018}, with \cite{Tsankov2018} grouping vulnerabilities by criticality and with \cite{Rameder2022} using conceptual groups to fit related vulnerabilities.

\item  There is a large diversity of names being used in state of the art to refer to the same vulnerability (e.g., both \textit{Integer bugs or arithmetic issues} and \textit{Integer over- or underflow} \cite{Rameder2022} refer to the same vulnerability). There are also cases in which very similar names refer to different vulnerabilities (e.g., \textit{unpredictable state} \cite{bauer_semantic_2018} refers to wrong class inheritance order defect while \textit{vulnerable state} \cite{krupp_teether_2018} refers to uninitialized storage variable defect). In some cases, the same name refers to different vulnerabilities, e.g., Transaction Order Dependency (TOD) is the name used in \cite{liao_soliaudit_2019} and in \cite{SAILFISH_2022}, which however refers respectively to "5.1.5 Transfer Amount Dependent on Transaction Order" and to "5.1.6 Transfer Recipient Dependent on Transaction Order".

\item Current classifications include several generic names that do not assist in the classification of specific defects (e.g., \textit{call to the unknown} \cite{Atzei2017} or \textit{unexpected function invocation} \cite{Chen2020a}). In several cases, unclear nomenclatures are used, such as \textit{entropy illusion}, \textit{constructors with care} \cite{SIGP2018}, or \textit{Improper Blockchain Magic Validation} \cite{Amiet2021}, which do not specify what the defect is. Another example is \textit{Style guide violation} \cite{zhang_soliditycheck_2019}, which is not even clear whether it is referring to bad practice or a vulnerability.

\item Regarding vulnerability classification, current research appears to be falling far behind the state of the practice. Current vulnerability detection tools identify several vulnerabilities (e.g., Securify2 \cite{Tsankov2018}) that do not fit in relatively well-established classifications, such as DASP \cite{DASP}, or SWC \cite{swc}.

\item Current classifications do not involve active community participation, and we observed little to no participation at all in several classifications. Thus, it is fundamental that a classification can be easily maintained and evolve to integrate new vulnerabilities or even has the possibility of structurally changing (i.e., versioning is also required). This reduced community participation is the main reason why the most popular classification initiatives, like SWC \cite{swc} or DASP \cite{DASP}, are currently far behind the detection capabilities of vulnerability detection tools. 

\item Classifications originated from vulnerability detection tools sometimes use names that are biased towards the tool's capabilities, which is fully acceptable from a tool perspective, but for broader goals (e.g., tool benchmarking), a vulnerability classification must be independent of specific tools' capabilities. For instance, Osiris \cite{torres_osiris_2018} is a tool for detecting vulnerabilities related with integer values and naturally focuses on a few types of issues affecting integer manipulation. Thus, the naming used is very specific of this context and also does not capture the larger picture (e.g., issues affecting other types of numbers may be related, but are not represented).

\item The high heterogeneity of names used across various tools, community efforts, and research initiatives creates a significant obstacle to understanding which tools perform better. Although initiatives exist to assess the effectiveness of the vulnerability detection tools, they all faced difficulties in adopting a uniform, fine-grained taxonomy for defects.

\item Many times, taxonomies mix the characteristics of a certain vulnerability with the effect of exploiting it or with how it is exploited, or its impact, and use category names like \textit{Denial of Service}, which is basically a consequence of the activation of a certain vulnerability. This is not necessarily wrong, but it may contribute to a non-uniform taxonomy and possibly error-prone from the point of view of the taxonomy's user.

\item Classification structures are often constructed with different degrees of granularity. Some structures have general categories, while others have more specific categories. This inconsistent categorization of  poses difficulties and complexities for practitioners and tool developers, as they end up creating new classifications. Overall, a broader view on vulnerability detection is needed to foster the longevity of a particular taxonomy, accompanied with the possibility of evolving it. 

\end{enumerate}
