\documentclass[aps,prl,twocolumn,superscriptaddress,preprintnumbers]{revtex4-1}% nofootinbib
\synctex=1 
%nofootinbib
\usepackage{epsf,epsfig}
\usepackage[utf8]{inputenc}
\usepackage{amssymb,amsmath,amsfonts}
\usepackage{color}
\usepackage{slashed}
\usepackage{tensor} 
\usepackage{braket}
\usepackage{float}
\usepackage[normalem]{ulem} %defines \sout{word} which strikes out {word}
%\usepackage{subcaption}
%\captionsetup{justification=justified,singlelinecheck=false}
\usepackage[colorlinks=true]{hyperref}  
\hypersetup{
    bookmarks=true,         % show bookmarks bar?
    unicode=false,          % non-Latin characters 
    pdftoolbar=true,        % show Acrobat
    pdfmenubar=true,        % show Acrobat 
    pdffitwindow=false,     % window fit to page when opened
    pdfstartview={FitH},    % fits the width of the page to the window
    colorlinks=true,       % false: boxed links; true: colored links
    linkcolor=magenta, %red,          % color of internal links (change box color with linkbordercolor)
    citecolor=blue,        % color of links to bibliography
    filecolor=magenta,      % color of file links
    urlcolor=cyan           % color of external links
} 
%%%%%%%%%%%%%%%%%%%%%%%
\newcommand{\labell}[1]{\label{#1}}
%\newcommand{\labell}[1]{\quad\mt{#1}\quad\label{#1}}
\newcommand{\etal}{{\it et.\ al. }}
\newcommand{\bea}{\begin{eqnarray}}
\newcommand{\arxiv}[1]{{\tt
\href{http://www.arXiv.org/abs/#1}{#1}}}
\newcommand{\diff}{\mathrm{d}}
\newcommand{\eea}{\end{eqnarray}}
\usepackage{lipsum} 
\newcommand{\ba}{\begin{eqnarray}}
\newcommand{\ea}{\end{eqnarray}}
\newcommand{\rh}{r_{\rm h}}
\newcommand{\nn}{\nonumber \\}
\newcommand{\eqn}[1]{(\ref{#1})}
\newcommand{\beq}{\begin{equation}}
\newcommand{\eeq}{\end{equation}}
\newcommand{\beqa}{\begin{eqnarray}}
\newcommand{\eeqa}{\end{eqnarray}}
\newcommand{\beqar}{\begin{eqnarray*}}
\newcommand{\eeqar}{\end{eqnarray*}}
\newcommand{\e}{\epsilon}
\newcommand{\reef}[1]{(\ref{#1})}
\newcommand{\ssc}{\scriptscriptstyle}
\newcommand{\eg}{{\it e.g.,}\ }
\newcommand{\ie}{{\it i.e.,}\ }
\newcommand{\equ}[1]{(\ref{#1})}
\newcommand{\mt}[1]{\textrm{\tiny #1}}
\newcommand{\ang}[1]{\left\langle #1 \right\rangle}
\newcommand{\veps}{\varepsilon}
\newcommand{\X}{\mathcal{X}}
\newcommand{\W}{\mathcal{W}}
\newcommand{\R}{\mathcal{R}}
\newcommand{\Z}{\mathcal{Z}}
\newcommand{\A}{\mathcal{A}}
\newcommand{\B}{\mathcal{B}}
\newcommand{\C}{\mathcal{C}}
\newcommand{\D}{\mathcal{D}}
\newcommand{\E}{\mathcal{E}}
\newcommand{\G}{\mathcal{G}}
\newcommand{\cO}{\mathcal{O}}
\newcommand{\cM}{\mathcal{M}}
\newcommand{\hH}{\mathcal{H}} %{{\widehat H}}
\newcommand{\nvec}{{\vec{n}}}
\newcommand{\lgb}{\lambda_{\text{\tiny{GB}}}}
\newcommand{\tL}{\tilde{L}}
\newcommand{\nc}{N_c}
\newcommand{\pd}{\partial}
\newcommand{\la}{\lambda}
\newcommand{\lp}{\ell_{\mt P}}
\newcommand{\mbf}{\mathbf}
\newcommand{\fin}{f_\infty}
\newcommand{\ff}{f_{\infty}}
\newcommand{\ct}{C_{T}} %{C_\mt{T}}
\newcommand{\hi}{{\hat \imath}}
\newcommand{\hj}{{\hat \jmath}}
\newcommand{\qn}{\textswab{q}}
\newcommand{\wn}{\textswab{w}}
\newcommand{\nw}{N_W}
\newcommand{\tr}{{\tilde \rho}}
\newcommand{\hr}{{\hat r}}
\newcommand{\vrho}{\varrho}
\newcommand{\trho}{{\tilde\varrho}}
\newcommand{\ttau}{{\tilde \tau}}
\newcommand{\ta}{{\tilde a}}
\newcommand{\tb}{{\tilde b}}
\newcommand{\tc}{{\tilde c}}

\usepackage{color}
\let\normalcolor\relax
\newcommand{\tcb}{\textcolor{blue}}
\newcommand{\tcr}{\textcolor{red}}
\newcommand{\tcg}{\textcolor{green}}
\newcommand{\red}{\textcolor{red}}
\newcommand{\blue}{\textcolor{blue}}
\newcommand{\orange}{\textcolor{orange}}
\newcommand{\green}{\textcolor{green}}


\newcommand{\ads}{a_d^*}
\newcommand{\pe}{\rho_\mt{E}}
\newcommand{\hC}{{\widehat C}}
\newcommand{\te}{t_\mt{E}}
\newcommand{\too}[1]{\mathrel{\mathop \to_{\scriptscriptstyle{#1}}}{}\!\!}
\newcommand{\dr}{{\Delta R}}
\newcommand{\m}{\text{min}}
\renewcommand{\c}{$c$}
\newcommand{\F}{$F$}
\newcommand{\cA}{{\cal A}}
\newcommand{\mv}{{\cal a}}
\newcommand{\Sig}{{\partial V}}
\newcommand{\un}{{\hat{\bf n}}}
\newcommand{\um}{{\widehat{\bf m}}}
\newcommand{\ut}{{\hat{\bf t}}}
\newcommand{\uu}{{\hat{\bf u}}}

\newcommand{\ren}{R\'enyi\ }
\newcommand{\sg}{\sigma}
\newcommand{\q}{a}
\newcommand{\qe}{\q_{\rm\ssc E}}
\newcommand{\see}{S} %{S_{\ssc EE}}
\newcommand{\ka}{\kappa}
\newcommand{\kae}{\ka_{\rm \ssc E}}
\newcommand{\sge}{\sg_{\rm \ssc E}}
\newcommand{\cs}{c_{\ssc S}}
\newcommand{\cse}{c_{\ssc S,E}}
\newcommand{\ctt}{C_{\ssc T}}
\newcommand{\ctte}{C^{\rm \ssc E}_{\ssc T}} 
\newcommand{\cttgb}{C_{\ssc T,GB}}
\newcommand{\lhat}{\hat\lambda}
\newcommand{\alhat}{\hat\alpha}

\newcommand{\mc}{\mathcal}
\newcommand{\al}{\alpha}
\newcommand{\de}{\delta}
\newcommand{\ts}{\thinspace{}}
\newcommand{\req}[1]{(\ref{#1})} %{Eq.\thinspace(\ref{#1})}
%\newcommand{\comment}[1]{{\it \color{blue}{[[[#1]]]}}}
\newcommand{\comment}[1]{{\bf *******{#1}*******}}

\newcommand{\bl}[1]{{\color{blue}#1}}
\newcommand{\rd}[1]{{\color{red}#1}}

\begin{document}

%\title{Holographic quantum critical transport with self-duality}
\title{Universal aspects of holographic quantum critical transport with self-duality}
\author{\'Angel J. Murcia}
\email{angel.murcia@pd.infn.it}
\affiliation{INFN, Sezione di Padova, Via Marzolo 8, 35131 Padova, Italy}

\author{Dmitri Sorokin} 
\email{dmitri.sorokin@pd.infn.it}
\affiliation{INFN, Sezione di Padova, Via Marzolo 8, 35131 Padova, Italy}
\affiliation{Dipartimento di Fisica ed Astronomia ``Galileo Galilei'',
Università degli Studi di Padova, Via Marzolo 8, 35131 Padova, Italy}


%\date{\today}
%We prove several universal properties of charge transport that hold for every CFT holographically dual to a four-dimensional theory of gravity non-minimally coupled to a gauge vector field in a duality-invariant way

\begin{abstract}
We prove several universal properties of charge transport in generic CFTs holographic to nonminimal extensions of four-dimensional Einstein-Maxwell theory with exact electromagnetic duality invariance. First, we explicitly verify that the conductivity of these theories at zero momentum is a universal frequency-independent constant. Then, we derive their analytical expressions for non-zero momentum in any holographic duality-invariant theory for large frequencies and in the limit of small frequencies and momenta. Next, in the absence of terms that couple covariant derivatives of the curvature to gauge field strengths, two universal features are proven. On the one hand, it is shown that for a general-relativity neutral black-hole background the conductivities at any frequency and momentum are independent of the choice of duality-invariant theory, thus coinciding with those in the Einstein-Maxwell case. On the other hand, if higher-curvature terms affect the gravitational background, the conductivities get modified, but the contributions from nonminimal couplings of the gauge field to gravity  are subleading. We illustrate this feature with an example. 

%in more general (higher-curvature gravity)
%backgrounds the conductivities get modified, but the contributions from nonminimal couplings of the gauge field to gravity are subleading.
%We illustrate this feature with an example. 




%On the other hand, for more generic {\color{red}(higher-curvature gravity)} backgrounds, it is proven that corrections to the conductivities coming from the specific choice of duality-invariant theory are subleading 


%in these theories with respect to the case of Maxwell theory minimally coupled to a higher-curvature gravity are extremely suppressed. We corroborate this feature in an explicit {\color{red} example} of duality-invariant theory with {\color{red} cubic higher-curvature} terms.

%Thirdly, in the absence of terms that couple covariant derivatives of the curvature to gauge field strengths, we prove that the conductivities at any frequency and momentum are independent of the choice of duality-invariant theory as long as a general-relativity neutral black hole background is considered. For more generic gravitational configurations, we observe that the conductivities get modified for non-zero momentum. However, we find that these corrections are dominated by the particular gravitational background selected, the effects associated to the specific choice of duality-invariant theory being subleading. We corroborate this last feature with an explicit instance of  duality-invariant theory with higher-order terms.

%We prove several universal properties of charge transport in CFTs which are holographically dual of four-dimensional theories of gravity non-minimally coupled to electromagnetism in a duality-invariant way. First, we show that the product of the transverse and longitudinal self-energies is a universal constant for all frequencies and momenta, independent of the CFT under study. Secondly, we demonstrate that the conductivity at zero momentum is the same frequency-independent constant for all holographic duality-invariant theories. Thirdly, we prove that the conductivities at any frequency and momentum are independent of the duality-invariant theory as long as a General Relativity gravitational background is considered. For more generic gravitational configurations, we observe that the conductivities get modified for non-zero momentum. We find that these corrections are dominated by the particular choice of gravitational background, the effects associated to a chosen \sout{the specific selection of} duality-invariant theory being highly suppressed. We corroborate this last feature with a specific example. 

%We prove several universal features of charge transport  CFTs which are holographically dual to four-dimensional duality-invariant theories of gravity and electromagnetism. 

%Using the classification of all duality-invariant theories which are quadratic in the Maxwell field strength provided in \cite{Cano:2021tfs}, it was possible 


%We prove that the conductivity of any self-dual holographic theory of gravity and electromagnetism, possibly including higher-order terms and nonminimal couplings, is exactly the same to that of Einstein-Maxwell theory for any frequency and momentum.
\end{abstract} 

\maketitle

The AdS/CFT correspondence \cite{Maldacena:1997re,Gubser:1998bc,Witten:1998qj} has become a powerful and fruitful tool for the study of strongly-coupled systems in the vicinity of quantum critical points, leading to the development of the so-called AdS/Condensed Matter (AdS/CMT) duality \cite{Hartnoll:2009sz,Sachdev:2010ch,Zaanen:2015oix,Hartnoll:2016apf,Baggioli:2016rdj,nastase_2017}. Among other aspects, it has been possible to identify a variety of holographic models which exhibit properties characteristic of condensed-matter systems, such as superfluidity, superconductivity or (quantum) Hall conductivity \cite{Hartnoll:2008vx,Keski-Vakkuri:2008ffv,Herzog:2008he, Hartnoll:2008kx,Herzog:2009xv,Bergman:2010gm}. Further scrutiny of such interesting features have turned AdS/CMT into a highly active topic of research --- see \emph{e.g.} \cite{Hartnoll:2020fhc,Arean:2021tks,Ammon:2021pyz,Sword:2021pfm,Donos:2022www,Flory:2022uzp,Donos:2022qao}.% for a necessarily incomplete list of references.
%in the recent years

On the other hand, the potential of higher-order theories of gravity to unveil generic aspects of CFTs has become evident in the recent years. Apart from being able to capture finite $N$ and finite coupling effects within the canonical holographic correspondence between Type IIB String Theory and $\mathcal{N}=4$ Super-Yang-Mills theory \cite{Buchel:2004di,Benincasa:2005qc,Myers:2008yi}, higher-order gravities make it possible to explore holographic CFTs whose correlators take the most generic form allowed by conformal symmetry \cite{deBoer:2009pn,Buchel:2009sk,Myers:2010jv,Bueno:2018xqc,Cano:2022ord} or to identify new universal relations that hold for arbitrary CFTs \cite{Myers:2010xs,Perlmutter:2013gua,Mezei:2014zla,Bueno:2015rda,Chu:2016tps,Li:2018drw,Bueno:2018yzo,Bueno:2022jbl,Baiguera:2022sao}. These features have motivated the study of (charge) transport properties of holographic duals of higher-order gravities, observing novel and intriguing phenomena in the shear viscosity to entropy density ratio \cite{Kats:2007mq,Brigante:2007nu,Ge:2008ni}, holographic superconductivity \cite{Gregory:2009fj,Pan:2010at,Edelstein:2022xlb} and (electrical) conductivities \cite{Myers:2010pk,Pal:2010sx,Witczak-Krempa:2012qgh,Witczak-Krempa:2013aea,Ling:2016dck}. 




%leading to intriguing results regarding the shear viscosity to entropy density ratio \cite{Kats:2007mq,Brigante:2007nu,Ge:2008ni}, the holographic superconductivity \cite{Gregory:2009fj,Pan:2010at,Edelstein:2022xlb} and the (electrical) conductivities \cite{Myers:2010pk,Pal:2010sx,Witczak-Krempa:2012qgh,Witczak-Krempa:2013aea,Ling:2016dck}. 

%\footnote{From an Effective Field Theory perspective, the existence of symmetries allows one to reduce the number of operators appearing in the effective action.}. 

In this work we explore various aspects of charge transport in CFTs holographic to
duality-invariant theories of electrodynamics with nonminimal couplings to gravity.  Duality invariance is a symmetry of the equations of motion of Einstein-Maxwell theory in vacuum, so it is justified to consider higher-order modifications that respect this symmetry. Explicit examples of duality-invariant theories are known to exist both with minimal couplings (see {\emph e.g.} \cite{Sorokin:2021tge} for a review) and with nonminimal couplings to gravity \cite{Cano:2021tfs,Cano:2021hje}.


%Various examples of duality-invariant nonlinear electrodynamics with minimal couplings to gravity are known (see {\emph e.g.} \cite{Sorokin:2021tge} for a review), while the classification of the theories with the most general duality-invariant non-minimal coupling of Maxwell's theory to gravity was carried out in . 

%We will prove that the following properties hold for every CFT dual to any duality-invariant theory whose equilibrium state is characterized by vanishing expectation values of all global charges, \emph{i.e.} for systems without chemical potentials:

%, as expected \cite{Myers:2010pk}

In particular, we study CFTs holographic to duality-invariant theories whose equilibrium state is characterized by vanishing expectation values of all global charges (\emph{i.e.} systems without chemical potentials). We begin by explicitly checking that the conductivity at zero momentum is a frequency-independent universal constant. Then we are able to derive the explicit expressions for the conductivities in any holographic duality-invariant theory in the regimes of large frequencies and for sufficiently small frequencies and momenta. The latter depend on both the gravitational background selected and the duality-invariant theory under study, so that conductivities at non-zero momentum will generically differ with respect to their Einstein-Maxwell values.

Nonetheless, when dealing with CFTs holographic to duality-invariant theories which do not couple covariant derivatives of the curvature to gauge field strengths, it turns out that conductivities do possess two remarkable universal features. First, whenever a general-relativity black hole background is considered, the conductivities for any frequency and momentum are the same for all such holographic theories. Secondly, if the black hole background is modified by higher-curvature terms, the conductivities get corrected, but the contributions coming from nonminimal couplings between the curvature and the gauge field are subleading. We corroborate this feature with an explicit example.


%the specific choice of duality-invariant theory does not change noticeably the conductivities, their corrections with respect to Maxwell theory (on top of this black hole background) being suppressed. We illustrate this feature in an explicit example.


%features the following universal properties:



%Regarding the last point, we observe that conductivities get corrected when higher-curvature terms affect the black hole background. However, after fixing such modified gravitational configuration, the specific choice of duality-invariant theory does not change noticeably the conductivities, their corrections with respect to Maxwell theory (on top of this black hole background) being suppressed. We illustrate this feature in an explicit example.

%(on top of the aforementioned black hole background)

%Regarding the last point, we will show that correcting the black hole background through higher-curvature \sout{terms} {\color{red} contributions} modifies noticeably the subsequent conductivities. However, a curious phenomenon occurs --- corrections associated to the specific choice of duality-invariant \sout{theory} {\color{red} non-minimal coupling of gauge fields to gravity} are subleading with respect to \sout{those arising from the particular gravitational} {\color{red} the higher-curvature corrections to the background}. We will demonstrate this feature in an explicit example.

%IIB String Theory and $\mathcal{N}=4$ Super-Yang-Mills theory

%and which could not have been guessed from the particular case of Einstein gravity  





%, they may be conceived as specific models to study more general classes of CFTs than those comprised by Einstein gravity
%they may be conceived as specific models with which to test more general classes of CFTs. In particular,




%This triggered the development of the AdS/Condensed Matter (AdS/CMT) duality 


%Mention the application of holographic toy models. Particle-vortex duality



\textbf{Duality-invariant bulk setup.} Let us consider generic nonminimal extensions of Einstein-Maxwell theory in four dimensions described by the following action:
\begin{equation}
 I=\kappa_N \int \diff^4 x \sqrt{\vert g \vert} \left[R+\frac{6}{L^2}-\chi^{\mu \nu \rho \sigma} F_{\mu \nu} F_{\rho\sigma}+\mathcal{L}_{\mathrm{grav}}^{\mathrm{high}}\right]\,,
    \label{eq:dualgen}
\end{equation}
where $\chi^{\mu \nu \rho\sigma}$ depends solely on the metric and the curvature, $\kappa_N=(16 \pi G)^{-1}$ and
\begin{equation}\label{Lhigher}
\mathcal{L}_{\mathrm{grav}}^{\mathrm{high}}=L^2 \sum_{i} \alpha_i^{(2)}\mathcal{R}_i^{(2)}+ L^4 \sum_{i}\alpha_i^{(3)} \mathcal{R}_i^{(3)}+\dots \,,
\end{equation}
where $\mathcal{R}_i^{(n)}$ stands for curvature invariants of $n$-th order --- the index $i$ denoting every such inequivalent term ---, $L$ is the cosmological-constant length scale and  $\alpha_i^{(n)}$ are dimensionless couplings characterizing the theory. Eq. \eqref{eq:dualgen} may also be interpreted as a generic effective action \cite{Weinberg:1995mt} obtained by adding to Einstein-Maxwell theory all possible terms quadratic in $F_{\mu \nu}$ which are compatible with diffeomorphism and gauge invariance. The reason why it suffices for our purposes to work at  $\mathcal{O}(F^2)$ will become apparent afterwards.



%Multiplied by $L^2$ they define the scale at which the higher order corrections should be taken into account. Such higher-derivative terms may be interpreted from an EFT perspective \cite{Weinberg:1995mt} to ensure that General Relativity is recovered at sufficiently low energies. 

%algebraic\footnote{We impose $\mathcal{T}_{\mu \nu}$ not to have covariant derivatives of the curvature. If it did, then the classical action would be equivalent, up to total derivatives, to a theory including covariant derivatives of $F_{\mu \nu}$. However, these theories lie beyond the class of theories we are considering --- quadratic in $F_{\mu \nu}$, without covariant derivatives thereof.}

Let $\mathcal{T}_{\mu \nu}$ be a traceless and symmetric tensor constructed from  contractions of the curvature tensor and its covariant derivatives, and let $b_n$ be the coefficients appearing in the Taylor series $\sqrt{1+x^2}=1+\sum_{n=1}^\infty b_n x^{2n}$. If we take the tensor $\tensor{\chi}{_{\mu \nu}^{\rho \sigma}}$ in \eqref{eq:dualgen} to be\footnote{Observe that $\tensor{\chi}{_{\mu \nu}^{\rho \sigma}}$ can be compactly expressed as $\tensor{\chi}{_{\mu \nu}^{\rho \sigma}}=\tensor{\Theta}{_{\mu \nu}^{\rho \sigma}}+\sqrt{\tensor{\delta}{_{[\mu}^{[\rho}} \tensor{\delta}{_{\nu]}^{\sigma]}}+\tensor{\Theta}{^{2}_{\mu \nu}^{\rho \sigma}}}$, as in \cite{Cano:2021hje}.}
\begin{align}\nonumber
    \tensor{\chi}{_{\mu \nu}^{\rho \sigma}}&= \tensor{\delta}{_{[\mu}^{[\rho}} \tensor{\delta}{_{\nu]}^{\sigma]}}+\tensor{\Theta}{_{\mu \nu}^{\rho \sigma}}+\sum_{n=1}^\infty b_n \tensor{\Theta}{^{2n}_{\mu \nu}^{\rho \sigma}}\,,\\ \tensor{\Theta}{^{2n}_{\mu \nu}^{\rho \sigma}}&=\tensor{\Theta}{_{\mu \nu}^{\alpha_1 \alpha_2}}\tensor{\Theta}{_{\alpha_1 \alpha_2}^{\alpha_3 \alpha_4}} \cdots \tensor{\Theta}{_{\alpha_{4n-3} \alpha_{4n-2}}^{\rho \sigma}}\,, \label{chiTheta} \\ \nonumber \tensor{\Theta}{_{\mu \nu}^{\rho \sigma}}&=\tensor{\mathcal{T}}{_{[\mu}^{[\rho}} \tensor{\delta}{_{\nu]}^{\sigma]}}\,,
    \end{align}
then the action \eqref{eq:dualgen} describes the most general exactly duality-invariant theory of electrodynamics with nonminimal couplings to gravity and having at most quadratic terms in the Maxwell field strength \cite{Cano:2021hje}. More concretely, this means that the set of equations formed by the equations of motion of  \eqref{eq:dualgen} and by the Bianchi identity for $F_{\mu \nu}$ is invariant under rigid $\mathrm{U}(1)$ rotations of the complex tensor $F'_{\mu\nu}+iH'_{\mu\nu}=e^{i\alpha}(F_{\mu\nu}+iH_{\mu\nu})$, where
\begin{equation}
\star H_{\mu \nu}=\frac 12 \frac{\delta I}{\delta F^{\mu\nu}}=-\tensor{\chi}{_{\mu \nu}^{\rho \sigma}} F_{\rho \sigma}\,.
\end{equation}
 In the case of Maxwell theory minimally coupled to a higher-curvature gravity, $\tensor{\chi}{_{\mu \nu}^{\rho \sigma}}=\tensor{\delta}{_{[\mu}^{[\rho}} \tensor{\delta}{_{\nu]}^{\sigma]}}$ and $H_{\mu\nu}$ is just the Hodge dual of $F_{\mu\nu}$.

With the goal of studying thermal CFTs in flat Minkowski space, we consider gravitational backgrounds of  \eqref{eq:dualgen} (\emph{i.e.} we take $F_{\mu \nu}=0$) which correspond to AdS black holes with a planar horizon --- usually called  (AdS) black branes in the literature:
\begin{equation}
    \diff s^2=\frac{r_0^2}{\tilde{L}^2 u^2}\left (-N^2(u)f(u) \diff t^2+\diff x^2+\diff y^2 \right )+\frac{\tilde{L}^2}{u^2 f(u)} \diff u^2\,,
    \label{eq:bhbackground}
\end{equation}
where $r_0$ is a constant of dimension of length, $\tilde{L}$ denotes the AdS length scale, generically differing from the cosmological-constant scale $L$ because of the higher-order corrections \cite{Buchel:2009sk,Bueno:2016xff,Bueno:2016ypa}, and
\begin{equation}
   N=1+\tilde{N}\, , \quad  f=1-u^3+\tilde{f}\,,
\end{equation}
where $\tilde{N}$ and $\tilde{f}$ are $u$-dependent functions encoding the higher-order corrections with respect to the GR solution ($\tilde{N}=\tilde{f}=0$) such that $\lim_{u \rightarrow 0} \tilde{f}= \lim_{u \rightarrow 0} \tilde{N}=0$. In this coordinate system the AdS boundary is located at $u=0$, while the horizon (which we assume to exist) is at $u=u_h$. The black hole temperature is
\begin{equation}\label{BHT}
T=-\frac{r_0 f'(u_h)}{4 \pi \tilde{L}^2}\,.
\end{equation}

\textbf{Retarded correlators from the AdS/CFT correspondence.}
We are interested in computing the retarded two-point current correlator $C_{ab}$ of CFTs at finite temperature which are holographically dual to exactly duality-invariant theories quadratic in $F_{\mu \nu}$.%, Eqs. \eqref{eq:dualgen} and \eqref{chiTheta}.

%Using the above bulk setup we will apply the AdS/CFT techniques to calculate retarded two-point current correlators $C_{ab}$ of boundary CFTs (at finite temperature) which are holographically dual to the duality-invariant theories \eqref{eq:dualgen}.

In a generic three-dimensional QFT with a current $J_a$ $(a=t,x,y)$, the retarded current-current correlator in momentum space $p^a=(\omega, \mathbf{k})$ is given by
\begin{equation}\label{Cab}
C_{ab}(p)=-i \int d^{3} x \, e^{-i p_a\, x^a} \Theta_H(t) \langle[J_a(x), J_b(0)] \rangle\,, 
\end{equation}
where $x^a=(t,x,y)$ are boundary coordinates and $\Theta_H(t)$ is the Heaviside step function.

Specifically, we will assume that the expectation values of all global conserved charges vanish in the equilibrium state, which is equivalent to exploring systems with no chemical potential \cite{Herzog:2007ij, Myers:2010pk}. In such a case, the correlator $C_{ab}$ can be derived holographically via studying linear perturbations $A_\mu$ which solve the classical equations of motion around a neutral black brane background \eqref{eq:bhbackground}. To this aim, we impose the gauge $A_u=0$ and decompose
the remaining non-vanishing components in momentum space:
%\footnote{This approach can also be applied whenever the corresponding chemical potential $\mu$ is sufficiently small; \emph{i.e.} $\ell_\ast \mu << 1$, $\ell_\ast$ being a length scale to be fixed by the particular duality \cite{Belin:2013uta,Belin:2014mva}.}
\begin{equation}\label{FA}
\begin{split}
 A_a &(t,u,\mathbf{x})=\int  \frac{\diff^3 p}{(2 \pi)^3}  \, e^{-i \omega t +i \mathbf{k} \cdot \mathbf{x} }A_a (u,\omega,\mathbf{k})\,.
 \end{split}
\end{equation}
 Working in momentum space and taking the spatial momentum vector to be $\mathbf{k}=(k,0)$, the equations of motion for $A_a$ in a duality-invariant theory given by Eqs. \eqref{eq:dualgen} and \eqref{chiTheta}  can be expressed in the following compact form\footnote{Expressed in this way, it might appear that the equations of motion are of third order in derivatives. However, they can be seen to be equivalent to the (second-order) equations $\nabla_\mu \left ( \chi^{\mu \nu \rho\sigma} F_{\rho \sigma} \right)=0$ in the gauge $A_u=0$ by appropriate manipulations which allow to integrate them into second-order equations, uniquely fixed after requiring $A_\mu=0$ to be a solution.}:
 %this is just a mere artifact of the compact notation employed: after manipulation of the equations, it can be seen that they are equivalent to the (second-order) equations $\nabla_\mu \left ( \chi^{\mu \nu \rho\sigma} F_{\rho \sigma} \right)=0$.  % can be integrated into the second-order equations  which are uniquely fixed after requiring $A_a=0$ to be a solution.}:
\begin{align}
\label{eq:symA}
 \mathcal{S} \mathbf{A}'' -\mathcal{S}'\mathbf{A}'+ \frac{\tilde{L}^4}{r_0^2} \mathcal{S}^2\left ( \omega^2 \mathcal{S}-\frac{k^2}{\mathcal{B}}\right)\mathbf{A}&=0\,, \\
 \omega\, \mathcal{S}\, \mathcal{B} A_t' +k A_x'&=0\,,
 \label{eq:symAxAt}
\end{align} 
\if{}
\begin{align}
\label{eq:sym1}
q^2 \mathcal{C} A_t+ q  \mathfrak{w} \mathcal{C} A_x - (\mathcal{B} A_t')'   &=0\,, \\  
  \label{eq:sym2}   \mathcal{B} A_t' +\frac{q}{\mathfrak{w}\mathcal{C}} A_x'&=0\, , \\ \label{eq:sym3}
   q \mathfrak{w}\mathcal{C} A_t + \mathfrak{w}^2 \mathcal{C} A_x
+\left (\frac{A_x'}{\mathcal{C}}\right )'&=0\,, \\ \label{eq:sym4}
     A_y \left (  \mathfrak{w}^2 \mathcal{C}-\frac{q^2}{\mathcal{B}}\right )+ \left (\frac{A_y'}{\mathcal{C}}\right )'&=0\,,
\end{align} 
\fi
where prime denotes derivative with respect to $u$, $\mathbf{A}=(\mathcal{B}A_t',A_y)$ and where $\mathcal{B}$ and $\mathcal{S}$ are identified after evaluation of $\mathcal{T}_{\mu \nu}$ on the black brane background \eqref{eq:bhbackground} as follows:
%are parts of components of \sout{are obtained by comparison with} the most general symmetric and traceless two-tensor $\mathcal{T}_{\mu \nu} \vert_{N,f}$ built from the curvature of \eqref{eq:bhbackground} and its covariant
% derivatives (see appendix):
\begin{align}
\label{eq:mostgendualinv}
  \hspace{-0.24cm}\left.  \mathcal{T}_{\mu}{}^{\nu}  \right \vert_{N,f}&=2(\theta+\varphi)\delta_\mu{}^t \delta_t{}^\nu+2(\theta-\varphi)\delta_\mu{}^u \delta_u{}^\nu-\theta\tensor{\delta}{_\mu^\nu}\,, \\ \theta&=\frac{N^2\mathcal{B}^2-1}{2N\mathcal{B}}\, , \quad \varphi=\frac{f^2 N^2 \mathcal{S}^2-1}{2fN \mathcal{S}}\,,
\end{align}
where we used that \eqref{eq:mostgendualinv} represents the most general form for a symmetric and traceless tensor built from the curvature of \eqref{eq:bhbackground} and its covariant derivatives (see appendix). The reason for the equations of motion of $A_\mu$ to take such a compact form is due to duality invariance \cite{Herzog:2007ij}.

Now, applying the holographic prescriptions originally presented in \cite{Son:2002sd,Policastro:2002se}, it is explained in the appendix that $C_{ab}$ can be obtained as follows:
\begin{equation}
    C_{ab}=-\left. \frac{4 r_0 \kappa_N}{ \tilde{L}^2}M'_{ab}\right \vert_{u=0}\,,
    \label{eq:corr0}
\end{equation}
where $M_{ab}$ is defined by\footnote{Boundary indices are lowered and raised with the Minkowski metric $\eta^{ab}$.} the relation $A_a=\tensor{M}{_{a}^{b}} A_b (0)$, with $A_b(0)=A_b \vert_{u=0}$. We impose infalling boundary conditions at the horizon for all components of $A_a$. This implies, on account of Eqs. \eqref{eq:symA} and \eqref{eq:symAxAt}, that 
both $\mathcal{B}A_t'$ and $A_x'/\mathcal{S}$ are proportional to $A_y$. Then, transforming these equations into an explicit second-order differential system for $A_a$ (see Footnote \cite{Note2}) and adapting the computations presented in \cite{Herzog:2007ij}, one may identify $M'_{ab}|_{u=0}$ and obtain the non-vanishing components of $C_{ab}$ from \eqref{eq:corr0}:
\begin{align}
\label{eq:clong}
    \frac{C_{tt}}{k^2}&=\frac{C_{xx}}{\omega^2}=-\frac{C_{tx}}{k \omega}=-\frac{C_{xt}}{k \omega}=\frac{4\tilde{L}^2 \kappa_N }{ r_0} \frac{A_y(0)}{A_y'(0)}\,,  \\
   C_{yy}&=-\frac{4 r_0 \kappa_N }{ \tilde{L}^2} \frac{A_y'(0)}{A_y(0)}\,. \label{eq:cyy}
\end{align}
 Having at our disposal the correlator $C_{ab}$, one may compute the so-called  longitudinal and transverse self-energies $K^L(\omega,\mathbf k)$ and $K^T(\omega,\mathbf k)$, defined as \cite{Herzog:2007ij}
\begin{equation}\label{KTL}
C_{xx}=-\frac{\omega^2}{\sqrt{k^2-\omega^2}} K^L\, ,\quad C_{yy}=\sqrt{k^2-\omega^2} K^T\,  .
\end{equation}
Comparing \eqref{eq:clong} with \eqref{eq:cyy}, we find that the product of $K^L$ and $K^T$ is the following universal constant for all frequencies and momenta:
\begin{equation}
K^L (\omega,\mathbf k) K^T(\omega,\mathbf k)=16 \kappa_N^2\,.
\label{eq:univresse}
\end{equation}
The above relation holds for all CFTs holographic to duality-invariant theories, since its derivation just requires to know their form up to quadratic order in the vector field, captured by the theories defined by Eqs. \eqref{eq:dualgen} and \eqref{chiTheta}. It matches\footnote{Up to a constant on the right-hand side of \eqref{eq:univresse}, which corresponds to the different conventions used for the electromagnetic duality transformation.} with the result obtained in \cite{Myers:2010pk} in the particular case of a background \eqref{eq:bhbackground} with $N=1$ after requiring duality invariance.


%presented in \cite{Myers:2010pk} for generic theories  \eqref{eq:dualgen} evaluated on the particular ansatz \eqref{eq:bhbackground} with $N=1$, after imposing the requisite of duality-invariance.

%Such a relation, characteristic of CFTs invariant under particle vortex-duality, was observed for Einstein-Maxwell theory in \cite{Herzog:2007ij} and generalized for theories of the form \eqref{eq:dualgen} in \cite{Myers:2010pk}. Imposing the theory to be self-dual, the results 


%, claiming that it was a consequence of electric-magnetic duality invariance. %With \eqref{eq:univresse}, we prove that 
%this is indeed the case. % {\color{red} for a generic duality-invariant theory}. %On a different front, note that \eqref{eq:univresse} is reminiscent of self-dual systems under 

%being a universal result valid for every holographic CFT associated to a duality-invariant theory. 

% captured by the theories \eqref{eq:dualgen}
%Since we are dealing with CFTs endowed with a current operator, it is natural to study the associated conductivity. We will be interested in the transverse and longitudinal conductivities $\sigma_y$ and $\sigma_x$. According to the Kubo formula \cite{kubo1957statistical}, they are given by
\textbf{Conductivity of holographic duality-invariant theories.} Following the usual holographic prescriptions, a gauge vector field on the bulk couples to a current on the boundary CFT. We are interested in studying the subsequent longitudinal and transverse conductivities $\sigma_x$ and $\sigma_y$ for the holographic theories defined by Eqs. \eqref{eq:dualgen} and \eqref{chiTheta}. According to the Kubo formula \cite{kubo1957statistical}, they are given by 
\begin{equation}
\label{sigmaTL}
\sigma_j (\omega,k)=-\mathrm{Im}\left ( \frac{C_{jj}}{\omega} \right)\,,\quad
j={x,y}\,.
\end{equation}
Particularly simple is the computation of  the conductivities at zero momentum $k=0$. In this case, spatial rotational invariance ensures that $K^L(\omega,0)=K^T(\omega,0)$ and $\sigma_x(\omega,0)=\sigma_y(\omega,0)$. Using \eqref{eq:univresse} and the expression for $C_{yy}$ given in  \eqref{eq:cyy}, one obtains\footnote{After choosing the sign for $K^L(\omega,0)=K^T(\omega,0)$ that guarantees a positive spectral function, defined by $-2\,  \mathrm{Im}(\sqrt{k^2-\omega^2} K^T)$. The same result could also be derived by direct resolution of \eqref{eq:symA}  for $k=0$.}
\begin{equation}
\sigma_x(\omega,0)=\sigma_y(\omega,0)=4 \kappa_N\,.
\label{eq:condzm}
\end{equation}
Therefore, the conductivity at zero momentum in any CFT holographic to a duality-invariant theory is a universal constant, independent of the frequency. Evidently, this is a consequence of the universal relation \eqref{eq:univresse}, as remarked in \cite{Herzog:2007ij,Myers:2010pk}. If duality symmetry is absent, Eq. \eqref{eq:condzm} may not necessarily hold --- see \cite{Myers:2010pk}.

%This was first noticed for Einstein-Maxwell theory in \cite{Herzog:2007ij} and now we have checked that it holds for every CFT holographic to a generic duality-invariant theory. 
%In particular, this implies that the membrane paradigm \cite{Kovtun:2003wp,Brigante:2007nu} provides the correct value for the conductivity at zero momentum and for any frequency (see appendix), which is not generically the case \cite{Iqbal:2008by}.


%Interestingly enough, this implies that $\sigma(\omega,0)$ coincides with the value $\sigma_{\mathrm{mb}}$ provided for the conductivity, for $\omega=k=0$, by the membrane paradigm \cite{Kovtun:2003wp,Brigante:2007nu,Iqbal:2008by}. %This is briefly reviewed in the appendix, where we also show that the membrane paradigm predicts the following value for the diffusion constant $D_c$:
%\begin{equation}
%D_c=\frac{\tilde{L}^2}{r_0} u_h\,,
%\label{eq:dc}
%\end{equation}
%so that a Nernst-Einstein relation $\sigma(\omega,0)=\chi_0 D_c$ with  $\chi_0=\frac{r_0}{4 \pi G \tilde{L}^2 u_h} $ holds. Result \eqref{eq:eq:dc} will be verified afterwards, when we will justify that $\chi_0$ is actually the charge susceptibility.

For non-zero momentum $k$, the longitudinal and transverse conductivities $\sigma_x$ and $\sigma_y$ are no longer the same and possess an explicit frequency dependence, as already observed in Einstein-Maxwell theory \cite{Herzog:2007ij}. Although an exact analytical expression for the conductivities at any frequency and momentum in an arbitrary duality-invariant theory seems currently out of reach (it remains elusive even in the Einstein-Maxwell case), it is in fact possible to obtain explicit results in certain limits. For large frequencies, straightforward application of the WKB approximation shows that
\begin{align}
\label{eq:lclfy}
\sigma_x(\omega,k)&=4 \kappa_N \frac{\omega}{\sqrt{\omega^2-k^2}}\, , \quad \omega^2>>k^2 \,, \\
\sigma_y(\omega,k)&=4 \kappa_N \frac{\sqrt{\omega^2-k^2}}{\omega}\, , \quad \omega^2>>k^2 \,.
\label{eq:lclf}
\end{align} 
Therefore, the behaviour of conductivities for large frequencies is theory-independent. Besides, we note that they tend to the universal value \eqref{eq:condzm} as $\omega \rightarrow \infty$. On the other hand, in the limit of sufficiently small frequencies and momenta $\omega,k << r_0/\tilde{L}^2$, one may generalize the results in \cite{Policastro:2002se} for the retarded correlators to obtain
\begin{align}
\label{eq:lcsfy}
\sigma_x(\omega,k)&= \frac{4\kappa_N \omega^2}{\omega^2+ D^2 k^4}\,, \quad \omega,k << \frac{r_0}{\tilde{L}^2}  \\
\sigma_y(\omega,k)&= \frac{4 \kappa_N}{1+c\, k^2} \,, \quad \quad \, \, \omega,k << \frac{r_0}{\tilde{L}^2} \,,
\label{eq:lcsf}
\end{align}
where we have implicitly defined
\begin{align}
\label{eq:intdc}
D=\frac{\tilde{L}^2}{r_0}\int_0^{u_h} \frac{\mathrm{d}z}{\mathcal{B}(z)}\,,\quad  c=\frac{2\tilde{L}^4}{r_0^2}\int_0^{u_h}\mathrm{d}z \, \mathcal{S}(z) \int_z^{u_h} \frac{\mathrm{d}w}{\mathcal{B}(w)}   \,.
\end{align}%(since they do in the limit of small frequencies and momenta)
These expressions show that the conductivities for non-zero momenta will generically depend on both the gravitational background and the particular choice of duality-invariant theory (since this is the case for small frequencies and momenta). Also, a closer look at Eqs. \eqref{eq:lclfy} and \eqref{eq:lcsfy} reveals that the longitudinal conductivity undergoes a hydrodynamic-to-collisionless crossover\footnote{This hydrodynamic-to-collisionless crossover manifests in the longitudinal conductivity and not in the transverse one since the longitudinal correlator $C_{xx}$ is related via Eq. \eqref{eq:clong} to the density-density correlator $C_{tt}$, which is a clear probe of hydrodynamic behaviour \cite{Herzog:2007ij}.} as we go from small to large frequencies. This is signaled by the fact that Eq.  \eqref{eq:lcsfy} possesses a pole at $\omega=-i D k^2$ (which is precisely the dispersion relation of diffusion modes in the heat equation), while \eqref{eq:lclfy} presents a pole at $\omega=k$ (which is the dispersion relation for free particles). Moreover, as a consistency check of our results, we have verified that the expression for the diffusion constant that can be derived from the membrane paradigm  \cite{Kovtun:2003wp,Brigante:2007nu,Iqbal:2008by} coincides with our formula \eqref{eq:intdc} for $D$.





%Nevertheless, expressions \eqref{eq:lcsfy} and \eqref{eq:lcsf} provide an analytical argument to explain why the effects of varying $\lambda$ are so small: with respect to the evaluations on the GR solution $f=1-u^3$, $\mathcal{I}(v)=0$ and $\lambda=0$, the corrections appear at order $\mathcal{O}(L^4)$.

%These expressions reveal that the behaviour of conductivities in this regime is not universal but allows one to derive that, for the model \eqref{eq:dualchoice} we are considering, both transverse and longitudinal conductivities will increase at small frequencies and momentum as $\lambda$ diminishes. 


Away from the small/large frequency regimes, it appears to be challenging to obtain specific formulae for the conductivities. Despite that, a universal statement regarding their form in generic duality-invariant theories can be made by noticing that every traceless and symmetric tensor $\mathcal{T}_{\mu \nu}$ built from algebraic combinations (\emph{i.e.} with no covariant derivatives) of the curvature of \eqref{eq:bhbackground} vanishes  identically when evaluated on the GR AdS black brane solution:
\begin{equation}
\mathcal{T}_{\mu \nu}\vert_{N=1, f=1-u^3}=0\,.
\label{eq:tmunueins}
\end{equation}
The explicit proof of this result is given in the appendix. Therefore, the retarded correlators $C_{ab}$ and the conductivities $\sigma_x(\omega,k)$ and $\sigma_y(\omega,k)$  will coincide with those of Einstein-Maxwell theory for any duality-invariant theory with no covariant derivatives of the curvature in $\mathcal{T}_{\mu \nu}$ as long as the GR AdS black brane background is considered. It is important to note that taking the background to be that of GR does not imply that $\mathcal{L}_{\mathrm{grav}}^{\mathrm{high}}=0$ in \eqref{eq:dualgen}. Indeed, there exist myriads of higher-order gravities which do not correct the GR AdS black brane solution (\emph{e.g.} the well-known $f(R)$ gravities \cite{Buchdahl:1970ynr,Sotiriou:2008rp}). Therefore, one may interpret duality invariance as a very powerful tool to constrain observables to have a simple and fixed expression: that of Einstein-Maxwell theory.




%If higher-curvature terms correct the GR solution, the retarded correlator, and hence the associated conductivities, will generically differ\footnote{If one insists on working with the GR solution, the lowest-order choice for $\mathcal{T}_{\mu \nu}$ which is nonzero on the AdS black brane solution of GR is the traceless part of $\nabla_\mu W_{\alpha \beta \rho \sigma} \nabla_\nu W^{\alpha \beta \rho\sigma}$, where $W_{\alpha \beta \rho \sigma}$ stands for the Weyl tensor. This term would in principle modify the conductivity of the dual CFT. Nevertheless, such a term is of order $\mathcal{O}(L^8)$ and there are other higher-curvature terms of less order that already modify the gravitational background.} from those of Einstein-Maxwell theory. In such a case, the subsequent charge transport will no longer be independent of the choice of duality-invariant theory. However, in spite of this lack of universality, it turns out that duality-invariance forces corrections with respect to the case $\mathcal{T}_{\mu \nu}=0$ to be highly suppressed, since Eq. \eqref{eq:tmunueins} implies that $\mathcal{T}_{\mu \nu}\vert_{N,f}$ is subleading with respect to the leading-order corrections in the gravitational background (leaving aside the subtleties raised in Footnote \cite{Note7}). Hence we conclude that corrections associated to the specific choice of $\mathcal{T}_{\mu \nu}$ are subleading with respect to those arising from the choice of the gravitational background (or equivalently, of $\mathcal{L}^{\mathrm{high}}_{\mathrm{grav}}$).
If higher-curvature terms correct the GR solution, the retarded correlator, and hence the associated conductivities, will generically differ from\footnote{If one insists on working with the GR solution, the lowest-order choice for $\mathcal{T}_{\mu \nu}$ which is nonzero on the AdS black brane solution of GR is the traceless part of $\nabla_\mu W_{\alpha \beta \rho \sigma} \nabla_\nu W^{\alpha \beta \rho\sigma}$, where $W_{\alpha \beta \rho \sigma}$ stands for the Weyl tensor. This term would in principle modify the conductivities of the dual CFT. Nevertheless, such a term is of order $\mathcal{O}(L^8)$ and there are other higher-curvature terms of less order that already modify the gravitational background.} those of Einstein-Maxwell theory. In such a case, the subsequent charge transport will no longer be independent of the choice of duality-invariant theory. However, in spite of this lack of universality, if $\mathcal{T}_{\mu \nu}$ contains no covariant derivatives of the curvature it turns out that duality invariance forces corrections with respect to the case $\mathcal{T}_{\mu \nu}=0$ to be highly suppressed, since Eq. \eqref{eq:tmunueins} implies that $\mathcal{T}_{\mu \nu}\vert_{N,f}$ is subleading with respect to the leading-order corrections in the gravitational background. Therefore, corrections associated to the specific choice of $\mathcal{T}_{\mu \nu}$ (without covariant derivatives of the curvature) are subleading with respect to those arising from the choice of the gravitational background (or equivalently, of $\mathcal{L}^{\mathrm{high}}_{\mathrm{grav}}$).
\begin{figure}[h]
\centering
\includegraphics[scale=0.36,trim={0.35cm 0 0 0},clip]{condlongknuevo.pdf}
\includegraphics[scale=0.36,trim={0.35cm 0 0 0},clip]{condtransknuevo.pdf}
\caption{Longitudinal (above) and transverse (below) conductivities in units of $L^2/r_0=1$ for Einstein-Maxwell ($E$-$M$) theory and for an ECG background. We have picked $\mu=1/10$, $k=1$  and several values of $\lambda$.}
\label{fig:cond}
\end{figure}


%However,  we observe that corrections associated to the specific choice of duality-invariant theory --- \emph{i.e.} of $\mathcal{T}_{\mu \nu}$ --- are subleading with respect to those arising from the choice of gravitational background --- \emph{i.e.} of $\mathcal{L}_{\mathrm{high}}^{\mathrm{grav}}$. 

%his can be seen from \eqref{eq:tmunueins}, since it implies that $\mathcal{T}_{\mu \nu} \vert_{N,f}$ will be of order $\mathcal{O}(L^6)$ on an ECG background.


\textbf{Holographic conductivities in an explicit example.}
Now we illustrate the previous aspects with the simplest non-trivial choices for $\mathcal{T}_{\mu \nu}$ and $\mathcal{L}_{\mathrm{grav}}^{\mathrm{high}}$. Regarding $\mathcal{T}_{\mu \nu}$, this corresponds to
\begin{equation}
\mathcal{T}_{\mu \nu}=\lambda L^2 \hat{R}_{\mu \nu}\,,
\label{eq:dualchoice}
\end{equation}
where $\hat{R}_{\mu \nu}$ denotes the traceless part of the Ricci tensor and $\lambda$ is a dimensionless coupling. Demanding $\mathcal{T}_{\mu \nu}$ to respect the Weak Gravity Conjecture \cite{Arkani-Hamed:2006emk} and causality \cite{Myers:2010pk}, we find that the acceptable range for $\lambda$ is $0 \geq \lambda \gtrsim -0.50105$ (see appendix) for the specific choice of $\mathcal{L}_{\mathrm{grav}}^{\mathrm{high}}$ we are about to make.%{Better to place this after WGC, since the analysis of $\lambda$ is for $\mu=1/10$}.

To pick a suitable $\mathcal{L}_{\mathrm{grav}}^{\mathrm{high}}$ one needs to consider gravitational theories which contain at least terms of cubic order in the curvature, since quadratic terms do not correct the (four-dimensional) GR AdS black brane solution. Among this class of theories, there is a unique subset admitting black brane solutions \eqref{eq:bhbackground} with $N(u)=1$ and second-order equation for $f$. All such theories are equivalent on ans\"atze of the form \eqref{eq:bhbackground}, so it is enough to select a convenient representative. We will choose it to be Einsteinian Cubic Gravity (ECG) \cite{Bueno:2016xff,Bueno:2019ltp}, whose higher order terms have the following form:
\begin{align}
\nonumber
\hspace{-0.1cm}-8 \mathcal{L}_{\mathrm{grav}}^{\mathrm{high}}&=\mu L^4 \left[ 12 R_{\mu}{}^\rho{}_\nu{}^\sigma R_\rho{}^\gamma{}_\sigma{}^\delta R_\gamma{}^\mu{}_\delta{}^\nu+8 R_{\mu \nu} R^{\nu \rho} R_{\rho}{}^\mu\right.\\ &\left. +R_{\mu \nu}{}^{\rho \sigma} R_{\rho \sigma}{}^{\gamma \delta} R_{\gamma \delta}{}^{\mu \nu}-12 R_{\mu \nu \rho \sigma} R^{\mu \rho} R^{\nu \sigma}  \right]\,,
\label{eq:ecg}
\end{align} 
where $\mu$ is a dimensionless coupling. The equation of motion for $f(u)$, though second-order, is too complicated to be solved analytically for generic $\mu$, so we will resort to numeric methods (details are given in the appendix). We will pick $\mu$ to be within the range $0 < \mu < 4/27$, since this ensures the existence of both a unique stable vacuum and black hole solutions \cite{Bueno:2018xqc}. %To the best of our knowledge, this will represent the first computation ever of higher-curvature corrections to the holographic conductivity.

 %The value of \eqref{eq:dualchoice} for the higher-order corrected black brane solution is of order
% $\mathcal{O}({L}^6)$.

%\textcolor{blue}{Mention that Hall effect does not appear since we are working on top of neutral gravitational background.}.

%First, we observe that corrections associated to the specific choice of duality-invariant theory --- \emph{i.e.} of $\lambda$ --- are clearly subleading with respect to those arising from the choice of gravitational background. This can be seen from \eqref{eq:tmunueins}, since it implies that $\mathcal{T}_{\mu \nu} \vert_{N,f}$ will be of order $\mathcal{O}(L^6)$ on an ECG background.
%higher order corrections to the gravitational background on top of which we consider the vector gauge field perturbations result in significant deviations from the Einstein-Maxwell conductivities, as can be seen from Fig. \ref{fig:cond}. % most notorious corrections with respect to the Einstein-Maxwell curves come from the modification of the gravitational background on top of which the vector gauge field perturbation is examined, being the corrections from \eqref{eq:dualchoice} not as relevant for the allowed values of $\lambda$. We have observed that this behaviour is not due to the particular value of $\mu$ chosen \textcolor{blue}{Check}, but rather it happens for generic allowed $\mu$ \textcolor{blue}{Improve physical justification}. 



%{\textcolor{blue}{Using the Nernst-Einstein relation $\sigma=\chi_c D$ to compute the charge susceptibility $\chi_c$, it is evident that it will not have a universal value either, being theory-dependent. }} 

%the charge susceptibility is given by $\chi_0=\frac{r_0}{4 \pi G \tilde{L}^2 u_h}$. Therefore, having computed independently (all at zero momentum) the conductivity, the diffusion constant (both from comparison with hydrodynamic limit and through the membrane paradigm) and the charge susceptibility, we prove that the Nernst-Einstein equation $\sigma(\omega,0)=\chi_0 D_c$ exactly holds. This justifies the appearance of the prefactor $\chi_0 D_c$ in expressions \eqref{eq:lcsfy}, \eqref{eq:lcsf}, \eqref{eq:lclfy} and \eqref{eq:lclf}, a feature that was argued to be due to duality-invariance in \cite{Herzog:2007ij} and which we rigorously prove here.  \, \\

%\begin{widetext} 
%Leave a blank space. Otherwise, the widetext for the images does not work %graf1b %graf3bb



In Fig. \ref{fig:cond} we present the longitudinal and transverse conductivities we get for Einstein-Maxwell theory and for the choices \eqref{eq:dualchoice} and \eqref{eq:ecg}. We have  set $\mu=1/10$, $L^2 k/r_0=1$ and $\lambda=0,-1/2,-1/4$, since the qualitative behaviour of the conductivities turns out to replicate for any $0 < \mu <4/27$ (approaching of course the Einstein-Maxwell case as $\mu \to 0$) and $k$ (approaching the constant universal value \eqref{eq:condzm} as $k \rightarrow 0$). By direct inspection of the graphs we check that corrections associated to the specific choice of $\lambda$ --- \emph{i.e.} of $\mathcal{T}_{\mu \nu}$ --- are clearly subleading with respect to those arising from the choice of the gravitational background (characterized, in this case, by the parameter $\mu$).
%\footnote{In the limit of small frequencies, the conductivities of Fig. \ref{fig:cond} do not fit very well to Eqs. \eqref{eq:lcsfy} and \eqref{eq:lcsf}. This was to be expected, since those expressions are only valid in the limit of small momentum. Choosing a sufficiently small $k$ (\emph{e.g.} $k \lesssim 0.2 \, r_0/L^2$), we have checked independently that the numerical conductivities agree with \eqref{eq:lcsfy} and \eqref{eq:lcsf} for small frequencies.}
%In particular, we have verified that the conductivities in Fig. \ref{fig:cond} are in full agreement with 
%Eqs. \eqref{eq:lclfy} and \eqref{eq:lclf} in the limit of large frequencies

%we have explicitly checked that the longitudinal and transverse self-energies satisfy the well-known universal relation \eqref{eq:univresse} for all frequencies and momenta

%, as already foreseen \cite{Myers:2010pk}

\textbf{Final comments.} We have examined various universal aspects of the holographic quantum critical transport associated to duality-invariant theories. In the first place, we have explicitly checked that the conductivity at zero momentum is a universal constant for all these theories. Next we have obtained the expressions for the conductivities in the limit of large frequencies and for small frequencies and momenta in every CFT holographic to a duality-invariant theory. From their form in this latter regime, we have concluded that conductivities at non-zero momentum generically depend on both the gravitational background and the theory under study.

%We observed that the latter regime depends on both the gravitational background and the theory under consideration, so that conductivities will be generically distinct to those of Einstein-Maxwell c

Despite that, we have proven that the conductivities in CFTs associated to duality-invariant theories which do not couple covariant derivatives of the curvature to gauge field strengths display two universal features. First, we have shown that, as long as a GR background is chosen, conductivities are universal and equal to those of Einstein-Maxwell theory for any frequency and momentum. Secondly, when the gravitational background is corrected by higher-curvature terms, we have proven that conductivities get modified in such a way that contributions from nonminimal couplings of the gauge field to gravity are subleading.




%in more generality, we have demonstrated that corrections to the conductivities with respect to the case of Maxwell theory minimally coupled to a higher-curvature are extremely suppressed.

%thus reinforcing the universality character of holographic Einstein-Maxwell theory.

%Specifically,
%after verifying that the longitudinal and transverse self-energies satisfy the universal relation \eqref{eq:univresse} for all frequencies and momenta, so that the conductivity at zero momentum is a universal constant for all frequencies, 




%we have checked that the longitudinal and transverse self-energies satisfy the universal relation \eqref{eq:univresse} for all frequencies and momenta, that the conductivity at zero momentum is a universal constant for all frequencies and that, for duality-invariant theories which do not couple gauge field strengths with covariant derivatives of the curvature, the corrections of the conductivities with respect to their value for Einstein-Maxwell theory either vanish (when a GR background is chosen) or are rather suppressed. Therefore, duality-invariance forces holographic charge transport not to deviate significantly from the behaviour predicted by Einstein-Maxwell theory, which thus acquires an even higher degree of universality.
%, reinforcing the universality character of the latter.




In another vein, there are several directions that would be interesting to address. Firstly, one could study other correlators of CFTs holographic to duality-invariant theories. For instance,  consider the Euclidean correlators $\langle J_a J_b \rangle_E$ and $\langle T_{ab} J_c J_d \rangle_E$ at zero temperature, where $T_{ab}$ is the stress-energy tensor. Conformal symmetry fixes the form of such correlators as follows \cite{Osborn:1993cr,Erdmenger:1996yc}: 
\begingroup
\begin{align}
   \langle J_a (x_1) J_b(x_2) \rangle_E&=\frac{C_J}{\vert x_{12} \vert^{4}} \mathcal{I}_{ab}\,, \\
   \langle T_{ab} (x_1) J_c (x_2) J_d(x_3) \rangle_E &=\frac{f_{abcd}(C_J,a_2)}{\vert x_{12} \vert^3 \vert x_{13} \vert^3 \vert x_{23} \vert}\,,
\end{align}
\endgroup
where $\mathcal{I}_{ab}$ and $f_{abcd}(C_J,a_2)$ are fixed tensorial structures, $x_{mn} =x_m-x_n$, $C_J$ is the current central charge and $a_2$ is a parameter that controls, together with $C_J$, the energy flux measured at infinity after the insertion of a current operator \cite{Hofman:2008ar}. The holographic expressions for $C_J$ and $a_2$ for the most general effective four-derivative theory were presented in \cite{Hofman:2008ar,Cano:2022ord}. Applying their results to the choice \eqref{eq:dualchoice} and $\mathcal{L}_{\mathrm{grav}}^{\mathrm{high}}=0$, one finds $a_2=0$ and that $C_J$ takes its Einstein-Maxwell value, so the Euclidean correlators at zero temperature are not modified to the fourth-derivative order.   This is another manifestation of the strength of duality invariance to constrain the form of correlators to be those of Einstein-Maxwell.

%Oe may ask whether this phenomenon replicates for different correlators. 

%where $\mathcal{I}_{ab}$ and $f_{abcd}(C_J,a_2)$ are fixed tensorial structures, $x_{mn} =x_m-x_n$, $C_J$ is the current central charge and $a_2$ is a parameter that controls, together with $C_J$, the energy flux measured at infinity after the insertion of a current operator \cite{Hofman:2008ar}. The holographic expressions for $C_J$ and $a_2$ for the most general effective four-derivative theory were presented in \cite{Hofman:2008ar,Cano:2022ord}. Applying their results to the choice \eqref{eq:dualchoice} and $\mathcal{L}_{\mathrm{grav}}^{\mathrm{high}}=0$, one finds that $C_J$ takes its Einstein-Maxwell value and $a_2=0$, so the Euclidean correlators at zero temperature would not be modified up to the fourth-derivative order.   This is another manifestation of the strength of duality-invariance to constrain the form of correlators to be those of Einstein-Maxwell.





%This is an extremely challenging task, since knowledge of all duality-invariant theories of 

%For instance, if attention is drawn to four-point correlators, 

%it might be of interest to characterize higher-point correlators. To this aim, one would need to classify all duality-invariant theories containing up to quartic terms in $F_{\mu \nu}$, which still remains an open problem. Although not an easy task at all, it is tantalizing to examine whether duality-invariance may simplify enough the problem so as to be able to infer generic properties of higher-point correlators.


%To compute, e.g. four-point correlators, it will be necessary to classify all duality-invariant theories of electrodynamics non-minimally coupled to gravity up to quartic terms in $F_{\mu \nu}$, which is still an open problem. Although not an easy task to complete, it is tantalizing to examine whether duality invariance may simplify the problem enough to derive generic properties of higher-point correlators.
% (or equivalently with non-vanishing expectation values of global conserved charges in the equilibrium state)
Secondly, it would be intriguing to extend our results to systems with chemical potentials (\emph{i.e.} with non-vanishing expectation values of global conserved charges in the equilibrium state). This is carried out by considering linear fluctuations of the vector field on top of a fixed charged gravitational background with a non-zero background electromagnetic field, as in \cite{Hartnoll:2007ai,Hartnoll:2007ih,Goldstein:2009cv,Blake:2014yla,Guo:2017bru,Wang:2018hwg,Kiczek:2020ngg}. 

Finally, one could also examine higher-point correlators. Indeed, it is natural to wonder what constraints duality invariance could impose on generic (current) $n$-point correlators. This would require the construction of (all) duality-invariant nonminimal extensions of Einstein-Maxwell theory of arbitrary order in $F_{\mu \nu}$, which remains as an outstanding open problem. 
 


%$n$-point correlators for $n>2$. 

%higher-point correlators could be examined


%. Indeed, it is tantalizing to examine which conditions duality-invariance may impose on the subsequent (current) $n$-point correlators. To this aim, knowledge of all exactly duality-invariant theories of order $n$ in $F_{\mu \nu}$ is required, which remains as an outstanding open problem as of today. 

%Finally, one could also wonder about the exact meaning of duality-invariance for the subsequent holographic CFTs. It is known to be related to particle-vortex duality \cite{Witten:2003ya,Herzog:2007ij,Myers:2010pk}, but the explicit construction of generic holographic duality-invariant CFTs away from Einstein-Maxwell theory is still {\color{red} to be accomplished}. Further examination of their bulk counterparts may help in this regard. 


%Finally, it is important to note that in this paper we considered systems without chemical potentials, corresponding to the vanishing of the expectation values of global conserved charges in the equilibrium state. This allowed us to fix a neutral gravitational background on top of which we computed the linearized equations of motion for the vector field and, subsequently, the transport properties of the holographic CFT. It will be interesting to extend these results to the case of a charged gravitational background and a non-zero gauge vector field, as e.g. in \cite{Hartnoll:2007ai,Hartnoll:2007ih,Goldstein:2009cv,Blake:2014yla,Guo:2017bru,Wang:2018hwg,Kiczek:2020ngg}. 


%%%%%%%%%%%%%Secondly, it could be of interest to study higher-point correlators of holographic duality-invariant theories. In the present document, it was possible to obtain general results valid for all duality-invariant theories since we analyzed current two-point correlators, which just require the retain terms quadratic in the Maxwell field strength in the action. For higher-point correlators, such as the four-point one, it would be necessary to classify all duality-invariant theories with up to quartic terms on $F_{\mu \nu}$, which has remained so far elusive. Although not an easy task at all, it is tantalizing to examine whether duality-invariance may simplify the problem enough so as to derive generic properties of higher-point correlators.

%Finally, it is important to remark a crucial assumption for our results: we considered systems without chemical potentials, corresponding to the vanishing of the expectation values of global conserved charges in the equilibrium state. This allowed to fix a neutral gravitational background on top of which we computed the linearized equations of motion for $A_\mu$ and, subsequently, the transport properties of the holographic CFT. Nevertheless, it would be intriguing to extend our results in the case we work over a charged gravitational background and a non-zero gauge vector field, for instance as in \cite{Hartnoll:2007ai,Hartnoll:2007ih,Goldstein:2009cv,Blake:2014yla,Guo:2017bru,Wang:2018hwg,Kiczek:2020ngg}. 

%Certainly, all these aspects deserve further research.
%%%%%%%%%%%%%
%%%%%%%%%%%%%
\vspace{1mm}
\noindent
\textbf{Acknowledgements}
\vspace{1mm}
%%%%%%%%%%%%%
%%%%%%%%%%%%%
%%%%%%%%%%%%%


%S_EE/\nu=a+2(c-a/3)(\mu R)^2 para U(1)
%S_EE/\nu=a+(2c-a)(\mu R)^2/6 para SU(2)
\noindent
We would like to thank Pablo A. Cano for very useful and enlightening discussions.
 Á. J. M. was supported by a postdoctoral fellowship from the INFN, Bando 23590. D.S. acknowledges partial support  of
%from the INFN Research Project “String Theory and Fundamental Interactions” (STEFI), 
the Spanish MICINN/FEDER grant PID2021-125700NB-C21 and the Basque Government Grant IT-979-16.




%Understand dual CFTs (for INTRODUCTION), More generic properties of duality invariant theories, the euclidean two and three-point correlators at T=0 are exactly the same for all duality-invariant theory... Comment on this.
%Another quantity of interest to derive within the membrane paradigm is that of the diffusion constant $D_c$. Such computation was carried out explicitly for Einstein-Maxwell theories \cite{Herzog:2002fn}, while for theories of the form \eqref{eq:dualgen} a general formula was derived \cite{Brigante:2007nu}. Nevertheless, for arbitrary duality-invariant theories \eqref{eq:dualgen} $D_c$ cannot be explicitly computed and it will generically depend on both the gravitational background and the choice of $\chi^{\mu \nu \rho \sigma}$. We will comment further on this afterwards.




%However, the charge susceptibility $\chi_c$ defimed by the Nernst-Einstein relation $\chi_c=\sigma_{\mathrm{mb}}/D_c$, where $D_c$ is the diffusion constant, cannot be computed analytically in general, since $D_c$ will depend on the 

%can be neither the same for all theory nor analytically computed in general, 



\onecolumngrid  \vspace{0.8cm} 
\begin{center}  
{\Large\bf Appendices} 
\end{center} 
\appendix 
%\tableofcontents

 \vspace{-0.25cm} 

\section{Traceless and symmetric two-tensors evaluated on black brane backgrounds}

Assume the following black brane ansatz for the metric:
\begin{equation}
    \diff s^2=\frac{r_0^2}{\tilde{L}^2 u^2}\left (-N^2(u)f(u) \diff t^2+\diff x^2+\diff y^2 \right )+\frac{\tilde{L}^2}{u^2 f(u)} \diff u^2\,,
    \label{eq:appbhbackground}
\end{equation}
for certain functions $N$ and $f$. We set the asymptotic (boundary) region to be at $u \rightarrow 0$, so that $\lim_{u \rightarrow 0}f=\lim_{u \rightarrow 0}N=1$, while the horizon (assuming it exists) is at $u=u_h$. Here we prove the following two features of traceless symmetric tensors $\mathcal{T}_{\mu \nu}$ evaluated on a black brane ansatz \eqref{eq:appbhbackground}:
\begin{itemize}
    \item Any $\mathcal{T}_{\mu \nu}$ built out from contractions of the curvature tensors and covariant derivatives thereof satisfies
    \begin{equation}
    \label{eq:approof1}
     \left. \mathcal{T}_{\mu}{}^{\nu}  \right \vert_{N,f}=2(\theta+\varphi)\delta_\mu{}^t \delta_t{}^\nu+2(\theta-\varphi)\delta_\mu{}^u \delta_u{}^\nu-\theta\tensor{\delta}{_\mu^\nu}\,,
    \end{equation}
    for certain functions $\theta$ and $\varphi$ of the coordinate $u$.% and where we defined the mutually orthogonal projectors $\tau_{\mu}{}^{\nu}=\tensor{\delta}{_\mu^t}\tensor{\delta}{_t^\nu}$, $\rho_{\mu}{}^{\nu}=\tensor{\delta}{_\mu^u}\tensor{\delta}{_u^\nu}$ and $\sigma_{\mu}{}^{\nu}=\tensor{\delta}{_\mu^x}\tensor{\delta}{_x^\nu}+\tensor{\delta}{_\mu^y}\tensor{\delta}{_y^\nu}$ onto the $t$, $u$ and $(x,y)$ directions, respectively.
    \item Any $\mathcal{T}_{\mu \nu}$ built out from algebraic contractions of the curvature tensors (without covariant derivatives) vanishes when evaluated on the GR black brane solution given by $N=1$ and $f=1-u^3$:
    \begin{equation}
    \left. \mathcal{T}_{\mu \nu} \right \vert_{N=1,f=1-u^3}=0\,.
    \label{eq:approof2}
\end{equation} 
\end{itemize}
Let us start by showing the first property. Using the Ricci decomposition of the Riemann tensor, $\mathcal{T}_{\mu \nu}$ may be entirely expressed in terms of the Ricci scalar $R$, the traceless part $\hat{R}_{\mu \nu}$ of the Ricci tensor and the Weyl tensor $W_{\mu \nu}{}^{\rho \sigma}$. If we define the mutually orthogonal projectors $\tau_{\mu}{}^{\nu}=\tensor{\delta}{_\mu^t}\tensor{\delta}{_t^\nu}$, $\rho_{\mu}{}^{\nu}=\tensor{\delta}{_\mu^u}\tensor{\delta}{_u^\nu}$ and $\sigma_{\mu}{}^{\nu}=\tensor{\delta}{_\mu^x}\tensor{\delta}{_x^\nu}+\tensor{\delta}{_\mu^y}\tensor{\delta}{_y^\nu}$ onto the $t$, $u$ and $(x,y)$ directions respectively, then the traceless Ricci tensor and the Weyl tensor on gravitational configurations \eqref{eq:appbhbackground} read
\begin{align}
\label{eq:wricapp}
\left.  W_{\mu \nu}{}^{\rho \sigma}\right \vert_{N,f}&=w \left [ \zeta_{[\mu}{}^{[\rho} \zeta_{\nu]}{}^{\sigma]} -\zeta_{[\mu}{}^{[\rho} \sigma_{\nu]}{}^{\sigma]} +\sigma_{[\mu}{}^{[\rho} \sigma_{\nu]}{}^{\sigma]}    \right]\,, \quad \left.\hat{R}_{\mu \nu}\right \vert_{N,f}=(X+Y) \tau_{\mu \nu}+
  (X-Y) \rho_{\mu \nu}-X \sigma_{\mu \nu}\,,\\
  \label{eq:wxy}
  w=-&\frac{u^2(3f'N'+Nf''+2 f N'')}{3 \tilde{L}^2 N}\,, \quad X=-\frac{u(3uf'N'+N(u f''-2 f')-2f(N'-u N'')}{4 \tilde{L}^2 N}\, , \quad Y=\frac{u f N'}{\tilde{L}^2 N}\,,
\end{align}
where $\zeta_{\mu \nu}=\tau_{\mu \nu}+\rho_{\mu \nu}$. If $\mathcal{T}_{\mu \nu}$ is constructed from algebraic combinations of curvature tensors (with no covariant derivatives), then \eqref{eq:approof1} follows by direct inspection of \eqref{eq:wricapp}, which ensures that any contraction of curvature tensors will be expressed in terms of the projectors $\tau_{\mu \nu}$, $\rho_{\mu \nu}$ and $\sigma_{\mu \nu}$. Now,  assume that $\mathcal{T}_{\mu \nu}$ contains covariant derivatives of the curvature. We observe that
\begin{align}
    \nabla_{\mu} \tau_{\eta \lambda}&=\left (\frac{2}{u}-\frac{f'}{f}-\frac{2N'}{N} \right) \tau_{\mu (\eta} \, \delta_{\lambda)}^u\, , \quad \nabla_\mu \rho_{\eta \lambda}=-\nabla_{\mu} \tau_{\eta \lambda}-\frac{2}{u} \sigma_{\mu (\eta} \, \delta_{\lambda)}^u\, , \quad \nabla_\mu \sigma_{\eta \lambda}=-\nabla_\mu \tau_{\eta \lambda}-\nabla_\mu \rho_{\eta \lambda}\,,
    \\ 
    \nabla_\mu \delta_\nu^{u}&=\frac{u}{2 \tilde{L}^2 N}(u Nf'-2f(N-uN')) \tau_{\mu \nu}+\frac{u(2f+u f')}{2 \tilde{L^2}}\rho_{\mu \nu}-\frac{u f}{\tilde{L}^2}\sigma_{\mu \nu}\,.
\end{align}
Consequently, it turns out that any contraction of curvature tensors and covariant derivatives thereof will be entirely expressed in terms of the projectors $\tau_{\mu \nu}$, $\rho_{\mu \nu}$, $\sigma_{\mu \nu}$ and in terms of $\delta_\rho^u$. Moreover, since each of the terms composing $\mathcal{T}_{\mu \nu}$ must possess an even number of covariant derivatives (otherwise, the number of indices does not match), we can always pair up tensors $\delta_\mu^u$ into projectors $\rho_{\mu \nu}$. Hence we deduce that
\begin{equation}
    \left. \mathcal{T}_{\mu \nu} \right \vert_{N,f}=F_1 \tau_{\mu \nu}+F_2 \rho_{\mu \nu}+F_3 \sigma_{\mu \nu}\,.
\end{equation}
Demanding $\left. \mathcal{T}_{\mu \nu} \right \vert_{N,f}$ to be traceless, we arrive at \eqref{eq:approof1}.

Regarding the proof of \eqref{eq:approof2}, observe that \eqref{eq:wricapp} and \eqref{eq:wxy} imply that $\left. \hat R_{\mu \nu}\right \vert_{N=1,f=1-u^3}=0$. Hence all potential non-trivial contributions to $\mathcal{T}_{\mu \nu}$ may only come from terms containing just Weyl tensors. However, from \eqref{eq:wricapp} we realize that the expression for the Weyl tensor is symmetric under the exchange of $\zeta_{\mu}{}^\nu$ and $\sigma_{\mu}{}^\nu$. Therefore, any contraction of Weyl tensors must be symmetric under this exchange, so that
\begin{equation}
    \mathcal{W}_{\mu \nu}^{(n)}=c_n ( \zeta_{\mu \nu}+\sigma_{\mu \nu})=c_n g_{\mu \nu}\,,
\end{equation}
where $\mathcal{W}_{\mu \nu}^{(n)}$ denotes any term formed by $n$ Weyl tensors. The traceless part of any $\mathcal{W}_{\mu \nu}^{(n)}$ is clearly zero and thus we arrive at \eqref{eq:approof2}. Note that \eqref{eq:approof2} is no longer true if $\mathcal{T}_{\mu \nu}$ possesses covariant derivatives of the curvature. An explicit counterexample is provided by the traceless part of $\nabla_\mu W_{\alpha \beta \eta \lambda} \nabla_\nu W^{\alpha \beta \eta \lambda}$, which is non-zero on the GR black brane solution $N=1$ and $f=1-u^3$.
%with an index of any other projector will have the structure $\sim \tau \, \delta^u +\sigma\, \delta^u$ (this is easily checked by a case-by-case study). Also, we observe that the contraction of any index of the covariant derivative of any projector with any index of another covariant derivative of any other projector will be entirely expressed in terms of $\tau_{\mu \nu}$, $\rho_{\mu \nu}$ and $\sigma_{\mu \nu}$.


\section{Holographic Prescription for Retarded Correlators}

Let us consider a generic duality-invariant theory with nonminimal couplings to gravity which is quadratic in the Maxwell field strength. According to \cite{Cano:2021hje}, it can be written as follows:
\begin{equation}
 I=\kappa_N \int \diff^4 x \sqrt{\vert g \vert} \left[R+\frac{6}{L^2}-\chi^{\mu \nu \rho \sigma} F_{\mu \nu} F_{\rho\sigma}+\mathcal{L}_{\mathrm{grav}}^{\mathrm{high}}\right]\,,
    \label{eq:appdualgen}
\end{equation}
where $\kappa_N=(16 \pi G)^{-1}$, $\mathcal{L}_{\mathrm{grav}}^{\mathrm{high}}$ stands for purely-gravitational higher-curvature terms and $\chi^{\mu \nu \rho \sigma}$ is given by
\begin{align}
\label{eq:appd1}
    \tensor{\chi}{_{\mu \nu}^{\rho \sigma}}&= \tensor{\delta}{_{[\mu}^{[\rho}} \tensor{\delta}{_{\nu]}^{\sigma]}}+\tensor{\Theta}{_{\mu \nu}^{\rho \sigma}}+\sum_{n=1}^\infty b_n \tensor{\Theta}{^{2n}_{\mu \nu}^{\rho \sigma}}\,,\\ \tensor{\Theta}{^{2n}_{\mu \nu}^{\rho \sigma}}&=\tensor{\Theta}{_{\mu \nu}^{\alpha_1 \alpha_2}}\tensor{\Theta}{_{\alpha_1 \alpha_2}^{\alpha_3 \alpha_4}} \cdots \tensor{\Theta}{_{\alpha_{4n-3} \alpha_{4n-2}}^{\rho \sigma}}\,, \quad  \tensor{\Theta}{_{\mu \nu}^{\rho \sigma}}=\tensor{\mathcal{T}}{_{[\mu}^{[\rho}} \tensor{\delta}{_{\nu]}^{\sigma]}}\,,
    \label{eq:appd2}
    \end{align}
where $\mathcal{T}_{\mu \nu}$ is a traceless and symmetric tensor constructed from the curvature tensor and covariant derivatives thereof. 

Consider now fluctuations of the bulk vector field $A_\mu$ on top of the neutral gravitational background \eqref{eq:appbhbackground}. After integration by parts, the piece $I_A$ of the action \eqref{eq:appdualgen} which is quadratic in the vector field reads
\begin{equation}
    I_A=-4\kappa_N \int \diff^4 x\,  \partial_\mu\left (  \sqrt{\vert g \vert} \chi^{\mu \nu \rho \sigma} A_{\nu} \partial_{\rho}A_\sigma \right )+2\kappa_N \int \diff^4 x \sqrt{\vert g \vert }\, \nabla_\mu\left ( \chi^{\mu \nu \rho \sigma} F_{\rho \sigma} \right) A_\nu \,.
\end{equation}
If we assume that  $A_\mu$ solves its field equations $\nabla_\mu\left ( \chi^{\mu \nu \rho \sigma} F_{\rho \sigma} \right)=0$, then
%\footnote{This represents the most general equation of motion satisfied by a linear perturbation in any higher-order gravity with nonminimal couplings to electromagnetism}
\begin{equation}
     I_A=-4 \kappa_N \int \diff^4 x\,  \partial_\mu\left (  \sqrt{\vert g \vert} \chi^{\mu \nu \rho \sigma} A_{\nu} \partial_{\rho}A_\sigma \right )\,.
     \label{eq:appstokes}
\end{equation}
Taking into account Eqs. \eqref{eq:appd2} and \eqref{eq:approof1}, one may derive that $\chi_{\mu \nu}{}^{\rho \sigma}$  evaluated on the metric \eqref{eq:appbhbackground} adopts the form
\begin{align}
\label{chiblack}
   \left.  \tensor{\chi}{_{\mu \nu}^{\rho \sigma}} \right \vert_{N,f}&=2 N \mathcal{B}\, \tensor{\tau}{_{[\mu}^{[\rho}} \tensor{\rho}{_{\nu]}^{\sigma]}}+2 f N \mathcal{S}\, \tensor{\tau}{_{[\mu}^{[\rho}} \tensor{\sigma}{_{\nu]}^{\sigma]}} +\frac{2}{f N \mathcal{S}}\, \tensor{\rho}{_{[\mu}^{[\rho}} \tensor{\sigma}{_{\nu]}^{\sigma]}} +\frac{1}{N \mathcal{B}}\,\tensor{\sigma}{_{[\mu}^{[\rho}} \tensor{\sigma}{_{\nu]}^{\sigma]}}\,, \\ N \mathcal{B}&=\theta+\sqrt{1+\theta^2}\, , \qquad  \qquad  \qquad f N \mathcal{S}=\varphi+\sqrt{1+\varphi^2}\,.
\end{align}
For instance, if one chooses $\mathcal{T}_{\mu \nu}=\lambda L^2 \hat{R}_{\mu \nu}$ (as in the main text), then $\theta=\lambda L^2 X$ and $\varphi=\lambda L^2 Y/2$, where $X$ and $Y$ were defined back at \eqref{eq:wxy}. Now, using the gauge $A_u=0$, direct application of Stokes' theorem on \eqref{eq:appstokes} reveals that
\begin{equation}\label{IA'}
    I_A=-4\kappa_N \int \diff^3 x   \left.  \sqrt{\vert g \vert} \left. \chi^{u a u b} \right \vert_{N,f} A_{a} A'_b \right \vert_{u=0}^{u=u_h} \,,
\end{equation}
where prime denotes differentiation with respect to the coordinate $u$ and where Latin indices refer to boundary coordinates $x^a=(t,x,y)$. If $p^a=(\omega, \mathbf{k})$, let us write $A_a$ in momentum space:
\begin{equation}\label{appFA}
 A_a (t,u,\mathbf{x})=\int  \frac{\diff^3 p}{(2 \pi)^3}  \, e^{-i \omega t +i \mathbf{k} \cdot \mathbf{x} }A_a (u,\omega,\mathbf{k})=\int  \frac{\diff^3 p}{(2 \pi)^3}  \, e^{-i \omega t +i \mathbf{k} \cdot \mathbf{x} } \tensor{M}{_{a}^{b}}(u,\omega,\mathbf{k}) A_b^0 (\omega,\mathbf{k})\,,
\end{equation}
where $A_a^0=A_a \vert_{u=0}$ and where we have implicitly defined the tensor $\tensor{M}{_{a}^{b}}$ satisfying that $\tensor{M}{_{a}^{b}}(u=0,\omega,\mathbf{k})=\tensor{\delta}{_a^b}$. Substituting \eqref{appFA} into \eqref{IA'} and performing the integration in the spacetime coordinates, we arrive to
%\begin{equation}
  %  I=-\frac{1}{4 \pi G} \int \frac{\diff^3 q}{(2 \pi)^3}   \left.  \sqrt{\vert g \vert} \chi^{u \nu u \sigma} \tensor{M}{_{\nu}^{\alpha}}(u,-\omega,-\mathbf{q}) A_\alpha^0 (-\omega,-\mathbf{q}) \tensor{{M'}}{_{\sigma}^{\beta}}(u,\omega,\mathbf{q}) A_\beta^0 (\omega,\mathbf{q}) \right \vert_{u=0}^{u=1} \,.
%    \label{eq:actionred}
%\end{equation}
%Observe that
%\begin{equation}
%\begin{split}
%     \sqrt{\vert g \vert} &\chi^{u \nu u \sigma}=\frac{r_0}{2 L^2} \hat{\chi}^{\nu \sigma}\,, 
% \\ \quad (\hat{\chi}^{\nu \sigma})&=B(u) \tensor{\delta}{_t^\nu} \tensor{\delta}{_t^\sigma}+f(u)(\tensor{\delta}{_x^\nu} \tensor{\delta}{_x^\sigma}+\tensor{\delta}{_y^\nu} \tensor{\delta}{_y^\sigma}) \,.
%     \end{split}
%\end{equation}
%Consequently, we may express \eqref{eq:actionred} as:
\begin{equation}
    I_A=\int \frac{\diff^3 p}{(2 \pi)^3}A_a^0 (-p^c) \mathcal{F}^{a b}(u, p^c)   A_b^0 (p^c) \Big \vert_{u=0}^{u=u_h}\,,
    \label{eq:prevcorr}
\end{equation}
where we have defined
 \begin{equation}
     \mathcal{F}^{ab}=-\frac{2r_0 \kappa_N }{\tilde{L}^2} \hat{\chi}^{cd}\tensor{M}{_{c}^{a}}(u,-p^e)\tensor{{M'}}{_{d}^{b}}(u,p^e)\,, \quad
\hat{\chi}^{ab}=\left (\frac{1}{\mathcal{S}}-\mathcal{B} \right) \tensor{\delta}{_t^a} \tensor{\delta}{_t^b}+\frac{1}{\mathcal{S}}\eta^{ab} \,,
 \end{equation}
 with $\eta^{ab}$ the Minkowski metric and $\tensor{{M'}}{_{d}^{b}}=\partial_u \tensor{{M}}{_{d}^{b}}$. Following the holographic prescription put forward in \cite{Son:2002sd,Herzog:2007ij}, we find the following result for the current-current retarded correlator $C_{ab}$:
\begin{equation}
    C_{ab}=2 \left. \mathcal{F}_{ab}\right \vert_{u=0}=-\left. \frac{4 r_0 \kappa_N }{ \tilde{L}^2}M'_{ab}\right \vert_{u=0}\,,
    \label{eq:appcorr}
\end{equation}
where we used that $\mathcal{B}$ and $\mathcal{S}$ must satisfy that $\lim_{u \rightarrow 0} \mathcal{B}=\lim_{u \rightarrow 0} \mathcal{S}=1$. It is important to note that \eqref{eq:appcorr} holds for all holographic duality-invariant theories (even with higher order terms in $F_{\mu\nu}$). The reason for this lies in the fact that one just needs to work with terms up to quadratic order in $A_\mu$ to fully determine $C_{ab}$. On a different front, we note that a similar derivation of such retarded correlators for generic theories \eqref{eq:appdualgen} (not necessarily duality-invariant) but in the particular case of $N(u)=1$ can be found in \cite{Myers:2010pk}.

%Therefore, all results regarding the two-point current correlator and obtained for any theory of the form \eqref{eq:appdualgen} will actually be valid for any duality-invariant theory which, at quadratic order in $A_\mu$, reduces to \eqref{eq:appdualgen}.



\section{Einsteinian Cubic Gravity}

%As shown in \cite{Bueno:2016xff}, \textcolor{red}{\'AM:THIS IS NOT CORRECT} there exists a unique non-trivial four-dimensional theory of cubic order in curvature, dubbed Einsteinian Cubic Gravity, which on maximally symmetric backgrounds propagates only a transverse and massless graviton. Its cubic terms are 
Up to topological terms, every four-dimensional six-derivative higher-order gravity may be mapped, via perturbative field redefinitions of the metric, to the following theory:
\begin{equation}
\begin{split}
    I&=\kappa_N \int \mathrm{d}^4 x \sqrt{\vert g \vert}\left[R+\frac{6}{L^2}-\frac{\mu L^4}{8}\mathcal{L}^{\mathrm{ECG}} \right]\, ,
    \label{eq:appactionecg} \\
    \mathcal{L}^{\mathrm{ECG}}&= 12 R_{\mu}{}^\rho{}_\nu{}^\sigma R_\rho{}^\gamma{}_\sigma{}^\delta R_\gamma{}^\mu{}_\delta{}^\nu+R_{\mu \nu}{}^{\rho \sigma} R_{\rho \sigma}{}^{\gamma \delta} R_{\gamma \delta}{}^{\mu \nu}-12 R_{\mu \nu \rho \sigma} R^{\mu \rho} R^{\nu \sigma} +8 R_{\mu \nu} R^{\nu \rho} R_{\rho}{}^\mu\,.
\end{split}
\end{equation}
where $\mu$ a dimensionless coupling and where $\mathcal{L}^{\mathrm{ECG}}$ receives the name of Einsteinian Cubic Gravity \cite{Bueno:2016xff}. The theory \eqref{eq:appactionecg} possesses two remarkable properties: it only propagates the usual massless  and traceless graviton on maximally symmetric backgrounds \cite{Bueno:2016xff} and admits solutions of the form \eqref{eq:appbhbackground} with $N=1$ and such that the equation of motion of $f$ can be integrated into a second-order differential equation \cite{Hennigar:2017ego,Bueno:2017sui}.
%of cubic order in the curvature may be mapped, through perturbative field redefinitions of the metric, to the 
%\begin{equation}
%   \mathcal{L}^{\mathrm{ECG}}= 12 R_{\mu}{}^\rho{}_\nu{}^\sigma R_\rho{}^\gamma{}_\sigma{}^\delta R_\gamma{}^\mu{}_\delta{}^\nu+R_{\mu \nu}{}^{\rho \sigma} R_{\rho \sigma}{}^{\gamma \delta} R_{\gamma \delta}{}^{\mu \nu}-12 R_{\mu \nu \rho \sigma} R^{\mu \rho} R^{\nu \sigma} +8 R_{\mu \nu} R^{\nu \rho} R_{\rho}{}^\mu\,.
 %  \label{eq:ecgapp}
%\end{equation}
%Afterwards, it was noticed that this theory admits  solutions with spherical, planar or hyperbolic sections characterized by a single metric function whose equation of motion can be integrated into a second-order differential equation, so that they belong to the class of theories called Generalized Quasitopological Gravities \cite{Hennigar:2017ego,Bueno:2017sui}. 

 Before examining gravitational backgrounds of the form of \eqref{eq:appbhbackground}, first one needs to derive the vacua of the theory, \emph{i.e.} its maximally-symmetric solutions. In GR, this is somewhat trivial, since the cosmological constant scale $L$ coincides with the corresponding AdS radius $\tilde{L}$. However, this is not the case when higher-curvature terms are present \cite{Bueno:2016xff,Bueno:2017sui}. For the theory \eqref{eq:appactionecg}, the relation between $L$ and the background scale $\tilde{L}$ reads as follows:
\begin{equation}
    1-\gamma+\mu \gamma^3=0\, , \qquad \gamma=\frac{L^2}{\tilde{L}^2}\,.
\end{equation}
This equation admits real and positive solutions for $\gamma$ when $\mu \leq 4/27$. Demanding the existence of both a unique stable vacuum and black hole solutions further restricts $\mu$ to lie within the range $0 \leq \mu \leq 4/27$, so that $1 \leq \gamma \leq 3/2$ \cite{Bueno:2018xqc}. When $\mu=0$ the theory \eqref{eq:appactionecg} reduces to GR, while $\mu=4/27$ corresponds to the so-called critical Einsteinian Cubic Gravity \cite{Feng:2017tev}, for which the effective Newton constant diverges. Therefore, we will assume $0 < \mu < 4/27$.





%{\color{red}[at $\mu=4/27$ we have $\gamma=3/2$, while the whole RANGE OF $\gamma$ is $1\leq\gamma<\infty$ AT $\gamma \to \infty$, $\mu \to 0$. IS THERE A PHYSICALLY MEANINGFUL SOLUTIONS FOR large $\gamma$? Asymptotically flat ones?]} On the other hand, the existence of a stable vacuum demands $\mu\geq 0$ \cite{Bueno:2017sui}, so we end up with $0 \leq \mu \leq \frac{4}{27}$. The case $\mu=0$ corresponds to GR and $\mu=\frac{4}{27}$ corresponds to the so-called critical Einsteinian Cubic Gravity \cite{Feng:2017tev}, for which the effective Newton constant diverges. Therefore, we will assume $0 < \mu < \frac{4}{27}$.

Let us now study black brane solutions of \eqref{eq:appactionecg}. The equations of motion on such backgrounds  may be obtained through the so-called reduced action method \cite{Palais:1979rca,Deser:2003up,Bueno:2017sui}, according to which one just needs to evaluate the action \eqref{eq:appactionecg} on \eqref{eq:appbhbackground} and vary with respect to $f$ and $N$. For the case at hands, if $I_{N,f}$ denotes such reduced action, one finds that
\begin{equation}
   \left.  \frac{\delta I_{N,f} }{\delta f} \right \vert_{N=1}=0\,, \qquad \forall f\,,
   \label{eq:gqgcond}
\end{equation}
so there exist black brane solutions with $N=1$. The equation of motion for $f$ is obtained by varying $I_{N,f}$ with respect to $N$ and then setting $N=1$. Eq. \eqref{eq:gqgcond} ensures that the equation for $f$  can be 
 integrated into the following second-order equation:
\begin{equation}
 4-4(3-2 \gamma) u^3-4 \gamma f+4\gamma^3  \mu f^3+\gamma^3 \mu  u^3 (f')^3-3 \gamma^3 \mu u^3 f f' f''=0\,,
 \label{eq:numecg}
\end{equation}
where the arbitrary constant arising from the integration has been carefully chosen so that, asymptotically (\emph{i.e.} as $u \rightarrow 0$), $f=1-u^3+ \mathcal{O}(u^6)$. 

Eq. \eqref{eq:numecg} is a non-linear second-order differential equation for $f$ whose analytical resolution is extremely challenging. Nevertheless, one can always obtain an asymptotic expansion for $f$, which reads
\begin{equation}
    f=1-u^3+\frac{21 (\gamma-1)}{4 \gamma-6} u^6-\frac{(\gamma-1) (3763 \gamma-3786)}{4 (3-2 \gamma)^2} u^9+\frac{3 (\gamma-1)^2 (535373 \gamma-540192) }{8 (2 \gamma-3)^3}u^{12}+ \mathcal{O}(u^{15})\,.
    \label{eq:fasymp}
\end{equation}
However, such asymptotic expansion does not provide an accurate description of the near-horizon profile for $f$, so it is necessary to solve \eqref{eq:numecg} numerically to capture both asymptotic and near-horizon behaviours.  

The differential equation \eqref{eq:numecg} becomes extraordinarily stiff as $u \rightarrow 0$ and one should proceed with utmost caution to extract a correct numerical solution.  In particular, as described in \cite{Bueno:2018xqc}, one should first demand $f$ to be regular around the horizon $u=u_h$:
\begin{equation}
    f(u)=4 \pi \frac{\tilde{L}^2}{r_0} T(u_h-u)+\sum_{n=2}^\infty a_n (u_h-u)^n\,,
\end{equation}
where $T$ is to be identified with the black brane temperature. Substituting this expression into \eqref{eq:numecg}, one fixes all coefficients $a_{n>2}$ in terms of $a_2$, which remains free. This is the parameter to adjust in order to guarantee the proper asymptotics for $f$. Through a shooting algorithm, 
 it is possible to find a unique value for $a_2$ which allows one to extend the solution from the horizon into the region where the asymptotic expansion \eqref{eq:fasymp} applies, thus finding the numerical solution for $f$ with the wanted physical properties.

 

%This is to be understood as fixing the mass to be in the units of r_0, and working with black holes which have the same mass regarding of the theory (the value of \mu) under consideration


 \section{Weak Gravity Conjecture and causality constraints}


Let us consider an exactly duality-invariant theory of the form \eqref{eq:appdualgen} with
\begin{equation}
    \mathcal{T}_{\mu \nu}=\lambda L^2 \hat{R}_{\mu \nu}\, , \quad \mathcal{L}_{\mathrm{grav}}^{\mathrm{high}}=-\frac{\mu L^4}{8} \mathcal{L}^{\rm ECG}\,,
    \label{eq:appdualwgc}
\end{equation}
where $\lambda$ is a dimensionless coupling. The goal of this appendix is to derive bounds for the dimensionless coupling $\lambda$ by demanding the theory defined by \eqref{eq:appdualwgc} to respect two physically-meaningful conditions: the Weak Gravity Conjecture and causality.
%\begin{equation}
 %   I=\kappa_N \int \mathrm{d}^4 x \sqrt{\vert g \vert} \left[R+ \frac{6}{L^2}- F^2- \lambda L^2 R_{\mu \nu} T_M^{\mu \nu} \right]\,,
 %   \label{eq:theorydualsmall}
%\end{equation}
\if{}
\begin{equation}
    I=\kappa_N \int \mathrm{d}^4 x \sqrt{\vert g \vert} \left[R+ \frac{6}{L^2}- F^2- \lambda L^2 R_{\mu \nu} T_M^{\mu \nu}-\frac{\lambda^2 L^4}{4}\left ( \hat{R}_{\mu}{}^\alpha \hat{R}_\nu{}^\beta+ \hat{R}_{\mu \rho} \hat{R}^{\rho \alpha} \delta_{\beta}{}^\nu \right ) F^{\mu \nu} F_{\alpha \beta} \right]-\frac{\mu L^4}{8}\mathcal{L}^{\mathrm{ECG}}\,,
    \label{eq:theorydualsmall}
\end{equation}
%\fi
where $\lambda$ is a dimensionless coupling $T_M^{\mu \nu}=F^{\mu \alpha} F^{\nu}{}_\alpha-\frac{1}{4} g^{\mu \nu}F^2$ the Maxwell stress-energy tensor. This theory corresponds to the $\mathcal{O}(L^2)$ truncation of an exactly duality-invariant theory of the form \eqref{eq:appdualgen} with $\mathcal{T}_{\mu \nu}=\lambda L^2 \hat{R}_{\mu \nu}$ ($\hat{R}_{\mu \nu}$ being the traceless Ricci tensor) and $\mathcal{L}_{\mathrm{grav}}^{\mathrm{high}}=\mathcal{O}(L^4)$. The goal of this appendix is to derive bounds for the dimensionless coupling $\lambda$ by demanding the theory \eqref{eq:theorydualsmall} to respect two physically-meaningful conditions: the Weak Gravity Conjecture and causality.
\fi

On the one hand, the Weak Gravity conjecture (WGC) was originally stated for asymptotically flat black holes \cite{Arkani-Hamed:2006emk,Harlow:2022gzl}. In its mild form, the WGC claims that the decay of an extremal black hole into smaller black holes should be possible, at least from the point of view of charge and energy conservation \cite{Cheung:2018cwt,Hamada:2018dde}. However, in AdS space such requirement is somewhat trivial, since perturbative (arbitrarily small) higher-order corrections cannot violate it \cite{Cremonini:2019wdk}. Instead, one could make use of the proposal put forward in \cite{Cheung:2018cwt}, according to which the corrections to the black hole entropy for any mass and charge ought to be positive for thermodynamically-stable black holes. This is a natural candidate for the generalization of the WGC to AdS spaces, since it has been shown that corrections to the near extremal entropy are related to corrections to the extremal mass \cite{Goon:2019faz,McPeak:2021tvu}. Therefore, this is the condition we will explore here. 

%We will here focus on the corrections to the near extremal entropy, which, interestingly enough, are connected to the corrections to the extremal mass \cite{Goon:2019faz,McPeak:2021tvu} and, therefore, in agreement with the spirit of the WGC for asymptotically flat black holes. This justifies to keep the name ``WGC'' for the requirement of having positive corrections to the entropy of AdS black holes. 


%Differently from the body of the document, for this argument we need to find (perturbative) solutions of the theory that take into account the interplay between the metric and the field strength --- \emph{i.e.} study the corresponding Reissner-Nordstr\"om-like solutions. For that, we consider the black brane background \eqref{eq:bhbackground} and\footnote{This form for $F$ is fixed after demanding it to respect the same symmetries as the background.}: 
Focusing on black brane configurations of the type \eqref{eq:appbhbackground}, the argument requires to consider a complete charged black brane solution that takes into account the backreaction from the gauge field strength (in other words, we must find the appropriate corrections to the Reissner-Nordstr\"om solution). Imposing $F$ to have the same symmetries as \eqref{eq:appbhbackground}, we may write
\begin{equation}
 F=-\frac{r_0}{L}\Phi'\mathrm{d}t \wedge \mathrm{d}u+\frac{r_0^2}{L^3}P \, \mathrm{d}x \wedge \mathrm{d}y\,,
\end{equation}
where $\Phi=\Phi(u)$ is the electric potential and $P$ is a constant related to the magnetic charge. To derive strong constraints on $\lambda$ from the WGC conjecture, it suffices to consider the following  $\mathcal{O}(L^2)$ truncation  of \eqref{eq:appdualwgc}:
\begin{equation}
    I=\kappa_N \int \mathrm{d}^4 x \sqrt{\vert g \vert} \left[R+ \frac{6}{L^2}- F^2- \lambda L^2 R_{\mu \nu} T_M^{\mu \nu} \right]\,,
    \label{eq:theorydualsmall}
\end{equation}
where $T_M^{\mu \nu}=F_{\mu \alpha}F_{\nu}{}^\alpha-g_{\mu \nu} F^2/4$ is the Maxwell stress-energy tensor. Expand the solution in terms of $\lambda$:
\begin{equation}
    f=f_0+f_1 \lambda + \mathcal{O}(\lambda^2)\, , \quad  N=N_0+N_1 \lambda + \mathcal{O}(\lambda^2)\, , \quad  \Phi=\Phi_0+\Phi_1 \lambda + \mathcal{O}(\lambda^2)\, .
    \label{eq:appexpan}
\end{equation}
%{\color{red} Observe that the solution is not expanded in terms of $L$ because it is dimensionful and could be set to unity in proper units.??? In fact, what is assumed is that the cosmological length scale $\ell_c$ is much larger than the scale above which higher-order corrections $\ell_h$ arise. Since \eqref{eq:appdualwgc} is expressed entirely in terms of the cosmological length scale $L=\ell_c$, we should interpret that appropriate powers of the ratio $\ell_h/L<<1$ have been absorbed into the definitions of the dimensionless couplings in \eqref{eq:appdualwgc} (\emph{i.e.} $\lambda <<1$).[I WOULD SKIP THIS PARAGRAPH]} Now, the zeroth-order solution is clearly the Reissner-Nordstr\"om one:
where the zeroth-order solution corresponds to the Reissner-Nordstr\"om one,
\begin{equation}
    f_0=1-u^3+ (Q^2+P^2)u^4\, ,\quad N_0=1 \, ,\quad \Phi_0=Q u\,,
\end{equation}
with $Q$ being a constant related to the electric charge. The computation of the leading-order corrections gives
\begin{equation}
    f_1=\frac{(Q^2+P^2) u^4}{20} \left (30 +  (14 (Q^2+P^2) u-15)u^3 \right )\, , \quad N_1=-\frac{u^4 (Q^2+P^2)}{4}\, , \quad \Phi_1=\frac{3 Q (Q^2+P^2) u^5}{20}\,.
\end{equation}
Observe that both $f_1$ and $N_1$ are invariant under rotations of $Q$ and $P$, as required by duality invariance. 

Let $u_h$ be the location of the outermost horizon. We express it as
\begin{equation}
    u_h=u_0+u_1 \lambda+\mathcal{O}(\lambda^2)\,.
\end{equation}
From the condition $f(u_h)=0$, one derives that 
\begin{equation}
    1-u_0^3+(Q^2+P^2) u_0^4=0\, , \quad u_1=\frac{u_0(16-u_0^3)(u_0^3-1)}{20(4-u_0^3)}\,.
\end{equation}
There exist positive solutions for $u_0$ only if and only if
\begin{equation}
    \frac{3}{4\sqrt[3]{4}} > Q^2+P^2>0\,.
    \label{eq:extcond}
\end{equation}
This imposes\footnote{The value $u_0=\sqrt[3]{4}$ corresponds to the extremal black brane and, in such a case, we observe that $u_1$ diverges. Therefore, the expansion in $\lambda$ is not reliable if we are arbitrarily close to extremality. Nevertheless, a near-extremal perturbative study of the black brane may be carried out if one remains in a regime for which $\vert \sqrt[3]{4}-u_0\vert >> \lambda$ (so that $\lambda u_1$ is still small compared to $u_0$).} $u_0 \in (1,\sqrt[3]{4})$, so that $u_1>0$. Since \eqref{eq:theorydualsmall} contains no higher-curvature terms, the entropy density $s$ associated to the black brane may be derived with the aid of the Bekenstein-Hawking formula, getting
\begin{equation}
    s=\frac{4 \pi \kappa_N r_0^2}{ L^2 u_h^2}=\frac{ 4 \pi \kappa_N r_0^2}{L^2 u_0^2}\left (1-\lambda \frac{(16-u_0^3)(u_0^3-1)}{10(4-u_0^3)} \right)+\mathcal{O}(\lambda^2)\,.
\end{equation}
For the correction to be positive, we must require\footnote{This constraint was obtained perturbatively and could not be reliable for large $\vert \lambda \vert$ (which would also imply the failure of the expansion \eqref{eq:appexpan}). Nevertheless, since causality will further restrict $\vert \lambda\vert $ to be less than $\sim 1/2$, we will not worry about further refinement.} $\lambda <0$. 
%\footnote{One could wonder at this point about the validity of this perturbative constraint. Nevertheless, we must bear in mind that we are dealing with a truncation of an effective theory to order $L^2$, so the coupling $\lambda$ must remain parametrically small in order to ensure the legitimacy of the process.}
%Hence, we may trust the constraint $\lambda <0$.


Let us now examine the constraints that may be derived  by requiring causality for the CFT dual to \eqref{eq:appdualwgc}. Following the analysis of \cite{Myers:2010pk}, one should guarantee that no superluminal modes of the vector field on top of a gravitational background of the form \eqref{eq:appbhbackground} are propagated, since this would signal causality violation in the holographic CFT. Such modes are absent when GR configurations are considered, but may appear after inclusion of higher-curvature terms. Therefore, we will explore which values of $\lambda$ ensure that causality is respected for the full theory \eqref{eq:appdualwgc}.

%\sout{Therefore, if we do not restrict ourselves to the truncation \eqref{eq:theorydualsmall} of \eqref{eq:appdualwgc}, requiring the theory to respect causality will impose further bounds on $\lambda$.} {\color{red} We will find these bounds on  $\lambda$ in the full theory \eqref{eq:appdualwgc}.}

To this end, we have to study the equations of motion of $A_\mu$. In the gauge $A_u=0$, such equations were presented in the main text in momentum space. It was explained that the equations for $A_y$, $\mathcal{B} A_t'$ and $A_x'/\mathcal{S}$ are completely equivalent, so here it suffices  to analyze that of $A_y$.  Defining $r_0 z=\tilde{L}^2\int_0^u \mathcal{S}(u') \mathrm{d}u'$, this equation can be written as:% follows:
%to examine the causality of the dual holographic CFT we shall take a look to the equations of motion \eqref{eq:sym1}, \eqref{eq:sym2}, \eqref{eq:sym3} and \eqref{eq:sym4} for the vector modes. As explained in the body of the document, the equations for $A_y$, $\mathcal{B} A_t'$ and $A_x'/\mathcal{C}$ are the same, so it suffices to analyze Eq. \eqref{eq:sym4}. Defining $z=\int_0^u \mathcal{C}(u') \mathrm{d}u'$, \eqref{eq:sym4} can be rewritten as follows:
\begin{equation}
    -\frac{\mathrm{d}^2 A_y}{\mathrm{d} z^2}+ k^2 V(z) A_y= \omega^2 A_y\,, \quad V(z)=\frac{1}{\mathcal{B}(z) \mathcal{S}(z)}\,.
    \label{eq:ondasAy}
\end{equation}
By direct inspection of \eqref{eq:ondasAy}, we observe that the existence of a region for $z$ in which $V(z)>1$ will generically give rise to modes with superluminal propagation \cite{Brigante:2008gz,Buchel:2009tt}. Indeed, if $z_0$ is a point for which $V(z_0)>1$, then around this point the renormalizable modes (which need to satisfy $\omega^2-k^2 V(z_0)>0$) will be such that $\omega^2/k^2>1$. Therefore, we need to determine which values of $\lambda$ ensure that $V(z)\leq 1$ in the domain of $z$ corresponding to $u \in (0,u_h)$.

%We should therefore find for which values of $\lambda$ the theory does not develop the aforementioned behaviour.


%We consider the theory under study in the manuscript:
%\begin{equation}
 %   I=\frac{1}{16 \pi G}\int \mathrm{d}^4 x \sqrt{\vert g \vert} \left[R+ \frac{6}{L^2}- F^2- \lambda L^2 R_{\mu \nu} T_M^{\mu \nu}-\frac{\mu L^4}{8}\mathcal{L}^{\mathrm{ECG}} \right]\,,
 %   \label{eq:theorydualsmallecg}
%\end{equation}
%where $\mathcal{L}^{\mathrm{ECG}}$ was defined in \eqref{eq:ecgapp}. Observe that now we do include cubic curvature terms in the action, since otherwise the analysis reduces to that of Einstein-Maxwell theory --- in particular, no causality violations exist. The theory \eqref{eq:theorydualsmallecg} possesses neutral black hole/brane solutions of the form \eqref{eq:bhbackground} with $N=1$, the equation for $f$ being second-order (see \eqref{eq:numecg}). This last equation can be resolved numerically, obtaining a subsequent numerical profile for the potential $V(z)$. It will depend on the ECG coupling $\mu$, so we will choose the value $\mu=\frac{1}{10}$, which corresponds to the specific choice made in the document.

The potential $V(z)$ depends on the parameter $\mu$ as well. We will fix it to be $\mu=1/10$, which is the value explored in the main text. Then, obtaining numerically the black brane background associated to \eqref{eq:appactionecg} (characterized by $N=1$ and the second order equation \eqref{eq:numecg} for $f$), in Fig. \ref{fig:pots} we present the profile for the potential for different values of $\lambda$. Only negative values of $\lambda$ are considered, since, as shown above, positive values are forbidden by the WGC conjecture. We observe that, as $\vert \lambda \vert$ increases, the function $V(z)$ develops a maximum which gets more and more peaked. In particular, we have checked that $V(z)\leq 1$ if
\begin{equation}
   0\geq \lambda \gtrsim -0.50105\,.
   \label{eq:lambdapermi}
\end{equation}
%This is the range for $\lambda$ considered in the main text. Note that the lower bound was obtained in the specific case $\mu=1/10$ and will change for different $\mu$ \textcolor{red}{¿Something else?}.
This is the range for $\lambda$ considered in the main text, obtained in the specific case $\mu=1/10$. Now, one might wonder whether additional conditions to further constrain the values of $\lambda$ could be found. For instance, one could explore if the energy flux measured at infinity after the local insertion of a current operator \cite{Hofman:2008ar} is positive in all directions \cite{Myers:2010pk} or study the existence of unstable modes of the vector field \cite{Myers:2007we}. However, none of these requirements provide additional constraints, since the energy flux is exactly the same as that of Einstein-Maxwell theory (as explained at the end of the body of the manuscript) and since the potential has no negative minima for the probed values of $\lambda$ \cite{Myers:2010pk}. %Therefore, the physically-sensible values for $\lambda$  are those ranging in the interval given at \eqref{eq:lambdapermi}. 

\begin{figure}[H]
\centering
\includegraphics[scale=0.4]{Graf_Pots_Ref.pdf}
\caption{Profile of $V(u)$. Here we have turned back to the use of the variable $u$ to make the representation clearer. We observe the appearance of a maximum that gets more and more pronounced as $\vert \lambda \vert$ increases.}
\label{fig:pots}
\end{figure}




%Although this result was obtained from a perturbative analysis, 


\bibliographystyle{apsrev4-1}
%\bibliographystyle{hsiam}
\vspace{1cm}
\bibliography{Gravities.bib}

\end{document}
















\section{Conductivity}

Let us recall the relation between the conductivity $\sigma$ and the retarded correlator $C_{ab}$ \cite{Myers:2010pk}:
\begin{equation}
    \sigma=-\mathrm{Im}\left ( \frac{C_{yy}}{\omega} \right)\,.
\end{equation}
Using \eqref{eq:corr}, it turns out that the conductivity is fully determined by:
\begin{equation}
    C_{yy}= \frac{r_0}{4 \pi G \tilde{L}^2} \frac{A_{y}'(0)}{A_y(0)}\,,
\end{equation}
where $A_y(u)$ is any solution of \eqref{eq:sym4} with \emph{ingoing} boundary conditions\footnote{We leave the normalization of $A_y$ unspecified, since it will be unimportant for the final result for the conductivity.} at the horizon located at $u=u_h$. 

\noindent
Let us first solve \eqref{eq:sym4} for $\mathfrak{q}=0$:
\begin{equation}
     a_0 \mathfrak{w}^2 \mathcal{C}+ \left (\frac{a_0'}{\mathcal{C}}\right )'=0\,.
\end{equation}
The most general solution compatible with ingoing boundary conditions at the horizon is:
\begin{equation}
    a_{0}(u)=\mathcal{A}_0\, e^{i \, \mathfrak{w} F(u) }\,,
\end{equation}
where $F(u)= \int_0^u \mathcal{C}(z) \mathrm{d}z$ and $\mathcal{A}_0$ is a constant. Now, for generic $\mathfrak{q} \neq 0$ write:
\begin{equation}
    A_y(u)=\sum_{n=0}^{\infty} a_n(u) \mathfrak{q}^{2n}\,.
\end{equation}
Substituting in \eqref{eq:sym4}:
\begin{equation}
     a_n \mathfrak{w}^2 \mathcal{C}+ \left (\frac{a_n'}{\mathcal{C}}\right )'=\frac{a_{n-1}}{\mathcal{B}}\,, \quad n \geq 1\,.
\end{equation}
This is an inhomogeneous linear differential equation whose most general solution is:
\begin{align}
\nonumber
    a_n(u)&=\mathcal{A}_{2n-1}\, e^{i \mathfrak{w} F(u)}+ \mathcal{A}_{2n}\, e^{-i \mathfrak{w} F(u)}\\ \label{eq:aneq} &-\frac{i}{2 \mathfrak{w}} \int_0^u \frac{a_{n-1}(z)}{\mathcal{B}} e^{-i \mathfrak{w} F(z)} \mathrm{d}z\,  e^{i \mathfrak{w} F(u)}\\ \nonumber&+\frac{i}{2 \mathfrak{w}} \int_0^u \frac{a_{n-1}(z)}{\mathcal{B}} e^{i \mathfrak{w} F(z)}  \mathrm{d}z \, e^{-i \mathfrak{w} F(u)}\,,
\end{align}
where $\mathcal{A}_{2n-1}$ and $\mathcal{A}_{2n}$ are constants which should be fixed properly in order to ensure that $A_y$ is ingoing at $u=u_h$. %However, its precise form is not of relevance for our purposes, as we are about to see next.

%\noindent
%Taking into account \eqref{eq:aneq} and the fact that the solution must reduce to that of Einstein-Maxwell theory asymptotically (observe that $\mathcal{C}(0)=\mathcal{B(0)}=1$), we arrive to:
%\begin{equation}
  %  \frac{A_y'(0)}{A_y(0)}=i \mathfrak{w}\dfrac{\mathcal{A}_0+\sum\limits_{n=1}^{\infty} (\mathcal{A}_{2n-1} -\mathcal{A}_{2n})\mathfrak{q}^{2n}}{\mathcal{A}_0+\sum\limits_{n=1}^{\infty} (\mathcal{A}_{2n-1} +\mathcal{A}_{2n}) \mathfrak{q}^{2n}}
%\end{equation}
%Consider the change of the previous ratio as the scale $\ell$ at which the effects of higher-order corrections should be taken into account. Direct computation reveals that:
%\begin{equation}
%  \hspace{-0.2cm}  \frac{\mathrm{d}}{\mathrm{d} \ell} \left (  \frac{A_y'(0)}{A_y(0)} \right)=\dfrac{ \dfrac{\mathrm{d}A_y'(0)}{\mathrm{d} \ell} A_y(0)-A_y'(0)\dfrac{\mathrm{d}A_y(0)}{\mathrm{d} \ell}}{(A_y(0))^2}=0\,,
%\end{equation}
%since:
%\begin{align}
%\frac{\mathrm{d}a_n'(0)}{\mathrm{d} \ell}&=i \mathfrak{w}\left (\frac{\mathrm{d}\mathcal{A}_{2n-1}}{\mathrm{d} \ell}-\frac{\mathrm{d}\mathcal{A}_{2n}}{\mathrm{d} \ell} \right )\\ 
%\frac{\mathrm{d}a_n(0)}{\mathrm{d} \ell}& =\frac{\mathrm{d}\mathcal{A}_{2n-1}}{\mathrm{d} \ell}+\frac{\mathrm{d}\mathcal{A}_{2n}}{\mathrm{d} \ell}\,.
%\end{align}
%Therefore, this proves that the conductivity of any holographic theory of gravity and electromagnetism, even including nonminimal couplings between gravity and electromagnetism, coincides identically to that of Einstein-Maxwell theory for any value of the momentum and frequency.


