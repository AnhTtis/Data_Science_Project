% mnras_template.tex
%
% LaTeX template for creating an MNRAS paper
%
% v3.0 released 14 May 2015
% (version numbers match those of mnras.cls)
%
% Copyright (C) Royal Astronomical Society 2015
% Authors:
% Keith T. Smith (Royal Astronomical Society)

% Change log
%
% v3.0 May 2015
%    Renamed to match the new package name
%    Version number matches mnras.cls
%    A few minor tweaks to wording
% v1.0 September 2013
%    Beta testing only - never publicly released
%    First version: a simple (ish) template for creating an MNRAS paper

%%%%%%%%%%%%%%%%%%%%%%%%%%%%%%%%%%%%%%%%%%%%%%%%%%
% Basic setup. Most papers should leave these options alone.
\documentclass[a4paper,fleqn,usenatbib]{mnras}

% MNRAS is set in Times font. If you don't have this installed (most LaTeX
% installations will be fine) or prefer the old Computer Modern fonts, comment
% out the following line
%\usepackage{newtxtext,newtxmath}
% Depending on your LaTeX fonts installation, you might get better results with one of these:
%\usepackage{mathptmx}
%\usepackage{txfonts}

% Use vector fonts, so it zooms properly in on-screen viewing software
% Don't change these lines unless you know what you are doing
\usepackage[T1]{fontenc}
\usepackage{ae,aecompl}
\usepackage[thinc]{esdiff}
\usepackage{chemformula}

\usepackage{multirow}
\usepackage{hyperref}
\usepackage{pgfplots}
\pgfplotsset{width=1cm}

\usepackage[normalem]{ulem}

%%%%% AUTHORS - PLACE YOUR OWN PACKAGES HERE %%%%%

% Only include extra packages if you really need them. Common packages are:
\usepackage{graphicx}	% Including figure files
\usepackage{amsmath}	% Advanced maths commands
\usepackage{amssymb}	% Extra maths symbols

\usepackage{isotope}
\usepackage{chemformula}
\usepackage{siunitx}
\usepackage{wrapfig,framed}
\usepackage{outlines}

\usepackage{natbib}
\defcitealias{Booth_disint22}{Booth22}
%%%%%%%%%%%%%%%%%%%%%%%%%%%%%%%%%%%%%%%%%%%%%%%%%%

%%%%% AUTHORS - PLACE YOUR OWN COMMANDS HERE %%%%%

% Please keep new commands to a minimum, and use \newcommand not \def to avoid
% overwriting existing commands. Example:
%\newcommand{\pcm}{\,cm$^{-2}$}	% per cm-squared


%%%%%%%%%%%%%%%%%%%%%%%%%%%%%%%%%%%%%%%%%%%%%%%%%%

%%%%%%%%%%%%%%%%%%% TITLE PAGE %%%%%%%%%%%%%%%%%%%


% Title of the paper, and the short title which is used in the headers.
% Keep the title short and informative.
\title[ ]{The evolution of catastrophically evaporating rocky planets}

% The list of authors, and the short list which is used in the headers.
% If you need two or more lines of authors, add an extra line using \newauthor
\author[]{Alfred Curry$^{1}$, Richard Booth$^{2,1}$, James E. Owen$^{1}$, Subhanjoy Mohanty$^{1}$
\\
% List of institutions
$^{1}$Astrophysics Group, Department of Physics, Imperial College London, Prince Consort Rd, London SW7 2AZ, UK\\
$^{2}${School of Physics and Astronomy, University of Leeds, Leeds, LS2 9JT, UK}}

% These dates will be filled out by the publisher
\date{Accepted XXX. Received YYY; in original form ZZZ}

% Enter the current year, for the copyright statements etc.
\pubyear{2022}

% Don't change these lines
\begin{document}
\label{firstpage}
\pagerange{\pageref{firstpage}--\pageref{lastpage}}
\maketitle

\begin{abstract}
%Understanding the interiors of rocky planets is an essential part of exoplanet science. They provide a tracer of planet formation, set the compositions of the secondary atmospheres of low mass planets through outgassing, and affect the thermal evolution of these atmospheres. One way of investigating the compositions of these rocky planets is through catastrophically evaporating planets. 
Catastrophically evaporating planets are observed through their dusty tails, formed through rocky material evaporated from their highly irradiated molten surfaces. The composition of these tails offers an avenue for studying the composition of rocky exoplanets, but only if the evolution of the underlying interior is understood. This is because it is the interior evolution that sets the composition at the base of the mass outflow. In this work, we present a model of the evolution of the interiors of catastrophically evaporating planets. Its basis is a one-dimensional code that takes into account energy flow through conduction and convection as well as melting. We find that these planets are likely to be entirely solid when significant mass loss occurs, other than a thin magma pool on the day side. Consequently, the outflows from the planets, and thus the dust tails, sample material only from the surface of the planet. We also use our model to investigate the occurrence rate of planets that can catastrophically evaporate, and find that, on average, a star in the {\it Kepler} sample has approximately one such planet. Our value is above, but within an order of magnitude of, the occurrence rate inferred from the {\it Kepler} statistics for Super-Earths, implying that exotic mechanisms to produce the catastrophically evaporating planet population may not be required. We also find that the range of substellar temperatures of the observed systems are well explained by recent theoretical models which only produce dust in a limited temperature region. 


\end{abstract}

\begin{keywords}
exoplanets - planets and satellites: interiors - planets and satellites: physical evolution - planets and satellites: composition
\end{keywords}
\section{Introduction}

Understanding rocky planet interiors is an essential part of exoplanet science. Firstly, their compositions lend insight into planet formation. Secondly, the interior chemistry determines the atmospheric chemistry through outgassing for planets below a few Earth masses that do not possess a \ch{H2}/He envelope. Atmospheric chemistry is influenced both through exchange with the early molten surface \citep[e.g.,][]{Abe1986,GAIL2014,Licht21verticalmagma}, and across the lifetime of the planet through volcanism \citep[e.g.,][]{Kite09,Noack2014,Tosi2017}. Furthermore, the thermal evolution of the interior affects these outgassing processes and acts as a boundary condition for the atmosphere's evolution. Thus, the interior is important to explain the range of possible exoplanetary atmospheres, including habitability. 

Chemical compositions of exoplanet interiors are most directly observed in the atmospheres of polluted white dwarfs \citep{Zuckerman2010}, which show metal lines due to planetary bodies that have recently been accreted, and have been shown to have compositions which are consistent with Solar System rock \citep[e.g.,][]{DZDwarfs1,Harrison2021}. However, as these systems can only track the bulk composition of the pollutant, little information about compositional structure or influence on the planet's atmosphere can be inferred. Another avenue of interest is the study of planet's atmospheres, particularly those of hot planets where the atmospheric composition is likely dominated by interaction with the surface and interior \citep[e.g.,][]{Zilinkas22}. However, the atmospheric interactions are still not fully understood, and made complicated by atmospheric processes such as loss and photochemistry, even if such tenuous atmospheres become observable. The catastrophically evaporating planets, however, are less dependent on such atmospheric processes because they are observed directly through solid material.

Three catastrophically evaporating planets -- Kepler 1520b, \citep[formerly KIC1255b,][]{KIC1255-discov}, KOI-2700b \citep{KOI2700b-discov} and K2-22b \citep{K2-22b-discov} -- were discovered by the {\it Kepler}/{\it K2} missions \citep{KEPLER,K2-mission}. The distinguishing features of the lightcurves of these systems are (i) all three systems have highly variable transits depths indicating that the orbiting body is not a single solid object. (ii) there is an increase in flux directly before transit, which can be explained  by the forward scattering of starlight by dust (iii) Kepler 1520b and KOI-2700b show a highly asymmetric averaged transit due to the extended tail \citep[see e.g.,][Figure 1]{vanLieshout16}. These features are well explained by the transit curves being due to tails of dusty material \citep{disint18}. 

The short periods of the dust tails imply high surface temperatures for any planet at that orbital distance (see \autoref{tab: systems}), sufficient to melt rocky materials. Therefore the origin of the dusty tails is thought to be the evaporation of rocky material from the molten surface of an underlying planet, which then expands and condenses as dust further from the surface. This idea was originally proposed by \citet{KIC1255-discov}, with further physical modelling by \citet{Perez-Becker13}. More recent models of the outflows have focused on photochemistry \citep{Ito-Ikoma21}, day to nightside flows \citep{Kang2021} and variability \citep{Bromely-Chiang23}. \citet{Booth_disint22} were notably able to show that dust can form self-consistently during this atmospheric escape under the right conditions. 

Since the material in the dusty tails comes directly from the solid portion of the planet, these systems are particularly interesting in the wider context of exoplanets because they may allow the study of interior composition. Thus far, attempts to constrain the composition of tail material have depended on modelling the light curves. For instance, \citet{vanLieshout14,vanLieshout16} find that the lightcurve of Kepler-1520b is consistent with corundum (\ch{Al3O2}) and KOI-2700b with corundum or fayalite (\ch{Fe2SiO4}), but both are inconsistent with pure iron or carbon, based on the chemicals' sublimation rates. In addition, the size of the forward scattering peak can be used to infer grain sizes, which are required to break degeneracies with composition in modelling, and are constrained to a range of 0.1 - \SI{1}{\micro\metre} \citep{Budaj13}.

Further, multi-wavelength observations will allow inferences to become more detailed. {\it James Webb Space Telescope} infra-red spectra may allow silicate minerals to be identified through their resonant features around 10$\mu$m \citep{Bodman18-JWST}, and it may be possible to detect atomic lines in the gas in the tails \citep{disint18}.

The mass loss models of \citet{Perez-Becker13} and \citet{Booth_disint22} agree that planets must be low enough mass for material to escape, $\lesssim 0.1 M_\oplus$, once they are observed, below the mass range of planets probed by conventional detection methods \citep[e.g.,][]{Christiansen2014}. This means that the catastrophically evaporating planets provide an opportunity to probe low-mass planets, which is otherwise inaccessible.

As has been noted in previous works \citep[e.g.,][]{Perez-Becker13}, and as we will also show, the high mass loss rates inferred for these planets means that they can be entirely destroyed within several Gyrs. Models of the mass loss process agree that the mass loss rates should generally increase with decreasing mass, leading to an accelerating mass loss before total destruction. A consequence of this is that the observed systems must be within a region where their mass loss rates are high enough to be observed, but the planet has not been destroyed yet. Therefore, the fact that any are observed at all may tell us about the occurrence rate of these planets' progenitors. This is of particular significance because, as noted above, these are low-mass planets when they are observed, but also, as we will show, must have started at low mass $\lesssim 0.3 M_\oplus$, so they are part of a distribution not otherwise observed.

In order to fully understand the catastrophically evaporating planets, both in their own right and to make full use of their potential as probes of interior physics, it is necessary to have a model of their interiors. The evolution of the underlying rocky planet may affect the thermal state at the bottom of the mass outflow. Additionally, and more significantly, the interior evolution sets the surface composition, which can be probed observationally through the dusty tails. Different compositions, for instance, through different depths that melting reaches, can vastly affect the chemicals that would be present in the outflows \citep[e.g.,][]{schaefer_fegley2009}.

In this work, we present our evolutionary model for the interior of catastrophically evaporating planets. In particular, we investigate the evolution of melt, since the molten regions are able to convect over much shorter timescales than solid regions, and so have an important influence on composition by circulating chemicals to the surface. We then go on to investigate the population of the progenitors of the catastrophically evaporating planets. 

In \S\ref{sec: Method}, we describe our 1D numerical model for rocky interiors and its specific application to these systems. In \S\ref{sec: results}, we present the results of our thermal evolution calculations. We discuss the results and their limitations in \S\ref{sec: discuss} and look at their implications for the distribution of low-mass planets in \S\ref{sec: occurrence}. We conclude in \S\ref{sec: conclusion}.

\section{Interior Model} \label{sec: Method}
We have developed a code for modelling in 1D the evolution of a rocky planet with a mass that can evolve over time, which we shall apply to the evolution of catastrophically evaporating planets. The numerical scheme is based on stellar structure codes, as described in \citet{Kippen} and \citet{Boden}. We solve for the structure in the planet's rocky mantle and iron core. Here we summarise the physics included and its basic operation.

\subsection{Basic Equations} \label{sec: struct_eqns}
The essential equations for the internal structure of a spherical body, written with mass, $m$, as the independent variable, are
\begin{enumerate}
    \item Hydrostatic Equilibrium
        \begin{equation}
           \diff{P}{m} = - \frac{Gm}{4 \pi r^4} \label{eq: hydrostat} 
        \end{equation}
    \item Mass conservation
        \begin{equation}
            \diff{r}{m} = \frac{1}{4 \pi r^2 \rho} \label{eq: masscon}
        \end{equation}
    \item Energy conservation
        \begin{equation}
            \diff{L}{m} = H -T\diff{S}{t} = H -C_p \diff{T}{t} +\frac{\delta}{\rho} \diff{P}{t} \label{eq: lumin}
    \end{equation}
    where $T\diff{S}{t}$ is the rate of heat exchange from a unit mass, $H$ is the heat generation per unit mass (through radioactive decay in the case of planets), $C_P$ is the specific heat capacity at constant pressure, and 
\begin{equation}
    \delta \equiv \left. -\diffp{{\ln{\rho}}}{{\ln{T}}}\right|_{P}
\end{equation} is a measure of thermal expansivity. 
    %The second equality comes from thermodynamic relations \citep[see, for instance,][chapter 4]{Kippen}.
\end{enumerate}
One also needs an equation for the temperature gradient, which will be related to heat transport. Following the conventions in stellar interiors, we define the temperature gradient, with respect to pressure, as
\begin{equation}
    \nabla \equiv \diff{\ln{T}}{\ln{P}} \label{eq: nabladef}
\end{equation}
and so, using hydrostatic equilibrium
\begin{equation}
    \diff{T}{m} = - \frac{GmT}{4 \pi r^4 P} \nabla \label{eq: Tmnabla}\\
\end{equation}
where we have simply made use of hydrostatic equilibrium.
We will show in \S\ref{sec: heat_flow} how we find the temperature gradient, $\nabla$, from consideration of the heat transport.

In order to find solutions to this system of differential equations, one requires, firstly, the equation of state and other material properties, which may be functions of temperature and pressure, which will be discussed in \S\ref{sec: physical properties}. Secondly, one requires four boundary conditions. The two
inner boundary conditions are simply $R=0$ and $L=0$ at $m=0$. At the outer boundary, we use a fixed $P=P_0$ and a relation between the outer temperature $T_0$ and the other outer properties. We explain these outer boundary conditions for the cases we investigate here in \S\ref{sec: BCs}.

In order to solve this system of equations, we follow a Henyey scheme, the details of which can be found in \citet{Boden}, chapter 5, for example. In summary, we set up a mass grid such that the differential equations become a series of simultaneous equations. We then solve these simultaneous equations through Newton's method, up to a given tolerance threshold. The tolerance criterion we use is that the difference in a dependent variable across a mass grid cell must have a relative accuracy of $10^{-6}$. The time dependence in the energy equation is solved implicitly.


\subsection{Melting}\label{sec: melting}
Rocks are chemical mixtures and thus can be partially molten even in thermodynamic equilibrium at fixed temperature and pressure. This process is complicated because the melt composition will generally differ from that of the solid rock, and will also depend strongly on both the local conditions and the overall composition of the rock. To simplify the full problem, we introduce a parameter for the mass fraction of melt, $\phi$, which is simply a function of $P$ and $T$ \citep[e.g.][]{abe1993thermal,Licht16,Bower17}. We are thus effectively assuming that melting is in equilibrium, which is not necessarily true; we will return to this point in the discussion (\S\ref{sec: discuss-other}).

Following other works \citep[e.g.,][]{abe1993thermal}, we use the simple linear function
\begin{equation}
    \phi = \frac{T - T_\text{sol}(P)}{T_\text{liq}(P) - T_\text{sol}(P)} \label{eq: melt_frac}
\end{equation}
where $T_\text{sol}$ is the solidus, the temperature below which the material is completely solid (i.e., $\phi$=0 for $T$$<$$T_{sol}$), and $T_\text{liq}$ is the liquidus, the temperature above which the material is completely liquid (i.e., $\phi$=1 for $T$$>$$T_{liq}$); both $T_{sol}$ and $T_{liq}$ are functions of $P$. For now, we consider these fixed functions, but again, the reality is far more complex and highly composition dependent.

For the mantle solidus and liquidus profiles, we use the fit to the Simon and Glatzel equation \citep{Simon-Glatzel}: from \citet{Andrault11} for high pressures, and the curves in \citet{LITASOV2002} for low pressures. The former are from experiments of an artificial chondritic composition, and the latter on peridotite KLB-1, representing mantle material. For use in our code, we tabulate these functions and access values using cubic Hermite interpolation.

\subsection{Heat flow in the mantle}\label{sec: heat_flow}
We include heat flow in our model by finding how the temperature gradient $\nabla$ depends on the energy flux. This can then be inserted into  \autoref{eq: Tmnabla}. In practice, we find how the energy flux, $F = \frac{L}{4\pi r^2}$, depends on the temperature gradient and invert this function using Brent's method.

\subsubsection{Conduction and Convection}
%For rocky interiors, both solid and liquid the dominant heat flow mechanism is convection, for any reasonable heat flux. This is because of the low conductivity, meaning the Schwarscild inequality is easily satisfied. Nevertheless we include conductivity, written as  a diffusion equation, where we take the conductivity to be constant.
Heat transport occurs in rocky interiors through conduction and convection. We model conduction using Fick's law and convection using a mixing length theory as described below.

The equation for conductive heat flux is thus
\begin{equation}
    F_\text{cond} = - k\diff{T}{r} \label{eq: conduction}
\end{equation}
assuming a constant conductivity, $k$.

Convection is extremely important for the evolution of rocky planets, both in the liquid and solid phases. To model convection we use mixing length theory \citep[see e.g.][]{Kippen, abe1995basic} and use the equation
\begin{equation}
    F_\text{conv} = - \rho l u C_P \left( \nabla - \nabla_\text{Ad} \right) \frac{T}{P} \diff{P}{r} \label{eq: Fconv}
\end{equation}
where $l$ is the mixing length, $u$ is the speed of convection and $\nabla_\text{Ad}$ is the adiabatic logarithmic temperature-pressure gradient (see \autoref{eq: nabladef}). 

The total flux is given by
\begin{equation}
    F = F_\text{conv} + F_\text{cond} \label{eq: totalF}
\end{equation}
which can be multiplied by $4\pi$ for our 1D model to give the luminosity $L$.

The mixing length prescription requires an estimate of the convective velocity $u$. To find the velcoity, we consider the forces $ \mathcal{F}_i$ acting on a parcel of the material moving due to convection. If the parcel is moving at a constant speed, the buoyancy force must be balanced by any drag forces. The drag forces are ram pressure, which is most important in the low viscosity limit, and viscous drag, which is more important in the high viscosity limit. These three forces may be given by the following formulae:
\begin{subequations}
  \begin{alignat}{5}
  &\mathcal{F}_\text{buoy} &=& \; V \frac{-\delta g \rho l}{P}\left( \nabla - \nabla_\text{Ad} \right) \diff{P}{r} \\
     &\mathcal{F}_\text{RAM} \; &=& \;  \rho u^2 A \\
     &\mathcal{F}_\text{visc} &=& \; 6\pi \nu\rho R u
    \end{alignat}
\end{subequations}
where $\nu = \eta/\rho$ is the  kinematic viscosity, and $\eta$ is the dynamic viscosity.  Here $R$, $A$ and $V$ are the radius, cross-sectional area and volume of the fluid parcel, respectively. 

Combining these results in a quadratic equation for $u$, yielding the solution
\begin{equation}
	u = \left( -1 + \sqrt{1- \frac{A V \delta g l \left( \nabla - \nabla_\text{Ad} \right)  }{\left(3\pi \nu R\right)^2 P}\diff{P}{r}}\right)\frac{3\pi\nu R}{A} \label{eq: quad_form}
\end{equation}
This has the limits
\begin{subequations}
    \begin{alignat}{4}
        u_\textit{visc} &=& -\frac{V \delta g l \left( \nabla - \nabla_\text{Ad} \right) }{6\pi \nu R P}\diff{P}{r} \\
        u_\textit{invisc} &=& \sqrt{-\frac{V \delta g l \left( \nabla - \nabla_\text{Ad} \right) }{PA}\diff{P}{r}}
    \end{alignat} 
\label{eq: our visc lims}
\end{subequations}
for high and low viscosity respectively.

We then choose geometric factors relating $R$, $A$ and $V$ to $l$, $l^2$ and $l^3$ in order to produce the velocity limits used by \citet{abe1995basic} for high and low viscosity. These are
\begin{subequations}
    \begin{alignat}{4}
        u_\textit{visc}  &=& \frac{-\delta g l^3}{18 \nu P}\left( \nabla - \nabla_\text{Ad} \right) \diff{P}{r} \label{eq: highviscv}\\
	    u_\textit{invisc} &=& \sqrt{\frac{-\delta g l^2}{16 P}\left( \nabla - \nabla_\text{Ad} \right) \diff{P}{r}} \label{eq: lowviscv}
    \end{alignat}
\end{subequations}
which it can be seen are equivalent to our \autoref{eq: our visc lims} up to numerical factors.

%\textcolor{red}{[confusingly written. eqns (14a,b) are presented above as if they are independently acquired somehow, but in fact, if i understand correctly, this is how $u_{visc}$ must be *defined* (apart from dimensional definitiions of R, AV, V in terms of l) in order to get eqn (13), correct? so should rephrase to make that clear. also, why does $u_{invisc}$ not appear in eqn (13) (or have an equiv eqn (13b)? ]}
Our formulation using \autoref{eq: quad_form} makes the transition smooth, which aids numerical convergence, as opposed to a switch at a critical value used in other works \citep[e.g.,][]{abe1995basic,Bower17}. 

\subsubsection{Application in partial melt}\label{sec: partial_conv}
A general formula for the adiabatic gradient in a medium is
\begin{equation}
    \nabla_\text{Ad} = \frac{P}{T} \frac{\delta}{C_P \rho} \label{eq: gen_adiabat}
\end{equation}
Under the assumption that the melt fraction $\phi$ is always in equilibrium, the adiabatic gradient is given by the ``wet adiabat'', where the latent heat alters the values of $\delta/\rho$ and $C_P$, and thus changes the value of $\nabla_\text{Ad}$ (\autoref{eq: gen_adiabat}). These properties are also adjusted throughout the other equations. 

The adjustments to the density $\rho$, thermal expansivity $\delta$ and heat capacity $C_P$ may be easily derived by assuming additive volumes and entropies of the melt and solid phases, i.e., $V = V_l\phi + V_s(1-\phi)$ and $S = S_l\phi + S_s(1-\phi)$, where $V$ and $S$ are specific volume and entropy. Consequently, the density is given by
\begin{equation}
    \rho = \frac{1}{\phi /\rho_l + (1-\phi) / \rho_s}
\end{equation}

To find $\delta$ and $C_P$ under partial melting one only requires the definitions
\begin{equation*}
    \delta = \frac{1}{V} \left.\diffp{V}{T}\right|_P  = -\frac{1}{\rho} \left.\diffp{\rho}{T}\right|_P \;\text{and  }\; C_p = T\left.\diffp{S}{T}\right|_P .
\end{equation*}
The results are
\begin{eqnarray}
    \frac{\delta}{\rho} &=& \phi \frac{\delta_l}{\rho_l} + (1-\phi)\frac{\delta_s}{\rho_s} + T\Delta V \left. \diffp{\phi}{T} \right|_P \label{eq: delv_prime} \\
    C_p &=& \phi C_{p,l} + (1-\phi)C_{p,s} + T\Delta S  \left. \diffp{\phi}{T} \right|_P \label{eq: Cp_prime}
\end{eqnarray}
where, subscripts $l$ and $s$ denote liquid and solid properties, respectively, $\Delta V \equiv \left(\frac{1}{\rho_l} - \frac{1}{\rho_s}\right)$ is the specific volume change of melting, and $\Delta S$ is the specific entropy change of melting. 
%\textcolor{red}{[subscript ``m" is now being used for the liquid (melt) phase, when ``liq" was used before (in the definition of $\phi$, i.e., for liquidus and solidus; also in eqn 23 etc below); use one consistently. i'd suggest ``liq" since ``m" is also being used to denote mass. also, solid is now denoted by ``s" when "sol" was used earlier; again, be consistent. ]}

Consequently, \autoref{eq: gen_adiabat} becomes
\begin{equation}
    \nabla_\text{Ad} = \frac{P}{T} \frac{ \phi \delta_l / \rho_l + (1-\phi)\delta_s / \rho_s + T\Delta V \left. \diffp{\phi}{T} \right|_P}{\phi C_{p,l} + (1-\phi)C_{p,s} + T\Delta S  \left. \diffp{\phi}{T} \right|_P } \label{eq: expanded ad_grad}
\end{equation}


For this work, we assume that the timescales of melting are shorter than those of mixing, so melt equilibrium is maintained, and our mixing length formulation fully captures the heat flow in the system. We return to this assumption in the discussion.
%{\bf I think that because of low fluxes in evolved systems, we can probably justify this.}

%{\bf OR}

%In reality, it is not clear that the timescales of mixing are much lower than the melting/crystalisation timescales. Consequently, we present cases where we use extra terms which take into account entrainment of solid in convective flow and its gravitational settling. These are taken from \citet{abe1993thermal} and \citet{Bower17}, and are as follows
%\begin{equation}
%    F_\textit{mix} = -\rho v l T \Delta S \diff{\phi}{r}
%\end{equation}
%for mixing, and
%\begin{equation}
 %   F_\textit{grac} = T \Delta S \frac{a^2 \rho g \left(\rho_s - \rho_l\right)F(\phi)}{\eta}
%\end{equation}
%for gravitational settling.
%{\bf The biggest issue is there are problems with actually solving with these in our code...}

\subsection{Physical properties of the mantle}\label{sec: physical properties}
To solve our problem we require the physical properties appropriate for the molten and solid rocky materials.

\subsubsection{Equations of state}
$\rho, C_P$ and $\delta$ are all obtained as functions of $P$ and $T$. For the solid phase, we use the equation of state for \ch{Mg Si O3} perovskite, the Earth's main mantle component, in \citet{Stixtrude2}. For the melt, we use the \ch{Mg Si O3} equation of state in \citet{RTpress}. 

For use in our numerical method, we precalculate these properties on a linear grid of temperature and pressure and retrieve values using bicubic interpolation. Under partial melting, density, thermal expansivity and heat capacity are given as a combination of the melt and solid properties as described above, in \S\ref{sec: partial_conv}.

\subsubsection{Volume and entropy change for melting}
In order to describe melting, it is necessary to know the change in specific volume, $\Delta V$, and entropy, $\Delta S$, between the solid and melt (see \autoref{eq: delv_prime} - \ref{eq: expanded ad_grad}). $\Delta V$ can be calculated from the densities given by the two equations of state. However, calculating $\Delta S$ from an equation of state requires an absolute entropy scale, which does not emerge naturally from the prescriptions we use. Instead, we calculate the entropy change through
\begin{equation}
    \Delta S = \Delta V \left.\diffp{P}{T}\right|_\phi \label{eq: DeltaS}
\end{equation}
which is derived in Appendix \ref{app: CC}. We use the estimate
\begin{equation}
    \left.\diffp{P}{T}\right|_\phi = \phi \,/\, \diff{T_\text{liq}}{P} + (1-\phi) \,/\, \diff{T_\text{sol}}{P}
\end{equation}
which simply interpolates between the $T-P$ gradient of the $\phi=0$ line, the solidus, and the $\phi=1$ line, the liquidus.

\subsubsection{Viscosity}\label{sec: viscosity}
The effective viscosity varies by many orders of magnitude as the melt fraction changes. A steep change from more liquid to solid-like behaviour is often taken to occur at a critical melt fraction, $\phi_c$ \citep[e.g.][]{SOLOMATOV-chapter}. This melt fraction is essentially the fraction of melt at the point when closely packed spheres of the typical size of crystals can no longer move past each other. We use an experimentally measured value of $\phi_c=0.4$ from \citet{Lejeune1995}. The transition between solid and liquid-like viscosities will be controlled by the value of the melt fraction relative to their critical value.

Additionally, the viscosity of the solid component is temperature and pressure dependent, which is an important influence on the long-term evolution of the mantle, and we model is using an Arrhenius law for diffusion creep \citep{Tackley13} 
\begin{equation}
    \eta_s(P,T) = \eta_0 \exp\left( \frac{E_0 + PV_0}{R_\text{gas}T} - \frac{E_0}{R_\text{gas}T_0}\right) \label{eq: arrhenius visc}
\end{equation}
where the $s$ subscript once again denotes solid and the values of parameters are shown in \autoref{tab: viscosity params}. This formula is valid when the material is solid, i.e., $\phi = 0$.

For low melt fractions ($ 0 < \phi < \phi_c$), we use an exponential parameterisation, for the dynamic viscosity \citep{kelemen1997}
\begin{equation} 
\eta = \eta_s(P,T_\text{sol}(P)) \exp(-\alpha_\eta\phi)
\end{equation} 
$ \eta_s(P,T_\text{sol}(P))$ denotes the viscosity of a solid at the solidus, i.e., just before any melting starts. We assume that diffusion creep is the dominant mechanism of solid deformation \citep[e.g.,][]{Tackley13}, and so use $\alpha_\eta = 26$, as found in \citet{Mei2002}. 

For high melt fractions ($\phi > \phi_c $), we use the formula from \citet{Roscoe52}
\begin{equation}
    \eta = \eta_l\left(\frac{1 - \phi_c }{\phi - \phi_c}\right)^{2.5} \label{eq: Roscoe}
\end{equation}
where the $l$ subscript denotes liquid. We take the liquid viscosity $\eta_l$ to be a constant, as its temperature dependence is small relative to the solid phase and to changes in melt fraction \citep{SOLOMATOV-chapter}. 

We combine these formulae into a smooth function, with no singularity at $\phi=\phi_c$, given by
\begin{equation}
    \eta = 
    \begin{cases}
        \eta_s(P,T) \;&, \; \phi = 0 \\
        \eta_s(P,T_\text{sol}) \exp(-\alpha_\eta\phi) \;&, \; 0 < \phi \leq \phi_c \\
        \frac{1}{\eta_s(P,T_\text{sol})^{-1}\exp(\alpha_\eta \phi) \, + \, \eta_l^{-1}\left(\frac{\phi - \phi_c}{1 - \phi_c }\right)^{2.5}} \;&, \; \phi_c < \phi < 1 \\
        \eta_l \;&, \; \phi = 1 
    \end{cases} \label{eq: full_viscosity}
\end{equation}
Since the solid viscosity is so much greater than the liquid (see \autoref{tab: viscosity params}) the $\phi > \phi_c$ formula is essentially the same as \autoref{eq: Roscoe}, unless the melt fraction is very close to $\phi_c$ .
\begin{table}
\resizebox{\columnwidth}{!}{%
\begin{tabular}{|p{0.35\columnwidth}|c|c|p{0.37\columnwidth}|}
\hline
{\bf Parameter} & {\bf Symbol [unit]} &  {\bf Value} & {\bf Reference }\\ \hline
Solid reference viscosity & $\eta_0$ [Pa s] & $\SI{1e21}{}$ & \citet{Tackley13}  \\ 
Viscosity activation energy & $E_0$ [kJ mol$^{-1}$] & $\SI{300}{}$ & \citet{Tackley13} \\
Activation temperature & $T_0$ [K]& $\SI{1600}{}$ & \citet{Tackley13} \\ 
Activation volume & $V_0$ [cm$^3$ mol$^{-1}$] & $\SI{5}{}$ & \citet{Tackley13} \\
Liquid viscosity & $\eta_l$ [Pa s]& $\SI{0.1}{}$ & \citet{SOLOMATOV-chapter}\\
Diffusion creep parameter& $\alpha_\eta$ & $26$ & \citet{Mei2002} \\
Critical melt fraction& $\phi_c$ & 0.4 & \citet{Lejeune1995} \\
Gas constant& $R_\text{gas}$ [JK$^{-1}$mol$^{-1}$]& 8.3145 &  \\\hline

\end{tabular}%
}
\caption{Values in our viscosity model.}\label{tab: viscosity params}
\end{table}

\subsection{The iron core}
Planets above a few thousand km in radii are likely to have formed an iron core through gravitational settling of the denser iron from the mantle \citep[e.g.,][]{Elkins-Tanton12-review}. We concentrate on planets with, a core mass fraction of 0.3, similar to the Earth. Were planets to have significantly different initial, silicate to iron ratios to the Earth, or undergo a major collision that might strip the mantle, then it is possible that the core mass fraction would be different for exoplanets. However, we find that differences in evolution are generally not significant to alter our conclusions.

%in order to account for the possibility of different compositions and formation channels. \textcolor{red}{[different composition and formation channels for what? clearly not for the core, since you're assuming it's always fully iron and formed via settling from the mantle! if mean different composition / formation channels for the planet as a whole, should say so, and also explain briefly how/why assuming different core mass fractions allows you to explore differences in composition/formation.]}}

We assume the core is pure iron and solve for its structure using our Henyey scheme in the same way as the mantle. For iron, we use physical properties for the $\gamma$ phase of iron from \citet{Dorogokupets2017}, as it is the appropriate phase at the pressures in the cores of the low mass planets we will consider \citep[e.g.,][]{TSUJINO2013}.%and see our \autoref{fig: internal no mdot}

For these calculations we neglect the fact that at early times the mantle will be molten, and set the temperature structure in the core to be adiabatic. The reason for these simplifications is that we are not interested in the evolution of the core, which itself is a topic with great uncertainty \citep[e.g.,][]{Zhang22}, only in the amount of energy it provides to the mantle, and this is likely not changed largely by these assumptions, as will be discussed in \S\ref{sec: discuss-other}.

\subsection{Radioactive heating}\label{sec: radioactivity}
For Gyr old planets, the energy generated by the decay of long-lived radioactive isotopes is an important contributor to the total luminosity. The significant elements in the Earth are \isotope[232]{Th}, \isotope[238]{U}, \isotope[40]{K} and \isotope[235]{U} \citep{turcotte2002geodynamics}. These elements are lithophiles, which means they preferentially dwell in the mantle. Over shorter periods, short-lived radioisotopes are more important; in the Solar System, the most significant of these were \isotope[26]{Al} and \isotope[60]{Fe}. For our fiducial case, we include in our models the equivalent radiogenic energy contribution per unit mass as is found in the Earth's mantle. We used data from \citet{turcotte2002geodynamics} normalised to give Earth levels at the age of the Earth (4.5 Gyrs). We also include \isotope[60]{Fe} in the core at the concentration estimated for the Solar System in \citet{lugaro18-review}. %These contributions can be varied within reasonable bounds set by galactic evolution.

\subsection{Outer boundary conditions}\label{sec: BCs}
\begin{figure}
    \centering
    \includegraphics[width=\linewidth]{L_T0func_Tss_2320.pdf}
    \caption{Temperature at a pressure $P_0 = 1$ GPa, $T_0$, for boundary conditions with no redistribution and $T_{ss} = 2320$K, as a function of the mean flux from the planet (luminosity divided by surface area). See \S\ref{sec: BCs}.}
    \label{fig: T0_func}
\end{figure}

The catastrophically evaporating systems we are investigating are very close to their host stars and so are likely tidally locked. They, therefore, have permanent daysides which are highly irradiated. This means that their outer boundaries are not spherically symmetric. We take this into account in our evolutionary models by adapting the outer boundary conditions. The full problem of heat transport in the planet is complex due to the mixture of radial and azimuthal heat transport and the effect of heat transport in any magma pool; we, therefore, make a series of assumptions as explained below. More detail of the calculations are shown in Appendix \ref{app: BC}, but we explain the principles of the approach here.

We first assume that below some depth within the interior, specified by a pressure $P_0$, the planet can be approximated as spherically symmetric. The pressure at the base of this outer layer is the outer edge of our numerical grid, and we find the temperature $T_0(L,R)$ at this pressure using our boundary conditions. We generally show models with $P_0 = 1GPa$ and discuss the consequences in \S\ref{sec: BC caveats}.

This approach is justified if any inward flux due to the star is much less than any outward flux from the planet cooling, which we will discuss further in \S\ref{sec: BC caveats}. \citet{Kite16} demonstrated that there will mostly be azimuthal heat transport in the magma pool, and as a consequence, the pool will be shallow ($\sim 150$ m). Since the base of the pool is, by definition, at a point where rock is mostly solid, it must already be much cooler than the surface (typically $\sim 1600$ K as opposed to $\sim 2000$ K).

Next, we consider the outer skin, where stellar irradiation does have an impact. Here the complexities of azimuthal heat transport are more important, so we proceed by making further simplifying assumptions. Specifically, we assume that there is no redistribution of the star's energy at the planet's surface. The first reason for this is that since the planets are tidally locked, there can be no redistribution due to rotation. Secondly, any atmosphere generated through outgassing volatiles will likely be thin \citep[for instance, the atmospheric escape models of][generate atmospheres with maximum pressures of $\sim 10^{-5}$ bar]{Booth_disint22}, and so we expect unable to transport heat efficiently. Thirdly, \citet{Kite16} showed that a surface magma ocean cannot transport enough heat laterally to decrease the temperature gradient imposed by irradiation.

As a result, there will be a steep temperature gradient from the day to the nightside. If we model the surface as acting like a black body, the surface temperature, $T_s$, as a function of angle $\theta$ from the substellar point is given by
\begin{equation}
    T_s(\theta) = \left(\frac{F(\theta) + F_*(\theta)}{\sigma}\right)^\frac{1}{4} \label{eq: F_surface BB}
\end{equation}
where $T_s(\theta)$ is the surface temperature and $F_*$ is the irradiation from the star, and $F(\theta)$ is heat flux from the interior. Assuming the planet is far enough away that irradiation is plane-parallel, this is given by
\begin{equation}
    F_*(\theta) = \begin{cases}
        \sigma T_*^4\frac{R_*^2}{a^2} \cos{\theta} \; &, \; 0 \leq \theta < \pi / 2 \,\,\,\,\,\text{(dayside)}\\
        0 \; &, \; \pi / 2 \leq \theta \leq \pi \,\,\,\,\,\text{(nightside)}
    \end{cases}
\end{equation}
where $R_*$, $T_*$ and $a$ are the stellar radius, effective temperature and planet's semi-major axis, respectively.

In the outer layer, we will make the further assumption that the azimuthal fluxes are small. This will also be justified later. The consequence of this assumption is that it allows us to find the temperature-pressure structure at any $\theta$, assuming constant $g$ and $F$, by considering the vertical heat flow at that angle. This is because the conduction/convection equation (Equations \ref{eq: conduction}-\ref{eq: totalF}) becomes simply a differential equation for $T(P)$ in this case, which is shown in full in Appendix \ref{sec: cond_conv for BC}.

For the boundary, to our 1D model, we require a relation between the temperature $T_0$ at the outer edge of the model and the luminosity $L$. We establish it by finding solutions to the conduction/convection equations which satisfy that the total flux integrated over the surface (integral $F(\theta)$ in \autoref{eq: F_surface BB}) is equal to $L$ and $T = T_0$ at $P_0$ for all angles. In other words, the total energy in at the bottom is equal to that at the top, and the bottom is spherically symmetric. The details of this procedure are documented in Appendix \ref{sec: BC grid_method}. 

In this calculation, we do not allow any inward heat flux and so set $F(\theta) = 0$ if the surface temperature is higher than $T_0$. As we have stated, we believe any inward heat flux will be small, so this assumption should not affect the overall evolution.

The result of these calculations is a relation between luminosity and temperature at the edge of the 1D domain, as shown in \autoref{fig: T0_func}. To aid the numerical procedure, we fit a smooth function to our calculated model, detailed in Appendix \ref{sec: fit BC}. 

The basic features of the plot are that at high temperatures, the flux increases with temperature, essentially because of black body cooling. There is then a large range of fluxes when the temperature changes little. This is because, when $T_0$ is close to the critical melt fraction, the viscosity at that point changes by orders of magnitude for very small changes in temperature (\S\ref{sec: viscosity}), thus the amount of energy that can be transported also drastically changes. At low fluxes, the planet has solidified, and the outer region is mostly conductive, so the temperature at $P_0$ again becomes a strong function of the flux. 

%\subsubsection{Thermal Boundary layer}
%Under vigorous convection, steep thermal boundary layer can develop on the edge of the convecting medium. This is because the size of the region over which convection can occur, shrinks closer to the boundary, making convection inefficient, resulting in a thin conductive layer \citep[e.g.][]{Kraichnan62}. We therefore, set the outermost temperature, $T_\text{out}$, of our grid to satisfy
%\begin{equation}
 %   T_\text{out} - T_s = 0.92 \, L^\frac{3}{4} \left(\frac{\eta T}{R^4 \left(\frac{\delta}{\rho}\right)\rho^3GMk^2C_P}\right)^\frac{1}{4}
%\end{equation}
%where $T_s$ is the surface temperature. This formula is based on the derivation in \citet{Kraichnan62}, and encodes this switch from convection to conduction near the boundary. 

%While we include this formula in our code, for evolved systems, its effect is small due to low luminosity, and $T_\text{out} \approx T_s$.

\subsection{Grid and timestepping}\label{sec: grid/time}
We solve the equations in \S\ref{sec: struct_eqns} on a mass grid where the size of the cells varies as $m^{\alpha}$. $\alpha$ is chosen to give a compromise between cells getting smaller in radius towards the edge, where high resolution is required to deal with the final crystallisation of the mantle and maintain a reasonable resolution in the centre of the planet. As density is close to constant, the first condition requires $\alpha \lesssim 3$. For the second condition, it helps to have $\alpha > 1$. We find $\alpha = 1.5$ works well.


\subsection{Mass Loss}\label{sec: massloss}
\begin{figure}
    \centering
    \includegraphics[width=\linewidth]{Mymdot_grid.pdf}
    \caption{Contours of gas mass loss rate for planets of different mass and substellar temperatures and fixed density as calculated by \citet{Booth_disint22}. It differs slightly from their Figure 5, which is calculated for a fixed mass-radius relation. Dust mass loss rates of $10^{-6}$, $10^{-4}$ and $10^{-2}$ $M_\oplus \text{Gyr}^{-1}$ are shown with the dotted, dashed and solid black contours. The irregular shape is due to the finite number of data points.}
    \label{fig: mdot_grid}
\end{figure}
Since our aim is to model the evolution of rocky planets undergoing evaporative mass loss, we incorporate mass loss as follows. 

At each timestep, we ascribe a mass loss rate determined by the current mass, radius and surface temperature. We use mass loss rates computed using the method described in \citet{Booth_disint22}, which assumes that the dusty outflows form $\SI{1}{\micro m}$ grains with properties similar to forsterite. An example is shown in \autoref{fig: mdot_grid}. We pre-tabulate mass-loss models for a given substellar temperature/ semi-major axis (related by equation \ref{eq: Tss}) for a grid of planet masses and densities. This allows us to find the mass loss rate at each timestep as the planet evolves in mass and density through its evolution.

When the mass loss rate has been determined, we then calculate the mass change to the next timestep according to our suggested timestep (see \S\ref{sec: grid/time}.) In order to make the numerical problem easier to solve, rather than removing the mass from only the outermost cell, as would be the closest to the physical reality, we instead shrink a finite number of the outermost cells and keep the mass contained in cells interior to this the same. This prevents a drastic change in the size of the outermost cell which might be numerically unstable, allowing more mass to be removed per timestep. Before doing so, we check that the mass loss results in the mass contained in these cells being reduced by no more than a certain fraction, and if the mass step is too great, we reduce the timestep so that this condition is satisfied. Our default values are the outermost $5\%$ of total cells being shrunk and by no more than $1\%$. This procedure requires us to merge cells so that we can continue to remove mass after the cells shrink and to split cells, to ensure that individual cells do not become too large a fraction of the total planet's mass. We make these grid changes at the beginning of each timestep, when appropriate thresholds are reached and use piecewise cubic Hermite interpolation \citep[PCHIP,][]{PCHIP}, where interpolation of values is required.

During mass loss, the gravitational potential of the planet is altered. Since mass loss occurs at the beginning of the timestep, this energy change is not included in \autoref{eq: lumin}. We, therefore, introduce an extra source term to that equation, using the prescription used in {\it MESA} \citep[see][\S3.3]{MESA19}. This approach considers the mass moving through and out of cells and the energy deposited by it, and we also assume that the mass loss timescale is longer than the thermal timescale. We generally find this energy contribution is small.

\section{Results} \label{sec: results}
\subsection{No mass loss}
\begin{figure*}
    \centering
    \includegraphics[width=\linewidth]{Evolve_Mdot0M0.15Tss2320_P.pdf}
    \caption{Evolution of the properties of the internal structure of a 0.15 $M_\oplus$ planet with core mass fraction 0.3, and a substellar temperature of 2320K (see \S\ref{sec: BCs}), but no mass loss. Note that the right-hand panels show only the mantle, so the pressure scale is different. The black solid lines on the temperature plot mark the liquidus (upper line) and solidus (lower line), which only apply to the mantle.} 
    \label{fig: internal no mdot}
\end{figure*}
\begin{figure}
    \centering
    \includegraphics[width=\linewidth]{Bulk_Mdot0_M0.15Tss2320.pdf}
    \caption{Evolution of the mean surface flux ({\it top panel}), temperature ({\it middle panel}) and the radius and molten state ({\it bottom panel}) of a 0.15 $M_\oplus$ planet with a core mass fraction of 0.3 and a substellar temperature of 2320K (see \S\ref{sec: BCs}), but no mass loss. In the bottom panel, regions to the left of the liquidus are entirely liquid, to the left of the $\phi = \phi_c$ line are partially molten, to left of the solidus behave like a solid but have some partial melting and to the right of the solidus are fully solid. The region above the 1GPa line is unresolved by the 1D model and will have melting on the dayside surface but be cold and solid on the night side. We also do not plot any partial melting at the core-mantle boundary, which does occur in our models, at certain times, due to the thermal boundary layer there (see \autoref{fig: internal no mdot}) because we are more interested in the state towards the surface.} 
    \label{fig: bulk no mdot}
\end{figure}
To demonstrate our model, we first show a fiducial case without mass loss, as the features of the thermal evolution are clearer in this case. We choose a substellar temperature of 2320 K as it is the fiducial value used in \citet{Booth_disint22}\footnote{\citet{Booth_disint22} choose this value as it is the substellar temperature of Kepler 1520b, for the stellar parameters they use. In \autoref{tab: systems} we show a temperature with more up to date parameters.}, where we take our mass loss models from (see next section and \S\ref{sec: massloss}). The substellar temperature is the temperature of the point on the planet closest to the star given no redistribution of energy or heat flow from the interior. As the assumption of no redistribution is likely well justified for the short period planets we consider (see \S\ref{sec: BCs}), it is, therefore, a proxy for the maximum temperature of a planet, and thus is informative in telling whether a planet might evaporate. It is defined by
\begin{equation}
    T_{ss}^4 = T_*^4 \frac{R_*^2}{a^2} \label{eq: Tss}
\end{equation}
%This is the temperature of Kepler 1520b, using stellar parameters of $M_* = 0.76 M_\odot$, $L_* = 0.22L_\odot$, $T_* = 4677$K \citep{Morton_2016} and a semi-major axis of $a = 0.0134$~au \citep[corresponding to a period of 15.7 hrs,][]{KIC1255-discov}.

Snapshots of the internal structure are shown in \autoref{fig: internal no mdot}. Time snapshots are roughly spaced logarithmically in time, but with extra times between $10^3 - 10^4$ yrs to demonstrate the crystallisation of the magma ocean and between $10^9 - 10^{10}$~yrs to show the late time evolution. Evolution of some overall properties are shown in \autoref{fig: bulk no mdot}.

As the planet's temperature decreases (\autoref{fig: internal no mdot}, {\it Panel a}), it crystallises from the inside out ({\it Panel c}), which is essentially due to the shape of the solidus /liquidus relative to the temperature structure, which is close to adiabatic (this is a well-known feature, e.g., \citealt{walker1975differentiation}, although, depending on the unconstrained shape of the liquidus, `middle-out' crystallisation has also been proposed, e.g., \citealt{Middle-out-Stix2009}). As the melt fraction passes through the critical point of $\phi = \phi_c = 0.4$ a many order of magnitude change in viscosity is observed ({\it Panel d} and see \S\ref{sec: viscosity}). The increase in melt fraction close to the core-mantle boundary and decrease towards the surface, seen at later times, is due to the fact that mixing length is equal to the distance to the nearest boundary, generating thermal boundary layers, which can also be seen in {\it Panel a}. In addition, the solid viscosity is highly temperature dependent (\autoref{eq: arrhenius visc}). This results in positive feedback, where the viscosity increases towards the surface, making convection less efficient and making the thermal boundary layer stronger, so temperatures lower and the viscosity increases. The effect is most noticeable at late times in {\it Panels a} and {\it d}. 

The density, meanwhile, increases as the planet cools ({\it Panel b}), resulting in an increase in pressure at the core-mantle boundary and at the centre (the horizontal extent of lines in {\it Panels a} and {\it b}). 

In the first two panels of \autoref{fig: bulk no mdot} are shown the evolution of flux and temperature, which are best understood with reference to \autoref{fig: T0_func}. There is an extended period of time where the temperature at $P_0$ does not change, but the luminosity continues to drop due to the large changes in viscosity around $\phi_c$. One can also note that the flux levels off slightly at $\sim 10^8$~yrs before decaying over the following $10^{10}$yrs, which is due to the radioactive decay of long-lived radioisotopes becoming the dominant heat source over primordial energy.

The final panel shows the planet shrinking, which is due to the density increase over time (see {\it Panel b} of \autoref{fig: internal no mdot}). Both the core and mantle shrink over the whole evolution, but the largest effect on the radius is the change from liquid to solid in the mantle, causing a marked decrease in total radius in the first few thousand years.

As well as the planet shrinking, the final panel of \autoref{fig: bulk no mdot} also shows the molten state of the planet. One sees that within a few hundred years, the planet is no longer fully molten, and within a few thousand, it is essentially solid (the critical melt fraction point has reached the surface - dot-dashed line). %However, the partially molten phase lasts Gyrs, which makes sense as partial melting is required for volcanism. 
In essence, however, despite the high substellar temperature, the planet is solid for the entire age of the universe because it can cool from the night side. What is not included in this plot, of course, is that a magma pool will persist on the dayside, from which evaporation can occur and will be from where mass loss occurs.

\subsection{Mass loss}
\begin{figure}
    \centering
    \includegraphics[width=\linewidth]{Mass_evoMs.pdf} 
    \caption{Evolution of the mass and mass loss rates of planets with different initial masses and substellar temperatures, which lose mass according to the method of \citet{Booth_disint22}. All planets start with a core mass fraction of 0.3.}
    \label{fig: mmdot}
\end{figure}

\begin{figure*}
    \centering
    \includegraphics[width=\linewidth]{crystalisation_evoMs.pdf}
    \caption{Evolution of the molten state of planets with different initial masses and substellar temperatures, which lose mass according to the method of \citet{Booth_disint22}. 
    %Note that the y-axis shows the fraction of the total radius, which decreases with time due to cooling and mass loss. This is why the core radius in \autoref{fig: crystalisation mdot} (solid red lines) increases since the mantle is being removed, but the core is not. 
    All planets start with a core mass fraction of 0.3.}
    \label{fig: crystalisation mdot}
\end{figure*}

\begin{figure}
    \centering
    \includegraphics[width=\linewidth]{depth_melt_evoMs.pdf}
    \caption{The top panel shows the evolution of the maximum depth of partial melting for planets with different initial masses and substellar temperatures. Increases in depth for certain times are due to an increase in heat flux due to the decrease in gravity, shown in the second panel. Simulations that end with a dot had full crystallisation up to 1GPa. Those that end with a cross reached the maximum density that mass loss rates were computed to. }
    \label{fig: depth}
\end{figure}
\autoref{fig: mmdot} shows the evolution of the total mass, radius and mass loss rate for a number of models with different initial planet masses and substellar temperatures. Mass loss rates increase with increasing substellar temperature and decreasing initial planet mass (as seen in \autoref{fig: mdot_grid}), which is seen in the faster mass and radius decrease for such planets. We mark on the mass loss rate panel an `observable' cut-off mass loss rate of $10^{-1} M_\oplus$~Gyr$^{-1}$, motivated by calculations by \citet{Perez-Becker13}. This mass loss rate is based on the amount of material expected to be lost now by the observed planets, and so is a simplified condition. We return to considering when the catastrophically evaporating planets are observable in more detail in \S\ref{sec: occurrence}. However, even with this simple condition, some interesting features can be noted about planets that are observed. Some planets are observable throughout their lifetimes, but their lifetimes are short ($\lesssim 10^8$yrs). These are planets with sufficiently high substellar temperatures (dotted lines). At the other extreme, much cooler planets (solid lines) or massive planets (red curves) are never observable within the age of the universe. There is also an intermediate regime of planets that become observable late in their lifetimes once they have lost sufficient mass. This was also noted by \citet{Perez-Becker13}, for their earlier mass loss models. We discuss the consequence of this on the number of systems we might observe in \S\ref{sec: occurrence}.

In \autoref{fig: crystalisation mdot}, we show the evolution of the molten state of the mantle. In all the models, the mantle still entirely crystallises to a melt fraction below $\phi_c$ within $\sim 10^4$yrs, meaning the planets are essentially solid at the point when evaporation occurs. Furthermore, in many models crystallisation occurs fully by around 1 Gyr. The exceptions are the highest temperature and lowest mass cases, where the planet is evaporated before the melt fraction is zero. 

However, even if the melt fraction is not zero, it is very small in all cases and the planet is essentially solid, as shown in the lower panel of \autoref{fig: depth}. In some cases, the melt fraction increases at late times. The reason for this is a subtle effect. The decrease in gravity allows more melting in the upper regions, although this should not allow more melting at a fixed pressure of $P_0$. The increased melting at $P_0$ is due to the fact that this increased melting allows an increase in flux to the surface, due to lower viscosity, as shown in the upper panel of \autoref{fig: depth}. This is also responsible for the small amount of remelting at the base of the mantle in the 2070 K, 0.04 $M_\oplus$ case. This effect may be important for any volcanism on the planet, and may alter the detailed composition through solid-melt partitioning, but since the amount of melting is small, it does not affect the overall conclusion that the catastrophically evaporating planets, are essentially solid, once major mass loss has occurred.

%The decrease in the maximum depth of melting at the end of the lifetime of the $0.04M_\oplus$, 2320K case in the first panel of \autoref{fig: depth} is explained by comparison with \autoref{fig: crystalisation mdot}, where it can be seen that the whole mantle becomes partially molten. This decrease is due to the planet's mass loss driven decrease in radius. The final panel of \autoref{fig: depth} shows that the melt fraction is small even in this case. 

To summarise, these highly irradiated planets solidify to a mostly molten state within a few thousand years and can solidify completely within a few Gyrs. This is because they are able to cool easily from their exposed nightsides. As a consequence, major melting is solely on the dayside, as has been proposed by e.g., \citet{Kite16}. Mass loss does not have a major effect on this, although may induce some more partial melting, due to the decrease in pressure. 

\section{Discussion} \label{sec: discuss}

\subsection{Depth of the magma pool and implications for chemistry}\label{sec: moltenmass}
\begin{figure}
    \centering
    \includegraphics[width=\linewidth]{Pool_depth1_M0.2Tss2320.pdf}
    \caption{The depth of a dayside magma pool under different assumptions, and using a boundary region extends to a different pressure, $P_0$. Firstly, a maximum depth is found by calculating the position of the critical melt fraction assuming conduction to the edge of the boundary layer. This naturally depends strongly on the value of $P_0$. Secondly, in the case where magma pool circulation is included, which produces a much shallower pool, as found in \citet{Kite16}. See \S\ref{sec: moltenmass} for further details of the calculations.}
    \label{fig: pooldepth}
\end{figure}

We found in \S\ref{sec: results} that catastrophically evaporating planets are likely to have almost entirely crystallised by the time a significant amount of mass has been lost. However, the peak day side temperature is higher than the liquidus, so there will be a molten region on this day side. 

The most naive way to calculate the depth of this pool is to assume a constant temperature gradient between the surface and a point in the interior where the temperature is known. The depth of the pool is then the point along this temperature gradient where the material becomes solid, best defined as when the melt fraction reaches the critical value ($\phi = \phi_c$). We use $T_0$, the temperature at a pressure $P_0$, the edge of our 1D model (see \S\ref{sec: BCs}) as the interior temperature, and thus the depth is dependent on what we choose for $P_0$, as shown in \autoref{fig: pooldepth}.

\citet{Kite16}, however, argue that lateral circulation in the lava pool greatly reduces the depth of the pool, compared to this value. They argue, using scaling laws from \citet{Vallis2006}, that the lava pool should have a thermal boundary layer at the surface, due to circulation, which extends to a depth of
\begin{equation}
    \delta_T = \left(\frac{k}{\rho C_P} \frac{4 \Omega \sin(\theta_p/2)\tan(\theta_p/2) R^2\theta_p}{(\Delta\rho/\rho) g} \right)^\frac{1}{3} \label{eq: pool BL}
\end{equation}
\citep[adapted from][Eq. 8]{Kite16}. Here $R$ is the planetary radius, $\Omega$ its orbital frequency and $\theta_p$ the angular size of the pool, centred at the substellar point, and $\Delta\rho$ the density change across the boundary layer. The total depth of the pool should not be more than $\sim 10$ times this boundary layer.

$\theta_p$, may be found, for a given substellar temperature, by using the fact that the surface temperature, as a function of $\theta$ will be approximately $\propto \cos^{1/4}\theta$ (see \S\ref{sec: BCs}) and finding the point that the critical melt fraction is reached on the surface. A value of 2100K, representative of the known systems KOI 2700b and K2-22b (\autoref{tab: systems}) gives, using our melting curves (\S\ref{sec: melting}), a very wide pool with $\theta_p \approx 0.4\pi$. If one takes surface values of the physical quantities, assuming $\Delta\rho = 10\%$, a typical value for melt-to-solid density change, at a one-day orbit, then one finds the scaling
\begin{equation}
     \delta_T \approx 18\text{m} \left(\frac{R}{R_\oplus}\right)^\frac{2}{3}\left(\frac{g}{g_\oplus}\right)^{-\frac{1}{3}} \label{eq: pool scaling}
\end{equation}
Since for an individual planet, the other terms in \autoref{eq: pool BL} do not change, \autoref{eq: pool scaling} may be used to track the evolution of the pool's depth for that planet.

Following \citet{Kite16} we assume that the depth of the pool is $10\delta_T$, and plot this depth for one case in \autoref{fig: pooldepth}. This demonstrates the scale difference between this case including circulation and the simple argument without, as noted by \citet{Kite16}. It is interesting to note that for early times, when the evolution is due to cooling, the depths of the pool with both estimates decreases, whereas at later stages, when mass loss is the most important factor, the basic estimate increases, whereas the estimate with circulation decreases. The reason for the different behaviour is that the only thing governing the estimate with circulation is the mass and radius of the planet (\autoref{eq: pool scaling}), and the planet is always getting smaller and denser, so its evolution is monotonic. However, for the simple estimate at early times, the most important factor is the temperature difference between the substellar point and $P_0$, which decreases as the planet cools, resulting in a shallower pool. However at late times, the decrease in gravity, means that the depth to a certain pressure increases, meaning the pool depth increases. However, it should be noted that the overall change for the circulation case is small, so the evolution is likely to be altered by effects not captured in the arguments it is derived from.

The important point, however, is that for either case it is likely that the planets have only a shallow molten region on their dayside, relative to the radius of the planet. This has consequences for the composition of the dusty tails which are the observational signatures of catastrophically evaporating planets are their dusty tails. Future measurements may be able to determine their chemistry in even more detail. The evaporation that generates the tails must come from the molten region of the planet. Therefore, our results show that the composition of the tails should reflect that only of the surface of the planet. This may mean that relatively volatile elements such as \isotope[]{Na} are still present in the winds because they are contained in the solid region and only a thin veneer of which is melted continually as the planet evaporates. A more detailed consideration of this chemistry is required to determine this, which will come in a future study. 

\subsection{Assumptions in the boundary conditions} \label{sec: BC caveats}
As discussed in \S\ref{sec: BCs}, due to the complexity of the radial and azimuthal heat distribution of the star's energy, we made some assumptions about the outer layers of the planets. We argued that the inward heat fluxes should be small compared to the outward ones, which justified having the inner regions of the planet evolve as an azimuthally symmetric structure, unaffected  by the star. At very late times when all internal energy has dissipated, this assumption is clearly untrue. Therefore, in the following, we shall check whether the inclusion of any inward or azimuthal fluxes would effect our conclusions.

At very late times, for tidally locked planets, the temperature should decrease from the dayside to the nightside, and energy should flow across. Heat flux in this case should be of the order
\begin{equation}
    F = k \frac{T_{ss}-T_n}{2R}
\end{equation}
where $T_n$ is the nightside temperature, and $k$ is again the conductivity. For a 1000 km planet with $T_{ss} = 2320$ K, and a negligible nightside temperature this gives a value of $\sim \SI{5e-3}{Wm^{-2}}$. The blackbody temperature required to emit this on the nightside is 17 K, justifying the assumption of the nightside temperature being negligible. One can see in \autoref{fig: bulk no mdot}, that around 10 Gyrs, the outward fluxes do get this low. However, the planet crystallises significantly before this point when the outward flux was at least 10 times higher. Therefore, the additional flux contribution would not be enough to melt the planet.
%As an additional test, we also recalculated our boundary condition, changing the assumption from no inward flux to an inward flux corresponding to conduction from the surface to $P_0$. This condition produces higher inward fluxes than one would expect physically, because the temperature at $P_0$ is still fixed at all angles, whereas in reality, the dayside should be warmer than the nightside, decreasing any inward flux for a given average temperature. In this case, the boundary condition is the same as our original version down to a 1 Gpa temperature of $\sim 1600$~K, which is below the solidus. At this point, the temperature begins to decrease less steeply with flux, before reaching an equilibrium between inward and outward flux at $\sim 1000$~K. Thus, even in this extreme case, our conclusion about the planet's crystallisation is unaffected.

We also assumed that azimuthal fluxes were small, which allowed us to estimate the flux from the nightside, simply as that coming from the interior, not the dayside. Azimuthal fluxes should also be of a similar order to the late-time day to nightside flux, estimated above, thus those fluxes become important at a similar time. Consequently, any adjustment to the nightside temperature would also only become significant once crystallisation has occurred, meaning azimuthal fluxes are also not able to affect our conclusion about the planet having solidified.

In our results, we have shown examples where the pressure at the edge of the 1D interior model is fixed at $P_0 = 1$ GPa. We choose this pressure as regions deeper than it are unlikely to be affected by irradiation (see above). Setting $P_0$ to be much deeper would result in the assumptions of constant gravity and flux in the outer region becoming less accurate since more than 10\% of the total mass of typical planets would be included in the boundary region. Therefore, to investigate the effect of choosing this arbitrary pressure, we instead compared to a lower pressure case of $P_0 = 0.5$ GPa. We found that in this case full crystallisation takes slightly longer but cooling proceeds slightly faster at late stages. Both these effects may be explained as consequences of some missing physics in the outer layer with pressure below $P_0$. The first point may be explained by the fact that the 0.5 GPa case the planet has more energy available. This is due to the fact that the outer layer, which does not include any energy production due to cooling or radioactive decay, is a smaller proportion of the planet. The second effect is due to the fact that at late times, the energy production is less important, and now the fact that the  outer layer effectively restricts the outward flux to the nightside is more significant. This means the case with a smaller value of $P_0$ has increased temperature loss. Overall, however, the effect of halving $P_0$ is small, so it is likely that the difference compared to a fully physically consistent model is also small.

\subsection{Further caveats}\label{sec: discuss-other}
We have used a relatively simple model to investigate the evolution of the catastrophically evaporating planets, thus there is necessarily some physics missing. For instance, the composition is fixed and uniform in the mantle. In reality, there may be planet-to-planet variation and compositional evolution in the planet. However, these are unlikely to change the properties of density and heat capacity enough to significantly alter conclusions. The composition would also affect the solidus and liquidus, which may make a more significant difference due to the difference between melt and solid having such a large effect on viscosity (see \S\ref{sec: viscosity}). A full investigation of this is beyond the scope of this work, but given the whole mantle passes below the critical melt fraction fairly early, such differences are likely unimportant in the long term. 

We also assumed that the melting is in equilibrium, meaning temperature and pressure directly determine the melt fraction. Non-equilibrium effects are potentially important for compositional evolution during the early magma ocean phase \citep[e.g.,][]{SOLOMATOV-chapter}, but once again since it passes through the critical melt fraction point early on, these effects are unlikely to be important for long term evolution.

Another simplification we made is in the treatment of the iron core. Since we are most interested in the evolution of the mantle, the precise state of the core is not important, we are simply interested in the flux from it. In reality the core will start molten and solidify over time, which generates extra energy through the latent heat of fusion and through gravitational potential energy as the iron density increases through the phase transition, resulting in contraction of the core. However, we do not include such a phase transition.

For our example in \autoref{fig: internal no mdot}, the melting point of iron at the core mantle boundary is around 2450K \citep{SLUITER2012}. This means that the planet must be between $10^8 - 10^9$ years old, and the mantle essentially crystallised, by the time any of these extra heat sources start to come into effect.

The total energy released as latent heat may be estimated as $E_L \sim M_c \Delta L_{Fe} $, where $\Delta L_{Fe}$ is the latent heat of fusion of iron, and $M_c$ the mass of the core. The energy from gravitational contraction can be estimated with $E_g \sim GM_c^2/R_c(\frac{1}{3}\Delta \rho_c/\rho_c)$, with $R_c$ the core radius, $\rho_c$ the density of iron and $\Delta \rho_c$ the solid melt density contrast. Typical values for, for instance a $0.15 M_\oplus$ planet with a core mass fraction of 0.3, give total values of several $10^{28}$ J for these two energy sources. This is of order the amount of energy lost by the core by cooling, as calculated by our models. Thus these additional energy should have some effect on the mantle at late times. However, it seems physically unlikely that the luminosity of the core should increase at this point, as the energy would instead remelt the core. Certainly, the core temperature cannot increase, since it cannot rise back above its melting point. Furthermore, the mantle itself typically has energies an order of magnitude above that of the core, both in thermal energy and radioactive elements. Therefore, it seems unlikely that the inclusion of these core freezing effects would affect the fact that the mantle has crystallised first. 

Another energy source we have neglected is tides. The short orbits of the planets, mean that tidal forces are potentially strong, although this also decreases the time it takes to circularise the orbits, at which point the tidal heating stops. \citet{Jackson2008}, suggest that even for small, but non zero eccentricities heating rates can be high, perhaps even high enough to keep the mantle molten. Considering these effects would require consideration of the orbital evolution, which is beyond the scope of this work.

Orbital evolution could also occur due to angular momentum exchange between the outflow and the planet, thus changing the planet's irradiation level. This might result in the planet spiralling outwards or inwards, cutting off mass loss or causing a runaway. Neglecting this effect is justified if the material from the planet is expelled from the system by the star, retaining its own angular momentum, meaning no angular momentum is exchanged with the planet. This fits with the current understanding of the dusty tails \citep[e.g.,][]{disint18}.

\section{Occurrence rate of low mass planets}\label{sec: occurrence}

Our evolutionary models demonstrate that there is only a small period of time, relative to the main sequence age of stars, that the catastrophically evaporating planets have observable mass loss but has not yet evaporated. The three systems that were found in the Kepler data must be in this region. By considering the detectability of these systems it is therefore possible to estimate the overall occurrence rate of small planets, within the region of parameter space that might produce catastrophic evaporation. This is of particular interest since it probes planets of lower mass than are generally undetectable by exoplanet surveys. 

The inferred abundance will depend on the precise mass loss model used as that will change the amount of time planets are observed and which initial masses are susceptible to evaporation. Here we present a calculation for the mass loss model of \citet{Booth_disint22}, as we have used in the rest of this work, but the method could equally be applied to any model. 


\subsection{Method}
\begin{figure}
    \centering
    \includegraphics[width=\linewidth]{optimisticTime_min_max.pdf}
    \caption{The observable time period of catastrophically evaporating planets using planets with core mass fractions of 0.3, mass loss rates calculated using the method of \citet{Booth_disint22} and assuming that a planet becomes observable if its mass loss rate exceeds 0.1~$M_\oplus \text{Gyr}^{-1}$ \citep{Perez-Becker13}.  {\it Top panel:} The dotted line shows the age at which a planet of a given substellar temperature gains a mass loss rate high enough to be observable. The dashed line shows the time when the whole of the planet's mantle evaporates, while the dash-dot shows when the mass loss rate becomes unobservable, which mostly coincides with the mantle evaporation line. {\it Bottom panel:} The length of time that a planet has an observable mass loss rate, as a function of initial mass and substellar temperature. Mass loss rates are observable when they are above a certain threshold, but the planet's mantle has not fully evaporated.}
    \label{fig: optimisticTime_min_max}
\end{figure}

\begin{figure}
    \centering
    \includegraphics[width=\linewidth]{optimisticN_T_Mp.pdf}
    \caption{The probability density that a planet of a given initial mass would be observable around a star, marginalised over the stellar mass and age distribution of stars observed in the Kepler primary mission. Here we consider a planet observable if its mass loss rate exceeds 0.1~$M_\oplus \text{Gyr}^{-1}$ \citep{Perez-Becker13}. Dashed white lines show the substellar temperatures of the known systems, with associated error bars. The position of the error bars does not indicate the planet initial mass, they are at an arbitrary mass.}
    \label{fig: optimisticProbTss_Mp}
\end{figure}
Kepler's primary mission observed stars with a distribution of masses ($M_*$) and ages ($t$) which we will write as $n_*(M_*,t)$, defined such that. 
\begin{equation}
    \int_{M_{*,\text{min}}}^{M_{*,\text{max}}}\int_{t_\text{min}}^{t_\text{max}} n_*(M_*,t) \; \mathrm{d}t \mathrm{d}M_* = N_*
\end{equation}
where $N_*$ is the total number of stars.

In order to isolate the dependency on the unknown number distribution of planets we first work out the chance of detecting a planet with initial mass $M_p$ and substellar temperature $T_{ss}$, per star, in the Kepler sample. This is given by
\begin{equation} 
    \begin{split}
    P_\text{detect}(M_p,T_{ss}) =  \\ \int_{M_{*,\text{min}}}^{M_{*,\text{max}}}\int_{t_{\text{obs},\text{min}}(T_{ss},M_p)}^{t_{\text{obs},\text{max}}(T_{ss},M_p)} \hat{n}_*(M_*,t) P_\text{trans}(a,R_*) \; \mathrm{d}t \mathrm{d}M_*
    \end{split}\label{eq: P_detect_a_T}
\end{equation}
$\hat{n}_*(M_*,t)$ is the normalised distribution of stellar ages and masses in the sample, i.e., $n_*(M_*,t) / N_*$. 

$P_\text{trans}(a,R_*)$ is the probability of a transit having the correct orientation relative to the Earth to be observed given by \citep{Borucki1984}
\begin{equation}
    P_\text{trans}(a,R_*) = \frac{R_*}{a}
\end{equation}
where $R_*$ is the stellar radius and $a$ the orbital separation. We only consider planets undergoing high mass loss, which would thus have deep transits like those observed, so we do not take into account any transit depth constraints.

$t_{\text{obs},\text{min/max}}(T_{ss},M_p)$ denote the earliest and latest times that the evaporation of a planet is observable. It is naturally a function of the planet's substellar temperature (see \autoref{eq: Tss}) and the planet's initial mass, $M_p$. These times are calculated through a grid of our coupled mass-loss/thermal evolution models. We consider two cases for this criterion. Firstly, an optimistic case (henceforth {\it Case 1}) where we consider planets detectable if their total mass loss rate exceeds 0.1 $M_\oplus \text{Gyr}^{-1}$, which is considered a lower limit on the detectable mass loss rate by \citet{Perez-Becker13}. In contrast, we also consider the threshold to be the region where dust is produced in the models of \citet{Booth_disint22} (henceforth {\it Case 2}). The consequences of these different cases are shown below.

In both cases, planets are considered no longer observable if their mantles have evaporated or the condition is no longer satisfied. We do not consider the evaporation of the core after the mantle has evaporated because the composition of the dust tails is thought to be inconsistent with iron \citep{vanLieshout14,vanLieshout16}. Formally, therefore, we are only considering the progenitors of planets with observed evaporating mantles. However, iron likely has a much higher mass loss rate \citep{Perez-Becker13}, due to its higher vapour pressure, and so a planet's evaporation should accelerate once the iron core is left, meaning the lifetime is potentially not significantly longer. This means that it is likely reflective of planets having both their mantles and cores evaporated. 

The number of detections can thus be written as
\begin{equation} 
    \begin{split}
    N_\text{detect} =  \\ N_* \int\int n_p(M_p,T_{ss}) P_\text{detect}(M_p,T_{ss}) \; \mathrm{d}M_p\mathrm{d}T_{ss} \geq 2 
    \end{split}\label{eq: n_detect}
\end{equation}
where $n_p(M_p,T_{ss})$ is the number planets per star with masses and substellar temperatures in the ranges $[M_p,M_p+\mathrm{d}M_p]$  and $[T_{ss},T_{ss}+\mathrm{d}T_{ss}]$, which has an unknown form. This is implicitly assumed to be universal over stellar type and age, which is unlikely, but adding a dependence would be an inappropriate level of complexity. We would expect the number distribution to be a function of orbital period, and $T_{ss}$ is a proxy for this around a given star, and also an independent variable in our model, so we make $n_p$ a function of $T_{ss}$ instead. %and there is no more reason to believe that the orbital separation distribution is universal over stellar type than the substellar temperature distribution.

We set $N_\text{detect}$ to be greater than or equal to two because two systems were found in the primary Kepler survey (Kepler 1520b and KOI-2700b), but it is possible that systems have been missed.

To find the distribution of stellar masses and ages in the Kepler sample ($N_*(M_*,t)$) we use the stellar properties derived for $\sim$ 180,000 stars from {\it GAIA} in the GKS sample \citep{Berger-GKS20} and apply the colour cut in \citet{Fulton17valley} to remove giant stars. To calculate $R_*$ and $L_*$ we use the {\it MIST} stellar isochrones \citep{MIST-Dotter16,MIST-Choi16,MESA1,MESA13,MESA15} with solar metalicity. While this neglects metallicity variation, our simple model does not warrant this complication. In order to evaluate \autoref{eq: P_detect_a_T} we binned the stars in the GKS sample according to stellar age and first integrated over the distribution of stellar masses for each age bin before integrating over the age distribution, where each distribution was normalised, thus giving the probability per star.

%For a given assumed planetary number distribution shape ($\alpha$ and $\beta$ in \autoref{eq: n_p dist}) one can thus find the normalisation $\tilde{N}$ that reproduces the 3 observations, and calculate the number of planets within a given mass/orbital period range by integrating  \autoref{eq: n_p dist} over that range.

To find $t_{\text{obs},\text{min/max}}(T_{ss},M_p)$ we need to know mass-loss rates as a function of time sampled across the parameter space where observable mass loss occurs. In order to produce a uniform sample we require a method of interpolating our mass loss models. We chose to fit the evolution of our models in mass- mass loss rate space ($M-\dot{M}$) using the function
\begin{equation}
    \dot{M} = \left[ \left(A M^{\alpha} \right)^{-\gamma} + \left( B M_\textit{init}^{a} \exp( - \lambda M^b) \right)^{-\gamma} \right]^{-\frac{1}{\gamma}} \label{eq: fit_mass_loss}
\end{equation}
where $M_\textit{init}$ is the initial mass of the planet and all other parameters are fit for and depend on substellar temperature. The physical reasoning behind this form is that the mass loss rate, for a given substellar temperature, is dependent on the mass and radius at that time. The main influence on the radius is the core mass fraction, which, if the initial core mass fraction is the same for all planets, will depend only on the mass and initial mass. At low masses, the mass loss rate increases with mass, which we assume to act as a power law, so we have $\beta > 0$. At high masses, the mass loss decreases with increasing mass, and does so approximately exponentially, hence the second bracketed term. 

The advantage of using this physically motivated form is that it not only allows interpolation but also extrapolation to higher densities. The mass loss models become more numerically difficult to run at higher densities, and therefore we did not run them to a high enough density for the entire mantle to have evaporated.

For each substellar temperature, we fit this function to a selection of our numerical outputs with different initial masses using {\texttt scipy}. To find the mass and mass loss rate evolution for a given, we integrate this function numerically. We find that this can reproduce our original input calculations as well as other calculations not when fitting the function to $\lesssim~8$\%. For the purposes of this occurrence rate calculation we deem this to be sufficient.

To interpolate between substellar temperatures where mass loss rates were calculated we use spline interpolation of the parameters in \autoref{eq: fit_mass_loss}.

\subsection{Trends in observability}
\subsubsection{Case 1: Fixed mass loss rate threshold}

We will first consider the case where the evaporating planets are considered observable if their total mass loss rates exceed 0.1 $M_\oplus \text{Gyr}^{-1}$ \citep{Perez-Becker13}. 

An important step in considering how likely it is to observe an evaporating planet is how long the planets are observable. In \autoref{fig: optimisticTime_min_max} we show the times that planets become observable and lose their entire mantles (top panel) and the length of time that they are observable, which is simply the difference between these times (bottom panel). 

For low-mass planets the mass loss rates are observable from birth and the length of time a planet is observable is limited by the time it takes either to evaporate or for the mass loss rate to decrease below observability. As a result the length of time observable increases with initial mass (lower panel of \autoref{fig: optimisticProbTss_Mp}). The mass range where this occurs is larger for higher temperatures. 

Once planets are too massive they take time to become observable, and the observability is now limited by the length of time it takes to become observable, which increases with initial mass. The time it takes to become observable and the time it takes to lose the entire mantle seem to converge (\autoref{fig: optimisticProbTss_Mp}, top panel), which can be understood conceptually because at high masses it is possible to lose the entire mantle without ever reaching a high enough mass loss rate to be observed. This only actually happens within the age of the universe for the lowest temperatures. 

The length of time that planets are observable is typically up to a few hundred Myrs, which is less than around 10\% of a star's lifetime. However, this is not all the information required to access the detectability of systems. The age of the system is also important. As can be seen in the top panel of \autoref{fig: optimisticTime_min_max}, higher masses for a given temperature become observable and evaporate later on, thus are more observable around older stars. Equally the observable masses for the same age shift upwards with increasing substellar temperature. This is why considering the observed stellar population is so important

In \autoref{fig: optimisticProbTss_Mp} we show the probability of observing an evaporating planet of a given initial mass and substellar temperature around a star in the GKS data. This is precisely the function in \autoref{eq: P_detect_a_T}. 

The contours demonstrate the trends we have described above. Higher-mass planets are observable at higher temperatures and it takes longer for the higher-mass planets to evaporate, thus they are more likely to be observed. However, the contours show two ridges for a given temperature. The ridge at lower initial mass is due to the peaks in the length of time observed, as seen in \autoref{fig: optimisticTime_min_max}. However, stars that are several Gyrs old are more common, which means at a population level it is more likely to observe a larger initial mass planet because they are more long-lived, even if the time they are observable might be shorter, which creates a second ridge at higher initial mass.

%In \autoref{fig: MP_min_max} we show the minimum mass a planet can be and not have already evaporated (labelled $M_\mathrm{min}$) and the maximum mass a planet can be and still have an observable mass loss rate ($M_\mathrm{max}$) as a function of age and substellar temperature. The bottom panel shows the fractional difference between these two masses. What we see is that a far higher range of masses are observable for younger stars, assuming the planets existed at their locations from the beginning of the  system's lifetime. This is driven by the initial mass range between observablity and destruction seen in the first panel of \autoref{fig: Time_min_max}. 

%Kepler typically did not observe young stars. Returning to \autoref{fig: Time_min_max} one can compare the top and bottom panel to see that for $\gtrsim 1$~Gyr the planets are in the regime where the length of time observable decreases with initial planet mass. Consequnelty, a typical length of time observable is perhaps closer to 100 Myrs or less, rather than the several hundred suggested above. This is why considering the whole observed population is important.

\subsubsection{Case 2: Modelled dust production threshold}
We now consider the case where planets are observable if \citet{Booth_disint22} predicts dust to be produced. The region of parameter space can be seen in \autoref{fig: mdot_grid}. Specifically, we use the threshold of dust mass loss rate exceeding $10^{-4} M_\oplus \text{Gyr}^{-1}$, although reference to \autoref{fig: mdot_grid} demonstrates that the dust production rate has a steep increase in parameter space from $10^{-6}$ to $10^{-2} M_\oplus \text{Gyr}^{-1}$, so the precise threshold is of little importance, as the dust is essentially either produced or not.

The difference compared to {\it Case 1} is, firstly, that at higher temperatures dust is no longer produced. The physical reason for this is that the dust condensation rate is too low compared to the flow timescale. Hence dust simply doesn't have time to form before the gas leaves the planet, as discussed further in \citet{Booth_disint22}. Furthermore, as can be seen in \autoref{fig: mdot_grid} the mass loss rates in the observable region are much higher for higher temperatures. The consequence of this can be seen in \autoref{fig: BoothTime_min_max} where the length of time observable decreases for higher temperatures due to the observable planets evolving quicker. In contrast for the simple cut off ({\it Case 1}) the timescale increased due to the fact that higher-mass planets can evaporate. 

For low temperatures, the observable timescale increases with temperature as the planets. In this case, however, it is driven by fewer the fact that it takes longer for slightly higher-temperature planets to leave the observable region, rather than the higher masses that can be evaporated

Another difference compared to {\it Case 1} is that the observable timescales are overall lower since the condition is generally more restrictive. Furthermore, the range of initial masses that become observable is much lower. This is because high-mass planets never reach the observable region (see \autoref{fig: mdot_grid}) before their mantles are evaporated.

In \autoref{fig: BoothProbTss_Mp} we show the probability per star that a planet is observed evaporating around a star in the GKS sample for {\it Case 2} (analogous to \autoref{fig: optimisticProbTss_Mp}, for {\it Case 1}). As should be expected, due to the lower observable time periods, the probability of detection is much lower than for {\it Case 1} (\autoref{fig: optimisticProbTss_Mp}). However, the most importantly difference between the two cases is that there is a cut-off for both high and low temperatures, due to restrictions in the dust-producing region (\autoref{fig: mdot_grid} and discussed above). For $T_{ss} \gtrsim 2050$ K higher initial masses can be observed because initial higher mass planets can enter the observable region, but the timescale they can be observed for decreases, for reasons explained above. Meanwhile, for $\lesssim 2050$ K, initial masses that become observable increase with decreasing temperature because of low-mass planets leaving the observable region.

\autoref{fig: BoothProbTss_Mp} also informs us about the range of progenitor planet masses that produce observable catastrophic evaporation under our model. The range is $\sim 0.04 - 0.08 M_\oplus$ for this case, which should be contrasted with the $\sim 0.1 - 0.3 M_\oplus$ in the case in the previous section (\autoref{fig: optimisticProbTss_Mp}).


%\begin{figure}
 %   \centering
 %   \includegraphics[width=\linewidth]{Mp_min_max.pdf}
  %  \caption{{\it Top panel:} The dotted line shows the minimum mass that a planet can be without having evaporated at a given age and temperature ($M_\mathrm{min}$). The dashed line shows the maximum mass that a planet can be and still have an observable mass loss rate. Thus planets between these lines are observable around stars of the given age a planet substellar temperature ($M_\mathrm{max}$). {\it Bottom panel:} The fractional difference between the maximum and minimum planet masses defined above. Younger stars can have a larger range of planets with observable mass loss rates.}
  %  \label{fig: MP_min_max}
%\end{figure}

\subsection{Occurrence}

\begin{figure}
    \centering
    \includegraphics[width=\linewidth]{BoothTime_min_max.pdf}
    \caption{The same as \autoref{fig: optimisticTime_min_max}, but instead planets are considered obsevable if the models of \citet{Booth_disint22} predict dust production. Note the difference in the range of initial masses compared to \autoref{fig: optimisticTime_min_max}.}
    \label{fig: BoothTime_min_max}
\end{figure}

\begin{figure}
    \centering
    \includegraphics[width=\linewidth]{BoothN_T_Mp.pdf}
    \caption{Same as \autoref{fig: optimisticProbTss_Mp}, but instead planets are considered obsevable if the models of \citet{Booth_disint22} predict dust production. Note the difference in the range of initial masses compared to \autoref{fig: optimisticProbTss_Mp}. Again the planets may have any initial mass, the postion of the error bars is arbitrary}
    \label{fig: BoothProbTss_Mp}
\end{figure}

The aim of this section is to estimate the number of planets per star required to explain the number of observed systems. The inferred occurrence of planets will naturally be a function of the relatively unconstrained number distribution, $n_p(M_p,T_{ss})$, as the behaviour of planets is a strong function of their initial mass and temperature. Furthermore, for {\it Case 1} the probability of observing a planet increases for increasing substellar temperature, so it will in that case also be a function of where we choose to cut off the distribution in temperature. 

It is, however, possible to get a handle on the number without the use of the number distribution in the following way. One can simply take the typical probability and multiply it by the total number of stars in the Kepler sample ($\sim 180,000$). This gives the total number of observations assuming that there is one planet per star, and thus the number of planets per star must be the number of observed systems (2) divided by this number.

For {\it Case 1} taking a typical probability of 0.005 gives $\sim0.22\%$ of stars having planets within the observable region in \autoref{fig: optimisticProbTss_Mp}, assuming a maximum temperature of 2500 K. On the other hand for {\it Case 2} a typical probability is closer to 0.00015, resulting in $\sim7.4\%$ having planets within the observable region of \autoref{fig: BoothProbTss_Mp}. Here no upper temperature need be assumed, as probability decreases for high temperatures. It should also be noted that both the implied number of planets is higher and the range of properties they have is lower.

For a more quantitative calculation, we do require the number distribution. We, therefore, assume the form
\begin{equation}
    n_p(M_p,T_{ss}) = \diffp{N_p}{{T_{ss}}{M_p}} =  \tilde{N} M_p^\alpha T_{ss}^\beta \label{eq: n_p dist}
\end{equation}


Because of the unbounded nature of the probability distribution for {\it Case 1} (\autoref{fig: optimisticProbTss_Mp}) it is not particularly instructive to do this calculation for that case. It is, on the other hand for {\it Case 2}. In \autoref{fig: Booth_alpha_beta} we show the number of inferred planets per logarithmic bin of substellar temperature and initial mass as a function of the parameters of this number distribution. This gives a measure that is independent of the size of the region that one believes the number distribution is valid for, and so may be used to estimate the number of planets within a chosen region of parameter space.

In \autoref{fig: Booth_alpha_beta} the $y$-axes show the dependence on substellar temperature, $T_{ss}$, and we allow this to take a range of values, as it is essentially unconstrained. On the $x$-axis we investigate more negative values for the derivative with respect to planet mass $M_p$, as the number distribution is unlikely to favour large planets. We ensure to encompass values of $\sim -8/3$ suggested for runaway growth \citep[e.g.,][]{Makino98-runaway} and -1.6 for the streaming instability \citep[e.g.,][]{Simon16}. 

%It is interesting to note the trends in the number of planets with the number distribution. For distributions with highly negative $\alpha$ -- which means more small planets -- it is less likely for a planet to be observed as many will have already evaporated or become unobservable, so the implied number of planets is higher. This has a much larger effect than the substellar temperature distribution. 
\autoref{fig: Booth_alpha_beta} implies that for an e-folding range of substellar temperature and initial planet mass around $M_p \sim 0.06 M_\oplus$ and $T_{ss} \sim 2050$~K there is approximately 1 planet per star, assuming the dust formation models of \citet{Booth_disint22}. This substellar temperature is equivalent to 0.016 a.u. and a period of 22 hrs for a K-type star like Kepler 1520, or 0.036 a.u. and 62 hrs for a sun-like star. For the more optimistic, in terms of detection probability, case the estimate is around an order of magnitude lower (see rough calculation above). One can compare this number to the equivalent number for Super-Earths, which has been directly observed. Using formulae derived empirically in \citet{Petigura2022}, for Super-Earths the number of planets per logarithmic bin of substellar temperature and planet mass, evaluated at the same point as our calculation, ranges between 0.1-0.18. Thus our number inferred occurrence rate is potentially a little large, though certainly not infeasible.

One can therefore not rule out the simple explanation for these systems, that they are formed close to their current locations and proceed to be evaporated over the lifetime of the star.

One explanation for the discrepancy between our figure and that for Super-Eaths may be that we are probing lower mass planets than conventional transits, and the smaller planet population could be larger, but would not have been detected in transit surveys. 

A way to decrease our inferred occurrence rate is to increase the probability of detection. We have assumed that the progenitor planets all have core mass fractions of 0.3. If there were a range of core masses the amount of time the mantle is evaporating would change. It is not immediately clear whether it would increase or decrease because a larger mantle has more material to evaporate, but evaporates quickly as it has a lower gravity for the same mass. We briefly investigated this by using models with the very low core mass fraction of 0.1. We found that these two effects almost cancel out and that the timescales of observable evaporation are very similar, although the lower core mass fraction examples become observable earlier, which would slightly decrease their observability around old stars. Consequently, we do not believe the effect would be significant, although further investigation could be done.

Another explanation might be that the observed planets have been scattered into their observed orbits, having formed further out. This would mean short lifetimes of lower-mass planets are more likely to be observed than using our assumptions, which penalise short lived planets that won't be observable around Gyr old main sequence stars if they formed at the same time as the star. This explanation has some physical motivation because of the strong tidal forces acting on such short period planets. As it is, our number is not so high that it is entirely inconsistent with the planets forming where they are observed, so we leave investigation of scattering and tidal inspiral to future works. As mentioned in \S\ref{sec: discuss-other}, it is possible orbital evolution could occur instead by angular momentum exchange, but we believe this to be unlikely.

\begin{figure}
    \centering
    \includegraphics[width=\linewidth]{BoothN_alpha_beta.pdf}
    \caption{The number of planets per star $N_p$ per natural logarithmic bin of initial planet mass, $M_p$, and substellar temperature, $T_ss$ as a function of $\alpha$ and $\beta$, the coefficients of $M_p$ and $T_{ss}$ in \autoref{eq: n_p dist}. Using the number of planets per logarithmic bin gives a measure that is independent of the range that one believes the number distribution is valid for. It is evaluated near the centre of the observable region: $T_{ss} = 2050$~K, $M_p = 0.06 M_\oplus$, around the mass of Mercury, since this is the peak of the observability contours in \autoref{fig: BoothProbTss_Mp}.
    }
    \label{fig: Booth_alpha_beta}
\end{figure}
\subsection{The known systems in the context of our calculations}
In \autoref{tab: systems} we show the properties of the three known catastrophically evaporating planets and their host stars. We also plot their substellar temperatures in \autoref{fig: optimisticProbTss_Mp} and \ref{fig: BoothProbTss_Mp} as dashed white lines. As ought to be the case, they all lie within the observable region, though some with low probabilities. As there are only three systems, it is inadvisable to draw to strong conclusions. However, for {\it Case 1} the probability of systems being detectable is strongly peaked towards high temperatures, whereas no such effect can be seen in the observed systems. This may, therefore, act as support towards the dust production models of \citet{Booth_disint22} being closer to the physical picture, since such models predict a higher chance of detecting lower temperature systems, compared to a higher temperature, so more closely resemble the data. The driving effects are the cut-off of dust production at high mass and temperature, which are probably independent of the details of the calculations.

This explanation for the temperatures of the systems is, of course, not unique. At least one alternative explanation could be that planets at very high substellar temperatures are intrinsically rare. This fits with the observed period distribution of Super-Earths \citep{Petigura2022}, which has a drop off at lower orbital periods, and can also be explained theoretically by tidal inspiral of planets onto the star.
\begin{table}
\resizebox{\columnwidth}{!}{%
\begin{tabular}{|l|c|c|c|}
\hline
{\bf Property} & {\bf Kepler 1520b} &  {\bf KOI-2700b} & {\bf K2-22b}\\ \hline
Stellar mass [$M_\odot$] & $0.712\pm0.031$ & $0.728 \pm^{0.033}_{0.03}$ & $0.6\pm 0.07$ \\\hline 
Stellar effective & \multirow{2}{*}{$4622.2\pm^{85.2}_{78.1}$} & \multirow{2}{*}{$4617.6\pm^{83.1}_{78.5}$} & \multirow{2}{*}{$3830\pm 100$} \\
 temperature [K] & & & \\\hline
Stellar radius [$R_\odot$] & $0.694\pm 0.017$ & $0.716\pm 0.015$ & $0.57 \pm 0.06$ \\ \hline
Orbital period [hrs] & 15.68 & 21.84 & 9.146 \\ \hline
Semi-major axis [a.u.] & $0.0132\pm 0.0002$ & $0.0165\pm 0.0002$ & $0.0087\pm0.0003$\\\hline
Planet substellar & \multirow{2}{*}{$2289\pm ^{53}_{50}$} & \multirow{2}{*}{$2072 \pm ^{45}_{44}$} & \multirow{2}{*}{$2116\pm ^{130}_{130}$} \\ 
temperature, $T_{ss}$ [K] & & & \\\hline

\end{tabular}%
}
\caption{Properties of the observed systems. For Kepler 1520b and KOI-2700b we take stellar properties from the GKS catalogue \citep{Berger-GKS20} and orbital periods from \citet{KIC1255-discov} and \citet{KOI2700b-discov} respectively. Properties for K2-22b are taken from \citet{K2-22b-discov}. Errors on the orbital period are small.}\label{tab: systems}
\end{table}

\section{Summary} \label{sec: conclusion}
In this work, we have presented a model for the thermal evolution of catastrophically evaporating planets. The model computes heat transport, through conduction and convection, and including the effects of melting, alongside hydrostatic equilibrium, allowing the mass of the model to be evolved self consistently. We have used this model to show that the catastrophically evaporating planets are likely almost entirely crystallised other than a shallow molten region on the day side of the planet. This means the chemistry of the outflows samples the surface of the planet only. This result is robust to the details of the redistribution of heat from the star, which is challenging to model in full. Therefore, we suggest that the composition of the observed dusty tails to sample the particular region of the mantle that the planet has evaporated to. Future work should include modelling of the chemical evolution of the mantle, in order to find the composition of the outer layer at a particular point in time. Additionally, further modelling of the chemical evolution of the lava pool, as well as the outflow itself, may be required to make a full link between the mantle composition and that of the dust. 

We also use our model to investigate the occurrence rate of the progenitors of the catastrophically evaporating planets, assuming they have remained at their current locations since birth. Our calculation implies, using the limit that two systems were found in the {\it Kepler} survey, that around 1 planet per star in a logarithmic bin centred at a substellar temperature of $\sim 2050$ K planetary mass of $\sim 0.06 M_\oplus$. This is around ten times that of Super-Earth's in the same region derived by {\it Kepler}, meaning the occurrence rate is high, but not infeasibly so. This inference is dependent on the choice of mass loss model but suggests that a simple scenario where the planets are formed at their current locations and then are evaporated over their lifetimes cannot be ruled out. 

When considering the observability of systems in general, we found that the substellar temperatures of the detected systems found are better explained by a dust production model, such as that of \citet{Booth_disint22}, than a simple model where planets become observable above a certain mass loss rate. In a more simple model, the hottest planets are most observable as they have the highest mass loss rates. However, in the model of \citet{Booth_disint22}, dust production is reduced at high temperatures. This is because dust condensation rates become lower than the mass flow rates, so dust is unable to form. Therefore, to observe dust either requires very high mass loss rates, thus reducing the time systems can be observed, or dust production is cut off all together.% While there is not enough data to make a strong claim, this observation adds support to such models.

\section{Data availability}
The code and simulated data underlying this article will be shared on reasonable request to the corresponding author. Any observational or experimental data used are available at the references in the main text.

\section{Acknowledgements}
 A.C. acknowledges support of an STFC PhD studentship. R.B. and J.E.O. are supported by the Royal Society through University Research Fellowships. J.E.O. has also received funding from the European Research Council (ERC) under the European Union’s Horizon 2020 research and innovation programme (Grant agreement No. 853022, PEVAP) and a 2020 Royal Society Enhancement Award. For the purpose of open access, the authors have applied a Creative Commons Attribution (CC-BY) licence to any Author Accepted Manuscript version arising.

\bibliographystyle{mn2e}
\bibliography{refs}

\appendix
\section{Entropy change of melting}\label{app: CC}

Here we demonstrate  a thermodynamically consistent way of calculating $\Delta S$, the entropy change from solid to melt, which does not require an absolute entropy scale.

We assume that the volume and entropy of the mixed solid and melt phases are additive, meaning that
\begin{equation}
    V = V_l\phi + V_s(1-\phi) \text{ and } S = S_l\phi + S_s(1-\phi) \label{eq: additive}
\end{equation}
Using this definition and the Maxwell relation
\begin{equation}
    \left.\diffp{V}{T}\right|_P = -\left.\diffp{S}{P}\right|_T  \label{eq: -dSP=dVT}
\end{equation}
one finds that
\begin{equation}
    \begin{split}
        \left.\diffp{V_l}{T}\right|_P\phi + \left.\diffp{V_s}{T}\right|_P (1-\phi) + \Delta V\left.\diffp{\phi}{T}\right|_P =  \\ -\left.\diffp{S_l}{P}\right|_T \phi -\left.\diffp{S_s}{P}\right|_T (1-\phi) -\Delta S \left.\diffp{\phi}{P}\right|_T 
    \end{split}
\end{equation}
If the melt and solid satisfy this Maxwell relation individually or if the latent heat terms dominate, which is generally the case, then
\begin{equation}
    \Delta V \left.\diffp{\phi}{T}\right|_P = -\Delta S \left.\diffp{\phi}{P}\right|_T \label{eq: dS_dV relation}
\end{equation}

We then also assume that the melt fraction, $\phi$, is simply a function of $P$ and $T$, which is true in equilibrium. This means we can use the triple product rule and write
\begin{equation}
    \left.\diffp{T}{P}\right|_\phi = - \left.\diffp{\phi}{P}\right|_T /\left.\diffp{\phi}{T}\right|_P \label{eq: cyclic}
\end{equation}

Combining \autoref{eq: dS_dV relation} and \autoref{eq: cyclic} gives
\begin{equation}
    \left.\diffp{T}{P}\right|_\phi = \frac{\Delta V}{\Delta S} \label{eq: genC-C}
\end{equation}
which is a generalised version of the Clausius-Clapeyron relation. This can then be used to calculate the entropy change through \autoref{eq: DeltaS} in the main text. 


\section{Asymmetric boundary condition} \label{app: BC}
\subsection{Simplification of conduction-convection equation at constant gravity} \label{sec: cond_conv for BC}

The equation of heat transport by conducition and convection (\autoref{eq: conduction}-\ref{eq: totalF}) is given, in full, by
\begin{equation}
    F = -k \diff{T}{P} \diff{P}{r}  - \rho l u C_P \left( \diff{T}{P} - \left.\diffp{T}{P}\right|_\text{Ad} \right) \diff{P}{r}
\end{equation}

Considering the viscous case, so $u$ is given by \autoref{eq: highviscv} this becomes 
\begin{equation}
     F = -k \diff{T}{P} \diff{P}{r}  - \frac{\rho C_P \delta g l^4}{18 \nu T} \left( \diff{T}{P} - \left.\diffp{T}{P}\right|_\text{Ad} \right)^2 \left(\diff{P}{r}\right)^2 \label{eq: cond_conv visc}
\end{equation}

Using $\diff{P}{r} = -\rho g$, we can rearrange this to
\begin{equation}
    A \left(\diff{T}{P}\right)^2 + \left(k\rho g - 2 A \left.\diffp{T}{P}\right|_\text{Ad} \right) \diff{T}{P} + A\left(\left.\diffp{T}{P}\right|_\text{Ad} \right)^2  = F
\end{equation}
where
\begin{equation}
    A \equiv \frac{\rho^3 C_P \delta g^3 l^4}{18 \nu T}
\end{equation}
Assuming that $g$ is constant, and density changes are small enough that we can take the mixing length, which will be the distance to the surface to be $l = \frac{P}{\rho g}$, then the root of this quadratic equation is simply \begin{equation}
    \diff{T}{P} = \frac{ 2 A \left.\diffp{T}{P}\right|_\text{Ad} - k\rho g + \sqrt{(k\rho g)^2 + 4A( F - k\rho g \left.\diffp{T}{P}\right|_\text{Ad})}}{2A} \label{eq: dTdP const g}
\end{equation}
i.e., an ODE for $T$ and $P$.

In deriving this equation, we made the assumption that the convection is viscous. Including the inviscid term in the way described in \S\ref{sec: heat_flow} would make this simplification impossible. We therefore instead just change $\diff{T}{P}$ to the corresponding value for pure inviscid convection
\begin{equation}
     \diff{T}{P} =  \left.\diffp{T}{P}\right|_\text{Ad} + \left(\frac{16F^2T}{\delta\rho^5 l^4 C_P^2 g^4}\right)^\frac{1}{3} 
\end{equation}
once the critical Reynolds number of 9/8 is reached.

The Reynolds number is given by
\begin{equation}
    Re = \frac{u_\textit{visc} l }{\nu}
\end{equation}
and determines how viscid/inviscid convection is, with high $Re$ corresponding to inviscid convection. The critical value is simply when the convective and conductive fluxes (\autoref{eq: conduction} and \ref{eq: Fconv}) are equal.

One does need to consider conduction in this case because if the fluid is inviscid it must also be convection dominated, as we shall show here.

The domination of conduction or convection is governed by the Peclet number 
\begin{equation}
    Pe = \frac{F_\text{conv}}{F_\text{cond}} = \frac{u l \rho C_P}{k} \label{eq: Peclet}
\end{equation}
where a high Peclet number corresponds to convection dominating heat transport. The second equality comes from considering when the temperature gradient is much higher than adiabatic. 

If one considers the ratio of the Peclet and Reynolds numbers in the viscous limit
\begin{equation}
    \frac{Pe}{Re} = \frac{\nu \rho C_P}{k}
\end{equation}
one sees that for any reasonable values the Peclet number is larger than the Reynolds number, meaning the fluid first becomes viscous and then conductive. Were the fluid inviscid then the velocities would be higher and this would be even more pronounced.

A simple cut-off for the inviscid region is not a major problem for two reasons. Firstly, the gradient is close to adiabatic, and the extra term is just an addition to it. Secondly, the transition occurs when the fluid becomes a liquid, which is indeed a steep transition. In our Henyey scheme for the 1D section of the planet, however, smoothness is a higher priority.

\subsection{Finding solutions to establish a temperature luminosity relation}\label{sec: BC grid_method}
The boundary to the 1D code requires a relation between the temperature at the boundary and luminosity (and other variables). In order to do so one must find self-consistent solutions to \ref{eq: dTdP const g} over $\theta$ at a given $L$ an $g$ and then repeat for a set a grid of $L$ and $g$. By self-consistent we mean $F$ and $T_s(\theta)$ (i.e., the flux and temperature at zero pressure) in \ref{eq: dTdP const g} must satisfy \autoref{eq: F_surface BB}. 

We approached this by integrating \autoref{eq: dTdP const g} for a grid of $T_\text{irr }$ and $F$ (which determines $T_s$ uniquely) thus giving a grid of possible $T_0$s. We used 5th-order Runge-Kutta integration. For a given $T_0$ we then used this grid, in combination with Brent's method, to find the flux that would correspond to each value of $T_\text{irr}(\theta)$ and integrated to find the total luminosity, $L$. If the surface temperature was higher than $T_0$ the flux was taken as 0. This was done for a grid of each $T_0$ and $g$. 

\subsection{Fitting of luminosity-temperature function} \label{sec: fit BC}
To aid the convergence of the Henyey scheme it is desirable for the $T_0(L)$ relation to be smooth. However, the calculation described in the previous section can produce numerical variation, particularly in the near-constant $T_0$ section (see \autoref{fig: T0_func}). 

We opted to fit a function to our result, motivated by limits to the equations' solutions. 

The solutions to the equations in \ref{sec: BC grid_method} have a large temperature increase in the conductive region close to the surface, followed by a less steep, close to adiabatic increase once conduction can dominate. The temperature $T_0$ can thus be approximated as 
\begin{equation}
    T_0 \approx T_s + \frac{F}{k \rho g} \min(P_\text{crit},P_0)
\end{equation}
where $\frac{F}{k \rho g}$ is the conductive gradient, and $P_\text{crit}$ is the point when conduction dominates. We can define this point using the Peclet number (\autoref{eq: Peclet}), which we can write as
\begin{equation}
    Pe  = \left( \frac{\rho^2 C_p \alpha g F l^4}{18 \eta k^2} \right)^{\frac{1}{2}}
\end{equation}
Taking $l = \frac{P}{\rho g}$ again gives
\begin{equation}
    T_0 \sim T_s + \left(\frac{18 k Pe_\text{crit}^2}{C_P \alpha \rho^2}\right)^\frac{1}{4} F^\frac{3}{4}g^{-\frac{1}{4}} \eta^\frac{1}{4}
\end{equation}
for the case where $P_\text{crit} < P_0$.
For low temperatures, $P_0 < P_\text{crit}$ so we take
\begin{equation}
    T_0 = T_s + A_1 F^{a_1} g^{b_1} \; , \; a_1\approx 1 \, , \, b_1\approx 1
\end{equation}
with $A_1$ a constant.
Once the system becomes partially melted the exponential part of the viscosity $\eta \propto \exp(-\alpha_\eta \phi)$ (see \autoref{eq: full_viscosity}) becomes the most important so
\begin{equation}
    T_0 \sim C\exp(-\lambda T_0) F^a_2 g^b_2
\end{equation}
with $C , \lambda , a_2 , b_2$ constants. We, therefore, fit the function
\begin{equation}
    T_0 = A_2 + a_2 \ln{F} + b_2 \ln{g}
\end{equation}
to this region.

Around $\phi \approx \phi_c$ the viscosity, and so the flux, is a very strong function of temperature, meaning that temperature is approximately constant for that a large range of fluxes, this region we approximate as a constant $T_\text{crit}$.

For high temperatures when the fluid is molten $\eta \sim \eta_l\left(\frac{\phi - \phi_c}{1 - \phi_c }\right)^{2.5}$. Since $\phi$ is linear in $T$ (for a given $P$, \autoref{eq: melt_frac}) this is not far from a power law in $T$ so we take
\begin{equation}
    T_0 = T_s + A_3 F^{a_3} g^{b_3} \; , \; a_3\approx \frac{3}{4} \, , \, b_3\approx -\frac{1}{4}
\end{equation}

We combine all these into a master formula of
\begin{equation}
    T_0 = \left(\left[\left(\left(T_s + A_1 F^{a_1} g^{b_1}\right)^{-\beta} + \left(A_2 + a_2 \ln{F} + b_2 \ln{g}\right)^{-\beta}\right)^\frac{\gamma}{\beta} + T_\text{crit}^{-\gamma} \right]^{-\frac{\alpha}{\delta}} +  \left(T_s + A_3 F^{a_3} g^{b_3}\right) ^\alpha \right)^\frac{1}{\alpha} \label{eq: fit_BC}
\end{equation}
where $\alpha,\beta,\gamma$ are smoothing constants.

We fit the coefficients including the smoothing values using {\texttt{ scipy curve\_fit}}. The only value we do not fit is $T_\textit{crit}$ which  we set to  1778.6. Coefficients for different substellar temperatures are shown in \autoref{tab: fit_params}.

As can be seen $a_1$ and $b_1$ are very close to our predicted values of 1 and -1. Meanwhile, $a_3$ and $b_3$, show a not insignificant deviation from our predicted values of 0.75 and -0.25. This is not surprising, as we entirely neglected any viscosity dependence. The values of $\alpha , \beta , \gamma$ are high, reflecting sharp transitions between regions.

%\autoref{fig: L_T0func_errors} shows the fractional error between our calculated $T_0$ values and our fitting function. 
Our fit produces fractional error on the calculated $T_0$ of $\lesssim 1.5\%$ which is probably well within errors in the physical parameters and heat transport assumptions.

%\begin{figure}
 %   \centering
  %  \includegraphics[width=\linewidth]{L_T0func_errors.pdf}
   % \caption{Fractional errors on $T_0$, the temperature at $P_0$. The error is defined as $\frac{T_{0,c}-T_{0,f}}{T_{0,c}}$, where $T_{0,c}$ is calculated by our procedure in \ref{sec: BC grid_method} and $T_{0,f}$, is that of the fitting function.}\label{fig: L_T0func_errors}
%\end{figure}

\begin{table}
\resizebox{\columnwidth}{!}{%}
\begin{tabular}{|l|c|c|c|c|c|}\hline
\multirow{2}{*}{\textbf{Parameter}} & \multicolumn{5}{c|}{\textbf{Substellar tempertature [K]}}                  \\ \cline{2-6}
                                    & \textbf{2070} & \textbf{2190} & \textbf{2320} & \textbf{2460} & \textbf{2600} \\ \hline
$A_1$                               & \num{6.79e+04}      & 6.87e+04      & \num{6.86e+04}      & \num{6.92e+04}      & \num{6.96e+04}      \\\hline
$a_1$                               & 1.04          & 1.04          & 1.04          & 1.04          & 1.04          \\\hline
$b_1$                               & -1.04         & -1.04         & -1.04         & -1.04         & -1.04         \\\hline
$A_2$                               & 1860      & 1870      & 1880      & 1880      & 1880      \\\hline
$a_2$                               & 81.1          & 82.8          & 84.2          & 85.3          & 86            \\\hline
$b_2$                               & 17.2          & 13.9          & 13.3          & 14.7          & 12.4          \\\hline
$A_3$                               & 0.41          & 0.734         & 0.66          & 0.38          & 0.201         \\\hline
$a_3$                               & 0.522         & 0.485         & 0.498         & 0.545         & 0.597         \\\hline
$b_3$                               & -0.369        & -0.326        & -0.298        & -0.281        & -0.273        \\\hline
$\alpha$                            & 6.98          & 7.23          & 7.79          & 7.99          & 8.21          \\\hline
$\beta$                             & 9.94          & 10.3          & 10.4          & 10.2          & 10          \\\hline
\end{tabular}
}
\caption{Best fit parameters for our boundary condition function \autoref{eq: fit_BC}.}\label{tab: fit_params}
\end{table}

\begin{table}\centering
\resizebox{!}{1.5cm}{%
\begin{tabular}{|l|c|c|c|c|}\hline
\multirow{2}{*}{\textbf{Parameter}} & \multicolumn{4}{c|}{\textbf{Substellar tempertature [K]}}                  \\ \cline{2-5}
                                    & \textbf{2070} & \textbf{2190} & \textbf{2320} & \textbf{2460} \\ \hline
$A$       &  4.99 &   4.1 &   100 &  78.9 \\\hline
$\alpha$   & 0.911 &  0.59 & 0.959 & 0.569 \\\hline
$B$       & 0.046 & 0.102 & 0.327 & 0.706 \\\hline
$a$       & -1.15 & -1.47 & -1.67 & -2.16 \\\hline
$b$       &  0.84 & 0.778 & 0.688 &   0.6 \\\hline
$\lambda$ &  30.8 &  26.5 &  21.3 &  16.3 \\\hline
$\gamma$  &  7.32 &  1.43 &  1.01 &  1.35 \\\hline
\end{tabular}
}
\caption{Best fit parameters for our boundary condition function \autoref{eq: fit_mass_loss}}\label{tab: mdot_fit_params}
\end{table}
%%%%%%%%%%%%%%%%%%%%%%%%%%%%%%%%%%%%%%%%%%%%%%%%%%


% Don't change these lines
\bsp	% typesetting comment
\label{lastpage}
\end{document}

% End of mnras_template.tex\documentclass{article}
\usepackage[utf8]{inputenc}


\end{document}