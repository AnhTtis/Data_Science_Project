

There are some basic questions that are unresolved. Perhaps the most basic question 
which is also unresolved in the classical setting without probing is 
to bridge the gap between the positive $1-1/e$ competitive ratio and in-approximations in the context of known i.d. random order arrivals. In terms of the single item prophet secretary problem (without probing), Correa et al. \cite {CorreaSZ19} obtain a $0.669$ competitive ratio following Azar et al. \cite{AzarCK18} who were the first to surpass the  $1-1/e$ ``barrier''.     
Correa et al. \cite{CorreaSZ19} also establish a $0.732$ in-approximation for the i.d. setting, and Huang et al. \cite{huang2022} recently established a $0.703$ in-approximation for i.i.d. arrivals in the multi-item case. 
Can we surpass $1-1/e$ in the probing setting for i.d. input arrivals or for the special case of i.i.d. input arrivals? As previously
mentioned, Yan \cite{Yan2022} recently proved that $0.645 > 1-1/e$ is attainable for known i.i.d. arrivals when probing is not required. 
Is there a provable difference between  stochastic bipartite matching (with probing constraints) and the 
classical online settings? Can we obtain the same competitive results against an optimal offline {\it  non-committal}  benchmark which respects the probing constraints  but doesn't operate in the probe-commit model? The $0.51$ in-approximation result of Fata et al. \cite{Fata2019MultiStageAM} suggests that $0.51$ may be the optimal competitive ratio against this stronger benchmark.





One interesting extension of the probing model is to 
allow non-Bernoulli edge random variables 
to describe edge uncertainty. Even for a single online
vertex in the unconstrained setting, this problem is interesting as it corresponds to computing an optimal
policy for the \textbf{free-order prophets problem},
which was recently studied by Segev and Singla in \cite{Segev2020}.






\subsection*{Acknowledgement}

We would like to thank Denis Pankratov, Rajan Udwani, and David Wajc for their very constructive comments on early versions of this paper. 
