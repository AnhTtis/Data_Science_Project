
\begin{abstract}
	We consider the classical online bipartite matching problem in the probe-commit model.
 	In this problem, when an online vertex arrives, its edges must be probed (queried) to determine if they exist, based on known edge probabilities. A probing algorithm must respect commitment, meaning that if a probed edge exists, it must be used in the matching. Additionally, each online vertex has a downward-closed probing constraint on its adjacent edges which indicates which sequences of edge probes are allowable. Our setting generalizes the commonly studied patience (or time-out) constraint  which limits the number of probes that can be made to an online vertex's adjacent edges. Most notably, this includes knapsack/budget constraints. We introduce a new configuration linear program (LP) which we prove is a relaxation of an optimal offline probing algorithm, the standard benchmark in the literature.
	Using this LP, we establish the following competitive ratios which depend on the model used to generate
	the instance graph, and the arrival order of its online vertices:
	\begin{enumerate}
		\item In the worst-case instance model, an optimal $1/e$ ratio when the vertices arrive in u.a.r. (uniformly at random) order.
		\item In the known independently distributed (i.d.) instance model, an optimal $1/2$ ratio when the vertices arrive
		in adversarial order, and a $1-1/e$ ratio when the vertices arrive in u.a.r. order.
	\end{enumerate}
	The latter two results improve upon the previous best competitive ratio of $0.46$ due to Brubach et al. (Algorithmica 2020), which only held in the more restricted known i.i.d. (independent and identically distributed) instance model. Our $1-1/e$ -competitive algorithm matches the best known result for the prophet secretary matching problem due to Ehsani et al. (SODA 2018). Our algorithm is efficient (i.e., polynomial time) and implies a $1-1/e$ approximation ratio for the special case when the graph is known. This is the offline stochastic matching problem,
	and we improve upon the $0.42$ approximation ratio for one-sided patience due to Pollner et al. (EC 2022), while also
	generalizing the $1-1/e$ approximation ratio for unbounded patience due to Gamlath et al. (SODA 2019). 
	
%	In the worst-case instance model, we generalize the secretary matching problem.
%In the known i.d. instance model, we generalize the prophet inequality matching problem 
%if the arrival order is adversarial
%and the prophet secretary matching problem if the arrival order is u.a.r. 	
%	

%, as well as the best known result when the instance is known to the algorithm and has unbounded patience due to Gamlath et al. (SODA 2019). 
\end{abstract}
