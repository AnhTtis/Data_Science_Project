\section{Limitations and Future Work}
\label{sec:limitations}

Here, we address the limitations of the proposed method that pave the way towards future directions. 
We identified two main limitations in the proposed method:
\begin{itemize}
    \item \emph{Style Adaptation:} 
    Our proposed method, contrary to the compared GAN-based methods~\cite{kang2020ganwriting,mattick2021smartpatch}, explicitly takes the writing style as an input embedding. 
    This way, the model can learn to recreate such styles accurately.
    Nonetheless, adaptation to new styles is not straightforward with this pipeline. 
    The interpolation concept is a work-around to generate ``new'' styles, and can be extended to even interpolating K different styles. 
    Nonetheless, even such ideas do not provide the ability to adapt to a specific given style via few word examples, as done in~\cite{kang2020ganwriting}. 
    An interesting future direction is the projection of different style embeddings to a common style representation space, using deep features extracted by a writer classification model as done in Section~\ref{sec:experiments}.  
    \item \emph{Sampling Complexity:} Generating realistic examples requires many iterations (timesteps) in the sampling process. To generate a single image requires $\sim12~$sec, when using $T=600$, making the creation of large-scale datasets impractical.
    We aim to explore ways to assist the generation of quality images in fewer steps, while also utilizing a more lightweight network to further reduce the time requirements of a single step.
    \item \emph{Fixed Image Size:}
    As the generation process is initiated by sampling a latent Gaussian noise of a fixed shape, our proposed method currently generates images of a fixed shape. Generating text images of arbitrary shapes is a possible future direction to explore.
\end{itemize}



