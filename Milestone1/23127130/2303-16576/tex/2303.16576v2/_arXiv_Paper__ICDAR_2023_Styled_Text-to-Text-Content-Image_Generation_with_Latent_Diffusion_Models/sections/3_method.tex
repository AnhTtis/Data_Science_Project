\section{Method}
\label{sec:method}

In this section, we present some general background information for the standard Diffusion and Latent Diffusion Models. 
We then illustrate in detail the proposed method that includes the forward process, model components, sampling and experimental setup for training and sampling from the model.

\subsection{Diffusion Models Background}

\subsubsection{\acrfull{ddpm}.} 
%paragraph checked
Diffusion Models are a type of generative model that employ Markov chains to add noise and disrupt the structure of data. 
The models then learn to reverse this process and reconstruct the data. Inspired by Thermodynamics~\cite{SohlDickstein2015DeepUL}, Diffusion Models have gained popularity in the field of image synthesis due to their ability to generate high quality samples.

The Diffusion Model consists of two phases: the forward (diffusion) process and the reverse (denoising) process.
In the forward process, a sample $x_0$ is initially drawn from a 
%real 
distribution $x_0\sim q(x_0)$ corresponding to the observed data.
This is subjected to Gaussian noise, which produces a latent variable $x_1$;
noise is again added to $x_1$, giving latent variable $x_2$, and so on, until some predefined hyper-parameter $T$.
This process forms a series of latent variables $x_1, x_2, \cdot, x_T$,
%At each timestep $t$ to produce latent variables $x_1,...,x_T$. 
Formally, we can write:
%The real distribution $q(x_0)$ is transformed into a tractable form using a Gaussian perturbation defined as: 
\begin{equation}
    q(x_{1:T}|x_0) = \prod_{t=1}^{T} q(x_t|x_{t-1}),\quad
    q(x_t|x_{t-1}) = N(x_t;\sqrt{1-\beta_t}x_{t-1}, \beta_tI),
\end{equation}
where we have $\beta_i \in [0, 1], \forall i \in [1,T]$.
Hyper-parameters $\beta_1,\beta_2,...,\beta_T$ collectively form 
a noise variance schedule, used to control the amount of noise added at each timestep.
In the final timestep, given large enough $T$ and suitable noise schedule, we will have $q(x_T|x_0) = q(x_T) \approx N(0, I)$, i.e. 
the end result becomes practically a pure Gaussian noise sample with no structure.
In the reverse (denoising) phase, a neural network learns to gradually remove the noise from the sampled by a stationary distribution until ending up with actual data.
Hence, image synthesis will be performed according to an ancestral sampling scheme.
This means that first we need to sample from $q(x_T)$, then we sample by the previous time-step conditioned on the sampled value of $x_T$, and so and so forth until we sample the required $x_0$.

The noise is gradually removed in reverse timesteps using the following transition:
\begin{equation}
\resizebox{.92\hsize}{!}{$
p_{\theta}(x_{0:T}) = p(x_T)\prod_{t=1}^{T} p_{\theta}(x_{t-1}|x_t),\quad p_{\theta}(x_{t-1}|x_t) = N(x_{t-1};\mu_\theta(x_t, t), \Sigma_\theta(x_t, t)) .$}
%\label{eq:backward_ddpm}
\end{equation}
The network is trained by optimizing the variational lower bound between the forward process posterior and the joint distribution of the reverse process $p_\theta$.
The training loss 
%This can be expressed as a Mean Squared Error (MSE) loss, such as $||\epsilon - \epsilon_\theta(x_t, t)||_2^2$.
\begin{equation}
%\resizebox{.92\hsize}{!}{$
L = \mathbb{E}_{x_0,t,\epsilon}[||\epsilon - \epsilon_\theta(x_t, t)||^2]
\end{equation}
is calculated as the reconstruction error between the actual noise, $\epsilon$, and the estimated noise, $\epsilon_\theta$.
In the case of Latent Diffusion models the loss will be adapted to the latent representation $z_t$.

\subsubsection{Latent Diffusion Models (LDM).} Diffusion Models have demonstrated remarkable performance in image generation and transformation tasks~\cite{ho2020denoising,kingma2021variational,kong2020diffwave,mittal2021symbolic}.
However, their computational cost is high due to the size of the input data and the use of cross-attention in images.
To address this issue, Latent Diffusion Models were introduced in~\cite{rombach2022high} to model the data distribution in a lower-dimensional latent representation space.
This is accomplished by mapping the input images to a latent representation using an encoder, and then decoding the sampled latents back into an image using a decoder, both from a variational autoencoder architecture. 



\subsection{Proposed Approach}

The goal of this work is to generate synthetic word-image samples given a word string and a style class as conditions from a known distribution.
We approach this problem with the use of latent diffusion models to minimize training time and computational cost.
To move to the latent space we use the pre-trained ``stable-diffusion" VAE implementation from the Hugging Face repository\footnote{\url{https://huggingface.co/CompVis/stable-diffusion}}.
Figure \ref{fig:model} presents the overall architecture of the proposed method.



% The visual abstract
\begin{figure}[t] % Always use [t] or [!t]
 \includegraphics[width=0.98\textwidth, scale=0.1]{images/model3.pdf}
    \centering
    \caption{\textbf{The overall architecture.} During training, an input image is fed to the encoder $V_E$ to create a latent representation $z$, then noise is added to the latent. The noisy latent $z_t$ is then fed to the U-Net noise predictor along with a style class index for the writer style and an encoded word as the content. The UNet predicts the noise of the noisy latent $z_{t-1}$ where $t-1$ is the corresponding timestep. During sampling, a random noise latent $z_T$ is given to predict its noise. Then the model uses the two noise predictions to reconstruct the latent of the image $z_0$ that is finally decoded by the decoder $V_D$ that creates the synthetic image.}
    \label{fig:model}
\end{figure}


\subsubsection{Forward Process and Training.}
For the forward process, the VAE encoder $V_E$ initially transforms an input image to a latent representation $z$.
A diffusion model $p_\theta(x|Y,c_\tau)$ is learned on the style $Y$ and text-condition $c_\tau$ pairs.
Timesteps $t$ are sampled from a uniform distribution and the latent representation $z$ gets gradually corrupted by the diffusion process in every timestep.
For the noise prediction, we use a U-Net architecture~\cite{ronneberger2015u} with Residual Blocks~\cite{he2016deep} and intermediate Transformer Blocks~\cite{vaswani2017attention} to add the text condition to the model, as typically used by Ho et al.~\cite{ho2020denoising}.
The network takes as input the noisy image latents, the corresponding timestep, and the desired conditions $Y$ and $\tau$.
Timesteps are encoded using a sinusoidal position embedding, similar to~\cite{vaswani2017attention} to inform the model about each particular timestep that is operating.
The training objective is to minimize the reconstruction error between the network's noise prediction and the noise present in the image.
For the diffusion process, a noise scheduler increases the amount of noise linearly from $\beta_1=10^{-4}$ to $\beta_T=0.02$ for $T = 1000$ timesteps.
%While most works use multiple ResNet blocks within the U-Net parts, our problem uses much less data than usual cases and the network becomes too complex and collapses.
While most works use multiple ResNet blocks within the U-Net components, in the context of the current problem we need to take into account that we must work with scarce data compared to other use-cases;
larger models correspond to larger parameter spaces, which are exponentially harder to explore.
Hence, we use $1$ ResNet block in every module of the U-Net.
To further reduce the parameters and complexity of the network we use an inner model dimension of $320$ and $4$ heads in the Multi-Headed Attention layers within the U-Net.


\subsubsection{Sampling.}
We generate synthetic samples by deploying the reverse denoising process learned from the model.
To this end, the noise of a random noisy sample $z_T$ is predicted by the learned network $p_\theta$ and gradually removed in every timestep of the reversed process starting from $T$ to $t=0$.
One of the main challenges associated with DDPM is the time required for sampling.
Our experiments indicate that reducing the number of time steps from 1,000 to 600 does not compromise the quality of the generated samples. 
The final image is obtained in pixel space by decoding the denoised latent variable using decoder $V_D$.
We demonstrate how the reduction of timesteps affects the quality of the generated sample in Figure \ref{tab:timesteps}.
The figure shows that below 500 timesteps the quality of the images is really affected, thus to make sure the generated samples are not affected dramatically we proceed with a value of 600.

\begin{figure}[ht]
\begin{center}

\centering
\renewcommand{\arraystretch}{1.5}
\addtolength{\tabcolsep}{2.5pt} 
%\begin{adjustbox}{scale=0.53,center}
\begin{tabular}{cccccccccc}

%Real IAM & Diffusion (ours) & SmartPatch & GANwriting\\

$T=100$ & $T=200$ &$T=300$ &$T=400$ &$T=500$ \\
 %1st row Labour
\includegraphics[scale=0.25]{images/time steps/what_254_100.png} & \includegraphics[scale=0.25]{images/time steps/what_254_200.png} & \includegraphics[scale=0.25]{images/time steps/what_254_300.png} & \includegraphics[scale=0.25]{images/time steps/what_254_400.png} &
\includegraphics[scale=0.25]{images/time steps/what_254_500.png}
\\
$T=600$ &$T=700$ &$T=800$ &$T=900$ &$T=1000$ \\
\includegraphics[scale=0.25]{images/time steps/what_254_700.png}
&
\includegraphics[scale=0.25]{images/time steps/what_254_800.png}
&
\includegraphics[scale=0.25]{images/time steps/what_254_900.png}
&
\includegraphics[scale=0.25]{images/time steps/what_254_1000.png}&
\includegraphics[scale=0.25]{images/time steps/what_254_600.png}\\


\end{tabular}
%\end{adjustbox}
\caption{Sampling outputs using various timesteps values in the reverse denoising process.}
\label{tab:timesteps}
\end{center}
\end{figure}

\subsubsection{Style and Text Conditions.}
The input style condition $Y$ is processed with an embedding layer and then added to the timestep embedding.
For the text condition, a content encoder $C_E$ is used to transform an input string $\tau$ into a meaningful context representation $c_\tau = C_E(\tau)$ for the model.
Initially, the string is tokenized using a unique index for each letter and then passed through an embedding layer to transform it to an appropriate embedding dimension according to the vocabulary size which is the number of characters present in the training set.
Then, positional encodings similar to~\cite{vaswani2017attention} are used to inform the model about the character position in the sequence with the use of sine and cosine functions as $PE_{pos, i} = \sin(pos/1000^{i/emb\_dim})$ and $PE_{pos, i+1} = \sin(pos/1000^{i/emb\_dim})$, where $pos$ is the position of each letter in the sequence and $emb\_dim$ is $320$ as mentioned previously. 
Finally, to create the text input condition $c_\tau$ a dot-product attention layer is used, defined as $Att(Q,K,V) = softmax(\frac{QK^T}{\sqrt{d_k}})V$, to create a weighted sum of the character representations.
To support the choice of the positional text encoding we present a few samples as ablation with and without the positional encoding and self-attention layers in Figure \ref{fig:encoder}.

\subsection{Experimental Setup}

We conducted extensive experiments using the IAM offline handwriting database on word-level~\cite{marti2002iam}.
Similar to \cite{mattick2021smartpatch} and \cite{kang2020ganwriting}, we used the Aachen split train set and included words of 2-7 characters to train the diffusion model.
Thus, during training the model sees 339 writer styles and approximately 45K words.
For consistency, all images were resized to a fixed height of $64$ pixels, retaining their aspect ratio. To handle variations in width, images of width smaller than $256$ pixels were center-padded, while larger ones were resized to the maximum width. Since the maximum number of characters is $7$, this resizing did not cause significant distortions in the images. Moreover, these images were intended for training other models, which could eventually lead to resizing or modifications of the original images.
AdamW~\cite{loshchilov2017decoupled} is used as the optimizer during training with a learning rate of $10^{-4}$.
To better understand the nature of the model, no augmentation is used on the images during training.
Each model was trained for 1K epochs with a batch size of 224 on a single A100 SXM GPU.

%The transformer uses ... .
%The embedding size is set to 320.

\section{Decoupled Regression as an Universal Encoding}
\label{sec:UE}

In this section, we propose a universal encoding model for point clouds.
Building on this encoder, in~\cref{sec:UCM}, we introduce Universal Continuous Mapping.

\subsection{Decoupled Gaussian Process Regression as Encoder-Decoder}% one page
\label{sec:encoder}

Inspired by DI-Fusion~\cite{huang2021di}, which introduces latent feature maps for continuous surface prediction, our goal is to generate a latent vector for each local patch.
%
% introduce the gaussian process
First, a universal encoder design comes from the Gaussian Process Regression (GPR), which is a widely used technique for low-dimensional regression.
It has been used in various mapping models~\cite{yuan2018fast,martens2016geometric}.
%
Given a set of $N$ observation points $\{(\V x_n, \V y_n)\}_{n=1}^{N}$ where point positions are $\V x_n \in \mathbb R^d$ ($d=3$ in this paper) and point properties are $\V y_n \in \mathbb R^c$, GPR is used to regress a function $f$ that best explains the data.
The $N$ points are aggregated in the $N\times d$ matrix $\V X$, and the targets are collected in the $N\times c$ matrix $\V Y$.
%The regression model estimation is $f(\V x)=E[\V y | \V X=\V x]$.
%based on the observation set. %Which is $\V y = f(\V x) + \espilon$ where $\epsilon$ is Gaussian noise. 

The Gaussian process assumes that the data is sampled from a multivariate Gaussian distribution, i.e.,
\begin{equation}
\begin{bmatrix}
\V Y\\
\V Y_{*}
\end{bmatrix}
	 \sim \mathcal{N}(\V 0, 
\begin{bmatrix}
k(\V X, \V X)\ k(\V X, \V X_{*})\\
k(\V X_{*}, \V X)\ k(\V X_{*}, \V X_{*})
\end{bmatrix}	 
	 ).
\end{equation}
where $(\V X, \V Y)$ and $(\V X_*, \V Y_*)$ represent the observation and inference pairs. 
For simplicity, we denote the matrix $\V K=k(\V X, \V X)$, $\V K_{*}=k(\V X, \V X_*)$, $\V K_{**}=k(\V X_*, \V X_*)$.

%
With the derivative in~\cite{williams2006gaussian}, we obtain 
\begin{equation}
	\V Y_{*} | \V X, \V Y, \V X_{*} \sim \mathcal{N} (\V K_{*}^T\V K^{-1}\V Y, \V K_{**}-\V K_{*}^T\V K^{-1}\V K_{*}).
\end{equation}
%
When an additional Gaussian error is introduced as $\V y=f(\V x)+\epsilon$, the covariance of $\V X$ is rewritten as $\V K+\delta^2_n\V I$.
%
The regression result is typically considered to be the mean
\begin{equation}
	\label{eq:mean}
	\V Y_{*} = \V K_{*}^T(\V K+\delta^2_n\V I)^{-1}\V Y.
\end{equation}
This well illustrates the challenge of using Gaussian process regression directly in large-scale reconstructions due to its $\mathcal{O}(n^3)$ time complexity, which is not feasible.
% with approxmiation
Furthermore, the formula (\cref{eq:mean}) is impractical since it requires the large input point cloud data to be maintained for the $ \V K_{*}$ computation. 

To address the issue of high complexity and avoid the need to maintain the entire point cloud, we propose to decouple the GPR to obtain the low-dimensional latent vectors for local regions.
%
The decoupling process involves approximating the kernel function as
\begin{equation}
	k(\V X,\V X_{*} ) \approx f_{posi}(\V X) f_\text{posi}(\V X_{*})^T
	\label{eq:k_approx}
\end{equation}
where $f_\text{posi}: \mathbb{R}^3\rightarrow \mathbb{R}^l$.
We refer to the function $f_\text{posi}$ as the position encoding function, consistent with the Neural Implicit Maps model, Di-Fusion~\cite{huang2021di}.
%
Thus, \cref{eq:mean} is rewritten as
\begin{equation}
	\label{eq:approx_K_mean}
	\begin{split}
	\V y_{*} %&= f_\text{posi}(\V X_*)^Tf_{posi}(\V X)(f_\text{posi}(\V X)^Tf_\text{posi}(\V X) +\delta^2_n\V I)^{-1}\V y\\
	%&
	=f_\text{posi}(\V X_*)f_\text{enc}(\V X, \V Y)
	\end{split}
\end{equation}
%With $X\in \mathbb{R}^{N\times 3}$, $y\in\mathbb{R}^{N\times c}$, we
where the content encoder function is
\begin{equation}
	\label{eq:encode}
	f_\text{enc}(\V X, \V Y) = f_\text{posi}(\V X)^T(f_\text{posi}(\V X)f_\text{posi}(\V X)^T +\delta^2_n\V I)^{-1}\V Y \in \mathbb{R}^{l \times c}.
\end{equation}

The encoded feature is denoted by $ \V F_{(\V X, \V Y)} = f_{enc}(\V X, \V Y) \in \mathbb{R}^{l\times c}$.
which serves as the basis for the construction of latent maps in the subsequent universal continuous mapping model (\cref{sec:UCM}).
Specifically, for geometry encoding we set $l=20$ and $c=1$, resulting in a $20$-dimensional vector feature.
Similarly, for color encoding, we have $l=20$ and $c=3$.

The decoding value for the inferred point $\V x_*$ is expressed as
\begin{equation}
	\label{eq:decode}
	f_{dec}(\V x_{*},\V F_{(\V X, \V Y)}) = f_{posi}(\V x_{*}) \V F_{(\V X, \V Y)}.
\end{equation}
Thus, a signed distance field or surface property field is approached.

In the following we derive the approximation function.

\begin{figure}[]
	\centering
    \psfragfig[width=1\linewidth]{im/eps/encoder1}{
	\psfrag{y1}{$\V y_1$}
	\psfrag{x1}{$\V x_1$}
	\psfrag{yn}{$\V y_n$}
	\psfrag{xn}{$\V x_n$}
	\psfrag{z1}{$\V z_1$}
	\psfrag{zn}{$\V z_n$}
	\psfrag{fp}{$f_{posi}$}
	\psfrag{fd}{$f_{dec}$}
	\psfrag{fe}{\color{mybronze}{$f_{enc}$}}
	\psfrag{ec}{\color{mybronze}{{Encoder}}}
	\psfrag{zm}{$\V Z_m$}
	\psfrag{ym}{$\V Y_m$}
	\psfrag{Fm}{$\V F_m$}
	\psfrag{x}{$\V x_{*}$}
	\psfrag{y}{$\V y_{*}$}
	\psfrag{f}{\tiny $\V Z_m^T(\V Z_m\V Z_m^T+\sigma_n\V I)^{-1}\V Y_m$}
}
	
	%\includegraphics[width=1\linewidth]{im/encoder}
	\caption{Interpreting formula with a graph that is coherent to the Encoder-decoder structure in Neural Implicit Maps~\cite{huang2021di}. }
	\label{fig:encoder}
	\vspace{-.3cm}
\end{figure}



\subsection{Position Encoding with Approximated Kernel Function} % one page
\label{sec:posi_encode}

Considering that our mapping needs to encode the local geometry \& property, the encoding function only requires to touch points in a limited region ($[-.5,.5]^3$ in our case).

As we discussed in related work, Nytr\"om methods offer greater accuracy than RFFs, because they depend on the given points. This property is well-suited to our application.

% introduce the Nytrom methods
The Nystr\"om method for kernel approximation begins with the use of eigenfunctions according to Mercer's theorem:
\begin{equation}
	 k(\V x_1, \V x_2) = \sum_{i\ge 1} \mu_i \psi_i(\V x_1)^T\psi_i(\V x_2)
	\label{eq:eigen}
\end{equation}
where $\psi_i$ and $\mu_i\ge 0$ are eigenfunctions and eigenvalues of kernel function $k$ with respect to the probability measure $q$.% With the top-$k$ eigenpairs, Nystr\"om method approximate kernel with $ k(\V x, \V y) = \sum^k_{i\ge 1} \alpha_i \varphi_i(\V x)\varphi_i(\V y)$.

Given a set of anchor samples $\hat{\V X}=\{\hat{\V x}_1,\cdots,\hat{\V x}_N\}$, 
% how we use it
we perform eigen-decomposition on the matrix $k(\hat{\V X}, \hat{\V X})$ to obtain its eigenpairs $\{(\lambda_i,\V u_i)\}_{i\in\{1,\cdots,l\}}$ with rank $l$.

Subsequently, Nystr\"om method produces
\begin{equation}
\label{eq:nytrom_1}
\psi_i(\V x) = \sum_{n}k(\V x, \hat{\V x}_n)\V u_{i,n}, i= 1,\cdots,l.
\end{equation}

% can be obtained with eigendecomposition of matrix $K(\hat{\V X}, \hat{\V X})$. 

%To approach
%$k(\V x, \V y) = \sum^{k}_{i= 1} \hat{\mu_i} \hat{\varphi_i(x)}\hat{\varphi_i(y)}$ 

%The \cref{eq:nytrom_1} is then rewriten as 
%\begin{equation}
%	\label{eq:nytrom_2}
%	\hat{\varphi}_i(\V x) = \frac{1}{N\hat{\mu_i}}\sum_{n=1}{N}k(\V x, \V x_n)\hat{\varphi}(\V x_n), i\in{1,\cdots,k}.
%\end{equation}


To simplify, we express the eigenfunction as 
\begin{equation}
\psi_i(\V x) = k(\V x, \hat{\V X})\V u_i .
\label{eq:nytrom_2}
\end{equation}
Similarly, the eigenvalue is written as $\mu_i=\frac{1}{\lambda_i}$.

For clarity, we introduce the notation $ \pmb \mu=[{\mu_1},\cdots,{\mu_l}]$ and
${\Psi}=[\psi_1,\cdots,{\psi_l}]^T$
 based on \cref{eq:eigen} where
${\Psi}:\mathbb{R}^3\rightarrow \mathbb{R}^l$.

To maintain consistency with \cref{eq:k_approx}, we set 
\begin{equation}
	f_{posi}(\V x) = diag(\sqrt {\pmb\mu})\Psi(\V x) 
	\label{eq:f_posi}
\end{equation}
where $diag$ produces diagonal matrix.
$f_{posi}$ above refers to the position encoder in \cref{eq:encode}.

In this paper, we employ the Mat\'ern kernel function~\cite{genton2001classes}\footnote{https://en.wikipedia.org/wiki/Mat\'ern\_covariance\_function}:
\begin{equation}
\label{eq:matern}
k(\V x_1, \V x_2) = \sigma^2 \frac{2^{1-\nu}}{\Gamma(\nu)}(\sqrt{2\nu}\frac{dist(\V x_1, \V x_2)}{\rho})K_{\nu}(\sqrt{2\nu}\frac{dist(\V x_1, \V x_2)}{\rho})
\end{equation}
where $\Gamma$ presents the gamma function, $K_{\nu}$ is the modified Bessel function of the second kind, $dist$ denotes the Euclidean distance, and $\sigma$ and $\rho$ are hyperparameters of the kernel function.
We utilize the half integer $\nu=3+\frac{1}{2}$, which results in the specific function:
\begin{equation}
\label{eq:matern_specific}
k(d) = \sigma^2(1+\frac{\sqrt{7}d}{\rho}+\frac{2}{5}(\frac{\sqrt{7}d}{\rho})^2+\frac{1}{15}(\frac{\sqrt{7}d}{\rho})^3)exp(-\frac{\sqrt{7}d}{\rho})
\end{equation}
where $d=dist(\V x_1, \V x_2)$ for short.

To approximate the above kernel function, anchor points are required.
We sample $N_a=256$ points uniformly 
from $[-.5,.5]^3$ cube (as $\hat{\V X}$) to compute the kernel matrix $\V K_{a}=k(\hat{\V X},\hat{\V X})$.
Subsequently, we perform an eigendecomposition on this kernel matrix, resulting in $\V K_{a}=\V U \Lambda \V U^T$. % with $U$ a $256\times $.
Lastly, $\V U$ and $\Lambda$ are utilized in \cref{eq:f_posi} and~\cref{eq:k_approx} to form the kernel function approximation.

It is important to note that the dimension of the encoded feature, $l$, used in \cref{sec:encoder}, depends on $\V U$. 
It further determines the size of the map in the next section (~\cref{sec:UCM}).% creating a trade-off between the approximation and map size.
%Therefore, the $l$ cannot be too large.
\begin{figure}[htbp]
	\centering
	\includegraphics[width=1.\linewidth]{im/eigenvalue}
	\caption{Sorted eigenvalues for $\V K_{a}$'s eigendecomposition.}
	\label{fig:eigenvalues}
\end{figure}

The eigenvalues of $\Lambda$ are plotted in \cref{fig:eigenvalues}, revealing that the matrix is primarily influenced by a small number of pairs with significant eigenvalues.
Most of the eigenvalues are less than 1. 
Therefore, we choose $l=20$ which is about $0.8$ in this plot to approximate the kernel while maintaining a compact feature dimension.

Note that we perform a single sampling and decomposition step. Subsequently, the encoding-decoding process in \cref{fig:encoder} only requires loading and reusing the parameters for $f_{posi}$, $f_{enc}$ and $f_{dec}$. In contrast, related works often involve pre-training on large datasets of objects~\cite{huang2021di,li2022bnv} or indoor scenes~\cite{zhu2022nice}. 
For our model, however, \textbf{no training is needed}.

%For the convenience of use in the next section, this section also encapsulate functions into API as in~\cref{alg:encode-decode}.
%\begin{algorithm}[H]
%	\caption{Encoding and decoding}\label{alg:encode-decode}
%	\begin{algorithmic}
%		\STATE 
%		\STATE {\textsc{Encode}}$(\V X,\ \V y)$
%		\STATE \hspace{0.5cm}$\V Z \gets f_{posi}(\V X) $ \hspace{1in} \# \cref{eq:f_posi}
%		\STATE \hspace{0.5cm}$ \V F \gets \V Z(\V Z^T\V Z+\sigma_n\V I)^{-1}\V y$ \hspace{1cm} \# \cref{eq:encode}
%		\STATE \hspace{0.5cm}\textbf{return}  $\V F$
%		\STATE 
%		\STATE {\textsc{Decode}}$(\V {X_*},\ \V F )$
%		\STATE \hspace{0.5cm}$\V y_*=f_{posi}(\V X_*)^T\V F$ \hspace{2cm}\# \cref{eq:decode}
%		\STATE \hspace{0.5cm}\textbf{return}  $\V y_*$
%	\end{algorithmic}
%\end{algorithm}


