% ****** Start of file apssamp.tex ******
%
%   This file is part of the APS files in the REVTeX 4.2 distribution.
%   Version 4.2a of REVTeX, December 2014
%
%   Copyright (c) 2014 The American Physical Society.
%
%   See the REVTeX 4 README file for restrictions and more information.
%
% TeX'ing this file requires that you have AMS-LaTeX 2.0 installed
% as well as the rest of the prerequisites for REVTeX 4.2
%
% See the REVTeX 4 README file
% It also requires running BibTeX. The commands are as follows:
%
%  1)  latex apssamp.tex
%  2)  bibtex apssamp
%  3)  latex apssamp.tex
%  4)  latex apssamp.tex
%
\documentclass[%
reprint,
%superscriptaddress,
%groupedaddress,
%unsortedaddress,
%runinaddress,
%frontmatterverbose, 
%preprint,
%preprintnumbers,
%nofootinbib,
%nobibnotes,
%bibnotes,
 amsmath,amssymb,
 aps,
%pra,
%prb,
%rmp,
%prstab,
%prstper,
%floatfix,
]{revtex4-2}

\usepackage{graphicx}% Include figure files
\usepackage{dcolumn}% Align table columns on decimal point
\usepackage{bm}% bold math
\usepackage{color}
%\usepackage{hyperref}% add hypertext capabilities
%\usepackage[mathlines]{lineno}% Enable numbering of text and display math
%\linenumbers\relax % Commence numbering lines

%\usepackage[showframe,%Uncomment any one of the following lines to test 
%%scale=0.7, marginratio={1:1, 2:3}, ignoreall,% default settings
%%text={7in,10in},centering,
%%margin=1.5in,
%%total={6.5in,8.75in}, top=1.2in, left=0.9in, includefoot,
%%height=10in,a5paper,hmargin={3cm,0.8in},
%]{geometry}

\begin{document}

\preprint{APS/123-QED}

\title{Gate-tunable giant superconducting nonreciprocal transport in few-layer $T_d$-MoTe$_2$}% Force line breaks with \\
%\thanks{A footnote to the article title}%

\author{T. Wakamura,$^1$ M. Hashisaka,$^1$ S. Hoshino$^2$, M. Bard$^1$, S. Okazaki,$^3$ T. Sasagawa,$^3$ T. Taniguchi,$^4$ K. Watanabe,$^5$ K. Muraki,$^1$}
\author{N. Kumada$^1$}
 %\altaffiliation[Also at ]{Physics Department, XYZ University.}%Lines break automatically or can be forced with \\
%\author{M. Hashisaka}%
%\author{K. Muraki}
%\author{N. Kumada}
 %\email{Second.Author@institution.edu}
\affiliation{$^1$NTT Basic Research Laboratories, NTT Corporation, 3-1 Morinosato-Wakamiya, Atsugi, 243-0198, Japan}
\affiliation{$^2$Department of Physics, Saitama University, Shimo-Okubo, Saitama 338-8570, Japan}
\affiliation{$^3$Laboratory for Materials and Structures, Tokyo Institute of Technology, Nagatsuta, 226-8503, Japan}
%\collaboration{MUSO Collaboration}%\noaffiliation

 %\homepage{http://www.Second.institution.edu/~Charlie.Author}
\affiliation{$^4$International Center for Materials Nanoarchitectronics, National Institute for Materials and Science, 1-1 Namiki, Tsukuba, 305-0044, Japan}

\affiliation{$^5$Research Center for Functional Materials, National Institute for Materials and Science, 1-1 Namiki, Tsukuba, 305-0044, Japan}%
%\author{Delta Author}
%\affiliation{%
 %Authors' institution and/or address\\
 %This line break forced with \textbackslash\textbackslash
%}%

\date{\today}% It is always \today, today,
             %  but any date may be explicitly specified

\begin{abstract}
We demonstrate gate-tunable giant field-dependent nonreciprocal transport (magnetochiral anisotropy) in a noncentrosymmetric superconductor $T_{\rm d}$-MoTe$_2$ in the thin limit. Giant magnetochiral anisotropy (MCA) with a rectification coefficient $\gamma$ = $3.1 \times 10^6$ T$^{-1}$ A$^{-1}$, is observed at 230 mK, below the superconducting transition temperature ($T_c$). This is one of the largest values reported so far and is likely attributed to the reduced symmetry of the crystal structure. The temperature dependence of $\gamma$ indicates that the ratchet-like motion of magnetic vortices is the origin of the MCA, as supported by our theoretical model. For bilayer $T_{\rm d}$-MoTe$_2$, we successfully perform gate control of the MCA and realize threefold modulation of $\gamma$. Our experimental results provide a new route to realizing electrically controllable superconducting rectification devices in a single material.  
%\begin{description}
%\item[Usage]
%Secondary publications and information retrieval purposes.
%\item[Structure]
%You may use the \texttt{description} environment to structure your abstract;
%use the optional argument of the \verb+\item+ command to give the category of each item. 
%\end{description}
\end{abstract}

%\keywords{Suggested keywords}%Use showkeys class option if keyword
                              %display desired
\maketitle

%\tableofcontents

%\section{\label{sec:level1}First-level heading:\protect\\ The line
%break was forced \lowercase{via} \textbackslash\textbackslash}



%Symmetry determines fundamental properties of crystals \cite{Tinkham1}. Systems with broken inversion symmetry offer an ideal testbed to investigate the role of symmetry for electrical transport in the system. 
Recent intensive studies on nonreciprocal transport have revealed the potential of using noncentrosymmetric materials or inversion-symmetry-breaking multilayer structures to develop novel rectification devices \cite{Tokura, Ideue, Ando}. In systems with broken inversion and time-reversal symmetries, Onsager's reciprocal theorem allows the electrical resistance to be different for opposite current directions. This is called magnetochiral anisotropy (MCA), and it leads to the rectification effect \cite{Tokura}.   


Broken inversion symmetry is more profitable in superconductors. Rectification via ratchet-like motion of magnetic vortices was reported more than a decade ago for superconductors with asymmetric artificial magnetic nanostructures or with asymmetric antidots as an asymmetric pinning potential, and it was found that as the asymmetry becomes stronger, rectification is more efficient \cite{Ratchet1, Ratchet2, Ratchet_PRB1, Ratchet_PRB2, Zhu, Villegas}. Recent studies have pointed out that such ratchet-like motion of magnetic vortices is also possible in noncentrosymmetric superconductors and provides large MCA as for other mechanisms such as paraconductivity \cite{Hoshino, Itahashi, Ideue2, Daido, Wakatsuki}. In such systems, the asymmetry of the crystal structure intrinsically induces asymmetric pinning potential for magnetic vortices. Taking the analogy with previous findings on superconductors with artificial asymmetric pinning potentials, the symmetry of the crystal should play a crucial role for the efficiency of the rectification. However, previous reports on MCA in noncentrosymmetric superconductors has been limited to those with trigonal symmetry \cite{MoS2, NbSe2, Itahashi, Ideue2, Itahashi2}. Therefore, it is particularly called for to explore MCA in noncentrosymmetric superconductors with different symmetries, especially with lower symmetry than trigonal symmetry for further enhancement of the efficiency.  

\begin{figure}[b]
\begin{center}
\includegraphics[width=8cm,clip]{Fig1_3.eps}
\caption{(a) Top view of the crystal structure, in which broken inversion symmetry is evident. For thin layers, only one mirror plane is present. (b) Optical microscope image of a 4 ML device. A thin $T_{\rm d}$-MoTe$_2$ flake is deposited on metallic contacts prepared in advance. (c) Temperature dependence of the resistance of 4 ML and bilayer samples. (d) Comparison of $R_{2\omega}$ when current is parallel to the $a$-axis (red) and $b$-axis (blue) taken of the 4 ML sample. }
\label{fig1}
\end{center}
\end{figure}
%Van der Waals (vdWs) materials play a significant role in recent condensed matter physics, and growing interests accelerate to find more novel materials. Whereas graphene is an archetype of vdW materials, transition-metal dichalcogenides (TMDs) are major building blocks owing to their diverse electrical and optical properties.
\begin{figure*}[tb]
\begin{center}
\includegraphics[width=16cm,clip]{Fig2_4.eps}
\caption{(a) Experimental data from the 4 ML sample. ~Top: Nonreciprocal resistance ($R_{\rm 2 \omega}$ = $V_{\rm 2 \omega}/I_{\rm \omega}$) measured at different temperatures. Bottom: $R_\omega$ signals measured simultaneously with $R_{2 \omega}$. 
(b) Temperature dependence of $\gamma$ taken from the 4 ML sample. The orange curve shows the fit based on equation (\ref{eq3}). Inset: Experimental data of $\gamma$ as a function of temperature obtained from the bilayer sample with the fit.
(c) Left: Schematic illustration of the motion of magnetic vortices with the velocity $\mathbf{v}$ driven by an external current $\mathbf{J}$, which generates an electric field $\mathbf{E} = \mathbf{B} \times \mathbf{v}$. Right: Image of sawtooth potential assumed as the ratchet pinning potential in (\ref{eq3}).}
\label{fig2}
\end{center}
\end{figure*} 
%Molybdenium-ditellulide (MoTe$_2$) is one of those attractive TMDs. MoTe$_2$ has the two phases stable at room temperature. $2H$- phase is semiconducting, and has the same structure as that of MoS$_2$. On the other hand, $1T'$- phase is semimetallic and preserves inversion symmetry.  When $1T'$-MoTe$_2$ is cooled down from room temperture, it exhibits a phase transition around 250 K, and below 250 K the stable crystal structure is $T_d$ phase, where inversion symmetry is broken. From previous first principle calculations and angle-resolved photoemission spectroscopy (ARPES) measurements, it is revealed that $T_d$-MoTe$_2$ is a type-II Weyl semimetal. Interestingly, it becomes superconducting around 100 mK. By contrast, WTe$_2$, a cousin of $T_d$-MoTe$_2$, is well known as a type-II Weyl semimetal in the bulk state and a quantum spin Hall insulator in the monolayer limit, but it does not exhibit superconductivity intrinsically. Therefore $T_d$-MoTe$_2$ is a promising candidate to realize topological superconductivity since it shows both superconductivity and topological order at low temperatures.


In this Letter, we demonstrate gate-tunable giant MCA in a noncentrosymmetric superconductor $T_{\rm d}$-MoTe$_2$ in the thin limit. $T_{\rm d}$-MoTe$_2$ lacks inversion symmetry and, for thin layers, has only one mirror plane normal to the $b$-axis as shown in Fig.~\ref{fig1}(a). This reduced symmetry of the crystal structure may make the pinning potential for magnetic vortices highly asymmetric, which can generate a large MCA. We exploit few-layer $T_{\rm d}$-MoTe$_2$ for our measurements. 
%The critical temperature ($T_c$) is largely enhanced in comparison to bulk $T_{\rm d}$-MoTe$_2$, whose $T_c$ is around 100 mK. 
Below $T_c$, we observe MCA under a perpendicular magnetic field. The rectification coefficient $\gamma$, i.e., the ratio of the nonreciprocal (second harmonic) resistance to the linear resistance, reaches 3.1 $\times$ 10$^6$ T$^{-1}$A$^{-1}$ at 230 mK, one of the largest values reported so far. The monotonic increase in $\gamma$ with decreasing temperature indicates that the giant MCA is due to the ratchet-like motion of magnetic vortices in the mixed state of the type-II superconductor \cite{Hoshino}. Interestingly, despite that  $T_{\rm d}$-MoTe$_2$ is a semimetal, in the bilayer sample we can successfully modulate the MCA via an external gate voltage and demonstrate threefold modulation of $\gamma$. This ability to produce a large variation in the nonreciprocal resistance by changing the gate voltage may provide key insights into the mechanisms behind the giant MCA by associating it with modulation of the superconducting properties.  

%Our experimental demonstration of the gate-tunable giant superconducting nonreciprocal effect provides a future prospect to realize novel superconducting device for efficient current rectification at low temperatures.

Thin semimetallic MoTe$_2$ samples are prepared from high-quality $T_{\rm d}$-MoTe$_2$ crystals with a residual-resistivity ratio (RRR) $\sim$ 1000 grown via the flux growth method. Mechanically exfoliated flakes are transferred onto prepatterned electrodes on a Si/SiO$_2$ chip or hexagonal boron-nitride (h-BN) deposited on a Si/SiO$_2$ chip by a typical dry transfer technique with polycarbonate (PC) and poli-dimethylpolysiloxane (PDMS) \cite{PC} in an argon-filled glovebox with a low concentration of O$_2$ and H$_2$O ($<$ 0.5 ppm). 
%Metal ellectrodes are fabricated via the typical electron-beam lithography and electron-beam evaporation of thin Pt (or Au) and Ti. Exfoliated thin flakes are identified by the optical microscope and the atomic-force microscope (AFM). Raman scattering measurements are exploited to determine the direction of the crystal axis of each flake \cite{Raman1}. 
Electrical measurements are performed with a lock-in amplifier and $^3$He low temperature measurement system. More details of the fabrication process and measurement setup are in the Supplemental Material \cite{SM}.
%\begin{figure*}[tb!]
%\begin{center}
%\includegraphics[width=17cm,clip]{Fig3.eps}
%\caption{(a) Temperature dependence of $\gamma$ taken from the 4 ML sample. The red curve shows the fit based on the equation (\ref{eq3}). (b) Schematic illustration of the motion of magnetic vortices driven by the external current. (c) Image of a sawtooth potential assumed as a ratchet pinning potential in (\ref{eq3}).}
%\label{fig3}
%\end{center}
%\end{figure*} 

In materials under broken inversion and time-reversal symmetries, Onsager's reciprocal theorem allows the linear longitudinal resistance to be different for opposite current directions \cite{Tokura}. Rikken \textit{et al}. heuristically found a general formula for the nonreciprocal transport, also called the MCA, expressed as \cite{Rikken}
\begin{equation}
R = R_0(1 + \gamma BI),
\label{eq1}
\end{equation}
where $\gamma$ is the rectification coefficient, which quantifies the efficiency of generating the nonreciprocal resistance. $R_0$, $B$ and $I$ are the linear resistance, magnetic field and excitation current, respectively. Substituting equation (\ref{eq1}) into Ohm's law $V = RI$ leads to
\begin{equation}
V = R_0 I+ \gamma BR_0I^2.
\label{eq2}
\end{equation}
The first term is the typical linear voltage response to the current and the second term is related to the nonreciprocal transport. Thus, the nonreciprocal response is obtained as a second harmonic signal for the ac excitation current $I_\omega \propto \sin(\omega t)$.



First, we show the temperature dependence of the resistance for the four monolayer (ML) and bilayer (2 ML) samples in Fig.~\ref{fig1}(c). While $T_c$ is low ($\sim$ 100 mK) for bulk $T_{\rm d}$-MoTe$_2$ \cite{Qi}, that for the 4 ML and bilayer samples is 750 mK and 2.2 K, respectively. This large enhancement in $T_c$ for thin layers is consistent with previous studies \cite{Rhodes, MoTe2cn}. Note that here $T_c$ is defined as the temperature where the resistance becomes half of that in the normal state.

%While superconductivity in $T_d$-MoTe$_2$ itself is an interesting subject to explore \cite{Uemura}, we discuss it in more detail elsewhere.

Now let us focus on measuring the nonreciprocal transport in the superconducting state. Figure~\ref{fig1}(d) shows the second-harmonic longitudinal resistance $R_{2\omega}$ for $I_\omega \parallel b$ and for $I_\omega \parallel a$ at 230 mK. $a$ and $b$ are the crystal axes as defined in Fig.~\ref{fig1}(a). A clear peak and dip are observed in $R_{2\omega}$ for $I_\omega \parallel b$. The field-asymmetric $R_{2 \omega}$ signals are in agreement with the MCA in (\ref{eq1}) and are consistent with previous experimental results \cite{MoS2, NbSe2, SrTiO3, Itahashi}. Note that the nonlinearity of the resistance due to the transition between the normal and superconducting state is symmetric in $B$, so it is excluded as the origin of $R_{2 \omega}$. In contrast to the case for $I_\omega \parallel b$, $R_{2 \omega}$ for $I_\omega \parallel a$ is dramatically suppressed. This is also consistent with the geometry of MCA, where the symmetry plane, the directions of the magnetic field and generated second-harmonic voltage are all perpendicular to each other \cite{Tokura}. Note that the finite signal for $I_\omega \parallel a$ is due to misalignment of the electrodes to the crystal axis (see Fig.~\ref{fig1}(b)) \cite{SM}. Below we focus on the geometry where $I_\omega \parallel b$.

The top part of Fig.~\ref{fig2}(a) displays $R_{2 \omega}$ as a function of perpendicular magnetic field measured at different temperatures. The amplitude of the signals monotonically decreases with increasing temperature. Above $T_c$, $R_{2 \omega}$ is completely suppressed, indicating that the effect is related to superconductivity. The bottom part of Fig.~\ref{fig2}(a) displays the $R_\omega$ signals measured simultaneously with the $R_{2 \omega}$ signals. 

Now that we have obtained $R_{2 \omega}$ and $R_{\omega}$, we can estimate the value of the rectification coefficient $\gamma = 2 R_{2 \omega}/(R_\omega B I_\omega)$ \cite{Tokura, MoS2, Hoshino}. To obtain $\gamma$, we use the values of $R_\omega$ and $R_{2 \omega}$ at $B$ where $R_{2 \omega}$ is at a peak. Figure~\ref{fig2}(b) shows that $\gamma$ continues to increase with decreasing temperature and reaches $\gamma$ = 3.1 $\times$ 10$^6$ T$^{-1}$ A$^{-1}$ at 230 mK, the lowest measurement temperature. This value is two to three orders of magnitude larger than that of other two-dimensional superconductors, such as MoS$_2$ and NbSe$_2$ as we will discuss later. In the inset of Fig.~\ref{fig2}(b) we also plot the temperature dependence of $\gamma$ for the bilayer sample, which shows the similar trend with slightly smaller amplitudes than those of the 4 ML sample.

So far, several mechanisms have been proposed to explain the MCA in the superconducting state \cite{Wakatsuki, Hoshino}. The temperature dependence of the signals and the direction of the applied magnetic field are clues with which to determine the mechanism. For example, paraconductivity is one of the mechanisms proposed as an origin of MCA under an in-plane magnetic field. Since it is relevant to thermal fluctuations of the superconducting order parameter, the nonreciprocal signal is slightly enhanced above $T_c$ and suppressed much below $T_c$. On the other hand, the ratchet-like motion of the magnetic vortices enhances the MCA below $T_c$ under a perpendicular magnetic field \cite{Hoshino}. In the mixed state of type-II superconductors, magnetic fluxes penetrate the superconductor, and they are usually trapped by pinning potentials induced by disorder. External current can drive the magnetic fluxes through the Lorenz force as schematically shown in the left image of Fig.~\ref{fig2}(c), if it is large enough to overcome the pinning potential \cite{Tinkham2, Vortex}. In superconductors with broken inversion symmetry, the asymmetry of the crystal structure locally affects the shape of the pinning potentials, making them asymmetric \cite{Rikken, Fente}. In this case, the magnetic vortices can exhibit ratchet-like motion, where the leftward and rightward motion of the vortex is not equivalent \cite{Ratchet1, Ratchet2}. This asymmetry provides a source for nonreciprocal transport \cite{Hoshino, Ideue2, Itahashi, NbSe2, Ratchet1, Ratchet2, Zhu, Ratchet_PRB1, Ratchet_PRB2}. Increasing $\gamma$ with decreasing temperature is consistent with the ratchet-like motion of the magnetic vortices as the origin of the MCA \cite{Hoshino}. Such a temperature dependence is because thermal fluctuations of the magnetic vortices inside the pinning potential, which disturb the ratchet motion, are suppressed with decreasing temperature, and also the coherence length, which determines the diameter of the vortex, becomes smaller, making the vortex more sensitive to the pinning potentials. 
%When a vertical magnetic field $B$ is applied to a type-II superconductor, magnetic fluxes are expelled from the superconductor when $B$ is small (Meissner effect) \cite{Tinkham2}. As $B$ increases, the magnetic fluxes start to penetrate the superconductor in the form of a superconducting magnetic flux quanta when $B$ becomes larger than the first critical magnetic magnetic field ($B_{\rm c1}$). When $B$ increases further and exceeds the second critical magnetic field ($B_{\rm c2}$), superconductivity is totally suppressed. 

\begin{figure*}[tb]
\begin{center}
\includegraphics[width=16cm,clip]{Fig4_2.eps}
\caption{(a) Gate voltage ($V_g$) dependence of $T_c$ for the bilayer sample taken at 230 mK. (b) $R_{2\omega}$ as a function of $B$ at different $V_g$ at 230 mK. (c) $\gamma$ as a function of $V_g$. Inset: Gate voltage dependence of $R_{2 \omega}$.}
\label{fig4}
\end{center}
\end{figure*} 

%In type-II superconductors, there are two critical magnetic fields, $B_{c1}$ and $B_{c2}$ \cite{Tinkham2}. For $B_{\rm c1} < B < B_{\rm c2}$, magnetic fluxes penetrate the superconductor, surrounded by screening supercurrents, and form magnetic vortices. Here superconductivity is sustained outside of magnetic vortices. This state is called the mixed state of type-II superconductors. When an external current is driven, Lorenz force is exerted on a magnetic vortex, and it is displaced in the direction normal to the magnetic field and the current \cite{Vortex}. This displacement of the magnetic vortex induces a finite electric field, normal to the magnetic field and the velocity of the magnetic vortex, which is equivalent to the direction parallel to the external current. While in an ideal superconductor without any disorder even an infinitesimal external current can drive magnetic vortices, in a real sample disorder is inevitable and vortices are pinned by the pinning potentials due to disorder. 


In Fig.~\ref{fig2}(b), we plot the theoretical fit to the experimental data based on the following theoretical expression assuming the ratchet-like motion of magnetic vortices as the origin of the MCA \cite{Hoshino, Itahashi2, SM}:
% Since the vortex number density is proportional to the magnetic field, it is convenience to consider the quantity $\gamma'=\gamma B$ which is $B$-independent and given by
% The nonreciprocal transport signal is given by as
\begin{equation}
\gamma = \frac{\phi_0^* \beta \ell } {W B}
\, 
\frac{g_2(\beta U)}{g_1(\beta U)}
, \label{eq3}
\end{equation}
where $W$ is the width of the sample, $\phi_0^*=h/2|e|$ is the flux quantum and $\beta = 1/k_B T$ is the inverse temperature.
$\ell$ and $U$ are the mean periodicity and the height of the pinning potential for a vortex, respectively. 
We take the simple potential shape shown in the right figure of Fig.~\ref{fig2}(c), where
the dimensionless parameter $f$ controls the asymmetry of the potential. % Equation~\eqref{eq3} .
$g_{1}$ and $g_2$ are dimensionless functions determined from the linear- and second-order responses. 
The ratio is given by $\frac{g_2(\beta U)}{g_1(\beta U)} \sim \frac{f(\beta U)^3}{180}$ for a moving vortex regime with a small ratchet potential.
The fits follow the experimental data qualitatively as shown in Fig.~\ref{fig2}(b),
% in the intermediate temperature range, 
which supports the ratchet-like motion of magnetic vortices as the dominant mechanism for the giant MCA in this system. There are two fitting parameters, and we emphasize that the fits shown in Fig.~\ref{fig2}(b) are obtained by using the same fitting parameters both for the 4 ML and bilayer sample, which also corroborates the validity of our theoretical model (see SM for more details \cite{SM}).
Note that the vortex picture may be justified specifically in the intermediate temperature range below $T_c$.
At higher temperatures, the vortex mechanism should be replaced by superconducting fluctuation and normal contributions \cite{MoS2}.
The origin of the deviation at lower temperatures can be explained by quantum effects of the ratchet-like motion of vortices \cite{PhysRevB.99.064307}, where quantum tunneling through the ratchet potential suppresses the MCA \cite{Itahashi}. Nonetheless, the monotonic enhancement of $\gamma$ with decreasing temperature represents the advantage of $T_{\rm d}$-MoTe$_2$ compared to other high-$\gamma$ noncentrosymmetric superconductors with paraconductivity-based nonreciprocal transport, because even larger $\gamma$ is expected at lower temperatures, and the temperature range for large $\gamma$ is much broader \cite{SrTiO3}.


% In Fig.~\ref{fig2}(b), we plot the thoretical fit to the experimental data based on the following theoretical expression assuming the ratchet motion of magnetic vortices as the origin of the MCA \cite{Hoshino}. 
% \begin{equation}
% \gamma = \frac{\phi_0 L}{BW} \frac{U_0^2 \beta^2 + \beta U_0 \sinh(\beta U_0) - 4 \cosh(\beta U_0) +4}{4 U_0 \sinh^2(\beta U_0/2)},
% \label{eq3}
% \end{equation}
% where $W$ is the width of the sample, $L$ and $U_0$ are the mean periodicity and the height of the pinning potential (see the inset of Fig.~\ref{fig2}(b)), respectively. $\beta = 1/k_B T$. It is easily found that the fit nicely follows the experimental data qualitatively. It provides $L$ = 13.5($\pm$2.2) nm, a rather smaller value in comparison to a rough estimate of $L$ ($\sim \sqrt{\phi_0/B_0}$ = 72 nm) with $B_0$ at which superconductivity is fully suppressed \cite{Hoshino}. 
% %This may be due to the assumption of a simple sawtooth potential as a ratchet potential in (\ref{eq3}), and further improvement of the shape of the potential should provide a better fit with larger $L$. 
% Good agreement for the temperature dependence of $\gamma$ between the theory and experiments supports the ratchet motion of magnetic vortices as the dominant mechanism for the giant MCA in this system. The monotonic enhancement of $\gamma$ with decreasing temperature represents the advantage of $T_{\rm d}$-MoTe$_2$ compared to other high-$\gamma$ noncentrosymmetric superconductors with paraconductivity-based nonreciprocal transport, because even larger $\gamma$ is expected at lower temperatures, and the temperature range for large $\gamma$ is much broader \cite{SrTiO3}. 




%As already reported in several studies, the ratchet motion of magnetic vortices can enhance the MCA in noncentrosymmetric superconductors. In comparison to those studies,  we observe several orders of magnitude larger value of $\gamma$. 
We next discuss the amplitude of $\gamma$. Superconductors with trigonal symmetry are often used in MCA measurements, and $\gamma$ = 8.0 $\times$ 10$^3$ T$^{-1}$ A$^{-1}$ and 3.4 $\times$ 10$^4$ T$^{-1}$ A$^{-1}$ have been obtained for MoS$_2$ and NbSe$_2$, respectively \cite{MoS2, NbSe2}. These values are two or three orders of magnitude smaller than the value of $\gamma$ = 3.1 $\times$ 10$^6$ T$^{-1}$ A$^{-1}$ obtained in our study. The only value comparable to ours reported in the previous studies is $\gamma=$ 3.2 $\times$ 10$^6$ T$^{-1}$ A$^{-1}$ from a SrTiO$_3$ Rashba superconductor under an in-plane magnetic field \cite{SrTiO3}. As for nonsuperconducting material with broken inversion symmetry, (Bi$_{1-x}$Sb$_x$)$_2$Te$_3$ (BST) topological nanowires provide $\gamma \sim 1.0 \times 10^5$ T$^{-1}$ A$^{-1}$ \cite{Ando}. Therefore, the value obtained in our study is one of the largest reported so far \cite{MoS2, NbSe2, Itahashi, Ideue2, Itahashi2}. We attribute this gigantic enhancement in $\gamma$ to the reduced symmetry of $T_{\rm d}$-MoTe$_2$ compared with other noncentrosymmetric superconductors employed in the previous studies. In comparison with other two-dimensional superconductors with trigonal symmetry, $T_{\rm d}$-MoTe$_2$ has reduced symmetry with only one mirror plane for thin layers. This reduced symmetry affects the asymmetry of the pinning potential. Since the symmetry of the pinning potential is crucial for the vortex dynamics, as reported previously \cite{Zhu, Villegas, Palau, Morgan}, the lower symmetry in the pinning potentials may generate larger nonreciprocal signals. 

%The theoretical formula (\ref{eq3}) assumes a purely one-dimensional periodic potential. To consider the reduced symmetry in our system, theoretical estimations based on a two-dimensional potential may provide a better fit to the experimental results. 

Finally, we demonstrate the gate modulation of the MCA for the bilayer $T_{\rm d}$-MoTe$_2$. While gate control of the MCA in the normal state has been studied in a BST topological nanowire \cite{Ando} and at the LaTiO$_3$/SrTiO$_3$ interface \cite{LAO}, it has not been reported yet in superconductors. The primary reason for this is that the concentration of charge carriers in a superconductor is typically high, making it challenging to employ a conventional solid gate to regulate superconducting characteristics due to the electric field screening on the nanometer scale within the material. We can overcome this problem by thinning down $T_{\rm d}$-MoTe$_2$ to a thickness comparable to the screening length \cite{Ma, Ferro}. Figure~\ref{fig4}(a) displays the gate dependence of $T_c$ obtained from the bilayer sample. Here the gate voltage ($V_g$) is applied through a h-BN (34 nm in thickness) as a gate insulator. $T_c$ is successfully modulated by $V_g$, and at $V_g$ = 8 V it is larger by around 20 $\%$ compared with at $V_g$ = $-$8 V. In addition to the variation of $T_c$, the MCA signals are also largely modulated by $V_g$ (Fig.~\ref{fig4}(b)). Figure~\ref{fig4}(c) plots $\gamma$ as a function of $V_g$. Here, $\gamma$ varies with $V_g$, and $\gamma$ at $V_g$ = $-$8 V is almost three times larger than at $V_g$ = 8 V. Note that this large variation is enabled by modulation of not only $R_{2 \omega}$, but also $R_\omega$ and $B$. 

The gate voltage can modulate some parameters relevant to superconductivity, such as $T_c$, $B_{c2}$, the magnetic penetration length $\lambda$ and the coherence length $\xi$. Interestingly, we found that superconductivity becomes more robust as the gate voltage is made more positive. This means that $B_{c2}$ becomes larger and $\xi$ smaller for more positive $V_g$. This trend is counterintuitive when we consider the variation in $R_{2 \omega}$ with $V_g$, because a larger $B_{c2}$ provides larger $U_0$, and a smaller $\xi$ should be more advantageous for the ratchet-like motion. By contrast, $\lambda$, which quantifies the scale for the vortex-vortex interaction, increases as $V_g$ decreases, concomitantly with the decline in superconductivity. Since the importance of the vortex-vortex interaction for the ratchet-like motion has already been discussed in the context of the artificial ratchet potentials \cite{Palau, Nori}, it may also play a role in intrinsic ratchet potentials in noncentrosymmetric superconductors. Although further theoretical studies are required to fully understand our experimental data, the demonstration of gate control of nonreciprocal transport illustrates the rich functionality of superconducting nonreciprocal devices for future applications and also provides key insights into exploring detailed mechanisms behind the ratchet-like motion of magnetic vortices. 
%Since gate tunability of the relative amplitude of the nonreciprocal signal makes superconducting rectification devices more functionable and may help to reveal the mechanism behind the giant MCA, we attempt to control the value of $\gamma$ via an external gate voltage. 

In conclusion, we have shown giant superconducting nonreciprocal transport (MCA) in thin samples of the noncentrosymmetric superconductor $T_{\rm d}$-MoTe$_2$. We obtain 3.1 $\times$ 10$^6$ T$^{-1}$ A$^{-1}$ at 230 mK, one of the largest values of $\gamma$ recorded so far. The temperature dependence of $\gamma$ supports the ratchet-like motion of magnetic vortices as the origin of the nonreciprocal transport. The giant nonreciprocal signal is likely due to the reduced symmetry of the crystal structure of $T_{\rm d}$-MoTe$_2$. We have also demonstrated gate modulation of the MCA in the superconducting state. In bilayer $T_{\rm d}$-MoTe$_2$, we obtain a threefold modulation of $\gamma$ using a typical solid gate. Simultaneous demonstration of the gigantic MCA and its gate modulation in the superconducting state reveals that $T_{\rm d}$-MoTe$_2$ is a promising candidate for realizing an electrically-tunable efficient superconducting rectification devices.  

We gratefully acknowledge M. Imai, S. Sasaki, H. Murofushi and S. Wang for their support in the experiments.
This project is financially supported in part by the JPSJ KAKENHI (grant no. 21H01022, 21H04652, 21K18181, 21H05236, 20H00354 and 19H05790).    



% The \nocite command causes all entries in a bibliography to be printed out
% whether or not they are actually referenced in the text. This is appropriate
% for the sample file to show the different styles of references, but authors
% most likely will not want to use it.
\nocite{*}

\bibliography{prl}% Produces the bibliography via BibTeX.

\end{document}
%
% ****** End of file apssamp.tex ******
