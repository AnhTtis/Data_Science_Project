% ****** Start of file apssamp.tex ******
%
%   This file is part of the APS files in the REVTeX 4.2 distribution.
%   Version 4.2a of REVTeX, December 2014
%
%   Copyright (c) 2014 The American Physical Society.
%
%   See the REVTeX 4 README file for restrictions and more information.
%
% TeX'ing this file requires that you have AMS-LaTeX 2.0 installed
% as well as the rest of the prerequisites for REVTeX 4.2
%
% See the REVTeX 4 README file
% It also requires running BibTeX. The commands are as follows:
%
%  1)  latex apssamp.tex
%  2)  bibtex apssamp
%  3)  latex apssamp.tex
%  4)  latex apssamp.tex
%
\documentclass[%
reprint,
%superscriptaddress,
%groupedaddress,
%unsortedaddress,
%runinaddress,
%frontmatterverbose, 
%preprint,
%preprintnumbers,
%nofootinbib,
%nobibnotes,
%bibnotes,
 amsmath,amssymb,
 aps,
%pra,
%prb,
%rmp,
%prstab,
%prstper,
%floatfix,
]{revtex4-2}
\usepackage{color}
\usepackage{graphicx}% Include figure files
\usepackage{dcolumn}% Align table columns on decimal point
\usepackage{bm}% bold math
\usepackage{amsmath,amsbsy,amssymb}
\usepackage{mathrsfs}
\usepackage{ulem}
\usepackage{mathcomp}
\usepackage{physics}
\usepackage{comment}
\usepackage{ascmac}
\usepackage{here}
\newcommand{\re}[1]{\textcolor{red}{#1}}
\newcommand{\bl}[1]{\textcolor{blue}{#1}}
% MY DEFINITIONS
\newcommand{\diff}{\mathrm{d}}
\newcommand{\imag}{\mathrm{Im}\,}
\newcommand{\imu}{\mathrm{i}}
\newcommand{\epn}{\mathrm{e}}
\newcommand{\sgn}{\mathrm{sgn}\,}
\newcommand{\ua}{\uparrow}
\newcommand{\da}{\downarrow}
\newcommand{\dg}{\dagger}
\newcommand{\la}{\langle}
\newcommand{\ra}{\rangle}
\newcommand{\al}{\alpha}
\newcommand{\sg}{\sigma}
\newcommand{\gm}{\gamma}
\newcommand{\ep}{\varepsilon}
%\usepackage{hyperref}% add hypertext capabilities
%\usepackage[mathlines]{lineno}% Enable numbering of text and display math
%\linenumbers\relax % Commence numbering lines

%\usepackage[showframe,%Uncomment any one of the following lines to test 
%%scale=0.7, marginratio={1:1, 2:3}, ignoreall,% default settings
%%text={7in,10in},centering,
%%margin=1.5in,
%%total={6.5in,8.75in}, top=1.2in, left=0.9in, includefoot,
%%height=10in,a5paper,hmargin={3cm,0.8in},
%]{geometry}

\begin{document}

\preprint{APS/123-QED}

\title{Gate-tunable giant superconducting nonreciprocal transport in few-layer $T_d$-MoTe$_2$}% Force line breaks with \\
%\thanks{A footnote to the article title}%

\author{T. Wakamura,$^1$ M. Hashisaka,$^{1,2}$ S. Hoshino,$^3$, M. Bard$^1$, S. Okazaki,$^4$ T. Sasagawa,$^4$ T. Taniguchi,$^5$ K. Watanabe,$^6$ K. Muraki,$^1$}
\author{N. Kumada$^1$}
 %\altaffiliation[Also at ]{Physics Department, XYZ University.}%Lines break automatically or can be forced with \\
%\author{M. Hashisaka}%
%\author{K. Muraki}
%\author{N. Kumada}
 %\email{Second.Author@institution.edu}
\affiliation{$^1$NTT Basic Research Laboratories, NTT Corporation, 3-1 Morinosato-Wakamiya, Atsugi, 243-0198, Japan}
\affiliation{$^2$Institute for Solid State Physics,University of Tokyo, 5-1-5 Kashiwa-no-ha, Kashiwa 277-8581, Japan}
\affiliation{$^3$Department of Physics, Saitama University, Shimo-Okubo, Saitama 338-8570, Japan}
\affiliation{$^4$Laboratory for Materials and Structures, Tokyo Institute of Technology, Nagatsuta, 226-8503, Japan}
%\collaboration{MUSO Collaboration}%\noaffiliation

 %\homepage{http://www.Second.institution.edu/~Charlie.Author}

\affiliation{$^5$International Center for Materials Nanoarchitectronics, National Institute for Materials and Science, 1-1 Namiki, Tsukuba, 305-0044, Japan}
\affiliation{$^6$Research Center for Functional Materials, National Institute for Materials and Science, 1-1 Namiki, Tsukuba, 305-0044, Japan}


%\author{Delta Author}
%\affiliation{%
 %Authors' institution and/or address\\
 %This line break forced with \textbackslash\textbackslash
%}%

\date{\today}% It is always \today, today,
             %  but any date may be explicitly specified

\begin{abstract}
 We demonstrate gate-tunable giant field-dependent nonreciprocal transport (magnetochiral anisotropy) in a noncentrosymmetric superconductor $T_{\rm d}$-MoTe$_2$ in the thin limit. Giant magnetochiral anisotropy (MCA) with a rectification coefficient (or a figure of merit) $\gamma$ = $3.1 \times 10^6$ T$^{-1}$ A$^{-1}$ is observed at 230 mK, below the superconducting transition temperature ($T_c$). This is one of the largest values reported so far and may be attributed to the reduced symmetry of the crystal structure. The temperature dependence of $\gamma$ indicates that ratchet-like motion of magnetic vortices is the origin of the MCA, as supported by our theoretical model. For bilayer (2 L) $T_{\rm d}$-MoTe$_2$, \textcolor{black}{we can successfully modulate $\gamma$ by gating}. Our experimental results provide a new route to realizing electrically controllable superconducting rectification devices in a single material.  
%\begin{description}
%\item[Usage]
%Secondary publications and information retrieval purposes.
%\item[Structure]
%You may use the \texttt{description} environment to structure your abstract;
%use the optional argument of the \verb+\item+ command to give the category of each item. 
%\end{description}
\end{abstract}

%\keywords{Suggested keywords}%Use showkeys class option if keyword
                              %display desired
\maketitle

%\tableofcontents

%\section{\label{sec:level1}First-level heading:\protect\\ The line
%break was forced \lowercase{via} \textbackslash\textbackslash}



%Symmetry determines fundamental properties of crystals \cite{Tinkham1}. Systems with broken inversion symmetry offer an ideal testbed to investigate the role of symmetry for electrical transport in the system. 
Recent intensive studies on nonreciprocal transport have revealed the potential of using noncentrosymmetric materials or inversion-symmetry-breaking multilayer structures to develop novel rectification devices \textcolor{black}{based on superconducting or Josephson diode effect} \cite{SCDiode_Ono, SCDiode_Ali, SCDiode_NbSe2}. In systems with broken inversion and time-reversal symmetries, Onsager's reciprocal theorem allows the electrical resistance to be different for opposite current directions. This is called magnetochiral anisotropy (MCA), which leads to the rectification effect \cite{Tokura, Ideue, Ando}.   


Broken inversion symmetry is more beneficial in superconductors. Rectification via ratchet-like motion of magnetic vortices was reported more than a decade ago for superconductors with asymmetric artificial magnetic nanostructures or with asymmetric antidots as an asymmetric pinning potential \cite{Ratchet1, Ratchet2, Ratchet_PRB1, Ratchet_PRB2, Zhu, Villegas}. These previous works revealed that as the asymmetry of the pinning potential for magnetic vortices becomes stronger, rectification becomes more efficient. Recent studies have pointed out that the ratchet-like motion of magnetic vortices is also possible in unpatterned noncentrosymmetric superconductors and provides large MCA \cite{Hoshino, Itahashi, Ideue2, Wakatsuki}. In such systems, the asymmetry of the crystal structure intrinsically induces asymmetric pinning potential. In contrast to superconducting films with artificial structures, noncentrosymmetric superconductors do not require complex fabrication processes, offering more facile accessibility to nonreciprocal transports. However, previous reports on MCA in noncentrosymmetric superconductors has mostly concentrated on those with trigonal symmetry \cite{MoS2, NbSe2, Itahashi, Ideue2, Itahashi2}, and other crystal symmetries have been poorly investigated. Toward more efficient rectification via MCA, it is essential to explore noncentrosymmetric superconductors with different symmetries, especially with lower crystal symmetry than trigonal symmetry. Furthermore, regarding future technological applications, facile electrical control of nonreciprocal signals is required, but no previous reports have addressed gate modulation of superconducting MCA.

%Taking the analogy with previous findings on superconductors with artificial asymmetric pinning potentials, the symmetry of the crystal should play a crucial role for the efficiency of the rectification. However, previous reports on MCA in noncentrosymmetric superconductors has been limited to those with trigonal symmetry \cite{MoS2, NbSe2, Itahashi, Ideue2, Itahashi2}. Therefore, it is particularly called for to explore MCA in noncentrosymmetric superconductors with different symmetries, especially with lower symmetry than trigonal symmetry for further enhancement of the efficiency.  


%Van der Waals (vdWs) materials play a significant role in recent condensed matter physics, and growing interests accelerate to find more novel materials. Whereas graphene is an archetype of vdW materials, transition-metal dichalcogenides (TMDs) are major building blocks owing to their diverse electrical and optical properties.

%Molybdenium-ditellulide (MoTe$_2$) is one of those attractive TMDs. MoTe$_2$ has the two phases stable at room temperature. $2H$- phase is semiconducting, and has the same structure as that of MoS$_2$. On the other hand, $1T'$- phase is semimetallic and preserves inversion symmetry.  When $1T'$-MoTe$_2$ is cooled down from room temperture, it exhibits a phase transition around 250 K, and below 250 K the stable crystal structure is $T_d$ phase, where inversion symmetry is broken. From previous first principle calculations and angle-resolved photoemission spectroscopy (ARPES) measurements, it is revealed that $T_d$-MoTe$_2$ is a type-II Weyl semimetal. Interestingly, it becomes superconducting around 100 mK. By contrast, WTe$_2$, a cousin of $T_d$-MoTe$_2$, is well known as a type-II Weyl semimetal in the bulk state and a quantum spin Hall insulator in the monolayer limit, but it does not exhibit superconductivity intrinsically. Therefore $T_d$-MoTe$_2$ is a promising candidate to realize topological superconductivity since it shows both superconductivity and topological order at low temperatures.


In this study, we demonstrate gate-tunable giant MCA in a noncentrosymmetric superconductor $T_{\rm d}$-MoTe$_2$ in the thin limit. $T_{\rm d}$-MoTe$_2$ lacks inversion symmetry and, for thin layers, has only one mirror plane normal to the $b$-axis as shown in Fig.~\ref{fig1}(a). This reduced symmetry of the crystal structure may make the pinning potential for magnetic vortices highly asymmetric, which can contribute to generating large MCA. From the MCA measurements under a perpendicular magnetic field in few-layer $T_{\rm d}$-MoTe$_2$ samples below $T_c$, we obtain 
%The critical temperature ($T_c$) is largely enhanced in comparison to bulk $T_{\rm d}$-MoTe$_2$, whose $T_c$ is around 100 mK. 
the rectification coefficient, i.e. a figure of merit $\gamma$ = 3.1 $\times$ 10$^6$ T$^{-1}$A$^{-1}$ at 230 mK, one of the largest values among those reported so far. The monotonic increase in $\gamma$ with decreasing temperature indicates that the giant MCA is due to the ratchet-like motion of magnetic vortices in the mixed state of the type-II superconductor \cite{Hoshino}. Interestingly, despite that  $T_{\rm d}$-MoTe$_2$ is a semimetal, in the 2 L sample we can successfully modulate the MCA via an external gate voltage and demonstrate the modulation of $\gamma$. This ability to produce a large variation in the nonreciprocal resistance by changing the gate voltage may provide key insights into the mechanisms behind the giant MCA by associating it with modulation of the superconducting properties.  

\begin{figure}[tb!]
\begin{center}
\includegraphics[width=8cm,clip]{Fig1_3_2.eps}
\caption{(a) Top view of the crystal structure, in which broken inversion symmetry is evident. For thin layers, only one mirror plane is present. (b) Optical microscope image of a 4 L device. A thin $T_{\rm d}$-MoTe$_2$ flake is deposited on metallic contacts prepared in advance. (c) Temperature dependence of the resistance of 4 L and 2 L samples. (d) Comparison of $R_{2\omega}$ when current is parallel to the $a$-axis (red) and $b$-axis (blue) taken of the 4 L sample. For the setup with $I \parallel a$, we drive the current between \textcircled{\scriptsize 1} and \textcircled{\scriptsize 5} and measure the voltage between \textcircled{\scriptsize 3} and \textcircled{\scriptsize 4}. For the $I \parallel b$ configuration, the current is driven between \textcircled{\scriptsize 3} and \textcircled{\scriptsize 7}, and the voltage is measured between \textcircled{\scriptsize 2} and \textcircled{\scriptsize 8} (see also Figs.~\ref{current_dist}).}
\label{fig1}
\end{center}
\end{figure}
%Our experimental demonstration of the gate-tunable giant superconducting nonreciprocal effect provides a future prospect to realize novel superconducting device for efficient current rectification at low temperatures.


%Metal ellectrodes are fabricated via the typical electron-beam lithography and electron-beam evaporation of thin Pt (or Au) and Ti. Exfoliated thin flakes are identified by the optical microscope and the atomic-force microscope (AFM). Raman scattering measurements are exploited to determine the direction of the crystal axis of each flake \cite{Raman1}. 

The few-layer $T_{\rm d}$-MoTe$_2$ flakes are mechanically exfoliated from high-quality $T_{\rm d}$-MoTe$_2$ crystals with a residual-resistivity ratio (RRR) $\sim$ 1000 grown via the flux growth method. The mechanical exfoliation is carried out inside an Ar-filled glovebox containing concentrations of O$_2$ and H$_2$O below 0.5 ppm. Independently from the flakes, we prepare metallic electrodes by using typical electron-beam (EB) lithography and EB evaporation on a SiO$_2$/Si substrate (for the 4 L sample) or hexagonal boron-nitride (h-BN) exfoliated on a SiO$_2$/Si substrate (for the 2 L sample). We use Au (for the 4 L sample) or Pt (for the 2 L sample) with Ti as a buffer layer for the electrodes, and the total thickness of the electrodes is set to less than 10 nm to avoid exerting extra strain on the flake. The exfoliated $T_{\rm d}$-MoTe$_2$ is transferred with h-BN picked up by using poli-dimethylpolysiloxane (PDMS) covered with polycarbonate (PC) film. The h-BN flake on $T_{\rm d}$-MoTe$_2$ also acts as a protective layer because $T_{\rm d}$-MoTe$_2$ easily deteriorates when it is exposed to air. The transfer process is also performed inside the glovebox. After taking the sample out of the glovebox, it is immediately mounted on the sample holder of the $^3$He insert and encapsulated by the internal vacuum can then cooled down. Raman scattering and atomic-force microscope (AFM) measurements are carried out at room temperature and in the atmosphere after the transport measurements. 

%\begin{figure*}[tb!]
%\begin{center}
%\includegraphics[width=17cm,clip]{Fig3.eps}
%\caption{(a) Temperature dependence of $\gamma$ taken from the 4 ML sample. The red curve shows the fit based on the equation (\ref{eq3}). (b) Schematic illustration of the motion of magnetic vortices driven by the external current. (c) Image of a sawtooth potential assumed as a ratchet pinning potential in (\ref{eq3}).}
%\label{fig3}
%\end{center}
%\end{figure*} 

In materials under broken inversion and time-reversal symmetries, Onsager's reciprocal theorem allows the linear longitudinal resistance to be different for opposite current directions \cite{Tokura}. Rikken \textit{et al}. heuristically found a general formula for the nonreciprocal transport, also called the MCA, expressed as \cite{Rikken}
\begin{equation}
R = R_0(1 + \gamma BI),
\label{eq1}
\end{equation}
where $\gamma$ is the rectification coefficient, which quantifies the efficiency of generating the nonreciprocal resistance. $R_0$, $B$ and $I$ are the linear resistance, magnetic field, and excitation current, respectively. Substituting equation (\ref{eq1}) into Ohm's law $V = RI$ leads to
\begin{equation}
V = R_0 I+ \gamma BR_0I^2.
\label{eq2}
\end{equation}
The first term is the typical linear voltage response to the current and the second term is related to the nonreciprocal transport. Thus, the nonreciprocal response is obtained as a second harmonic signal for the ac excitation current $I_\omega \propto \sin(\omega t)$.

\begin{figure*}[tb]
\begin{center}
\includegraphics[width=16cm,clip]{Fig2_5.eps}
\caption{(a) Experimental data from the 4 L sample. ~Top: Nonreciprocal resistance ($R_{\rm 2 \omega}$ = $V_{\rm 2 \omega}/I_{\rm \omega}$, $I_\omega$ = 100 nA) measured at different temperatures. Middle: $R_\omega$ signals measured simultaneously with $R_{2 \omega}$. \textcolor{black}{Bottom: $\gamma$ as a function of $B$ at different temperatures. The $B$ values where $R_{2 \omega}$ shows a peak ($B_{\rm peak}$) are marked with open triangles.} \textcolor{black}{Inset in the middle figure: $I_\omega-V_\omega$ curve at $B$ = 0 T and $T$ = 230 mK. Zero resistance is observed around $I_\omega$ = 0, but the shape of the curve is somewhat rounded, partly because of the 2D nature of superconductivity.} 
(b) Temperature dependence of $\gamma$ taken from the 4 L sample. The orange curve shows the fit based on equation (\ref{eq3}). Inset: Experimental data of $\gamma$ as a function of temperature obtained from the 2 L sample with the fit. \textcolor{black}{Yellow shaded regions in the main figure and the inset represent the quantum metal (QM) phase (see also Appendix E).}
(c) Top: Schematic illustration of the motion of magnetic vortices with the velocity $\mathbf{v}$ driven by an external current $\mathbf{J}$, which generates an electric field $\mathbf{E} = \mathbf{B} \times \mathbf{v}$. Bottom: Image of a sawtooth potential assumed as the ratchet pinning potential in (\ref{eq3}).}
\label{fig2}
\end{center}
\end{figure*} 

First, we show the temperature dependence of the resistance for the four layer [4 L, see Fig.~\ref{fig1}(b)] and bilayer (2 L) samples in Fig.~\ref{fig1}(c). While $T_c$ is low ($\sim$ 100 mK) for bulk $T_{\rm d}$-MoTe$_2$ \cite{Qi}, that for the 4 L and 2 L samples is 750 mK and 2.2 K, respectively. This large enhancement in $T_c$ for thin layers is consistent with previous studies \cite{Rhodes, MoTe2cn}. Note that here $T_c$ is defined as the temperature where the resistance becomes half of that in the normal state. 

%While superconductivity in $T_d$-MoTe$_2$ itself is an interesting subject to explore \cite{Uemura}, we discuss it in more detail elsewhere.

Now let us focus on measuring the nonreciprocal transport in the superconducting state. Figure~\ref{fig1}(d) shows the second-harmonic longitudinal resistance $R_{2\omega}$ for $I_\omega \parallel b$ and for $I_\omega \parallel a$ at 230 mK. $a$ and $b$ are the crystal axes as defined in Fig.~\ref{fig1}(a), and the $b$-axis is orthogonal to the mirror plane. A clear peak and dip are observed in $R_{2\omega}$ for $I_\omega \parallel b$. The field-asymmetric $R_{2 \omega}$ signals are in agreement with the MCA in (\ref{eq1}) and are consistent with previous experimental results \cite{MoS2, NbSe2, SrTiO3, Itahashi}. \textcolor{black}{The suppression of $R_{2 \omega}$ at higher fields is due to the suppression of superconductivity by a magnetic field.} Note that the nonlinearity of the resistance due to the transition between the normal and superconducting state is symmetric in $B$, so it is excluded as the origin of $R_{2 \omega}$. In contrast to the case for $I_\omega \parallel b$, $R_{2 \omega}$ for $I_\omega \parallel a$ is dramatically suppressed. This is also consistent with the geometry of MCA, where the symmetry plane, the directions of the magnetic field, and generated second-harmonic voltage are all perpendicular to each other \cite{Tokura}. Note that the finite signal for $I_\omega \parallel a$ is due to misalignment of the electrodes to the crystal axis [see Fig.~\ref{fig1}(b) and also Appendices B, F and G]. Below we focus on the geometry where $I_\omega \parallel b$.

The top part of Fig.~\ref{fig2}(a) displays $R_{2 \omega}$ as a function of perpendicular magnetic field measured at different temperatures. The amplitude of the signals monotonically decreases with increasing temperature. Above $T_c$, $R_{2 \omega}$ is completely suppressed, indicating that the effect is related to superconductivity. The middle part of Fig.~\ref{fig2}(a) shows the $R_\omega$ signals measured simultaneously with the $R_{2 \omega}$ signals. \textcolor{black}{The inset is $I_\omega-V_\omega$ curve measured at $B$ = 0 T and $T$ = 230 mK, showing the exactly zero resistance state at low $I_\omega$. At $I_\omega$ = 100 nA, $R_\omega$ starts to deviate from zero immediately after a magnetic field is turned on.}

Now that we have obtained $R_{2 \omega}$ and $R_{\omega}$, we can estimate the value of the rectification coefficient $\gamma = 2 R_{2 \omega}/(R_\omega B I_\omega)$ \cite{Tokura, MoS2, Hoshino}. \textcolor{black}{As shown in the bottom of Fig.~\ref{fig2}(a), $\gamma$ depends on $B$ and increases rapidly as $B \rightarrow$ 0, particularly at the lowest temperature. This is due to the decrease in $R_\omega$ and $B$ in the denominator. In the following, we use $\gamma$ at $B$ ($\equiv B_{\rm peak}$), at which $R_{2 \omega}$ is at a peak [triangles in the bottom of Fig.~\ref{fig2}(a)] as a representative value. This definition of representative $\gamma$ is often used in previous studies and useful for quantitative discussion. \cite{Masuko, MoS2, NbSe2, Itahashi, Ideue2}.} Figure~\ref{fig2}(b) shows that $\gamma$ continues to increase with decreasing temperature and reaches $\gamma$ = 3.1 $\times$ 10$^6$ T$^{-1}$ A$^{-1}$ at 230 mK, the lowest measurement temperature. This value is two to three orders of magnitude larger than that of other two-dimensional superconductors, such as MoS$_2$ and NbSe$_2$ as we will discuss later. In the inset of Fig.~\ref{fig2}(b), we also plot the temperature dependence of $\gamma$ for the 2 L sample, which shows the similar trend with slightly smaller amplitudes. \textcolor{black}{The smaller $\gamma$ in the 2 L sample is due to larger $B_{\rm peak}$ caused by more robust superconductivity (see Appendix D)}. 

\begin{table*}[tb!]
\caption{\label{tab:table3}Summary of $\gamma$, \textcolor{black}{$\gamma W$, $\gamma Wt$} and $\gamma B_{\rm peak}$ for different 2D superconductors}
\begin{ruledtabular}
\begin{tabular}{cccccccccc}
 %&\multicolumn{2}{c}{$D_{4h}^1$}&\multicolumn{2}{c}{$D_{4h}^5$}\\
 Material&Symmetry&$B_{\rm peak}$ [T] & $R_{2 \omega} [\Omega]$ & $W$ [$\mu$m] & $\gamma$ [A$^{-1}$T$^{-1}$] & \textcolor{black}{$\gamma W$ [A$^{-1}$T$^{-1}$m]} & \textcolor{black}{$\gamma Wt$ [A$^{-1}$T$^{-1}$m$^2$]} &$\gamma B_{\rm peak}$ [A$^{-1}$] & ref.\\ \hline
 MoTe$_2$ 2 L & $C_{s}$ & 0.35 & 1.8 & 3 & 6.7 $\times 10^5$ & 2.0 & 2.8 $\times 10^{-9}$ & 2.34 $\times 10^5$ & This study \\
 MoTe$_2$ 4 L & $C_{s}$ & 0.050 & 1.3 & 6 & 3.1 $\times 10^6$ & 19 & 1.1 $\times 10^{-7}$ & 1.54 $\times 10^5$ & This study \\
MoS$_2$\footnotemark[1]& $C_{3v}$ & 0.50 & 0.65 & 3 & 4.6$\times10^3$ & 1.4 $\times 10^{-2}$ & - & 2.32$\times10^3$ & \cite{Wakatsuki}\\
 NbSe$_2$ 5 L& $D_{3h}\footnotemark[2]$ & 3.0 & 0.062 & 5 & 2.8$\times10^2$ & 1.4$\times 10^{-3}$ &4.7 $\times 10^{-11}$ & 8.49$\times10^2$ & \cite{NbSe2} \\
 SrTiO$_3$\footnotemark[1]& - & 0.050 & 0.29 & 80 & 3.2 $\times 10^6$ & 2.6 $\times 10^2$ &-& 1.60 $\times 10^5$ & \cite{Itahashi} \\
\end{tabular}
\end{ruledtabular}
\footnotetext[1]{\textcolor{black}{Because superconductivity is induced by the ionic liquid gating close to the surface, the exact value of $t$ is difficult to define.}}
\footnotetext[2]{The point group of bulk NbSe$_2$ is $D_{6h}$. On the other hand, $D_{3h}$ is the point group for thin NbSe$_2$ with the odd number of layers.}
\end{table*}

So far, several mechanisms have been proposed to explain the MCA in the superconducting state \cite{Wakatsuki, Hoshino}. The temperature dependence of the signals and the direction of the applied magnetic field are clues for identifying the mechanism. For example, paraconductivity is a mechanism proposed as an origin of MCA under an in-plane magnetic field \cite{Hoshino,SrTiO3}. Since it is relevant to thermal fluctuations of the superconducting order parameter, the nonreciprocal signal is slightly enhanced above $T_c$ and suppressed much below $T_c$. On the other hand, ratchet-like motion of magnetic vortices enhances the MCA below $T_c$ under a perpendicular magnetic field \cite{Hoshino}. In the mixed state of type-II superconductors, magnetic fluxes penetrate the superconductor, and they are usually trapped by pinning potentials induced by disorder \textcolor{black}{such as underlying discrete lattice structure, defects, and impurities}. External current can drive the magnetic fluxes through the Lorenz force as schematically shown in the top image of Fig.~\ref{fig2}(c), if it is large enough to overcome the pinning potential \cite{Tinkham2, Vortex}. In superconductors with broken inversion symmetry, the asymmetry of the crystal structure locally affects the shape of the pinning potentials, making them asymmetric \cite{Rikken, Fente}. In this case, the magnetic vortices can exhibit ratchet-like motion, where the leftward and rightward motion of the vortex is not equivalent \cite{Hoshino, Ideue2, Itahashi, NbSe2, Ratchet1, Ratchet2, Zhu, Ratchet_PRB1, Ratchet_PRB2}. This asymmetry provides a source for nonreciprocal transport. 
The ratchet-like motion of magnetic vortices provides increasing $\gamma$ with decreasing temperature because thermal fluctuations of the magnetic vortices inside the pinning potential, which disturb the ratchet-like motion, are suppressed with decreasing temperature, and also the coherence length, which determines the diameter of the vortex, becomes smaller, making the vortex more sensitive to the pinning potentials.
%\textcolor{black}{Note that previous studies pointed out that the asymmetric spin-orbit coupling together with an inplane magnetic field also induces the asymmetric deformation of the pinning potential \cite{Vortex_Ind1}.}

%\textcolor{blue}{Recently, the deformation of vortex pinning potential and enhanced pinning induced by inplane magnetic field has been revealed via the vortex inductance measurements \cite{Vortex_Ind1}, which has been becoming a powerful probe for exploring nonreciprocal transport in superconductors \cite{Vortex_Ind2, Vortex_Ind3}. Note that in our system, on the other hand, the deformation of the pinning potential is generated by the asymmetric crystal structure considering the absence of inplane magnetic fields during the measurements.} 

%\textcolor{black}{Note that the asymmetric surface barrier for magnetic vortices due to the inequivalency of the edge shape cannot explain our experimental data. Previous studies revealed that the surface barrier becomes dominant in vortex dynamics close to $T_c$, at which the bulk pinning becomes extremely weak, and its effect diminishes with decreasing temperature \cite{SB1, SB2, SB3, SB4} This temperature dependence is at odds with increasing $\gamma$ with decreasing temperature observed in our experiments. Detailed discussions are found in \cite{SM}.}

%When a vertical magnetic field $B$ is applied to a type-II superconductor, magnetic fluxes are expelled from the superconductor when $B$ is small (Meissner effect) \cite{Tinkham2}. As $B$ increases, the magnetic fluxes start to penetrate the superconductor in the form of a superconducting magnetic flux quanta when $B$ becomes larger than the first critical magnetic magnetic field ($B_{\rm c1}$). When $B$ increases further and exceeds the second critical magnetic field ($B_{\rm c2}$), superconductivity is totally suppressed. 

\begin{figure*}[tb]
\begin{center}
\includegraphics[width=16.5cm,clip]{Fig4_2_2.eps}
\caption{(a) Gate voltage ($V_g$) dependence of $T_c$ for the 2 L sample taken at 230 mK. The inset shows the $R$-$T$ curves for the different $V_g$. (b) $R_{2\omega}$ as a function of $B$ at different $V_g$ at 230 mK. (c) $\gamma$ as a function of $V_g$. Inset: Gate voltage dependence of $R_{2 \omega}$.}
\label{fig4}
\end{center}
\end{figure*} 

%In type-II superconductors, there are two critical magnetic fields, $B_{c1}$ and $B_{c2}$ \cite{Tinkham2}. For $B_{\rm c1} < B < B_{\rm c2}$, magnetic fluxes penetrate the superconductor, surrounded by screening supercurrents, and form magnetic vortices. Here superconductivity is sustained outside of magnetic vortices. This state is called the mixed state of type-II superconductors. When an external current is driven, Lorenz force is exerted on a magnetic vortex, and it is displaced in the direction normal to the magnetic field and the current \cite{Vortex}. This displacement of the magnetic vortex induces a finite electric field, normal to the magnetic field and the velocity of the magnetic vortex, which is equivalent to the direction parallel to the external current. While in an ideal superconductor without any disorder even an infinitesimal external current can drive magnetic vortices, in a real sample disorder is inevitable and vortices are pinned by the pinning potentials due to disorder. 
\textcolor{black}{We next discuss the temperature dependence of $\gamma$ in more detail by comparing the experimental data with a theoretical model based on the ratchet-like motion of magnetic vortices. Solid lines in Fig.~\ref{fig2}(b) represent the theoretical curve following the expression (see Appendix H for details) \cite{Hoshino, Itahashi2}}:
% Since the vortex number density is proportional to the magnetic field, it is convenience to consider the quantity $\gamma'=\gamma B$ which is $B$-independent and given by
% The nonreciprocal transport signal is given by as
\begin{equation}
\gamma = \frac{\phi_0^* \beta \ell } {W B}
\, 
\frac{g_2(\beta U)}{g_1(\beta U)}
, \label{eq3}
\end{equation}
where $W$ is the width of the sample, $\phi_0^*=h/2|e|$ is the flux quantum and $\beta = 1/k_B T$ is the inverse temperature.
$\ell$ and $U$ are the mean periodicity and the height of the pinning potential for a vortex, respectively. \textcolor{black}{Here, in order to account for vortex dynamics qualitatively, we phenomenologically introduce the pinning potential and employ the Langevin dynamics.}
We take the simple potential shape shown in the bottom figure of Fig.~\ref{fig2}(c), where
the dimensionless parameter $f$ controls the asymmetry of the potential. % Equation~\eqref{eq3} .
$g_{1}$ and $g_2$ are dimensionless functions determined from the linear- and second-order responses. 
The ratio is given by $\frac{g_2(\beta U)}{g_1(\beta U)} \sim \frac{f(\beta U)^3}{180}$ for a moving vortex regime with a small ratchet potential.
The curves follow the experimental data qualitatively in the intermediate temperature region as shown in Fig.~\ref{fig2}(b),
% in the intermediate temperature range, 
which supports the ratchet-like motion of magnetic vortices as the dominant mechanism for the giant MCA in this system. 
In the theoretical curves, there are two fitting parameters $\alpha$ and $f$, and the former defines the exponent in the temperature dependence of the ratchet potential $U \sim U_0 [(T_c - T)/T_c]^\alpha$ with $U_0$ as $U(T = 0)$. Fitting the experimental results in the intermediate temperature region provides $\alpha$ and $f$. By using these values and $U$ estimated from the critical current density $j_c$ at a small magnetic field, we obtain $U_0$ = 0.10 (0.17) meV for the 4 L (2 L) sample. Note that these $U_0$ values are consistent with $U_0$ = 0.10 (0.56) meV, alternatively obtained from the temperature dependence of the resistance under different magnetic fields (see Appendix E).  %There are two fitting parameters $\alpha$ and $f$, and the former defines the exponent in the temperature dependence of the ratchet potential $U$ (see Section S9 in SI). 
%We emphasize that the fits shown in Fig.~\ref{fig2}(b) are obtained by using the same fitting parameters both for the 4 L and 2 L sample, which also corroborates the validity of our theoretical model (see Section S10 in SI for more details).
%Note that the vortex picture can be justified particularly in the intermediate temperature range below $T_c$.

\textcolor{black}{In contrast to the intermediate temperature region, the fits substantially overestimate $\gamma$ at lower temperatures. This can be explained by appearance of the quantum metal phase, in which quantum tunneling of vortices through the pinning potential suppresses the ratchet-like motion and thus MCA \cite{Itahashi, Hamamoto}. Indeed, the temperature range where the experimental data deviate from the theoretical curve corresponds to the region for the quantum metal phase [yellow shaded region on Fig. \ref{fig2}(b)] in the vortex phase diagram (see Appendix E). Note that the current density in the present measurements is small enough to preserve the quantum metal phase \cite{Itahashi}.
%This indicates that the emergence of the quantum metal phase is revealed by comparing the experimental data with the theoretical model on the basis of the ratchet motion of magnetic vortices. 
At higher temperatures, on the other hand, rectification mechanism is replaced by superconducting fluctuation and normal contributions \cite{Hoshino,SrTiO3}.}



% In Fig.~\ref{fig2}(b), we plot the thoretical fit to the experimental data based on the following theoretical expression assuming the ratchet motion of magnetic vortices as the origin of the MCA \cite{Hoshino}. 
% \begin{equation}
% \gamma = \frac{\phi_0 L}{BW} \frac{U_0^2 \beta^2 + \beta U_0 \sinh(\beta U_0) - 4 \cosh(\beta U_0) +4}{4 U_0 \sinh^2(\beta U_0/2)},
% \label{eq3}
% \end{equation}
% where $W$ is the width of the sample, $L$ and $U_0$ are the mean periodicity and the height of the pinning potential (see the inset of Fig.~\ref{fig2}(b)), respectively. $\beta = 1/k_B T$. It is easily found that the fit nicely follows the experimental data qualitatively. It provides $L$ = 13.5($\pm$2.2) nm, a rather smaller value in comparison to a rough estimate of $L$ ($\sim \sqrt{\phi_0/B_0}$ = 72 nm) with $B_0$ at which superconductivity is fully suppressed \cite{Hoshino}. 
% %This may be due to the assumption of a simple sawtooth potential as a ratchet potential in (\ref{eq3}), and further improvement of the shape of the potential should provide a better fit with larger $L$. 
% Good agreement for the temperature dependence of $\gamma$ between the theory and experiments supports the ratchet motion of magnetic vortices as the dominant mechanism for the giant MCA in this system. The monotonic enhancement of $\gamma$ with decreasing temperature represents the advantage of $T_{\rm d}$-MoTe$_2$ compared to other high-$\gamma$ noncentrosymmetric superconductors with paraconductivity-based nonreciprocal transport, because even larger $\gamma$ is expected at lower temperatures, and the temperature range for large $\gamma$ is much broader \cite{SrTiO3}. 

%\begin{table}[tb!]
%\caption{\label{tab:table3}Summary of $B_{\rm peak}$, $\gamma$, $\gamma W/L$ and $\gamma B_{\rm peak}$ for different superconductors}
%\begin{ruledtabular}
%\begin{tabular}{ccccc}
 %&\multicolumn{2}{c}{$D_{4h}^1$}&\multicolumn{2}{c}{$D_{4h}^5$}\\
 %Material&$B_{\rm peak}$ &$\gamma$ &$\gamma W/L$&$\gamma B_{\rm peak}$\\ 
 %&[T] & [A$^{-1}$T$^{-1}$]&[A$^{-1}$T$^{-1}$m] &[A$^{-1}$] \\ 
%\hline
 %MoTe$_2$ 2 L & 0.35 & 6.7 $\times 10^5$ & 1.0 $\times 10^6$ & 2.34 $\times 10^5$ \\
 %MoTe$_2$ 4 L & 0.050 & 3.1 $\times 10^6$ & 4.6 $\times 10^6$ & 1.54 $\times 10^5$ \\
%MoS$_2$ \cite{MoS2} & 0.50 & 4.6$\times10^3$ & 7.0$\times10^2$ & 2.32$\times10^3$\\
 %NbSe$_2$ \cite{NbSe2}& 3.0 & 2.8$\times10^2$ & 7.1$\times10^2$ & 8.49$\times10^2$\\
 %SrTiO$_3$ \cite{SrTiO3} & 0.050 & 3.2 $\times 10^6$ & - & 1.60 $\times 10^5$\\
%\end{tabular}
%\end{ruledtabular}
%\end{table}


%As already reported in several studies, the ratchet motion of magnetic vortices can enhance the MCA in noncentrosymmetric superconductors. In comparison to those studies,  we observe several orders of magnitude larger value of $\gamma$. 
We then compare the amplitude of $\gamma$ with those in different materials. The summary of $\gamma$ in different noncentrosymmetric superconductors is shown in Table I. $T_{\rm d}$-MoTe$_2$ exhibits larger $\gamma$ by several orders of magnitude than those for trigonal superconductors such as MoS$_2$ and NbSe$_2$. The only value comparable to ours reported in the previous studies is that from a SrTiO$_3$ Rashba superconductor under an in-plane magnetic field \cite{SrTiO3}. 
%As for nonsuperconducting material with broken inversion symmetry, (Bi$_{1-x}$Sb$_x$)$_2$Te$_3$ (BST) topological nanowires provide $\gamma \sim 1.0 \times 10^5$ T$^{-1}$ A$^{-1}$ \cite{Ando}. 
Therefore, the value obtained in our study is one of the largest reported so far \cite{MoS2, NbSe2, Itahashi, Ideue2, Itahashi2}. 

In Table I, we also show other quantities taking into account the difference in the sample geometry and also $B_{\rm peak}$. Because $\gamma$ depends on the sample width \textcolor{black}{and thickness}, we compare $\gamma W$ and $\gamma W\textcolor{black}{t}$, where $W$ \textcolor{black}{and $t$} are the width \textcolor{black}{and the thickness} of the sample, respectively. It is evident that the values for $T_{\rm d}$-MoTe$_2$ surpasses substantially those for other noncentrosymmetric superconductors, \textcolor{black}{except for SrTiO$_3$ due to the much larger sample width.}
%In addition, generation of large $\gamma$ with smaller current density at smaller magnetic field is of technological advantage, because large nonreciprocal signal is more easily accessible. In comparison to the current density in the previous studies (such as $\sim$ 5 A/m for MoS$_2$ \cite{Itahashi} and $\sim$ 4 A/m for NbSe$_2$), we can successfully obtain large nonreciprocal signals with much smaller current density (0.014 A/m for the 4 L and 0.040 A/m for the 2 L sample). The current density for the 4 L sample is comparable to that for SrTiO$_3$ ($\sim$ 0.012 A/m). 
%Note that the large values of $\gamma$ are not obtained because of small current (or current density). From the relation $\gamma = 2 R_{2 \omega}/(R_\omega B I_\omega)$, one may naively think that $\gamma$ becomes infinite as $I_\omega \rightarrow 0$ (or $B \rightarrow 0$). However, this is not correct, because $R_{2 \omega}$ ($R_\omega$) is also a function of these parameters, namely, $R_{2 \omega}(I,B) (R_\omega(B))$. 
The values of $\gamma B_{\rm peak}$ are also compared, which consider the difference in $B_{\rm peak}$. $T_{\rm d}$-MoTe$_2$ shows much larger values than those for MoS$_2$ and NbSe$_2$, and is comparable or slightly larger than the value for SrTiO$_3$. These comparisons corroborate the enhanced MCA in $T_{\rm d}$-MoTe$_2$ among van der Waals superconductors. Note that while our lowest measurement temperature ($\equiv T_{\rm meas}$) is lower than that in previous studies for MoS$_2$ or NbSe$_2$ \cite{MoS2, NbSe2, Itahashi}, $T_c$ of $T_{\rm d}$-MoTe$_2$ is lower, and the energy scale ($k_B T_{\rm meas})$ relative to the superconducting gap is comparable. Thus we can rule out lower measurement temperature as an origin for the substantial nonreciprocal signals. 
%\textcolor{black}{If we use the current density ($j =I/W$, where $W$ is the width of the sample) instead of the current $I$, another definition of the figure of merit $\gamma' = \gamma W$ can be employed. In this case, owing to the larger width of the sample $\gamma W$ = 2.56 $\times$ 10$^2$ A$^{-1}$ T$^{-1}$m for SrTiO$_3$, larger than 2.00 A$^{-1}$ T$^{-1}$m and 18.5 A$^{-1}$ T$^{-1}$m for the 2 L and 4 L sample, respectively. Nevertheless, the figure of merit of $T_{\rm d}$-MoTe$_2$ is still larger than those for MoS$_2$ (1.39$\times 10^{-3}$ A$^{-1}$ T$^{-1}$m) and NbSe$_2$ (1.41$\times 10^{-3}$ A$^{-1}$ T$^{-1}$m) by several orders of magnitude, representing the advantage of $T_{\rm d}$-MoTe$_2$ for larger MCA.} 
%Detailed discussions are found in \cite{SM} including more detailed comparison between different noncentrosymmetric superconductors.


%Nonreciprocal signals obtained from $T_{\rm d}$-MoTe$_2$ are two or three orders of magnitude larger than those for MoS$_2$ and NbSe$_2$ employed in the previous studies. 
Difference in the crystal symmetry is likely to be the origin of the large variation in $\gamma$. In comparison with other two-dimensional trigonal superconductors, $T_{\rm d}$-MoTe$_2$ has reduced symmetry with only one mirror plane for thin layers. This reduced symmetry affects the asymmetry of the pinning potential. Since the symmetry of the pinning potential is crucial for the vortex dynamics, as reported previously \cite{Zhu, Villegas, Palau, Morgan}, the lower symmetry in the pinning potentials may generate larger nonreciprocal signals. Further theoretical study is required to scrutinize the effect of the symmetry reduction on the amplitude of nonreciprocal signals.

%The theoretical formula (\ref{eq3}) assumes a purely one-dimensional periodic potential. To consider the reduced symmetry in our system, theoretical estimations based on a two-dimensional potential may provide a better fit to the experimental results. 

Finally, we demonstrate the gate modulation of the MCA for the 2 L $T_{\rm d}$-MoTe$_2$. While gate control of the MCA in the normal state has been studied in a BST topological nanowire \cite{Ando} and at the LaAlO$_3$/SrTiO$_3$ interface \cite{LAO}, it has not been reported yet in superconductors. The primary reason is that the concentration of charge carriers in a superconductor is typically high, making it challenging to employ a conventional solid gate to regulate superconducting characteristics due to the electric field screening on the nanometer scale within the material. We can overcome this problem by thinning down $T_{\rm d}$-MoTe$_2$ to a thickness comparable to the screening length \cite{Ma, Ferro}. \textcolor{black}{Note that the screening length is estimated to be $\sim$ 0.4 nm, the same order of the length as the one layer thickness of $T_{\rm d}$-MoTe$_2$.} Figure~\ref{fig4}(a) displays the gate dependence of $T_c$ obtained from the 2 L sample. Here the gate voltage ($V_g$) is applied through a h-BN (34 nm in thickness) as a gate insulator. $T_c$ is successfully modulated by $V_g$, and at $V_g$ = 8 V it is larger by around 20 $\%$ compared with at $V_g$ = $-$8 V. In addition to the variation of $T_c$, the MCA signals are also modulated by $V_g$ [Fig.~\ref{fig4}(b)]. We find that not only the height of the peak but $B_{\rm peak}$ is also modulated by $V_g$, suggesting that the gating largely affects the vortex dynamics. Figure~\ref{fig4}(c) plots $\gamma$ as a function of $V_g$, showing the large variation of $\gamma$.

The gate voltage can modulate some parameters relevant to superconductivity, such as $T_c$, $B_{c2}$, the magnetic penetration length $\lambda$ and the coherence length $\xi$. \textcolor{black}{$\lambda$ is a characteristic scale for vortex-vortex interaction, whose crucial role was previously pointed out in the ratchet-like motion \cite{Nori, Palau}, whereas nonreciprocal signals in the superconducting state are largely affected by the variation of $T_c$ \cite{Wakatsuki, Hoshino, Itahashi2}. As the gating modulates these parameters in a complex manner, at present we cannot identify the dominant contribution to the large variation of $\gamma$. We hope that our results stimulate further theoretical as well as experimental investigations to reveal the role of the crystal symmetry for MCA and vortex dynamics in noncentrosymmetric superconductors.}

%\textcolor{red}{One possible explanation for the large variation of $\gamma$ with $V_g$ is that it is ascribed to the variation of $T_c$. The ratio between nonreciprocal transport signals in the superconducting state and that in the normal state is roughly controlled by the factor $\epsilon_F/k_B T_c$ with the Fermi energy $\epsilon_F$, which qualitatively explains why a huge enhancement of nonreciprocal signals is expected in the superconducting state \cite{MoS2, Itahashi2}. Assuming that $T_c$ is more sensitively affected by gating compared to $\epsilon_F$ (or the Fermi level) in the normal state, the transport signal may become larger if $T_c$ becomes smaller. This behavior is consistent with the observed tendency in Fig.~\ref{fig4}(c). However, we stress that the large enhancement of $\gamma$ in $T_{\rm d}$-MoTe$_2$ in comparison to MoS$_2$ and NbSe$_2$ cannot be explained simply by the smaller $T_c$, considering that $\gamma$ is 2-3 orders of magnitude larger in $T_{\rm d}$-MoTe$_2$ than the others whilst $T_c$ is different by just several factors. Note also that due to the different position of the Fermi level, we cannot attribute the difference in $T_c$ to the difference in $\gamma$ between the 2 L and 4 L sample. Because the gating affects not only $T_c$ but also the other quantities such as pinning and/or vortex-vortex interactions, microscopic descriptions are required to clarify the more detailed origins of these observations.}

 %This means that $B_{c2}$ becomes larger and $\xi$ smaller for more positive $V_g$. This trend is counterintuitive when we consider the variation in $R_{2 \omega}$ with $V_g$, because a larger $B_{c2}$ provides larger $U_0$, and a smaller $\xi$ should be more advantageous for the ratchet-like motion. By contrast, $\lambda$, which quantifies the scale for the vortex-vortex interaction, increases as $V_g$ decreases, concomitantly with the decline in superconductivity. Since the importance of the vortex-vortex interaction for the ratchet-like motion has already been discussed in the context of the artificial ratchet potentials \cite{Palau, Nori}, it may also play a role in intrinsic ratchet potentials in noncentrosymmetric superconductors. Although further theoretical studies are required to fully understand our experimental data, the demonstration of gate control of nonreciprocal transport illustrates rich functionality of superconducting nonreciprocal devices for future applications and also provides key insights into exploring detailed mechanisms behind the ratchet-like motion of magnetic vortices. 
%Since gate tunability of the relative amplitude of the nonreciprocal signal makes superconducting rectification devices more functionable and may help to reveal the mechanism behind the giant MCA, we attempt to control the value of $\gamma$ via an external gate voltage. 
%\begin{figure*}[tb!]
%\begin{center}
%\includegraphics[width=16cm,clip]{Fig_Appendix1.eps}
%\caption{(a) Angular dependence of the Raman intensity at the 163 cm$^{-1}$ Raman shift for the 4 L sample. (b) Optical image of the 4 L sample. The orientation of the sample corresponds to the orientation of the angle in (a). The scale bar is 5 $\mu$m. 
%\textcolor{black}{(c) Angular dependence for the 2 L sample. (d) Sample picture of the 2 L sample, with 5 $\mu$m scale bar.}}
%\label{figAppx}
%\end{center}
%\end{figure*}

In conclusion, we showed giant superconducting nonreciprocal transport (MCA) in thin samples of the noncentrosymmetric superconductor $T_{\rm d}$-MoTe$_2$, which has only one mirror plane. We obtained 3.1 $\times$ 10$^6$ T$^{-1}$ A$^{-1}$ at 230 mK, one of the largest values of $\gamma$ recorded so far. The temperature dependence of $\gamma$ supports the ratchet-like motion of magnetic vortices as the origin of the nonreciprocal transport. The reduced symmetry of the crystal structure of $T_{\rm d}$-MoTe$_2$ may contribute to the large nonreciprocal signals. We also demonstrated gate modulation of the MCA in the superconducting state. In the 2 L $T_{\rm d}$-MoTe$_2$, we obtain a substantial modulation of $\gamma$ using a typical solid gate. Simultaneous demonstration of the gigantic MCA and its gate modulation in the superconducting state reveals that $T_{\rm d}$-MoTe$_2$ is a potential candidate for realizing electrically-tunable efficient superconducting rectification devices. 

We gratefully acknowledge M. Imai, S. Sasaki, H. Murofushi and S. Wang for their support in the experiments.
This project is financially supported in part by the JPSJ KAKENHI (Grant Number JP21H01022, JP21H04652, JP21K18181, JP21H05236, JP20H00354 and JP19H05790).



\appendix

%\section{Details of the sample fabrication and measurements}
% Put \label in argument of \section for cross-referencing
%\section{\label{}}


%The few-layer $T_{\rm d}$-MoTe$_2$ samples are fabricated via the mechanical exfoliation inside an Ar-filled glovebox containing concentrations of O$_2$ and H$_2$O below 0.5 ppm. Figure~\ref{figMoTe2} shows some examples of exfoliated flakes a few layers in thickness on a SiO$_2$(285 nm)/Si substrate. Independently from the samples, we prepare metallic electrodes by using typical electron-beam (EB) lithography and EB evaporation on a SiO$_2$/Si substrate (for the 4 L sample) or hexagonal boron-nitride (h-BN) exfoliated on a SiO$_2$/Si substrate (for the 2 L sample). We use Au (for the 4 L sample) or Pt (for the 2 L sample) with Ti as a buffer layer for the electrodes, and the total thickness of the electrodes is set to less than 10 nm to avoid exerting extra strain on the flake. The exfoliated $T_{\rm d}$-MoTe$_2$ is transferred with h-BN picked up by using poli-dimethylpolysiloxane (PDMS) covered with polycarbonate (PC) film. The h-BN flake on $T_{\rm d}$-MoTe$_2$ also acts as a protective layer because $T_{\rm d}$-MoTe$_2$ easily deteriorates when it is exposed to air. The transfer process is also performed inside the glovebox. After taking the sample out of the glovebox, it is immediately mounted on the sample holder of the $^3$He insert and encapsulated by the internal vacuum can then cooled down. Raman scattering and atomic-force microscope (AFM) measurements are carried out at room temperature and in the atmosphere after the transport measurements. 
%In the transport measurements for the 4 L sample for the $I \parallel b$ setup, the middle electrodes are employed as a current source and drain, and the neighboring contacts act as voltage probes. 
 
\section{Thickness identification via atomic force microscope (AFM)}

Here we show the AFM data on the thickness of the thin $T_{\rm d}$-MoTe$_2$ flakes. Figures~\ref{figMoTe2} show some examples of exfoliated flakes a few layers in thickness on a SiO$_2$(285 nm)/Si substrate. As shown in Fig.~\ref{figAFM}(a) and (b), the two samples employed in this study are 4 L and 2 L. Note that AFM measurements are performed after the transport measurements, thus $T_{\rm d}$-MoTe$_2$ flakes are encapsulated by hBN.

\begin{figure}[tb!]
\begin{center}
\includegraphics[width=8cm,clip]{Fig_MoTe2.eps}
\caption{Optical microscope images of thin $T_{\rm d}$-MoTe$_2$ flakes obtained via mechanical exfoliation.(a): 4 L (b): 2 L.}
\label{figMoTe2}
\end{center}
\end{figure}






\section{Determination of the crystal axes by Raman scattering}
\begin{figure}[b!]
\begin{center}
\includegraphics[width=8cm,clip]{FigAFM_2.eps}
\caption{Height profiles obtained from AFM scan of the 4 L (a) and 2 L (b) samples.}
\label{figAFM}
\end{center}
\end{figure} 
\begin{figure*}[tb!]
\begin{center}
\includegraphics[width=14cm,clip]{Fig_Raman5.eps}
\caption{(a) Angular dependence of the Raman intensity at the 163 cm$^{-1}$ Raman shift for the 4 L sample. (b) Optical image of the 4 L sample. The orientation of the sample corresponds to the orientation of the angle in (a). The scale bar is 5 $\mu$m. 
\textcolor{black}{(c) Angular dependence for the 2 L sample. (d) Sample picture of the 2 L sample, with 5 $\mu$m scale bar.}
}
\label{figRaman}
\end{center}
\end{figure*}

Since $T_{\rm d}$-MoTe$_2$ is a highly anisotropic material, it is important to determine the crystal axes of the sample and associate them with its transport properties. The polarization-angle dependence of the Raman intensity is a powerful tool for identifying the crystal axes \cite{Song2017, Zhou2017}. We perform Raman scattering measurements using HeNe laser light at 633 nm and measure the polarization-angle dependence of the Raman intensity at 163 cm$^{-1}$ Raman shift. Previous studies on angle-resolved Raman scattering measurements reported that the Raman intensity at 163 cm$^{-1}$ exhibits a maximum when the polarization is parallel to the Mo-zigzag chain ($b$-axis), and a minimum when it is perpendicular to it ($a$-axis). Figure~\ref{figRaman}(a) displays the angle dependence of the Raman intensity at 163 cm$^{-1}$ for the 4 L sample. Due to the reduced thickness of the sample in comparison with bulk, the otherwise typical two-lobe structure is deformed, but the in-plane anisotropy and the orientation of the axes are clearly discernible. The corresponding orientation of the 4 L sample is shown in Fig.~\ref{figRaman}(b). We can see that the electrodes are slightly misaligned from the direction of the principal crystal axes, which is the reason for a finite $R_{2 \omega}$ for $I \parallel a$ [$\equiv R_{2 \omega}(I \parallel a)$] shown in the main text. The misalignment angle is estimated to be around 15$^{\circ}$, leading to $\sin(-15^{\circ}) \sim -0.26$. This value is in close agreement with the ratio $R_{2 \omega}(I \parallel a)/R_{2 \omega}(I \parallel b) = -0.23$, taking into account that the peak (dip) is observed in the positive magnetic field for $R_{2 \omega}(I \parallel a)$ [$R_{2 \omega}(I \parallel b)]$.
\textcolor{black}{Figures~\ref{figRaman}(c) and (d) display the angular dependence of the Raman intensity and the sample picture for the 2 L sample. Misalignment between the electrodes and the crystal axes is larger, probably leading to a slight suppression of the nonreciprocal signals compared with those for the 4 L sample.}   

%\section{S4. Two-dimensional superconducting propeperties for few-layer $T_d$-MoTe$_2$}
%Since $T_{\rm d}$-MoTe$_2$ employed in our study is few-layer thick, superconductivity exhibits two-dimensional (2D) nature, which is drastically different from that for three-dimensional (3D) superconductors [S3, S4]. In 2D superconductors, thermal fluctuations become larger than 3D superconductors, leading to the formation of vortex-antivortex pairs. With increasing temperature, these vortex-antivortex pairs are unbounded at a certain temperature, and free vortices proliferate with temperature. Motion of these vortices induces dissipation even though the system is still in the superconducting state. This transition is called the Berezinskii-Kosterlitz-Thouless (BKT) transition, and the transition temperature is termed as the BKT transition temperature ($T_{\rm BKT}$). $T_{\rm BKT}$ can be estimated from the power-law dependence of the voltage on the current ($V = I^\alpha$). $T_{\rm BKT}$ is defined as the temperature at which $\alpha$ = 3 [S5, S6].
\begin{figure*}[htb!]
\begin{center}
\includegraphics[width=14cm,clip]{Fig_freq.eps}
\caption{(a) $R_{2 \omega}$ at different frequencies. No frequency dependencies are observed. (b) Driving current ($I_\omega$) dependence of $R_{2 \omega}$. $I_\omega$ = 100 nA provides the largest $R_{2 \omega}$. The data in (a) and (b) are both from the 4 L sample.}
\label{figfreq}
\end{center}
\end{figure*} 


%Figure S\ref{figBKT}(a) is a logarithmic plot of the $I-V$ relations taken from the 4 ML sample at different temperatures below the intrinsic superconducting transition temperature ($T_c$). By using the results in (a), we plot the exponent $\alpha$ as a function of temperature ($T$) in Fig. S\ref{figBKT}(b). $\alpha$=3 at $T$ = 420 mK is obtained from the interpolation of the experimental data. This temperature corresponds to the temperature at which the resistance becomes finite (see Fig. 2(a) in the main text), in agreement with the BKT transition temperature as explained above.

%\begin{figure*}[h!]
%\begin{center}
%\includegraphics[width=16cm,clip]{FigBKT.eps}
%\caption{(a) Logarithmic plot of the current-voltage ($I-V$) characteristics at different temperatures taken from the 4 ML sample. The dashed line shows an example of the fitting by assuming a power-law dependence of $V$ on $I$ ($V$ = $I^\alpha$) (b) Temperature dependence of the exponent $\alpha$. The horizontal dashed line denotes $\alpha$ = 3.}
%\label{figBKT}
%\end{center}
%\end{figure*} 


\section{Current and frequency dependence of the nonreciprocal resistance}
The main text shows $R_{2 \omega}$ as a function of $B$ with $I_\omega$ at 18 Hz. One may assume that the second harmonic signals are due to some spurious effect such as a capacitive coupling of the sample to the surrounding conductive environment. To rule out this possibility, we carry out $R_{2 \omega}$ measurements by driving $I_\omega$ at different frequencies. Figure~\ref{figfreq}(a) displays the results, demonstrating that $R_{2 \omega}$ signals are independent of the frequency of $I_\omega$. This is corroboration that $R_{2 \omega}$ signals we measure derive from the intrinsic transport properties of $T_{\rm d}$-MoTe$_2$, namely, magnetochiral anisotropy (MCA).

As mentioned in the main text, the efficiency of generating the nonreciprocal resistance is evaluated as $\gamma = 2 R_{2 \omega}/(R_\omega B I_\omega)$. Because of the definition of $\gamma$, one might naively think that $\gamma$ is divergent in the limit of $I_\omega \rightarrow 0$. However, this would be incorrect considering that the origin of the nonreciprocal transport is the ratchet-like motion of the magnetic vortices, because the vortices are not driven when the Lorenz force exerted by $I_\omega$ does not overcome the pinning force. On the other hand, superconductivity is suppressed if $I_\omega$ is too large. Therefore, it is expected that there is an intermediate value of $I_\omega$ which provides the largest signal of $R_{2 \omega}$. Indeed, as shown in Fig.~\ref{figfreq}(b), $R_{2 \omega}$ is suppressed as $I_\omega \rightarrow 0$, and there is an optimal value to obtain the largest $R_{2 \omega}$, which is 100 nA for the 4 L sample. $R_{2 \omega}$ diminishes when $I_\omega$ is larger than this value. Therefore, $I_\omega$ = 100 nA was used for the nonreciprocal transport measurements shown in the main text. Note that the optimum value of $I_\omega$ for the 2 L sample is 200 nA reflecting higher $T_c$. 




%\section{S6. Estimation of $U_0$ from $R$-$T$ curves}
%\begin{figure*}[tb!]
%\begin{center}
%\includegraphics[width=17.5cm,clip]{Fig_U0.eps}
%\caption{}
%\label{figU0}
%\end{center}
%\end{figure*} 
\section{MCA for the 2 L ${\rm MoTe}_2$ sample and comparison of $\gamma$ as a function of $B$}
Whereas the main text we principally shows the data from the 4 L sample, similar nonreciprocal transport results are also obtained for the 2 L sample. Figure~\ref{figbilayer}(a) shows $R_{2 \omega}$ curves taken at different temperatures from the 2 L sample. Increasing $R_{2 \omega}$ with decreasing temperature is visible explicitly, while the value of $B_{\rm peak}$ is different from that of the 4 L sample because of the higher $T_c$. 
%The temperature dependence of $\gamma$ is also shown in Fig. S\ref{figbilayer}(b). The maximum $\gamma$ at the lowest temperature is slightly smaller than for the 4 ML sample, but still on the order of 10$^5$ T$^{-1}$ A$^{-1}$ and the plot exhibits a similar trend in the temperature dependence. 

%The maximum $\gamma$ at the lowest temperature is slightly smaller than for the 4 L sample, but still on the order of 10$^5$ T$^{-1}$ A$^{-1}$ as shown in the main text. Naively, one might expect that the rectification effect is enhanced with increasing $T_c$. The experimental results are thus counterintuitive, and a smaller $R_{2 \omega}$ with increasing $T_c$ is consistent with the gate dependence of $R_{2 \omega}$ shown in the main text. 


%In our study, $T_c$ is increased by decreasing the thickness of $T_{\rm d}$-MoTe$_2$. The quality of the film declines with decreasing thickness, and it may provide additional pinning traps, such as dislocations, which do not contribute to the ratchet-like motion. Other possibilities are also considerable, and further theoretical studies are required to understand the relation between $T_c$ and $R_{2 \omega}$. 

%\begin{figure*}[h!]
%\begin{center}
%\includegraphics[width=10cm,clip]%{Fig_gamma_vs_B_2L_4L.eps}
%\caption{\textcolor{red}{Comparison of the relation between $\gamma$ and $B$ between the 2 L and 4 L sample} . %(b) Rectification coefficient $\gamma$ as a function of temperature ($T$). Increasing $\gamma$ with decreasing $T$ indicates that the ratchet-like motion of the magnetic vortices is an origin of the nonreciprocal transport.
%}
%\label{figgamma}
%\end{center}
%\end{figure*} 
%However, decreasing the thickness of a superconductor enhances the two dimensionality of the system, which induces larger fluctuations of superconducting order parameter. Such fluctuations can decline the ratchet effect.
\textcolor{black}{In the main text we discuss the value of $\gamma$ using the values of $R_{2 \omega}$ and $R_\omega$ at $B_{\rm peak}$. With this estimate, $\gamma$ for the 2 L sample is slightly smaller than the 4 L sample. By contrast, we can also plot $\gamma$ as a function of $B$ as we show in the bottom panel of Fig. 2(a). Figure~\ref{figbilayer}(b) displays the evolution of $\gamma$ with $B$ from the 2 L and 4 L samples. It is evident that at any $B$, $\gamma$ from the 2 L sample is larger than that from the 4 L sample. This indicates that the smaller $\gamma$ for the 2 L sample is mainly due to the larger $B_{\rm peak}$.}

%\section{S6. Comparison of $\gamma$ with those reported in other noncentrosymmetric superconductors}

%\begin{table*}[h!]
%\caption{\label{tab:table3}Summary of $\gamma$, $\gamma W$ and $\gamma B$ for different superconductors}
%\begin{ruledtabular}
%\begin{tabular}{ccccccccc}
 %&\multicolumn{2}{c}{$D_{4h}^1$}&\multicolumn{2}{c}{$D_{4h}^5$}\\
 %Material&Symmetry&$B_{\rm peak}$ [T] & $R_{2 \omega} [\Omega]$ & $W$ [$\mu$m]&$\gamma$ 
 %[A$^{-1}$T$^{-1}$]&$\gamma W$ [A$^{-1}$T$^{-1}$m] &$\gamma B$ [A$^{-1}$] & ref.\\ \hline
 %MoTe$_2$ 2 L & $C_{2v}$ & 0.35 & 1.8 & 3 & 6.7 $\times 10^5$ & 2.00 & 2.34 $\times 10^5$ & This study \\
 %MoTe$_2$ 4 L & $C_{2v}$ & 0.050 & 1.3 & 6 & 3.1 $\times 10^6$ & 18.5 & 1.54 $\times 10^5$ & This study \\
 %NbSe$_2$& $D_{6h}$ & 3.0 & 0.062 & 5 & 2.8$\times10^2$ & 1.41$\times10^{-3}$ & 8.49$\times10^2$ & [S6] \\
 %MoS$_2$& $C_{3v}$ & 0.50 & 0.65 & 3 & 4.6$\times10^3$ & 1.39$\times10^{-3}$ & 2.32$\times10^3$ & [S7] \\
 %SrTiO$_3$& - & 0.050 & 0.29 & 80 & 3.2 $\times 10^6$ & 2.56$\times10^2$ & 1.60 $\times 10^5$ & [S8] \\
%\end{tabular}
%\end{ruledtabular}
%\end{table*}

%As discussed in [S3], the efficiency for generating a nonreciprocal signal is evaluated by $\gamma = 2 R_{2 \omega}/(R_\omega B I_\omega)$. Since this definition includes $B$, the magnetic field at which $R_{2 \omega}$ take a peak, and an excitation current $I$, one may wonder that we can compare $\gamma$ between different systems because the values of $B$ and $I$ are different. In addition, it is also important to account for the sample geometry, especially the width of the sample, considering that the current density is more essential rather than the current itself (here we mainly consider two-dimensional systems). Taking these issues into account, we show in Table~S\ref{tab:table3} $B_{\rm peak}$, $R_{2 \omega}$, the width of the sample ($W$), $\gamma$, $\gamma W$ and $\gamma B$ for each system. $\gamma W$ corresponds to replacing $I$ with $j$, the current density, as discussed in [S4]. We can find that while the values of $\gamma$ are comparable between our $T_{\rm d}$-MoTe$_2$ samples and SrTiO$_3$, because of the larger sample width (80 $\mu$m), $\gamma W$ for SrTiO$_3$ is much larger than for MoTe$_2$. By contrast, $T_{\rm d}$-MoTe$_2$ exhibits 3-4 orders of magnitude larger $\gamma W$ compared with other van der Waals superconductors (NbSe$_2$ and MoS$_2$).

%In the case of MCA induced by the ratchet motion of magnetic vortices, it was also proposed to use $\gamma B$ for evaluating the efficiency to generate nonreciprocal signals [S5]. Our $T_{\rm d}$-MoTe$_2$ samples show comparable values of $\gamma B$ in comparison to SrTiO$_3$, and 2-3 orders of magnitude larger than NbSe$_2$ and MoS$_2$ (note that although the mechanism behind the nonreciprocal signals is not the ratchet motion of magnetic vortices for SrTiO$_3$, we estimated $\gamma B$ for reference). Interestingly, when we consider $B$, the resulting coefficient ($\gamma B$) for the 2 L sample is larger than that for the 4 L sample, contrary to the relative amplitudes of $\gamma$. 
%The evaluation based on $\gamma W$ or $\gamma B$ indicates that the efficiency for $T_{\rm d}$-MoTe$_2$ to obtain large nonreciprocal signals is comparable to SrTiO$_3$, and it is much larger than that for other van der Waals superconductors such as MoS$_2$ and NbSe$_2$. 

%Note that uncertainty in the current distribution in a sample (e.g. the 4 L sample), even if it exists, does not affect our conclusion. Suppose there exists a current inhomogeneity inside the sample because of the complex electrode geometry, the net width of the sample will reduce. Then, $\gamma W$ will also reduce by several factors, but it is still much larger than that for MoS$_2$ and NbSe$_2$. 

\section{Vortex phase diagram and estimation of $U_0$}
Since giant MCA observed in $T_{\rm d}$-MoTe$_2$ is likely attributed to the ratchet-like motion of magnetic vortices, we show the vortex phase diagram in Fig.~\ref{vortex_phase}(a) to identify the vortex state at each temperature and magnetic field where we observe a large nonreciprocal signal. At lower temperatures and lower magnetic fields, vortices are in the "quantum metal" phase, in which a finite resistance remains even much below $T_c$ under a magnetic field \cite{Saito2015}. Vortices are in the thermal creep (or thermally assisted flux flow) regime at higher temperatures and higher magnetic fields up to $T_c$ and the upper critical magnetic field ($H_{c2}$), where vortices are mobile and can plastically flow under an excitation current. Ratchet-like motion of magnetic vortices is effective in the thermal creep regime. The experimental data points which separates the quantum metal phase and thermal creep phase are obtained following \cite{Saito2015}. Arrhenius plots of the temperature dependence of the resistance are prepared under a magnetic field, and the temperature at which the plot deviates from the exponential decay is defined as the boundary temperature [see Fig. \ref{vortex_phase}(b)]. Repeating this procedure for different magnetic fields provides sets of data ($T$,~$B$) for the boundary between the thermal creep and the quantum metal phase, as plotted in Fig.~\ref{vortex_phase}(a). The cross marks in Fig.~\ref{vortex_phase}(a) compose sets of temperature and magnetic field condition at which the peaks in $R_{2\omega}$ are observed. Most of the points are contained in the thermal creep region, except for the two points at lower temperature, giving another evidence that vortex dynamics affected by the ratchet-like potential plays a principle role for giant nonreciprocal signals. In the phase diagram it is visible that the quantum metal phase extends to relatively larger magnetic fields at lower temperatures. The remaining two points overlap this region, indicating that quantum effects may affect the nonreciprocal signals. This explains the suppression of $\gamma$ obtained from experiments in comparison to the theoretical fit based on the ratchet-like motion of magnetic vortices as we discuss in the main text.  



\textcolor{black}{Temperature ($T$) dependence of the resistance under a magnetic field also enables to estimate the potential height $U_0$. In the thermal creep regime, the resistance $R$ depends on the activation energy $U(B)$ as $R = R_0 \exp[-U(B)/k_B T]$ with the normal state resistance $R_0$ \cite{Saito2015}. Thus the slope in a logarithmic plot of $R$ with $1/T$ provides $U(B)$ [see also Fig.~\ref{vortex_phase}(b)]. Measuring $T$-dependence of $R$ under different $B$ provides the relation between $U(B)$ and $B$, from which we can obtain $U_0$ = 0.10(0.56) meV by using the relation $U(B) = U_0 \ln(B_0/B)$ for the 4 L (2 L) samples [Fig.~\ref{vortex_phase}(c)] \cite{Saito2015}.}

\begin{figure*}[tb!]
\begin{center}
\includegraphics[width=14cm,clip]{Fig_bilayer3.eps}
\caption{(a) MCA signals obtained from the 2 L sample at different temperatures. The characteristics are similar to those for the 4 L sample. (b) Comparison of the relation between $\gamma$ and $B$ at 230 mK between the 2 L (at $V_g$ = 0) and 4 L sample.%(b) Rectification coefficient $\gamma$ as a function of temperature ($T$). Increasing $\gamma$ with decreasing $T$ indicates that the ratchet-like motion of the magnetic vortices is an origin of the nonreciprocal transport.
}
\label{figbilayer}
\end{center}
\end{figure*} 

\textcolor{black}{\section{Axes dependence of the nonreciprocal signal}}
\textcolor{black}{In the main text we show the experimental results of MCA from the 2 L and 4 L samples. While these samples provide a sufficient amount of data sets, the crystal axes are not perfectly aligned parallel to the current direction, making it difficult to identify the axes dependence of the nonreciprocal signals. Here, we provide additional data from other samples whose crystal axes are almost parallel to the current direction. Figures~\ref{axes_dep}(a)-(c) are from another 4 L sample (4 L$\#$2) and the current direction is parallel to the $b$-axis ($I \parallel b$). The crystal axes are determined from the Raman intensity profile shown in Fig.~\ref{axes_dep}(b). As seen in Fig.~\ref{axes_dep}(c), clear peak and dip structures are observed. Figures~\ref{axes_dep}(d)-(f) display the second harmonic signal from the 6 L sample where $I \parallel a$. We can easily find in Fig.~\ref{axes_dep}(f) that while slowly oscillating background is visible, no peak and dip structures typical for MCA are observed. This also corroborates that peak and dip structures that we observe in the other samples arise from MCA.}

\section{Other possible effects to generate nonreciprocal signals}
While we have shown a number of additional experimental data which support MCA as the origin of the giant nonreciprocal signals, some may still wonder other effects can be considered to explain those nonreciprocal resistances. Let us rule out some other possibilities as an origin of the second harmonic signals.

\subsection{Thermal effects}
If there existed a thermal gradient $\bm{\nabla} T$, the Nernst effect would generate an electric field $\mathbf{E} \propto \mathbf{B} \times \bm{\nabla} T$ under a magnetic field $B$. The thermal gradient may be due to the Joule heating effect, therefore $\bm{\nabla} T \propto \mathbf{j}^2$, where $\mathbf{j}$ is current density, and $\bm{\nabla} T$ should be parallel to $\mathbf{j}$. Indeed, such a superconducting Nernst effect which generates a second harmonic voltage ($V_{2 \omega}$) was observed in another van der Waals superconductor NbSe$_2$, using a thermal gradient driven by a heater mounted close to the sample \cite{Li2020}. 
$V_{2 \omega}$ exhibits similar magnetic field dependence as those observed in our samples, derived from the (vortex) Nernst effect \cite{Behnia2016}. We can rule out the Nernst effect as a possible origin of MCA due to the following reasons: i) Temperature gradient assumed above should not exist considering that our $T_{\rm d}$-MoTe$_2$ is highly crystalline so that excitation currents pass almost homogeneously through the sample. ii) $V_{2 \omega}$ from the Nernst effect is reduced to zero at lower temperatures, opposite to our observations. Furthermore, the Nernst signal persists even above $T_c$, in contrast to the suppressed nonreciprocal signals above $T_c$. iii) $V_{2 \omega}$ induced by the Joule heating should be orthogonal to $\mathbf{j}$, inconsistent with our observations of large longitudinal nonreciprocal signals. 
\begin{figure*}[tb!]
\begin{center}
\includegraphics[width=14cm,clip]{vortex_phase5.eps}
\caption{(a) Vortex phase diagram obtained from the 4 L sample. Thermal creep (thermally activated flux flow) phase transits into the quantum metal phase at lower temperatures. Blue squares are determined from the upper critical field $B_{c2}$, and red triangles are obtained from the temperature dependence of the resistance under perpendicular magnetic field. Orange cross marks express the points at which $R_{2 \omega}$ takes a peak. All the cross marks apart from two of them at lower temperatures are inside the thermal creep phase, supporting the vortex dynamics plays a central role for giant nonreciprocal signals. (b) Arrhenius plot of the temperature dependence of the resistance. The boundary point between the quantum metal and thermal creep phase is determined as a point (the yellow point) at which the Arrhenius plot deviates from the exponential decay (black solid line). This yellow point corresponds to the yellow point in (a). (c) The slope shown by the solid line in (b) provides the activation energy $U(B)$ at different magnetic fields. From the slope of $U(B)/k_B$ as a function of $B$ (orange solid line), we can obtain $U_0$.} 
\label{vortex_phase}
\end{center}
\end{figure*} 
\begin{figure*}[tb!]
\begin{center}
\includegraphics[width=14cm,clip]{Fig_axes_dep.eps}
\caption{\textcolor{black}{(a) Optical microscope image of the 4 L$\#$2 device. The directions of the crystal axes and excitation current are denoted. (b) Angular dependence of the Raman scattering intensity. The sample is set as shown in (a), thus 0 degree corresponds to the $b$-axis of the sample. (c) $R_{2 \omega}$ signal as a function of perpendicular magnetic field $B$ for $I \parallel b$. Large peak and dip structures are clearly visible. Inset shows the liner resistance $R_\omega$ simultaneously measured with $R_{2 \omega}$. (d) Optical microscope image of another sample with 6 L in thickness. Here the direction of a current $I$ is parallel to the $a$-axis, determined from the angular dependence of the Raman scattering intensity shown in (e). (f) $R_{2 \omega}$ for $I \parallel a$. There is a slowly oscillating background, but no peak and dip are observed. $R_\omega$ signal is displayed in the inset. Note that because of the problems in the electrodes, we cannot perform similar measurements driving $I$ parallel to the $b$-axis in this sample. In (a) and (d) $T_{\rm d}$-MoTe$_2$ is highlighted by a yellow dotted line, and the scale bar corresponds to 5 $\mu$m.}}
\label{axes_dep}
\end{center}
\end{figure*} 

\subsection{Geometrical effects}
Vortex rectification effect induced by the asymmetric sample geometry (e.g. asymmetric edge shape of the sample) was previously proposed \cite{Vodolazov2005} and experimentally confirmed \cite{Cerbu2013, Ji2016}. The main idea is that since magnetic vortices always enter from the edge (or surface in the case of three dimensional superconductors) of the sample and cannot nucleate inside the superconductor, the edge asymmetry between the opposite sides of the sample generates the asymmetric surface potential barrier for vortices. Because the edge selected for the vortex entry depends on the polarity of the current, the inequivalency in the potential barrier between the edges leads to the different current condition for the vortex entry thus rectification. One may wonder that the nonreciprocal signals observed in this study are attributed to the asymmetry between the edges of the sample. It is true that slight asymmetry between the edges is inevitable in our samples employing mechanically exfoliated $T_{\rm d}$-MoTe$_2$ flakes. If the edge asymmetry is a dominant contribution in our samples, larger nonreciprocal signals are expected for samples with more asymmetric edges. We do not see the correlation between the edge asymmetry and the amplitude of the nonreciprocal signal for different samples with different edge shapes. As an example, the edge asymmetry is more peculiar in the sample shown in Fig.~\ref{axes_dep}(d) than (a), but the nonreciprocal signal is much suppressed. Moreover, in our samples electrodes are always aligned symmetrically, ruling out the possibility of the asymmetry induced by electrodes. Therefore, we can draw a conclusion that vortex rectification due to the geometrical asymmetry in the sample is not an origin for the giant nonreciprocal signals.



\textcolor{black}{
\subsection{Surface barrier effect}}

\textcolor{black}{In relation to the geometrical effect, one may argue that the asymmetry in the surface barrier for magnetic vortices is the origin of MCA. As for the dynamics of magnetic vortices in superconductors, we must consider two effects: Bulk pinning and surface barrier. The important point is which effect is dominant in a certain condition. We highlight that the surface barrier effect plays a major role in the vortex dynamics only when the temperature is close to $T_c$, at which the bulk pinning becomes extremely weak. This important point has been already demonstrated by many previous studies \cite{SB1, SB3, SB4, Zeldov}. This behavior is inconsistent with our observation, indicating that bulk pinning effect is dominant in our samples.}

%\subsection{4. Inhomogeneous current effect}

%The current and voltage probe configuration for the 4 L and 2 L samples is shown in Figs.~S\ref{current_dist}. Specifically for the configuration for $I \parallel a$ in the 4 L sample, one may wonder that large $\gamma$ is obtained due to the inhomogeneous current flow inside the sample. However, because $\gamma$ is estimated from the current and not from the current density, the value of $\gamma$ is not affected by the distribution of the current inside the sample. Note that due to the much larger resistivity of the metallic layer in the electrodes than that of $T_{\rm d}$-MoTe$_2$, parallel conduction is negligible in the setup in Fig.~S\ref{current_dist}(b).

%To confirm the effect of the current distribution, we consider $\gamma W$ already shown in Table SI. In the case of an inhomogeneous current distribution, the net transversal width $W$ is expected to be reduced. Suppose that the actual $W$ ($\equiv W_{\rm net}$) is half of the sample width $W$. In this case, $\gamma W$ becomes half the value shown in Table SI. Even though, it is clear that $\gamma W_{\rm net}$ is orders of magnitude larger than those in other van der Waals materials such as NbSe$_2$ and MoS$_2$. 

\section{Theoretical description using ratchet model}

\begin{figure}[b!]
\begin{center}
\includegraphics[width=8cm,clip]{current_dist.eps}
\caption{(a) The current and voltage probes configuration for $I \parallel a$ in the 4 L sample. (b) The current and voltage probes configuration for $I \parallel b$ in the 4 L sample. (c) The current and voltage probes configuration in the 2 L sample. }
\label{current_dist}
\end{center}
\end{figure} 

Here we describe the nonreciprocal transport signal originating from the vortex motion \cite{Hoshino, Itahashi2} .
For simplicity, we regard the vortex as a point particle.
The relation between velocity and force is written as
\begin{align}
v_{x} &= q_{1xx} F_x + q_{2xxy} F_x F_y 
\\
v_{y} &= q_{1yy} F_y + q_{2yxx} F_x^2 + q_{2yyy} F_y^2
\end{align}
where the mirror symmetry along the $y$ axis is assumed (for a trigonal symmetry, there are the relations $q_{1xx}=q_{1yy}$ and $q_{2xxy}=-2q_{2yxx}=2q_{2yyy}$).
If we take the configuration with $F_x=0$, only the $y$ direction is involved.
Hence we consider the one-dimensional equation of motion for the estimation of $q_{2yyy}$:
\begin{align}
    \eta \dot y = F_y -\frac{\partial \mathcal U(y)}{\partial y} + \xi(t)
\end{align}
where $ \displaystyle \eta = \frac{1.45 \pi \hbar^2\sigma_{\rm n}}{2e^2\xi^2}$ is a damping coefficient 
% \cite{Bardeen,Tinkham} 
with the normal conductivity $\sigma_{\rm n}$ and in-plane coherence length $\xi$ \cite{Kopnin}.
The random force $\xi (t)$ represents a thermal noise satisfying $\la \xi (t) \xi(t') \ra = 2\eta k_{\rm B}T\delta(t-t')$ where the bracket indicates the random average.
The asymmetric potential $\mathcal U(y)$ for vortex is responsible for the non-reciprocal transport signal.
We take the periodic potential of the height $U$ and the periodicity $\ell$ given in Fig.~\ref{figTheor}(a).
% As shown in the figure, the parameter $f$ controls the asymmetry.
% $   \mathcal U(y) = Uy/\ell$ ($0<y<\ell$) which is strongly asymmetric.
The response coefficients are given by $\displaystyle q_{1yy}=\frac{1}{\eta} g_1(\beta U)$ and $\displaystyle q_{2yyy} = \frac{\beta \ell}{\eta} g_2(\beta U)$.
The functional forms of $g_{1,2}$ are explicitly given by
\begin{align}
&g_1(x) = \frac{x^2}{2(\cosh x-1)},\\
&g_2(x) = \frac{f \, x\, (4+x^2-4\cosh x+x\sinh x)}{2(\cosh x-1)^2}.
\end{align}
The parameter $f$ ($\leq \tfrac 1 2$) controls the asymmetry of the potential.



The force acting on the vortex is given by $F_y=j \phi_0^*$ where $j$ is a current density along $x$ direction and $\displaystyle \phi_0^* = \frac{h}{2|e|}$ is the flux quantum for superconductors.
The number density of vortices is given by $\displaystyle n=\frac{B}{\phi_0^*}$.
The voltage along $x$ direction is then given by the Josephson relation
$V_x = \phi_0^* L n v_y = R_1I + R_2I$ where $I=jW$ is a current with the sample width $W$ and length $L$.
The linear and non-linear transport coefficients are 
\begin{align}
R_1 &= \frac{\phi_0^* L B}{\eta W} g_1(\beta U)
,\ \ \ 
R_2 = \frac{(\phi_0^*)^2 ILB\ell}{\eta k_{\rm B}TW^2} g_2(\beta U)
.
\end{align}
Since both the signals are proportional to the vortex number density, their ratio is written in a simple form 
\begin{align}
\gm' &= \frac{R_{2}}{R_{1} I} 
= \frac{\phi_0^* \ell}{Wk_{\rm B}T} 
\cdot \frac{g_2(\beta U)}{g_1(\beta U)}
\end{align}
which is determined from the profile of the potential of vortices.
Note also the relations $R_{\omega}=R_1$ and $R_{2\omega}=R_2/2$.


%Now we estimate the order of the magnitude of the signal.
Now we discuss characteristic parameters used in this model.
For the characteristic length $\ell$, we consider the value of the magnetic field $B_{\rm pin}$ at which all the pinning centers are occupied at low temperature.
The length scale is then given by $\ell(T=0) \sim \sqrt{\phi_0^*/B_{\rm pin}}$.
The potential height $U$ is estimated by the critical current density $j_{\rm c}$ at small magnetic field by the relation $U=j_c \phi_0^*\ell (\tfrac 1 2 + f)$.
We also consider the temperature dependence of these parameters since the size of vortices changes as the coherence length varies with increasing temperature.
We assume the temperature dependence of $\displaystyle U(T)\sim U_0 \Big( \frac{T_c-T}{T_c} \Big)^{\al}$ and $\displaystyle 
\ell(T) \sim \ell_0 \Big( \frac{T_c-T}{T_c} \Big)^{\al-\frac 1 2}$, which results in $j_c \propto \sqrt{T_c-T}$ ($\alpha=1$ is used in \cite{Hamamoto}).
Since the microscopic origin of the vortex potential is not clear at present, 
% Since the profiles of pinning should be strongly dependent on each system, 
here we take $\alpha$ as a parameter phenomenologically for a better fit to the experimental data.
The temperature dependence of $U$ is more strongly reflected in the signal compared to that of $\ell$, because $\beta U$ is the argument of the non-linear function $g_{2}$.
% Assuming the same temperature dependence used in Ref.~[S4], we write $U(T)\sim U_0 \frac{T_c-T}{T_c}$ and $\ell(T) \sim \ell_0 \sqrt{\frac{T_c-T}{T_c}}$.
% For comparison, we further consider the model with constant $U(T)=U_0$ and $\ell(T)=\ell_0$ (model B).
 
% [We may compare this with the temperature-independent model if necessary. The result is also plotted in the figure as "No T-dep" in Fig. S7.]
\begin{figure*}[tb!]
\begin{center}
\includegraphics[width=14cm,clip]{figure.eps}
\caption{
(a) Spatial dependence of the ratchet potential, where $f$ ($0\leq f\leq 1/2$) controls the asymmetry.
(b) Temperature dependence of (b1) $\gm'=\gm B$, and (b2) $R_{2\omega}$ for the 4 L sample.
Analogous plots for the 2 L sample are shown in (c1,c2).
% [TO BE DELETED?] Original $R_1,R_2$ data for (a,b) the model A and (c,d) model B. The dotted lines indicate asymptotic forms near the transition temperature where the vortices are free from pinning.
% Temperature-dependent model is nicer in the sense that R2 increases with decreasing temperature. The region with $T\lesssim 0.2$ shows suppression of the signals due to the vortex pinning.
}
\label{figTheor}
\end{center}
\end{figure*} 

Now we estimate the magnitude of the non-linear transport signal.
Since the nonreciprocal signals in magnetic field dependence is maximized in a moving vortex regime, we assume the expression for a small pinning potential
% where the vortex is free from pinning: 
$g_1(\beta U) \sim 1$ and $\displaystyle g_2(\beta U) \sim f \frac{ (\beta U)^3}{180}$, with which the vortices are not fixed in the pinning potential.
% (In the pinned regime at low temperatures, the resistance becomes exponentially small).
We use the parameters for the 4 L (2 L) sample such as
$W=5\ (2.5)\ \mu{\rm m}$, $T_c=0.75\ (2.2)$ K, $B_{\rm pin}(T=0) \sim 0.025\ (0.2)$ T, and $I_c(T=0)=j_cW\sim 200\ (500)$ nA. 
$B_{\rm pin}$ is estimated from the magnetic-field dependence of the resistance, and $I_c$ from the current-voltage characteristics under a magnetic field measured for each sample. 
% Since we have used the strongly asymmetric potential $\mathcal U(y)$ defined above, this result gives a maximum of the signal, and it may be reduced by changing its shape. 
In order to be compatible with experiments, we choose the vortex potential parameters as $f=0.15$ and $\alpha = 0.5$, which are used for both the 4 L and 2 L samples.
With these parameters, the pinning potential height at zero temperature is estimated as $U_0 \sim 0.1\ (0.17)$ meV. Note that the estimated values of $U_0$ are close to those experimentally obtained from the temperature dependence of the resistance under a perpendicular magnetic field \textcolor{black}{(see Appendix E)}.  
Since $\gm'$ is proportional to $(T_c-T)^{4\al-\frac 1 2}$ in our model, the temperature dependence becomes more convex downward if we take larger $\al$.

The temperature dependence of $\gm' = \gamma B$ is shown in Fig.~\ref{figTheor} (b1) for the 4 L sample and in (c1) for the 2 L sample.
The vortex ratchet model with two adjustable parameters $\alpha$ and $f$ reproduces the magnitude of signals observed in experiments.
We note that the present model can be justified in the middle temperature range.
For high temperature range $T\gtrsim T_{c}$, the vortex picture should be replaced by superconducting fluctuation mechanism and/or normal contribution.
At low temperatures, on the other hand, a quantum effect on ratchet-like motion \cite{Hamamoto} is needed for more accurate estimate, as supported by the vortex phase diagram discussed above.
% multiply the scaling factor, which corresponds to a more symmetric potential.



% We have considered both the models with temperature-dependent $U,\ell$ (model A) and with $T$-independent $U_0,\ell_0$ (model B).
% Both the model can produce the magnitude of $\gm'$ observed in experiment.


With the above setup, we also estimate the values of $R_{2\omega}$ by using
% the sample length $L=4.5\mu$m 
the normal resistance $R_{\rm n} = \frac{L}{W}\sg_{\rm n} \sim 330\ (380)\ \Omega$, the coherence length $\xi(T) \sim \xi_0 \sqrt{T_c/(T_c-T)}$ with $\xi_0\sim 45\ (24)$ nm, and the current $I=100\ (200)$ nA.
The results are shown in Fig.~\ref{figTheor}(b2) for the 4 L sample and (c2) for the 2 L sample.
Since the experimental data plotted here are obtained using different magnetic fields ($0.04$--$0.05$ T for 4 L and $0.25$--$0.42$ T for 2 L), we show several lines for different magnetic fields.
% (Note that $\gm'$ is $B$-independent).
We confirm that the present model can roughly reproduce the magnitude $R_{2\omega}$ in a middle temperature range below $T_c$.
% It is also notable that 
% $R_{1\omega}$ is increasing function of temperature while
$R_{2\omega}$ is decreasing function of temperature, which
% This tendency 
is also consistent with experiments.

% With the scaling factor $f=0.1$ multiplied, 
% The value of $R_2$ at $I=200$nA and $B=0.3$T is the order of 1$\Omega$ for the both models A and B, as in the experimental results.
% On the other hand, the temperature dependence of $R_2$ is different: $R_2$ increases with decreasing $T$ for the model A in the relatively high temperature regime, while $R_2$ decreases for the model B.
% In this respect, the model A with $T$-dependent parameters $U,\ell$ is consistent with experimental results.



% \begin{figure*}[h!]
% \begin{center}
% \includegraphics[width=12cm,clip]{R2R1T.pdf}
% \caption{Temperature dependence of $\gm'$ based on the ratchet model with the factor $f=0.1$ multiplied. The two theoretical curves are calculated based on
%  $T$-dependent (model A) or $T$-independent (model B) pinning potential $U$ and length $\ell$.}
% \label{figTheor}
% \end{center}
% \end{figure*} 


%\textcolor{red}{\section{S11. Magnetic field dependence of $\gamma$: Comparison between the 2 L and 4 L sample}}






































%\begin{thebibliography}{99}

%\end{thebibliography}



%\subsection{}
%\subsubsection{}
%\nocite{*}

%\bibliography{sm}

% If in two-column mode, this environment will change to single-column
% format so that long equations can be displayed. Use
% sparingly.
%\begin{widetext}
% put long equation here
%\end{widetext}

% figures should be put into the text as floats.
% Use the graphics or graphicx packages (distributed with LaTeX2e)
% and the \includegraphics macro defined in those packages.
% See the LaTeX Graphics Companion by Michel Goosens, Sebastian Rahtz,
% and Frank Mittelbach for instance.
%
% Here is an example of the general form of a figure:
% Fill in the caption in the braces of the \caption{} command. Put the label
% that you will use with \ref{} command in the braces of the \label{} command.
% Use the figure* environment if the figure should span across the
% entire page. There is no need to do explicit centering.

% \begin{figure}
% \includegraphics{}%
% \caption{\label{}}
% \end{figure}

% Surround figure environment with turnpage environment for landscape
% figure
% \begin{turnpage}
%\begin{figure}
%\includegraphics{FigBKT.eps}%
%\caption{\label{}}
%\end{figure}
%\end{turnpage}

% tables should appear as floats within the text
%
% Here is an example of the general form of a table:
% Fill in the caption in the braces of the \caption{} command. Put the label
% that you will use with \ref{} command in the braces of the \label{} command.
% Insert the column specifiers (l, r, c, d, etc.) in the empty braces of the
% \begin{tabular}{} command.
% The ruledtabular enviroment adds doubled rules to table and sets a
% reasonable default table settings.
% Use the table* environment to get a full-width table in two-column
% Add \usepackage{longtable} and the longtable (or longtable*}
% environment for nicely formatted long tables. Or use the the [H]
% placement option to break a long table (with less control than 
% in longtable).
% \begin{table}%[H] add [H] placement to break table across pages
% \caption{\label{}}
% \begin{ruledtabular}
% \begin{tabular}{}
% Lines of table here ending with \\
% \end{tabular}
% \end{ruledtabular}
% \end{table}

% Surround table environment with turnpage environment for landscape
% table
% \begin{turnpage}
% \begin{table}
% \caption{\label{}}
% \begin{ruledtabular}
% \begin{tabular}{}
% \end{tabular}
% \end{ruledtabular}
% \end{table}
% \end{turnpage}

% Specify following sections are appendices. Use \appendix* if there
% only one appendix.
%\appendix
%\section{}

% If you have acknowledgments, this puts in the proper section head.
%\begin{acknowledgments}
% put your acknowledgments here.
%\end{acknowledgments}

% Create the reference section using BibTeX:
%\bibliography{basename of .bib file}


\bibliography{prr}% Produces the bibliography via BibTeX.

\end{document}
%
% ****** End of file apssamp.tex ******
