%%
%% This is file `sample-sigconf.tex',
%% generated with the docstrip utility.
%%
%% The original source files were:
%%
%% samples.dtx  (with options: `sigconf')
%% 
%% IMPORTANT NOTICE:
%% 
%% For the copyright see the source file.
%% 
%% Any modified versions of this file must be renamed
%% with new filenames distinct from sample-sigconf.tex.
%% 
%% For distribution of the original source see the terms
%% for copying and modification in the file samples.dtx.
%% 
%% This generated file may be distributed as long as the
%% original source files, as listed above, are part of the
%% same distribution. (The sources need not necessarily be
%% in the same archive or directory.)
%%
%% Commands for TeXCount
%TC:macro \cite [option:text,text]
%TC:macro \citep [option:text,text]
%TC:macro \citet [option:text,text]
%TC:envir table 0 1
%TC:envir table* 0 1
%TC:envir tabular [ignore] word
%TC:envir displaymath 0 word
%TC:envir math 0 word
%TC:envir comment 0 0
%%
%%
%% The first command in your LaTeX source must be the \documentclass command.
\documentclass[sigconf]{acmart}
\usepackage{enumitem}
\usepackage[ruled,linesnumbered]{algorithm2e}
\usepackage{mathrsfs}
\usepackage{makecell}
\usepackage{subcaption}
\usepackage{tablefootnote}
\usepackage{booktabs,tabularx}
% \usepackage{graphicx}
% \usepackage{newclude}

%% NOTE that a single column version may be required for 
%% submission and peer review. This can be done by changing
%% the \doucmentclass[...]{acmart} in this template to 
%% \documentclass[manuscript,screen]{acmart}
%% 
%% To ensure 100% compatibility, please check the white list of
%% approved LaTeX packages to be used with the Master Article Template at
%% https://www.acm.org/publications/taps/whitelist-of-latex-packages 
%% before creating your document. The white list page provides 
%% information on how to submit additional LaTeX packages for 
%% review and adoption.
%% Fonts used in the template cannot be substituted; margin 
%% adjustments are not allowed.
%%
%%
%% \BibTeX command to typeset BibTeX logo in the docs
\AtBeginDocument{%
  \providecommand\BibTeX{{%
    \normalfont B\kern-0.5em{\scshape i\kern-0.25em b}\kern-0.8em\TeX}}}

%% Rights management information.  This information is sent to you
%% when you complete the rights form.  These commands have SAMPLE
%% values in them; it is your responsibility as an author to replace
%% the commands and values with those provided to you when you
%% complete the rights form.
\setcopyright{acmcopyright}
\copyrightyear{2023}
\acmYear{2023}
\acmDOI{XXXXXXX.XXXXXXX}

%% These commands are for a PROCEEDINGS abstract or paper.
\acmConference[SIGMOD ’24]{Proceedings of the 2024 International Conference on Management of Data}{June 11--16,2024}{Santiago, Chile}
%
%  Uncomment \acmBooktitle if th title of the proceedings is different
%  from ``Proceedings of ...''!
%
\acmBooktitle{Proceedings of the 2023 International Conference on Management of Data (SIGMOD ’24), June 11--16,2023, Santiago, Chile} 
\acmPrice{15.00}
\acmISBN{978-1-4503-XXXX-X/18/06}
\newcommand{\LC}{\textsc{FasCo}}
\newcommand{\MSCN}{\textsc{MSCN}}
\newcommand{\DeepDB}{\textsc{DeepDB}}
\newcommand{\TLSTMCost}{\textsc{TLSTMCost}}
\newcommand{\QPPNet}{\textsc{QPPNet}}
%%
%% Submission ID.
%% Use this when submitting an article to a sponsored event. You'll
%% receive a unique submission ID from the organizers
%% of the event, and this ID should be used as the parameter to this command.
% \acmSubmissionID{253}

%%
%% For managing citations, it is recommended to use bibliography
%% files in BibTeX format.
%%
%% You can then either use BibTeX with the ACM-Reference-Format style,
%% or BibLaTeX with the acmnumeric or acmauthoryear sytles, that include
%% support for advanced citation of software artefact from the
%% biblatex-software package, also separately available on CTAN.
%%
%% Look at the sample-*-biblatex.tex files for templates showcasing
%% the biblatex styles.
%%

%%
%% The majority of ACM publications use numbered citations and
%% references.  The command \citestyle{authoryear} switches to the
%% "author year" style.
%%
%% If you are preparing content for an event
%% sponsored by ACM SIGGRAPH, you must use the "author year" style of
%% citations and references.
%% Uncommenting
%% the next command will enable that style.
%%\citestyle{acmauthoryear}

%%
%% end of the preamble, start of the body of the document source.
\newcounter{siqiang}
\numberwithin{siqiang}{section}
    \newcommand{\siqiang}[1]{\stepcounter{siqiang}
    \textcolor{red}{Siqiang [\arabic{section}.\arabic{siqiang}]: #1} }
\begin{document}

%%
%% The "title" command has an optional parameter,
%% allowing the author to define a "short title" to be used in page headers.
\title{Less is More: Towards Lightweight Cost Estimator \\for Database Systems [Data Management for Data Science]}

\author{Weiping Yu}
\affiliation{%
  \institution{Nanyang Technological University}
  \country{Singapore}
}
\email{weiping001@e.ntu.edu.sg}

\author{Siqiang Luo}
\affiliation{%
  \institution{Nanyang Technological University}
  \country{Singapore}
}
\email{siqiang.luo@ntu.edu.sg}
%%
%% The "author" command and its associated commands are used to define
%% the authors and their affiliations.
%% Of note is the shared affiliation of the first two authors, and the
%% "authornote" and "authornotemark" commands
%% used to denote shared contribution to the research.


%%
%% By default, the full list of authors will be used in the page
%% headers. Often, this list is too long, and will overlap
%% other information printed in the page headers. This command allows
%% the author to define a more concise list
%% of authors' names for this purpose.
\renewcommand{\shortauthors}{Weiping Yu and Siqiang Luo, et al.}

%%
%% The abstract is a short summary of the work to be presented in the
%% article.
\begin{abstract}
%In a database system, estimating the cost of an execution plan is a fundamental operator and has been proven important to the query optimizer. Recent studies show that machine learning(ML)-based cost estimation methods can achieve promising results. However, state-of-the-art ML-based models tend to incorporate large models, leading to days of training or high model inference latency. 

We present {\LC}, a simple yet effective learning-based estimator for the cost of executing a database query plan. {\LC} uses significantly shorter training time and a lower inference cost than the state-of-the-art approaches, while achieving higher estimation accuracy. The effectiveness of {\LC} comes from embedding abundant explicit execution-plan-based features and incorporating a novel technique called cardinality calibration. Extensive experimental results show that {\LC} achieves orders of magnitude higher efficiency than the state-of-the-art methods: on the JOB-M benchmark dataset, it cuts off training plans by 98\%, reducing training time from more than two days to about eight minutes while entailing better accuracy. Furthermore, in dynamic environments, {\LC} can maintain satisfactory accuracy even without retraining, narrowing the gap between learning-based estimators and real systems.
\end{abstract}

%%
%% The code below is generated by the tool at http://dl.acm.org/ccs.cfm.
%% Please copy and paste the code instead of the example below.
%%
% \begin{CCSXML}
% <ccs2012>
% <concept>
% <concept_id>10002951.10002952</concept_id>
% <concept_desc>Information systems~Data management systems</concept_desc>
% <concept_significance>500</concept_significance>
% </concept>
% </ccs2012>
% \end{CCSXML}
% <ccs2012>
%  <concept>
%   <concept_id>10010520.10010553.10010562</concept_id>
%   <concept_desc>Computer systems organization~Embedded systems</concept_desc>
%   <concept_significance>500</concept_significance>
%  </concept>
%  <concept>
%   <concept_id>10010520.10010575.10010755</concept_id>
%   <concept_desc>Computer systems organization~Redundancy</concept_desc>
%   <concept_significance>300</concept_significance>
%  </concept>
%  <concept>
%   <concept_id>10010520.10010553.10010554</concept_id>
%   <concept_desc>Computer systems organization~Robotics</concept_desc>
%   <concept_significance>100</concept_significance>
%  </concept>
%  <concept>
%   <concept_id>10003033.10003083.10003095</concept_id>
%   <concept_desc>Networks~Network reliability</concept_desc>
%   <concept_significance>100</concept_significance>
%  </concept>
% </ccs2012>
% \end{CCSXML}


% \begin{CCSXML}
% <ccs2012>
%    <concept>
%        <concept_id>10002951.10002952</concept_id>
%        <concept_desc>Information systems~Data management systems</concept_desc>
%        <concept_significance>300</concept_significance>
%        </concept>
%    <concept>
%        <concept_id>10010147.10010257</concept_id>
%        <concept_desc>Computing methodologies~Machine learning</concept_desc>
%        <concept_significance>300</concept_significance>
%        </concept>
%  </ccs2012>
% \end{CCSXML}

\ccsdesc[300]{Information systems~Data management systems}
\ccsdesc[300]{Computing methodologies~Machine learning}
% \ccsdesc[500]{Computer systems organization~Embedded systems}
% \ccsdesc[300]{Computer systems organization~Redundancy}
% \ccsdesc{Computer systems organization~Robotics}
% \ccsdesc[100]{Networks~Network reliability}

%%
%% Keywords. The author(s) should pick words that accurately describe
%% the work being presented. Separate the keywords with commas.
\keywords{database management, cost estimation, machine learning}

%% A "teaser" image appears between the author and affiliation
%% information and the body of the document, and typically spans the
%% page.
% \begin{teaserfigure}
%   \includegraphics[width=\textwidth]{sampleteaser}
%   \caption{Seattle Mariners at Spring Training, 2010.}
%   \Description{Enjoying the baseball game from the third-base
%   seats. Ichiro Suzuki preparing to bat.}
%   \label{fig:teaser}
% \end{teaserfigure}

%%
%% This command processes the author and affiliation and title
%% information and builds the first part of the formatted document.
\maketitle








\let\clearpage\relax
\section{Introduction}
\label{sec:introduction}
% \begin{itemize}
%     % Diffusion of FL
%     \item {\st{Diffusion of FL}}
%     % Security threats to FL
%     \item {\st{Security threats to FL with particular focus on model poisoning}}
%     % Limitations of existing countermeasures
%     \item {\st{Current countermeasures (e.g., KRUM) and their limitations}}
%     % Proposed method and its advantages
%     \item {\st{Intuitive description of the proposed method and its difference (i.e., advantages) w.r.t. state of the art}}
%     % Main contributions
%     \item {\st{Summary of the main contributions of this work}}
%     % Paper's structure and organization
%     \item {\st{Paper's structure and organization}}
% \end{itemize}

% Diffusion of FL
Recently, {\em federated learning} (FL) has emerged as the leading paradigm for training distributed, large-scale, and privacy-preserving machine learning (ML) systems~\cite{mcmahan2017googleai,mcmahan2017aistats}. 
The core idea of FL is to allow multiple edge clients to collaboratively train a shared, global model without disclosing their local private training data.
%Specifically, an FL system consists of a central server and many edge clients; 
A typical FL round involves the following steps: {\em(i)} the server randomly picks some clients and sends them the current, global model; {\em(ii)} each selected client locally trains its model with its own private data; then, it sends the resulting local model to the server;\footnote{Whenever we refer to global/local model, we mean global/local model {\em parameters}.} {\em(iii)} the server updates the global model by computing an \emph{aggregation function}, usually the average (FedAvg), on the local models received from clients.
% \begin{enumerate}
%     \item[{\em(i)}] the server sends the current, global model to the clients and appoints some of them for training;
%     \item[{\em(ii)}] each selected client locally trains its copy of the global model with its own private data; then, it sends the resulting local model back to the server;\footnote{Whenever we refer to global/local model, we mean global/local model {\em parameters}.}
%     \item[{\em(iii)}] the server updates the global model by computing an \emph{aggregation function} on the local models received from clients (by default, the average, also referred to as FedAvg~\cite{mcmahan2017aistats}).
% \end{enumerate}
This process goes on until the global model converges. %(e.g., after a certain number of rounds or other similar stopping criteria).
%\\
% The advantages of FL over the traditional, centralized learning paradigm are undoubtedly clear in terms of flexibility/scalability (clients can join/disconnect from the FL network dynamically), network communications (only model weights\footnote{We will use \textit{parameters} and \textit{weights} interchangeably.} are exchanged between clients and server), and privacy (each client's private training data is kept local at the client's end and not uploaded to the server).
\\
% Security threats to FL
%However, the growing adoption of FL also raises security concerns~\cite{costa2022covert}, particularly about its confidentiality, integrity, and availability.
Although its advantages over standard ML, FL also raises security concerns~\cite{costa2022covert}. %, particularly about its confidentiality, integrity, and availability~\cite{costa2022covert}.
% OLD, LONG VERSION
% Indeed, some work deals with privacy leakage that may expose the local data of some clients~\cite{melis2019sp}. 
% A large body of work, instead, investigates attacks that usually aim to detriment the predictive accuracy of the learned global model. For instance, \emph{data poisoning} attacks achieve this goal by letting an adversary pollute the training set of some corrupt FL clients with maliciously crafted examples~\cite{jagielski2018sp}.
% Similarly, in \emph{model poisoning} the attacker attempts to tweak the global model weights~\cite{bhagoji2019pmlr} by directly perturbing the local model's weights of some infected FL clients before these are sent to the central server for aggregation, usually via so-called Byzantine attacks. 
% It turns out that Byzantine model poisoning attacks severely impact standard FedAvg; therefore, more robust aggregation functions must be designed to make FL systems secure.
Here, we focus on \emph{untargeted model poisoning} attacks~\cite{bhagoji2019pmlr}, where an adversary attempts to tweak the global model weights %\footnote{We will use the terms \textit{parameters} and \textit{weights} interchangeably.} 
by directly perturbing the local model's parameters of some infected clients before these are sent to the central server for aggregation.
In doing so, the adversary aims to jeopardize the global model \textit{indiscriminately} at inference time.
Such model poisoning attacks severely impact standard FedAvg; therefore, more robust aggregation functions must be designed to secure FL systems.
\\
% In this paper, we focus on designing a novel robust aggregation scheme at the server's end to contrast the effect of Byzantine model poisoning attacks.
%
% Current countermeasures and their limitations
%Several countermeasures have been proposed in the literature to combat model poisoning attacks on FL systems.
% Some methods use simple statistics more robust than plain average to smooth the impact of malicious updates (e.g., Trimmed Mean and FedMedian~\cite{yin2018icml}). 
% Other defenses implement outlier detection techniques to discard malicious updates from the aggregation performed at the server's end. Those are either based on heuristics (e.g., Krum/Multi-Krum~\cite{blanchard2017nips} and Bulyan~\cite{mhamdi2018pmlr}) or data-driven approaches (e.g., K-means clustering~\cite{shen2016acm} or DnC via spectral analysis~\cite{shejwalkar2021ndss}). 
% Finally, some strategies rely on a centralized ``source of trust'' to spot potential malicious updates (e.g., FLTrust~\cite{cao2020fltrust}).
% Several countermeasures have been proposed in the literature to combat model poisoning attacks on FL systems, i.e., to discard possible malicious local updates from the aggregation performed at the server's end. 
% These techniques range from simple statistics more robust than plain average (e.g., Trimmed Mean and FedMedian~\cite{yin2018icml}) to outlier detection heuristics (e.g., Krum/Multi-Krum~\cite{blanchard2017nips} and Bulyan~\cite{mhamdi2018pmlr}) or data-driven approaches (e.g., spectral analysis via K-means clustering~\cite{shen2016acm} or spectral analysis), or methods based on ``source of trust'' (e.g., FLTrust~\cite{cao2020fltrust}).
% OLD, LONG VERSION
%Several countermeasures have been proposed in the literature to combat Byzantine model poisoning attacks on FL systems.
% Descriptive statistics
% For example, Trimmed Mean and FedMedian aggregate local model updates using more robust statistics than standard average~\cite{yin2018icml}.
%
% % Heuristics for outlier detection
% Many existing Byzantine-resilient strategies implement some outlier detection heuristics to discard the model updates sent by potentially malicious clients from the input of the aggregation function.
% One of the most popular heuristics is Krum~\cite{blanchard2017nips}.
% This strategy tries to mitigate the impact of Byzantine attacks by selecting as a global model the local model with the smallest sum of Euclidean distances to {\em all} the other local models.
% Although powerful, Krum requires the server to know (or, at least, estimate) the number of malicious FL clients upfront, which is generally impossible in a realistic attack scenario. %
% Moreover, Krum may become ineffective for complex, high-dimensional model parameter spaces due to the curse of dimensionality.
% Bulyan~\cite{mhamdi2018pmlr} tries to overcome this issue by combining Krum with a variant of Trimmed Mean.
% % Data-driven outlier detection
% Other strategies use data-driven outlier detection techniques -- e.g., via K-means clustering~\cite{shen2016acm} -- to spot potential malicious local model updates. 
% %For instance, Shen et al. propose to cluster local model updates with K-means and thus identify outliers.
%
% % Other techniques
% As far as the server is concerned, any local model received can be from a potential malicious client. 
% FLTrust~\cite{cao2020fltrust} assumes the server acts as a client, i.e., trains a local model on an additional {\em trustworthy} dataset at the server's end and compares it against all the local models from other clients. 
% This way, the server can rely on some ``source of trust'' when discarding potentially malicious clients.
%\\
% Limitations of existing Byzantine-resilient strategies
Unfortunately, existing defense mechanisms either rely on simple heuristics (e.g., Trimmed Mean and FedMedian by~\cite{yin2018icml}) or need strong and unrealistic assumptions to work effectively (e.g., foreknowledge or estimation of the number of malicious clients in the FL system, as for Krum/Multi-Krum~\cite{blanchard2017nips} and Bulyan~\cite{mhamdi2018pmlr}, which, however, cannot exceed a fixed threshold).
Furthermore, outlier detection methods using K-means clustering~\cite{shen2016acm} or spectral analysis like DnC~\cite{shejwalkar2021ndss} do not directly consider the temporal evolution of local model updates received.
Finally, strategies like FLTrust~\cite{cao2020fltrust} require the server to collect its own dataset and act as a proper client, thereby altering the standard FL protocol.
\\
% OLD, LONG VERSION
% Overall, existing Byzantine-resilient strategies are either simple heuristics (e.g., FedMedian) or, if they are more complex, they rely on strong and unrealistic assumptions to work effectively (e.g., knowing the number of malicious clients in the FL system in advance, as for Krum and alike).
% Furthermore, data-driven outlier detection methods do not consider the temporary evolution of local model updates received (e.g., K-means clustering). 
% Finally, strategies like FLTrust requires the server to collect its own dataset and act as a proper client, thereby altering the standard FL protocol.
%
% Description of the proposed method
This work introduces a novel pre-aggregation \textit{filter} robust to untargeted model poisoning attacks. Notably, this filter $(i)$ operates without requiring prior knowledge or constraints on the number of malicious clients and $(ii)$ inherently integrates temporal dependencies. 
The FL server can employ this filter as a preprocessing step before applying \textit{any} aggregation function, be it standard like FedAvg or robust like Krum or Bulyan.
Specifically, we formulate the problem of identifying corrupted updates as a multidimensional (i.e., matrix-valued) time series anomaly detection task. 
The key idea is that legitimate local updates, resulting from well-calibrated iterative procedures like stochastic gradient descent (SGD) with an appropriate learning rate, show \textit{higher predictability} compared to malicious updates. This hypothesis stems from the fact that the sequence of gradients (thus, model parameters) observed during legitimate training exhibit regular patterns, as validated in Section~\ref{subsec:intuition}. %until convergence. 
%This regularity may be more pronounced for smooth convex loss functions, but it can still be captured within an appropriate time window, even for more complex and convoluted loss surfaces. 
%We provide evidence of this claim in Appendix~B, where we show that the average mutual information (i.e., ``predictability''), calculated over pairs of legitimate model updates sent at different FL rounds, is significantly higher than the corresponding computation for a malicious client.
\\
Inspired by the matrix autoregressive (MAR) framework for multidimensional time series forecasting~\cite{chen2021je}, we propose the FLANDERS ({\em \textbf{F}ederated \textbf{L}earning meets \textbf{AN}omaly \textbf{DE}tection for a \textbf{R}obust and \textbf{S}ecure}) filter.
The main advantages of FLANDERS over existing strategies like FLDetector~\cite{zhao2020multivariate} are its resilience to large-scale attacks, where $50\%$ or more FL participants are hostile, and the capability of working under realistic non-iid scenarios.
We attribute such a capability to two key factors: $(i)$ FLANDERS works without knowing a priori the ratio of corrupted clients, and $(ii)$ it embodies temporal dependencies between intra- and inter-client updates, quickly recognizing local model drifts caused by evil players. Below, we summarize our main contributions:

\begin{itemize}
\item[{\em(i)}]
We provide empirical evidence that the sequence of models sent by legitimate clients is more predictable than those of malicious participants performing untargeted model poisoning attacks.
\\
\item[{\em(ii)}] 
We introduce FLANDERS, the first pre-aggregation filter for FL robust to untargeted model poisoning based on multidimensional time series anomaly detection.
\\
\item[{\em(iii)}] 
We integrate FLANDERS into Flower,\footnote{\scriptsize{\url{https://flower.dev/}}} a popular FL simulation framework for reproducibility.
\\
\item[{\em(iv)}] 
We show that FLANDERS improves the robustness of the existing aggregation methods under multiple settings: different datasets, client's data distribution (non-iid), models, and attack scenarios.
\\
\item[{\em(v)}] 
We publicly release all the implementation code of FLANDERS along with our experiments.\footnote{\scriptsize{\url{https://anonymous.4open.science/r/flanders_exp-7EEB}}}
\end{itemize}

% Paper's structure and organization
The remainder of the paper is structured as follows. %some related work and the current state-of-the-art solutions to security issues that FL entails. 
Section~\ref{sec:background} covers background and preliminaries. 
In Section~\ref{sec:related}, we discuss related work.
Section~\ref{sec:problem} and Section~\ref{sec:method} describe the problem formulation and the method proposed. % to tackle it. 
Section~\ref{sec:experiments} gathers experimental results. %, and Section~\ref{sec:limitations} discusses some limitations of this work.
Finally, we conclude in Section~\ref{sec:conclusion}.
 %discusses the limitations of this work and draws future research directions.
%reports conclusions and draws perspectives for future research directions.

%%%%%%% OLD %%%%%%%
%to overcome the resilience of Byzantine failures in distributed Stochastic Gradient Descent computations. 
% The strength of Krum is its time complexity, which is linear in the gradient dimension. 
% However, the robustness of the approach is guaranteed for gradient-based learning applications only when the majority of the clients are not compromised. 
% Besides, the aggregation mechanism of Krum, as well as that of similar methods, is robust from a coarse-grained perspective and does not provide solutions to errors and perturbations that may occur at inference time.
%A related approach to~\cite{blanchard2017nips} is the work of Su et al.~\cite{su2016dc}. Here, the authors propose an iterated approximate agreement to tackle a multi-layer scenario attacked by Byzantine agents. 
%However, the method works efficiently on the sole discrete context and it is inapplicable to continuous state environments.
%\gabri{Maybe, we should just talk about the main limitations of existing countermeasures without digging into their details (or, we can just mention Krum as this is the most popular one). I will move the description of all these methods to the Related Work section.}
\section{The Semi-Oblivious Chase Procedure}\label{sec:semi}
%

The semi-oblivious chase (or simply chase) takes as input a database $D$ and a set $\dep$ of TGDs, and constructs an instance that contains $D$ and satisfies $\dep$.
%
A central notion in this context is that of trigger.
%are those of trigger, active trigger, and trigger application.

\begin{definition}%[\textbf{Trigger Application}]
	Given a set $\dep$ of TGDs and an instance $I$, a {\em trigger} for $\dep$  on $I$ is a pair $(\sigma,h)$, where $\sigma \in \dep$ and $h$ is a homomorphism from $\body{\sigma}$ to $I$.
	%
	The {\em result} of $(\sigma,h)$, denoted $\result{\sigma}{h}$, is the set $\mu(\head{\sigma})$, where $\mu : \var{\head{\sigma}} \ra \ins{C} \cup \ins{N}$ is defined as follows:
	%
	%$\mu(x) = h(x)$ if $x \in \fr{\sigma}$, and $\mu(x) = \bot_{\sigma,h_{|\fr{\sigma}}}^{x}$ otherwise,
	\[
	\mu(x)\
	=\ \left\{
	\begin{array}{ll}
	h(x) & \quad \text{if } x \in \fr{\sigma}\\
	&\\
	\bot_{\sigma,h_{|\fr{\sigma}}}^{x} & \quad \text{otherwise}
	\end{array} \right.
	\]
	where $\bot_{\sigma,h_{|\fr{\sigma}}}^{x} \in \ins{N}$.  Let $T(\dep,I)$ be the set of triggers for $\dep$ on $I$.	\hfill\markfull
\end{definition}




Observe that in the definition of $\result{\sigma}{h}$, each existentially quantified variable $x$ of $\head{\sigma}$ is mapped by $\mu$ to a null value of $\ins{N}$ whose name is uniquely determined by the trigger $(\sigma,h)$ and the variable $x$ itself. This means that, given a trigger $(\sigma,h)$, we can unambiguously construct the set of atoms $\result{\sigma}{h}$.
%
The central idea of the chase is, starting from a database $D$, to exhaustively apply triggers for the given set $\dep$ of TGDs on the instance constructed so far.
%
More precisely, given a database $D$ and a set $\dep$ of TGDs, let
\[
\mathsf{chase}^{0}(D,\dep)\ =\ D,
\]
and for each $i>0$, let
\[
\mathsf{chase}^{i}(D,\dep)\ =\ \mathsf{chase}^{i-1}(D,\dep)\ \cup\ \bigcup_{(\sigma,h) \in S} \result{\sigma}{h},
\]
where $S = T(\dep,\mathsf{chase}^{i-1}(D,\dep))$. 
%
We finally define {\em the result of the chase of $D$ w.r.t.~$\dep$} as the (possibly infinite) instance
\[
\chase{D}{\dep}\ =\ \bigcup_{i \geq 0} \mathsf{chase}^{i}(D,\dep).
\]


\ignore{
The semi-oblivious chase procedure (or simply chase) takes as input a database $D$ and a set $\dep$ of TGDs, and constructs an instance that contains $D$ and satisfies $\dep$.
%
Central notions in this context are those of trigger, active trigger, and trigger application.

\begin{definition}%[\textbf{Trigger Application}]
	Given a set $\dep$ of TGDs and an instance $I$, a {\em trigger} for $\dep$  on $I$ is a pair $(\sigma,h)$, where $\sigma \in \dep$ and $h$ is a homomorphism from $\body{\sigma}$ to $I$.
	%
	The {\em result} of $(\sigma,h)$, denoted $\result{\sigma}{h}$, is the set $\mu(\head{\sigma})$, where $\mu : \var{\head{\sigma}} \ra \ins{C} \cup \ins{N}$ is defined as follows:
	%
	%$\mu(x) = h(x)$ if $x \in \fr{\sigma}$, and $\mu(x) = \bot_{\sigma,h_{|\fr{\sigma}}}^{x}$ otherwise,
	\[
	\mu(x)\
	=\ \left\{
	\begin{array}{ll}
	h(x) & \quad \text{if } x \in \fr{\sigma}\\
	&\\
	\bot_{\sigma,h_{|\fr{\sigma}}}^{x} & \quad \text{otherwise}
	\end{array} \right.
	\]
	where $\bot_{\sigma,h_{|\fr{\sigma}}}^{x}$ is a null value from $\ins{N}$.
	%
	The trigger $(\sigma,h)$ is {\em active} if $\result{\sigma}{h} \not\subseteq I$.
	%
	The {\em application} of $(\sigma,h)$ to $I$ returns the instance $J = I \cup \result{\sigma}{h}$ and is denoted as $I \app{\sigma}{h} J$.
	\hfill\markfull
\end{definition}


Observe that in the definition of $\result{\sigma}{h}$ above, each existentially quantified variable $x$ of $\head{\sigma}$ is mapped by $\mu$ to a null value of $\ins{N}$ whose name is uniquely determined by the trigger $(\sigma,h)$ and the variable $x$ itself. This means that, given a trigger $(\sigma,h)$, we can unambiguously extract the set of atoms 
$\result{\sigma}{h}$.



%\medskip

%\noindent
%\textbf{Semi-Oblivious Chase.}
The central idea of the chase is, starting from a database $D$, to exhaustively apply active triggers for the given set $\dep$ of TGDs on the instance constructed so far. This is formalized via the notion of (semi-oblivious) chase derivation, which can be finite or infinite.


\begin{definition}
	Consider a database $D$ and a set $\dep$ of TGDs.
	%We consider the two cases where a derivation is finite or infinite:
	\begin{itemize}
		\item A finite sequence $(I_i)_{0 \leq i \leq n}$ of instances, with $D = I_0$ and $n \geq 0$, is a {\em chase derivation} of $D$ w.r.t.~$\dep$ if, for each $i \in \{0,\ldots,n-1\}$, there is an active trigger $(\sigma,h)$ for $\dep$ on $I_i$ with $I_i \app{\sigma}{h} I_{i+1}$, and there is no active trigger for $\dep$ on $I_n$. The {\em result} of such a chase derivation is the instance $I_n$.
		
		
		\item An infinite sequence $(I_i)_{i \geq 0}$ of instances, with $D = I_0$, is a {\em chase derivation} of $D$ w.r.t.~$\dep$ if, for each $i \geq 0$, there is an active trigger $(\sigma,h)$ for $\dep$ on $I_i$ such that $I_i \app{\sigma}{h} I_{i+1}$. Moreover, $(I_i)_{i \geq 0}$ is {\em fair} if, for each $i \geq 0$, and for every active trigger $(\sigma,h)$ for $\dep$ on $I_i$, there exists $j > i$ such that $(\sigma,h)$ is not an active trigger for $\dep$ on $I_j$. 
		%The latter is known as the {\em fairness condition}, and guarantees that all the active triggers will be deactivated. %
		The {\em result} of such a chase derivation is the instance $\bigcup_{i \geq 0} \, I_i$.
	\end{itemize}
	%
	%The {\em result} of a chase derivation is defined as the union of all the instances occurring in it. 
	A chase derivation is {\em valid} if it is finite or infinite and fair.  \hfill\markfull
\end{definition}


Let us stress that infinite but unfair chase derivations are not considered as valid ones since they do not serve the main purpose of the chase, that is, to build an instance that satisfies the given set of TGDs. Indeed, given the set $\dep$ consisting of the TGDs
\[
\sigma\ =\ R(x,y) \ra \exists z \, R(y,z) \qquad \sigma'\ =\ R(x,y) \ra P(x,y),
\]
the result of the unfair chase derivation of $D = \{R(a,b)\}$ w.r.t.~$\dep$ that involves only triggers of the form $(\sigma,\cdot)$, i.e., only the TGD $\sigma$ is used, does not satisfy $\sigma'$, and thus, it does not satisfy $\dep$.
%
Interestingly, for every database $D$ and set $\dep$ of TGDs, any two valid chase derivations of $D$ w.r.t.~$\dep$ have always the same result, which implies that all valid chase derivations are either finite or infinite~\cite{GrOn18}. Therefore, in the rest of the paper, we can safely refer to {\em the} result of the chase of $D$ w.r.t. $\dep$, which we will denote by $\chase{D}{\dep}$. 
}


%\subsection{Non-Uniform Chase Termination}\label{sec:problem}
%

\medskip

\noindent
\textbf{Chase Termination.}
The result of the chase may be infinite even for very simple settings: it is easy to see that for $D = \{R(a,b)\}$ and $\dep = \{R(x,y) \ra \exists z \, R(y,z)\}$, $\chase{D}{\dep}$ is infinite.
%; in particular, $\chase{D}{\dep} = \{R(a,b),R(b,\bot_1),R(\bot_1,\bot_2),R(\bot_2,\bot_3),\ldots\}$, where $\bot_1,\bot_2,\ldots$ are null values.
%
This leads to the following problem, parameterized by a class $\class{C}$ of TGDs such as $\class{SL}$ (the class of simple-linear TGDs) and $\class{L}$ (the class of linear TGDs):


\medskip

\begin{center}
	\fbox{
		\begin{tabular}{ll}
			%{\small PROBLEM} : & %$\mathsf{ChaseTermination}(\class{C})$
			%\\
			{\small INPUT} : & A database $D$ and a set $\dep$ of TGDs from $\class{C}$.
			\\
			{\small QUESTION} : &  Is the instance $\chase{D}{\dep}$ finite?
	\end{tabular}}
\end{center}

\medskip

\noindent This problem has been recently studied in~\cite{CaGP22} for the classes of simple-linear and linear TGDs. Interestingly, for both classes, the finiteness of the result of the chase has been syntactically characterized by exploiting the notion of non-uniform weak-acyclicity. 
%
We proceed to recall this acyclicity notion, and then present the characterizations established in~\cite{CaGP22}, which in turn lead to simple algorithms for checking the finiteness of the result of the chase.
%
Note that, for the sake of clarity, in the rest of the paper we assume TGDs with a non-empty frontier, i.e., we assume that there is at least one variable in a TGD $\sigma$ that occurs both in $\body{\sigma}$ and $\head{\sigma}$. This assumption can be made without loss of generality since, given a database $D$ and a set $\dep$ of TGDs, we can easily construct a set $\dep'$ of TGDs with a non-empty frontier by slightly modifying $\dep$ such that $\chase{D}{\dep}$ is finite iff $\chase{D}{\dep'}$ is finite.


\medskip

\noindent
\textbf{Non-Uniform Weak-Acyclicity.} Weak-acyclicity was introduced in~\cite{FKMP05} as the main formalism for data exchange purposes, which guarantees the finiteness of the result of the chase for {\em every} input database. Non-uniform weak-acyclicity is the database-dependent variant of weak-acyclicity introduced in~\cite{CaGP22}. We proceed to give the formal definitions.
%
We first need to recall the notion of the {\em dependency graph} of a set $\dep$ of TGDs, 
%which symbolically encodes how terms may propagate during the chase.
%The {\em dependency graph} of set $\dep$ of TGDs 
defined as a directed multigraph $\depg{\dep}=(N,E)$, where $N = \pos{\sch{\dep}}$ and $E$ contains {\em only} the following edges.
%
For each TGD $\sigma \in \dep$ with $\head{\sigma} = \{\alpha_1,\ldots,\alpha_k\}$, for each $x \in \frontier{\sigma}$, and for each position $\pi \in \posvar{\body{\sigma}}{x}$:
\begin{itemize}
	\item For each $i \in [k]$ and for each $\pi' \in \posvar{\alpha_i}{x}$, there exists a \emph{normal} edge $(\pi,\pi') \in E$.
	%
	\item For each existentially quantified variable $z$ in $\sigma$, $i \in [k]$, and $\pi' \in \posvar{\alpha_i}{z}$, there is a \emph{special} edge $(\pi,\pi') \in E$.
\end{itemize}
%
We further need to define when a predicate is reachable from another predicate. 
%
Given predicates $R,P \in \sch{\dep}$, {\em $P$ is reachable from $R$ (w.r.t.~$\dep$)} if $R = P$, or there exists a path in $\depg{\dep}$ from a position of the form $(R,i)$ to a position of the form $(P,j)$.
%
%we write $R \ra_\dep P$  if $R = P$, or there exists a TGD $\sigma \in \dep$ such that $R$ occurs in $\body{\sigma}$ and $P$ occurs in $\head{\sigma}$. We say that {\em $P$ is reachable from $R$ (w.r.t.~$\dep$)}, denoted $R \reach{\dep} P$, if (i) $R \ra_\dep P$, or (ii) there exists $T \in \sch{\dep}$ such that $R \reach{\dep} T$ and $T \ra_\dep P$.
%in $\depg{\dep}$, denoted $R \reach{\dep} P$, if there exists a path in $\depg{\dep}$ from a position $(R,i)$ to a position $(P,j)$, for some $i \in [\arity{R}]$ and $j \in [\arity{P}]$.
Given a database $D$, we say that a (not necessarily simple and possibly cyclic) path $C$ in $\depg{\dep}$ is \emph{$D$-supported} if there exists an atom $R(\bar t) \in D$ and a node of the form $(P,i)$ in $C$ such that $P$ is reachable from $R$.
%
We are now ready to recall (non-uniform) weak-acyclicity.



\begin{definition}\label{def:dwa}
	Consider a database $D$ and a set $\dep$ of TGDs. We say that $\dep$ is {\em weakly-acyclic w.r.t.~$D$}, or {\em $D$-weakly-acyclic}, if there is no $D$-supported cycle in $\depg{\dep}$ with a special edge. 
	%
	We say that $\dep$ is {\em weakly-acyclic} if there is no cycle in $\depg{\dep}$ with a special edge. \hfill\markfull
\end{definition}


\smallskip

\noindent
\textbf{Characterizing the Finiteness of the Chase.}
It is not very difficult to show that whenever a set $\dep$ of TGDs (not necessarily linear) is $D$-weakly-acyclic, then the instance $\chase{D}{\dep}$ is finite. In other words, the $D$-weak-acyclicity of $\dep$ is a sufficient condition for the finiteness of $\chase{D}{\dep}$. What is more interesting is that, assuming that $\dep$ is a set of simple-linear TGDs, the $D$-weak-acyclicity of $\dep$ is also a necessary condition for the finiteness of $\chase{D}{\dep}$. This leads to the following characterization established in~\cite{CaGP22}:

\begin{theorem}\label{the:characterization-simple-linear}
	Consider a database $D$ and a set $\dep \in \class{SL}$ of TGDs. It holds that $\chase{D}{\dep}$ is finite iff $\dep$ is $D$-weakly-acyclic.
\end{theorem}

For linear TGDs, it turned out that non-uniform weak-acyclicity is not powerful enough for characterizing the finiteness of the chase instance. Here is an example given in~\cite{CaGP22} that illustrates this fact:
%This is illustrated by the following example.


\begin{example}
	Consider the database $D = \{R(a,b)\}$ and the singleton set $\dep$ consisting of the (non-simple) linear TGD
	\[
	R(x,x)\ \ra\ \exists z \, R(z,x). 
	\]
	It is easy to see that there is no trigger for $\dep$ on $D$. This means that $\chase{D}{\dep} = D$ is finite, whereas $\dep$ is {\em not} $D$-weakly-acyclic. \hfill\markfull
\end{example}


To obtain a characterization analogous to Theorem~\ref{the:characterization-simple-linear}, the authors of~\cite{CaGP22} used the technique of {\em simplification} to convert linear TGDs into simple-linear TGDs, while preserving the finiteness of the chase instance. We proceed to recall this technique.
%
Let $\bar t = (t_1,\ldots,t_n)$ be a tuple of (not necessarily distinct) terms. We write $\unique{\bar t}$ for the tuple obtained from $\bar t$ by keeping only the first occurrence of each term in $\bar t$.
%
For example, if $\bar t = (x,y,x,z,y)$, then $\unique{\bar t} = (x,y,z)$.
%
For each $i \in [n]$, the \emph{identifier of $t_i$ in $\bar t$}, denoted $\id{\bar t}{t_i}$, is the integer that identifies the position of $\unique{\bar t}$ at which $t_i$ appears. 
%
We write $\id{}{\bar t}$ for the tuple $(\id{\bar t}{t_1},\ldots,\id{\bar t}{t_n})$.
%
For example, if $\bar t = (x,y,x,z,y)$, then $\id{}{\bar t} = (1,2,1,3,2)$.
%
For an atom $\alpha = R(\bar t)$, the {\em simplification of $\alpha$}, denoted $\simple{\alpha}$, is the atom $R_{\id{}{\bar t}}(\unique{\bar t})$, whereas the {\em shape of $\alpha$}, denoted $\shape{\alpha}$, is the predicate $R_{\id{}{\bar t}}$. We can naturally refer to the simplification and the shape of a set of atoms.
%
For a tuple of variables $\bar x = (x_1,\ldots,x_n)$, a \emph{specialization of $\bar x$} is a function $f$ from $\bar x$ to $\bar x$ such that $f(x_1) = x_1$, and $f(x_i) \in \{f(x_1),\ldots,f(x_{i-1}),x_i\}$, for each $i \in \{2,\ldots,n\}$.
We write $f(\bar x)$ for $(f(x_1),\ldots,f(x_n))$. We are now ready to recall how a set of linear TGDs is converted into a set of simple-linear TGDs.

\begin{definition}\label{def:simplification}
	Consider a linear TGD $\sigma$ of the form
	\[
	R(\bar x) \ra \exists \bar z\, \psi(\bar y,\bar z), 
	\]
	where $\bar y \subseteq \bar x$, and a specialization $f$ of $\bar x$. The {\em simplification of $\sigma$ induced by $f$} is the simple-linear TGD
	\[
	\simple{R(f(\bar x))} \rightarrow \exists \bar z\, \simple{\psi(f(\bar y),\bar z)}.
	\]
	We write $\simple{\sigma}$ for the set of all simplifications of $\sigma$ induced by some specialization of $\bar x$.
	%
	For a set $\dep \in \class{L}$ of TGDs, the {\em simplification of $\dep$} is defined as the set
	\[
	\simple{\dep}\ =\ \bigcup_{\sigma \in \dep} \simple{\sigma}
	\]
	consisting only of simple-linear TGDs. \hfill\markfull
\end{definition}

We can now recall the characterization for the finiteness of the chase instance for linear TGDs, established in~\cite{CaGP22}, which is similar to the one for simple-linear TGDs, with the key difference that first we need to simplify both the database and the set of linear TGDs:

\begin{theorem}\label{the:characterization-linear}
	Consider a database $D$ and a set $\dep \in \class{L}$ of TGDs. Then, $\chase{D}{\dep}$ is finite iff $\simple{\dep}$ is $\simple{D}$-weakly-acyclic.
\end{theorem}

It is clear that Theorems~\ref{the:characterization-simple-linear} and~\ref{the:characterization-linear} provide simple algorithms for checking whether the chase instance is finite. In particular, given a database $D$ and a set $\dep$ of simple-linear TGDs, we simply need to check whether $\dep$ is $D$-weakly-acyclic, in which case the algorithm returns \true; otherwise, it returns \false. The same holds when $\dep$ is a set of linear TGDs, with the difference that the algorithm first needs to simplify $D$ and $\dep$, and then perform the acyclicity check.
%
Our goal is to experimentally evaluate the above algorithms with the aim of understanding which input parameters affect their performance, clarifying whether they can be applied in a practical context, and revealing their performance limitations. Of course, a naive implementation of the above algorithms, especially for linear TGDs where the expensive simplification must be applied, will lead to poor performance, and thus, will not be very useful towards our goal. Hence, we need to somehow convert the above theoretical algorithms into practical algorithms that are amenable to efficient implementations. This is the subject of the next section.
\section{FasCo}\label{sec:LC}

%In this section, we will introduce the light tree-based cardinality-estimation-free model for cost estimation, called LightCost. First, we discuss how we select and utilize the features in Section~\ref{sec:feature}. Next, we present the model design of LightCost in Section~\ref{sec:modeldesign}. Finally, we will describe how we leverage fewer training samples to train LightCost in Section~\ref{sec:training}.
%
%In this section, we introduce {\LC}, a simple yet effective cost estimator which uses much fewer training execution plans than the state of the arts. We will discuss how we select and utilize the features in Section~\ref{sec:feature}, present the main model design in Section~\ref{sec:modeldesign}, and describe how we leverage fewer execution plans for training in Section~\ref{sec:training}.

% \subsection{Main Idea}
%  \label{sec:mainidea}

% As shown in Figure~\ref{fig:modeldesign}, the keys LightCost can keep both high efficiency and accuracy are efficient model design, explicit feature input and reasonable training. In addition, cardinality correction can improved accuracy significantly.
%
{\LC} is a model designed to accurately and efficiently estimate the runtime (cost) of an execution plan generated by a database system. The model consists of four main stages, which are outlined in Figure~\ref{fig:modeldesign}: First, {\LC} takes an execution plan corresponding to a SQL statement as input. The execution plan is a tree structure with nodes representing specific operations (e.g., {\it Sequential Scan}) required to execute the SQL statement. For each node, {\LC} then generates an embedding that conveys abundant information. The embedding of a parent node is generated by feeding the embeddings from its two children into a simple neural network. This allows {\LC} to be lightweight and efficient, using 98\% fewer training plans, incurring $40$ times lower training cost, and achieving higher accuracy than previous methods. While {\LC} incorporates a tree-structured network similar to QPPNet~\cite{marcus2019plan}, we highlight the main differences and novelties of {\LC} as follows. (1) \textbf{Efficient model design.} To accelerate training and inference, we propose a lightweight network architecture. First, we simplify the execution plan tree structure by merging each unary node and its only child node. As such, a unary node is no longer treated as an independent node and hence reduces the training and inference costs which are proportional to the number of nodes. Then, we share the information between the nodes effectively so that the model can be compressed to three sets of MLPs with only a few layers (as shown in Figure~\ref{fig:modeldesign}(d)). This allows us to avoid learning a separate model for each operator as in~\cite{marcus2019plan}. %This architecture can maintain high accuracy because we use more explicit features and more reasonable training methods, which will also be discussed in this section. 
We find that such a simple architecture becomes surprisingly powerful when integrated with the features we carefully selected. Due to the simplicity of the model, this architecture significantly speeds up the training and inference. In addition, we incorporate a weighted Q-error into the loss function, making our model more accurate (see Section~\ref{sec:modeldesign}).
%
% Third, to further improve convergence speed and accuracy, we design a weighted Q-error for loss function. This loss function can make the model prioritize learn more important and accurate information.
%
%
%
%
% \noindent \textbf{Explicit features as input.} To improve the efficiency and accuracy of the model, we input more explicit features from both the current node and leaves (as shown in Figure~\ref{fig:modeldesign} (c)). What needs to be highlighted is that we add the cardinalities and costs of the leaves into the feature. Inputting more explicit information can speed up the convergence and improve the accuracy.
%
%
% \vspace{1mm}
% \noindent \textbf{Cherry-picked Explicit features.} {\color{blue} We integrate more features as explicit input as shown in Figure~\ref{fig:modeldesign} (c). We highlight that we also add the cardinalities and costs of the child nodes into the feature, and we also carefully design the initialization of cardinalities. We find that these features are important to enhance the accuracy of the model. More details of feature extraction will be discussed in Section~\ref{sec:feature}.}
(2) \textbf{Cherry-picked explicit features.} { We integrate more features directly related to the execution plan as explicit inputs as shown in Figure~\ref{fig:modeldesign} (c). %We highlight that we also add the cardinalities and costs of the child nodes into the feature, and we also carefully design the initialization of cardinalities. 
Previous works rarely rely on these explicit features (e.g., \textit{Subquery} and \textit{Cardinality}) for model training and some of them rely on adding hidden features to secure accuracy. We find that by incorporating these explicit features, we can significantly reduce the size of hidden features. We note that these explicit features are important to enhance the accuracy of the model. More details of feature extraction will be discussed in Section~\ref{sec:feature}.}
%
%
% \noindent \textbf{Reasonable training} We adapt the pipeline for generating training data 
% and adopt tree-structure back propagation to optimize LightCost.  We will elaborate the process in Section~\ref{sec:training}.
%
% \noindent \textbf{Cardinality correction.} Besides above techniques, we also investigate cardinality given by histograms and proposed cardinality correction to further improve the accuracy.  The correction is based on sampling without sacrificing much efficiency.  We will elaborate the process in Section~\ref{sec:correction}.
(3) \textbf{Cardinality calibration.} We also investigate the cardinality modeling given by histograms and propose {\it cardinality calibration} to further improve the accuracy.  This technique is based on sampling without sacrificing much efficiency.


\vspace{-2mm}
\subsection{Model Design}   
\label{sec:modeldesign}
%{\color{red} what exactly is the model here? A model should usually have an objective function and parameters to train. All these are unclear.}
%In this section, we will describe the flow of data through {\LC} from the generated plans to the cost estimates, including preprocessing, model architecture, and loss function.


% \noindent \textbf{Preprocessing.} To make full use of the information provided by the database system, the base structure of {\LC} follows the tree of execution plans. However, we modify the process about materialization nodes to improve efficiency. Among the nodes of all types of sub-plans, we observe that the materialization nodes contain less information than the other two types of nodes (scan nodes and merge nodes). Moreover, each materialized node has one and only one leaf (merge node has two, scan node has no leaves), and almost all its information comes from its leaf. For example, In the 3-rd node of the example tree in Figure~\ref{fig:modeldesign}, the Aggregate node computes the count of a set of input values from the Hash Join node. So we merge the materialization nodes to their leaves as shown in Figure~\ref{fig:preprocess}. The method is to take the total cost of the two nodes as the cost of the merged node, and other information comes from the leaf before the merge. The materialization nodes no longer act as independent node, reducing model inference latency and training cost.

% \vspace{1mm}
\noindent \textbf{Preprocessing.} The base structure of {\LC} is geared to the tree of the input execution plan. To improve the efficiency, we simplify the tree to have fewer nodes, making the subsequent operators simpler and more efficient. Particularly, among the nodes of all types of sub-plans, the unary node with a single child node contains less information than the other types of nodes. This is because most of its information comes from its immediate child node. For example, In Node 3 of the example tree in Figure~\ref{fig:modeldesign} (a), the {\it Aggregate} node computes the count of a set of input 
tuples from the {\it Hash Join} node. Hence, we merge each unary node to its child node as shown in Figure~\ref{fig:modeldesign} (a). By merging we mean to take the total cost of the two nodes as the cost of the merged node, and the merged node also contains other information that comes from the original child node. After merging, the unary node no longer acts as an independent node, and the whole structure is simplified, thereby reducing model inference latency and training cost. %{\color{blue} This is because the forward propagation (will be discussed in the next part) of {\LC} is calculated once for each tree node. A materialization node and its child node need to be calculated twice before merging while the merged node only needs to be calculated once. }

% Besides merging nodes, we also encode the text-type features by ordinal encoding in preprocessing. For example, we encode the ``Outer'' to 0 and ``Inner'' to 1 in the feature ``parent realtionship''. We will explain the details shortly in Section~\ref{sec:feature}.

% The input of each node is the concatenated vector, containing embeddings of the selected features, as well as cardinality and cost from its leaf nodes. The information of the leaf node is transferred to its root node implicitly and explicitly through hidden states and direct input, respectively. 
% We simulate the tree-structure model as a stack, which can be formalized as Algorithm~\ref{alg:stack}.
% \begin{algorithm}[t]
% \caption{Simulating LightCost with stack}
% \label{alg:stack}
% \KwIn{A post-order list of nodes in an execution plan $\{n_i\}$;  The parameters of LightCost model $\theta$} 
% \KwOut{The set of cost $C$}
% \Begin{
% Initialize the stack of cost $C$\;
% Initialize an stack of plan nodes $P$\;
% \For{$n_i \in \{n_i\}$}{
% \If{$length(P) >=2$ $and$ $level(P[-1])=level(P[-2])$}{
% $leaf_{l} = pop(P)$ \;
% $leaf_{r} = pop(P)$ \;
% Concat the selected features in $leaf_{l},leaf_{r}$ and $n_i$ as $input$\;
% $state,cost = inference(input,\theta)$\;
% Add $state$ and $cost$ into $n_i$ as features\;
% $P=push(n_i,P)$\;
% $C=push(cost,C)$\;
% }
% \Else{
% Initialize the feature of null node $f$\;
% Concat the selected features in $n_i$ and $f$ as $input$\;
% $state,cost = inference(input,\theta)$\;
% Add $state$ and $cost$ into $n_i$ as features\;
% $P=push(n_i,P)$\;
% $C=push(cost,C)$\;
% }
% }
% }
% \end{algorithm}
% The step of concatenating the features from leaves and $i$th current node to get the input can be formalized as:
% \begin{displaymath}
% \mathcal{F}(leaf_i^l) = Concat(E_{op}(Op_i^l),Cost_i^l,Card_i^l,E_{jk}(JK_i^l),State_i^l)
% \end{displaymath}
% \begin{displaymath}
% \mathcal{F}(leaf_i^r) = Concat(E_{op}(Op_i^r),Cost_i^r,Card_i^r,E_{jk}(JK_i^r),State_i^r)
% \end{displaymath}
% \begin{displaymath}
% \mathcal{F}(cur_i) = Concat(E_{op}(Op_{i}),Card_{i},Filter_{i},Parent_{i})
% \end{displaymath}
% \begin{displaymath}
% \mathcal{I}_i = Concat(\mathcal{F}(leaf_i^l),\mathcal{F}(cur_i),\mathcal{F}(leaf_i^r)),
% \end{displaymath}
% where $Op$ denotes the physical operation, $Card$ denotes the cardinality, $State$ denotes the hidden state, $JK$ denotes the join keys, and $Parent$ denotes parent relationship which are all described in Section~\ref{sec:feature}. 
\begin{figure*}[ht]
\vspace{-1.5mm}
  \centering
  \includegraphics[width=0.86\linewidth]{figure/model.pdf}
  \vspace{-3mm}
  \caption{Model design of {\LC}.}
\label{fig:modeldesign}
\vspace{-5mm}
\end{figure*}

% After feature integration, the input is fed into the backbone MLP layer and activation layer sequentially. The activation layer here we apply is $Tanh$, because it is zero-centered and all neurons can be fully utilized than $ReLU$ without the dead ReLU problem. Then the state and cost will be calculated respectively by state MLP and cost MLP. The activation layer we use when calculating $Cost$ is $Sigmoid$, because it needs to be bounded and strictly greater than 0.
% Figure~\ref{fig:modeldesign} shows an example of how we process an execution plan. It can be seen from the right side that we only adopt MLP and activation layer in the model, which simplifies the calculation extremely and keep the model size under 20 kilobytes. 
% \begin{figure}[h!]
%   \centering
%   \includegraphics[width=0.9\linewidth]{figure/preprocess.pdf}
%   \caption{Preprocessing.}
% \label{fig:preprocess}
% \end{figure}

% \vspace{1mm}
\noindent \textbf{Model architecture.} {\LC} adopts a tree-based model architecture because (1) it resembles the structure of the execution plan and hence gives a nice one-one correspondence between the network node and plan node; and (2) it was proved to be effective~\cite{marcus2019plan,sun13end,hilprecht2022zero}. Further, to make the model lightweight, we use MLP, which is the simplest neural network. Our experiments demonstrate that this simplicity does not lead to worse effectiveness. 
%\textcolor{black}{We follow the tree-based model because it resembles the structure of execution-plan trees well. From the view of each node, we try to use only the simplest neural network, MLP, we will elaborate as follows. (R1.D1)}


The training follows a post-order traversal of the execution tree, with each plan node corresponding to a model node that takes input from its two child nodes. For example, the three-node plan in Figure~\ref{fig:modeldesign}(a) (right) is transformed into a model with three corresponding components in Figure~\ref{fig:modeldesign}(d) (left).
The $j$-th and $k$-th nodes are the left and right child nodes of the current $i$-th node, and in this concrete example, $j$,$k$, and $i$ are 1,2 and 3 respectively. We collect three sets of information from them. The {\bf first one} is the feature $x_i$, which denotes the full information of the current node, and some information (e.g., cardinality) of its two child nodes. $x_i$ can be obtained in the execution plan tree and we defer the details to Section~\ref{sec:feature}. The {\bf second one} is the hidden states $s_j$ and $s_k$, which denote the implicit information derived from its left and right child nodes respectively. These hidden states help correct the cost and cardinality estimates and provide more information of context beyond just using the explicit features $x_i$. % They can detect the abnormal estimations due to purely using feature $x_i$ as input, and calibrate the estimation automatically during learning. For example, sometimes the cardinality we input to the current node is extremely abnormal. The hidden state will record information such as joins or operations of the previous nodes, and judge whether the cardinality is too large or too small, thereby helping us to correct the results.
The {\bf third one} is the estimated costs $c_j$ and $c_k$ that are respectively from its left and right child node. As the ultimate goal is to get an accurate total estimated cost for the whole execution plan, the accumulation of the cost of each node is also crucial in establishing the final estimate.

% The {estimated cost} is the cost for executing the sub-plan rooted at the node. As the ultimate goal is to get an accurate estimated node cost for the root of the whole execution plan, and the estimated cost is derived from the children node, the estimated  cost for each node is also crucial in establishing the final estimation. 
% For normalization, we set an upper-bound on the maximum allowable execution time, denoted as $c_{max}$. $c_{max}$ can be set based on the maximum time in the generated training execution plans. Then, the normalization can be formalized as $norm(c) = \frac{\log c}{\log c_{max}}$.
% \begin{displaymath}
%   norm(c) = \frac{\log c}{\log c_{max}}
% \end{displaymath}
% The advantage of this logarithm normalization over uniform normalization is that it can get a more even distribution, rectifying the excessively long execution time incurred by extreme cases.

The inference based on the model is performed in a propagation following the execution plan tree in a post-order manner. %All the above information will be descried in detail in Section~\ref{sec:feature}. Figure~\ref{fig:back} shows the process of forward propagation related to the $i$-th node. If omitting the embedding layers and the biases, we only need to compute three multiplications of threthe e sets of weight matrices and their input. To simplify the formula, we use $W$,$U$,$V$ to denote the weights of backbone, state and cost MLP in Figure~\ref{fig:modeldesign}(d) respectively. Note that they do not necessarily represent only one linear layer, but can be a combination of multiple linear layers and activation layers. Then the forward propagation of the non-leaf $i$-th node can be formalized as:
In Figure~\ref{fig:modeldesign}(d), the feature $x_3$ (for Node 3), together with ($s_1$, $c_1$) from Node 1 and ($s_2$, $c_2$) from Node 2, is input into an MLP (named Backbone MLP) to produce an output $o_3$, which is then fed into two MLPs (named State MLP and Cost MLP) to generate the state $s_3$ and cost $c_3$. These values are propagated upward along the tree similarly and finally, we get the estimated cost at the root node. 
% \vspace{-1mm}

The internal structure of these MLPs is designed to be concise with a stack of linear layers and $Tanh$ activation function. Meanwhile, the network size can be compacted to a few layers with only thousands of parameters, leading to extremely low latency. 

% More specifically, let us respectively denote $l_{bone}$,$l_{state}$,$l_{cost}$ the number of layers in the Backbone, State, and Cost MLPs, and $Tanh$, $Sigmoid$ are common activation layers~\cite{nwankpa2018activation}. The variables for Node $i$ can be calculated as follows:
% \begin{align}
% o_i&=MLP_{bone}([s_k,c_k,x_i,c_j,s_j];l_{bone},Tanh,Tanh)\nonumber\\
% s_i&=MLP_{state}(o_i;l_{state},Tanh,Tanh)\nonumber\\
% c_i&= MLP_{cost}(o_i;l_{cost},Tanh,Sigmoid)\nonumber
% \end{align}
% \begin{algorithm}[t]
% \LinesNumbered  
% \SetAlgoVlined
% % 
% \caption{The function of $MLP(X_0; l,\sigma_m,\sigma_f)$ \label{alg:mlp}}
% \KwIn{Input $X_0$; Number of layers $l$; Intermediate activation layer $\sigma_m$; The final activation layer $\sigma_f$; A weight matrix $W$; A bias matrix $b$;}
% \KwOut{Output $X_l$}
% \For{i = $1$ to $l-1$ }{
%         $X_i = \sigma_m(W_i \times X_{i-1} + b_i)$ 
%     }
%           $X_l = \sigma_f(W_l \times X_{l-1} + b_l)$ 
% return $X_l$\;
% \end{algorithm}
% %For the leaf nodes, the input $s$, $c$ and some part of $x$ will be initialized, which will be discussed in Section~\ref{sec:feature}.
% The function of $MLP()$ is defined as Algorithm~\ref{alg:mlp}, where the choice of the number of layers will be discussed in the experimental section. The input values for leaf nodes will be discussed in Section~\ref{sec:feature}.

% \noindent \textbf{Estimated Node Cost.} is defined as the estimated cost for executing the sub-plan rooted at the node. Our ultimate goal is to get an accurate estimated node cost for the root of the whole execution plan. As in our model, the estimated node cost is derived from the children node. Hence, this feature is also crucial as it is passed between the parent node and children nodes. %We input the costs of the child nodes into each current node. Because the cost estimate of the current node is based on the cost of its child nodes.

% The database system will give two types of time at each node: the startup time and the total time, both of which start timing when the entire plan starts executing. The startup time may overlap with the execution time of the previous nodes, and there is some randomness, which will confuse the model and increase the training cost. Instead, the total times increase strictly in post-order, and the model can directly learn the final target. So we only select the total time as one of the features. 

%For normalization, we assume that execution time is bounded in practical use, whose bound is $(0,Max]$ ($Max$ is the maximum execution time in all generated training samples), then the normalization can be formalized as:
% We set an upper-bound on the maximum allowable execution time, denoted as $c_{max}$. $c_{max}$ can be set based on the maximum time in the generated training execution plans. Then, the normalization can be formalized as:
% \begin{displaymath}
%   norm(c) = \frac{\log c}{\log c_{max}}
% \end{displaymath}
% The advantage of this logarithm normalization over uniform normalization is that it can get a more evenly distribution, rectifying the excessively long execution time incurred by some extreme cases.% At each non-leaf node we input the execution time of its leaf nodes, while at leaf nodes we initialize the inputs as 0.  
% and fluctuates greatly under the influence of the database system itself. 
% This is because (1) in general, a leaf node usually starts from time 0 because of its independence; (2) occasionally if it starts during the execution of other nodes, then this time should also be counted as part of the execution time of this node, because it is waiting for other resources, which can also be learned by the model.


\noindent \textbf{Loss Function.} Based on the aforementioned model and the corresponding inference, our goal is to minimize the estimate error at each node. The estimated cost $c_i$ for Node $i$ can be compared with the ground-truth cost $l_i$, giving us a loss as: $\text{Q-error}(c_{i},l_{i}) = max(\frac{c_{i}}{l_{i}},\frac{l_{i}}{c_{i}})$.
% \begin{align}
% q_i=\text{Q-error}(c_{i},l_{i}) = max(\frac{c_{i}}{l_{i}},\frac{l_{i}}{c_{i}})\label{eq:qerror}
% \end{align}
% where $c_{i}$ and $l_{i}$ respectively denote the estimated cost and ground truth. 
% The reason why Q-error can be used directly as a loss function is that when the distributions of $c_{i}$ and $l_{i}$ are roughly the same, not only are the magnitude similar in the two cases, but the magnitudes of the gradient are also similar. Let the input of the last MLP be $x$, then the estimated cost is:
% \begin{displaymath}
% c_{i}= \sigma(V \times o_i + b),
% \end{displaymath}
% in which $\sigma$ denotes the $Sigmoid$ function, $W$ and $b$ denote the weight and bias of the last MLP. Then the gradient of $q_i$ can be formalized as below, 
% The gradient is well defined,
% when $c_{i}>l_{i}$:
% \begin{displaymath}
% \frac{ \partial q_i}{\partial V} = \frac{1}{l_{i}}\sigma'(V \times o_i + b)o_i;
% \end{displaymath}
% when $c_{i}<l_{i}$:
% \begin{displaymath}
% \frac{ \partial q_i}{\partial V} = -\frac{l_{i}}{c_{i}^2}\sigma'(V \times o_i + b)o_i
% \end{displaymath}
% When the distributions of $l_{i}$ and $c_{i}$ are the same, $\frac{1}{l_{i}}$ and $\frac{l_{i}}{c_{i}^2}$ are equal-order, leading to a stable gradient.
As each node may have different impacts on the final estimation, we add weights to aggregate the final loss as:
\vspace{-2mm}
\begin{equation}
\label{equ:3}
L=\frac{1}{n}\sum_i \lambda_i \text{Q-error}(c_{i},l_{i}),
\vspace{-1mm}
\end{equation}
% \vspace{-1mm}
where $n$ is the number of nodes in an execution plan and $\lambda_i$ is the weight of $i$-th node. %As we will discuss in the experiments, it is sufficient to set only two different values of $\lambda's$ which respectively correspond to the leaf nodes and non-leaf nodes.
%
%
% \begin{figure}[h]
%   \centering
%     \begin{subfigure}{\linewidth}
%         \includegraphics[height=3.6cm]{figure/plan-actual.pdf}
%         \caption{Original plan cardinalities vs. actual cardinalities.}
%         \label{fig:plan-actual}
%     \end{subfigure}
%     \begin{subfigure}{\linewidth}
%         \includegraphics[height=3.6cm]{figure/correct-actual.pdf}
%         \caption{Corrected plan cardinalities vs. actual cardinalities.}
%         \label{fig:correct-actual}
%     \end{subfigure}
%   \caption{Plan cardinalities vs. actual cardinalities.}
% \label{fig:planrows}
% \end{figure}
%
 To determine how to set $\lambda$, we investigate the actual costs for different types of nodes. From a statistical perspective, we observe that the execution time of nodes using an index (index nodes, e.g., {\it Index Scan} and {\it Bitmap Scan}) are usually short and subject to fluctuation, which increases the difficulties in model learning. 
 On the contrary, the nodes without using an index (non-index nodes, e.g., {\it Seq Scan} and { \it Nested Loop}) would take longer, bringing more reliable cost information. Moreover, the index nodes' runtime is only a small fraction of the total execution time. Since our final target is to estimate the total cost, the estimation accuracy largely depends on the estimates at the non-index nodes. Hence, we assign larger $\lambda$ values for nodes without using indexes. The setting of $\lambda$'s will be further discussed in the experimental section.

\vspace{-3mm}
\subsection{Feature Extraction}   
\label{sec:feature}
% Different from existing methods, we adopt more explicit inputs at each node instead of letting the model learn excessive implicit information (as shown in Figure~\ref{fig:modeldesign} (c)) to reduce the difficulty of model learning. The features we selected all have a direct impact on execution time, so as to avoid useless features generating meaningless costs. In this section, we will describe in detail the reason why we select and how to process these features.

% Existing methods use excessive implicit variables 
% This section introduces the details of feature $x_i$ for Node $i$. Existing tree-based methods~\cite{sun13end,marcus2019plan} usually regard excessive features as implicit variables to train the model. This unavoidably brings significant training or inference overheads. Different from existing methods, we adopt more explicit input features to reduce the model complexity and ease the learning process. Our intuition is that there are existing explicit plan-related features involving which significantly boost the model accuracy and training efficiency. We next describe the features we selected in detail.

This section introduces the details of feature $x_i$ for Node $i$. Unlike existing tree-based methods~\cite{sun13end,marcus2019plan} that employ excessive implicit features, resulting in significant training or inference overhead, we adopt more explicit input features to reduce complexity and improve the learning process. Our intuition is that involving more explicit plan-related features can significantly boost accuracy and efficiency. We will describe the selected features in detail.

% \vspace{1mm}
% \noindent \textit{\underline{Physical Operator}} is the most influential factor in the execution process which is also known as node type in an execution plan. The reason we include it as the first feature is that the execution time of different types of operators varies widely which has been discussed in Section~\ref{sec:problem}.
\noindent \textit{\underline{Physical Operator}} is an influential factor in the execution process since the runtime of different types of operators varies widely.
We use an embedding layer to process this feature. The advantage is that the embedding layer avoids the dependency on the order of features when encoding. Instead, the neural network will automatically learn the relationship and distinction between them. 
% Embedding layer is a special form of multi-layer perceptron (MLP). Supposing the $i$-th physical operation $x_i \in \{x_0,x_1,\ldots,x_{n-1}\}$, in which $n$ is the number of features that need to be embedded, we first convert it to one-hot encoding: $x_i^T = [0_{(0)},0_{(1)},\ldots,1_{(i)},\ldots,0_{(n-1)},]$. Then a MLP layer is adopted:
% \begin{displaymath}
% E(x_i)= Tanh(W_e \times x_i   + b_e),
% \end{displaymath}
% \noindent
% where $W_e$, $b_e$ are the weight and bias of the embedding layer. 

% After this process, taking PostgreSQL~\cite{postgresql1996postgresql} on the JOB and JOB-light~\cite{leis2015good} as an example, there are a total of 32 physical operations, and we use an embedding layer whose dimension equals 5 (according to the experimental estimation of $\log_2(n)$) for embedding.




% \vspace{1mm}
% \noindent \textit{\underline{Parent Relationship}} is a feature describing the relationship between the sibling sub-plans under the same parent node. The relationship can be either {\it outer} or {\it inner}. {\it Outer} means that the node queries the data from a raw table. {\it Inner} means that the node queries the data from the result of its sibling node, which means the node is {\it based on} its sibling node. We simply encode the {\it Outer} and {\it Inner} to ``0'' and ``1'' without embedding layers, and pad ``0'' to the nodes without this feature.
\noindent \textit{\underline{Subquery}} in the context refers specifically to the relationship between sibling sub-plan nodes. If the current node is the second child of its parent and it is a {\it Subquery} of its sibling, it means that the node queries the data from the result of its sibling node. Otherwise, the node will query the data from a raw table. We simply encode the two conditions to ``1'' and ``0'' without embedding layers.


% \vspace{1mm}
\noindent \textit{\underline{Cardinality}} associated in a node is the number of rows that are qualified as results after the execution of the operations of the sub-plan rooted at the node. We find that it is a vital factor in estimating the cost of an execution plan. %In {\LC}, we input the child nodes' cardinalities and the cardinality of the node itself as an explicit supplement to hidden information. For example, the cardinalities of the input of two child nodes are large, but the cardinality of the current node is small. If the current node cannot obtain this information, the output cost of the current node will be small.


% Compared with neural networks in other applications, the initialization of the cost estimation model should be paid more attention to due to its higher interpretability. For hidden states, we initialize them to 0, as other recurrent networks like RNN and LSTM often do.
% At the same time, the costs we input into leaf nodes are also assigned to 0, which is also in line with our model design. For the join key and physical operations of the leaves, we respectively set a class of "initialization" as additional inputs of the embedding layers for training.

%The initialization of cardinality of the leaf nodes is worth diving into, which is also one of the explicit information we offer to the model. 
% We find the initialization of cardinality of the leaf nodes is vital and worth investigating, which can be divided into three cases: (1) When the current node is not {\it based on} its sibling nodes, which means the current node directly queries from a table, we set its first cardinality input that corresponds to the left child as the count of rows of the table it is querying, and the second that corresponds to the right child as 1; (2) When the query of the current node is {\it based on} its sibling node and only has the filtering predicate, which means the query target of the current node comes from the result of its sibling node, we set its first cardinality as the row count of the result tuples of its sibling node, and the second as 1;  (3) When the query of the current node is {\it based on} its sibling node but has the join predicate, which means the current node will merge the tuples in the sibling node with a table by the join keys, we set its left-child cardinality as the cardinality of its sibling node, and the right one as the row count of the table.
The initialization of leaf node cardinalities is vital and can be divided into three cases: (1) When the current node is not a subquery of its sibling nodes, we set its first cardinality input that corresponds to the left child as the count of rows of the table it is querying, and the second that corresponds to the right child to 1; (2) When the current node is subquery of its sibling node and only has the filtering predicate, we set its first cardinality as the row count of the result tuples of its sibling node, and the second to 1;  (3) When the current node is a subquery of its sibling but has the join predicate, meaning it merges the results in its sibling node with a raw table, we set its left-child cardinality to the cardinality of its sibling node, and the right one to the row count of the table.
% The initialized cardinality of a leaf node usually represents a table stored on disk, while the cardinality of a merge node denotes the row count of tuples in the cache. Nonetheless, with other information initialized differently, the model can learn how to deal with these cardinalities.

%We also normalize the cardinality. Since it is difficult to estimate its maximum value, we simply divide it by a constant to avoid overwhelming other features in the small model. For the inaccuracy of statistical methods in the database system, we proposed a correction method, which will be introduced in Section~\ref{sec:correction}. 

% \noindent \textbf{Filter} is the number of condition in predicates, which has been discussed in Section~\ref{sec:problem}. Although we do not need to represent the entire predicate, it is useful for {\LC} to know how many filters are in the predicates, because the execution times of
% queries with different numbers of filters are different. For example, queries with ``BETWEEN'' (which can be regarded as ``>='' and ``<='') take longer time than queries with only one ``>=''. For example, in the example plan tree in Figure~\ref{fig:modeldesign}, the first, second, and third nodes have 2, 0, and 1 filters respectively.


% \vspace{1mm}
\noindent \textit{\underline{Filter}} is the number of conditions in predicates. To make it clear, in the example plan tree in Figure~\ref{fig:modeldesign}, the first, second, and third nodes have 2, 0, and 1 filters respectively. We find that it is useful for {\LC} to integrate this feature, because the query execution time is positively related to the number of filters used. %For example, queries with ``BETWEEN'' (which can be regarded as ``>='' and ``<='') take longer time than queries with only one ``>=''. 

% For example, there are two filters in ``company\_name.country\_code != '[pl]' AND company\_type.kind = 'production companies' '' while one in ``company\_type.kind = 'production companies' '', which can be directly got in the execution plan node. So the filter feature is the count of the conditions in the predicate.

\noindent \textit{\underline{Join Keys}} denote the attributes that are used when the tuples from two tables are joined. This feature conveys the information related to the execution cost because the join cost can be related to which keys to be joined between the tables. %Due to the difference in the row counts and storage of tables, there will still be a gap in the cost of different joins. We did not adopt embeddings of tables because individual join key provides more granular information. 
We also use an embedding layer to process this feature. For example, in the third node in Figure~\ref{fig:modeldesign} (a), the two join keys {\it t.id} and {\it ci.movie\_id} are converted to a vector. Note that the leaf nodes can also have join keys. Because some scan nodes are subqueries of their sibling nodes and have join predicates as mentioned in the feature \textit{Cardinality}. Similarly, some merge nodes may not have join keys because the join operation is done in one of their child nodes. In this case, we embed the join key as an extra category. 
% For example, there are 30 join keys in the IMDB dataset (including the initialization key for leaf nodes). 
% Also according to the estimation of $\log_2(n)$), we use an embedding layer of size 5.



% is the value we see most often which means ``take in the rows from this operation as input, process them and pass them on''; (2) ``inner'' is only ever seen on the second leaf of join operations, and is always seen there. This is the “inner” part of the loop. i.e., for each outer row, we look up its match according to the first leaf. 


% Since information cannot be passed between leaf nodes, we input this feature at the current node and pass it through the hidden state. It should be noted that there is a special case where some materialization nodes such as ``Hash'' will be marked "inner". However, it actually refers to getting data from its leaf node, not from the previous leaf node. This is one of the reasons why we want to merge nodes: we can use the parent relationship of its leaf node as the attribute of the merged node.



% \vspace{-5pt}
% \subsection{Training}   
% \label{sec:training}

% To train the model, {\LC} collects training samples which are execution plans with ground-truth costs at each node. Compared to previous studies, our model can be trained with much fewer training execution plans. 
% % In this paper, training samples refers to execution plans for training, which contain plan (devised by DBMS) information and actual information. After we collect training samples, we can process and fed them into the model as described in Section~\ref{sec:modeldesign}.
% To generate a training execution plan for real-world databases, the first step is to collect all the potential join keys to form the join graph, where the vertices are tables and each edge connects two joinable tables. Then we randomly select $N$ connected tables from the join graph, in which the distribution of $N$ is determined by the test set. A predicate is composed of columns, operators, and values. Thus, we select each column in the columns that are queried in the test set with uniform probability. Next, we randomly pick all operators that are valid to be applied in numerical or string column.
% % including ``=,<,>,>=,<=,!=,BETWEEN'' for columns with numerical values and ``=,!=,LIKE,NOT LIKE,IN,IS NULL,IS NOT NULL'' for columns with string values. 
% Then the values are also randomly retrieved from the corresponding columns with uniform probability. After obtaining the training queries, we generate execution plans by using the plan analysis tool in the database system. Then we serialize the nodes in the execution plan in post-order for training. 

% In real-world datasets, since randomly generated training queries may lead to skew distribution of results, some generated execution plans will not retrieve any matching tuple from the first node at all. This will result in zero cardinality and the costs in all subsequent nodes approximately being 0. This type of execution plans is repetitive and not useful in training the model, hence we generate extra plans and queries to exclude these zero-cardinality execution plans. 
% % For example, if we want to train 2,000 samples on JOB workloads, we will generate about 20\% more data. Then from the total 2400 data, we extract samples according to whether the cardinality of last node is 0 or not. Finally, we can get 1,800 samples whose last cardinalities are not 0 while the other 200 ones are allowed to have 0 cardinality in the nodes.  
% This practice can not only reduce the training cost, but also improve accuracy by alleviating the gap between the distribution of training and test execution plans due to the random generation.

% Note that if the test set is generated from templates (e.g. TPC-H~), then our training plans are also generated from the templates to ensure that all types of complex queries are included.

% \begin{algorithm}[t]
% \LinesNumbered  
% \SetAlgoVlined
% % 
% \caption{Generate a training plan for real-world database \label{alg:training}}
% \KwIn{Join graph $J$; numerical operator sets $Op^{num}$; string operator sets $O^{str}$; Samples of every tables $S$;}
% \KwOut{A training plan $\mathcal{P}$}
% Randomly select $N$ connected tables $T$ from $J$\;
% Initialize the predicate set $P: {(col_x,op_x,val_x)}$\; 
% Add all the join predicates into $P$\;
% \ForEach{$t_i \in T$}{
%         \ForEach{\text{column} $col_j \in t_i$}{
%             \If {$col_j$ is selected randomly}{
%                 Randomly select $val_j$ from $s_{i,j}$ in $S$\;
%                 \If {$col_j$ is a numerical columns}{
%                     Randomly select $op_j$ from $Op^{num}$\;
%                 }
%                 \Else{
%                     Randomly select $op_j$ from $Op^{str}$\;
%                 }
%                 Add $(col_j,op_j,val_j)$ into $P$
%             }
%             \Else{
%                 continue;
%             }
%         }
% }
% Gather $P$ to form a query $Q$\;
% Input $Q$ to the database system and get $\mathcal{P}$\;
% return $\mathcal{P}$\;
% \end{algorithm}

  
% \begin{figure}[h!]
%   \centering
%   \includegraphics[width=0.9\linewidth]{figure/back.pdf}
%   \caption{Forward propagation and back propagation.{\color{red} where is this figure mentioned?}}
% \label{fig:back}
% \end{figure}
% \noindent \textbf{Initialization.} Compared with neural networks in other applications, the initialization of the cost estimation model should be paid more attention to due to its higher interpretability. For hidden states, we initialize them to 0, as other recurrent networks like RNN and LSTM often do.
% At the same time, the costs we input into leaf nodes are also assigned to 0, which is also in line with our model design. For the join key and physical operations of the leaves, we respectively set a class of "initialization" as additional inputs of the embedding layers for training.

% Cardinality is more special, and is divided into three cases: (1) When the query of the current node does not depend on other nodes, we set its first cardinality 
% as the count of rows of the table it is querying, and the second as 1; (2) When the query of the current node depends on its previous node (usually the first leaf under the same root) and only has the filtering function (the query target of the second node only comes from the result of the first node), we set its first cardinality as the row count of the result tuples of the first node, and the second as 1;  (3) When the query of the current node depends on its previous node but has the join function (will query and match the tuples of the first node from other sources by the join key), we set its first cardinality as the row count of the result tuples of the first node, and the second as the row count of the other source; The initialized cardinality of a leaf node usually represents a table stored on disk, while the cardinality of a merge node denotes the row count of tuples in the cache. Nonetheless, with other information initialized differently, the model can learn how to deal with these cardinalities.

% \vspace{1mm}
% \noindent \textbf{Back propagation.} Although we have added many explicit features during the inference and tree-based structure seems complicated, the computational graph of the model is deterministic and differentiable, which has a well defined gradient. In machine learning, back propagation is a widely used algorithm for training deep neural networks. Similar to forward propagation (see Section~\ref{fig:modeldesign}), we only need to  optimize three sets of weights $V$, $W$ and $U$ besides embedding layers and biases. The back propagation is shown in Figure~\ref{fig:back}. The gradient of  $V$ is related to the current loss $q_i$ and cost $c_i$:
% \begin{displaymath}
% \frac{\partial q_i}{\partial V}=\frac{\partial q_i}{\partial c_i}\frac{\partial c_i}{\partial V}
% \end{displaymath}
% And the gradient of $W$ can be formalized as:
% \begin{displaymath}
% \frac{\partial q_i}{\partial W}=\frac{\partial q_i}{\partial c_i}\frac{\partial c_i}{\partial o_i}\frac{\partial o_i}{\partial W}
% \end{displaymath}
% Since $o_i$ is derived from concatenating the current input $x_i$ and its leaf nodes' states $s_k$ and $s_j$, and costs $c_k$ and $c_j$, its partial derivative of $W$ can be further inferred as:
% \begin{displaymath}
% \begin{aligned}
% \frac{\partial o_i}{\partial W}=&\frac{\partial (W [c_k,s_k,x_i,s_j,c_j])}{\partial W}\\
% =&[c_{k}+W\frac{\partial c_{k}}{\partial o_{k}}\frac{\partial o_{k}}{\partial W},s_{k}+W\frac{\partial s_{k}}{\partial o_{k}}\frac{\partial o_{k}}{\partial W},x_i,\\
% &s_{j}+W\frac{\partial s_{j}}{\partial o_{j}}\frac{\partial o_{j}}{\partial W},c_{j}+W\frac{\partial c_{j}}{\partial o_{j}}\frac{\partial o_{j}}{\partial W}]
% \end{aligned}
% \end{displaymath}
% So $\frac{\partial o_{i}}{\partial W}$ is a recursive term. When $i=0$ or $i \bmod 2 = 1$,that is, when the node is a leaf node, it reaches the end point because the state $s_{-1}$ and cost $c_{-1}$ of the input are initialized.
% Similarly, the partial derivative of $U$ can be derived as:
% \begin{displaymath}
% \frac{\partial q_i}{\partial U}=\frac{\partial q_i}{\partial c_i}\frac{\partial c_i}{\partial o_i}\frac{\partial o_i}{\partial U},
% \end{displaymath}\
% where $\frac{\partial o_{i}}{\partial U}$ is also a recursive term:
% \begin{displaymath}
% \begin{aligned}
% \frac{\partial o_i}{\partial U}=&\frac{\partial (W [c_{k},s_{k},x_i,s_{j},c_{j}])}{\partial U}\\
% =&[W\frac{\partial c_{k}}{\partial o_{k}}\frac{\partial o_{k}}{\partial W},W\frac{\partial s_{k}}{\partial o_{k}}\frac{\partial o_{k}}{\partial W},0,
%  W\frac{\partial s_{j}}{\partial o_{j}}\frac{\partial o_{j}}{\partial U},W\frac{\partial c_{j}}{\partial o_{j}}\frac{\partial o_{j}}{\partial W}]
% \end{aligned}
% \end{displaymath}
% In addition, the gradient of the embedding layer is also easy to calculate, just continue to chain the derivative in the back direction. As the depth of the tree increases, the gradient brought by the last node becomes smaller. And since the last node will give the final estimation directly, we will also adjust its $\lambda$ in Equation~\ref{equ:3}.

% \vspace{1mm}
% \noindent \textbf{Batch Training.} Batch training is a technique that combines forward propagation and backward propagation of several training plans into one to accelerate the training or inference process~\cite{ruder2016overview}. Forward propagation refers to calculating the intermediate variables of the model from input to output, and backward propagation is the process of computing the gradient of the network from output to input, both of which are classical and effective methods~\cite{rumelhart1995backpropagation}. However, because we calculate the loss of all nodes in each iteration, it is already equivalent to batch training in the way of backward propagation. We found in our experiments that if we continued to use the batch technique, it was difficult for our model to converge in expected iterations. This is because in batch training the gradient is averaged over the entire batch, so the process of descent is slower. In the inference, due to great diversity in the number of nodes in real-world plan trees, lots of padding needs to be adopted, so the acceleration is also not obvious. However, for possible further research, we still provide a batch version of {\LC}. First, we sample $N$ plans with similar depth in the training data to avoid excessive padding which will reduce efficiency. Let $D$ denote the max depth of the training execution plans, then we rearrange the training execution plans as a $D*N$ tensor, in which each column is a sequence of nodes in the post order of the plan tree and short plans will be padded with 0. This way every time we infer the {\LC} model, we're going to input a $N$-dimensional tensor instead of 1 which theoretically reduces training time by $N$ times.
\vspace{-4mm}
\subsection{Cardinality Calibration}
\label{sec:correction}
The cardinality of executing a sub-plan (i.e., a sub-tree in the plan tree) is an important feature for estimating the total plan execution cost. As the actual cardinality corresponding to each node in the plan tree is not known beforehand, an {\it estimated cardinality} is often used. Among the learning-based methods employing such an approach, there are two issues. First, the classic histogram-based cardinality estimator embedded in the common DBMS can be vulnerable, rendering the final estimate of the execution cost unreliable. Second, these methods often require a large number of training samples to fit the model. %The work of~\cite{marcus2019plan,wu2013predicting} tries to leverage the original cardinality predictor embedded in a database system (e.g., PostgreSQL) and learn to map the prediction to the actual execution costs. %These methods require a large number of training samples to fit the two distributions that differ widely and the accuracy of cost estimates will be hurt. To alleviate this problem, in this part, we will present our correction method of cardinality. Since our method is a modified version of the histogram-based method, we first briefly introduce the histogram-based cardinality models in DBMS.
%Unfortunately, these methods require a large number of training samples to fit the model. 
To alleviate these issues, we present a simple but effective cardinality calibration technique based on the classic histogram-based model. 

We first briefly introduce the histogram-based cardinality models adopted in common databases.
% The main idea is to transform cardinality estimation to selectivity estimation, where the {\it selectivity} of a table is defined as the percentage of the rows that are qualified for the predicates applied in the table. Particularly, given an SQL query, we let %$a_i$ denote the $i$-th column queried in the SQL query and 
% $\theta_i$ denote the $i$-th predicate (i.e., filtering predicate or join predicate) for the $i$-th table. The {\it selectivity} regarding the $i$-th table is the normalized form of the cardinality, formalized as:
% \vspace{-2mm}
% \begin{equation}
% \label{equ:sel-card}
% %\begin{displaymath}
% sel(\theta_i)=card(\theta_i)/|T|,
% %\end{displaymath}
% \end{equation}
% where $|T|$ denotes the total row count of the table. Since $sel(\theta_i)$ can be estimated by a histogram-based method based on several statistical techniques~\cite{postgresql1996postgresql}, the cardinality for the node corresponding to the $i$-th table can be estimated by $sel(\theta_i)\times |T|$. 
\textcolor{black}{The main idea is to first estimate the row count of each predicate by a histogram-based method based on several statistical techniques~\cite{postgresql1996postgresql}. Then the cardinality of a merge node can be estimated by the {\it product} of the cardinalities under its 
sub-predicates, and this propagates upward along the tree to estimate the cardinality of the root node of the whole plan. We note that the {\it product} is used based on the assumption that the predicates applied in different tables are independent. }
% 
%
% \begin{equation}
% \label{equ:indepent-card}
% card(\theta)=\Pi_{1\leq i\leq n} card(\theta_i).
% \end{equation}
% The cardinality of each merge node can be estimated by the product of two cardinalities under two predicates $\theta$, and this propagates upward along the tree to estimate the cardinality of the root node of the whole plan. 
% Note that the cardinality of a merge node is not necessarily equal to the product of the cardinality of its two children when the children are not {\it based on} each other. This is because in this case, one of the child nodes does not go through the join predicate. % We note that Equation~\ref{equ:indepent-card} is used based on the assumption that the predicates applied in different tables are independent. 


% :
% \begin{equation}
% \label{equ:1}
% card(\theta)=\Pi_{1\leq i\leq n} sel(a_i,\theta_i).
% \end{equation}
% Then, the overall cardinality can be estimated by $sel(a,\theta)\times |T|$.



% {\color{blue} 
% Particularly, given an SQL query, and we let $\theta_i$ denote the set of predicates that query the $i$-th table queried in the SQL (e.g. {\it title.production\_year $\geq$ 1980} and {\it title.production\_year $\leq$ 2010} are regarded as one set of predicates). The final cardinality of a query is calculated as the product of the cardinalities of each set of predicates:
% \begin{equation}
% \label{equ:indepent-card}
% card(\theta)=\Pi_{1\leq i\leq n} card(\theta_i).
% \end{equation}
% }

% \vspace{-1mm}
\begin{figure}[h]
\vspace{-4mm}
\centering
  \includegraphics[width=0.85\linewidth]{figure/calib.pdf}
  \vspace{-3mm}
  \caption{The process of cardinality calibration.}
\label{fig:correct}
\vspace{-4mm}
\end{figure}
% The major advantage of the histogram-based methods is efficient with no training and extreme low model inference latency. Existing learning-based cardinality estimators not only require extra resources to train, but also have high model inference latency. For tree-structure cost estimators, the more serious problem is that due to the independence of some leaf nodes, the cardinality model must be inferred at almost every leaf nodes, which will impose significant long model inference latency. Whereas estimated cardinalities in database systems can be used directly, but are often ignored because of the inaccuracy. 

The histogram-based method for cardinality estimation is known to be efficient as it requires no training. The downside is its low estimation accuracy for cardinalities. If we directly use this estimator for estimating the execution cost, we may suffer from low accuracy as well. %In contrast, existing learning-based cardinality estimators not only require extra resources to train, but also have a high model inference latency. As a return, they often have higher estimation accuracies. 
{\it We argue that it is possible to apply a simple calibration of the classic cardinality estimator so that the cardinality estimation accuracy can be enhanced and becomes suitable to contribute positively to the cost estimator.} %Taking tree-structure cost estimators as an example, the more serious problem is that due to the independence of some leaf nodes, the cardinality model must be inferred at almost every leaf nodes, which will impose significant long model inference latency. Whereas estimated cardinalities in database systems can be used directly, but are often ignored because of the inaccuracy. 

% To prove that the cardinalities given by histogram is not useless, we have sampled some execution plans in the JOB workload. As shown on the left side of Figure~\ref{fig:planrows}, we find that the gap between the two is indeed huge. But when we calculate the cosine similarity, the result shows different views. The cosine similarity can be formalized as:
\begin{figure}[t]
  \centering
  \includegraphics[width=0.9\linewidth]{figure/plan-actual.pdf}
  \vspace{-4mm}
  \caption{Q-error of vanilla estimated cardinalities vs. calibrated cardinalities on JOB-M workload.}
\label{fig:plan-actual}
\vspace{-6mm}
\end{figure}
Our main insight is that there is a strong correlation between the estimated cardinality of histogram-based methods and the actual cardinality. However, the accuracy of a histogram-based method is sometimes disrupted by disastrous estimates. As shown in Figure~\ref{fig:plan-actual}, we observe a significant gap between the estimated and actual cardinalities, but we do not that their cosine similarity is more than 0.90, which indicates a strong positive correlation between the two sets of cardinalities. 

%While the strong correlation shows that the mapping between the cardinality and execution cost is {\it learnable}, it may still suffer from disastrous estimates. 
The strong correlation indicates that the estimated cardinality can be calibrated. Serious errors that occur at merge nodes whose two children are both scan nodes often propagate to the remaining nodes. For example, Node {\it n} in Figure~\ref{fig:correct} is a merge node of this type, which are referred to as {\it lowest-level merge nodes}. The errors at {\it lowest-level merge nodes} propagate because the final estimate is the product of the cardinality estimates of all corresponding nodes. This also enlightens us that calibrating these rooted errors is more cost-effective. Therefore, we propose to {\it 
calibrate} the cardinality in these {\it lowest-level merge nodes} to avoid propagating the errors to the final estimate.
% Let us take an extreme example to illustrate this. 
%
% So the cardinality of a current node is obtained by multiplying the cardinalities of its two child nodes. 
%
% To explain, we first briefly introduce how the histogram helps in estimating cardinalities. Let $a_i$ denote the $i$-th column queried in the SQL and $\theta_i$ denote the $i$-th predicate (i.e., filtering conditions). To simplify the formulation, we introduce the {\it selectivity}, which can be regarded as the normalized form of the cardinality, and it can be formalized as:
% \begin{displaymath}
% sel(a_i,\theta_i)=card(a_i,\theta_i)/|T|,
% \end{displaymath}
% where $|T|$ denotes the total row count of the table. Based on the histogram method, the selectivity of a query is calculated as the product of the selectivity of each predicate:
% \begin{equation}
% \label{equ:1}
% sel(a,\theta)=\Pi_{1\leq i\leq n} sel(a_i,\theta_i).
% \end{equation}
% In this equation, each table and column is assumed to be independent for simplicity. 
% {\color{blue} For example, if the cardinality of the first scan node is 0, according to the Equation~\ref{equ:1}, the estimated cardinalities of its ancestor nodes are all 0. It is an extreme bad case for histogram method, and it will hurt the cost estimates badly.}
% To avoid a node with 0 cardinality leading to a ``zero'' plan, for example, PostgreSQL add an extra constant when $const$calculating each single selectivity. For histogram method, if the value $val$ of the predication fall into the $j$th bucket $bkt$, then we can calculate the selectivity as: 
% \begin{displaymath}
% sel(a_i,\theta_i)=\frac{1}{num(bkt)} \left[  const + \frac{val - min(bkt_j)}{max(bkt_j) - min(bkt_j)} \right]
% \end{displaymath}
% This method is acceptable for computing continuous numerical columns. For queries on columns which have more common values, the list of most common values (MCVs) is used to determine the selectivity. The database gather the statistics about the $n$ values with the highest probability of occurrence and stores their probability in the $mcv$ list. If the value in the predicate exists in $mcv$, it can be taken directly from it; If not, the selectivity is calculated as:
% \begin{displaymath}
% sel(a_i,\theta_i)=\frac{const-\sum_j^nmcv_j}{num(dst_i)-n},
% \end{displaymath}
% where $dst_i$ denotes the all distinct values in $i$th column.
% The real problem comes when the two sets of tuples are joined by index. Because all the values in index column are unique, so the database system can only use an equation that relies only on the number of distinct values for both relations together with their null fractions:
% \begin{displaymath}
% \begin{split}
% sel(a_i,\theta_i)*sel(a_{i+1},\theta_{i+1})=(const-null)(const-null) \\
% \times min(\frac{1}{num(dst_i)},\frac{1}{num(dst_{i+1})}),
% \end{split}
% \end{displaymath}
% where $null$ means null fractions usually assigned 0. For nodes using index, this method will get the same results regardless of the input tuples.
% \subsection{Correction of Cardinality}
% \label{sec:sample}

%It is not realistic to improve the estimates of each node due to huge computational complexity, so we need a cost-effective method. It can be seen from the above case that if there is a serious error in the cardinality estimates in the first merge node, the error will be accumulated in subsequent calculations. Moreover, we should avoid excessive high latency caused by the correction process. So correcting the cardinalities of first merge node in each execution plan by querying the samples of its actual result is cost-effective. We propose a sampling method to achieve it. 

% \vspace{1mm}
\noindent
{\bf Calibration Procedure.} Our method is based on sampling. First, in preprocessing we sample rows from the inner join results of each pair of tables with common attributes. The samples are stored in a {\it lookup list} and we set a low sample rate to store it in the main memory. Let us denote the sample rate as $\frac{1}{p}$ for some $p>1$. The sampling can be done in parallel to generate training plans. In {\LC}'s training and inference, we can query the corresponding lookup list to get a second cardinality estimate of the lowest-level merge node in the plan tree besides of vanilla estimate from DBMS. Particularly, we filter the samples in the lookup list with the predicate of the lowest-level merge node and get the qualified row counts, $c$. The second estimate for the merge node is then calculated as $c\cdot p$. If $\tilde{c}$ is the 
vanilla cardinality estimate, then the bias factor is $\frac{c\cdot p+{p}}{\tilde{c}+p}$.
%
% \begin{equation}
% \label{equ:bias}
% bias=\frac{c\cdot p+{p}}{\tilde{c}+p}
% \end{equation}
%
In the numerator and denominator we both add a value $p$, which is the inverse of the sample rate. This helps counter the error that can be introduced when the sample rate is small, as it reduces the weight given to $c$.
% Next we calculate the bias between the corrected cardinality and the original plan cardinality given by the database system. The bias is the ratio of the corrected cardinality and original cardinality. Then we propagate the bias to all nodes based on the first merge node by multiplying the original cardinality by the bias. 

% Let $p(a_0)=sel(a_0,\theta_i)$ and $p(a_0)=sel(a_0,\theta_i)$, then 


% According to Equation~\ref{equ:1}, the cardinality of the first merge node estimated by the database system is 
% % \begin{displaymath}
% % sel_d(a_0,a_1,\theta_0,\theta_1)=sel(a_0,\theta_0)sel(a_1,\theta_1)
% % \end{displaymath}
% \begin{displaymath}
% card_d(\theta_0,\theta_1)=card(\theta_0)card(\theta_1)
% \end{displaymath}
% So the actual meaning of the bias is:
% % \begin{equation}
% % \label{equ:2}
% % bias=\frac{sel_a(a_0,a_1)}{sel_d(a_0,a_1)} = \frac{sel((a_0,\theta_0)|(a_1,\theta_1))}{sel(a_1,\theta_1)}
% % \end{equation}
% \begin{equation}
% \label{equ:2}
% bias=\frac{card_a(\theta_0,\theta_1)}{card_d(\theta_0,\theta_1)} = \frac{card(\theta_0|\theta_1)}{card(\theta_1)}
% \end{equation}

% \begin{algorithm}[t]
% \LinesNumbered  
% \SetAlgoVlined
% % 
% \caption{Cardinality Correction from a merge node \label{alg:card}}
% \KwIn{Sub-plan nodes $N$; Vanilla estimated cardinalities $V$; Lookup lists $L$;}
% \KwOut{Corrected estimated cardinalities $C$}
% Find a {\it low-level merge node} $n_x$ from $N$\;
% Query the cardinality $c_x$ of $n_x$ from $L$\; 
% Calculate the $bias$\;% based on Equation~\ref{equ:bias}\;
% \ForEach{$n_i \in N$}{
% Draw the corresponding vanilla est. card. $v_i$ from $V$\;
%     \If{$n_i$ is the ancestor of $n_x$}{
%         $c_i = bias \times v_i$;
%     }
%     \ElseIf{$n_i$ is the sibling of $n_x$ or the sibling of ancestors of $n_x$ or the child of $n_x$}{
%          \If{$n_i$ is related with $n_x$}{
%             $c_i = bias \times v_i$;
%          }
%          \Else{
%             $c_i = v_i$;
%          }
%     }
%     \ElseIf{$i \neq x$}{
%         $c_i = v_i$;
%     }
% }
% $C = \{c_i\}$\;
% return $C$\;
% \end{algorithm}

% \begin{algorithm}[t]
% \LinesNumbered  
% \SetAlgoVlined
% % 
% \caption{Cardinality Correction from a merge node \label{alg:card}}
% \KwIn{Sub-plan nodes $N$; Vanilla estimated cardinalities $V$; Lookup lists $L$;}
% \KwOut{Corrected estimated cardinalities $C$}
% Find a {\it low-level merge node} $n_x$ from $N$\;
% Query the cardinality $c_x$ of $n_x$ from $L$\; 
% Calculate the $bias$\;% based on Equation~\ref{equ:bias}\;
% \ForEach{$n_i \in N$}{
% Draw the corresponding vanilla est. card. $v_i$ from $V$\;
%     \If{$n_i$ is related of $n_x$}{
%         $c_i = bias \times v_i$;
%     }
%      \Else{
%         $c_i = v_i$;
%      }
    

% }
% $C = \{c_i\}$\;
% return $C$\;
% \end{algorithm}


Figure~\ref{fig:correct} illustrates the calibration process. In Node {\it n} there is a large estimate error (200 vs. 20k), causing an unacceptable error (500 vs. 50k) finally. To address the issue, during inference or training, our calibration method identifies Node {\it n} as a {\it lowest-level merge node} and inspects its lookup list. Then, we query the sampled lookup list and obtain a value of 239. Suppose that the sample rate is $\frac{1}{100}$, i.e., $p=100$. Then, the $factor$ is computed to be $\frac{239\times 100+{100}}{200+100}=80$. The calibration will be propagated to other nodes related to Node {\it n}.

% For example, in the left of Figure~\ref{fig:correct}, we have a plan tree with original and actual cardinalities, whose cardinality of the root node differs from the actual value by more than 100 times (6k vs. 700k). According to the steps of cardinality correction, in the right of Figure~\ref{fig:correct}, first we collect the samples and save the lookup lists. In inference or training, when we meet the first Nested Loop node (the third node in the  Figure~\ref{fig:correct}), we query from the corresponding look lists and get a corrected cardinality ``30k'', which is pretty closer to the real cardinality ``30k''. Then we can get bias by dividing the two value and get 96. In practice, the corrected cardinality may fluctuate around the actual ones due to random sampling. So to avoid the huge errors caused by very little actual cardinalities, we have added a constant to the numerator and denominator respectively. For instance, in IMDB dataset, we sample each join result by a ratio of one percent considering the inference efficiency. So if the real cardinality is 1, we might get the corrected rows to be 20. Then if we calculate the bias directly and get 20, we will impose an irrational factor on subsequent cardinalities. As the fluctuation will increase as the sampling rate decreases, we refer to the inverse of sample rate and assign this constant as 100 in this case. 

%After the calculation of the bias, we propagate it to the nodes related with the 3rd node. 
%We need to further find out whether other nodes are related to the 3rd node to complete the correction. 
To identify the related nodes of the Node {\it n}, the following conditions are used: First, all ancestor nodes of Node {\it n} are related to it, because their results come from it; Second, the sibling nodes of Node {\it n} and its ancestor nodes can be identified based on the {\it Subquery} (see Section~\ref{sec:feature}). If the node is the {\it Subquery} of Node {\it n} or its ancestors, it is relevant. Finally, the two child nodes of the Node {\it n} can also be distinguished by {\it Subquery} as the second condition. The estimated cardinalities of the related nodes are then calibrated by multiplying the factor, while the estimated cardinalities of unrelated nodes are left unchanged. This calibration improves the overall Q-error, as shown in Figure~\ref{fig:plan-actual}.%{\color {blue} Note that if the two child nodes of Node 3 are not {\it based on} each other, the estimated cardinality of the Node 3 is not necessarily equal to the product of the cardinalities of its two child nodes. The reason is that the estimated cardinality of one of the nodes has not gone through the join predicate, whereas the join predicate is applied on Node 3.}

\noindent
\textcolor{black}{\textbf{Remarks.} Compared with fully sampling-based cardinality estimators, each lookup list in cardinality calibration only includes the merge results of two tables, which avoids huge memory overhead. Compared with other sampling strategies embedded in ML-based models, such as sampling bitmaps~\cite{wu2021unified,sun13end}, our cardinality calibration provides more precise and explicit information, which is more efficient for a lightweight model without extra embeddings of predicates.}
%\vspace{-2mm}
%
% In this example, Node 2, Node 5, and Node 7 are related to Node 3. Hence, they are corrected by multiplying the bias computed by Equation~\ref{equ:bias}. Other nodes are not related to Node 3, so they keep their vanilla estimated cardinalities. We summarize the procedure in Algorithm~\ref{alg:card}. %{\color {blue} We note that we only start the cardinality correction on the nodes whose child nodes are both leaf nodes, such as Node 3. If there are multiple such nodes in one execution plan (rare in database systems), our correction starts from each of such nodes and propagation the biases respectively.}
% The remaining two merge nodes (the 5th and 7th node in Figure~\ref{fig:correct}) are based on the third node. The first Seq Scan (the first node in Figure~\ref{fig:correct}) is not based on the third node, so it is not corrected. In addition, the second Index Scan and the second Seq Scan nodes (the 4th and 6th node in Figure~\ref{fig:correct}) are independent of the third node, which means they do not need to be corrected. The first Index Scan node is based on the first Seq Scan node, whose cardinality is expected to be $p(a_1|a_0)$ while it is assigned as $p(a_1)$ in the database system. So it can be corrected by multiplying the bias according to Equation~\ref{equ:2}.

%Although this method cannot completely solve the problem of inaccurate predictions, it can effectively alleviate extreme errors. As shown on the right in Figure~\ref{fig:correct}, several plans with original cardinalities of almost 0 were corrected in this way. In the other hand, other several plans with abnormally high rows estimations were also improved. Quantitatively, the cosine similarity between plan and actual rows increased from 0.90 to 0.95. Using this method makes it easier for the model to learn the mapping from actual cardinality to plan cardinality.

% \noindent
% {\bf Inference Complexity.} We assume that the average row count of the join result of every two joinable tables is $m$, and the sample rate is $\frac{1}{p}$, then the complexity of querying one lookup list is $\mathcal{O}(\frac{m}{p})$. Let $n$ denote the number of nodes in the execution tree, then the total complexity of the correction procedure is about $\mathcal{O}(\frac{m}{p}+n)$. The propagation time of the $bias$ ($\mathcal{O}(n)$) is negligible compared to the querying time.  Compared with calculating the sampling cardinalities of each merge node (approximately $\mathcal{O}(\frac{m}{p} \times \frac{n}{2})$), our proposed cardinality correction reduces the inference time by about $\mathcal{O}(\frac{n}{2})$ times.

% \noindent
% {\bf Remarks.} We note that the purpose of the correction is not to fully eliminate the inaccuracies. Instead, we aim to remove the disastrous estimates that would finally lead to a significant bias of the actual cost estimates. As shown in Figure~\ref{fig:plan-actual}, the corrected estimated cardinalities are much closer to the actual cardinalities than the vanilla ones. We find that such a correction procedure can already produce satisfactory performance when it is incorporated into our model introduced in Section~\ref{sec:modeldesign}.
We present in section~\ref{ssec:faces} an application of PnP-HVAE on face images, using a pretrained state-of-the-art hierarchical VAE. 
Next, we study the application of our framework to natural images. To that end, we introduce  in section~\ref{ssec:patchVDVAE}  a patch hierachical VAE architecture, that is able to model natural images of different resolutions. In section~\ref{ssec:app_nat}, we provide deblurring, super-resolution and inpainting experiments to demonstrate the relevance of the proposed method.

Additional results are presented in Appendix~\ref{app:add}. All experiments can be reproduced using the code available at \url{https://github.com/jprost76/PnP-HVAE}.



\subsection{Face Image restoration (FFHQ)}\label{ssec:faces}
We first demonstrate the effectiveness of PnP-HVAE on highly structured data, by performing face image restoration.
Latent variable generative models can accurately model structured images such as face images \cite{karras2019style,vahdat2020nvae,child2021very,kingma2018glow}, and then be used to produce high quality restoration of such data. 
In our experiments, we use the VDVAE model of~\cite{child2021very}, pre-trained on the FFHQ dataset~\cite{karras2019style}, as our hierarchical VAE prior.
VDVAE has $L=66$ latent variable groups in its hierarchy and generates images at resolution $256\times256$.

We compare PnP-HVAE with the intermediate layer optimization algorithm (ILO)~\cite{daras2021intermediate} that is based on a different class of generative models than HVAE. ILO is a GAN inversion method which optimizes the image latent code along with the intermediate layer representation of a StyleGAN to generate an image consistent with a degraded observation.
We use the official implementation of ILO, along with a StyleGAN2 model~\cite{karras2020analyzing, stylegan2pytorch}, that was trained for 550k iterations on images of resolution $256\times256$ from FFHQ.  
As VDVAE and StyleGAN models are not trained on the same train-test split of FFHQ, we chose to evaluate the methods on a subset of 100 images from the CelebA dataset~\cite{liu2018large}. 
For super-resolution, the degradation model corresponds to the application of a gaussian low-pass filter followed by a $\times 4$ sub-sampling, and the addition of a gaussian white noise with $\sigma=3$.
For the deblurring, we considered motion blur and  gaussian kernels, both with a noise level $\sigma=8$. %

We provide quantitative comparisons in table~\ref{table:comp_ILO}, along with a visual comparison of the results in figure~\ref{fig:face_restoration}.
PnP-HVAE has the best  PSNR and SSIM results for all the considered restoration tasks, while ILO provides better results  for the perceptual distance.
By jointly optimizing the image and its latent variable, PnP-HVAE provides  results that are both realistic and consistent with the degraded observation.
On the other hand,  ILO  only optimizes on an extended latent space. This method generates  sharp and realistic images with better LPIPS scores,   
but the results lack  of consistency with respect to the observation, which explains the overall lower PSNR performance. 






\subsection{PatchVDVAE: a HVAE for natural images}\label{ssec:patchVDVAE}
Available generative models in the literature operate on images of  fixed resolutions and
are either restrained to datasets of limited diversity, or even to registered face images~\cite{kingma2018glow,child2021very, vahdat2020nvae, karras2019style}, or requiring additional class information~\cite{brock2018large, dhariwal2021diffusion, song2020score, luhman2022optimizing}.
Fitting an unconditional model on natural images appears to be a more difficult task, as their resolution can change, and their content is highly diverse.
The complexity of the problem can be reduced by learning a prior model on patches of reduced dimension. 
For image restoration problems, the patch model can be reused on images of higher dimensions~\cite{zoran2011learning,prost2021learning,altekruger2022patchnr}. When the model is a full CNN, the prior on the set of the  patches can  be computed efficiently by applying the network on the full image~\cite{prost2021learning}.

We thus introduce  patchVDVAE, a fully convolutional hierarchical VAE.
Contrary to existing HVAE models whose resolution is constrained by the constant tensor at the input of the top-down block, patchVDVAE can generate images of different resolutions by controlling the dimension of the input latent. 
This amounts to defining a prior on patches whose dimension corresponds to the receptive field of the VAE. A similar model is used for image denoising in~\cite{prakash2021interpretable}.

 
For PatchVDVAE architecture, we use the same bottom-up and top-down blocks as VDVAE~\cite{child2021very}, and replace the constant trainable input in the first top-down block by a latent variable, to make the model fully convolutional (details on the  architecture are given in Appendix~\ref{app:details}). 
The training dataset is composed of $128\times 128$ patches extracted from a combination of DIV2K~\cite{agustsson2017ntire} and Flickr2K~\cite{Lim_2017_CVPR_workshops} datasets.
We perform data augmentation by extracting  patches at $3$ resolutions: HR-images and $\times 2$ and $\times 4$ downscaled images. 
The model is trained for $7.10^5$ iterations with a batch size of $64$. Following the recommendation of~\cite{hazami2022efficient}, we use Adamax optimizer with an exponential moving average and gradient smoothing of the variance.
We set the decoder model to be a gaussian with diagonal covariance, as in~\cite{luhman2022optimizing}.
PatchVDVAE is fully convolutional and can generate images of dimension that are multiples of $64$ as illustrated by
figure~\ref{fig:vdvae}.

\newlength{\patchwidth}
\setlength{\patchwidth}{0.135\columnwidth}
\begin{figure}[!ht]
    \centering
    \begin{subfigure}[t]{.34\columnwidth}\hspace{0.1cm}
        \setlength{\tabcolsep}{0.02pt}
\renewcommand{\arraystretch}{0}
        \begin{tabular}{*{2}{p{1.03\patchwidth}}}
            \includegraphics[width=\patchwidth]{figures_arxiv/patchVDVAE/samples/generated/64x64/setup-5-image-0018.png} &
            \includegraphics[width=\patchwidth]{figures_arxiv/patchVDVAE/samples/generated/64x64/setup-5-image-0016.png} \\
            \includegraphics[width=\patchwidth]{figures_arxiv/patchVDVAE/samples/generated/64x64/setup-5-image-0008.png} &
            \includegraphics[width=\patchwidth]{figures_arxiv/patchVDVAE/samples/generated/64x64/setup-5-image-0019.png}   
        \end{tabular}
    \end{subfigure}\hspace{-0.15cm}
    \begin{subfigure}[t]{.64\columnwidth}
\begin{tabular}{cc}\vspace{-0.1cm}
\includegraphics[width=2\patchwidth]{figures_arxiv/patchVDVAE/samples/generated/256x256/setup-2-image-0009.png}&
        \includegraphics[width=2\patchwidth]{figures_arxiv/patchVDVAE/samples/generated/256x256/setup-2-image-0002.png}\end{tabular}

    \end{subfigure}
    \caption{\label{fig:vdvae} Left: $64\times64$ patches samples from our patchVDVAE model trained on patches from natural images.
    Right: PatchVDVAE is fully convolutional and it can generate images of higher resolution (here: $128\times128$).\vspace{-0.2cm}}
\end{figure}

\subsection{Natural images restoration}\label{ssec:app_nat}
We  evaluate PnP-HVAE on natural image restoration.
For each task, we report the average value of the PSNR, the SSIM, and the LPIPS metrics on $20$ images from the test set of the BSD dataset~\cite{MartinFTM01}.\\


\noindent
{\bf Image deblurring.}
In the experiments, we consider $2$ gaussian kernels and $2$ motion blur kernels from~\cite{levin2009understanding}, with $3$ different noise levels 
$\sigma \in \{2.55, 7.65, 12.75\}$.
As a baseline we consider  EPLL~\cite{zoran2011learning}, which learns a prior on image patches with a gaussian mixture model.
We also compare PnP-HVAE  with PnP-MMO and GS-PnP, $2$ competing convergent Plug-and-Play methods based on CNN denoisers.
PnP-MMO~\cite{pesquet2021learning} restricts the denoiser to be contraction in order to guarantee the convergence of the PnP forward-backard algorithm. GS-PnP~\cite{hurault2022gradient} considers a gradient step denoiser and reaches state-of-the-art performances of non converging methods~\cite{zhang2021plug}.
We set the temperature $\tau$  in our method as $0.95$, $0.8$ and $0.6$ for noise levels $2.55$, $7.65$ and $12.75$ respectively, and we let it run for a maximum of $50$ iterations. 
For the three compared methods we use the official implementations and pre-trained models provided by the respective authors. 
Details on the choice of hyperparameters for the concurrent methods are provided in the Appendix~\ref{app:details}
Figure~\ref{fig:deblurring_bsd} illustrates that our method provides correct deblurring results. 

According to table~\ref{tab:deb}, the performance of PnP-HVAE is between those of EPLL and GS-PnP and it outperforms PnP-MMO for large noise levels.\\

\begin{table}
\begin{center}\footnotesize
    \begin{tabular}{>{\centering}m{.3cm}*{5}{c}}
    $\sigma$ &Method & PSNR$\uparrow$ & SSIM$\uparrow$ & LPIPS$\downarrow$  \\ 
    \hline
    \multirow{4}{*}{\vcell{$2.55$}}
    & PnP-HVAE & $27.75$ & $0.79$ & $0.31$\\
    & GS-PNP \cite{hurault2022gradient} & $\mathbf{29.59}$ & $\mathbf{0.84}$ & $\mathbf{0.22}$\\
    & EPLL \cite{zoran2011learning} & $26.49$ & $0.71$ & $0.36$\\ 
    & PnP-MMO \cite{pesquet2021learning} & $\underbar{29.50}$ & $\underbar{0.83}$ & $\underbar{0.20}$ \\ \hline
    \multirow{4}{*}{\vcell{$7.65$}}
    & PnP-HVAE & $\underbar{26.36}$ & $\underbar{0.72}$ & $\underbar{0.40}$\\
    & GS-PNP \cite{hurault2022gradient} & $\mathbf{27.33}$ & $\mathbf{0.77}$ & $\mathbf{0.31}$\\
    & EPLL \cite{zoran2011learning} & $24.04$ & $0.66$ & $0.45$ \\ 
    & PnP-MMO \cite{pesquet2021learning} & $25.34$ & $0.69$ & $0.34$\\
    \hline
    \multirow{4}{*}{\vcell{$12.75$}}
    & PnP-HVAE & $\underbar{25.12}$ & $\mathbf{0.73}$ & $\underbar{0.47}$\\
    & GS-PNP \cite{hurault2022gradient} & $\mathbf{26.32}$ & $\mathbf{0.73}$ & $\mathbf{0.37}$\\
    & EPLL \cite{zoran2011learning} & $23.28$ & $0.61$ & $0.51$ \\ 
    & PnP-MMO \cite{pesquet2021learning} & $22.42$ & $0.53$& $0.54$ \\
    \hline
    &\vspace*{-.3cm}\\
            \multicolumn{2}{c}{Blur and motion kernels}& \multicolumn{3}{c}{
        \includegraphics*[scale=1]{figures_arxiv/kernels/4.png}\;\includegraphics*[scale=1]{figures_arxiv/kernels/7.png}\;\includegraphics*[scale=1]{figures_arxiv/kernels/9.png}\;\includegraphics*[scale=1]{figures_arxiv/kernels/11.png}} 
    \end{tabular}
        \caption{\label{tab:deb}Comparison  of PnP-HVAE  and other restoration methods on deblurring. Results are averaged on $4$ kernels.\vspace{-0.2cm}}% on image deblurring.}
    \end{center}
\end{table}

\begin{figure}
    
    \begin{subfigure}[h]{\linewidth}
        \centering
        \includegraphics*[width=\columnwidth]{figures_arxiv/deb_s255_k7.pdf}\vspace{-0.1cm}
        \caption{Gaussian blur, $\sigma=2.55$}
    \end{subfigure}
    \begin{subfigure}[h]{\linewidth}
        \centering
        \includegraphics*[width=\columnwidth]{figures_arxiv/deb_s765_k11.pdf}\vspace{-0.1cm}
        \caption{Motion blur, $\sigma=7.65$}
    \end{subfigure}\vspace*{-0.1cm}
    \caption{\label{fig:deblurring_bsd} Natural image deblurring\vspace{-0.1cm}}
\end{figure}

\noindent {\bf Effect of the temperature.}
PnP-HVAE gives control on the temperature of the prior over the latent space.
In figure~\ref{fig:temp_effect}, we illustrate that reducing the temperature increases the strength of the regularization prior. In this example the tuning $\tau=0.7$ produces the best performance.\\
\begin{figure}[!ht]
   
    \includegraphics[width=\columnwidth]{figures_arxiv/demo_temp.pdf}\vspace{-0.15cm}
    \caption{ \label{fig:temp_effect} Effect of the temperature in PnP-VAE on a deblurring problem, with $\sigma=7.65$.\vspace{-0.15cm}}
\end{figure}


\noindent
{\bf Image inpainting.}
Next we consider the task of noisy image inpainting. 
We compose a test-set of 10 images from the validation set of BSD~\cite{MartinFTM01} and we create masks
  by occluding diverse objects of small size in the images. 
A gaussian white noise with $\sigma=3$ is added to the images.
As a comparaison, we still consider GS-PnP and EPLL.
For PnP-HVAE, the temperature is set to $\tau=0.6$, and the algorithm is run for a maximum of $200$ iterations, unless the residual $||\x_{k+1}-\x_k||$ is on a plateau.
We provide on Table~\ref{tab:inpainting_bsd} the distortion metrics with the ground truth, as well as a visual
\begin{table}



\begin{center}
    \begin{tabular}{cccc}
        & PSNR$\uparrow$ & SSIM$\uparrow$ &LPIPS$\downarrow$ \\\hline
        PnP-HVAE  & $\mathbf{29.54}$ & $\mathbf{0.93}$ & $\mathbf{0.06}$\\
        GS-PNP & $28.52$ & $\mathbf{0.93}$ & $0.09$\\
        EPLL & $\underline{29.16}$ & $\mathbf{0.93}$ & $\mathbf{0.06}$\\
    \end{tabular}
    \caption{\label{tab:inpainting_bsd}Quantitative evaluation for inpainting on BSD.}
    \end{center}
\end{table}
comparison on figure~\ref{fig:inpainting_bsd}. 
With its hierarchical structure,  PnP-HVAE outperforms the compared methods. \vspace{0.05cm}



\begin{figure}[!h]
    \includegraphics[width=\columnwidth]{figures_arxiv/demo_inp_bsd2.pdf}\vspace{-0.1cm}
    \caption{\label{fig:inpainting_bsd}Natural image inpainting\vspace{-0.3cm}}
\end{figure}












\section{RELATED WORK}
\label{sec:related}
\section{Related work}
\noindent \textbf{Video foundation models.}
With sufficient computational power and an abundant source of data, there have been attempts to build a single large-scale foundation model that can be adapted to diverse downstream tasks.
Along with the success of foundations models in the natural language processing domain~\cite{brown2020language,chen2021evaluating,devlin2019bert} and in computer vision~\cite{bertasius2021space,jia2021scaling,radford2021learning}, video data has become another data type of interest, as it has grown in scale due to numerous internet video-sharing platforms.
Accordingly, several methods to train a video foundation model have been proposed.
Due to the innate multi-modality of video data, \textit{i.e.}, a combination of visual $\cdot$ vocal $\cdot$ textual context, most works have centered around the variations of the cross-modal attention mechanism \cite{akbari2021vatt,bertasius2021space,gabeur2020multi,luo2020univl,neimark2021video,tan2021look,wei2020multi,yang2021taco}.
In addition, as most video data lack proper labels or descriptions, contrastive learning methods were studied to learn meaningful feature representations or enhance video-text alignment in a self-supervised manner \cite{akbari2021vatt,kuang2021video,luo2020univl,yang2021taco}.

More specifically, MERLOT \cite{zellers2021merlot} proposed a multi-modal representation learning method for visual commonsense reasoning, which also performed well in twelve video reasoning tasks.
VATT \cite{akbari2021vatt} introduced a multi-modal learning method via contrastive learning. 
The pre-trained model performed well in a variety of vision tasks from image classification to video action recognition and zero-shot video retrieval.
Another representative work, UniVL \cite{luo2020univl} proposed a straightforward pre-training method with auxiliary loss functions. 
After fine-tuning on a specific task, the pre-trained model performed outstandingly in a wide range of tasks of text-to-video retrieval, action segmentation, action step localization, video sentiment analysis, and video captioning.
Other foundation models for multiple video tasks include \cite{li2020hero,sun2019learning,sun2019videobert,zhu2020actbert,fu2021violet,wang2022all}. 

\noindent \textbf{Auxiliary learning.}
In order to enhance the performance of one or a multitude of primary tasks, auxiliary learning methods can be incorporated.
\cite{ruder2017overview} introduced Multi-task learning (MTL) to the deep neural networks by training a single model with multiple task losses to assist learning on the main task.
Such a method is generally adapted to pre-train the foundation models in the self-supervised manner~\cite{li2020hero,sun2019learning,sun2019videobert,zhu2020actbert,fu2021violet,wang2022all}.
However, these various pretext task losses used in the pre-training phase are ignored in the fine-tuning phase, and only the primary task loss is minimized.

Recently, meta-learning methods have been introduced for auxiliary learning.
\cite{liu2019self,navon2020auxiliary,shu2019meta} proposed a meta-learning method in which the model learns auxiliary tasks to generalize well to unseen data. 
In these settings, a separate subset of data is held out as the primary task, while the others are used as auxiliary tasks that aid the primary task's performance.
Similar methods were adopted for computer vision tasks such as semantic segmentation \cite{xu2021leveraging}.
Other domain applications include navigation tasks with reinforcement learning \cite{ye2021auxiliary}, or self-supervised learning methods on graph data \cite{hwang2020self}.

\section{Conclusion}
\label{sec:conclusion}
\section{Conclusion}\label{sec:conclusion}
In this work, we focus on addressing the fundamental challenge of OOD detection tasks, which is how to fully understand the semantic discrepancy between the ID/OOD samples. We reveal that the key to success in the realistic SCOOD task is to allocate as many ID samples in the unlabeled set correctly as possible. To this end, we propose a novel uncertainty-aware optimal transport scheme that introduces class-specific energy scores as guidance for effective label assignment. Experimental results show that our method achieves better performance than previous state-of-the-art methods on SCOOD benchmarks.

\textbf{Limitations.} In addition to temperature scaling, other techniques such as feature clipping applied in ReAct~\cite{sun2021react} also enhance the performance of energy score, so how to obtain an OOD score that best fits the SCOOD task can be further explored. Moreover, a setting highly related to SCOOD has been proposed in \cite{katz2022training} and formulated as a constrained optimization problem. We will also theoretically analyze these practical OOD settings in our feature work.

% \section*{Acknowledgments}
\textbf{Acknowledgments.} 
This work is supported by National Key R\&D Program of China under Grant 2020AAA0105701, National Natural Science Foundation of China (NSFC) under Grants 61872327, Major Special Science and Technology Project of Anhui, National Natural Science Foundation of China (62033012) and Ant Group through Ant Research Intern Program.


%%
%% The acknowledgments section is defined using the "acks" environment
%% (and NOT an unnumbered section). This ensures the proper
%% identification of the section in the article metadata, and the
%% consistent spelling of the heading.


%%
%% The next two lines define the bibliography style to be used, and
%% the bibliography file.
\bibliographystyle{ACM-Reference-Format}
\bibliography{sample-base}
%%
%% If your work has an appendix, this is the place to put it.
\newpage
\mbox{}
\newpage

\end{document}

