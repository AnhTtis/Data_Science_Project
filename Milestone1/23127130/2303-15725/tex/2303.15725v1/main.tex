 \def\isarxiv{1} %%% for icml submission version, we comment this line

\ifdefined\isarxiv
\documentclass[11pt]{article}

\usepackage[numbers]{natbib}

\else
%\documentclass[nohyperref]{article}
\documentclass{article}
%\usepackage{icml2022}
\usepackage{neurips_2022}
\fi


\usepackage{amsmath}
\usepackage{amsthm}
\usepackage{amssymb}
\usepackage{algorithm}
\usepackage{subfig}
\usepackage{algpseudocode}
\usepackage{graphicx}
\usepackage{grffile}
\usepackage{wrapfig,epsfig}
\usepackage{url}
\usepackage{xcolor}
\usepackage{epstopdf}


\usepackage{bbm}
\usepackage{dsfont}

 %%% print refs in table of contents
%\displaybreak
\allowdisplaybreaks

%\usepackage[lmargin=1in,rmargin=1in,tmargin=0.8in,bmargin=0.8in]{geometry}

\ifdefined\isarxiv

\let\C\relax
\usepackage{tikz}
\usepackage{hyperref}  %%% arxiv don't allow this.
\hypersetup{colorlinks=true,citecolor=blue,linkcolor=blue} %%% Zhao : maybe we should comment this in submission.
\usetikzlibrary{arrows}
\usepackage[margin=1in]{geometry}

\else

\usepackage{microtype}
\usepackage{hyperref}
\definecolor{mydarkblue}{rgb}{0,0.08,0.45}
\definecolor{mydarkgreen}{rgb}{0,0.8,0}
\hypersetup{colorlinks=true, citecolor=mydarkblue,linkcolor=mydarkblue}
%\usepackage[capitalize,noabbrev]{cleveref}
%\usepackage{colortbl}

\fi
%\linespread{1}
%\newcommand{\QED}{\hfill$\qed$}
%\graphicspath{{./figs/}}


\newtheorem{theorem}{Theorem}[section]
\newtheorem{lemma}[theorem]{Lemma}
\newtheorem{definition}[theorem]{Definition}
\newtheorem{notation}[theorem]{Notation}
%\newtheorem{proof}[theorem]{Proof}
\newtheorem{proposition}[theorem]{Proposition}
\newtheorem{corollary}[theorem]{Corollary}
\newtheorem{conjecture}[theorem]{Conjecture}
\newtheorem{assumption}[theorem]{Assumption}
\newtheorem{observation}[theorem]{Observation}
\newtheorem{fact}[theorem]{Fact}
\newtheorem{remark}[theorem]{Remark}
\newtheorem{claim}[theorem]{Claim}
\newtheorem{example}[theorem]{Example}
\newtheorem{problem}[theorem]{Problem}
\newtheorem{open}[theorem]{Open Problem}
\newtheorem{property}[theorem]{Property}
\newtheorem{hypothesis}[theorem]{Hypothesis}

\newcommand{\wh}{\widehat}
\newcommand{\wt}{\widetilde}
\newcommand{\ov}{\overline}
\newcommand{\N}{\mathcal{N}}
\newcommand{\R}{\mathbb{R}}
\newcommand{\RHS}{\mathrm{RHS}}
\newcommand{\LHS}{\mathrm{LHS}}
\renewcommand{\d}{\mathrm{d}}
\renewcommand{\i}{\mathbf{i}}
\renewcommand{\tilde}{\wt}
\renewcommand{\hat}{\wh}
\newcommand{\Tmat}{{\cal T}_{\mathrm{mat}}}

\DeclareMathOperator*{\E}{{\mathbb{E}}}
\DeclareMathOperator*{\var}{\mathrm{Var}}
\DeclareMathOperator*{\Z}{\mathbb{Z}}
\DeclareMathOperator*{\C}{\mathbb{C}}
\DeclareMathOperator*{\D}{\mathcal{D}}
\DeclareMathOperator*{\median}{median}
\DeclareMathOperator*{\mean}{mean}
\DeclareMathOperator{\OPT}{OPT}
\DeclareMathOperator{\supp}{supp}
\DeclareMathOperator{\poly}{poly}

\DeclareMathOperator{\nnz}{nnz}
\DeclareMathOperator{\sparsity}{sparsity}
\DeclareMathOperator{\rank}{rank}
\DeclareMathOperator{\diag}{diag}
\DeclareMathOperator{\dist}{dist}
\DeclareMathOperator{\cost}{cost}
\DeclareMathOperator{\vect}{vec}
\DeclareMathOperator{\tr}{tr}
\DeclareMathOperator{\dis}{dis}
\DeclareMathOperator{\cts}{cts}
\DeclareMathOperator{\reg}{reg}


\makeatletter
\newcommand*{\RN}[1]{\expandafter\@slowromancap\romannumeral #1@}
\makeatother
\newcommand{\Zhao}[1]{{\color{red}[Zhao: #1]}}
%\newcommand{\Danyang}[1]{{\color{purple}[Danyang: #1]}}
\newcommand{\Tianyi}[1]{{\color{purple}[Tianyi: #1]}} %%%Change to intern name
\newcommand{\Zhihang}[1]{{\color{blue}[Zhihang: #1]}}


\usepackage{lineno}
\def\linenumberfont{\normalfont\small}





\begin{document}

\ifdefined\isarxiv

\date{}


\title{Solving Regularized Exp, Cosh and Sinh Regression Problems}
\author{
Zhihang Li\thanks{\texttt{lizhihangdll@gmail.com}. Huazhong Agriculture University.}
\and
Zhao Song\thanks{\texttt{zsong@adobe.com}. Adobe Research.}
\and
Tianyi Zhou\thanks{\texttt{t8zhou@ucsd.edu}. University of California San Diego.}
}




\else

\title{Intern Project} 
\maketitle 
\iffalse
\icmltitlerunning{????}
%\linenumbers

\twocolumn[

\icmltitle{???}
% It is OKAY to include author information, even for blind
% submissions: the style file will automatically remove it for you
% unless you've provided the [accepted] option to the icml2019
% package.

% List of affiliations: The first argument should be a (short)
% identifier you will use later to specify author affiliations
% Academic affiliations should list Department, University, City, Region, Country
% Industry affiliations should list Company, City, Region, Country

% You can specify symbols, otherwise they are numbered in order.
% Ideally, you should not use this facility. Affiliations will be numbered
% in order of appearance and this is the preferred way.
\icmlsetsymbol{equal}{*}

\begin{icmlauthorlist}
\icmlauthor{Aeiau Zzzz}{equal,to}
\icmlauthor{Bauiu C.~Yyyy}{equal,to,goo}
\icmlauthor{Cieua Vvvvv}{goo}
\icmlauthor{Iaesut Saoeu}{ed}
\icmlauthor{Fiuea Rrrr}{to}
\icmlauthor{Tateu H.~Yasehe}{ed,to,goo}
\icmlauthor{Aaoeu Iasoh}{goo}
\icmlauthor{Buiui Eueu}{ed}
\icmlauthor{Aeuia Zzzz}{ed}
\icmlauthor{Bieea C.~Yyyy}{to,goo}
\icmlauthor{Teoau Xxxx}{ed}\label{eq:335_2}
\icmlauthor{Eee Pppp}{ed}
\end{icmlauthorlist}

\icmlaffiliation{to}{Department of Computation, University of Torontoland, Torontoland, Canada}
\icmlaffiliation{goo}{Googol ShallowMind, New London, Michigan, USA}
\icmlaffiliation{ed}{School of Computation, University of Edenborrow, Edenborrow, United Kingdom}

\icmlcorrespondingauthor{Cieua Vvvvv}{c.vvvvv@googol.com}
\icmlcorrespondingauthor{Eee Pppp}{ep@eden.co.uk}

% You may provide any keywords that you
% find helpful for describing your paper; these are used to populate
% the "keywords" metadata in the PDF but will not be shown in the document
\icmlkeywords{Machine Learning, ICML}

\vskip 0.3in
]

\printAffiliationsAndNotice{\icmlEqualContribution} 
\fi
\fi





\ifdefined\isarxiv
\begin{titlepage}
  \maketitle
  \begin{abstract}


Over the past few years, there has been a significant amount of research focused on studying the ReLU activation function, with the aim of achieving neural network convergence through over-parametrization. However, recent developments in the field of Large Language Models (LLMs) have sparked interest in the use of exponential activation functions, specifically in the attention mechanism.

Mathematically, we define the neural function $F: \R^{d \times m} \times  \mathbb{R}^d \rightarrow \mathbb{R}$ using an exponential activation function. Given a set of data points with labels $\{(x_1, y_1), (x_2, y_2), \dots, (x_n, y_n)\} \subset \mathbb{R}^d \times \mathbb{R}$ where $n$ denotes the number of the data. Here $F(W(t),x)$ can be expressed as $F(W(t),x) := \sum_{r=1}^m a_r \exp(\langle w_r, x \rangle)$, where $m$ represents the number of neurons, and $w_r(t)$ are weights at time $t$. It's standard in literature that $a_r$ are the fixed weights and it's never changed during the training. We initialize the weights $W(0) \in \mathbb{R}^{d \times m}$ with random Gaussian distributions, such that $w_r(0) \sim \mathcal{N}(0, I_d)$ and initialize $a_r$ from random sign distribution for each $r \in [m]$.

Using the gradient descent algorithm, we can find a weight $W(T)$ such that $\| F(W(T), X) - y \|_2 \leq \epsilon$ holds with probability $1-\delta$, where $\epsilon \in (0,0.1)$ and $m = \Omega(n^{2+o(1)}\log(n/\delta))$. To optimize the over-parametrization bound $m$, we employ several tight analysis techniques from previous studies [Song and Yang arXiv 2019, Munteanu, Omlor, Song and Woodruff ICML 2022]. 

 


  \end{abstract}
  \thispagestyle{empty}
\end{titlepage}

{\hypersetup{linkcolor=black}
\tableofcontents
}
\newpage

\else

\begin{abstract}


Over the past few years, there has been a significant amount of research focused on studying the ReLU activation function, with the aim of achieving neural network convergence through over-parametrization. However, recent developments in the field of Large Language Models (LLMs) have sparked interest in the use of exponential activation functions, specifically in the attention mechanism.

Mathematically, we define the neural function $F: \R^{d \times m} \times  \mathbb{R}^d \rightarrow \mathbb{R}$ using an exponential activation function. Given a set of data points with labels $\{(x_1, y_1), (x_2, y_2), \dots, (x_n, y_n)\} \subset \mathbb{R}^d \times \mathbb{R}$ where $n$ denotes the number of the data. Here $F(W(t),x)$ can be expressed as $F(W(t),x) := \sum_{r=1}^m a_r \exp(\langle w_r, x \rangle)$, where $m$ represents the number of neurons, and $w_r(t)$ are weights at time $t$. It's standard in literature that $a_r$ are the fixed weights and it's never changed during the training. We initialize the weights $W(0) \in \mathbb{R}^{d \times m}$ with random Gaussian distributions, such that $w_r(0) \sim \mathcal{N}(0, I_d)$ and initialize $a_r$ from random sign distribution for each $r \in [m]$.

Using the gradient descent algorithm, we can find a weight $W(T)$ such that $\| F(W(T), X) - y \|_2 \leq \epsilon$ holds with probability $1-\delta$, where $\epsilon \in (0,0.1)$ and $m = \Omega(n^{2+o(1)}\log(n/\delta))$. To optimize the over-parametrization bound $m$, we employ several tight analysis techniques from previous studies [Song and Yang arXiv 2019, Munteanu, Omlor, Song and Woodruff ICML 2022]. 

 

\end{abstract}

\fi



\section{Introduction}

The increasing complexity of source code poses a key challenge to the reliability of large-scale software systems. Software bugs in these systems can lead to safety issues~\cite{bug_safety} for users around the world as well as cause non-negligible financial losses~\cite{bug_loss}. As such, developers have to spend a large amount of time and effort on bug fixing. Consequently, \aprfull (\apr), designed to automatically generate patches to fix software bugs, has attracted wide attention from both academia and industry~\cite{long2016prophet, legoues2012genprog, long2015spr, lou2020can, tufano2018empstudy}. 


To achieve \apr, one popular approach is known as Generate-and-Validate (G\&V)~\cite{qi2015gv, ghanbari2019prapr, lou2020can, le2016hdrepair, legoues2012genprog, wen2018capgen, hua2018sketchfix, martinez2016astor, koyuncu2020fixminder, liu2019tbar, liu2019avatar}, which is typically based on the following pipeline: First, fault localization techniques~\cite{wong2016fl, abreu2007ochiai, zhang2013injecting, papadakis2015metallaxis, li2019deepfl, li2017transforming} are applied to determine the suspicious locations in programs where bugs are likely to exist. Then, the buggy locations are used by the \apr tools to generate a list of patches that replace buggy lines with correct lines. Afterward, each patch is validated against the original test suite to identify any \emph{plausible patches} (i.e., passing all tests in the test suite). Finally, to determine the \emph{correct patches}, developers examine the list of plausible patches to see if any of them can correctly fix the bug. 

Traditional \apr tools can mainly be categorized into heuristic-based~\cite{legoues2012genprog, le2016hdrepair, wen2018capgen}, constraint-based~\cite{mechtaev2016angelix, le2017s3, demacro2014nopol, long2015spr} and \template~\cite{ghanbari2019prapr, hua2018sketchfix, martinez2016astor, liu2019tbar, liu2019avatar}. Among these traditional tools, \template \apr tools~\cite{ghanbari2019prapr, liu2019tbar, benton2020effectiveness} have been able to achieve state-of-the-art results. \Template \apr tools typically leverage pre-defined templates (e.g., adding a nullness check) for bug fixing. However, since these fix templates are typically handcrafted, the number and types of bugs they are able to fix can be limited. 



To address the limitations of traditional \apr, researchers have proposed various \learning \apr tools~\cite{li2020dlfix, chen2018sequencer, jiang2021cure, lutellier2020coconut, zhu2021recoder, ye2022rewardrepair} based on the \nmtfull (\nmt) architecture~\cite{sutskever2014mt} where the input is the buggy code snippets and the goal is to translate the buggy code snippets into a fixed version. To accomplish this, \learning \apr tools require supervised training datasets with pairs of both buggy and fixed code snippets in order to learn how to perform this translation step. These training data are usually obtained by mining historical bug fixes using heuristics/keywords~\cite{dallmeier2007benchmark}, which can be imprecise for identifying bug-fixing commits; even the actual bug-fixing commits can include irrelevant code changes, leading to further pollution in the dataset~\cite{xia2022alpharepair}.
% 
Moreover, it can be hard for such \apr tools to generalize and fix bug types unseen during training. 



To better leverage recent advances in \plmfull{s} (\plm{s}), researchers~\cite{xia2022alpharepair, xia2023repairstudy, kolak2022patch, prenner2021codexws} have directly applied \plm{s} to generate patches without bug-fixing datasets. These \llm-based \apr tools work by either directly generating a complete code function~\cite{prenner2021codexws, xia2023repairstudy} or predict/infill the correct code snippet given its surrounding context~\cite{xia2022alpharepair, xia2023repairstudy}. By directly using \llm{s} that are pre-trained on billions of open-source code snippets, \llm-based \apr tools can achieve state-of-the-art performance on many repair datasets~\cite{xia2022alpharepair}. 


% 
%
%

Traditional \apr tools have long used the insight of the \emph{plastic surgery hypothesis}~\cite{barr2014plastic} where it states that the code ingredients to fix a bug already exist within the same project. Traditional \apr tools have manually designed pattern-~\cite{ghanbari2019prapr, saha2017elixir} or heuristic-based~\cite{jiang2018simfix, legoues2012genprog} approaches to finding and using such relevant code ingredients to generate fixes for bugs. However, the plastic surgery hypothesis has been largely ignored in \llm-based \apr. In fact, \llm provides a unique opportunity to fully automate the plastic surgery hypothesis idea via fine-tuning (learning project-specific information via model updates from the buggy project) and prompting (directly providing relevant code ingredients to the model), and make it directly applicable to different languages (since the \llm{s} are typically multi-lingual).%
Moreover, despite the intensive manual efforts involved, traditional \apr tools still cannot fully leverage project-specific information due to large search space for leveraging/composing existing code ingredients. In contrast, the project-specific information can effectively leveraged by \llm{s} due to their power in code understanding/vectorization, e.g., even partial/imprecise information may still guide \llm{s} in correct patch generation!
 To this end, we ask the question: \emph{How useful is the plastic surgery hypothesis in the era of \plm{s}}?








\mypara{Our Work.} To answer the question, we present \ourtech{\xspace} -- a \llm-based approach that automatically utilizes the plastic surgery hypothesis by systematically combining multiple fine-tuning and prompting strategies for \apr. \ourtech fine-tunes \plm{s} using two novel domain-specific training strategies: \textbf{\epfinetune} -- we fine-tune using the original buggy project by aggressively masking out a high percentage of tokens, which allows \plm to learn project-specific code tokens and programming styles; and \textbf{\rofinetune} -- which only masks out a single continuous code sequence per training sample, allowing the model to get used to the final \csapr task of predicting a single continuous code sequence. Furthermore, we directly leverage the ability for \plm{s} to understand natural language instructions and introduce a novel prompting strategy, \textbf{\idprompting}, which uses information retrieval and static analysis to obtain a list of relevant identifiers for the buggy lines. While such relevant identifiers are critical for fixing some difficult bugs, they may not be seen by the \llm during inference due to limited context window size. Through the use of prompting, we directly tell the model to use these extracted identifiers (relevant code ingredients) to generate the correct code. Finally, to perform repair, we combine all four model variants (including the base model, both fine-tuned models and the base model with prompting) for the final repair.





While our insight of leveraging the plastic surgery hypothesis for \llm-based \apr is generalizable across different types of \plm{s}, to implement \ourtech, we choose a recent \plm{\xspace}, \ctfive~\cite{wang2021codet5}, which is pre-trained on millions of open-source code snippets. \ctfive is an encoder-decoder model trained using \mspfull (\msp) objective where a percentage of tokens are masked out and each continuous masked token sequence is referred to as a masked span. Also, although we only extract relevant identifiers from the current buggy project (since this paper focuses on the plastic surgery hypothesis), our work can be easily extended to obtain other code information (such as relevant statements or functions) from other sources, such as  the massive pre-training corpora~\cite{husain2020codesearchnet} or historical bug-fixing datasets~\cite{jiang2019infer}, which can provide more coding knowledge for \llm{s}. Besides, although we mainly focus on using traditional string comparison algorithms for information retrieval in this paper, these techniques can be easily replaced by other frequency-based retrieval~\cite{robertson2009probabilistic} and neural search (or embedding-based search)~\cite{reimers2019sentence}.
  In summary, this paper makes the following contributions:


%


\begin{itemize}[noitemsep, leftmargin=*, topsep=0pt]
    \item \textbf{Dimension.} This paper is the first to revisit the important plastic surgery hypothesis in the era of \llm{s}. It opens up a new dimension for \llm-based \apr to incorporate previously neglected information from the buggy project itself to boost \apr performance. Furthermore, it demonstrates the promising future of retrieval-based prompting for modern \llm-based \apr.
    \item \textbf{Implementation.} We implement \ourtech based on the recent \ctfive model. We augment the model using two novel fine-tuning strategies: \epfinetune and \rofinetune, along with a novel prompting strategy based on information retrieval and static analysis: \idprompting. We combine the patches generated by all four models together and perform patch ranking to speed up \apr.% 
    \item \textbf{Evaluation Study.} We conduct an extensive evaluation against state-of-the-art \apr tools. On the widely studied \dfj 1.2 and 2.0 datasets~\cite{just2014dfj}, \ourtech is able to achieve the new state-of-the-art results of 89 and 44 correct bug fixes (15 and 8 more than best baseline) respectively.  Furthermore, we perform a broad ablation study to justify our design. \ourtech demonstrates for the first time that the plastic surgery hypothesis can substantially boost \llm-based \apr and advance state-of-the-art \apr, while being fully automated and general. Moreover, even partial/imprecise code ingredients may still effectively guide \llm{s} for \apr!
\end{itemize}

 %%% Section 1. Introduction

%\vspace{-2mm}
\section{Preliminary}\label{sec:preli}
\paragraph{Notations.}For a positive integer, we use $[n]$ to denote set $\{ 1,2,\cdots,n\}$. 
We use $\cosh(x) =\frac{1}{2}( e^x + e^{-x})$ and $\sinh(x) = \frac{1}{2}(e^x - e^{-x} )$.
For a square matrix, we use $\tr[A]$ to denote the trace of $A$.
An $n \times n$ symmetric real matrix $A$ is said to be positive-definite if $x^{\top} A x > 0$ for all non-zero $x \in \R^n$.
An $n \times n$ symmetric real matrix $A$ is said to be positive-semidefinite if $x^{\top} A x \geq 0$ for all non-zero $x \in \R^n$. For any function $f$, we use $\wt{O}(f) = f \cdot \poly(\log f)$.



\subsection{Matrix hyperbolic functions}
\begin{definition}[Matrix function]
Let $f:\R \rightarrow \R$ be a real function and $A\in \R^{n\times n}$ be a real symmetric function with eigendecomposition 
\begin{align*} 
A=Q\Lambda Q^{-1}
\end{align*}
where $\Lambda\in \R^{n\times n}$ is a diagonal matrix. Then, we have
\begin{align*}
    f(A):=Qf(\Lambda) Q^{-1},
\end{align*}
where $f(\Lambda)\in \R^{n\times n}$ is the matrix obtained by applying $f$ to each diagonal entry of $\Lambda$.
\end{definition}

We have the following lemma to bound $\cosh(A)$ and delay the proof to Appendix~\ref{sec:cosh_bound_proof}.
\begin{lemma}\label{lem:cosh_bound}
Let $A$ be a real symmetric matrix, then we have
\begin{align*}
    \|\cosh(A)\| = \cosh(\|A\|) \leq \tr[\cosh(A)].
\end{align*}
We also have 
\begin{align*}
\|A\| \leq 1+\log(\tr[\cosh(A)]).
\end{align*}
\end{lemma}



\subsection{Properties of \texorpdfstring{$\sinh$}{} 
and \texorpdfstring{$\cosh$}{}
}


We have the following lemma for properties of $\sinh$ and $\cosh$. 
\begin{lemma}[Scalar version]\label{lem:property_sinh_cosh_scalar}
Given a list of numbers $x_1, \cdots x_n$, we have
\begin{itemize}
    \item $( \sum_{i=1}^n \cosh^2(x_i) )^{1/2} \leq \sqrt{n} + ( \sum_{i=1}^n \sinh^2(x_i) )^{1/2}$,
    \item $(\sum_{i=1}^n \sinh^2(x_i) )^{1/2} \geq \frac{1}{\sqrt{n}} (\sum_{i=1}^n \cosh(x_i) - n)$.
\end{itemize}
\end{lemma}
\begin{proof}
For the first equation, we can bound $( \sum_{i=1}^n \cosh^2(x_i) )^{1/2}$ by:
\begin{align*}
     ( \sum_{i=1}^n \cosh^2(x_i) )^{1/2} 
    = &~ (n + \sum_{i=1}^n \sinh^2(x_i))^{1/2} \\
    \leq &~\sqrt{n} + ( \sum_{i=1}^n \sinh^2(x_i) )^{1/2}
\end{align*}
where the first step comes from fact~\ref{fact:cosh_sinh_1}, and the second step follows from $\sqrt{a + b} \leq \sqrt{a} + \sqrt{b}$.

For the second equation, we can bound $(\sum_{i=1}^n \sinh^2(x_i) )^{1/2}$ by:
\begin{align*}
    (\sum_{i=1}^n \sinh^2(x_i) )^{1/2} 
    \geq &~ \frac{1}{\sqrt{n}}(\sum_{i=1}^{n} \sinh(x_i)) \\
    \geq &~ \frac{1}{\sqrt{n}}(\sum_{i=1}^{n} \cosh(x_i) -n)
\end{align*}
where the first step follows that $\sqrt{\frac{\sum_{i=1}^{n} x_i^2}{n}} \geq \frac{\sum_{i=1}^{n} x_i}{n}$,
and the second step follows from fact~\ref{fact:cosh_sinh_1} and $\sqrt{x^2 -1} \geq \sqrt{x} - 1$.
\end{proof}


We also have a lemma for the matrix version. 
\begin{lemma}[Matrix version]\label{lem:property_sinh_cosh_matrix}
For any real symmetric matrix $A$, we have
\begin{itemize}
    \item $ (\tr[\cosh^2(A)])^{1/2} \leq \sqrt{n} + \tr[ \sinh^2(A) ]^{1/2}$,
    \item $(\tr[ \sinh^2(A) ])^{1/2} \geq \frac{1}{\sqrt{n}} ( \tr[ \cosh(A) ] - n ) $.
\end{itemize}
\end{lemma}

\begin{proof}

{\bf Part 1.}
We have
\begin{align*}
    (\tr[\cosh^2(A)])^{1/2} = & ~ ( n+  \tr[\sinh^2(A)] )^{1/2} \\
    \leq & ~ \sqrt{n} + \tr[ \sinh^2(A) ]^{1/2}.
\end{align*}
where the first step follows from $ \cosh^2(A) - \sinh^2(A) = I$.

{\bf Part 2.}
Let $\sigma_i$ denote the singular value of $\cosh(A)$
\begin{align*}
    ( \tr[  \sinh^2(A) ] )^{1/2} 
    = & ~  ( \tr[ \cosh^2(A) ] - n )^{1/2} \\
    = & ~  ( \sum_{i=1}^n \sigma_i^2 - 1 )^{1/2} \\ 
    \geq & ~   \frac{1}{\sqrt{n}} \sum_{i=1}^n \sqrt{ \sigma_i^2  -1 } \\
    \geq & ~    \frac{1}{\sqrt{n}} (\sum_{i=1}^n \sigma_i - 1 ) \\
    = & ~ \frac{1}{\sqrt{n}} ( \tr[  \cosh(A) ] -  n ) 
\end{align*}
where the second step follows from $\| \cdot \|_2 \geq \frac{1}{\sqrt{n}} \| \cdot \|_1$, the third step follows from $\sigma_i \geq 1$.

\end{proof}





\newpage
\section{Experiments}
\label{sec:exp}

In this section, we demonstrate the wide range of applications and the high capabilities of Uni-Fusion. 
First, we evaluate Uni-Fusion in application 1) Incremental surface and color reconstruction, comparing its performance with SOTAs.
%
For applications 2) and 5), which are new topics, no specific benchmarks are available. 
Therefore, we showcase the performance on existing results.
%
Next, we implement application 3) and compare it with SOTA zero-shot semantic segmentation models.
%
Finally, for application 4), since infrared data is not commonly used, we collect our own dataset containing infrared values and show all applications on this data.

\subsection{Implementation Details}
\label{sec:exp:details}

In the experiments, we use our sample-based GPIS for local geometry encoding.
For each point, two additional points are sampled along normal direction, one positive and one negative, with distance $d_s=0.1$ in the local voxel's normalized space. 
Compared to derivative-based GPIS, our sample-based GPIS is more efficient in both space and time. 
For the encoder, we randomly sample $256$ anchor points from the range $[-0.5,0.5]^3$.
We utilize the first $20$ eigenpairs, resulting in a feature dimension of $20$.
The model selection process is discussed in the ablation study.

Different latent maps use different granularities.
For the surface LIM, we use a voxel size of $5\si{\centi\meter}$. 
For color which requires later comparison to NeRF, we use a voxel size of $2\si{\centi\meter}$.
For other property LIM and feature LIM, we use a voxel size of $10\si{\centi\meter}$.

For smooth reconstruction, the encoded voxel is designed overlapped following~\cite{huang2021di}.
The encoded voxel uses twice the voxel size, resulting in a half-space overlap with each neighboring voxel.
During meshing, SDFs are retrieved and interpolated from the overlapped voxels~\cite{huang2021di}.
While for the remaining properties, we sample only from its own voxel part.

The implementation runs on a PC with AMD Ryzen 9 5950X 16-core CPU and an Nvidia Geforce RTX 3090 GPU (24 GB).

\subsection{Datasets}

We evaluate incremental reconstruction on the ScanNet dataset~\cite{dai2017scannet}, TUM RGB-D dataset~\cite{sturm2012benchmark}, and Replica dataset~\cite{sucar2021imap}.
Using MSG-Net~\cite{zhang2018multi}'s material set, we transfer styles to the 3D canvas.
For open-vocabulary scene understanding, we evaluate on ScanNet segmentation data~\cite{qi2017pointnet++} and S3DIS dataset~\cite{armeni20163d}.

\subsubsection{ScanNet~\cite{dai2017scannet}}

ScanNet is a densely annotated RGB-D video dataset.
It is captured with the structure sensor~\cite{occipital} and contains 1513 scenes for training and validation.
For each scene, both images and a 3D mesh is provided, along with their 2D and 3D semantic annotations. 

ScanNet provides 312 scenes for validation, which contains a wide range of different room structures.
It has now been widely used in the thorough evaluation of the performance of reconstruction and semantic segmentation.

\subsubsection{TUM RGB-D~\cite{sturm2012benchmark}}

TUM RGB-D is a benchmark to mainly evaluate the tracking performance.
It is captured with Microsoft Kinect sensor together with ground-truth trajectory from the sensor.

\subsubsection{Replica~\cite{sucar2021imap}}

The Replica dataset refers to iMAP's pre-processed dataset~\cite{sucar2021imap}.
It is a synthetic rendered RGB-D dataset from given 3D models.
The advantage of including this dataset is that Replica does not have motion blur. 
This is better to evaluate the capability of the algorithms on reconstructing surface color.

\subsubsection{MSG-Net Style~\cite{zhang2018multi}}

MSG-Net provides material images for transfering the styles.
We select 21style fold for demonstration.
These images are given in \cref{fig:style} together with our result.

\subsubsection{ScanNet Point Cloud Segmentation Data~\cite{qi2017pointnet++}}

For point cloud semantic segmentation benchmarking, PointNet++~\cite{qi2017pointnet++} preprocesses the original ScanNet~\cite{dai2017scannet} and generates subsampled point clouds and corresponding annotations for each scene.

\subsubsection{S3DIS~\cite{armeni20163d} and 2D-3D-S~\cite{armeni2017joint}}

S3DIS is a semantic segmentation dataset for 3D point clouds.
Which is also a subset of the 2D-3D-S dataset.
The 2D-3D-S dataset is a multi-modality dataset containing 2D, 2.5D and 3D domains. 
This dataset is densely annotated with semantic labels.

Note that 2D-3D-S's 2D captures is not a RGB-D video as ScanNet.
2D-3D-S's images only have small overlap. 
Therefore, it is only suitable for semantic segmentation and not for incremental reconstruction.

\subsection{Baselines}

For online surface mapping evaluation, we select TSDF-Fusion~\cite{curless1996volumetric}, iMAP~\cite{sucar2021imap}, SOTA DI-Fusion~\cite{huang2021di} and BNV-Fusion~\cite{li2022bnv} as four baseline methods.

For the color field, we choose TSDF-Fusion~\cite{curless1996volumetric}, $\sigma$-Fusion~\cite{rosinol2023probabilistic}, iMAP~\cite{sucar2021imap}, NICE-SLAM~\cite{zhu2022nice} and even the recent hot Neural Radiance Fields model NeRF-SLAM~\cite{rosinol2022nerf} as five baselines.
While including NeRF in the comparison may not be entirely fair, we want to show how Uni-Fusion narrows the performance gap.

For the scene understanding application, we evaluate generalized zero-shot point cloud semantic segmentation with ZSLPC~\cite{cheraghian2019zero}, DeViSe~\cite{frome2013devise} and SOTA 3DGenZ~\cite{michele2021generative} for comparison.

\subsection{Metrics}

For incremental reconstruction, we evaluate the geometric reconstruction using \textbf{Accuracy}, \textbf{Completeness}, and \textbf{F1 score} according to SOTA BNV-Fusion. It firstly uniformly samples $100,000$ points from the reconstruction and ground truth meshes respectively.
Then \textbf{Accuracy} (\textbf{Completeness}) measures the percentage of reconstruction-to-groundtruth (groundtruth-to-reconstruction) distances that are lower than $2.5\si{\centi\meter}$ threshold. \textbf{F1 score} is the harmonic mean of accuracy and completeness.
For tracking performance, we use \textbf{ATE RMSE}.

To evaluate color reconstruction, we follow SOTA on this topic, NeRF to render both depth and RGB images to compare the image level \textbf{Depth L1} and \textbf{RGB PSNR}.

To compare scene understanding, we follow zero-shot point cloud semantic segmentation SOTA 3DGenZ to evaluate the \textbf{Intersection-of-Union (IoU)} and \textbf{Accuracy}.


\subsection{Reconstruction Results}

For evaluation, we first use the ScanNet validation set with 312 sequences to thoroughly test the geometric reconstruction on a large variant of scenes.
%
Then, we use TUM RGB-D to compare our modified tracking model with related works.
Because this part is not the main contribution, we give a rough overview of the tracking results.
%
To fairly evaluate the color reconstruction, we use the high quality rendered Replica dataset to compare with related works, including NeRF.

%\subsubsection{Object}
% on instance-gp
% Objective data usually has more fine detail
% 1. for detail precision
% A: no, object reconstruction is not as good as instance-ngp, so cancelled.
\begin{table*}[!]
	\centering
	\caption{Comparison to ScanNet~\cite{dai2017scannet}.
       Our method generalizes better to various scenes.
       $^*$ indicates the result from our runs of the official BNV-Fusion code.}
	\small
	%\setlength{\tabcolsep}{5mm}
	\setlength{\tabcolsep}{0.9em}
		%\resizebox{\textwidth}{!}{
		\begin{tabular}{l  c c c| c c c }
			\toprule
			Method & \begin{tabular}{@{}c@{}}Pre-Train\\ with extra dataset\end{tabular} & \begin{tabular}{@{}c@{}}Train \\ with sequences\end{tabular} & Real-time & Accuracy (\%) $\uparrow$ & Completeness (\%) $\uparrow$ & F1 score $\uparrow$ \\
			\midrule
			TSDF Fusion~\cite{zhou2018open3d} & None & None & $\checkmark$ &73.83 & 85.85 & 78.84 \\
			iMAP~\cite{sucar2021imap} & None & Online train& &68.96 & 82.12 & 74.96 \\
			DI-Fusion~\cite{huang2021di} &Object Pretrain & None & $\checkmark$&66.34 & 79.65 & 72.97 \\
			BNV-Fusion~\cite{li2022bnv} &Object Pretrain &  Post Optimization& &{74.90} & \textbf{88.12} & {80.56} \\
			BNV-Fusion$^{*}$~\cite{li2022bnv} &Object Pretrain & Post Optimization &&{73.42} & {81.75} & {77.18} \\
			\textbf{Uni-Fusion (Ours)} &None &None &$\checkmark$&\textbf{80.43} & {84.91} & \textbf{82.44} \\
			\bottomrule
		\end{tabular}
	  %}
	\label{tab:scannet}
	\vspace{-.6cm}
\end{table*}
\begin{figure*}[t]
	\subfloat[width=.33\textwidth][Accuracy]{
		\centering
		\includegraphics[width=.22\linewidth]{im/exp/recons/scannet/scannet_acc.png}
		\includegraphics[width=.1\linewidth]{im/exp/recons/scannet/scannet_acc_box.png}
	}
	\subfloat[width=.33\textwidth][Completeness]{
		\centering
		\includegraphics[width=.22\linewidth]{im/exp/recons/scannet/scannet_comp.png}
		\includegraphics[width=.1\linewidth]{im/exp/recons/scannet/scannet_comp_box.png}
	}
	\subfloat[width=.33\textwidth][F1 score]{
		\centering
		\includegraphics[width=.22\linewidth]{im/exp/recons/scannet/scannet_F1.png}
		\includegraphics[width=.1\linewidth]{im/exp/recons/scannet/scannet_F1_box.png}
	}
	\label{fig:recon:scannet:elementwise}
	\caption{Quantitative comparison on 312 scenes of the ScanNet validation set.
       We demonstrate the performance of SOTA BNV-Fusion and our Uni-Fusion.
       We sort our evaluation value and reordered all of the scores.
       The zigzag pink is the BNV-Fusion result;
       we also plot a deep-pink smoothed curve for better visualization.}
\end{figure*}

\subsubsection{Evaluation on ScanNet Dataset~\cite{dai2017scannet}}
\label{sec:exp:scannet}

We use the 312 diversified scenes from the ScanNet validation set to evaluate the performance of surface reconstruction. 
We follow the pure mapping SOTA BNV-Fusion to take every 10th posed frame as input. 
%
Without using any learning (in contrast iMAP, DI-Fusion, and BNV-Fusion do) or any post optimization (as BNV-Fusion does), our Uni-Fusion is capable to achieve precise continuous mapping performance. 

As shown in~\cref{tab:scannet}, our Uni-Fusion achieves \textbf{$+6$ higher accuracy} than the incremental surface reconstruction SOTA BNV-Fusion.
Our model does not exceed on completeness comparing to BNV-Fusion that support completion in post-optimization.
Though, Uni-Fusion's completion is still much higher than one other optimization based model iMAP.
%We consider it because our model does not support hole-completion as the optimization based models iMap and BNV-Fusion.
Overall, our Uni-Fusion model achieves higher F1-scores that quantifies the overall quality.

Please note that, SOTA BNV-Fusion is not real-time capable, since it requires post optimization of all fed frames.
Without the post-optimization, the real-time model Di-Fusion shows much worse results.
However, our \textbf{real-time} model \textbf{Uni-Fusion} is able to achieves \textbf{much better} reconstruction quality than these approaches even without post-optimization. 

\newcommand{\scannetImSize}{.16}
\begin{figure*}[t!]
	\centering
	\setlength{\tabcolsep}{0.1em}
	\renewcommand{\arraystretch}{.1}
	\begin{tabular}{|c | c |c |||c |c | c|}
		\hline
		{\Large{BNV-Fusion}} & {\Large{Uni-Fusion}} &{\Large{Ground Truth}} & {\Large{BNV-Fusion}} &{\Large{Uni-Fusion}} & {\Large{Ground Truth}} \\ \hline \hline
		
\includegraphics[width=\scannetImSize\linewidth]{im/exp/recons/scannet_qualifi/scene0568_00_bnv.png}
		&\includegraphics[width=\scannetImSize\linewidth]{im/exp/recons/scannet_qualifi/scene0568_00_mine.png}
		&\includegraphics[width=\scannetImSize\linewidth]{im/exp/recons/scannet_qualifi/scene0568_00_gt.png}
		&		\includegraphics[width=\scannetImSize\linewidth]{im/exp/recons/scannet_qualifi/scene0164_00_bnv.png}
		&\includegraphics[width=\scannetImSize\linewidth]{im/exp/recons/scannet_qualifi/scene0164_00_mine.png}
		&\includegraphics[width=\scannetImSize\linewidth]{im/exp/recons/scannet_qualifi/scene0164_00_gt.png}\\
		
\includegraphics[width=\scannetImSize\linewidth]{im/exp/recons/scannet_qualifi/scene0249_00_bnv.png}
		&\includegraphics[width=\scannetImSize\linewidth]{im/exp/recons/scannet_qualifi/scene0249_00_mine.png}
		&\includegraphics[width=\scannetImSize\linewidth]{im/exp/recons/scannet_qualifi/scene0249_00_gt.png}
		&		\includegraphics[width=\scannetImSize\linewidth]{im/exp/recons/scannet_qualifi/scene0435_00_bnv.png}
		&\includegraphics[width=\scannetImSize\linewidth]{im/exp/recons/scannet_qualifi/scene0435_00_mine.png}
		&\includegraphics[width=\scannetImSize\linewidth]{im/exp/recons/scannet_qualifi/scene0435_00_gt.png}\\
		
\includegraphics[width=\scannetImSize\linewidth]{im/exp/recons/scannet_qualifi/scene0046_00_bnv.png}
		&\includegraphics[width=\scannetImSize\linewidth]{im/exp/recons/scannet_qualifi/scene0046_00_mine.png}
		&\includegraphics[width=\scannetImSize\linewidth]{im/exp/recons/scannet_qualifi/scene0046_00_gt.png}
		&		\includegraphics[width=\scannetImSize\linewidth]{im/exp/recons/scannet_qualifi/scene0050_00_bnv.png}
		&\includegraphics[width=\scannetImSize\linewidth]{im/exp/recons/scannet_qualifi/scene0050_00_mine.png}
		&\includegraphics[width=\scannetImSize\linewidth]{im/exp/recons/scannet_qualifi/scene0050_00_gt.png}\\
		\hline
	\end{tabular}
	%\captionof{figure}
	\caption{Surface reconstruction on ScanNet dataset.}
	\label{fig:recons:scannet_demo}
	\vspace{-.5cm}
\end{figure*}

We additionally run BNV-Fusion's official implementation (emphasized with $^*$) on the 312 videos of ScanNet and do a post element-wise comparison in \cref{fig:recon:scannet:elementwise}. 
Our result is the {\color{Cyan}light blue} curve, BNV-Fusion's result is colored with {\color{Lavender}pink}.
Scene index is sorted corresponding to the score value of Uni-Fusion.
For better visualization, we smooth BNV-Fusion's curve and plot it with dark pink.
It is obvious that the score of Uni-Fusion is overall higher than BNV-Fusion's. 
Moreover, we use box-plot to conclude the statistics besides the curve plot. Uni-Fusion's scores are distributed in a higher region. For completeness which is less obvious better, Uni-Fusion's box is smaller while in a relative higher position. This means that Uni-Fusion has more stable completeness result while BNV-Fusion is more likely to get low completeness in some cases.

To summarize, our model is almost better on all 312 scenes on all accuracy, completeness and F1-score.
Which is also revealed in \cref{tab:scannet} with BNV-Fusion$^*$, that the BNV-Fusion's official implementation does not exceed Uni-Fusion on all metrics.

We plot reconstruction on selected scenes from ScanNet in~\cref{fig:recons:scannet_demo}. 
Both BNV-Fusion and our Uni-Fusion are able to produce high quality reconstruction.
We see that BNV-Fusion gives lots of small meshes on walls, which are shown as small particles in the reconstruction. 
We consider it is because BNV-Fusion use very small voxel size ($0.02\si{\meter}$) to get a high score.
This is also revealed by their \textbf{\SI{247}{MB}} mesh in average, while ours is only \textbf{\SI{54}{Mb}} in average.
Furthermore, our Uni-Fusion's mesh is more smooth and
%Both BNV-Fusion and Uni-Fusion demonstrate high quality result.
also provides high-precise color to the mesh which is not available for the Surface SOTA.

%In this test, we purely evaluate the surface reconstruction capacity with SOTAs. 
%While reconstruction is not merely surface.  
%Thus in the following, we find benchmarks for both surface and color.


\subsubsection{Tracking Evaluation on TUM RGB-D Dataset~\cite{sturm2012benchmark}}
% follow nice-slam

In the above test, we compare the performance of pure mapping.
While tracking is not the contribution focus in our paper, it is part of the reconstruction model. We follow the novel reconstruction model NICE-SLAM~\cite{zhu2022nice} to evaluate the camera tracking on the small-scale TUM RGB-D dataset.
Our Uni-Fusion uses a coarse-to-fine strategy for 3D reconstruction tracking.
From~\cref{tab:tum_rmse}, it demonstrates overall better ATE RMSE than other implicit representation models.

\begin{table}[]
		\caption{Tracking on TUM RGB-D~\cite{sturm2012benchmark}.
		ATE RMSE [$\si{\centi\meter}$] ($\downarrow$) is used as the evaluation metric.
	}
	\centering
	\footnotesize
	\setlength{\tabcolsep}{0.7em}
	\resizebox{\linewidth}{!}{
		\begin{tabular}{l|ccc}
			\hline
			& \tt{fr1/desk} &  \tt{fr2/xyz} &  \tt{fr3/office} \\
			
			\hline
			{iMAP}~\cite{sucar2021imap}      & 4.9 & 2.0 & 5.8  \\
			{iMAP$^*$}~\cite{sucar2021imap} & 7.2 & 2.1  & 9.0 \\
			{DI-Fusion~\cite{huang2021di}} & 4.4 & 2.3 & 15.6 \\
			NICE-SLAM~\cite{zhu2022nice}           & 2.7 & 1.8 & 3.0 \\
			Ours& 1.8& 0.5& 2.1 \\
			\hline
			{BAD-SLAM}\cite{schops2019bad} & 1.7  & 1.1  & 1.7 \\
			{Kintinuous}\cite{whelan2012kintinuous} & 3.7  &  2.9  & 3.0 \\
			{ORB-SLAM2}\cite{mur2017orb} & \bf 1.6  & \bf 0.4  & \bf 1.0 \\
			\hline
	\end{tabular}}
	\vspace{2pt}

	\label{tab:tum_rmse}
\end{table}

On the other hand, there also exist high accuracy algorithms from SLAM. 
By additional using Bundle Adjustment and Loop-closing techniques, their tracking quality is much better than all of the implicit based models.

%But it is dangerous to directly apply SLAM result on reconstruction. Please find our demonstration in Fig [?]. Which explains the more widely used frame-to-model strategy in 3D reconstruction.
Even though, our coarse-to-fine strategy firstly ensure not easy to lose track. Secondly, it is more suitable for surface fitting.

Which further support our test in Replica dataset.



\begin{table*}[t!]
	\centering
	\caption{Geometric (L1) and Photometric (PSNR) evaluation on the Replica dataset~\cite{sucar2021imap}.}
	\footnotesize
	\setlength{\tabcolsep}{0.36em}
	\renewcommand{\arraystretch}{1.2}
	\begin{tabular}{clcccccccccccccccccc}
		\toprule
		& & \multicolumn{1}{c}{\makecell{\tt{office-0}}} & \multicolumn{1}{c}{\makecell{\tt{office-1}}} & \multicolumn{1}{c}{\makecell{\tt{office-2}}}& \multicolumn{1}{c}{\makecell{\tt{office-3}}} & \multicolumn{1}{c}{\makecell{\tt{office-4}}} & \multicolumn{1}{c}{\makecell{\tt{room-0}}} & \multicolumn{1}{c}{\makecell{\tt{room-1}}} &  \multicolumn{1}{c}{\makecell{\tt{room-2}}} & Avg. \\
		\midrule
		\multicolumn{5}{l}{\textit{Non-continuous mapping method}}\\
		\multirow{2}{*}{\makecell{\textbf{TSDF-Fusion}~\cite{curless1996volumetric}}}
		& {\bf Depth L1} [$\si{\centi\meter}$] $\downarrow$
	 & 14.11 & 10.50 & 30.89 & 28.92 & 22.83	& 23.51 & 20.94 & 23.34 & 21.88 \\
		& {\bf PSNR } [$\si{\dB}$] $\uparrow$
		& 11.16 & 15.92 & 4.86 & 5.68 & 5.46 & 3.43 & 4.51 & 5.57 & 7.07 \\
		
		\midrule
		\multirow{2}{*}{\makecell{\textbf{$\sigma$-Fusion}\cite{rosinol2023probabilistic} }}
		& {\bf Depth L1} [$\si{\centi\meter}$] $\downarrow$
		 & 13.80 & 10.21 & 22.27 & 28.70 & 22.21& 21.92 & 19.28 & 22.40 & 20.10 \\
		& {\bf PSNR } [$\si{\dB}$] $\uparrow$
		 & 11.16 & 15.92 & 4.86 & 5.69 & 5.46& 3.45  & 4.51 & 5.57 & 7.08 \\
		
		
		
		
		
		\midrule
		\midrule
		\multicolumn{5}{l}{\textit{Continuous mapping method}}\\
		\multirow{2}{*}{\makecell{\textbf{iMAP$^*$}~\cite{sucar2021imap}}}
		& {\bf Depth L1} [$\si{\centi\meter}$] $\downarrow$
		 & 6.43 & 7.41 & 14.23 & 8.68 & 6.80& 5.70 & 4.93 & 6.94 & 7.64\\
		& {\bf PSNR } [$\si{\dB}$] $\uparrow$
		& 7.39 & 11.89 & 8.12 & 5.62 & 5.98& 5.66 & 5.31 & 5.64  & 6.95\\
		\midrule
		\multirow{2}{*}{{\makecell{\textbf{Nice-SLAM}~\cite{zhu2022nice} }}}
		& {\bf Depth L1} [$\si{\centi\meter}$] $\downarrow$
		& { 1.51 } & { 0.93 } & { 8.41 } & { 10.48 } & {2.43} & { 2.53 } & { 3.45 } & { 2.93 }  & { 4.08 } \\
		& {\bf PSNR } [$\si{\dB}$] $\uparrow$
		 & { 22.44 } & { 25.22 } & { 22.79 } & { 22.94 } & { 24.72 } & \textbf{ 29.90 } & \textbf{ 29.12 } & { 19.80 }& { 24.61 } \\
		
		
		
		
		\midrule	
		\multirow{2}{*}{{\makecell{\textbf{Uni-Fusion} (Ours) }}}
		% using abs(diff)
		%	& {\bf Depth L1} [$\si{\centi\meter}$] $\downarrow$ &\textbf{1.98}&\textbf{1.18}&\textbf{1.64}&\textbf{1.23}&\textbf{0.84}&\textbf{1.61}&\textbf{3.01}&\textbf{1.60} &\textbf{1.64}
		% follow nerf-slam to remove outlier gt first
		& {\bf Depth L1} [$\si{\centi\meter}$] $\downarrow$
		& \textbf{0.79}&\textbf{0.56}&\textbf{1.59}&\textbf{2.71}&\textbf{1.66}&\textbf{1.94}&\textbf{0.69}&\textbf{1.80}& \textbf{1.47}
		\\
		& {\bf PSNR } [$\si{\dB}$] $\uparrow$ &\textbf{33.88}&\textbf{33.31}&\textbf{25.84}&\textbf{26.01}&\textbf{28.14}&24.02&26.20&\textbf{27.17} &\textbf{28.07}
		\\
		
		\midrule
		\midrule
		\multicolumn{5}{l}{\textit{Neural radiance field method}}\\
		\multirow{2}{*}{{\makecell{\textbf{NeRF-SLAM}~\cite{rosinol2022nerf} }}}
		& {\bf Depth L1} [$\si{\centi\meter}$] $\downarrow$
	 & {2.49}   & {1.98}  & {9.13}  & {10.58} & {3.59}	& {2.97}  & {2.63}  & {2.58}  & {4.49} \\
		& {\bf PSNR } [$\si{\dB}$] $\uparrow$
	 & \textbf{48.07}  & \textbf{53.44} & \textbf{39.30} & \textbf{38.63} & \textbf{39.21} 	& \textbf{34.90} & \textbf{36.95} & \textbf{40.75}& \textbf{41.40} \\
		
		\bottomrule
	\end{tabular}%
	
	\label{tab:replica_per_scene}
\end{table*}


\begin{table*}[t!]
	\centering
	\caption{Differences among different Surface \& Color reconstruction models.}
	\small
	\setlength{\tabcolsep}{.6em}
	%{
		%\resizebox{\textwidth}{!}{
			\begin{tabular}{l | c c c c c c }
				\toprule
				Method & 
				\begin{tabular}{@{}c@{}}Pre-Train\\ with extra dataset\end{tabular}
				& \begin{tabular}{@{}c@{}}Train\\ with sequences\end{tabular}
				& Real-time	
				& Direct Output &  \begin{tabular}{@{}c@{}}Light\\ direction\end{tabular} 
				&Render\\
				\hline 
				TSDF-Fusion & None & None & $\checkmark$& Discrete TSDF &  &Ray Rasterization\\\hline
				$\sigma$-Fusion & None & None &$\checkmark$&Discrete TSDF  && Ray Rasterization\\\hline
				iMAP & None & Online Train && MLPs  & &Volumetric Rendering\\\hline
				NICE-SLAM & \begin{tabular}{@{}c@{}}Pretrain\\ with indoor dataset\end{tabular} & Online Train&& Neural Implicit Grid&  & Volumetric Rendering\\\hline
				
				NeRF-SLAM & None & Train hundred epochs &-&NeRF & $\checkmark$ &Volumetric Rendering \\\hline
				
				\textbf{Uni-Fusion} & None & None&$\checkmark$& Latent Implicit Map && Ray Rasterization\\				
				\hline
			\end{tabular}
		%}
	%}
	\label{tab:replica_diff}
\end{table*}
\subsubsection{Evaluation on Replica RGB-D Dataset~\cite{sucar2021imap}}
In this evaluation, we compare with implicit reconstruction (TSDF-Fusion, $\sigma$-Fusion) and latent implicit reconstruction models (iMAP, NICE-SLAM) that support colors. 
We also add a large-scale NeRF model, NeRF-SLAM in to the table.  
NeRF is SOTA in view-synthesis task, which is unfair to direct compare with the rest. As the rest model does not even model light directions.
We add NeRF in this part to demonstrate that Uni-Fusion strongly reduce the gap.
Note that, NeRF-SLAM embeds external tracking model ~\cite{teed2021droid} to provide poses while using SOTA NeRF implementation Instance-ngp~\cite{muller2022instant} for NeRF construction.
%Therefore it is considered the SOTA to model the colors.

Uni-Fusion track and follow our previous setting in ScanNet test to take every 10 frames for mapping.
NICE-SLAM and NeRF-SLAM produce depth and color by rendering,
To obtain result from Uni-Fusion, we cast rays from virtual camera to our result surface mesh for depth image. 
Then Uni-Fusion infer the cast points in Uni-Fusion's color LIM for color result.

From~\cref{tab:replica_per_scene}, Uni-Fusion demonstrate
best Depth L1 on all scenes with an average of \textbf{$\pmb{1.47}$$\si{\centi\meter}$ depth L1}. Which is \textbf{$\pmb{177\%}$ boost} comparing to the second best.

Moreover, excluding NeRF, our Uni-Fusion also shows the best performance to model the colors with an average of $28.07$$\si{\dB}$ PSNR.

However, it is strange that NICE-SLAM lost details while in two cases, it shows better PSNR than Uni-Fusion. 
To highlight the true result,
we plot the rendered image in \cref{fig:replica_render}.
It is obvious that our Uni-Fusion models the details of painting, carpet and quilt well. 
While NICE-SLAM just roughly models the average color.

Moreover, from the  \cref{fig:replica_render}, our Uni-Fusion's rendering quality is as precise as NeRF. 
Please also find the painting, carpet and quilt, Uni-Fusion recovered the original appearance well.
Please find the {\color{green} green window} for the emphasized region.
Uni-Fusion reproduce the high-quality appearance which is very close to NeRF on qualitative evaluation.
%It can hardly find difference between the results from NeRF-SLAM, Uni-Fusion and Ground Truth.
%
But, Uni-Fusion still has a quantitative score gap to the color rending of NeRF ($41.4$$\si{\dB}$).
Though the Uni-Fusion's rendering result is highly close to NeRF and ground truth.
%
We consider the main reasons are that \textbf{1.} Uni-Fusion does not model the light directions to points, which is essential to NeRF.
\textbf{2.} NeRF optimizes on the rendering image quality by focussing mainly on color while less on depth.
It can be revealed by the higher color rendering score with much worse depth rendering than our Uni-Fusion.
\textbf{3.} our Uni-Fusion does not support hole filling.
This directly leads to black holes in our rendered images.

We summarize the differences in \cref{tab:replica_diff}.
Similar to TSDF-Fusion and $\sigma$-Fusion, our Uni-Fusion is a forward method which, does not need any training, i.e., pre- or online training. 
Uni-Fusion also produces similar to NICE-SLAM and NeRF-SLAM an implicit map with set of latent that outputs results at arbitrary resolution.
However, we differ on the extracting of the signed distance field.
%FIXME: I do not understand the next sentence.
Uni-Fusion's latent feature rule its own region independently.
Each query value is directly inferred with the corresponding ruling latent.
While NICE-SLAM and NeRF-SLAM use a much denser grid to interpolate query features. This requires volumetric rendering for inference.

Similar to TSDF-Fusion, $\sigma$-Fusion, our Uni-Fusion is also a real-time algorithm.
iMAP, NICE-SLAM and NeRF-SLAM run hardly in real-time.
NeRF-SLAM is claiming to be real-time, which is questionable as they still need hundreds of epochs training after feeding the data.

Nevertheless, optimization with backpropagation learns pixel-to-pixel well.
It is theoretically advanced for a regression-and-fusion strategy. 
Though Uni-Fusion demonstrates its high capability to model the color, NeRF-like post-optimization would still be a good direction for further improvements of Uni-Fusion.

\newcommand{\replicaImSize}{.24}
\begin{figure*}[t]
	\centering
	\setlength{\tabcolsep}{0.1em}
	\renewcommand{\arraystretch}{.1}
	\begin{tabular}{|c | c |c |c| }
		 \hline
		{\Large{NICE-SLAM}} &{\Large{NeRF-SLAM}}&\textbf{\Large{Uni-Fusion}}&\Large{Ground Truth}\\
		%		\hline
		%		\includegraphics[width=\replicaImSize\linewidth]{im/exp/recons/replica/nice-slam/of2_1286.png} &
		%		\includegraphics[width=\replicaImSize\linewidth]{im/exp/recons/replica/mine/of2_1286.jpg} &
		%		\includegraphics[width=\replicaImSize\linewidth]{im/exp/recons/replica/mine/of2_1286.jpg} &
		%		\includegraphics[width=\replicaImSize\linewidth]{im/exp/recons/replica/gt/of2_1286.jpg} \\
		
		\hline
		\includegraphics[width=\replicaImSize\linewidth]{im/exp/recons/replica/nice-slam/rm0_769_window.png} &
		\includegraphics[width=\replicaImSize\linewidth]{im/exp/recons/replica/nerf-slam/rm0_769_window.jpg} &
		\includegraphics[width=\replicaImSize\linewidth]{im/exp/recons/replica/mine/rm0_769_window.jpg} &
		\includegraphics[width=\replicaImSize\linewidth]{im/exp/recons/replica/gt/rm0_769_window.jpg} \\
		\hline
		\includegraphics[width=\replicaImSize\linewidth]{im/exp/recons/replica/nice-slam/of3_575_window.png} &
		\includegraphics[width=\replicaImSize\linewidth]{im/exp/recons/replica/nerf-slam/of3_575_window.jpg} &
		\includegraphics[width=\replicaImSize\linewidth]{im/exp/recons/replica/mine/of3_575_window.jpg} &
		\includegraphics[width=\replicaImSize\linewidth]{im/exp/recons/replica/gt/of3_575_window.jpg} \\
		\hline
		\includegraphics[width=\replicaImSize\linewidth]{im/exp/recons/replica/nice-slam/rm1_425_window.png} &
		\includegraphics[width=\replicaImSize\linewidth]{im/exp/recons/replica/nerf-slam/rm1_425_window.jpg} &
		\includegraphics[width=\replicaImSize\linewidth]{im/exp/recons/replica/mine/rm1_425_window.jpg} &
		\includegraphics[width=\replicaImSize\linewidth]{im/exp/recons/replica/gt/rm1_425_window.jpg} \\
		\hline
		\includegraphics[width=\replicaImSize\linewidth]{im/exp/recons/replica/nice-slam/rm2_1085_window.png} &
		\includegraphics[width=\replicaImSize\linewidth]{im/exp/recons/replica/nerf-slam/rm2_1085_window.jpg} &
		\includegraphics[width=\replicaImSize\linewidth]{im/exp/recons/replica/mine/rm2_1085_window.jpg} &
		\includegraphics[width=\replicaImSize\linewidth]{im/exp/recons/replica/gt/rm2_1085_window.jpg} \\		
		
	\end{tabular}
	%\captionof{figure}
	\caption{Demonstration of color rendering on the Replica dataset. Fine appearances are highlighted in {\color{green}green window}. Small flaws are in a {\color{red}red} box.}
	\label{fig:replica_render}
	\vspace{-.5cm}
\end{figure*}

%(2) NeRF model learning radiance field that model the light on different direction on surface. While Uni-Fusion naturally treat different directional light the same color.

%\begin{table*}[t!]
%	\centering
%	\setlength{\tabcolsep}{0.1em}
%	\renewcommand{\arraystretch}{.1}
%	\begin{tabular}{c | c |c |c |c }
%		\hline 
%		\rotatebox{90}{\large{NICE-SLAM}} &
%		\includegraphics[width=\replicaImSize\linewidth]{im/exp/recons/replica/nice-slam/of3_575.png} &
%		\includegraphics[width=\replicaImSize\linewidth]{im/exp/recons/replica/nice-slam/rm0_769.png} &
%		\includegraphics[width=\replicaImSize\linewidth]{im/exp/recons/replica/nice-slam/rm1_425.png} &
%		\includegraphics[width=\replicaImSize\linewidth]{im/exp/recons/replica/nice-slam/rm2_1085.png} \\
%		\hline
%		\rotatebox{90}{\large{NeRF-SLAM}} &
%		\includegraphics[width=\replicaImSize\linewidth]{im/exp/recons/replica/mine/of3_575.jpg} &
%		\includegraphics[width=\replicaImSize\linewidth]{im/exp/recons/replica/mine/rm0_769.jpg} &
%		\includegraphics[width=\replicaImSize\linewidth]{im/exp/recons/replica/mine/rm1_425.jpg} &
%		\includegraphics[width=\replicaImSize\linewidth]{im/exp/recons/replica/mine/rm2_1085.jpg} \\	
%		\hline
%		\rotatebox{90}{\textbf{\Large{Uni-Fusion}}} &
%		\includegraphics[width=\replicaImSize\linewidth]{im/exp/recons/replica/mine/of3_575.jpg} &
%		\includegraphics[width=\replicaImSize\linewidth]{im/exp/recons/replica/mine/rm0_769.jpg} &
%		\includegraphics[width=\replicaImSize\linewidth]{im/exp/recons/replica/mine/rm1_425.jpg} &
%		\includegraphics[width=\replicaImSize\linewidth]{im/exp/recons/replica/mine/rm2_1085.jpg} \\
%		\hline
%		\rotatebox{90}{\large{Ground Truth}} &
%		\includegraphics[width=\replicaImSize\linewidth]{im/exp/recons/replica/gt/of3_575.jpg} &
%		\includegraphics[width=\replicaImSize\linewidth]{im/exp/recons/replica/gt/rm0_769.jpg} &
%		\includegraphics[width=\replicaImSize\linewidth]{im/exp/recons/replica/gt/rm1_425.jpg} &
%		\includegraphics[width=\replicaImSize\linewidth]{im/exp/recons/replica/gt/rm2_1085.jpg} \\
%		\hline		
%		
%	\end{tabular}
%	\captionof{figure}{Demonstration of color rendering on Replica dataset.}
%\end{table*}

\subsection{Ablation study}
\label{exp:surface:ablation}

\begin{figure}[]
	\centering
%		\subfloat[width=\textwidth][Sample based]{
%		\centering
%		\includegraphics[width=.7\linewidth]{im/exp/ablation/GPIS/seq3_sample_color.png}
%	}\\
%	\subfloat[width=\textwidth][Derivative based]{
%		\centering
%		\includegraphics[width=.7\linewidth]{im/exp/ablation/GPIS/seq3_derivative_color.png}
%	}
		\includegraphics[width=.7\linewidth]{im/exp/ablation/GPIS/seq3_sample_color_a.png}
		\includegraphics[width=.7\linewidth]{im/exp/ablation/GPIS/seq3_derivative_color_b.png}
	\caption{Ablation study on surface construction basis. (a) Sample based. (b) Derivative based.}
	\label{fig:ablation:GPIS}
\end{figure}


\begin{table}[]
	\caption{Ablation study on tracking.
	}
	\centering
	\footnotesize
	\setlength{\tabcolsep}{0.7em}
	\resizebox{\linewidth}{!}{
		\begin{tabular}{l|ccc}
			\hline
			& \tt{fr1/desk} &  \tt{fr2/xyz} &  \tt{fr3/office} \\
			\hline
			External& 2.1& 0.5& 2.5 \\
			External+Internal&1.8& 0.5& 2.1 \\
			\hline
	\end{tabular}}
	\vspace{-2pt}
	%\vspace{-1cm}
	\label{tab:tum_rmse2}
\end{table}

\begin{figure}
	\centering
	\includegraphics[width=.7\linewidth]{im/exp/ablation/voxel_size/seq_voxel_size.png}
	%	\subfloat[width=.33\textwidth][0.1]{
		%		\centering
		%		\includegraphics[width=.33\linewidth]{im/exp/ablation/voxel_size/seq3_0_1_color.png}
		%	}
	%	\subfloat[width=.3\textwidth][0.05]{
		%		\centering
		%		\includegraphics[width=.33\linewidth]{im/exp/ablation/GPIS/seq3_sample_color.png}
		%	}
	%	\subfloat[width=.33\textwidth][0.02]{
		%	\centering
		%	\includegraphics[width=.33\linewidth]{im/exp/ablation/voxel_size/seq3_0_02_color.png}
		%	}
	\caption{Ablation study on voxel size.}
	\label{fig:ablation:voxel_size}
\end{figure}




 \newcommand{\styleImSize}{.2}
\begin{figure*}[b!]
	\vspace{-.5cm}
	\centering
	\setlength{\tabcolsep}{0.1em}
	\renewcommand{\arraystretch}{.1}
	\resizebox{\textwidth}{!}{\begin{tabular}{ccccc}
			%		\includegraphics[width=\styleImSize\line]{im/exp/style/style/0} &
			%		\includegraphics[width=\styleImSize\linewidth]{im/exp/style/style/1} &
			%		\includegraphics[width=\styleImSize\linewidth]{im/exp/style/style/2} &
			%		\includegraphics[width=\styleImSize\linewidth]{im/exp/style/style/3} &
			%		\includegraphics[width=\styleImSize\linewidth]{im/exp/style/style/4} &
			%		\includegraphics[width=\styleImSize\linewidth]{im/exp/style/style/5} &
			%		\includegraphics[width=\styleImSize\linewidth]{im/exp/style/style/6} \\
			\hline\hline
			\includegraphics[width=\styleImSize\linewidth]{im/exp/style/processed/office0_0.png} &
			\includegraphics[width=\styleImSize\linewidth]{im/exp/style/processed/office0_1.png} &
			\includegraphics[width=\styleImSize\linewidth]{im/exp/style/processed/office0_2.png} &
			\includegraphics[width=\styleImSize\linewidth]{im/exp/style/processed/office0_3.png} &
			\includegraphics[width=\styleImSize\linewidth]{im/exp/style/processed/office0_4.png} \\
			\includegraphics[width=\styleImSize\linewidth]{im/exp/style/processed/office0_5.png} &
			\includegraphics[width=\styleImSize\linewidth]{im/exp/style/processed/office0_6.png} &
			\includegraphics[width=\styleImSize\linewidth]{im/exp/style/processed/office0_7.png} &
			\includegraphics[width=\styleImSize\linewidth]{im/exp/style/processed/office0_8.png} &
			\includegraphics[width=\styleImSize\linewidth]{im/exp/style/processed/office0_9.png} \\
			\includegraphics[width=\styleImSize\linewidth]{im/exp/style/processed/office0_10.png} &
			\includegraphics[width=\styleImSize\linewidth]{im/exp/style/processed/office0_11.png} &
			\includegraphics[width=\styleImSize\linewidth]{im/exp/style/processed/office0_12.png} &
			\includegraphics[width=\styleImSize\linewidth]{im/exp/style/processed/office0_13.png} &
			\includegraphics[width=\styleImSize\linewidth]{im/exp/style/processed/office0_14.png} \\
			\includegraphics[width=\styleImSize\linewidth]{im/exp/style/processed/office0_15.png} &
			%\includegraphics[width=\styleImSize\linewidth]{im/exp/style/processed/office0_16.png} &
			\includegraphics[width=\styleImSize\linewidth]{im/exp/style/processed/office0_17.png} &
			\includegraphics[width=\styleImSize\linewidth]{im/exp/style/processed/office0_18.png} &
			\includegraphics[width=\styleImSize\linewidth]{im/exp/style/processed/office0_19.png} &
			\includegraphics[width=\styleImSize\linewidth]{im/exp/style/processed/office0_20.png} 
			\\	\hline
		\end{tabular}
	}
	%\captionof{figure}
	\caption{Style transfer on 3D canvas.}
	\label{fig:style}
\end{figure*}


\subsubsection{ Sample-based or Derivative-based}

We select the surface model with our own captured sequences. 
All settings are detailed in \cref{sec:exp:details}.
As shown in~\cref{fig:ablation:GPIS}, reconstruction of Yijun's office is demonstrated. 
Both models are able to construct, but the derivative-based model produces a lot of noise on the surface.
This is because for smoothness purpose, we build voxels that are overlapped to its neighbor, which causes redundant voxels near the surface.
For those redundant voxels, no center sample is provided and thus the derivative based surface construction builds bad SDFs on unknow region of the voxels.


Instead, sample-based surface construction does not have this problem as it adds more points in voxels and is able to construct highly-smooth surfaces.
From which, we find well constructed and colored white board, chair, school bag and even the oranges.

\subsubsection{Tracking}


Our Uni-Fusion use a coarse-to-fine strategy for tracking. 
An external tracking model is running in one thread aside from the mapping thread.
In the mapping thread, it takes pose result from the external tracking and applies the internal tracking for colored point cloud.

The result is demonstrated in~\cref{tab:tum_rmse2}. 
The coarse-to-fine is relatively better on trajectory estimation.

\subsubsection{Voxel size}

Testing the office scene, we vary the voxel size from low to high. 
From~\cref{fig:ablation:voxel_size}, when low voxel size $0.02$m is used, the surface is rough.
Then when voxel size goes larger, the smoothness is improved.
However, when we use $0.1$m voxel size, the surface color is blur. 
Considering Uni-Fusion produces a surface color field, the quality of surface directly affect the coloring.
Thus, continuing enlarging the voxel size also results in worse surface results.

Therefore, in the above experiments, $0.05$m voxel size is utilized for surface construction.
In addition, each voxel for encoding are actually with size $0.1$m, since we use overlapped voxel.

%\subsubsection{Anchor number and feature dimension}


\subsection{Application: 2D-to-3D Transfer}
\label{sec:fabircated_prop}

Applications such as 2) and 4) can be easily integrated with application 1) incremental reconstruction (\cref{sec:incremental_reconstruction}) by incorporating the fabricated result together with the point cloud.
%
For instance, given RGB-D frames, we detect saliency or transfer image styles to generate a fabricated $X$ image. Here, $X$ represents saliency, style, or other properties. 
By combining $X$ with depth information through unprojection,
we assign
the fabricated values to corresponding points, resulting in point pairs ($\V X$, $\V Q_{X}$).

Similar to the reconstruction pipeline in~\cref{fig:recons_and_scene_understanding}, we employ encoding (\cref{sec:encoder}) and fusion (\cref{eq:fuse}) to construct a global LIM for the fabricated properties $X$.
This global LIM represents a surface $X$ fields that is utilized for subsequent inference.

While it is possible to similarly transfer a 2D semantic image to 3D,
it may not be feasible in practice due to the need for multiple passes of different categories of semantic information 
 on the same dataset (such as object, usability, etc.).
Therefore, in the following section, we demonstrate the construction of a surface feature field for scene understanding application that satisfies various 
requirements through a single mapping pass.

\begin{table*}[b!]
	%\renewcommand{\arraystretch}{0.9}
	%\setlength{\tabcolsep}{3pt}
	\caption{GZSL semantic segmentation results. Scores are in \%.
	  $^\dagger$ indicate 3DGenZ's adaption of the method.
       Note that, Uni-Fusion-SU does not even train with the seen classes.}
	\centering
	\begin{tabular}{l|c|c |c ||ccc|ccc}
		\toprule
		\multicolumn{1}{c}{}& \multicolumn{2}{c|}{Training set} & Inference input &\multicolumn{3}{c|}{ScanNet } & \multicolumn{3}{c}{S3DIS}\\
		& Backbone & Classifier & &$Seen$& $Unseen$ & $All$&$Seen$& $Unseen$ & $All$
%		\multicolumn{3}{c|}{mIoU} & 
%		\multicolumn{3}{c|}{mIoU} \\ 
%		&&&& $Seen$& $Unseen$ & $All$&$Seen$& $Unseen$ & $All$
		%\cellcolor{white}{}  
		%\cellcolor{white}{}
		\\
		\midrule
		
		\multicolumn{5}{l}{\textit{Supervised methods with different levels of supervision}}\\
		
		Full supervision & $seen \cup unseen$ & $seen \cup unseen$ & Point Cloud &43.3&51.9 &45.1&74.0&50.0&66.6 \\
		
		ZSL backbone & $seen$ & $seen \cup unseen$  &Point Cloud&41.5&39.2 & 40.3&60.9& 21.5&  48.7 \\
		
		ZSL-trivial & $seen$ & $seen$ &Point Cloud&39.2&0.0&31.3&70.2 &0.0&48.6  \\
		\midrule
		\multicolumn{5}{l}{\textit{Generalized zero-shot-learning methods}}\\
		
		ZSLPC-Seg~\cite{cheraghian2019zero}$^\dagger$ & $seen$ & $unseen$  &Point Cloud&28.2&0.0& 22.6&65.6 &0.0& 45.3\\
		
		DeViSe-3DSeg~\cite{frome2013devise}$^\dagger$ & $seen$ & $unseen$   &Point Cloud &20.0&0.0&16.0&70.2&0.0& 48.6\\ 
		%ZSLPC-Seg~\cite{cheraghian2019zero}$^\dagger$ & $seen$ & $unseen$  &  4.0&13.9\\
		%DeViSe-3DSeg~\cite{frome2013devise}$^\dagger$ & $seen$ & $unseen$   &  3.0&10.9\\
		3DGenZ~\cite{michele2021generative} & $seen$ & $seen \cup \hat{unseen}$  &Point Cloud &32.8&7.7& {27.8}&53.1&7.3&   \textbf{39.0} \\
		\midrule
		\multicolumn{5}{l}{\textit{Zero-shot learning + map fusion}}\\
		Uni-Fusion-SU (Ours) &None&None&Sparse Frames&31.0&\textbf{41.9}&\textbf{32.9} &31.3&\textbf{24.0}&29.0\\
		\bottomrule
		\multicolumn{1}{l}{}\\[-7pt]
	\end{tabular}

	\label{tab:sem_seg_overview}
\end{table*}

\begin{figure*}[t!]
	\centering
	\setlength{\tabcolsep}{0.1em}
	\renewcommand{\arraystretch}{.1}
	\begin{tabular}{|c | c |c |||c |c | c|}
		\toprule
		{\Large{3DGenZ}} & {\Large{Uni-Fusion}} &{\Large{Ground Truth}} & {\Large{3DGenZ}} &{\Large{Uni-Fusion-SU}} & {\Large{Ground Truth}} \\ \midrule
		
		\includegraphics[width=\scannetImSize\linewidth]{im/exp//ss/gen3dz_0568.png}
		&\includegraphics[width=\scannetImSize\linewidth]{im/exp//ss/mine_0568.png}
		&\includegraphics[width=\scannetImSize\linewidth]{im/exp//ss/gt_0568.png}
		&		\includegraphics[width=\scannetImSize\linewidth]{im/exp//ss/gen3dz_0164.png}
		&\includegraphics[width=\scannetImSize\linewidth]{im/exp//ss/mine_0164.png}
		&\includegraphics[width=\scannetImSize\linewidth]{im/exp//ss/gt_0164.png}\\
		
		
		\includegraphics[width=\scannetImSize\linewidth]{im/exp//ss/gen3dz_0249.png}
		&\includegraphics[width=\scannetImSize\linewidth]{im/exp//ss/mine_0249.png}
		&\includegraphics[width=\scannetImSize\linewidth]{im/exp//ss/gt_0249.png}
		&		\includegraphics[width=\scannetImSize\linewidth]{im/exp//ss/gen3dz_0435.png}
		&\includegraphics[width=\scannetImSize\linewidth]{im/exp//ss/mine_0435.png}
		&\includegraphics[width=\scannetImSize\linewidth]{im/exp//ss/gt_0435.png}\\
		
		
		\includegraphics[width=\scannetImSize\linewidth]{im/exp//ss/gen3dz_0046.png}
		&\includegraphics[width=\scannetImSize\linewidth]{im/exp//ss/mine_0046.png}
		&\includegraphics[width=\scannetImSize\linewidth]{im/exp//ss/gt_0046.png}
		&		\includegraphics[width=\scannetImSize\linewidth]{im/exp//ss/gen3dz_0050.png}
		&\includegraphics[width=\scannetImSize\linewidth]{im/exp//ss/mine_0050.png}
		&\includegraphics[width=\scannetImSize\linewidth]{im/exp//ss/gt_0050.png}\\
		\bottomrule
		
	\end{tabular}
	\includegraphics[width=\linewidth]{im/ss_colorbar}
	%\captionof{figure}
	\caption{Demonstration of semantic segmentation on the ScanNet dataset.
       Selected scenes are consistent with~\cref{fig:recons:scannet_demo}}
	\label{fig:segmentation_demo}
	
\end{figure*}

\subsection{Scene Understanding Results}

Saliency detection effectively highlights the objects of interest.
This is also considered part of 3D semantic understanding.
However, as the semantics categories vary, fusing different categories of semantics into multiple LIMs can be inefficient.
%
Therefore, in this section, we utilize Uni-Fusion to fuse and construct a surface field for high-dimensional CLIP embeddings.
With a single LIM, we can generate different semantic results based on corresponding commands.
%
Since now our Uni-Fusion works with OpenSeg for scene understanding purposes, we call it Uni-Fusion-SU.

\subsubsection{Semantic Segmentation}
\label{sec:exp:semantic}

We first evaluate our model on generalized zero-shot point cloud semantic segmentation application.
Generalized Zero-Shot Learning (GZSL) differs from Zero-Shot Learning (ZSL) in that ZSL only predicts classes unseen during training, while GZSL predicts both unseen and seen classes~\cite{michele2021generative}.
Therefore, comparing our results with GZSL SOTAs provides a better understanding of the potential of Uni-Fusion-SU, as it does not train on both seen and unseen. 

This test uses ScanNet and S3DIS datasets for benchmarking. 
It is important to note that the \textbf{compared baselines are trained on the corresponding datasets}.
Our Uni-Fusion-SU uses OpenSeg to provide the 2D image level feature ebmedding.
Although \textbf{Uni-Fusion-SU} is also zero-shot, \textbf{it does not touch any ScanNet or S3DIS annotations}.

We demonstrate the mIoU scores in~\cref{tab:sem_seg_overview}.
In particular, our model achieves best results among the zero-shot learning methods on the ScanNet dataset and remains competitive with fully supervised methods.

Furthermore, we provide results specifically for the unseen classes in~\cref{sup:tab:sn_acc_miou}.
Although not as good as the fully supervised approach, Uni-Fusion-SU performs much better than 3DGenZ.
In addition, our Uni-Fusion-SU demonstrates high precision in classes such as sofa and Toilet, even when compared to the fully supervised model.

\begin{table}[htbp]
		\caption{Classwise GZSL semantic segmentation performance (\%) on the ScanNet unseen split.}
	\centering
	\newcommand*\rotext{\multicolumn{1}{R{45}{1em}}}
	\setlength{\tabcolsep}{1.7pt}
	\begin{tabular}{@{}l@{~}c|rrrr|r@{}}
		\toprule		
		& &
		{\textbf{Bookshelf}} & {\textbf{Desk}} & {\textbf{Sofa}} & {\textbf{Toilet}} & \stackbox{mean} \\
		
		\midrule
		FSL (Fully supervise) & IoU & 	56.9&	30.0&	57.4&	63.4 & 51.9
		\\ 
		3DGenZ (Zero-shot) & IoU & 	6.3&	3.3&	13.1&	8.1 & 7.7
		\\
		Uni-Fusion-SU (Ours) & IoU &38.3&16.8&51.7&60.9&41.9
	\\ \midrule 
	3DGenZ (Zero-shot)& Acc. & 	13.4&	5.9&	49.6&	26.3 &23.8
	\\
	Uni-Fusion-SU (Ours) & Acc. &61.9&29.6&67.4&91.6& 62.6
		\\
		\bottomrule
	\end{tabular}

	\label{sup:tab:sn_acc_miou}
\end{table}

However, in the S3DIS dataset, our model does not outperform 3DGenZ and other methods as shown in~\cref{tab:sem_seg_overview}.

Even in the result of unsceened data, as presented in \cref{sup:tab:s3dis_acc_miou}, we observe that Uni-Fusion-SU hardly finds some classed, e.g. Beam and Column, which are not commonly annotated objects. 
However, for common objects like sofa and window, our model performs much better.

\begin{table}[htbp]
		\caption{Classwise GZSL semantic segmentation performance (\%) on the S3DIS unseen split.}
	\centering
	\newcommand*\rotext{\multicolumn{1}{R{45}{1em}}}
	\setlength{\tabcolsep}{1.7pt}
	\begin{tabular}{@{}l@{~}c|rrrr|r@{}}
		\toprule		
		& &
		{\textbf{Beam}} & {\textbf{Column}} & {\textbf{Sofa}} & {\textbf{Window}} & \stackbox{mean} \\
		
		\midrule
		FSL (Fully supervise) & IoU & 	63.1&	10.2&	54.1&	72.4 & 50.0
		\\ 
		3DGenZ (Zero-shot) & IoU & 	13.9&	2.4&4.9&	8.1 &7.3
		\\
		Uni-Fusion-SU (Ours) & IoU &5.5&0.02&57.4&32.9&	24.0
		\\ \midrule 
		3DGenZ (Zero-shot) & Acc. & 	20.0&	9.1&	62.4&	23.7 &28.8
		\\
		Uni-Fusion-SU (Ours) & Acc. &41.5&0.02&78.3&42.1& 40.5
		\\	
		\bottomrule
	\end{tabular}

	\label{sup:tab:s3dis_acc_miou}
\end{table}

We present the results of the semantic segmentation in~\cref{fig:segmentation_demo}. 
It is evident that, 3DGenZ's result contains more noise, as seen in the spotted sofa, bed and other objects.
Conversely, Uni-Fusion-SU's results are generally smoother and more precise.

%
%\begin{figure*}[htbp]
%	\centering
%	\includegraphics[width=.3\linewidth]{example-image-golden}
%	\includegraphics[width=.3\linewidth]{example-image-golden}
%	\includegraphics[width=.3\linewidth]{example-image-golden}
%	\\
%	\includegraphics[width=.3\linewidth]{example-image-golden}
%	\includegraphics[width=.3\linewidth]{example-image-golden}
%	\includegraphics[width=.3\linewidth]{example-image-golden}
%	
%	\caption{Semantic segmentation result on ScanNet.}
%\end{figure*}
%
%\begin{figure*}[htbp]
%	\centering
%	\includegraphics[width=.3\linewidth]{example-image-golden}
%	\includegraphics[width=.3\linewidth]{example-image-golden}
%	\includegraphics[width=.3\linewidth]{example-image-golden}
%	\\
%	\includegraphics[width=.3\linewidth]{example-image-golden}
%	\includegraphics[width=.3\linewidth]{example-image-golden}
%	\includegraphics[width=.3\linewidth]{example-image-golden}
%	
%	\caption{Semantic segmentation result on S3DIS.}
%\end{figure*}

\subsubsection{Scene Understanding with Different Properties}

\begin{figure*}[t!]
	\centering
	\setlength{\tabcolsep}{0.1em}
	\renewcommand{\arraystretch}{.1}
	\resizebox{\textwidth}{!}{\begin{tabular}{|c | c | c | c | c | c|}
			\toprule 
			& \textbf{scene0568\_00} & \textbf{scene0249\_00} & \textbf{scene0435\_00} & \textbf{office3} & \textbf{room0}\\
			\midrule
			{} &
			\raisebox{-.5\height}{\includegraphics[width=\fabImSize\linewidth]{im/exp/fab/scannet/0568_color.png}} & %\raisebox{-.5\height}{\includegraphics[width=\fabImSize\linewidth]{im/exp/fab/scannet/0164_color.png}} &
			\raisebox{-.5\height}{\includegraphics[width=\fabImSize\linewidth]{im/exp/fab/scannet/0249_color.png}} & \raisebox{-.5\height}{\includegraphics[width=\fabImSize\linewidth]{im/exp/fab/scannet/0435_color.png}}
			&
			\raisebox{-.5\height}{\includegraphics[width=\fabImSize\linewidth]{im/exp/fab/replica/office3_color.png}}
			&
			\raisebox{-.5\height}{\includegraphics[width=\fabImSize\linewidth]{im/exp/fab/replica/room0_color.png}}\\ %\raisebox{-.5\height}{\includegraphics[width=\fabImSize\linewidth]{im/exp/fab/scannet/0050_color.png}} %\includegraphics[width=\fabImSize\linewidth]{im/exp/fab/replica/office3_color.png}
			\\
			\textbf{Desk}  &
			\raisebox{-.5\height}{\includegraphics[width=\fabImSize\linewidth]{im/exp/fab/scannet/0568_lt_desk.png}}&
			%\raisebox{-.5\height}{\includegraphics[width=\fabImSize\linewidth]{im/exp/fab/scannet/0164_lt_desk.png}}&
			\raisebox{-.5\height}{\includegraphics[width=\fabImSize\linewidth]{im/exp/fab/scannet/0249_lt_desk.png}}&
			\raisebox{-.5\height}{\includegraphics[width=\fabImSize\linewidth]{im/exp/fab/scannet/0435_lt_desk.png}}
			&
			\raisebox{-.5\height}{\includegraphics[width=\fabImSize\linewidth]{im/exp/fab/replica/office3_lt_desk.png}}
			&
			\raisebox{-.5\height}{\includegraphics[width=\fabImSize\linewidth]{im/exp/fab/replica/room0_lt_desk.png}}\\
			%\raisebox{-.5\height}{\includegraphics[width=\fabImSize\linewidth]{im/exp/fab/scannet/0050_lt_desk.png}}
			%\includegraphics[width=\fabImSize\linewidth]{im/exp/fab/replica/office3_saliency.png}
			\\
			
			\textbf{Sofa} &
			\raisebox{-.5\height}{\includegraphics[width=\fabImSize\linewidth]{im/exp/fab/scannet/0568_lt_sofa.png}} &
			%\raisebox{-.5\height}{\includegraphics[width=\fabImSize\linewidth]{im/exp/fab/scannet/0164_lt_sofa.png}} &
			\raisebox{-.5\height}{\includegraphics[width=\fabImSize\linewidth]{im/exp/fab/scannet/0249_lt_sofa.png}} &
			\raisebox{-.5\height}{\includegraphics[width=\fabImSize\linewidth]{im/exp/fab/scannet/0435_lt_sofa.png}}&
			\raisebox{-.5\height}{\includegraphics[width=\fabImSize\linewidth]{im/exp/fab/replica/office3_lt_sofa.png}}
			&
			\raisebox{-.5\height}{\includegraphics[width=\fabImSize\linewidth]{im/exp/fab/replica/room0_lt_sofa.png}}\\
			%\raisebox{-.5\height}{\includegraphics[width=\fabImSize\linewidth]{im/exp/fab/scannet/0050_lt_sofa.png}}
			%\includegraphics[width=\fabImSize\linewidth]{im/exp/fab/replica/office3_style.png}
			\\
			\textbf{Work} &
			\raisebox{-.5\height}{\includegraphics[width=\fabImSize\linewidth]{im/exp/fab/scannet/0568_lt_work.png}} &
			%\raisebox{-.5\height}{\includegraphics[width=\fabImSize\linewidth]{im/exp/fab/scannet/0164_lt_work.png}} &
			\raisebox{-.5\height}{\includegraphics[width=\fabImSize\linewidth]{im/exp/fab/scannet/0249_lt_work.png}} &
			\raisebox{-.5\height}{\includegraphics[width=\fabImSize\linewidth]{im/exp/fab/scannet/0435_lt_work.png}}&
			\raisebox{-.5\height}{\includegraphics[width=\fabImSize\linewidth]{im/exp/fab/replica/office3_lt_work.png}}
			&
			\raisebox{-.5\height}{\includegraphics[width=\fabImSize\linewidth]{im/exp/fab/replica/room0_lt_work.png}}\\
			%\raisebox{-.5\height}{\includegraphics[width=\fabImSize\linewidth]{im/exp/fab/scannet/0050_lt_work.png}}
			%\includegraphics[width=\fabImSize\linewidth]{im/exp/fab/replica/office3_style.png}
			\\
			\textbf{Sittable} &
			\raisebox{-.5\height}{\includegraphics[width=\fabImSize\linewidth]{im/exp/fab/scannet/0568_lt_sit.png}} &
			%\raisebox{-.5\height}{\includegraphics[width=\fabImSize\linewidth]{im/exp/fab/scannet/0164_lt_sit.png}} &
			\raisebox{-.5\height}{\includegraphics[width=\fabImSize\linewidth]{im/exp/fab/scannet/0249_lt_sit.png}} &
			\raisebox{-.5\height}{\includegraphics[width=\fabImSize\linewidth]{im/exp/fab/scannet/0435_lt_sit.png}}&
			\raisebox{-.5\height}{\includegraphics[width=\fabImSize\linewidth]{im/exp/fab/replica/office3_lt_sit.png}}
			&
			\raisebox{-.5\height}{\includegraphics[width=\fabImSize\linewidth]{im/exp/fab/replica/room0_lt_sit.png}}\\
			%\raisebox{-.5\height}{\includegraphics[width=\fabImSize\linewidth]{im/exp/fab/scannet/0050_lt_sit.png}}
			%\includegraphics[width=\fabImSize\linewidth]{im/exp/fab/replica/office3_style.png}
			\\
			\textbf{Wood} &
			\raisebox{-.5\height}{\includegraphics[width=\fabImSize\linewidth]{im/exp/fab/scannet/0568_lt_wood.png}} &
			%\raisebox{-.5\height}{\includegraphics[width=\fabImSize\linewidth]{im/exp/fab/scannet/0164_lt_wood.png}} &
			\raisebox{-.5\height}{\includegraphics[width=\fabImSize\linewidth]{im/exp/fab/scannet/0249_lt_wood.png}} &
			\raisebox{-.5\height}{\includegraphics[width=\fabImSize\linewidth]{im/exp/fab/scannet/0435_lt_wood.png}}&
			\raisebox{-.5\height}{\includegraphics[width=\fabImSize\linewidth]{im/exp/fab/replica/office3_lt_wood.png}}
			&
			\raisebox{-.5\height}{\includegraphics[width=\fabImSize\linewidth]{im/exp/fab/replica/room0_lt_wood.png}}\\
			%\raisebox{-.5\height}{\includegraphics[width=\fabImSize\linewidth]{im/exp/fab/scannet/0050_lt_wood.png}}
			%\includegraphics[width=\fabImSize\linewidth]{im/exp/fab/replica/office3_style.png}
			\\
			
			
			\bottomrule
		\end{tabular}
	}
	%\captionof{figure}
	\caption{Demonstration of the original mesh, highlighted semantic mesh given various queries.}
	\label{fig:fab_lt}
	\vspace{-.5cm}
\end{figure*}

The main contribution of this application is that, Uni-Fusion is the first model to construct a continuous mapping of high-dimensional embeddings onto the surface without the need for any training of the map representation.
%
In the previous experiment (\cref{sec:exp:semantic}), we evaluate the performance of generalized zero-shot semantic segmentation.
However, the potential of Uni-Fusion goes beyond semantic segmentation.
%
By constructing a LIM, we obtain a surface CLIP feature field.
This enables us to query various semantic categories such as 
%without the need of multiple LIMs or rerun for other properties, we query 
\textbf{Object, Room Type, Material, Affordance and Activity} without requiring multiple LIMs or re-running the model.

We present the results in \cref{fig:fab_lt}, 
where we query object (desk, sofa), activity (work), affordance (sittable), and material (wood).
Uni-Fusion-SU accurately identifies and highlights the object and material regions.
However, for less specific commands such as work or sittable, the model provides a wider range of results with less confidence (indicated by dull yellow).
Nevertheless, the suggested options are also roughly correct.









\subsection{Time}

We run all of the applications in a single pass using our captured office sequences and evaluate the time cost of construction and fusion of each LIM. 
The average time cost across frames is shown in~\cref{tab:time}.

\begin{table}[htbp]
	\caption{Time required for each frame.
	}
	\centering
	\footnotesize
	\setlength{\tabcolsep}{0.7em}
	\resizebox{\linewidth}{!}{
		\begin{tabular}{l|ccccccc}
			\toprule
			&Surface & Color & Infrared & Style & Saliency & Latent&Internal Track \\ \midrule
			Time ($\si{\second}$)&0.100 & 0.038 & 0.045 & 0.048 & 0.045 &0.011 &0.225 \\ \bottomrule
	\end{tabular}}
	
	\label{tab:time}
\end{table}

\newcommand{\mineImSize}{.32}
%\begin{table*}[t!]
%	\centering
%	\setlength{\tabcolsep}{0.1em}
%	\renewcommand{\arraystretch}{.1}
%	\begin{tabular}{|c | c |c |}
%		\hline 
%		{Color} &{Infrared} & {Saliency} \\
%		\includegraphics[width=\mineImSize\linewidth]{im/exp/fab/mine/office/seq3_color.png} &
%		\includegraphics[width=\mineImSize\linewidth]{im/exp/fab/mine/office/seq3_color.png} &
%		\includegraphics[width=\mineImSize\linewidth]{im/exp/fab/mine/office/seq3_saliency.png} \\
%		{Style 1}&{Style 1}&{Style 1}\\
%		\includegraphics[width=\mineImSize\linewidth]{im/exp/fab/mine/office/seq3_style.png}&
%		\includegraphics[width=\mineImSize\linewidth]{im/exp/fab/mine/office/seq3_style.png}&
%		\includegraphics[width=\mineImSize\linewidth]{im/exp/fab/mine/office/seq3_style.png}\\
%		{Sofa}&{Desk}&{Soft}\\
%		\includegraphics[width=\mineImSize\linewidth]{im/exp/fab/mine/office/seq3_lt_sofa.png}&
%		\includegraphics[width=\mineImSize\linewidth]{im/exp/fab/mine/office/seq3_lt_desk.png}&
%		\includegraphics[width=\mineImSize\linewidth]{im/exp/fab/mine/office/seq3_lt_soft.png}\\		
%		\hline
%	\end{tabular}
%	\captionof{figure}{Demonstration on captured office data.}
%	\label{fig:mine_demo}
%\end{table*}
%\begin{figure*}[t!]
%	\centering
%	\setlength{\tabcolsep}{0.1em}
%	\renewcommand{\arraystretch}{.1}
%	\begin{tabular}{|c | c |c |}
%		\hline \hline
%		\includegraphics[width=\mineImSize\linewidth]{im/exp/fab/mine/office/seq3_w_slam_color.png}&	\includegraphics[width=\mineImSize\linewidth]{im/exp/fab/mine/office/seq3_w_slam_ir.png}&	\includegraphics[width=\mineImSize\linewidth]{im/exp/fab/mine/office/seq3_w_slam_saliency.png}\\
%		{Color} &{Infrared} & {Saliency}\\
%<<<<<<< HEAD


%=======
%		
%		\includegraphics[width=\mineImSize\linewidth]{im/exp/fab/mine/office/seq3_w_slam_style.png}
%		&\includegraphics[width=\mineImSize\linewidth]{im/exp/fab/mine/office/seq3_w_slam_lt_desk.png}
%		&\includegraphics[width=\mineImSize\linewidth]{im/exp/fab/mine/office/seq3_w_slam_lt_wood.png}\\
%		{Style} & {Object-desk} & {Material-wood} \\\hline
%>>>>>>> e014bc950c14dec9ffa1d2d7a6de9b7abfefabdd
%	\end{tabular}
%	%\captionof{figure}
%	\caption{Demonstration on captured Office data.}
%	\label{fig:office}
%\end{figure*}

\begin{figure*}[]
	\centering
	\setlength{\tabcolsep}{0.1em}
	\renewcommand{\arraystretch}{.1}
	\begin{tabular}{|c | c |c |}
	\hline \hline
	\includegraphics[width=\mineImSize\linewidth]{im/exp/fab/mine/appartment2/appartment2_color.png}&	\includegraphics[width=\mineImSize\linewidth]{im/exp/fab/mine/appartment2/appartment2_ir.png}&	\includegraphics[width=\mineImSize\linewidth]{im/exp/fab/mine/appartment2/appartment2_saliency.png}\\
		{Color} &{Infrared} & {Saliency}\\
	\includegraphics[width=\mineImSize\linewidth]{im/exp/fab/mine/appartment2/appartment2_style.png}
&\includegraphics[width=\mineImSize\linewidth]{im/exp/fab/mine/appartment2/appartment2_lt_sofa.png}
&\includegraphics[width=\mineImSize\linewidth]{im/exp/fab/mine/appartment2/appartment2_lt_desk.png}
\\
{Style} & {Object-sofa} & {Object-desk}\\
\includegraphics[width=\mineImSize\linewidth]{im/exp/fab/mine/appartment2/appartment2_lt_coat.png}
&\includegraphics[width=\mineImSize\linewidth]{im/exp/fab/mine/appartment2/appartment2_lt_sit.png}
&\includegraphics[width=\mineImSize\linewidth]{im/exp/fab/mine/appartment2/appartment2_lt_wood.png}\\
{Object-coat} & {Affordance-sit} & {Material-wood} \\\hline
	\end{tabular}
%\captionof{figure}
\caption{Demonstration on the captured apartment data.}
\label{fig:appartment}
%\vspace{-.5cm}
\end{figure*}


Using depth and property images of size $720\times1280$ as input, it is evident from the table, that our model operates at a frequency of $\sim10\si{\hertz}$ for  surface (sample mode) LIM construction and integration. 
It alse achieves a frequency of over $20\si{\hertz}$ for color, infrared, style, and saliency.
These results demonstrate the suitability of Uni-Fusion for real-time applications.

However, our internal tracking process takes around $0.225\si{\second}$ per frame, which is relatively slower compared to the mapping module. 
Nevertheless, Uni-Fusion uses external tracking to prevent tracking loss, enabling our internal tracking and mapping to operate at a lower frequency.
As a result, the entire model can be effectively applied in real-time in various scenarios.

\section{Extensive experiment on our own dataset}

In previous experiments, we evaluate the capabilities of Uni-Fusion in different applications. 
To further demonstrate its effectiveness in robotic environmental understanding, we capture our own dataset to show all applications together.

We capture two scenes: The office and apartment of the first author using a Microsoft Kinect Azure. 
%
RGB-D and infrared video are captured. After calibration, RGB, depth, infrared inputs have resolution of $720\times1280$.
Uni-Fusion tracks and reconstructs all applications in one pass.
%
While office data has been involved in ablation study (\cref{exp:surface:ablation}), we showcase all applications using the apartment dataset, as depicted in~\cref{fig:appartment}.

For better visualization, the ceiling of reconstruction is removed.
The top row of images presents the colored mesh with room details, the infrared mesh revealing the lighting effect, and the saliency reconstruction highlighting objects crucial for navigation.
Additionally, we select the second style from~\cref{fig:style} for style transfer to the apartment canvas.
%
As a result, the wooden floor in the room is colored with dark green.
The whole apartment is in a warm style.

The remaining results are generated from the surface field of the CLIP embeddings. 
We issue commands to locate objects, e.g., where is the sofa, desk and coat.
In addition, it easily identifies affordances such as being sittable.
For material, it successfully detects the wooden floor in each room.


\newpage
\section{Cosh Regression}\label{sec:cosh}
 
In this section, we provide detailed analysis of $L_{\cosh}$. In Section~\ref{sec:cosh:definition} we define the loss function $L_{\cosh}$ based on $\cosh(x)$. In Section~\ref{sec:cosh:gradient} we compute the gradient of $L_{\cosh}$ by detail. In Section~\ref{sec:cosh:hessian} we compute the hessian of $L_{\cosh}$ by detail. In Section~\ref{sec:cosh:gradient_hessian}, we summarize the result of Section~\ref{sec:cosh:gradient} and Section~\ref{sec:cosh:hessian} and aquire the gradient $\nabla L_{\cosh}$ and hessian $\nabla^2 L_{\cosh}$ for $L_{\cosh}$. In Section~\ref{sec:cosh:loss_and_reg} we define $L_{\cosh,\reg}$ by adding the regularization term $L_{\reg}$ in Section~\ref{sec:preli:regularization} to $L_{\cosh}$ and compute the gradient $\nabla L_{\cosh,\reg}$ and hessian $\nabla^2 L_{\cosh,\reg}$ of $L_{\cosh,\reg}$. In Section~\ref{sec:cosh:convex} we proved that $\nabla^2 L_{\cosh,\reg} \succ 0$ and thus showed that $L_{\cosh,\reg}$ is convex. In Section~\ref{sec:cosh:lipschitz} we provide the upper bound for $\|\nabla^2 L_{\cosh,\reg}(x)-\nabla^2 L_{\cosh,\reg}(y)\|$ and thus proved $\nabla^2 L_{\cosh,\reg}$ is lipschitz.
\subsection{Definition}\label{sec:cosh:definition}

\begin{definition}\label{def:L_cosh}
Given $A \in \R^{n \times d}$ and $b \in \R^n$. For a vector $x \in \R^d$, we define loss function $L_{\cosh}(x)$ as follows:
\begin{align*}
L_{\cosh}(x) := 0.5 \cdot \| \cosh(Ax) - b \|_2^2
\end{align*}
\end{definition}


\subsection{Gradient}\label{sec:cosh:gradient}

\begin{lemma}[Gradient for Cosh]\label{lem:gradient_cosh}
We have
\begin{itemize}
    
    \item Part 1.
    \begin{align*}
        \frac{ \d ( \cosh(Ax) - b ) }{ \d t } = \sinh(Ax) \circ \frac{A \d x }{ \d t}
    \end{align*}
    \item Part 2. 
    \begin{align*}
        \frac{\d L_{\cosh} }{ \d t} = (\cosh(Ax) - b )^\top \cdot ( \sinh(Ax) \circ \frac{A \d x}{ \d t } )
    \end{align*}
\end{itemize}
Further, we have for each $i \in [d]$
\begin{itemize}
    
    \item Part 3.
    \begin{align*}
        \frac{ \d ( \cosh(Ax) - b ) }{ \d x_i } = \sinh(Ax) \circ A_{*,i}
    \end{align*}
    \item Part 4. 
    \begin{align*}
        \frac{\d L_{\cosh} }{ \d x_i}  
        = & ~ (\cosh(Ax) - b )^\top \cdot ( \sinh(Ax) \circ A_{*,i} )
    \end{align*}
    \item Part 5.
    \begin{align*}
        \frac{\d L_{\cosh} }{ \d x} =  A^\top \diag( \sinh(Ax) ) (\cosh(Ax) - b)
    \end{align*}
\end{itemize}
\end{lemma}

\begin{proof}


{\bf Proof of Part 1.}

For each $i \in [n]$, we have
\begin{align*}
 \frac{ \d ( \cosh(Ax) - b )_i }{ \d t } 
 = & ~ \sinh(Ax)_i \cdot \frac{\d (Ax)_i}{\d t} \\
 = & ~ \sinh(Ax)_i \cdot \frac{ (A \d x)_i}{\d t} 
\end{align*}
where the first and second step follow from the differential chain rule.

Thus, we complete the proof.

{\bf Proof of Part 2.}

We have
 
\begin{align*}
    \frac{\d L_{\cosh} }{ \d t} 
    = & ~ (\cosh(Ax) - b )^\top \cdot \frac{ \d ( \cosh(Ax) - b ) }{\d t} \\
    = & ~ (\cosh(Ax) - b )^\top \cdot ( \sinh(Ax) \circ \frac{A \d x}{ \d t } )
\end{align*}
where the first step follows from the differential chain rule, and the second step follows from $\frac{ \d ( \cosh(Ax) - b ) }{ \d t } = \sinh(Ax) \circ \frac{A \d x }{ \d t}$ in Part 1.



{\bf Proof of Part 3.}

We have
\begin{align*}
    \frac{\d (\cosh(Ax)-b)}{\d x_i}
    = & ~ \frac{\d (\cosh(Ax))}{\d x_i} - \frac{\d b}{\d x_i}\\
    = & ~ \sinh(Ax) \circ \frac{\d Ax}{\d x_i} - 0\\
    = & ~ \sinh(Ax) \circ A_{*,i}
\end{align*}
where the first step follows from the property of the gradient, the second step follows from the differential chain rule, and the last step directly follows from Lemma~\ref{lem:Ax_gradient_hessian}.

{\bf Proof of Part 4.}

By substitute $x_i$ into $t$ of Part 3, we get
\begin{align*}
    \frac{\d L}{\d x_i}
    = & ~ (\cosh(Ax) - b )^\top \cdot ( \sinh(Ax) \circ \frac{A \d x}{ \d x_i } )\\
    = & ~ (\cosh(Ax) - b )^\top \cdot ( \sinh(Ax) \circ A_{*,i} ) \\
    = & ~ A_{*,i}^\top \diag( \sinh(Ax) ) (\cosh(Ax) - b )
\end{align*}
where the first step follows from the result of Part 2 and the second step follows from the result of Lemma~\ref{lem:Ax_gradient_hessian}, the last step follows from Fact~\ref{fac:circ_diag}.

{\bf Proof of Part 5.}

We have
\begin{align*}
    \frac{\d L}{ \d x}
    = & ~ A^\top \diag( \sinh(Ax) ) (\cosh(Ax) - b)
\end{align*}
where this step follows from the result of Part 4 directly.
\end{proof}

\subsection{Hessian}\label{sec:cosh:hessian}

\begin{lemma}\label{lem:hessian_cosh}
\begin{itemize}

    \item Part 1.
    \begin{align*}
        \frac{ \d^2 ( \cosh(Ax) - b ) }{ \d x_i^2 }
        = & ~ A_{*,i} \circ \cosh(Ax) \circ A_{*,i}
    \end{align*}
    \item Part 2.
    \begin{align*}
        \frac{ \d^2 ( \cosh(Ax) - b ) }{ \d x_i \d x_j }
        = & ~ A_{*,j} \circ \cosh(Ax) \circ A_{*,i}
    \end{align*}
    \item Part 3. 
    \begin{align*}
        \frac{\d^2 L_{\cosh} }{ \d x_i^2}
        = & ~ A_{*,i}^\top \diag( 2 \cosh(Ax) \circ \cosh(Ax) - b \circ \cosh(Ax) - {\bf 1}_n )  A_{*,i}
    \end{align*}
    \item Part 4. 
     \begin{align*}
        \frac{\d^2 L_{\cosh} }{ \d x_i \d x_j} = A_{*,i}^\top \diag( \sinh^2(Ax) + \cosh^2(Ax) - b \circ \cosh(Ax) ) A_{*,j}
    \end{align*}
\end{itemize}
\end{lemma}
\begin{proof}

    
{\bf Proof of Part 1.}
\begin{align*}
    \frac{ \d^2 ( \cosh(Ax) - b ) }{ \d x_i^2 }
        = & ~ \frac{\d}{\d x_i}\bigg(\frac{\d (\cosh(Ax) - b)}{\d x_i} \bigg) \\
        = & ~ \frac{\d (\sinh(Ax) \circ A_{*,i})}{\d x_i} \\
        = & ~ A_{*,i} \circ \frac{\d \sinh(Ax)}{\d x_i} \\
        = & ~ A_{*,i} \circ \cosh(Ax) \circ A_{*,i}
\end{align*}
where the first step is an expansion of the Hessian, the second step follows from the differential chain rule, the third step extracts the matrix $A_{*,i}$ with constant entries out of the derivative, and the last step also follows from the chain rule.

{\bf Proof of Part 2.}
\begin{align*}
    \frac{ \d^2 ( \cosh(Ax) - b ) }{ \d x_i \d x_j }
        = & ~ \frac{\d}{\d x_i}\bigg(\frac{\d}{\d x_j}\bigg(\cosh(Ax) - b\bigg) \bigg) \\
        = & ~ \frac{\d}{\d x_i}\bigg(\sinh(Ax) \circ A_{*,j} \bigg) \\
        = & ~ A_{*,j} \circ \cosh(Ax) \circ A_{*,i}
\end{align*}
where the first step is an expansion of the Hessian, the second and third steps follow from the differential chain rule.

{\bf Proof of Part 3.}

Here in the proof, for simplicity, we let use $\cosh^2(Ax) = \cosh(Ax) \circ \cosh(Ax)$.

We have
\begin{align*}
    \frac{\d^2 L }{ \d x_i^2}
        = & ~ \frac{\d}{\d x_i}\bigg(\frac{\d L}{\d x_i} \bigg) \\
        = & ~ \frac{\d}{\d x_i}\bigg((\cosh(Ax) - b )^\top \cdot ( \sinh(Ax) \circ A_{*,i} ) \bigg) \\
        = & ~ (\sinh(Ax) \circ A_{*,i})^\top \cdot ( \sinh(Ax) \circ A_{*,i} )+(\cosh(Ax) - b )^\top \cdot (A_{*,i} \circ \cosh(Ax) \circ A_{*,i}) \\ 
        = & ~ A_{*,i}^\top \diag( \sinh(Ax) \circ \sinh(Ax) + \cosh(Ax) \circ \cosh(Ax) - b \circ \cosh(Ax) ) A_{*,i} \\
        = & ~ A_{*,i}^\top \diag( 2 \cosh(Ax) \circ \cosh(Ax) - b \circ \cosh(Ax) - {\bf 1}_n ) A_{*,i}
\end{align*}
where the first step is an expansion of the Hessian, and the third step follows from the product rule of calculus, the last step follows from Fact~\ref{fac:vector_norm}.

{\bf Proof of Part 4.}
\begin{align*}
    \frac{\d^2 L }{ \d x_i \d x_j}
        = & ~ \frac{\d}{\d x_i}\bigg(\frac{\d L}{\d x_j} \bigg) \\
        = & ~ \frac{\d}{\d x_i}\bigg((\cosh(Ax) - b )^\top \cdot ( \sinh(Ax) \circ A_{*,j} ) \bigg) \\
        = & ~ (\sinh(Ax) \circ A_{*,i})^\top \cdot ( \sinh(Ax) \circ A_{*,j} )+(\cosh(Ax) - b )^\top \cdot (A_{*,j} \circ \cosh(Ax) \circ A_{*,i}) \\
        = & ~ A_{*,i}^\top \diag( \sinh(Ax) \circ \sinh(Ax) + \cosh(Ax) \circ \cosh(Ax) - b \circ \cosh(Ax) ) A_{*,j} \\
        = & ~ A_{*,i}^\top \diag( 2 \cosh(Ax) \circ \cosh(Ax) - b \circ \cosh(Ax) - {\bf 1}_n ) A_{*,j}
\end{align*}
where the first step is an expansion of the Hessian,  and the third step follows from the product rule of calculus.  
\end{proof}


\subsection{Gradient and Hessian of the Loss function for Cosh Function}\label{sec:cosh:gradient_hessian}

\begin{lemma}\label{lem:gradient_hessian_cosh}
    Let $L: \R^d \to \R_{\geq 0}$ be defined in Definition~\ref{def:L_exp}. Then for any $i, j \in [d]$, we have
    \begin{itemize}
        \item Part 1. Gradient
    \begin{align*}
        \nabla L_{\cosh} = A^\top \diag(\sinh(Ax)) \diag(\cosh(Ax) - b) {\bf 1}_n
    \end{align*}
        \item Part 2. Hessian
        \begin{align*}
            \nabla^2 L_{\cosh} = A^\top \diag( 2 \cosh(Ax) \circ \cosh(Ax) - b \circ \cosh(Ax) ) A
        \end{align*}
    \end{itemize}
\end{lemma}
 
\begin{proof}

{\bf Part 1.}
We run Lemma~\ref{lem:gradient_cosh} and Fact~\ref{fac:circ_diag} directly.

{\bf Part 2.}
It follows from Part 5 of Lemma~\ref{lem:hessian_cosh}.
\end{proof}



\subsection{Loss Function with a Regularization Term}\label{sec:cosh:loss_and_reg}

\begin{definition}\label{def:L_cosh_and_regularized}
Given matrix $A \in \R^{n \times d}$ and $b \in \R^n$, $w \in \R^n$. For a vector $x \in \R^d$, we define loss function $L(x)$ as follows
\begin{align*}
L_{\cosh,\reg}(x): = 0.5 \cdot \| \cosh(Ax) - b \|_2^2+ 0.5 \cdot \| W A x \|_2^2
\end{align*}
where $W = \diag(w)$.
\end{definition}

\begin{lemma}
Let $L$ be defined as Definition~\ref{def:L_cosh_and_regularized}, then we have
\begin{itemize}
    \item Part 1. Gradient
    \begin{align*}
        \frac{\d L_{\cosh,\reg}}{\d x} = A^\top \diag(\sinh(Ax)) ( \diag(\cosh(Ax) - b) ) {\bf 1}_n + A^\top W^2 A x
    \end{align*}
    \item Part 2. Hessian
    \begin{align*} 
        \frac{\d^2 L_{\cosh,\reg}}{\d x^2} = A^\top \diag( 2  \cosh(Ax) \circ \cosh(Ax) - b \circ \cosh(Ax) ) A +  A^\top W^2 A
    \end{align*}
    
\end{itemize}
\end{lemma}
\begin{proof}
{\bf Proof of Part 1.}
We run Lemma~\ref{lem:gradient_hessian_exp} and Lemma~\ref{lem:regularization} directly.

{\bf Proof of Part 2.}
We run Lemma~\ref{lem:gradient_hessian_exp} and Lemma~\ref{lem:regularization} directly.

 


\end{proof}


\subsection{Hessian is Positive Definite}\label{sec:cosh:convex}

\begin{lemma}\label{lem:hessian_is_pd_cosh}
 If $w_{i}^2 > 0.5 b_{i}^2 + l/\sigma_{\min}(A)^2 + 1$ for all $i \in [n]$, then 
    \begin{align*}
        \frac{\d^2 L}{\d x^2} \succeq l \cdot I_d
    \end{align*}
\end{lemma}

\begin{proof}
We define diagonal matrix $D \in \R^{n \times n}$
\begin{align*}
D = \diag( \sinh^2(Ax) + \cosh^2(Ax) - b \circ \cosh(Ax) ) + W^2
\end{align*}

Then we can rewrite Hessian as 
\begin{align*}
\frac{\d^2 L}{\d x^2} = A^\top D A.
\end{align*} 

 
Then we have
\begin{align*}
D_{i,i} 
= & ~ ( \sinh^2( (Ax)_i ) + \cosh^2((Ax)_i) - b_i \cosh^2((Ax)_i) ) ) + w_{i,i}^2 \\
= & ~ ( z_i^2 - 1 + z_i^2 - b_i z_i ) + w_{i}^2 \\
= & ~ 2 z_i^2 - b_i z_i + w_i^2 -1 \\
> & ~ 2 z_i^2 - b_i z_i + 0.5 b_{i}^2 + l/\sigma_{\min}(A)^2 \\
= & ~ 0.5 ( 2z_i - b_i )^2 + l/\sigma_{\min}(A)^2 \\
\geq & ~ l/\sigma_{\min}(A)^2
\end{align*}
where the first step follows from simple algebra, the second step follows from replacing $\cosh(Ax)$ with $z = \cosh(Ax)$ and $\sinh^2() = \cosh^2()-1$ (Fact~\ref{fac:e_cosh_sinh_exact}), the third step follows from $w_{i}^2 > 0.5b_{i}^2 + l/\sigma_{\min}(A)^2 + 1$, the fourth step follows from simple algebra, the fifth step follows from $x^2\geq0, \forall x$.


Since we know $D_{i,i} > 0$ for all $i \in [n]$ and Lemma~\ref{lem:ADA_pd}, we have 
\begin{align*}
A^\top D A \succeq (\min_{i \in [n]} D_{i,i}) \cdot \sigma_{\min}(A)^2 I_d \succeq l \cdot I_d
\end{align*}
Thus, Hessian is positive definite forever and thus the function is convex.
\end{proof}


\subsection{Hessian is Lipschitz}\label{sec:cosh:lipschitz}
\begin{lemma}[Hessian is Lipschitz]\label{lem:hessian_is_lipschitz_cosh}
If the following condition holds
 
\begin{itemize}
    \item Let $H(x) = \frac{\d^2 L_{\cosh,\reg}}{\d x^2}$
    \item Let $R > 2$
    \item $\|x \|_2 \leq R, \| y \|_2 \leq R$
    \item $\| A (x-y) \|_{\infty} < 0.01$
    \item $\| A \| \leq R$
    \item $\| b \|_2 \leq R$
\end{itemize}
Then we have
    \begin{align*}
        \| H(x) - H(y) \| \leq \exp(6R^2) \cdot \| x- y \|_2
    \end{align*}
\end{lemma}
\begin{proof}

We have
\begin{align}\label{eq:rewrite_H_diff_cosh}
& ~ \| H(x) - H(y) \| \notag \\
= & ~ \| A^\top \diag(2 \cosh(Ax) - b) \diag(\cosh(Ax)) A -  A^\top \diag(2 \cosh(Ay) - b) \diag(\cosh(Ay)) A \| \notag \\
\leq & ~ \| A \|^2 \cdot \|  (2 \cosh(Ax) - b) \circ \cosh(Ax) - (2 \cosh(Ay) - b) \circ \cosh(Ay) \|_2  \notag \\
= & ~ \| A \|^2 \cdot \| 2 (\cosh(Ax) + \cosh(Ay) )\circ ( \cosh(Ax) - \cosh(Ay) ) - b \circ ( \cosh(Ax) - \cosh(A y) ) \|_2 \notag \\
= & ~ \| A \|^2 \cdot \| ( 2 \cosh(Ax) + 2 \cosh(Ay) - b ) \circ ( \cosh(Ax) - \cosh(Ay) ) \|_2  \notag \\
\leq & ~ \| A \|^2 \cdot \| ( 2 \cosh(Ax) + 2 \cosh(Ay) - b ) \|_{\infty} \cdot \| \cosh(Ax) - \cosh(Ay) \|_2
\end{align}
where the first step follows from $H(x) = \nabla^2L$ and simple algebra, 
the second step follows from Fact~\ref{fac:matrix_norm}, the third step follows from simple algebra, the fourth step follows from simple algebra, the last step follows from Fact~\ref{fac:vector_norm}.
 

For the first term in Eq.~\eqref{eq:rewrite_H_diff_cosh}, we have
\begin{align}\label{eq:upper_bound_H_x_H_y_step_1_cosh}
\| A \|^2 \leq R^2
\end{align}

For the second term in Eq.~\eqref{eq:rewrite_H_diff_cosh}, we have
\begin{align}\label{eq:upper_bound_H_x_H_y_step_2_cosh}
\| ( 2 \cosh(Ax) + 2 \cosh(Ay) - b ) \|_{\infty} \notag
\leq & ~ \| 2 \cosh(Ax)\|_{\infty} + \|2 \cosh(Ay)\|_{\infty} + \|b\|_\infty \notag \\
\leq & ~ \| 2 \cosh(Ax)\|_2 + \|2 \cosh(Ay)\|_2 + \|b\|_\infty \notag \\
\leq & ~ 2\exp(\|Ax\|_2) + 2\exp(\|Ay\|_2) + \|b\|_\infty \notag \\
\leq & ~ 4 \exp(\| A \| R) + \| b \|_{\infty} \notag \\
\leq & ~ 4 \exp(R^2) + R \notag \\
\leq & ~ 5 \exp(R^2)
\end{align}

where the first step follows from Fact~\ref{fac:vector_norm}
, the second step follows from Fact~\ref{fac:vector_norm}, the third step follows from Fact~\ref{fac:vector_norm}, the fourth step follows from $\|x\|_2 \leq R,\|y\|_2 \leq R$ and Fact~\ref{fac:vector_norm}, the fifth step follows from $\|b\|_2 \leq R$, the last step follows from $R \geq 2$.
 


For the third term in Eq.~\eqref{eq:rewrite_H_diff_cosh}, we have
\begin{align}\label{eq:upper_bound_H_x_H_y_step_3_cosh}
\| \cosh(Ax) - \cosh(Ay) \|_2 
\leq & ~ \| \cosh(Ax) \|_2 \cdot 2 \| A (y-x) \|_{\infty} \notag \\
\leq & ~ \exp(\| A \| R)  \cdot 2 \| A (y-x) \|_{\infty} \notag \\
\leq & ~ \exp(\| A \| R)  \cdot 2 \| A (y-x) \|_2 \notag \\
\leq & ~ \exp(\| A \| R)  \cdot 2 \| A \| \cdot \| y - x \|_2 \notag\\
\leq & ~ 2 R \exp(R^2) \cdot \|y - x\|_2
\end{align}
where the first step follows  from $\| A (y-x) \|_{\infty} < 0.01$ and Fact~\ref{fac:vector_norm}, the second step follows from Fact~\ref{fac:vector_norm}, the third step follows from Fact~\ref{fac:vector_norm},  the forth step follows from Fact~\ref{fac:matrix_norm}, the fifth step follows from $\|A\| \leq R$.
 
 

Putting it all together, we have
\begin{align*}
 \| H(x) - H(y) \| 
 \leq & ~ R^2 \cdot 5 \exp(R^2) \cdot 2 \exp(R^2) \| y - x \|_2 \\
= & ~ 10 R^2 \exp(2R^2) \cdot \| y - x \|_2 \\
 \leq & ~ \exp(4R^2) \cdot  \exp(2R^2) \cdot \| y - x \|_2 \\
 = & ~ \exp(6R^2) \cdot \| y - x \|_2
\end{align*}
where the first step follows from by applying Eq.~\eqref{eq:upper_bound_H_x_H_y_step_1_cosh}, Eq.~\eqref{eq:upper_bound_H_x_H_y_step_2_cosh}, and Eq.~\eqref{eq:upper_bound_H_x_H_y_step_3_cosh}, the second step follows from simple algebra, the third step follows from $R \geq 2$, the last step follows from simple algebra.

\end{proof}

\newpage
\section{Sinh Regression}\label{sec:sinh}
 
In this section, we provide detailed analysis of $L_{\sinh}$. In Section~\ref{sec:sinh:definition} we define the loss function $L_{\sinh}$ based on $\sinh(x)$. In Section~\ref{sec:sinh:gradient} we compute the gradient of $L_{\sinh}$ by detail. In Section~\ref{sec:sinh:hessian} we compute the hessian of $L_{\sinh}$ by detail. In Section~\ref{sec:sinh:gradient_hessian}, we summarize the result of Section~\ref{sec:sinh:gradient} and Section~\ref{sec:sinh:hessian} and aquire the gradient $\nabla L_{\sinh}$ and hessian $\nabla^2 L_{\sinh}$ for $L_{\sinh}$. In Section~\ref{sec:sinh:loss_reg} we define $L_{\sinh,\reg}$ by adding the regularization term 
$L_{\reg}$ in Section~\ref{sec:preli:regularization} to $L_{\sinh}$ and compute the gradient $\nabla L_{\sinh,\reg}$ and hessian $\nabla^2 L_{\sinh,\reg}$ of $L_{\sinh,\reg}$. In Section~\ref{sec:sinh:convex} we proved that $\nabla^2 L_{\sinh,\reg} \succ 0$ and thus showed that $L_{\sinh,\reg}$ is convex. In Section~\ref{sec:sinh:lipschitz} we provide the upper bound for $\|\nabla^2 L_{\sinh,\reg}(x)-\nabla^2 L_{\sinh,\reg}(y)\|$ and thus proved $\nabla^2 L_{\sinh,\reg}$ is lipschitz.
\subsection{Definition}\label{sec:sinh:definition}
\begin{definition}\label{def:L_sinh}
Given $A \in \R^{n \times d}$ and $b \in \R^n$. For a vector $x \in \R^d$, we define loss function $L(x)$ as follows:
\begin{align*}
L(x) := 0.5 \cdot \| \sinh(Ax) - b \|_2^2
\end{align*}
\end{definition}



\subsection{Gradient}\label{sec:sinh:gradient}

\begin{lemma}[Gradient for Sinh]\label{lem:gradient_sinh}
We have
\begin{itemize}
    
    \item Part 1.
    \begin{align*}
        \frac{ \d ( \sinh(Ax) - b ) }{ \d t } = \cosh(Ax) \circ \frac{A \d x }{ \d t}
    \end{align*}
    \item Part 2. 
    \begin{align*}
        \frac{\d L_{\sinh} }{ \d t} = (\sinh(Ax) - b )^\top \cdot ( \cosh(Ax) \circ \frac{A \d x}{ \d t } )
    \end{align*}
\end{itemize}
Further, we have for each $i \in [d]$
\begin{itemize}
    
    \item Part 3.
    \begin{align*}
        \frac{ \d ( \sinh(Ax) - b ) }{ \d x_i } = \cosh(Ax) \circ A_{*,i}
    \end{align*}
    \item Part 4. 
    \begin{align*}
        \frac{\d L_{\sinh} }{ \d x_i}  
        = & ~ (\sinh(Ax) - b )^\top \cdot ( \cosh(Ax) \circ A_{*,i} )
    \end{align*}
    \item Part 5.
    \begin{align*}
        \frac{\d L_{\sinh} }{ \d x} =  A^\top \diag( \cosh(Ax) ) (\sinh(Ax) - b)
    \end{align*}
\end{itemize}
\end{lemma}

\begin{proof}


{\bf Proof of Part 1.}

For each $i \in [n]$, we have
\begin{align*}
 \frac{ \d ( \sinh(Ax) - b )_i }{ \d t } 
 = & ~ \cosh(Ax)_i \cdot \frac{\d (Ax)_i}{\d t} \\
 = & ~ \cosh(Ax)_i \cdot \frac{ (A \d x)_i}{\d t} 
\end{align*}
where the first and second step follow from the differential chain rule.

Thus, we complete the proof.

{\bf Proof of Part 2.}

We have
 
\begin{align*}
    \frac{\d L_{\sinh} }{ \d t} 
    = & ~ (\sinh(Ax) - b )^\top \cdot \frac{ \d ( \sinh(Ax) - b ) }{\d t} \\
    = & ~ (\sinh(Ax) - b )^\top \cdot ( \cosh(Ax) \circ \frac{A \d x}{ \d t } )
\end{align*}
where the first step follows from the differential chain rule, and the second step follows from $\frac{ \d ( \sinh(Ax) - b ) }{ \d t } = \cosh(Ax) \circ \frac{A \d x }{ \d t}$ in {\bf Part 1}.



{\bf Proof of Part 3.}

We have
\begin{align*}
    \frac{\d (\sinh(Ax)-b)}{\d x_i}
    = & ~ \frac{\d (\sinh(Ax))}{\d x_i} - \frac{\d b}{\d x_i}\\
    = & ~ \cosh(Ax) \circ \frac{\d Ax}{\d x_i} - 0\\
    = & ~ \cosh(Ax) \circ A_{*,i}
\end{align*}
where the first step follows from the property of the gradient, the second step follows from the differential chain rule, and the last step directly follows from Lemma~\ref{lem:Ax_gradient_hessian}.

{\bf Proof of Part 4.}

By substitute $x_i$ into $t$ of {\bf Part 3}, we get
\begin{align*}
    \frac{\d L}{\d x_i}
    = & ~ (\sinh(Ax) - b )^\top \cdot ( \cosh(Ax) \circ \frac{A \d x}{ \d x_i } )\\
    = & ~ (\sinh(Ax) - b )^\top \cdot ( \cosh(Ax) \circ A_{*,i} ) \\
    = & ~ A_{*,i}^\top \diag( \cosh(Ax) ) (\sinh(Ax) - b )
\end{align*}
where the first step follows from the result of {\bf Part 2} and the second step follows from the result of Lemma~\ref{lem:Ax_gradient_hessian}, the last step follows from Fact~\ref{fac:circ_diag}.

{\bf Proof of Part 5.}

We have
\begin{align*}
    \frac{\d L}{ \d x}
    = & ~ A^\top \diag( \cosh(Ax) ) (\sinh(Ax) - b)
\end{align*}
where this step follows from the result of {\bf Part 4} directly.
\end{proof}

\subsection{Hessian}\label{sec:sinh:hessian}

\begin{lemma}\label{lem:hessian_sinh}
\begin{itemize}

    \item Part 1.
    \begin{align*}
        \frac{ \d^2 ( \sinh(Ax) - b ) }{ \d x_i^2 }
        = & ~ A_{*,i} \circ \sinh(Ax) \circ A_{*,i}
    \end{align*}
    \item Part 2.
    \begin{align*}
        \frac{ \d^2 ( \sinh(Ax) - b ) }{ \d x_i \d x_j }
        = & ~ A_{*,j} \circ \sinh(Ax) \circ A_{*,i}
    \end{align*}
    \item Part 3.
    \begin{align*}
        \frac{\d^2 L_{\sinh} }{ \d x_i^2}
        = & ~ A_{*,i}^\top \diag( 2\sinh(Ax) \circ \sinh(Ax) - b \circ \sinh(Ax) + {\bf 1}_n )  A_{*,i}
    \end{align*}
    \item Part 4. 
     \begin{align*}
        \frac{\d^2 L_{\sinh} }{ \d x_i \d x_j} = A_{*,i}^\top \diag( 2 \sinh(Ax) \circ \sinh(Ax) - b \circ \sinh(Ax) - {\bf 1}_n ) A_{*,j}
    \end{align*}
\end{itemize}
\end{lemma}
\begin{proof}

    
{\bf Proof of Part 1.}
\begin{align*}
    \frac{ \d^2 ( \sinh(Ax) - b ) }{ \d x_i^2 }
        = & ~ \frac{\d}{\d x_i}\bigg(\frac{\d (\sinh(Ax) - b)}{\d x_i} \bigg) \\
        = & ~ \frac{\d (\cosh(Ax) \circ A_{*,i})}{\d x_i} \\
        = & ~ A_{*,i} \circ \frac{\d \cosh(Ax)}{\d x_i} \\
        = & ~ A_{*,i} \circ \sinh(Ax) \circ A_{*,i}
\end{align*}
where the first step is an expansion of the Hessian, the second step follows from the differential chain rule, the third step extracts the matrix $A_{*,i}$ with constant entries out of the derivative, and the last step also follows from the chain rule.

{\bf Proof of Part 2.}
\begin{align*}
    \frac{ \d^2 ( \sinh(Ax) - b ) }{ \d x_i \d x_j }
        = & ~ \frac{\d}{\d x_i}\bigg(\frac{\d}{\d x_j}\bigg(\sinh(Ax) - b\bigg) \bigg) \\
        = & ~ \frac{\d}{\d x_i}\bigg(\cosh(Ax) \circ A_{*,j} \bigg) \\
        = & ~ A_{*,j} \circ \sinh(Ax) \circ A_{*,i}
\end{align*}
where the first step is an expansion of the Hessian, the second and third steps follow from the differential chain rule.

{\bf Proof of Part 3.}
\begin{align*}
    \frac{\d^2 L }{ \d x_i^2}
        = & ~ \frac{\d}{\d x_i}\bigg(\frac{\d L}{\d x_i} \bigg) \\
        = & ~ \frac{\d}{\d x_i}\bigg((\sinh(Ax) - b )^\top \cdot ( \cosh(Ax) \circ A_{*,i} ) \bigg) \\
        = & ~ (\cosh(Ax) \circ A_{*,i})^\top \cdot ( \cosh(Ax) \circ A_{*,i} )+(\sinh(Ax) - b )^\top \cdot (A_{*,i} \circ \sinh(Ax) \circ A_{*,i})\\ 
        = & ~ A_{*,i}^\top \diag( \cosh^2(Ax) + \sinh^2(Ax) - b \circ \sinh(Ax) ) A_{*,i} \\
        = & ~ A_{*,i}^\top \diag( 2\sinh(Ax) \circ \sinh(Ax) - b \circ \sinh(Ax) + {\bf 1}_n )  A_{*,i}
\end{align*}
where the first step is an expansion of the Hessian,
the second step follows from {\bf Part 4} of Lemma~\ref{lem:gradient_sinh},
the third step follows from the product rule of calculus,
the fourth step follows from Fact~\ref{fac:circ_diag},
the last step follows from Fact~\ref{fac:circ_diag}.


{\bf Proof of Part 4.}
\begin{align*}
    \frac{\d^2 L }{ \d x_i \d x_j}
        = & ~ \frac{\d}{\d x_i}\bigg(\frac{\d L}{\d x_j} \bigg) \\
        = & ~ \frac{\d}{\d x_i}\bigg((\sinh(Ax) - b )^\top \cdot ( \cosh(Ax) \circ A_{*,j} ) \bigg) \\
        = & ~ (\cosh(Ax) \circ A_{*,i})^\top \cdot ( \cosh(Ax) \circ A_{*,j} )+(\sinh(Ax) - b )^\top \cdot (A_{*,j} \circ \sinh(Ax) \circ A_{*,i}) \\
        = & ~ A_{*,i}^\top \diag( \cosh^2(Ax) + \sinh^2(Ax) - b \circ \sinh(Ax) ) A_{*,j} \\
        = & ~ A_{*,i}^\top \diag( 2\sinh(Ax) \circ \sinh(Ax) - b \circ \sinh(Ax) + {\bf 1}_n )  A_{*,j}
\end{align*}
where the first step is an expansion of the Hessian,
the second step follows from {\bf Part 4} of Lemma~\ref{lem:gradient_sinh},
the third step follows from the product rule of calculus,
the fourth step follows from Fact~\ref{fac:circ_diag},
the last step follows from Fact~\ref{fac:circ_diag}.

 
\end{proof}


\subsection{Gradient and Hessian of the Loss function for Sinh Function}\label{sec:sinh:gradient_hessian}

\begin{lemma}\label{lem:gradient_hessian_sinh}
    Let $L: \R^d \to \R_{\geq 0}$ be defined in Definition~\ref{def:L_exp}. Then for any $i, j \in [d]$, we have
    \begin{itemize}
        \item Part 1. Gradient
    \begin{align*}
        \nabla L_{\sinh} = A^\top \diag(\cosh(Ax)) \diag(\sinh(Ax) - b) {\bf 1}_n
    \end{align*}
        \item Part 2. Hessian
        \begin{align*}
            \nabla^2 L_{\sinh} = A^\top \diag( 2\sinh(Ax) \circ \sinh(Ax) - b \circ \sinh(Ax) + {\bf 1}_n ) A
        \end{align*}
    \end{itemize}
\end{lemma}
 
\begin{proof}

{\bf Part 1.}
We run Lemma~\ref{lem:gradient_sinh} and Fact~\ref{fac:circ_diag} directly.

{\bf Part 2.}
It follows from Part 5 of Lemma~\ref{lem:hessian_sinh}.
\end{proof}



\subsection{Loss Function with a Regularization Term}\label{sec:sinh:loss_reg}

\begin{definition}\label{def:L_sinh_and_regularized}
Given matrix $A \in \R^{n \times d}$ and $b \in \R^n$, $w \in \R^n$. For a vector $x \in \R^d$, we define loss function $L(x)$ as follows
\begin{align*}
L_{\sinh,\reg}(x): = 0.5 \cdot \| \sinh(Ax) - b \|_2^2+ 0.5 \cdot \| W A x \|_2^2
\end{align*}
where $W = \diag(w)$.
\end{definition}

\begin{lemma}
Let $L$ be defined as Definition~\ref{def:L_sinh_and_regularized}, then we have
\begin{itemize}
    \item Part 1. Gradient
    \begin{align*}
        \frac{\d L_{\sinh,\reg}}{\d x} = A^\top \diag(\cosh(Ax)) ( \diag(\sinh(Ax) - b) ) {\bf 1}_n + A^\top W^2 A x
    \end{align*}
    \item Part 2. Hessian
    \begin{align*} 
        \frac{\d^2 L_{\sinh,\reg}}{\d x^2} = A^\top \diag( 2\sinh(Ax) \circ \sinh(Ax) - b \circ \sinh(Ax) + {\bf 1}_n ) A +  A^\top W^2 A
    \end{align*}
\end{itemize}
\end{lemma}
\begin{proof}
{\bf Proof of Part 1.}
We run Lemma~\ref{lem:gradient_hessian_exp} and Lemma~\ref{lem:regularization} directly.

{\bf Proof of Part 2.}
We run Lemma~\ref{lem:gradient_hessian_exp} and Lemma~\ref{lem:regularization} directly.
\end{proof}
 



\subsection{Hessian is Positive Definite}\label{sec:sinh:convex}
\begin{lemma}\label{lem:hessian_is_pd_sinh}
Let $l > 0$ denote a parameter. 
 If $w_{i}^2 > 0.5 b_{i}^2 + l/\sigma_{\min}(A)^2 - 1$ for all $i \in [n]$, then
    \begin{align*}
        \frac{\d^2 L}{\d x^2} \succeq l \cdot I_d
    \end{align*}
\end{lemma}

\begin{proof}



We define $D$
\begin{align*}
D = \diag( 2\sinh(Ax) \circ \sinh(Ax) - b \circ \sinh(Ax) + {\bf 1}_n ) + W^2
\end{align*}

Then we can rewrite Hessian as 
\begin{align*}
\frac{\d^2 L}{\d x^2} = A^\top D A.
\end{align*} 

We define
\begin{align*}
z_i = \sinh( ( Ax )_i )
\end{align*}
 
Then we have
\begin{align*}
D_{i,i} 
= & ~ ( 2 \sinh^2((Ax)_i) +1 - b_i \sinh((Ax)_i) ) ) + w_{i,i}^2 \\
= & ~ ( 2z_i^2 + 1  - b_i z_i ) + w_{i}^2 \\
= & ~ 2 z_i^2 - b_i z_i + w_i^2 + 1 \\
> & ~ 2 z_i^2 - b_i z_i + 0.5 b_{i}^2 + l/\sigma_{\min}(A)^2\\
= & ~ 0.5 ( 2z_i - b_i )^2 + l/\sigma_{\min}(A)^2\\
\geq & ~ l/\sigma_{\min}(A)^2
\end{align*}
where the first step follows from simple algebra, the second step follows from replacing $\sinh(Ax)$ with $z = \sinh(Ax)$ and $\cosh^2() = \sinh^2()+1$ (Fact~\ref{fac:e_cosh_sinh_exact}), the third step follows from $w_{i}^2 > 0.5b_{i}^2 + 1/\sigma_{\min}(A)^2-1$, the fourth step follows from simple algebra, the fifth step follows from $x^2\geq0, \forall x$.
 

Since we know $D_{i,i} > l$ for all $i \in [n]$ and Lemma~\ref{lem:ADA_pd}, we have 
\begin{align*}
A^\top D A \succeq (\min_{i \in [n]} D_{i,i}) \cdot \sigma_{\min}(A)^2 I_d \succeq l \cdot I_d
\end{align*}
Thus, Hessian is positive definite forever and thus the function is convex.
\end{proof}



\subsection{Hessian is Lipschitz}\label{sec:sinh:lipschitz}
\begin{lemma}[Hessian is Lipschitz]\label{lem:hessian_is_lipschitz_sinh}
If the following condition holds
 
\begin{itemize}
    \item Let $H(x) = \frac{\d^2 L_{\sinh,\reg}}{\d x^2}$
    \item Let $R > 2$
    \item $\|x \|_2 \leq R, \| y \|_2 \leq R$
    \item $\| A (x-y) \|_{\infty} < 0.01$
    \item $\| A \| \leq R$
    \item $\| b \|_2 \leq R$
\end{itemize}
Then we have
    \begin{align*}
        \| H(x) - H(y) \| \leq \sqrt{n} \exp(6R^2) \cdot \| x- y \|_2
    \end{align*}
\end{lemma}
\begin{proof}

We have
\begin{align}\label{eq:rewrite_H_diff_sinh}
& ~ \| H(x) - H(y) \| \notag \\
= & ~ \| A^\top \diag(2 \sinh(Ax) - b) \diag(\sinh(Ax)) A -  A^\top \diag(2 \sinh(Ay) - b) \diag(\sinh(Ay)) A \| \notag \\
\leq & ~ \| A \|^2 \cdot \|  (2 \sinh(Ax) - b) \circ \sinh(Ax) - (2 \sinh(Ay) - b) \circ \sinh(Ay) \|_2  \notag \\
= & ~ \| A \|^2 \cdot \| 2 (\sinh(Ax) + \sinh(Ay) )\circ ( \sinh(Ax) - \sinh(Ay) ) - b \circ ( \sinh(Ax) - \sinh(A y) ) \|_2 \notag \\
= & ~ \| A \|^2 \cdot \| ( 2 \sinh(Ax) + 2 \sinh(Ay) - b ) \circ ( \sinh(Ax) - \sinh(Ay) ) \|_2  \notag \\
\leq & ~ \| A \|^2 \cdot \| ( 2 \sinh(Ax) + 2 \sinh(Ay) - b ) \|_{\infty} \cdot \| \sinh(Ax) - \sinh(Ay) \|_2
\end{align}
where the first step follows from $H(x) = \nabla^2L$ and simple algebra, 
the second step follows from Fact~\ref{fac:matrix_norm}, the third step follows from simple algebra, the fourth step follows from simple algebra, the last step follows from Fact~\ref{fac:vector_norm}.
 

For the first term in Eq.~\eqref{eq:rewrite_H_diff_sinh}, we have
\begin{align}\label{eq:upper_bound_H_x_H_y_step_1_sinh}
\| A \|^2 \leq R^2
\end{align}

For the second term in Eq.~\eqref{eq:rewrite_H_diff_sinh}, we have
\begin{align}\label{eq:upper_bound_H_x_H_y_step_2_sinh}
\| ( 2 \sinh(Ax) + 2 \sinh(Ay) - b ) \|_{\infty} \notag
\leq & ~ \| 2 \sinh(Ax)\|_{\infty} + \|2 \sinh(Ay)\|_{\infty} + \|b\|_\infty \notag \\
\leq & ~ \| 2 \cosh(Ax)\|_{\infty} + \|2 \cosh(Ay)\|_{\infty} + \|b\|_\infty \notag \\
\leq & ~ 2\exp(\|Ax\|_2) + 2\exp(\|Ay\|_2) + \|b\|_\infty \notag \\
\leq & ~ 4 \exp(R^2) + \| b \|_{\infty} \notag \\
\leq & ~ 4 \exp(R^2) + R \notag \\
\leq & ~ 5 \exp(R^2)
\end{align}

where the first step follows from Fact~\ref{fac:vector_norm}
, the second step follows from Fact~\ref{fac:vector_norm}, the third step follows from Fact~\ref{fac:vector_norm},  the fourth step follows from $\|Ax\|_2 \leq R^2, \|Ay\|_2 \leq R^2$, the fifth step follows from $\|b\|_\infty \leq R$, the last step follows from $R \geq 2$.



For the third term in Eq.~\eqref{eq:rewrite_H_diff_sinh}, we have
\begin{align}\label{eq:upper_bound_H_x_H_y_step_3_sinh}
\| \sinh(Ax) - \sinh(Ay) \|_2 
\leq & ~ \| \cosh(Ax) \|_2 \cdot 2 \| A (y-x) \|_{\infty} \notag \\
\leq & ~ \sqrt{n}\| \cosh(Ax) \|_\infty \cdot 2 \| A (y-x) \|_{\infty} \notag \\
\leq & ~ \sqrt{n}\exp(\|Ax\|_2) \cdot 2 \| A (y-x) \|_{2} \notag \\
\leq & ~ \sqrt{n}\exp(R^2)  \cdot 2 \| A (y-x) \|_2 \notag \\
\leq & ~ \sqrt{n}\exp(R^2)  \cdot 2 \| A \| \cdot \| y - x \|_2 \notag\\
\leq & ~ 2 \sqrt{n}R \exp(R^2) \cdot \|y - x\|_2
\end{align}
where the first step follows  from $\| A (y-x) \|_{\infty} < 0.01$ and Fact~\ref{fac:vector_norm}, the second step follows from Fact~\ref{fac:vector_norm}, the third step follows from Fact~\ref{fac:vector_norm},   the fourth step follows from $\|Ax\|_2 \leq R^2$, the fifth step follows from Fact~\ref{fac:matrix_norm}, the last step follows from $\|A\| \leq R$.
 


Putting it all together, we have
\begin{align*}
 \| H(x) - H(y) \| 
 \leq & ~ R^2 \cdot 5 \exp(R^2) \cdot 2 \sqrt{n} R\exp(R^2) \| y - x \|_2 \\
= & ~ 10 \sqrt{n} R^3 \exp(2R^2) \cdot \| y - x \|_2 \\
 \leq & ~ \sqrt{n}\exp(4R^2) \cdot  \exp(2R^2) \cdot \| y - x \|_2 \\
 = & ~ \sqrt{n}\exp(6R^2) \cdot \| y - x \|_2
\end{align*}
where the first step follows from by applying Eq.~\eqref{eq:upper_bound_H_x_H_y_step_1_sinh}, Eq.~\eqref{eq:upper_bound_H_x_H_y_step_2_sinh}, and Eq.~\eqref{eq:upper_bound_H_x_H_y_step_3_sinh}, the second step follows from simple algebra, the third step follows from $R \geq 2$, the last step follows from simple algebra.

\end{proof}

\newpage
\section{Newton Method}\label{sec:newton}
 
In this section, we provide an approximate version of the Newton method for solving convex optimization problem and provide detailed analysis of such method. In Section~\ref{sec:newton:definitions} we define some assumptions under which we can tackle the optimization problem efficiently. In Section~\ref{sec:newton:connection} we state a simple lemma which is useful in Section~\ref{sec:newton:shrink}.In Section~\ref{sec:newton:approximation} we provides a approximation variant for the update step of newton method for convex optimization. In Section~\ref{sec:newton:hessian_property} we provide the upper bound of $\|H(x_k)\|$.  In Section~\ref{sec:newton:shrink} we provide the upper bound for $\|r_{k+1}\|$ and thus showed that our approximate update step is effective in solving the optimization problem. In Section~\ref{sec:newton:induction} we provide a lemma that showed our update step is effective. In Section~\ref{sec:newton:main}, we prove our main result.


\subsection{Definition and Update Rule}\label{sec:newton:definitions}
Let us study the local convergence of the Newton method. Consider the problem
\begin{align*}
    \min_{x \in \R^d } f(x)
\end{align*}
under the following assumptions:
\begin{definition}\label{def:f_ass}
We have
\begin{itemize}
    \item {\bf $l$-local Minimum.} Let $l > 0$ denote a parameter. There is a vector $x^* \in \R^d$ such that
    \begin{itemize}
        \item $\nabla f(x^*) = {\bf 0}_d$.
        \item $\nabla^2 f(x^*) \succeq l \cdot I_d$.
    \end{itemize}
    \item {\bf Hessian is $M$-Lipschitz.} Let $M>0$ denote a parameter that  \begin{align*}
        \| \nabla^2 f(y) - \nabla^2 f(x) \| \leq M \cdot \| y - x \|_2 
    \end{align*}
    \item {\bf Good Initialization Point.} Let $r_0:=\| x_0 -x_*\|_2$ such that
    \begin{align*}
        r_0 M \leq 0.1 l
    \end{align*}    
\end{itemize}
\end{definition}

We define gradient and Hessian as follows
\begin{definition}[Gradient and Hessian]=
We define gradient function $g : \R^d \rightarrow \R^d$ as
\begin{align*}
    g(x) := \nabla f(x)
\end{align*}
We define the Hessian function $H : \R^d \rightarrow \R^{d \times d}$ ,
\begin{align*}
    H(x) := \nabla^2 f(x)
\end{align*}
\end{definition}

Using the $g: \R^d \rightarrow \R^d$ and $H : \R^d \rightarrow \R^{d \times d}$, we can rewrite the exact process as follows 
:
\begin{definition}[Exact update]\label{def:exact_update_variant}
\begin{align*}
    x_{k+1} = x_k - H(x_k)^{-1} \cdot g(x_k)
\end{align*}
\end{definition}
\subsection{Connection between Gradient and Hessian}\label{sec:newton:connection}

\begin{lemma}[folklore]\label{lem:integral_gradient_hessian}
Let $g$ denote the gradient function and let $H: \R^d \rightarrow \R^{d \times d}$ denote the hessian function, then for any $x,y$, we have
\begin{align*}
g(y) - g(x) = \int_0^1 H( x + \tau(y-x) ) \cdot (y-x) \d \tau
\end{align*}
\end{lemma}
\begin{proof}

We have
\begin{align*}
\int_0^1 H( x + \tau(y-x) ) \cdot (y-x) \d \tau
= & ~ g( x+ \tau(y-x) ) \big|_0^1 \\
= & ~ g(x + 1\cdot (y-x)) - g(x+ 0 \cdot(y-x) ) \\
= & ~ g(y) - g(x)
\end{align*}
\end{proof}

\subsection{Approximate of Hessian and Update Rule}\label{sec:newton:approximation}
In many optimization applications, computing $\nabla^2 f(x_k )$ or $(\nabla^2 f(x_k))^{-1}$ is quite expensive. Therefore, a natural motivation is to approximately formulate its Hessian or inverse of Hessian.

\begin{definition}[Approximate Hessian]\label{def:wt_H}
For any $H(x_k)$, we define $\wt{H}(x_k)$ to satisfy the following condition 
\begin{align*}
 (1-\epsilon_H) \cdot H(x_k) \preceq \wt{H}(x_k) \preceq (1+\epsilon_H) \cdot H(x_k) .
\end{align*}
\end{definition}

To efficiently compute $\wt{H}(x_k)$, we use a standard tool from the literature
\begin{lemma}[\cite{syyz22,dsw22}]\label{lem:subsample}
Let $\epsilon_H = 0.01$. 
Given a matrix $A \in \R^{n \times d}$, for any positive diagonal matrix $D \in \R^{n \times n}$, there is an algorithm that runs in time
\begin{align*}
O( (\nnz(A) + d^{\omega} ) \poly(\log(n/\delta)) )
\end{align*}
output a $O(d \log(n/\delta))$ sparse diagonal matrix $\wt{D} \in \R^{n \times n}$ such that 
\begin{align*}
(1- \epsilon_H) A^\top D A \preceq A^\top \wt{D} A \preceq (1+\epsilon_H) A^\top D A.
\end{align*}
\end{lemma}





\begin{definition}[Approximate update]\label{def:update_x_k+1}
We consider the following process
\begin{align*}
    \underbrace{ x_{k+1} }_{d \times 1} = \underbrace{ x_k }_{d \times 1} - \underbrace{ \wt{H}(x_k)^{-1} }_{d \times d} \cdot \underbrace{ g(x_k) }_{d \times 1}
\end{align*}
\end{definition}

 
\subsection{Property of Hessian}\label{sec:newton:hessian_property}
\begin{lemma}\label{lem:lower_bound_spectral_hessian}
If the following conditions hold
\begin{itemize}
    \item Let $f$ be function that Hessian is $M$-Lipschitz (see Definition~\ref{def:f_ass}) 
    \item Suppose the optimal solution $x^*$  satisfy that $\| H(x_*) \| \geq l$ (see Definition~\ref{def:f_ass})
    \item Let $r_k := \| x_k - x^* \|_2$
\end{itemize}
 
We have
\begin{align*}
\| H(x_k) \| \geq l - M \cdot r_k
\end{align*}
\end{lemma}
\begin{proof}
We can show that
\begin{align*}
\| H(x_k) \| \geq & ~ \| H(x_*) \| - \| H(x_*) - H(x_k) \| \\
\geq & ~ \| H(x_*) \| - M \cdot \| x_* - x_k \|_2 \\
\geq & ~ l - M r_k
\end{align*}
where the first step follows from Fact~\ref{fac:matrix_norm}, the second step follows from $f$ is a $M$-bounded function, the third step follows from $\|H(x_k)\|\geq l$.
\end{proof}



\subsection{One Step Shrinking Lemma}\label{sec:newton:shrink}
\begin{lemma}\label{lem:one_step_shrinking}
If the following condition hold
\begin{itemize}
    \item Function $f$ follows from Definition~\ref{def:f_ass}. 
    \item Let $H$ denote the Hessian of $f$
    \item Let $g$ denote the gradient of $f$
    \item Let $r_k:= \| x_k - x^* \|_2$
\end{itemize}
 Then we have
\begin{align*}
r_{k+1} \leq 2(\epsilon_H + \frac{M r_k}{ l - M r_k } ) \cdot r_k
\end{align*} 
\end{lemma}

\begin{proof}
We have
\begin{align}\label{eq:variant_of_r_k_plus_1}
    x_{k+1} - x^* 
    = & ~ x_k - x^* - \wt{H}(x_k)^{-1} \cdot g(x_k)  \notag \\
    = & ~ x_k - x^* - \wt{H}(x_k)^{-1} \cdot ( g(x_k) - g(x^*) ) \notag\\ 
    = & ~ x_k - x^* - \wt{H}(x_k)^{-1} \cdot \int_0^1 H ( x^* + \tau (x_k-x^*) ) (x_k - x^*) \d \tau \notag\\ 
    = & ~ \wt{H}(x_k)^{-1} ( \wt{H}(x_k) ( x_k - x^*) ) - \wt{H}(x_k)^{-1} \cdot \int_0^1 H ( x^* + \tau (x_k-x^*) ) (x_k - x^*) \d \tau \notag\\ 
    = & ~  \wt{H}(x_k)^{-1} \bigg(\wt{H}(x_k) - \int_0^1 H ( x^* + \tau (x_k-x^*) ) \d \tau \bigg) \cdot (x_k - x^*) \notag\\ 
    = & ~ \wt{H}(x_k)^{-1} \bigg(\int_0^1 \wt{H}(x_k) \d \tau - \int_0^1 H ( x^* + \tau (x_k-x^*) ) \d \tau \bigg) \cdot (x_k - x^*) \notag\\ 
    = & ~  \bigg(\wt{H}(x_k)^{-1} \int_0^1 (\wt{H}(x_k) - H ( x^* + \tau (x_k-x^*)) \d \tau \bigg) \cdot (x_k - x^*) \notag\\ 
    = & ~ G_k \cdot (x_k - x^*)
\end{align}
where the first step follows from Definition~\ref{def:update_x_k+1}, the second step follows from $g(x^*) = {\bf 0}_d$, the third step follows from  Lemma~\ref{lem:integral_gradient_hessian}, the forth step follows from $H^{-1} H = I$, the fifth step follows from simple algebra, the sixth step follows from simple algebra, 
 the last step follows from rewrite the equation using $G_k$ below
:
\begin{align*}
    G_k := \wt{H}(x_k)^{-1} \cdot \int_0^1 \wt{H}(x_k) - H(x^*+ \tau (x_k - x^*)) \d \tau
\end{align*}

 

Then we can bound $\| G_k \|$ as follows
\begin{align}\label{eq:upper_bound_G_k_divide}
    \| G_k \|
    = & ~ \Big\| \wt{H}(x_k)^{-1} \cdot  \int_0^1 ( \wt{H}(x_k) - H( x^* + \tau (x_k - x^*) ) ) \d \tau \Big\| \notag \\ 
    = & ~ \Big\| \wt{H}(x_k)^{-1} \cdot  \int_0^1 ( \wt{H}(x_k) - H(x_k) + H(x_k) - H( x^* + \tau (x_k - x^*) ) ) \d \tau \Big\| \notag \\
    = & ~ \Big\| \wt{H}(x_k)^{-1} \cdot  \Big(\int_0^1 ( \wt{H}(x_k) - H(x_k)) \d \tau + \int_0^1(H(x_k) - H( x^* + \tau (x_k - x^*) ) ) \d \tau \Big) \Big\| \notag \\
    \leq & ~\Big\| \wt{H}(x_k)^{-1} \cdot \int_0^1 ( \wt{H}(x_k) - H(x_k) ) \d \tau \Big\| + \Big\| \wt{H}(x_k)^{-1} \cdot \int_0^1 (H(x_k) - H( x^* + \tau (x_k - x^*) )) \d \tau \Big\|
\end{align}
where the first step follows from the definition of $G_k$, the second step follows from simple algebra, the third step follows from simple algebra, the last step follows from Fact~\ref{fac:matrix_norm}.
 

For the first term, we have

\begin{align}\label{eq:upper_bound_G_k_1st_step}
    \Big\| \wt{H}(x_k)^{-1} \cdot \int_0^1 ( \wt{H}(x_k) - H(x_k) ) \d \tau \Big\| \notag
    = & ~ \Big\| \wt{H}(x_k)^{-1} \cdot ( \wt{H}(x_k) - H(x_k) ) \int_0^1 \d \tau \Big\| \\ \notag
    = & ~ \| \wt{H}(x_k)^{-1}  ( \wt{H}(x_k) - H(x_k) ) \|  \\
    \leq & ~ 2\epsilon_H
\end{align}
where the first step follows from simple algebra, the second step follows from $\int_0^1 \d \tau = 1$, the third step follows from Fact~\ref{fac:psd}.

For the second term, we have
\begin{align}\label{eq:upper_bound_G_k_2nd_step}
    & ~\Big\| \wt{H}(x_k)^{-1} \cdot \int_0^1 (H(x_k) - H( x^* + \tau (x_k - x^*) ))  \d \tau \Big\|  \notag \\
    \leq & ~ \| \wt{H}(x_k)^{-1} \| \cdot \Big\| \int_0^1 (H(x_k) - H( x^* + \tau (x_k - x^*) ) ) \d \tau \Big\| \notag \\ 
    \leq & ~ (1+\epsilon_H) \cdot \| H(x_k)^{-1} \| \cdot \Big\| \int_0^1 (H(x_k) - H( x^* + \tau (x_k - x^*) ) )  \d \tau \Big\| \notag\\ 
    \leq & ~ (1+\epsilon_H) \cdot \| H(x_k)^{-1} \| \cdot \int_0^1 \Big\|  H(x_k) - H( x^* + \tau (x_k - x^*) )  \Big\| \d \tau \notag\\ 
    \leq & ~ (1+\epsilon_H) \cdot \| H(x_k)^{-1} \| \cdot \max_{\tau \in [0,1]} \Big\|  H(x_k) - H( x^* + \tau (x_k - x^*) )  \Big\| \notag\\ 
    \leq & ~ (1+\epsilon_H) \cdot \| H(x_k)^{-1} \| \cdot r_k M \notag\\ 
    \leq & ~ (1+\epsilon_H) \cdot (l - M r_k)^{-1} \cdot r_k M \notag\\
    \leq & ~ 2 \frac{M r_k}{l- M r_k}
\end{align}
where the first step follows from Fact~\ref{fac:matrix_norm}, the second step follows from $(1+\epsilon_H)H(x_k)<\wt{H}(x_k)$ and Fact~\ref{fac:matrix_norm}
, the third step follows from $\|\int \d \tau \| \leq \int \| \| \d \tau$, the forth step follows from $\int_0^1 f(\tau) \d \tau \leq \max_{\tau \in [0,1]} f(\tau)$
, the fifth step follows from Definition~\ref{def:f_ass}
, the sixth step follows from $\|H(x_k)\| \ge l-Mr_k$ (see Lemma~\ref{lem:lower_bound_spectral_hessian}), the last step follows from $\epsilon_H \in (0,1)$.


Thus, we have, 
    \begin{align*}
        r_{k+1}
        = & ~ \|G_k \cdot (x_k - x^*)\| \\
        \leq & ~ \|G_k\| \cdot \|(x_k - x^*)\| \\
        = & ~ \|G_k\| \cdot r_k \\
        \leq & ~ 2(\epsilon_H +  \frac{M r_k}{l- M r_k}) \cdot r_k
    \end{align*}

 
where the first step follows from Eq.~\eqref{eq:variant_of_r_k_plus_1} and by definition of $r_k$, the second step follows form $\|ab\| \leq \|a\|\|b\|, \forall a,b$, the third step follows from $r_k = \|x_k - x^*\|$, the last step follows from Eq.~\eqref{eq:upper_bound_G_k_divide}, Eq.~\eqref{eq:upper_bound_G_k_1st_step}, and Eq.~\eqref{eq:upper_bound_G_k_2nd_step}.

\end{proof}


\subsection{Induction}\label{sec:newton:induction}

\begin{lemma}\label{lem:newton_induction}
If the following condition hold
\begin{itemize}
    \item $\epsilon_H = 0.01$
    \item $r_{i} \leq 0.4 r_{i-1}$, for all $i \in [k]$
    \item $M \cdot r_i \leq 0.1 l$, for all $i \in [k]$
\end{itemize}
Then we have
\begin{itemize}
    \item $r_{k+1} \leq 0.4 r_k$
    \item $M \cdot r_{k+1} \leq 0.1 l$
\end{itemize}
\end{lemma}
\begin{proof}

{\bf Proof of Part 1.}

 We have
 \begin{align*}
r_{k+1} \leq &  2(\epsilon_H + \frac{M r_k}{ l - M r_k } ) \cdot r_k \\
\leq & ~ 2 ( 0.01 + \frac{M r_k}{ l - M r_k }  ) r_k \\
\leq & ~ 2 (0.01 + \frac{0.1 l}{l - 0.1l}) r_k \\
\leq & ~ 0.4 r_k
 \end{align*}
 where the first step follows from Lemma~\ref{lem:one_step_shrinking}.


{\bf Proof of Part 2.}

We have 
\begin{align*}
M \cdot r_{k+1} 
\leq & ~ M \cdot 0.4 r_{k} \\
\leq & ~ 0.4 \cdot 0.1 l \\
\leq & ~ 0.1 l
\end{align*}

\end{proof}

\subsection{Main Result}\label{sec:newton:main}

We state our main result as follows.
\begin{theorem}[Formal version of Theorem~\ref{thm:main_informal}]\label{thm:main_formal}
Given matrix $A \in \R^{n \times d}$, $b \in \R^n$, and $w \in \R^n$. 

Let $f$ be any of functions $\exp, \cosh$ and $\sinh$.

Let $x^*$ denote the optimal solution of 
\begin{align*}
\min_{x \in \R^d} 0.5 \| f(Ax) - b \|_2^2 + 0.5 \| \diag(w) A x \|_2^2
\end{align*}
that $g(x^*) = {\bf 0}_d$ and $\| x^* \|_2 \leq R$.

Let $\| A \| \leq R, \| b \|_2 \leq R$.

\begin{itemize}
    \item Let $w_{i}^2 \geq 0.5 b_i^2 + l$ for all $i \in [n]$. (If $f=\exp$, see Lemma~\ref{lem:hessian_is_pd_exp})
    \item Let $w_{i}^2 \geq 0.5 b_i^2 + l/\sigma_{\min}(A)^2 + 1$ for all $i \in [n]$. (If $f=\cosh$, see Lemma~\ref{lem:hessian_is_pd_cosh})
    \item Let $w_{i}^2 \geq 0.5 b_i^2 + l/\sigma_{\min}(A)^2 - 1$ for all $i \in [n]$. (If $f=\sinh$, see Lemma~\ref{lem:hessian_is_pd_sinh})
    
\end{itemize}


Let $M = \exp(6R^2)$.

Let $x_0$ denote an initial point such that $M \| x_0 - x^* \|_2 \leq 0.1 l$.

For any accuracy parameter $\epsilon \in (0,0.1)$ and failure probability $\delta \in (0,0.1)$.  There is a randomized algorithm (Algorithm~\ref{alg:main}) that runs $\log(\| x_0 - x^* \|_2/ \epsilon)$ iterations and spend  
\begin{align*}
O( (\nnz(A) + d^{\omega} ) \cdot \poly(\log(n/\delta)) 
\end{align*}
time per iteration, and finally outputs a vector $\wt{x} \in \R^d$ such that
 
\begin{align*}
\| \wt{x} - x^* \|_2 \leq \epsilon
\end{align*}
holds with probability at least $1-\delta$.
 
\end{theorem}
\begin{proof}
{\bf Proof of $\exp$ function.}


It follows from combining Lemma~\ref{lem:hessian_is_pd_exp}, Lemma~\ref{lem:newton_induction}, Lemma~\ref{lem:subsample}, and
Lemma~\ref{lem:one_step_shrinking}.

After $T$ iterations, we have
\begin{align*}
\| x_T - x^* \|_2 \leq 0.4^T \cdot \| x_0 - x^* \|_2
\end{align*}
By choice of $T$, we get the desired bound. The failure probability is following from union bound over $T$ iterations.

{\bf Proof of $\cosh$ function.}


It follows from combining Lemma~\ref{lem:hessian_is_pd_cosh}, Lemma~\ref{lem:newton_induction}, Lemma~\ref{lem:subsample}, and
Lemma~\ref{lem:one_step_shrinking}.

{\bf Proof of $\sinh$ function.}


It follows from combining Lemma~\ref{lem:hessian_is_pd_sinh}, Lemma~\ref{lem:newton_induction}, Lemma~\ref{lem:subsample}, and
Lemma~\ref{lem:one_step_shrinking}.

\end{proof}

\ifdefined\isarxiv
%\section*{Acknowledgments}
\bibliographystyle{alpha}
\bibliography{ref}
\else
\bibliography{ref}
%\bibliographystyle{icml2022}
\bibliographystyle{alpha}

\fi



\newpage
\onecolumn
\appendix





%%%% Cut-line between first 10 pages and appendix







%%% some writing rules

%% Writing rule for creating tags.
%% Tags :
%% Theorem    \ref{thm:bla_bla}
%% Lemma      \ref{lem:bla_bla}
%% Claim      \ref{cla:bla_bla}
%% Corollary  \ref{cor:bla_bla}
%% Fact       \ref{fac:bla_bla}
%% Definition \ref{def:bla_bla}
%% Section    \ref{sec:bla_bla}
%% Subsection \ref{sub:bla_bla}
%% Equation   \ref{eq:bla_bla}



\end{document}



%%%%%%%%%%%%%%%%%%%%%%%%%%%%%%%%%%%%%%%%%%%%%%%%%%%%%%%%%%%%%%%%%%%%%%%%%%%%%%%%%%%%%%%%%%%%%%%%%%%%%%%%%%%%%%%%%%%%%%%%%%%%%%%%%%%%%%%%%%%%%%%%%%%%%%%%%%%%%%%%%%%%%%%%%%%%%%%%%%%%%%%%%%%%%%%%%%%%%%%%%%%%%%%%%%%%%%%%%%%%%%%%%%%%%%%%%%%%%%%%%%%%%%%%%%%%%%%%%%%%%%%%%%%%%%%%%%%%%%%%%%%%%%%%%%%%%%%%%%%%%%%%%%%%%%%%%%%%%%%%%%%%%%%%%%%%%%%%%%%%%%%%%%%%%%%%%%%%%%%%%%%%%%%%%%%%%%%%%%%%%%%%%%%%%%%%%%%%%%%%%%%%%%%%%%%%%%%%%%%%%%%%%%%%%%%%%%%%%%%%%%%%%%%%%%%%%%%%%%%%%%
