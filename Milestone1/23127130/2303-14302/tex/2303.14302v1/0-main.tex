% CVPR 2023 Paper Template
% based on the CVPR template provided by Ming-Ming Cheng (https://github.com/MCG-NKU/CVPR_Template)
% modified and extended by Stefan Roth (stefan.roth@NOSPAMtu-darmstadt.de)

\documentclass[10pt,twocolumn,letterpaper]{article}

%%%%%%%%% PAPER TYPE  - PLEASE UPDATE FOR FINAL VERSION
% \usepackage[review]{cvpr}      % To produce the REVIEW version
\usepackage{cvpr}              % To produce the CAMERA-READY version
%\usepackage[pagenumbers]{cvpr} % To force page numbers, e.g. for an arXiv version

% Include other packages here, before hyperref.
\usepackage{graphicx}
\usepackage{amsmath}
\usepackage{amssymb}
\usepackage{booktabs}
\usepackage{pifont}

\usepackage{bm}
\usepackage{enumitem}
\usepackage{makecell}
\usepackage{multirow}
\usepackage{soul}
\usepackage{tabularx}
\usepackage{xcolor}



\newcommand{\best}[1]{{\color{blue}{\textbf{#1}}}} % for best result
\newcommand{\second}[1]{{\color{black}{\textbf{#1}}}}% for 2nd result
\newcommand{\yek}[1]{\textcolor{blue}{TODO(yek@): #1}}
\newcommand{\junjiek}[1]{\textcolor{orange}{TODO(junjiek@): #1}}
\newcommand{\fengyang}[1]{\textcolor{red}{TODO(fengyang@): #1}}
\newcommand{\ours}{VILA}
\newcommand{\vect}[1]{\boldsymbol{#1}}

\newcommand{\cmark}{\ding{51}}%
\newcommand{\xmark}{\ding{55}}%

% It is strongly recommended to use hyperref, especially for the review version.
% hyperref with option pagebackref eases the reviewers' job.
% Please disable hyperref *only* if you encounter grave issues, e.g. with the
% file validation for the camera-ready version.
%
% If you comment hyperref and then uncomment it, you should delete
% ReviewTempalte.aux before re-running LaTeX.
% (Or just hit 'q' on the first LaTeX run, let it finish, and you
%  should be clear).
\usepackage[pagebackref,breaklinks,colorlinks]{hyperref}


% Support for easy cross-referencing
\usepackage[capitalize]{cleveref}
\crefname{section}{Sec.}{Secs.}
\Crefname{section}{Section}{Sections}
\Crefname{table}{Table}{Tables}
\crefname{table}{Tab.}{Tabs.}


%%%%%%%%% PAPER ID  - PLEASE UPDATE
\def\cvprPaperID{2077} % *** Enter the CVPR Paper ID here
\def\confName{CVPR}
\def\confYear{2023}


\begin{document}

%%%%%%%%% TITLE - PLEASE UPDATE
\title{\ours: Learning Image Aesthetics from User Comments \\ with Vision-Language Pretraining}

\author{
Junjie Ke, Keren Ye, Jiahui Yu, Yonghui Wu, Peyman Milanfar, Feng Yang \\
Google Research\\
% For a paper whose authors are all at the same institution,
% omit the following lines up until the closing ``}''.
% Additional authors and addresses can be added with ``\and'',
% just like the second author.
% To save space, use either the email address or home page, not both
{\tt\small  {\{junjiek, yek, jiahuiyu, yonghui, milanfar, fengyang\}@google.com}}
}

\maketitle

%%%%%%%%% ABSTRACT
\begin{abstract}
Assessing the aesthetics of an image is challenging, as it is influenced by multiple factors including composition, color, style, and high-level semantics. Existing image aesthetic assessment (IAA) methods primarily rely on human-labeled rating scores, which oversimplify the visual aesthetic information that humans perceive. Conversely, user comments offer more comprehensive information and are a more natural way to express human opinions and preferences regarding image aesthetics. In light of this, we propose learning image aesthetics from user comments, and exploring vision-language pretraining methods to learn multimodal aesthetic representations. Specifically, we pretrain an image-text encoder-decoder model with image-comment pairs, using contrastive and generative objectives to learn rich and generic aesthetic semantics without human labels.  To efficiently adapt the pretrained model for downstream IAA tasks, we further propose a lightweight rank-based adapter that employs text as an anchor to learn the aesthetic ranking concept. Our results show that our pretrained aesthetic vision-language model outperforms prior works on image aesthetic captioning over the AVA-Captions dataset, and it has powerful zero-shot capability for aesthetic tasks such as zero-shot style classification and zero-shot IAA, surpassing many supervised baselines. With only minimal finetuning parameters using the proposed adapter module, our model achieves state-of-the-art IAA performance over the AVA dataset.
\end{abstract}

\section{Introduction}


Recent years have witnessed the rise of human digitization~\cite{habermannDeepCapMonocularHuman2020,alexanderCREATINGPHOTOREALDIGITAL,pengNeuralBodyImplicit2021,alldieckDetailedHumanAvatars2018, rajANRArticulatedNeural2020}. This technology greatly impacts the entertainment, education, design, and engineering industry.
There is a well-developed industry solution for this task.
High-fidelity reconstruction of humans can be achieved either with full-body laser scans~\cite{saitoSCANimateWeaklySupervised2021}, dense synchronized multi-view cameras~\cite{xiangModelingClothingSeparate2021a,xiangDressingAvatarsDeep2022a}, or light stages~\cite{alexanderCREATINGPHOTOREALDIGITAL}.
However, these settings are expensive and tedious to deploy and consist of a complex processing pipeline, preventing the technology's democratization.

Another solution is to view the problem as inverse rendering and learn digital humans directly from custom-collected data.
Traditional approaches directly optimize explicit mesh representation~\cite{loperSMPLSkinnedMultiperson2015, fangRMPERegionalMultiperson2018, pavlakosExpressiveBodyCapture2019} which suffers from the problems of smooth geometry and coarse textures~\cite{prokudinSMPLpixNeuralAvatars2020,alldieckVideoBasedReconstruction2018}. Besides, they require professional artists to design human templates, rigging, and unwrapped UV coordinates.
Recently, with the help of volumetric-based implicit representations~\cite{mildenhallNeRFRepresentingScenes2020, parkDeepSDFLearningContinuous2019, meschederOccupancyNetworksLearning2019} and neural rendering~\cite{laineModularPrimitivesHighPerformance2020, liuSoftRasterizerDifferentiable2019, thiesDeferredNeuralRendering2019}, 
one can easily digitize a quality-plausible human avatar from video footage~\cite{jiangNeuManNeuralHuman2022,wengHumanNeRFFreeviewpointRendering}.
Particularly, volumetric-based implicit representations~\cite{mildenhallNeRFRepresentingScenes2020, pengNeuralBodyImplicit2021} can reconstruct scenes or objects with much higher fidelity against previous neural renderer~\cite{thiesDeferredNeuralRendering2019,prokudinSMPLpixNeuralAvatars2020}, and is more user-friendly as it does not need any human templates, pre-set rigging, or UV coordinates.
Captured visual footage and corresponding skeleton tracking are enough for training.
However, better reconstructions and more friendly usability are at the expense of the following factors.
1) \textbf{Inefficiency:}
They require longer optimization times (typically tens of hours or days) and inference slowly.
Volume rendering~\cite{mildenhallNeRFRepresentingScenes2020,lombardiNeuralVolumesLearning2019} formulates images by querying the densities and colors of millions of spatial coordinates. 
In the training stage, due to memory constraints, only a small fraction of points are sampled which leads to slow convergence speed.
2) \textbf{Entangled representations}:
The geometry, materials, and motion dynamics are entangled in the neural networks. 
Due to the implicit nature of neural nets, one can hardly edit one property without touching the others~\cite{yuanNeRFEditingGeometryEditing2022a,liuEditingConditionalRadiance2021}.
3) \textbf{Graphics incompatibility}:
Volume rendering is incompatible with the current popular graphic pipeline,
which renders triangular/quadrilateral meshes efficiently with the rasterization technique.
Many downstream applications require mesh rasterization in their workflow (\eg, editing~\cite{foundationBlenderOrgHome}, simulation~\cite{benderPositionBasedSimulationMethods2015}, real-time rendering~\cite{akenine2019real}, ray-tracing~\cite{waldRTXRayTracing}).
Although there are approaches~\cite{lorensenMarchingCubesHigh,labelleIsosurfaceStuffingFast2007} can convert volumetric fields into meshes, the gaps from discrete sampling degrade the output quality in terms of both meshes and textures.


To address these issues, we present \textbf{EMA}, a method based on \textbf{E}fficient \textbf{M}eshy neural fields to reconstruct animatable human \textbf{A}vatars.
Our method enjoys flexibility from implicit representations and efficiency from explicit meshes, yet still maintains high-fidelity reconstruction quality.
Given video sequences and the corresponding pose tracking, our method digitizes humans in terms of canonical triangular meshes, physically-based rendering (PBR) materials, and skinning weights \textit{w.r.t.} skeletons.
We jointly learn the above components via inverse rendering~\cite{laineModularPrimitivesHighPerformance2020,chenDIBRLearningPredict2021,chenLearningPredict3D2019} in an end-to-end manner.
Each of them is derived from a separate neural field, which relaxes the requirements of a preset human template, rigging, or UV coordinates.
Specifically, we predict a canonical mesh out of a signed distance field (SDF) by differentiable marching tetrahedra~\cite{shenDeepMarchingTetrahedra2021,gaoGET3DGenerativeModel,gaoLearningDeformableTetrahedral2020,munkbergExtractingTriangular3D2022}, then we extend the marching tetrahedra~\cite{shenDeepMarchingTetrahedra2021} for spatial-varying materials by utilizing a neural field to predict PBR materials \textit{on the mesh surfaces} after rasterization~\cite{munkbergExtractingTriangular3D2022,hasselgrenShapeLightMaterial2022,laineModularPrimitivesHighPerformance2020}.
To make the canonical mesh animatable, we take another neural field to model the forward linear blend skinning for the meshes. 
Given a posed skeleton, the canonical mesh is then transformed into the corresponding poses.
Finally, we shade the mesh with a rasterization-based differentiable renderer~\cite{laineModularPrimitivesHighPerformance2020} and train our models with a photo-metric loss.
After training, we export the mesh with materials and discard the neural fields.

\looseness=-1
There are several merits of our method design.
1) \textbf{Efficiency}:
Powered by efficient mesh rendering, our method can render in real-time.
Besides, the training speed is boosted as well, 
since we compute loss holistically on the whole image and the gradients only flow on the mesh surface. In contrast, volume rendering takes limited pixels for loss computation and back-propagates the gradients in the whole space.
Our method only needs about an hour of training and minutes of optimization are enough for plausible avatar reconstruction.
2) \textbf{Disentangled representations}:
Our shape, materials, and motion modules are disentangled naturally by design, which facilitates editing. 
Besides, Canonical meshes with forward skinning modeling handle the out-of-distribution poses better.
3) \textbf{Graphics compatibility}:
Our derived mesh representation is compatible with 
the prominent graphic pipeline, which leads to instant downstream applications (\eg, the shape and materials can be edited directly in design software~\cite{foundationBlenderOrgHome}).
To further improve reconstruction quality, we additionally optimize image-based environment lights and non-rigid motions.


We conduct extensive experiments on standards benchmarks H36M~\cite{ionescuHuman36MLarge2014b} and ZJU-MoCap~\cite{pengNeuralBodyImplicit2021}.
Our method achieves very competitive performance for novel view synthesis, generalizes better for novel poses, 
and significantly improves both training time and inference speed against previous arts.
Our research-oriented code reaches real-time inference speed (100+ FPS for rendering $512\times512$ images).
We in addition showcase applications including novel pose synthesis, material editing, and relighting.
\section{Related Work} \label{sec:related work}
\vspace{-0.2cm}
{\noindent \bf Vision-Language Pre-training.} In the early literature, \cite{Mori99,Frome13,Weston11} explore jointly training image-text embeddings using paired text documents. Recently, some studies have further scaled up the training with large-scale web data to form ``the \textbf{foundation} models'', {\em e.g.}, CLIP~\cite{Radford21}, ALIGN~\cite{Jia21}, Florence~\cite{yuan2021florence}, FILIP~\cite{yao2021filip}, VideoCLIP~\cite{xu2021videoclip}, and LiT~\cite{zhai2022lit}. These foundation models usually contain one visual encoder and one textual encoder, which are trained using simple noise contrastive learning for powerful cross-modal representations. They have shown promising potential in many tasks, such as image classification and detection, action recognition, and retrieval. In this paper, we use CLIP for low-shot temporal action localization, but the same technique should be applicable to other foundation models as well.



\vspace{0.1cm}
{\noindent \bf Prompting} refers to leveraging input instructions to steer foundation models for desired outputs. In the NLP domain, early papers~\cite{Gao21,Jiang20,Timo21,Shin20} focus on handcrafted prompt templates. To avoid labor and increase flexibility, some studies~\cite{Lester21,li21-prefixtuning,li2021prefix} propose learnable prompt tuning at the textual stream, showing strong low-shot generalization. In the CV domain, some recent papers~\cite{zhou2019learn,zhou2022conditional,ju2022prompting} introduce such randomly initialized prompt tuning to handle visual tasks, {\em e.g.}, image understanding~\cite{zhu2022prompt,lu2022prompt,yang2022learning,ma2023diffusionseg} and video understanding~\cite{jia2022visual,nag2022zero,ni2022expanding}. However, these studies ignore lexical ambiguity of category names, and cases that are not easy to describe in text. This paper designs novel conditional prompt tuning and language descriptions from LLMs, to solve these issues. 



\vspace{0.1cm}
{\noindent \bf Closed-set Temporal Action Localization} considers to detect and classify action instances from one pre-defined category list. Specifically, existing methods can be divided into two popular supervisions, {\em i.e.}, strong~\cite{zeng2019graph,lin2021learning,qing2021temporal} and weak~\cite{wang2017untrimmednets,ju2023constraint,ju2020point,yudistira2022weakly}. Strong supervision gives precise boundary labels and category labels for training. There are two detailed pipelines: the top-down framework~\cite{shou2016temporal,shou2017cdc,gao2017turn,chao2018rethinking,lin2017single,xu2017r,tan2021relaxed,zhu2021enriching,wang2022rcl,xu2020g} pre-defines extensive anchors, adopts fixed-length sliding windows to produce initial proposals, then regresses to refine boundaries; the bottom-up framework~\cite{zhao2017temporal,lin2018bsn,lin2019bmn,vo2023aoe,zhao2020bottom,bai2020boundary} learns frame-wise boundary detectors for the boundary frames, then groups extreme frames or estimates action lengths for proposal generation. In addition, several works~\cite{gao2018ctap,liu2019multi,yang2020revisiting} used various fusion strategies to complement these frameworks. On the other hand, weak supervision trains without boundary labels to alleviate annotation costs. The video-level setting learns from category labels~\cite{paul2018w,ju2022distilling}, the CAS-based framework~\cite{liu2019completeness,ju2021adaptive,min2020adversarial,narayan2021d2,lee2019background,lee2021weakly,zhao2021soda} and attention-based framework~\cite{nguyen2018weakly,luo2021action,nguyen2019weakly,shi2020weakly,gao2022fine,he2022asm,huang2021foreground,luo2020weakly,ma2022weakly} have been well studied. To generate better results from CAS or attention, some studies~\cite{shou2018autoloc,liu2019weakly} improved post-processing. To balance cost and performance, some papers introduced single-frame annotations~\cite{ju2021divide,ma2020sf,lee2021learning,yang2021background,mettes2019pointly} or instance-number annotations~\cite{narayan20193c,xu2019segregated}. 

Nevertheless, all the above methods assume that action categories remain identical for training and testing, which is an over-simplification of real application scenarios, limiting practical uses of the vision system.



\vspace{0.1cm} 
{\noindent \bf Low-Shot Temporal Action Localization} considers more realistic scenarios: generalize TAL towards action categories that are unseen (zero-shot) or with several support samples (few-shot). Existing methods~\cite{ju2022prompting,nag2022zero,zhang2022ow,bao2022opental} most rely on foundational models pre-trained on large-scale image-caption pairs for help. Typically, E-Prompt~\cite{ju2022prompting} is the first to construct wide baselines with popular prompt tuning~\cite{Lester21,li21-prefixtuning} and vanilla temporal modeling. STALE~\cite{nag2022zero} explores the one-stage framework to further simplify usage. Although promising, all above methods meet two main challenges: (1) For category semantics, the definition may be vague, inaccurate, or incomplete. (2) For visual motions, temporal modeling may be insufficient. In this paper, for detailed category understanding, we design novel language descriptions from LLMs and vision-conditional prompt tuning; for clearer motion understanding, we introduce optical flows to provide explicit motion inputs. 





\vspace{-0.3em}
\section{Method}
\vspace{-0.3em}

Our sensitivity-aware visual parameter-efficient fine-tuning consists of two stages. In the first stage, SPT measures the task-specific sensitivity for the pre-trained parameters (Section~\ref{subsec:sensitivity}). Based on the parameter sensitivity and a given parameter budget, SPT then adaptively allocates trainable parameters to task-specific important positions (Section~\ref{subsec:SPT}).

\vspace{-0.3em}
\subsection{Task-specific Parameter Sensitivity}
\label{subsec:sensitivity}
\vspace{-0.3em}

Recent research has observed that pre-trained backbone parameters exhibit varying feature patterns~\cite{raghu2021vision,naseer2021intriguing} and criticality~\cite{zhang2019all,chatterji2019intriguing} at distinct positions. 
Moreover, when transferred to downstream tasks, their efficacy varies depending on how much pre-trained features are reused and how well they adapt to the specific domain gap~\cite{yosinski2014transferable,kumar2022finetuning,neyshabur2020being}. Motivated by these observations, we argue that not all parameters contribute equally to the performance across different tasks in PEFT and propose a new criterion to measure the sensitivity of the parameters in the pre-trained backbone for a given task.

Specifically, given the training dataset $\gD_t$ for the $t$-th task and the pre-trained model weights $\vw=\left\{w_1, w_2, \ldots, w_N\right\}\in \sR^N$ where $N$ is the total number of parameters, the objective for the task is to minimize the empirical risk: $\min_{\vw} E(\gD_t, \vw)$.
We denote the parameter sensitivity \bohan{set} as $\gS=\{s_1, \ldots, s_N\}$ and the sensitivity $s_n$ for parameter $w_n$ is measured by the empirical risk difference when tuning it:
\begin{equation}
\vspace{-0.3em}
    \begin{aligned}
        s_n = E(\gD_t, \vw)-E(\gD_t, \vw\mid w_n=w_n^*),
    \end{aligned}
\label{eq:sensitivity}
\end{equation}
where $w_n^*=\underset{w_n}{\rm argmin}(E(\gD_t, \vw))$. We can reparameterize the tuned parameters as  $w_n^*=w_n+\Delta_{w_n}$, where $\Delta_{w_n}$ denotes the update for $w_n$ after tuning. Here we individually measure the sensitivity of each parameter, which is reasonable given that most of the parameters are frozen during fine-tuning in PEFT. However, it is still computationally intensive to compute Eq.~(\ref{eq:sensitivity}) for two reasons. Firstly, getting the empirical risk for $N$ parameters requires forwarding the entire network $N$ times, which is time-consuming. Secondly, it is challenging to derive $\Delta_{w_n}$, as we have to tune each individual $w_n$ until convergence.

{\begin{algorithm}[t!]
\caption{\label{alg:tps} Computing task-specific parameter sensitivities}
\begin{algorithmic}
    \STATE \textbf{Input:} Pre-trained model with network parameters $\vw$, training set $\gD_t$ for the $t$-th task, and number of training samples $C$ used to calculate the parameter sensitivities
    \STATE \textbf{Output:} Sensitivity set $\gS=\{s_1, \ldots, s_N\}$
    \STATE Initialize $\gS=\{0\}^N$
    \FOR{$i\in\{1,\ldots,C\}$}
        \STATE Get the $i$-th training sample of $\gD_t$
	    \STATE Compute loss $E$
		\STATE Compute gradients $\vg$
		\FOR{$n\in\{1,\ldots,N\}$}
                \STATE Update sensitivity for the $n$-th parameter: $s_{n} = s_{n} + g_n^2$
		    \ENDFOR
    \ENDFOR
\end{algorithmic}
\end{algorithm}}


\begin{figure*}[t]
\begin{center}
    \includegraphics[width=\linewidth]{main_figure.pdf}
\end{center}\vspace{-2em}
\caption{Overview of our trainable parameter allocation strategy. With the parameter sensitivity \bohan{set} $\gS$, we first get the top-$\tau$ sensitive parameters. Instead of directly tuning these sensitive parameters, we also boost the representational capability by replacing unstructured tuning with structured tuning at sensitive weight matrices that have a large number of sensitive parameters, which can be implemented by an existing structured tuning method, \eg, LoRA~\cite{hu2022lora} and Adapter~\cite{houlsby2019parameter}. Red lines and blocks represent trainable parameters and modules, while blue lines represent frozen parameters.}
\label{fig:main}
\vspace{-1.5em}
\end{figure*}


To overcome the first barrier, we simplify the empirical loss by approximating $s_n$ in the vicinity of $\vw$ by its first-order Taylor expansion
\vspace{-0.3em}
\begin{equation}
\vspace{-0.5em}
    \begin{aligned}
        s_n^{(1)} = -g_n\Delta_{w_n},
    \end{aligned}
\label{eq:first-order}
\end{equation}
where the gradients $\vg=\partial E/\partial\vw$, and $g_n$ is the gradient of the $n$-th element of $\vg$. 
To address the second barrier, following~\cite{liu2018darts,cai2018proxylessnas}, we take the one-step unrolled weight as the surrogate for $w_n^*$ and approximate $\Delta_{w_n}$ in Eq.~(\ref{eq:first-order}) with a single step of gradient descent. We can accordingly get $s_n^{(1)} \approx g_n^2\epsilon$,
where $\epsilon$ is the learning rate. Since $\epsilon$ is the same for all parameters, we can eliminate it when comparing the sensitivity with the other parameters and finally get 
\vspace{-0.5em}
\begin{equation}
\vspace{-0.3em}
    \begin{aligned}
        s_n^{(1)} \approx g_n^2.
    \end{aligned}
\label{eq:first-order-simp}
\end{equation}
Therefore, the sensitivity of a parameter can be efficiently measured by its potential to reduce the loss on the target domain. Note that although our criterion draws inspiration from pruning work~\cite{molchanov2019importance}, it is distinct from it. \cite{molchanov2019importance} measures the parameter importance by the squared change in loss when removing them, \ie, $\left( E(\gD_t, \vw)-E(\gD_t, \vw\mid w_n=0) \right)^2$ and finally derives the parameter importance by $\left( g_n w_n \right)^2$, which is different from our formulations in Eqs.~(\ref{eq:sensitivity}) and~(\ref{eq:first-order-simp}).

In practice, we accumulate $\gS$ from a total number of $C$ training samples ahead of fine-tuning to generate accurate sensitivity as shown in Algorithm~\ref{alg:tps}, where $C$ is a pre-defined hyper-parameter. In Section~\ref{subsec:abl}, we show that employing only 400 training samples is sufficient for getting reasonable parameter sensitivity, which requires only 5.5 seconds with a single GPU for any VTAB-1k dataset with ViT-B/16 backbone~\cite{vit}.

\vspace{-0.3em}
\subsection{Adaptive Trainable Parameters Allocation}
\label{subsec:SPT}
\vspace{-0.2em}

Our next step is to allocate trainable parameters based on the obtained parameter sensitivity set $\gS$ and a desired parameter budget $\tau$. A straightforward solution is to directly tune the top-$\tau$ most sensitive unstructured connections (parameters) \rev{while keeping the rest frozen}, which we name unstructured tuning. Specifically, we select the top-$\tau$ most sensitive weight connections in $\gS$ to form the sensitive weight connection set $\gT$. Then, for \rev{a} weight matrix $\mW\in \sR^{d_{\rm in}\times d_{\rm out}}$, we can get a binary mask $\mM\in \sR^{d_{\rm in}\times d_{\rm out}}$ computed by
\vspace{-0.5em}
\begin{equation}
\vspace{-0.5em}
    {\begin{array}{ll}
    \small
    \begin{aligned}
    \mM^j =
    \left\{\begin{array}{ll} 
    1 ~~~~~ \mW^j \in \gT \\
    0 ~~~~~ \mW^j \notin \gT
    \end{array}\right.
    \end{aligned},
    \small
    \end{array}}
\label{eq:mask}
\end{equation}
where $\mW^j$ and $\mM^j$ are the $j$-th element in $\mW$ and $\mM$, respectively. Accordingly, we can train the sensitive parameters by gradient descent and the updated weight matrix can be formulated as $\mW'\leftarrow \mW - \epsilon\vg_{\mW}\odot\mM$, where $\vg_{\mW}$ is the gradient for $\mW$.

However, considering PEFT approaches generally limit the proportion of trainable parameters to less than 1\%, tuning only a small number of unstructured weight connections might not have enough representational capability to handle the downstream datasets with large domain gaps from the source pre-training data. Therefore, to improve the representational capability, we propose to replace unstructured tuning with structured tuning at the sensitive weight matrices that have a high number of sensitive parameters. To preserve the parameter budget, we can implement structured tuning with an existing efficient structured tuning PEFT method~\cite{hu2022lora,chen2022adaptformer,houlsby2019parameter,jie2022convolutional} that learns to directly adjust \rev{all hidden dimensions at once}. We depict an overview of our trainable parameter allocation strategy in Figure~\ref{fig:main}. For example, we can employ the low-rank reparameterization trick LoRA~\cite{hu2022lora} to the sensitive weight matrices \rev{and the one-step update for $\mW$ can be formulated as}
\vspace{-0.4em}
\begin{equation}
\vspace{-0.4em}
    {\begin{array}{ll}
    \small
    \begin{aligned}
    \mW' = \left\{\begin{array}{ll} 
    \mW + \mW_{\rm down}\mW_{\rm up} & ~~ \text { if } ~~ \sum_{j=0}^{d_{\rm in}\times d_{\rm out}} \mM^j \geq \sigma_{\rm opt} \\
    \mW - \epsilon\vg_{\mW}\odot\mM & ~~ {\rm otherwise}
    \end{array}\right.
    \end{aligned},
    \small
    \end{array}}
\label{eq:weight_updat}
\end{equation}
where $\mW_{\rm down}\in \sR^{d_{\rm in}\times r}$ and $\mW_{\rm up}\in \sR^{r\times d_{\rm out}}$ are two learnable low-rank matrices to approximate the update of $\mW$ and rank $r$ is a hyper-parameter where $r \ll {\rm min}(d_{\rm in},d_{\rm out})$. In this way, we perform structured tuning on $\mW$ when its number of sensitive parameters exceeds $\sigma_{\rm opt}$, whose value depends on the pre-defined type of structured tuning method. For example, since implementing structured tuning with LoRA requires $2\times d_{\rm in} \times d_{\rm out} \times r$ trainable parameters for each sensitive weight matrix, we set $\sigma_{\rm LoRA} \leftarrow 2\times d_{\rm in} \times d_{\rm out} \times r$ to ensure that the number of trainable parameters introduced by structured tuning is always equal to or lower than the number of sensitive parameters.

In this way, our SPT adaptively incorporates both structured and unstructured tuning granularities to enable higher flexibility and stronger representational power, simultaneously. In Section~\ref{subsec:abl}, we show that structured tuning is important for the downstream tasks with larger domain gaps and both unstructured and structured tuning contribute clearly to the superior performance of our SPT.
%%%%%%%%%%%%%%%%%%%%%%%%%%%%%%%%%%%%%%%%%%%%%%%%%

\begin{table*}[t!]
\centering
\caption{{Main Results on OV-COCO and OV-LVIS:} We evaluate box AP with IoU threshold 0.5 ($\mathrm{mAP^{50}}$) on OV-COCO, and box AP ($\mathrm{mAP^{box}}$) and mask AP ($\mathrm{mAP^{mask}}$) on OV-LVIS. Note that $\mathrm{mAP_{novel}}$ and $\mathrm{mAP}$ indicate the performance of zero-shot and the entire of categories, respectively. Lastly, latency implies inference time per image in seconds for OV-LVIS.}
 \vspace*{-0.25cm}
\label{tab:main}
\resizebox{1.0\linewidth}{!}{%
\begin{tabular}{@{}lrrrrrrrrrrr@{}}
\toprule 
& & & & \multicolumn{3}{c}{OV-COCO}  & \multicolumn{5}{c}{OV-LVIS}  \\
\cmidrule(lr){5-7} \cmidrule(lr){8-12}  
{Methods}& Backbone & CLIP & Res. & $\mathrm{mAP^{50}_{novel}}$ & $\mathrm{mAP^{50}_{base}}$ & $\mathrm{mAP^{50}}$ & $\mathrm{mAP^{box}_{novel}}$ & $\mathrm{mAP^{box}}$ & $\mathrm{mAP^{mask}_{novel}}$ & $\mathrm{mAP^{mask}}$ & Latency\,(s)\\ 
\midrule
\multicolumn{1}{r}{\normalsize {\sffamily DETR-based}} \vspace*{0.1cm} \\
% OWL-ViT & ViT-L/14 & ViT-L/14 & 25.6 & 34.7 & - & - \\ 
OV-DETR~\cite{zang2022open} & RN50 & ViT-B/32 & 1333  & 29.4 & 61.0 & 52.7 & 18.0 & 27.4 & 17.4 & 26.6 & 12.28 \\
\textbf{Prompt-OVD} & ViT-B/16 & ViT-L/14 & 840 & \textbf{30.6} & \textbf{63.5} & \textbf{54.9} & \textbf{29.4} & \textbf{33.0} & \textbf{23.1} & 24.2 & 0.58 \\
\cmidrule(lr){1-12}
% & %(memory update / robust learning) &
\multicolumn{1}{r}{\normalsize {\sffamily RCNN-based}}&  \multicolumn{4}{r}{\normalsize \!\!\!\!\!\!\!\!{\sffamily(Latency Range: 0.40 -- 0.70 seconds)}}\vspace*{0.1cm} \\ 
% ViLD-text & 10.1 & 24.9 & 5.9 & 49.3 \\
Detic~\cite{zhou2022detic}     & RN50 & ViT-B/32 & 1333 & 27.8 & 51.0 & 45.0 & 23.6 & 30.4 & 21.4 & \textbf{26.9} & 0.47 \\
ViLD~\cite{guopen2022vild}      & RN50 & ViT-B/32 & 1333 & 27.6 & 59.6 & 51.3& 16.7 & 27.8 & 16.6 & 25.5 & 0.48  \\
F-VLM~\cite{kuoopen2023fvlm}     & RN50 & RN50 & 1024 & 28.0 & 43.7 & 39.6& 20.3 & 27.8 & 18.6 & 24.2 & 0.50 \\
DetPro~\cite{du2022detpro}    & RN50 & ViT-B/32 & 1333  & - & -& - & 20.8 & 28.4 & 19.8 & 25.9 & 0.67\\
% F-VLM~\cite{kuoopen2023fvlm}    & RN50x4 & RN50x4 & 1024 & - & - & 26.3 & 28.5 & 0.72\\ 
% \cmidrule(lr){1-9} 
\bottomrule
\end{tabular}%
}
\vspace*{-0.1cm}
\end{table*}

\begin{table*}[t!]
% \begin{wraptable*}{r}{3cm}

\parbox{0.3\linewidth}{
\centering
\caption{Latency change by modifying OV-DETR to \algname{}.}
\vspace*{-0.25cm}
\label{tab:inference_study}
\resizebox{1.0\linewidth}{!}{
\begin{tabular}{@{}llr@{}}
\toprule 
& {Modification}& Latency (s)\\ 
\midrule
 & OV-DETR & 12.28\\
% \cmidrule{1-3}
\,\,\,(1)\!\! & ResNet $\xrightarrow{}$ ViT & 12.36\\ 
\,\,\,(2)\!\! & ViT $\xrightarrow{}$ ViTDet & 8.75\\ 
\,\,\,(3)\!\! & Prompt-based Decoding & 2.89\\
\,\,\,(4)\!\! & Ensemble with CLIP & 3.03\\
\,\,\,(5)\!\! & RoI Pruning ($\epsilon=0.3)$ & 0.58\\
\bottomrule
\label{table:modification}
\vspace*{-0.4cm}
\end{tabular}%
}}
{\color{white} \,}
\hfill
\parbox{0.33\linewidth}{
\centering
\caption{Performance with varying $\alpha$ when fixing $\beta=0.4$.}
\vspace*{-0.3cm}
\label{tab:alpha}
\resizebox{1.0\linewidth}{!}{%
\begin{tabular}{@{}lrrrr@{}}
\toprule 
{$\alpha$}& $\mathrm{mAP^{box}_{novel}}$ & $\mathrm{mAP^{box}}$ & $\mathrm{mAP^{mask}_{novel}}$ & $\mathrm{mAP^{mask}}$ \\ 
\midrule
% & %(memory update / robust learning) &
% \multicolumn{1}{r}{\footnotesize {\sffamily RCNN-based}} \\
% ViLD-text & 10.1 & 24.9 & 5.9 & 49.3 \\
0.0 & 28.1 & 30.8 & 21.9 & 22.4 \\
0.1 & 28.7 & 32.3 & 22.6 & 23.6\\
% \rowcolor{LightCyan}
\textbf{0.2} & \textbf{29.4} & \textbf{33.0} & \textbf{23.1} & \textbf{24.2}  \\
0.3 & 29.5 & 32.2 & 23.2 & 23.7 \\
0.4 & 29.7 & 30.6 & 23.3 & 22.5 \\
0.5 & 29.9 & 28.8 & 23.5 & 21.2\\
1.0 & 30.5 & 14.5 & 24.0 & 10.8 \\
\bottomrule
\label{table:alpha_search}
\vspace*{-0.4cm}
\end{tabular}%
}}
{\color{white} \,}
\hfill
\parbox{0.33\linewidth}{
\centering
\caption{Performance with varying $\beta$ when fixing $\alpha=0.2$.}
\vspace*{-0.3cm}
\label{tab:beta}
\resizebox{1.0\linewidth}{!}{
\begin{tabular}{@{}lrrrr@{}}
\toprule 
{$\beta$}& $\mathrm{mAP^{box}_{novel}}$ & $\mathrm{mAP^{box}}$ & $\mathrm{mAP^{mask}_{novel}}$ & $\mathrm{mAP^{mask}}$ \\ 
\midrule
% & %(memory update / robust learning) &
% \multicolumn{1}{r}{\footnotesize {\sffamily RCNN-based}} \\
% ViLD-text & 10.1 & 24.9 & 5.9 & 49.3 \\
0.0 & 15.9 & 30.0 & 12.1 & 21.7\\
0.1 & 22.8 & 31.4 & 17.9 & 22.9 \\
0.2 & 29.0 & 32.7 & 22.5 & 23.9 \\
0.3 & 29.3 & 33.0 & 23.0 & 24.1 \\
\textbf{0.4} & \textbf{29.3} & \textbf{33.0} & \textbf{23.1} & \textbf{24.2} \\
0.5 & 28.6 & 33.0 & 22.4 & 24.1 \\
1.0 & 19.6 & 31.5 & 15.3 & 22.9 \\
\bottomrule
\label{table:beta_search}
\vspace*{-0.4cm}
\end{tabular}%
}
}


\vspace*{-0.35cm}
\end{table*}

\section{Evaluation} 

\noindent\textbf{Datasets.} We evaluate our approach on two popularly used benchmark datasets, namely OV-COCO and OV-LVIS, each of which is modified from MS-COCO~\cite{lin2014microsoft} and LVIS\,(v1)~\cite{gupta2019lvis}; OV-COCO has 121K images with 64 classes while OV-LVIS has 100K images with 1,203 classes.
Following OV-DETR~\cite{zang2022open}, COCO is split into 17 novel classes and 48 base classes. LVIS is split into three categories: 337 novel classes, and 866 common or base classes based on the number of training images. Note that we refer to the two datasets as OV-COCO and OV-LVIS, respectively, and only base classes are used for training. 


\smallskip\smallskip
\noindent\textbf{Algorithms.} We compare \algname{} with an end-to-end OVD detection model named OV-DETR\,\cite{zang2022open} (baseline) and four two-stage OVD models. However, it should be noted that the two-stage OVD models are based on Mask-RCNN or allow the use of external large-scale data, which makes a fair comparison with the end-to-end Transformer-based detectors difficult. Therefore, to ensure a fair comparison, we follow two criteria: (1) the results should be obtained by only using base categories in training, i.e., the restricted OVD setup and (2) the models' inference speed should be in the range of 0.4~--~0.7 seconds/image, which is similar to that of our proposed framework. 

%To validate the effectiveness of our method, we chose OV-DETR and four Mask-RCNN-based methods as baselines. Since Mask-RCNN-based approaches have a completely different model architecture from ours, we chose these methods as baselines based on two criteria for a fair comparison: (i) using only the dataset for base categories during training (not allowing to use the additional dataset), and (ii) having inference times between 0.4 and 0.7 seconds, which are similar to Prompt-OVD.

\smallskip\smallskip
\noindent\textbf{Implementation.} The proposed \algname{} builds upon Deformable DETR\,\cite{zhu2020deformable}, similar to OV-DETR. However, we merge the independent ResNet backbone and Transformer encoder into a single ViT encoder using ViTDet~\cite{li2022exploring}. As a result, our architecture is a purely Transformer encoder-decoder structure, following recent fully Transformer detection pipeline\,\cite{litransformer, songvidt}. 

For training, we initialize the backbone weights with a plain ViT backbone that has been pre-trained as Masked Autoencoders on ImageNet-1K. The entire model is then trained end-to-end for 50 epochs with a batch size of 32, a weight decay of 1e-4, and an AdamW optimizer. We set the initial learning rate to 2e-4 while using an image size of 840$\times$840. We implement and test all algorithms using PyTorch on eight NVIDIA V100 GPUs. 

For inference, there are three hyperparameters: the weights $\alpha$ and $\beta$ for ensembling in Eq.\,\eqref{eq:ensemble}; the threshold $\epsilon$ for RoI pruning in Eq.\,\eqref{eq:pruning}. The former weights are set to be $(0.2, 0.35)$ and $(0.2, 0.4)$ for OV-COCO and OV-LVIS, while the latter pruning threshold is set to 0.125 and 0.3 for OV-COCO and OV-LVIS. As for the ensemble for classification, we leverage the CLIP model that uses ViT-L/14 as its image encoder with a image size of 336$\times$336, which adds very little computational overhead with our RoI-based masked attention and RoI pruning. The detailed analysis of the hyperparameters and additional overhead due to using CLIP is provided in Section \ref{sec:abl_study}. 

In addition, we need to incorporate the mask head into our model for the evaluation on OV-LVIS, as RPN-based two-stage methods have reported both box and mask APs. Following the recent literature\,\cite{dong2021solq, song2022extendable}, we extend our DETR-based detector using SOLQ\,\cite{dong2021solq}, which can perform a joint training of object detection and instance segmentation by simply adding a unified query representation module. The mask vector size is set to be 1,024 while keeping remaining hyperparameters to be the same as SOLQ.

\smallskip\smallskip
\noindent\textbf{Evaluation Metrics.} We evaluate the detection accuracy of our method following exactly the same metrics used in prior OVD studies\,\cite{guopen2022vild, zang2022open, zhong2022regionclip, minderer2022owlvit}. Specifically, for OV-COCO, we use $\mathrm{mAP^{50}}$ which is a measure of the box average precision\,(AP) with an IoU threshold of 0.5. On the other hand, for OV-LVIS, we use both box mAP ($\mathrm{mAP^{box}}$) and mask mAP ($\mathrm{mAP^{mask}}$) obtained by the joint learning of object detection and instance segmentation, respectively. 

Inference time is also a crucial metric for practical applications. To compare the efficiency of different models, we compute the inference time of all methods using the same hardware environment, consisting of a single NVIDIA V100 GPU and six Intel(R) Xeon(R) Gold 5120 CPUs. To ensure an accurate inference time measurement, we calculate the average time of 100 iterations after five initial iterations, using a batch size of 1.

%\noindent\textbf{Implementation.} To enhance the performance of our model, we opt for the DETR~\cite{carion2020end} architecture based on OV-DETR~\cite{zang2022open}, and we replace the backbone and encoder with ViT-DET~\cite{li2022exploring}. The initial weights of the backbone start from the MAE pre-trained model. Our model is trained for 50 epochs, with a 32 batch size, a weight decay of 1e-4, an optimizer AdamW~\cite{loshchilov2017decoupled}, and an initial learning rate of 2e-4, while using an image size of 840x840. It is worth noting that the training process is carried out using 8 NVIDIA A100 GPUs.

%When performing inference, we use the values of ($\alpha$, $\beta) = (0.2, 0.35)$ and (0.2, 0.4) for OV-COCO and OV-LVIS, respectively, to ensemble with CLIP. In addition, we set the values of $\epsilon$ to 0.125 and 0.3 for OV-COCO and OV-LVIS, respectively, for RoI pruning. We limit the number of detections per image to a maximum of 300 and 1500, and the temperature is set to 0.065 and 0.01 for OV-COCO and OV-LVIS, respectively. Finally, we measure the inference time of all methods using the same hardware environment consisting of a NVIDIA V100 GPU and 6 Intel(R) Xeon(R) Gold 5120 CPUs. To ensure accurate inference time measurements, we calculate the average time of 100 iterations after five initial iterations using a batch size of 1.


%\smallskip\smallskip
%\noindent\textbf{Instance Segmentation.} In order to calculate the mask mAP for OV-LVIS, we need to incorporate the mask head into our model. Since the mask head in DETR is a FPN-style network, and our backbone cannot combine with it, we adopt the SOLQ method~\cite{dong2021solq}, which segments objects by jointly learning the unified query representation for three tasks (classification, localization, and segmentation). We utilize the 1024 dimension of the mask vector, while keeping all other parameters the same as SOLQ.

%\smallskip\smallskip
%\noindent\textbf{Evaluation Metrics.} We follow the same metrics as earlier open vocabulary studies. Specifically, for OV-COCO, they use a metric called $\mathrm{mAP^{50}}$ which measures the box AP with an IoU threshold of 0.5. For OV-LVIS, the metrics use both mask mAP ($\mathrm{mAP^{mask}}$) and box mAP ($\mathrm{mAP^{box}}$). 


\subsection{Main Experiment}

We present a comprehensive comparison of \algname{} with other five OVD methods in terms of detection accuracy and speed. To ensure a fair comparison, we only include the results of RCNN-based methods that can operate at a similar inference speed as ours. Table \ref{tab:main} summarizes the results of \algname{} and other five OVD methods.

In general, \algname{} outperforms the previous end-to-end OVD method, OV-DETR, on both datasets. Notably, \algname{}'s inference speed is {$21.2$ times} faster than OV-DETR, thanks to its prompt-based decoding approach. Refer to Section \ref{sec:inference_speed} for an in-depth comparison of efficiency with OV-DETR. %
%
Moreover, Prompt-OVD exhibits superior performance in terms of box mAP, even compared to RCNN-based OVD methods. These results support the effectiveness of our design that utilizes ViT-based CLIP with RoI-based mask attention and RoI pruning, improving the overall performance. Further investigation of the two techniques can be found in Section~\ref{sec:masked_attention} and ~\ref{sec:pruning}.

Although we observe a larger gap between box mAP and mask mAP (29.4 $\mathrm{mAP^{box}_{novel}} \rightarrow 23.1 \mathrm{mAP^{mask}_{novel}})$, this is a result of inheriting the limitation of SOLQ\,\cite{dong2021solq}. Specifically, the vector encoding of 2D segmentation masks using discrete cosine transformation loses object details compared to the conventional FPN-style mask head\,\cite{song2022extendable}.  
%
Furthermore, despite using a larger ViT-B/16 backbone than ResNet-50, \algname{} exhibits comparable inference time to RCNN-based methods, thanks to its simple encoding-decoding pipeline. Therefore, our results demonstrate that \algname{} shows a potential of the end-to-end Transformer-based framework for OVD. 

In Appendices B and C, we discuss potential enhancements to our method and present the results of our experiments on using image queries other than text queries for open-vocabulary object detection, respectively.


%Prompt-OVD performs better than both Mask-RCNN-based and DETR-based models for all metrics on OV-LVIS. Although Prompt-OVD significantly outperforms the other models in terms of box mAP, F-VLM and Detic achieve the highest scores for mask mAP in novel and entire categories, respectively. We believe that SOLQ has limitations in instance segmentation because it only uses 1024 mask vectors that differ from the FPN-style mask head. Consequently, Prompt-OVD has a larger gap between box mAP and mask mAP than models that utilize the FPN-style mask head.

%Also, Table~\ref{tab:main} shows that Prompt-OVD outperforms all other baselines on OV-COCO. It is noteworthy that Prompt-OVD surpasses the others in terms of box $\mathrm{mAP^{50}}$ for both novel and overall categories. We believe that the RoI pruning and score fusion with CLIP may be crucial in improving the overall performance, and we investigate the effects of these techniques in the Section~\ref{sec:abl_study}.


%\smallskip\smallskip
%\noindent\textbf{Inference Time.} Table~\ref{tab:main} also shows the inference time per image on OV-LVIS. Surprisingly, despite using a larger backbone compared to Mask-RCNN-based baselines, there is little difference in inference time. Furthermore, Prompt-OVD reduces the inference time for OV-DETR, which is the method based on DETR, by up to $\mathsf{24x}$.


% \begin{table*}[t!]
% \centering
% \caption{Main Results on OV-LVIS}
% \vspace*{-0.2cm}
% \label{tab:main_lvis}
% \resizebox{1.0\linewidth}{!}{%
% \begin{tabular}{@{}lrrrrrrrr@{}}
% \toprule 
% % & & & \multicolumn{2}{c}{OV-LVIS} & \multicolumn{2}{c}{OV-COCO} \\
% % \cmidrule(lr){4-5} \cmidrule(lr){6-7}  
% {Methods}& Backbone & CLIP & Res. & $\mathrm{mAP^{box}_{novel}}$ & $\mathrm{mAP^{box}}$ & $\mathrm{mAP^{mask}_{novel}}$ & $\mathrm{mAP^{mask}}$ & s/img\\ 
% \midrule
% \multicolumn{1}{r}{\footnotesize {\sffamily DETR-based}} \\
% % OWL-ViT & ViT-L/14 & ViT-L/14 & 25.6 & 34.7 & - & - \\ 
% OV-DETR~\cite{zang2022open} & RN50 & ViT-B/32 & 1024 & 18.0 & 27.4 & 17.4 & 26.6 & 12.28\\
% \textbf{Prompt-OVD} & ViT-B/16 & ViT-L/14 & 840 & \textbf{29.4} & \textbf{33.0} & \textbf{23.1} & 24.2 & 0.58\\
% \cmidrule(lr){1-9}

% % & %(memory update / robust learning) &
% \multicolumn{1}{r}{\footnotesize {\sffamily RCNN-based}} \\
% % ViLD-text & 10.1 & 24.9 & 5.9 & 49.3 \\
% Detic~\cite{zhou2022detic}     & CenterNet2 & ViT-B/32 & 1333 & 23.6 & 30.4 & 21.4 & \textbf{26.9} & 0.47 \\
% ViLD~\cite{guopen2022vild}      & RN50 & ViT-B/32 & 1333 & 16.7 & 27.8 & 16.6 & 25.5 & 0.48 \\
% F-VLM~\cite{kuoopen2023fvlm}     & RN50 & RN50 & 1024 & 20.3 & 27.8 & 18.6 & 24.2 & 0.50 \\

% DetPro~\cite{du2022detpro}    & RN50 & ViT-B/32 & 1333 & 20.8 & 28.4 & 19.8 & 25.9 & 0.67 \\

% % F-VLM~\cite{kuoopen2023fvlm}    & RN50x4 & RN50x4 & 1024 & - & - & 26.3 & 28.5 & 0.72\\ 


% % \cmidrule(lr){1-9} 

% \bottomrule
% \end{tabular}%
% }
% % \vspace*{-0.5em}
% \end{table*}



% \begin{table}[t!]
% \centering
% \caption{Main Results on OV-COCO}
% \vspace*{-0.2cm}
% \label{tab:main_coco}
% \resizebox{0.9\linewidth}{!}{%
% \begin{tabular}{@{}lrrr@{}}
% \toprule 
% % & & & \multicolumn{2}{c}{OV-LVIS} & \multicolumn{2}{c}{OV-COCO} \\
% % \cmidrule(lr){4-5} \cmidrule(lr){6-7}  
% {Methods}& $\mathrm{mAP^{novel}_{50}}$ & $\mathrm{mAP^{base}_{50}}$ & $\mathrm{mAP_{50}}$ \\ 
% \midrule
% % & %(memory update / robust learning) &
% \multicolumn{1}{r}{\footnotesize {\sffamily DETR-based}} \\
% % OWL-ViT & ViT-L/14 & ViT-L/14 & 25.6 & 34.7 & - & - \\ 
% OV-DETR~\cite{zang2022open} & 29.4 & 61.0 & 52.7 \\
% \textbf{Prompt-OVD} & \textbf{30.6}& \textbf{63.5} &\textbf{54.9} \\
% \cmidrule(lr){1-4}

% \multicolumn{1}{r}{\footnotesize {\sffamily RCNN-based}} \\
% % ViLD-text & 10.1 & 24.9 & 5.9 & 49.3 \\
% Detic~\cite{zhou2022detic}      &  27.8 & 51.0 & 45.0 \\
% ViLD~\cite{guopen2022vild}      &  27.6 & 59.6 & 51.3 \\
% F-VLM~\cite{kuoopen2023fvlm}     &  28.0 & 43.7 & 39.6 \\

% \bottomrule
% \end{tabular}%
% }
% \end{table}
\begin{figure*}[t]
\begin{center}
\includegraphics[width=16.7cm]{figures/ablation_study.pdf}
\end{center}
\vspace*{-0.5cm}
\begin{subfigure}{0.3\textwidth}
\label{fig:gt}
\caption{GT}
\end{subfigure}
\begin{subfigure}{0.15\textwidth}
\label{fig:rpn}
\caption{RPN}
\end{subfigure}
\begin{subfigure}{0.31\textwidth}
\caption{No pruning ($\epsilon = 0$)}
\label{fig:noprune}
\end{subfigure}
\begin{subfigure}{0.16\textwidth}
\caption{Pruning ($\epsilon =0.3$)}
\label{fig:prune}
\end{subfigure}
\vspace*{-0.4em}
\caption{Box predictions: (a) ground-truth boxes, (b) boxes estimated by RPN from DetPro~\cite{du2022detpro}, (c)--(d) boxes estimated by \algname{} without and with RoI pruning. Due to numerous predicted boxes in (b) and (c), we limit the number of boxes to 40 for better visualization. Red and green boxes are the ground-truth of novel and base classes, while blue and white ones represent predicted boxes that are either in close proximity or not in close proximity to the ground-truth, respectively.}
\label{fig:pruning_abl}
\vspace*{-0.4cm}
\end{figure*}


\subsection{Main Ablation Study}
\label{sec:abl_study}

\subsubsection{Inference Speed Up from OV-DETR} 
\label{sec:inference_speed} 
Table \ref{table:modification} summarizes the change in inference speed when replacing each design component of OV-DETR with our proposed ones on OV-LVIS. (1) Despite having more parameters, using ViT-B/16 incurs little additional latency, as it rather reduces the computational burden for the multi-scale deformable attention of the DETR encoder. (2) The use of local attention and the replacement of the DETR encoder with a simple feature pyramid network, as suggested by \cite{li2022exploring, song2022extendable}, result in a meaningful reduction in latency. (3) The primary speedup comes from replacing the decoding of OV-DETR with prompt-based decoding. (4) The ensemble with ViT-based CLIP only adds very little latency thanks to our efficient RoI-based masked attention. (6) RoI pruning also significantly contributes to speeding up by reducing the number of RoI candidates for detection and segmentation, without sacrificing detection accuracy. Overall, \algname{} speeds up inference by $21.2$ times over OV-DETR.

\begin{table}[t!]
\centering

\caption{Performance between RoI Align and RoI-based Masked Attention (RMA) on OV-LVIS.}% We set $\epsilon$ as 0.3 while pruning.} % If RoI pruning is employed, $\epsilon$ is assigned a value of 0.3; otherwise, it should be set to 0.0.
\vspace*{-0.3cm}
\label{tab:roi_pool}
\resizebox{1.0\linewidth}{!}{%
\begin{tabular}{@{}lrrrrrr@{}}
\toprule 
{RoI Proc.}& $\mathrm{mAP^{box}_{novel}}$ & $\mathrm{mAP^{box}}$ & $\mathrm{mAP^{mask}_{novel}}$ & $\mathrm{mAP^{mask}}$ & Latency\\ 
\midrule
% & %(memory update / robust learning) &
% \multicolumn{1}{r}{\footnotesize {\sffamily RCNN-based}} \\
% ViLD-text & 10.1 & 24.9 & 5.9 & 49.3 \\
Naive & 12.8 & 28.3 & 9.9 & 20.3 & 13.51 \\
\hspace{3mm}+Pruning & 13.7 & 28.3 & 10.2 & 20.4 & 1.85\\
Align~\cite{he2017mask} & 24.2 & 31.7 & 19.1 & 23.0 & 3.06\\
\hspace{3mm}+Pruning & 26.2 & 32.0 & 20.8 & 23.4 & 0.60 \\
RMA (ours) & 26.6 & 32.5 & 21.0 & 23.8 & 3.03 \\
\hspace{3mm}+Pruning & \textbf{29.4} & \textbf{33.0} & \textbf{23.1} & \textbf{24.2} & 0.58 \\


\bottomrule
\end{tabular}%
}
\vspace*{-0.3cm}
\label{tab:rma_analysis}
\end{table}

%\vspace*{-0.18cm}
\subsubsection{Ensemble Coefficient}
%\label{sec:hyperparams}
We investigate the influence of the ensembling weights $\alpha$ and $\beta$ for base and novel classes. We vary the value of each hyperparameter while keeping the other constant, as summarized in Tables \ref{tab:alpha} and \ref{tab:beta}, with a fixed RoI pruning threshold $\epsilon$ of 0.3. In general, the overall performance increases and then reaches at their maximum values when $\alpha=0.2$ and $\beta=0.4$. Using extreme values 0.0 or 1.0 for either coefficient results in significantly worse performance than using a more balanced ensemble. Moreover, the results show that the ensemble is more effective for novel classes than base classes, as evidenced by the significant impact of $\beta$ on the results of $\mathrm{mAP_{novel}}$. This suggests that the knowledge from CLIP has a greater positive impact on novel classes than on base classes. We set the values of $\alpha$ and $\beta$ to 0.2 and 0.4 for all experiments.
%

%Also, for getting optimal probability weights $\alpha$ and $\beta$, we report the performance change as $\alpha$ and $\beta$ increase in Table~\ref{tab:alpha} and ~\ref{tab:beta}, respectively. Table~\ref{tab:alpha} illustrates that while the zero-shot performance ($\mathrm{mAP_{novel}}$) increases monotonically, the overall performance ($\mathrm{mAP}$) increases and peaks at $\alpha=0.2$, then decreases. Interestingly, Table~\ref{tab:beta} demonstrates that both zero-shot and overall performances increase and reach a maximum value when $\beta=0.4$. Therefore, we anticipate that the knowledge from CLIP have a greater positive impact on new classes than the base classes, as the optimal value of $\beta$ is larger than that of $\alpha$.
%Note that we set the values of $\alpha$ and $\beta$ to 0.2 and 0.4, respectively, for all the experiments.



\subsubsection{RoI-based Masked Attention} 
\label{sec:masked_attention}
We conduct a comparison between our RoI-based masked attention method with both the naive approach and the commonly used RoI Align method, as summarized in Table \ref{tab:rma_analysis}. The naive approach implies that CLIP infers all the cropped images of RoIs according to Eq.~\eqref{eq:iter_infer}. To apply the RoI Align method to CLIP's ViT encoder, we reconstruct its patch tokens into a 2D feature map prior to the final Transformer layer. Compared to the naive approach, our RoI-based masked attention method has a significantly smaller computational overhead. Additionally, the efficiency and effectiveness of our method can be further improved by utilizing RoI pruning, which removes background RoIs. In contrast, RoI Align shows substantially lower $\mathrm{mAP^{box}_{novel}}$ and $\mathrm{mAP^{box}}$ than our method, as it is not optimized for the Transformer structure. Surprisingly, the naive crop method did not perform well, likely due to resizing a small object to be too large. Therefore, using masked attention is a more appropriate approach for Transformers than others. 

%To validate our proposed RoI-based masked attention, we compare it with RoI Align, which is utilized by Mask-RCNN in Table~\ref{tab:roi_pool}. Masked attention overcomes RoI Align for all the metrics with a large margin. Moreover, we analyze the optimal usage of the attention layer number in Table~\ref{tab:blk_num}, and using the last layer has the best results compared to the others for all the metrics. Through those emperical studies, therefore, we select RoI-based masked attention with the last attention layer instead of RoI align. 
\label{sec:eval_RMA}
\begin{table}[t!]
\centering
\caption{Performance trade-off with varying $\epsilon$ on OV-LVIS.}
\vspace*{-0.2cm}
\label{tab:roi_thres}
\resizebox{1.0\linewidth}{!}{%
\begin{tabular}{@{}lrrrrr@{}}
\toprule 
{$\epsilon$}& $\mathrm{mAP^{box}_{novel}}$ & $\mathrm{mAP^{box}}$ & $\mathrm{mAP^{mask}_{novel}}$ & $\mathrm{mAP^{mask}}$ & Latency (s)\\ 
\midrule
% & %(memory update / robust learning) &
% \multicolumn{1}{r}{\footnotesize {\sffamily RCNN-based}} \\
% ViLD-text & 10.1 & 24.9 & 5.9 & 49.3 \\
0.0 & 26.6 & 32.5 & 21.0 & 23.8 & 3.03\\
0.1 & 27.7 & 32.8 & 21.9 & 24.0 & 1.38\\
0.2 & 28.3 & 33.0 & 22.3 & 24.1 & 0.86\\
% \rowcolor{LightCyan}
\textbf{0.3} & \textbf{29.4} & \textbf{33.0} & \textbf{23.1} & \textbf{24.2} & 0.58\\
0.4 & 28.3 & 31.9 & 22.5 & 23.4 & 0.43\\
0.5 & 25.3 & 28.1 & 19.8 & 20.8 & 0.37\\
\bottomrule
\end{tabular}%
}
\vspace*{-0.2cm}
\end{table}



\subsubsection{Box Regression over RPN}
%\label{sec:eval_prune}
% tab2에서 masked attention과 align과의 비교. 
% block number 마지막으로 선택한 이유. 
%\vspace*{-0.1cm}
We validate the effectiveness of \algname{} in terms of box regression compared with the existing RPN-based method. Figure \ref{fig:pruning_abl} compares their estimated bounding boxes based on the ground-truth ones. %Specifically, Figure \ref{fig:pruning_abl}(a) shows the base and novel objects with their ground-truth bounding boxes. 
Figure \ref{fig:pruning_abl}(b) is an example of the scenario where the RPN method fails to accurately localize objects, i.e., a  missing box for carriage (novel class) and two deviated boxes from the ground-truth for the bread (novel class) and plate (base class). In contrast, \algname{} successfully localizes all base and novel objects with high recall using the prompt-guided decoding, as shown in Figure \ref{fig:pruning_abl}(c). Despite the presence of background or inaccurate boxes in the box candidates, \algname{} successfully covers all the ground-truth boxes with its predictions. Furthermore, as seen in Figure \ref{fig:pruning_abl}(d), RoI pruning effectively excludes such irrelevant boxes from the detection process.


%In order to validate the effectiveness of RoI pruning, we compare the proposals before and after pruning in Fig.~\ref{fig:pruning_abl}. Prior to pruning, there are numerous false positive boxes that did not match the ground truth, and those might negatively affect the performance due to score fusion with CLIP. However, pruning significantly reduces the number of non-matching boxes. In addition, since object detection consumes their time for handling lots of proposals, RoI pruning improves both performance and inference speed. We believe that RoI pruning narrow the gap in inference time between our method and the baseline despite using a larger backbone. 
%\vspace*{-0.3cm}
\subsubsection{RoI Pruning Threshold}
\label{sec:pruning}
%\vspace*{-0.1cm}
We investigate the trade-off between detection performance and computational efficiency by varying the pruning threshold $\epsilon$, as summarized in Table \ref{tab:roi_thres}. As the threshold increases, fewer bounding boxes are retained, as the number of boxes with object scores greater than the threshold decreases. For instance, when the threshold is set to be 0.0, which means RoI pruning is not applied, the detection accuracy deteriorates due to the inclusion of background boxes, also resulting in a high latency. On the contrary, the detection accuracy improves as $\epsilon$ increases within the reasonable range of 0.0~--~0.3, but deteriorates with a larger threshold of 0.4~--~0.5. That is, as $\epsilon$ increases, more false positive boxes begins to be excluded, leading to improved performance. However, further increasing the threshold leads to the removal of true positive boxes, causing performance degradation. Therefore, we set the value of $\epsilon$ to 0.3 for all experiments.

%, we analyze the trend of performance and inference time as $\epsilon$ increases and report the results in Table~\ref{tab:roi_thres}. It shows that latency decreases as $\epsilon$ increases, since the number of boxes of which score is bigger than $\epsilon$ decreases. Especially, the inference time is drastically higher when not using RoI prunning ($\epsilon=0.0$). On the contrary, the performance increases as $\epsilon \in [0.0, 0.3]$, and decreases as $\epsilon \in [0.3, 0.5]$. We expect that, as $\epsilon$ increases, it drops more false positive boxes to improve the performance at $\epsilon \in [0.0, 0.3]$, then it causes degradation of performance by dropping true positive boxes at $\epsilon \in [0.3, 0.5]$. Hence, we select $\epsilon$ as 0.3, which has the best performance with valid inference time. 


\begin{table}[t!]
\centering
\caption{Performance when using different CLIP models for ensembling on OV-LVIS.}
\vspace*{-0.25cm}
\label{tab:clip_arch}
\resizebox{1.0\linewidth}{!}{%
\begin{tabular}{@{}lrrrrr@{}}
\toprule 
{CLIP}& $\mathrm{mAP^{box}_{novel}}$ & $\mathrm{mAP^{box}}$ & $\mathrm{mAP^{mask}_{novel}}$ & $\mathrm{mAP^{mask}}$ & Latency (s)\\ 
\midrule
% & %(memory update / robust learning) &
% \multicolumn{1}{r}{\footnotesize {\sffamily RCNN-based}} \\
% ViLD-text & 10.1 & 24.9 & 5.9 & 49.3 \\
None     & 15.4 & 28.1 & 11.7 & 20.3 & 0.54 \\
ViT-B/32 & 20.5 & 30.1 & 16.4 & 21.9 & 0.56 \\
ViT-B/16 & 22.3 & 31.1 & 17.5 & 22.6 & 0.57\\
\textbf{ViT-L/14} & \textbf{29.4} & \textbf{33.0} & \textbf{23.1} & \textbf{24.2} & 0.58 \\


\bottomrule
\end{tabular}%
}
\label{tab:clip_size}
\vspace{-1.2em}
\end{table}


\smallskip\smallskip
\subsection{Additional Design Choice}
%\vspace*{-0.1cm}
\noindent We explore two supplementary design choices to utilize the ViT-based CLIP model in a manner that achieves the best balance between OVD performance and inference speed. %Following this, we further examine the impact of utilizing different initial weights for the backbone. %on the overall performance of \algname{}.
We provide more supplementary analysis on applying RoI-based masked attention to different attention layers and using different pre-trained ViT backbones with \algname{} in Appendix D.

%\vspace*{-0.3cm}
\subsubsection{CLIP Model Size} 
%\vspace*{-0.1cm}
% We conduct experiments to examine the impact of classification ensemble using CLIP with respect to its model size. 
Table \ref{tab:clip_size} summarizes the performance obtained after the ensemble with three different sizes of ViT-based CLIP models, including the scenario where CLIP is not used at all. We observe that without using the ensemble technique, the zero-shot performance is very poor compared to when CLIP is employed. However, as the size of the CLIP model increases, both zero-shot and overall performance improve, albeit with slightly higher inference speed. The difference in inference time is negligible due to our proposed efficient techniques; RoI-based masked attention and RoI pruning. Therefore, we conclude that the benefits obtained from using a larger CLIP encoder with ViT-L/14 outweigh the minimal increase in inference time.

%Since we expect that the zero-shot results ($\mathrm{mAP_{novel}}$) could be influenced by the utilization of the CLIP model for ensembling, we report the performance on OV-LVIS in Table~\ref{tab:clip_arch} as the CLIP model is varied. Specifically, the zero-shot performance is noticeably worse when not utilizing CLIP for ensembling compared to when CLIP is used. As the size of the CLIP model increases, both zero-shot and overall performance increase, and the inference time also increases slightly. However, the difference in inference time is negligible in comparison to the performance benefits gained. As such, we can conclude that boxes from DETR architecture contains not only base classes but also novel classes, and CLIP can more contribute the classification for unseen classes as the size increases without increasing latency much.

%\vspace*{-0.3cm}
\subsubsection{CLIP Input Resolution} 
%$\vspace*{-0.1cm}
Another design consideration is the resolution of input to the ViT-based CLIP. To evaluate the performance trade-off between detection accuracy and latency of CLIP, we vary the input image size and report the results in Table \ref{tab:image_size}. Similar to the model size, the latency change by input resolution is negligible thanks to our efficient methodological design. This indicates that we can use a variety of image resolutions without sacrificing the latency of \algname{}. However, the best image resolution for the ViT-L/14 encoder is $336 \times 336$, which is the original input size used to train the CLIP model, while the detection accuracy for base and novel classes drops with a larger $672\times672$. The $336 \times 336$  provides a good balance between detection accuracy and inference time, so it is the recommended input resolution. %for the proposed \algname{} framework. 

%Here, we evaluate the performance trade-off between detection accuracy and latency with of CLIP by varying the input image size, and report the results in Table~\ref{tab:clip_img_res}. Although there is a significant difference in performance, the latency is negligible when using image resolutions of 168 and 336. However, when compared to image resolutions of 336 and 672, the performance difference is minimal, but there is a significant gap in inference time. As a result, we choose an image resolution of 336 when feeding images into the CLIP model. 



%e analyze the performance is affected by using different initial weights for the backbone, specifically ImageNet and MAE. Table~\ref{tab:pretrained_model} displays the performance results when training the backbone using these two pretrained models. As a result, starting from the MAE pretrained model yields better performance than starting from the ImageNet pretrained model. Therefore, we use the trained model that starts from the MAE pretrained model for all experiments.







\begin{table}[t!]
\centering
\caption{Performance when using different input image resolution for ViT-based CLIP on OV-LVIS.}
\vspace*{-0.25cm}
\label{tab:clip_img_res}
\resizebox{1.0\linewidth}{!}{%
\begin{tabular}{@{}lrrrrr@{}}
\toprule 
{img. res.}& $\mathrm{mAP^{box}_{novel}}$ & $\mathrm{mAP^{box}}$ & $\mathrm{mAP^{mask}_{novel}}$ & $\mathrm{mAP^{mask}}$ & Latency \\ 
\midrule
% & %(memory update / robust learning) &
% \multicolumn{1}{r}{\footnotesize {\sffamily RCNN-based}} \\
% ViLD-text & 10.1 & 24.9 & 5.9 & 49.3 \\

168$\times$168 & 28.5 & 30.1 & 20.9 & 22.0 & 0.57 \\ 
\textbf{336$\times$336} & \textbf{29.4} & \textbf{33.0} & \textbf{23.1} & \textbf{24.2} & 0.58 \\ 
672$\times$672 & 29.2 & 33.0 & 23.0 & 24.2 & 0.83 \\


\bottomrule
\end{tabular}%
}
\vspace*{-0.35cm}
\label{tab:image_size}
\end{table}


% \begin{table}[t!]
% \centering
% \caption{Latency change as modifying OV-DETR to Prompt-OVD on OV-LVIS.}
% \label{tab:inference_study}
% \resizebox{0.7\linewidth}{!}{%
% \begin{tabular}{@{}llr@{}}
% \toprule 
% & {Modification}& Latency (s)\\ 
% \midrule
%  & OV-DETR & 12.28\\
% % \cmidrule{1-3}
% (1) & ResNet $\xrightarrow{}$ ViT & \\ 
% (2) & ViT $\xrightarrow{}$ ViTDET & \\ 
% (3) & Encoder $\xrightarrow{}$ FPN & \\
% (4) & prompt decode & \\
% (5) & + Ensemble with CLIP & \\


% \bottomrule
% \end{tabular}%
% }

% \end{table}


%%%%%%%%%%%%%%%%%%%%%%%%%%%%%%%%%%%%%%%%%%%%%%%%%

\section{Conclusion} 
We introduce \algname{}, a novel object detection method that achieves highly efficient inference speed while also improving zero-shot generalization compared with existing methods. The prompt-based decoding approach reduces the computational burden of object queries. The RoI-based masked attention and RoI pruning techniques allow us to efficiently leverage a large ViT-based CLIP model, enhancing detection performance through classification prediction ensembling. Comprehensive experiments show that \algname{} is $21.2$ times faster than OV-DETR while achieving comparable or higher APs on base and novel classes compared to two-stage OVD methods. %We believe that our work will inspire future work to explore the benefits of using Transformers.


%\paragraph{Ethics Statement.} 
%The focus of this paper is on open-vocabulary object detection. Our approach involves the integration of Transformer-based object detector and CLIP. We have not identified any foreseeable negative social impact associated with our work to share our findings with the scientific community. Nonetheless, we will continue to monitor and consider any potential concerns that may arise. 
\clearpage
%%%%%%%%% REFERENCES
{\small
\bibliographystyle{ieee_fullname}
\bibliography{egbib}
}

\clearpage
\begin{center}
    \Large \textbf{Supplementary Material}
\end{center}

\section{Effect of Hyper-parameter $\alpha$, $\beta$}
\noindent Tab.~\ref{tab:abaltion_alpha_beta} presents the ablation study on the use of different hyper-parameter $\alpha, \beta$ ratios in Eq. (2) during the pretraining of \ours. The results demonstrate that the setting of $\alpha : \beta = 1: 2$ yields the best performance, which is consistent with the recommended setting in the CoCa~\cite{yu2022coca} paper.


\begin{table}[!htp]
    \centering
    \footnotesize
    \begin{tabular}{lcccc}
    \toprule
        $\alpha : \beta$ &SRCC &PLCC \\
    \midrule
        $2:1$                            &0.645 &0.651 \\
        $1:1$                            &0.656 &0.661 \\
        $1:2$                             &\textbf{0.657} &\textbf{0.663} \\
    \bottomrule
    \end{tabular}
    % \vspace{-2.5mm}
    \caption{Effect of $\alpha$ and $\beta$ ratios in aesthetic pretraining, comparing zero-shot IAA performance with an ensemble of prompts on the AVA dataset.}
    \label{tab:abaltion_alpha_beta}
    % \vspace{-6mm}
\end{table}


\section{Effect of Margin Hyper-parameter $m$}
\noindent Tab.~\ref{tab:abaltion_margin} presents an ablation study for using different margin hyper-parameter $m$ in Eq. (7) when finetuning \ours-R. The results show that a margin of $m=0.1$ achieves the best performance, and we adopt this value as the default for all other experiments.


\begin{table}[!htp]
    \centering
    \footnotesize
    \begin{tabular}{lcccc}
    \toprule
        Margin &SRCC &PLCC \\
    \midrule
        $m=0.01$                            &0.769 &0.769 \\
        $m=0.05$                            &0.770 &0.770 \\
        $m=0.1$                             &\textbf{0.774} &\textbf{0.774} \\   
        $m=0.15$                            &0.772 &0.771 \\
        $m=0.2$                             &0.772 &0.770 \\
    \bottomrule
    \end{tabular}
    % \vspace{-1mm}
    \caption{Ablation for different margin hyper-parameter $m$  for the proposed rank-based adapter tuning on AVA dataset.}
    \label{tab:abaltion_margin}
    % \vspace{-4mm}
\end{table}

\section{Effect of Random Sampling Comments}
\noindent During training, we create image-comment pairs by randomly selecting one comment from the available list of comments for an image, if there are multiple comments associated with the same image. Tab.~\ref{tab:abaltion_sampling_comment} shows the effect of such random sampling during aesthetic pretraining. When a fixed comment is used for each image, the AVA ZSL performance drops from 0.663 PLCC to 0.596. Random sampling is an effective approach since different comments may cover different aesthetic aspects of the same image, allowing the model to fully expose itself to diverse and rich aesthetic information in the noisy dataset. This strategy enables the mining of open-set aesthetic concepts automatically.

\begin{table}[!htp]
    \centering
    \footnotesize
    \begin{tabular}{ccccc}
    \toprule
        Comment Sampling &SRCC &PLCC \\
    \midrule
        Random                             &\textbf{0.657} &\textbf{0.663} \\
        Fixed                             &0.585 &0.596 \\
    \bottomrule
    \end{tabular}
    \caption{Effect of random sampling comment in aesthetic pretraining, comparing zero-shot IAA performance with an ensemble of prompts on the AVA dataset.}
    \label{tab:abaltion_sampling_comment}
\end{table}


\section{Per-class Evaluation on AVA-Style}
\noindent We show the per-class evaluation on AVA-Style in Tab.~\ref{tab:ava_style_prediction_per_class}, comparing to the same baselines as in our main paper. 

\begin{table*}[ht!]
    \scriptsize
    \centering
    \setlength\tabcolsep{1.2pt} % default value: 6pt
    \begin{tabularx}{1.0\linewidth}{l*{14}{>{\centering\arraybackslash}X}|*{1}{>{\centering\arraybackslash}X}}
    \Xhline{2\arrayrulewidth}
        & Compl. Colors
        & Duo tones
        & HDR
        & Image Grain
        & Light On White
        & Long Expos.
        & Macro
        & Motion Blur
        & Negative Image
        & Rule of Thirds
        & Shallow DOF
        & Silhouet.
        & Soft Focus
        & Vanish. Point
        & mAP \\
    \Xhline{1\arrayrulewidth}
        Murray \etal [32]                 & -&-&-&-&-&-&-&-&-&-&-&-&-&-& \textcolor{lightgray}{53.9} \\
        Karayev \etal [18]                & \textcolor{lightgray}{46.9} & \textcolor{lightgray}{67.6} & \textcolor{lightgray}{66.9} & \textcolor{lightgray}{64.7} & \textcolor{lightgray}{90.8} & \textcolor{lightgray}{45.3} & \textcolor{lightgray}{47.8} & \textcolor{lightgray}{47.8} & \textcolor{lightgray}{59.5} & \textcolor{lightgray}{35.2} & \textcolor{lightgray}{62.4} & \textcolor{lightgray}{79.1} & \textcolor{lightgray}{31.2} & \textcolor{lightgray}{68.4} & \textcolor{lightgray}{58.1} \\
        Lu \etal [29]                     & -&-&-&-&-&-&-&-&-&-&-&-&-&-& \textcolor{lightgray}{64.1} \\
        MNet [42]                         & -&-&-&-&-&-&-&-&-&-&-&-&-&-& \textcolor{lightgray}{65.5} \\
        Sal-RGB [10]                      & \textcolor{lightgray}{61.4} & \textcolor{lightgray}{87.6} & \textcolor{lightgray}{72.9} & \textcolor{lightgray}{82.2} & \textcolor{lightgray}{83.0} & \textcolor{lightgray}{61.9} & \textcolor{lightgray}{66.6} & \textcolor{lightgray}{62.0} & \textcolor{lightgray}{87.7} & \textcolor{lightgray}{41.7} & \textcolor{lightgray}{82.4} & \textcolor{lightgray}{93.1} & \textcolor{lightgray}{46.4} & \textcolor{lightgray}{76.8} & \textcolor{lightgray}{71.8} \\
    \Xhline{1\arrayrulewidth}
        \textbf{Zero-shot Learning} \\
        General Pretraining (single prompt)   & 36.5 & 21.0 & 23.9 & 7.1 & 37.0 & 34.6 & 49.9 & 32.8 & 12.7 & 14.3 & 33.5 & 67.9 & 15.6 & 23.4 & 29.3 \\
        General Pretraining (ensemble prompts) & 36.0 & 51.3 & 36.9 & 8.1 & 30.5 & 40.4 & 55.0 & 33.4 & 13.8 & 14.5 & 27.0 & 64.9 & 18.0 & 27.3 & 32.6 \\
        \ours{}-P (single prompt)           & 48.1 & 55.8 & 76.6 & 76.0 & 72.9 & 66.1 & 70.8 & 67.6 & 34.9 & 25.8 & 77.9 & 81.6 & \textbf{51.2} & 67.3 & 62.3 \\
        \ours{}-P (ensemble prompt)         & \textbf{53.6} & \textbf{81.8} & \textbf{79.3} & \textbf{86.7} & \textbf{75.4} & \textbf{69.2} & \textbf{72.9} & \textbf{74.1} & \textbf{58.6} & \textbf{30.9} & \textbf{78.6} & \textbf{85.4} & 51.0 & \textbf{67.8} & \textbf{69.0}\\
    \Xhline{2\arrayrulewidth}
    \end{tabularx}
    \caption{AVA-Style per-class evaluation results. Supervised baselines are shown in \textcolor{lightgray}{gray} color.}
    \label{tab:ava_style_prediction_per_class}
\end{table*}


\section{Details on ZSL for AVA-Style Classification}


\begin{table*}[!htp]
    \centering
    \footnotesize
    \begin{tabular}{cc}
    \toprule
        Style &Prompts \\
    \midrule
        \multirow{5}{*}{Complementary\_Colors}
        & \textit{``complementary colors''} \\
        & \textit{``complementary color''} \\
        & \textit{``great complementary colors''} \\
        & \textit{``great use of complementary colors''} \\
        & \textit{``good use of complementary colors''} \\
    \midrule
        \multirow{5}{*}{Duotones}
        & \textit{``duo tones''} \\
        & \textit{``duotone''} \\
        & \textit{``nice duotone''} \\
        & \textit{``duotone works very well''} \\
        & \textit{``use of duotone''} \\
    \midrule
        \multirow{5}{*}{HDR}
        & \textit{``hdr''} \\
        & \textit{``i like the hdr''} \\
        & \textit{``great job with the hdr''} \\
        & \textit{``hdr done well''} \\
        & \textit{``love the hdr shot''} \\
    \midrule
        \multirow{5}{*}{Image\_Grain}
        & \textit{``image grain''} \\
        & \textit{``i like the image grain''} \\
        & \textit{``nice use of image grain''} \\
        & \textit{``a good job with the image grain''} \\
        & \textit{``excellent use of image grain''} \\
    \midrule
        \multirow{5}{*}{Light\_On\_White}
        & \textit{``light on white''} \\
        & \textit{``great for the light on white''} \\
        & \textit{``nice light on white''} \\
        & \textit{``love the light on white''} \\
        & \textit{``like the light on white''} \\
    \midrule
        \multirow{5}{*}{Long\_Exposure}
        & \textit{``long exposure''} \\
        & \textit{``nice long exposure''} \\
        & \textit{``nice use of long exposure''} \\
        & \textit{``enjoy these long exposure shots''} \\
        & \textit{``look great with the long exposure''} \\
    \midrule
        \multirow{5}{*}{Macro}
        & \textit{``macro''} \\
        & \textit{``excellent detailed macro''} \\
        & \textit{``nice macro''} \\
        & \textit{``good macro shot''} \\
        & \textit{``great macro''} \\
    \bottomrule
    \end{tabular}
    \begin{tabular}{cc}
    \toprule
        Style &Prompts \\
    \midrule
        \multirow{5}{*}{Motion\_Blur}
        & \textit{``motion blur''} \\
        & \textit{``nice motion blur''} \\
        & \textit{``great use of the motion blur''} \\
        & \textit{``i love the motion blur''} \\
        & \textit{``cool motion blur''} \\
    \midrule
        \multirow{5}{*}{Negative\_Image}
        & \textit{``negative image looks good''} \\
        & \textit{``love the negative images''} \\
        & \textit{``the negative image is captivating''} \\
        & \textit{``use of the negative image is interesting''} \\
        & \textit{``fan of the negative image''} \\
    \midrule
        \multirow{5}{*}{Rule\_of\_Thirds}
        & \textit{``rule of thirds''} \\
        & \textit{``benefited from the rule of thirds''} \\
        & \textit{``followed the rule of thirds nicely''} \\
        & \textit{``use of the rule of thirds is fantastic''} \\
        & \textit{``great use of rule of thirds''} \\
    \midrule
        \multirow{5}{*}{Shallow\_DOF}
        & \textit{``shallow dof''} \\
        & \textit{``nice shallow dof''} \\
        & \textit{``i love the shallow DOF''} \\
        & \textit{``lovely use of shallow DOF''} \\
        & \textit{``shallow dof works perfect here''} \\
    \midrule
        \multirow{5}{*}{Silhouettes}
        & \textit{``silhouettes''} \\
        & \textit{``like the silhouettes''} \\
        & \textit{``great silhouettes''} \\
        & \textit{``i really like silhouettes''} \\
        & \textit{``silhouettes are lovely''} \\
    \midrule
        \multirow{5}{*}{Soft\_Focus}
        & \textit{``soft focus''} \\
        & \textit{``love the soft focus''} \\
        & \textit{``love the effect of soft focus''} \\
        & \textit{``excellent use of soft focus''} \\
        & \textit{``lovely soft focus''} \\
    \midrule
        \multirow{5}{*}{Vanishing\_Point}
        & \textit{``vanishing point''} \\
        & \textit{``i like the lines and fading or vanishing''} \\
        & \textit{``i love the lines and vanishing point''} \\
        & \textit{``nice to see the vanishing point off of center''} \\
        & \textit{``the background with the vanishing point is nice''} \\
    \bottomrule
    \end{tabular}
    \vspace{-1mm}
    \caption{Text prompts used in the ensemble approach for AVA-Style ZSL.}
    \label{tab:style_prompts}
    \vspace{-4mm}
\end{table*}
\noindent\textbf{Single prompt.}
In this approach, we use the 14 photographic style names as the language prompts: \{``complementary colors'', ``duo tones'', ``hdr'', ``image grain'', ``light on white'', ``long exposure'', ``macro'', ``motion blur'', ``negative image'', ``rule of thirds'', ``shallow dof'', ``silhouettes'', ``soft focus'', ``vanishing point''\}. The cosine similarity between the prompt text embedding and the image embedding is used as the  prediction score.

\vspace{+3mm}
\noindent\textbf{Ensemble of prompts.} In this approach, we manually curate five sentences/phrases that are frequently mentioned in the AVA-Caption user comments, for each of the styles. These prompts either use synonyms (\emph{e.g.} ``color'' and ``colors'') of the styles or add more text contexts (\emph{e.g.}, ``i like the lines and fading or vanishing''). Tab.~\ref{tab:style_prompts} shows these prompts.

\section{Details on ZSL for IAA}
\noindent To effectively perform zero-shot learning for IAA, we use a pair of prompts with opposite meanings (``good" v.s. ``bad").

\vspace{+3mm}
\noindent\textbf{Single prompt.} In this approach, we use \{``good image", ``bad image"\} as input prompts. Let $\vect{p}_g$ and $\vect{p}_b$ be the normalized unimodal text embedding for the ``good" and ``bad" prompts respectively, $\vect{v}$ be the normalized image contrastive embedding. We compute the cosine similarity and use the softmax normalized score for ``good image" as the final score $r$ for IAA.
\begin{align*}
r = \frac{e^{\vect{v}^\top\vect{p}_g}}{e^{\vect{v}^\top\vect{p}_g} + e^{\vect{v}^\top\vect{p}_b}}
\end{align*}

\noindent\textbf{Ensemble of prompts.} In this approach, we similarly construct six pairs of ``good" v.s. ``bad" prompts for \{``image", ``lighting", ``content", ``background", ``foreground", ``composition"). The second group in Tab.~\ref{tab:iaa_prompts} shows these pairs of prompts. For each pair, we can obtain a score $r_i, i = 1,..., 6$. Then we use the average ensemble of the scores to get the final score $r$ for IAA.

\begin{table}[!htp]
    \centering
    \footnotesize
    \setlength\tabcolsep{1.2pt}
    \begin{tabular}{lccc}\toprule
    &\multicolumn{2}{c}{Prompts} \\\cmidrule{2-3}
        &$\vect{p}_g$ &$\vect{p}_b$ \\\midrule
    Single Prompt
        &\textit{``good image"} &\textit{``bad image"} \\\midrule
    \multirow{6}{*}{Ensemble of Prompts}
        &\textit{``good image"} &\textit{``bad image"} \\
        &\textit{``good lighting"} &\textit{``bad lighting"} \\
        &\textit{``good content"} &\textit{``bad content"} \\
        &\textit{``good background"} &\textit{``bad background"} \\
        &\textit{``good foreground"} &\textit{``bad foreground"} \\
        &\textit{``good composition"} &\textit{``bad composition"} \\
    \bottomrule
    \end{tabular}
    \vspace{-1mm}
    \caption{Text prompts used in ZSL for IAA.}
    \label{tab:iaa_prompts}
    \vspace{-1mm}
\end{table}



\section{Results on KonIQ-10k}

\begin{table}
\begin{center}
\footnotesize
\begin{tabular}{lccc}\toprule
Method &SRCC &PLCC \\\midrule
BRISQUE \cite{mittal2012no} &0.665 &0.681 \\
ILNIQE \cite{zhang2015feature} &0.507 &0.523 \\
HOSA \cite{xu2016blind} &0.671 &0.694 \\
BIECON \cite{kim2016fully} &0.618 &0.651 \\
WaDIQaM \cite{bosse2017deep} &0.797 &0.805 \\
PQR \cite{zeng2017probabilistic} &0.880 &0.884 \\
SFA \cite{li2018has} &0.856 &0.872 \\
DBCNN \cite{zhang2018blind} &0.875 &0.884 \\
MetaIQA \cite{zhu2020metaiqa}  &0.850 &0.887 \\
BIQA \cite{su2020blindly} &0.906 &0.917 \\
CLIP-IQA$^+$ \cite{wang2022exploring} &0.895 &0.909 \\ 
KonCept512 \cite{koniq10k} &0.921 &0.937 \\
% MUSIQ \cite{Ke_2021_ICCV} &\second{0.916} &\best{0.928} \\
MUSIQ \cite{Ke_2021_ICCV} &0.924 &0.937 \\\midrule
\ours{}-R &0.919 &0.932 \\
\bottomrule
\end{tabular}
\end{center}
\vspace{-2mm}
\caption{Results on KonIQ-10k~\cite{koniq10k} dataset.  We take numbers from \cite{wang2022exploring, Ke_2021_ICCV} for results of the reference methods.} \label{tab:koniq-results}
\vspace{-3mm}
\end{table}

\noindent Table~\ref{tab:koniq-results} presents additional results on the image quality dataset KonIQ-10k~\cite{koniq10k}. We adopt the same data split as~\cite{koniq10k} and and employ a batch size of 32 to finetune the rank-based adapter for 30k steps, with a learning rate of 5e-4 and linear decay to zero, and 0.04 weight decay. Our proposed VILA-R outperforms CLIP-IQA$^+$ \cite{wang2022exploring} which trains a prompt tuning module on top of CLIP features. While CLIP features only use general pretraining, VILA-R benefits from the aesthetic pretraining which learns rich perceptual quality information, highlighting the importance of the proposed aesthtic pretraining. Remarkably, with only 0.1\% tunable parameters, VILA-R's performance is competitive with KonCept512 \cite{koniq10k} and MUSIQ \cite{Ke_2021_ICCV}, which rely on much larger resolutions. It is worth noting that KonIQ-10k~\cite{koniq10k} is not solely focused on aesthetics quality, and it includes images with technical quality problems such as compression and blur. There is limited user comments mentioning such aspects on the AVA-Captions dataset. Despite the gap, our model demonstrates competitive performance on KonIQ-10k, showcasing its robustness in capturing the visual appeal of the image across different datasets.


\section{More Qualitative Examples}

\noindent Fig.~\ref{fig:ava_styles_all} displays additional style retrieval results (top-5) on KonIQ-10k~\cite{koniq10k} using AVA-style names as the query. In order to provide clear attribution to the image sources, we have opted to showcase images from the KonIQ-10k dataset instead of the AVA dataset. Attribution to the images are provided in  Table~\ref{tab:koniq-attribution}. Overall, the retrieved results align with our aesthetic perspective. Notably, \ours{} accuratly captures the lighting or color related information. For example, images retrieved for ``Silhouettes" and ``Complementary colors'' accurately depict the corresponding concepts. Additionally, \ours{} recognizes concepts aesthetic concepts like ``Motion blur'' with high accuracy. However, there are also some failure cases where improvements are possible. For example, among the images retrieved using the query ``Rule of thirds", the last three images are centered rather than following the rule of thirds, which may be attributed to the random cropping augmentation during training. Augmentation improvement may help mitigate this issue.  For ``Duo tones", the top retrieved images have a yellowish tone, possibly due to training data bias in the AVA-Captions dataset. Thus, using a more diverse aesthetic pretraining dataset may further enhance the model's performance.


\begin{figure}[!t]
    \centering
    \includegraphics[width=8cm]{Figures/ava-styles-supp.png}
    \caption{More examples for the top-5 images retrieved using style name query on KonIQ-10k~\cite{koniq10k}. The source of the displayed images are provided in Table~\ref{tab:koniq-attribution}.}
    \label{fig:ava_styles_all}
    \vspace{-2mm}
\end{figure}

\section{KonIQ-10k Images Attribution}
\noindent In this paper, we display several images from KonIQ-10k~\cite{koniq10k}. The Flickr links and the license information for these images can be found in Table~\ref{tab:koniq-attribution}. We extend our gratitude to the original photographers for sharing their images.


\begin{table}
\begin{center}
\tiny
\setlength\tabcolsep{1.2pt} % default value: 6pt
\begin{tabular}{l|c|c}\toprule
Flickr Link &User &License \\\midrule
\textbf{Figure 3 (from left to right, top to bottom)} & & \\
http://www.flickr.com/photos/43437767@N00/7499578096/ &43437767@N00 &CC BY-SA 2.0 \\
http://www.flickr.com/photos/12708857@N00/228617373/ &12708857@N00 &CC BY-SA 2.0 \\
http://www.flickr.com/photos/39443895202@N01/4295525241/ &39443895202@N01 &CC BY-NC 2.0 \\
http://www.flickr.com/photos/43343993@N00/6814873580/ &43343993@N00 &CC BY-NC-SA 2.0 \\
http://www.flickr.com/photos/93656595@N00/5212354067/ &93656595@N00 &CC BY-NC 2.0 \\
http://www.flickr.com/photos/19761391@N06/6190643783/ &19761391@N06 &CC BY-NC-SA 2.0 \\
http://www.flickr.com/photos/8490344@N04/5805954950/ &8490344@N04 &CC BY-NC-SA 2.0 \\
http://www.flickr.com/photos/28577026@N02/4732322374/ &28577026@N02 &CC BY 2.0 \\
http://www.flickr.com/photos/40595948@N00/4125943270/ &40595948@N00 &CC BY 2.0 \\
http://www.flickr.com/photos/8397802@N05/6782627736/ &8397802@N05 &CC BY-NC-SA 2.0 \\\midrule
\textbf{Figure 4 (from left to right, top to bottom)} & & \\
http://www.flickr.com/photos/28081633@N00/3388712525/ &28081633@N00 &CC BY-SA 2.0 \\
http://www.flickr.com/photos/69078621@N00/2501256504/ &69078621@N00 &CC BY-NC 2.0 \\
http://www.flickr.com/photos/21657526@N00/8460154333/ &21657526@N00 &CC BY-NC-SA 2.0 \\
http://www.flickr.com/photos/31990116@N03/8746457937/ &31990116@N03 &CC BY-SA 2.0 \\
http://www.flickr.com/photos/61585804@N00/4639050491/ &61585804@N00 &CC BY-NC-SA 2.0 \\
http://www.flickr.com/photos/78135748@N00/3824485636/ &78135748@N00 &CC BY 2.0 \\
http://www.flickr.com/photos/86381710@N00/163977327/ &86381710@N00 &CC BY-NC-SA 2.0 \\
http://www.flickr.com/photos/11152520@N03/5700224418/ &11152520@N03 &CC BY 2.0 \\
http://www.flickr.com/photos/77175355@N07/10862487886/ &77175355@N07 &CC BY-NC 2.0 \\
http://www.flickr.com/photos/76042652@N00/9467289840/ &76042652@N00 &CC BY-NC-SA 2.0 \\
http://www.flickr.com/photos/74167788@N00/2746983219/ &74167788@N00 &CC BY-NC 2.0 \\
http://www.flickr.com/photos/16409072@N08/7107776883/ &16409072@N08 &CC BY-NC-SA 2.0 \\
http://www.flickr.com/photos/13447407@N00/205662428/ &13447407@N00 &CC BY-NC 2.0 \\
http://www.flickr.com/photos/60635600@N08/6070993297/ &60635600@N08 &CC BY-NC 2.0 \\
http://www.flickr.com/photos/30626457@N00/9629668489/ &30626457@N00 &CC BY 2.0 \\
http://www.flickr.com/photos/31916492@N02/9702908989/ &31916492@N02 &CC BY-NC 2.0 \\
http://www.flickr.com/photos/24742305@N00/3561351919/ &24742305@N00 &CC BY 2.0 \\
http://www.flickr.com/photos/19072679@N00/8216243918/ &19072679@N00 &CC BY-NC-SA 2.0 \\
http://www.flickr.com/photos/50795598@N02/8532549490/ &50795598@N02 &CC BY-NC-SA 2.0 \\
http://www.flickr.com/photos/40573754@N04/4162338388/ &40573754@N04 &CC BY-NC-SA 2.0 \\\midrule
\textbf{Figure 6 (from left to right, top to bottom)} & & \\
http://www.flickr.com/photos/33602849@N00/1348094685/ &33602849@N00 &CC BY-NC-SA 2.0 \\
http://www.flickr.com/photos/32842313@N00/4435718106/ &32842313@N00 &CC BY-NC-SA 2.0 \\
http://www.flickr.com/photos/37306288@N02/6788530155/ &37306288@N02 &CC BY 2.0 \\
http://www.flickr.com/photos/32535586@N07/8120735257/ &32535586@N07 &CC BY-NC 2.0 \\
http://www.flickr.com/photos/62528187@N00/8596361580/ &62528187@N00 &CC BY 2.0 \\
http://www.flickr.com/photos/36543005@N00/242395156/ &36543005@N00 &CC BY 2.0 \\
http://www.flickr.com/photos/47100034@N08/8968242975/ &47100034@N08 &CC BY-NC-SA 2.0 \\
http://www.flickr.com/photos/8397802@N05/6856483689/ &8397802@N05 &CC BY-NC-SA 2.0 \\
http://www.flickr.com/photos/40355539@N00/4948021193/ &40355539@N00 &CC BY-NC-SA 2.0 \\
http://www.flickr.com/photos/10734170@N08/4049047636/ &10734170@N08 &CC BY-NC-SA 2.0 \\
http://www.flickr.com/photos/83670786@N03/8184338593/ &83670786@N03 &CC BY-NC-SA 2.0 \\
http://www.flickr.com/photos/24328811@N00/5091406011/ &24328811@N00 &CC BY-NC-SA 2.0 \\
http://www.flickr.com/photos/46318514@N06/4911877298/ &46318514@N06 &CC BY-NC-SA 2.0 \\
http://www.flickr.com/photos/16482030@N00/5593982816/ &16482030@N00 &CC BY-NC 2.0 \\
http://www.flickr.com/photos/68683191@N00/7915207374/ &68683191@N00 &CC BY-SA 2.0 \\
http://www.flickr.com/photos/51963363@N00/5921291430/ &51963363@N00 &CC BY-NC-SA 2.0 \\
http://www.flickr.com/photos/52713160@N00/6422846525/ &52713160@N00 &CC BY-NC-SA 2.0 \\
http://www.flickr.com/photos/28658116@N02/7914158128/ &28658116@N02 &CC BY-SA 2.0 \\
http://www.flickr.com/photos/10957255@N08/9365989820/ &10957255@N08 &CC BY-NC-SA 2.0 \\
http://www.flickr.com/photos/38439215@N06/4809080599/ &38439215@N06 &CC BY-NC-SA 2.0 \\
http://www.flickr.com/photos/92755733@N00/3017896883/ &92755733@N00 &CC BY 2.0 \\
http://www.flickr.com/photos/33455872@N05/7443887988/ &33455872@N05 &CC BY-NC-SA 2.0 \\
http://www.flickr.com/photos/8833673@N05/4462477090/ &8833673@N05 &CC BY-NC-SA 2.0 \\
http://www.flickr.com/photos/54852753@N05/5766806996/ &54852753@N05 &CC BY 2.0 \\
http://www.flickr.com/photos/90088957@N00/6434876963/ &90088957@N00 &CC BY-NC 2.0 \\
\bottomrule
\end{tabular}
\end{center}
\vspace{-2mm}
\caption{Flickr links to the KonIQ-10k~\cite{koniq10k} images shown in the paper. } \label{tab:koniq-attribution}
\vspace{-3mm}
\end{table}

\end{document}
