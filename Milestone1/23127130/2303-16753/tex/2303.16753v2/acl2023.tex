% This must be in the first 5 lines to tell arXiv to use pdfLaTeX, which is strongly recommended.
\pdfoutput=1
% In particular, the hyperref package requires pdfLaTeX in order to break URLs across lines.

\documentclass[11pt]{article}

% Remove the "review" option to generate the final version.
% \usepackage[review]{ACL2023}
\usepackage[]{ACL2023}


% Standard package includes
\usepackage{times}
\usepackage{latexsym}

% For proper rendering and hyphenation of words containing Latin characters (including in bib files)
\usepackage[T1]{fontenc}
% For Vietnamese characters
% \usepackage[T5]{fontenc}
% See https://www.latex-project.org/help/documentation/encguide.pdf for other character sets

% This assumes your files are encoded as UTF8
\usepackage[utf8]{inputenc}

% This is not strictly necessary and may be commented out.
% However, it will improve the layout of the manuscript,
% and will typically save some space.
\usepackage{microtype}

% This is also not strictly necessary, and may be commented out.
% However, it will improve the aesthetics of text in
% the typewriter font.
\usepackage{inconsolata}
% =========== custom package
% add by gzf
\usepackage{amssymb}
\usepackage{amsmath} 
\usepackage{booktabs}
\usepackage{enumerate}
\usepackage{graphicx}
\usepackage{subfigure}
\usepackage{xspace}
\usepackage{float}
\usepackage{bbm}
\usepackage{bm}
\usepackage{multirow}
\usepackage{booktabs}
\usepackage{color}
\usepackage{framed}
\usepackage{stfloats}
\usepackage{iitem}
% End by gzf
\usepackage{makecell}
\definecolor{shadecolor}{RGB}{180,180,180}
\usepackage{colortbl}
\usepackage{color, xcolor}
% \newtheorem{theorem}{Theorem}[section]
% \newtheorem{lemma}[theorem]{Lemma}
% \newtheorem{corollary}[theorem]{Corollary}
\newtheorem{definition}{Definition}[section]
\newtheorem{thm}{\bf Theorem}
\newtheorem{lem}{\bf Lemma}
\newtheorem{defi}{\bf Definition}
\newtheorem{cor}{\bf Corollary}
\newcommand{\paratitle}[1]{\vspace{1.5ex}\noindent\textbf{#1}}
\newcommand\Vector{\bm}
\newcommand\Matrix{\mathbf} 
\newcommand\Tensor{\mathcal}
\newcommand{\ie}{\emph{i.e.,}\xspace}
\newcommand{\aka}{\emph{i.e.,}\xspace}
\newcommand{\eg}{\emph{e.g.,}\xspace}
\newcommand{\etal}{\emph{et al.}\xspace}
\newcommand{\ignore}[1]{}
\newcommand{\tabincell}[2]{\begin{tabular}{@{}#1@{}}#2\end{tabular}}
% =========== custom package
%%%%%%%%% algorithm
\usepackage{algorithm}
\newtheorem{theorem}{Theorem}[section]
\newtheorem{corollary}[theorem]{Corollary}
\usepackage{algpseudocode}
% \usepackage{amsthm}
\usepackage{amsmath}
\usepackage{tikz}
\renewcommand{\algorithmicrequire}{\textbf{Require:}}
\renewcommand{\algorithmicensure}{\textbf{Output}}
% If the title and author information does not fit in the area allocated, uncomment the following
%
%\setlength\titlebox{<dim>}
%
% and set <dim> to something 5cm or larger.

% \title{Instructions for ACL 2023 Proceedings}

% Author information can be set in various styles:
% For several authors from the same institution:
% \author{Author 1 \and ... \and Author n \\
%         Address line \\ ... \\ Address line}
% if the names do not fit well on one line use
%         Author 1 \\ {\bf Author 2} \\ ... \\ {\bf Author n} \\
% For authors from different institutions:
% \author{Author 1 \\ Address line \\  ... \\ Address line
%         \And  ... \And
%         Author n \\ Address line \\ ... \\ Address line}
% To start a seperate ``row'' of authors use \AND, as in
% \author{Author 1 \\ Address line \\  ... \\ Address line
%         \AND
%         Author 2 \\ Address line \\ ... \\ Address line \And
%         Author 3 \\ Address line \\ ... \\ Address line}

% \author{First Author \\
%   Affiliation / Address line 1 \\
%   Affiliation / Address line 2 \\
%   Affiliation / Address line 3 \\
%   \texttt{email@domain} \\\And
%   Second Author \\
%   Affiliation / Address line 1 \\
%   Affiliation / Address line 2 \\
%   Affiliation / Address line 3 \\
%   \texttt{email@domain} \\}
% Scaling Transformers more deeper base on MPO
% \title{Scaling Transformers to Deeper with Fewer Parameters 
% \\ and Faster Convergence}
\title{Scaling Pre-trained Language Models to Deeper \\ via Parameter-efficient Architecture}

\author{
	Peiyu Liu$^{1,3}$\thanks{$\ $ Authors contributed equally.},
	Ze-Feng Gao$^{1*}$,
        Yushuo Chen$^{1,3}$,
	Wayne Xin Zhao$^{1,3}$\thanks{$\ $ Corresponding author.},\and
	\textbf{Ji-Rong Wen}$^{1,2,3}$
	\\
	$^1$Gaoling School of Artificial Intelligence, Renmin University of China\\
	$^2$ School of Information, Renmin University of China\\
	$^3$Beijing Key Laboratory of Big Data Management and Analysis Methods\\
	% $^4$Beijing Academy of Artificial Intelligence, Beijing, 100084, China\\
	{\tt\{liupeiyustu,zfgao,jrwen\}@ruc.edu.cn, }\\ 
	{\tt batmanfly@gmail.com,chenyushuo1999@foxmail.com}
}

\begin{document}
\maketitle

\begin{abstract}
In this paper, we propose a highly parameter-efficient approach to scaling pre-trained language models~(PLMs) to a deeper model depth. 
Unlike prior work that shares all parameters or uses extra blocks, we design a more capable parameter-sharing architecture based on  matrix product operator~(MPO). 
MPO decomposition can reorganize and factorize the information of a parameter matrix into two parts: the major part that contains the major information (\emph{central tensor}) and the supplementary part that only has a small proportion of parameters (\emph{auxiliary tensors}). Based on such a decomposition,  our architecture shares the central tensor 
across all layers for reducing the model size and meanwhile keeps layer-specific auxiliary tensors (also using adapters) for enhancing  the adaptation flexibility.   To improve the model training,  
we further propose a stable initialization algorithm tailored for the MPO-based architecture.
Extensive experiments have demonstrated the effectiveness of our proposed model in reducing the model size and achieving highly competitive performance. 

\end{abstract}

\section{Introduction}

Text-to-image (T2I) generative models have made great progress in the last few years thanks to algorithmic advances and the availability of large-scale paired training datasets
~\cite{ramesh2022hierarchical,yu2022scaling,rombach2022high,schuhmann2021laion,schuhmann2022laion}. Diffusion-based T2I generative models in particular 
have achieved remarkable results in terms of image quality
~\cite{ho2022classifier,nichol2021glide,balaji2022ediffi,saharia2022photorealistic,mou2023t2i,zhang2023adding}. 
Despite these strong results, controllable generation for these methods is still challenging: generated images are often not faithful to the captions, compositional capabilities are lacking, and prompt engineering is often required to achieve the desired results~\cite{dall-e-prompt-book}. Moreover, most large-scale models have only been trained on English text captions, greatly limiting their use across the world.

\begin{figure}[t]
\vspace{-2mm}
\includegraphics[width=1\linewidth]{   figs/fig_demo_results.pdf}
% \vspace{-3mm}
\caption{Setting of GlueGen. GlueNet is trying to provide an adaptable portal for the Stable Diffusion model to input multi-modal data, such as text, audio, \ie, (a) and (b), or text-audio hybrid signals, \ie (c), for X-to-image generation.}\label{fig:demo-resuls}
% \vspace{-4mm}
\end{figure}

Recent research has emphasized the crucial role of text encoders in improving Text-to-Image (T2I) models' performance, and their ability to comprehend and represent text is considered a bottleneck for image generation~\cite{saharia2022photorealistic, croitoru2022diffusion}. However, the current T2I models' text encoders are often trained on short image captions, which limits their performance on complex prompts and challenges their quality of feature extraction~\cite{rombach2022high}. Furthermore, T2I models' capacities are limited to generating images from text, and they cannot incorporate multimodal conditions such as sound and audio easily. Nevertheless, replacing the text encoder in existing T2I models is challenging since the text encoder and image generator's representation spaces are tightly coupled~\cite{rombach2022high, ramesh2022hierarchical}. This severe domain gap between the new conditions and the existing model impedes the image generation's final performance, and training the entire T2I model from scratch, with higher quality image-caption pairs, would be prohibitively expensive~\cite{edwards_2022}.\footnote{The cost of training a Stable Diffusion model is around 600K USD.}


As seen in Fig.~\ref{fig:demo-resuls}, we propose GlueNet to address the challenge of efficiently replacing or upgrading the text encoder in existing diffusion-based T2I models. With GlueNet, off-the-shelf pre-trained language models and multimodal encoders can be easily aligned with image encoders of T2I models, greatly enlarging their functionalities at a low cost. Importantly, this can be achieved without requiring retraining from scratch or even finetuning, maintaining the representation alignment between the text and image encoders. The proposed method follows an encoder-decoder structure. The encoder of GlueNet first aligns the representation space of the new condition encoder with that of the T2I model's image generator, minimizing both element-wise and global-wise discrepancy. Then, the decoder of GlueNet maps the aligned condition representations back to the original representation space of the new condition encoder by minimizing the reconstruction loss, preserving rich semantics captured by the pre-trained model during alignment training. Align existing models would inevitably decrease feature discriminability~\cite{pmlr-v97-chen19i,cui2020towards}, which makes the feature decoder necessary.
The entire training of GlueNet requires only a parallel corpus with the same content but different modalities or languages. At inference time, only the encoder of GlueNet is applied on top of the new condition encoder for representation alignment.



To verify the effectiveness of the proposed framework, we conducted three major experiments ranging from single- and multi-modal encoders. Firstly, we upgraded the existing text encoders of the Latent Diffusion Model~\cite{rombach2022high} using a stronger language model, T5-3B~\cite{raffel2020exploring}. Our model showed competitive improvements in FID score and user study ranking compared to the baselines but it still required finetuning for the overall performance boost. Secondly, we aligned a multilingual language model, XLM-Roberta-L~\cite{conneau2019unsupervised}, using our approach, enabling multilingual text-to-image generation. It achieved competitive results of translation-based models under a significantly lower training cost. Finally, we demonstrated GlueNet's capability to bring new functionalities beyond text signals into existing T2I models. 
The alignment of the AudioClip~\cite{guzhov2022audioclip} encoder enables sound-to-image generation without requiring any parameter finetuning of the image generator. This new capability allows the existing Stable Diffusion model to generate high-quality images that correspond to sound signals such as dogs barking and street music. This new capability goes beyond the traditional T2I generation and opens up new possibilities for creating multimedia content towards X-to-Image (X2I) generation.

\begin{figure*}[t]
\begin{center}
\includegraphics[height=0.3\linewidth,width=\linewidth]{   figs/framework_new_iccv.pdf}
\end{center}
\vspace{-4mm}
\caption{
Illustration of our desired GlueGen framework. With the proposed GlueNet model of the GlueGen framework, the pre-trained image generator (\ie UNet) can be bridged to off-the-shelf single- or multi-modal encoders to expand their functionalities, \ie, multilingual/sound-to-image generation, within a limited budget. GlueNet is trained offline and does not require back-propagation of UNet and image-text pairs for training. Therefore, GlueGen is flexible and efficient to achieve.
}
\vspace{-4mm}
\label{fig:setting}
\end{figure*}

Our contributions can be summarized as follows:

\begin{itemize}
\item To the best of our knowledge, this is the first work to consider the problem of efficiently aligning a pre-trained audio model with a pre-trained T2I diffusion model for sound-to-image generation.

\item  Extensive experiments on text-to-image generation benchmarks demonstrate the superiority of our model over the baseline LDM method on both image quality and language controllability.

\item Our framework also enables text-to-image generation beyond English prompts without the need of multilingual image-text pairs for retraining.

\end{itemize}

Visual encodings in multiclass scatterplots significantly affect people's ability to interpret categorical data correctly.
However, we still do not understand the perceptual impact of encoding choices across varying numbers of categories.
We briefly review the topics of graphical perception in scatterplots, color palette design, and tasks in scatterplots to ground our work.

\subsection{Graphical Perception in Scatterplots}

Understanding categorical perception is a fundamental task in both cognitive science~\cite{harnad2003categorical} and visualization~\cite{munzner2014visualization}. Past work has introduced a range of techniques for eliciting patterns in categorical data, such as Flexible Linked Axes~\cite{kosara2006parallel}, Parallel Sets~\cite{lex2010comparative}, and Matchmaker~\cite{claessen2011flexible}.
However, these techniques leverage specialized approaches with high learning costs, 
%and required knowledge base make it hard to employ these techniques 
making them difficult for lay audiences to work with.
Scatterplots, alternatively, are more familiar for many audiences and commonly encode categorical data~\cite{sarikaya2018scatterplots}. Consequently, 
understanding how to best design scatterplots for categorical datasets is essential for effective data communication.
%the effectiveness of visual design for scatterplots is crucial including evaluating the graphical perception.
%since scatterplot is still the most commonly used categorical visualization technique~\cite{sarikaya2018scatterplots}.
%We mainly review perceptual studies in scatterplots and a complete review of such studies in different visualizations is beyond the scope of this paper, we refer to Quadri and Rosen~\cite{quadri2021survey}.

Graphical perception studies investigate how effectively people can estimate different properties from visualized data (see Quadri \& Rosen~\cite{quadri2021survey} for a survey). 
%Graphical perception in scatterplot has been studied for decades since Cleveland and McGill~\cite{cleveland1984graphical, cleveland1986experiment} analyzed several low-level judgment tasks in bi-variate graphs.
Scatterplots are commonly used in graphical perception experiments as they are sufficiently complex to reflect real-world challenges and simultaneously sufficiently simple to control~\cite{rensink2014prospects, rensink2010perception, harrison2014ranking, kay2016beyond}.   
%Heer and Bostock~\cite{heer2010crowdsourcing} further confirmed that Weber’s Law can capture perceptual precision in graphical perception problems.
Existing studies have analyzed how scatterplots can support a variety of perceptual tasks across a range of channels. 
For example, Kim \& Heer use scatterplots as a means to assess how different visual channels support 
%a range of 
various tasks~\cite{kim2018assessing}.
Hong et al.~\cite{hong2021weighted} found that varying point size and lightness can lead to perceptual bias in mean judgments in scatterplots. Scatterplot studies commonly investigate how design influences people's abilities to estimate aggregate statistics, such as correlation~\cite{harrison2014ranking,rensink2010perception,kay2016beyond}, clustering~\cite{quadri2022automatic,sedlmair2012taxonomy,quadri2020modeling}, and means~\cite{hong2021weighted,gleicher2013perception,wei2019evaluating,whitlock2020graphical}. 
Other studies model the influence of different channels on scatterplot design, such as opacity~\cite{micallef2017towards}, color~\cite{szafir2018modeling}, and shape~\cite{burlinson2017open}. 

Most graphical perception studies focus on statistical relationships within a single category of scatterplots. However, studies of multiclass scatterplots often characterize people's abilities to separate classes by measuring just-noticeable differences in categorical encodings~\cite{smart2019measuring,burlinson2017open}. Alternatively, Gleicher et al.~\cite{gleicher2013perception} studied how different categorical encodings influenced people's abilities to compare the means of different classes 
%relative mean values of different classes were perceived in multiclass scatterplots. They conducted a crowd-sourced user study to compare the accuracy of mean value judgments 
with varying numbers of points and
%$hardness$ levels (denoted by $\Delta$ and computed by the pixel distance between classes' centers), 
differences in means, colors, and shapes.
%of scatterplots. 
%Their results revealed that hardness level and color are the most significant factors that impact human judgment, while the numbers of points are less influential.
%Based on the empirical results, they suggested the following guidelines: 1) scatterplots are capable to reveal the inter-class differences and viewers can extract that information, 2) conflicting cues do not hinder performance in the assessment of aggregates, and 3) since additional classes have little impact on performance, multi-class scatterplots should not be exempted in visualization design.
They found that scatterplots can effectively reveal interclass differences and that the design of a scatterplot influenced people's abilities to compare classes, with color being the strongest categorical cue. However, in contrast to other work on categorical visualization~\cite{haroz2012capacity}, they found that increasing the number of classes from two to three did not decrease performance.  
%Prior work in categorical visualization indicates that 
%However, they neither studied complex multiclass scatterplots with more than 3 classes nor took various color palettes into consideration.
%Since color palettes~\cite{zhou2015survey} and the number of classes~\cite{haroz2012capacity} both are important factors for human perception in visualizations, in this paper, we conduct a crowd-sourced experiment to evaluate various factors in multiclass scatterplots.
%Our studies focus on complex multiclass scatterplots from
We build on these observations to explore how robust people's estimates are in scatterplots with between 2 and 10 classes with varying hardness levels, color palettes, and numbers of points, (see Section \ref{sec-methodology}) to more deeply understand factors involved in effective multiclass scatterplot design. 
%aiming to provide a guideline with the human accuracy rate based on the number of categories, color palettes, and other variables such as hardness level (see \autoref{sec-discussion}).


\subsection{Color Palette Design}
Gleicher et al.'s findings about the effectiveness of color in multiclass scatterplots echo existing design guidance and results from other studies of categorical data encodings~\cite{gleicher2013perception,haroz2012capacity,trumbo1981theory,munzner2014visualization}.
%Color is one of the most common encoding channels in visualization. 
Choosing a proper categorical color palette\footnote{We define a color \emph{palette} as a set of colors specifically designed for categorical data.} for visualizing categorical data is a crucial task~\cite{trumbo1981theory, zhou2015survey}. Designers employ a combination of color models and heuristics to generate palettes (see Zhou \& Hansen~\cite{zhou2015survey}, Kovesi~\cite{kovesi2015good}, Bujack et al.~\cite{bujack2017good}, and Nardini et al.~\cite{nardini2019making} for surveys).
A range of studies has explicitly examined color perception for continuous data, such as characterizing limitations of rainbow colormaps \cite{ware1988color,reda2020rainbows,borland2007rainbow,quinan2019examining}, comparing the task-based effectiveness of continuous colormap designs~\cite{padilla2016evaluating,reda2018graphical,liu2018somewhere}, modeling color discrimination~\cite{ware2018measuring}, examining color semantics~\cite{anderson2021affective}, quantifying the impact of size and shape on encoding perception~\cite{smart2019measuring, szafir2018modeling} and examining perceptual biases~\cite{schloss2018mapping}.
However, significantly fewer studies have characterized color use for categorical data encoding.  

%Various aspects related to color perception have been studied empirically.
%Cleveland and McGill~\cite{cleveland1984graphical} found that compared to size and position, human perception of visual encoding of color channels would be at a lower precision in the most basic bi-variate graphs.
%For multiclass scatterplots, Gleicher et al.~\cite{gleicher2013perception} suggest that in the mean judgment task, color schema shows a stronger cue compared to other visual channels such as shape or orientation.
%Several existing studies focused on the co-effect between color and other visual encodings, such as size~\cite{szafir2018modeling}, shape~\cite{smart2019measuring} and uncertainty-relevant factors~\cite{maceachren2012visual}.
%For more details about color perception, we refer to recent surveys~\cite{zhou2015survey, kovesi2015good, nardini2019making}. 

Several principles and metrics of effective color palette design have been proposed~\cite{brewer1994guidelines,harrower2003colorbrewer,stone2006choosing,gramazio2016colorgorical}. 
%For example, Trumbo~\cite{trumbo1981theory} suggested that i) the order of colors should be comparable if presenting an ordered statistical variable and ii) the difference among colors should be obvious if presenting the differences of a variable.
Past work recommends that color palettes optimize the mapping between data semantics and color semantics~\cite{lin2013selecting,schloss2020semantic,setlur2016linguistic}; select colors that emphasize color harmonies~\cite{stone2006choosing,zeileis2009escaping}, affect~\cite{bartram2017affective}, or pair preference~\cite{schloss2011aesthetic}; and maximize perceptual and categorical separability between colors~\cite{healey1996choosing} (see Silva et al. \cite{silva2011using} for a survey).
%More recently, Zeileis et al.~\cite{zeileis2009escaping} recommended that colors should be attractive and harmonic with each other.
Designers can use predefined metrics to describe aesthetic 
%are some optimizing metrics proposed to measure aesthetic preference such as Pair Preference
(e.g., pair preference~\cite{schloss2011aesthetic}), perceptual (e.g., %or to measure color discrimination, such as Perceptual Distance (
CIEDE 2000~\cite{sharma2005ciede2000}), 
and categorical (e.g., color name difference or uniqueness~\cite{heer2012color})
%, and Name Uniqueness~\cite{heer2012color}.
attributes of color to implement these guidelines and constrain effective palette design. 
While these metrics underlie many palette design guidelines, implementing these guidelines effectively takes significant expertise. 
%We utilize the above-mentioned 3 color discrimination measures in our analysis.
%Additionally, the length or variance on paths of color spaces such as CIELCh~\cite{ihaka2003colour} can be used to model color distributions and differences in a palette.
%CIELCh is an equal space to CIELAB, but with perceptual path representations where $L*$ counts the \emph{lightness} of a color, $C*$ approximates its \emph{chroma}, and $h*$ measures the \emph{hue}.
%Compared to CIELAB ($\alpha*$ counts red-to-green ratio and $\beta*$ counts blue-to-yellow ratio) or RGB spaces, the perception-based CIELCh can better represent the property of colors~\cite{zeileis2009escaping}.
%As a consequence, we utilize the variance and length of the CIELCh paths to evaluate color palettes in our paper.

%Besides, Haroz and Whitney~\cite{haroz2012capacity} analyzed categorical binned images with different colors and found that the $subitizing$ phenomenon~\cite{kaufman1949discrimination,mandler1982subitizing}, can significantly impact the accuracy and responding time in categorical visualization.
%Their findings suggest that 5 may be the capacity limit of human attention in category numbers which could heavily impact the effectiveness of visualization design.

Several methods for creating effective color palettes have been introduced.
For example, Healey~\cite{healey1996choosing} considers linear separability, color difference, and color categorization to design discriminable color palettes.
Harrower and Brewer~\cite{harrower2003colorbrewer} introduced ColorBrewer for providing designer-crafted distinguishable color palettes for cartography.
Gramazio et al.~\cite{gramazio2016colorgorical} developed Colorgorical, which can generate categorical palettes by optimizing several perceptual and aesthetic metrics. 
%Lin et al. ~\cite{lin2013selecting} introduced categorical colors assignment with semantically-resonant colors by mapping values to representative images, and Schloss et al. ~\cite{schloss2020semantic} models semantic space distance on semantic discriminability. Bartram et al. ~\cite{bartram2017affective} and Anderson et al. ~\cite{9318559} proposed proper affective color properties selection can enhance visual communications. 
Recent efforts have also explored how palettes might be extracted from images~\cite{zheng2022image} or colors from a given palette optimally assigned to a visualization~\cite{lee2012perceptually, lin2013selecting,wang2018optimizing}. Tools such as Colorgorical~\cite{gramazio2016colorgorical} and ColorBrewer~\cite{harrower2003colorbrewer} enable people to generate or choose from a range of palette designs (see Zhou \& Hansen~\cite{zhou2015survey} for a survey).  
%Smart et al.~\cite{smart2019color} created effective color encodings based on a corpus of 222 expert-designed seed color ramps. 
%Wang et al.\cite{} combined the impact of spatial relation, density, cluster overlap, and background to assign colors.
In this study, we compare preconstructed palettes from a range of sources, 
%Since our studies do not contain comparisons of color assignment technologies, we choose pre-defined colors from existing designer-crafted palettes 
including ColorBrewer~\cite{harrower2003colorbrewer}, 
%Paul Tol~\cite{tol2012colour}, and Swiss Federal Statistical Office (SFSO)~\cite{sfso} but also from commercial visualization tools~\cite{4376133} like 
Tableau~\cite{tableau}, D3~\cite{6064996}, Stata Graphics~\cite{statagraphics19}, and Carto~\cite{carto} (see \autoref{fig:palettes} for the details of our selected color palettes). Following the model for comparing the effectiveness of continuous color ramps in Liu \& Heer \cite{liu2018somewhere}, we leverage these palettes to understand how effectively common best-practice color palettes encode data over a range of data parameters.  


%\subsection{Tasks in Scatterplot}
% Moved to 2 sentences in Sec. 4.1.

%Amar et al.~\cite{amar2005low} generalized a task-based taxonomy such as cluster, sort, and value retrieval that might impact analytics in information visualization.
%Since then, a number of perceptual studies have been proposed to conduct various low-level tasks to assess the effectiveness of scatterplots, such as assessing trend estimation in multivariate scatterplots~\cite{nguyen2016correlation}, modeling correlation perception with Weber's Law~\cite{harrison2014ranking}, evaluating outlier perception~\cite{sarikaya2018design}, and modeling cluster perception topologically~\cite{quadri2020modeling}.

%Among them, the relative mean judgment task, which is required to estimate the mean position of classes and compare its value in scatterplots, is commonly employed in perceptual experiments to evaluate the differences across multiple classes of points~\cite{sarikaya2018scatterplots}.
%For example, Gleicher et al.~\cite{gleicher2013perception} performed this task to evaluate perception accuracy for basic multiclass scatterplots.
%Karmer et al.~\cite{kramer2017visual} found by comparing the mean and variances of variables over time, people can capture trend information within data.
%Hong et al.~\cite{hong2021weighted} introduced perceptual biases in judging mean positions in scatterplots with varying colors and sizes of points.
%For multiclass scatterplots, the mean judgment task requires  making accurate judgments of mean localization and mean comparison, thus it combines the ability of both value retrieval and sort tasks, and enables to assess of the visual aggregation from human perception~\cite{gleicher2013perception, sarikaya2018scatterplots}.
%As a consequence, we perform the mean judgment task to assess human perception in multiclass scatterplots.

% \section{Preliminary}
In this paper, scalars, vectors and matrices are denoted by lowercase letters  (\eg $a$,  $\Vector{v}$ and  $\Matrix{M}$) respectively.
The high-order~(order three or higher) tensors are denoted by boldface Euler script letters (\eg $\Tensor{T}$). 
An $n$-order tensor $\Tensor{T}_{i_1,i_2,...i_n}$ can be considered as an~(potentially multidimensional) array with $n$ indices $\{ i_1,i_2,...,i_n \}$.


\subsection{Matrix Product Operators}
\label{subsec-matrix-product-operators}
We will give the procedure of the matrix product operator decomposition method.
\paragraph{Matrix Product Operator Decomposition}
Originating from quantum many-body physics~\cite{gao2020compressing}, the MPO decomposition can effectively reorganize and aggregate the information of the matrix. The parameter matrix is decomposed into a central tensor and a set of auxiliary tensors~\citep{liu2021enabling, gao2022parameter}.

To clarify the MPO decomposition process, we assume that a weight matrix $\Matrix{W}\in \mathbb{R}^{I\times J}$ is a matrix with size $I\times J$.
Given two arbitrary factorizations of its dimensions into natural numbers, we can reshape and transpose this matrix into an $n$-dimension tensor $\Matrix{W}_{i_1,\dots, i_n, j_1, \dots, j_n}$ in which:
\begin{equation}
\small
    \prod_{k=1}^{n} i_k = I, \quad \prod_{k=1}^{n}j_k = J.
\end{equation}
This decomposition can be written as:
\begin{equation}
    \small
   \Matrix{W}_{i_1,\dots , i_n, j_1, \dots, j_n} = \Tensor{T}^{(1)}[i_1, j_1]\cdots \Tensor{T}^{(n)}[i_n, j_n],
\label{eq:mpo-decom}
\end{equation}
where the $\Tensor{T}^{(k)}[i_k, j_k]$ is a 4-dimensional tensor with size $d_{k-1}\times i_k \times j_k \times d_k$ in which $d_k$ is a bond dimension linking $T^{(k)}$ and $T^{(k+1)}$ with $d_0=d_n=1$.
with size $d_{k-1}\times i_k \times j_k \times d_k$ in which $\prod_{k=1}^{n}i_k=I, \prod_{k=1}^{n}j_k=J$ and $d_0=d_n=1$.
% Following~\citep{liu-etal-2021-enabling}, the tensor right in the middle is the \emph{central tensor}, and the rest is the \emph{auxiliary tensor}.

According to ~\citep{gao2020compressing}, the original matrix $\Matrix{M}$ may be precisely reconstructed using tensor contraction without the connecting bond $\{d_k\}_{k=1}^m$ being truncated.
After MPO decomposition, the central tensor can encode the essential data from the original matrix, while the other auxiliary tensors serve as its complement.




\subsection{BERT}
\label{subsec-bert}
We will provide a brief review of the BERT~\citep{devlin2018bert} pre-training approach as well as some of the training options.
\paragraph{Architecture and Setup}
BERT uses the transformer architecture~\citep{vaswani2017attention}. 
We use an MPO-based transformer architecture with $L$ layers. 
Each block uses $A$ attention heads and hidden dimension $H$.
In the setup step, BERT uses the concatenation of two segments as input~(\ie $ x_1, \dots, x_N$ and $y_1, \dots, y_M$, $N$ and $M$ are the length of input and output respectively, which constrained such that $M+N<T$, where $T$ is a parameter that controls maximum sequence length during training).
Note that each segment usually consists of more than one natural sentence.
Special tokens are used to distinguish the two segments as a single input sequence to BERT:
$[CLS], x_1, \dots, x_N, [SEP], y_1,\dots, y_M,[EOS]$.
% 模型会先在几个无标签大数据集进行预训练,紧接着在下游任务上进行微调。
The models are first pre-trained on several large unlabeled datasets and subsequently fine-tuned on downstream tasks.
\paragraph{Training Objectives}
BERT uses two objectives during the pre-training stage, \ie masked language modeling~(MLM) and next sentence prediction~(NSP).

In the input sequence, MLM is to replace a random sample of tokens with special tokens $[MASK]$. Predicting the cross-entropy loss of the masked tokens is the objective of MLM. 15\% of the input tokens are consistently chosen for replacement by BERT. Out of the chosen tokens, 80\% are changed to $[MASK]$, 10\% are left alone, and 10\% are changed to randomly chosen lexical tokens.

In order to determine if two text fragments from the original work agree with one another, NSP, a binary classification loss, is utilized. Extracting consecutive sentences from a text corpus results in the creation of uplifting instances. Pairing pieces from other sources produce negative instances. The same probability is used to sample both positive and negative cases. The objective of the NSP is to enhance the performance of downstream tasks, such as natural language inference~\citep{bowman2015large}.

\paragraph{Datasets and Optimization}
Following~\citep{devlin2018bert}, BERT is trained on 16GB of uncompressed text from English Wikipedia and BOOKCORPUS~\citep{zhu2015aligning} combined.
BERT is optimized with Adam~\citep{kingma2014adam} and using GELU activation function~\citep{hendrycks2016gaussian}.
The learning rate is warmed up over the first 10,000 steps to a peak value of 1e-4, and then linearly decayed.
In the pre-training stage, models are trained for 1,000,000 updates with mini-batches containing $B=256$ sequences of maximum length $T=512$ tokens.

\section{Method}
%In this section, we describe our proposed MPOBERT for building the deep Transformer-based model with fewer parameters and enhanced training efficiency. With parameter sharing, it is possible to save a substantial number of parameters when scaling models along the depth. In addition, we propose a theoretically derived tensor initialization for training deep Transformers. Thus, we will explain MPOBERT in the two parts mentioned above.
In this section, we describe the proposed \emph{MPOBERT} approach for building deep PLMs via a highly parameter-efficient architecture.
Our approach follows the classic \emph{weight sharing} paradigm, while introducing a  principled mechanism for sharing informative parameters across layers and also enabling  layer-specific  weight adaptation. 
%the flexibility for capturing layer-wise variation. 


\subsection{Overview of Our Approach}
\label{sec-framework}
%The basic idea of MPOBERT is to share \emph{more informative} parameters among different Transformer layers to build  deep PLMs in a parameter-efficient way. 
Although weight sharing has been widely explored for building  compact PLMs~\cite{lan2019albert}, existing studies either  share  all the parameters across  layers~\cite{lan2019albert} or incorporate additional blocks to facilitate  the sharing~\cite{zhang2022minivit,nouriboriji@2022minialbert}. They  either have limited model capacity with a rigid architecture  or require additional efforts for maintenance.   

Considering the above issues, we motivate our approach in two aspects.  Firstly, only informative parameters should be shared across layers, instead of all the parameters.  Second, it should not affect the capacity to capture layer-specific  variations. 
To achieve this, we utilize the MPO decomposition from multi-body physics~\cite{gao2020compressing} to develop a parameter-efficient architecture by sharing informative components  across layers and keeping layer-specific supplementary components  (Section~\ref{subsec-crosslayer}). 
%we develop our approach based on the MPO decomposition from multi-body physics~\cite{gao2020compressing}. 
 %while also employed for parameter-efficient  PLMs~\cite{liu2021enabling, gao2022parameter}.  % To develop a more principled weight sharing mechanism, we utilize MPO decomposition to obtain shared parameters  across layers and layer-specific parameters (Section~\ref{subsec-crosslayer}). 
As another potential issue, it  is difficult to optimize  deep PLMs due to unstable training~\cite{wang@2022deepnet}, especially when weight sharing~\cite{lan2019albert} is involved. 
We further propose a simple yet effective method to stabilize the training of MPOBERT (Section~\ref{sec-efficient-training}).  Next, we introduce the technical details of our approach. 

  %Next, we introduce the 

%Transformer-based models
% We first introduce the procedure of the matrix product operator decomposition method and then detail the MPOBERT with two parts: scaling to deeper models with parameter sharing and training deep models with tensor initialization.

% \subsection{Matrix Product Operator Decomposition}
% \label{subsec-matrix-product-operators}
% We will briefly describe the matrix product operator decomposition method. Originating from quantum many-body physics~\cite{gao2020compressing}, the MPO decomposition can effectively reorganize and aggregate the information of the matrix. The parameter matrix is decomposed into a central tensor and a set of auxiliary tensors~\citep{liu2021enabling, gao2022parameter}.

% To clarify the MPO decomposition process, we assume that a weight matrix $\Matrix{W}\in \mathbb{R}^{I\times J}$ is a matrix with size $I\times J$.
% Given two arbitrary factorizations of its dimensions into natural numbers, we can reshape and transpose this matrix into an $n$-dimension tensor $\Matrix{W}_{i_1,\dots, i_n, j_1, \dots, j_n}$ in which:
% \begin{equation}
% \small
%     \prod_{k=1}^{n} i_k = I, \quad \prod_{k=1}^{n}j_k = J.
% \end{equation}
% This decomposition can be written as:
% \begin{equation}
%     \small
%    \Matrix{W}_{i_1,\dots , i_n, j_1, \dots, j_n} = \Tensor{T}^{(1)}[i_1, j_1]\cdots \Tensor{T}^{(n)}[i_n, j_n],
% \label{eq:mpo-decom}
% \end{equation}
% where the $\Tensor{T}^{(k)}[i_k, j_k]$ is a 4-dimensional tensor with size $d_{k-1}\times i_k \times j_k \times d_k$ in which $d_k$ is a bond dimension linking $T^{(k)}$ and $T^{(k+1)}$ with $d_0=d_n=1$.
% with size $d_{k-1}\times i_k \times j_k \times d_k$ in which $\prod_{k=1}^{n}i_k=I, \prod_{k=1}^{n}j_k=J$ and $d_0=d_n=1$.

% According to ~\citet{gao2020compressing}, the original matrix $\Matrix{M}$ may be precisely reconstructed using tensor contraction without the connecting bond $\{d_k\}_{k=1}^m$ being truncated.
% After MPO decomposition, the central tensor can encode the essential data from the original matrix, while the other auxiliary tensors serve as its complement.

% \subsection{Parameter-efficient Architecture}
% \subsection{Weight Sharing with Layer-wise Adaption}
% \subsection{Lightweight Layer-wise Adaption}

%\subsection{Scaling to Deeper Models based on MPO}
%\label{subsec-parameter-efficient-architecture}
% 回顾一下Transformer结构和参数共享策略,以及表示。当前的问题是这个共享参
% \textcolor{blue}{Add a summary sentence in there.}
%In this subsection, we first describe the overview of our approach, and then present the technical details for the proposed architecture.    
%introduce our proposed MPOBERT in order to build a deep Transformer-based model with fewer parameters.

\ignore{The general idea of MPOBERT is to share parameters among different Transformer layers to build parameter-efficient PLMs. 
% Prior studies have demonstrated weight sharing as an effective method to compress large Transformer-based models~\cite{}. For example, \citet{lan2019albert} shares all of the Transformer layers in BERT to build a highly compact pre-trained language model.
Prior studies have demonstrated weight sharing as an effective method to build highly compact PLMs~\cite{lan2019albert}. However, a serious problem encountered with the weight-sharing technique is performance degradation and the main cause is the strict identity of weights across different layers. Thus, \citet{zhang2022minivit} and~\citet{nouriboriji@2022minialbert} consider supplementing the shared weights by adding an extra block for each layer. As a comparison, we decompose model weights into tensors containing common and specific information, which makes it potentially possible to consider only sharing common information across layers to alleviate the performance degradation issue. 
% Unlike these works, we split model weight into common and specific parameters and only share common parameters across different layers. 
}

%To develop a more principled weight sharing mechanism, our approach is based on  two motivations.


\ignore{To achieve this, we introduce a novel matrix decomposition method, \ie MPO decomposition. An important merit of MPO decomposition is that it can reorganize and aggregate information in central tensors.
Thus we propose an MPO-based Transformer layer containing two major parts: First we can share the central tensor parameters across different layers. Then, we design layer-specific adapters to supplement the capacity of each layer in MPOBERT. We will describe each part in detail.
}

% Thus we can share the central tensor parameters across different layers. In addition, we design layer-specific adapters to supplement the capacity of each layer in MPOBERT. We will describe each part in detail.
% To be concrete, we denote shared parameters as $\Matrix{W}_0$ while specific parameters as $\Matrix{W}^{'}$ and weight-sharing models can be formulate as Eq.~\ref{eq-weight-sharing}:
% 探测试验
% Based on our probing test in~\ref{}, we observe that the 
% 共享所有层参数会限制模型的表达能力,导致无法构建更加深层的模型。替代全部共享,一些工作提出了部分共享,同时增加额外的参数来增加模型的表达能力。因此共享哪些参数成为了一个核心问题,来实现参数不随参数增加的太大,同时保证模型性能的提升。
% 同时也暴露出来参数共享遇到的核心问题就是深层效果下降。语义探测实验发现共享参数后导致模型的表达能力下降,是核心原因。为了解决这个问题,cite{} 均采用增加额外的模块来增加模型参数以强化模型的表达能力。与这些方法不同,我们关注共享的策略本身,核心挑战是到底该共享哪些参数。我们将要共享的参数表示为\theta_0,其他参数表示为\theta^',那么基于参数共享的模型可以被表达为:
% \begin{equation}
%     \Vector{h}_{i+1}=f(\Vector{h}_i; \Matrix{W}_0, \Matrix{W}^{'}), i=0,1,\cdots,L-1.
% \label{eq-weight-sharing}
% \end{equation}

% Given this condition, the problem can be regarded as determining parameters $\Matrix{W}_0$ to be shared across layers in order to reduce model redundancy while maintaining performance in downstream tasks. Next, we will describe two proposed techniques to share parameters and supplement the capacity of each layer to enhance downstream performance.

% In order to solve this problem, we propose two major techniques, which can largely reduce total parameters when scaling to deeper models and effectively train deep models.

\subsection{MPO-based Transformer Layer}
\label{subsec-crosslayer}
In this section, we first introduce the MPO decomposition and introduce how to utilize it for building parameter-efficient deep PLMs.  
%We will first introduce the MPO decomposition method. Next, we will describe two proposed techniques: cross-layer parameter sharing and layer-specific adapters to eventually build our parameter-efficient deep Transformer-based model.
% An important merit of MPO decomposition is that can reorganize and aggregate information in central tensors.
% Thus we can share the central tensor parameters across different layers. In addition, we design layer-specific adapters to supplement the capacity of each layer in MPOBERT. We will describe each part in detail.

\subsubsection{MPO Decomposition}

%First, we will briefly describe the matrix product operator decomposition method. 
%From quantum many-body physics~\cite{gao2020compressing}, the MPO decomposition 

Given a weight matrix $\Matrix{W}\in \mathbb{R}^{I\times J}$,   MPO decomposition~\cite{gao2020compressing} can  decompose a matrix into a product of $n$ tensors by reshaping the two dimension sizes $I$ and $J$:
\begin{equation}
    \small
   \Matrix{W}_{i_1,\dots , i_n, j_1, \dots, j_n} = \Tensor{T}^{(1)}[i_1, j_1]\cdots \Tensor{T}^{(n)}[i_n, j_n],
\label{eq:mpo-decom}
\end{equation}
where we have $I=\prod_{k=1}^{n} i_k$, $J=\quad \prod_{k=1}^{n}j_k$, and  $\Tensor{T}^{(k)}[i_k, j_k]$ is a 4-dimensional tensor with size $d_{k-1}\times i_k \times j_k \times d_k$ in which $d_k$ is a bond dimension linking $T^{(k)}$ and $T^{(k+1)}$ with $d_0=d_n=1$. For simplicity, we omit the bond dimensions  in Eq.~\eqref{eq:mpo-decom}.
When $n$ is odd, the middle tensor contains the most parameters (with the largest bond dimensions), while the parameter sizes of the rest decrease with the increasing distance to the middle tensor. 
Following \citep{liu2021enabling}, we further simplify the decomposition results of a matrix as a central tensor  $\mathcal{C}$ (the middle tensor) and auxiliary tensors $\{ \mathcal{A}_i \}_{i=1}^{n-1}$ (the rest tensor). 


As a major merit, such a decomposition
can effectively reorganize and aggregate the information of the matrix~\cite{gao2020compressing}:   central tensor  $\mathcal{C}$ can encode the essential information of the original matrix, while  auxiliary tensors $\{ \mathcal{A}_i \}_{i=1}^{n-1}$ serve as its complement to exactly reconstruct the matrix.  
%effectively reorganize and aggregate the information of the matrix. The parameter matrix is decomposed into a \emph{central tensor} and a set of \emph{auxiliary tensors}~\citep{liu2021enabling, gao2022parameter}. 

\ignore{Formally,  given a weight matrix $\Matrix{W}\in \mathbb{R}^{I\times J}$,  
we can factorize the two dimensions into a product of natural numbers, and reshape it into an $n$-dimension tensor $\Matrix{W}_{i_1,\dots, i_n, j_1, \dots, j_n}$, which  satisfies:
\begin{equation}
\small
    \prod_{k=1}^{n} i_k = I, \quad \prod_{k=1}^{n}j_k = J.
\end{equation}
This decomposition can be written as:
\begin{equation}
    \small
   \Matrix{W}_{i_1,\dots , i_n, j_1, \dots, j_n} = \Tensor{T}^{(1)}[i_1, j_1]\cdots \Tensor{T}^{(n)}[i_n, j_n],
\label{eq:mpo-decom}
\end{equation}
where the $\Tensor{T}^{(k)}[i_k, j_k]$ is a 4-dimensional tensor with size $d_{k-1}\times i_k \times j_k \times d_k$ in which $d_k$ is a bond dimension linking $T^{(k)}$ and $T^{(k+1)}$ with $d_0=d_n=1$.
with size $d_{k-1}\times i_k \times j_k \times d_k$ in which $\prod_{k=1}^{n}i_k=I, \prod_{k=1}^{n}j_k=J$ and $d_0=d_n=1$. This bond dimension indicates the associative strength between two adjacent tensors. 
For clarity, we can rewrite the decomposition results as central tensor  $\mathcal{C}$ and auxiliary tensors $\{ \mathcal{A}_i \}_{i=1}^{n-1}$. As an important merit, such a decomposition
can effectively reorganize and aggregate the information of the matrix~\cite{gao2020compressing}:   central tensor  $\mathcal{C}$ can encode the essential information from the original matrix, while  auxiliary tensors $\{ \mathcal{A}_i \}_{i=1}^{n-1}$ serve as its complement to precisely reconstruct the matrix. 
}


\ignore{According to ~\citet{gao2020compressing}, the original matrix $\Matrix{W}$ may be precisely reconstructed using tensor contraction without the connecting bond $\{d_k\}_{k=1}^n$ being truncated.
After MPO decomposition, the central tensor can encode the essential data from the original matrix, while the other auxiliary tensors serve as its complement.
}

\subsubsection{MPO-based Scaling to Deep Models }
\label{subsec-mpobased_scaling}
Based on MPO decomposition, the essence of our scaling method is to share the central tensor across layers (\emph{capturing the essential information}) and keep layer-specific auxiliary tensors (\emph{modeling layer-specific  variations}). Fig.~\ref{fig:main} shows the overview architecture of the proposed MPOBERT. 
%Since central tensor can encode the major information of the original matrix, our idea is to share the central tensor across layers,  

% \begin{figure*}[t]
%     \centering
%     \includegraphics[width=0.8\textwidth]{section/figs/main2.pdf}
%     \caption{Overview architecture of MPOBERT and MPOBERT$_{+}$. We use blocks with dashed borderlines to represent shared central tensors. Central tensors are shared across all $L$ Layers in MPOBERT and within groups in MPOBERT$_{+}$.}
%     % \textcolor{blue}{~(Left) We share the parameter $\Matrix{W}^{(1)}$ for all the layers.~(Middle) The tensor $\Tensor{C}$ is shared across all blue blocks while the auxiliary tensors are kept as layerwise parameters. We also add layer-specific adapters for each layer in yellow blocks.~(Right) We divided all the layers into three groups and shared the same set of central tensors in each group.}}
%     \label{fig:main}
% \end{figure*}

\begin{figure}[t]
    \centering
    \hspace{-0.33cm}\includegraphics[width=0.5\textwidth]{section/figs/main10.pdf}
    \caption{Overview architecture of MPOBERT and MPOBERT$_{+}$. We use blocks with dashed borderlines to represent shared central tensors. Central tensors are shared across all $L$ Layers in MPOBERT and within groups in MPOBERT$_{+}$.}
    % \textcolor{blue}{~(Left) We share the parameter $\Matrix{W}^{(1)}$ for all the layers.~(Middle) The tensor $\Tensor{C}$ is shared across all blue blocks while the auxiliary tensors are kept as layerwise parameters. We also add layer-specific adapters for each layer in yellow blocks.~(Right) We divided all the layers into three groups and shared the same set of central tensors in each group.}}
    \label{fig:main}
\end{figure}

\paratitle{Cross-layer Parameter Sharing}. To introduce our architecture, we  consider a simplified structure of $L$ layers, each consisting of a single matrix. With the five-order MPO decomposition (\ie $n=5$), we can obtain the decomposition results for a weight matrix ($\Matrix{W}^{(l)}$), denoted as $\{\Tensor{C}^{(l)}, \Tensor{A}_1^{(l)}, \Tensor{A}_2^{(l)}, \Tensor{A}_3^{(l)}, \Tensor{A}_4^{(l)}\}_{l=1}^{L}$, where   $\Tensor{C}^{(l)}$ and $\{\Tensor{A}_i^{(l)}\}_{i=1}^{4}$ are the central tensor and auxiliary tensors of the $l$-th layer. Our approach is to set a shared central tensor $\Tensor{C}$ across layers, which means that  $\Tensor{C}^{(l)}=\Tensor{C}$ ($\forall l=1\cdots L$). 
As shown in \citet{gao2020compressing}, the central tensor contains the major proportion of  parameters (more than 90\%), and thus our method can largely reduce the parameters when scaling a PLM to very deep architecture.  Note that this strategy 
can be easily applied to multiple matrices in a Transformer layer, and we omit the discussion for the multi-matrix extension.  Another extension is to share the central tensor by different grouping layers. 
% 老师写的
% We implement a layer-grouping MPOBERT, named \emph{MPOBERT$_{+}$}, in which it divides the layers into two parts (upper and down) and sets two unique shared central tensors. 
% 我改的
We implement a layer-grouping MPOBERT, named \emph{MPOBERT$_{+}$}, which divides the layers into multiple parts and sets unique shared central tensors in each group.

\ignore{To be more specific, we present a conceptual formulation about such a weight-sharing mechanism: 
\begin{equation}
    \Vector{h}_{l+1}=f(\Vector{h}_l; \Matrix{W}_0, \Matrix{W}^{'(l)}), l=0,1,\cdots,L-1, 
\label{eq-weight-sharing}
\end{equation}
where  $\Vector{h}_l$ denotes the hidden representation from the output of the $l$-layer, and 
shared parameters  (\ie central tensors) and layer-specific parameters (\ie auxiliary tensors and other adaptation parameters) are denoted as  $\Matrix{W}_0$ and $\Matrix{W}^{'(l)}$, respectively.  
}


%In particular, we propose the utilization of MPOBERT+, an extension of the all-layer sharing approach, which employs a layer-wise grouping strategy to enhance model capacity. That is, we partition all layers into multiple distinct groups and implement cross-layer parameter sharing within each group~(Fig.~\ref{fig:main}).
%To be concrete, we consider $L$ layers, so we have $L$ parameter matrix in total, denote by $\Matrix{W}^{L}_{l=1}$. Specifically, we denote shared parameters as $\Matrix{W}_0$ while specific parameters as $\Matrix{W}^{'(l)}$. Therefore, weight-sharing models can be formulated as Eq.~\ref{eq-weight-sharing}:
\ignore{\begin{equation}
    \Vector{h}_{i+1}=f(\Vector{h}_i; \Matrix{W}_0, \Matrix{W}^{'(i)}), i=0,1,\cdots,L-1.
\label{eq-weight-sharing}
\end{equation}
}


\ignore{
As discussed in the MPO decomposition process, a matrix can be decomposed into $n$ tensors, \ie one central tensor, and $n-1$ auxiliary tensors.
We consider five decomposed tensors~(\ie $n=5$) in this work for convenience.
The decomposition results can be denoted as $\{\Tensor{C}^{(l)}, \Tensor{A}_1^{(l)}, \Tensor{A}_2^{(l)}, \Tensor{A}_3^{(l)}, \Tensor{A}_4^{(l)}\}_{l=1}^{L}$, where $\Tensor{C}^{(l)}$ and $\{\Tensor{A}_i^{(l)}\}_{i=1}^{4}$ are the central tensor and auxiliary tensors of the $l$-th layer.
The fundamental concept behind developing a parameter-efficient PLM is to share the central tensor across multiple layers and to maintain layer-specific auxiliary tensors as layer-wise parameters, and we designate the global central tensor as $\Tensor{C}^{(l)}$.
% To develop a parameter-efficient BERT model, the core idea is to share the central tensor across different layers and keep layer-specific auxiliary tensors as layerwise parameters, and we denote the global central tensor as $\Tensor{C}^{(l)}$.
In this way, we can only keep one central tensor for each Transformer layer.
}

\ignore{
Second, we propose a simple yet effective cross-layer parameter sharing based on MPO decomposition. 
For Transformer-based models, there are mainly two major structures, namely feed-forward network~(FFN) and multi-headed attention~(MHA). 
Without loss of generality, we can consider a simple case in which each Transformer layer contains exactly one parameter matrix. It is easy to extend to multi-matrix cases.
}

\paratitle{Layer-specific Weight Adaptation}. Unlike ALBERT~\cite{lan2019albert}, our MPO-based architecture  enables layer-specific adaptation by keeping layer-specific auxiliary tensors ($\{\Tensor{A}_i^{(l)}\}_{i=1}^{4}$). 
These auxiliary tensors are decomposed from the original matrix, instead of extra blocks~\cite{zhang2022minivit}. They only contain a very small proportion of parameters, which does not significantly increase the model size. While, another merit of MPO decomposition is that these tensors are highly correlated via bond dimensions, and a small perturbation on an auxiliary tensor can reflect the whole matrix~\cite{liu2021enabling}.  If the downstream task requires more layer specificity, we can further  incorporate low-rank adapters~\cite{hu2021lora} for layer-specific adaptation. Specifically,  %we simplify the discussion by considering the weight matrix $\Matrix{W}^{(l)}$ of the $l$-th layer. 
we denote $\Matrix{W}^{(l)}_{Adapter}$ as the low-rank adapter for $\Matrix{W}^{(l)}$. 
In this way, $\Matrix{W}^{(l)}$ can be formulated as a set of tensors: $\{\Tensor{C}^{(l)},  \Tensor{A}_1^{(l)}, \Tensor{A}_2^{(l)},  \Tensor{A}_3^{(l)},  \Tensor{A}_4^{(l)}, \Matrix{W}^{(l)}_{Adapter}\}$. The parameter scale of  adapters,  $L \times r \times d_{total}$, is determined by the layer number $L$, the rank $r$, and the shape of  the original matrix ($d_{total}=d_{in}+d_{out}$ is the sum of the input and output dimensions of a Transformer Layer). Since we employ low-rank adapters, we can effectively control the number of  additional parameters from  adapters. 


\ignore{
While, auxiliary tensors only contain a very small proportion of parameters, which has limited capacity in fulfilling layer-specific variations.   
% With the cross-layer parameter-sharing strategy, we can supplement the capacity of different layers by utilizing layer-wise auxiliary tensors.
%As discussed in section~\ref{sec-framework}, the expressive ability of auxiliary tensors is limited due to the small number of parameters. 
Inspired by recent studies on adapters~\cite{hu2021lora}, we propose to add the low-rank adapters at each layer of MPOBRT. 
%for the MPO-based Transformer layer to further improve the capacity of $\Matrix{W}^{'}$ in each layer of MPOBERT. 
Specifically, we simplify the discussion by considering the weight matrix $\Matrix{W}^{(l)}$ of the $l$-th layer. We denote $\Matrix{W}^{(l)}_{Adapter}$ as the low-rank adapter for $\Matrix{W}^{(l)}$. 
In this way, $\Matrix{W}^{(l)}$ can be formulated as a set of tensors: $\{\Matrix{W}_0,  \Tensor{A}_1^{(l)}, \Tensor{A}_2^{(l)},  \Tensor{A}_3^{(l)},  \Tensor{A}_4^{(l)}, \Matrix{W}^{(l)}_{Adapter}\}$.
In inference, we first obtain the reconstruction matrix $\Matrix{W}^{(l)}$ by integrating the central tensor with the auxiliary tensor at the $l$-th layer and then add it by the adapter $\Matrix{W}^{(l)}_{Adapter}$. 
%In other words, the $\Matrix{W}^{(l)}_{Adapter}$ replaces the existing weight matrix with a trainable low-rank decomposition matrix. 
The parameter scale of  adapters is determined by the layer number, the rank $r$, and the shape of  the original matrix.
It can be computed as 
 $L \times r \times d_{total}$, where $d_{total}=d_{in}+d_{out}$ is the sum of the input and output dimension size of a Transformer Layer. Since we employ low-rank adapters, we can effectively control the number of  additional parameters from  adapters. 
}

%are determined by the layer number, the rank $r$, and the shape of the original matrix, which is $L \times r \times d_{total}$, where $d_{total}=d_{in}+d_{out}$ is the sum of the input and output dimension size of a Transformer Layer.

\subsection{Stable Training for MPOBERT}
\label{sec-efficient-training}

With the above MPO-based approach, we can scale a PLM to a deeper architecture  in a highly  parameter-efficient way. 
However, as shown in prior studies~\cite{lan2019albert,wang@2022deepnet}, it is difficult to optimize very deep PLMs, especially when shared components are involved. In this section, we introduce a simple yet stable training algorithm for MPOBERT and then discuss how it addresses the training instability issue.  

%Directly optimizing deep Transformers is difficult due to the training instability issue and large computation cost. In this subsection, we introduce how to solve the above challenges with theoretically derived initializing methods and acceleration techniques.
% \subsubsection{Tensor Initialization for Stable Optimization}
\subsubsection{MPO-based Network Initialization}
\label{sec-mpo-based-network-initialization}
Existing work has found that parameter initialization is important for training deep models~\cite{huang2020initialize,zhang2019fixup,wang@2022deepnet}, which can help  alleviate the training instability. 
To better optimize the scaling MPOBERT, we propose a specially  designed  initialization method  based on the above MPO-based architecture. 


\paratitle{Initialization with MPO Decomposition}. Since MPOBERT shares global components (\ie the central tensor) across all layers, our idea is to employ  existing well-trained PLMs based on weight sharing for improving parameter initialization.  Here, we use the released 24-layer ALBERT with all the parameters shared across layers. The key idea is to perform MPO decomposition on the parameter matrices of ALBERT, and obtain the corresponding central and auxiliary tensors. Next, we discuss the initialization of MPOBERT in two aspects. %, namely  central tensors ($\Tensor{C}^{(l)}$) and auxiliary tensors ($\Tensor{A}_i^{(l)}$). 
 For \emph{central tensors}, we directly initialize them  (each for each matrix) by the derived central tensors from the MPO decomposition results of ALBERT. Since they are globally shared, one single copy is only needed for initialization regardless of the layer depth. Similarly,  for  \emph{auxiliary tensors}, we can directly copy the auxiliary tensors from the  MPO decomposition results of ALBERT.
 
\paratitle{Scaling the Initialization}. A potential issue is that ALBERT only provides a 24-layer architecture, and such a strategy no longer supports the  initialization for an architecture of more than 24 layers (without corresponding auxiliary tensors). As our solution, we borrow the idea in \citet{wang@2022deepnet} that  avoids the exploding update by incorporating an additional scaling coefficient and multiplying  the randomly initialized values for the auxiliary tensors (those in higher than 24 layers) with a coefficient of $(2L)^{-\frac{1}{4}}$, where $L$ is the layer number. Next, we present a theoretical analysis  of  training stability. 

%we consider employing the well-trained 24-layer ALBERT model for helping initializing MPOBERT.  


\ignore{To address the instability problem during training, we propose a weight initialization method for deep MPO-based Transformers that involves initializing the central tensors with decomposed pre-trained weights and the auxiliary tensors with theoretically derived initial values.
Then we deliver a theoretical analysis to demonstrate the effectiveness of stabilizing the optimization.
We consider the typical $L$-layer MPOBERT architecture discussed in Section~\ref{subsec-parameter-efficient-architecture} and introduce our approach in two parts, the central tensors and the auxiliary tensors, respectively.
\textbf{(1) For the central tensors}, we only need to consider a single layer, regardless of the overall depth of the model. This means that we can potentially use a pre-trained shallow model to initialize very deep models. To do this, we utilize the central tensors obtained from the decomposed weights in ALBERT, as it is the only PLM that has a single layer of weights.
\textbf{(2) For the auxiliary tensors}, we find by a theoretical derivation that scaling the randomly initialized values for the auxiliary tensors with $(2N)^{-\frac{1}{4}}$ can ensure that the stability of the model during training will not be compromised as the depth increases. 
}


% --- v1
% In this part, we propose a theoretically derived initialization method for the auxiliary tensors and the central tensors respectively based on the analysis of the training stability issue of deep Transformers.
% Our primary technique for addressing the instability problem during training is limiting the scope of model changes using adaptive initialization.
% Previous studies illustrate that directly optimizing deep Transformer-based models encounters performance degradation. To address this problem,~\citet{wang@2022deepnet} has shown its effectiveness in stabilizing training by improving weight initialization. 
% However, neither of them discusses how to leverage PLMs for better initialization. 
% As a comparison, we can achieve better training stability by using initializing the central tensors with decomposed pre-trained weights and the auxiliary tensors with theoretically derived initial values. 
% Thus we start with a theoretical examination of the training instability issue, attributing it to a bound on the model updates. Then we propose our initialization method to optimize this bound.

% \begin{theorem}
%     In $N$-layer MPOBERT, we assume that $\left\| \frac{\partial F}{\partial c_1} \right\|=\mathcal{O}(1)$, \ie the gradient signal of $c_1$ from the layers above is bounded, then $\bigtriangleup F$ satisfies $\left\| \bigtriangleup F\right\|=\mathcal{O}(1)$ when for all $i=1,\cdots,N$:
%     \begin{equation}
%         (v_i^2+u_i^2)(u_Nv_N)=\mathcal{O}(\frac{1}{N})
%     \label{eq-thm2}
%     \end{equation}
% \label{theorem-2}
% \end{theorem}
% According to Theorem~\ref{theorem-2}, we get that $\{\left\|v_{i}^*-v_{i}\right\|\}_{i=1}^N$ and $\{\left\|u_{i}^*-u_{i}\right\|\}_{i=1}^N$ are bounded if Eq.~\ref{eq-thm2} is satisfied. Due to symmetry, we set $u_i=u$, $v_i=v$. 
% Thus, from Eq.~\ref{eq-thm2}, we set $\{u,v\}$ as following for initializing the auxiliary tensors: 
% \begin{equation}
%     u=v=(2N)^{-\frac{1}{4}}.
% \end{equation}
% Finally, by appropriately initializing local tensors in MPOBERT, we can ensure that the magnitude of model updates is bounded independent of model depth $N$, $\ie \left\| \bigtriangleup F\right\|=\mathcal{O}(1)$.

\subsubsection{Theoretical Analysis}

To understand the issue of training instability from a theoretical perspective, we consider a Transformer-based model $F(\Vector{x},\Matrix{W})$ with $\Vector{x}$ and $\Matrix{W}$ as input and parameters, and consider one-step update $\bigtriangleup F$\footnote{$\bigtriangleup F \overset{\bigtriangleup}{=}F(\Vector{x},\Matrix{W}-\eta\frac{\partial}{\partial \Matrix{W}}\mathcal{L}(F(\Vector{x})-y))-F(\Vector{x};\Matrix{W}).$}. %and write the (one-step) model update $\bigtriangleup $ as: $\bigtriangleup F \overset{\bigtriangleup}{=}F(\Vector{x},\Matrix{W}-\eta\frac{\partial}{\partial \Matrix{W}}\mathcal{L}(F(\Vector{x})-y))-F(\Vector{x};\Matrix{W}).$
\ignore{
\begin{equation}
\small
\bigtriangleup F \overset{\bigtriangleup}{=}F(\Vector{x},\Matrix{W}-\eta\frac{\partial}{\partial \Matrix{W}}\mathcal{L}(F(\Vector{x})-y))-F(\Vector{x};\Matrix{W}).
\end{equation} 
}
According to \citet{wang@2022deepnet}, a large model update ($\bigtriangleup F$) at the beginning of training is likely to cause the training instability of deep Transformer models. 
%the instability starts from the large model update, \ie $\left\| \bigtriangleup F\right\|$, at the beginning of training.
To mitigate the exploding update problem,  the update should be bounded by a constant, \ie $\left\| \bigtriangleup F\right\|=\mathcal{O}(1)$.
Next, we study how the $\bigtriangleup F$ is bounded with the MPOBERT. 

\ignore{
\textcolor{blue}{We first express the model updates in terms of $\bigtriangleup F$. That is, }given a learning rate $\eta$, the update to the model can be written as 
$\bigtriangleup F \overset{\bigtriangleup}{=}F(\Vector{x},\Matrix{W}-\eta\frac{\partial}{\partial \Matrix{W}}\mathcal{L}(F(\Vector{x})-y))-F(\Vector{x};\Matrix{W})$. 
% It has been shown that an exploding model update at the beginning of training can cause instability~\citep{wang@2022deepnet}.
\textcolor{blue}{~\citet{wang@2022deepnet} demonstrates the instability starts from the large model update, \ie $\left\| \bigtriangleup F\right\|$, at the beginning of training. In order to mitigate the exploding update problem, we need to make the update bounded by a constant, \ie $\left\| \bigtriangleup F\right\|=\mathcal{O}(1)$.}
% Therefore, our goal is to bound the magnitude of $\left\| \bigtriangleup F\right\|$ independently of the model depth.
}

\paratitle{MPO-based Update Bound}. Without loss of generality, we consider a simple case of low-order MPO decomposition: $n=3$ in Eq.~\eqref{eq:mpo-decom}.  Following the derivation method in \citet{wang@2022deepnet}, we simplify the matrices $\Matrix{W}$, $\Tensor{A}_1$,  $\Tensor{C}$ and $\Tensor{A}_2$ to scalars $w$,$u$,$c$,$v$, which means the parameter $w_l$ at the $l$-th layer   can be decomposed as $w_l=u_l\cdot c_l\cdot v_l$. Based on these notations, we consider $L$-layer Transformer-based model $F(x,w)(w=\{w_1, w_2, ...,w_L\})$, where each sub-layer is normalized with Post-LN: $x_{l+1}=LN(x_l+G_l(x_l,w_l))$. Then we can prove $\left\| \bigtriangleup F\right\|$ satisfies (see Theorem~\ref{app-thm1} in the Appendix):
%we next present Theorem~\ref{theorem-1} for the magnitude of $\left\| \bigtriangleup F\right\|$ for an $L$-layer MPOBERT.  Given an $L$-layer Transformer-based model $F(x,w)(w=\{w_1, w_2, ...,w_L\})$, where $w_l$ denotes the parameters in $l$-th layer and each sub-layer is normalized with Post-LN: $x_{l+1}=LN(x_l+G_l(x_l,w_l))$. 
       \begin{align}
    \left\| \bigtriangleup F\right\|\leq
    % &\sum_{i=1}^{N}\frac{1-u_ic_1v_i}{(1+u_i^2c_1^2v_i^2)^{\frac{3}{2}}}(c_1v_i\left\|u_i^*-u_i \right\| \nonumber\\
    % &+ c_1u_i\left\|v_i^*-v_i \right\| + u_iv_i\left\|c_1^*-c_1 \right\|)
    &\sum_{l=1}^{L}(c_1v_l \left\|u_{l}^*-u_{l}\right\|+ c_1u_l \left\|v_{l}^*-v_{l}\right\| \nonumber\\
    &+ v_lu_l \left\|c_1^*-c_1\right\|),
    \label{eq-theorem1}
    \end{align}
\ignore{
\begin{theorem}
    Given an $L$-layer Transformer-based model $F(x,w)(w=\{w_1, w_2, ...,w_L\})$, where $w_l$ denotes the parameters in $l$-th layer and each sub-layer is normalized with Post-LN: $x_{l+1}=LN(x_l+G_l(x_l,w_l))$. In MPOBERT, $w_l$ is decomposed by MPO to local tensors: $w_l=u_l\cdot c_l\cdot v_l$, and we share $\{c_i\}_{i=1}^{L}$ across $L$ layers: $c_l=c_1, l=1,2,\cdots,L$. Then $\left\| \bigtriangleup F\right\|$ satisfies:
\label{theorem-1}
\end{theorem}
}
The above equation bounds the model update in terms of the central and auxiliary tensors. 
Since central tensors ($c_l$) can be initialized using the pre-trained weights, we can further simplify the above bound by reducing them. With some derivations (See Corollary~\ref{app-thm2} in the Appendix), we can obtain $(v_i^2+u_i^2)(u_Lv_L)=\mathcal{O}(\frac{1}{L})$ in order to guarantee that $\left\| \bigtriangleup F\right\|=\mathcal{O}(1)$.   For simplicity, we set $u_i=v_i=(2L)^{-\frac{1}{4}}$ to  bound the magnitude of each update  independent of layer number $L$. In the implementation, we first adopt the Xavier method for initialization, and then scale the parameter values with the coefficient of $(2L)^{-\frac{1}{4}}$. 


\ignore{Since central tensors ($c_1$) can be initialized using the pre-trained weights, we can further simplify the above bound by reducing the term of $c_i$. For simplicity, we set $u_i=v_i=(2L)^{-\frac{1}{4}}$ to  bound the magnitudes of each update  independent of layer number $L$, \ie $\left\| \bigtriangleup F\right\|=\mathcal{O}(1)$.
The complete derivations for Theorem~\ref{theorem-1} and Corollary~\ref{theorem-2} are given in the Appendix.  
In the implementation, we first adopt the Xavier method for initialization, and then scale the values with the coefficient of $(2L)^{-\frac{1}{4}}$. }
%it is reasonable to assume that updates of $c_1$ are well-bounded. Then the bound in Eq.~\ref{eq-theorem1} based on MPO decomposition is actually tighter than other methods with random initialization. In order to scaling 
% --------- v1
% With this derivation we find that with the proper initialization of $\{\left\|v_{i}^*-v_{i}\right\|\}_{i=1}^N$, $\{\left\|u_{i}^*-u_{i}\right\|\}_{i=1}^N$ and $\left\|c_1^*-c_1\right\|$, we can restrict model updates to a constrained range. 

% Finally, our proposed initialization method is able to easily achieve this bound. Pre-trained weights are generally superior to randomly initialized weights, as demonstrated by previous research~\cite{huang2020initialize,zhang2019fixup} showing that initialization can reduce the difficulty of training deep models.~\citet{gong@2019efficient_stack} specifically demonstrated the effectiveness of shallow initialization of deep layers in accelerating convergence, and highlighted the superiority of pre-trained weights over random initialization. This suggests that our proposed method for initializing the central tensors will be particularly effective. For the auxiliary tensors, we only need to adjust their initialization value by a factor of $(2N)^{-\frac{1}{4}}$ to ensure depth-independence on the right-hand side of the bound. See the appendix~\ref{app-thm2} for more details on the theory behind these steps.

% --------- v2
\ignore{
\begin{corollary}
    % In $L$-layer MPOBERT, we assume that $\left\| \frac{\partial F}{\partial c_1} \right\|=\mathcal{O}(1)$, \ie the gradient signal of $c_1$ from the layers above is bounded, 
    Given that we initialize $c_1$ in MPOBERT with well-trained weights, it is reasonable to assume that updates of $c_1$ are well-bounded.
    Then $\bigtriangleup F$ satisfies $\left\| \bigtriangleup F\right\|=\mathcal{O}(1)$ when for all $i=1,\cdots,N$:
    \begin{equation}
        (v_i^2+u_i^2)(u_Nv_N)=\mathcal{O}(\frac{1}{N})
    \label{eq-thm2}
    \end{equation}
\label{theorem-2}
\end{corollary}
}
%For simplicity, we set $u_i=v_i=(2L)^{-\frac{1}{4}}$ to  bound the magnitudes of each update  independent of layer number $L$, \ie $\left\| \bigtriangleup F\right\|=\mathcal{O}(1)$. The complete derivations for Theorem~\ref{theorem-1} and Corollary~\ref{theorem-2} are given in the Appendix.   In implementation, we first adopt the Xavier method for initialization, and then scale the values with the coefficient of $(2L)^{-\frac{1}{4}}$. 
% Due to symmetry, we set $u_i=u$, $v_i=v$, and $u=v=(2N)^{-\frac{1}{4}}$ to achieve bound the magnitudes of each update to be independent of model depth $N$, \ie $\left\| \bigtriangleup F\right\|=\mathcal{O}(1)$.

% --------- v2

\paratitle{Comparison}. Previous research has shown that using designed values for random initialization can improve the training of deep models~\citep{huang2020initialize,zhang2019fixup,wang@2022deepnet}. These methods  aim to 
improve the initialization of general Transformer architectures  for training from scratch. 
 %$While these approaches aim to improve initialization for training from scratch, 
% our proposed method is easier to use and achieve the same goal by leveraging pre-trained weights in the central tensors rather than random values. 
As a comparison, we explore the use of pre-trained weights  and  employ the MPO decomposition results for initialization.  
%our suggested method achieves the same goal with less computational costs by utilizing pre-trained weights in the central tensors instead of random values.
In particular,~\citet{gong@2019efficient_stack} have demonstrated the effectiveness of stacking pre-trained shallow layers for deep models in  accelerating convergence, also showing performance  superiority  of pre-trained weights over random initialization.

 %in accelerating convergence and highlighted the superiority of pre-trained weights over random initialization. %Therefore, our proposed method for initializing the central tensors is expected to be particularly effective.
% The closest work to ours in training deep models is DeepNet~\citep{wang@2022deepnet}, which proposes a new normalization function, Deepnorm, and a theoretically derived initialization method to address the training instability issue in deep Transformers. The initialization can only work when training from scratch and thus prevent DeepNet to leverage pre-trained models to accelerate the pre-training procedure. Different from DeepNet, we initialize the central tensors in MPOBERT from decomposed weights in pre-trained ALBERT which prove to largely reduce the computational cost in the pre-training procedure.

% \subsubsection{Combined Speedup}
\subsubsection{Training and Acceleration}
% 基于我们提出的MPOBERT模型,
% Based on the MPO-based Transformer layer and initialization methods, we summarize the training process in Algorithm~\ref{alg-overall-process}.
% We summarize our proposed MPO-based Transformer layer, corresponding initialization methods, and efficient training techniques, in a single Algorithm~\ref{alg-overall-process}, which can be easily applied to different PLMs for scaling along the model depth and efficient training.
% --- V2
% Generally speaking, our proposed MPO-based Transformer layer and corresponding initialization methods can be easily applied to different PLMs for scaling along the model depth and efficient training, which we summarize in a single Algorithm~\ref{alg-overall-process}.
% --- v3
\ignore{In general, our approach can be easily adapted to various PLMs for scaling along the model depth and efficient training~(see Algorithm~\ref{alg-overall-process}). 
We implemented various strategies to reduce the pre-training cost of MPOBERT during the process of realization. The results are summarized in Table~\ref{tab-accelerating}.
}

To instantiate our approach, we pre-train a 48-layer BERT model (\ie MPOBERT$_{48}$). For a fair comparison with BERT$_{\rm{BASE}}$ and BERT$_{\rm{LARGE}}$, we adopt the same pre-training corpus~(BOOKCORPUS~\citep{zhu2015aligning} and English Wikipedia~\citep{devlin2018bert}) and pre-training tasks~(masked language modeling, and sentence-order prediction).  
We first perform MPO decomposition on the weights of ALBERT and employ the initialization algorithm in Section~\ref{sec-mpo-based-network-initialization} to set the parameter weights. During the training, we need to keep an updated copy of central tensors and auxiliary tensors: we optimize them according to the pre-training tasks in an end-to-end way and combine  them to derive the original parameter matrix for forward computation (taking a relatively small cost of parallel matrix multiplication). 

\ignore{
In general, our suggested MPO-based Transformer layer and accompanying initialization methods can be easily applied to various PLMs for scaling along the model depth and efficient training, as summarised in a single Algorithm~\ref{alg-overall-process}. Furthermore, we implemented various strategies to reduce the pre-training cost of MPOBERT and summarize the results in Table~\ref{tab-accelerating}.}
% 通过多种方法组合,我们将MPOBERT的训练时间减少了80%,让48层的MPOBERT训练时间从15天减少到3.8天,具体的实验数据我们统计在了表格中。我们参考了MPOP的模型中MPO的实现作为基础,增加了混合精度训练策略,算子融合方法,整合在了deepspeed的框架内来实现并行训练。在BERT base的结构上,原有的官方实现需要steps,训练days达到的效果,现在只需要steps,训练days即可达到。我们的方法支持48层的BERT,仅需要3.8天就可以训练完成。
% By utilizing multiple ways to accelerate the training of MPOBERT, we are able to reduce the $70\%$ training cost in pre-training MPOBERT from scratch. Following~\citet{liu2021enabling}, we implement the baseline version of MPOBERT and it cost 12.5 days for pre-training. First, we rebuild the Transformer layer in MPOBERT to fully adapt to the DeepSpeed package, which includes various optimizations, such as I/O pre-fetching, and mixed precision training. Second, we also use fused implementations for all linear-activation-bias operations and layer norms. The comparison of training costs is summarized in Table~\ref{tab-accelerating}. Compared to the baseline version~($\ie w/o DS, fp16$) we find that the pre-training of our optimized version reduces the total training cost from 12.5 days to 3.8 days while also saving 2.9$G$ Mem in total.
% --- v1
% We have implemented various strategies to reduce the pre-training cost of MPOBERT by 70\% as shown in Table~\ref{tab-accelerating}. 
% First, we utilize a mixed precision training technique to reduce memory consumption during training and speed up the multiplication process since FP16 format has a narrower dynamic range than FP32 and 
% To achieve this improvement, we adapted the Transformer layer in MPOBERT to the DeepSpeed package~\cite{rasley2020deepspeed}, which includes optimization techniques such as I/O prefetching and mixed precision training. 
% We also adopted fused implementations for linear-activation-bias operations and layer norms. These changes resulted in a reduction of total training costs~(MPOBERT 3.8 days v.s. Baseline 12.5 days~\cite{liu2021enabling}) and a saving of 2.9GB of memory.
% --- v2
% The speed of the pre-training process is typically limited by one of three factors: arithmetic bandwidth, memory bandwidth, or latency. Mixed precision training addresses two of these
% limiters. Memory bandwidth pressure is lowered by using fewer bits~(FP16) to store the same number of values. Arithmetic time can also be lowered on processors that offer higher throughput for reduced
% precision math. In addition, we also adopted fused implementations for linear-activation-bias operations and layer norms to reduce the latency since 
% --- v3
% We have implemented various strategies to reduce the pre-training cost of MPOBERT and summarize the results in Table~\ref{tab-accelerating}.
% Typically, one of three factors—arithmetic bandwidth, memory bandwidth, or latency—limits the speed of the pre-training procedure. 

% Table~\ref{tab-accelerating} describes in detail the effects of the different acceleration techniques we propose.
% ---v4


Typically, the speed of the pre-training process is affected by three major factors: arithmetic bandwidth, memory bandwidth, or latency. We further utilize a series of efficiency optimization ways to accelerate the pre-training, such as {mixed precision training with FP16} (reducing memory and arithmetic bandwidth) and {fused implementation of activation and normalization} (reducing latency). Finally, we can train the 48-layer MPOBERT at a time cost of 3.8 days (compared with a non-optimized cost of 12.5 days) on our server configuration (8 NVIDIA V100 GPU cards and 32GB memory). More training details are can be found in the experimental setup Section~\ref{sec-experimental-setup} and  Appendix~\ref{add-trianing-detail}~(Table~\ref{tab-strongest_variants} and Algorithm~\ref{alg-overall-process}). 


\ignore{First, we employ mixed precision training to alleviate memory and arithmetic bandwidth limitations. In particular, we use fewer bits~(FP16) to store the same number of variables, hence decreasing the memory bandwidth constraint. In addition, reduced arithmetic time is also possible on processors with higher throughput for reduced precision arithmetic. Second, we adopt the fused implementation technique for linear-activation-bias operations and layer norms 
which decreases the latency by reducing the number of memory accesses.
% in order to eliminate unnecessary materialization of intermediate results. 
More details are included in Appendix. In general, these changes resulted in a reduction of total training costs~(MPOBERT 3.8 days v.s. Baseline 12.5 days~\cite{liu2021enabling}) and a saving of 2.9GB of memory.
}

\ignore{
\begin{table}[h]
\centering
\small
\begin{tabular}{llcll}
\toprule
\multicolumn{1}{c}{Exp}         & BS & Mem~(GB)  & Days \\ \midrule
\multirow{2}{*}{MPOBERT$_{48}$} & 4  & 11.6     & 4.9     \\ 
                                & 8  & 16.3     & 3.8     \\ \midrule
w/o FI                          & 4  & 11.4     & 5.4     \\ 
w/o FP16                        & 4  & 18.7     & 12.2     \\ 
w/o FI,FP16                     & 4  & 19.2     & 12.5     \\ \bottomrule
\end{tabular}
\caption{A speed comparison between our optimized training framework of MPOBERT and original implementation from MPOP~\cite{liu2021enabling}. Specifically, ``FI'' is for fused implementation approach, whereas ``FP16'' stands for mixed precision training.}
\label{tab-accelerating}
\end{table}
}
\section{Experiments}
In this section, we first set up the experiments and then evaluate the efficiency of MPOBERT on a variety of tasks with different model settings.
% report the results and the detailed analysis to demonstrate the effectiveness of MPOBERT. 
% For all experiments, we use NVIDIA Tesla V100.

\subsection{Experimental Setup}
\label{sec-experimental-setup}
\paratitle{Pre-training Setup}.
% For the datasets, we follow BERT~\citep{devlin2018bert} setup and use BOOKCORPUS~\citep{zhu2015aligning} and English Wikipedia~\citep{devlin2018bert} for pre-training our models in order to keep the comparison as meaningful as possible.
For the architecture, we denote the number of layers as $L$, the hidden size as $H$, and the number of self-attention heads as $A$. We report results on four model sizes: \textbf{MPOBERT$_{\textbf{12}}$}~($L$=12, $H$=768, $A$=12), \textbf{MPOBERT$_{\textbf{24}}$} ($L$=24, $H$=1024, $A$=16), \textbf{MPOBERT$_{\textbf{48}}$}~($L$=48, $H$=1024, $A$=16) and \textbf{MPOBERT$_{\textbf{48+}}$} that 
% only adds two sets of central tensors on  MPOBERT$_{48}$.
implement cross-layer parameter sharing in three distinct groups as discussed in subsection~\ref{subsec-mpobased_scaling}.
% We pre-train all of the models from scratch with batch size of 4096 for 10$k$ steps.
We pre-train all of the models with a batch size of 4096 for 10$k$ steps. Our code will be released after the review period.

\paratitle{Fine-tuning Datasets}.
To evaluate the performance of our model, we conduct experiments on the GLUE~\citep{wang2018glue} and SQuAD v1.1~\citep{rajpurkar2016squad} benchmarks.
% GLUE benchmark covers multiple datasets~(MNLI, QNLI, QQP, CoLA, RTE, MRPC, SST-2)~\footnote{In line with~\citet{raffel2020exploring}, we do not test WNLI due to its adversarial character with respect to the training set.}. 
% The SQuAD is a collection of 100$k$ crowd-sourced question/answer pairs. Given a question and a passage, the task is to predict the answer text span in the passage. 
Since fine-tuning is typically fast, we run an exhaustive parameter search and choose the model that performs best on the development set to make predictions on the test set. 
We include the details in the Appendix(see Appendix~\ref{add-detail_dataset} for the datasets and Appendix~\ref{add-detail_metric} for evaluation metrics)
% The results are summarized in Table~\ref{tab-main_results}.
% We report the results of both the development set and the test set in GLUE. 
% Since the original test sets are not accessible, we divide the original validation set in half and use one half for validation and the other for the test for datasets with fewer than 10,000 samples~(RTE, MRPC, STS-B, CoLA)~\citep{zhang2020revisiting}.
\begin{table*}[ht]
\centering
\small
\begin{tabular}{l|rrrrrrrrr|rr}                                     
\toprule[1pt]
\small
\multirow{2}{*}{Experiments}    & \makebox[0.04\textwidth][c]{MRPC} & \makebox[0.04\textwidth][c]{SST-2} & \makebox[0.04\textwidth][c]{CoLA} & \makebox[0.04\textwidth][c]{RTE}   & \makebox[0.04\textwidth][c]{STS-B} & \makebox[0.04\textwidth][c]{QQP} & \makebox[0.04\textwidth][c]{MNLI} & \makebox[0.04\textwidth][c]{QNLI} & \makebox[0.04\textwidth][c]{SQuAD} &  \makebox[0.04\textwidth][c]{Avg.} &\makebox[0.04\textwidth][c]{\#To~(M)}  \\ 
                                & F1   & Acc.  & Mcc.   & Acc.   & Spear. & F1/Acc. & Acc. & Acc. & F1\\ \midrule
\rowcolor{gray!10}\multicolumn{12}{c}{\it \textbf{Development set}}\\
\rowcolor{gray!10}\multicolumn{12}{l}{\textbf{Tiny Models}~(\rm{\#To < 50M)}}\\
ALBERT$_{12}$                      & 89.0           & 90.6              & 53.4             & 71.1             & 88.2             & -/89.1             & 84.5             & 89.4             & 89.3          & 82.7             & 11 \\
ALBERT$_{24}$                      & 84.6           & \underline{93.6}  & 52.5             & \textbf{79.8}    & 90.1             & -/88.1             & 85.0             & \underline{91.7} & 90.6          & \underline{84.0} & 18 \\ 
\textcolor{purple}{MPOBERT$_{12}$} & \underline{90.3} & 92.3            & \underline{55.2} & 71.8             & \underline{90.5} & \underline{-/90.1} & \underline{84.7} & 91.2             & 90.1          & \underline{84.0} & 20 \\ 
\textcolor{purple}{MPOBERT$_{24}$} & \textbf{90.3}  & \textbf{94.4}     & \textbf{58.1}    & \underline{75.5} & \textbf{91.1}    & \textbf{-/90.2}    & \textbf{87.0}    & \textbf{92.6}    & \textbf{92.3} & \textbf{85.7}    & 46 \\\midrule
\rowcolor{gray!10}\multicolumn{12}{l}{\textbf{Small Models}~(\rm{50M < \#To < 100M)}}\\
T5$_{12}$                           & \underline{89.2}  & 94.7  & \underline{53.5}  & \underline{71.7} & \underline{91.2}   & \textbf{-/91.1}    & \textbf{87.8}    & \textbf{93.8}     & \underline{90.0}    & \underline{84.8} & 60\\ 
\textcolor{purple}{MPOBERT$_{48}$}  & \textbf{90.8}     & 94.7	& \textbf{58.3}     & \textbf{77.3}	   & \textbf{91.4}      & \underline{-/89.5} & \underline{86.3} & \underline{92.0}  & \textbf{92.3}       & \textbf{85.8}    & 75\\ \midrule
\rowcolor{gray!10}\multicolumn{12}{l}{\textbf{Base Models}~(\rm{\#To > 100M)}}\\
BERT$_{12}$                         & 90.7              & 91.7             & 48.9             & 71.4             & \underline{91.0} & \underline{-/90.8} & 83.7             & 89.3             & 88.5             & 82.9             & 110  \\
XLNet$_{12}$                        & 85.3              & \underline{94.4} & 49.3             & 63.9             & 85.6             & -/90.7             & \textbf{90.9}    & 91.8             & 90.2             & 82.5             & 117 \\
RoBERTa$_{12}$                      & \textbf{91.9}     & 92.2             & \textbf{59.4}    & 72.2             & 89.4             & \textbf{-/91.2}    & \underline{88.0} & \textbf{92.7}    & 91.2             & \underline{85.4} & 125\\
BART$_{12}$                         & \underline{91.4}  & 93.8             & 56.3             & \underline{79.1} & 89.9             & \underline{-/90.8} & 86.4             & \underline{92.4} & \textbf{91.6}    & 82.8             & 140 \\
\textcolor{purple}{MPOBERT$_{48+}$} & 89.7              & \textbf{94.4}    & \underline{57.4} & \textbf{79.8}    & \textbf{91.1}    & -/89.3             & 87.1             & \underline{92.4} & \underline{91.4} & \textbf{86.0}    & 102\\ \midrule\midrule
\rowcolor{gray!10}\multicolumn{12}{c}{\it \textbf{Test set}}\\
\rowcolor{gray!10}\multicolumn{12}{l}{\textbf{Tiny Models}~(\rm{\#To < 50M)}}\\
ALBERT$_{12}$                        & 89.2             & 93.2             & \underline{53.6} & 70.2            & \underline{87.3}  & 70.3/-            & 84.6              & \underline{92.5}   & 89.3             & 81.1              & 11\\
ALBERT$_{24}$                        & 88.7             & \underline{94.0} & 51.7	          & \textbf{73.7}	& 86.9              & 69.1/-	        & 84.9	            & 91.8	             & \underline{90.6}	& \underline{81.2}              & 18 \\
MobileBERT$_{24}$$\blacklozenge$     & 88.8             & 92.6             & 51.1             & 70.4            & 84.8              & \underline{70.5/-}& 83.3              & 91.6               & 90.3             & 80.4              & 25\\
\textcolor{purple}{MPOBERT$_{12}$}   & \textbf{89.2}    & 91.9             & 52.7             & 70.6            & 87.1              & 69.6/-            & \underline{85.0}  & 91.0               & 90.1             & 80.8              & 20     \\  
\textcolor{purple}{MPOBERT$_{24}$}   & \underline{89.0}	& \textbf{94.5}    & \textbf{55.5}	  & \underline{73.4}& \textbf{88.2}     & \textbf{71.0/-}	& \textbf{86.3}     & \textbf{93.0}      & \textbf{92.3}    & \textbf{82.6}     & 46   \\\midrule
\rowcolor{gray!10}\multicolumn{12}{l}{\textbf{Small Models}~(\rm{50M < \#To < 100M)}}\\
T5$_{12}$                           & \underline{89.7}  & 91.8              & 41.0             & 69.9               & 85.6              & 70.0/-            & 82.4              & 90.3              & 90.0              & 78.7              & 60\\ 
TinyBERT$_{6}$$\clubsuit$           & 87.3              & \underline{93.1}  & \underline{51.1} & \underline{70.0}   & \underline{83.7}  & \textbf{71.6/-}   & \underline{84.6}  & \underline{90.4}  & \underline{87.5}  & \underline{79.9}  & 67\\
DistilBERT$_{6}$$\clubsuit$         & 86.9              & 92.5              & 49.0             & 58.4               & 81.3              & 70.1/-            & 82.6              & 88.9              & 86.2              & 77.3              & 67\\
\textcolor{purple}{MPOBERT$_{48}$}  & \textbf{90.0}	    & \textbf{94.0}	    & \textbf{55.0}    & \textbf{74.0}	    & \textbf{88.7}     & \underline{71.0/-}& \textbf{86.5}     & \textbf{91.8}     & \textbf{92.3}     & \textbf{82.6}     & 75\\ \midrule
\rowcolor{gray!10}\multicolumn{12}{l}{\textbf{Base Models}~(\rm{\#To > 100M)}}\\
BERT$_{12}$$\spadesuit$              & 88.9             & 93.5              & 52.1          & 66.4              & 85.8              & 71.2/-            & 84.6             & 90.5              & 88.5               & 79.1     & 110    \\
XLNet$_{12}$                         & 89.2             & \underline{94.3}  & 47.3          & 66.5              & 85.4              & \underline{71.9/-}& \underline{87.1} & 91.4              & 90.2               & 80.4      & 117   \\
RoBERTa$_{12}$                       & 89.9	            & 93.2	            & \textbf{57.9} & 69.9	            & \underline{88.3}	& \textbf{72.5/-}   & \textbf{87.7}	   & \underline{92.5}	   & 91.2	& \underline{82.6}      & 125\\
BART$_{12}$                          & 89.9             & 93.7              & 49.6          & \underline{72.6}  & 86.9              & 71.7/-            & 84.9             & 92.3              & \textbf{91.6}               & 81.5      & 140   \\
\textcolor{purple}{MPOBERT$_{48+}$}  & \textbf{89.9}    & \textbf{94.5}     & \underline{56.0}  & \textbf{74.5} & \textbf{88.4}     & 70.5/-            & 86.5             & \textbf{92.6}  & \underline{91.4}      & \textbf{82.7}  & 102\\ \bottomrule
\end{tabular}
\caption{Performance comparison of different models on natural language understanding tasks~(in percent). ``\# To~(M)'' denote the number~(in millions) of total parameters. 
We compare MPOBERT with PLMs~(\ie BERT and ALBERT) and Parameter-efficient Transformers~(\ie MobileBERT, TinyBERT and DistilBERT), respectively. The best and the second-best performance in each task are highlighted in bold and underlined.
$\blacklozenge$: Experimental results by~\citet{sun2020mobilebert}.
$\clubsuit$: Experimental results by~\citet{jiao2019tinybert}.
$\spadesuit$: Experimental results by~\citet{devlin2018bert}.}
\label{tab-main_results}
\end{table*}

\paratitle{Baseline Models}.
We compare our proposed MPOBERT to the existing competitive deep PLMs and parameter-efficient models. In order to make fair comparisons, we divide the models into three major categories based on their model sizes: 

$\bullet$~{Tiny Models~(\rm{\#To < 50M}).} ALBERT$_{12}$~\cite{lan2019albert} is the most representative PLM that achieves competitive results with only 11M.
% We fine-tune all of the parameters in deep PLMs including BERT~\citep{devlin2018bert} and ALBERT~\citep{lan2019albert}. 

$\bullet$~{Small models~(50M< \#To <100M).}
% T5$_{12}$ is a small variant of T5~\cite{raffel2020exploring} which has only 6 encoder layers and 6 decoder layers. In addition, there are three parameter-efficient Transformer models that have similar parameters, namely MobileBERT~\citep{sun2020mobilebert}, DistilBERT~\citep{sanh2019distilbert} and TinyBERT~\citep{jiao2019tinybert}. We compare with these compressed models to show the benefit of scaling to deeper models over compressing large models to small variants.
We consider PLMs~(T5$_{12}$) and compressed models~(MobileBERT~\citep{sun2020mobilebert}, DistilBERT~\citep{sanh2019distilbert} and TinyBERT~\citep{jiao2019tinybert}).

$\bullet$~{Base models~(\#To > 100M).} We compare with BERT$_{12}$, XLNet$_{12}$, RoBERTa$_{12}$ and BART$_{12}$ for this category. 
Note that we only include the base variants that have similar model sizes in order to make a fair comparison. 
% More details about the comparison with the strongest variants are described in Appendix~\ref{app-exp}.
% $\bullet$~\underline{Parameter-efficient Models.} We compare with parameter-efficient models based on compression techniques including MobileBERT~\citep{sun2020mobilebert}, DistilBERT~\citep{sanh2019distilbert} and TinyBERT~\citep{jiao2019tinybert}.

More details about the baseline models are described in Appendix~\ref{add-detail_baseline}. 
% We use the same environment for all approaches without any additional techniques like label smoothing and multi-task learning.

\subsection{Main Results}
\paratitle{Fully-supervised setting}.
We present the results of MPOBERT and other baseline models on GLUE and Squad for fine-tuning in Table~\ref{tab-main_results}. 
% We find that MPOBERT demonstrates its superiority in terms of layer-specific parameters and parameter efficiency. 

% --v1
% First, we compare MPOBERT to PLMs with different model sizes. For tiny models, MPOBERT outperforms ALBERT on all datasets. Compared to the best ALBERT variant, ALBERT$_{24}$, MPOBERT$_{24}$ achieves prominent gains for both the development set~(85.7 vs. 84.0) and the test set~(82.6 vs. 81.2). These considerable gains clearly demonstrate the benefit of layer-specific parameters~(\ie the auxiliary tensors and layer-specific adapters). For small models, MPOBERT consistently achieves better results than T5$_{12}$. For base models, MPOBERT still achieves comparable results while having fewer parameters. By zooming in on specific tasks, we find that the baseline model performance varies widely across tasks with lower data size~(\ie RTE and CoLA). Large models tend to have better results~(63.6 for RoBERTa$_{12}$ on CoLA and 79.1 for BART$_{12}$ on RTE) due to sufficient model capacity. By contrast, our MPOBERT$_{48}$ with the least parameters still achieves comparable~(57.4 on CoLA) or even better~(79.8 on RTE) results. 
% This demonstrates that the deep model constructed by our method can effectively stimulate the capability of PLMs without significantly increasing the number of model parameters. This is good news for PLMs that currently promote scaling law and can be expanded to a larger PLM by a more parameter-efficient way to explore the capability of models.
% --v2
% We first compare MPOBERT to PLMs with different model sizes. We find that MPOBERT outperforms ALBERT, and achieves considerable gains compared to ALBERT${24}$ on both the development~(85.7 \emph{v.s.} 84.0) and test sets~(82.6 \emph{v.s.} 81.2). This demonstrates the benefit of layer-specific parameters~(\ie the auxiliary tensors and layer-specific adapters) in MPOBERT. In addition, MPOBERT consistently achieves better results than T5$_{12}$ for small models and comparable results to larger models while having fewer parameters. 
% In particular, we consider that for the same number of layers~($L$=12), we can still obtain comparable results compared to the PLMs in small models, or even better~(+1.7 for BERT$_{12}$ and +0.4 for XLNet$_{12}$).
% This further illustrates the superiority and efficiency of our method in constructing deep models without significantly increasing the number of parameters.
% % --v3
% In our comparison of MPOBERT to other PLMs with different model sizes, we found that MPOBERT demonstrates its parameter efficiency when compared to other PLMs within the same size category.
% Specifically, for tiny models, MPOBERT$_{24}$ outperforms ALBERT$_{24}$, and achieves substantial improvements on both the development~(85.7 \emph{v.s.} 84.0) and test sets~(82.6 \emph{v.s.} 81.2), highlighting the benefits of MPOBERT's layer-specific parameters, such as the auxiliary tensors and layer-specific adapters.  
% Furthermore, MPOBERT consistently achieves better results than T5$_{12}$ for small models and comparable results to larger models while having fewer parameters. 
% Additionally, we found that MPOBERT also demonstrates significant benefits when scaling along the model depth. For instance, when considering models with $L$=12 layers, MPOBERT achieves comparable results or even outperforms~(+1.7 for BERT$_{12}$ and +0.4 for XLNet$_{12}$) PLMs while having fewer parameters. 
% Overall, these results demonstrate the superiority and efficiency of our method in constructing deep models without significantly increasing the number of parameters.

% --v4
% We first compare MPOBERT to other PLMs with varying model sizes. We found that MPOBERT demonstrates its superiority in terms of layer-specific parameters and parameter efficiency. 
% Specifically, for tiny models, MPOBERT$_{24}$ outperforms ALBERT$_{24}$, and achieves substantial improvements on both the development~(85.7 \emph{v.s.} 84.0) and test sets~(82.6 \emph{v.s.} 81.2), highlighting the benefits of increased capacity from layer-specific parameters~(\ie the auxiliary tensors and layer-specific adapters). Furthermore, MPOBERT demonstrates significant benefits of scaling along the model depth with layer-specific parameters. For instance, MPOBERT$_{48}$ consistently achieves better results than T5$_{12}$ for small models and comparable results to other larger 12-layer PLMs with a reduced number of parameters.
% Additionally, we found that MPOBERT also demonstrates its parameter efficiency when compared to other PLMs within the same model depth. For instance, when considering models with $L$=12 layers, MPOBERT achieves comparable results or even outperforms~(+1.7 for BERT$_{12}$ and +0.4 for XLNet$_{12}$) PLMs while having fewer parameters. 

% Second, we compare MPOBERT to parameter-efficient Transformers. We find that MPOBERT outperforms MobileBERT, TinyBERT, and DistilBERT on almost all considered datasets. This further proves the advantage of the parameter-efficient scaling along the model depth over compressing existing large PLMs to smaller models.

% Overall, these results demonstrate the superiority and efficiency of our method in constructing deep models without significantly increasing the number of parameters.
%  ---v5
% We first compare MPOBERT to other PLMs with varying model sizes. 
% Specifically, for tiny models, MPOBERT$_{24}$ outperforms ALBERT$_{24}$, and achieves substantial improvements on both the development~(85.7 \emph{v.s.} 84.0) and test sets~(82.6 \emph{v.s.} 81.2), highlighting the benefits of increased capacity from layer-specific parameters~(\ie the auxiliary tensors and layer-specific adapters). Furthermore, MPOBERT demonstrates significant benefits of scaling along the model depth with layer-specific parameters. For instance, MPOBERT$_{48}$ consistently achieves better results than T5$_{12}$ for small models and comparable results to other larger 12-layer PLMs with a reduced number of parameters.

% Second, we compare MPOBERT to parameter-efficient Transformers. We find that MPOBERT outperforms MobileBERT, TinyBERT, and DistilBERT on almost all considered datasets. This further proves the advantage of the parameter-efficient scaling along the model depth over compressing existing large PLMs to smaller models.

% Finally, we found that MPOBERT also demonstrates its parameter efficiency when compared to other PLMs within the same model depth. For instance, when considering models with $L$=12 layers, MPOBERT achieves comparable results or even outperforms~(+1.7 for BERT$_{12}$ and +0.4 for XLNet$_{12}$) PLMs while having fewer parameters. 
% --- v6
Firstly, we evaluate MPOBERT's performance in comparison to other models with similar numbers of parameters. In particular, for tiny models, MPOBERT$_{24}$ outperforms ALBERT$_{24}$, and achieves substantial improvements on both the development set~(85.7 \emph{v.s.} 84.0) and test sets~(82.6 \emph{v.s.} 81.2). This highlights the benefits of increased capacity from layer-specific parameters~(\ie the auxiliary tensors and layer-specific adapters) in MPOBERT. Furthermore, for small and base models, 48-layer MPOBERT consistently achieves better results than T5$_{12}$ and all parameter-efficient models, while also achieving comparable results to other 12-layer PLMs with a reduced number of parameters. This demonstrates the significant benefits of scaling along the model depth with layer-specific parameters in MPOBERT.

Secondly, we assess MPOBERT's parameter efficiency by comparing it to other PLMs within the same model depth. For instance, when considering models with $L$=12 layers, MPOBERT achieves comparable results or even outperforms~(+1.7 for BERT$_{12}$ and +0.4 for XLNet$_{12}$) PLMs while having fewer parameters. This further highlights the advantages of MPOBERT's parameter-efficient approach in constructing deep models.

\paratitle{Multitask Fine-tuning Setting}.
% multitask finetuning是啥。核心是否可以强化多任务之间的辅助弱化多任务之间的干扰。
To demonstrate the effectiveness of our proposed parameter-sharing model in learning shared representations across multiple tasks, we fine-tune MPOBERT, BERT and ALBERT on the multitask GLUE benchmark and report the results in Table~\ref{tab:multi-task}. 
% In order to illustrate the effectiveness of MPOBERT, 
Specifically, we design two groups of experiments.~(1) Deep vs. shallow models. Comparing with BERT$_{12}$, MPOBERT$_{48}$ has much deeper Transformer layers but still fewer total parameters~(\ie 75M vs. 110M). We find that MPOBERT$_{48}$ achieves 1.4 points higher on average GLUE score than BERT$_{12}$.~(2) Central tensors sharing vs. all weight sharing. Comparing with ALBERT$_{12}$, MPOBERT$_{12}$ only shares part of weights, \ie central tensors, while ALBERT$_{12}$ shares all of the weights. 
% Since previous studies~\citep{rabeeh2021hyperformer, gao2022parameter} have observed the positive transfer effects provided by sharing parameters in the multi-task fine-tuning setting, we only focus on those tasks where the difference in task performance is greater than 2 points~(\ie MRPC, CoLA, and RTE).
% We have shown that sharing central tensors may improve the results more than sharing all weights~(89.9 \emph{v.s.} 89.4 for MRPC).
We find that sharing central tensors may effectively improve the average results than sharing all weights~(82.0 \emph{v.s.} 81.4 for MRPC).
% while also mitigating the performance decrease~(54.9 \emph{v.s.} 40.7 for CoLA) that would otherwise occur.
% 为了说明MPOBERT的有效性,我设计了2种对比变种,效果体现在表格中。(1)对比BERT和MPOBERT_48,为了说明增加深度对多任务训练的优势。我们发现,(1)对比ALBERT和MPOBERT_12为了说明,为了说明相比较全参数共享的优势。我们发现,
\begin{table}[t]
\centering
\small
\begin{tabular}{lrrrr} 
\toprule
    \multicolumn{1}{c}{\multirow{1}{*}{Datasets}} & \multicolumn{1}{c}{B$_{12}$} & \multicolumn{1}{c}{M$_{48}$} & \multicolumn{1}{c}{M$_{12}$} & \multicolumn{1}{c}{A$_{12}$} \\    \midrule
        MNLI~(Acc.)                  &83.9  & 85.4   & 82.8  & 82.7\\
        QNLI~(Acc.)                  &90.8  & 91.1   & 90.0  & 89.4\\
        SST-2~(Acc.)                 &91.7  & 93.0   & 90.9  & 90.6\\
        RTE~(Acc.)                   &81.2  & 82.0   & 79.8  & 79.1\\
        QQP~(Acc.)                   &91.2  & 87.6   & 90.4  & 89.7\\
        CoLA~(Mcc.)                  &53.6  & 54.9   & 45.0  & 35.9\\
        MRPC~(F1)                  &84.2  & 91.8   & 89.9  & 89.2\\
        STS-B~(Spear.)             &87.4  & 89.0   & 86.9  & 87.5\\\midrule
        Avg.                        &83.0  & 84.4   & 82.0  & 80.5\\ 
         {\#To~(M)}                 &110   & 75     & 20    & 11\\
\bottomrule
\end{tabular}
\caption{Performance of multi-task learning on GLUE benchmark obtained by fine-tuning BERT$_{12}$~(B$_{12}$), MPOBERT$_{48}$~(M$_{48}$), MPOBERT$_{12}$~(M$_{12}$) and ALBERT$_{12}$~(A$_{12}$)~(in percent).}
\label{tab:multi-task}
\end{table}

\paratitle{Few-shot Learning Setting}.
\begin{table}[t]
\small
\begin{tabular}{l|ccc|ccc}
\midrule
\multicolumn{1}{c}{}                        & \multicolumn{3}{c}{SST-2} & \multicolumn{3}{c}{MNLI}            \\ \midrule
\multicolumn{1}{c}{Shots~(K)} & 10      & 20    & 30              & 10    & 20    & 30    \\ \midrule
BERT$_{12}$                              & 54.8  & \underline{59.7}  & \underline{61.6}      & \textbf{37.0}  & 35.6  & 35.7  \\ \midrule
ALBERT$_{12}$                            & \underline{56.7}  & 59.3  & 60.0      & 36.3  & 35.6  & \underline{36.5}   \\ \midrule
MPOBERT$_{12}$                           & \textbf{58.9}  & \textbf{65.4}  & \textbf{64.6}      & \underline{36.7}  & \textbf{36.7}  & \textbf{37.1}   \\ \midrule
\end{tabular}
% \caption{Few-shot performance of BERT, ALBERT and MPOBERT.}
\caption{Comparison of few-shot performance.}
\label{tab-few_shot}
\end{table}
% ---v2
We evaluate the performance of our proposed model, MPOBERT, in few-shot learning setting~\cite{huang-etal-2022-clues} on two tasks, SST-2 and MNLI, using a limited number of labeled examples. Results in Table~\ref{tab-few_shot} show that MPOBERT outperforms BERT, which suffers from over-fitting, and ALBERT, which does not benefit from its reduced number of parameters. These results further demonstrate the superiority of our proposed model in exploiting the potential of large model capacity under limited data scenarios.

\begin{figure}[t]
\centering
\subfigure[Pre-training from scratch]{
\begin{minipage}[t]{0.5\columnwidth}
\label{fig2:left}
\centering
\includegraphics[width=\columnwidth]{section/figs/scratch2.pdf} 
\end{minipage}%
}%
\subfigure[Continual Pre-training]{
\begin{minipage}[t]{0.5\columnwidth}
\label{fig2:right}
\centering
\includegraphics[width=\columnwidth]{section/figs/initialize2.pdf} 
\end{minipage}%
}
\caption{Comparison of the SST-2 accuracy achieved through pre-training from scratch and pre-training with the initialization of decomposed ALBERT weights.}
\label{fig:fig2}
\end{figure}
\subsection{Detailed Analysis}
\paratitle{Analysis of Initialization Methods}.
% This experiment was performed to exclude the effect of initialized pre-trained weight on the final results. We initialized MPOBERT with the local tensors from ALBERT and then we continue to train both models in the pre-training datasets. We compare the downstream performance on GLUE shown in Table~\ref{tab-continue}. For ALBERT, the gains from continued training are negligible. On the contrary, MPOBERT achieves obvious improvement at the first 10$k$ training steps. This is a reassuring result that demonstrates the improvement of MPOBERT is not brought about by initialized pre-trained weights.
% This experiment is performed to exclude the effect of initialized pre-trained weight on the fine-tuning results. We plot the performance of the model on the downstream tasks at different training steps. Specifically, we initialized MPOBERT with different initialization methods~(Xavier in~\ref{fig2:left} and decomposed weights of ALBERT in~\ref{fig2:right}) and then pre-trained it. For pre-training from scratch, MPOBERT needs around 50$k$ steps to achieve similar performance with BERT$_{BASE}$ while initializing with ALBERT can significantly speed up the convergence and achieve obvious improvement at the first 10$k$ training steps. On the contrary, the gains of continual pre-training for ALBERT are negligible. This is a reassuring result that demonstrates the improvement of MPOBERT is not brought about by initialized pre-trained weights.
This experiment aims to exclude the effect of initialized pre-trained weights on fine-tuning results. We plot the performance of the model on SST-2 \emph{w.r.t} training steps. In particular, we compare the performance of MPOBERT using different initialization methods (Xavier in Fig.~\ref{fig2:left} and decomposed weights of ALBERT in Fig.~\ref{fig2:right}) for pre-training. The results demonstrate that pre-training MPOBERT from scratch requires around 50$k$ steps to achieve performance comparable to BERT$_{\rm{BASE}}$, while initializing with the decomposed weights of ALBERT significantly accelerates convergence and leads to obvious improvements within the first 10$k$ training steps. In contrast, the gains from continual pre-training for ALBERT are negligible. These results provide assurance that the improvements observed in MPOBERT are not solely attributed to the use of initialized pre-trained weights.
\begin{figure}[htb]
    \centering
    \includegraphics[width=0.48\textwidth]{section/figs/linguistic_v3.pdf}
    \caption{A visualization of layer-wise linguistic patterns. Each column represents a probing task, and each row represents a Transformer layer. The red dashed box indicates the layer that performs best.}
    \label{fig:linguistic}
\end{figure}

\begin{table}[ht]
\centering
\small
\begin{tabular}{lrrrr}                                                                                              
\toprule
Experiment      & SST-2 & RTE     & MRPC  &\#To~(M) \\ \midrule
MPOBERT$_{12}$  & 92.8  & 72.9    & 91.8  & 20.0\\ \midrule
w/o Adapter     & 92.3  & 71.8    & 90.3  & 19.4\\  
w/o PS    & 91.4  & 67.9    & 85.8  & 11.9\\ \bottomrule
\end{tabular}
\caption{Ablation study on the SST-2, RTE, and MRPC datasets~(in percent).}
\label{tab-ablation}
\end{table}
\paratitle{Ablation Analysis}.
% To verify the effectiveness of each component, we perform the ablation study using the SST-2, RTE, and MRPC datasets.
% We choose accuracy as the assessment metric and consider removing the layer-specific adapter~(short as Adapter) and cross-layer parameter sharing strategy~(short as PS), respectively.
% The ablation results are shown in Table~\ref{tab-ablation}. 
% We can see that removing any component would lead to a decrease in the model performance. 
% It demonstrates the effectiveness of each component in our strategy.
% ---v2
% To evaluate the contribution of each component to the performance of the MPOBERT model, 
% we conduct an ablation study by removing either the layer-specific adapter or the cross-layer parameter-sharing strategy and show the results in Table~\ref{tab-ablation}. 
% The results indicate that removing either component leads to a decrease in the model's performance, demonstrating the effectiveness of each component in our proposed strategy.
% ---v3
To assess the individual impact of the components in our MPOBERT model, we conduct an ablation study by removing either the layer-specific adapter or the cross-layer parameter-sharing strategy. The results, displayed in Table~\ref{tab-ablation}, indicate that the removal of either component results in a decrease in the model's performance, highlighting the importance of both components in our proposed strategy. While the results also indicate that cross-layer parameter sharing plays a more crucial role in the model's performance.
\begin{table}[]
\centering
\small
\begin{tabular}{lrrrrr}
\toprule
Rank                & SST-2     & RTE     & MRPC  &\#To~(M)      \\ \midrule
4                   & 91.9      & 69.7    & 88.2  & 19.7        \\  
8                   & 92.8      & 72.9    & 91.8  & 20.0        \\ 
64                  & 91.6      & 69.3    & 88.1  & 24.3        \\ \bottomrule
\end{tabular}
\caption{Comparison of different adapter ranks on three GLUE tasks~(in percent). ``Rank'' denotes the adapter rank in MPOBERT.}
\label{tab-rank}
\end{table}

\paratitle{Performance Comparison \emph{w.r.t} Adapter Rank}.
To compare the impact of the adapter rank in layer-specific adapters on MPOBERT's performance, we trained MPOBERT with different ranks~(4,8 and 64) and evaluate the model on downstream tasks in Table~\ref{tab-rank}. The results demonstrate that a rank of 8 is sufficient for MPOBERT, which further shows the necessity of layer-specific adapters. However, we also observe a decrease in the performance of the variant with adapter rank 64. This illustrates that further increasing the rank may increase the risk of over-fitting in fine-tuning process. Therefore, we set a rank of 8 for MPOBERT in the main results.

\paratitle{Analysis of Linguistic Patterns}.
% We aim to compare the difference in linguistic patterns captured by MPOBERT, BERT and ALBERT. 
% Following~\citet{Tinny2019probe}, we use a suite of probing tasks to quantify where specific types of linguistic information are encoded. Specifically, the probing tasks are divided into three categories to probe surface, syntactic and semantic information, respectively. 
% That is the representations that encode more information will exhibit better-probing task performance.
% Fig.~\ref{fig:linguistic} shows a difference in the distribution of linguistic information captured by Transformer layers.
% We find that BERT encodes more local syntax in low layers while capturing more complex semantics in high layers. However, ALBERT does not exhibit such a monotonic increase. The topmost layer of ALBERT is the best while other layers perform evenly~(which we expected, since the weights in all layers are shared).
% Different from all-layer sharing in ALBERT, MPOBERT keeps layer-wise parameters~(\ie the layer-wise auxiliary tensors and adapters) and shares information only in central tensors. Thus, we observe not only similar layer-wise behavior with BERT~(\ie task 0,2,4) but also better results for lower Transformer layers~(\ie task 3). 
% ---v2
To investigate the linguistic patterns captured by MPOBERT, BERT, and ALBERT, we conduct a suite of probing tasks, following the methodology of~\citet{Tinny2019probe}. These tasks are designed to evaluate the encoding of surface, syntactic, and semantic information in the models' representations. The results, shown in Fig.~\ref{fig:linguistic}, reveal that BERT encodes more local syntax in lower layers and more complex semantics in higher layers, while ALBERT does not exhibit such a clear trend.
% MPOBERT, on the other hand, uses layer-wise parameters and shares information only through central tensors. As a result, it exhibits similar layer-wise behavior to BERT in some tasks~(\ie task 0,2,4), and improved results in lower layers for others~(\ie task 3). Overall, these results demonstrate that MPOBERT captures linguistic information differently than other models, and its layer-wise parameters play an important role in this difference.
However, MPOBERT exhibits similar layer-wise behavior to BERT in some tasks~(\ie task 0,2,4), and improved results in lower layers for others~(\ie task 3) which is similar to ALBERT. The result demonstrates that MPOBERT captures linguistic information differently than other models, and its layer-wise parameters play an important role in this difference.
\section{Conclusion}
% We develop MPOBERT, a parameter-efficient pre-trained language model that allows for fewer parameters and efficient training.
% MPOBERT develops a novel mechanism with which we can easily scale BERT to deep models without increasing parameters or computational costs.
% During training, we propose initialization methods for both the central tensors and the auxiliary tensors based on our theoretical analysis to alleviate the training instability issue.
% We validate the effectiveness via supervised, few-shot and multitask experiments. 
% With fewer and less training costs, MPOBERT outperforms several competing models.
% --v2
We develop MPOBERT, a parameter-efficient pre-trained language model that allows for the efficient scaling of deep models without the need for additional parameters or computational resources. 
We achieve this by introducing an MPO-based Transformer layer and sharing the central tensors across layers. During training, we propose initialization methods for the central and auxiliary tensors, which are based on theoretical analysis to address training stability issues. 
The effectiveness of MPOBERT is demonstrated through various experiments, such as supervised, multitasking, and few-shot where it consistently outperforms other competing models.

\section*{Limitations}
% ACL 2023 requires all submissions to have a section titled ``Limitations'', for discussing the limitations of the paper as a complement to the discussion of strengths in the main text. This section should occur after the conclusion, but before the references.
The results presented in our study are limited by some natural language processing tasks and datasets that are evaluated, and further research is needed to fully understand the interpretability and robustness of our MPOBERT models. Additionally, there is subjectivity in the selection of downstream tasks and datasets, despite our use of widely recognized categorizations from the literature. 
% Due to computational constraints, the scaling performance of our method at deeper model depth~(\ie 96 layers or deeper) will be studied for our future work.
Furthermore, the computational constraints limited our ability to study the scaling performance of the MPOBERT model at deeper depths such as 96 layers or more. This is an area for future research.
\section*{Ethics Statement}
% Scientific work published at ACL 2023 must comply with the ACL Ethics Policy.\footnote{\url{https://www.aclweb.org/portal/content/acl-code-ethics}} We encourage all authors to include an explicit ethics statement on the broader impact of the work, or other ethical considerations after the conclusion but before the references. The ethics statement will not count toward the page limit (8 pages for long, 4 pages for short papers).
The use of a large corpus for training large language models may raise ethical concerns, particularly regarding the potential for bias in the data. In our study, we take precautions to minimize this issue by utilizing only standard training data sources, such as BOOKCORPUS and Wikipedia, which are widely used in language model training~\cite{devlin2018bert,lan2019albert}. However, it is important to note that when applying our method to other datasets, the potential bias must be carefully considered and addressed. Further investigation and attention should be given to this issue in future studies.

% Entries for the entire Anthology, followed by custom entries
\bibliography{anthology,custom}
\bibliographystyle{acl_natbib}

\appendix
\clearpage
\section{Appendix}
\label{sec:appendix}
\subsection{Proofs}
\label{app:proof}
\paragraph{Notations.} We denote $\mathcal{L}(\cdot)$ as the loss function. $LN(x)$ as the standard layer normalization with scale $\gamma=1$ and bias $\beta =0$. Let $\mathcal{O}(\cdot)$ denote standard Big-O notation that suppresses multiplicative constants. $\overset{\Theta}{=} $ stands for equal bound of magnitude. 
We aim to study the magnitude of the model updates. We define the model update as $\left\| \bigtriangleup F\right\|$.
% \begin{definition}
%     $F(x,\theta)$ is updated by $\Theta(\eta)$ per SGD step after initialization as $\eta\to 0$. That is, $\left\| \bigtriangleup F(x)\right\|=\Theta(\eta)$ where $\bigtriangleup F(x)$ can be calculated through $F(x,\theta-\eta\frac{\partial}{\partial \theta}\mathcal{L}(F(x)-y))-F(x;\theta)$.
% \end{definition}

\paratitle{Definition}
    $F(x,\theta)$ is updated by $\Theta(\eta)$ per SGD step after initialization as $\eta\to 0$. That is, $\left\| \bigtriangleup F(x)\right\|=\Theta(\eta)$ where $\bigtriangleup F(x)$ can be calculated through $F(x,\theta-\eta\frac{\partial}{\partial \theta}\mathcal{L}(F(x)-y))-F(x;\theta)$.

\begin{theorem}
    Given an $N$-layer transformer-based model $F(x,\theta)(\theta=\{\theta_1, \theta_2, ...,\theta_N\})$, where $\theta_l$ denotes the parameters in $l$-th layer and each sub-layer is normalized with Post-LN: $x_{l+1}=LN(x_l+G_l(x_l,\theta_l))$. In MPOBERT, $\theta_l$ is decomposed by MPO to local tensors: $\theta_l=u_l\cdot c_l\cdot v_l$, and we share $\{c_i\}_{i=1}^{N}$ across $N$ layers: $c_l=c_1, l=1,2,\cdots,N$. Then $\left\| \bigtriangleup F\right\|$ satisfies:
    \begin{align}
    \left\| \bigtriangleup F\right\|\leq
    % &\sum_{i=1}^{N}\frac{1-u_ic_1v_i}{(1+u_i^2c_1^2v_i^2)^{\frac{3}{2}}}(c_1v_i\left\|u_i^*-u_i \right\| \nonumber\\
    % &+ c_1u_i\left\|v_i^*-v_i \right\| + u_iv_i\left\|c_1^*-c_1 \right\|)
    &\sum_{i=1}^{N}(c_1v_i \left\|u_{i}^*-u_{i}\right\|+ c_1u_i \left\|v_{i}^*-v_{i}\right\| \nonumber\\
    &+ v_iu_i \left\|c_1^*-c_1\right\|)
    \end{align}
    \label{app-thm1}
\end{theorem}
$Proof.$ 
We follow~\cite{zhang2019fixup} and make the following assumptions to simplify the derivations:
\begin{enumerate}
    \item Hidden dimension $d$ equals to $1$;
    \item $var(x+G_l(x))\overset{\Theta}{=}var(x)+var(G_l(x))$;
    \item All relevant weights $\theta$ are positive with magnitude less than $1$.
\end{enumerate}
Given Assumption 1, if $G_l(x)$ is MLP with the weight $\theta_l$, then $G_l(x)\overset{\Theta}{=}\theta_l x$. With assumption 2, we have:
\begin{align}
    x_{l+1}&=f_l(x_l, \theta_l)=\frac{x+G_l(x)}{\sqrt{Var(x+G_l(x))}}\\
    &\overset{\Theta}{=}\frac{1+\theta_l}{\sqrt{1+\theta_l^2}}x_l,
    \label{eq:base}
\end{align}
Then, with Taylor expansion, the model update $\left\|\bigtriangleup F\right\|$satisfies:
\begin{align}
    \left\|\bigtriangleup F\right\|=&\left \|F(x,\theta^*)-F(x, \theta\right \|\nonumber\\
    =&\left\|x_{N+1}^*-x_{N+1}\right\| \nonumber\\
    =&\left\|f(x_{N}^*,\theta_{N}^*) -f(x_{N},\theta_{N})\right\|\nonumber\\
    =&\left \| f(x_{N}^*, U_{N}^*,C_{N}^*,V_{N}^*)\nonumber\right.\\
    &\left.-f(x_N, U_{N},C_{N},V_{N}) \right \| \nonumber\\
    \approx&\left \|\frac{\partial f }{\partial x}(x_{N}^*-x_{N})\nonumber\right.\\
    &\left.+\frac{\partial f}{\partial \theta }\frac{\partial\theta}{\partial U_{N}}(U_{N}^*-U_{N})^T \nonumber\right.\\
    &\left.+\frac{\partial f}{\partial \theta }\frac{\partial\theta}{\partial C_{N}}(C_{N}^*-C_{N})^T\nonumber\right.\\
    &\left.+\frac{\partial f}{\partial \theta }\frac{\partial\theta}{\partial V_{N}}(V_{N}^*-V_{N})^T  \right \|
    \label{eq:9}
\end{align}
With Eq.~\eqref{eq:base}, the magnitude of $\frac{\partial f_l}{\partial x}$ and $\frac{\partial f_l}{\partial \theta}$ is bounded by:
\begin{align}
    & \frac{\partial f_l}{\partial x}\overset{\Theta}{=}\frac{1+\theta_l}{\sqrt{1+\theta_l^2}} \\
    & \frac{\partial f_l}{\partial \theta_l}\overset{\Theta}{=}\frac{1-\theta_l}{(1+\theta_l^2)^{\frac{3}{2}}}x_l
\end{align}
Since we apply MPO decomposition to $\theta_l$, we get:
\begin{align}
    % &\theta_l=u_lc_lv_l
    \theta_l=U_l\cdot C_l \cdot V_l
\end{align}
For simplicity, we reduce the matrices $U$,$C$,$V$ to the scalars $u$,$c$,$v$. 
% In MPOBERT, we share $\{c_i\}_{i=1}^{N}$ across $N$ layers, so it goes to $c_l=c_1$. Considering that $c_1$ is initialized with well-trained parameters, the magnitude of the term $(c_N^*-c_N)$ is negligible compared with others. 
Thus with Assumption 3, Eq.~\eqref{eq:9} is reformulated as:
Finally, with Assumption 3 we have:
\begin{align}
    \left\|\bigtriangleup F \right\|=
    &\left\| x_{N+1}^*-x_{N+1} \right\| \\
    \leq &\sum_{i=1}^{N}\frac{1-u_ic_1v_i}{({{1+u_{i}^2c_1^2v_{i}^2}})^{\frac{3}{2}}}(c_1v_i \left\|u_{i}^*-u_{i}\right\|\nonumber\\
    &+ c_1u_i \left\|v_{i}^*-v_{i}\right\|) + v_iu_i \left\|c_1^*-c_1\right\|) \nonumber\\
    \approx&\sum_{i=1}^{N}(c_1v_i \left\|u_{i}^*-u_{i}\right\|+ c_1u_i \left\|v_{i}^*-v_{i}\right\| \nonumber\\
    &+ v_iu_i \left\|c_1^*-c_1\right\|)
\end{align}
\rightline{$\Box$}
\begin{corollary}
\label{app-thm2}
    % In $N$-layer MPOBERT, we assume that $\left\| \frac{\partial F}{\partial c_1} \right\|=\mathcal{O}(1)$, \ie the gradient signal of $c_1$ from the layers above is bounded, 
    Given that we initialise $c_1$ in MPOBERT with well-trained weights, it is reasonable to assume that updates of $c_1$ are well-bounded.
    Then $\bigtriangleup F$ satisfies $\left\| \bigtriangleup F\right\|=\mathcal{O}(1)$ when for all $i=1,\cdots,N$:
    \begin{equation}
        (v_i^2+u_i^2)(u_Nv_N)=\mathcal{O}(\frac{1}{N})
    \end{equation}
\end{corollary}

$Proof.$ 
For an $N$-layer MPOBERT, we have:
\begin{align}
    \left \|\bigtriangleup F\right \|
    \leq &\sum_{i=1}^{N}(v_i \left\|u_{i}^*-u_{i}\right\|+{u_i \left\|v_{i}^*-v_{i}\right\|}) \\
    \leq &\eta\sum_{i=1}^{N}(v_i \left\|\frac{\partial \mathcal{L}}{\partial F}\right\|\cdot \left\|\frac{\partial F}{\partial \theta_i}\right\|\cdot \left\|\frac{\partial \theta_i}{\partial u_i}\right\|\nonumber\\
    &+{u_i \left\|\frac{\partial \mathcal{L}}{\partial F}\right\|\cdot \left\|\frac{\partial F}{\partial \theta_i}\right\|\cdot\left\|\frac{\partial \theta_i}{\partial v_i}\right\|})
\end{align}
By assumption $\left\| \frac{\partial \mathcal{L}}{\partial F}\right\|=\mathcal{O}(1)$ and $\left\|\frac{\partial F}{\partial {\theta_i}} \right\|\leq\left\| \frac{\partial F}{\partial \theta_N} \right\|\overset{\Theta}{=}\left\| \theta_{N}\right\|$, we achieve:
\begin{align}
    % &{\eta\sum_{i=1}^{N}\frac{1-u_iv_i}{({{1+u_{i}^2v_{i}^2}})^{\frac{3}{2}}}(v_i \left\|\frac{\partial \mathcal{L}}{\partial F}\right\|\cdot \left\|\frac{\partial F}{\partial u_i}\right\|}+{u_i \left\|\frac{\partial \mathcal{L}}{\partial F}\right\|\cdot \left\|\frac{\partial F}{\partial v_i}\right\|})\\
    &\eta\sum_{i=1}^{N}(v_i \left\|\frac{\partial \mathcal{L}}{\partial F}\right\|\cdot \left\|\frac{\partial F}{\partial \theta_i}\right\|\cdot \left\|\frac{\partial \theta_i}{\partial u_i}\right\|\nonumber\\
    &+{u_i \left\|\frac{\partial \mathcal{L}}{\partial F}\right\|\cdot \left\|\frac{\partial F}{\partial \theta_i}\right\|\cdot\left\|\frac{\partial \theta_i}{\partial v_i}\right\|})\\
    =&\eta\sum_{i=1}^{N}(v_i^2u_Nv_N+u_i^2u_Nv_N) \nonumber\\
    = &\mathcal{O}(\sum_{i=1}^{N}(v_i^2+u_i^2)(u_Nv_N))=\mathcal{O}(1),
\end{align} 
\label{eq:bound}
Finally, we achieve:
\begin{equation}
    (v_i^2+u_i^2)(u_Nv_N)=\mathcal{O}(\frac{1}{N})
\end{equation}

Due to symmetry, we set $u_i=u$, $v_i=v$. Thus, from~\ref{eq:bound}, we set $u=v=(2N)^{-\frac{1}{4}}$ to achieve to bound the magnitudes of each update to be independent of model depth $N$, \ie $\left\| \bigtriangleup F\right\|=\mathcal{O}(1)$.
\rightline{$\Box$}

\begin{algorithm}[htb]
    \caption{The MPOBERT training procedure.}
    \begin{algorithmic}[1] %每行显示行号
    \small
        % \Require  $\Matrix{W}$: initial pre-trained weight 
        \Require $\Matrix{W}^{(l)}$: Weight matrix of $l$-th layer in MPOBERT.
        $\Matrix{W}_{A}^{(0)}$: Pre-trained weight matrix in ALBERT.
        % $\Matrix{W}_{Adapter}^{(l)}$: Low rank adapter containing $\Matrix{U}^{(l)}$ and $\Matrix{D}^{(l)}$.
        $\Matrix{U}^{(l)}$ and $\Matrix{D}^{(l)}$: Matrices in low-rank adapter.
        $\eta$: Learning rate.
        $\mathcal{L}$: Stochastic objection function.
        $L$: Model layers number.
        % \Require time step $t\gets 0$~(Initialize timestep)
        \Statex (MPO decomposition)
        \State
        $\{\Tensor{A}_1^{(l)},\Tensor{A}_2^{(l)},\Tensor{C}^{(l)},\Tensor{A}_3^{(l)},\Tensor{A}_4^{(l)}\}$ $\gets$ MPO ($\Matrix{W}^{(l)}$)
        \State
        $\{\Tensor{A}_1^{(0)},\Tensor{A}_2^{(0)},\Tensor{C}^{(0)},\Tensor{A}_3^{(0)},\Tensor{A}_4^{(0)}\}$ $\gets$ MPO ($\Matrix{W}_{A}^{(0)}$)
        \Statex (Initialization Procedure)
        \For {$0<l\leq 24$}
            \State  $\Tensor{C}^{(l)} \gets \Tensor{C}^{(0)} , 
             \{\Tensor{A}_{j}^{(l)}\}_{j=1}^{4} \gets \{\Tensor{A}_{j}^{(0)}\}_{j=1}^{4}$ 
        \EndFor
        \For {$ 24<l\leq L$}
            \State  $\Tensor{C}^{(l)} \gets \Tensor{C}^{(0)} ,    \{\Tensor{A}_{j}^{(l)}\}_{j=1}^{4} \gets \{(2L)^{-\frac{1}{4}}\Tensor{A}_{j}^{(0)}\}_{j=1}^{4}$ 
        \EndFor
        \State $\Matrix{U}^{(l)} \gets \Matrix{0}$, $\Matrix{D}^{(l)} \gets \mathcal{N}(0, \sigma^2)$
        \State $\Matrix{W}^{(l)}=\Tensor{A}_1^{(l)}\Tensor{A}_2^{(l)}\Tensor{C}^{(l)}\Tensor{A}_3^{(l)}\Tensor{A}_4^{(l)}+\Matrix{W}_{Adapter}^{(l)}$
        \Statex (Training procedure with mixed precision and fused implementation techniques.)
        \While {not converged}
            \State $t \gets t+1$
            \State $g_t \gets \frac{\partial\mathcal{L}(\Matrix{W}^{(l)}_t)}{\partial(\Matrix{W}^{(l)}_t)}$
            \State $\Matrix{W}^{(l)}_t \gets \Matrix{W}^{(l)}_{t-1} - \eta \cdot g_t$
        \EndWhile
        \State \Return Converged model
    \end{algorithmic}
\label{alg-overall-process}
\end{algorithm}
\subsection{Training Details}
\label{add-trianing-detail}

\subsubsection{Details of Training}
Here we describe the details of the pre-training process in Algorithm~\ref{alg-overall-process}. 
For pre-training, we tune the learning rate in the range of [$1.0\times 10^{-5}$, $1.0\times 10^{-6}$] and use the LAMB optimizer~\cite{you2020lamb}. Since fine-tuning is typically fast, we run an exhaustive parameter search~(\ie learning rate in the range of [$2.0\times 10^{-4}$, $2.0\times 10^{-6}$], batch size in \{8,16,32\}) and choose the model that performs best on the development set to make predictions on the test set. 


\subsubsection{Details of Training Configurations}
In this part, we list the training configurations of MPOBERT and other representative PLMs in Table~\ref{tab-strongest_variants}.
\begin{table*}[t]
\centering
\begin{tabular}{lrrrcrrr}                       
\toprule
Models    & \#To~(M) & Depth & Samples  & Training time  & GLUR Dev.  &GLUE Test      \\ \midrule
T5$\rm{_{11B}}$        & 11000  & 24     & -   & -    & - & 89.0       \\  \\
T5$\rm{_{BASE}}$        & 220  & 24     & 128$\times$ 524$k$   & \multirow{2}{*}{\thead{16 TPU v3\\ 1 Day~(t5-base)}}  & 84.1    & 82.5       \\  \\
BERT$\rm{_{LARGE}}$      & 330     & 24     & 256$\times$ 1000$k$   & \multirow{2}{*}{\thead{16 Cloud TPUs\\ 4 Days}}  & 84.1    & 81.6       \\  \\
ALBERT$\rm{_{XXLARGE}}$    & 235     & 1     & 4096$\times$ 1.5$M$   & \multirow{2}{*}{\thead{TPU v3\\ 16 Days}}  & 90.0    & -       \\  \\
BART$\rm{_{LARGE}}$      & 407     & 24     & 8000$\times$ 500$k$   & -  & 88.8    & -       \\  \\
RoBERTa$\rm{_{LARGE}}$   & 355     & 24     & 8000$\times$ 500$k$   & \multirow{2}{*}{\thead{1024 V100 GPUs\\ 1 Day}}  & 88.9    & -       \\  \\
XLNet$\rm{_{LARGE}}$     & 361     & 24     & 8192$\times$ 500$k$   & \multirow{2}{*}{\thead{512 TPU v3\\ 5.5 Days}}  & 87.4    & -       \\  \\
MPOBERT$_{48+}$   & 102     & 48    & 4096$\times$ 10$k$    & \multirow{2}{*}{\thead{8 V100 GPUs\\ 3.8 Days}}    & 85.6  & 81.7  \\ \\ \bottomrule
\end{tabular}
\caption{Comparison with the strongest variants of popular PLMs. Since T5$\rm{_{11B}}$ has far more parameters than other candidates, it's reasonable to use T5$\rm{_{base}}$ for a fair comparison.}
\label{tab-strongest_variants}
\end{table*}


\subsection{Experimental Details}
\subsubsection{Details of Fine-tuning Datasets}
\label{add-detail_dataset}
GLUE benchmark covers multiple datasets~(MNLI, QNLI, QQP, CoLA, RTE, MRPC, SST-2)~\footnote{In line with~\citet{raffel2020exploring}, we do not test WNLI due to its adversarial character with respect to the training set.}. 
The SQuAD is a collection of 100$k$ crowd-sourced question/answer pairs. Given a question and a passage, the task is to predict the answer text span in the passage. 

\subsubsection{Details of Evaluation Metrics}
\label{add-detail_metric}
Following~\citet{gao2022parameter}, we employ Matthew's correlation for CoLA, Spearman for STS-B, F1 for MRPC, and accuracy for the remaining tasks as the metrics for the GLUE benchmark.
We compute and present the average scores across all test samples for each of the aforementioned metrics.

\subsubsection{Details of Baseline Models}
\label{add-detail_baseline}
We compare our proposed MPOBERT to the existing competitive deep PLMs and parameter-efficient models. In order to make fair comparisons, we divide the models into three major categories based on their model sizes: 
$\bullet$~{Tiny Models~(\rm{\#To < 50M}).} ALBERT$_{12}$~\cite{lan2019albert} is the most representative PLM that achieves competitive results with only 11M.

$\bullet$~{Small models~(50M< \#To <100M).} T5$_{12}$ is a small variant of T5~\cite{raffel2020exploring} which has only 6 encoder layers and 6 decoder layers. In addition, there are three parameter-efficient Transformer models that have similar parameters, namely MobileBERT~\citep{sun2020mobilebert}, DistilBERT~\citep{sanh2019distilbert} and TinyBERT~\citep{jiao2019tinybert}. We compare with these compressed models to show the benefit of scaling to deeper models over compressing large models to small variants.

$\bullet$~{Base models~(\#To > 100M).} We compare with BERT$_{12}$, XLNet$_{12}$, RoBERTa$_{12}$ and BART$_{12}$ for this category. Note that we only include the base variants that have similar model sizes in order to make a fair comparison. More details about the comparison with the strongest variants are described in Appendix~\ref{app-exp}.
% $\bullet$~\underline{Parameter-efficient Models.} We compare with parameter-efficient models based on compression techniques including MobileBERT~\citep{sun2020mobilebert}, DistilBERT~\citep{sanh2019distilbert} and TinyBERT~\citep{jiao2019tinybert}.

\label{app-exp}



\end{document}
