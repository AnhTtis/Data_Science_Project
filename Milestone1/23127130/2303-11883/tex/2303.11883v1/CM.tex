\documentclass[times,doublespace]{oupau}
% % \usepackage[latin1]{inputenc}
\usepackage[british]{babel}
\usepackage[all]{xy}
\usepackage{amscd}
\usepackage{amssymb}
\usepackage{amsthm}
\usepackage{enumitem}
\usepackage{mathrsfs,bbm}
\usepackage{xcolor,graphicx}
\usepackage{graphics}
\usepackage{soul}
\usepackage{comment}
\usepackage[all]{xy}
\usepackage{amscd}
\usepackage{amssymb,amsmath,latexsym}
\usepackage{amsthm}
\usepackage{enumitem}
\usepackage{mathrsfs,bbm}
\usepackage{dsfont}
\usepackage{tikz-cd}
\usepackage[T1]{fontenc}
\usepackage[utf8]{inputenc}  
 %
%%%%%%%%%%%%%%%%%%%%%%%%%%%%%%%%%%
%pagestyle
%%%%%%%%%%%%%%%%%%%%%%%%%%%%%%%%%%
%\pagestyle{plain}
\textwidth=430pt
\headsep=.7cm
\evensidemargin=15pt
\oddsidemargin=15pt
\leftmargin=0cm
\rightmargin=0cm
%%
%%%%%%%%%%%%%%%%%%%%%%%
\newcommand*\fixitem {\item[]%
  \refstepcounter{enumi}\hskip-\leftmargin\labelenumi\hskip\labelsep}
\newtheorem*{mainthm}{Main Theorem}
\newtheorem*{mainthm1}{Theorem}
\newtheorem*{maincor}{Corollary}
\usepackage[colorlinks=true]{hyperref}
\DeclareMathOperator{\Forall}{\forall}
\DeclareMathOperator{\Exists}{\exists}
\DeclareMathOperator{\ord}{ord}
\newcommand{\phiD}{\varphi_D}
\newcommand{\phiDI}{\varphi_{\mathbf{D}_I}}
\newcommand{\phiDIj}{\varphi_{\mathbf{D}_I (j)}}
\newcommand{\phiH}{\varphi_H}
\newcommand{\phiTimes}{\phiD \otimes \phiH}
\newcommand{\phiTimesDI}{\varphi_{\mathbf{D}_I} \otimes \phiH}
\newcommand{\R}{\mathscr{A}}
\newcommand{\X}{\mathscr{X}}
\newcommand{\Xf}{\mathscr{X}_{(k_0 ,i)}[r_0]}
\newcommand{\Xfr}{\mathscr{X}_{(k_0,i)}[r]}
\newcommand{\hotimes}{\widehat{\otimes}}
\newcommand{\C}{\mathbb{C}_p}
\newcommand{\V}{\mathscr{V}}
\newcommand{\B}{\mathscr{B}}
\newcommand{\dualD}{\mathfrak{D}}
\newcommand{\Dg}{\mathbf{D}}
\newcommand{\DD}{\mathcal{D}^0}
\newcommand{\DDg}{\mathcal{D}}
\newcommand{\DV}{\mathcal{D}}
\newcommand{\W}{\mathscr{W}_N}
\newcommand{\Ao}{\mathbf{A}^\circ}
\newcommand{\AoK}{\mathbf{A}^\circ_{\K}}
\newcommand{\AK}{\mathbf{A}_{/\K}}
\newcommand{\OOO}{\mathscr{A}^\circ}
\newcommand{\K}{\mathcal{K}} 
\newcommand{\OK}{\mathcal{O}_{\K}}
\newcommand{\varprojlog}[1]{\underleftarrow{\log\!^{#1}}}
\newcommand{\T}{\mathscr{T}}
\newcommand{\TT}{\mathbf{T}}
\newcommand{\VV}{\mathbf{V}}
\newcommand{\HH}{\mathcal{H}}
\newcommand{\hh}{\mathcal{H}^+}
\newcommand{\HG}[2]{\mathcal{H}_{#1}(#2)}
\newcommand{\hhl}{\mathcal{H}^{+,[l]}}
\newcommand{\hhj}{\mathcal{H}^{+,[j]}}
\newcommand{\hhjj}{\mathcal{H}^{+,[l,l']}}
\newcommand{\GS}{G_{\mathbb{Q},S}}
\newcommand{\Rf}{R_{(k_0 ,i)}[r_0]}
\newcommand{\Rfr}{R_{(k_0 ,i)}[r]}
\newcommand{\parT}{\langle T\rangle}
\newcommand{\Zf}{Z_{(k_0 ,i)}[r_0]}
\newcommand{\Zfr}{\mathscr{Z}_{(k_0 ,i)}[r]}
\newcommand{\ZFf}{\mathscr{Z}_{(k_0 ,i)}[r_0]}
\newcommand{\ZFfr}{\mathscr{Z}_{(k_0 ,i)}[r]}
\newcommand{\ZF}{\mathscr{Z}}

 %\usepackage{tgpagella,euler}
 \usepackage{tikz-cd}
 \usetikzlibrary{graphs,decorations.pathmorphing,decorations.markings}
 \usepackage{amsmath}
% \usepackage{amssymb}

\newcommand\hmmax{0}
\newcommand\bmmax{0}
\usepackage{bm}
% \usepackage{textcomp}

% \usepackage[numbers]{natbib}

\usepackage{pifont}
% To use mathlcal
\usepackage[charter]{mathdesign}
\DeclareFontFamily{U}{dutchcal}{\skewchar\font=45 }
\DeclareFontShape{U}{dutchcal}{m}{n}{<-> s*[1.0] dutchcal-r}{}
\DeclareFontShape{U}{dutchcal}{b}{n}{<-> s*[1.0] dutchcal-b}{}
\DeclareMathAlphabet{\mathlcal}{U}{dutchcal}{m}{n}
\SetMathAlphabet{\mathlcal}{bold}{U}{dutchcal}{b}{n}
 

\DeclareMathAlphabet{\mathpzc}{OT1}{pzc}{m}{it}



% we still want to use ordinary \mathcal.
\DeclareSymbolFont{cmcal}{OMS}{cmsy}{m}{n}
\SetSymbolFont{cmcal}{bold}{OMS}{cmsy}{b}{n}
\DeclareSymbolFontAlphabet{\mathcal}{cmcal}



% \usepackage{tikz}
% \usepackage{tikz-cd}
% \usetikzlibrary{calc,decorations.pathmorphing,shapes}
% \usetikzlibrary{arrows,arrows.meta} %preamble
% \tikzcdset{arrow style=tikz, diagrams={>=stealth}}

%%%%%%%%%%%%%%%%%%%%%%THIS PAPER%%%%%%%%%%%%%%%%%%%%%%%%%%%%%%%%%%%%%%%%%%%%%%%%%%%%

% \textheight=8in
% \textwidth=6.3in
% \voffset=.6in
% \hoffset=.00in
% \voffset=.6in
% \hoffset=.5in

% \renewcommand{\baselinestretch}{1.3}
% \setcounter{tocdepth}{2}

% \newcommand{\functor}[1]{\mathbf{#1}}
% \newcommand{\dgcat}[1]{\overline{\category{#1}}\,}
% \newcommand{\gdcat}[1]{\underline{\category{#1}}\,}
% \newcommand{\category}[1]{\emph{\textbf{#1}}\,}

% \def\HOM{\HOM}
% \def\THOM{\category{THom}}

% %\DeclareMathOperator{\Hom}{Hom}
% \newcommand{\HOM}{\hbox{$\mathop{\bm{\Hom}}$}}
% \newcommand{\END}{\hbox{$\mathop{\bm{\End}}$}}

% \DeclareMathOperator{\THom}{THom}
% \newcommand{\THOM}{\hbox{$\mathop{\bm{\THom}}$}}






% \def\gh{\mathit{gh}}
% \def\wt{\mathit{wt}}
% \def\fm{\mathit{fm}}
% \def\cl{\mathit{cl}}
% \def\dR{\mathit{dR}}
% \def\co{\mathit{co}}
% \def\al{\mathit{al}}
% \def\ho{\mathit{ho}}
% \def\un{\mathit{un}}




% \newcommand{\QFTS}{{\mathit{QFTS}_{\hbar}}}
% \newcommand{\QFTA}{{\mathit{QFTA}_{\hbar}}}
% \def\QFTSS{\mathit{QFTS}}
% \def\QFTAA{\mathit{QFTA}}

% \def\dbullet{\hbox{\scriptsize\ding{70}}}


% \def\kbar{{\color{blue}\hbar}}
% \def\kkbar{[\![\kbar]\!]}
% \def\lkbar{[\![\kbar,\kbar^{-1}]\!]}
% \def\llbar{(\!(\kbar)\!)}
% \def\lbar{[\![\kbar^{-1}]\!]}


% \newcommand{\bell}{%
%   \mathbin{{\bm{\uell}}\mkern-12.6mu{\bm{\uell}}}%
% }

% \newcommand{\ttriangle}{%
%   \mathbin{\triangle\mkern-26.6mu\vartriangle}%
%   }

% \newcommand{\tttriangle}{%
%   \mathbin{\triangle\mkern-22mu\vartriangle}%
%   }


% \newcommand{\sbigotimes}{%
%   \mathop{\mathchoice{\textstyle\bigotimes}{\bigotimes}{\bigotimes}{\bigotimes}}%
% }\newcommand{\sbigwedge}{%
%   \mathop{\mathchoice{\textstyle\bigwedge}{\bigwedge}{\bigwedge}{\bigwedge}}%
% }


% \def\cp{{\mathrel\vartriangle}}
% \def\bcp{{\mathrel\blacktriangle}}

% \def\dga{{\category{dgA}(\Bbbk)}\!}
% \def\cdga{{\category{cdgA}(\Bbbk)}\!}
% \def\hcdga{{\mathit{ho}\category{cdgA}(\Bbbk)}\!}
% \def\ccdgc{{\category{ccdgC}(\Bbbk)}}
% \def\hccdgc{{\mathit{ho}\category{ccdgC}(\Bbbk)}\!}
% \def\cdgh{{\category{cdgH}(\Bbbk)}}
% \def\hcdgh{{\mathit{ho}\category{cdgH}(\Bbbk)}\!}
% \def\ccdgh{{\category{ccdgH}(\Bbbk)}}
% \def\hccdgh{{\mathit{ho}\category{ccdgH}(\Bbbk)}\!}


% \def\shh{{\hbox{\it\scriptsize\color{blue}shH}}}
% \def\tsh{{\hbox{\it\scriptsize\color{blue}tsH}}}

% \def\Betti{{\hbox{\it\scriptsize\color{blue}Betti}}}
% \def\deRham{{\hbox{\it\scriptsize\color{blue}de Rham}}}





% \def\ide{\mathpzc{e}}
% \def\asx{\!\star\!}


% \def\sEo{\bm{\sE}_{\!\!\bm{\check{\o}}}}


% \newcommand\coPi{\rotatebox[origin=c]{180}{$\bm{\Pi}$}}
% \newcommand\bmPi{\bm{\Pi}}
% \newcommand\copi{\rotatebox[origin=c]{180}{$\bm{\pi}$}}
% \newcommand\bmpi{\bm{\pi}}




%%%%%%%%%%%%%%%%%%%%%%BODY%%%%%%%%%%%%%%%%%%%%%%%%%%%%%%%%%%%%%%%%%%%%%%%%%%%%


% \def\cpT{\bm{\cp}_{\bm{\mT}}}
% \def\tsT{\bm{\otimes}_{\!\bm{\mT}}}
% \def\tsTpr{\bm{\otimes}_{\!\bm{\mT}^\pr}}
% \def\epTpr{\bm{\ep}_{\bm{\mT}^\pr}}
% \def\unT{{1}_{\bm{\mT}}}
% \def\unTpr{{1}_{\bm{\mT}^\pr}}
% \def\epT{\bm{\ep}_{\bm{\mT}}}


% \def\me{\mathfrak{e}}
% \def\tw{{\mathit{tw}}}
% \def\tst{{\bm{\otimes}}_{\!{}_{\category{T}}}}
% \def\unt{{\bm{1}}_{\!{}_{\category{T}}}}

% \def\gd{{{}^{\natural}\g}}
% \def\gdt{{{}^{\natural}\tilde{\g}}}

% \def\tsC{{\bm{\otimes}}_{\!{\cp_{\!C}}}}

% \def\Modop{\category{Mod}_{\!\!L}\!\big(\mb{\o}(\CP)\!\big)}
% \def\Modp{\category{Mod}_{\!\!L}\!(\sP\!)}
% \def\Repp{\category{Rep}_{\!\Bbbk}(\bm{\mP})}
% \def\Chk{\category{Ch}\!(\Bbbk)}

% \def\I{\mathbb{I}}

\DeclareMathOperator{\I}{\mathbb{I}}
 
\DeclareMathOperator{\Ens}{Ens}

\DeclareMathOperator{\Obj}{Obj}
\DeclareMathOperator{\Eq}{Eq}
\DeclareMathOperator{\Lax}{Lax}
\DeclareMathOperator{\cMon}{cMon}
\DeclareMathOperator{\ev}{ev}
\DeclareMathOperator{\id}{id}
\DeclareMathOperator{\Nat}{Nat}
\DeclareMathOperator{\op}{op}
\DeclareMathOperator{\Rep}{Rep}

\DeclareMathOperator{\Vect}{Vect}
\DeclareMathOperator{\Mod}{Mod}
\DeclareMathOperator{\Comod}{Comod}
\DeclareMathOperator{\Spec}{Spec}

\DeclareMathOperator{\Mnd}{Mnd}
\DeclareMathOperator{\Monad}{Monad}
\DeclareMathOperator{\Comonad}{Comonad}
\DeclareMathOperator{\Free}{Free}
\DeclareMathOperator{\Cofree}{Cofree}
\DeclareMathOperator{\Forget}{Forget}
\DeclareMathOperator{\Funct}{Funct}

\DeclareMathOperator{\Fib}{Fib}
\DeclareMathOperator{\Aff}{Aff}
\DeclareMathOperator{\fl}{fl}
\DeclareMathOperator{\Ind}{Ind}
\DeclareMathOperator{\Pro}{Pro}

\DeclareMathOperator{\Ex}{Ex}
\DeclareMathOperator{\pr}{\prime}


% \DeclareMathOperator{\lHom}{\mathlcal{H}\!\mathit{om}}
% \DeclareMathOperator{\End}{End}
% \DeclareMathOperator{\Aut}{Aut}
% \DeclareMathOperator{\Hom}{Hom}
\DeclareMathOperator{\Isom}{Isom}

\begin{document}

%
%\title{Enriched Morita theory of monoids in a closed symmetric monoidal category}
%%\title{Tensor enriched categorical generalization of the Eilenberg-Watts theorem}
%
%\author{
%Jaehyeok Lee
% \and
% Jae-Suk Park
%%\thanks{
%%}
%}
%
%\institute{
%Department of Mathematics, POSTECH, Pohang, Republic of Korea 
%}
%
%
%
%
%\maketitle
%\tableofcontents



% Enter full title and short title for running headers
% \title{A Demonstration of the \LaTeXe\ Class File for the \textit{Oxford University Press Ltd Journal}}
\title{Enriched Morita theory of monoids in a closed symmetric monoidal category}
\shorttitle{Enriched Morita theory of monoids in a closed symmetric monoidal category}

% Enter the publication year and the ID number of the paper
\volumeyear{2023}
\paperID{rnn999}

% Author name(s)
\author{Jaehyeok Lee\affil{1} and Jae-Suk Park\affil{2,*}}
% Abbreviated author name for running headers
\abbrevauthor{J. Lee and J.-S. Park}
% Abbreviated author name for first page header
\headabbrevauthor{J. Lee and J.-S. Park}

\address{%
\affilnum{1,2}37673, Department of Mathematics, POSTECH, Pohang, Republic of Korea}
% and
% \affilnum{2}Complete Second Author Address}

% Address / e-mail address of corresponding author
\correspdetails{jaesuk@postech.ac.kr}

% Received/revised/accepted dates will be entered by the publisher during production of an accepted paper. Please do not edit these placeholders for submission.
\received{1 Month 20XX}
\revised{11 Month 20XX}
\accepted{21 Month 20XX}

% Enter details of editor communicating this article
\communicated{A. Editor}

\begin{abstract}
We develop Morita theory
of monoids in a closed symmetric monoidal category,
in the context of enriched category theory.
\end{abstract}

\maketitle









\section{Introduction}
We consider a closed symmetric monoidal category
$\mathlcal{C}=\bigl(\mathscr{C},\otimes,c,[-,-]\bigr)$
whose underlying category $\mathscr{C}$ is finite complete and finite cocomplete.
Some examples are the closed symmetric monoidal categories
$\mathlcal{Set}/\mathlcal{fSet}/\mathlcal{sSet}$ of small sets/finite sets/simplicial sets,
% $\mathlcal{Grp}$ of groups,
$\mathlcal{Cat}$ of small categories,
$\mathlcal{Ab}/\mathlcal{fAb}$ of abelian groups/finitely generated abelian groups,
$\mathlcal{Vec}_K/\mathlcal{fVec}_K$ of vector spaces/finite dimensional vector spaces over a field $K$,
$\mathlcal{Mod}_R$/$\widehat{\mathlcal{Mod}}_R$/$\mathlcal{dgMod}_R$
of modules/$L$-complete modules/dg-modules over a commutative ring $R$,
$\mathlcal{CGT}$/$\mathlcal{CGT}_{\!\!*}$ of unbased/based compactly generated topological spaces, 
$\mathlcal{S\!p}^{\Sigma}_{\mathit{CGT}_{\!*}}$ of topological symmetric spectra,
$\mathlcal{CGWH}$/$\mathlcal{CGWH}_{\!\!*}$ of unbased/based compactly generated weakly Hausdorff spaces,
and $\mathlcal{Ban}$ of Banach spaces with linear contractions.
Every elementary topos is also an example.

We first provide the following generalization of the Eilenberg-Watts theorem.
We denote enriched categorical structures with bold letters.
For each monoid $\mathfrak{b}$ in $\mathlcal{C}$,
we denote $\bm{\mathcal{M}}_{\mathfrak{b}}$ as the
$\mathlcal{C}$-enriched category of right $\mathfrak{b}$-modules.

\begin{theorem} \label{thm Intro EWthm}
  Let $\bm{T}$ be a tensored $\mathlcal{C}$-enriched category which has weighted coequalizers.
  For each monoid $\mathfrak{b}$ in $\mathlcal{C}$,
  we have an adjoint equivalence of categories
  \begin{equation*}
    \begin{tikzcd}[row sep=0]
      \text{ }\!\!_{\mathfrak{b}}T
      \ar[r,squiggly,"\simeq"]
      &
      {\mathlcal{C}\text{-}\mathit{Funct}_{\mathit{cocon}}(\bm{\mathcal{M}}_{\mathfrak{b}},\bm{T})}
      \\
      \text{ }\!\!_{\mathfrak{b}}X
      \ar[r,mapsto]
      &
      -\circledast_{\mathfrak{b}}\! \text{ }\!\!_{\mathfrak{b}}X:\bm{\mathcal{M}}_{\mathfrak{b}}\rightsquigarrow\bm{T}
    \end{tikzcd}
  \end{equation*}
  from the category
  % $\text{ }\!\!_{\mathfrak{b}}T$
  of left $\mathfrak{b}$-module objects $\text{ }\!\!_{\mathfrak{b}}X$ in $\bm{T}$
  to the category 
  % $\mathlcal{C}\text{-}\mathit{Funct}_{\mathit{cocon}}(\bm{\mathcal{M}}_{\mathfrak{b}},\bm{T})$
  of cocontinuous $\mathlcal{C}$-enriched functors 
  $\bm{\mathcal{F}}:\bm{\mathcal{M}}_{\mathfrak{b}}\rightsquigarrow\bm{T}$.
\end{theorem}

\begin{corollary} \label{cor Intro EWthm}
  Let $\mathfrak{b}$, $\mathfrak{b}'$ be monoids in $\mathlcal{C}$.
  We have an adjoint equivalence of categories
  \begin{equation*}
    \begin{tikzcd}[row sep=0]
      \text{ }\!\!_{\mathfrak{b}}\mathcal{M}_{\mathfrak{b}'}
      \ar[r,squiggly,"\simeq"]
      &
      {\mathlcal{C}\text{-}\mathit{Funct}_{\mathit{cocon}}(\bm{\mathcal{M}}_{\mathfrak{b}},\bm{\mathcal{M}}_{\mathfrak{b}'})}
      \\
      \text{ }\!\!_{\mathfrak{b}}x_{\mathfrak{b}'}
      \ar[r,mapsto]
      &
      -\circledast_{\mathfrak{b}}\! \text{ }\!\!_{\mathfrak{b}}x_{\mathfrak{b}'}
      :\bm{\mathcal{M}}_{\mathfrak{b}}\rightsquigarrow\bm{\mathcal{M}}_{\mathfrak{b}'}
    \end{tikzcd}
  \end{equation*}
  from the category
  % $\text{ }\!\!_{\mathfrak{b}}\mathcal{M}_{\mathfrak{b}'}$
  of $(\mathfrak{b},\mathfrak{b}')$-bimodules $\text{ }\!\!_{\mathfrak{b}}x_{\mathfrak{b}'}$
  to the category 
  % $\mathlcal{C}\text{-}\mathit{Funct}_{\mathit{cocon}}(\bm{\mathcal{M}}_{\mathfrak{b}},\bm{T})$
  of cocontinuous $\mathlcal{C}$-enriched functors 
  $\bm{\mathcal{F}}:\bm{\mathcal{M}}_{\mathfrak{b}}\rightsquigarrow\bm{\mathcal{M}}_{\mathfrak{b}'}$.
\end{corollary}
The Eilenberg-Watts theorem \cite{Eilenberg1960}, \cite{Watts1960}
states that the adjoint equivalence of categories
in Corollary~\ref{cor Intro EWthm} is essentially surjective when $\mathlcal{C}=\mathlcal{Ab}$.

We say monoids $\mathfrak{b}$, $\mathfrak{b}'$ in $\mathlcal{C}$ are \emph{Morita equivalent}
if we have an equivalence of $\mathlcal{C}$-enriched categories
$\bm{\mathcal{M}}_{\mathfrak{b}}\simeq\bm{\mathcal{M}}_{\mathfrak{b}'}$.
The main goal of this paper is to prove the following theorem.
We say an object $X$ in a $\mathlcal{C}$-enriched category $\bm{T}$
is a \emph{$\mathlcal{C}$-enriched compact generator} if 
the $\mathlcal{C}$-enriched functor $\bm{T}(X,-):\bm{T}\rightsquigarrow\bm{\mathscr{C}}$
is cocontinuous and conservative.

\begin{theorem} \label{thm Intro CosMorita}
  Let $\mathfrak{b}$, $\mathfrak{b}'$ be monoids in $\mathlcal{C}$.
  The following are equivalent:
  \begin{itemize}
    \item[(i)]
    The monoids $\mathfrak{b}$, $\mathfrak{b}'$ in $\mathlcal{C}$ are Morita equivalent.

    \item[(ii)]
    There exists a $\mathlcal{C}$-enriched compact generator $x_{\mathfrak{b}'}$ in $\bm{\mathcal{M}}_{\mathfrak{b}'}$
    together with an isomorphism of monoids $\mathfrak{b}\cong \mathit{End}_{\bm{\mathcal{M}}_{\mathfrak{b}'}}(x_{\mathfrak{b}'})$
    in $\mathlcal{C}$.

    \item[(iii)]
    There exists a $(\mathfrak{b},\mathfrak{b}')$-bimodule $\text{ }\!\!_{\mathfrak{b}}x_{\mathfrak{b}'}$
    and a $(\mathfrak{b}',\mathfrak{b})$-bimodule $\text{ }\!\!_{\mathfrak{b}'}y_{\mathfrak{b}}$
    together with isomorphisms of bimodules
    $\text{ }\!\!_{\mathfrak{b}}x_{\mathfrak{b}'}\circledast_{\mathfrak{b}'}\text{ }\!\!_{\mathfrak{b}'}y_{\mathfrak{b}}
    \cong\text{ }\!\!_{\mathfrak{b}}b_{\mathfrak{b}}$
    and
    $\text{ }\!\!_{\mathfrak{b}'}y_{\mathfrak{b}}\circledast_{\mathfrak{b}}\text{ }\!\!_{\mathfrak{b}}x_{\mathfrak{b}'}
    \cong\text{ }\!\!_{\mathfrak{b}'}b'\text{ }\!\!\!\!_{\mathfrak{b}'}$.
  \end{itemize}
\end{theorem}
When $\mathlcal{C}=\mathlcal{Ab}$, we recover the result of Morita \cite{Morita1958}.

We denote morphisms as $\to$,
functors as $\rightsquigarrow$
and natural transformations as $\Rightarrow$.
In section~\ref{sec CocomCcat},
we introduce $\mathlcal{C}$-enriched categories and explain basic notations.
In section~\ref{sec EW-thm},
we prove Theorem~\ref{thm EWthm cocomCcat}
which is a stronger statement than Theorem~\ref{thm Intro EWthm}.
In section~\ref{sec Morita},
we prove Theorem~\ref{thm Morita CharacterizingMb}
which characterize when a $\mathlcal{C}$-enriched category is equivalent to $\bm{\mathcal{M}}_{\mathfrak{b}}$.
We also give a proof of Theorem~\ref{thm Intro CosMorita}.








\section{Enriched Categories}\label{sec CocomCcat}
Let
$\mathlcal{C}=\bigl(\mathscr{C},\otimes,c,[-,-]\bigr)$
be a closed symmetric monoidal category whose underlying category
$\mathscr{C}$ is finite complete and finite cocomplete.
Let $z$, $x$, $y$ be objects in $\mathscr{C}$.
We have a functor $\otimes:\mathscr{C}\times\mathscr{C}\rightsquigarrow \mathscr{C}$
and an object $c$ in $\mathscr{C}$
together with coherence isomorphisms in $\mathscr{C}$
\begin{equation}\label{eq EnCat asij}
  \begin{aligned}
    a_{z,x,y}
    &:
    \begin{tikzcd}[cramped,column sep=20]
      z\!\otimes\! \bigl(x\!\otimes\! y\bigr)
      \ar[r,pos=0.4,"\cong"]
      &\bigl(z\!\otimes\! x\bigr)\!\otimes\! y
      ,
    \end{tikzcd}
    \\%1
    s_{x,y}
    &:
    \begin{tikzcd}[cramped,column sep=20]
      x\!\otimes\! y
      \ar[r,pos=0.4,"\cong"]
      &y\!\otimes\! x
      ,
    \end{tikzcd}
  \end{aligned}
  \qquad\quad
  \begin{aligned}
    \imath_x
    &:
    \begin{tikzcd}[cramped,column sep=20]
      c\!\otimes\! x
      \ar[r,pos=0.4,"\cong"]
      &x
      ,
    \end{tikzcd}
    \\%1
    \jmath_x
    &:
    \begin{tikzcd}[cramped,column sep=20]
      x\!\otimes\! c
      \ar[r,pos=0.4,"\cong"]
      &x
    \end{tikzcd}
  \end{aligned}
\end{equation}
natural in variables $z$, $x$ and $y$.
The functor $-\otimes x:\mathscr{C}\rightsquigarrow\mathscr{C}$
admits a right adjoint which we denote as $[x,-]:\mathscr{C}\rightsquigarrow\mathscr{C}$,
and we have a functor $[-,-]:\mathscr{C}^{\op}\times\mathscr{C}\rightsquigarrow\mathscr{C}$.

We refer \cite{Borceux1994}, \cite{Kelly2005} for the basics of enriched category theory.
Enriched categorical structures are denoted with bold letters.
We have the $\mathlcal{C}$-enriched category $\bm{\mathscr{C}}$
whose objects are those of $\mathscr{C}$,
and Hom objects are given by $\bm{\mathscr{C}}(x,y)=[x,y]$.
Given a $\mathlcal{C}$-enriched category $\bm{T}$ and $X$, $Y$, $Z\in\Obj(\bm{T})$,
we denote the identity, composition morphisms as
$\mathit{id}_X:c\to \bm{T}(X,X)$,
$\mu_{X,Y,Z}:\bm{T}(Y,Z)\otimes \bm{T}(X,Y)\to \bm{T}(X,Z)$.
A morphism in $\bm{T}$ means a morphism in the underlying category $T$ of $\bm{T}$.
For each morphism $l:X\to Y$ in $\bm{T}$, we have morphisms
$l_{\star}:\bm{T}(Z,X)\to\bm{T}(Z,Y)$, $l^{\star}:\bm{T}(Y,Z)\to\bm{T}(X,Z)$ in $\bm{\mathscr{C}}$.
We denote a $\mathlcal{C}$-enriched functor as
$\bm{\alpha}:\bm{T}\rightsquigarrow\bm{S}$,
where $\bm{\alpha}_{X,Y}:\bm{T}(X,Y)\to \bm{S}\bigl(\bm{\alpha}(X),\bm{\alpha}(Y)\bigr)$
is the morphism between Hom objects.
Given $\mathlcal{C}$-enriched functors
$\bm{\alpha}$, $\bm{\beta}:\bm{T}\rightsquigarrow\bm{S}$,
we denote a $\mathlcal{C}$-enriched natural transformation as
$\bm{\xi}:\bm{\alpha}\Rightarrow\bm{\beta}:\bm{T}\rightsquigarrow\bm{S}$.
The component of $\bm{\xi}$ at $X\in\Obj(\bm{T})$
is denoted by $\xi_X:\bm{\alpha}(X)\to\bm{\beta}(X)$,
which is not bolded.

A $\mathlcal{C}$-enriched category $\bm{T}$ is \emph{tensored} if for each $X\in\Obj(\bm{T})$,
the $\mathlcal{C}$-enriched functor $\bm{T}(X,-):\bm{T}\rightsquigarrow\bm{\mathscr{C}}$
admits a left adjoint $\mathlcal{C}$-enriched functor
$-\circledast X:\bm{\mathscr{C}}\rightsquigarrow\bm{T}$.
We denote the components of the unit, counit $\mathlcal{C}$-enriched natural transformations
at $z\in \Obj(\bm{\mathscr{C}})$, $Y\in \Obj(\bm{T})$ as
\begin{equation*}
  \begin{tikzcd}[cramped,column sep=45]
    \bm{\mathscr{C}}
    \ar[r,squiggly,shift left=1.5,bend left=15,"{-\text{ }\!\circledast X}"]
    &\bm{T}
    \ar[l,squiggly,shift left=1.5,bend left=15,"{\bm{T}(X,-)}"]
  \end{tikzcd}
  \qquad\qquad
  \mathit{Cv}_{z,X}:
  \begin{tikzcd}[cramped,column sep=16]
    z
    \ar[r]
    &\bm{T}\bigl(X,z\!\circledast\! X\bigr)    
  \end{tikzcd}
  ,
  \quad
  \mathit{Ev}_{X,Y}:
  \begin{tikzcd}[cramped,column sep=16]
    \bm{T}(X,Y)\!\circledast\! Y
    \ar[r]
    &X
  \end{tikzcd}
  .
\end{equation*}
For each morphism $l:z\circledast X\to Y$ in $\bm{T}$,
we denote $\bar{l}:z\to \bm{T}(X,Y)$ as the corresponding morphism in $\bm{\mathscr{C}}$.
We have a unique isomorphism
$\imath_X:c\circledast X\xrightarrow[]{\cong}X$ in $\bm{T}$
which corresponds to the morphism $\bar{\imath}_X=\mathit{id}_X:c\to \bm{T}(X,X)$ in $\bm{\mathscr{C}}$.

Let $\bm{\beta}:\bm{T}\rightsquigarrow\bm{S}$ be a $\mathlcal{C}$-enriched functor
between tensored $\mathlcal{C}$-enriched categories $\bm{T}$, $\bm{S}$.
Given $z\in\Obj(\bm{\mathscr{C}})$, $X\in\Obj(\bm{T})$,
we have the following morphism in $\bm{S}$
\begin{equation*}
  \begin{aligned}
    &t^{\bm{\beta}}_{z,X}
    =
    \mathit{Ev}_{\bm{\beta}(X),\bm{\beta}(z\circledast X)}
    \circ
    \bigl(\bm{\beta}_{X,z\circledast X}\circledast \I_{\bm{\beta}(X)}\bigr)
    \circ
    \bigl(\mathit{Cv}_{z,X}\circledast\I_{\bm{\beta}(X)}\bigr)
    \\
    &:
    \begin{tikzcd}[cramped,column sep=18]
      z\!\circledast\! \bm{\beta}(X) 
      \ar[r]
      &\bm{T}\bigl(X,z\!\circledast\! X\bigr)\!\circledast\! \bm{\beta}(X)
      \ar[r]
      &\bm{S}\Bigl(\bm{\beta}(X),\bm{\beta}\bigl(z\!\circledast\! X\bigr)\Bigr)\!\circledast\! \bm{\beta}(X)
      \ar[r]
      &\bm{\beta}\bigl(z\!\circledast\! X\bigr)
    \end{tikzcd}      
  \end{aligned}
\end{equation*}
which is called the \emph{tensorial strength associated to $\bm{\beta}$}.
We say $\bm{\beta}:\bm{T}\rightsquigarrow\bm{S}$ \emph{preserves tensored objects}
if the tensorial strength $t^{\bm{\beta}}_{z,X}$ is an isomorphism in $\bm{S}$
for every pair $z\in\Obj(\bm{\mathscr{C}})$, $X\in\Obj(\bm{T})$.
Given another tensored $\mathlcal{C}$-enriched category $\bm{S}'$
and a $\mathlcal{C}$-enriched functor
$\bm{\beta}':\bm{S}\rightsquigarrow \bm{S}'$,
the tensorial strength of $\bm{\beta}'\bm{\beta}:\bm{T}\rightsquigarrow\bm{S}'$
is given by
$t^{\bm{\beta}'\bm{\beta}}_{z,X}
=\bm{\beta}'(t^{\bm{\beta}}_{z,X})\circ t^{\bm{\beta}'}_{z,\bm{\beta}(X)}
:z\circledast \bm{\beta}'\bm{\beta}(X)
\to \bm{\beta}'\bigl(z\circledast \bm{\beta}(X)\bigr)
\to \bm{\beta}'\bm{\beta}\bigl(z\circledast X\bigr)$.

\begin{example}
  The $\mathlcal{C}$-enriched category $\bm{\mathscr{C}}$ is tensored.
  Let $z$, $x$, $y\in\Obj(\bm{\mathscr{C}})$.
  The tensored object of $z$, $x$ in $\bm{\mathscr{C}}$
  is given by $z\circledast x=z\otimes x$.
  The coherence isomorphism $\imath_x:c\otimes x\xrightarrow[]{\cong}x$ in (\ref{eq EnCat asij})
  corresponds to the unique isomorphism $\imath_x:c\circledast x\xrightarrow[]{\cong}x$ in $\bm{\mathscr{C}}$.
  The coherence isomorphism $a_{z,x,y}:z\otimes\bigl(x\otimes y\bigr)\xrightarrow[]{\cong}\bigl(z\otimes x\bigr)\otimes y$
  in (\ref{eq EnCat asij})
  corresponds to the tensorial strength
  $t^{-\circledast y}_{z,x}:z\circledast\bigl(x\circledast y\bigr)\xrightarrow[]{\cong}\bigl(z\circledast x\bigr)\circledast y$
  associated to the $\mathlcal{C}$-enriched functor $-\circledast y:\bm{\mathscr{C}}\rightsquigarrow\bm{\mathscr{C}}$.
  We are going to identify $\otimes$ and $\circledast$ throughout this paper.
  See \cite[\textsection 3]{Lee2023} for the details.
  For instance, we denote a monoid in $\mathlcal{C}$ as a triple
  $\mathfrak{b}=\bigl(b,u_b,m_b\bigr)$
  where $b$ is the underlying object in $\bm{\mathscr{C}}$
  and $u_b:c\to b$, $m_b:b\circledast b\to b$
  are the unit, product morphisms in $\bm{\mathscr{C}}$.    
\end{example}

Let $\bm{T}$ be a tensored $\mathlcal{C}$-enriched category
and let $w$, $z\in\Obj(\bm{\mathscr{C}})$, $X\in\Obj(\bm{T})$.
We denote the tensorial strength associated to the $\mathlcal{C}$-enriched functor
$-\circledast X:\bm{\mathscr{C}}\rightsquigarrow\bm{T}$ as
$a_{w,z,X}:=t^{-\circledast X}_{w,z}:
w\circledast \bigl(z\circledast X\bigr)
\xrightarrow[]{\cong}\bigl(w\circledast z\bigr)\circledast X$,
which is an isomorphism in $\bm{T}$.
We omit this isomorphism unless necessary
and denote $w\circledast z\circledast X\in\Obj(\bm{T})$.    

\begin{example}
  Let $\mathfrak{b}=\bigl(b, u_b, m_b\bigr)$ be a monoid in $\mathlcal{C}$.
  A \emph{right $\mathfrak{b}$-module}
  is a pair $z_{\mathfrak{b}}=(z,\gamma_z)$
  of $z\in\Obj(\bm{\mathscr{C}})$
  and a morphism 
  $\gamma_z:z\circledast b\to z$ in $\bm{\mathscr{C}}$
  satisfying the right $\mathfrak{b}$-action relations.
  For instance, we have the right $\mathfrak{b}$-module 
  $b_{\mathfrak{b}}:=\bigl(b,m_b:b\circledast b\to b\bigr)$.
  Let $\tilde{z}_{\mathfrak{b}}=(\tilde{z},\gamma_{\tilde{z}})$
  be another right $\mathfrak{b}$-module.
  We have the $\mathlcal{C}$-enriched category $\bm{\mathcal{M}}_{\mathfrak{b}}$
  of right $\mathfrak{b}$-modules,
  whose Hom object between $z_{\mathfrak{b}}$, $\tilde{z}_{\mathfrak{b}}$ is given by the following equalizer
  in the underlying category $\mathscr{C}$ of $\bm{\mathscr{C}}$.
  \begin{equation*}
    \begin{tikzcd}[cramped,column sep=50]
      \bm{\mathcal{M}}_{\mathfrak{b}}(z_{\mathfrak{b}},\tilde{z}_{\mathfrak{b}})
      \ar[r,hook,"\mathit{eq}_{z_{\mathfrak{b}},\tilde{z}_{\mathfrak{b}}}"]
      &\bm{\mathscr{C}}(z,\tilde{z})
      \ar[r,shift left=0.7,"{(\gamma_z)}^{\star}"]
      \ar[r,shift right=0.7,"{(\gamma_{\tilde{z}})_{\star}\circ (-\circledast b)_{z,\tilde{z}}}"']
      &\bm{\mathscr{C}}\bigl(z\!\circledast\! b,\tilde{z}\bigr)
    \end{tikzcd}
  \end{equation*}  
  We also have the forgetful $\mathlcal{C}$-enriched functor
  $\bm{\mathcal{M}}_{\mathfrak{b}}\rightsquigarrow\bm{\mathscr{C}}$
  whose morphism on Hom objects is given by
  $\mathit{eq}_{z_{\mathfrak{b}},\tilde{z}_{\mathfrak{b}}}
  :\bm{\mathcal{M}}_{\mathfrak{b}}(z,\tilde{z})\hookrightarrow \bm{\mathscr{C}}(z,\tilde{z})$.
  The $\mathlcal{C}$-enriched category $\bm{\mathcal{M}}_{\mathfrak{b}}$ is tensored:
  see \cite[\textsection 3]{Lee2023}.
  The tensored object of $w\in\Obj(\bm{\mathscr{C}})$, $z_{\mathfrak{b}}\in\Obj(\bm{\mathcal{M}}_{\mathfrak{b}})$
  is the pair
  $w\circledast z_{\mathfrak{b}}=\bigl(w\circledast z,\gamma_{w\circledast z}\bigr)$
  where 
  $\gamma_{w\circledast z}=\I_w\circledast\gamma_z:
  \begin{tikzcd}[cramped,column sep=16]
    w\circledast z\!\circledast\! b 
    \ar[r]
    &w\circledast z
  \end{tikzcd}$.
  The morphism $\gamma_z:z\circledast b\to z$ in $\bm{\mathscr{C}}$ becomes a morphism
  $\gamma_{z_{\mathfrak{b}}}:z\circledast b_{\mathfrak{b}}\to z_{\mathfrak{b}}$ in $\bm{\mathcal{M}}_{\mathfrak{b}}$,
  and we have an isomorphism
  $\bar{\gamma}_{z_{\mathfrak{b}}}:z\xrightarrow[]{\cong} \bm{\mathcal{M}}_{\mathfrak{b}}(b_{\mathfrak{b}},z_{\mathfrak{b}})$
  in $\bm{\mathscr{C}}$.
\end{example}

We refer \cite[\textsection 3.1]{Lee2023}
for the following properties of tensorial strengths.
Let $\bm{T}$, $\bm{S}$ be tensored $\mathlcal{C}$-enriched categories
and let $\bm{\beta}:\bm{T}\rightsquigarrow\bm{S}$ be a $\mathlcal{C}$-enriched functor.
The tensorial strength of $\bm{\beta}:\bm{T}\rightsquigarrow\bm{S}$ satisfies the relations below.
\begin{equation}\label{eq strength funct}
  \begin{tikzcd}[cramped,row sep=20,column sep=18]
    c\!\circledast\! \bm{\beta}(X)
    \ar[r,pos=0.33,"t^{\bm{\beta}}_{c,X}"]
    \ar[dr,bend right=20,"\imath_{\bm{\beta}(X)}"',"\cong"]
    &\bm{\beta}\bigl(c\!\circledast\! X\bigr)
    \ar[d,"\text{ }\bm{\beta}(\imath_X)","\cong\text{ }"']
    \\%1
    \text{ }
    &\bm{\beta}(X)
  \end{tikzcd}
  \qquad\qquad
  \begin{tikzcd}[cramped,row sep=13,column sep=27]
    w\!\circledast\! \bigl(z\!\circledast\! \bm{\beta}(X)\bigr)
    \ar[d,"a_{w,z,\bm{\beta}(X)}\text{ }"',"\text{ }\cong"]
    \ar[r,pos=0.37,"\I_w\circledast t^{\bm{\beta}}_{z,X}"]
    &w\!\circledast\! \bm{\beta}\bigl(z\!\circledast\! X\bigr)
    \ar[r,pos=0.4,"t^{\bm{\beta}}_{w,z\circledast X}"]
    &\bm{\beta}\Bigl(w\!\circledast\! \bigl(z\!\circledast\! X\bigr)\Bigr)
    \ar[d,"\text{ }\bm{\beta}(a_{w,z,X})","\cong\text{ }"']
    \\%1
    \bigl(w\!\circledast\! z\bigr)\!\circledast\! \bm{\beta}(X)
    \ar[rr,"t^{\bm{\beta}}_{w\circledast z,X}"]
    &\text{ }
    &\bm{\beta}\Bigl(\bigl(w\!\circledast\! z\bigr)\!\circledast\! X\Bigr)
  \end{tikzcd}
\end{equation}
Conversely, suppose we have a functor
$\beta:T\rightsquigarrow S$ between the underlying categories
together with a collection of morphisms
$t_{z,X}:z\circledast\beta(X)\to \beta\bigl(z\circledast X\bigr)$
in $\bm{S}$ natural in variables $z$, $X$ satisfying the relations (\ref{eq strength funct}).
Then we have a unique $\mathlcal{C}$-enriched functor $\bm{\beta}:\bm{T}\rightsquigarrow\bm{S}$
whose underlying functor is equal to $\beta$ and $t^{\bm{\beta}}_{z,X}=t_{z,X}$.

Let $\bm{\alpha}$, $\bm{\beta}:\bm{T}\rightsquigarrow\bm{S}$ be $\mathlcal{C}$-enriched
functors between tensored $\mathlcal{C}$-enriched categories $\bm{T}$, $\bm{S}$.
For each $\mathlcal{C}$-enriched natural transformation $\bm{\xi}:\bm{\alpha}\Rightarrow\bm{\beta}$,
the tensorial strengths of $\bm{\alpha}$, $\bm{\beta}$ satisfy the following relation
for all $z\in\Obj(\bm{\mathscr{C}})$, $X\in\Obj(\bm{T})$.
\begin{equation}\label{eq strength nat}
  \begin{tikzcd}[cramped,row sep=13,column sep=30]
    z\!\circledast\! \bm{\alpha}(X)
    \ar[d,"\I_z\circledast \xi_X\text{ }"']
    \ar[r,"t^{\bm{\alpha}}_{z,X}"]
    &\bm{\alpha}\bigl(z\!\circledast\! X\bigr)
    \ar[d,"\text{ }\xi_{z\circledast X}"]
    \\%1
    z\!\circledast\! \bm{\beta}(X)
    \ar[r,"t^{\bm{\beta}}_{z,X}"]
    &\bm{\beta}\bigl(z\!\circledast\! X\bigr)
  \end{tikzcd}
\end{equation}
Conversely, a natural transformation
$\xi:\alpha\Rightarrow \beta$
between the underlying functors $\alpha$, $\beta:T\rightsquigarrow S$
becomes a $\mathlcal{C}$-enriched natural transformation
$\bm{\xi}:\bm{\alpha}\Rightarrow\bm{\beta}:\bm{T}\rightsquigarrow\bm{S}$
precisely when the collection of morphisms $\xi_X$ in $\bm{S}$ satisfies the relation (\ref{eq strength nat}).

A $\mathlcal{C}$-enriched category $\bm{T}$ \emph{has weighted coequalizers}
if for each pair
$\begin{tikzcd}[cramped,column sep=18]X_1\ar[r,shift left=0.7]\ar[r,shift right=0.7]& X_2\end{tikzcd}$
of parallel morphisms in $\bm{T}$,
there exists a morphism
$\begin{tikzcd}[cramped,column sep=18]X_2\ar[r]& X\end{tikzcd}$
in $\bm{T}$ such that for every $Y\in\Obj(\bm{T})$
we have an equalizer diagram
$\begin{tikzcd}[cramped,column sep=18]\bm{T}(X,Y)\ar[r]&\bm{T}(X_2,Y)\ar[r,shift left=0.7]\ar[r,shift right=0.7]& \bm{T}(X_1,Y)\end{tikzcd}$
in the underlying category $\mathscr{C}$ of $\bm{\mathscr{C}}$.
In this case, we automatically have a coequalizer diagram
$\begin{tikzcd}[cramped,column sep=18]X_1\ar[r,shift left=0.7]\ar[r,shift right=0.7]& X_2\ar[r]&X\end{tikzcd}$
in the underlying category $T$ of $\bm{T}$.

\begin{example}
  The $\mathlcal{C}$-enriched category $\bm{\mathscr{C}}$ has weighted coequalizers.
  For each monoid $\mathfrak{b}=\bigl(b,u_b,m_b\bigr)$ in $\mathlcal{C}$,
  the $\mathlcal{C}$-enriched category $\bm{\mathcal{M}}_{\mathfrak{b}}$ has weighted coequalizers.
  For each $z_{\mathfrak{b}}=(z,\gamma_z)\in\Obj(\bm{\mathcal{M}}_{\mathfrak{b}})$,
  we have the following weighted coequalizer diagram in $\bm{\mathcal{M}}_{\mathfrak{b}}$.
  \begin{equation*}
    \begin{tikzcd}[cramped, column sep=50]
      z\!\circledast\! b\!\circledast\! b_{\mathfrak{b}}
      \ar[r,shift left=0.7,"{\gamma_z\circledast\I_{b_{\mathfrak{b}}}}"]
      \ar[r,shift right=0.7,"{\I_z\circledast \gamma_{b_{\mathfrak{b}}}}"']
      &z\!\circledast\! b_{\mathfrak{b}}
      \ar[r,"\gamma_{z_{\mathfrak{b}}}"]
      &z_{\mathfrak{b}}
    \end{tikzcd}
  \end{equation*}    
\end{example}

Let $\mathfrak{b}=\bigl(b,u_b,m_b\bigr)$ be a monoid in $\mathlcal{C}$
and let $\bm{T}$ be a tensored $\mathlcal{C}$-enriched category.
A \emph{left $\mathfrak{b}$-module object in $\bm{T}$}
is a pair $\text{ }\!\!_{\mathfrak{b}}X=(X,\rho_X)$
of $X\in\Obj(\bm{T})$
and a morphism $\rho_X:b\circledast X\to X$ in $\bm{T}$
satisfying the left $\mathfrak{b}$-action relations.
A morphism $\text{ }\!\!_{\mathfrak{b}}X\to\text{ }\!\!_{\mathfrak{b}}\tilde{X}$
of left $\mathfrak{b}$-module objects in $\bm{T}$
is a morphism $X\to \tilde{X}$ in $\bm{T}$ which is compatible with $\rho_X$, $\rho_{\tilde{X}}$.
We denote $\text{ }\!\!_{\mathfrak{b}}T$
as the ordinary category of left $\mathfrak{b}$-module objects in $\bm{T}$.

For each $X\in\Obj(\bm{T})$, the triple
$\mathit{End}_{\bm{T}}(X)=\bigl(\bm{T}(X,X),\mathit{id}_X,\mu_{X,X,X}\bigr)$
is a monoid in $\mathlcal{C}$.
A pair $\bigl(X,\rho_X:b\circledast X\to X\bigr)$
is a left $\mathfrak{b}$-module object in $\bm{T}$
if and only if the corresponding morphism
$\bar{\rho}_X:b\to \bm{T}(X,X)$ in $\bm{\mathscr{C}}$
is a morphism of monoids
$\mathfrak{b}\to\mathit{End}_{\bm{T}}(X)$
in $\mathlcal{C}$.

For each left $\mathfrak{b}$-module object 
$\text{ }\!\!_{\mathfrak{b}}X=(X,\rho_X)$ in $\bm{T}$,
the $\mathlcal{C}$-enriched functor 
$\bm{T}(X,-):\bm{T}\rightsquigarrow\bm{\mathscr{C}}$
factors through the forgetful $\mathlcal{C}$-enriched functor 
$\bm{\mathcal{M}}_{\mathfrak{b}}\rightsquigarrow\bm{\mathscr{C}}$.
We have the $\mathlcal{C}$-enriched functor
$\bm{T}(\text{ }\!\!_{\mathfrak{b}}X,-):\bm{T}\rightsquigarrow\bm{\mathcal{M}}_{\mathfrak{b}}$
which sends each $Y\in\Obj(\bm{T})$ to the pair
$\bm{T}(\text{ }\!\!_{\mathfrak{b}}X,Y)=\bigl(\bm{T}(X,Y),\gamma_{\bm{T}(X,Y)}\bigr)\in\Obj(\bm{\mathcal{M}}_{\mathfrak{b}})$
where
\begin{equation*}
  \begin{tikzcd}[cramped,column sep=60]
    \bm{T}
    \ar[r,rightsquigarrow,"{\bm{T}(\text{ }\!\!_{\mathfrak{b}}X,-)}"]
    \ar[dr,bend right=18,rightsquigarrow,"{\bm{T}(X,-)}"']
    &\bm{\mathcal{M}}_{\mathfrak{b}}
    \ar[d,rightsquigarrow]
    \\%1
    \text{ }
    &\bm{\mathscr{C}}
  \end{tikzcd}
  \qquad\qquad
  \begin{aligned}
    \gamma_{\bm{T}(X,Y)}
    &:=
    \mu_{X,X,Y}
    \circ
    \bigl(\I_{\bm{T}(X,Y)}\circledast\bar{\rho}_X\bigr)
    \\
    &:
    \begin{tikzcd}[cramped,column sep=16]
      \bm{T}(X,Y)\!\circledast\! b
      \ar[r]
      &\bm{T}(X,Y)\!\circledast\! \bm{T}(X,X)
      \ar[r]
      &\bm{T}(X,Y)
      .
    \end{tikzcd}  
  \end{aligned}
\end{equation*}

\begin{example}
  Let $\mathfrak{b}$, $\mathfrak{b}'$
  be monoids in $\mathlcal{C}$.
  A \emph{$(\mathfrak{b},\mathfrak{b}')$-bimodule}
  $\text{ }\!\!_{\mathfrak{b}}x_{\mathfrak{b}'}=(x_{\mathfrak{b}'},\rho_{x_{\mathfrak{b}'}})$
  is a left $\mathfrak{b}$-module object in $\bm{\mathcal{M}}_{\mathfrak{b}'}$,
  which is a pair of $x_{\mathfrak{b}'}\in\Obj(\bm{\mathcal{M}}_{\mathfrak{b}'})$
  and a morphism $\rho_{x_{\mathfrak{b}'}}:b\circledast x_{\mathfrak{b}'}\to x_{\mathfrak{b}'}$ in $\bm{\mathcal{M}}_{\mathfrak{b}'}$.
  We denote $\text{ }\!\!_{\mathfrak{b}}\mathcal{M}_{\mathfrak{b}'}$ as
  the ordinary category of $(\mathfrak{b},\mathfrak{b}')$-bimodules.
  For instance, the pair
  $\text{ }\!\!_{\mathfrak{b}}b_{\mathfrak{b}}
  :=\bigl(b_{\mathfrak{b}},\gamma_{b_{\mathfrak{b}}}:b\circledast b_{\mathfrak{b}}\to b_{\mathfrak{b}}\bigr)$
  is a $(\mathfrak{b},\mathfrak{b})$-bimodule.
\end{example}

For the rest of this section,
$\mathfrak{b}=\bigl(b,u_b,m_b\bigr)$ is a monoid in $\mathlcal{C}$
and 
$\bm{T}$ is a tensored $\mathlcal{C}$-enriched category
which has weighted coequalizers.

\begin{definition}\label{def tensorCfunct}
  Given $z_{\mathfrak{b}}=(z,\gamma_z)\in\Obj(\bm{\mathcal{M}}_{\mathfrak{b}})$
  and a left $\mathfrak{b}$-module object $\text{ }\!\!_{\mathfrak{b}}X=(X,\rho_X)$ in $\bm{T}$,
  we define $z_{\mathfrak{b}}\circledast_{\mathfrak{b}}\text{ }\!\!_{\mathfrak{b}}X\in\Obj(\bm{T})$
  as the weighted coequalizer in $\bm{T}$
  \begin{equation*}
    \begin{tikzcd}[cramped, column sep=50]
      z\!\circledast\! b\!\circledast\! X
      \ar[r,shift left=0.7,"{\gamma_z\circledast\I_X}"]
      \ar[r,shift right=0.7,"{\I_z\circledast \rho_X}"']
      &z\!\circledast\! X
      \ar[r,pos=0.45,two heads,"{\mathit{coeq}_{z_{\mathfrak{b}},\text{ }\!\!_{\mathfrak{b}}\!X}}"]
      &z_{\mathfrak{b}}\!\circledast_{\mathfrak{b}}\! \text{ }\!\!_{\mathfrak{b}}X
      .
    \end{tikzcd}
  \end{equation*}
  For each pair of a morphism $l:z_{\mathfrak{b}}\to\tilde{z}_{\mathfrak{b}}$ in $\bm{\mathcal{M}}_{\mathfrak{b}}$
  and a morphism $\tilde{l}:\text{ }\!\!_{\mathfrak{b}}X\to \text{ }\!\!_{\mathfrak{b}}\tilde{X}$
  of left $\mathfrak{b}$-module objects in $\bm{T}$,
  we have a unique morphism
  $l\circledast_{\mathfrak{b}}\tilde{l}
  :z_{\mathfrak{b}}\circledast_{\mathfrak{b}}\text{ }\!\!_{\mathfrak{b}}X
  \to \tilde{z}_{\mathfrak{b}}\circledast_{\mathfrak{b}}\text{ }\!\!_{\mathfrak{b}}\tilde{X}$ in $\bm{T}$
  satisfying the relation 
  \begin{equation*}
    \begin{tikzcd}[cramped,row sep=13,column sep=45]
      z\!\circledast\! X
      \ar[d,two heads,"\mathit{coeq}_{z_{\mathfrak{b}},\text{ }\!\!_{\mathfrak{b}}X}\text{ }"']
      \ar[r,"l\circledast \tilde{l}"]
      &\tilde{z}\!\circledast\! \tilde{X}
      \ar[d,two heads,"\text{ }\mathit{coeq}_{\tilde{z}_{\mathfrak{b}},\text{ }\!\!_{\mathfrak{b}}\tilde{X}}"]
      \\%1
      z_{\mathfrak{b}}\!\circledast_{\mathfrak{b}}\! \text{ }\!\!_{\mathfrak{b}}X
      \ar[r,pos=0.4,dotted,"\exists!\text{ }l\circledast_{\mathfrak{b}}\tilde{l}"]
      &\tilde{z}_{\mathfrak{b}}\!\circledast_{\mathfrak{b}}\! \text{ }\!\!_{\mathfrak{b}}\tilde{X}
    \end{tikzcd}
    \qquad
    \mathit{coeq}_{\tilde{z}_{\mathfrak{b}},\text{ }\!\!_{\mathfrak{b}}\tilde{X}}
    \circ
    \bigl(l\circledast \tilde{l}\bigr)
    =
    \bigl(l\circledast_{\mathfrak{b}} \tilde{l}\bigr)
    \circ
    \mathit{coeq}_{z_{\mathfrak{b}},\text{ }\!\!_{\mathfrak{b}}X}
    .
  \end{equation*}
  We have a $\mathlcal{C}$-enriched functor
  $-\circledast_{\mathfrak{b}}\text{ }\!\!_{\mathfrak{b}}X:\bm{\mathcal{M}}_{\mathfrak{b}}\rightsquigarrow\bm{T}$.
  For each $w\in\Obj(\bm{\mathscr{C}})$,
  the tensorial strength
  $a_{w,z_{\mathfrak{b}},\text{ }\!\!_{\mathfrak{b}}X}
  :=t^{-\circledast_{\mathfrak{b}}\text{ }\!\!_{\mathfrak{b}}X}_{w,z_{\mathfrak{b}}}$
  is the unique isomorphism in $\bm{T}$ satisfying the relation
  \begin{equation*}
    \begin{tikzcd}[cramped,row sep=13,column sep=50]
      w\!\circledast\! \bigl(z\!\circledast\! X\bigr)
      \ar[d,two heads,"\I_w\circledast \mathit{coeq}_{z_{\mathfrak{b}},\text{ }\!\!_{\mathfrak{b}}X}\text{ }"']
      \ar[r,"a_{w,z,X}","\cong"']
      &\bigl(w\!\circledast\! z\bigr)\!\circledast\! X
      \ar[d,two heads,"\text{ }\mathit{coeq}_{w\circledast z_{\mathfrak{b}},\text{ }\!\!_{\mathfrak{b}}X}"]
      \\%1
      w\!\circledast\! \bigl(z_{\mathfrak{b}}\!\circledast_{\mathfrak{b}}\! \text{ }\!\!_{\mathfrak{b}}X\bigr)
      \ar[r,pos=0.45,dotted,"\exists!\text{ }a_{w,z_{\mathfrak{b}},\text{ }\!\!_{\mathfrak{b}}X}","\cong"']
      &\bigl(w\!\circledast\! z_{\mathfrak{b}}\bigr)\!\circledast_{\mathfrak{b}}\! \text{ }\!\!_{\mathfrak{b}}X
    \end{tikzcd}
    \quad
    \begin{aligned}
      &
      a_{w,z_{\mathfrak{b}},\text{ }\!\!_{\mathfrak{b}}X}
      \circ
      \bigl(\I_w\circledast\mathit{coeq}_{z_{\mathfrak{b}},\text{ }\!\!_{\mathfrak{b}}X}\bigr)
      \\
      &=
      \mathit{coeq}_{w\circledast z_{\mathfrak{b}},\text{ }\!\!_{\mathfrak{b}}X}
      \circ
      a_{w,z,X}
      .
    \end{aligned}
  \end{equation*}
\end{definition}
For each morphism $\text{ }\!\!_{\mathfrak{b}}X\to\text{ }\!\!_{\mathfrak{b}}\tilde{X}$
of left $\mathfrak{b}$-module objects in $\bm{T}$,
the collection of morphisms
$z_{\mathfrak{b}}\circledast_{\mathfrak{b}}\text{ }\!\!_{\mathfrak{b}}X
\to z_{\mathfrak{b}}\circledast_{\mathfrak{b}}\text{ }\!\!_{\mathfrak{b}}\tilde{X}$
in $\bm{T}$ is $\mathlcal{C}$-enriched natural in variable $z_{\mathfrak{b}}\in\Obj(\bm{\mathcal{M}}_{\mathfrak{b}})$.
Thus we have a well-defined functor
\begin{equation}\label{eq leftadj}
  \begin{tikzcd}[row sep=0]
    \text{ }\!\!_{\mathfrak{b}}T
    \ar[r,squiggly]
    &
    {\mathlcal{C}\text{-}\mathit{Funct}(\bm{\mathcal{M}}_{\mathfrak{b}},\bm{T})}
    ,
    \\%1
    \text{ }\!\!_{\mathfrak{b}}X
    \ar[r,mapsto]
    &
    -\circledast_{\mathfrak{b}}\! \text{ }\!\!_{\mathfrak{b}}X:\bm{\mathcal{M}}_{\mathfrak{b}}\rightsquigarrow\bm{T}
    .
  \end{tikzcd}
\end{equation}

For each left $\mathfrak{b}$-module object 
$\text{ }\!\!_{\mathfrak{b}}X=(X,\rho_X)$ in $\bm{T}$,
we have a unique isomorphism
$\imath^{\mathfrak{b}}_{\text{ }\!\!_{\mathfrak{b}}X}
:b_{\mathfrak{b}}\circledast_{\mathfrak{b}}\text{ }\!\!_{\mathfrak{b}} X\xrightarrow[]{\cong} X$
in $\bm{T}$ which satisfies the relation
$\rho_X=
\imath^{\mathfrak{b}}_{\text{ }\!\!_{\mathfrak{b}}X}
\circ
\mathit{coeq}_{b_{\mathfrak{b}},\text{ }\!\!_{\mathfrak{b}}X}$.
The inverse is given by
\begin{equation*}
  \begin{tikzcd}[cramped,row sep=13,column sep=50]
    b\!\circledast\! X
    \ar[d,two heads,"\mathit{coeq}_{b_{\mathfrak{b}},\text{ }\!\!_{\mathfrak{b}}X}\text{ }"']
    \ar[dr,bend left=20,"\rho_X"]
    \\%1
    b_{\mathfrak{b}}\!\circledast_{\mathfrak{b}}\! \text{ }\!\!_{\mathfrak{b}}X
    \ar[r,pos=0.35,dotted,"\exists!\text{ }\imath^{\mathfrak{b}}_{\text{ }\!\!_{\mathfrak{b}}X}","\cong"']
    &\text{ }X
  \end{tikzcd}
  \qquad\qquad
  \begin{aligned}
    (\imath^{\mathfrak{b}}_{\text{ }\!\!_{\mathfrak{b}}X})^{-1}
    &=
    \mathit{coeq}_{b_{\mathfrak{b}},\text{ }\!\!_{\mathfrak{b}}X}
    \circ
    \bigl(u_b\!\circledast\! \I_X\bigr)
    \circ
    \imath^{-1}_X
    \\%1
    &:    
    \begin{tikzcd}[cramped,column sep=20]
      X
      \ar[r,pos=0.4,"\cong"]
      &c\!\circledast\! X
      \ar[r]
      &b\!\circledast\! X
      \ar[r,two heads]
      &b_{\mathfrak{b}}\!\circledast_{\mathfrak{b}}\! \text{ }\!\!_{\mathfrak{b}}X
      .
    \end{tikzcd}  
  \end{aligned}
\end{equation*}

\begin{proposition} \label{prop bXtensorHomCadj}
  For each left $\mathfrak{b}$-module object $\text{ }\!\!_{\mathfrak{b}}X$ in $\bm{T}$,
  we have a $\mathlcal{C}$-enriched adjunction
  \begin{equation*}
    \begin{tikzcd}[cramped,column sep=45]
      \bm{\mathcal{M}}_{\mathfrak{b}}
      \ar[r,squiggly,shift left=1.5,bend left=15,"{-\text{ }\!\circledast_{\mathfrak{b}}\text{ }\!\!_{\mathfrak{b}}X}"]
      &
      \bm{T}
      \ar[l,squiggly,shift left=1.5,bend left=15,"{\bm{T}(\text{ }\!\!_{\mathfrak{b}}X,-)}"]
    \end{tikzcd}
    \qquad\quad
    -\circledast_{\mathfrak{b}}\text{ }\!\!_{\mathfrak{b}}X
    \dashv
    \bm{T}(\text{ }\!\!_{\mathfrak{b}}X,-)
    :\bm{\mathcal{M}}_{\mathfrak{b}}\rightsquigarrow\bm{T}.
  \end{equation*}
  \begin{itemize}
    \item
    The component of the unit $\mathlcal{C}$-enriched natural transformation at $z_{\mathfrak{b}}\in\Obj(\bm{\mathcal{M}}_{\mathfrak{b}})$ is
    the unique morphism
    $\eta_{z_{\mathfrak{b}}}:\begin{tikzcd}[cramped,column sep=15]
      z_{\mathfrak{b}}
      \ar[r]&
      \bm{T}\bigl(\text{ }\!\!_{\mathfrak{b}}X,z_{\mathfrak{b}}\!\circledast_{\mathfrak{b}}\! \text{ }\!\!_{\mathfrak{b}}X\bigr)
    \end{tikzcd}$
    in $\bm{\mathcal{M}}_{\mathfrak{b}}$ whose morphism in $\bm{\mathscr{C}}$ is
    \begin{equation*}
      \eta_{z_{\mathfrak{b}}}:
      \begin{tikzcd}[cramped,column sep=40]
        z
        \ar[r,"{\mathit{Cv}_{z,X}}"]
        &\bm{T}\bigl(X,z\!\circledast\! X\bigr)
        \ar[r,pos=0.43,"{(\mathit{coeq}_{z_{\mathfrak{b}},\text{ }\!\!_{\mathfrak{b}}X})_{\star}}"]
        &\bm{T}\bigl(X,z_{\mathfrak{b}}\!\circledast_{\mathfrak{b}}\! \text{ }\!\!_{\mathfrak{b}}X\bigr)
        .
      \end{tikzcd}
    \end{equation*}

    \item
    The component of the counit $\mathlcal{C}$-enriched natural transformation at $Y\in\Obj(\bm{T})$ is
    the unique morphism
    $\varepsilon_Y:\bm{T}(\text{ }\!\!_{\mathfrak{b}}X,Y)\circledast_{\mathfrak{b}}\text{ }\!\!_{\mathfrak{b}}X\to Y$ in $\bm{T}$
    which satisfies the relation
    \begin{equation*}
      \begin{tikzcd}[cramped,row sep=13,column sep=60]
        \bm{T}(X,Y)\!\circledast\! X
        \ar[d,two heads,"{\mathit{coeq}_{\bm{T}(\text{ }\!\!_{\mathfrak{b}}X,Y),\text{ }\!\!_{\mathfrak{b}}X}\text{ }}"']
        \ar[dr,pos=0.7,bend left=20,"{\mathit{Ev}_{X,Y}}"]
        \\
        \bm{T}(\text{ }\!\!_{\mathfrak{b}}X,Y)\!\circledast_{\mathfrak{b}}\! \text{ }\!\!_{\mathfrak{b}}X
        \ar[r,pos=0.45,dotted,"{\exists!\text{ }\varepsilon_{Y}}"]
        &
        Y
      \end{tikzcd}
      \qquad
      \mathit{Ev}_{X,Y}
      =
      \varepsilon_{Y}
      \circ
      \mathit{coeq}_{\bm{T}(\text{ }\!\!_{\mathfrak{b}}X,Y),\text{ }\!\!_{\mathfrak{b}}X}
      .
    \end{equation*}
  \end{itemize}
\end{proposition}
\begin{proof}
  Omitted.
\end{proof}









\section{The Eilenberg-Watts Theorem} \label{sec EW-thm}
Let $\mathfrak{b}=\bigl(b,u_b,m_b\bigr)$ be a monoid in $\mathlcal{C}$
and let $\bm{T}$ be a tensored $\mathlcal{C}$-enriched category which has weighted coequalizers.
The goal of this section is to prove the following theorem,
which is a stronger statement than Theorem~\ref{thm Intro EWthm}.
Let us recall the functor
$\text{ }\!\!_{\mathfrak{b}}T\rightsquigarrow
\mathlcal{C}\text{-}\mathit{Funct}(\bm{\mathcal{M}}_{\mathfrak{b}},\bm{T})$
in (\ref{eq leftadj}).

\begin{theorem} \label{thm EWthm cocomCcat}
  We have a fully faithful left adjoint functor
  \begin{equation*}
    \begin{tikzcd}[row sep=0]
      \text{ }\!\!_{\mathfrak{b}}T
      \ar[r,squiggly]
      &
      {\mathlcal{C}\text{-}\mathit{Funct}(\bm{\mathcal{M}}_{\mathfrak{b}},\bm{T})}
      \\%1
      \text{ }\!\!_{\mathfrak{b}}X
      \ar[r,mapsto]
      &
      -\circledast_{\mathfrak{b}}\! \text{ }\!\!_{\mathfrak{b}}X:\bm{\mathcal{M}}_{\mathfrak{b}}\rightsquigarrow\bm{T}
    \end{tikzcd}
  \end{equation*}
  whose essential image is the full subcategory
  $\mathlcal{C}\text{-}\mathit{Funct}_{\mathit{cocon}}(\bm{\mathcal{M}}_{\mathfrak{b}},\bm{T})$
  of cocontinuous $\mathlcal{C}$-enriched functors
  $\bm{\mathcal{F}}:\bm{\mathcal{M}}_{\mathfrak{b}}\rightsquigarrow \bm{T}$. 
  In particular:
  \begin{itemize}
    \item[(i)]
    we have an equivalence of categories
    $\begin{tikzcd}[cramped,column sep=30]
      \text{ }\!\!_{\mathfrak{b}}T
      \ar[r,squiggly,"\simeq"]
      &{\mathlcal{C}\text{-}\mathit{Funct}_{\mathit{cocon}}(\bm{\mathcal{M}}_{\mathfrak{b}},\bm{T})}
    \end{tikzcd}$;

    \item[(ii)]
    $\mathlcal{C}\text{-}\mathit{Funct}_{\mathit{cocon}}(\bm{\mathcal{M}}_{\mathfrak{b}},\bm{T})$
    is a coreflective full subcategory of
    $\mathlcal{C}\text{-}\mathit{Funct}(\bm{\mathcal{M}}_{\mathfrak{b}},\bm{T})$.
  \end{itemize}
\end{theorem}

We explain what the right adjoint functor is.
For each $\mathlcal{C}$-enriched functor $\bm{\mathcal{F}}:\bm{\mathcal{M}}_{\mathfrak{b}}\rightsquigarrow\bm{T}$,
we have a left $\mathfrak{b}$-module object
$\text{ }\!\!_{\mathfrak{b}}\bm{\mathcal{F}}(b_{\mathfrak{b}})
=\bigl(\bm{\mathcal{F}}(b_{\mathfrak{b}}),\rho_{\bm{\mathcal{F}}(b_{\mathfrak{b}})}\bigr)$
in $\bm{T}$
where $\rho_{\bm{\mathcal{F}}(b_{\mathfrak{b}})}
:=\bm{\mathcal{F}}(\gamma_{b_{\mathfrak{b}}})\circ t^{\bm{\mathcal{F}}}_{b,b_{\mathfrak{b}}}
:b\circledast \bm{\mathcal{F}}(b_{\mathfrak{b}})\to \bm{\mathcal{F}}\bigl(b\circledast b_{\mathfrak{b}}\bigr)\to\bm{\mathcal{F}}(b_{\mathfrak{b}})$.
We have a functor
\begin{equation}\label{eq rightCadj}
  \begin{tikzcd}[row sep=0]
    {\mathlcal{C}\text{-}\mathit{Funct}(\bm{\mathcal{M}}_{\mathfrak{b}},\bm{T})}
    \ar[r,squiggly]
    &\text{ }\!\!_{\mathfrak{b}}T
    ,
    \\%1
    \bm{\mathcal{F}}:\bm{\mathcal{M}}_{\mathfrak{b}}\rightsquigarrow\bm{T}
    \ar[r,mapsto]
    &\text{ }\!\!_{\mathfrak{b}}\bm{\mathcal{F}}(b_{\mathfrak{b}})
    .
  \end{tikzcd}
\end{equation}

\begin{proposition} \label{prop lambdaF}
  For each $\mathlcal{C}$-enriched functor $\bm{\mathcal{F}}:\bm{\mathcal{M}}_{\mathfrak{b}}\rightsquigarrow\bm{T}$,
  we have a $\mathlcal{C}$-enriched natural transformation
    $\bm{\lambda}^{\bm{\mathcal{F}}}:
    \begin{tikzcd}[cramped,column sep=20]
      -\!\circledast_{\mathfrak{b}}\! \text{ }\!\!_{\mathfrak{b}}\bm{\mathcal{F}}(b_{\mathfrak{b}})
      \ar[r,Rightarrow]
      &\bm{\mathcal{F}}:\bm{\mathcal{M}}_{\mathfrak{b}}\rightsquigarrow\bm{T}      
    \end{tikzcd}$
  whose component $\lambda^{\bm{\mathcal{F}}}_{z_{\mathfrak{b}}}$
  at $z_{\mathfrak{b}}\in\Obj(\bm{\mathcal{M}}_{\mathfrak{b}})$ is the unique morphism in $\bm{T}$
  satisfying the relation
  \begin{equation} \label{eq prop lambdaF}
    \begin{tikzcd}[cramped,row sep=20,column sep=60]
      z\!\circledast\! b\!\circledast\! \bm{\mathcal{F}}(b_{\mathfrak{b}})
      \ar[d,shift left=1,"\text{ }\I_z\circledast\rho_{\bm{\mathcal{F}}(b_{\mathfrak{b}})}"]
      \ar[d,shift right=1,"\gamma_z\circledast\I_{\bm{\mathcal{F}}(b_{\mathfrak{b}})}\text{ }"']
      \ar[r,"t^{\bm{\mathcal{F}}}_{z\circledast b,b_{\mathfrak{b}}}"]
      &\bm{\mathcal{F}}\bigl(z\!\circledast\! b\!\circledast\! b_{\mathfrak{b}}\bigr)
      \ar[d,shift left=1,"\text{ }\bm{\mathcal{F}}(\I_z\circledast \gamma_{b_{\mathfrak{b}}})"]
      \ar[d,shift right=1,"\bm{\mathcal{F}}(\gamma_z\circledast\I_{b_{\mathfrak{b}}})\text{ }"']
      \\%1
      z\!\circledast\! \bm{\mathcal{F}}(b_{\mathfrak{b}})
      \ar[d,two heads,"\mathit{coeq}_{z_{\mathfrak{b}},\text{ }\!\!_{\mathfrak{b}}\bm{\mathcal{F}}(b_{\mathfrak{b}})}\text{ }"']
      \ar[r,"t^{\bm{\mathcal{F}}}_{z,b_{\mathfrak{b}}}"]
      &\bm{\mathcal{F}}\bigl(z\!\circledast\! b_{\mathfrak{b}}\bigr)
      \ar[d,"\text{ }\bm{\mathcal{F}}(\gamma_{z_{\mathfrak{b}}})"]
      \\
      z_{\mathfrak{b}}\!\circledast_{\mathfrak{b}}\! \text{ }\!\!_{\mathfrak{b}}\bm{\mathcal{F}}(b_{\mathfrak{b}})
      \ar[r,pos=0.4,dotted,"\exists!\text{ }\lambda^{\bm{\mathcal{F}}}_{z_{\mathfrak{b}}}"]
      &\bm{\mathcal{F}}(z_{\mathfrak{b}})
    \end{tikzcd}
    \qquad
    \begin{aligned}
      &
      \bm{\mathcal{F}}(\gamma_{z_{\mathfrak{b}}})
      \circ t^{\bm{\mathcal{F}}}_{z,b_{\mathfrak{b}}}
      \\
      &=
      \lambda^{\bm{\mathcal{F}}}_{z_{\mathfrak{b}}}
      \circ \mathit{coeq}_{z_{\mathfrak{b}},\text{ }\!\!_{\mathfrak{b}}\bm{\mathcal{F}}(b_{\mathfrak{b}})}
      .
    \end{aligned}
  \end{equation}
  We have
  $\lambda^{\bm{\mathcal{F}}}_{b_{\mathfrak{b}}}
  =\imath^{\mathfrak{b}}_{\text{ }\!\!_{\mathfrak{b}}\bm{\mathcal{F}}(b_{\mathfrak{b}})}:
  b_{\mathfrak{b}}\circledast_{\mathfrak{b}} \text{ }\!\!_{\mathfrak{b}}\bm{\mathcal{F}}(b_{\mathfrak{b}})
  \xrightarrow[]{\cong}\bm{\mathcal{F}}(b_{\mathfrak{b}})$.
  Moreover, the following are equivalent:
  \begin{itemize}
    \item[(i)]
    The $\mathlcal{C}$-enriched natural transformation $\bm{\lambda}^{\bm{\mathcal{F}}}$ is a $\mathlcal{C}$-enriched natural isomorphism.

    \item[(ii)]
    The $\mathlcal{C}$-enriched functor $\bm{\mathcal{F}}:\bm{\mathcal{M}}_{\mathfrak{b}}\rightsquigarrow\bm{T}$ is a left adjoint.

    \item[(iii)]
    The $\mathlcal{C}$-enriched functor $\bm{\mathcal{F}}:\bm{\mathcal{M}}_{\mathfrak{b}}\rightsquigarrow\bm{T}$ is cocontinuous.

    \item[(iv)]
    The $\mathlcal{C}$-enriched functor $\bm{\mathcal{F}}:\bm{\mathcal{M}}_{\mathfrak{b}}\rightsquigarrow\bm{T}$ 
    preserves tensored objects and weighted coequalizers.
  \end{itemize}
\end{proposition}
\begin{proof}
  We leave for the readers to check that 
  such morphism $\lambda^{\bm{\mathcal{F}}}_{z_{\mathfrak{b}}}$ in $\bm{T}$ exists,
  and that the diagram in (\ref{eq prop lambdaF}) commutes.
  To show that the collection of $\lambda^{\bm{\mathcal{F}}}_{z_{\mathfrak{b}}}$
  is $\mathlcal{C}$-enriched natural in variable $z_{\mathfrak{b}}\in\Obj(\bm{\mathcal{M}}_{\mathfrak{b}})$,
  we need to verify the following relation for every $w\in\Obj(\bm{\mathscr{C}})$.
  \begin{equation*}
    \begin{tikzcd}[cramped,row sep=13,column sep=50]
      w\!\circledast\! \bigl(z_{\mathfrak{b}}\!\circledast_{\mathfrak{b}}\! \text{ }\!\!_{\mathfrak{b}}\bm{\mathcal{F}}(b_{\mathfrak{b}})\bigr)
      \ar[d,"\I_w\circledast \lambda^{\bm{\mathcal{F}}}_{z_{\mathfrak{b}}}\text{ }"']
      \ar[r,pos=0.45,"a_{w,z_{\mathfrak{b}},\text{ }\!\!_{\mathfrak{b}}\bm{\mathcal{F}}(b_{\mathfrak{b}})}","\cong"']
      &\bigl(w\!\circledast\! z_{\mathfrak{b}}\bigr)\!\circledast_{\mathfrak{b}}\! \text{ }\!\!_{\mathfrak{b}}\bm{\mathcal{F}}(b_{\mathfrak{b}})
      \ar[d,"\text{ }\lambda^{\bm{\mathcal{F}}}_{w\circledast z_{\mathfrak{b}}}"]
      \\%1
      w\!\circledast\! \bm{\mathcal{F}}(z_{\mathfrak{b}})
      \ar[r,"t^{\bm{\mathcal{F}}}_{w,z_{\mathfrak{b}}}"]
      &\bm{\mathcal{F}}\bigl(w\!\circledast\! z_{\mathfrak{b}}\bigr)          
    \end{tikzcd}
  \end{equation*}
  We obtain this relation after right-cancelling the epimorphism
  $\I_w\circledast \mathit{coeq}_{z_{\mathfrak{b}},\text{ }\!\!_{\mathfrak{b}}\bm{\mathcal{F}}(b_{\mathfrak{b}})}$
  in the diagram below.
  \begin{equation*}
    \hspace*{-1.6cm}
    \begin{tikzcd}[cramped,row sep=12,column sep=4]
      &w\!\circledast\! \bigl(z\!\circledast\! \bm{\mathcal{F}}(b_{\mathfrak{b}})\bigr)
      \ar[d,two heads,"\text{ }\I_w\circledast \mathit{coeq}_{z_{\mathfrak{b}},\text{ }\!\!_{\mathfrak{b}}\bm{\mathcal{F}}(b_{\mathfrak{b}})}"]
      \ar[r,equal]
      &w\!\circledast\! \bigl(z\!\circledast\! \bm{\mathcal{F}}(b_{\mathfrak{b}})\bigr)
      \ar[dd,"\text{ }a_{w,z,\bm{\mathcal{F}}(b_{\mathfrak{b}})}","\cong\text{ }"']
      &w\!\circledast\! \bigl(z\!\circledast\! \bm{\mathcal{F}}(b_{\mathfrak{b}})\bigr)
      \ar[dd,"\text{ }a_{w,z,\bm{\mathcal{F}}(b_{\mathfrak{b}})}","\cong\text{ }"']
      \ar[r,equal]
      &w\!\circledast\! \bigl(z\!\circledast\! \bm{\mathcal{F}}(b_{\mathfrak{b}})\bigr)
      \ar[d,"\text{ }\I_w\circledast t^{\bm{\mathcal{F}}}_{z,b_{\mathfrak{b}}}"]
      &w\!\circledast\! \bigl(z\!\circledast\! \bm{\mathcal{F}}(b_{\mathfrak{b}})\bigr)
      \ar[d,"\text{ }\I_w\circledast t^{\bm{\mathcal{F}}}_{z,b_{\mathfrak{b}}}"]
      &w\!\circledast\! \bigl(z\!\circledast\! \bm{\mathcal{F}}(b_{\mathfrak{b}})\bigr)
      \ar[d,"\text{ }\I_w\circledast t^{\bm{\mathcal{F}}}_{z,b_{\mathfrak{b}}}"]
      \ar[r,equal]
      &w\!\circledast\! \bigl(z\!\circledast\! \bm{\mathcal{F}}(b_{\mathfrak{b}})\bigr)
      \ar[d,two heads,shift left=7,"\I_w\circledast \mathit{coeq}_{z_{\mathfrak{b}},\text{ }\!\!_{\mathfrak{b}}\bm{\mathcal{F}}(b_{\mathfrak{b}})}\text{ }"']
      \\%1
      &w\!\circledast\! \bigl(z_{\mathfrak{b}}\!\circledast_{\mathfrak{b}}\! \text{ }\!\!_{\mathfrak{b}}\bm{\mathcal{F}}(b_{\mathfrak{b}})\bigr)
      \ar[dd,"\text{ }a_{w,z_{\mathfrak{b}},\text{ }\!\!_{\mathfrak{b}}\bm{\mathcal{F}}(b_{\mathfrak{b}})}","\cong\text{ }"']
      &\text{ }
      &\text{ }
      &w\!\circledast\! \bm{\mathcal{F}}\bigl(z\!\circledast\! b_{\mathfrak{b}}\bigr)
      \ar[d,"\text{ }t^{\bm{\mathcal{F}}}_{w,z\circledast b_{\mathfrak{b}}}"]
      &w\!\circledast\! \bm{\mathcal{F}}\bigl(z\!\circledast\! b_{\mathfrak{b}}\bigr)
      \ar[d,"\text{ }t^{\bm{\mathcal{F}}}_{w,z\circledast b_{\mathfrak{b}}}"]
      \ar[r,equal]
      &w\!\circledast\! \bm{\mathcal{F}}\bigl(z\!\circledast\! b_{\mathfrak{b}}\bigr)
      \ar[dd,"\text{ }\I_w\circledast \bm{\mathcal{F}}(\gamma_{z_{\mathfrak{b}}})"]
      &w\!\circledast\! \bigl(z_{\mathfrak{b}}\!\circledast_{\mathfrak{b}}\! \text{ }\!\!_{\mathfrak{b}}\bm{\mathcal{F}}(b_{\mathfrak{b}})\bigr)
      \ar[dd,"\text{ }\I_w\circledast \lambda^{\bm{\mathcal{F}}}_{z_{\mathfrak{b}}}"]
      \\%2
      &\text{ }
      &\bigl(w\!\circledast\! z\bigr)\!\circledast\! \bm{\mathcal{F}}(b_{\mathfrak{b}})
      \ar[d,two heads,"\text{ }\mathit{coeq}_{w\circledast z_{\mathfrak{b}},\bm{\mathcal{F}}(b_{\mathfrak{b}})}"]
      \ar[r,equal]
      &\bigl(w\!\circledast\! z\bigr)\!\circledast\! \bm{\mathcal{F}}(b_{\mathfrak{b}})
      \ar[d,"\text{ }t^{\bm{\mathcal{F}}}_{w\circledast z,b_{\mathfrak{b}}}"]
      &\bm{\mathcal{F}}\Bigl(w\!\circledast\! \bigl(z\!\circledast\! b_{\mathfrak{b}}\bigr)\Bigr)
      \ar[d,"\text{ }\bm{\mathcal{F}}(a_{w,z,b_{\mathfrak{b}}})","\cong\text{ }"']
      \ar[r,equal]
      &\bm{\mathcal{F}}\Bigl(w\!\circledast\! \bigl(z\!\circledast\! b_{\mathfrak{b}}\bigr)\Bigr)
      \ar[dd,"\text{ }\bm{\mathcal{F}}(\I_w\circledast \gamma_{z_{\mathfrak{b}}})"]
      &\text{ }
      &\text{ }
      \\%3
      &\bigl(w\!\circledast\! z_{\mathfrak{b}}\bigr)\!\circledast_{\mathfrak{b}}\! \text{ }\!\!_{\mathfrak{b}}\bm{\mathcal{F}}(b_{\mathfrak{b}})
      \ar[d,"\text{ }\lambda^{\bm{\mathcal{F}}}_{w\circledast z_{\mathfrak{b}}}"]
      \ar[r,equal]
      &\bigl(w\!\circledast\! z_{\mathfrak{b}}\bigr)\!\circledast_{\mathfrak{b}}\! \text{ }\!\!_{\mathfrak{b}}\bm{\mathcal{F}}(b_{\mathfrak{b}})
      \ar[d,"\text{ }\lambda^{\bm{\mathcal{F}}}_{w\circledast z_{\mathfrak{b}}}"]
      &\bm{\mathcal{F}}\Bigl(\bigl(w\!\circledast\! z\bigr)\!\circledast\! b_{\mathfrak{b}}\Bigr)
      \ar[d,"\text{ }\bm{\mathcal{F}}(\gamma_{w\circledast z_{\mathfrak{b}}})"]
      \ar[r,equal]
      &\bm{\mathcal{F}}\Bigl(\bigl(w\!\circledast\! z\bigr)\!\circledast\! b_{\mathfrak{b}}\Bigr)
      \ar[d,"\text{ }\bm{\mathcal{F}}(\gamma_{w\circledast z_{\mathfrak{b}}})"]
      &\text{ }
      &w\!\circledast\! \bm{\mathcal{F}}(z_{\mathfrak{b}})
      \ar[d,"\text{ }t^{\bm{\mathcal{F}}}_{w,z_{\mathfrak{b}}}"]
      \ar[r,equal]
      &w\!\circledast\! \bm{\mathcal{F}}(z_{\mathfrak{b}})
      \ar[d,"\text{ }t^{\bm{\mathcal{F}}}_{w,z_{\mathfrak{b}}}"]
      \\%4
      &\bm{\mathcal{F}}\bigl(w\!\circledast\! z_{\mathfrak{b}}\bigr)          
      &\bm{\mathcal{F}}\bigl(w\!\circledast\! z_{\mathfrak{b}}\bigr)          
      \ar[r,equal]
      &\bm{\mathcal{F}}\bigl(w\!\circledast\! z_{\mathfrak{b}}\bigr)          
      &\bm{\mathcal{F}}\bigl(w\!\circledast\! z_{\mathfrak{b}}\bigr)          
      \ar[r,equal]
      &\bm{\mathcal{F}}\bigl(w\!\circledast\! z_{\mathfrak{b}}\bigr)          
      \ar[r,equal]
      &\bm{\mathcal{F}}\bigl(w\!\circledast\! z_{\mathfrak{b}}\bigr)          
      &\bm{\mathcal{F}}\bigl(w\!\circledast\! z_{\mathfrak{b}}\bigr)          
    \end{tikzcd}
  \end{equation*}
  Thus we have a well-defined $\mathlcal{C}$-enriched natural transformation $\bm{\lambda}^{\bm{\mathcal{F}}}$.
  From the definition, we have
  $\lambda^{\bm{\mathcal{F}}}_{b_{\mathfrak{b}}}
  =\imath^{\mathfrak{b}}_{\text{ }\!\!_{\mathfrak{b}}\bm{\mathcal{F}}(b_{\mathfrak{b}})}:
  b_{\mathfrak{b}}\circledast_{\mathfrak{b}} \text{ }\!\!_{\mathfrak{b}}\bm{\mathcal{F}}(b_{\mathfrak{b}})
  \xrightarrow[]{\cong}\bm{\mathcal{F}}(b_{\mathfrak{b}})$.
    
  We are left to check that the statements (i)-(iv) are equivalent.
  By Proposition~\ref{prop bXtensorHomCadj}, 
  (i) implies (ii).
  We immediately see that (ii) implies (iii)
  and (iii) implies (iv).
  We claim that (iv) implies (i).
  Assume that the $\mathlcal{C}$-enriched functor
  $\bm{\mathcal{F}}:\bm{\mathcal{M}}_{\mathfrak{b}}\rightsquigarrow\bm{T}$
  preserves tensored objects and weighted coequalizers.
  If we look at the diagram in (\ref{eq prop lambdaF}),
  we see that
  the top, middle horizontal morphisms are isomorphisms in $\bm{T}$
  and
  the right vertical morphisms form a weighted coequalizer diagram in $\bm{T}$.
  This shows that $\lambda^{\bm{\mathcal{F}}}_{z_{\mathfrak{b}}}$
  is an isomorphism in $\bm{T}$ for every $z_{\mathfrak{b}}\in\Obj(\bm{\mathcal{M}}_{\mathfrak{b}})$.
  This completes the proof of Proposition~\ref{prop lambdaF}.
\end{proof}

\begin{remark}  
  Let $\mathfrak{b}'$ be another monoid in $\mathlcal{C}$.
  In \cite{Hovey2009},
  Hovey defined the underlying natural transformation of
  $\bm{\lambda}^{\bm{\mathcal{F}}}
  :-\circledast_{\mathfrak{b}}\text{ }\!\!_{\mathfrak{b}}\bm{\mathcal{F}}(b_{\mathfrak{b}})\Rightarrow \bm{\mathcal{F}}:\bm{\mathcal{M}}_{\mathfrak{b}}\rightsquigarrow\bm{\mathcal{M}}_{\mathfrak{b}'}$
  for each $\mathlcal{C}$-enriched functor
  $\bm{\mathcal{F}}:\bm{\mathcal{M}}_{\mathfrak{b}}\rightsquigarrow\bm{\mathcal{M}}_{\mathfrak{b}'}$.
\end{remark}

\begin{lemma} \label{lem lambdaF CnaturalinF}
  The collection of $\mathlcal{C}$-enriched natural transformations
  \begin{equation*}
    \Bigl\{
      \text{ }
      \bm{\lambda}^{\bm{\mathcal{F}}}:-\!\circledast_{\mathfrak{b}}\! \text{ }\!\!_{\mathfrak{b}}\bm{\mathcal{F}}(b_{\mathfrak{b}})\Rightarrow \bm{\mathcal{F}}
      \text{ }\Big|\text{ }
      \bm{\mathcal{F}}:\bm{\mathcal{M}}_{\mathfrak{b}}\rightsquigarrow\bm{T}
      \text{ }
    \Bigr\}
  \end{equation*}
  is natural in variable $\bm{\mathcal{F}}:\bm{\mathcal{M}}_{\mathfrak{b}}\rightsquigarrow\bm{T}$.
\end{lemma}
\begin{proof}
  For each $\mathlcal{C}$-enriched natural transformation
  $\bm{\xi}:\bm{\mathcal{F}}\Rightarrow\tilde{\bm{\mathcal{F}}}$
  between $\mathlcal{C}$-enriched functors
  $\bm{\mathcal{F}}$, $\tilde{\bm{\mathcal{F}}}:\bm{\mathcal{M}}_{\mathfrak{b}}\rightsquigarrow\bm{T}$,
  we need to verify the relation
  \begin{equation} \label{eq lambdaF CnaturalinF}
    \begin{tikzcd}[cramped,row sep=13,column sep=30]
      -\!\circledast_{\mathfrak{b}}\! \text{ }\!\!_{\mathfrak{b}}\bm{\mathcal{F}}(b_{\mathfrak{b}})
      \ar[d,Rightarrow,"-\text{ }\! \circledast_{\mathfrak{b}}\xi_{b_{\mathfrak{b}}}\text{ }"']
      \ar[r,Rightarrow,"\bm{\lambda}^{\bm{\mathcal{F}}}"]
      &\bm{\mathcal{F}}
      \ar[d,Rightarrow,"\text{ }\bm{\xi}"]
      \\%1
      -\!\circledast_{\mathfrak{b}}\! \text{ }\!\!_{\mathfrak{b}}\tilde{\bm{\mathcal{F}}}(b_{\mathfrak{b}})
      \ar[r,Rightarrow,"\bm{\lambda}^{\tilde{\bm{\mathcal{F}}}}"]
      &\tilde{\bm{\mathcal{F}}}
    \end{tikzcd}
    \qquad
    \bm{\xi}\circ\bm{\lambda}^{\bm{\mathcal{F}}}
    =\bm{\lambda}^{\tilde{\bm{\mathcal{F}}}}
    \circ
    \bigl(-\text{ }\! \!\circledast_{\mathfrak{b}}\! \xi_{b_{\mathfrak{b}}}\bigr):
    -\text{ }\! \!\circledast_{\mathfrak{b}}\! \text{ }\!\!_{\mathfrak{b}}\bm{\mathcal{F}}(b_{\mathfrak{b}})
    \Rightarrow
    \tilde{\bm{\mathcal{F}}}.
  \end{equation}
  For each $z_{\mathfrak{b}}=(z,\gamma_z)\in\Obj(\bm{\mathcal{M}}_{\mathfrak{b}})$,
  we have the following diagram.
  \begin{equation*}
    \hspace*{-0.5cm}
    \begin{tikzcd}[cramped,row sep=13,column sep=10]
      &z\!\circledast\! \bm{\mathcal{F}}(b_{\mathfrak{b}})
      \ar[d,two heads,"\text{ }\mathit{coeq}_{z_{\mathfrak{b}},\text{ }\!\!_{\mathfrak{b}}\bm{\mathcal{F}}(b_{\mathfrak{b}})}"]
      \ar[r,equal]
      &z\!\circledast\! \bm{\mathcal{F}}(b_{\mathfrak{b}})
      \ar[d,"\text{ }t^{\bm{\mathcal{F}}}_{z,b_{\mathfrak{b}}}"]
      &z\!\circledast\! \bm{\mathcal{F}}(b_{\mathfrak{b}})
      \ar[d,"\text{ }t^{\bm{\mathcal{F}}}_{z,b_{\mathfrak{b}}}"]
      \ar[r,equal]
      &z\!\circledast\! \bm{\mathcal{F}}(b_{\mathfrak{b}})
      \ar[d,"\text{ }\I_z\circledast \xi_{b_{\mathfrak{b}}}"]
      &z\!\circledast\! \bm{\mathcal{F}}(b_{\mathfrak{b}})
      \ar[d,"\text{ }\I_z\circledast \xi_{b_{\mathfrak{b}}}"]
      \ar[r,equal]
      &z\!\circledast\! \bm{\mathcal{F}}(b_{\mathfrak{b}})
      \ar[d,two heads,"\text{ }\mathit{coeq}_{z_{\mathfrak{b}},\text{ }\!\!_{\mathfrak{b}}\bm{\mathcal{F}}(b_{\mathfrak{b}})}"]
      \\%1
      &z_{\mathfrak{b}}\!\circledast_{\mathfrak{b}}\! \text{ }\!\!_{\mathfrak{b}}\bm{\mathcal{F}}(b_{\mathfrak{b}})
      \ar[d,"\text{ }\lambda^{\bm{\mathcal{F}}}_{z_{\mathfrak{b}}}"]
      &\bm{\mathcal{F}}\bigl(z\!\circledast\! b_{\mathfrak{b}}\bigr)
      \ar[d,"\text{ }\bm{\mathcal{F}}(\gamma_{z_{\mathfrak{b}}})"]
      \ar[r,equal]
      &\bm{\mathcal{F}}\bigl(z\!\circledast\! b_{\mathfrak{b}}\bigr)
      \ar[d,"\text{ }\xi_{z\circledast b_{\mathfrak{b}}}"]
      &z\!\circledast\! \tilde{\bm{\mathcal{F}}}(b_{\mathfrak{b}})
      \ar[d,"\text{ }t^{\tilde{\bm{\mathcal{F}}}}_{z,b_{\mathfrak{b}}}"]
      \ar[r,equal]
      &z\!\circledast\! \tilde{\bm{\mathcal{F}}}(b_{\mathfrak{b}})
      \ar[d,two heads,"\text{ }\mathit{coeq}_{z_{\mathfrak{b}},\text{ }\!\!_{\mathfrak{b}}\tilde{\bm{\mathcal{F}}}(b_{\mathfrak{b}})}"]
      &z_{\mathfrak{b}}\!\circledast_{\mathfrak{b}}\! \text{ }\!\!_{\mathfrak{b}}\bm{\mathcal{F}}(b_{\mathfrak{b}})
      \ar[d,"\text{ }\I_{z_{\mathfrak{b}}}\circledast_{\mathfrak{b}} \xi_{b_{\mathfrak{b}}}"]
      \\%2
      &\bm{\mathcal{F}}(z_{\mathfrak{b}})
      \ar[d,"\text{ }\xi_{z_{\mathfrak{b}}}"]
      \ar[r,equal]
      &\bm{\mathcal{F}}(z_{\mathfrak{b}})
      \ar[d,"\text{ }\xi_{z_{\mathfrak{b}}}"]
      &\tilde{\bm{\mathcal{F}}}\bigl(z\!\circledast\! b_{\mathfrak{b}}\bigr)
      \ar[d,"\text{ }\tilde{\bm{\mathcal{F}}}(\gamma_{z_{\mathfrak{b}}})"]
      \ar[r,equal]
      &\tilde{\bm{\mathcal{F}}}\bigl(z\!\circledast\! b_{\mathfrak{b}}\bigr)
      \ar[d,"\text{ }\tilde{\bm{\mathcal{F}}}(\gamma_{z_{\mathfrak{b}}})"]
      &z_{\mathfrak{b}}\!\circledast_{\mathfrak{b}}\! \text{ }\!\!_{\mathfrak{b}}\tilde{\bm{\mathcal{F}}}(b_{\mathfrak{b}})
      \ar[d,"\text{ }\lambda^{\tilde{\bm{\mathcal{F}}}}_{z_{\mathfrak{b}}}"]
      \ar[r,equal]
      &z_{\mathfrak{b}}\!\circledast_{\mathfrak{b}}\! \text{ }\!\!_{\mathfrak{b}}\tilde{\bm{\mathcal{F}}}(b_{\mathfrak{b}})
      \ar[d,"\text{ }\lambda^{\tilde{\bm{\mathcal{F}}}}_{z_{\mathfrak{b}}}"]
      \\%3
      &\tilde{\bm{\mathcal{F}}}(z_{\mathfrak{b}})
      &\tilde{\bm{\mathcal{F}}}(z_{\mathfrak{b}})
      \ar[r,equal]
      &\tilde{\bm{\mathcal{F}}}(z_{\mathfrak{b}})
      &\tilde{\bm{\mathcal{F}}}(z_{\mathfrak{b}})
      \ar[r,equal]
      &\tilde{\bm{\mathcal{F}}}(z_{\mathfrak{b}})
      &\tilde{\bm{\mathcal{F}}}(z_{\mathfrak{b}})
    \end{tikzcd}
  \end{equation*}
  After right-cancelling the epimorphism
  $\mathit{coeq}_{z_{\mathfrak{b}},\text{ }\!\!_{\mathfrak{b}}\bm{\mathcal{F}}(b_{\mathfrak{b}})}$
  in the above diagram, we obtain the relation (\ref{eq lambdaF CnaturalinF}).
  This completes the proof of Lemma~\ref{lem lambdaF CnaturalinF}.
\end{proof}
Let $\text{ }\!\!_{\mathfrak{b}}X$ be a left $\mathfrak{b}$-module object in $\bm{T}$.
The functor
$\mathlcal{C}\text{-}\mathit{Funct}(\bm{\mathcal{M}}_{\mathfrak{b}},\bm{T})
\rightsquigarrow \text{ }\!\!_{\mathfrak{b}}T$
in (\ref{eq rightCadj}) sends the $\mathlcal{C}$-enriched functor
$-\circledast_{\mathfrak{b}}\text{ }\!\!_{\mathfrak{b}} X:\bm{\mathcal{M}}_{\mathfrak{b}}\rightsquigarrow\bm{T}$
to the left $\mathfrak{b}$-module object 
$\text{ }\!\!_{\mathfrak{b}}b_{\mathfrak{b}}\circledast_{\mathfrak{b}} \text{ }\!\!_{\mathfrak{b}}X
=\bigl(
  b_{\mathfrak{b}}\circledast_{\mathfrak{b}} \text{ }\!\!_{\mathfrak{b}}X,
  \rho_{b_{\mathfrak{b}}\circledast_{\mathfrak{b}}\text{ }\!\!_{\mathfrak{b}}X}
\bigr)$
in $\bm{T}$, where
\begin{equation*}
  \rho_{b_{\mathfrak{b}}\!\circledast_{\mathfrak{b}}\! \text{ }\!\!_{\mathfrak{b}}X}:
  \begin{tikzcd}[cramped,column sep=45]
    b\!\circledast\! \bigl(b_{\mathfrak{b}}\!\circledast_{\mathfrak{b}}\! \text{ }\!\!_{\mathfrak{b}}X\bigr)
    \ar[r,pos=0.45,"a_{b,b_{\mathfrak{b}},\text{ }\!\!_{\mathfrak{b}}X}","\cong"']
    &\bigl(b\!\circledast\! b_{\mathfrak{b}}\bigr)\!\circledast_{\mathfrak{b}}\! \text{ }\!\!_{\mathfrak{b}}X
    \ar[r,pos=0.45,"\gamma_{b_{\mathfrak{b}}}\circledast_{\mathfrak{b}}\I_{\text{ }\!\!_{\mathfrak{b}}X}"]
    &b_{\mathfrak{b}}\!\circledast_{\mathfrak{b}}\! \text{ }\!\!_{\mathfrak{b}}X
    .
  \end{tikzcd}
\end{equation*}
The isomorphism
$\imath^{\mathfrak{b}}_{\text{ }\!\!_{\mathfrak{b}}X}:b_{\mathfrak{b}}\circledast_{\mathfrak{b}}\text{ }\!\!_{\mathfrak{b}}X \xrightarrow[]{\cong} X$
in $\bm{T}$ becomes an isomorphism
$\imath^{\mathfrak{b}}_{\text{ }\!\!_{\mathfrak{b}}X}:
\text{ }\!\!_{\mathfrak{b}}b_{\mathfrak{b}}\!\circledast_{\mathfrak{b}}\! \text{ }\!\!_{\mathfrak{b}}X
\xrightarrow[]{\cong}\text{ }\!\!_{\mathfrak{b}}X$
of left $\mathfrak{b}$-module objects in $\bm{T}$.
We are ready to prove Theorem~\ref{thm EWthm cocomCcat}.

\begin{proof}
  (of Theorem~\ref{thm EWthm cocomCcat})
  We claim that the functor
  $\text{ }\!\!_{\mathfrak{b}}T
  \rightsquigarrow \mathlcal{C}\text{-}\mathit{Funct}(\bm{\mathcal{M}}_{\mathfrak{b}},\bm{T})$
  in (\ref{eq leftadj}) is left adjoint to the functor
  $\mathlcal{C}\text{-}\mathit{Funct}(\bm{\mathcal{M}}_{\mathfrak{b}},\bm{T})
  \rightsquigarrow \text{ }\!\!_{\mathfrak{b}}T$ in (\ref{eq rightCadj})
  with the following unit, counit natural transformations.
  Let $\text{ }\!\!_{\mathfrak{b}}X$ be a left $\mathfrak{b}$-module object in $\bm{T}$
  and let $\bm{\mathcal{F}}:\bm{\mathcal{M}}_{\mathfrak{b}}\rightsquigarrow\bm{T}$
  be a $\mathlcal{C}$-enriched functor.
  \begin{equation}\label{eq EWthm cocomCcat}
    \begin{tikzcd}[cramped,column sep=50]
      \text{ }\!\!_{\mathfrak{b}}T
      \ar[r,shift left=2,squiggly]
      &{\mathlcal{C}\text{-}\mathit{Funct}(\bm{\mathcal{M}}_{\mathfrak{b}},\bm{T})}
      \ar[l,shift left=2,squiggly]
    \end{tikzcd}
  \end{equation}
  \begin{itemize}
    \item 
    The component of the unit natural transformation at
    $\text{ }\!\!_{\mathfrak{b}}X$
    is the isomorphism 
    $(\imath^{\mathfrak{b}}_{\text{ }\!\!_{\mathfrak{b}}X})^{-1}:
    \begin{tikzcd}[cramped,column sep=18]
      \text{ }\!\!_{\mathfrak{b}}X
      \ar[r,pos=0.45,"\cong"]
      &\text{ }\!\!_{\mathfrak{b}}b_{\mathfrak{b}}\!\circledast_{\mathfrak{b}}\! \text{ }\!\!_{\mathfrak{b}}X
    \end{tikzcd}$
    of left $\mathfrak{b}$-module objects in $\bm{T}$.
    
    \item
    The component of the counit natural transformation at
    $\bm{\mathcal{F}}$
    is the $\mathlcal{C}$-enriched natural transformation
    $\bm{\lambda}^{\bm{\mathcal{F}}}:
    \begin{tikzcd}[cramped,column sep=18]
      -\!\circledast_{\mathfrak{b}}\! \text{ }\!\!_{\mathfrak{b}}\bm{\mathcal{F}}(b_{\mathfrak{b}})
      \ar[r,Rightarrow]
      &\bm{\mathcal{F}}
    \end{tikzcd}$
    defined in Proposition~\ref{prop lambdaF}.
  \end{itemize}
  We leave for the readers to check that
  $(\imath^{\mathfrak{b}}_{\text{ }\!\!_{\mathfrak{b}}X})^{-1}:
  \text{ }\!\!_{\mathfrak{b}}X
  \xrightarrow[]{\cong}
  \text{ }\!\!_{\mathfrak{b}}b_{\mathfrak{b}}\circledast_{\mathfrak{b}}\text{ }\!\!_{\mathfrak{b}}X$
  is natural in variable $\text{ }\!\!_{\mathfrak{b}}X$.
  We showed in Lemma~\ref{lem lambdaF CnaturalinF}
  that the collection of $\bm{\lambda}^{\bm{\mathcal{F}}}$
  is natural in variable $\bm{\mathcal{F}}$.
  We are left to check the following unit-counit relations.
  \begin{equation*}
    \begin{tikzcd}[cramped,row sep=13,column sep=40]
      -\!\circledast_{\mathfrak{b}}\! \text{ }\!\!_{\mathfrak{b}}X
      \ar[r,Rightarrow,"-\circledast_{\mathfrak{b}}(\imath^{\mathfrak{b}}_{\text{ }\!\!_{\mathfrak{b}}X})^{-1}"]
      \ar[dr,bend right=18,equal]
      &-\!\circledast_{\mathfrak{b}}\! \bigl(\text{ }\!\!_{\mathfrak{b}}b_{\mathfrak{b}}\!\circledast_{\mathfrak{b}}\! \text{ }\!\!_{\mathfrak{b}}X\bigr)
      \ar[d,Rightarrow,"\text{ }\bm{\lambda}^{-\circledast_{\mathfrak{b}}\text{ }\!\!_{\mathfrak{b}}X}"]
      \\%1
      \text{ }
      &-\!\circledast_{\mathfrak{b}}\! \text{ }\!\!_{\mathfrak{b}}X        
    \end{tikzcd}
    \qquad\quad
    \begin{tikzcd}[cramped,row sep=13,column sep=40]
      \text{ }\!\!_{\mathfrak{b}}\bm{\mathcal{F}}(b_{\mathfrak{b}})
      \ar[r,"(\imath^{\mathfrak{b}}_{\text{ }\!\!_{\mathfrak{b}}\bm{\mathcal{F}}(b_{\mathfrak{b}})})^{-1}","\cong\text{ }"']
      \ar[dr,bend right=18,equal]
      &\text{ }\!\!_{\mathfrak{b}}b_{\mathfrak{b}}\!\circledast_{\mathfrak{b}}\! \text{ }\!\!_{\mathfrak{b}}\bm{\mathcal{F}}(b_{\mathfrak{b}})
      \ar[d,"\text{ }\lambda^{\bm{\mathcal{F}}}_{b_{\mathfrak{b}}}"]
      \\%1
      \text{ }
      &\text{ }\!\!_{\mathfrak{b}}\bm{\mathcal{F}}(b_{\mathfrak{b}})
    \end{tikzcd}
  \end{equation*}
  The relation on the right is immediate.
  To verify the other relation, let $z_{\mathfrak{b}}\in\Obj(\bm{\mathcal{M}}_{\mathfrak{b}})$
  and consider the following diagram.
  \begin{equation*}
    \hspace*{-0.6cm}
    \begin{tikzcd}[cramped,row sep=13,column sep=10]
      &z\!\circledast\! X
      \ar[d,two heads,"\text{ }\mathit{coeq}_{z_{\mathfrak{b}},\text{ }\!\!_{\mathfrak{b}}X}"]
      \ar[r,equal]
      &z\!\circledast\! X
      \ar[ddd,"\text{ }\I_z\circledast(\imath^{\mathfrak{b}}_{\text{ }\!\!_{\mathfrak{b}}X})^{-1}","\cong\text{ }"']
      \ar[r,equal]
      &z\!\circledast\! X
      \ar[d,"\text{ }\I_z\circledast \imath_X^{-1}","\cong\text{ }"']
      &z\!\circledast\! X
      \ar[d,"\text{ }\I_z\circledast \imath_X^{-1}","\cong\text{ }"']
      &z\!\circledast\! X
      \ar[d,"\text{ }\I_z\circledast \imath_X^{-1}","\cong\text{ }"']
      \ar[r,equal]
      &z\!\circledast\! X
      \ar[dd,"\text{ }\jmath^{-1}_z\circledast\I_X","\cong\text{ }"']
      \ar[r,equal]
      &z\!\circledast\! X
      \ar[dddd,equal]
      \\%1
      &z_{\mathfrak{b}}\!\circledast_{\mathfrak{b}}\! \text{ }\!\!_{\mathfrak{b}}X
      \ar[ddd,"\text{ }\I_{z_{\mathfrak{b}}}\circledast_{\mathfrak{b}}(\imath^{\mathfrak{b}}_{\text{ }\!\!_{\mathfrak{b}}X})^{-1}","\cong\text{ }"']
      &\text{ }
      &z\!\circledast\! \bigl(c\!\circledast\! X\bigr)
      \ar[d,"\text{ }\I_z\circledast(u_b\circledast\I_X)"]
      &z\!\circledast\! \bigl(c\!\circledast\! X\bigr)
      \ar[d,"\text{ }\I_z\circledast(u_b\circledast\I_X)"]
      \ar[r,equal]
      &z\!\circledast\! \bigl(c\!\circledast\! X\bigr)
      \ar[d,"\text{ }a_{z,c,X}","\cong\text{ }"']
      &\text{ }
      &\text{ }
      \\%2
      &\text{ }
      &\text{ }
      &z\!\circledast\! \bigl(b\!\circledast\! X\bigr)
      \ar[d,two heads,"\text{ }\I_z\circledast \mathit{coeq}_{b_{\mathfrak{b}},\text{ }\!\!_{\mathfrak{b}}X}"]
      \ar[r,equal]
      &z\!\circledast\! \bigl(b\!\circledast\! X\bigr)
      \ar[d,"\text{ }a_{z,b,X}","\cong\text{ }"']
      &\bigl(z\!\circledast\! c\bigr)\!\circledast\! X
      \ar[d,"\text{ }(\I_z\circledast u_b)\circledast\I_X"]
      \ar[r,equal]
      &\bigl(z\!\circledast\! c\bigr)\!\circledast\! X
      \ar[d,"\text{ }(\I_z\circledast u_b)\circledast\I_X"]
      &\text{ }
      \\%3
      &\text{ }
      &z\!\circledast\! \bigl(b_{\mathfrak{b}}\!\circledast_{\mathfrak{b}}\! \text{ }\!\!_{\mathfrak{b}}X\bigr)
      \ar[d,two heads,"\text{ }\mathit{coeq}_{z_{\mathfrak{b}},\text{ }\!\!_{\mathfrak{b}}b_{\mathfrak{b}}\circledast_{\mathfrak{b}}\text{ }\!\!_{\mathfrak{b}}X}"]
      \ar[r,equal]
      &z\!\circledast\! \bigl(b_{\mathfrak{b}}\!\circledast_{\mathfrak{b}}\! \text{ }\!\!_{\mathfrak{b}}X\bigr)
      \ar[d,"\text{ }t^{-\circledast_{\mathfrak{b}}\text{ }\!\!_{\mathfrak{b}}X}_{z,b_{\mathfrak{b}}}=a_{z,b_{\mathfrak{b}},\text{ }\!\!_{\mathfrak{b}}X}","\cong\text{ }"']
      &\bigl(z\!\circledast\! b\bigr)\!\circledast\! X
      \ar[d,two heads,"\text{ }\mathit{coeq}_{z\circledast b_{\mathfrak{b}},\text{ }\!\!_{\mathfrak{b}}X}"]
      \ar[r,equal]
      &\bigl(z\!\circledast\! b\bigr)\!\circledast\! X
      \ar[d,"\text{ }\gamma_z\circledast \I_X"]
      &\bigl(z\!\circledast\! b\bigr)\!\circledast\! X
      \ar[d,"\text{ }\gamma_z\circledast \I_X"]
      &\text{ }
      \\%4
      &z_{\mathfrak{b}}\!\circledast_{\mathfrak{b}}\! \bigl(\text{ }\!\!_{\mathfrak{b}}b_{\mathfrak{b}}\!\circledast_{\mathfrak{b}}\! \text{ }\!\!_{\mathfrak{b}}X\bigr)
      \ar[d,"\text{ }\lambda^{-\circledast_{\mathfrak{b}}\text{ }\!\!_{\mathfrak{b}}X}_{z_{\mathfrak{b}}}"]
      \ar[r,equal]
      &z_{\mathfrak{b}}\!\circledast_{\mathfrak{b}}\! \bigl(\text{ }\!\!_{\mathfrak{b}}b_{\mathfrak{b}}\!\circledast_{\mathfrak{b}}\! \text{ }\!\!_{\mathfrak{b}}X\bigr)
      \ar[d,"\text{ }\lambda^{-\circledast_{\mathfrak{b}}\text{ }\!\!_{\mathfrak{b}}X}_{z_{\mathfrak{b}}}"]
      &\bigl(z\!\circledast\! b_{\mathfrak{b}}\bigr)\!\circledast_{\mathfrak{b}}\! \text{ }\!\!_{\mathfrak{b}}X
      \ar[d,"\text{ }\gamma_{z_{\mathfrak{b}}}\circledast_{\mathfrak{b}}\I_{\text{ }\!\!_{\mathfrak{b}}X}"]
      \ar[r,equal]
      &\bigl(z\!\circledast\! b_{\mathfrak{b}}\bigr)\!\circledast_{\mathfrak{b}}\! \text{ }\!\!_{\mathfrak{b}}X
      \ar[d,"\text{ }\gamma_{z_{\mathfrak{b}}}\circledast_{\mathfrak{b}}\I_{\text{ }\!\!_{\mathfrak{b}}X}"]
      &z\!\circledast\! X
      \ar[d,two heads,"\text{ }\mathit{coeq}_{z_{\mathfrak{b}},\text{ }\!\!_{\mathfrak{b}}X}"]
      &z\!\circledast\! X
      \ar[d,two heads,"\text{ }\mathit{coeq}_{z_{\mathfrak{b}},\text{ }\!\!_{\mathfrak{b}}X}"]
      \ar[r,equal]
      &z\!\circledast\! X
      \ar[d,two heads,"\text{ }\mathit{coeq}_{z_{\mathfrak{b}},\text{ }\!\!_{\mathfrak{b}}X}"]
      \\%5
      &z_{\mathfrak{b}}\!\circledast_{\mathfrak{b}}\! \text{ }\!\!_{\mathfrak{b}}X
      &z_{\mathfrak{b}}\!\circledast_{\mathfrak{b}}\! \text{ }\!\!_{\mathfrak{b}}X
      \ar[r,equal]
      &z_{\mathfrak{b}}\!\circledast_{\mathfrak{b}}\! \text{ }\!\!_{\mathfrak{b}}X
      &z_{\mathfrak{b}}\!\circledast_{\mathfrak{b}}\! \text{ }\!\!_{\mathfrak{b}}X
      \ar[r,equal]
      &z_{\mathfrak{b}}\!\circledast_{\mathfrak{b}}\! \text{ }\!\!_{\mathfrak{b}}X
      &z_{\mathfrak{b}}\!\circledast_{\mathfrak{b}}\! \text{ }\!\!_{\mathfrak{b}}X
      &z_{\mathfrak{b}}\!\circledast_{\mathfrak{b}}\! \text{ }\!\!_{\mathfrak{b}}X
    \end{tikzcd}
  \end{equation*}
  We obtain the other relation after right-cancelling the epimorphism
  $\mathit{coeq}_{z_{\mathfrak{b}},\text{ }\!\!_{\mathfrak{b}}X}$ in the diagram above.
  Thus we obtain the adjunction (\ref{eq EWthm cocomCcat}) whose unit is a natural isomorphism.
  By Proposition~\ref{prop lambdaF},
  $\bm{\lambda}^{\bm{\mathcal{F}}}$
  is a $\mathlcal{C}$-enriched natural isomorphism
  if and only if the $\mathlcal{C}$-enriched functor
  $\bm{\mathcal{F}}:\bm{\mathcal{M}}_{\mathfrak{b}}\rightsquigarrow\bm{T}$ is cocontinuous.
  We conclude that the left adjoint functor 
  $\text{ }\!\!_{\mathfrak{b}}T
  \rightsquigarrow \mathlcal{C}\text{-}\mathit{Funct}(\bm{\mathcal{M}}_{\mathfrak{b}},\bm{T})$
  in (\ref{eq leftadj}) is fully faithful,
  whose essential image is the coreflective full subcategory
  $\mathlcal{C}\text{-}\mathit{Funct}_{\mathit{cocon}}(\bm{\mathcal{M}}_{\mathfrak{b}},\bm{T})$
  of cocontinuous $\mathlcal{C}$-enriched functors
  $\bm{\mathcal{F}}:\bm{\mathcal{M}}_{\mathfrak{b}}\rightsquigarrow\bm{T}$.
  This completes the proof of Theorem~\ref{thm EWthm cocomCcat}.
\end{proof}

Theorem~\ref{thm EWthm cocomCcat} implies Theorem~\ref{thm Intro EWthm}.
Let $\mathfrak{b}'$ be another monoid in $\mathlcal{C}$.
By substituting $\bm{T}=\bm{\mathcal{M}}_{\mathfrak{b}'}$
in Theorem~\ref{thm Intro EWthm},
we obtain Corollary~\ref{cor Intro EWthm}.





\section{Morita Theory} \label{sec Morita}

\begin{definition}
  Let $\bm{T}$ be a $\mathlcal{C}$-enriched category
  and let $X\in\Obj(\bm{T})$.
  We say
  \begin{itemize}
    \item[(i)]
    $X$ is $\mathlcal{C}$-enriched compact
    if the $\mathlcal{C}$-enriched functor
    $\bm{T}(X,-):\bm{T}\rightsquigarrow \bm{\mathscr{C}}$
    is cocontinuous;
 
    \item[(ii)]
    $X$ is a $\mathlcal{C}$-enriched generator
    if the $\mathlcal{C}$-enriched functor
    $\bm{T}(X,-):\bm{T}\rightsquigarrow \bm{\mathscr{C}}$
    is conservative;
  
    \item[(iii)]
    $X$ is a $\mathlcal{C}$-enriched compact generator
    if the $\mathlcal{C}$-enriched functor
    $\bm{T}(X,-):\bm{T}\rightsquigarrow \bm{\mathscr{C}}$
    is cocontinuous and conservative.
  \end{itemize}    
\end{definition}

\begin{example}
  Consider the case of $\mathlcal{C}=\mathlcal{Ab}$.
  Let $R$ be a ring and let $\mathit{Mod}_R$ be the category of right $R$-modules.
  For each object $N_R$ in $\mathit{Mod}_R$,
  \begin{itemize}
    \item[(i)]
    $N_R$ is $\mathlcal{Ab}$-compact if and only if it is a finitely generated projective right $R$-module;
    
    \item[(ii)]
    $N_R$ is an $\mathlcal{Ab}$-generator if and only if it is a generator in $\mathit{Mod}_R$;

    \item[(iii)]
    $N_R$ is an $\mathlcal{Ab}$-compact generator if and only if it is a finitely generated projective generator
    in $\mathit{Mod}_R$.
  \end{itemize}
\end{example}

For each monoid $\mathfrak{b}=\bigl(b,u_b,m_b\bigr)$ in $\mathlcal{C}$,
we have a $\mathlcal{C}$-enriched compact generator $b_{\mathfrak{b}}$ in $\bm{\mathcal{M}}_{\mathfrak{b}}$.
This is because the forgetful $\mathlcal{C}$-enriched functor
$\bm{\mathcal{M}}_{\mathfrak{b}}\rightsquigarrow\bm{\mathscr{C}}$ is cocontinuous and conservative,
and the collection of isomorphisms $\bar{\gamma}_{z_{\mathfrak{b}}}:z\xrightarrow[]{\cong}\bm{\mathcal{M}}_{\mathfrak{b}}(b_{\mathfrak{b}},z_{\mathfrak{b}})$
in $\bm{\mathscr{C}}$
is $\mathlcal{C}$-enriched natural in variable $z_{\mathfrak{b}}\in\Obj(\bm{\mathcal{M}}_{\mathfrak{b}})$.
Moreover, the isomorphism
$\bar{\gamma}_{b_{\mathfrak{b}}}:b\xrightarrow[]{\cong} \bm{\mathcal{M}}_{\mathfrak{b}}(b_{\mathfrak{b}},b_{\mathfrak{b}})$
in $\bm{\mathscr{C}}$
becomes an isomorphism of monoids
$\mathfrak{b}\cong\mathit{End}_{\bm{\mathcal{M}}_{\mathfrak{b}}}(b_{\mathfrak{b}})$
in $\mathlcal{C}$.

\begin{theorem} \label{thm Morita CharacterizingMb}
  Let $\mathfrak{b}$ be a monoid in $\mathlcal{C}$.
  A tensored $\mathlcal{C}$-enriched category $\bm{T}$ which has weighted coequalizers
  is equivalent to 
  $\bm{\mathcal{M}}_{\mathfrak{b}}$ if and only if
  there exists a $\mathlcal{C}$-enriched compact generator $X$ in $\bm{T}$
  together with an isomorphism of monoids 
  $\mathfrak{b}\cong \mathit{End}_{\bm{T}}(X)$ in $\mathlcal{C}$.
\end{theorem}
\begin{proof}
  We only need to prove the if part.
  Let us denote 
  $f:b\xrightarrow[]{\cong}\bm{T}(X,X)$
  as the isomorphism in $\bm{\mathscr{C}}$
  and let $\rho_X:=\mathit{Ev}_{X,X}\circ\bigl(f\circledast\I_X\bigr)
  :b\circledast X\xrightarrow[]{\cong} \bm{T}(X,X)\circledast X\to X$
  be the corresponding morphism in $\bm{T}$.
  Then the pair $\text{ }\!\!_{\mathfrak{b}}X=(X,\rho_X)$
  is a left $\mathfrak{b}$-module object in $\bm{T}$.
  By Proposition~\ref{prop bXtensorHomCadj}, we have the following 
  adjoint pair of $\mathlcal{C}$-enriched functors.
  \begin{equation} \label{eq Morita CharacterizingMb}
    \begin{tikzcd}[column sep=70]
      \bm{\mathcal{M}}_{\mathfrak{b}}
      \ar[r,squiggly,shift left=1,bend left=15,"{\bm{\alpha}\text{ }\!:=\text{ }\!-\text{ }\!\circledast_{\mathfrak{b}}\text{ }\!\!_{\mathfrak{b}}X}"]
      &
      \bm{T}
      \ar[l,squiggly,shift left=1,bend left=15,"{\bm{\beta}\text{ }\!:=\text{ }\!\bm{T}(\text{ }\!\!_{\mathfrak{b}}X,-)}"]
    \end{tikzcd}
  \end{equation}
  The right adjoint $\mathlcal{C}$-enriched functor
  $\bm{\beta}:\bm{T}\rightsquigarrow\bm{\mathcal{M}}_{\mathfrak{b}}$
  preserves tensored objects and weighted coequalizers.
  This is because the $\mathlcal{C}$-enriched functor
  $\bm{T}(X,-):\bm{T}\rightsquigarrow\bm{\mathscr{C}}$ is cocontinuous,
  and the forgetful $\mathlcal{C}$-enriched functor
  $\bm{\mathcal{M}}_{\mathfrak{b}}\rightsquigarrow\bm{\mathscr{C}}$
  is cocontinuous and conservative.
  By Proposition~\ref{prop lambdaF},
  we obtain that the $\mathlcal{C}$-enriched functor
  $\bm{\beta}\bm{\alpha}:\bm{\mathcal{M}}_{\mathfrak{b}}\rightsquigarrow\bm{\mathcal{M}}_{\mathfrak{b}}$
  is cocontinuous.

  We claim that the adjunction (\ref{eq Morita CharacterizingMb})
  is an equivalence of $\mathlcal{C}$-enriched categories.
  We first show that the unit
  $\bm{\eta}:\bm{I}_{\bm{\mathcal{M}}_{\mathfrak{b}}}\Rightarrow \bm{\beta}\bm{\alpha}:\bm{\mathcal{M}}_{\mathfrak{b}}\rightsquigarrow\bm{\mathcal{M}}_{\mathfrak{b}}$
  is a $\mathlcal{C}$-enriched natural isomorphism.
  By Corollary~\ref{cor Intro EWthm},
  it suffices to show that
  $\eta_{b_{\mathfrak{b}}}:b_{\mathfrak{b}}\to \bm{T}\bigl(\text{ }\!\!_{\mathfrak{b}}X,b_{\mathfrak{b}}\circledast_{\mathfrak{b}} \text{ }\!\!_{\mathfrak{b}}X\bigr)$
  is an isomorphism in $\bm{\mathcal{M}}_{\mathfrak{b}}$.
  The morphism 
  $\eta_{b_{\mathfrak{b}}}:b\to \bm{T}\bigl(X, b_{\mathfrak{b}}\circledast_{\mathfrak{b}} \text{ }\!\!_{\mathfrak{b}}X\bigr)$ 
  in $\bm{\mathscr{C}}$ is equal to
  $(\imath^{\mathfrak{b}}_{\text{ }\!\!_{\mathfrak{b}}X})^{-1}_{\star}\circ f
  :b\xrightarrow[]{\cong}\bm{T}(X,X)\xrightarrow[]{\cong}\bm{T}\bigl(X,b_{\mathfrak{b}}\circledast_{\mathfrak{b}} \text{ }\!\!_{\mathfrak{b}}X\bigr)$
  as we can see from the diagram below.
  \begin{equation*}
    \hspace*{-0.3cm}
    \begin{tikzcd}[cramped,row sep=13,column sep=8]
      &b
      \ar[d,"\text{ }\mathit{Cv}_{b,X}"]
      \ar[dd,shift right=7,bend right=30,"\eta_{b_{\mathfrak{b}}}"']
      &b
      \ar[d,"\text{ }\mathit{Cv}_{b,X}"]
      &b
      \ar[d,"\text{ }\mathit{Cv}_{b,X}"]
      \ar[r,equal]
      &b
      \ar[d,"\text{ }f","\cong\text{ }"']
      &b
      \ar[d,"\text{ }f","\cong\text{ }"']
      \\%1
      &\bm{T}\bigl(X,b\!\circledast\! X\bigr)
      \ar[d,"\text{ }{(\mathit{coeq}_{b_{\mathfrak{b}},\text{ }\!\!_{\mathfrak{b}}X})_{\star}}"]
      \ar[r,equal]
      &\bm{T}\bigl(X,b\!\circledast\! X\bigr)
      \ar[dd,"\text{ }{(\rho_X)_{\star}}"]
      \ar[r,equal]
      &\bm{T}\bigl(X,b\!\circledast\! X\bigr)
      \ar[d,"\text{ }{(f\circledast\I_X)_{\star}}","\cong\text{ }"']
      &\bm{T}(X,X)
      \ar[d,"\text{ }\mathit{Cv}_{\bm{T}(X,X),X}"]
      \ar[r,equal]
      &\bm{T}(X,X)
      \ar[dd,equal]
      \\%2
      &\bm{T}\bigl(X,b_{\mathfrak{b}}\!\circledast_{\mathfrak{b}}\! \text{ }\!\!_{\mathfrak{b}}X\bigr)
      \ar[d,"\text{ }{(\imath^{\mathfrak{b}}_{\text{ }\!\!_{\mathfrak{b}}X})_{\star}}","\cong\text{ }"']
      &\text{ }
      &\bm{T}\bigl(X,\bm{T}(X,X)\!\circledast\! X\bigr)
      \ar[d,"\text{ }{(\mathit{Ev}_{X,X})_{\star}}"]
      \ar[r,equal]
      &\bm{T}\bigl(X,\bm{T}(X,X)\!\circledast\! X\bigr)
      \ar[d,"\text{ }{(\mathit{Ev}_{X,X})_{\star}}"]
      &\text{ }
      \\%3
      &\bm{T}(X,X)
      \ar[r,equal]
      &\bm{T}(X,X)
      \ar[r,equal]
      &\bm{T}(X,X)
      &\bm{T}(X,X)
      \ar[r,equal]
      &\bm{T}(X,X)
    \end{tikzcd}
  \end{equation*}    
  This shows that the unit $\bm{\eta}:\bm{I}_{\bm{\mathcal{M}}_{\mathfrak{b}}}\Rightarrow \bm{\beta}\bm{\alpha}$
  is a $\mathlcal{C}$-enriched natural isomorphism.

  We are left to show that the counit $\bm{\varepsilon}:\bm{\alpha}\bm{\beta}\Rightarrow\bm{I}_{\bm{T}}:\bm{T}\rightsquigarrow\bm{T}$
  is a $\mathlcal{C}$-enriched natural isomorphism.
  For each $Y\in\Obj(\bm{T})$, the component
  $\varepsilon_Y:\bm{\alpha}\bm{\beta}(Y)\to Y$
  of the counit is explicitly given as follows.
  \begin{equation*}
    \begin{tikzcd}[cramped,row sep=20,column sep=60]
      \bm{T}(X,Y)\!\circledast\! b\!\circledast\! X
      \ar[d,"\I_{\bm{T}(X,Y)}\circledast f\circledast \I_X\text{ }"',"\text{ }\cong"]
      \ar[r,shift left=0.7,"{\gamma_{\bm{T}(X,Y)}\circledast \I_X}"]
      \ar[r,shift right=0.7,"{\I_{\bm{T}(X,Y)}\circledast \rho_X}"']
      &\bm{T}(X,Y)\!\circledast\! X
      \ar[d,equal]
      \ar[r,two heads,pos=0.45,"{\mathit{coeq}_{\bm{T}(\text{ }\!\!_{\mathfrak{b}}X,Y),\text{ }\!\!_{\mathfrak{b}}X}}"]
      &\bm{\alpha}\bm{\beta}(Y)
      \ar[d,equal]
      \\%1
      \bm{T}(X,Y)\!\circledast\! \bm{T}(X,X)\!\circledast\! X
      \ar[r,shift left=0.7,"{\mu_{X,X,Y}\circledast \I_X}"]
      \ar[r,shift right=0.7,"{\I_{\bm{T}(X,Y)}\circledast \mathit{Ev}_{X,X}}"']
      &\bm{T}(X,Y)\!\circledast\! X
      \ar[dr,bend right=17,pos=0.4,"{\mathit{Ev}_{X,Y}}"']
      \ar[r,two heads,"{\mathit{coeq}}"]
      &\bm{\alpha}\bm{\beta}(Y)
      \ar[d,dotted,"{\text{ }\exists!\text{ }\varepsilon_{Y}}"]
      \\%2
      \text{ }
      &\text{ }
      &Y
    \end{tikzcd}
  \end{equation*}
  After applying the cocontinuous $\mathlcal{C}$-enriched functor
  $\bm{T}(X,-):\bm{T}\rightsquigarrow\bm{\mathscr{C}}$,
  we obtain the following diagram in $\bm{\mathscr{C}}$.
  \begin{equation*}
    \hspace*{-0cm}
    \begin{tikzcd}[cramped,row sep=13,column sep=55]
      \bm{T}(X,Y)\!\circledast\! \bm{T}(X,X)\!\circledast\! \bm{T}(X,X)
      \ar[r,pos=0.45,shift left=0.7,"{\mu_{X,X,Y}\circledast \I_{\bm{T}(X,X)}}"]
      \ar[r,pos=0.45,shift right=0.7,"{\I_{\bm{T}(X,Y)}\circledast \mu_{X,X,X}}"']
      &\bm{T}(X,Y)\!\circledast\! \bm{T}(X,X)
      \ar[dr,pos=0.35,bend right=18,"{\mu_{X,X,Y}}"']
      \ar[r,two heads,"{\mathit{coeq}}"]
      &\bm{T}\bigl(X,\bm{\alpha}\bm{\beta}(Y)\bigr)
      \ar[d,"\text{ }{(\varepsilon_{Y})_{\star}}"]
      \\%1
      \text{ }
      &\text{ }
      &\bm{T}(X,Y)
    \end{tikzcd}
  \end{equation*}
  Since the lower cofork in the above diagram is also a weighted coequalizer diagram in $\bm{\mathscr{C}}$,
  we obtain that $(\varepsilon_Y)_{\star}:\bm{T}\bigl(X,\bm{\alpha}\bm{\beta}(Y)\bigr)\to \bm{T}(X,Y)$
  is an isomorphism in $\bm{\mathscr{C}}$.
  As the $\mathlcal{C}$-enriched functor $\bm{T}(X,-):\bm{T}\rightsquigarrow\bm{\mathscr{C}}$
  is conservative,
  we conclude that $\varepsilon_Y:\bm{\alpha}\bm{\beta}(Y)\to Y$
  is an isomorphism in $\bm{T}$ for every $Y\in\Obj(\bm{T})$.
  This completes the proof of Theorem~\ref{thm Morita CharacterizingMb}.
\end{proof}

\begin{remark}
  Let us weaken the assumption of Theorem~\ref{thm Morita CharacterizingMb}
  and assume that $X\in\Obj(\bm{T})$ is merely $\mathlcal{C}$-enriched compact.
  Then the left adjoint $\mathlcal{C}$-enriched functor
  $\bm{\alpha}:\bm{\mathcal{M}}_{\mathfrak{b}}\rightsquigarrow\bm{T}$
  in (\ref{eq Morita CharacterizingMb})
  induces an equivalence of $\mathlcal{C}$-enriched categories
  from $\bm{\mathcal{M}}_{\mathfrak{b}}$
  to a coreflective full $\mathlcal{C}$-enriched subcategory of $\bm{T}$.
\end{remark}

For each monoid $\mathfrak{b}=\bigl(b,u_b,m_b\bigr)$ in $\mathlcal{C}$,
we have a $\mathlcal{C}$-enriched natural isomorphism
\begin{equation}\label{eq jmathb}
  \begin{tikzcd}[cramped,row sep=13,column sep=50]
    z\!\circledast\! b_{\mathfrak{b}}
    \ar[d,two heads,"\mathit{coeq}_{z_{\mathfrak{b}},\text{ }\!\!_{\mathfrak{b}}b_{\mathfrak{b}}}\text{ }"']
    \ar[dr,pos=0.7,bend left=20,"\gamma_{z_{\mathfrak{b}}}"]
    \\%1
    z_{\mathfrak{b}}\!\circledast_{\mathfrak{b}}\! \text{ }\!\!_{\mathfrak{b}}b_{\mathfrak{b}}
    \ar[r,pos=0.4,dotted,"\exists!\text{ }\jmath^{\mathfrak{b}}_{z_{\mathfrak{b}}}","\cong"']
    &z_{\mathfrak{b}}
  \end{tikzcd}
  \qquad\quad
  \bm{\jmath}^{\mathfrak{b}}:
  \begin{tikzcd}[cramped,column sep=20]
    -\!\circledast_{\mathfrak{b}}\! \text{ }\!\!_{\mathfrak{b}}b_{\mathfrak{b}}    
    \ar[r,Rightarrow,"\cong"]
    &\bm{I}_{\bm{\mathcal{M}}_{\mathfrak{b}}}:
    \bm{\mathcal{M}}_{\mathfrak{b}}\rightsquigarrow\bm{\mathcal{M}}_{\mathfrak{b}}
  \end{tikzcd}
\end{equation}
whose component
at $z_{\mathfrak{b}}=(z,\gamma_z)\in\Obj(\bm{\mathcal{M}}_{\mathfrak{b}})$
is the unique isomorphism
$\jmath^{\mathfrak{b}}_{z_{\mathfrak{b}}}:z_{\mathfrak{b}}\circledast_{\mathfrak{b}}\text{ }\!\!_{\mathfrak{b}}b_{\mathfrak{b}}\xrightarrow[]{\cong}z_{\mathfrak{b}}$
in $\bm{T}$ satisfying the relation
$\gamma_{z_{\mathfrak{b}}}=
\jmath^{\mathfrak{b}}_{z_{\mathfrak{b}}}
\circ
\mathit{coeq}_{z_{\mathfrak{b}},\text{ }\!\!_{\mathfrak{b}}b_{\mathfrak{b}}}$.

Let $\mathfrak{b}'$, $\mathfrak{b}''$ be additional monoids in $\mathlcal{C}$.
Given a $(\mathfrak{b},\mathfrak{b}')$-bimodule
$\text{ }\!\!_{\mathfrak{b}}x_{\mathfrak{b}'}$
and a $(\mathfrak{b}',\mathfrak{b}'')$-bimodule $\text{ }\!\!_{\mathfrak{b}'}y_{\mathfrak{b}''}$,
we have the $(\mathfrak{b},\mathfrak{b}'')$-bimodule
\begin{equation*}
  \text{ }\!\!_{\mathfrak{b}}x_{\mathfrak{b}'}\!\circledast_{\mathfrak{b}'}\! \text{ }\!\!_{\mathfrak{b}'}y_{\mathfrak{b}''}
  =\Bigl(
    x_{\mathfrak{b}'}\!\circledast_{\mathfrak{b}'}\! \text{ }\!\!_{\mathfrak{b}'}y_{\mathfrak{b}''}
    ,\text{ }
    \rho_{x_{\mathfrak{b}'}\circledast_{\mathfrak{b}'} \text{ }\!\!_{\mathfrak{b}'}y_{\mathfrak{b}''}}:
    \begin{tikzcd}[cramped,column sep=20]
      b\!\circledast\! \bigl(x_{\mathfrak{b}'}\!\circledast_{\mathfrak{b}'}\! \text{ }\!\!_{\mathfrak{b}'}y_{\mathfrak{b}''}\bigr)
      \ar[r]
      &x_{\mathfrak{b}'}\!\circledast_{\mathfrak{b}'}\! \text{ }\!\!_{\mathfrak{b}'}y_{\mathfrak{b}''}
    \end{tikzcd}
  \Bigr)
\end{equation*}
whose left $\mathfrak{b}$-action is given by
\begin{equation*}
  \rho_{x_{\mathfrak{b}'}\circledast_{\mathfrak{b}'} \text{ }\!\!_{\mathfrak{b}'}y_{\mathfrak{b}''}}:
  \begin{tikzcd}[cramped,column sep=45]
    b\!\circledast\! \bigl(x_{\mathfrak{b}'}\!\circledast_{\mathfrak{b}'}\! \text{ }\!\!_{\mathfrak{b}'}y_{\mathfrak{b}''}\bigr)
    \ar[r,pos=0.45,"a_{b,x_{\mathfrak{b}'},\text{ }\!\!_{\mathfrak{b}'}y_{\mathfrak{b}''}}","\cong"']
    &\bigl(b\!\circledast\! x_{\mathfrak{b}'}\bigr)\!\circledast_{\mathfrak{b}'}\! \text{ }\!\!_{\mathfrak{b}'}y_{\mathfrak{b}''}
    \ar[r,pos=0.4,"\rho_{x_{\mathfrak{b}'}}\!\circledast_{\mathfrak{b}'} \I_{\text{ }\!\!_{\mathfrak{b}'}y_{\mathfrak{b}''}}"]
    &x_{\mathfrak{b}'}\!\circledast_{\mathfrak{b}'}\! \text{ }\!\!_{\mathfrak{b}'}y_{\mathfrak{b}''}
    .
  \end{tikzcd}
\end{equation*}
We have a $\mathlcal{C}$-enriched natural isomorphism 
\begin{equation} \label{eq associatorisom}
  \bm{a}_{-,\text{ }\!\!_{\mathfrak{b}}x_{\mathfrak{b}}\text{ }\!\!\!\!_{\text{ }^{\pr}},\text{ }\!\!_{\mathfrak{b}}\text{ }\!\!\!\!_{\text{ }^{\pr}}y_{\mathfrak{b}}\text{ }\!\!\!\!_{\text{ }^{\pr\pr}}}:
  \begin{tikzcd}[cramped,column sep=30]
    -\!\circledast_{\mathfrak{b}}\! \bigl(\text{ }\!\!_{\mathfrak{b}}x_{\mathfrak{b}'}\!\circledast_{\mathfrak{b}'}\! \text{ }\!\!_{\mathfrak{b}'}y_{\mathfrak{b}''}\bigr)
    \ar[r,Rightarrow,"\cong"]
    &\bigl(-\!\circledast_{\mathfrak{b}}\! \text{ }\!\!_{\mathfrak{b}}x_{\mathfrak{b}'}\bigr)\!\circledast_{\mathfrak{b}'}\! \text{ }\!\!_{\mathfrak{b}'}y_{\mathfrak{b}''}
  \end{tikzcd}
  :\bm{\mathcal{M}}_{\mathfrak{b}}\rightsquigarrow\bm{\mathcal{M}}_{\mathfrak{b}''}
\end{equation}
whose component $a_{z_{\mathfrak{b}},\text{ }\!\!_{\mathfrak{b}}x_{\mathfrak{b}}\text{ }\!\!\!\!_{\text{ }^{\pr}},\text{ }\!\!_{\mathfrak{b}}\text{ }\!\!\!\!_{\text{ }^{\pr}}y_{\mathfrak{b}}\text{ }\!\!\!\!_{\text{ }^{\pr\pr}}}$
at $z_{\mathfrak{b}}\in\Obj(\bm{\mathcal{M}}_{\mathfrak{b}})$
is the unique morphism in $\bm{\mathcal{M}}_{\mathfrak{b}''}$
which makes the following diagram commutative.
\begin{equation*}
  \begin{tikzcd}[cramped,row sep=13,column sep=70]
    z\!\circledast\! \bigl(x_{\mathfrak{b}'}\!\circledast_{\mathfrak{b}'}\! \text{ }\!\!_{\mathfrak{b}'}y_{\mathfrak{b}''}\bigr)
    \ar[d,two heads,"\mathit{coeq}_{z_{\mathfrak{b}},\text{ }\!\!_{\mathfrak{b}}x_{\mathfrak{b}}\text{ }\!\!\!\!\!_{\text{ }^{\pr}}\circledast_{\mathfrak{b}}\text{ }\!\!\!\!\!_{\text{ }^{\pr}}\text{ }\!\!_{\mathfrak{b}}\text{ }\!\!\!\!\!_{\text{ }^{\pr}}y_{\mathfrak{b}}\text{ }\!\!\!\!\!_{\text{ }^{\pr\pr}}}\text{ }"']
    \ar[r,"a_{z,x_{\mathfrak{b}'},\text{ }\!\!_{\mathfrak{b}'}y_{\mathfrak{b}''}}","\cong"']
    &\bigl(z\!\circledast\! x_{\mathfrak{b}'}\bigr)\!\circledast_{\mathfrak{b}'}\! \text{ }\!\!_{\mathfrak{b}'}y_{\mathfrak{b}''}
    \ar[d,two heads,"\text{ }\mathit{coeq}_{z_{\mathfrak{b}},\text{ }\!\!_{\mathfrak{b}}x_{\mathfrak{b}}\text{ }\!\!\!\!\!_{\text{ }^{\pr}}}\circledast_{\mathfrak{b}'}\I_{\text{ }\!\!_{\mathfrak{b}'}y_{\mathfrak{b}''}}"]
    \\%1
    z_{\mathfrak{b}}\!\circledast_{\mathfrak{b}}\! \bigl(\text{ }\!\!_{\mathfrak{b}}x_{\mathfrak{b}'}\!\circledast_{\mathfrak{b}'}\! \text{ }\!\!_{\mathfrak{b}'}y_{\mathfrak{b}''}\bigr)
    \ar[r,pos=0.45,dotted,"\exists!\text{ }a_{z_{\mathfrak{b}},\text{ }\!\!_{\mathfrak{b}}x_{\mathfrak{b}}\text{ }\!\!\!\!\!_{\text{ }^{\pr}},\text{ }\!\!_{\mathfrak{b}}\text{ }\!\!\!\!\!_{\text{ }^{\pr}}y_{\mathfrak{b}}\text{ }\!\!\!\!\!_{\text{ }^{\pr\pr}}}","\cong"']
    &\bigl(z_{\mathfrak{b}}\!\circledast_{\mathfrak{b}}\! \text{ }\!\!_{\mathfrak{b}}x_{\mathfrak{b}'}\bigr)\!\circledast_{\mathfrak{b}'}\! \text{ }\!\!_{\mathfrak{b}'}y_{\mathfrak{b}''}
  \end{tikzcd}
\end{equation*}
We are ready to prove Theorem~\ref{thm Intro CosMorita}.

\begin{proof}
  (of Theorem~\ref{thm Intro CosMorita})
  By substituting $\bm{T}=\bm{\mathcal{M}}_{\mathfrak{b}'}$ in Theorem~\ref{thm Morita CharacterizingMb},
  we immediately obtain that statements (i), (ii) are equivalent.
  We are left to show that statements (i), (iii) are equivalent.  
  The monoids $\mathfrak{b}$, $\mathfrak{b}'$ in $\mathlcal{C}$
  are Morita equivalent if and only if
  there exist cocontinuous $\mathlcal{C}$-enriched functors
  $\bm{\alpha}:\bm{\mathcal{M}}_{\mathfrak{b}}\rightsquigarrow \bm{\mathcal{M}}_{\mathfrak{b}'}$,
  $\bm{\beta}:\bm{\mathcal{M}}_{\mathfrak{b}'}\rightsquigarrow \bm{\mathcal{M}}_{\mathfrak{b}}$
  together with $\mathlcal{C}$-enriched natural isomorphisms 
  $\bm{\beta}\bm{\alpha}\cong \bm{I}_{\bm{\mathcal{M}}_{\mathfrak{b}}}$,
  $\bm{\alpha}\bm{\beta}\cong \bm{I}_{\bm{\mathcal{M}}_{\mathfrak{b}'}}$.
  By Corollary~\ref{cor Intro EWthm}
  and using the $\mathlcal{C}$-enriched natural isomorphisms
  (\ref{eq jmathb}), (\ref{eq associatorisom}),
  we obtain that the existence of such pair $\bm{\alpha}$, $\bm{\beta}$
  is equivalent to the existence of 
  bimodules $\text{ }\!\!_{\mathfrak{b}}x_{\mathfrak{b}'}$, $\text{ }\!\!_{\mathfrak{b}'}y_{\mathfrak{b}}$
  together with isomorphisms of bimodules
  $\text{ }\!\!_{\mathfrak{b}}x_{\mathfrak{b}'}\circledast_{\mathfrak{b'}}\text{ }\!\!_{\mathfrak{b}'}y_{\mathfrak{b}}\cong \text{ }\!\!_{\mathfrak{b}}b_{\mathfrak{b}}$,
  $\text{ }\!\!_{\mathfrak{b}'}y_{\mathfrak{b}}\circledast_{\mathfrak{b}}\text{ }\!\!_{\mathfrak{b}}x_{\mathfrak{b}'}\cong \text{ }\!\!_{\mathfrak{b}'}b'\text{ }\!\!\!\!_{\mathfrak{b}'}$.
  This completes the proof of Theorem~\ref{thm Intro CosMorita}.
\end{proof}


\ack{
This paper is supported by Basic Science Research Institute Fund, whose NRF grant number is 2021R1A6A1A10042944.}



\begin{thebibliography}{00}

\bibitem{Borceux1994}
F.~Borceux.
\newblock {\em Handbook of categorical algebra. 2}, volume~51 of {\em
  Encyclopedia of Mathematics and its Applications}.
\newblock Cambridge University Press, Cambridge, 1994.
\newblock Categories and structures.

\bibitem{Eilenberg1960}
S.~Eilenberg.
\newblock Abstract description of some basic functors.
\newblock {\em The Journal of the Indian Mathematical Society. New Series},
  24:231--234 (1961), 1960.

\bibitem{Hovey2009}
M.~Hovey.
\newblock The {E}ilenberg-{W}atts theorem in homotopical algebra.
\newblock {\em arXiv e-prints}, page arXiv:0910.3842, Oct. 2009.

\bibitem{Kelly2005}
G.~M. Kelly.
\newblock Basic concepts of enriched category theory.
\newblock {\em Reprints in Theory and Applications of Categories}, (10):vi+137,
  2005.
\newblock Reprint of the 1982 original [Cambridge Univ. Press, Cambridge;
  MR0651714].

\bibitem{Lee2023}
J.~Lee.
\newblock Tensor enriched categorical generalization of the {E}ilenberg-{W}atts
  theorem.
\newblock arXiv:2302.11001.

\bibitem{Morita1958}
K.~Morita.
\newblock Duality for modules and its applications to the theory of rings with
  minimum condition.
\newblock {\em Science Reports of the Tokyo Kyoiku Daigaku, Section A},
  6:83--142, 1958.

\bibitem{Watts1960}
C.~E. Watts.
\newblock Intrinsic characterizations of some additive functors.
\newblock {\em Proceedings of the American Mathematical Society}, 11:5--8,
  1960.

\end{thebibliography}  
\end{document}

