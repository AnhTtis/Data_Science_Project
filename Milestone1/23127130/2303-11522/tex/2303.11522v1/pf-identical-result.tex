\section{Proof of Theorem~\ref{thm:IdenticalResult}} \label{sec:pf-identical-result}

%\begin{proof}
Let $p^*$ be the optimal price, and let $x_{it}^*$ be the optimal production quantity for each suppliers $i$ at each period $t$. We first show that the payment and cost regret are of order $O(\log \log T)$. To this end, first observe by monotonicity of the optimal supplier production in the prices that both payment and cost regret are only accumulated when $p_t>p^*$. Then, as we repeatedly square the precision parameter, the number of sub-phases of repeated squaring necessary to get from a step size of $0.5$ to a step size of $\frac{1}{T}$ is $O(\log \log T)$. Next, observe that the price $p_t>p^*$ at most once in each sub-phase. By the boundedness of the supplier production quantities for prices in the range $[0, 1]$, it follows that the total payment and cost regret are $O(\log \log T)$.

\begin{comment}
    Recall that, given a sequence of prices $p_t$ of Algorithm~\ref{alg:fixed-demand}, we seek to bound the following quantities:
\begin{enumerate}
    \item Unmet Demand: $\sum_{t = 1}^T \left( d - \sum_{i = 1}^n x_{it}^*(p_t) \right)_+$,
    \item Payment Regret: $\sum_{t = 1}^T \left( p_t \sum_{i = 1}^n x_{it}^*(p_t) - p^* d \right)_+$, and
    \item Cost Regret: $\sum_{t = 1}^T \left( \sum_{i = 1}^n c_i^t(x_{it}^*(p_t)) - \sum_{i = 1}^n c_i^t(x_{it}^*(p^*)) \right)_+$.
\end{enumerate}
\end{comment}




We now show that the unmet demand is also $O(\log \log T)$. To this end, first note that there is unmet demand at each period when $p_t < p^*$. However, in each sub-phase other than the first sub-phase and last phase when the price is fixed at the lower end of the feasible price interval, it holds that the length of the feasible interval $[a,b]$ is $\sqrt{\epsilon}$, with sub-intervals of length $\epsilon$. As there are $\frac{1}{\sqrt{\epsilon}}$ sub-intervals at most $\frac{1}{\sqrt{\epsilon}}$ offers are made. Next, we note by Lemma~\ref{lem:ProductionLipschitz} that the optimal production quantity for each supplier, given by the solution to Problem~\eqref{eq:supObj}, is Lipschitz continuous, i.e., $|x_{it}^*(p_t) - x_{it}^*(p^*)| \leq \frac{1}{\mu} |p-p^*|$, where $\mu = \min_{i \in [n]} \mu_i$ where $\mu_i$ is the strong convexity modulus for supplier $i$'s cost function $c_i(\cdot)$. As a result, it holds that
\begin{align*}
    |x_{it}(p_t) - x_{it}(p^*)| \leq \frac{1}{\mu} |p-p^*| \leq \frac{1}{\mu} (b-a) \leq \frac{1}{\mu} \sqrt{\epsilon},
\end{align*}
for all suppliers $i \in [n]$. Next, the total unmet demand at period $t$ is given by
\begin{align*}
    \left( d - \sum_{i = 1}^n x_{it}^*(p_t) \right)_+ \leq \sum_{i = 1}^n |x_{it}^*(p_t) - x_{it}^*(p^*)| \leq n \frac{1}{\mu} \sqrt{\epsilon} = O(n \frac{1}{\mu} \sqrt{\epsilon}).
\end{align*}
Then, since $\frac{1}{\sqrt{\epsilon}}$ offers are made in each sub-phase, it follows that the total unmet demand is $O(n \frac{1}{\mu})$ in each phase. As there are $O(\log \log T)$ sub-phases, it thus holds that the total unmet demand is $O(n \frac{1}{\mu} \log \log T)$ before the final phase when the price is kept fixed at the lower end of the feasible interval. 

In the final phase, note that the length of the feasible interval is less than $\frac{1}{T}$ and by Lemma~\ref{lem:ProductionLipschitz}, it holds that $|x_{it}^*(p_t) - x_{it}^*(p^*)| \leq \frac{1}{\mu T}$. Since there are at most $T$ offers in the final phase, the total unmet demand in this phase is at most $O(n \frac{1}{\mu})$. Thus, we have established that the total unmet demand is $O(n \frac{1}{\mu} \log \log T) = O(\log \log T)$, which proves our claim.
%\end{proof}