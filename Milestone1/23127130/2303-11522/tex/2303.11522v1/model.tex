\section{Model}
\label{sec:model}

In this section, we present the offline model of a market operator seeking to set equilibrium prices in the market to satisfy a customer demand for a commodity (Section~\ref{sec:market-model}) and performance metrics used to evaluate the efficacy of a pricing policy in the online setting (Section~\ref{sec:perf-measures}).

\subsection{Market Model and Equilibrium Pricing} \label{sec:market-model}

We study a market run by an operator seeking to meet the customer demand $d>0$ for a commodity, e.g., energy, by purchasing the required amount from $n$ competing suppliers. Each supplier $i \in [n]$ has a cost function $c_i: \mathbb{R}_{\geq 0} \rightarrow \mathbb{R}_{\geq 0}$, where $c_i(x_i)$ represents the cost incurred by supplier $i$ for producing $x_i$ units of the commodity. Furthermore, to meet the customer demand, the market operator posts a price $p$ in the market, which represents the payment made by the market operator for each unit of the commodity produced by a given supplier. In particular, for producing $x_i$ units of the commodity, a supplier $i$ receives a payment of $p x_i$ from the market operator. Then, given a posted price $p$ for the commodity and a cost function $c_i(\cdot)$, each supplier makes an individual decision on the optimal production quantity $x_i^*(p)$ to maximize their total profit, as described through the following optimization problem
\begin{maxi}|s|[2]                   % mini! = minimize 
    {x_i \geq 0}                               % optimization variable
    {p x_i - c_i(x_i). \label{eq:supObj}}   % objective function and label
    {}             % label for optimizatio problem
    {}                                % optimization result
\end{maxi}

The posted price that suppliers best respond to is set by a market operator that seeks to determine an equilibrium price $p^*$ that satisfies the following three desirable properties: %In particular, denoting $x_i^*(p)$ as the production quantity for each supplier $i \in [n]$ given by the solution to Problem~\eqref{eq:supObj} under a posted price $p$, the goal of the market operator is set a price $p^*$ to achieve the following three desirable properties:

\begin{enumerate}
    \item Market Clearing: The total supply equals the total demand, i.e., $\sum_{i = 1}^n x_i^*(p^*) = d$.
    \item Minimal Supplier Production Cost: The total production cost of all suppliers, given by $\sum_{i = 1}^n c_i(x_i^*(p^*))$, is minimal among all feasible production quantities $x_i \geq 0$ for all suppliers $i \in [n]$ satisfying the customer demand, i.e., $\sum_{i = 1}^n x_i = d$.
    \item Minimal Payment: The total payment made to all suppliers, given by $\sum_{i = 1}^n p^* x_i^*(p^*)$, is minimal among all feasible production quantities $x_i \geq 0$ for all suppliers $i \in [n]$ satisfying the customer demand, i.e., $\sum_{i = 1}^n x_i = d$.
\end{enumerate}
While these properties are, in general, not possible to achieve simultaneously, e.g., in markets where the supplier cost functions are non-convex~\cite{azizan2020optimal}, in markets where the cost functions $c_i(\cdot)$ of all suppliers are convex, there exists an equilibrium price $p^*$ that satisfies the above three properties. Moreover, in markets with convex cost functions, the equilibrium price can be computed through the dual variables of the market clearing constraint of the following convex optimization problem
\begin{mini!}|s|[2]                   % mini! = minimize 
    {x_{i} \geq 0, \forall i \in [n]}                               % optimization variable
    {\sum_{i = 1}^n c_i(x_{i}), \label{eq:supObj2}}   % objective function and label
    {\label{eq:minCost}}             % label for optimizatio problem
    {C^* = }                                % optimization result
    \addConstraint{\sum_{i = 1}^n x_{i}}{= d, \label{eq:demand-con}} 
\end{mini!}
where~\eqref{eq:supObj2} is the minimum supplier production cost objective and~\eqref{eq:demand-con} is the market clearing constraint. 
In particular, from the KKT condition, the optimal solution to Problem~\eqref{eq:supObj2}-\eqref{eq:demand-con} satisfies
\[
\begin{cases}
    \sum_{i=1}^n x^*_i = d, x_i^* \geq 0, \forall \,  i = 1, \dots, n, \\
    \frac{\partial c_i}{\partial x_i}(x^*_i) \ge p^*, \forall \,  i = 1, \dots, n, \\
    \frac{\partial c_i}{\partial x_i}(x^*_i) = p^*, \forall \,  i \text{ s.t. } x_i^*>0,
\end{cases}
\]
so that the optimal dual variable $p^*$ satisfies all conditions of equilibrium pricing.
While the equilibrium price $p^*$ has several desirable properties, such an equilibrium price typically cannot be directly computed by solving Problem~\eqref{eq:supObj2}-\eqref{eq:demand-con} as the cost functions of suppliers are, in general, unknown to the market operator. Furthermore, both the cost functions of the suppliers and the customer demands tend to be time-varying and thus would involve the market operator periodically re-solving Problem~\eqref{eq:supObj2}-\eqref{eq:demand-con} to determine equilibrium prices at short time intervals, which may be computationally prohibitive. To overcome these challenges, in this work, we propose online learning algorithms to learn equilibrium prices over multiple periods in the incomplete information setting when the cost functions of the suppliers are unknown (or only partially known) to the market operator.



\subsection{Performance Metrics to Set Equilibrium Prices in Online Setting} \label{sec:perf-measures}

We now introduce the online learning setting, wherein the market operator sets prices for the commodity over multiple periods, and present the performance metrics to evaluate the efficacy of an online pricing policy. In particular, we consider the setting when the market operator seeks to satisfy the customer demand over multiple periods $t = 1, \ldots, T$. At each period $t \in [T]$, the customer demand for the commodity is given by $d_t$ and each supplier $i \in [n]$ has a private cost function $c_{it}(\cdot)$ that is increasing, continuously differentiable, strongly convex, and normalized to satisfy $c_{it}(0) = 0$. We assume that the demand at each period $t$ lies in a bounded interval, i.e., $d_t \in [\underline{d}, \Bar{d}]$ for all $t$ for some $\underline{d}, \Bar{d}>0$. Furthermore, for ease of exposition, we normalize the set of feasible prices corresponding to any customer demand and realization of supplier cost functions to be such that the corresponding optimal price of the commodity belongs to the normalized interval $[0, 1]$. In addition, we note that we consider strongly convex cost functions of suppliers, as opposed to general convex costs, due to the performance limitations of any online algorithm under the incomplete information setting studied in this work for non-strongly convex cost functions (see Section~\ref{sec:fixed-convex-limitations} for further details). 


In this work, we begin by considering the informational setting wherein the cost functions of the suppliers are fixed over time (see Sections~\ref{sec:fixed-setting} and~\ref{sec:vary-demand}) and upon observing the customer demand $d_t$, the market operator makes a pricing decision $p_t$ that depends on the past observations of supplier productions, i.e., revealed preference feedback in response to set prices as in~\cite{roth2016watch,ji2018social,jalota2022onlinetraffic,jalota2022online}, and the realized customer demands. In particular, over the $T$ periods, the market operator sets a sequence of prices given by the pricing policy $\ppi = (\pi_1, \ldots, \pi_T)$, where $p_t = \pi_t(\{ (x_{it'}^*)_{1 = 1}^n, d_{t'} \}_{t'=1}^{t-1}, d_t)$, where $x_{it}^*$ represents the optimal production quantity corresponding to the solution of Problem~\eqref{eq:supObj} for supplier $i$ at period $t$. 
When the pricing policy is evident from the context, we will overload the notation and simply write $\ppi = (p_1, p_2, \dots, p_T)$.
We then consider the informational setting when suppliers' cost functions are time-varying (see Section~\ref{sec:time-vary-cost-main}), for which we introduce an augmented problem setting and the corresponding class of online pricing policies that we consider in Section~\ref{sec:contextual-bandit}.



We evaluate the efficacy of an online pricing policy $\ppi$ using three regret metrics: (i) unmet demand, (ii) cost regret, and (iii) payment regret. These regret metrics represent the performance loss of the policy $\ppi$ relative to the optimal offline algorithm with complete information on the three desirable properties of equilibrium prices elucidated in Section~\ref{sec:market-model}. We also note that these performance metrics naturally generalize to the augmented problem setting we consider when suppliers' cost functions are time-varying and present the corresponding generalizations of the regret metrics in Appendix~\ref{sec:new-regret-defs} for completeness.

\paragraph{Unmet Demand:} We evaluate the unmet demand of an online pricing policy $\ppi$ as the sum of the differences between the demand and the total supplier productions corresponding to the pricing policy $\ppi$ at each period $t$. In particular, for an online pricing policy $\ppi$ that sets a sequence of prices $p_1, \ldots, p_T$, the cumulative unmet demand is given by
\begin{align*}
    U_T(\ppi) = \sum_{t = 1}^T \left( d_t - \sum_{i = 1}^n x_{it}^*(p_t) \right)_+,
\end{align*}
where $x_{it}^*(p_t)$ is the optimal production quantity corresponding to the solution of Problem~\eqref{eq:supObj} for supplier $i$ at period $t$. 

\paragraph{Cost Regret:} We evaluate the cost regret of an online pricing policy $\ppi$ through the difference between the total supplier production cost corresponding to algorithm $\ppi$ and the minimum total production cost, given complete information on the supplier cost functions. In particular, the cost regret $C_T(\ppi)$ of an algorithm $\ppi$ is given by
\begin{align*}
    C_T(\ppi) = \sum_{t = 1}^T \sum_{i = 1}^n \left(c_{it}(x_{it}^*(p_t)) - c_i^t(x_{it}^*(p_t^*))\right),
\end{align*}
where the price $p_t^*$ for each period $t \in [T]$ is the optimal dual variable of the market clearing constraint of Problem~\eqref{eq:supObj2}-\eqref{eq:demand-con} given the demand $d_t$ and cost functions $c_{it}$ for all $i \in [n]$.

\paragraph{Payment Regret:} Finally, we evaluate the payment regret of online pricing policy $\ppi$ through the difference between the total payment made to all suppliers corresponding to algorithm $\ppi$ and the minimum total payment, given complete information on the supplier cost functions. In particular, the payment regret $P_T(\ppi)$ of an algorithm $\ppi$ is given by
\begin{align*}
    P_T(\ppi) = \sum_{t = 1}^T \sum_{i = 1}^n \left(p_t x_{it}^*(p_t) -  p^* x_{it}^*(p^*) \right).
\end{align*}

In this work, we focus on developing algorithms that jointly optimize these three regret metrics over $T$ periods. 
In particular, because it is desirable that the performance of the pricing policy improves as the market operator receives more information, we intend to design algorithms that guarantee all three regret metrics are sub-linear in $T$.
Note that achieving good performance on one of these metrics is typically easy as setting low prices will lead to low cost and payment regrets while setting very high prices will lead to no unmet demand. Thus, the challenge in simultaneously optimizing these different regret metrics in the incomplete information setting is to find the right price, i.e., the equilibrium price, at which all these regret metrics are kept small. 

%Jointly optimizing multiple regret metrics is common in the online learning literature and has been extensively studied in the context of online constrained convex optimization with long-term constraints wherein the constraints only need to be approximately satisfied in the long run~\cite{yi2021regret,liakopoulos2019cautious,jenatton2016adaptive,mahdavi2012trading,valls2020online}. However, our chosen unmet demand metric is in contrast to the standard constraint violation regret metrics used in these works as well as other studies in online learning~\cite{li2020simple,jalota2022online}. In particular, in these works, resource constraints only need to be approximately satisfied in the long run, but in our setting, unmet demand is accumulated only when the supplier production is less than the demand. In other words, our unmet demand metric considers the setting when over-production by suppliers at particular periods cannot compensate for unmet demand at subsequent periods, in alignment with several real-world markets, such as electricity markets, where the customer demand needs to be satisfied at each period. Formally, we have $U_T(\ppi) = \sum_{t = 1}^T \left( d_t - \sum_{i = 1}^n x_{it}^*(p_t) \right)_+ \geq \left[ \sum_{t = 1}^T \left( d_t - \sum_{i = 1}^n x_{it}^*(p_t) \right) \right]_+$, where the term $\left[ \sum_{t = 1}^T \left( d_t - \sum_{i = 1}^n x_{it}^*(p_t) \right) \right]_+$ corresponds to the setting when the customer demand only needs to be satisfied in the long-run.
%Further, since we obtain regret guarantees for the above unmet demand metric, using techniques from parametric optimization, our regret guarantees naturally extend for the corresponding stronger notions of the payment and cost regret metrics as well. However, we present our payment and cost regret metrics in alignment with the classical regret metrics in the literature, wherein lower payments (costs) at particular periods can compensate for excess payments (costs) at other periods.

A few comments about our regret metrics are in order. First, our unmet demand metric aligns with real-world markets, e.g., electricity markets, where the demand needs to be satisfied at each period, and over-production at particular periods cannot compensate for unmet demand at subsequent periods. 
%Note that our unmet demand metric contrasts and serves as a stronger benchmark than constraint violation metrics in the literature of jointly optimizing multiple regret metrics~\cite{yu2017online,jenatton2016adaptive,mahdavi2012trading}, where resource constraints only need to be approximately satisfied in the long run. 
Therefore, we define our unmet demand metric as a stronger benchmark than the typical constraint violation metrics in the literature of jointly optimizing multiple regret metrics~\cite{yu2017online,jenatton2016adaptive,mahdavi2012trading}, where resource constraints only need to be approximately satisfied in the long run. 
Formally, $U_T(\ppi) = \sum_{t = 1}^T \left( d_t - \sum_{i = 1}^n x_{it}^*(p_t) \right)_+ \geq \left[ \sum_{t = 1}^T \left( d_t - \sum_{i = 1}^n x_{it}^*(p_t) \right) \right]_+$, where the latter term corresponds to the setting when the customer demand only needs to be satisfied in the long-run. Further, since we obtain regret guarantees for the above unmet demand metric, using techniques from parametric optimization, our regret guarantees naturally extend for the corresponding stronger notions of the payment and cost regret metrics as well. However, we present our payment and cost regret metrics in alignment with the classical regret metrics in the literature, wherein lower payments (costs) at particular periods can compensate for excess payments (costs) at other periods.