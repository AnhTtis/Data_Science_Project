\documentclass{article}
\usepackage{amsmath,inputenc}

\title{Online Learning for Equilibrium Pricing in Markets under Incomplete Information}
\author{Devansh Jalota$^{*1}$, Haoyuan Sun$^{*2}$, Navid Azizan$^2$%\\
\thanks{$^*$Equal Contribution}
\thanks{$^1$Institute for Computational and Mathematical Engineering, Stanford University; {\tt djalota@stanford.edu}.}%
\thanks{$^2$Laboratory for Information and Decision Systems, Massachusetts Institute of Technology; {\tt \{haoyuans,azizan\}@mit.edu}}%
%\\
%$^1$Stanford University, Stanford, CA\\
%$^2$Massachusetts Institute of Technology, Cambridge, MA
}
%\thanks{$^1$ {\tt djalota@stanford.edu}}
%\thanks{$^2$ {\tt \{haoyuans,azizan\}@mit.edu}}
\date{March 2023}

% \usepackage[latin1]{inputenc}
\usepackage[british]{babel}
\usepackage[all]{xy}
\usepackage{amscd}
\usepackage{amssymb}
\usepackage{amsthm}
\usepackage{enumitem}
\usepackage{mathrsfs,bbm}
\usepackage{xcolor,graphicx}
\usepackage{graphics}
\usepackage{soul}
\usepackage{comment}
\usepackage[all]{xy}
\usepackage{amscd}
\usepackage{amssymb,amsmath,latexsym}
\usepackage{amsthm}
\usepackage{enumitem}
\usepackage{mathrsfs,bbm}
\usepackage{dsfont}
\usepackage{tikz-cd}
\usepackage[T1]{fontenc}
\usepackage[utf8]{inputenc}  
 %
%%%%%%%%%%%%%%%%%%%%%%%%%%%%%%%%%%
%pagestyle
%%%%%%%%%%%%%%%%%%%%%%%%%%%%%%%%%%
%\pagestyle{plain}
\textwidth=430pt
\headsep=.7cm
\evensidemargin=15pt
\oddsidemargin=15pt
\leftmargin=0cm
\rightmargin=0cm
%%
%%%%%%%%%%%%%%%%%%%%%%%
\newcommand*\fixitem {\item[]%
  \refstepcounter{enumi}\hskip-\leftmargin\labelenumi\hskip\labelsep}
\newtheorem*{mainthm}{Main Theorem}
\newtheorem*{mainthm1}{Theorem}
\newtheorem*{maincor}{Corollary}
\usepackage[colorlinks=true]{hyperref}
\DeclareMathOperator{\Forall}{\forall}
\DeclareMathOperator{\Exists}{\exists}
\DeclareMathOperator{\ord}{ord}
\newcommand{\phiD}{\varphi_D}
\newcommand{\phiDI}{\varphi_{\mathbf{D}_I}}
\newcommand{\phiDIj}{\varphi_{\mathbf{D}_I (j)}}
\newcommand{\phiH}{\varphi_H}
\newcommand{\phiTimes}{\phiD \otimes \phiH}
\newcommand{\phiTimesDI}{\varphi_{\mathbf{D}_I} \otimes \phiH}
\newcommand{\R}{\mathscr{A}}
\newcommand{\X}{\mathscr{X}}
\newcommand{\Xf}{\mathscr{X}_{(k_0 ,i)}[r_0]}
\newcommand{\Xfr}{\mathscr{X}_{(k_0,i)}[r]}
\newcommand{\hotimes}{\widehat{\otimes}}
\newcommand{\C}{\mathbb{C}_p}
\newcommand{\V}{\mathscr{V}}
\newcommand{\B}{\mathscr{B}}
\newcommand{\dualD}{\mathfrak{D}}
\newcommand{\Dg}{\mathbf{D}}
\newcommand{\DD}{\mathcal{D}^0}
\newcommand{\DDg}{\mathcal{D}}
\newcommand{\DV}{\mathcal{D}}
\newcommand{\W}{\mathscr{W}_N}
\newcommand{\Ao}{\mathbf{A}^\circ}
\newcommand{\AoK}{\mathbf{A}^\circ_{\K}}
\newcommand{\AK}{\mathbf{A}_{/\K}}
\newcommand{\OOO}{\mathscr{A}^\circ}
\newcommand{\K}{\mathcal{K}} 
\newcommand{\OK}{\mathcal{O}_{\K}}
\newcommand{\varprojlog}[1]{\underleftarrow{\log\!^{#1}}}
\newcommand{\T}{\mathscr{T}}
\newcommand{\TT}{\mathbf{T}}
\newcommand{\VV}{\mathbf{V}}
\newcommand{\HH}{\mathcal{H}}
\newcommand{\hh}{\mathcal{H}^+}
\newcommand{\HG}[2]{\mathcal{H}_{#1}(#2)}
\newcommand{\hhl}{\mathcal{H}^{+,[l]}}
\newcommand{\hhj}{\mathcal{H}^{+,[j]}}
\newcommand{\hhjj}{\mathcal{H}^{+,[l,l']}}
\newcommand{\GS}{G_{\mathbb{Q},S}}
\newcommand{\Rf}{R_{(k_0 ,i)}[r_0]}
\newcommand{\Rfr}{R_{(k_0 ,i)}[r]}
\newcommand{\parT}{\langle T\rangle}
\newcommand{\Zf}{Z_{(k_0 ,i)}[r_0]}
\newcommand{\Zfr}{\mathscr{Z}_{(k_0 ,i)}[r]}
\newcommand{\ZFf}{\mathscr{Z}_{(k_0 ,i)}[r_0]}
\newcommand{\ZFfr}{\mathscr{Z}_{(k_0 ,i)}[r]}
\newcommand{\ZF}{\mathscr{Z}}
\definecolor{purple}{rgb}{1, 0, 1}

\newcommand{\ie}{\emph{i.e.,}\xspace}
\newcommand{\eg}{\emph{e.g.,}\xspace}
\newcommand{\abr}{\emph{abbr.}\xspace}
\newcommand{\ea}{\emph{et al.}\xspace}
\newcommand{\gensync}{\emph{GenSync}\xspace}
\newcommand{\colosseum}{\emph{Colosseum}\xspace}
\newcommand{\srep}{\emph{SREP}\xspace} % Set Reconciliation Enhances
\newcommand{\srepsim}{\emph{SREPSim}\xspace}
% Propagation
\newcommand{\esrep}{\emph{E-SREP}\xspace}
\newcommand{\epsrep}{\emph{EP-SREP}\xspace}
\newcommand{\mesrep}{\emph{ME-SREP}\xspace}
\newcommand{\mempoolsync}{\emph{MempoolSync}}

\newcommand{\fref}[1]{Fig.~\ref{#1}}
\newcommand{\tref}[1]{Table~\ref{#1}}
\newcommand{\aref}[1]{Algorithm~\ref{#1}}
\newcommand{\procref}[1]{Procedure~\ref{#1}}
\newcommand{\sref}[1]{Section~\ref{#1}}
\newcommand{\lineref}[1]{line~\ref{#1}}
\newcommand{\appref}[1]{Appendix~\ref{#1}}

% Change \eqref
\LetLtxMacro{\originaleqref}{\eqref}
\renewcommand{\eqref}{Eq.~\originaleqref}

% Theorems and corollaries
\newcounter{theoremcount}
\setcounter{theoremcount}{0}
\DeclareRobustCommand{\theorem}[1]{%
  \refstepcounter{theoremcount}%
  \noindent\textit{\textbf{Theorem \thetheoremcount\label{theorem:#1}: }}%
}
\DeclareRobustCommand{\theoremref}[1]{Theorem~\ref{theorem:#1}}

\DeclareRobustCommand{\proof}{\emph{Proof:}\xspace}
\DeclareRobustCommand{\qqed}{\hfill$\blacksquare$}

\newcounter{corollcount}
\setcounter{corollcount}{0}
\DeclareRobustCommand{\coroll}[1]{%
  \refstepcounter{corollcount}%
  \noindent\textit{\textbf{Corollary \thecorollcount\label{coroll:#1}: }}%
}
\DeclareRobustCommand{\corollref}[1]{Corollary~\ref{coroll:#1}}

\newcounter{lemmacount}
\setcounter{lemmacount}{0}
\DeclareRobustCommand{\lemma}[1]{%
  \refstepcounter{lemmacount}%
  \noindent\textit{\textbf{Lemma \thelemmacount\label{lemma:#1}: }}%
}
\DeclareRobustCommand{\lemmaref}[1]{Lemma~\ref{lemma:#1}}

\newcounter{definitioncount}
\setcounter{definitioncount}{0}
\DeclareRobustCommand{\definition}[1]{%
  \refstepcounter{definitioncount}%
  \noindent\textit{\textbf{Definition \thedefinitioncount\label{definition:#1}: }}%
}
\DeclareRobustCommand{\defref}[1]{Definition~\ref{definition:#1}}

%notes of different authors
\newif\ifnotes
\notestrue
\notesfalse

\newif\ifdiff
\difftrue
\difffalse

\newcommand{\anote}[1]{\ifnotes $\ll$\textsf{\textcolor{purple}{Ari: {#1}}}$\gg$ \fi}
\newcommand{\nnote}[1]{\ifnotes $\ll$\textsf{\textcolor{orange}{Novak: {#1}}}$\gg$ \fi}
\newcommand{\diff}[1]{\ifdiff\textcolor{orange}{#1}\else#1\fi}

%%% Local Variables:
%%% mode: latex
%%% TeX-master: "main"
%%% End:


\begin{document}

\maketitle


\begin{abstract}

The study of market \emph{equilibria} is central to economic theory, particularly in efficiently allocating scarce resources. However, the computation of \emph{equilibrium} prices at which the supply of goods matches their demand typically relies on having access to complete information on private attributes of agents, e.g., suppliers' cost functions, which are often unavailable in practice. Motivated by this practical consideration, we consider the problem of setting equilibrium prices in the incomplete information setting wherein a market operator seeks to satisfy the customer demand for a commodity by purchasing the required amount from competing suppliers with privately known cost functions unknown to the market operator. In this incomplete information setting, we consider the online learning problem of learning equilibrium prices over time while jointly optimizing three performance metrics - \emph{unmet demand}, \emph{cost regret}, and \emph{payment regret} – pertinent in the context of equilibrium pricing over a horizon of $T$ periods. We first consider the setting when suppliers' cost functions are fixed and develop algorithms that achieve a regret of $O(\log \log T)$ when the customer demand is constant over time, or $O(\sqrt{T} \log \log T)$ when the demand is variable over time. Next, we consider the setting when the suppliers' cost functions can vary over time and illustrate that no online algorithm can achieve sublinear regret on all three metrics when the market operator has no information about how the cost functions change over time. Thus, we consider an augmented setting wherein the operator has access to hints/contexts that, without revealing the complete specification of the cost functions, reflect the variation in the cost functions over time and propose an algorithm with sublinear regret in this augmented setting.


\end{abstract}

\section{Introduction}
\label{sec:introduction}
% \begin{itemize}
%     % Diffusion of FL
%     \item {\st{Diffusion of FL}}
%     % Security threats to FL
%     \item {\st{Security threats to FL with particular focus on model poisoning}}
%     % Limitations of existing countermeasures
%     \item {\st{Current countermeasures (e.g., KRUM) and their limitations}}
%     % Proposed method and its advantages
%     \item {\st{Intuitive description of the proposed method and its difference (i.e., advantages) w.r.t. state of the art}}
%     % Main contributions
%     \item {\st{Summary of the main contributions of this work}}
%     % Paper's structure and organization
%     \item {\st{Paper's structure and organization}}
% \end{itemize}

% Diffusion of FL
Recently, {\em federated learning} (FL) has emerged as the leading paradigm for training distributed, large-scale, and privacy-preserving machine learning (ML) systems~\cite{mcmahan2017googleai,mcmahan2017aistats}. 
The core idea of FL is to allow multiple edge clients to collaboratively train a shared, global model without disclosing their local private training data.
%Specifically, an FL system consists of a central server and many edge clients; 
A typical FL round involves the following steps: {\em(i)} the server randomly picks some clients and sends them the current, global model; {\em(ii)} each selected client locally trains its model with its own private data; then, it sends the resulting local model to the server;\footnote{Whenever we refer to global/local model, we mean global/local model {\em parameters}.} {\em(iii)} the server updates the global model by computing an \emph{aggregation function}, usually the average (FedAvg), on the local models received from clients.
% \begin{enumerate}
%     \item[{\em(i)}] the server sends the current, global model to the clients and appoints some of them for training;
%     \item[{\em(ii)}] each selected client locally trains its copy of the global model with its own private data; then, it sends the resulting local model back to the server;\footnote{Whenever we refer to global/local model, we mean global/local model {\em parameters}.}
%     \item[{\em(iii)}] the server updates the global model by computing an \emph{aggregation function} on the local models received from clients (by default, the average, also referred to as FedAvg~\cite{mcmahan2017aistats}).
% \end{enumerate}
This process goes on until the global model converges. %(e.g., after a certain number of rounds or other similar stopping criteria).
%\\
% The advantages of FL over the traditional, centralized learning paradigm are undoubtedly clear in terms of flexibility/scalability (clients can join/disconnect from the FL network dynamically), network communications (only model weights\footnote{We will use \textit{parameters} and \textit{weights} interchangeably.} are exchanged between clients and server), and privacy (each client's private training data is kept local at the client's end and not uploaded to the server).
\\
% Security threats to FL
%However, the growing adoption of FL also raises security concerns~\cite{costa2022covert}, particularly about its confidentiality, integrity, and availability.
Although its advantages over standard ML, FL also raises security concerns~\cite{costa2022covert}. %, particularly about its confidentiality, integrity, and availability~\cite{costa2022covert}.
% OLD, LONG VERSION
% Indeed, some work deals with privacy leakage that may expose the local data of some clients~\cite{melis2019sp}. 
% A large body of work, instead, investigates attacks that usually aim to detriment the predictive accuracy of the learned global model. For instance, \emph{data poisoning} attacks achieve this goal by letting an adversary pollute the training set of some corrupt FL clients with maliciously crafted examples~\cite{jagielski2018sp}.
% Similarly, in \emph{model poisoning} the attacker attempts to tweak the global model weights~\cite{bhagoji2019pmlr} by directly perturbing the local model's weights of some infected FL clients before these are sent to the central server for aggregation, usually via so-called Byzantine attacks. 
% It turns out that Byzantine model poisoning attacks severely impact standard FedAvg; therefore, more robust aggregation functions must be designed to make FL systems secure.
Here, we focus on \emph{untargeted model poisoning} attacks~\cite{bhagoji2019pmlr}, where an adversary attempts to tweak the global model weights %\footnote{We will use the terms \textit{parameters} and \textit{weights} interchangeably.} 
by directly perturbing the local model's parameters of some infected clients before these are sent to the central server for aggregation.
In doing so, the adversary aims to jeopardize the global model \textit{indiscriminately} at inference time.
Such model poisoning attacks severely impact standard FedAvg; therefore, more robust aggregation functions must be designed to secure FL systems.
\\
% In this paper, we focus on designing a novel robust aggregation scheme at the server's end to contrast the effect of Byzantine model poisoning attacks.
%
% Current countermeasures and their limitations
%Several countermeasures have been proposed in the literature to combat model poisoning attacks on FL systems.
% Some methods use simple statistics more robust than plain average to smooth the impact of malicious updates (e.g., Trimmed Mean and FedMedian~\cite{yin2018icml}). 
% Other defenses implement outlier detection techniques to discard malicious updates from the aggregation performed at the server's end. Those are either based on heuristics (e.g., Krum/Multi-Krum~\cite{blanchard2017nips} and Bulyan~\cite{mhamdi2018pmlr}) or data-driven approaches (e.g., K-means clustering~\cite{shen2016acm} or DnC via spectral analysis~\cite{shejwalkar2021ndss}). 
% Finally, some strategies rely on a centralized ``source of trust'' to spot potential malicious updates (e.g., FLTrust~\cite{cao2020fltrust}).
% Several countermeasures have been proposed in the literature to combat model poisoning attacks on FL systems, i.e., to discard possible malicious local updates from the aggregation performed at the server's end. 
% These techniques range from simple statistics more robust than plain average (e.g., Trimmed Mean and FedMedian~\cite{yin2018icml}) to outlier detection heuristics (e.g., Krum/Multi-Krum~\cite{blanchard2017nips} and Bulyan~\cite{mhamdi2018pmlr}) or data-driven approaches (e.g., spectral analysis via K-means clustering~\cite{shen2016acm} or spectral analysis), or methods based on ``source of trust'' (e.g., FLTrust~\cite{cao2020fltrust}).
% OLD, LONG VERSION
%Several countermeasures have been proposed in the literature to combat Byzantine model poisoning attacks on FL systems.
% Descriptive statistics
% For example, Trimmed Mean and FedMedian aggregate local model updates using more robust statistics than standard average~\cite{yin2018icml}.
%
% % Heuristics for outlier detection
% Many existing Byzantine-resilient strategies implement some outlier detection heuristics to discard the model updates sent by potentially malicious clients from the input of the aggregation function.
% One of the most popular heuristics is Krum~\cite{blanchard2017nips}.
% This strategy tries to mitigate the impact of Byzantine attacks by selecting as a global model the local model with the smallest sum of Euclidean distances to {\em all} the other local models.
% Although powerful, Krum requires the server to know (or, at least, estimate) the number of malicious FL clients upfront, which is generally impossible in a realistic attack scenario. %
% Moreover, Krum may become ineffective for complex, high-dimensional model parameter spaces due to the curse of dimensionality.
% Bulyan~\cite{mhamdi2018pmlr} tries to overcome this issue by combining Krum with a variant of Trimmed Mean.
% % Data-driven outlier detection
% Other strategies use data-driven outlier detection techniques -- e.g., via K-means clustering~\cite{shen2016acm} -- to spot potential malicious local model updates. 
% %For instance, Shen et al. propose to cluster local model updates with K-means and thus identify outliers.
%
% % Other techniques
% As far as the server is concerned, any local model received can be from a potential malicious client. 
% FLTrust~\cite{cao2020fltrust} assumes the server acts as a client, i.e., trains a local model on an additional {\em trustworthy} dataset at the server's end and compares it against all the local models from other clients. 
% This way, the server can rely on some ``source of trust'' when discarding potentially malicious clients.
%\\
% Limitations of existing Byzantine-resilient strategies
Unfortunately, existing defense mechanisms either rely on simple heuristics (e.g., Trimmed Mean and FedMedian by~\cite{yin2018icml}) or need strong and unrealistic assumptions to work effectively (e.g., foreknowledge or estimation of the number of malicious clients in the FL system, as for Krum/Multi-Krum~\cite{blanchard2017nips} and Bulyan~\cite{mhamdi2018pmlr}, which, however, cannot exceed a fixed threshold).
Furthermore, outlier detection methods using K-means clustering~\cite{shen2016acm} or spectral analysis like DnC~\cite{shejwalkar2021ndss} do not directly consider the temporal evolution of local model updates received.
Finally, strategies like FLTrust~\cite{cao2020fltrust} require the server to collect its own dataset and act as a proper client, thereby altering the standard FL protocol.
\\
% OLD, LONG VERSION
% Overall, existing Byzantine-resilient strategies are either simple heuristics (e.g., FedMedian) or, if they are more complex, they rely on strong and unrealistic assumptions to work effectively (e.g., knowing the number of malicious clients in the FL system in advance, as for Krum and alike).
% Furthermore, data-driven outlier detection methods do not consider the temporary evolution of local model updates received (e.g., K-means clustering). 
% Finally, strategies like FLTrust requires the server to collect its own dataset and act as a proper client, thereby altering the standard FL protocol.
%
% Description of the proposed method
This work introduces a novel pre-aggregation \textit{filter} robust to untargeted model poisoning attacks. Notably, this filter $(i)$ operates without requiring prior knowledge or constraints on the number of malicious clients and $(ii)$ inherently integrates temporal dependencies. 
The FL server can employ this filter as a preprocessing step before applying \textit{any} aggregation function, be it standard like FedAvg or robust like Krum or Bulyan.
Specifically, we formulate the problem of identifying corrupted updates as a multidimensional (i.e., matrix-valued) time series anomaly detection task. 
The key idea is that legitimate local updates, resulting from well-calibrated iterative procedures like stochastic gradient descent (SGD) with an appropriate learning rate, show \textit{higher predictability} compared to malicious updates. This hypothesis stems from the fact that the sequence of gradients (thus, model parameters) observed during legitimate training exhibit regular patterns, as validated in Section~\ref{subsec:intuition}. %until convergence. 
%This regularity may be more pronounced for smooth convex loss functions, but it can still be captured within an appropriate time window, even for more complex and convoluted loss surfaces. 
%We provide evidence of this claim in Appendix~B, where we show that the average mutual information (i.e., ``predictability''), calculated over pairs of legitimate model updates sent at different FL rounds, is significantly higher than the corresponding computation for a malicious client.
\\
Inspired by the matrix autoregressive (MAR) framework for multidimensional time series forecasting~\cite{chen2021je}, we propose the FLANDERS ({\em \textbf{F}ederated \textbf{L}earning meets \textbf{AN}omaly \textbf{DE}tection for a \textbf{R}obust and \textbf{S}ecure}) filter.
The main advantages of FLANDERS over existing strategies like FLDetector~\cite{zhao2020multivariate} are its resilience to large-scale attacks, where $50\%$ or more FL participants are hostile, and the capability of working under realistic non-iid scenarios.
We attribute such a capability to two key factors: $(i)$ FLANDERS works without knowing a priori the ratio of corrupted clients, and $(ii)$ it embodies temporal dependencies between intra- and inter-client updates, quickly recognizing local model drifts caused by evil players. Below, we summarize our main contributions:

\begin{itemize}
\item[{\em(i)}]
We provide empirical evidence that the sequence of models sent by legitimate clients is more predictable than those of malicious participants performing untargeted model poisoning attacks.
\\
\item[{\em(ii)}] 
We introduce FLANDERS, the first pre-aggregation filter for FL robust to untargeted model poisoning based on multidimensional time series anomaly detection.
\\
\item[{\em(iii)}] 
We integrate FLANDERS into Flower,\footnote{\scriptsize{\url{https://flower.dev/}}} a popular FL simulation framework for reproducibility.
\\
\item[{\em(iv)}] 
We show that FLANDERS improves the robustness of the existing aggregation methods under multiple settings: different datasets, client's data distribution (non-iid), models, and attack scenarios.
\\
\item[{\em(v)}] 
We publicly release all the implementation code of FLANDERS along with our experiments.\footnote{\scriptsize{\url{https://anonymous.4open.science/r/flanders_exp-7EEB}}}
\end{itemize}

% Paper's structure and organization
The remainder of the paper is structured as follows. %some related work and the current state-of-the-art solutions to security issues that FL entails. 
Section~\ref{sec:background} covers background and preliminaries. 
In Section~\ref{sec:related}, we discuss related work.
Section~\ref{sec:problem} and Section~\ref{sec:method} describe the problem formulation and the method proposed. % to tackle it. 
Section~\ref{sec:experiments} gathers experimental results. %, and Section~\ref{sec:limitations} discusses some limitations of this work.
Finally, we conclude in Section~\ref{sec:conclusion}.
 %discusses the limitations of this work and draws future research directions.
%reports conclusions and draws perspectives for future research directions.

%%%%%%% OLD %%%%%%%
%to overcome the resilience of Byzantine failures in distributed Stochastic Gradient Descent computations. 
% The strength of Krum is its time complexity, which is linear in the gradient dimension. 
% However, the robustness of the approach is guaranteed for gradient-based learning applications only when the majority of the clients are not compromised. 
% Besides, the aggregation mechanism of Krum, as well as that of similar methods, is robust from a coarse-grained perspective and does not provide solutions to errors and perturbations that may occur at inference time.
%A related approach to~\cite{blanchard2017nips} is the work of Su et al.~\cite{su2016dc}. Here, the authors propose an iterated approximate agreement to tackle a multi-layer scenario attacked by Byzantine agents. 
%However, the method works efficiently on the sole discrete context and it is inapplicable to continuous state environments.
%\gabri{Maybe, we should just talk about the main limitations of existing countermeasures without digging into their details (or, we can just mention Krum as this is the most popular one). I will move the description of all these methods to the Related Work section.}

\section{Proposed Framework: {\ourmodel}}
\label{model}


In this section, we introduce a novel self-supervised co-training framework {\ourmodel}.
The proposed framework is illustrated in Figure~\ref{fig:intro_model} and works in three phases.
Phase one automatically generates two sets of pseudo labels.
We use a combination of off-the-shelf pre-trained POS and NER taggers, knowledge graph, and GPT-2 scorer for generating the first set of pseudo labels automatically without any hand-crafted rules for matching the slot values.
The other set of pseudo labels is acquired through a zero-shot slot filling model~\cite{liu2020coach}, trained on the out-of-domain dataset.
It is critical to emphasize that both sets of labels are noisy and incomplete which poses serious challenges to training effective models for the task of open-domain slot filling.
Phase two fine-tunes the pre-trained BERT to the slot filling task that effectively transfers the knowledge from the pre-trained language model~(LM) to overcome the issue of label incompleteness to some extent. 
Further, we employ the early stopping technique to minimize the noise in the labels.
The output of this phase is two BERT models that can generate soft labels for self-supervision during co-training in phase three.
Phase three leverages the fine-tuned models and further trains them in an iterative fashion.
Specifically, the proposed peer training approach facilitates high-confidence soft label selection for the other peer to perform training. This phase progressively reduces the noise in the labels and enables effective model fitting. 



\subsection{Phase One: Automatic Label Generation}
To acquire the first set of labels, we perform the following steps.
First of all, off-the-shelf trained POS and NER taggers are used to predict initial estimates of the slot values irrespective of the slot types. Then, the type information of the slot values is queried from the KG and the slot value is tagged for the most appropriate slot in the target domain.
This approach, however, produces low recall. 
To expand the candidate slot values, we generate n-grams of the natural language text and employ a partial matching scheme to query the KG for type information (e.g., \myspecial{Jason} \myspecial{Aldean} = \myspecial{American} \myspecial{singer}) of the n-grams if the entry exists.
This process generates multiple overlapping hypotheses about the slot values.
We replace a span of text that corresponds to a slot value by its type information and a GPT-2 based scorer (see Section~\ref{sec:nlpmodels}) is used to select the best candidate based on the fluency of the text.
Naturally, if a token (or span of tokens) is replaced by its type, the sentence should score higher as compared to the case where an inappropriate substitution is performed. 
We select the best hypothesis if the score is greater than the threshold.
Intuitively, the candidate selection threshold can automatically be searched based on a small validation set from the target domain, making the label generation process fully automatic. 
The other set of noisy labels is acquired by the zero-shot slot filling model~\cite{liu2020coach} that has been trained using an out-of-domain dataset. It is important to highlight that the zero-shot slot filling model does not require any labeled in-domain training example. 
To summarize the automatic label generation phase, both sets of labels are acquired in a fully automatic fashion without any hand-crafting.


In contrast to previous work in weak supervision~\cite{ren2015clustype,he2017autoentity,fries2017swellshark,giannakopoulos2017unsupervised} that obtains a single set of noisy labels and then propose techniques to overcome the challenge of fitting an effective model to the noisy labels, we acquire two sets of complementary labels.
The choice of these two sets of labels is guided by the intuition that they should be complementary and the models trained on these sets of labels should be able to share complementary information with the other to improve the performance in the later phases of the framework.
Essentially, the first set of labels carries information from external knowledge sources, whereas the labels generated through the pre-trained zero-shot slot filling model capture how the slot values are mentioned in other domains.
%
To further elaborate on the motivation and our process for the first set of labels (i.e., labels using KG and other NLP models), the pre-trained LMs have been shown to have a great deal of knowledge~\cite{petroni2019language}, thus should be capable of generating automatic labels with no need of external KG. 
To the best of our knowledge, there exists no work that shows that accurate token-level automatic labeling (e.g., slot filling task) is possible with pre-trained LMs. 
Moreover, such approaches would require heavy prompting in each new target domain, whereas our label generation process is fully automatic and only relies on the readily-available pre-trained NLP models and external KG.

\subsection{Phase Two: LM-assisted Weak Supervision}
Since we do not have access to dataset $\{(\mathbf{X}_n,\mathbf{Y}_n)\}_{n=1}^N$ with true ground-truth labels.
We use pseudo labels generated in phase one, $\{(\mathbf{X}_n,\mathbf{D}_n)\}_{n=1}^N$, to learn 
$f_{m,c}(\cdot; \cdot)$ that outputs the probability of the $m$-th token to take on class $c$. 
We learn $f_{m,c}(\cdot; \cdot)$ by minimizing the following loss over the noisy dataset $\{(\mathbf{X}_n,\mathbf{D}_n)\}_{n=1}^N$: 
$$
\hat\theta = \argmin_{\theta}\frac{1}{N}\sum_{n=1}^{N} \ell(\mathbf{D}_n, f(\mathbf{X}_{n}; \theta)),
\label{eq:stage1}
$$
where $\ell(\mathbf{D}_n, f(\mathbf{X}_{n}; \theta)) = \frac{1}{M} \sum_{m=1}^{M} -\log{f_{m,d_{n, m}}(\mathbf{X}_{n}; \theta)}$. 
We employ the pre-trained multilingual BERT with token-level classification head that uses Adam optimizer \cite{kingma2014adam,Liu2019} with early stopping and multiple random initializations. 


Since slot filling task is similar to the MLM training objective of the BERT, we employ pre-trained BERT as the backbone model.
That is, MLM's goal is to predict the masked tokens using bidirectional contexts. Similarly, slot filling tries to predict the label for a token leveraging both left and right contexts simultaneously, which makes the pre-trained BERT an ideal model of choice that greatly facilitates minimizing incomplete labels.
It is important to highlight that our automatically generated labels are not only incomplete but also potentially wrong.
The training strategies employed in this phase minimize the noise in the label to some extent. 
Specifically, early stopping can provide a strong regularization and would not let the model overfit to the noisy labels, especially wrong labels. 
Moreover, early stopping does not let the model forget the knowledge in the pre-trained model.
Similarly, multiple random initializations enforce robustness. 
Since the model is fine-tuned on the noisy labels, averaging the predictions of multiple models for each token ensures that wrong labels end up with low probabilities and true labels consistently achieve high probabilities.
Using the above-mentioned strategies, we train two slot filling models, which we call the peers. The peer one is trained on the first set of pseudo labels that were generated using POS and NER taggers, KG, and the GPT-2 scorer in phase one. Similarly, peer two is trained using the predictions of the zero-shot slot filling model~\cite{liu2020coach}.
Both models have the same architecture and follow the same training procedures.

\begin{table*}[t!]
\centering
\caption{Dataset statistics.}
\vspace{-7pt}
\label{tab:dataset}
\begin{tabular}{lccccc}
\toprule
\textbf{Dataset}  & \textbf{Dataset Size} & \textbf{Vocab. Size} & \textbf{Avg. Length} & \textbf{\# of Domains} & \textbf{\# of Slots} \\ \hline
\textbf{SGD}      & 188K                  & 33.6K                & 13.8                 & 20                     & 240                  \\
\textbf{MultiWoZ} & 67.4K                 & 10.5K                & 13.3                 & 8                      & 61 \\
\bottomrule
\end{tabular}
\vspace{-7pt}
\end{table*}

\subsection{Phase Three: Self-supervised Co-training}
We introduce an iterative peer training algorithm where both peers generate high-confidence soft labels for training the other peer in the next iteration. 
Theoretically, these peers can be anything, but in this work, 
we explore two of the most promising directions that have shown the promise to minimize the need for manual labeling for the task: zero-shot learning and distant supervision.
This phase uses a self-supervised co-training scheme to exploit the patterns of slot values from other domains through the labels generated by the zero-shot filling model (i.e., peer two)~\cite{liu2020coach} as well as utilize the knowledge in external KGs and pre-trained models via labels provided by the peer one.
Specifically, we initialize the peers trained in phase two and use their pseudo labels to kick-start training in this phase.
Specifically, peer one $f_{m,c}(\cdot; \theta_{\textrm{p1}})$ would generate labels $\{\tilde{\mathbf{Y}}^{(t)}_n = [\tilde{y}_{n,1}^{(t)}, ..., \tilde{y}_{n,m}^{(t)}]\}_{n=1}^{N}$ for peer two $f_{m,c}(\cdot; \theta_{\textrm{p2}})$ at the $t$-th iteration by:
$$
\tilde{y}_{n,m}^{(t)} = \argmax_{c}{f_{m,c}(\mathbf{X}_n; \theta_{\textrm{p1}}^{(t)})}. 
\label{eq:pseudo}
$$

Based on these labels, the peer two can be fine-tuned by: 
$$
\hat\theta_{\textrm{p2}}^{(t+1)} = \argmin_{\theta}\frac{1}{N}\sum_{n=1}^N \ell(\tilde{\mathbf{Y}}_n^{(t)}, f(\mathbf{X}_{n}; \theta)).
\label{eq:self_train1}
$$

Similarly, peer two $f_{m,c}(\cdot; \theta_{\textrm{p2}})$ would generate pseudo labels for peer one $f_{m,c}(\cdot; \theta_{\textrm{p1}})$ that are used to fine-tune peer one. 
We also notice that it is beneficial to stop early during this phase as well, to improve the model fitting and gradually reduce the noise associated with the automatically generated labels.
Since pseudo labels are refined gradually in an iterative way, both peers can benefit from the knowledge contained within the labels of the other while avoiding overfitting.
Furthermore, as an alternative to pseudo labels, we also generate soft labels that are used for confidence re-weighting. 
The high-confidence soft label selection strategy enables better model fitting and efficient learning via better quality of the automatic labels.
Specifically, for the given $m$-th token in the $n$-th training example, the probability for all classes $C$ is $[f_{m,1}(\mathbf{X}_n;\theta),...,f_{m,C}(\mathbf{X}_n;\theta)]$. 
Following ~\cite{xie2016unsupervised}, at $t$-th iteration, peer one generates soft labels, $\{\mathbf{S}_n^{(t)} = [\mathbf{s}_{n,m}^{(t)}]_{m=1}^M \}_{n=1}^N$, as given below:
$$
\mathbf{s}_{n,m}^{(t)} = [s_{n,m,c}^{(t)}]_{c=1}^{C} = \Bigg[  \frac{f_{m,c}^2(\mathbf{X}_n;\theta_{\textrm{peer1}}^{(t)})/p_{c}}{\sum_{c'=1}^C f_{m,c'}^2(\mathbf{X}_n;\theta_{\textrm{peer1}}^{(t)})/p_{c'}}\Bigg]_{c=1}^{C}
\label{eq:soft}
$$ 
where $p_{c} = \sum_{n=1}^N \sum_{m=1}^M f_{m,c}(\mathbf{X}_n;\theta_{\textrm{p1}}^{(t)})$ computes the frequency of the tokens for the $c$-th class. 
Then, peer two $f(\cdot; \theta_{\textrm{p2}}^{(t+1)})$ is fine-tuned by:
$$
\theta_{\textrm{p2}}^{(t+1)} = \argmin_{\theta} \frac{1}{N} \sum_{n=1}^{N} \ell_{\rm KL}(\mathbf{S}_n^{(t)}, f(\mathbf{X}_{n}; \theta)),
$$
where $\ell_{\rm KL}(\cdot,\cdot)$ is the KL-divergence-based loss:
$$
\ell_{\rm KL}(\mathbf{S}_n^{(t)}, f(\mathbf{X}_{n}; \theta))=\frac{1}{M}\sum_{m=1}^M\sum_{c=1}^C - s_{n,m,c}^{(t)} \log f_{m,c}(\mathbf{X}_{n}; \theta).
\label{eq:klloss}
$$

Moreover, we also investigate selecting tokens that have high confidence. 
For instance, we pick high-confidence tokens from the $m$-th input example at the $t$-th iteration by  
$
H^{(t)}_n = \{m : \max_{c} s_{n,m,c}^{(t)} > \epsilon \},
$
where $\epsilon\in [0,1]$ is a threshold that can be searched based on a small validation set. 
Then, peer two $f(\cdot; \theta_{\textrm{p2}}^{(t+1)})$ is fine-tuned by:
$$
\theta_{\textrm{p2}}^{(t+1)} %&= \argmin_{\theta} \frac{1}{N} \sum_{n=1}^{N} \ell_{\rm S-KL}(\bS_n^{(t)}, f(\bX_{n}; \theta)) \\
= \argmin_{\theta} \frac{1}{N|H^{(t)}_n|}\sum_{n=1}^{N} \sum_{m\in H^{(t)}_n}\sum_{c=1}^C - s_{n,m,c}^{(t)} \log f_{m,c}(\mathbf{X}_{n}; \theta).
$$

This phase improves the robustness to effectively fit the model for tokens with high confidence. 
Both peers keep sharing information and their confidence by producing soft labels for their counterparts until they approximate to the true labels while employing early stopping and scheduled learning rates.
It is important to remind that phase three is the most important phase that progressively reduces noise from the labels to a great extent and enables superior performance for the task of open-domain slot filling.

\section{Fixed Cost Functions and Demand} \label{sec:fixed-setting}

We now investigate the design of online pricing policies achieving good performance on the three regret metrics, i.e., sub-linear unmet demand, cost regret, and payment regret in the number of periods $T$. As a warm-up, we first consider the setting when the cost functions of the suppliers and customer demand are fixed over the $T$ periods. 
Formally, the supplier cost functions satisfy $c_{it}(\cdot) = c_{it'}(\cdot)$ for all $t, t' \in [T]$ and the demands satisfy $d_t = d_{t'}$ for all periods $t, t' \in [T]$. For ease of exposition, in this section, we drop the subscript (superscript) $t$ in the notation for the customer demand (and supplier cost functions) and denote $d_t = d$ (and $c_{it}(\cdot) = c_i(\cdot)$ for all suppliers $i$) for all periods $t \in [T]$.
In this setting, we develop an algorithm that achieves a regret of $O(\log \log T)$ on the three regret metrics when the suppliers' cost functions are strongly convex (Section~\ref{sec:fixed-algorithm}). We further present an example to demonstrate that if the strong convexity condition on suppliers' cost functions is relaxed, then no sub-linear regret guarantee on all three regret metrics is, in general, possible for any online algorithm (Section~\ref{sec:fixed-convex-limitations}).



\subsection{Algorithm with Sub-linear Regret for Strongly Convex Cost Functions} \label{sec:fixed-algorithm}

In this section, we consider the setting of fixed supplier cost functions and customer demands and present an algorithm that achieves a regret of $O(\log \log T)$ on the three regret metrics when the suppliers' cost functions are strongly convex. 
To motivate our algorithm, we first note that since the customer demand and the supplier cost functions are fixed over time, the optimal price $p^* \in [0, 1]$ for all periods $t \in [T]$ is also fixed and given by the dual of the demand constraint of Problem~\eqref{eq:supObj2}-\eqref{eq:demand-con}. %For ease of exposition of our algorithm (Algorithm~\ref{alg:fixed-demand}) and its corresponding analysis, we normalize the set of feasible prices such that the optimal price $p^*$ of the commodity belongs to the normalized interval $[0, 1]$, i.e., $p^* \in [0, 1]$. \haoyuan{This should go into the models section. Also for boundedness of the demand etc.} \devansh{Done!}
%Crucially, we note that we are only concerned with locating the equilibrium price which clears the market and it is not necessary to learn other features of the cost function $c_i(\cdot)$.
Furthermore, the cumulative production $x_t^*(p) = \sum_{i = 1}^n x_{it}^*(p)$ is monotonically non-decreasing in the price $p$ because suppliers' cost functions are increasing. Utilizing this monotonicity property, we note that if we set two prices $p_1, p_2 \in [0, 1]$ such that the cumulative production $\sum_{i = 1}^n x_{it}^*(p_1)>d$ and $\sum_{i = 1}^n x_{it}^*(p_2)<d$, then $p_1$ and $p_2$ respectively serve as upper and lower bounds on the optimal price $p^*$ when the supplier cost functions and customer demands are fixed over time.

Following these observations, we present Algorithm~\ref{alg:fixed-demand}, akin to the algorithm in~\cite{oppa}, 
which maintains a feasible interval for the optimal price $p^*$ and sets a sequence of prices for each arriving user to continuously shrink this feasible price set. In particular, the feasible price interval $[a, b]$ is initialized to $\S_p = [0, 1]$ and a precision parameter $\varepsilon$ is set to $0.5$. 
%Then, in any given algorithm sub-phase, 
Then, for a given algorithm sub-phase associated with feasible price interval $[a, b]$, the operator posts prices $a, a+\varepsilon, a+2\varepsilon, \ldots$ (up to $b$) at each period until the total supply exceeds the demand at the offered price. If $a+k\varepsilon$ for some $k \in \mathbb{N}$ was the last price such that $x_{t}^*(a+k\varepsilon) \leq d$, then $[a+k\varepsilon, a+(k+1)\varepsilon]$ is set as the new feasible interval for the optimal price, and the precision parameter is re-set to $\varepsilon^2$. This process of shrinking the feasible interval and updating the precision parameter is repeated until the length of the feasible interval is smaller than $\frac{1}{T}$, following which the market operator posts the price at the lower end of the feasible interval for the remaining periods. This process is presented formally in Algorithm~\ref{alg:fixed-demand}.


\begin{algorithm} 
\SetAlgoLined
\SetKwInOut{Input}{Input}\SetKwInOut{Output}{Output}
\Input{Feasible set of prices $\S_p = [0, 1]$, Precision Parameter $\varepsilon = \frac{1}{2}$}
Set the lower and upper bounds of the feasible price set: $a \leftarrow 0, b \leftarrow 1$ \;
 \While{\text{length of feasible price set is greater than $\frac{1}{T}$}}{
 Offer prices $a, a+\varepsilon, \ldots, a+(k+1)\varepsilon$ (all of which are $\le b$) to each subsequent user where $a+k\varepsilon$ is the last price such that $\sum_{i = 1}^n x_{it}^*(a+k\varepsilon) < d$ \;
 Set the new feasible interval to $[a+k\varepsilon, a+(k+1)\varepsilon]$ and reduce the precision parameter to $\varepsilon^2$ \;
 }
 \For{\text{the remaining time periods}}{
  Offer price $p_t = a$ \;
}
\caption{Feasible Price Set Tracking under Fixed Demand and Costs} 
\label{alg:fixed-demand}
\end{algorithm} 

While Algorithm~\ref{alg:fixed-demand} is similar to the corresponding algorithm in~\cite{oppa} for the setting of fixed user valuations, our market setting is considerably different than the revenue maximization setting in~\cite{oppa}. First, in this work, suppliers have a continuous rather than a binary action space as in the revenue maximization setting in~\cite{oppa}, where consumers either purchase one unit of the resource at the given price or do not purchase it. Furthermore, as opposed to the single regret measure analyzed in~\cite{oppa}, we consider and analyse three different regret measures that often compete against each other. 

We now present the main result of this section, which establishes that Algorithm~\ref{alg:fixed-demand} simultaneously achieves an $O(\log \log T)$ regret on the three regret measures studied in this work.

\begin{theorem} \label{thm:IdenticalResult}
The unmet demand, cost regret, and payment regret of Algorithm~\ref{alg:fixed-demand} are $O(\log \log T)$ if the cost functions of the suppliers are strongly convex.
\end{theorem}

The proof of Theorem~\ref{thm:IdenticalResult} relies on the following Lipshitzness condition between the optimal supplier production and the prices set by the market operator.

\begin{lemma} [Lipschitzness of Production in Prices] \label{lem:ProductionLipschitz}
Suppose that the suppliers' cost functions $c_i(\cdot)$ are $\mu_i$-strongly convex. Then, at any period $t$, the optimal production quantity for supplier $i$ corresponding to the solution of Problem~\eqref{eq:supObj} is Lipschitz in the price $p$, i.e., $|x_{it}^*(p_1) - x_{it}^*(p_2)| \leq L |p_1 - p_2|$ for some constant $L>0$ for all $p_1, p_2 \in [0, 1]$.
\end{lemma}
\begin{proof}
Fix a period $t \in [T]$. Then, by computing the first-order optimally condition of Problem~\eqref{eq:supObj} for each supplier $i$, we have that 
\[ p = c_i'(x_{it}^*(p)) \implies x_{it}^*(p) = (c_i')^{-1}(p). \]
Then, by the inverse function theorem, we have that
    \[(x_{it}^*)'(p) = \frac{d}{dp}(c_i')^{-1}(p) = \frac{1}{c_i''(x_{it}^*(p))} \le \frac{1}{\mu_i},\]
where the inequality follows since the function $c_i(\cdot)$ is $\mu_i$-strongly convex.
Hence, at each period $t$, the optimal supplier production $x^*_{it}$ is $(1/\mu_i)$-Lipschitz in the prices.
\end{proof}

Lemma~\ref{lem:ProductionLipschitz} establishes that small changes in the price set by the market operator correspond to small changes in the optimal production of suppliers. Using Lemma~\ref{lem:ProductionLipschitz}, we now present a proof sketch of Theorem~\ref{thm:IdenticalResult} and present its complete proof in Appendix~\ref{sec:pf-identical-result}.

\begin{hproof}
To establish this result, we first note that we need $O(\log \log T)$ sub-phases of repeated squaring of the parameter $\varepsilon$ to reduce $\varepsilon$ from $0.5$ to $\frac{1}{T}$. 
Due to the monotonicity of the optimal supplier production in the prices that both payment and cost regret are only accumulated when $p_t>p^*$ (and that $p_t>p^*$ at most once in each sub-phase), the total payment and cost regret are $O(\log \log T)$.
Next, to bound the unmet demand, we use Lemma~\ref{lem:ProductionLipschitz} to map prices to productions and show that there is a constant unmet demand accumulated in each sub-phase in Algorithm~\ref{alg:fixed-demand}, resulting in an $O(\log \log T)$ unmet demand when the length of the feasible price interval is more than $\frac{1}{T}$, as there are $O(\log \log T)$ sub-phases. In the final phase, when the length of the feasible price interval is less than $\frac{1}{T}$ and the price is fixed, we again use Lemma~\ref{lem:ProductionLipschitz} to show that the unmet demand through this phase is constant.
Thus, the unmet demand is $O(\log \log T)$, establishing our claim.
\end{hproof}


We reiterate that the proof of Theorem~\ref{thm:IdenticalResult} crucially relies on the strong convexity of the cost functions of the suppliers, which was necessary to establish the Lipshitzness relation between the optimal production of suppliers and the prices set by the market operator (Lemma~\ref{lem:ProductionLipschitz}). As a result, in addition to leveraging tools from the analysis of the corresponding algorithm in~\cite{oppa}, our regret analysis additionally uses tools from parametric optimization to develop the necessary sensitivity relations required to analyse the three regret metrics considered in this work.
We also note that the $O(\log \log T)$ regret guarantee obtained in Theorem~\ref{thm:IdenticalResult} indicates that in the setting with fixed supplier cost functions and customer demand, Algorithm~\ref{alg:fixed-demand} incurs little performance loss on all three regret metrics as compared to when equilibrium prices are set with complete knowledge of the cost functions of the suppliers, where Algorithm~\ref{alg:fixed-demand}'s prices converge to the equilibrium price $p^*$ super-exponentially.
Furthermore, the obtained upper bound on the regret of Algorithm~\ref{alg:fixed-demand} compares favorably to the $\Omega(\log \log T)$ regret lower bound for any online algorithm in the revenue maximization setting with fixed user valuations studied in~\cite{oppa}.



\subsection{Performance Limitations for Non-Strongly Convex Cost Functions} \label{sec:fixed-convex-limitations}

While Algorithm~\ref{alg:fixed-demand} achieved sub-linear regret on all three performance measures in the incomplete information setting for strongly convex cost functions, we note that this result does not generalize to the setting of general convex costs. In particular, in this section, we show that if the cost functions of the suppliers are linear, then no online algorithm can achieve sub-linear regret guarantees for the unmet demand, cost regret, and payment regret metrics simultaneously. This result highlights the difficulty of the incomplete information setting compared to that with complete information, where the equilibrium price satisfying the three desirable properties in Section~\ref{sec:market-model} exists and can be computed through the dual variable of the market clearing constraint of Problem~\eqref{eq:supObj2}-\eqref{eq:demand-con} when the cost functions of all suppliers are convex (and, not necessarily, strongly convex).

 
To motivate why sub-linear regret cannot be attained simultaneously on the three desirable metrics for linear cost functions, we note that the optimal production of suppliers is zero if $p<p^*$ and their optimal production is the maximum feasible if $p>p^*$. Thus, there is a jump discontinuity in the production of suppliers at the price $p = p^*$ and so the production is not Lipschitz in the prices, which breaks a key property (Lemma~\ref{lem:ProductionLipschitz}) required to prove Theorem~\ref{thm:IdenticalResult}. We now formally present an example demonstrating that, even in a market with a single supplier, no online algorithm can achieve sub-linear regret on all three regret measures if the supplier's cost function is linear.

\begin{example} [Sub-linear Regret is not Possible for Linear Cost Functions] \label{eg:linear-cost}
We consider a market with one supplier with a linear cost function that is fixed over time. In this setting, suppose that the optimal price is $p^*$ and the cost function of the supplier is given by $c(x) = c x$. Then, given a price $p$ set by the market operator, the individual decision making problem for the supplier is to produce a quantity $x^*(p) \geq 0$ that maximizes $(p-c) \cdot x$. Note that if $p<c$, then $x^*(p) = 0$ and if $p>c$, then the supplier generates the maximum amount that a supplier can feasibly generate. Thus, it is only when $p = p^* = c$ that the supplier will generate the amount equal to the customer demand $d$. Since the perfect identification of the optimal price $p^*$ is, in general, not possible under incomplete information on the cost coefficient $c$, we have that the market operator will either set prices $p_t<p^*$ or $p_t>p^*$. 

Next, we observe that there is an unmet demand of $d$ at any period when $p_t<p^*$ (as the supplier generates nothing if the price is below $p^*$ by linearity of the cost function). Thus, to achieve sub-linear unmet demand, the market operator must set the price $p > p^*$ for $O(T)$ periods; however, doing so results in a linear cost and payment regret as the supplier produces strictly more than $d$ units of the commodity when $p>p^*$. Thus, no online algorithm can achieve a sub-linear regret on all three regret measures if the cost functions of the suppliers is linear.
\qed
\end{example}


\section{Fixed Cost Functions and Time-Varying Demand} \label{sec:vary-demand}

%The setting considered in the previous section of fixed supplier cost functions and customer demands over time provided intuition on designing algorithms with sub-linear regret guarantees. 

Having studied the setting of fixed supplier cost functions and customer demands over time, in this section, we investigate a more general market setting when the suppliers' cost functions are static while customer demands can vary across the $T$ periods. 
%\haoyuan{need citation for real-world examples.} 
In particular, we suppose that the customer demands for the commodity are time-varying and lie in a continuous but bounded interval, i.e., the customer demand at each period $t$ is some variable quantity $d_t \in [\underline{d}, \overline{d}]$. In this setting, we extend the algorithm developed for fixed supplier cost functions and customer demands (Algorithm~\ref{alg:fixed-demand}) and show that it achieves a regret of $O(\sqrt{T} \log \log T)$ on all three performance measures for strongly convex cost functions.


Our approach for the time-varying demand setting builds upon the algorithmic ideas for the fixed demand setting. To address the challenge that the demands can vary between the interval $[\underline{d}, \Bar{d}]$, we first consider a direct extension of Algorithm~\ref{alg:fixed-demand} to the time-varying demand setting, wherein a feasible price set is maintained for each realized demand. However, as there may be up to $O(T)$ different demand realizations over the $T$ periods, the worst-case regret of such an algorithm is $O(T)$. To resolve this issue, we leverage the intuition that customer demands that are close to each other correspond to equilibrium prices that are also close together. Thus, we uniformly partition the demand interval $[\underline{d}, \Bar{d}]$ into sub-intervals of width $\gamma$ and consider any demand in the same sub-interval the same. In particular, any demand lying in a given sub-interval, i.e., $d_t \in [\underline{d} + k \gamma, \underline{d} + (k+1) \gamma]$ for some $k \in \mathbb{N}$, is considered as a demand equal to the lower bound of that interval. Note then that from the perspective of the algorithm, there are $O(\frac{1}{\gamma})$ distinct demands, as opposed to $O(T)$ possible demand realizations, as the feasible demand interval is partitioned into $O(\frac{1}{\gamma})$ sub-intervals. Finally, for these $O(\frac{1}{\gamma})$ distinct demands, corresponding to the lower bounds of the $O(\frac{1}{\gamma})$ sub-intervals, we apply the aforementioned direct extension of Algorithm~\ref{alg:fixed-demand}. 
Our algorithmic approach is formally presented in Algorithm~\ref{alg:time-varying-demand-new}.

\begin{algorithm}
\SetAlgoLined
\SetKwInOut{Input}{Input}\SetKwInOut{Output}{Output}
\Input{Discretized demand intervals $I_1, \dots, I_K$ with $I_k = \{\underline{d} + (k-1)\gamma, \underline{d} + k\gamma\}$ such that $K\gamma = \overline{d} - \underline{d}$}
 Initialize a feasible price set $\S_{k} = (0, 1]$, current price $p_k = 0$, and price precision $\varepsilon_k = 1/2$ for each demand interval $I_{k}$\;
 \For{$t = 1, \dots, T$}{
 %\If{$d_t = 0$}{
 %Offer price 0\;
 %\textbf{continue}
 %}
 Determine $k_t$ such that $d_t \in I_{k_t} =: [a_{k_t}, b_{k_t}]$\;
 Offer price $p_{k_t}$ to the supplier\;
 \If{\text{width of feasible price set $|\S_{k_t}|$ is greater than $\frac{1}{\sqrt{T}}$}}{
 \tcc{If production exceeds target demand, then narrow down the search interval}
 \uIf{$\sum_{i = 1}^n x_{it}^*(p_{k_t}) \ge a_{k_t}$}{
 Set $\S_{k_t} \gets (p_{k_t} - \varepsilon_{k_t}, p_{k_t}]$\;
 Set next price $p_{k_t} \gets p_{k_t} - \varepsilon_{k_t}$\;
 Reset the precision to $\varepsilon_{k_t} \gets \varepsilon_{k_t}^2$\;
 }
 \uElse{
 Set next price $p_{k_t} \gets p_{k_t} + \varepsilon_{k_t}$
 }
 }
 }
\caption{Feasible Price Set Tracking for Time-Varying Demands}
\label{alg:time-varying-demand-new}
\end{algorithm}



We now present the main result of this section, which establishes that Algorithm~\ref{alg:time-varying-demand-new} achieves a regret of $O(\sqrt{T} \log \log T)$ if the sub-interval width $\gamma = \frac{1}{\sqrt{T}}$ for strongly convex cost functions of suppliers. We note that choosing $\gamma = \frac{1}{\sqrt{T}}$ optimally balances between two different sources of regret in the time-varying demand setting, as is elucidated through the proof sketch of the following theorem.


\begin{theorem} \label{thm:VaryingDemandResult}
Let the demand sub-interval width $\gamma = \frac{1}{\sqrt{T}}$. Then, the unmet demand, cost regret, and payment regret of Algorithm~\ref{alg:time-varying-demand-new} are $O(\sqrt{T} \log \log T)$ if the cost functions of the suppliers are strongly convex. %With the discretization of the demand set to $\gamma = 1/\sqrt{T}, K = (\overline{d} - \underline{d})\sqrt{T}$, the unmet demand, cost regret, and payment regret of Algorithm~\ref{alg:time-varying-demand-new} are $O(\sqrt{T} \log \log T)$ if the cost functions of the suppliers are strongly convex.
\end{theorem}

\begin{hproof}
For each of the three regret metrics, the regret incurred by Algorithm~\ref{alg:time-varying-demand-new} can be broken down into two parts: (i) the regret incurred by the Algorithm~\ref{alg:fixed-demand} sub-routine for each demand sub-interval, and (ii) the inaccuracies of considering all demands in a given sub-interval to be equal to the lower bound of that sub-interval. By invoking Theorem~\ref{thm:IdenticalResult}, the first part is of order $O(K \log\log T)$ for all three regret measures, where $K := \lceil(\overline{d} - \underline{d})/\gamma\rceil$. Next, since all demands in a given sub-interval are treated as a demand equal to the lower bound of that sub-interval and the suppliers' optimal production is monotonic in the price, every price $p_t$ offered by Algorithm~\ref{alg:time-varying-demand-new} is an under-estimate to the equilibrium price for demand $d_t$.
Thus, the second part of the regret is only positive for the unmet demand and is at most $O(\gamma T)$, as the width of each demand sub-interval is $\gamma$, and regret is only accumulated over $T$ periods. Finally, choosing $\gamma = \frac{1}{\sqrt{T}}$ achieves an optimal balance (up to logarithmic terms) between the two quantities above, i.e., $O(\gamma^{-1} \log\log T)$ and $O(\gamma T)$, which establishes the $O(\sqrt{T} \log\log T)$ regret bound.
\end{hproof}

For a complete proof of Theorem~\ref{thm:VaryingDemandResult}, see Appendix~\ref{sec:vary-demand-pf}. We reiterate that Theorem~\ref{thm:VaryingDemandResult} applies to strongly convex cost functions as with Theorem~\ref{thm:IdenticalResult} and that extending this result to general convex cost functions, e.g., linear functions, is, in general, not possible (see Example~\ref{eg:linear-cost} in Section~\ref{sec:fixed-convex-limitations}). Furthermore, compared to the regret guarantee obtained in Theorem~\ref{thm:IdenticalResult}, Theorem~\ref{thm:VaryingDemandResult} establishes that the time-varying nature of the customer demand incurs an additional factor of $O(\sqrt{T})$ in the regret guarantee as compared to the setting with fixed demands. However, we do note that if the set of demand realizations $D$ is known \emph{a priori} to be $o(\sqrt{T})$, then the regret guarantee in Theorem~\ref{alg:time-varying-demand-new} can be improved to $O(|D| \log \log T)$ by running the direct extension of Algorithm~\ref{alg:fixed-demand}, wherein a feasible price interval is maintained for each realized demand. Finally, we note that the regret guarantee obtained in Theorem~\ref{thm:VaryingDemandResult} compares favorably to classical $O(\sqrt{T})$ regret guarantees in the OCO or MAB literature~\cite{OPT-013}. 



\section{Time-Varying Cost Functions} \label{sec:time-vary-cost-main}
In this section, we consider the general setting, where, in addition to customer demands changing over time, suppliers' cost functions can also vary across the $T$ periods. Formally, at each period $t$, each supplier $i$ has a privately known and time-varying cost function $c_{it}(\cdot)$. Compared to the setting with fixed cost functions, the fundamental ideas underlying the performance guarantees of Algorithms~\ref{alg:fixed-demand} and \ref{alg:time-varying-demand-new} do not directly apply to this setting, as suppliers' production may not remain the same when the operator offers the same price at different periods due to the time-varying nature of their cost functions. In fact, in Section~\ref{sec:counter-example}, we show that if the operator does not know the process that governs the variation of the cost functions, then sub-linear regret is impossible to achieve on all three regret metrics.
%that even if the cost functions of the suppliers are drawn i.i.d. from a distribution, no online algorithm can achieve sub-linear regret on all three regret metrics. 
%\haoyuan{I think the way you stated is somewhat misleading, as the distribution is a form of information.} \devansh{I have changed the distribution part here to the statement you had so this should be better}
%This result highlights that if the operator does not know the process that deterministically governs the variation of the cost functions, then sub-linear regret is impossible to achieve.
%This result highlights that if the operator does not have any information on the process that governs the variation of the cost functions, even if the cost functions are drawn i.i.d. from a distribution, then sub-linear regret is impossible to achieve.
To this end, in Section~\ref{sec:contextual-bandit}, in alignment with real-world markets, we consider an augmented problem setting wherein the market operator is provided with a hint (i.e., context) on the variation in suppliers' cost functions over time, e.g., due to weather conditions in electricity markets, while still keeping the full description of the costs away from the operator. In this setting, where the operator has access to additional hints on suppliers' cost functions, we then develop an algorithm with sub-linear on all three regret metrics (see Sections~\ref{sec:igw-alg-sol} and~\ref{sec:igw-alg-sketch}) through an adaptation of an algorithm in the contextual bandits literature~\cite{foster2020beyond}.


\subsection{Impossibility of Setting Equilibrium Price Under Time-Varying Costs}
\label{sec:counter-example}


We initiate our study of the setting of time-varying costs by presenting an example that illustrates the impossibility of setting equilibrium prices if the market operator has no information on how the cost functions of suppliers change over time. In particular, Proposition~\ref{prop:time-varying-cost-countereg} presents a counterexample establishing that even if suppliers' cost functions are drawn i.i.d. from a known distribution, no online algorithm can achieve a sub-linear regret on all three regret metrics as long as the operator is not informed about the outcome of the random draws from the distribution.

%the difficulty of the setting with time-varying supplier cost functions arises from the fact that the market operator cannot observe a suppliers' cost function


\begin{proposition} [Impossibility of Sub-linear Regret for Time-varying Costs] \label{prop:time-varying-cost-countereg}
There exists an instance with fixed time-invariant demand and a single supplier whose cost functions are drawn i.i.d. from some (potentially known) distribution such that no online algorithm can achieve sub-linear regret on all three regret metrics.
\end{proposition}

\begin{hproof}
We consider a setting with a fixed demand of $d=1$ at every period and a single supplier whose cost functions at each period are drawn from a distribution such that at each period $t$, its cost function could be either $c_1(x) = \frac{1}{8}x^2$ or $c_2(x) =\frac{1}{16} x^2$, each with probability $0.5$. We suppose that the market operator has knowledge of the distribution from which the supplier's cost function is sampled i.i.d. but does not know the outcome of the random draw at any period and show that any pricing strategy adopted by the operator must incur a linear regret on at least one of the three regret metrics for this instance. To this end, we first note that the equilibrium price corresponding to the cost function $c_1(x)$ is $p_1^* = \frac{1}{4}$ while that corresponding to the cost function $c_2(x)$ is $p_2^* = \frac{1}{8}$. Then, we analyze the total regret, i.e., the sum of the unmet demand, payment regret, and cost regret, for three different price ranges - (i) $p<\frac{1}{8}$, (ii) $1/8 \le p \le 1/4$, and (iii)  $p > 1/4$ - and show that irrespective of the set price $p$ at any period $t$, the expected total regret at any period is at least $\frac{7}{64}$, i.e., the total regret is at least $\frac{7}{64} T$. Finally, since the sum of the three regret metrics is linear in $T$, at least one of the three metrics must be linear in $T$, establishing our claim.%with two possible cost functions $c_1(x) = \frac{1}{8}x^2$ and $c_2(x) =\frac{1}{16} x^2$
\end{hproof}

For a complete proof of Proposition~\ref{prop:time-varying-cost-countereg}, see Appendix~\ref{sec:pf-prop-countereg}. While it was possible to achieve sub-linear regret in the setting with time-varying customer demands (see Theorem~\ref{thm:VaryingDemandResult}), Proposition~\ref{prop:time-varying-cost-countereg} establishes that such a result is, in general, not possible in the setting with time-varying cost functions. The setting with time-varying cost functions is more challenging because the market operator observes customer demands, which it can use to make pricing decisions, but does not observe the cost functions of suppliers.
%Additionally, the example in the proof of Proposition~\ref{prop:time-varying-cost-countereg} illustrates the difficulty in jointly balancing the three regret metrics.
%In particular, even though both the payment regret and cost regret are negative when the price is set below the equilibrium price, setting such prices corresponds to a higher unmet demand.
%Thus, any decrease in the payment or cost regret would correspond to a similar increase in the unmet demand and vice versa, which results in a total regret, i.e., the sum of the unmet demand, payment regret, and cost regret, which is linear in the number of periods $T$.
Further, in contrast to the settings in~\cite{yu2017online,balseiro2022best}, where online gradient descent approaches can simultaneously achieve sub-linear regret for multiple performance metrics, we note that our definition of unmet demand is considerably stronger as over-production at particular periods cannot compensate for unmet demand at other periods (see Section~\ref{sec:perf-measures} for a further discussion).
Thus, Proposition~\ref{prop:time-varying-cost-countereg} shows that, with the stronger unmet demand metric, it is impossible to jointly optimize the three regret metrics, and illustrates the difficulty of balancing the three regret metrics, as decreasing the payment or cost regret causes an increase in the unmet demand and vice versa.


\subsection{Adding Contexts for Time-varying Costs}
\label{sec:contextual-bandit}

Proposition~\ref{prop:time-varying-cost-countereg} highlights that if the operator does not have any information on the change in suppliers' cost functions over time, it is impossible to achieve sub-linear regret on all three regret metrics. To this end, in this section, we consider a natural augmented problem setting wherein the market operator, without knowing the complete specification of cost functions of suppliers, additionally has access to a hint (i.e., context) that reflects the variation in cost functions of suppliers over time. We note that such a setting aligns with real-world markets, e.g., electricity markets, wherein the cost functions of suppliers are private information yet will typically vary over time based on observed quantities, such as changes in the ambient weather conditions.  
%\devansh{can consider adding a reference here...}

To specify the augmented problem setting with contexts, we first introduce some notation regarding the cost functions of suppliers. In particular, we assume that each supplier's cost function is composed of two parts: (i) an unknown component that is time-invariant, and (ii) a time-varying component that is revealed to the marker operator. More precisely, the cost function of each supplier $i$ is parameterized as follows:
\[c_{it}(\cdot) = c_i(\cdot; \phi_i, \theta_{it}),\]
where $\phi_i$ is private information and $\theta_{it}$ is the time-varying component of the cost function given to the operator as \textit{contexts}.
Note that for any fixed $\phi_i$, the context $\theta_{it}$ uniquely determines the cost function of supplier $i$ at time $t$.
We stress that we do not assume any structure on the parameterization of the cost functions and so the time-varying and time-invariant components of the cost functions need not be separable.
Further, since $\phi_i$'s are unknown, the market operator cannot directly solve Problem~\eqref{eq:supObj2}-\eqref{eq:demand-con} to obtain the equilibrium prices in the market.

For the simplicity of exposition, for the remainder of this section, we aggregate all suppliers' cost functions into a combined cost $c_t(\cdot; \theta_t) = \sum_{i=1}^n c_{it}(\cdot; \phi_i, \theta_{it})$, %where we denote $\phi$ as the private information and $\theta_t$ as the time-varying context for all suppliers associated with the combined cost function.
where $\theta_t = (\theta_{1t}, \dots, \theta_{nt})$ is the time-varying context associated with the combined cost function. 
Note that doing so is without loss of generality as all suppliers have convex costs and observe the same prices in the market.
Furthermore, we note that since the private information $\phi_1, \dots, \phi_n$ are unknown, the market operator cannot directly solve Problem~\eqref{eq:supObj2}-\eqref{eq:demand-con} to obtain the equilibrium prices in the market.

In this augmented problem setting, at each period $t$, in addition to receiving the customer demand $d_t$, the market operator observes a context $\theta_t$, which it can use along with the prior history of supplier production quantities, customer demands, and contexts, to set a price $p_t$. In particular, with access to sequentially arriving contexts, the market operator sets a sequence of prices given by the pricing policy $\ppi = (\pi_1, \ldots, \pi_T)$, where $p_t = \pi_t(\{ (x_{t'}^*)_{1 = 1}^n, d_{t'}, \theta_{t'} \}_{t'=1}^{t-1}, d_t, \theta_t)$, where $x_{t}^*$ represents the sum of optimal production quantity corresponding to the solution of Problem~\eqref{eq:supObj} for each supplier at period $t$. We then evaluate the performance of this class of pricing policies on three regret metrics introduced in Section~\ref{sec:perf-measures}.
Note that we can naturally extend these three metrics to the augmented setting with contexts by plugging in $c_{it}(\cdot) = c_i(\cdot; \phi_i, \theta_{it})$ and for completeness, we present the corresponding definitions explicitly in Appendix~\ref{sec:new-regret-defs}. 


\subsection{Algorithm for Time-Varying Costs with Contexts}
\label{sec:igw-alg-sol}

We now present an algorithm that simultaneously achieves sub-linear regret for the unmet demand, payment regret, and cost regret metrics for the augmented problem setting introduced in Section~\ref{sec:contextual-bandit}. Our algorithmic approach is inspired by recent ideas in the contextual bandits literature (e.g.~\cite{agarwal2014taming,foster2020beyond}) and involves two building blocks. 
First, we seek to learn how to associate the arriving contexts with the relevant properties of suppliers' cost functions. Next, based on the information inferred from the arriving contexts, our algorithm offers prices to suppliers in the next period.
We note that the first step of our approach fundamentally differs from the fixed cost setting, as time-varying supplier costs, unlike time-varying demands, are not observed and thus unavailable to the operator making pricing decisions.
Therefore, the first step of our approach is crucial in learning a descriptive model on the supplier's response to various contexts and prices.
%Hence, the first step differs from the setting with varying demands and fixed costs, as the market operator observes the customer demand and uses it to make subsequent pricing decisions.

The task of learning to associate the arriving contexts with the relevant properties of the cost function is accomplished by an \textit{online regression oracle}. In particular, an online regression oracle performs real-valued online regression and achieves a prediction error guarantee, with a bound denoted $\est(T)$, relative to the best function in a class $\mathcal{F}$.


\begin{definition}[Online Regression Oracle]
Consider a function class $\mathcal{F} : \mathcal{A} \to \mathcal{B}$, at each time $t$, the online regression oracle receives an input $a_t$ and computes an estimate $\hat{b}_t = \hat{f}_t(a_t)$, where $\hat{f}$ depends on the past history
\[ \mathcal{H}_{t-1} = (a_1, b_1), \dots, (a_{t-1}, b_{t-1}). \]
%and not necessarily in $\mathcal{F}$\footnote{Some literature also require that $\hat{f} \in \mathcal{F}$, but this typically is not an issue for most problem settings, e.g. when $\mathcal{F}$ is convex.}. 
Then, the oracle receives the true output $b_t$.

The predictors $\hat{f}_t$ of the oracle are almost as accurate as any function in $\mathcal{F}$ in the sense that:
\[\sum_{t=1}^T (\hat{b}_t - b_t)^2 - \inf_{f \in \mathcal{F}} \sum_{t=1}^T (f(a_t) - b_t)^2 \le \est(T)\]
\end{definition}

The prediction error $\est(T)$ of the online regression oracle scales with the ``size'' of the $\mathcal{F}$ in a statistical sense. As an example, if the function class $\mathcal{F}$ is finite, then the exponential weights update algorithm achieves $\est(T) \le \log |\mathcal{F}|$ (see e.g.~\cite{vovk1995game}).
Therefore, the function class $\mathcal{F}$ should be rich enough to capture the map between contexts and the variation in cost functions and also not be too large so that the estimation error is small.

To construct a function class $\mathcal{F}$ appropriate for our problem setting, we first note that the market operator cannot observe the supplier's costs but can instead use the oracle to regress on the supplier's production $x^*(\cdot; \theta_t)$, which is directly observable.
Note that by the first-order optimality condition on Problem~\eqref{eq:supObj} that the production and price are related as follows:
\[c_t'(x^*(p; \theta_t); \theta_t) = p \implies x^*(p; \theta_t) = (c_t')^{-1}(p; \theta_t).\]
Note that when the cost functions are strongly convex (i.e., $c_t'$ is invertible), the production level $x^*(\cdot; \theta_t)$ is well-defined as a function of the price $p_t$ and context $\theta_t$.
Thus, we define the function class $\mathcal{F}$ as the possible mappings from the price-context tuple $(p, \theta)$ to the production $x^*$, i.e., the oracle tries to determine the amount of the supplier's production given a price $p_t$ and context $\theta_t$.


With our choice of the online regression oracle, we now present our algorithm for the setting of time-varying costs with contexts based on the inverse gap weighing method introduced in~\cite{foster2020beyond}. To present our algorithmic approach, we restrict the algorithm's choices to a finite set of $K$ prices that are uniformly spaced on the interval $[0, 1]$, where the performance of our algorithm will depend on the choice of $K$ (see Theorem~\ref{thm:igw-bound-informal}).
Given the oracle's output $\hat{f}_t$ that estimates the production quantity $x^*(p_t; \theta_t)$, the greedy choice at each period $t$ is to match the requested demand $d_t$ as closely as possible, i.e. choose a price $\hat{p}_t$ such that the quantity $|\hat{f}_t(\hat{p}_t; \theta_t) - d_t|$ is minimized.
To balance between both exploration and exploitation, Algorithm~\ref{alg:time-varying-cost} instead samples each price $p_t$ from the set of $K$ discrete prices according to a probability distribution $\Delta_t$. 
%In particular, rather than deterministically choosing the price $\hat{p}_t$ greedily at each period based on the observed context $\theta_t$ and demand $d_t$, Algorithm~\ref{alg:time-varying-cost} assigns a probability distribution $\Delta_t(\cdot)$ to sample one of the $K$ prices in the discretized price set. 
In effect, Algorithm~\ref{alg:time-varying-cost} achieves good exploration by choosing any of the $K$ prices with some positive probability, which ensures that the online regression oracle has access to a wide-ranging history $\mathcal{H}_{t-1}$, so it can achieve a low prediction error even when the market operator receives a new context or demand. Furthermore, to minimize the penalty for exploration, the probability distribution $\Delta_t$ is chosen such that it assigns the highest probability to the greedy choice $\Hat{p}_t$ under the current oracle estimate $\hat{f}_t$ while assigning a probability to every other price that is roughly inversely proportional to the gap between its unmet or excess demand and that of the greedy choice $\Hat{p}_t$. Then, for each period $t$, given this choice of $\Delta_t$, a price $p_t$ is sampled from this distribution, following which suppliers produce an optimal quantity $x^*(p_t; \theta_t)$ of the commodity as given by the solution of Problem~\eqref{eq:supObj}. Finally, the oracle is updated with the new context $\theta_t$, customer demand $d_t$, and optimal supplier production to generate a new estimator $\Hat{f}_{t+1}$ for the next period. This process is presented formally in Algorithm~\ref{alg:time-varying-cost}.
%The above process is repeated for the $T$ period horizon.

 

\begin{algorithm}
\SetAlgoLined
\SetKwInOut{Input}{Input}\SetKwInOut{Output}{Output}
\Input{Online regression oracle $\mathcal{O}$ with input pairs $(\theta_t, p_t)$ and output $x_t$; uniform $K$-cover of possible prices $0 = p_1 < p_2 < \cdots < p_K = 1$; exploration parameter $\gamma > 0$}
 \For{$t = 1, \dots, T$}{
 Query the oracle $\mathcal{O}$ for an estimator $\hat{f}_t$\;
 Receive context $\theta_t$ and demand $d_t$\;
 Sample price $p_t$ from the probability distribution 
    \[\Delta_t(p_i) = \frac{1}{\lambda + 2 \gamma \left(|\hat{f}_t(p_i; \theta_t) - d_t| - |\hat{f}_t( \hat{p}_t; \theta_t) - d_t|\right)},\]
    with $\hat{p}_t = \argmin_{p \in \{p_1, \dots, p_K\}} |\hat{f}_t(p; \theta_t) - d_t|$ and $\lambda \in (0, K)$ as the normalization constant\;
Commit $p_t$ and observe the production $x_t = x^*(p_t; \theta_t)$ corresponding to the solution of Problem~\eqref{eq:supObj} given the price $p_t$\; 
Update the oracle $\mathcal{O}$ with $((\theta_t, p_t), x_t)$\;
}
\caption{Online Equilibrium Pricing for Time-Varying Costs}
\label{alg:time-varying-cost}
\end{algorithm}

We now present the main result of this section, which establishes that for an appropriate choice of the discretization $K$ of the price set, Algorithm~\ref{alg:time-varying-cost} can achieve sub-linear regret on all three regret metrics, as is elucidated through the following theorem. 
We highlight that this theorem holds for any sequence of contexts $\theta_t$, which means that $\theta_t$ can be derived from some physical dynamics, drawn from a probability distribution, or even chosen adversarially.
%\haoyuan{add references for these settings.} 

\begin{theorem}[Informal]
    \label{thm:igw-bound-informal}
    With high probability, for any sequence of contexts $\theta_t$ and customer demands $d_t$ over $T$ periods, the unmet demand, payment regret, and cost regret satisfy:
    \[ \EE_{p_t \sim \Delta_t, t \in [T]}[U_T(p_1, \dots, p_T)] \le O\left(\sqrt{KT \cdot \est(T)} + \frac{T}{K}\right),\]
    and similarly for $\EE_{p_t \sim \Delta_t, t \in [T]}[P_T(p_1, \dots, p_T)]$ and $\EE_{p_t \sim \Delta_t, t \in [T]}[C_T(p_1, \dots, p_T)]$, where $K$ is the number of prices in the uniformly discretized price set.
\end{theorem}
We note that because the pricing policy in Algorithm~\ref{alg:time-varying-cost} is probabilistic, we present the regret bounds in expectation with respect to the distributions $\Delta_t$.
For the simplicity of exposition, we only present Theorem~\ref{thm:igw-bound-informal} as an informal statement and present the analysis of a rigorous theorem statement, which involves introducing additional notation, e.g., quantifying the high probability bound, in Appendix~\ref{sec:igw-alg-proof}. Furthermore, we present a sketch of the main ideas used to analyze Algorithm~\ref{alg:time-varying-cost} in Section~\ref{sec:igw-alg-sketch}.

\begin{remark}
Recall that, when the function class $\mathcal{F}$ is finite, e.g. there are only finitely many possible cost functions, the exponential weights algorithm can achieve $\est(T) = \log |\mathcal{F}|$.
Therefore, picking $K = \sqrt[3]{T / \log |\mathcal{F}|}$ achieves a sublinear regret bound of $O(T^{2/3} \sqrt[3]{\log |\mathcal{F}|})$.


Additionally, in Appendix~\ref{sec:igw-alg-proof},  we discuss several other function classes $\mathcal{F}$, including parametric classes, bounded Lipschitz functions, and neural networks with bounded spectral norm, for which Algorithm~\ref{alg:time-varying-cost} achieves a sub-linear regret guarantee by leveraging results from the statistical learning literature (e.g. see~\cite{wainwright2019high,rakhlin2014online}).
\end{remark}

Note that the regret bound obtained for finite function classes corresponds to an additional factor of $O(T^{1/6})$ than the guarantee for the setting with time-varying demands and fixed costs (Theorem~\ref{thm:VaryingDemandResult}). The additional loss in the regret for the setting of time-varying costs is because Algorithm~\ref{alg:time-varying-cost} utilizes an online regression oracle to associate the contexts with properties of the supplier's varying cost functions that are unknown to the market operator.
Note that by performing regression, Algorithm~\ref{alg:time-varying-cost} learns all of the features about the supplier's optimal production as a function of the contexts and prices.
In contrast, Algorithm~\ref{alg:time-varying-demand-new} only learns the equilibrium prices for a discrete set of demands and is not concerned with accurately predicting the supplier's production for other demands.
Thus, the additional factor of $O(T^{1/6})$ in the regret is attributable to the fact that Algorithm~\ref{alg:time-varying-cost} attempts to solve a more complex statistical problem than in the fixed cost setting.
Nevertheless, the guarantee in Theorem~\ref{thm:igw-bound-informal} still compares favorably to the best known algorithms in related problem settings, e.g., no online algorithm can achieve a regret better than $O(T^{2/3})$ for a Lipschitz bandit with contexts in $\RR$ (see e.g.~\cite{slivkins2011contextual,foster2020beyond}).

\begin{comment}
A few comments about the bound obtained in Theorem~\ref{thm:igw-bound-informal} and the choice of the parameter $K$ to achieve a sub-linear regret bound on all three performance metrics are in order. In particular, the first term of the regret bound corresponds to the regret associated with achieving a balance between exploration and exploitation if the optimal price at each period belongs to one of the $K$ prices in the discretized set.
On the other hand, the second term $\frac{T}{K}$ of the regret bound corresponds to the error due to the discretization of the price set, as, in general, the equilibrium price at each period belongs to a continuous interval and may not correspond to one of the $K$ prices chosen in Algorithm~\ref{alg:time-varying-cost}. %Observe that, as expected, the regret due to the discretization of the price set is lower for larger values of $K$. 
Then, the choice of $K$ to achieve a sub-linear regret bound by balancing between the two sources of regret depends on the function class $\mathcal{F}$, which influences the prediction error $\est(T)$ corresponding to the online regression oracle. 

As previously illustrated, if the function class $\mathcal{F}$ is finite, we can achieve a regret bound of $O(T^{2/3} \sqrt[3]{\log |\mathcal{F}|})$, which is sub-linear in the number of periods $T$. 
Note that this regret bound corresponds to an additional factor of $O(T^{1/6})$ than the guarantee for the setting with time-varying demands and fixed costs (Theorem~\ref{thm:VaryingDemandResult}). The additional loss in the regret for the setting of time-varying costs is because Algorithm~\ref{alg:time-varying-cost} utilizes an online regression oracle to associate the contexts with properties of the supplier's varying cost functions that are unknown to the market operator.
Note that by performing regression, Algorithm~\ref{alg:time-varying-cost} learns all of the features about the supplier's optimal production as a function of the contexts and prices.
In contrast, Algorithm~\ref{alg:time-varying-demand-new} only learns the equilibrium prices for a discrete set of demands and is not concerned with accurately predicting the supplier's production for other demands.
Thus, the additional factor of $O(T^{1/6})$ in the regret is attributable to the fact that Algorithm~\ref{alg:time-varying-cost} attempts to solve a more complex statistical problem than in the fixed cost setting.
Nevertheless, the guarantee in Theorem~\ref{thm:igw-bound-informal} still compares favorably to the best known algorithms in related problem settings, e.g., no online algorithm can achieve a regret better than $O(T^{2/3})$ for a Lipschitz bandit with contexts in $\RR$ (see e.g.~\cite{slivkins2011contextual,foster2020beyond}).    
\end{comment}





\subsection{Sketch of Main Ideas to Analyze Algorithm~\ref{alg:time-varying-cost}}
\label{sec:igw-alg-sketch}


In this section, we shall describe the key ideas behind the analysis of Algorithm~\ref{alg:time-varying-cost}.
We first establish the necessary Lipshitzness relations between the problem parameters, which enables us to simultaneously optimize over multiple objectives, as opposed to standard contextual bandit algorithms that typically optimize a single regret metric. We then define a notion of ``proxy'' regret and show that Algorithm~\ref{alg:time-varying-cost} achieves the desired regret bound in the statement of Theorem~\ref{thm:igw-bound-informal}. Finally, Theorem~\ref{thm:igw-bound-informal} follows as the ``proxy'' regret serves as an upper bound on the unmet demand, payment regret, and cost regret due to the derived Lipshitz relations.


Our first lemma establishes the Lipschitzness between the optimal prices corresponding to the solution of Problem~\eqref{eq:supObj2}-\eqref{eq:demand-con} and customer demands utilizing techniques from parametric optimization. 

\begin{lemma} [Lipschitzness of Prices in Demands] \label{lem:PriceLipschitz}
The optimal prices corresponding to the dual variables of the market clearing constraint of Problem~\eqref{eq:supObj2}-\eqref{eq:demand-con} are Lipschitz in the demand $d$, i.e., $|p^*(d_1) - p^*(d_2)| \leq L_1 |d_1 - d_2|$ for some constant $L_1>0$ for all $d_1, d_2 \in [\underline{d}, \Bar{d}]$. Here, $p^*(d)$ is the optimal price corresponding to the dual variable of the market clearing constraint of Problem~\eqref{eq:supObj2}-\eqref{eq:demand-con} with a customer demand of $d$.
\end{lemma}

Next, we use Lemma~\ref{lem:PriceLipschitz} to show that both the payment and cost regret metrics are Lipschitz in the prices. That is, Lemma~\ref{lem:lipschitzPricesRegret} establishes that small changes in the prices result in only small changes in the cost and payment regret metrics.

\begin{lemma} [Lipschitzness of Regret Metrics in Prices] \label{lem:lipschitzPricesRegret}
Consider an online pricing policy $\ppi$
The payment and cost regret are upper bounded by the absolute difference in prices $p_t$ corresponding to the online pricing policy $\ppi$ and equilibrium prices $p^*_t$.
Namely, there exists some constant $L_2, L_3 > 0$ so that 
\[P_T(\ppi) \le L_2 \sum_{t=1}^T |p_t - p^*_t|, \text{and }  C_T(\ppi) \le L_3 \sum_{t=1}^T |p_t - p^*_t|. \]
\end{lemma}
For proofs of Lemmas~\ref{lem:PriceLipschitz} and~\ref{lem:lipschitzPricesRegret}, we refer the readers to Appendix~\ref{sec:lipschitz-lemma}.

By employing these two lemmas, we can upper bound unmet demand, payment regret, and cost regret with the following proxy regret metric:
\[\textsc{Reg}(T) := \sum_{t=1}^T \EE_{p_t \sim \Delta_t} \left[\left|x^*(p_t; \theta_t) - d_t\right| \mid \mathcal{H}_{t-1} \right],\]
where $x^*(p_t; \theta_t)$ is the optimal production quantity of the suppliers and $p_t$ is chosen according to the distribution $\Delta_t$ as defined in Algorithm~\ref{alg:time-varying-cost}. 
Note that the choice of distribution $\Delta_t$ depends on the past history $\mathcal{H}_{t-1} = ((p_1, \theta_1), x_1), \dots ((p_{t-1}, \theta_{t-1}), x_{t-1})$.
We can employ Lemmas~\ref{lem:PriceLipschitz} and~\ref{lem:lipschitzPricesRegret} to show that the proxy regret upper bounds the unmet demand, payment regret, and cost regret metrics.

First, we note that this proxy regret upper bounds the unmet demand because
\[\left|x^*(p_t; \theta_t) - d_t\right| \ge \left(x^*(p_t; \theta_t) - d_t\right)_+.\]
Next, let $p^*_t$ be the equilibrium price for demand $d_t$, and note that $p_t$ is the equilibrium price when demand is equal to $x^*(p_t; \theta_t)$.
So, by Lemma~\ref{lem:PriceLipschitz}, it follows that 
\[ |p_t  - p^*_t| \le L_1 \left|x^*(p_t; \theta_t) - d_t\right| \] 
for some constant $L$.
And by applying Lemma~\ref{lem:lipschitzPricesRegret} to the previous inequality, we conclude that 
\[\EE_{p_t \sim \Delta_t, t \in [T]}[P_T] \le L_1 \sum_{t=1}^T \EE_{p_t \sim \Delta_t} \left|p_t - p^*_t\right| \le L_1 L_2 \sum_{t=1}^T \EE_{p_t \sim \Delta_t} \left|x^*(p_t; \theta_t) - d_t\right| = L_1 L_2 \textsc{Reg}(T).\]
Using a similar line of reasoning, we can also show that the proxy regret serves as an upper bound, up to constants, for the cost regret $\EE_{p_t \sim \Delta_t, t \in [T]}[C_T]$.


Given the above observation that the proxy regret is an upper bound on the three regret metrics, it suffices to show that Algorithm~\ref{alg:time-varying-cost} achieves a sub-linear regret, as is elucidated by the following proposition.
\begin{proposition}[informal]
\label{thm:igw-regret-informal}
    With high probability, for any sequence of context $\theta_t$ and demand $d_t$, the proxy regret is bounded by
    \[ \textsc{Reg}(T) \le O\left(\sqrt{KT \cdot \est(T)} + \frac{T}{K}\right).\]
\end{proposition}
Note that Theorem~\ref{thm:igw-bound-informal} is an immediate consequence of Proposition~\ref{thm:igw-regret-informal}. For a complete proof of Proposition~\ref{thm:igw-regret-informal}, see Appendix~\ref{sec:igw-alg-proof}.

\begin{comment}

Note that Theorem~\ref{thm:igw-bound-informal} is an immediate consequence of Proposition~\ref{thm:igw-regret-informal}. In the following, we present an outline of the key ideas for proving Proposition~\ref{thm:igw-regret-informal} and defer the complete proof to Appendix~\ref{sec:igw-alg-proof}.


Depending on the structure of the contexts, we can invoke well-known results from statistical learning to give explicit bounds on $\est(T)$ and therefore provide sub-linear regret guarantee on vaious problems.
Before we proceed, we first define $\mathcal{N}_p(\mathcal{F}, \varepsilon)$ as the $\varepsilon$-covering number of the function class $\mathcal{F}$ with respect to norm $\norm{\cdot}_p$.
\haoyuan{Need to double check the constants here.}
\begin{itemize}
    \item If $\mathcal{F}$ is finite, then with exponential weights algorithm, we get $\est(T) = \log |\mathcal{F}|$ and a choice of $K = T^{1/3}$ results in regret of $O(T^{2/3} \log |\mathcal{F}|)$.
    \item If $\mathcal{F}$ is parametric in the sense that $\mathcal{N}_2(\mathcal{F}, \varepsilon) \asymp \varepsilon^{-d}$ for some $d > 0$, then~\cite{rakhlin2014online} shows that $\est(T) \in O(d \log T)$. So, the regret becomes $\widetilde{O}(d \cdot T^{2/3})$.
    \item If $\mathcal{F}$ are the set of bounded Lipschitz functions over $(\theta, p) \in \RR^{d+1}$, then $\log \mathcal{N}_\infty(\mathcal{F}, \varepsilon) = \varepsilon^{-d-1} $ (see Example 5.10 and 5.11 in~\cite{wainwright2019high} for an explicit construction).
    And~\cite{rakhlin2014online} shows that we then have $\est(T) \in O(T^{(d+1)/(d+2)})$.
    A choice of $K = T^{1/(d+4)}$ results in $\textsc{Reg}(T) \lesssim T^{(d+3)/(d+4)}$.
    \item If $\mathcal{F}$ is a neural network with appropriately bounded spectral norm, then~\cite{rakhlin2014online, bartlett2017spectrally} show that $\est(T) \in O(T^{-1/2})$, and this results in a regret of $O(T^{4/5})$.
\end{itemize}

For more details, please refer to the analysis in Appendix~\ref{sec:igw-alg-proof}.


The key motivation behind Algorithm~\ref{alg:time-varying-cost} is to optimally balance between exploiting the best price given the known information and exploring the price range so the algorithm does not incur a large regret when a new demand $d_t$ is different from all previous demands $d_{1:t-1}$.
We note that the distribution $\Delta_t$ accomplishes this goal by assigning exploration probability to the prices according to how closely they match the demand $d_t$.
This way, under the algorithm's estimate $\hat{f}_t$, the penalty for exploration is up to
\[\sum_{i=1}^K \frac{|\hat{f}_t(p_i; \theta_t) - d_t| - |\hat{f}_t( \hat{p}_t; \theta_t) - d_t|}{\lambda + 2 \gamma \left(|\hat{f}_t(p_i; \theta_t) - d_t| - |\hat{f}_t( \hat{p}_t; \theta_t) - d_t|\right)} \le \frac{K-1}{2\gamma}.\]
After bounding the difference between the estimated production $\hat{f}_t$ and the actual production function
Recall that $\hat{p}_t \argmin_{p \in \{p_1, \dots, p_K\}} |x^*(p; \theta_t) - d_t|$, so the summation term is 0 for one of the price $p_i$.
After bounding the difference in the estimated production function $\hat{f}_t$ and the actual production function $x^*$, we have the following inequality:
\begin{align*}
    &\EE_{p_t \sim \Delta_t} \left[|x^*(\theta_t, p_t) - d_t| \, \bigg| \, \mathcal{H}_{t-1}\right] - \min_{p \in \{p_1, \dots, p_K\}} |x^*(p; \theta_t) - d_t| \\
    \le{}& \frac{K}{\gamma} + \frac{\gamma}{2} \cdot \EE_{p_t\sim\Delta_t} \left[ (|x^*(p_t; \theta_t) - d_t| - |\hat{f}_t(p_t; \theta_t) - d_t|)^2 \right]
\end{align*}
We now try to relate the quantity $(|\hat{f}_t(p_i; \theta_t) - d_t| - |\hat{f}_t( \hat{p}_t; \theta_t) - d_t|)^2$ with the guarantee of the online regression oracle $\est(T)$.
First, we bridge the difference between the regret metric we try to optimize and the output of online regression oracle by computing
\[(|x^*(\theta_t, p_t) - d_t| - |\hat{f}_t(\theta_t, p_t) - d_t|)^2 \le (x^*(\theta_t, p_t) - \hat{f}_t(\theta_t, p_t))^2.\]
Next, through standard techniques in probability theory, we can show that the online regression oracle also satisfies 
    \[\sum_{t=1}^T \EE_{p_t \sim \Delta_t} \left[(x^*(\theta_t, p_t) - \hat{f}_t(\theta_t, p_t))^2\right] \le \est(T)\]
with high probability.
This means that 
\[\sum_{t=1}^T \EE_{p_t \sim \Delta_t} \left[|x^*(\theta_t, p_t) - d_t| \, \bigg| \, \mathcal{H}_{t-1}\right] - \min_{p \in \{p_1, \dots, p_K\}} |x^*(p; \theta_t) - d_t| \le \frac{KT}{\gamma} + \gamma \cdot \est(T).\]
Due to Lemma~\ref{lem:ProductionLipschitz}, we know that there exists constant $L > 0 $ where
\[ \min_{p \in \{p_1, \dots, p_K\}} |x^*(p; \theta_t) - d_t| = \min_{p \in \{p_1, \dots, p_K\}} |x^*(p; \theta_t) - x^*(p^*_t, \theta_t)|\le L \cdot \min_{p \in \{p_1, \dots, p_K\}} |p - p^*_t| \le \frac{L}{K}. \]
Hence,
\[\sum_{t=1}^T \min_{p \in \{p_1, \dots, p_K\}} |x^*(p; \theta_t) - d_t| \in O\left(\frac{T}{K}\right).\]
Finally, a choice of $\gamma = \sqrt{\est(T)/(KT)}$ leads to the bound in Proposition~\ref{thm:igw-regret-informal}.
\end{comment}

\section{Conclusion}\label{sec:conclusion}
In this work, we focus on addressing the fundamental challenge of OOD detection tasks, which is how to fully understand the semantic discrepancy between the ID/OOD samples. We reveal that the key to success in the realistic SCOOD task is to allocate as many ID samples in the unlabeled set correctly as possible. To this end, we propose a novel uncertainty-aware optimal transport scheme that introduces class-specific energy scores as guidance for effective label assignment. Experimental results show that our method achieves better performance than previous state-of-the-art methods on SCOOD benchmarks.

\textbf{Limitations.} In addition to temperature scaling, other techniques such as feature clipping applied in ReAct~\cite{sun2021react} also enhance the performance of energy score, so how to obtain an OOD score that best fits the SCOOD task can be further explored. Moreover, a setting highly related to SCOOD has been proposed in \cite{katz2022training} and formulated as a constrained optimization problem. We will also theoretically analyze these practical OOD settings in our feature work.

% \section*{Acknowledgments}
\textbf{Acknowledgments.} 
This work is supported by National Key R\&D Program of China under Grant 2020AAA0105701, National Natural Science Foundation of China (NSFC) under Grants 61872327, Major Special Science and Technology Project of Anhui, National Natural Science Foundation of China (62033012) and Ant Group through Ant Research Intern Program.


\bibliographystyle{unsrt}
\bibliography{main}


\appendix

\section{Proof of Theorem~\ref{thm:IdenticalResult}} \label{sec:pf-identical-result}

%\begin{proof}
Let $p^*$ be the optimal price, and let $x_{it}^*$ be the optimal production quantity for each suppliers $i$ at each period $t$. We first show that the payment and cost regret are of order $O(\log \log T)$. To this end, first observe by monotonicity of the optimal supplier production in the prices that both payment and cost regret are only accumulated when $p_t>p^*$. Then, as we repeatedly square the precision parameter, the number of sub-phases of repeated squaring necessary to get from a step size of $0.5$ to a step size of $\frac{1}{T}$ is $O(\log \log T)$. Next, observe that the price $p_t>p^*$ at most once in each sub-phase. By the boundedness of the supplier production quantities for prices in the range $[0, 1]$, it follows that the total payment and cost regret are $O(\log \log T)$.

\begin{comment}
    Recall that, given a sequence of prices $p_t$ of Algorithm~\ref{alg:fixed-demand}, we seek to bound the following quantities:
\begin{enumerate}
    \item Unmet Demand: $\sum_{t = 1}^T \left( d - \sum_{i = 1}^n x_{it}^*(p_t) \right)_+$,
    \item Payment Regret: $\sum_{t = 1}^T \left( p_t \sum_{i = 1}^n x_{it}^*(p_t) - p^* d \right)_+$, and
    \item Cost Regret: $\sum_{t = 1}^T \left( \sum_{i = 1}^n c_i^t(x_{it}^*(p_t)) - \sum_{i = 1}^n c_i^t(x_{it}^*(p^*)) \right)_+$.
\end{enumerate}
\end{comment}




We now show that the unmet demand is also $O(\log \log T)$. To this end, first note that there is unmet demand at each period when $p_t < p^*$. However, in each sub-phase other than the first sub-phase and last phase when the price is fixed at the lower end of the feasible price interval, it holds that the length of the feasible interval $[a,b]$ is $\sqrt{\epsilon}$, with sub-intervals of length $\epsilon$. As there are $\frac{1}{\sqrt{\epsilon}}$ sub-intervals at most $\frac{1}{\sqrt{\epsilon}}$ offers are made. Next, we note by Lemma~\ref{lem:ProductionLipschitz} that the optimal production quantity for each supplier, given by the solution to Problem~\eqref{eq:supObj}, is Lipschitz continuous, i.e., $|x_{it}^*(p_t) - x_{it}^*(p^*)| \leq \frac{1}{\mu} |p-p^*|$, where $\mu = \min_{i \in [n]} \mu_i$ where $\mu_i$ is the strong convexity modulus for supplier $i$'s cost function $c_i(\cdot)$. As a result, it holds that
\begin{align*}
    |x_{it}(p_t) - x_{it}(p^*)| \leq \frac{1}{\mu} |p-p^*| \leq \frac{1}{\mu} (b-a) \leq \frac{1}{\mu} \sqrt{\epsilon},
\end{align*}
for all suppliers $i \in [n]$. Next, the total unmet demand at period $t$ is given by
\begin{align*}
    \left( d - \sum_{i = 1}^n x_{it}^*(p_t) \right)_+ \leq \sum_{i = 1}^n |x_{it}^*(p_t) - x_{it}^*(p^*)| \leq n \frac{1}{\mu} \sqrt{\epsilon} = O(n \frac{1}{\mu} \sqrt{\epsilon}).
\end{align*}
Then, since $\frac{1}{\sqrt{\epsilon}}$ offers are made in each sub-phase, it follows that the total unmet demand is $O(n \frac{1}{\mu})$ in each phase. As there are $O(\log \log T)$ sub-phases, it thus holds that the total unmet demand is $O(n \frac{1}{\mu} \log \log T)$ before the final phase when the price is kept fixed at the lower end of the feasible interval. 

In the final phase, note that the length of the feasible interval is less than $\frac{1}{T}$ and by Lemma~\ref{lem:ProductionLipschitz}, it holds that $|x_{it}^*(p_t) - x_{it}^*(p^*)| \leq \frac{1}{\mu T}$. Since there are at most $T$ offers in the final phase, the total unmet demand in this phase is at most $O(n \frac{1}{\mu})$. Thus, we have established that the total unmet demand is $O(n \frac{1}{\mu} \log \log T) = O(\log \log T)$, which proves our claim.
%\end{proof}

\section{Proof of Theorem~\ref{thm:VaryingDemandResult}}
\label{sec:vary-demand-pf}

%\begin{proof}[Proof of Theorem~\ref{thm:VaryingDemandResult}]
We first show that payment and cost regrets are at most $O(\sqrt{T} \log\log T)$.
Note that at any period $t$, payment or cost regrets are accumulated only if $x^*(p_{k_t}) > d_t \ge a_{k_t}$, where $x^*(p) = \sum_{i = 1}^n x_i^*(p)$.
Due to the assumption on the boundedness of prices, the amount of payment or cost regret incurred at any period is at most a constant.
Thus, it suffices to bound the number of times the feasible price set is shrunk.
For any interval $I_k, k \in [K]$, the number of times we repeatedly square the precision parameter $\epsilon_k$ to shrink its associated feasible price set to below $1/\sqrt{T}$ is $O(K \log\log T) = O(\sqrt{T} \log\log T)$.
It thus follows that the total payment and cost regret of Algorithm~\ref{alg:time-varying-demand-new} is at most $O(K \log\log T) = O(\sqrt{T} \log\log T)$.

Next, we show that the unmet demand is also at most $O(\sqrt{T} \log\log T)$.
For each period $t$, there are two possibilities:
\begin{itemize}
    \item $|\S_{k_t}| \le 1/\sqrt{T}$: From the construction of our algorithm, as all demands in each sub-interval are considered equal to the lower bound of that interval, $p^*(a_{k_t}) \in \S_{k_t}$ is always satisfied. 
    By Lemma~\ref{lem:ProductionLipschitz} and monotonicity of the cumulative production of all suppliers in the prices, we have
    \[a_k \le x^*(\max \S_{k_t}) \le x^* \left(p_{k_t} + \frac{1}{\sqrt{T}} \right) \le x^*(p_{k_t}) + O(1/\sqrt{T})\]
    Since $d_t \le a_{k_t} + \gamma$, we have that $d_t \le x^*(p_{k_t}) + O(1/\sqrt{T})$.
    Thus, over $T$ periods, the total amount of unmet regret incurred in this case is then $O(T \cdot 1/\sqrt{T}) = O(\sqrt{T})$.
    \item $|\S_{k_t}| > 1/\sqrt{T}$: Since $|\S_{k_t}|^2 = \varepsilon_{k_t}$, there are at most $1/\sqrt{\varepsilon_{k_t}}$ periods where the demand sub-interval $I_{k_t}$ is associated with the feasible set $\S_{k_t}$.
    By Lemma~\ref{lem:ProductionLipschitz} and the monotonicity of the cumulative production of all suppliers in the prices, we have
    \[a_k \le x^*(\max \S_{k_t}) \le x^*(p_{k_t} + \sqrt{\varepsilon_{k_t}}) \le x^*(p_{k_t}) + O(\sqrt{\varepsilon_{k_t}})\]
    Since $d_t \le a_{k_t} + \gamma < a_{k_t} + |\S_{K_t}|$, we have that $d_t \le x^*(p_{k_t}) + O(\sqrt{\varepsilon_{k_t}})$.
    Therefore, in total, the unmet demand incurred for interval $I_{k_t}$ that are associated with $\S_{k_t}$ is up to $O(\frac{1}{\sqrt{\varepsilon_{k_t}}} \cdot \sqrt{\varepsilon_{k_t}}) = O(1)$.
    Furthermore, for any interval $I_k, k \in [K]$, it takes up to $O(\log\log \sqrt{T}) = O(\log\log T)$ times to shrink its associated feasible price set to below $1/\sqrt{T}$.
    It thus follows that the unmet demand incurred in this case is at most $O(K \log\log T \cdot 1) = O(\sqrt{T} \log\log T)$.
\end{itemize}
Therefore the unmet demand is also at most $O(\sqrt{T} \log\log T)$, which establishes our claim.
%\end{proof}

\section{Proof of Proposition~\ref{prop:time-varying-cost-countereg}} \label{sec:pf-prop-countereg}

%\begin{proof}
We consider a setting with a fixed demand of $d=1$ at every period and a single supplier whose cost functions at each period are drawn from a distribution such that at each period $t$, its cost function could be either $c_1(x) = \frac{1}{8}x^2$ or $c_2(x) =\frac{1}{16} x^2$, each with probability $0.5$. We suppose that the market operator has knowledge of the distribution from which the supplier's cost function is sampled i.i.d. but does not know the outcome of the random draw at any period and show that any pricing strategy adopted by the operator must incur a linear regret one at least one of the three regret measures for this instance. 

To prove this claim, we first define the \emph{total regret} as the sum of the unmet demand, payment regret, and cost regret and note that if the total regret is linear in the number of periods $T$, then at least one of the three regret measures must be linear in $T$. To analyse the total regret, we first analyse the each of the regret measures for a given price $p$ for both the cost functions.

\begin{enumerate}
    \item For the first cost function $c_1(x)$, the optimal production level given a price $p$ is $x^*(p) = 4p$, so the equilibrium price is $p^* = \frac{1}{4}$.
    \begin{itemize}
        \item The payment regret at price $p$ is $p(4p) - 1/4 = 4p^2 - 1/4$.
        \item The cost regret at price $p$ is $\frac{1}{8}(4p)^2 - 1/8 = 2p^2 - 1/8$.
        \item The unmet demand is $1-4p$ if $p < 1/4$ and 0 otherwise.
    \end{itemize}
    \item For the second cost function $c_2(x)$, the optimal production level given a price $p$ is $x^*(p) = 8p$, so the equilibrium price is $p^* = \frac{1}{8}$.
    \begin{itemize}
        \item The payment regret at price $p$ is $p(8p) - 1/8 = 8p^2 - 1/8$.
        \item The cost regret at price $p$ is $\frac{1}{16}(8p)^2 - 1/16 = 4p^2 - 1/16$.
        \item The unmet demand is $1-8p$ if $p < 1/8$ and 0 otherwise.
    \end{itemize}
\end{enumerate}
Then, the expected total regret at each period $t$ is as follows:
\begin{itemize}
    \item If $p < 1/8$: expected total regret is $\frac{1}{2}(18p^2 - 9/16 + (1-4p) + (1-8p)) = 9p^2 - 6p + 23/32$.
    \item If $1/8 \le p \le 1/4$: expected total regret is $\frac{1}{2}(18p^2 - 9/16 + (1-4p)) = 9p^2 - 2p + 7/32$.
    \item If $p > 1/4$: expected total regret is $\frac{1}{2}(18p^2 - 9/16) = 9p^2 - 9/32$.
\end{itemize}
From the above obtained relations, we can derive that the expected total regret at any period $t$ is at least $7/64$, which is attained when $p = 1/8$.
It thus follows that, regardless of the pricing strategy adopted by the market operator, that the total expected regret is at least $\frac{7}{64} T$.
Hence, either the unmet demand, payment regret, or cost regret are not sublinear, which establishes our claim.
%\end{proof}

\section{Regret Measures for Time-Varying Cost Functions} \label{sec:new-regret-defs}
In this section, we will re-define the problem setting and performance metrics with respect to the augmented setting as described in Section~\ref{sec:contextual-bandit}.
For brevity, this section focuses on clarifying the mathematical definitions and we refer the readers to Section~\ref{sec:model} for a complete discussion of the motivations and reasoning behind these definitions.
Recall that, in Section~\ref{sec:contextual-bandit}, we parameterized the suppliers' cost functions with an unknown time-invariant component $\phi_i$ and a time-varying component $\theta_{it}$ that is revealed to the market operator, i.e., 
\[ c_{it}(\cdot) = c(\cdot; \phi_i, \theta_{it}). \]

With this definition in mind, at each period $t$, the suppliers seek to maximize their profits at a given price $p$ through the following optimization problem:
\begin{maxi}|s|[2]                   % mini! = minimize 
    {x_{it} \geq 0}                               % optimization variable
    {p x_{it} - c_i(x_{it}; \phi_i, \theta_{it}). \label{eq:supObj-aug}}   % objective function and label
    {}             % label for optimization problem
    {x^*_{i}(p; \phi_i, \theta_{it}) = }                                % optimization result
\end{maxi}
And when the cost functions $c_{it}$'s are convex, we can find the market equilibrium price by solving for the dual variables of the following optimization problem:
\begin{mini!}|s|[2]                   % mini! = minimize 
    {x_{it} \geq 0, \forall i \in [n]}                               % optimization variable
    {\sum_{i = 1}^n c_i(x_{it}; \phi_i, \theta_{it}), \label{eq:supObj2-aug}}   % objective function and label
    {\label{eq:minCost-aug}}             % label for optimization problem
    {}                                % optimization result
    \addConstraint{\sum_{i = 1}^n x_{it}}{= d_t, \label{eq:demand-con-aug}} 
\end{mini!}
Note that Problems~\eqref{eq:supObj-aug} and~\eqref{eq:supObj2-aug}-\eqref{eq:demand-con-aug} differ from their counterparts in Section~\ref{sec:market-model} (i.e. Problems~\eqref{eq:supObj} and~\eqref{eq:supObj2}-\eqref{eq:demand-con}) only in the parametrization of cost functions $c_{it}$'s.



Like in Section~\ref{sec:perf-measures}, we evaluate the efficacy of an online algorithm for this setting with three regret metrics---unmet demand, payment regret, and cost regret.
Specifically, over the $T$ periods, the market operator sets a sequence of prices $p_t$ according to the online algorithm's policy $\ppi = (\pi_1, \ldots, \pi_T)$, where $p_t = \pi_t(\{ (x_{it'}^*)_{1 = 1}^n, d_{t'}, \theta_{t'} \}_{t'=1}^{t-1}, d_t, \theta_t)$ depends on the past history on the suppliers' production, consumer demands, and contexts. 
The three regret metrics represent the sub-optimality of the policy $\ppi$ relative to the optimal offline algorithm with complete information on the three desirable properties of equilibrium prices as described in Section~\ref{sec:market-model} (i.e. market clearing, minimal supplier cost, and minimal payment).

\paragraph{Unmet Demand:} We evaluate the unmet demand of an online pricing policy $\ppi$ as the sum of the differences between the demand and the total supplier productions corresponding to the pricing policy $\ppi$ at each period $t$. In particular, for an online pricing policy $\ppi$ that sets a sequence of prices $p_1, \ldots, p_T$, the cumulative unmet demand is given by
\begin{align*}
    U_T(\ppi) = \sum_{t = 1}^T \left( d_t - \sum_{i = 1}^n x_{i}^*(p_t; \phi_i, \theta_{it}) \right)_+.
\end{align*}

\paragraph{Cost Regret:} We evaluate the cost regret of an online pricing policy $\ppi$ through the difference between the total supplier production cost corresponding to algorithm $\ppi$ and the minimum total production cost, given complete information on the supplier cost functions. In particular, the cost regret $C_T(\ppi)$ of an algorithm $\ppi$ is given by
\begin{align*}
    C_T(\ppi) = \sum_{t = 1}^T \sum_{i = 1}^n c_{i}(x_{it}^*(p_t; \phi_i, \theta_{it}); \phi_i, \theta_{it}) - c_{i}(x_{it}^*(p^*_t; \phi_i, \theta_{it}); \phi_i, \theta_t),
\end{align*}
where the price $p_t^*$ for each period $t \in [T]$ is the optimal price corresponding to the solution of Problem~\eqref{eq:supObj2}-\eqref{eq:demand-con} given the demand $d_t$ and supplier cost functions $c_{it}$ for all $i \in [n]$.

\paragraph{Payment Regret:} Finally, we evaluate the payment regret of online pricing policy $\ppi$ through the difference between the total payment made to all suppliers corresponding to algorithm $\ppi$ and the minimum total payment, given complete information on the supplier cost functions. In particular, the payment regret $P_T(\ppi)$ of an algorithm $\ppi$ is given by
\begin{align*}
    P_T(\ppi) = \sum_{t = 1}^T \sum_{i = 1}^n p_t x_{i}^*(p_t; \phi_i, \theta_{it})-  p^* x_{it}^*(p^*_t; \phi_i, \theta_{it}).
\end{align*}

We note that, compared to the corresponding definitions in Section~\ref{sec:perf-measures}, we merely explicitly write out the suppliers' cost functions and production levels in terms of the parameterization of an unknown time-invariant component and a known time-varying component as discussed in Section~\ref{sec:contextual-bandit}.



\section{Proof of Lemmas~\ref{lem:PriceLipschitz} and \ref{lem:lipschitzPricesRegret}}
\label{sec:lipschitz-lemma}

\begin{proof}[Proof of Lemma~\ref{lem:PriceLipschitz}]
Defining the conjugate function $f_{it}^*(p) = \min_{x_{it} \geq 0} \{ c_{it}(x_{it}) - p x_{it} \}$, we first formulate the following dual of Problem~\eqref{eq:supObj2}-\eqref{eq:demand-con}:
\begin{maxi!}|s|[2]                   % mini! = minimize 
    {p}                               % optimization variable
    {g(p, d_t) := p d_t + \sum_{i = 1}^n f_{it}^*(p). \label{eq:supObj2Dual}}   % objective function and label
    {\label{eq:minCostDual}}             % label for optimizatio problem
    {}                                % optimization result
\end{maxi!}
Since the cost functions $c_{it}$ are continuously differentiable with a compact domain, they are Lipschitz smooth.
Then, from the properties of the conjugate function, the dual function is strongly concave (see~\cite{zhou2018fenchel}).
By the strong concavity of the dual function it follows that
\begin{align}
    \frac{\mu}{2} (p^*(d_1) - p^*(d_2))^2 
    &\leq -g(p^*(d_1), d_2) - (-g(p^*(d_2), d_2)) + \nabla (-g(p^*(d_2), d_2)) (p^*(d_2) - p^*(d_1)) \nonumber \\
    &= -g(p^*(d_1), d_2) + g(p^*(d_2), d_2), \label{eq:strong-convexity-relation}
\end{align}
where the equality follows as $\nabla g(p^*(d_2), d_2) = d_2 - \sum_{i = 1}^n x_{it}(p^*(d_2)) = 0$.

Next, we observe that
\begin{align*}
    - g(p^*(d_1), d_2) + g(p^*(d_2), d_2) &= [- g(p^*(d_1), d_2) + g(p^*(d_1), d_1)] - [- g(p^*(d_2), d_2) + g(p^*(d_2), d_1)] \\
    &\hspace{10em} + [- g(p^*(d_1), d_1) + g(p^*(d_2), d_1)], \\
    &\stackrel{(a)}{\leq} [-g(p^*(d_1), d_2) + g(p^*(d_1), d_1)] - [-g(p^*(d_2), d_2) + g(p^*(d_2), d_1)], \\
    &\stackrel{(b)}{=} \left[ -p^*(d_1) d_2 - \sum_{i = 1}^n f_{it}^*(p^*(d_1)) + p^*(d_1) d_1 + \sum_{i = 1}^n f_{it}^*(p^*(d_1)) \right] \\
    &\hspace{2em}- \left[- p^*(d_2) d_2 - \sum_{i = 1}^n f_{it}^*(p^*(d_2)) + p^*(d_2) d_1 + \sum_{i = 1}^n f_{it}^*(p^*(d_2)) \right] \\
    &= (p^*(d_2) - p^*(d_1)) (d_2 - d_1).
\end{align*}
where (a) follows as $g(p^*(d_1), d_1) \geq g(p^*(d_2), d_1)$ by the optimality of the dual function $g$ at the optimal price $p^*(d_1)$ for the demand $d_1$, and (b) follows by the definition of $g$.

From the above inequality and the strong concavity relation in Equation~\eqref{eq:strong-convexity-relation}, we obtain that
\begin{align*}
    \frac{\mu}{2} (p^*(d_1) - p^*(d_2))^2 \leq (p^*(d_1) - p^*(d_2)) (d_2 - d_1) \leq |p^*(d_1) - p^*(d_2)| |d_2 - d_1|,
\end{align*}
which in turn implies our desired Lipschitz condition that
\begin{align*}
    |p^*(d_1) - p^*(d_2)| \leq \frac{2}{\mu} |d_1 - d_2|,
\end{align*}
which establishes our claim for the Lipschitz constant $L = \frac{2}{\mu}$, where $\mu$ is the strong concavity modulus of the dual function $g$.
\end{proof}

\begin{proof}[Proof of Lemma~\ref{lem:lipschitzPricesRegret}]
We have the following relation for the payment regret:
\begin{align*}
    p_t \sum_{i = 1}^n x_{it}(p) - p^* d_t &\stackrel{(a)}{=} (p_t - p^*) \sum_{i = 1}^n x_{it}(p_t) + p^* \left(\sum_{i = 1}^n x_{it}(p_t) - \sum_{i = 1}^n x_{it}(p^*) \right), \\
    &\stackrel{(b)}{\leq} (p_t - p^*) \sum_{i = 1}^n x_{it}(p_t) + p^* \left(\sum_{i = 1}^n | x_{it}(p_t) - x_{it}(p^*) | \right), \\
    &\stackrel{(c)}{\leq} (p_t - p^*) n \Bar{x} + n \lambda (p_t - p^*) = (n \Bar{x} + n \lambda) (p_t - p^*) = O(\gamma),
\end{align*}
where (a) follows from adding and subtracting $p^* \sum_{i = 1}^n x_{it}(p_t)$, (b) follows as $x_{it}(p_t) - x_{it}(p^*) \leq |x_{it}(p_t) - x_{it}(p^*)|$ for all $t \in [T]$, and (c) follows by the boundedness of the production $x_{it}(p_t) \leq x_{it}(1) \leq \Bar{x}$ for all prices $p_t \in [0, 1]$ and the lipshitzness of the production in the prices (see Lemma~\ref{lem:ProductionLipschitz}).

Analogously, we have the following relation for the cost regret:
\begin{align*}
    \sum_{i = 1}^n c_i^t(\mathbf{x}_{it}(p_t)) - \sum_{i = 1}^n c_i^t(\mathbf{x}_{it}(p^*)) &\stackrel{(a)}{\leq} \sum_{i = 1}^n \nabla c_i^t(\mathbf{x}_{it}(p_t)) (\mathbf{x}_{it}(p_t) - \mathbf{x}_{it}(p^*)), \\
    &\stackrel{(b)}{\leq} \nabla c_i^t(\mathbf{x}_{it}(1)) \sum_{i = 1}^n \lambda (p_t-p^*) = O(\gamma),
\end{align*}
where (a) follows by the convexity of the cost functions and (b) follows as the cost functions are monotonically increasing and by the Lipshitzness of the production in the prices (see Lemma~\ref{lem:ProductionLipschitz}).

The above relations establish our claim that if $|p_t-p^*| = O(\gamma)$ for a given period $t$, then the payment and cost regret for period $t$ are also $O(\gamma)$.
\end{proof}

\section{Technical tools: Freedman Inequality}
The following is a Freedman-type inequality, which are generalization of Bernstein's inequality to martingale~\cite{freedman1975tail}.
Such inequality has been extensively employed in the bandit literature, e.g. \cite{agarwal2014taming, foster2020beyond}.
\begin{lemma}\label{lem:freedman1}
    Let $\{X_t\}_{t \le T}$ be a martingale difference sequence adapted to a filtration $\{\mathfrak{F}_t\}_{t \le T}$.
        If $|X_t| \le R$ almost surely, then for any $\eta \in (0, 1/R)$,
        \[\sum_{t=1}^{T} X_t \le \eta \sum_{t=1}^{T} \EE[X_t^2 \mid \mathfrak{F}_{t-1}] + \frac{\log(1/\delta)}{\eta}\]
        with probability at least $1-\delta$.
\end{lemma}

By applying this lemma to $X_t - \EE[X_t \mid \mathfrak{F}_{t-1}]$, we have the following
\begin{lemma}\label{lem:freedman2}
    Let $\{X_t\}_{t \le T}$ be a sequence adapted to a filtration $\{\mathfrak{F}_t\}_{t \le T}$. If $0 \le X_t \le R$ almost surely, then
        \[\sum_{t=1}^{T} X_t \le \frac{3}{2} \sum_{t=1}^{T} \EE[X_t \mid \mathfrak{F}_{t-1}] + 4 \log(2/\delta)\]
        and
        \[\sum_{t=1}^{T} \EE[X_t \mid \mathfrak{F}_{t-1}] \le 2 \sum_{t=1}^{T} X_t + 8 R \log(2/\delta)\]
        with probability at least $1-\delta$.
\end{lemma}

\section{Formal Regret Analysis of Algorithm~\ref{alg:time-varying-cost}}
\label{sec:igw-alg-proof}
In this section, we shall formalize the results in Section~\ref{sec:igw-alg-sol} and present a rigorous proof leveraging ideas from~\cite{foster2020beyond}.
Additionally, we will derive the regret guarantees for several additional examples of function classes $\mathcal{F}$ on the suppliers' optimal production with respect to the context $\theta_t$ and the price $p_t$.
First, we quantify the descriptiveness of the function class $\mathcal{F}$ with the following property.
\begin{definition}
A function class $\mathcal{G} : \mathcal{A} \to \mathcal{B}$ is \textit{well-specified} with respect to the ground truth $g^*$ if $g^* \in \mathcal{G}$, and $\mathcal{G}$ is $\varepsilon$-miss-specified with respect to the ground truth $g^*$ if:
\[\exists \bar{g} \in \mathcal{G} \text{, s.t. } \forall a \in \mathcal{A} \text{, we have } \norm{\bar{g}(a) -  g^*(a)}_\mathcal{B} \le \varepsilon. \]
Note that being well-specified is equivalent to being $0$-miss-specified.
\end{definition}

Recall that in Section~\ref{sec:igw-alg-sketch}, we showed that the three desired regret metrics are all upper bounded by the proxy regret (up to some positive constants):
\[ \textsc{Reg}(T) = \sum_{t=1}^T \EE_{p_t \sim \Delta_t} \left[ |x^*(p_t; \theta_t) - d_t|\mid \mathcal{H}_{t-1} \right]. \]
For ease of notation, we use $\lesssim$ as a shorthand notation that the left-hand side is smaller than some fixed constant times the right-hand side.
And we can bound this proxy regret as follows.

\begin{theorem}\label{thm:igw-regret}
    If the function class $\mathcal{F}$ is $\varepsilon$-miss-specified with respect to the suppliers' optimal production $x^*(p_t; \theta_t)$, then with probability $1-\delta$, and for any sequence of contexts $\theta_t$ and demands $d_t$, Algorithm~\ref{alg:time-varying-cost} achieves the following proxy regret bound:
    \[ \textsc{Reg} \lesssim \sqrt{KT \cdot \est(T)} + \varepsilon \sqrt{K} \cdot T + \frac{T}{K} + \sqrt{KT \log(1/\delta)}.\]
\end{theorem}

Noting from Section~\ref{sec:igw-alg-sketch} that the three desired regret metrics are all upper bounded by the proxy regret (up to some positive constants), Theorem~\ref{thm:igw-regret} allows us to formally state the result given by Theorem~\ref{thm:igw-bound-informal} as follows.
\begin{theorem} \label{thm:main-contextual}
    If the function class $\mathcal{F}$ is $\varepsilon$-miss-specified with respect to the suppliers' optimal production $x^*(p_t; \theta_t)$, then with probability $1-\delta$, and for any sequence of contexts $\theta_t$ and demands $d_t$, Algorithm~\ref{alg:time-varying-cost} achieves the following bound on our regret metrics:
    \[ \EE_{p_t \sim \Delta_t, t \in [T]}[U_T(p_1, \dots, p_T)] \lesssim \sqrt{KT \cdot \est(T)} + \varepsilon \sqrt{K} \cdot T + \frac{T}{K} + \sqrt{KT \log(1/\delta)},\]
    and similarly for $\EE_{p_t \sim \Delta_t, t \in [T]}[P_T(p_1, \dots, p_T)]$ and $\EE_{p_t \sim \Delta_t, t \in [T]}[C_T(p_1, \dots, p_T)]$, where $K$ is the number of prices in the uniformly discretized price set.
\end{theorem}

We refer to Appendix~\ref{sec:full-pf-contextual-thm} for a complete proof of Theorem~\ref{thm:igw-regret}. Furthermore, in Appendix~\ref{sec:implications-thm-contextual}, we present explicit regret bounds corresponding to Theorem~\ref{thm:igw-regret} for various function classes $\mathcal{F}$.

\subsection{Regret Bounds for Different Function Classes $\mathcal{F}$} \label{sec:implications-thm-contextual}

%Before presenting the proof of Theorem~\ref{thm:igw-regret}, 

In this section, we shall combine Theorem~\ref{thm:main-contextual} with results in the statistical learning literature to establish concrete regret bounds for several examples of function classes $\mathcal{F}$.
To this end, first recall from Section~\ref{sec:igw-alg-sol} that we already obtained an explicit bound on the regret for finite function classes $\mathcal{F}$. Thus, we focus the following discussion on infinite function classes.
To do so, we first quantify the ``size'' of an infinite function class with the notion of \textit{sequential covering}.

\begin{definition}
    Given a real-valued function space $\mathcal{G} : \mathcal{A} \to \RR$ and a sample set $S = \{a_1, \dots, a_n\}$, we say that a finite set of functions $\mathcal{G}'$ is an \textit{$\varepsilon$-sequential cover} of $\mathcal{G}$ with respect to $S$ if 
    \[ \forall \, g \in \mathcal{G}, \exists \, g' \in \mathcal{G}' \text{ s.t. } \left(\frac{1}{n} \sum_{i=1}^n (g(a_i) - g'(a_i))^2\right)^{1/2} < \varepsilon. \]
    Then, the \textit{$\varepsilon$-sequential covering number} of $\mathcal{G}$ is the size of the smallest $\varepsilon$-sequential cover with respect to the sample set $S$, 
    \[ \mathcal{N}_2(\mathcal{G}, \varepsilon, S) = \min \{|\mathcal{G}'| : \mathcal{G}' \text{ is a $\varepsilon$-sequential cover of $\mathcal{G}$ with respect to $S$}\}. \]
    Finally, denote $\mathcal{N}_2(\mathcal{G}, \varepsilon) = \sup_{S \text{ finite}} \mathcal{N}_2(\mathcal{G}, \varepsilon, S)$.
\end{definition}

As shown in~\cite{rakhlin2014online}, the prediction accuracy can be expressed in terms of the sequential covering number of the function class $\mathcal{F}$.
\begin{theorem}[see \cite{rakhlin2014online}]
    There exist online regression oracles achieving the following bounds:
    \begin{itemize}
        \item If $\mathcal{F}$ is finite, then $\est(T) \le \log |\mathcal{F}|$.
        \item If $\mathcal{F}$ is parametric in the sense that $\mathcal{N}_2(\mathcal{F}, \varepsilon) \in O(\varepsilon^{-m})$, then $\est(T) \lesssim m \cdot \log(T)$.
        \item If $\mathcal{F}$ is non-parametric in the sense that $\log \mathcal{N}_2(\mathcal{F}, \varepsilon) \in O(\varepsilon^{-m})$, then $\est(T) \lesssim T^{1 - 2/(2+m)}$ if $m \in (0, 2)$ and $\est(T) \lesssim T^{1 - 1/m}$ if $m \ge 2$.
    \end{itemize}
\end{theorem}
In many cases, we can easily construct efficient algorithms that match or nearly match these bounds.
For example, for finite $\mathcal{F}$, we can achieve the bound $\est(T) \le \log |\mathcal{F}|$ with the classical exponential weights update algorithm~\cite{vovk1995game}.
And when $\mathcal{F}$ is a linear class in the sense that
\[\mathcal{F} =\{(p, \theta) \mapsto \langle \phi, \sigma(p, \theta) \rangle : \phi \in B^m_2\},\]
where $\sigma$ is a fixed feature map, then the Vovk-Azoury-Warmuth forecaster achieves $\est(T) \lesssim m \cdot \log(T)$~\cite{vovk1997competitive,azoury2001relative}.
For general function class $\mathcal{F}$, we can nearly achieve the preceding bounds by performing the exponential weights update algorithm on a sequential cover of $\mathcal{F}$ (see e.g.,~\cite{vovk2006metric}).


With these results in mind, we can derive the exact regret bounds for various instances of function class $\mathcal{F}$.
In particular, if the suppliers' optimal production function $x^*$ is contained in one of the function classes listed below, we can compute the regret bounds as follows:
\begin{itemize}
    \item If $\mathcal{F}$ is finite, we have $\est(T) = \log |\mathcal{F}|$ and a choice of $K = (\frac{T}{\log |\mathcal{F}|})^{1/3}$ results in the regret bound $\textsc{Reg}(T) \lesssim T^{2/3} \left(\sqrt[3]{\log |\mathcal{F}|} + \sqrt{\log(1/\delta)}\right)$ with probability $1-\delta$.
    \item If the cost functions are quadratic functions with time-varying coefficients in the sense that
    \[c(x; \phi, \theta_t) = \frac{1}{2\langle \phi, \sigma(\theta_t)\rangle} x^2, \; \phi \in B^m_2,\]
    for some fixed feature map $\sigma$, then the suppliers' optimal production can be expressed as
    \[x^*(p; \phi, \theta_t) = \langle \phi, p \cdot \sigma(\theta_t) \rangle.\]
    So, in this case, $\mathcal{F}$ is a linear function class and we have $\est(T) \lesssim m \cdot \log(T)$.
    If $K = (\frac{T}{d\log T})^{1/3}$, then we get the regret bound $\textsc{Reg}(T) \lesssim T^{2/3} \left(\sqrt[3]{m \cdot \log T} + \sqrt{\log(1/\delta)}\right)$ with probability $1-\delta$.
    \item Consider an Euclidean context $\theta_t \in \RR^m$ encapsulates similarity information on the suppliers' behavior such that $\mathcal{F}$ is the set of bounded Lipschitz functions over $(p, \theta) \in \RR^{m+1}$.
    From this well-specified function class $\mathcal{F}$, we find a subset of $\mathcal{F}$ with sequential covering and apply the miss-specified version of Theorem~\ref{thm:igw-regret} on this subset.
    We can explicitly construct an $\varepsilon$-covering so that $\log \mathcal{N}_2(\mathcal{F}, \varepsilon) \lesssim \varepsilon^{-m-1} $ (see Examples 5.10 and 5.11 in~\cite{wainwright2019high}).
    If we run the exponential weights update algorithm over this covering, we have $\est(T) = \varepsilon^{-m-1}$.
    Since, by construction, the covering is $\varepsilon$-miss-specified with respect to the optimal suppliers' production, we have
    \[\textsc{Reg}(T) \lesssim \sqrt{KT \varepsilon^{-m-1}} + \varepsilon \sqrt{K} \cdot T + \frac{T}{K} + \sqrt{KT \log(1/\delta)}.\]
    If we pick $K = \varepsilon^{-2/3}$ and $\varepsilon = T^{-1/(m+2)}$, then we get that
    \[\textsc{Reg}(T) \lesssim T^{(3m+4)/(3m+6)} + T^{1/2 + 1/(3m+6)} \sqrt{\log(1/\delta)}\]
    with probability $1-\delta$.
    \item If $\mathcal{F}$ is a neural network whose weight matrices' spectral norms are at most 1, then it is known that $\log \mathcal{N}_2(\mathcal{F}, \varepsilon) \lesssim \varepsilon^{-2}$~\cite{bartlett2017spectrally}. 
    Then, we have $\est(T) \in O(T^{1/2})$, and a choice of $K = T^{1/6}$ gives the regret bound $\textsc{Reg} \lesssim T^{5/6} + T^{7/12} \sqrt{\log(1/\delta)}$ with probability $1-\delta$.
\end{itemize}

We can analogously work out the regret bounds if each of the examples is miss-specified.

%We now turn our attention to the proof of Theorem~\ref{thm:igw-regret}.

\subsection{Proof of Theorem~\ref{thm:igw-regret}} \label{sec:full-pf-contextual-thm}

%\begin{proof}[Proof of Theorem~\ref{thm:igw-regret}]
Recall that $x^*$ is the suppliers' optimal production, and since $\mathcal{F}$ is $\varepsilon$-miss-specified with respect to $x^*$, there exists a function $\bar{x} \in \mathcal{F}$ so that
\[\forall (p, \theta), \text{we have } |\bar{x}(p; \theta) -  x^*(p; \theta)| \le \varepsilon.\]
From Lemma~\ref{lem:ProductionLipschitz}, we know that $x^*$ is Lipschitz in price, so we know that for any context $\theta$ and price $p$, there exists $\bar{p}$ from the list of Algorithm~\ref{alg:time-varying-cost}'s choices $\{p_i\}_{i=1}^K$ such that $|x^*(\theta, p) - x^*(\theta, \bar{p})| < O(1/K)$.
Note that for the market-clearing price $p^*_t$, we have $|x^*(p^*_t; \theta_t) - d_t| = 0$.
Therefore, from the triangle inequality we have
\begin{align*}
    \textsc{Reg}(T) 
    &= \sum_{t=1}^T \EE_{p_t \sim \Delta_t} \left[ |x^*(p_t; \theta_t) - d_t|\mid \mathcal{H}_{t-1} \right] - |x^*(p^*_t; \theta_t) - d_t| \\
    &\le \sum_{t=1}^T \EE_{p_t \sim \Delta_t} \left[ |\bar{x}(p_t; \theta_t) - d_t| \mid \mathcal{H}_{t-1} \right] - |\bar{x}( \bar{p}_t; \theta_t) - d_t| + 2\varepsilon T + O(T/K)
\end{align*}
%
Now we attempt to upper bound the quantity
\[ \preg(T) := \sum_{t=1}^T \EE_{p_t \sim \Delta_t} \left[ |\bar{x}(p_t; \theta_t) - d_t| \mid \mathcal{H}_{t-1} \right] - |\bar{x}(\bar{p}_t; \theta_t) - d_t|. \]
%
Recall that $\hat{f}_t$ is the online regression oracle's output at time $t$.
For simplicity, denote
\begin{align*}
    \bar{g}_t(p) &= |\bar{x}(p; \theta_t) - d_t|, \\
    \hat{g}_t(p) &= |\hat{f}_t(p; \theta_t) - d_t|.
\end{align*}
%
Let $\hat{p}_t = \argmin \hat{g}_t(\cdot)$, then we have
\begin{align*}
    & \EE_{p_t \sim \Delta_t} [\bar{g}_t(p_t) \mid \mathcal{H}_{t-1} ] - \bar{g}_t(\bar{p}_t) \\
    ={}& \underbrace{\EE_{p_t \sim \Delta_t} [\hat{g}_t(p_t) - \hat{g}_t(\hat{p}_t) \mid \mathcal{H}_{t-1} ]}_{\textcircled{1}} + \underbrace{\EE_{p_t \sim \Delta_t} [\bar{g}_t(p_t) - \hat{g}_t(p_t) \mid \mathcal{H}_{t-1} ]}_{\textcircled{2}} + \underbrace{(\hat{g}_t(\hat{p}_t) - \bar{g}_t(\bar{p}_t))}_{\textcircled{3}}
\end{align*}
%
Plugging in the chosen distribution $\Delta_t$, we get that the first term is
\[\textcircled{1} = \sum_{i=1}^K \frac{\hat{g}_t(p_i) - \hat{g_t}(\hat{p_t})}{\lambda + 2\gamma (\hat{g}_t(p_i) - \hat{g_t}(\hat{p_t}))} \le \frac{K-1}{2\gamma}. \]
%
By convexity and then AM-GM, the second term can be bounded as 
\[\textcircled{2} \le \sqrt{\EE_{p_t \sim \Delta_t} [(\bar{g}_t(p_t) - \hat{g}_t(p_t))^2 \mid \mathcal{H}_{t-1}]} \le \frac{1}{2\gamma} + \frac{\gamma}{2} \EE_{p_t \sim \Delta_t} [(\bar{g}_t(p_t) - \hat{g}_t(p_t))^2 \mid \mathcal{H}_{t-1}].\]
%
And the third term can be rewritten as
\begin{align*}
    \textcircled{3}
    &= \hat{g}_t(\bar{p}_t) - \bar{g}(\bar{p}_t) - (\hat{g}_t(\bar{p}_t) - \hat{g}_t(\hat{p}_t)) \\
    &\le \frac{\gamma \Delta_t(\bar{p}_t)}{2}(\hat{g}_t(\bar{p}_t) - \bar{g}(\bar{p}_t))^2 + \frac{1}{2\gamma \Delta_t(\bar{p}_t)} - (\hat{g}_t(\bar{p}_t) - \hat{g}_t(\hat{p}_t)) \\
    &\le \frac{\gamma}{2} \EE_{p_t \sim \Delta_t}\left[ (\hat{g}_t(\bar{p}_t) - \bar{g}(\bar{p}_t))^2 \mid \mathcal{H}_{t-1} \right] + \frac{1}{2\gamma \Delta_t(\bar{p}_t)} - (\hat{g}_t(\bar{p}_t) - \hat{g}_t(\hat{p}_t)) \\
    &= \frac{\gamma}{2} \EE_{p_t \sim \Delta_t}\left[ (\hat{g}_t(\bar{p}_t) - \bar{g}(\bar{p}_t))^2 \mid \mathcal{H}_{t-1} \right] + \frac{\lambda + 2\gamma (\hat{g}_t(\bar{p}_t) - \hat{g}_t(\hat{p}_t))}{2\gamma } - (\hat{g}_t(\bar{p}_t) - \hat{g}_t(\hat{p}_t)) \\
    &= \frac{\gamma}{2} \EE_{p_t \sim \Delta_t}\left[ (\hat{g}_t(\bar{p}_t) - \bar{g}(\bar{p}_t))^2 \mid \mathcal{H}_{t-1} \right] + \frac{\lambda}{2\gamma} \\
    &\le \frac{\gamma}{2} \EE_{p_t \sim \Delta_t}\left[ (\hat{g}_t(\bar{p}_t) - \bar{g}(\bar{p}_t))^2 \mid \mathcal{H}_{t-1} \right] + \frac{K}{2\gamma}
\end{align*}
%
Now, summing these three terms yields that
\[ \EE_{p_t \sim \Delta_t} [\bar{g}_t(p_t) \mid \mathcal{H}_{t-1} ] - \bar{g}_t(\bar{p}_t) \le \frac{K}{\gamma} + \gamma \cdot \EE_{p_t \sim \Delta_t} [(\bar{g}_t(p_t) - \hat{g}_t(p_t))^2 \mid \mathcal{H}_{t-1}]. \]
%
After summing over $t =1, \dots, T$, we have
\[ \preg(T) = \sum_{t=1}^T \EE_{p_t \sim \Delta_t} [\bar{g}_t(p_t) \mid \mathcal{H}_{t-1} ] - \bar{g}_t(\bar{p}_t) \le \frac{KT}{\gamma} + \gamma \sum_{t=1}^T \EE_{p_t \sim \Delta_t} [(\bar{g}_t(p_t) - \hat{g}_t(p_t))^2 \mid \mathcal{H}_{t-1}]. \]
%
Next, we upper bound the RHS with some case work.
\begin{itemize}
    \item  $\bar{x}(p_t; \theta_t) \ge d_t \ge \hat{f}_t(p_t; \theta_t)$:
    \begin{align*}
        (|\bar{x}(p_t; \theta_t) - d_t| - |\hat{f}_t(p_t; \theta_t) - d_t|)^2 
        &\le (|\bar{x}(p_t; \theta_t) - d_t| + |\hat{f}_t(p_t; \theta_t) - d_t|)^2 \\
        &= (\bar{x}(p_t; \theta_t) - d_t + d_t - \hat{f}_t(p_t; \theta_t))^2 \\ 
        &= (\bar{x}(p_t; \theta_t) - \hat{f}_t(p_t; \theta_t))^2
    \end{align*}
    \item $\hat{f}_t(p_t; \theta_t) \ge d_t \ge \bar{x}(p_t; \theta_t)$, similar to the previous case, we have
    \[(|\bar{x}(p_t; \theta_t) - d_t| - |\hat{f}_t(p_t; \theta_t) - d_t|)^2 \le (\bar{x}(p_t; \theta_t) - \hat{f}_t(p_t; \theta_t))^2 \]
    \item $\hat{f}_t(p_t; \theta_t), \bar{x}(p_t; \theta_t) \ge d_t$ or $\bar{x}(p_t; \theta_t), \hat{f}_t(p_t; \theta_t) \le d_t$:
    \[(|\bar{x}(p_t; \theta_t) - d_t| - |\hat{f}_t(p_t; \theta_t) - d_t|)^2 = (\bar{x}(p_t; \theta_t) - d_t + d_t - \hat{f}_t(p_t; \theta_t))^2 = (\bar{x}(p_t; \theta_t) - \hat{f}_t(p_t; \theta_t))^2 \]
\end{itemize}
Therefore,
\begin{equation}\label{equ:reg-to-est}
    \preg(T) = \sum_{t=1}^T \EE_{p_t \sim \Delta_t} [\bar{g}_t(p_t) \mid \mathcal{H}_{t-1} ] - \bar{g}_t(\bar{p}_t) \le \frac{KT}{\gamma} + \gamma \sum_{t=1}^T \EE_{p_t \sim \Delta_t} [(\bar{x}(p_t; \theta_t) - \hat{f}_t(p_t; \theta_t))^2 \mid \mathcal{H}_{t-1}].
\end{equation}
%
We finish the proof by claiming that the final term can be bounded by the online regression oracle's prediction error guarantee with high probability.
Since $\bar{x}$ and $\hat{f}_t$ are bounded, we can apply Lemma~\ref{lem:freedman2} to get
\begin{equation}\label{equ:freedman1}
    \sum_{t=1}^T \EE_{p_t \sim \Delta_t} [(\bar{x}(p_t; \theta_t) - \hat{f}_t(p_t; \theta_t))^2 \mid \mathcal{H}_{t-1}] \le 2 \sum_{t=1}^T (\bar{x}(p_t) - \hat{f}_t(p_t))^2 + O(\log(1/\delta))
\end{equation}
%
Let $x_t$ be our observation on suppliers' production at time $t$.
We expand the RHS as follows:
\begin{align*}
    (\bar{x}(p_t; \theta_t) - \hat{f}_t(p_t; \theta_t))^2
    &= \bar{x}(p_t; \theta_t)^2 - 2\bar{x}(p_t; \theta_t)\hat{f}_t(p_t; \theta_t) + \hat{f}_t(p_t; \theta_t)^2 \\
    &= 2 \bar{x}(p_t; \theta_t)^2 - 2\bar{x}(p_t; \theta_t)\hat{f}_t(p_t; \theta_t) + 2 x_t (\hat{f}_t(p_t; \theta_t) - \bar{x}(p_t; \theta_t)) \\
    &\hspace{10em}+ \hat{f}_t(p_t; \theta_t)^2 - 2 x_t (\hat{f}_t(p_t; \theta_t) - \bar{x}(p_t; \theta_t)) - \bar{x}(p_t; \theta_t)^2 \\
    &= 2(x_t - \bar{x}(p_t; \theta_t))(\hat{f}_t(p_t; \theta_t)  - \bar{x}(p_t; \theta_t)) + (\hat{f}_t(p_t; \theta_t) - x_t)^2 - (\bar{x}(p_t; \theta_t) - x_t)^2 
\end{align*}
%
Therefore,
\begin{equation}\label{equ:diff-expand}
\begin{aligned}
    & \sum_{t=1}^T (\bar{x}(p_t; \theta_t) - \hat{f}_t(p_t; \theta_t))^2 \\
    ={}& \sum_{t=1}^T (\hat{f}_t(p_t; \theta_t) - x_t)^2 - \sum_{t=1}^T (\bar{x}(p_t; \theta_t) - x_t)^2 + \sum_{t=1}^T 2(x_t - \bar{x}(p_t; \theta_t))(\hat{f}_t(p_t; \theta_t)  - \bar{x}(p_t; \theta_t)) \\
    \le{}& \sum_{t=1}^T (\hat{f}_t(p_t; \theta_t) - x_t)^2 - \inf_{f \in \mathcal{F}}\sum_{t=1}^T (f(p_t; \theta_t) - x_t)^2 + \sum_{t=1}^T 2(x_t - \bar{x}(p_t; \theta_t))(\hat{f}_t(p_t; \theta_t)  - \bar{x}(p_t; \theta_t)) \\
    ={}& \est(T) + \sum_{t=1}^T 2(x_t - \bar{x}(p_t; \theta_t))(\hat{f}_t(p_t; \theta_t)  - \bar{x}(p_t; \theta_t))
\end{aligned}
\end{equation}
%
The last term can be expanded as
\begin{align*}
    &\sum_{t=1}^T (x_t - \bar{x}(p_t; \theta_t))(\hat{f}_t(p_t; \theta_t)  - \bar{x}(p_t; \theta_t)) \\
    ={}& \sum_{t=1}^T (x_t - x^*(p_t; \theta_t) + x^*(p_t; \theta_t) - \bar{x}(p_t; \theta_t))(\hat{f}_t(p_t; \theta_t)  - \bar{x}(p_t; \theta_t)) \\
    \le{}& \sum_{t=1}^T (x_t - x^*(p_t; \theta_t))(\hat{f}_t(p_t; \theta_t)  - \bar{x}(p_t; \theta_t)) + \sum_{t=1}^T \varepsilon\cdot |\hat{f}_t(p_t; \theta_t)  - \bar{x}(p_t; \theta_t)| \\
    \le{}& \sum_{t=1}^T (x_t - x^*(p_t; \theta_t))(\hat{f}_t(p_t; \theta_t)  - \bar{x}(p_t; \theta_t)) + \sum_{t=1}^T 6 \varepsilon^2 + \frac{1}{24}(\hat{f}_t(p_t; \theta_t)  - \bar{x}(p_t; \theta_t))^2
\end{align*}
Since $\EE[x_t - x^*(p_t; \theta_t) \mid \mathcal{H}_{t-1}] = 0$ and both $x_t - x^*(p_t; \theta_t)$ and $\hat{f}_t(p_t; \theta_t)  - \bar{x}(p_t; \theta_t)$ are bounded above by $2\bar{d}$, we can apply Lemma~\ref{lem:freedman1} to the first sum with $\eta = 1/(64\bar{d}^2)$.
And for the second sum, we can apply Lemma~\ref{lem:freedman2}.
Then, with probability at least $1-\delta$, we have
\begin{equation}\label{equ:freedman2}
\begin{aligned}
    &\sum_{t=1}^T (x_t - \bar{x}(p_t; \theta_t))(\hat{f}_t(p_t; \theta_t)  - \bar{x}(p_t; \theta_t)) \\
    \le{}& \sum_{t=1}^T \left(\frac{1}{64 \bar{d}^2} \EE[(x_t - f^*(p_t; \theta_t))^2(\hat{f}_t(p_t; \theta_t)  - \bar{x}(p_t; \theta_t))^2 \mid \mathcal{H}_{t-1}] + \frac{1}{16}\EE[(\hat{f}_t(p_t; \theta_t)  - \bar{x}(p_t; \theta_t))^2 \mid \mathcal{H}_{t-1}]\right) \\
    &\hspace{26em}+ O(\varepsilon^2 T + \log(1/\delta)) \\
    \le{}& \frac{1}{8} \sum_{t=1}^T  \EE[(\hat{f}_t(p_t; \theta_t)  - \bar{x}(p_t; \theta_t))^2 \mid \mathcal{H}_{t-1}] + O(\varepsilon^2 T + \log(1/\delta))
\end{aligned}
\end{equation}
%
Putting \eqref{equ:freedman1}, \eqref{equ:diff-expand} and \eqref{equ:freedman2} together gives us
\begin{align*}
    & \sum_{t=1}^T \EE_{p_t \sim \Delta_t} [(\bar{x}(p_t; \theta_t) - \hat{f}_t(p_t; \theta_t))^2 \mid \mathcal{H}_{t-1}] \\
    \le{}& 2 \est(T) + \frac{1}{2} \sum_{t=1}^T \EE_{p_t \sim \Delta_t} [(\bar{x}(p_t; \theta_t) - \hat{f}_t(p_t; \theta_t))^2 \mid \mathcal{H}_{t-1}] + O(\varepsilon^2 T + \log(1/\delta))
\end{align*}
%
Therefore, 
\begin{equation}\label{equ:est-to-oracle}
    \sum_{t=1}^T \EE_{p_t \sim \Delta_t} [(\bar{x}(p_t; \theta_t) - \hat{f}_t(p_t; \theta_t))^2 \mid \mathcal{H}_{t-1}] \le 4 \est(T) + O(\varepsilon^2 T + \log(1/\delta))
\end{equation}
%
Combining \eqref{equ:reg-to-est} and \eqref{equ:est-to-oracle}, and taking
\[\gamma = \sqrt{\frac{KT}{\est(T) + \varepsilon^2 T + \log(1/\delta)}},\]
we conclude the following regret bound with probability $1-\delta$:
\begin{align*}
    \preg(T)
    &\lesssim \frac{KT}{\gamma} + \gamma \cdot \left(\est(T) + \varepsilon^2 T + \log(1/\delta)\right) \\
    &\lesssim \sqrt{KT(\est(T) + \varepsilon^2 T + \log(1/\delta))} \\
    &\lesssim \sqrt{KT \cdot\est(T)} + \varepsilon\sqrt{K} \cdot T + \sqrt{KT \log(1/\delta)}
\end{align*}
%
Finally, we have
\begin{align*}
    \textsc{Reg}(T) 
    &\lesssim \sqrt{KT \cdot\est(T)} + \varepsilon\sqrt{K} \cdot T + \sqrt{KT \log(1/\delta)} + \varepsilon T + \frac{T}{K} \\
    &\lesssim \sqrt{KT \cdot\est(T)} + \frac{T}{K} + \varepsilon\sqrt{K} \cdot T + \sqrt{KT \log(1/\delta)},
\end{align*}
with probability $1-\delta$.
so we are done.



%\end{proof}

\end{document}
