\section{Proof of Theorem~\ref{thm:VaryingDemandResult}}
\label{sec:vary-demand-pf}

%\begin{proof}[Proof of Theorem~\ref{thm:VaryingDemandResult}]
We first show that payment and cost regrets are at most $O(\sqrt{T} \log\log T)$.
Note that at any period $t$, payment or cost regrets are accumulated only if $x^*(p_{k_t}) > d_t \ge a_{k_t}$, where $x^*(p) = \sum_{i = 1}^n x_i^*(p)$.
Due to the assumption on the boundedness of prices, the amount of payment or cost regret incurred at any period is at most a constant.
Thus, it suffices to bound the number of times the feasible price set is shrunk.
For any interval $I_k, k \in [K]$, the number of times we repeatedly square the precision parameter $\epsilon_k$ to shrink its associated feasible price set to below $1/\sqrt{T}$ is $O(K \log\log T) = O(\sqrt{T} \log\log T)$.
It thus follows that the total payment and cost regret of Algorithm~\ref{alg:time-varying-demand-new} is at most $O(K \log\log T) = O(\sqrt{T} \log\log T)$.

Next, we show that the unmet demand is also at most $O(\sqrt{T} \log\log T)$.
For each period $t$, there are two possibilities:
\begin{itemize}
    \item $|\S_{k_t}| \le 1/\sqrt{T}$: From the construction of our algorithm, as all demands in each sub-interval are considered equal to the lower bound of that interval, $p^*(a_{k_t}) \in \S_{k_t}$ is always satisfied. 
    By Lemma~\ref{lem:ProductionLipschitz} and monotonicity of the cumulative production of all suppliers in the prices, we have
    \[a_k \le x^*(\max \S_{k_t}) \le x^* \left(p_{k_t} + \frac{1}{\sqrt{T}} \right) \le x^*(p_{k_t}) + O(1/\sqrt{T})\]
    Since $d_t \le a_{k_t} + \gamma$, we have that $d_t \le x^*(p_{k_t}) + O(1/\sqrt{T})$.
    Thus, over $T$ periods, the total amount of unmet regret incurred in this case is then $O(T \cdot 1/\sqrt{T}) = O(\sqrt{T})$.
    \item $|\S_{k_t}| > 1/\sqrt{T}$: Since $|\S_{k_t}|^2 = \varepsilon_{k_t}$, there are at most $1/\sqrt{\varepsilon_{k_t}}$ periods where the demand sub-interval $I_{k_t}$ is associated with the feasible set $\S_{k_t}$.
    By Lemma~\ref{lem:ProductionLipschitz} and the monotonicity of the cumulative production of all suppliers in the prices, we have
    \[a_k \le x^*(\max \S_{k_t}) \le x^*(p_{k_t} + \sqrt{\varepsilon_{k_t}}) \le x^*(p_{k_t}) + O(\sqrt{\varepsilon_{k_t}})\]
    Since $d_t \le a_{k_t} + \gamma < a_{k_t} + |\S_{K_t}|$, we have that $d_t \le x^*(p_{k_t}) + O(\sqrt{\varepsilon_{k_t}})$.
    Therefore, in total, the unmet demand incurred for interval $I_{k_t}$ that are associated with $\S_{k_t}$ is up to $O(\frac{1}{\sqrt{\varepsilon_{k_t}}} \cdot \sqrt{\varepsilon_{k_t}}) = O(1)$.
    Furthermore, for any interval $I_k, k \in [K]$, it takes up to $O(\log\log \sqrt{T}) = O(\log\log T)$ times to shrink its associated feasible price set to below $1/\sqrt{T}$.
    It thus follows that the unmet demand incurred in this case is at most $O(K \log\log T \cdot 1) = O(\sqrt{T} \log\log T)$.
\end{itemize}
Therefore the unmet demand is also at most $O(\sqrt{T} \log\log T)$, which establishes our claim.
%\end{proof}