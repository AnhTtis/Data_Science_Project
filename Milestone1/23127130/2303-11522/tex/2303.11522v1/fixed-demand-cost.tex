\section{Fixed Cost Functions and Demand} \label{sec:fixed-setting}

We now investigate the design of online pricing policies achieving good performance on the three regret metrics, i.e., sub-linear unmet demand, cost regret, and payment regret in the number of periods $T$. As a warm-up, we first consider the setting when the cost functions of the suppliers and customer demand are fixed over the $T$ periods. 
Formally, the supplier cost functions satisfy $c_{it}(\cdot) = c_{it'}(\cdot)$ for all $t, t' \in [T]$ and the demands satisfy $d_t = d_{t'}$ for all periods $t, t' \in [T]$. For ease of exposition, in this section, we drop the subscript (superscript) $t$ in the notation for the customer demand (and supplier cost functions) and denote $d_t = d$ (and $c_{it}(\cdot) = c_i(\cdot)$ for all suppliers $i$) for all periods $t \in [T]$.
In this setting, we develop an algorithm that achieves a regret of $O(\log \log T)$ on the three regret metrics when the suppliers' cost functions are strongly convex (Section~\ref{sec:fixed-algorithm}). We further present an example to demonstrate that if the strong convexity condition on suppliers' cost functions is relaxed, then no sub-linear regret guarantee on all three regret metrics is, in general, possible for any online algorithm (Section~\ref{sec:fixed-convex-limitations}).



\subsection{Algorithm with Sub-linear Regret for Strongly Convex Cost Functions} \label{sec:fixed-algorithm}

In this section, we consider the setting of fixed supplier cost functions and customer demands and present an algorithm that achieves a regret of $O(\log \log T)$ on the three regret metrics when the suppliers' cost functions are strongly convex. 
To motivate our algorithm, we first note that since the customer demand and the supplier cost functions are fixed over time, the optimal price $p^* \in [0, 1]$ for all periods $t \in [T]$ is also fixed and given by the dual of the demand constraint of Problem~\eqref{eq:supObj2}-\eqref{eq:demand-con}. %For ease of exposition of our algorithm (Algorithm~\ref{alg:fixed-demand}) and its corresponding analysis, we normalize the set of feasible prices such that the optimal price $p^*$ of the commodity belongs to the normalized interval $[0, 1]$, i.e., $p^* \in [0, 1]$. \haoyuan{This should go into the models section. Also for boundedness of the demand etc.} \devansh{Done!}
%Crucially, we note that we are only concerned with locating the equilibrium price which clears the market and it is not necessary to learn other features of the cost function $c_i(\cdot)$.
Furthermore, the cumulative production $x_t^*(p) = \sum_{i = 1}^n x_{it}^*(p)$ is monotonically non-decreasing in the price $p$ because suppliers' cost functions are increasing. Utilizing this monotonicity property, we note that if we set two prices $p_1, p_2 \in [0, 1]$ such that the cumulative production $\sum_{i = 1}^n x_{it}^*(p_1)>d$ and $\sum_{i = 1}^n x_{it}^*(p_2)<d$, then $p_1$ and $p_2$ respectively serve as upper and lower bounds on the optimal price $p^*$ when the supplier cost functions and customer demands are fixed over time.

Following these observations, we present Algorithm~\ref{alg:fixed-demand}, akin to the algorithm in~\cite{oppa}, 
which maintains a feasible interval for the optimal price $p^*$ and sets a sequence of prices for each arriving user to continuously shrink this feasible price set. In particular, the feasible price interval $[a, b]$ is initialized to $\S_p = [0, 1]$ and a precision parameter $\varepsilon$ is set to $0.5$. 
%Then, in any given algorithm sub-phase, 
Then, for a given algorithm sub-phase associated with feasible price interval $[a, b]$, the operator posts prices $a, a+\varepsilon, a+2\varepsilon, \ldots$ (up to $b$) at each period until the total supply exceeds the demand at the offered price. If $a+k\varepsilon$ for some $k \in \mathbb{N}$ was the last price such that $x_{t}^*(a+k\varepsilon) \leq d$, then $[a+k\varepsilon, a+(k+1)\varepsilon]$ is set as the new feasible interval for the optimal price, and the precision parameter is re-set to $\varepsilon^2$. This process of shrinking the feasible interval and updating the precision parameter is repeated until the length of the feasible interval is smaller than $\frac{1}{T}$, following which the market operator posts the price at the lower end of the feasible interval for the remaining periods. This process is presented formally in Algorithm~\ref{alg:fixed-demand}.


\begin{algorithm} 
\SetAlgoLined
\SetKwInOut{Input}{Input}\SetKwInOut{Output}{Output}
\Input{Feasible set of prices $\S_p = [0, 1]$, Precision Parameter $\varepsilon = \frac{1}{2}$}
Set the lower and upper bounds of the feasible price set: $a \leftarrow 0, b \leftarrow 1$ \;
 \While{\text{length of feasible price set is greater than $\frac{1}{T}$}}{
 Offer prices $a, a+\varepsilon, \ldots, a+(k+1)\varepsilon$ (all of which are $\le b$) to each subsequent user where $a+k\varepsilon$ is the last price such that $\sum_{i = 1}^n x_{it}^*(a+k\varepsilon) < d$ \;
 Set the new feasible interval to $[a+k\varepsilon, a+(k+1)\varepsilon]$ and reduce the precision parameter to $\varepsilon^2$ \;
 }
 \For{\text{the remaining time periods}}{
  Offer price $p_t = a$ \;
}
\caption{Feasible Price Set Tracking under Fixed Demand and Costs} 
\label{alg:fixed-demand}
\end{algorithm} 

While Algorithm~\ref{alg:fixed-demand} is similar to the corresponding algorithm in~\cite{oppa} for the setting of fixed user valuations, our market setting is considerably different than the revenue maximization setting in~\cite{oppa}. First, in this work, suppliers have a continuous rather than a binary action space as in the revenue maximization setting in~\cite{oppa}, where consumers either purchase one unit of the resource at the given price or do not purchase it. Furthermore, as opposed to the single regret measure analyzed in~\cite{oppa}, we consider and analyse three different regret measures that often compete against each other. 

We now present the main result of this section, which establishes that Algorithm~\ref{alg:fixed-demand} simultaneously achieves an $O(\log \log T)$ regret on the three regret measures studied in this work.

\begin{theorem} \label{thm:IdenticalResult}
The unmet demand, cost regret, and payment regret of Algorithm~\ref{alg:fixed-demand} are $O(\log \log T)$ if the cost functions of the suppliers are strongly convex.
\end{theorem}

The proof of Theorem~\ref{thm:IdenticalResult} relies on the following Lipshitzness condition between the optimal supplier production and the prices set by the market operator.

\begin{lemma} [Lipschitzness of Production in Prices] \label{lem:ProductionLipschitz}
Suppose that the suppliers' cost functions $c_i(\cdot)$ are $\mu_i$-strongly convex. Then, at any period $t$, the optimal production quantity for supplier $i$ corresponding to the solution of Problem~\eqref{eq:supObj} is Lipschitz in the price $p$, i.e., $|x_{it}^*(p_1) - x_{it}^*(p_2)| \leq L |p_1 - p_2|$ for some constant $L>0$ for all $p_1, p_2 \in [0, 1]$.
\end{lemma}
\begin{proof}
Fix a period $t \in [T]$. Then, by computing the first-order optimally condition of Problem~\eqref{eq:supObj} for each supplier $i$, we have that 
\[ p = c_i'(x_{it}^*(p)) \implies x_{it}^*(p) = (c_i')^{-1}(p). \]
Then, by the inverse function theorem, we have that
    \[(x_{it}^*)'(p) = \frac{d}{dp}(c_i')^{-1}(p) = \frac{1}{c_i''(x_{it}^*(p))} \le \frac{1}{\mu_i},\]
where the inequality follows since the function $c_i(\cdot)$ is $\mu_i$-strongly convex.
Hence, at each period $t$, the optimal supplier production $x^*_{it}$ is $(1/\mu_i)$-Lipschitz in the prices.
\end{proof}

Lemma~\ref{lem:ProductionLipschitz} establishes that small changes in the price set by the market operator correspond to small changes in the optimal production of suppliers. Using Lemma~\ref{lem:ProductionLipschitz}, we now present a proof sketch of Theorem~\ref{thm:IdenticalResult} and present its complete proof in Appendix~\ref{sec:pf-identical-result}.

\begin{hproof}
To establish this result, we first note that we need $O(\log \log T)$ sub-phases of repeated squaring of the parameter $\varepsilon$ to reduce $\varepsilon$ from $0.5$ to $\frac{1}{T}$. 
Due to the monotonicity of the optimal supplier production in the prices that both payment and cost regret are only accumulated when $p_t>p^*$ (and that $p_t>p^*$ at most once in each sub-phase), the total payment and cost regret are $O(\log \log T)$.
Next, to bound the unmet demand, we use Lemma~\ref{lem:ProductionLipschitz} to map prices to productions and show that there is a constant unmet demand accumulated in each sub-phase in Algorithm~\ref{alg:fixed-demand}, resulting in an $O(\log \log T)$ unmet demand when the length of the feasible price interval is more than $\frac{1}{T}$, as there are $O(\log \log T)$ sub-phases. In the final phase, when the length of the feasible price interval is less than $\frac{1}{T}$ and the price is fixed, we again use Lemma~\ref{lem:ProductionLipschitz} to show that the unmet demand through this phase is constant.
Thus, the unmet demand is $O(\log \log T)$, establishing our claim.
\end{hproof}


We reiterate that the proof of Theorem~\ref{thm:IdenticalResult} crucially relies on the strong convexity of the cost functions of the suppliers, which was necessary to establish the Lipshitzness relation between the optimal production of suppliers and the prices set by the market operator (Lemma~\ref{lem:ProductionLipschitz}). As a result, in addition to leveraging tools from the analysis of the corresponding algorithm in~\cite{oppa}, our regret analysis additionally uses tools from parametric optimization to develop the necessary sensitivity relations required to analyse the three regret metrics considered in this work.
We also note that the $O(\log \log T)$ regret guarantee obtained in Theorem~\ref{thm:IdenticalResult} indicates that in the setting with fixed supplier cost functions and customer demand, Algorithm~\ref{alg:fixed-demand} incurs little performance loss on all three regret metrics as compared to when equilibrium prices are set with complete knowledge of the cost functions of the suppliers, where Algorithm~\ref{alg:fixed-demand}'s prices converge to the equilibrium price $p^*$ super-exponentially.
Furthermore, the obtained upper bound on the regret of Algorithm~\ref{alg:fixed-demand} compares favorably to the $\Omega(\log \log T)$ regret lower bound for any online algorithm in the revenue maximization setting with fixed user valuations studied in~\cite{oppa}.



\subsection{Performance Limitations for Non-Strongly Convex Cost Functions} \label{sec:fixed-convex-limitations}

While Algorithm~\ref{alg:fixed-demand} achieved sub-linear regret on all three performance measures in the incomplete information setting for strongly convex cost functions, we note that this result does not generalize to the setting of general convex costs. In particular, in this section, we show that if the cost functions of the suppliers are linear, then no online algorithm can achieve sub-linear regret guarantees for the unmet demand, cost regret, and payment regret metrics simultaneously. This result highlights the difficulty of the incomplete information setting compared to that with complete information, where the equilibrium price satisfying the three desirable properties in Section~\ref{sec:market-model} exists and can be computed through the dual variable of the market clearing constraint of Problem~\eqref{eq:supObj2}-\eqref{eq:demand-con} when the cost functions of all suppliers are convex (and, not necessarily, strongly convex).

 
To motivate why sub-linear regret cannot be attained simultaneously on the three desirable metrics for linear cost functions, we note that the optimal production of suppliers is zero if $p<p^*$ and their optimal production is the maximum feasible if $p>p^*$. Thus, there is a jump discontinuity in the production of suppliers at the price $p = p^*$ and so the production is not Lipschitz in the prices, which breaks a key property (Lemma~\ref{lem:ProductionLipschitz}) required to prove Theorem~\ref{thm:IdenticalResult}. We now formally present an example demonstrating that, even in a market with a single supplier, no online algorithm can achieve sub-linear regret on all three regret measures if the supplier's cost function is linear.

\begin{example} [Sub-linear Regret is not Possible for Linear Cost Functions] \label{eg:linear-cost}
We consider a market with one supplier with a linear cost function that is fixed over time. In this setting, suppose that the optimal price is $p^*$ and the cost function of the supplier is given by $c(x) = c x$. Then, given a price $p$ set by the market operator, the individual decision making problem for the supplier is to produce a quantity $x^*(p) \geq 0$ that maximizes $(p-c) \cdot x$. Note that if $p<c$, then $x^*(p) = 0$ and if $p>c$, then the supplier generates the maximum amount that a supplier can feasibly generate. Thus, it is only when $p = p^* = c$ that the supplier will generate the amount equal to the customer demand $d$. Since the perfect identification of the optimal price $p^*$ is, in general, not possible under incomplete information on the cost coefficient $c$, we have that the market operator will either set prices $p_t<p^*$ or $p_t>p^*$. 

Next, we observe that there is an unmet demand of $d$ at any period when $p_t<p^*$ (as the supplier generates nothing if the price is below $p^*$ by linearity of the cost function). Thus, to achieve sub-linear unmet demand, the market operator must set the price $p > p^*$ for $O(T)$ periods; however, doing so results in a linear cost and payment regret as the supplier produces strictly more than $d$ units of the commodity when $p>p^*$. Thus, no online algorithm can achieve a sub-linear regret on all three regret measures if the cost functions of the suppliers is linear.
\qed
\end{example}
