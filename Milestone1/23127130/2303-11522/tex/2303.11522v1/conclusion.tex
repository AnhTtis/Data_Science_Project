\section{Conclusion and Future Work}

In this work, we studied the problem of setting equilibrium prices to satisfy the customer demand for a commodity in markets where the cost functions of suppliers are unknown to the market operator. Since centralized optimization approaches to compute equilibrium prices are typically not conducive in this incomplete information setting, we studied the problem of learning equilibrium prices online under several informational settings regarding the time-varying nature of the customer demands and supplier cost functions. We first considered the setting when suppliers' cost functions are fixed over the $T$ periods and developed algorithms with regret guarantees of $O(\log \log T)$ (and $O(\sqrt{T} \log \log T)$) when the customer demand is fixed (or can vary across the periods in a continuous interval) for strongly convex cost functions. Next, when suppliers' cost functions are time-varying, we showed that no online algorithm achieves sub-linear regret on all three regret metrics when suppliers' cost functions are sampled i.i.d. from a distribution. Thus, we studied an augmented contextual bandit setting where the operator has access to hints (contexts) on how the cost functions change over time and developed an algorithm that with sub-linear regret on all three regret metrics in this setting.

There are several directions for future research. First, we note that the discretization in Algorithms~\ref{alg:time-varying-demand-new} and~\ref{alg:time-varying-cost} results in memory requirements that depend on the number of periods $T$. Thus, it would be interesting to investigate the design of algorithms whose memory requirement does not scale with $T$ but still achieves similar performance guarantees to Algorithms~\ref{alg:time-varying-demand-new} and \ref{alg:time-varying-cost} in their respective settings. Next, the unmet demand metric adopted in this work involved satisfying the customer demand at each period without the possibility of the rollover of excess supply of the commodity to subsequent periods. Since there is often storage capacity for excess supply in many application settings, it would be worthwhile to study the design of algorithms under a more relaxed unmet demand notion with the possibility of using excess supply at earlier periods to satisfy customer demand at subsequent periods. Furthermore, there is a scope to generalize the model when suppliers' cost functions are non-convex, in which case the operator may need different pricing strategies for each supplier~\cite{azizan2020optimal}.

%There are several directions for future research. First, we note that discretization in Algorithms~\ref{alg:time-varying-demand-new} and \ref{alg:time-varying-cost} result in memory requirements that depend on the number of periods $T$. So, it would be interesting to investigate the design of algorithms whose memory requirement does not scale with $T$ and yet achieves similar performance guarantees to Algorithms~\ref{alg:time-varying-demand-new} and \ref{alg:time-varying-cost} in their respective settings. Next, the unmet demand metric adopted in this work involved satisfying customer demand at each period and does not allow for the possibility of the roll-over of excess supply of the commodity to subsequent periods. Since there is often storage capacity for excess supply, particularly in energy markets, it would be worthwhile to study the design of algorithms under a more relaxed unmet demand notion with the possibility of excess supply being rolled over to satisfy customer demand at subsequent time periods. Furthermore, we can potentially study more general model where the suppliers' cost functions are non-convex.
%Since a single price for all suppliers typically does not lead to equilibrium outcomes in markets with non-convex costs, we may want the operator to have different pricing strategy for each supplier.

%algorithm design and limitations in setting with non-convex costs

%when relaxing unmet demand metric to allow for roll-over stock in subsequent periods

%reduce dependence on gridding to reduce memory scaling, which depends on T