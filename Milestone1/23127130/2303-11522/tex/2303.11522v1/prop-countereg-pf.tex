\section{Proof of Proposition~\ref{prop:time-varying-cost-countereg}} \label{sec:pf-prop-countereg}

%\begin{proof}
We consider a setting with a fixed demand of $d=1$ at every period and a single supplier whose cost functions at each period are drawn from a distribution such that at each period $t$, its cost function could be either $c_1(x) = \frac{1}{8}x^2$ or $c_2(x) =\frac{1}{16} x^2$, each with probability $0.5$. We suppose that the market operator has knowledge of the distribution from which the supplier's cost function is sampled i.i.d. but does not know the outcome of the random draw at any period and show that any pricing strategy adopted by the operator must incur a linear regret one at least one of the three regret measures for this instance. 

To prove this claim, we first define the \emph{total regret} as the sum of the unmet demand, payment regret, and cost regret and note that if the total regret is linear in the number of periods $T$, then at least one of the three regret measures must be linear in $T$. To analyse the total regret, we first analyse the each of the regret measures for a given price $p$ for both the cost functions.

\begin{enumerate}
    \item For the first cost function $c_1(x)$, the optimal production level given a price $p$ is $x^*(p) = 4p$, so the equilibrium price is $p^* = \frac{1}{4}$.
    \begin{itemize}
        \item The payment regret at price $p$ is $p(4p) - 1/4 = 4p^2 - 1/4$.
        \item The cost regret at price $p$ is $\frac{1}{8}(4p)^2 - 1/8 = 2p^2 - 1/8$.
        \item The unmet demand is $1-4p$ if $p < 1/4$ and 0 otherwise.
    \end{itemize}
    \item For the second cost function $c_2(x)$, the optimal production level given a price $p$ is $x^*(p) = 8p$, so the equilibrium price is $p^* = \frac{1}{8}$.
    \begin{itemize}
        \item The payment regret at price $p$ is $p(8p) - 1/8 = 8p^2 - 1/8$.
        \item The cost regret at price $p$ is $\frac{1}{16}(8p)^2 - 1/16 = 4p^2 - 1/16$.
        \item The unmet demand is $1-8p$ if $p < 1/8$ and 0 otherwise.
    \end{itemize}
\end{enumerate}
Then, the expected total regret at each period $t$ is as follows:
\begin{itemize}
    \item If $p < 1/8$: expected total regret is $\frac{1}{2}(18p^2 - 9/16 + (1-4p) + (1-8p)) = 9p^2 - 6p + 23/32$.
    \item If $1/8 \le p \le 1/4$: expected total regret is $\frac{1}{2}(18p^2 - 9/16 + (1-4p)) = 9p^2 - 2p + 7/32$.
    \item If $p > 1/4$: expected total regret is $\frac{1}{2}(18p^2 - 9/16) = 9p^2 - 9/32$.
\end{itemize}
From the above obtained relations, we can derive that the expected total regret at any period $t$ is at least $7/64$, which is attained when $p = 1/8$.
It thus follows that, regardless of the pricing strategy adopted by the market operator, that the total expected regret is at least $\frac{7}{64} T$.
Hence, either the unmet demand, payment regret, or cost regret are not sublinear, which establishes our claim.
%\end{proof}