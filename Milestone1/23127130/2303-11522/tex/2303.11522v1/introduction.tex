\section{Introduction}

The study of market mechanisms for efficiently allocating scarce resources traces back to the seminal work of Walras~\cite{walras1954elements}. In his work, Walras investigated the design of pricing schemes to mediate the allocation of scarce resources such that the economy operates at an \emph{equilibrium}, i.e., the supply of each good matches its demand. Market equilibria exist under mild conditions on agents' preferences~\cite{arrow-debreu}, and, under convexity assumptions on their preferences, can often be computed by solving a large-scale centralized optimization problem. As a case in point, in electricity markets with convex supplier cost functions, the equilibrium prices correspond to the shadow prices of a convex optimization problem that minimizes the sum of the supplier costs subject to a market clearing (or load balance) constraint~\cite{azizan2020optimal}.


While methods such as convex programming provide computationally tractable approaches to computing market equilibria, the efficacy of such centralized optimization approaches for equilibrium computation suffers from several inherent limitations. First, centralized optimization approaches rely on complete information on agents' utilities and cost functions that are typically unavailable to a market operator. For instance, with the deregulation of electricity markets, suppliers' cost functions are private information, which has led to strategic bidding practices by suppliers seeking to maximize their profits~\cite{liberopoulos2016critical,VENTOSA2005897,david2000strategic} and has been associated with tens of millions of dollars of over-payments to suppliers~\cite{strategic-bidding-jp-morgan}. Moreover, even if a market operator has access to some information on agents' utilities and cost functions, such information can typically only provide a noisy or imperfect estimate of their preferences due to inadequate information or uncertainty~\cite{weitzman-pvq}. In the context of electricity markets, the advent of renewables and distributed energy resources has accompanied a high degree of uncertainty in the supply of energy to meet customer demands at different times of the day and year, as these energy sources are sensitive to weather conditions. To further compound these challenges, agents' preferences in markets such as electricity markets may also be time-varying, e.g., in electricity markets, customer demands may change over time and the cost functions of suppliers may depend on fluctuating weather conditions. Thus, a market operator may need to periodically collect agents' preferences and solve a large-scale centralized optimization at each time period to set equilibrium prices, which may be computationally challenging.

Motivated by these practical considerations that limit the applicability of centralized optimization approaches to computing equilibrium prices, in this work, we study the problem of setting equilibrium prices in the incomplete information setting where a market operator seeks to satisfy customer demand for a commodity by purchasing the required amount from competing suppliers with privately known cost functions. We investigate this problem under several informational settings regarding the time-varying nature of the customer demands and supplier cost functions and develop online learning algorithms for each of these settings that iteratively adjust the prices in the market over time. Our proposed algorithms employ the observation that a market operator can effectively learn information on suppliers' costs and equilibrium prices through observations of their cumulative production relative to the customer demand given different market prices. To analyze the performance of our algorithms, we combine techniques from online learning and parametric optimization as we seek to simultaneously optimize multiple, often competing, performance metrics pertinent in the context of equilibrium pricing. 
%In particular, in addition to using tools from online learning, our analysis leverages techniques from parametric optimization to establish necessary sensitivity relations between problem parameters required to obtain guarantees across multiple performance measures.


\subsection{Contributions}

In this work, we study the problem of setting equilibrium prices faced by a market operator that seeks to satisfy an inelastic customer demand for a commodity by purchasing the required amount from $n$ competing suppliers. Crucially, we study this problem in the incomplete information setting when the cost functions of suppliers are private information and thus unknown to the market operator. Since traditional centralized methods of setting equilibrium prices are typically not conducive in this incomplete information setting, we consider the problem of learning equilibrium prices over $T$ periods to minimize three performance (regret) metrics: (i) \emph{unmet demand}, (ii) \emph{cost regret}, and (iii) \emph{payment regret}. Here unmet demand refers to the cumulative difference between the demand and the total production of the commodity corresponding to an online pricing policy. Furthermore, cost regret (payment regret) refers to the difference between the total cost of all suppliers (payment made to all suppliers) corresponding to the online allocation and that of the offline oracle with complete information on suppliers' cost functions. For a more thorough discussion of these regret metrics, we refer to Section~\ref{sec:perf-measures}.

In this incomplete information setting, we investigate the design of online algorithms to set a sequence of prices that achieve sub-linear regret, in the number of periods $T$, on the above three performance metrics. To this end, we first consider the setting when suppliers' cost functions are fixed over the $T$ periods and develop algorithms that achieve a regret of $O(\log \log T)$ when the customer demand is constant over time (Section~\ref{sec:fixed-setting}), and $O(\sqrt{T} \log \log T)$ when the demand is variable over time (Section~\ref{sec:vary-demand}), for strongly convex cost functions. To establish these regret guarantees for the three performance metrics, we leverage and combine techniques from parametric optimization and online learning. We further demonstrate through an example that if the strong convexity condition
on suppliers' cost functions is relaxed, no online algorithm can achieve a sub-linear regret guarantee on all three regret metrics.


Then, we consider the setting when suppliers' cost functions can vary across the $T$ periods (Section~\ref{sec:time-vary-cost-main}) and show that if the operator does not know the process that governs the variation of the cost functions, no online algorithm can achieve sub-linear regret on all three regret metrics. Thus, in alignment with real-world markets, e.g., electricity markets, we consider an augmented setting, wherein the market operator has access to some hints (contexts) that, without revealing the complete specification of the cost functions, reflect the change of the cost functions over time. In this setting, we propose an algorithm that achieves sub-linear regret on all three performance metrics, where the exact dependence of the regret guarantee on $T$ relies on the statistical properties of the function class that suppliers' cost functions belong to.


\section{Literature Review}

The design of market mechanisms to efficiently allocate resources under incomplete information on agents' preferences and costs has received considerable attention in the operations research, economics, and computer science communities. For instance, mechanism design has enabled designing optimal resource allocation strategies even in settings when certain information is privately known to agents~\cite{akbarpour2020redistributive,NIPS2004_fc03d482,heydaribeni2018distributed,pmlr-v119-deng20d}. Furthermore, inverse game theory~\citep{data-inverse-opt} and revealed preference based approaches~\cite{balcan2014learning,bei2016learning,beigman2006learning,zadimoghaddam2012efficiently} have emerged as methods to learn the underlying utilities and costs of agents given past observations of their actions. While, in line with these works, we consider an incomplete information setting wherein suppliers' cost functions are private information, we do not directly learn or elicit suppliers' cost functions to make pricing decisions as in these works and instead study the problem of learning equilibrium prices as an online decision-making problem.

%In line with the literature on revealed preferences, several of our algorithms rely on past observations of suppliers' production to make subsequent pricing decisions. However, in contrast to these works, we do not directly learn or elicit suppliers' cost functions to make pricing decisions.

The paradigm of online-decision making has enabled the allocation of scarce resources in settings with incomplete information where data is revealed sequentially to an algorithm designer and has found several applications~\cite{msvv,manshadi2021fair,lien2014sequential}. Two of the most well-studied classes of online-decision making problems include online linear programming (OLP) and online convex optimization (OCO). While OLP has been studied extensively under different informational settings, including the adversarial~\cite{msvv,TCS-057}, random permutation~\cite{online-agrawal,devanur-adwords}, and stochastic input models~\cite{li2020simple,li2021symmetry,chen2021linear,li2021online}, in this work, we consider the setting when suppliers' cost functions are convex and, in general, non-linear. Given the prevalence of non-linear objectives in various resource allocation settings, there has been a growing interest in OCO~\cite{OPT-013}, wherein several works~\cite{AgrawalD15,balseiro2022best,balseiro2021regularized} have investigated the design of algorithms with near-optimal regret guarantees under the adversarial, random permutation, and stochastic input models. 
Additionally, there have been many works on a smoothed variant of OCO where the agent must pay switching costs for changing decisions between rounds~\cite{lin2012online,goel2019beyond,chen2018smoothed,bansal20152}.
As in the works on OCO, we also consider general non-linear convex objectives; however, as opposed to the resource constraints that need to be satisfied over the entire time horizon in these works, we adopt a stronger performance metric where we accumulate regret at each period when the customer demand is not satisfied (see Section~\ref{sec:perf-measures} for more details on our performance metrics).
Thus, our algorithms are considerably different from the dual-based algorithms based on applying dual sub-gradient descent developed in these works on OCO~\cite{AgrawalD15,balseiro2022best,balseiro2021regularized}.  


Our algorithms are inspired by the multi-armed bandit literature and involve a tradeoff between exploration and exploitation in an unknown environment~\cite{freund1999adaptive,auer2002nonstochastic}.
In a typical multi-armed bandit (MAB) setting, a decision-making agent performs sequential trials on a set of permissible actions (arms), observes the outcome of the actions, and maximizes its rewards.
Several extensions of MAB have been proposed over the years, including bandits with partial observations~\cite{bartok2014partial}, contextual bandits~\cite{langford2007epoch,slivkins2011contextual}, Lipschitz bandits~\cite{mab-ms-2008}, and bandits with constrained resources~\cite{agrawal2016efficient, badanidiyuru2018bandits}.
These results have contributed to many applications, such as online posted-price auctions~\cite{oppa}, dynamic pricing with limited supply~\cite{babaioff2015dynamic}, and dark pools in stock market~\cite{agarwal2010optimal}.
In typical bandit frameworks, the decision maker's objective is to optimize a single reward function, wherein the rewards are revealed sequentially as part of the observation to the decision maker.
However, in our setting, we seek to jointly optimize multiple performance metrics where suppliers' cost functions are not revealed to the market operator (see Section~\ref{sec:perf-measures} for details).
%Therefore, our solutions leverage tools from parametric optimization to simultaneously achieve several desirable performance metrics.
