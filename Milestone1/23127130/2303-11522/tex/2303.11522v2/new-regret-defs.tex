\section{Regret Measures for Time-Varying Cost Functions} \label{sec:new-regret-defs}
In this section, we will re-define the problem setting and performance metrics with respect to the augmented setting as described in Section~\ref{sec:contextual-bandit}.
For brevity, this section focuses on clarifying the mathematical definitions and we refer the readers to Section~\ref{sec:model} for a complete discussion of the motivations and reasoning behind these definitions.
Recall that, in Section~\ref{sec:contextual-bandit}, we parameterized the suppliers' cost functions with an unknown time-invariant component $\phi_i$ and a time-varying component $\theta_{it}$ that is revealed to the market operator, i.e., 
\[ c_{it}(\cdot) = c(\cdot; \phi_i, \theta_{it}). \]

With this definition in mind, at each period $t$, the suppliers seek to maximize their profits at a given price $p$ through the following optimization problem:
\begin{maxi}|s|[2]                   % mini! = minimize 
    {x_{it} \geq 0}                               % optimization variable
    {p x_{it} - c_i(x_{it}; \phi_i, \theta_{it}). \label{eq:supObj-aug}}   % objective function and label
    {}             % label for optimization problem
    {x^*_{i}(p; \phi_i, \theta_{it}) = }                                % optimization result
\end{maxi}
And when the cost functions $c_{it}$'s are convex, we can find the market equilibrium price by solving for the dual variables of the following optimization problem:
\begin{mini!}|s|[2]                   % mini! = minimize 
    {x_{it} \geq 0, \forall i \in [n]}                               % optimization variable
    {\sum_{i = 1}^n c_i(x_{it}; \phi_i, \theta_{it}), \label{eq:supObj2-aug}}   % objective function and label
    {\label{eq:minCost-aug}}             % label for optimization problem
    {}                                % optimization result
    \addConstraint{\sum_{i = 1}^n x_{it}}{= d_t, \label{eq:demand-con-aug}} 
\end{mini!}
Note that Problems~\eqref{eq:supObj-aug} and~\eqref{eq:supObj2-aug}-\eqref{eq:demand-con-aug} differ from their counterparts in Section~\ref{sec:market-model} (i.e. Problems~\eqref{eq:supObj} and~\eqref{eq:supObj2}-\eqref{eq:demand-con}) only in the parametrization of cost functions $c_{it}$'s.



Like in Section~\ref{sec:perf-measures}, we evaluate the efficacy of an online algorithm for this setting with three regret metrics---unmet demand, payment regret, and cost regret.
Specifically, over the $T$ periods, the market operator sets a sequence of prices $p_t$ according to the online algorithm's policy $\ppi = (\pi_1, \ldots, \pi_T)$, where $p_t = \pi_t(\{ (x_{it'}^*)_{1 = 1}^n, d_{t'}, \theta_{t'} \}_{t'=1}^{t-1}, d_t, \theta_t)$ depends on the past history on the suppliers' production, consumer demands, and contexts. 
The three regret metrics represent the sub-optimality of the policy $\ppi$ relative to the optimal offline algorithm with complete information on the three desirable properties of equilibrium prices as described in Section~\ref{sec:market-model} (i.e. market clearing, minimal supplier cost, and minimal payment).

\paragraph{Unmet Demand:} We evaluate the unmet demand of an online pricing policy $\ppi$ as the sum of the differences between the demand and the total supplier productions corresponding to the pricing policy $\ppi$ at each period $t$. In particular, for an online pricing policy $\ppi$ that sets a sequence of prices $p_1, \ldots, p_T$, the cumulative unmet demand is given by
\begin{align*}
    U_T(\ppi) = \sum_{t = 1}^T \left( d_t - \sum_{i = 1}^n x_{i}^*(p_t; \phi_i, \theta_{it}) \right)_+.
\end{align*}

\paragraph{Cost Regret:} We evaluate the cost regret of an online pricing policy $\ppi$ through the difference between the total supplier production cost corresponding to algorithm $\ppi$ and the minimum total production cost, given complete information on the supplier cost functions. In particular, the cost regret $C_T(\ppi)$ of an algorithm $\ppi$ is given by
\begin{align*}
    C_T(\ppi) = \sum_{t = 1}^T \sum_{i = 1}^n c_{i}(x_{it}^*(p_t; \phi_i, \theta_{it}); \phi_i, \theta_{it}) - c_{i}(x_{it}^*(p^*_t; \phi_i, \theta_{it}); \phi_i, \theta_t),
\end{align*}
where the price $p_t^*$ for each period $t \in [T]$ is the optimal price corresponding to the solution of Problem~\eqref{eq:supObj2}-\eqref{eq:demand-con} given the demand $d_t$ and supplier cost functions $c_{it}$ for all $i \in [n]$.

\paragraph{Payment Regret:} Finally, we evaluate the payment regret of online pricing policy $\ppi$ through the difference between the total payment made to all suppliers corresponding to algorithm $\ppi$ and the minimum total payment, given complete information on the supplier cost functions. In particular, the payment regret $P_T(\ppi)$ of an algorithm $\ppi$ is given by
\begin{align*}
    P_T(\ppi) = \sum_{t = 1}^T \sum_{i = 1}^n p_t x_{i}^*(p_t; \phi_i, \theta_{it})-  p^* x_{it}^*(p^*_t; \phi_i, \theta_{it}).
\end{align*}

We note that, compared to the corresponding definitions in Section~\ref{sec:perf-measures}, we merely explicitly write out the suppliers' cost functions and production levels in terms of the parameterization of an unknown time-invariant component and a known time-varying component as discussed in Section~\ref{sec:contextual-bandit}.

