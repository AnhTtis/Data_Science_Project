\section{Fixed Cost Functions and Time-Varying Demand} \label{sec:vary-demand}

%The setting considered in the previous section of fixed supplier cost functions and customer demands over time provided intuition on designing algorithms with sub-linear regret guarantees. 

Having studied the setting of fixed supplier cost functions and customer demands over time, in this section, we investigate a more general market setting when the suppliers' cost functions are static while customer demands can vary across the $T$ periods. 
%\haoyuan{need citation for real-world examples.} 
In particular, we suppose that the customer demands for the commodity are time-varying and lie in a continuous but bounded interval, i.e., the customer demand at each period $t$ is some variable quantity $d_t \in [\underline{d}, \overline{d}]$. In this setting, we extend the algorithm developed for fixed supplier cost functions and customer demands (Algorithm~\ref{alg:fixed-demand}) and show that it achieves a regret of $O(\sqrt{T} \log \log T)$ on all three performance measures for strongly convex cost functions.


Our approach for the time-varying demand setting builds upon the algorithmic ideas for the fixed demand setting. To address the challenge that the demands can vary between the interval $[\underline{d}, \Bar{d}]$, we first consider a direct extension of Algorithm~\ref{alg:fixed-demand} to the time-varying demand setting, wherein a feasible price set is maintained for each realized demand. However, as there may be up to $O(T)$ different demand realizations over the $T$ periods, the worst-case regret of such an algorithm is $O(T)$. To resolve this issue, we leverage the intuition that customer demands that are close to each other correspond to equilibrium prices that are also close together. Thus, we uniformly partition the demand interval $[\underline{d}, \Bar{d}]$ into sub-intervals of width $\gamma$ and consider any demand in the same sub-interval the same. In particular, any demand lying in a given sub-interval, i.e., $d_t \in [\underline{d} + k \gamma, \underline{d} + (k+1) \gamma]$ for some $k \in \mathbb{N}$, is considered as a demand equal to the lower bound of that interval. Note then that from the perspective of the algorithm, there are $O(\frac{1}{\gamma})$ distinct demands, as opposed to $O(T)$ possible demand realizations, as the feasible demand interval is partitioned into $O(\frac{1}{\gamma})$ sub-intervals. Finally, for these $O(\frac{1}{\gamma})$ distinct demands, corresponding to the lower bounds of the $O(\frac{1}{\gamma})$ sub-intervals, we apply the aforementioned direct extension of Algorithm~\ref{alg:fixed-demand}. 
Our algorithmic approach is formally presented in Algorithm~\ref{alg:time-varying-demand-new}.

\begin{algorithm}
\SetAlgoLined
\SetKwInOut{Input}{Input}\SetKwInOut{Output}{Output}
\Input{Discretized demand intervals $I_1, \dots, I_K$ with $I_k = \{\underline{d} + (k-1)\gamma, \underline{d} + k\gamma\}$ such that $K\gamma = \overline{d} - \underline{d}$}
 Initialize a feasible price set $\S_{k} = (0, 1]$, current price $p_k = 0$, and price precision $\varepsilon_k = 1/2$ for each demand interval $I_{k}$\;
 \For{$t = 1, \dots, T$}{
 %\If{$d_t = 0$}{
 %Offer price 0\;
 %\textbf{continue}
 %}
 Determine $k_t$ such that $d_t \in I_{k_t} =: [a_{k_t}, b_{k_t}]$\;
 Offer price $p_{k_t}$ to the supplier\;
 \If{\text{width of feasible price set $|\S_{k_t}|$ is greater than $\frac{1}{\sqrt{T}}$}}{
 \tcc{If production exceeds target demand, then narrow down the search interval}
 \uIf{$\sum_{i = 1}^n x_{it}^*(p_{k_t}) \ge a_{k_t}$}{
 Set $\S_{k_t} \gets (p_{k_t} - \varepsilon_{k_t}, p_{k_t}]$\;
 Set next price $p_{k_t} \gets p_{k_t} - \varepsilon_{k_t}$\;
 Reset the precision to $\varepsilon_{k_t} \gets \varepsilon_{k_t}^2$\;
 }
 \uElse{
 Set next price $p_{k_t} \gets p_{k_t} + \varepsilon_{k_t}$
 }
 }
 }
\caption{Feasible Price Set Tracking for Time-Varying Demands}
\label{alg:time-varying-demand-new}
\end{algorithm}



We now present the main result of this section, which establishes that Algorithm~\ref{alg:time-varying-demand-new} achieves a regret of $O(\sqrt{T} \log \log T)$ if the sub-interval width $\gamma = \frac{1}{\sqrt{T}}$ for strongly convex cost functions of suppliers. We note that choosing $\gamma = \frac{1}{\sqrt{T}}$ optimally balances between two different sources of regret in the time-varying demand setting, as is elucidated through the proof sketch of the following theorem.


\begin{theorem} \label{thm:VaryingDemandResult}
Let the demand sub-interval width $\gamma = \frac{1}{\sqrt{T}}$. Then, the unmet demand, cost regret, and payment regret of Algorithm~\ref{alg:time-varying-demand-new} are $O(\sqrt{T} \log \log T)$ if the cost functions of the suppliers are strongly convex. %With the discretization of the demand set to $\gamma = 1/\sqrt{T}, K = (\overline{d} - \underline{d})\sqrt{T}$, the unmet demand, cost regret, and payment regret of Algorithm~\ref{alg:time-varying-demand-new} are $O(\sqrt{T} \log \log T)$ if the cost functions of the suppliers are strongly convex.
\end{theorem}

\begin{hproof}
For each of the three regret metrics, the regret incurred by Algorithm~\ref{alg:time-varying-demand-new} can be broken down into two parts: (i) the regret incurred by the Algorithm~\ref{alg:fixed-demand} sub-routine for each demand sub-interval, and (ii) the inaccuracies of considering all demands in a given sub-interval to be equal to the lower bound of that sub-interval. By invoking Theorem~\ref{thm:IdenticalResult}, the first part is of order $O(K \log\log T)$ for all three regret measures, where $K := \lceil(\overline{d} - \underline{d})/\gamma\rceil$. Next, since all demands in a given sub-interval are treated as a demand equal to the lower bound of that sub-interval and the suppliers' optimal production is monotonic in the price, every price $p_t$ offered by Algorithm~\ref{alg:time-varying-demand-new} is an under-estimate to the equilibrium price for demand $d_t$.
Thus, the second part of the regret is only positive for the unmet demand and is at most $O(\gamma T)$, as the width of each demand sub-interval is $\gamma$, and regret is only accumulated over $T$ periods. Finally, choosing $\gamma = \frac{1}{\sqrt{T}}$ achieves an optimal balance (up to logarithmic terms) between the two quantities above, i.e., $O(\gamma^{-1} \log\log T)$ and $O(\gamma T)$, which establishes the $O(\sqrt{T} \log\log T)$ regret bound.
\end{hproof}

For a complete proof of Theorem~\ref{thm:VaryingDemandResult}, see Appendix~\ref{sec:vary-demand-pf}. We reiterate that Theorem~\ref{thm:VaryingDemandResult} applies to strongly convex cost functions as with Theorem~\ref{thm:IdenticalResult} and that extending this result to general convex cost functions, e.g., linear functions, is, in general, not possible (see Example~\ref{eg:linear-cost} in Section~\ref{sec:fixed-convex-limitations}). Furthermore, compared to the regret guarantee obtained in Theorem~\ref{thm:IdenticalResult}, Theorem~\ref{thm:VaryingDemandResult} establishes that the time-varying nature of the customer demand incurs an additional factor of $O(\sqrt{T})$ in the regret guarantee as compared to the setting with fixed demands. However, we do note that if the set of demand realizations $D$ is known \emph{a priori} to be $o(\sqrt{T})$, then the regret guarantee in Theorem~\ref{alg:time-varying-demand-new} can be improved to $O(|D| \log \log T)$ by running the direct extension of Algorithm~\ref{alg:fixed-demand}, wherein a feasible price interval is maintained for each realized demand. Finally, we note that the regret guarantee obtained in Theorem~\ref{thm:VaryingDemandResult} compares favorably to classical $O(\sqrt{T})$ regret guarantees in the OCO or MAB literature~\cite{OPT-013}. 

