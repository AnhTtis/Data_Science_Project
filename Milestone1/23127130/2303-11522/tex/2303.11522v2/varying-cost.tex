\section{Time-Varying Cost Functions} \label{sec:time-vary-cost-main}
In this section, we consider the general setting, where, in addition to customer demands changing over time, suppliers' cost functions can also vary across the $T$ periods. Formally, at each period $t$, each supplier $i$ has a privately known and time-varying cost function $c_{it}(\cdot)$. Compared to the setting with fixed cost functions, the fundamental ideas underlying the performance guarantees of Algorithms~\ref{alg:fixed-demand} and \ref{alg:time-varying-demand-new} do not directly apply to this setting, as suppliers' production may not remain the same when the operator offers the same price at different periods due to the time-varying nature of their cost functions. In fact, in Section~\ref{sec:counter-example}, we show that if the operator does not know the process that governs the variation of the cost functions, then sub-linear regret is impossible to achieve on all three regret metrics.
%that even if the cost functions of the suppliers are drawn i.i.d. from a distribution, no online algorithm can achieve sub-linear regret on all three regret metrics. 
%\haoyuan{I think the way you stated is somewhat misleading, as the distribution is a form of information.} \devansh{I have changed the distribution part here to the statement you had so this should be better}
%This result highlights that if the operator does not know the process that deterministically governs the variation of the cost functions, then sub-linear regret is impossible to achieve.
%This result highlights that if the operator does not have any information on the process that governs the variation of the cost functions, even if the cost functions are drawn i.i.d. from a distribution, then sub-linear regret is impossible to achieve.
To this end, in Section~\ref{sec:contextual-bandit}, in alignment with real-world markets, we consider an augmented problem setting wherein the market operator is provided with a hint (i.e., context) on the variation in suppliers' cost functions over time, e.g., due to weather conditions in electricity markets, while still keeping the full description of the costs away from the operator. In this setting, where the operator has access to additional hints on suppliers' cost functions, we then develop an algorithm with sub-linear on all three regret metrics (see Sections~\ref{sec:igw-alg-sol} and~\ref{sec:igw-alg-sketch}) through an adaptation of an algorithm in the contextual bandits literature~\cite{foster2020beyond}.


\subsection{Impossibility of Setting Equilibrium Price Under Time-Varying Costs}
\label{sec:counter-example}


We initiate our study of the setting of time-varying costs by presenting an example that illustrates the impossibility of setting equilibrium prices if the market operator has no information on how the cost functions of suppliers change over time. In particular, Proposition~\ref{prop:time-varying-cost-countereg} presents a counterexample establishing that even if suppliers' cost functions are drawn i.i.d. from a known distribution, no online algorithm can achieve a sub-linear regret on all three regret metrics as long as the operator is not informed about the outcome of the random draws from the distribution.

%the difficulty of the setting with time-varying supplier cost functions arises from the fact that the market operator cannot observe a suppliers' cost function


\begin{proposition} [Impossibility of Sub-linear Regret for Time-varying Costs] \label{prop:time-varying-cost-countereg}
There exists an instance with fixed time-invariant demand and a single supplier whose cost functions are drawn i.i.d. from some (potentially known) distribution such that no online algorithm can achieve sub-linear regret on all three regret metrics.
\end{proposition}

\begin{hproof}
We consider a setting with a fixed demand of $d=1$ at every period and a single supplier whose cost functions at each period are drawn from a distribution such that at each period $t$, its cost function could be either $c_1(x) = \frac{1}{8}x^2$ or $c_2(x) =\frac{1}{16} x^2$, each with probability $0.5$. We suppose that the market operator has knowledge of the distribution from which the supplier's cost function is sampled i.i.d. but does not know the outcome of the random draw at any period and show that any pricing strategy adopted by the operator must incur a linear regret on at least one of the three regret metrics for this instance. To this end, we first note that the equilibrium price corresponding to the cost function $c_1(x)$ is $p_1^* = \frac{1}{4}$ while that corresponding to the cost function $c_2(x)$ is $p_2^* = \frac{1}{8}$. Then, we analyze the total regret, i.e., the sum of the unmet demand, payment regret, and cost regret, for three different price ranges - (i) $p<\frac{1}{8}$, (ii) $1/8 \le p \le 1/4$, and (iii)  $p > 1/4$ - and show that irrespective of the set price $p$ at any period $t$, the expected total regret at any period is at least $\frac{7}{64}$, i.e., the total regret is at least $\frac{7}{64} T$. Finally, since the sum of the three regret metrics is linear in $T$, at least one of the three metrics must be linear in $T$, establishing our claim.%with two possible cost functions $c_1(x) = \frac{1}{8}x^2$ and $c_2(x) =\frac{1}{16} x^2$
\end{hproof}

For a complete proof of Proposition~\ref{prop:time-varying-cost-countereg}, see Appendix~\ref{sec:pf-prop-countereg}. While it was possible to achieve sub-linear regret in the setting with time-varying customer demands (see Theorem~\ref{thm:VaryingDemandResult}), Proposition~\ref{prop:time-varying-cost-countereg} establishes that such a result is, in general, not possible in the setting with time-varying cost functions. The setting with time-varying cost functions is more challenging because the market operator observes customer demands, which it can use to make pricing decisions, but does not observe the cost functions of suppliers.
%Additionally, the example in the proof of Proposition~\ref{prop:time-varying-cost-countereg} illustrates the difficulty in jointly balancing the three regret metrics.
%In particular, even though both the payment regret and cost regret are negative when the price is set below the equilibrium price, setting such prices corresponds to a higher unmet demand.
%Thus, any decrease in the payment or cost regret would correspond to a similar increase in the unmet demand and vice versa, which results in a total regret, i.e., the sum of the unmet demand, payment regret, and cost regret, which is linear in the number of periods $T$.
Further, in contrast to the settings in~\cite{yu2017online,balseiro2022best}, where online gradient descent approaches can simultaneously achieve sub-linear regret for multiple performance metrics, we note that our definition of unmet demand is considerably stronger as over-production at particular periods cannot compensate for unmet demand at other periods (see Section~\ref{sec:perf-measures} for a further discussion).
Thus, Proposition~\ref{prop:time-varying-cost-countereg} shows that, with the stronger unmet demand metric, it is impossible to jointly optimize the three regret metrics, and illustrates the difficulty of balancing the three regret metrics, as decreasing the payment or cost regret causes an increase in the unmet demand and vice versa.


\subsection{Adding Contexts for Time-varying Costs}
\label{sec:contextual-bandit}

Proposition~\ref{prop:time-varying-cost-countereg} highlights that if the operator does not have any information on the change in suppliers' cost functions over time, it is impossible to achieve sub-linear regret on all three regret metrics. To this end, in this section, we consider a natural augmented problem setting wherein the market operator, without knowing the complete specification of cost functions of suppliers, additionally has access to a hint (i.e., context) that reflects the variation in cost functions of suppliers over time. We note that such a setting aligns with real-world markets, e.g., electricity markets, wherein the cost functions of suppliers are private information yet will typically vary over time based on observed quantities, such as changes in the ambient weather conditions.  
%\devansh{can consider adding a reference here...}

To specify the augmented problem setting with contexts, we first introduce some notation regarding the cost functions of suppliers. In particular, we assume that each supplier's cost function is composed of two parts: (i) an unknown component that is time-invariant, and (ii) a time-varying component that is revealed to the marker operator. More precisely, the cost function of each supplier $i$ is parameterized as follows:
\[c_{it}(\cdot) = c_i(\cdot; \phi_i, \theta_{it}),\]
where $\phi_i$ is private information and $\theta_{it}$ is the time-varying component of the cost function given to the operator as \textit{contexts}.
Note that for any fixed $\phi_i$, the context $\theta_{it}$ uniquely determines the cost function of supplier $i$ at time $t$.
We stress that we do not assume any structure on the parameterization of the cost functions and so the time-varying and time-invariant components of the cost functions need not be separable.
Further, since $\phi_i$'s are unknown, the market operator cannot directly solve Problem~\eqref{eq:supObj2}-\eqref{eq:demand-con} to obtain the equilibrium prices in the market.

For the simplicity of exposition, for the remainder of this section, we aggregate all suppliers' cost functions into a combined cost $c_t(\cdot; \theta_t) = \sum_{i=1}^n c_{it}(\cdot; \phi_i, \theta_{it})$, %where we denote $\phi$ as the private information and $\theta_t$ as the time-varying context for all suppliers associated with the combined cost function.
where $\theta_t = (\theta_{1t}, \dots, \theta_{nt})$ is the time-varying context associated with the combined cost function. 
Note that doing so is without loss of generality as all suppliers have convex costs and observe the same prices in the market.
Furthermore, we note that since the private information $\phi_1, \dots, \phi_n$ are unknown, the market operator cannot directly solve Problem~\eqref{eq:supObj2}-\eqref{eq:demand-con} to obtain the equilibrium prices in the market.

In this augmented problem setting, at each period $t$, in addition to receiving the customer demand $d_t$, the market operator observes a context $\theta_t$, which it can use along with the prior history of supplier production quantities, customer demands, and contexts, to set a price $p_t$. In particular, with access to sequentially arriving contexts, the market operator sets a sequence of prices given by the pricing policy $\ppi = (\pi_1, \ldots, \pi_T)$, where $p_t = \pi_t(\{ (x_{t'}^*)_{1 = 1}^n, d_{t'}, \theta_{t'} \}_{t'=1}^{t-1}, d_t, \theta_t)$, where $x_{t}^*$ represents the sum of optimal production quantity corresponding to the solution of Problem~\eqref{eq:supObj} for each supplier at period $t$. We then evaluate the performance of this class of pricing policies on three regret metrics introduced in Section~\ref{sec:perf-measures}.
Note that we can naturally extend these three metrics to the augmented setting with contexts by plugging in $c_{it}(\cdot) = c_i(\cdot; \phi_i, \theta_{it})$ and for completeness, we present the corresponding definitions explicitly in Appendix~\ref{sec:new-regret-defs}. 


\subsection{Algorithm for Time-Varying Costs with Contexts}
\label{sec:igw-alg-sol}

We now present an algorithm that simultaneously achieves sub-linear regret for the unmet demand, payment regret, and cost regret metrics for the augmented problem setting introduced in Section~\ref{sec:contextual-bandit}. Our algorithmic approach is inspired by recent ideas in the contextual bandits literature (e.g.~\cite{agarwal2014taming,foster2020beyond}) and involves two building blocks. 
First, we seek to learn how to associate the arriving contexts with the relevant properties of suppliers' cost functions. Next, based on the information inferred from the arriving contexts, our algorithm offers prices to suppliers in the next period.
We note that the first step of our approach fundamentally differs from the fixed cost setting, as time-varying supplier costs, unlike time-varying demands, are not observed and thus unavailable to the operator making pricing decisions.
Therefore, the first step of our approach is crucial in learning a descriptive model on the supplier's response to various contexts and prices.
%Hence, the first step differs from the setting with varying demands and fixed costs, as the market operator observes the customer demand and uses it to make subsequent pricing decisions.

The task of learning to associate the arriving contexts with the relevant properties of the cost function is accomplished by an \textit{online regression oracle}. In particular, an online regression oracle performs real-valued online regression and achieves a prediction error guarantee, with a bound denoted $\est(T)$, relative to the best function in a class $\mathcal{F}$.


\begin{definition}[Online Regression Oracle]
Consider a function class $\mathcal{F} : \mathcal{A} \to \mathcal{B}$, at each time $t$, the online regression oracle receives an input $a_t$ and computes an estimate $\hat{b}_t = \hat{f}_t(a_t)$, where $\hat{f}$ depends on the past history
\[ \mathcal{H}_{t-1} = (a_1, b_1), \dots, (a_{t-1}, b_{t-1}). \]
%and not necessarily in $\mathcal{F}$\footnote{Some literature also require that $\hat{f} \in \mathcal{F}$, but this typically is not an issue for most problem settings, e.g. when $\mathcal{F}$ is convex.}. 
Then, the oracle receives the true output $b_t$.

The predictors $\hat{f}_t$ of the oracle are almost as accurate as any function in $\mathcal{F}$ in the sense that:
\[\sum_{t=1}^T (\hat{b}_t - b_t)^2 - \inf_{f \in \mathcal{F}} \sum_{t=1}^T (f(a_t) - b_t)^2 \le \est(T)\]
\end{definition}

The prediction error $\est(T)$ of the online regression oracle scales with the ``size'' of the $\mathcal{F}$ in a statistical sense. As an example, if the function class $\mathcal{F}$ is finite, then the exponential weights update algorithm achieves $\est(T) \le \log |\mathcal{F}|$ (see e.g.~\cite{vovk1995game}).
Therefore, the function class $\mathcal{F}$ should be rich enough to capture the map between contexts and the variation in cost functions and also not be too large so that the estimation error is small.

To construct a function class $\mathcal{F}$ appropriate for our problem setting, we first note that the market operator cannot observe the supplier's costs but can instead use the oracle to regress on the supplier's production $x^*(\cdot; \theta_t)$, which is directly observable.
Note that by the first-order optimality condition on Problem~\eqref{eq:supObj} that the production and price are related as follows:
\[c_t'(x^*(p; \theta_t); \theta_t) = p \implies x^*(p; \theta_t) = (c_t')^{-1}(p; \theta_t).\]
Note that when the cost functions are strongly convex (i.e., $c_t'$ is invertible), the production level $x^*(\cdot; \theta_t)$ is well-defined as a function of the price $p_t$ and context $\theta_t$.
Thus, we define the function class $\mathcal{F}$ as the possible mappings from the price-context tuple $(p, \theta)$ to the production $x^*$, i.e., the oracle tries to determine the amount of the supplier's production given a price $p_t$ and context $\theta_t$.


With our choice of the online regression oracle, we now present our algorithm for the setting of time-varying costs with contexts based on the inverse gap weighing method introduced in~\cite{foster2020beyond}. To present our algorithmic approach, we restrict the algorithm's choices to a finite set of $K$ prices that are uniformly spaced on the interval $[0, 1]$, where the performance of our algorithm will depend on the choice of $K$ (see Theorem~\ref{thm:igw-bound-informal}).
Given the oracle's output $\hat{f}_t$ that estimates the production quantity $x^*(p_t; \theta_t)$, the greedy choice at each period $t$ is to match the requested demand $d_t$ as closely as possible, i.e. choose a price $\hat{p}_t$ such that the quantity $|\hat{f}_t(\hat{p}_t; \theta_t) - d_t|$ is minimized.
To balance between both exploration and exploitation, Algorithm~\ref{alg:time-varying-cost} instead samples each price $p_t$ from the set of $K$ discrete prices according to a probability distribution $\Delta_t$. 
%In particular, rather than deterministically choosing the price $\hat{p}_t$ greedily at each period based on the observed context $\theta_t$ and demand $d_t$, Algorithm~\ref{alg:time-varying-cost} assigns a probability distribution $\Delta_t(\cdot)$ to sample one of the $K$ prices in the discretized price set. 
In effect, Algorithm~\ref{alg:time-varying-cost} achieves good exploration by choosing any of the $K$ prices with some positive probability, which ensures that the online regression oracle has access to a wide-ranging history $\mathcal{H}_{t-1}$, so it can achieve a low prediction error even when the market operator receives a new context or demand. Furthermore, to minimize the penalty for exploration, the probability distribution $\Delta_t$ is chosen such that it assigns the highest probability to the greedy choice $\Hat{p}_t$ under the current oracle estimate $\hat{f}_t$ while assigning a probability to every other price that is roughly inversely proportional to the gap between its unmet or excess demand and that of the greedy choice $\Hat{p}_t$. Then, for each period $t$, given this choice of $\Delta_t$, a price $p_t$ is sampled from this distribution, following which suppliers produce an optimal quantity $x^*(p_t; \theta_t)$ of the commodity as given by the solution of Problem~\eqref{eq:supObj}. Finally, the oracle is updated with the new context $\theta_t$, customer demand $d_t$, and optimal supplier production to generate a new estimator $\Hat{f}_{t+1}$ for the next period. This process is presented formally in Algorithm~\ref{alg:time-varying-cost}.
%The above process is repeated for the $T$ period horizon.

 

\begin{algorithm}
\SetAlgoLined
\SetKwInOut{Input}{Input}\SetKwInOut{Output}{Output}
\Input{Online regression oracle $\mathcal{O}$ with input pairs $(\theta_t, p_t)$ and output $x_t$; uniform $K$-cover of possible prices $0 = p_1 < p_2 < \cdots < p_K = 1$; exploration parameter $\gamma > 0$}
 \For{$t = 1, \dots, T$}{
 Query the oracle $\mathcal{O}$ for an estimator $\hat{f}_t$\;
 Receive context $\theta_t$ and demand $d_t$\;
 Sample price $p_t$ from the probability distribution 
    \[\Delta_t(p_i) = \frac{1}{\lambda + 2 \gamma \left(|\hat{f}_t(p_i; \theta_t) - d_t| - |\hat{f}_t( \hat{p}_t; \theta_t) - d_t|\right)},\]
    with $\hat{p}_t = \argmin_{p \in \{p_1, \dots, p_K\}} |\hat{f}_t(p; \theta_t) - d_t|$ and $\lambda \in (0, K)$ as the normalization constant\;
Commit $p_t$ and observe the production $x_t = x^*(p_t; \theta_t)$ corresponding to the solution of Problem~\eqref{eq:supObj} given the price $p_t$\; 
Update the oracle $\mathcal{O}$ with $((\theta_t, p_t), x_t)$\;
}
\caption{Online Equilibrium Pricing for Time-Varying Costs}
\label{alg:time-varying-cost}
\end{algorithm}

We now present the main result of this section, which establishes that for an appropriate choice of the discretization $K$ of the price set, Algorithm~\ref{alg:time-varying-cost} can achieve sub-linear regret on all three regret metrics, as is elucidated through the following theorem. 
We highlight that this theorem holds for any sequence of contexts $\theta_t$, which means that $\theta_t$ can be derived from some physical dynamics, drawn from a probability distribution, or even chosen adversarially.
%\haoyuan{add references for these settings.} 

\begin{theorem}[Informal]
    \label{thm:igw-bound-informal}
    With high probability, for any sequence of contexts $\theta_t$ and customer demands $d_t$ over $T$ periods, the unmet demand, payment regret, and cost regret satisfy:
    \[ \EE_{p_t \sim \Delta_t, t \in [T]}[U_T(p_1, \dots, p_T)] \le O\left(\sqrt{KT \cdot \est(T)} + \frac{T}{K}\right),\]
    and similarly for $\EE_{p_t \sim \Delta_t, t \in [T]}[P_T(p_1, \dots, p_T)]$ and $\EE_{p_t \sim \Delta_t, t \in [T]}[C_T(p_1, \dots, p_T)]$, where $K$ is the number of prices in the uniformly discretized price set.
\end{theorem}
We note that because the pricing policy in Algorithm~\ref{alg:time-varying-cost} is probabilistic, we present the regret bounds in expectation with respect to the distributions $\Delta_t$.
For the simplicity of exposition, we only present Theorem~\ref{thm:igw-bound-informal} as an informal statement and present the analysis of a rigorous theorem statement, which involves introducing additional notation, e.g., quantifying the high probability bound, in Appendix~\ref{sec:igw-alg-proof}. Furthermore, we present a sketch of the main ideas used to analyze Algorithm~\ref{alg:time-varying-cost} in Section~\ref{sec:igw-alg-sketch}.

\begin{remark}
Recall that, when the function class $\mathcal{F}$ is finite, e.g. there are only finitely many possible cost functions, the exponential weights algorithm can achieve $\est(T) = \log |\mathcal{F}|$.
Therefore, picking $K = \sqrt[3]{T / \log |\mathcal{F}|}$ achieves a sublinear regret bound of $O(T^{2/3} \sqrt[3]{\log |\mathcal{F}|})$.


Additionally, in Appendix~\ref{sec:igw-alg-proof},  we discuss several other function classes $\mathcal{F}$, including parametric classes, bounded Lipschitz functions, and neural networks with bounded spectral norm, for which Algorithm~\ref{alg:time-varying-cost} achieves a sub-linear regret guarantee by leveraging results from the statistical learning literature (e.g. see~\cite{wainwright2019high,rakhlin2014online}).
\end{remark}

Note that the regret bound obtained for finite function classes corresponds to an additional factor of $O(T^{1/6})$ than the guarantee for the setting with time-varying demands and fixed costs (Theorem~\ref{thm:VaryingDemandResult}). The additional loss in the regret for the setting of time-varying costs is because Algorithm~\ref{alg:time-varying-cost} utilizes an online regression oracle to associate the contexts with properties of the supplier's varying cost functions that are unknown to the market operator.
Note that by performing regression, Algorithm~\ref{alg:time-varying-cost} learns all of the features about the supplier's optimal production as a function of the contexts and prices.
In contrast, Algorithm~\ref{alg:time-varying-demand-new} only learns the equilibrium prices for a discrete set of demands and is not concerned with accurately predicting the supplier's production for other demands.
Thus, the additional factor of $O(T^{1/6})$ in the regret is attributable to the fact that Algorithm~\ref{alg:time-varying-cost} attempts to solve a more complex statistical problem than in the fixed cost setting.
Nevertheless, the guarantee in Theorem~\ref{thm:igw-bound-informal} still compares favorably to the best known algorithms in related problem settings, e.g., no online algorithm can achieve a regret better than $O(T^{2/3})$ for a Lipschitz bandit with contexts in $\RR$ (see e.g.~\cite{slivkins2011contextual,foster2020beyond}).

\begin{comment}
A few comments about the bound obtained in Theorem~\ref{thm:igw-bound-informal} and the choice of the parameter $K$ to achieve a sub-linear regret bound on all three performance metrics are in order. In particular, the first term of the regret bound corresponds to the regret associated with achieving a balance between exploration and exploitation if the optimal price at each period belongs to one of the $K$ prices in the discretized set.
On the other hand, the second term $\frac{T}{K}$ of the regret bound corresponds to the error due to the discretization of the price set, as, in general, the equilibrium price at each period belongs to a continuous interval and may not correspond to one of the $K$ prices chosen in Algorithm~\ref{alg:time-varying-cost}. %Observe that, as expected, the regret due to the discretization of the price set is lower for larger values of $K$. 
Then, the choice of $K$ to achieve a sub-linear regret bound by balancing between the two sources of regret depends on the function class $\mathcal{F}$, which influences the prediction error $\est(T)$ corresponding to the online regression oracle. 

As previously illustrated, if the function class $\mathcal{F}$ is finite, we can achieve a regret bound of $O(T^{2/3} \sqrt[3]{\log |\mathcal{F}|})$, which is sub-linear in the number of periods $T$. 
Note that this regret bound corresponds to an additional factor of $O(T^{1/6})$ than the guarantee for the setting with time-varying demands and fixed costs (Theorem~\ref{thm:VaryingDemandResult}). The additional loss in the regret for the setting of time-varying costs is because Algorithm~\ref{alg:time-varying-cost} utilizes an online regression oracle to associate the contexts with properties of the supplier's varying cost functions that are unknown to the market operator.
Note that by performing regression, Algorithm~\ref{alg:time-varying-cost} learns all of the features about the supplier's optimal production as a function of the contexts and prices.
In contrast, Algorithm~\ref{alg:time-varying-demand-new} only learns the equilibrium prices for a discrete set of demands and is not concerned with accurately predicting the supplier's production for other demands.
Thus, the additional factor of $O(T^{1/6})$ in the regret is attributable to the fact that Algorithm~\ref{alg:time-varying-cost} attempts to solve a more complex statistical problem than in the fixed cost setting.
Nevertheless, the guarantee in Theorem~\ref{thm:igw-bound-informal} still compares favorably to the best known algorithms in related problem settings, e.g., no online algorithm can achieve a regret better than $O(T^{2/3})$ for a Lipschitz bandit with contexts in $\RR$ (see e.g.~\cite{slivkins2011contextual,foster2020beyond}).    
\end{comment}





\subsection{Sketch of Main Ideas to Analyze Algorithm~\ref{alg:time-varying-cost}}
\label{sec:igw-alg-sketch}


In this section, we shall describe the key ideas behind the analysis of Algorithm~\ref{alg:time-varying-cost}.
We first establish the necessary Lipshitzness relations between the problem parameters, which enables us to simultaneously optimize over multiple objectives, as opposed to standard contextual bandit algorithms that typically optimize a single regret metric. We then define a notion of ``proxy'' regret and show that Algorithm~\ref{alg:time-varying-cost} achieves the desired regret bound in the statement of Theorem~\ref{thm:igw-bound-informal}. Finally, Theorem~\ref{thm:igw-bound-informal} follows as the ``proxy'' regret serves as an upper bound on the unmet demand, payment regret, and cost regret due to the derived Lipshitz relations.


Our first lemma establishes the Lipschitzness between the optimal prices corresponding to the solution of Problem~\eqref{eq:supObj2}-\eqref{eq:demand-con} and customer demands utilizing techniques from parametric optimization. 

\begin{lemma} [Lipschitzness of Prices in Demands] \label{lem:PriceLipschitz}
The optimal prices corresponding to the dual variables of the market clearing constraint of Problem~\eqref{eq:supObj2}-\eqref{eq:demand-con} are Lipschitz in the demand $d$, i.e., $|p^*(d_1) - p^*(d_2)| \leq L_1 |d_1 - d_2|$ for some constant $L_1>0$ for all $d_1, d_2 \in [\underline{d}, \Bar{d}]$. Here, $p^*(d)$ is the optimal price corresponding to the dual variable of the market clearing constraint of Problem~\eqref{eq:supObj2}-\eqref{eq:demand-con} with a customer demand of $d$.
\end{lemma}

Next, we use Lemma~\ref{lem:PriceLipschitz} to show that both the payment and cost regret metrics are Lipschitz in the prices. That is, Lemma~\ref{lem:lipschitzPricesRegret} establishes that small changes in the prices result in only small changes in the cost and payment regret metrics.

\begin{lemma} [Lipschitzness of Regret Metrics in Prices] \label{lem:lipschitzPricesRegret}
Consider an online pricing policy $\ppi$
The payment and cost regret are upper bounded by the absolute difference in prices $p_t$ corresponding to the online pricing policy $\ppi$ and equilibrium prices $p^*_t$.
Namely, there exists some constant $L_2, L_3 > 0$ so that 
\[P_T(\ppi) \le L_2 \sum_{t=1}^T |p_t - p^*_t|, \text{and }  C_T(\ppi) \le L_3 \sum_{t=1}^T |p_t - p^*_t|. \]
\end{lemma}
For proofs of Lemmas~\ref{lem:PriceLipschitz} and~\ref{lem:lipschitzPricesRegret}, we refer the readers to Appendix~\ref{sec:lipschitz-lemma}.

By employing these two lemmas, we can upper bound unmet demand, payment regret, and cost regret with the following proxy regret metric:
\[\textsc{Reg}(T) := \sum_{t=1}^T \EE_{p_t \sim \Delta_t} \left[\left|x^*(p_t; \theta_t) - d_t\right| \mid \mathcal{H}_{t-1} \right],\]
where $x^*(p_t; \theta_t)$ is the optimal production quantity of the suppliers and $p_t$ is chosen according to the distribution $\Delta_t$ as defined in Algorithm~\ref{alg:time-varying-cost}. 
Note that the choice of distribution $\Delta_t$ depends on the past history $\mathcal{H}_{t-1} = ((p_1, \theta_1), x_1), \dots ((p_{t-1}, \theta_{t-1}), x_{t-1})$.
We can employ Lemmas~\ref{lem:PriceLipschitz} and~\ref{lem:lipschitzPricesRegret} to show that the proxy regret upper bounds the unmet demand, payment regret, and cost regret metrics.

First, we note that this proxy regret upper bounds the unmet demand because
\[\left|x^*(p_t; \theta_t) - d_t\right| \ge \left(x^*(p_t; \theta_t) - d_t\right)_+.\]
Next, let $p^*_t$ be the equilibrium price for demand $d_t$, and note that $p_t$ is the equilibrium price when demand is equal to $x^*(p_t; \theta_t)$.
So, by Lemma~\ref{lem:PriceLipschitz}, it follows that 
\[ |p_t  - p^*_t| \le L_1 \left|x^*(p_t; \theta_t) - d_t\right| \] 
for some constant $L$.
And by applying Lemma~\ref{lem:lipschitzPricesRegret} to the previous inequality, we conclude that 
\[\EE_{p_t \sim \Delta_t, t \in [T]}[P_T] \le L_1 \sum_{t=1}^T \EE_{p_t \sim \Delta_t} \left|p_t - p^*_t\right| \le L_1 L_2 \sum_{t=1}^T \EE_{p_t \sim \Delta_t} \left|x^*(p_t; \theta_t) - d_t\right| = L_1 L_2 \textsc{Reg}(T).\]
Using a similar line of reasoning, we can also show that the proxy regret serves as an upper bound, up to constants, for the cost regret $\EE_{p_t \sim \Delta_t, t \in [T]}[C_T]$.


Given the above observation that the proxy regret is an upper bound on the three regret metrics, it suffices to show that Algorithm~\ref{alg:time-varying-cost} achieves a sub-linear regret, as is elucidated by the following proposition.
\begin{proposition}[informal]
\label{thm:igw-regret-informal}
    With high probability, for any sequence of context $\theta_t$ and demand $d_t$, the proxy regret is bounded by
    \[ \textsc{Reg}(T) \le O\left(\sqrt{KT \cdot \est(T)} + \frac{T}{K}\right).\]
\end{proposition}
Note that Theorem~\ref{thm:igw-bound-informal} is an immediate consequence of Proposition~\ref{thm:igw-regret-informal}. For a complete proof of Proposition~\ref{thm:igw-regret-informal}, see Appendix~\ref{sec:igw-alg-proof}.

\begin{comment}

Note that Theorem~\ref{thm:igw-bound-informal} is an immediate consequence of Proposition~\ref{thm:igw-regret-informal}. In the following, we present an outline of the key ideas for proving Proposition~\ref{thm:igw-regret-informal} and defer the complete proof to Appendix~\ref{sec:igw-alg-proof}.


Depending on the structure of the contexts, we can invoke well-known results from statistical learning to give explicit bounds on $\est(T)$ and therefore provide sub-linear regret guarantee on vaious problems.
Before we proceed, we first define $\mathcal{N}_p(\mathcal{F}, \varepsilon)$ as the $\varepsilon$-covering number of the function class $\mathcal{F}$ with respect to norm $\norm{\cdot}_p$.
\haoyuan{Need to double check the constants here.}
\begin{itemize}
    \item If $\mathcal{F}$ is finite, then with exponential weights algorithm, we get $\est(T) = \log |\mathcal{F}|$ and a choice of $K = T^{1/3}$ results in regret of $O(T^{2/3} \log |\mathcal{F}|)$.
    \item If $\mathcal{F}$ is parametric in the sense that $\mathcal{N}_2(\mathcal{F}, \varepsilon) \asymp \varepsilon^{-d}$ for some $d > 0$, then~\cite{rakhlin2014online} shows that $\est(T) \in O(d \log T)$. So, the regret becomes $\widetilde{O}(d \cdot T^{2/3})$.
    \item If $\mathcal{F}$ are the set of bounded Lipschitz functions over $(\theta, p) \in \RR^{d+1}$, then $\log \mathcal{N}_\infty(\mathcal{F}, \varepsilon) = \varepsilon^{-d-1} $ (see Example 5.10 and 5.11 in~\cite{wainwright2019high} for an explicit construction).
    And~\cite{rakhlin2014online} shows that we then have $\est(T) \in O(T^{(d+1)/(d+2)})$.
    A choice of $K = T^{1/(d+4)}$ results in $\textsc{Reg}(T) \lesssim T^{(d+3)/(d+4)}$.
    \item If $\mathcal{F}$ is a neural network with appropriately bounded spectral norm, then~\cite{rakhlin2014online, bartlett2017spectrally} show that $\est(T) \in O(T^{-1/2})$, and this results in a regret of $O(T^{4/5})$.
\end{itemize}

For more details, please refer to the analysis in Appendix~\ref{sec:igw-alg-proof}.


The key motivation behind Algorithm~\ref{alg:time-varying-cost} is to optimally balance between exploiting the best price given the known information and exploring the price range so the algorithm does not incur a large regret when a new demand $d_t$ is different from all previous demands $d_{1:t-1}$.
We note that the distribution $\Delta_t$ accomplishes this goal by assigning exploration probability to the prices according to how closely they match the demand $d_t$.
This way, under the algorithm's estimate $\hat{f}_t$, the penalty for exploration is up to
\[\sum_{i=1}^K \frac{|\hat{f}_t(p_i; \theta_t) - d_t| - |\hat{f}_t( \hat{p}_t; \theta_t) - d_t|}{\lambda + 2 \gamma \left(|\hat{f}_t(p_i; \theta_t) - d_t| - |\hat{f}_t( \hat{p}_t; \theta_t) - d_t|\right)} \le \frac{K-1}{2\gamma}.\]
After bounding the difference between the estimated production $\hat{f}_t$ and the actual production function
Recall that $\hat{p}_t \argmin_{p \in \{p_1, \dots, p_K\}} |x^*(p; \theta_t) - d_t|$, so the summation term is 0 for one of the price $p_i$.
After bounding the difference in the estimated production function $\hat{f}_t$ and the actual production function $x^*$, we have the following inequality:
\begin{align*}
    &\EE_{p_t \sim \Delta_t} \left[|x^*(\theta_t, p_t) - d_t| \, \bigg| \, \mathcal{H}_{t-1}\right] - \min_{p \in \{p_1, \dots, p_K\}} |x^*(p; \theta_t) - d_t| \\
    \le{}& \frac{K}{\gamma} + \frac{\gamma}{2} \cdot \EE_{p_t\sim\Delta_t} \left[ (|x^*(p_t; \theta_t) - d_t| - |\hat{f}_t(p_t; \theta_t) - d_t|)^2 \right]
\end{align*}
We now try to relate the quantity $(|\hat{f}_t(p_i; \theta_t) - d_t| - |\hat{f}_t( \hat{p}_t; \theta_t) - d_t|)^2$ with the guarantee of the online regression oracle $\est(T)$.
First, we bridge the difference between the regret metric we try to optimize and the output of online regression oracle by computing
\[(|x^*(\theta_t, p_t) - d_t| - |\hat{f}_t(\theta_t, p_t) - d_t|)^2 \le (x^*(\theta_t, p_t) - \hat{f}_t(\theta_t, p_t))^2.\]
Next, through standard techniques in probability theory, we can show that the online regression oracle also satisfies 
    \[\sum_{t=1}^T \EE_{p_t \sim \Delta_t} \left[(x^*(\theta_t, p_t) - \hat{f}_t(\theta_t, p_t))^2\right] \le \est(T)\]
with high probability.
This means that 
\[\sum_{t=1}^T \EE_{p_t \sim \Delta_t} \left[|x^*(\theta_t, p_t) - d_t| \, \bigg| \, \mathcal{H}_{t-1}\right] - \min_{p \in \{p_1, \dots, p_K\}} |x^*(p; \theta_t) - d_t| \le \frac{KT}{\gamma} + \gamma \cdot \est(T).\]
Due to Lemma~\ref{lem:ProductionLipschitz}, we know that there exists constant $L > 0 $ where
\[ \min_{p \in \{p_1, \dots, p_K\}} |x^*(p; \theta_t) - d_t| = \min_{p \in \{p_1, \dots, p_K\}} |x^*(p; \theta_t) - x^*(p^*_t, \theta_t)|\le L \cdot \min_{p \in \{p_1, \dots, p_K\}} |p - p^*_t| \le \frac{L}{K}. \]
Hence,
\[\sum_{t=1}^T \min_{p \in \{p_1, \dots, p_K\}} |x^*(p; \theta_t) - d_t| \in O\left(\frac{T}{K}\right).\]
Finally, a choice of $\gamma = \sqrt{\est(T)/(KT)}$ leads to the bound in Proposition~\ref{thm:igw-regret-informal}.
\end{comment}