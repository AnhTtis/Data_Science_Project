% General mathematical macros.
% The version current as of 14 January 1990.
% Many general-purpose macros moved to myadd.sty

%%%%%%%%%%ARROWS%%%%%%%%%%

%Function macros: #1 : #2 --> #3
\newcommand{\function}[3]{ #1 : #2 \rightarrow #3 }
%\newcommand{\fn}{\function}
 \ifdefinedfix{fn}{}{\function}
 \ifdefinedfix{myfn}{}{\function}

%Element function macros:  #1 : #2 |--> #3
\newcommand{\elfunction}[3]{ #1  : #2 \mapsto #3 }
\newcommand{\elfn}[3]{\elfunction{#1}{#2}{#3}}

%Functional dependencies:  #1 --> #2
\newcommand{\fd}[2]{ #1 \rightarrow #2 }

%Afunctional dependency:   #1 -/-> #2
\newcommand{\afd}[2]{ #1 \not \! \not \, \rightarrow #2 }

%Elementwise functional dependencies:  #1 |--> #2
\newcommand{\elfd}[2]{ #1 \mapsto #2 }

%Embedding function: #1 : #2 c--> #3
\newcommand{\ifunction}[3]{ #1 \, : \, #2 \hookrightarrow #3 }

%Embedding functional dependencies:  #1 c--> #2
\newcommand{\ifd}[2]{ #1 \hookrightarrow #2 }

%Assignment of value #1 <-- #2. 
 \newcommand{\assignment}[2]{#1 \leftarrow #2}

%%%%%%%%%%SETS%%%%%%%%%%

%Powerset macro.
%\newcommand{\powerset}[1]{{\displaystyle\cal P}(#1)}
 \ifdefinedfix{powerset}{[1]}{{\displaystyle\cal P}(#1)}

%Set-of macro.
\newcommand{\setdef}[2]{\{#1 \;|\; #2\}}

%Set brackets macro.
\newcommand{\setbr}[1]{\{#1\}}

% Cardinality.
 \newcommand{\card}{{\sf Card}}

%%%%%%%%%%OPERATIONS%%%%%%%%%%

%Unions.
%\newcommand{\union}{\cup}
 \ifdefinedfix{union}{}{\cup}
\newcommand{\bigunion}{\bigcup}

%Intersections.
%\newcommand{\intersect}{\cap}
 \ifdefinedfix{intersect}{}{\cap}
\newcommand{\bigintersect}{\bigcap}

%Joins.
%\newcommand{\join}{\vee}
 \ifdefinedfix{join}{}{\vee}
\newcommand{\bigjoin}{\bigvee}

%Meets.
\newcommand{\meet}{\wedge}
\newcommand{\bigmeet}{\bigwedge}

%Complement in Boolean Algebra.
%\newcommand{\complement}[1]{{\sf c}(#1)}
 \ifdefinedfix{complement}{[1]}{{\sf c}(#1)}

%Alternate complement for center of lattice.
\newcommand{\cencomp}[1]{#1^\ast}

%Isomorphic relation.
\newcommand{\isomorphic}{\cong}

%Composition.
\newcommand{\comp}{\circ}

%Supremum in parentheses.
\newcommand{\msup}[1]{\sup (#1)}

%Infimum in parentheses.
\newcommand{\minf}[1]{\inf (#1)}

%Kernel in parentheses.
\newcommand{\mker}[1]{\ker (#1)}

%Specially spaced operations.
\newcommand{\narroweq}{\! = \!}
\newcommand{\narrowneq}{\! \neq \!}

%angle brackets around the argument <#1>.
\newcommand{\abr}[1]{\langle #1 \rangle}



%%%%%%%%%%LOGIC%%%%%%%%%%%

%Redefinition of connectives
\newcommand{\mland}{{\scriptstyle \land}}
\newcommand{\mbigland}{{\scriptstyle \bigwedge}}
\newcommand{\mlandsp}{\;{\scriptstyle \land}\;}
\newcommand{\mlor}{{\scriptstyle \lor}}
\newcommand{\mbiglor}{{\scriptstyle \bigvee}}
\newcommand{\mlorsp}{\;{\scriptstyle \lor}\;}
\newcommand{\mlnot}{\lnot}
 \newcommand{\mimplies}{\Rightarrow}
 \ifdefinedfix{implies}{}{\mimplies}
%\newcommand{\implies}{\mimplies}
\newcommand{\miff}{\Leftrightarrow}
\newcommand{\syntails}{\vdash}
\newcommand{\sentails}{\;{\scriptstyle \models}\;}
\newcommand{\notsentails}{\;{\scriptstyle \not\models}\;}
 \newcommand{\compl}[1]{\overline{#1}}
% \newcommand{\true}{{\bf true}}
 \ifdefinedfix{true}{}{{\bf true}}
% \newcommand{\false}{{\bf false}}
 \ifdefinedfix{false}{}{{\bf false}}

%The language denoted by a symbol, and its components
\newcommand{\lang}[1]{\cat #1}
\newcommand{\lvar}[1]{{\sf Var}(\lang{#1})}
\newcommand{\lpred}[1]{{\sf Pred}(\lang{#1})}
\newcommand{\lpredn}[2]{{\sf Pred}_{#2}(\lang{#1})}
\newcommand{\lcsymb}[1]{{\sf CS}(\lang{#1})}
\newcommand{\predarity}[1]{{\sf Ar}(#1)}
%\newcommand{\ar}[1]{\predarity{#1}} %An abbreviation.
 \ifdefinedfix{ar}{[1]}{\predarity{#1}} %An abbreviation.
\newcommand{\lterms}[1]{{\sf Terms}({\lang #1})}
\newcommand{\pterms}{{\sf Terms}}
\newcommand{\lwffs}[1]{{\sf Wff}({\lang #1})}
\newcommand{\pwffs}{{\sf Wff}}
\newcommand{\domain}[1]{{\cat D}(#1)}
\newcommand{\relofstruct}[2]{#1^{#2}}
\newcommand{\funcofstruct}[2]{#1^{#2}}
\newcommand{\struct}[1]{{\sf Struc}(\lang{#1})}
\newcommand{\sentence}[1]{{\sf Sen}(\lang{#1})}
\newcommand{\termfn}[2]{{\sf Term}(#1,#2)}
\newcommand{\wfffn}[2]{{\sf Wff}(#1,#2)}
\newcommand{\subst}[3]{{\sf Subst}(#1,#2\rightarrow#3)}
\newcommand{\modelsof}[1]{{\sf Mod}(#1)}
\newcommand{\fmodelsof}[1]{{\sf Mod}_f(#1)}
\newcommand{\theoryof}[1]{{\sf Th}(#1)}
\newcommand{\alttheoryof}[1]{#1^+}
\newcommand{\eequiv}{\equiv_{\scriptscriptstyle ee}}
\newcommand{\karyinterp}[3]{{}^{#2}#1^{#3}}
\newcommand{\reductof}[3]{{\sf Reduct}(\lang{#1},#2,#3)}
\newcommand{\reductoftwo}[2]{{\sf Reduct}(\lang{#1},#2)}
\newcommand{\expansof}[3]{{\sf Exp}(\lang{#1},#2,#3)}
\newcommand{\expansoftwo}[2]{{\sf Exp}(\lang{#1},#2)}

%The language defined by a schema or whatever: Lang(-).
 \newcommand{\langof}[1]{{\sf Lang}(#1)}

%%%%%%%%%%CATEGORY THEORY%%%%%%%%%%

%Notation for a category.
\newcommand{\cat}[1]{{\cal #1}}

%The objects of a category.
\newcommand{\obj}[1]{{\sf obj}(\cat{#1})}
\newcommand{\objpure}[1]{{\sf obj}(#1)}

%The morphisms of a category.
\newcommand{\mor}[1]{{\sf mor}(\cat{#1})}
\newcommand{\morpure}[1]{{\sf mor}(#1)}

%The morphism set relative to a pair of objects.
\newcommand{\morset}[3]{\cat{#1}(#2,#3)}
\newcommand{\morsetpure}[3]{#1(#2,#3)}

%The category of sets and functions.
\newcommand{\catset}{{\bf Set}}

%The category of sets and relations.
\newcommand{\catrel}{{\bf Rel}}

%Notation for a functor.
\newcommand{\func}[1]{{\cal #1}}

%Identity morphism, subscripted.
\newcommand{\sidmor}[1]{{\bf 1}_{#1}}

%Image factorization system.
 \newcommand{\imfsys}{({\cal E},{\cal M})}


%%%%%%%%%%SPECIAL OPERATORS%%%%%%%%%%

 \newcommand{\bemph}[1]{{\bf #1}}
 \newcommand{\sfemph}[1]{{\sf #1}}
 %Section sign \S for math mode symbol.
 \mathchardef\Smath="0278

 \newcommand{\fixpoint}{{\sf Fixpoint}}
