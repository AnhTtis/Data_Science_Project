%%%%%%%%%%%%%%%%%%%%%%%%%%%%%%%%%%%%%%%%%%%%%%%%%%%%%%%%%%%%%%%%%%%%%%%%%%%
%%%%%%%%%%%%%%%%%%%%%%%%%%%%%%%%%%%%%%%%%%%%%%%%%%%%%%%%%%%%%%%%%%%%%%%%%%%
%%% Inference rules for binary predicates
%%% Stephen J. Hegner
%%% Section 5
%%% 29 July 2021
%%%%%%%%%%%%%%%%%%%%%%%%%%%%%%%%%%%%%%%%%%%%%%%%%%%%%%%%%%%%%%%%%%%%%%%%%%%
%%%%%%%%%%%%%%%%%%%%%%%%%%%%%%%%%%%%%%%%%%%%%%%%%%%%%%%%%%%%%%%%%%%%%%%%%%%



 \section{Negation in the Armstrong Context}\label{sec:negarm}

%%% \section{Transformation of Proof Rules via Swapping}\label{sec:swap}

     In the remainder of this paper, the goal is to show how to
augment the inference system $\bininfpos{G}$ of \envref{def:bininfpos}
to include all assertions in $\allbinconstrgrsch{G}$; that is, to
include those in $\nbinconstrgrsch{G}$ as well as those of
$\pbinconstrgrsch{G}$ which are already covered by $\bininfpos{G}$.
     The approach taken is a general one, not limited to
$\pbinconstrgrsch{G}$.  Rather, it applies to any Armstrong context
$\armctxt{C}$.  There is absolutely no added complexity in taking this
more general approach.  On the contrary, it is easier to develop the
ideas independently of a specific context.  This approach not only
avoids involving irrelevant properties specific to
$\allbinconstrgrsch{G}$, but also opens the door for these results to
be applied in other contexts.  Nevertheless, explicit guidance on how
the general results apply to $\allbinconstrgrsch{G}$ will always be
kept at the forefront.
      \par
     In this section, some essential properties of negation in an
Armstrong Context are developed.  In the next section,
Sec.\ \ref{sec:ninfrules}, the construction of a set of inference
rules which are complete for both positive and negative assertions is
provided.

 \begin{metalabpara}{context}{}
     {Context}\envlabel{ctxt:armctxt}
   Unless stated specifically to the contrary, for the rest of this
paper, $\logicsys{L}$ will denote a general logical system (see
\envref{disc:logicsys}), with $\armctxt{C} \subseteq \wffof{L}$ an
Armstrong context (see \envref{summ:armstrong}).  In other words,
every satisfiable subset of $\armctxt{C}$ will admit an Armstrong
model.
  For readers not interested in any generalization, it suffices to
take $\armctxt{C} = \pbinconstrgrsch{G}$, with $\logicsys{L}$ the
logic which includes all assertions in $\allbinconstrgrsch{G}$.
 \end{metalabpara}


 \begin{metalabpara}{mydefinition}{}
         {Armstrong constraint sets}\envlabel{def:armconstrset}
    It is useful to have an explicit notation for the negations of the
assertions in $\armctxt{C}$.  To this end, for any
 $\Phi \subseteq \armctxt{C}$, define
 $\notset{\Phi} = \setdef{\mlnot\varphi}{\varphi \in \Phi}$,
 and define
  $\allarm{\armctxt{C}} = \armctxt{C} \union \notset{\armctxt{C}}$.
  For the specific case of $\armctxt{C}=\pbinconstrgrsch{G}$, it is
immediate that
 $\notset{\pbinconstrgrsch{G}} = \nbinconstrgrsch{G}$
 and
 $\allarm{\pbinconstrgrsch{G}} = \allbinconstrgrsch{G}$.
   \par
    An \emph{extended Armstrong constraint set} over $\armctxt{C}$ is
is any satisfiable subset of $\allarm{\armctxt{C}}$.  In
particular,for $\armctxt{C}=\pbinconstrgrsch{G}$, an extended
Armstrong constraint set is a subset of $\allbinconstrgrsch{G}$.
    \par
    For $\Phi \subseteq \allarm{\armctxt{C}}$, define
  $\posof{\Phi} = \Phi \intersect \armctxt{C}$, and define
  $\negof{\Phi} = \Phi \intersect \notset{\armctxt{C}}$.
  Thus, if $\armctxt{C}=\pbinconstrgrsch{G}$, then
 $\posof{\Phi} = \Phi \intersect \pbinconstrgrsch{G}$
 and
 $\negof{\Phi} = \Phi \intersect \nbinconstrgrsch{G}$.
 \end{metalabpara}


 \begin{metalabpara}{mydefinition}{}
         {Equivalent and complementary pairs}\envlabel{def:cfree}
    Two WFFs may be equivalent without being identical.  For example,
in $\pbinconstrgrsch{G}$, $\subrulep{\botgrsch{G}}{g_1}$ and
$\subrulep{g_2}{\topgrsch{G}}$ are equivalent for any
 $g_1, g_2 \in \granulesofsch{G}$, since both are tautologies and so
true in any granular structure.
    More generally, $\varphi_1, \varphi_2 \in \wffof{L}$ are
\emph{equivalent} if they have the same models.  In this case, write
$\equivwff{\varphi_1}{\varphi_2}$,
 and say that $\setbr{\varphi_1,\varphi_2}$ forms an \emph{equivalent
pair}.
    \par
   A two-element set $\setbr{\varphi_1,\varphi_2} \subseteq \wffof{L}$
forms a \emph{complementary pair} if
$\setbr{\varphi_1,\mlnot\varphi_2}$ forms an equivalent pair, with
$\varphi_1$ and $\varphi_2$ called \emph{complements} of one another.
 \end{metalabpara}


 \begin{metalabpara}{myexample}{Example}
         {Complementary pairs in $\pbinconstrgrsch{G}$}\envlabel{ex:cpair}
    For any $g_1, g_2 \in \granulesofschnb{G}$,
 $\setbr{\subrulep{\botgrsch{G}}{g_1},\disjrulep{g_2}{g_2}}$
 forms a complementary pair of elements in
 $\armctxt{C}=\pbinconstrgrsch{G}$,
 with $\subrulep{\botgrsch{G}}{g_1}$ a tautology and
 $\disjrulep{g_2}{g_2}$ unsatisfiable.
 \end{metalabpara}
    \parvert

    It holds generally that a complementary pair in $\armctxt{C}$ must
consist of a tautology and an an unsatisfiable assertion.  Central to
proving this is to note that the empty set admits an Armstrong model,
as established below.


 \begin{metaemphlabpara}{lemma}{Lemma}
       {Armstrong model for $\emptyset$}\envlabel{lem:armempty}
  In the context identified in \envref{ctxt:armctxt}, the empty set
$\emptyset \subseteq \armctxt{C}$ admits an Armstrong model.
   \begin{proof}
   $\logicsys{L}$ is assumed to be nontrivial in the sense that there
is a $\varphi \in \wffof{L}$ which admits a model (see
\envref{disc:logicsys}).
   Since any model of $\varphi$ is also a model of $\emptyset$, it
follows that the empty set is satisfiable.  Since $\armctxt{C}$ is an
Armstrong context and $\emptyset \subseteq \armctxt{C}$, $\emptyset$
must have an Armstrong model, as required.
   \end{proof}
 \end{metaemphlabpara}


 \begin{metaemphlabpara}{lemma}{Lemma}
   {Complementary pairs in the Armstrong context}\envlabel{lem:cparm}
  Let $\setbr{\varphi_1,\varphi_2} \subseteq \armctxt{C}$.  If
$\setbr{\varphi_1,\varphi_2}$ is a complementary pair, then one must
be a tautology, with the other unsatisfiable.
   \begin{proof}
     Let $M_{\emptyset}$ be an Armstrong model for $\emptyset$, which
must exist in view of \envref{lem:armempty}, and let
 $\setbr{\varphi_1,\varphi_2} \subseteq \armctxt{C}$ be a
complementary pair.  $M_{\emptyset}$ is a model of $\varphi_1$ iff it
is a tautology (just by definition of tautology; see,
\mycite[Def.\ 8.22]{Monk76} or \mycite[p.\ 23]{Enderton01_book}), and
similarly for $\varphi_2$.  Thus, if neither is a tautology, then
$M_{\emptyset}$ is a model of neither, and so it must be a model of
both $\mlnot\varphi_1$ and $\mlnot\varphi_2$, which is impossible,
since they are complements of one another.  Thus, one of $\varphi_1$
or $\varphi_2$ must be a tautology, completing the proof.
   \end{proof}
 \end{metaemphlabpara}

  \parvert
    The characterization of overlap of $\armctxt{C}$ and
$\notset{\armctxt{C}}$ now follows easily.


 \begin{metaemphlabpara}{proposition}{Proposition}
   {Overlap of $\armctxt{C}$ and $\notset{\armctxt{C}}$ in the
Armstrong context}\envlabel{prop:ovlparm}
  If $\varphi_1 \in \armctxt{C}$ and
  $\varphi_2 \in \notset{\armctxt{C}}$ with
$\equivwff{\varphi_1}{\varphi_2}$, then $\varphi_1$ must be either a
tautology or else unsatisfiable.
   \begin{proof}
 Assume that $\varphi_1 \in \armctxt{C}$,
  $\varphi_2 \in \notset{\armctxt{C}}$ with
$\equivwff{\varphi_1}{\varphi_2}$.  Then $\varphi_2$ is of the form
$\mlnot\varphi_3$ for some $\varphi_3 \in \armctxt{C}$, whence
 $\equivwff{\varphi_1}{\mlnot\varphi_3}$.
 It then follows that $\setbr{\varphi_1,\varphi_3}$ is a complementary
pair, whence by \envref{lem:cparm}, one of $\varphi_1$, $\varphi_3$
must be a tautology, with the other unsatisfiable.
   \end{proof}
 \end{metaemphlabpara}

  \parvert
   With the preceding results in hand, the focus is now turned to the
more general topic of incorporating negation in inference.  Although
the idea of contraposition is well-known in logic, it is worthwhile to
recall it precisely here, since the idea is central to results of this
section.


 \begin{metaemphlabpara}{lemma}{Lemma}
       {General swapping of premise and conclusion}\envlabel{lem:genswap}
     Let $\wffset{A} \subseteq \wffof{L}$ be closed under logical
negation; \ie, $\notset{\wffset{A}} \subseteq \wffset{A}$.
(Note that this holds in particular for
  $\wffset{A} = \allarm{\armctxt{C}}$.)
 In addition, let $\Phi \subseteq \wffset{A}$ be finite with
 $\varphi, \psi \in \wffset{A}$,
  \baxblkc
    \axitem{(a)} 
 $\Phi \union \setbr{\varphi} \sentails \psi$
     ~~iff~~
 $\Phi \union \setbr{\mlnot\psi} \sentails \mlnot\varphi$.
    \axitem{(b)} If $\Phi \union \setbr{\varphi} \sentails \psi$
 and $\Phi\union\setbr{\varphi}$ is satisfiable, with
$\varphi$ is essential in the sense that
   $\Phi \not\sentails \psi$,
 then $\Phi \union \setbr{\mlnot\varphi}$ is also satisfiable.
  \eaxblk
   In particular, taking $\psi$ to be unsatisfiable, the following
special cases are obtained.
   \baxblkc
     \axitem{(a$'$)} 
 $\Phi \union \setbr{\varphi}$ is unsatisfiable
     ~~iff~~
 $\Phi \sentails \mlnot\varphi$,
     \axitem{(b$'$)} If  $\Phi \union \setbr{\varphi}$ is unsatisfiable
and $\varphi$ is essential to that unsatisfiability in the sense that 
   $\Phi$ is satisfiable, then $\Phi \union \setbr{\mlnot\varphi}$ is
also satisfiable.
  \eaxblk
   \begin{proof}
   Part (a) is just \emph{contraposition}
\mycite[Cor.\ 24D]{Enderton01_book}.
  For part (b), since $\Phi \not\sentails \psi$, there is a model $M$
of $\Phi \union \setbr{\mlnot\psi}$, which must also be a model of
$\mlnot\varphi$ by (a).  Hence $M$ is also a model of
 $\Phi \union \setbr{\mlnot\varphi}$.
   Parts (a$'$) and (b$'$) follow by taking $\psi$ to be the identically
false assertion $\false$ in (a) and (b).
   \end{proof}
 \end{metaemphlabpara}


 \begin{metalabpara}{remark}{Remark}
     {Swapping in the presence of unsatisfiability}\envlabel{rmk:ruleswap}
    In the first part of \envref{lem:genswap}, it is important to note
that if $\Phi \sentails \psi$; that is, if $\varphi$ is superfluous in
$\Phi \union \setbr{\varphi} \sentails \psi$, then
 $\Phi \union \setbr{\mlnot\psi}$ must be unsatisfiable, so that
 $\Phi \union \setbr{\mlnot\psi} \sentails \mlnot\varphi$
 holds in a rather trivial way, as $\false \sentails \mlnot\varphi$.
 Just take $\varphi$ to be the identically true assertion $\true$ to
see this.
    \par
    Similarly, in the second part, if $\Phi$ is unsatisfiable, so that
 $\Phi \union \setbr{\varphi}$ is unsatisfiable regardless of $\varphi$,
 then $\Phi \sentails \mlnot\varphi$ holds again in a rather trivial
way, since false implies anything.
    \par
    The point is that while the result holds in general, it is, for
the most part, interesting only in the case that $\varphi$ is
essential on the left-hand side for the entailment to hold.
 \end{metalabpara}

   \parvert
     When applied within an Armstrong context, the general idea of
swapping premise and conclusion leads to remarkable results.
Specifically, negative premises are of no consequence in deriving
positive conclusions, and in the derivation of a negative conclusion,
at most one negative premise is necessary.  These ideas are formalized
in the next two results.


 \begin{metaemphlabpara}{lemma}{Lemma}
    {Satisfiability and unsatisfiability
             in extended Armstrong constraint sets}\envlabel{lem:suarmstrong}
     Let $\Phi$ be an extended Armstrong constraint set over
$\armctxt{C}$,
 \baxblkc
  \axitem{(a)} If $\Phi$ satisfiable, then any Armstrong model of
$\posof{\Phi}$ relative to $\armctxt{C}$ is also a model of all of
$\Phi$.
  \axitem{(b)} If $\Phi$ is not satisfiable, then either
$\posof{\Phi}$ is unsatisfiable or else there is some
 $\mlnot\varphi \in \negof{\Phi}$ with
 $\posof{\Phi} \union \setbr{\mlnot\varphi}$ unsatisfiable.
 \eaxblk
 \begin{proof}
     For part (a), if $\Phi = \posof{\Phi}$, then there is nothing
further to prove.  Otherwise, let $M$ be an Armstrong model of $\Phi$
relative to $\armctxt{C}$, and let $\mlnot\varphi \in \negof{\Phi}$.
Since $\Phi \union \setbr{\mlnot\varphi}$ is satisfiable, it follows
from \envref{lem:genswap}(a) that $\Phi \not\sentails \varphi$.  Thus,
since $M$ is an Armstrong model of $\Phi$, it cannot be a model of
$\varphi$, whence it is a model of $\mlnot\varphi$.  Since
$\mlnot\varphi$ was chosen arbitrarily from $\negof{\Phi}$, it follows
that $M$ is a model of all of $\Phi$, as required.
   \par
   For part (b), assume that $\Phi$ is not satisfiable.  If
$\posof{\Phi}$ is not satisfiable, there is nothing further to prove.
So, assume that $\posof{\Phi}$ is satisfiable, and let $M$ be an
Armstrong model of $\posof{\Phi}$ relative to $\armctxt{C}$.  Since
$\Phi$ is not satisfiable, there must be some
 $\mlnot\varphi \in \negof{\Phi}$ with
 $M \not\in \modelsof{\mlnot\varphi}$,
 whence $M \in \modelsof{\varphi}$.
 Thus, by the definition of Armstrong model,
 $\posof{\Phi} \sentails \varphi$.
 Finally, applying \envref{lem:genswap}(a$'$),
 $\posof{\Phi} \union \setbr{\mlnot\varphi}$
 is unsatisfiable, completing the proof.
 \end{proof}
 \end{metaemphlabpara}


 \begin{metaemphlabpara}{proposition}{Proposition}
    {Entailment in extended Armstrong constraint sets}\envlabel{prop:entarm}
     Let $\Phi$ be an extended Armstrong constraint set over
$\armctxt{C}$,
 and let $\beta \in \armctxt{C}$.
   \baxblkc
     \axitem{(a)} If $\Phi \sentails \beta$,
  then it must be the case that $\posof{\Phi} \sentails \beta$.
     \axitem{(b)} If $\Phi \sentails \mlnot\beta$,
 then either $\posof{\Phi} \sentails \mlnot\beta$, or else there is
 a $\mlnot\varphi \in \negof{\Phi}$ with the property that
    $\posof{\Phi} \union \setbr{\mlnot\varphi} \sentails \mlnot\beta$
 (and hence
    $\posof{\Phi} \union \setbr{\beta} \sentails \varphi$)
 holds,  with both $\posof{\Phi} \union \setbr{\mlnot\varphi}$ and
   $\posof{\Phi} \union \setbr{\beta}$ satisfiable.
  \eaxblk
  \begin{proof}
    Part (a): If $\Phi$ is not satisfiable, the result holds
trivially.  So, assume that $\Phi$ is satisfiable with
 $\Phi \sentails \beta$, and let $M$ be an Armstrong model for
$\posof{\Phi}$.  By \envref{lem:suarmstrong}(a), $M$ is a model of all
of $\Phi$, and hence a model of $\beta$ as well.  Finally, by the
definition of Armstrong model, the only constraints in $\armctxt{C}$
which hold in $M$ are those implied by $\posof{\Phi}$, whence
 $\posof{\Phi} \sentails \beta$.
     \par
    Part (b): If $\Phi$ is not satisfiable, then the result is
immediate.  So, assume that $\Phi$ is satisfiable with
 $\Phi \sentails \mlnot\beta$.  In view of \envref{lem:genswap}(a$'$),
$\Phi \union \setbr{\beta}$ must be unsatisfiable.  Then, applying
\envref{lem:suarmstrong}(b), it must be the case that either
 $\posof{\Phi}\union\setbr{\beta}$ is unsatisfiable or else there is
$\mlnot\varphi \in \negof{\Phi}$ with
 $\posof{\Phi} \union \setbr{\beta} \union \setbr{\mlnot\varphi}$
 unsatisfiable.
 If $\posof{\Phi}\union\setbr{\beta}$ is unsatisfiable, a second
application of \envref{lem:genswap}(a$'$) yields $\posof{\Phi}
\sentails \mlnot\beta$.
 If $\posof{\Phi}\union\setbr{\beta}$ is satisfiable but there is
$\mlnot\varphi \in \negof{\Phi}$ with
 $\posof{\Phi} \union \setbr{\beta} \union \setbr{\mlnot\varphi}$
 unsatisfiable, an application of \envref{lem:genswap}(a) yields 
 $\posof{\Phi} \union \setbr{\mlnot\varphi} \sentails \mlnot\beta$,
 completing the proof.
  \end{proof}
 \end{metaemphlabpara}


