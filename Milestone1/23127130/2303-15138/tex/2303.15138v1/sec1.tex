%%%%%%%%%%%%%%%%%%%%%%%%%%%%%%%%%%%%%%%%%%%%%%%%%%%%%%%%%%%%%%%%%%%%%%%%%%%
%%%%%%%%%%%%%%%%%%%%%%%%%%%%%%%%%%%%%%%%%%%%%%%%%%%%%%%%%%%%%%%%%%%%%%%%%%%
%%% Notes on multigranular formalism
%%% Stephen J. Hegner
%%% Section 1
%%% 22 October 2018
%%%%%%%%%%%%%%%%%%%%%%%%%%%%%%%%%%%%%%%%%%%%%%%%%%%%%%%%%%%%%%%%%%%%%%%%%%%
%%%%%%%%%%%%%%%%%%%%%%%%%%%%%%%%%%%%%%%%%%%%%%%%%%%%%%%%%%%%%%%%%%%%%%%%%%%



 \section{Introduction}\label{sec:intro}

   In a multigranular relational database system such as
\mgdbsys\ \mycite{HegnerRo17_inform} \mycite{HegnerRo18_dexa}, the
ordinary flat attributes are replaced with ones which admit a
hierarchical structure.  Specifically, a \emph{constrained granulated
attribute schema}, or \emph{CGAS} includes a set $\granules{\gsn}$,
called the \emph{granules} of $\gsn$, together with a set of
constraints $\constr{\gsn}$ which relate the granules to one another.
The knowledge model is \emph{open world} in the sense that an
assertion $\varphi$ may be true
  (if $\constr{\gsn} \sentails \varphi$),
false
  (if $\constr{\gsn} \sentails \mlnot\varphi$),
or unknown
  (if neither $\constr{\gsn} \sentails \varphi$ nor
              $\constr{\gsn} \sentails \mlnot\varphi$)
holds.
    Although it can be shown that inference is decidable, if the
nature of the constraints is not restricted carefully, it becomes
highly intractable.  To address this issue, the constraints which are
permitted are required to lie in one of two groups.  The first group,
which is the main topic of this paper, consists of the binary forms of
subsumption 
 $\subrulep{\granvar{g}_1}{\granvar{g}_2}$
 (equivalent to containment, both proper and improper, in the
underlying semantics)
 as well as of disjointness
 $\disjrulep{\granvar{g}_1}{\granvar{g}_2}$,
 together with their negations
 $\nsubrulep{\granvar{g}_1}{\granvar{g}_2}$
 and
 $\ndisjrulep{\granvar{g}_1}{\granvar{g}_2}$.
    In the context of geographic entities of Chile, simple examples of
positive binary statements include the statements that the province of
Llanquihue lies within the region of Los Lagos, and that the provinces
of Llanquihue and Osorno are disjoint, represented in
(\ref{sec:intro}-1) below.
    \[
      \tag{\ref{sec:intro}-1}
      \subrulep{\llanquihuep}{\loslagosr} \qquad
      \disjrulep{\llanquihuep}{\osornop} 
    \] 
 Similarly, simple examples of negative binary statements include that
Puyehue National Park is neither contained in nor disjoint from the
region of Los Lagos, represented in (\ref{sec:intro}-2) below.
    \[
      \tag{\ref{sec:intro}-2}
      \nsubrulep{\puyehuenpk}{\loslagosr} \qquad
      \ndisjrulep{\puyehuenpk}{\loslagosr} 
    \] 

      The second group consists of assertions which express that a
single granule is either subsumed by or else equal to the join ( or
union) of a (finite) set of other granules.
    A typical example, shown in (\ref{sec:intro}-3) below, asserts
that the region of Los Lagos is the disjoint union of its constituent
provinces.
  \[
   \tag{\ref{sec:intro}-3}
   \loslagosr =
           \biggrlatjoinddisp{}{}
        \setbr{\osornop, \llanquihuep, \chiloep, \palenap}
  \]
     Management of complexity of inference for this type of constraint
requires that the granules be grouped into \emph{granularities}, and
that such assertions be \emph{bigranular}.  For details on this,
including the meaning of bigranular, the reader is referred to
\mycite{HegnerRo18_dexa}.  Despite their importance, such assertions
will not be considered further in this paper.
   \par
      The focus of this paper is the first group of constraints; more
precisely, to provide a consistent and complete inference system for
the two forms of positive binary constraints
 $\subrulep{\granvar{g}_1}{\granvar{g}_2}$
 and
 $\disjrulep{\granvar{g}_1}{\granvar{g}_2}$,
 together with the associated negative binary constraints
 $\nsubrulep{\granvar{g}_1}{\granvar{g}_2}$
 and
 $\ndisjrulep{\granvar{g}_1}{\granvar{g}_2}$.
   The construction of this inference system is divided into two
steps.  In the first step, an inference system for positive binary
constraints only is developed, and in the second, that system is
extended to include negative binary constraints.
   \par
    For the first step, the inference rules shown in
Figs.\ \ref{fig:infrulespmain} and \ref{fig:infrulesptaut}, taken
together, are shown to be \emph{sat-conditionally complete}; that is,
they are complete when applied to a consistent set of constraints.  In
these rules, $\granvar{g}$, $\granvar{g}_1$, $\granvar{g}_2$,
$\granvar{g}_3$, $\granvar{g}_1'$, and $\granvar{g}_2'$ are variables
which may take on the value of any granule, while $\bot$ and $\top$
represent the special bottom (least) and top (greatest) granules.
  \begin{figure}[htb]
  \belowdisplayskip=0em
  \begin{align*}
    \begin{prooftreem}
       \hypoii{\subrulep{\granvar{g}_1}{\granvar{g}_2}}
              {\subrulep{\granvar{g}_2}{\granvar{g}_3}}
       \inferi{\subrulep{\granvar{g}_1}{\granvar{g}_3}}
    \end{prooftreem}
    &&
    \begin{prooftreem}
      \hypoii{\subrulep{\granvar{g}_1}{\granvar{g}_1'}}
              {\disjrulep{\granvar{g}_1'}{\granvar{g}_2}}
      \inferi{\disjrulep{\granvar{g}_1}{\granvar{g}_2}}
    \end{prooftreem}
    \end{align*}
  \caption{Main inference rules for positive binary constraints}\label{fig:infrulespmain}
  \end{figure}

  \begin{figure}[htb]
  \belowdisplayskip=0em
    \begin{align*}
    \begin{prooftreem}
      \hypoi{\phantom{X}}
      \inferi{\subrulep{\granvar{g}}{\granvar{g}}}
    \end{prooftreem}
    &&
    \begin{prooftreem}
      \hypoi{\phantom{X}}
      \inferi{\subrulep{\granvar{g}}{\top}}
    \end{prooftreem}
    &&
    \begin{prooftreem}
      \hypoi{\phantom{X}}
      \inferi{\subrulep{\bot}{\granvar{g}}}
    \end{prooftreem}
    &&
    \begin{prooftreem}
      \hypoi{\phantom{X}}
      \inferi{\disjrulep{\bot}{\granvar{g}}}
    \end{prooftreem}
   \end{align*}
  \caption{Tautological inference rules for positive binary constraints}\label{fig:infrulesptaut}
  \end{figure}

    The rules of Fig.\ \ref{fig:infrulespmain} are the main ones ---
asserting respectively that subsumption is closed under transitivity,
and that if two granules are disjoint, so too are any of their
subgranules under subsumption.
    The four rules of Fig.\ \ref{fig:infrulesptaut} represent
tautologies --- statements which are always true.  The conclusion
holds without any premises.  Although these tautological rules may
seem trivial, they are necessary to ensure complete sat-conditional
inference.
     \par
    In order to derive unsatisfiability of a set of positive
assertions, the rule of Fig.\ \ref{fig:infrulespunsat} must be
included as well.  It requires that no granule, other than $\bot$, may
have empty semantics, so such a granule may never be disjoint from
itself.
  \begin{figure}[htb]
  \belowdisplayskip=0em
    \begin{align*}
    \begin{prooftreem}
      \hypoi{\disjrulep{\granvar{g}}{\granvar{g}}}
      \inferi{\false}
    \end{prooftreem}
    {\scriptstyle\abr{|{\scriptscriptstyle(\granvar{g} \neq \bot)}}}
   \end{align*}
  \caption{Unsatisfiability inference rule on positive binary constraints}\label{fig:infrulespunsat}
  \end{figure}

    This first part of the paper is based substantially upon the
earlier work reported in \mycite{AtzeniPa88_dke}, which provides
overlapping but not identical rules in a context in which neither
$\bot$ nor $\top$ exist as special granules, and in which any granule
may have empty semantics.  Since these differences are substantial,
the full development of this inference system for positive assertions
requires considerable development.
    \par
     In the second part of this paper, it is shown that a consistent
and complete set of inference rules for both positive and negative
constraints may be obtained by augmenting the rules for positive
constraints with \emph{swapped} versions, in which a single antecedent
is swapped with the consequent, and each negated.  Shown in
Fig.\ \ref{fig:infrulespmainswap} are the main rules so obtained from
Fig.\ \ref{fig:infrulespmain}.
  \begin{figure}[htb]
   \begin{align*}
    &\begin{prooftreem}
      \hypoi{\subrulep{\granvar{g}_1}{\granvar{g}_2} \hspace*{1cm}
             \nsubrulep{\granvar{g}_1}{\granvar{g}_3}}
      \inferi{\nsubrulep{\granvar{g}_2}{\granvar{g}_3}}
      \end{prooftreem} %\hspace*{2em}
   &&\begin{prooftreem}
      \hypoi{\subrulep{\granvar{g}_2}{\granvar{g}_3} \hspace*{1cm}
             \nsubrulep{\granvar{g}_1}{\granvar{g}_3}}
      \inferi{\nsubrulep{\granvar{g}_1}{\granvar{g}_2}}
    \end{prooftreem} % \hspace*{2em}
   &&&\begin{prooftreem}
     \hypoii{\subrulep{\granvar{g}_1}{\granvar{g}_1'}}
             {\ndisjrulep{\granvar{g}_11}{\granvar{g}_2}}
     \inferi{\nsubrulep{\granvar{g}_1'}{\granvar{g}_2'}}
    \end{prooftreem}
   \end{align*}
  \caption{Swapped main inference rules for binary constraints}\label{fig:infrulespmainswap}
  \end{figure}
     Similarly, Fig.\ \ref{fig:infrulesntaut} shows the swapped
versions of the rules of Fig.\ \ref{fig:infrulespunsat}.  Informally
speaking, the $\false$ conclusion becomes a $\true$ premise under a
swap, which is trivial and so may be omitted.  The resulting rule thus
expresses a tautology.
     Together, the rules of
Figs.\ \ref{fig:infrulespmain}--\ref{fig:infrulesptaut}
  and
     \ref{fig:infrulespmainswap}--\ref{fig:infrulesntaut}
 provide a sat-conditionally complete inference system for all binary
constraints, positive and negative.
     (For technical reasons to be explained later, the tautological
rules of Fig.\ \ref{fig:infrulesptaut} do not need to be swapped.)
  \begin{figure}[htb]
  \belowdisplayskip=0em
    \begin{align*}
    \begin{prooftreem}
      \hypoi{\phantom{X}}
      \inferi{\ndisjrulep{\granvar{g}}{\granvar{g}}}
    \end{prooftreem}
    {\scriptstyle\abr{|{\scriptscriptstyle(\granvar{g} \neq \bot)}}}
   \end{align*}
  \caption{Tautological inference rule for negative binary constraints}\label{fig:infrulesntaut}
  \end{figure}

     \par
     In order to derive unsatisfiability for a mix of negative and
positive constraints, a simple augmentation of
Figs.\ \ref{fig:infrulespmain}--\ref{fig:infrulesntaut} suffices.
Namely, for every positive assertion $\alpha$, a rule which states
that $\alpha$ and $\mlnot\alpha$ cannot both hold is added.
     Since there are only two types of assertions in the binary logic
of granules, the two rules of Fig.\ \ref{fig:infrulesunsatpr}
suffice.

  \begin{figure}[htb]
   \begin{align*}
    \begin{prooftreem}
      \hypoi{\subrulep{\granvar{g}_1}{\granvar{g}_2} \hspace*{1cm}
             \nsubrulep{\granvar{g}_1}{\granvar{g}_2}}
      \inferi{\false}
    \end{prooftreem}
   &&
   \begin{prooftreem}
     \hypoi{\disjrulep{\granvar{g}_1}{\granvar{g}_2} \hspace*{1cm}
            \ndisjrulep{\granvar{g}_1'}{\granvar{g}_2'}}
     \inferi{\false}
    \end{prooftreem}
   \end{align*}
  \caption{Unsatisfiable-pair rules for binary constraints}\label{fig:infrulesunsatpr}
  \end{figure}

    The combined rules given in
Figs.\ \ref{fig:infrulespmain}--\ref{fig:infrulesunsatpr} are
consistent and complete for inference on all binary constraints,
positive and negative.
    \par
    An important aspect of the augmentation procedure which is
developed is that it is not limited to binary constraints of granule
logic.  Rather, it applies to any \emph{Armstrong context} in which
the positive constraints satisfy certain properties.  This opens the
door to the application of these results to other contexts.
    \par
    This report is organized as follows.  Sec.\ \ref{sec:setting}
provides a summary of granule spaces, which retain the essential
features of CGASs which are necessary for this paper, while
eliminating the unnecessary details.
     In Sec.\ \ref{sec:proofsys}, essential ideas for proof systems
are reviewed and/or developed.
     In Sec.\ \ref{sec:pinfrules}, the consistency and completeness of
the binary inference rules of
Figs.\ \ref{fig:infrulespmain}--\ref{fig:infrulespunsat} for positive
binary constraints is established.
     In Sec.\ \ref{sec:negarm}, the semantics of negation in the
setting in an Armstrong setting, which includes the logic of binary
predicates on granules, is developed.
     In Sec.\ \ref{sec:ninfrules}, the results of
Sec.\ \ref{sec:negarm} are used to develop an associated proof system,
establishing that the rules of
Figs.\ \ref{fig:infrulespmainswap}--\ref{fig:infrulesunsatpr} provide
a sufficient augmentation of those of
Figs.\ \ref{fig:infrulespmain}--\ref{fig:infrulespunsat} to obtain a
complete set of inference rules for both positive and negative granule
constraints.
    Sec.\ \ref{sec:lit} contains to information the relationship of
this work to other publications.
    Sec.\ \ref{sec:cfd} contains conclusions and further directions.

    
