%%%%%%%%%%%%%%%%%%%%%%%%%%%%%%%%%%%%%%%%%%%%%%%%%%%%%%%%%%%%%%%%%%%%%%%%%%%
%%%%%%%%%%%%%%%%%%%%%%%%%%%%%%%%%%%%%%%%%%%%%%%%%%%%%%%%%%%%%%%%%%%%%%%%%%%
%%% Notes on multigranular formalism
%%% Stephen J. Hegner
%%% Section Cover
%%% 22 October 2018
%%%%%%%%%%%%%%%%%%%%%%%%%%%%%%%%%%%%%%%%%%%%%%%%%%%%%%%%%%%%%%%%%%%%%%%%%%%
%%%%%%%%%%%%%%%%%%%%%%%%%%%%%%%%%%%%%%%%%%%%%%%%%%%%%%%%%%%%%%%%%%%%%%%%%%%

% \titlerunning{Information-Optimal Updates}
% \authorrunning{S.\ J.\ Hegner}

 \title{Inference Rules for Binary Predicates
                           \\ in a Multigranular Framework}

 \author{
         Stephen J. Hegner \\
         DBMS Research of New Hampshire \\
            PO Box, 2153, New London, NH 03257, USA \\
            {\tt dbmsnh@gmx.com}
         \vspace*{1em} \\
         M. Andrea Rodr{\'{\i}}guez\footnote{
              Author to whom all correspondence regarding this paper
              should be directed.} \\
            Millennium Institute for Foundational Research on Data \\
            Departamento Ingenier{\'i}a Inform{\'a}tica y
            Ciencias de la Computaci{\'o}n \\
            Edmundo Larenas 219,
            Universidad de Concepci{\'o}n \\
            4070409 Concepci{\'o}n, Chile \\
            {\tt andrea@udec.cl}
        }

 \date{}  

  \maketitle

 \thispagestyle{simpleheadingsone}
 \begin{abstract}
     In a multigranular framework, the two most important binary
predicates are those for subsumption and disjointness.  In the first
part of this work, a sound and complete inference system for
assertions using these predicates is developed.  It is customized for
the granular framework; particularly, it models both bottom and top
granules correctly, and it requires all granules other then the bottom
to be nonempty.  Furthermore, it is single use, in the sense that no
assertion is used more than once as an antecedent in a proof.
      \par
    In the second part of this work, a method is developed for
extending a sound and complete inference system on a framework which
admits Armstrong models to one which provides sound and complete
inference on all assertions, both positive and negative.  This method
is then applied to the binary granule predicates, to obtain a sound
and complete inference system for subsumption and disjointness, as
well as their negations.
% \keywordname\ update, view
 \end{abstract}

 \thispagestyle{empty}
 \thispagestyle{simpleheadingsone}

%%%%%%%%%%%%%%%%%%%%%%%%%%%%%%%%%%%%%%%%%%%%%%%%%%%%%%%%%%%%%%%%%%%%%%%%%%
%%%%%%%%%%%%%%%%%%%%%%%%%%%%%%%%%%%%%%%%%%%%%%%%%%%%%%%%%%%%%%%%%%%%%%%%%%

