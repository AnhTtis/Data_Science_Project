%%%%%%%%%%%%%%%%%%%%%%%%%%%%%%%%%%%%%%%%%%%%%%%%%%%%%%%%%%%%%%%%%%%%%%%%%%%
%%%%%%%%%%%%%%%%%%%%%%%%%%%%%%%%%%%%%%%%%%%%%%%%%%%%%%%%%%%%%%%%%%%%%%%%%%%
%%% Inference rules for binary predicates
%%% Stephen J. Hegner
%%% Section 2
%%% 06 february 2021
%%%%%%%%%%%%%%%%%%%%%%%%%%%%%%%%%%%%%%%%%%%%%%%%%%%%%%%%%%%%%%%%%%%%%%%%%%%
%%%%%%%%%%%%%%%%%%%%%%%%%%%%%%%%%%%%%%%%%%%%%%%%%%%%%%%%%%%%%%%%%%%%%%%%%%%


 \section{The Semantics of Granules}\label{sec:setting}

   In the framework reported \mycite{HegnerRo17_inform} and
\mycite{HegnerRo18_dexa}, a central notion is that of
\emph{granularity}, which provides a classification system for the
underlying granules.  For example, relative to the examples of Chilean
spatial entities described in Sec.\ \ref{sec:intro}, three principal
granularities are $\regiong$, $\provinceg$, and $\parkg$, with
$\loslagosr$ a granule of granularity $\regiong$, $\osornop$,
$\llanquihuep$, $\chiloep$, and $\palenap$ granules of granularity
$\provinceg$, and $\puyehuenpk$ a granule of granularity $\parkg$.
From the perspective of a multigranular DBMS, such classification of
granules into granularities is essential not only for the purposes of
organizing concepts, but also for the systematic expression and
tractable inference of constraints of the form (\ref{sec:intro}-3).
    Nevertheless, within the context of inference on binary
constraints of the forms illustrated in
 Figs.\ \ref{fig:infrulespmain}--\ref{fig:infrulesunsatpr},
 they are of secondary importance at best.
    Therefore, in this section, a simplified version of the CGAS
(constrained granulated attribute schema) of \mycite{HegnerRo18_dexa}
is developed, in which granularities are ignored, with only binary
constraints considered.  The result is termed a \emph{simple
monogranular attribute schema}, or \emph{SMAS}.
    This approach has the advantage of allowing a much simpler
development and presentation of the main results of this report.  In
addition, it allows this report to be read and understood without
first gaining a thorough understanding of the concepts developed in
\mycite{HegnerRo17_inform} and \mycite{HegnerRo18_dexa}.


 \begin{metalabpara}{definition}{}
          {Granule spaces}\envlabel{def:gransp}
   A \emph{granule space} is a triple $\granspacedef{G}$ in which
$\granulesof{G}$ is a set, called the set of \emph{granules}.  with
$\botgrsp{G},\topgrsp{G} \in \granulesof{G}$.
     $\botgrsp{G}$ is called the \emph{bottom granule} of
$\granspacename{G}$ while $\topgrsp{G}$ is called the \emph{top granule}
of $\granspacename{G}$); they are always distinct.
  The special notation $\granulesofnb{G}$ is used to denote the set
$\granulesof{G} \setminus \setbr{\botgrsp{G}}$, while
$\granulesofnbnt{G}$ is used to denote the set
 $\granulesof{G} \setminus \setbr{\botgrsp{G},\topgrsp{G}}$.
 \end{metalabpara}


 \begin{metalabpara}{mydefinition}{}
     {Granule structures and binary constraints}\envlabel{def:granstrsem}
   A \emph{granule structure} over $\granspacename{G}$ is a pair
$\gnlestrpr{\sigma}$ in which $\gnledom{\sigma}$ is a nonempty set and
  $\fn{\gnletodom{\sigma}}{\granulesof{G}}
                          {\powerset{\gnledom{\sigma}}}$
 is a function which assigns a subset of $\gnledom{\sigma}$ to each
granule, subject to the conditions that
   $\gnletodom{\sigma}(\botgrsp{G}) = \emptyset$,
   $\gnletodom{\sigma}(\topgrsp{G}) = \gnledom{\sigma}$,
 and for every $g \in \granulesofnb{G}$,
   $\gnletodom{\sigma}(g) \neq \emptyset$.
     \par
   The \emph{positive binary constraints} over $\granspacename{G}$ are
of two forms.  For $g_1, g_2 \in \granulesof{G}$, the
\emph{subsumption rule} \mbox{$\subrulep{g_1}{g_2}$} holds in the
granule structure $\sigma$ iff
 \preformat{\linebreak}
 $\gnletodom{\sigma}(g_1) \subseteq \gnletodom{\sigma}(g_2)$.
   Likewise, the \emph{disjointness rule} $\disjrulep{g_1}{g_2}$ holds
in $\sigma$ iff
 $\gnletodom{\sigma}(g_1) \intersect \gnletodom{\sigma}(g_2) = \emptyset$.
    The set of all such positive subsumption rules (resp.\ positive
disjointness rules) over $\granspacename{G}$ is denoted
$\psubconstrgr{G}$ (resp.\ $\pdisjconstrgr{G}$).
  Combining these, the set of all positive binary constraints over
$\granspacename{G}$ is
 \nlrightt
 $\pbinconstrgr{G} = \psubconstrgr{G} \union \pdisjconstrgr{G}$.
    \par
   The \emph{negative binary constraints} over $\granspacename{G}$ are the
negations of the positive constraints.  For
 $g_1, g_2 \in \granulesof{G}$, $\nsubrulep{g_1}{g_2}$ denotes that
$\subrulep{g_1}{g_2}$ does not hold; \ie, that
 $\gnletodom{\sigma}(g_1) \not\subseteq \gnletodom{\sigma}(g_2)$,
 with $\nsubconstrgr{G}$ denoting the set of all such negative
subsumption rules.
   Similarly, $\ndisjrulep{g_1}{g_2}$ denotes that
$\disjrulep{g_1}{g_2}$ does not hold; \ie, that
  $\gnletodom{\sigma}(g_1) \intersect \gnletodom{\sigma}(g_2) \neq \emptyset$.
  The set of all negative binary constraints over $\granspacename{G}$
is $\nbinconstrgr{G} = \nsubconstrgr{G} \union \ndisjconstrgr{G}$,
 with $\ndisjconstrgr{G}$ denoting the set of all such negative
subsumption rules.
   \par
   $\allbinconstrgr{G} = \pbinconstrgr{G} \union \nbinconstrgr{G}$
 is the set of \emph{all binary constraints} over $\granspacename{G}$.
   \par
%   It will also be necessary to work with the logical assertion
%$\true$, which is always true, and $\false$, which is always false.
%In order to regard them as binary constraints, it suffices to
%associate them with statements which are always true or false.  To be
%concrete, regard $\true$ to be a synonym for
%$\subrulep{\botgrsp{G}}{\topgrsp{G}}$ and $\false$ to be a synonym for
%$\subrulep{\topgrsp{G}}{\botgrsp{G}}$.
%   \par
   Following standard logical terminology, the granule structure
$\sigma$ is a \emph{model} of
 $\varphi \in
 \preformat{\linebreak}
 \allbinconstrgr{G}$ if $\varphi$ holds in $\sigma$, and $\sigma$ is a
\emph{model} of $\Phi \subseteq \allbinconstrgr{G}$ if it is a model
of every $\varphi \in \Phi$.  $\varphi$ (resp.\ $\Phi$) is
\emph{satisfiable} if it has at least one model.  The set of all
models of $\varphi$ (resp.\ $\Phi$) is denoted $\modelsof{\varphi}$
(resp.\ $\modelsof{\Phi}$).
   The constraint $\varphi$ (resp.\ the set $\Phi$ of constraints) is
\emph{satisfiable} if
 $\modelsof{\varphi} \neq \emptyset$
 (resp.\ $\modelsof{\Phi} \neq \emptyset$).
    \par
   For $\varphi_1, \varphi_2 \in \allbinconstrgr{G}$,
 $\varphi_1$ \emph{semantically entails} $\varphi_2$, written
 $\varphi_1 \sentails \varphi_2$, just in case
 $\modelsof{\varphi_1} \subseteq \modelsof{\varphi_2}$.
   Likewise, for $\Phi \subseteq \allbinconstrgr{G}$
and $\varphi \in \allbinconstrgrsch{G}$
   (resp.\ $\Phi' \subseteq \allbinconstrgr{G}$),
 write $\Phi \sentails \varphi$ (resp.\ $\Phi \sentails \Phi'$)
 just in case
 $\modelsof{\Phi} \subseteq \modelsof{\varphi}$.
 (resp.\  $\modelsof{\Phi} \subseteq \modelsof{\Phi'}$.
 \end{metalabpara}

 \begin{metalabpara}{mydefinition}{}
     {Positive binary tautologies}\envlabel{def:pbintaut}
    In standard logic terminology, a \emph{tautology} is a statement
which is true in every model.  Define the \emph{positive binary
tautologies} of $\granspacename{G}$ as follows.
  \begin{equation*}
  \begin{split}
     \textstyle
    \pbintautgr{G} =\;\phantom{\union}
        &\setdef{\subrulep{g}{g}}{g \in \granulesof{G}} \\
        \union\;
        &\setdef{\subrulep{\botgrsp{G}}{g}}{g \in \granulesof{G}} \\
        \union\;
        &\setdef{\subrulep{g}{\topgrsp{G}}}{g \in \granulesof{G}} \\
        \union\;
        &\setdef{\disjrulept{\botgrsp{G}}{g}}{g \in \granulesof{G}}
  \end{split}
  \end{equation*}
    That $\pbintautgr{G}$ consists of exactly those members of
$\pbinconstrgr{G}$ which are tautologies is established next.
 \end{metalabpara}


 \begin{metaemphlabpara}{proposition}{Proposition}
     {Characterization of positive binary tautologies}\envlabel{prop:pbintaut}
    Let $\granspacename{G}$ be a granule space.  Then $\pbintautgr{G}$
is precisely the subset of $\pbinconstrgr{G}$ consisting of
tautologies.  More precisely, if $\varphi \in \pbinconstrgr{G}$, then
$\varphi \in \pbintautgr{G}$ iff $\sigma \in \modelsof{\varphi}$ for
every granule structure $\sigma$ over $\granspacename{G}$.
 \begin{proof}
   First of all, it is immediate from the definition of constraint
satisfaction in \envref{def:granstrsem} that every
 $\varphi \in \pbintautgr{G}$ is a tautology; that
 $\sigma \in \modelsof{\varphi}$ for every granule structure $\sigma$
over $\granspacename{G}$.
   To show they are the only tautologies, choose any
 $\varphi \in \pbinconstrgr{G} \setminus \pbintautgr{G}$.
 If $\varphi \in \psubconstrgr{G}$, then it must be of the form
$\subrulep{g_1}{g_2}$ with
 $g_1 \neq \botgrsp{G}$, $g_2 \neq \topgrsp{G}$, and
 $g_1 \not\ideq g_2$.
 Define $\gnlestrpr{\sigma'}$ to have
 $\gnledom{\sigma'} = \setbr{x_1,x_2}$ (any two-element set) with
 $\gnletodom{\sigma'}(g_1) = \setbr{x_1,x_2}$ and
 $\gnletodom{\sigma'}(g_2) = \setbr{x_2}$.
 Clearly, $\sigma'$ is a granule structure which is not a model of
$\varphi$, so the latter cannot be a tautology.
 On the other hand, if $\varphi \in \pdisjconstrgr{G}$, then it must
be of the form $\disjrulep{g_1}{g_2}$ with
 $g_1 \neq \botgrsp{G}$ and $g_2 \neq \botgrsp{G}$.
 Then $\sigma'$, as defined above, is not a model of $\varphi$, so
again the latter is not a tautology.
 \end{proof}
 \end{metaemphlabpara}


 \begin{metalabpara}{mydefinition}{}
     {Unsatisfiable positive binary constraints}\envlabel{def:pbinunsat}
   Define the \emph{unsatisfiable positive binary constraints} of
$\granspacename{G}$ as follows.
  \begin{equation*}
  \begin{split}
     \textstyle
    \pbinunsatgr{G} =\;\phantom{\union}
        &\setdef{\subrulep{g}{\botgrsp{G}}}{g \in \granulesofnb{G}} \\
        \union\;
        &\setdef{\disjrulept{g}{g}}{g \in \granulesofnb{G}}
  \end{split}
  \end{equation*}
    That $\pbinunsatgr{G}$ consists of exactly those members of
$\pbinconstrgr{G}$ which are unsatisfiable is established next.
 \end{metalabpara}


 \begin{metaemphlabpara}{proposition}{Proposition}
     {Characterization of unsatisfiable positive binary constraints}\envlabel{prop:pbinunsat}
    Let $\granspacename{G}$ be a granule space.  Then
$\pbinunsatgr{G}$ is precisely the subset of $\pbinconstrgr{G}$
consisting of unsatisfiable assertions.  More precisely, if
 $\varphi \in \pbinconstrgr{G}$, then $\varphi \in \pbinunsatgr{G}$
iff for no granule structure $\sigma$ over $\granspacename{G}$ is it
the case that $\sigma \in \modelsof{\varphi}$.
 \begin{proof}
   First of all, it is immediate from the definition of constraint
satisfaction in \envref{def:granstrsem} that every
 $\varphi \in \pbinunsatgr{G}$ is unsatisfiable; that for no granule
structure $\sigma$ over $\granspacename{G}$ is it the case that
 $\sigma \in \modelsof{\varphi}$ .
   To show they are the only unsatisfiable members of
$\pbinconstrgr{G}$, choose any
 $\varphi \in \pbinconstrgr{G} \setminus \pbinunsatgr{G}$.
 If $\varphi \in \psubconstrgr{G}$, then it must be of the form
$\subrulep{g_1}{g_2}$ with $g_2 \neq \botgrsp{G}$.
 If $g_1 = \botgrsp{G}$, then $\varphi$ is a tautology, and so
trivially satisfiable.
 Otherwise, define $\gnlestrpr{\sigma'}$ to have
 $\gnledom{\sigma'} = \setbr{x_1,x_2}$ (any two-element set) with
 $\gnletodom{\sigma'}(g_1) = \setbr{x_1}$.
 and
 $\gnletodom{\sigma'}(g_1) = \setbr{x_1,x_2}$.
 Clearly, $\sigma'$ is a granule structure which is a model of
$\varphi$, so the latter is satisfiable.
 On the other hand, if $\varphi \in \pdisjconstrgr{G}$, then it must
be of the form $\disjrulep{g_1}{g_2}$ with either $g_1 \not\ideq g_2$
or else $g_1 = g_2 = \botgrsp{G}$.
 If $g_1 = g_2 = \botgrsp{G}$, then $\disjrulep{g_1}{g_2}$ is a
tautology and there is nothing further to prove.  Otherwise, define
$\gnlestrpr{\sigma''}$ to have
 $\gnledom{\sigma''} = \setbr{x_1,x_2}$ (any two-element set) with
 $\gnletodom{\sigma'}(g_1) = \setbr{x_1}$ and
 $\gnletodom{\sigma'}(g_2) = \setbr{x_2}$.
 Clearly, $\sigma''$ is a granule structure which is a model of
$\varphi$, rendering it satisfiable.
 \end{proof}
 \end{metaemphlabpara}


 \begin{metalabpara}{definition}{}
     {Simple monogranular attribute schemata}\envlabel{def:smas}
   A \emph{simple monogranular attribute
   \preformat{\linebreak}
   schema}, or \emph{SMAS}, is a
pair $\smasdef{G}$ in which $\granspace{G}$ is a granule space and
 $\constrgrsch{G} \subseteq  \allbinconstr{\granspace{G}}$.
   To understand the concepts of \envref{def:granstrsem} in the
context of this SMAS, it is only necessary to replace
$\granspacename{G}$ with $\granspace{G}$ everywhere.
   However, as they occur frequently, $\botgrsch{G}$
(resp.\ $\topgrsch{G}$) will often be used as synonyms for the more
cumbersome $\botgrschfull{G}$ (resp.\ $\topgrschfull{G}$).
   \par
   Define
   $\pconstrgrsch{G} = \constrgrsch{G} \intersect \pbinconstrgrsch{G}$.
 Thus, $\pconstrgrsch{G}$ is the set of all positive constraints in
$\constrgrsch{G}$.
   Similarly, define
   $\nconstrgrsch{G} = \constrgrsch{G} \intersect \nbinconstrgrsch{G}$.
    \par
   Define the \emph{closure} of $\constrgrsch{G}$ to be
  \nlrightt
    $\clconstrgrsch{G}
       = \setdef{\varphi \in \allbinconstrgrsch{G}}
                {\constrgrsch{G} \sentails \varphi}$.
    \par
   A granular structure $\sigma$ is a \emph{model} of
$\granschemaname{G}$ precisely in the case that it is a model of
$\constrgrsch{G}$.
   Call an SMAS $\granschemaname{G}$ \emph{satisfiable} if
$\constrgrsch{G}$ has that property, and \emph{unsatisfiable}
otherwise.
 \end{metalabpara}


 \begin{metalabpara}{definition}{}
     {Equivalence and identity of granules}\envlabel{def:eqidsep}
     It is important to distinguish two distinct notions of equality
between granules.
     Relative to $\granschemaname{G}$, two granules
 $g_1, g_2 \in \granulesofsch{G}$ are \emph{equivalent} if
  $\constrgrsch{G} \sentails
         \setbr{\subrulep{g_1}{g_2},\subrulep{g_2}{g_1}}$.
   Thus, two granules are equivalent if they have the same value under
every granule structure which is a model.
   On the other hand, the two granules
 $g_1, g_2 \in \granulesofsch{G}$ are \emph{identical} if they are the
same granule; that is, if they have the same name.
   Clearly, two distinct granules may be equivalent without being
identical.
   In order to distinguish these two, the expression
$\eqrulep{g_1}{g_2}$ will always mean that $g_1$ and $g_2$ are
equivalent.
   To express identity, the notation $\idrulep{g_1}{g_2}$ is always
used.
%ote that $\idrulep{g_1}{g_2}$ is actually a bit of a
%ontradiction, since $g_1$ and $g_2$ are already different names.  Of
%ourse, the intent is that they are variables representing granule
%ames.  As such, it is better technically to use granule variables in
%uch expressions, \eg, $\idrulep{\granvar{g}_1}{\granvar{g}_2}$, as
%laborated in \envref{def:granwff}.
    \par
%   An SMAS is separating if it is not possible for distinct granules
%to be equivalent.  Formally, $\granschemaname{G}$ is \emph{separating}
%if at least one of
%  $\constrgrsch{G} \sentails \nsubrulep{g_1}{g_2}$,
%  $\constrgrsch{G} \sentails \nsubrulep{g_2}{g_1}$
% holds for any pair $\setbr{g_1,g_2} \in \granulesofsch{G}$
% of nonidentical granules.
 \end{metalabpara}


 \begin{metalabpara}{convention}{}
     {Implicit symmetry of disjunction}\envlabel{conv:disjsym}
    Throughout this work, for any $g_1, g_2 \in
 \preformat{\linebreak}
 \granulesofsch{G}$, $\disjrulep{g_1}{g_2}$ and $\disjrulep{g_2}{g_1}$
will always be regarded as the same constraint.  Clearly, the
semantics does not depend upon any order on $g_1$ and $g_2$.  This
will eliminate the need for largely trivial inference rules which
would otherwise be required to assure that
 $\disjrulep{g_1}{g_2}$ and $\disjrulep{g_2}{g_1}$,
 as well as
 $\ndisjrulep{g_1}{g_2}$ and $\ndisjrulep{g_2}{g_1}$,
  are the same.
   \par
   It will also permit writing constraints of the form
$\disjrulesetp{S}$ and $\ndisjrulesetp{S}$, in which $S$ is a nonempty
set consisting of at most two granules.  If $S = \setbr{g_1,g_2}$,
then $\disjrulesetp{S}$ represents indifferently both
$\disjrulep{g_1}{g_2}$ and $\disjrulep{g_2}{g_1}$, while if
 $S = \setbr{g}$, then $\disjrulesetp{S}$ represents
$\disjrulep{g}{g}$.
   The meaning of $\ndisjrulesetp{S}$ is analogous.
 \end{metalabpara}


 \begin{metalabpara}{discussion}{}
     {Relationship of SMASs to CGASs}\envlabel{disc:cmascgas}
     This section provides a brief comparison of the SMAS, as defined
in \envref{def:smas} and the CGAS, as defined in
\mycite[Sec.\ 2]{HegnerRo18_dexa} and the earlier \emph{granularity
schema}, as defined in \mycite[Sec.\ 3]{HegnerRo17_inform}.  The
reader who is not familiar with these latter concepts, or who is not
interested in the comparison, may safely skip this discussion.
     \par
    A CGAS $\cgaschdefii{\gsn}$ consists of three components,
   a \emph{granularity poset} $\granposet{\gsn}$,
   a \emph{granule assignment} $\grasgn{\gsn}$,
   and a set $\allconstr{\gsn}$ of constraints.
   Relative to such a CGAS, an SMAS $\smasdef{G}$ differs in three
fundamental ways.
    \begin{axiomsd}{1em}{1em}{0em}
     \axitem{(a)} In an SMAS, there is no granularity poset, since
there is no classification of granules into granularities.
     \axitem{(b)} The granule space $\granspace{G}$ of the SMAS
corresponds roughly to the granule assignment $\grasgn{\gsn}$ of the
CGAS.  However, since there is no granularity hierarchy in an SMAS,
the classification of granules into granularities, which is embedded
in $\grasgn{\gsn}$ in the CGAS, is absent in $\granspace{G}$.
     \axitem{(c)} In a CGAS, some binary constraints, are already
embedded in $\grasgn{\gsn}$, including in particular those which arise
from the classification of granules into granularities.  More
specifically, the granule preorder $\gnleleqb{\gsn}$, as well as the
constraints arising from the requirement that granules of distinct
granularities be disjoint, are already embedded in the granule
assignment.  On the other hand, in a SMAS, no constraints are embedded
in the granule space; constraints are only embedded in
$\constrgrsch{G}$.
%, and only those of $\essconstrgrsch{G}$ are included
%automatically.
     \end{axiomsd}
 \end{metalabpara}

