% \documentclass[preprint]{elsarticle}
% \documentclass[letterpaper, 10 pt, journal, twoside]{IEEEtran}
% \pagestyle{empty}
% \documentclass[journal,twoside,web]{ieeecolor}
\documentclass[10pt,conference]{ieeeconf}


% \usepackage{lineno,hyperref}
% \modulolinenumbers[5]

\usepackage{mathrsfs}
\usepackage{graphics} % for pdf, bitmapped graphics files
% \usepackage{epsfig} % for postscript graphics files
\usepackage{amsmath} % assumes amsmath package installed
\usepackage{amssymb}  % assumes amsmath package installed
\let\proof\relax
\let\endproof\relax
\usepackage{amsthm}

\usepackage{cite}
\usepackage{bm}
% \usepackage{subfigure}
\usepackage{acronym}
\usepackage{paralist}
\usepackage{float}
\usepackage{color}
\usepackage{epstopdf}
\usepackage{multicol}
\usepackage{tikz}
% \usepackage{subcaption}
\usepackage{graphicx}
% \usepackage{caption}
\usepackage{mathtools}
\usepackage{soul}
\usepackage[hang, flushmargin]{footmisc}
\usepackage[colorlinks=true]{hyperref}


\usepackage{amsmath} % assumes amsmath package installed
\usepackage{amssymb}  % assumes amsmath package installed
\usepackage{graphicx}
\usepackage{epstopdf}
\usepackage{amsmath}
\usepackage{mathtools}
\usepackage{dsfont}
\usepackage{tikz}
\usepackage{siunitx}
\usepackage{xcolor}


\newcommand{\pn}[1]{{\color{black} #1}}%red 
\newcommand{\pno}[1]{{\color{black} #1}}%cyan

% \usepackage{mathtools}
\DeclarePairedDelimiter\ceil{\lceil}{\rceil}
\DeclarePairedDelimiter\floor{\lfloor}{\rfloor}


\usepackage{ifthen}
\newboolean{Disturbances}
\setboolean{Disturbances}{false}
\newcommand{\NotDist}[1]{\ifthenelse{\boolean{Disturbances}}{}{{\color{black}#1}}}
\newcommand{\IfDist}[1]{\ifthenelse{\boolean{Disturbances}}{\color{red}#1}{}}  
%
\newboolean{ACCFinal}
\setboolean{ACCFinal}{True}
\newcommand{\NotACC}[1]{\ifthenelse{\boolean{ACCFinal}}{}{{\color{magenta}#1}}}
\newcommand{\IfACC}[2]{\ifthenelse{\boolean{ACCFinal}}{{\color{black}#1}}{{\color{magenta}#2}}}  %magenta 

%
\newboolean{Integrity}
\setboolean{Integrity}{false}
\newcommand{\NotInt}[1]{\ifthenelse{\boolean{Integrity}}{}{{\color{black}#1}}}%blue
\newcommand{\IfInt}[1]{\ifthenelse{\boolean{Integrity}}{{\color{red}#1}}}  

\newcommand{\al}[1]{\textcolor{magenta}{\sf {\bf AC:}  #1}}

\makeatletter
\hypersetup{colorlinks=true}
\AtBeginDocument{\@ifpackageloaded{hyperref}
  {\def\@linkcolor{blue}
  \def\@anchorcolor{red}
  \def\@citecolor{red}
  \def\@filecolor{red}
  \def\@urlcolor{black}
  \def\@menucolor{red}
  \def\@pagecolor{red}
\begingroup
  \@makeother\`%
  \@makeother\=%
  \edef\x{%
    \edef\noexpand\x{%
      \endgroup
      \noexpand\toks@{%
        \catcode 96=\noexpand\the\catcode`\noexpand\`\relax
        \catcode 61=\noexpand\the\catcode`\noexpand\=\relax
      }%
    }%
    \noexpand\x
  }%
\x
\@makeother\`
\@makeother\=
}{}}
\makeatother


\newtheorem{Theorem}{Theorem}

\newtheorem*{Problem}{Problem $(\star)$}

\newtheorem{Remark}{Remark}

\newtheorem{Corollary}{Corollary}

\newtheorem{Lemma}{Lemma}

\newtheorem{Example}{Example}

\newtheorem{Assumption}{Assumption}

\newtheorem{Definition}{Definition}

\newcommand{\Zp}{\mathbb{N}_{0+}}
\newcommand{\Rp}{\mathbb{R}_{0+}}
\newcommand{\Xgoal}{\mathcal{X}_T^k_\textrm{goal}}

\DeclareMathOperator*{\argmax}{arg\,max}
\DeclareMathOperator*{\argmin}{arg\,min}

\IEEEoverridecommandlockouts

\def\BibTeX{{\rm B\kern-.05em{\sc i\kern-.025em b}\kern-.08em
    T\kern-.1667em\lower.7ex\hbox{E}\kern-.125emX}}

\makeatletter
\newcommand{\fixed@sra}{$\vrule height 2\fontdimen22\textfont2 width 0pt\shortrightarrow$}
\newcommand{\shortarrow}[1]{%
  \mathrel{\text{\rotatebox[origin=c]{\numexpr#1*45}{\fixed@sra}}}
}
\makeatother

\begin{document}

\title{\LARGE{\bf An Observer-based Switching Algorithm for Safety \\\pno{under Sensor} \NotInt{Denial-of-Service} Attacks}}
%Safe Recovery from Sensor \NotInt{Denial-of-Service} Attacks
\author{Santiago Jimenez Leudo, Kunal Garg, Ricardo G. Sanfelice and Alvaro A. Cardenas
\thanks{S. J. Leudo and R. G. Sanfelice are with the Department of Electrical and Computer Engineering, and  A. A. Cardenas is with the Department of Computer Science and Engineering,
University of California, Santa Cruz, CA 95064.
K. Garg is with the Department of Aeronautics and Astronautics, Massachusetts Institute of Technology, Cambridge, MA 02139.
      Email: {\tt\small \{sjimen28, ricardo, alacarde\}@ucsc.edu}, \tt\small\{kgarg\}@mit.edu}
\IfACC{ \thanks{ 
\pno{
Research partially supported by the NSF Grants no. ECS-1710621, CNS-2039054, and CNS-2111688, by the AFOSR Grants no. FA9550-19-1-0169, FA9550-20-1-0238, and FA9550-23-1-0145,
 by the AFRL Grant nos. FA8651-22-1-0017 and FA8651-23-1-0004,
by ARO Grant no. W911NF-20-1-0253, and by Fulbright Colombia - MinTIC.
 The views and conclusions of this document are those of the authors and should not be interpreted as representing the official policies of the ARO or the U.S. Government. The U.S. Government is authorized to reproduce and distribute reprints for Government purposes, notwithstanding any copyright notation herein.
 }
% K. Garg is with the Department of Aeronautics and Astronautics, Massachusetts Institute of Technology, Cambridge, MA 02139.
% Email: {\tt\small kgarg@mit.edu}
}}{}
}
\maketitle
\thispagestyle{empty}

\begin{abstract}
%The abstract of a paper should contain a description of the problem addressed, a statement of the main result, a key conclusion that you want to highlight, and perhaps some comment on an application/numerical example included. One or two sentences for each of these items should be sufficient.
%The abstract should not include literature reviews, should not attempt to justify why the problem is important/timely, and should not discuss problems not solved in the paper.
%Abstracts are often included in searchable databases and therefore should be self-contained, with neither bibliographic citations nor references to figures/tables in the paper. Since they are often automatically converted to pure text, equations and non-text symbols should be avoided.
\pno{The design of} safe-critical control algorithms for systems under Denial-of-Service (DoS) \IfInt{and integrity}{} attacks on the system output \pno{is} studied in this work. We aim to address scenarios where attack-mitigation approaches are not feasible, and the system needs to maintain safety under adversarial attacks. We propose an attack-recovery strategy by designing a \pno{switching} observer and characterizing bounds in the error of a state estimation scheme by specifying tolerable limits on the time length of attacks. Then, we propose a \pno{switching} control algorithm that renders forward invariant a set for the \pno{observer}. Thus, by satisfying the error bounds of the state estimation, we guarantee that the safe set is rendered conditionally invariant with respect to a set of initial conditions. A numerical example \pno{illustrates} the efficacy of the approach.
\end{abstract}


% \section{Things to do (plan for 5-6 weeks)}
% Under the assumption that $e(0) = |x(0)- \hat x(0)|\leq E$. 

% \begin{itemize}
%     \item[1)] Show the maximum length (timewise) of an attack if we have a specific set $E_2$ that we want the error to remain in.
%     \item[2)] Show the minimum amount of time without an attack such that the error can return to an smaller set $E$.
%     \item Revise proofs. Find $\gamma_1, c_2, \lambda$.
%     \item[3)] Currently only estimation design (let's focus on that first). Later: What about control design? Given the best performance you can achieve with your observer, you can guarantee this and this in terms of control design. We want to guarantee that $x(t) \in S$ for all $t$.
% \end{itemize}


\section{Introduction}

% \textbf{Idea:} Extension of PID-Piper paper: for this idea, we use a similar approach as in PID-Piper (i.e., attacks on sensors), but with safety guarantees. Here, the idea is to switching between a nominal (feedback) control and a control under attack (open-loop) that does not rely on measurements at all time We include error accumulation due to model-mismatch in the state propagation and maybe use a hybrid observer to minimize this error, and accommodate it in the safety analysis. This can have a implementation (simulation)-based performance analysis on top of theoretical guarantees.

%The first paragraph should briefly describe the problem. It is essentially an expanded version of the initial sentence(s) in the abstract, but it is best leave the formal problem definition (with equations) for later in the paper. Sometimes a paragraph that precedes the problem description justifies in very broad terms the need to study the problem addressed by the paper.

%The need to study this problem:

The security of Cyber-Physical Systems from a control-theoretic perspective is a growing area of research~\cite{chong2019tutorial}. Various types of attacks on a control system can occur, such as sensor data or system actuators getting compromised~\cite{teixeira2015secure,cardenas2008secure}.  Attackers can \IfInt{inject malicious (false) data in the output or actuator signals, thus compromising the integrity of the system~\cite{Mo2014IntegrityAttacks}. Alternatively, attackers can}{} disable the transmission of signals between devices, causing a Denial of Service (DoS) attack\IfACC{~\cite{Amin2009DoSNetControl}}{}. Such attacks can lead to violation of safety requirements, such as avoiding obstacles or keeping the system trajectories in a desired region of the state space\IfACC{~\cite{krotofil2014cps}}{}.
\NotACC{As an example, under a sensor-based DoS attack~\cite{Amin2009DoSNetControl}, Programmable Logic Controllers (PLCs) can compute control signals with \emph{stale data}~\cite{krotofil2014cps}, leading the system to unsafe regions.} 

Mitigating and responding to attacks is an active area of research. Some efforts have focused on developing robust \pno{observers} to prevent compromised data from affecting the \pno{feedback loops.} %view of the system. 
Secure estimation uses redundant observers to reconstruct the state, but they assume that only a certain number of sensors (in particular, less than half of the sensors) have been compromised~\cite{fawzi2014secure}. An alternative to reduce this level of redundancy \NotACC{necessary for guaranteeing a satisfactory reconstruction of the state }is to reject outliers with the use of robust statistics~\cite{leblanc2013resilient}. \IfACC{This approach}{When rejecting sensor signals, an alternative is to use \emph{virtual sensors} from digital twins of the system~\cite{piedrahita2017leveraging}. However, that} requires precise knowledge of the system's dynamical model\IfACC{.}{\> for provable guarantees for safety-critical systems.} The control \NotACC{or actuation} signal can also be constrained to prevent attackers from causing damages~\cite{kafash2018constraining}. However, such approaches \pno{may negatively} affect the system's performance. There is a plethora of work on resilient or robust control design, see, e.g., \cite{fawzi2014secure,yan2017resilient,bai2017kalman}, that focuses on system performance under attacks, however, without any consideration or guarantees on safety. 

%Other approaches have focused on how to reconfigure a controller when the system is under attack. 

%In the last two decades, the impact of cyberattacks in control systems has increased significantly. A classification of attacks is based on its location in the control system scheme. We have input attacks [][], and sensor attacks []. A remarkable issue with sensor attacks is that it compromises the information used for feedback control loops impacting the whole integrity of the system [] and reducing success of mitigation strategies. These type of attacks include DoS, integrity attacks, ...
%and are designed by malicious agents aiming to cause as much harm as possible without being detected which motivates them to launch finite-time attacks only at some instants of time.
%Current literature has focused on attack-detection and prevention algorithms, see [] and the references therein, but there is a lack of tools that guarantee successful recovery from an imminent attack. 
Safety is perhaps the most important system property we need to maintain while undergoing an attack. Control barrier function (CBF)-based approaches can help design control algorithms for forward invariance of a safe set \cite{ames2014control}. The authors in \cite{clark2020control} introduce the notion of fault-tolerant CBF for handling attacks on stochastic systems. In \cite{amin2009safe}, the authors study safe control design under DoS attacks. 

In this paper\pno{,} we focus on the problem of safely recovering from output attacks, i.e., keeping the system trajectories in a safe set even under \NotInt{DoS} attacks\NotACC{\>on the system output}. 
%In safe-critical scenarios, the state of the system is kept from entering to an undesired set of the state space, referred to as an ``unsafe region". 
The proposed formulation is applicable to several use cases with objectives including \pno{obstacle avoidance} and collision-free navigation for autonomous vehicles, reach-avoid control problems, surveillance, and convoy of multi-agent systems, among others. 
%convergence of consensus \cite{senejohnny2017jamming}
%practical responses \cite{piedrahita2017leveraging}
%constraining actuators \cite{kafash2018constraining}
%
We \pno{propose a control scheme based on the information available, namely, the uncompromised outputs, that assures safety for systems with outputs experiencing DoS \IfInt{and integrity }{}attacks.} 
We consider scenarios {in which %some of the outputs remain unattacked at all instants of time} and 
every attack has finite \pno{duration}, succeeded by an interval \pno{of time} without attacks.
We are interested in finding the set of initial conditions and the control action such that the state trajectory remains in the safe set at all times. \pno{During} attacks, the controller relies only on the uncompromised outputs, \pno{from which we generate an estimate of the state,} whereas the \pno{entire output} is used when attacks are not present.
%The second paragraph typically relates this problem to the relevant literature. In short papers, all the literature review could be included here (in this case, perhaps needing two paragraphs). 
%The paragraphs that follow the literature review should clearly state the contribution of the paper

\IfACC{
In this paper, we design a \pno{switching} observer scheme that uses the complete output information when there is no attack and uncompromised output information \pno{during} an attack in the sensors.
 We provide sufficient conditions \pno{involving key} properties of the system, such as the maximum tolerable length of the DoS attack and the minimum required length of the interval without an attack for recovery, \pno{guaranteeing that the} state estimation error remains uniformly bounded. % for all times.
Furthermore, we design CBF-based observer-based feedback laws to render a properly defined set forward invariant for the \pno{observer} so that with bounded estimation error, \pno{the system is safe. This is obtained provided conditional invariance of a set of interest with respect to a set of initial states.}
Due to space constraints, proofs and other details are not included and will be published elsewhere.
}{The main contributions of this paper are as follows.
\begin{itemize}
\item We design a \pno{switching} observer scheme that switches according to the presence of attack in the sensors. %The designed observer 
It uses the complete output information when there is no attack and uncompromised output information \pno{during} an attack.
\item We provide sufficient conditions \pno{involving key} properties of the system, such as the maximum tolerable length of the DoS attack and the minimum required length of the interval without an attack for recovery, \pno{guaranteeing that the} state estimation remains uniformly bounded.% for all times.
\item We design CBF-based observer-based feedback laws to render a properly defined set forward invariant for the \pno{observer} so that with bounded estimation error, \pno{the system is safe. This is obtained provided conditional invariance of a set of interest with respect to a set of initial states.}
\end{itemize}}
%Choose your words carefully, skip over unimportant details, and focus the attention of the reader on your most important achievements. In these paragraph you may want to refer back to the literature review and emphasize the relationship between your contribution and previous results.


% The reminder of the paper is organized as follows. In section II we formulate the problem of recovering from DoS attacks while rendering a set safe and we introduce the solution approach followed in this paper. The design of a state estimator is presented in Section III. In Section IV we describe the process followed for control design to render a set of interest safe via control barrier methods. A numerical example focused on the double integrator is presented in Section V. Section VI provides conclusions, closing remarks, and future work.

\textbf{Notation.} {The symbols $\mathbb{R}$, $\mathbb{R}_{\geq 0}$, and $\mathbb{N}_{>0}$ denote the sets of real numbers, nonnegative reals, and positive natural numbers, respectively. Let $|x|$ be the \pno{E}uclidean norm of the vector $x$. 
Let $\overline{\mathcal{A}}$ denote the closure of the set ${\mathcal{A}}$.
Let $|A|$ be the \pno{induced matrix $2-$norm of $A$, $\textup{rank}(A)$ denote its rank,} and $\lambda_m(A), \lambda_M(A)$ denote the  eigenvalues with minimum and maximum real part, respectively. Let $\mathbb{B} \subset \mathbb{R}^n$ denote the \pno{closed} unit ball centered at the origin and $p+r\mathbb{B}$ the ball of radius $r\geq 0$ centered at $p\in \mathbb R^n$. We denote by $\tilde{\mathcal O} (C,A)$ the observability matrix of the pair $(C,A)$ and by $\tilde{\mathcal C} (A,B)$ the controllability matrix of the pair $(A,B)$.
}
\section{Preliminaries}
{
%%%%%%%%%%%%%%%%%%%%%%%%%%%
Consider the nonlinear system %$\mathcal F$ given as
\begin{align}\label{eq: nl system}
% \hspace{-40pt}\text{\textit{Actual system}:} &   \hspace{30pt} \dot x = \Phi(t,x,u), \quad x(0) = x_0,
\mathcal F : %\begin{cases}  %NotConf
\quad\quad %IfConf
\dot z = F(t,z) \IfDist{+ d(t,x)},
\IfACC{\quad \quad}{\\}
%\\ %NotConf
y = H(t,z)%,\\ %NotConf
%z\in \mathcal D
%\end{cases} %NotConf
\end{align}
%with $z(0) = z_0\in \mathbb R^{n}$, 
where $z\in %\mathcal D$
\mathbb{R}^n$ is the system state, $y\in \mathbb R^p$ is the system output, % with $\mathcal D\subset\mathbb R^{n}$, 
 $F: \mathbb{R}_{\ge0} \times \mathbb{R}^{n} \rightarrow \mathbb{R}^n$ \pno{is the (potentially nonsmooth) flow map} and $H:\mathbb{R}_{\ge0} \times \mathbb{R}^{n} \rightarrow \mathbb{R}^p$ \pno{is the output map}.

 A solution to the system $\mathcal{F}$ is defined as follows.

{
\begin{Definition}[Solution to $\mathcal{F}$]
A \pno{locally absolutely continuous function $t \mapsto z(t)$} defines a solution to the %hybrid 
system $\mathcal{F}$ in (\ref{eq: nl system}) 
  if
%\begin{itemize}
%    \item $z(0) \in \overline{\mathcal{D}} $,
%    \item For all $t \in \textup{dom} \> z$, $z(t) \in \mathcal{D}$,
%    and
 %   \item 
 %for almost all $t \in \textup{dom} \> z$,
    %\begin{eqnarray*}
     $   \frac{d}{dt}z(t)= F(t,z(t))$
    %\end{eqnarray*}
    \pno{for almost all $t \in \mathbb{R}_{\geq 0}$.}%\textup{dom} \> z$. 
%    \item For all $(t,j)\in \textup{dom} x$ such that $(t,j+1)\in \textup{dom} x$,  
%    \begin{eqnarray*}
%	x(t,j) &\in& D_\kappa \\
%	x(t,j+1) &=& G(x(t,j),\kappa_D({x}(t,j))) 
%\end{eqnarray*}
%\end{itemize}
%A solution $x$ is a compact solution if $x$ is a compact hybrid arc.
\label{SolutiontocalH}
\end{Definition}}
  We say that a solution $z$ to $\mathcal{F}$ is maximal if it cannot be extended and we say it is complete when $\textup{dom} \> z = [0, \infty)$. 

\begin{Definition}[Safety] %Given an observer-based feedback law $\kappa$, the  closed-loop system (\ref{eq: cl system}) is said to be safe with respect to $(X_0,\hat X_0,X_u)$, with $X_0,\hat X_0 \subset \mathbb{R}^n \setminus X_u$, if for each$x_0 \in X_0$, $\hat x_0 \in \hat X_0$, the solution to (\ref{eq: cl system}) starting from $(x_0, \hat x_0)$ satisfies $x(t) \in \mathbb{R}^n\setminus X_u$ and \pn{$|x(t)- \hat x(t)| \leq \bar E$} for all$t \in \textup{dom}\> (x, \hat x)$  with \pn{$\bar E >0$}.
The system (\ref{eq: nl system}) is said to be safe with respect to $(X_0,X_u)$, with $X_0 \subset \mathbb{R}^{n} \setminus X_u$, if for each $z_0 \in X_0$, \pno{each} solution $t \mapsto z(t)$ to (\ref{eq: nl system}) \pno{with $z(0)=z_0$} satisfies $z(t) \in \mathbb{R}^{n}\setminus X_u$ for all $t \in \textup{dom}\> z$.
\end{Definition}

\begin{Definition}[Conditional invariance] %Given an observer-based feedback law $\kappa$, a closed set $S \subset \mathbb{R}^n$ is said to be conditionally invariant for the  closed-loop system (\ref{eq: cl system}) with respect to the sets $M,\hat M \subset S$ if, for each $x_0 \in M$, $\hat x_0 \in \hat M$, the solution starting from $x_0, \hat x_0$ satisfies $x(t) \in S$, and \pn{$|x(t)- \hat x(t)| \leq \bar E$} for all $t \in \textup{dom}\> x$ \pn{with $\bar E >0$}.
A closed set $S \subset \mathbb{R}^{n}$ is said to be conditionally invariant for system (\ref{eq: nl system}) with respect to $M\subset S$ if, for each $z_0 \in M$, any solution $t \mapsto z(t)$ to (\ref{eq: nl system}) from $z_0$ satisfies $z(t) \in S$ for all $t \in \textup{dom}\> z$. 
\end{Definition}

It is immediate that the system (\ref{eq: nl system}) is safe with respect to $(X_0,X_u)$ if and only if the set $S := \mathbb{R}^n\setminus X_u$ is conditionally invariant for (\ref{eq: nl system}) with respect to  $X_0$.} \pno{For more details see \cite{198}.}
%%%%%%%%%%%%%%%%%%%%%%%%%555



\section{Problem formulation}
\subsection{System Model}
Consider \pno{the} linear \pno{time-invariant} control system 
\begin{align}\label{eq: actual system}
% \hspace{-40pt}\text{\textit{Actual system}:} &   \hspace{30pt} \dot x = \Phi(t,x,u), \quad x(0) = x_0,
\mathcal S : %\begin{cases} %NotConf
\quad\quad %IfConf
\dot x = Ax + Bu \IfDist{+ d(t,x)},
\IfACC{\quad \quad}{\\}
%\\ %NotConf
y = Cx%,\\ 
%x\in \mathcal D, u\in \mathcal U,
%\end{cases} %NotConf
\end{align}
%with $x(0) = x_0\in \mathbb{R}^n$, %\overline{\mathcal D}$, 
where $x\in \mathbb{R}^n$ %\mathcal D $
is the system state, $y\in \mathbb R^p$ is the system output, %and 
$u\in \mathcal U$ is the control input, %with $\mathcal D\subset\mathbb R^n$
and $\mathcal U\subset \mathbb R^m$. Here, \pno{$A\in \mathbb R^{n\times n}$, $B\in \mathbb R^{n\times m}$, %are such that $\mathcal S$ has unique solutions, %are the system matrices,
\NotDist{and} $C\in \mathbb R^{p\times n}.$}% is the output matrix\NotDist{.} \IfDist{and $d:\mathbb R_{\geq 0}\times\mathbb R^n\rightarrow\mathbb R^n$ is unknown and represents the unmodeled dynamics.}


\subsection{Attack Model}
In this work, we consider attacks on the system output $y$. In particular, we consider an attack where a subset of the components of the system output is compromised. Under such an attack model, the \textit{measured} system output {$\bar y$} takes the form
%\begin{align}\label{eq: attack model}
%   \bar y =  \begin{cases}\begin{bmatrix}\tilde Cx\\ \bar C x\end{bmatrix} & \textrm{if} \; t\notin \mathcal T_a, \vspace{3pt}\\
%    \begin{bmatrix}\tilde Cx\\ Y(t,x)\end{bmatrix} & \textrm{if} \; t \in \mathcal T_a, \end{cases}
%\end{align}
%
\begin{equation}\label{eq: attack model}
    \bar {y} = %\begin{bmatrix}
    (y_s , y_a) 
    %\end{bmatrix},
\end{equation}
where 
%\begin{equation}
 $  y_s=\tilde Cx,
$ and, \pno{for each solution $t \mapsto x(t)$ to \eqref{eq: actual system},} %\end{equation}

\begin{align}\pno{
    y_a(t) = \begin{cases}\bar C x(t) & \textrm{if} \; t\notin \mathcal T_a, \\ Y(t,x(t)) & \textrm{if} \; t\in \mathcal T_a\end{cases}}
\end{align}
The quantity $\tilde Cx$ denotes the \textit{secured} output components that \textit{cannot} be attacked with $\tilde C\in \mathbb R^{\tilde p\times n}$ and $0\leq \tilde p<p$, $\bar Cx$ denotes the \textit{vulnerable} output components that \textit{can} be attacked with $\bar C\in \mathbb R^{(p-\tilde p)\times n}$ such that $C = \begin{bmatrix}\tilde C\\ \bar C\end{bmatrix}$, and $Y:\mathbb R_{\geq 0}\times\mathbb R^n\rightarrow\mathbb R^{p-\tilde p}$ denotes the attacked output signal.
% (see Figure \ref{fig:overview fig}). 
We denote with $\mathcal T_a\subset\mathbb R_{\geq 0}$ the set of times when an attack is present on the system output, \pno{which is assumed to be known provided a DoS attack detection mechanism}. 
% \begin{figure}[b]
% 	\centering
% 	\includegraphics[width=\columnwidth,clip]{sensor_attach_overview.g}
% 	\caption{Overview of the attack model and the recovery mechanism.}
% 	\label{fig:overview fig}
% \end{figure}
The attack model \eqref{eq: attack model} captures \IfInt{both}{} Denial-of-Service (DoS) \IfInt{and integrity}{} attacks on the system output\IfInt{, and thus, models a large class of cyber attacks \cite{Amin2009DoSNetControl,Mo2014IntegrityAttacks}}{}. Let $[t_1^i, t_2^i)$ with $t_2^i>t_1^i\geq 0$ denote the interval of time \pno{over which} the \pno{$i-$th} DoS attack \pno{occurs}, with \pn{$i \in \mathbb{N}_{>0}$.} 
\pno{Define $\mathcal T_a \coloneqq \bigcup\limits_i[t_1^i, t_2^i)$,
$\mathcal T_1 = \bigcup\limits_i\{t_1^i\}$, and $\mathcal T_2 = \bigcup\limits_i \{t_2^i\}$ as the intervals of attack, and the sets of the starting and ending time instants of attacks, respectively.
To provide sufficient conditions to guarantee safety, we characterize the attacks by defining}
\IfInt
{\begin{subequations}
\begin{align}
    T_a & \coloneqq \max_{i\in \{1,2,\dots\} }(t_2^i-t_1^i),\\
    T_{na} & \coloneqq \min_{i\in \{2,3,\dots\}}(t_1^i-t^{i-1}_2),
\end{align}
\end{subequations}}{
 $T_a  \coloneqq \max_{i\in \{1,2,\dots\} }(t_2^i-t_1^i)$ and $
    T_{na}  \coloneqq \min_{i\in \{2,3,\dots\}}(t_1^i-t^{i-1}_2)$
}
as the maximum length of the DoS attack and the minimum length of the interval without an attack, %on the system sensors, 
respectively. Notice that $t_2^0:=0$, and when $t_1^1>0$, we have $t_1^1\geq T_{na}$.

% Note that under a output-feedback law $u = k(y)$, an attack on the system output $y$ affects the behavior of the system through the input $u$. 
\subsection{Problem Statement}
%
Given a nonempty, closed set $S\subset\mathbb R^n$, referred to as \pno{ the {\em safe}} set, the problem to solve is the design of an algorithm such that the set $S$ is conditionally invariant for (\ref{eq: actual system}) with respect to the set $X_0$.
\IfDist{We make the following assumption on the unmodeled dynamics $d$ in \eqref{eq: actual system}:
\begin{Assumption}\label{assum: d bound}
There exists $\gamma>0$ such that $|d(t, x)|\leq \gamma$ for all $t\geq 0$ and $x\in \mathcal D$.
\end{Assumption}}
%
% \noindent We consider the \textit{safety} property when designing the detection mechanism for an attack on the system input. The problem we study in this paper is as follows. 
%
Formally, the control design problem studied in this paper is stated as follows.

\begin{Problem}\label{Problem 1}
Given system \eqref{eq: actual system}\IfDist{ with unmodeled dynamics $d$ that satisfies Assumption \ref{assum: d bound}}, %the observer \eqref{eq: Generic Observer},
a closed set $S \subset \mathbb{R}^n$, and the attack model in \eqref{eq: attack model}, 
\begin{enumerate}
    \item Find a set of initial states $X_0\subset S$, %and initial estimates $\hat X_0\subset S$, 
    and
    \item Design a control %n observer-based feedback 
    law {$\kappa$} assigning the input $u$ of \eqref{eq: actual system} \pno{using measurements of $\bar y$}%, and defining the closed-loop system %\eqref{eq: cl system},
\end{enumerate}
    such that\pno{,} for each $x_0 \in X_0$, %\hat x_0 \in \hat X_0$,
     %pair $(x,\hat x)$
    the %$x$ component of each 
    solution to the resulting closed-loop system, %\eqref{eq: cl system}, 
    namely $t \mapsto x(t)$, with $x(0)=x_0$, %(x_0, \hat x_0)$,
    %of the closed-loop system resulting from the composition of \eqref{eq: actual system}
    %and the designed input
    satisfies $x(t) \in S$ % and \pn{$|x(t)- \hat x(t)| \leq \bar E$}
    for all $t\geq 0$.% with \pn{$\bar E >0$}.
\end{Problem}


% \subsection{Linear time-invariant case}
% Consider the LTI system:
% \begin{subequations}\label{eq: LTI system}
% \begin{align}
%     \dot x & = Ax + Bu + d(t,x), \\
%     y & = Cx, 
% \end{align}
% \end{subequations}
% with $x(0) = x_0\in \mathbb R^n$, where $A\in \mathbb R^{n\times n}, B\in \mathbb R^{n\times m}, C\in \mathbb R^{p\times n}$ and $d:\mathbb R_{\geq 0}\times\mathbb R^n\rightarrow\mathbb R^n$ is unknown and represents the unmodeled dynamics. 

\subsection{Proposed Solution}

%The control design objective, per Problem $(\star)$, {is to design an {observer-based feedback law} to keep the $x$ component of the solution to \eqref{eq: cl system} starting from $X_0$} in the safe set %and \pn{ estimation error bounded} 
%at all times, or,
%equivalently to render the set $S$ {conditionally} %forward
%invariant with respect to $X_0$ %, \hat X_0$ 
%for the closed-loop system \eqref{eq: cl system}.
To solve Problem $(\star)$, we propose the design of an observer-based feedback law that induces conditional invariance of $S$ with respect to $X_0$. 
Most CBF-based methods for forward invariance rely on measurement of the entire state \cite{ames2017control}. 
%To design an observer-based feedback law,
We propose to employ a state estimator that reconstructs the system state using the measured output $\bar y$. The observer is given as
% \begin{subequations}
\begin{align}\label{eq: pred model}
    \dot{\hat x} \NotACC{&}
    %&
    = A\hat x + Bu + g(\bar y,\hat y), \IfACC{\quad \quad}{\\}
    %\\
    \hat y \NotACC{&}
    %&
    = C\hat x,
\end{align}
% \end{subequations}
%with $\hat x(0) = \hat x_0
\pno{where $\hat x\in \mathbb R^n$ is the estimate of $x$} and $g:\mathbb R^p\times\mathbb R^p\rightarrow\mathbb R^n$ is the innovation term to be designed \pno{such that $g(\bar{y},\hat{y})=0$ at $\bar y = \hat y$}. 
% In the special case of a Luenberger observer, the function $g$ is defined as $g(y) = L(y-C\hat x)$ for an appropriate matrix $L:\mathbb R^{n\times p}$. 
% and $L\in \mathbb R^{n\times p}$ be such that each of the eigenvalue of the matrix $(A-LC)$ is in open left-half plane. 
% When the system output is under an attack according to the attack model \eqref{eq: attack model}, the state information is not available for the feedback synthesis. To this end, we design a control algorithm that uses the actual state $x$ when the system output is not under an attack, and the predictor state $\hat x$, when the system output is under an attack. More specifically, the control input under the attack model \eqref{eq: attack model} is given as
% % \begin{align}\label{eq: u switch}
% %     u = \begin{cases}Kx & t\notin \mathcal T_a; \\
% %     K\hat x & t\in \mathcal T_a;\end{cases},
% % \end{align}
% \begin{align}\label{eq: u switch}
%     u = \begin{cases}h_1(x) & t\notin \mathcal T_a; \\
%     h_2(\hat x) & t\in \mathcal T_a;\end{cases},
% \end{align}
% where $h_1, h_2: \mathbb R^{n}\rightarrow\mathbb R^m$ are the feedback laws under \textit{nominal} condition (i.e., when the system is not under an attack) and \textit{adversarial} condition (i.e., when the system is under an attack) to be defined. Thus, the closed-loop system \eqref{eq: LTI system} under \eqref{eq: u switch} reads
% \begin{align}
%     \dot x = \begin{cases} Ax + Bh_1(x) + d(t,x) & t\notin \mathcal T_a; \\
%     Ax + Bh_2(\hat x) & t\in \mathcal T_a; \end{cases}.
% \end{align}
When the system output is under an attack according to the attack model \eqref{eq: attack model}, the actual output information is not available to the state observer. Thus, the observer needs to take into account the attacks on the system output. To this end, we design an observer that uses the \textit{complete output} vector when there is no attack and only the \textit{non-attacked output} components when the system output is under attack. More specifically, the proposed observer under the attack model \eqref{eq: attack model} is given as
\begin{align}\label{eq: hat x switch general}
    \dot {\hat x} = \begin{cases} A\hat x + Bu + g_1(Cx,C\hat x) & \textrm{if} \quad  t\notin \mathcal T_a, \\
    A\hat x + Bu + g_2(\tilde Cx, C\hat x) & \textrm{if} \quad t\in \mathcal T_a \end{cases}
\end{align}
where $g_1, g_2:\mathbb R^p\times \mathbb R^p\rightarrow\mathbb R^n$ are to be designed. 
% \begin{align}\label{eq: u switch}
%     u = \begin{cases}Kx & t\notin \mathcal T_a; \\
%     K\hat x & t\in \mathcal T_a;\end{cases},
% \end{align}
% In other words, if the observable state-space for the pair $(\tilde C, A)$ has dimension $\tilde n$, then the ma 
%The {observer-based feedback law} assigning $u$ is %defined using the estimate $\hat x$ and is given by
% \begin{align}\label{eq: u switch}
%     u = \begin{cases}Kx & t\notin \mathcal T_a; \\
%     K\hat x & t\in \mathcal T_a;\end{cases},
% \end{align}
%\begin{align}\label{eq: u switch}
%    \kappa(\hat x,\bar y) = \begin{cases}
%    \kappa_1(\hat x,\bar y) & \textrm{if} \quad t\notin \mathcal T_a, \\
%    \kappa_2(\hat x,\bar y) & \textrm{if} \quad t\in \mathcal T_a\end{cases}
%\end{align}
%\pn{defined as in \eqref{eq: u switch} as a function of the measured output $\bar y$, namely, 
%\begin{equation} \label{eq: u switch}
%(\hat x, \bar y) \mapsto \kappa(\hat x, \bar y),
%\end{equation}
Given a set $\mathcal{T}_a \subset \mathbb{R}$, the %observer-based 
feedback law $\kappa$ assigning $u$ is defined as 
\begin{align}\label{eq: u switch} 
    \kappa(t,\hat x, y) = \begin{cases}
    \kappa_1(\hat x, y) & \textrm{if} \quad t\notin \mathcal T_a, \\
    \kappa_2(\hat x, y) & \textrm{if} \quad t\in \mathcal T_a,
    \end{cases}
\end{align}\\
where $\kappa_1, \kappa_2: \mathbb R^{n} \times \mathbb{R}^p \rightarrow\mathbb R^m$ are functions to be designed under \textit{nominal} operation (i.e., when the system is not under an attack) and \pno{under attack,} respectively.
Notice that the closed-loop system
resulting from the composition of \eqref{eq: actual system} and \eqref{eq: hat x switch general} with $\kappa$ as in \eqref{eq: u switch} can be expressed \pno{as in} \eqref{eq: nl system} with $z = (x, \hat x)$.
%} 
% 
% Thus, the closed-loop system \eqref{eq: LTI system} under \eqref{eq: u switch} reads
% \begin{align}
%     \dot x = \begin{cases} Ax + Bh_1(\hat x) + d(t,x) & \textrm{if} \quad t\notin \mathcal T_a; \\
%     Ax + Bh_2(\hat x) + d(t,x) & \textrm{if} \quad t\in \mathcal T_a; \end{cases}.
% \end{align}
% \pn{%To ensure during attack-free intervals that for every initial condition, there exists a control action that can drive the state to a desired point, and seeking to take advantage of the history of previous control inputs and sensor measurements, 
% To establish minimum requirements in the design process of a controller and a state estimator, 

We make the following assumption on $\mathcal{S}$ in \eqref{eq: actual system}.
% }
\begin{Assumption}\label{assum A B cont}
The pair $(A,B)$ is controllable and the pair $(C,A)$ is {detectable.}
\end{Assumption}

Based on the structure of the observer in \eqref{eq: actual system} and the {observer-based feedback law} in \eqref{eq: u switch}, the approach followed in this paper for \pno{safety under attacks for} system \eqref{eq: actual system} is as follows.\\
%\pn{Should we state explicitly that the input of \eqref{eq: hat x switch general} is $u_\kappa$? In other words, should we define a closed loop system for the observer?}
\textbf{Approach}: 
% \begin{Problem}
Given a closed set $S \subset \mathbb{R}^n$, the system \eqref{eq: actual system}% satisfying Assumption \ref{assum A B cont}
,\IfDist{ with unmodeled dynamics $d$ satisfying Assumption \ref{assum: d bound},} and the attack model \eqref{eq: attack model}, %and the \pn{feedback law} $u_\kappa$ as in \eqref{eq: u switch}, 
our approach is to compute sets $X_0, \hat X_0,\hat S_0\subset S$ and design functions $g_1,g_2$ for the observer in \eqref{eq: hat x switch general} and functions $\kappa_1, \kappa_2$ for the {observer-based feedback law} $\kappa$ as in \eqref{eq: u switch} such that
\pno{each} solution pair $t \mapsto (x(t), \hat x(t))$ to the closed-loop system resulting from the composition of \eqref{eq: actual system} and \eqref{eq: hat x switch general} with  $\kappa$ satisfies \pno{the following properties:}
\begin{itemize}
    \item[1)] For each $t_0\in \mathcal T_1$ such that $x(t_0)\in X_0$ and $\hat x(t_0)\in \hat X_0$, the $x$ component of the resulting closed-loop solution satisfies $x(t)\in S$ for all $t\in [t_0, t_0+T_a)$;
 % Under an attack starting at X_0 you will remain in S for at least T_a 
 % Notice that the attack can be shorter than T_a and this condition might imply multiple intervals
    %\item[2)] For each $t_0\in \mathcal T_2$  such that $x(t_0)\in S$ and \pn{$\hat x(t_0)\in  \hat S_0$}, $x(t_ 0+T_{na})\in X_0$ and $x(t)\in S$ for all $t\in [t_0, t_0+T_{na})$
    % At the end of an attack, state starting at S, will be back in X_0 in T_{na}, and S is invariant
%%%%%%%%%%%%%%%%%%%%%%%% 
    \item[2)] For each $t_0\in \mathcal T_2$ such that $x(t_0)\in S$ and $\hat x(t_0)\in  \hat S_0$, 
    and for $\hat t_0 =\max \{t_0, \inf_{t\geq t_0} \mathcal T_1 \}$,
    %which accounts for the interval after the last attack
    the $x$ component of the resulting closed-loop solution satisfies %$\pn{x(t_ 0+T_{na})}, 
    $x(\hat t_0)\in X_0$ and $x(t)\in S$ for all $t\in [t_0, \hat t_0)$.
    % For intervals without attacks: 
    % state starting at S, will 
    %    - be back in X_0 in T_{na} seconds (Not anymore), 
    %    - be in X_0 at the beginning of next attack and
    %    - will stay at S (invariant) at all time of the interval
%%%%%%%%%%%%%%%%%%%%%%%    
%%%%%%%%%%%%%%%%%%%%%%% 
    %\item[3)] For each $t_0\notin \mathcal T_a$ such that $x(t_0)\in X_0$ and \pn{$\hat x(t_0)\in \hat X_0$, $x(t)\in S$ for all $t\in [t_0, \hat t_0)$ and $x(\hat t_0)\in X_0$} where $\hat t_0 =\max \{t_0, \inf_{t\geq t_0} \mathcal T_1 \}$,
    % CInv_v: Under no attack, state starting at X_0, will come back to X_0 at the time of the next attack.
%%%%%%%%%%%%%%%%%%%%%%% 
   
    %\item[3)] For each $t_0\notin \mathcal T_a$ such that $x(t_0)\in X_0$ and \pn{$\hat x(t_0)\in \hat X_0$, $x(t)\in X_0$} for all $t\in [t_0, \hat t_0)$ where $\hat t_0 =\max \{t_0, \inf_{t\geq t_0} \mathcal T_1 \}$,
    % FInv_v: Under no attack, state starting at X_0, will remain at X_0 until the next attack.
%%%%%%%%%%%%%%%%%%%%%%%
%%%%%%%%%%%%%%%%%%%%%%% 
%
%
% All of this is under the assumption that I know the times t_1^i, t_2^i.
\end{itemize}
%\pn{where $x$ is the state of \eqref{eq: cl system} and $\hat x$ is the estimate of $x$ according to the observer $\eqref{eq: hat x switch general}$.}
\begin{Remark}
\pno{The sets $\hat{X}_0$ and $\hat{S}_0$ denote the sets of estimates before and after an attack, respectively. We will design these sets in the next section.}
Item \pno{1} in our solution approach encodes conditional invariance of the set $S$ %, \hat{X}_0$
for system \eqref{eq: actual system} with respect to $X_0$, under an attack with maximum duration.
%
 Upon the requirement of the state to be in $S$ at the end of every attack, item \pno{2} encodes safety of  system \eqref{eq: actual system} with respect to $(X_0, \mathbb{R}^n \setminus S)$ during the %shortest
 time-intervals with no attacks,
 %
 {and the state to be in $X_0$ %after the minimum time without attack and 
 at the beginning of the next attack.}
%
%Finally, item 3) 
% CInv_v:
%\pn{encodes the requirement of the state to come back to the set $X_0$} before the next attack,  
%\pn{
%%If $\hat{t}_0$ is infinite, the state is not required to go back to $X_0$.
%covering the case in which $\hat{t}_0$ is infinite.
%}
% FInv_v:\pn{encodes invariance of the set $X_0$} in the absence of an attack, which helps guarantee conditional invariance via item 1).  \pn{This last item covers the case in which $\hat{t}_0$ is infinite.}
\end{Remark}

\NotACC{In the next section, we present the design of the observer in \eqref{eq: hat x switch general}, and in the following section, we present the design of the {observer-based feedback law} in \eqref{eq: u switch}.}

% \end{Problem}

% \section{Preliminary analysis}
% Define $e = x -\hat x$ as the difference between the system states $x$ and the predictor state $x$ to obtain the error dynamics as
% \begin{align}
%     \dot e = \begin{cases} (A-LC)e +d(t,x), & t\notin \mathcal T_a; \\
%     Ae + LC\hat x+d(t,x) & t\in \mathcal T_a;\end{cases}.,
% \end{align}
% with $e(0) = x(0) - \hat x(0) = x_0-\hat x_0$. 
% First, we analyze the bounds on error $e$ when there is no attack, i.e., $t\notin \mathcal T_a$. Let $t_0 \in \cup \{t_2^i\}\cup \{0\}$ be the starting instant of the interval when there is no attack on the system output and define $e_0 \coloneqq e(t_0)$. Since the eigenvalues of the matrix $(A-LC)$ are in the open left-half plane, it follows that there exists positive definite matrices $P, Q$ such that $V:\mathbb R^n\rightarrow\mathbb R_{\geq 0}$ defined as $V(e) \coloneqq e^TPe$ satisfies
% \begin{align*}
%     \dot V(e)  = -e^TQe + 2e^TPd(t,x).
% \end{align*}
% Let $|A|$ be a matrix norm of a matrix $A$ and $\lambda_m(P), \lambda_M(P)$ denote the minimum and the maximum eigenvalue of a positive definite matrix $P$, respectively. Then, under Assumption \ref{assum: d bound}, we obtain
% \begin{align*}
%     \dot V(e)\leq -|e|^2\lambda_m(Q) + 2|e||P|\gamma.
% \end{align*}
% Let $c_1 = \lambda_m(Q), c_2 = 2|P|\gamma$ and $\theta\in (0,1)$ and re-write the above bound on $\dot V$ as
% \begin{align*}
%     \dot V(e)\leq -(1-\theta)c_1|e|^2 -\theta c_1|e|^2 + c_2|e|.
% \end{align*}
% Note that for $|e|\geq \frac{c_2}{\theta c_1} = \frac{2\gamma|P|}{\theta\lambda_m(Q)}$, it holds that $-\theta c_1|e|^2 + c_2|e|$. Define $E = \coloneqq \left\{e\in \mathbb R^n\; |\; |e|\leq \frac{2\gamma|P|}{\theta \lambda_m(Q)}\right\}$ so that for $e_0\in E$, it holds that
% \begin{align*}
%     \dot V(e) \leq -(1-\theta)c_1|e|^2,
% \end{align*}
% and thus, it follows that
% \begin{align*}
%     |e(t)|\leq \frac{\lambda_M(P)}{\lambda_m(P)}\exp\left({-\frac{(1-\theta)c_1}{\lambda_M(P)}(t-t_0)}|e(t_0)|.
% \end{align*}
% Furthermore, it holds that for each $e(t_0)\in E $, $e(t)\in E$ for all $t\geq t_0$. 

% % \subsection{Prediction error during attack}
% Now we compute the maximum possible growth in the prediction error $e$ during the system output is attacked, i.e., $t\in \mathcal T_a$. Let $t_0\in \mathcal T_a$ be the instant the attack happens on the system output and $e(t_0) = e_0$. Without loss of generality, assume that $0\in \textrm{S}$ and that the control input $u = h_2(\hat x)$ is such that the origin is exponentially stable for \eqref{eq: pred model}. Thus, assume that there exists $\lambda, c>0$ such that $|\hat x(t)|\leq \lambda \exp\left({-c(t-t_0)}|\hat x(t_0)|$. Under this setup, it can be shown that the error $e$ can be bounded as
% \begin{align*}
%     |e(t)|\leq \exp\left({c_1(t-t_0)}|e(t_0)|+\frac{\exp\left({c_1(t-t_0)}-1}{c_1}\gamma + \frac{1}{c_1+c}(\exp\left({c_1(t-t_0)}-\exp\left({-c(t-t_0)})
% \end{align*}
% for some $c_1>0$. 

% In particular, the maximum error for an attack of time length $T$ is
%  \begin{align*}
%     |e_M|\leq \exp\left({c_1T}|e(t_0)|+\frac{\exp\left({c_1T}-1}{c_1}\gamma + \frac{1}{c_1+c}(\exp\left({c_1T}-\exp\left({-cT}) 
%  \end{align*}
 
% In brief (let $\{t^i_1\}, \{t^i_2\}$ be the sequences when the attacks start and stop: 
% \begin{itemize}
%     \item Assume $e(0)\in E$ so that $e(t^1_1)\in E$, i.e., $$|e^1_1|\leq \frac{2\gamma|P|}{\theta \lambda_m(Q)}$$ 
%     \item 
%     % $$|e(t^1_2)|\leq \exp\left({c_1T}|e(t^1_1)|+\frac{\exp\left({c_1T}-1}{c_1}\gamma + \frac{1}{c_1+c}(\exp\left({c_1T}-\exp\left({-cT})$$
%     $$|e(t^1_2)|\leq \exp\left({c_1T}\frac{2\gamma|P|}{\theta \lambda_m(Q)}+\frac{\exp\left({c_1T}-1}{c_1}\gamma + \frac{1}{c_1+c}(\exp\left({c_1T}-\exp\left({-cT})$$
%     \item $$|e(t^2_1)|\leq \frac{\lambda_M(P)}{\lambda_m(P)}\exp\left({-\frac{(1-\theta)c_1}{\lambda_M(P)}T}|e(t^1_2)|$$
%     \item If $e(t^2_1)\in E$, then it holds for all $i\geq 1$
%     \begin{align*}
%         |e(t^i_2)|& \leq \underbrace{\exp\left({c_1T}\frac{2\gamma|P|}{\theta \lambda_m(Q)}+\frac{\exp\left({c_1T}-1}{c_1}\gamma + \frac{1}{c_1+c}(\exp\left({c_1T}-\exp\left({-cT})}_{\rho}\\
%         |e^i_1| & \leq \frac{2\gamma|P|}{\theta \lambda_m(Q)}
%     \end{align*}
%     \item Main conclusion: 
%     \begin{align*}
%         \frac{\lambda_M(P)}{\lambda_m(P)}\exp\left({-\frac{(1-\theta)c_1}{\lambda_M(P)}T}\rho \leq \frac{2\gamma|P|}{\theta \lambda_m(Q)}
%         \implies |e(t)| \leq \frac{\lambda_M(P)}{\lambda_m(P)}\rho \quad \forall t \geq 0
%     \end{align*}
% \end{itemize}
 
% \begin{figure}[b]
% 	\centering
% 	\includegraphics[width=\columnwidth,clip]{error_image.png}
% 	\caption{Evolution of error under the conditions discussed above.}
% 	\label{fig:err e}
% \end{figure}

%  \subsection{Reduced-order observer}
%  Main idea:
 
%  \begin{align}
%      \dot x & = Ax + Bu, \\
%      y & = Cx, 
%  \end{align}
%  Let's assume that after an attack, the system output takes the form:
% \begin{align}
%     y = \begin{cases}Cx & \textrm{if} \; t\notin \mathcal T_a, \\
%     \begin{bmatrix}\tilde Cx\\ O(t,x)\end{bmatrix} & \textrm{if} \; t \in \mathcal T_a, \end{cases}
% \end{align}
% Assume that the matrix $\tilde C\in \mathbb R^{\tilde p\times n}$ is full-rank. Then, it is possible to find a transformation $Q = \begin{bmatrix}
% \tilde C\\ R \end{bmatrix}$ such that $Q$ is invertible. Using this, define $z = \begin{bmatrix}
% z_1\\z_2\end{bmatrix} = \begin{bmatrix}\tilde Cx\\ Rx\end{bmatrix} = Qx$ to obtain
% \begin{align}
%     \dot z & = QAQ^{-1}z + QBu, \\
%     y & = z_1, 
% \end{align}
% when the system output is under attack. Now, if $(\tilde C, A)$ is observable, then an observer-based controller can be designed for $t\in \mathcal T_a$. On the other hand, if $(C, A)$ is detectable but $(\tilde C, A)$ is not observable, then an observer cannot be designed. 

% Discussion: 
% \begin{itemize}
%     \item If we assume that $(\tilde C, A)$ is observable, then the redundant input in $y= Cx$ are not necessary, and so, effectively, the attack is not affecting the system. 
%     \item If we assume that $(\tilde C, A)$ is not observable, then a reduced-order observer cannot be designed. 
% \end{itemize}

% % If the matrix $A$ is invertible, then it follows under Assumption \ref{assum: d bound} that
% % \begin{align}
% %     |e(t)|\leq \gamma |A^{-1}(\exp\left({A T}-I))|, 
% % \end{align}
% % for each $T>0$. 

% % System
% % \begin{align}
% % \dot x & = Ax + Bk(\hat x),\\
% %  \dot{\hat x} & = A\hat x + L(y-C\hat x)+ Bk(\hat x),
% % \end{align}

% % Requirement: $x(t)\in S$ for all $t\geq 0$ and all $x(0)\in X_0\subset S$ with $|x(0)-\hat x(0)|\leq \gamma$ for some carefully chosen $\gamma>0$.


\section{Switching Observer Design}
Under an attack on the system output of the form \eqref{eq: attack model}, it might not be possible to reconstruct the state \pno{of \eqref{eq: actual system}} for a full-state feedback control design. Specifically, under the considered attack model, the rank of the observability matrix $\tilde{\mathcal O}$ for the pair $(\tilde C,A)$, namely, $\textup{rank}(\tilde{\mathcal O}) = \tilde n$, potentially \pno{smaller than $n$.} Thus, there might be \pno{$n-\tilde n>0$} eigenvalues in the closed right-half plane for the \pno{dynamics of the estimation error resulting for} any observer design under attack. Keeping this in mind, \pno{the \pno{switching} observer in \eqref{eq: hat x switch general} is defined as}
\begin{align}\label{eq: hat x switch}
    \dot {\hat x} = \begin{cases} A\hat x+ Bu  + L(Cx-C\hat x) & \textrm{if} \quad  t\notin \mathcal T_a, \\
    A\hat x + Bu +\tilde L(\tilde Cx-\tilde C\hat x)& \textrm{if} \quad t\in \mathcal T_a,\end{cases}
\end{align}
where \pno{$L\in \mathbb{R}^{n\times p}$ and} $\tilde L \in \mathbb R^{n\times \tilde p}$ is such that $\tilde n$ (with $\tilde n\leq n$) eigenvalues of the matrix $A-\tilde L\tilde C$ lie in the open left-half plane. %, where $\tilde n$ is the rank of the observability matrix for the pair $(\tilde C, A)$. 
%
On the other hand, since $(C, A)$ is detectable under Assumption \ref{assum A B cont}, we can design $L$ such that all the eigenvalues of $(A-LC)$ are in the open left-half plane.
% \subsection{Convergence when there is no attack}
Now, define $e = x -\hat x$ as the \pno{estimation} error to obtain the error dynamics given as  
\begin{align} \label{eq:errordyn}
    \dot e = \begin{cases} (A-LC)e \IfDist{+d(t,x)} & \textrm{if} \quad t\notin \mathcal T_a, \\
    (A-\tilde L\tilde C)e\IfDist{+d(t,x)} & \textrm{if} \quad t\in \mathcal T_a\end{cases}
\end{align}
with $ e(0) = x(0) - \hat x(0)$. Next, we analyze the error bounds when there is no attack, i.e., at each $t\notin \mathcal T_a$.

\subsection{Analysis under No Attacks}

%Without loss of generality, let $\pn{t_0} \in %\underset{i}{\bigcup} \{t_2^i\}
Consider the starting instant of an interval during which there is no attack on the system output, namely 
$t_2^i \in
\mathcal{T}_2\cup \{0\}$, with $i \in \mathbb{N}$
. %and define $e_0 \coloneqq e(t_0)$. 
%\pn{Since the system is time-invariant, it is possible to consider $t_0 = 0$ without loss of generality.} 
{The following result is the initial step to guarantee conditional invariance of $S$ with respect to $X_0$ %, \hat X_0$ 
for the  system \eqref{eq: actual system} when there are no attacks.}
% ----------------------------------------------------------------------------------------------------------------
% ----------------------------------------------------------------------------------------------------------------
% ----------------------------------------------------------------------------------------------------------------

\begin{Lemma}
\pno{Given system (\ref{eq: actual system}), suppose Assumption \IfDist{\ref{assum: d bound}-}\NotDist{\ref{assum A B cont}} %and \ref{ass:bounderror}
holds.}
For given $\IfDist{\gamma,}
T_{na}, \bar e_0>0$,
an associated \pno{observer} (\ref{eq: hat x switch}),
and corresponding error dynamics (\ref{eq:errordyn}),
%with initial error \NotDist{satisfying} $|e_0|\NotDist{\leq \bar{E}}$, 
%
\pno{if at the} $i-$th interval of no attacks with {$i \in \mathbb{N}$,} 
% 
%for  $T_{na}$ seconds 
%from $t = 0$, 
{ $|e(t_2^i)| \leq \bar e_0$ with $t_2^i\in \mathcal{T}_2$,  then}
the \pno{state estimation} error satisfies %$|e(t)| \leq \gamma_1(t) \bar E$ for all $t \in [0, t_1^1]$, and 
$|e(t)|\leq \pn{\gamma_1}(t{-t_2^i})%(T_{na})
\bar e_0$ for all $t \in  [t_2^i, t_1^{i+1}]$, %[T_{na}, t_2^1]$, 
%\pn{$i \in \mathbb{N}$,} 
where  
% $t_2^i = 0$ for $i=0$ and
\begin{align}\label{eq: C1}
    \pn{\gamma_1}(t)%(T_{na})
    \coloneqq c_1 \exp\left({-\bar \lambda_1
    %T_{na}
    t}\right)
\end{align}
with %where
$\bar \lambda_1 = \frac{\lambda_m(Q)}{2\lambda_M(P)}$, ${c_1} = \sqrt{\frac{\lambda_M(P)}{\lambda_m(P)}}$, 
\pno{ and $L$ such that}
for some \pno{symmetric} positive definite matrices $P$ \pno{and} $Q$, { $-Q=(A-LC)^\top P + P(A-LC)$} \pno{holds}.
%\left(\frac{\lambda_M(P)}{\lambda_m(P)}\right)^{1/2}\exp\left({-\frac{(1-\theta) \lambda_m(Q)\lambda_m(P)}{2\lambda_M(P)}T_{na}}\right) \bar E$ 
% and $\theta \in (0,1)$.
%\NotDist{and } the matrices $P, Q$ are symmetric, positive definite and such that $-Q=(A-LC)^\top P + P(A-LC)$
%\IfDist{, and  $\bar E= \frac{2\gamma|P|}{\theta \lambda_m(Q)} $}.
% 0) Show the maximum error during T_{na} seconds without an attack for a given bound in the initial error.
 \label{Lemma: Bound un attack}
\end{Lemma}

% ----------------------------------------------------------------------------------------------------------------
% ----------------------------------------------------------------------------------------------------------------
% ----------------------------------------------------------------------------------------------------------------
\NotACC{
\begin{proof}
Since the pair $(C, A)$ is detectable, it follows that there exist positive definite matrices $P, Q$ such that $V:\mathbb R^n\rightarrow\mathbb R_{\geq 0}$, defined as $V(e) \coloneqq e^TPe$, satisfies
%\begin{align*}
%    \dot V(e)  = -e^TQe \IfDist{+ 2e^TPd(t,x)}
%\end{align*}
$$
\begin{aligned}
\dot{V}(e)&=\dot{e}^{\top} P e+e^{\top} P \dot{e}\\ &=[(A-L C) e\IfDist{+d}]^{\top} P e+e^{\top} P[(A-L C) e \IfDist{+d}] \\
&=e^{\top}(A-L C)^{\top} P e\IfDist{+d^{\top} P e}+e^{\top} P(A-L C) e\IfDist{+e^{\top} P d} \\
&=-e^{\top} Q e\IfDist{+2 e^{\top} P d}.
\end{aligned}
$$
%where $-Q=(A-L C)^{\top} P+P(A-L C)$\IfDist{ and $P$ is symmetric}.
%
\IfDist{Under Assumption \ref{assum: d bound}, and g}\NotDist{G}iven that $ \lambda_m(Q) e^\top  e  \leq e^\top Q e$ for all $e \in \mathbb{R}^n$, we obtain
\begin{align}
    \dot V(e)\leq -|e|^2\lambda_m(Q) \IfDist{+ 2|e||P| \pn{\gamma}}.
    \label{bound1Vd}
\end{align}
%Let $c_1 = \lambda_m(Q)$. 
% By denoting $c_1 = \lambda_m(Q)$, \IfDist{$c_d = 2|P|\gamma$, }for each $\theta\in (0,1)$, re-write (\ref{bound1Vd}) as
% \begin{align*}
%     \dot V(e)\leq -(1-\theta)c_1|e|^2 -\theta c_1|e|^2 \IfDist{+ c_d|e|}.
% \end{align*}
% Note that \IfDist{for $|e|\geq \frac{c_d}{\theta c_1} = \frac{2\gamma|P|}{\theta\lambda_m(Q)}$, it holds that }$-\theta c_1|e|^2 \IfDist{+ c_d|e|}\leq 0$\IfDist{.}\NotDist{,} 
% \IfDist{Define the set $E \coloneqq \left\{e\in \mathbb R^n\; :\; |e|\geq \frac{2\gamma|P|}{\theta \lambda_m(Q)}\right\}$ }so that it holds that
% \begin{align}
%     \dot V(e) \leq -(1-\theta)c_1|e|^2 \IfDist{\quad \quad e\in  E}.
%     \label{eq:boundVdote}
% \end{align}
% Since $P$ is symmetric, we have 
% \begin{equation} 
% \lambda_m(P)|e|^2 \leq e^\top P e \leq \lambda_M(P)|e|^2
% \label{eq:boundP}
% \end{equation}
% which, together with (\ref{eq:boundVdote}) imply that 
% $$\dot{V}(e)\leq -\frac{\lambda_m(P)(1-\theta)c_1}{\lambda_M(P)}|e|^2     \IfDist{\quad \quad e(t)\in  E}.$$
% Note that it follows that 
% $$V(e(t))\leq V(e(0)) \exp\left(-\frac{\lambda_m(P)(1-\theta)c_1}{\lambda_M(P) \pn{\lambda_m(P)}}t %(t-t_0)}
% \right)     \IfDist{\quad \quad e(t)\in  E}.$$
% %
% Since $V(e_0) \leq \lambda_M (P) |e_0|^2$, using the left inequality in (\ref{eq:boundP}), and by denoting $c_2=\frac{\lambda_m(P)}{\lambda_M(P)}$, it follows that
% \begin{align*}\lambda_m (P) |e(t)|^2\leq \lambda_M (P) \exp\left({-\frac{(1-\theta)c_1c_2}{\pn{\lambda_m(P)}} t%(t-t_0)
% }\right) |e_0|^2    \IfDist{\\ \quad \quad e(t)\in  E}\end{align*}
% which implies that
% \begin{align*}
%     |e(t)|\leq c_2^{\pn{-1/2}} \exp \left({-\frac{(1-\theta)c_1 c_2}{2\pn{\lambda_m(P)}}t%(t-t_0)
%     }\right)|e_0|%|E|
%     \IfDist{\quad \quad e(t)\in  E}.
% \end{align*}
% the error is bounded and the bound decreases with time.
Since $P$ is symmetric, we have 
\begin{equation} 
\lambda_m(P)|e|^2 \leq V(e) \leq \lambda_M(P)|e|^2,
\label{eq:boundP}
\end{equation}
for all $e\in \mathbb R^n$, which, together with \eqref{bound1Vd} implies that 
$\dot{V}(e)\leq -\frac{\lambda_m(Q)}{\lambda_M(P)}V(e)    \IfDist{\quad \quad e(t)\in  E}$. Let $t \mapsto e(t)$ be a solution of \eqref{eq:errordyn} and consider the interval $[t_2^i, t_1^{i+1}]$ for $i \in \mathbb{N}_{>0}$. It follows that 
$V(e(t))\leq V(e(t_2^i)) \exp\left(-\frac{\lambda_m(Q)}{\lambda_M(P)}(t-t_2^i) %(t-t_0)}
\right)     \IfDist{\quad \quad e(t)\in  E}$ for all  $t \in [t_2^i, t_1^{i+1}]$.
%
Since $V(e(t_2^i)) \leq \lambda_M (P) |e(t_2^i)|^2$, using the left inequality in (\ref{eq:boundP}), it follows that
\begin{align*}\lambda_m (P) |e(t)|^2\leq \lambda_M (P) \exp\left({-\frac{\lambda_m(Q)}{\lambda_M(P)} (t{-t_2^i})
}\right) |e{(t_2^i)}|^2    \IfDist{\\ \quad \quad e(t)\in  E}\end{align*}
{for all $t \in [t_2^i, t_1^{i+1}]$}, 
which thanks to $|e{(t_2^i)}|\leq \bar e_0$, %$c_2=\frac{\lambda_M(P)}{\lambda_m(P)}$ 
implies that
\begin{align*}
    |e(t)|\leq c_1 \exp \left({-\bar \lambda_1(t{-t_2^i})
    }\right)|e{(t_2^i)}|%|E|
    \IfDist{\quad \quad e(t)\in  E}
    \leq \gamma_1(t{-t_2^i})\bar e_0,
\end{align*}
for all  $t \in [t_2^i, t_1^{i+1}]$,
completing the proof. 
\end{proof}
}
% ----------------------------------------------------------------------------------------------------------------
% ----------------------------------------------------------------------------------------------------------------
% ----------------------------------------------------------------------------------------------------------------

\IfDist{Furthermore, given that the rate of growth of the bound is negative, it holds that for each $e(t_0)\in \mathbb R^n\setminus E$, $e(t)\in \mathbb R^n\setminus E$ for all $t\geq t_0$ without an attack. }

Notice that the above analysis (with a nominal Luenberger observer) can be used to show that starting from $e(t_2^i) $ {with $t_2^i \in \mathcal{T}_2 \cup \{0\}, i\in \mathbb{N}$}, the error exponentially converges to  $\delta \mathbb B$ in time $T_{na}$, where
$
    \delta = %c_2^{1/2}\exp\left({-\frac{(1-\theta)c_1 c_2}{2}T_{na}}\right)|e_0| = 
    \gamma_1(T_{na})|e(t_2^i)|
$, and stays in that ball \pno{until} the next attack starts at $ t^{i+1}_1$. %\pn{Though this is not what we do, because we bound only based on the error at time zero.}

% {\color{blue}\textbf{Future Work}: Design a finite-time observer and get a faster convergence.} 

\begin{Remark}
The Luenberger observer used when there are no attacks is just one choice of a \pno{state estimator}. It is \pno{also} possible to use a finite-time stable \pno{state estimator} \cite{shen2008semi}, or any other observer %, in place of a Luenberger observer for the case when there is no attack,
that has faster convergence guarantees. 
% In particular, a finite-time state-estimator will give a better convergence bound than the one obtained in Lemma \ref{Lemma: Bound un attack}.%; however, in the interest of space, we leave using a finite-time observer for future work.  
\end{Remark}
%Next, we analyze the error bound under attack. 

\subsection{Analysis under Attacks}
During the attack on the output, we use a different observer gain designed for the pair \pno{$(\tilde C, A)$}. Since it might not be possible to place all the eigenvalues of $A-\tilde L\tilde C$ in the open left-half plane, {the matrix $\tilde L$ in \eqref{eq: hat x switch} can be designed to minimize the maximum eigenvalue of $A-\tilde L\tilde C$, which minimizes the rate of growth of the error during \pno{attacks}. %Formally, {the matrix $\tilde L$ can be given as}
%\begin{align}
%    \tilde L = \argmin_{\hat L\in \mathbb R^{n\times \tilde p}}\left(\lambda_M(A-\hat L\tilde C)\right)
%\end{align}}\\
%and 
Based on $\tilde L$, we compute the maximum growth rate possible in the estimation error $e$ during intervals of attacks in the system output, %i.e., for $t \in [t_0, t_0 +T_a]%$ such that $[t_a,t_b]\in \mathcal T_a$, 
assuming a worst-case attack.

Under the attack model \eqref{eq: attack model}, a subset of the state space may still be \pno{detectable} for the pair $(\tilde C, A)$. Thus, under the \pno{observer} \eqref{eq: hat x switch} for $t\in \mathcal T_a$, it is possible that some of the eigenvalues of the matrix $A-\tilde L\tilde C$ are in the open left-half plane. \pno{To bound} the error growth during the attack, we \pno{consider the general case in which we can decompose} the matrix $A-\tilde L\tilde C$ into submatrices $\hat A_{11}$ and $\hat A_{22}$, such that the eigenvalues of $\hat A_{11}$ are in the open left-half plane. To this end, let $\Phi \in \mathbb{R}^{n \times n}$ be an invertible matrix consisting of the generalized eigenvectors of the matrix $A-\tilde L\tilde C$ such that %$\Phi \Phi^{-1} =I$ and 
%
\small{
\begin{align}\label{eq: Phi mat}{\small
\Phi^{-1} (A-\tilde L\tilde C)\Phi = \begin{bmatrix}\hat A_{11} & 0_{\tilde n \times (n-\tilde n)} \\ 0_{(n-\tilde n) \times \tilde n} & \hat A_{22}\end{bmatrix}}
\end{align}}\normalsize
where $\hat A_{11}$ and $\hat A_{22}$ are Jordan blocks such that $\lambda_M (\hat A_{11})<0$ and $0_{p\times q}\in \mathbb R^{p\times q}$ is a matrix consisting of zeros\footnote{Note that it is always possible to find the Jordan form of the matrix $A-\tilde L\tilde C$, even when it is not diagonalizable.}. Also, let {$\Phi^{-1}=\big [\hat{\Phi}_1^\top,  \hat{\Phi}_2^\top \big ]$}, and define the change of coordinates $z=\Phi^{-1} e$. Then, $e=\Phi z$, and in the new coordinates, the error dynamics are expressed as
% 
\pno{\small
\begin{align*}
%
    \dot z = \Phi^{-1} \dot e %=\Phi^{-1} (A-\tilde L \tilde C)e \IfDist{ + \Phi^{-1} d(t,x)} & = \Phi^{-1} (A-\tilde L \tilde C)\Phi z \IfDist{+ \Phi^{-1} d(t,x)} \\
        & =  \begin{bmatrix}\hat A_{11} & 0_{\tilde n \times (n-\tilde n)} \\ 0_{(n-\tilde n) \times \tilde n} & \hat A_{22}\end{bmatrix} z \IfDist{+ \Phi^{-1} d(t,x)}.
%       
\end{align*}}

Define $z=(z_{11},z_{22})$, where $z_{11} \in \mathbb{R}^{\tilde n}$ and $z_{22} \in \mathbb{R}^{n-\tilde n}$ so that we have
\IfACC
{
$\dot{z}=(\dot z_{11},\dot z_{22})  = (\hat A_{11} z_{11}, \hat A_{22} z_{22}).$
}
{
\begin{align}
 \dot z_{11} &= \hat A_{11} z_{11} \IfDist{+ \Phi_1 d(t,x)} \label{eq:dz11}  \\
      \dot z_{22} &= \hat A_{22} z_{22}. \IfDist{+ \Phi_2 d(t,x)} \label{eq:dz22}
\end{align}
}
We can now state the following result \pno{providing a} bound on the state estimation error under attacks. 
 %where $d_{1}$ and $d_{2}$ are the vectors composed by the first $l$ and last $n-l$ elements of $d(t,x)$, respectively. 
% Thus, we have
  %\begin{align*}
  %  z_{11}(t) &= \exp (\hat A_{11} t) (z_{11}(0)+\hat{A}_{11}^{-1} \Phi_1 d)- \hat{A}_{11}^{-1}\Phi_1 d\\
  %        z_{22}(t) &= \exp (\hat A_{22} t) (z_{22}(0)+\hat{A}_{22}^{-1} \Phi_2 d) - \hat{A}_{22}^{-1} \Phi_2 d
 %\end{align*}
 
%   \begin{align*}
%    z(t) =& \begin{bmatrix}\exp (\hat A_{11} t) && 0 \\ 0 &&  \exp (\hat A_{22} t)\end{bmatrix} (z(0)+\begin{bmatrix} %\hat{A}_{11}^{-1} &&0  \\0&& \hat{A}_{22}^{-1} \end{bmatrix} \Phi^{-1} d)\\&- 
%    \begin{bmatrix} \hat{A}_{11}^{-1} &&0  \\0&& \hat{A}_{22}^{-1} \end{bmatrix} \Phi^{-1} d
% \end{align*}
% \subsection{Divergence when there is an attack}
% Under an attack, a partial state-observer can be designed as given in \eqref{eq: hat x switch} with $\tilde L \in \mathbb R^{n \times \tilde p}$ chosen such that there are $\tilde n$ (with $\tilde n\leq n$) eigenvalues of the matrix $(A-\tilde L\tilde C)$ in the open left-half plane, where $\tilde n$ is the rank of the observability matrix for the pair $(\tilde C, A)$. 
% Consider an attack launched at time $t_0 \in \mathcal{T}_1$ of maximum duration $T_a$, then the ultimate bound on the error satisfies:
%\begin{align*}
%    |e(t)| \leq (\tilde c_1 \exp\left({-\lambda_1 (t-t_0)}\right) + \tilde c_2 \lambda_2\exp\left({\lambda_2 (t-t_0)}\right)) |e(t_0)|
%    \IfDist{\\+\frac{\tilde c_2 \gamma }{\lambda_2} \exp(\lambda_2(t-t_0))}
    % \quad \\t\in [t_0, t_0+T_a],
% \end{align*}
% where  $\lambda_1$, $\lambda_2$, $\tilde c_1$, and $\tilde c_2$ are constants. 
% ----------------------------------------------------------------------------------------------------------------
% ----------------------------------------------------------------------------------------------------------------
% ---------------------------------------------------------------------------------------------------------------
\begin{Lemma}
\pno{Given system (\ref{eq: actual system}), suppose Assumption \IfDist{\ref{assum: d bound}-}\NotDist{\ref{assum A B cont}} %and \ref{ass:bounderror} 
holds.}
For given $%\gamma,
T_{a}, \bar{e}_0>0$, 
%$\hat c_1, \hat c_2, \lambda_1, \lambda_2>0$
%initialized in the set $X_0$, \pn{\textbf{where $X_0$ is}}
\pno{ an associated observer} (\ref{eq: hat x switch}), %initialized at $\hat{X}_0=X_0 \oplus \frac{\epsilon}{\gamma_2}$,
and corresponding error dynamics (\ref{eq:errordyn}),
%with $e(0)=e_0$ \NotDist{satisfying} $|e_0| \NotDist{\leq \bar{E}}$, 
% under 
%
\pno{if at the} $i-$th interval of attack with {$i \in \mathbb{N}_{>0}$} 
%If there is an attack launched at a time $t_0 \in \mathcal{T}_1$ 
and maximum length $T_{a}$,
{ $|e(t_1^i)| \leq \bar e_0$ with $t_1^i\in \mathcal{T}_1$,  then}
%
%the state estimation is initialized at \pn{$\hat{X}_0=X_0 \oplus \frac{\epsilon}{\gamma_2}$,} where $T_a \mapsto \gamma_2(T_a):=\max_{t\in [0, T_a]}(\hat c_1 \exp\left({-\lambda_1 t}\right) + \hat c_2 \lambda_2\exp\left({\lambda_2 t}\right))$,
%
%then
%
the state estimation error satisfies $|e(t)|\leq { %\epsilon(\bar E, T_{na})=
\gamma_2(T_a) \bar e_0}$ for all $t\in [t_1^i, t_2^i]$, 
where 
\begin{align}\label{eq: C2}
    \gamma_2(T_a) \coloneqq \max\limits_{t\in [0, T_a]} \hat c_1 \exp\left({-\hat\lambda_1 t}\right) + \hat c_2 \exp\left({\hat\lambda_2 t}\right)
\end{align}
%
with 
\footnotesize{
\IfInt{
\begin{align}\label{eq: c1 c2 params}
    & \hat c_1 = |\Phi||\hat \Phi_1|\sqrt{\frac{\lambda_M(\hat P)}{\lambda_m(\hat P)}}, &
    \hat c_2 & = |\Phi||\hat \Phi_2|,
    \\ 
    & \hat\lambda_1 = \frac{\lambda_m(\hat Q)}{2\lambda_M(\hat P)},
    &\hat \lambda_2 & = |\hat A_{22}|,
\end{align}
}{
$$
\hat c_1 = |\Phi||\hat \Phi_1|\sqrt{\frac{\lambda_M(\hat P)}{\lambda_m(\hat P)}}, 
\>\>\>\>
    \hat c_2  = |\Phi||\hat \Phi_2|,
    \>\>\>\>
     \hat\lambda_1 = \frac{\lambda_m(\hat Q)}{2\lambda_M(\hat P)},
     \>\>\>\>
    \hat \lambda_2  = |\hat A_{22}|,
$$}}\normalsize
\noindent \pno{and $\tilde L$ such that} for some \pno{symmetric} positive definite matrices $\hat P$ \pno{and} $ \hat Q$, $-\hat Q=\hat{A}_{11}^{\top} \hat P+\hat P\hat{A}_{11}$ \pno{holds}.
%
% $\hat c_1, \hat c_2, \lambda_1, \lambda_2>0$.
%$\theta \in (0,1)$, 
%the matrices $P, Q$ are positive definite and such that $V:\mathbb R^n\rightarrow\mathbb R_{\geq 0}$ defined as $V(e) \coloneqq e^TPe$ satisfies
%\begin{align*}
%    \dot V(e)  = -e^TQe + 2e^TPd(t,x)
%\end{align*} along the error dynamics (\ref{eq:errordyn}), and 
%$\bar E= \frac{2\gamma|P|}{\theta \lambda_m(Q)} $.
%
 % Under an attack starting at X_0 at t_0,  you will remain in S for at least T_a 
 \label{Lemma: Bound u attack}
\end{Lemma}
%
\NotACC{
\begin{proof} 
%At a time $t$, during the interval of attack, the solution to the error dynamics (\ref{eq:errordyn}) is given by
%\begin{equation*}
%    e(t-t_0)=\exp\left({A-\tilde L \tilde C}\right) e(t_0) + \text{ a term with } d(t,x)
%\end{equation*}
%we want to prove that 
%\begin{equation*}
%    |\exp\left({A-\tilde L \tilde C}\right)| \leq
%    (c_1 \exp\left({-\lambda_1 (t-t_0)}\right) + c_2\exp\left({\lambda_2 (t-t_0)}\right))
%\end{equation*}
Consider the dynamics \eqref{eq:dz11}. Since by construction, the eigenvalues of the matrix $\hat{A}_{11}$ are in the open left-half plane, it follows that there exist positive definite matrices  $\hat P, \hat Q$ such that $\hat V:\mathbb R^l\rightarrow\mathbb R_{\geq 0}$ defined as $\hat V(z_{11}) \coloneqq z_{11}^\top \hat P z_{11}$ satisfies
\begin{align*}
    \dot {\hat V}(z_{11})  = -z_{11}^\top \hat Q z_{11} \IfDist{+ 2 z_{11}^\top \hat P \Phi_1 d(t,x)}
\end{align*}
%\pn{Proof. TO BE FIXED
%$$
%\begin{aligned}
%\dot{V}=\dot{e}^{\top} P e+e^{\top} P \dot{e} &=[(A-L C) e+d]^{\top} P e+e^{\top} P[(A-L C) e+d] \\
%&=e^{\top}(A-L C)^{\top} P e+d^{\top} P e+e^{\top} P(A-L C) c+e^{\top} P d \\
%&=-e^{\top} Q e+2 e^{\top} P d
%\end{aligned}
%$$
%where $-\hat Q=\hat{A}_{11}^{\top} \hat P+\hat P\hat{A}_{11}$ and $\hat P$ is symmetric.
%
% ------------------
\IfDist{Under Assumption \ref{assum: d bound}, and g}\NotDist{G}iven that $ \lambda_m(\hat Q) z_{11}^\top  z_{11} \leq z_{11}^\top \hat Q z_{11} $ for all $z_{11} \in \mathbb{R}^{\tilde n}$, we obtain
\begin{align}
    \dot{ \hat V}(z_{11})\leq -|z_{11}|^2\lambda_m(\hat Q) \IfDist{+ 2|z_{11}||\hat P| |\Phi_1| \gamma}.
    \label{bound1Vhd}
\end{align}
%By denoting $\breve c_1 = \lambda_m(\hat Q), \IfDist{\breve c_d = 2|\hat P||\Phi_1| \gamma}$ re-write (\ref{bound1Vhd}) as $\dot{\hat V}(z_{11})\leq -\breve c_1|z_{11}|^2$. 
% Note that \IfDist{for $|z_{11}|\geq \frac{\breve c_d}{\theta \breve c_1} = \frac{2\gamma|\hat P||\Phi_1|}{\theta\lambda_m(\hat Q)}$, it holds that} $-\theta \breve c_1|z_{11}|^2 \IfDist{+ \breve c_d|z_{11}|}\leq 0$\IfDist{.}\NotDist{,} 
% \IfDist{Define the set $\hat E \coloneqq \left\{z_{11}\in \mathbb R^l\; :\; |z_{11}|\geq \frac{2\gamma|\hat P||\Phi_1|}{\theta \lambda_m(\hat Q)}\right\}$} so that it holds that
% \begin{align}
%     \dot{\hat V}(z_{11}) \leq -(1-\theta)\breve c_1|z_{11}|^2 \IfDist{\quad \quad z_{11}\in  \hat E},
%     \label{eq:boundVhdote}
% \end{align}
% and thus, 
Let $t\mapsto (z_{11}(t), z_{22}(t))$ denote a solution of \eqref{eq:dz11}-\eqref{eq:dz22}. 
Denote 
$\hat \lambda_1 = \frac{\lambda_m(\hat Q)}{2\lambda_M(\hat P)}$ and 
$\breve c_1=\sqrt{\frac{\lambda_M(\hat P)}{\lambda_m(\hat P)}}$ to obtain that
\begin{align*}
    |z_{11}(t)|\leq \breve c_1 \exp \left({-\hat \lambda_1(t-t_1^i)}\right)|z_{11}(t_1^i)|%|E|
    \IfDist{\quad\\ z_{11}(t)\in  \hat E}.
\end{align*}
% ------------------
On the other hand, for \eqref{eq:dz22},
%
%we introduce the change of coordinates $\zeta_{22}=\frac{1}{2}z_{22}^\top z_{22}=\frac{1}{2}|z_{22}|^2$. Then, (\ref{eq:dz22}) expressed in the new coordinates is
%\begin{align*}
%    \dot{\zeta}_{22}&= \frac{1}{2}(z_{22}^\top A_{22} z_{22} + z_{22}^\top \Phi_2^\top d(t,x))
%                        +\frac{1}{2}(z_{22}^\top A_{22}^\top z_{22} + d^\top (t,x) \Phi_2 z_{22} )\\
%\end{align*}
%Which implies
%\begin{align*}
%    \dot{\zeta}_{22}& \leq |z_{22}|^2 |A_{22}|+ |z_{22}| |\Phi_{2}| \gamma   \\
%                      & = 2 \zeta_{22} |A_{22}| + \sqrt{\zeta_{22}} |\Phi_{2}| \gamma
%\end{align*}
given that $|\hat A_{22} z_{22} \IfDist{+ \Phi_2 d(t,x)}| \leq  |\hat A_{22}| |z_{22}| \IfDist{+ |\Phi_2| \gamma }$, we have
\begin{align*}
    |z_{22}(t)|\leq {|z_{22}(t_1^i)|}\IfDist{+\frac{|\Phi_2| \gamma} }{}\exp(|\hat{A}_{22}| (t-t_1^i)) \IfDist{- \frac{|\Phi_2| \gamma }{|\hat{A}_{22}|}}
\end{align*}
\IfDist{So 
\begin{align*}
    |z_{22}(t)|\leq \exp(|\hat{A}_{22}| (t-t_0)) \left({|z_{22}(t_0)|}+\frac{|\Phi_2| \gamma }{|\hat{A}_{22}|}\right)
\end{align*}}
Thus, by denoting ${\hat c_1=|\Phi||\hat \Phi_1|\breve c_1},% \lambda_1=\frac{\breve c_1}{2 \lambda_M(\hat P)},
$ and 
${\hat c_2=|\Phi||\hat \Phi_2|}, \hat\lambda_2=|\hat{A}_{22}| $, in the original coordinates we have
   \begin{align*}
    |e(t)| \leq& |\Phi| (|z_{11}(t)| + |z_{22}(t)|)\\
    \leq&  {|\Phi|\breve c_1
     |z_{11}(t_1^i)|
     \exp \left({-\hat\lambda_1(t-t_1^i)}\right)  }
     \\
    &+{|\Phi|
    {|z_{22}(t_1^i)|}\IfDist{+\frac{|\hat \Phi_2| \gamma} }{}}
    \exp(|\hat{A}_{22}| (t-t_1^i)) 
    \\ 
    =
    &  {|\Phi|\breve c_1
    |{\hat{\Phi}_{1}}e(t_1^i)|
    \exp \left({-\hat\lambda_1(t-t_1^i)}\right)  }
    \\
    &+{|\Phi|
    {|{\hat{\Phi}_2}e(t_1^i)|}\IfDist{+\frac{|\Phi_2| \gamma} }{}}
    \exp(\hat\lambda_2 (t-t_1^i)) 
    \\
    \leq & \left({\hat c_1} \exp ({-\hat\lambda_1(t-t_1^i)})  +\hat c_2 \exp(\hat\lambda_2 (t-t_1^i)) \right)|e(t_1^i)|
       \IfDist{
       \\ 
    &+\frac{\hat c_2 \gamma }{\lambda_2} \exp(\lambda_2 (t-t_1^i))}
 \end{align*}
Thus, thanks to $|e(t_1^i)| \leq \bar e_0$, for all $t\in [t_1^i, t_2^i]$ it follows that
\begin{align*}
    |e(t)|\leq % & \max_{t\in [t_0, t_0+T_a]}\left({\hat c_1} \exp ({-\lambda_1(t-t_0)})  \right.\\
    %& \hspace{40pt}\left.+\hat c_2 \exp(\lambda_2 (t-t_0)) \right)|e(t_0)| \\
     %=      \\ \leq &
     \max_{t\in [0, T_a]}\left({\hat c_1} \exp ({-\hat\lambda_1 t}) %\right. \\
     %& \hspace{40pt}\left. 
     +\hat c_2 \exp(\hat\lambda_2 t) \right)\bar e_0%|e(t_1^i)|
\end{align*}
which completes the proof. 
%or
%  \begin{align*}
%    |e(t)| =& |\Phi z(t)| = |\Phi \begin{bmatrix}z_{11}(t) \\ z_{22}(t) \end{bmatrix}| \\
%    \leq& |\Phi| \sqrt{|z_{11}(t)|^2 + |z_{22}(t)|^2}\\
%     \leq& |\Phi| \sqrt{c_1^2 \exp \left({-2\lambda_1(t-t_0)}\right)|z_{11}(t_0)|^2  +}\\
    % &\overline{\exp(2|\hat{A}_{22}| (t-t_0)) \left({|z_{22}(t_0)|}+\frac{|\Phi_2| \gamma }{|\hat{A}_{22}|}\right)^2}\\
    % =& |\Phi| \sqrt{c_1^2 \exp \left({-2\lambda_1(t-t_0)}\right)|\Phi_{1}e(t_0)|^2  +}\\
    % &\overline{\exp(2\lambda_2 (t-t_0)) \left({|\Phi_2e(t_0)|}+\frac{|\Phi_2| \gamma }{|\hat{A}_{22}|}\right)^2}\\
    % =& |\Phi| \sqrt{c_1^2|\Phi_{1}|^2 \exp \left({-2\lambda_1(t-t_0)}\right)|e(t_0)|^2  +}\\
    % &\overline{|\Phi_2|^2\exp(2\lambda_2 (t-t_0)) )\left({|e(t_0)|^2 +
%     \frac{2{|e(t_0)|\gamma}}{|\hat{A}_{22}|}+\frac{ \gamma^2 }{|\hat{A}_{22}|^2}}\right)}\\
%     \leq& |\Phi| \left({c_1}|\Phi_{1}| \exp \big({-\lambda_1(t-t_0)}\right)  +|\Phi_2|\exp(\lambda_2 (t-t_0)) \big)|e(t_0)|\\ 
%     &+|\Phi| |\Phi_2|\exp(\lambda_2 (t-t_0))\left(\left(\frac{2{|e(t_0)|\gamma}}{|\hat{A}_{22}|} \right)^{1/2}
%     +\frac{ \gamma }{|\hat{A}_{22}|}\right)
%  \end{align*}
\end{proof}
}
%
\NotACC{Note that it is possible that $\tilde n = 0$, i.e., all the eigenvalues of the matrix $A-\tilde L\tilde C$ are in the closed right-half plane. In that case, $\hat c_1 = 0$ in \eqref{eq: C2}.} 
%
\pno{ \subsection{Global Bound on Estimation Error}
Before we state the first main result of the paper, we make the following assumption on the initial state estimation error. 
\begin{Assumption}
The closed set $S\subset \mathbb{R}^n$ %\mathcal D$
is such that there exists $\bar E>0$ {such that}, for the initial state $x(0)\in S$ %\mathbb R^n$ %the observer is initialized at 
and initial estimate 
$\hat x(0) \in S$, %\mathbb R^n$, % such that 
the error satisfies 
$|e(0)|=|x(0)- \hat x(0)|\leq \bar E$.
\label{ass:bounderror}
\end{Assumption}
%
A pre-defined initial error bound helps us guarantee the existence of a \pno{switching} observer of the form \eqref{eq: hat x switch general} such that \pno{safety is guaranteed}.

Now, we provide a result on bounds on the state estimation error under the proposed switching observer algorithm.
}

\begin{Theorem}
\pno{
Given system (\ref{eq: actual system}), suppose Assumptions \IfDist{\ref{assum: d bound}-}\NotDist{\ref{assum A B cont}} and \ref{ass:bounderror} hold for $\bar{E}>0$.
}
For given $T_{na}, T_a>0$,
an associated observer (\ref{eq: hat x switch}),
and corresponding error dynamics (\ref{eq:errordyn}), let $c_1,\bar \lambda_1,\hat c_1, \hat c_2, \hat\lambda_1, \hat\lambda_2>0$ be defined as per Lemma \ref{Lemma: Bound un attack} and Lemma \ref{Lemma: Bound u attack}.
%\pn{ If $T_{na}$ is such that $\gamma_1(T_{na})\leq 1$
%with $\gamma_1$ as in \eqref{eq: C1}, then 
%$  |e(t)|    \leq \bar E$ for all $t \in \mathcal{T}_1$.} 
    %On the other hand, i
    If $T_{na}$ and $T_a$ are such that
$\gamma_1(T_{na})\gamma_2(T_a) \leq 1$ with $\gamma_1$ as in \eqref{eq: C1} and $\gamma_2$ as in \eqref{eq: C2}, then 
    \pn{$|e(t)|\leq  \gamma_1(0)\gamma_2(T_a) \bar E$} for all $t\geq 0$. In addition,
    \begin{itemize}
    \item\pno{if there is an attack at time $t=0$, then
    $|e(t)|\leq  \bar E$ for all $t\in \mathcal{T}_1 \cup \{0\}$, and
    \item if the first attack is launched after at least $T_{na}$ seconds, then
    $|e(t)|\leq  \bar E$ for all $t\in \mathcal{T}_2 \cup \{0\}$.}
    \end{itemize}
\label{Th:TBoundError}
\end{Theorem}
\NotACC{
\begin{proof}
From Lemma \ref{Lemma: Bound un attack}, we have that for each 
interval without attacks starting at $t_2^i \in \mathcal{T}_2$ and ending at $t_1^{i+1} \in \mathcal{T}_1$, with $i \in \mathbb{N}$,
the estimation error satisfies 
\begin{multline}
%\scriptstyle 
|e(t_1^{i+1})|\leq c_1 \exp\left(-\bar \lambda_1 (t_1^{i+1} - t_2^i)
\right) |e(t_2^i)| \IfDist{+ \tilde c_2\gamma \exp(\lambda_2 T_a)\frac{1}{\lambda_2}}
\\
\leq c_1 \exp\left(-\bar \lambda_1 T_{na}\right) |e(t_2^i)|
=
\gamma_1(T_{na})|e(t_2^i)|
\label{eq: proofth1}
\end{multline}
%
The right-hand bound holds for any $t_1^{i+1}\geq T_{na}$ thanks to $c_1,\bar \lambda_1 >0$ % and $P\succ 0$ 
in Lemma \ref{Lemma: Bound un attack}, namely, thanks to the exponential being decrescent.
%Thanks to $\gamma_1(T_{na})\leq 1$, the error satisfies $  |e(\underline t)|
 %   \leq \bar E$ for any $\underline t$, which implies $|e( t)|$ for all $t \in \mathcal{T}_1$.\\
%
For each attack interval starting at $t_1^{i} \in \mathcal{T}_1$ and ending at $t_2^{i} \in \mathcal{T}_2$, with $i \in \mathbb{N}_{>0}$,
the estimation error satisfies 
\begin{multline}
{\scriptstyle 
|e(t_2^{i})|\leq \max\limits_{t\in [t_1^i, t_1^i+T_a]} \left\{{\hat c_1} \exp ({-\hat\lambda_1(t-t_1^i)})  +\hat c_2 \exp(\hat\lambda_2 (t-t_1^i)) \right\}|e(t_1^i)|}
\\
{\scriptstyle
=
\max\limits_{t\in [0, T_a]} \left\{{\hat c_1} \exp (-\hat\lambda_1(t))  +\hat c_2 \exp(\hat\lambda_2 (t)) \right\}|e(t_1^i)|
=\gamma_2(T_a)|e(t_1^i)|}
\label{eq: proofth2}
\end{multline}
%
%
Thus, %looking at the error after the first attack, notice that 
if there is an attack at the initial time, namely $t_1^1=0$,
%\pn{(I think this case should not be allowed; otherwise it will grow right away potentially to $\gamma_2(T_a)\bar E>\bar E$)}
from Assumption \ref{ass:bounderror}, we have $|e(t_1^1)|\leq \bar E$, 
from \eqref{eq: proofth2} we have $|e(t_2^1)|\leq \gamma_2(T_a) \bar E$, and 
from  
\eqref{eq: proofth1} we have $|e(t_1^2)|\leq \gamma_1(T_{na})\gamma_2(T_a) \bar E$. {If
$T_{na}$ and $T_a$ are such that $\gamma_1(T_{na})\gamma_2(T_a) \leq 1$, then $|e(t_1^2)|\leq \bar E$,} namely, at the beginning of the second attack, the error will be bounded by $\bar E$.
Recursively, this implies that $|e(t)| \leq \bar E$ at the beginning of every attack, namely, for all $t\in \mathcal{T}_1 \cup \{0\}$.
Notice that %since the function $\hat c_1 \exp({-\hat\lambda_1 t}) + \hat c_2 \exp({\hat\lambda_2 t})$ in \eqref{eq: C2} with $t\in [t_1^1, t_2^1]$ is convex, we have that 
the error is bounded during the first interval of attack by 
$|e(t)| \leq \gamma_2(T_a)\bar E$ for all $t \in [0, t_2^1]$.
Given that 
$\max\limits_{t\in [0, T_{na}]} \gamma_1(t)=\gamma_1(0)$,  we have that
$|e(t)|\leq \gamma_1(0)\gamma_2(T_a)\bar E$ for all $t\in [0,t_1^2]$.
Recursively, 
%the error satisfies$|e(t)|\leq  \gamma_1(0)\gamma_2(T_a) \bar E$ for all $t \in [t_2^i, t_2^{i+1}], i \in \mathbb{N}$.
this implies that $|e(t)| \leq \gamma_1(0)\gamma_2(T_a)\bar E$ for all $t \geq 0$.
%
%

On the other hand, if there is no attack at the beginning, we have at least $T_{na}$ seconds before the first attack is launched, namely $t_1^1>T_{na}$, 
and from Assumption \ref{ass:bounderror}, we have $|e(t_2^0)|\leq \bar E$, 
from \eqref{eq: proofth1} we have $|e(t_1^1)|\leq \gamma_1(T_{na}) \bar E$, and 
from \eqref{eq: proofth2}, $|e(t_2^1)|\leq \gamma_1(T_{na})\gamma_2(T_a) \bar E$. If
$T_{na}$ and $T_a$ are such that $\gamma_1(T_{na})\gamma_2(T_a) \leq 1$, then $|e(t_2^1)|\leq \bar E$, i.e., at the end of the first attack, the error will be bounded by $\bar E$. 
Recursively, this implies that $|e(t)| \leq \bar E$ at the end of every attack, namely, for all $t\in \mathcal{T}_2 \cup \{0\}$.
Notice that since 
$\max\limits_{t\in [0, T_{na}]} \gamma_1(t)=\gamma_1(0)$,  the error is bounded during the first interval without attacks by 
$|e(t)|\leq \gamma_1(0)\bar E$ for all $t\in [0,t_1^1]$. 
Given that the function $\hat c_1 \exp({-\hat\lambda_1 t}) + \hat c_2 \exp({\hat\lambda_2 t})$ in \eqref{eq: C2} with $t\in [t_1^1, t_2^1]$ is convex, we have that $|e(t)| \leq \gamma_1(0)\gamma_2(T_a)\bar E$ for all $t \in [0, t_2^1]$. Recursively, 
%the error satisfies$|e(t)|\leq  \gamma_1(0)\gamma_2(T_a) \bar E$ for all $t \in [t_2^i, t_2^{i+1}], i \in \mathbb{N}$.
this implies that $|e(t)| \leq \gamma_1(0)\gamma_2(T_a)\bar E$ for all $t \geq 0$.
\end{proof}
}

% \begin{figure}[t]
% 	\centering
% 	\includegraphics[width=\columnwidth,clip]{AttackBounds2.png}%error_bound_fig.png}
% 	\caption{Evolution of error under the conditions discussed above. Here, $\{t^i_1\}, \{t^i_2\}$ are the sequences when the attacks start and stop.}
% 	\label{fig:err e}
% \end{figure}

\begin{Remark}
{Consider a set $X_0$, %$S$, define $X_0 = S\ominus \varepsilon$ and
{and for a given $x_0\in X_0$ such that  $x(0)=x_0$, define the set $\hat X_0(x_0) :=\{x \in \mathbb{R}^n : x \in x_0+\bar E\mathbb{B} \}$. } Notice that thanks to Theorem \ref{Th:TBoundError}, for %each 
{each} $x_0\in X_0$, 
    %there exists
and each $\hat x_0\in  \hat X_0(x_0)$ %such that 
we have that each solution pair $t \mapsto \pno{(x(t),\hat x(t))}$ to \eqref{eq: actual system} from $x(0)=x_0, \hat x(0) = \hat x_0$ satisfies %$\gamma_2(T_a)|e_0|\leq\epsilon $,and 
%the error satisfies:
\begin{itemize}
    \item[1)] Boundedness of error at all times: $|x(t)-\hat x(t)|\leq \bar E$ for all $t\geq 0$\pno{;}
    \item[2)] Maximum error at the beginning of each attack: $|x(t_1^i)-\hat x(t_1^i)|\leq \gamma_1(T_{na})$ for each $i \in \mathbb{N}_{> 0}$ . 
\end{itemize}}
%Figure \ref{fig:err e} pictorially represents the bound on the error $e$ with time, where, 
Under \pno{an} attack, it is possible that the error grows, and when there is no attack, \pno{the error} decreases. However, \pno{using} the proposed observer, the norm of the error always remains bounded by $\gamma_1(0)\gamma_2(T_a)\bar E$, \pno{as long as Assumption \ref{ass:bounderror} on the initial estimation error holds}.
\end{Remark}




%\begin{align*}
%    \gamma_2 & = \max\{(c_1+c_2), (c_1 \exp\left({-\lambda_1 T_a}\right) + c_2\exp\left({\lambda_2 T_a}\right))\}
%\end{align*}

%For a given $\gamma_2, c_1, c_2$, solve for 
%$ (c_1 \exp\left({-\lambda_1 T_a}\right) + c_2\exp\left({\lambda_2 T_a}\right)) = \max\{\gamma_2, (c_1+c_2)\}$. 

% ----------------------------------------------------------------------------------------------------------------
% ----------------------------------------------------------------------------------------------------------------
%\begin{Corollary}
%Under similar conditions, if the maximum allowed error for an interval of attack is $XX$, the the maximum time of an attack allowed is ... 
%\end{Corollary}

% ----------------------------------------------------------------------------------------------------------------


\pno{\section{Observer-Based Feedback Law Design}}
\NotACC{In this section, we present a set construction process to solve part 1 of Problem $(\star)$ and, based on it,  a control design scheme that solves part 2. First, we define the sets that are going to be used in the control design. }

\pno{\subsection{Construction of Sets of Initial Conditions}}
% From the results in the previous section, we have that $|\hat x(t)-x(t)|\leq \varepsilon = \gamma_1\gamma_2\bar E$ for all $t\geq 0$. 
%
%Set $X_0 = S\ominus \varepsilon$ and \pn{for a given $x_0\in X_0$ set  $x(0)=x_0$, and define the set $\hat X_0(x_0) :=\{x \in \mathbb{R}^n : x \in x_0+\bar E\mathbb{B} \cap X_0 \}$. }  Thus, for each $x_0 \in X_0$ and $\hat x_0\in \hat X_0$, it holds that $|x_0-\hat x_0|\leq \bar E$ \pn{and $\hat x(0)\in X_0$}. Furthermore, given that $|\hat x(t)-x(t)|\leq \varepsilon = \gamma_1\gamma_2 \bar E$ (which follows from Theorem \ref{Th:TBoundError}), it holds that $\hat x(t)\in X_0 \implies x(t)\in S$. Thus, the control objective is to keep the trajectory $\hat x(\cdot)$ in the set $X_0$ at all times. 
%
Consider a closed set $S  \subset \mathbb{R}^n$, %\mathcal{D}$,  
$T_a, T_{na}>0$, maps $\gamma_1$ and $ \gamma_2$ as in \eqref{eq: C1} and \eqref{eq: C2}, and $\bar E>0$ in Assumption \ref{ass:bounderror}. 
\pno{Pick} \pn{$\varepsilon >(1+\gamma_1(0)\gamma_2(T_a))\bar E$}
%\bar E(\gamma_1(0)\gamma_2(T_a)+1)$. 
Define the set of initial states as 
\small{
\begin{align}\label{eq: set X0}
 X_0 \coloneqq S \setminus (\partial S + \varepsilon \mathbb B) .
\end{align}
}\normalsize
Note that under Assumption \ref{ass:bounderror}, \pno{$X_0$ is nonempty.} Now, given $x_0\in X_0$, set  $x(0)=x_0$ and define the \pno{set-valued map} 
\small{
\begin{align}\label{eq: set hat X0}
    \hat X_0(x_0) \coloneqq x_0 + %\gamma_1(0)\gamma_2(T_a)
    \bar E\mathbb{B}.
\end{align}
}\normalsize
Thus, for each $x_0 \in X_0$ and $\hat x_0\in \hat X_0(x_0)$, it holds that $|x_0-\hat x_0|\leq \bar E$. Additionally, notice that {$\hat x_0\in \tilde{X}$}, 
where 
\small{
\begin{align}\label{eq: tilde X}
    \tilde X \coloneqq X_0 + %\gamma_1(0)\gamma_2(T_a)
    \bar E \mathbb B
\end{align}
}\normalsize
\pno{which is an inflation of $X_0$ by $ %\gamma_1(0)\gamma_2(T_a)
\bar E$.}
\pno{This} construction of the sets of initial conditions, namely, $X_0$ and $\hat X_0$, \pno{leads to} conditional invariance of $S$, as shown below.

\begin{Lemma}\label{lemma: bounded error}
Given the system \eqref{eq: actual system}, the observer \eqref{eq: hat x switch}, the observer-based feedback law $\kappa$ \eqref{eq: u switch}, a closed set $S \subset \mathbb{R}^n$, %\mathcal{D}$,
$X_0$ as in \eqref{eq: set X0}, and $\hat X_0$ as in \eqref{eq: set hat X0}, consider the solution $t \mapsto (x(t),\hat x(t))$ to the resulting closed-loop system from the composition of \eqref{eq: actual system} and \eqref{eq: hat x switch} with $\kappa$ from $x(0)\in X_0$, $\hat x(0)\in \hat X_0(x(0))$ and $T_a, T_{na}, \bar E$,  such that conditions of Theorem \ref{Th:TBoundError} are satisfied. If $S \setminus (\partial S + \pn{(1+\gamma_1(0)\gamma_2(T_a))\bar E\mathbb B})\neq \emptyset$ and $ \hat x(t) \in \tilde X$ for all $t \geq 0$, then $ x(t) \in S$ for all $t \geq 0$. 
\end{Lemma}
%
\NotACC{
\begin{proof}
 From the definition of the sets $X_0$ and $\tilde X$, and $\varepsilon \geq (1+\gamma_1(0) \gamma_2(T_a)) \bar E $ 
 it follows that 
 $ X_0 \subset S \setminus (\partial S + (1+ \gamma_1(0) \gamma_2(T_a)) \bar E \mathbb B)$ and 
  $\tilde X \subset  S \setminus (\partial S + \gamma_1(0) \gamma_2(T_a)\bar E\mathbb B )$.
 Thus, under the assumption that $\hat x(t) \in \tilde X$ for all $t \geq 0$
with 
$|\hat {x}(t) - x(t)| \leq \gamma_1(0)\gamma_2(T_a)\bar E$ for all $t \geq 0$,
%From Theorem \ref{Th:TBoundError}, it follows that $|e(t)|\leq \gamma_1(0)\gamma_2(T_a)\bar E$, which implies that {$|\hat x(t)-x(t)| \leq \gamma_1(0)\gamma_2(T_a) \bar E$} at all times. 
 it follows that
%With $\varepsilon \geq \bar E (\gamma_1(0) \gamma_2(T_a) +1)$, if we can guarantee that $\hat x(t)\in \tilde {X}$ at all times, then it would follow that $x(t)\in S$ at all times. 
 $x(t) \in S$ for all $t \geq 0$.
\end{proof}
}
%
In words, the set of initial states $X_0$ and the set of initial estimates $\hat X_0$ are defined such that the initial estimation error is upper bounded by $\bar E$. 
%
Furthermore, we define $\tilde X$ in \eqref{eq: tilde X} as the set resulting from an inflation of $X_0$ by $%\gamma_1(0)\gamma_2(T_a)
\pn{\bar E}$. Under this construction, 
\pno{
for the resulting closed-loop system from the composition of \eqref{eq: actual system} and \eqref{eq: hat x switch} with $\kappa$,}
forward invariance of $\tilde X$ for the observer \eqref{eq: hat x switch} implies conditional invariance of the set $S$ for the system \eqref{eq: actual system}
with respect to $X_0$.
%, \hat X_0$.
% When the conditions of Theorem \ref{Th:TBoundError} are satisfied, and the set $X_0$ is a deflation of the set $S$ of more than $2\bar E$ (the maximum initial error plus the maximum error during the evolution of the system), invariance of the set $S$ for \eqref{eq: cl system} follows from the invariance of the set $\tilde{X}$ for \eqref{eq: hat x switch general}. 
%
Thus, the control objective is to enforce {the estimate $\hat x$ in the set $\tilde X$} at all times to guarantee safety of $S$.
%

\subsection{QP-based Feedback Law Synthesis}
We use a control barrier function (CBF)-based approach for guaranteeing forward invariance of a subset $\bar X$ of the set $\tilde X$ in \eqref{eq: tilde X} for \eqref{eq: hat x switch} (see \cite{ames2017control}). In order to use CBF for forward invariance, we need a zero sublevel set representation of the set $\bar X$. To this end, consider the function $h:\mathbb R^n\rightarrow\mathbb R$ and define a set
{\small
\begin{align}\label{eq: bar X}
    \bar X \coloneqq \{\hat x\; |\; h(\hat x)\leq 0\}\subset {\tilde X}.
\end{align}
}
% {namely, a function whose 0-sublevel set is contained in the set $\tilde X$}. 

Given an observer-based feedback law $\kappa$ assigning the input $u = \kappa(t, \hat x,\bar y)$ of \eqref{eq: hat x switch}, consider a solution $t \mapsto \hat x(t)$ to \eqref{eq: hat x switch} from $\hat x(0)\in \bar X$. \pno{For the given measurement $\bar y$,} it is sufficient to ensure that for each $\hat x(0)\in \bar X$, %the solution $\hat x(\cdot)$ of \eqref{eq: hat x switch} 
the estimate satisfies $\hat x(t)\in \bar X \subset \tilde X$, for all $t\geq 0$. The CBF condition for guaranteeing this %\eqref{eq: hat x switch} 
when there is no attack is:
%
\small{
\begin{align}\label{eq: CBFCondnA}
    \frac{\partial}{\partial \hat x}h(\hat x(t))& \left(A\hat x(t)+ B\kappa_1(\hat x(t), \bar y(t))\right. \nonumber \\
    & \left.+ L(\bar y(t)-C\hat x(t))\right)\leq \alpha_1(-h(\hat x(t))), 
\end{align}}\normalsize
for all $t\geq 0$, where $t \mapsto \bar y(t)$ is the measured output signal, and the CBF condition under attack is
\small{
\begin{align}\label{eq: CBFConduA}
    \frac{\partial}{\partial \hat x}h(\hat x(t))& \left(A\hat x(t)+ B\kappa_2(\hat x(t), \bar y(t))\right. \nonumber \\
    & \left.+ \tilde L(\bar y_s(t)-\tilde C\hat x(t))\right)\leq \alpha_1(-h(\hat x(t))), 
\end{align}}\normalsize 
for all $t\geq 0$, where $t \mapsto y_s(t)$ is the secured output signal and $\alpha_1, \alpha_2$ are class-$\mathcal K$ functions. 
% Using the results from \cite{garg2022fixed}, 
We can use a Quadratic Programming (QP) formulation to compute the input $u$ in the respective cases. 

Consider the following QP for each $\hat x\in {\bar X}$ {and $\bar y$ such that $x \in S$} for input synthesis when there is no attack:
\small{
\begin{subequations}\label{QP no attack}
\begin{align}
\hspace{10pt}\min_{(v, \eta)} \quad \frac{1}{2}|v-K\hat x|^2 + & \frac{1}{2}\eta^2\\
     \textrm{s.t.} \;\quad \; \; %Av  \leq &  \; b, \label{C1 cont const}\\
    \frac{\partial}{\partial \hat x}h(\hat x)\left(A\hat x+ Bv + L(\bar y-C\hat x)\right) \leq & -\eta h(\hat x),
\end{align}
\end{subequations}}\normalsize
where $K$ is the optimal LQR gain for the pair $(A,B)$. 
%, $l_B$ is the Lipschitz constants of the function $B$. 
Next, we use a similar QP to compute the input under attack. Consider the following QP for each $\hat x \in {\bar X}$ {and $y_s = \tilde C x$ such that $x \in S$}:
\small{
\begin{subequations}\label{QP under attack}
\begin{align}
\hspace{10pt}\min_{(v_s, \zeta)} \quad \frac{1}{2}|v_s-K\hat x|^2 + & \frac{1}{2}\zeta^2\\
     \textrm{s.t.} \;\quad \; \; %Av_s  \leq &  \; b, \label{C1 cont const k}\\
    \frac{\partial}{\partial \hat x}h(\hat x)\left(A\hat x+ Bv_s + {\tilde L}(y_s-\tilde C\hat x)\right) \leq & -\zeta h(\hat x).
\end{align}
\end{subequations}}\normalsize  
% \pn{where $\zeta>0$ is a constant.}
{The objective functions in \eqref{QP no attack} and \eqref{QP under attack} set the convex minimization problem to obtain the closest control action to the LQR control that satisfies the constraints. The additional decision variables, namely $(\eta, \zeta)$, respectively, are slack variables.}
Denote the solutions to \eqref{QP no attack} and \eqref{QP under attack} as {$t \mapsto u_1^*(\hat x(t),\bar y(t))$ and $t \mapsto u_2^*(\hat x(t), \bar y(t))$}, respectively. \pno{To} guarantee continuity of these solutions with respect to $\hat x$, we need to impose the strict complementary slackness condition (see \cite{garg2022fixed}). In brief, if the $i-$th constraint of \eqref{QP no attack} (or \eqref{QP under attack}), with $i \in \{1, 2\}$, is written as $G_i(\hat x,\bar y,u_{QP})\leq 0$ with $u_{QP} = (v,\eta)$ (respectively, $u_{QP} = (v_s, \zeta)$ for \eqref{QP under attack}), and the corresponding Lagrange multiplier is $\bar\lambda_i\in \mathbb R_{\geq 0}$, then strict complementary slackness requires that $\bar\lambda_i^*G(\hat x,\bar y,u_{QP}^*)<0$, 
where $u_{QP}^*$ and $\bar\lambda_i^*$ denote the optimal solution and the corresponding optimal Lagrange multiplier, respectively. We are now ready to state the second main result of the paper. 

\begin{Theorem}\label{TH: 2}
\pno{Given system \eqref{eq: actual system},
suppose that Assumptions \ref{assum A B cont} and \ref{ass:bounderror} hold.}
For the attack model \eqref{eq: attack model}, the observer \eqref{eq: hat x switch}, and a closed set $S\subset \mathbb{R}^n$, %\mathcal{D}$,
let $X_0$, $\hat X_0$, $\tilde X$ and $\bar X$ be given as in \eqref{eq: set X0}-\eqref{eq: bar X}, and assume that the strict complementary slackness holds for the QPs \eqref{QP no attack} and \eqref{QP under attack} for all $\hat x\in {\tilde X}$.
The following holds\pno{:}
\begin{enumerate}
\item If $S \setminus (\partial S + \pn{(1+}\gamma_1(0)\gamma_2(T_a))\bar E\mathbb B)\neq \emptyset$, then, for each $\hat x \in \textnormal{int}( \bar X)$, the QPs \eqref{QP no attack} and \eqref{QP under attack} are feasible and their respective solutions $t \mapsto u_1^*(\hat x(t), \bar y(t)), t \mapsto u_2^*(\hat x(t), \bar y(t))$ are continuous on $\textnormal{int}(\bar X)$. 

%Furthermore, 
\item For each $x_0\in X_0$ and $\hat x_0\in \bar X \cap \hat X_0(x_0)$, %it holds that
each solution pair $t \mapsto (x(t), \hat x(t))$ to the closed-loop system resulting from assigning the input $u$ of \eqref{eq: actual system} and \eqref{eq: hat x switch} to the {observer-based feedback law} $\kappa$ in \eqref{eq: u switch} with $\kappa_1(\hat {x}, \bar y) = u_1^*(\hat x, \bar y)$ and $\kappa_2(\hat {x}, \bar y) = u_2^*(\hat x, \bar y)$,
 satisfies
%
$\hat x(t)\in \bar X$ %for all $t\geq 0$ 
and $x(t) \in S$ %is conditionally invariant with respect to $X_0$ %, \hat X_0$ 
%for the closed-loop system \eqref{eq: cl system}. 
for all $t\geq 0$.
\end{enumerate}
\end{Theorem}
%
%
\NotACC{
\begin{proof}
Feasibility and continuity of the solutions of the QPs \eqref{QP no attack} and \eqref{QP under attack} follow from \cite[Lemma 5]{garg2022fixed} and \cite[Theorem 1]{garg2022fixed}, respectively. 
\pn{From feasibility of the QPs and continuity of its solutions, there exists a continuous control input $u$ such that the CBF condition \eqref{eq: CBFConduA} holds along the closed-loop trajectory $\hat x(t)$.
Thus, it follows that 
 the set $\tilde X$ is forward invariant for the observer \eqref{eq: hat x switch}, and hence}
$\hat x(t)\in \tilde X$ for all $t\geq 0$ and for each $\hat x(0)\in \tilde X_0$. 
Thanks to Theorem \ref{Th:TBoundError}, for each $x(0)\in X_0$, one has $|e(t)|\leq \gamma_1(0)\gamma_2(T_a)\bar E$, which from Lemma \ref{lemma: bounded error} implies $x(t)\in S$ for all $t\geq 0$. 
\end{proof}
}

\NotACC{Thus, the proposed observer-based feedback framework, based on a \pno{switching} observer and \pno{a switching control scheme}, can keep the system safe even under output attacks. Next, we evaluate our proposed scheme via numerical experiments.} 

% Given $e_0$, we have
% \begin{itemize}
%     \item Under no attack, $e(t)\leq \gamma_1 e_0$, with $e_1 = \gamma_1e_0$ at end of no-attack
%     \item Under attack, $e(t)\leq \gamma_2 e_1$ with $e_2 = \gamma_1\gamma_2e_0$ at end of attack 
% \end{itemize}

% % $S_1 = S\ominus \gamma_1\gamma_2e_0$: where we want to remain before the attack starts

% Controller objective is to satisfy $\hat x(t)\in S_1 = S\ominus \gamma_1\gamma_2e_0$ for all $t$

% we get $x(t) \in \hat x(t) \oplus \gamma_1\gamma_2e0 \in S\ominus \gamma_1\gamma_2e_0 \oplus \gamma_1\gamma_2e_0 = S$

% Now that we have bounds on state estimation error, we can define the sets $S_1 = S\ominus \delta$ and $S_2 = S\ominus \epsilon$ and the control objectives for the \pn{feedback law} in \eqref{eq: u switch} are
% \begin{itemize}
%     \item \textbf{No attack}: Design $h_1$ so that $\hat x$ reach the set $S_1$ within a finite time $T_{na}$ for each initial condition in $\hat x(0)\in S_2$;
%     \item \textbf{Under attack}: Design $h_2$ so that the set $S_2$ is forward invariant, i.e., $\hat x(t)\in S_2$ for all $t$. 
% \end{itemize}


% \subsection{(Preliminary) Analysis under attack}
% Let $t_0\in \mathcal T_a$ be the instant the attack happens on the system output and $e(t_0) = e_0$. Without loss of generality, assume that $0\in \textrm{S}$ and that the control input $u = h_2(\hat x)$ is such that the origin is exponentially stable for \eqref{eq: pred model}. Thus, assume that there exists $\lambda, c>0$ such that $|\hat x(t)|\leq \lambda \exp\left({-c(t-t_0)}\right)|\hat x(t_0)|$. Under this setup, it can be shown that the error $e$ can be bounded as
% \begin{align*}
%     |e(t)|\leq \exp\left({c_1(t-t_0)}\right)|e(t_0)|+\frac{\exp\left({c_1(t-t_0)}\right)-1}{c_1}\gamma + \frac{\lambda}{c_1+c}(\exp\left({c_1(t-t_0)}\right)-\exp\left({-c(t-t_0)}\right))
% \end{align*}
% for some $c_1>0$. 

% In particular, the maximum error for an attack of time length $T$ is
%  \begin{align*}
%     |e_M|\leq \exp\left({c_1T}\right)|e(t_0)|+\frac{\exp\left({c_1T}\right)-1}{c_1}\gamma + \frac{\lambda}{c_1+c}(\exp\left({c_1T}\right)-\exp\left({-cT}\right)) 
%  \end{align*}
 
% In brief (let $\{t^i_1\}, \{t^i_2\}$ be the sequences when the attacks start and stop: 
% \begin{itemize}
%     \item Assume $e(0)\in E$ so that $e(t^1_1)\in E$, i.e., $$|e^1_1|\leq \frac{2\gamma|P|}{\theta \lambda_m(Q)}$$ 
%     \item 
%     % $$|e(t^1_2)|\leq \exp\left({c_1T}|e(t^1_1)|+\frac{\exp\left({c_1T}-1}{c_1}\gamma + \frac{1}{c_1+c}(\exp\left({c_1T}-\exp\left({-cT})$$
%     $$|e(t^1_2)|\leq \exp\left({c_1T}\right)\frac{2\gamma|P|}{\theta \lambda_m(Q)}+\frac{\exp\left({c_1T}\right)-1}{c_1}\gamma + \frac{\lambda}{c_1+c}(\exp\left({c_1T}\right)-\exp\left({-cT}\right))$$
%     \item $$|e(t^2_1)|\leq \frac{\lambda_M(P)}{\lambda_m(P)}\exp\left({-\frac{(1-\theta)c_1}{\lambda_M(P)}T}\right)|e(t^1_2)|$$
%     \item If $e(t^2_1)\in E$, then it holds for all $i\geq 1$
%     \begin{align*}
%         |e(t^i_2)|& \leq \underbrace{\exp\left({c_1T}\right)\frac{2\gamma|P|}{\theta \lambda_m(Q)}+\frac{\exp\left({c_1T}\right)-1}{c_1}\gamma + \frac{\lambda}{c_1+c}(\exp\left({c_1T}\right)-\exp\left({-cT}\right))}_{\rho}\\
%         |e^i_1| & \leq \frac{2\gamma|P|}{\theta \lambda_m(Q)}
%     \end{align*}
%     \item Main conclusion: 
%     \begin{align*}
%         \frac{\lambda_M(P)}{\lambda_m(P)}\exp\left({-\frac{(1-\theta)c_1}{\lambda_M(P)}T}\right)\rho \leq \frac{2\gamma|P|}{\theta \lambda_m(Q)}
%         \implies |e(t)| \leq \frac{\lambda_M(P)}{\lambda_m(P)}\rho \quad \forall t \geq 0
%     \end{align*}
% \end{itemize}
 
 
% % \begin{figure}[b]
% % 	\centering
% % 	\includegraphics[width=\columnwidth,clip]{error_image.png}
% % 	\caption{Evolution of error under the conditions discussed above.}
% % 	\label{fig:err e}
% % \end{figure}

% %  \subsection{Reduced-order observer}
% %  Main idea:
 
% %  \begin{align}
% %      \dot x & = Ax + Bu, \\
% %      y & = Cx, 
% %  \end{align}
% %  Let's assume that after an attack, the system output takes the form:
% % \begin{align}
% %     \bar y = \begin{cases}Cx & \textrm{if} \; t\notin \mathcal T_a, \\
% %     \begin{bmatrix}\tilde Cx\\ O(t,x)\end{bmatrix} & \textrm{if} \; t \in \mathcal T_a, \end{cases}
% % \end{align}
% % Assume that the matrix $\tilde C\in \mathbb R^{\tilde p\times n}$ is full-rank. Then, it is possible to find a transformation $Q = \begin{bmatrix}
% % \tilde C\\ R \end{bmatrix}$ such that $Q$ is invertible. Using this, define $z = \begin{bmatrix}z_1\\z_0\end{bmatrix} = \begin{bmatrix}\tilde Cx\\ Rx\end{bmatrix} = Qx$ to obtain
% % \begin{align}
% %     \dot z & = QAQ^{-1}z + QBu, \\
% %     y_1 & = z_1
% % \end{align}
% % when the system output is under attack. Define $\bar A = QAQ^{-1}$ and $\bar B = QB$, and re-write the above system as
% % \begin{align}
% %     \dot z & = \bar Az + \bar Bu, \\
% %     y_1 & = z_1.
% % \end{align}
% % Let $z = \begin{bmatrix}z_1\\ z_2\\ z_3\end{bmatrix}$, where $z_2$ is the subset of the states observable through the output $y = z_1$ and $z_3$ is the subset that is not observable. Thus, it is possible to design an observer such that
% % \begin{align}
% %     |\hat z_1(t)-z_1(t)| & = 0\\
% %     |\hat z_2(t)-z_2(t)| & \leq \lambda_1 \exp\left({-c_1t}|\hat z_2(0)-z_2(0)|,\\
% %     |\hat z_3(t)-z_3(t)| & \leq \lambda_2 \exp\left({c_2t}|\hat z_3(0)-z_3(0)|
% % \end{align}
% % for some $\lambda_1, \lambda_2, c_1, c_2>0$. Now, let the observer for the actual state $x$ be given as $\hat x = Q^{-1}\hat z$ so that the estimation error in terms of the system states $x$ can be computed as
% % \begin{align*}
% %     e & = \hat x-x  = Q^{-1}(\hat z-z)\\
% %     \implies |e| & \leq |Q^{-1}(\hat z_1-z_1)|+|Q^{-1}(\hat z_2-z_2)|+|Q^{-1}(\hat z_3-z_3)|\\
% %     & \leq |Q^{-1}|(\lambda_1 \exp\left({-c_1t}|\hat z_2(0)-z_2(0)|+\lambda_2 \exp\left({c_2t}|\hat z_3(0)-z_3(0)|)
% % \end{align*}

% % In brief: the estimation error under a reduced-order observer is bounded as
% % \begin{align}
% %     |e(t)| \leq a_1\exp\left({-c_1t} + a_2\exp\left({c_2t}
% % \end{align}
% % for some $a_1, a_2>0$. The estimation error without a reduced-order observer is bounded as
% % \begin{align*}
% %      |e(t)|& \leq \exp\left({c_3(t-t_0)}|e(t_0)| + \frac{\lambda}{c_3+c_4}(\exp\left({c_3(t-t_0)}-\exp\left({-c_4(t-t_0)})\\ 
% %      & \leq a_3\exp\left({c_3t} + a_4\exp\left({-c_4t}
% % \end{align*}
% % for some $a_3, a_4, c_3,c_4>0$.
% % % Now, if $(\tilde C, A)$ is observable, then an observer-based controller can be designed for $t\in \mathcal T_a$. On the other hand, if $(C, A)$ is observable but $(\tilde C, A)$ is not observable, then an observer cannot be designed. 

% % % Discussion: 
% % % \begin{itemize}
% % %     \item If we assume that $(\tilde C, A)$ is observable, then the redundant input in $y= Cx$ are not necessary, and so, effectively, the attack is not affecting the system. 
% % %     \item If we assume that $(\tilde C, A)$ is not observable, then a reduced-order observer cannot be designed. 
% % % \end{itemize}

% % % If the matrix $A$ is invertible, then it follows under Assumption \ref{assum: d bound} that
% % % \begin{align}
% % %     |e(t)|\leq \delta |A^{-1}(\exp\left({A T}-I))|, 
% % % \end{align}
% % % for each $T>0$. 

% % % System
% % % \begin{align}
% % % \dot x & = Ax + Bk(\hat x),\\
% % %  \dot{\hat x} & = A\hat x + L(y-C\hat x)+ Bk(\hat x),
% % % \end{align}

% % % Requirement: $x(t)\in S$ for all $t\geq 0$ and all $x(0)\in X_0\subset S$ with $|x(0)-\hat x(0)|\leq \delta$ for some carefully chosen $\delta>0$.

\section{Numerical Example}
%\subsection{Example 1: Double Integrator}
Consider a system $\mathcal S $ as in \eqref{eq: actual system}, with state $x=(x_1,x_2) \in \mathbb{R}^2$, input $u \in \mathbb{R}$, and %double integrator 
dynamics
%\begin{subequations}
%\begin{align}\label{eq: double integrator}
%% \hspace{-40pt}\text{\textit{Actual system}:} &   \hspace{30pt} \dot x = \Phi(t,x,u), \quad x(0) = x_0,
%\dot x_1 &= x_2, \quad  
%\quad \dot x_2 = u ,\\
$\dot x = (x_2, u), 
y = (x_1, x_2)$ %\begin{bmatrix} x_1\\ x_2 \end{bmatrix},
%\end{align}
%\end{subequations}
where %the output that cannot be attacked is $y_s=x_1$, while 
\pno{$y_a=x_1$} is only available when there are no attacks. 
DoS attacks have maximum duration of $T_a=1.6$ seconds and are launched only after at least $T_{na}=0.047$ seconds without an attack. Here, $u$ is designed such that every response $t \mapsto x(t)$ to %\eqref{eq: double integrator} 
$\mathcal{S}$
satisfies $ x(t) \in S:=\{(x_1,x_2) \in \mathbb{R}^2 : x_1^2 + 2 x_2^2+2 x_1 x_2-35\leq 0\}$ for all $t\geq 0$, given that $x(0) \in X_0:= S \setminus (\partial S + \varepsilon \mathbb B)$, with {$\varepsilon=2.01$}.

An observer as in \eqref{eq: hat x switch} is designed. Given that Assumption \ref{assum A B cont}
is satisfied, and by setting $L=\left[\begin{smallmatrix} 
32 && 0.5 
\\ 
0.5 && 32
\end{smallmatrix}\right]$ 
and 
$\tilde L =\left[\begin{smallmatrix} 0.05 \\ 3.2 \end{smallmatrix}\right] $, we have $\lambda(A- L  C)=-31.75\pm\textit{i}0.43$, %$\lambda_M(A- L  C)=-2-\textit{i}\sqrt{2}$, 
and
$\lambda(A-\tilde L \tilde C)=\pno{\{0.5,-3.2\}}$. %, and $\lambda_M(A-\tilde L \tilde C)=0.821$. 
%
Given $x_0=(5.3,-2.4)$, $\hat x_0=(4.9,-2.1)$, and {$\bar E=0.55$}, we have that $|e(0)|=0.5 \leq \bar E$, so Assumption \ref{ass:bounderror} holds.

{Thus, by applying Lemma \ref{Lemma: Bound un attack}, with %$\pn{c_1=\lambda_m (Q), c_2 = \frac{\lambda_m (P)}{\lambda_M (P)}}, 
$P=\left[\begin{smallmatrix} 1 && 0 \\ 0 && 1 \end{smallmatrix}\right], 
Q=\left[\begin{smallmatrix}
63 && 0 
\\ 
0 && 64
\end{smallmatrix}\right] $, %(that satisfy $P=P^\top \succeq 0$ and $Q=Q^\top \succeq 0$), %and $\theta=0.1$, 
and given that every pair of subsequent attacks are separated by at least $T_{na}$ seconds, the estimation error satisfies 
%
$|e(t)|\leq \gamma_1(t-t^{i}_2) e(t^{i}_2)$ for all $t \in [t^{i}_2,t_1^{i+1}]$, $i\in \mathbb{N}$,  $\gamma_1( T_{na}=0.047)=
%c_2^{-1/2} \exp \left({-\frac{(1-\theta)c_1 c_2}{2\lambda_m(P)}T_{na}%(t-t_0)
 %   }\right)=
    0.226,$ and is displayed in green\footnote{Code at 
    https://github.com/HybridSystemsLab/SafeRecovery-DoSAttacks} 
    %https://github.com/sjleudo/SafeRecovery} 
    in Figure \ref{fig:statesex}}.
%    
 Given that the growth rate of the exponential defining the function $\gamma_1$ is negative, % in Figure \ref{fig:statesex},
 {the bound on the error norm} decreases at each interval without attacks.%, denoted with \textit{Attack=}$0$ in the fourth subplot.
%    

%{During attacks, denoted with \textit{Attack=}$1$, the norm of the error grows but satisfies the upper bound.}    

\begin{figure}[t]
\hspace{-0.5cm}
\includegraphics[width=1.05\columnwidth,clip]{AttackPlotsBV.eps}%error_bound_fig.png}
	%\vspace{-0.2cm}
	\caption{Solutions to the 2D system and state estimation error during worst-case attacks of  $T_a=1.6s$, for $x_0=(5.3,-2.4)$, $\hat x_0=(4.9,-2.1)$, and  $\bar E=0.55$. {In the third plot, the bound (purple) is defined as in Theorem 1.}}
	\label{fig:statesex}
\end{figure}

{In addition, by applying Lemma \ref{Lemma: Bound u attack} %at every attack instant $t \in \mathcal{T}_1$, 
with 
$\hat c_1%=|\Phi||\Phi_1|\breve c_1
=1.12,
\hat c_2 %= |\Phi||\Phi_2|
=1.19, 
\hat \lambda_1=3.2,
{\hat \lambda_2%= |\hat{A}_{22}|
=0.5,} \> 
%\breve c_1 = \lambda_m (\hat Q), \breve c_2 = \frac{\lambda_m (\hat P)}{\lambda_M (\hat P)},}
\hat P=\left[\begin{smallmatrix}
1 && 0
\\ 
0 && 1 
\end{smallmatrix}\right], 
\hat Q=\left[\begin{smallmatrix} 6.4 && 0 
\\ 
0 && 6.4 
\end{smallmatrix}\right] $, %(that satisfy $\hat P=\hat P^\top \succeq 0$ and $\hat Q=\hat Q^\top \succeq 0$),
$\Phi=\left[\begin{smallmatrix} 
1 && -0.25 
\\ 0 && 0.97 \end{smallmatrix}\right]$, and
{$\hat A_{22}= 0.5$,}
 given that every attack has a maximum duration of $T_{a}$ seconds, the estimation error satisfies 
%
$|e(t)|\leq \gamma_2(T_a)|e(t_1^i)| $ for all $t \in [t_1^i,t^i_2]$, $i\in \mathbb{N}_{>0}$ 
where  $\gamma_2( T_{a})
%=\max_{t\in [0, T_a]}(\hat c_1 \exp\left({-\lambda_1 t}\right) + \hat c_2 \lambda_2\exp\left({\lambda_2 t}\right))
=2.65,$ and is displayed in light blue in Figure \ref{fig:statesex}.}
 %   
%In Figure \ref{fig:statesex}, the norm of the error $|e(t)|$ shows the satisfaction of this bound over every interval of attack, denoted with \textit{Attack=}$1$ in the fourth subplot.
%    
%Given that the growth rate of the increasing exponential defining $\gamma_2$ is considerably greater than the growth rate of the decreasing exponential, namely $\hat \lambda_1>\hat \lambda_2$, Figure \ref{fig:statesex} \pn{displays the bound on the error} increasing at each interval of attack. 
%
\pn{Thanks to Theorem 1, given that $\gamma_1(T_{na})\gamma_2(T_1) \leq 1$, the error satisfies $|e(t)|\leq c_1 \gamma_2(T_a)\bar E = 1.46 $ for all $t\geq 0$.}

In Figure \ref{fig:ppex}, the set $X_0$ is a deflation of the set $S$ by $\varepsilon$, and the set $\tilde X$ is an inflation of the set $X_0$ by $\bar E$. The set of initial estimations, $\hat X_0(x_0)$, is defined as the ball of radius $\bar E$ centered at $x_0$.
Thus, the estimator $\hat x$ is initialized at $X_0(x_0)\subset \tilde X$.  
The set $\bar X :=\{(x_1,x_2) \in \mathbb{R}^2 : h(x)\leq 0\}\subset  \tilde X$ is defined by the barrier function $h(x)=x_1^2 + 2 x_2^2+2 x_1 x_2-12.5$.
Given that {the set $S\subset \mathbb R^n$ is such that $S \setminus (\partial S + \pn{(1+\gamma_1(0)\gamma_2(T_a))\bar E\mathbb B})\neq \emptyset$}, by assigning $K=[2.3016,\>\>    2.3671]$ and solving the QPs \eqref{QP no attack} and \eqref{QP under attack} at every point of the trajectory $\hat x(t) \in \bar X$ to assign the input action, %for the intervals without and with attacks, 
thanks to Theorem \ref{TH: 2}, we ensure that $\hat x(t) \in \bar X$ for all $t$, and consequently, $x(t) \in S$ for all $t$. 
%Given that $\varepsilon \geq \bar E (\gamma_1(0)\gamma_2(T_a) +1) $, thanks to Theorem \ref{TH: 2}, by ensuring $\hat x(t) \in \bar X$ for all $t$, we guarantee $x(t) \in S$ for all $t$.

\begin{figure}[t]
	\hspace{-0.7cm}
	\includegraphics[width=1.14\columnwidth,clip]{PhasePortraitBV.eps}%error_bound_fig.png}
	%\vspace{-0.4cm}
	\caption{Phase portrait of $\dot x = (x_2, u), 
y = (x_1, x_2)$ with state estimation for safe recovery of DoS attacks in the measurements of $x_1$. 
%In yellow, we plot the safe set $S$, in green the set $\tilde X$ where the initial state estimators are allowed, in purple the set $X_0$ of initial states, in blue the set $\tilde X_0 (x_0)$ of initial estimators with respect to the given initial state $x_0$, and in scarlet, the set $\bar X$ rendered forward invariant for the estimates via the control barrier function.	%Here, $T_a=1.2, T_{na}=1, x_0=(-5,2), \hat x_0=(-5.4,2.3), \varepsilon=1.1$, and  $\bar E=0.55$
By initializing the estimation $\hat x$ in the $\bar E-$ball (green) around $x(0)$, the set $\bar X$ (purple) is rendered forward invariant for $\hat x$ (orange), and the safe set $S$ (yellow) conditionally invariant for $x$ (dark blue) with respect to the set of allowed initial states, namely $X_0$ (light blue), via the control barrier function.
The set $\tilde X$ (scarlet) denotes the allowed initial \pno{observer} states.
	}
	\label{fig:ppex}
\end{figure}

\NotACC{\subsection{Example 2: 10-D System}
To show the potential of scalability of our design, consider a system with state $x \in \mathbb R^n$, input $u \in \mathbb{R}^m$, and dynamics as in \eqref{eq: actual system}, with randomly generated parameters $A \in \mathbb R^ {n \times n}, B \in \mathbb R^ {n \times m},$ and $C \in \mathbb R^ {p \times n}$, where $n=10, m=5, p=5$, satisfying
$\textup{rank} (\tilde{\mathcal O} (C,A))=n, \textup{rank} (\tilde{\mathcal C} (A,B))=n, \textup{rank} (\tilde{\mathcal O} (\tilde C,A))=n-\tilde p$, namely, $\tilde p$ randomly chosen outputs are attacked.

DoS attacks have maximum duration of $T_a=\pn{0.2}$ seconds and are launched only after at least $T_{na}=\pn{18.9}$ seconds without an attack.

Here, $u$ is designed such that every response $t \mapsto x(t)$ to %\eqref{eq: double integrator}
$\mathcal{S}$ 
satisfies $x(t) \in S:=\{x \in \mathbb{R}^{10} : x_1^2 + 2 x_2^2 + x_3^2 + x_4^2 + x_5^2 + x_6^2 + x_7^2 + x_8^2 + x_9^2 + x_{10}^2 + 2 x_1 x_2-25\leq 0\}$ for all $t\geq 0$, given that $x(0) \in X_0:= S \setminus (\partial S + \varepsilon \mathbb B)$, with $\varepsilon=\pn{7}$.

An observer as in \eqref{eq: hat x switch} is designed. Given that Assumption \ref{assum A B cont}
is satisfied, and by optimally calculating $L$ as the gain of the action $u=-L^\top x$ that minimizes the quadratic cost function $J(u)=\int_0^\infty (x^\top 100I_{n \times n}x+u^\top 0.01 I_{p\times p} u) dt$ subject to the dynamics $\dot x = A^\top x+ C^\top u$, 
and 
$\tilde L $ as the corresponding values to $L$ after deleting the attacked signals, we have $\lambda_m(A- L  C)=-371.1$, 
$\lambda_M(A- L  C)=-1.61 \pm 1.36\textit{i} $, 
$\lambda_m(A-\tilde L \tilde C)=-332.2$, and $\lambda_M(A-\tilde L \tilde C)=1.91$. 

%For $x_0=(2, -1, 1, 1, 1, 1, 1, 1, 1, 1)$, $\hat x_0=x_0+\textup{rand}(n,1)$, and $\bar E=\pn{3.5}$, we have that $|e_0|\leq \pn{3.16} \leq \bar E$.
%In Figure \ref{fig:statesex2}, the norm of the error $|e(t)|$ shows the satisfaction of the upper bound over both intervals with and without attack, denoted with \textit{Attack=}$1$ and \textit{Attack=}$0$, respectively, in the fourth subplot.
%    
%\pn{Given that the rate of growth of the exponential defining $\gamma_1$ is negative, the norm of the error decreases at each interval without attacks. }   


\begin{figure}[t]
	\centering
	\includegraphics[width=0.8\columnwidth,clip]{AttackPlots2.png}%error_bound_fig.png}
	\vspace{-1.3cm}
	\caption{Solutions to a larger scale system and error of the state estimator with respect to the actual state during \pn{longest possible}  attacks and no attacks intervals for  $T_a=0.2, T_{na}=18.9, x_0=(2, -1, 1, 1, 1, 1, 1, 1, 1, 1)$, $\hat x_0=x_0+\textup{rand}(n,1)$, and $\bar E=3.5$. In the third plot, the bound in yellow is defined as in Theorem 1. 
	}
	\label{fig:statesex2}
\end{figure}

%In Figure \ref{fig:statesex2}, the norm of the error $|e(t)|$ shows the satisfaction of this bound over every interval of attack, denoted with \textit{Attack=}$1$ in the fourth subplot.
%    
%\pn{Given that the rate of growth of the second exponential defining $\gamma_2$ is positive, the norm of the error increases at each interval of attack.}  

%The set $X_0$ is a deflation of the set $S$ by $\varepsilon$, and the set $\tilde X$ is an inflation of the set $X_0$ by $\bar E$. The set of initial estimations, $\hat X_0(x_0)$, is defined as the ball of radius $\bar E$ centered at $x_0$.
The estimator $\hat x$ is initialized at $X_0(x_0)\subset \tilde X$.  
The set $\bar X :=\{(x \in \mathbb{R}^10 : h(x)\leq 0\}\subset  \tilde X$ is defined by the barrier function $h(x)=x_1^2 + 2 x_2^2 + x_3^2 + x_4^2 + x_5^2 + x_6^2 + x_7^2 + x_8^2 + x_9^2 + x_{10}^2 + 2 x_1 x_2-17$.
\pn{Given that $\varepsilon \geq \bar E (\gamma_1(0) \gamma_2(T_a) +1) $, by solving the QPs \eqref{QP no attack} and \eqref{QP under attack} at every point of the trajectory $\hat x(t) \in \bar X$, %to assign the input action for the intervals without (and with) attacks, 
we show that thanks to Theorem \ref{TH: 2}, by rendering $\hat x(t) \in \bar X$ for all $t$, we guarantee $x(t) \in S$ for all $t$.}
}{}
\section{Conclusion and Future Work}
% The first paragraph of the conclusion section should summarize the main results. To some extent it will be similar to the abstract, except that at this point you have already introduced additional terminology and therefore you can be more precise (and still brief) in summarizing your work. Your goal is to reinforce in the reader’s mind what the main points are.
In this paper, we present a switched controller design that, together with a switched observer, ensures a linear time-invariant system to recover safely from finite-time DoS attacks in some of the system outputs. Conditional invariance of a set is guaranteed with respect to a subset of initial conditions by employing a barrier function approach and bounding the estimation error at all times. 
%The second paragraph should discuss future work related to the paper. The goal of this paragraph is to tell others about interesting problems that remain open in this area. If you want to be egotistical about it, you can keep the following in mind: “I’m giving the reader a few nice problems to work on. If they do work on these problems, my work will be cited.”
Future works include studying safe-recovery controllers under uncertainty in the model parameters, noise in the unattacked sensors, nonlinearities in the system dynamics, and only approximate information on the attack times.
In addition, an implementation of a finite-time observer and a tighter bound to relax the conservatism of the conditions are to be considered.
\bibliographystyle{IEEEtran}
\bibliography{myreferences}

% % % Find $\hat x(0), x(0), X_0, \hat X_0$..



\end{document}