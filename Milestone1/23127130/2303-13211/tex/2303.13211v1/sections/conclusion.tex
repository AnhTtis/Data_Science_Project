\section{Conclusion}\label{sec:conclusion}

\textbf{Limitations.} While our analysis sheds light on the behavior of neural networks when presented with clean and poisoned samples from a frequency-domain standpoint, we acknowledge that the proposed FREAK method is only one possible approach for leveraging these insights, and it may not necessarily be the most optimal. Furthermore, although we opted to use Wasserstein distance, other more suitable distances may be available for this particular problem. Lastly, although our findings encompass all frequency-based backdoor attacks, we acknowledge that we only evaluated a fraction of spatial attacks, and thus further research is required to obtain a more comprehensive assessment.


\textbf{Conclusion.} In conclusion, this paper presents a comprehensive investigation into the frequency sensitivity of Deep Neural Networks when exposed to clean versus poisoned samples. Our results reveal significant differences in frequency sensitivity between these two types of samples. Based on these findings, we propose FREAK, a novel frequency-based poisoned sample detection algorithm that is both simple and effective. Our experimental results demonstrate that FREAK is not only successful against frequency-based backdoor attacks but also some spatial attacks. While our work represents a critical first step towards leveraging these insights, we anticipate that our analysis and proposed defense mechanism will establish a basis for future research and development of backdoor defenses. One possible future direction is sample purification which is presented in the supplementary material.



\section{Acknowledgements} 

This work was supported by the King Abdullah University of Science and Technology (KAUST) Office of Sponsored Research through the Visual Computing Center (VCC) funding, the SDAIA-KAUST Center of Excellence in Data Science and Artificial Intelligence (SDAIA-KAUST AI), and UKRI grant: Turing AI Fellowship EP/W002981/1. We also thank the Royal Academy of Engineering and FiveAI for their support. Adel Bibi has received funding from the Amazon Research Awards.