\section{Experiments}\label{sec:experiments}

\begin{table*}[th!]
\centering
\renewcommand{\arraystretch}{1.1}
\scalebox{0.85}{
\begin{tabular}{@{}cccccccc@{}}
\hhline{--------} 
         \multicolumn{3}{c}{\textbf{FIBA}~\cite{Feng2021FIBAFB}}  & \multicolumn{3}{a}{\textbf{FTrojan}~\cite{Wang2021BackdoorAT} }      & \multicolumn{2}{c}{\textbf{CYO}~\cite{Hammoud2021CheckYO}}           \\ \hhline{--------} 
         $h$  & $\alpha$  & PSNR$\uparrow$/SSIM$\uparrow$  & {Locations} & {Magnitude} & PSNR$\uparrow$/SSIM$\uparrow$ & $k$ & \multicolumn{1}{l}{PSNR$\uparrow$/SSIM$\uparrow$} \\ \hhline{--------} 

  50 &    0.2    &    23.98/0.9010       &     (223,224), (111,111)      &     30.0      &     44.89/0.9943      & 1000 &   49.51/0.9981                            \\ \bottomrule
\end{tabular}}
\vspace{5pt}
\caption{\textbf{Parameters of Frequency Backdoor Attacks.} The parameters of each frequency backdoor attack are chosen such that an ASR $>95\%$ is achieved. These parameters, along with the invisibility metrics for each attack, are summarized here.}
\label{hyperparameters}
\end{table*}

\subsection{Experimental Setup \& Metrics} 
\noindent \textbf{Setup.} Similar to \cite{Hammoud2021CheckYO, Doan2021LIRALI}, we conduct our experiments on ImageNet dataset \cite{Russakovsky2015ImageNetLS}. All models are trained using a ResNet18 trained from scratch using an SGD optimizer with initial learning rate of 0.1 that decays by a factor of 0.25 every 15 epochs. The poisoning rate is fixed to 5.0\%.

\noindent \textbf{Backdoor Attack Metrics.}  To evaluate the performance of the trained backdoor attacked models, we use two commonly used metrics: clean data accuracy (CDA), which measures the DNN’s performance on clean samples, and attack success rate (ASR), which measures the effectiveness of the backdoor attack in instigating the target label. A good backdoored model should have a high ASR and a high CDA.

\noindent \textbf{Detector Metrics.} To evalute the performance of the proposed FREAK detector, we use True Positive Rate (TPR) and False Positive Rate (FPR) as metrics. TPR is a measure of how often a detector correctly identifies a poisoned sample. It is calculated as the number of true positive instances divided by the total number of positive instances. FPR is a measure of how often a detector incorrectly identifies a clean samples as poisoned. It is calculated as the number of false positive results divided by the total number of negative instances. Both TPR and FPR are important metrics in evaluating the performance of a detector. TPR helps us to assess how effective the detector is at identifying poisoned samples, while FPR helps us to identify cases where the detector is misclassifying clean samples as poisoned. A good detector should have a high TPR and a low FPR.

\noindent \textbf{Invisibility Metrics.} Following \cite{Hammoud2021CheckYO}, we measure the imperceptibility of an attack using peak signal-to-noise ratio (PSNR) and structural similarity (SSIM). SSIM is a perceptual metric that compares the structural similarity between two images, while PSNR measures the peak signal-to-noise ratio between the two images. A higher SSIM or PSNR value indicates a higher quality image, \textit{i.e poisoned image looks close to clean one}, while a lower value indicates a more noticeable attack.

\subsection{FREAK against Frequency Backdoor Attacks}

We evaluate our backdoor defense against three frequency backdoor attacks, namely, CYO \cite{Hammoud2021CheckYO}, FTrojan \cite{Wang2021BackdoorAT}, and FIBA \cite{Feng2021FIBAFB}. As mentioned earlier, the models are trained from scratch using a poisoning rate of 10.0\%. Table \ref{hyperparameters} shows the invisibility metrics (PSNR and SSIM) and hyperparameters selected for each attack. The hyperparameters were chosen to achieve an ASR $>95.0\%$. 

To test our proposed defense, we fix $\alpha=1$, $|\mathcal{D}_h|= 32$, $|\mathcal{D}_c| = 128$, $\beta = 12$ (pooling filter size), and $k=5000$. We randomly select 5000 samples from $\mathcal{D}_{\text{test}}$ to be poisoned and another 5000 samples to compute the false positive rate. 

% FREAK is capable to achieving a very high TPR while maintaining a low FPR rate on all frequency backdoor attacks. Particularly, against CYO \cite{Hammoud2021CheckYO} and FTrojan \cite{Wang2021BackdoorAT} the TPR is close to 100\% with a FPR close to 0\%. Since FIBA poisons low frequency content, which generally overlaps with the frequencies a clean network attends to, the TPR drops to 90\% with a FPR of 5\%. Results for ResNet34 and for different hyperparameters are presented in the supplementary material.
FREAK has demonstrated remarkable capabilities in achieving a remarkably high true positive rate (TPR) while simultaneously maintaining an impressively low false positive rate (FPR) in response to all frequency backdoor attacks. Specifically, against CYO \cite{Hammoud2021CheckYO} and FTrojan \cite{Wang2021BackdoorAT}, FREAK attains a TPR that is close to perfect, i.e., 100\%, with an accompanying FPR that is insignificantly close to zero. However, because FIBA \cite{Feng2021FIBAFB} corrupts low-frequency data, which typically coincides with the frequencies that a clean network processes, the TPR decreases to 90\% with an FPR of 5\%. Further details about the results obtained for ResNet34 and ablations of different hyperparameters can be found in the supplementary material.







\begin{table}[h!]
\centering
\scalebox{0.9}{
\begin{tabular}{@{}ccc@{}}
\cmidrule(l){2-3}
\textbf{}        & \textbf{TPR (\%)} & \textbf{FPR (\%)} \\ \midrule
\textbf{CYO}     & 99.25             & 1.56              \\
\textbf{FTrojan} & 100.00            & 1.39              \\
\textbf{FIBA}    & 90.15             & 5.31              \\ \bottomrule
\end{tabular}}
\caption{\textbf{FREAK Against Frequency Backdoor Attacks.} FREAK proves to be capable of achieving a high TPR while maintaining a low FPR against frequency backdoor-attacks.}
\end{table}

\subsection{FREAK against Spatial Backdoor Attacks}

We subjected FREAK to a series of spatial backdoor attacks, including BadNet \cite{Gu2019BadNetsEB}, Blend \cite{Chen2017TargetedBA}, SIG \cite{Barni2019ANB}, and WaNet \cite{Nguyen2021WaNetI}, and summarized the outcomes in Table \ref{spatial}. Interestingly, FREAK was able to achieve a high true positive rate against BadNet and SIG while maintaining a low false positive rate; however, this was not the case for Blend and WaNet, where a significant decline in performance was observed. We discuss this further in the limitations section. The robust performance of FREAK against SIG attacks may be due to the fact that the sinusoidal signal utilized for poisoning the model is of high frequency, creating distinct artifacts in the frequency domain compared to a clean sample.



% Please add the following required packages to your document preamble:
% \usepackage{booktabs}
\begin{table}[h!]
\centering
\scalebox{0.9}{
\begin{tabular}{@{}ccc@{}}
\cmidrule(l){2-3}
                & \textbf{TPR (\%)} & \textbf{FPR (\%)} \\ \midrule
\textbf{BadNet} & 96.60             & 2.73              \\
\textbf{SIG}    & 84.51             & 4.74              \\
\textbf{WaNet}  & 2.34              & 1.95              \\
\textbf{Blend}  & 9.11              & 3.91              \\ \bottomrule
\end{tabular}}
\caption{\textbf{FREAK Against Spatial Backdoor Attacks.} FREAK proves to be capable of achieving a high TPR while maintaining a low FPR on BadNet and SIG, however, this is not the case for WaNet and Blend.}
\label{spatial}
\end{table}

