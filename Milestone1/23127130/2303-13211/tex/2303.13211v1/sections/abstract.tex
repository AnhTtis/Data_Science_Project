\begin{abstract}
%Recently, a growing body of research has demonstrated the effectiveness of creating backdoor attacks in the frequency domain. These attacks have been shown to be both highly stealthy and capable of evading existing spatial backdoor defenses while achieving a high attack success rate. 
%In this work, we analyze the frequency sensitivity of deep neural networks (DNNs) when presented with clean samples versuses poisoned samples. Our analysis reveals that a DNN's frequency sensitivity to poisoned samples is considerably different than that for clean samples. Those findings form the basis of FREAK, a simple yet extremely powerful frequency-based poisoned sample detection algorithm. Through extensive experiments, we show that FREAK is capable of detecting not only frequency backdoor-attacked samples but also those attacked with spatial backdoor attacks. 

% In this work, we conduct an in-depth analysis of the frequency sensitivity of deep neural networks (DNNs) when presented with clean samples versus poisoned samples. Our analysis uncovers significant disparities in the frequency sensitivity of DNNs between the two types of samples. Building upon these findings, we introduce FREAK, a simple and highly effective frequency-based poisoned sample detection algorithm. Our extensive experimental results demonstrate the exceptional performance of FREAK, not only in detecting frequency backdoor-attacked samples but also in identifying those that are subject to spatial backdoor attacks. 


% 1. We analyze the frequency sensitivty of deep neural networks when presented with clean samples versus poisoned samples.

% 2. Our analysis uncovers significant disparities in the frequency sensitivity of DNNs between the two types of samples. 

% 3. Building upon these findings, we introduce FREAK, a simple and highly effective frequency-based poisoned sample detection algorithm.

% 4.  Our extensive experimental results demonstrate the effectiveness of FREAK, but also in identifying some spatial backdoor attacks. 

% 5. We hope our analysis serves as basis for future backdoor defenses.


% In this paper, we investigate the frequency sensitivity of DNNs when presented with clean samples versus poisoned samples. Our analysis reveals significant disparities in the frequency sensitivity of DNNs between the two types of samples. Building upon these findings, we propose FREAK, a simple and highly effective frequency-based poisoned sample detection algorithm. Our extensive experimental results demonstrate the effectiveness of FREAK in detecting poisoned samples and even identifying some spatial backdoor attacks. We hope our analysis and proposed defense mechanism will serve as a basis for future research and development of backdoor defenses in DNNs.

In this paper we investigate the frequency sensitivity of Deep Neural Networks (DNNs) when presented with clean samples versus poisoned samples. Our analysis shows significant disparities in frequency sensitivity between these two types of samples. Building on these findings, we propose FREAK, a frequency-based poisoned sample detection algorithm that is simple yet effective. Our experimental results demonstrate the efficacy of FREAK not only against frequency backdoor attacks but also against some spatial attacks. Our work is just the first step in leveraging these insights. We believe that our analysis and proposed defense mechanism will provide a foundation for future research and development of backdoor defenses.







\end{abstract}