\section{Experiments}

\subsection{Dataset}
We obtain a dataset of 80 CBCT dental scans for our study to simulate PX imaging and obtain the ground truth for oral structure. We allocate 60 cases to train end-to-end models, while the remaining 20 cases are reserved for evaluation and optimization for NeRF-based models. We resize all CBCT scans into $288\times256\times160$ using trilinear interpolation to facilitate easy comparison.


\subsection{Panoramic Imaging Simulation from CBCT}
To generate projection images, we fit the focal curve using a beta function and adopt the same approach as Oral-3D. We simulate projection on 576 small curves with equal distances along the focal curve. The projection rays are simulated in a range of $\frac{\pi}{4}$ and $\frac{3\pi}{4}$ relative to each segment.

\subsection{Network Architecture and Hyperparameters}
We select $C=1000$ and $S=1200$ in Equation (\ref{eq:render_discrete}) according to the HU value of air and soft tissue. The sampling rate $N_s$ for each radiation ray during training follows a uniform distribution in $[0.25, 1,25]$. We use frequency embedding for the positional encoder with digital frequency at 32, and normalize the coordinates into $[-1, 1]$. For NeXF model, we use a 12-layer MLP with residual connections and set the number of heads as 160 to be consistent with CBCT data.

\subsection{Training and evaluation}
 The model has been trained for 100k iterations with a batch size of 64 rays. The model is optimized by Adam optimizer with a learning rate starting at 0.001 decrease to 0.0001 after 20k iterations. We use structural similarity index measure (SSIM\cite{ssim}), dice coefficient, and peak signal-to-noise ratio (PSNR) to evaluate the reconstruction results. We also use the averaged score proposed in \cite{oral_3d} as the overall metric.