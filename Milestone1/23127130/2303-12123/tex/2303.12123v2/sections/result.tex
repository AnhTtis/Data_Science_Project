\section{Results}
\label{sec:results}

\begin{figure*}[t]
    \centering
    \includegraphics[width=\textwidth]{figs/Nexf-3D-compare.png}
    \caption{Comparison of 3D oral reconstruction by different methods from PX imaging. The reconstruction results are shown by maximum projection to compare density details. We could easily find that our method show the best performance with clear density density distributions and teeth boundaries.
    }
    \label{fig:3d_compare}
\end{figure*}

\begin{table*}[tp]
    \centering
    \caption{Ablation study by removing each component in our proposed method. M: Multi-head Prediction, D: Dynamic Sampling, P: Change $\hat{f}(\cdot)$ to $f(\cdot)$ in training}
    \label{tab:ablation}
    \begin{tabular}
     {p{0.8cm}<{\centering}p{0.8cm}<{\centering}p{0.8cm}<{\centering}p{2.2cm}<{\centering}p{2.2cm}<{\centering}p{2.2cm}<{\centering}p{1.5cm}<{\centering}p{1.5cm}<{\centering}}
    \hline
    M&D&P
    &PSNR&SSIM(\%)&Dice(\%)&Overall&Drop\cr
    \hline
    \xmark&\cmark&\cmark&16.68$\pm$0.74&73.62$\pm$5.49&61.25$\pm$4.57&72.76&-13.28\cr
    \cmark&\xmark&\cmark&16.80$\pm$0.71&61.44$\pm$5.87&73.29$\pm$3.24&72.91&-13.13\cr
    \cmark&\cmark&\xmark&16.57$\pm$1.08&63.63$\pm$3.07&70.28$\pm$3.28&72.25&-13.79\cr
    \hline
    \end{tabular}
\end{table*}

\subsection{Qualitative Comparison}
We first show qualitative comparison in Fig.~\ref{fig:3d_compare} to compare the reconstruction results of baseline models. We can see that although ResEncoder and GAN could restore the curved shape of mandible without any prior knowledge, these models fail to recover the detail density distribution in the reconstruction results. For NAF, the model could recover the curved shape and density variance. But the results contain too much noise and is hard to identify the teeth shape. For Oral-3D, the model could restore both shape and teeth details with the help of individual dental arch shape. However, its reconstruction quality is obviously lower than our method, especially for the details of density change between teeth root and the mandible.

\subsection{Quantitative Comparison}
We then show the quantitative comparison by the proposed metrics in Table \ref{tab:compare}. The dice score is computed by setting a threshold at 500 HU to extract the bone from soft tissues. We could see our model could significantly outperform other models, with improvement of $+5$ in SSIM and $+7.5$ in the overall score against the state-of-the-art method without training on paired images or deformation by individual prior knowledge. To be noted, Oral-3D has a better Dice score but lower performance in PSNR and SSIM. This is consistent with the visualized results shown in Fig.~\ref{fig:3d_compare}, where Oral-3D restores less density details than our model.


\subsection{Ablation Study}
\label{sec:ablation}
We conduct an ablation study to evaluate the contribution of each component in our model: 1) replace the multi-head field function with a single-head predictor and taking in 3D coordinates as input for the positional encoder; 2) use a fixed sampling rate of $N_s=1$ to generate sample points on projection rays; 3) change the rendering function in Eq. (\ref{eq:soft_render_discrete}) to Eq. (\ref{eq:render_function}) when training the model. We use the letters M, D, and P to represent these changes. Results are shown in Table~\ref{tab:ablation}, where the performance drops significantly (about -13 in Overall) when changing any module. We could see the dynamic sampling strategy can greatly improve the reconstruction quality without introducing additional models. And the multi-head architecture has stronger ability in implicit representation in radiation imaging. 
% These results have demonstrated that our proposed multi-head implicit representation model and dynamic sampling strategy are both efficient and effective in single-image reconstruction from PX.


\subsection{Complexity Analysis}
We also conduct the complexity analysis for NeXF and NeRF with the result in Table \ref{tab:complexity}, where the models run on a Nvidia A100 GPU with the same batch size. Combined with experiments in the ablation study, we see that NeXF is both efficient and effective over NeRF, i.e., $\times3.3$ faster in training and inference. To be noted, the training process only takes 30 minutes, which is quite acceptable in clinical application compared to a CBCT scan that takes about 20 minutes \cite{cbct_time}.
\begin{table}[tp]
    \centering
    \setlength\tabcolsep{2pt}
    \caption{Complexity analysis between NeXF and NeRF.}
    \begin{tabular}{p{1.6cm}<{\centering}p{1.8cm}<{\centering}p{1.8cm}<{\centering}p{1.8cm}<{\centering}}
    \hline
    Method&Train(ms/iter)&Infer(s)&Learn(min)\cr
    \hline
    NeXF &90&20&30\cr
    NeRF &300&60&100\cr
    \hline
    \end{tabular}
    
    \label{tab:complexity}
\end{table}