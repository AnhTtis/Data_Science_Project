%%%%%%%%%%%%%%%%%%%%%%%%%%%%%%%%%%%%%%%%%%%%%%%%%%%%%%%%%%%%%%%%%%%%%%%%%%%%%%%%
%2345678901234567890123456789012345678901234567890123456789012345678901234567890
%        1         2         3         4         5         6         7         8

\documentclass[letterpaper, 10 pt, conference]{ieeeconf}  % Comment this line out if you need a4paper

%\documentclass[a4paper, 10pt, conference]{ieeeconf}      % Use this line for a4 paper

\IEEEoverridecommandlockouts  % This command is only needed if you want to use the \thanks command

\overrideIEEEmargins   % Needed to meet printer requirements.

%In case you encounter the following error:
%Error 1010 The PDF file may be corrupt (unable to open PDF file) OR
%Error 1000 An error occurred while parsing a contents stream. Unable to analyze the PDF file.
%This is a known problem with pdfLaTeX conversion filter. The file cannot be opened with acrobat reader
%Please use one of the alternatives below to circumvent this error by uncommenting one or the other
%\pdfobjcompresslevel=0
%\pdfminorversion=4

% See the \addtolength command later in the file to balance the column lengths
% on the last page of the document

\let\labelindent\relax  % compatibility with enumitem package
\let\endtable\relax     % compatibility with subfig package
%\let\endfigure\relax    % compatibility with subfig package

% The following packages can be found on http:\\www.ctan.org
\usepackage{graphics} % for pdf, bitmapped graphics files
\usepackage{epsfig} % for postscript graphics files
\usepackage{mathptmx} % assumes new font selection scheme installed
\usepackage{times} % assumes new font selection scheme installed
\usepackage{amsmath} % assumes amsmath package installed
\usepackage{amssymb}  % assumes amsmath package installed

% my packages
\usepackage{comment}    % for comment environment
\usepackage{amssymb}    % for \mathbb{}
\usepackage{mathtools}  % more math commands (e.g \coloneqq)
\usepackage[inline]{enumitem}   % for latin numbers enumeration
\usepackage[svgnames]{xcolor}
\usepackage{bm}
%\usepackage[normalem]{ulem}       % to cancel text with a line (\sout{})
%\usepackage{setspace}   % for \doublespacing
%\usepackage{graphicx}
\usepackage[caption=false]{subfig}
\usepackage{cite}
%\usepackage{lipsum}

\let\proof\relax \let\endproof\relax
\usepackage{amsthm}     % for definition and theorem environment

\makeatletter
\let\NAT@parse\undefined
\makeatother
\usepackage[hidelinks]{hyperref}   % for clickable references

\DeclareMathAlphabet{\mathcal}{OMS}{cmsy}{m}{n} % normal mathematical symbols

% my commands
\theoremstyle{definition}

\makeatletter
\def\thm@space@setup{%
  \thm@preskip=\parskip \thm@postskip=\parskip
}
\makeatother

\newtheoremstyle{mythmstyle}% <name>
{}% <Space above>
{}% <Space below>
{\itshape}% <Body font>
{}% <Indent amount>
{\bfseries}% <Theorem head font>
{:}% <Punctuation after theorem head>
{.5em}% <Space after theorem headi>
{{\bfseries\thmname{#1}\thmnumber{ #2}}\thmnote{ (#3)}}% <Theorem head spec (can be left empty, meaning `normal')>

\theoremstyle{mythmstyle}

\newtheorem{theorem}{Theorem}
\newtheorem{definition}{Definition}
\newtheorem{corollary}{Corollary}
\newtheorem{lemma}{Lemma}
\newtheorem{assumption}{Assumption}
\newtheorem{observation}{Observation}
\newtheorem{rem}{Remark}

\renewcommand{\vec}[1]{\mathbf{#1}}         % redefine vectors notation
\newcommand{\T}{^{\mathsf{T}}}
\newcommand{\euler}{\mathsf{e}}
\newcommand{\imaginary}{\mathsf{i}}
\newcommand{\B}[1]{\if#1\relax\bm{#1}\else\mathbf{#1}\fi} % bold text
\newcommand{\R}[1]{\mathrm{#1}}						      % regul. text
\newcommand{\C}[1]{\mathcal{#1}}
\newcommand{\BB}[1]{\mathbb{#1}}
\newcommand{\unitvec}[1]{ \vec{\hat{#1}} }
\newcommand{\norm}[1]{\left\lVert #1 \right\rVert}
\newcommand{\abs}[1]{\left\lvert #1 \right\rvert}
\newcommand{\tr}[0]{\mathrm{trace}}
\newcommand{\remark}[1]{{\color{red} [#1]}}

\allowdisplaybreaks
\urlstyle{same}

% correct bad hyphenation here
\hyphenation{op-tical net-works semi-conduc-tor}

%% Formatting for manual annotation
%\doublespacing  % increase interline spacing 
%\onecolumn      % single column

%%%%%%%%%%%%%%%%%%%%%%%%%%%%%%%%%%%%%%%%%%%%%%%%%%%%%%%%%%%%%%%%%%%%%%%%%%%%%%%%
\title{\LARGE \bf
%Emergence and stability of triangular pattern formations 
Local convergence of %large-scale 
multi-agent systems towards triangular patterns}%


\author{{Andrea Giusti$^{1}$, Marco Coraggio$^{2}$, and Mario di Bernardo$^{1, 2}$}% <-this % stops a space
\thanks{This work was in part supported by the Research Project ``SHARESPACE'' funded by the European Union (EU HORIZON-CL4-2022-HUMAN-01-14. SHARESPACE. GA 101092889 - http://sharespace.eu),
%Views and opinions expressed are those of the author(s) only and do not necessarily reflect those of the European Union, which cannot be held responsible for them.
and by the Research Project ``Centro Nazionale HPC, Big Data e Quantum Computing Italian Center for Super Computing (ICSC)'', funded by European Union (PNRR CN00000013).
}% <-this % stops a space
\thanks{$^{1}$Department of Electrical Engineering and Information Technology, University of Naples Federico II, Via Claudio 21, Naples, 80125, Italy.}
\thanks{$^{2}$Scuola Superiore Meridionale, School for Advanced Studies, Largo S. Marcellino 10, Naples, 80138, Italy.}
\thanks{Contacts: 
        {\{andrea.giusti, marco.coraggio, mario.dibernardo\}@unina.it.}
        % {\tt\footnotesize mario.dibernardo@unina.it,}
        % {\tt\footnotesize andrea.giusti@unina.it,}
        % {\tt\footnotesize marco.coraggio@unina.it.}
}
}

\begin{document}



\maketitle
\thispagestyle{empty}
\pagestyle{empty}


%%%%%%%%%%%%%%%%%%%%%%%%%%%%%%%%%%%%%%%%%%%%%%%%%%%%%%%%%%%%%%%%%%%%%%%%%%%%%%%%
% {\color{blue}Nota: Il testo in blu è da rifinire.}
% {\color{Green} Il testo in verde è nuovo.}

\begin{abstract}
%purpose, methods, scope, conclusions
Geometric pattern formation is an important emergent behavior in many applications involving large-scale multi-agent systems, such as sensor
networks deployment and collective transportation.
Attraction/repulsion virtual forces are the most common control approach to achieve such behavior in a distributed and scalable manner.
Nevertheless, for most existing solutions only numerical and/or experimental evidence of their convergence is available.
Here, we revisit the problem of achieving pattern formation giving sufficient conditions to prove analytically that under the influence of appropriate virtual forces, a large-scale multi-agent swarming system locally converges towards a stable and robust triangular lattice configuration. 
%
%Specifically, we use
%{\color{Blue}perturbation analysis} 
%{\color{blue}LaSalle's invariance principle} and geometry-based arguments to derive sufficient conditions for a set of local control actions to guarantee the emergence and local asymptotic stability of the desired formation.
%
Specifically, the proof is carried out using LaSalle's invariance principle and geometry-based arguments.
%
Our theoretical results are complemented by exhaustive numerical simulations confirming their effectiveness and estimating the region of asymptotic stability of the triangular configuration.
%Moreover, we numerically validate the theoretical results and estimation the basin of attraction.
% The proof is carried out exploiting graph theory and LaSalle's theorem, while numerical analysis is used to validate the result and estimate the basin of attraction.
\end{abstract}

%%%%%%%%%%%%%%%%%%%%%%%%%%%%%%%%%%%%%%%%%%%%%%%%%%%%%%%%%%%%%%%%%%%%%%%%%%%%%%%%
\section{Introduction}
\label{sec::intro}

%-------------------------------------------
%\subsection{Problem description and motivation}

% Large scale MAS and swarms ------
% a variety of systems
% The behavior of a rich variety of systems depends on the individual dynamics and the reciprocal interactions of their components. 
Many natural and artificial systems consist of multiple interacting agents; their behavior being determined by both the individual agent dynamics and their interaction.
%This is the case for both natural systems, such as gene networks in cells, flocks of birds, or global climate; and artificial ones, among others robotic swarms and internet \remark{REF}.
In some applications the number of \emph{agents} can be extremely large (\emph{large-scale multi-agent systems}) and the role played by their interconnections becomes predominant over their individual dynamics \cite{PShi2021}. Examples include cell populations \cite{Grandel2021}, swarming multi-robot systems \cite{Heinrich2022}, social networks \cite{Jusup2022} among many others.
%
% the growth of the field
%The growing interest around such systems caused an increase in the study of their \emph{emerging behaviors} and the relative features, namely \emph{scalability, robustness}, and \emph{flexibility}.
%These same features fostered the development of new technologies and methodologies with applications spanning from synthetic biology to swarm robotics, from opinion dynamics to infrastructures' resilience.
%
% The tasks and the Applications -------
% tasks
Some of the most relevant emerging behavior exhibited by these systems involve their 
%belong to the categories of 
\emph{spatial organization, coordination}, and \emph{cooperation} \cite{Brambilla2013}.
%
% geometric pattern formation
A notable case is \emph{geometric pattern formation} \cite{HOh2017} where the agents are required to self-organize
%a relevant example of spatial organization.
%It consists in 
%the self-organization of the agents
%to drive their relative positions 
into some desired \emph{pattern}, such as, for example, triangular lattices consisting of repeating adjacent triangles.
%\sout{This extends the more classical problem of formation control allowing for a greater scalability.}%
%\footnote{\color{red}[MC: aren't they just different? are you sure geometric pattern formation is more general? Forse eliminerei proprio la frase]}
%
% applications (copiato da altro articolo)
Applications of pattern formation include sensor networks deployment \cite{Zhao2019}, collective transportation and construction \cite{Rubenstein2013, Gardi2022}, and exploration and mapping \cite{Kegeleirs2021}.

% The study of stability ---------
%Interestingly, although many control algorithms for geometric pattern formation have been proposed in the literature (e.g. \cite{refs}), they are typically validated only through simulations or experiments, while a formal study of their convergence and stability properties is often lacking.
%The study of stability offers unparalleled insights in the dynamics of the systems, useful in both the improvement of existing solutions and the development of new ones.
%Moreover, proving the stability of a behavior opens the door to applications in more critical scenarios. 
%
% Open challenges and motivation --------
%For these reasons the stability analysis has been the pillar of the study on dynamical systems, nevertheless the diffusion of large-scale multi-agent systems, their inherent complexity and the wide spectrum of emerging behaviors they can express left the formal analysis trudging behind.
% Indeed, a variety of control algorithms for geometric pattern formation has been proposed and validated by simulations or experiments, but the analytical study of their stability remains unaddressed.
%The geometric pattern formation is no exception.

%-------------------------------------------
%\subsection{State of the art}

%-------------------------------------------
%\subsubsection*{Geometric organization in multi-agent systems}
%The geometric organization of multiple agents is studied in different fields, including \emph{geometric pattern formation, formation control} and \emph{flocking}.

Most of the existing distributed control algorithms for geometric pattern formation rely on the use of \emph{virtual forces} (or \emph{virtual potentials}), \cite{Giusti2022, Spears2004, Casteigts2012, Zhao2019, Torquato2009, Olfati-Saber2002IFAC, Mesbahi2010, Sakurama2021, Olfati-Saber2006}.
Within this framework, agents move under the effect of forces generated by the presence of their neighboring agents and the environment, causing attraction, repulsion, alignment, etc. 
%A sub-class in this category are the approaches based on \emph{virtual potentials}, where the virtual force applied on an agent is the negative gradient of a virtual potential \cite{Olfati-Saber2002IFAC, Olfati-Saber2006, Mesbahi2010, Sakurama2021}; thus, only conservative forces can appear.

Interestingly, most strategies are validated only numerically or experimentally
\cite{Giusti2022, Spears2004, Casteigts2012, Zhao2019}.
%\cite{Giusti2022, Spears1999, Spears2004, Balch2000, Balch2000a, Fujibayashi2002, Li2009, Casteigts2012, Song2014, Zhao2019}
Among the exceptions, in \cite{Lee2008}, a geometric control approach based on trigonometric functions is proposed to build triangular lattices, and its  global convergence is proved. % using the direct Lyapunov method.
The extension to 3D spaces is validated analytically in \cite{Lee2010}.
%
Moreover, \emph{harmonic approximation} \cite{Hinsen2005} provides necessary conditions for the local stability of a lattice. 
These conditions are used in \cite{Torquato2009} to numerically design a virtual force that locally stabilizes an hexagonal lattice.
%In \cite{Torquato2009}, the authors solve a numerical optimization to obtain a Lennard-Jones-like function that locally stabilize hexagonal patterns; in particular, \emph{harmonic approximation} \cite{Hinsen2005} is used to generate necessary conditions for the local stability of the pattern, and these conditions are imposed as constraints for the optimization.
%
%Harmonic approximation is a popular tool for the stability analysis in the Physics community \cite{Hinsen2005}. Given a gradient-flow system, made up of particles (agents) with second order dynamics, the harmonic approximation consists in approximating the potential function, around a stationary point, with a second order function. This approximation is used to obtain necessary conditions for the local stability of the system. Such necessary conditions are used in \cite{Torquato2009} as constraints for a numerical optimization procedure to obtain a Lennard-Jones-like function that locally stabilize hexagonal patterns.
%
A general analysis of the effects of attraction/repulsion virtual forces is carried out in \cite{Gazi2002}, where the authors prove that the agents converge inside a bounded region, even though the specific equilibrium configuration is not characterized.
%whose radius depends on the interaction functions and, eventually, the number of agents. 
%Notably this works do not provide any results on where the agents will stop.
%In \cite{Gazi2002}, an extension to formation control is proposed.
%:
%The formation is described by the desired distance between all the possible couple of agents. 
%the stability of the formation is proved, but unique agents identifiers are needed and a specific interaction function is defined for each pair of agents.
%so to have a zero at the desired length of the link.
%------------------------------------------
%\subsubsection{Formation Control}
%\emph{Formation control} 
We wish to remark here that formation control \cite{Sakurama2021, Olfati-Saber2002IFAC,  Mesbahi2010}
%\cite{Sakurama2021, Olfati-Saber2002IFAC, Cortes2009, Mesbahi2010, Singh2015, Porfiri2007, Wang2012}
differs from geometric pattern formation because of a typically smaller number of agents (order of tens) with, possibly, unique identifiers, numerous roles for the agents
%(often one for each agent).
and often some coordinated motion of the agents.
%Moreover, the control laws used for formation control are independent on the target formation, while most solutions to geometric pattern formation are designed ad-hoc to build one or two specific lattices.
%Hence, normally algorithms designed for formation control cannot be directly applied to pattern formation problems.
%but need to be extended with  formation selection and role assignment strategies. Moreover, to preserve the nature of these control algorithms, both this strategies must be dynamic (i.e. can be executed at run time) and distributed.
%Notably,
%A potential exception is contained 
%in \cite{Sakurama2021} the authors propose a general formalism to describe multi-agent problems and design control laws based on virtual potential minimization, but the complexity of the formalism makes it difficult to extend to pattern formation.
%They apply this approach to solve formation control, formation selection and local position assignment, therefore it seems reasonable that it may be extended to geometric pattern formation.
%
%-------------------------------------------
%\subsubsection{Flocking}
Similarly, when solving \emph{flocking} control problems, the emergence of coordinated motion is the crucial concern \cite{Olfati-Saber2006, FWang2022, GWang2022}.
%For instance, in \cite{Olfati-Saber2006} the emergence of ``$\alpha$-lattices'' is proved as a by-product of a flocking algorithm. 
%Such structures resemble regular geometric lattices, but they allow for disconnected or non-rigid components, and the coexistence of heterogeneous patterns (e.g., triangular and square).


%-------------------------------------------
%\subsubsection*{Control algorithms}

%\remark{questa sottosezione sembra non integrarsi bene con le precedenti. Quelle erano una divisione per are di ricerca, qui si parla di un'altra categorizzazione}

% A similar approach, very popular for formation control \cite{Sakurama2021, Olfati-Saber2002IFAC, Mesbahi2010} 
% and flocking \cite{Olfati-Saber2006}, is based on \emph{virtual potentials}, where the virtual force applied on an agent is the negative gradient of a virtual potential. 
% This is a specific instance of the more general category of approaches based on virtual forces, as it can only produce conservative forces.
%\sout{Therefore, we focus on the virtual forces approach, as the results will have a wide applicability.}

%------------------------------------------
%\subsection{Contribution}
%In this work,
In this paper, we revisit the problem of  geometric pattern formation using \emph{attraction/repulsion} virtual forces with the aim of bridging a gap in the existing literature and deriving a general proof of convergence when considering the formation of triangular lattice  configurations.
%as in this case this approach is equivalent to the virtual potential, and, hence, the results will have a wide applicability.
%Specifically, we derive a set of sufficient conditions guaranteeing that a swarm of planar and anonymous agents with first order dynamics locally asymptotically converge towards  robust \emph{triangular lattice}%
%(Sometimes known as hexagonal lattice \cite{Spears2004}.}
%configurations.
When compared to previous work, our stability results (i) can be applied to most control laws based on virtual forces (or potentials), rather than only holding for specific algorithms, e.g. \cite{Lee2008}, (ii) are sufficient rather than being only necessary \cite{Hinsen2005}, (iii) characterize the asymptotic configuration of the agents, rather than just proving its  boundedness \cite{Gazi2002}, and (iv) guarantee the emergence of triangular lattices rather than less regular ones, e.g. $\alpha$-lattices studied in \cite{Olfati-Saber2006}.

%%%%%%%%%%%%%%%%%%%%%%%%%%%%%%%%%%%%%%%%%%%%%%%%%%%%%%%%%%%%%%
\section{Mathematical preliminaries}
\label{sec:preliminaries}
 
%\paragraph*{Notation} 
%We refer to $\mathbb{R}^2$ as the \emph{plane}.
Given a vector $\vec{v} \in \BB{R}^d$, we denote by $[\vec{v}]_i$ its $i$-th element, by $\norm{\vec{v}}$ its Euclidean norm, and by $\unitvec{v}\coloneqq\frac{\vec{v}}{\norm{\vec{v}}}$ its direction. 
$\vec{0}$ denotes a column vector of appropriate dimension with all elements equal to 0.
Given a matrix $\vec{A}$, $[\vec{A}]_{ij}$ is its $(i, j)$-th element. 


%--------------------------------------------------
%\subsection{Dynamical systems}
%We start by recalling the concepts of equilibrium set and stability.
Given a continuous-time, autonomous dynamical system
\begin{equation}\label{eq:dynamical_system}
    \dot{\vec{x}}(t) = \vec{f}(\vec{x}(t)), \quad \vec{x}(0) = \vec{x}_0,
\end{equation}
 with state vector $\vec{x}(t) \in \BB{R}^d$, and $\vec{x}_0 \in \BB{R}^d$, we term as $\vec{\phi}(t, \vec{x}_0)$ its trajectory starting from $\vec{x}(0) = \vec{x}_0$.

\begin{definition}[Equilibrium set]
\label{def:equilibrium_set}
A set $\Xi  \subset \BB{R}^d$ is an \emph{equilibrium set} for system \eqref{eq:dynamical_system} if $ \vec{f}(\vec{x}) = \vec{0} \ \forall \vec{x} \in \Xi$.
\end{definition}

\begin{definition}[Local asymptotic stability {\cite[Definition~1.8]{Kuznetsov2004}}]
\label{def:LAS}
An equilibrium set $\Xi$ for system \eqref{eq:dynamical_system} is \emph{locally asymptotically stable} if $\forall \epsilon>0, \exists \delta > 0$ such that if $\min_{\vec{y} \in \Xi}\norm{\vec{x}_0 - \vec{y}} < \delta $, then
\begin{enumerate}
    \item 
    $\min_{\vec{y} \in \Xi}\norm{\vec{\phi}(t, \vec{x}_0)-\vec{y}} < \epsilon,  \ \forall t>0$, and
    \item
    $ \lim_{t \rightarrow +\infty} \vec{\phi}(t, \vec{x}_0) \in \Xi$.
\end{enumerate}
\end{definition}

%\begin{theorem}[LaSalle's]
%    \remark{Bisogna dare il Teorema di LaSalle o basta dare una ref? forse solo ref}
%\end{theorem}

%\subsection{Graphs, frameworks and rigidity}

%Now, let us review the concepts of incidence matrix of a graph, frameworks and (infinitesimal) rigidity.


\begin{definition}[Incidence matrix]
\label{def:incidence_mat}
Given a digraph with $n$ vertices and $m$ edges, its \emph{incidence matrix} $\vec{B}\in \mathbb{R}^{n\times m}$ has elements defined as
\begin{equation*}
    [\vec{B}]_{ij} \coloneqq \begin{dcases}
    + 1, &\text{if edge $j$ starts from vertex $i$},\\
    - 1, &\text{if edge $j$ ends in vertex $i$},\\
    0,   &\text{otherwise}.
    \end{dcases}
\end{equation*}
\end{definition}
%
\begin{comment}
\begin{definition}[Edge Laplacian]
\label{def:edge_laplacian}
Given a digraph with $m$ edges, its \emph{edge Laplacian} is $\vec{L}_{\R{e}}\coloneqq\vec{B}\T \vec{B} \in \mathbb{R}^{m\times m}$, and its elements are
\begin{equation}
    [\vec{L}_{\R{e}}]_{ij}\coloneqq\begin{cases}
    2, &\mbox{if } i=j, \\
    + 1, & \parbox[t]{5.5cm}{ if edge $i$ and $j$ share the starting or ending vertex,}\\
    - 1, & \parbox[t]{5.5cm}{ if the starting vertex of edge $i$ is the ending vertex of edge $j$, or vice versa,}\\
    0, &\mbox{otherwise.}
    \end{cases}
    \label{eq:edgeLaplacian}
\end{equation}
\end{definition}
\end{comment}

%--------------------------------------------------
%\subsection{Frameworks and rigidity}

%Next, we introduce frameworks and (infinitesimal) rigidity.

\begin{definition}[Framework {\cite[p.~120]{Mesbahi2010}}]\label{def:framework}
Consider a (di\mbox{-)}graph $\C{G}=(\C{V}, \C{E})$ with $n$ vertices, and a set of positions $\vec{p}_1, \dots, \vec{p}_n \in \mathbb{R}^d$ associated to its vertices, with $\vec{p}_i \neq \vec{p}_j \ \forall i,j \in \{1, \dots, n\}$.
A \emph{$d$-dimensional framework} is the pair $(\C{G},\bar{\vec{p}})$, where $\bar{\vec{p}} \coloneqq [\vec{p}_1\T \ \cdots \ \vec{p}_n\T]\T \in \mathbb{R}^{dn}$.
Moreover, the \emph{length} of an edge, say $(i,j)\in \C{E}$, is $\norm{\vec{p}_i - \vec{p}_j}$.
\end{definition}

\begin{definition}[Congruent frameworks {\cite[p. 3]{Jackson2007}}]\label{def:congurent_framework}
Given a graph $\C{G}=(\C{V},\C{E})$ and two frameworks $(\C{G},\bar{\vec{p}})$ and $(\C{G},\bar{\vec{q}})$, these are \emph{congruent} if $\norm{\vec{p}_i - \vec{p}_j}=\norm{\vec{q}_i - \vec{q}_j}\ \forall i,j \in \C{V}$.
\end{definition}
%

%Given a framework with positions $\bar{\vec{p}}^0$, we call a \emph{continuous motion of its vertices} a continuous transformation $\bar{\vec{q}} : \BB{R}_{\ge 0} \rightarrow \BB{R}^{dn}$, with $\bar{\vec{q}}(0) = \bar{\vec{p}}^0$, that returns new positions after a time $t \in \BB{R}_{\ge 0}$ has passed.
%With slight abuse of notation, we denote these new positions by $\vec{p}_1(t), \dots, \vec{p}_n(t)$.
%We let $\C{Q}$ be the set of all such continuous motions.

\begin{definition}[Rigidity matrix {\cite[p.~5]{Jackson2007}}]
\label{def:rigidity_matrix}
Given a $d$-dimensional framework with $n\geq2$ vertices and $m$ edges, its \emph{rigidity matrix} $\vec{M}\in \mathbb{R}^{m\times dn}$ has elements defined as
\begin{equation}\label{eq:rigidity_matrix}
    [\vec{M}]_{e,(jd-d+k)}\coloneqq\begin{cases}
    [\vec{p}_j-\vec{p}_i]_k, & \parbox[t]{4cm}{ if edge $e$ starts from vertex $i$ and ends in vertex $j$,}\\
    [\vec{p}_i-\vec{p}_j]_k, & \parbox[t]{4cm}{ if edge $e$ starts from vertex $j$ and ends in vertex $i$,}\\
    0, &\mbox{otherwise.}
    \end{cases}
\end{equation}
with $k=1, \dots, d$.
\end{definition}

\begin{definition}[Infinitesimal rigidity {\cite[p.~122]{Mesbahi2010}}]
\label{def:inf_rigidity}
A framework with rigidity matrix $\vec{M}$ is \emph{infinitesimally rigid} if, for any infinitesimal motion, say $\vec{u}$,%
\footnote{$\vec{u}$ can be interpreted as either a velocity or a small displacement.}
of its vertices, such that the length of the edges is preserved, it holds that $\vec{M u} = 0$.
\end{definition}

To give a geometrical intuition of the concept of infinitesimal rigidity, we note that an infinitesimally rigid framework is also rigid \cite[p.~122]{Mesbahi2010}, according to the definition below.%
\footnote{In rare cases, a rigid graph is not infinitesimally rigid; e.g. \cite[p.~7]{Jackson2007}.}

\begin{definition}[Rigidity {\cite[p. 3]{Jackson2007}}]
\label{def:rigidity}
A framework is \emph{rigid} if every continuous motion of the vertices, that preserves the length of the edges, also preserves the distances between all pairs of vertices.
%Formally, $\forall \bar{\vec{q}} \in \C{Q}$, it holds that
%$\frac{\R{d}}{\R{d}t}\norm{\vec{p}_i(t) - \vec{p}_j(t)} = 0 \ \forall (i, j) \in \C{E}$
%implies that
%$\frac{\R{d}}{\R{d}t} \norm{\vec{p}_i(t) - \vec{p}_j(t)} = 0 \ \forall i, j \in \{1, \dots, n\}$.
\end{definition}
%Notice that in a rigid framework all the vertices must have at least 2 edges.
As a consequence, in a rigid framework, any continuous motion that does \emph{not} preserve the distance between any two pairs of vertices also does \emph{not} preserve the length of at least one edge.
%\footnote{This is the case because if a motion altered the distance between a pair of vertices while preserving the length of all edges, then by definition of rigidity, it should preserve all the distances between any two pairs of vertices, which is absurd.}

%To more easily assess the infinitesimal rigidity of a framework, it is possible to use the following result.
\begin{theorem}[{\cite[p. 122]{Mesbahi2010}}]
\label{th:rigidity}
A 2-dimensional framework with $n\geq2$ vertices and rigidity matrix $\vec{M}$ is infinitesimally rigid if and only if $\R{rank}(\vec{M})=2n-3$.
%

\end{theorem}

% \begin{corollary}
% \label{corollary:rigidity}
% Given a rigid framework, applying small movements to the vertices, the framework remains rigid.
% \remark{Formulazione informale provvisoria, vedere lemma 4.1 di \cite{Jackson2007}}
% \end{corollary}
% {\color{blue}[MC: serve, ma va dimostrato]}

%--------------------------------------------------
%\subsection{Swarms and lattices}

\begin{definition}[Swarm]
\label{def:swarm}
A \emph{(planar) swarm} $\C{S} \coloneqq \{1,2,\dots,n\}$ is a set of $n \in \mathbb{N}_{>0}$ identical agents that can move on the plane.
For each agent $i \in \C{S}$, $\vec{x}_i(t)\in \mathbb{R}^2$ denotes its position in the plane at time $t \in \mathbb{R}_{\geq0}$.
\end{definition}

Moreover, we call $\bar{\vec{x}}(t) \coloneqq [\vec{x}_1\T(t) \ \cdots \ \vec{x}_n\T(t)]\T \in \mathbb{R}^{2n}$ the \emph{configuration} of the swarm, define $\vec{x}_{\R{c}}(t) \coloneqq \frac{1}{n} \sum_{i = 1}^n \vec{x}_i(t) \, \in \mathbb{R}^2$ as its \emph{center}, and
denote by $\vec{r}_{ij}(t) \coloneqq \vec{x}_{i}(t)-\vec{x}_{j}(t) \in \mathbb{R}^2$ the relative position of agent $i$ with respect to agent $j$.

%{\color{blue}
%Moreover, let us introduce a different representation of the configuration $\bar{\vec{X}}= [\vec{x}_1 \ \cdots \ \vec{x}_n]\T \in \BB{R}^{n\times 2}$.
%}

%Let us introduce the \emph{maximum link length} $R_\R{a} \in \BB{R}_{>0}$, which is used to build a lattice of connections, according to the next definition.

\begin{definition}[Adjacency set]\label{def:adjacency_set}
Given a swarm $\C{S}$, the \emph{adjacency set} of agent $i$ at time $t$ is 
$%\begin{equation}
    \C{A}_i(t) \coloneqq \{ j \in \C{S} \setminus \{i\} : \Vert \vec{r}_{ij}(t)\Vert \leq R_\mathrm{a} \}
$,%\end{equation}
where $R_\R{a} \in \BB{R}_{>0}$ is the \emph{maximum link length}.
\end{definition}

\begin{definition}[Links]
\label{def:links}
A \emph{link} is a pair $(i,j) \in \C{S} \times \C{S}$ such that $j \in \C{A}_i(t)$; $\norm{\vec{r}_{ij}(t)}$ is its \emph{length}.
The set of all links existing in a certain configuration $\bar{\vec{x}}$ is denoted by $\C{E}(\bar{\vec{x}})$.
\end{definition}
Notice that $(i,j) \in \C{E}(\bar{\vec{x}}) \iff (j,i) \in \C{E}(\bar{\vec{x}})$.

% \begin{definition}[Virtual Forces]
% \label{def:VF}
% Given a swarm, the \emph{virtual forces} control law gives the control input of agent $i$ at time $t$ as 
% \begin{equation}
%     \label{eq:control_law}
%     \vec{u}_i(t)  \coloneqq \sum_{j \in \C{I}_i(t)} f(\norm{\vec{r}_{ij}(t)})\, \unitvec{r}_{ij}(t),
% \end{equation}
% where $f : \mathbb{R}_{\geq 0} \rightarrow \mathbb{R}$ is the \emph{interaction function}.
% \end{definition}

% \begin{comment}
% \begin{definition}[Triangular lattice]
% \label{def:triangular_lattice}
% Given a \emph{desired link length} $R \in \mathbb{R}_{>0}$, a \emph{triangular lattice} is a set of planar positions $\{ \vec{x}_i \}_{i=1\dots n} \subset \mathbb{R}^2$ such that, if $\norm{\vec{x}_i -\vec{x}_j}<R\sqrt{3} $, then $ \norm{\vec{x}_i -\vec{x}_j}=R$.
% \end{definition}
% This definition allows for the presence of isolated agents, i.e. those being farther than $R\sqrt{3}$ from all the others, to be in any position, and agents with only one adjacent agent closer than $R\sqrt{3}$ ($\exists! j \in \C{S} \setminus \{i\} : \norm{\vec{x}_i -\vec{x}_j}<R\sqrt{3}$) to lay on a circle \remark{MC: potrebbe essere un problema?}; but forces other agents to actually lay on a triangular lattice.
% Notice that multiple groups of agents will, eventually, form distinct lattices.
% \end{comment}

\begin{definition}[Swarm graph and framework]
\label{def:swram_graph}
The \emph{swarm graph} is the digraph $\C{G}(\bar{\vec{x}})\coloneqq(\C{S},\C{E}(\bar{\vec{x}}))$.
%, whose vertices correspond to the agents in the swarm and whose edges correspond to the links.
The \emph{swarm framework} is $\C{F}(\bar{\vec{x}})\coloneqq(\C{G}(\bar{\vec{x}}), \bar{\vec{x}})$.
\end{definition}

\begin{comment}
Next, let us define the \emph{desired link length} $R \in \BB{R}_{>0}$, which is the distance between adjacent agents in a triangular lattice, as stated in the next definition.
Moreover, let us say that $R_\R{a}$ is such that
\begin{equation}
    R_\R{a}\in ]R, R\sqrt{3}[,
    \label{eq:R_a}
\end{equation}
so that, when the swarm is in a triangular configuration (see Definition \ref{def:triangular_lattice} below), links exist only between agents at distance $R$. 
\remark{aggiungere figura lattice con R e Ra}
\end{comment}

\begin{definition}[Triangular lattice configuration]
\label{def:triangular_lattice}
Consider a planar swarm $\C{S}$ with framework $\C{F}(\bar{\vec{x}}^*)$.
$\bar{\vec{x}}^*$ 
 is a \emph{triangular (lattice) configuration} if
\begin{enumerate}[label=(\Alph*)]
    \item\label{condition:rigidity}
    $\C{F}(\bar{\vec{x}}^*)$
    is infinitesimally rigid, and
    \item\label{condition:link_length} 
    $\norm{\vec{r}_{ij}}=R,\ \forall (i,j)\in \C{E}(\bar{\vec{x}}^*)$,
\end{enumerate}
where $R \in \BB{R}_{>0}$ denotes the \emph{desired link length}.
\end{definition}
Here, we assume that
\begin{equation}\label{eq:R_a}
    R_\R{a}\in ]R, R\sqrt{3}[,
\end{equation}
so that, when the swarm is in a triangular configuration, the adjacency set (Definition~\ref{def:adjacency_set}) of any agent includes only the agents in its immediate surroundings, and all the links (Definition~\ref{def:links}) have length $R$ (see Fig. \ref{fig:triangular_lattices_examples}).

We denote by $\C{T} \subset \BB{R}^{2n}$ the set of all triangular lattice configurations; it is immediate to verify that $\C{T}$ is unbounded and disconnected.  

\begin{figure}[t]
    \centering
    \subfloat[]{
    \includegraphics[width=0.35\columnwidth]{triangular_lattice_scheme.pdf}\label{fig:R_a_R}} \qquad
    \subfloat[]{
    \includegraphics[trim=4mm 4mm 4mm 4mm, clip, width=0.35\columnwidth]{N=100_sparse_rep2_small.pdf}}
    
    \caption{
    Triangular configurations. 
    (a) Schematic representation of a triangular lattice; red agents belong to the adjacency set of the black agent.
    (b) Example of a triangular configuration with $n=100$ agents.}
    \label{fig:triangular_lattices_examples}
\end{figure} 

\begin{definition}[Congruent configurations]\label{def:congruent_conf}
Given a configuration $\bar{\vec{x}}^\diamond$, we define \emph{the set of its congruent configurations} $\Gamma(\bar{\vec{x}}^\diamond)$ as the set of configurations with congruent associated frameworks (see Definition~\ref{def:congurent_framework}), that is
\begin{equation*}
    \Gamma(\bar{\vec{x}}^\diamond) \coloneqq \{ \bar{\vec{x}} \in \BB{R}^{2n} : \norm{\vec{x}_{i}-\vec{x}_{j}} = \norm{\vec{x}_{i}^\diamond-\vec{x}_{j}^\diamond}, \forall i,j \in \C{S} \}.
\end{equation*}
\end{definition}
%
These configurations are obtained by translations and rotations of the framework $\C{F}(\bar{\vec{x}}^\diamond)$; thus, it is immediate to verify that $\Gamma(\bar{\vec{x}}^\diamond)$ is connected and unbounded for any $\bar{\vec{x}}^\diamond$ (see Fig. \ref{fig:triangular_lattices}).
%Notice that $\Gamma(\bar{\vec{x}}^*)$ is connected and unbounded for any configuration $\bar{\vec{x}}^*$
Also, note that $\bar{\vec{x}}^* \in \C{T} \iff \Gamma(\bar{\vec{x}}^*) \subset \C{T}$, and
\begin{equation}
    \label{eq:TasUnion}
    \C{T}=\bigcup_{\bar{\vec{x}}^* \in \C{T}} \Gamma(\bar{\vec{x}}^*).
\end{equation}

\begin{figure}[t]
    \centering
    \subfloat[]{
    \includegraphics[width=0.45\columnwidth]{triangular_lattices_v02.pdf}
    \label{fig:triangular_lattices}}
    \subfloat[]{
    \includegraphics[width=0.45\columnwidth]{all_sets_v05.pdf}
    \label{fig:all_sets}}
    \caption{(a): Sets of triangular lattices configurations. (b): Sets used in the proof of Theorem \ref{th:local_stability_lyap}.}
\end{figure}

In the following, we 
%(i) assume that $R_\R{a}$ is \emph{proper}, i.e. $R_\R{a}\in ]R, R\sqrt{3}[$, and (ii) 
omit the dependence on time when clear from the context.

%%%%%%%%%%%%%%%%%%%%%%%%%%%%%%%%%%%%%%%%%%%%%%%%%%%%%%%%%%%%%%%%%%%%%%%%
\section{Problem Statement}
Consider a swarm $\C{S}$ of $n$ agents, with agents' dynamics described by
%
\begin{equation}\label{eq:model}
    \dot{\vec{x}}_i(t) = \vec{u}_i(t), \ \ \forall i \in \C{S},
\end{equation}
%
where 
%$\vec{x}_i(t)$ was introduced in Definition \ref{def:swarm} and 
$\vec{u}_i(t)\in \mathbb{R}^2$ is a control law to be designed.

Let $R_\R{s} \in \BB{R}_{>0}$ be a \emph{sensing radius} and define the \emph{interaction set} of agent $i$ at time $t$ as 
\begin{equation}
    \C{I}_i(t) \coloneqq \{ j \in \C{S} \setminus \{i\} : \Vert \vec{r}_{ij}(t)\Vert \leq R_{\mathrm{s}} \}.
\end{equation}
For the term $\vec{u}_i(t)$ in \eqref{eq:model}, we consider the distributed \emph{virtual forces} control law, given by
\begin{equation}\label{eq:control_law}
    \vec{u}_i(t) \coloneqq \sum_{j \in \C{I}_i(t)} f\left( \norm{\vec{r}_{ij}(t)} \right)\, \unitvec{r}_{ij}(t),
\end{equation}
where $f : \mathbb{R}_{\geq 0} \rightarrow \mathbb{R}$ is the \emph{interaction function}.

Note that in general there is no specific relation between $\C{I}_i$ and $\C{A}_i$ (see Definition~\ref{def:adjacency_set}); however, we reasonably assume that $R_{\R{s}}\geq R_\R{a}$, so that
\begin{equation}\label{eq:A_subset_I}
    \C{A}_i \subseteq \C{I}_i, \quad \forall i\in \C{S}.
\end{equation}

%Under these assumptions, the swarm forms a triangular lattice (see Definition~\ref{def:triangular_lattice}) if and only if the length of all its links is $R$, that is $\norm{\vec{r}_{ij}}=R \ \forall (i,j)\in \C{E}$.

The following result slightly extends the one reported in {\cite[Lemma 1]{Gazi2002}}.
% \footnote{\color{Red}[MC: non è più o meno uguale?]AG: è uguale ma rilassa le ipotesi sulla funzione, loro assumono che la funzione abbia una serie di propietà. MC ok, grazie}
\begin{lemma}\label{th:invariant_center}
The position of the center of the swarm (see Definition~\ref{def:swarm}), say $\vec{x}_{\R{c}}$, under the control law \eqref{eq:control_law} is invariant, that is 
$%\begin{equation}
    %\label{eq:invariant_center}
    \dot{\vec{x}}_{\R{c}} = \vec{0}\ \forall \bar{\vec{x}} \in \BB{R}^{2n}.
$%\end{equation}
\end{lemma}
\begin{proof}
Exploiting \eqref{eq:model} and \eqref{eq:control_law}, the dynamics of the center of the swarm is given by
\begin{equation}
    \label{eq:velocity_cm}
    \dot{\vec{x}}_{\R{c}} \coloneqq \frac{1}{n} \sum_{i = 1}^n \dot{\vec{x}}_i = \frac{1}{n} \sum_{i = 1}^n {\vec{u}}_i =
    \frac{1}{n} \sum_{i = 1}^n \sum_{j \in \C{I}_i} f(\norm{\vec{r}_{ij}})\, \unitvec{r}_{ij}.
\end{equation}
Since in a swarm the existence of any link $(i,j)$ implies the existence of link $(j,i)$ (see Definition~\ref{def:adjacency_set}), in \eqref{eq:velocity_cm}, for any term $f(\norm{\vec{r}_{ij}}) \, \unitvec{r}_{ij}$ there exists a term $f(\norm{\vec{r}_{ji}}) \, \unitvec{r}_{ji}=-f(\norm{\vec{r}_{ij}}) \, \unitvec{r}_{ij}$ (because $\norm{\vec{r}_{ij}} = \norm{\vec{r}_{ji}}$ and $\unitvec{r}_{ij} = - \unitvec{r}_{ji}$).
Therefore, the sum of the two is zero, yielding the thesis.
\end{proof}

%----------------------------------------------------
\section{Convergence to a triangular configuration}
\label{sec:main_results}
%Next, we study the local stability of the swarm around the triangular configurations. %(see Definition~\ref{def:triangular_lattice}).
We can now state the main result of this work, showing that, given an interaction function $f$ (in \eqref{eq:control_law}) that generates short range repulsion and long range attraction, the set of triangular configurations of the swarm is a locally asymptotically stable equilibrium set (see Definitions \ref{def:equilibrium_set} and \ref{def:LAS}).

\begin{comment}
    
\end{comment}

\begin{assumption}\label{ass:interaction_function}
    $f$ (in \eqref{eq:control_law}) is such that:
\begin{enumerate}[label=(a\arabic*)]
    \item \label{hp:null_point} $f(R)=0$,
    %\item \label{hp:negative_derivative} $f(z)$ is differentiable in $z = R$ and $f'(R)=-k$ with $k>0$,
    \item \label{hp:attraction_repulsion} $f(z) > 0$ for $z \in [0;R [$ and $f(z) < 0$ for $z>R$, 
    \item \label{hp:integrable} $f(z)$ is continuous in $[0; R_\R{a}]$,
    \item \label{hp:vanishing} $f(z) = 0$ for any $z > R_\R{a}$,
    % \item \label{hp:equilibria} {\color{Blue}\sout{ in an arbitrary small neighborhood of a triangular configuration, all equilibria are triangular configurations.}}
    \end{enumerate}
\end{assumption}

An exemplary interaction function fulfilling the assumption above is portrayed in Fig. \ref{fig:interaction_function_and_potential}.

Without loss of generality, we further assume that, under Assumption \ref{ass:interaction_function}, in a sufficiently small neighborhood of a triangular configuration, all other equilibria are also triangular (supporting evidence showing that this assumption is not restrictive is reported in the \hyperref[sec:appendix]{Appendix}).

\begin{figure}[t]
    \centering
    \includegraphics{interaction_function_and_potential_v02.pdf}
    \caption{Example of an interaction function $f$ (top panel) and its corresponding potential $P$ (bottom panel).}
    \label{fig:interaction_function_and_potential}
\end{figure}

\begin{theorem}\label{th:local_stability_lyap}
Let Assumption \ref{ass:interaction_function} hold.
Then, for any triangular configuration $\bar{\vec{x}}^*$, 
$\Gamma(\bar{\vec{x}}^*)$ is a locally asymptotically stable equilibrium set.
Consequently, $\C{T}$ is also a locally asymptotically stable equilibrium set.
\end{theorem}
%
\begin{proof}%[Proof of Theorem \ref{th:local_stability_lyap}]
Let us consider \emph{any} triangular configuration $\bar{\vec{x}}^* \in \C{T}$, 
with center $\vec{x}^*_{\R{c}} \coloneqq \frac{1}{n} \sum_{i = 1}^n \vec{x}_i^*$ and relative positions $\B{r}_{ij}^*$,
and the set $\Gamma(\bar{\vec{x}}^*)$ of its congruent configurations.
Recalling Definition~\ref{def:triangular_lattice}.\ref{condition:link_length} and \ref{hp:null_point}, we have that $\bar{\vec{x}}^*$ is an equilibrium point of \eqref{eq:model}--\eqref{eq:control_law}; thus, $\Gamma(\bar{\vec{x}}^*)$ and $\C{T}$ are equilibrium sets.
%(Definition~\ref{def:equilibrium_set}).
% {\color{Blue}Without loss of generality, we assume that in a sufficiently small neighborhood of a triangular configuration, all other equilibrium configurations are also triangular.}
Next, we will prove local asymptotic stability
%(Definition~\ref{def:LAS}) 
of $\Gamma(\bar{\vec{x}}^*)\subset \C{T}$, which implies local asymptotic stability of $\C{T}$ through \eqref{eq:TasUnion}.
%The rest of the proof is organized as follows.
%In Step 1, we define a Lyapunov function $V$;
%in Step 2, we restrict the analysis to a neighborhood of $\Gamma(\bar{\vec{x}}^*)$ where certain properties hold on $V$;
%in Step 3; we derive and study the expression for $\dot{V}$;
%finally, in Step 4, we apply LaSalle's theorem to obtain the thesis.

%--------------------------------------------------
\paragraph*{Step 1 (Lyapunov function)}

Given a configuration $\bar{\vec{x}} \in \BB{R}^{2n}$ with center $\vec{x}_{\R{c}}$ and inducing the links in $\C{E}(\bar{\vec{x}})$ according to Definition~\ref{def:links}, let $m \coloneqq \lvert \C{E}(\bar{\vec{x}}) \rvert$ and order the links in $\C{E}(\bar{\vec{x}})$ arbitrarily, so that $\vec{r}_1, \dots, \vec{r}_m$ refer to the relative positions $\vec{r}_{ij}$ for $(i,j)\in \C{E}(\bar{\vec{x}})$.
%and define $\vec{d} \coloneqq \left[\norm{\vec{r}_1}, \dots, \norm{\vec{r}_m} \right]\T \in \mathbb{R}_{\geq 0}^m$.%
Recalling \ref{hp:integrable}, we can define the potential function
$P:[0, R_\R{a}]\to \BB{R}$ given by $P(z)=- \int_{R}^{z} f(y) \, \R{d}y$ (see Fig. \ref{fig:interaction_function_and_potential}). 
Note that $P(R)=0$, $\frac{\R{d}P}{\R{d} z}(z)= -f(z)$, and, from \ref{hp:attraction_repulsion},
\begin{equation} \label{eq:P_positive}
    P(z) > 0 \quad \forall z \in \BB{R}_{\geq0} \setminus \{R\} .    
\end{equation}

Then, let us consider the candidate Lyapunov function
\begin{equation}\label{eq:V+cm}
\begin{aligned}
    V(\bar{\vec{x}}) &\coloneqq \norm{\vec{x}^*_\R{c} - \vec{x}_\R{c} }^2 + \sum_{k\in\C{E}(\bar{\vec{x}})} P(\norm{\vec{r}_{k}}).
\end{aligned}
\end{equation}
By \eqref{eq:P_positive}, it holds that
%\begin{equation}
    %\label{eq:Vpositivedef1}
    $V(\bar{\vec{x}})\geq0 \ \forall \bar{\vec{x}} \in \BB{R}^{2n}$,
%\end{equation}
and $V = 0$ if and only if both $\vec{x}_{\R{c}} =\vec{x}_{\R{c}}^*$ and Definition~\ref{def:triangular_lattice}.\ref{condition:link_length} holds%
%(i.e., $\norm{\vec{r}_i}=R \ \forall i\in \C{E}(\bar{\vec{x}})$)
. %i.e.,
%and have the same center of $\bar{\vec{x}}^*$, 
% \begin{equation}
%     %\label{eq:Vpositivedef2}
%     V(\bar{\vec{x}})=0 \iff \norm{\vec{r}_i}=R \ \forall i\in \C{E}(\bar{\vec{x}}) \ \land \ \vec{x}_{\R{c}} = \vec{x}_{\R{c}}^* .
% \end{equation}

%--------------------------------------------------
\paragraph*{Step 2 (Properties of $V$)}

$V(\bar{\vec{x}})$ is discontinuous over $\BB{R}^{2n}$ (because $\C{E}(\bar{\vec{x}})$ changes when links (dis-)appear).
However, $V(\bar{\vec{x}})$ is continuous and differentiable in any subset of $\BB{R}^{2n}$ where the set $\C{E}(\bar{\vec{x}})$ of links is constant.
To find such a set, we seek conditions on $\vec{\bar{x}}$ such that $\C{E}(\vec{\bar{x}}) = \C{E}(\vec{\bar{x}}^*)$ (see Definitions \ref{def:adjacency_set} and \ref{def:links}), i.e.,
\begin{subequations}
\begin{align}
    \norm{\vec{r}_{ij}} &< R_\R{a}, \quad \forall (i, j) \in \C{E}(\vec{\bar{x}}^*),\label{eq:links_preserved}\\
    \norm{\vec{r}_{ij}} &> R_\R{a}, \quad \forall (i, j) \not\in \C{E}(\vec{\bar{x}}^*).\label{eq:no_links_created}
\end{align}
\end{subequations}
\eqref{eq:links_preserved} means that all links in $\C{E}(\vec{\bar{x}}^*)$ are preserved in $\C{E}(\vec{\bar{x}})$, while \eqref{eq:no_links_created} means that no new links are created in $\C{E}(\vec{\bar{x}})$ with respect to $\C{E}(\vec{\bar{x}}^*)$.
\begin{comment} equations rewritten
We can rewrite \eqref{eq:links_preserved} as
\begin{equation}\label{eq:links_preserved_rewritten}
    \norm{\vec{r}_{ij}} - \norm{\vec{r}_{ij}^*} < R_\R{a} - \norm{\vec{r}_{ij}^*} \le \abs{R_\R{a} - \norm{\vec{r}_{ij}^*}}.
\end{equation}
On the other hand, \eqref{eq:no_links_created} can be rewritten as
\begin{equation}\label{eq:no_links_created_rewritten}
    - \norm{\vec{r}_{ij}} + \norm{\vec{r}_{ij}^*} < - R_\R{a} + \norm{\vec{r}_{ij}^*} \le \abs{R_\R{a} - \norm{\vec{r}_{ij}^*}}.
\end{equation}
\end{comment}
With simple algebraic manipulations it is possible to show that  \eqref{eq:links_preserved} and \eqref{eq:no_links_created} hold if $\bar{\vec{x}} \in \C{B}$, where
\begin{equation}
    \label{eq:setB}
    \C{B} \coloneqq \{ \bar{\vec{x}} \in \mathbb{R}^{2n} : \abs{ \norm{\vec{r}_{ij}} - \norm{\vec{r}_{ij}^*} } < \beta, \ \forall i,j \in \C{S} \},
\end{equation}
and $\beta < \min_{i,j\in \C{S}}  \abs{ R_\R{a} - \norm{\vec{r}_{ij}^*} }$.
Note that $\C{B}$ can be interpreted as a ``neighborhood'' of $\Gamma(\bar{\vec{x}}^*)$ with ``width'' $\beta$ (see Fig. \ref{fig:all_sets}). %(obviously, $\Gamma(\bar{\vec{x}}^*) \subset \C{B}$).
Hence, $\C{E}(\vec{\bar{x}}) = \C{E}(\vec{\bar{x}}^*)$ in $\C{B}$, and thus $V$ is continuously differentiable in $\C{B}$.
%Moreover, it is straightforward to verify that $\C{B}$ is unbounded and connected, and that $\Gamma(\bar{\vec{x}}^*) \subset \C{B}$.
%($\C{D}$ and the other relevant sets in the Proof are portrayed in Fig.~\ref{fig:all_sets}.


% Namely, this happens if%
% \begin{enumerate}
%     \item for some $(i, j) \in \C{E}(\vec{\bar{x}}^*)$, $\norm{\vec{x}_i - \vec{x}_j} > R_\R{a}$ (i.e., the existing link between $i$ and $j$ disappears), or
%     \item for some $(i, j) \not\in \C{E}(\vec{\bar{x}}^*)$, $\norm{\vec{x}_i - \vec{x}_j} < R_\R{a}$ (i.e., a link between $i$ and $j$ is formed).
% \end{enumerate}
%From these considerations, we define the open set
%(see Fig.~\ref{fig:all_sets})
% \begin{equation}
%     \label{eq:setD}
%     \C{D}\coloneqq \{ \bar{\vec{x}} \in \mathbb{R}^{2n} : \abs{ \norm{\vec{r}_{ij}} - \norm{\vec{r}_{ij}^*} } < \abs{ R_\R{a} - \norm{\vec{r}_{ij}^*} }, \, \forall i,j \in \C{S} \},
% \end{equation}
%


%(ii) Since the framework $\C{F}(\bar{\vec{x}})$ is infinitesimally rigid for all $\bar{\vec{x}} \in \Gamma(\bar{\vec{x}}^*)$, exploiting \cite[Lemma 6.6]{Mesbahi2010}, there exists a small $\beta$ such that $\C{F}(\bar{\vec{x}})$ is infinitesimally rigid also for all $\bar{\vec{x}} \in \C{B}$ (see Definitions \ref{def:framework}, \ref{def:inf_rigidity} and \ref{def:swram_graph}).




%--------------------------------------------------
\paragraph*{Step 3 (Analysis of $\dot{V}$)}

At this point, we can restrict our analysis to the set $\C{B}$ to study the attractivity of $\Gamma(\bar{\vec{x}}^*)$.
Let us start by studying the dynamics of the agents.
From \eqref{eq:model}--\eqref{eq:control_law}, we have
\begin{equation}\label{eq:x_i_dot}
    \dot{\vec{x}}_{i}= \sum_{j\in \C{I}_i} f(\Vert \vec{r}_{ij}\Vert)  \unitvec{r}_{ij}.
\end{equation}
Hypothesis \ref{hp:vanishing} and \eqref{eq:A_subset_I} imply that in \eqref{eq:x_i_dot} we have
\begin{equation}\label{eq:replace_I_with_A}
    \sum_{\substack{j\in \C{I}_i}} f(\norm{\vec{r}_{ij}})  \unitvec{r}_{ij} =
    \sum_{\substack{j\in \C{A}_i}} f(\norm{\vec{r}_{ij}})  \unitvec{r}_{ij}.
\end{equation}
Then, exploiting \eqref{eq:replace_I_with_A} and the incidence matrix $\vec{B}$ (Definition~\ref{def:incidence_mat}) of the swarm graph,
%(Definition~\ref{def:swram_graph}),
\eqref{eq:x_i_dot} can be rewritten as%
%\footnote{\color{Green}Provided that the orientation of the $\unitvec{r}_{j}$'s is selected properly.}
\begin{equation} \label{eq:x_i_dot_Incidence}
\begin{aligned} 
    \dot{\vec{x}}_{i}= \sum_{j\in \C{A}_i} f(\Vert \vec{r}_{ij}\Vert)  \unitvec{r}_{ij} = \sum_{k=1}^m [\vec{B}]_{ik} f(\Vert \vec{r}_{k}\Vert) \unitvec{r}_{k}.
\end{aligned}
\end{equation}
Moreover we can write the dynamics of the relative positions along a link $k$ as
$%\begin{equation}\label{eq:r_i_dot}
    \dot{\vec{r}}_{k}= \sum_{i=1}^n [\vec{B}]_{ik} \dot{\vec{x}}_{i}.
$ %\end{equation} 
%
Thus, exploiting Lemma \ref{th:invariant_center} and \eqref{eq:x_i_dot_Incidence},
%and \eqref{eq:r_i_dot}
we get
\begin{equation*}%\label{eq:Vdot}
\begin{aligned}
    \dot{V}(&\bar{\vec{x}}) = \sum_{k=1}^m \frac{\partial V}{\partial \norm{\vec{r}_k}}\ \frac{\partial \norm{\vec{r}_k}}{\partial \vec{r}_k} \ \dot{\vec{r}}_k 
    = \sum_{k=1}^m P'(\norm{\vec{r}_k}) \ \unitvec{r}_k\T  \sum_{i=1}^n [\vec{B}]_{ik} \dot{\vec{x}}_{i} \\
    &=-\sum_{i=1}^n  \sum_{k=1}^m [\vec{B}\T]_{ki}  \ f(\norm{\vec{r}_k}) \ \unitvec{r}_k\T \dot{\vec{x}}_{i} 
    =-\sum_{i=1}^n \dot{\vec{x}}_i\T \dot{\vec{x}}_i = - \dot{\bar{\vec{x}}}\T \dot{\bar{\vec{x}}}.
\end{aligned}
\end{equation*}
%
\begin{comment} % erivazione di V con edge Laplacian
\begin{equation}\label{eq:r_ij_dot}
\begin{aligned}
    \dot{\vec{r}}_{ij}&=\dot{\vec{x}}_{i}-\dot{\vec{x}}_{j}= \sum_{k\in \C{I}_i} f(\Vert \vec{r}_{ik}\Vert)  \unitvec{r}_{ik} - \sum_{k\in \C{I}_j} f(\Vert \vec{r}_{jk}\Vert)  \unitvec{r}_{jk}=\\ 
    &=2 f(\Vert \vec{r}_{ij}\Vert) \unitvec{r}_{ij} + \sum_{\substack{k\in \C{I}_i \\ k\neq j}}  f(\Vert \vec{r}_{ik}\Vert)  \unitvec{r}_{ik} -  \sum_{\substack{k\in \C{I}_j \\ k\neq i}} f(\Vert \vec{r}_{jk}\Vert)  \unitvec{r}_{jk}, % = \\
    %&=2 \ g(\Vert \vec{r}_{ij}\Vert) \vec{r}_{ij} + \sum_{\substack{k\in \C{I}_i \\ k\neq j}} g(\Vert \vec{r}_{ik}\Vert)  \vec{r}_{ik} -  \sum_{\substack{k\in \C{I}_j \\ k\neq i}} g(\Vert \vec{r}_{jk}\Vert)  \vec{r}_{jk}
\end{aligned}
\end{equation}
where we exploited again that $f(\norm{\vec{r}_{ji}})\unitvec{r}_{ji}=-f(\norm{\vec{r}_{ij}})\unitvec{r}_{ij}$.
Then, exploiting \eqref{eq:replace_I_with_A} and the edge Laplacian $\vec{L}_{\R{e}}$ (Definition~\ref{def:edge_laplacian}) of the swarm graph, \eqref{eq:r_ij_dot} can be rewritten as
\begin{equation}
    \dot{\vec{r}}_{i} = \sum_{j=1}^m \left[\vec{L}_{\R{e}} \right]_{ij} f(\Vert \vec{r}_{j}\Vert) \unitvec{r}_{j} .
    \label{eq:relPositionsMicroModel}
\end{equation}
Next, exploiting Lemma \ref{th:invariant_center} and \eqref{eq:relPositionsMicroModel}, we get
\begin{align*}
    \dot{V}(\bar{\vec{x}}) &= \sum_{i=1}^m \frac{\partial V}{\partial \norm{\vec{r}_i}}\ \frac{\partial \norm{\vec{r}_i}}{\partial \vec{r}_i} \ \dot{\vec{r}}_i \\
    &= \sum_{i=1}^m P'(\norm{\vec{r}_i}) \unitvec{r}_i\T \sum_{j=1}^m [\vec{L}_{\R{e}}]_{ij} f(\norm{\vec{r}_j}) \unitvec{r}_j\\
    &=-\sum_{i=1}^m\sum_{j=1}^m [\vec{L}_{\R{e}}]_{ij} \ f(\norm{\vec{r}_i}) \ f(\norm{\vec{r}_j}) \ \unitvec{r}_i\T \unitvec{r}_j \\
    %&=-\sum_{i=1}^m\sum_{j=1}^m \left( [\vec{L}_{\R{e}}]_{ij} \ f(\norm{\vec{r}_i}) \right) \left( f(\norm{\vec{r}_j}) \ \unitvec{r}_j\T \unitvec{r}_i \right) .
    &=-\sum_{i=1}^m\sum_{j=1}^m \sum_{k=1}^n [\vec{B}\T]_{ik} [\vec{B}]_{kj} \ f(\norm{\vec{r}_i}) \ f(\norm{\vec{r}_j}) \ \unitvec{r}_i\T \unitvec{r}_j \\
    &=-\sum_{k=1}^n \dot{\vec{x}}_k\T \dot{\vec{x}}_k = - \dot{\bar{\vec{x}}}\T \dot{\bar{\vec{x}}}.
\end{align*}
\end{comment}
%
\begin{comment} % Forma matriciale
Now, let us define the diagonal matrix $\B{F} \in \BB{R}^{m \times m}$, such that $[\B{F}]_{ii} = f(\norm{\vec{r}_i})$, and the matrix $\B{Q} \coloneqq [\unitvec{r}_1 \ \cdots \ \unitvec{r}_m] \in \BB{R}^{2 \times m}$.
Then, we obtain {\cite[p.~479]{Horn2012}} 
\begin{equation}
\begin{aligned}
    \dot{V}(\bar{\vec{x}}) 
    &=-\sum_{i=1}^m\sum_{j=1}^m [\vec{F} \vec{L}_{\R{e}}]_{ij} [\vec{F} \vec{Q}\T \vec{Q}]_{ji}=\\
    &=-\tr \left( \vec{F} \vec{L}_{\R{e}} \vec{F} \vec{Q}\T \vec{Q} \right).
\end{aligned}
\end{equation}
By the cyclic property of the trace {\cite[p.~55, problem 1.2.P2]{Horn2012}}, we have
\begin{equation}
\begin{aligned}
    \dot{V}(\bar{\vec{x}}) &= -\tr\left( \vec{Q} \vec{F} \vec{L}_{\R{e}} \vec{F} \vec{Q}\T  \right) .
\end{aligned}
\label{eq:Vdot}
\end{equation}

Now, we show that $\dot{V}(\bar{\vec{x}}) = 0$ implies that $\bar{\B{x}}$ is an equilibrium point.
To do so, define 
$ \B{A} \coloneqq \vec{Q} \vec{F} \vec{L}_{\R{e}}\vec{F}\vec{Q}\T \in \BB{R}^{2 \times 2}$ and
$\bar{\vec{X}} \coloneqq [\vec{x}_1 \ \cdots \ \vec{x}_n]\T \in \BB{R}^{n\times 2}$.
It is straightforward to verify that the model \eqref{eq:model}--\eqref{eq:control_law} can be rewritten as
\begin{equation}
            \dot{\bar{\vec{X}}} = \vec{B} \vec{F} \vec{Q}\T
\end{equation}
and that $\vec{A}= \dot{\bar{\vec{X}}}\T \dot{\bar{\vec{X}}}$.
$\vec{A}$ is symmetric and positive semi-definite, therefore its eigenvalues are all real and non-negative; moreover $\dot{V}(\bar{\vec{x}})=-\tr(\vec{A})=0$ if and only if $\dot{\bar{\vec{X}}}=\vec{0}$, i.e., in correspondence of equilibrium configurations.
\end{comment}
%
%\remark{Derivazione alternativa, forse è più chiara ed un po' più breve perchè non richiede la matrice di incidenza. Però va aggiustata per prendere in considerazione il primo termine di $V$}
% {\color{Green}
% \begin{equation} \label{eq:grad_V}
% \begin{aligned}
%    \frac{\partial V}{\partial \vec{x}_i} &= \frac{\partial}{\partial \vec{x}_i} \sum_{(i,j)\in \C{E}} P(\norm{\vec{r}_{ij}})
%    = \sum_{j\in \C{A}_i} \frac{\partial P(\norm{\vec{r}_{ij}})}{\partial \vec{x}_i} \\
%    &= \sum_{j\in \C{A}_i} P'(\norm{\vec{r}_{ij}}) \frac{\partial \norm{\vec{r}_{ij}}}{\partial \vec{r}_{ij}} \ \frac{\partial \vec{r}_{ij}}{\partial \vec{x}_{i}} 
%    \\
%    &=-\sum_{j\in \C{A}_i} f(\norm{\vec{r}_{ij}}) \ \unitvec{r}_{ij}\T
%    =-\dot{\vec{x}}_i\T.
% \end{aligned}
% \end{equation}

% Therefore, $ \frac{\partial V}{\partial \bar{\vec{x}}} = - \dot{\bar{\vec{x}}}\T$ and 
% \begin{equation}
%    \dot{V}(\bar{\vec{x}}) =\frac{\partial V}{\partial \bar{\vec{x}}}\ \dot{\bar{\vec{x}}}
%    = - \dot{\bar{\vec{x}}}\T \dot{\bar{\vec{x}}}.
% \end{equation}
% }
%
We can hence conclude that $\dot{V}(\bar{\vec{x}})=0$ if and only if $\dot{\bar{\vec{x}}}=\vec{0}$, i.e., in correspondence of equilibrium configurations.

%\ref{hp:equilibria} 
Now, choosing $\beta$ small enough, we can exclude the presence of equilibrium configurations not belonging to $\Gamma(\bar{\vec{x}}^*)$, and therefore 
\begin{equation}
\begin{cases}\label{eq:characterization_V_dot}
    \dot{V}(\bar{\vec{x}})=0, 
    &\mbox{if }\bar{\vec{x}} \in \Gamma(\bar{\vec{x}}^*),\\
    \dot{V}(\bar{\vec{x}}) < 0,
    &\mbox{if }\bar{\vec{x}} \in \C{B} \setminus \Gamma(\bar{\vec{x}}^*).
\end{cases}
\end{equation}

%$\dot{V}(\bar{\vec{x}})<0 \ \forall \bar{\vec{x}} \in \C{B}_{\beta} \setminus \Gamma(\bar{\vec{x}}^*)$ and $\dot{V}(\bar{\vec{x}})=0 \ \forall \bar{\vec{x}} \in \Gamma(\bar{\vec{x}}^*)$.

%--------------------------------------------------
\paragraph*{Step 4 (Applying LaSalle's invariance principle)}
To complete the proof, we define a forward invariant neighborhood of $\bar{\vec{x}}^*$ and then apply LaSalle's invariance principle. 
%\cite[Theorem~4.4]{Khalil2002}.
Given some $\omega \in \BB{R}_{>0}$, let $\Omega$ be the largest connected set containing $\bar{\vec{x}}^*$ such that
$ V(\bar{\vec{x}}) \leq \omega \ \forall \bar{\vec{x}} \in \Omega$ (see Fig. \ref{fig:all_sets}).
In particular, we select $\omega$ small enough that $\Omega \subseteq \C{B}$.%
\footnote{Such a value of $\omega$ exists because $\C{B}$ is a ``neighborhood'' of $\Gamma(\bar{\vec{x}}^*)$ (in the sense of \eqref{eq:setB}) and,  by the rigidity of framework $\C{F}(\vec{\bar{x}}^*)$ (Definition~\ref{def:rigidity}), any continuous motion of the vertices that changes the distance between any two vertices also changes the length of at least one link, causing $V$ to increase.}
Since $V(\bar{\vec{x}}) \leq \omega$ and $\dot{V}(\bar{\vec{x}}) \leq 0$ for all $ \bar{\vec{x}} \in \Omega$, then $\Omega$ is forward invariant.
%Now, recall that \eqref{eq:Vpositivedef1} and \eqref{eq:Vpositivedef2} imply that starting from $\vec{\bar{x}}^*$, any change in length of an edge will cause $V$ to increase.
%Moreover, by the rigidity of framework $\C{F}(\vec{\bar{x}}^*)$ (Definition~\ref{def:rigidity}), any continuous motion of the vertices that changes the distance between any two vertices will also change the length of at least one edge, and thus increase $V$.
% Therefore, we can state that, for all $\beta$ small enough that $\C{B}_\beta \subseteq \C{D}$, there exists some $\omega$ such that $\Omega_{\omega} \subseteq \C{B}_{\beta}$
% \begin{equation}
% \begin{aligned}
%     &\forall \beta\in \ ]0, \beta_{\C{E}}[ \ \exists \omega>0 : \\ 
%     &V(\bar{\vec{x}})\leq \omega \implies \abs{\norm{\vec{r}_{ij}}-\norm{\vec{r}_{ij}^*}} < \beta \ \forall i,j \in \C{S},
% \end{aligned}
% \end{equation}
% or equivalently
% \remark{semplificare}
% \begin{equation}
%     \forall \beta \in \ ]0, \beta_\C{E}[ \ \exists \omega>0 : \Omega_{\omega} \subseteq \C{B}_{\beta}.
% \end{equation}
Moreover, 
%Now, choosing $\beta<\beta_{\max}$ and $\omega$ such that  $\Omega_{\omega}  \subseteq \C{B}_{\beta}$ (see Fig. \ref{fig:all_sets}), 
$\Omega$ is closed, because $V$ is continuous in $\Omega$, and $\Omega$ is the inverse image of the closed set $[0,\omega]$.
$\Omega$ is also bounded because (i) translations too far from $\bar{\vec{x}}^*$ cause $V$ to increase beyond $\omega$ (see \eqref{eq:V+cm}), and (ii) $\Omega \subseteq \C{B}$ implies that the deformations of the framework are bounded (see \eqref{eq:setB}).
Since $\Omega$ is closed and bounded, it is also compact.
%, meaning that any trajectory starting in $\Omega_{\omega}$ will remain in it.

As $\Omega$ is compact and forward invariant, we can apply LaSalle's invariance principle \cite[Theorem~4.4]{Khalil2002}, and noting that, in $\Omega$, $\dot{V}(\bar{\vec{x}})=0$ if and only if $\bar{\vec{x}} \in \Gamma(\bar{\vec{x}}^*)$ (see \eqref{eq:characterization_V_dot}), we get that all the trajectories starting in $\Omega$ converge to $\Gamma(\bar{\vec{x}}^*) \cap \Omega$.
This and the forward invariance of $\Omega$ imply that $\Gamma(\bar{\vec{x}}^*)$ is locally asymptotically stable,
%(see Definition~\ref{def:LAS})
%for all $\bar{\vec{x}}^*\in \C{T}$.
and so is $\C{T}$ because of \eqref{eq:TasUnion}.
\end{proof}

%\begin{rem}
%\label{rem:isolated_eqilibria}
% In Section~\ref{sec:validation} below, we verified numerically that the results of the theorem still hold when \ref{hp:vanishing} only holds approximately. 
%While \ref{hp:equilibria} might appear difficult to verify, in reality, in all cases we considered, it held true as a consequence of \ref{hp:null_point}, \ref{hp:attraction_repulsion}, \ref{hp:vanishing}, and $f(z)$ being differentiable in $z=R$.
%This was verified through a numerical eigenvector analysis of the linearized dynamics of \eqref{eq:model}--\eqref{eq:control_law}, repeated for $10$ triangular configurations for each value of $n \in \{25, 26, \dots, 100\}$.
%\end{rem}

%%%%%%%%%%%%%%%%%%%%%%%%%%%%%%%%%%%%%%%%%%%%%%%%%%%%%%%%%%%%%%%%%%%
\section{Numerical validation}
\label{sec:validation}

In this section, we validate numerically the result presented in Section \ref{sec:main_results} and estimate the basin of attraction of $\C{T}$.
%See the \hyperref[sec:appendix]{Appendix} for further evidence.

%--------------------------------------
\subsection{Simulation setup}

We set the desired link length to $R=1$, the maximum link length to $R_\R{a}=  (1+\sqrt{3})/2 \approx 1.37$, the sensing radius to $R_\R{s}=3$, and the number of agents to $n=100$.

The interaction function $f$ is chosen as the Physics-inspired Lennard-Jones function \cite{Brambilla2013, Giusti2022},
%typically used in the Literature e.g. 
given by
\begin{equation}\label{eq::Lennard-Jones}
    f(z) = \min \left\lbrace \left( \frac{a}{z^{2c}}-\frac{b}{z^c}\right), \ 1 \right\rbrace,
\end{equation}
where we select $a = b = 0.5$ and $c=12$; see Fig.~\ref{fig:interaction_function}.
In \eqref{eq::Lennard-Jones}, $f$ is saturated to $1$ to avoid divergence of $f$ for $z \to 0$.
Concerning Assumption \ref{ass:interaction_function}, the interaction function $f$ satisfies \ref{hp:null_point}, \ref{hp:attraction_repulsion}, and \ref{hp:integrable}.
Also, as shown in Fig.~\ref{fig:interaction_function}, it quickly tends to zero so that we can assume it practically satisfies \ref{hp:vanishing}.
The choice of not setting $f(z)$ exactly equal to zero for $z\geq R_a$ is intentional as it allows to account for long range attraction between the agents, which is frequently required in swarm robotics applications \cite{Gazi2002}.

\begin{figure}[t]
    \centering
    \includegraphics[width=0.9\columnwidth]{interaction_function_ylabel.pdf}
    \caption{Plot of the interaction function defined by \eqref{eq::Lennard-Jones}. The zero of the function is highlighted by a red dot.}
    \label{fig:interaction_function}
\end{figure}

To assess if the swarm is in a triangular configuration, we check the conditions in Definition~\ref{def:triangular_lattice}.
%We define the Boolean variable $\rho(t)$ as $1$ if Definition~\ref{def:triangular_lattice}.\ref{condition:rigidity} holds and 0 otherwise,
%To compute $\rho$ we use Theorem \ref{th:rigidity}.
%
To evaluate whether a configuration is infinitesimally rigid, we use Theorem \ref{th:rigidity}.
Moreover, we define the \emph{error} 
$e(t) \coloneqq \max_{k\in\mathcal{E}(t)} \abs{\norm{\vec{r}_{k}(t)} - R}$,
which is zero when the configuration is triangular. 
Also, as long as $e(t)$ is lower than $R_\R{a}-R$, links in the configuration of interest are neither created nor destroyed.

For each simulation, the initial positions of the agents are obtained by picking a random triangular configuration and then applying, to each agent, a random displacement drawn from a uniform distribution over a disk of radius $\delta \in \BB{R}_{\ge 0}$.
%Notice that $e(0)\leq 2\delta$.



%existing links are preserved and no new ones are established 
%(see Definition~\ref{def:links}).
%The simultaneous presence of rigidity and a small value of $e$ will denote a triangular configuration.

All simulation are run in {\sc Matlab}%
\footnote{Code available at {\url{https://github.com/diBernardoGroup/SwarmSimPublic}}.} and last $20\, \text{s}$;
the agents' dynamics \eqref{eq:model}--\eqref{eq:control_law} are integrated using the forward Euler method with a fixed time step equal to $0.01\, \text{s}$.

\begin{rem}
\label{rem:extension_3D}
Theorem~\ref{th:local_stability_lyap}, together with Definitions~\ref{def:swarm} and \ref{def:triangular_lattice} allow a straightforward extension of the analysis to the three-dimensional case ($d = 3$). The only cumbersome step is to assess the infinitesimal rigidity of the 3D framework of interest as  Theorem~\ref{th:rigidity} can no longer be applied.
\end{rem}


\begin{comment}
\begin{figure}[t] % initial configurations for various values of delta
    \centering

    % \subfloat[$\delta=0$, $e=0$]{
    % \includegraphics[width=0.4\columnwidth]{delta=0_small.pdf}
    % \label{subfig:delta=0}}
    % \subfloat[$\delta=0.1$, $e=0.19$]{
    % \includegraphics[width=0.4\columnwidth]{delta=010_small.pdf}
    % \label{subfig:delta=0.1}}
    
    % \subfloat[$\delta=(R_a-1)/2$, $e=0.31$]{
    % \includegraphics[width=0.4\columnwidth]{delta=018_small.pdf}
    % \label{subfig:delta=0.18}}
    % \subfloat[$\delta=0.2$, $e=0.34$]{
    % \includegraphics[width=0.4\columnwidth]{delta=020_small.pdf}
    % \label{subfig:delta=0.2}}
    
    % \subfloat[$\delta=0.3$, $e=0.45$]{
    % \includegraphics[width=0.4\columnwidth]{delta=030_small.pdf}
    % \label{subfig:delta=0.3}}
    % \subfloat[$\delta=0.5$, $e=0.79$]{
    % \includegraphics[width=0.4\columnwidth]{delta=050_small.pdf}
    % \label{subfig:delta=0.5}}

    \subfloat[$\delta=0$, $e=0$]{
    \includegraphics[width=0.32\columnwidth]{delta=0_small.pdf}
    \label{subfig:delta=0}}
    \subfloat[$\delta=0.2$, $e=0.34$]{
    \includegraphics[width=0.32\columnwidth]{delta=020_small.pdf}
    \label{subfig:delta=0.2}}
    \subfloat[$\delta=0.5$, $e=0.79$]{
    \includegraphics[width=0.32\columnwidth]{delta=050_small.pdf}
    \label{subfig:delta=0.5}}
    
    \caption{ Initial configurations obtained from the same triangular configuration, for different values of $\delta$, with the corresponding values of $e(0)$.
    \remark{incorporare in figura 6}
    }
    \label{fig:values_of_delta}
\end{figure}
\end{comment}


\subsection{Numerical results}

To validate Theorem \ref{th:local_stability_lyap} and estimate the basin of attraction of the set of triangular configurations, we performed extensive simulations for various values of $\delta$, and observed the steady state configurations. The results are reported in Fig.~\ref{fig:varyingdelta}.
Namely, we see that for $\delta \leq \delta^{\R{thres}} \coloneqq 0.25$ all simulations converge to a triangular configuration, with a rigid framework and a negligible value of $e$.
Then, as $\delta$ increases beyond $\delta^{\R{thres}}$, the average number of simulations converging to triangular configurations decreases, until for $\delta> 0.45$ no simulation converges to a triangular configuration.
Notice that $e(0)\leq 2\delta$, therefore $\delta = 0.25$ corresponds to a perturbation of up to 50\% of the initial length of the links, providing an estimation of the basin of attraction (region of asymptotic stability) of $\C{T}$.
%This value is greater than $R_\R{a}-R \approx 0.36$, which is the value of $e$ such that the links in the initial configuration are ensured to connect the same nodes as those in the unperturbed triangular configuration $\bar{\vec{x}}^*$.
%Interestingly, $\delta^{\R{thres}}$ is slightly larger than $(R_\R{a}-R)/2 \approx 0.18$, which is the value of $\delta$ such the links in the initial configuration are ensured to connect the same nodes as those in the unperturbed triangular configuration $\bar{\vec{x}}^*$. 
%This suggests that the set $\{ \bar{\vec{x}} \in \mathbb{R}^{2n} : \C{E}(\bar{\vec{x}}) = \C{E}(\bar{\vec{x}}^*) \}$ might be a (conservative) estimate of the basin of attraction of $\Gamma(\bar{\vec{x}}^*)$.
%
%Notice that $e(0)\leq 2\delta$, therefore, choosing $\delta < (R_\R{a}-R)/2$ implies that only the links of the triangular configuration are present, that is $\C{E}(\bar{\vec{x}}) = \C{E}(\bar{\vec{x}}^*)$.
%
%These results show that basin of attraction of the equilibrium set $\C{T}$ extends far beyond the restrictive limits applied during the proof of Th. \ref{th:local_stability_lyap}.

\begin{figure}[t]
    \centering
    \subfloat[Terminal values of the metrics]{\includegraphics[width=1\columnwidth]{e_rho.pdf}}\\[-0.5ex]
    \subfloat[Initial configurations]{
    \captionsetup[subfigure]{labelformat=empty}
    \subfloat[$\delta=0.2$]{\includegraphics[trim={2mm 2mm 2mm 12mm},clip,width=0.32\columnwidth]{d=02_x_0.pdf}}
    \subfloat[$\delta=0.4$]{\includegraphics[trim={2mm 2mm 2mm 12mm},clip,width=0.32\columnwidth]{d=04_x_0.pdf}}
    \subfloat[$\delta=0.6$]{\includegraphics[trim={2mm 2mm 2mm 12mm},clip,width=0.32\columnwidth]{d=06_x_0.pdf}}
    \setcounter{subfigure}{2}}% Reset subfigure counter
    \\[-0.5ex]
    \subfloat[Final configurations]{
    \captionsetup[subfigure]{labelformat=empty}
    \subfloat[$\delta=0.2$]{\includegraphics[trim={2mm 2mm 2mm 12mm},clip,width=0.32\columnwidth]{d=02_x_20.pdf}}
    \subfloat[$\delta=0.4$]{\includegraphics[trim={2mm 2mm 2mm 12mm},clip,width=0.32\columnwidth]{d=04_x_20.pdf}}
    \subfloat[$\delta=0.6$]{\includegraphics[trim={2mm 2mm 2mm 12mm},clip,width=0.32\columnwidth]{d=06_x_20.pdf}}
    \setcounter{subfigure}{3}}% Reset subfigure counter
    %
    %\subfloat[]{\includegraphics[width=0.9\columnwidth]{varyingdelta_rigidity.eps}}
    \caption{Simulations for different values of $\delta$.
    (a): Terminal values of $e$ and $\rho$. 
    $\rho$ is the fraction of trials converging to an infinitesimally rigid configuration.
    For $e$, the solid line is the mean; the shaded area is the minimum and maximum.
    20 simulations with random initial conditions are performed for each value of $\delta$.
    (b), (c): Initial and final configurations of representative simulations for specific values of $\delta$.}
    \label{fig:varyingdelta}
\end{figure}

Moreover, we analysed the time evolution of $e(t)$ in the case $\delta = 0.2$.
The results of $10$ simulations are shown in Fig.~\ref{fig:simulation}.
We find that the rigidity is preserved during all simulations, and at steady state $e$ reaches zero, meaning that the swarm, when locally perturbed, quickly converges back to a triangular configuration, as expected from Theorem~\ref{th:local_stability_lyap}.
% Then agents are let evolve under law \eqref{eq:control_law} for a sufficiently long time to reach steady state.
% The links and the infinitesimal rigidity of the framework are preserved along all the simulations, while the value of $e$ decreases and settles to a negligible value (around $2\cdot 10^{-3}$) (see Fig.~\ref{subfig:metric}), 
%\footnote{Such residual value is due to the approximation implicit in assumption \ref{hp:vanishing} of Theorem \ref{th:local_stability_lyap}.}
% showing the effectiveness of control law \eqref{eq:control_law} and validating the theoretical result presented in Section \ref{sec:main_results}.

\begin{figure}[t]
    \centering
    % \subfloat[Initial configuration.]{
    % \includegraphics[trim={2mm 2mm 2mm 2mm},clip,width=0.35\columnwidth]{d=02 x_0.pdf}
    % \label{subfig:initial}}
    % \subfloat[Final configuration.]{
    % \includegraphics[trim={2mm 2mm 2mm 2mm},clip,width=0.35\columnwidth]{d=02 x_20.pdf}
    % \label{subfig:final}}
    
    %\subfloat[Error.]{
    \includegraphics[width=0.9\columnwidth]{d=02_e_max.pdf}
    %\label{subfig:metric}
    %}
    %\subfloat[]{    %\includegraphics[width=0.49\textwidth]{V_v2.eps}
    %\label{subfig:V_v2}}    
    \caption{%
    %10 simulations with random initial conditions and $\delta = 0.2$.
    %(a) and (b): Initial and final configurations of a representative simulation.
    %\remark{These two figures are the same of Fig \ref{fig:varyingdelta} for $\delta=0.2$}
    %(c) 
    Time evolution of the error $e$ in 10 simulations with random initial conditions and $\delta = 0.2$; the solid line is the mean, while the shaded area is the maximum and minimum.
    %Panel \ref{subfig:V_v2} shows the evolution of V \eqref{eq:V+cm}. 
    %Parameters: $\delta=(R_a-1)/2$. 
    }
    \label{fig:simulation}
\end{figure}







%%%%%%%%%%%%%%%%%%%%%%%%%%%%%%%%%%%%%%%%%%%%%%%%%%%%%%%%%%%%%%%%%%%
\section{Conclusions}

We proved analytically local asymptotic stability of triangular lattice configurations for planar swarms under the action of a distributed control action based on virtual attraction/repulsion forces.
The theoretical derivations were supported by exhaustive numerical simulations validating the theoretical results and providing an estimate of the basin of attraction.
The mild hypotheses required on the interaction function that were used to prove convergence allow for wide applicability of the theoretical results.
%that addresses an important gap in the Literature on geometric pattern formation, providing a theoretical support for the numerous existing solutions.

Future work will focus on the formalization of the three-dimensional case and the extension of the results to other geometric lattices, such as squares and hexagons.
%This work sets the stage for a natural extension to the  multidimensional case, with Th. \ref{th:rigidity} being the only aspect restricted to the planar case.
%Moreover, in the future, we plan to extend this result to other lattices, such as the square one.

%\addtolength{\textheight}{-12cm}   % This command serves to balance the column lengths
                                  % on the last page of the document manually. It shortens
                                  % the textheight of the last page by a suitable amount.
                                  % This command does not take effect until the next page
                                  % so it should come on the page before the last. Make
                                  % sure that you do not shorten the textheight too much.

%%%%%%%%%%%%%%%%%%%%%%%%%%%%%%%%%%%%%%%%%%%%%%%%%%%%%%%%%%%%%%%%%%%%%%%%%%%%%%%%


%%%%%%%%%%%%%%%%%%%%%%%%%%%%%%%%%%%%%%%%%%%%%%%%%%%%%%%%%%%%%%%%%%%%%%%%%%%%%%%%
%\section*{APPENDIX}

%Appendixes should appear before the acknowledgment.


\appendix
\label{sec:appendix}

To confirm the effectiveness of our theoretical results, we provide below further semi-analytical evidence that the set of triangular configurations $\C{T}$, is locally asymptotically stable, which also excludes the presence of other equilibria in an arbitrarily small neighborhood of it.
To do so, we linearize
%($f(z)$ is differentiable in $z=R$) 
system \eqref{eq:model}--\eqref{eq:control_law} around a triangular configuration, say $\bar{\vec{x}}^*$, obtaining
$%\begin{equation}
    \label{eq:linearization}
    \dot{\bar{\vec{x}}} \approx \vec{J}(\bar{\vec{x}}^*) \, (\bar{\vec{x}}-\bar{\vec{x}}^*)
$, %\end{equation}
with $\vec{J}(\bar{\vec{x}}^*) \in \BB{R}^{2n \times 2n}$ derived as follows.

\paragraph*{Jacobian of \eqref{eq:model}--\eqref{eq:control_law}}
System \eqref{eq:model}--\eqref{eq:control_law} can be recast as
\begin{equation}\label{eq:model_stack}
    \dot{\bar{\vec{x}}} = ((\B{B} \vec{F} \B{G}^{-1} \B{B}\T) \otimes \B{I}_2) \bar{\vec{x}}= ((\B{B} \vec{H} \B{B}\T) \otimes \B{I}_2) \bar{\vec{x}},
\end{equation}
where $\B{F}, \B{G}, \B{H} \in \BB{R}^{m \times m}$ are diagonal matrices; $[\B{F}]_{ii} \coloneqq f(\norm{\vec{r}_i})$, $[\B{G}]_{ii} \coloneqq \norm{\vec{r}_i}$, and $\vec{H} \coloneqq \B{F} \B{G}^{-1}$. 
%
The Jacobian of \eqref{eq:model_stack} is 
%$\vec{J}(\bar{\vec{x}})=\frac{\partial \dot{\bar{\vec{x}}}}{\partial \bar{\vec{x}}} \in \BB{R}^{2n \times 2n}$, and is given by
\begin{equation}
\label{eq:Jacobian}
\begin{aligned}    
    \vec{J}&=\left(
    \vec{B} \frac{\partial \vec{H}}{\partial \bar{\vec{x}}}  \vec{B}\T \otimes \vec{I}_2 \right) \vec{\bar{x}} +
    (\B{B} \vec{H} \B{B}\T) \otimes \B{I}_2 \eqqcolon \vec{J}_1 +\vec{J}_2,
\end{aligned}
\end{equation}
where  $\frac{\partial \vec{H}}{\partial \bar{\vec{x}}} \in \BB{R}^{m\times m \times 2n}$ is a tensor, and 
%the tensor-matrix product%
%\footnote{\color{red}[MC: see eg https://math.stackexchange.com/questions/1953185/tensors-and-matrices-multiplication, https://math.stackexchange.com/questions/2005612/tensor-mode-product], ho cercato ma non ho trovato nulla}
%is defined so that \remark{add ref}
\begin{equation*}
    \left[\frac{\partial \vec{H}}{\partial \bar{\vec{x}}}  \B{B}\T \right]_{:,:,k} =  \left[ \frac{\partial \vec{H}}{\partial \bar{\vec{x}}} \right]_{:,:,k} \B{B}\T \quad \in \BB{R}^{m\times n},
\end{equation*}
with notation $[\ \cdot\ ]_{:,:,k}$ denoting the matrix obtained by fixing the third index of the tensor.
From \ref{hp:null_point}, for all triangular configurations we have $\vec{J}_2=(\B{B} \vec{H} \B{B}\T) \otimes \B{I}_2 =\vec{0}$.
Then,
$%\begin{equation*}
    \left[ \vec{J}_1 \right]_{:,k} 
    = \left(
    \vec{B} \left[ \frac{\partial \vec{H}}{\partial \bar{\vec{x}}} \right]_{:,:,k}  \vec{B}\T \otimes \vec{I}_2 \right) \vec{\bar{x}}
$. %\end{equation*}
From {\cite[p. 20]{Jackson2007} %\cite[p. 176]{Asimow1979}
}%
we have 
$\frac{\partial \norm{\vec{r}_i}^2}{\partial [\bar{\vec{x}}]_k}  = 2 [\vec{M}]_{i,k}$ (see Definition \ref{def:rigidity_matrix}),
that is
$\frac{\partial \norm{\vec{r}_i}}{\partial [\bar{\vec{x}}]_k}  = \frac{1}{\norm{\vec{r}_i}} [\vec{M}]_{i,k}$, and thus
\begin{subequations}\label{eq:matrix_H}
\begin{align}
    \left[ \frac{\partial \vec{H}}{\partial \bar{\vec{x}}} \right]_{i,i,k} 
    &= \frac{\partial [f(\norm{\vec{r}_i})/\norm{\vec{r}_i}]}{\partial \norm{\vec{r}_i}} \frac{\partial \norm{\vec{r}_i}}{\partial [\bar{\vec{x}}]_k}\nonumber\\
    &= [f'(\norm{\vec{r}_i}) \norm{\vec{r}_i}  - f(\norm{\vec{r}_i})] \norm{\vec{r}_i}^{-3} [\vec{M}]_{i,k},\\
    \left[ \frac{\partial \vec{H}}{\partial \bar{\vec{x}}} \right]_{i,j,k} &= 0, \quad \text{if} \ i \ne j.
\end{align}
\end{subequations}

\paragraph*{Numerical analysis}
%To verify that the equilibrium set corresponding to triangular configurations is locally asymptotically stable, 
We set $R = 1$ and generated $760$ random triangular configurations ($10$ per each number of agents 
$n$ between $25$ and $100$).
%Then assuming the interaction function is of the form 
For each of these configurations, assuming $f$ (in \eqref{eq:control_law}) is in the form \eqref{eq::Lennard-Jones}, we computed $\B{J}$ using \eqref{eq:Jacobian}--\eqref{eq:matrix_H} and found that in all cases $\vec{J}$ has $3$ zero eigenvalues with eigenvectors $\{\vec{w}_i^{0}\}_{i}$, and $2n-3$ negative eigenvalues with eigenvectors $\{\vec{w}_j^{\pm}\}_j$.
Moreover, $\B{M} \B{w}_i^{0} = \B{0}$ and $\B{M} \B{w}_j^{\pm} \ne \B{0}$; thus, from Definition~\ref{def:inf_rigidity}, the span of $\{\B{w}_i^{0}\}$ corresponds to roto-translations and is a hyperplane locally tangent to $\Gamma(\bar{\vec{x}}^*)$ (see Definition \ref{def:congruent_conf}), while $\{\B{w}_j^{\pm}\}$ correspond to other motions.
Therefore, the \emph{center manifold theorem} \cite[Theorem 5.1]{Kuznetsov2004} yields that $\Gamma(\bar{\vec{x}}^*)$ is a \emph{center manifold} of system \eqref{eq:model}--\eqref{eq:control_law}.
Moreover, as expected from Theorem \ref{th:local_stability_lyap}, the \emph{reduction principle} \cite[Theorem 5.2]{Kuznetsov2004} confirms that the dynamics locally converge onto the %invariant subspace spanned by $\{\vec{w}_i^{0}\}_{i}$, that is locally tangent to $\C{T}$.
equilibrium set $\Gamma(\bar{\vec{x}}^*)$, and excludes the presence of other equilibria in an  arbitrarily small neighborhood of it.

%%%%%%%%%%%%%%%%%%%%%%%%%%%%%%%%%%%%%%%%%%%%%%%%%%%%%%%%%%%%%%%%%%%%%%%%%%%%%%%%
%\section*{ACKNOWLEDGMENT}

%Acknowledgments ...



%%%%%%%%%%%%%%%%%%%%%%%%%%%%%%%%%%%%%%%%%%%%%%%%%%%%%%%%%%%%%%%%%%%%%%%%%%%%%%%%


\bibliographystyle{IEEEtran}
\bibliography{LSMAS, Synt_Bio}


%\begin{thebibliography}{99 }
%\end{thebibliography}

\end{document}
