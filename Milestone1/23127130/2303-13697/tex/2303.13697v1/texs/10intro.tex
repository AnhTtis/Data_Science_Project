\section{Background and Motivation}
\label{sec:intro}

Piecewise-affine (PWA) systems~\cite{sontag1981nonlinear} are widely used to
model highly nonlinear behaviors such as contact dynamics in
robotics. Given an initial condition and a goal, a trajectory that drives the state to the goal while respecting the PWA dynamics and state/control constraints can be obtained by solving a mixed-integer convex programming (MICP) problem.  This approach has seen success in important robotic applications such as push recovery~\cite{han2017feedback} and footstep planning~\cite{deits2014footstep} (as illustrated in Fig.~\ref{fig:motivation}). The MICP approaches are sound and complete for a discrete-time PWA model and a fixed horizon - they find feasible solutions if they exist. However, completeness comes at a high computational cost due to the inherent complexity of solving MICP problems.

Previous work on controlling PWA systems has focused on obtaining efficient and strong formulations of the MICP problems in different application scenarios~\cite{andrikopoulos2013piecewise,marcucci2019mixed,han2017feedback,marcucci2021shortest,deits2014footstep,aceituno2017simultaneous,landry2016aggressive}. The actual solving is typically off-loaded to \emph{general-purpose} off-the-shelf MICP solvers~\cite{mosek,cplex,gurobi,scip}. While the performance of off-the-shelf solvers has improved dramatically in the past decade, these solvers were originally developed with non-robotic applications (e.g., operations research) in mind. Little has been done to tailor solvers specifically to robotics applications. A natural question is then: \textbf{can we exploit the structure of problems arising from PWA systems to design specialized MICP solvers that are faster than general-purpose ones?}

\begin{figure}
    \centering
    \includegraphics[height=2.8cm]{imgs/motivation.png}
    \quad
    \includegraphics[height=2.8cm]{imgs/motivation-1.png}
    \caption{Footstep planning (from~\cite{deits2014footstep}): the robot must find a path using only the available ``stepping stones.'' We evaluate a similar problem later (see Fig.~\ref{fig:stepping_stone}).}
    \label{fig:motivation}
    \vspace{-0.5cm}
\end{figure}

In MICP encodings of problems from the planning domain, logical constraints are typically encoded using arithmetic. When it comes to PWA dynamics, a ubiquitous logical constraint type is the \emph{one-hot constraint}, which encodes the fact that at any time ``the system is in exactly one mode.'' For example, each footstep of the robot in Fig.~\ref{fig:motivation} must be on exactly one of the ``stepping stones.''

Another way to encode one-hot constraints, however, is to use propositional logic directly.
\emph{Our key insight is that reasoning about modes explicitly at the logical level can be beneficial.} To get some intuition for why this is the case, suppose we know that a certain mode combination is infeasible.  To rule out this mode combination, we could either encode it using integer arithmetic constraints or as a single disjunction at the propositional logic level. Empirically, it can be observed that the addition of a few hundred arithmetic (linear) constraints can significantly increase the runtime of a solver.  At the same time, propositional solvers can process thousands of disjunctions in milliseconds.

A tight integration of propositional and theory (e.g., arithmetic) reasoning is at the heart of the highly successful satisfiability modulo theories (SMT) paradigm~\cite{barrett2018satisfiability}, with most implementations based on the popular DPLL(T) framework~\cite{ganzinger2004dpll}. In this paper, we adapt the DPLL(T) framework for the setting of PWA planning. In particular, we develop a sound and complete DPLL(T) procedure which integrates propositional reasoning with reasoning about \emph{mixed-integer linear programming} (MILP) problems (a subset of MICP problems) with one-hot constraints.

The underlying convex solver in our approach can only operate on convex relaxations of the one-hot constraints, i.e., integer variables are relaxed to real variables.
To further tailor our solver to our problem domain, 
we propose to ``softly'' guide the convex solver with information about the precise one-hot constraints before branching on them. Inspired by the sum-of-infeasibilities method in convex optimization~\cite{boyd2004convex}, we define a cost function which represents the degree to which the current solution violates the one-hot constraints. If an assignment is found with cost zero, then not only is the assignment a solution for the convex relaxation, but it also solves the precise problem.

%Including the cost function results in a system of constraints that is only 
The aforementioned cost function is concave piecewise-linear, which is challenging to minimize directly. We observe, however, that for any specific mode sequence, the system collapses into a set of \emph{linear} constraints, which can be optimized by an LP solver. Minimizing the linear cost function provides a way to evaluate ``how feasible'' the corresponding mode sequence is.  Leveraging this insight, we propose to use Markov chain Monte Carlo (MCMC) sampling to efficiently navigate towards mode sequences at the global minimum of the cost function. In addition, we propose a novel propagation-based proposal strategy for MCMC sampling, which guarantees that the sampled mode sequence is 1) non-repetitive; and 2) does not match any known infeasible mode combinations. 

Our end result is a specialized solver, \sys,\footnote{\sys is available at \href{https://github.com/stanford-centaur/Soy}{https://github.com/stanford-centaur/Soy}} that combines the strength of SMT, MILP, and stochastic local search to efficiently reason about PWA systems. \sys takes in MILP problems defined in the standard MPS format, which is supported by most MICP solvers. This makes it easy for users of MICP solvers to try their problem on \sys.
While \sys is still an early prototype, it can already be used to solve PWA control problems appearing in the literature significantly faster than was previously possible (using existing MILP solvers alone). The closest related work is \cite{shoukry2017smc}, which also shows that combining logical and arithmetic reasoning can be beneficial. Our work goes further by proposing domain-specific solutions for PWA dynamics, combining complete search with local search, and implementing a tool that is friendly to practitioners accustomed to using MICP solvers.

To summarize, our contributions include:
\begin{enumerate}
    \item an instantiation of the DPLL(T) framework for MILP problems with one-hot constraints;
    \item \deepsoi, a novel local search procedure based on the sum-of-infeasibilities method and MCMC sampling;
    \item a propagation-based strategy for MCMC sampling;  
    \item \sys, a specialized MILP solver for PWA control that combines the proposed techniques;
    \item an evaluation of \sys on PWA-control benchmarks.
\end{enumerate}

%The rest of the paper is organized as follows. We start with definitions and notation in Sec.~\ref{sec:prelim}. Sec.~\ref{sec:dpllt} describes our instantiation of the DPLL(T) procedure for MILP problems with one-hot constraints. Sec.~\ref{sec:soi} defines the sum-of-infeasibilities for one-hot constraints, discusses its minimization with MCMC sampling, and proposes a propagation-based proposal strategy. Sec.~\ref{sec:overview} describes the architecture of the implementaion in \sys. We present our experimental evaluation of \sys in Sec.~\ref{sec:experiments}, and conclude in Sec.~\ref{sec:next-steps}.

