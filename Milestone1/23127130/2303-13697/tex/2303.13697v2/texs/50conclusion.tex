\section{Conclusion and Future Directions}
\label{sec:next-steps}

We have introduced \sys, a specialized MILP solver for PWA control problems. We instantiated the DPLL(T) procedure for deciding the feasibility of combinations of linear and one-hot constraints. We also presented \deepsoi, a specialized optimization procedure that stochastically minimizes the sum-of-infeasibilities to search for feasible mode sequences. \sys is already competitive against highly-optimized off-the-shelf MICP solvers, suggesting that this direction of designing specialized MICP solvers for PWA control is promising. We hope the work will garner interest in \sys and welcome contributions of new benchmarks for \sys. 

\smallskip\noindent \textbf{Limitations and Future Work.} \sys is still a research prototype, and as such has many limitations compared to mature solvers: \begin{enumerate*} [1) ]
    \item We do not yet support more general convex constraints, such as quadratic constraints, which many MICP applications (e.g., \cite{deits2014footstep}) require; 
    \item We do not yet support other logical constraints common in robotic applications (e.g., ``at-least-one'', ``at-most-one'');
    \item We only perform feasibility checks instead of finding optimal solutions.
\end{enumerate*}

We hope to address all these limitations in the future. We note that the proposed techniques are compatible with more general convex constraints and can be in principle extended to other logical constraints, though actually doing these efficiently requires non-trivial research and engineering effort. On the other hand, a strong feasibility checker is the first step towards a strong optimizer. In the SMT community, there is an active effort to extend SMT solvers to do optimization (a line of work called ``optimization modulo theories'')~\cite{sebastiani2020optimathsat} and insights there could potentially be borrowed.

Supporting parallelism, incrementality, and APIs to encode constraints are also important next steps for our tool.

\smallskip\noindent \textbf{Acknowledgment.} This work was partially supported by grants from the National Science Foundation (2211505 and 2218760) and by the Stanford Center for AI Safety.
We thank Tobia Marcucci,
Gustavo Araya,
Richard McDaniel,
J\"org Hoffmann,
Aina Niemetz,
Mathias Preiner,
and Makai Mann for the helpful discussions.