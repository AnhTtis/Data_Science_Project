\section{Experiments}
\subsection{Experiment Configurations}
\paragraph{Implementation details.} The segmentation model in ScribbleSeg could be an arbitrary semantic segmentation network. In this work, we choose the SegFormer~\cite{xie2021segformer} for experiments following previous SOTAs~\cite{focalclick} of click-based segmentation. The input size of ScribbleSeg is kept as $384\times384$ during training and inference. 

During inference, we apply the zoom-in strategy proposed in \cite{fbrs}. Concretely, starting from the second stroke of scribble, we calculate the external box according to the previous mask and the current scribbles and expand it with a ratio of 1.4. Then, we crop the model input according to this expanded box and resize it to $384\times384$. This allows the model to focus on the target region, which is especially effective when the target is small in the image.

\vspace{-3mm}
\paragraph{Training configurations.} Following previous works of click-based segmentation~\cite{focalclick,sofiiuk2021ritm}, we train our model on the combined dataset of COCO~\cite{lin2014coco} and LVIS~\cite{gupta2019lvis}. For data augmentation, we use random flip and random resize with a scale ratio from 0.75 to 1.4. We take 3,000 images for each epoch and train ScribbleSeg with 150 epochs. The initial learning rate is set as 0.0005, and we add two lr decay with the ratio of 0.1 at the epoch of 110 and 130. For the optimizer, we pick ADAM with $\beta_{1}=0.9$ and $\beta_{2}=0.999$.


 



\begin{table*}[t]
\begin{center}
\scalebox{0.85}{
\begin{tabular}{l|l|c|ccc|ccc|ccc}
\toprule[1pt]
& \multicolumn{1}{l|}{} & & \multicolumn{3}{c|}{ADE-Stuff} & \multicolumn{3}{c}{ADE-Thing} & \multicolumn{3}{c}{ADE-Full}  \\

& Method & Train Interaction & NoI~85 & NoI~90 & NoF~90 & NoI~85 & NoI~90 & NoF~90 & NoI~85 & NoI~90 & NoF~90 \\
\hline
1 & AppearanceSim~\cite{appearancesimilarity} & None & 9.61  & 15.34 & 92 & 10.01  & 12.12 & 186 & 9.87 & 13.18 & 278  \\
2 & RITM-scribble~\cite{sofiiuk2021ritm} & Click points & 7.20  & 10.33 & 76 & 7.91  & 10.82 & 166 & 7.73 & 10.65 & 242  \\
3 & RITM-scribble~\cite{sofiiuk2021ritm}  & Linked points~\cite{IFIS}   & 6.11 & 7.92  & 58 & 6.99 &  8.74 & 131 & 6.69 & 8.47 & 189 \\

\hline
4 & ScribbleBase-B0 & Composed scribbles   & 4.89 & 7.14  & 41 & 5.61 &  8.19 & 111 & 5.37  & 7.84 & 152   \\
5 & ScribbleSeg-B0  & Composed scribbles   & 4.77 & 6.90  & 41 & 5.08 & 7.63  & \bf105 &  4.97  & 7.38 & \bf146   \\
\rowcolor{gray!20} 
6 & ScribbleSeg-B3  & Composed scribbles  & \bf4.41 & \bf6.57  & \bf39 & \bf5.06 & \bf7.46  & 109 & \bf4.84  & \bf7.16 & 148   \\
\bottomrule
\end{tabular}
}
\end{center}
\vspace{-2mm}
\caption{Comparison results on our proposed ADE20K benchmark. We report the performance for stuff and thing categories respectively. All models are trained on COCO+LVIS. `NoI~85/90' denotes the average Number of Interactions required the get IoU of 85/90\%. `B0/B3' denotes using SegFormer-B0/B3~\cite{xie2021segformer} as the segmentation model.  } 
\label{tab:ade}
\vspace{0mm}
\end{table*}

\subsection{ Quantitative Analysis} 
We first give a quantitative analysis for ScribbleSeg on our newly constructed ADE20K benchmark.  Using the proposed evaluation protocol, we evaluate the model performance with NoI@IoU, which means the average Number of Interactions required to reach the target IoU. 

As explained in Sec.~\ref{overall}, previous works are not suitable for comparison. Hence, we reproduce some representative baselines for comparison in Tab.~\ref{tab:ade}. In row~1, we train 
a similarity-based model like~\cite{appearancesimilarity}, which shows poor performance as it could not deal with the fine details. Row~2 corresponds to using the click-based solution~\cite{sofiiuk2021ritm} to deal with scribbles. We find that, although clicks could be regarded as short scribbles, directly using click-based models could not get satisfactory results. In row~3, we follow IFIS~\cite{IFIS} to simulate scribbles via linking randomly sampled points. This strategy brings improvements  compared to using click disks. However, it still shows a big gap compared with our work which is demonstrated in the second part.  

Row~4 denotes the baseline version without PAM and CRM, and row~5 and 6 show the full ScribbleSeg using SegFormer-B0/B3~\cite{xie2021segformer} as the segmentation model. It could be observed that ScribbleSeg achieves significantly better performance than its counterparts. 
On stuff categories, the advantage of ScribbleSeg is even larger. This is because scribbles can provide more indications compared to clicks in stuff categories that cover a relatively big region of arbitrary shapes and complicated semantics. 



\begin{table}[t]
\begin{center}
\scalebox{0.7}{
\begin{tabular}{l|cc|cc|cc}
\toprule[1pt]
\multicolumn{1}{l|}{} & \multicolumn{2}{c|}{Berkeley} & \multicolumn{2}{c|}{DAVIS} & \multicolumn{2}{c}{ADE-full}  \\
Pred : Pertub   & NoI~90 & NoF~90 & NoI~90 & NoF~90 & NoI~90 & NoF~90 \\
\hline
0 : 0  & 2.30 & 3 & 6.02 & 64 & 8.95  & 179 \\
1 : 0  & 2.06 & 1 & 5.46 & 61 & 8.61  & 176 \\
1 : 0.2  & 1.79 & 0 & 5.21 & 52 & 8.01  & 156 \\
\rowcolor{gray!20} 
1 : 0.4  & 1.77 & 0 & 5.18 & 51 & 7.98  & 153 \\
1 : 0.6  & 1.81 & 0 & 5.33 & 54 & 8.12 & 160 \\

\bottomrule
\end{tabular}
}
\end{center}
\vspace{-2mm}
\caption{ Comparison results of using different ratios of predicted masks~(generated by iterative training) and perturbed ground truth mask to simulate the previous mask.  } 
\label{tab:prev_mask}
\vspace{-2mm}
\end{table}

\begin{table}[t]
\begin{center}
\scalebox{0.7}{
\begin{tabular}{l|cc|cc|cc}
\toprule[1pt]
\multicolumn{1}{l|}{} & \multicolumn{2}{c|}{Berkeley} & \multicolumn{2}{c|}{DAVIS} & \multicolumn{2}{c}{ADE-full}  \\
%\cline{1-9}
Max Number  & NoI~90 & NoF~90 & NoI~90 & NoF~90 & NoI~90 & NoF~90 \\
\hline
8 & 1.92 & 1 & 5.34 & 56 & 8.33  & 164 \\
12 & 1.81 & 0 & 5.17 & 51 & 8.11  & 158 \\
\rowcolor{gray!20} 
16 & 1.77 & 0 & 5.18 & 51 & 7.98  & 153 \\
20  & 1.76 & 0 & 5.23 & 52 & 8.01  & 155 \\
\bottomrule
\end{tabular}
}
\end{center}
\vspace{-2mm}
\caption{Comparisons for the maximum number of strokes for the simulated scribbles during training.} 
\label{tab:num_scr}
\vspace{-2mm}
\end{table}

\begin{table}[t]
\begin{center}
\scalebox{0.73}{
\begin{tabular}{l|cc|cc|cc}
\toprule[1pt]
\multicolumn{1}{l|}{} & \multicolumn{2}{c|}{Berkeley} & \multicolumn{2}{c|}{DAVIS} & \multicolumn{2}{c}{ADE-full}  \\
%\cline{1-9}
Proportions & NoI~90 & NoF~90 & NoI~90 & NoF~90 & NoI~90 & NoF~90 \\
\hline
0    & 1.74 & 0 & 5.26 & 52 & 8.12  & 155 \\
\rowcolor{gray!20} 
0.2  & 1.77 & 0 & 5.18 & 51 & 7.98  & 153 \\
0.4  & 1.79 & 0 & 5.29 & 53 & 8.09  & 153 \\
\bottomrule
\end{tabular}
}
\end{center}
\vspace{-2mm}      
\caption{We set the Bezier curve as the principle strategy for scribble simulation, and analyze the proportions for the axial and boundary scribble during training.} 
\label{tab:scribble}
\vspace{-2mm}
\end{table}

\begin{table}[t]
\begin{center}
\scalebox{0.7}{
\begin{tabular}{l|cc|cc|cc}
\toprule[1pt]
\multicolumn{1}{l|}{} & \multicolumn{2}{c|}{Berkeley} & \multicolumn{2}{c|}{DAVIS} & \multicolumn{2}{c}{ADE-full}  \\
%\cline{1-9}
Scribble Type  & NoI~90 & NoF~90 & NoI~90 & NoF~90 & NoI~90 & NoF~90 \\
\hline
Allow Error  & 2.01 & 1 & 5.45 & 57 & 8.98  & 183 \\
Clean Boundary  & 1.77 & 0 & 5.18 & 51 & 7.98  & 153 \\
\rowcolor{gray!20} 
Protect Boundary  & 1.66 & 0 & 5.14 & 53 & 7.84 & 152 \\
\bottomrule
\end{tabular}
}
\end{center}
\vspace{-2mm}
\caption{Different strategies for dealing with the simulated scribbles that are near the boundaries of the ground truth mask.} 
\label{tab:boudary}
\vspace{-2mm}
\end{table}


\subsection{Ablations Studies}
After verifying our promising performance, in this section, we dive into the details of our framework, including the basic settings of scribble-simulation, and the two novel components: Prototype Adaption Module~(PAM) and Corrective Refine Module~(CRM). 

We first make an analysis of the basic settings to explore what makes a strong baseline for scribble-based interactive segmentation. We use a vanilla model without PAM and CRM and mainly focus on exploring the strategies of simulating the previous masks and the scribbles during training.  

\vspace{-4mm}
\paragraph{Flawed masks simulation.} We first analyze the simulation of previous masks. In Tab.~\ref{tab:prev_mask}, we explore the combined ratio for two kinds of previous mask generation methods introduced in Sec.~\ref{sec:train}. `Pred' denotes the predicted masks of the current model, which is generated by iterative training~\cite{mahadevan2018iteratively}. `Pertub' means the perturbed mask of the ground truth. The results show that the previous masks are important for interactive segmentation, and we choose the combination ratio that achieves the best performance.

\vspace{-4mm}
\paragraph{Scribble simulation.} Afterwards, we make explorations for the scribble simulation strategies. In Tab.~\ref{tab:num_scr}, we report the performance of using different numbers of scribble strokes. During training, we set the maximum number of strokes, and randomly pick a stroke number with a probability decay of 0.8, which means that the probability of choosing $n$ strokes is 0.8 of $n-1$. 

In Tab.~\ref{tab:scribble}, we tune the combined ratio for the three types of scribbles introduced in Sec.~\ref{sec:train}. We use the Bezier curve as the principle strategy, and tune the ratios of axial and boundary scribbles. As the Bezier curve is the closest to human-drawn scribbles, a higher portion results in better performance. At the same time, a small portion of the axial and boundary scribbles could increase the training diversity, which is also beneficial.

Tab.~\ref{tab:boudary} shows that dealing with scribbles in boundary regions is also important. `Allow Error' means allowing the simulated scribbles to slightly exceed the ground truth masks. `Clean Boundary' denotes removing the exceeded parts of simulated scribbles. `Protect Boundary' describes eroding the ground truth mask as the target region to simulate scribbles, and is proven to be effective.



\begin{table}[t]
\begin{center}
\scalebox{0.7}{
\begin{tabular}{l|cc|cc|cc}
\toprule[1pt]
\multicolumn{1}{l|}{} & \multicolumn{2}{c|}{Berkeley} & \multicolumn{2}{c|}{DAVIS} & \multicolumn{2}{c}{ADE-full}  \\
%\cline{1-9}
Method  & NoI~90 & NoF~90 & NoI~90 & NoF~90 & NoI~90 & NoF~90 \\
\hline
StrongBase   & 1.66 & 0 & 5.14 & 53 & 7.84 & 152 \\
+PAM  & 1.68 & 0 & 4.98 & 52 & 7.52  & 152 \\
+CRM  & 1.53 & 0 & 4.76 & 51 & 7.46  & 148 \\
\rowcolor{gray!20} 
+PAM+CRM & 1.49 & 0 & 4.68 & 50 & 7.38 & 146 \\
\bottomrule
\end{tabular}
}
\end{center}
\vspace{-3mm}
\caption{Ablation studies for our novel components. } 
\label{tab:ablation}
\vspace{-3mm}
\end{table}


\begin{table}[t]
\begin{center}
\scalebox{0.7}{
\begin{tabular}{l|cc|cc|cc}
\toprule[1pt]
\multicolumn{1}{l|}{} & \multicolumn{2}{c|}{Berkeley} & \multicolumn{2}{c|}{DAVIS} & \multicolumn{2}{c}{ADE-full}  \\
%\cline{1-9}
Method  & NoI~90 & NoF~90 & NoI~90 & NoF~90 & NoI~90 & NoF~90 \\
\hline
StrongBase   & 1.66 & 0 & 5.14 & 53 & 7.84 & 152 \\
+Scribble Pool  & 1.66 & 0 & 5.01 & 51 & 7.60  & 151 \\
+Mask Pool  & 1.64 & 0 & 5.06 & 53 & 7.63  & 153 \\
\rowcolor{gray!20} 
+Full PAM  & 1.68 & 0 & 4.98 & 52 & 7.52  & 152 \\
\bottomrule
\end{tabular}
}
\end{center}
\vspace{-3mm}
\caption{Ablation studies for the details of PAM. } 
\label{tab:PAM}
\vspace{-3mm}
\end{table}



\begin{table}[t]
\begin{center}
\scalebox{0.65}{
\begin{tabular}{l|cc|cc|cc}
\toprule[1pt]
\multicolumn{1}{l|}{} & \multicolumn{2}{c|}{Berkeley} & \multicolumn{2}{c|}{DAVIS} & \multicolumn{2}{c}{ADE-full}  \\
%\cline{1-9}
Method  & NoI~90 & NoF~90 & NoI~90 & NoF~90 & NoI~90 & NoF~90 \\
\hline
StrongBase  & 1.66 & 0 & 5.14 & 53 & 7.84 & 152 \\
+Refiner~\cite{focalclick}  & 1.61 & 0 & 5.01 & 52 & 7.66 & 153 \\
+CRM w/o Detach & 1.55 & 0 & 4.95 & 52 & 7.64 & 151 \\
+CRM w/o Error Map  & 1.52 & 0 & 4.87 & 52 & 7.65 & 152 \\
+CRM w/o Scribbles  & 1.58 & 0 & 4.93 & 51 & 7.56 & 150 \\
\rowcolor{gray!20} 
+Full CRM  & 1.53 & 0 & 4.76 & 51 & 7.46 & 148 \\
\bottomrule
\end{tabular}
}
\end{center}
\vspace{-3mm}
\caption{Ablation studies for the details of CRM. } 
\label{tab:CRM}
\vspace{-2mm}
\end{table}


\begin{figure}[t]
\newcommand{\image}{\includegraphics[width=0.31\columnwidth]}
\centering 
\tabcolsep=0.02cm
\renewcommand{\arraystretch}{0.06}
\begin{tabular}{ccc}
\vspace{3pt}
\image{Figures/compare/11.png} &
\image{Figures/compare/12.png} &
\image{Figures/compare/13.png} \\
\vspace{3pt}
{\footnotesize (a)~Ground Truth } &{\footnotesize (a)~1 Scr~:~92.3\% } & {\footnotesize (a)~3 Clicks~:~83.5\% } \\
\vspace{3pt}
\image{Figures/compare/21.png} &
\image{Figures/compare/22.png} &
\image{Figures/compare/23.png} \\
\vspace{3pt}
{\footnotesize (b)~Ground Truth } &{\footnotesize (b)~1 Scr~:~86.9\% } & {\footnotesize (b)~3 Clicks~:~37.8\% } \\
\vspace{3pt}
\image{Figures/compare/31.png} &
\image{Figures/compare/32.png} &
\image{Figures/compare/33.png} \\
\vspace{3pt}
{\footnotesize (c)~Ground Truth } &{\footnotesize (c)~1 Scr~:~76.8\% } & {\footnotesize (c)~3 Clicks~:~67.9\% } \\
\end{tabular}
\vspace{0mm}
\caption{Comparisons of scribble-based method~(ScribbleSeg) and click-based method~(FocalClick~\cite{focalclick}). }
\vspace{-3mm}
\label{fig:compare}
\end{figure}


After tuning the settings explored above, we get a strong baseline for scribble-based interactive segmentation, which already displays great performance according to Tab.~\ref{tab:evaluation sota}. In the next section, we add PAM and CRM  to make further improvements and analyze the details of these two modules.
In Tab.~\ref{tab:ablation}, we show that PAM and CRM could enhance the performance of the strong baseline independently, and combining both of them could make further improvements. 

\vspace{-4mm}
\paragraph{Prototype Adaption.} PAM gathers information from the scribble-marked regions and the mask regions to update the projection kernel. This assists ScribbleSeg in making more consistent predictions. PAM is composed of mask-pooling and scribble-pooling-guided prototype adaption. The results in Tab.~\ref{tab:PAM} demonstrate that both of these two modules are effective.

\vspace{-4mm}
\paragraph{Corrective Refine.} CRM makes detailed refinement in the predicted error regions. In Tab.~\ref{tab:CRM}, we make analyses for the different implementations. The results show that the error map is important for CRM, as it enables CRM to focus on the fine details. Detaching the feature and masks from the previous stage also brings improvements. 


\begin{table}[t]
\begin{center}
\scalebox{0.7}{
\begin{tabular}{ll|c|c|c|c|c}
\toprule[1pt]
\multicolumn{2}{l|}{} & Berkeley~\cite{berkeley} & \multicolumn{2}{c|}{SBD~\cite{SBD}} & \multicolumn{2}{c}{DAVIS~\cite{davis}} \\

\multicolumn{2}{l|}{Method }  & NoI~90 & NoI~85 & NoI~90 & NoI~85 & NoI~90 \\
\hline
\multicolumn{2}{l|}{f-BRS-B-hr32~\cite{fbrs}}   & 2.44 & 4.37 & 7.26 & 5.17 & 6.50 \\
\multicolumn{2}{l|}{ RITM-hr18s~\cite{sofiiuk2021ritm}}   & 2.60 & 4.04 &  6.48 & 4.70 & 5.98 \\
\multicolumn{2}{l|}{ RITM-hr32~\cite{sofiiuk2021ritm}}     & 2.10 & 3.59 & 5.71 & 4.11 & 5.34 \\
\multicolumn{2}{l|}{ FocalClick-hr18s-S2~\cite{focalclick} }   & 2.66 & 4.43 &  6.79 & 3.90 & 5.25 \\

\multicolumn{2}{l|}{ FocalClick-B0-S2~\cite{focalclick}}   & 2.27 & 4.56 & 6.86 & 4.04 & 5.49 \\
\multicolumn{2}{l|}{ FocalClick-B3-S2~\cite{focalclick}}  &  {1.92}  & {3.53} & {5.59} & {3.61} & {4.90} \\

\hline
\multicolumn{2}{l|}{ SribbleBase-B0}   &  1.66  & 2.18 & 4.50 & 3.67 & 5.14  \\
\multicolumn{2}{l|}{ ScribbleSeg-B0}  &  1.49 & 2.56 & 4.21 & 3.29 & 4.68  \\
\rowcolor{gray!20} 
\multicolumn{2}{l|}{ ScribbleSeg-B3} & \bf1.35 & \bf2.42 & \bf3.99 & \bf3.10 & \bf4.45  \\
\bottomrule[1pt]
\end{tabular}
}
\end{center}
\vspace{-2mm}
\caption{Evaluation results on Berkeley, SBD and DAVIS datasets. 
 `NoI~85/90' denotes the average Number of Interactions~(clicks or scribbles) required the get IoU of 85/90\%.  }
\label{tab:evaluation sota}
\vspace{0mm}
\end{table}


\begin{figure*}[t]
\newcommand{\image}{\includegraphics[width=0.49\columnwidth]}
\centering 
\tabcolsep=0.04cm
\renewcommand{\arraystretch}{0.06}
\begin{tabular}{cccc}
\vspace{3pt}
\image{Figures/Demo/21.png} &
\image{Figures/Demo/22.png} &
\image{Figures/Demo/23.png} &
\image{Figures/Demo/24.png} \\
\vspace{3pt}
{\footnotesize (1)~Ground Truth } &{\footnotesize (1)~1 Scribble~:~85.2\% } & {\footnotesize (1)~2 Scribbles~:~94.6\% } & {\footnotesize (1)~3 Scribbles~:~96.8\% }\\
\vspace{3pt}
\image{Figures/Demo_compressed/11.png} &
\image{Figures/Demo_compressed/12.png} &
\image{Figures/Demo_compressed/13.png} &
\image{Figures/Demo_compressed/14.png} \\
\vspace{3pt}
{\footnotesize (2)~Ground Truth } &{\footnotesize (2)~1 Scribble~:~51.5\% } & {\footnotesize (2)~3 Scribbles~:~90.1\% } & {\footnotesize (2)~8 Scribbles~:~92.4\% }\\
\vspace{3pt}
\image{Figures/Demo_compressed/31.png} &
\image{Figures/Demo_compressed/32.png} &
\image{Figures/Demo_compressed/33.png} &
\image{Figures/Demo_compressed/34.png} \\
\vspace{3pt}
{\footnotesize (3)~Ground Truth } &{\footnotesize (3)~1 Scribble~:~83.9\% } & {\footnotesize (3)~3 Scribbles~:~92.5\% } & {\footnotesize (3)~8 Scribbles~:~94.9\% }\\

\vspace{3pt}
\image{Figures/Demo_compressed/51.png} &
\image{Figures/Demo_compressed/52.png} &
\image{Figures/Demo_compressed/53.png} &
\image{Figures/Demo_compressed/54.png} \\
\vspace{3pt}
{\footnotesize (4)~1 Scribble~:~90.2\% } &{\footnotesize (5)~2 Scribbles~:~96.7\% } & {\footnotesize (6)~1 Scribble~:~95.5\% } & {\footnotesize (7)~1 Scribble~:~95.6\% }\\

\vspace{3pt}
\image{Figures/Demo/61.png} &
\image{Figures/Demo/62.png} &
\image{Figures/Demo/63.png} &
\image{Figures/Demo/64.png} \\
\vspace{3pt}
{\footnotesize (8)~1 Scribble~:~91.4\% } &{\footnotesize (9)~2 Scribbles~:~93.4\% } & {\footnotesize (10)~1 Scribble~:~98.2\% } & {\footnotesize (11)~1 Scribble~:~94.0\% }\\

\end{tabular}
\vspace{0mm}
\caption{ Visualization results for ScribbleSeg-B3 on ADE20K~\cite{ade20k} and DAVIS~\cite{davis}. The numbers of scribbles and the IOU is marked below each image. The positive and negative scribbles are marked in green and blue. Demos 1-7 show the deterministic scribbles, which demonstrate our automatic evaluation procedure.  Demos 8-11 show the user customer scribbles. }
\vspace{-4mm}
\label{fig:demo}
\end{figure*}

\subsection{ Comparisons with Click-based Methods} 
Considering that our evaluation protocol introduced in Sec.~\ref{val_protocol} is compatible with click-based methods in a general form, we could directly compare our ScribbleSeg with previous click-based solutions. We first compare ScribbleSeg with the SOTA solutions for the click-based setting to show the benefits of using scribbles as the interaction form and prove the effectiveness of our method.

\vspace{0mm}
\paragraph{Quatitative comparisons.} In Tab.~\ref{tab:evaluation sota}, we list the click-based SOTA methods, and use our generalized metrics introduced in Sec.~\ref{val_protocol} to perform comparisons for our ScribbleSeg. We also report the performance of our strong baseline without PAM and CRM. The results show that our baseline already surpasses all click-based methods. This reflects the superiority of using scribbles as the interaction format. With PAM and CRM, the full version ScribbleSeg gets steady improvements. 

\vspace{-4mm}
\paragraph{Qualitative results.} In Fig.~\ref{fig:compare}, we compare  ScribbleSeg with the click-based SOTA method FocalClick~\cite{focalclick}. The results show that when given only one stroke of scribble, ScribbleSeg could outperform FocalClick with 3 clicks. It is clear that scribbles could provide significantly more indications than clicks. We hope ScribbleSeg could serve as a preferred choice for interactive segmentation.

\subsection{Qualitative Results}
The qualitative results for ScribbleSeg are demonstrated in Fig.~\ref{fig:demo}, where we use SegFormer-B3~\cite{xie2021segformer} as the segmentation model and make predictions on DAVIS~\cite{davis} and ADE20K~\cite{ade20k} benchmarks. 

\vspace{-4mm}
\paragraph{Evaluation procedure.}
In 1-4 rows, we show the evaluation procedure for sequentially added scribbles with the deterministic simulator. Results show that ScribbleSeg performs well on both things and stuff across diverse scenes. 

\vspace{-4mm}
\paragraph{User customer scribbles.}
The examples in the demo 8-11 show the user given scribbles with arbitrary shapes and thicknesses.
It demonstrates the robustness and generalization ability of ScribbleSeg for different interactions.