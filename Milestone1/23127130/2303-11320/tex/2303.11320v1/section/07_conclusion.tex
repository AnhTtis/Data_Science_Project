\vspace{-3pt}
\section{Conclusion}
\vspace{-3pt}
We are the first to formally address the task of scribble-based interactive image segmentation. We have constructed 
the standard train/val protocol and propose ScribbleSeg. Our method shows clear advantages compared with previous click-based models. We hope this work could serve as the baseline and assist the community in making further explorations on scribble-based interactive image segmentation and developing more powerful mask annotation tools.