\section{Introduction}

Interactive image segmentation requires users to indicate the target by providing simple annotations such as boxes, scribbles, and clicks. Compared with traditional annotation tools like the lasso or brush, interactive models could largely reduce the time and cost of creating masks, which is especially important in the era of big data.  

Demonstrations of common forms of interactions are shown in Fig.~\ref{fig:forms}. Among them, we claim that drawing scribbles is the most flexible and practical way to indicate the foreground and background regions. As shown in Fig.~\ref{fig:forms} (c), although boxes could indicate the size and rough location of the target, they could not make further indications inside the rectangle. As in (b), though clicks could provide in-depth annotations, a small number of clicks is unable to indicate the shape and size of the object accurately. Thus, click-based models often require extensive interactions, especially when annotating large and complicated objects. Scribbles, on the other hand, have the combined advantages of boxes and clicks. Long scribbles can accurately indicate the shape and size of the target, while short scribbles can make detailed corrections. Scribbles are therefore regarded as an extension of clicks as they encode more information about the user's intention.


\begin{figure}[t]
\newcommand{\image}{\includegraphics[width=0.31\columnwidth]}
\centering 
\tabcolsep=0.05cm
\renewcommand{\arraystretch}{0.06}
\begin{tabular}{ccc}
\vspace{1mm}
\image{Figures/Figure1/1.png} &
\image{Figures/Figure1/2.png} &
\image{Figures/Figure1/3.png} \\
\vspace{3mm}
{\footnotesize (a)~Scribble } & {\footnotesize (b)~Click } & {\footnotesize (c)~Box} \\
\end{tabular}
%\vspace{-2mm}
\caption{Comparisons of different interaction forms for interactive image segmentation. Foreground scribbles/clicks are marked in green and background scribbles/clicks in blue. }
\vspace{-3mm}
\label{fig:forms}
\end{figure}

Although drawing scribbles is more practical and favorable, this topic is rarely discussed by researchers. The settings for the few existing works~\cite{bai2014error,appearancesimilarity,DeepIGeoS,IFIS} vary greatly. They use different dataset, scribble-simulation methods, evaluation metrics, and does not provide code. This makes it hard to make comparisons and hinders the development of scribble-based interactive segmentation. 
In contrast, click-based interactive segmentation is flourishing with booming works~\cite{DIOS, li2018latentdiversity, mahadevan2018iteratively, sofiiuk2021ritm, firstclick, jang2019brs,fbrs,chen2021cdnet, focalclick}.
We believe a core reason is that DIOS~\cite{DIOS}~(CVPR'16) formulated a training pipeline and evaluation protocol, thus other researchers could follow the standard setting and focus on specific points to improve the performance.   

In this work, we attempt to reference the successes of click-based methods to reformulate the task of scribble-based interactive segmentation. However, there exist gaps between these two tasks, and in our exploration, we tackle the following challenges: 

\vspace{-5mm}
\paragraph{How to get diversified scribbles for training? }
Clicks could simply be represented by a pair of coordinates, but scribbles have arbitrary shapes and complicated representations.  Previous scribble-based works majorly use feature similarities to guide the segmentation, thus they~\cite{bai2014error,appearancesimilarity,DeepIGeoS} do not need training scribbles. \cite{IFIS} simply links randomly sampled points to simulate scribbles, but is too naive to cover different real-world user interactions. 

In this paper, we design multiple meta-simulators to simulate different kinds of annotating behavior and make compositions to ensure the diversity of training samplers. We also
adapt the iterative training strategy~\cite{mahadevan2018iteratively} to add scribbles on the error regions of the last prediction.



\vspace{-4mm}
\paragraph{How to fairly evaluate the model?} 

During evaluation, we design a deterministic simulator to generate the scribble 
according to the shape and size of the given mask. We use this method to add positive/negative scribbles automatically on the max difference region between the ground truth and the predicted masks. Thus, similar to click-based settings, we could measure the Number of Interactions required to reach the target IOU. This protocol provides a unified benchmark for different types of interactions, and enables us to compare the performance among models with different interaction forms. Besides, we construct a benchmark based on ADE20K~\cite{ade20k} to evaluate the model's ability in diversified scenarios and categories.


\begin{figure}[t]
\newcommand{\image}{\includegraphics[width=0.94\columnwidth]}
\centering 
\image{Figures/small_pipe.pdf} 
\vspace{-3mm}
\caption{The pipeline of ScribbleSeg. `PAM' denotes Prototype Adaption Module, and `CRM' means Corrective Refine Module.}
\label{fig:small_pipe}
\vspace{-5mm}
\end{figure}

\vspace{-5mm}
\paragraph{How to fully utilize the indications in scribbles? } Scribbles could be regarded as an extension of clicks, thus many designs from click-based methods could be transferred.  Differently, scribbles contain more indications than clicks that could be explored.
Thus, we first build a vanilla pipeline adapting from click-based methods and make specific designs considering the characteristics of scribbles. 
As in Fig.~\ref{fig:small_pipe}.  We first represent positive and negative scribbles into two binary masks. Then, we feed the 3-channel image, along with two scribble masks, and the previous prediction masks into a segmentation model. This method could be regarded as a vanilla solution. 
Starting from this baseline, we add two components to further improve its performance.

As scribbles cover more pixels than clicks, they could not only provide the location priors but also the appearance indications~(the scribble-covered regions),  
thus, we develop a Prototype Adaption Module~(PAM) to update the final projection kernel according to the user-provided scribbles. 
Besides, to produce high-quality masks, we design a Corrective Refine Module~(CRM), which takes the prediction of the segmentation model as input to estimate the probable error region and make corrections for the details. 

Our contribution could be summarized in three folds:
1)~We reformulate the task of scribble-based interactive segmentation and provide a standard train/validation protocol and benchmark. 2)~We propose ScribbleSeg, which shows strong performance for scribble-based interactive segmentation. 3)~We design PAM and CRM, which are simple and effective modules for interactive segmentation.