%%%%%%%%%%%%%%%%%%%%%%%%%%%%%%%%%%%%%%%%%%%%%%%%%%%%%%%%%%%%%%%%%%%%%%%%%%%%%%%%
%2345678901234567890123456789012345678901234567890123456789012345678901234567890
%        1         2         3         4         5         6         7         8

\documentclass[letterpaper, 10 pt, conference]{ieeeconf}  % Comment this line out if you need a4paper

%\documentclass[a4paper, 10pt, conference]{ieeeconf}      % Use this line for a4 paper

\IEEEoverridecommandlockouts                              % This command is only needed if 
                                                          % you want to use the \thanks command

\overrideIEEEmargins                                      % Needed to meet printer requirements.

% numbers option provides compact numerical references in the text. 
\usepackage[colorlinks = true,
            bookmarks=true,
            linkcolor = magenta,
            urlcolor  = magenta,
            citecolor = cyan,
            anchorcolor = magenta]{hyperref}
\usepackage{url}            % simple URL typesetting
\usepackage{booktabs}       % professional-quality tables
\usepackage{amsfonts}       % blackboard math symbols
\usepackage{nicefrac}       % compact symbols for 1/2, etc.
\usepackage{microtype}      % microtypography
\usepackage{multicol,lipsum}
\usepackage{subcaption}
\usepackage[skip=2pt,font=footnotesize]{caption}
\usepackage{comment}
\usepackage{booktabs, tabularx} % for professional tables
\usepackage{makecell}
\usepackage{multirow}
% Set the typeface to Times Roman
\usepackage{wrapfig}
\usepackage{times}
\usepackage{url}
\usepackage{algorithm}
\usepackage{algorithmic}
\usepackage{graphicx}
\usepackage{mathtools}
\usepackage{amsmath}
\usepackage{amssymb}
\usepackage{pifont}% http://ctan.org/pkg/pifont
\usepackage{array}
\usepackage{soul}

%\setcitestyle{square, comma, numbers, sort&compress, super}
\newcommand{\norm}[1]{\| #1 \|}
\newcommand{\abs}[1]{| #1 |}

\newcommand{\epssub}{\delta}
\newcommand{\cH}{\mathcal{H}}
\newcommand{\cS}{\mathcal{S}}
\newcommand{\cA}{\mathcal{A}}
\newcommand{\cB}{\mathcal{B}}
\newcommand{\cM}{\mathcal{M}}
\newcommand{\cW}{\mathcal{W}}

\newcommand{\chen}[1]{\textcolor{cyan}{#1}}
\newcommand{\addressed}[1]{\textcolor{red}{#1}}

\newcolumntype{P}[1]{>{\arraybackslash}p{#1}}
\newcolumntype{M}[1]{>{\arraybackslash}m{#1}}

\title{\LARGE \bf
ReorientDiff: Diffusion Model based Reorientation for Object Manipulation
}


\author{Utkarsh A. Mishra and Yongxin Chen% <-this % stops a space
% \thanks{*This work was not supported by any organization}% <-this % stops a space
\thanks{Utkarsh A. Mishra and Yongxin Chen are affiliated to the Institute for Robotics and Intelligent Machines~(IRIM), Georgia Institute of Technology
        {\tt\small umishra31@gatech.edu}, {\tt\small yongchen@gatech.edu}}%
}


\begin{document}



\maketitle
\thispagestyle{empty}
\pagestyle{empty}


%%%%%%%%%%%%%%%%%%%%%%%%%%%%%%%%%%%%%%%%%%%%%%%%%%%%%%%%%%%%%%%%%%%%%%%%%%%%%%%%
\begin{abstract}
% The ability to manipulate objects in a desired configurations is a fundamental requirement for robots to complete various practical applications. While certain goals can be achieved by picking and placing the objects of interest directly, object reorientation is needed for precise placement in most of the tasks. In such scenarios, the object must be reoriented and re-positioned into intermediate poses that facilitate accurate placement at the target pose. To solve this, we propose ReorientDiff, a diffusion model based reorientation planner which uses visual inputs from the scene and goal-specific language prompts to plan intermediate reorientation poses. The scene and the language-task is mapped to a joint scene-task representation feature space which is further used to condition a diffusion model. The diffusion model samples based on the representation using classifier-free guidance and then uses gradients of learned feasibility-score models for implicit iterative pose-refinement. After obtaining the sampled poses, a motion planner is used to execute the two-step operation. We achieve 95.1\% success rate and show our method's performance in simulation for a set of YCB-objects and a suction gripper.

The ability to manipulate objects in desired configurations is a fundamental requirement for robots to complete various practical applications. While certain goals can be achieved by picking and placing the objects of interest directly, object reorientation is needed for precise placement in most of the tasks. In such scenarios, the object must be reoriented and re-positioned into intermediate poses that facilitate accurate placement at the target pose. To this end, we propose a reorientation planning method, ReorientDiff, that utilizes a diffusion model-based approach. The proposed method employs both visual inputs from the scene, and goal-specific language prompts to plan intermediate reorientation poses. Specifically, the scene and language-task information are mapped into a joint scene-task representation feature space, which is subsequently leveraged to condition the diffusion model. The diffusion model samples intermediate poses based on the representation using classifier-free guidance and then uses gradients of learned feasibility-score models for implicit iterative pose-refinement. The proposed method is evaluated using a set of YCB-objects and a suction gripper, demonstrating a success rate of 95.2\% in simulation. Overall, we present a promising approach to address the reorientation challenge in manipulation by learning a conditional distribution, which is an effective way to move towards generalizable object manipulation. More results can be found on our website: \url{https://utkarshmishra04.github.io/ReorientDiff}.
\end{abstract}


%%%%%%%%%%%%%%%%%%%%%%%%%%%%%%%%%%%%%%%%%%%%%%%%%%%%%%%%%%%%%%%%%%%%%%%%%%%%%%%%
\section{Introduction}
\label{sec:introduction}

\begin{figure*}[t]
\centering
\includegraphics[width=0.9\linewidth]{figures/pitcher_drill.pdf}
    \caption{\textbf{Reorientation for precise target placement} The above figure represents the phenomenon of reorientation in which an object from a cluttered file has to be placed precisely in a shelf~(target position shown). As the object cannot be directly placed at the target location, our proposed method, ReorientDiff, samples a reorientation pose using a learned conditional distribution by a diffusion model. Such a proposed reorientation pose acts as a transition for facilitating successful placement. We also consider and take advantage of the object dynamics, as introduced by Wada~\textit{et al.}~\cite{wada2022reorientbot}, by which we ensure that un-grasping an object in an unstable pose will eventually allow the object to settle at some favourable pose.}
    \label{fig:train_pitcher_drill}
    \vspace{-1em}
\end{figure*}

Rearranging objects into specific poses is a fundamental task. It's not only essential for everyday activities at home but also plays a critical role in industrial applications like packing and assembly lines. Performing such a task requires extracting object information from visual-sensor data and planning a pick-place sequence~\cite{zeng2021transporter, tang2022selective}. While a single-step pick-place sequence is a viable solution, placing the object at a specific position and orientation is not always feasible. \emph{Reorientation} is an effective strategy when successfully changing an object's pose allows its placement at the target pose~\cite{wada2022reorientbot}. Such a strategy ensures feasible intermediate transition poses in scenarios without common grasps between the current pose and an object's desired placement pose.

The problem of finding reorientation poses is traditionally approached via rejection sampling based on finding successful grasps between the current pose-intermediate pose and intermediate pose-target pose. While previous classical approaches achieve this by using trajectory planners~\cite{kuffner2000rrt} to plan motion from the current pose to the desired pose via diverse candidate intermediate poses, such an exhaustive search is expensive on time and is limited by choice of the number of intermediate pose options. Recently, there have been efforts to improve the reorientation process via a data-driven rejection sampling solution using learned models~\cite{wada2022reorientbot} that predict the feasibility score of an intermediate pose w.r.t. feasible grasps in the current and target pose. 
% and help in early evaluation for rejection sampling. 
While their method improves the success rate and planning time, the algorithm requires processing significantly large number of candidate random samples and specifying the target object's placement pose. The former limits \emph{scalability}, and the latter challenges \emph{generalizability}. Lately, with the advances in language descriptor foundation models like CLIP~\cite{radford2021learning}, which projects images and texts to a common feature space, target object specifications can be directly correlated between visual information and suitable language commands, thus empowering human-robot interaction. This motivated us to explore grounding the problem statement of reorientation on language and hence embed semantic knowledge of the task with the spatial structure of the scene~\cite{shridhar2022cliport}.

With recent advances in artificial intelligence, an increasing number of scientific artifacts are being publicly released, from code, to data, to models \citep{piwowar2011shares,10.1145/3442188.3445922}. As a result, the reproducibility of these artifacts is becoming increasingly important for the research community \citep{Gundersen_Kjensmo_2018,wieling-etal-2018-squib,Gundersen_2019,belz-etal-2021-systematic}. This has led many scientific organizations and, associated with them, journals and conferences such as NeurIPS\footnote{\url{https://neurips.cc/Conferences/2021/PaperInformation/PaperChecklist}}, and AAAI\footnote{\url{https://aaai.org/Conferences/AAAI-22/reproducibility-checklist/}} but also the entire ACL comunity\footnote{\url{https://aclrollingreview.org/responsibleNLPresearch/}} to require the completion of checklists attesting the reproducibility of the released material \citep{dodge-etal-2019-show,pineau2021improving,rogers-etal-2021-just-think}. Additionally, members of the scientific community are becoming increasingly aware of this issue and organizing related workshops\footnote{\url{http://4real.di.fc.ul.pt/},\url{https://rrpr2022.sciencesconf.org/},\url{https://rohanalexander.com/reproducibility.html}} and shared tasks \citep{Pineau:2019,belz-etal-2021-reprogen}.
More recently, \citep{EMNLP} questioned that the reproducibility of the published works is not only related to the resources to be of public access, advocating for the addition of small-scale experiments and ad-hoc scripts to generate the results reported in the paper by the authors during the papers submissions.
Also, \citet{ulmer2022experimental} raise the need for experimental standards in the context of natural language processing (NLP) research, providing suggestions and best practices for the release of data, code, and models and for publishing experimental papers.
However, all these works have always focused on reproducibility while never pushing toward the actual correctness of the codebases and, consequently, of the models built from them.

In this paper, we claim that experimental correctness should be considered a fundamental aspect, prior to reproducibility, when sharing scientific artifacts in order to avoid completely wrong findings for the research community. As a proof of concept and through extensive experiments on the automatic speech recognition (ASR) and speech translation (ST) state-of-the-art architecture Conformer \citep{gulati20_interspeech}, we show that even only partially incorrect codes can lead to wrong conclusions, incorrectness that are also shared among the most commonly used codebases by researchers. 
Moreover, in addition to encouraging the research community to produce more correct material, we summarize a list of best practices that should be required to be listed in papers along with experimental settings to check the reliability of the released resources and the soundness of the results.
Specifically, our contributions are:
\begin{itemize}
    \item We analyze commonly used open-source implementations of the Conformer architecture, including Fairseq \citep{ott2019fairseq} and ESPnet \citep{watanabe2018espnet}, and discover that all the repositories are affected by at least one bug in their code (\S \ref{sec:analysis});
    \item We show that the presence of bug(s) can incredibly impact the results, leading to completely inaccurate findings which, however, can be hidden by high-quality scores of the model itself (\S \ref{sec:impact});
    \item We provide a list of guidelines to be included in the experimental settings of the paper to show the reliability of the experiments and the validity of their results (\S \ref{sec:checklist}).
\end{itemize}

\section{Related Work}
\label{sec:relatedwork}

\section{Related work}
% There is extensive recent work on speeding up analytical queries due to the need for consistent execution times in the face of the explosive growth in the volume of available data.
% In this section, we divide existing work into two categories: maintaining data freshness in MVs (\Cref{sec:server_side}) and optimizations for minimizing ad-hoc query latency (\Cref{sec:client_side}).

% \subsection{Maintaining Data Freshness in MVs}
% \label{sec:server_side}
% There exists a variety of data warehousing applications aimed at supporting low-latency analytical queries on fresh data.
% In particular, these applications require efficiency in the propagation of newly ingested data into downstream MVs.
 
\mypara{Efficient MV Refresh}
Incremental view maintenance (IVM) aims to update MVs to reflect newly ingested data, taking advantage of already computed results to perform the update in a manner more efficient than computing from scratch (full refresh)
~\cite{ahmad2012dbtoaster,mcsherry2013differential,armbrust2013generalized,zeng2016iolap, palpanas2002incremental, griffin1995incremental, agiwal2021napa, braun2015analytics}. 
There is an abundance of work in IVM, including incremental updates on duplicate values~\cite{griffin1995incremental}, non-distributive aggregate functions~\cite{palpanas2002incremental}, higher-order views~\cite{ahmad2012dbtoaster}, and sliding windows~\cite{braun2015analytics}. 
More recent works also investigate the scalability aspect of IVM, proposing scale-independent updates~\cite{armbrust2013generalized} and sampled views~\cite{zeng2016iolap}. Since \system is applicable to arbitrary SQL statements, \system is orthogonal to and is fully compatible with existing IVM techniques.

\mypara{MV Refresh Scheduling}
There exist works on scheduling the refresh of a MV set focusing on resolving cyclic dependencies~\cite{folkert2005optimizing}, minimizing weighted average staleness~\cite{golab2009scheduling}, and prioritizing MVs with the highest speedups on predicted future queries~\cite{ahmed2020automated}.
\system's scheduling to speed up the end-to-end refresh of the MV set is not addressed in existing works.

\mypara{DAG Workflow Scheduling}
The execution of workloads consisting of individual jobs with acyclic dependencies is a well-studied topic~\cite{apacheoozie,sparkdag,marchal2018parallel,bathie2020revisiting,baruah2022ilp}; many of these techniques can be applied to MV refresh runs studied in this paper.
Existing workflow scheduling systems such as Apache Oozie~\cite{apacheoozie}, Apache Airflow~\cite{airflow}, and Spark DAG scheduler~\cite{sparkdag} automate the execution of user-defined workflows following a topological order.
There are recent works aimed at finding more optimal execution orders in terms of peak memory usage~\cite{marchal2018parallel, bathie2020revisiting} and execution time on parallel platforms~\cite{baruah2022ilp}.
While \system is designed for use with MV refresh runs/workloads, our technique on joint scheduling and optimization can be reasonably applied to general workloads as a possible future direction.

% \paragraph{Incremental MV indexing}
% Update-optimized indices such as the log-structured merge-trees (LSM)~\cite{o1996log} are used for indexing MVs due to frequent updates induced by data ingestion~\cite{gupta2016mesa,agiwal2021napa}.
% \system is orthogonal to indexing: \system is capable of efficiently performing MV refresh runs regardless of whether the individual MVs are indexed or not.

% \subsection{Ad-hoc Query Latency Reduction}
% \label{sec:client_side}

% The minimization of ad-hoc analytical query response times is a well-studied topic due to latency being negatively correlated with the productivity of a data analyst during a data analysis session~\cite{liu2014effects}.
% Sessions are commonly conducted within visualization systems that contain a variety of optimization techniques to ensure that query response times fall within a certain latency tolerance.

% \mypara{Data prefetching}
% Data is often loaded into memory on a by-need basis in visualization systems to minimize interference with user-issued query computations~\cite{mani2017effective,xin2021enhancing,galakatos2017revisiting, yan2020auto, battle2016dynamic, crotty2016case, jalaparti2018netco}. 
% Query-time data retrieval can be significantly expedited by anticipating the data usage of the user in future queries and pre-loading the data into memory during the downtime between user queries (`think time'). SMART~\cite{mani2017effective} prefetches data for modified versions of current user-issued queries with different filters and dimensions. A-WARE~\cite{crotty2016case} maintains a sample store constantly refined through ingesting data based on speculations of future plots.
% ForeCache~\cite{battle2016dynamic} uses an SVM to predict the user's current analysis phase and accordingly prefetches data tiles partitioned based on different numerical values. NetCo predicts future queries via log analysis, and solves an ILP formulation to prefetch data to maximize the number of SLO-meeting queries~\cite{jalaparti2018netco}.
% In the case of MV refresh workloads, `think time' is nonexistent as individual MVs are refreshed back-to-back, rendering data prefetching techniques non-applicable.

\mypara{Intermediate Data Caching}
Some existing data visualization systems cache user-defined variables to support the typical incremental construction of data visualizations~\cite{zgraggen2016progressive, eichmann2020idebench} during data analysis sessions~\cite{jupyter, rstudio, colab}. 
Recent work proposes a management scheme for these cached variables under a memory constraint that greedily keeps variables with the highest estimated time savings based on predicted future user behavior via neural networks~\cite{xin2021enhancing}.
While useful for data visualization, a greedy approach to memory management fails to achieve satisfactory results compared to \system.

\mypara{Intermediate Result Reuse}

There exist works on storing intermediate results from computations to speedup future computations~\cite{yang2018intermediate, dursun2017revisiting, nagel2013recycling, michiardi2019memory, galakatos2017revisiting}.
Studied topics include the identification of reuse opportunities by finding overlaps in computation graphs of successive jobs~\cite{yang2018intermediate, michiardi2019memory},
selective storage under a space constraint with heuristics such as reuse probability~\cite{dursun2017revisiting}, expected savings~\cite{yang2018intermediate}, and recompute-storage cost difference~\cite{nagel2013recycling},
and rewriting incoming jobs to take advantage of stored intermediates~\cite{galakatos2017revisiting}.
These works share similarity with \system in their selection of items to store under a memory constraint, however, \system's problem setting requires it to uniquely consider the joint (re)ordering of job executions along with the selection of items.

% work that considers both job execution (re)order as well as intermediate result caching with a bounded amount of memory. but notably lack the joint aspect of \system and cannot be used to achieve immediate speedup on an incoming MV refresh run if no intermediates are stored beforehand. 

\mypara{Incremental Query Processing} Incremental processing (IQP) is useful for cases where not all data required for a query is immediately available. Similar to online aggregation~\cite{hellerstein1997online}, initial results of a query are computed on a subset of required data and progressively refined as the rest of the required data arrives in a predictable pattern~\cite{tang2019intermittent,wangtempura}. Tang et al. propose a dynamic programming formulation to pick intermediate states to store in memory given a limited memory budget~\cite{tang2019intermittent}. Tempura rewrites the query plan for more efficient execution based on predicted data arrival patterns~\cite{wangtempura}. While similarities exist between the problem setting of IQP and \system, such as management of bounded memory, \system notably includes additional joint optimization for the order of MV updates.

% \paragraph{Sampling}
% Sampling has seen wide use in visualization systems for reducing the computation time of ad-hoc queries by computing an approximate result over a subset of data as exact results are not always required by the user~\cite{crotty2016case, mani2017effective, zgraggen2014panoramicdata, kraska2021northstar, galakatos2017revisiting, kandula2016quickr}. 
% Commonly studied topics in sampling for ad-hoc queries include complex query sampling~\cite{kandula2016quickr}, rare event aggregation~\cite{kraska2021northstar, galakatos2017revisiting}, and maintaining consistency between related sampled visualizations~\cite{zgraggen2014panoramicdata}.
% Sampling server-side at the MV level compromises the assumptions of downstream applications and is thus not considered in \system.

% \paragraph{Progressive visualization}
% The latency tolerance for time-consuming queries can be circumvented by presenting a partially-computed visualization to the user within the tolerance, which is then incrementally refined until it is fully accurate~\cite{rahman2017ve, zgraggen2016progressive, crotty2015vizdom, kraska2021northstar, kamat2017infiniviz}.
% Example plots which benefit from progressive visualization include bar charts~\cite{kamat2017infiniviz} and heatmaps~\cite{rahman2017ve}.
% Similar to sampling, study on this topic is orthogonal to \system as pushing out partially-updated MVs compromises downstream assumptions.

% \subsection{Object Manipulation: Pick and Place}

% % The ability to manipulate objects in an intended way has been an extensive research direction in the literature. 
% While traditional methods have tried to solve the pick-and-place task using grasp planning~\cite{mahler2017dex, mahler2018dex, mahler2019learning} with known object geometries or using pose estimation methods~\cite{zhu2014single, zeng2017multi, deng2020self}, the recent literature has focused more on vision-based object manipulation~\cite{zeng2021transporter, seita2021learning}. Solving single-step pick and place tasks is typically achieved by planning grasp poses using segmentation and depth maps of the object, where it is considered that a picked object can be placed within the region of interest~(like in a box)~\cite{pinto2016supersizing, zeng2022robotic}. Recent studies have also shown object rearrangement planning capabilities~\cite{shridhar2022cliport, shridhar2022perceiver} where a target location is sampled based on some user-specified goal. Then a planner is responsible for generating a collision-free trajectory from the current to the target location. Some works have proposed object rearrangement as a long horizon problem~\cite{tang2022selective} consisting of multiple sequential pick and place actions to achieve a desired configuration.

% \subsection{Language Models for Robotics}

% Language models like GPT-2~\cite{radford2019language} and GPT-3~\cite{brown2020language} have proven to be quite effective in grounding the task's semantics with the scene's spatial features using several foundation models. One such foundation model is CLIP~\cite{radford2021learning} which encodes visual and language information into common representation space and has been helpful in learning policies for generalized pick-place tasks in planar tabletop~\cite{shridhar2022cliport} and 3D~\cite{shridhar2022perceiver} manipulation and for control of embodied AI agents~\cite{khandelwal2022simple, batra2020objectnav}. Further, language models have also been used in language-conditioned object rearrangement planning~\cite{liu2022structdiffusion, liu2022structformer} and supplying high-level instructions for long-horizon planning~\cite{ahn2022can}.  

% \subsection{Reorientation and Regrasping}

% Reorientation is a vital capability required for solving complex manipulation tasks. Prior research have explored this direction by planning to reorient objects using extrinsic supports~\cite{cheng2021learning, xu2022planar}, which enables them to re-grasp the object in a desired way. While \cite{cheng2021learning} proposed a graph neural network structure for pose sequencing and \cite{xu2022planar} used an end-to-end point-cloud based model for predicting reorientation poses, \cite{wermelinger2021grasping} proposed a heuristic based method for reorienting rock structures in excavation. Recently, ReorientBot~\cite{wada2022reorientbot} was proposed to solve the reorientation task using learned feasibility prediction models and rejection sampling. 
% % An essential aspect of reorientation task introduced in Wada~\textit{et al.}~\cite{wada2022reorientbot} is to use the dynamics of the target object such that if the object is un-grasped at a particular pose, it settles down to a favourable pose, enabling future grasps.

% \subsection{Generative Models for Robotics}

% Generative models like VAE have been used for planning grasps~\cite{mousavian20196dof} using visible point-cloud of objects and for constructing embedding space for high-level tasks for various downstream planning. Recently, diffusion models have been used extensively in literature for trajectory planning from imitation data~\cite{janner2022diffuser, ajay2022conditional} and for genre ating target poses for language-conditioned object rearrangement tasks~\cite{liu2022structdiffusion}. With language-guided scene and video generation applications, such models have been used for generating task-videos for robot learning~\cite{dai2023learning} and generalizing to unseen scenarios~\cite{yu2023scaling}.

\begin{figure*}[t]
\centering
% \includegraphics[width=0.9\linewidth]{figures/embedding_training.pdf}
\includegraphics[clip=True, trim={0 0.5cm 0 0}, width=0.9\linewidth]{figures/REORIENTDIFF_OVERVIEW.pdf}
    \caption{\textbf{Method Overview} (a) \textbf{Forward and reverse diffusion process.} ReorientDiff uses a combination of classifier-free guidance with classifier-based implicit refinement to sample from the learned distribution of intermediate poses. It ensures high-success feasibility with minimal variance by guiding the scene-task conditioned sampling using feasibility score gradients. (b) \textbf{Conditioned score function.} ReorientDiff learns the target distribution of feasible reorientation poses conditioned on the scene (pile of objects) and task (language prompt) jointly represented as $\Phi$. We use the pre-trained frozen CLIP text and image embeddings to formulate a joint embedding, trained end-to-end to encode information about placement pose, target object and current pose. Further, the current pose and target poses are processed to obtain feasible grasps~($\eta_1$ and $\eta_2$), which are used to calculate the feasibility gradients~$g_k$ in (a). The joint embedding is used as a sequence to condition the transformer-based score network $\epsilon_\theta(\textbf{q}_k, k, \Phi)$ via cross-attention to obtain the classifier-free score estimate in (a).}
    \label{fig:embedding}
    \vspace{-1em}
\end{figure*}

\section{Preliminary: Diffusion Models}
\label{sec:preliminary}

Consider samples $x_0$ from an unknown data distribution $q(x_0)$; diffusion models~\cite{ho2020denoising} learn to estimate the distribution by a parameterized model $p_\theta(x_0)$ using the given samples. The procedure is completed in two steps: the forward and the reverse diffusion processes. The former continuously injects Gaussian noise in $x_0$ to create a Markov chain with latents $x_{1:K}$ following transitions:
\begin{equation}
    q(x_{1:K}|x_0) = \prod_{k=1}^K q(x_k|x_{k-1}),
\end{equation}
where $q(x_k|x_{k-1})~=~\mathcal{N}(x_{k};\sqrt{1-\beta_k}x_{k-1}, \beta_k \mathbf{I})$ is the per-step noise injection following variance schedule $\beta_1, \dots, \beta_K$. This leads to the distribution $q(x_k|x_0)~=~\mathcal{N}(x_{k};\sqrt{\bar{\alpha}_k}x_{0},~(1-\bar{\alpha}_k)~\mathbf{I})$ following notations introduced in~\cite{song2020denoising} as $\alpha_k = 1 - \beta_k$ and $\Bar{\alpha}_k = \prod_{i=1}^k \alpha_i$. Note that $\bar \alpha_K \approx 0$ and thus $x_K \sim \mathcal{N}(0, \mathbf{I})$. The reverse diffusion learns to denoise the data starting from $x_K$ and following $p_\theta(x_{k-1}|x_k)~=~\mathcal{N}(x_{k-1};\mu_\theta(x_{k}, k), \beta_k \mathbf{I})$ where
\begin{equation}\label{eq:mu}
    \mu_\theta(x_{k}, k) = \frac{1}{\sqrt{\alpha_k}} \Big( x_k - \frac{\beta_k}{\sqrt{1 - \bar{\alpha}_k}} \epsilon_\theta(x_k, k) \Big).
\end{equation}
The parameterized model $\epsilon_\theta(x_k, k)$ is called the score-function, and it is trained to predict the perturbations and the noising schedule by the score-matching objective~\cite{song2020score}
\begin{equation}
    \arg\min_\theta \mathbb{E}_{x_0\sim q, \epsilon\sim \mathcal{N}(0, \mathbf{I})} \Big[\| \epsilon - \epsilon_\theta( \sqrt{\bar{\alpha}_k}x_{0} + \sqrt{1-\bar{\alpha}_k} \epsilon,k) \|^2\Big]
\end{equation}
In particular, such a score function represents the gradients of the learned probability distribution as
\begin{equation}
    \nabla_{x_k} \log p_\theta(x_k) = - \frac{1}{\sqrt{1-\bar{\alpha}_k}} \epsilon_\theta(x_k, k).
\end{equation}

% \begin{figure*}[t]
% \centering
% \includegraphics[width=\linewidth]{figures/forward_rev_process.pdf}
%     \caption{\textbf{Forward and Reverse Diffusion Process} The above figure shows the forward diffusion and the reverse denoising and sampling process of ReorientDiff. As described in~\autoref{sec:reorientdiff}, following classifier-free guidance will result in high-likelihood samples with high-variance in terms of success feasibility of the samples. Using the feasibility score gradients, we realize an implicit iterative pose refinement, as marked by the blue box in the figure. This significantly decrease variance and ensure high success feasibility of the samples.}
%     \label{fig:reorientdiff}
%     \vspace{-1em}
% \end{figure*}

\section{Reorientation}
\label{sec:reorientation}
% \subsection{Reorientation}
% \chen{this section introduces the overall pipeline. we can briefly mention how generative model is used in this pipeline}

Reorientation consists of solving two problems simultaneously, finding a pose that is reachable from the current pose and, after the effect of gravity, results in a pose that makes placement at the target pose achievable~(as shown in \autoref{fig:train_pitcher_drill}). 
% \chen{what? This makes} the problem challenging. 
Once we have an estimate of the current and target pose, it is intuitive that there will be a set of poses that will satisfy reorientability. However, only a small subset of such reorientable poses will be valid with provided kino-dynamic constraints on grasp poses. Identifying a candidate sample from this subset by either brute force sampling or optimization is computationally expensive and has to be done for every new scenario. 

To circumvent the above challenges, we propose a generative modeling approach to sample from the subset of valid reorientation poses. More specifically, our method learns the distribution of all reorientable poses using a conditional diffusion model and use classifiers to guide sampling towards valid poses directly during inference based on provided grasp poses. Hence, we divide the problem into three segments: i) regression-based end-to-end learning for finding the target object and placement pose from the scene and task description~(scenario), ii) learning the distribution of all reorientable poses for a given scenario once the object specifications are known and iii) learning grasp feasibility classifiers for selecting only the valid reorientation poses. To achieve this, we discuss our formulation for constructing scene-task representation, calculating grasp poses from object poses and learning grasp feasibility classifiers below. The diffusion model training and inference is discussed in the next section.

\subsection{Constructing Generic Scene-Task Representations}

A scene-task representation is a compact embedding of all available information present in the scene and specified by the user. We define a scene as the location and occupancy of the place from where a target object should be picked and a task as the language prompt containing the descriptions for selecting the target object and deciding placement poses. A top-down RGB-D camera provides an image~$\mathcal{I}\in\mathbb{R}^{H\times W\times3}$ and a heightmap~$\mathcal{H}\in\mathbb{R}^{H\times W \times 1}$ as the description of the pile. For learning the semantic and spatial embeddings~\cite{shridhar2022cliport, shridhar2022perceiver}, we use pre-trained CLIP foundation model and obtain semantic embeddings from the image~$\mathcal{I}$ and language~$\mathcal{L}$. We sequence the embeddings with the 
% trained ResNet50~(spatial) 
spatial embeddings for target object segmentation 
% with RGBD~($[\mathcal{I}, \mathcal{H}]$) 
to get a joint embedding sequence~$\Phi$ as generic scene-task representation as shown in~\autoref{fig:embedding}(b). 
The embedding is further used to predict the target object and the final placement pose.

\subsection{Sampling Grasp Poses}

We generate grasp poses by following the classical approach of converting the heightmap into a point cloud representation and eventually to a point-normal representation~\cite{wada2022reorientbot}. The predicted target object segmentation of the scene is then used to obtain the surface normals of the target object. After performing an edge masking using the Laplacian of the surface normals, the remaining point-normals on the surface are feasible grasp poses. While we sample grasp poses~$\eta_1$ for picking the object from the pile in the aforementioned manner, we assume that we have the mesh of the selected object
% \footnote{\chen{this could be too much for a footnote. why not include it in the main paper?} \addressed{no space} For complicated manipulation, as one shown in this work, the whole shape of the object plays an important role. While previous single-step pick and place tasks have all the necessary information about the object from a singe-view perspective, such an information is incomplete for reorientation tasks~(multi-step) which require knowledge of parts of the object not visible in the current pose. Previous related works~\cite{wada2022reorientbot, cheng2021learning, xu2022planar, wermelinger2021grasping} have proposed their approach under similar assumptions.} 
for sampling grasp poses~$\eta_2$ for placing the object at the predicted pose.

\subsection{Feasibility Score Models}
% \chen{is feasibility score model a good name? we already have score function}
Following prior works~\cite{mousavian20196dof, wada2022reorientbot, liu2022structformer}, a feasibility prediction model is important for early-evaluation and rejection of unfavorable samples. Such a feasibility model predicts the probability of success of a given grasp pose in successfully grasping an object in some candidate pose for a specified scene representation. The phenomenon of grasp success evaluation in dynamic reorientation pose, as addressed by~\cite{wada2022reorientbot}, is particularly interesting for our setup. Modelling dynamics for every object is indeed non-trivial and adds to the complexity; hence the feasibility model implicitly takes care of the dynamics of the object after deactivating the grasp. For checking feasibility or the probability of success~($y$) of sampled grasps for candidate reorientation poses~$\mathbf{q}$, we train two models: 
% \addressed{clarify what $y$ is}
\begin{itemize}
    \item For predicting success of reorientation from the current pose in a pile to a candidate pose given pick grasp poses~($\eta_1$) and scene representation, denoted as~$\mathcal{M}_1(y|\eta_1, \mathbf{q}, \Phi)$
    \item For predicting success of post-grasp deactivation pose from the candidate pose and placement grasp poses~($\eta_2$), denoted as~$\mathcal{M}_2(y|\eta_2, \mathbf{q}, \Phi)$
\end{itemize}

\section{ReorientDiff: Diffusion for Reorientation}
\label{sec:reorientdiff}
% \addressed{maybe we should include a diagram for reorientdiff}
We aim to generate intermediate reorientation poses for the target object, which enables successive placement at the desired pose and is reachable from the current pose.
% While prior works have used rejection sampling~\cite{wada2022reorientbot} and end-to-end supervised learning~\cite{xu2022planar, cheng2021learning}, 
We introduce a diffusion model-based approach to sample the most probable successful reorientation poses~($\mathbf{q}$) conditioned on the scene representation priors~($\Phi$), denoted as $p(\mathbf{q}|\Phi)$, which already contains the spatial and semantic information about the scene and the task. The denoising process can be further flexibly conditioned by sampling from modified distributions of the form
\begin{equation}
    \label{eq:dist}
    p_{h}(\mathbf{q}) \propto p(\mathbf{q}|\Phi) h(\mathbf{q}, \Phi),
\end{equation}
where $h(\mathbf{q}, \Phi)$ can represent several grasp success probability heuristics. By separating the grasp success from reorientation candidate sampling, the diffusion model trained for reorientation poses can be reused for varied selection of picking~($\eta_1$) and placement grasp poses~($\eta_2$).

\subsection{Classifier-free Conditional Pose Generation}

Following the distribution defined in ~\eqref{eq:dist}, we use classifier-free guidance~\cite{ho2022classifier} to sample high-likelihood reorientation poses for a particular scene-task representation. We train a score-network~\cite{song2020score}, $\epsilon_\theta(\mathbf{q}_k, \Phi) \propto \nabla_{\mathbf{q}_k} \log p(\mathbf{q}_k|\Phi)$ 
% \addressed{or $\nabla_{\mathbf{q}_k} \log p(\mathbf{q}_k| \Phi)$?}
, to denoise from~$\mathbf{q}_K \sim \mathcal{N}(\bf{0}, \mathbf{I})$ to possible reorientation poses~$\mathbf{q}_0$ from a $K$-step reverse diffusion denoising process. For each step, we calculate $\Tilde{\epsilon}_k$ as
\begin{equation}
    \Tilde{\epsilon}_k = \epsilon_\theta(\mathbf{q}_k, \Phi) + w_c \Big(\epsilon_\theta(\mathbf{q}_k, \Phi) - \epsilon_\theta(\mathbf{q}_k, \o)\Big)
\end{equation}
The scalar $w_c$ implicitly guides the reverse-diffusion towards poses that best satisfy the scene-task representations. Further, we calculate the successive samples for the next $(k-1)^{th}$ step using the DDIM~\cite{song2020denoising} sampling strategy and $\Tilde{\epsilon}_k$ as follows:
\begin{align}
\label{eq:update_diffusion}
    \tilde{\mathbf{q}}_{k-1} &\longleftarrow \sqrt{\Bar{\alpha}_{k-1}}\Big(\frac{\mathbf{q}_k - \sqrt{1 - \Bar{\alpha}_k} \;\;\Tilde{\epsilon}_k }{\sqrt{\Bar{\alpha}_k}}\Big) + \sqrt{1 - \Bar{\alpha}_{k-1}}\;\;\Tilde{\epsilon}_k
\end{align}
where, $\Bar{\alpha}_{k}$ is as described in~\autoref{sec:preliminary}.

\subsection{Feasibility Guided Pose Refinement}

We use the two feasibility-score prediction models ($\mathcal{M}_1$ and $\mathcal{M}_2$), which are pre-trained for predicting grasp feasibility for picking grasp, reorientation pose pairs and placement grasp, reorientation pose pairs, respectively. In such a case, the scores can be converted into probability distributions for each heuristic, defined as, for each $i=1,2$,
% \addressed{minus sign $-$ missing?}
\[
    h_i \equiv \ p(y=1|\eta_i, \mathbf{q}, \Phi)|_{\mathcal{M}_i} =  \text{exp}\Big(-(1 - \mathcal{M}_i(y|\eta_i, \mathbf{q}, \Phi))^2\Big) 
    % \\
    % h_2 \equiv \ p(y=1|\eta_2, q, \Phi)|_{\mathcal{M}_2} = \ \text{exp}\Big(-(1 - \mathcal{M}_2(y|\eta_2, q, \Phi))^2\Big)
\]

Following classifier-based guidance~\cite{dhariwal2021diffusion} formulation for the heuristics, the reverse diffusion can be formulated as: 
% \chen{I thought $h_1, h_2$ only depend on $\mathbf{q}_0$} 
\begin{multline}
\label{eq:new_dist}
    p_{h}(\mathbf{q}_k|\mathbf{q}_{k+1}, y, \Phi) \propto \\  p(\mathbf{q}_k|\mathbf{q}_{k+1},\Phi) \;\; p(y|\eta_1, \hat{\mathbf{q}}_0^k, \Phi)|_{\mathcal{M}_1} \ p(y|\eta_2, \hat{\mathbf{q}}_0^k, \Phi)|_{\mathcal{M}_2}
\end{multline}
where, $\hat{\mathbf{q}}_0^k$ is the sample proposed at diffusion step $k$ and defined as: 
% \chen{the noise below maybe incorporated by using a stochastic version of DDIM directly}
\begin{equation}
    \hat{\mathbf{q}}_0^k = \frac{\mathbf{q}_k - \sqrt{1 - \Bar{\alpha}_k} \;\;\Tilde{\epsilon}_k }{\sqrt{\Bar{\alpha}_k}}
\end{equation}
Considering Taylor first order approximations for heuristics and standard reverse process Gaussian $(\mu_\theta(\mathbf{q}_k, k, \Phi), \beta_k \mathbf{I})$ as described in~\autoref{sec:preliminary}, we get the new mean~($\mu_{\theta, h}(\mathbf{q}_k, k, \Phi)$) for the distribution~$p_{h}(\mathbf{q}_k|\mathbf{q}_{k+1}, y, \Phi)$ in \eqref{eq:new_dist} as: 
% \addressed{what's $\sigma$?} \addressed{missing sign?}
\begin{align}
     & \mu_{\theta, h}(\mathbf{q}_k, k, \Phi) \nonumber \\
     &= \mu_\theta(\mathbf{q}_k, k, \Phi) + \beta_k \sum_{i=1}^2 w_i\nabla_{\mathbf{q}_k} \log p(y|\eta_i, \mathbf{q}_k, \Phi)|_{\mathcal{M}_i} \nonumber \\
     &= \mu_\theta(\mathbf{q}_k, k, \Phi) - \beta_k \sum_{i=1}^2 w_i\nabla_{\mathbf{q}_k} \Big[1 - \mathcal{M}_i(y|\eta_i, \hat{\mathbf{q}}_0^k, \Phi)\Big]^2. \nonumber
\end{align}
In view of \eqref{eq:mu}, we then obtain the modified score
    \[
    \epsilon_k \longleftarrow \Tilde{\epsilon}_k - \sqrt{1 - \Bar{\alpha}_k} \ g_k
    \]
where $g_k = -\beta_k\sum_{i=1}^2 w_i\nabla_{\mathbf{q}_k} \Big[1 - \mathcal{M}_i(y|\eta_i, \hat{\mathbf{q}}_0^k, \Phi)\Big]^2$. We notice that injecting noise to $g_k$, as in stochastic DDIM, can slightly improve the performance. 
% Now, to incorporate the feasibility-score gradients into the standard score function, we calculate the gradients as follows:
% \[
%     g_k = -\beta_k\sum_{i=1}^2 w_i\nabla_{\mathbf{q}_k} \Big[1 - \mathcal{M}_i(y|\eta_i, \hat{\mathbf{q}}_0^k, \Phi)\Big]^2 + \sqrt{\zeta \sigma} \Tilde{q},
% \]
% and adapt the score function according to the following:
% \begin{align}
%     \epsilon_k &\longleftarrow \Tilde{\epsilon}_k - \sqrt{1 - \Bar{\alpha}_k} \;\; g_k
% \end{align}
We calculate the final $\mathbf{q}_{k-1}$ using the refined $\epsilon_k$ in \eqref{eq:update_diffusion}. A visual clarification of the forward and reverse diffusion is shown in~\autoref{fig:embedding}(a).

\begin{figure}[t]
\centering
\includegraphics[clip=True, trim={0 0.3cm 0 0}, width=\linewidth]{figures/predictions.pdf}
    \caption{\textbf{Visual Analysis of Scene-Task Network Performance} The scene-task network maps the visual~(row $2$) image of the pile~(row $1$) and language~(bottom row) inputs to a feature space which is used to predict the placement location~(row $4$) and target object segmentation~(row $3$).}
    \label{fig:networkeval}
    \vspace{-1em}
\end{figure}

\begin{figure*}[t]
\includegraphics[clip=true, trim={0 0.3cm 0 0}, width=\linewidth]{figures/all_reorient_results.pdf}
    \caption{\textbf{Reverse Diffusion for Reorientation Pose Generation} The reverse sampling process for $4$ $k$-steps at $k=20, 12, 4, 0$ for $K=20$ 
    % \addressed{$k=100$ to $k=0$?}
    in four different scene-task scenarios comprising of the Cracker Box, Mustard Bottle and Sugar Box in different target orientations are shown above. The scenes are shown in the left-side of every sub-figure and consists of the pile with the target object and the predicted placement location on the shelf. The language prompt defining each of the tasks is mentioned below each sub-figure. It consists of either the absolute (the object's name) or the relative~(heaviest/lightest) reference to the object and details about the target placement.}
    \label{fig:diffusionObjects}
\vspace{-1em}
\end{figure*}

\section{Results: Simulation}
\label{sec:resultssim}

Based on the environment setup as discussed in~\autoref{sec:reorientation}, we create datasets, train diffusion and feasibility score models and evaluate them in simulation.
% for proper placement conditions.

\subsection{Dataset Generation and Training}

We use PyBullet~\cite{coumans2021} and an OMPL~\cite{sucan2012ompl} based motion planner to solve for collision-free path between current pose and a candidate reorientation pose and from the reorientation pose to the ground-truth placement pose for diverse set of YCB-objects and target locations. We sampled approximately $40000$ candidate poses following Wada~\textit{et al.}~\cite{wada2022reorientbot}. The goal properties were converted into modular language instructions, and the success of pick and place for both the steps was recorded. The scene and task properties were used to construct the joint visual-language embedding space, which was further used to train the feasibility score models using binary success labels. Eventually, we train a conditional diffusion model using only the successful reorientation poses. Such a diffusion model is reusable for diverse set of grasp poses when combined with the feasibility score models.

\subsection{Performance Evaluation: Scene-Task Representation}

To evaluate the quality of the scene-task embedding network, we analyze the accuracy of the object selection and placement pose prediction along with the error in the predicted segmentation. We show a visual analysis in~\autoref{fig:networkeval} where the output segmentation and the predicted placement pose in the shelf are shown for three scenes and tasks. For accurate shelf-level estimation, we round each object's predicted height to the nearest shelf-level height, and a similar post-processing is conducted for the object orientation. 
% To add complexity, although we consider only four orientations: front, back, left and right, we discretize the possible orientations into $8$ possible options and round the predicted orientation to the nearest option. 
In our experiments, the object selection network was $100\%$ accurate, and the number of pixels wrongly classified was about $1\%$ of the complete image on average over $100$ random samples. The average error in predicting the height of the target placement after post-processing is around $8$~mm, and the mean error in the yaw angle of the predicted pose is $0.3$~rad.


% Average Pose Loss: 0.008775011206108865
% Average Orientation Loss: [0.01257151 0.         0.0210217  0.00975848]
% Average Object Loss: 0.0
% Average Segmentation Loss: 808.66

\subsection{Performance Evaluation: Diffusion with Guidance}

% \addressed{how many samples does the DM generate?}
The trained classifier-free conditional diffusion model and the score feasibility models are used to perform the reverse diffusion using the classifier-free guidance with and without feasibility score guidance. Experiments comparing the performance of both the methods are shown in~\autoref{fig:diffusionObjects} for a set of YCB Objects~\cite{calli2015ycb} and different scene-task scenarios where only $40$ candidate poses are sampled and top $10$ high-likelihood poses are selected. The comparison shows that while the classifier-free guidance is good enough to sample high-likelihood reorientation poses, the primary purpose of the feasibility score gradients is to reduce the variance in the pose generation and ensure high success probability. A numerical analysis of the overall success is shown and compared with the rejection sampling based baseline~\cite{wada2022reorientbot} in~\autoref{tab:performance}. 

\begin{table}[h]
\caption{Success evaluation of the proposed method as compared to the rejection sampling based baseline ReorientBot. The ReorientDiff algorithm was tested for more than 100 different scene task settings consisting of equal distribution of the selected objects and all the orientations. A task is considered a success if it is completed at-least once in 3 random seeds.}
\label{tab:performance}
\centering
\begin{tabular}{c|c|c|c}
\hline
\textbf{Method}               & \textbf{\begin{tabular}[c]{@{}c@{}}Success (\%)\\ Reorient\end{tabular}} & \textbf{\begin{tabular}[c]{@{}c@{}}Success (\%)\\ Place\end{tabular}} & \textbf{\begin{tabular}[c]{@{}c@{}}Success (\%)\\ Overall\end{tabular}} \\ \hline
Random   & 43.4  & 40.8  & 40   \\ 
ReorientBot   & 97.9  & 95.1  & 93.2   \\ 
\begin{tabular}[c]{@{}c@{}}ReorientDiff\\ (w/o Guide)\end{tabular}         & 97.4   & 92.3  & 90.8          \\ 
\textbf{ReorientDiff} & \textbf{98.9}   & \textbf{96.5}    & \textbf{95.2}  \\
\hline                                                          
\end{tabular}
\vspace{-1em}
\end{table}

The reorientation success percentage holds different relevance as compared to the baseline. The baseline does two step reverse rejection sampling where reorientation search is conducted over candidates which are feasible for placement, so there might be a scenario where there is no solution. For the case of ReorientDiff, the reorientation success measures the capability of the diffusion model to generalize to poses which ensure reorientability and scope for future placement. Higher reorientation success and lower placement success is an indication that the model is short-sighted and is giving importance to a single step success metric. From~\autoref{tab:performance}, we ensure high reorientability success along with better placement success. The overall success is based on the accurate placement of the object from the reoriented pose, and it represents the successful completion of a task. The metric is measured by calculating the difference between the desired and the pose after final placement.

% \begin{table}[h]
% \caption{(Incomplete) Ablation and Comparison Performance \chen{why is our planning time so long?}}
% \label{tab:performance}
% \centering
% \begin{tabular}{c|c|c|c}
% \textbf{Method} & \textbf{\begin{tabular}[c]{@{}c@{}}Success (\%)\\ Place\end{tabular}} & \textbf{\begin{tabular}[c]{@{}c@{}}Success (\%)\\ Overall\end{tabular}} & \textbf{\begin{tabular}[c]{@{}c@{}}Planning \\ Time (sec)\end{tabular}} \\ \hline
% ReorientBot           &         95.1  & 93.2 & \textbf{2.5}     \\ 
% ReorientDiff (w/o Guide) & 86.3 & 85.8 &  2.7 \\ 
% \textbf{ReorientDiff} & \textbf{95.1} & \textbf{93.7} & 5.3          
% \end{tabular}
% \end{table}

\begin{table}[h]
\caption{Success evaluation with different levels of discretization while sampling using ReorientDiff.}
% \chen{the performance is too sensitive to $K$. the result should be better if we use DEIS}}
\label{tab:kperformance}
\centering
\begin{tabular}{c|c|c|c}
\hline
\textbf{ReorientDiff K}               & \textbf{\begin{tabular}[c]{@{}c@{}}Success (\%)\\ Reorient\end{tabular}} & \textbf{\begin{tabular}[c]{@{}c@{}}Success (\%)\\ Place\end{tabular}} & \textbf{\begin{tabular}[c]{@{}c@{}}Success (\%)\\ Overall\end{tabular}} \\ \hline
$K = 10$   &  97.4  & 94.5  & 93.9   \\ 
\textbf{$K = 20$}   &  \textbf{98.9}  & \textbf{96.5}  & \textbf{95.2}   \\ 
% $K = 200$    & 97.2   & 93.2       & 92.9  \\ 
% $K = 50$ & 98.3  & 94.8 & 93.9          \\
% % $K = 100$ &     97.5         & 91.1  & 88.6 \\
\hline
\end{tabular}
\vspace{-1em}
\end{table}

\subsection{Performance Evaluation: K-Step Reverse Diffusion}

% \begin{table}[h]
% \caption{(Incomplete) Ablation and Comparison Performance \chen{why is our planning time so long?}}
% \label{tab:timeperf}
% \centering
% \begin{tabular}{c|c|c}
% \textbf{Method} & \textbf{\begin{tabular}[c]{@{}c@{}}Success (\%)\\ Place\end{tabular}} & \textbf{\begin{tabular}[c]{@{}c@{}}Planning \\ Time (sec)\end{tabular}} \\ \hline
% \textbf{ReorientDiff @ $K = 256$} & \textbf{95.1} & \textbf{5.3}  \\
% ReorientDiff @ $K = 200$ & TBD & 4.3                   \\
% ReorientDiff @ $K = 100$ & TBD & 2.5               \\
% ReorientDiff @ $K = 50$ & TBD & 1.5                
% \end{tabular}
% \end{table}

Sampling from a trained diffusion models is flexible and can be achieved using different levels of discretization between $x_K \sim \mathcal{N}(0, \mathbf{I})$ to meaningful reorientation poses. We perform the complete analysis for multiple values of the number of reverse denoising steps $K$ as shown in~\autoref{tab:kperformance}. ReorientDiff performs well with only 20 sampling steps.  

Following our analysis on performance, we explored the time consumption for the overall planning of a successful reorientation pose from a given scene and corresponding task information. We provide the recorded timings for all of our ablations and the baseline in~\autoref{tab:tperformance}.
\begin{table}[h]
\caption{Computational analysis of the planning time for ReorientDiff~($K=20$) with and without feasibility score guidance along with the baseline.}
\label{tab:tperformance}
\centering
\begin{tabular}{c|c}
\hline
\textbf{Method}               & \textbf{\begin{tabular}[c]{@{}c@{}}Planning \\ Time (sec)\end{tabular}} \\ \hline
ReorientBot   & 2.5 \\ 
ReorientDiff (w/o Guide)   & 0.3 \\ 
% ReorientDiff @ $K = 10$ & 5.3 \\
\textbf{ReorientDiff}         &      \textbf{1.05}\\ 
% % ReorientDiff @ $K = 200$    & 4.3 \\ 
% ReorientDiff @ $K = 50$ & 2.5\\
% ReorientDiff @ $K = 100$ & 5.3 \\
\hline
\end{tabular}
\vspace{-1em}
\end{table}

Our findings show that ReorientDiff leverages fast sampling strategies of FastDPM~\cite{kong2021fast} to recover from computationally heavy gradient calculations for reverse denoising steps. Without using the guidance from the feasibility-score models, classifier-free guidance requires even less time as compared to the baseline, ReorientBot, as shown in~\autoref{tab:tperformance}. Hence, from our visual and empirical analysis, ReorientDiff successfully proves that formulating the problem of reorientation as learning a conditional distribution is an efficient and scalable way to move towards more generalizable object manipulation.

\section{Conclusion}
\label{sec:conclusion}

Diffusion models are powerful generative models capable of modeling (conditional) distributions. Our proposed method ReorientDiff exploits the capabilities of such models to predict reorientation poses conditioned on a compact scene-task representation embedding containing information about the target object and its placement location. Further, the samples are refined using learned feasibility-score models to reduce uncertainty and ensure the success of the planned intermediate poses. With only $10$ candidate reorientation poses, we achieved an overall success rate of $95.2$\% across various objects. With the possible inclusion of point-cloud-based object representations~\cite{simeonov2023shelving}, such a method can generalize to a more diverse set of objects.
% We believe that with the incorporation of several high-order solvers~\cite{zhang2022fast} for solving the reverse diffusion sampling, we can drastically reduce the planning time without trading off the final performance. 
% We consider incorporating more efficient sampling schemes and better generalization performance for unseen objects and placement goals as a potential future work.


\bibliographystyle{ieeetr}
\bibliography{references}

\end{document}
