\documentclass[final,UKenglish,a4paper,thm-restate,autoref,numberwithinsect]{lipics-v2021}

\nolinenumbers                  % Do we have to use line numbers?
%\linenumbers

%%%%%%%%%%%%%%%
% Usepackages %
%%%%%%%%%%%%%%%


\usepackage{amssymb,amsmath,mathrsfs}
\usepackage{mathtools}
\usepackage{thmtools}
\usepackage{xspace}
\usepackage{dsfont}
\usepackage{tikz}
\usetikzlibrary{cd,fit,calc,positioning,arrows,automata,shapes}
\usepackage[all]{xy}
\usepackage{ifthen}
\usepackage[notref,notcite]{showkeys}
\usepackage[noadjust]{cite}
\usepackage{rotating}


%\usepackage[all]{xy}    
%\newcommand{\toKl}{\xymatrix{\ar[0,1]|-{\circ} &}}
%\def\arKl[#1]{\ar[#1]|-{\circ}}
%\SelectTips{cm}{}
%\usepackage{times}   % condensed fonts
\usepackage{microtype}
\usepackage{graphicx}
\usepackage{amsbsy,eucal}  
\usepackage{dsfont}
\usepackage{xspace}
\newcommand{\TT}{\mathcal{T}}
\newcommand{\TTcse}{\TT_{\mathsf{cse}}}
\newcommand{\treepartial}{\rho}
\newcommand{\tbar}{\overline{t}}
\newcommand{\cut}{\partial}
\newcommand{\muf}{\approx}
%\newcommand{\height}{\mbox{\sf ht}}
\renewcommand{\root}{\mbox{\sf root}}
\newcommand{\rootnode}{\root}
\renewcommand{\phi}{\varphi}
\newcommand{\phibar}{\overline{\phi}}
\usepackage{ifdraft}
%\ifdraft{
%  \overfullrule=0mm % disable marking of overfull lines
%}{}

%%%% LM: I took out the \ifdraft below
%% \newcommand{\blue}[1]{\ifdraft{\textcolor{blue}{#1}}{#1}}

%
% Theorem environments; needs to be loaded before hyperref so that
% \autoref works
%
%\usepackage{amsthm,thmtools}
%
%\newcommand{\ownthmSpaceAbove}{5pt}
%\newcommand{\ownthmSpaceBelow}{5pt}
%% all thmtools hooks are within the theorem group, so we need to adjust global definitions
\newcommand{\resetCurThmBraces}{%
\gdef\curThmBraceOpen{(}%
\gdef\curThmBraceClose{)}}
\resetCurThmBraces
\newcommand{\removeThmBraces}{%
\gdef\curThmBraceOpen{}%
\gdef\curThmBraceClose{}}
\resetCurThmBraces

\newenvironment{notheorembrackets}{\removeThmBraces}{\resetCurThmBraces}

\usepackage{etoolbox}
\patchcmd{\thmhead}{(#3)}{\curThmBraceOpen #3\curThmBraceClose }{}{}

%
% Nice enumerate environments.
%
% LIPIcs explicitly tells us not to use those. 
%
%\usepackage[inline]{enumitem}
%\setlist[enumerate,1]{label=(\arabic*),font=\normalfont,align=left,leftmargin=0pt,labelindent=0pt,listparindent=\parindent,labelwidth=0pt,itemindent=!,topsep=0pt,parsep=0pt,itemsep=0pt,start=1}
%\setlist[enumerate,2]{label=(\alph*),font=\normalfont,labelindent=*,leftmargin=*,start=1}
%\setlist[itemize]{labelindent=*,leftmargin=*,topsep=5pt,itemsep=3pt}
%\setlist[description]{labelindent=*,leftmargin=*,itemindent=-1 em}


%
% Hyperref
%
\usepackage{hyperref}
\hypersetup{hidelinks,final,bookmarks}

%\usepackage[UKenglish]{babel}
\addto\extrasUKenglish{% only necessary if babel is used
  \renewcommand{\sectionautorefname}{Section}
  \renewcommand{\subsectionautorefname}{Section}
}


% show sections and subsections in toc (needed for hyperref)
\setcounter{tocdepth}{2}

%%%% Text wrapping for showkeys
\usepackage{seqsplit}
\usepackage{xstring}
\newcommand{\defaultshowkeysformat}[1]{%
% instead of \textvisiblespace you can also put in ~
% if you want to keep a plain space at space characters
\StrSubstitute{#1}{ }{\textvisiblespace}[\TEMP]%
\parbox[t]{\marginparwidth}{\raggedright\normalfont\small\ttfamily\(\{\){\color{red!50!black}\expandafter\seqsplit\expandafter{\TEMP}}\(\}\)}%
}
\newenvironment{hideshowkeys}{%
  \renewcommand*\showkeyslabelformat[1]{%
  \noexpandarg%
  }
}{%
  \renewcommand*\showkeyslabelformat[1]{%
  \noexpandarg%
  \defaultshowkeysformat{##1}%
  }
}

\renewcommand*\showkeyslabelformat[1]{%
\noexpandarg%
\defaultshowkeysformat{#1}%
}

\renewcommand\itemautorefname{Item}
\newcommand{\itemref}[2]{\autoref{#1}.\ref{#2}}

\hyphenation{homo-morphism}
\hyphenation{mono-morphism}
\hyphenation{cate-gori-cal}
%
% Marginal notes
%
\usepackage[footnote,marginclue,nomargin]{fixme}


%
% Allow page breaks in math displays
%
\allowdisplaybreaks

%%%%%%%%%%%%%%%%
% New commands %
%%%%%%%%%%%%%%%%

%
% nicer overlines for math symbols
%
\newcommand{\overbar}[1]{\mkern 1.5mu\overline{\mkern-1.5mu#1\mkern-1.5mu}\mkern 1.5mu}
% Usage \mybar{<scale factor>}{<right shift>}{<math expression>}
%
\newcommand{\mybar}[3]{%
  \mathrlap{\hspace{#2}\overline{\scalebox{#1}[1]{\phantom{\ensuremath{#3}}}}}\ensuremath{#3}
}

\newcommand{\myhat}[3]{%
  \mathrlap{\hspace{#2}\widehat{\scalebox{#1}[1]{\phantom{\ensuremath{#3}}}}}\ensuremath{#3}
}

\newcommand{\mytilde}[3]{%
  \mathrlap{\hspace{#2}\widetilde{\scalebox{#1}[1]{\phantom{\ensuremath{#3}}}}}\ensuremath{#3}
}

\newcommand{\barM}{\mybar{0.7}{2.9pt}{M}}
\newcommand{\bpartial}{\overline{\partial}}
\newcommand{\tildeM}{\mytilde{0.7}{2.9pt}{M}}


%\makeatletter
%\newcommand{\mysubsec}[1]{%
%  \par\medskip\noindent{\bfseries #1}\hspace{3mm}%
%  \@ifnextchar\par{\@gobble}{}% this eats a following \par if any
%}
%\makeatother
\newcommand{\mysubsec}{\subparagraph*}

\newcommand{\xra}[1]{\xrightarrow{~#1~}}
\newcommand{\xla}[1]{\ensuremath{\xleftarrow{~#1~}}}

%
% Macros
%
\newcommand{\smooth}{smooth\xspace}
\newcommand{\opcit}[1][.]{\textit{op.cit#1}\xspace}

\newcommand{\ol}[1]{\overline{#1}}

\DeclareMathOperator{\Sub}{\mathsf{Sub}}

\newcommand{\op}[1]{\operatorname{\mathsf{#1}}}
\newcommand{\id}{\op{id}}

\newcommand{\inj}{\op{in}}
\newcommand{\inl}{\op{inl}}
\newcommand{\inr}{\op{inr}}
\newcommand{\pr}{\op{pr}}

\newcommand{\monoto}{\ensuremath{\rightarrowtail}}
\newcommand{\epito}{\ensuremath{\twoheadrightarrow}}
\newcommand{\subto}{\ensuremath{\hookrightarrow}}

% Categories
\newcommand{\cat}[1]{\mathscr{#1}}
\def\A{\cat A}
\def\B{\cat B}
\def\C{\cat C}
\def\D{\cat D}
\newcommand{\E}{\mathcal{E}}
\newcommand{\FF}{\mathcal{F}}
\newcommand{\MM}{\mathcal{S}}
\newcommand{\Set}{\mathsf{Set}}
\newcommand{\Pos}{\mathsf{Pos}}
\newcommand{\Pfn}{\mathsf{Pfn}}
\newcommand{\Gra}{\mathsf{Gra}}
\newcommand{\KVec}{K\text{-}\mathsf{Vec}}
\newcommand{\CPO}{\mathsf{CPO}}
\newcommand{\DCPO}{\mathsf{DCPO}}
\newcommand{\DCPOb}{\mathsf{DCPO}_\bot}
\newcommand{\CMS}{\mathsf{CMS}}
\newcommand{\Met}{\MS}
\newcommand{\MS}{\mathsf{Met}}
\newcommand{\UMet}{\mathsf{UMet}}
\newcommand{\CMet}{\mathsf{CMet}}
\newcommand{\Ord}{\mathsf{Ord}}
\newcommand{\Top}{\ensuremath{\mathsf{Top}}}
\newcommand{\KHaus}{\ensuremath{\mathsf{KHaus}}}
\newcommand{\Khaus}{\KHaus}
\newcommand{\Haus}{\mathsf{Haus}}
\newcommand{\Stone}{\mathsf{Stone}}

% Functors
\newcommand{\Id}{\mathsf{Id}}
\newcommand{\Pow}{\mathscr{P}}
\def\pow{\Pow}
\newcommand{\Powf}{\Pow_\mathsf{f}} % leichter als omega (ordinal number!)
\def\powf{\Powf}
\def\powfin{\Powf}
\newcommand{\powcl}{\Pow_{\mathsf{cl}}}
\newcommand{\V}{\mathscr{V}}
\renewcommand{\H}{\mathcal{H}}

% Class M
\newcommand{\M}{\mathcal{M}}

\newcommand{\N}{\mathds{N}}
\def\Nat{\N}
\newcommand{\R}{\mathds{R}}
\newcommand{\Z}{\mathds{Z}}
\newcommand{\fpair}[1]{\ensuremath{\langle #1 \rangle}}

\newcommand{\Coalg}{\mathop{\mathsf{Coalg}}}
\newcommand{\Alg}{\mathop{\mathsf{Alg}}}
\newcommand{\colim}{\mathop{\mathsf{colim}}}

% structures of initial algebras and terminal coalgebras
\newcommand{\ter}{\tau}
\newcommand{\ini}{\iota}

% for backwards functions
\newcommand*\cocolon{%
        \nobreak
        \mskip6mu plus1mu
        \mathpunct{}%
        \nonscript
        \mkern-\thinmuskip
        {:}%
        %\mskip2mu
        \relax
}

% Others
\newcommand{\HF}{H\!F}
\newcommand{\arity}{\mathop{\mathsf{ar}}}
\newcommand{\set}[1]{\{#1\}}

\newcommand{\Abar}{B}
\newcommand{\dbar}{d'}
\newcommand{\eps}{\varepsilon}
\newcommand{\opp}{\mathsf{op}}

%
%
%
\newcommand{\mhat}{\widehat{m}}
\newcommand{\ehat}{\widehat{e}}
\newcommand{\chat}{\widehat{c}}  
\renewcommand{\o}{\cdot}
\newcommand{\temptree}{\mathsf{tr}}
%
% For taking out stuff
%
\newcommand{\takeout}[1]{\empty}

% utility for tikz-cd
\newcommand{\descto}[3][]{\arrow[phantom]{#2}[#1]{\text{\footnotesize{}\begin{tabular}{c}#3\end{tabular}}}}
\newcommand{\desctox}[4][]{\arrow[phantom,#2]{#3}[#1]{\text{\footnotesize{}\begin{tabular}{c}#4\end{tabular}}}}

\newcommand{\pair}[1]{\langle {#1} \rangle}

\tikzset{shiftarr/.style={
        rounded corners,%
        to path={--([#1]\tikztostart.center)
                     -- ([#1]\tikztotarget.center) \tikztonodes
                     -- (\tikztotarget)},
}}

% usage: \pullbacklabel{angle}
\newcommand{\pullbackangle}[2][]{\arrow[phantom,to path={
                     -- ($ (\tikztostart)!1cm!#2:([xshift=8cm]\tikztostart) $)
                        node[anchor=west,pos=0.0,rotate=#2,
                        inner xsep = 0]
                        {\begin{tikzpicture}[minimum
                        height=1mm,baseline=0,#1]
    \draw[-] (0,0) -- (.5em,.5em) -- (0,1em);
                        \end{tikzpicture}}}]{}}

%%% Some additional commands needed in this paper
\renewcommand{\P}{\Pow}


%%%%%%%%%%%%%%%%%%%%%%%%
% Theorem Environments %
%%%%%%%%%%%%%%%%%%%%%%%%
%\theoremstyle{plain}
%\newtheorem{theorem}{Theorem}[section]
%\newtheorem{proposition}[theorem]{Proposition}
%\newtheorem{lemma}[theorem]{Lemma}
%\newtheorem{corollary}[theorem]{Corollary}
\theoremstyle{definition}
\newtheorem{defn}[theorem]{Definition} %% definitions in non-italic
\newtheorem{rem}[theorem]{Remark} %% remarks in non-italic                  
\newtheorem{construction}[theorem]{Construction}
\newtheorem{assumption}[theorem]{Assumption}
\newtheorem{oproblem}[theorem]{Open Problem}
\newtheorem{notation}[theorem]{Notation}
%\newtheorem{example}[theorem]{Example}


\title{On Kripke, Vietoris and Hausdorff Polynomial Functors}
\authorrunning{J.~Ad\'amek S.~Milius, L.~S.~Moss}

% Funding only in accepted version! We like need to save space.
\author{Ji\v{r}\'i Ad\'amek}%
  {Czech Technical University in Prague, Czech Republic and Technische
  Universität Braunschweig} %, Germany} %% SM: Let's save a line.
  {j.adamek@tu-braunschweig.de}%
  {}{Supported by the grant No.~22-02964S of the Czech Grant Agency}
\author{Stefan Milius}%
  {Friedrich-Alexander-Universität Erlangen-Nürnberg, Germany}%
  {stefan.milius@fau.de}%
  {https://orcid.org/0000-0002-2021-1644}%{}%
  {Supported by Deutsche Forschungsgemeinschaft (DFG) under project
    \mbox{MI~717/9-1} and as part of the Research and Training Group 2475 ``Cybercrime and Forensic Computing'' (393541319/GRK2475/1-2019)}% and MI~717/5-2}
\author{Lawrence S.~Moss}%
  {Indiana University, Bloomington IN, USA}%
  {larry.moss@gmail.com}%
  {}{Supported by grant \#586136 from the Simons Foundation.}%

%%%%%%%%%%%%%%%%%%%%%
% Document Metadata %
%%%%%%%%%%%%%%%%%%%%%

\category{(Co)algebraic pearls}

\Copyright{Ji\v{r}\'i Ad\'amek, Stefan Milius, Lawrence S.~Moss}

% TODO Chosen by fair dice roll. Do we need more/different?
\ccsdesc[500]{Theory of computation~Models of computation}
\ccsdesc[500]{Theory of computation~Logic and verification}
\keywords{Hausdorff functor, Vietoris functor, initial algebra,
  terminal coalgebra}

%\hideLIPIcs %uncomment to remove references to LIPIcs series (logo, DOI, ...), e.g. when preparing a pre-final version to be uploaded to arXiv or another public repository

\takeout{
\EventEditors{Paolo Baldan and Valeria de Paiva}
\EventNoEds{2}
\EventLongTitle{10th Conference on Algebra and Coalgebra in Computer Science (CALCO 2023)}
\EventShortTitle{CALCO 2023}
\EventAcronym{CALCO}
\EventYear{2023}
\EventDate{June 19--22, 2023}
\EventLocation{Bloomington IN, USA}
\EventLogo{}
\SeriesVolume{211}
\ArticleNo{}}% end takeout

%%%%%%%%%%%%%%%%%%
% Document Start %
%%%%%%%%%%%%%%%%%%

\begin{document}
\FXRegisterAuthor{sm}{asm}{SM}%Stefan
\FXRegisterAuthor{ja}{aja}{JA}%Jirka
\FXRegisterAuthor{lm}{alm}{LM}%Larry

\maketitle



\begin{abstract}
  The Vietoris space of compact subsets of a given Hausdorff space yields an endofunctor~$\V$ on the category of Hausdorff spaces. Vietoris polynomial endofunctors on that category are built from~$\V$, the identity and constant functors by forming products, coproducts and compositions. These functors are known to have terminal coalgebras and we deduce that they also have initial algebras.  We present an analogous class of endofunctors on the category of extended metric spaces, using in lieu of $\V$ the Hausdorff functor $\H$.  We prove that the ensuing Hausdorff polynomial functors have terminal coalgebras and initial algebras. Whereas the canonical constructions of terminal coalgebras for Vietoris polynomial functors takes $\omega$ steps, one needs $\omega + \omega$ steps in general for Hausdorff ones.  We also give a new proof that the closed set functor on metric spaces has no fixed points.%
  \smnote{I vote against the sentence about the Worrell result; for me
    this is appendix material.}
\end{abstract}

\section{Introduction}
\label{S:intro}

This paper presents results on terminal coalgebras and initial
algebras for certain endofunctors on the categories $\Haus$ of
Hausdorff topological spaces and $\Met$ of extended metric spaces.
These results are based on the terminal coalgebra construction first
presented by Ad\'amek~\cite{A74} (in dual form) and independently by
Barr~\cite{barr}.  Given an endofunctor $F$, iterate $F$ on the unique
morphism $!\colon F1 \to 1$ to obtain the following
$\omega^\opp$-chain
\begin{equation}\label{eq:oop-chain-3}
  1 \xla{!} F1 \xla{F!} FF1 \xla{FF!} FFF1 \xla{FFF!} \cdots
\end{equation}
Assume that the limit exists, and denote it
 by $V_\omega$ and the limit cone by $\ell_n\colon V_\omega \to F^n 1$ ($n
< \omega$).
We obtain a unique morphism $m\colon FV_\omega \to V_\omega$ such that
for all $n \in \omega^\opp$ we have%
\smnote{@Jirka: sorry, this cannot be inlined; we refer to this
  equation later (as is clear from the number and label).}
\begin{equation}
  \label{diag:m}
  F\ell_n = \big(FV_\omega \xra{m} V_\omega \xra{\ell_{n+1}} F^{n+1} 1\big).
%  \begin{tikzcd}
%    FV_\omega
%    \ar{r}{m}
%    \ar{rd}[swap]{F\ell_n}
%    &
%    V_\omega
%    \arrow{d}{\ell_{n+1}}
%    \\
%    & F^{n+1} 1
%  \end{tikzcd}
\end{equation}
If $F$ preserves the limit $V_\omega$, then $m$ is an isomorphism.
Its inverse yields the terminal coalgebra $m^{-1}\colon
V_\omega \to FV_\omega$; shortly $\nu F = V_\omega$, and we say
that the terminal coalgebra is \emph{obtained in $\omega$ steps}.

\smnote{Jirka proposes to replace this text by a shorter, uninspiring
  and technical text; I like the current version better.} %
This technique of \emph{finitary iteration} is the most basic and
prominent construction of terminal coalgebras.  However finite
iteration requires that the limit in (\ref{eq:oop-chain-3}) exists and
also that it is preserved by the functor.  It does \emph{not} apply on
$\Set$ to the finite power set functor $\powf$.  For that functor
$FV_{\omega} \not\cong V_{\omega}$.  However, a modification of finite
iteration does apply, as shown by Worrell~\cite{worrell:05}.  One
makes a \emph{second infinite iteration}, iterating $F$ on the
morphism $m\colon FV_{\omega} \to V_{\omega}$ rather than on
$!\colon F1 \to 1$, obtaining a chain
\begin{equation}\label{diag:mm}
  V_\omega
  \xla{m}
  V_{\omega + 1}
  \xla{Fm}
  V_{\omega + 2}
  \xla{FFm}
  \cdots
\end{equation}
Its limit is denoted by
$V_{\omega+\omega} = \lim_{n < \omega} V_{\omega+n}$ with the limit
cone $\bar l_n\colon V_{\omega+\omega} \to V_{\omega+n}$, for
$n < \omega$.  Worrell's insight was that this second limit,
$V_{\omega+\omega}$, is preserved by every finitary functor. We prove
that this also works for functors built from~$\powf$ using product,
coproduct, and composition (which may be non-finitary).%
\smnote{We should add this; it won't be clear for all readers that
  Kripke polynomial functors might be non-finitary.}
\smnote{I am against mixing plurals `products', `coproducts' with one
  singular `composition'; the constructions that are used are
  product, coproduct and composition (singular), and it is clear that
  a particular functor can be build by applying each of them more than once.}
These are
the \emph{Kripke polynomial functors} mentioned in our title.

\smnote{I like the current version of this paragraph better than
  Jirka's proposal.}
We are interested in other settings where terminal coalgebras may be
built using either the limit in~\eqref{eq:oop-chain-3} or the limit in
\eqref{diag:mm}.  We study fixed points of naturally occurring
endofunctors on Hausdorff spaces and metric spaces, endofunctors built
from the Vietoris and Hausdorff functors and several other natural
constructions.


In the category~$\Top$ a good analogy of $\powf$ is the \emph{Vietoris
  functor} $\V$ assigning to every space $X$ the space of all compact
subsets equipped with the Vietoris topology (\autoref{S:Vietoris}).
Hofmann et al.~\cite{HNN19} define \emph{Vietoris polynomial} functors
as those endofunctors on $\Top$ built from $\V$, the constant
functors, and the identity functor, using product, coproduct, and
composition. We study this on the subcategory $\Haus$ of Hausdorff
spaces and use that $\V\colon \Haus \to \Haus$ preserves limits of
$\omega^{\opp}$-chains, a fact for which we present a new proof.%
\smnote{I added that we present a new proof; one of our contributions
  in this paper.}  This implies that for \emph{Vietoris polynomial
  functors} (defined as above but with $\V$ in lieu of~$\powf)$, the
terminal coalgebra exists and is the limit
of~\eqref{eq:oop-chain-3}. The existence of initial algebras follows.
 
We also present a result for the category $\MS$ of metric spaces and
nonexpanding maps. The role of the Vietoris functor is played by the
\emph{Hausdorff functor} $\H$ assigning to every space $X$ the space
$\H X$ of all compact subsets with the Hausdorff metric.

%% taken out stuff moved to another file
  
\mysubsec{Other contributions.}
% The aforementioned results are the main ones in this paper.
In addition to the aforementioned results we show results obtained by
either varying the category or the endofunctor.  For example, consider
again the Hausdorff polynomial functors.  Whenever $F$ is such a
functor and the constants involved in its construction are complete
spaces, $\nu F$ again turns out to be complete. Analogous results hold
for compact spaces, or ultrametric spaces. Finally, we present a proof
of the description of $\nu\powfin$ and $V_\omega$ for $\powfin$
exhibited by Worrell~\cite{worrell:05} (the latter without a proof).%
\smnote{Worrell did have a proof for $\nu
  \powfin$ (showing that it follows
  from $V_\omega$).}

We simplify a proof of a known negative result:  the variation of $\H$ obtained
by moving from compact sets to closed sets has no fixed points.

\mysubsec{Related work.}
%
Our work is more general and hence improves results of
Abramsky~\cite{abramsky}, Hofmann et al.~\cite{HNN19}, and
Worrell~\cite{worrell:05}.

There are numerous results about the existence and construction of
terminal coalgebras in the literature. At several places we discuss
other possible approaches to our results.



\section{Preliminaries}
\label{section-preliminaries}

We review a few preliminary points. We assume that readers are
familiar with basic notions of category theory as well as algebras and
coalgebras for an endofunctor. 
We denote by $\Set$ the category of 
sets and functions, $\Top$ is the category of  topological spaces and
continuous functions, and $\Met$ is the category of \emph{(extended)
  metric spaces} (so we might have $d(x,y) = \infty$) and
non-expanding maps: the functions $f\colon X \to Y$ where
$d(f(x),f(x')) \leq d(x,x')$ holds for every pair $x, x' \in X$.
Note that this class of morphisms is smaller than the class of 
continuous functions between metric spaces.
%
\begin{rem}\label{P:chain}
  Consider an $\omega^\opp$-chain
  \begin{equation}\label{Xsequence}
    X_0 \xla{f_0} X_1 \xla{f_1} X_2 \xla{f_2} \cdots
  \end{equation}
  % 
  \begin{enumerate}
  \item In $\Set$, the limit $L$ consists of all sequences
    $(x_n)_{n < \omega}$, $x_n \in X_n$ that are \emph{compatible}:
    $f_{n}(x_{n+1}) = x_n$ for every $n$.  The limit projections are the functions
    $\ell_n\colon L \to X_n$ defined by $\ell_n((x_i)) = x_n$.
    
  \item\label{R:Vietoris-tech:1} In $\Top$, the limit is again carried
    by the same set $L$ as in $\Set$, and the limit projections $\ell_n$ are
    also the same.  The topology on $L$ has as a base the sets
    $\ell_n^{-1}(U)$, for $U$ open in $X_n$.
    \takeout{ %% taken out
      This collection
      forms a base since it contains $L$ as an element and is closed
      under intersections: if $n \leq m$, then
      $\ell_n^{-1}(U_1) \cap \ell_m^{-1}(U_2) = \ell_m^{-1}(U_2\cap W)$,
      where $W = f_{m,n}^{-1}(U_1)$ and $f_{m,n}\colon X_m\to X_n$ is
      the evident composition.  With this topology on $L$, the maps
      $\ell_n$ mentioned above are continuous.}% end takeout
    
  \item\label{R:Haus:4} In $\Met$, the limit is again carried by the
    same set $L$, and the same limit projections $\ell_n$.  The metric on $L$
    is defined by $d((x_n), (y_n)) = \sup_{n <\omega} d(x_n,y_n)$.
    % With this metric, the projections are non-expanding maps.
  \end{enumerate}
\end{rem}

\subparagraph*{Smooth Monomorphisms.}
%
In addition to terminal coalgebras, we also study initial algebras for the functors
of interest in this paper.  For this, we call on a general result which allows one
to infer the existence of the initial algebra for an endofunctor $F$ from the existence
of a terminal coalgebra for $F$ (or in fact of any algebra with monic
structure).

For a class $\M$ of monomorphisms we denote by $\Sub_\M(A)$ the
collection of subobjects of~$A$ represented by monomorphisms from
$\M$. To say that this is a dcpo means that it is a set which (when
ordered by factorization in the usual way) is a poset having directed joins.

\removeThmBraces
\begin{defn}[{\cite[Def.~3.1]{amm21}}]\label{D:smooth}
  Let $\M$ be a class of monomorphisms closed under isomorphisms and
  composition.
  \begin{enumerate}
  \item\label{D:smooth:1} We say that an object $A$ has \emph{\smooth
      $\M$-subobjects} provided that $\Sub_\M(A)$ is a dcpo with
    bottom $\bot$, where the least element
    and directed joins are given by colimits of the corresponding
    diagrams of subobjects.

  \item\label{D:smooth:2} The class $\M$ is \emph{smooth} if every
    object of $\A$ has smooth $\M$-subobjects.
  \end{enumerate}
  A category has \emph{smooth monomorphisms} if the class of all
  monomorphisms is smooth.
\end{defn}
\resetCurThmBraces
%
%
\begin{example}\label{E:cms}
  \begin{enumerate}
  \item \label{E:cms:set} The categories $\Set$ and $\Top$ have smooth
    monomorphisms, and so does the full subcategory of Hausdorff
    spaces. This is easy to see.
    
  \item\label{E:cms:met}
    The category $\Met$ also has smooth monomorphisms (these are the injective
    non-expanding maps)~\cite[Lemma~A.1]{amm21}.
    
    The full subcategory $\CMS$ of complete metric spaces does not have
    smooth monomorphisms. However, strong monomorphisms (isometric
    embeddings) are smooth in both $\Met$ and $\CMS$~\cite[Lemma~A.2]{amm21}.
    
  \item Strong monomorphisms (subspace embeddings) in $\Top$ are not
    smooth~\cite[Ex.~3.5]{ahrt23}. 
    
  \end{enumerate}
\end{example}
%
\removeThmBraces
\begin{theorem}[{\cite[Cor.~4.4]{amm21}}]\label{T:initial}
  Let $\M$ be a smooth class of monomorphisms. If an endofunctor~$F$
  preserving~$\M$ has a terminal coalgebra, then it has an initial algebra.
\end{theorem}
\resetCurThmBraces
%
\noindent
Note that loc.~cit.~states more: given any algebra
$m\colon FA \monoto A$ where $m$ lies in $\M$, the initial algebra
exists and is a subalgebra of $(A,m)$.

\section{Kripke Polynomial Functors}
\label{section-Kripke}

We turn to the first collection of functors mentioned in the title of
this paper: the Kripke polynomial functors on $\Set$.  The name stems
from \emph{Kripke structures} used in modal logic. Our definition below is a
slight generalization of the (finite) Kripke polynomial functors
presented by Jacobs~\cite[Def.~2.2.1]{jacobs}.  (Kripke polynomial
functors using the full power-set functor were originally introduced
by R\"{o}{\ss}iger~\cite{Roessiger00}.)  We admit arbitrary products in lieu of just arbitrary
exponents.%
%\smnote{I reformulated this, since we should be more careful what we
%  say here: both Rößiger and Jacobs use the full power-set; only
%  Jacobs speaks of \emph{finite} Kripke polynomial functors, when
%  $\pow$ is replaced by $\powfin$.}
%
\begin{defn}\label{D:Kripke}
  The \emph{Kripke polynomial functors} $F$ are the set functors built
  from the finite power-set functor, constant functors and the
  identity functor, by using product, coproduct and composition. In
  other words, Kripke polynomial functors are built according to the
  following grammar:
  \[
    \textstyle
    F ::= \powfin \mid  A \mid \Id \mid \prod_{i\in I} F_i \mid
    \coprod_{i\in I} F_i \mid FF,
  \]
  where $A$ ranges over all sets (and is interpreted as a constant
  functor) and $I$ is an arbitrary index set.
\end{defn}
% 
\begin{rem}
  The constant functors could be omitted from the grammar since they
  are obtainable from the rest of the grammar.  The constant functor
  with value $1$ is the empty product.  For each set $A$, the constant
  functor with value $A$ is then a coproduct: $A = \coprod_{a \in A} 1$.
\end{rem}
%
\begin{example}\label{E:fin-branch} 
  The Kripke polynomial functor $FX = \powfin(A \times X)$ is the type
  functor of finitely branching labelled transition systems with a set
  $A$ of actions.
\end{example}
% 
\begin{rem}\label{R:finitary-Kripke}
  An endofunctor is \emph{finitary} if it preserves directed colimits.
  Worrell~\cite{worrell:05} proved that for every finitary set functor
  the terminal coalgebra is obtained in $\omega+\omega$ steps.  We
  prove a version of Worrell's result but for Kripke polynomial
  functors.

  There are Kripke polynomial set functors which are not finitary. One
  example of such a functor is $F(X) = X^{\Nat}$, where $\Nat$ is the
  set of natural numbers.  There are also finitary set functors which
  are not Kripke polynomial functors.  One example (see~\autoref{omitted})
  is the functor assigning to a set $X$ the set of \emph{nonempty}
  finite subsets of $X$.%
  \smnote{Omitted proofs should come in an appendix (as usual for
    conference papers); so I brought back Larry's proof in Appendix~D.}
\end{rem}


Our proof below uses ideas from Worrell's work~\cite{worrell:05}.
%
\begin{theorem}\label{T:Kripke}
  Every Kripke polynomial functor $F$ has a terminal coalgebra
  obtained in $\omega+\omega$ steps: $\nu F = V_{\omega + \omega}$.%
\end{theorem}
%
\begin{proof}%
  \begin{enumerate}
  \item\label{P:Kripke:1} We first observe that $F$ preserves
    monomorphisms and intersections of monomorphisms.  This is clear
    for constant functors and for $\Id$, and it%
    \smnote{My feeling tells me that `it' sounds better than `this'
      here; @Larry? LM: I think `it' is better here.}
    is easy to see for
    $\powf$. Moreover, these properties are clearly preserved by
    product, coproduct and composition.

  \item\label{P:Kripke:2} Let $(X_n)_{n < \omega}$ be an
    $\omega^\opp$-chain in $\Set$. Then the canonical morphism
    $m\colon F(\lim X_n) \to \lim FX_n$ is monic.  This is obvious for
    constant functors and $\Id$.  Let us check it for $\powfin$.
    Denote the limit projections by $\ell_n\colon \lim X_n \to X_n$
    and $p_n\colon \lim \powf X_n \to \powf X_n$ ($n < \omega$); the
    canonical morphism $m$ is unique such that
    $p_n \cdot m = \powf\ell_n$. Now given $S \neq T$ in
    $\powf(\lim X_n)$, without loss of generality we can pick
    $x \in T \setminus S$. Using that the $\ell_n$ are jointly monic,
    for every $s \in S$ we can choose $n< \omega$ such that
    $\ell_n(x) \neq \ell_n(s)$. Since $S$ is finite, this choice can
    be performed independently of $s \in S$. Thus
    $\ell_n(x) \not\in \ell_n[S]$, and hence
    $\powf\ell_n(T)\neq \powf(S)$. Thus, $\powf\ell_n$ is a jointly
    monic family.  Since $p_n \cdot m = \powf\ell_n$, we see that $m$
    is monic.

  \item\label{P:Kripke:2.5} An induction on Kripke polynomial functors
    $F$ now shows that $m\colon V_{\omega+1}\to V_{\omega}$ is monic.
    We have seen this for the base case functors in
    \autoref{P:Kripke:2}.  The desired property that $m$ is monic is
    preserved by products, coproducts and composition. In particular,
    for a composition $FG$ note that the canonical morphism for $FG$
    is the composition
    \[
      FG(\lim X_n) \xra{Fm} F(\lim GX_n) \xra{m'} \lim FGX_n,
    \]
    where $m$ is the canonical morphism for $G$ w.r.t.~the given
    $\omega^\opp$-chain and $m'$ the one for $F$ and the
    $\omega^\opp$-chain $(GX_n)_{n< \omega}$. So this morphism
    $m'\o Fm$  is monic since
    both $m$ and $m'$ are so and $F$ preserves monomorphisms by
    \autoref{P:Kripke:1}. 

  \item\label{P:Kripke:3} Since $F$ preserves monomorphisms, we see
    that $Fm$, $FFm$ etc.~are monic. We obtain a decreasing chain of
    subobjects $ V_{\omega+n } \monoto V_\omega$.  Therefore, the
    limit $V_{\omega + \omega} = \lim_{n < \omega} V_{\omega + n}$ is
    simply the intersection of these subobjects. From
    \autoref{P:Kripke:1} we know that $F$ preserves this limit. It
    follows that $\nu F = V_{\omega + \omega}$, as desired.\qedhere
  \end{enumerate}
\end{proof}
%
\begin{corollary}\label{C:K-I}
  Every Kripke polynomial functor $F$ on $\Set$ has an initial algebra.
\end{corollary}

\noindent
This follows from \autoref{T:Kripke}, \autoref{E:cms}.\ref{E:cms:set},
and \autoref{T:initial} since $F$ preserves monomorphisms.

\removeThmBraces
\begin{example}[{\cite{worrell:05}}]\label{E:strong-ext}
  The functor $\powfin$ has a terminal coalgebra consisting of all
  finitely branching strongly extensional trees (up to isomorphism of
  trees). Moreover, the limit $V_\omega$ consists of all compactly
  branching strongly extensional trees. We present a proof of these results
  in \autoref{S:app-worrell} (\autoref{T:Worrell}).
  %\renewcommand{\theoremautorefname}{Theorems}
  %(\autoref{T:worrell} and~\ref{T:Worrell-main}).
  %\renewcommand{\theoremautorefname}{Theorem}%
\end{example}
\resetCurThmBraces

\section{Vietoris Polynomial Functors}
\label{S:Vietoris}

Hofmann et
al.~\cite{HNN19} proved that Vietoris polynomial functors on the category
$\Haus$ of Hausdorff spaces have terminal coalgebras obtained in
$\omega$ steps. Our proof is slightly different from theirs
because we wish to avoid a result stated by
Zenor~\cite{zenor70} 
whose proof is incomplete.



Recall that a \emph{base} of a topology is a collection $\mathcal{B}$
of open sets such that every open set is a union of members of
$\mathcal{B}$.  A \emph{subbase} is a collection of open sets whose
finite intersections form a base.  For every collection $\mathcal{B}$
of subsets of the space, there is a smallest topology for which
$\mathcal{B}$ is a (sub)base, the family of unions of finite
intersections from $\mathcal{B}$.
%
\begin{defn}\label{D:Vietoris}
  \begin{enumerate}
  \item Let $X$ be a topological space.  We denote by $\V X$  the space of
    compact subsets of $X$ equipped with the ``hit-and-miss''
    topology.  This topology has as a subbase all sets of the following forms:
    \begin{equation}\label{eq:predlift}
      \begin{aligned}[c]
        U^{\Diamond}
        & = \set{R \in VX : R \cap U\neq \emptyset}
        & \text{($R$ hits $U$)}, \\
        U^{\Box}
        &= \set{R \in VX : R \subseteq U}
        & \text{($R$ misses $X \setminus U$),}
      \end{aligned}
    \end{equation}
    where $U$ ranges over the open sets of $X$.    We call $\V X$ the \emph{Vietoris space of
      $X$}, also known as the \emph{hyperspace} of $X$.
    
  \item  Recalling that the image of a compact set
  under a continuous function is also compact,
for a continuous function $f\colon X \to Y$ we put $\V f (A) =
    f[A]$ for every compact subset $A$ of $X$. 
  \end{enumerate}
\end{defn}
%
\begin{rem}\label{R:Vietoris}
  \begin{enumerate}
  \item For a compact Hausdorff space $X$, Vietoris~\cite{Vietoris22}
    defined $\V X$ to consist of all \emph{closed} subsets of $X$.  These
 are   the same as the compact subsets in this case. In the
    coalgebraic literature, $\V X$ has also mostly been studied for spaces
    $X$ which are compact Hausdorff. However,  
    the ``classic Vietoris space'' (using closed subsets) does not
    yield a functor on $\Top$ (see Hofmann et
    al.~\cite[Rem.~2.28]{HNN19}). Hofmann et
    al.~\cite[Def.~2.27]{HNN19} call the functor $\V$ in
    \autoref{D:Vietoris} the \emph{compact Vietoris functor}. 

  \item\label{R:Vietoris:2} Michael~\cite[Thm.~4.9.8]{Michael51}
    proved that $X$ is Hausdorff iff so is $\V X$.
    \takeout{ Indeed, for `only
    if', given $K_i \in \V X$, $i = 1, 2$, satisfying $K_1 \neq K_2$,
    assume without loss of generality that $K_1 \setminus K_2 \neq \emptyset$ and let
    $x \in K_1 \setminus K_2$. Since $X$ is Hausdorff and $K_2$ is compact,
    there are
    disjoint open sets $U_i$, $i =1,2$, such that $x \in U_1$ and
    $K_2 \subseteq U_2$. Hence, $K_1 \in U_1^\Diamond$ and
    $K_2 \in U_2^\Box$. Moreover, $U_1^\Diamond$ and $U_2^\Box$ are
    disjoint, since for $R \subseteq U_2$ we have
    $R \cap U_1 =
    \emptyset$. %, thus $U_2^\Box \subseteq X\setminus U_1^\Diamond$.
    We omit the proof of `if'.
 }
 \takeout{   
    \item Recall that a topological space is $T_1$ if for any two distinct points
    $x$ and $y$, there is an open set containing $x$ but not $y$.
    Equivalently, all singletons are closed.
     If $X$ is $T_1$, then so is $\V X$.
     To see this, let $K_1$ and $K_2$ be distinct compact sets.
     Let us first assume that  $K_2 \setminus K_1 \neq \emptyset$.
     Let $x\in K_2 \setminus K_1$.  Since $K_1$ is compact,
     there is an open subset $U$ such that $K_1\subseteq U $ and $x\notin U$.
     Then $U^\Box$ contains $K_1$ but not $K_2$.
     Let $V$ be the open set $X\setminus\set{x}$.
     
     
     The second case is when $K_2\subseteq K_1$.  Let $x\in K_1\setminus K_2$,
     and let $U$ be the open set $X\setminus\set{x}$.
     Then 
}    
    
       
  \item\label{R:Vietoris:3} Vietoris~\cite{Vietoris22}
    originally proved that for a compact Hausdorff space $X$ (the
    classic Vietoris space) $\V X$ is compact Hausdorff, too. 



  \item\label{R:Vietoris:4}
  A \emph{Stone space} is a compact Hausdorff space 
    having a base of clopen sets.   
    If $X$ is a Stone
    space, so is $\V X$; 
see~\cite[Thm.~4.9.9]{Michael51} or~\cite[Section~III.4]{Johnstone86}.
  \end{enumerate}
\end{rem}
%
\begin{proposition}\label{prVnat}
  For every continuous function $f\colon X\to Y$ and every open
  $U\subseteq Y$, $(f^{-1}(U))^{\Diamond} = (\V f)^{-1}(U^{\Diamond})$,
  and  $(f^{-1}(U))^{\Box} = (\V f)^{-1}(U^{\Box})$.
\end{proposition}
%
\begin{proof}
  Let $R\in \V X$.   Observe that
  \[
    R \cap f^{-1}(U) \neq \emptyset
    \iff 
    f[R] \cap U \neq \emptyset
    \iff
    f[R]\in U^{\Diamond}
    \iff
    R\in (\V f)^{-1}( U^{\Diamond}).
  \]
  This proves our first assertion for all $R$. For the second assertion, we have
  \[
    R \subseteq f^{-1}(U)
    \iff 
    f[R] \subseteq U
    \iff
    f[R]\in U^{\Box}
    \iff
    R\in (\V f)^{-1}(U^{\Box}).
    \tag*{\qedhere}
  \]
\end{proof}
%
\begin{corollary}\label{V-functor}
  The mappings $X \mapsto \V X$ and $f \mapsto \V f$ form a
 functor~$\V$ on~$\Top$.
\end{corollary}
%
\noindent
Indeed, \autoref{prVnat} shows that
for every subbasic open set of $\V Y$ its inverse image under~$\V f$
is open in $\V X$.  This establishes continuity of $\V f$. 
%
\begin{notation}\label{N:V}
  We denote by $\Haus$, $\Khaus$ and $\Stone$ the full subcategories
  of $\Top$ given by all Hausdorff spaces, all compact Hausdorff spaces and
  all Stone spaces, respectively. By
  \autoref{R:Vietoris}.\ref{R:Vietoris:2}--\ref{R:Vietoris:4}, $\V$
  restricts to these three full subcategories, and we denote the
  restrictions by $\V$ as well.
\end{notation}
%
\begin{rem}\label{R:Vietoris-tech}
  \begin{enumerate}     
  \item\label{R:Vietoris-tech:2} The full subcategories $\Haus$,
    $\KHaus$ and $\Stone$ are closed under limits in $\Top$.
    In particular, the inclusion functors preserve and reflect
    limits. In fact, $\KHaus$ is a full reflective
    subcategory: the reflection of a space is its Stone-\v{C}ech
    compactification.

  \item\label{R:Vietoris-tech:3} If an $\omega^\opp$-chain
    as in~\eqref{Xsequence} consists of surjective continuous maps
    between compact Hausdorff spaces, then each limit projection 
    $\ell_n\colon \lim_{k<\omega} X_k \to X_n$ is surjective, too.  
    \takeout{To see this, let
    $a\in X_n$.  Choose a sequence $(x_i)$ so that for $i\leq n$,
    $x_i$ is the image of $a$ in $X_i$, and use recursion so that for
    $i > n$, $x_{i+1}$ is any preimage of $x_i$ under $f_i$.}
    Moreover, Eilenberg and Steenrod~\cite[Cor.~3.9]{ES:1952} prove
    the surjectivity of projections for all codirected limits of
    surjections between compact Hausdorff spaces; see also Ribes and
    Zalesskii~\cite[Prop.~1.1.10]{RibesZ10}).
    % 
    \takeout{No need to argue this way. SM: Yes, there is!
      Because we'd like to make Rem.~4.9(1). Because this is what
      other people prove in the literature; they are not so interested
      in $\omega^\opp$-chains but codirected limits.}

  \item\label{R:Vietoris-tech:4} If $X$ has a base
    $\mathcal B$ which is closed under finite unions, then the sets
    $U^\Diamond$ and $U^\Box$ for $U \in \mathcal B$ already form a
    subbase of $\V X$.
    %\smnote{This is~\cite[Lemma~3.13(2)]{HNN19}; not sure we need to
    % cite this.}
    Indeed, given a set $\mathcal S$ of open subsets of
    $X$ we have
    $(\bigcup \mathcal S)^\Diamond = \bigcup \set{ U^\Diamond : U \in
      \mathcal S}$. Moreover, it is easy to see that
    \[\textstyle
      (\bigcup \mathcal S)^\Box = \bigcup \set{\big( \bigcup \mathcal
        F\big)^\Box : \text{$\mathcal F \subseteq \mathcal S$ finite} };
    \]
    ``$\supseteq$'' is trivial, and for ``$\subseteq$'' use
    compactness of $R \in \V X$.  Hence, if $\mathcal S$ consists of
    basic open sets from $\mathcal B$, then $\bigcup F \in \mathcal B$ due to
    its closure under finite unions. Thus, $(\bigcup \mathcal S)^\Box$
    is a union of sets of the form $U^\Box$ for $U \in \mathcal B$.
  \end{enumerate}
\end{rem}



\takeout{
\begin{lemma}\label{lemma-needed}
  Suppose that~\eqref{Xsequence} is an $\omega^\opp$-chain in $\Haus$,
  and let $W\in \V L$.  Then $W$ contains every $x\in L$ with the
  property that for all $n$, $\ell_n(x)\in \V \ell_n(W)$.
\end{lemma}
%
\begin{proof}
  Since $W$ is a compact subset of the Hausdorff space $L$, we know
  that $W$ is closed. Hence, it is sufficient to prove that every open
  neighborhood of $x$ meets $W$.  We may restrict ourselves to the
  neighborhoods $\ell_n^{-1}(U)$ where $U$ is open in $X_n$ (since
  they form a base).  Thus $\ell_n(x)\in U$.  From
  $\ell_n(x)\in \V \ell_n(W) = \ell_n[W]$, we get some $y\in W$ with
  $\ell_n(y) = \ell_n(x)\in U$.  Thus $y\in W\cap\ell_n^{-1}(U)$.
\end{proof}
}

%\removeThmBraces
\begin{proposition}
  \label{P:limit-Haus}
  The functor $\V\colon \Haus \to \Haus$ preserves limits of
  $\omega^{\opp}$-chains.
\end{proposition}
%\resetCurThmBraces
%
\begin{proof}
  Consider an $\omega^\opp$-chain as in~\eqref{Xsequence}. Let
  $M = \lim \V X_n$, with limit cone $r_n \colon M \to \V X_n$.  Let
  $m\colon \V L \to M$ be the unique continuous map such that
  $\V \ell_n = r_n \o m$ for all $n < \omega$. We shall prove that $m$
  is a bijection and then that its inverse is continuous, which proves
  that $m$ is an isomorphism.
  \begin{enumerate}
  \item\label{P:limit-Haus:1} Injectivity of $m$ follows from the
    fact that $\V\ell_n$ ($n < \omega$) forms a jointly monic family,
    as we will now prove. Suppose that $A, B \in \V L$ satisfy
    $\ell_n[A ] = \ell_n[B]$ for every $n < \omega$. We prove that $A
    \subseteq B$; by symmetry $A = B$ follows. Given $a \in A$, we show that every
    open neighbourhood of $a$ has a nonempty intersection with
    $B$. Since $B$ is closed, we then have $a \in B$ (otherwise $L
    \setminus B$ would be an open neighbourhood of $a$ disjoint from
    $B$). It suffices to prove the desired property for the basic open
    neighbourhoods $\ell_n^{-1}(U)$ of $a$, for~$U$ open in $X_n$ (see
    \autoref{P:chain}.\ref{R:Vietoris-tech:1}). Since
    $\ell_n[A] = \ell_n[B]$ we have some $b \in B$ which satisfies $\ell_n(a) =
    \ell_n(b)$. Then $b \in \ell_n^{-1}(U) \cap B$. 

  \item Surjectivity of $m$.  An element of $M$ is a sequence
    $(K_n)_{n < \omega}$ of compact (hence closed) subsets
    $K_n \subseteq X_n$ such that $f_n[K_{n+1}] = K_n$ for every
    $n < \omega$. We need to find a compact set $K \subseteq L$ such
    that $\ell_n[K] = K_n$ for every $n < \omega$.  With the subspace
    topology, $K_n$ is itself a compact space.  The connecting maps
    $f_n\colon X_{n+1} \to X_n$ restrict to surjective continuous maps
    $K_{n+1} \epito K_n$.
    Thus, the%
    \smnote{To me `the' sound more correct then `all'; @Larry?
    LM: I know that Jirka prefers 'all' for constructions like this.  I prefer `the'.
    I would prefer 'the spaces' even more.}
    $K_n$ form an $\omega^\opp$-chain
    of surjections in $\KHaus$. Let $K$ be the limit with
    projections $p_n \colon K \epito K_n$.  Then $K$ is a subset of
    $L$, and each projection $p_n$ is the restriction of $\ell_n$ to
    $X_n$.

    Let us check that the topology on $K$ is the subspace topology
    inherited from $L$.
    %; this will show that $K$ is an element of $\V L$, and we are done.  
    A base of the topology on $K$ is the
    family of sets $p_n^{-1}(U)$ as $U$ ranges over the open subset
    of $K_n$.  Each $U$ is of the form $V\cap K_n$ for some open
    $V$ of $X_n$, and $p_n^{-1}(U) = \ell_n^{-1}(V)\cap K$. Thus
    $p_n^{-1}(U)$ is open in the subspace topology, and the converse
    holds as well.
    %We see that the topology on $K$ is the subspace topology inherited from $L$, as desired.

    The maps $p_n$ are surjective by
    \autoref{R:Vietoris-tech}.\ref{R:Vietoris-tech:3}. Moreover, $K$
    is a compact space by
    \autoref{R:Vietoris-tech}.\ref{R:Vietoris-tech:2}.  Thus, $K$ is
    the desired compact set in $\V L$ such that $p_n[K] = K_n$ for all $n$.

    \takeout{ %% Not needed, I think.
      We conclude by showing that this space is a compact subset of
      $L$. Let $j\colon K \to L$ be the inclusion.  So $j$ is a
      monomorphism in $\Top$, and we check that $j$ is a subspace
      inclusion.  Let $z\colon Z\to K$ be a function in $\Set$ with the
      property that $j\o z$ is continuous; we show that $z$ is itself
      continuous.
      % Let us check that  each composite $p_n\o z$ is continuous.
      The inclusion $i_n\colon K_n\to X_n$ has the universal property of
      subspace inclusions, and $i_n \o (p_n \o z) = \ell_n \o (j \o
      z)$. This last morphism is continuous, and so each $p_n\o z$ is
      continuous.  The family of morphisms $p_n\o z$ is a cone in $\Top$
      and in $\Set$.  In $\Set$, $z$ itself is the morphism of cones.
      In $\Top$, we know that there is also a unique morphism of cones;
      call it $y\colon Z \to K$.  Since the forgetful functor
      $U\colon \Top\to\Set$ preserves limits, $Uy = z$.  Thus $z =y$ is
      continuous, as desired.
    }%    
    
  \item Finally, we prove that the inverse $k\colon M \to \V L$, say,
    of $m$ is continuous. We know that the sets $\ell_n^{-1}(U)$, for
    $U$ open in $X_n$, form a base of $L$. Moreover, this base is
    closed under finite unions. By
    \autoref{R:Vietoris-tech}.\ref{R:Vietoris-tech:4} and using
    \autoref{prVnat} we obtain that $\V L$ has a subbase given by the following
    sets
    \[
      (\V \ell_n)^{-1}(U^\Diamond) = (\ell_n^{-1}(U))^\Diamond
      \quad
      \text{and}
      \quad
      (\V \ell_n)^{-1}(U^\Box) = (\ell_n^{-1}(U))^\Box
      \quad
      \text{for $U$ open in $X_n$.}
    \]
    It suffices to show that the inverse images of these subbasic open
    sets of $\V L$ are open in $M$. For $\V\ell_n^{-1} (U^\Diamond)$
    with $U$ open in $X_n$ we use that $\V \ell_n \cdot k = r_n$
    clearly holds to obtain
    \[
      k^{-1}\big(\V\ell_n^{-1}(U^\Diamond) = r_n^{-1}(U^\Diamond),
    \]
    which is a basic open set of $M$ by
    \autoref{P:chain}.\ref{R:Vietoris-tech:1}. For the subbasic
    open sets $\V\ell_n^{-1}(U^\Box)$ the proof is similar.
    \qedhere
  \end{enumerate}
  \takeout{
  \smnote[inline]{Old proof starts here.}
  Consider a chain of surjections as in~\eqref{Xsequence}.  Let
  $M = \lim \V X_n$, with limit cone $r_n \colon M \to \V X_n$.  Let
  $m\colon \V L \to M$ be the unique continuous map such that for all
  $n$, $\V \ell_n = r_n \o m$.  Define $k\colon M \to \V L$ by
  \[
    k(y) = \set{x\in L : (\forall n) \, \ell_n(x) \in r_n(y)}.
  \]
  Notice that $k(y) = \bigcap_n (\ell_n)^{-1} (r_n(y))$.  So $k(y)$ is
  an intersection of closed sets, hence it is itself closed.  We shall
  prove that $k =m^{-1}$ and then that $k$ is continuous.

  For all $y\in M$ and $n\in\omega$,
  $\V \ell_n(k(y)) \subseteq r_n(y)$.  Since all of the maps $\ell_n$
  are surjective, the reverse inclusion also holds.  Hence
  $\V \ell_n \o k = r_n$.  It follows that
  $r_n \o m \o k = \V \ell_n\o k = r_n$.  Since the cone $(r_n)_n$ is
jointly monic, $m\o k = \id_M$.  To see that
  $k\o m = \id_{\V L}$, let $W \in\V L$.  Then from
  $\V \ell_n\o k = r_n$ we get
  \begin{align*}
    k(m(W)) &=  \set{x\in L : (\forall n)\, \ell_n(l) \in \V \ell_n(W)}  
    % & =  W
  \end{align*}
  This is $W$ by \autoref{lemma-needed}.

  Finally, since $m$ and $k$ are bijections,
  with $m$ continuous, $\V L$ compact, and $M$ Hausdorff, 
  it follows that $m$ is a homeomorphism.}% end takeout
\end{proof}
%
\begin{corollary}\label{C:limit-KHaus}
  The restrictions of $\V$ to $\KHaus$ and $\Stone$ preserve limits of
  $\omega^\opp$-chains. % of surjections. SM: not needed
\end{corollary}
%
Indeed, use \autoref{R:Vietoris-tech}.\ref{R:Vietoris-tech:2}.
%
\begin{rem}
A \emph{codirected limit} is the 
  limit of a diagram
    whose scheme is of the form $P^\opp$ for a directed poset $P$.
 \autoref{P:limit-Haus} and \autoref{C:limit-KHaus} hold more
    generally for codirected limits.     
The argument is the same.  This proves a result stated in 
    Zenor~\cite{zenor70}, but with an incomplete proof.
 \end{rem}



%
% CONTINUE EDITING HERE
%



The following definition is due to Kupke et al.~\cite{kkv} for Stone
spaces, whereas Hofmann et al.~\mbox{\cite[Def.~2.29]{HNN19}} use arbitrary
topological spaces, but they later essentially restrict constants to be
(compact) Hausdorff, stably compact or spectral spaces.
%
%\removeThmBraces
\begin{defn}%[\cite{kkv}]
  \label{D:Vietoris-poly}
  The \emph{Vietoris polynomial functors} are the endofunctors on
  $\Top$ built from the Vietoris functor $\V$, the constant functors,
  and the identity functor, using product, coproduct, and composition.
  Thus, the Vietoris polynomial functors are built according to the
  following grammar
  \[
    \textstyle
    F ::= \V \mid A \mid \Id \mid \prod_{i\in I} F_i \mid
    \coprod_{i\in I} F_i \mid FF,
  \]
  where $A$ ranges over all topological spaces and $I$ is an arbitrary
  index set.
\end{defn}




%\lmnote{Jirka recommends dropping the point ``The following result is new'' which I had included.  I guess we should check with Hofmann or coauthors to be sure that this is really right.  Our proof seems to be simpler than anything that's out there.  I do think that C:Stone (SM: ??) should be in their paper. LM: Yes, it's in their paper and I said that.  I still say that our result is new here and it's not crazy to say this.}
 
%
\begin{theorem}\label{T:Vietoris}
Let $F\colon \Top\to \Top$ be a Vietoris polynomial functor, and assume that 
all constants in $F$ are Hausdorff spaces.
Then $F$ has a terminal coalgebra obtained in $\omega$ steps, and $\nu
F = V_\omega$ is a Hausdorff space.
%\begin{enumerate}
%\item The set $V_{\omega}$ carries a terminal coalgebra structure, so $\nu F = V_{\omega}$.
%Moreover, 
%\item \label{part:KHaus}  If  all constants in $F$ are compact, then $\nu F$ is a compact Hausdorff space.
%\item \label{part:Stone} If all constants in $F$ are Stone spaces, then $\nu F$ is a Stone space.
%\end{enumerate}
\end{theorem}
%
\begin{proof}
  An easy induction on Vietoris polynomial functors $F$  
  shows that:
  \begin{enumerate}
  \item The functor $F$ has a restriction $F_0\colon \Haus \to \Haus$,
  \item The restriction $F_0$ preserves surjective maps; the most
    important step being for $\V$ itself, and this uses the fact when
    $f\colon X\to Y$ is continuous and $X$ and $Y$ are Hausdorff, the inverse
    images of compact sets are compact.
  \item  The functor $F_0$ preserves limits of
    $\omega^\opp$-chains; the most important step is done in
    \autoref{P:limit-Haus}.
  \end{enumerate}
  The terminal coalgebra result for $F_0$ follows from the fact which
  we have mentioned in
  \autoref{section-preliminaries}: $\nu F$ is
  the limit of the terminal-coalgebra $\omega^\opp$-chain $F_0^n 1$
  ($n < \omega$).  Since $\Haus$ is closed under limits in~$\Top$ and
  $F_0^n 1 = F^n 1$, the functor $F$ has the same terminal coalgebra
  $\nu F = \lim F^n 1$.
\end{proof}
%
\begin{corollary}
  Let $F\colon \Top\to \Top$ be a Vietoris polynomial functor, and
  assume that all constants in $F$ are Hausdorff spaces. Then $F$ has
  an initial algebra.
\end{corollary}
%
\noindent
This follows from \autoref{T:Vietoris},
\autoref{E:cms}.\ref{E:cms:set} and \autoref{T:initial}, since an easy
induction shows that $F$ preserves monomorphisms.
%
\begin{corollary}\label{C:KHaus}
  Let $F\colon \Top\to \Top$ be a Vietoris polynomial functor in which
  all constants are \emph{compact} Hausdorff spaces and only finite
  coproducts are used. Then the terminal coalgebra $\nu F$ is a
  compact Hausdorff space.
\end{corollary}
%
\begin{proof}
  The functor $F$ restricts to an endofunctor on $\KHaus$.%
  \smnote{Let us not introduce $F_0$ (and repeat the proof of 4.11);
    this is not clearer, just makes the notation more heavy and 
    it is absolutely not needed.}
  Thus, the terminal-coalgebra $\omega^\opp$-chain $F^n 1$ lies in
  $\KHaus$. Moreover, $\KHaus$ is closed under limits in $\Top$
  because it is a full reflective subcategory
  (\autoref{R:Vietoris-tech}.\ref{R:Vietoris-tech:2}). Thus,
  $\nu F = \lim_{n < \omega} F^n 1$ is compact Hausdorff.  \qedhere
\end{proof}

\begin{corollary}\label{C:Stone}
  Let $F\colon \Top\to \Top$ be a Vietoris polynomial functor
  in which all constants are Stone spaces and only finite
  coproducts are used.%
  \smnote{I reformulated this to settle a dispute about a comma, which
    should not be here in my opinion.}
  Then the terminal coalgebra $\nu F$ is a Stone
  space.
\end{corollary}
%
\noindent
The proof is similar.

\begin{rem}
  \renewcommand{\itemautorefname}{Item} \autoref{C:KHaus} essentially
  appears in work by Hofmann et al.~\cite[Thm.~3.42]{HNN19} (except
  for the convergence ordinal). \autoref{C:Stone} is due to Kupke et
  al.~\cite{kkv}.  Our proof using convergence of the
  terminal-coalgebra chain is different than the previous ones.
  \renewcommand{\itemautorefname}{item}%
\end{rem}

\begin{example}\label{ex:Vnu}
  The terminal coalgebra for $\V$ itself was identified by
  Abramsky~\cite{abramsky}.  By what we have shown, it is
  $V_{\omega} = \lim \V^n 1$. An easy induction on $n$ shows that
  $\V^n 1$ is $\powf^n 1$ with the discrete topology; the key point is
  that each set $\powf^n 1$ is finite. The topology was described in
  \autoref{P:chain}.\ref{R:Vietoris-tech:1}: it has as a base the sets
  $\partial_n^{-1}(U)$ as $U$ ranges over the subsets of $\powf^n 1$.
  By \autoref{T:Vietoris}, $\nu F$ is a Stone space.

  In \autoref{S:app-worrell}, we present a concrete description of
  $V_\omega$ as the set of compactly branching strongly extensional
  trees.%
  \smnote{Ok, I moved the description of the topology here; trees are
    not needed for this.}
\end{example}
%
% 
\begin{rem}
  Note that \autoref{T:Vietoris} also holds for Vietoris polynomial
  functors when we take $\Haus$ as our base category.
  Hofmann et al.~\cite{HNN19} consider other full subcategories of
  $\Top$, and they
  also study the completeness of the category of coalgebras for Vietoris polynomial
  functors $F$ (however, they restrict to using finite products and
  finite coproducts in their definition of Vietoris polynomial
  functors).
  %\begin{enumerate}
  %\item
  For a Vietoris polynomial functor $F$ on $\Haus$, the category of
  coalgebras is complete~\mbox{\cite[Cor.~3.41]{HNN19}}. Moreover,
  every subfunctor of $F$ has a terminal
  coalgebra~\mbox{\cite[Cor.~4.6]{HNN19}}.
  %\end{enumerate}
\end{rem}
% 
\begin{rem}\lmnote{This seems like a niche topic, and I propose
    dropping the remark. SM: It does not hurt us to discuss related
    work since we have the space; this makes our paper
    stronger. Readers who are not interested may skip this remark.}
  Hofmann et al.~\cite[Ex.~2.27(2)]{HNN19} also consider a related
  construction called the \emph{lower Vietoris space} of $X$. It is
  the set of all closed subsets of $X$ with the topology generated by
  all sets $U^\Diamond$, cf.~\eqref{eq:predlift}. This again yields a
  functor on $\Top$: a given continuous function is mapped to
  $A \mapsto \ol{f[A]}$, where $\ol{f[A]}$ denotes the closure
  of~$f[A]$. Furthermore, one has a corresponding notion of lower
  Vietoris polynomial functors. They prove that for such functors $F$
  on the category of stably compact spaces
  (defined in~\cite{HNN19}), $\Coalg F$ is
  complete~\cite[Thm.~3.35]{HNN19}. Furthermore, if a lower Vietoris
  polynomial functor $F$ on $\Top$ can be restricted to that category,
  then it has a terminal coalgebra obtained by finite iteration:
  $\nu F = V_\omega$~\cite[Thm.~3.36]{HNN19}. Similar results hold for
  the category of spectral spaces and spectral maps.
\end{rem}

 

\begin{rem}\label{R:notAcc}
  Let us mention a very general result which applies in many
  situations to deliver a terminal coalgebra: Makkai and Par\'e's
  Limit Theorem~\cite[Thm.~5.1.6]{makkaipare}.  It implies that every
  accessible endofunctor $F\colon\A\to\A$ on a locally presentable
  category has an initial algebra and a terminal coalgebra.  (Indeed,
  the theorem implies that the category of $F$-coalgebras is
  cocomplete.)  This result cannot be used here because $\Haus$ is not
  locally presentable:  it does not have
  a small set of objects that is colimit-dense~\cite[Prop.~8.2]{ahrt23}.
\end{rem}
%
\begin{oproblem}
  \begin{enumerate}
  \item\label{OP:1} Does every Vietoris polynomial functor on $\Top$ have a
    terminal coalgebra?
  \item Does every Vietoris polynomial functor on $\KHaus$ as in
    \autoref{C:KHaus}  have an initial algebra?
  \end{enumerate}
\end{oproblem}

\noindent
\renewcommand{\itemautorefname}{Item}%
\autoref{OP:1} above is equivalent to asking whether the result that $\nu F$ exists
for every Vietoris polynomial functor would remain true if we allowed
non-Hausdorff constants.
\renewcommand{\itemautorefname}{item}%

\section{Hausdorff Polynomial Functors}
\label{S:Hausdorff}


Analogously to the Vietoris polynomial functors on $\Top$, we
introduce Hausdorff polynomial functors on $\MS$.  Closer to the
situation of Kripke polynomial functors on $\Set$ than to Vietoris
polynomial functors on $\Top$, the Hausdorff polynomial functors on $\MS$ have
terminal coalgebras obtained in $\omega+\omega$ steps.



\begin{notation}\label{N:Haus}
  The Hausdorff functor $\H\colon \Met \to \Met$ maps a metric space
  $X$ to the space $\H X$ of all compact subsets of $X$
  equipped with the Hausdorff distance\footnote{The definition goes
    back to Pompeiu~\cite{Pompeiu05} and was popularized by
    Hausdorff~\cite[p.~293]{Hausdorff14}.} given by
  \[
    \bar{d}(S,T)
    =
    \max \left(\sup\nolimits_{x \in S} d(x,T), \sup\nolimits_{y \in T}d(y,S)\right),
    \qquad\text{for $S, T \subseteq X$ compact},
  \]
  where $d(x,S) = \inf_{y \in S} d(x,y)$. In particular
  $\bar d(\emptyset, T) = \infty$ for nonempty compact sets $T$. For
  a non-expanding map $f\colon X \to Y$ we have $\H f\colon S \mapsto f[S]$.
\end{notation}
%

\begin{rem}
  \begin{enumerate}
  \item The functors $\V\colon\Top\to\Top$ and $\H\colon \Met \to \Met$ are
    closely related: for compact metric spaces $X$ the Vietoris space
    $\V X$ is precisely the topological space induced by the Hausdorff
    space $\H X$.
    
  \item Some authors define $\H X$ to consist of all \emph{nonempty}
    compact subsets of $X$. However, Hausdorff~\cite{Hausdorff14} did not exclude
    $\emptyset$, and the above formula works (as already indicated)
    without such an exclusion.
  \end{enumerate}
\end{rem}
%


\begin{rem}\label{R:Haus}
  \begin{enumerate}
  \item For a complete metric space, $\H X$ is complete again (see
    e.g.~Barnsley~\mbox{\cite[Thm.~7.1]{Barnsley93})}.  Thus $\H$
    restricts to a functor on the category $\CMS$ of complete metric
    spaces, which we denote by the same symbol $\H$.

  \item Let $\UMet$ denote the category of (extended) ultrametric
    spaces: the full subcategory of $\MS$ given by spaces satisfying
    the following stronger version of the triangle inequality:
    \[
      d(x,z) \leq \max\set{d(x,y),d(y,z)}.
    \]
    If $X$ is an ultrametric space, then so is $\H X$.  To see this,
    let $S, T, U\in \H X$.  Write $p$ for
    $\max\set{\bar{d}(S,T), \bar{d}(T,U)}$.  For each $x\in S$, there
    is some $y\in T$ such that $d(x,y)\leq \bar{d}(S,T)$.  For this
    $y$, there is some $z\in U$ such that $d(y,z)\leq \bar{d}(T,U)$.
    So
    \[
      d(x,z) \leq \max\set{d(x,y), d(y,z)} \leq \max\set{\bar{d}(S,T),
        \bar{d}(T,U)} = p.
    \]
    It follows that $d(x,U) \leq p$.  This for all $x\in X$ shows that
    $d(S,U) \leq p$.  Note that
    $p = \max\set{\bar{d}(U,T), \bar{d}(T,S)}$.  The same argument
    shows that $\sup_{z\in U} d(z,S) \leq p$.  So we have
    $\bar{d}(S,U) \leq p$.  This proves the ultrametric inequality.

    We again denote the restriction of the Hausdorff functor to
    $\UMet$ is denoted by $\H$.

  \item For a discrete metric space $X$ (where all distances are $0$
    or $1$), $\H X$ is the discrete space formed by all 
    finite subsets of $X$.

  \item\label{R:Haus:3} For an arbitrary metric space $X$, the
    nonempty finite subsets of $X$ form a dense set in~$\H X$. Indeed,
    given a nonempty compact set $S \subseteq X$, for every $\eps > 0$, there
    exists a nonempty finite set $T \subseteq S$ such that  $S$ is covered
    by $\eps$-balls around the points in $T$. Therefore
    $d(x, T) \leq \eps$ for all $x \in S$, and we have $d(y,S) = 0$
    for all $y \in T$. This implies that $\bar d(S,T) \leq \eps$.
  \end{enumerate}
\end{rem}
%
\begin{example}\label{E:Haus-ter}
  \begin{enumerate}    
  \item\label{E:Haus-ter:2} For the Hausdorff functor, a terminal
    coalgebra is carried by the space of all finitely branching
    strongly extensional trees equipped with the discrete metric. This
    follows from the finite power-set functor $\powfin$ having its
    terminal coalgebra formed by those trees
    (\autoref{E:strong-ext}).%
    \smnote{I prefer to repeat what is needed; this way readers do not
      have to look back but may if they feel the need.}  Indeed, the
    terminal-coalgebra chain $V_i$ ($i \in \Ord$) for
    $\H$ is obtained by equipping the sets in the
    terminal-coalgebra chain for $\powfin$ with the discrete
    metric. Furthermore, since limits in $\Met$ (or $\CMS$) are
    set-based, we see that both chains converge in exactly
    $\omega + \omega$ steps.  Therefore
    $\nu\H = V_{\omega+ \omega}$.
  \end{enumerate}
\end{example}

It follows that, unlike the Vietoris functor, the Hausdorff functor
does not preserve limits of $\omega^\opp$-chains:%
\smnote{@Jirka: I disagree that we repeat something here; we state for
  the first time that $\H$ does not preserve those limits and then
  give a proof based on the above example (which does not state the
  fact about $\H$ we prove here).} %
the terminal-coalgebras chain for
$\H(-)$ does not converge before $\omega + \omega$
steps (see \autoref{E:Haus-ter}.\ref{E:Haus-ter:2}). Thus this functor
does not preserve the limit $V_\omega = \lim_{n <\omega} V_n$. 

\begin{defn}
  Let $(X_n)_{n <\omega}$ be an $\omega^\opp$-chain in $\MS$. A cone
  $r_n\colon M \to X_n$ is \emph{isometric} if for all $x, y \in M$ we
  have $d(x,y) = \sup_{n \in \Nat} d (r_n(x), r_n(y))$.
\end{defn}

By \autoref{P:chain}.\ref{R:Haus:4}, limit cones of
$\omega^\opp$-chains in $\MS$ are isometric.


\begin{proposition}\label{P:Haus-limit}
  The Hausdorff functor preserves isometric cones of
  $\omega^\opp$-chains.
\end{proposition}
%
\begin{proof}
  Let $(X_n)_{n < \omega}$ be an $\omega^\opp$-chain with connecting
  maps $f_n\colon X_{n+1} \to X_n$. Given an isometric cone
  $\ell_n\colon M \to X_n$ ($n < \omega$), we prove that the cone
$\H \ell_n\colon \H M \to \H X_n$ is also isometric:
  \[
    \bar d(S, T) = \sup_{n < \omega} \bar d (\ell_n(S), \ell_n(T))
    \qquad
    \text{for all compact subset $S, T \subseteq M$.}
  \]
  We can assume that $S$ and $T$ are nonempty and finite: since finite sets are
  dense in $\H M$ by \autoref{R:Haus}.\ref{R:Haus:3}, and the maps
  $\ell_n$ are (non-expanding whence) continuous, the desired equality
  then holds for all pairs in $\H M$. The case where $S$ or $T$ is
  empty is trivial.

  Since every $\ell_n$ is non-expanding, we only need to prove that
  $\bar d(S,T) \leq c$ holds for
  $c = \sup_{n < \omega} \bar d(\ell_n[S], \ell_n[T])$.  For this,
we   show that for every $\eps > 0$, $\bar d(S,T) \leq c +
  \eps$. By the definition of the Hausdorff metric $\bar d$, it
  suffices to prove that for every $x \in S$ we have
  $d(x,T) \leq c + \eps$. By symmetry, we then also have
  $d(y, S) \leq c + \eps$ for every $y \in T$.

  Given $y \in T$ we have $d(x,y) = \sup_{n < \omega} d(\ell_n(x),
  \ell_n(y))$. Thus, there is a $k < \omega$ such that
  \[
    d(x,y) \leq d(\ell_k(x),\ell_k(y)) + \eps. 
  \]
  Since $T$ is finite, we can choose $k$ such that this inequality
  holds for all $y \in T$. By definition, % of $\bar d$,
  \[
   \bar d(\ell_k(x), \ell_k[T])
    =
    \inf_{y \in T} d(\ell_k(x), \ell_k(y))
    \qquad\text{in $X_k$}.
  \]
  Again using that  $T$ is finite, we can pick some $y \in T$ such that
  $d(\ell_k(x), \ell_k[T]) = d(\ell_k(x),\ell_k(y))$. With this $y$ we
  conclude that \\[5pt]
  % \begin{align*}
  \(\hspace*{1.5em}
  d(x,T)
  % &
  \leq d(x,y) \leq d(\ell_k(x), \ell_k(y)) + \eps
  % \\
  % &
  = d(\ell_k(x),
  \ell_k[T]) + \eps \leq \bar d(\ell_k[S], \ell_k[T]) + \eps
  % \\
  % &
  \leq c + \eps. %\tag*{\qedhere}
  \)
  % \end{align*}  
\end{proof}
%
\begin{rem}\label{R:Haus-pres}
  The Hausdorff functor preserves isometric embeddings and their
  intersections. Indeed, for every subspace $X$ of a metric space $Y$,
  a set $S \subseteq X$ is compact in $X$ iff it is so in
  $Y$. Moreover, given $S, T \in \H X$, their distances in $\H X$ and
  $\H Y$ are the same. Thus, $\H$ preserves isometric embeddings.

  Given a collection $X_i \subseteq Y$ ($i \in I$) of subspaces, a set
  $S \subseteq \bigcap_{i\in I} X_i$ is compact in $X$ iff it is so in
  $Y$ (and therefore in every $X_i$). Thus $\H$ preserves that
  intersection.
\end{rem}
%
\begin{defn}\label{D:Haus-poly}
  The \emph{Hausdorff polynomial functors} are the endofunctors on
  $\Met$ built from the Hausdorff functor, the constant functors, and
  the identity functor, using product, coproduct, and composition.
  Thus, the Hausdorff polynomial functors are built according to the following
  grammar (cf.~\autoref{D:Kripke}):
  \[
    \textstyle
    F ::= \H \mid A \mid \Id \mid \prod_{i\in I} F_i \mid
    \coprod_{i\in I} F_i \mid FF,
  \]
  where $A$ ranges over all metric spaces and $I$ is an arbitrary
  index set.
\end{defn}
%
\begin{theorem}\label{T:Haus-poly}
Every Hausdorff polynomial functor $F\colon \Met \to \Met$ has a 
  terminal coalgebra obtained in $\omega+\omega$ steps:
  $\nu F = V_{\omega +\omega}$.
\end{theorem}
%
\begin{proof}
  An easy induction over the structure of Hausdorff polynomial
  functors shows that each such functor $F$ preserves:
  \begin{enumerate}
  \item\label{T:Haus-poly:1}
    %for every $\omega^\opp$-chain $(X_n)$ in $\Met$ the canonical
    %morphism $F(\lim X_n) \to \lim FX_n$ is an isometric embedding,
    %and
 isometric cones of $\omega^\opp$-chains, and
  \item\label{T:Haus-poly:2}  isometric embeddings and their
    intersections.
  \end{enumerate}
  The most important step is done in \autoref{P:Haus-limit} and
  \autoref{R:Haus-pres}.%
    
  We conclude that in the terminal-coalgebra chain, the map
  $m\colon V_{\omega + 1} \to V_\omega$ from~\eqref{diag:m} in the
  Introduction is an isometric embedding by~\autoref{T:Haus-poly:1}.
  By \autoref{T:Haus-poly:2}, all of the maps $m, Fm, FFm, \ldots$ in
  the chain $(V_{\omega+n})_{n<\omega}$ are isometric
  embeddings. Hence $F$ preserves the intersection of the ensuing
  subspaces of $V_\omega$ viz.~the limit
  $V_{\omega+\omega} = \lim_{n <\omega} V_{\omega + n}$. Consequently,
  we have $\nu F = V_{\omega + \omega}$.
  %
  % 
  % 
  \takeout{ %% taken out - why do we need to formulate a new proof?
\begin{enumerate}
\item\label{T:Haus-poly:1} The functor $F$ preserves isometric embeddings and their intersections.
  Indeed, $\H$ does (Remark~???).%
  \smnote{TODO: make reference}
  Thus, the statement follows by an easy
  induction on Hausdorff polynomial functors.  the constant functors
  and the identity functor clearly preserve isometric embeddings and
  their intersections, and this property is stable under product,
  coproduct, and composition.

\item The functor $F$ preserves isometric cones of
  $\omega^{op}$-chains.  For $\H$, this is ?????.  The recursion is
  analogous to what we saw in~\ref{T:Haus-poly:1}.

\item The morphism $m\colon V_{\omega+1} \to V_\omega$ is an
  isometric embedding.  This follows from the fact that the limit cone
  $\ell_n\colon V_\omega \to F^n 1$ is isometric, and thus so is
  $F\ell_n: FV_\omega \to F^{n+1} 1$.  From
  $\ell_{n+1} \o m = F\ell_n$, we conclude that $j$ is an isometric
  embedding.

\item The rest of the proof is completely analogous to
  items~\ref{P:Kripke:2} and~\ref{P:Kripke:3} in the proof of
  \autoref{T:Kripke}.
  \qedhere
\end{enumerate}
}% end takeout
\end{proof}
%
\begin{rem}\label{R:Hausdorff-poly-cms}
  Note that if a Hausdorff polynomial functor $F$ uses only contants
  given by complete metric spaces $A$,
  then it has a restriction to an endofunctor on
  $\CMS$. Indeed, by an easy induction on the
  structure of $F$ one shows that $FX$ is complete whenever~$X$ is
  complete. Similarly, when $F$ uses constants which are ultrametric
  spaces, then $F$ has a restriction on $\UMet$. 
\end{rem}

Since $\CMS$ and $\UMet$ are closed under limits of
$\omega^\opp$-chains in $\Met$, we obtain the following

\begin{corollary}
  Every Hausdorff polynomial functor on $\CMet$ or $\UMet$ has a
  terminal coalgebra obtained in $\omega+\omega$ steps.
\end{corollary}
%
\begin{corollary}\label{C:HP-I}
  Every Hausdorff polynomial functor $F$ on $\Met$ or $\CMS$ has an
  initial algebra.
  % Moreover, $F$ is a varietor.
\end{corollary}
%
\noindent
Indeed, since Hausdorff polynomial functors preserve isometric embeddings, 
this follows from~\autoref{T:Haus-poly},
\autoref{E:cms}.\ref{E:cms:met}, and \autoref{T:initial}.
%
% SM: sentence proving that $F$ is a covarietor.
%
%For the second statement note that if $F$ is a
%Hausdorff polynomial functor, then so is every functor $F(-) + A$.

\begin{rem}\label{MP2}
  We mentioned another possible approach to terminal coalgebras in
  \autoref{R:notAcc}.  Let us comment on the situation regarding the
  results on $\Met$ here.  The category~$\Met$ is locally presentable
  (see e.g.~\cite[Ex.~2.3]{AdamekR22}).  The Limit Theorem does
  imply that on $\MS$, the Hausdorff polynomial functors have terminal
  coalgebras.  In more detail, the Hausdorff functor is finitary: this
  was proved for its restriction to $1$-bounded metric
  spaces~\cite[Sec.~3]{AdamekEA15}, and the proof for $\H$ itself is
  the same. An easy induction then shows that every Hausdorff
  polynomial functor is accessible, so that the Limit Theorem can be
  applied.  However, our elementary proof shows that the terminal
  coalgebra chain converges in $\omega + \omega$ steps.  The proof of
  Makkai and Par\'e's Limit Theorem does not yield such a bound.
\end{rem}


\section{Variation: the Closed Subset Functor on $\Met$}

We have been concerned with the Hausdorff functor taking a metric
space $M$ to the space of its nonempty compact subsets.  For two
variations, let us consider $\powcl\colon \MS\to\MS$ taking $M$ to the
set of its \emph{closed} subsets, and its subfunctor
$\powcl'\colon \MS \to \MS$ taking $M$ to the set of its nonempty
closed subsets.  Both $\powcl M$ and $\powcl' M$ are given the
Hausdorff metric. For a non-expanding map $f\colon X \to Y$, the
non-expanding map $\powcl f\colon \powcl X \to \powcl Y$ sends a
closed subset $S$ of $X$ to the closure of $f[S]$.  This makes
$\powcl$ and $\powcl'$ functors.  Due to the empty set, $\powcl$ is a
closer analog of $\H$.%
\smnote{The proposed mention of $\powfin$ seems off to me; I vote
  against it.}  It is natural to ask whether the positive results of
Section~\ref{S:Hausdorff} hold for these functors $\powcl$ and
$\powcl'$.  As proved by van Breugel~\cite[Prop.~8]{vanBreugel}, the
functor $\powcl$ has no terminal coalgebra.  Turning to $\powcl'$,
this functor has an initial algebra given by the empty metric space
and a terminal coalgebra carried by a singleton metric space. But
$\powcl'$ has no other fixed points (see van Breugel et
al.~\cite[Cor.~5]{vBCW04}), where an object $X$ is a \emph{fixed
  point} of an endofunctor $F$ if $FX \cong X$. We provide below a
different, shorter proof.
%
\begin{rem}\label{R:ubd-below-i}
\begin{enumerate}
\item A subset $X$ of a metric space is \emph{$\delta$-discrete} if
whenever $x\neq y$ are elements of $X$, $d(x,y) \geq \delta$.
Every subset of a $\delta$-discrete set is
$\delta$-discrete, and every such set is closed.  Moreover, if
$C$ and $D$ are different subsets of a $\delta$-discrete set,
then $\bar d(C,D)\geq \delta$.
\item
  A subset $S$ of an ordinal $i$ is \emph{cofinal} if for all $j<i$
  there is some $k\in S$ with $j \leq k < i$. If $S$ is not cofinal,
  then its complement $i\setminus S$ must be so. (But it is possible
  that both $S$ and $i\setminus S$ are cofinal in $i$.)
\end{enumerate}
\end{rem}

\removeThmBraces
\begin{theorem}\label{T:Pcl}
  There is no isometric embedding $\powcl' M \to M$ when $|M| \geq
  2$.
  \end{theorem}
\resetCurThmBraces
%
\begin{proof}%
  Suppose towards a contradiction that $\iota\colon \powcl M \to M$
  were an isometric embedding where $|M| \geq 2$.  If all distances in
  $M$ are $0$ or $\infty$, then $\powcl' M$ is the nonempty power set
  of $M$.  In this case, our result follows from the fact that for
  $|M|\geq 2$, $M$ has more nonempty subsets than
  elements.
  Thus we fix distinct points $a, b\in M$ of finite
  distance, and put $\delta = d(a,b)/2$.  Let
  $A = \set{x\in M : d(x,a) \leq \delta}$, and let $B = M\setminus A$.
  (In case $d(a,b) = \infty$, we need to adjust this by setting
  $\delta = \infty$, and $B$ to be the points whose distance to $a$ is
  finite.  But we shall not present the argument in this case.)

  % We will exhibit a strictly increasing sequence of subsets
  %$S_i \subseteq M$ indexed by all ordinal numbers $i$, a
  %contradiction. To this end
  We proceed to define an ordinal-indexed sequence of elements
  $x_i\in M$. We also prove that each set $S_i = \set{ x_j : j < i}$
  is $\delta$-discrete, and we put
  \[
    X_i =
    \begin{cases}
      A & \text{if $\set{j < i : x_j \in A}$ is cofinal in $i$} \\
      B & \text{else}.
    \end{cases}
  \]   
  For $i =0$, put $x_0 = \iota(\set{a,b})$.  Given an ordinal $i>0$,
  we put 
  \[
    x_i = \iota(X_i \cap S_i).
  \]
  Being nonempty (since $i>0$) and
  $\delta$-discrete, $X_i \cap S_i$ lies in~$\powcl' M$.
  %(Note also that when $i$ is a
  %successor ordinal, say $i = j+1$, then $m_i = A$ iff $x_j\in A$.)%
  %\smnote{We don't need this remark and so might want to drop it?}

  The remainder of our proof consists of showing that for every ordinal $i$:
  \[
    d(x_j, x_k) \geq \delta
    \qquad\text{for $0 \leq j < k \leq i$}.
  \]
  We proceed by transfinite induction.
  \takeout{  
  \begin{enumerate}
  \item\label{P:Pcl-claim:1} For $0 \leq j < k \leq i$, $d(x_j, x_k) \geq \delta$. 
    % Thus $S_i\cup\set{x_i}$ is $\delta$-discrete, and $x_i\notin S_i$.

  \item\label{P:Pcl-claim:2} %If $x_i \in A$,%
    %\smnote{Should this be ``If $m_i = A$ \dots''; otherwise I have a
    % problem with the proof below; same for the next item.}
    If $x_i \in A$,
    then for $1\leq j \leq i$, either $x_j= \iota(Z)$ for
    some $\emptyset\neq Z\subsetneq A\cap (S_j\cup\set{x_j})$
    or
    $x_j = \iota(Z)$ for some
    $\emptyset\subsetneq Z\subseteq B \cap S_j$.
    
  \item\label{P:Pcl-claim:3} %If $x_i \in B$,
    If $x_i \in B$, then for $1\leq j \leq i$, either $x_j= \iota(Z)$ for
    some  $\emptyset\neq Z\subsetneq B\cap (S_j\cup\set{x_j)}$ or
    $x_j = \iota(Z)$ for some 
    $\emptyset\neq Z\subseteq A\cap S_j$.
  \end{enumerate}}% end takeout
% 
  Assume that our claim holds for every $k < i$ and then prove it for
  $i$. The base case $i = 0$ is trivial. For $i > 0$, note first that it
  follows from the induction hypothesis that $S_i$ is $\delta$-discrete.

  Hence, we need only verify that $d(x_j, x_i) \geq \delta$ when
  $0\leq j < i$.  We argue the case $X_i = A$; when $X_i = B$, the
  argument is similar, mutatis mutandis.  For $j= 0$, recall that
  $x_0 = \iota(\set{a,b})$ and $x_i = \iota(A\cap S_i)$.  Since $b$ has
  distance at least $\delta$ from every element of $A$, we obtain
  $\bar d(\set{a,b}, A\cap S_i)\geq \delta$. As $\iota$ is an
  isometric embedding, this distance is also $d(x_0,x_i)$.  Now let $j >0$.
  Since we have $X_i = A$, let $k$ be such that $j \leq k < i$ and $x_k\in A$.
  Now either $x_j = \iota(A \cap S_j)$ or else
  $x_j = \iota(B \cap S_j)$.

  In the first case, note that $x_k\in A\cap S_i$ since $k < i$, and
  $x_k\notin S_j$ by the definition of $S_j$ since $k \geq j$.  So
  $A\cap S_j$ and $A\cap S_i$ are different nonempty subsets of the
  $\delta$-discrete set $S_i$. Hence, the distance between these sets
  is at least $\delta$, and therefore we have
  $d(x_j, x_i) \geq \delta$.
 
  In the second case, $B\cap S_j$ is a nonempty subset of $B$, and
  thus again it not equal to $A\cap S_i$.  So again we see that
  $d(x_j, x_i) = \bar d( B\cap S_j, A\cap S_i) \geq \delta$.

  We now obtain the desired contradiction since $(x_i)$ is an
  ordinal-indexed sequence of pairwise distinct elements of $M$.
\end{proof}

\begin{corollary}\label{C:Pcl}
  \begin{enumerate}
  \item The functor $\powcl'\colon \MS\to \MS$ has no fixed points
    except the empty set and the singletons.
  \item\label{C:Pcl:2} The functor $\powcl\colon \MS \to \MS$ admits
    no isometric embedding $\powcl M \to M$, whence has no fixed point.
  \end{enumerate}
\end{corollary}
%
\begin{proof}
  The first item is immediate from \autoref{T:Pcl}. For the second
  one, observe that the inclusion map $e\colon \powcl' M \to \powcl M$
  is an isometric embedding. Assuming that there were an isometric
  embedding $\iota\colon \powcl M \to M$, we see that $M$ cannot be
  empty (since $\powcl M$ is nonempty) or a singleton (since then
  $|\powcl M| = 2$). Hence $|M| \geq 2$. Moreover, we obtain an
  isometric embedding $\iota \cdot e\colon \powcl' M \to M$,
  contradicting \autoref{T:Pcl}.%
\end{proof}



\section{Summary}

We have investigated versions of the finite power set functor for the
categories $\Haus$ and $\Met$.  Our main results are that the Vietoris
functor $\V$, and indeed all Vietoris polynomial functors, have
terminal coalgebras obtained in $\omega$ steps of the
terminal-coalgebra chain.  The same holds for the Hausdorff polynomial
functors on $\Met$, but the iteration takes $\omega+\omega$ steps and
so the underlying reasons are different.

Our work on the Kripke and Hausdorff polynomial functors highlights a
technique which we feel could be of wider interest.  To prove that a
terminal coalgebra exists in a situation where the limit of the
$\omega^\opp$-chain~\eqref{eq:oop-chain-3} is not preserved by the
functor, one could try to find preservation properties which imply
that the limit of the $\omega^\opp$-chain $(V_{\omega+n})_n$ was
preserved.  In $\Set$, we used finitarity and preservation of
monomorphisms and intersections, and in $\Met$ we used preservation of
intersections, isometric embeddings, and isometric cones.

We have also seen that for the functor $\powcl$ on $\Met$, there is no
fixed point and hence no terminal coalgebra.  We leave open the
question of whether every Vietoris polynomial functor on $\Top$ has a
terminal coalgebra.


\mysubsec{Acknowledgement} We are grateful to Pedro Nora for
discussions on the proof of
\renewcommand{\propositionautorefname}{Prop.}%
\autoref{P:limit-Haus}.
\renewcommand{\propositionautorefname}{Proposition}%


%
% Appendix (if needed)
%
%\clearpage
\appendix

\section{Trees and the Limit of the Terminal-Coalgebra Chain for $\powf$}
\label{S:app-worrell}

We give the description of $V_{\omega}$
for $\powfin$ due to Worrell~\cite{worrell:05}. 
%The result is
%interesting in its own right, and it will also be used in our work on
%the Vietoris functor in \autoref{dcb}.  
We provide a full
exposition to the results which Worrell stated without proof.%


A \emph{tree} is a directed graph $t$ with a distinguished node
$\root(t)$ from which every other node can be reached by a unique
directed path.
% We often write $x\to y$ to mean that $y$ is a child of $x$ in $t$.
Every tree in our sense must have a root, so there is no
empty tree.  All of our trees are \emph{unordered}.  We always
identify isomorphic trees.

 \takeout{ 
We use trees to represent elements of $\powf^n 1= V_n$. 
We represent $V_0$ as $\set{t}$, where $t$ is the one-point tree (with a root).
Given $V_n$ represented by trees, we represent $V_{n+1} = \powf(V_n)$ by using
tree-tupling: a set $\set{t_1, \ldots, t_k}$ represents the tree with maximum subtrees
$t_1,\ldots, t_k$.   



The \emph{depth} of a node of a tree is its distance from the root.  

}



%
\removeThmBraces
\begin{defn}
 \label{D-SE}
  \begin{enumerate}
  \item We use the notation $t_x$ for the subtree of $t$ rooted in the
    node $x$ of $t$.
    \item A tree $t$ is \emph{extensional} if for every node $x$
      distinct children $y$ and $z$ of $x$ give different (that is,
      non-isomorphic) subtrees $t_y$ and $t_z$.
  %  \item Let $T_n$ be the set of trees of height at most $n$.

  \item A \emph{graph bisimulation} between two trees $t$ and $u$ is a
    relation between the nodes of $t$ and the nodes of $u$ with the
    property that  whenever $x$ and $y$ are related: (a) every child of $x$
    is related to some child of $y$, and (b) every child of $y$ is
    related to some child of $x$.
    
  \item A \emph{tree bisimulation} between two trees $t$ and $u$ is a
    graph bisimulation $R$ with the additional properties that that
    \begin{enumerate}
    \item The nodes $\root(t)$ and $\root(u)$
    are related; the roots are not
      related to other nodes;
      and
    \item  whenever two nodes are related, their parents are also related.
    \end{enumerate}
    
  \item Two trees are \emph{tree bisimilar} if there is a tree
    bisimulation between them.

  \item A tree $t$ is \emph{strongly extensional}
    if every tree bisimulation on it is a
    subset of the diagonal
    \[
      \Delta_t = \set{(x,x): x\in t}.
    \]
    In other words, $t$ is strongly extensional iff distinct children
    $x$ and $y$ of the same node define subtrees $t_x$ and $t_y$ which
    are \emph{not} tree bisimilar.
%\item For every tree $t$, we form its extensional quotient by successively merging every pair of
%sibling nodes (nodes with the same parent) $x$ and $y$ such that $t_x = t_y$.
%\item
%We denote by $\partial_n t$ the extensional quotient of the cutting of $t$ at level $n$.

  \end{enumerate}
\end{defn}
\resetCurThmBraces




\begin{rem}
 \label{rem:strongext}
  \begin{enumerate}
\takeout{  \item Every strongly extensional tree is extensional.
  \item\label{rem:strongext:2} Consider the following trees (where the right-hand one has $n$
    children for the $n$-th vertex in the breadth-first search):
    \begin{equation}
      \label{eq:exttrees}
      \vcenter{
        \xy
        \POS (000,000) *{\bullet} = "b1"
        ,    (000,-05) *{\bullet} = "b2"
        ,    (000,-10) *{\bullet} = "b3"
        ,    (000,-15) *{\bullet} = "b4"
        ,    (000,-20) *+{\vdots} = "d"
        %%% 
        \ar@{-} "b1";"b2"
        \ar@{-} "b2";"b3"
        \ar@{-} "b3";"b4"
        \endxy
      }
      \qquad\qquad
      \vcenter{
        \xy
        \POS (000,000) *{\bullet} = "b1"
        ,    (000,-05) *{\bullet} = "b2"
        ,    (-10,-10) *{\bullet} = "b3"
        ,    (010,-10) *{\bullet} = "b4"
        ,    (-15,-15) *{\bullet} = "b5"
        ,    (-10,-15) *{\bullet} = "b6"
        ,    (-05,-15) *{\bullet} = "b7"
        ,    (004,-15) *{\bullet} = "b8"
        ,    (008,-15) *{\bullet} = "b9"
        ,    (012,-15) *{\bullet} = "ba"
        ,    (016,-15) *{\bullet} = "bb"
        ,    (-10,-20) *+{\vdots}
        ,    (010,-20) *+{\vdots}
        %%% 
        \ar@{-} "b1";"b2"
        \ar@{-} "b2";"b3"
        \ar@{-} "b2";"b4"
        \ar@{-} "b3";"b5"
        \ar@{-} "b3";"b6"
        \ar@{-} "b3";"b7"
        \ar@{-} "b4";"b8"
        \ar@{-} "b4";"b9"
        \ar@{-} "b4";"ba"
        \ar@{-} "b4";"bb"
        \endxy
      }
    \end{equation} 
    Both are extensional; the left-hand one is strongly extensional, the
    right-hand one is not, since the relation relating all nodes of the
    same depth is a tree bisimulation on it.
}% end takeout

\item  Every composition and every union of
  tree bisimulations is again a tree bisimulation. In addition, the
  opposite relation of every tree bisimulation is a tree bisimulation:
  if $R$ is a tree bisimulation from $t$ to $u$, then $R^{\opp}$ is a
  tree bisimulation from $u$ to $t$. Consequently, the largest tree
  bisimulation on every tree is an equivalence relation.
 
\takeout{   
\item The notion of tree bisimulation is different from the usual
  graph bisimulation. For example, the following tree
    \begin{equation}
      \label{tree-graph}
      \begin{tikzpicture}[baseline = (B.base),
        %x=25mm,y=5mm,
        scale=.4,
        dot/.style={circle, fill, minimum size=#1, inner sep = 0, outer sep =0},
        dot/.default=4pt % default radius of dots
        ]
        \tikzstyle{level 1}=[sibling distance=20mm]
        \node[dot] {}
        child{node(B)[dot] {}}
        child{
          node[dot] {}
          child{node[dot] {}}
        };
      \end{tikzpicture}
    \end{equation}
    is strongly extensional, but there is a graph bisimulation
    relating the two leaves.
}

  \item\label{rem:strongext:5} A subtree $s$ of a strongly extensional tree $t$ is strongly
    extensional. Indeed, 
    if $R$ is a 
    tree bisimulation on $s$,
    then $R\cup\Delta_t$ is a tree bisimulation on $t$. 
    Since $R\cup\Delta_t\subseteq \Delta_t$, we have $R\subseteq \Delta_s$.
  \end{enumerate}
\end{rem}

\takeout{
\begin{proposition}\label{P:finite}
  A finite tree is extensional iff it is strongly extensional.
 \label{prop-equiv-extensionality}
\end{proposition}
\index{tree!extensional}
 %
\begin{proof}
  `If' is clear. For `only if' let $t$ be an extensional finite tree,
  and let $R$ be a tree bisimulation on it.  We claim that if
  $x \mathbin{R} y$, then the corresponding subtrees $t_x$ and $t_y$
  are equal. First notice that every node of $t_x$ must be related by
  $R$ to some node of $t_y$ (to see this, use induction on the depth
  of nodes, i.e.~their distance from the root) and vice versa. Thus,
  $t_x$ and $t_y$ have the same height, $n$ say. We now prove
  $t_x = t_y$ by induction on $n$. For $n = 0$, the result is obvious
  because the nodes of height $0$ are leaves.  Assume our result for
  $n$, and let $x$ and $y$ be related by $R$ and of height $n+1$. Then
  by the induction hypothesis and extensionality of $t$, for every
  child $x'$ of $x$ there is a \emph{unique} child $y'$ of $y$ and
  $t_{x'} = t_{y'}$; and vice-versa. This implies that $t_x = t_y$.
  It now follows that $t$ is strongly extensional.
\end{proof}
}

\takeout{
\begin{defn}
  The \emph{strongly extensional quotient} $\ol t$ of a tree $t$
  is the quotient tree of~$t$ modulo its largest tree bisimulation. 
\end{defn}

\begin{lemma}
  For every tree $t$,  $\ol t$ is a tree.
\end{lemma}
%
\begin{proof}
Let $R$ be the largest tree bisimulation on $R$, and 
  let $[x]$ denote the equivalence class of $x$. If $x$ is the
  root of $t$, then $[x]$ is the root of $\ol t$. Moreover, the children
  of $[y]$ for a node $y$ of $t$ are the classes $[z]$ for all children $z$
  of $y$.   The main point is the verification that the quotient is a tree,
  and this uses the fact that if two nodes are related by a tree bisimulation, 
  then they have the same distance from the root.
\end{proof}
}% end takeout


%\smnote[inline]{I propose that we salvage A.6--A.8 from Jirka's (now
%  taken out) version of this appendix. Who will make a concrete proposal?}

\begin{lemma}
 \label{lem:w1} 
  If $t$ and $u$ are strongly extensional and related by a tree
  bisimulation, then we have $t = u$.
\end{lemma}
%
\begin{proof}
  Let $R$ be a tree bisimulation between $t$ and
  $u$.  By \autoref{rem:strongext}, $R^{\opp} \o R$ is a tree
  bisimulation on $t$, whence $R^{\opp} \o R \subseteq \Delta_t$ by
  strong extensionality.
But every node of $t$ is related to at least
  one node of $u$ (use induction on the depth of nodes) implying that $R^{\opp} \o R =
  \Delta_t$. Similarly, $R \o R^{\opp} = \Delta_u$. Thus, $R$ (is a
  function and it) is an isomorphism of trees, and we identify such
  trees.
\end{proof}



\begin{notation}\label{N:partial}
\begin{enumerate}
\item Let $\TT$ be the class of trees.  We define maps
  $\partial_n\colon \TT \to V_n= \powf^n 1$ as follows: $\partial_0$
  is the unique map to $1$, and given the map $\partial_n$ and a tree
  $t$, we put
  \[
    \partial_{n+1}(t) = \set{\partial_n(t_x ): \text{$x$ is a child of
        the root of $t$}}. 
  \]
  On the right we have a subset of $\powf^n 1$, and this is an element
  of $\powf^{n+1}1$.
  
\item The trees $t$ and $u$ are \emph{Barr equivalent} provided that
  $\partial_n t = \partial_n u$ for all $n$.  We write $t\approx u$ in
  this case.\lmnote{We only need this relation $\approx$ in one place
    in this appendix, in~\autoref{E-sat}.  We could do without
    $\approx$.  We also could do more with it if we want. SM: Since we
    only use it once, I vote for dropping $\approx$.  LM: On the other hand,
    if we add material to the later appendix that uses $\approx$ we need
    the definition someplace.  See~\autoref{T:compact}.}
  
\item For every tree $t$, we define maps
  $\treepartial^t_n \colon t \to V_n = \powf^n 1$ in the following
  way: $\treepartial^t_0$ is the unique map $t\to 1$, and for all
  nodes $x$ of $t$,
  $\treepartial^t_{n+1}(x) = \set{\treepartial^t_n(y): \mbox{$y$ is a
      child of $x$ in $t$}}$.  This family of maps $\treepartial^t_n$
  is a cone: we have
  $\treepartial^t_n = v_{m,n} \cdot \treepartial^t_m$ for every
  connecting map $v_{m,n}\colon \powfin^m 1 \to\powfin^n 1$,
  $m \geq n$. Hence, there is a unique map
  $\treepartial^t_{\omega}\colon t\to V_{\omega}$ such that
  $\ell_n\o\treepartial^t_\omega = \treepartial^t_n$ for all $n$.
\end{enumerate}
\end{notation}
%
\begin{rem}
  Note that $V_n = \powfin^n 1$ may be described as the set of all
  extensional trees of height at most $n$. Indeed, $1$ is described as
  the singleton set consisting of the root-only tree, and every finite
  set of extensional trees in $V_{n+1} = \powfin V_n$  is represented by
  the extensional tree obtained by tree-tupling the trees from the given set.
\end{rem}
% 
\begin{rem}\label{R:treepartial}
  \begin{enumerate}
  \item\label{R:treepartial:1} If
    $\treepartial^t_{n+1}(a) = \treepartial^t_{n+1}(b)$, then for all
    children~$a'$ of $a$, there is some child $b'$ of $b$ and
    $\treepartial^t_{n}(a') = \treepartial^t_{n}(b')$.  This is easy
    to see from the definition of $\treepartial^t_{n+1}$.%
  \item For all trees $t$,
    \( \treepartial^t_i(\root(t))= \partial_i(t).  \)
Furthermore, let $b\colon t \to \TT$ be given
    by $b(x) = t_x$.  Then
    $\treepartial^t_i = \partial_i\o b$.%
  \end{enumerate}
\end{rem}
%
\begin{defn} \label{dcb} Let $x_0, x_1, \ldots, $ be a sequence of
  nodes in a tree $t$, and let $y$ also be a node in $t$.  We write 
  \[
    \lim x_n = y
  \]
  to mean
  that for every $n$ there is some $m$ such that
  $\treepartial^t_n(x_p) = \treepartial^t_n(y)$ whenever $p \geq m$.

  A tree $t$ is \emph{compactly branching} if for all nodes $x$ of
  $t$, the set of children of $x$ is \emph{sequentially compact}:
  for every sequence of $(y_n)$ of children of $x$
  there is 
  a subsequence $(w_n)$ of $(y_n)$ and 
  some child $z$ of $x$ such that $\lim {w_n} = {z}$.
\end{defn}

\begin{example}\label{E-sat}
  The following tree $t$ is not compactly branching:
  \[
    t\colon \qquad
    \begin{tikzpicture}[
      x=25mm,scale=.4,
      dot/.style={circle, fill, minimum size=#1, inner sep = 0, outer sep =0},
      dot/.default=4pt, % default radius of dots
      baseline = (B.base),
      sibling distance=60
      ]
      \node[dot] {}
      child{node[dot,label=left:$y_0$] (B) {}}
      child{node[dot,label=left:$y_1$] {}
        child{node[dot] {}}
      }
      child{node[dot,label=right:$\quad\cdots$,label=left:$y_2$] {}
        child{node[dot] {}
          child{node[dot] {}}
        }
      };
    \end{tikzpicture}
    \]
    To see this, consider the sequence $y_0$, $y_1$, $\ldots$.
    % $t_{y_0}$, $t_{y_1}$, $\ldots$.
    Note that for $n \geq m$, 
    $\treepartial^{t}_n(y_n) = 
    \partial_i(t_{y_n}) =
     t_{y_m}$.   
 %   $\partial_n(t_{y_m}) = t_{y_m}$.  
 We
    claim that  for every subsequence 
    $(y_{k_n})$
    of this sequence $(y_n)$ 
    there is no $y_p$ such that $\lim_n  y_{k_n} = y_p$.
      To simplify the notation, we only verify this for
    the sequence $(y_n)$ itself.  It does not converge to any
    fixed element $y_m$ because for $p > m$,
    \[
      \treepartial^{t}_p(y_m)
      = 
      \partial_p(t_{y_m})
      \neq
      \partial_p(t_{y_p})
      =
      \treepartial^{t}_p(y_p).
    \]
 
    In contrast, the following tree is compactly branching (also observe also that
    $t \approx t'$):
    \[
      t'\colon \qquad
      \begin{tikzpicture}[
        x=25mm,scale=.4,
        dot/.style={circle, fill, minimum size=#1, inner sep = 0, outer sep =0},
        dot/.default=4pt, % default radius of dots
        baseline = (B.base),
        sibling distance=70
        ]
        \node[dot] {}
        child{node[dot,label=left:$z$] (B) {}
          child{node [dot] {}
            child{node[dot] {}
              child{node[dot,label=below:$\vdots$] {}
              }
            }
          }
        }
        child{node[dot,label=left:$y_0$] {}}
        child{node[dot,label=left:$y_1$] {}
          child{node[dot] {}}
        }
        child{node[dot,label=right:$\quad\cdots$,label=left:$y_2$] {}
          child{node[dot] {}
            child{node[dot] {}}
          }
        };
      \end{tikzpicture}
    \]
    To check the compactness, consider a sequence of children of the
    root, say $({x_n})$.  If there is an infinite subsequence which
    is constant, then of course that sequence converges.  If not, then
    there is a subsequence of $({x_n})$, say $({w_n})$, where each
    $w_n$ is $y_k$ for some $k\geq n$.  In this case,
    $\lim_n ({w_n}) =z$.  This is because for all but
    finitely many $n$,
    $\treepartial^t_n(z) =
    \partial_n(t_z) = t_{w_n} = \partial_n(t_{w_n}) = \treepartial^t_n(w_n) $.
\end{example}



\begin{lemma}\label{seqK-part}
  If $t$ and $u$ are compactly branching, and if
  $\treepartial^t_{\omega}(\root(t)) = \treepartial^u_{\omega}(\root(u)) $, then
  there is a tree bisimulation between $t$ and $u$ which includes
  \(
    \set{(x,y) \in t\times u \colon \treepartial^t_{\omega}(x) =
      \treepartial^u_{\omega}(y)}.
  \)
\end{lemma}

\begin{proof}
  Given compactly branching trees $t$ and $u$, we define a relation 
  $R\subseteq t\times u$ inductively by
  \begin{align*}
    x\mathbin{R} y \quad \mbox{iff} \quad &
    \text{(1)~$x = \root(t)$ and $y= \root(u)$, or $x$ and $y$ have $R$-related parents, and} \\
    &
    \text{(2)~$\treepartial^t_{\omega}(x) = \treepartial^u_{\omega}(y) $}.
  \end{align*}
  Let us check that $R$ is a tree bisimulation.  Suppose that $(x,y)$
  are related by $R$ as above, and let $x'$ be a child of $x$ in $t$.
  Using \autoref{R:treepartial}.\ref{R:treepartial:1} we see that for each
  $n$, there is some child $y'_n$ of $y$ in $u$ with
  $\treepartial^t_{n}(x') = \treepartial^u_{n}(y_n')$.  Consider the
  sequence $y'_0$, $y'_1$, $\ldots$.  Now
  $\treepartial^t_{n}(x') = \treepartial^u_{n}(y_m')$ if
  $m \geq n$, since $\treepartial^t_n$ and $\treepartial^u_n$ form
  cones:
  \(
    \treepartial^t_n(x')
    =
    v_{m,n} \cdot \treepartial^t_m(x')
    =
    v_{m,n} \cdot \treepartial^u_m(y_m')
    =
    \treepartial^u_n(y_m').
  \)
  By sequential compactness, there is a
  subsequence $z_0$, $z_1$, $\ldots$, and also some child $z^*$ of $y$
  such that $\lim z_n = z^*$.  Being a subsequence,
  $\treepartial^t_{n}(x') = \treepartial^u_{n}(z_m)$ whenever
  $m \geq n$.  Let us check that for all $n$,
  $\treepartial^t_n(x') = \treepartial^u_n(z^*)$.  To see this, fix
  $n$ and let $m\geq n$ be large enough so that for $p\geq m$,
  $\treepartial^u_n(z_p) = \treepartial^u_n(z^*)$.  Thus,
  $\treepartial^t_{n}(x') = \treepartial^u_{n}(z_m)
  =\treepartial^u_n(z^*) $.
  Thus, $\treepartial^t_\omega(x') = \treepartial^u_\omega(z^*)$,
  which shows $x' \mathbin{R} z^*$, as desired. 

  The other half of the verification that $R $ is a tree bisimulation is similar.
\end{proof}

\takeout{
\removeThmBraces
\begin{theorem}[\cite{worrell:05}] For the finite power set functor $\powf$,
\begin{enumerate}
\item $V_\omega$ is the set $T$ 
with projections $\partial_n\colon T \to T_n$ 
\item $(D,\delta)$ is the terminal coalgebra.
\end{enumerate}
\end{theorem}
\resetCurThmBraces


\begin{proof}
\begin{enumerate}
\item
It is clear that the maps $\partial_n$ constitute a cone.
For the limit property, let $t_n\in T_n$ be any sequence of trees
such that $f_n(t_{n+1}) = t_n$ for all $n$.  
Let $t$ be the disjoint union of the trees $t_n$, but identifying their roots.
The strongly extensional quotient $\ol t$ has the property that $\partial_n(\ol t) = t_n$ 
for all $n$.  For the uniqueness, let $u$ be a strongly extensional tree such that 
 $\partial_n(u) = t_n$ 
for all $n$.   Thus, $\ol t$ and $u$ are related by a tree
  bisimulation.  By \autoref{lem:w1}.\ref{lem:w1:2},  $\ol t = u$.
  
\item
We use \autoref{T:Kripke}.
An easy induction on $n$ shows that $V_{\omega+n}$ is the set of 
strongly extensional trees $t$ with the property that the topmost $n$ levels of $t$ are
finitely branching.   With this description, 
$V_{\omega+n} \subseteq D$, and indeed $D = \bigcap_n V_{\omega+n}$.
This shows that the carrier set of $\nu \powf$ is $D$.
\end{enumerate}  
\end{proof}
}% end takeout


\begin{notation}\label{N:tx}
  In this section, $V_{\omega}$ denotes the limit of
  \eqref{eq:oop-chain-3} for the finite power set functor.
  \begin{enumerate}
  \item We take the elements of $V_{\omega}$ to be compatible
    sequences $(x_n)$. That is, $x_n \in \powfin^n 1$ and
    $\powfin^{n}!(x_{n+1}) = x_n$ for every $n < \omega$. To save on
    notation, we write $x$ for $(x_n)$.  We consider the relation
    $\leadsto$ on $V_{\omega}$ defined by
    \begin{equation}\label{leadsto}
      x\leadsto y \quad\mbox{iff}\quad
      \mbox{for all $n$, $y_{n}\in x_{n+1}$}.
    \end{equation}

  \item Let $L^+$ be the set of nonempty finite sequences from
    $V_{\omega}$.  We write such a sequence with the notation
    $\pair{x^1, \ldots, x^n}$.  We consider the relation $\Rightarrow$
    on $L^+$ defined by
    \[
      \pair{x^1, \ldots, x^n} \Rightarrow \pair{y^1, \ldots, y^m}
      \quad\mbox{iff}\quad \mbox{$m = n+1$, $x^1 = y^1$, $\ldots$,
        $x^n = y^n$, and $x^n \leadsto y^{n+1}$}.
    \]
    In other words, $m = n+1$,
    $ \pair{y^1, \ldots, y^{m-1}} = \pair{x^1, \ldots, x^n}$, and
    $x^n \leadsto y^m$.

  \item For each $x\in V_{\omega}$, let $\temptree_x$%
    \smnote{This notation is used for something else already! We need
      to come up with a different notation here.  LM: Okay, I am
      trying.  The macro is ``temptree''. SM: I changed it to $\mathsf{tr}_x$.  LM: looks good!} %
    be the tree whose nodes are the sequences
    $\pair{x,x^2, \ldots, x^n} \in L^+$ whose first entry is $x$, with
    root the one-point sequence $\pair{x}$, and with graph relation
    the restriction of $\Rightarrow$.
For readers familiar with tree unfoldings of pointed graphs, $\temptree_x$ is the tree unfolding
of the graph $(V_\omega,\leadsto)$ at the point $x$.
  \item Finally, let
    \begin{equation}\label{eq:T}
      T = \set{\temptree_{x} : x\in V_{\omega}}.
    \end{equation}
  \end{enumerate}
\end{notation}

Recall the connecting maps $\powfin^n!\colon \powfin^{n+1} 1 \to
\powfin^n 1$.
%
\begin{lemma}\label{lemma-seqseq}
  Let $x\in V_{\omega}$.  
  \begin{enumerate}
  \item\label{lemma-seqseq:1} For all $k$ and all
    $\pair{x,x^2, \ldots, x^n}\in \temptree_x$,
    $\treepartial^{\temptree_x}_k(\pair{x,x^2, \ldots, x^n}) = x^n_k$.

  \item\label{lemma-seqseq:2} Let $R$ be a tree bisimulation on $\temptree_x$.
    If $\pair{x, x^2, \ldots, x^n}\ R\ \pair{x, y^2, \ldots, y^n}$, then
    for all $k$,
    \[
      \treepartial^{\temptree_x}_k(\pair{x, x^2, \ldots, x^n}) =
      \treepartial^{\temptree_x}_k(\pair{x, y^2, \ldots, y^n}).
    \]
    
  \item\label{lemma-seqseq:3} The tree $\temptree_x$ is strongly
    extensional and compactly branching, and
    $ \partial_\omega(\temptree_x) =
    \treepartial^{\temptree_x}_{\omega}(\pair{x}) = x$.
  \end{enumerate}
\end{lemma}
%
\begin{proof}
\begin{enumerate}
\item By induction on $k$.  For $k = 0$, our result is clear: the
  codomain of $\treepartial_k$ is $1$.  Assume our result for $k$, fix
  $x\in L^+$ and $\pair{x^1, \ldots, x^n}\in \temptree_x$. We first
  prove that
  \begin{equation}\label{eq:aux}
    \set{ y_k : x^n \leadsto y} = x^n_{k+1}.
  \end{equation}
  Indeed, if $x^n\leadsto y$, then
  $y_k\in x^n_{k+1}$. Conversely, if $a\in x^n_{k+1}$, we construct
  $y\in V_{\omega}$ such that $x^n\leadsto y$ with $y_k = a$.  Note
  that
  \[
    x^n_k = \powf^{k}!(x^n_{k+1}) = \pow\powf^{k-1}!(x^n_{k+1}) =
    \powf^{k-1}![x^n_{k+1}].
  \]
  Since $a\in x^n_{k+1}$,  we have $\powf^{k-1}!(a) \in x^n_k$.  So we let
  $y_{k-1} = \powf^{k-1}!(a)$.  We repeat this argument to define
  $y_{k-2}$, $\ldots$, $y_1$, $y_0$; the point is that
  $y_{k-i} \in x^n_{k-i+1}$ for $i = 0,\ldots, k$.  Choices are needed
  when we go the other way from $k$.  Note that
  \[
    \powf^{k+1}![x^n_{k+2}]
    =
    \powf(\powf^{k+1}!)(x^n_{k+2}) 
    =
    \powf^{k+2}!(x^n_{k+2}) = x^n_{k+1}.
  \]
  Every set functor preserves surjective functions, and so
  $\powf^{k+1}!$ is surjective.  Thus there is some
  $y_{k+1} \in x^n_{k+2}$ such that $\powf^{k+1}!(y_{k+1}) = y_k$.
  The same argument enables us to find by recursion on $i$ a sequence
  $y_{k+i + 1} \in x^n_{k+i+2}$ such that
  $\powf^{k+i+1}!(y_{k+i+1 }) = y_{k+i}$.  This defines $y$ such that
  $x^n \leadsto y$ according to~\eqref{leadsto}
  with $y_k = a$.  

  The induction step is now easy:
  \begin{align*}
    \treepartial^{\temptree_x}_{k+1}(\pair{x,x^2, \ldots, x^n})
    &= \set{\treepartial^{\temptree_x}_k(\pair{x,x^2, \ldots, x^n,y}): x^n \leadsto y}\\
    &= \set{ y_k : x^n \leadsto y} &\text{by induction hypothesis} \\
    & = x^n_{k+1} &\text{by~\eqref{eq:aux}}. 
  \end{align*}


\item This again is an induction on $k$, and the steps are similar to
  what we have just seen.  We also note that tuples in $\temptree_x$
  related by a tree bisimulation must have the same length.

\item Note first that by \autoref{lemma-seqseq:1} with $n = 1$, we have
  $\treepartial^{\temptree_x}_k(\pair{x}) = x_k$ for all $k$. This implies that
  $\treepartial^{\temptree_x}_{\omega}(\pair{x}) = x$.  For the strong extensionality,
  let $R$ be a tree bisimulation on $\temptree_x$.  Suppose that
  $\pair{x, x^2, \ldots, x^n}$ and $\pair{x, y^2, \ldots, y^n}$ are
  related by $R$.  Using
  \renewcommand{\itemautorefname}{items}%
  \autoref{lemma-seqseq:1} and~\ref{lemma-seqseq:2},
  \renewcommand{\itemautorefname}{item}%
  we see that for all $k$, we have $x^n_k = y^n_k$.  Thus $x^n = y^n$.  In
  addition, since $R$ is a tree bisimulation, the parents of the two
  nodes under consideration are also related by $R$.  So the same
  argument shows that $x^{n-1}= y^{n-1}$.  Continuing in this way
  shows that $x^{n-2}= y^{n-2}$, $\ldots$, $x^2 = y^2$.  Hence
  $\pair{x, x^2, \ldots, x^n}=\pair{x, y^2, \ldots, y^n}$.

  Finally, we verify that $\temptree_x$ is compactly branching.  To simplify
  the notation a little, we shall show this for children of the root
  $\pair{x}$.  So suppose we have an infinite sequence
  ${\pair{x,y^1}}$, ${\pair{x,y^2}}$, $\ldots$.  Recall that each set
  $\powf^n 1$ is finite.  By successively thinning the sequence $y^1$,
  $y^2$, $\ldots$, we may assume that for all $n\in \omega$ and all
  $p, q \geq n$, $y^n_p = y^n_q$.  Let $z\in V_{\omega}$ be the
  `diagonal' sequence $z_n = y^n_n$. Since every $\pair{x,y^n}$ is a
  child of the root $\pair{x}$ (in symbols:
  $\pair{x} \Rightarrow \pair{x,y^n}$), we have $x \leadsto y^n$. This
  implies that for all $n$, we have $z_n = y^n_n \in x_{n+1}$, whence
  $x\leadsto z$. Thus, $\pair{x,z}$ is a child of the root of
  $\temptree_x$. Recall from \autoref{lemma-seqseq:1} that
  $\treepartial^{\temptree_x}_n(\pair{x,z}) = z_n$.  So we obtain the desired conclusion:
  $\lim \pair{x,y^n} = \pair{x,z}$.\qedhere
\end{enumerate}
\end{proof}


\begin{lemma}\label{forBarr}
For every tree $t$ there is a Barr-equivalent tree
    $t^*\in T$ such that $t^*$ is strongly extensional and compactly
    branching.
\end{lemma}
%
\begin{proof}
  Given any tree $t$, we have $x = \partial_\omega(t)\in V_{\omega}$.
  For all $n$, $x_n = \partial_n(t)$.  The tree $t^* = \temptree_x$ in
  \autoref{lemma-seqseq}.\ref{lemma-seqseq:3} is strongly extensional
  and compactly branching.  Recall that the root of $t^*$ is
  $\pair{x}$.  By \autoref{lemma-seqseq}.\ref{lemma-seqseq:1}, we have
  that for all $n<\omega$,
  \[
    \partial_n(t^*)
    =
    \treepartial^{t^*}_n(\root(\temptree_x))
    =
    \treepartial^{t^*}_n(\pair{x})
    =
    x_n
    =
    \partial_n(t).
  \]
  This completes the proof.
  \qedhere
\end{proof}
%
\begin{lemma}
  The set $T$ defined in~\eqref{eq:T} is the set of all compactly
  branching, strongly extensional trees.
\end{lemma}
%
\begin{proof}
  By \autoref{lemma-seqseq}.\ref{lemma-seqseq:3} we know that every
  tree in $T$ is strongly extensional and compactly branching. For the
  reverse inclusion, let $t$ be compactly branching and strongly
  extensional.  Let $t^*$ be as in Lemma~\ref{forBarr} for $t$. By
  \renewcommand{\lemmaautorefname}{Lemmas}%
  \autoref{lem:w1} and~\ref{seqK-part},
  \renewcommand{\lemmaautorefname}{Lemma}%
  $t = t^*$.  Thus $t\in T$.
\end{proof}

\begin{defn}
  Let $D$ be the set of finitely branching strongly extensional trees.
  Let $\delta\colon D\to \powf D$ take a strongly extensional tree $t$
  to the (finite) set of its subtrees $t_x$.
\end{defn}

\noindent
In this definition, we use
\autoref{rem:strongext}.\ref{rem:strongext:5}: a subtree of a strongly
extensional tree is strongly extensional.%
\smnote{I prefer to have the sentence proposed by Larry rather than
  Jirkas proposal `(recalling \autoref{rem:strongext}.\ref{rem:strongext:5})'.}

\removeThmBraces
\begin{theorem}[\cite{worrell:05}] \label{T:Worrell}
  For the finite power set functor $\powf$ the following hold:
  \begin{enumerate}
  \item the maps $\partial_n\colon T \to \powf^n 1$ given by
    $\partial_n(\temptree_x) = x_n$ form a limit
    of~\eqref{diag:m}; thus, $V_\omega \cong T$,
  \item\label{T:Worrell:2} the coalgebra $(D,\delta)$ is terminal.
  \end{enumerate}
\end{theorem}
\resetCurThmBraces
%
\begin{proof}
\begin{enumerate}
\item The map $\phi\colon V_{\omega} \to T$ given by
  $\phi(x) = \temptree_x$ is obviously surjective.  Suppose that
  $\temptree_x = \temptree_y$.  The roots of these trees are
  $\pair{x}$ and $\pair{y}$.  For all $n$, we have that 
\[ x_n = \treepartial^{\temptree_x}_n(\pair{x}) = \treepartial^{\temptree_y}_n(\pair{y}) = y_n.
\]
 Thus
  $\partial_\omega(\pair{x}) = \partial_\omega(\pair{y})$.  By
  \renewcommand{\lemmaautorefname}{Lemmas}%
  \autoref{lem:w1} and~\ref{seqK-part}, $x = y$.
  \renewcommand{\lemmaautorefname}{Lemma}%
  So $\phi$ is
  injective.  The formula for $\partial_n$ comes from
  \autoref{lemma-seqseq}.\ref{lemma-seqseq:1}.


\item We use \autoref{T:Kripke}. The map $m\colon V_{\omega + 1} \to
  V_\omega$ in~\eqref{diag:m} assigns to a finite set of trees in
  $V_\omega$ their tree-tupling. Its image is the set of all strongly
  extensional, compactly branching trees which are finitely branching
  at the root. An easy induction on $n$ shows that
  $V_{\omega+n}$ is the set of all compactly branching, strongly
  extensional trees $t$ with the property that the topmost $n$ levels
  of $t$ are finitely branching.  With this description,
  $V_{\omega+n} \subseteq D$, and the limit $V_{\omega+\omega}$ is
  simply the intersection $D = \bigcap_n V_{\omega+n}$.
  This shows that the carrier set of $\nu \powf$ is $D$.  For the
  structure map $\delta$, note that
  $m\colon \powf V_{\omega} \to V_{\omega}$ in (\ref{diag:m}) is
  tree-tupling, as are $\powf m$, $\powf \powf m$, etc.  It follows
  that in the intersection, $D$, the coalgebra structure is the
  inverse of tree-tupling.\qedhere
\end{enumerate}  
\end{proof}

This concludes our work showing that for the finite power set functor
$\powf$, $V_{\omega}$ is the set $T$ of strongly extensional,
compactly branching trees, and the terminal coalgebra $\nu \powf$ is
the set $D$ of finitely branching, strongly extensional  trees.

\clearpage

%
% Bibliography
%
\bibliographystyle{plainurl}% the mandatory bibstyle
\bibliography{refs}

\clearpage

\section{Cofree Comonads}

A closely related topic to terminal coalgebras are cofree comonads.
Given an endofunctor $F$ on a category $\A$, a comonad $F_{\sharp}$
together with a natural transformation $\eps\colon F_{\sharp}\to F$ is
\emph{cofree} provided that for every comonad $C$ and every natural
transformation $\phi\colon F_{\sharp}\to F$, there is a unique comonad
morphism $\phibar\colon C\to F_{\sharp}$ for which the triangle below
commutes:
\[
\begin{tikzcd}
    &
   C
    \arrow[dashed]{d}{\phi}
    \\
    F_{\sharp}
    \ar{r}[swap]{\eps}
    \ar{ru}{\phibar}  & F
  \end{tikzcd}
\]

%\begin{notheorembrackets}
  \begin{proposition}[Dual to Barr~\cite{barr0}]\label{P:cofree}
    An endofunctor $F$ on a cocomplete category generates cofree
    comonads iff all cofree coalgebras exist, in other words,
    $U_F\colon \Coalg F \to \A$ has a right adjoint.
  \end{proposition}
%\end{notheorembrackets}
%
\takeout{
Moreover, if the given category has finite products, then a cofree
coalgebra on an object $Y$ is precisely a terminal coalgebra for the
endofunctor $F(-) \times Y$. 
}
\begin{corollary}\label{C:cofree}
  Let $\A$ be a cocomplete category with finite products. An
  endofunctor $F$ generates a cofree comonad iff for every object $Y$
  of $\A$ a terminal coalgebra for $F(-) \times Y$ exists.
  Moreover, the cofree comonad $F_\sharp$ is given by
  \(
  F_\sharp Y = \nu(F(-)\times Y).
  \)
\end{corollary}


\mysubsec{Cofree Comonad Construction}

From the above the following construction of $F_\sharp$ was
derived~\cite{takr}.%
\smnote{Is this the right citation; in the book we cite
  Kelly~\cite[Thm.~23.3]{kelly}; can we give a precise place
  in~\cite{takr}?}
Let $F$ be an endofunctor on a complete category $\A$. Define an
ordinal-indexed chain $F_i$ ($i \in \Ord^\opp$) of endofunctors and
connection natural transformations $f_{i,j}\colon F_i \to F_j$ ($i\geq
j$) by transfinite recursion: put
\[
  \begin{array}{l@{\,}l@{\qquad}l}
    F_0 &= \Id,\\
    F_{j+1} &= FF_j \times \Id& \text{for all ordinals $j$,}\\
    F_j &= \lim_{i<j} F_i & \text{for all limit ordinals $j$, and}
    \\[10pt]
    \multicolumn{3}{l}{
      \text{$f_{1,0}\colon F_1 = F\times \Id \to \Id$ is the projection,}}\\
    \multicolumn{3}{l}{f_{k+1,j+1} = Ff_{k,j} \times \id\colon FF_k
      \times \Id \to FF_j \times \Id,}\\
    \multicolumn{3}{l}{\text{$f_{j,i}\colon F_j \to F_i$ ($j>i$) is the limit cone for
        every limit ordinal $j$.}}
  \end{array}
\]
If $F$ preserves the limit $F_\omega X = \lim_{n<\omega} F_nX$ for
every object $X$, then it generates a cofree comonad carried by
$F_\omega$, and we say that the cofree comonad is \emph{obtained in
  $\omega$ steps.}

Analogously, if $F$ preserves the limit $F_{\omega + \omega} X=
\lim_{n< \omega} F_{\omega +n} X$ for every object $X$, then the
cofree comonad is carried by $F_\sharp = F_{\omega+\omega}$, and we
say that it is \emph{obtained in $\omega+\omega$ steps}.



%
\begin{corollary}
  Every Kripke polynomial functor has a cofree comonad obtained in
  $\omega + \omega$ steps. 
\end{corollary}
%
\noindent
This follows from \autoref{T:Kripke} and \autoref{C:cofree}: if $F$ is a
Kripke polynomial functor, then so is $F(-) \times Y$ for every set $Y$.

\begin{example}
  \begin{enumerate}
  \item For the set functor $FX = X+1$ a cofree comonad
    $F_\sharp$ is obtained in $\omega$ steps: 
    \[
      F_\sharp X = X^* + X^\omega
    \]
    is the set of all finite and infinite words on the set $X$.
    
  \item For a polynomial set functor $H_\Sigma$, a cofree comonad is
    obtained in $\omega$ steps: $F_\sharp Y$ is the set of all
    $\Sigma_Y$-trees where $\Sigma_Y$ is the signature obtained from
    $\Sigma$ by adding a new constant symbol for every $y \in Y$.
  \end{enumerate}
\end{example}
%
\begin{corollary}
  The restriction of a $T_2$ Vietoris polynomial functor to $\Haus$
  generates a cofree comonad obtained in $\omega$ steps.
\end{corollary}
%
\noindent
Indeed, if $F$ is a $T_2$ Vietoris polynomial functor, then so is the
functor $F(-) \times A$ for every Hausdorff space $A$. Now apply
\autoref{C:cofree} and \autoref{T:Vietoris}. 

\begin{corollary}
  Every Hausdorff polynomial functor $F$ on $\Met$, $\CMS$ or $\UMet$
  generates a cofree comonad obtained in $\omega+\omega$ steps. 
\end{corollary}
%
\noindent
The same argument works, but we use \autoref{T:Haus-poly}.

\noindent
\lmnote[inline]{We could add free monads in this section. SM: Let's
  maybe do it when we prepare the journal version.}

\section{Omitted Proofs and Details}

\subsection*{Details for \autoref{R:finitary-Kripke}}
\label{omitted}

Consider the functor $G$
assigning to a set $X$ the set of \emph{nonempty} finite subsets of
$X$.  We verify  that $G$ is not naturally isomorphic to a Kripke
polynomial functor.  This follows from the result below and the observation that
$|G\set{\emptyset}| = 1$.


\begin{proposition}
  Let $F$ be a Kripke polynomial functor $F$. Suppose that $|FX|=1$%
  \smnote{@Larry: we generally omit parenthesis and write $FX$ in lieu
    of $F(X)$ (for functors but not for functions). Below, in
    particular, this makes everything easier readable.}  for some
  nonempty set $X$.  Then $F$ is naturally isomorphic to the constant
  functor with value~$1$.
\end{proposition}

\begin{proof}
  With each Kripke polynomial functor $F$ we associate a number
  $n(F)\in \set{0,1,2,3}$.  The inductive definition of $n(F)$ follows
  the grammar in~\autoref{D:Kripke}.  Following the definition, we
  show four points by induction on $F$:
  \begin{enumerate}
  \item If $n(F) = 0$, then $FX = \emptyset$ for all sets $X$.
  \item If $n(F) = 1$, then $|FX| = 1$ for all sets $X$.
  \item If $n(F) = 2$, then $FX$ is finite for all finite sets $X$,
    $FX$ is infinite for all infinite sets $X$;
    and for all sets $X$ with $|X| \geq 1$, $|FX| > 1$.
  \item If $n(F) = 3$, then $FX$ is infinite for all sets $X$.
  \end{enumerate}

  Here is our definition:
  \begin{align*}
    n(\powf) &= 2\\
    n(\Id) & = 2 \\
    n(\prod_{i\in I} F_i)
    &=
    \begin{cases}
      0 & \text{if for some $i\in I$, $n(F_i) = 0$}\\
      1 & \text{if $n(F_i) = 1$ for all $i\in I$ (including the case $I = \emptyset$)}\\
      2 & \parbox[t]{10cm}{if $n(F_i)\in\set{1, 2}$ for all $i$, and
        $n(F_i) = 1$ for all but finitely many $i$ and $n(F_i) = 2$ for some $i\in I$,} 
      \\
      3  & \parbox[t]{10cm}{if $n(F_i)\in\set{1, 2, 3}$ for all $i$,  and either $n(F_i) = 2$ for infinitely many $i$, or there is some $i\in I$ with $n(F_i) = 3$}
    \end{cases}
    \\
    n(\coprod_{i\in I} F_i)
    & = \begin{cases}
      0 & \text{if $n(F_i) = 0$ for all $i\in I$ (including the case $I = \emptyset$)}
      \\
      1 & \text{if $I$ is a singleton set $\set{i}$ and $n(F_i) = 1$}
      \\
      2 & \parbox[t]{10cm}{$n(F_i) \in\set{0,1,2}$ for all $i\in I$,
        and $n(F_i) = 0$ for all but finitely many $i$; and either 
        $n(F_i)  = 1$ for two distinct $i \neq j\in I$, or $n(F_i) =
        2$ for some $i\in I$}
      \\
      3 & \text{if for some $i\in I$, $n(F_i) = 3$, or if for infinitely many $i$, $n(F_i)\in \set{1,2}$}
    \end{cases}
    \\
    n(FG) &=
    \begin{cases}
      0 & \text{if $n(F) = 0$} \\
      1 & \text{if either $n(F) = 1$; or $n(F) = 2$, $n(G) = 1$, and $|F2| = 1$}\\
      2 & \parbox[t]{10cm}{if either $n(F) = 2$ and $n(G) = 2$; 
        or $n(F) = 2$, $n(G) = 1$, and $|F2| \neq 1$}\\
      3 & \text{if either $n(F) = 3$, or
        $n(F) \geq 2$ and $n(G) = 3$}\\
    \end{cases}
  \end{align*}

  We now turn to the assertions stated before the definition of
  $n(F)$.  For $\powf$ and $\Id$, our assertions are clear.  For a
  functor $\prod_{i\in I}F_i$, the verifications use the induction
  hypothesis and the following facts: the product of any family of
  sets is empty if any factor in the product is empty; the product of
  any family of singleton sets is a singleton set; an arbitrary
  product is finite and non-empty if all but finitely many factors are
  singletons and the remaining factors are finite; a product of
  non-empty sets is infinite if either there are infinitely many
  finite factors of size $\geq 2$, or if one of the factors is
  infinite.

  The verifications for the coproduct and composition steps are
  similar.

  We return to the proof our our result.  If $|F(X)| = 1$ for some
  $X\neq 1$, then $n(F)$ cannot be $0$, $2$ or $3$.  Thus $n(F) = 1$,
  and our result follows.
\end{proof}

\subsection*{Details for \autoref{ex:Vnu}}

\smnote{@Larry: I think this text needs a revision; see Jirka's scan
  from March 4.   LM: I saw the scan from March 5.}
The terminal coalgebra for $\V$ itself was identified by
Abramsky~\cite{abramsky}.  By what we have shown, it is
$V_{\omega} = \lim \V^n 1$. An easy induction on $n$ shows that
$\V^n 1$ is $\powf^n 1$ with the discrete topology; the key point is
that each set $\powf^n 1$ is finite.  We know that $V_{\omega}$ is
thus the set $T$ in~\eqref{eq:T} of all compactly branching, strongly
extensional trees.  The topology was described in
\autoref{P:chain}.\ref{R:Vietoris-tech:1}: it has as a base the sets
$\partial_n^{-1}(U)$ as $U$ ranges over the subsets of $\powf^n 1$.
By \autoref{T:Vietoris}, $\nu F$ is a Stone space.

We also can explicate the terminology of `compactly branching'
(\autoref{dcb}).  We have formulated this using convergence in a
formal sense, and here we connect this to a metric.  The collection
$\mathcal{C}$ of all strongly extensional trees is a proper class.
Setting this aside, it has a pseudo-metric $d$ given by
\[
  d(s,t) = \inf \set{ 2^{-n} : \partial_n(t) =  \partial_n(u)}.
\]
This pseudo-metric has been defined by Worrell~\cite{worrell:05}. 

\begin{proposition} The relation $\approx$ on the class $\TT$ of all
  trees has $2^{\aleph_0}$ equivalence classes, and $d$ is a metric on
  $\TT/{\approx}$.
\end{proposition}

\begin{proof}
  We have a well-defined injective map
  \[
    b\colon \TT/{\approx} \to V_\omega
  \]
  assigning to an equivalence class of a tree $t$ the sequence
  \(
  (\partial_n t)_{n < \omega}$. Hence,
  $|\TT/{\approx}| \leq 2^{\aleph_0}.
  \)
  We check the reverse
  inequality.  For every set $A\subseteq \Nat \setminus \set{0}$, let
  $t_A$ be an infinite path with additional leaves of depth $n$ for
  each $n\in A$.  Then one can check that for $A \neq B$, $t_A$ and
  $t_B$ are not Barr equivalent. Hence
  $|\TT/{\approx}| \geq 2^{\aleph_0}$ since the set
  $\Nat \setminus \set{0}$ has $2^{\aleph_0}$ subsets.

The general construction of a metric space from a pseudo-metric space 
takes equivalence classes of points of distance $0$.  In our setting, points (trees)
have distance $0$ in $\TT$ exactly when they are Barr-equivalent.
\end{proof}


%
\begin{theorem}\label{T:compact}
  The metric space $(\TT/{\approx},d)$ is compact.
\end{theorem}
%
\begin{proof}
\begin{enumerate}
\item The space $\TT/{\approx}$ is totally bounded: for every
  $\eps > 0$, there is a finite cover by $\eps$-balls. Indeed,
  choose~$n$ with $2^{-n} < \eps$ and take the (finite) set~$A$ of all
  extensional trees of height at most $n$. Then every tree $t$
  satisfies $\partial_n t \in A$ and
  $d(t, \partial_n t) \leq 2^{-n} < \eps$.

\item We recall that a metric space is compact iff it is totally
  bounded and complete. Towards showing that $\TT/{\approx}$ is
  complete, let $(t_n)$ be a Cauchy sequence.  For each fixed~$k$, the
  sequence $(\partial_k(t_n))$ is eventually constant.  Let
  $x_k\in V_k$ be such that for all but finitely many~$n$,
  $(\partial_k(t_n)) =x_k$.  We may choose the sequence $(x_k)$ so
  that it is compatible: $\partial_k(x_{k+1}) = x_k$.  Then
  \autoref{forBarr} provides us with a single tree~$t$ such that
  $\partial_k(t) = x_k$ for all $k$.  This tree $t$ is a limit of the
  original sequence $(t_n)$.  \qedhere
\end{enumerate}
\end{proof}
%
\begin{corollary}
  A set $M \subseteq \TT$ is compact iff for every sequence $(t_n)$
  from $M$, if $\lim_n t_n = s$, there exists $t\in M$ such that
  $t\approx s$.
\end{corollary}
%
\begin{proof}
  The quotient map $q\colon \TT\to \TT/{\approx}$ is continuous and
  closed, and thus $M$ is compact iff its image $q[M]$ is.  The latter
  holds iff $q[M]$ is a closed set.  This is what the condition in our
  corollary states.
\end{proof}

It is a standard fact that for metric spaces, compactness and
sequential compactness are equivalent.  
 Thus, for all
trees $t$ and all nodes $x$ of $t$,
$\set{t_y : \text{$y$ a child of $x$ in $t$}}$ is compact iff it is sequentially
compact.  This explains why we were able to define the compactly
branching trees using sequential compactness.

\subsection*{Example: Terminal Coalgebra for a Hausdorff Polynomial Functor}

Here we present a description of a terminal coalgebra for a Hausdorff
polynomial functor related to the type functor for finitely branching
labelled transition systems (\autoref{E:fin-branch}). First note that
the terminal coalgebra for the Kripke polynomial set functor given by
$FX = \powfin(\Sigma \times X)$ has a description which is analogous to the
one for $\powfin$. We work, for the label alphabet $\Sigma$, with
trees having edges labelled in $\Sigma$, and identify again isomorphic
trees. The concept of extensionality is defined w.r.t.~isomorphism of
labelled trees, and tree bisimulation also takes into
account the labelling of egdes:
%
\begin{definition}
  A \emph{tree bisimulation} between two $\Sigma$-labelled trees $t$
  and $u$ is a relation between the nodes of $t$ and the nodes of $u$
  with the following properties:
  \begin{enumerate}
  \item whenever $x$ and $y$ are related: for every $a \in \Sigma$~(a)
    every $a$-child\footnote{An $a$-child of a node $x$ is a child
      node connected to $x$ by an edge labelled by $a$.} of $x$ is
    related to some $a$-child of $y$, and~(b) every $a$-child of $y$
    is related to some $a$-child of $x$;
  \item the roots of $t$ and $u$ are related; the roots are not
    related to other nodes; and
  \item whenever two nodes are related, their parents are also related.
  \end{enumerate}
\end{definition}

The notions of tree bisimilarity and strong extensionality are now
defined completely analogously as for unlabelled trees using the above
notion of tree bisimulation.  The terminal coalgebra for
$\powfin(\Sigma \times X)$ consists of all finitely branching strongly
extensional labelled trees,
and the limit $V_\omega$ consists of all compactly branching strongly
extensional trees, with the limit cone given by $\partial_n$ $(n<\omega)$.

 The proof is completely analogous to that
in \autoref{S:app-worrell}.


\newcommand{\Vbar}{\overline{V}}
\newcommand{\vbar}{\overline{v}}

\begin{example}
  Let $\Sigma = \set{0,1}$ be the metric space with
  $d(0,1) = \delta< 1$.  We describe the terminal coalgebra of
  $F(X) = \H(\Sigma\times X) $.  The elements of $\nu F$ are finitely
  branching strongly extensional trees $t$ whose edges are labelled in
  $\Sigma$.  We denote by $|t|$ the underlying non-labelled tree.  The
  metric is the given for $t\neq s$ by
  \begin{equation}
    \label{eq:deltaTrees}
    d(t, s) = \biggl\{
    \begin{array}{ll}
      \delta & \mbox{if $|t| = |s|$,} \\
      \infty & \mbox{otherwise.}
    \end{array}
    \biggr.
  \end{equation}
  The coalgebra structure assigns to a tree $t$ with maximum subtrees
  $t_1, \ldots, t_n$ the set of pairs
  $\set{(\sigma_1, t_1), \ldots, (\sigma_n, t_n)} $, where $\sigma_i$
  is the label of the corresponding edge from the root.

  Here is the proof.  We denote by $V_i$ the $i^{th}$ entry in the
  terminal-coalgebra chain for $F$ (in~$\Met$) and by $\Vbar_i$ the
  corresponding entry in the terminal-coalgebra chain for
  $\powf(\Sigma\times X)$ (in~$\Set$).  Analogously, we denote by
  $v_{i,j}\colon V_i \to V_j$ and $\vbar_{i,j}\colon \Vbar_i \to \Vbar_j$
  the connecting maps for all $j \leq i < \omega +\omega$. (Thus,
  $v_{\omega+1},\omega$ is $m$ from~\eqref{diag:m} in the
  Introduction.)  We prove by transfinite induction on $i$ that the
  $V_i$ can be chosen to have the underlying set $\Vbar_i$ with the
  metric in (\ref{eq:deltaTrees}), and that the connecting morphisms
  $v_{ij}$ are carried by the corresponding functions $\vbar_{ij}$.
  %
  \begin{enumerate}
  \item For $i=0$, we can choose $\Vbar_0 = \set{t_r}$, the root-only tree
    as a metric space, and $V_0 = \set{t_r}$.

  \item The isolated step.  Since all positive distances in
    $\Sigma\times V_i$ are $\delta$ or $\infty$, every compact subset
    of this space is finite.  Thus the underlying set of the space $V_{i+1}$ is
    $\powf(\Sigma\times \Vbar_i) = \Vbar_{i+1}$.  Given distinct trees
    $t$ and $s$, where $t = \set{(\tau_i, t_i) : i = 1, \ldots, n}$
    and $s = \set{(\sigma_i, s_i) : i = 1, \ldots, m}$ in
    $\H(\Sigma\times V_i) $, we prove that their
    distance is $d(t,s)$ as given above.  This is clear if $n =0$ or
    $m=0$.  In that case, $|t|$ and $|s|$ are not bisimilar, and
    $d(t,s) = \infty$.

    Let $n>0$ and $m>0$.  Then $d(s,t)$ is the Hausdorff distance of
    the given sets in $\H(\Sigma\times V_i)$.  Suppose that the
    corresponding maximum distance between a maximum branch of one
    tree to the other tree is obtained at $t_1$, so that
    $\bar d(t,s) = d((\tau_1,t_1), s)$.  We know that distances in
    $\Sigma\times V_i$ are $0$, $\delta$, or $\infty$.  It follows
    that $\H(\Sigma\times V_i)$ has this property also.  We thus have
    two cases: $\bar d(t,s) =\delta$, or $\bar d(t,s) =\infty$.
    \begin{enumerate}
    \item Case $\bar d(t,s) =\delta$. Since $i = 1$ maximizes the
      distance of a branch of $t$ to $s$, for every $i = 1, \ldots, n$
      there exists $j = 1, \ldots, m$ such that $d(t_i, s_j) \leq
      \delta$. Conversely, for every $j$ there exists $i$ such that
      $d(t_i,s_j) \leq \delta$. Put
      \[
        T= \set{(i,j) : \text{$i\leq n$, $j \leq m$ and
            $d(z_i,s_j)\leq \delta$}}.
      \]
      By the induction hypothesis, for each $(i,j) \in T$ there exists
      a tree bisimulation $R_{i,j} \subseteq |t_i| \times |s_j|$. Then
      the following is a tree bisimulation $R \subseteq |t| \times
      |s|$:
      \[
        R = \set{(x_r,y_r) } \cup \bigcup_{(i,j)\in T} R_{i,j}.
      \]
      Therefore the above distance formula agrees: $d(t,s) = \delta =
      \bar d(t,s)$.
      
    \item Case $\bar d(t,s) = \infty$. Since the minimum distance of
      $(\tau_1, t_1)$ to $(\sigma_j, s_j)$ is $\infty$, we conclude
      from $d(\tau_i,\sigma_j) \leq\delta < \infty$ that $d(t_1, s_j)
      = \infty$ for $j = 1, \ldots, m$.
      Our task is to prove that $|t|$ is not tree-bisimilar to
      $|s|$. Assuming the contrary, $|t_1|$ is tree-bisimilar to
      $|s_j|$ for some $j$. By the induction hypothesis, this
      contradicts $d(t_1, s_j) = \infty$.
    \end{enumerate}
    

  \item The limit step.  Since the chain $\Vbar_i$ converges in
    $\omega + \omega$ steps, we need only consider the cases
    $i = \omega$ and $i = \omega+\omega$.

    For $i = \omega$, note that the forgetful functor $\Met\to\Set$
    preserves limits.  Thus we may assume that the underling set of
    $V_{\omega} = \lim_{m<\omega} V_n$ is
    $\lim_{m<\omega} \Vbar_n = \Vbar_{\omega}$. This is the set of all
    compactly branching strongly extensional, $\Sigma$-edge-labelled
    trees (cf.~\autoref{E:strong-ext}).%
    \smnote{@Jirka: it must be `cf.'; for labelled trees the
      definition of tree bisimulation has to be adjusted. But this is
      straightforward and uninspiring, and I suggest to leave this
      boring task to readers.}
    The limit cone is given by 
    \[
      \ell_{\omega, n} = \partial_n\colon V_{\omega} \to V_{n} \quad (n<\omega).
    \]
    The distance between trees $t\neq s$ is therefore
    \(
      d(t,s) = \sup_{n<\omega} d_n(\partial_n t, \partial_n s).
    \)
    Whenever the trees $|t|$ and $|s|$ are
    tree-bisimilar, $|\partial_n(t)|$ and $|\partial_n(s)|$ are also
    tree-bisimilar.  Thus, the supremum above is $\delta$, as
    required.  If $|t|$ and $|s|$ are not tree-bisimilar, then there
    exists some $n< \omega$ such that $|\partial_n(t)|$ and
    $|\partial_n(s)|$ are not tree-bisimilar.  In this case,
    $d(t,s) = \infty$, as required.

    Let $i = \omega + \omega$. We know that
    $\vbar_{\omega+1}\colon \Vbar_{\omega+1} \to \Vbar_{\omega}$
    represents the subset of all trees that are finitely branching at
    the root (cf.~the proof of \autoref{T:Worrell}.\ref{T:Worrell:2}).%
    \smnote{@Jirka: cf.~since for labelled trees the notion of tree
      bisimulation is adjusted.}
    The connecting map
    $\bar v_{\omega+1, \omega}\colon \Vbar_{\omega+1} \to
    \Vbar_{\omega}$ preserves distances because for all $n$ we have
    $\partial_n \o v_{\omega+1,\omega} = F\partial_n$.  (For this, use
    $F\partial_n = Fv_{\omega,n} = v_{\omega+1,n+1} = v_{\omega,n} \o
    v_{\omega+1, \omega}$.)  Thus $v_{\omega+1,\omega}$ represents the
    metric subspace of $V_{\omega}$ on trees which are finitely
    branching at the root.  The functor $F$ preserves isometric
    embeddings.  Hence
    $v_{\omega+2,\omega} = Fv_{\omega+1,\omega,\omega}\o
    v_{\omega+1,\omega}$ is an isometric embedding.  Analogously,
    $v_{\omega+n,\omega}\colon V_{\omega+n}\to V_{\omega}$ represents
    the metric subspace of all trees finitely branching up to level
    $n$.  The $\omega$-chain $V_{\omega+n}$ ($n < \omega$) consists of
    isometric embeddings.  Thus the limit
    $V_{\omega+\omega} = \lim_n V_{\omega+n}$ is the metric subspace
    of $V_{\omega}$ on the intersection of the subspaces
    $V_{\omega+n}$.  This is precisely the metric space described at
    the beginning of this discussion.
  \end{enumerate}
\end{example}


\end{document}

% Local Variables:
% mode: latex
% TeX-master: t
% End:
