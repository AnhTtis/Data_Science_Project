\documentclass[prd,twocolumn,superscriptaddress,amsfonts,amssymb,amsmath,showpacs]{revtex4-2}
\usepackage{bm}
\usepackage{amsfonts}
\usepackage{latexsym}
\usepackage[latin1]{inputenc}
\usepackage{graphicx}
\usepackage{amsmath}
\usepackage{palatino}
\usepackage{mathpazo}
\usepackage[british]{babel}
\usepackage{array}
\usepackage{hhline}
\usepackage{multirow}
\usepackage{textcomp}
\linespread{1.12}
\usepackage{float}
\usepackage{booktabs}
\usepackage{dcolumn}
%\usepackage{multicol,tabularx,capt-of}
\usepackage{hhline}
\usepackage{multirow}
\usepackage{ragged2e}
\usepackage{hyperref}
\hypersetup{colorlinks,citecolor=blue}
\hypersetup{colorlinks=true,linkcolor=red,filecolor=magenta,    urlcolor=cyan}
\usepackage{amsmath}
\usepackage{xcolor}
\usepackage{orcidlink}
\usepackage{epsfig}
\usepackage{caption}
\usepackage{subcaption}
%\usepackage[caption=false]{subfig}
\usepackage{commath}
\captionsetup[subfigure]{labelformat=brace}
\def \nn  {\nonumber}
% \setlength{\arrayrulewidth}{0.5mm}
% \setlength{\tabcolsep}{18pt}
% \renewcommand{\arraystretch}{1.5}
%%%%%%%%%%%%%%%%%%%%  AAS MACROS LIKELY TO BE USED IN THIS PAPER  %%%%%%%%%%%%%%
\def\jnl@style{\it}
\def\aaref@jnl#1{{\jnl@style#1}}

\def\aaref@jnl#1{{\jnl@style#1}}


\def\aj{\aaref@jnl{AJ}}                   % Astronomical Journal
\def\apj{\aaref@jnl{ApJ}}                 % Astrophysical Journal
\def\apjl{\aaref@jnl{ApJ}}                % Astrophysical Journal, Letters
\def\apjs{\aaref@jnl{ApJS}}               % Astrophysical Journal, Supplement
\def\apss{\aaref@jnl{Ap\&SS}}             % Astrophysics and Space Science
\def\aap{\aaref@jnl{A\&A}}                % Astronomy and Astrophysics
\def\aapr{\aaref@jnl{A\&A~Rev.}}          % Astronomy and Astrophysics Reviews
\def\aaps{\aaref@jnl{A\&AS}}              % Astronomy and Astrophysics, Supplement
\def\mnras{\aaref@jnl{Mon.~Not.~Roy.~Astron.~Soc.}}             % Monthly Notices of the RAS
\def\prd{\aaref@jnl{Phys.~Rev.~D}}        % Physical Review D
\def\prc{\aaref@jnl{Phys.~Rev.~C}}  % Physical Review C
\def\prl{\aaref@jnl{Phys.~Rev.~Lett.}}    % Physical Review Letters
\def\qjras{\aaref@jnl{QJRAS}}             % Quarterly Journal of the RAS
\def\skytel{\aaref@jnl{S\&T}}             % Sky and Telescope
\def\ssr{\aaref@jnl{Space~Sci.~Rev.}}     % Space Science Reviews
\def\zap{\aaref@jnl{ZAp}}                 % Zeitschrift fuer Astrophysik
\def\nat{\aaref@jnl{Nature}}              % Nature
\def\aplett{\aaref@jnl{Astrophys.~Lett.}} % Astrophysics Letters
\def\apspr{\aaref@jnl{Astrophys.~Space~Phys.~Res.}} % Astrophysics Space Physics Research
\def\physrep{\aaref@jnl{Phys.~Rep.}}      % Physics Reports
\def\physscr{\aaref@jnl{Phys.~Scr}}       % Physica Scripta
\def\commat{\aaref@jnl{Comm.~Math.~Phys.}}              % Communications in Mathematical Physics
\def\science{\aaref@jnl{Science}}               % Science
\def\cqg{\aaref@jnl{Classical Quant.~Grav.}}            % Classical and Quantum Gravity
\def\jpcs{\aaref@jnl{JPCS}}                                     % Journal of Physics Conference Series
\def\ijmpd{\aaref@jnl{Int.~J.~Mod.~Phys.~D}}                    % International Journal of Modern Physics D
\def\grg{\aaref@jnl{Gen.~Relat.~Gravit.}}               % General Relativity and Gravitation
\def\rpp{\aaref@jnl{Rep.~Prog.~Phys.}}          % Reports on Progress in Physics
\def\npa{\aaref@jnl{Nucl.~Phys.~A}}        % Nuclear Physics A
\def\lrr{\aaref@jnl{Living Rev.~Rel.}}                   % Living reviews in relativity
\def\jcap{\aaref@jnl{J.~Cosmology Astropart.~Phys.}}    % Journal of cosmology and astroparticle physics
\def\rmp{\aaref@jnl{Rev.~Mod.~Phys.}}   %Reviews of modern physics
\def\epjc{\aaref@jnl{Eur.~Phys.~J.~C}} 
\def\plb{\aaref@jnl{~Phy.~Lett.~B}} 
\def\mpla{\aaref@jnl{Mod.~Phy.~Lett.~A}} 
\def\arxiv{\aaref@jnl{arxiv.org}}
\let\astap=\aap
\let\apjlett=\apjl
\let\apjsupp=\apjs
\let\applopt=\ao


%----------------------------------------------------------------------------
% Some own style rules
%----------------------------------------------------------------------------
% space units properly
%\newcommand{\lc}[1]{\accentset{\circ}{#1}}%Levi-Civita connection
\newcommand{\unit}[1]{\ensuremath{\, \mathrm{#1}}}
% allow equations to be split onto two pages (needed for the lengthy evolequ.s)
\allowdisplaybreaks[1]
% stretch tables a bit
\renewcommand{\arraystretch}{1.1}
\addtolength{\tabcolsep}{6pt}

\DeclareMathOperator{\sgn}{sgn}


\begin{document}
%\color{red}
\color{black}       %% For one column
%
\title{\bf Observational constrained $F(R, \mathcal{G})$ gravity cosmological model and the dynamical system analysis}

%\end{document}

\author{Santosh V Lohakare \orcidlink{0000-0001-5934-3428}}
\email{lohakaresv@gmail.com}
\affiliation{Department of Mathematics,
Birla Institute of Technology and Science-Pilani, Hyderabad Campus,
Hyderabad-500078, India.}

\author{Krishna Rathore \orcidlink{0009-0009-1158-0569}}
\email{rathorekrishnaa2000@gmail.com}
\affiliation{Department of Mathematics,
Birla Institute of Technology and Science-Pilani, Hyderabad Campus,
Hyderabad-500078, India.}

 \author{B. Mishra \orcidlink{0000-0001-5527-3565}}
 \email{bivu@hyderabad.bits-pilani.ac.in}
 \affiliation{Department of Mathematics, Birla Institute of Technology and Science-Pilani, Hyderabad Campus, Hyderabad-500078, India.}

\begin{abstract}
{\textbf{Abstract:} The geometrical and dynamical parameters of the $F(R, \mathcal{G})$ gravity cosmological model are constrained through the cosmological data sets. The functional form of $F(R, \mathcal{G})$ involves the square Ricci scalar and the higher power of the Gauss-Bonnet invariant. The observed value of the free parameters in the expression of $H(z)$, the Hubble parameter, indicates a different phase of the evolution of the Universe. In all the data sets, the early deceleration and late time acceleration behavior of the Universe has been observed. We develop a set of dynamical equations for a given physical system and find the numerical solutions, along with phase-space solutions, and the stability of individual critical points. We also discuss the asymptotic behavior of the critical points of the system.}
\end{abstract}

\maketitle
%\textbf{PACS number}: \\
\textbf{Keywords}: Gauss-Bonnet invariant, Cosmological data set, Dynamical system analysis, Critical points.

\section{Introduction} \label{SEC-I}

The Friedmann-Lema\^{i}tre-Robertson-Walker (FLRW) geometry describes the four-dimensional structure of the Universe on a large scale as isotropic and homogeneous. Through the prism of cold dark matter (CDM), $\Lambda$CDM cosmology offers an excellent model structure for studying galactic dynamics \cite{Baudis_2016_43} in contrast to the cosmological constants ($\Lambda$) \cite{weinberg_2008}. The $\Lambda$CDM cosmology can reproduce most of the correct observations that occur at various scales of the Universe with an inflationary epoch \cite{Perenon_2015_11}. However, recently there has been a growing issue with the effectiveness of the $\Lambda$CDM model. The $\Lambda$CDM model was convinced to explain Hubble data but has been challenged by the $H_0$ tension problem, i.e. discrepancy observed between the values of $H_0$ from the early Universe predicted by model-independent measurements \cite{Riess_2019_876, Wong_2019_498} and observed by model-independent measurements \cite{Aghanim_2018_641}. A significant issue with fine-tuning also affects the $\Lambda$CDM model \cite{Carroll_2001_4}.\\

The Einstein field equations can be modified to fit the matter-energy content of the observable Universe by changing the geometrical sector. On the explanation of the evolution of the Universe, various modified gravity theories have been proposed \cite{Capozziello_2011_509, Faraoni_2010_170, Nojiri_2010_505, Carroll_2004_70}. One of the important findings is that it is possible to define early inflation with different coupling parameters and describe the late-time dark energy (DE) dominated era with precision \cite{Starobinsky_1980_91}. As the modified gravity can reproduce General Relativity (GR) within certain bounds, modifications extending the Hilbert-Einstein action attracted much attention in this paradigm. In $f(R)$ gravity \cite{Carroll_2004_70, Nojiri_2011_505}, the gravitational action generalizes the Hilbert-Einstein action by introducing a generic function of the Ricci scalar curvature $R$ and GR can be restored by assuming $f(R) = R$. Without using $\Lambda$, the fourth-order field equations that define the $f(R)$ gravity models can simulate the behavior of DE under some circumstances. None of the $f(R)$ gravity models can fit all of the experimental data at once or reproduce cosmic history more accurately than a $\Lambda$CDM model. Additionally, some $f(R)$ models show ghosts in their Hamiltonian structure, leading to higher-order field equations, which makes a self-consistent quantisation scheme impossible. The general relativistic gravitational Lagrangian may be modified to include a broader range of curvature invariants, such as $R$, $R_{i\,j}R^{i\,j}$ and $R_{i\,j\,k\,l}R^{i\,j\,k\,l}$, among others. The gravitational Lagrangian in Gauss-Bonnet (GB) gravity theories is a $F(R, \mathcal{G})$ function, where the GB invariant $\mathcal{G}$ is defined as $\mathcal{G} \equiv R^2-4R^{i\,j} R_{i\,j}+R^{i\,j\,k\,l}R_{i\,j\,k\,l}$. The Gauss-Bonnet invariant arises as a mathematical expression in differential geometry and topology and often used while modifying the Einstein-Hilbert action that governs the dynamics of gravity in GR. In \cite{Baojiu_2007_76, Lattimer_2014_784, De_Felice_2009_675}, $F(R,\mathcal{G})$ was proposed to incorporate $R$ and $\mathcal{G}$ into a bivariate function that supports the double inflationary scenario \cite{Laurentis_2015_91} and is strongly supported by observation \cite{Capozziello_2014_29}. Besides its stability, the $F(R,\mathcal{G})$ theory is well-suited to describe the crossing of the phantom divide line and the transformation between an accelerating and decelerating state of celestial bodies. The parametrization methodology in modified teleparallel Gauss-Bonnet gravity has been recently studied to analyze cosmological models \cite{Lohakare_2023_39}.\\

In recent days, the use of the cosmological data sets to constrain the cosmological parameters has been widely used to confront the theoretical result with the results of the cosmological observations. One such cosmological observation is the Cosmic Chronometers (CC) approach. This approach attempts to obtain the age of the Universe or the look back time and is also used to estimate the expansion rate of the Universe. The base of the CC technique is made up of three basic components: i) the definition of a sample of optimal CC tracers, ii) the determination of the differential age, and iii) the assessment of the systematic effects \cite{Moresco_2022_25}. Accurately determining the Hubble parameter $H(z)$ has recently been a major driving force in modern cosmology, providing fundamental information about the energy content of the Universe and the physical mechanisms driving its acceleration. Although many works have attempted to estimate its local value at $z=0$. But few methods are used to determine the $H(z)$ such as detection of the BAO signal in the clustering of galaxies and quasars, or on analyzing SN data, Ref. \cite{Riess_2018_853, Riess_2021_908, Font_Ribera_2014_2014_027, Raichoor_2020_500, Hou_2020_500}). \textit{Pantheon}$^+$ is the successor to the original Pantheon analysis \cite{Scolnic_2018_859} and expands on the original Pantheon analysis framework to combine an even larger number of SN Ia samples, including those in galaxies with measured Cepheid distances, in order to constrain parameters describing the full expansion history with the local expansion rate $(H_0)$. We are motivated here to use some of the cosmological observations data sets to obtain the value of the geometrical parameters. 

The $F(R, \mathcal{G})$ gravity model has been able to address some of the issues of early and late Universe and it is good to know its general phase space structure. The field equations of $F(R, \mathcal{G})$ are one of the most complicated field equations among higher-order theories of gravity, and the study of dynamical system analysis has become an important method to understand the physical behavior of the model. Often, the dynamical systems analysis are performed to find the stability of the model and the presence of fixed points \cite{Olmo_2005_72, Santos_da_Costa_2018_35}. In addition, this may help in avoiding the challenges of solving nonlinear cosmological equations and will allow to examine the asymptotic behavior of cosmological models. Analyzing the asymptotic behavior of critical points of the dynamical system, the overall dynamic of the Universe in terms of cosmological epochs can be described. Hence, we are intending to take up the dynamical system analysis of the model obtained by constraining the parameters from the cosmological data sets. Further from the phase-space analysis, one can assess the stability of the critical points. In the literature, some of the dynamical system analysis approach has been used in the modified generalised cosmological models \cite{Dent_2011_009, Mirza_2017_011, Bahamonde_2019_100_8, Duchaniya_2023_83, Duchaniya_2022_82, Narawade_2022_36, Kadam_2022_82}.

The article is organised as follows: In Section \ref{SEC-II}, we present the mathematical formalism of $F(R, \mathcal{G})$ gravity. In Section \ref{SEC-III}, we discuss and use the observational data sets derived from CC sample, \textit{Pantheon$^+$} samples and BAO. The geometrical and dynamical parameters are also constrained by using these data sets.  Dynamical system analysis has been performed for the model in Section \ref{SEC-IV}. Finally, we summarize our results in Section \ref{SEC-V} with the conclusion.

%%%%%%%%%%%%%%%%%%%%%%%

\section{Basic Formalism of $F(R,\mathcal{G})$ Gravity and Cosmology} \label{SEC-II}
The action of $F(R, \mathcal{G})$ gravity, a modification of Einstein's GR \cite{Laurentis_2015_91, Wu_2015_92, Santos_da_Costa_2018_35, ODINTSOV_2019_938_935, Kumar_Sanyal_2020_37, Brout_2022_938} is,
\begin{equation}\label{1}
S=\int \sqrt{-g}\left[\frac{1}{2k^2}F(R,\mathcal{G})+\mathcal{L}_m\right] d^{4}x,
\end{equation}
Where $g$ is a metric determinant, $\mathcal{L}_m$ describes Lagrangian matter, $k^2=8\pi G_N$, $G_N$ is the gravitational constant. The Gauss-Bonnet invariant is defined as
\begin{equation}\label{2}
\mathcal{G} \equiv R^2-4 R^{i\,j} R_{i\,j} + R^{i\,j\,k\,l} R_{i\,j\,k\,l},
\end{equation}
with the Ricci tensor and Riemann tensor, respectively, denoted by $R^{i\,j}$ and $R^{i\,j\,k\,l}$. The definition of $\mathcal{G}$ in differential geometry is
\begin{equation}\label{3}
\int_{\mathcal{M}} \mathcal{G} d^{n}x=\chi (\mathcal{M}),
\end{equation}
where the manifold $\mathcal{M}$ in $n$ dimensions has $\chi(\mathcal{M})$ Euler properties. Because $\chi(\mathcal{M})=0$ for $n=4$, it may be regarded as a surface term that does not affect dynamics. By modifying the action (\ref{1}) with respect to the metric tensor $g_{i\,j}$, the field equations of $F(R,\mathcal{G})$ gravity may be written as
\begin{eqnarray} \label{4}
\nonumber F_R{G}_{i\,j}&=&k^2{T}_{i\,j}+\frac{1}{2}g_{i\,j}[F(R,\mathcal{G})-R F_{R}]+\nabla_{i} \nabla_{j} F_{R} \nonumber\\&-&g_{i\,j} \Box F_{R} + F_{\mathcal{G}}({-2R}{R_{i\,j}}+4R_{i\,k}R^{k}_{j}-2R^{k\,l\,m}_{i}R_{j\,k\,l m} \nonumber\\&+&4g^{k\,l} g^{m\,n} R_{i\,k\,j\,m} R_{l\,n})+2(\nabla_{i}\nabla_{j}F_{\mathcal{G}})R-2g_{i\,j}(\Box F_{\mathcal{G}})R \nonumber\\&+&4(\Box F_{\mathcal{G}})R_{i\,j}-4(\nabla_{k} \nabla_{i} F_{\mathcal{G}})R^{k}_{j}-4(\nabla_{k} \nabla_{j} F_{\mathcal{G}})R^{k}_{i} \nonumber\\&+&4g_{i\,j}(\nabla_{k} \nabla_{l} F_{\mathcal{G}})R^{kl}-4(\nabla_{l} \nabla_{n} F_{\mathcal{G}})g^{kl}g^{mn}R_{i\,k\,j\,m}
\end{eqnarray}
where $G_{i\,j}$ represents the Einstein tensor, $\nabla_{i}$ describes the covariant derivative operator associated with $g_{i\,j}$, $\Box \equiv g^{i\,j}\nabla_{i}\nabla_{j}$ represents the covariant d'Alembert operator, and ${T}_{i\,j}$ represents the energy-momentum tensor. Additionally, the following quantities have been specified.
\begin{equation*} 
F_R\equiv \frac{\partial F(R,\mathcal{G})}{\partial R},\hspace{1cm} F_\mathcal{G}\equiv \frac{\partial F(R,\mathcal{G})}{\partial \mathcal{G}},
\end{equation*}

The space-time for the flat FLRW metric can be given as
\begin{equation} \label{5}
ds^{2}=-dt^{2}+a^{2}(t)(dx^{2}+dy^{2}+dz^{2}),
\end{equation}
where $a(t)$ is the scale factor and the Hubble parameter is $H\equiv\frac{\dot{a}(t)}{a(t)}$, where the over-dot denotes the derivative with respect to cosmic time $t$. Subsequently, the Ricci scalar and the Gauss-Bonnet invariant respectively becomes

\begin{equation} \label{6}
R=6(\dot{H}+2H^{2}), \hspace{1cm} \mathcal{G}=24H^{2}(\dot{H}+H^{2})
\end{equation}
Using an energy-momentum tensor, the Einstein equations and continuity equation are determined by the presence of an isotropic perfect fluid
\begin{eqnarray} \label{7}
    T_{i}^{j}=diag (-\rho, p, p, p)
\end{eqnarray}
where $\rho$ denotes the matter-energy density and $p$ is the pressure of matter.
By substituting Eqn. \eqref{5} and Eqn. \eqref{6} into the gravitational field Eqn. \eqref{4} we obtain the field equations of $F(R,\mathcal{G})$ gravity as,

% \begin{eqnarray} 
% 3H^{2}&=&\frac{\kappa^{2}}{F_{R}} \rho+\frac{1}{2}\left[R+\mathcal{G} \frac{F_{\mathcal{G}}}{F_{R}} - \frac{F(R,\mathcal{G})}{F_{R}}\right]\nonumber\\&-&3H \frac{\dot{F}_{R}}{F_{R}}-12H^{3}\frac{\dot{F}_{\mathcal{G}}}{F_{R}},\label{8}\\
% 2\dot{H}+3H^{2} &=& -\frac{\kappa^{2}}{F_{R}} p +\frac{1}{2}\left[R+\mathcal{G} \frac{F_{\mathcal{G}}}{F_{R}} - \frac{F(R,\mathcal{G})}{F_{R}}\right]\nonumber\\&-&2H\frac{\dot{F}_{R}}{F_{R}}-\frac{\ddot{F}_{R}}{F_{R}}-4H^{2}\frac{\ddot{F}_{\mathcal{G}}}{F_{R}}-8H\left(\dot{H}+H^2\right)\frac{\dot{F}_{\mathcal{G}}}{F_{R}}.\label{9}\nonumber\\
% \end{eqnarray}

\begin{eqnarray} 
3H^{2} F_{R}&=&{\kappa^{2}} \rho+\frac{1}{2}\left[R {F_{R}}+\mathcal{G} {F_{\mathcal{G}}} - {F(R,\mathcal{G})}\right]\nonumber\\& &-3H \dot{F}_{R} -12H^{3} \dot{F}_{\mathcal{G}},\label{8}\\
(2\dot{H}+3H^{2}) F_{R} &=& -\kappa^{2} p +\frac{1}{2}\left[R F_{R}+\mathcal{G} F_{\mathcal{G}} - F(R,\mathcal{G})\right]\nonumber\\& & -2H\dot{F}_{R}-\ddot{F}_{R}-8H\left(\dot{H}+H^2\right) \dot{F}_{\mathcal{G}}\nonumber\\& & -4H^{2} \ddot{F}_{\mathcal{G}}.\label{9}
\end{eqnarray}

Background cosmology can be simplified by rewriting these equations as effective fluids, embodying additional terms due to higher-order curvature terms incorporated into the expression. We consider mapping, $F(R,\mathcal{G}) \longrightarrow R+\mathcal{F}(R,\mathcal{G})$ and accordingly, the equations of motion in Eqn. (\ref{8}) and Eqn. (\ref{9}) reduce to the following form:
\begin{eqnarray}
    3 H^2 &=& \kappa^2 (\rho + \rho_{\text{eff}}) \label{10}\\
    3 H^2+2 \dot{H} &=& -\kappa^2 (p+p_{\text{eff}}), \label{11}
\end{eqnarray}
and the effective fluid can be recovered as
\begin{eqnarray}
    \kappa^2 \rho_{\text{eff}} &=& -3 H^2 \mathcal{F}_R+\frac{1}{2}\big(R \mathcal{F}_R + \mathcal{G} \mathcal{F}_\mathcal{G}-\mathcal{F}(R,\mathcal{G}) \nonumber\\ & & -6 H \dot{\mathcal{F}}_R-24 H^3 \dot{\mathcal{F}}_\mathcal{G}\big) \label{12}\\
    \kappa^2 p_{\text{eff}} &=& (2 \dot{H}+3 H^2) \mathcal{F}_R-\frac{1}{2} \big(R \mathcal{F}_R+\mathcal{G} \mathcal{F}_{\mathcal{G}}-\mathcal{F}(R,\mathcal{G}) \nonumber\\&-&4 H \dot{\mathcal{F}}_R-2\ddot{\mathcal{F}}_R-8 H^2 \ddot{\mathcal{F}}_{\mathcal{G}}-16 H \dot{H} \dot{\mathcal{F}}_{\mathcal{G}}-16 H^3 \dot{\mathcal{F}}_{\mathcal{G}}\big) \nonumber \label{13}\\
\end{eqnarray}

Now, the $\mathcal{F}(R, \mathcal{G})$ form needs to be specified to frame the cosmological model, and hence we consider,
\begin{eqnarray} \label{14}
\mathcal{F}(R,\mathcal{G})=\alpha R^2 \mathcal{G}^\beta
\end{eqnarray}
where $\alpha$ and $\beta$ are arbitrary constants. It is a double inflationary scenario connected to the existence of Noether symmetries \cite{Capozziello_2014_29}. We rewrite $R$ and $\mathcal{G}$ in the redshift parameter to get the expansion rate [$(1 + z)H(z) = -\frac{dz}{dt}$] as
\begin{eqnarray}
    R&=&6\left(2 H_0^2 E(z)-\frac{H_0^2 (1+z) E'(z)}{2}\right), \nonumber\\ \mathcal{G}&=&24H_0^2 E(z) \left(H_0^2 E(z)-\frac{H_0^2 (1+z) E'(z)}{2}\right), \label{15}
\end{eqnarray}
where $H^2 (z)=H_0^2 E(z)$, $H_0$ represents the present value of Hubble parameter and the prime denotes the derivative to the redshift parameter. We use the following functional form for $E(z)$ \cite{Lemos_2018_483},
\begin{eqnarray} \label{16}
    E (z) = A\, (1+z)^3 + B + C\, z + D\, ln(1+z)
\end{eqnarray}
where $A$, $B$, $C$, and $D$ are free parameters.

\section{Observational Constraints} \label{SEC-III}
In cosmology, the Hubble and \textit{Pantheon$^+$} data sets are important to study the expansion history of the Universe and the properties of DE. In this problem,we shall use the early-type galaxies expansion rate data such as, the $H(z)$ and \textit{Pantheon$^+$} data and the baryon-acoustic oscillations (BAO) and the Cosmic Microwave Background (CMB) distance priors. Since $H(z)$ provides the basic information about the energy content and the main physical mechanisms driving the present acceleration of the Universe, therefore the accurate determination of the expansion rate of the Universe has become important. In CC measurement, the expansion rate of the Universe is directly and cosmology-independently estimated without any assumptions about the origin of the Universe. There is no direct correlation between the observations and cosmological models. Therefore, these data sets serve as an independent tool to estimate the parameters of cosmological models.

\begin{figure*} [!htb]
\centering
\includegraphics[width=120mm]{HandP.pdf}
\caption{The contour plots with $1-\sigma$ and $2-\sigma$ errors for the parameters $H_0$, $A$, $B$, $C$ and $D$ by using CC samples and {\textit{Pantheon$^+$}} samples.}
\label{FIG2}
\end{figure*}

\begin{figure*} [!htb]
\centering
\includegraphics[width=120mm]{HPBAO.pdf}
\caption{The contour plots with $1-\sigma$ and $2-\sigma$ errors for the parameters $H_0$, $A$, $B$, $C$ and $D$ by using CC + {\textit{Pantheon$^+$}} + BAO.}
\label{FIG3}
\end{figure*}
\begin{table*} [!htb]
\centering % used for centering table
\begin{tabular}{|c|c|c|c|c|} % centered columns (3 columns)
\hline\hline %inserts double horizontal lines
Coefficients & CC Sample  & \textit{Pantheon$^+$} & CC + \textit{Pantheon$^+$} & CC + \textit{Pantheon$^+$} + BAO\\ [0.5ex] % inserts table
%heading
% inserts single horizontal line
\hline
$H_0$ & 70.2 $\pm$ 4.6 & 69.1 $\pm$ 4.8 & $68.69_{-0.59}^{+0.67}$ & $69.26_{-0.53}^{+0.57}$ \\
\hline
$A$ & 0.297 $\pm$ 0.04 & 0.28 $\pm$ 0.11 & $0.285^{+0.050}_{-0.048}$ & $0.264^{+0.039}_{-0.036}$ \\
\hline
$B$ & 0.66$^{+0.11}_{-0.13}$ & 0.64 $\pm$ 0.16 & $0.689^{+0.071}_{-0.067}$ & $0.698^{+0.070}_{-0.071}$ \\
\hline
$C$ & 0.0099 $\pm$ 0.0053 & 0.02 $\pm$ 0.011 & $0.012^{+0.98}_{-1.1}$ & $0.012 \pm 0.71$ \\[0.5ex] % [1ex] 
\hline %inserts single line
$D$ & 0.0037 $\pm$ 0.0019 & 0.0099 $\pm$ 0.056 & $0.014^{+1.1}_{-0.99}$ & $0.0025^{+0.81}_{-0.61}$ \\[0.5ex] % [1ex] adds vertical space
\hline %inserts single line
\end{tabular}
\caption{These marginalized constraints are based on the CC samples, \textit{Pantheon$^+$} samples, and BAO data sets.} % title of Table
\label{TABLE I}
\end{table*}

\subsection{CC Dataset}
To estimate the expansion rate of the Universe at redshift $z$, we use the widely used differential age (DA) method. In this way, it is possible to predict $H(z)$ using $(1+z) H(z)=-\frac{dz}{dt}$. The Hubble parameter is modeled on 32 data points for a redshift range of $0.07 \leq z \leq 1.965$ \cite{Moresco_2022_25}. The mean values of the parameters $H_0, A, B, C,$ and $D$ are determined by minimizing the chi-square value. Using Hubble data, the chi-square function is as follows.
\begin{equation} \label{eq.hubbdef}
    \chi_{\text{Hubble}}^{2}=\sum_{i=1}^{32}\frac{\left[H_{th}(z_i)-H_{\text{obs}}(z_i)\right]^2}{\sigma_{i}^{'2}},
\end{equation}
A standard error in Hubble function experimental values is denoted by $\sigma_{i}^{'}$. The $H_{th}(z_i)$ and $H_{\text{obs}}(z_i)$ respectively indicates the theoretical and observable values of the Hubble parameter.

\subsection{\textit{Pantheon$^+$} Sample}
Among the \textit{Pantheon$^+$} sample data set are 1701 light curves of 1550 distinct Type Ia supernovae with redshifts between $(0.00122, 2.2613)$ \cite{Brout_2022_938}. The observed and theoretical distance moduli are compared to fit the model parameters. According to the \textit{Pantheon$^+$}, the SNe Ia functions for 1701 are
\begin{equation} \label{eq.pantheondef}
\chi_{\textit{Pantheon$^+$}}^{2}=\sum_{i=1}^{1701}\frac{\left[\mu_{th}(\mu_0,z_i)-\mu_{\text{obs}}(z_i)\right]^2}{\sigma_{i}^{'2}}
\end{equation}

In addition, the standard error in the actual value of $H$ is indicated with the symbol $\sigma_{i}^{'}$. The theoretical distance modulus $\mu_{th}$ is defined as $\mu_{th}^i=\mu(D_{L})=m-M=5 log_{10}D_L(z)+\mu_0$, where $m$ and $M$ are represented by apparent and absolute magnitudes, and the nuisance parameter $\mu_0$ is defined as $\mu_0=5log(\frac{H_0^{-1}}{Mpc})+25$. The luminosity distance $D_L$ is defined by $D_L(z)=(1+z) H_0 \int \frac{1}{H(z^*)} dz^*$. The $H(z)$ series is limited to the tenth term and approximately integrates the limited series to obtain the luminosity distance.
\subsection{BAO Dataset}
By studying a fluid consisting of photons, baryons, and dark matter tightly coupled through Thompson scattering, we can explore the oscillations produced in the early phase of the Universe due to cosmological perturbations. There are three types of BAO measurements: High-resolution Sloan Digital Sky Surveys (SDSS), Six Degree Field Galaxy Surveys (6dFGS), and Baryon Oscillation Spectroscopic Surveys (BOSS) \cite{Percival_2010_401}. Here, we incorporate BAO data and the following cosmology as follows.

\begin{eqnarray}
    d_A (z)&=&\int_{0}^{z} \frac{dz^*}{H(z^*)} \label{19}\\
    D_v (z)&=&\left( \frac{d_A (z)^2 z}{H(z)}\right)^{\frac{1}{3}} \label{20}
\end{eqnarray}
and
\begin{eqnarray} \label{21}
    \chi^2_{BAO}=X^T C^{-1} X,
\end{eqnarray}
where the angular diameter distance and the dilation scale are represented by $d_A(z)$, $D_V(z)$, respectively, and $C$ represents the covariance matrix \cite{Giostri_2012_2012_027}.

\begin{figure*} [!htb]
\centering
\includegraphics[width=8.9cm,height=6cm]{Herrorbar31.pdf}
\includegraphics[width=8.9cm,height=6cm]{Pntheonerror.pdf}
\caption{The blue error bars are from the 32 points of CC sample, and the solid red line is of the model, and the broken black line is for the $\Lambda$CDM (left panel). In (right panel), the red line is the plot of the model's distance modulus $\mu(z)$ versus $z$, which exhibits a better fit to the 1701 points of the \textit{Pantheon$^+$} data sets along with its error bars.}
\label{FIG1}
\end{figure*}
\begin{figure*} [!htb]
\centering
\includegraphics[scale=0.46]{q.pdf}
\includegraphics[scale=0.56]{EoS.pdf}
\caption{Deceleration parameter and the EoS with CC, \textit{Pantheon$^+$} and BAO datasets for the parameters $\alpha=1.1$, $\beta=4$.}
\label{FIG4}
\end{figure*}

\begin{table*}[!htb]
%\small\addtolength{\tabcolsep}{-5pt}
%\resizebox{\textwidth}{!}{%
\centering % used for centering table
\begin{tabular}{|c|c|c|c|c|} % centered columns (3 columns)
\hline\hline %inserts double horizontal lines
Parameters & CC Sample & \textit{Pantheon$^+$} & CC + \textit{Pantheon$^+$} & CC + \textit{Pantheon$^+$} + BAO\\ [0.5ex] % inserts table
%heading
% inserts single horizontal line
\hline
$q$ & -0.526 ($z_t \approx 0.636$) & -0.529 ($z_t \approx 0.656$) & -0.548 ($z_t \approx 0.691$) & -0.579 ($z_t \approx 0.74$) \\
\hline
$\omega_{\text{eff}}$ & -0.8478 & -1.02 & -1.224 & -1.47 \\
\hline
\end{tabular}
\caption{Present value of deceleration and EoS parameter based on the CC samples, \textit{Pantheon$^+$} samples, and BAO datasets.} % title of Table
\label{TABLE II}
\end{table*}
The deceleration parameter $q=-1-\frac{\dot{H}}{H^2}$ describes the rate of acceleration of the Universe, where a positive $q$ indicates that the Universe is in a decelerated phase, while a negative $q$ indicates that the Universe is in an accelerated phase. The constrained values of model parameters in the Hubble, \textit{Pantheon$^+$}, and BAO data sets result in $q$ changing from a positive value in the past, suggesting an early slowdown, to a negative value in the present, indicating an acceleration in the present, as seen in Fig. \ref{FIG4}. In the current cosmic epoch, Hubble and Pantheon data are relatively consistent with the range $q_0=-0.528^{+0.092}_{-0.088}$ determined by recent observations \cite{Christine_2014_89} and a redshift from deceleration to acceleration occurs at $z_t=0.8596^{+0.2886}_{-0.2722}$, $z_t=0.65^{+0.19}_{-0.01}$ \cite{Yang_2020_2020_059, Capoziello_2008_664, Capozziello_2014_90_044016}. The deceleration parameter $q_0= -0.526$, $q_0= -0.529$, $q_0=-0.548$ and $q_0=-0.579$ at the current cosmic epoch and our derived model shows a smooth transition from a deceleration phase of expansion to an acceleration phase, at $z_t = 0.636$, $z_t = 0.656$, $z_t = 0.691$ and $z_t = 0.74$ for CC, \textit{Pantheon$^+$}, CC+\textit{Pantheon$^+$} and CC+\textit{Pantheon$^+$}+BAO datasets respectively, is mostly in line with the current findings. The recovered transition redshift value $z_t$ is consistent with certain current constraints based on 11 $H(z)$ observations reported by Busca et al. \cite{Busca_2013_552} between the redshifts $0.2 \leq z \leq 2.3$, $z_t = 0.74 \pm 0.5$ from Farooq et al. \cite{Farooq_2013_766}, $z_t = 0.7679^{+0.1831}_{-0.1829}$ by Capozziello et al. \cite{Capozziello_2014_90_044016} and $z_t = 0.60^{+0.21}_{-0.12}$ by Yang et al. \cite{Yang_2020_2020_059}

Among the parameters that define the behavior of the Universe is the deceleration parameter, which determines whether the Universe continuously decelerates or accelerates constantly, has a single phase of transition or several, etc. Energy sources play a similar role in the evolution of the Universe according to the effective EoS parameter $\big(\omega_{\text{eff}}=\frac{p_{\text{eff}}}{\rho_{\text{eff}}}\big)$. Calculating the related energy density and pressure of the dark energy, as illustrated in Fig. \ref{FIG4}, allows us to see the variations in the effective EoS of DE [ Eqn. (\ref{13})] regarding the redshift variable. For the current value of the effective EoS ($z = 0$) to match $-0.8478$, $-1.02$, $-1.224$ and $-1.47$ for CC, \textit{Pantheon$^+$}, CC + \textit{Pantheon$^+$} and CC + \textit{Pantheon$^+$} + BAO datasets respectively, this figure shows the phantom behavior (at $z \leq -0.015$) and its approach to $-1$ at late times. The numerical value of the EoS parameter has also been restricted by several cosmological investigations, including the Supernovae Cosmology Project $\omega_{\text{eff}}=-1.035^{+0.055}_{-0.059}$ \cite{Amanullah_2010_716}, Plank 2018, $\omega_{\text{eff}}=-1.03\pm 0.03$ \cite{Aghanim_2018_641} and WAMP+CMB, $\omega_{\text{eff}}=-1.079^{+0.090}_{-0.089}$ \cite{Hinshaw_2013_208}.

\section{Dynamical System Analysis} \label{SEC-IV}
The dynamical system analysis for the future behavior of the system can predict cosmological models based on dynamical systems. There may be an equation of the type $x= f(x)$ representing the dynamical system, where $x$ represents the column vector, and $f(x)$ represents the equivalent column vector of the autonomous equations. The prime represents the derivative with respect to $N = ln a$. This method can generate the general form of the dynamical system for the modified FLRW equations, which are defined by Eqn. (\ref{9}). As an autonomous system, the set of cosmological equations of the model is written with the following dimensionless variables \cite{Santos_da_Costa_2018_35}:
\begin{eqnarray} \label{22}
    u_1=\frac{H \dot{F}_R}{F_R},\hspace{0.5cm} u_2=\frac{F}{6 H^2 F_R},\hspace{0.5cm} u_3=\frac{R}{6 H^2},\nonumber\\ u_4=\frac{\mathcal{G} F_{\mathcal{G}}}{6 H^2 F_R},\hspace{0.5cm} u_5=\frac{4 H \dot{F}_\mathcal{G}}{F_R},
\end{eqnarray}
with the energy density parameters
\begin{eqnarray} \label{23}
    u_6=\Omega_r= \frac{\kappa^2 \rho_r}{3 H^2 F_R},\hspace{1cm} u_7=\Omega_m= \frac{\kappa^2 \rho_m}{3 H^2 F_R}
\end{eqnarray}
Thus, we have the algebraic identity
\begin{equation} \label{24}
    1=-u_1-u_2+u_3+u_4-u_5+\Omega_r+\Omega_m\\
\end{equation}
The dynamical system is
\begin{eqnarray}
    \frac{du_1}{dN}&=& \frac{\ddot{F}_{R}}{F_{R}H^{2}} -u_1^{2}-u_1\frac{\dot{H}}{H^{2}}, \label{25}\\
    \frac{du_2}{dN}&=& \frac{\dot{F}}{6F_{R}H^{3}}-u_1u_2-2u_2\frac{\dot{H}}{H^{2}}, \label{26}\\
    \frac{du_3}{dN}&=& \frac{\dot{R}}{6H^{3}}-2u_3 \frac{\dot{H}}{H^{2}}, \label{27}\\
    \frac{du_4}{dN}&=& \frac{\dot{\mathcal{G}}}{\mathcal{G}H} u_4+\frac{\mathcal{G}}{24H^{4}} u_5-u_1u_4-2u_4(u_3-2), \label{28}\\
    \frac{du_5}{dN}&=& u_5 \frac{\dot{H}}{H^{2}}+4\frac{\ddot{F_{\mathcal{G}}}}{F_{R}}-u_1u_5 \label{29}\\
    \frac{du_6}{dN}&=& -2u_3 u_6-u_1 u_6, \label{30}\\
    \frac{du_7}{dN}&=& -u_7 \left(3+u_1+2 \frac{\dot{H}}{H^{2}}\right) \label{31}
\end{eqnarray}

To close the system, all terms on the right-hand side of the above equations must be expressed in terms of variables specified in Eqn. (\ref{14}). Thus, we find
\begin{eqnarray}
 \frac{\dot{H}}{H^{2}}&=&u_3 \label{32}\\
 \frac{\dot{F}}{6F_{R}H^{3}}&=&{-u_1 u_3} \label{33}\\
 \frac{\dot{R}}{6H^{3}}&=&{u_1 u_3} \label{34}\\
 \frac{\mathcal{G}}{24H^{4}}&=&u_3-1 \label{35}\\
 \frac{\dot{\mathcal{G}}}{\mathcal{G}H}&=&\frac{1}{u_3-1}\left[{u_1 u_3}+2(u_3-2)^{2}\right] \label{36}
\end{eqnarray}

\begin{table*} [!htb]
    \centering % used for centering table
    \begin{tabular}{|c|c|c|c|c|c|c|} % centered columns (5 columns)
    \hline\hline %inserts double horizontal lines
    C.P. & ${u_3}_c$ & ${u_4}_c$ & ${u_6}_c$ & ${u_7}_c$ & Exists for \\ [0.5ex] % inserts table %heading
    \hline\hline % inserts single horizontal line
    $\mathcal{P}_{1}$ & 0 & 0 & 1 & 0 & always \\
     \hline
    $\mathcal{P}_{2}$ & 0 & $\frac{1-x_4}{5}$ & $x_4$ & 0 & $3+2x_4 \neq 0$, $\beta=\frac{1}{2}$ \\
     \hline
    $\mathcal{P}_{3}$ & 2 & -2 & 0 & 0 & $-1+4\beta \neq 0$ \\
   \hline
    $\mathcal{P}_{4}$ & 0 & $x_6$ & 0 & 0 & $-1+x_6 \neq 0, -1+2x_6 \neq 0, \beta =\frac{-3-x_6}{8(-1+x_6)}$ \\
    \hline
    $\mathcal{P}_{5}$ & $x_7$ & $\frac{1}{2}(-6+x_7)$ & 0 & 0 & $-1+x_7\neq 0, -2+x_7 \neq 0, 14-12 x_7+3 x_7^2 \neq 0, \beta =0$ \\
    \hline
    \end{tabular}
     \caption{ The critical points of the dynamical system.} % title of Table
    \label{TABLE-III}
\end{table*}

A theory specified by $\Gamma=\frac{\ddot{F}_{R}}{F_{R}H^{2}}$ is used. It can be inferred that the system can only be considered complete once it is expressed in terms of dynamical variables (\ref{22}), (\ref{23}). From Eqns. (\ref{14}), (\ref{22}) and Eqn. (\ref{29}), we can get
\begin{eqnarray}
    u_3&=&2u_2, \label{37}\\
    u_5&=&\frac{u_4}{u_3-1}\left[{2u_1}+\frac{\beta-1}{u_3-1}\left[2(u_3-3)^{2}+{u_1u_3}\right]\right] \label{38}
\end{eqnarray}
Using these relations and the constraint [Eqn. \eqref{24}], the system can be reduced to a set of four equations as
\begin{eqnarray}
    \frac{du_3}{dN}&=&u_1u_3-2u_3(u_3-2), \label{39}\\
    \frac{du_4}{dN}&=&\frac{\beta u_4}{u_3-1}\left[2(u_3-3)^{2}+{u_1u_3}\right]+u_1u_4-2u_4(u_3-2),\nonumber\\ \label{40}\\
    \frac{du_6}{dN}&=&-2u_3 u_6-u_1 u_6, \label{41}\\
    \frac{du_7}{dN}&=&-u_7(2u_3+u_1-1) \label{42}
\end{eqnarray}
where
\begin{equation} \label{43}
    u_1 = \frac{-1+\frac{3}{2}u_3+u_6+u_7+u_4-2(\beta-1)\frac{(u_3-2)^{2}}{(u_3-1)^{2}}u_4}{1+\frac{u_4}{(u_3-1)}\left[2+u_3\frac{(\beta-1)}{(u_3-1)}\right]}
\end{equation}
and
\begin{eqnarray} \label{omega_total}
    \omega_{\text{tot}}=-1-\frac{2\dot{H}}{3H^2}=-1-\frac{2}{3}u_3
\end{eqnarray}

\begin{table} [H]
    \centering % used for centering table
    \begin{tabular}{|c|c|c|c|c|c|c|} % centered columns (5 columns)
    \hline\hline %inserts double horizontal lines
    C.P. & $\Omega_m$ & $\Omega_r$ & $\Omega_{de}$ & $q$ & $\omega_{\text{tot}}$ \\ [0.5ex] % inserts table %heading
    \hline\hline % inserts single horizontal line
    $\mathcal{P}_{1}$ & 0 & 1 & 0 & 1 & $\frac{1}{3}$ \\
     \hline
    $\mathcal{P}_{2}$ & 0 & $x_4$ & $1-x_4$ & 1 & $\frac{1}{3}$ \\
     \hline
    $\mathcal{P}_{3}$ & 0 & 0 & 1 & -1 & $-1$ \\
   \hline
    $\mathcal{P}_{4}$ & 0 & 0 & 1 & 1 & $\frac{1}{3}$ \\
    \hline
    $\mathcal{P}_{5}$ & 0 & 0 & 1 & $1-x_7$ & $\frac{1}{3} (1-2x_7)$ \\
    \hline
    \end{tabular}
     \caption{The deceleration, EoS and density parameters for the critical points.} % title of Table
    \label{TABLE-IV}
\end{table}

\begin{table} [H]
\begin{tabular}{|*{2}{c|}}\hline
\parbox[c][0.5cm]{0.5cm}{C.P.} & \parbox[c][1cm]{8cm}{Eigenvalues}\\\hline
\hline
\parbox[c][0.5cm]{0.5cm}{$\mathcal{P}_1$} & \parbox[c][1cm]{8cm}{$\big\{4, -1, 1, -4(-1+2 \beta)\big\}$}\\\hline
\parbox[c][0.5cm]{0.5cm}{$\mathcal{P}_2$} & \parbox[c][1cm]{8cm}{$\big\{0, \frac{-5 (-1+2 x_4)}{3+2 x_4}, 1, 4\big\}$}\\\hline
\parbox[c][0.5cm]{0.5cm}{$\mathcal{P}_3$} & \parbox[c][1cm]{8cm}{$\big\{-4, -3, \frac{3-12x_5-\sqrt{9-136x_5+400 x_6^2}}{2(-1+4 x_6)}, \frac{3-12x_5+\sqrt{9-136x_5+400 x_6^2}}{2(-1+4 x_6)}\big\}$}\\\hline
\parbox[c][0.5cm]{0.5cm}{$\mathcal{P}_4$} & \parbox[c][1cm]{8cm}{$\big\{\frac{4x_6}{(-1+x_6)(-1+2 x_6)}, \frac{-3+x_6}{-1+x_6}, \frac{2(-1+3 x_6)}{-1+x_6}, \frac{1-7 x_6+10 x_6^2}{(-1+x_6)(-1+2 x_6)} \big\}$}\\\hline
\parbox[c][0.5cm]{0.5cm}{$\mathcal{P}_5$} & \parbox[c][1cm]{8cm}{$\big\{0, \frac{-6(1-2 x_7+x_7^2)}{14-12 x_7+3 x_7^2}, -4(-1+x_7), (5-4x_7) \big\}$}\\\hline
\end{tabular}
\caption{Equivalent eigenvalues for fixed points} % title of Table
    \label{TABLE-VI}
\end{table}

The physical properties and existence of the critical points of these systems are shown in Table \ref{TABLE-III}, \ref{TABLE-IV}. The critical points can be calculated to analyze their features and behavior. Below we will discuss the properties of each critical point and their potential connection with different evolutionary eras of our Universe, which are divided into five critical points.

\subsection{Visualization of Phase Portraits}
For a complete understanding of the distinguishing features of each critical point, it is crucial to describe its behavior in proper diagrams. The phase portraits for each critical point are presented in this section, along with the critical steps involved in their derivation and whether they are compatible with the analysis of Table \ref{TABLE-III} and \ref{TABLE-VI}.The properties of each of the five critical points separately discussed and explores their possible connections to the eras of the evolution of the Universe that they represent.

\begin{itemize}
    \item {\textbf{Point $\mathcal{P}_1$:}} In a radiation-dominated Universe, the first critical point $\mathcal{P}_1$ occurs. Table \ref{TABLE-III} shows that the critical point exists for all values of the free parameters. Based on Table \ref{TABLE-IV}, $\Omega_r=1$, this critical point applies to any free model parameter. The total equation of state $\omega_{\text{tot}}=\frac{1}{3}$ and deceleration parameter $q=1$ demonstrates that the background level does not experience late-time acceleration in this solution. Table \ref{TABLE-III} shows that our critical point is a saddle hyperbolic. Point $\mathcal{P}_1$ possesses a 2D local unstable manifold with boundaries defined only within the neighbourhood of the critical point, whereas the description of local indicates that these boundaries are determined only within the neighbourhood of the critical point.
\end{itemize}

\begin{figure*} [!htb]
    \centering
    \includegraphics[width=65mm]{p1p2.pdf}
    \includegraphics[width=65mm]{p3.pdf}\\
    \includegraphics[width=65mm]{p4.pdf}
    \includegraphics[width=65mm]{p5.pdf}
    \caption{$2D$ phase portrait for the dynamical system.} \label{FIG5}
\end{figure*}
\begin{table*} [!htb]
%    \small\addtolength{\tabcolsep}{15pt}
    \centering % used for centering table
    \begin{tabular}{|c|c|c|c|c|c|c|} % centered columns (5 columns)
    \hline\hline %inserts double horizontal lines
    C.P. & Acceleration equation & Phase of the Universe & Stability condition \\ [0.5ex] % inserts table %heading
    \hline\hline % inserts single horizontal line
    $\mathcal{P}_{1}$ & $\dot{H}=-2 H^2$ & $a(t)= t_{0} (2 t+c_{1})^\frac{1}{2}$ & Unstable  \\
     \hline
    $\mathcal{P}_{2}$ & $\dot{H}=-2 H^2$ & $a(t)= t_{0} (2 t+c_{1})^\frac{1}{2}$ & Unstable  \\
     \hline
    $\mathcal{P}_{3}$ & $\dot{H}=0$ & $a(t)=t_0 e^{c_1 t}$ & Stable \\
   \hline
    $\mathcal{P}_{4}$ & $\dot{H}=-2 H^2$ & $a(t)= t_{0} (2 t+c_{1})^\frac{1}{2}$ & Unstable \\
    \hline
    $\mathcal{P}_{5}$ & $\dot{H}=(-2+x_7) H^2$ & $a(t)=t_0 \left((2-x_7)t+c_1\right)^\frac{1}{2-x_7}$ & Stable \\
    \hline
    \end{tabular}
     \caption{Phase of the Universe with stability conditions} % title of Table
    \label{TABLE-V}
\end{table*}
\begin{itemize}
    \item {\textbf{Point $\mathcal{P}_2$:}} Table \ref{TABLE-III} shows that the second critical point $\mathcal{P}_2$ exists for $3+2 x_4 \neq 0$ and $\beta=\frac{1}{2}$. At this point, the Universe is in a radiation-dominated phase with $\Omega_r=x_4$, $\Omega_m=0$, and $\Omega_{de}=1-x_4$. This is further evidenced by the EoS parameter ($\omega_{\text{tot}}$) being equal to $\frac{1}{3}$ and the deceleration parameter $q$ having a value of 1. The Jacobian matrices associated with these critical points have real positive and negative parts and zero eigenvalues, indicating that it has an unstable saddle behavior.\\
    \item {\textbf{Point $\mathcal{P}_3$:}} Under the conditions in Table \ref{TABLE-III}, this point $\mathcal{P}_3$ corresponds to a Universe dominated by dark energy. Since it is stable under the conditions shown, it can be considered a late-time state of the Universe. Interestingly, under conditions with $-1+4 \beta \neq 0$, the equation parameter of state ($\omega_{\text{tot}}$) equals the value of the cosmological constant $-1$ at this critical point, where $\Omega_{de} = 1$, $\omega_{\text{tot}} = -1$, and the deceleration prameter $q = -1$ are the background levels. Since these features are compatible with observations, they are a great advantage of the scenario under consideration; furthermore, they can only be obtained by using $F(R,\mathcal{G})$ gravity without explicitly including a cosmological constant or a canonical or phantom scalar field.\\
    \item {\textbf{Point $\mathcal{P}_4$:}} This critical point exists in a radiation-dominated Universe for $-1+x_6 \neq 0$, $-1+2 x_6 \neq 0$ and $\beta =\frac{-3-x_6}{8(-1+x_6)}$, leading to a decelerating phase of the Universe with an equation of state parameter $\omega_{\text{tot}} =\frac{1}{3}$ and deceleration parameter $q = 1$. The corresponding density parameters are $\Omega_r=0$, $\Omega_r=0$, and $\Omega_r=1$. The eigenvalues associated with this critical point reveal positive and negative signs by taking some restrictions on $x_6$, indicating that it is an unstable node.\\
    \item {\textbf{Point $\mathcal{P}_5$:}} At late times, Point $\mathcal{P}_5$ could attract the Universe due to its stability under the conditions presented in Table \ref{TABLE-III}. There are similarities between it and point $\mathcal{P}_3$, but there are differences in parameter regions. In particular, it suggests an accelerating Universe dominated by dark energy. The negative value of the deceleration parameter indicates the accelerating phase of the Universe, and $\omega_{\text{tot}} = -1$ behaves as a cosmological constant at this critical point, where $\Omega_{de} = 1$ and $q = -1$ are the background levels. The corresponding eigenvalue is\\ $\{0, \frac{-6(1-2 x_7+x_7^2)}{14-12 x_7+3 x_7^2}, -4(-1+x_7), (5-4x_7) \}$.
\end{itemize}

\begin{figure}[H]
    \centering
    \includegraphics[width=80mm]{evolution_curve.pdf}
    \caption{Evolution of density parameters DE (magenta line), matter (blue line) and radiation (cyan line) for the initial conditions: $u_3=10^{-9.45}$, $u_4=0.01$, $u_6=1.28999$, $u_7=0.448 \times 10^{-1.2}$.} 
    \label{FIG6}
\end{figure}

Fig. \ref{FIG6} shows the cosmic evolution of the density parameter for matter, radiation, and dark energy for the model (\ref{14}), where the initial conditions $u_3=10^{-9.45}$, $u_4=0.01$, $u_6=1.28999$ and $u_7=0.448 \times 10^{-1.2}$ are taken. The behavior is consistent with recent cosmic observations on the evolution of density parameters. To obtain the current densities, $\Omega_m \approx 0.28$, $\Omega_{de} \approx 0.679$, and $\Omega_r \approx 0.047$ were calculated. Radiation dominance is shown in Fig. \ref{FIG6} at the beginning, followed by a brief phase of matter dominance and, at the end, the de-Sitter phase.

\section{Conclusion} \label{SEC-V}

This study aims to investigate the cosmological behavior of a specific class of modified Gauss-Bonnet gravity models. In order to achieve this, we first developed a general description of a gravitational action involving a Ricci scalar and a Gauss-Bonnet invariant. The parameterization method also discussed the effective equation of state ($\omega_\text{eff}$) for $F(R, \mathcal{G})$ gravity models. The Hubble parameter formula coefficients have been constrained using the CC sample, the largest \textit{Pantheon$^+$} and the BAO dataset. The values of the model parameters that best match the data are displayed in Table \ref{TABLE I}, following the testing of our cosmological solutions in Section \ref{SEC-III}. We derive the deceleration parameter $q$ from our constrained values, and our derived model shows a smooth transition from a decelerating Universe expansion phase to an accelerated Universe expansion phase. For CC, \textit{Pantheon$^+$}, CC + \textit{Pantheon$^+$} and CC + \textit{Pantheon$^+$} + BAO data, the transition redshifts are $z_t=0.636$, $z_t=0.656$, $z_t=0.691$ and $z_t=0.74$, respectively. The EoS parameter shows that the expansion of the Universe has increased since it is within the phantom region for $z \leq -0.015$. Our result for the effective EoS parameter at $z = 0$ is $-0.8478$, $-1.02$, $-1.224$ and $-1.47$ for CC, \textit{Pantheon$^+$}, CC + \textit{Pantheon$^+$} and CC + \textit{Pantheon$^+$} + BAO datasets respectively, is mostly in line with the current observational findings.

Considering this, we focused our analysis on a specific type of function $F(R, \mathcal{G})$, which allowed us to analyze global behavior and stability in terms of the cosmological model. We found some interesting preliminary findings for the finite phase space of a power-law class of fourth-order gravity models $\mathcal{F}(R, \mathcal{G})=\alpha R^2 \mathcal{G}^\beta$. There are a total of five critical points obtained, two of which ($\mathcal{P}_3, \mathcal{P}_5$) are stable and five ($\mathcal{P}_1, \mathcal{P}_2, \mathcal{P}_4$) of which are unstable. During the de-Sitter phase of the Universe, stable critical points appeared, whereas unstable behavior was observed during the radiation-dominated phase. A signature of eigenvalues and a phase-space portrait support the behavior of critical points. The trajectory behavior indicates that the unstable critical points act as release points while the stable ones act as attractor points. We confirm the accelerating model when the EoS and deceleration parameters are $-1$. Eqns. (\ref{39})-(\ref{42}) presents the dynamical system for the mixed power law $\mathcal{F}(R, \mathcal{G})$ gravity model. Table-\ref{TABLE-III} provides critical points and existing conditions for the model. At the same time, Table \ref{TABLE-IV} presents a value for the deceleration, EoS, and density parameters are $\omega_{\text{tot}}$, $\Omega_{m}$, $\Omega_{r}$ and $\Omega_{de}$ for studying the stability of critical points at various phases of Universe evolution. It is consistent with recent cosmic observations regarding density parameter evolution. Current densities were obtained by subtracting $\Omega_m \approx 0.28$, $\Omega_{de} \approx 0.679$, and $\Omega_r \approx 0.047$ from the above equation. Initially, Figure \ref{FIG6} shows radiation dominance, followed by matter dominance, and finally, de-Sitter dominance.

\begin{widetext}
\begin{appendix}
\section*{Appendix}
\begin{table}[H]
\small\addtolength{\tabcolsep}{-2pt}
\centering % used for centering table
\begin{tabular}{c c c c c | c c c c c | c c c c c} % centered columns (10 columns)
\hline\hline %inserts double horizontal lines
No. & $z_{i}$ & $H(z)$ & $\sigma_{H(z)}$ & Ref. & No. & $z_{i}$ & $H(z)$ & $\sigma_{H(z)}$ & Ref. &  No. & $z_{i}$ & $H(z)$ & $\sigma_{H(z)}$ & Ref.\\ [0.5ex] % inserts table
%heading
\hline % inserts single horizontal line
1. & 0.070 & 69.00 & 19.6 & \cite{Zhang_2014_14} & 12. & 0.400 & 95.00 & 17.00 &  \cite{Simon_2005_71} & 23. & 0.875 & 125.00 & 17.00 &  \cite{Moresco_2012_2012_006}\\ 
2. & 0.090 & 69.00 & 12.0 & \cite{Simon_2005_71} & 13. & 0.4004 & 77.00 & 10.20 &  \cite{Moresco_2016_2016_014} & 24. & 0.880 & 90.00 & 40.00 & \cite{Stern_2010_2010_008}\\
3. & 0.120 & 68.60 & 26.2 & \cite{Zhang_2014_14} & 14. & 0.425 & 87.10 & 11.20 &  \cite{Moresco_2016_2016_014}  & 25. & 0.900 & 117.00 & 23.00 &  \cite{Simon_2005_71}\\
4. & 0.170 & 83.00 & 8.00 &  \cite{Simon_2005_71}  & 15. & 0.445 & 92.80 & 12.90 &  \cite{Moresco_2016_2016_014}  & 26. & 1.037 & 154.0 & 20.00 &  \cite{Moresco_2012_2012_006}\\
5. & 0.179 & 75.00 & 4.00 & \cite{Moresco_2012_2012_006} & 16. & 0.47 & 89.00 & 49.60 &  \cite{Ratsimbazafy_2017_467}  & 27. & 1.300 & 168.0 & 17.00 &  \cite{Simon_2005_71} \\
6. & 0.199 & 75.00 & 5.00 &  \cite{Moresco_2012_2012_006} & 17. & 0.4783 & 80.90 & 9.00 &  \cite{Moresco_2016_2016_014}  & 28. & 1.363 & 160.00 & 33.60 &  \cite{Moresco_2015_450} \\
7. & 0.200 & 72.90 & 29.60 &  \cite{Zhang_2014_14} & 18. & 0.48 & 97.00 & 62.00 &  \cite{Stern_2010_2010_008}  & 29. & 1.430 & 177.0 & 18.00 &  \cite{Simon_2005_71} \\ 
8. & 0.270 & 77.00 & 14.00 &  \cite{Simon_2005_71}  & 19. & 0.593 & 104.00 & 13.00 &  \cite{Moresco_2012_2012_006} & 30. & 1.530 & 140.0 & 14.00 &  \cite{Simon_2005_71}\\
9. & 0.280 & 88.80 & 36.60 & \cite{Zhang_2014_14} & 20. & 0.680 & 92.00 & 8.00 &  \cite{Moresco_2012_2012_006} & 31. & 1.750 & 202.0 & 40.00 & \cite{Simon_2005_71}\\ 
10. & 0.352 & 83.00 & 14.00 &  \cite{Moresco_2012_2012_006}& 21. & 0.750 & 98.80 & 33.60 &  \cite{Borghi_2022_928} & 32. & 1.965 & 186.5 & 50.4 & \cite{Simon_2005_71}\\
11. & 0.380 & 83.00 & 13.50 &  \cite{Moresco_2016_2016_014} & 22. & 0.781 & 105.00 & 12.00 & \cite{Moresco_2012_2012_006} & &  &  &  & \\[0.5ex] % [1ex] adds vertical space
\hline %inserts single line
\end{tabular}
\caption{$H(z)$ measurements were made using the CC technique, expressed in [$km\, s^{-1} Mpc^{-1}$] units, along with the corresponding errors.} % title of Table
\label{table: Table II} % is used to refer this table in the text
\end{table}
\end{appendix}
\end{widetext}
\section*{Acknowledgement} SVL acknowledges the financial support provided by University Grants Commission (UGC) through Senior Research Fellowship  (UGC Ref. No.:191620116597) to carry out the research work. BM acknowledges the support of IUCAA, Pune (India) through the visiting associateship program.

\bibliographystyle{utphys}
\bibliography{ref_short}
\end{document}