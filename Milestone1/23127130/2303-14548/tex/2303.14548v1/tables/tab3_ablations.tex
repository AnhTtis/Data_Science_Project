\begin{table*}
  \setlength\tabcolsep{4pt}
  \label{tab:ablations}
  \hfill%
  \subfloat[\textbf{Fourier frequency.}]{
    \label{tab:ablate_freq}
    \begin{tabular}{lcc}
    \toprule
     $f_{\mathrm{max}}$ &   mAP$\uparrow$ & NDS$\uparrow$ \\
    \midrule
       4 & 0.446 & 0.521 \\
       \rowcolor{lightgray} 8 & \textbf{0.451} & \textbf{0.527} \\
       16 & 0.447 & 0.521 \\
       32 & 0.442 & 0.517 \\
    \bottomrule
    \end{tabular}}
  \hfill%
  \subfloat[\textbf{\# Fourier bands.}]{
    \label{tab:ablate_bands}
    \begin{tabular}{lcc}
    \toprule
     $k$ &   mAP$\uparrow$ & NDS$\uparrow$ \\
    \midrule
       32 & 0.446 & 0.522 \\
       \rowcolor{lightgray} 64 & \textbf{0.451} & \textbf{0.527} \\
       96 & 0.445 &  0.520 \\
       128 & 0.438 &  0.512 \\
    \bottomrule
    \end{tabular}}
  \hfill%
  \subfloat[\textbf{\# virtual views}]{%
    \label{tab:ablate_views}
    \resizebox{0.31\columnwidth}{!}{
    \begin{tabular}{lcc}
    \toprule
     $V$ &   mAP$\uparrow$ & NDS$\uparrow$ \\
    \midrule
       0 & 0.432 & 0.495 \\
       1 & 0.441 & 0.518 \\
       \rowcolor{lightgray} 2 & \textbf{0.451} & \textbf{0.527} \\
       4 & 0.448 & 0.526 \\
       6 & 0.437 & 0.511 \\
    \bottomrule
    \end{tabular}}}
  \hfill%
  \subfloat[\textbf{virtual view weights.}]{
    \label{tab:ablate_weights}
    \begin{tabular}{lcc}
    \toprule
     $\lambda_{\mathrm{v}}$ &   mAP$\uparrow$ & NDS$\uparrow$ \\
    \midrule
       0.1 & 0.442 & 0.522 \\
       \rowcolor{lightgray} 0.2 & \textbf{0.451} & \textbf{0.527} \\
       0.4 & 0.439 & 0.517 \\
       0.6 & 0.431 & 0.499 \\
    \bottomrule
    \end{tabular}}
  \hfill%
  \vspace{12pt}
  \hfill
  \subfloat[\textbf{components addition.}]{
    \label{tab:ablate_component_add}
    \hspace{30pt}
    \resizebox{0.99\columnwidth}{!}{
    \begin{tabular}{lccccc}
    \toprule
     \#  & Fourier+MLP GPE. & 2 virtual views & 2-frame &  mAP$\uparrow$ & NDS$\uparrow$ \\
    \midrule
      1 & & & & 0.404 & 0.447 \\
      2 & \checkmark & & & 0.420 & 0.464 \\
      3 & \checkmark & \checkmark & & 0.434 & 0.488 \\
      4 & \checkmark & & \checkmark & 0.432 & 0.495 \\
     \rowcolor{lightgray} 5 & \checkmark & \checkmark & \checkmark & \textbf{0.451} & \textbf{0.527} \\
    \bottomrule
    \end{tabular}}}
  \hfill%
  \subfloat[\textbf{components variation}]{%
    \label{tab:ablate_component_var}
    \resizebox{0.49\columnwidth}{!}{
    \begin{tabular}{lcc}
    \toprule
     method &   mAP$\uparrow$ & NDS$\uparrow$ \\
    \midrule
       \rowcolor{lightgray} \Acronym & \textbf{0.451} & \textbf{0.527} \\
       no Fourier & 0.382 & 0.476 \\
       no $\mathbf{\bar{q}}$ & 0.436 & 0.515 \\
       no $\mathbf{t}$ & 0.353 & 0.453 \\
       no joint match & 0.425 & 0.483 \\
    \bottomrule
    \end{tabular}}}
  \hfill%

  \hfill
  \caption{\textbf{Ablation studies and analyses.} In \cref{tab:ablate_freq,tab:ablate_bands,tab:ablate_views,tab:ablate_weights} we first analyze important hyper-parameters relevant to the algorithm, and choose the ones giving the best performance. In \cref{tab:ablate_component_add} we ablate and show the importance of the proposed geometric embeddings and virtual views. They not only monotonically bring improvements to the 3D detection performance, but also mutually benefit adding time frames. In \cref{tab:ablate_component_var} we ablate some variations in the components, showing the critical usage of Fourier encoding and some optimal settings for queries during training and inference.}
  \end{table*}
