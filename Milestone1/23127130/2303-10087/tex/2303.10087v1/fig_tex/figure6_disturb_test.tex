\begin{figure}[t]
    \centering
   \begin{subfigure}{.5\linewidth}
        \centering
       \includegraphics[width=\linewidth]{pdf/head_rot.pdf}
       \vspace{-18pt}
       \caption{\small{Rotation test (Indoor)}}
    \vspace*{6pt}
   \end{subfigure}%
   \begin{subfigure}{.5\linewidth}
       \centering
       \includegraphics[width=\linewidth]{pdf/head_trans.pdf}
       \vspace{-18pt}
       \caption{\small{Translation test (Indoor)}}
    \vspace*{6pt}
   \end{subfigure}
     \begin{subfigure}{.5\linewidth}
        \centering
       \includegraphics[width=\linewidth]{pdf/shop_rot.pdf}
       \vspace{-18pt}
       \caption{\small{Rotation test (Outdoor)}}
   \end{subfigure}%
   \begin{subfigure}{.5\linewidth}
       \centering
       \includegraphics[width=\linewidth]{pdf/shop_trans.pdf}
       \vspace{-18pt}
       \caption{\small{Translation test (Outdoor)}}
   \end{subfigure}
\caption{
Experiments on pose refinement bounds of our method in indoor and outdoor scenes. Each plot shows errors before (x-axis) and after (y-axis) refinement when ground-truth pose is perturbed by varying magnitudes.\ Dashed \textcolor{OliveGreen}{green} line is `$y\!\!=\!\!x$'. Points below this line indicate a reduction in pose error using our refinement method.
%Plots of the tests of model's capability on correcting rotational and translational errors. We plot the error after refinement against the error before refinement. The \textcolor{OliveGreen}{green} line is the reference line. Point below the reference line means that our method has a positive effect on the initial error. 
% Point below the reference line means that our method has a positive effect on the initial error. 
%\YB{Quick comment: can we make lines thicker and points larger here? also, better to use brighter colours like magenta or red}
}
\label{fig:perturb_test}
\end{figure}