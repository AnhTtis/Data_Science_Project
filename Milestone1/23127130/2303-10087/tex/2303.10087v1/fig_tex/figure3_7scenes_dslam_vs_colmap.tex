\begin{figure}[t]
    \centering
   \begin{subfigure}{0.4\linewidth}
        \centering
       \includegraphics[width=\linewidth]{fig/000_dslam.png}
   \end{subfigure}%
   \begin{subfigure}{0.4\linewidth}
       \centering
       \includegraphics[width=\linewidth]{fig/000_colmap.png}
       % \caption{Colmap NeRF}
       %\label{fig:three sin x}
   \end{subfigure}\vspace{-1mm}
   \begin{subfigure}{0.4\linewidth}
       \centering
       \includegraphics[width=\linewidth]{fig/000_disp_dslam.png}
       \caption{dSLAM NeRF}
      %  \label{fig:y equals x}
   % \end{subfigure}&\begin{subfigure}[b]{.5\columnwidth}
   \end{subfigure}%
   \begin{subfigure}{0.4\linewidth}
       \centering
       \includegraphics[width=\linewidth]{fig/000_disp_colmap.png}
       \caption{SfM NeRF}
       %\label{fig:three sin x}
   \end{subfigure}
  % \end{tabular}
% \caption{Comparison between the NeRF rendering quality between dSLAM GT training pose and SfM GT training pose. As illustrated, SfM NeRF (PSNR 19.94 dB) can render superior geometric details (bottom row) than dSLAM NeRF (PSNR 16.11 dB).}

\caption{Qualitative comparison between the NeRFs trained by dSLAM GT pose (a) vs. SfM GT pose (b). As illustrated, SfM NeRF (PSNR 19.94 dB) can render superior geometric details (bottom row) than dSLAM NeRF (PSNR 16.11 dB).}


%We hypothesis the original NeRF suffers from severe artifacts because 1. the camera pose of the scene is outward-looking; 2. the training set and testing set are separate trajectory sequences; 3. the training set contains blurry and deformable contents.(PSNR 19.94 dB) (PSNR 16.11 dB)
\label{fig:7scenes_dslam_vs_colmap}
\end{figure}
