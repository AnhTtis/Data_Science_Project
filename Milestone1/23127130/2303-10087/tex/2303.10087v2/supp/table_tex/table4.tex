\begin{table}[t]
\centering
% \resizebox{\linewidth}{!}{
\small
\begin{tabular}{c|lc}
\toprule
&LR Settings & Shop-20\% (+NeFeS$_{20}$) \\
\midrule
&Initial Pose Error          & 0.58m/3.14$\degree$\\
% \midrule
\multirow{5}{*}{\makecell[c]{Same lr}}&$lr_{R}=lr_t=0.1$              & 0.91m/22.70$\degree$\\
&$lr_{R}=lr_t=0.01$             & 0.49m/1.51$\degree$\\
&$lr_{R}=lr_t=0.003$            & 0.54m/2.44$\degree$\\
&$lr_{R}=lr_t=0.001$         & 0.57m/2.48$\degree$\\
\midrule
\multirow{1}{*}{\makecell[c]{Different lr}}&$lr_R=0.01$, $lr_t=0.1$     & \boldred{0.27m}/\boldred{1.77$\degree$}\\
\bottomrule
\end{tabular}
% }
\caption{We use a toy example to show the benefit of using \textit{different} learning rates over \textit{same} learning rates for translation and rotation components during direct pose refinement. We show four cases for \textit{same} learning rate including two settings that are used in prior works. Our pose refinement results are evaluated by using $20$\% test data of \textit{Cambridge: Shop Facade} and $20$ iterations of optimization.}
\label{supp:table:split-LRs}
\end{table}
