\section{Related Work} \label{sec:related-work}

Group-level analytics with data about people involves data mining techniques to identify groups using observed individual features, with a variety of visual analytic techniques proposed to help users understand the identified groups. This has been joined by the recent progress in explainable AI or interpretable machine learning, with a goal to enable human users to understand the data-driven decision-making processes. We briefly review works that are the most related to ours in the three areas.

\subsection{Group identification from observed data}
Online user-generated contents such as users' social media posts and activities have been increasingly used in many domains for understanding the characteristics of different sociodemographic groups. Groups can be characterized in a deductive or inductive manner. The deductive approach starts with certain hypotheses (e.g., groups are different in terms of some attributes) and uses top-down or supervised learning methods to find groups with certain attributes \cite{phillips2017using}. On the other hand, unsupervised learning methods, such as community detection, are used to group individuals with similar patterns without presumptions of the group characteristics \cite{lin2009metafac,tang2011group}. In the context of social media data where the hypothesized attributes are not readily available, text mining techniques have been utilized to extract more explicit attributes into theorized constructs to analyze individuals, with the use of sociolinguistic features, words and hashtags \cite{an_greysanatomy_nodate, carpenter_real_2017, conover2011political} and other social media behavior \cite{bamman_gender_2014, filippova_user_nodate, wood-doughty_how_2017}. Other types of data, such as demographic metadata \cite{sloan_who_2015} or network structure \cite{rao_classifying_2010}, may be used in addition to the user-generated text based features, such as linguistic occurrences based on LIWC dictionary \cite{fink_inferring_nodate}, words \cite{burger_discriminating_nodate, jensen2012political}, or embeddings \cite{hovy_demographic_2015, demszkyAnalyzingPolarizationSocial2019}, to algorithmically identify target groups, including predicting gender \cite{fink_inferring_nodate, sap_developing_2014}, age \cite{guimaraes_age_2017, lopez_predicting_2017} and political preferences \cite{cohen_classifying_nodate, volkova_inferring_2014, volkova2015online}.

Many of the features extracted from user-generated text, such as the use of specific words and phrases, may not have a theoretical justification, which hinders the interpretation of the generated groups -- for example, whether a group has characteristics as previously hypothesized. Several works have especially focused on extracting the interpretable, theoretically-grounded, or higher-level differences between groups, such as sentiment and emotion \cite{volkova_predicting_2015, volkova_inferring_2016}, personality \cite{golbeck_predicting_nodate, wei_beyond_2017}, or moral values \cite{kalimeri_predicting_2019}. For example, Svitlana et al. \cite{volkova_inferring_2016} examined how users' emotional dimensions are different from their social contacts and found such differences are discriminative in predicting socio-demographic groups such as ethnicity and gender. Kalimeri et al. \cite{kalimeri_predicting_2019} analyzed the psychometric questionnaires and web browsing behavior to infer moral trait and human values.

Such inferred, higher-level variables may create new challenge in interpretation as the variables may not correspond to a particular example in the raw text. In this work, we introduce an analytic tool that bridges the high-level, theoretically-grounded attributes and the low-level, language instances that enables users to find evidence directly from user-generated text for how the inferred attributes reflect the concept it purports to measure.


\subsection{Visual analytics for group difference}
Visual analytics has been designed to support group-level analysis in many application domains. Typical analysis tasks include 
(1) selecting a group of people that meet certain criteria (e.g., constructing patient cohorts to support cohort analysis or intervention design) \cite{klemm_interactive_2014, kwon_retainvis_2019, zhang2015iterative,krause_supporting_2016}, (2) identifying certain socio-demographic groups, or (3) clustering people using observed data over multivariate attributes \cite{cavallo_clustrophile_2019, kwon_clustervision_2018, pearlman2007visualizing, pham2010visualization, pham2014visualization}. For example, Klemm et al. \cite{klemm_interactive_2014} visualized the averaging medical images of groups with respect to socio-demographic metadata or clustering analysis. Interactive techniques are introduced in sub-setting cohorts who have certain temporal constraints \cite{krause_supporting_2016, zhang2015iterative}. Krause et al. \cite{krause_supporting_2016} proposed a visual filtering layout for specifying cohorts, while CAVA \cite{zhang2015iterative} proposed a flow visualization where a subset of cohorts can be aggregated and divided by features over discrete timepoints.

Although most visualization works support the interactive exploration of group patterns, fewer tools focus on supporting both the group classification and group differences. One such work is DemographicVis \cite{dou_demographicvis_2015}, which proposed a visual text analytics system for exploring the group inference task and the underlying topical patterns of user-generated contents across demographic groups. While this work showed that a group classification model may be improved by adding linguistic and topical features, the process of group inference and the identified important features are not interpretable to the end-users. Some of studies on visual analytic tools with the classification task investigate the interative subgrouping but with the focus of discovering bias and model failure \cite{ahn2019fairsight, cabrera2019fairvis, gleicher2020boxer, wexler2019if, chung2019slice}. In this work, we introduce a new visual analytic suite that enables not only the exploration of group patterns but also the reasoning of important attributes and of individuals in a group.

\subsection{Deep learning and interpretable machine learning}
Deep neural networks has been shown to help with identifying group differences, such as sentiment \cite{can2018multilingual,chen2018twitter}, emotion \cite{abdullah2018sedat,gupta2017sentiment}, affect \cite{abid2019sentiment}, and morality \cite{rezapour2019enhancing}. The flexible architectures in deep neural networks make it effective in taking diverse types of data input ranged from structured variables, text \cite{sundermeyer2012lstm}, to image \cite{kalchbrenner2014convolutional}, into predictive features. For example, Recurrent Neural Networks (RNN) \cite{sundermeyer2012lstm}  can take a sequence of tokens (words or phrases) from natural languages as input, and recursively encode and transformed the textual input to a latent space while keeping the language structures. However, the transformed features resulted from the learning layers are often difficult to interpret. Thus, how to enhance the interpretability of neural network models has drawn considerable attention. One example technique is the use of the attention mechanism, first introduced by Cho et al. \cite{cho2014learning} on RNNs for machine translation. In this approach, the recursively encoded tokens are weighted in the learning of the model, so that contributions of individual token inputs to the model prediction are discoverable through the weights. The attention mechanism has been extended in different contexts. Xu et al. \cite{xu2015show} used attention to identify informative pixel areas from images. Hermann et al. \cite{hermann2015teaching} used attention for the extraction of important sections from text paragraphs and Bahdanau et al. \cite{bahdanau2014neural} used attention for word alignment in machine translation.

In this work, we introduce a new multi-task prediction task that leverages RNN architecture and attention mechanism with a joint objective to simultaneously predict group labels and attributes. As a result, the attention mechanism enables the finding of language cues as evidence to support the interpretation of groups' sociolinguistic attributes.
