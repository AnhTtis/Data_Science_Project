
%%
%% This is file `sample-acmsmall.tex',
%% generated with the docstrip utility.
%%
%% The original source files were:
%%
%% samples.dtx  (with options: `acmsmall')
%% 
%% IMPORTANT NOTICE:
%% 
%% For the copyright see the source file.
%% 
%% Any modified versions of this file must be renamed
%% with new filenames distinct from sample-acmsmall.tex.
%% 
%% For distribution of the original source see the terms
%% for copying and modification in the file samples.dtx.
%% 
%% This generated file may be distributed as long as the
%% original source files, as listed above, are part of the
%% same distribution. (The sources need not necessarily be
%% in the same archive or directory.)
%%
%% The first command in your LaTeX source must be the \documentclass command.
\documentclass[acmsmall]{acmart}

\usepackage[boxruled, linesnumbered]{algorithm2e}
\usepackage{xspace}

%%
%% \BibTeX command to typeset BibTeX logo in the docs
\AtBeginDocument{%
  \providecommand\BibTeX{{%
    \normalfont B\kern-0.5em{\scshape i\kern-0.25em b}\kern-0.8em\TeX}}}

%% Rights management information.  This information is sent to you
%% when you complete the rights form.  These commands have SAMPLE
%% values in them; it is your responsibility as an author to replace
%% the commands and values with those provided to you when you
%% complete the rights form.
\setcopyright{acmlicensed}
\copyrightyear{2021}
\acmYear{2021}
\acmDOI{10.1145/1122445.1122456}

%%
%% These commands are for a JOURNAL article.
\acmJournal{TIIS}
\acmVolume{1} 
\acmNumber{1} 
\acmArticle{1} 
\acmMonth{1}
\acmPrice{15.00}
\acmDOI{10.1145/3484509}

%%
%% Submission ID.
%% Use this when submitting an article to a sponsored event. You'll
%% receive a unique submission ID from the organizers
%% of the event, and this ID should be used as the parameter to this command.
%%\acmSubmissionID{123-A56-BU3}

%%
%% The majority of ACM publications use numbered citations and
%% references.  The command \citestyle{authoryear} switches to the
%% "author year" style.
%%
%% If you are preparing content for an event
%% sponsored by ACM SIGGRAPH, you must use the "author year" style of
%% citations and references.
%% Uncommenting
%% the next command will enable that style.
%%\citestyle{acmauthoryear}

%%
%% end of the preamble, start of the body of the document source.
\begin{document}

%%
%% The "title" command has an optional parameter,
%% allowing the author to define a "short title" to be used in page headers.
\title{Tribe or Not? Critical Inspection of Group Differences Using TribalGram}

%%
%% The "author" command and its associated commands are used to define
%% the authors and their affiliations.
%% Of note is the shared affiliation of the first two authors, and the
%% "authornote" and "authornotemark" commands
%% used to denote shared contribution to the research.
\author{Yongsu Ahn}
\affiliation{%
  \institution{University of Pittsburgh}
  \city{Pittsburgh}
  \country{United States}}
\email{yongsu.ahn@pitt.edu}

\author{Muheng Yan}
\affiliation{%
  \institution{University of Pittsburgh}
  \city{United States}
  \country{United States}}
  \email{muheng.yan@pitt.edu}

\author{Yu-Ru Lin}
\affiliation{%
  \institution{University of Pittsburgh}
  \city{United States}
  \country{United States}}
  \email{yurulin@pitt.edu}
  
\author{Wen-Ting Chung}
\affiliation{%
  \institution{University of Pittsburgh}
  \city{United States}
  \country{United States}}
  \email{wtchung@pitt.edu}
  
\author{Rebecca Hwa}
\affiliation{%
  \institution{University of Pittsburgh}
  \city{United States}
  \country{United States}}
  \email{hwa@pitt.edu}

%%
%% By default, the full list of authors will be used in the page
%% headers. Often, this list is too long, and will overlap
%% other information printed in the page headers. This command allows
%% the author to define a more concise list
%% of authors' names for this purpose.
\renewcommand{\shortauthors}{Ahn et al.}

%%
%% The abstract is a short summary of the work to be presented in the
%% article.
%% Comments
\newcommand{\ys}[1]{{\color{blue}{[YS: #1]}}}
\newcommand{\panpan}[1]{{\color{green}{[PX: #1]}}}
\newcommand{\rev}[1]{{\color{blue}{#1}}}
\newcommand{\yrl}[1]{{\color{red}{[YL: #1]}}}

\renewcommand{\vec}[1]{\mathbf{#1}}
\newcommand{\mat}[1]{\mathbf{#1}}

%% System
\newcommand{\name}{\texttt{ESCAPE}\xspace}
\newcommand{\namee}{\texttt{TribalGram}\xspace}

%% Components
\newcommand{\diagnosisview}{\texttt{Misclassification Dignosis View}\xspace}
\newcommand{\performancechart}{\texttt{Performance Chart}\xspace}
\newcommand{\instancespace}{\texttt{Instance Space}\xspace}
\newcommand{\confidencescorefilter}{\texttt{Confidence Score Filter}\xspace}
\newcommand{\confusionmatrix}{\texttt{Confusion Matrix View}\xspace}
\newcommand{\confusionspace}{\texttt{Confusion Space}\xspace}
\newcommand{\contrastiveview}{\texttt{Contrastive Analysis View}\xspace}
\newcommand{\conceptlist}{\texttt{Concept List}\xspace}
\newcommand{\segmentview}{\texttt{Segment View}\xspace}

\newcommand{\conceptinspectionview}{\texttt{Concept Inspection View}\xspace}
\newcommand{\conceptassociationplot}{\texttt{Concept Association Plot}\xspace}
\newcommand{\conceptdetailview}{\texttt{Concept Detail View}\xspace}
\newcommand{\debiasview}{\texttt{Debias View}\xspace}
\newcommand{\debiasplot}{\texttt{Debias Plot}\xspace}



\begin{abstract}
% \vspace{-1em}
The diffusion-based generative models have achieved remarkable success in text-based image generation. However, since it contains enormous randomness in generation progress, it is still challenging to apply such models for real-world visual content editing, especially in videos. 
In this paper, we propose \texttt{FateZero}, a zero-shot text-based editing method on real-world videos without per-prompt training or use-specific mask. 
\RM{Specifically, different from a pipeline of two independent inversion and then generation stages, we find the intermediate attention maps during inversions store better structure and motion information. We thus reform them to temporally casual attention and replace them in the generation progress. To further reduce the unnecessary semantic leakage of source video and enhance the editing quality, we then remix the temporally casual attentions via the cross-attention features of the source prompt as the mask.}
To edit videos consistently, we propose several techniques based on the pre-trained models. Firstly, in contrast to the straightforward DDIM inversion technique, our approach captures intermediate attention maps during inversion, which effectively retain both structural and motion information. These maps are directly fused in the editing process rather than generated during denoising. To further minimize semantic leakage of the source video, we then fuse self-attentions with a blending mask obtained by cross-attention features from the source prompt. Furthermore, we have implemented a reform of the self-attention mechanism in denoising UNet by introducing spatial-temporal attention to ensure frame consistency.
Yet succinct, our method is the first one to show the ability of zero-shot text-driven video style and local attribute editing from the trained text-to-image model. We also have a better zero-shot shape-aware editing ability based on the text-to-video model~\cite{tuneavideo}. \RM{Besides video, our unified method also achieves state-of-the-art performance in zero-shot image editing.\chenyang{Need exp or remove the zero-shot image}} Extensive experiments demonstrate our superior temporal consistency and editing capability than previous works.
% The code will be released.
% \chenyang{emphasize: our observation at inversion time} \xiaodong{replacing the bold part to the actual pipeline: \textbf{Specifically, we work on replacing and mixing the attention maps between the inversion and generation since the self-attention map keeps the structure of the original natural image and the cross-attention is semantic-related, after remixing, we replace them in the corresponding generation steps for denoising.}}
% \footnote{Since there is no general video diffusion model is publicly available, we use one-shot video generation method~(Tune-A-Video~\cite{tuneavideo}) as the pretrained video diffusion model for zero-shot video editing\xiaodong{can be removed if we actually zero-shot on video}.}.
\end{abstract}

%%
%% The code below is generated by the tool at http://dl.acm.org/ccs.cfm.
%% Please copy and paste the code instead of the example below.
%%
\begin{CCSXML}
<ccs2012>
 <concept>
  <concept_id>10010520.10010553.10010562</concept_id>
  <concept_desc>Human-centered computing</concept_desc>
  <concept_significance>500</concept_significance>
 </concept>
 <concept>
  <concept_id>10003033.10003083.10003095</concept_id>
  <concept_desc>Visual Analytics</concept_desc>
  <concept_significance>100</concept_significance>
 </concept>
</ccs2012>
\end{CCSXML}

\ccsdesc[500]{Human-centered computing~Visual analytics}
\ccsdesc[500]{Human-centered computing~Interactive systems and tools}

%%
%% Keywords. The author(s) should pick words that accurately describe
%% the work being presented. Separate the keywords with commas.
\keywords{group analysis, group difference, group profiling, visual analytics, interpretable machine learning, contrastive explanation}

\begin{teaserfigure}
\centering
  \includegraphics[width=0.9\columnwidth]{figures/teaser.pdf}
  \caption{
  \name provides a critical inspection of group differences to facilitate accountable group analytics with a visual analytic suite. It allows users to (a) visually contrast groups' main characteristics (\grouptrend), (b) examine the within-group variance (\variscope), (c) capture group-level attribute importance (\depscope), (d) retrieve qualitative language details (\languagescope), and (e) explore nested patterns and diversity within a group. It also supports users to (g) assess the analytic model quality against ground-truth data whenever available (\evalscope), and (f) obtain the rationale behind the group assignment of any individual instance (\rationalescope).
  }
  \label{fig:teaser}
\end{teaserfigure}

\maketitle

\section{Introduction}

The ability to reason about plans is critical for performing long-horizon tasks \citep{erol1996hierarchical, sohn2018hierarchical, sharma-etal-2022-skill}, compositional generalization \citep{corona-etal-2021-modular} and generalization to unseen tasks and environments \citep{shridhar2020alfred}.
Consider a simple long-horizon planning scenario where a robot is tasked with preparing a meal and serving it on the table. 
This presents a non-trivial planning problem since the agent needs to understand the sequence of operations required to perform the task and search for the relevant objects in the unfamiliar environment by interacting with various objects. %



Large language models have been recently shown to possess commonsense knowledge about the world such as object affordances and physical dynamics \citep{ouyang2022training,chowdhery2022palm}.
Early approaches considered text based environments and fine-tuned PLMs to predict actions given the history of past observations and actions \citep{jansen-2020-visually,micheli-fleuret-2021-language,yao-etal-2020-keep}.
Recent work has used this ability to reason about plans from text instructions in simulated household environments with simplifying assumptions such as text-only environment observations or feedback \citep{huang2022language,ahn2022can,li2022pre,logeswaran-etal-2022-shot}.


We focus on \emph{visually grounded planning} with PLMs --- the ability to adapt plans based on interaction and visual feedback from the environment.
While PLMs have strong planning commonsense priors, predictions from a PLM may not be directly realizable in the environment since the observation and action spaces are unknown.
This requires \emph{grounding} the PLM in the environment and adapting it to observe visual feedback, which is highly non-trivial.
Some prior works assume the availability of a pre-trained affordance function \citep{ahn2022can} or a success detector \citep{mirchandani2021ella}.
Notably, SayCan \citep{ahn2022can} completely decouples the PLM from observation information by selecting actions that have both high affordability (through a pre-trained affordance model) and high PLM likelihood.
Although this partially addresses the grounding problem, the use of visual feedback for action affordance alone is limited.
Often an agent must choose one of many affordable actions using information from observations.
For example, a driving agent should re-navigate and possibly turn around when encountering a ``road closed'' sign, but both turning around and driving forward are indistinguishable to SayCan because they are both affordable and the PLM is blind to observations.

Another workaround explored in prior work is translating the information in the visual observations to text using a pre-trained captioning system \citep{shridhar2021alfworld,huang2022language}.
However, it can be difficult to faithfully describe an image in words and information is lost in this inherently noisy process, which limits the information available to the planner.



Recent work shows that PLMs can be adapted for various natural language tasks by inserting tunable embeddings or soft prompts at the input of the PLM (also called prompt tuning or prefix tuning)~\citep{li-liang-2021-prefix,lester-etal-2021-power}.
This approach also extends to multi-modal understanding tasks such as image captioning \citep{mokady2021clipcap} and VQA \citep{tsimpoukelli2021multimodal} where images are encoded as soft prompts and finetuned for the target task.
Transformer based architectures have also been successfully applied to offline Reinforcement Learning in recent work \citep{chen2021decision,janner2021offline,li2022pre,reid2022can}.

Taking inspiration from these works, we propose the simple approach of embedding visual observations (`visual prompts') and \textit{directly inserting them as PLM input embeddings}.
The visual encoder and PLM are jointly trained for the target task, an approach we call \textbf{\oursfull}~(\ours).
By teaching the PLM to use observations for planning in an end to end manner, we remove the dependency on external data such as captions and affordability information that was used in prior work.
We show that this simple approach performs better than prior PLM-based planning approaches on two embodied planning benchmarks based on ALFWorld~\citep{shridhar2021alfworld} and Virtualhome~\cite{puig2018virtualhome}.



\section{Related Work}

\subsection{Pattern discovery on systematic AI errors}

Systematic errors, sometimes coined as blind spots or unknown-unknowns \cite{BeatMachineChallengingHumans}, refer to model's failure over a group of instances that share similar semantics. There are various approaches for discovering such patterns, including algorithmic, human, or hybrid techniques.

A number of studies have shown that fully algorithmic techniques can help automatically discover unknown-unknowns \cite{lakkaraju2017identifying, coveragebasedutility}. Recently, several studies have also been proposed to advance the methods towards discovering automatic slices or subclasses that are semantically coherent \cite{DominoDiscoveringSystematicErrorsCrossModal, SpotlightGeneralMethodDiscoveringSystematic}, or to propose a framework for evaluating blindspot discovery methods in a unified manner \cite{EvaluatingSystemicErrorDetectionMethods}.

On the other hand, researchers have also explored how human intelligence can identify blind spots where automatic techniques alone do not work. Several studies \cite{BeatMachineChallengingHumans, ContradictMachineHybridApproach, HybridHumanAIWorkflowsUnknown, InvestigatingHumanMachineComplementarity} demonstrated that a well-designed crowdsourcing study can detect problematic instances. Hybrid workflows to leverage the abilities of both humans and machines \cite{HybridHumanAIWorkflowsUnknown, lakkaraju2017identifying, han2021iterative, chung2018unknownexamples} have also been explored throughout several studies in proposing collaborative human-AI workflow \cite{HybridHumanAIWorkflowsUnknown} or generating text descriptions \cite{han2021iterative} about spurious patterns.

While these studies demonstrate how human intelligence plays a significant role, tool support is still lacking to guide practitioners to inspect, identify, and mitigate systematic errors. In our study, we provide a workflow and systematic support for inspecting which systematic errors are attributed to interpretable concepts.

\subsection{Visual analytics for ML diagnostics}
Visual analytics tools in recent years have evolved to offer interactive ways for inspecting the machine learning process. In general, these tools aim to better visualize the predictive results in a model-agnostic manner or present the structure of the model in a model-specific way. Model-agnostic approaches propose to better visualize machine learning results regardless of model types. Many visualizations among them are largely designed on the grounds of confusion matrix as tree or flow diagram \cite{shen2020designing, VisualizingSurrogateDecisionTrees}, comparative visual design \cite{ManifoldModelAgnosticFrameworkInterpretation, ExplainExploreVisualExplorationMachine, olson2021contrastive, kaul2021improvingcounterfactuals, krause2017workflow}, radial \cite{VisualMethodsAnalyzingProbabilistic} or multi-axes based layout \cite{SquaresSupportingInteractivePerformance}. On the other hand, model-specific inspections also gained attention to support the inspection of a deep neural network inside its layers, neurons, or activations \cite{liu2017analyzingtraining, ShapeShopUnderstandingDeepLearning, TopoActVisuallyExploringShape, DeepVIDDeepVisualInterpretation}.

Visual analytic tools can also help inspect and explain the potential cause of systematic failures such as a shifted or skewed distribution of the training examples termed as out-of-distribution \cite{OoDAnalyzerInteractiveAnalysisOutofDistribution}, covariate or concept shift \cite{DiagnosingConceptDriftVisual} or machine biases \cite{FairVisVisualAnalyticsDiscovering, FairSightVisualAnalyticsFairness, WhatIfToolInteractiveProbing}. The OoD analyzer \cite{OoDAnalyzerInteractiveAnalysisOutofDistribution} presented a grid-based layout to visualize the distributional differences in training and test sets. The problem of concept drift was tackled and presented as visualizations in a 2D heatmap visualization \cite{DiagnosingConceptDriftVisual} or distribution-based scatterplot \cite{ConceptExplorerVisualAnalysisConcept}. Other interactive tools such as Deblinder \cite{DiscoveringValidatingAIErrorsCrowdsourced}, SEAL \cite{SEALInteractiveToolSystematicError}, or Error Analysis \cite{erroranalysis} have recently been proposed to mitigate systematic errors with subclass labeling or user-generated report. Compared to previous work, our study aims to promote a human-in-the-loop workflow consisting of tasks to identify biased patterns and their association/attribution aspects with the perspective of spurious associations.

% Recent visualization studies also proposed how to better explain them with counterfactuals [BF-1], or to present them in a form of report [BF-3]. 


\subsection{Understanding model with concept interpretability}

The XAI methods to explain the behavior of black box models \cite{InterpretabilityFeatureAttributionQuantitative, AutomaticConceptbasedExplanations, BayesianCaseModelGenerative, ConceptWhiteningInterpretable2020} have been recently expanded to a concept-level sensitivity. The method called TCAV (Testing Concept Activation Vector) \cite{InterpretabilityFeatureAttributionQuantitative} provides a post-hoc method to explain the global influence of a concept in a pre-trained model. ACE (Automatic Concept Extraction) \cite{AutomaticConceptbasedExplanations} was proposed to identify and filter interpretable concepts from the meaningful clusters of segments on the basis of TCAV. In \cite{ConceptWhiteningInterpretable2020}, Concept Whitening (CW) purposefully alters batch normalization layers to a concept whitening layer to learn an interpretable latent space. Especially, the whitening step in this method points out that the concept space needs to be preprocessed to better align concept vectors.

These concept-level interpretability methods, however, require the human ability to observe and extract semantically meaningful concepts \cite{AutomaticConceptbasedExplanations}. There are various ways to identify and extract concepts in collaboration with humans and systems \cite{AutomaticConceptbasedExplanations, NeuroCartographyScalableAutomaticVisualSummarization, zhao2021humanintheloopextraction, DASHVisualAnalyticsDebiasingImage,  ConceptExplorerVisualAnalysisConcept, ProtoSteerSteeringDeepSequence, AnchorVizFacilitatingSemanticData, ConceptVectorTextVisualAnalytics, VisualConceptProgrammingVisualAnalytics}. ConceptExtract \cite{zhao2021humanintheloopextraction} aimed to support concept extraction and classification in a human-in-the-loop workflow and visual tools. In DASH \cite{kwon2022dash}, problematic biases from irrelevant concepts can be identified through observations from users, which were proposed to be mitigated through random image generation using GAN techniques. ConceptExplainer \cite{ConceptExplorerVisualAnalysisConcept} was designed to explore the concept associations focusing on validating conceptual overlapping between classes, especially serving as a concept exploration tool for non-expert users. In \cite{VisualConceptProgrammingVisualAnalytics}, a self-supervised technique was proposed to automatically extract visual vocabulary to allow experts to refine the labeled data and understand the concepts.

Unlike existing work, our study proposes an interactive workflow of exploring concepts for the purpose of inspecting systematic errors and spurious concept associations behind them. Similar to \cite{WhatDidMyAILearn}, our human-in-the-loop workflow aims to promote the sensemaking of practitioners specifically in the problem of systematic errors where they can iteratively work on subsetting, contrasting patterns in instances, and hypothesizing spurious associations.


% All these methods including.. share the idea of defining a concept vector with a group of semantically coherent segments. While we take the approach of pre-processing steps on concept space in [] and sensitivity, we expand the utility of concept exploration towards inspecting model's false behaviors. In our study, we demonstrate that using concept interpretability can help not only interpreting the concept association towards misclassificaitons, and tracing back ... then removing the biases to further improve the quality of classification.






\section{Design Guidelines \& Tasks} \label{sec:goalsandtasks}


\begin{table}[]
\resizebox{\textwidth}{!}{%
\begin{tabular}{@{}lllll@{}}
\toprule
 &
  \textbf{Expertise} &
  \textbf{Goal of analysis} &
  \textbf{Workflow} &
  \textbf{Challenges and Limitations} \\ \midrule
\textbf{Expert A} &
  \begin{tabular}[c]{@{}l@{}}Online movements \\ and campaigns\end{tabular} &
  \textit{\begin{tabular}[c]{@{}l@{}}"How do groups hold \\ different emotional and \\ moral attitudes towards \\ social issues and capture \\ a concrete evidence \\ from their utterances?"\end{tabular}} &
  \begin{tabular}[c]{@{}l@{}}Count the word occurrences in R;\\ Fit multiple models with \\ filtered words for each attribute;\end{tabular} &
  \begin{tabular}[c]{@{}l@{}}Finding anecdotal evidence \\ for complex psychological \\ dimensions is not feasible, \\ or it is shallow when available\end{tabular} \\
\textbf{Expert B} &
  \begin{tabular}[c]{@{}l@{}}Team \\ communication\end{tabular} &
  \textit{\begin{tabular}[c]{@{}l@{}}"How do demographic groups \\ form the faultline with respect \\ to team members' attributes \\ within or across the groups?"\end{tabular}} &
  \begin{tabular}[c]{@{}l@{}}Run a linear regression with STATA; \\ Do clustering analysis with Tableau;\\ Examine subgroup members \\ in spreadsheet;\end{tabular} &
  \begin{tabular}[c]{@{}l@{}}Interaction between deductive \\ and inductive analysis is \\ inconvenient; (i.e., how is a \\ subgroup characterized \\ (inductive) or attributed \\ (deductive) to each attribute?\end{tabular} \\ \bottomrule
\end{tabular}%
}
\caption{Examples of the pilot study result.}
\label{tbl:pilot_study}
\end{table}

In this section, we propose a set of design guidelines of group-level analytic tools for facilitating a conscientious practice of analyzing group characteristics. To formulate the design of a group analysis tool, we collect the feedback and thoughts from potential users in using existing tools. We assume that such tools will be used by data analysts and experts who conduct a group-level analysis in different domains. In order to understand the current practice and goal of group analysis and the challenges from it, we conducted the pilot study with domain experts and identified the concerns and limitations of the group-level analysis in their workflow. Based on the findings of the pilot study, our design guideline, which consists of three goals and six specific tasks, was formulated by compiling common aspects across the interviewees’ comments. In the following paragraphs, we describe the process of deriving design guidelines in detail.

\textbf{Pilot study.} Five interviewees were chosen from a variety of domains and specializations, ranging from data science, psychology, education, language, to anthropology. All these domain experts had prior experiences with human group research using digital trace data. Each interview lasted about an hour, in the form of a semi-structured interview. During the interview, we engaged the domain experts to consider a simple scenario of using structured and unstructured data to conduct group analyses, encouraged them to apply their current practices, and reflect on the limitations and concerns of the analytic tools they currently used. We organized the interview session with three primary questions to facilitate the interviewees’ thinking process:

\begin{itemize}
    \item \textbf{Expertise/Workflow}: “What is your expertise and in what way typically do you get insights on such analysis?”
    \item \textbf{Goal of analysis}: “What insights do you anticipate to find out?”
    \item \textbf{Challenges/Limitations}: “Despite your current practice of group analysis with available tools, how does it fail to meet your needs?”
\end{itemize}


While interviewees reported a set of goals and challenges/limitations in their own context of group analysis as shown in Table \ref{tbl:pilot_study}, we were able to find that they had experienced similar difficulties in their goals and tasks. For example, a typical analysis goal of three interviewees is not only to test their hypothesis over the group/subgroup characteristics but also to find qualitative evidence to back up their test statistics. As summarized in Table \ref{tbl:pilot_study}, many existing tools they currently used in their workflow, such as STATA, R, Tableau, or simple software for dealing with spreadsheet, did not fully support the range of group analytic function -- from overviewing group characteristics, running a regression/prediction model, to identifying subgroup patterns across attributes.


After all interviews were conducted, we compiled a list of pairs of goal and challenges for each interviewee (within-interview), and grouped common facets of pairs (e.g., (goal) \textit{capture both quantitative and qualitative evidence} - (challenge) \textit{hard to capture it with respect to each attribute}) shared by interviewees (between-interviews). Based on the result of the interview study, we identified six concerns that were commonly mentioned by interviewees (denoted as \textbf{C}):

\begin{itemize}
    \setlength{\itemsep}{0.2pt}
    \item[\textbf{C1.}] Hard to capture the overview of group differences.
    \item [\textbf{C2.}] No links or traces between deductive and inductive analysis.
    \item [\textbf{C3.}] Lack of capability in analyzing both within-variance of groups and instances.
    \item [\textbf{C4.}] Unreliability without model quality inspection or attribute importance.
    \item [\textbf{C5.}] Less informative qualitative details.
    \item [\textbf{C6.}] Lack of supports in providing rationales on individual decisions.
\end{itemize}

These six common facets of concerns can be grouped and summarized into three main implications:
{\bf (1) A lack of bridge between top-down and bottom-up group analysis:} During the interviews, we found the discrepancy in common practices of conducting group analyses among experts. Some of our interviewees prefer to use deductive approach, e.g., using regression or other predictive modeling to test hypotheses, while others mostly use inductive approach, e.g., using clustering techniques to discover patterns not previously hypothesized. Especially for some research topics (e.g., analyzing team or demographic faultlines), clustering task tends to be a prevalent way of characterizing subgroups that are distinctive in their traits. Based on the interviews, we learned that a better tool may facilitate users not only to see the big picture (\textbf{C1}), but to efficiently trace the links between the higher-level patterns (hypothesized or not) (\textbf{C2}) and the instances located in the database (\textbf{C3}), which  %in order to 
allows in-depth analysis by crossing over the current two practices of top-down and bottom-up group analysis.
%A researcher ``{would like to see face validity in the raw texts before testing my hypotheses},'' and another commented ``{I often questioned how many different snowflakes in the data may have the shapes I just found}.'' 
{\bf (2) A lack of confidence in sophisticated tools and the analysis results:} Several interviewees expressed concerns about the lack of transparency when using sophisticated machine learning tools. For example, these techniques are not helpful for explaining the group culture with hypotheses (\textbf{C4}) or offering qualitative evidence (\textbf{C5}).
%A researcher prefers to ``{\it know how to explain the group culture with my hypotheses, but found it difficult with sophisticated techniques},'' and another ``{it's too hard to grapple with anecdotal evidence with those sophisticated statistics.}."
{\bf (3) A concern about the implication for group-level analyses:} A general concern repeatedly stated by the interviewees is that the correlations learned from the big data by sophisticated tools may be translated to decisions that benefit or harm certain individuals (\textbf{C6}). In particular, several of them referred to the concern of profiling under GDPR \cite{gdpr}. 

Based on the feedback from domain experts, we determine six tasks (denoted as {\bf T}) that address the aforementioned concerns, which were grouped to four guidelines (denoted as \textbf{G}) in a bottom-up manner.

\paragraph{\bf G1. Identify the shared characteristics of a group of people and the differences between groups.} The analytic tool should allow users to see how groups of people share the common characteristics and how groups differ from each other in terms of key attributes of interest.

\begin{itemize}
    \setlength{\itemsep}{0.2pt}
\item[{\bf T1}] \textbf{Group trend}: in our design, the characteristics of groups will be visualized as ``group trend'' and the differences between groups will be contrasted through visual encoding.
\item[{\bf T2}] \textbf{Inference reliance}: the tool will support users to assess the analytic model quality against ground-truth data whenever available. 
\end{itemize}

\paragraph{\bf G2. Inspect the shared characteristics and variance of groups in both quantitative and qualitative ways for hypothesis testing and searching.} The tool should allow users to closely inspect the group differences in two ways: it should provide the quantitative evidence showing how the key characteristics or attributes differentiate the groups, and the qualitative details where specific instances from the data can be retrieved to corroborate the identified characteristics. It should also allow users to discover variance from the explicit grouping to reduce the possible overgeneralization of the group distinctions.

\begin{itemize}
    \setlength{\itemsep}{-0.5pt}
\item[{\bf T3}] \textbf{Group variance}: the tool will extract and visualize subgroup characteristics to support the examination of the within-group trends and variation.
\item[{\bf T4}] \textbf{Attribute importance}: how groups are differentiable by the key characteristics or attributes will be visualized as ``attribute importance'' -- the dependence of a given attribute when predicting a group. 
\item[{\bf T5}] \textbf{Qualitative details}: the tool will enable users to retrieve the qualitative cues from individual data instances that are representative for each quantitative measured attribute.
\end{itemize}

\paragraph{\bf G3. Provide the rationale for the prediction of an instance as a group member.} The tool should allow users to understand the rationale behind each individual's group assignment produced by any analytic/predictive models. The rationale should allow users to connect to or verify the identified group characteristics. The tool should provide a module to offer instance-level explanation, which is complementary to the approaches for describing model behavior with feature importance (in \textbf{G2}) or aggregated individual explanations \cite{krause2018user, stumpf2016explanations, poursabzi2018manipulating}.

\begin{itemize}
    \setlength{\itemsep}{0.2pt}
\item[{\bf T6}] \textbf{Grouping rationale}: in our design, the group rationale will be offered through a ``contrastive explanation'' -- to explain why a member is considered to belong to one group rather than the other.
\end{itemize}

\section{Application Scenario and Data} \label{sec:appanddata}

We take the aforementioned data journalist example as an application scenario to design a visual analytic system following the design guideline. In this scenario, a user (the data journalist) is interested in exploring Twitter users' communications related to gun and gun-control issues. The user wonders: ``{\it how do social media users with different political leanings talk differently on the gun issues?}'' ``{\it how do the differences revealed through the hypothesized attributes -- particularly certain sociolinguistic characteristics -- and manifested on users' tweets?}'' and ``{\it how can I make sense of and trust the group analysis results?''}

We make use of a publicly available Twitter dataset from a prior study \cite{yan2017quantifying}. This dataset contains more than 600k Twitter users that have been identified with liberal or conservative leaning based on their following behaviors \cite{yan2017quantifying,yan2020mimicprop}. From this data, we have identified a total of 3,100 tweets and 2,256 users that are related to ``gun rights'' or ``gun control'' discussions. In addition to standard keyword matching method, we use a PU learning algorithm \cite{fusilier2015detecting} to identify related tweets, then validate the results manually. In order to verify the machine-inferred sociolinguistic characteristics from the tweet text, each tweet was manually annotated with seven sociolinguistic attributes, including \textit{Care}, \textit{Authority}, \textit{Purity}, \textit{Fairness}, and \textit{Loyalty} as described by Graham et al. \cite{graham2009liberals} as moral values, and \textit{Valence} (the happiness expressed) and \textit{Dominance} (the degree of control exerted) \cite{warriner2013norms} as affects. We summarize the definitions of these seven attributes and their annotation process in the Appendix (Section~\ref{sec:attr-def} and \ref{sec:annotation}). This carefully curated dataset allows us to design and test our new visual analytic toolkit with the set of {\it ground-truth} group labels and attribute values in a real analysis scenario. In the following sections, we use ``\red'' and ``\blue''  as the group labels.

\begin{figure}
    \vspace{-1em}
    \includegraphics[width=\columnwidth]{figures/system-task-pipeline.pdf}
    \vspace{-1em}
    \caption{\label{fig:system-task-analytic-pipeline}
    Our system in (a) operates on the analytic pipeline in (b) consisting of: \textit{contrastive explanatory models} for predicting and discovering the group difference being rendered in \grouptrend and \evalscope, and providing rationales for explaining the prediction of any instance in \rationalescope, and \textit{multi-task prediction models} for generating language cues from the tweet dataset to support the visual inspection in \languagescope.
    }
\end{figure}

\section{\name}
\name is built following the design guideline described earlier. As shown in Fig.~ \ref{fig:system-task-analytic-pipeline}, \name consists of two main components:  (a) visualization and (b) analytic pipeline. The visualization component is comprised of a number of visual interactive tools to support the six main tasks described in Section~\ref{sec:goalsandtasks}, including \grouptrend ({\bf T1}), \evalscope ({\bf T2}), \variscope ({\bf T3}), \depscope ({\bf T4}), \languagescope ({\bf T5}), and \rationalescope ({\bf T6}) (Fig.~\ref{fig:teaser}). Together, these six ``scopes'' allow users to access to qualitative differences ({\bf G1}) and variability ({\bf G2}) of groups, qualitative details from user-generated text ({\bf G2}), and model explanation ({\bf G3}). The analytic component (Fig. \ref{fig:system-task-analytic-pipeline}) includes two major machine learning modules: (1) {\it multi-task prediction models}  generate language cues from the tweet dataset to support the visual inspection in \languagescope, and (2) {\it contrastive explanatory models} generate rationales for explaining the membership prediction. Built up as a web-based tool, our system was implemented as a full-stack application with a python based back-end framework called Django \footnote{https://www.djangoproject.com/} to process API calls and data processing, and a front-end framework called ReactJS and React hooks\footnote{https://reactjs.org/}, and Postgres database \footnote{https://www.postgresql.org/}. Any dataset with all features and metadata is required to be stored in a file and processed into the database before running the tool. Below, we summarize our implementation of the visualization and analytic components.

\subsection{Visualization}\label{sec:vis}
Fig.~\ref{fig:system-overview} captures the user interface of \name. To facilitate users to navigate the dataset and select data/attributes of interest, two interactive tools, \instanceviewer and \scopecontroller are provided as shown on the left and the top of the user interface (Fig.~ \ref{fig:system-overview}a,b).
Once users specify the attributes of interest (e.g., which features, or sociopsychological dimensions) through the \scopecontroller, the main visualization panel with various ``scopes'' will be updated accordingly. These scopes provide distinct functionality to support tasks ({\bf T1}--{\bf T6}) in the design requirements. 
The main visualization panel seamlessly integrates multiple scopes so that the users' data exploration can be loosely guided. For example, \grouptrend can guide users' gaze from left to right through its polylines, and from top to bottom through its vertical axes. This integrated visual layout is designed to make users be contentious about what the group patterns entail. When staring at certain group patterns along the horizontal direction, users can be immediately hinted by the variability and predictive confidence of the patterns via \variscope and \evalscope. Along the vertical direction, users can easily find evidence for the patterns via \languagescope or \rationalescope.
%The main visualization panel seamlessly integrates multiple scopes while users' data exploration is loosely guided by the visual flow created by the featured component, \grouptrend -- with the polylines guiding users' gaze from left to right, and the vertical axes from top to bottom. This integrated visual layout is designed to make users be contentious about what the group patterns entail. For example, when staring at certain group patterns, users should be immediately hinted by the variability and predictive confidence of the patterns (through the horizontal direction with \variscope and \evalscope) and by the readily available evidence for the patterns (through the vertical direction with \languagescope or \rationalescope below the \grouptrend). 
These views also help users to make connection between the specific observations from an individual scope view and the overall group patterns. We describe the specific functions of each scope below.
% Note that the users will be able to see both the visualization components that support showing the overall trend derived from a top-down, or quantitative approach (Fig.~\ref{fig:system-overview}c-i,ii,iii,iv) and that alternatively support showing qualitative details for further bottom-up inquiry (Fig.~\ref{fig:system-overview}d).}

%visual components that are dedicated to the overview of group trend and variance (Fig.~\ref{fig:system-overview}c-i,ii,iii,iv) and qualitative details (Fig.~\ref{fig:system-overview}d) will be updated accordingly. Note that the visual component showing the overall trend and qualitative difference are displaced adjacently to each other, so users can inspect both at the first glance, decide the components they would like to focus depending on their approaches and meanwhile inspect the patterns from both the top-down (more quantitative, big trends) and bottom-approach (more qualitative, language cues.
%\rev{After adjusting the group analysis setting, visual components dedicated to the overview of the group trend and variance (Fig.~\ref{fig:system-overview}c-i,ii,iii,iv) and qualitative details (Fig.~\ref{fig:system-overview}d) are updated accordingly. 

%Note that visual components are displayed together being adjacent to each other, in order to reflect the feedback from the pilot study (refer to \textbf{C3} in Section \ref{sec:goalsandtasks}), so that our system allows to trace the link between the top-down and bottom-approach or quantitative group difference and qualitative language cues by overviewing them at a glance. 

%The data and attributes of interest, once identified by users, are visualized in the main visualization panel through various scopes. 

Unless otherwise specified, \redclr and \blueclr colors are used to differentiate the group membership, with color saturation representing the proportion of tweets in the \red or \blue groups. 

\begin{figure*}[!ht]
    \centering
    \includegraphics[width=0.95\linewidth]{figures/system-overview.pdf}
    % \vspace{-1.8em}
    \caption{\label{fig:system-overview}
    The system overview of \name. The system integrates visualization and analytic pipeline to support the group analysis tasks. On the user interface, (a) \instanceviewer and (b) \scopecontroller support navigation, data retrieval, selection, and control. The data and attributes of interest are visualized in the main visualization panel through various ``scopes.'' For example, (c) \grouptrend visually captures the major ``trends'' of groups across attributes, and (d) \languagescope provides a visual summary of the language evidence for every sociolinguistic attribute, enabling users to further retrieve qualitative details from the tweet instances. (e) \rationalescope provides the instance-level explanation on why a tweet was classified as certain group in comparison with another tweet.
    % The overview of \name and analytic pipeline to support the group analysis tasks. (a) \instanceviewer provides the list of tweets to support the exploration of individual tweets. (b) In \scopecontroller, users can (i) overview the summary of group trend, (ii) select attribute of interest, and (iii) adjust the weights of the language-level criteria to extract the important sequences. The system represents the group difference (c) in the attribute-level features quantitatively by (i) visualizing the group difference in \grouptrend, (ii) representing the attribute importance in \depscope and (iii) inference quality in \evalscope from (f) the contrastive explanatory model, and (iv) clusters identified from Agglomerative Hierarchical clustering with the selected attributes in \clusterview. (d) \languagescope visualizes the most important qualitative details in the language-level extracted from (g) the multi-task model described in Fig. \ref{fig:multi-model}.  (e) \rationalescope provides the instance-level explanation on why a tweet was classified as certain group in comparison with another tweet. 
    }
    \label{fig:system-overview}
    \vspace{-0.8em}
\end{figure*}

\subsubsection{\bf Retrieve, Select \& Control}
Once the system is initialized, it allows users to retrieve tweets and select a particular tweet to take a closer look at it (in \instanceviewer), and control which tweets and attributes to include in the analysis (in \scopecontroller). First, \instanceviewer (Fig.~\ref{fig:system-overview}a) allows users to search and retrieve tweets by search terms. The retrieved tweets are displayed as a list of boxes. Each box contains a tweet with its group information on the top-left corner (a square glyph with its annotated group label colored as \redclr or \blueclr), the ``psycholinguistic score chart'' on the top-right corner (where each bar height represents annotated attribute value), and the tweet text. This \instanceviewer also serves as a selection tool allowing users to look into a particular tweet. For example, users can click the ``tweet handle'' located at the bottom-right corner to highlight the tweet in \grouptrend, or click an attribute on the psycholinguistic score chart to highlight the language sequence corresponding to the selected attribute in the tweet text. \scopecontroller (Fig.~\ref{fig:system-overview}b) provides an overview of all the available group attributes with statistical significance information in a ``Psycholinguistic Summary Bar Chart'' (Fig. \ref{fig:system-overview}b-i). Such overview helps users to determine which attributes to be included in (or excluded from) further group analysis. The selection of attributes can be done using the ``Features'' menu (Fig.~\ref{fig:system-overview}b-ii). The ``Sequence'' menu includes four sliders allowing users to determine the important aspects of retrieved language cues, which will be described later in the analytic modules. 

\begin{figure}
    \vspace{-1em}
    \includegraphics[width=0.95\columnwidth]{figures/system-overview-depscope-variscope.pdf}
    \vspace{-1em}
    \caption{\label{fig:system-overview-depscope-variscope}
    Two modes of Axes in \grouptrend: \depscope and \variscope.
    }
    \vspace{-1.5em}
\end{figure}

\subsubsection{\bf Overview and Inspect group difference and variance.} After setting up the configuration of analysis in \scopecontroller, users can overview the group difference in \grouptrend (Fig. \ref{fig:system-overview}c-i) with \depscope (Fig. \ref{fig:system-overview-depscope-variscope}a) or \variscope (Fig. \ref{fig:system-overview-depscope-variscope}b) as its axes in rectangular boxes allowing the inspection of feature importance and subgroups. The system also supports the inspection of the model inference results in \evalscope (Fig. \ref{fig:system-overview}c-iii).
\paragraph{\bf \grouptrend}
\grouptrend (Fig. \ref{fig:system-overview}c-i) allows users to visually capture the ``trends'' of groups and contrast the differences between them through a parallel set ({\bf T1}), where the trends are captured when polylines from a group, representing the multidimensional attribute values of data instances, agglomerate due to close attribute values. In \grouptrend, each psycholinguistic attribute is represented as a vertical axis; tweets are represented as polylines, colored by their annotated group membership, and bundled whenever appropriated to reduce the visual clutter and to enhance the rendering performance.

\paragraph{\bf \depscope \& \variscope}
The parallel axes in the \grouptrend are further augmented with \depscope and \variscope, to provide an in-context inspection of group variability on top of the group trends. Users can switch between the two modes as shown in Fig.~\ref{fig:system-overview-depscope-variscope}. The \depscope (Fig.~\ref{fig:system-overview-depscope-variscope}a) allows users to inspect the attribute importance through a ``conditional partial dependence plot'' (PDP) where the marginal probability density of each group is shown on the $x$-axes against the attribute values on the $y$-axes. This enables users to visually assess the extent to which the groups are differentiable by a given single attribute ({\bf T4}), which supports the evaluation of a hypothesis relevant to this attribute. \variscope (Fig.~\ref{fig:system-overview-depscope-variscope}b), on the other hand, allows users to inspect the group variance through a set of ``subgroup attribute glyphs'' ({\bf T3}), where each rectangle glyph represents the attribute summary of a subgroup with vertical position indicating the central tendency, height indicating the variance, width indicating the size of the subgroup, and color reflecting the probability of group membership. The subgroups were automatically detected based on the attribute values of tweets using Agglomerative Hierarchical Clustering method with the number of subgroups determined by the elbow method evaluated with the total intra-cluster variation. This enables users to visually capture the coherence or variability within a group and across attributes -- e.g., a less coherent group will have several subgroups spreading vertically along one or more attribute axes. Moreover, it serves as a hypothesis evaluation and seeking tool as the subgroup patterns may support/disconfirm an existing hypothesis, and any emerging, cross-attribute subgroup tendency may inform a new hypothesis. 

\paragraph{\bf \evalscope}
\evalscope (Fig. \ref{fig:system-overview}c-iii) allows users to closely examine the model inference results of tweet group membership by comparing the predictive membership against the ground-truth (human annotated) labels (\textbf{T2}). The predictive membership is generated from a decision tree model, which is the same as the contrastive explanatory model described later in Section~\ref{sec:contrastive-exp}. The comparison is achieved through a ``dual-sided histograms'' of classification probability (from top to bottom: from the most likely \blue to the most likely \red), where correct and wrong classifications are separately shown on the left and right side of the histogram plot. To facilitate the inspection of particular prediction cases, users can click any location of the histogram bars to highlight a particular set of instances within the corresponding range of classification probabilities.  

\subsubsection{\bf Back up with qualitative details.}
\languagescope (Fig. \ref{fig:system-overview}d) provides a visual summary of the language evidence for every sociolinguistic attribute, which enables users to capture and further retrieve qualitative details from the tweet instances (\textbf{T5}). \languagescope is comprised of multiple parallel axes arranged in the same way as the \grouptrend. The language cues are text snippets extracted from the tweet text to represent a particular attribute having a particular value (or value range) -- e.g., the text snippet \tweet{out until the house votes to address gun violence} are extracted to represent an expression for attribute \domi around the value 0.85 (Fig. \ref{fig:teaser}d). The language cues are represented as squared glyphs along each attribute axis. Each glyph is represented with its size indicating the sequence importance (described in Section~\ref{sec:multi}), with color indicating the group membership probability, and vertical position indicating the mean attribute value. These language cues are learned automatically from the multi-task prediction models, which will be described in a later section. The extracted cues are associated with an importance score that reflect both the model prediction and users' preference (as described in the \scopecontroller). By default, the system displays ten most important language cues for each attribute and the cues with an importance score greater than a threshold will be shown with the corresponding text snippets. 

\subsubsection{\bf Offer the rationale behind group profiling.}
\rationalescope (Fig. \ref{fig:system-overview}d) allows users to closely examine how individual tweets may be predicted to be in a certain group (or not) according to its attribute values, through a ``contrastive explanation dialog box'' ({\bf T6}). The dialog box presents a ``rationale'' as an answer to a question about the prediction. Users can select tweets of interest (from browsing the \instanceviewer or other scopes) and ask two types of reasoning questions \cite{van2002remote}: (1) {\it p-mode}: focuses on the ``properties'' of an object or instance (e.g., ``{\it Why is tweet X classified as \red rather than \blue}?''), and (2) {\it o-mode}: focuses on the contrast of two objects (e.g, ``{\it Why are tweet X classified as \red whereas another tweet Y classified as \blue }?''). Such rationales are automatically generated from the contrastive explanatory models, which will be described in a later section. The extracted rationales are shown using natural language with counterfactual examples in the dialog box. \\

\begin{table}[]
\resizebox{\textwidth}{!}{%
\begin{tabular}{llll}
\hline \\[-10pt]
\textbf{Layout} &
  \textbf{Design scheme} &
  \textbf{Advantage} &
  \textbf{Disadvantage} \\[2pt] \hline \\[-7pt]
\textbf{\begin{tabular}[c]{@{}l@{}}Scatterplot\\ matrix\end{tabular}} &
  \begin{tabular}[c]{@{}l@{}}A layout with a set of scatter plots \\ representing bivariate relationships \\ in a two-dimensional matrix\end{tabular} &
  \begin{tabular}[c]{@{}l@{}}Well-represents the pairwise \\ relationship between attributes\end{tabular} &
  \begin{tabular}[c]{@{}l@{}}Does not show the trends \\ over multiple attributes\end{tabular} \\[15pt]
\textbf{\begin{tabular}[c]{@{}l@{}}Radial\\ layout\end{tabular}} &
  \begin{tabular}[c]{@{}l@{}}A two-dimensional plot encoding \\ multivariate attributes of instances \\ along with radial axes\end{tabular} &
  \begin{tabular}[c]{@{}l@{}}Reveals the instance-level \\ differences by the overall \\ association with attributes\end{tabular} &
  \begin{tabular}[c]{@{}l@{}}Does not reveal \\ the attribute-wise trend\end{tabular} \\[18pt]
\textbf{\begin{tabular}[c]{@{}l@{}}Glyph-based\\ layout\end{tabular}} &
  \begin{tabular}[c]{@{}l@{}}Instances are represented as glyphs \\ encoding attribute values\\ (e.g., in case of a glyph design \\ with small radial bars)\end{tabular} &
  \begin{tabular}[c]{@{}l@{}}Visualize both instance-wise \\ characteristics and the differences \\ over multivariate attributes \\ between instances \\ in two-dimensional plot\end{tabular} &
  \begin{tabular}[c]{@{}l@{}}Hard to show attribute-wise \\ trend and group differences\end{tabular} \\[30pt]
\textbf{\begin{tabular}[c]{@{}l@{}}Parallel\\ coordinates/\\ sets\end{tabular}} &
  \begin{tabular}[c]{@{}l@{}}A axis-based layout with its polylines\\ as instances passing through multiple \\ axes for multivariate characteristics\end{tabular} &
  \begin{tabular}[c]{@{}l@{}}Provides the overall trends \\ of instances' attributes\end{tabular} &
  \begin{tabular}[c]{@{}l@{}}Less space-efficient compared \\ to other layouts\end{tabular} \\[15pt] \hline
\end{tabular}%
}
\label{table:design_choice}
\caption{\label{table:design_choice} The summary of visual layouts in the design process of \grouptrend.}
\end{table}

\subsubsection{\bf Design choice and consideration.} To decide the proper design of visual components and integrated layout methods, we went through multiple phases of the design process defining the underlying visualization tasks/problems and examining prior research especially in multi-dimensional visualization. Following the Munzner’s Nested Model \cite{munzner2009nested}, we started with defining the core domain problem, “visualizing group difference”, and the specific design requirements listed in Section \ref{sec:goalsandtasks} above. We then identified the possible data types (i.e., continuous, ordinal, or categorical attributes of group characteristics, and text data) and the operations (i.e., predictive analysis and its attribute-wise interpretation) to be considered in our system. 


To support the visualization of domain problems, tasks, data types, and operations all together, we break down the whole design process into two phases: First, we define a visual layout for it to serve the core domain problem, ``visualizing the group differences''. In the system, we let \grouptrend play an central role, and investigated multi-dimensional visualization to determine the design of 
\grouptrend. Second, we define other visual components in accordance with \grouptrend to serve the aforementioned tasks and operations. In summary, in these steps we compared alternative choices of organizing visual space and layouts for multidimensional dataset and their potential to be extended to support machine learning and interpretability functionality. In the final layout, other layout components such as \depscope, \variscope, \evalscope are tightly coupled and integrated with the design of \grouptrend so that the visual layout as a whole not only serves the requirements of our system but visually associates with each other. We illustrate the two considerations of the design process in detail below.

First, to find out the design of \grouptrend for visualizing the group differences, we investigated 10 designs in the multi-dimensional visualization studies. In the literature review, we first categorized visual layouts in four types of multi-dimensional visualization based on the classifications in two surveys \cite{liuVisualizingHighDimensionalData2017a, hoffman2002survey}, then searched for literature by the name of layout types and selected 10 visual layouts. The design alternatives as a result of this process can be categorized into four typical multidimensional visual layouts: parallel coordinates and sets \cite{kosara2006parallel, richerEnablingHierarchicalExploration, vosoughParallelHierarchiesVisualization2018, weideleAutoAIVizOpeningBlackbox2020, novotnyOutlierPreservingFocusContext2006} and radial layout \cite{albuquerque2010improving, wang2019polarviz}, scatterplot matrix \cite{wilkinson2006high}, and glyph-based layout \cite{zhaoSkyLensVisualAnalysis2018, cao2018z}. 

Table \ref{table:design_choice} provides a summary of advantages and disadvantages of visual layouts examined in the design decision. Among them, scatterplot matrix is a well-known visual layout with a set of scatter plots representing bivariate relationships in a two-dimensional matrix. While it effectively reveals how data is correlated with respect to any combinations of two variables, we find that it has a limitation of showing trends throughout multiple variables in our application. Radial and glyph-based layout are other types of layout which encode multivariate attributes of an instance as a vector with a point or glyph being projected and coordinated in two dimensional plot. It is advantageous by its two-sided strategy, encoding the overall similarity between instances by their coordinates and attribute-wise properties of instances by radial axes or glyphs with small radial bars, however, it does not facilitate the overview of agglomerative group-wise trends.

We compare the pros and cons of all possible visual layout candidates, as summarized in Table \ref{table:design_choice}, and decided that the parallel sets/coordinates are most suitable to meet the required data types and operations.
It provides the overview of multivariate group trends especially with polylines colored by group memberships along with multiple axes. To support heterogeneous data types, we combine the layout of parallel sets and coordinates. For example, in our dataset, the {\it affect} attributes (e.g., \vale) are continuous variables and the {\it moral} attributes (e.g., \care, \fair) are categorical/ordinal variables. When different types of variables are involved, a polyline indicates either an instance (for continuous variable) or a group of instances belonging to a category (for categorical/ordinal variable). These polylines pass through the vertical parallel axes that are arranged in a way that the highest to lowest possible values of continuous variables are shown from the top to the bottom, while the values in categorical variables are shown in the same or similar orders (e.g., ``virtue,'' ``both,'' ``none,'' and ``vice''). 

Second, considering the visualization of classification results and attribute-wise interpretation, we find that the choice of the parallel sets as a visual layout benefits from its extendability, with which we turned it into “parallel sets for classification” with several visual components being integrated and connected to \grouptrend as an extension of traditional parallel sets. Specifically, we extend the visual space of parallel sets to incorporate the classification results and interpretation in two ways: 1) Utilizing unused visual space - by utilizing the axes of parallel sets for attribute-wise interpretation (presented as \depscope or \variscope). The vertical parallel axes in the traditional layout are typically of no use without any functionality. We expand it to a vertical space to encode the attribute-wise interpretation (details in the \depscope or \variscope section), 2) Connecting to other layouts - with the histogram plot of prediction results (\evalscope) being aligned and connected on the right side of \grouptrend (as shown in Fig. \ref{fig:system-overview}c-i and Fig. \ref{fig:system-overview}c-iii). The polylines in \grouptrend flow through the vertical axes and lead to \evalscope, which can show how group trends are associated with the inference results (details in the \evalscope section).


\subsection{Analytic pipeline}\label{sec:analytic}

The analytic pipeline, as shown in Fig.~ \ref{fig:system-task-analytic-pipeline}b, includes data processing and machine learning modules to extract information to be shown on the visualization interface. As complete details of the implementation are beyond the scope of this paper, here we provide our methodology for implementing the two major machine learning modules. We propose two machine learning algorithms to enhance the interpretability and explainability of group analysis: (1) Multi-task predictive model: We introduce a multi-task prediction neural architecture predicting jointly both group membership and attribute values from language sequences enables to extract attribute-wise linguistic cues as qualitative evidence from the attention mechanism with better predictive performance. (2) Contrastive explanatory model: We provide the module of generating contrastive explanations to present the minimal and sufficient information of group classification results. By leveraging a contrastive explanation approach \cite{van2018contrastive}, our pipeline introduces our criteria and methods to retrieve counterfactual examples in fact-foil tree in addition to explanation itself for better explainability.

\subsubsection{Generating Language Cues via Multi-task Prediction}\label{sec:multi}
\begin{figure}[!ht]
    \centering
    \includegraphics[width=0.95\columnwidth]{figures/multi-model.pdf}
    \caption{
    The neural network architecture for generating language cues. An input of word sequence in a tweet is transformed to word embedding and encoded by a Bi-LSTM layer, with predictive weights learned through an attention mechanism. Weighted latent vectors are taken into the dense layers to jointly predict the group and attribute value. Informative language cues are generated from a function of the learned attention weights. 
    }
    \label{fig:multi-model}
\end{figure}
The {\it multi-task prediction models\footnote{The code is available at: \url{https://github.com/picsolab/TRIBAL-multi-task-prediction}}} are developed to extract text snippets as language cues from the tweet dataset. Compared to language models with a bottom-up approach for discovering latent dimensions of semantics such as topic modeling, our model was designed to enable the theory-driven analysis where attributes to be included in the analysis are given by users in a top-down manner, and allow them to find linguistic cues with respect to each attribute. Our model is thus advantageous in interpreting the group difference based on (a) attributes that are  theoretically meaningful (e.g., emotional/moral attributes) or (b) data-driven features indicative of behavioral patterns (e.g., the number of retweets). Language cues identified from the model allows users to make sense of an attribute (e.g., ``{\it What dose high \vale mean in this context}?'') through retrieving qualitative details from the tweet instances (e.g., ``{\it What would the language expressing high \vale look like from the \red or \blue groups}?'')  ({\bf T5}), which are shown in the \languagescope. To automatically learn the language cues that are {\it representative for both groups and attribute values}, we introduce a new multi-task prediction neural network architecture, where the objective is to jointly predict (a) attribute values and (b) group labels (\red or \blue) from the language sequence of a given tweet.  As a result, the neural network architecture can learn to identify what language sequences are more predictive to a particular group and attribute information. Because the representative language cues for different attributes will be different, we train a set of models with similar architectures but different objective functions (one model for each attribute).

Take the \vale attribute as an example. Fig~\ref{fig:multi-model} illustrates the neural network architecture that jointly predicts \vale value and group label. The input tweet text is represented by a pre-trained embedding, where each tweet word is represented as a high-dimensional vector. Due to the specific emphasis on sociolinguistic values in this context, we leverage two pre-trained embedding methods: (1) a word2vec \cite{mikolov2013efficient} embedding trained on a standard Twitter corpus \cite{baziotis2018ntua}, and (2) the attribute-aligned embedding trained with MimicProp algorithm \cite{yan2020mimicprop} that is optimized for sociolinguistic lexicons. We concatenate the two equal-sized embeddings to generate a 600-dimensional vector for each word.

Let the $\mat{X}= \{\vec{x}_{1}, \vec{x}_2, …, \vec{x}_t, ... \vec{x}_n\}$ represent the embedding language sequence for a tweet with $n$ words,  where $\vec{x}_t$ represents the embedding vector for the $t^{th}$ word in the tweet. Let $l$ and $s$ denote the ground-truth label and attribute value of a tweet. The objective is to minimize the total loss per tweet:
\begin{equation}
    loss = \lambda\cdot loss_{group}  + (1 - \lambda)\cdot loss_{attr},
\end{equation}

where the $loss_{group} = CrossEntropy(pr_{group}, l)$ is the cross-entropy loss for predicting group label by comparing the posterior label probability $pr_{group}$ with the true label $l$, and $loss_{attr} = MSE(logit_{attr}, s)$ is the mean squared error for predicting attribute value by comparing the logit $logit_{attr}$ with the true value $s$, and $\lambda$ is a hyper-parameter determining the trade-off between two types of loss. 

For continuous attributes (e.g., \vale in this case), we use $MSE(\cdot)$ to model the loss, whereas for categorical attributes, the loss is computed using $CrossEntropy(\cdot)$ similar to the loss for group prediction. We leverage a bi-LSTM encoder \cite{huang2015bidirectional} with an attention mechanism \cite{cho2014learning} in order to learn {\it which specific portion} of the language sequence serves predictive features to the prediction. The bi-LSTM encoder learns a latent representation recursively for an input word at location $t$, as $\vec{h}_t = f[(\vec{x}_t) , \vec{h}_{t-1}]$, with $\vec{h}_{0}$ a trainable bias parameter. The latent representation is then taken as the input for the attention layer to learn the attention weight $a_{t}$ by: $a_{t} = \frac{exp(e_{t})}{\sum_{k = 1}^{n}exp(e_{k})}, e_{t} = attn(\vec{h}_{t})$, where $attn(\cdot)$ is a dense layer with trainable weights $\vec{w}$ and bias parameter $b$ which transforms $\vec{h}_{t}$ as $e_{t} = \vec{w} \cdot \vec{h}_{t} + b$.
The latent vector $\vec{h}_{t}$ is then weighted by the learned attention weight $a_t$ as an input for the dense layers for computing the loss. 

The attention weights can be seen as the importance of each word in the prediction. As our goal is to retrieve a sequence (of consecutive words) rather than individual words, we compute the cumulative attention for a length-$k$ sequence $\vec{w} = \{w_i … w_{i+k-1}\}$ with attention weights $\vec{a} = \{a_i ... a_{i+k-1}\}$ where the length is determined when adding the next word to the current sequence decreases the mean attention of the sequence to a great extent (specifically, $k$ is automatically detected when increase of $k$ leads to the drop of mean attention exceeding a threshold $\theta = a_t - \frac{1}{2} \sigma^2(\vec{a})$). 

We evaluate our multi-task prediction models using a hold-out experiment. The experiment results suggest that our multi-task models can significantly improve group prediction without sacrificing the performance for single attribute prediction. 
In the experiments, we determine the hyperparameter $\lambda$ by searching on the space from 0 to 1 with 0.1 interval.
Our empirical results suggest 0.5 to be the optimal $\lambda$.
Experiment results reported in Section~\ref{sec:holdout} are all from models with $\lambda=0.5$.

Lastly, to compare the ``representativeness'' among all language sequences extracted from the data through this process, we define an importance score for each sequence based on four information criteria: (1) {\it Predictive Impact} reflects the predictive contribution of the sequence across all training tweet samples, which is computed as the percentile rank of the aforementioned cumulative attention for an extracted sequence. (2) {\it Concept Representativeness} captures how closely the tweet containing the sequence is aligned with the joint objective, which is measured using the normalized posterior probability of the tweet. (3) {\it Length} is the desired length of the extracted sequences, where longer sequences are usually preferred. (4) {\it Prevalence} reflect the frequency of an extracted sequence. While similar or identical sequences in the training data may aggregately achieve higher predictive power, retrieving similar sequences adds little to human interpretation. We thus consider the relative occurrence of a sequence in the data as an indicator for its (lack of) uniqueness. Together the four criteria are used to compute the sequence importance score as:
$\vec{w}_{seq} = \frac{p + log(l + \epsilon)}{r + log(f + \epsilon)}$,
where $\epsilon$ is a smoothing term, $p, l, r$ and $f$ denote the \textit{\textbf{p}osterior score}, \textit{sequence \textbf{l}ength}, \textit{attention weight \textbf{r}anking} and \textit{sequence \textbf{f}requency}, respectively. To enable users to retrieve language cues with different characteristics, the weights of the four criteria can be adjusted in the \scopecontroller as described in Section~\ref{sec:vis} (Fig.~\ref{fig:system-overview}b-iii).

\subsubsection{Generating Rationales via Contrastive Explanatory Models}\label{sec:contrastive-exp}

The {\it contrastive explanatory models} was developed to generate a rationale behind the membership classification of any given tweet ({\bf T6}). Instead of offering a more complete and well-rounded explanation, in this work, we leverage the contrastive explanation approach proposed by van der Waa et al. \cite{van2018contrastive} to generate a simple interpretation in a more user-friendly manner -- to present the minimal and sufficient information required to understand the current output by contrasting with another one that is absent. Specifically, we use fact-foil trees (locally trained one-versus-all decision trees) to identify the disjoint set of rules on important features to answer a question like ``{\it why this output (the fact) instead of that output (the foil)}?'' We further contextualize such question in our application scenario to produce two types of questions -- the {\it p-mode} (``{\it group \red instead of group \blue}?'') and {\it o-mode} (``{\it tweet X instead of tweet Y}?'') questions as described in \rationalescope (Section~\ref{sec:vis}). Our \textit{contrastive explanatory model} consists of three steps. The first and second steps were to generate a set of contrastive explanations from the decision rules based on the fact-foil tree model \cite{van2018contrastive}. However, there could be numerous instances falling into this explanation set, and showing all the instances will be overwhelming. Therefore, we introduce a third step to select contrastive examples that best represent a contrastive explanation.
\begin{enumerate}
\vspace{-0.2em}
\item {\it Identifying a foil leaf}: In a fact-foil decision tree, the leaves are considered as a contrasting unit of classification. 
We first identify a leaf that includes the selected instance of our interest (e.g., tweet $X$ classified as \red), and then find the counterfactual leaf to contrast the selected leaf (e.g., tweet $Y$ classified as \blue), with their differences extracted as an explanation. If a tweet $X$ is placed in leaf $l$, the contrastive leaf is the closest one but classified as \blue. 
\item {\it Generate contrastive explanation with decision rules}: In the decision tree, the decision rules along the path from the root to a selected leaf can be considered as a full explanation on why a tweet was classified as a certain class. To generate a minimal and sufficient explanation, we extract only the difference between the two paths to the fact and foil leaves. 
\item {\it Select a contrastive example}: To generate a contrastive explanation to show in the \rationalescope, we need not only the minimal information from the decision rules (about what attribute or attributes are most important) but also a tweet example to illustrate such decision rules. We identify the most appropriate tweet example within the set of instances that belong to the foil leaves based on two criteria -- the closest (base on Gower distance between the tweets' attribute values \cite{gower1971general}) and reliable (correctly classified as in another class/group) tweet. 
\end{enumerate}

\section{Use Case Scenario}\label{sec:case}
\begin{figure}[!t]
    \centering
    \vspace{-0.9em}
    \includegraphics[width=0.95\columnwidth]{figures/case-study-individual.pdf}
    \vspace{-0.9em}
    \caption{\label{fig:case-study-individual}
    \textbf{Use case scenario (T5)}: While browsing tweets related to the ``Orlando shooting'' event, (a)-(b) the user explores two tweets from each group with different \domi values, and identified the expression of \domi from the psycholinguistic score chart. The user (c) explores the expression of \auth, and (d) can further compare the tweet against others within the subgroup 3 with its polyline highlighted as green in \grouptrend.}
    \vspace{-1.3em}
\end{figure}

We now present a use case scenario for how \name facilitates accountable group-level analyses. Consider the earlier data journalist example -- Erin is trying to examine the public's sentiments on social media after a mass shooting event. In particular, she would like to determine whether online users with liberal or conservative-leaning might talk differently over the topics of gun violence, gun policies, and related issues. She hopes a deep dive into online users' conversation would give her more insights in addition to the polls that have been reported elsewhere. However, merely searching tweets through a search engine does little for her goals. In this scenario, we use aforementioned Twitter data (see Section \ref{sec:appanddata} for the details of the dataset).

\paragraph{\bf Retrieve relevant tweets and qualitative cues (T5)}
Fig. \ref{fig:case-study-individual}a-b shows how she can use \name to quickly identify relevant tweets with diverse expressions. She entered ``orlando'' as a search term, expecting to find tweets about the Orlando shooting incidents happening in 2016. This event has provoked intense social media reactions, nationwide debates, and subsequent legislative actions. As a result of search, the keyword query returned 166 tweets shown on the \instanceviewer, from both the \blue and \red camps (e.g., Fig. \ref{fig:case-study-individual}a and b, respectively). To look at how these users talked about the event differently, she explored the {\it psycholinguistic summary} bar chart as shown at the top-right corner of each tweet. She found the first two tweets seemed to be quite different in terms of expressing \domi (a sense of feeling in control or losing control of a certain state). This prompted her to look for more tweets that express \domi: ``Can I find such expressions from both groups to compare?'' She clicked the bar `D' (abbr. for \domi) from those tweets (Fig. \ref{fig:case-study-individual}a), the first tweet (from User2560 in the \blue camp) had a highlighted language \tweet{must be love and stronger gun controls}, and the second (from User2185 in the \red camp) had \tweet{even mention that the Orlando shooting had islamic ties} (Fig. \ref{fig:case-study-individual}b). These illustrate how the two users had expressed the sense of in control or losing control differently when commenting on the Orlando shootings. Out of curiosity, she clicked `A' (abbr. for \auth) to see what an expression of \auth may look like (Fig. \ref{fig:case-study-individual}c) -- e.g., a retrieved tweet with highlighted text  \tweet{letting my senator know that I support gun control} suggests the user called his/her senator (authority) to take the leadership. 

% Erin noticed that, on \grouptrend, the two dimensions, \vale and \domi, tend to have blues lines on the top (Fig. \ref{fig:case-study-group}b), suggesting she may find more expressions of \vale and \domi from the \blue camp.

\begin{figure}
  \centering
  \vspace{-0.8em}
  \includegraphics[width=0.95\columnwidth]{figures/case-study-group.pdf}
  \vspace{-1em}
    \caption{\label{fig:case-study-group}
    \textbf{Use case scenario (T1, T4, T5)}: The user (a)-(d) explores the major trends in \grouptrend and the group-wide attribute importance in \depscope, and (e) identifies the language cues in \languagescope.}
    \label{fig:case-study-group}
\end{figure}

\paragraph{\bf Overview group trend (T1), check attribute importance (T4), and retrieve qualitative evidence (T5)}
Erin used \grouptrend to get an overview of the major differences between groups. 
In addition to the \blue camp's general associations with a higher value of \vale and \domi (Fig. \ref{fig:case-study-group}c), she found that the two camps seemed to have mixed scales in \auth and \loya, as shown by crossed edges (Fig. \ref{fig:case-study-group}d). With such observations, she now wondered whether she should focus on the two more distinguishing dimensions to determine if the two political camps had talked about the event with distinct \vale and \domi tones. To find it out, she used the \depscope to check the two dimensions separately. The partial dependence plot (PDP) in the \depscope indicated that, for \vale, a lower value tends to be associated with \red lines (Fig. \ref{fig:case-study-group}a), whereas a higher value may be mixed. Counter to her expectation, the \depscope suggested that this single dimension would not be sufficient to distinguish the two camps, as the \red appeared to have diverse values and even extreme values on both positive and negative side of \vale. On the other hand, the \domi dimension shown in \depscope was more consistent, where the \blue appeared to be associated with a higher value of \domi (Fig. \ref{fig:case-study-group}b). She became interested in telling a story about this collective tendency she observed from the \blue camp. ``{\it How can I tell the story?}'' The \languagescope allows her to track the texts in tweets with specific sociolinguistic tones. Using \languagescope, she found that the word ``vote'' was recurrently shown in the tweets with a higher \domi from the \blue camp (Fig.~\ref{fig:case-study-group}e) -- e.g., \tweet{stay strong we must have a vote against gun}. Such evidence allows her to come out with a story about how liberal-leaning users made a call to action in response to this mass shooting incident.
 
\paragraph{\bf Inspect group variation (T3)}
She now wondered if her story applied to all users from the \blue camp. The \variscope mode in the \grouptrend allows her to see the trends of subgroups across different sociolinguistic dimensions (Fig. \ref{fig:case-study-subgroup}). Using \variscope, she can see how subgroups (each as a rectangle bar) within the two camps may possess a higher or lower value in a particular dimension (indicated by the vertical position of the bar), and how each subgroup may have a more or less diverse pattern (indicated by the bar height) and a varying group size (indicated by the bar width). For example, she found \blue subgroups (1, 2) to have shorter bars in the \care and \fair dimensions, while the \blue subgroup 3 had a taller bar, indicating the latter subgroup had expressed different and varying tones from the rest of the \blue camp in terms of the two dimensions. Such subgroup differences prompted her to look for the subset of users who possess very similar characteristics. For the \blue camp, she found subgroup 1 seemed to be very consistent with bar positions far from other \red subgroups. For the \red camp, she found the largest \red subgroup 10 appeared to be a coherent set, located at the lower end of many features from all other \blue subgroups. 

This scenario demonstrates the major features of \name. More features will be covered in the later sections.

\begin{figure}[!t]
    \vspace{-1.5em}
    \setlength\tabcolsep{2pt} % default value: 6pt
    \begin{tabular}{c}\\
    \includegraphics[width=0.95\linewidth]{figures/case-study-subgroup.pdf}
    \end{tabular}  
    \vspace{-1em}
    \caption{\label{fig:case-study-subgroup}
    \textbf{Use case scenario (T3)}: Exploring the variance of subgroup differences using \variscope in \grouptrend.}
    \vspace{-1.5em}
\end{figure}
\section{Expert Interview}\label{sec:expert}

We conducted expert interviews to better understand whether the proposed system achieves its design goals, as well as its strengths and limitations. 
Based on the feedback from the pilot study where domain experts expressed their concerns in using existing tools that are limited by basic or surface-level group analyses, we see the evaluation process needs to be formulated in a way that demonstrates how users can gain the insights that are more complex (i.e., involving several pieces of data as evidence \textit{"in a synergistic way"} rather than simple individual data), relevant (i.e., \textit{"deeply embedded"} in the relevant domain), and deep (i.e., \textit{“accumulating and building on itself”}) while using our system. It is referred as the insight-based evaluation as termed from prior research in evaluating visualization \cite{northMeasuringVisualizationInsight2006, plaisantPromotingInsightBasedEvaluation2008}. For this purpose, instead of a quantitative evaluation we chose to conduct a more elaborate semi-structured interview to let the interviewees facilitate their thinking process enough to derive the context-specific and insightful findings simulating their workflows where they can test their own the hypothesis and find out quantitative evidence.

We invited three domain experts -- a political scientist, a social psychologist, and a machine learning expert specialized in natural language processing. All three experts had experience in working with social media data. Two of them had participated in our pilot interviews and their concerns and desired analytic support have been incorporated into our design guideline. In these open-ended interviews, we aimed to evaluate \name in a realistic group analysis workflow.

Each interview lasted about 90 minutes. The first was conducted in person, while the other two were via video conferencing. The system was running on a Chrome browser from both computers of the interviewer and interviewee. For each interview, we first provided a guided tutorial of the system and dataset, followed by a walkthrough of the system and a semi-structured interview. To emulate a realistic workflow, we asked the participants to think aloud. They were asked to consider: (1) a research question they would like to explore, or any hypothesis they may want to test or generate with the system, (2) how the system may facilitate the exploration of their question, and (3) the limitation or desired activities of current system. This section summarizes our findings from the three interviews.

\begin{figure}[!t]
    \vspace{-1.5em}
    \setlength\tabcolsep{2pt} % default value: 6pt
    \begin{tabular}{c}\\
    \includegraphics[width=0.95\linewidth]{figures/expert-interview-1.pdf}
    \end{tabular}  
    \vspace{-1em}
    \caption{\label{fig:expert-interview-1}
    \textbf{Expert interview 1}: Analyzing the within-group language variability of \care. (a) After observing the variance within three subgroups in \variscope, the expert retrieved the relevant language-level evidence to explore the aspect of issue polarization in \languagescope.}
    \vspace{-1em}
\end{figure}

\subsection{Interview 1: Exploring ways of political polarization}
Expert 1 is a political scientist interested in studying {\it the varying aspects of political polarization}. He would like to use \name to capture how social media users are polarized on gun-related issues. In particular, he wanted to explore whether the increasingly polarized online space is more of a reflection of {\it issue polarization} or {\it non-issue polarization}. He explained that in non-issue polarization, such as {\it affective} or {\it identity} polarization, the divide is driven by ideology, partisanship, or group identity, whereas in issue polarization, the group difference reflects different issue positions or policy attitudes. He hypothesized that in the case of non-issue polarization, {\it the language patterns will be more similar in one camp but mutually disjoint between camps; in contrast, users' languages will be more diverse in general if the concern is issue based}.

\paragraph{\bf Identify typical group behaviors (T1) and representative subgroups (T3)}
He started with the \scopecontroller and found the two camps were largely dissimilar in most of the sociolinguistic attributes. While confirming that the attributes of \blue were statistically significant from the \red in most of the dimensions (except for \puri) (Fig. \ref{fig:system-overview}b-i), he commented that the clear overall differences could be a sign for non-issue polarization. Next, the \variscope on \grouptrend caught his attention, ``{\it [it allows me to] take a closer look at each attribute and observe the subtle differences in each group.}'' He observed that the overall subgroup trends showed how the two groups were separated, and that the subgroups, 1 and 10, were quite ``representative'' of each camp, which represented how the attribute values of one camp were far from the other side. Having seen the varying patterns, he commented ``{\it [this could be] a useful tool for observing the partisan divide not just from political pundits but also from normal citizens.}''

\paragraph{\bf Refine initial hypotheses with language cues (T5) and help mitigate overgeneralization on group characterization}
Continuing on his exploration, he found that he learned more about ``{\it group variation rather than group coherence}'' in \variscope. For example, two \blue-dominant subgroups (2 and 3) had quite different values in the \care dimension (Fig. \ref{fig:expert-interview-1}i,ii), while \red-dominant subgroup (10) was likely to express with a tone contrary to \care (i.e., {\it harm}), which was similar to that of the \blue-dominant subgroup (3) (Fig. \ref{fig:expert-interview-1}iii). Uncertain about what such similarity means, he used \languagescope to retrieve relevant language cues (Fig. \ref{fig:expert-interview-1}b). The highlighted text from the subgroup 2, \tweet{thoughts and prayers are} clearly expressed \care; on the other hand, the texts \tweet{chicago has lost so many to gun violence} from the subgroup 3 and \tweet{gun law country because they cannot defend themseleves} from the subgroup 10 both concerned the harm but there was a difference in what was responsible for the harm (gun violence vs. gun law). ``{\it [These languages differences] did show the varying aspect of concerns [on this gun issue]},'' but after observing the language cues (Fig. \ref{fig:expert-interview-1}b), he felt he needed to be more cautious in interpreting the ``similar'' language patterns. While he found more evidence for the issue polarization hypothesis, he felt his original set up through comparing the language similarity was insufficient and can be misleading if not inspecting the subtleness of how the languages are used in the issue contexts. He concluded, ``{\it [this tool] offers enough depth and information to allow me to learn from the complexity of messages}.'' 

\begin{figure}[!t]
    \vspace{-1.5em}
    \setlength\tabcolsep{2pt} % default value: 6pt
    \begin{tabular}{c}\\
    \includegraphics[width=0.95\linewidth]{figures/expert-interview-2.pdf}
    \end{tabular}
    \vspace{-1em}
    \caption{\label{fig:expert-interview-2}
    \textbf{Expert interview 2}: Summary of representative language cues for \fair and \domi from \blue-dominated and \red-dominated subgroups. The language cues from the subgroups represent various aspects of online campaigns and debates about guns and gun control policies. }
    \vspace{-1em}
\end{figure}

\subsection{Interview 2: Language insights for online activism and campaigns}

Expert 2 is a social psychologist interested in studying language use and narratives in online movements and campaigns. Having learned about the gun-issue dataset, she was eager to use \name to see how the two political camps differ in psycholinguistic dimensions, particularly in \fair and \domi. She hypothesized the two camps would show different patterns in \domi because she perceived a gradual shift in  public opinions (with recent polls showing increasing support for gun regularization policies), and ``{\it in this backdrop, conservatives may express a lower level of feeling in control}.'' Her hypothesis of the difference in \fair came from her understanding of the central argument on both sides: liberals view the gun regulation as a justified means to fairly guard the public safety ({\it fair}), whereas conservatives view the restriction on gun ownership and rights as putting people in danger ({\it unjust}). She was curious about how her hypothesized differences may reflect in the language used in tweets.

\paragraph{\bf Glance over the group patterns (T1), focus on specific attributes and nuance subgroup patterns (T3)}
Her attention was first drawn to \grouptrend (T1), ``{\it so nice…you can see the overall patterns for the two major group of tweets only at a few glances}.'' She further used the \scopecontroller to select the two focal dimensions (by unchecking others). After looking closely, she confirmed that the differences between the two camps were aligned with her initial hypotheses, and meanwhile, she noticed that the distinction in \fair seems to be greater than that in \domi. Observing this, she was now interested in adding more attributes in \grouptrend to examine whether other dimensions may have better distinguishing power than the two she originally focused on, ``{\it [this makes it] easier to inspect which [additional] dimensions could be more useful in differentiating the groups}.'' She noticed that the \blue tweets tend to cluster more closely around higher values in most dimensions, whereas the \red lines spread wider in all the dimensions, which suggests that some tweets from \red camps might be similar to those from \blue. ``{\it [This shows] a more complex picture [of the \red camp]}.'' Observing this, she concluded that the impressions based on the overall patterns may be too overgeneralized. To examine the complexity, she praised \variscope--subgroup for not displaying a simple, dichotomous picture of the two camps but capturing the varying patterns across subgroups, ranging from the most \blue-dominant group, to in-between purple ones, and to the most \red-dominant group.

\paragraph{\bf Establish test validity with language cues, generate a new hypothesis (T5) and help mitigate overgeneralization on group characterization} 
To examine the patterns beyond the dichotomy and to test her hypotheses, she decided to pick subgroups with distinct colors and compared their languages by the \languagescope. ``{\it [It is] so convenient [that it allows for] a quick check on the language sequences from the subgroups.}'' She mentioned that it was usually a complicated and even a tedious process to check the test validity from the natural language signals, and ``{\it a system like yours really facilitates people to navigate more qualitative, complicated messages beyond numbers.}'' From the \languagescope, she found several tweets supported her original hypotheses. For example (as shown in Fig. \ref{fig:expert-interview-2}), texts from \blue-dominant groups (Fig. \ref{fig:expert-interview-2}a-c) mentioning \tweet{to demand vote,} \tweet{join filibuster,} and \tweet{shout with on voice,} expressing a strong \fair tone about righteousness in advocating legal means to make a change, and texts from \red tweets (Fig. \ref{fig:expert-interview-2}d-e) like \tweet{failure of strict gun laws} and \tweet{laws not affect criminals} expressing a low level of \fair (unfair or unjust) tone. After looking at the language cues more closely, she pointed out that the languages within the \red-dominant subgroup were less coherent and direct, which matched the previous observations that the distribution of the scores varied more widely among \red tweets. For example, one \red tweet was actually in favor of gun regulation, ``\textit{vote on gun violence prevention legislation.}'' More, after reviewing more closely to the \red tweets at the lower end, she concluded, ``{\it this gives [new] insights too! ... makes me think of a new hypothesis that for \textit{Dominance}, at the lower end, the languages used in \red tweets may be less coherent. They shared less common narratives.}'' Our design -- which offers non-dichotomous exploration, together with the chance to inspect the language patterns and their variations -- enables her to engage in the kind of sense-making regarding within-group variations against over-generalized conclusions.


%may not have shared narratives in terms of gun issues.}''
\begin{figure}[!t]
    \vspace{-2em}
    \setlength\tabcolsep{2pt} % default value: 6pt
    \begin{tabular}{c}\\
    \includegraphics[width=0.95\linewidth]{figures/expert-interview-3.pdf}
    \end{tabular}  
    \vspace{-1em}
    \caption{\label{fig:expert-interview-3}
    \textbf{Expert interview 3}: Four edge cases from the subgroup 5 identified through the dual-sided histogram chart in \evalscope. The user selected a set of instances to conduct contrastive analysis: (a) comparing true cases (true \blue vs. true \red), and (b) comparing positive cases (true \red vs. false \blue).
    }
    \vspace{-1.6em}
\end{figure}

\subsection{Interview 3: Interpretable machine learning for discovering common ground and edge cases}

Expert 3 is a natural language processing (NLP) researcher who wishes to %understanding how interpretable ML helps with reasoning about the classification results behind the groups' psycholinguistic differences. 
better understanding the relationship between the interpretable ML's predictions and the groups' psycholinguistic differences.  
While walking through the system, she was particularly interested in examining the subgroup 5, which is a borderline subgroup, having roughly similar portions of members from both camps. She hoped this subgroup might reveal ``{\it what is the psycholinguistic common ground between the two sides?}'' She found \evalscope useful as ``{\it [it] provides an overview of the predictive quality and lets [her] inspect the false and true predictions more closely.}'' She noticed that the subgroup 5 had more edge cases (more false and true predictions with the posterior probabilities close to 0.5), which she thought would bring the interesting finding in understanding the classifier.

\paragraph{\bf Identify edge cases (T2) and examine the inference variability (T3)}
From \grouptrend, she found this particular subgroup had many \blue and \red lines in the middle-ranged values across all dimensions (Fig. \ref{fig:expert-interview-3}). She used \evalscope to find four sets of edge cases and clicked to select each set to see the attribute values across different sociolinguistic dimensions. She found that, when comparing the two true cases (true \blue and true \red on the right), the system gave correct prediction because the differences between the two edge cases, while small, were consistent with the major differences between the two camps -- in terms of \domi and \fair (Fig. \ref{fig:expert-interview-3}a-i and a-ii). When comparing with the two positive cases (false \blue and true \red at the bottom), she found the two sets of instances had similar values of \vale and \domi, but the overall \grouptrend showed that the highlight \blue lines had bigger variance in the two dimensions (Fig. \ref{fig:expert-interview-3}b-i). The lack of coherence in these dimensions from the \blue camp ``{\it would make it trickier for the classifier to do correct prediction.}'' She elaborated that the common ground would be likely to appear from the true edge cases rather than from the incorrect prediction resulted from the noisier basis on either side, and considered the system's ability to tell apart the edge cases very valuable. 

\paragraph{\bf Retrieve similar language patterns (T3) and compare the prediction rationales (T6)}
Noticing some tweets in the two camps had similar sociolinguistic attribute values, she was curious about how the system would explain the differences. She picked two tweets (that have very similar attribute values) and used \rationalescope to check why one tweet was classified as \blue and the other as \red (by selecting the o-mode in the contrastive explanation, as shown in Fig. \ref{fig:teaser}f). She was satisfied when the system returned a rationale indicating \fair as the most discriminative feature for the predictions. She concluded that such level of interpretability that {\it directly links the plain text to sociolinguistic features to group prediction} could be useful to help to determine whether the results were from machine behaviors or human behaviors. 

While using the system, she felt that system helped her gain a better understanding of the interactions between the data and the underlying ML model. She commented that ``{\it the suite of tools allowed [her] to both keep a global view about the data while drilling down to the more interesting subgroups.}'' She was particularly positive about \rationalescope for its contrastive explanation: ``{\it it is useful to be given not just the most discriminative feature but also two contrastive samples; even if I might not personally agree with a particular characterization (say if I don't think this tweet strongly expresses \fair) I at least get a sense for the range that the system is operating under by comparing it against the contrastive tweet.}'' As a suggestion for further development, she anticipate the future system might allow users to construct more directed queries beyond semantic similarity so that a user might dynamically create new data subsets and test new hypotheses.\\

\textbf{Mitigate the overgeneralization on group characteristics (T3-T6).} After exploring functions of \name, she compared it with other relevant tools she had experience with, such as the machine learning tool Weka \cite{witten2016weka}, or the visualization NLPReViz~\cite{trivedi2018nlpreviz} designed for similar purpose. She appreciated the ``analytic engagement'' in the current design -- not simply offering a model concluding what properties may be attributed to a group, but also analytical tools for
users to engage conversations with ``\textit{machine learned classifiers that do over-generalize by its nature and the predictions may not be always correct in general.}'' She explained, ``\textit{I feel that, in the typical group analysis, the burden is on the user’s side. It is usually user’s role to make sure not to jump to conclusion}.'' But, in \name, ``\textit{the overall framework gives the analysts more tools to visualize and study those predictions.}'' She elaborated on the point in details mentioning her experience that, ``{\it Some tools available out there like Weka, for example, offers statistics such as contingency table or uncertainty estimates, which is in the high-level but conveys just one-sided explanations. It does not help break down the reasoning behind it or back up evidences enough to provide the details on identified group characteristics.}'' She particularly highlighted our tool's capability to provide a variety of mechanisms and opportunities to inspect across subgroups at multiple levels of granularity including actual tweet instances and language patterns, so that ``\textit{the analysts could investigate the prediction outcome, the model's underlying rationales, and make informed judgment about whether the hypothesis holds.}'' She added that many tools such as NLPReViz only offered two views, either globally or at individual instances. Finally, in terms of what particular users may benefit from our design, she pointed out,  ``\textit{the contrasting explanation greatly helped non-content experts [who lack prior knowledge of the sociopsychological dimensions] ... it helped them make sense of why the system predicted certain labels}.''

%%%%%%%%%%%%%%%%%%%%%%%%%%%%%%
%         Discussion         %
%%%%%%%%%%%%%%%%%%%%%%%%%%%%%%

%--------------------------% 
%----- shock breakout -----%
%--------------------------% 
\subsection{Shock Breakout} \label{subsec:shock_breakout}

Just as with shock breakout and shock interaction for core-collapse supernovae, if the early emission is thermal, it can be used to probe characteristics of the progenitor and its immediate surroundings as well as the structure of the outflow. In the context of gamma-ray bursts, thermal emission can probe the progenitor-star photosphere (which is set by the mass loss and stellar radius), inhomogeneities of the mass-loss and structure of the jet, and its cocoon. For this paper, we focus on more fundamental aspects behind a thermal component to determine whether it is a reasonable explanation of the observed emission, deferring a detailed comparison of the data to models for a later paper.

If we assume the observed triggering emission is produced by the Lorentz-boosted thermal emission of shock breakout, the limits and shape of the emission can be used to constrain the properties of the shock as it emerges from the star. In the strong shock limit for a highly-relativistic gas, the pressure of the shock ($P_{\rm shock}$) is,

\begin{equation} \label{eq:P_shock}
    P_{\rm shock} \approx \Gamma^2 \rho_{\rm CSM} c^2
\end{equation}

\noindent
where $\Gamma$ is the Lorentz factor, $\rho_{\rm CSM}$ is the density in the region of shock breakout (in the wind of the massive star), and $c$ is the speed of light. Assuming the pressure is radiation-dominated, the corresponding temperature ($T_{\rm shock}$) of the emitting gas in the gas comoving frame is,

\begin{equation} \label{eq:T_shock}
    T_{\rm shock} \approx 1.9 (\Gamma/100)^{0.5} (\rho_{\rm CSM}/10^{-10}\,g\,cm^{-3})^{0.25} \; \textrm{keV}
\end{equation}

\noindent
and the corresponding peak energy of the emitted photons ($\nu_{\rm peak}$) in the observer frame is,

\begin{equation} \label{eq:nu_shock}
    \nu_{\rm peak} \approx 1 (\Gamma/100)^{1.5} (\rho_{\rm CSM}/10^{-10}\,g\,cm^{-3})^{0.25} \; \textrm{MeV}
\end{equation}

In the thermal shock breakout paradigm, the broad range and relatively flat spectra for the prompt emission requires a distribution of Lorentz factors. The observed peak emission around 15\,MeV places strong constraints on the upper limit of the Lorentz factor, corresponding to peak Lorentz factors lying between 300--1000, with corresponding densities of $70$--$0.05 \times 10^{-10} \, {\rm g \, cm^{-3}}$.

We consider if this triggering pulse would be identified in other GRBs. Some bright long GRBs have weak trigger intervals followed by quiescence before the main emission episode, but most do not. The 1.024\,s peak-flux interval of this pulse would trigger \gbm out to z$\approx$1.3. Approximately half of \gbm GRBs with measured redshifts are within this range. Thus, a similar pulse should have been recovered in other bursts. The particularly high $\Gamma$ for this burst would produce a higher luminosity shock breakout, which may explain the lack of identification in other collapsars.

%--------------------------% 
%----- lorentz factor -----%
%--------------------------% 
\subsection{Lorentz Factor} \label{subsec:lorentz_factor}

With the requirement that the prompt emission site be optically thin, we can derive lower limits on the Lorentz factor. For this method we usually use the minimum variability timescale in conjunction with the spectral shape. Here we can only derive the MVT for the time intervals without PPU. The spectrum within the \gbm BTIs however yields stronger constraints on $\Gamma$ even with conservative assumptions on the minimum variability timescale. Based on Figure\,\ref{fig:mvt} and the discussions in Section\,\ref{subsec:mvt}, we assume a conservative variability timescale estimate of $\Delta t_{\rm var}=0.1 $\,s.

The Lorentz factor limits involve pair-production ($\Gamma_{\rm pp}$) for the highest energy photons interacting with the \gbm prompt spectrum and the overall requirement that the \gbm spectrum be optically thin ($\Gamma_{\rm ot}$; \citealt{lithwick01}). \lat observed a 99.3 GeV photon at \t0+240\,s \citep{gcn32658}. Assuming the \gbm emission emanates from the same volume, the requirement that such a photon can escape without producing $e^{\pm}$ pairs yields a Lorentz factor constraint of $\Gamma_{\rm pp, min}\gtrsim 1560$. The calculation of $\Gamma_{\rm min}$ by \citealt{Lithwick2001} is done under the assumption of a single zone. If we consider a more realistic situation, taking into account the angular, temporal and spatial dependence of the radiation field, our values of $\Gamma_{\rm min}$ could be lower by a factor of \abt2 (i.e., $\Gamma_{\rm pp, min}\gtrsim 780$) \citep{Hascoet+12gamgam,Gill_2018, Vianello_2018, Arimoto_2020}.

The requirement that the \gbm emission is optically thin, given the spectrum in the brightest interval, yields a limit of $\Gamma_{\rm ot, min}\gtrsim 1040$. If we instead use $\Delta t_{\rm var}=0.05 $\,s, which is still reasonable given the MVT values surrounding region IV, we find a Lorentz factor limit of $\Gamma_{\rm ot, min}\gtrsim 1470$. Although both MVT-derived Lorentz factor lower limits fall below the single zone pair-production Lorentz factor lower limit, the two methods independently produce consistent results.

We can also place constraints on the Lorentz factor from the triggering pulse. Based on our COMP function spectral fits in region Ia ($\alpha=-1.69\pm0.01$, $E_{\rm peak}=3.98\pm0.37$\,MeV, and energy flux=$(1.98\pm0.03)\times 10^{-6}$ erg cm$^{-2}$ s$^{-1}$ in the 10-1000 keV range) and requiring the source be optically thin, we find $\Gamma_{\rm trig, min}\gtrsim 188$. Although less constraining, this value is consistent with the value of 300-1000 derived via a shock-breakout assumption. There are no photons in the GeV range that are unambiguously associated with the GRB during this time interval, so the pair-creation Lorentz factor limit is not constraining in this case.

The peak of the afterglow emission denotes the beginning of the external shock deceleration time. By identifying this peak for ISM and wind-type external media as $t^{\rm ag, ISM}_{peak}\gtrsim t_{0}+(140\pm1.5)$\,s and $t^{\rm ag, wind}_{peak}\gtrsim t_{0}+(120\pm6.5)$\,s respectively we can derive upper limits for the Lorentz factor of the afterglow via,

\begin{equation} \label{eq:Lorentz_factor_general}
    \Gamma_{\rm ag} = \pp{\p{\frac{(17-4s)}{16\pi(4-s)}}\p{\frac{E_{k}}{n_0 m_{\rm p}c^{5-s}}}}^{\frac{1}{8-2s}}\p{\frac{t_{\rm peak}}{(1+z)}}^{-\frac{3-s}{8-2s}}
\end{equation}

\noindent
where $s=0$ assumes an ISM external density profile and $s=2$ assumes a wind-type external medium \citep{GammaNappo,Ghirlanda+18Lorentz}. For both cases we use the VLT reported redshift of z = 0.151 \citep{gcn32648} and assume the kinetic energy, $E_k$ is approximately the isotropic-equivalent gamma-ray energy ($E_{k}\approx E_{iso}$).

Assuming ISM ($s=0$) we have:

\begin{equation} \label{eq:Lorentz_factor_ISM}
    \Gamma_{\rm ag, ISM} \gtrsim 260 \left(\frac{E_{k,55}}{n/1~ {\rm cm}^{-3}}\right)^{1/8} \left(\frac{t_{\rm peak}}{120 ~{\rm s} \times 1.151}\right)^{-3/8}
\end{equation}
Using the $t_{\rm peak} = 105 $ s value, increases slightly to $\Gamma_{\rm ag, ISM}\gtrsim 270$ (all scaling parameters are the same)

For the wind-type external medium ($s=2$) the density parameter $n= 3\times 10^{35}~A_\star~ {\rm cm}^{-1}$ \citep{grb221009a_fermi_lat_collaboration_2023} which gives us,

\begin{equation} \label{eq:Lorentz_factor_wind}
    \Gamma_{\rm ag, wind} \gtrsim 282 \left(\frac{E_{k,55}}{A_{\star,-1}}\right)^{1/4} \left(\frac{t_{\rm peak}}{ 140 ~{\rm s} \times 1.151}\right)^{-1/4}
\end{equation}

For a spherically emitting shell, the MVT relates the radius of the shell to the Lorentz factor via,

\begin{equation} \label{eq:min_radius}
    \textrm{R} \sim \Delta t_{\textrm{min}} \p{\frac{\Gamma^{2} c}{\p{1+z}}}
\end{equation}

\noindent
where R is the radius of the spherically emitting shell, $\Gamma$ is the Lorentz factor, $c$ is the speed of light, and $z$ is the measured redshift. Along these lines, the MVT and the bulk Lorentz factor can be used to place limits on emitting shell radius, assuming a singular emitting region. In interval Ia, the average value of MVT is \abt0.1\,s. Using the afterglow-derived bulk Lorentz factor for the ISM ($\Gamma_{\rm ag, ISM} \gtrsim 260$) and the wind-type external medium ($\Gamma_{\rm ag, wind} \gtrsim 282$) as a proxy for the bulk Lorentz factor in this region gives us an estimate on the radius of the emitting star. We find $R_{\star, ISM} \gtrsim 1.7 \times 10^{14}$\,cm and $R_{\star, wind} \gtrsim 2.0 \times 10^{14}$\,cm for the ISM and wind-type media respectively. Both of these values are roughly consistent with the radius of the outer wind from the Wolf-Rayet progenitor star ($R_{\star} > 1 \times 10^{13}$\,cm) \citep{Crowther2007}.

Using this same equation, the lower limit of the shock breakout derived triggering pulse Lorentz factor ($\Gamma=300$), and a delay time of \abt220\,s between regions I and III, we get an internal dissipation radius of \abt$6\times10^{17}$\,cm. This value is larger than is typically discussed but could be consistent with the ICMART model of $10^{16}$\,cm, which is consistent with the non-detection of neutrinos \citep{gcn32665, gcn32741} and a Poynting flux dominted jet \citep{Zhang2010}.

%--------------------------% 
%------- neutrinos --------%
%--------------------------% 
% \subsection{Neutrinos} \label{subsec:neutrinos}


%--------------------------% 
%---------- SSC -----------%
%--------------------------% 
% \subsection{Synchrotron Self-Compton} \label{subsec:ssc}


\section{Conclusion}

In this paper, we proposed \name, a visual analytic system for group differences. Our work is a first attempt at creating a data visualization that aims at promoting a conscientious, interactive experience for users to negotiate with and ponder about analytical results from computational predictive models. The challenge resides in how to retain the complex statistical results to a level that could indicate the group difference patterns derived from computational models succinctly, but not conclusively. Our interface design affords the users opportunities to engage in further analytical thinking beyond what the computational models have offered. Our evaluation by expert interviews suggests \name is a promising design for hypothesis generation and testing for data analysts. 

%In this paper, we proposed \name, a visual analytic system for group difference. Our study is designed for the purpose of maximizing the interpretability of group analysis taken by experts: The analytic pipeline helps decompose the factors in a hierarchical manner in both high-level and language-level, and our explainability modules supports exploring global and local explanations, where experts can proactively conduct the analysis that is responsible and insightful. To evaluate the usefulness of our system, first, the case study demonstrated the exploration of different granularity of instances in the group, subgroup, and instance-level. The three expert interviews showcased the process of expert’s hypothesis generation and validation, where they were able to find the higher-level group differences, then validate it with language-level patterns, upon theories of expert’s interest in multiple domains.

%We plan to improve the capability of our system reflecting the feedbacks described for more generalized and insightful group analysis. We expect that our system as an exemplary tool can shed light on a set of analytic modules and pipelines for experts’ insights in better understanding groups, which will bring better public policies and services that benefit and encompass heterogeneous groups, across public and private sectors in our society.

\section*{Acknowledgement}
We thank the anonymous referees for their useful suggestions. The authors would like to acknowledge the support by the grants from the PICSO Lab, including DARPA UGB, NSF \#1739413, \#2027713, AFOSR awards, and Adobe Research Grant. Any opinions, findings, and conclusions or recommendations expressed in this material do not necessarily reflect the views of the funding sources.
\section{Appendix}

\subsection{Psycholinguistic Attributes}\label{sec:attr-def}

Drawing upon literature~\cite{graham2009liberals, shepherd2018guns,mendez2017neurology}, we identify seven most relevant sociolinguistic attributes that could potentially predict how two ideological groups talk differently on the gun issues.

Two {\it affect} dimensions:
\begin{itemize}
\item \vale: emotions can range from positive (e.g., pleasant, happy, hopeful) to negative (e.g., unhappy, annoyed, despairing)
\item \domi: emotions can range from the most dominant (e.g., feeling-in-control, influential, autonomous) to the least dominant (e.g, weak, submissive, and guided)
\end{itemize}

Five {\it moral} foundations:
\begin{itemize}
\item \care: the virtue of caring, nurturing, and protecting the vulnerable
\item \fair: the virtue of reciprocal altruism, including justice, rights, and welfare
\item \auth: the virtue of respect for authority
\item \loya: the virtue of being loyal to your identified groups
\item \puri: the virtue of seeing the human bodies as holly temples that should not be contaminated
\end{itemize}

\subsection{Human Annotated Attribute Values}\label{sec:annotation}

The human annotation included two phases: (1) creating reliable coding rules, and (2) coding. In the first phase, a major task is to the create the inclusion criteria for human annotators to identify the language signals that correspond to the theorized attributes in tweets. To do so, we sampled a subset of tweets (10-40\%) for each attribute from the total of 3100 relevant tweets. Through an iterative process, one of our authors who is in the field of social psychology began with open-coding to evaluate how the theoretical constructs and categories can be manifested in tweets' language use. She identified the discourse features and themes, and then built, tested, and refining the rules with a graduate research assistant. The inclusion criteria were created with 100\% agreement between the two criteria developers. Once the criteria were set up, each of these tweets was then coded by two independent annotators by a group of four research assistants who did not participate in the the criteria development stage but were trained to follow the coding schemes. These research assistants were chosen because they had been trained prior to this project and developed skills to analyze social media discussions that involve complex politics and social contexts. For each tweet, the annotators determined whether the tweet texts involved each of the seven attributes as a set of binary outcomes. The coding in this phase resulted in fair to substantial agreements between the annotators, with inter-rater reliability in terms of the Cohen's kappa ranging from 0.32 to 0.88 across all attributes. Any disagreement was reconciled after discussion and the coding criteria and procedure were formulated through the process. In the second phase, every tweet (from the 3100 relevant set) was annotated. For {\it moral} attribute values (e.g., \fair, \auth), we followed the coding schemes developed from the first phase. The annotation generated categorical values for each of the moral attribute. For {\it affect} (e.g., \vale), we determined to adopt the Best-Worse Scaling used by Mohammad et al.~\cite{mohammad2018obtaining} after testing it in the first phase. This annotation scheme employed comparative annotation method, which can be used to generate continuous rating for an attribute. We adopted this method and implemented the coding through crowdsourcing on Amazon Mturk. In the crowdsourcing annotation, three annotations are required for each of the $2N$ tweet-tuples (where each tuple contains 4 randomly-grouped tweets, and $N=3100$ in our case) in order generate reliable annotation results. Finally, the annotated scores for affect attributes are normalized to range from -1 to 1.

%The human annotation of the present sociolinguistic attributes involved: (1) an open coding phase, and (2) a categorical coding phase. First, the open coding phase is to generate the coding criteria and procedure for coding a tweet on each attribute based on the theoretical definitions in literature \cite{XX}. To do so, we sampled a subset of tweets (10-40\%) from the total of 3100 relevant tweets. Each of these tweets was coded by two independent annotators who were familiar with the theoretical definitions. For each tweet, the annotators determined whether the tweet texts involved each of the seven attributes as a set of binary outcomes. The coding in this phase resulted in fair to substantial agreements between the coders, with inter-rater reliability in terms of the Cohen's kappa ranging from 0.32 to 0.88 across all attributes. Any disagreement was reconciled after discussion and the coding criteria and procedure were formulated through the process. In the second phase, every tweet (from the 3100 relevant set) was annotated. For {\it moral} attribute values (e.g., \fair, \auth), we followed the coding schemes developed from the first phase. The annotation generated categorical values for each of the moral attribute. For {\it affect} (e.g., \vale), we determined to adopt the Best-Worse Scaling used by Mohammad et al.~\cite{mohammad2018obtaining} after testing it in the first phase. This annotation scheme employed comparative annotation method, which can be used to generate continuous rating for an attribute. We adopted this method and implemented the coding through crowdsourcing on Amazon Mturk. In the crowdsourcing annotation, three annotations are required for each of the $2N$ tweet-tuples (where each tuple contains 4 randomly-grouped tweets, and $N=3100$ in our case) in order generate reliable annotation results. Finally, the annotated scores for affect attributes are normalized to range from -1 to 1.
%\yrl{WT: please check and fix issues; also could you provide a short/brief layman description about each attribute in the bullet points below?}

%\begin{itemize}
%\item \vale:
%\item \domi:
%\item \care:
%\item \fair:
%\item \puri:
%\item \auth:
%\item \loya:
%\end{itemize}

% \wtc{The annotation involved two phases. The first was to develop the operational definitions and coding schemes of all the attributes. A subset of tweets (10-40\% of the total of 3100 tweets) were randomly sampled, used to generate operational definitions, and each tweet was coded by two independent annotators for obtaining reliable results. For each tweet, the annotators determined whether the tweet texts involved the seven attributes, yes or no, as a binary code.  Reliability testing indicated fair to substantial agreements, with the values of Cohen's kappa ranging from .32 to .88. In the second phase, every tweet was annotated. For moral values, we followed the coding schemes developed in phase 1. For affect, we further adapt the Best-Worse Scaling ~\cite{mohammad2018obtaining}, an annotation scheme that employs comparative annotation, which can generate continuous values instead of binary codes, for each dimension, that allows the direct comparisons among tweets. Prior empirical studies  ~\cite{mohammad2018obtaining} have developed a reliable procedure to annotate emotions from words, which suggested to use three annotations to annotate 2N 4-randomly-grouped-tweets (N is the numbers of data item; in our study, the total number of tweets, 3100) is sufficient to have reliable scores. We followed the recommendations and implemented through crowd-sourcing on Mturk. The scores are normalized so range from -1 to 1.}. 


\subsection{Evaluation for the multi-task prediction}\label{sec:holdout}

\begin{table*}[ht]
\footnotesize
\begin{center}
\caption{\textbf{Results of Multi-task Prediction.} We report results of the Attribute Prediction tasks and Group Label prediction tasks. Performance changes of the Multi-task Predictions compared to baselines are reported in parentheses}\label{tb:pred_performance}
\begin{tabular}{c|c|cc}

\toprule
& \bf Attribute Prediction &\multicolumn{2}{c}{\bf Group Label Prediction} \\

& Acc. (CLF) or Pearson \textit{r}(REG)& Accuracy & F1 \\
\midrule

Dominance (REG) & 0.804 (-0.035) & \bf 0.814 (+ 0.009) & \bf 0.809 (+ 0.014)\\
Valence(REG) & 0.768 (- 0.015) & 0.809  & 0.805\\
Harm(CLF) & 0.623 (+ 0.012) & 0.812 & \bf 0.809 (+ 0.014)\\
Fairness (CLF) & 0.68 (+ 0.001) & 0.803 & 0.802\\
Authority (CLF) & 0.742 (+ 0.007) & 0.799 & 0.794\\
Purity (CLF) & 0.975 (+ 0.003) & 0.793 & 0.791\\
Loyalty (CLF) & 0.825 (+ 0.009) & 0.802 & 0.798\\ \midrule
Dominance Baseline & 0.839 & N/A  & N/A \\
Valence Baseline & 0.783 & N/A & N/A \\
Harm Baseline & 0.611 & N/A & N/A \\
Fairness Baseline & 0.679 & N/A & N/A \\
Authority Baseline & 0.735 & N/A & N/A \\
Purity Baseline & 0.972 & N/A  & N/A \\
Loyalty Baseline & 0.816 & N/A & N/A \\ 
\midrule

Group Baseline & N/A & 0.805 & 0.795\\

\bottomrule
\end{tabular}
\end{center}



\end{table*}


We evaluate the multi-task prediction models using a hold-out experiment on the 3100 relevant tweets, where 50\% samples are used for training, 15\% samples for validating, and the remaining samples for testing. 
We consider two types of baseline models: (a) group prediction baseline 1: to predict group labels with all the annotated attributes on sample tweets using standard machine learning method; (b) group prediction baseline 2: to predict group labels with tweet texts on sample tweets using single-task neural network architecture; and (c) attribute prediction baseline: to predict a single attribute value with tweet texts on sample tweets using single-task neural network architecture. 
The group prediction task is evaluated using {\it accuracy} as our dataset is balanced. 
The attribute prediction tasks are evaluated in terms of the {\it Pearson correlation coefficient} for continuous attributes, and by {\it accuracy} for categorical attributes. 
Table~\ref{tb:pred_performance} report performances of all models and baselines.
The performance gain (or loss) of multi-task models compared to the baselines are reported in the parentheses. 
We highlight key observations from the results: (1) For group prediction, our best model achieves accuracy 0.814, outperforming the two group prediction baselines by up to 30\%. (2) For attribute prediction, the performance measures of our models range 0.768--0.804 in terms of \textit{Pearson} correlation (for the two continuous attributes) and 0.623--0.975 in terms of accuracy (for the five categorical attributes), which is very close to the attribute baseline (with only 0.3\% differences on average). 
Such results suggest that our multi-task models can significantly improve group prediction without sacrificing the performance for attribute prediction.


\bibliographystyle{ACM-Reference-Format}
\bibliography{references}

\end{document}
