\section{Design Guidelines \& Tasks} \label{sec:goalsandtasks}


\begin{table}[]
\resizebox{\textwidth}{!}{%
\begin{tabular}{@{}lllll@{}}
\toprule
 &
  \textbf{Expertise} &
  \textbf{Goal of analysis} &
  \textbf{Workflow} &
  \textbf{Challenges and Limitations} \\ \midrule
\textbf{Expert A} &
  \begin{tabular}[c]{@{}l@{}}Online movements \\ and campaigns\end{tabular} &
  \textit{\begin{tabular}[c]{@{}l@{}}"How do groups hold \\ different emotional and \\ moral attitudes towards \\ social issues and capture \\ a concrete evidence \\ from their utterances?"\end{tabular}} &
  \begin{tabular}[c]{@{}l@{}}Count the word occurrences in R;\\ Fit multiple models with \\ filtered words for each attribute;\end{tabular} &
  \begin{tabular}[c]{@{}l@{}}Finding anecdotal evidence \\ for complex psychological \\ dimensions is not feasible, \\ or it is shallow when available\end{tabular} \\
\textbf{Expert B} &
  \begin{tabular}[c]{@{}l@{}}Team \\ communication\end{tabular} &
  \textit{\begin{tabular}[c]{@{}l@{}}"How do demographic groups \\ form the faultline with respect \\ to team members' attributes \\ within or across the groups?"\end{tabular}} &
  \begin{tabular}[c]{@{}l@{}}Run a linear regression with STATA; \\ Do clustering analysis with Tableau;\\ Examine subgroup members \\ in spreadsheet;\end{tabular} &
  \begin{tabular}[c]{@{}l@{}}Interaction between deductive \\ and inductive analysis is \\ inconvenient; (i.e., how is a \\ subgroup characterized \\ (inductive) or attributed \\ (deductive) to each attribute?\end{tabular} \\ \bottomrule
\end{tabular}%
}
\caption{Examples of the pilot study result.}
\label{tbl:pilot_study}
\end{table}

In this section, we propose a set of design guidelines of group-level analytic tools for facilitating a conscientious practice of analyzing group characteristics. To formulate the design of a group analysis tool, we collect the feedback and thoughts from potential users in using existing tools. We assume that such tools will be used by data analysts and experts who conduct a group-level analysis in different domains. In order to understand the current practice and goal of group analysis and the challenges from it, we conducted the pilot study with domain experts and identified the concerns and limitations of the group-level analysis in their workflow. Based on the findings of the pilot study, our design guideline, which consists of three goals and six specific tasks, was formulated by compiling common aspects across the interviewees’ comments. In the following paragraphs, we describe the process of deriving design guidelines in detail.

\textbf{Pilot study.} Five interviewees were chosen from a variety of domains and specializations, ranging from data science, psychology, education, language, to anthropology. All these domain experts had prior experiences with human group research using digital trace data. Each interview lasted about an hour, in the form of a semi-structured interview. During the interview, we engaged the domain experts to consider a simple scenario of using structured and unstructured data to conduct group analyses, encouraged them to apply their current practices, and reflect on the limitations and concerns of the analytic tools they currently used. We organized the interview session with three primary questions to facilitate the interviewees’ thinking process:

\begin{itemize}
    \item \textbf{Expertise/Workflow}: “What is your expertise and in what way typically do you get insights on such analysis?”
    \item \textbf{Goal of analysis}: “What insights do you anticipate to find out?”
    \item \textbf{Challenges/Limitations}: “Despite your current practice of group analysis with available tools, how does it fail to meet your needs?”
\end{itemize}


While interviewees reported a set of goals and challenges/limitations in their own context of group analysis as shown in Table \ref{tbl:pilot_study}, we were able to find that they had experienced similar difficulties in their goals and tasks. For example, a typical analysis goal of three interviewees is not only to test their hypothesis over the group/subgroup characteristics but also to find qualitative evidence to back up their test statistics. As summarized in Table \ref{tbl:pilot_study}, many existing tools they currently used in their workflow, such as STATA, R, Tableau, or simple software for dealing with spreadsheet, did not fully support the range of group analytic function -- from overviewing group characteristics, running a regression/prediction model, to identifying subgroup patterns across attributes.


After all interviews were conducted, we compiled a list of pairs of goal and challenges for each interviewee (within-interview), and grouped common facets of pairs (e.g., (goal) \textit{capture both quantitative and qualitative evidence} - (challenge) \textit{hard to capture it with respect to each attribute}) shared by interviewees (between-interviews). Based on the result of the interview study, we identified six concerns that were commonly mentioned by interviewees (denoted as \textbf{C}):

\begin{itemize}
    \setlength{\itemsep}{0.2pt}
    \item[\textbf{C1.}] Hard to capture the overview of group differences.
    \item [\textbf{C2.}] No links or traces between deductive and inductive analysis.
    \item [\textbf{C3.}] Lack of capability in analyzing both within-variance of groups and instances.
    \item [\textbf{C4.}] Unreliability without model quality inspection or attribute importance.
    \item [\textbf{C5.}] Less informative qualitative details.
    \item [\textbf{C6.}] Lack of supports in providing rationales on individual decisions.
\end{itemize}

These six common facets of concerns can be grouped and summarized into three main implications:
{\bf (1) A lack of bridge between top-down and bottom-up group analysis:} During the interviews, we found the discrepancy in common practices of conducting group analyses among experts. Some of our interviewees prefer to use deductive approach, e.g., using regression or other predictive modeling to test hypotheses, while others mostly use inductive approach, e.g., using clustering techniques to discover patterns not previously hypothesized. Especially for some research topics (e.g., analyzing team or demographic faultlines), clustering task tends to be a prevalent way of characterizing subgroups that are distinctive in their traits. Based on the interviews, we learned that a better tool may facilitate users not only to see the big picture (\textbf{C1}), but to efficiently trace the links between the higher-level patterns (hypothesized or not) (\textbf{C2}) and the instances located in the database (\textbf{C3}), which  %in order to 
allows in-depth analysis by crossing over the current two practices of top-down and bottom-up group analysis.
%A researcher ``{would like to see face validity in the raw texts before testing my hypotheses},'' and another commented ``{I often questioned how many different snowflakes in the data may have the shapes I just found}.'' 
{\bf (2) A lack of confidence in sophisticated tools and the analysis results:} Several interviewees expressed concerns about the lack of transparency when using sophisticated machine learning tools. For example, these techniques are not helpful for explaining the group culture with hypotheses (\textbf{C4}) or offering qualitative evidence (\textbf{C5}).
%A researcher prefers to ``{\it know how to explain the group culture with my hypotheses, but found it difficult with sophisticated techniques},'' and another ``{it's too hard to grapple with anecdotal evidence with those sophisticated statistics.}."
{\bf (3) A concern about the implication for group-level analyses:} A general concern repeatedly stated by the interviewees is that the correlations learned from the big data by sophisticated tools may be translated to decisions that benefit or harm certain individuals (\textbf{C6}). In particular, several of them referred to the concern of profiling under GDPR \cite{gdpr}. 

Based on the feedback from domain experts, we determine six tasks (denoted as {\bf T}) that address the aforementioned concerns, which were grouped to four guidelines (denoted as \textbf{G}) in a bottom-up manner.

\paragraph{\bf G1. Identify the shared characteristics of a group of people and the differences between groups.} The analytic tool should allow users to see how groups of people share the common characteristics and how groups differ from each other in terms of key attributes of interest.

\begin{itemize}
    \setlength{\itemsep}{0.2pt}
\item[{\bf T1}] \textbf{Group trend}: in our design, the characteristics of groups will be visualized as ``group trend'' and the differences between groups will be contrasted through visual encoding.
\item[{\bf T2}] \textbf{Inference reliance}: the tool will support users to assess the analytic model quality against ground-truth data whenever available. 
\end{itemize}

\paragraph{\bf G2. Inspect the shared characteristics and variance of groups in both quantitative and qualitative ways for hypothesis testing and searching.} The tool should allow users to closely inspect the group differences in two ways: it should provide the quantitative evidence showing how the key characteristics or attributes differentiate the groups, and the qualitative details where specific instances from the data can be retrieved to corroborate the identified characteristics. It should also allow users to discover variance from the explicit grouping to reduce the possible overgeneralization of the group distinctions.

\begin{itemize}
    \setlength{\itemsep}{-0.5pt}
\item[{\bf T3}] \textbf{Group variance}: the tool will extract and visualize subgroup characteristics to support the examination of the within-group trends and variation.
\item[{\bf T4}] \textbf{Attribute importance}: how groups are differentiable by the key characteristics or attributes will be visualized as ``attribute importance'' -- the dependence of a given attribute when predicting a group. 
\item[{\bf T5}] \textbf{Qualitative details}: the tool will enable users to retrieve the qualitative cues from individual data instances that are representative for each quantitative measured attribute.
\end{itemize}

\paragraph{\bf G3. Provide the rationale for the prediction of an instance as a group member.} The tool should allow users to understand the rationale behind each individual's group assignment produced by any analytic/predictive models. The rationale should allow users to connect to or verify the identified group characteristics. The tool should provide a module to offer instance-level explanation, which is complementary to the approaches for describing model behavior with feature importance (in \textbf{G2}) or aggregated individual explanations \cite{krause2018user, stumpf2016explanations, poursabzi2018manipulating}.

\begin{itemize}
    \setlength{\itemsep}{0.2pt}
\item[{\bf T6}] \textbf{Grouping rationale}: in our design, the group rationale will be offered through a ``contrastive explanation'' -- to explain why a member is considered to belong to one group rather than the other.
\end{itemize}
