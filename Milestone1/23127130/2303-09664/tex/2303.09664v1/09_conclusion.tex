\section{Conclusion}

In this paper, we proposed \name, a visual analytic system for group differences. Our work is a first attempt at creating a data visualization that aims at promoting a conscientious, interactive experience for users to negotiate with and ponder about analytical results from computational predictive models. The challenge resides in how to retain the complex statistical results to a level that could indicate the group difference patterns derived from computational models succinctly, but not conclusively. Our interface design affords the users opportunities to engage in further analytical thinking beyond what the computational models have offered. Our evaluation by expert interviews suggests \name is a promising design for hypothesis generation and testing for data analysts. 

%In this paper, we proposed \name, a visual analytic system for group difference. Our study is designed for the purpose of maximizing the interpretability of group analysis taken by experts: The analytic pipeline helps decompose the factors in a hierarchical manner in both high-level and language-level, and our explainability modules supports exploring global and local explanations, where experts can proactively conduct the analysis that is responsible and insightful. To evaluate the usefulness of our system, first, the case study demonstrated the exploration of different granularity of instances in the group, subgroup, and instance-level. The three expert interviews showcased the process of expert’s hypothesis generation and validation, where they were able to find the higher-level group differences, then validate it with language-level patterns, upon theories of expert’s interest in multiple domains.

%We plan to improve the capability of our system reflecting the feedbacks described for more generalized and insightful group analysis. We expect that our system as an exemplary tool can shed light on a set of analytic modules and pipelines for experts’ insights in better understanding groups, which will bring better public policies and services that benefit and encompass heterogeneous groups, across public and private sectors in our society.

\section*{Acknowledgement}
We thank the anonymous referees for their useful suggestions. The authors would like to acknowledge the support by the grants from the PICSO Lab, including DARPA UGB, NSF \#1739413, \#2027713, AFOSR awards, and Adobe Research Grant. Any opinions, findings, and conclusions or recommendations expressed in this material do not necessarily reflect the views of the funding sources.