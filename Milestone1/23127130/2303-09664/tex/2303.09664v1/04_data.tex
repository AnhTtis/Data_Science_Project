\section{Application Scenario and Data} \label{sec:appanddata}

We take the aforementioned data journalist example as an application scenario to design a visual analytic system following the design guideline. In this scenario, a user (the data journalist) is interested in exploring Twitter users' communications related to gun and gun-control issues. The user wonders: ``{\it how do social media users with different political leanings talk differently on the gun issues?}'' ``{\it how do the differences revealed through the hypothesized attributes -- particularly certain sociolinguistic characteristics -- and manifested on users' tweets?}'' and ``{\it how can I make sense of and trust the group analysis results?''}

We make use of a publicly available Twitter dataset from a prior study \cite{yan2017quantifying}. This dataset contains more than 600k Twitter users that have been identified with liberal or conservative leaning based on their following behaviors \cite{yan2017quantifying,yan2020mimicprop}. From this data, we have identified a total of 3,100 tweets and 2,256 users that are related to ``gun rights'' or ``gun control'' discussions. In addition to standard keyword matching method, we use a PU learning algorithm \cite{fusilier2015detecting} to identify related tweets, then validate the results manually. In order to verify the machine-inferred sociolinguistic characteristics from the tweet text, each tweet was manually annotated with seven sociolinguistic attributes, including \textit{Care}, \textit{Authority}, \textit{Purity}, \textit{Fairness}, and \textit{Loyalty} as described by Graham et al. \cite{graham2009liberals} as moral values, and \textit{Valence} (the happiness expressed) and \textit{Dominance} (the degree of control exerted) \cite{warriner2013norms} as affects. We summarize the definitions of these seven attributes and their annotation process in the Appendix (Section~\ref{sec:attr-def} and \ref{sec:annotation}). This carefully curated dataset allows us to design and test our new visual analytic toolkit with the set of {\it ground-truth} group labels and attribute values in a real analysis scenario. In the following sections, we use ``\red'' and ``\blue''  as the group labels.
