\section{Use Case Scenario}\label{sec:case}
\begin{figure}[!t]
    \centering
    \vspace{-0.9em}
    \includegraphics[width=0.95\columnwidth]{figures/case-study-individual.pdf}
    \vspace{-0.9em}
    \caption{\label{fig:case-study-individual}
    \textbf{Use case scenario (T5)}: While browsing tweets related to the ``Orlando shooting'' event, (a)-(b) the user explores two tweets from each group with different \domi values, and identified the expression of \domi from the psycholinguistic score chart. The user (c) explores the expression of \auth, and (d) can further compare the tweet against others within the subgroup 3 with its polyline highlighted as green in \grouptrend.}
    \vspace{-1.3em}
\end{figure}

We now present a use case scenario for how \name facilitates accountable group-level analyses. Consider the earlier data journalist example -- Erin is trying to examine the public's sentiments on social media after a mass shooting event. In particular, she would like to determine whether online users with liberal or conservative-leaning might talk differently over the topics of gun violence, gun policies, and related issues. She hopes a deep dive into online users' conversation would give her more insights in addition to the polls that have been reported elsewhere. However, merely searching tweets through a search engine does little for her goals. In this scenario, we use aforementioned Twitter data (see Section \ref{sec:appanddata} for the details of the dataset).

\paragraph{\bf Retrieve relevant tweets and qualitative cues (T5)}
Fig. \ref{fig:case-study-individual}a-b shows how she can use \name to quickly identify relevant tweets with diverse expressions. She entered ``orlando'' as a search term, expecting to find tweets about the Orlando shooting incidents happening in 2016. This event has provoked intense social media reactions, nationwide debates, and subsequent legislative actions. As a result of search, the keyword query returned 166 tweets shown on the \instanceviewer, from both the \blue and \red camps (e.g., Fig. \ref{fig:case-study-individual}a and b, respectively). To look at how these users talked about the event differently, she explored the {\it psycholinguistic summary} bar chart as shown at the top-right corner of each tweet. She found the first two tweets seemed to be quite different in terms of expressing \domi (a sense of feeling in control or losing control of a certain state). This prompted her to look for more tweets that express \domi: ``Can I find such expressions from both groups to compare?'' She clicked the bar `D' (abbr. for \domi) from those tweets (Fig. \ref{fig:case-study-individual}a), the first tweet (from User2560 in the \blue camp) had a highlighted language \tweet{must be love and stronger gun controls}, and the second (from User2185 in the \red camp) had \tweet{even mention that the Orlando shooting had islamic ties} (Fig. \ref{fig:case-study-individual}b). These illustrate how the two users had expressed the sense of in control or losing control differently when commenting on the Orlando shootings. Out of curiosity, she clicked `A' (abbr. for \auth) to see what an expression of \auth may look like (Fig. \ref{fig:case-study-individual}c) -- e.g., a retrieved tweet with highlighted text  \tweet{letting my senator know that I support gun control} suggests the user called his/her senator (authority) to take the leadership. 

% Erin noticed that, on \grouptrend, the two dimensions, \vale and \domi, tend to have blues lines on the top (Fig. \ref{fig:case-study-group}b), suggesting she may find more expressions of \vale and \domi from the \blue camp.

\begin{figure}
  \centering
  \vspace{-0.8em}
  \includegraphics[width=0.95\columnwidth]{figures/case-study-group.pdf}
  \vspace{-1em}
    \caption{\label{fig:case-study-group}
    \textbf{Use case scenario (T1, T4, T5)}: The user (a)-(d) explores the major trends in \grouptrend and the group-wide attribute importance in \depscope, and (e) identifies the language cues in \languagescope.}
    \label{fig:case-study-group}
\end{figure}

\paragraph{\bf Overview group trend (T1), check attribute importance (T4), and retrieve qualitative evidence (T5)}
Erin used \grouptrend to get an overview of the major differences between groups. 
In addition to the \blue camp's general associations with a higher value of \vale and \domi (Fig. \ref{fig:case-study-group}c), she found that the two camps seemed to have mixed scales in \auth and \loya, as shown by crossed edges (Fig. \ref{fig:case-study-group}d). With such observations, she now wondered whether she should focus on the two more distinguishing dimensions to determine if the two political camps had talked about the event with distinct \vale and \domi tones. To find it out, she used the \depscope to check the two dimensions separately. The partial dependence plot (PDP) in the \depscope indicated that, for \vale, a lower value tends to be associated with \red lines (Fig. \ref{fig:case-study-group}a), whereas a higher value may be mixed. Counter to her expectation, the \depscope suggested that this single dimension would not be sufficient to distinguish the two camps, as the \red appeared to have diverse values and even extreme values on both positive and negative side of \vale. On the other hand, the \domi dimension shown in \depscope was more consistent, where the \blue appeared to be associated with a higher value of \domi (Fig. \ref{fig:case-study-group}b). She became interested in telling a story about this collective tendency she observed from the \blue camp. ``{\it How can I tell the story?}'' The \languagescope allows her to track the texts in tweets with specific sociolinguistic tones. Using \languagescope, she found that the word ``vote'' was recurrently shown in the tweets with a higher \domi from the \blue camp (Fig.~\ref{fig:case-study-group}e) -- e.g., \tweet{stay strong we must have a vote against gun}. Such evidence allows her to come out with a story about how liberal-leaning users made a call to action in response to this mass shooting incident.
 
\paragraph{\bf Inspect group variation (T3)}
She now wondered if her story applied to all users from the \blue camp. The \variscope mode in the \grouptrend allows her to see the trends of subgroups across different sociolinguistic dimensions (Fig. \ref{fig:case-study-subgroup}). Using \variscope, she can see how subgroups (each as a rectangle bar) within the two camps may possess a higher or lower value in a particular dimension (indicated by the vertical position of the bar), and how each subgroup may have a more or less diverse pattern (indicated by the bar height) and a varying group size (indicated by the bar width). For example, she found \blue subgroups (1, 2) to have shorter bars in the \care and \fair dimensions, while the \blue subgroup 3 had a taller bar, indicating the latter subgroup had expressed different and varying tones from the rest of the \blue camp in terms of the two dimensions. Such subgroup differences prompted her to look for the subset of users who possess very similar characteristics. For the \blue camp, she found subgroup 1 seemed to be very consistent with bar positions far from other \red subgroups. For the \red camp, she found the largest \red subgroup 10 appeared to be a coherent set, located at the lower end of many features from all other \blue subgroups. 

This scenario demonstrates the major features of \name. More features will be covered in the later sections.

\begin{figure}[!t]
    \vspace{-1.5em}
    \setlength\tabcolsep{2pt} % default value: 6pt
    \begin{tabular}{c}\\
    \includegraphics[width=0.95\linewidth]{figures/case-study-subgroup.pdf}
    \end{tabular}  
    \vspace{-1em}
    \caption{\label{fig:case-study-subgroup}
    \textbf{Use case scenario (T3)}: Exploring the variance of subgroup differences using \variscope in \grouptrend.}
    \vspace{-1.5em}
\end{figure}