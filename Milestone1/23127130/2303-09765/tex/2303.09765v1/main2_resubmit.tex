\documentclass[%
 reprint,
%superscriptaddress,
%groupedaddress,
%unsortedaddress,
%runinaddress,
%frontmatterverbose, 
% preprint,
% preprintnumbers,
%nofootinbib,
%nobibnotes,
%bibnotes,
 amsmath,amssymb,
 aps,
%rmp,
%prstab,
%prstper,
%floatfix,
]{revtex4-2}


%%%%%%%%%% usepackage %%%%%%%%%%
\usepackage{graphicx}% Include figure files
\usepackage{dcolumn}% Align table columns on decimal point
\usepackage{bm}% bold math
\usepackage[colorlinks=true,urlcolor=blue,citecolor=blue, linkcolor=blue]{hyperref}
\usepackage{siunitx} % ishiyama
\usepackage{braket}
\usepackage{physics}
\usepackage{comment}
\usepackage{lineno}

%\usepackage{hyperref}% add hypertext capabilities
%\usepackage[mathlines]{lineno}% Enable numbering of text and display math
%\linenumbers\relax % Commence numbering lines

%\usepackage[showframe,%Uncomment any one of the following lines to test 
%%scale=0.7, marginratio={1:1, 2:3}, ignoreall,% default settings
%%text={7in,10in},centering,
%%margin=1.5in,
%%total={6.5in,8.75in}, top=1.2in, left=0.9in, includefoot,
%%height=10in,a5paper,hmargin={3cm,0.8in},
%]{geometry}

% \subsection{\label{sec:level2}}
% \subsubsection{Wide text (A level-3 head)}
% \subsection{\label{sec:citeref}Citations and References}
% \subsubsection{Citations}
% \paragraph{Syntax}


%%%%%%%%%% substitute %%%%%%%%%%
% |g\rangle --> \ket{g}
% |e\rangle --> \ket{e}


%%%%%%%%%% Document %%%%%%%%%%
\begin{document}
\renewcommand{\thefigure}{S\arabic{figure}}
\renewcommand{\thetable}{S\Roman{table}}
\renewcommand{\thesection}{S\arabic{section}}
\renewcommand{\theequation}{S\arabic{equation}}

% \linenumbers

\preprint{APS/123-QED}

\title{Supplemental Material for\\``Observation of an Inner-Shell Orbital Clock Transition in Neutral Ytterbium Atoms''}

\maketitle
\section{EXPERIMENTAL DETAIL}\label{sec:experiment}
Figure~\ref{FigS1} shows the experimental configuration of our measurements. 
We note that the lifetime measurement is done for atoms loaded into a 3D optical lattice, although not shown in Fig.~\ref{FigS1}. 

\begin{figure}[t]
\includegraphics[width = 0.9\linewidth]{FigS1_experimental_setup_ver2.pdf}
\caption{\label{FigS1}
Schematic of our experimental setup. 
Our experiments are performed with Yb atoms in a crossed FORT consisting of 1064-nm and 1070-nm laser beams. 
The linearly polarized excitation light, shown as the blue arrow, is irradiated from the $y$-axis. 
For polarizability measurements, a linearly polarized Ti-sapphire laser, shown as the red arrow, shines along the $x$-axis. 
Note that the quantization axis defined by the magnetic field is changed for each measurement, as well as the polarization angles of the Ti:sapphire and excitation lasers.
}
\end{figure}

For all isotopes, Yb atoms are decelerated by a Zeeman slower, collected in a magneto-optical trap using the $^1S_0$ $\leftrightarrow$ $^3P_1$ transition, and then evaporatively cooled in a crossed far off-resonance trap (FORT).
Since $^{171} \mbox{Yb}$($^{176} \mbox{Yb}$) has a small $s$-wave scattering length and thus evaporation cooling is inefficient, it is sympathetically cooled with $^{173}$Yb($^{174} \mbox{Yb}$)~\cite{Taie2010, Ono2022}.
The number of atoms and the temperature are measured by an absorption imaging method using the $^1S_0$ $\leftrightarrow$ $^1P_1$ transition.

The laser for the excitation at a wavelength of 431~nm is obtained by second harmonic generation of an interference-filter-stabilized external-cavity diode laser~\cite{Baillard2006} at a wavelength of 862~nm.
This 862-nm laser beam is offset locked to an optical cavity made of ultra-low expansion glass with the Pound-Drever-Hall method~\cite{Drever1983}.
%To generate the error signal, a 16 MHz electrical signal is added to a fiber electro-optic modulator.
%We mix this 16 MHz electrical signal with another signal in a power combiner to frequency stabilize it with an offset to the resonance %frequency of the optical resonator.
%During spectroscopy, we swept the frequency of the latter electrical signal.
%Note that the RF signal is generated by a signal generator from a 10 MHz GPS-synchronized reference.
Note that the signal generator, driving a fiber electro-optic modulator for the offset lock, is referenced by the 10-MHz oscillator disciplined by the global positioning system.    
The output power of the 862-nm laser is about 30~mW, and then amplified to about 1~W by a tapered amplifier.
Using a periodically poled lithium niobate waveguide, the wavelength is converted to 431~nm and about 5~mW is obtained for irradiation of the atoms.
The peak intensity at the atomic position is up to about 0.5~kW/cm$^2$.

%\paragraph{Zeeman and hyperfine spectra.}





\section{ISOTOPE SHIFT}\label{sec:isotope}
%%%%%%%%%%%% Isotope shifts %%%%%%%%%%%%
%\indent
%{\it Isotope shifts.{\bf \textemdash}}
%Next, we report on isotope shift measurements between Bose isotopes.
As mentioned in the introduction, the isotope shift is an important physical quantity directly related to the search for new physics.
Here, we have measured the isotope shift of the transition to the $m_J=0$ state using five isotopes $^{168} \mbox{Yb}$, $^{170} \mbox{Yb}$, $^{172} \mbox{Yb}$, $^{174} \mbox{Yb}$, and $^{176} \mbox{Yb}$ with $^{174} \mbox{Yb}$ as a reference.
In order to minimize the systematic uncertainty of the isotope shift measurements, we use the following measurement technique:
after performing spectroscopy of one isotope, we re-lock the excitation laser frequency and perform the spectroscopy of the other isotope.
We repeat this sequence by multiple times.
By repeating the spectroscopy alternately, the effect of excitation light frequency drift during the measurement is minimized.
Note that the interleaved spectroscopy used in the polarizability and $g$-factor measurements is not applicable because the isotope shift is as large as several GHz, and it is difficult to change the laser frequency with the frequency-lock kept on.
When re-locking the excitation laser frequency, care should be taken to keep the offset voltage of the error signal and the magnitude of the transmitted signal of the optical resonator constant.
The FORT light intensity is also kept constant during the excitation to make the light shift due to the FORT light common in two isotopes.
Since the resonance shift is also atom-temperature dependent, the temperatures of the two isotopes are adjusted to be as close as possible.
The effect of residual temperature deviations is estimated to be smaller than 10~kHz.
%taken into account later as a systematic uncertainty.
The interaction shift due to collisions between atoms depends on the scattering length between atoms, which is isotope-dependent.
As is explained in the main text, no evidence of the collisional shift is found in our experimental condition.
%Theoretical calculations for the scattering length are not available and must be estimated experimentally.
%Other effects are not isotope dependent and are expected to cancel out in isotope shift measurements.
%Results are shown in Table 1.





\section{$g$-FACTOR}\label{sec:g-factor}
%%%%%%%%%%%% g-factor %%%%%%%%%%%%
%\indent
%{\it $g$-factor for bosonic isotopes.{\bf \textemdash}}
The measurement of the $g$-factor of the excited state $4f^{13}5d6s^2 \: (J = 2)$ is performed using the bosonic isotope of $^{174} \mbox{Yb}$.
%The g-factor is important because it is a quantity that characterizes the interaction with an external magnetic field, and its experimental measurement is a test for atomic many-body calculations.
%We have obtained the $g$-factor by measuring the Zeeman shift of the magnetic sublevel $m_{J'} = +1, +2$ of $^{174} \mbox{Yb}$.
The measurement sequence is as follows.
We apply a magnetic field along the $z$-axis to $^{174} \mbox{Yb}$ in a crossed FORT and excite the atoms to $m_{J}$ = +1 or +2.
% Interleaved measurements are made at four field magnitudes 2.91, 5.82, 8.73, and 11.6 G to measure the Zeeman shift differences at field differences of 2.91, 5.82, and 8.73 G.
% These fields are calibrated using $^1S_0 \: \rightarrow \: ^3P_1$ transitions.
Interleaved spectroscopy is performed to measure the Zeeman shift differences between four field strengths of 0.291(1), 0.582(2), 0.873(3), and 1.164(4)~mT, calibrated by spectroscopy with the $^1S_0$ $\leftrightarrow$ $^3P_1$ transition, in which the $g$-factor of the $^3P_1$ state $g_J(^3\text{P}_1)$ = 1.49282(2) is obtained from the weighted average of the two measured values in Refs.~\cite{Budick1967, Baumann1968}.
A linear fit to the magnetic field difference and the Zeeman shift difference, with the slope divided by the Bohr magneton $\mu_B$ and $|m_{J}|$, gives the excited state $g$-factor.
Note that the quadratic Zeeman shift is calculated to be -0.43~Hz/(mT$^2$) for $m_J=0$, and -0.34~Hz/(mT$^2$) for $\abs{m_J}=1,2$~\cite{Dzuba2018}, negligibly small compared to the uncertainty of our measurement.
Figure~\ref{FigS2} shows a typical result of such measurements for $m_J=+2$.
%The solid line in the upper panel shows the linear fit to the data, and the lower panel shows the deviation from the linear fit.
%The error bars on the vertical axis are the standard errors of the measurement.
We perform the $m_{J} = +1$ and $+2$ measurements twice, and adopt the weighted average of the four measurements.
In this way, we obtain the $g$-factor as $1.463(2)$.
The uncertainty is determined from the propagation of the uncertainty of the magnetic field calibration and the fitting error in Fig.~\ref{FigS2}.
Note that the former, with a magnitude of about $0.3\%$, is dominant, and the latter hardly comes into play.
%We also performed theoretical calculations using the DIRAC program to obtain measured and consistent values.

\begin{figure}
\includegraphics[width = 0.9\linewidth]{FigS3_g_factor.pdf}
\caption{\label{FigS2}
Measurement of the $g$-factor of the $4f^{13}5d6s^2 (J = 2)$ state.
In the upper panel, the differential Zeeman shifts of $^{174}$Yb ($m_{J}=+2$) are plotted as a function of the differential magnetic field, 
The solid green line is the linear fit to the data, the slope of which corresponds to the $g$-factor of the $4f^{13}5d6s^2 (J = 2)$ state.
The lower panel shows residuals from the linear fit.
The error bars on the vertical axis are the standard error.
%Note that the measurement accuracy is limited by the magnetic field uncertainty.
}
\end{figure}

\begin{figure*}
\includegraphics[width = 0.9\linewidth]{FigS4_scalar_tensor_polarizabilities.pdf}
\caption{\label{FigS3}
Differential scalar polarizability $\Delta\alpha^S$ and tensor polarizability $\alpha^T$ around the transitions: (a) $4f^{13}5d6s^2 \: (J = 2)$ $\leftrightarrow$  $4f^{13} 6s^2 6p_{3/2}\: (J=3)$ at $792.5$~nm, and (b) $4f^{13}5d6s^2 \: (J = 2)$ $\leftrightarrow$ $4f^{13} 6s^2 6p_{3/2} \:(J=2)$ at $833.7$~nm.
The solid curves are the scalar and tensor differential polarizabilities
expressed as the linear combinations of the curves in Figs.~3(a, b) of the main text using Eq.~(1).
For reference, the linear combination of the data points in Figs.~3(a, b) are shown.
We plot the average of the wavelengths of the two points for pairs which have slightly different wavelengths within $0.12$~nm.
}
\end{figure*}


\section{HYPERFINE CONSTANTS}\label{sec:hyperfine}
Here, we describe the details of the analysis to evaluate the hyperfine constants of the $4f^{13}5d6s^2 \: (J = 2)$ state for $^{171} \mbox{Yb}$ and  $^{173} \mbox{Yb}$. The hyperfine structure arises from the interaction between an electron with a total angular momentum $J$ and a nucleus with a spin $I$, and is expressed as a multipole series~\cite{Charles1955}. The hyperfine-structure Hamiltonian $\hat{H}_\text{HFS}$, truncated at the third term, is given by
\begin{widetext}
\begin{align}
    \hat{H}_\text{HFS} &= \hat{H}_\text{dip} + \hat{H}_\text{quad} + \hat{H}_\text{oct},
    \label{HFS}
    \\
    \hat{H}_\text{dip}  &= A_\text{HFS}\hat{\vb{I}}\cdot\hat{\vb{J}},
    \label{A}
    \\
    \hat{H}_\text{quad}  &= B_\text{HFS}\frac{3(\hat{\vb{I}}\cdot\hat{\vb{J}})^2+\frac{3}{2}(\hat{\vb{I}}\cdot\hat{\vb{J}})-I(I+1)J(J+1)}{2I(2I-1)J(2J-1)},
    \label{B}
    \\
    \hat{H}_\text{oct}  &= C_\text{HFS}\frac{10(\hat{\vb{I}}\cdot\hat{\vb{J}})^3+20(\hat{\vb{I}}\cdot\hat{\vb{J}})^2+2(\hat{\vb{I}}\cdot\hat{\vb{J}})[I(I+1)+J(J+1)+3-3I(I+1)J(J+1)]-5I(I+1)J(J+1)}{I(I-1)(2I-1)J(J-1)(2J-1)},
    \label{C}
\end{align}
\end{widetext}
where $\hbar\hat{\vb{J}}$ ($\hbar\hat{\vb{I}}$) is an operator for an electronic angular momentum (a nuclear spin). 
Eqs.~\eqref{A}, \eqref{B}, and \eqref{C} are associated with a magnetic dipole moment, an electric quadrupole moment, and a magnetic octupole moment, respectively. We consider an atom interacting with an external magnetic field $B$ along the $z$-axis, and the Zeeman Hamiltonian $\hat{H}_\text{Z}$ is given by
\begin{equation}
    \hat{H}_\text{Z} = \mu_\text{B}(g_J\hat{J}_z+g_I\hat{I}_z)B,
    \label{Zeeman}
\end{equation}
where $\hat{J}_z$ ($\hat{I}_z$) is the $z$-component of $\hat{\vb{J}}$ ($\hat{\vb{I}}$). As shown in Sec.~\ref{sec:g-factor}, the $g$-factor of the $4f^{13}5d6s^2 (J = 2)$ state is measured to be $g_J=$ 1.463(2), and $g$-factor of a nuclear spin is represented as $g_I=\mu_I/(\mu_BI)$, where a nuclear magnetic moment $\mu_I$ in units of the nuclear magneton $\mu_\text{N}$ is given by 0.49367(1) for $^{171}$Yb, and -0.67989(3) for $^{173}$Yb~\cite{Olschewski1972ZP, Stone2005ADNDT}. 
Here we do not consider the correction to $g_\text{I}$ due to mixing with other states caused by the hyperfine interaction~\cite{Porsev2004, Boyd2007}, since the correction is not known and, in the first place, the contribution of $g_I$ to the determination of the hyperfine constants is negligibly small, considering the uncertainties of the obtained hyperfine constants.
To determine the hyperfine structure constants $A_\text{HFS}$, $B_\text{HFS}$, and $C_\text{HFS}$, the eigenenergies of $\hat{H}_\text{HFS} + \hat{H}_\text{Z}$ are used for the fits to the data shown in Fig.~2 in the main text. As a result, we obtain $A_\text{HFS}(^{171} \mbox{Yb})=1123.3(3)$~MHz, and $A_\text{HFS}(^{173}\mbox{Yb})=-309.46(4)$~MHz, and $B_\text{HFS}(^{173} \mbox{Yb})= -1700.6(9)$~MHz, with the uncertainties due to the fitting error. 
%Note that the systematic effects such as the collision shift are not taken into account. It is also noted introducing $C_\text{HFS}$ for $^{173} \mbox{Yb}$ does not improve the fitting. 
% Assuming that the correction factor for $g_I$ due to the hyperfine coupling is common as in the $^3P_0$ state~\cite{Porsev2004}, the magnetic dipole constants between two distinct isotopes have the following relation: $A_\text{HFS}(^{173}\mbox{Yb})=A_\text{HFS}(^{171} \mbox{Yb})\times g_I(^{173}\text{Yb})/g_I(^{171}\text{Yb})$.
% The right-hand side of the above equation is obtained as -309.32(1)~MHz. The slight discrepancy with the measured $A_\text{HFS}(^{173}\mbox{Yb})$ would be ascribed to the charge distribution difference between the isotopes, called the hyperfine anomaly~\cite{Persson2013}.

\section{POLARIZABILITY}\label{sec:scalartensor}
\subsection{Calculation of magic wavelength}
Here we estimate magic wavelengths for the $^1S_0\leftrightarrow 4f^{13}5d6s^2\;(J=2)$ transition. An atom interacting with a laser field with intensity $I_0$ has an energy shift, called the AC Stark shift $\Delta E = -\alpha I_0/(2\epsilon_0c)$, where $\epsilon_0$ is the permittivity in vacuum. The polarizability for an atom in a fine-structure level $J$ is given by Eq.~(1) in the main text when the laser field is linearly polarized.
The scalar and tensor possibilities are expressed as follows~\cite{LeKien2013}:
\begin{equation}
\begin{aligned}
\alpha_{J}^{S} &= \frac{1}{\sqrt{3(2J+1)}}\alpha_J^{(0)},\\
\alpha_{J}^{T} &= -\sqrt{\frac{2J(2J-1)}{3(J+1)(2J+1)(2J+3)}}\alpha_J^{(2)},
\end{aligned}
\end{equation}
where
\begin{equation}
\begin{aligned}
\alpha^{(K)}_J = &(-1)^{K+J+1}\sqrt{2K+1}
\sum_{i}
\begin{Bmatrix}
1 & K & 1 \\
J & J_i & J
\end{Bmatrix}
\nonumber\\
&\times\frac{|\langle J_i\|\hat{D}\|J\rangle|^2}{\hbar}
\left(\frac{1}{\omega_i-\omega}+\frac{(-1)^K}{\omega_i+\omega}\right).
\end{aligned}
\end{equation}
Here $\{\cdot\}$ represents the Wigner 6-$j$ symbol, and $\langle J_i\|\hat{D}\|J\rangle$ is the reduced transition dipole matrix element.
To calculate the polarizability of the excited state, we consider two E1-allowed optical transitions of $4f^{13}5d6s^2 \: (J = 2)$ $\leftrightarrow$  $4f^{13} 6s^2 6p_{3/2} \:(J=3)$ at a wavelength of 792.5~nm and $4f^{13}5d6s^2 \: (J = 2)$ $\leftrightarrow$ $4f^{13} 6s^2 6p_{3/2}\; (J=2)$ at 832.7~nm, whose transition wavelength are within a tunable range of a Ti:sapphire laser. The corresponding transition dipole moments are presented in Sec.~\ref{sec:results}. We calculate magic wavelengths, where the polarizability of the excited state coincides with that of the ground state $^1\text{S}_0$~\cite{Tang2018}. As a result, we obtain two magic wavelengths for $m_J = 0$ and $\theta=90^\circ$: $\lambda_\text{magic}$ = 795.5~nm and 833.7~nm. 
Similarly, the polarizability of an atom in a hyperfine-structure level $F$ is given by
\begin{equation}
\alpha = \alpha_F^S + \frac{3\cos^2\theta-1}{2}\frac{3m_F^2-F(F+1)}{F(2F-1)}\alpha_F^T,
\label{polarizability: hyperfine}
\end{equation}
where
\begin{align}
\alpha_F^S=&\alpha_J^S,\\
\alpha_F^T = &-(-1)^{J+I+F}\sqrt{\frac{2F(2F-1)(2F+1)}{3(F+1)(2F+3)}}
\nonumber\\
&\times
\begin{Bmatrix}
F & 2 & F \\
J & I & J
\end{Bmatrix}
\alpha_J^{(2)}.
\end{align}

As shown in Fig.~3 in the main text, we measured the total differential polarizability $\Delta\alpha$, and one can retrieve the differential scalar polarizability $\Delta\alpha^S$ and tensor polarizability $\alpha^T$ from the measured $\Delta\alpha$ using the following simultaneous equations:
\begin{equation}
    \begin{cases}
    \Delta\alpha(m_J=0,\theta=0)=\Delta\alpha^S-\alpha^T, \\
     \Delta\alpha(m_J=0,\theta=90^\circ)=\Delta\alpha^S+\frac{1}{2}\alpha^T,\\
     \Delta\alpha(m_J=2, \theta=90^\circ)=\Delta\alpha^S-\frac{1}{2}\alpha^T. 
    \end{cases}
    \label{simultaneous equations}
\end{equation}
Figure \ref{FigS3} shows the retrieved $\Delta\alpha^S$ and $\alpha^T$ from the measured $\Delta\alpha$ using Eq.~\ref{simultaneous equations}.

%The measurement technique is as follows.
\subsection{Experimental procedure}
In our measurement, we apply a high magnetic field of 14.6~mT to $^{174} \mbox{Yb}$ in a crossed FORT and irradiate $\pi$-polarized excitation light at 431~nm to induce a magnetic-field-induced E1 transition~\cite{Taichenachev2006} to the $m_J = 0$ state in the $4f^{13}5d6s^2 \: (J = 2)$.
%A Ti:sapphire laser beam is irradiated along the x-axis direction.
%For polarizability measurements, we additionally apply a linearly polarized Ti:sapphire laser along the x-axis.
The wavelength is tunable between 750~nm and 860~nm, with a maximum intensity of 8~kW/cm$^2$ at the atom position.
The laser power is stabilized by feedback control using an acoustic-optic modulator, and the beam pointing is monitored to evaluate the uncertainty of the laser intensity at the atom position.
%The wavelength is $\lambda$ and the angle between the quantization axis $\vec{e_z}$ and the polarization is $\theta$.
By interleaved measurements with three light intensities $0, I_0, 2I_0$ of the Ti:sapphire laser, we determine the differential light shift $\Delta \nu$ with respect to the light intensity difference.
This interleaved measurement minimizes systematic effects such as the drifts of the excitation light frequency and the Ti:sapphire light intensity.
From a linear fit of the differential light shift $\Delta \nu$ with a light intensity difference, we obtain the differential polarizability $\Delta \alpha$ as its slope.
%Here, we perform two measurements with different $\theta$ or $m_{J'}$ at each wavelength to obtain the scalar and tensor polarizabiliies $\Delta \alpha ^S (\lambda)$ and $\alpha ^T (\lambda)$.
%As can be seen from Eq[], by measuring the polarizability difference $\Delta \alpha (\lambda)$ under two conditions with different $\theta$ or $m_{J'}$, two parameters $\Delta \alpha ^S (\lambda)$, $\Delta \alpha ^T (\lambda)$ can be obtained.
% Figures~\ref{FigS3}(a) and (b) show the differential polarizabilities of scalar and tensor parts $\Delta \alpha ^S $ and $\alpha ^T$.
%The solid lines are $\Delta \alpha ^S$ and $\alpha ^T$ obtained as a linear combination of the fitting curves in Figs.~\ref{Fig3}(a) and (b) using Eq.~(1).
%For reference, the linear combination of the data points in Fig. 3 (a) and (b) is also shown.

\subsection{Possible trap geometries for optical atomic clock}
Different from the case of the $^1S_0$ $\leftrightarrow$ $^3P_0$ transition, the $^1S_0$ $\leftrightarrow$ $4f^{13}5d6s^2 \: (J = 2)$ transition is influenced by the tensor polarizability, which depends on $\theta$, and so fluctuations in the magnetic field or the polarization of the laser light will cause fluctuations of the light shift.
Since $d(\cos^2\theta)/d\theta=0$ at $\theta =0$ or $90^\circ$, $\theta =0$ or $90^\circ$ is preferable.
Here, we estimate how much this contributes to the future clock operation using the following trap geometries.
\begin{enumerate}
\item 
Let us consider a 3D optical lattice clock with a single occupancy~\cite{Akatsuka2008} using $^{174} \mbox{Yb}$ at the condition of $m_J=0$ and $\theta=90^{\circ}$, and the operation at a magic wavelength $797.2(4)$~nm with a depth of $25E_\text{R}$ for each lattice axis.
\item
Another case of interest is a 1D optical lattice clock~\cite{Takamoto2006} operated with excitation to the stretched state $\ket{F=5/2, m_F = \pm5/2}$ of spin-polarized $^{171}$Yb atoms, which is useful to reject the higher-order Zeeman effect induced by the hyperfine structure.
From the scalar and tensor polarizabilities, shown in Figs.~\ref{FigS3}(a) and Eq.~\ref{polarizability: hyperfine}, the magic wavelength for $^{171}$Yb is calculated as 797.3~nm at the condition of $\theta = 90^\circ$.
\item
In addition, single-atom arrays trapped by optical tweezers are a promising platform for an optical atomic clock~\cite{Young2020}. We consider tweezer-trapped $^{174}$Yb atoms at the condition of $m_J=0$ and $\theta= 90^{\circ}$, and the operation at a magic wavelength $797.2(4)$~nm with a depth of $25E_\text{R}$.
\end{enumerate}
For these three cases shown above, the fluctuation $\delta\theta$ of $\theta$ causing a fractional uncertainty at the $10^{-18}$ level is estimated to be about 1~mrad, which is manageable in the current experimental technique.


\begin{comment}
Here, we estimate how much this contributes to the future clock operation.
Let us consider a 3D optical lattice clock~\cite{Akatsuka2008} using $^{174} \mbox{Yb}$ at the condition of $m_J=0$ and $\theta=90^{\circ}$, and the operation at a magic wavelength $797.2(4)$~nm with a depth of $25E_\text{R}$ for each lattice axis.
In this case, the fluctuation $\delta \theta$ of $\theta$ causing a fractional uncertainty of $1\times10^{-18}$ is estimated to be 0.5~mrad, which is manageable in the current experimental technique.
Another case of interest is a 1D optical lattice clock~\cite{Takamoto2006} using $^{171} \mbox{Yb}$ operated with the excitation to the $\ket{F=5/2, m_F = \pm5/2}$ states.
From the scalar and tensor polarizabilities, shown in Figs.~\ref{FigS3}(a) and (b), the magic wavelength for $^{171}$Yb is obtained as 797.3~nm at the condition of $\theta = 90^\circ$. In this case, we estimates the fluctuation $\delta\theta = 0.9$~mrad results in a fractional uncertainty of $1\times10^{-18}$ at a depth of 25$E_\text{R}$.
In addition, optical tweezers offer a platform for an optical atomic clock~\cite{Young2020}, as well as an optical lattice. Considering tweezer-trapped $^{174}$Yb atoms at the condition of $m_J=0$ and $\theta=90^{\circ}$, and the operation at a magic wavelength $797.2(4)$~nm with a depth of $25E_\text{R}$, the fluctuation $\delta\theta = 0.8$~mrad results in a fractional uncertainty of $1\times10^{-18}$.
\end{comment}
 \begin{figure}[t]
\includegraphics[width = 0.9\linewidth]{FigS2_230118_spectroscopy_pulse_sequence.pdf}
\caption{\label{FigS4}
(a) Excitation spectrum of $^{174}$Yb ($m_J=0$) in a 3D optical lattice.
The horizontal axis shows the detuning of the clock laser, and the vertical axis shows the normalized number of atoms after the irradiation by the excitation laser.
%We observe two peaks consisting of the carrier and the blue sideband.
The solid red curve is the fit with a double-peak Lorentzian function. The FWHM of the carrier spectrum is 12~kHz.
%The FWHM of the carrier spectrum is $12$~kHz.
% The optical lattice depths of the $x$, $y$, and $z$ axes are 20, 50, and 30 $E_r$, respectively.
(b) Pulse sequence of the lifetime measurement.
The blue and purple pulses correspond to the $^1S_0\;\leftrightarrow\;4f^{13}5d6s^2\;(J=2)$, and $^1S_0\:\leftrightarrow\:^1P_1$ transitions, respectively.
First, the excitation pulse of 3 ms is applied. 
Then, the remaining $^1S_0$ state is blown up with 399-nm light resonant to the $^1S_0$ $\rightarrow$ $^1P_1$ transition.
Then, we wait for variable hold time.
At the end of hold time, again we blow off the atoms that have decayed from the excited state to $^1S_0$.
Finally, we count the atoms remaining in the excited state by absorption imaging using the $^1S_0$ $\leftrightarrow$ $^1P_1$ transition, by the returning the excited atoms to the ground state with the de-excitation pulse of 20~ms light pulse.
}
\end{figure}
%
%%%%%%%%%  lifetime  %%%%%%%%%%%%%
\section{Lifetime}\label{sec:scalartensor}
The lifetime of atoms in the $4f^{13}5d6s^2 \: (J = 2)$ state is measured using a 3D optical lattice. Figure~\ref{FigS4}(a) shows the spectrum with resolved carrier and blue-sideband components.
%It can be seen that a double peak was obtained.
%The low frequency side is considered to be the carrier and the high-frequency side is the blue sideband.
%The solid red line shows the fitting by the double-peak Lorenz function. 
The full width at half maximum (FWHM) of the carrier is about 12~kHz, narrower than the 30~kHz linewidth for atoms in a crossed FORT and
% and the frequency difference between the carrier and blue sideband is about 18 kHz.
%Compared to the FWHM of 30 kHz in the crossed FORT, the FWHM has been narrowed to less than half.
%We believe that the current linewidth
is limited by the linewidth of the excitation laser and the effect of the differential light shift at the non-magic wavelength of the 759-nm optical lattice light.
%The magnetic field is $\vec{e_z}$ oriented, and all three axes have linear polarization perpendicular to the magnetic field.
%Note that 797.2 nm is the magic wavelength for $^{174} \mbox{Yb} (m_{J'} = 0, \theta=90^{\circ})$.
%In order to simultaneously evaluate photon scattering by the lattice light at the magic wavelength along with lifetime measurements of the excited states, we formed an optical lattice at 797.2 nm in only one axis.
%This is because our Ti:Sa laser is not powerful enough to form a 3D optical lattice.

%
Figure~\ref{FigS4}(b) shows the pulse sequence of the lifetime measurement.
Spin-polarized $^{173} \mbox{Yb} (m_F=+5/2)$ atoms are loaded into a 3D optical lattice and then excited to the $\ket{F' = 5/2, m_{F'} = +5/2}$ state by 3 ms excitation light, where $F'$ is the total angular momentum in the excited state.
The excitation light is linearly polarized along the $z$-axis, enabling the hyperfine-induced E1 transition with $\Delta m_{F} = 0$.
The excitation and de-excitation sequence depicted in Fig.~\ref{FigS4}(b) enables the measurement of the lifetime.
We also measure the lifetime of the ground state $^1S_0$.
%The measurement procedure is simple;
After loading the lattice, we wait for the hold time, and finally observe the number of atoms with absorption imaging using the $^1S_0$ $\leftrightarrow$ $^1P_1$ transition. 
The lattice depths are 28$E_\text{R}$ for the 2D optical lattice at 759.4~nm, and 25$E_\text{R}$ for the 1D optical lattice at 797.2~nm. 
The lattice depth is calibrated using a pulsed-lattice method applied to a $^{174}$Yb Bose-Einstein condensate.

We estimate the photon-scattering rate for the atoms in the $4f^{13}5d6s^2 \: (J = 2)$ state due to the 1D magic-wavelength optical lattice laser beam at 797.2~nm, which is close to the resonance wavelength of the $4f^{13}5d6s^2 \: (J = 2)$ $\leftrightarrow$  $4f^{13} 6s^2 6p_{3/2}\: (J'=3)$ transition at 792.5~nm. The scattering rate $\gamma_\text{sc}$ is given by
\begin{equation}
    \gamma_\text{sc}=\frac{3\pi c^2\Gamma^2}{2\hbar\omega_0^3}\frac{\beta^2}{\Delta^2}I_\text{lat},
\end{equation}
where $\Delta = \omega-\omega_0$ is the detuning of the lattice beam from the resonance, and $I_\text{lat}$ is the intensity of the lattice laser. The transition strength factor, or the square of the Clebsch-Gordan coefficient $\beta^2 = \abs{\bra{J'=3, m_{J'}=\pm1}\ket{J=2, m_J = 0;1,\pm1}}^2 = 0.4$ is considered, since the polarization of the lattice light is perpendicular to the quantization axis. The spontaneous decay rate $\Gamma$ is given by
\begin{equation}
    \Gamma = \frac{2J+1}{2J'+1}\frac{\omega_0^3}{3\pi\epsilon_0\hbar c^3}|\langle J'\|\hat{D}\|J\rangle|^2.
\end{equation}
As a result, the rate of the photon scattering is obtained as $\gamma_\text{sc}=0.15$~s$^{-1}$. In the case of another magic wavelength $\lambda_\text{magic}=834.2(1)$~nm, the photon-scattering rate at the same condition is calculated as $\gamma_\text{sc}=0.066$~s$^{-1}$.


\section{Electronic structure calculations}\label{sec:results}
All of the electronic-structure calculations were carried out using the DIRAC program package~\cite{saue2020dirac,DIRAC22} (git hash dbeb14f, d86857b, 721ac5b). 
The Dirac-Coulomb-Gaunt Hamiltonian was employed in all calculations. 
The dyall v3z basis set~\cite{Dyall_Gomes_2010} was employed in the uncontracted form. 
We employed the KR-CI module~\cite{Fleig2003JCP,Knecht2010JCP,Knecht_thesis,Denis2015NJP} for the relativistic configuration-interaction calculation.
The electrons in $5s$, $5p$, $4f$, and $6s$ orbitals were correlated in the calculations of the $g$-factor. The electrons in $5p$, $4f$, and $6s$ orbitals were correlated in the calculations to obtain the polarizabilities, 
that is, the E1 transition dipole moments (TDM) from the $4f^{13}6s^25d \: (J=2)$ state to the $4f^{13}6s^26p_{3/2} \: (J=3)$ and $4f^{13}6s^26p_{3/2} \: (J=2)$ states. The reduced matrix elements of the TDM for $4f^{13}6s^26p_{3/2} \: (J=3)$ and $4f^{13}6s^26p_{3/2} \: (J=2)$ states are 0.8587~a.u. and 0.3203~a.u., respectively, where a.u. denotes atomic unit.

% \begin{table}
% \caption{\label{tab:table2}%
% Hyperfine constants.}
% \begin{ruledtabular}
% \begin{tabular}{ccr}
% Isotope & ABC & Hyperfine constants $(\mathrm{MHz})$ \\
% \hline
% 171 & A & 1123.3(5)\\
% 173 & A & -309.46(6)\\
% 173 & B &  -1700.6(6)\\
% 173 & C &  -0.00(4)\\
% \end{tabular}
% \end{ruledtabular}
% \end{table}



%%%%%%%%%% bibtex %%%%%%%%%%
% \bibliography{main_resub}% Produces the bibliography via BibTeX.

%apsrev4-2.bst 2019-01-14 (MD) hand-edited version of apsrev4-1.bst
%Control: key (0)
%Control: author (8) initials jnrlst
%Control: editor formatted (1) identically to author
%Control: production of article title (0) allowed
%Control: page (0) single
%Control: year (1) truncated
%Control: production of eprint (0) enabled
\providecommand{\noopsort}[1]{}\providecommand{\singleletter}[1]{#1}%
\begin{thebibliography}{25}%
\makeatletter
\providecommand \@ifxundefined [1]{%
 \@ifx{#1\undefined}
}%
\providecommand \@ifnum [1]{%
 \ifnum #1\expandafter \@firstoftwo
 \else \expandafter \@secondoftwo
 \fi
}%
\providecommand \@ifx [1]{%
 \ifx #1\expandafter \@firstoftwo
 \else \expandafter \@secondoftwo
 \fi
}%
\providecommand \natexlab [1]{#1}%
\providecommand \enquote  [1]{``#1''}%
\providecommand \bibnamefont  [1]{#1}%
\providecommand \bibfnamefont [1]{#1}%
\providecommand \citenamefont [1]{#1}%
\providecommand \href@noop [0]{\@secondoftwo}%
\providecommand \href [0]{\begingroup \@sanitize@url \@href}%
\providecommand \@href[1]{\@@startlink{#1}\@@href}%
\providecommand \@@href[1]{\endgroup#1\@@endlink}%
\providecommand \@sanitize@url [0]{\catcode `\\12\catcode `\$12\catcode
  `\&12\catcode `\#12\catcode `\^12\catcode `\_12\catcode `\%12\relax}%
\providecommand \@@startlink[1]{}%
\providecommand \@@endlink[0]{}%
\providecommand \url  [0]{\begingroup\@sanitize@url \@url }%
\providecommand \@url [1]{\endgroup\@href {#1}{\urlprefix }}%
\providecommand \urlprefix  [0]{URL }%
\providecommand \Eprint [0]{\href }%
\providecommand \doibase [0]{https://doi.org/}%
\providecommand \selectlanguage [0]{\@gobble}%
\providecommand \bibinfo  [0]{\@secondoftwo}%
\providecommand \bibfield  [0]{\@secondoftwo}%
\providecommand \translation [1]{[#1]}%
\providecommand \BibitemOpen [0]{}%
\providecommand \bibitemStop [0]{}%
\providecommand \bibitemNoStop [0]{.\EOS\space}%
\providecommand \EOS [0]{\spacefactor3000\relax}%
\providecommand \BibitemShut  [1]{\csname bibitem#1\endcsname}%
\let\auto@bib@innerbib\@empty
%</preamble>
\bibitem [{\citenamefont {Taie}\ \emph {et~al.}(2010)\citenamefont {Taie},
  \citenamefont {Takasu}, \citenamefont {Sugawa}, \citenamefont {Yamazaki},
  \citenamefont {Tsujimoto}, \citenamefont {Murakami},\ and\ \citenamefont
  {Takahashi}}]{Taie2010}%
  \BibitemOpen
  \bibfield  {author} {\bibinfo {author} {\bibfnamefont {S.}~\bibnamefont
  {Taie}}, \bibinfo {author} {\bibfnamefont {Y.}~\bibnamefont {Takasu}},
  \bibinfo {author} {\bibfnamefont {S.}~\bibnamefont {Sugawa}}, \bibinfo
  {author} {\bibfnamefont {R.}~\bibnamefont {Yamazaki}}, \bibinfo {author}
  {\bibfnamefont {T.}~\bibnamefont {Tsujimoto}}, \bibinfo {author}
  {\bibfnamefont {R.}~\bibnamefont {Murakami}},\ and\ \bibinfo {author}
  {\bibfnamefont {Y.}~\bibnamefont {Takahashi}},\ }\bibfield  {title} {\bibinfo
  {title} {{Realization of a
  $\mathrm{SU}(2)\ifmmode\times\else\texttimes\fi{}\mathrm{SU}(6)$ System of
  Fermions in a Cold Atomic Gas}},\ }\href
  {https://doi.org/10.1103/PhysRevLett.105.190401} {\bibfield  {journal}
  {\bibinfo  {journal} {Phys. Rev. Lett.}\ }\textbf {\bibinfo {volume} {105}},\
  \bibinfo {pages} {190401} (\bibinfo {year} {2010})}\BibitemShut {NoStop}%
\bibitem [{\citenamefont {Ono}\ \emph {et~al.}(2022)\citenamefont {Ono},
  \citenamefont {Saito}, \citenamefont {Ishiyama}, \citenamefont {Higomoto},
  \citenamefont {Takano}, \citenamefont {Takasu}, \citenamefont {Yamamoto},
  \citenamefont {Tanaka},\ and\ \citenamefont {Takahashi}}]{Ono2022}%
  \BibitemOpen
  \bibfield  {author} {\bibinfo {author} {\bibfnamefont {K.}~\bibnamefont
  {Ono}}, \bibinfo {author} {\bibfnamefont {Y.}~\bibnamefont {Saito}}, \bibinfo
  {author} {\bibfnamefont {T.}~\bibnamefont {Ishiyama}}, \bibinfo {author}
  {\bibfnamefont {T.}~\bibnamefont {Higomoto}}, \bibinfo {author}
  {\bibfnamefont {T.}~\bibnamefont {Takano}}, \bibinfo {author} {\bibfnamefont
  {Y.}~\bibnamefont {Takasu}}, \bibinfo {author} {\bibfnamefont
  {Y.}~\bibnamefont {Yamamoto}}, \bibinfo {author} {\bibfnamefont
  {M.}~\bibnamefont {Tanaka}},\ and\ \bibinfo {author} {\bibfnamefont
  {Y.}~\bibnamefont {Takahashi}},\ }\bibfield  {title} {\bibinfo {title}
  {{Observation of Nonlinearity of Generalized King Plot in the Search for New
  Boson}},\ }\href {https://doi.org/10.1103/PhysRevX.12.021033} {\bibfield
  {journal} {\bibinfo  {journal} {Phys. Rev. X}\ }\textbf {\bibinfo {volume}
  {12}},\ \bibinfo {pages} {021033} (\bibinfo {year} {2022})}\BibitemShut
  {NoStop}%
\bibitem [{\citenamefont {Baillard}\ \emph {et~al.}(2006)\citenamefont
  {Baillard}, \citenamefont {Gauguet}, \citenamefont {Bize}, \citenamefont
  {Lemonde}, \citenamefont {Laurent}, \citenamefont {Clairon},\ and\
  \citenamefont {Rosenbusch}}]{Baillard2006}%
  \BibitemOpen
  \bibfield  {author} {\bibinfo {author} {\bibfnamefont {X.}~\bibnamefont
  {Baillard}}, \bibinfo {author} {\bibfnamefont {A.}~\bibnamefont {Gauguet}},
  \bibinfo {author} {\bibfnamefont {S.}~\bibnamefont {Bize}}, \bibinfo {author}
  {\bibfnamefont {P.}~\bibnamefont {Lemonde}}, \bibinfo {author} {\bibfnamefont
  {P.}~\bibnamefont {Laurent}}, \bibinfo {author} {\bibfnamefont
  {A.}~\bibnamefont {Clairon}},\ and\ \bibinfo {author} {\bibfnamefont
  {P.}~\bibnamefont {Rosenbusch}},\ }\bibfield  {title} {\bibinfo {title}
  {Interference-filter-stabilized external-cavity diode lasers},\ }\href
  {https://doi.org/https://doi.org/10.1016/j.optcom.2006.05.011} {\bibfield
  {journal} {\bibinfo  {journal} {Optics Communications}\ }\textbf {\bibinfo
  {volume} {266}},\ \bibinfo {pages} {609} (\bibinfo {year}
  {2006})}\BibitemShut {NoStop}%
\bibitem [{\citenamefont {Drever}\ \emph {et~al.}(1983)\citenamefont {Drever},
  \citenamefont {Hall}, \citenamefont {Kowalski}, \citenamefont {Hough},
  \citenamefont {Ford}, \citenamefont {Munley},\ and\ \citenamefont
  {Ward}}]{Drever1983}%
  \BibitemOpen
  \bibfield  {author} {\bibinfo {author} {\bibfnamefont {R.~W.~P.}\
  \bibnamefont {Drever}}, \bibinfo {author} {\bibfnamefont {J.~L.}\
  \bibnamefont {Hall}}, \bibinfo {author} {\bibfnamefont {F.~V.}\ \bibnamefont
  {Kowalski}}, \bibinfo {author} {\bibfnamefont {J.}~\bibnamefont {Hough}},
  \bibinfo {author} {\bibfnamefont {G.~M.}\ \bibnamefont {Ford}}, \bibinfo
  {author} {\bibfnamefont {A.~J.}\ \bibnamefont {Munley}},\ and\ \bibinfo
  {author} {\bibfnamefont {H.}~\bibnamefont {Ward}},\ }\bibfield  {title}
  {\bibinfo {title} {Laser phase and frequency stabilization using an optical
  resonator},\ }\href {https://doi.org/10.1007/BF00702605} {\bibfield
  {journal} {\bibinfo  {journal} {Applied Physics B}\ }\textbf {\bibinfo
  {volume} {31}},\ \bibinfo {pages} {97} (\bibinfo {year} {1983})}\BibitemShut
  {NoStop}%
\bibitem [{\citenamefont {Budick}\ and\ \citenamefont
  {Snir}(1967)}]{Budick1967}%
  \BibitemOpen
  \bibfield  {author} {\bibinfo {author} {\bibfnamefont {B.}~\bibnamefont
  {Budick}}\ and\ \bibinfo {author} {\bibfnamefont {J.}~\bibnamefont {Snir}},\
  }\bibfield  {title} {\bibinfo {title} {$g_j$ values of excited states of the
  ytterbium atom},\ }\href
  {https://www.sciencedirect.com/science/article/pii/0375960167910316}
  {\bibfield  {journal} {\bibinfo  {journal} {Physics Letters A}\ }\textbf
  {\bibinfo {volume} {24}},\ \bibinfo {pages} {689} (\bibinfo {year}
  {1967})}\BibitemShut {NoStop}%
\bibitem [{\citenamefont {Baumann}\ and\ \citenamefont
  {Wandel}(1968)}]{Baumann1968}%
  \BibitemOpen
  \bibfield  {author} {\bibinfo {author} {\bibfnamefont {M.}~\bibnamefont
  {Baumann}}\ and\ \bibinfo {author} {\bibfnamefont {G.}~\bibnamefont
  {Wandel}},\ }\bibfield  {title} {\bibinfo {title} {$g_j$ factors of the 6s6p
  $^3\text{P}_1$ and 6s6p $^1\text{P}_1$ states of ytterbium},\ }\href
  {https://www.sciencedirect.com/science/article/pii/0375960168902004}
  {\bibfield  {journal} {\bibinfo  {journal} {Physics Letters A}\ }\textbf
  {\bibinfo {volume} {28}},\ \bibinfo {pages} {200} (\bibinfo {year}
  {1968})}\BibitemShut {NoStop}%
\bibitem [{\citenamefont {Dzuba}\ \emph {et~al.}(2018)\citenamefont {Dzuba},
  \citenamefont {Flambaum},\ and\ \citenamefont {Schiller}}]{Dzuba2018}%
  \BibitemOpen
  \bibfield  {author} {\bibinfo {author} {\bibfnamefont {V.~A.}\ \bibnamefont
  {Dzuba}}, \bibinfo {author} {\bibfnamefont {V.~V.}\ \bibnamefont
  {Flambaum}},\ and\ \bibinfo {author} {\bibfnamefont {S.}~\bibnamefont
  {Schiller}},\ }\bibfield  {title} {\bibinfo {title} {Testing physics beyond
  the standard model through additional clock transitions in neutral
  ytterbium},\ }\href {https://doi.org/10.1103/PhysRevA.98.022501} {\bibfield
  {journal} {\bibinfo  {journal} {Phys. Rev. A}\ }\textbf {\bibinfo {volume}
  {98}},\ \bibinfo {pages} {022501} (\bibinfo {year} {2018})}\BibitemShut
  {NoStop}%
\bibitem [{\citenamefont {Schwartz}(1955)}]{Charles1955}%
  \BibitemOpen
  \bibfield  {author} {\bibinfo {author} {\bibfnamefont {C.}~\bibnamefont
  {Schwartz}},\ }\bibfield  {title} {\bibinfo {title} {{Theory of Hyperfine
  Structure}},\ }\href {https://doi.org/10.1103/PhysRev.97.380} {\bibfield
  {journal} {\bibinfo  {journal} {Phys. Rev.}\ }\textbf {\bibinfo {volume}
  {97}},\ \bibinfo {pages} {380} (\bibinfo {year} {1955})}\BibitemShut
  {NoStop}%
\bibitem [{\citenamefont {Olschewski}(1972)}]{Olschewski1972ZP}%
  \BibitemOpen
  \bibfield  {author} {\bibinfo {author} {\bibfnamefont {L.}~\bibnamefont
  {Olschewski}},\ }\bibfield  {title} {\bibinfo {title} {Messung der
  magnetischen kerndipolmomente an
  {freien$^{43}$Ca-,$^{87}$Sr-,$^{135}$Ba-,$^{137}$Ba-,$^{171}$Yb-}
  {und$^{173}$Yb-Atomen} mit optischem pumpen},\ }\href
  {https://doi.org/10.1007/BF01400226} {\bibfield  {journal} {\bibinfo
  {journal} {Zeitschrift f{\"u}r Physik}\ }\textbf {\bibinfo {volume} {249}},\
  \bibinfo {pages} {205} (\bibinfo {year} {1972})}\BibitemShut {NoStop}%
\bibitem [{\citenamefont {Stone}(2005)}]{Stone2005ADNDT}%
  \BibitemOpen
  \bibfield  {author} {\bibinfo {author} {\bibfnamefont {N.~J.}\ \bibnamefont
  {Stone}},\ }\bibfield  {title} {\bibinfo {title} {Table of nuclear magnetic
  dipole and electric quadrupole moments},\ }\href
  {https://doi.org/10.1016/j.adt.2005.04.001} {\bibfield  {journal} {\bibinfo
  {journal} {At. Data Nucl. Data Tables}\ }\textbf {\bibinfo {volume} {90}},\
  \bibinfo {pages} {75} (\bibinfo {year} {2005})}\BibitemShut {NoStop}%
\bibitem [{\citenamefont {Porsev}\ \emph {et~al.}(2004)\citenamefont {Porsev},
  \citenamefont {Derevianko},\ and\ \citenamefont {Fortson}}]{Porsev2004}%
  \BibitemOpen
  \bibfield  {author} {\bibinfo {author} {\bibfnamefont {S.~G.}\ \bibnamefont
  {Porsev}}, \bibinfo {author} {\bibfnamefont {A.}~\bibnamefont {Derevianko}},\
  and\ \bibinfo {author} {\bibfnamefont {E.~N.}\ \bibnamefont {Fortson}},\
  }\bibfield  {title} {\bibinfo {title} {Possibility of an optical clock using
  the $6{}^{1}{S}_{0}\ensuremath{\rightarrow}6{}^{3}{P}_{0}^{o}$ transition in
  ${}^{171,173}\mathrm{Yb}$ atoms held in an optical lattice},\ }\href
  {https://doi.org/10.1103/PhysRevA.69.021403} {\bibfield  {journal} {\bibinfo
  {journal} {Phys. Rev. A}\ }\textbf {\bibinfo {volume} {69}},\ \bibinfo
  {pages} {021403} (\bibinfo {year} {2004})}\BibitemShut {NoStop}%
\bibitem [{\citenamefont {Boyd}\ \emph {et~al.}(2007)\citenamefont {Boyd},
  \citenamefont {Zelevinsky}, \citenamefont {Ludlow}, \citenamefont {Blatt},
  \citenamefont {Zanon-Willette}, \citenamefont {Foreman},\ and\ \citenamefont
  {Ye}}]{Boyd2007}%
  \BibitemOpen
  \bibfield  {author} {\bibinfo {author} {\bibfnamefont {M.~M.}\ \bibnamefont
  {Boyd}}, \bibinfo {author} {\bibfnamefont {T.}~\bibnamefont {Zelevinsky}},
  \bibinfo {author} {\bibfnamefont {A.~D.}\ \bibnamefont {Ludlow}}, \bibinfo
  {author} {\bibfnamefont {S.}~\bibnamefont {Blatt}}, \bibinfo {author}
  {\bibfnamefont {T.}~\bibnamefont {Zanon-Willette}}, \bibinfo {author}
  {\bibfnamefont {S.~M.}\ \bibnamefont {Foreman}},\ and\ \bibinfo {author}
  {\bibfnamefont {J.}~\bibnamefont {Ye}},\ }\bibfield  {title} {\bibinfo
  {title} {Nuclear spin effects in optical lattice clocks},\ }\href
  {https://doi.org/10.1103/PhysRevA.76.022510} {\bibfield  {journal} {\bibinfo
  {journal} {Phys. Rev. A}\ }\textbf {\bibinfo {volume} {76}},\ \bibinfo
  {pages} {022510} (\bibinfo {year} {2007})}\BibitemShut {NoStop}%
\bibitem [{\citenamefont {Le~Kien}\ \emph {et~al.}(2013)\citenamefont
  {Le~Kien}, \citenamefont {Schneeweiss},\ and\ \citenamefont
  {Rauschenbeutel}}]{LeKien2013}%
  \BibitemOpen
  \bibfield  {author} {\bibinfo {author} {\bibfnamefont {F.}~\bibnamefont
  {Le~Kien}}, \bibinfo {author} {\bibfnamefont {P.}~\bibnamefont
  {Schneeweiss}},\ and\ \bibinfo {author} {\bibfnamefont {A.}~\bibnamefont
  {Rauschenbeutel}},\ }\bibfield  {title} {\bibinfo {title} {Dynamical
  polarizability of atoms in arbitrary light fields: general theory and
  application to cesium},\ }\href {https://doi.org/10.1140/epjd/e2013-30729-x}
  {\bibfield  {journal} {\bibinfo  {journal} {The European Physical Journal D}\
  }\textbf {\bibinfo {volume} {67}},\ \bibinfo {pages} {92} (\bibinfo {year}
  {2013})}\BibitemShut {NoStop}%
\bibitem [{\citenamefont {Tang}\ \emph {et~al.}(2018)\citenamefont {Tang},
  \citenamefont {Yu}, \citenamefont {Jiang},\ and\ \citenamefont
  {Dong}}]{Tang2018}%
  \BibitemOpen
  \bibfield  {author} {\bibinfo {author} {\bibfnamefont {Z.-M.}\ \bibnamefont
  {Tang}}, \bibinfo {author} {\bibfnamefont {Y.-M.}\ \bibnamefont {Yu}},
  \bibinfo {author} {\bibfnamefont {J.}~\bibnamefont {Jiang}},\ and\ \bibinfo
  {author} {\bibfnamefont {C.-Z.}\ \bibnamefont {Dong}},\ }\bibfield  {title}
  {\bibinfo {title} {Magic wavelengths for the transition in ytterbium atom},\
  }\href {https://doi.org/10.1088/1361-6455/aac181} {\bibfield  {journal}
  {\bibinfo  {journal} {Journal of Physics B: Atomic, Molecular and Optical
  Physics}\ }\textbf {\bibinfo {volume} {51}},\ \bibinfo {pages} {125002}
  (\bibinfo {year} {2018})}\BibitemShut {NoStop}%
\bibitem [{\citenamefont {Taichenachev}\ \emph {et~al.}(2006)\citenamefont
  {Taichenachev}, \citenamefont {Yudin}, \citenamefont {Oates}, \citenamefont
  {Hoyt}, \citenamefont {Barber},\ and\ \citenamefont
  {Hollberg}}]{Taichenachev2006}%
  \BibitemOpen
  \bibfield  {author} {\bibinfo {author} {\bibfnamefont {A.~V.}\ \bibnamefont
  {Taichenachev}}, \bibinfo {author} {\bibfnamefont {V.~I.}\ \bibnamefont
  {Yudin}}, \bibinfo {author} {\bibfnamefont {C.~W.}\ \bibnamefont {Oates}},
  \bibinfo {author} {\bibfnamefont {C.~W.}\ \bibnamefont {Hoyt}}, \bibinfo
  {author} {\bibfnamefont {Z.~W.}\ \bibnamefont {Barber}},\ and\ \bibinfo
  {author} {\bibfnamefont {L.}~\bibnamefont {Hollberg}},\ }\bibfield  {title}
  {\bibinfo {title} {{Magnetic Field-Induced Spectroscopy of Forbidden Optical
  Transitions with Application to Lattice-Based Optical Atomic Clocks}},\
  }\href {https://doi.org/10.1103/PhysRevLett.96.083001} {\bibfield  {journal}
  {\bibinfo  {journal} {Phys. Rev. Lett.}\ }\textbf {\bibinfo {volume} {96}},\
  \bibinfo {pages} {083001} (\bibinfo {year} {2006})}\BibitemShut {NoStop}%
\bibitem [{\citenamefont {Akatsuka}\ \emph {et~al.}(2008)\citenamefont
  {Akatsuka}, \citenamefont {Takamoto},\ and\ \citenamefont
  {Katori}}]{Akatsuka2008}%
  \BibitemOpen
  \bibfield  {author} {\bibinfo {author} {\bibfnamefont {T.}~\bibnamefont
  {Akatsuka}}, \bibinfo {author} {\bibfnamefont {M.}~\bibnamefont {Takamoto}},\
  and\ \bibinfo {author} {\bibfnamefont {H.}~\bibnamefont {Katori}},\
  }\bibfield  {title} {\bibinfo {title} {Optical lattice clocks with
  non-interacting bosons and fermions},\ }\href
  {https://doi.org/10.1038/nphys1108} {\bibfield  {journal} {\bibinfo
  {journal} {Nature Physics}\ }\textbf {\bibinfo {volume} {4}},\ \bibinfo
  {pages} {954} (\bibinfo {year} {2008})}\BibitemShut {NoStop}%
\bibitem [{\citenamefont {Takamoto}\ \emph {et~al.}(2006)\citenamefont
  {Takamoto}, \citenamefont {Hong}, \citenamefont {Higashi}, \citenamefont
  {Fujii}, \citenamefont {Imae},\ and\ \citenamefont {Katori}}]{Takamoto2006}%
  \BibitemOpen
  \bibfield  {author} {\bibinfo {author} {\bibfnamefont {M.}~\bibnamefont
  {Takamoto}}, \bibinfo {author} {\bibfnamefont {F.-L.}\ \bibnamefont {Hong}},
  \bibinfo {author} {\bibfnamefont {R.}~\bibnamefont {Higashi}}, \bibinfo
  {author} {\bibfnamefont {Y.}~\bibnamefont {Fujii}}, \bibinfo {author}
  {\bibfnamefont {M.}~\bibnamefont {Imae}},\ and\ \bibinfo {author}
  {\bibfnamefont {H.}~\bibnamefont {Katori}},\ }\bibfield  {title} {\bibinfo
  {title} {{Improved Frequency Measurement of a One-Dimensional Optical Lattice
  Clock with a Spin-Polarized Fermionic $^{87}$Sr Isotope}},\ }\href
  {https://doi.org/10.1143/JPSJ.75.104302} {\bibfield  {journal} {\bibinfo
  {journal} {Journal of the Physical Society of Japan}\ }\textbf {\bibinfo
  {volume} {75}},\ \bibinfo {pages} {104302} (\bibinfo {year}
  {2006})}\BibitemShut {NoStop}%
\bibitem [{\citenamefont {Young}\ \emph {et~al.}(2020)\citenamefont {Young},
  \citenamefont {Eckner}, \citenamefont {Milner}, \citenamefont {Kedar},
  \citenamefont {Norcia}, \citenamefont {Oelker}, \citenamefont {Schine},
  \citenamefont {Ye},\ and\ \citenamefont {Kaufman}}]{Young2020}%
  \BibitemOpen
  \bibfield  {author} {\bibinfo {author} {\bibfnamefont {A.~W.}\ \bibnamefont
  {Young}}, \bibinfo {author} {\bibfnamefont {W.~J.}\ \bibnamefont {Eckner}},
  \bibinfo {author} {\bibfnamefont {W.~R.}\ \bibnamefont {Milner}}, \bibinfo
  {author} {\bibfnamefont {D.}~\bibnamefont {Kedar}}, \bibinfo {author}
  {\bibfnamefont {M.~A.}\ \bibnamefont {Norcia}}, \bibinfo {author}
  {\bibfnamefont {E.}~\bibnamefont {Oelker}}, \bibinfo {author} {\bibfnamefont
  {N.}~\bibnamefont {Schine}}, \bibinfo {author} {\bibfnamefont
  {J.}~\bibnamefont {Ye}},\ and\ \bibinfo {author} {\bibfnamefont {A.~M.}\
  \bibnamefont {Kaufman}},\ }\bibfield  {title} {\bibinfo {title}
  {Half-minute-scale atomic coherence and high relative stability in a tweezer
  clock},\ }\href {https://doi.org/10.1038/s41586-020-3009-y} {\bibfield
  {journal} {\bibinfo  {journal} {Nature}\ }\textbf {\bibinfo {volume} {588}},\
  \bibinfo {pages} {408} (\bibinfo {year} {2020})}\BibitemShut {NoStop}%
\bibitem [{\citenamefont {Saue}\ \emph {et~al.}(2020)\citenamefont {Saue},
  \citenamefont {Bast}, \citenamefont {Gomes}, \citenamefont {Jensen},
  \citenamefont {Visscher}, \citenamefont {Aucar}, \citenamefont {Di~Remigio},
  \citenamefont {Dyall}, \citenamefont {Eliav}, \citenamefont {Fasshauer} \emph
  {et~al.}}]{saue2020dirac}%
  \BibitemOpen
  \bibfield  {author} {\bibinfo {author} {\bibfnamefont {T.}~\bibnamefont
  {Saue}}, \bibinfo {author} {\bibfnamefont {R.}~\bibnamefont {Bast}}, \bibinfo
  {author} {\bibfnamefont {A.~S.~P.}\ \bibnamefont {Gomes}}, \bibinfo {author}
  {\bibfnamefont {H.~J.~A.}\ \bibnamefont {Jensen}}, \bibinfo {author}
  {\bibfnamefont {L.}~\bibnamefont {Visscher}}, \bibinfo {author}
  {\bibfnamefont {I.~A.}\ \bibnamefont {Aucar}}, \bibinfo {author}
  {\bibfnamefont {R.}~\bibnamefont {Di~Remigio}}, \bibinfo {author}
  {\bibfnamefont {K.~G.}\ \bibnamefont {Dyall}}, \bibinfo {author}
  {\bibfnamefont {E.}~\bibnamefont {Eliav}}, \bibinfo {author} {\bibfnamefont
  {E.}~\bibnamefont {Fasshauer}}, \emph {et~al.},\ }\bibfield  {title}
  {\bibinfo {title} {{The DIRAC code for relativistic molecular
  calculations}},\ }\href {https://doi.org/10.1063/5.0004844} {\bibfield
  {journal} {\bibinfo  {journal} {J. Chem. Phys.}\ }\textbf {\bibinfo {volume}
  {152}},\ \bibinfo {pages} {204104} (\bibinfo {year} {2020})}\BibitemShut
  {NoStop}%
\bibitem [{DIR()}]{DIRAC22}%
  \BibitemOpen
  \href@noop {} {}\bibinfo {note} {{DIRAC}, a relativistic ab initio electronic
  structure program, Release {DIRAC22} (2022), written by H.~J.~{\relax
  Aa}.~Jensen, R.~Bast, A.~S.~P.~Gomes, T.~Saue and L.~Visscher et al.
  (available at \url{http://dx.doi.org/10.5281/zenodo.6010450}, see also
  \url{http://www.diracprogram.org}), accessed Sep. 29, 2022.}\BibitemShut
  {Stop}%
\bibitem [{\citenamefont {Dyall}\ and\ \citenamefont
  {Gomes}(2010)}]{Dyall_Gomes_2010}%
  \BibitemOpen
  \bibfield  {author} {\bibinfo {author} {\bibfnamefont {K.~G.}\ \bibnamefont
  {Dyall}}\ and\ \bibinfo {author} {\bibfnamefont {A.~S.}\ \bibnamefont
  {Gomes}},\ }\bibfield  {title} {\bibinfo {title} {{Revised relativistic basis
  sets for the 5d elements Hf-Hg}},\ }\href
  {https://doi.org/10.1007/s00214-009-0717-7} {\bibfield  {journal} {\bibinfo
  {journal} {Theoretical Chemistry Accounts}\ }\textbf {\bibinfo {volume}
  {125}},\ \bibinfo {pages} {97–100} (\bibinfo {year} {2010})}\BibitemShut
  {NoStop}%
\bibitem [{\citenamefont {Fleig}\ \emph {et~al.}(2003)\citenamefont {Fleig},
  \citenamefont {Olsen},\ and\ \citenamefont {Visscher}}]{Fleig2003JCP}%
  \BibitemOpen
  \bibfield  {author} {\bibinfo {author} {\bibfnamefont {T.}~\bibnamefont
  {Fleig}}, \bibinfo {author} {\bibfnamefont {J.}~\bibnamefont {Olsen}},\ and\
  \bibinfo {author} {\bibfnamefont {L.}~\bibnamefont {Visscher}},\ }\bibfield
  {title} {\bibinfo {title} {{The generalized active space concept for the
  relativistic treatment of electron correlation. II. Large-scale configuration
  interaction implementation based on relativistic 2- and 4-spinors and its
  application}},\ }\href {https://doi.org/10.1063/1.1590636} {\bibfield
  {journal} {\bibinfo  {journal} {J. Chem. Phys.}\ }\textbf {\bibinfo {volume}
  {119}},\ \bibinfo {pages} {2963} (\bibinfo {year} {2003})}\BibitemShut
  {NoStop}%
\bibitem [{\citenamefont {Knecht}\ \emph {et~al.}(2010)\citenamefont {Knecht},
  \citenamefont {Jensen},\ and\ \citenamefont {Fleig}}]{Knecht2010JCP}%
  \BibitemOpen
  \bibfield  {author} {\bibinfo {author} {\bibfnamefont {S.}~\bibnamefont
  {Knecht}}, \bibinfo {author} {\bibfnamefont {H.~J.~A.}\ \bibnamefont
  {Jensen}},\ and\ \bibinfo {author} {\bibfnamefont {T.}~\bibnamefont
  {Fleig}},\ }\bibfield  {title} {\bibinfo {title} {{Large-scale parallel
  configuration interaction. II. Two- and four-component double-group general
  active space implementation with application to BiH}},\ }\href
  {https://doi.org/10.1063/1.3276157} {\bibfield  {journal} {\bibinfo
  {journal} {J. Chem. Phys.}\ }\textbf {\bibinfo {volume} {132}},\ \bibinfo
  {pages} {014108} (\bibinfo {year} {2010})}\BibitemShut {NoStop}%
\bibitem [{Kne()}]{Knecht_thesis}%
  \BibitemOpen
  \href@noop {} {}\bibinfo {note} {Stefan R. Knecht. Parallel Relativistic
  Multiconfiguration Methods: New Powerful Tools for Heavy-Element
  Electronic-Structure Studies. PhD thesis, Mathematisch-Naturwissenschaftliche
  Fakult{\"{a}}t, Heinrich-Heine- Universit{\"{a}}t D{\"{u}}sseldorf, 2009.
  URL:
  http://docserv.uni-duesseldorf.de/servlets/DocumentServlet?id=13226.}\BibitemShut
  {Stop}%
\bibitem [{\citenamefont {Denis}\ \emph {et~al.}(2015)\citenamefont {Denis},
  \citenamefont {N{\o}rby}, \citenamefont {Jensen}, \citenamefont {Gomes},
  \citenamefont {Nayak}, \citenamefont {Knecht},\ and\ \citenamefont
  {Fleig}}]{Denis2015NJP}%
  \BibitemOpen
  \bibfield  {author} {\bibinfo {author} {\bibfnamefont {M.}~\bibnamefont
  {Denis}}, \bibinfo {author} {\bibfnamefont {M.~S.}\ \bibnamefont {N{\o}rby}},
  \bibinfo {author} {\bibfnamefont {H.~J.~A.}\ \bibnamefont {Jensen}}, \bibinfo
  {author} {\bibfnamefont {A.~S.~P.}\ \bibnamefont {Gomes}}, \bibinfo {author}
  {\bibfnamefont {M.~K.}\ \bibnamefont {Nayak}}, \bibinfo {author}
  {\bibfnamefont {S.}~\bibnamefont {Knecht}},\ and\ \bibinfo {author}
  {\bibfnamefont {T.}~\bibnamefont {Fleig}},\ }\bibfield  {title} {\bibinfo
  {title} {{Theoretical study on ThF$^+$, a prospective system in search of
  time-reversal violation}},\ }\href
  {https://doi.org/10.1088/1367-2630/17/4/043005} {\bibfield  {journal}
  {\bibinfo  {journal} {New Journal of Physics}\ }\textbf {\bibinfo {volume}
  {17}},\ \bibinfo {pages} {043005} (\bibinfo {year} {2015})}\BibitemShut
  {NoStop}%
\end{thebibliography}%

\end{document}