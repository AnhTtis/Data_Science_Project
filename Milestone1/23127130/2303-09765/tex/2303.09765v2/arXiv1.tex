\documentclass[%
 reprint,
%superscriptaddress,
%groupedaddress,
%unsortedaddress,
%runinaddress,
%frontmatterverbose, 
% preprint,
% preprintnumbers,
%nofootinbib,
%nobibnotes,
%bibnotes,
 amsmath,amssymb,
 aps,
 prl,
%prb,
%rmp,
%prstab,
%prstper,
%floatfix,
]{revtex4-2}


%%%%%%%%%% usepackage %%%%%%%%%%
\usepackage{graphicx}% Include figure files
\usepackage{dcolumn}% Align table columns on decimal point
\usepackage{bm}% bold math
\usepackage[colorlinks=true,urlcolor=blue,citecolor=blue, linkcolor=blue]{hyperref}
\usepackage{siunitx} % ishiyama
\usepackage{braket}
\usepackage{comment}
\usepackage{mathtools}
\usepackage{multirow}
\usepackage{lineno}

% \usepackage{physics}

%\usepackage{hyperref}% add hypertext capabilities
%\usepackage[mathlines]{lineno}% Enable numbering of text and display math
%\linenumbers\relax % Commence numbering lines

%\usepackage[showframe,%Uncomment any one of the following lines to test 
%%scale=0.7, marginratio={1:1, 2:3}, ignoreall,% default settings
%%text={7in,10in},centering,
%%margin=1.5in,
%%total={6.5in,8.75in}, top=1.2in, left=0.9in, includefoot,
%%height=10in,a5paper,hmargin={3cm,0.8in},
%]{geometry}

% \subsection{\label{sec:level2}}
% \subsubsection{Wide text (A level-3 head)}
% \subsection{\label{sec:citeref}Citations and References}
% \subsubsection{Citations}
% \paragraph{Syntax}


%%%%%%%%%% substitute %%%%%%%%%%
% |g\rangle --> \ket{g}
% |e\rangle --> \ket{e}


%%%%%%%%%% Document %%%%%%%%%%
\begin{document}
\preprint{APS/123-QED}

\title{Observation of an Inner-Shell Orbital Clock Transition in Neutral Ytterbium Atoms}

\author{Taiki Ishiyama}
% \email{ishiyama.taiki.88e@st.kyoto-u.ac.jp}
\author{Koki Ono}
 \email{koukiono3@yagura.scphys.kyoto-u.ac.jp}
\author{Tetsushi Takano}
\author{Ayaki Sunaga}
\author{Yoshiro Takahashi}
\affiliation{Department of Physics, Graduate School of Science, Kyoto University, Kyoto 606-8502, Japan}
\date{\today}

% \linenumbers

%%%%%%%%%% Abstract %%%%%%%%%%
\begin{abstract}
We observe a weakly allowed optical transition of atomic ytterbium 
from the ground state to the metastable state $4f^{13}5d6s^2 \: (J=2)$ for all five bosonic and two fermionic isotopes with resolved Zeeman and hyperfine structures. 
This inner-shell orbital transition has been proposed as a new frequency standard as well as a quantum sensor for new physics. 
We find magic wavelengths through the measurement of the scalar and tensor polarizabilities and reveal that the measured trap lifetime in a three-dimensional optical lattice is 1.9(1)~s, which is crucial for precision measurements. We also determine the $g$ factor by an interleaved measurement, consistent with our relativistic atomic calculation.
This work opens the possibility of an optical lattice clock with improved stability and accuracy 
as well as novel approaches for physics beyond the standard model.
\end{abstract}
\maketitle

%%%%%%%%%%%% Introduction %%%%%%%%%%%%
Recent development of optical atomic clocks using ions and neutral atoms has established a high fractional accuracy at the $10^{-18}$ level~\cite{Chou2010, Bloom2014, Ushijima2015}.
In addition to the contribution to metrology 
such as the redefinition of the second~\cite{Grebing2016, Milner2019} and geodesy~\cite{Takano2016, Lisdat2016}, 
a highly stable atomic clock enables various applications 
ranging from quantum simulations~\cite{Kato2016, Kolkowitz2017} 
to fundamental physics~\cite{SafronovaRMP2018} including gravitational wave detection~\cite{Kolkowitz2016}, and others. 
The development of even more precise clocks benefits all of these applications.

Recently, the optical transition of atomic ytterbium (Yb) to its metastable state $4f^{13}5d6s^2 \: (J=2)$ with an energy of 23188.518~cm$^{-1}$~\cite{NIST2022}, 
has been proposed as a new frequency standard with high stability and accuracy~\cite{Safronova2018, Dzuba2018}. 
As shown in Fig.~\ref{Energy diagram}, the radiative lifetime of the $4f^{13}5d6s^2 \: (J=2)$ state is calculated to be much longer than that of the other metastable states, which potentially improves the quality factor of an optical lattice clock.
In addition, the quadratic Zeeman shift as well as the black-body-radiation shift of the $^1S_0$ $\leftrightarrow$ $4f^{13}5d6s^2 \: (J=2)$ transition are calculated to be considerably small compared to those of the $^1S_0$ $\leftrightarrow$ $^3P_0$ and $^1S_0$ $\leftrightarrow$ $^3P_2$ transitions~\cite{Dzuba2018}, suggesting that the inner-shell orbital clock transition has a potential to reduce systematic uncertainties.
The dual clock operation~\cite{Akamatsu2018} combined with the well-established $^1S_0$ $\leftrightarrow$ $^3P_0$ transition, 
or already observed $^1S_0$ $\leftrightarrow$ $^3P_2$ transition~\cite{Yamaguchi2010} of Yb atoms, 
possibly can further reduce systematic uncertainties in a clock comparison. 

\begin{figure}[t]
\centering
\includegraphics[width = 0.95\linewidth]{Fig1_energy_diagram_rev2.pdf}
\caption{\label{Energy diagram}
Clock transitions in an Yb atom. 
The relevant transition wavelengths and lifetimes $\tau$~\cite{Dzuba2018} are shown. 
%In this letter, we study an ultra-narrow transition $4f^{14}6s^2 \: ^1S_0 \rightarrow 4f^{13}5d6s^2 \: (J = 2)$.
%(b) Schematic of our experimental setup. 
%Our experiments are performed with Yb atoms in a crossed FORT consisting of 1064 nm and 1070 nm laser beams. 
%The linear-polarized excitation light, shown as the blue arrow, is irradiated from the $y$-axis. 
%For polarizability measurements, linear-polarized titanium (Ti)-sapphire laser, shown as the red arrow, is shined along the $x$-axis. 
%Note that the quantization axis defined by the magnetic field is changed for each measurement, as well as the polarization angles of the Ti:Sapphire and excitation lasers.
}
\end{figure}

This metastable state of Yb also attracts considerable interest from the viewpoints of new physics searches~\cite
{Shaniv2018, Safronova2018, Dzuba2018}.
The $^3P_0$ $\leftrightarrow$ $4f^{13}5d6s^2 \: (J=2)$ as well as $^3P_2$ $\leftrightarrow$ $4f^{13}5d6s^2 \: (J=2)$~\cite{Tang2022} transitions show some of the highest sensitivities in neutral atoms to the variation of the fine-structure constant, 
providing also the possibilities of searching for ultralight scalar dark matter~\cite{Arvanitaki2015} and testing Einstein’s equivalence principle~\cite{SafronovaRMP2018}. 
The metastable state has also high sensitivity to the violation of local Lorentz invariance in the electron-photon sector~\cite{Kostelecky1999, Shaniv2018}, 
so far studied using single or two ions~\cite{Pruttivarasin2015, Megidish2019, Sanner2019, Dreissen2022} and dysprosium atoms~\cite{Hohensee2013}. 
Furthermore, establishing a new clock transition associated with the metastable state, in addition to the $^1S_0$ $\leftrightarrow$ $^3P_0$ and $^1S_0$ $\leftrightarrow$ $^3P_2$ transitions, 
provides a unique possibility to study a nonlinearity of the King plot using five bosonic isotopes of Yb 
in the search for a hypothetical particle mediating a force between electrons and neutrons beyond the Standard Model~\cite{Berengut2018, Ono2022}.
Clearly, the first step towards these experiments is to experimentally observe the associated transition and characterizations such as the lifetime~\cite{Safronova2018, Dzuba2018, Tang2022} and possible existence of a magic wavelength~\cite{Dzuba2018,Tang2022}.
 %which have not been studied experimentally yet.

In this Letter, we report the observation of the $^1S_0$ $\leftrightarrow$ $4f^{13}5d6s^2 \: (J=2)$ optical transition of Yb.  
% (hereafter $\ket{e}$) 
The resonances for all five bosonic and two fermionic isotopes are clearly observed with resolved Zeeman and hyperfine structures. 
We find two magic wavelengths of 797.2(4)~nm and 834.2(1)~nm for the practical condition of $m_J$ = 0 and the laser polarization perpendicular to a magnetic field, through the measurement of the polarizability. 
%and our calculation shows that the photon scattering rates at these wavelengths in the expected operating conditions are on the order of 0.1~s$^{-1}$, comparable with the state-of-the-art optical lattice clocks~\cite{Takamoto2009}. 
Here $m_J$ is the projection of the total electronic angular momentum $J$ along the quantization axis.
In addition, our excitation and de-excitation sequence measurement of atoms in a three-dimensional optical lattice enables us to obtain the trap lifetime to be 1.9(1)~s.
Through an interleaved measurement with different magnetic fields, we determine the $g$ factor to be 1.463(4), consistent with our relativistic many-body calculation. 
%Furthermore, our calculation for the $4f^{14}6s^2 \: ^1S_0$ - $4f^{13}5d6s^2 \: (J=2)$ transition reveals a two-orders of magnitude larger sensitivity for a new hypothetical particle mediating the force between an electron and a neutron than that of the already studied $4f^{14}6s^2 \: ^1S_0$ - $4f^{14}6s6p \: ^3P_0$ transition, 
%owing to the largely different electronic configurations between the $4f^{14}6s^2 \: ^1S_0$ and $4f^{13}5d6s^2 \: (J=2)$ states. 
These findings are promising for the development of an optical lattice clock with improved stability and accuracy, 
and its various applications for searching for physics beyond the Standard Model.



%%%%%%%%%%%% A new clock transition in Yb %%%%%%%%%%%%
%\indent
%{\it A new clock transition in Yb.{\bf \textemdash}}
%Figure~\ref{Fig1}(a) shows three clock transitions of the ytterbium atom.

%The ground state is [Xe] $4f^{14}6s^2 \: ^1S_0$ (hereafter $\ket{g}$)
%where [Xe] represents the electron configuration of xenon, which is omitted in the figure.
%The states $4f^{14}6s6p \: ^3P_0$ and $^3P_2$, where one of the outermost $6s^2$ electrons in the ground state is %excited to 6p, are metastable states with a lifetime of a few tens of seconds.
%Since the natural width and quality factor of the transitions are inversely proportional to the lifetime, there exist %ultra-narrow linewidth transitions between the ground state and the metastable states.
%As mentioned in the introduction, these metastable states and ultra-narrow linewidth transitions have been extensively %utilized from precision measurements[] to quantum simulations[] and calculations[].
%The target of this study is the $4f^{13}5d6s^2 \: (J = 2)$ (hereafter $\ket{e}$) state in which the inner-shell electron %$4f$ is excited to $5d$.
%This state is magnetic quadrupole (M2)-allowed, with electric dipole (E1), magnetic dipole (M1), and electric quadrupole (E2) transitions to and from the ground state prohibited.
%Transition wavelength is $431$ nm.
%Theoretical calculations predict a lifetime of about $200$ s [], and the corresponding quality factor $10^{17}$ is one order of magnitude higher than $^1S_0$ $\rightarrow$ $^3P_0$, $^3P_2$.

\paragraph*{\label{Experimental setup}Experimental setup.}
All spectroscopic measurements other than the lifetime measurement are done using ultracold Yb atoms evaporatively cooled in a crossed far-off resonance trap (FORT).
%Yb atoms are cooled and captured in a Zeeman cooled magnetooptic trap and then evaporatively cooled in crossed FORT.
%Since $^{171} \mbox{Yb}$ and $^{176} \mbox{Yb}$ have small s-wave scattering lengths between ground states and %evaporation cooling is inefficient, they were co-cooled with $^{174} \mbox{Yb}$.
%Atomic number and temperature are measured by absorption imaging method using $^1S_0$ $\rightarrow$ $^1P_1$ transitions.
The number of atoms is about $2\times10^4 \sim 1.3\times10^5$ 
and the temperature is about $0.3 \sim \SI{1}{\mu \kelvin}$, depending on the isotope.
%The gray arrows represent magnetic fields, the magnitude and direction of which are adjusted according to each measurement.
%
We apply a linearly polarized excitation light beam at a wavelength of 431~nm.
%along the y-axis.
%, as indicated by the blue arrows.
We measure the number of atoms remaining in the ground state after the irradiation of the excitation laser, and identify the optical resonance through the resonant atom loss. 
%Since there are several excitation mechanisms (magnetic-quadrupole (M2), hyperfine-induced electric-dipole(E1), and magnetic-field-induced E1-allowed transitions), 
The angle of linear polarization, light intensity, and irradiation time are adjusted to optimize each spectroscopic measurement.
%
%The intensity and angle of linear polarization are adjusted in the same way as for the excitation light.
We note that the lifetime measurement is done for atoms loaded into a three-dimensional (3D) optical lattice.
%, although not shown in Fig.~\ref{FigS1}. 
See Sec.~S1 in Supplemental Material (SM) for the detail of the experimental system.

\begin{figure}
\includegraphics[width = 0.91\linewidth]{Fig2_spectroscopy_1col_wide.pdf}
\caption{\label{Excitation spectra}
Excitation spectra of all stable isotopes.
(a) Five bosonic isotopes.
The upper figure shows all spectra of the five isotopes for $m_J=0$ state, except for $^{174}$Yb, and the lower shows a magnified view around $^{174} \mbox{Yb}$.
The horizontal axis is the detuning from the resonance frequency of $^{174}$Yb ($m_J=0$).
(b, c) Two fermionic isotopes, (b) $^{171}$Yb($I=1/2$) and (c) $^{173}$Yb($I=5/2$).
The top figures show all spectra of all hyperfine states,
and the lower show $m_{F}$-resolved spectra of each hyperfine state. 
The horizontal axis is the detuning from the average of $m_F=\pm 1/2$ in $F=3/2$ for $^{171}$Yb and $F=1/2$ for $^{173}$Yb.
The solid curves are the fits with a Gaussian function.
Note that not all the magnetic sublevels for $F=5/2(9/2)$ state of $^{171}$Yb($^{173}$Yb) are shown.}
\end{figure} 
%%%%%%%%%%%% Excitation spectra %%%%%%%%%%%%
%Introduction, we report on resonance search and spectroscopy of all stable isotopes.
%There are seven total stable isotopes of the ytterbium atom.
%Bose isotopes are $^{168} \mbox{Yb}$, $^{170} \mbox{Yb}$, $^{172} \mbox{Yb}$, $^{174} \mbox{Yb}$ and $^{176} \mbox{Yb}$ with the nuclear spin $I$ zero.
%The Fermi isotopes are $^{171} \mbox{Yb}$ and $^{173} \mbox{Yb}$, and the nuclear spins $I$ are $1/2$ and $5/2$, respectively.
%Since the total electron angular momentum of the excited state is $J'=2$, the finite nuclear spin $I$ gives rise to the hyperfine structure $F'\: (|I-2| \leq F'\leq |I+2|)$.

\paragraph*{\label{Zeeman and hyperfine spectra.}Zeeman and hyperfine spectra.}
Figure~\ref{Excitation spectra} summarizes the isotope shifts, Zeeman, and hyperfine spectra of the five bosonic isotopes ($^{168} \mbox{Yb}, ^{170} \mbox{Yb}, ^{172} \mbox{Yb}, ^{174} \mbox{Yb}$, and $^{176} \mbox{Yb}$ with nuclear spin $I=0$) and the two fermionic isotopes ($^{171} \mbox{Yb}$ and $^{173} \mbox{Yb}$, with $I=1/2$ and $5/2$, respectively).
The upper panel of Fig.~\ref{Excitation spectra}(a) shows the spectra of the five bosonic isotopes, and the lower panel shows $m_J$-resolved spectra of $^{174}$Yb.
%Fig.~\ref{Fig2}(a) is the spectrum of all stable Bose isotopes.
% where the normalized number of atoms is plotted as a function of the detuning from $^{174} \mbox{Yb}$ with $m_{J}=0$. 
% the solid red line is the fitting by Gaussian function.
%A magnetic field of 0.15~mT is applied along the $(\vec{e_x}+\vec{e_y})/\sqrt{2}$ direction and the polarization of the excited light is tilted by $66 ^{\circ}$ from $\vec{e_z}$ to $\vec{e_x}$,
%According to the selection rule for M2 transitions, this experimental configuration  $\Delta m_J = 0, \pm1 , \pm2$
%allowing the excitation to all magnetic sublevels of the $J=2$ state, according to the selection rule for a magnetic quadrapole (M2) transition.
% A magnetic field of 0.136~mT defines a quantization axis, and the propagation and polarization angles of the excitation light with respect to the quantization axis are 45$^\circ$ and 24$^\circ$, respectively. 
The angle between the propagation (polarization) direction of the excitation light and the quantization axis, defined by a magnetic field, is 45$^\circ$(24$^\circ$).
This configuration allows the excitation to all magnetic sublevels of the $4f^{13}5d6s^2 \: (J = 2)$ state, according to the selection rule for a magnetic quadrupole (M2) transition.
%
%As the figure shows, a resonance search for five isotopes and a magnetic sublevel $m_{J'}$-resolved spectrum were successfully obtained~\cite{NIST2022}.
The typical full width at half maximum (FWHM) is about 30~kHz, limited by the Doppler broadening of about 20~kHz and the finite laser linewidth of about 10~kHz. 
%which is enough to obtain a magnetic sublevel $m_{J'}$-resolved spectrum.
%The minimum point of the spectrum is less than 0.5, which suggests the presence of a small trapping potential felt by the excited state and a loss mechanism due to collisions between atoms.
Note that more than half of the atoms are lost at the peak, suggesting some atom loss mechanism in the $4f^{13}5d6s^2 \: (J = 2)$ state, possibly caused by a shallow trapping potential and inelastic atom collisions.

%%%% isotope shift  %%%%%%%%%%%%%
Successful observation of the spectra of all bosonic isotopes enables us to determine the isotope shifts of the $^1S_0$ $\leftrightarrow$ $4f^{13}5d6s^2 \: (J = 2)$ transition.
See Sec.~S2 in SM for the detail of the measurement procedure.
The determined isotope shifts are summarized in Table~\ref{Isotope shift}, where only the statistical errors are evaluated.
Note that many systematic effects, such as the light shift and the quadratic Zeeman shift, are common among the isotopes within the accuracy of the present measurement.
In order to evaluate a possible line shift due to $s$-wave collisions between trapped ultracold atoms, which would be the largest systematic effect among many, we perform interleaved measurements with two different numbers of atoms by adjusting the evaporative cooling process. 
For the five bosonic isotopes, the resonance shifts, corrected for the difference in the atomic temperature, are within the uncertainties of about 5~kHz, and we find no evidence of atomic collision shifts at our measurement conditions.

\begingroup
\setlength{\tabcolsep}{10pt} % Default value: 6pt
\renewcommand{\arraystretch}{1.5} % Default value: 1
\begin{table}[tb]
\centering
\caption{\label{Isotope shift}%
Measured isotope shifts $\nu^{A'A}:=\nu^{A'}-\nu^{A}$ of the $4f^{14}6s^2 \: ^1S_0 \leftrightarrow 4f^{13}5d6s^2 \: (J = 2)$ transition, where $\nu^{A}$ is the transition frequency of an Yb isotope with a mass number $A$. Statistical 1$\sigma$ uncertainties are shown as $(\cdot)_\text{stat}$.}

\begin{tabular}{cr}\hline\hline
Isotope pair $(A',A)$ & Isotope shift $\nu^{A'A}$ $(\mathrm{MHz})$ \\
\hline
% w/ correction
% $(168,174)$ & $-4564.595 \: (2)_{\text{stat}} $ \\ 
% $(170,174)$ & $-2810.665 \: (2)_{\text{stat}} $ \\
% $(172,174)$ & $-1180.610 \: (2)_{\text{stat}} $ \\
% $(174,176)$ & $ 1115.768 \: (6)_{\text{stat}} $ \\
% w/o correction
$(168,174)$ & $-4564.596(2)_{\text{stat}} $ \\ 
$(170,174)$ & $-2810.666(2)_{\text{stat}} $ \\
$(172,174)$ & $-1180.614(2)_{\text{stat}} $ \\
$(174,176)$ & $ -1115.766(6)_{\text{stat}} $ \\
\hline\hline
\end{tabular}
\end{table}
\endgroup

%%%%  g-factor  %%%%%%%%%%%%%%%%%
In addition, we precisely determine the $g$ factor of the $4f^{13}5d6s^2 \: (J = 2)$ state as $g_J = 1.463(4)$ with the uncertainty determined by the propagation of the fitting error and the uncertainty of the magnetic field calibration.
%Interleaved measurements are made to measure the Zeeman shift differences between four field strengths of 0.291, 0.582, 0.873, and 1.16~mT, <- Moved to SM
%at field differences of 2.91, 5.82, and 8.73 G.
%which are calibrated by the spectroscopy with the $^1S_0$ $\leftrightarrow$ $^3P_1$ transition.
%A linear fit to the magnetic field difference and the Zeeman shift difference, with the slope divided by the Bohr magneton $\mu_B$ and $|m_{J'}|$, gives the excited state g factor $g_{J'}$.
%Note that the former, with a magnitude of about 0.3\%, is dominant, and the latter hardly comes into play.
Our theoretical calculation of the $g$ factor using the DIRAC program~\cite{saue2020dirac,DIRAC22} gives $g_J = 1.465(2)$, which is consistent with the experiment.
See Sec.~S3 in SM for the detail of the measurement procedure and Sec.~S7 for the theoretical calculations.

%%%%  Hyperfine  %%%%%%%%%%%%%%%%%
Figures~\ref{Excitation spectra}(b) and (c) shows the hyperfine (top panel) and Zeeman spectra (lower panels) of $^{171} \mbox{Yb}$ and  $^{173} \mbox{Yb}$, respectively.
The observed Zeeman spectra for fermionic isotopes are well explained by the measured $g_J$.
Here, we identify an excitation mechanism other than the M2 transition.
%$4f^{14}6s^2 \: ^1S_0$ $\rightarrow$ $4f^{13}5d6s^2 (J = 2)$ transitions are forbidden for E1, M1, and E2 transitions and allowed for M2 transitions, as mentioned above. However, hyperfine interactions, i.e., 
A hyperfine interaction can cause the $4f^{13}5d6s^2 (J = 2)$ state to mix with other states such as $^3P_1$ and $4f^{13}5d6s^2 (J=1)$ states, making the electric-dipole (E1) transition partially allowed,
which is called a hyperfine-induced E1 transition~\cite{Boyd2007}.
%For example, in the $^1S_0 \rightarrow ^3P_0$ and $^3P_2$ transitions of alkaline earth metals, E1 transitions are known to occur when the excited states mix with $^3P_1$ and $^1P_1$ states.
%In the spectroscopy of  isotopes, we observed E1 transitions by this mechanism.
%The states in which the excited state $\ket{e}$ mixes are not known, but $^3P_1$ and $4f^{13}5d6s^2 (J=1)$ are possible.
%Note that the the selection rule of the E1 transition $\Delta F = 0, \pm1$ for the total angular momentum $F$.
%In the present case, this corresponds to $|I-1| \leq F'\leq |I+1|$.
The transition to the $F=3/2$ state of $^{171} \mbox{Yb}$  and those to the $F=3/2$, 5/2, and $7/2$ of $^{173} \mbox{Yb}$ are E1-allowed, and give resonance signals much stronger than other transitions that are allowed only by M2.
%The upper panel shows the spectrum of the hyperfine structure $F'$, and the lower panel shows a magnified view around each F' resonance with resolved Zeeman sublevels.
%(b) and (c) show the abscissa of $F'=3/2 (m_{F'}=\pm1/2)$ and $F'=1/2 (m_{F'}=\pm1/2)$, respectively.
%As the figure shows, we succeeded in obtaining a resonance search for the total hyperfine structure of the two isotopes and a spectrum with resolved magnetic sublevels.

%%%%%% Hyperfine constants %%%%%%%%%%%%%%%%%
The data shown in Figs.~\ref{Excitation spectra}(b) and (c) enable us to determine the hyperfine constants of $^{171} \mbox{Yb}$ and $^{173} \mbox{Yb}$ of the $4f^{13}5d6s^2 \: (J = 2)$ state, respectively.
We obtain the magnetic dipole constant $A_\text{hfs}(^{171} \mbox{Yb})=1123.3(3)$~MHz, and $A_\text{hfs}(^{173}\mbox{Yb})=-309.46(4)$~MHz, and the electric quadrupole constant $B_\text{hfs}(^{173} \mbox{Yb})= -1700.6(9)$~MHz, with the uncertainties due to the fitting error. Note that the presumably much smaller systematic effects such as the collision shift are not evaluated at present. The careful evaluation of the systematic effects and precise determination of the hyperfine constants will be an important future work.
It is also noted that introducing a magnetic octupole constant $C_\text{hfs}$ for $^{173} \mbox{Yb}$ does not improve the fitting. 
See Sec.~S4 in SM for the detail.

%The E1 transition is induced not only by the internal magnetic field due to each spin, but also by
%An external magnetic field also induces an E1 transition,
%which is called a magnetic-field-induced-E1 transition~\cite{Taichenachev2006}.
%Using this method, E1 transitions can also be induced for Bose isotopes.
%Note that, in order to obtain a reasonably large signal for each resonance shown in Fig.~\ref{Fig2}, we optimize the intensity and duration of the excitation laser as well as the relative orientation between the quantization axis and polarization of the excitation light, considering the different selection rules of M2, hyperfine-induced E1, and magnetic field-induced E1 transitions~\cite{Taichenachev2006}. 
%the best of the three excitation mechanisms (M2, hyperfine-induced E1, and magnetic field-induced E1 transitions) for each measurement.
%The quantization axis and polarization of the excitation light are also adjusted to change the selection rule depending on whether M2 or E1 is used.

\begin{figure}
\includegraphics[width = 0.95\linewidth]{Fig3_20230123_polarizability_1col.pdf}
\caption{\label{polarizability measurement}
Differential polarizability measurement between $^1S_0$ and $4f^{13}5d6s^2 \: (J = 2)$ states.
Measurements around the transitions: (a) $4f^{13}5d6s^2 \: (J = 2)$ $\leftrightarrow$  $4f^{13} 6s^2 6p_{3/2}\: (J=3)$ at $792.5$~nm, and (b) $4f^{13}5d6s^2 \: (J = 2)$ $\leftrightarrow$ $4f^{13} 6s^2 6p_{3/2} \:(J=2)$ at $833.7$~nm.
The differential polarizability, shown on the vertical axis as arbitrary units, 
is obtained from the linear fit to the measured differential light shifts $\Delta \nu$ as a function of the Ti:sapphire laser intensity.
The error bars on the vertical axis are obtained from the error propagation of 
the standard error of $\Delta \nu$ and the uncertainty of the Ti:sapphire laser intensity.
The solid curves are fits to the data.
}
\end{figure}

%%%%%%%%%%%% Differential polarizability %%%%%%%%%%%%
%\indent
%{\it Differential polarizability.{\bf \textemdash}}
\paragraph{\label{Differential polarizability.}Differential polarizability.}
%From here, we will investigate various properties of excited states and clock transitions.
We perform a measurement of the wavelength-dependent polarizability of the $4f^{13}5d6s^2 \: (J = 2)$ state.
%over the wavelength range from 750 nm to 860 nm.
%between the $\ket{g}$ and $\ket{e}$.
This is very important in the search for a magic wavelength at which the polarizabilities of the ground and  $4f^{13}5d6s^2 \: (J = 2)$ states coincide, and thus the line shift due to the trapping light is largely suppressed,
%Trapping atoms in an optical grating consisting of laser light at the magic wavelength suppresses fluctuations caused by the trapped light,
enabling ultra-narrow-linewidth spectroscopy and accurate clock operation~\cite{Katori2003}.
See Sec.~S5 in SM for the details of the experimental procedure.

In the case of linear polarization, the total polarizability $\alpha $ of the state $|J, m_{J}\rangle$ is
\begin{equation}
\alpha = \alpha^S  + \frac{3\mbox{cos}^2\theta -1}{2} \frac{3m_J^2 - J(J+1)}{J(2J-1)} \alpha^T,
\label{polarizability}
\end{equation}
where the superscripts $S$ and $T$ denote scalar and tensor, respectively, and $\theta$ is the polarization angle with respect to the quantization axis~\cite{LeKien2013}.
% $\theta = \arccos(\hat{\epsilon} \cdot \hat{B})$.
% Unit vectors of $\hat{B}$ and $\hat{\epsilon}$ denote the quantization axis defined by the applied magnetic field and polarization vector, respectively.
%$A$ is a constant that depends on the polarization vector of light, and is 0 for linear polarization and $\pm1$ for circular polarization.
Using this formula, the differential polarizability $\Delta \alpha $ between the ground state with $J=0$ and the $4f^{13}5d6s^2 \: (J = 2)$ state is given in a straightforward manner.
%\begin{eqnarray}
%    \Delta \alpha (\lambda) = \Delta \alpha^S (\lambda) + \alpha ^T (\lambda) \frac{3 \mbox{cos}^2\theta-1}{2} \frac{m_{J'}^2-2}{2}
%\end{eqnarray}
%, where $\theta$ denotes $\hat{\epsilon} \cdot \hat{B}$.
% Note that the polarizability difference $\Delta \alpha(\lambda)$ is represented by two quantities $\Delta \alpha^S (\lambda)$ and $\alpha^T (\lambda)$.

%Before making the measurement, we estimated the magic wavelength by theoretical calculation.
In this work, we search for a magic wavelength within the tunable range of a titanium(Ti):sapphire laser which can provide enough power for optical lattices.
In particular, in this wavelength range, there are two E1-allowed optical transitions of $4f^{13}5d6s^2 \: (J = 2)$ $\leftrightarrow$  $4f^{13} 6s^2 6p_{3/2} \:(J=3)$ at a wavelength of 792.5~nm and $4f^{13}5d6s^2 \: (J = 2)$ $\leftrightarrow$ $4f^{13} 6s^2 6p_{3/2}\: (J=2)$ at 832.7~nm~\cite{NIST2022}, which will resonantly change the polarizability.
In fact, according to our calculation for the excited state and Ref.~\cite{Tang2018} for the ground state, the magic wavelengths are expected to be around 795.5~nm and 833.7~nm for $m_J=0$ and $\theta=90^\circ$.
See Sec.~S7 in SM for the theoretical calculation of the transition dipole moments for the two E1-allowed transitions.

%See Appendix for details of the theoretical calculations.
%and are within the tunable range of our Ti:Sapphire laser.

%We measured the polarizability difference according to previous studies~\cite{Golovizin2019}.

\begin{comment} %move to SM
%The measurement technique is as follows.
In our measurement, we apply a high magnetic field of 14.6~mT to $^{174} \mbox{Yb}$ in a crossed FORT and irradiate $\pi$-polarized excitation light at 431~nm to induce a magnetic-field-induced E1 transition to the $m_J = 0$ state in the $4f^{13}5d6s^2 \: (J = 2)$.
%A Ti:sapphire laser beam is irradiated along the x-axis direction.
%For polarizability measurements, we additionally apply a linearly polarized Ti:sapphire laser along the x-axis.
The wavelength is tunable between 750~nm and 860~nm, with a maximum intensity of 8~kW/cm$^2$ at the atom position.
%The wavelength is $\lambda$ and the angle between the quantization axis $\vec{e_z}$ and the polarization is $\theta$.
By interleave measurements with three light intensities $0, I_0, 2I_0$ of the Ti:sapphire laser, we determine the differential light shift $\Delta \nu$ with respect to the light intensity difference.
This interleave measurement minimizes systematic effects such as the drifts of the excitation light frequency and Ti:sapphire light intensity.
From a linear fit of the differential light shift $\Delta \nu$ with a light intensity difference, we obtain the differential polarizability $\Delta \alpha$ as its slope.
%Here, we perform two measurements with different $\theta$ or $m_{J'}$ at each wavelength to obtain the scalar and tensor polarizabiliies $\Delta \alpha ^S (\lambda)$ and $\alpha ^T (\lambda)$.
%As can be seen from Eq[], by measuring the polarizability difference $\Delta \alpha (\lambda)$ under two conditions with different $\theta$ or $m_{J'}$, two parameters $\Delta \alpha ^S (\lambda)$, $\Delta \alpha ^T (\lambda)$ can be obtained.
\end{comment}

As shown in Fig.~\ref{polarizability measurement}(a) and (b), we observe the resonant changes of the polarizability in the vicinity of the transition wavelengths of 792.5~nm and 832.7~nm. The solid curves are the fits to the data using the following equation: $\Delta \alpha (\omega) = a/(\omega_0^2-\omega^2)  + b$,
% Figures~\ref{polarizability measurement}(a) and (b) show the measurement results around the wavelengths of 798~nm and 833~nm, respectively.
%The error bars on the vertical axis are evaluated as the standard error of $\Delta \nu$ and the error propagation of the Ti:sapphire light intensity uncertainty.
% Each point corresponds to the measured data and the solid curves are the fits to the data using the following equation:
% \begin{equation}
%     \Delta \alpha (\omega) = a \left(\frac{1}{\omega_0-\omega} + \frac{1}{\omega_0+\omega}\right) + b,
%     \label{fit}
% \end{equation}
where $a$ and $b$ are fitting parameters, $\omega$ is the angular frequency of a Ti:sapphire laser, and $\omega_0$ is the resonance angular frequency of the $4f^{13}5d6s^2 \: (J = 2)$ $\leftrightarrow$ $4f^{13} 6s^2 6p_{3/2}\; (J=3)$ or $4f^{13}5d6s^2 \: (J = 2)$ $\leftrightarrow$ $4f^{13} 6s^2 6p_{3/2}\; (J=2)$ transition, given in Ref.~\cite{NIST2022}.
%$A$ is a constant corresponding to the transition strength at the resonance, and $B$ is an offset constant for the polarization rate difference.
Note that this model assumes that only one resonant transition with $\omega_0$ dominantly contributes to the wavelength dependence, while the effects of other off-resonant transitions contribute as wavelength-independent offset constant $b$.
From the fitting, we can determine the magic wavelengths $\lambda_\text{magic}$ as follows:
% \begin{equation}
%     \lambda_\text{magic}=
%     \begin{cases}
%     \SI{798.9(4)}{nm} & (m_J=0, \;\theta=0^{\circ}),\\
%     \SI{797.2(4)}{nm} & \multirow{2}{*}{$(m_J=0, \;\theta=90^{\circ})$,}\\
%     \SI{834.2(1)}{nm} & \\
%     \SI{833.28(4)}{nm} & (m_J=2, \;\theta=90^{\circ}).
%     \end{cases}
%     \label{magic wavelength}
% \end{equation}
\begin{equation}
    \lambda_\text{magic}=
    \begin{cases}
    \SI{798.9(4)}{nm} & (m_J=0, \;\theta=0^{\circ}),\\
    \begin{rcases*}
    \SI{797.2(4)}{nm}\\ \SI{834.2(1)}{nm}
    \end{rcases*}
    &(m_J=0, \;\theta=90^{\circ}),\\
    \SI{833.28(4)}{nm} & (m_J=2, \;\theta=90^{\circ}).
    \end{cases}
    \label{magic wavelength}
\end{equation}
Since two measurements with different $\theta$ or $m_J$ at each wavelength are performed, we can determine the differential scalar polarizability $\Delta \alpha ^S$ and the tensor polarizability $\alpha ^T $ from Eq.~\eqref{polarizability}.
% See Fig. S3 in SM for the scalar and tensor polarizability difference.
%Figure\ref{Fig3}(c) and (d) show the scalar and tensor parts $\Delta \alpha ^S (\lambda)$ and $\alpha ^T (\lambda)$.
%The solid lines are $\Delta \alpha ^S (\lambda)$ and $\Delta \alpha ^T (\lambda)$ obtained as a linear combination of the fitting curves in Fig.\ref{Fig3}(a) and (b) using Eq[].
%For reference, the linear combination of the data points in Fig3(a) and (b) is also shown.
%There are two advantages of obtaining $\Delta \alpha ^S (\lambda)$ and $\Delta \alpha ^T (\lambda)$.
%First, when the vector polarizability vanishes, the magic wavelength for arbitrary condition $(\theta, m_{J'})$ can be calculated.
%Note that this condition is satisfied when the laser light is linearly polarized, as mentioned above.
%Second, we can estimate the effect of the tensor part on the magic wavelength.
Possible trap geometries to minimize the uncertainty of the tensor light shift on the performance of clock operation using the 797.2-nm optical lattice are discussed in Sec.~S5 in SM.
It should be noted that the differential higher-order light shift due to M1 and E2 multipolar polarizabilities and E1 hyperpolarizability~\cite{Porsev2018} should exist even at an E1 magic wavelength, where the differential light shift due to E1 polarizability vanishes~\cite{Katori2003}. The investigation of the operational magic condition for wavelength and intensity, where the overall light shift is insensitive to the lattice-intensity variation~\cite{Ushijima2018}, will be an important future work for clock operation.

%%%%%%%%%%%% Lifetime %%%%%%%%%%%%
%\indent
%{\it Lifetime.{\bf \textemdash}}
\paragraph{\label{Lifetime}Lifetime.}
%Next, we report on the lifetime measurement of the $4f^{13}5d6s^2 \: (J = 2)$ state.
The lifetime of the excited state of a clock transition is important because it is directly related to the quality factor of the clock transition.
In this measurement, we use spin-polarized $^{173}\mbox{Yb}$ atoms in the $\ket{F=5/2, m_F=5/2}$ state loaded into a 3D optical lattice to minimize the possible atom loss due to inelastic interatomic collisions.
%A 759 nm semiconductor laser was used as a light source for the $x$ and $z$ axes, and a 797.2 nm Ti:Sa laser was used for the $y$ axis, the excitation axis.
The 3D optical lattice consists of a 2D optical lattice at the wavelength $\lambda_1=759.4$~nm and a 1D optical lattice at the wavelength $\lambda_2=797.2$~nm, superimposed on the excitation light.
% The laser wavelength for forming optical lattices along the $x$ and $z$ axes is 759.x nm, while that for the $y$ axis is 797.2 nm.
\begin{comment} %moved to SM
Figure~\ref{lifetime}(a) shows the spectrum with resolved carrier and blue-sideband components.
%It can be seen that a double peak was obtained.
%The low frequency side is considered to be the carrier and the high frequency side is the blue sideband.
%The solid red line shows the fitting by the double-peak Lorenz function. 
The FWHM of the carrier is about 12~kHz, narrower than the 30~kHz linewidth for atoms in a crossed FORT and
% and the frequency difference between the carrier and blue sideband is about 18 kHz.
%Compared to the FWHM of 30 kHz in the crossed FORT, the FWHM has been narrowed to less than half.
%We believe that the current linewidth
is limited by the linewidth of the excitation laser and the effect of the differential light shift at the non-magic wavelength of the 759~nm optical lattice light.
%The magnetic field is $\vec{e_z}$ oriented, and all three axes have linear polarization perpendicular to the magnetic field.
%Note that 797.2 nm is the magic wavelength for $^{174} \mbox{Yb} (m_{J'} = 0, \theta=90^{\circ})$.
%In order to simultaneously evaluate photon scattering by the lattice light at the magic wavelength along with lifetime measurements of the excited states, we formed an optical lattice at 797.2 nm in only one axis.
%This is because our Ti:Sa laser is not powerful enough to form a 3D optical lattice.
\end{comment}
See Sec.~S6 in SM for the detail of the experimental sequence.
\begin{figure}
\includegraphics[width = 0.95\linewidth]{Fig4_20230118_lifetime.pdf}
\caption{\label{lifetime}
%(b) Pulse sequence for lifetime measurement.
Lifetime measurement of atoms in $4f^{13}5d6s^2 \: (J = 2)$ state.
% The optical lattice depth is set to $25 E_r$ in all three axes.
The horizontal axis shows the time to hold the excited atoms in the optical lattice, and the vertical axis shows the number of atoms in the excited state after the hold time.
The measurements are repeated four times, and the mean and standard deviation are plotted.
%The error bars on the vertical axis are the standard deviation of four repetitions.
%The solid green line is a single exponential fit and the corresponding lifetime is 1.9(1)~s.
}
\end{figure}
% The result of the measurement is shown in Fig.\ref{Fig4}(b).
Figure.~\ref{lifetime} shows the number of excited atoms as a function of the holding time in the optical lattice.
The lattice depths are 28$E_\text{R}$ for the 2D optical lattice at $\lambda_1$, and 25$E_\text{R}$ for the 1D optical lattice at $\lambda_2$. Here $E_\text{R} = h^2/(2m\lambda_2^2)=h\times1.8$~kHz represents the recoil energy at the wavelength $\lambda_2$, where $h$ is the Planck constant, and $m$ is the mass of an Yb atom.
%The solid line is a fit of the data with an exponential function, and the corresponding lifetime is 1.9(1)~s.
From the fit to the data with a single exponential function, the trap lifetime of the excited atoms is obtained as 1.9(1)~s.
As a reference, we also investigate the trap loss of atoms in the ground state $^1S_0$ in the same experimental setup, and find the trap lifetime to be 2.8(2)~s, which would be limited by some mechanism such as collisions with background gases.
Assuming that the same mechanism also limits the trap lifetime of the $4f^{13}5d6s^2 \: (J = 2)$ state, we infer the intrinsic state lifetime of the excited state to be 5.9(1.3)~s.
This is compared with the theoretical calculations in Refs.~\cite{Safronova2018}, \cite{Dzuba2018}, and \cite{Tang2022}, which predict a lifetime of about 60, 200, and 190~s, respectively.
Our preliminary estimation shows that the rate of the photon scattering $\gamma_\text{sc}$ due to the lattice laser beam of near resonant 797.2-nm light at 25$E_\text{R}$ is $\gamma_\text{sc}=0.15$~s$^{-1}$, which seemingly explains the measured result.
Further systematic measurements with various experimental conditions will clarify the intrinsic lifetime of this state.  



%%%%%%%%%%%% Summary %%%%%%%%%%%%
%\indent
%{\it Summary.{\bf \textemdash}}
\paragraph{\label{Summary}Summary.}
We report the observation of the $^1S_0$ $\leftrightarrow$ $4f^{13}5d6s^2 (J = 2)$ optical transition of all seven Yb isotopes with resolved Zeeman and hyperfine structures. 
Two magic wavelengths of 797.2(4)~nm and 834.2(4)~nm for practical conditions are found and the measured lifetime is 1.9(1)~s, promising for applications for precision measurements. 
We also determine the $g$ factor to be $1.463(4)$, and compare the result with our relativistic many-body calculation with good agreement. 
%Furthermore, our calculation reveals that the observed transition has sensitivity for a new hypothetical particle, which is utilized by a King plot experiment. 
A 3D optical lattice experiment with many ultracold atoms with isotope mixtures for the observed $4f^{13}5d6s^2 \:(J = 2)$ state is promising
to test the violation of local Lorentz invariance of electron-photon sector and Einstein’s weak equivalence principle.

\begin{acknowledgments}
% This work was supported by the Grant-in-Aid for Scientific Research of JSPS (Nos.~JP17H06138, JP18H05405, JP18H05228, JP20K14383, JP21H01014, JP22K20356), JST CREST (No.~JP-MJCR1673), and MEXT Quantum Leap Flagship Program (MEXT Q-LEAP) Grant No.~JPMXS0118069021, and , JST Moonshot R\&D-MILLENNIA Program (Grant No.~JPMJMS2269). K.O. was supported by Graduate School of Science, Kyoto University under Ginpu Fund. A.S. acknowledges support from the JSPS KAKENHI Grant No.~21K14643. In this research work, A.S. used the computer resource offered under the category of General Projects by the Research Institute for Information Technology, Kyushu University.
This work was supported by the Grant-in-Aid for Scientific Research of JSPS (No. JP17H06138, No. JP18H05405, No. JP18H05228, No. JP22K20356), JST CREST (No. JP-MJCR1673), and MEXT Quantum Leap Flagship Program (MEXT Q-LEAP) Grant No. JPMXS0118069021, and JST Moonshot R\&D Grant No. JPMJMS2269. K. O. was supported by Graduate School of Science, Kyoto University under Ginpu Fund. A. S. acknowledges support from the JSPS KAKENHI Grant No. 21K14643. In this research work, A. S. used the computer resource offered under the category of General Projects by the Research Institute for Information Technology, Kyushu University.
\end{acknowledgments}

%%%%%%%%%% bibtex %%%%%%%%%%
%apsrev4-2.bst 2019-01-14 (MD) hand-edited version of apsrev4-1.bst
%Control: key (0)
%Control: author (8) initials jnrlst
%Control: editor formatted (1) identically to author
%Control: production of article title (0) allowed
%Control: page (0) single
%Control: year (1) truncated
%Control: production of eprint (0) enabled
\providecommand{\noopsort}[1]{}\providecommand{\singleletter}[1]{#1}%
\begin{thebibliography}{35}%
\makeatletter
\providecommand \@ifxundefined [1]{%
 \@ifx{#1\undefined}
}%
\providecommand \@ifnum [1]{%
 \ifnum #1\expandafter \@firstoftwo
 \else \expandafter \@secondoftwo
 \fi
}%
\providecommand \@ifx [1]{%
 \ifx #1\expandafter \@firstoftwo
 \else \expandafter \@secondoftwo
 \fi
}%
\providecommand \natexlab [1]{#1}%
\providecommand \enquote  [1]{``#1''}%
\providecommand \bibnamefont  [1]{#1}%
\providecommand \bibfnamefont [1]{#1}%
\providecommand \citenamefont [1]{#1}%
\providecommand \href@noop [0]{\@secondoftwo}%
\providecommand \href [0]{\begingroup \@sanitize@url \@href}%
\providecommand \@href[1]{\@@startlink{#1}\@@href}%
\providecommand \@@href[1]{\endgroup#1\@@endlink}%
\providecommand \@sanitize@url [0]{\catcode `\\12\catcode `\$12\catcode
  `\&12\catcode `\#12\catcode `\^12\catcode `\_12\catcode `\%12\relax}%
\providecommand \@@startlink[1]{}%
\providecommand \@@endlink[0]{}%
\providecommand \url  [0]{\begingroup\@sanitize@url \@url }%
\providecommand \@url [1]{\endgroup\@href {#1}{\urlprefix }}%
\providecommand \urlprefix  [0]{URL }%
\providecommand \Eprint [0]{\href }%
\providecommand \doibase [0]{https://doi.org/}%
\providecommand \selectlanguage [0]{\@gobble}%
\providecommand \bibinfo  [0]{\@secondoftwo}%
\providecommand \bibfield  [0]{\@secondoftwo}%
\providecommand \translation [1]{[#1]}%
\providecommand \BibitemOpen [0]{}%
\providecommand \bibitemStop [0]{}%
\providecommand \bibitemNoStop [0]{.\EOS\space}%
\providecommand \EOS [0]{\spacefactor3000\relax}%
\providecommand \BibitemShut  [1]{\csname bibitem#1\endcsname}%
\let\auto@bib@innerbib\@empty
%</preamble>
\bibitem [{\citenamefont {Chou}\ \emph {et~al.}(2010)\citenamefont {Chou},
  \citenamefont {Hume}, \citenamefont {Koelemeij}, \citenamefont {Wineland},\
  and\ \citenamefont {Rosenband}}]{Chou2010}%
  \BibitemOpen
  \bibfield  {author} {\bibinfo {author} {\bibfnamefont {C.~W.}\ \bibnamefont
  {Chou}}, \bibinfo {author} {\bibfnamefont {D.~B.}\ \bibnamefont {Hume}},
  \bibinfo {author} {\bibfnamefont {J.~C.~J.}\ \bibnamefont {Koelemeij}},
  \bibinfo {author} {\bibfnamefont {D.~J.}\ \bibnamefont {Wineland}},\ and\
  \bibinfo {author} {\bibfnamefont {T.}~\bibnamefont {Rosenband}},\ }\bibfield
  {title} {\bibinfo {title} {{Frequency Comparison of Two High-Accuracy
  ${\mathrm{Al}}^{+}$ Optical Clocks}},\ }\href
  {https://doi.org/10.1103/PhysRevLett.104.070802} {\bibfield  {journal}
  {\bibinfo  {journal} {Phys. Rev. Lett.}\ }\textbf {\bibinfo {volume} {104}},\
  \bibinfo {pages} {070802} (\bibinfo {year} {2010})}\BibitemShut {NoStop}%
\bibitem [{\citenamefont {Bloom}\ \emph {et~al.}(2014)\citenamefont {Bloom},
  \citenamefont {Nicholson}, \citenamefont {Williams}, \citenamefont
  {Campbell}, \citenamefont {Bishof}, \citenamefont {Zhang}, \citenamefont
  {Zhang}, \citenamefont {Bromley},\ and\ \citenamefont {Ye}}]{Bloom2014}%
  \BibitemOpen
  \bibfield  {author} {\bibinfo {author} {\bibfnamefont {B.~J.}\ \bibnamefont
  {Bloom}}, \bibinfo {author} {\bibfnamefont {T.~L.}\ \bibnamefont
  {Nicholson}}, \bibinfo {author} {\bibfnamefont {J.~R.}\ \bibnamefont
  {Williams}}, \bibinfo {author} {\bibfnamefont {S.~L.}\ \bibnamefont
  {Campbell}}, \bibinfo {author} {\bibfnamefont {M.}~\bibnamefont {Bishof}},
  \bibinfo {author} {\bibfnamefont {X.}~\bibnamefont {Zhang}}, \bibinfo
  {author} {\bibfnamefont {W.}~\bibnamefont {Zhang}}, \bibinfo {author}
  {\bibfnamefont {S.~L.}\ \bibnamefont {Bromley}},\ and\ \bibinfo {author}
  {\bibfnamefont {J.}~\bibnamefont {Ye}},\ }\bibfield  {title} {\bibinfo
  {title} {An optical lattice clock with accuracy and stability at the
  $10^{-18}$ level},\ }\href {https://doi.org/10.1038/nature12941} {\bibfield
  {journal} {\bibinfo  {journal} {Nature}\ }\textbf {\bibinfo {volume} {506}},\
  \bibinfo {pages} {71} (\bibinfo {year} {2014})}\BibitemShut {NoStop}%
\bibitem [{\citenamefont {Ushijima}\ \emph {et~al.}(2015)\citenamefont
  {Ushijima}, \citenamefont {Takamoto}, \citenamefont {Das}, \citenamefont
  {Ohkubo},\ and\ \citenamefont {Katori}}]{Ushijima2015}%
  \BibitemOpen
  \bibfield  {author} {\bibinfo {author} {\bibfnamefont {I.}~\bibnamefont
  {Ushijima}}, \bibinfo {author} {\bibfnamefont {M.}~\bibnamefont {Takamoto}},
  \bibinfo {author} {\bibfnamefont {M.}~\bibnamefont {Das}}, \bibinfo {author}
  {\bibfnamefont {T.}~\bibnamefont {Ohkubo}},\ and\ \bibinfo {author}
  {\bibfnamefont {H.}~\bibnamefont {Katori}},\ }\bibfield  {title} {\bibinfo
  {title} {Cryogenic optical lattice clocks},\ }\href
  {https://doi.org/10.1038/nphoton.2015.5} {\bibfield  {journal} {\bibinfo
  {journal} {Nature Photonics}\ }\textbf {\bibinfo {volume} {9}},\ \bibinfo
  {pages} {185} (\bibinfo {year} {2015})}\BibitemShut {NoStop}%
\bibitem [{\citenamefont {Grebing}\ \emph {et~al.}(2016)\citenamefont
  {Grebing}, \citenamefont {Al-Masoudi}, \citenamefont {D\"{o}rscher},
  \citenamefont {H\"{a}fner}, \citenamefont {Gerginov}, \citenamefont {Weyers},
  \citenamefont {Lipphardt}, \citenamefont {Riehle}, \citenamefont {Sterr},\
  and\ \citenamefont {Lisdat}}]{Grebing2016}%
  \BibitemOpen
  \bibfield  {author} {\bibinfo {author} {\bibfnamefont {C.}~\bibnamefont
  {Grebing}}, \bibinfo {author} {\bibfnamefont {A.}~\bibnamefont {Al-Masoudi}},
  \bibinfo {author} {\bibfnamefont {S.}~\bibnamefont {D\"{o}rscher}}, \bibinfo
  {author} {\bibfnamefont {S.}~\bibnamefont {H\"{a}fner}}, \bibinfo {author}
  {\bibfnamefont {V.}~\bibnamefont {Gerginov}}, \bibinfo {author}
  {\bibfnamefont {S.}~\bibnamefont {Weyers}}, \bibinfo {author} {\bibfnamefont
  {B.}~\bibnamefont {Lipphardt}}, \bibinfo {author} {\bibfnamefont
  {F.}~\bibnamefont {Riehle}}, \bibinfo {author} {\bibfnamefont
  {U.}~\bibnamefont {Sterr}},\ and\ \bibinfo {author} {\bibfnamefont
  {C.}~\bibnamefont {Lisdat}},\ }\bibfield  {title} {\bibinfo {title}
  {Realization of a timescale with an accurate optical lattice clock},\ }\href
  {https://doi.org/10.1364/OPTICA.3.000563} {\bibfield  {journal} {\bibinfo
  {journal} {Optica}\ }\textbf {\bibinfo {volume} {3}},\ \bibinfo {pages} {563}
  (\bibinfo {year} {2016})}\BibitemShut {NoStop}%
\bibitem [{\citenamefont {Milner}\ \emph {et~al.}(2019)\citenamefont {Milner},
  \citenamefont {Robinson}, \citenamefont {Kennedy}, \citenamefont {Bothwell},
  \citenamefont {Kedar}, \citenamefont {Matei}, \citenamefont {Legero},
  \citenamefont {Sterr}, \citenamefont {Riehle}, \citenamefont {Leopardi} \emph
  {et~al.}}]{Milner2019}%
  \BibitemOpen
  \bibfield  {author} {\bibinfo {author} {\bibfnamefont {W.~R.}\ \bibnamefont
  {Milner}}, \bibinfo {author} {\bibfnamefont {J.~M.}\ \bibnamefont
  {Robinson}}, \bibinfo {author} {\bibfnamefont {C.~J.}\ \bibnamefont
  {Kennedy}}, \bibinfo {author} {\bibfnamefont {T.}~\bibnamefont {Bothwell}},
  \bibinfo {author} {\bibfnamefont {D.}~\bibnamefont {Kedar}}, \bibinfo
  {author} {\bibfnamefont {D.~G.}\ \bibnamefont {Matei}}, \bibinfo {author}
  {\bibfnamefont {T.}~\bibnamefont {Legero}}, \bibinfo {author} {\bibfnamefont
  {U.}~\bibnamefont {Sterr}}, \bibinfo {author} {\bibfnamefont
  {F.}~\bibnamefont {Riehle}}, \bibinfo {author} {\bibfnamefont
  {H.}~\bibnamefont {Leopardi}}, \emph {et~al.},\ }\bibfield  {title} {\bibinfo
  {title} {{Demonstration of a Timescale Based on a Stable Optical Carrier}},\
  }\href {https://doi.org/10.1103/PhysRevLett.123.173201} {\bibfield  {journal}
  {\bibinfo  {journal} {Phys. Rev. Lett.}\ }\textbf {\bibinfo {volume} {123}},\
  \bibinfo {pages} {173201} (\bibinfo {year} {2019})}\BibitemShut {NoStop}%
\bibitem [{\citenamefont {Takano}\ \emph {et~al.}(2016)\citenamefont {Takano},
  \citenamefont {Takamoto}, \citenamefont {Ushijima}, \citenamefont {Ohmae},
  \citenamefont {Akatsuka}, \citenamefont {Yamaguchi}, \citenamefont
  {Kuroishi}, \citenamefont {Munekane}, \citenamefont {Miyahara},\ and\
  \citenamefont {Katori}}]{Takano2016}%
  \BibitemOpen
  \bibfield  {author} {\bibinfo {author} {\bibfnamefont {T.}~\bibnamefont
  {Takano}}, \bibinfo {author} {\bibfnamefont {M.}~\bibnamefont {Takamoto}},
  \bibinfo {author} {\bibfnamefont {I.}~\bibnamefont {Ushijima}}, \bibinfo
  {author} {\bibfnamefont {N.}~\bibnamefont {Ohmae}}, \bibinfo {author}
  {\bibfnamefont {T.}~\bibnamefont {Akatsuka}}, \bibinfo {author}
  {\bibfnamefont {A.}~\bibnamefont {Yamaguchi}}, \bibinfo {author}
  {\bibfnamefont {Y.}~\bibnamefont {Kuroishi}}, \bibinfo {author}
  {\bibfnamefont {H.}~\bibnamefont {Munekane}}, \bibinfo {author}
  {\bibfnamefont {B.}~\bibnamefont {Miyahara}},\ and\ \bibinfo {author}
  {\bibfnamefont {H.}~\bibnamefont {Katori}},\ }\bibfield  {title} {\bibinfo
  {title} {Geopotential measurements with synchronously linked optical lattice
  clocks},\ }\href {https://doi.org/10.1038/nphoton.2016.159} {\bibfield
  {journal} {\bibinfo  {journal} {Nature Photonics}\ }\textbf {\bibinfo
  {volume} {10}},\ \bibinfo {pages} {662} (\bibinfo {year} {2016})}\BibitemShut
  {NoStop}%
\bibitem [{\citenamefont {Lisdat}\ \emph {et~al.}(2016)\citenamefont {Lisdat},
  \citenamefont {Grosche}, \citenamefont {Quintin}, \citenamefont {Shi},
  \citenamefont {Raupach}, \citenamefont {Grebing}, \citenamefont {Nicolodi},
  \citenamefont {Stefani}, \citenamefont {Al-Masoudi}, \citenamefont
  {D{\"o}rscher} \emph {et~al.}}]{Lisdat2016}%
  \BibitemOpen
  \bibfield  {author} {\bibinfo {author} {\bibfnamefont {C.}~\bibnamefont
  {Lisdat}}, \bibinfo {author} {\bibfnamefont {G.}~\bibnamefont {Grosche}},
  \bibinfo {author} {\bibfnamefont {N.}~\bibnamefont {Quintin}}, \bibinfo
  {author} {\bibfnamefont {C.}~\bibnamefont {Shi}}, \bibinfo {author}
  {\bibfnamefont {S.~M.~F.}\ \bibnamefont {Raupach}}, \bibinfo {author}
  {\bibfnamefont {C.}~\bibnamefont {Grebing}}, \bibinfo {author} {\bibfnamefont
  {D.}~\bibnamefont {Nicolodi}}, \bibinfo {author} {\bibfnamefont
  {F.}~\bibnamefont {Stefani}}, \bibinfo {author} {\bibfnamefont
  {A.}~\bibnamefont {Al-Masoudi}}, \bibinfo {author} {\bibfnamefont
  {S.}~\bibnamefont {D{\"o}rscher}}, \emph {et~al.},\ }\bibfield  {title}
  {\bibinfo {title} {A clock network for geodesy and fundamental science},\
  }\href {https://doi.org/10.1038/ncomms12443} {\bibfield  {journal} {\bibinfo
  {journal} {Nature Communications}\ }\textbf {\bibinfo {volume} {7}},\
  \bibinfo {pages} {12443} (\bibinfo {year} {2016})}\BibitemShut {NoStop}%
\bibitem [{\citenamefont {Kato}\ \emph {et~al.}(2016)\citenamefont {Kato},
  \citenamefont {Inaba}, \citenamefont {Sugawa}, \citenamefont {Shibata},
  \citenamefont {Yamamoto}, \citenamefont {Yamashita},\ and\ \citenamefont
  {Takahashi}}]{Kato2016}%
  \BibitemOpen
  \bibfield  {author} {\bibinfo {author} {\bibfnamefont {S.}~\bibnamefont
  {Kato}}, \bibinfo {author} {\bibfnamefont {K.}~\bibnamefont {Inaba}},
  \bibinfo {author} {\bibfnamefont {S.}~\bibnamefont {Sugawa}}, \bibinfo
  {author} {\bibfnamefont {K.}~\bibnamefont {Shibata}}, \bibinfo {author}
  {\bibfnamefont {R.}~\bibnamefont {Yamamoto}}, \bibinfo {author}
  {\bibfnamefont {M.}~\bibnamefont {Yamashita}},\ and\ \bibinfo {author}
  {\bibfnamefont {Y.}~\bibnamefont {Takahashi}},\ }\bibfield  {title} {\bibinfo
  {title} {{Laser spectroscopic probing of coexisting superfluid and insulating
  states of an atomic Bose--Hubbard system}},\ }\href
  {https://doi.org/10.1038/ncomms11341} {\bibfield  {journal} {\bibinfo
  {journal} {Nature Communications}\ }\textbf {\bibinfo {volume} {7}},\
  \bibinfo {pages} {11341} (\bibinfo {year} {2016})}\BibitemShut {NoStop}%
\bibitem [{\citenamefont {Kolkowitz}\ \emph {et~al.}(2017)\citenamefont
  {Kolkowitz}, \citenamefont {Bromley}, \citenamefont {Bothwell}, \citenamefont
  {Wall}, \citenamefont {Marti}, \citenamefont {Koller}, \citenamefont {Zhang},
  \citenamefont {Rey},\ and\ \citenamefont {Ye}}]{Kolkowitz2017}%
  \BibitemOpen
  \bibfield  {author} {\bibinfo {author} {\bibfnamefont {S.}~\bibnamefont
  {Kolkowitz}}, \bibinfo {author} {\bibfnamefont {S.~L.}\ \bibnamefont
  {Bromley}}, \bibinfo {author} {\bibfnamefont {T.}~\bibnamefont {Bothwell}},
  \bibinfo {author} {\bibfnamefont {M.~L.}\ \bibnamefont {Wall}}, \bibinfo
  {author} {\bibfnamefont {G.~E.}\ \bibnamefont {Marti}}, \bibinfo {author}
  {\bibfnamefont {A.~P.}\ \bibnamefont {Koller}}, \bibinfo {author}
  {\bibfnamefont {X.}~\bibnamefont {Zhang}}, \bibinfo {author} {\bibfnamefont
  {A.~M.}\ \bibnamefont {Rey}},\ and\ \bibinfo {author} {\bibfnamefont
  {J.}~\bibnamefont {Ye}},\ }\bibfield  {title} {\bibinfo {title}
  {Spin--orbit-coupled fermions in an optical lattice clock},\ }\href
  {https://doi.org/10.1038/nature20811} {\bibfield  {journal} {\bibinfo
  {journal} {Nature}\ }\textbf {\bibinfo {volume} {542}},\ \bibinfo {pages}
  {66} (\bibinfo {year} {2017})}\BibitemShut {NoStop}%
\bibitem [{\citenamefont {Safronova}\ \emph
  {et~al.}(2018{\natexlab{a}})\citenamefont {Safronova}, \citenamefont
  {Budker}, \citenamefont {DeMille}, \citenamefont {Kimball}, \citenamefont
  {Derevianko},\ and\ \citenamefont {Clark}}]{SafronovaRMP2018}%
  \BibitemOpen
  \bibfield  {author} {\bibinfo {author} {\bibfnamefont {M.~S.}\ \bibnamefont
  {Safronova}}, \bibinfo {author} {\bibfnamefont {D.}~\bibnamefont {Budker}},
  \bibinfo {author} {\bibfnamefont {D.}~\bibnamefont {DeMille}}, \bibinfo
  {author} {\bibfnamefont {D.~F.~J.}\ \bibnamefont {Kimball}}, \bibinfo
  {author} {\bibfnamefont {A.}~\bibnamefont {Derevianko}},\ and\ \bibinfo
  {author} {\bibfnamefont {C.~W.}\ \bibnamefont {Clark}},\ }\bibfield  {title}
  {\bibinfo {title} {Search for new physics with atoms and molecules},\ }\href
  {https://doi.org/10.1103/RevModPhys.90.025008} {\bibfield  {journal}
  {\bibinfo  {journal} {Rev. Mod. Phys.}\ }\textbf {\bibinfo {volume} {90}},\
  \bibinfo {pages} {025008} (\bibinfo {year} {2018}{\natexlab{a}})}\BibitemShut
  {NoStop}%
\bibitem [{\citenamefont {Kolkowitz}\ \emph {et~al.}(2016)\citenamefont
  {Kolkowitz}, \citenamefont {Pikovski}, \citenamefont {Langellier},
  \citenamefont {Lukin}, \citenamefont {Walsworth},\ and\ \citenamefont
  {Ye}}]{Kolkowitz2016}%
  \BibitemOpen
  \bibfield  {author} {\bibinfo {author} {\bibfnamefont {S.}~\bibnamefont
  {Kolkowitz}}, \bibinfo {author} {\bibfnamefont {I.}~\bibnamefont {Pikovski}},
  \bibinfo {author} {\bibfnamefont {N.}~\bibnamefont {Langellier}}, \bibinfo
  {author} {\bibfnamefont {M.~D.}\ \bibnamefont {Lukin}}, \bibinfo {author}
  {\bibfnamefont {R.~L.}\ \bibnamefont {Walsworth}},\ and\ \bibinfo {author}
  {\bibfnamefont {J.}~\bibnamefont {Ye}},\ }\bibfield  {title} {\bibinfo
  {title} {Gravitational wave detection with optical lattice atomic clocks},\
  }\href {https://doi.org/10.1103/PhysRevD.94.124043} {\bibfield  {journal}
  {\bibinfo  {journal} {Phys. Rev. D}\ }\textbf {\bibinfo {volume} {94}},\
  \bibinfo {pages} {124043} (\bibinfo {year} {2016})}\BibitemShut {NoStop}%
\bibitem [{\citenamefont {Kramida}\ \emph {et~al.}(2022)\citenamefont
  {Kramida}, \citenamefont {Ralchenko}, \citenamefont {Reader},\ and\
  \citenamefont {{NIST ASD Team}}}]{NIST2022}%
  \BibitemOpen
  \bibfield  {author} {\bibinfo {author} {\bibfnamefont {A.}~\bibnamefont
  {Kramida}}, \bibinfo {author} {\bibfnamefont {Y.}~\bibnamefont {Ralchenko}},
  \bibinfo {author} {\bibfnamefont {J.}~\bibnamefont {Reader}},\ and\ \bibinfo
  {author} {\bibnamefont {{NIST ASD Team}}},\ }\href@noop {} {\bibinfo {title}
  {{NIST Atomic Spectra Database (version 5.10) [Online]}}},\ \bibinfo
  {howpublished} {Available: \url{https://physics.nist.gov/asd} [2023, January
  03]. National Institute of Standards and Technology, Gaithersburg, MD.}
  (\bibinfo {year} {2022})\BibitemShut {NoStop}%
\bibitem [{\citenamefont {Safronova}\ \emph
  {et~al.}(2018{\natexlab{b}})\citenamefont {Safronova}, \citenamefont
  {Porsev}, \citenamefont {Sanner},\ and\ \citenamefont {Ye}}]{Safronova2018}%
  \BibitemOpen
  \bibfield  {author} {\bibinfo {author} {\bibfnamefont {M.~S.}\ \bibnamefont
  {Safronova}}, \bibinfo {author} {\bibfnamefont {S.~G.}\ \bibnamefont
  {Porsev}}, \bibinfo {author} {\bibfnamefont {C.}~\bibnamefont {Sanner}},\
  and\ \bibinfo {author} {\bibfnamefont {J.}~\bibnamefont {Ye}},\ }\bibfield
  {title} {\bibinfo {title} {{Two Clock Transitions in Neutral Yb for the
  Highest Sensitivity to Variations of the Fine-Structure Constant}},\ }\href
  {https://doi.org/10.1103/PhysRevLett.120.173001} {\bibfield  {journal}
  {\bibinfo  {journal} {Phys. Rev. Lett.}\ }\textbf {\bibinfo {volume} {120}},\
  \bibinfo {pages} {173001} (\bibinfo {year} {2018}{\natexlab{b}})}\BibitemShut
  {NoStop}%
\bibitem [{\citenamefont {Dzuba}\ \emph {et~al.}(2018)\citenamefont {Dzuba},
  \citenamefont {Flambaum},\ and\ \citenamefont {Schiller}}]{Dzuba2018}%
  \BibitemOpen
  \bibfield  {author} {\bibinfo {author} {\bibfnamefont {V.~A.}\ \bibnamefont
  {Dzuba}}, \bibinfo {author} {\bibfnamefont {V.~V.}\ \bibnamefont
  {Flambaum}},\ and\ \bibinfo {author} {\bibfnamefont {S.}~\bibnamefont
  {Schiller}},\ }\bibfield  {title} {\bibinfo {title} {Testing physics beyond
  the standard model through additional clock transitions in neutral
  ytterbium},\ }\href {https://doi.org/10.1103/PhysRevA.98.022501} {\bibfield
  {journal} {\bibinfo  {journal} {Phys. Rev. A}\ }\textbf {\bibinfo {volume}
  {98}},\ \bibinfo {pages} {022501} (\bibinfo {year} {2018})}\BibitemShut
  {NoStop}%
\bibitem [{\citenamefont {Akamatsu}\ \emph {et~al.}(2018)\citenamefont
  {Akamatsu}, \citenamefont {Kobayashi}, \citenamefont {Hisai}, \citenamefont
  {Tanabe}, \citenamefont {Hosaka}, \citenamefont {Yasuda},\ and\ \citenamefont
  {Hong}}]{Akamatsu2018}%
  \BibitemOpen
  \bibfield  {author} {\bibinfo {author} {\bibfnamefont {D.}~\bibnamefont
  {Akamatsu}}, \bibinfo {author} {\bibfnamefont {T.}~\bibnamefont {Kobayashi}},
  \bibinfo {author} {\bibfnamefont {Y.}~\bibnamefont {Hisai}}, \bibinfo
  {author} {\bibfnamefont {T.}~\bibnamefont {Tanabe}}, \bibinfo {author}
  {\bibfnamefont {K.}~\bibnamefont {Hosaka}}, \bibinfo {author} {\bibfnamefont
  {M.}~\bibnamefont {Yasuda}},\ and\ \bibinfo {author} {\bibfnamefont {F.-L.}\
  \bibnamefont {Hong}},\ }\bibfield  {title} {\bibinfo {title} {{Dual-Mode
  Operation of an Optical Lattice Clock Using Strontium and Ytterbium Atoms}},\
  }\href {https://doi.org/10.1109/TUFFC.2018.2819888} {\bibfield  {journal}
  {\bibinfo  {journal} {IEEE Transactions on Ultrasonics, Ferroelectrics, and
  Frequency Control}\ }\textbf {\bibinfo {volume} {65}},\ \bibinfo {pages}
  {1069} (\bibinfo {year} {2018})}\BibitemShut {NoStop}%
\bibitem [{\citenamefont {Yamaguchi}\ \emph {et~al.}(2010)\citenamefont
  {Yamaguchi}, \citenamefont {Uetake}, \citenamefont {Kato}, \citenamefont
  {Ito},\ and\ \citenamefont {Takahashi}}]{Yamaguchi2010}%
  \BibitemOpen
  \bibfield  {author} {\bibinfo {author} {\bibfnamefont {A.}~\bibnamefont
  {Yamaguchi}}, \bibinfo {author} {\bibfnamefont {S.}~\bibnamefont {Uetake}},
  \bibinfo {author} {\bibfnamefont {S.}~\bibnamefont {Kato}}, \bibinfo {author}
  {\bibfnamefont {H.}~\bibnamefont {Ito}},\ and\ \bibinfo {author}
  {\bibfnamefont {Y.}~\bibnamefont {Takahashi}},\ }\bibfield  {title} {\bibinfo
  {title} {{High-resolution laser spectroscopy of a Bose–Einstein condensate
  using the ultranarrow magnetic quadrupole transition}},\ }\href
  {https://doi.org/10.1088/1367-2630/12/10/103001} {\bibfield  {journal}
  {\bibinfo  {journal} {New Journal of Physics}\ }\textbf {\bibinfo {volume}
  {12}},\ \bibinfo {pages} {103001} (\bibinfo {year} {2010})}\BibitemShut
  {NoStop}%
\bibitem [{\citenamefont {Shaniv}\ \emph {et~al.}(2018)\citenamefont {Shaniv},
  \citenamefont {Ozeri}, \citenamefont {Safronova}, \citenamefont {Porsev},
  \citenamefont {Dzuba}, \citenamefont {Flambaum},\ and\ \citenamefont
  {H\"affner}}]{Shaniv2018}%
  \BibitemOpen
  \bibfield  {author} {\bibinfo {author} {\bibfnamefont {R.}~\bibnamefont
  {Shaniv}}, \bibinfo {author} {\bibfnamefont {R.}~\bibnamefont {Ozeri}},
  \bibinfo {author} {\bibfnamefont {M.~S.}\ \bibnamefont {Safronova}}, \bibinfo
  {author} {\bibfnamefont {S.~G.}\ \bibnamefont {Porsev}}, \bibinfo {author}
  {\bibfnamefont {V.~A.}\ \bibnamefont {Dzuba}}, \bibinfo {author}
  {\bibfnamefont {V.~V.}\ \bibnamefont {Flambaum}},\ and\ \bibinfo {author}
  {\bibfnamefont {H.}~\bibnamefont {H\"affner}},\ }\bibfield  {title} {\bibinfo
  {title} {{New Methods for Testing Lorentz Invariance with Atomic Systems}},\
  }\href {https://doi.org/10.1103/PhysRevLett.120.103202} {\bibfield  {journal}
  {\bibinfo  {journal} {Phys. Rev. Lett.}\ }\textbf {\bibinfo {volume} {120}},\
  \bibinfo {pages} {103202} (\bibinfo {year} {2018})}\BibitemShut {NoStop}%
\bibitem [{\citenamefont {Tang}\ \emph {et~al.}()\citenamefont {Tang},
  \citenamefont {Yu}, \citenamefont {Sahoo}, \citenamefont {Dong},
  \citenamefont {Yang},\ and\ \citenamefont {Zou}}]{Tang2022}%
  \BibitemOpen
  \bibfield  {author} {\bibinfo {author} {\bibfnamefont {Z.-M.}\ \bibnamefont
  {Tang}}, \bibinfo {author} {\bibfnamefont {Y.-m.}\ \bibnamefont {Yu}},
  \bibinfo {author} {\bibfnamefont {B.~K.}\ \bibnamefont {Sahoo}}, \bibinfo
  {author} {\bibfnamefont {C.-Z.}\ \bibnamefont {Dong}}, \bibinfo {author}
  {\bibfnamefont {Y.}~\bibnamefont {Yang}},\ and\ \bibinfo {author}
  {\bibfnamefont {Y.}~\bibnamefont {Zou}},\ }\href
  {https://doi.org/10.48550/ARXIV.2208.09200} {\bibinfo {title} {{A New Clock
  Transition with the Highest Sensitivity to $\alpha$ Variation and
  Simultaneous Magic Trapping Conditions with Other Clock Transitions in Yb,
  arXiv:2208.09200}}}\BibitemShut {NoStop}%
\bibitem [{\citenamefont {Arvanitaki}\ \emph {et~al.}(2015)\citenamefont
  {Arvanitaki}, \citenamefont {Huang},\ and\ \citenamefont
  {Van~Tilburg}}]{Arvanitaki2015}%
  \BibitemOpen
  \bibfield  {author} {\bibinfo {author} {\bibfnamefont {A.}~\bibnamefont
  {Arvanitaki}}, \bibinfo {author} {\bibfnamefont {J.}~\bibnamefont {Huang}},\
  and\ \bibinfo {author} {\bibfnamefont {K.}~\bibnamefont {Van~Tilburg}},\
  }\bibfield  {title} {\bibinfo {title} {Searching for dilaton dark matter with
  atomic clocks},\ }\href {https://doi.org/10.1103/PhysRevD.91.015015}
  {\bibfield  {journal} {\bibinfo  {journal} {Phys. Rev. D}\ }\textbf {\bibinfo
  {volume} {91}},\ \bibinfo {pages} {015015} (\bibinfo {year}
  {2015})}\BibitemShut {NoStop}%
\bibitem [{\citenamefont {Kosteleck\'y}\ and\ \citenamefont
  {Lane}(1999)}]{Kostelecky1999}%
  \BibitemOpen
  \bibfield  {author} {\bibinfo {author} {\bibfnamefont {V.~A.}\ \bibnamefont
  {Kosteleck\'y}}\ and\ \bibinfo {author} {\bibfnamefont {C.~D.}\ \bibnamefont
  {Lane}},\ }\bibfield  {title} {\bibinfo {title} {Constraints on lorentz
  violation from clock-comparison experiments},\ }\href
  {https://doi.org/10.1103/PhysRevD.60.116010} {\bibfield  {journal} {\bibinfo
  {journal} {Phys. Rev. D}\ }\textbf {\bibinfo {volume} {60}},\ \bibinfo
  {pages} {116010} (\bibinfo {year} {1999})}\BibitemShut {NoStop}%
\bibitem [{\citenamefont {Pruttivarasin}\ \emph {et~al.}(2015)\citenamefont
  {Pruttivarasin}, \citenamefont {Ramm}, \citenamefont {Porsev}, \citenamefont
  {Tupitsyn}, \citenamefont {Safronova}, \citenamefont {Hohensee},\ and\
  \citenamefont {H{\"a}ffner}}]{Pruttivarasin2015}%
  \BibitemOpen
  \bibfield  {author} {\bibinfo {author} {\bibfnamefont {T.}~\bibnamefont
  {Pruttivarasin}}, \bibinfo {author} {\bibfnamefont {M.}~\bibnamefont {Ramm}},
  \bibinfo {author} {\bibfnamefont {S.~G.}\ \bibnamefont {Porsev}}, \bibinfo
  {author} {\bibfnamefont {I.~I.}\ \bibnamefont {Tupitsyn}}, \bibinfo {author}
  {\bibfnamefont {M.~S.}\ \bibnamefont {Safronova}}, \bibinfo {author}
  {\bibfnamefont {M.~A.}\ \bibnamefont {Hohensee}},\ and\ \bibinfo {author}
  {\bibfnamefont {H.}~\bibnamefont {H{\"a}ffner}},\ }\bibfield  {title}
  {\bibinfo {title} {{Michelson--Morley analogue for electrons using trapped
  ions to test Lorentz symmetry}},\ }\href
  {https://doi.org/10.1038/nature14091} {\bibfield  {journal} {\bibinfo
  {journal} {Nature}\ }\textbf {\bibinfo {volume} {517}},\ \bibinfo {pages}
  {592} (\bibinfo {year} {2015})}\BibitemShut {NoStop}%
\bibitem [{\citenamefont {Megidish}\ \emph {et~al.}(2019)\citenamefont
  {Megidish}, \citenamefont {Broz}, \citenamefont {Greene},\ and\ \citenamefont
  {H\"affner}}]{Megidish2019}%
  \BibitemOpen
  \bibfield  {author} {\bibinfo {author} {\bibfnamefont {E.}~\bibnamefont
  {Megidish}}, \bibinfo {author} {\bibfnamefont {J.}~\bibnamefont {Broz}},
  \bibinfo {author} {\bibfnamefont {N.}~\bibnamefont {Greene}},\ and\ \bibinfo
  {author} {\bibfnamefont {H.}~\bibnamefont {H\"affner}},\ }\bibfield  {title}
  {\bibinfo {title} {{Improved Test of Local Lorentz Invariance from a
  Deterministic Preparation of Entangled States}},\ }\href
  {https://doi.org/10.1103/PhysRevLett.122.123605} {\bibfield  {journal}
  {\bibinfo  {journal} {Phys. Rev. Lett.}\ }\textbf {\bibinfo {volume} {122}},\
  \bibinfo {pages} {123605} (\bibinfo {year} {2019})}\BibitemShut {NoStop}%
\bibitem [{\citenamefont {Sanner}\ \emph {et~al.}(2019)\citenamefont {Sanner},
  \citenamefont {Huntemann}, \citenamefont {Lange}, \citenamefont {Tamm},
  \citenamefont {Peik}, \citenamefont {Safronova},\ and\ \citenamefont
  {Porsev}}]{Sanner2019}%
  \BibitemOpen
  \bibfield  {author} {\bibinfo {author} {\bibfnamefont {C.}~\bibnamefont
  {Sanner}}, \bibinfo {author} {\bibfnamefont {N.}~\bibnamefont {Huntemann}},
  \bibinfo {author} {\bibfnamefont {R.}~\bibnamefont {Lange}}, \bibinfo
  {author} {\bibfnamefont {C.}~\bibnamefont {Tamm}}, \bibinfo {author}
  {\bibfnamefont {E.}~\bibnamefont {Peik}}, \bibinfo {author} {\bibfnamefont
  {M.~S.}\ \bibnamefont {Safronova}},\ and\ \bibinfo {author} {\bibfnamefont
  {S.~G.}\ \bibnamefont {Porsev}},\ }\bibfield  {title} {\bibinfo {title}
  {{Optical clock comparison for Lorentz symmetry testing}},\ }\href
  {https://doi.org/10.1038/s41586-019-0972-2} {\bibfield  {journal} {\bibinfo
  {journal} {Nature}\ }\textbf {\bibinfo {volume} {567}},\ \bibinfo {pages}
  {204} (\bibinfo {year} {2019})}\BibitemShut {NoStop}%
\bibitem [{\citenamefont {Dreissen}\ \emph {et~al.}(2022)\citenamefont
  {Dreissen}, \citenamefont {Yeh}, \citenamefont {F{\"u}rst}, \citenamefont
  {Grensemann},\ and\ \citenamefont {Mehlst{\"a}ubler}}]{Dreissen2022}%
  \BibitemOpen
  \bibfield  {author} {\bibinfo {author} {\bibfnamefont {L.~S.}\ \bibnamefont
  {Dreissen}}, \bibinfo {author} {\bibfnamefont {C.-H.}\ \bibnamefont {Yeh}},
  \bibinfo {author} {\bibfnamefont {H.~A.}\ \bibnamefont {F{\"u}rst}}, \bibinfo
  {author} {\bibfnamefont {K.~C.}\ \bibnamefont {Grensemann}},\ and\ \bibinfo
  {author} {\bibfnamefont {T.~E.}\ \bibnamefont {Mehlst{\"a}ubler}},\
  }\bibfield  {title} {\bibinfo {title} {{Improved bounds on Lorentz violation
  from composite pulse Ramsey spectroscopy in a trapped ion}},\ }\href
  {https://doi.org/10.1038/s41467-022-34818-0} {\bibfield  {journal} {\bibinfo
  {journal} {Nature Communications}\ }\textbf {\bibinfo {volume} {13}},\
  \bibinfo {pages} {7314} (\bibinfo {year} {2022})}\BibitemShut {NoStop}%
\bibitem [{\citenamefont {Hohensee}\ \emph {et~al.}(2013)\citenamefont
  {Hohensee}, \citenamefont {M\"uller},\ and\ \citenamefont
  {Wiringa}}]{Hohensee2013}%
  \BibitemOpen
  \bibfield  {author} {\bibinfo {author} {\bibfnamefont {M.~A.}\ \bibnamefont
  {Hohensee}}, \bibinfo {author} {\bibfnamefont {H.}~\bibnamefont {M\"uller}},\
  and\ \bibinfo {author} {\bibfnamefont {R.~B.}\ \bibnamefont {Wiringa}},\
  }\bibfield  {title} {\bibinfo {title} {{Equivalence Principle and Bound
  Kinetic Energy}},\ }\href {https://doi.org/10.1103/PhysRevLett.111.151102}
  {\bibfield  {journal} {\bibinfo  {journal} {Phys. Rev. Lett.}\ }\textbf
  {\bibinfo {volume} {111}},\ \bibinfo {pages} {151102} (\bibinfo {year}
  {2013})}\BibitemShut {NoStop}%
\bibitem [{\citenamefont {Berengut}\ \emph {et~al.}(2018)\citenamefont
  {Berengut}, \citenamefont {Budker}, \citenamefont {Delaunay}, \citenamefont
  {Flambaum}, \citenamefont {Frugiuele}, \citenamefont {Fuchs}, \citenamefont
  {Grojean}, \citenamefont {Harnik}, \citenamefont {Ozeri}, \citenamefont
  {Perez},\ and\ \citenamefont {Soreq}}]{Berengut2018}%
  \BibitemOpen
  \bibfield  {author} {\bibinfo {author} {\bibfnamefont {J.~C.}\ \bibnamefont
  {Berengut}}, \bibinfo {author} {\bibfnamefont {D.}~\bibnamefont {Budker}},
  \bibinfo {author} {\bibfnamefont {C.}~\bibnamefont {Delaunay}}, \bibinfo
  {author} {\bibfnamefont {V.~V.}\ \bibnamefont {Flambaum}}, \bibinfo {author}
  {\bibfnamefont {C.}~\bibnamefont {Frugiuele}}, \bibinfo {author}
  {\bibfnamefont {E.}~\bibnamefont {Fuchs}}, \bibinfo {author} {\bibfnamefont
  {C.}~\bibnamefont {Grojean}}, \bibinfo {author} {\bibfnamefont
  {R.}~\bibnamefont {Harnik}}, \bibinfo {author} {\bibfnamefont
  {R.}~\bibnamefont {Ozeri}}, \bibinfo {author} {\bibfnamefont
  {G.}~\bibnamefont {Perez}},\ and\ \bibinfo {author} {\bibfnamefont
  {Y.}~\bibnamefont {Soreq}},\ }\bibfield  {title} {\bibinfo {title} {Probing
  new long-range interactions by isotope shift spectroscopy},\ }\href
  {https://doi.org/10.1103/PhysRevLett.120.091801} {\bibfield  {journal}
  {\bibinfo  {journal} {Phys. Rev. Lett.}\ }\textbf {\bibinfo {volume} {120}},\
  \bibinfo {pages} {091801} (\bibinfo {year} {2018})}\BibitemShut {NoStop}%
\bibitem [{\citenamefont {Ono}\ \emph {et~al.}(2022)\citenamefont {Ono},
  \citenamefont {Saito}, \citenamefont {Ishiyama}, \citenamefont {Higomoto},
  \citenamefont {Takano}, \citenamefont {Takasu}, \citenamefont {Yamamoto},
  \citenamefont {Tanaka},\ and\ \citenamefont {Takahashi}}]{Ono2022}%
  \BibitemOpen
  \bibfield  {author} {\bibinfo {author} {\bibfnamefont {K.}~\bibnamefont
  {Ono}}, \bibinfo {author} {\bibfnamefont {Y.}~\bibnamefont {Saito}}, \bibinfo
  {author} {\bibfnamefont {T.}~\bibnamefont {Ishiyama}}, \bibinfo {author}
  {\bibfnamefont {T.}~\bibnamefont {Higomoto}}, \bibinfo {author}
  {\bibfnamefont {T.}~\bibnamefont {Takano}}, \bibinfo {author} {\bibfnamefont
  {Y.}~\bibnamefont {Takasu}}, \bibinfo {author} {\bibfnamefont
  {Y.}~\bibnamefont {Yamamoto}}, \bibinfo {author} {\bibfnamefont
  {M.}~\bibnamefont {Tanaka}},\ and\ \bibinfo {author} {\bibfnamefont
  {Y.}~\bibnamefont {Takahashi}},\ }\bibfield  {title} {\bibinfo {title}
  {{Observation of Nonlinearity of Generalized King Plot in the Search for New
  Boson}},\ }\href {https://doi.org/10.1103/PhysRevX.12.021033} {\bibfield
  {journal} {\bibinfo  {journal} {Phys. Rev. X}\ }\textbf {\bibinfo {volume}
  {12}},\ \bibinfo {pages} {021033} (\bibinfo {year} {2022})}\BibitemShut
  {NoStop}%
\bibitem [{\citenamefont {Saue}\ \emph {et~al.}(2020)\citenamefont {Saue},
  \citenamefont {Bast}, \citenamefont {Gomes}, \citenamefont {Jensen},
  \citenamefont {Visscher}, \citenamefont {Aucar}, \citenamefont {Di~Remigio},
  \citenamefont {Dyall}, \citenamefont {Eliav}, \citenamefont {Fasshauer} \emph
  {et~al.}}]{saue2020dirac}%
  \BibitemOpen
  \bibfield  {author} {\bibinfo {author} {\bibfnamefont {T.}~\bibnamefont
  {Saue}}, \bibinfo {author} {\bibfnamefont {R.}~\bibnamefont {Bast}}, \bibinfo
  {author} {\bibfnamefont {A.~S.~P.}\ \bibnamefont {Gomes}}, \bibinfo {author}
  {\bibfnamefont {H.~J.~A.}\ \bibnamefont {Jensen}}, \bibinfo {author}
  {\bibfnamefont {L.}~\bibnamefont {Visscher}}, \bibinfo {author}
  {\bibfnamefont {I.~A.}\ \bibnamefont {Aucar}}, \bibinfo {author}
  {\bibfnamefont {R.}~\bibnamefont {Di~Remigio}}, \bibinfo {author}
  {\bibfnamefont {K.~G.}\ \bibnamefont {Dyall}}, \bibinfo {author}
  {\bibfnamefont {E.}~\bibnamefont {Eliav}}, \bibinfo {author} {\bibfnamefont
  {E.}~\bibnamefont {Fasshauer}}, \emph {et~al.},\ }\bibfield  {title}
  {\bibinfo {title} {{The DIRAC code for relativistic molecular
  calculations}},\ }\href {https://doi.org/10.1063/5.0004844} {\bibfield
  {journal} {\bibinfo  {journal} {J. Chem. Phys.}\ }\textbf {\bibinfo {volume}
  {152}},\ \bibinfo {pages} {204104} (\bibinfo {year} {2020})}\BibitemShut
  {NoStop}%
\bibitem [{DIR()}]{DIRAC22}%
  \BibitemOpen
  \href@noop {} {}\bibinfo {note} {H.~J.~{\relax Aa}.~Jensen, R.~Bast,
  A.~S.~P.~Gomes, T.~Saue and L.~Visscher et al., {DIRAC}, A relativistic ab
  initio electronic structure program, release {DIRAC22},
  \url{https://doi.org/10.5281/zenodo.6010450}; see also
  \url{http://www.diracprogram.or.g}), accessed 2022; see also
  \url{http://www.diracprogram.org}}\BibitemShut {NoStop}%
\bibitem [{\citenamefont {Boyd}\ \emph {et~al.}(2007)\citenamefont {Boyd},
  \citenamefont {Zelevinsky}, \citenamefont {Ludlow}, \citenamefont {Blatt},
  \citenamefont {Zanon-Willette}, \citenamefont {Foreman},\ and\ \citenamefont
  {Ye}}]{Boyd2007}%
  \BibitemOpen
  \bibfield  {author} {\bibinfo {author} {\bibfnamefont {M.~M.}\ \bibnamefont
  {Boyd}}, \bibinfo {author} {\bibfnamefont {T.}~\bibnamefont {Zelevinsky}},
  \bibinfo {author} {\bibfnamefont {A.~D.}\ \bibnamefont {Ludlow}}, \bibinfo
  {author} {\bibfnamefont {S.}~\bibnamefont {Blatt}}, \bibinfo {author}
  {\bibfnamefont {T.}~\bibnamefont {Zanon-Willette}}, \bibinfo {author}
  {\bibfnamefont {S.~M.}\ \bibnamefont {Foreman}},\ and\ \bibinfo {author}
  {\bibfnamefont {J.}~\bibnamefont {Ye}},\ }\bibfield  {title} {\bibinfo
  {title} {Nuclear spin effects in optical lattice clocks},\ }\href
  {https://doi.org/10.1103/PhysRevA.76.022510} {\bibfield  {journal} {\bibinfo
  {journal} {Phys. Rev. A}\ }\textbf {\bibinfo {volume} {76}},\ \bibinfo
  {pages} {022510} (\bibinfo {year} {2007})}\BibitemShut {NoStop}%
\bibitem [{\citenamefont {Katori}\ \emph {et~al.}(2003)\citenamefont {Katori},
  \citenamefont {Takamoto}, \citenamefont {Pal'chikov},\ and\ \citenamefont
  {Ovsiannikov}}]{Katori2003}%
  \BibitemOpen
  \bibfield  {author} {\bibinfo {author} {\bibfnamefont {H.}~\bibnamefont
  {Katori}}, \bibinfo {author} {\bibfnamefont {M.}~\bibnamefont {Takamoto}},
  \bibinfo {author} {\bibfnamefont {V.~G.}\ \bibnamefont {Pal'chikov}},\ and\
  \bibinfo {author} {\bibfnamefont {V.~D.}\ \bibnamefont {Ovsiannikov}},\
  }\bibfield  {title} {\bibinfo {title} {{Ultrastable Optical Clock with
  Neutral Atoms in an Engineered Light Shift Trap}},\ }\href
  {https://doi.org/10.1103/PhysRevLett.91.173005} {\bibfield  {journal}
  {\bibinfo  {journal} {Phys. Rev. Lett.}\ }\textbf {\bibinfo {volume} {91}},\
  \bibinfo {pages} {173005} (\bibinfo {year} {2003})}\BibitemShut {NoStop}%
\bibitem [{\citenamefont {Le~Kien}\ \emph {et~al.}(2013)\citenamefont
  {Le~Kien}, \citenamefont {Schneeweiss},\ and\ \citenamefont
  {Rauschenbeutel}}]{LeKien2013}%
  \BibitemOpen
  \bibfield  {author} {\bibinfo {author} {\bibfnamefont {F.}~\bibnamefont
  {Le~Kien}}, \bibinfo {author} {\bibfnamefont {P.}~\bibnamefont
  {Schneeweiss}},\ and\ \bibinfo {author} {\bibfnamefont {A.}~\bibnamefont
  {Rauschenbeutel}},\ }\bibfield  {title} {\bibinfo {title} {Dynamical
  polarizability of atoms in arbitrary light fields: general theory and
  application to cesium},\ }\href {https://doi.org/10.1140/epjd/e2013-30729-x}
  {\bibfield  {journal} {\bibinfo  {journal} {The European Physical Journal D}\
  }\textbf {\bibinfo {volume} {67}},\ \bibinfo {pages} {92} (\bibinfo {year}
  {2013})}\BibitemShut {NoStop}%
\bibitem [{\citenamefont {Tang}\ \emph {et~al.}(2018)\citenamefont {Tang},
  \citenamefont {Yu}, \citenamefont {Jiang},\ and\ \citenamefont
  {Dong}}]{Tang2018}%
  \BibitemOpen
  \bibfield  {author} {\bibinfo {author} {\bibfnamefont {Z.-M.}\ \bibnamefont
  {Tang}}, \bibinfo {author} {\bibfnamefont {Y.-M.}\ \bibnamefont {Yu}},
  \bibinfo {author} {\bibfnamefont {J.}~\bibnamefont {Jiang}},\ and\ \bibinfo
  {author} {\bibfnamefont {C.-Z.}\ \bibnamefont {Dong}},\ }\bibfield  {title}
  {\bibinfo {title} {Magic wavelengths for the transition in ytterbium atom},\
  }\href {https://doi.org/10.1088/1361-6455/aac181} {\bibfield  {journal}
  {\bibinfo  {journal} {Journal of Physics B: Atomic, Molecular and Optical
  Physics}\ }\textbf {\bibinfo {volume} {51}},\ \bibinfo {pages} {125002}
  (\bibinfo {year} {2018})}\BibitemShut {NoStop}%
\bibitem [{\citenamefont {Porsev}\ \emph {et~al.}(2018)\citenamefont {Porsev},
  \citenamefont {Safronova}, \citenamefont {Safronova},\ and\ \citenamefont
  {Kozlov}}]{Porsev2018}%
  \BibitemOpen
  \bibfield  {author} {\bibinfo {author} {\bibfnamefont {S.~G.}\ \bibnamefont
  {Porsev}}, \bibinfo {author} {\bibfnamefont {M.~S.}\ \bibnamefont
  {Safronova}}, \bibinfo {author} {\bibfnamefont {U.~I.}\ \bibnamefont
  {Safronova}},\ and\ \bibinfo {author} {\bibfnamefont {M.~G.}\ \bibnamefont
  {Kozlov}},\ }\bibfield  {title} {\bibinfo {title} {{Multipolar
  Polarizabilities and Hyperpolarizabilities in the Sr Optical Lattice
  Clock}},\ }\href {https://doi.org/10.1103/PhysRevLett.120.063204} {\bibfield
  {journal} {\bibinfo  {journal} {Phys. Rev. Lett.}\ }\textbf {\bibinfo
  {volume} {120}},\ \bibinfo {pages} {063204} (\bibinfo {year}
  {2018})}\BibitemShut {NoStop}%
\bibitem [{\citenamefont {Ushijima}\ \emph {et~al.}(2018)\citenamefont
  {Ushijima}, \citenamefont {Takamoto},\ and\ \citenamefont
  {Katori}}]{Ushijima2018}%
  \BibitemOpen
  \bibfield  {author} {\bibinfo {author} {\bibfnamefont {I.}~\bibnamefont
  {Ushijima}}, \bibinfo {author} {\bibfnamefont {M.}~\bibnamefont {Takamoto}},\
  and\ \bibinfo {author} {\bibfnamefont {H.}~\bibnamefont {Katori}},\
  }\bibfield  {title} {\bibinfo {title} {{Operational Magic Intensity for Sr
  Optical Lattice Clocks}},\ }\href
  {https://doi.org/10.1103/PhysRevLett.121.263202} {\bibfield  {journal}
  {\bibinfo  {journal} {Phys. Rev. Lett.}\ }\textbf {\bibinfo {volume} {121}},\
  \bibinfo {pages} {263202} (\bibinfo {year} {2018})}\BibitemShut {NoStop}%
\end{thebibliography}%

\end{document}