\documentclass[a4paper]{amsart}

\theoremstyle{plain}
\newtheorem{thm}{Theorem}
\theoremstyle{definition}
\newtheorem{defi}{Definition}
\theoremstyle{plain}
\newtheorem{lem}{Lemma}[section]
\theoremstyle{plain}
\newtheorem*{nclem}{The $C^r$ Closing Lemma}
\theoremstyle{plain}

\newcommand\EU{\mathbb{R}}
\newcommand\SVF{\mathfrak{X}(M)}
\newcommand\CFB{F}
\newcommand\FB{long flow box}
\newcommand\NW{\Omega}
\newcommand\NB{\mathcal{U}}
\newcommand\LIE{\mathcal{L}}
\newcommand\CD{\nabla}
\newcommand\oY{\overline{Y}}
\newcommand\oX{\overline{X}}
\newcommand\orho{\overline{\rho}}
\newcommand\mT{\mathcal{T}}
\newcommand\mP{\mathcal{P}}
\newcommand\mI{\mathcal{I}}
\newcommand\mO{\mathcal{O}}


\DeclareMathOperator{\dis}{dist}

\usepackage{amsthm,amsmath,amsfonts,amssymb,amsrefs,mathtools,graphicx}
\usepackage{caption,subcaption}

\graphicspath{D:/Program Files (x86)/MiKTeX/ARTICLES OF MINE/a general iterative algorithm}

\numberwithin{equation}{section}

\begin{document}

\title{general iterative approximation to differential equations.}
\author{Chang Gao}
\begin{abstract}
This article provides a general iterative approximation to partial differential equations, and thus establish existence of smooth solution. The heart of the method is to contract (or expand) the boundary conditions uniformly in the domain, then using local and global correspondence to transform discrete step function to successive integration on the domain. Numerical scheme is discussed and tested for Navier-Stokes equation.
\end{abstract}
\maketitle

\section{introduction}
Let $\Omega \subset \EU^n$ be a domian with smooth boundary $\partial \Omega$, let $P = {P_1, P_2, \ldots, P_k}$ be a partition of $\EU^n$. 
 Denote by $u_i: \Omega \to P_i$ unknown vector functions defined on the chart $x$, denote $y_i$ a one dimensional parameter of the chart $y$. Let $\phi$ be a coordinate transform from the chart $y$ to $x$ so that 
$$span_{j\neq i} \frac{\partial \phi}{\partial y_j}(\partial \Omega)=T\partial \Omega_i
 \quad %\mathrm{and} \quad 
span_{j=1}^n \frac{\partial \phi}{\partial y_j}(\Omega)=T \Omega 
\quad %\mathrm{and} \quad 
\cup_{i=1}^n T\partial \Omega_i=T\partial \Omega,$$
 where $\partial \Omega_i$ has dimension $dim(P_i)-1$. Consider a system of differential equations that can be written in the form:
\begin{equation}\label{pro0}\begin{aligned}
\frac{\partial u_i(\phi y)}{\partial y_i}=& F_i(u, y)
\quad %\mathrm{and} \quad
u_i(\partial \Omega_i)=w_0 \quad i=1,\ldots,l  \\
\frac{\partial u_j(\phi y)}{\partial y_j}=& F_j(u, y)
\quad %\mathrm{and} \quad
u_j(\partial \Omega_j)=W_j(u_i, y) \quad  j=l+1,\ldots,k
\end{aligned}\end{equation}
where the functionals $F_i$, $F_j$ and $W_j$ are completely determined, $w_0$ represents initial and boundary conditions. Equation ($\ref{pro0}$) covers many mathematical physical equations that appear in nature and industry, such as hydrodynamic equations, wave equations and statistical equations. %We will illustrate this with a hydrodynamic equation example in later sections.
 For example, the imcompressible Navier-Stokes equation can be written in the form of equation (\ref{pro0}):
\[\begin{aligned}
\frac{\partial u}{\partial t}&=-u\cdot \nabla u+\mu \Delta u+ f-grad\, p \quad u(t=t_0)=u_0 \\
\frac{\partial p}{\partial y_j}&=q \quad p(\partial \Omega)=p_0 \\
\frac{\partial q}{\partial y_j}&=-\frac{\partial q}{\partial y_k}- g(y)\, q+ div\,(-u\cdot \nabla u+\mu \Delta u+ f) \quad q(\partial \Omega)=q_0(u) \quad k\neq j
\end{aligned}\]

 The problem of determining the general conditions for existence of solution to equation ($\ref{pro0}$) is a long standing open problem in mathematics \cite{Fe2006}. Besides theoretical interests, an iterative algorithm that approximate solution to such equations accurately and efficiently has potentially a wide range of applications. %and provides a better alternative to finite element methods in certain cases. 

Current progress on this problem is too vast to survey thoroughly,  to name a few aspects: weak solutions, finite element methods and various strong solutions in special cases. Examples of non-uniqueness and finite time blow-up can be found in literature \cite{BuDrShVi2022}. %These phenomena may occur when their initial or boundary conditions does not match ones defined in equation (\ref{pro0}). %The general initial and boundary conditions presented in equation (\ref{pro0}) could explain
The goal of this article is to prove that the conditions  specified in equation (\ref{pro0}) are necessary and sufficient for the existence of smooth solution, by constructing an iterative approximation that converge to it.  %The inductive sequence may provide helpful insight into relevent researches metioned above. 
\begin{thm}\label{con0}
The following sequences converge to the solution of equation (\ref{pro0}) as $\lim m,p\to \infty$:
\begin{equation}\label{con1}\begin{aligned}
U(i,0)=&w_0 \quad \\
U(i,m+1)=&U(i,0)+ \\ 
+\lim_{p \to \infty} \int_{y_i\, proj\, \partial \Omega}^{y_i} &F_i(U(1,m),\ldots, U(l,m), U(l+1,m,p),\ldots, U(k,m,p) , y) \, dy_i \\ \mathrm{for} \: i=1,\ldots, l. \: \mathrm{And} \\
U(j,m,0)=&W_j(U(1,m),\ldots,U(l,m), y) \quad \\
U(j,m,p+1)=&U(j,m,0)+ \\ 
+\int_{y_j\, proj\, \partial \Omega}^{y_j} &F_j(U(1,m),\ldots, U(l,m), U(l+1,m,p),\ldots, U(k,m,p) , y) \, dy_j
\end{aligned}\end{equation}
for j=l+1,\ldots,k. %where $y_i\, proj\, \partial \Omega$ denotes the projection of $y_i$ on the boundary $\partial \Omega$.
\end{thm}
 Different partition index $i<k$ corresponds to dependent sequences, i.e. each sequence takes limit in each step of another sequence to converge to the solution $u$. Notice that setting $U(i,m+1)=U(i,m)$ in the inductive expression (\ref{con1}) automatically implies $U(i,m)$ is a solution to equation (\ref{pro0}), the problem reduce to proving the convergence of the sequences (\ref{con1}).   

A corresponding step method for numerical computations is also introduced. The step method provides a more efficient alternative to finite elements, since each step takes only linear operations.
%It is assumed that the functionals $F$ and $W$ and the boundary conditions are smooth, the domain is simply connected and has smooth boundary.


\section{convergence and error estimate}
 Notations for $F,\,W,\,y$ and others follow previous section. Let $L$ be a Lipschitz constant that bounds the functionals $W(u,y)$, $F(u,y)$ and boundary conditions $w_0$. Recall that $\frac{\partial \phi}{\partial y_i}$ are transvesal to the boundary $\partial \Omega$, so that $\frac{\partial u}{\partial y_j}(\partial \Omega) \quad j\neq i$ is completely determined. Let $T\geq |y_i-y_i \, proj \, \partial \Omega|$ be an integration bound for all $i$. The convergence for the inductive sequences (\ref{con1}) may now be proved. %to the solution of the general differential equation (\ref{pro0}).
\begin{proof}[proof of theorem \ref{con0}]
For simplicity denote $f(m)=|U(m+1)-U(m)|$ the maximum of $|U(i,m+1)-U(i,m)|$ for all partition index $i=1,\ldots, k$. %it is clear that 
%$$U(i,m)\leq \sum_{i=0}^m f_i(m).$$
 Substitute the inductive expression (\ref{con1}) into $f(m)$ %and estimating the bound for $f_i(m)$:
\begin{equation*}\begin{aligned}
f(m)=&|\int^{y_i}_{y_i\, proj\, \partial \Omega}(F(U(m),y)-F(U(m-1),y))\, dy_i| \\
\leq& \int^{y_i}_{y_i\, proj\, \partial \Omega}L|U(m)-U(m-1)|\,dy_i \\
\leq& \frac{1}{2} L\, T \, f(m-1)
\end{aligned}\end{equation*}
By induction on $m$, a bound for $f(m)$ is obtained
\[
f(m)\leq \frac{1}{m!}\, L^m \int^{y_i}_{y_i\, proj\, \partial \Omega} |F(U(0))|\, y_i^m \, dy_i \leq \frac{1}{(m+1)!}\, L^{m+1}\, T^{m+1}
\]
Since
$$\lim_{m \to \infty}U(i,m)\leq U(0)+\sum_{m=0}^\infty f(m)\leq w_0+ e^{L\, T},$$
I now conclude that $U(i,m)$ converge uniformly as $m \to \infty$, by Weierstrass M test. The same reasoning hold for $U(j,m,p)$.
\end{proof}

Denote the error term $R(m')=\sum^\infty_{m=m'+1} f(m)$ for order $m'$, which is bounded by the Lagrange remainder term:
\begin{equation}\label{rem}
R(m')\leq \sum^\infty_{m=m'+1} \frac{1}{m!} (L\, T)^m \leq \frac{1}{(m'+1)!}\, (L\, T)^{m'+1}\, e^{L\, T}\, .
\end{equation}
Notice that the error may increase initially when $L\,T$ is large, but eventually converge.


\section{step method}
To obatin numerical solutions efficiently, there is a corresponding step method to the iterative approximation (\ref{con1}), which sometimes takes less computational resource than successive integration. The basic idea is that when the step size $dy$ is small enough,  the following may be equalized:
\[
u(y_0+dy)=u(y_0)+\int_{y_0}^{y_0+dy} F(u,y)\, dy \approx u(y_0)+ F(u(y_0),y_0)\, dy .
\]


Consider a simplified differential system of equation (\ref{pro0}), $\frac{\partial u}{\partial y}=F(u,y)$. Suppose the integration interval is $T$, and the total step number is $N=\frac{T}{dy}$. Denote $u(y_0+m\, dy)$ by $U(m)$, the iterative expression for each step is obtained:
\begin{equation}\label{stp01}\begin{aligned}
U(0)&=u(y_0) \\
U(m+1)&=U(m)+F(U(m),m)\, dy
\end{aligned}\end{equation}
with $U(N)$ the numerical solution sought. To prove convergence, it suffice to prove that the step method (\ref{stp01}) is actually equivalent to the iterative approximation (\ref{con1}) when $\lim dy \to 0$. 
%\begin{lem}
%The inductive expression of step method (\ref{stp01}) converge to that of iterative approximation (\ref{con1}) as $dy \to 0$.
%\end{lem}
%\begin{proof}
Observe that
\[\begin{aligned}
U(m+1)&=U(m)+F(U(m),m)\, dy \\
&= U(m-1)+(F(U(m-1),m-1)+F(U(m),m))\, dy .
\end{aligned}\]
By induction on $m$
\[
U(m+1)=U(0)+\sum^m_{m'=0}F(U(m'),m')\, dy .
\]
Then take $dy \to 0$
\[
\lim_{dy \to 0} U(m+1)=U(0)+\int_{y_0}^y F(U(m),y)\, dy \quad ,
\]
which has exactly the same form as the iterative approximation (\ref{con1}). I may conclude that the step method (\ref{stp01}) converges to the iterative approximation (\ref{con1}) as $dy \to 0$ by this local and global correspondence.
%\end{proof}
%It is in fact this local-global correpondence that led the author to the iterative approximation (\ref{con1}). Convergence is however easier to prove for the successive integration.

\subsection*{error bound}

Let $L$ be the Lipschitz constant that bounds $F(u)$ as before. In case of uniform step size $dy$, denote by $\overline{u}(m)$ the solution obtained by step method at $m$-th step. Denote by $u(m)$ the actual solution. Denote by $R(m)=|\overline{u}(m)-u(m)|$ the error term. Then 
\[\begin{aligned}
R(m+1)&=|\overline{u}(m)+F(\overline{u}(m))\, dy- u(m) - \int_m^{m+1} F(u(y))\, dy| \\
%&\leq R(m)+\int_m^{m+1} \sup| F(u(m))-F(u(m+1))|\, dy \\
&\leq R(m)+ | F(\overline{u}(m))-F(u(m+1))|\, dy \\
&\leq R(m)+ | F(\overline{u}(m))-F(u(m))|\, dy + | F(u(m))-F(u(m+1))|\, dy \\
&\leq R(m)+ L\, R(m)\, dy + L^2\, dy.
\end{aligned}\]
Set $R(0)=0$, by induction on $m$
\[
R(N)\leq (1+L\, dy)^N\, R(0) + L^2\, dy \sum^{N-1}_{m'=0} (1+L\, dy)^{m'} = L^2\, dy \sum^{N-1}_{m'=0} (1+L\, dy)^{m'}.
\]
This is the error bound for the step method at $N$-th step. The error increases further away from the boundary. %vanishes as $dy \to 0$.


\section{example: Navier-Stokes equations}
To demonstrate  intuitively the accuracy of the step method (\ref{stp01}), consider the imcompressible Navier-Stokes equations: 
\begin{equation}\label{ns0}\begin{aligned}
\frac{\partial u}{\partial t}&= -u\cdot \nabla u - grad\, p +  \Delta u \\
\Delta p &= div ( -u\cdot \nabla u  + \Delta u)\, ,
\end{aligned}\end{equation}
where the Poisson equation for $p$ is obtained by taking the derivative of incompressible condition $div(u_t)=0$ and substituted into the first equation. Finding numerical solution to Navier-Stokes equation is then equivalent to iterate step function (\ref{stp01}) for Poisson equation of $p$ at each step of $u$. The step function for $u$ is
\[
u(i+1)=u(i)+(-u(i)\cdot \nabla u(i)- grad \, p(i) + \Delta u(i))\, dt.
\]


To compare intuitively the accuracy of the step method, consider for example the Taylor-Green vortex, an analytic solution to Navier-Stokes equation
\[\begin{aligned}
u &= (\sin(x) \cos(y)\, e^{- 2 t}, -\cos(x) \sin(y)\, e^{- 2 t}) \\
p &=\frac{1}{4}\cos(2 x)\, \cos(2 y)\, e^{-4 t} .
\end{aligned}\]

Assuming the solution to Poisson equation of $p$ is known for each $u$, the step function for $u$ may be iterated from $t=0$, see figure \ref{tgf1}.  


\begin{figure}[h]
\begin{subfigure}[t]{0.4\linewidth}
\includegraphics[width=\linewidth]{fig1}
\caption{Analytic solution at time 1 exhibits saddle connections.}
\end{subfigure}
\begin{subfigure}[t]{0.4\linewidth}
\includegraphics[width=\linewidth]{fig2}
\caption{Step function at time 1 with step size 0.25 .}
\end{subfigure}
\caption{Taylor-Green vortex.}\label{tgf1}
\end{figure}

The central difference between solving ODEs' and PDEs' with step method is that PDEs' take derivatives at each step, while ODEs' are purely numerical. One way to optimize this algorithm is to find the closed-form solution to the step function. This is achieved by treating the differential operators as linear coefficients, and then solve the corresponding linear difference equation. Though the specific computation time for a given goal depends on the form of the equation, the shape of boundary and boundary conditions, substituting derivatives into closed form significantly reduces computation time.

 Consider the step function for Poisson equation of $p$ with circular boundary conditions, denote $f(j)=div ( -u(i)\cdot \nabla u(i)  + \Delta u(i))$ the right hand side of poisson equation at $j$-th step away from boundary 
\[\begin{aligned}
%\frac{\partial}{\partial R}\, 
p(j+1)&= p(j)+q(j)\, dr \\
q(j+1)&= q(j)+(f(j)+ \frac{q(j)}{r(j)} - \frac{1}{r(j)^2} \frac{\partial^2 p(j)}{\partial \theta^2} )\, dr
\end{aligned}\]
where $q=\frac{\partial p}{\partial r}$. The right hand side of $q$ is just laplacian in polar coordinates, where the differential operators satisfy $\frac{\partial}{\partial r}=\frac{-x}{\sqrt{x^2+y^2}} \frac{\partial}{\partial x}+\frac{-y}{\sqrt{x^2+y^2}} \frac{\partial}{\partial y}$ and $\frac{\partial}{\partial \theta}=-y\frac{\partial}{\partial x}+x\frac{\partial}{\partial y}$. 

Another optimization approach is to consider stable solution of the corresponding linear system, which will not be addressed in this article. The step method introduced above adapts to systems of arbitrary dimensions and order, should enough boundary conditions be supplmented.



 %With boundary conditions $u_0$ and $q_0$ defined on the cube $\Omega=\{ (x,y,z,t)|-\pi \leq x,y,z \leq \pi,\, 0\leq t \}$, I may write the inductive solution of (\ref{ns0}) in terms of (\ref{stp01}).
%\[\begin{aligned}
%U(0)&= u_0 \quad P(0)= p_0 \quad Q(0)= q_0 \\
%U(m+1)&= U(m)+ (-U(m)\cdot \nabla U(m) - grad\, P(m) + \frac{1}{3} \Delta U(m))\, dy \\
%P(m+1)&=P(m)+Q(m)\, dy \\
%Q(m+1)&=Q(m)+ div ( -U(m)\cdot \nabla U(m)  + \frac{1}{3} \Delta U(m))\, dy
%\end{aligned}\] 





\section{conclusions}
%Arbitrary order. step method use only linear computation. Both guaranteed to converge. Fast and accurate. 
%further research: optimizing the algorithm for various boundaries. It is possible that there is an efficient optimization scheme for linear equations. 
The iterative approximation described in this article estabishes the existence of smooth solution to a wide class of, if not all, partial differential equations. Additionaly, the corresponding step method provides a potentially more efficient numerical scheme for solving equations than existing methods, if the algorithm is properly optimized. No doubt the method opens up new paths to the understanding of turbulence bifurcations phenomena.

 One of the key element of the method is choosing the appropriate differentiation transformation, as described at the beginning of this article, so the entire boundary continuously contract to a single point. The local and global correspondences connects discrete steps of this contraction to successive integration on the entire domain.


\begin{bibdiv}\begin{biblist}

\bib{BuDrShVi2022}{article}{
doi={10.4171/ICM2022/210},
 author = {Tristan Buckmaster},
author={Theodore Drivas},
author={Steve Shkoller},
author={Vlad Vicol},
 journal = {Proceedings of the International Congress of Mathematicians 2022},
 pages = {},
 title = {Formation and development of singularities for the compressible Euler equations.},
 volume = {},
 year={}
}

\bib{Fe2006}{article}{
 author = {Charles L. Fefferman},
 journal = {The Millennium Prize Problems, Clay Mathematics Institute, https://www.claymath.org/sites/default/files/navierstokes.pdf },
 url={https://www.claymath.org/sites/default/files/navierstokes.pdf}
 title = {Existence and Smoothness of the Navier-Stokes Equation},
 volume = {},
 year={2006}
}

\end{biblist}\end{bibdiv}

\end{document}

