\begin{abstract}
Foundation models (\eg ChatGPT, StableDiffusion) pervasively influence society, warranting immediate social attention.
While the models themselves garner much attention, to accurately characterize their impact, we must consider the broader sociotechnical ecosystem.
We propose \EG as a documentation framework to transparently centralize knowledge of this ecosystem.
\EG is composed of \textit{assets} (datasets, models, applications) linked together by \textit{dependencies} that indicate technical (\eg how Bing relies on GPT-4) and social (\eg how Microsoft relies on OpenAI) relationships.
To supplement the graph structure, each asset is further enriched with fine-grained metadata (\eg the license or training emissions).
% Through this documentation, \EG makes explicit what the current status quo is (including \textit{dark matter}: information that must exist, yet the public is unaware) as well as how the ecosystem evolves (through its sustained maintenance).
We document the ecosystem extensively at \websiteURL: as of \releasedate, we annotate \numnodes assets (\numdatasets datasets, \nummodels models, \numapplications applications) from \numorganizations organizations linked by \numedges dependencies.
We show \EG functions as a powerful abstraction and interface for achieving the minimum transparency required to address myriad use cases.
Therefore, we envision \EG will be a community-maintained resource that provides value to stakeholders spanning AI researchers, industry professionals, social scientists, auditors and policymakers.
\end{abstract}