\section{Introduction}
\label{sec:introduction}

% \tldr{Rapid development + growing impact of FMs}
Foundation models (FMs) are the defining paradigm of modern AI \citep{bommasani2021opportunities}.
Beginning with language models 
\citep{
% peters2018elmo, 
devlin2019bert,
brown2020gpt3,
% rae2021gopher,
chowdhery2022palm}, 
the paradigm has expanded to
images \citep{chen2020imagegpt, ramesh2021dalle, radford2021clip},
videos \citep{singer2022makeavideo, wang2022internvideo},
code \citep{chen2021codex},
% speech \citep{},
% music \citep{},
proteins \citep{jumper2021alphafold, verkuil2022esm2},
% robotics \citep{},
and more.
Beyond rapid technology development, foundation models have entered broad social discourse \citep{nyt2020, nature2021, economist2022, cnn2023}.
Given their remarkable capabilities, we are witnessing unprecedented adoption:
ChatGPT amassed 100 million users in just 50 days \citep[the fastest-growing consumer application in history;][]{hu2023chatgpt}
and
Stable Diffusion accrued 30k+ GitHub stars in 90 days \citep[much faster than Bitcoin and Spark;][]{appenzeller2022stablediffusion},
% HuggingFace \dots,
% CoPilot \dots.
% Skip for brevity
As a bottom line, over 200 foundation model startups have emerged, collectively raising \$3.5B as of October 2022 \citep{kaufmann2022scaleindex}.
In fact, the influx of funding continues to accelerate: Character received \$200M from Andreesen Horowitz, Adept received \$350M from General Catalyst, and OpenAI received \$10B from Microsoft just in Q1 of 2023. 

% \tldr{Confusion about status quo}
Foundation models are changing society but what is the nature of this impact?
Who reaps the benefits, who shoulders the harms, and how can we characterize these societal changes?
Further, how do trends in research correspond to outcomes in practice (\eg how do emergent abilities \citep{wei2022emergent} influence deployment decisions, how do documented risks \citep{abid2021persistent} manifest as concrete harms)?
Overall, there is pervasive confusion on the status quo, which breeds further uncertainty on how the space of foundation models will evolve and what change is necessary.
Currently, the AI community and broader public tolerate the uncomfortable reality that models are deployed ubiquitously through products yet we know increasingly little about the models, how they were built, and the mechanisms (if any) in place to mitigate and address harm.

    \begin{figure*}
        \centering
        \begin{subfigure}[b]{0.475\textwidth}
            \centering
            \includegraphics[width=\textwidth]{figures/pile_hub.png}
            \caption{\assetURL{The Pile} dataset \citep{gao2021thepile}}    
        \end{subfigure}
        \hfill
        \begin{subfigure}[b]{0.475\textwidth}  
            \centering 
            \includegraphics[width=\textwidth]{figures/p3_hub.png}
            \caption{\assetURL{P3} dataset \citep{sanh2021multitask}}
        \end{subfigure}
        \vskip\baselineskip
        \begin{subfigure}[b]{0.475\textwidth}   
            \centering 
            \includegraphics[width=\textwidth]{figures/palm_hub.png}
            \caption{\assetURL{PaLM} model \citep{chowdhery2022palm}}
        \end{subfigure}
        \hfill
        \begin{subfigure}[b]{0.475\textwidth}   
            \centering 
            \includegraphics[width=\textwidth]{figures/chatgpt_hub.png}
            \caption{\assetURL{ChatGPT API} \citep{openai2023chatgptapi}}
        \end{subfigure}
\caption{\textbf{Hubs in the ecosystem.} 
To demonstrate the value of \EG, we highlight \textit{hubs}: assets that feature centrally in that many assets directly depend on them.
(a) The Pile is an essential resource for training foundation models from a range of institutions (\eg EleutherAI, Meta, Microsoft, Stanford, Tsinghua, Yandex).
(b) P3 is of growing importance as interest in instruction-tuning grows, both directly used to train models and as component in other instruction-tuning datasets.
(c) PaLM features centrally in Google's internal foundation models for vision (PALM-E), robotics (PaLM-SayCan), text (FLAN-U-PaLM), reasoning (Minerva), and medicine (Med-PaLM), making the recent announcement of an API for external use especially relevant.
(d) The ChatGPT API profoundly accelerates deployment with downstream products spanning a range of industry sectors.
}
\label{fig:hubs}
\end{figure*}

% \FigTop{example-image-a-a}{1.0}{hubs}{\textbf{Hubs in the ecosystem.} To demonstrate the value of \EG, we highlight \textit{hubs}: assets that feature centrally in that many assets directly depend on them.} 

% \tldr{Assets}
To clarify the societal impact of foundation models, we propose \EG as a centralized knowledge graph for documenting the foundation model \textit{ecosystem} (\reffig{ecosystem-diagram}).
\EG consolidates distributed knowledge to improve the ecosystem's transparency.
\EG operationalizes the insight that significant understanding of the societal impact of FMs is already possible if we centralize available information to analyze it collectively. 

Each node in the graph is (roughly) an \textit{asset} (a dataset, model, or application).
Simply being aware of assets is an outstanding challenge: new datasets are being built, new models are being trained, and new products are being shipped constantly, often with uneven public disclosure.
While attention centers on the foundation model, the technical underpinnings and the social consequences of a foundation model depend on the broader ecosystem-wide context.
To link nodes, we specify \textit{dependencies}: in its simplest form, models require training data and applications require models.
Dependencies are technical relationships between assets  (\eg different ways of training or adapting a foundation model) that induce social relationships between organizations (\eg Microsoft depends on OpenAI because Bing
depends on GPT-4).
Especially for products, surfacing these dependencies
is challenging yet critical: products determine much of the direct impact and dependencies indicate the flow of resources, money, and power.

% \tldr{Documentation}
To supplement the graph structure, we further document each node with an \textit{ecosystem card}, drawing inspiration from other documentation frameworks (\eg data sheets \citep{gebru2018datasheets}, data statements \citep{bender2018data}, model cards \citep{mitchell2018modelcards}).
The ecosystem card contextualizes the node not only in isolation (\eg when was it built), but also with respect to the graph structure (\eg the license affects downstream use, data filters interact with upstream dependencies). 
Documenting applications concretizes societal impact: structural analyses (\eg which organizations wield outsized power) requires grounding out into how people are affected, which is directly mediated by applications. 
We also make explicit new challenges faced in documentation such as 
(i) \textit{maintenance} practices to synchronize the ecosystem graph with the ecosystem,
and (ii) \textit{incentives} that may inhibit or facilitate documentation. 

% \tldr{Discuss concrete EG + findings/analysis}
Given our framework, we concretely document the existing foundation model ecosystem through \numnodes nodes linked by \numedges dependencies and annotated with \numdocumentationentries metadata entries as of \releasedate.
This amounts to \numdatasets datasets (\eg \assetURL{The Pile},  \assetURL{LAION-5B}), \nummodels models (\eg \assetURL{BLOOM}, \assetURL{Make-A-Video}), and \numapplications applications (\eg \assetURL{GitHub CoPilot}, \assetURL{Notion AI}) that span \numorganizations organizations (\eg OpenAI, Google)  and \nummodalities modalities (\eg music, genome sequences).
To briefly demonstrate the value of \EG, we highlight the \textit{hubs} in the graph (\reffig{hubs}), drawing inspiration from the widespread analysis of hubs across other graphs and networks \cite[][\textit{inter alia}]{kleinberg1999hubs, hendricks1995hubs, franks2008extremism, van2013network}.
For asset developers, hubs indicate their assets are high impact; for economists, hubs communicate emergent market structure and potential consolidation of power; for investors, hubs signal opportunities to further support or acquire; for policymakers, hubs identify targets to scrutinize to ensure their security and safety.
In general, \EG functions as a rich interface and suitable abstraction to provide needed transparency on the foundation model ecosystem (\refsec{uses}).
We encourage further exploration at \websiteURL and are actively building \EG by collaborating with the community at \githubURL.