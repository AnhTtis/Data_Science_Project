\section{Ecosystem Graph}
\label{sec:ecosystem-graph}
The ecosystem graph that we have documented thus far is available at \websiteURL.
As of \releasedate, the graph is defined by \numnodes assets linked together by \numedges edges, with the assets amounting to \numdocumentationentries annotation entries.
We contextualize the current state of \EG in \reftab{assets}: the assets are built by \numorganizations organizations and break down into \numdatasets datasets, \nummodels models, and \numapplications applications.

\begin{table*}[htp]
\resizebox{\textwidth}{!}{
\begin{tabular}{lllll}
\toprule
\textbf{Name} & \textbf{Type} & \textbf{Organization} & \textbf{Date} & \textbf{URL} \\
\midrule
\assetURL{GPT-3} & Model & OpenAI & May 28, 2020 & \url{https://arxiv.org/abs/2005.14165} \\
\bottomrule 
\end{tabular}}
\caption{\textbf{Existing Assets.} 
As of \releasedate, we document these assets in \EG.}
\label{tab:assets}
\end{table*}

\paragraph{Views.}

\begin{figure*}
\centering
  \includegraphics[width=\textwidth]{example-image-a}
  \caption{\textbf{Graph view} for \EG as of \releasedate.
  }
  \label{fig:graph-view}
\end{figure*}

\begin{figure*}
\centering
  \includegraphics[width=\textwidth]{example-image-a}
  \caption{\textbf{Table view} for \EG, sorted by recency as of \releasedate.
  }
  \label{fig:table-view}
\end{figure*}

To visualize the graph structure of the \EG, we provide a simple graph interface shown in \reffig{graph-view}.
Users can zoom into specific regions of the graph to better understand specific subgraphs.
Alternatively, we provide an interactive table (\reffig{table-view}) to search for assets or filter on specific metadata, which can also be exported to a CSV. 
Users can include or exclude specific fields as well as sort by column (\eg on the organization or the artifact's size).
Clicking on an asset name in either the graph or table views will take the user to the associated asset card.

\paragraph{Asset cards.}

\begin{figure*}
\centering
  \includegraphics[width=\textwidth]{figures/asset-card.png}
  \caption{\textbf{GPT-3 Asset Card.
  \pl{way too small to read, I think we'll have to cut this off}
  } 
% The asset card for GPT-3 as an example. 
  }
  \label{fig:asset-card}
\end{figure*}

Each asset is associated with an asset card.
To navigate between adjacent assets, an asset's dependencies (upstream) and dependents (downstream) are linked to at the top of the page. 
In \reffig{asset-card}, we provide the asset card for GPT-3 from \assetlongURL{GPT-3}: each field includes a help icon that clarifies what the field refers to (see \reftab{all-properties}).
We highlight that the general design of the asset card is to centralize useful information, as can be seen in the abundance of the links, rather than to extensively replicate the information.

\paragraph{Implementation.}
On the back end, \EG is a collection of \texttt{YAML} files that store the annotation metadata against a pre-specified schema of fields that matches \reftab{all-properties}.
All aspects of asset selection are handled by the annotator in choosing what to specify in the \texttt{YAML} file: all specified assets are rendered.
For constructing the graph, the dependencies field is used to build edges between the graph: if a dependency is specified but no asset card has been annotated, a stub node is created in the graph.
Anyone can contribute (\eg adding new assets, editing existing assets) by submitting a pull request at \githubURL that will be reviewed by a verified maintainer of the \EG effort. 

\pl{I'd advocate for moving this section farther up, especially given that this is a pretty short section;
I think of the ideal structure of the paper as:
(i) general goals / principles,
(ii) concrete implementation,
(iii) implications, ruminations, aspirations, etc.;
I think currently we have a bit too much of (i) and (iii) mixed in together; and it'd nice to get (ii) farther up.
}

\pl{
Another way to say it:
we sometimes have text that describes - well, here are the nuances, and here's how things could go this way or that way, so here's why we did X; and sometimes it's better if we can just declare what we do, and then justify it (kind of like Theorem/Proof).
The core EG is very simple, and I think we should highlight and celebrate that simplicity and elegance; and then proceed to talk about the complexities.
It's like: first describe the rules of Go (simple), and then talk about all the strategies (complex).
}