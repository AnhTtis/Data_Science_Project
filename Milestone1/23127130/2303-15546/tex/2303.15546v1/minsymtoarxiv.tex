%
%. Now I am found
\documentclass[12pt]{article}
\usepackage{graphicx}
\usepackage{amssymb,amsmath,amsfonts,palatino,amsthm}
\usepackage{amssymb}
\usepackage{epstopdf}
%\usepackage{feynman}
\usepackage{slashed} 
\DeclareGraphicsRule{.tif}{png}{.png}{`convert #1 `dirname #1`/`basename #1 .tif`.png}
\setlength{\textwidth}{6in}
\setlength{\oddsidemargin}{0.5\paperwidth}
\addtolength{\oddsidemargin}{-.5\textwidth}
\addtolength{\oddsidemargin}{-1in}
\setlength{\evensidemargin}{\oddsidemargin}
\setlength{\topmargin}{-.311in}
\setlength{\headheight}{6.2pt}
\setlength{\textheight}{\paperheight}
\addtolength{\textheight}{-2.5in}
\newcommand{\f}{\begin{equation}}
\newcommand{\ff}{\end{equation}}
\newcommand{\blankline}{\vskip .3cm}
% This fixes the margins and page sizes
\setlength{\hoffset}{0pt}
\setlength{\voffset}{0pt}
\setlength{\topmargin}{-20pt}
\setlength{\headsep}{30pt}
\addtolength{\headsep}{-\headheight}
\setlength{\textheight}{9in}
\addtolength{\textheight}{-40pt}
\setlength{\footskip}{30pt}
\setlength{\oddsidemargin}{0pt}
\setlength{\textwidth}{6.5in}
%%%%%%%%%%%%%%%%%%%%%%%%%%%%%%%%%%%%%%%%%%%%%%%%%
%.  ffffffff
\begin{document}
\title{The path integral in energetic causal set models of the universe}
\author{Lee Smolin\thanks{lsmolin@perimeterinstitute.ca} 
\\
Perimeter Institute for Theoretical Physics,\\
31 Caroline Street North, Waterloo, Ontario N2J 2Y5, Canada}
\date{\today}
\maketitle
%\vfill
\begin{abstract}

I study several aspects of the path(st) integral we formulated in previous papers on energetic causal sets with Cortes and others.
The focus here is on quantum field theories, including the standard model of particle physics.

I show that the the theory can be extended to a quantum field theory,  cut off in momentum space.    Fields of spin 
$0, \frac{1}{2}. $ and $1$ may be included as in a perturbative treatment of the standard model.      

The theory is at first formulated in momentum space.      Under  certain conditions, spacetime can emerge in a semiclassical limit.  
The theory comes with a $uv$ cutoff in momentum space, $\mu$, hence that is also a scale for  lorentz invariance to break down.    
Traditionally,  $\mu$ is taken to be. a Planck energy, but we 
explore as a possibility making $\mu $ smaller.

% on the order of a  $ev$, ie the weak scale.  


\end{abstract}
%\newpage
\tableofcontents

%\newpage

%\makepaper

\section{Introduction} We continue the series of papers on the energetic causal set approach to quantum foundations , quantum gravity and
cosmology\cite{ECS1}-\cite{VVQM}.   Our focus in this paper is to extend the range of the theory to cut off formulations of (nearly) relativistic
quantum field theories. 

We will make several  assumptions, which shape our treatment of the theory.  Some of the motivation for choices made come from quantum gravity,
although the theory I present here is not (yet) a quantum theory of gravity.

\begin{itemize}

\item{}Usually we define a quantum gravity theory by fixing a uv cutoff, $\mu$ and then looking for fixed point in the space of couplings which the theory
may approach in a limit as $\mu \rightarrow \infty$.  This is thought 
to be necessary to restore symmetries or gauge invariances that are lost for fixed $\mu$.    

But, experimentally, to my knowledge, Lorentz invariance is only checked explicitly  up to
$\gamma = \frac{1}{1- \frac{v^2}{c^2}}  \approx 10^{12}\cite{howhigh,howhigh2}$.    Moreover, there is a preferred simultaneity for the universe as a whole.
These and other ideas suggest that we might formulate quantum gravity in a way in which both global and local lorentez are broken at some
intermediate energy scale,  allowing just
enough to preserve current experiments.   The fact is that there is a lot of room between the Planck scale and the weak scale, and the lower we can put $\mu$,
the more opportunity we can open up for cutoff scale quantum gravity.   The question then is then, just {\it  how low  }  can $\mu$ go?  

\item{}As proposed originally by Cohl\cite{Cohl1},  the histories of events we sum over in the model are those of the  standard model.  
There are not two sets of events, the causal set events and the standard model events. In the usual way of thinking about it, the first  set of events define the history of the quantum spacetime. Then the standard model particles and fields propagate in the background of events defied by the first set. 
But there is no need for two sets of histories, and two path integrals.
  The second set of histories, where we include the matter degrees of freedom, is already summing over a sufficient number of paths to define the spacetime.  We don't have to do that twice.

%\item{}The densities of events involving the diverse fields are taken from nature according to the standard model.    Normally, the causal set and CDT models define a quantum spacetime as the limit
%as a uv cutoff scale is taken to zero.  In the present model there is no limit taken, which is to say that all the symmetries are badly broke  already around a scale not too far above the weak scale. $\epsilon_0$.   

%It is true that if we want to define the amplitudes so that they turn out to be invariant under all boosts, no matter how large, we need to take the ultraviolet cut offs all to infinity.   But there is no need for this, as the data that checks boost invariance only goes up only
%to around the weak scale.  So our model only needs to be defined up to there.   When I say the model is only defined up to the weak scale, I mean that nothing happens or exists which is above that scale. This is not a cutoff, to be taken to zero, it is a part of the definition of the dynamics.

%%item{}None the less,  {\it this is a quantum theory of gravity, but it is a model in which gravity is mostly weak.} That is, our model spacetimes only have structure defined up to the cut-off scale.   

\item{} Fundamentally, our model is just defined on pieces of momentum space.  There is no spacetime, quantum or classical.  Can we presume that
spacetime instead, emerges at around the cutoff scale?  How low can we make that scale?

Classical spacetime emerges, carrying a scale which is $l_{o} = \epsilon^{-1}$.

\item{}. Our theory comes with a tiny number,   
\f
\epsilon _0 \approx    {l_{Planck}}{\epsilon}
\ff

There is no avoiding this number; after all it is in the data.  Its origin is unknown!  The hierarchy problem is re-expressed but remains unresolved.  
But let us use it in the theory somewhere it will do us some good.

If we can make the fundamental cutoff scale, somewhere between the Planck scale and  the weak  scale, we have to explain how quantum gravity gets to be so weak.    

\item{}The very tricky question is whether there are possible observations that might be made in a theory with a weak scale cutoff that could indicate
that there are not degrees of freedom far above that scale, in energy units.   This is the converse of the question of whether current experiments, made below the weak scale,
could establish the existence of modes of a smooth spacetime, far above the weak scale.   In other words could we already rule out the scenario proposed here.
Some relevant papers  are \cite{howfar,howfar2}.

\item{} The way we treat energy, momentum and spacetime in this matter is inspired by how they are treated in {\it relative locality }models
of quantum gravity phenomenology\cite{RL1,RL2}.  .   

Usually we take spacetime ${\cal M}$ as fundamental, while momentum space is an aspect of the cotangent bundle over spacetime.  
This means that there is a momentum space for every pointy of spacetime.   Here we reverse this, we take momentum space
as fundamental.   Spacetime is not defined at first, then it emerges from the tangent space of momentum space.  This means that
if several particles interact, each carrying a different momenta, they also move in different spacetimes.  
  This means that there is
a single momentum space, but there is, strictly speaking a spacetime for every point in momentum space.   In [
\cite{RL1,RL2} we showed how this works,
giving rise to a beautiful picture of quantum gravity phenomenology, emerging with the emergence of space. 

\end{itemize}

%Many theoretical studies of the path integral in quantum gravity postpone the matter contributions and focus only on the path integral in pure quantum gravity.   They then assume, explicitly or implicitly, that the path integral in our universe is dominated by the gravitational sum.  That is to say,  from the weak scale. down to the Planck length they assume that the path integral is dominated by the gravitational terms and factors, which. are growing in proportional to the matter contributions.

Down at the Planck scale, the gravitational terms are order unity, which is to say strongly 
coupled, whereas due to asymptotic freedom, the gauge matter couplings are practically negligible. 
So it make sense to think about a strongly coupled gravitational plasma around the plank scale and ignore matter there.  

But if the fundamental cutoff scale is closer to the weak scale than the Planck scale, then there is nothing that forces us to solve this very hard problem, of quantum gravity in the Planck scale regime.

%pling\footnote{I owe a great deal to John F. Donoghue. \cite{Donoghue},  who has made
%a different, but closely related point.  }.




%\begin{itemize}


%\item{} As opposed to what is the case in many models, we will assume that our model is asymetric.  There is no more symmetry in the mo0del then is forced on us by the data.     

%\end{itemize}
%\begin{center}


\section{The dynamics}

\subsection{Energetic causal dynamics}

Let us first review the dynamics of the energetic causal theory, in a generalized form.

Our model is defined by a path integral, from a fixed initial state  to a fixed final state.    Those states are labeled by representations of the Poincare group,
$p_a^{cutoff} $
up to a fixed cutoff $\mu$.    Dynamics is imposed by constraints; which are represented in the path integral by
delta functionals.  

The theory thus computes probability amplitudes 
 in momentum space.
 
We preserve rotational invariance in a specified frame, which is to say; that we include in our sums over momentum modes for momentum $| p_i | \leq \epsilon$.
\f 
Z= < p_a^{i}  |  \sum_\Gamma  \Pi_{ \Gamma}   ( \int dp_c^{p^{max}}  \delta ({\cal P} (p))  \delta ({\cal C} ) 
%e^{\imath  {\cal W} (p)   }
 |   q_b^{j } >   
 \label{past}
\ff
where the constraints are
\f
{\cal P}_a^I = \sum_{k \in node K}  p_a^{k} =0
\ff
\f
{\cal C}_K^L = p_{a K}^L h^{ab} p_{b K}^L =0
\label{C=0}
\ff


It will be important to understand how the edges and vertices are produced in the above action.    

The interactions come only from the ${\cal P}_a = 0$ constraints.   To see how this happens let's remove the terms in these constraints
\begin{eqnarray}
Z^{-{\cal P} }  &= &  < p_a^{i}  |  \sum_\Gamma  \Pi_{ \Gamma}   \int dp_c^{p^{max}}   \delta ({ \cal C }) | p_a^{i}  >  
\nonumber \\
 &  = &   < p_a^{i}  |   \sum_\Gamma  \Pi_{ \Gamma}   \int dp_c^{p^{max}}   \int d{\cal N}^J_K.  e^{\imath  \sum_{J<K}  {\cal N}^K_L {\cal C}_K^L    }  | q_b^{j } >   
\end{eqnarray}

Since the ${\cal C}_J^K$ constraints act each on one momentum, there is no coupling of the different momentum to each other.
%  It is more correct to say that the integral is over momentum space to the power of the number of particles.  
The only place interactions are introduced
in our model is  by the conservation laws imposed by delta functions in the measure of the path integrals. 

\section{Generation of  the causal process by the path integral}

Our diagrams represent a causal process.   They are embedded neither in spacetime nor in momentum space.   Each diagram contains a
partial ordering amongst its vertices,.   

\begin{center}
\begin{figure}[!h]
   % \includegraphics[scale=0.4]{except.jpg}
\includegraphics[scale=.09]{fun11.jpeg}
%\end{center}
  \caption{Two basic moves, one after another}
  %\label{cpvscon}
\end{figure}
\end{center}




\begin{figure}[!h]
%\begin{center}
   % \includegraphics[scale=0.4]{except.jpg}
\includegraphics[scale=0.09]{fun13.jpeg}
%\end{center}
  \caption{Another basic move.  In these diagrams we see described the partially ordered casual structure of the nodes.}
  %\label{cpvscon}
\end{figure}



\begin{figure}[!h]
   % \includegraphics[scale=0.4]{except.jpg}
\includegraphics[scale=.2]{fun12.jpeg}
%\end{center}
  \caption{Another basic move.}
  %\label{Fig3}
\end{figure}

A diagram in our theory is a contribution to the past integral \ref{past}.   We see that the diagrams are made up of nodes, each one of which has 2, 3 or 4 
inputs or outputs.  They are connected by ppropagators.  There are only two kinds of particles, which are chiral fermions and real scalar particles
In a later section, we show how to extend the  notation to incorporate all the fields of the standard model
ie lptons, quarks, W bosons, photons, gluons and Higgs
particles.        
%We have at each time $\tau$ a moment in the propagation (or growth) of our system, 

A diagram starts off with a specific number of free particle states, which are eigenstates of the momentum operators.    Each free particle represents
a propagator, constructed by summing up an infinite number of two point functions, as in equation (\ref{sumup}) . 

%starting off, as we said, with an input state consisting of a finite number of unconnected particles, with definite momenta. 





\subsection{The two point function}


At each stage in the causal order, you see a static diagram, which contains, first, the initial incoming  states, then a diagram built
on them. 

Each incoming particle is represented by an appropriate line, ending at a node (these initial nodes have just a single past input each.)

After one has performed the sum
\f
{\cal J}   = \delta^I_{I+1}    +   (p_a )^2   p^2 + p_{a}^2 p^4 + \ldots = \frac{1}{1 - p^2 }
\label{sumup}
\ff


%The same strategy works in the case of the standard model. That is, the partition function can be written in terms purely of momentum space.  
To extend the path integral () to the standard model, we need to  first derive expressions for particles to propagate.  These come  from summing up all the two
point functions. This is illustrated in Equation (\ref{sumup}). 

% FIgure !?           An in initial. state.


\subsection{Causality and the $\imath \epsilon $ rule}

There is only one place that position spacetime is referred to in the path integral, which is that we must recover the  $+ \imath \epsilon $ prescription in the
propagators, as that refers to positive frequency goes to the future and negative frequency goes to the past.  
These $+ \imath \epsilon $ factors are imposed on the Fourier transforms of our momentum space amplitudes; if we fail to replicate
the effects of this, the Feynman path integral is not reproduced.

However, the causal structure is already there in our pure momentum space factors.  The reason is that we are required to go through the path integral
in a causal order.  Our problem is just to make sure that the factors coming from this, correct ordering,  is preserved when we write out the Fourier expansion
of the expansion of our path integral (\ref{past}).
So we see that the causal structure of each amplitude is preserved in the embedding of the Fourier transformed map into spacetime.
\f
%\begin{eqnarray}
Z^{SM}    =  < p_a^{i}  |  \sum_\Gamma  \Pi_{ \Gamma}   \int dp_c^{p^{max}}  \prod_{vertices}   \delta ({ \cal P }_{a \ vert})   e^{\imath {\cal W}^{top}    } | p_a^{i}  >  
\ff

% &  = &   < p_a^{i}  |   \sum_\Gamma  \Pi_{ \Gamma}   \int dp_c^{p^{max}}   \int D{\cal N}^J_K.  e^{\imath  \sum_{J<K}  {\cal C}_K^L    }  | q_b^{j } >   


\begin{itemize}

\item{} Our next step is to extend the constraint that gives the scalar particle its transformation properties, in order to introduce the action of the $SU(2)_{Left}$ group.
Instead of a single scalar field $\Phi $ we have a doublet from the electroweak  group $H$.  This is represented by
\f
    {\cal C}^{H} = H^{\dagger}_{A'} ( p_a   g^{ab} p_b )  \eta^{A'A}  H_{A} +V
\ff
where
\f
V= -\frac{\mu^2 }{2} H_{A}  \eta^{A'A}  H_A  +    \frac{\lambda}{24}   [ H^{\dagger}_{A'}   \eta^{A'A}  H_A ]^2
\ff
is the scalar potential.

Again we sum over all two point functions for the Higgs.
A Higgs has an arrow next to its edge.
 
\item{} Next in complexity is the pure fermion, sector,  which is to start with the form,
\f
S^{\psi} = \sum_{I}^{\ K}  \Psi^{\dagger A^\prime } (p)  \sigma_{A^\prime }^{a B}  p_a  \Psi (p)_{B}
\ff
We can sum over an infinite number of two point diagrams to construct the propagator, as we see in  eq(\ref{sumup}). 
A left handed chiral fermion has an arrow on to its edge.


\subsection{The vertices}

To construct the vertices of the standard model, one needs to expand the path integral.

\begin{figure}[!h]
   % \includegraphics[scale=0.4]{except.jpg}
\includegraphics[scale=.05]{creature.jpeg}
%\end{center}
  \caption{The first few moves in an evolution beginning with one initial chiral fermion and two bosons.}
  \label{FIg4}
\end{figure}




%\item{}  We start with the original $ECS$ model
%\f
%S = \sum{\Gamma} \Pi_{I \in {node }} \delta  ({\cal P}_a  ) \sum_{I < J } \delta (  {\cal C}_{particles} )   e^{\imath {\cal  W} }
%\ff  
%This is an interacting scalar theory with one component.   

It is interesting also to note that locality in the emergent spacetime is a consequence of the linearity of the conservation laws. 
It is important that all these processes satisfy the energy- momentum  conservation rules; this is accomplished by inserting the 
appropriate delta functions downstairs, in the measure of the path integral.


The graph then grows in a series of steps.  In each step one of several things happen. 
\begin{itemize}

\item{} One, two or three edges are added moving to the future of a free present node.   (See Figures 1 and 2 ). 

The nodes correspond to events.  They are time ordered, in the sense that the causal, partial order is constructed according to the
partial order of adding nodes to the graphs.

\item{} Two present nodes come together and become a single node with two or more input edges.   (See Figure (\ref{Fig3} ).

\item{} Nodes are in the present if they may be built on by the assignment of further nodes to them.  

%\item{}The causal structure in the graphs translated into Fourier transform is always preserved, as we have just noted.


\item{}As in the expansion of the "past" integral in the original energetic causal set models, a slight modification of our
model allows us to  interpret it in the context of
a presentist view of time.  We may put a restriction on nodes to limit future growth from it, for example no more than $m$ nodes directly
to its future.  When a mode saturates this restriction, it can no longer be a pathway to the future; and we move it to a set of  {\bf past nodes}.
Once a mode is in the past, it cannot move back into the future\cite{ECS1}, \cite{ECS2}.

We feel justified in saying that only present nodes are part of the real, but we recognize this may be a matter of taste\cite{ECSinthefuture}.

In ordinary causal set models, we cannot impose such a restriction because it makes demonstrations of lorentz invariance impossible\cite{CS1}.
We have no need to preserve lorentz invariance at arbitrary distances from a given point, so we easily live with the dropping of Lorentz
invariance.

In the diagrams  drawn here there are two kinds of edges, those corresponding to chiral fermions, and those corresponding to real
scalar fields.  The chiral fermion modes never end, and they preserve a chiral fermion number.   The real scalar modes can
start or stop at a node.




\end{itemize}


%\section{new patterns}

%\end{itemize}
%\f
%S^{first} = a | \hat{Z}_{initial, final }   \sum_\Gamma \sum_{p_a I}^J 
%\ff


In the simple model we are notating, the chiral fermion may be in one of four momentum eigenstates, which correspond to parity and charge.
while the scalar may be in one of two states.  This is sufficient to show the elementary logic behind the $CPT$ theorem.  
The theorem still works when we increase the complexity to incorporate the standard model. 

\subsection{Discrete symmetries and $CPT$ theorem.}

There are four charged states that a particle may find itself in, with respect to the present:  
$Q=1,-1, \pm= + $ or $-$.  We can write there four states like
$\frac{\pm 1}{\pm}= \frac{Q}{\pm}$.  On the other hand, if we have a charge neutral state, it only can change the direction of its arrow, so it has two states
$\pm = \frac{0}{\pm}$.   
We define the following three discrete symmetries:

Parity. ${\cal P}$:
\f
{\cal P}  Q = Q,     {\cal P}\pm = -\pm. ,  \ \ \ \ \ \ \ \ ie \  \   \ \   {\cal P}:  \frac{Q}{\pm} =\frac{Q}{-\pm}. 
\ff
Charge conjugation:  {\cal C}:
\f
{\cal C}: {\cal Q} = - {\cal Q}; \ \ \ \ \ \ {\cal C}: \pm = \pm.  \ \ \ \ \ \ \ \   ie \  \   \ \   {\cal C}:  \frac{Q}{\pm} =\frac{-Q}{\pm}. 
\ff
Time reversal {\cal T}  
\f
{\cal T}: Q = -Q , \ \ \ \ \ {\cal  T}:  \pm = \mp.   \ \ \ \   ie \  \   \ \   {\cal T} \  \frac{Q}{\pm} =\frac{-Q}{-\pm}.
\ff

It is easy to check that these transformations are not trivial on both kinds of states we are working with.

On the other hand $CTP$ is the identity on our states,
\f
{\cal T}{\cal C} {\cal P}= {\cal I}
\ff


%\begin{figure}[!h]
   % \includegraphics[scale=0.4]{except.jpg}
%\includegraphics[scale=.05]{CPT.jpeg}
%\end{center}
 % \caption{The $CPT$ theorem}
  %\label}
%\end{figure}

\subsection{The gauge fields}


\item{} We next introduce the the gauge field, $SU(2)_{Left}  $.
\f
p_a \phi (p)  \rightarrow {\cal D}_a \phi (p)  ) =  {\partial_a + A_a} \phi (p)  
\ff


\f
Z= < p_a^{i \ldots } |  \sum_\Gamma     ( \int dp_c^{p^{max}}  (  \int p^{max}  dp_d   )     d ( dp )  d (dr) d(ds)  \delta  (p+ w - q)   |   q_b^{j \ldots } >   
\ff     
Crucial to our theory is the incorporation of the gauge fields to turn global symmetries into local symmetries.
\f
p_a  \rightarrow D_a = p_a + \hat{A}_a
\ff
\item{} The next step is to introduce the right handed lepton fields.

\item{} Next we follow the same procedure to introduce the $SU(3)$ colours of quarks, along with an  $SU(3)$  gauge field.

\item{}Next, we introduce the $U(1)$ gauge fields. 


%\item{}. 



%\item{}

\item{Next add the flavor symmetries}

We multiply the fermion fields, $\Psi $ and Higgs fields, $H $ by an $SU_"LEFT" (2)$. by putting the in fundamental representations of $SU_{Left}(2)$
\f
\phi \rightarrow H^{\  B}.  
\ff
The Hamiltonian constraint just changes by  putting it in a fundamental of $SU_{left} (2) .$,

\f
{\cal C} =  p^2 \rightarrow ,    \ \ \ \     | p_a p_b  I ^2 = p_b  g^{ab}  p_b
\ff

\item{Add colour symmetry. }
\f
P_{1  a}^b = {p^2} (  h^{a}_{b} + \frac{p^a p_b}{p^2}  )
\ff

%\item{} Add in $SU(2)_{Left} $ symmetry by extending the scalar fields to live in an $SU_L (S )$ representation
%\f
%{\cal P}_0 \rightarrow {\cal P}_I^{\ J} = 
  %{\cal N_0 }_{A}^{\ B} {\cal C}_{A}^{B} = \delta_A^{\ B}  p_{aA}  p_{bB} \delta^{AB}
%\ff

%\subsection{Colour}
%\item{}  

 \item {Complete the gauge coupling}

Each trivalent and higher vertex involving gage fields comes from completion of the covariant derivative.  

%\item{}: 

\item{Impose the uv cutoffs  in the internal legs.}

Note that the integral will not be exactly gauge invariant, due to the cutoff. 

%\end{enumerate}

\end{itemize}

\section{Comments}

After we have put in the various interaction terms, what do we have?   We may note that we have put into the action all the terms
in the action for the standard model, as we would write it in momentum space.  Let us suppose we write these $N_{SM}$ terms as,   ${\cal L}_\alpha$.
\f
Z^{SM}= Tr    < p_a^{i}  |   \sum_\Gamma  \Pi_{ \Gamma}   \int dp_c^{p^{max}} \delta_{interactions} (\sum_{K a} ^ F ( p_{a L}   ) )  e^{\imath  \sum_{\alpha=1}^{N_{SM}}
{\cal L}_\alpha }
|p_b^{j } >  
\nonumber 
\ff
We can  make a diagrammatic expansion of by expanding around the bivalent terms, as usual.  We may first note that the
resulting series are not exactly the Feynman diagrams, because they do not have all the gauge invariance of Feynman diagrams.  For example, the cubic
vertex, gotten from grafting a gauge field only a quadratic propagator may have an independent coupling constant ; there is, it seems, no principle that ties the value
of the cubic graph to that of the quadratic graph.  

{\bf Dominance of gauge theories by gauge invariant terms at low energies.}

We are reminded of a informal idea in lattice gauge theory, sometimes called "multi-critical dynamics or "random dynamics",
which assumes that nature is governed by a theory with a finite, but large, uv cutoff, where it is described by a combination of
lattice gauge theory dynamics,  which are a random mixture of gauge and non-gauge invariant terms.  Then authors then claim that
the non-gauge  invariant couplings go away, when compared with the gauge invariant coupling such that in the infrared limit
the theory is governed by the gauge invariant sub-theory, alone\cite{Shenker, Nielsen}.     


\section{Conclusions}

We close with a few remarks on on--going and future work.

\begin{enumerate}

\item{} We have now explored tentatively several corners of the framework of energetic causal  sets.  LLet us put it in some oersoective.

Counting the cosmological constant, $\Lambda = \frac{c^2}{R^2}$ as a fundamental observable, we have four fundamental, dimensional observables:      $\hbar, G, R, c $.  This is, famously, one too many.   But actually, there is one more: $N$,  the number of degrees of freedom.

Within any  proposed framework for quantum gravity, are then at least ten ways of reducing the description of our world to a simpler world, that still must be consistent, because it follows from a consistent set by taking one or more of the parameters to zero or infinity.

$ECS$ are one such proposed framework,  how are we doing?
First of alll, we have the well known corners:

\begin{itemize}



\item{Newtonian physics}

\item{Newtonian physics}
\f
c \rightarrow \infty,   \ \ \ \ \   \hbar \rightarrow 0, \ \ \ \ \ \ R \rightarrow \infty
\ff

\item{}Classical relativistic physics
\f
  \ \ \ \ \   \hbar \rightarrow 0, \ \ \ \ \ \ R \rightarrow \infty
\ff
\item{} Classical General Relativity
\f
\ \ \ \   \hbar \rightarrow 0, \ \
\ff

As the subject of Energetic Causal has developed we have scanned a ranges of theories.  In our first papers we studied
models of spacial relativistic particle based on a mechanism for the emergence of flat spacetime\cite{ECS1}-\cite{ECS4}.
Later we studied models of hidden variables in which we studied how non-relativistic quantum mechanics
emerges in a   limit $N \rightarrow \infty$
\cite{QMMV, VVQM}.  

%ers we studied a regime in which $G \rightarrow 0  $ and $c \rightarrow 1$. while $R \rightarrow \infty$.

%Then in [] we studied a regime in which 
%\f \rightarrow \infty \ \ \ \ \ \ \ \ \ R \rightarrow \infty \ \ \ \ \ \ \ \hbar \rightarrow \frac{1}{N}. \ff
These limits gave us non-relativistic quantum many-body theory.

In this paper we have broken through to the regieme  of $QFT$, although it appears we only recover lorentz invariance at low energies, compared to a 
fixed $uv$ cutoff.  

Clearly we have still some way way to go, most importantly we have to get off the  $G=0$ axis, to get gravity into the game.


One way to do this is to consider the $RL$ regieme in which $G$ and $\hbar $ both go to zero, with their ratio fixed.
\f
G \rightarrow 0, \ \ \ \ \hbar \rightarrow 0  \ \ \ \ \  \frac{\hbar}{G}  =   M_{Planck}^2. = const. 
\ff
\item{}Here we studied the case in which momentum space is flat, but the formalism we've developed may be easily extended to the
case. of non-linear momentum spaces. 
One way to do this is to deform the metric of momentum space in the hamiltonian constraints (\ref{C=0} )this forces new interactions amongst the
particles, as discussed in \cite{RL1}.  Whether this can be connected to a dynamical curvature on spacetime is
presently unknown.





\end{itemize}

\end{enumerate}

%\section{Conclusions}
\section*{ACKNOWLEDGEMENTS}

II would like to thank Stefano Liberachi,   Laurent Freidel,  Clelia Verde, Ted Jacobson, Marina Cortes,  and Joao Magueijo for important conversations.  I am grateful to
Kai Smolin for the figures.
 This research was supported in part by Perimeter Institute for Theoretical Physics. Re- search at Perimeter Institute is supported by the Government of Canada through Industry Canada and by the Province of Ontario through the Ministry of Research and Innovation. This research was also partly supported by grants from NSERC, FQXi and the John Templeton Foundation.

\begin{thebibliography}{99}

\bibitem{Cohl1}Cohl Furey, Notes on algebraic causal sets, unpublished notes (2011); Cambridge Part III research essay, (2006).

\bibitem{Donoghue}John F.  Donoghue,  {\it Quantum general relativity and effective field theory}, arXiv:2211.09902.   

\bibitem{ECS1}Marina Cortes, Lee Smolin, The Universe as a Process of Unique Events, arXiv:1307.6167 [gr-qc], Phys. Rev. D 90, 084007 (2014).


\bibitem{EC2}Marina Cortes and Lee Smolin, Quantum Energetic Causal Sets,  arXiv:1308.2206 [gr-qc], Physical Review D, volume 90, eid 044035.


\bibitem{ECS3}Marina Cortes, Lee Smolin, Spin foam models as energetic causal sets. arXiv:1407.0032, Phys. Rev. D 93, 084039 (2016) DOI: 10.1103/PhysRevD.93.084039.

\bibitem{ECS4} Marina Cortes and Lee. Smolin,   {\it Reversing the irreversible,  from limit cyclres to emergent time symmetery},  arXiv:1703.09696v1 , 
https://doi.org/10.48550/arXiv.1703.09696
Phys. Rev. D 97, 026004 (2018)
Related DOI:  https://doi.org/10.1103/PhysRevD.97.026004

\bibitem{VVCS} Lee Smolin,   {\it Views, variety and celestial spheres}, arXiv:2202.00594.

\bibitem{VVQM} Lee Smolin  , {\it Views, variety and quantum mechanics}, arXiv:2105.03539  

\bibitem{DD}. Lee Smolin,      {\it The dynamics of difference}.    arXiv:1712.04799 ,    
doi
10.1007/s10701-018-0141-8.
 
\bibitem{QMMV}Lee Smolin, {\it. Quantum mechanics and the principle of maximal variety,}   ArXiv:1506.02938,  
doi
10.1007/s10701-016-9994-x

\bibitem{RL1}G. Amelino-Camelia, L. Freidel, J. Kowalski-Glikman and L. Smolin, {\it The principle of relative locality,} Phys. Rev. D 84 (2011) 084010 [arXiv:1101.0931 [hep-th]].

\bibitem{RL2}  Laurent Freidel, Lee Smolin, Gamma ray burst delay times probe the geometry of momentum space,  arxiv:hep-th/arXiv:1103.5626:
A.E. McCoy, Lee Smolin, Gamma Ray Bursts, The Principle of Relative Locality and Connection Normal Coordinates,  arXiv:1201.1255 [gr-qc].

\bibitem{CS1} Luca Bombelli, Joohan Lee, David Meyer, Rafael D. Sorkin, Spacetime as a Causal Set, Phys.Rev.Lett. 59 (1987) 521-524.

\bibitem{Shenker}Eduardo H. Fradkin, B.A. Huberman, Stephen H. Shenker
{\it Gauge Symmetries in Random Magnetic Systems}
(Cornell U., LNS), Phys.Rev.B 18 (1978) 4789
DOI: 10.1103/PhysRevB.18.4789
Report number: SLAC-PUB-2112

\bibitem{Nielsen}Z. Koba, ]H. G. Nielsen, {\it Reaction amplitude for n-mesons a generalization of the Veneziano-Bardakci Ruegg-Virasoro model} 
Nuclear Physics BVolume 10, Issue 4, 15 May 1969, Pages 633-655

\bibitem{ECSinthefuture} L Smolin, C Verde The quantum mechanics of the present, arXiv preprint arXiv:2104.09945,

\bibitem{howhigh}Stefano Liberati,   {\it  Tests of Lorentz invariance: a 2013 update},
  arXiv:1304.5795, 
doi 10.1088/0264-9381/30/13/133001


\bibitem{howhigh2}Ted Jacobson, Stefano Liberati,David Mattingly
Lorentz violation at high energy: Concepts, phenomena and astrophysical constraints
Published in: Annals Phys. 321 (2006) 150-196 : astro-ph/0505267 [astro-ph]

\end{thebibliography}

\end{document}


\end{document}


In particular:
\begin{enumerate}

\item{}{Minimal translation symmetry}

Most cosmologists begin by assuming that the universe has some global translation symmetry, which is broken to some finite extant by observations.
We instead assume there is no global translation symmetry, exact at any scale.  Thus the universe has at large scale, a preferred reference frame, and  a preferred moment of time and only approximate spatial symmetries.   

\item{} However, there may be finite regions of limited approximate translation symmetries, extending  in time and space for finite parameters.
One relatively simple scheme for symmetry is given by figure 2.  The universe has 

\begin{eqnarray}
Length  &   L (t)      & \delta (t)  
\\
Time & T=       & \delta \tau (\tau )
\\
\delta \theta  \approx \delta \theta  
\end{eqnarray}

This is a universe in which for $\rho < \rho_0$, and for a range of $ \theta $ down to $\delta \rho > 1/200 $  we have an approximate 
Symmetry.
$| \rho(x)  - \rho(x') |  \approx $.


Now it is often more elegant and powerful to speak in terms some mathematically exact form of symmetry.
But in  most applications of dynamics to the real world,  this is what we actully mean when we say we have an approximate symmetry.
See Figure 1, for examples of such symmetries.
%\item{}Possible rotation symmetry.}


\end{enumerate}

\subsection{Limited Lorentzian symmetries}

Let us consider a universe with arbitrary infrared boost symmetry around an event, $\rho$.


\section{Approximate symmetries}


Let's contrast this with universe $\bf B$, a universe with arbitrary symmetry around a timeline line $t^a$.  



\begin{enumerate}

\item{}We consider a breakdown of Lorentz symmetry generated by $U_{\lambda, L}^{A_{short,   long}}$ where 
The zero points $U_{0, 0}^{0,0}$  are given by a particle in a universal rest frame.

\item{} Breakdown of spatial translations at short distances

\item{} Breakdown of  spatial translations at long distances

\item{Breakdown of boost symmetries}

\item{} We consider that boost symmetries hold from 
$\gamma $  up to $\gamma^{top} $, but no higher  than  $\gamma = 10$ .
\item{}We consider that boost symmetries do not how below $\gamma_\gamma < \gamma^{bottom} $.
\end{enumerate}

We then postulate that the $'propagating state"$
\f
| propagating  state (K ) > =U_{trans } U_{d.{rot} \cdot }|o> 
\ff

Has a depth, $d$ of a few, at the relativistic edges of the cloud.   

Note that the propagation is always in the future direction.   Also no particle


\section{Figures}


\begin{figure}[!h]
  \begin{center}
  \subfloat[]{\label{braided}\includegraphics[scale=0.4]{braided.jpg}}
  \subfloat[]{\label{twisted}\includegraphics[scale=0.4]{twisted.jpg}}
\end{center}
  \caption{borrowed:  Equivalence move by which braids are equivalent to twists}
  \label{braidntwist}
\end{figure}






\end{document}

%$ \sqrt{\eta_{ab}} $


But let us note that there is then a puzzle:   Let us assume that we work with a fixed cutoff, $M$, but integrate over all the fields, gravitational and not.   Then we are interested in the
effective field theory for the standard model, including gravity.
\f
e^{\imath {\cal W}[\hat{\phi}, \hat {g}_{ab}  , M]}  = \int d \delta g_{ab M }  d\delta \phi_M e^{\imath S [ \phi = \hat{\phi} + \delta \phi,   g_{ab} = \hat{g}_{ab} + \delta{g}_{ab} ) ]
\nonumber
\ff

Now, let us compare this with the effective action, in a theory in which the metric is not dynamical
\f
e^{\imath {\cal W}^{nograv}[\hat{\phi}, \hat {g}_{ab}  , M]}  =  d \delta \cancelb{g_{ab M}} \int d\delta \phi_M e^{\imath S^{nograv} [ \phi = \hat{\phi} + \delta \phi,   g_{ab} = \hat{g}_{ab} + \delta{g}_{ab} ) ]
\nonumber
\ff

______

To compute the  probability for a in state to turn into an out state we proceed as follows.  

\begin{enumerate}

\item{}  Choose the in and out states in momentum space

\item{} Write down a finite set of processes that take $< in |$ to $ | out >$ using diagrams such as, [], obtained by an expansion of the past integral (\ref{past} to some chosen order of approximation.  




++++


I) two point functions:  


\f
P_{Total} = P_{ 0} \oplus  P_{\frac{1}{2}}^{AB} \oplus P_{1 c}^{d}
\ff
where
\f
P_{ 0} = {\cal N_0 }{\cal C}
\ff
\f
P_{\frac{1}{2}  A }^B = \sigma_{A}^{a \  B}  p_a
\ff




\section{Figure tryouts}



\begin{figure}[!h]
  \begin{center}
\includegraphics[scale=.4]{111.jpg}
 \begin{center}
    \includegraphics[scale=0.4]{except.jpg}
  \end{center}
\end{center}
  \caption{Figure 1}
  %\label{cpvscon}
\end{figure}


\begin{figure}[!h]
   % \includegraphics[scale=0.4]{except.jpg}
\includegraphics[scale=.3]{scan2.jpeg}
%\end{center}
  \caption{Figure 2}
  %\label{cpvscon}
\end{figure}



\f
R \rightarrow \infty,  \ \ \ \ \ \ \ \ \ R \rightarrow 0
\ff

gives us two worlds we might think we understand.
\f
\hbar \rightarrow 0
\ff
gives us classical relativistic  physics, which is from the present point of view, a known entity.

Similarly 
\f
c \rightarrow \infty       \ \ \ \ \ \ \  c \rightarrow 0.   
\ff
Gives us each a world we think we believe we understand, in the case then the second, or work out,  in the first.

In a number of places we studied limits in which



Then there are the relative locality limits

\f
\lim_{\hbar \rightarrow 0}  \lim_{G \rightarrow} 0     \lim_{Planck \rightarrow constant}   \hbar G , 
\ff 
keeping $\mu^2 = \hbar G$ fixed.

This is a limit that picks out certain high energy behavior.

Now think about limits that hold an acceleration
\f
a_0=  {c^2} \sqrt{\Lambda} 
\ff 
fixed, 
while we take either
\f
c \rightarrow \infty,      \mbox{.  and. }   \Lambda \rightarrow 0 
\ff
 or the reverse
\f
c \rightarrow 0,      \mbox{.  and. }   \Lambda \rightarrow \infty 
\ff
In both cases, we can choose any limit in the $\hbar, G$ plane.

picking out $a_0$ picks the MOND acceleration.

\end{enumerate}



If this is the case then we may expect our theory to agree at low energies with the perturbative $QFT$ of general relativity, coupled to the
standard model. 

But there is more than one gauge invariant theory our theory may flow to.  Lets see what happens if we make a 
RG picture of how the couplings on 

 flow in the space of couplings.   We start with our theory as posited in its initial diagram, but
with random couplings.   If we leave any relevant local relevant couplings, the renormalization group will insert them, step by step as we go.
Given our idea that the theory is most likely to choose infrared couplings which lead to the largest infrared symmetry group, these
couplings may be chosen whether or not they are in the original list of couplings\cite{Shenker,Nielsen}'.

The two strongest couplings are the cosmological constant and Newton's constant.  Even if we have left them out of our initial list of
couplings, unless they are blocked initially by some extra symmetry from mixing with the couplings which we do introduce, they will appear and couple as we take the 
infrared cutoff smaller and smaller-and ultimately to zero.   

Now many $QFT$'s are Lorentz and Poincare invariant, and hence have symmetries that will prevent the emergence of gravity from such a process.
But our theories do not have these symmetries - because they are formulated in momentum space.  
The meaning of this is that our theory is a quantum theory of gravity.   Indeed some cut-off QFT treated in this way may be quantum theories
of gravity.    


\subsection{On symmetries}

%\begin{itemize}

%\item{}

Note that our model fails to be Lorentz invariant, because we are missing term necessary to complete the sums over all boosts.
Indeed, breaking Lorentz invariance is unavoidable, given the finite cutoff.   Instead, we make sure that the quadratic modes are complete, and that 
cubic nodes in the interactions arise from quadratic terms in the two point functions to leading order by the standard
substitution.

%\item{} 
We see that this cutoff version on the standard model is set in momentum space.
We impose an ultraviolet cutoff in momentum space, which is no higher than it has to be to reproduce nature (we assume is around
$10^{10} Gev = \epsilon_{weak}$.)  
 
%\item{} Several different heuristics govern which gage invariant couplings dominates.

%\end{itemize}


\subsection{On scales}

%\subsection{Scales}

It is natural to think that the special scale $\omega_0$ that is the doorway from the background dependent scales below which we can apply 
conventional QFT is the Planck scale, $~ 10^{-33}$ of a centemeter.     But this opens up a gap of some $10$ orders of magnitude between
the scales of elementary particles, where we can begin to make detailed observations at the fundamental Planck scale. 
So what if we take a different point of view and declare that the quantum gravity processes are the same as the elementary particle processes.

%The only processes the theory allows are that are both carrying quantum numbers of the standard model and general relativity.  See Figure ??.

That is to say, let us make a path integral quantization of the standard model, ignoring gravity.

How many terms are these?  Regarding the standard model we want no more than allow the standard model to make sense of the present observations.
Another way to say this is to regard the standard model as a effective field theory with a finite cutoff scale; present observations put the
effective field theory scale of the standard model at around $10^{10} Gev$, or about $ 10^{-9} M_{Planck} $.  Thus, amplitudes will typically be bounded by
\f
{\cal A} < G M^2    < 10^{-17}
\ff
Thus, once we have the point of view, we see that most quantum gravity amplitudes are tiny.   At the same time, all of these amplitudes are non-perturbative.

\subsection{Other limits}

%\subsection{Details}


The main conclusion is that, rather surprisingly, what I have presented here has a chance to be a viable quantum theory of gravity,
coupled to the standard model.
   
Its main assumptions are:

\begin{enumerate}

\item{} No background spacial views, but there are background (ie fixed) pieces of momentum space.  

\item{}A physical process evolves from vacuum to vacuum.   ie one recovers pieces of an S-matrix in momentum space.

\item{}We have an expilicit, physical cutoff which may be around  ssome $\epsilon << E_{Planck} $.  Near the cutoff the amplitudes loose lorentz invariance and gauge
invariance.

\item{}But these are restored for $E  << E_0,$  $p_i << \frac{E_0}{c}$ to be consistent with present experimental bounds on lorentz
symmetry breaking.

\item{}Spacetime emerges as a description of the semclassical limit of diagrams at low energies.    There is no single spacetime picture for
energies about $E_0$.   (ie the theory behaves like a discretized version of relative locality at and above $E_0$.)   

\end{enumerate}




%\section{FIGURES rehersals}
\begin{center}
\begin{figure}[!h]
   % \includegraphics[scale=0.4]{scan12.jpeg}
\includegraphics[scale=.4]{scan1.jpeg}
  \caption{An energetic causal set mimicking a relativistic field theory}
  %\label{cpvscon}
 % \end{center}
\end{figure}
\end{center}

