% This must be in the first 5 lines to tell arXiv to use pdfLaTeX, which is strongly recommended.
\pdfoutput=1
% In particular, the hyperref package requires pdfLaTeX in order to break URLs across lines.

\documentclass[11pt]{article}

% Remove the "review" option to generate the final version.
\usepackage{acl}
%[review]

% Standard package includes
\usepackage{times}
\usepackage{latexsym}

% For proper rendering and hyphenation of words containing Latin characters (including in bib files)
\usepackage[T1]{fontenc}
% For Vietnamese characters
% \usepackage[T5]{fontenc}
% See https://www.latex-project.org/help/documentation/encguide.pdf for other character sets


\usepackage{graphicx}
\usepackage{csquotes}
\usepackage{algorithm}
\usepackage{algpseudocode}
\usepackage{multirow}
\usepackage{booktabs}
\usepackage{microtype}
\usepackage{colortbl}
\usepackage{subcaption}
%\usepackage[table,x11names,dvipsnames]{xcolor}
\usepackage{twemojis}
\usepackage{tikz}
\usepackage{threeparttable}
\usepackage{ragged2e}

%\usepackage[margin=20mm]{geometry}
\usepackage{enumitem}
\usepackage{etoolbox}
\AtBeginEnvironment{tcolorbox}{%
\setlist[itemize]{nosep,
                 leftmargin=*,
                 label=\textbullet,
                 before=\begin{minipage}[t]{\linewidth}, % <---
                 after=\end{minipage}\medskip}                   % <---
                            }

\usepackage[most]{tcolorbox}
\tcbuselibrary{raster}

\usepackage{lipsum}
\urlstyle{same}

\interfootnotelinepenalty=10000

% This assumes your files are encoded as UTF8
\usepackage[utf8]{inputenc}

% This is not strictly necessary, and may be commented out,
% but it will improve the layout of the manuscript,
% and will typically save some space.
\usepackage{microtype}

% If the title and author information does not fit in the area allocated, uncomment the following
%
%\setlength\titlebox{<dim>}
%
% and set <dim> to something 5cm or larger.

\newcommand{\bug}{\scalebox{1.75}{\twemoji{beetle}}}
\newcommand{\bugone}{\scalebox{1.25}{\twemoji{beetle}}\textsubscript{1}}
\newcommand{\bugtwo}{\scalebox{1.25}{\twemoji{beetle}}\textsubscript{2}}
\newcommand{\bugthree}{\scalebox{1.25}{\twemoji{beetle}}\textsubscript{3}}
\newcommand{\bugall}{\scalebox{1.25}{\twemoji{beetle}}\textsubscript{1,2,3}}
\newcommand{\smallbug}{\scalebox{1.25}{\twemoji{beetle}}}
\newcommand{\correct}{\scalebox{1.75}{\twemoji{check mark button}}}
\newcommand{\smallcorrect}{\scalebox{1.25}{\twemoji{check mark button}}}
\newcommand{\wrong}{\scalebox{1.75}{\twemoji{cross mark}}}
\newcommand{\smallwrong}{\scalebox{1.25}{\twemoji{cross mark}}}
\newcommand{\checkinbox}{\scalebox{1}{\twemoji{black square button}}}
\newcommand{\idea}{\scalebox{1.25}{\twemoji{light bulb}}}
\newcommand{\code}{\scalebox{1.25}{\twemoji{laptop}}}
\newcommand{\results}{\scalebox{1.25}{\twemoji{bar chart}}}
\newcommand{\school}{\scalebox{1.25}{\twemoji{school}}}
% \newcommand{\torchimpl}{PyTorch}
\newcommand{\torchimpl}{TorchAudio}

\newcommand\blfootnote[1]{%
  \begingroup
  \renewcommand\thefootnote{}\footnote{#1}%
  \addtocounter{footnote}{-1}%
  \endgroup
}

%\newcommand{\mg}{\textcolor{olive}}
%\newcommand{\sara}{\textcolor{blue}}
%\newcommand{\mn}{\textcolor{red}}

\newcommand{\mg}{\textcolor{black}}
\newcommand{\sara}{\textcolor{black}}
\newcommand{\mn}{\textcolor{black}}


\title{Reproducibility is Nothing without Code Correctness:\\Conformer Bugs, }

\title{Reproducibility is Nothing without Code Correctness:\\Debugging Incorrect findings and Pushing for Best Practices}

\title{Reproducibility is Nothing without  Correctness:\\Conformer Bugs, Incorrect }


\title{Secure the Science:\\ A Push for Proper Code Verification in NLP}


%%%% debugging, debunking
%%% debugging conformer, debunking findings
%%% debug and debunk: how errors in 
%%% debug and debunk: conformer errors, misleading results 
%%% Reproducibility is Nothing without Correctness:\\A Case Study  with Conformer, a Few Bugs
%%% Reproducibility is Nothing without Code Correctness:\\A Short Story of Three Bugs
%%% Reproducibility is Nothing without Code Correctness:\\Short Story of Three Bugs and Wrong Conclusions
%%% Reproducibility is Nothing without Code Correctness:\\Short Stories of Three Bugs and Wrong Happy Ends 
%%% Reproducibility is Nothing without Code Correctness:\\Three Conformer Bugs, Wrong Happy Ends and a Call to Action


\title{Reproducibility is Nothing without Correctness:\\The Importance of Testing Code in NLP}
%\title{Reproducibility is Nothing without Correctness:\\\sara{The Importance of Code Correctness in NLP}}

%\title{\mn{Reproducibility is Nothing without  Correctness:\\A Call to Action Towards Code Verification in NLP}}


% Open and Reproducible is not Enough: Best Practices to avoid m


% Author information can be set in various styles:
% For several authors from the same institution:
% \author{Author 1 \and ... \and Author n \\
%         Address line \\ ... \\ Address line}
% if the names do not fit well on one line use
%         Author 1 \\ {\bf Author 2} \\ ... \\ {\bf Author n} \\
% For authors from different institutions:
% \author{Author 1 \\ Address line \\  ... \\ Address line
%         \And  ... \And
%         Author n \\ Address line \\ ... \\ Address line}
% To start a seperate ``row'' of authors use \AND, as in
% \author{Author 1 \\ Address line \\  ... \\ Address line
%         \AND
%         Author 2 \\ Address line \\ ... \\ Address line \And
%         Author 3 \\ Address line \\ ... \\ Address line}

\author{Sara Papi\textsuperscript{\idea\results\school}, Marco Gaido\textsuperscript{\idea\results\school}, Matteo Negri\textsuperscript{\results}, Andrea Pilzer\textsuperscript{\code} \\
  \results Fondazione Bruno Kessler \\
  \school University of Trento \\
  \code NVIDIA \\
  \texttt{\{spapi,mgaido,negri\}@fbk.eu,apilzer@nvidia.com}
}

\begin{document}
\maketitle


\begin{abstract} 
%Despite its paramount importance in the dissemination of research outcomes, code correctness (a component of technical soundness) is often presumed to be valid based on the perceived quality of the experimental results. This comes with the risk of erroneous outcomes that can propagate and reinforce misleading findings. To address this issue, we posit that the current focus on result reproducibility should go hand in hand with the emphasis on the correctness of research software. We bolster our call to the NLP community by presenting a case study, in which we identify (and correct) three bugs in widely used open-source implementations of the state-of-the-art Conformer architecture. Through comparative experiments on automatic speech recognition and translation in various language settings, we demonstrate that the existence of bugs yields results that, though easily reproducible, are nonetheless erroneous. In light of these findings, recognizing the need to avoid overburdening the peer-review process with costly code quality verification procedures, this paper is a call to action towards the adoption of best practices aimed at fostering correctness during code development, prior to its release.

% The peer-review process currently evaluates the technical correctness of a paper solely on the basis of the perceived quality of research results. The code used to obtain these results, instead, is presumed to be correct at the risk of erroneous outcomes and inaccurate, potentially misleading findings. To raise awareness of this problem, we argue that the current emphasis on result reproducibility should go hand in hand with the emphasis on the quality and correctness of research software. We bolster our call to the scientific community by presenting a case study that identifies (and corrects) three bugs in open-source implementations of the state-of-the-art Conformer architecture for two tasks (automatic speech recognition and translation).  The results of our experiments on 8 language pairs demonstrate that the presence of bugs yields reproducible and comparable results, which are nonetheless erroneous and potentially misleading. In response to this, we advise the adoption of best practices aimed at ensuring code correctness and improving software quality within the NLP community.

% CORRECTNESS IS EVALUATED ON RESULTS
% PRESENCE OF BUGS LEAD TO WRONG RESULTS THAT IS NOT EVIDENT 
%\chatgpt{Ensuring code correctness in research software is of paramount importance to avoid the risk of erroneous outcomes and potentially misleading findings. This paper emphasizes that the current focus on result reproducibility should be complemented with the emphasis on the quality and correctness of research software, and presents a case study identifying and correcting three bugs in open-source implementations of the state-of-the-art Conformer architecture. Comparative experiments on automatic speech recognition and translation in various language settings demonstrate that the existence of bugs yields easily reproducible but erroneous results, highlighting the need for best practices aimed at fostering correctness during code development prior to its release in the NLP community.}

%\sara{The code correctness of a paper is often presumed to be valid based on the perceived quality of the experimental results. This comes with the risk of erroneous outcomes that can propagate and reinforce misleading findings. To raise awareness of this problem, we posit that the current focus on result reproducibility should go hand in hand with the emphasis on the quality and correctness of research software. We bolster our call to the NLP community by presenting a case study in which we identify (and correct) three bugs in widely used open-source implementations of the state-of-the-art Conformer architecture.  Comparative experiments on automatic speech recognition and translation in various language settings demonstrate that the presence of bugs yields good and reproducible results, which are however erroneous and potentially misleading. In response to this, we advise the adoption of best practices aimed at fostering code correctness during code development and improving software quality within the NLP community.}




%\mn{Despite its paramount importance in the dissemination of research outcomes, code correctness 
%(a crucial aspect of technical soundness) 
%is often presumed to be valid based on the perceived quality of the experimental results. This comes with the risk of erroneous outcomes that can propagate and reinforce misleading findings. To address this issue, we posit that the current focus on result reproducibility should go hand in hand with the emphasis on the correctness of research software. We bolster our call to the NLP community by presenting a case study, in which we identify (and correct) three bugs in widely used open-source implementations of the state-of-the-art Conformer architecture. Through comparative experiments on automatic speech recognition and translation in various language settings, we demonstrate that the existence of bugs yields results that, though easily reproducible, are nonetheless erroneous. In response to this, recognizing the need to not add undue burden to the peer-review process with costly code quality verification procedures, this paper is a plea towards the adoption of best practices aimed at fostering correctness during code development, prior to its release.}


Despite its pivotal role in research experiments, code correctness is often presumed only on the basis of the perceived quality of the results. This comes with the risk of erroneous outcomes and potentially misleading findings. To address this issue, we posit that the current focus on result reproducibility should go hand in hand with the emphasis on coding best practices. We bolster our call to the NLP community by presenting a case study, in which we identify (and correct) three bugs in widely used open-source implementations of the state-of-the-art Conformer architecture. Through comparative experiments on automatic speech recognition and translation in various language settings, we demonstrate that the existence of bugs does not prevent the achievement of good and reproducible results and can lead to incorrect conclusions that potentially misguide future research. In response to this, this study is a call to action toward the adoption of coding best practices aimed at fostering correctness and improving the quality of the developed software.

\end{abstract}

\blfootnote{\idea The authors contributed equally.}

\section{Introduction}
Conferences and journals rely on peer reviews to ensure the quality of published material \citep{Ziman1968-ZIMPKA,ravetz1971scientific,meadows-1974-science}. Currently, the peer-review process evaluates the technical correctness of a paper by examining the 
%validity
soundness of the hypotheses, conclusions, and experimental design,
%the hypothesis
% and conclusions, 
% identifying 
% %\mn{checking for}
% flaws
% in the experimental design,
and by comparing results with other
%studies. 
works.
Specifically, the correctness of a new algorithm 
%proposed in a paper is assessed by
is assessed by
%assessing the correctness of a new algorithm proposed in a paper and its code involves 
\enquote{\textit{establishing consistency of results versus existing implementations, standard benchmarks, or sanity checks via statistically significant experimental results}} \citep{Rozier-2014-repro}. 
%SHORTER ALTERNATIVE: However, the actual functioning of the code is never assessed during this process, leading to challenges in reproducing published papers even when their code, data, and models are released \citep{piwowar2011shares,10.1145/3442188.3445922}. 
%However, the actual functioning of the code is never assessed during this process. As a result, the reproducibility of papers, despite the release of their code, data, and models \citep{piwowar2011shares,10.1145/3442188.3445922}, remains a challenge \citep{arvan-etal-2022-reproducibility-code,Chen2019} 

%Accordingly, the actual functioning of the code is never assessed during this process, with several consequences.
The actual functioning of the underlying code, 
%though,
however,
is never assessed during this process, 
%with
and this has
several undesired consequences.
%\sara{During this process, the actual functioning of the code is never evaluated, leading to several consequences.}
One 
%problem is the lack of reproducibility of the published works, 
%\mn{is the so-called ``reproducibility crisis''}
is the lack of reproducibility of the published works, 
%\sara{that,}
%, which has recently become an increasing and widespread concern within the research community, leading to the rise of dedicated initiatives to mitigate the problem \citep{dodge-etal-2019-show,pineau2021improving,rogers-etal-2021-just-think}.
%Indeed, 
%
%
%
%which, despite the increase in the number of scientific artifacts (code, data, and models) that are released open source \citep{piwowar2011shares,10.1145/3442188.3445922}, still remains a challenge.
which remains a well-known issue despite the increasing number of scientific artifacts (code, data, and models) that are released open source \citep{piwowar2011shares,10.1145/3442188.3445922}.
Indeed, a considerable percentage of codebases fail to run without errors and/or miss dependencies \citep{Chen2019,arvan-etal-2022-reproducibility-code}.
%
%
%
%
% However, this is not the only issue, as reproducibility alone is not sufficient to \enquote{\textit{guarantee the quality, correctness, or validity of the published results}}~\citep{Peng-2011-reproducible}, and obtaining good and reproducible results does not imply that the code is \textbf{correct}, i.e. it performs what it is stated in the paper. Even worse, when a work achieves interesting results and findings and is easily reproducible, although the code is not correct, building future research on them is facilitated and could lead to incorrect conclusions \citep{McCullough-2008-replicable}.
% [L'UNICA ALTERNATIVA CHE TROVO A QUESTO HOWEVER E' NEVERTHLESS]
%
%
%\mn{However, this 
%This
But this
is not the only issue. %\mg{, though}. 
Akin to good performance, reproducibility alone is not sufficient to \enquote{\textit{guarantee the quality, correctness, or validity of the published results}}~\citep{Peng-2011-reproducible} or, in other words, to ensure that the code used to obtain the results is \textbf{correct} (i.e. it actually performs what it is supposed to do).
%
%
%
% \mg{The attainment of good results does not imply that the code is \textbf{correct} (i.e. it actually performs what it is supposed to do) either.}
% \mg{Furthermore,
% when a work obtains noteworthy results and is easily reproducible, future research is likely built on it \citep{McCullough-2008-replicable}, and, in case the code is flawed, the subsequent findings could be misleading.}
%
%%%%%%%\mn{Good and easily reproducible results, in fact, do not guarantee that the code used to obtain them is \textbf{correct} (i.e. it actually performs what it is supposed to do).
%
% The risk, when this condition does not hold, is 
% %the dissemination of erroneous, misleading findings
% that subsequent research will likely build on them
% The risk, when subsequent research builds noteworthy results that rest on flawed code \citep{McCullough-2008-replicable}
Therefore, since the replicability of good results 
%\mn{(especially if they are good)} 
%is often the starting point
facilitates the adoption of the related code as the starting point
of subsequent research \citep{McCullough-2008-replicable}, the risk is that 
% flawed code will contribute to propagate misleading findings.
building on flawed code 
%will propagate misleading findings.
can yield unreliable and misleading findings.
%can yield wrong results that propagate misleading findings.}
%
%
%
%\mg{Even worse, when a work achieves interesting results and is easily reproducible, building future research on it is facilitated \citep{McCullough-2008-replicable}. In case the code is not correct, this can lead to misleading findings.}
%
%although the code is not correct, building future research on them is facilitated and could lead to incorrect conclusions \citep{McCullough-2008-replicable}.}
% %Also, since 
%Since obtaining good and reproducible results does not imply that the code is \textbf{correct} 
%%(i.e. it performs what it is stated in the paper), 
%(i.e. it actually performs what it is supposed to do), 
%building future research on 
%easily reproducible results obtained with software affected by errors 
%them can lead to incorrect conclusions and to the propagation of misleading findings \citep{McCullough-2008-replicable}.



%In this paper, as a countermeasure to the problems mentioned above and a method to increase confidence in current findings and future research directions based on them, we posit that code correctness should be assessed on its own and promote the adoption of best practices of software quality assurance (SQA) in the research community. 
%In this paper, as a countermeasure to the aforementioned issues and to bolster confidence in published results and subsequent findings, as well as future research based on them, we posit that code correctness \mg{should not be assumed on the basis of good results}
%should undergo an independent assessment 
%and advocate for the adoption of best practices aimed at enhancing software quality within the research community.
%
%With the goal of enhancing confidence in published findings,
%and future research based on them, 
%in this paper, we
% \sara{In this paper, with the goal of enhancing confidence in published findings, we argue that good results alone are not an adequate measure to ensure the code correctness,}
%
%
%
%
% With the goal of enhancing confidence in published findings, in this paper we argue that good results alone are not an adequate measure to ensure the code correctness,
% %argue against the assumption of code correctness based solely on good results, 
% and propose the adoption of best practices aimed at enhancing software quality within the NLP research community.
% Our contributions can be summarized as follows:
%
%
%
% \mn{In light of the above,  this paper is a call to action, supported by empirical evidence, toward enhancing the reliability of published NLP findings through the adoption of best practices during code development. Specifically our contributions are:}
In light of the above,  this study is a call to action, underpinned by empirical evidence,
to enhance the dependability of published NLP findings through the 
implementation of best practices during code development. In particular, our contributions are:
%
\begin{enumerate}[leftmargin=14pt]
    %\setlength\itemsep{-3pt}
    \item %We show that the code correctness of papers is given for granted in current peer-review processes based on the quality of the results, 
    We demonstrate that the code correctness of a paper in current peer reviews is assumed solely based on the quality of the results,
    whereas we posit that it should undergo independent assessment, similarly to reproducibility (\S\ref{sec:core-idea});
%    \item \mn{[INVERTIREI: THROUGH A CASE STUDY...CONFORMER...DIMOSTRIAMO CHE IL TEMA E' URGENTE ...]} By using the open-source implementations of the widespread Conformer architecture \citep{gulati20_interspeech} as a case study, we prove that:
    \item Through a case study on  open-source implementations of the widespread Conformer architecture \citep{gulati20_interspeech}, we prove that:
    \begin{itemize}[leftmargin=8pt]
    \item[-] All the analyzed implementations contain at least one bug in the codebase (\S\ref{subsec:analysis});
    \item[-] The 
    %presence
    existence
    of bugs does not prevent from achieving reproducible results that outperform those of other 
    %architectures and works, 
    architectures
    across 
    %various tasks and languages
    different tasks and language settings
    %even in presence of bugs, it is possible to obtain reproducible results that are competitive with those of other architectures and works, across various tasks and languages 
    %pairs \mn{[non avendo detto su quali task e' fatta quest'analisi, suona strano leggere ``language pairs''.]}
    (\S\ref{subsec:impact_bug});
    \item[-] 
    The presence of bugs
    %\mn{This}
    can lead to incorrect findings when incorporating a new technique into the architecture (\S\ref{subsec:impact_code}).
    \end{itemize}
    \item We release an open-source version of Conformer 
    %implementation that 
    that
    is not affected by bugs, as well as all the pre-trained models, at \texttt{\url{https://github.com/hlt-mt/fbk-fairseq}} under the Apache 2.0 Licence;
    \item We propose a checklist of best practices to 
    %enforce code correctness
    foster
    %favour 
    code correctness prior to its release
    (\S \ref{sec:checklist}).
\end{enumerate}








\section{Correctness is not Evaluated}
\label{sec:core-idea}
%\url{https://aaai.org/conference/aaai/aaai-23/reproducibility-checklist/}

In NLP, as in many 
%experimental 
empirical
fields, a scientific contribution typically
builds on
%\mg{has}
three main constituent elements:
%(Fig. \ref{fig:scientificartifact}):
\textbf{concept} (the
%original idea
idea
or hypothesis being tested), \textbf{implementation} (the software developed to test the concept), and \textbf{outcomes} (the results
%obtained from the
of the
experiments carried out by using the code).

%The concept is the foundation of the publication. In the reviewing process, it 
%\mn{In the reviewing process, the concept}
The concept, the foundation of the 
%publication,
work,
is typically evaluated based on its 
%\textit{novelty} (i.e., how original the idea is),
%\textit{novelty}  \mn{(how original the idea is),}
novelty (\textit{is the idea original?}),
%\textit{soundness} (i.e., if the idea is grounded on solid theoretical bases),
%\textit{soundness} \mn{(if the idea is grounded on solid theoretical bases),}
soundness (\textit{is the idea grounded on solid theoretical bases?}),
and  
%\textit{impact} (i.e., how important the idea is for the scientific community).
%\textit{impact} \mn{(how important the idea is for the scientific community).}
impact (\textit{is the idea important for the scientific community?}).
%The implementation is a scientific artifact created to realize and test the concept. When present, it 
The implementation, when present,
is 
%generally 
evaluated 
%by looking at its \textit{accessibility} (i.e., if the code is released open source), and its \textit{usefulness}  (i.e., if the research community will benefit from the use of the software).
in terms of 
% \textit{accessibility} (if the code is released open source) 
accessibility (\textit{is the code released open-source?})
and  
%\textit{usefulness} (if the research community will benefit from the use of the software).
usefulness (\textit{will the research community benefit from the use of the software?}).
% The outcomes are usually checked in comparison either with the  state of the art (\textit{are the results  competitive with those reported in recent literature?}) or with a strong baseline in absence of related works, looking at their significance (\textit{are the improvements robust to statistical fluctuations?}).
The outcomes are usually assessed in terms of significance (\textit{are the reported improvements robust to statistical fluctuations?}) with respect 
%either to 
to either
the state of the art (\textit{are the results competitive with those reported in recent literature?}) or strong baselines.

%Aside from 
Besides considering these elements separately, it is also important to consider how they are interconnected.
Evaluating the link between implementation and outcomes
corresponds to answering the question: \textit{``Can the results be actually produced by the codebase?''} or, in other words, \textit{``Are the results \textbf{reproducible} with the code?''.}
According to the Association for Computing Machinery,
%(ACM), 
%a code is reproducible if
reproducibility holds when
\enquote{\textit{an independent group can obtain the same result using the author’s own artifacts}},\footnote{\url{https://www.acm.org/publications/policies/artifact-review-and-badging-current}} which is coherent with the definition given 
both in other fields \citep{doi:10.1128/mBio.00525-18} and in NLP \citep{ulmer2022experimental}.
%in various fields \citep{doi:10.1128/mBio.00525-18} including NLP \citep{ulmer2022experimental}.
%by related works on experimental standards in NLP 
Several works have been dedicated to this topic, 
%claiming the presence of a \enquote{reproducibility crisis}, as their systematic analyses questioned the reproducibility of a large portion of published papers
and 
%stigmatized
denounced a \enquote{reproducibility crisis} 
% problem that hinders the 
% \sara{reproducibility}
% %\mn{replication}
% of 
that affects
a large portion of published papers
\citep{prinz2011believe,Baker2016,Gundersen_Kjensmo_2018,NEURIPS2019_c429429b,Gundersen_2019}, also in the specific context of 
%
%
%
% NLP \citep{wieling-etal-2018-squib,belz-etal-2021-systematic,belz-etal-2022-quantified}.
% Most of them argue that the published works lack consistent evaluation settings \citep{marie-etal-2021-scientific,gehrmann2022repairing} and cannot be reproduced \citep{narang-etal-2021-transformer}, even 
% %if codebases are available 
% with codebases available
% \citep{arvan-etal-2022-reproducibility-code}.
%
NLP \citep{wieling-etal-2018-squib,belz-etal-2021-systematic,marie-etal-2021-scientific,narang-etal-2021-transformer,gehrmann2022repairing,belz-etal-2022-quantified}, and even 
when codebases are available
\citep{arvan-etal-2022-reproducibility-code}.
%
%
%
%To mitigate this issue, 
To tackle this problem,
numerous scientific organizations and associated journals and conferences, such as NeurIPS,\footnote{\url{https://neurips.cc/Conferences/2021/PaperInformation/PaperChecklist}} AAAI,\footnote{\url{https://aaai.org/Conferences/AAAI-22/reproducibility-checklist/}} and the ACL community,\footnote{\url{https://aclrollingreview.org/responsibleNLPresearch/}} have recently introduced 
%the completion of checklists
checklists
aimed at enhancing the reproducibility of published material \citep{dodge-etal-2019-show,pineau2021improving,rogers-etal-2021-just-think}.
%The emphasis on the topic is further confirmed 
Such a growing concern is also attested
by the organization of dedicated workshops and shared tasks on this theme
\citep{Pineau:2019,belz-etal-2021-reprogen}, and its inclusion as part of
%recent 
conference theme tracks.\footnote{\url{https://2022.emnlp.org/calls/main_conference_papers/\#emnlp-2022-theme-track}, \url{https://2023.aclweb.org/calls/main_conference/\#theme-track-reality-check}}

%\begin{figure}[!t]
%    \centering
%    \includegraphics[width=0.45\textwidth]{img/idea.png}
%    \caption{Elements of a scientific publication.}
%    \label{fig:scientificartifact}
%\end{figure}

\begin{table}[!tb]
    \centering
    \small
    \setlength{\tabcolsep}{8pt}
    \begin{tabular}{l||c|c}
    \specialrule{.1em}{.05em}{.05em} 
        %\multirow{2}{*}{\textbf{Venue}} & \multicolumn{2}{c}{\textbf{Review Form Content}} \\
        %\cline{2-3}
        \textbf{Venue} & \textbf{Reproducibility} & \textbf{Correctness} \\
        \specialrule{.1em}{.05em}{.05em} 
        %\multirow{2}{*}{AAAI} & Reproducibility & \smallcorrect{} Yes/No questions for each record of the Reproducibility Checklist\tnote{*} \\
        %    & Correctness & Evaluated by... \\
        % \hline
        *ACL &  \smallcorrect{} & \smallwrong{} \\
        ARR & \smallcorrect{} & \smallwrong{} \\
        ICASSP & \smallwrong{} & \smallcorrect{} \\
        ICML & \smallwrong{} & \smallwrong{} \\
        ICLR & \smallcorrect{} & \smallcorrect{} \\
        Interspeech & \smallcorrect{} & \smallwrong{} \\
        NeurIPS & \smallwrong{} & \smallwrong{} \\
        TACL & \smallcorrect{} & \smallcorrect{} \\
    \specialrule{.1em}{.05em}{.05em} 
    \end{tabular}
    % \caption{Which major conferences/journals in NLP explicitly require a score for reproducibility and correctness in the review forms?}
    \caption{Reproducibility and correctness in the review forms of major conferences/journals in the NLP field.}
    %\mn{[PAURA! Sicuri, vero, che a NeurIPS e ICML non c'e' alcun check nemmeno sulla riproducibilita'???]}
    \label{tab:nlp}
    %\enquote{\smallcorrect{}} means the aspect is mentioned in the form, \enquote{\smallwrong{}} otherwise.
\end{table}

\begin{figure*}[!t]
    \centering
    \includegraphics[width=0.95\textwidth]{img/conformer_con.drawio.png}
    \caption{Convolution module in the Conformer encoder layer. All convolutional blocks are 1D convolutions.}
    \label{fig:conf_conv}
\end{figure*}

But what about the link between 
%the 
concept and 
%the 
implementation?
Evaluating this connection corresponds to 
%answer
answering
the question: \textit{``Does the 
%developed
code exactly 
%realize the idea
implement the concept?''.}
This 
equals to
%\mn{implies}
assessing the \enquote{\textit{extent to which a program satisfies its specifications}}  \citep{McCall1977FactorsIS}, which is a fundamental attribute of a codebase, known as \textit{functionality} \citep{ISO9126,ISO/IEC2010} or functional \textbf{correctness} in the field of software quality assurance \citep{Buckley-1984-sqa,tripathy2011software}.
If this characteristic is not enforced, there is no guarantee that the code actually does what it is 
%expected, i.e. 
expected and, in turn,
that the results of a paper actually validate the concept.
%
%
%
%
% Therefore, we posit that 
% %enforcing
% \mn{fostering}
% code correctness represents a crucial aspect of research code. 
However, differently from reproducibility, code correctness has been so far 
%neglected 
overlooked by the research community.
%in the peer-review process.
%is not a currently evaluated aspect of the reviewing process of a paper.

To demonstrate this phenomenon, we scraped the latest review forms\footnote{Checked on January 15th 2023.} of the most popular conferences/journals in the NLP field, namely: 
*ACL (AACL-IJCNLP, ACL, EACL, EMNLP, NAACL), ACL Rolling Review (ARR),
ICASSP, ICML, ICLR, Interspeech, NeurIPS, and TACL.
Table \ref{tab:nlp} shows that an explicit score is assigned to reproducibility in most of the 
%conferences/journals
venues
(5 out of 8). 
% In addition, at NeurIPS, although there is no explicit score, reproducibility is listed
% among the factors that should contribute to the overall score.
%\mn{Although
In addition, although
there is no explicit score at NeurIPS, reproducibility is listed
among the factors that should contribute to the overall evaluation.
Conversely, correctness is 
%only mentioned in TACL, ICASSP, and ICLR, 
mentioned in fewer cases (3)
and its definition varies. 
%In
At ICLR and ICASSP, the scope of the term is not clearly defined, while in TACL it is 
% \sara{considered together with the \textit{soundness}} of the approach and results/experiments. 
included in the broader concept of \textit{soundness} of the experiments/results.
%The soundness of the experiments/results
The soundness is also 
%present in 
assessed at
NeurIPS and ICML but, again, 
%the 
correctness is never explicitly mentioned. Notably, 
%Interspeech's review form 
Interspeech
contemplates a score for \enquote{Technical Correctness}, which, however, refers 
%,though, 
to the reproducibility of the paper (\enquote{\textit{[...] are enough details provided to be able to reproduce the experiments?}}).
% A score related to the software is present in the ARR form, but it comprises only the usefulness and documentation of newly-released open-source code, not its correctness.
A score related to the software is present in the ARR form, but it only refers to the usefulness and documentation of newly-released open-source code rather than to its correctness.
%
%
%We can conclude that code correctness is largely neglected in the peer-review process and it is assessed only in terms of the soundness of the results and experimental settings.

% \mn{[ALLORA: QUI CI VUOLE UN PO' DI ATTENZIONE. COME GIA' DISCUSSO OGGI, SEMBRA CHE IL PROBLEMA SIA NEL PROCESSO DI BLIND REVIEW, COSA CHE NON VOGLIAMO ARRIVARE A DIRE PER NON METTERE IL CULO SULLE PEDATE...PIUTTOSTO, LE INDICAZIONI PER LE REVIEW, ANCHE NELLE TOP VENUES (E DIREI INEVITABILMENTE PERCHE' SAREBBE COMPLICATO FARE ALTRIMENTI), DIMOSTRANO UNA SCARSA ATTENZIONE ALLA CORRETTEZZA CODICE CHE E' POSSIBILE SI RIFLETTA SUGLI AUTORI...LA CHIUSA DI QUESTA SEZIONE DOVREBBE RUOTARE SU QUESTA POSSIBILITA' (\textbf{SENZA BLAMING DEL PEER REVIEW}), CHE VERRA' VERIFICATA NELLA PROSSIMA SEZIONE.]}

%In the next section, through a case study based on the open-source implementations of the Conformer architecture, we discuss the risks of this practice of evaluating the correctness solely by looking at the results. Specifically, we show how results can provide misleading indications about the correctness of the code and how building on incorrect code can lead to incorrect findings.

It is important to remark that
%our
the above discussion of 
%the
current
%above 
%peer-review
review guidelines is not meant to advocate for
code quality checks by peers, which would be hard and costly (if not impossible) to  implement systematically. Its purpose, instead, is to 
highlight that 
%
%
%
%
% the current reproducibility-centered
% %evaluation conditions
% evaluation reflects and can induce
their common emphasis on reproducibility-centered evaluation reflects and can induce
%in the authors
a reduced sensitivity towards code correctness, which 
comes with
%carries
inherent risks.
%\mg{assess the scarce sensitivity towards code correctness that comes with inherent risks.}  
%The next section discusses these risks through a case study based on open-source implementations of the widely used Conformer architecture.
In the next section, we dive into these risks by analyzing, as a case study, the open-source implementations of the state-of-the-art architecture for two tasks.
% In 
% %the next section, 
% \S\ref{sec:case-study},
% we dive into these risks by discussing a case study involving 
% %open-source implementations of 
% a state-of-the-art architecture used in two tasks and different language settings. In \S\ref{sec:checklist}, we  propose a list of recommendations aimed at mitigating these risks.}
% \mg{IO METTEREI SOLO: In the next section we dive into .... Abbiamo bisogno di spazio e forse in un paper basta collegarsi alla sezione dopo secondo me}










%%%%% NON CANCELLARE :-)
% \mn{It is important to remark that our discussion of the above peer-review guidelines is not meant to stigmatize the lack of code quality checks, which would be hard and costly (if not impossible) to  implement systematically. Its purpose, instead, is to acknowledge that these unavoidable ``environmental'' conditions can induce a reduced sensitivity towards code correctness that comes with inherent risks. 
% %The next section discusses these risks through a case study based on open-source implementations of the widely used Conformer architecture.
% In the next section, we dive into these risks by discussing a case study involving 
% %two tasks and different language settings.
% open-source implementations of a state-of-the-art architecture used in two tasks and different language settings. In Section QUATTRO...we will propose a chcklist of ... BEST PRATICES...aimed at mitigating these risks.}


\section{The Case Study}%: Conformer}
%\section{\mn{Case Study: Using Conformer for Speech Processing Tasks}}
\label{sec:case-study}
As a case study, we examine existing open-source implementations of the Conformer \citep{gulati20_interspeech} architecture, which is the state-of-the-art solution for speech processing tasks \citep{conformer-sota,ma2021end,conformer-sota2,conformer-sota3} such as automatic speech recognition (ASR) and speech-to-text translation (ST). 
%After a brief description of the ASR and ST tasks and the Conformer architecture (\S\ref{subsec:speech_background}), 
After introducing the ASR and ST tasks object of our study, and the basics of the Conformer architecture (\S\ref{subsec:speech_background}),
we analyze several widely-used open-source codebases that contain a Conformer implementation, showing that all the analyzed repositories have at least one bug (\S\ref{subsec:analysis}). 
Through extensive experiments on the two tasks and on all the language pairs of the MuST-C v1.0 corpus \cite{CATTONI2021101155}, we demonstrate that the presence of 
%\mn{such} 
bugs can be hidden by good -- but incorrect -- results (\S\ref{subsec:impact_bug}) that consequently lead to wrong findings (\S\ref{subsec:impact_code}).


\subsection{Background}
\label{subsec:speech_background}
% ASR is the task in which an audio containing speech content is transcribed in its original language, while in ST the source content is translated into a different language.
ASR is the task in which an audio containing speech content is transcribed in its original language. In ST, instead, the source audio is translated into text in a different language.
Nowadays, both tasks are commonly performed with end-to-end (or direct) models \citep{pmlr-v32-graves14,Chorowski-2014-asr,berard_2016,weiss2017sequence}, whose architecture is based on the Transformer~\citep{NIPS2017_3f5ee243}. The Transformer has been adapted to work with audio inputs~\citep{8462506,gangi19_interspeech} by introducing two convolutional layers that shrink the length of the input sequence by a factor of 
%4.
$4$, so as to reduce the otherwise excessive memory requirements.
%make it manageable thanks to affordable memory requirements.}
% \mg{More recently, \citet{gulati20_interspeech} proposed the Conformer} 
% \sara{architecture} \mg{by modifying the structure of the encoder layers, with significant improvements both in ASR and in ST \citep{inaguma2021non}.}
%
%
%
% \mn{More recently, \citet{gulati20_interspeech} proposed the Conformer: a novel architecture
% %in which 
% with a  modified structure of the encoder layers that led to significant improvements  in both ASR and ST \citep{inaguma2021non}.}
More recently, \citet{gulati20_interspeech} proposed the Conformer: a novel architecture with a modified encoder that led to significant improvements in both ASR and ST \citep{inaguma2021non}.



The changes introduced in the Conformer encoder 
%layers
layer structure can be summarized as follows:
\textit{i)} relative sinusoidal positional encodings \citep{dai-etal-2019-transformer} are introduced in the self-attention for improved generalization with respect to varying input lengths;
\textit{ii)} the FFN sublayer is replaced by two FFNs that wrap the self-attention, inspired by the Macaron-Net~\citep{lu-et-al-2016-macaron-net};
\textit{iii)}~a convolution module (depicted in Figure \ref{fig:conf_conv}) is added immediately after the self-attention, before the second FFN layer.
The convolution module, which is wrapped in a residual connection, applies layer normalization and then a point-wise convolution to each feature 
vector, doubling its dimension that is restored to its original size by a Gated Linear Unit (GLU) activation function~\citep{Dauphin-2017-glu}. 
% vector that doubles its dimension \mn{(which is later restored to its original size by a Gated Linear Unit (GLU) activation function~\citep{Dauphin-2017-glu})}. 
% \mn{vector. This doubles its dimension, which is later restored to its original size by a Gated Linear Unit (GLU) activation function~\citep{Dauphin-2017-glu}.}
%After,
Then,
a depth-wise convolution with 31 kernel size is applied before a batch normalization \citep{ioffe-2015-batchnorm}, followed by the Swish activation function \citep{swish-2017}, and another point-wise convolution.
Lastly, a dropout module \citep{Srivastava-2014-dropout} randomly masks (i.e. zeroes out) a percentage of the feature values to prevent the network from overfitting.

\subsection{Analysis of the Codebases}
\label{subsec:analysis}

We analyze the behavior of the open-source implementations of the Conformer 
%when
by systematically varying a parameter that should not affect the results: the inference batch size (IBS).
With 
%a high
high
IBSs, multiple samples are collected in the same batch, allowing for their parallel processing on 
%GPU, thus reducing the overall computational cost.
GPU 
%and an overall reduction of computational costs.
to reduce the overall computational cost.
When samples of different lengths are collected in the same batch
%, 
--
a frequent situation in speech tasks where the input length largely varies
%, 
--
the input sequences have to be brought to the same dimension by filling them with padding. 
% \sara{For this reason, research papers often include details only on the training batch size, which is an important hyperparameter for the stability of the training, while the IBS is not reported.}
%Since with correct implementations the effect of padding is assumed to be independent of the IBS, 
Since with correct implementations the results are independent of the presence of padding (and, therefore,
%from
of
the IBS),
research papers usually
%include details only about
include only
the training batch size (which, in contrast, is an important hyperparameter for the stability of the training). However, as we
%will
demonstrate in this section, 
%the presence of bugs
the bugs present in widely-used Conformer implementations 
%breaks the above assumption.}
undermines the above assumption.

We studied six open-source implementations from widely adopted codebases, namely:
Fairseq-ST~\citep{wang2020fairseqs2t}, ESPnet-ST~\citep{inaguma-etal-2020-espnet}, NeMo~\citep{kuchaiev2019nemo}, SpeechBrain~\citep{speechbrain}, 
%an open source implementation on github\footnote{\url{https://github.com/sooftware/conformer}} named \enquote{Conformer}, and the official PyTorch implementation from TorchAudio~\citep{yang2021torchaudio}.
an open source codebase named \enquote{Conformer}\footnote{\url{https://github.com/sooftware/conformer}}, and \torchimpl{}~\citep{yang2021torchaudio}.
We discovered that all these implementations return different results with different IBSs, showing that the presence of padding has unpredictable effects on the results.\footnote{We 
%would like to 
emphasize 
that our intention is not to
single out the shortcomings of individual libraries. Conversely, we are extremely thankful for the 
%important and worthy 
invaluable
contribution they represent for our community. Our analysis is only intended to further improve the correctness of codes and, consequently, of the experimental results, which we
believe is of utmost importance.}
%
%
%
% \footnote{We 
% %would like to emphasize 
% remark
% that our intention is not to blame any specific 
% % codebase or 
% codebase, nor
% its developers. Conversely, we are extremely thankful for the important 
% %and worthy contribution 
% contribution
% of these open-source libraries to our community. Our analysis is not intended 
% %to point out the issues of the single libraries but is intended to further improve the correctness of codes and, consequently, of the experimental results which we believe is of utmost importance.
% to single out the shortcomings of individual libraries, 
% but rather to promote best practices aimed at increasing
% %improved
% code correctness and, as a result,
% %more reliable
% the reliability of 
% experimental findings, which we consider to be of paramount importance.}
%
%
%
%
%
%
Upon inspection of the codes relative to the Conformer architecture, we 
%isolate
isolated
three types of bugs:


\begin{figure}[!tb]
\centering
\renewcommand*\thesubfigure{\arabic{subfigure}} 
     \begin{subfigure}[b]{0.46\textwidth}
        \centering
         \includegraphics[width=0.4\textwidth]{img/start.png}
         \caption{Before shifting, the Relative PE matrix ($P_{00},...,P_{22}$) is padded (zero values).}
     \end{subfigure}
     %\par\medskip
     \begin{subfigure}[b]{0.46\textwidth}
        \centering
         \includegraphics[width=0.4\textwidth]{img/wrong.png}
         \caption{When relative shift is applied to the Relative PE matrix without considering padding, some values of the padding area (in \textcolor{red}{\textbf{red}}) are incorrectly 
         %included in 
         moved to
         the non-padding area.}
         %\smallbug\textsubscript{3}, 
        % some values ($P_{11}, P_{12}, P_{20}, P_{21}$) are moved to the padding part (in \textcolor{red}{\textbf{red}}), hence are not considered in the following computation, while some padding is instead incorrectly considered in the following computation (in \textcolor{teal}{\textbf{green}}). Please notice that the first row is always discarded.}
     \end{subfigure}
     \begin{subfigure}[b]{0.46\textwidth}
        \centering
         \includegraphics[width=0.4\textwidth]{img/correct.png}
         \caption{When relative shift is applied to the Relative PE matrix considering padding, the values $P_{00},...,P_{22}$ are not moved to the padding area.}
     \end{subfigure}
    \caption{Example of relative shift operation starting from a Relative PE matrix containing padding (1), both considering a codebase with \bugthree{} (2) and without (3) bug. The first row is always discarded.}
    \label{fig:relativePEs}
\end{figure}

\paragraph{Convolution Module Bug (\bugone)} The depth-wise and point-wise convolutions of the Conformer convolution module do not consider the presence of padding and produce a non-padded output with non-zero values adjacent to the input sample.
These values 
%alter 
modify
the behavior of the subsequent batch normalization and of the other convolutions, leading to incorrect alterations of all the valid values.

\paragraph{Initial SubSampling Bug (\bugtwo)} The two initial convolutions that subsample the input sequence by a factor of 4 do not consider padding. For this reason, the second convolution is fed with non-zero values adjacent to the input sequence that lead to a wrong computation of the last valid elements.

\paragraph{Positional Encodings Bug (\bugthree)} The relative sinusoidal positional encodings (PEs), which are added to the attention matrix, are computed by shifting a sinusoidal matrix. This shifting operation first prepends a zero column to the sinusoidal matrix and then reshapes it so that the last element of the first row becomes the first element of the second row, the last two elements of the second row become the first ones of the third row, and so on. By doing this, this operation assumes that all elements are valid. However, when a 
%sentence
sequence
is padded, only a part of the attention matrix is valid (in green in Figure~\ref{fig:relativePEs}.1) and spurious values are moved to the beginning of the next row (Figure~\ref{fig:relativePEs}.2). 
In Figure~\ref{fig:relativePEs}, for the sake of clarity of the example, we pretend that existing implementations set to 0 the PE in the padding area. While this is not what happens in practice (as the padding area contains other sinusoidal PEs), it shows that the correct values are discarded and the final matrix significantly differs from 
%that
the one
obtained without padding, which is
%does not contain the correct values, which are 
instead shown in Figure~\ref{fig:relativePEs}.3.
%\mg{(i.e., those obtained without padding)}

\begin{table}[!ht]
\small
    \centering
    \setlength{\tabcolsep}{6pt}
    \begin{tabular}{l|c|c|c}
    \specialrule{.1em}{.05em}{.05em} 
         \textbf{Repository} & \textbf{Conv. Mod.} & \textbf{SubSampl.} & \textbf{Pos. Enc.} \\
         \specialrule{.1em}{.05em}{.05em} 
         Fairseq-ST & \bugone & \bugtwo & \bugthree \\
         ESPnet-ST & \bugone & \bugtwo & \bugthree \\
         NeMo & & \bugtwo & \bugthree \\
         SpeechBrain & \bugone & \bugtwo & \bugthree \\
         Conformer & \bugone & \bugtwo & \bugthree \\
         \torchimpl{} & \bugone & NA & NA \\
    \specialrule{.1em}{.05em}{.05em} 
    \end{tabular}
    \caption{Presence of bug(s) in the codebase of the analyzed repositories. NA stands for \enquote{Not Applicable}.}
    \label{tab:bug}
\end{table}

In Table \ref{tab:bug}, we report the presence (or absence) of these bugs for each analyzed codebase in its current version.\footnote{Checked on January 8th 2023.}
All the implementations but one (NeMo) are affected by \bugone.
%Moreover, 
%
%
%
%
% Also,
% all the implementations are affected by \bugtwo{} and \bugthree, except for the implementation by Pytorch, which does not introduce relative PEs in the attention and replaces the initial sub-sampling convolutional layers with linear layers that map the input sequence in matrices of fixed dimensions.
% %\mg{We} can conclude that all the \mg{analyzed implementations} contain at least one bug.
% Having ascertained that all the analyzed implementations contain at least one bug, the next sections will concentrate on their impact on ASR and ST results and related findings.
%
Also, 
all are affected by \bugtwo{} and \bugthree, except for
%implementation by Pytorch,
%\mn{PyTorch one,}
%the \torchimpl{} implementation,
\torchimpl,
which 
%does not introduce
includes neither relative PEs in the attention nor
%and 
%\sara{does not include}
%%replaces 
the initial sub-sampling convolutional layers 
in its implementation.
%with linear layers that map the input sequence in matrices of fixed dimensions.
Having ascertained that all the analyzed implementations contain at least one bug, the next sections will concentrate on their impact on ASR and ST results 
%and
and, in turn, the
related findings.

% \mn{In Table \ref{tab:bug}, we report the presence (or absence) of these bugs for each 
% %repository
% \mn{analyzed codebase}
% in its current version.\footnote{Checked on January 8th 2023.} As it can be seen, all the implementations contain at least one bug. In particular, all but one (NeMo) are affected by \bugone.
% %Moreover, 
% \mn{Also,}
% all the implementations are affected by \bugtwo{} and \bugthree, except for the implementation by Pytorch\mg{, which} does not \mg{introduce} relative PEs in the attention and replaces the initial sub-sampling convolutional layers with linear layers that map the input sequence in matrices of fixed dimensions.}




\subsection{Experimental Settings}

We train and evaluate ASR and ST models on MuST-C v1.0 \cite{CATTONI2021101155}, which contains parallel speech-to-text data with English (en) as source language and 8 target text languages, namely Dutch (nl), French (fr), German (de), Italian (it), Portuguese (pt), Romanian (ro), Russian (ru), and Spanish (es). 
The ASR model is 
trained on the transcripts of the MuST-C en-es  train set,
%\mn{built on the transcripts of the MuST-C en-es training set,}
as it is the largest section of the corpus
%\sara{of the dataset}.
%, containing $\sim$260K samples.
%Data
(for data statistics,
%are presented in 
see Appendix \ref{sec:data_stats}).
For ST, 8 different models are trained, one for each language direction en$\rightarrow$\{de, es, fr, it, nl, pt, ro, ru\}. 
All the experimental settings, including model architecture, and training hyperparameters, are described in Appendix \ref{sec:exp_sett}.
Evaluation is performed on
%\sara{, whose outputs are generated using a single NVIDIA A40 with 40GB of RAM,}
%MuST-C tst-COMMON with word error rate (WER) metric for ASR and with sacreBLEU \citep{post-2018-call}
the MuST-C tst-COMMON, 
by computing  word error rate (WER) for ASR and SacreBLEU \citep{post-2018-call}\footnote{BLEU|\#:1|c:mixed|e:no|tok:13a|s:exp|v:2.0.0} for ST.
All our 
trainings and inferences 
%have been 
were respectively
performed on two and one NVIDIA Ampere A40 GPUs.

%\paragraph{The Case of TF32}
%Another issue that potentially causes unexpected differences when padding is introduced does not relate to the code but to the use of NVIDIA Ampere GPUs (A40/A100) during training/inference.
It is worth remarking 
that, by default, on the Ampere GPUs,
%\mn{that, by default, on the Ampere GPUs,}
the PyTorch backend computes convolutions and matrix multiplications with TensorFloat-32\footnote{\url{https://pytorch.org/docs/stable/notes/cuda.html\#tensorfloat-32-tf32-on-ampere-devices}.} (\textbf{TF32}) cores. TF32 speeds up the computation but introduces numeric errors that can cause small 
%but 
random fluctuations, e.g. in presence of padding. 
In the following, we hence experiment both with and without TF32 (both at training and inference time) 
%as
because
%the presence of padding 
padding
has no effect on the final outputs only when 
%disabling TF32.
TF32 is disabled.


%\subsection{Impact of Single Bugs}
\subsection{Impact of the Identified Bugs}
\label{subsec:impact_bug}


% \sara{To quantify the problems caused by the presence of bugs in the code,}
% we analyze the impact of the three bugs described in \S\ref{subsec:analysis} on the performance of the models and we compare 
% \sara{the results with those obtained by}
% previous works
% \sara{on}
% different architectures. 
To assess the impact of 
%bugs in 
flawed
code, we examine the effect of the three bugs described in \S\ref{subsec:analysis} on the performance of the models and compare our results to those from previous studies using different architectures.
To this aim, we evaluate different IBSs, as increasing the batch size brings more padding, thus amplifying the effects of bugs. 
% \sara{The experiments are first conducted on the correct codebase (\smallcorrect), successively single precision is enabled (TF32), then bugs are introduced one by one in the code during both training and inference (\bugone, \bugtwo{}, and \bugthree) to evaluate their individual contribution, and are then introduced all together (\bugall).}
The experiments are first conducted on the correct codebase (\smallcorrect); successively, by enabling single precision (TF32); then by reintroducing bugs one by one during both training and inference (\bugone, \bugtwo{}, and \bugthree) to evaluate their individual contribution; finally by reintroducing them all together (\bugall).




\begin{table}[!tb]
\small
\setlength{\tabcolsep}{12pt}
    \centering
    \footnotesize
    \begin{tabular}{l|ccc}
    \specialrule{.1em}{.05em}{.05em} 
        \multirow{2}{*}{\textbf{Code}} & \multicolumn{3}{c}{\textbf{IBS}} \\
        \cline{2-4}
         & 1 & 10 & 100 \\
        \specialrule{.1em}{.05em}{.05em} 
        \smallcorrect & 10.52 & 10.52 & 10.52 \\
        \hline
        + TF32 & 10.73 & 10.73 & 10.73 \\
        \quad + \bugone & 10.72 & 11.25* & 19.50* \\
        \quad + \bugtwo & 10.73 & 10.74 & 10.74 \\
        \quad + \bugthree & \textbf{10.46} & 10.62 & 10.73 \\
        \quad + \smallbug\textsubscript{1,2,3} & 11.32* & 14.25* & 54.56* \\
    \specialrule{.1em}{.05em}{.05em} 
    \end{tabular}
    \caption{WER ($\downarrow$) scores for ASR obtained with TF32 and bugs
    %the different
    %types of bug \bugall{} 
    %\mg{bugs}
    %and TF32 for \mg{ASR}
    %the ASR task
    % on 
    % MuST-C 
    % %en-es 
    % tst-COMMON 
    as IBS varies (1, 10, and 100 sentences). *~indicates that the difference with \smallcorrect{} is statistically significant, computed with bootstrap resampling \citep{koehn-2004-statistical} with 95\% confidence interval (CI).
    %(computed with bootstrap resampling with 95\% confidence interval).
    }
    \label{tab:ablationASR}
\end{table}

\paragraph{ASR}

%The impact of TF32 and of the different bugs on the performance of the Conformer for the ASR task is shown in Table \ref{tab:ablationASR}. First, TF32 introduces only limited variations in the generated output (the number of generated words is slightly different), as the IBS varies. In addition, the small increase in WER (0.21) compared to \smallcorrect{} is not statistically significant.
Table \ref{tab:ablationASR} shows, in comparison to \smallcorrect{}, the impact of TF32 and of the different bugs on ASR performance. First, for each IBS, TF32 alone causes an identical, not statistically significant quality drop (+0.21 WER). This is due to minor variations in the output, attested by a slightly different number of generated words.
%
%
%
%\sara{First of all, the TF32 seems not to introduce impactful errors in the output computation as the WER remains constant when IBS varies. Although the 0.21 WER increase compared to codebase \smallcorrect{} is not statistically significant, it 
%highlights that we already obtain different outputs, thus the code exhibits an incorrect behavior, when only TF32 is activated.}
%However, we manually verified that the outputs obtained by varying IBS are not identical, which is anyway an incorrect behavior \mn{[QUESTA E' UN PO' BUTTATA LI': in che senso ``manually verified''? in che senso ``non identical''? Con diverse IBS e' sempre 10.73...che pero', vero, e' 0.21 piu' della versione bug-free...Qualunque cosa sia, e' un fatto grave, no? mi pare che quasi sorvoliate.]}.} 
% First of all, the TF32 seems not to introduce relevant errors in the output computation as the IBS varies, as the WER is constant,  although the outputs are not identical (which is, anyway, not a correct behavior). The WER increase of 0.21 is not statistically significant, thus we cannot assert that it degrades the performance. 
%Differently, 
%\sara{Moreover,}
%as
%
%
%
%
%As soon as any of the bugs (\bugone, \bugtwo, \bugthree) is introduced in the code, instead, the performance becomes highly influenced by the IBS. This phenomenon is particularly evident in the case of \bugone, for which we observe a major performance degradation (+8.78 WER) when we introduce a considerable amount of padding by increasing IBS from 1 to 100 sentences. 
As soon as any of the bugs 
%(\bugone, \bugtwo, \bugthree) 
is reintroduced in the code, the performance becomes highly sensitive to the IBS. This is particularly evident with \bugone, which causes a significant performance drop (+8.78 WER) when we introduce a considerable amount of padding by increasing IBS from 1 to 100 sentences.
%
%
%
% Noticeably,  the best result is obtained with \bugthree{} and 1 as IBS, and most of the differences with the bug-free version (\smallcorrect{}) are not statistically significant. 
It is noteworthy that most of the differences compared to the bug-free version (\smallcorrect{}) are not statistically significant, and the best ASR result is achieved with \bugthree{} and 1 as IBS.
%
%
%
% Only the presence of all bugs \bugall{} leads to a consistent and significant performance drop, although the results are more than
% %very 
% competitive with those obtained with a Transformer-based architecture on the same benchmark: \citet{CATTONI2021101155} report 26.61 WER, while \citet{gaido-etal-2021-ctc} obtain 15.6. 
Only the presence of all bugs \bugall{} 
%leads to
produces
consistent and statistically significant performance drops. Despite this, the results with 1 and 10 as IBS are still far better than those obtained with a Transformer-based architecture on the same benchmark (i.e. 26.61 by \citealt{CATTONI2021101155} and 15.6 by \citealt{gaido-etal-2021-ctc}). 
% %\mg{In addition, the results can be easily reproduced by setting the IBS to 1.}
% %
% %
% %
% %
% Notice 
% %that the bugs do not prevent the results from being reproducible: by setting the IBS to a specified value, the same result is obtained consistently and can be easily replicated.
% that the presence of  bugs does not hinder the reproducibility of the results,
%and
Moreover,
their reproducibility is not hindered by the presence of bugs,
as setting the IBS to any particular value consistently yields the same score.
%
%
%In light of these results, we can conclude that \textbf{\textit{competitive and reproducible results can be achieved even in presence of bugs}}, and focusing only on these two aspects is not enough.
In light of these results, we can conclude that \textbf{\textit{even flawed code can produce competitive and reproducible results}} and, therefore, focusing only on these two aspects is not enough
%to guarantee reliable findings.
to ensure the trustworthiness of the code.
%All in all, we can conclude that\mn{, contrary to the idea that a scientific product should ...,} \mg{\textbf{\textit{the presence of bugs can be hidden by very competitive results}} and \textbf{\textit{the reproducibility of results does not ensure code correctness}} either.} 

%\textit{\textbf{the presence of each 
% of the bugs described above cannot be noticed from the results
%\sara{bug is hidden by the competitive results}}}\sara{, even when bugs are present altogether,}
%and even the contemporaneous presence of all bugs cannot be spotted if compared with other works on the same benchmark with different architectures, 
%unless different IBSs are tested.



\begin{table}[!tb]
\small
\setlength{\tabcolsep}{2.5pt}
    \centering
    \footnotesize
    \begin{tabular}{l|ccc|ccc}
        \cline{2-7}
         & \multicolumn{3}{c|}{\textbf{en-de}} & \multicolumn{3}{c|}{\textbf{en-es}} \\
         \specialrule{.1em}{.05em}{.05em} 
        \multirow{2}{*}{\textbf{Code}} & \multicolumn{3}{c|}{\textbf{IBS}} & \multicolumn{3}{c}{\textbf{IBS}} \\
        \cline{2-7}
        & 1 & 10 & 100 & 1 & 10 & 100 \\
        \specialrule{.1em}{.05em}{.05em} 
        \smallcorrect & 24.67 & 24.67 & 24.67 & 30.34 & 30.34 & 30.34 \\
        \hline
        + TF32 & \textbf{24.84} & \textbf{24.84} & 24.83 & \textbf{30.63} & 30.62 & \textbf{30.63} \\
        \quad + \bugone & 24.52 & 24.65 & 24.67 & 29.53* & 29.41* & 27.71* \\
        \quad + \bugtwo & 24.56 & 24.57 & 24.58 & 30.53 & 30.53 & 30.53 \\
        \quad + \bugthree & 24.53 & 24.46 & 24.42 & 30.33 & 30.35 & 30.24\\
        \quad + \bugall & 24.68 & 24.58 & 23.23* & 28.57* & 27.81* & 21.15* \\
    \specialrule{.1em}{.05em}{.05em} 
    \end{tabular}
    % \caption{BLEU ($\uparrow$) \mg{scores} obtained with the different
    % %types of bug \bugall{}
    % bugs and TF32 for
    % %the ST task on MuST-C tst-COMMON
    % \mg{ST}
    % as IBS varies (1, 10, and 100 sentences). * indicates that the difference with \smallcorrect{} is statistically significant, computed with bootstrap resampling (95\% CI).
    % %(computed with bootstrap resampling with 95\% confidence interval).}
    % }
    \caption{BLEU ($\uparrow$) scores for ST obtained with TF32 and bugs as IBS varies (1, 10, and 100 sentences).
    %the st task. 
    *~indicates that the difference with \smallcorrect{} is statistically significant, computed with bootstrap resampling (95\% CI).
    %(computed with bootstrap resampling with 95\% confidence interval).}
    }
    \label{tab:ablationST}
\end{table}



\paragraph{ST} Table \ref{tab:ablationST} reports the same study on the two 
%of the 
most used language pairs of the MuST-C benchmark (en-de, and en-es).
The behavior is quite different 
%between the two language directions, although
between
%\mg{them,}
the two,
although the best results are always obtained with TF32 and without bugs. Indeed, 
%in
on
en-es the presence of \bugone{} causes statistically significant drops, which
increase 
with the IBS
%\sara{as the IBS increases}
and are exacerbated if combined with the other two bugs (\bugall).
%In
On
en-de, instead, 
%which is by far the most widely used benchmark in ST, 
none of the bugs has a significant impact on the results. Interestingly,
%with 1 as the IBS, 
the result obtained with all bugs (\bugall) and 1 as IBS is slightly 
%(0.01) 
higher (+0.01) than 
%\sara{those}
that 
without bugs (\smallcorrect).
Furthermore, 
%by 
comparing the scores 
obtained with all bugs (\bugall) and 1 as IBS with those
of previous works on ST
%and architectures with those obtained with \smallbug\textsubscript{1,2,3} and IBS 1 
(Table \ref{tab:comparison_others}), we can notice 
%that\sara{, similar to the findings for the ASR task,}
that, as previously observed for 
%the ASR task,
ASR,
%, as we have already seen for the ASR task,
%, as in ASR, 
the presence of bugs 
%is again not evident from the results\sara{, even in the case of the en-es language pair} where they cause a significant drop.
is  not evident from the results, which 
%can still be 
are still
competitive with those achieved by other 
%approaches and 
architectures.
%, \sara{as we have already seen for the ASR task, even when they cause a significant drop (en-es).}
%not even in en-es where they cause a significant drop.
%In light of this, we can strengthen 
This supports 
our previous conclusion that \textbf{\textit{good (and reproducible) results do not imply the correctness of the code}}, as this statement holds for different tasks and language pairs. 
%\sara{We reiterate that the presence of bugs can be spotted only when using higher IBS which, however, should be in theory an irrelevant hyperparameter for the performance of the system.}
We reiterate that the presence of bugs is evident only by testing with different IBS, which should be 
%an irrelevant hyperparameter for the performance of the systems.
%uncorrelated with
irrelevant for
the performance of the systems.
%Again, the presence of bugs was evident only by experimenting with different (high) IBSs, which should be an irrelevant hyperparameter.

%\mg{DA DIRE DA QUALCHE PARTE CHE TUTTI QUESTI RISULTATI SONO RIPRODUCIBILI!}
%
%For the ST task, we present the impact of TF32 and of the different bugs \smallbug\textsubscript{1,2,3} on two language directions of MuST-C v1.0, en-de, and en-es, having respectively different and similar word ordering with respect to the source language. Results are presented in Table \ref{tab:ablationST}.
%Differently from the ASR case, there is no bug that has a prevalent role over the others, except for bug \smallbug\textsubscript{1} in en-es that shows a BLEU degradation from 0.81 to 2.63 points compared to the correct code (\smallcorrect) as the batch size increases. In all the other cases, the influence of every single bug seems marginal if considered alone but is exacerbated when all the bugs are inserted in the code (\smallbug\textsubscript{1,2,3}). Therefore, we can conclude that the additive effect of bugs on performance degradation does not depend on the particular task but is valid for both ASR and ST.



\begin{table}[t]
\setlength{\tabcolsep}{10pt}
\centering
\small
\begin{tabular}{l|cc}
\specialrule{.1em}{.05em}{.05em} 
\textbf{Model} & \textbf{en-de} & \textbf{en-es} 
 \\
\specialrule{.1em}{.05em}{.05em} 
 ESPNet \cite{inaguma-etal-2020-espnet} & 22.9 & 28.0 \\
 Fairseq \cite{wang2020fairseqs2t} & 22.7 & 27.2 \\
% AFS \citep{zhang-etal-2020-adaptive} & 22.4 & 26.9 \\
 Speechformer \cite{papi-etal-2021-speechformer} & 23.6 & 28.5 \\
 E2E + ML \citep{zhao-etal-2021-mutual} & - & 28.5 \\
% MultiLang Adapters \citep{le-etal-2021-lightweight} & \textbf{24.7} & \textbf{28.7} \\
 SATE (no KD) \citep{xu-etal-2021-stacked} & 24.1 & - \\
 E2E-ST-FS \citep{pmlr-v162-zhang22i} & 23.0 & 28.0 \\
 \hline
 Conformer \bugall & \textbf{24.7} & \textbf{28.6} \\
\specialrule{.1em}{.05em}{.05em} 
\end{tabular}
% \caption{Comparison of the BLEU scores on MuST-C en-de and en-es of the Conformer implementation with all bugs using 1 as IBS with other models presented in the literature.}
\caption{
%\mn{BLEU scores on MuST-C en-de and en-es of the Conformer implementation with all bugs using 1 as IBS, and other models presented in literature.}
BLUE ($\uparrow$) scores of models trained on MuST-C en-de and en-es compared to the Conformer implemented with all bugs and 1 as IBS.
}
\label{tab:comparison_others}
\end{table}


\begin{table*}[t!]
\setcounter{table}{6}
\setlength{\tabcolsep}{5.7pt}
    \centering
    %\footnotesize
    \small
    \begin{tabular}{c|c|c||c|c|c|c|c|c|c|c||c}
    \specialrule{.1em}{.05em}{.05em} 
        \textbf{Code} & \textbf{Model} & \textbf{IBS} & \textbf{en-de} & \textbf{en-es} & \textbf{en-fr} & \textbf{en-it} & \textbf{en-nl} & \textbf{en-pt} & \textbf{en-ro} & \textbf{en-ru} & \textbf{Avg} \\
        \specialrule{.1em}{.05em}{.05em} 
        \multirow{6}{*}{\correct} & \multirow{3}{*}{Conformer} & 1 & \multirow{3}{*}{24.67} & \multirow{3}{*}{30.34} & \multirow{3}{*}{36.22} & \multirow{3}{*}{25.73} & \multirow{3}{*}{30.04} & \multirow{3}{*}{\textbf{30.55}} & \multirow{3}{*}{23.43} & \multirow{3}{*}{17.29} & \multirow{3}{*}{27.28} \\
        %\cline{3-11}
        & & 10 & & & & & & & & \\
        %\cline{3-11}
        & & 100 & & & & & & & & \\
        \cline{2-12}
        & \multirow{3}{*}{\shortstack{Conformer\\+\\CTC Compr.}} & 1 & \multirow{3}{*}{24.97} & \multirow{3}{*}{30.48} & \multirow{3}{*}{\textbf{36.43}} & \multirow{3}{*}{\textbf{26.25*}} & \multirow{3}{*}{\textbf{30.31}} & \multirow{3}{*}{30.09\textsuperscript{$\dagger$}} & \multirow{3}{*}{\textbf{24.67*}} & \multirow{3}{*}{\textbf{17.35}} & \multirow{3}{*}{\textbf{27.57}} \\
        %\cline{3-11}
        & & 10 & & & & & & & & \\
        %\cline{3-11}
        & & 100 & & & & & & & & \\
        \hline
        \multirow{6}{*}{\bug\textsubscript{1,2,3}} & \multirow{3}{*}{Conformer} & 1 & 24.68 & 28.57 & 35.70 & 25.81 & 29.68 & 30.22 & 23.52 & 15.83 & 26.75 \\
        %\cline{3-11}
        & & 10 & 24.58 & 27.81 & 35.65 & 25.70 & 29.35 & 30.02 & 23.43 & 15.36 & 26.49 \\
        %\cline{3-11}
        & & 100 & 23.23 & 21.15 & 31.70 & 23.42 & 24.92 & 27.72 & 22.68 & 11.05 & 23.23 \\
        \cline{2-12}
        & \multirow{3}{*}{\shortstack{Conformer\\+\\CTC Compr.}} & 1 & 24.95 & 30.49* & 36.27* & 25.84 & 29.42 & 30.04 & 23.96* & 17.05* & 27.25 \\
        %\cline{3-11}
        & & 10 & 25.21* & \textbf{30.72}* & 36.18* & 26.01 & 29.64 & 30.14 & 23.95* & 17.06* & 27.36 \\
        %\cline{3-11}
        & & 100 & \textbf{25.26}* & 30.52* & 36.36* & 25.88* & 29.66* & 30.16* & 23.92* & 16.87* & 27.33 \\
    \specialrule{.1em}{.05em}{.05em} 
    \end{tabular}
    \caption{BLEU ($\uparrow$) scores for ST
    %over all the 8 languages of MuST-C v1.0
    %tst-COMMON 
    %based on 
    of
    the correct/incorrect codebase 
    %of the Conformer model (
    with and without CTC compression
    %) 
    as IBS varies (1, 10, and 100). * and \textsuperscript{$\dagger$} indicate that, respectively, the improvement or the degradation 
    %given by 
    of
    %the 
    CTC compression is statistically significant, computed with bootstrap resampling (95\% CI).
    %(computed with bootstrap resampling with 95\% confidence interval).}
    }
    \label{tab:ST}
\end{table*}

\subsection{Impact of Building on Incorrect Code}
\label{subsec:impact_code}
%\mn{TO SHOWCASE...\\
%1) FACCIAMO ESPERIMENTI MIRATI.\\
%2) QUESTI ESP. MIRATI COINVOLGONO LA CTC COMPRESSION, UNA TECNICA UTILIZZATA in studi recenti.\\
%3) LA SCELTA DI CTC COMPR. E' MOTIVATA DALLE SUE CARATTERISTICHE INTRINSECHE CHE LA RENDONO POTENZIALMENTE SENSISBILE AI TRE BUG TROVATI.\\
%4) INFATTI, LA CTC COMPR. FUNZIONA COSI'...BLA.\\
%5) COME VEDEREMO, QUESTA CARATTERISTICA (IL SUO FUNZINAM.) FA SI' CHE ESPERIMENTI COL CONFORMER BACATO PORTINO A RISULTATI MOLTO POSITIVI (IN QUALCHE MODO SOSPETTI) IN CONTRADDIZIONE CON QUELLI DI ALTRI LAVORI RECENTI...\\
%6) MA QUESTI RISULTATI SONO VERITIERI????
%%%%NO, COL CONFORMER NON BACATO, L'HYPE RIENTRA...
%}

To showcase how the development of new solutions on top of incorrect code can result in misleading conclusions, we conducted ad-hoc experiments on both the correct (\smallcorrect) and incorrect (\bugall) codebases.
These experiments evaluate the effect of adding a technique that, given its intrinsic characteristics, we %speculate is 
hypothesize to be
potentially sensitive to the bugs identified in \S\ref{subsec:analysis}: the CTC compression \citep{liu2020bridging,gaido-etal-2021-ctc}.
This technique has been proposed for Transformer-based models, as it significantly reduces training and inference costs while providing only limited (if any) gains in output quality.
%, and 
It is based on the CTC mechanism \citep{Graves2006ConnectionistTC}, which 
%.
enables predicting 
%an 
output sequences of variable length
%\sara{s}
but shorter than the input ones, as in our case where transcript sub-words are predicted from audio frames.
For each input frame, the CTC produces a probability distribution over the possible target labels augmented with a \texttt{<blank>} symbol. This probability distribution is leveraged by the CTC compression (here applied to the 8\textsuperscript{th} layer) to collapse contiguous vectors corresponding to the same label by averaging them.
As a result, the CTC compression dynamically reduces the sequence length of the input, and, in turn, the amount of padding.
This leads experiments with the flawed Conformer implementations to reward the introduction of this technique with much larger gains than those shown on the Transformer architecture.
But are these improvements due to the architecture change or to the bugs?

%%%%%%%%%%%%%%%%%%%%
% \mg{These experiments evaluate}
% \sara{
%% The aim of the experiments is to investigate 
% the effect of adding the CTC compression \citep{liu2020bridging,gaido-etal-2021-ctc}, a technique proposed for Transformer-based models that significantly reduces training and inference costs  while only providing limited gains in output quality.
% This technique is based on the CTC mechanism \citep{Graves2006ConnectionistTC} that allows the prediction of output sequences of variable length shorter than the input ones, as in the case where transcript sub-words are predicted from audio frames.
% For each input time step, the CTC produces a probability distribution over the possible target labels augmented with a \texttt{<blank>} symbol. This probability distribution is leveraged by the CTC compression (here applied to the 8\textsuperscript{th} layer) to collapse contiguous vectors corresponding to the same label by averaging them.
% As a result, the CTC compression dynamically reduces the sequence length of the input, and, in turn, the amount of padding.
% Given its intrinsic characteristics, the CTC compression is potentially sensitive to the bugs identified in \S\ref{subsec:analysis} and, as we will show, its introduction to the Conformer encoder leads experiments with the flawed code to yield very positive results, contradicting those performed on Transformer. But are these improvements also valid for the correct Conformer implementation?}
%%%%%%%%%%%%%%%%%%%%
%what will the results of the experiments reveal about the effect of adding the CTC compression to the correct Conformer implementation?
% which we presume to be potentially impacted by the bugs identified in \S\ref{subsec:analysis} due to its intrinsic characteristics.
%%%%%%%%%
% These experiments 
%%involve the use of CTC compression,
% focus on the addition of the CTC compression technique, which, given its intrinsic characteristics, we speculate to be potentially sensitive to the bugs identified in \S\ref{subsec:analysis}. 
%The CTC compression has been proposed
%%%%%%%%
% \mg{\citet{liu2020bridging} and \citet{gaido-etal-2021-ctc} proposed the CTC compression}
% for Transformer-based models, showing that it greatly reduces training and inference costs
%%, contextually bringing 
% \sara{but brings} limited gains in terms of output quality\mg{.} 
%%%%%%%%
%\citep{liu2020bridging,gaido-etal-2021-ctc}.
%
%a technique introduced for ST that greatly reduces training and inference costs, but brings limited gains in terms of output quality of Transformer-based models \citep{liu2020bridging,gaido-etal-2021-ctc}.}
%a popular technique adopted in speech processing that, given its intrinsic characteristics, is potentially sensitive to the bugs identified in \S\ref{subsec:analysis}.
%Specifically, CTC compression aims at reducing the sequence length of the input while maintaining the original content. 
%%%%%%%%%%%%%
% It is based on the CTC mechanism \citep{Graves2006ConnectionistTC}, which enables the prediction of output sequences of variable length shorter than the input ones, as in the case in which the transcript sub-words are predicted from audio frames.
%%source audio frames are used to predict the transcript sub-words.
% For each input time step, the CTC produces a probability distribution over the possible target labels augmented with a \texttt{<blank>} symbol. This probability distribution is leveraged by the CTC compression (here applied to the 8\textsuperscript{th} layer) to collapse contiguous vectors corresponding to the same label by averaging them.
% As a result, the CTC compression dynamically reduces the sequence length of the input, and, in turn, the amount of padding.
% As we will see, this characteristic leads experiments with the flawed code to yield very positive results, 
%%\mg{with significantly larger gains than those previously assessed. In the following, hence, we will examine: are these improvements real?}
% \sara{contradicting those performed on Transformer. In the following, we examine: are these improvements also valid for the correct Conformer implementation?} 
%%%%%%%%%
%contradicting those of other recent works on Transformers \citep{liu2020bridging,gaido-etal-2021-ctc} indicating that CTC compression greatly reduces training and inference costs but brings limited gains in terms of output quality.
%In the following, we will examine whether these improvements are also valid for the correct Conformer implementation.


\begin{table}[t]
\setcounter{table}{5}
%\setlength{\tabcolsep}{4pt}
    \centering
    \small
    \begin{tabular}{l|c|ccc}
    \specialrule{.1em}{.05em}{.05em} 
        \multirow{2}{*}{\textbf{Model}} & \multirow{2}{*}{\textbf{Code}} & \multicolumn{3}{c}{\textbf{IBS}} \\
        \cline{3-5}
         & & 1 & 10 & 100 \\
        \specialrule{.1em}{.05em}{.05em} 
        Conformer & \multirow{2}{*}{\smallcorrect} & 10.52 & 10.52 & 10.52 \\
        \hspace{0.5em}+ CTC Compr. & & 10.64 & 10.64 & 10.64 \\
        \hline
        Conformer & \multirow{2}{*}{\bugall} & 11.32 & 14.25 & 54.56 \\
        \hspace{0.5em}+ CTC Compr. & & 10.39* & \textbf{10.34}* & 10.81* \\ 
    \specialrule{.1em}{.05em}{.05em} 
    \end{tabular}
    \caption{WER ($\downarrow$) scores for ASR
    %on
    %\mg{MuST-C}
    %the transcripts of MuST-C en-es tst-COMMON
    %based
    %on the correct/incorrect codebase of the Conformer model
    of the correct/incorrect codebase
    %Conformer implementations 
    with 
    %(\bugall) 
    and without %(\smallcorrect) 
    %bugs
    %(with and without 
    CTC compression
    %) 
    as IBS varies (1, 10, and 100). * indicates that the improvement 
    %given by the 
    of
    CTC compression is statistically significant, computed with bootstrap resampling (95\% CI).
    %(computed with bootstrap resampling with 95\% confidence interval).}
    }
    \label{tab:ASR}
\end{table}





\paragraph{ASR}
Table \ref{tab:ASR} shows the effects on the ASR performance of the introduction of the CTC compression to both codebases (\bugall{} and \smallcorrect).
%on the ASR performance of introducing the CTC compression both with and without the bugs. 
We can notice that
%without the bugs, 
the CTC compression causes a small 
% \sara{(+0.12 WER)}
%(
and not statistically significant
%) 
performance degradation
(+0.12 WER)
when the correct implementation (\smallcorrect{}) is used (in accordance with the findings of \citealt{gaido-etal-2021-ctc} on the Transformer architecture).
%, as the WER increases by 0.12 (in accordance with the findings of \citealt{gaido-etal-2021-ctc} on the Transformer architecture). 
%When the bugs are introduced, 
When bugs are present in the codebase (\bugall),
instead, the conclusion is overturned:
%completely overturned:
the CTC compression brings statistically significant improvements (nearly -1 WER even with 1 as IBS). In addition, the best overall result is 
%obtained
achieved with 
%\sara{10 as IBS and}
%the 
%CTC compression \mg{and the \bugall{} codebase}
the \bugall{} codebase (with 10 as IBS and CTC compression).
This suggests that the introduction of CTC compression yields significant performance improvements, a finding that however is not verified when the correct codebase is used.
%\mn{[A voi sembrera' banale/inutile, ma io espliciterei quindi il ``misleading finding'' a cui si giunge a causa dei bachi (la CTC porta anche a miglioramenti di prestazioni e non solo di riduzione dei costi, giusto? Quindi si sopravvaluta la CTC?)]}
Therefore, we can conclude that \textit{\textbf{building on incorrect code can produce misleading findings}}.
%, \sara{since the introduction of bugs does not necessarily degrade the results.}
%and all the bugs
%\sara{implemented in the \bugall{} codebase.}
%(and 10 as IBS). 
%As such, not only we can reiterate 
%\sara{Therefore, we can affirm that}
%, but we also showed that 
%\sara{\textit{\textbf{good results not only can hide incorrect codebases but also lead to misleading findings}} since the}
%\textit{\textbf{building on incorrect code can produce misleading findings}}, 
%and the introduction of noise \mn{[Sorry ma non capisco: cos'e' questo ``noise''?]} at inference time does not necessarily degrade the results (as the padding introduced with 10 as IBS improves the performance with respect to 1 as IBS). 
%\mn{[Siamo in un altro punto ``All in all'': limare al massimo e far brillare il messaggio. Quest'ultima frase e' una chiusa un po' criptica a non so quanto efficace. Il passaggio si dovrebbere chiudere con qualcosa di splendente su cosa significa avere bachi che portano a misleading findings. Ora, invece, recepisco la chiusa allo stesso modo in cui recepisco il testo della canzone di Madame a Sanremo ;-)]}
%\sara{compared to that obtained with IBS 1).}





\paragraph{ST}
Table \ref{tab:ST} reports the same analysis on the 8 language pairs of MuST-C v1.0. As in ASR, the presence of 
%the 
bugs (\bugall) unfairly rewards the CTC compression mechanism, which achieves statistically significant gains 
%with IBS 100
%among
on
all the languages with 100 as IBS and 
%also on (at least) 4 out of 8
on 4/5 out of 8
languages with 1/10 as IBS.
%1 or 10.
%
%
%on at least 4 out of 8 language pairs at every IBS and on all languages with 100 sentences as IBS. 
With the 
%correct code \mn{[
bug-free version
%of Conformer?]}
(\smallcorrect), instead, the improvements are 
%only 
statistically significant only on two language pairs (en-it, and en-ro), while on en-pt there is a statistically significant degradation. 
%Interestingly, en-it is one of the only two languages where the improvement is significant with the correct code, while is among the languages where the improvements are statistically significant only with 100 as IBS with the bugs. 
% On average 
On average,
on all the language directions, the 
% improvement achieved
gain brought 
by the CTC compression
%\sara{on codebase \bugall{}}
in presence of 
%the 
bugs (\bugall)
ranges from 0.5 BLEU 
%(IBS=1) 
%\sara{(with IBS 1)}
(with 1 as IBS)
to 4.1 BLEU 
%(IBS=100) 
%\sara{(with IBS 100)}
(with 100 as IBS),
%in the presence of the bugs 
while 
%the gain 
it is only of 0.29 BLEU with 
the correct code~(\smallcorrect). 
%\sara{codebase~\smallcorrect.}
Moreover, the best scores for en-de and en-es
%(which curiously are the most widespread benchmarks) 
are 
%\sara{obtained with CTC Compession on the buggy code~(\bugall).}
achieved with the presence of 
%the 
bugs 
%(\bugall+CTC compression).
(\bugall) and CTC compression.
We can hence affirm that
%utilizing flawed codebases
\textbf{\textit{the presence of 
%the 
bugs
led to erroneous findings}} also 
%in the case of the ST task,
in the ST task,
as the introduction of CTC compression appears to significantly improve 
% the translation quality
translation quality
%,
%result in significant performance improvements, 
while this is not the case with the correct implementation.
%which are instead not supported by the results obtained from a correct codebase.
%We can hence confirm that the use of incorrect codebases can lead to misleading findings because, also in the ST task, the CTC compression seems to bring statistically significant gains while, instead, these gains are not verified on a correct codebase.
%This further demonstrates that
In addition, it shows that
\textit{\textbf{it is impossible to assess code correctness only by looking at the results}} since, for instance, the average performance gap between 
%\bugall{} and \smallcorrect{} codebases
\bugall{} and \smallcorrect{}
can be as little as 0.21 BLEU (when IBS is 10) and
% can further be
may be further
narrowed, or even overturned, with a 
%simple IBS
IBS
\enquote{tuning}.

%We can hence confirm on the ST task the finding of the previous ASR section, i.e. that \textit{\textbf{building on incorrect code can produce misleading findings}}, as the presence of bugs may lead to erroneously concluding that the CTC compression brings statistically significant gains on most of the language pairs without degrading on any of them, while this is not true. In addition, it further demonstrates that is virtually \textit{\textbf{impossible to assess code correctness from the results}}, as, for instance, the difference between correct and buggy code with CTC compression can be as little as 0.21 BLEU (with IFS=10) on average over 8 language pairs, and we cannot exclude that a ``tuning'' of the IBS could further narrow the gap (or even overturn it).


\section{Correctness Checklist}
\label{sec:checklist}
After showing that code correctness is currently 
%evaluated in peer reviews
assessed only on the basis of the results shown in the paper (\S\ref{sec:core-idea}) and that results are not a valid measure of code correctness (\S\ref{sec:case-study}),
in this section, we introduce a checklist of recommendations (Table \ref{tab:checklist})
%In addition, we would emphasize that the scope (and the name) of our \enquote{Correctness Checklist} should be interpreted analogously to that of the Reproducibility Checklists present in literature: its intent is to promote and improve code correctness not to certify or guarantee it.}
that we encourage researchers to adopt to 
%improve and enforce
improve the correctness of research codebases.

%the Correctness Checklist (Table \ref{tab:checklist}), a list of several recommendations that we encourage the authors to check before developing their codebases and declare when submitting their scientific artifacts. 



%First of all,
First,
as testing a piece of code is the only way to enforce that it works correctly,
% we argue that \mn{[SICCOME E' ABBASTANZA OVVIO, PIU' CHE ARGUE TROVEREI UN'ESPRESSIONE CHE INDICHI CHE CI RIFACCIAMO ALL'ABC DELLA PROGRAMMAZIONE. NON SO COME DIRE MEGLIO QUESTA COSA; SOLO CHE ``ARGUE'' MI DA' L'IDEA DI UNA GRAN SCOPERTA.]} the only way to enforce that a
%\sara{piece of code}
%\mg{code unit} 
%behaves correctly is by testing it. Specifically, 
we recommend that researchers write Unit Tests
%or UTs \citep{10.1145/987305.987309,10.1145/800027.808473,10.5555/1349795,8048665}
(UTs) to check that their code has the expected behavior \citep{10.1145/987305.987309,10.1145/800027.808473,10.5555/1349795,8048665}.
%,
%or even first write the tests and then the actual code, 
%possibly 
Possibly, this should be done even before writing the actual code as per the popular test-driven development 
%(TDD) 
practice~\citep{beck2002driven}.
%The UTs
UTs should cover all the assumptions about how the code works, e.g. the presence of padding does not alter the 
%result.
results.
In this case, a UT should enforce that the output is the same when the batch size changes.
%ensure that the results do not vary when batch size changes.
%
%
%
%
%One of the best practices we strongly suggest adopting to improve the code \textit{traceability} is the use of Unit Tests or UTs \citep{10.5555/1349795,8048665} to check every portion of it, if not first writing the tests and then the actual code, as per the popular test-driven development (TDD) practice \citep{beck2002driven}.
%For example, more than one batch size can be tested when developing a code of a new architecture to check the consistency among the obtained results since they do have not to vary when the batch size varies.
%For example, testing multiple batch sizes when developing a new architecture code ensures consistency in its implementation as the results should not vary with batch size changes.
%
%
%
%
% Although researchers may initially perceive that adding UTs is an additional and undesirable cost, which sums to all the other activities, this feeling is exaggerated and the actual overhead is lower \citep{oro3667,Ellims2006}. \citet{Williams-2003-TDD} proved that writing UTs does not impact the productivity in writing code, as the initial overhead\footnote{Estimated in 16\%-35\% of the overall software development cost \citep{George-2003-tdd,google-ut}.} pays back by saving time spent in (unsuccessful) manual experiments \citep{google-ut}. 
%
Although researchers may initially 
%perceive
view
%\mg{consider}
%that adding UTs is
writing UTs as
an excessive and undesirable cost added to all the other activities,
%this feeling should be downsized and put into perspective.
%\mg{\citet{oro3667,Ellims2006} show that this feeling is exaggerated.}
\citet{Williams-2003-TDD} proved that 
%%writing
adding UTs
%\sara{it}
does not impact the productivity in writing code and \citet{oro3667,Ellims2006} showed that 
%this feeling is exaggerated.
the perceived cost is \enquote{\textit{exaggerated}}.
%\sara{the perceived costs of UTs are \enquote{\textit{exaggerated}}.}
%, as
Indeed,
the initial overhead\footnote{Estimated in 16\%-35\% of the overall software development cost \citep{George-2003-tdd,google-ut}.} pays back by saving time spent in (unsuccessful) manual experiments \citep{google-ut}. 
%Other studies \citep{oro3667,Ellims2006} conclude that the actual overhead is even lower.
%In addition,
Moreover, since 
%manual 
%NLP 
experiments often involve costly hardware with a significant environmental footprint
%in NLP
\citep{strubell-etal-2019-energy}, UTs may reduce the computing and environmental costs of NLP research.
%
%

As a complete test coverage of the code is utopian, we also encourage researchers 
%%%%to state in the \enquote{Limitations} section which assumptions they have on the code behavior that were not tested.
%\mn{to state in the \enquote{Limitations} section which assumptions they have made regarding the code behavior that were not tested.}
to include in the \enquote{Limitations} section any assumptions they have made about the code behavior that have not been tested.
In addition, it is important to notice that UTs should be executed every time 
%a modification is introduced in the codebase, even if it 
%%is seemingly
%\mg{seems} an
the codebase is modified, even in case of a seemingly
\textit{unrelated change}.
% a new change is introduced in the codebase, even though the change is apparently unrelated. \mn{[RISPETTO A COSA? INTENDEVATE INNOCUO - harmless?]}. 
Indeed, their validity ends when a new version of the code is used. This is commonly enforced through continuous integration (CI), which executes all UTs at every code change~\citep{Duvall-2007-ci}. 
%The presence of a (running and successful)
A running and successful CI has the additional benefit of enforcing that 
% the open-source version of the code released by the authors works as they expect
any open-source release of the code works as the authors expect
and provides implicit guidance on how to install and successfully run the code %to
for
people trying to reproduce a study.
%paper.


\begin{table}[!tb]
\setcounter{table}{7}
\begin{tcbitemize}[%
        raster columns=1,
        raster equal height,
        raster width=.47\textwidth,
        before=,after=\hfill,
        boxsep=3pt, left=6pt, right=6pt,
        colframe=teal!75!black,colback=white,
        fonttitle=\large\bfseries,
        halign=left,
        ]
\tcbitem
\begin{itemize}
\small
\justifying
\item[\checkinbox] All the assumptions about code behavior have been tested with Unit Tests (UTs).
\item[\checkinbox] All the assumptions that were not tested with UTs have been included in the \enquote{Limitations} section.
\item[\checkinbox] UTs have been executed on the last version of the code, which has been used for the experiments and has been released.
\item[\checkinbox] Every contribution to the codebase has been reviewed (and approved) by one or more people.
\item[\checkinbox] %(Preferable) 
The repository contains a continuous integration 
%(or CI) 
tool
%CI \mn{[ESPANDERE NEL CASO UNO, COME ME, SI FOSSE PERSO CHE CI = CONTINUOUS INTEGRATION?]}, 
which installs the software and runs the UTs (this also helps reproducibility as it forces listing all dependencies and showing how to install the code).
%
%lists all the dependencies needed to correctly install the software and run the experiments.
%\item[\checkinbox] More than one batch size was tested to check the consistency among the obtained results (i.e., the results do not vary when the batch size varies). 
\end{itemize}
%\raggedright\textbf{Browse All Courses in Programming}
\end{tcbitemize}
\caption{
Correctness
%\mg{Coding best-practice}
Checklist.}
\label{tab:checklist}
\end{table}


%At last,
Lastly,
we promote the adoption of a code reviewing practice \citep{7589787}, 
%which requires that 
in which
all changes are reviewed and approved by a person different from the 
%author of the code.
code author.\footnote{The reviewer(s) can be any person who is not the author of the code with basic knowledge of the codebase, such as lab teammates or advisors.} 
This 
%practice aims
aims
not only at avoiding bugs but also at improving 
%the 
code 
%readability and 
readability,
documentation,\footnote{Example of documentation standards can be found at \url{https://help.github.com/articles/about-readmes/}.} 
% which improve its reusability \citep{Chen2019,Bahaidarah-2022-toward,Trisovic2022}, and hence reproducibility.
reusability \citep{Chen2019,Bahaidarah-2022-toward,Trisovic2022} and, in turn,
%results'
the
reproducibility of results.

%The reviewer(s) can be any person who is not the author of the code with basic knowledge of the codebase, such as lab teammates or advisors.







%Along this line, we recommend listing all the dependencies of the released repository to allow a correct installation \citep{bestpractices} and improve code \textit{completeness}.

% FOOTNOTE
%The name of our \enquote{Correctness Checklist} should be interpreted analogously to that of the \enquote{Reproducibility Checklist} required by many conferences: its intent is promoting and improving code correctness rather than ensuring it.


% OLD CONCLUSION
%To conclude, we hope that the NLP research community will devote more attention to coding best practices and to the role of code correctness in research. As a first step, we propose the adoption of a checklist similar to the one we propose here, as it was done for reproducibility. Moreover, we strongly encourage considering coding practices and efforts toward code correctness as an important contribution while evaluating a paper.

In conclusion, this study 
%seeks to raise awareness among NLP researchers 
\sara{aims to increase researchers' awareness}
regarding the importance of coding best practices and the role of code correctness. 
%As an initial step, we recommend the adoption of the above \enquote{Correctness Checklist}, which is intended to promote and enhance code correctness rather than to certify it (which would be utopian), similar to the \enquote{Reproducibility Checklist} that many top-tier venues require. 
As a first step, we 
%propose
\mn{encourage}
the adoption of 
%a
\mn{the above}
\enquote{Correctness Checklist}, whose name 
should be interpreted 
%analogously
\mn{in analogy}
to that of the \enquote{Reproducibility Checklist} \mn{now} required by many 
%conferences
\mn{top-tier venues}: its intent is %promoting and improving 
\mn{fostering}
code correctness rather than 
%ensuring it, which would be utopian.
\mn{certifying it.}
Furthermore, to address the current lack of incentives for code quality, we urge the NLP community to consider it an important contribution while evaluating new works.


% \mg{To conclude, this study aims at sensitizing \mn{NLP} researchers to devote more attention to coding best practices and to the role of code correctness in research.
% As a first step, we 
% %propose
% \mn{encourage}
% the adoption of 
% %a
% \mn{the above}
% \enquote{Correctness Checklist}, whose name should be interpreted 
% %analogously
% \mn{in analogy}
% to that of the \enquote{Reproducibility Checklist} \mn{now} required by many 
% %conferences
% \mn{top-tier venues}: its intent is %promoting and improving 
% \mn{fostering}
% code correctness rather than 
% %ensuring it, which would be utopian.
% \mn{certifying it.}
% In addition, as a countermeasure to the current lack of incentives toward code quality, we encourage considering it as an important contribution while evaluating a work
% \mn{research contribution}.}

%In conclusion, the importance of code correctness cannot be overstated in NLP research. As we have demonstrated in this paper, code bugs can have serious consequences on the validity of research findings. Therefore, we call upon the NLP community to give more attention to coding best practices and to consider code correctness as a crucial component of their research. Similar to the Reproducibility Checklist that has been widely adopted, we propose the adoption of a \enquote{Correctness Checklist} as a first step towards this goal. This checklist can serve as a tool to promote and improve code correctness in NLP research, and we strongly encourage the evaluation of papers to take into account coding practices and efforts towards code correctness. By working towards a culture of code correctness in NLP research, we can ensure that our findings are not only reproducible, but also accurate and trustworthy.

%this 
%
%aspect into the review forms by explicitly asking the reviewers to answer/give a score/give a judgment on it but also by encouraging the authors to compile the Correctness Checklist during their submissions.


%On these aspects, plenty of literature already discussed the need to go beyond code openness \citep{Chen2019,Trisovic2022}, highlighting the role of proper documentation and testing in different environments. We reiterate the benefits brought by improving code readability and documentation of research repositories. While there is currently no incentive to do so \citep{barba-2019-praxis}, an ethical commitment toward more reusable code brings benefits to the whole community and increases  the reproducibility of the works. Similarly, they increase \textit{maintainability}, which measures the effort needed to make specific modifications to the code. This eases the code reuse by other researchers, as well as it eases building future works on the same codebase.


%Although the community showing interest in discussing the ways in which reported performance improvements on NLP benchmarks are meaningful, as demonstrated by the last years' theme tracks of conferences like EMNLP\footnote{\url{https://2022.emnlp.org/calls/main_conference_papers/\#emnlp-2022-theme-track}} and ACL\footnote{\url{https://2023.aclweb.org/calls/main_conference/\#theme-track-reality-check}}, little efforts have been devoted to push towards improved technical correctness of the published scientific artifacts.

%We hope that the Checklist will be also useful as a guideline for better software.

%Therefore, we suggest the authors to 
%and  the Correctness Checklist which we suggest being checked before submitting papers containing scientific artifacts. The guidelines are presented in Table \ref{tab:checklist}.





\section{Conclusions}
%%% MIA
% \mg{We demonstrated that, among the constituent elements of a scientific work, the code is the only one whose quality and correctness are currently largely neglected. Through a use case involving the widespread Conformer architecture, we empirically demonstrated in two tasks and different language settings that the risks of this disregard include the potential for drawing incorrect or misleading conclusions.
% %As a countermeasure,
% In response, we proposed a Correctness Checklist aimed at sensitizing researchers to the importance of adopting coding best practices, with the
% %hope
% ultimate goal of increasing the quality and correctness of research codebases in the NLP community.}
% % of coding best practices, whose adoption we hope will spread in the NLP community, so as to increase the quality and correctness of research codebases.

%%% ALTERNATIVA CHATGPT
%\mg{Our study has shown that, among the different constituents of a scientific work, the quality and correctness of code are often overlooked. Using the widely-used Conformer architecture as a case study, we have empirically demonstrated in multiple tasks and languages that neglecting the quality and correctness of code can lead to the risk of drawing incorrect or misleading conclusions. To address this issue, we have proposed a Correctness Checklist to encourage researchers to adopt coding best practices, and ultimately improve the quality and correctness of research codebases in the NLP community.}


%%% MERGE
Our study has shown that, among the different constituents of a scientific work, the code is the only one whose quality and correctness are currently overlooked. 
%Using the implementations of the widespread Conformer architecture as a case study, 
Through a case study involving implementations of the widespread Conformer architecture,
we empirically demonstrated in two tasks and multiple languages that the risks of %this
such
disregard include drawing incorrect or misleading conclusions.
%As a countermeasure,
In response, we proposed a Correctness Checklist aimed at sensitizing researchers to the importance of adopting coding best practices, with the ultimate goal of increasing the quality and correctness of research codebases in the NLP community.


\section*{Limitations}



% \sara{Our case study focused on the use of the Conformer architecture 
% %for the two most popular speech processing tasks: 
% \mn{in the two most popular speech processing tasks in which it is used:}
% speech recognition and translation.} 
To back up our call to action toward the adoption of coding best practices aimed at fostering correctness and improving the quality of the developed software, we presented a case study involving the use of the Conformer architecture in the two most popular speech processing tasks\sara{:}
%in which it is used: 
speech recognition and translation.
Although the effects of the presence of bugs might be found also in other 
%\mn{speech-related} 
scenarios, such as 
%\mn{like}
text-to-speech, speech emotion recognition, spoken language understanding, and speech separation, we did not cover them in this paper. 
%\mn{Although this was out of the scope of our study, its natural extension is to explore the undesired impact of the bug we isolated on these tasks.}
While the undesired effect of the 
%bug
bugs
we isolated (and corrected) was 
%sufficiently 
empirically
demonstrated, extending the analysis to 
%more tasks
other research areas
%is
would be
a natural extension of our study,
%\sara{
which could provide a more comprehensive understanding of the impact of
%code correctness
the identified bugs
on the broader NLP community working on speech-related tasks.
%}
%with a high potential impact on the broader community working on speech-related tasks.

Moreover, in our case study, we examined the open-source implementations of Conformer,
%Our case study focused on the open-source implementations of the Conformer architecture, 
%from 
in which we identified three types of bugs related to the Convolution Module, Initial Subsampling, and Positional Encodings.
While we found efficient solutions for the first two bugs, 
%in the last case 
for the last one
our fix introduces a significant overhead. As a result, the implementation
%In correcting the bugs, we found efficient solutions for the first two types of bugs but fail for the last one. Currently, the implementation that
we release, although correct,
%is slower compared to the implementation with bugs of about XXXX.
increases the training time of the models.
%We also hope that, as we release open-source our code, other members of the research community can optimize our solution, overcoming this limitation.
We are confident that, by open-sourcing our code, the community will soon find a way to optimize it and overcome this limitation, capitalizing on our findings and spreading the use of more reliable versions of a state-of-the-art architecture.



%Nonetheless, we believe that the importance of building on correct code far outweighs the additional computational cost and, in addition, we cannot exclude that the implementation can be improved through a software engineering process.

%% TTS, Speech emotion recognition, spoken language understanding, speech separation



% Entries for the entire Anthology, followed by custom entries
\bibliography{custom}
\bibliographystyle{acl_natbib}


\appendix

\section{Data Statistics}
\label{sec:data_stats}

%\begin{table}[!htb]
%    \centering
%    \small
%    \setlength{\tabcolsep}{2pt}
%    \begin{tabular}{l|cccccccc}
%    \specialrule{.1em}{.05em}{.05em} 
%        \textbf{Split} & \textbf{de} & \textbf{es} & \textbf{fr} & \textbf{it} & \textbf{nl} & \textbf{pt} & \textbf{ro} & \textbf{ru} \\      
%        \specialrule{.1em}{.05em}{.05em} 
%        train & 225.3 & 260.1 & 269.3 & 248.2 & 243.5 & 201.5 & 231.5 & 259.5 \\
%        dev & 1.4 & 1.3 & 1.4 & 1.3 & 1.4 & 1.4 & 1.4 & 1.3 \\
%        test & 2.6 & 2.5 & 2.6 & 2.6 & 2.6 & 2.5 & 2.6 & 2.5 \\
%    \specialrule{.1em}{.05em}{.05em} 
%    \end{tabular}
%    \caption{Number of samples (in thousands) of the train, dev, and test (tst-COMMON) sets for each language of MuST-C v1.0.}
%    \label{tab:data-stats}
%\end{table}


\begin{table}[!htb]
    \centering
    \small
    \setlength{\tabcolsep}{2pt}
    \begin{tabular}{l|cccccccc}
    \specialrule{.1em}{.05em}{.05em} 
        \textbf{Split} & \textbf{de} & \textbf{es} & \textbf{fr} & \textbf{it} & \textbf{nl} & \textbf{pt} & \textbf{ro} & \textbf{ru} \\      
        \specialrule{.1em}{.05em}{.05em} 
        train & 387.5 & 478.6 & 469.5 & 441.5 & 421.4 & 363.7 & 409.8 & 466.4 \\
        dev & 2.5 & 2.5 & 2.5 & 2.5 & 2.5 & 2.5 & 2.5 & 2.5 \\
        test & 4.1 & 4.1 & 4.1 & 4.1 & 4.1 & 4.1 & 4.1 & 4.1 \\
    \specialrule{.1em}{.05em}{.05em} 
    \end{tabular}
    \caption{Number of hours of the train, dev, and test (tst-COMMON) sets for each language of MuST-C v1.0.}
    \label{tab:data-stats}
\end{table}

\section{Architecture and Training Settings}
\label{sec:exp_sett}

%\subsection{Architecture and Training Settings}
Our Conformer-based architecture is composed of 12 Conformer \citep{gulati20_interspeech} Encoder layers and 6 Transformer \citep{NIPS2017_3f5ee243} Decoder layers, with 8 attention heads each. Embedding size is set to 512 and hidden neurons in the feed-forward layers to 2,048, with a total of 
%$\sim$115M
114,894,730
parameters for the model. Dropout is set to 0.1 for feed-forward, attention, and convolution layers. The kernel size of the Convolution Module is set to 31 for both point- and depth-wise convolutions. 
We train all the models using Adam \citep{journals/corr/KingmaB14} optimizer (betas $(0.9, 0.98)$) and label-smoothed cross-entropy (LSCE) loss (smoothing factor 0.1).
We also use an auxiliary Connectionist Temporal Classification or CTC loss \citep{Graves2006ConnectionistTC} during training to ease convergence and obtain competitive results without pre-training the encoder with that of an ASR model \citep{gaido-etal-2022-efficient}. The auxiliary loss is summed to the LSCE with 0.5 relative weight. 
The learning rate is set to $2\cdot10^{-3}$ with Noam scheduler \citep{NIPS2017_3f5ee243} and 25k warm-up steps. 
The vocabularies are based on SentencePiece models \citep{sennrich-etal-2016-neural} with size 5,000 for the English source \citep{inaguma-etal-2020-espnet} and 8,000 for each of the ST target languages \citep{di-gangi-etal-2020-target}. We set 100k maximum updates but we early stop training after 10 epochs without loss decrease on the dev set and average 5 checkpoints around the best (best, two preceding, and two following). All trainings are performed 
%on 2 NVIDIA A40 40GB of RAM, 
with 40k tokens as batch size and 4 as update frequency on two GPUs.
All other settings are the default of Fairseq-ST~\citep{wang2020fairseqs2t}, which we forked as a base of our implementation.
SpecAugment \citep{Park2019} is applied during training, while utterance-level Cepstral mean and variance normalization is performed both at training and inference time. 
%The implementation is a modified version of Fairseq-ST \citep{wang2020fairseqs2t} that is released at \url{[anonymous\_url]}.
Trainings lasted 18-33 hours depending on the model configuration (e.g., with the fixes or not, with or without CTC compression) and the language pair due to the different sizes of the training data.




\end{document}
