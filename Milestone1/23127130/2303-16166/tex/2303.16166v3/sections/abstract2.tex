% % %Despite the crucial role of code in scientific endeavors, the evaluation of code correctness is normally disregarded in the peer-review process.
% % %\mn{Despite its crucial role}
% % \mn{Despite its paramount importance
% % in scientific endeavours, the assessment of code correctness is 
% % frequently neglected in the peer-review process.} 
% % %Rather, code correctness is often assumed based on the quality of research results, resulting in the potential for erroneous outcomes and inaccurate findings, even when the results are easily reproducible
% % \mn{Instead, code correctness is often presumed to be valid 
% % based on the perceived quality of research results, at the risk of  erroneous outcomes and inaccurate, potentially misleading findings even when the results are easily reproducible.}
% \mg{The peer-review process currently evaluates the technical correctness of a paper solely on the basis of the perceived quality of research results. The code used to obtain these results, instead, is presumed to be correct}
% at the risk of  erroneous outcomes and inaccurate, potentially misleading findings\mg{.}
% %\mg{,} even when the results are easily reproducible.}
% %potentially erroneous outcomes and inaccurate findings even when the results are easily reproducible.}
% %
% %
% %\mn{[Sarei partito dalla riproducibilita' (che ``non e' tutto'') per poi passare alla correttezza, in parallelo con il titolo. Messa cosi' ``even when...'' non si capisce bene la logica del vostro ragionamento.]}. 
% % This paper illustrates the importance of independent \mn{[Indipendente da cosa? Non mi pare che il concetto ritorni nel paper]} code evaluation through a case study analyzing the existence of bugs in open-source implementations of the Conformer architecture. The study demonstrates that the presence of bugs yields \mn{[girare: non dire che ``la presenza dei bachi porta a risultati riproducibili ma...''; se mai, che ``nonostante i risultati siano riproducibili e confrontabili, la presenza dei bachi rende i findings sbagliati e inaffidabili'']} reproducible and comparable results, which are nonetheless erroneous \mn{[sa di ripetitivo rispetto alla seconda frase]}. 
% To raise awareness of this problem, we argue that the current emphasis on result reproducibility should go hand in hand with
% \mg{the 
% %attention to the 
% \sara{emphasis on}
% quality and correctness of research software.}
% %a thorough examination of the code used to obtain these results.
% We bolster our call to the scientific community by presenting a case study that 
% %compares different widely used 
% \mg{identifies (and corrects) three bugs in}
% %the} 
% open-source implementations of the state-of-the-art Conformer architecture for two tasks (automatic speech recognition and %speech 
% translation). 
% The results of our experiments 
% %in various language settings
% \mg{on 
% %eight
% \sara{8}
% language pairs} demonstrate that the 
% %existence
% \mg{presence} of bugs yields reproducible and comparable results, which are nonetheless erroneous
%%%% ALTERNATIVA \mn{The results of our experiments in various language settings demonstrate that the existence of bugs yields reproducible and comparable results, which are nonetheless erroneous.}
% % Through comparative experiments in various language settings, we demonstrate that the 
% % %existence 
% % presence
% % of bugs yields results that, though easily reproducible, are nonetheless erroneous.}
% and potentially misleading.
% %
% %
% % To mitigate this issue, we advocate for the implementation of a set of best practices aimed at ensuring code correctness and promoting software quality within the scientific community \mn{[IN generale o ci rivolgiamo espressamente a quella NLP? Visto che l'analisi delle review guidelines attuali e' in NLP/ML conferences, io terrei quel profilo li' senza pretendere di rivolgermi urbi et orbi.]}.
% In 
% %light of
% response to
% this, we 
% %advocate for the implementation 
% %recommend
% \mg{advise}
% the adoption
% of best practices aimed at ensuring code correctness and 
% %promoting
% improving
% software quality within the 
% %scientific 
% NLP
% community.


%%%%%%%%%%%%%%%%%%%%%%%%%%%%%%%%%%%
\mn{Despite its paramount importance in the dissemination of  
 research outcomes, code correctness 
%(a crucial aspect of the technical quality of most scientific contributions) 
(a 
%crucial aspect of
component of 
%overall technical quality)
technical soundness)
is often presumed to be valid based on the perceived quality of the experimental results. 
%Even when the results reported are easily reproducible, this 
This comes with the risk of erroneous outcomes 
%and 
that can propagate and reinforce
%inaccurate,  potentially 
misleading findings. 
%To raise awareness of this problem, we argue that the current emphasis on result reproducibility should go hand in hand,  with a thorough examination of the code used to obtain these results. 
To address this issue, we 
%propose
%argue 
posit 
that the current focus on result reproducibility 
%(in the peer-review process) must be complemented by a thorough assessment (before paper submission) of the code used to obtain these results.
should go hand in hand with the emphasis on the 
%quality and 
correctness of research software.
%Our call to action to the NLP community is supported by  
We bolster our call to the NLP community by presenting 
%a case study that  compares different widely used open-source implementations of the state-of-the-art Conformer architecture for two tasks (automatic speech recognition and translation).
a case study, in which we identify (and correct) three bugs in widely used open-source implementations of the state-of-the-art Conformer 
architecture.  
Through comparative experiments 
%in two tasks (automatic speech recognition and translation) and 
on automatic speech recognition and translation 
in various language settings, we demonstrate that the existence of bugs yields results that, though easily reproducible, are nonetheless erroneous. 
% In response to this, this paper is a plea towards the adoption of best practices aimed at 
% %enforcing
% favouring code correctness. 
% Recognizing the need to not add undue burden to the peer-review process with 
% %extensive 
% costly code quality verification procedures, we emphasize the importance of giving equal weight to code correctness by developers and authors as other essential elements of a reputable scientific contribution in NLP.
%In response to this, 
%% SUGGERIMENTO A. PILZER: In light of these findings, this paper is a call to action towards the adoption of code development best practices. Fostering correctness during code development, prior to its release and and recognizing the need to avoid overburdening the peer-review process.
In light of these findings, recognizing the need 
%to not add undue burden to 
%not to overburden
to avoid overburdening 
the peer-review process with costly code quality verification procedures, this paper is 
%a plea
call to action
towards the adoption of best practices aimed at fostering correctness during code development, prior to its release.}