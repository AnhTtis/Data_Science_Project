%In this paper, we introduced the concept of code correctness, an aspect of the evaluation process of scientific artifacts that has been largely neglected to date. 
%This paper introduces 
We 
\sara{discussed the importance}
%introduced the concept 
of code correctness, 
%\mn{[NON L'ABBIAMO INTRODOTTO NOI; NE ABBIAMO DISCUSSO L'IMPORTANZA IN AMBITO NLP.]}
a largely neglected aspect of the current evaluation of scientific artifacts in NLP. 
%We proved that 
Through a use case involving the \sara{widespread state-of-the-art} Conformer architecture, we empirically demonstrated 
%on
in
two tasks and in different language settings that
%, apart from not being considered during the peer review \mn{[Questo ``apart from not being considered during the peer review'' SA DI RIPETITIVO E POI AVEVAMO DETTO CHE E' DA DIRE CON CAUTELA PER EVITARE DI METTERE IL CSLP...]}, 
underestimating its importance when developing our research can lead to completely wrong findings 
%which can also be hidden 
hidden
by good results. For this reason, we 
%propose 
\mn{proposed}
a 
%\mn{[our last contribution is a?]} 
Correctness Checklist aimed at sensitizing authors on better code practices that 
%we hope the community will check and adopt when submitting their scientific artifacts.
we hope the community will adopt when submitting scientific artifacts and evaluated by peer review.




