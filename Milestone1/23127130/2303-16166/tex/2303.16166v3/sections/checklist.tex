In Section \ref{sec:core-idea}, we claimed the importance of the correctness of the code and we pointed out that this aspect is not currently evaluated in the peer-reviewing process 
%and that the correctness of a paper 
or
is only judged on the basis of the results shown in the paper. In Section \ref{sec:case-study}, we proved through extensive experiments that the results are not a valid measure of code correctness, invoking the need for ad-hoc guidelines aimed at improving this aspect.
%paying more attention to this important but neglected aspect.
%introduction of ad-hoc question(s) aimed at evaluating this important but neglected aspect.
%invoking ad-hoc guidelines aimed at improving this aspect and 
%encouraging the community to consider this aspect in the evaluation of a paper. 
In this section, we introduce the Correctness Checklist (Table \ref{tab:checklist}), a list of several recommendations that we encourage the authors to check before developing their codebases and declare when submitting their scientific artifacts. 

\begin{table}[!tb]
\begin{tcbitemize}[%
        raster columns=1,
        raster equal height,
        before=,after=\hfill,
        boxsep=3pt, left=10pt,   right=10pt,
        colframe=teal!75!black,colback=white,
        fonttitle=\large\bfseries,
        halign=left
        ]
\tcbitem[title=Correctness Checklist]
\begin{itemize}
\small
\justifying
\item[\checkinbox] The authors made use of Unit Tests to write their code (i.e., the codebase was isolated in individual units which were tested to show their correctness).
\item[\checkinbox] The codebase underwent a code reviewing process in which one or several people (who are not the author of the code) checked the program mainly by viewing and reading parts of its source code to ensure its correctness.
\item[\checkinbox] The documentation is sufficient to understand the code and follows documentation standards.
\item[\checkinbox] The repository lists all the dependencies needed to correctly install the software and run the experiments.
%\item[\checkinbox] More than one batch size was tested to check the consistency among the obtained results (i.e., the results do not vary when the batch size varies). 
\end{itemize}
%\raggedright\textbf{Browse All Courses in Programming}
\end{tcbitemize}
\caption{Correcntess Checklist.}
\label{tab:checklist}
\end{table}

One of the best practices we strongly suggest adopting to improve the code \textit{traceability} is the use of Unit Tests or UTs \citep{10.5555/1349795,8048665} to check every portion of it, if not first writing the tests and then the actual code, as per the popular test-driven development (TDD) practice \citep{beck2002driven}.
%For example, more than one batch size can be tested when developing a code of a new architecture to check the consistency among the obtained results since they do have not to vary when the batch size varies.
For example, testing multiple batch sizes when developing a new architecture code ensures consistency in its implementation as the results should not vary with batch size changes.
Although researchers may initially perceive that adding UTs is an additional and undesirable cost, which sums to all the other activities that have to be carried out, this feeling is exaggerated and the actual overhead is lower \citep{oro3667,Ellims2006}. \citet{Williams-2003-TDD} proved that writing UTs does not impact the productivity in writing code, as the initial overhead\footnote{Estimated in 16\%-35\% of the overall software development cost \citep{George-2003-tdd,google-ut}.} pays back by saving time spent in (unsuccessful) manual experiments \citep{google-ut}. In addition, since manual experiments often involve costly hardware with a significant environmental footprint in NLP \citep{strubell-etal-2019-energy}, UTs may reduce the computing and environmental costs of NLP research.
%
Another best practice that we promote when possible is the code-reviewing process \citep{7589787} to have external (with respect to the author(s) of the code) checks aimed not only at avoiding bugs but also at improving the code readability and documentation\footnote{Example of documentation standards can be found at \url{https://help.github.com/articles/about-readmes/}.} to make it more \textit{consistent} and  reusable \citep{Chen2019,Trisovic2022}. Along this line, we recommend listing all the dependencies of the released repository to allow a correct installation \citep{bestpractices} and improve code \textit{completeness}.

Ultimately, we hope that correctness will be an aspect to be sensitized by conferences and journals in the future by integrating this aspect into the review forms by explicitly asking the reviewers to answer/give a score/give a judgment on it but also by encouraging the authors to compile the Correctness Checklist during their submissions.


%On these aspects, plenty of literature already discussed the need to go beyond code openness \citep{Chen2019,Trisovic2022}, highlighting the role of proper documentation and testing in different environments. We reiterate the benefits brought by improving code readability and documentation of research repositories. While there is currently no incentive to do so \citep{barba-2019-praxis}, an ethical commitment toward more reusable code brings benefits to the whole community and increases  the reproducibility of the works. Similarly, they increase \textit{maintainability}, which measures the effort needed to make specific modifications to the code. This eases the code reuse by other researchers, as well as it eases building future works on the same codebase.


%Although the community showing interest in discussing the ways in which reported performance improvements on NLP benchmarks are meaningful, as demonstrated by the last years' theme tracks of conferences like EMNLP\footnote{\url{https://2022.emnlp.org/calls/main_conference_papers/\#emnlp-2022-theme-track}} and ACL\footnote{\url{https://2023.aclweb.org/calls/main_conference/\#theme-track-reality-check}}, little efforts have been devoted to push towards improved technical correctness of the published scientific artifacts.

%We hope that the Checklist will be also useful as a guideline for better software.

%Therefore, we suggest the authors to 
%and  the Correctness Checklist which we suggest being checked before submitting papers containing scientific artifacts. The guidelines are presented in Table \ref{tab:checklist}.



