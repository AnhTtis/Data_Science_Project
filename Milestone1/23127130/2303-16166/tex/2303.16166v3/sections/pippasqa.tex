%\mg{The attributes that determine the quality of software have been studied for many years \citep{mccall-1977-factors}, and defined in the ISO 9126 standard \citep{ISO9126}, which has been extended and superseded by ISO 25010 \citep{ISO/IEC2010} that added two characteristics (\textit{security} and \textit{compatibility}). Although their definition has targeted mostly production code, all of them have an impact also in code released to the research community: \textit{functionality}, \textit{reliability}, \textit{usability}, \textit{efficiency}, \textit{maintainability}, and \textit{portability}.}
%
%Functionality - "A set of attributes that bear on the existence of a set of functions and their specified properties. The functions are those that satisfy stated or implied needs."
%Reliability - "A set of attributes that bear on the capability of software to maintain its level of performance under stated conditions for a stated period of time."
%Usability - "A set of attributes that bear on the effort needed for use, and on the individual assessment of such use, by a stated or implied set of users."
%Efficiency - "A set of attributes that bear on the relationship between the level of performance of the software and the amount of resources used, under stated conditions."
%Maintainability - "A set of attributes that bear on the effort needed to make specified modifications."
%Portability - "A set of attributes that bear on the ability of software to be transferred from one environment to another."


\mg{The attributes that determine the quality of software have been studied for many years \citep{McCall1977FactorsIS}, and defined in the ISO 9126 standard \citep{ISO9126}, extended and superseded by ISO 25010 \citep{ISO/IEC2010}. Although their definition has targeted mostly production code, all of them have an impact also in code released to the research community. Below we analyze each of them with the implication it has on research:}

\mg{\textit{Functionality} is the first characteristic and involves the correctness of the code, i.e. its compliance with the specifications (i.e., what is stated in a paper that the code does) and accuracy in performing the operations. This is the attribute that enforces the validity of the findings of a paper: indeed, if it is not met, it means that the code does not perform what the authors stated.
Software functionality can be ensured only through tests that assert and enforce the correct behavior. As such, we promote the adoption of unit tests (UTs) in research software and encourage each researcher to consider tests when writing their code, if not first writing the tests and then the actual code, as per the popular test-driven development (TDD) practice \citep{beck2002driven}. Although tests should have the broadest coverage possible, the most important aspect is that the assumptions of a paper are tested and that ``given-for-granted'' elements, such as the invariance of the results by changing the batch size of a model, are guaranteed with UTs to enforce the correctness of the code. We invite researchers to state clearly what was tested, as no other assumption about the functioning of the code can be done. In addition, tests should be executed after every change, even seemingly unrelated ones, as suggested in the continuous integration (CI) practice \citep{Duvall-2007-ci}, otherwise their validity is lost.}


\mg{\textit{Portability} and \textit{usability}, instead, pertain to the replicability of a paper, as they respectively concern the ability to execute the same experiments in a different hardware/software environment and the effort needed by an individual to use the software, i.e. how easy is to execute the code. On these aspects, plenty of literature already discussed the need to go beyond code openness \citep{Chen2019,Trisovic2022}, highlighting the role of proper documentation and testing in different environments. We reiterate the benefits brought by improving code readability and documentation of research repositories. While there is currently no incentive to do so \citep{barba-2019-praxis}, an ethical commitment toward more reusable code brings benefits to the whole community and increases  the reproducibility of the works. Similarly, they increase \textit{maintainability}, which measures the effort needed to make specific modifications to the code. This eases the code reuse by other researchers, as well as it eases building future works on the same codebase.}

\mg{Lastly, \textit{reliability} defines the capability of the software to operate with adequate performance in many conditions and for long time. While research work cannot target this objective, it is a desirable property as it ensures that computing resources are not wasted because of unexpected failures. Likewise, \textit{efficiency} refers to the amount of resource used and is not the main focus of many research efforts. Although both these properties would contribute to reduce the environmental footprint of NLP research \citep{strubell-etal-2019-energy}, an appreciable ethical commitment in this direction constitutes a research direction on its own and can hardly be considered a pre-requisite for orthogonal research efforts.}


