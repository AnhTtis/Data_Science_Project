After showing that code correctness is currently 
%evaluated in peer reviews
assessed only on the basis of the results shown in the paper (\S\ref{sec:core-idea}) and that results are not a valid measure of code correctness (\S\ref{sec:case-study}),
in this section, we introduce a checklist of recommendations (Table \ref{tab:checklist})
%In addition, we would emphasize that the scope (and the name) of our \enquote{Correctness Checklist} should be interpreted analogously to that of the Reproducibility Checklists present in literature: its intent is to promote and improve code correctness not to certify or guarantee it.}
that we encourage researchers to adopt to 
%improve and enforce
improve the correctness of research codebases.

%the Correctness Checklist (Table \ref{tab:checklist}), a list of several recommendations that we encourage the authors to check before developing their codebases and declare when submitting their scientific artifacts. 



%First of all,
First,
as testing a piece of code is the only way to enforce that it works correctly,
% we argue that \mn{[SICCOME E' ABBASTANZA OVVIO, PIU' CHE ARGUE TROVEREI UN'ESPRESSIONE CHE INDICHI CHE CI RIFACCIAMO ALL'ABC DELLA PROGRAMMAZIONE. NON SO COME DIRE MEGLIO QUESTA COSA; SOLO CHE ``ARGUE'' MI DA' L'IDEA DI UNA GRAN SCOPERTA.]} the only way to enforce that a
%\sara{piece of code}
%\mg{code unit} 
%behaves correctly is by testing it. Specifically, 
we recommend that researchers write Unit Tests
%or UTs \citep{10.1145/987305.987309,10.1145/800027.808473,10.5555/1349795,8048665}
(UTs) to check that their code has the expected behavior \citep{10.1145/987305.987309,10.1145/800027.808473,10.5555/1349795,8048665}.
%,
%or even first write the tests and then the actual code, 
%possibly 
Possibly, this should be done even before writing the actual code as per the popular test-driven development 
%(TDD) 
practice~\citep{beck2002driven}.
%The UTs
UTs should cover all the assumptions about how the code works, e.g. the presence of padding does not alter the 
%result.
results.
In this case, a UT should enforce that the output is the same when the batch size changes.
%ensure that the results do not vary when batch size changes.
%
%
%
%
%One of the best practices we strongly suggest adopting to improve the code \textit{traceability} is the use of Unit Tests or UTs \citep{10.5555/1349795,8048665} to check every portion of it, if not first writing the tests and then the actual code, as per the popular test-driven development (TDD) practice \citep{beck2002driven}.
%For example, more than one batch size can be tested when developing a code of a new architecture to check the consistency among the obtained results since they do have not to vary when the batch size varies.
%For example, testing multiple batch sizes when developing a new architecture code ensures consistency in its implementation as the results should not vary with batch size changes.
%
%
%
%
% Although researchers may initially perceive that adding UTs is an additional and undesirable cost, which sums to all the other activities, this feeling is exaggerated and the actual overhead is lower \citep{oro3667,Ellims2006}. \citet{Williams-2003-TDD} proved that writing UTs does not impact the productivity in writing code, as the initial overhead\footnote{Estimated in 16\%-35\% of the overall software development cost \citep{George-2003-tdd,google-ut}.} pays back by saving time spent in (unsuccessful) manual experiments \citep{google-ut}. 
%
Although researchers may initially 
%perceive
view
%\mg{consider}
%that adding UTs is
writing UTs as
an excessive and undesirable cost added to all the other activities,
%this feeling should be downsized and put into perspective.
%\mg{\citet{oro3667,Ellims2006} show that this feeling is exaggerated.}
\citet{Williams-2003-TDD} proved that 
%%writing
adding UTs
%\sara{it}
does not impact the productivity in writing code and \citet{oro3667,Ellims2006} showed that 
%this feeling is exaggerated.
the perceived cost is \enquote{\textit{exaggerated}}.
%\sara{the perceived costs of UTs are \enquote{\textit{exaggerated}}.}
%, as
Indeed,
the initial overhead\footnote{Estimated in 16\%-35\% of the overall software development cost \citep{George-2003-tdd,google-ut}.} pays back by saving time spent in (unsuccessful) manual experiments \citep{google-ut}. 
%Other studies \citep{oro3667,Ellims2006} conclude that the actual overhead is even lower.
%In addition,
Moreover, since 
%manual 
%NLP 
experiments often involve costly hardware with a significant environmental footprint
%in NLP
\citep{strubell-etal-2019-energy}, UTs may reduce the computing and environmental costs of NLP research.
%
%

As a complete test coverage of the code is utopian, we also encourage researchers 
%%%%to state in the \enquote{Limitations} section which assumptions they have on the code behavior that were not tested.
%\mn{to state in the \enquote{Limitations} section which assumptions they have made regarding the code behavior that were not tested.}
to include in the \enquote{Limitations} section any assumptions they have made about the code behavior that have not been tested.
In addition, it is important to notice that UTs should be executed every time 
%a modification is introduced in the codebase, even if it 
%%is seemingly
%\mg{seems} an
the codebase is modified, even in case of a seemingly
\textit{unrelated change}.
% a new change is introduced in the codebase, even though the change is apparently unrelated. \mn{[RISPETTO A COSA? INTENDEVATE INNOCUO - harmless?]}. 
Indeed, their validity ends when a new version of the code is used. This is commonly enforced through continuous integration (CI), which executes all UTs at every code change~\citep{Duvall-2007-ci}. 
%The presence of a (running and successful)
A running and successful CI has the additional benefit of enforcing that 
% the open-source version of the code released by the authors works as they expect
any open-source release of the code works as the authors expect
and provides implicit guidance on how to install and successfully run the code %to
for
people trying to reproduce a study.
%paper.


\begin{table}[!tb]
\setcounter{table}{7}
\begin{tcbitemize}[%
        raster columns=1,
        raster equal height,
        raster width=.47\textwidth,
        before=,after=\hfill,
        boxsep=3pt, left=6pt, right=6pt,
        colframe=teal!75!black,colback=white,
        fonttitle=\large\bfseries,
        halign=left,
        ]
\tcbitem
\begin{itemize}
\small
\justifying
\item[\checkinbox] All the assumptions about code behavior have been tested with Unit Tests (UTs).
\item[\checkinbox] All the assumptions that were not tested with UTs have been included in the \enquote{Limitations} section.
\item[\checkinbox] UTs have been executed on the last version of the code, which has been used for the experiments and has been released.
\item[\checkinbox] Every contribution to the codebase has been reviewed (and approved) by one or more people.
\item[\checkinbox] %(Preferable) 
The repository contains a continuous integration 
%(or CI) 
tool
%CI \mn{[ESPANDERE NEL CASO UNO, COME ME, SI FOSSE PERSO CHE CI = CONTINUOUS INTEGRATION?]}, 
which installs the software and runs the UTs (this also helps reproducibility as it forces listing all dependencies and showing how to install the code).
%
%lists all the dependencies needed to correctly install the software and run the experiments.
%\item[\checkinbox] More than one batch size was tested to check the consistency among the obtained results (i.e., the results do not vary when the batch size varies). 
\end{itemize}
%\raggedright\textbf{Browse All Courses in Programming}
\end{tcbitemize}
\caption{
Correctness
%\mg{Coding best-practice}
Checklist.}
\label{tab:checklist}
\end{table}


%At last,
Lastly,
we promote the adoption of a code reviewing practice \citep{7589787}, 
%which requires that 
in which
all changes are reviewed and approved by a person different from the 
%author of the code.
code author.\footnote{The reviewer(s) can be any person who is not the author of the code with basic knowledge of the codebase, such as lab teammates or advisors.} 
This 
%practice aims
aims
not only at avoiding bugs but also at improving 
%the 
code 
%readability and 
readability,
documentation,\footnote{Example of documentation standards can be found at \url{https://help.github.com/articles/about-readmes/}.} 
% which improve its reusability \citep{Chen2019,Bahaidarah-2022-toward,Trisovic2022}, and hence reproducibility.
reusability \citep{Chen2019,Bahaidarah-2022-toward,Trisovic2022} and, in turn,
%results'
the
reproducibility of results.

%The reviewer(s) can be any person who is not the author of the code with basic knowledge of the codebase, such as lab teammates or advisors.







%Along this line, we recommend listing all the dependencies of the released repository to allow a correct installation \citep{bestpractices} and improve code \textit{completeness}.

% FOOTNOTE
%The name of our \enquote{Correctness Checklist} should be interpreted analogously to that of the \enquote{Reproducibility Checklist} required by many conferences: its intent is promoting and improving code correctness rather than ensuring it.


% OLD CONCLUSION
%To conclude, we hope that the NLP research community will devote more attention to coding best practices and to the role of code correctness in research. As a first step, we propose the adoption of a checklist similar to the one we propose here, as it was done for reproducibility. Moreover, we strongly encourage considering coding practices and efforts toward code correctness as an important contribution while evaluating a paper.

In conclusion, this study 
%seeks to raise awareness among NLP researchers 
\sara{aims to increase researchers' awareness}
regarding the importance of coding best practices and the role of code correctness. 
%As an initial step, we recommend the adoption of the above \enquote{Correctness Checklist}, which is intended to promote and enhance code correctness rather than to certify it (which would be utopian), similar to the \enquote{Reproducibility Checklist} that many top-tier venues require. 
As a first step, we 
%propose
\mn{encourage}
the adoption of 
%a
\mn{the above}
\enquote{Correctness Checklist}, whose name 
should be interpreted 
%analogously
\mn{in analogy}
to that of the \enquote{Reproducibility Checklist} \mn{now} required by many 
%conferences
\mn{top-tier venues}: its intent is %promoting and improving 
\mn{fostering}
code correctness rather than 
%ensuring it, which would be utopian.
\mn{certifying it.}
Furthermore, to address the current lack of incentives for code quality, we urge the NLP community to consider it an important contribution while evaluating new works.


% \mg{To conclude, this study aims at sensitizing \mn{NLP} researchers to devote more attention to coding best practices and to the role of code correctness in research.
% As a first step, we 
% %propose
% \mn{encourage}
% the adoption of 
% %a
% \mn{the above}
% \enquote{Correctness Checklist}, whose name should be interpreted 
% %analogously
% \mn{in analogy}
% to that of the \enquote{Reproducibility Checklist} \mn{now} required by many 
% %conferences
% \mn{top-tier venues}: its intent is %promoting and improving 
% \mn{fostering}
% code correctness rather than 
% %ensuring it, which would be utopian.
% \mn{certifying it.}
% In addition, as a countermeasure to the current lack of incentives toward code quality, we encourage considering it as an important contribution while evaluating a work
% \mn{research contribution}.}

%In conclusion, the importance of code correctness cannot be overstated in NLP research. As we have demonstrated in this paper, code bugs can have serious consequences on the validity of research findings. Therefore, we call upon the NLP community to give more attention to coding best practices and to consider code correctness as a crucial component of their research. Similar to the Reproducibility Checklist that has been widely adopted, we propose the adoption of a \enquote{Correctness Checklist} as a first step towards this goal. This checklist can serve as a tool to promote and improve code correctness in NLP research, and we strongly encourage the evaluation of papers to take into account coding practices and efforts towards code correctness. By working towards a culture of code correctness in NLP research, we can ensure that our findings are not only reproducible, but also accurate and trustworthy.

%this 
%
%aspect into the review forms by explicitly asking the reviewers to answer/give a score/give a judgment on it but also by encouraging the authors to compile the Correctness Checklist during their submissions.


%On these aspects, plenty of literature already discussed the need to go beyond code openness \citep{Chen2019,Trisovic2022}, highlighting the role of proper documentation and testing in different environments. We reiterate the benefits brought by improving code readability and documentation of research repositories. While there is currently no incentive to do so \citep{barba-2019-praxis}, an ethical commitment toward more reusable code brings benefits to the whole community and increases  the reproducibility of the works. Similarly, they increase \textit{maintainability}, which measures the effort needed to make specific modifications to the code. This eases the code reuse by other researchers, as well as it eases building future works on the same codebase.


%Although the community showing interest in discussing the ways in which reported performance improvements on NLP benchmarks are meaningful, as demonstrated by the last years' theme tracks of conferences like EMNLP\footnote{\url{https://2022.emnlp.org/calls/main_conference_papers/\#emnlp-2022-theme-track}} and ACL\footnote{\url{https://2023.aclweb.org/calls/main_conference/\#theme-track-reality-check}}, little efforts have been devoted to push towards improved technical correctness of the published scientific artifacts.

%We hope that the Checklist will be also useful as a guideline for better software.

%Therefore, we suggest the authors to 
%and  the Correctness Checklist which we suggest being checked before submitting papers containing scientific artifacts. The guidelines are presented in Table \ref{tab:checklist}.



