Taxonomy is formulated as directed acyclic graphs or trees, which consist of \emph{hyponym-hypernym} relations between concepts.
One concept is a hypernym of another concept if the meaning of the former covers the latter\cite{sang2007extracting}.
A well-constructed taxonomy can assist various downstream tasks, including web content tagging \cite{liu2020giant,liu2019tencent_taxonomy,peng2019RCNN4taxonomy}, personalized recommendation \cite{karamanolakis2020txtract, huang2019taxonomy}, query understanding  \cite{yang2020co} and so on. 

Manually constructing and maintaining a taxonomy is labor-intensive, expensive, and time-consuming.
So the task of automated taxonomy expansion is hence proposed \cite{jurgens2016semeval}, which aims to automatically assign an existing concept (anchor concept) as a hypernym concept to the newly input concept (query concept).
The mainstream taxonomy expansion methods tend to use the self-supervised method to generate data from the existing taxonomy \cite{shen2020taxoexpan, yu2020steam, wang2021enquire, zhang2021taxonomy, mao2020octet, ma2021hyperexpan, wang2022qen}.
The model will learn from the existing hypernym-hyponym relation, and  to assign anchor concepts to the new query concepts.
These models achieve great success in saving a lot of time from automated taxonomy construction and maintenance.

However, we argue that the assumption made by the previous taxonomy expansion researches that \emph{the query concepts are most clean} is not true in the real-scenarios.
We observe that the number of query concepts in real-scenarios are much larger than the number in the experiments of previous researches, and \emph{most query concepts are noisy}, which means they shouldn't be added to the existing taxonomy.
% We observe that huge amount of noisy query concepts are expected to be added to the existing taxonomy in the industry application, which make the amount of query concepts much larger than the experiment setting in the previous researches.
Tens of thousands of concepts are involved in the experiments of previous researches\cite{ zhang2021taxonomy, ma2021hyperexpan, wang2022qen}, and there are millions of new concepts expected to be added in Taobao taxonomy, which is the largest Chinese shopping platform, in a single time \cite{jiang2019towards}, and each year there will be billions of new concepts generated in Meituan online application \cite{cheng2022learning}, which is a leading Chinese e-commerce platform for ordering take-outs, and only thousands of them are finally added in the existing taxonomy.

The large amount of noisy concepts will greatly slow the efficiency and effectiveness of the previous methods.
These methods rely heavily on good concept representations, which can judge the hypernym-hyponym relationships of concepts through the similarity between representations.
Even with only tens of thousands of concepts in their experiments, the previous methods also spent a lot of time training the model to get a representation for each concept that satisfies the above condition\cite{yu2020steam, zhang2021taxonomy, wang2021enquire}.
The time expense of such progress will be unacceptable when the data scale increases from tens of thousands to billions.
We summarize the challenges of facing such large amount of noisy concepts and list as follows:
\begin{itemize}
    \item Large amount of concepts will consume huge amount of time in learning representation for every concept
    \item Even with a good representation, a large number of noisy query concepts is still a huge challenge to current taxonomy expansion models which do not even consider the problem of noisy concepts.
\end{itemize}


In this paper, we propose a plugin task called taxonomy enterance evaluation.
The taxonomy enterance evaluation aims at scoring the query concept to indicate whether this concept should be added to the existing taxonomy in an efficient way, the noisy concept is expected to get a low score, while the real concept is expected to be scored high.
The input will meet the assumption of taxonomy expansion task when low-score concepts are all excluded, the efficiency and effectiveness of taxonomy expansion will hence increase greatly.

We model the taxonomy enterance evaluation as a binary classification task, and obtain the representations of concepts by pretrained language models (PLMs).
With the rapid development of PLM, it has achieved a great success in the field of text understanding \cite{devlin2018bert, yang2019xlnet}.
With representation of every word is pretrained in PLM, it will quickly output the representation of whole text.
In order to efficiently get the representation of the concept, many works use language models to quickly obtain the representation through the textual information of the concept \cite{cheng2022learning, takeoka2021low}.
Although the language model has a limited ability in representing conceptual text, classifer will have strong discriminative ability with a large amount of high quality training data.
To ensure the quality of the training data, large amount of deceptive negative data and fedility positive data are needed, and simply sampling from existing Taxonomy is not enough.


Here, together with the taxonomy enterance evaluation task, we propose a corresponding method called \textbf{G}enerative \textbf{A}dversarial \textbf{N}etwork for \textbf{T}axonomy \textbf{E}nterance \textbf{E}valuation (GANTEE).
The discriminative model $\mathcal{D}$ in Generative Adversarial Network learns to give supervised signals of telling the generated text apart from the text in real query concept, and the generative model $\mathcal{G}$ learns to confuse $\mathcal{D}$ by generating text which has a deceptively ``hyponym'' relation to the given anchor concept.
The generated texts are expected to be a fake query concept which is a hyponym to the given anchor concept, but a question proposed by the previous research\cite{guo2018leakgan} that when the discriminator is too strict, the effective gradient won't be calculated, so we use a loose rollout discriminator and a strict hyper discriminator to provide supervised signal during the generating process.
Rollout discriminator is proposed to alleviate the problem of gradient passing back during the rollout process \cite{yu2017seqgan}, which is a process proposed to get the reward of unfinished text.
A hyper discriminator is used to detect whether the generated text has a ``hyponym'' relation to the given anchor concept after the generating process.

In the experiments, we benchmark the taxonomy entrance evaluation task on three real-world taxonomies in two different languages, English and Chinese.
We make new State-Of-Art score in the experiments by improve the hit@1 from \red{xxx} to \red{xxx}, MR from \red{xxx} to \red{xxx} and MRR from \red{xxx} to \red{xxx}.
Then we test the efficiency of our method, and GANTEE still outperforms the baselines by a large margin on all three datasets.
Finally, extensive experiments have been conducted to study how the parameters affect the result of GANTEE, and provide an effective tuning scheme for the following researchers who will use our method in the future.

\subsubsection{Contribution.}
To summarize, our major contributions include: 
(1): a more realistic task called taxonomy entrance evaluation which judge whether a query concept should be added to an existing taxonomy in an efficient way;
(2): a novel, effective method called GANTEE to generate confusing negative and fidelity positive data which boosts the performance and the efficiency for both traditional taxonomy expansion tasks and taxonomy enterance evaluation tasks.
(3): extensive experiments that verify the effectiveness of GANTEE on three datasets in two languages, and an effective tuning method was proposed by the experiment.