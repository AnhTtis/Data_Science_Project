

\subsection{Taxonomy Expansion}
In taxonomy expansion task, an existing taxonomy such as WordNet\cite{miller1995wordnet} or Probase\cite{wu2012probase,chen2019cnprobase} is provided as both the guidance and the base for adding new terms. 
Aly \emph{et al.}\cite{aly2019every} adopt hyperbolic embedding to capture hierarchical lexical-semantic relations.
Fauceglia \emph{et al.}\cite{fauceglia2019automatic} use a hybrid method to combine linguistic patterns, semantic web, and neural network for taxonomy expansion.
Manzoor \emph{et al.}\cite{manzoor2020expanding} model the implicit edge semantics to score the hyponymy relevance between node pairs. 
However, the above methods fail to exploit the structural information of the existing taxonomy.
To better maintain the structure of the existing taxonomy, Shen \emph{et al.}\cite{shen2020taxoexpan} propose position-enhanced graph neural networks to encode the relative position of terms and improve the overall quality of taxonomy.
Song \emph{et al.}\cite{song2021should} design a concept sorting model to extract hyponymy relations and sort their insertion order by utilizing the relationship between the newly mined concepts.
Wang \emph{et al.}\cite{wang2021enquire} utilize the hierarchical information of the existing taxonomy by extracting tree-exclusive features in the taxonomy for better taxonomy coherence. 
One limitation of these approaches is that they mainly focus on the general-purpose taxonomies or utilize the general text corpora. Thus, they cannot be easily generalized to specific taxonomies.
Mao \emph{et al.}\cite{mao2020octet} leverage heterogeneous sources of signals such as lexical semantics and structural information to train an end-to-end online catalog taxonomy enrichment model.
The structural information is captured by modeling the structure of the core taxonomy and the query-item-taxonomy interaction in user behavior. However, it has to work with extra item-category information and cannot represent the semantics of new concepts.
Compared with these methods, our proposed method designs relational and structural representations learned from user-generated content and user click logs to model the semantics of in-domain concepts.

Study like \cite{cheng2022learning} have done the candidate query filtering, but their filtering method is mainly based on user generated content.
In our work, we formulate the task of taxonomy entering, and use generative model to self-supervised generate graph-level negative samples, which boost both the performance of traditional taxonomy expansion and the taxonomy entering.

\subsection{Text GAN}
The GANs in discrete domains encounter a problem in propagating the gradient into the generator.
SeqGAN\cite{yu2017seqgan} has overcome this problem by using a REINFORCE-like\cite{williams1992simple} algorithm in the training of the generator.
It assumes the generator as an agent that receives more reward from the discriminator to generate more realistic sentences and uses the Monte Carlo tree search to estimate the expected reward.
RankGAN\cite{juefei2018rankgan} proposes a ranking model to replace the original binary classifier as the discriminator.
LeakGAN\cite{guo2018leakgan} designs a mechanism to provide intermediate information about text generation for generator, where the discriminator can leak its features through a manager module.
MaskGAN\cite{lee2020maskgan} introduces an actor-critic conditional GAN that fills in missing text conditioned on the surrounding context by resorting to a seq2seq model.
FM-GAN\cite{amodio2021fmgan} proposes to match the latent feature distributions of real and synthetic sentences using the feature-movers distance.
REL-GAN\cite{nie2018relgan} incorporates a relational memory as a new component for modeling the long-distance dependency.
Furthermore, its discriminator utilizes multiple representations to preapre a more informative signal for the generator.

Furthermore, apart from the RL-based algorithms, very recent works, such as AutoGAN-distiller\cite{fu2020autogan}, AdversarialNAS\cite{gao2020adversarialnas}, DEGAS\cite{doveh2021degas} and alphaGAN\cite{tian2020alphagan}, employ the gradient-based algorithm to search the GAN architecture. 
It is noteworthy that gradient-based frameworks could better search generator and discriminator by playing a two-player min-max game at the same time to achieve good results. 
Kobayashi \emph{et al.}\cite{shi2022multi} adopted the evolutionary algorithm to search GAN and also achieved better results than the manual design. 
Zhou \emph{et al.}\cite{zhou2020searching} successfully applied gradient-based GANAS algorithm to the conditional GAN search.

