Taxonomy is formulated as directed acyclic graphs or trees, which consist of \emph{hyponym-hypernym} relations between concepts.
One concept is a hypernym of another concept if the meaning of the former covers the latter \cite{sang2007extracting}.
A well-constructed taxonomy assist various downstream tasks, including web content tagging \cite{liu2020giant,liu2019tencent_taxonomy,peng2019RCNN4taxonomy}, personalized recommendation \cite{karamanolakis2020txtract, huang2019taxonomy}, query understanding  \cite{yang2020co} and so on. 

\begin{figure*}
    \centering
    \includegraphics[width=0.8\linewidth]{intro.png}
    \caption{An example of the GANTEE framework. This framework is implemented as a plugin before taxonomy expansion. The demand of generating high-quality data and taxonomy enterance evaluation are formulated as a generative adversarial network in GANTEE.}
    \label{fig:introduction}
    \vspace{-5mm}
\end{figure*}

Manually maintaining a taxonomy is labor-intensive and time-consuming.
For example, millions of new concepts are expected to be added to the taxonomy of one of the largest shopping platforms in the world each month \cite{jiang2019towards}.
So the task of automated taxonomy expansion is proposed \cite{jurgens2016semeval}, which aims to automatically assign an existing concept (anchor concept) as a hypernym concept to the newly input concept (query concept).
These models achieve great success in saving a lot of time from labor annotation in taxonomy maintenance.
%The prevailing taxonomy expansion methods assume that most query concepts are clean.
These methods predict the hypernym-hyponym relationships mainly based on the representation similarity, so they sample the training data based on the existing taxonomy to learn good representations for each concept. 

However, we argue that these methods cannot be applied to real applications due to two drawbacks. 
\textbf{First}, these methods suffer from low effectiveness in real applications due to limited ability to represent semantics of concepts. Only sampling from existing taxonomy provides limited data to train representations for zero-shot concepts, where most of the query concepts are zero-shot concepts.
%there is a few-shot problem in real scenarios that only sampling from the existing taxonomy does not provide enough diverse data to train the model which can mine the hypernym-hyponym relationships in real scenarios.
Previous work shows that xx~\cite{cheng2022learning}.
%Since the data in real scenarios is usually thousands of times the size of the concept in the existing taxonomy, only sampling from the existing taxonomy will greatly harm the effectiveness of the previous taxonomy expansion methods in the real scenarios.
\textbf{Second}, these methods suffer from low efficiency in real applications because most of the query concepts are noisy. %Previous methods assume that the query concepts are clean.
For instance, in a leading ordering take-outs online platform, only thousands of concepts out of billions of queries are finally added to their taxonomy\cite{cheng2022learning}.
Previous methods ignore the noisy concepts and waste a lot of time to find the anchor concept for noisy query concepts.
%A large number of noisy concepts will result in a huge time expense in learning a good representation during the predicting stage in taxonomy expansion.


In this paper, we propose a plugin framework called \textbf{G}enerative \textbf{A}dversarial \textbf{N}etwork for \textbf{T}axonomy \textbf{E}nterance \textbf{E}valuation (GANTEE) to boost both the effectiveness and efficiency of the taxonomy expansion methods in real-scenarios as is shown in Figure \ref{fig:introduction}.
To improve the effectiveness, GANTEE proposes a generative model to generate more high-quality training data, which are required to train the high-quality representation for taxonomy expansion. 
To improve the efficiency, GANTEE introduces a new task called taxonomy entrance evaluation, which removes the large amount of noisy query concepts before finding the suitable anchor concept for them.
Intuitively, the generative model and the taxonomy entrance evaluation model are two adversarial models which can improve each other. The generative model produces more fake samples which are similar to the real samples to confuse the discriminative model, and the taxonomy entrance evaluation model learns to generate high quality samples. Therefore, these two tasks can be formulated as a generative adversarial task\cite{goodfellow2014generative} to further improve their performance.

Specifically, in order to improve efficiency, the taxonomy entrance evaluation model is expected to be more light-weight than taxonomy expansion model. We introduce two mechanisms to ensure the efficiency of taxonomy entrance evaluation model. First, instead of determining where to extend the query concept, GANTEE introduce two easier tasks xx and xx to determine whether the query concept is a concept and whether the query concept should be added to the taxonomy.
Second, instead of training a new representation for emerging query concepts, we use pre-trained language models (PLMs) to directly obtain a representation of each concept based on the textual features of the concept. Although PLMs have limited ability in representing the fine-grained semantics based on conceptual text\cite{lauscher2019specializing}, the experimental results verify that PLMs are suitable to generate representation for xx and xx.



%This approach is a trade-off between efficiency and effectiveness.
%PLMs can obtain a representation from the textual features of concepts in a very short time\cite{devlin2018bert}, the reason why previous works do not use PLMs is that PLMs have limited ability in representing the fine-grained semantics based on conceptual text\cite{lauscher2019specializing}.
%We compensate for the lack of PLMs representation capability by reducing the denoising problem into a coarse-grained evaluation problem and generating a large amount of high-quality training data by the generative model.

In the experiments, we benchmark the taxonomy entrance evaluation task on three real-world taxonomies in two different languages, English and Chinese.
We make new State-Of-Art score in the experiments by improve the hit@1 from \red{xxx} to \red{xxx}, MR from \red{xxx} to \red{xxx} and MRR from \red{xxx} to \red{xxx}.
Then we test the efficiency of our method, and GANTEE still outperforms the baselines by a large margin on all three datasets.
Finally, extensive experiments have been conducted to study how the parameters affect the result of GANTEE, and provide an effective tuning scheme for the following researchers who will use our method in the future.

\subsubsection{Contribution.}
To summarize, our major contributions include: 
(1) a more realistic task called taxonomy entrance evaluation which judge whether a query concept should be added to an existing taxonomy in an efficient way;
(2) a novel, effective method called GANTEE to generate confusing negative and fidelity positive data which boosts the performance and the efficiency for both traditional taxonomy expansion tasks and taxonomy enterance evaluation tasks.
(3) extensive experiments that verify the effectiveness of GANTEE on three datasets in two languages, and an effective tuning method was proposed by the experiment.