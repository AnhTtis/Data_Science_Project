In this section, we first give the definitions of the preliminary concepts used in this paper, then formally define the taxonomy enterance evaluation task and traditional taxonomy expansion problems.

\subsection{Preliminaries} \label{sec:define}

\begin{myDef}
\textbf{(Taxonomy and Anchor Concept.)}
A taxonomy $\mathcal{T}=(\mathcal{C}_{\mathcal{A}}, \mathcal{E})$ is a directed acyclic graph.
Each node $c_q \in \mathcal{C}_{\mathcal{A}}$ denotes an anchor concept and each directed edge $\langle n_{q}^{p}, n_{q}^{c} \rangle \in \mathcal{E}$ represents the hyponymy relation between the parent (hypernym) concept $n^c_p$ and the child (hyponym) concept $n^c_q$.
\end{myDef}

\begin{myDef}
\textbf{(Query Concept.)}
Query concept is the concept prepared to be added into the existing taxonomy.
The set of query concepts is represented as $\mathcal{C}_{\mathcal{Q}} = \{c^1_q, c^2_q, \cdots, c^m_q\}$.
While the previous studies assume the query concepts can all be added into the existing taxonomy, we argue that would not be true in the real scenario, and we will discuss the detail of it in the Problem \ref{taxonomy enterance evaluation}.
\end{myDef}


\subsection{Problem Definition}


\begin{myproblem}
\label{taxonomy enterance evaluation}
\textbf{(Taxonomy Enterance Evaluation.)}
Given an existing taxonomy $\mathcal{T}^{0}=(\mathcal{C}^{0}_{\mathcal{A}}, \mathcal{E}^{0})$ and the set of query concepts $\mathcal{C}_{\mathcal{Q}}$,
the goal of taxonomy enterance evaluation task is to score a query concept $c_q \in \mathcal{C}_{\mathcal{Q}}$ of the existing taxonomy $\mathcal{T}^{0}$, and the score bigger than the threhold $\gamma$ are considered as the concepts should be added into the taxonomy and are taken into the following steps.
\end{myproblem}
Formally, we consider each query concept $c_q \in \mathcal{C}_{\mathcal{Q}}$ and the taxonomy $\mathcal{T}^{0}$ as two random variables.
The task of taxonomy enterance evaluation can be formulated as follows:

% \red{
% \begin{equation}
% \small
% \begin{aligned}
% \mathcal{C}^{*} &=\underset{\mathcal{C}}{\arg \max } \mathbf{P}(C\mid\mathcal{T}^{0} ; \Theta_1)\\
% &=\underset{\mathcal{C}}{\arg \max } \sum_{i=1}^{\left|\mathcal{C}\right|} \log \mathbf{P}\left(c_{i} \mid \mathbf{T}^{0}; \Theta_1\right).
% \end{aligned}
% \end{equation}
% }
\begin{equation}
    \mathcal{C}_\mathcal{Q}^*=\{c_q^i\}\ \forall c_q^i \ s.t.\ p(c_q^i|\mathcal{T}^{0};\Theta_1)>\gamma
\end{equation}
where $\Theta_1$ denotes model parameters, and the $\mathcal{C}_\mathcal{Q}^*$ are the clean query concepts we have found.
% \red{
% We then find the optimal query concepts set $\mathcal{C}^{*}$ larger than a minimized size to maximize the likelihood.
% }

\begin{myproblem}
\textbf{(Taxonomy Expansion.)}
Given an existing taxonomy $\mathcal{T}^{0}=(\mathcal{C}^{0}_{\mathcal{A}}, \mathcal{E}^{0})$, the set of clean query concepts $\mathcal{C}_\mathcal{Q}^*$, the goal of taxonomy expansion task is to attach every query concept $c_q \in \mathcal{C}_\mathcal{Q}^*$ to the existing taxonomy $\mathcal{T}^{0}$ and expand it to obtain an enriched taxonomy $\mathcal{T}^{*}=(\mathcal{C}^{0}_{\mathcal{A}} \cup  \mathcal{C}_\mathcal{Q}^*, \mathcal{E}^{0} \cup \mathcal{R})$, where $\mathcal{R}$ is a set of new hyponymy relations $\langle c_a, c_q \rangle$, $c_a \in \mathcal{C}^{0}_{\mathcal{A}}$, $c_q \in \mathcal{C}_\mathcal{Q}^*$.
\end{myproblem}

Formally, we consider each node $c_q^{i} \in \mathcal{C}^{0}_\mathcal{Q}$ as a random variable and the taxonomy $\mathcal{T}^{0}$ as a Bayesian network.
The probability of taxonomy $\mathcal{T}^{0}$ can be formulated as follows.
\begin{equation}
\small
\mathbf{P}(\mathcal{T} \mid \Theta)=\mathbf{P}(\mathcal{C}_\mathcal{Q} \mid \mathcal{T}^{0}; \Theta)=\prod_{i=1}^{|\mathcal{C}_\mathcal{Q}|} \mathbf{P}\left(c^{i}_q \mid p\left(c^{i}_q\right); \Theta\right).
\end{equation}
where $\Theta$ denotes model parameters, and $p\left(c_q^{i}\right)$ represents the parent concept(s) of $c_q^{i}$.
We then find the optimal taxonomy $\mathcal{T}^{*}$ by maximizing the likelihood.
\begin{equation}
\small
\begin{aligned}
\mathcal{T}^{*} &=\underset{\mathcal{T}}{\arg \max } \mathbf{P}(\mathcal{T} \mid \Theta) \\
&=\underset{\mathcal{T}}{\arg \max } \sum_{i=1}^{\left|\mathcal{C}^{0}_{\mathcal{A}} \cup  \mathcal{C}_\mathcal{Q}^*\right|} \log \mathbf{P}\left(c_{i} \mid p\left(c_{i}\right); \Theta\right).
\end{aligned}
\end{equation}

However, for every node in the existing taxonomy, all query concepts in $\mathcal{C}^*$ can be taken as candidates of hyponym, thus the potential search space is incredibly enormous.
We follow the previous studies \cite{manzoor2020expanding, shen2020taxoexpan, yu2020steam, zhang2021taxonomy, mao2020octet} address the above problem via an assumption that the input set of new concepts contains only one element (i.e., $|\mathcal{C}^*_\mathcal{Q}|=1$), and aims to find one single parent node in the existing taxonomy $\mathcal{T}^{0}$ of this new concept (i.e., $|\mathcal{R}|=1$).
As a result, we divide the above computationally intractable problem into the following set of $|C^*|$ tractable optimization problems:
$$c_a^{i*}=\underset{c_a^i\in \mathcal{C}^0_\mathcal{A}}{\arg\max}logP(c_q^j|c_a^i,\Theta),\space \forall j\text{ or }i\in \{1,2,\dots,|C_\mathcal{Q}^*|\},$$
where $c_a^{i*}$ is the parent node of a new concept $c_q^i\in \mathcal{C}_\mathcal{Q}$.



