Taxonomy is formulated as directed acyclic graphs or trees, which consist of \emph{hyponym-hypernym} relations between concepts.
One concept is a hypernym of another concept if the meaning of the former covers the latter \cite{sang2007extracting}.
A well-constructed taxonomy can assist various downstream tasks, including web content tagging \cite{liu2020giant,liu2019tencent_taxonomy,peng2019RCNN4taxonomy}, personalized recommendation \cite{karamanolakis2020txtract, huang2019taxonomy}, query understanding  \cite{yang2020co} and so on. 

Manually constructing and maintaining a taxonomy is labor-intensive and time-consuming.
For example, there are millions of new concepts expected to be added in Taobao taxonomy, which is the largest Chinese shopping platform, in a single time \cite{jiang2019towards}.
So the task of automated taxonomy expansion is proposed \cite{jurgens2016semeval}, which aims to automatically assign an existing concept (anchor concept) as a hypernym concept to the newly input concept (query concept).
The mainstream taxonomy expansion methods assume that most query concepts are clean.
They tend to learn from the hypernym-hyponym relation in the existing taxonomy, and assign anchor concepts to all query concepts \cite{shen2020taxoexpan, yu2020steam, wang2021enquire, zhang2021taxonomy, mao2020octet, ma2021hyperexpan, wang2022qen}.
These models achieve great success in saving a lot of time from labor annotation in taxonomy construction and maintenance.

However, we argue that there are two main differences between the previous studies and real-scenarios. 
\textbf{First}, only tens of thousands of query concepts were covered in the experiments of previous taxonomy expansion studies, which is far from the real-scenarios.
\textbf{Second}, the assumption made by the previous studies that most query concepts are clean is not true in the real-scenarios.
For instance, there have billions of new query concepts generated in Meituan online application \cite{cheng2022learning}, which is a leading Chinese e-commerce platform for ordering take-outs, and only thousands of them are clean and being added to the existing taxonomy.
Even with only tens of thousands of concepts in their experiments, it still take the previous methods a lot of time during the training and the predicting stage \cite{yu2020steam, zhang2021taxonomy, wang2021enquire}.
Because most of the previous methods rely heavily on concept representation techniques, these techniques use vector to represent conceptual features instead of text.
The representations of the concepts are expected to denote the hypernym-hyponym relationships based on the vectorized similarity.
And the time expense of learning a good representation will be unacceptable when the data scale increases from tens of thousands to billions during the predicting stage in real-scenarios.
One step more, even with a good representation, a large number of noisy query concepts is still a huge challenge to current taxonomy expansion models which do not even consider the problem of noisy concepts.

In this paper, we propose a plugin task called taxonomy enterance evaluation to alleviate the above problems, and a corresponding method called \textbf{G}enerative \textbf{A}dversarial \textbf{N}etwork for \textbf{T}axonomy \textbf{E}nterance \textbf{E}valuation (GANTEE).
From the case of Meituan, we find that most of query concepts are noisy in real-scenarios.
So the task of taxonomy enterance evaluation aims at excluding noisy query concepts before taxonomy expansion, leaving clean query concepts to increase the efficiency of taxonomy expansion.
Intuitively, taxonomy enterance evaluation task can be modeled as a binary classification task, but such modeling brings two challenges.
\begin{itemize}
    \item Large amount of high quality positive and negative data are needed to do the training.
    \item The efficiency of this task should be high enough to reduce the time expense of whole automated taxonomy construction and maintenance task.
\end{itemize}

The existing hypernym-hyponym can be used as training data, which can be on solution to the challenge of the training data.
And with the rapid growth of pretrained language models (PLMs), one can quickly fetch the representation of concept based on the text of concept since PLMs have pretrained every token of the text, which can be one solution to the challenge of the efficiency.
However, simply sampled from the existing taxonomy only produce weak training data, and PLMs have limited ability in representing hypernym-hyponym relation based on its pretrained representation.
Combine the above methods, GANTEE will do much better not only taxonomy enterance evaluation, but also taxonomy expansion.

GANTEE use Generative Adversarial Network (GAN) to produce deceptively negative data and fidelity positive data with theoretically infinite number.
PLMs are used during the adversarial process to enforce the discriminative model $\mathcal{D}$ to distinguish the hypernym-hyponym relationships based on the representation of PLMs, rather than relying on similarity as in the previous method.
And designed discriminative model $\mathcal{D}$ learns to give supervised signals of telling the generated text apart from the text in real query concept, and the generative model $\mathcal{G}$ learns to confuse $\mathcal{D}$ by generating text which has a deceptively ``hyponym'' relation to the given text of an anchor concept.
The generated texts are expected to be a fake query concept which is a hyponym to the given anchor concept, but a question proposed by the previous research\cite{guo2018leakgan} that when the discriminator is too strict, the effective gradient won't be calculated, so we use a loose rollout discriminator and a strict hyper discriminator to provide supervised signal during the generating process.
Rollout discriminator is proposed to alleviate the problem of gradient passing back during the rollout process \cite{yu2017seqgan}, which is a process proposed to get the reward of unfinished text.
A hyper discriminator is used to detect whether the generated text has a ``hyponym'' relation to the given anchor concept after the generating process.

In the experiments, we benchmark the taxonomy entrance evaluation task on three real-world taxonomies in two different languages, English and Chinese.
We make new State-Of-Art score in the experiments by improve the hit@1 from \red{xxx} to \red{xxx}, MR from \red{xxx} to \red{xxx} and MRR from \red{xxx} to \red{xxx}.
Then we test the efficiency of our method, and GANTEE still outperforms the baselines by a large margin on all three datasets.
Finally, extensive experiments have been conducted to study how the parameters affect the result of GANTEE, and provide an effective tuning scheme for the following researchers who will use our method in the future.

\subsubsection{Contribution.}
To summarize, our major contributions include: 
(1): a more realistic task called taxonomy entrance evaluation which judge whether a query concept should be added to an existing taxonomy in an efficient way;
(2): a novel, effective method called GANTEE to generate confusing negative and fidelity positive data which boosts the performance and the efficiency for both traditional taxonomy expansion tasks and taxonomy enterance evaluation tasks.
(3): extensive experiments that verify the effectiveness of GANTEE on three datasets in two languages, and an effective tuning method was proposed by the experiment.