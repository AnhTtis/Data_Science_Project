Die jüngsten technologischen Fortschritte haben Head Mounted Displays (HMDs) kleiner und kabellos gemacht und fördern so die Vision von allgegenwärtiger Interaktion mit Informationen in einer digital erweiterten physikalischen Welt. Zur Interaktion mit solchen Geräten wird bislang Eingabe-seitig – neben wenig intuitiven Fingergesten – vor allem dreierlei verwendet: 1) Touch-Eingabe auf dem Gehäuse der Geräte oder 2) auf Zubehör (Controller) sowie 3) Spracheingabe. Während diese Techniken, abhängig von der aktuellen Situation des Benutzers, sowohl Vor- als auch Nachteile haben, so ignorieren sie weitgehend die Fähigkeiten und Geschicklichkeit, die wir im Umgang mit der realen Welt zeigen: Während unseres ganzen Lebens haben wir ausgiebig trainiert unsere Gliedmaßen zu benutzen, um mit der physischen Welt um uns herum zu interagieren und sie zu manipulieren.

Diese Arbeit untersucht, wie sich diese Fertigkeiten und Geschicklichkeit unserer oberen und unteren Gliedmaßen, die in der Interaktion mit der realen Welt erworben und trainiert wurden, auf die Interaktion mit HMDs übertragen lassen. So entwickelt diese Arbeit die Vision der \emph{Around-Body Interaction}, in der wir den Raum um unseren Körper, definiert durch die Reichweite unserer Gliedmaßen, für eine schnelle, genaue und angenehme Interaktion mit solchen Geräten nutzen. Diese Arbeit trägt vier Interaktionstechniken bei, jeweils zwei für die oberen und zwei für die unteren Gliedmaßen: Der erste Beitrag zeigt, wie der räumliche Abstand zwischen Kopf und Hand genutzt werden kann, um mit HMDs zu interagieren. Der zweite Beitrag erweitert die Interaktion mit den oberen Gliedmaßen auf mehrere Benutzer und veranschaulicht, wie die Registrierung von augmentierten Informationen in der realen Welt kooperative Anwendungsfälle unterstützen kann. Der dritte Beitrag verlagert den Fokus auf die unteren Gliedmaßen und diskutiert, wie Fußberührungen als Eingabemodalität für HMDs genutzt werden können. Der vierte Beitrag stellt vor, wie seitliche Verschiebungen des Gehweges für die mobile und freihändige Interaktion mit HMDs während des Gehens genutzt werden können.