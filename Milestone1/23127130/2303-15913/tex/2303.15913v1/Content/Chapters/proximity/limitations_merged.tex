\section{Limitations and Future Work}
\label{sec:proximity:limitations}

The study design, as well as the results of the controlled experiment, hint at some limitations and directions for future work. This section lists these points and provides possible solutions.

\subsection{Beyond the One-Dimensional Interaction Space}

This chapter focused on rectilinear interactions alongside the user's line of sight. This limitation was deliberately chosen in order to provide a sound and rigorous assessment of human capabilities for such types of interaction. However, the other degrees of freedom provided by the shoulder, elbow and wrist joints allow for a multitude of other movements and, thus, other proximity-based interactions in front of the user, that may be beneficial for future use of \acp{HMD}. The design space opened here offers many starting points for future contributions in this area.

\subsection{Combination with other Input Modalities}

This chapter mainly focused on the \emph{selection} of information. Future systems will, therefore, also require means to \emph{manipulate} information. There are several approaches to address this: First, the second hand could be used for on-body touch input on the currently active layer, similar to~\ncite{Dezfuli2012}. Second, proximity-based interaction could be supplemented, similar to~\ncite{Whitmire2017}, with finger gestures of the same hand: Users can use their thumbs to provide discrete and continuous input on the remaining fingers or the palm. Furthermore, a combination with further input modalities would be conceivable, e.g., eye tracking.

\subsection{Anchor Point of Proximity}

This chapter focused on interpreting the proximity between the user's hand and head as an input modality. There are also other possible anchor points of proximity conceivable, such as the abdomen or chest of the user. The relocation of the anchor point could alleviate the problem of unpleasant interactions directly in front of the user's face. In addition, multiple anchor points could be active simultaneously to provide the user with multiple dimensions for interaction. For future use of such interfaces, it is, therefore, necessary to establish an understanding of the influence of the anchor point on the accuracy, efficiency and overall user experience of such interfaces.