\section{Experiment I: Continuous Interaction}
\label{sec:proximity:exp1}

The following section presents the methodology (see section \ref{sec:proximity:methodology}) and the results (see section \ref{sec:proximity:results}) of a controlled experiment investigating the human capabilities for a proximity-based one-hand input modality in multi-layer information spaces.

\subsection{Methodology}
\label{sec:proximity:methodology}

This section presents the methodology of a controlled experiment focusing on proximity-based continuous interaction. More specifically, the controlled experiment addressed three main research questions:

\begin{enumerate}
	\item[RQ1] How accurate and efficient users can interact with the layered information space in a search task scenario?
	\item[RQ2] How does the direction of interaction and the side of the hand affect the efficiency and accuracy of the interaction?
	\item[RQ3] How to design the interaction space in terms of layer thickness, number of layers, and convenient boundaries of the physical interaction volume?
\end{enumerate}

For this, 14 participants (P1-P14: 4 female, 1 left-handed), aged between 24 and 29 years ($\mu=26$, $\sigma=1.6$), were recruited. The average height was 177cm ($\sigma=9.5$cm) with an average arm length (measured from armpit to carpus) of 59cm ($\sigma=3.6cm$). No compensation was provided.

\subsubsection{Design and Task}

\textfigH{proximity/conditions}{The information space alongside the participants' line of sigh used in the experiment, grouped by the distance to the starting point.}

To answer the research questions presented above, the design of the task was based on a basic multi-layer information space alongside the participants' line of sight (see figure \ref{fig:proximity/conditions}) consisting of randomized integer numbers (each layer displayed one number) similar to~\ncite{Spindler2012}. The conditions varied the \emph{number of layers} in the available interaction space (which directly correlates with the layers' thickness) as an independent variable with the values of \emph{12, 24, 36, 48, 60} and \emph{72}. In addition, the conditions also varied the \emph{direction of interaction} between \emph{flexion} and \emph{extension} as a second as well as the \emph{side of the hand} as (\emph{palm} or \emph{backside}) as a third independent variable. The subsequent analysis examined the influence of the individual factors on the participants' performance in terms of accuracy and efficiency. 

The experiment varied the independent variables with 6 levels for \emph{numbers of layers}, 2 different \emph{hand sides}, and 2 \emph{directions of interaction} with 6 repetitions (two from each zone) for each combination in a repeated measure design, resulting in $6\times2\times2\times6 = 144$ trials per participant. Informal pre-tests suggested that these levels would provide the highest accuracy. The order of the conditions was counterbalanced using a Balanced Latin Square design for the number of layers and the direction of the interaction. For practicality reasons, the \emph{side of the hand} condition was excluded from the Latin Square design because remounting the trackable marker resulted in also recalibrating the system. However, half of the participants performed all palm-side trials first, while the other half started with the backside trials.

\begin{figure}[t!]
	\begin{center}
		\includegraphics[width=\linewidth]{Content/Figures/proximity/concept}
		\caption{Traveling distance zones (a) and setup of the experiment (b-d).}
		\label{fig:proxi:concept}
	\end{center}
\end{figure}

The participants' first task was to search for the one red colored number in the stack of white colored numbers (see fig. \ref{fig:study}). Once found, participants confirmed the discovery by pressing a button with their non-interacting hand. Directly afterwards, as the second task, participants had to hold the hand steady at the respective position for 3 seconds to measure the accuracy while trying to hold on a layer.

The system defined the maximum boundary of the interaction space with the participant's individual arm-length and the minimum boundary as the near point of the human's eye of young adults (not closer than 12.5cm to the user's face~\ncite{Kulp1999}). Furthermore, the starting point of all trials was defined as half of the distance between the minimum and the maximum interaction distance, resulting in an elbow joint deflection of around 100 degrees. Informal pre-tests showed this to be a natural and relaxed holding position for the hand. To systematically analyze influences of the traveling distance of the user's hand, the total available interaction space in each direction was divided into three equal-sized zones for later analysis: near, medium, and far as shown in Figure \ref{fig:proximity/conditions}).

\subsubsection{Experiment Setup and Apparatus}

The setup used an optical tracking system (OptiTrack, see fig. \ref{fig:proxi:concept} b) to precisely measure the linear distance between the participant's hand and eyes alongside the participant's line of sight. To reliably track the position and orientation of the participant's head and hand, the participants wore two trackable apparatuses: A glasses frame and a glove, each augmented with a number of small retro-reflective markers (see fig. \ref{fig:proxi:concept} d). The system further used the real-time tracking information to fit the projected feedback to the participant's hand (see fig. \ref{fig:study}) to simulate an \ac{AR} system. For each trial, we measured:

\begin{enumerate}
	\item the \emph{task completion time (TCT)} as the timespan between starting the trial and confirming the discovery of the target.
	\item the \emph{overshooting error} as the maximum deviation in the distance (in mm) between the center of the target layer and the participant's hand after first reaching the target layer before confirming the discovery.
	\item the \emph{holding error} as the maximum distance (in mm) from the starting point of the holding task.
\end{enumerate}

\subsubsection{Procedure}

\begin{figure}[t!]
	\begin{center}
		\includegraphics[width=\linewidth]{Content/Figures/proximity/study}
		\caption{Visual feedback in the experiment: After reaching the starting position (a), the system showed the direction of interaction (b). The participants task was to browse through a stack of white colored numbers (c) to find the one red colored number (d).}
		\label{fig:study}
	\end{center}
\end{figure}

The investigator introduced the participants to the concept and experiment setup and asked the participants to put on the two trackable apparatuses before calibrating the system to adapt it to the respective arm size. Before starting each trial, the system guided the user to the starting position through visual feedback displayed on the user's hand. Once in the starting position, the system displayed the direction of the interaction. Each trial started by pressing the button. Once the target was found, the participant confirmed the discovery through another click. After that, the system informed participants to hold their current position for three seconds. Participants did not receive any feedback during the holding task and were not informed on the current layer thickness.

After each condition, participants took a 30 seconds break. The experiment concluded with a semi-structured interview focusing on the participants' overall opinion about the concept, preferred interaction boundaries (minimum/maximum distance), and differences between the tested conditions. The experiment took about 60 minutes per participant.

\textfigStudybox{proximity/studybox_proximity1}