\section{Concept}
\label{sec:proximity:intro}

\cptteaser{proximity/teaser_small}{A map application as an example of one-handed (a) proximity-based interaction with a linear layered information space. The user can browse map layers by moving his hand through the space (b).}

Especially in mobile situations, one hand of users is often busy interacting with the world, rendering the large body of related work on two-handed interaction techniques unsuitable for such situations. One-handed interaction techniques, in contrast, can reduce the interference with regular interactions with the real world and, thus, cover a larger amount of interaction situations.

Therefore, this chapter focuses on how the large number of degrees of freedom offered by our hands and arms can support one-handed interactions. 

The degrees of freedom of movement of the arm and hand are defined by the degrees of freedom of the joints involved. In particular, this includes the shoulder joint, the elbow joint, the wrist as well as the countless joints of the individual fingers. Since movements of the wrist and fingers are relatively close to (on-body) touch interactions already explored in the body of related work, and movements controlled by the shoulder joint are known to cause fatigue~\ncite{Hincapie-Ramos2014}, this work focuses as a first step on the degrees of freedom of the elbow joint. The elbow joint is movable by \emph{flexion} (i.e., moving the hand towards the body) and \emph{extension} (i.e., moving the hand away from the body). 

This chapter, therefore, explores the proximity dimension defined by the elbow joint as an additional input modality for one-handed mobile interaction: The interaction space alongside the user's line of sight can be divided into multiple parallel planes. Similar to \ncite{Subramanian2006}, each plane corresponds to a layer with visual content that can be displayed on the user's hand. The user can move the hands to browse through successive layers (see figure \ref{fig:proximity/teaser_small}). Further, through the sense of proprioception, users can perform these actions unconsciously, reducing the mental load of interaction and allowing peripheral or completely eyes-free interactions.

This chapter explores two different interaction techniques for \emph{continuous} and \emph{discrete} interaction in this one-dimensional interaction space: For \emph{continuous} interaction, the user can move his hands to browse through successive layers (see figure \ref{fig:proximity/teaser_small}). This movement can, for example, represent scrolling through a list. For discrete interaction, the user can raise his arm at a specific distance and, thus, directly select a layer. Each layer can be mapped to a shortcut action, allowing for fast and immediate interaction.






