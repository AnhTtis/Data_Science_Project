The previous chapter introduced the emergence of a new class of body-based interfaces for mobile interaction with \acp{HMD} that extend the interaction from the surface \emph{on} our body into the space \emph{around} our body defined by the reachable range of our limbs. The discussion of such interfaces led to a classification by limbs used for input and stabilization point of the output. This chapter discusses \emph{upper limb} interfaces for use with \emph{body-stabilized} interfaces (see section \ref{sec:relatedwork:hmds:implementation}) with an emphasis on support for the interaction situations \mobilityLower{}, \singleuserLower{}, \discreteLower{} and \continuousLower{}.

As outlined in section \ref{sec:rw:around:upper}, it is with our hands and arms that we show the greatest dexterity in interacting with the real world. Further, the body stabilization of the visualization allows the user to \emph{carry along} an interface, thus affording mobile interaction. Therefore, interaction techniques in this quadrant of the design space (see figure \ref{fig:proximity/overview_proximity}) are particularly suitable for precise interactions in the mobile context.

The contribution of this chapter is three-fold. First, the chapter presents the results of two controlled experiments investigating a one-handed hand input modality for 1) continuous and 2) discrete interaction with \emph{body-stabilized} interfaces. Second, based on the findings of the experiments, this chapter presents a set of guidelines for designing such \aroundbodyinteraction{} techniques in this quadrant of the design space. Third, building on the design guidelines, this chapter presents use cases for such interfaces and shows the applicability of the presented interaction technique beyond \acp{HMD} through a prototype implementation for smartwatches.

The remainder of this chapter is structured as follows: After reviewing the related works (section \ref{sec:proximity:rw}) and based on the established requirements, the chapter presents the concept for an interaction technique (section \ref{sec:proximity:intro}). Afterward, section \ref{sec:proximity:exp1} and \ref{sec:proximity:exp2} present the methodology and results of two controlled experiments investigating the interaction technique presented. Based on the results, section \ref{sec:proximity:discussion} presents guidelines for the future use of such interfaces. Section \ref{sec:proximity:applicability} provides hints for the applicability of the presented interaction technique for \acp{HMD} and other wearable device classes. The chapter concludes with limitations and guidelines for future work (section \ref{sec:proximity:limitations}).

\bigskip

\begin{mdframed}[style=infoboxstyle]
	\textbf{Publication:} This chapter is based on the following publications:
	
	\begin{small}
		\longfullcite{muller2015a}
		
		\longfullcite{muller2016proxiwatch}
	\end{small}
	
	\textbf{Contribution Statement:} I led the idea generation, implementation, and performed the data evaluation. The student \emph{Sebastian Günther} implemented the study client application. \emph{Mohammadreza Khalilbeigi}, \emph{Niloofar Dezfuli}, \emph{Alireza Sahami Shirazi} and \emph{Max Mühlhäuser} supported the conceptual design and contributed to the writing process.
\end{mdframed}