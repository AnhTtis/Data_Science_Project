\subsection{Results}
\label{sec:proximity:results}

This section reports the results of the controlled experiment investigating the research questions RQ1, RQ2, and RQ3 as described in section \ref{sec:proximity:methodology}. The analysis of the data was performed as described in section \ref{sec:rw:methodology}. 

\subsubsection{Task Completion Time}

\begin{figure}[t!]
	\begin{center}
		\includegraphics[width=\linewidth]{Content/Figures/proximity/taskCompletionTime}
		\caption{Mean TCT and SD for different numbers of layers.}
		\label{fig:taskcompletiontime}
	\end{center}
	\vspace{-.5cm}
\end{figure}

The analysis unveiled that the traveling distance (measured in the three groups \emph{near}, \emph{medium} and \emph{far}) of the hand had a significant effect on the \ac{TCT} (\anovaWithoutEffect{2}{62}{23.27}{<.001}). Post-hoc tests confirmed that the \ac{TCT} for near (\valSi{4.7}{3.6}{s}) and medium zone (\valSi{4.8}{2.7}{s}) targets were significantly smaller ($p<.001$) than for those in the far zone (\valSi{6.4}{3.4}{s}). Post-hoc test did not indicate significantly different \acp{TCT} between medium and near zone targets. Table \ref{tab:proximity/tct_distance} lists the \acp{TCT} for all zones.

Further, the number of layers and, thus, the size of the individual layers had a significant effect on the \ac{TCT} (\anovaCorWithoutEffect{2.45}{31.36}{45.68}{<.001}{.49}). Post-hoc tests confirmed a significantly ($p<.01$) larger \ac{TCT} for higher numbers of layers between all groups. The mean \ac{TCT} increased from  \valSi{3.7}{1.8}{s} for 12 layers to \valSi{7.2}{4.7}{s} for 72 layers. While the mean \ac{TCT} was faster for extension (\valSi{5.5}{3.5}{s}) than flexion (\valSi{5.1}{3.1}{s}), the analysis could not proof any significant effects (\anovaWithoutEffect{1}{13}{2.8}{>.05}). Also, no significant effect of the hand orientation on \ac{TCT} was found (\anovaWithoutEffect{1}{13}{.15}{>.05}, Palm: \valSi{5.2}{3.2}{s}, Back: \valSi{5.3}{3.4}{s}). Also, the analysis could not find interaction effects between the conditions. Figure \ref{fig:taskcompletiontime} shows the \ac{TCT} for the explored numbers of layers and target zones.


\begin{table}[t!]
	\centering

\begin{tabularx}{\linewidth}{YYYYY}
	&   &   & \multicolumn{2}{c}{\textbf{95\% Confidence Interval}}\\
  	\cmidrule(lr){4-5}
	\textbf{Zone} & $\pmb{\mu}$ & $\pmb{\sigma_{\overline{x}}}$ & \textbf{Lower} & \textbf{Upper}\\
	\midrule
	Near   & 4.7s  & .9s      & 2.81s                     & 6.56s                     \\
	Medium & 4.8s  & .7s      & 3.39s                     & 6.21s                     \\
	Far    & 6.4s  & .9s      & 4.62s                     & 8.18s                     \\
	\bottomrule
	\end{tabularx}

	\caption{The Task Completion Time (TCT) per distance zone (in seconds). The table reports the mean value $\mu$, the standard error $\sigma_{\overline{x}}$ and the 95\% confidence interval.}
	\label{tab:proximity/tct_distance}
\end{table}

\subsubsection{Overshooting Error}

\begin{figure}[b!!]
	\begin{center}
		\includegraphics[width=\linewidth]{Content/Figures/proximity/combinedError}
		\caption{Error measurements for the three traveling distance zones.}
		\label{fig:errorperinteractionzone}
	\end{center}
\end{figure}

The traveling distance had a significant effect on the overshooting error (\anovaCorWithoutEffect{1.26}{16.38}{39.44}{<.001}{.63}). Post-hoc tests confirmed significant differences between all zones (all $p<.05$). The observations during the experiment showed that participants initially started with fast movements and slowed down towards their physical boundaries in the far zones, resulting in higher overshooting errors in the near (\valSi{4.4}{1.7}{cm}) and medium ($\mu$=2.1cm, $\sigma$=1.0cm) zones compared to the far ($\mu$=1.6cm, $\sigma$=0.7cm) zone. Table \ref{tab:proximity/overshoot_distance} lists the mean overshoot and corresponding standard errors per target zone, figure \ref{fig:errorperinteractionzone} compares the values graphically.

The analysis showed neither any significant influence of the direction of interaction on the overshooting error (\anovaWithoutEffect{1}{13}{.0008}{>.05}, flexion: \valSi{2.5}{3.0}{cm}, extension: \valSi{2.6}{3.2}{cm}) nor the hand orientation (\anovaWithoutEffect{1}{13}{.11}{>.05}, palm: \valSi{2.6}{3.1}{cm}, back: \valSi{2.6}{3.0}{cm}). Furthermore, the analysis could not show any significant influence (\anovaWithoutEffect{5}{64}{.64}{>.05}) of the number of layers (Min: \valSi{2.2}{3.1}{cm} for 12 layers, Max: \valSi{2.8}{3.4}{cm} for 36 layers). Also, the analysis did not show any significant correlation between the participants' arm-length and their accuracy ($r(166) = -0.8376, p>.05$) in the recorded data.


\begin{table}[h!]
	\centering

\begin{tabularx}{\linewidth}{YYYYY}
	&   &   & \multicolumn{2}{c}{\textbf{95\% Confidence Interval}}\\
  	\cmidrule(lr){4-5}
	\textbf{Zone} & $\pmb{\mu}$ & $\pmb{\sigma_{\overline{x}}}$ & \textbf{Lower} & \textbf{Upper}\\
	\midrule
	Near   & 4.4cm & .45cm    & 3.51cm                    & 5.29cm                    \\
	Medium & 2.1cm & .27cm    & 1.58cm                    & 2.62cm                    \\
	Far    & 1.6cm & .19cm    & 1.23cm                    & 1.97cm                    \\
	\bottomrule
	\end{tabularx}

	\caption{The overshooting error per distance zone (in cm). The table reports the mean value $\mu$, the standard error $\sigma_{\overline{x}}$ and the 95\% confidence interval CI.}
	\label{tab:proximity/holding_distance}
\end{table}

\subsubsection{Holding Error}


The analysis showed significant effects of the distance between the starting point and the holding point on the holding error (\anovaCorWithoutEffect{1.12}{14.56}{5.53}{<.05}{.56}). Post-hoc tests confirmed significant effects (all $p<.05$) between targets in the far (\valSi{1.6}{1.8}{cm}) and the medium (\valSi{1.0}{.9}{cm}) zone as well as between the far and near (\valSi{1.1}{1.1}{cm}) zone. The difference between near and medium zones, however, was not significant ($p>.05$). Table \ref{tab:proximity/holding_distance} lists the mean overshoot and corresponding standard errors per target zone, figure \ref{fig:errorperinteractionzone} compares the values graphically.

The analysis showed neither any significant influences of the direction of interaction (\anovaWithoutEffect{1}{13}{1.65}{>.05}, flexion: \valSi{.7}{.8}{cm}, extension: \valSi{.8}{.8}{cm}) nor of the hand orientation (\anovaWithoutEffect{1}{13}{1.37}{>.05}, palm: \valSi{.8}{.8}{cm}, back: \valSi{.7}{.7}{cm}) on the holding error.

However, the analysis could show a significant (\anovaCorWithoutEffect{1.45}{18.85}{7.21}{<.001}{.29}) influence on the number of layers. Post-hoc tests confirmed a significant ($p<.01$) bigger holding error for 12 layers (\valSi{1.2}{.9}{cm}) compared to all higher numbers of layers. The mean holding error further decreased for increasing numbers of layers (min: \valSi{.6}{.4}{cm} for 72 layers), but was not significant.


\begin{table}[t!]
	\centering
	
\begin{tabularx}{\linewidth}{YYYYY}
	&   &   & \multicolumn{2}{c}{\textbf{95\% Confidence Interval}}\\
		\cmidrule(lr){4-5}
		\textbf{Zone} & $\pmb{\mu}$ & $\pmb{\sigma_{\overline{x}}}$ & \textbf{Lower} & \textbf{Upper}\\
		\midrule
		Near   & 1.1cm & .29cm    & .52cm                     & 1.68cm                    \\
		Medium & 1.0cm & .24cm    & .53cm                     & 1.47                      \\
		Far    & 1.6cm & .48cm    & .66cm                     & 2.54cm                    \\
		\bottomrule
	\end{tabularx}
	
	\caption{The holding error per distance zone (in cm). The table reports the mean value $\mu$, the standard error $\sigma_{\overline{x}}$ and the 95\% confidence interval CI.}
	\label{tab:proximity/overshoot_distance}
\end{table}

\subsubsection{Qualitative Results}

In general, all participants appreciated the idea of being able to interact with multi-layer information spaces through movements of their hand. There was a strong consensus among participants (11 out of 14) that this input modality is suitable for immediate and short-term interactions, such as the serendipitous discovery of contents, fast peeking into information or executing a shortcut. The participants' comments suggested that the convenient boundaries for interaction are approximately the near and middle zones in each direction. Far zones turned out to cause more fatigue on the arm and upper arm muscles.

Regarding the hand side, participants mentioned mixed opinions. Six participants preferred the back side of the hand for interactions. P9 commented, e.g., \enquote{I know this movement, that is like looking at my watch}. The remaining eight participants preferred interactions with the palm side of the hand. Participants did not feel an influence of the direction of the interaction. P13 commented on this: \enquote{Both directions are okay for me, as long as the target is not too far away or too close to my head}. 
