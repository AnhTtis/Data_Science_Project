\section{Conclusion}

This chapter explored an interaction design for one-handed and proximity-based gestures for interacting with \acp{HMD}. More precisely, this chapter investigated the efficiency and accuracy of the interaction in a layered information space for 1) continuous interaction and 2) discrete interaction. The results confirmed the viability and feasibility of this input modality. The traveling distance to the target layer proved to be the primary influence for accuracy and efficiency. 

This chapter added to the body of research in multiple areas:

\begin{enumerate}
	\item This chapter considered the spatial position of the hand in relation to the body as an input dimension for \acp{HMD}. This style of interaction has not been considered by previous work on interacting with \acp{HMD}. Thus, the work presented in this chapter opens up a new field of research for future body-based and one-hand interfaces for \acp{HMD}.
	\item This chapter contributed the design and results of two controlled experiments assessing discrete and continuous proximity-based interaction in a layered information space in front of the body. The results confirmed the viability of this interaction style for accurate and fast interactions. The results, as well as the guidelines derived from the results, provide a reliable basis for the future use and further refinement of such interaction techniques.
	\item  Furthermore, this chapter presented four use cases for proximity-based interactions with \acp{HMD}. Additionally, the chapter demonstrated how the presented interaction techniques could be implemented for the interaction with a smartwatch. Thereby, the chapter highlighted that the presented body-based interaction technique could also be utilized in other areas beyond \acp{HMD}. 
	
\end{enumerate}

\subsection{Integration}

\textfigH{proximity/alice}{Alice uses proximity-based interaction to select a phone contact.}

\interactionbox{alice_proximity}{At the Mall}{
	Alice is at the mall carrying a shopping bag. She wants to take a break and thinks of her friend Bob, who lives nearby. She decides to call Bob. For this, she raises her free hand and scrolls through her contact list by moving her arm back and forth (see figure \ref{fig:proximity/alice}). After she has found the contact, she confirms the call by clicking with her thumb. Bob wants to have some coffee. Alice uses her free hand to scroll through the list of applications and selects the map application. In the following, she uses proximity-based interaction to set filters on coffee shops and set the radius of the card. The two agree on a coffee shop and set off.
	
	The entire interaction takes place one-handed and allows Alice to interact quickly and concurrently with information while still holding the shopping bag in her hand.
}

The interaction technique presented in this chapter allows for fast and direct interaction with \emph{body-stabilized} interfaces, leveraging our \emph{upper limbs} for one-handed input. In particular, the one-handed operation allows mobile interaction with information, while the second hand is still available for interaction with the real world (supporting \mobilityLower{}). The two interaction styles presented allow the operation of different common types of interfaces, such as cascading menus (\discreteLower{}) or sliders (\continuousLower{}). Therefore, the chapter contributes to the vision of ubiquitous \aroundbodyinteraction{} with \acp{HMD} as introduced in section \ref{sec:introduction:aroundbodyinteraction}.

As described in the section \ref{sec:proximity:limitations}, the design space of interactions with \emph{body-stabilized} interfaces using the \emph{upper limbs} exhibits much more degrees of freedom than could be covered in the scope of this work. The extension of the concept by further input dimensions (i.e., 2D or 3D movements) as well as the connection with on-body touch input (carried out by the fingers of the same hand) could enhance the expressiveness while potentially maintaining the presented advantages of proximity-based interactions. While these ideas were outside the scope of this work, the ideas and quantitative results presented in this chapter can serve as a reliable baseline for further explorations of the design space.

\subsection{Outlook}

Interfaces registered to the body of the user have an inherent focus on \singleuserLower{} as they move together with the user by design, rendering such interfaces unsuitable for multiple users. Chapter \ref{ch:cloudbits} discuss how \emph{world-stabilized} interfaces can be used to support such \multiuserLower{} situations.

Furthermore, interaction with the \emph{upper limbs} cannot support situations in which both hands are occupied, e.g., when something is carried in both hands. Chapters \ref{ch:cheesyfoot} and \ref{ch:walktheline} present solutions for such situations leveraging the \emph{lower limbs}.