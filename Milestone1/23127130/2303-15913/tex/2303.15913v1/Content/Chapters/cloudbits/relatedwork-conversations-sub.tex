\subsection{Related Work}
\label{sec:cloudbits:relatedwork}

The following section discusses the relevant prior work regarding information retrieval in conversations with regard to both, prevalent problems and approaches to overcome them. In the following, this section compares the approaches to the requirements established through the exploratory user study (see table \ref{tab:req:cloudbits}).

\subsubsection{Information Retrieval in Conversations}

Many studies investigated the influence of information retrieval using mobile devices on the quality of conversations. \typcite{Su2015} found that smartphones help to enhance conversations through additional information but can also cause disruptions to the ad-hoc and informal nature of conversations. The use of such devices \enquote{force[s] people to isolate themselves rather than engage in their immediate surroundings}. Continuing on this, \typcite{Porcheron2016} found that, while additional information retrieval may help to solve open issues during conversations, the process of information retrieval also causes people to get distracted from the actual conversation. After the transient focus on the mobile device, people also showed problems to re-engage with the discussion. \typcite{Brown2015} found that information retrieval can be a vivid part of a conversation and \enquote{rather than search being solely about getting correct information, conversations around search may be just as important.}

The perseverative interaction with mobile devices can lead to encapsulation in a mobile bubble, a phenomenon defined as \emph{phubbing}~\ncite{Coehoorn2014}. Emphasizing the influence on the quality of conversations, \typcite{Przybylski2013} found that the interaction with mobile devices reduces closeness and trust as well as interpersonal understanding and empathy between the participants. Regarding family meal situations, \typcite{Moser2016} found that \enquote{attitudes about mobile phone use at meals differ depending on the particular phone activity and on who at the meal is engaged in that activity, children versus adults.}%

Regarding meeting scenarios, \typcite{Bohmer2013} found that phone usage interferes with and decreases productivity and collaboration. Individuals have the feeling that they make productive use of their smart devices but perceive the usage of others as unrelated.

\subsubsection{Approaches for Conversation Support Systems}

Various approaches have been presented to overcome the presented problems and to support information retrieval in conversations.

\typcite{SusLundgren8601} proposed to use a tablet as a public display to provide awareness for the activity of persons working on their smartphones. \typcite{Ferdous2016} proposed to use personal devices as a combined shared display to support interactions and conversations at the family dinner table. To support conversations between strangers, \typcite{Nguyen2015} proposed to display potential conversation topics of mutual interest through \acp{HMD}.

Further approaches focus on managing the time users focus on their mobile devices. \typcite{Lopez-Tovar2015} propose to assess the importance of notifications and whether the user needs to be interrupted. As another approach, \typcite{Eddie2015} presented a solution that proactively interrupts users to discourage excessive mobile phone usage during conversations.

While all of the presented approaches are practical and helpful in several ways, none of the approaches offers comprehensive support of the requirements for conversation support systems established in the explorative study (see table \ref{tab:req:cloudbits}).

Highly related, \typcite{Suh2007} evaluated such a shared information space for collaboration, sharing, and interaction with contents. However, the authors a) did not focus on the area of content retrieval and b) focused on mobile \ac{AR} using smartphones. In this work, we argue that \acp{HMD} are a better fit for this use case as they leave the user's hands-free for interaction.

\subsubsection{Zero-Query Search}

To reduce the time needed to retrieve data, zero-query search has been proposed as a proactive means to retrieve necessary information~\ncite{Rhodes2000}. Such systems use contextual cues such as location, time, or usage history to retrieve and proactively present information to the user. In recent years, zero-query search-based systems such as Google Assistant or Microsoft Cortana were broadly implemented in consumer devices. This was accompanied by a stream of research focusing on how contextual cues can be used to derive search queries and when they should be presented to the user~\ncite{Yang2016, Shokouhi2015}.

\renewcommand*\theadfont{\bfseries}
\settowidth\rotheadsize{\theadfont Unsolicited and Real-Time}
\renewcommand\theadgape{}
\renewcommand\theadalign{lc}
\renewcommand\rotheadgape{}
\begin{table}[t!]
	\centering
	\begin{tabular}{m{5cm}cccc}
		& \rothead{R4.1: Unsolicited and Real-Time Service} & \rothead{R4.2: Supporting Fluid Transition and Re-Engagement} & \rothead{R4.3: Selective Sharing from the Public-Private Information Spectrum} \\
		\midrule
		\cite{SusLundgren8601} & \reqYes & \reqPartially & \reqNo  \\
		\cite{Ferdous2016} & \reqNo & \reqNo & \reqYes \\
		\cite{Nguyen2015} & \reqYes & \reqNo & \reqYes \\
		\cite{Lopez-Tovar2015} & \reqPartially & \reqPartially & \reqNo \\
		\cite{Eddie2015} & \reqPartially & \reqPartially & \reqNo \\
		\cite{Radeck-Arneth} & \reqYes & \reqPartially & \reqNo \\
		\cite{Andolina2015} & \reqYes & \reqPartially & \reqNo \\
		\cite{Suh2007} & \reqNo & \reqYes & \reqYes \\
		\bottomrule
	\end{tabular}
	\caption{Fulfillment of requirements of the related works. \reqYes~ indicates that a requirement is fulfilled, \reqPartially~indicates partial fulfillment.}
	\label{tab:req:cloudbits}
\end{table} 

Building on the concept of zero-query search, work on ambient voice search~\ncite{Radeck-Arneth} supports users in a conversation scenario by providing relevant information to all participants of the conversation on a public display. This allows users to interact with the information through direct (touch) interaction on the display. Focusing on collaborative idea generation, \typcite{Andolina2015} presented a similar system to support users through displaying related keywords based on the topics of their conversation. However, the presented concepts do not provide complete support for fluid transition and, further, do not support the sharing of private information.