\section{Guidelines}
\label{sec:cloudbits:guidelines}

Based on the results of the study, the following section presents a set of guidelines for the design of user interfaces for conversation support systems:

\subsection{Leverage the Surrounding Physical Space}

The analysis of the results of the study presented showed a great interest of the participants in the possibility of arranging information in space, categorizing it and using it for cooperation. This interest manifested itself both in the observed interactions of the participants and in the qualitative feedback given by the participants in the semi-structured interviews (see section~\ref{sec:cloudbits:study2results:spatialarrangement}). The participants used the entire available space of the room in all three dimensions to work with information in \emph{working} and \emph{storage} areas (see section  \ref{sec:cloudbits:study2results:zones}). Therefore, conversation support systems should enable the usage of the entire available space for categorization of and interaction with the information.

\subsection{Provide Means for Fluid Transition and Re-Engagement}

The results of the study suggested that today's conversation support systems do not adequately support users in (re)-engaging in conversations. The analysis found that the lack of awareness of the activities of other conversation partners can lead to a feeling of exclusion, which in turn can manifest itself in immersing oneself in interaction with technology and losing connection to the conversation (see section \ref{sec:cloudbits:study2results:awareness}). This becomes a problem especially after interruptions of the conversation, whether caused by unwanted external influences or by the temporary focus on the reception of information (see section \ref{sec:cloudbits:study2results:reengagement}). It is, therefore, necessary for conversation support systems to provide the context to help users get back into the conversation after breaks. Furthermore, such systems should provide awareness of the actions of other participants during the conversation in order to prevent participants from drifting away.

\subsection{Support Selective Sharing from the Public-Private Information Spectrum}

The analysis of the related works and the results of the study presented in this chapter showed that today's systems do not offer sufficient possibilities for fast and easy sharing of information (see section \ref{sec:cloudbits:study2results:sharing}). In particular, the analysis showed privacy concerns when sharing information by handing around the private smartphone. Furthermore, the analysis also highlighted the importance of sharing information in collaboration. Therefore, conversation support systems should support selective sharing for privacy-preserving sharing of private information without the need to share the complete device.