\section{Related Work}
\label{sec:cloudbits:relatedwork:hmd}

The related works regarding interaction with \acp{HMD} were covered in chapter \ref{ch:relatedwork} already. The related works concerning interaction using the upper limbs were addressed in section \ref{sec:proximity:rw}. This section further adds related work in the area of \acp{HMD} in social interaction.

\subsection{Head-Mounted Displays in Social Interaction}

As discussed in section \ref{sec:relatedwork:hmds}, \acp{HMD} are a promising technology for immediate and direct interaction with information. Despite all benefits, research showed that the use of such interfaces introduces problems in social interactions: The form factor of \acp{HMD}, as well as the visibility of the \ac{HMD}’s output restricted to the wearer only, can have a negative impact on attentiveness, concentration, and eye-contact, and, thus, lead to less natural conversations~\ncite{McAtamney2006, Due2015}.


While some of the presented problems can be solved through technological advances (e.g., better eye contact through less bulky devices), other problems (e.g., the private experience of information) are inherently connected to the use of \acp{HMD}. As a possible way to increase social acceptance and to mitigate for these effects, \typcite{Koelle2015} showed that the offering of awareness to \enquote{communicate the intention of use} helps to build interfaces that overcome problems presented above. Following this stream of research, \typcite{Gugenheimer2017} proposed interaction techniques between \acp{HMD} persons wearing \acp{HMD} and external persons and \typcite{Chan2017} showed how an additional display on the \acp{HMD} could help to involve other persons.

As another approach to overcome the effects of the inherently private display, several \acp{HMD} can synchronously access a shared information space (i.e., Interface elements are displayed for multiple users at a synchronized real-world position) as proposed by Billinghurst et al~\ncite{Billinghurst1998a, Billinghurst2000}. Based on this inspirational work, such shared information spaces have been used in various contexts such as gaming~\ncite{Szalavari1998} or learning~\ncite{Bacca2015}. Recent advances in computer vision allow remote attendees to seamlessly integrate into a conversation as a lifelike 3D model, overcoming spatial boundaries ~ \ncite{Orts-Escolano2016}.

However, to the best of our knowledge, no prior work has been concerned with the retrieval of content in such shared information spaces. Retrieving information and the associated involvement in the interaction with technology can, however, cause participants to lose contact with the conversation~\ncite{Su2015}. Therefore, we have chosen the area of information retrieval as the use case for this chapter and will discuss this in more detail in the next section.