\section{Concept}
\label{sec:cloudbits:introduction}

The following section presents an interaction technique for the \emph{upper limbs} to interact with \emph{world-stabilized} interfaces. Since the focus of this contribution is on the utilization of such world-stabilized interfaces to support \multiuserLower{} situations, this chapter is build around a use case for collaborative \ac{AR} systems. As the review of related work revealed a gap in the field of information retrieval support for such collaborative \ac{AR} concepts, this chapter centers around this use case. However, the interaction techniques presented are not tied to this particular use case and, thus, can support other situations as outlined in section \ref{sec:cloudbits:limitations:usecase}.

The remainder of this section is structured as follows: After a more detailed introduction to the use case (section \ref{sec:cloudbits:introduction:introduction}), the section presents the methodology (section \ref{sec:cloudbits:study1}) and results (section \ref{sec:cloudbits:study1:results}) of an exploratory study to establish requirements for a system to support this use case. Next, the section presents related work in the area of conversation support systems and classifies them according to the requirements (section \ref{sec:cloudbits:relatedwork}). Last, the section presents the concept for \acp{HMD}-based conversation support leveraging \emph{world-stabilized} interfaces that are operated using the \emph{upper limbs} (section \ref{sec:cloudbits:cloudbits}).

\subsection{Introduction and Background}
\label{sec:cloudbits:introduction:introduction}

Today, the retrieval of digital information during conversations and meetings (for both, private and shared use) happens using personal (smart) devices such as smartphones or laptops. However, the interaction with the smart device requires the user to shift the (visual) attention to the device and, thus, away from the conversation and other tasks. This cognitive focus switching between conversation and smart device can hamper the flow of the conversation: Users can lose the connection to the conversation~\ncite{Przybylski2013}, or even favor the interaction with the smart device over the actual conversation, a phenomenon known as \emph{phubbing}~\ncite{Coehoorn2014}. This can decrease mutual awareness of user's activities or otherwise hamper the joint experience. Furthermore, sharing retrieved information with other participants of the conversation can be cumbersome: Users need to connect their device to a public display or pass round the device which imposes privacy issues~\ncite{Karlson2009}.) 

Prior work proposed ambient voice search~\ncite{Radeck-Arneth} as a first step towards supporting conversation scenarios through proactive information retrieval. Such systems automatically retrieve relevant auxiliary information through voice recognition and topic extraction and present it on a shared (large-scale) public display to all users. This can help to diminish the need for individual information retrieval and, thus, to mitigate the challenges set out above. However, interaction with the presented information is limited to touch-based interaction on the public display itself. Furthermore, such a system cannot provide individual per-user output and, therefore, support private information.

In contrast to prior systems that present auxiliary information on screens (including mobile phones and public displays), we argue in this chapter that the representation of information using \acp{HMD} in the periphery of users has a great potential to unobtrusively support the interaction with information. Therefore, this chapter presents a novel approach to visualize and interact with public and private information in the  user's periphery to support conversations. 

In order to establish the requirements for such a system, a focus group-based study was conducted, which will be presented in the next section (section \ref{sec:cloudbits:study1}). Based on the results of the study, section \ref{sec:cloudbits:relatedwork} presents a classification of the related work in this area. Building up on the restults on the study and the review of the related works, section \ref{sec:cloudbits:cloudbits} presents the final design of CloudBits.