\section{Results and Discussion}
\label{sec:cloudbits:study2results}

The following section reports on the results of the evaluation with respect to the research questions presented above. Therefore, the section reports the results grouped by 1) the spatial arrangement and use of the 3D space, 2) the participants' usage of working and storage zones, 3) the mutual awareness of participants, 4) CloudBits' support for (re-) engagement, and 5) the selective sharing from the public-private information spectrum. The analysis of the data was performed as described in section \ref{sec:rw:methodology}.

\subsection{Spatial Arrangement and Use of the 3D Space}
\label{sec:cloudbits:study2results:spatialarrangement}

The participants highly appreciated the general idea of in-situ support through augmented information drops. Notably, the possibility for an individual arrangement of information drops, visualized in a shared 3D space, was received enthusiastically. The analysis found that most participants used the complete available ($\sim$ 30 square meter) space to sort the information and that participants used all available dimensions (top/down, left/right and front/back). Nine participants expressed their satisfaction of using a wide space or even the whole room as an information space in the semi-structured interview. P6 commented: \enquote{Compared to the mobile scenario, where the information space is restricted to a very small screen, CloudBits big scene filled with information is extremely desired.} 

\textfig{cloudbits/working_zones_public}{Spatial arrangement of shared information in the study. Participants divided the information drops into working (W) and storage (S) zones based on their current task.}

The human capabilities allow creating a cognitive map that contains relative positions and orientations of objects~\ncite{Manns2009}. This so-called spatial memory allowed participants to place, arrange, and relocate information drops naturally. P8 compared this to pervasive work practices for knowledge workers on desks: \enquote{It is like... I could arrange my documents on the desktop and easily memorize where to find them. However, doing that in 3D is much more fun.} P10 added: \enquote{I can easily categorize the retrieved information in space. Since the arrangement is personally customized, I can immediately remember where is what.} The participants used this shared augmented information space to refer to information drops through natural gestures such as pointing and looking (see figure \ref{fig:cloudbits/study_setting}, c). P10 commented on this: \enquote{I just accidentally pointed at the information and said \emph{Look there!}. It was amazing that my [P9] could also see the same information on the same place and understood what I meant.}

Participants found that the smartphone condition required constant focus switches between different information sources on the smartphone and between the smartphone and the actual conversation and, thus, constant re-engagement. P3 commented: \enquote{When I need to search for information, using the mobile phone required me to constantly switch my focus from one application or piece of information to the other. So I will lose detail of one information when I switch to the next and need to repeatedly do the switching.}

In comparison, CloudBits allowed the participants to \enquote{see more information at a glance} (P7) while still being able to focus on the actual conversation, supporting the \emph{focus+context} (see section \ref{sec:cloudbits:cloudbits:focuspluscontext}) nature of CloudBits.

\subsection{Working and Storage Zones}
\label{sec:cloudbits:study2results:zones}

Participants used different spatial configurations to sort and categorize information drops. The spatial arrangements were created collaboratively in an on-demand manner. While the participant groups created individual categories for categorization, we found that participants across all groups divided information drops into 

\begin{description}
	\item[Working Zones] containing the information drops that participants were actively using for their current task.
	\item[Storage Zones] containing the information drops not used at that time, but that participants kept for later use.
	
\end{description}

While the spatial layout of these zones differed over all participant groups (see figure \ref{fig:cloudbits/working_zones_public}), the general usage of these zones proofed to be consistent over all participants, a behavior of users that was already found when interacting with tabletops~\ncite{Scott2003}.

The participants' tasks required them to make decisions and re-retrieve this information later on. Participants used the storage zones to pin relevant drops for later use while explicitly removing or letting fade out unused information drops. Five participants pointed out that they found CloudBits pin concept \enquote{very intuitive} (P2, P6) as \enquote{when I pin my to-do post-it on the kitchen board} (P2).

In the smartphone condition, participants reacted to the requirement to keep information for later access with different techniques: Participants wrote the information down on paper or created screenshots on the smartphones. P12 commented on the problems: \enquote{I need to browse and remember which snapshots are relevant as they look all similar and include a lot of text.}

\subsection{Awareness}
\label{sec:cloudbits:study2results:awareness}

The analysis showed that participants followed the actions of their partner through brief glances at their actions. In the following interview, all participants reported that they could gain insights about the current state of the work of their partner. P8 explained that \enquote{While using CloudBits, I was really happy that I could see what my partner is looking at and interacting with.}

The observations showed that the missing awareness in the smartphone condition caused a management overhead in the conversation. Participants were forced to give regular updates about their current actions and whether they were ready to continue the conversation with regard to the content. P3 described the problems: \enquote{We both wanted to search [...] each using our own mobile devices. [...] when I was still in the search process, she found her desired answer and started speaking about the next step we needed to do. But I was still engaged with the searching process of the last needed information piece and could not get what she was talking about.}

\subsection{Supporting (Re)-Engagement}
\label{sec:cloudbits:study2results:reengagement}

During both conditions, the investigator enforced a distraction through faked technical problems in the study setup. After five minutes, the investigator told the participants to continue from where they left off.

The participants' comments, as well as the observations, showed that CloudBits provided them with means for easy and fast re-engagement. After the break, participants started the re-engagement process in the CloudBits condition by looking around the room, using the information drops to get back to the conversation. The analysis showed that participants used both, 1) the falling information drops (covering the latest topics of the conversation) as well as the pinned information drops in their working zones to re-engage with the conversation and their individual tasks. During the interviews, seven participants explicitly appreciated that the necessary information to re-engage was directly available without the need to interact with the system actively. 

In contrast, the observations, as well as the comments from the participants, clearly showed that the smartphone condition did not provide sufficient support for re-engagement. P4 said that \enquote{If I lose my attention to the topic of conversation, I need to concentrate for a while in order to be able to switch back to the topic, using my mobile phone does not help at all and might be even more distracting}. P10 added \enquote{I usually have lots of open information tabs on my mobile device which needs to be browsed to skim them, but I am not able to immediately remember where I have stopped.}

\subsection{Selective Sharing from the Public-Private Information Spectrum}
\label{sec:cloudbits:study2results:sharing}

\textfig{cloudbits/working_zones_private_back}{Spatial arrangement of public (W,S) and private (P1, P2) information in the study.}

All participants showed enthusiasm regarding the possibility to access both public and private information in a shared workspace at the same time. When asked for the reasons, participants reported that this enabled them to selectively share information without the need to share the complete device. P9 explained: \enquote{CloudBits let me share a part of the information which needs to become public [...]. I always have concerns about other persons having access to all my data while sharing information with others through my mobile phone.} P10 further added: \enquote{I really did not want to share my personal device to my partner, but it was also kind of impolite to ask him to search for the same information himself. This meant I have no trust in you or I do not want to help you.}

Participants did not mix public and private information in the same zones (see figure \ref{fig:cloudbits/working_zones_private_back}). In particular, participants chose spatial arrangements to keep private information drops far apart from the shared work zones. We observed two basic patterns for the spatial arrangement of private information: Half of the participants positioned private information drops close to themselves (see figure \ref{fig:cloudbits/working_zones_private_back} P1), while the other half of the participants moved them to an unused space preferably far away (see figure \ref{fig:cloudbits/working_zones_private_back} P2).

While the study setup did not impose any restriction on the physical arrangement of the participants in the room, the observations showed that participants chose different arrangements for the conditions to support the process of information sharing. In the CloudBits condition, all participants arranged themselves in a face-to-face setting (see figure \ref{fig:cloudbits/study_setting} a). The interview revealed that this provided them with a comfortable position for the conversation and shared information access and, further, gave them the necessary space to perform mid-air gestures.

In the smartphone condition, participants showed two different approaches for the spatial arrangement to support information sharing: Three pairs constantly changed their position between face-to-face for individual work and side-by-side for sharing information through each other's smartphone screen (see figure \ref{fig:cloudbits/study_setting} b). The other three pairs stayed in a face-to-face arrangement during the whole session and tried to exchange the found information orally.

Comparing both conditions, participants commented that information sharing felt more immediate and efficient in the CloudBits condition. P5 explained: \enquote{Similar to the real world, sharing information using CloudBits occurs by just naturally changing the virtual position of the information to where my partner is. This experience reminds me exactly to when I pass a physical object to someone to share it.}  P8 added that \enquote{CloudBits information sharing saves the effort currently is needed [...] to share [...].}

\subsection{AttrakDiff}
\label{sec:cloudbits:study2results:attrakdiff}

\sameheightpic{cloudbits/attrak_diff}{AttrakDiff Comparison}
{cloudbits/attrak_diff_details}{Detailed Results}
{General (Smartphone: blue, CloudBits: Orange) (a) and detailed (b) results of the AttrakDiff comparison between the two conditions for the four AttrakDiff quality dimensions (left to right: pragmatic, hedonic-identity, hedonic-stimulation, attractiveness).}

The AttrakDiff~\ncite{Hassenzahl2003} questionnaire indicated higher qualities for CloudBits in both hedonic dimensions (HQ-I: Identity and HQ-S: Stimulation) compared to traditional information retrieval using smartphones. The pragmatic qualities were rated on a similar level with slight advantages for the more traditional smartphone condition. In total, CloudBits achieved a higher result for attractiveness (ATT, see figure \ref{fig:cloudbits/attrak_diff_details}).

Based on the feedback of the participants, the lower values for the practical quality dimensions could be based on the new and unfamiliar interaction with \acp{HMD}. Furthermore, the technical limitations of today's \acp{HMD} and of the study client implementation also influenced the results and might let to participants transferring those problems to the general concepts. Further work is needed in this area to identify the detailed reasons for these differences.
