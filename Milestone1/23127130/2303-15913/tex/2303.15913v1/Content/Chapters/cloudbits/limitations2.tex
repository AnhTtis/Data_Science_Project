\section{Limitations and Future Work}
\label{sec:cloudbits:limitations}

The study design, as well as the results of the controlled experiment, hint at some limitations and directions for future work. This section lists these points and provides possible solutions.

\subsection{Scope of the Study}

This chapter presented a Wizard-Of-Oz style evaluation of the concepts. This restriction was necessary in order to focus the study on evaluating the concepts presented in this chapter. Furthermore, this design decision eliminated the influences of uncontrollable variables (e.g. the success rate of speech recognition). However, due to the Wizard-Of-Oz style of the study, it was necessary to closely confine the scope of the study in terms of scenario and relationship. A large-scale study in the wild might therefore yield further insights into the problem area.

\subsection{Interaction Concepts for Larger Groups}

This chapter presented concepts and an evaluation of interaction techniques for groups of two conversation partners. While some of the concepts are directly transferable to larger groups (e.g, \emph{Grab\&Show}, some of the concepts are geared to the interaction between two persons. For example, the \emph{Grab\&Share} gesture employs the metaphor of handing over an object to a person. For larger groups, sharing must at least distinguish between sharing with one user and making information generally accessible to all users. Further, in the current concept, the \emph{Grab\&Show} gesture opens information drops for all participating users. For larger groups, this might worsen the overview. Therefore, an additional technique for privately opening information drops might be beneficial. In summary, future work is necessary to focus on concepts beyond the 1-on-1 collaboration scenarios addressed in this chapter.

\subsection{Interaction in the 3D Space}

The presented study mainly focused on the evaluation of the interaction techniques introduced earlier in this chapter. However, the analysis of the observations and the qualitative feedback of the participants found many interesting aspects regarding the use of a shared 3D space. Hereby, this chapter offers interesting starting points for a deeper and more focused look on how people interact in a shared 3D information space to arrange themselves and information spatially. In addition, further work is needed to understand the impact of such working practices on users' accuracy and efficiency.

\subsection{CloudBits Interaction Beyond the Specific Use Case}
\label{sec:cloudbits:limitations:usecase}

This section covered the presented interaction techniques in the context of the specific use case of information retrieval in co-located conversation scenarios. However, we are convinced that the CloudBits interaction techniques can also be used in other areas. For example, CloudBits can also be beneficial for individual users: Such a system could visualize notifications for the user. Depending on the urgency of the notification, color and size, as well as spawn position (between peripheral area and directly in the line of sight) could be varied. Future work is needed on how to adopt the interaction techniques to the requirements of individual users.