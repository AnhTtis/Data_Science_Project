The previous chapter explored how the degrees of freedom offered by the \emph{upper limbs} and, in particular, the elbow joint can be used for interactions with a \emph{body-stabilized} interface. The presented interaction technique registers the visualization to the body (in this specific case, the palm) of the user and, thus, enables the interface to be available anytime and anywhere for fast and immediate interactions.

This chapter shifts the stabilization point of the visualization from the user's body to the real world, focusing on interaction techniques for \emph{world-stabilized} interfaces (see section \ref{sec:relatedwork:hmds:implementation}) which are still operated using the \emph{upper limbs} (see figure \ref{fig:cloudbits/overview_cloudbits}). Since information is no longer bound to the position and movements of the user (or specific body-parts), the information can be freely positioned in space, and the user can view it from different perspectives by moving around. This positioning enables new use cases for \inplaceLower{} scenarios that give meaning to the spatial location of information: Like documents in the real world, users can sort and group information and use the spatial layout to add meta-information to the actual information implicitly. The \emph{world-stabilization} is particularly relevant for \multiuserLower{} scenarios if the information is anchored to the same real-world position for all participating users, allowing collaborative interaction with information. To gain insights into this problem domain, this chapter focuses on co-located conversation and meeting scenarios as an exemplary application domain, where users frequently interact with public and private information. Despite the focus on this application domain, the resulting interaction technique can as well as the insights gained can also be transferred to other domains, as highlighted in section \ref{sec:cloudbits:limitations:usecase}.

This chapter 1) contributes the results of an exploratory study investigating the requirements for the design of a user interface to support the interaction with information during conversations. Based on the results of the study, this chapter 2) presents an interaction technique for collaborative interaction with \emph{world-stabilized} interfaces using the \emph{upper limbs}, along with a prototype implementation. Finally, this chapter 3) reports the findings of a qualitative evaluation of the interaction techniques and concludes with guidelines for the design of such user interfaces.

The remainder of this chapter is structured as follows: First, this section presents a review of the related works in the area of \acp{HMD} in social interactions (section \ref{sec:cloudbits:relatedwork:hmd}). Second, based on a focus-group study to establish requirements and the review of related works, the chapter introduces an interaction technique to support collaborative interactions with information (section \ref{sec:cloudbits:introduction}). In the following, section \ref{sec:cloudbits:study2} presents the methodology and section \ref{sec:cloudbits:study2results} reports the results of the lab study on the behavior of users when interacting in a shared information space. Based on the results, section \ref{sec:cloudbits:guidelines} provides guidelines for the future use of such interfaces. The chapter concludes with a discussion of limitations and with guidelines for the future use of such interfaces (section \ref{sec:cloudbits:limitations}).

\newpage

\cptIntroBox{muller2017cloudbits}{The student \emph{Azita Hosseini Nejad} implemented the study client application and supported the conduct of the study. \emph{Sebastian Günther}, \emph{Niloofar Dezfuli}, \emph{Mohammadreza Khalilbeigi} and \emph{Max Mühlhäuser} supported the conceptual design and contributed to the writing process.}