\section{Evaluation}
\label{sec:cloudbits:study2}

This section presents the methodology of a laboratory study to investigate if and how CloudBits supports co-located conversation scenarios. In particular, the experiment focused on if and how\ldots

\begin{enumerate}
	\item \ldots users leverage the surrounding space for acquiring and interacting with information,
	\item \ldots CloudBits provides mutual awareness of activities and eases the (re)-engagement into the conversation and
	\item \ldots CloudBits enables selective sharing from the private-public information spectrum.
\end{enumerate}

For this, 12 participants (P1-P12: seven female, aged between 25 and 35 years) were recruited in six groups of two persons each. The two-person pairs knew each other before in order to stimulate conversation and collaboration similar to real-life situations. The recruitment included pairs of persons in various relationships: work colleagues, close friends, and spouses. None of them had prior experience with augmented reality glasses. No compensation was provided.

\subsection{Design and Task}

Inspired by the study design presented by~\typcite{Lissermann2014}, the overarching goal for the participants in the study was to plan a vacation trip collaboratively. The design of the scenario was chosen to require participants to search, explore, and share both private and public information with their partners and individually. 

The study tested CloudBits and information retrieval via smartphones as two conditions in a within-subjects design. In both conditions, the participants received a destination name, the available budget, and a list of tasks. Participants received all tasks upfront, and they were free to choose the order of processing. The investigator asked the participants to note down their decisions on a provided paper. The four tasks were:

\begin{description}
	\item[Task 1] required participants to agree on the departure date and length of the trip by reviewing their personal calendars and finding possible time slots.
	\item[Task 2] required participants to agree on a flight for their trip based on the selected dates and the price. Therefore, participants had to check the offers they personally received from their travel agencies via e-mail.
	\item[Task 3] required participants to select a hotel based on their budget, the location of the hotel, and online reviews. Furthermore, both participants received personal e-mails with suggestions from friends who traveled to the respective destination before. 
	\item[Task 4] required participants to select a restaurant for the first evening based on location, reviews, and the type of food served.  
\end{description}

To understand if and how CloudBits supports (re)-engagement in the conversation, the investigator induced an (for the participants) unanticipated break in both condition to interrupt the discussions and distract the participants from their current tasks. These interruptions were realized through faked technical problems. After five minutes, the investigator pretended to have fixed the problems and asked the participants to continue from where they left off.

\subsection{Study Setup and Apparatus}

\textfig{cloudbits/study_setting}{The setting of the study. Participants were free to choose a spatial arrangement for both, the CloudBits (a) and the smartphone (b) condition.}

The study setup used the prototype application, as presented in section \ref{sec:cloudbits:prototype} deployed to two Microsoft HoloLens devices. For the smartphone condition, the study used two Google/LG Nexus 5X devices. 

The study was conducted in Wizard-of-Oz style to have full control on when and what the participants saw during the study and to eliminate system errors caused by problems in the speech recognition or language processing as these parts were not the focus of the evaluation. Therefore, a wizard application was implemented that allowed the investigator to prepare information drops upfront and to send them on demand to the individual participants (see figure \ref{fig:cloudbits/fig_card_study}).

Similarly, the smartphones used by the participants during the study were prepared with the content (such as e-mails and calendar entries) that they needed to complete the tasks. 

The sessions were videotaped with an external camera and, for the CloudBits condition, the personal views of the participants were recorded through the HoloLens \enquote{Mixed-Reality Capture}\footnote{\url{https://docs.microsoft.com/en-US/windows/mixed-reality/mixed-reality-capture}, [last downloaded 10.07.2019]}. This allowed recording the participants view into the real world together with the augmented information drops. The study concluded with a semi-structured interview for each participant pair. The recorded data from the study was analyzed using an open coding~\ncite{Strauss1998} approach. 

\subsection{Procedure}

\textfig[.7]{cloudbits/fig_card_study}{Content of an information drop as used in the evaluation: A hotel in New York City.}

The order of the two conditions was counterbalanced by randomly assigning the starting condition to the participant pairs. Further, the destination and date of the task were changed between both conditions to avoid learning effects. 

After welcoming the participants, the investigator introduced the participants to the setup of the study and gave them 15 minutes to acclimatize to the Hololens and its general interaction and visualization techniques.

For the smartphone condition, the investigator handed them the two prepared smartphones and informed them about their task. For the CloudBits condition, the investigator observed the conversation of the two participants and sent them appropriate information drops during the study. 

After both conditions, participants took a five-minute break. The experiment concluded with a semi-structured interview focusing on the participants' overall opinion about the CloudBits concept and the differences between the tested conditions. The experiment took 180 minutes per participant pair.

\textfigStudybox{cloudbits/studybox_cloudbits}