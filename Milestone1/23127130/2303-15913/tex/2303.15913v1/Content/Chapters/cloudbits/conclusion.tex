\newpage
\section{Conclusion}
\label{sec:cludbits:conclusion}

This chapter explored how multiple users interact with information in a shared information space and presented CloudBits together with its prototype implementation. We evaluated our concepts in a qualitative lab study and presented guidelines for the design of user interfaces for in-situ collaboration in a shared information space. The results of the controlled experiment confirmed the feasibility of the concept. 

To conclude, this chapter added to the body of research on interacting with \acp{HMD} in multiple areas:

\begin{enumerate}
	\item This chapter demonstrated how information could be visualized time-dependently based on the current information needs of users: Therefore, relevant information appears unsolicited in the peripheral view of the user. As time passes, the information drops fall down until they finally disappear, playfully visualizing the transience of information. While this visualization was used in this chapter to provide additional information during conversations, there are many more areas of application where such an unobtrusive time-dependent visualization of information could be of use (e.g., incoming e-mails or calendar appointments). Therefore, this work has paved the way for further research investigating the viability of such interfaces in other scenarios.
	\item In addition, this chapter contributed interaction techniques for collaborative interaction in a shared information space. Therefore, this work added to the body of research on interacting with \acp{HMD} by providing interaction techniques that break through the isolation of the inherently single-user focused device class. This work can thus serve as a foundation for further contributions in this area.
	\item Finally, this work contributes to the area of conversation support by unobtrusively providing additional information based on the topics of the conversation. The evaluation of the concept showed great advantages over state-of-the-art information retrieval during conversations. Thus this work opened a promising way to support future (co-located and remote) conversations.
	
\end{enumerate}

\subsection{Integration}

\textfigH{cloudbits/alice}{Alice and Bob share memories during a conversation. Cloudbits provides them with retrieved digital information.}

\interactionbox{alice_cloudbits}{In the Coffee Shop}{Alice meets her friend Bob for a coffee. The two talk about their shared memories of the past year. During the conversation CloudBits automatically retrieves images of the two from their respective photo albums, which the two subsequently share with each other (see figure \ref{fig:cloudbits/alice}). The two of them come to the conclusion that you should once again do something together. During their discussion about different possibilities, CloudBits automatically displays appropriate information from the Internet and displays the respective calendars privately to help them schedule the appointment.
	
	Over the course of the conversation, Alice and Bob interact quickly and easily with the information, grouping and sharing it through natural hand movements. Through the shared information space, the other person understands what the partner is doing.}

The interaction technique presented in this chapter allows for interactions with \emph{world-stabilzed} interfaces leveraging our \emph{upper limbs} for input. The interaction technique uses the world-stabilization to provide a shared information space between users, supporting \continuousLower{} in co-located \inplaceLower{} \multiuserLower{} scenarios. This contribution is particularly important due to the natural focus of \acp{HMD} on single user interactions, which results from the limited visibility of displayed information restricted to the wearing user (see section \ref{sec:relatedwork:hmds}). Thus, the contribution extends the applicability of \aroundbodyinteraction{} (see section \ref{sec:introduction:aroundbodyinteraction}) to multi-user situations and contributes to the overall vision.

In this chapter, the presented interaction techniques were investigated based on the specific application area of information retrieval and interaction with information in a co-located two-person meeting. While this restriction to the specific use case was necessary to the scope of this work, other use cases may yield different results. As discussed in section \ref{sec:cloudbits:limitations}, further modifications to the design (e.g., by supporting more than two users) are necessary for future interaction with \acp{HMD}. The ideas and qualitative results presented in this chapter can serve as a reliable baseline for further explorations of the respective design spaces.

\subsection{Outlook}

This chapter, together with chapter \ref{ch:proximity:merged}, investigated interaction techniques with the \emph{upper limbs}. However, there are situations in which interaction with the upper limbs is not desired (e.g., while using the hands to interact with the real world) or even physically possible (e.g., while carrying something in both hands). Chapters \ref{ch:cheesyfoot} and \ref{ch:walktheline} therefore present hands-free interaction techniques that draw on our \emph{lower limbs}.