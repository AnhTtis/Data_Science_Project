\subsection{Exploratory Study}
\label{sec:cloudbits:study1}

This section presents the design and results of an exploratory study to gain insights into the design of a user interface to support conversations and to establish requirements for such a system. More specifically, the study investigated the following research questions:

\begin{description}
	\item[RQ1] How can a system effectively support information retrieval in co-located conversations?
	\item[RQ2] What are the requirements for the user interface of such a system?
\end{description}

For the study, 7 participants (4 male, 3 female, 30 years on average) were invited for individual semi-structured interview sessions. No compensation was provided.

\subsubsection{Design and Procedure}

As a starting point for the brainstorming sessions, five different conversation scenarios (S1-S5) were defined. Based on~\ncite{Pask1976, Dubberly2009}, the five scenarios were designed to include a wide variety of circumstances of conversations in terms of (1) \emph{location}, (2) \emph{objectives} and (3) \emph{mood} of the participants as well as different (4) \emph{relationships}.

\begin{description}
	\item[S1: Consultation] A conversation between persons with different levels of information and understanding of a problem space, e.g., a medical consultation.
	\item[S2: Meeting] A conversation between peers with the same level of information, e.g., a meeting between coworkers.
	\item[S3: Authority Gradient] A conversation between persons with different levels of information and an authority gradient, e.g., a trainer teaching a trainee.
	\item[S4: Informal Talk] A conversation between peers in an informal setting, e.g., friends at a bar.
	\item[S5: Different Intentions] A conversation between persons with different intentions, e.g., a sales meeting with an estate agent.
\end{description}

Remembering special experiences (both positive and negative) is easier than remembering ordinary experiences.~\ncite{Sharp2007}. Therefore, the investigator asked participants about their positive and negative experiences with information retrieval in the respective scenarios, focusing on problems with the current systems. If participants did not have specific experiences in the respective scenario, the scenario was skipped. The study lasted around two hours per single-user session. For data gathering, the sessions were recorded on video.

\subsection{Results and Requirements}
\label{sec:cloudbits:study1:results}

The recorded sessions were analyzed using an open coding approach. The coders selected salient quotes for further analysis. The following section presents the results of our study with respect to the research questions.

In general, all participants stated that they currently use mobile information retrieval in conversation scenarios. When asked about the kind of retrieved information, participants stated that they primarily looked up unknown terms or abbreviations, factual information from public sources and personal information such as appointments or e-mails. The comments of the participants showed clusters in three areas, which led to the identification of three main requirements for the design of a user interface to support information retrieval in conversations.

\subsubsection{R4.1: Unsolicited and Real-Time Service}
In the study, participants stated that the shame of nescience is one of the major reasons for information retrieval using personal devices in all of the discussed situations. This includes not only formal situations but also informal talk with friends. P4 said: \enquote{If I think that it’s too easy or I don’t listen to something, I won’t ask anybody because it’s embarrassing}, P7 added: \enquote{I don’t ask other people because of shyness}. As another reason, participants remembered multiple situations in which fast and immediate retrieval of relevant information was necessary for the continuation of the conversation. Participants stated that breaks during the conversation, caused by the necessity for information retrieval, were \enquote{really upsetting} (P4). Additionally, the interviews showed that information should stay available for immediate re-retrieval as the same information might be needed again within short time frames. 

To support the presented situations, a system should provide direct and unsolicited service to all participants without the need to explicitly ask for information. The information should be available in real-time (i.e., available at the right moment) and time-varying (i.e., available as long as needed) fashion.

\subsubsection{R4.2: Supporting Fluid Transition and Re-Engagement} 

We found that participants have the feeling that they spend a significant amount of time for information retrieval in conversations which \enquote{leads to missing other parts} (P2) of the conversation. This even led participants to refrain from searching (P2, P5) in multiple situations. Participants felt that the time spent on the mobile device caused them to \enquote{lose connection} (P3) to the actual conversation because their focus shifted towards the interaction with the device and the retrieved information. Even more, participants felt \enquote{let down} (P5) when the other person in a conversation focused on their mobile device. Participants named other instances (such as having to leave to room) that caused them to lose the connection to the topics of the conversation and, thus, forced an immediate re-engagement process after returning to the conversation. 

Therefore, a system should provide a means for a fast and smooth transition between information retrieval and the actual conversation to prevent users from losing the connection. In the case of inevitable disruptions, the system should support the user in the re-engagement process. As a further consequence, systems should avoid being a source of distraction from conversation through their visualization.

\subsubsection{R4.3: Selective Sharing from the Public-Private Information Spectrum}

In the analysis, we found the sharing of the retrieved information with other participants of the conversation to be cumbersome. The retrieved information is only available on the personal device of the retrieving user and, thus, shared through sharing the complete device by handing the mobile phone to someone. Participants felt \enquote{uncomfortable} (P3) doing this, not only in formal but also in more intimate situations. Besides privacy issues, participants recalled multiple situations (particular regarding S1 and S5) where this turned out to be frustrating for users because of the limited screen space.

Thus, a system should support 1) selective sharing of specific contents and 2) collaborative interaction with information in a large shared information space. 