\section{Contributions}
\label{sec:conclusion:summary}

\subsection{Upper Limbs}

The first two main contributions of this thesis focused on interaction techniques for the \emph{upper limbs} to operate \emph{body-stabilized} and \emph{world-stabilized} interfaces, respectively. 

\subsubsection{Body-Stabilized Interfaces}

First, \projProximity explored a one-handed interaction technique for \emph{body-stabilized} interfaces to be operated using the \emph{upper limbs}. The chapter presented an interaction technique that leverages the degree of freedom of the elbow joint to allow users to explore a one-dimensional interaction space along the line of sight. By flexing or extending the arm, the user can browse through successive layers. The visual content of the layers is anchored to the user's hand, enabling quick and immediate access to information, anytime and anywhere. The approach was evaluated in a controlled experiment, which showed short interaction times and high accuracy rates. 

As illustrated in the chapter, the support for both, \discreteLower{} and \continuousLower{}, enables the usage of this technique for a wide range of \singleuserLower{} applications. The contribution focuses entirely on one-handed interaction and, thus, considerably increases the \mobilityLower{} of fine-grained mobile interaction with digitally augmented information, as the second hand is still available for interaction with the real world. Therefore, the presented interaction technique contributes to the vision of \aroundbodyinteraction{} by providing fast and accurate one-handed interactions for \acp{HMD}.

\subsubsection{World-Stabilized Interfaces}

Second, \projCloudbits shifted the focus from body-stabilized to \emph{world-stabilized} interfaces, where information is not anchored to the user, but the environment. The chapter presented interaction techniques that leverage this world-stabilization of interfaces to support collaborative use cases where multiple users can exploit the spatial layout of information for collaborative access and manipulation.

The interaction techniques support \continuousLower{} in \inplaceLower{} and \multiuserLower{} scenarios. As shown in the evaluation, the contributed techniques alleviate the burdens connected to the retrieval and ease the interaction with information in conversation scenarios. Therefore, the presented interaction techniques contribute to the vision of \aroundbodyinteraction{} by providing by allowing users to interact with, sort, and share information easily and intuitively using the degrees of freedom offered by the joints of their body.

\subsection{Lower Limbs}

The third and the fourth main contributions of this thesis focused on interaction techniques for the \emph{lower limbs} to operate \emph{body-stabilized} and \emph{world-stabilized} interfaces, respectively. 

\subsubsection{Body-Stabilized Interfaces}

Third, \projCheesyfoot assessed the viability of foot-tapping as an interaction technique for direct and indirect interaction with \acp{HMD}. This technique visualizes a semicircular grid on the ground (directly) or in the air in front of the user (indirectly). In both cases, the user can select input options using foot taps. The analysis showed promising results that indicated foot-tapping as a viable interaction technique for hands-free interaction with \acp{HMD}.

Therefore, the presented interaction technique allows \singleuserLower{} to perform \discreteLower{} in situations, where the hands are not available, supporting \mobilityLower{}. Depending on the requirements of the application, the two interaction styles presented support both highly accurate (direct) as well as fast and casual interactions (indirect), where the user does not need to lower his head. Therefore, the presented interaction technique contributes to the vision of \aroundbodyinteraction{} by providing an alternative input modality for situations where the users' hands are busy interacting with the real world.

\subsubsection{World-Stabilized Interfaces}

Forth, \projCheesyfootToGo focused on foot-based interaction while walking and presented an approach to leverage locomotion as an additional input modality. The approach augments world-anchored lanes parallel to the walking path of the user to the ground. Each of the lanes represents an input option. Through lateral displacements of the user's walking path, the user can walk on a line to select it. The analysis proved this to be a promising approach for hands-free interaction while walking without detracting the user's visual attention.

Therefore, the presented interaction technique supports \singleuserLower{}s during \mobilityLower{} situations and provides \discreteLower{}. This interaction technique is of particular importance for situations in which the other techniques presented in this paper are not available to the user due to the situational impairments (e.g., walking with things carried in hand). Therefore, the presented interaction technique contributes to the vision of \aroundbodyinteraction{} by providing a highly mobile interaction technique that allows for hands-free interaction during locomotion.