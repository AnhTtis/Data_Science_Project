This thesis argued for \aroundbodyinteraction{} (see section \ref{sec:introduction:aroundbodyinteraction}) leveraging 1) the \emph{upper} and \emph{lower} limbs to interact with 2) \emph{body}- and \emph{world-stabilized} interfaces. For each combination in this design space, this thesis presented a suitable interaction technique based on the individual requirements of the combinations and evaluated them in terms of efficiency and accuracy.

In this work, a concrete interaction technique was derived from the respective general requirements for each quadrant of the presented design space. These interaction techniques do not represent the only possible interaction techniques, and other approaches may yield different results depending on the situation. However, we are convinced that - while each of the presented interaction techniques is tailored towards a specific setting - the combination of these interaction techniques already supports a wide range of situations encountered during everyday usage of such devices.

This chapter shortly summarizes the main contributions and, based on the presented interaction techniques, outlines a vision for an integrated concept for joyful and efficient mobile interaction with \acp{HMD} using \aroundbodyinteraction{} (section \ref{sec:conclusion:summary}). Further, section \ref{sec:conclusion:futurework} highlights open questions and gives directions for future research.