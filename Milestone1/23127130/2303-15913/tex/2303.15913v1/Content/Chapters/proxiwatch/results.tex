\subsection{Results}
\label{sec:proximitywatch:results}

For the analysis, the recorded data was normalized to the respective arm length of the participant into a scale from $0\ldots1$. In this scale, $0$ refers to the near point of the human eye (\SI{12.6}{cm}) and $1$ to the arm length of the participant from shoulder to wrist. This maximum arm reach was measured in the calibration process of the system with the same optical tracking system.  The analysis of the data was performed as described in section \ref{sec:rw:methodology}. 

\subsubsection{Task Completion Time (TCT)}

The analysis showed a significant (\anovaCorWithoutEffect{2.92}{40.85}{5.17}{<.01}{.486}) influence of the number of layers on the \ac{TCT}. Post-hoc tests confirmed significantly rising \acp{TCT} for higher numbers of layers when comparing 2 to 4 and 6 (both $p<.05$), 7 ($p<.01$) and 8 ($p<.001$) layers. Further, the analysis showed a significant effect between 3 and 8 layers ($p<.05$). Table \ref{tab:proxiwatch/tct_post_hoc} lists the mean differences for all post-hoc comparisons together with the significant conditions. The analysis showed generally small \acp{TCT}, ranging from \valSi{1.7}{.14}{s} for 2 target layers to \valSi{2.37}{.14}{s} for 8 target layers. Table \ref{tab:proxiwatch/tct_mean} lists all mean \acp{TCT} for the conditions.


\begin{table}[t!]
	\centering
	
	\begin{tabularx}{\linewidth}{YYYY}
		
		\multicolumn{2}{c}{\textbf{Comparison}} & & \\
		\cmidrule(lr){1-2}
		\textbf{Layers} & \textbf{Layers} & $\pmb{\Delta	\mu}$ & \textbf{sig} \\
		\midrule
		2 & 4 & -.42s & *\\
		2 & 6 & -.46s & *\\
		2 & 7 & -.59s & **\\
		2 & 8 & -.64s & ***\\
		3 & 8 & -.44s & *\\
		\bottomrule
	\end{tabularx}
	
	\caption{The post-hoc tests for the mean \acp{TCT} between the tested numbers of layers. Only the significant comparisons are listed. \textbf{sig} denotes the significance level: * $p<.05$, ** $p<.01$, *** $p<.001$}
	\label{tab:proxiwatch/tct_post_hoc}
\end{table}

\begin{table}[b!]
	\centering

\begin{tabularx}{\linewidth}{YYYYY}
	&   &   & \multicolumn{2}{c}{\textbf{95\% Confidence Interval}}\\
	\cmidrule(lr){4-5}
	\textbf{Nr of Layers} & $\pmb{\mu}$ & $\pmb{\sigma_{\overline{x}}}$ & \textbf{Lower} & \textbf{Upper}\\
	\midrule
	2    & 1.73s & .14s     & 1.44s                     & 2.02s        				\\
	3    & 1.93s & .14s     & 1.64s                     & 2.22s                     \\
	4    & 2.15s & .14s     & 1.86s                     & 2.44s         			\\
	5    & 2.07s & .14s     & 1.78s                     & 2.36s                     \\
	6    & 2.20s & .14s     & 1.91s                     & 2.48s                     \\
	7    & 2.32s & .14s     & 2.03s                     & 2.61s                     \\
	8    & 2.37s & .14s     & 2.08s                     & 2.66s						\\ 
	\bottomrule
\end{tabularx}

	\caption{The Task Completion Times (TCT, in seconds) for different numbers of layers. The table reports the mean value $\mu$, the standard error $\sigma_{\overline{x}}$ and the 95\% confidence interval.}
	\label{tab:proxiwatch/tct_mean}
\end{table}



\subsubsection{Personal Interaction Space}

\textfigH{proxiwatch/interactionSpace}{The interaction space of the participants. Black dots show the recorded distances normalized to the arm size, and red dots show the center point of the interaction space. The used space differs significantly between participants.}

The investigator told the participants to use the area and to separate the interaction space into layers in a way that is convenient for them. The analysis showed that the interaction space used by the participants as well as the center point of all interactions differs significantly (\anovaWithoutEffect{14}{84}{8.6}{<.001}) between participants (Min: P14, $.0-.51$ with one outlier, center point of interaction $\mu=.24$, Max: P4, $.0-.98$, center point of interaction $\mu=.54$). Figure~\ref{fig:proxiwatch/interactionSpace} shows the interaction space of all participants normalized to their arm length. This personal interaction space for each participant remained constant for different layer subdivisions. For all further evaluations, the data was scaled based on the personal interaction space of each participant with 0 as the closest and 1 the most distant data point.

\subsubsection{Directly Accessible Layers}

\textfigH{proxiwatch/targetZones4}{The distances for three exemplary participants for layer subdivision $n=4$, scaled to their personal interaction space. Finding: A global model to classify points to a target layer is not possible. However, individual models per user to classify point with regards to the target layer seems feasible.}

The analysis showed that the size and the location of the center points of the layers differ significantly (Size: \anovaWithoutEffect{14}{98}{2.6}{<.01}, Location: \anovaWithoutEffect{14}{98}{6.9}{<.001}) between participants even after scaling the data to the personal interaction space of each participant. A generalized model that is able to map points from every user into the respective target layer is, therefore, not feasible for layer subdivisions $>2$, as the layers largely overlap between participants. Within the data of individual participants, however, a more fine-grained differentiation between the layers can be archived with no overlapping layers for a subdivision of at least 4 for all participants. As an example of this finding, figure~\ref{fig:proxiwatch/targetZones4} shows the data points for three participants for a subdivision of four layers. The analysis further showed smaller layers for the outer regions (i.e.,\ close to and far away from the body) compared to the inner regions (Inner: $\mu=.16,~\sigma=.06$, Outer: $\mu=.12,~\sigma=.07$). This is not influenced by the personal interaction spaces of the participants.