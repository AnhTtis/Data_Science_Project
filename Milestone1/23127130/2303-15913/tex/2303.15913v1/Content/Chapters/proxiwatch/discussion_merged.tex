\section{Discussion and Guidelines}
\label{sec:proximity:discussion}

The quantitative, as well as the qualitative results of both experiments, indicated that one-handed proximity-based interactions could be a viable interaction technique for \acp{HMD} that can support convenient and fun interactions for both, discrete and continuous interactions. Based on the results of the experiments and the related works, this section proposes three guidelines for the future design of such interfaces.

\subsection{Partition the Space by Layer Thickness}
\label{sec:proximity:discussion:guideline1}


\begin{figure}[h!]
	\centering
	\includegraphics[width=\textwidth]{Content/Figures/proximity/guideline4}
	\caption{\emph{Design Guideline: Partition the space by layer thickness} The interaction space should be designed based on the absolute layer thickness, not on a desired number of layers to account for different arm sizes.}
	\label{fig:proximity:guideline:1}
\end{figure}

The results indicate that various sizes of participants' arms do not influence the accuracy - measured as an error of absolute distance - of hand movement. Therefore, for users with smaller arms, too many and, thus, thin layers would decrease the accuracy. On the other hand, for taller users with greater arm length, insufficient numbers of layers would result in greater traveling distances and, therefore, decreased efficiency. 

Hence, as a result of the experiment, the interaction space should be designed based on the layers specific thickness. This way, the design results in different numbers of layers for different arm sizes, allowing the user to interact within the borders of their physical abilities (see figure \ref{fig:proximity:guideline:1}).

\subsection{Use an Uneven Layer Thickness}
\label{sec:proximity:discussion:guideline2} 


\begin{figure}[h!]
	\centering
	\includegraphics[width=\textwidth]{Content/Figures/proximity/guideline1}
	\caption{\emph{Design Guideline: Use an uneven layer thickness} The interaction space should be designed with a decreasing layer size towards the outer layers to account for the higher overshooting errors found in the experiment.}
	\label{fig:proximity:guideline:2}
\end{figure}

The traveling distance of the hand proved to be the most critical factor to influence the depended values in both experiments. The first experiment focusing on continuous interactions showed that the typical overshooting error decreases towards outer regions. Furthermore, the second experiment indicated that the outer layers of the mental model of participants were smaller compared to the inner layers. Therefore, this chapter proposes to use uneven and descending layer thicknesses towards outer regions (see figure \ref{fig:proximity:guideline:2}).

This layer subdivision allows for smaller layers in outer regions without increasing the interaction time that is introduced due to overshooting the target. Based on the quantitative results, a layer thickness of 7.8 cm for near, 4.2 cm for medium and 3.0 cm for far targets (the respective mean overshoot plus the double standard deviation) would result in $>$95\% accuracy for all traveling distances.

\subsection{Respect the Personal Interaction Space}
\label{sec:proximity:discussion:guideline3}

\begin{figure}[h!]
	\centering
	\includegraphics[width=\textwidth]{Content/Figures/proximity/guideline3}
	\caption{\emph{Design Guideline: Respect the Personal Interaction Space} The experiment showed differing convenient interaction spaces for each participant. This should be reflected by a system implementation.}
	\label{fig:proximity:guideline:3}
\end{figure}

The experiment showed that different users have different personal convenient interaction spaces. These convenient interaction spaces are not generalizable over multiple users. Thus, a system should not force the user into a fixed set of layers that spans larger or smaller than the user's personal interaction space (see figure \ref{fig:proximity:guideline:3}).

Therefore, a general model over all users is not feasible and, thus, a personal model is necessary to achieve high recognition rates for higher subdivisions than two.

\subsection{Focus on the Convenient Range} 
\label{sec:proximity:discussion:guideline4}

\begin{figure}[h!]
	\centering
	\includegraphics[width=\textwidth]{Content/Figures/proximity/guideline2}
	\caption{\emph{Design Guideline: Focus on the Convenient Range} Use the near and medium layers for frequent and common interactions, use the far layers for irreversible interactions.}
	\label{fig:proximity:guideline:4}
\end{figure}

The qualitative feedback from participants showed that interactions in the far zones are less convenient compared to the closer regions. Therefore, we propose to focus on the near and medium zones for frequent and common interactions. (see figure \ref{fig:proximity:guideline:4}) 

As proposed by \typcite{Benford2012}, the slightly uncomfortable hand position in the far zones can be leveraged for important and not reversible actions such as deleting a file or sending an e-mail.