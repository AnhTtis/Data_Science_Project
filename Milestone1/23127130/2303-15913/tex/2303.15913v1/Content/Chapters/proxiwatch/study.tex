\section{Experiment II: Discrete Interaction}
\label{sec:proximity:exp2}

The following section presents the methodology (see section \ref{sec:proximitywatch:methodology}) and the results (see section \ref{sec:proximitywatch:results}) of a controlled experiment investigating discrete interactions in a multi-layered interaction space.

\subsection{Methodology}
\label{sec:proximitywatch:methodology}

\textfigH{proxiwatch/setup}{The setup of the controlled experiment with two retro-reflective apparatuses mounted on the participant’s head and wrist, and the display showing the current task.}

This section presents the methodology of the second controlled experiment focusing on discrete interactions for proprioceptive interactions without visual feedback. More specifically, the second experiment addressed the following research questions:

\begin{enumerate}
	\item[RQ4] How accurate and efficient users can raise the hand to a given target position in the space in front of them without any visual feedback on their performance?
	\item[RQ5] Where are the targets located in the participants’ mental model?
\end{enumerate}

For this, 15 participants (5 female, 2 left-handed), aged between 19 and 30 years, were recruited. No compensation was provided.

\subsubsection{Design and Task}
The design of the experiment was similar to the design used in the first experiment. Again, the experiment defined a basic information space alongside the participants’ line of sight, evenly split into multiple layers and numbered in ascending order. The participants’ task was to raise their arm at a specified target layer without any visual feedback. The conditions varied the number of layers as an independent variable with integer values from 2 to 8. As the results from the first experiment did not show an effect of the hand side on any of the dependent variables, the second experiment disregarded the hand side as an independent variable. As in the first experiment, the system defined the maximum boundary of the interaction space as the participant’s individual arm-length and the minimum boundary as the near point of the human’s eye (not closer than 12.5cm to the user’s face). However, the investigator told the participants to use the space that is most comfortable for them as an interaction space.

The experiment used a repeated measure design with 7 levels for the number of layers (2, 3, \ldots, 7, and 8). For each level, the participants targeted each layer with 5 repetitions. This resulted in a total of $(2+3+4+5+6+7+8) * 5 = 175$ trials per participant. The order of conditions, as well as the order of targets within each condition, was counterbalanced using a Balanced Latin Square design (see section \ref{sec:rw:methodology}).

\subsubsection{Experiment Setup}

The system used an optical tracking system (OptiTrack) to measure the distance alongside the line of sight between the participant’s wrist and eyes. Participants wore a wristband on their non-dominant hand and a pair of glasses, each augmented with a set of retro-reflective markers, during the experiment (\reffig{fig:proxiwatch/setup}). A display in front of the participants showed the current task (layer subdivision and target layer within this subdivision). Additionally, a button was mounted within reach of the participant’s dominant hand. For each trial, the system recorded 

\begin{enumerate}
	\item the \emph{distance} between wrist and eyes after completing the task.
	\item the \emph{task completion time (TCT)} as the timespan between starting the trial until pressing the confirmation button.
	\item the \emph{target layer} of the condition.
	\item the \emph{total number of layers} in the current condition.
\end{enumerate}

\subsubsection{Procedure}

After welcoming the participants, the investigator introduced them to the concept and the setup of the experiment and asked them to put on the two trackable apparatuses. Then, the system was calibrated to adapt it to the respective arm length. Before each condition, the system informed the participants about the layer subdivision for this task. Each trial was started by asking the participant to stand relaxed and lower the non-dominant arm. Once ready, the participant pressed the button to start the trial.

After that, the system showed the target layer as a number from 1 (nearest layer to the body) to the highest layer of the current condition (2-8). Then, the participants raised their hand at the position where they imagined the respective layer. The investigator told the participants to look at the center of the trackable apparatus on their wrist in order to to keep the measured distances comparable. After raising their hand, participants had to confirm their action by pressing the nearby mounted button with their non-interacting hand. In the following, the system asked the user to take their hand down and enforced a 5-second break before starting the next trial.

The investigator told the participants to focus on the accuracy instead of the speed. Participants did not receive any feedback during the experiment. After each condition, participants took a 30-second break. The complete experiment took about 30 minutes for each participant.

\textfigStudybox{proxiwatch/studybox_proximity2}