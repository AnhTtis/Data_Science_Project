\subsection{Research Methodology}
\label{sec:rw:methodology}

The contributions of this thesis are situated in the field of \acp{HCI}, a field \enquote{concerned with the design, evaluation and implementation of interactive computing systems for human use and with the study of major phenomena surrounding them}~\ncite{Hewett2014}. Each of the main contributions presented in this thesis is substantiated in one or more observational studies or controlled experiments. All of the studies and experiments were designed, conducted, and analyzed according to widely accepted standards of the \ac{HCI} community~\ncite{Lazar2010}. This section describes the general approach to the design and analysis, which applies to all studies presented. Any exceptions are mentioned and justified in the respective chapters.

\subsubsection{Study Design}

In this thesis, two general types of study designs are applied, controlled experiments and observational studies.

\paragraph{Controlled Experiment}

Controlled experiments are used to gain quantitative insights into a domain while excluding external factors as far as possible. These experiments, depending on the experiment design, vary one or more independent variables to assess their influence on measured dependent variables. In order to avoid first order carryover effects (e.g., learning effects between the conditions), the experiments vary the sequence of conditions between participants according to a balanced Latin square as proposed by \typcite{Williams1949}. As dependent variables, performance metrics (e.g., \acf{TCT}, accuracy) accepted and widely used by the \ac{HCI} community were recorded. The respective methodology sections name and define them with regards to the presented experiment. In addition to the performance metrics recorded by tracking the users, the experiment designs use standardized and accepted questionnaires to collect further data:

\begin{description}
	\item[NASA TLX] as proposed by \typcite{Hart1988} to quantify the perceived mental load of participants. For the analysis of the NASA TLX questionnaires, the raw method indicating an overall workload as described by \typcite{Hart2006} is used.
	\item[AttrakDiff] as proposed by~\typcite{Hassenzahl2003} to quantify the participants’ opinions about the user experience of concepts.
	\item[Custom Questionnaires in Likert Scales] as proposed by~\typcite{likert1932technique} to assess participants’ attitudes towards certain aspects of the proposed concepts.
\end{description}

Besides the resulting quantitative results, additional qualitative feedback was collected between or after the respective experiments in semi-structured interviews or focus groups~\ncite{longhurst2003semi} to gain further insights into the user experience of the participants.

\paragraph{Observational Study}

In addition to controlled experiments, this work also uses observational studies where users are observed interacting with systems without actively intervening. This type of study is used in this thesis to collect qualitative findings of the interaction of users with a system. As for the controlled experiments, the observational studies were concluded with semi-structured interviews to collect further data.

\subsubsection{Analysis}

The collected data were carefully analyzed using widely accepted quantitative and qualitative methods. This section describes the methods used for analysis.

\paragraph{Parametric Analysis}

The recorded continuous dependent variables were analyzed using (multi-way) repeated measures analysis of variance (RM ANOVA) to unveil significant effects of the influence of the respective factors. This is an accepted approach to hypothesis testing in frequentist statistics~\ncite{girden1992anova}. Before analyzing, the data were tested for the fulfillment of the assumptions of RM ANOVA using the standard tests~\ncite{Field2003}: First, the data were tested for normality using Shapiro-Wilk’s test. If the test indicated a violation of the assumption of normality, the data were treated as non-parametric. Second, the data were tested for sphericity using Mauchly’s test. If Mauchly’s test indicated a violation of the assumption of sphericity, the degrees of freedom of the RM ANOVA were corrected using the Greenhouse-Geisser method, and this thesis reports the respective \gge. 

When RM ANOVA revealed significant effects, the analysis used paired-samples t-tests for pairwise post-hoc comparisons. To assure the reliability of the results of the t-tests, the results were corrected using the conservative Bonferroni method. 

\paragraph{Non-Parametric Analysis}

For the non-parametric hypothesis testing, the data was analyzed using the test as proposed by \typcite{Friedman1937} (for single-factor designs) or using an Aligned Rank Transformation (ART) followed by a RM ANOVA as proposed by \typcite{Wobbrock2011} (for multi-factorial designs).

When significant effects were revealed, pairwise signed-rank tests as proposed by \typcite{Wilcoxon1945} were performed for post-hoc analysis and, as for the parametric results, corrected using Bonferroni’s method. Wilcoxon's pairwise signed rank test is a non-parametric alternative to the t-test that does not assume a normal distribution.

\paragraph{Qualitative Analysis}

The qualitative feedback in semi-structured interviews and the conversations of participants in observational studies were recorded and transcribed. Afterward, the data were analyzed using an open coding~\ncite{Strauss1998} approach to, in the next step, identify common themes across participants.

\subsubsection{Reporting of the Results}
This thesis reports the eta-squared $\eta^{2}$ as an estimate of the effect size and uses Cohen’s suggestions to classify the effect size~\ncite{Cohen1988}. As an estimate of the influence of the individual factors, the thesis reports the \ac{EMM} as proposed by \typcite{Searle1980}. For the measured raw values as well as for the \acp{EMM}, the thesis reports the mean value $\mu$, the standard deviation $\sigma$ and the standard error $\sigma_{\overline{x}}$.