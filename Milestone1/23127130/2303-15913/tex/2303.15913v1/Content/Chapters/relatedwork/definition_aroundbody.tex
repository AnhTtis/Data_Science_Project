\typcite{Chen2014} shaped the term of \aroundbodyinteraction{} for interactions that expand \enquote{the input space beyond the device’s screen, [situating] interaction in the space within arm’s reach around the body}. The authors proposed to sense the distance and orientation of a hand-held device relative to the user’s body and presented example applications to use this information for around body interaction techniques. However, the authors only focused on the area around the upper body, ignoring the lower limbs, and mainly proposed interaction techniques tailored explicitly to smartphones. 

This work in the area of \aroundbodyinteraction{} shaped this thesis to a great extent. The central idea of expanding the interaction area to the space \emph{around} the user, leaving behind the limited interaction areas of devices and the surface of the user’s body is a fundamental building block of the work presented here. However, this thesis goes beyond the State-of-the-art and the prior definition of around-body interaction by 1) also exploring the lower limbs for interaction, 2) considering the effects of different forms of visualization, and 3) tailoring the interaction techniques to the requirements of \acp{HMD}.

Therefore, this thesis proposes the following extended definition of \aroundbodyinteraction{}:

\defbox{aroundbody}{Around-Body Interaction}{Around-body interaction leverages the movement of our upper and lower limbs to interact with information in the space around our body, defined by the reach of our limbs.}