\section{Discussion and Guidelines}
\label{sec:cheesyfoot:discussion}

The results of our controlled experiments suggest that foot-taps provide a viable interaction technique for \acp{HMD}. In both experiments, the evaluation showed \acp{TCT} suitable for fast interactions. While the analysis found significantly increasing \acp{TCT} for finer subdivisions of \direct{} interfaces (see section \ref{sec:cheesyfoot:experiment1:tct}), the \acp{TCT} of \indirect{} interfaces were stable across all conditions with only slight differences (see section \ref{sec:cheesyfoot:experiment2:tct}). Interestingly, for higher subdivisions, the \ac{TCT} seem to converge between both styles.

Based on the analysis of the two interaction styles, this section presents a set of guidelines.

\subsection{Favour the Division into Columns over Rows}

Our results suggest that more granular subdivisions through higher numbers of rows have a larger impact on the accuracy than finer subdivisions through the addition of columns. This impression was further supported for \indirect{} interfaces by investigating the overlap of the individual target areas: The analysis found a significantly larger overlap within a column (i.e., between several rows) compared with the overlap within a row (i.e., between several columns, see section \ref{sec:cheesyfoot:experiment2:overlap}). Also, in both experiments, the analysis showed a significantly growing spread of the tapping points for targets in more distant target rows (\reffig{fig:cheesyfoot/scatters}).

Therefore, the division into columns over rows should be favored when designing such interfaces.

\subsection{Use indirect interfaces for longer-term interactions that require less accuracy}

As expected, the accuracy rates for \indirect{} interactions were significantly lower compared to \direct{} interactions (see section \ref{sec:cheesyfoot:comparison:accuracy}). However, the difference was very low for the \row{1} conditions, in particular for 2 and 4 targets (see figure \ref{fig:cheesyfoot/accuracy}). Together with the differing overlaps in the row and column directions discussed above, this leads us to the conclusion that the participants - despite different self-perception - had great difficulties in distinguishing between different rows and, thus, the use of multiple rows for \indirect{} interfaces is not feasible. Regarding the Likert-questionnaires and the qualitative feedback, the analysis found greater popularity of the \indirect{} interfaces (see section \ref{sec:cheesyfoot:experiment1:qualitative} and \ref{sec:cheesyfoot:experiment2:qualitative}).

Taken together the greater enthusiasm, as well as the lower TLX scores (for \row{1} subdivisions), the use of \indirect{} interfaces is preferable in most situations. In particular, this applies to situations where 1) a lower number of options is sufficient and 2) a restricted view (as in the \direct{} interfaces, where the head is directed to the floor) could be problematic. Based on the analysis, a \row{1}, \col{4} layout for \indirect{} interfaces is feasible.

\subsection{Use direct interfaces for short-term and fine-grained interactions}

Direct interfaces delivered significantly higher accuracy rates compared to indirect interfaces (see sections \ref{sec:cheesyfoot:comparison:accuracy}). However, the analysis of qualitative feedback and answers in the Likert questionnaires showed a clear preference of participants for \indirect{} interfaces. The limitations of the hardware used in the experiment (e.g., weight, field of view) might have exerted a considerable influence on the opinion of the participants. However, in particular, the downward head posture seems to be rejected by the participants for longer-term interactions in general (see section \ref{sec:cheesyfoot:experiment1:qualitative}).

Therefore, for the tested design, \direct{} interfaces proved to be best suited for short-term interactions requiring high accuracy and a large number of input options. For such interfaces, a high degree of accuracy is still achieved with \row{3}, \col{6} layouts.


