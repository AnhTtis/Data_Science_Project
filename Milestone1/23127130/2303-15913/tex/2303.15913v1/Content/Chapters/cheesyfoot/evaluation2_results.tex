\section{Experiment II: Indirect Interaction}
\label{sec:cheesyfoot:evaluation2_results}

This section reports the results of the second experiment focusing on RQ2 and, thus, on \indirect{} interfaces using the visualization in front of the participant as described in section \ref{sec:methodology:heads-up}. For this, 18 participants (5 female), aged between 21 and 31 years (\val{23.3}{2.8}), 3 left-footed, were recruited using our University’s mailing list. None of them had prior experience with \ac{AR}. During the analysis, 16 out of 8748 trials were excluded as outliers due to technical problems during recording. The analysis of the data was performed as described in section \ref{sec:rw:methodology}. 

Section \ref{sec:cheesyfoot:comparison} compares the results of this experiment (focusing on \indirect{} interfaces) to the results for \direct{} interfaces. Section \ref{sec:cheesyfoot:discussion} discusses the results of both experiments with regards to the research questions.



\subsection{Classification}

In the first experiment, the analysis used the physical dimensions of the targets (visible as \direct{} feedback on the floor) to calculate the accuracy rates. However, this approach could not be transferred directly to the second experiment, as the participants interacted with an \indirect{} visualization. There was, therefore, no direct definition of the accuracy of the participants’ hits and misses. As a result, the analysis was started with the construction of suitable classifiers.

The data was classified using \acp{SVM} and trained nine \ac{SVM} classifiers according to our nine conditions. For this, each corresponding partial data set was divided into an 80\% training set and a 20\% test set. The training sets were used to train per-condition \acp{SVM} with radial kernels. To avoid over-fitting to the data, the process used 10-fold cross-validation with 3 repetitions and predictions on the 20\% test sets to assess the quality of the \acp{SVM}. Furthermore, per-participant \acp{SVM} were trained and compared to the results of the models trained with the data of all participants. However, as there were only minor differences in the accuracy rates (+/- 2\%, depending on the condition), this section uses the generalized models for further analysis.

\subsection{Accuracy}

\begin{table}
	\centering
	
	\begin{tabularx}{\linewidth}{ccYYYY}
		& &   &   & \multicolumn{2}{c}{\textbf{95\% CI\textsuperscript{1}}}\\
		\cmidrule(lr){5-6}
		\textbf{Rows} & \textbf{Columns} & $\pmb{\mu}$ & $\pmb{\sigma}$ & \textbf{Lower} & \textbf{Upper}\\
		\midrule

1 & 2 & .995 & .008 & .931 & 1.0 \\
& 4 & .977 & .037 & .912 & 1.0 \\
& 6 & .846 & .122 & .782 & .911 \\
2 & 2 & .821 & .141 & .756 & .885 \\
& 4 & .746 & .148 & .682 & .810 \\
& 6 & .675 & .161 & .611 & .739 \\
3 & 2 & .675 & .172 & .611 & .740 \\
& 4 & .641 & .117 & .577 & .705 \\
& 6 & .585 & .148 & .521 & .650 \\

		\bottomrule
	\end{tabularx}
	
	\caption{The accuracy rates of indirect interfaces as measured in the second experiment. The table reports the recorded mean values $\mu$ together with the standard deviation $\sigma$. \textsuperscript{1} The confidence interval CI is based on the fitted \ac{EMM} model.}
	\label{tab:cheesyfoot/tables/accuracy_indirect}
\end{table}


The analysis showed that both independent variables, the number of rows (\anova{2}{30}{60.87}{<.001}{.460}) and the number of columns (\anova{2}{30}{11.61}{<.001}{.082}) had a significant influence on the accuracy with large and small effect size, respectively. Post-hoc tests confirmed significantly lower accuracy rates for a higher number of rows between all groups (all $p <.001$) and between 2 and 6 ($p<.001$) as well as 4 and 6 columns ($p<.05$). The analysis could not find any interaction effects between the variables (\anova{4}{60}{2.37}{>.05}{.007}).

Interestingly, a closer look revealed that, as the number of rows and columns increases, the falling accuracy is not directly dependent on the number of resulting targets: In both, the \row{1}, \col{4} condition as well as the \row{2}, \col{2} condition, the participants had to hit 4 different targets. However, we found a significantly higher accuracy rate for the \row{1}, \col{4} (\val{98 \%}{1.1 \%}) condition compared to the \row{2}, \col{2} (\val{83.6 \%}{13.4 \%}) condition ($p<.01$). The analysis found the same effect for the \row{1}, \col{6} (\val{85.1 \%}{12.3 \%}) condition compared to the \row{3}, \col{2} (\val{69.7 \%}{17.2 \%}) condition ($p<.01$). This indicates that the number of rows has a greater influence on the accuracy than the number of columns.

Considering the \ac{EMM} for the individual conditions, the analysis found a overall high accuracy rate for the \row{1} (\emmSiCI{94.3}{3}{89.0}{99.0}{\%}) conditions. Therefore, the accuracy for the \row{1} conditions of \indirect{} interfaces proved to be comparable to the accuracy found for \direct{} interfaces (see \ref{sec:cheesyfoot:experiment1:accuracy}). Section \ref{sec:cheesyfoot:comparison} presents a more detailed comparison of both interface styles. Figure \ref{fig:cheesyfoot/accuracy} (\studyTwoColor{}) depicts the measured accuracy rates for all conditions, table \ref{tab:cheesyfoot/tables/accuracy_indirect} lists the \acp{EMM} for the individual factors.

\subsection{Task Completion Time}
\label{sec:cheesyfoot:experiment2:tct}


\begin{table}
	\centering
	
	\begin{tabularx}{\linewidth}{ccYYYY}
		& &   &   & \multicolumn{2}{c}{\textbf{95\% CI\textsuperscript{1}}}\\
		\cmidrule(lr){5-6}
		\textbf{Rows} & \textbf{Columns} & $\pmb{\mu}$ & $\pmb{\sigma}$ & \textbf{Lower} & \textbf{Upper}\\
		\midrule
		1 & 2 & 1.44s & .17s & 1.34s & 1.55s  \\
		& 4   & 1.53s & .19s & 1.43s & 1.64s  \\
		& 6   & 1.58s & .17s & 1.48s & 1.69s  \\
		2 & 2 & 1.53s & .21s & 1.42s & 1.63s  \\
		& 4   & 1.50s & .20s & 1.40s & 1.61s  \\
		& 6   & 1.60s & .27s & 1.50s & 1.71s  \\
		3 & 2 & 1.51s & .19s & 1.41s & 1.62s  \\
		& 4   & 1.55s & .26s & 1.44s & 1.66s  \\
		& 6   & 1.57s & .17s & 1.46s & 1.68s  \\
		\bottomrule
	\end{tabularx}
	
	\caption{The task-completion times of indirect interfaces as measured in the second experiment (in seconds). The table reports the recorded mean values $\mu$ together with the standard deviation $\sigma$. \textsuperscript{1} The confidence interval CI is based on the fitted \ac{EMM} model.}
	\label{tab:cheesyfoot/tables/tct_indirect}
\end{table}

The analysis unveiled that the number columns of the condition had a significant (\anova{2}{30}{7.698}{<.01}{.032}) effect on the \acl{TCT} with a small effect size. Post-hoc tests confirmed a significantly lower \ac{TCT} for the \col{2} conditions (\emmSiCI{1.495}{.044}{1.402}{1.587}{s}) compared to the \col{6} conditions (\emmSiCI{1.585}{.044}{1.492}{1.677}{s}), $p<.001$. With regard to the number of rows (\ac{EMM} $\mu$ between $\SI{1.52}{s}$ and $\SI{1.54}{s}$), the analysis could not find any significant influence (\anova{2}{30}{.307}{>.05}{.003}). Also, the analysis could not find any interaction effects between the factors (\anova{4}{60}{1.314}{>.05}{.282}). Figure \ref{fig:cheesyfoot/tct} (\studyTwoColor{}) depicts the \acp{TCT} for all conditions, table \ref{tab:cheesyfoot/tables/tct_indirect} lists the \acp{EMM} for the individual factors.

\subsection{Footedness and Foot Used for the Interaction}

The analysis could not find any influence of the footedness of the participants on the accuracy (\anova{1}{14}{.145}{>.05}{.002}) nor on the TCT (\anova{1}{14}{2.08}{>.05}{arg5.08}).

As in the first experiment, almost all targets to the left of the participants’ line of sight were performed with the left foot and vice versa ($\mu > .97$ for all conditions). Again, the analysis found no significant influences of the number of rows (\anovaCor{1.27}{17.74}{.044}{>.05}{.633}{.001}), the number of columns (\anovaCor{1.33}{18.61}{.344}{>.05}{.665}{.003}) or the footedness of the participant (\anova{1}{14}{.048}{>.05}{.001}) on the foot used for interaction.

\subsection{Size of the Target Areas}
\label{sec:study2:size}

\textfig{cheesyfoot/scatters}{Scatter plots with 95\% data probability ellipses for the \col{4} conditions with \indirect{} interfaces in the second experiment. While the data points for four columns can be separated, this is not possible for more than one row.}

Again, the analysis showed a siginificant influence of the target row on the area of the targets (\anova{2}{32}{8.90}{<.001}{.027}) with a small effect size. Post-hoc tests confirmed significantly smaller areas if the target was in \row{1} (\emmSiCI{.042}{.009}{.024}{.060}{m^2}) compared to \row{3} (\emmSiCI{.074}{.009}{.056}{.092}{m^2}), $p<.001$. The analysis could not find a significant influence of the target column (\anova{11}{176}{1.58}{>.05}{.029}) nor interaction effects (\anova{22}{352}{1.02}{>.05}{.028}).

\subsection{Overlap}
\label{sec:cheesyfoot:experiment2:overlap}

For conditions with multiple rows, there were noticeable overlaps in the distribution of the tapping points (\reffig{fig:cheesyfoot/scatters} for the 4 column conditions). As a measure for these overlaps, the analysis compared the number of points from adjacent targets in the row direction and in the column direction that fell into the 95\% data ellipse of each target.

The analysis showed a significant difference between the overlap in row and column direction (\anovaCor{1}{17}{324}{<.001}{.890}) with a large effect size. Post-hoc tests confirmed a significantly lower overlap in row direction (\valSi{4.0}{3.7}{\%}) compared to the column direction (\valSi{55.0}{12.7}{\%}), $p<.001$.

\subsection{TLX and Questionnaire}


The analysis showed a significant influence of the number of rows (\anova{2}{34}{31.02}{<.001}{.125}) with a medium effect size. Post-hoc tests confirmed a significantly higher perceived cognitive load for higher numbers of rows ($p<.001$ comparing \row{1} and \row{3}, $p<.01$ otherwise) from \emm{19.6}{2.55}{14.3}{24.9} (\row{1}) over \emm{25.0}{2.55}{19.7}{30.4} (\row{2}) to \emm{30.5}{2.55}{25.2}{35.8} (\row{3}).

The analysis further found a significant influence of the number of columns (\anova{2}{34}{10.481}{<.001}{.035}) with a small effect size. The post-hoc analysis showed rising estimated marginal means (\col{2}: \emm{22.6}{2.53}{17.4}{27.9}, \col{4}: \emm{24.2}{2.53}{18.9}{29.5}, \col{6}: \emm{28.3}{2.53}{23.0}{33.5}) with significant differences between 2 and 6 columns ($p<.001$) as well as between 4 and 6 columns ($p<.05$). We could not observe interaction effects between the number of rows and the number of columns (\anova{4}{68}{.447}{>.05}{.002}). Figure \ref{fig:cheesyfoot/tlx} (\studyTwoColor{}) depicts the measured values.

\subsubsection{Confidence}

\textfig{cheesyfoot/likert-study2}{The participant’s answers to our questions for indirect interfaces on a 5-point Likert-scale.}

The questionnaire asked the participants how confident they felt to have hit the correct targets. The analysis found significant effects for both, the number of rows (\anovaWithoutEffect{2}{34}{22.711}{<.001}) as well as the number of columns (\anovaWithoutEffect{2}{34}{35.345}{<.001}). Post-hoc tests confirmed significantly higher confidence ratings for \row{1} conditions compared to \row{2} and \row{3} conditions (both $p<.001$). For the number of columns, the analysis found significantly rising ratings between all levels (all $p<.001$). The analysis could not find interaction effects (\anovaWithoutEffect{4}{68}{.185}{>.05}).


The absolute numbers (\reffig{fig:cheesyfoot/likert-study2}) show a high agreement for all \row{1} conditions with decreasing confidence for higher numbers. Interestingly, the majority of the participants were convinced that they could keep the targets apart for all conditions (except \row{3}, \col{6}).

\subsubsection{Convenience}

The questionnaire further asked the participants how convenient the layout felt to interact with information. The analysis showed significant effects for both, the number of rows (\anovaWithoutEffect{2}{34}{56.462}{<.001}) and the number of columns (\anovaWithoutEffect{2}{34}{8.203}{<.01}). Post-hoc tests confirmed significantly falling ratings for higher numbers of rows between all levels (all $p<.001$). Regarding the number of columns, we found significantly lower ratings for the \col{6} conditions compared to the \col{2} ($p<.01$) and \col{4} ($p<.05$) conditions. The analysis could not find interaction effects (\anovaWithoutEffect{4}{68}{1.947}{>.05}).


All but the \row{3}, \col{6} condition were rated predominantly positive.

\subsubsection{Willingness to Use}


As the last question, the questionnaire asked the participants if they would like to use this arrangement for interacting with HMDs. The analysis showed a significant effect for the number of rows (\anovaWithoutEffect{2}{34}{26.849}{<.001}) and the number of columns (\anovaWithoutEffect{2}{34}{3.600}{<.05}) as well as interaction effects between the factors (\anovaWithoutEffect{4}{68}{3.286}{<.05}). Post-hoc tests confirmed significantly lower ratings for the \row{3} conditions compared to the \row{1} and \row{2} conditions (both $p<.001$). For the number of columns, post-hoc tests did not confirm significant differences.


\subsection{Qualitative Feedback}
\label{sec:cheesyfoot:experiment2:qualitative}

The participant's feedback proved to be more enthusiastic compared to the \direct{} interfaces as presented in section \ref{sec:cheesyfoot:experiment1:qualitative}. See section \ref{sec:cheesyfoot:comparison} and \ref{sec:cheesyfoot:discussion} for more detailed comparisons and discussion.

In general, all participants appreciated the idea of being able to interact with \acp{HMD} using their feet without looking at the floor. When asked for the reasons, participants told us that this interaction modality felt \pquote{novel}{1}, \pquote{fun to use}{12} and \pquote{very easy to perform in addition to other tasks}{15} as the \pquote{hands are not needed}{18} and \pquote{it's a low effort extension [\ldots] to interact}{9}. Participants found the \pquote{radial placement}{11} of the targets \pquote{nice}{11} and had the feeling that different columns were \pquote{relatively easy to discern}{7}. Four of the participants felt \mpquote{unsure}{P2, P5, P11, P12} about their performance with multiple rows. P11 even perceived more than one row as \enquote{\emph{inconvenient}}. P18 summarized: This \enquote{\emph{feels quite naturally in comparison to the strange in-air gestures that are used for the Hololens}}.
