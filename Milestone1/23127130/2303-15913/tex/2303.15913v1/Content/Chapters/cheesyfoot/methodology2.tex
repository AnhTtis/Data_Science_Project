\section{Methodology}
\label{sec:cheesyfoot:methodology}


This chapter presents the methodology of two controlled experiments assessing the accuracy and efficiency of \direct{} and \indirect{} interfaces for foot-based interactions with \acp{HMD} as presented in section \ref{sec:cheesyfoot:concept}. More specifically, the experiments investigated the following research questions:

\begin{description}
	
	\item[RQ1] How does the layout of the semicircle in terms of number of columns and rows affect the accuracy, efficiency and user experience of \direct{} interfaces?
	\item[RQ2] How does the layout of the semicircle in terms of number of columns and rows affect the accuracy, efficiency and user experience of \indirect{} interfaces?
	\item[RQ3] How does the the visualization (as \direct{} or \indirect{}) affect the accuracy, efficiency and user experience of such an interface?
	
\end{description}

To avoid learning effects, RQ1 and RQ2 were addressed in two separate experiments, although the basic design and procedure show large overlaps. To keep the results of both experiments comparable, only the visualization technique was changed as \direct{} and \indirect{} between the experiments. No participant took part in both experiments. The analysis used the results of both experiments to address RQ3. The following description of the methodology applies to both experiments unless stated otherwise.

\colfigVspace{cheesyfoot/rowcolumns2}{The independent variables (number of rows and number of columns) tested in the two experiments.}{0em}

\subsection{Design and Task}

The conditions varied the number of \emph{rows} and \emph{columns} that divide the semicircular grid into several targets (see section \ref{sec:cheesyfoot:concept}) as independent variables in a repeated measures design. The independent variables were varied in three levels for the number of rows (1,2,3) and three levels for the number of columns (2,4,6). Therefore, the experiment tested grids from $1*2=2$ to $3*6=18$ targets (\reffig{fig:cheesyfoot/rowcolumns2}). We considered these variables to assess their impact on participants’ performance regarding accuracy and efficiency. The experiment required at least three repetitions of each target (i.e., based on the most complex condition \row{3}, \col{6}: $3*6*3 = 54$). To prevent the influence of fatigue, the experiment was designed with an equal number of trials in each condition. This design resulted in a total of $3*3*54=486$ trials per participant. The order of conditions was counterbalanced using a Balanced Latin Square design. For each condition, the series of targets were randomized while maintaining an equal number for each target.

\begin{figure*}[ht!]
\subfloat[Foot Tracking\label{fig:study:feet}]
  {\includegraphics[width=.49\linewidth]{Content/Figures/cheesyfoot/study_feet_cropped}}\hfill
\subfloat[HoloLens Tracking\label{fig:study:head}]
  {\includegraphics[width=.49\linewidth]{Content/Figures/cheesyfoot/study_head_cropped}}\hfill
\subfloat[Direct Visualization\label{fig:study:direct}]
  {\includegraphics[width=.49\linewidth]{Content/Figures/cheesyfoot/study_direct}}\hfill
\subfloat[Indirect Visualization\label{fig:study:indirect}]
    {\includegraphics[width=.49\linewidth]{Content/Figures/cheesyfoot/study_indirect}}
\caption{We tracked the position and orientation of the feet (a) and the hololens (b). During the first experiment, we used a \direct{} visualization on the floor (c). In the second experiment, we used an \indirect{} floating visualization (d).}
\end{figure*}


\subsubsection{Experiment I: Direct Visualization}
\label{sec:methodology:floor}

The system visualized the semicircular grid within leg reach on the floor in front of the participant. Depending on the condition, the semicircle was divided into a grid with 2-6 horizontal columns of equal size and 1-3 rows of equal size. The columns filled the complete semicircle (\reffig{fig:study:direct}). Based on the average human leg length \ncite{Eveleth1990}, the system used a fixed height of \SI{8.5}{cm} for each row. This size was chosen to allow all participants to reach the goals within the \row{3} conditions comfortably. The participants’ task was to look at the floor in front of them and to tap highlighted targets.

\subsubsection{Experiment II: Indirect Visualization}
\label{sec:methodology:heads-up}

The second experiment used an \indirect{} \ac{HUD} visualization, floating in front of the eyes of the user (\reffig{fig:study:indirect}). The ultimate goal of this experiment was to understand how the participants would naturally map the presented target areas to the ground in front of them. Therefore, the system did not give the participants feedback about the position of their feet. Such feedback would have given the participants an indication of the size of the target areas, thereby distorting the results. The participants’ task was to tap the floor at the position where they expected the targets highlighted in the floating visualization.

\subsection{Study Setup and Apparatus}

The system used an optical tracking system (OptiTrack) to measure the position of the participant’s feet. For this, the participants wore 3D-printed parts, each augmented with a set of retro-reflective markers, on both feet (\reffig{fig:study:feet}) and a Microsoft Hololens (also with retro-reflective markers, \reffig{fig:study:head}) which displayed the respective visualization.

A study client application was implemented that allowed the investigator to set the task from a desktop located next to the participant. For each trial, the system logged the trace of the participants’ \emph{feet movements} and \emph{head (HoloLens) movements} to establish a matching between the visual feedback and the foot-taps. Furthermore, the system measured the time between displaying the task and touching the floor with the foot as the \emph{\ac{TCT}} and logged it together with the \emph{foot used for interaction}, the \emph{tap position} (relative to the participant), the \emph{target} and the \emph{condition} for later analysis.

\subsection{Procedure}
\label{study1:procedure}

After welcoming the participants, the investigator introduced them to the concept and the setup of the study. During this, the method proposed by \typcite{Chapman1987} was used to measure the foot preference of the participants. For this, the investigator asked the participants to write their names with their feet, like they would in the sand on the beach. The investigator observed the participants and noted which foot they used. Further, the investigator measured the height and leg length of the participants to analyze their impact on the participants’ performance later. Then, the participants mounted the trackable apparatuses on their feet and were asked them to put the Hololens on their head. To avoid learning effects, the experiment started with five minutes warming phase without data recording to get accustomed to the hardware and the interfaces.


The system was calibrated with the participants standing relaxed and looking straight ahead. After starting the condition, the participants saw the respective visualization. Once ready and in starting position (both feet together), the investigator started the condition to be evaluated. The system then colored the respective target to be reached in blue (\reffig{fig:study:direct}, \ref{fig:study:indirect}) and informed the participant about the start of the trial with an additional audio signal. Then, the participant moved the foot and tapped the floor on the target position. The investigator did not enforce the usage of a specific foot but told the participants to use the foot that seemed most comfortable for each trial. After tapping the target, the system changed the target color to green to inform the participant that the measurement was recorded and that the participant should move the foot back to the starting position. Once reached, the system waited 2 seconds before proceeding to the next target.

Participants were instructed to focus on the accuracy (tapping the center of the target) instead of the speed. Participants did not receive any feedback regarding their performance during the study. After each condition, participants completed a NASA TLX~\ncite{Hart2006} questionnaire and answered questions regarding their experiences on a 5-point Likert-scale (1: strongly disagree, 5: strongly agree). Further, the system enforced a 5-minute break between the conditions during which participants gave qualitative feedback in a semi-structured interview. Each experiment took about 60 minutes per participant.

\textfigStudybox{cheesyfoot/studybox_mindthetap}


