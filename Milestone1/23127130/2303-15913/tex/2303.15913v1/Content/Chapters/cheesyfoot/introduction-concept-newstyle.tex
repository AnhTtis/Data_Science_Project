\FloatBarrier
\section{Concept}
\label{sec:cheesyfoot:introduction}
\label{sec:cheesyfoot:concept}

\cptteaser{cheesyfoot/teaser_sepp}{This chapter presents two interaction techniques, leveraging foot-taps as a \direct{} (a) and \indirect{} (b) input modality for interacting with \acp{HMD}.}

As outlined above, foot-based interaction techniques frequently emerge as a potential input modality for novel computing systems. In recent years, research started to transfer the ideas of foot-based interaction to the field of interacting with \acp{HMD}~\ncite{Matthies2013, Fukahori2015}. This chapter contributes to this promising stream of research by exploring the feasibility of using foot-taps as an input modality for \acp{HMD}, assessing the benefits and drawbacks of 1) \emph{direct interaction} with interfaces that are displayed on the floor and require the user to look down to interact (see figure~\ref{fig:cheesyfoot/teaser_sepp} a) and 2) \emph{indirect interaction} with interfaces that, although operated by the user's feet, are displayed as a two-dimensional window floating in the space in front of the user (see figure~\ref{fig:cheesyfoot/teaser_sepp} b).

For this, we consider a semicircular interaction wheel that is anchored to the dominant foot of the user’s standing position. The interaction wheel is divided into a grid by multiple rows and columns. Each cell of the grid represents an option the user can select though a foot tap (see figure~\ref{fig:cheesyfoot/teaser_sepp}). This chapter explores two different styles of interaction with such an interactive grid that share the same style of input, but vary the visualization:

\begin{description}
	\item[Direct Interaction] The semicircular grid is visualized within leg reach on the floor in front of the participant. Therefore, there is a \direct{} connection between the location of input and output. The user can interact with the system by looking to the ground and tapping the location where the respective grid cell is visualized.
	\item[Indirect Interaction] The \indirect{} interface moves the visualization from the floor to the air in front of the user. However, despite the changed location of the visualization, users still can interact with the system using foot taps through the sense of proprioception. This sense allows users to move their feet without looking at them.
\end{description}