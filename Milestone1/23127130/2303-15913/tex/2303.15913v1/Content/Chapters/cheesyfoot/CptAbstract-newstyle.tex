The previous chapters of this thesis investigated the use of our \emph{upper limbs} for interacting with \acp{HMD} for \emph{body-stabilized} and \emph{world-stabilized} visualizations. As introduced in chapter \ref{ch:introduction} and \ref{ch:relatedwork}, interaction techniques utilizing the upper limbs are not sufficient to use \acp{HMD} in all everyday situations: Such interaction techniques cannot support a wide range of situations where both of the user's hands are busy interacting with the real world (e.g., due to carrying something in hand).

As a possible solution for such situations, this chapter proposes the usage of the \emph{lower limbs} to interact with \emph{body-stabilized} interfaces (see section \ref{sec:relatedwork:hmds:implementation}) with an emphasis on support for the interaction situations \mobilityLower{}, \singleuserLower{}, and \discreteLower{}. Interaction techniques in this quadrant of the design space allow users to keep their hands free to interact with the real world, while their feet can provide input for a system. By stabilizing the interface to the body of the user, it is always available for interaction, further increasing the mobility of interaction techniques in this quadrant of the design space (see figure \ref{fig:cheesyfoot/overview_mindthetap}). While there is a long history of using the lower limbs for foot-based interfaces in various areas of computing systems, to the best of our knowledge, foot-based interactions have not yet been systematically evaluated for interaction with \acp{HMD}.

This chapter aims to close this gap and to add to the body of research on interacting with \acp{HMD} by exploring a foot-based input modality for \acp{HMD}. The contribution of this chapter is twofold: First, the chapter presents the results of two controlled experiments, assessing the benefits and drawbacks of two styles of interaction leveraging the \emph{lower limbs} to interact with \emph{body-stabilized} interfaces. Second, based on the results of the two experiments, this chapter provides a set of guidelines for designing such user interfaces for both types of interaction.

The remainder of this chapter is structured as follows: After the review of related works (section \ref{sec:cheesyfoot:rw}), section \ref{sec:cheesyfoot:concept} presents two interaction styles for foot-based interaction with \acp{HMD}. In the following, \ref{sec:cheesyfoot:methodology} describes the methodology and research questions of two controlled experiments focusing on \direct{} and \indirect{} interaction with content, respectively. After that, sections \ref{sec:cheesyfoot:evaluation1_results} and \ref{sec:cheesyfoot:evaluation2_results} present the results of the two controlled experiments. Based on a comparison of both styles of interaction (section \ref{sec:cheesyfoot:comparison}), section \ref{sec:cheesyfoot:discussion} provides design recommendations for \direct{} and \indirect{} foot-based user interfaces for interacting with \acp{HMD}. The chapter concludes with the limitations of the approach and directions for future work (section \ref{sec:cheesyfoot:limitations}).

\cptIntroBoxAward{Muller2019}{The student \emph{Joshua McManus} implemented the study client application. \emph{Sebastian Günther}, \emph{Martin Schmitz}, \emph{Markus Funk} and \emph{Max Mühlhäuser} supported the conceptual design and contributed to the writing process.}