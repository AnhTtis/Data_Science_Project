\section{Limitations and Future Work}
\label{sec:cheesyfoot:limitations}

The design and results of our experiments impose some limitations and directions for future work.


\subsection{Layout of the Targets}

The experiment used a fixed semicircular grid of targets. This layout was chosen because of the natural reachability of targets from a fixed standing position. However, other shapes (e.g., rectangular, oval) and arrangements (e.g., not equally sized targets) or adoptions to the footedness of the user could also be considered for future work. This is of particular interest as our experiment showed a larger spread for targets further away from the participant. These extensions enlargen the design space of such interfaces and, despite being outside of the scope of this work, provide interesting direction for future work. 

\subsection{Feedback for Indirect Interaction}

The goal of the presented experiment was to investigate the ability of users to use \indirect{} interfaces without visual feedback and, thus, create a baseline for future work. Therefore, the participants received no feedback about the position of their feet during the \indirect{} experiment as such additional feedback could strongly influence the performance of the participants.

Future Work is necessary to understand the implications of different forms of more direct feedback for users. Such direct feedback could be indicated by highlighting the currently selected option or by displaying a cursor moved by the foot.

\subsection{Other Styles of Interaction}

The experiment concentrated on interfaces, which, as an analogy to the traditional point-and-click interfaces, are operated with foot-taps. Other interaction styles, such as gestures for fine-granular control or taps with different parts of the foot (e.g., heel) may be beneficial for the future use of \acp{HMD}. 

Additionally, the presented experiment focused on one-time interactions. As a possible addition, cascading menus could help to keep the grid size small  and reduce necessary foot movements while maintaining a large set of options.



\subsection{The Midas Tap Problem}
\label{sec:cheesyfoot:limitations:midas}

Similar to the Midas Touch Problem~\ncite{Jacob1995} in eye gaze tracking, it is challenging to separate intentional input from natural motion when using foot-based input. A possible solution could be a special foot input mode, activated using a secondary input modality such as a toggle on the HMD or gaze interaction in the user interface. For \direct{} interfaces, just looking at the ground may be sufficient to activate this mode, as actions are only triggered after a subsequent tap. Further, sensor-based gait detection~\ncite{Jacob1995, Derawi2010} allows to only enable foot input while standing and, thus, help to prevent erroneous activation. Further work in this field is necessary to conclude on the Midas Tap problem.
