\section{Experiment I: Direct Interaction}
\label{sec:cheesyfoot:evaluation1_results}



This section reports the results of a controlled experiment investigating RQ1 and, thus, focusing on \direct{} interfaces using the visualization on the floor as described in section \ref{sec:methodology:heads-up}. For this, 18 participants (6 female), aged between 21 and 30 years (\val{24.9}{3.0}), were recruited using our University’s mailing list. Three of them had prior experience with \ac{AR}. 7 out of 8748 trials were excluded in the analysis as outliers due to technical problems. The analysis of the data was performed as described in section \ref{sec:rw:methodology}. 

Section \ref{sec:cheesyfoot:comparison} compares the results of this experiment (focusing on \direct{} interfaces) to the results for \indirect{} interfaces. Section \ref{sec:cheesyfoot:discussion} discusses the results of both experiments with regards to the research questions.

\subsection{Accuracy}
\label{sec:cheesyfoot:experiment1:accuracy}

\textfigH{cheesyfoot/accuracy}{The measured accuracy rates in both experiments. All error bars depict the standard error.}

The physical dimensions of the targets (visible on the floor through the \ac{HMD}) were used to classify the taps of the participants as hits and errors to obtain an accuracy rate. The analysis revealed that the number of rows had a significant (\anovaCor{1.32}{22.51}{4.068}{<.05}{.662}{.099}) influence on the accuracy with a small effect size. Post-hoc tests confirmed significantly higher accuracy rates for the \row{1} (\emmSiCI{98.9}{.6}{97.7}{100.2}{\%}) and \row{2} (\emmSiCI{99.1}{.6}{97.9}{100.4}{\%}) conditions compared to the \row{3} (\emmSiCI{96.8}{.6}{95.6}{98.1}{\%}) conditions.



The analysis did not show any significant influence of the number of columns (\anova{2}{34}{.515}{>.05}{.003}) or interaction effects between both factors (\anovaCor{2.63}{44.70}{1.699}{>.05}{.657}{.019}). Overall, the results showed high accuracy rates up to the highest (\row{3}, \col{6}) condition (\val{95.9 \%}{.5 \%}). Figure \ref{fig:cheesyfoot/accuracy} (\studyOneColor{}) depicts the measured accuracy rates for all conditions, table \ref{tab:cheesyfoot/tables/accuracy} lists the \acp{EMM} for the individual factors.


\begin{table}
	\centering
	
	\begin{tabularx}{\linewidth}{ccYYYY}
		& &   &   & \multicolumn{2}{c}{\textbf{95\% CI\textsuperscript{1}}}\\
		\cmidrule(lr){5-6}
		\textbf{Rows} & \textbf{Columns} & $\pmb{\mu}$ & $\pmb{\sigma}$ & \textbf{Lower} & \textbf{Upper}\\
		\midrule
		1 & 2 & 0.990 & 0.021 &  0.938 & 1.043  \\
		
		& 4 & 0.982 & 0.035 & 			0.930 & 1.035  \\
		
		& 6 & 0.925 & 0.112 & 			0.872 & 0.978  \\
		
		2 & 2 & 0.911 & 0.129 & 			0.858 & 0.963  \\
		
		& 4 & 0.878 & 0.161 & 			0.825 & 0.930  \\
		
		& 6 & 0.841 & 0.193 & 			0.789 & 0.894  \\
		
		3 & 2 & 0.834 & 0.193 & 			0.782 & 0.887  \\
		
		& 4 & 0.816 & 0.189 & 			0.763 & 0.868  \\
		
		& 6 & 0.783 & 0.217 & 			0.730 & 0.836  \\
		\bottomrule
	\end{tabularx}
	
	\caption{The accuracy rates of direct interfaces as measured in the first experiment. The table reports the recorded mean values $\mu$ together with the standard deviation $\sigma$. \textsuperscript{1} The confidence interval CI is based on the fitted \ac{EMM} model.}
	\label{tab:cheesyfoot/tables/accuracy}
\end{table}


\subsection{Task Completion Time}
\label{sec:cheesyfoot:experiment1:tct}
\textfigH{cheesyfoot/tct}{The measured task-completion times in both experiments. All error bars depict the standard error.}


The analysis unveiled that both, the number of rows (\anova{2}{34}{14.47}{<.001}{.059}) and the number columns (\anova{2}{34}{43.39}{<.001}{.203}) had a significant influence on the \acl{TCT} with medium and large effect size, respectively. Further, the analysis found interaction effects between the number of rows and the number of columns (\anovaCor{2.16}{36.67}{3.22}{<.05}{.539}{.024}) with a medium effect size.

Post-hoc tests confirmed significantly rising \acp{TCT} for higher numbers of rows (\row{1}: \emmSiCI{1.163}{.046}{1.068}{1.258}{s}, \row{2}: \emmSiCI{1.243}{.046}{1.149}{1.338}{s}, \row{3}: \emmSiCI{1.445}{.046}{1.351}{1.540}{s}) between all levels ($p<.05$ between \row{1} and \row{2}, $p<.001$ otherwise). For the number of columns, post-hoc tests showed significant differences between the \col{2} (\emmSi{1.196}{.045}{1.102}{1.290}{s}) and \col{6} (\emmSiCI{1.346}{.045}{1.252}{1.440}{s}) conditions ($p<.001$) as well as between the \col{4} (\emmSiCI{1.310}{.45}{1.252}{1.440}{s}) and the \col{6} conditions. Figure \ref{fig:cheesyfoot/tct} (\studyOneColor{}) shows the \acp{TCT} for all conditions, table \ref{tab:cheesyfoot/tables/tct} lists the \acp{EMM} for the individual factors.


\begin{table}
	\centering
	
	\begin{tabularx}{\linewidth}{ccYYYY}
		& &   &   & \multicolumn{2}{c}{\textbf{95\% CI\textsuperscript{1}}}\\
		\cmidrule(lr){5-6}
		\textbf{Rows} & \textbf{Columns} & $\pmb{\mu}$ & $\pmb{\sigma}$ & \textbf{Lower} & \textbf{Upper}\\
		\midrule
			1 & 2 & 1.09s & .26s & 0.98s & 1.20s  \\
& 4   & 1.19s & .17s & 1.08s & 1.30s  \\
& 6   & 1.20s & .19s & 1.10s & 1.31s  \\
2 & 2 & 1.18s & .20s & 1.07s & 1.29s  \\
& 4   & 1.30s & .27s & 1.19s & 1.41s  \\
& 6   & 1.24s & .18s & 1.14s & 1.36s  \\
3 & 2 & 1.31s & .29s & 1.20s & 1.42s  \\
& 4   & 1.43s & .22s & 1.33s & 1.54s  \\
& 6   & 1.58s & .33s & 1.48s & 1.69s  \\
		\bottomrule
	\end{tabularx}
	
	\caption{The task-completion times of direct interfaces as measured in the first experiment (in seconds). The table reports the recorded mean values $\mu$ together with the standard deviation $\sigma$. \textsuperscript{1} The confidence interval CI is based on the fitted \ac{EMM} model.}
	\label{tab:cheesyfoot/tables/tct}
\end{table}



\subsection{Footedness and Foot Used for the Interaction}

The analysis could not find any influence of the footedness of the participants on the accuracy (\anova{1}{16}{.570}{>.05}{.007}) nor on the TCT (\anova{1}{16}{1.42}{>.05}{.042}). Interestingly, although the system left it up to the participants to decide which foot they wanted to use, virtually all targets to the left of the participants’ line of sight were performed with the left foot and vice versa ($\mu > 96\%$ for all conditions). Matching this, the results showed no significant influences of the number of rows (\anova{2}{32}{.408}{>.05}{.002}), the number of columns (\anovaCor{1.21}{19.28}{.292}{>.05}{.603}{.003}) or the footedness (\anova{1}{16}{.451}{>.05}{.014}) on the foot used for interaction.

\subsection{Size of the Target Areas}

\textfig{cheesyfoot/scatters_study1}{Scatter plots with 95\% data probability ellipses for the \col{4} conditions with \direct{} interfaces in the first experiment. All target areas can be separated. The outer (\row{3}) target areas are larger than the nearer targets.}

This section analyzes the influence of the target position (as target row and target column) on the spread of the recorded tapping positions. As a measurement for the spread of data, individual 95\% data probability ellipses (i.e., ellipses containing 95\% of the recorded points for this target) were calculated per participant and target position and compared the other data ellipses.

The analysis showed a significant influence of the target row on the area of the targets (\anova{2}{34}{13.36}{<.001}{.04}) with a small effect size. Post-hoc tests confirmed significantly larger areas if the target was located in \row{3} (\emmSiCI{.0454}{.006}{.033}{.058}{m^2}) compared to \row{1} (\emmSiCI{.005}{.006}{.000}{.017}{m^2}) and \row{2} (\emmSiCI{.008}{.006}{.000}{.020}{m^2}), both $p<.001$. Despite rising means, the analysis could not show significant effects between targets in the \row{1} and \row{2} conditions.

The analysis could not find a significant influence of the target column (\anova{11}{187}{1.62}{>.05}{.026}) nor interaction effects between the number of rows and the number of columns (\anova{22}{374}{1.62}{>.05}{.0051}). The overlap between the target areas was not analyzed as the \direct{} visualization limited the size of the target areas. Figure \ref{fig:cheesyfoot/scatters_study1} depicts the 95\% data probability ellipses for the \col{4} conditions and illustrates the rising area sizes for targets in outer rows.

\subsection{TLX and Questionnaire}

\textfigH{cheesyfoot/tlx}{The measured Raw-TLX rates in both experiments. All error bars depict the standard error.}

The \ac{RTLX} questionaire showed a significant influence of the number of rows (\anova{2}{34}{16.82}{<.001}{.047}) with a small effect size. Post-hoc tests confirmed a significant effect for the number of rows between the \row{1} (\emm{21.4}{3.28}{14.5}{28.3}) and \row{2} (\emm{25.1}{3.28}{18.3}{32.0}) conditions ($p<.05$), the \row{1} and \row{3} (\emm{29.2}{3.28}{22.4}{36.1}) conditions ($p<.001$) as well as between the \row{2} and \row{3} conditions ($p<.05$).

The analysis further showed a significant influence of the number of columns (\anovaCor{1.34}{22.85}{6.83}{<.01}{.672}{.023}) with a small effect size. Post-hoc tests showed significant differences between the \col{2} (\emm{22.2}{3.3}{15.3}{29.1}) and \col{4} (\emm{26.0}{3.3}{19.1}{32.9}) conditions as well as between the \col{2} and \col{6} (\emm{27.5}{3.3}{20.6}{34.4}) conditions. The analysis could not find any interaction effects between the factors (\anova{4}{68}{2.28}{>.05}{.005}). Figure \ref{fig:cheesyfoot/tlx} (\studyOneColor{}) depicts the measured values for all conditions.

\subsubsection{Confidence}

\textfig{cheesyfoot/likert-study1}{The participant’s answers to our questions for direct interfaces on a 5-point Likert-scale.}

Matching the quantitative results, the participants felt very confident that they hit the correct targets across all conditions (\reffig{fig:cheesyfoot/likert-study1}). The analysis showed a significant effect for the number of columns (\anovaWithoutEffect{2}{34}{5.259}{<.05}). Post-hoc tests confirmed a significantly higher confidence for \col{4} conditions compared to \col{2} and \col{6} conditions (both $p<.05$). The analysis could not find effects for the number of rows (\anovaWithoutEffect{2}{34}{.831}{>.05}) but interaction effects between the two factors (\anovaWithoutEffect{4}{68}{5.057}{<.01}).


\subsubsection{Convenience}

The questionnaire asked the participants how convenient they felt with the layout to interact with information considering the number of input options and the physical and mental effort required to use them.. The analysis showed significant effects for both, the number of rows (\anovaWithoutEffect{2}{34}{23.984}{<.001}) as well as the number of columns (\anovaWithoutEffect{2}{34}{7.891}{<.01}). Post-hoc tests confirmed significantly lower ratings for the \row{3} conditions compared to the other levels (both $p<.001$). Regarding the number of columns, the analysis found a significant difference between the \col{2} and \col{6} conditions ($p<.01$). The analysis could not find interaction effects (\anovaWithoutEffect{4}{68}{1.065}{>.05}).


A closer look at the answers supports the statistical results and, thus, the strong influence of the number of rows: All but the \row{3} conditions are rated predominantly positively. Figure \ref{fig:cheesyfoot/likert-study1} depicts all answers from the participants.

\subsubsection{Willingness to Use}

Further, the questionnaire asked the participants if they would like to use this arrangement for interacting with HMDs. The analysis showed a significant effect for both, the number of rows (\anovaWithoutEffect{2}{34}{8.938}{<.001}) as well as the number of columns (\anovaWithoutEffect{2}{34}{6.087}{<.01}). The analysis could not find interaction effects (\anovaWithoutEffect{2}{68}{1.370}{>.05}). Post-hoc tests confirmed significantly lower ratings for the \row{3} conditions compared to \row{1} ($p<.001$) and \row{2} ($p<.05$) conditions. For the number of columns, the analysis found a significant higher rating for the \col{4} conditions compared to the \col{6} conditions ($p<.01$).


Again, the participants' ratings for all but the \row{3} conditions were predominantly positive (\reffig{fig:cheesyfoot/likert-study1} for all results).

\subsection{Qualitative Feedback}
\label{sec:cheesyfoot:experiment1:qualitative}

In general, all participants appreciated the idea of foot-based interactions with \acp{HMD} because it is \mpquote{easy to use}{P6, P8, P11, P12}, and \mpquote{not tiring [compared to the standard air-tap interface of the Hololens]}{P8, P17}.

Participants commented that the limitations of the used hardware - \mpquote{weight}{P1, P3, P4, P9}, \mpquote{field of view}{P5, P6, P7, P8, P15} - had a strong influence on their comfort because it forced them into an \pquote{unnatural}{14} posture during the study. P17 summarized: \enquote{Looking down all the time is a bit exhausting for the neck. So I wouldn’t use it for longer-lasting [interactions], but I would love this for quick and short [interactions]}.
