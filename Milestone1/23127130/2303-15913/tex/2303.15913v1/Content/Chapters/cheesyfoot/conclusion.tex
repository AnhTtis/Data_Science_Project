\section{Conclusion}

This chapter explored foot-taps as an input modality for \acp{HMD}. More precisely, the chapter investigated two different interaction styles: 1)~\emph{direct interaction} with interfaces that are displayed on the floor and require the user to look down to interact and 2)~\emph{indirect interaction} with interfaces that, although operated by the user's feet, are displayed as a floating window in front of the user. The results confirmed the viability of foot-taps for accurate and pleasant interaction with \acp{HMD}.

To conclude, this chapter added to the body of research on interacting with \acp{HMD} in multiple areas:

\begin{enumerate}
	\item This chapter contributed interaction techniques for \acp{HMD} \emph{on the move}. These techniques allow for hands-free interaction and, thus, enable novel use cases for \acp{HMD} when the user's hands are encumbered.
	\item Second, this chapter contributed two controlled experiments, proving the viability of the presented concepts for accurate, efficient, and joyful interactions. Therefore, the chapter presented a first evaluation of hands-free and mobile interfaces for interacting with \acp{HMD}, opening up a new research field for future developments.
	\item Third, based on the results of the two experiments, this chapter presented a set of guidelines that can inform future development of interfaces for interacting with \acp{HMD} \emph{on the go}. 
\end{enumerate}

\subsection{Integration}

\textfigH{cheesyfoot/alice}{Alice calls a taxi using foot-based interaction with her hands occupied.}

\interactionbox{alice_mindthetaps}{At the Traffic Light}{Alice is on her way home. She has taken a coffee with her and is still carrying her shopping, so both her hands are occupied. She reaches the exit of the mall and stands at a traffic light.  She wants to call an autonomous taxi for a ride home. She looks at the floor and chooses the taxi app with her foot (see figure \ref{fig:cheesyfoot/alice}). With a second foot tap, she confirms the pick-up location.
	
	The interaction takes place quickly and easily and purely by foot taps, without having to free the hands.}

The interaction technique presented in this chapter allows for fast interaction with with \emph{body-stabilized} interfaces leveraging our \emph{lower limbs} for input. This style of interaction supports \singleuserLower{}s with \discreteLower{} in situations, where the user's hands are encumbered (e.g., while carrying something during \mobilityLower{}). The two interaction styles presented thus cover a large number of situations, which could not be supported by the previous contributions of this thesis due to their focus on the \emph{upper limbs} (see chapter \ref{ch:proximity:merged} and chapter \ref{ch:cloudbits}) and, thus, contributes to the vision of ubiquitous \aroundbodyinteraction{} (see section \ref{sec:introduction:aroundbodyinteraction}).

As discussed in section \ref{sec:cheesyfoot:limitations}, there are multiple yet unexplored areas in the design space of such mobile and foot-based interactions that go beyond the scope of this work. Further modifications of the concept could potentially widen the applicability of foot-based interaction to more situations: By supporting continuous interaction (e.g., to adjust a slider precisely) and distinguishing the part of the foot that performed the interaction, the expressiveness could be considerably increased. The results presented in this chapter can provide a baseline for future work in these areas, providing the first step towards more comfortable and safer interaction with \acp{HMD} \emph{on the go}. 

\subsection{Outlook}

As a result of the Midas Tap problem identified in section \ref{sec:cheesyfoot:limitations:midas}, foot gestures, as presented in this chapter, cannot effectively be used for interaction while walking (and, thus, during an important part of \mobilityLower{}). Chapter \ref{ch:walktheline} discusses how this limitation can be mitigated by the use of \emph{world-stabilized} operated using the \emph{lower limbs}.




