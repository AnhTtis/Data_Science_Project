\section{Conclusion}

This chapter explored a way to leverage minimal changes in the user's walking path as an additional input modality for \acp{HMD}. The proposed concept augments lanes parallel to the user's walking direction on the floor in front of the user, representing individual options. The user can select one of these options by shifting the walking path sideways. The results of the controlled experiment confirmed the viability of such interfaces for fast, accurate, and fun interactions. Based on the results, the chapter presented a prototype implementation for Microsoft Hololens together with three example applications. 

To conclude, this chapter added to the body of research on interacting with \acp{HMD} in multiple areas:

\begin{enumerate}
	\item This chapter contributed an interaction technique for \acp{HMD} \emph{on the go}. The interaction is hands-free and can be performed while walking without interfering with locomotion. Thus, this work contributes to advancing the vision of ubiquitous interaction with information in a digitally augmented physical world.
	\item This chapter investigated the accuracy and efficiency of the envisioned interface in a controlled experiment, focusing on the influence of \factorLanes{} and the \factorTime{}. Furthermore, the chapter identified interdependencies between these factors that influence the future design of such interfaces. Therefore, this chapter contributed an initial evaluation of an interface designed specifically for interaction while walking, opening up a new field of research. For future research, the quantitative results contributed in this chapter provide a valuable baseline.
	\item Finally, this chapter presented three example applications illustrating the interaction with \acp{HMD} while walking in an urban context. Thus, this chapter demonstrated a safer and easier way to interact with information on the go compared to today's usage of smartphones.
\end{enumerate}

\subsection{Integration}
\textfigH{walktheline/alice}{Alice answers a text message by selecting a predefined message through walking on the corresponding lane.}


\interactionbox{alice_walktheline}{On the Go}{Alice is on her way to the pick-up point of the autonomous taxi. Meanwhile, she gets a message from Bob who thanks her for the great time they spent together. Automatically, four predefined responses appear as lanes in front of Alice, suggested as possible responses to Bob's message from her \ac{HMD}. Alice shifts her walkway to the side and, therby, chooses one of the options (see figure \ref{fig:walktheline/alice}).
	
Alice's interaction takes place as she walks without having to stop or free her hands.}

The interaction technique presented in this chapter allows for interaction with \emph{world-stabilized} interfaces leveraging our \emph{lower limbs} for input. The interaction is hands-free and takes place while walking, closing the gap left by previous interaction techniques for the lower limbs (see chapter \ref{ch:cheesyfoot}). This interaction technique thus supports users by providing \discreteLower{} in \mobilityLower{} situations, contributing to the vision of ubiquitous \aroundbodyinteraction{} (see section \ref{sec:introduction:aroundbodyinteraction}).

As described in section \ref{sec:walktheline:limitations}, the design space of such \interactionStyleBased{} interaction modalities for \acp{HMD} offers many more degrees of freedom that go beyond the scope of this work. The possible improvements of the concept listed in the section could further enable a broader applicability of the concept, e.g. by adapting the lanes to the real world or by integrating continuous interaction. However, the concepts as well as quantitative and qualitative results presented in this chapter provide a first step towards more comfortable and safe interaction with \acp{HMD} \emph{on the go} and, thus, can inform for future work in this area.

\subsection{Outlook}

This chapter presented the last main contribution of this work. Chapter \ref{ch:conclusion} concludes this work by integrating the individual contributions and presenting directions for future work.


