\section{Limitations and Future Work}
\label{sec:walktheline:limitations}

The presented results provide valuable insights to the applicability of \interactionStyleBased{} input for the interaction with \acp{HMD}. However, the study design, as well as the results of the experiment, impose some limitations and directions for future work.


\subsection{Continuous Interaction}

The experiment focused on discrete interaction steps, that is, the sequential calling of options. The approach was chosen to define the basic requirements for the design of such interfaces in terms of minimum width and time needed to interact. However, such \interactionStyleBased{} interfaces can also be of great use for continuous interaction as suggested in the \emph{Music Walker} example (see section \ref{sec:walktheline:prototype:music}): For such interfaces, a) the deviation of the user from the direct path or b) the time spent on a lane could be mapped directly to a cursor or other interface elements. Future work in this area is necessary to asses the accuracy and efficiency of such interfaces.


\subsection{Shapes beyond Straight Lines}

This chapter investigated the deviation from a straight line in front of the user as an input modality. In many real-world scenarios, however, a straight line may not be a suitable baseline for interaction (because of e.g., obstacles, directional changes of the user). Therefore, further work in this field is necessary to conclude on these challenges.



\subsection{Other modes of Locomotion}

Beyond walking investigated in this chapter, this type of interaction can also be of great use for other modes of locomotion such as jogging, cycling, riding e-scooters, or when using wheelchairs. In particular, a study of the influence of the real world (obstacles, oncoming traffic) on the interaction with such a system is of interest. Future work is necessary to assess the influence of other modes of locomotion and, thus, also speeds on the feasibility, accuracy, efficiency, and safety of such interfaces. In addition, when using driving equipment, users often have their hands on a steering mechanism (e.g., steering wheel) that can be included in the interaction. Further work is necessary to evaluate the possibilities of such an extended design space.