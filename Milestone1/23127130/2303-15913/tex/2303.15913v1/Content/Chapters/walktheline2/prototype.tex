\section{Implementation and Example Applications}
\label{sec:walktheline:prototype}

\begin{figure*}[ht!]
\subfloat[Assistant\label{fig:walktheline:uc:assistant}]
  {\includegraphics[width=.33\linewidth]{Content/Figures/walktheline/uc1}}\hfill
\subfloat[\prototypeCamera\label{fig:walktheline:uc:camera}]
  {\includegraphics[width=.33\linewidth]{Content/Figures/walktheline/uc2}}\hfill
\subfloat[Music Walker\label{fig:walktheline:uc:music}]
  {\includegraphics[width=.33\linewidth]{Content/Figures/walktheline/uc3}}
\caption{Three example applications to illustrate the presented concepts. \emph{Assistant} (a) allows users to explore nearby services (e.g., coffee shops, hotels, restaurants) after walking on the assistant lane. \emph{\prototypeCamera} (b) allows users to take pictures, apply filters and share the results to social media platforms. \emph{Music Walker} (c) allows users to walk through the playlist and select songs by leaving the lane. Further, users can continuously change the volume by walking on the respective lane.}
\end{figure*}


To eliminate influences of the restrictions of the current hardware generations of \acp{HMD}, the experiment used an artificial setup consisting of projectors and external tracking of the participants (see section \ref{sec:walktheline:methodology:setup}). However, the results of our experiment indicated that the accuracy of the inside-out tracking used by the Microsoft Hololens (average deviation of \SI{1.25}{cm}~\ncite{liu2018technical}) would be sufficient for a real-world implementation of the concepts. Therefore, based on the results of the experiment, this section presents the implementation of a \interactionStyleBased~input modality for the Microsoft Hololens. 

The software augments lanes onto the ground in front of the user, parallel to their walking path. Using the internal inside-out tracking of the Hololens, the system calculates the intersection between the orthogonal projection of the user's head position and the augmented lanes to identify the currently selected lane. The implementation works as a standalone application without modifications to the Hololens or additional external tracking. 

\subsection{Example Applications}

To show the practical applicability of the concept, this section presents three example applications: \nameref{sec:walktheline:prototype:assistant}, \nameref{sec:walktheline:prototype:camera} and \nameref{sec:walktheline:prototype:music}. The presented applications are not restricted to the general interaction concept presented in this chapter, but extend it by additional elements to make interaction possible in an urban context.

\subsubsection{Assistant}
\label{sec:walktheline:prototype:assistant}


Assistant is a personal assistance service, which - similar to commercial assistance solutions like Google Assistant, Alexa, or Cortana - can offer personalized recommendations. For this, an unobtrusively visualized assistant lane is displayed at the right edge of the user's field of view. This lane serves as a minimalist means of starting to interact with the system. If users want to access the service, they shift their walking path to the lane. This movement opens various options that allow the user to access personalized local services such as recommended restaurants or shops (see Figure \ref{fig:walktheline:uc:assistant}). By selecting an element through walking on it, the user can walk further down the options tree of a cascading menu.

\subsubsection{\prototypeCamera}
\label{sec:walktheline:prototype:camera}

By entering the photo lane (i.e., a specific lane within \emph{Assistant} or a standalone lane), the user can activate the camera. Exiting this lane to the \enquote{take a picture} side starts a countdown to take a picture of the current view of the participant without the augmented content. In the following, the user can apply various filters to the image, which are displayed as new lanes. The effect of each filter is previewed as soon as the user enters the corresponding lane (see Figure \ref{fig:walktheline:uc:camera}). By walking on a lane for a longer time, the user can select one of the filters and, in the next step, share the edited photo to different social media platforms, again visualized as newly appearing lanes.






\subsubsection{Music Walker}
\label{sec:walktheline:prototype:music}


Besides discrete inputs investigated in the experiment, walking-based interfaces can also be used to control continuous interfaces (see section \ref{sec:walktheline:limitations}). This section presents \emph{Music Walker} as an example of such a type of interaction.

Music Walker is a music player application. The user can continuously change the volume by walking on the \emph{volume up} or \emph{volume down} lane. The longer the user stays on a lane, the further the volume is increased or decreased, respectively. Further, the \emph{playlist} lane allows the user to walk up a list of the upcoming tracks. Leaving the lane allows the user to select a new song. Figure \ref{fig:walktheline:uc:music} depicts the interaction.