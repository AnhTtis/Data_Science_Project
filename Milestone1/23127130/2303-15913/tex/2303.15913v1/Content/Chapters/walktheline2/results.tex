\section{Results}
\label{sec:walktheline:results}


The following section reports the results of the controlled experiment investigating the research questions RQ1 - RQ3. The analysis of the data was performed as described in section \ref{sec:rw:methodology}. 

\newpage
\subsection{Accuracy}

\colfig{walktheline/accuracy_small}{The measured accuracy rates in the controlled experiment. All error bars depict the standard error.}



The accuracy of participants was analyzed as the rate of successful trials. The analysis revealed that the \emph{number of lanes} had a significant (\anova{2}{34}{27.05}{<.001}{.134}) influence on the participants' accuracy with a medium effect size. Post-hoc tests confirmed significant differences between the \lane{8} (\emmP{85.8}{2.2}) and \lane{16} (\emmP{70.1}{2.2}) conditions as well as between the \lane{12} (\ac{EMM} \emmP{80.4}{2.2}) and \lane{16} conditions (both $p<.001$).

Further, the analysis showed a significant (\anovaCor{1.44}{24.45}{37.57}{<.001}{.719}{.307}) effect for the \emph{selection time} on the participants' accuracy with a large effect size. Post-hoc tests confirmed significant differences between the \seltime{1} (\emmP{65.2}{2.4}) and both, the \seltime{2} (\emmP{88.5}{2.4}) and the \seltime{3} (\emmP{82.6}{2.4}) conditions (both $p<.001$).


Additionally, the analysis showed significant (\anovaCor{2.57}{43.74}{4.28}{<.05}{.643}{.033}) interaction effects between both factors with a small effect size.

In the experiment, the measurements showed accuracy rates ranging from \emmP{93.6}{3.1} (\lane{8}, \seltime{3}) to \emmP{58.9}{3.1} (\lane{16}, \seltime{1}). Table \ref{tab:walktheline/accuracy_table} lists the measured accuracy rates for the tested conditions, figure \ref{fig:walktheline/accuracy_small} depicts the mean values.

\begin{table}[ht]
\centering
\caption{Accuracy for different prompting strategies (averaged on $5_{0,\cdots,4}$ different seeds, where Top-$k$ and Greedy indicate \topk with $k$ demonstrations and \greedy respectively).}
\label{tab:four-topk-greedy}
\resizebox{1.0\textwidth}{!}{\begin{tabular}{c|c|ccc||ccc} \toprule
\multirow{2}{*}{\textbf{Model}}                  & \multirow{2}{*}{\textbf{Dataset}}                  & \multirow{2}{*}{\textbf{Random}} & 
 \multirow{2}{*}{\textbf{Diversity}} & \multirow{2}{*}{\textbf{Similarity\tnote{1} }} & \multicolumn{3}{c}{\textbf{Ours}} \\ & & & &  &\textbf{Top-2} & \textbf{Top-4} & \textbf{Greedy}                  \\ \midrule \multirow{7}{*}{BLOOM (176B)} & {SST2}   & $92.7_{2.3}$&$\best{95.0_{0.9}}$ &$94.0_{0.9}$   & ${94.6_{0.5}}$      & $93.8_{2.1}$     & ${91.2_{4.0}}$  \\ \cmidrule{2-8} &    {AGNews}                            & $73.9_{5.9}$  & $70.2_{10.1}$ &$74.8_{3.8}$  &$75.4_{2.2}$      & $74.8_{2.3}$       & $\best{79.6_{1.4}}$     \\ \cmidrule{2-8}
                                & {TREC}   & $47.9_{14.6}$&$46.0_{8.7}$ &$31.4_{3.1}$   &$55.4_{13.3}$            & $39.2_{19.3}$     & $\best{66.8_{2.5}}$  \\ \cmidrule{2-8}
                                & {RTE}  & $62.4_{4.2}$ &$\best{69.2_{1.9}}$&$67.2_{3.5}$  &$55.6_{1.0}$            & $57.6_{1.9}$      & ${63.0_{2.1}}$ \\  \cmidrule{2-8}
                                & {CoLA}   & {$68.4_{4.8}$} & \best{$71.0_{3.7}$} & $69.8_{2.5}$ &$66.4_{8.6}$            & $66.8_{3.7}$       & ${68.2_{6.2}}$      \\  \midrule\multirow{5}{*}{LLaMA (33B)}& {SST2}   & $82.5_{11.8}$&$\best{90.0_{2.7}}$ &$72.8_{4.4}$   & ${82.0_{11.1}}$      & $80.0_{12.2}$     & ${85.6_{8.2}}$  \\ \cmidrule{2-8} &   {AGNews}           & {$75.2_{5.0}$}            &$75.0_{5.1}$      & {$75.0_{2.4}$}        &$73.2_{3.9}$      & $69.8_{4.4}$            & $\best{76.4_{4.6}}$    \\ \cmidrule{2-8}
                                & {TREC} & $68.1_{11.1}$ &$68.2_{4.7}$            & $60.6_{3.4}$ &$71.4_{11.1}$            & $57.8_{17.3}$        & $\best{80.2_{5.3}}$  \\ \cmidrule{2-8}
                                & {CoLA}   & $66.9_{11.0}$ & $68.8_{6.8}$ & $72.8_{2.0}$ &$63.8_{13.3}$            & $69.8_{3.9}$       & $\best{70.6_{4.2}}$      \\ \midrule\multirow{5}{*}{LLaMA (65B)} & {SST2}   & $90.0_{7.7}$&$90.8_{9.0}$ &$87.4_{3.1}$   & ${88.2_{8.6}}$      & $\best{95.8_{1.5}}$     & ${87.8_{9.0}}$  \\ \cmidrule{2-8}
                                &    {AGNews}                          & $76.8_{5.0}$   &$\best{78.2_{3.1}}$      & \best{$78.2_{1.8}$}  &${77.0_{3.4}}$      & $76.2_{4.9}$           & $76.0_{4.0}$    \\ \cmidrule{2-8}
                                & {TREC} & $63.6_{14.2}$ &$65.2_{10.9}$            & $64.0_{5.5}$   &$65.8_{13.0}$            & $57.4_{19.9}$       & $\best{74.0_{12.2}}$  \\ \cmidrule{2-8}
                                & {CoLA} & $66.2_{9.8}$ &$62.6_{8.6}$& $59.2_{14.0}$  &$67.6_{11.7}$            & $62.6_{6.5}$       & $\best{72.0_{4.5}}$      \\ \bottomrule
\end{tabular}}
\end{table}


\subsection{Stabilizing Error}

\colfigH{walktheline/overshoot_small}{The measured stabilizing error rates in the controlled experiment. All error bars depict the standard error.}


The stabilizing error rate was calculated by counting the number of trials when participants left the target lane after initially reaching it. The analysis showed a significant (\anova{2}{34}{127.3}{<.001}{.45}) influence of the \emph{number of lanes} on the stabilizing error rate with a large effect size. Post-hoc tests confirmed significantly higher stabilizing error rates for higher \emph{numbers of lanes} (and thus smaller lanes) between all levels (\lane{8}: \emmP{16.2}{3.5}, \lane{12}: \emmP{36.8}{3.5}, \lane{16}: \emmP{61.7}{3.5}, all $p<.001$). 

Further, the \emph{selection time} also proved to have an significant (\anova{2}{34}{67.07}{<.001}{.164}) influence on the stabilizing error rate in the experiment with a large effect size. Post-hoc tests confirmed significantly higher stabilizing error rates for longer selection times between all levels (\seltime{1}: \emmP{24.3}{3.3}, \seltime{2}: \emmP{38.7}{3.3}, \seltime{3}: \emmP{51.8}{3.3}, all $p<.001$).

Lastly, the analysis also showed significant (\anova{4}{68}{6.73}{<.001}{.023}) interaction effects between both factors with a small effect size. 

The analysis found stabilizing error rates ranging from \emmP{7.8}{4.1} (\lane{8}, \seltime{1}) to \emmP{79.8}{4.1} (\lane{16}, \seltime{3}). Table \ref{tab:walktheline/overshoot_table} lists the measured accuracy rates for the tested conditions, figure \ref{fig:walktheline/overshoot_small} depicts the mean values.


\begin{table}
	\centering
	
	\begin{tabularx}{\linewidth}{ccYYYY}
		& &   &   & \multicolumn{2}{c}{\textbf{95\% CI\textsuperscript{1}}}\\
		\cmidrule(lr){5-6}
		\textbf{Number of Lanes} & \textbf{Selection Time} & $\pmb{\mu}$ & $\pmb{\sigma}$ & \textbf{Lower} & \textbf{Upper}\\
		\midrule
		8 & \seltime{1} & .074 & .116 & .0   & .155 \\
		  & \seltime{2} & .172 & .137 & .091 & .254 \\
		  & \seltime{3} & .240 & .159 & .158 & .321 \\
		12& \seltime{1} & .250 & .141 & .169 & .332 \\
		  & \seltime{2} & .338 & .218 & .256 & .419 \\
		  & \seltime{3} & .516 & .219 & .435 & .597 \\
		16& \seltime{1} & .404 & .187 & .323 & .486 \\
		  & \seltime{2} & .649 & .191 & .568 & .731 \\
		  & \seltime{3} & .798 & .145 & .716 & .879 \\
		
		\bottomrule
	\end{tabularx}
	
	\caption{The stabilizing error rates per combination of \emph{selection time} and \emph{number of lanes} as measured in the experiment. The table reports the recorded mean values $\mu$ together with the standard deviation $\sigma$. \textsuperscript{1} The confidence interval CI is based on the fitted \ac{EMM} model.}
	\label{tab:walktheline/overshoot_table}
\end{table}

\subsection{Task Completion Time}
\label{sec:walktheline:results:tct}

\colfig{walktheline/tct_small}{The measured task-completion times (in seconds) in the controlled experiment. All error bars depict the standard error.}


The \ac{TCT} was measured as the time to successful activation of a lane after subtracting the respective selection time to keep the \acp{TCT} comparable. The time was measured from the moment the target was displayed to the participant. The analysis only considered the \acp{TCT} of the successful trails, as the different accuracy rates would otherwise influence the results. 

The analysis showed a significant (\anova{2}{34}{117.8}{<.001}{.262}) influence of the \emph{number of lanes} on the \ac{TCT} with a large effect size. Post-hoc tests confirmed rising \acp{TCT} for higher numbers of lanes between all levels (\lane{8}: \emmSi{1.81}{.07}{s}, \lane{12}: \emmSi{2.11}{.07}{s}, \lane{16}: \emmSi{2.79}{.07}{s} all $p<.001$). 

Interestingly, despite subtracting of the selection time from the \ac{TCT}, the analysis also showed a significant (\anova{2}{34}{123.3}{<.001}{.413}) effect of the \emph{selection time} on the \ac{TCT} with a large effect size. Post-hoc tests confirmed significantly higher \acp{TCT} for higher selection times between all levels (\seltime{1}: \emmSi{1.57}{.08}{s}, \seltime{2}: \emmSi{2.30}{.08}{s}, \seltime{3}: \emmSi{2.83}{.08}{s}, all $p<.001$). As depicted in figure \ref{fig:walktheline/tct_small}, the \acp{TCT} for the different selection times are close together for the \lane{8} conditions. For higher numbers of lanes, the \acp{TCT} grow faster for longer selection times.

Further, the analysis again showed significant (\anovaCor{2.71}{46.06}{25.3}{<.001}{.677}{.073}) interaction effects between the factors with a medium effect size.%

The graphical analysis of the \acp{TCT} showed strong visual correlations with the stabilizing error rates as presented above (see Figure \ref{fig:walktheline/overshoot_small} and \ref{fig:walktheline/tct_small}). Calculating Pearson's $r$ supported the visual impression by confirming a very strong~\ncite{Evans1996} correlation between \emph{stabilizing error rate} and \emph{\ac{TCT}} (\pearson{.925}{<.001}).

The analysis found \acp{TCT} ranging from \emmSi{1.41}{.10}{s} (\lane{8}, \seltime{1}) to \emmSi{3.71}{.08}{s} (\lane{16}, \seltime{3}). Table \ref{tab:walktheline/tct_table} and figure \ref{fig:walktheline/tct_small} depict the measured \acp{TCT} for all conditions.


\newcommand{\myunit}[1]{#1~s}

\begin{table}
	\centering
	
	\begin{tabularx}{\linewidth}{ccYYYY}
		& &   &   & \multicolumn{2}{c}{\textbf{95\% CI\textsuperscript{1}}}\\
		\cmidrule(lr){5-6}
		\textbf{Number of Lanes} & \textbf{Selection Time} & $\pmb{\mu}$ & $\pmb{\sigma}$ & \textbf{Lower} & \textbf{Upper}\\
		\midrule
		8 & \seltime{1} & \myunit{1.410} & \myunit{.170} & \myunit{1.216} & \myunit{1.605} \\
		  & \seltime{2} & \myunit{1.858} & \myunit{.226} & \myunit{1.663} & \myunit{2.052} \\
	      & \seltime{3} & \myunit{2.149} & \myunit{.409} & \myunit{1.955} & \myunit{2.344} \\
		12& \seltime{1} & \myunit{1.578} & \myunit{.187} & \myunit{1.384} & \myunit{1.773} \\
		  & \seltime{2} & \myunit{2.111} & \myunit{.327} & \myunit{1.916} & \myunit{2.305} \\
		  & \seltime{3} & \myunit{2.643} & \myunit{.491} & \myunit{2.448} & \myunit{2.837} \\
		16& \seltime{1} & \myunit{1.732} & \myunit{.123} & \myunit{1.537} & \myunit{1.926} \\
		  & \seltime{2} & \myunit{2.929} & \myunit{.548} & \myunit{2.735} & \myunit{3.124} \\
		  & \seltime{3} & \myunit{3.705} & \myunit{.773} & \myunit{3.511} & \myunit{3.900} \\
		
		\bottomrule
	\end{tabularx}
	
	\caption{The task-completion times (in seconds) per combination of \emph{selection time} and \emph{number of lanes} as measured in the experiment. The table reports the recorded mean values $\mu$ together with the standard deviation $\sigma$. \textsuperscript{1} The confidence interval CI is based on the fitted \ac{EMM} model.}
	\label{tab:walktheline/tct_table}
\end{table}

\subsection{Walked Distance}


\colfig{walktheline/distance_small}{The measured distances to selection in the controlled experiment. All error bars depict the standard error.}

To take into account different walking speeds of the participants, the analysis assessed the walking distance necessary to activate a target. Similar to the \ac{TCT}, the analysis only considered the distance that was necessary to select a lane without the distance walked during the \factorTime~of the respective condition. Therefore, the system measured the distance the participants walked from the beginning of the task within the \ac{TCT} as defined above. %

The analysis showed a significant (\anovaCor{1.45}{24.65}{84.0}{<.001}{.725}{.159}) influence of the \emph{number of lanes} on the distance with a large effect size. Post-hoc tests confirmed significantly higher distances for higher \emph{numbers of layers} between all levels (\lane{8}: \emmSi{2.06}{.18}{m}, \lane{12}: \emmSi{2.43}{.18}{m}, \lane{16}: \emmSi{3.11}{.18}{m} , all $p<.001$).

As for the \ac{TCT}, the analysis also showed a significant (\anovaCor{1.41}{23.89}{102.5}{<.001}{.703}{.263}) effect for the \emph{selection time} on the distance with a large effect size. Again, post-hoc tests confirmed significantly higher distances for higher selection times between all levels (\seltime{1}: \emmSi{1.80}{.18}{m}, \seltime{2}: \emmSi{2.63}{.18}{m}, \seltime{3}: \emmSi{3.17}{.18}{m}, all $p<.001$).

Further, the analysis showed significant (\anovaCor{2.96}{50.32}{21.9}{<.001}{.74}{.042}) interaction effects between both factors with a small effect size. 

As for the \ac{TCT}, the visual analysis of the measured distances showed correlations with the stabilizing error rates (see Figure \ref{fig:walktheline/overshoot_small} and \ref{fig:walktheline/distance_small}). Again, calculating Pearson's $r$ supported the visual impression, confirming a strong correlation between the \emph{stabilizing error rate} and the needed \emph{distance} to walk (\pearson{.924}{<.001}).

The analysis found distances ranging from \emmSi{1.62}{.19}{m} (\lane{8}, \seltime{1}) to \emmSi{4.07}{.19}{m} (\lane{16}, \seltime{3}). Table \ref{tab:walktheline/distances_table} and figure \ref{fig:walktheline/distance_small} depicts the measured walking distances for all conditions in the experiment.

The results presented in this section can only provide an approximation to a potential lower limit of interaction distances. In an urban environment, environmental influences such as obstacles or road conditions can influence these results.


\newcommand{\meterunit}[1]{#1~m}

\begin{table}
	\centering
	
	\begin{tabularx}{\linewidth}{ccYYYY}
		& &   &   & \multicolumn{2}{c}{\textbf{95\% CI\textsuperscript{1}}}\\
		\cmidrule(lr){5-6}
		\textbf{Number of Lanes} & \textbf{Selection Time} & $\pmb{\mu}$ & $\pmb{\sigma}$ & \textbf{Lower} & \textbf{Upper}\\
		\midrule
		8 & \seltime{1} & \meterunit{1.616} & \meterunit{.505} & \meterunit{1.217} & \meterunit{2.015} \\
	      & \seltime{2} & \meterunit{2.153} & \meterunit{.648} & \meterunit{1.753} & \meterunit{2.552} \\
		  & \seltime{3} & \meterunit{2.402} & \meterunit{.707} & \meterunit{2.002} & \meterunit{2.801} \\
		12& \seltime{1} & \meterunit{1.811} & \meterunit{.549} & \meterunit{1.411} & \meterunit{2.210} \\
		  & \seltime{2} & \meterunit{2.457} & \meterunit{.748} & \meterunit{2.057} & \meterunit{2.856} \\
		  & \seltime{3} & \meterunit{3.034} & \meterunit{1.02} & \meterunit{2.634} & \meterunit{3.433} \\
		16& \seltime{1} & \meterunit{1.981} & \meterunit{.580} & \meterunit{1.582} & \meterunit{2.381} \\
		  & \seltime{2} & \meterunit{3.291} & \meterunit{1.08} & \meterunit{2.892} & \meterunit{3.690} \\
		  & \seltime{3} & \meterunit{4.066} & \meterunit{1.25} & \meterunit{3.667} & \meterunit{4.466} \\
		
		\bottomrule
	\end{tabularx}
	
	\caption{The walked distances to complete a task (in meters) per combination of \emph{selection time} and \emph{number of lanes} as measured in the experiment. The table reports the recorded mean values $\mu$ together with the standard deviation $\sigma$. \textsuperscript{1} The confidence interval CI is based on the fitted \ac{EMM} model.}
	\label{tab:walktheline/distances_table}
\end{table}

\subsection{TLX}

\colfig{walktheline/tlx_small}{The \acl{RTLX} measured in the controlled experiment. All error bars depict the standard error.}


To assess the differences in the mental load induced by the two factors, the analysis considered the influence of the factors on the \acf{RTLX}. The analysis showed a significant (\anovaCor{1.43}{24.36}{18.96}{<.001}{.716}{.104}) influence of the \emph{number of lanes} with a medium effect size. Post-hoc tests confirmed significantly higher values for the \lane{16} (\emm{43.9}{2.99}) conditions compared to both, the \lane{8} (\emm{29.4}{2.99}) and \lane{12} (\emm{33.7}{2.99}) conditions (both $p<.001$).

Further, the analysis showed a significant (\anovaCor{1.18}{20.01}{21.7}{<.001}{.588}{.182}) effect for the \emph{selection time} on the \ac{RTLX} with a large effect size. Post-hoc tests confirmed significantly higher values for the \seltime{1} (\emm{47.0}{3.16}) conditions compared to the \seltime{2} (\emm{29.1}{3.16}) and \seltime{3} (\emm{30.9}{3.16}) conditions (both $p<.001$).

Lastly, the analysis showed significant (\anova{4}{68}{3.12}{<.05}{.022}) interaction effects between the factors with a small effect size. 

The analysis found \ac{RTLX} values ranging from \emm{20.7}{3.8} (\lane{8}, \seltime{3}) to \emm{51.9}{3.8} (\lane{16}, \seltime{1}). Table \ref{tab:walktheline/tlx_table} and figure \ref{fig:walktheline/tlx_small} show the mean raw TLX values for the tested conditions.


\begin{table}
	\centering
	
	\begin{tabularx}{\linewidth}{ccYYYY}
		& &   &   & \multicolumn{2}{c}{\textbf{95\% CI\textsuperscript{1}}}\\
		\cmidrule(lr){5-6}
		\textbf{Number of Lanes} & \textbf{Selection Time} & $\pmb{\mu}$ & $\pmb{\sigma}$ & \textbf{Lower} & \textbf{Upper}\\
		\midrule
		8 & \seltime{1} & 43.47 & 22.50 & 35.87 & 51.07 \\
		  & \seltime{2} & 24.12 & 12.03 & 16.52 & 31.72 \\
		  & \seltime{3} & 20.69 & 11.17 & 13.09 & 28.30 \\
		12& \seltime{1} & 45.65 & 17.54 & 38.05 & 53.25 \\
		  & \seltime{2} & 27.87 & 15.20 & 20.27 & 35.47 \\
		  & \seltime{3} & 27.59 & 16.88 & 19.99 & 35.19 \\
		16& \seltime{1} & 51.85 & 20.73 & 44.25 & 59.45 \\
		  & \seltime{2} & 35.32 & 10.89 & 27.72 & 42.93 \\
		  & \seltime{3} & 44.49 & 13.98 & 36.89 & 52.09 \\
		
		\bottomrule
	\end{tabularx}
	
	\caption{The raw \ac{RTLX} values per combination of \emph{selection time} and \emph{number of lanes} as measured in the experiment. The table reports the recorded mean values $\mu$ together with the standard deviation $\sigma$. \textsuperscript{1} The confidence interval CI is based on the fitted \ac{EMM} model.}
	\label{tab:walktheline/tlx_table}
\end{table}

\subsection{Height}

To assess the influence of the participants height and, thus, differences in step sizes, the analysis assessed the between subject effects of the \emph{height} on the independent variables. However, the analysis did not show any influence on the \emph{accuracy} (\anova{1}{16}{1.78}{>.05}{.031}), the stabilizing error (\anova{1}{16}{1.31}{>.05}{.041}), the \ac{TCT} (\anova{1}{16}{.09}{>.05}{.002}) nor on the distance (\anova{1}{16}{1.15}{>.05}{.053}).

\subsection{Location of the Target Lane}


The analysis assessed the effect of the location of the target lane by comparing the measurements grouped by \emph{outer} (i.e., lanes on the far left and right as well as the lanes next to the central \emph{zero-lane}) and \emph{inner} (i.e., all other lanes) target lanes.

The analysis showed a significant influence of the target location on the accuracy (\anova{1}{17}{35.95}{<.001}{.058}) with a small effect size. Post-hoc tests confirmed significantly higher accuracy rates for outer (\emmP{85.0}{2.01}) compared to inner (\emmP{75.4}{2.01}) target lanes ($p<.001$).

Besides the accuracy, the analysis did not show any significant effects for the stabilizing error rate (\anova{1}{17}{3.49}{>.05}{.003}), the \ac{TCT} (\anova{1}{17}{2.43}{>.05}{.001}) nor the walked distance (\anova{1}{17}{1.92}{>.001}{.001}).

\subsection{Questionnaire}


After each condition, the participants filled a questionnaire asking questions regarding their experiences on a 5-point Likert-scale (1: strongly disagree, 5: strongly agree). The following section analyses the participants' answers.

\subsubsection{Confidence}

\textfigH{walktheline/likert_q1_diss}{The answers of the participants the \emph{Confidence} question in the questionnaire.}

The Likert questionnaire asked the participants about their \emph{confidence} to have successfully hit the target lanes in the condition. The analysis showed a significant (\anovaWithoutEffect{2}{34}{61.92}{<.001}) effect of the \emph{number of lanes} on the participants' confidence. Post-hoc tests confirmed significantly lower approval for the \lane{16} conditions compared to the \lane{8} and \lane{12} conditions (both $p<.001$).

Additionally, the analysis showed a significant (\anovaWithoutEffect{2}{34}{16.67}{<.001}) effect for the \emph{selection time} on the participants' confidence. Post-hoc tests revealed significantly lower approval rates for the \seltime{1} conditions compared to the \seltime{2} and \seltime{3} conditions (both $p<.001$).

The analysis found significant (\anovaWithoutEffect{4}{68}{6.11}{<.01}) interaction effects. Figure \ref{fig:walktheline/likert_q1_diss} depicts all the answers of the participants.

\subsubsection{Convenience}

\textfigH{walktheline/likert_q2_diss}{The answers of the participants the \emph{convenience} question in the questionnaire.}

Further, the Likert questionnaire asked the participants if the combination of \emph{number of lanes} and \emph{selection time} was convenient to use. The analysis showed a significant (\anovaWithoutEffect{2}{34}{48.53}{<.001}) effect for the \emph{number of lanes} on the participants' ratings of the convenience. Post-hoc tests revealed significantly higher convenience ratings for \lane{8} and \lane{12} conditions compared to \lane{16} conditions (both $p<.001$).

Further, the analysis found a significant (\anovaWithoutEffect{2}{34}{11.47}{<.001}) influence of the \emph{selection time} on the ratings. Post-hoc tests confirmed significantly higher approval ratings for \seltime{2} ($p<.001$) and \seltime{3} ($p<01$) compared to \seltime{1} conditions.

The analysis found no interaction effects (\anovaWithoutEffect{4}{68}{0.16}{>.05}). Figure \ref{fig:walktheline/likert_q2_diss} depicts all the answers of the participants.

\subsubsection{Willingness to Use}

\textfigH{walktheline/likert_q3_diss}{The answers of the participants the \emph{Willingness to Use} question in the questionnaire.}

As a last question, the questionnaire asked the participants if they would like to use this combination of the \emph{number of lanes} and the \emph{selection time} for interacting with \acp{HMD}. The analysis showed a significant (\anovaWithoutEffect{2}{34}{28.13}{<.001}) influence of the \emph{number of lanes} on the participants' ratings. Post-hoc tests revealed significantly rising approval ratings for lower numbers of lanes between all levels ($p<.01$ comparing \lane{8} and \lane{12}, otherwise $p<.001$).

Further, the analysis unveiled a significant (\anovaWithoutEffect{2}{34}{17.86}{<.001}) influence of the \emph{selection time} on the ratings. Post-hoc tests showed significantly lower approval rates for \seltime{1} conditions compared to \seltime{2} and \seltime{3} conditions (both $p<.001$).

The analysis did not indicate any significant (\anovaWithoutEffect{4}{68}{1.59}{>.05}) interaction effects between the two factors. Figure \ref{fig:walktheline/likert_q2_diss} depicts all the answers of the participants.

\subsection{Qualitative Results}

In general, all participants showed strong approval for the idea of hands-free interaction with \acp{HMD} through walking. Asked for the reasons, participants told that it felt \pquote{fun}{8}, \pquote{novel}{15}, \pquote{fast}{12} and \pquote{convenient}{1,8}, and would be especially \pquote{helpful [...] while doing other things}{8}.

The participants noted that the \factorLanes{} had a strong influence on their experience. P14 summarized: \enquote{With many lanes it is frustrating. I have to concentrate a lot to accomplish that.} P8 added: \enquote{With the small lanes, it almost feels like I have to walk on a balance beam.} 

Concerning the \factorTime{}, the opinions of the participants diverged. While almost all participants agreed that \seltime{1} is \mpquote{too short}{P1, P2, P8, P13, P17}, both other selection times were equally popular. P7 explained the problem of identifying the \enquote{best} selection time: \enquote{It's complicated. With the thin lanes, I'm annoyed [...] by too much [selection] time because balancing is difficult. With the wide lanes, on the other hand, I find longer [selection] times easier.}