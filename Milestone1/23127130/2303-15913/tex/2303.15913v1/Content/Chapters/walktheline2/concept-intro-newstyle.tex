\section{Concept}
\label{sec:walktheline:concept}
\label{sec:walktheline:intro-short}

\cptteaser{walktheline/teaser_big}{Walk the Line leverages lateral shifts of the walking path as an input modality for \acp{HMD}. Options are visualized as \emph{lanes} on the floor parallel to the user's walking path. Users select options by shifting the walking path sideways. Following a selection, sub-options of a cascading menu appear as new \emph{lanes}.}

While walking, we routinely respond to changes in the environment by adapting the trajectory of our walking path to avoid obstacles, such as oncoming pedestrians or pavement damages. These trajectory changes occur quickly and accurately and without changing the original direction of travel, but by laterally shifting the walking path. This chapter argues that such lateral shifts of the user can be leveraged as a novel input modality for interaction \emph{on the go}.

Today, a large number of pedestrians interact with their smartphones as they walk, losing touch with the world around them~\ncite{Lin2017a}. Similar to distracted driving, distracted walking also leads to potentially dangerous situations: The lack of (visual) attention causes pedestrians to walk into obstacles, to collide with other persons or otherwise endanger themselves~\ncite{Schabrun2014a, Thompson2013}. As another approach to interaction while walking, related work proposed voice-based interfaces. However, such systems may perform badly depending on background noise and have social implications~\ncite{Koelle2017, Starner2002}. In addition, voice-based interfaces interfere with the communication between people, whether it is a local conversation or a phone call. To overcome the limitations of interaction while walking, research proposed ways to mitigate for the situational hindrances~\ncite{sears03} induced by walking, leveraging increased button sizes~\ncite{Kane2008} or content stabilization~\ncite{Rahmati2009}. 

The contribution of this chapter goes beyond the state-of-the-art by not only compensating for such situational hindrances but by actively exploiting the process of locomotion as an input modality: Mobile \ac{AR} interfaces potentially enable more comfortable and safer interaction while walking, as the visual attention is no longer captured purely by a display~\ncite{Lucero2014}. This chapter proposes to use \acp{HMD} to visualize different input options as augmented lanes on the ground parallel to the walking path of the user. By laterally shifting the path onto a lane and, subsequently, walking on the lane, users can select an option (see figure~\ref{fig:walktheline/teaser_big}).

For this, this chapter considers a system that displays multiple lanes parallel to the walking path of the user. Each lane represents an option the user can select. The lanes can be arranged on both sides of the user's walking path (see figure \ref{fig:walktheline/teaser_big}). The specific visualization of the lanes can be tailored to the application and adapted to the current situation of the user. For example, it can contain icons or text or can be connected to bubbles floating in the air which describe the information to be selected.

To interact with the system, users shift their path sideways until they walk on the desired option lane without a need to change their walking speed, just as they would when avoiding an obstacle on the sidewalk. The system highlights the lane the user is currently walking on by changing the visualization (e.g., the color) of the respective lane. This change affects the entire lane, which is also visible in front of the user. Therefore, users do not have to look to the ground to interact with the system, but can keep their head up. By walking along one of the lanes for a certain period of time, the respective option can be selected, analogously to the concept of selection by dwell time in eye-gaze interaction~\ncite{Qian2017}. This section refers to the time-to-select that users need to walk on a lane as the \emph{selection time}. In our concept, this \emph{selection time} is visualized to the user by changing the opacity of the lanes: While walking on a lane, all other lanes are gradually faded out.

In addition to the option lanes, the concept proposes a non-active \emph{null lane} that covers the path directly in front of the user, which remains free. Therefore, if users continue walking straight ahead without adjusting their path, the system does not interpret this as an interaction and does not trigger any actions.