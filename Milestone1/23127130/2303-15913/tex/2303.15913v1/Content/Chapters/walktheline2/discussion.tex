\section{Discussion}
\label{sec:walktheline:discussion}

The results of the controlled experiment suggest that the usage of lateral shifts of the walking path of users provides a viable interaction technique for \acp{HMD}. The analysis showed the highest accuracy rates ($\approx 94\%$) for 8 interaction lanes (with an additional inactive \emph{zero lane} in the middle, resulting in a lane width of ~\SI{11}{cm}) with a selection time of \seltime{2}. The following section discusses the results of the experiment with respect to the research questions as presented above.

\subsection{RQ1: Influence of the Number of Lanes}

The analysis revealed a strong dependence of both, the accuracy and the efficiency, on the \emph{number} and - since the experiment varied the number of lanes on a fixed-width area - the \emph{width} of lanes. Higher numbers of lanes reduced the accuracy across all conditions. Further, higher numbers of lanes also led to higher \acp{TCT} and also increased the required walking distance to activate a target, decreasing efficiency.

The reduced accuracy and efficiency for higher numbers of lanes can be attributed to the higher stabilizing error rates through overshooting and swing-back errors caused by thinner lanes. This effect was further amplified by the natural lateral oscillation of the head that occurs during walking: The steps cause the head to constantly move slightly to the left and right of the actual path while walking, causing participants to oscillate out of the target lane. At comfortable walking speeds of around 1-1.5 m/s~\ncite{bohannon1997}, this effect occurs at the stride frequency of approximately \SI{1}{Hz}, and is responsible for a lateral translation of \SIrange{10}{15}{mm} in each direction~\ncite{MOORE2006}. This equates to 9-14\% (for the \lane{8} conditions) and up to 17-25\% (for the \lane{16} conditions) of the lane widths tested in the experiment. 

Further, the analysis showed significantly higher \ac{RTLX} values indicating a higher mental load for higher numbers of lanes. The results of the Likert-questionnaires supported the general discomfort of the participants with higher numbers of lanes. The participants answered all three questions - regarding confidence, convenience, and willingness to use - with significantly lower scores for \lane{16} compared to \lane{12} and \lane{8} conditions. The qualitative feedback of the participants further supported these findings, most of whom were in favor of lower lane numbers. 

\subsection{RQ2: Influence of the Selection Time}

Concerning the \factorTime, the analysis found a more complicated relationship to the accuracy than with the \factorLanes. The analysis showed that different selection times had a strong influence on the accuracy. Surprisingly, the middle selection time (\seltime{2}) was the one that achieved the highest accuracy rates. 

On the one hand, too short selection times led to participants accidentally selecting wrong lanes, as they spent too much time over an intermediate lane when changing lanes, thus decreasing accuracy. On the other hand, too long selection times increased the chance of participants accidentally leaving the target lane before the selection, as indicated by increased stabilizing error rates for higher selection times measured in the experiment. The observations during the experiment showed that - after such an incident - participants very carefully re-approached the target lane in order not to overshoot again, spending long periods of time on the adjacent lane. This behavior increased the chance of accidentally selecting a wrong target lane and, thereby, again reduced the accuracy.

Interestingly, the analysis also showed an influence of the \factorTime{} on the efficiency of participants, even though the respective \factorTime{} was explicitly subtracted from the \acl{TCT}. This effect can be attributed to the extended periods of time participants had to stay on a lane, increasing the chance of accidentally oscillating out of the lane and, thus, restarting the \factorTime. The restarted timer increased the \ac{TCT} and, thereby, the necessary distance. The analysis of the data provided further support for the assumption that the increased \ac{TCT} for higher numbers of lanes and higher selection times are related to oscillating out of the target lane: The analysis showed a strong correlation between stabilizing error rates, measured as the rate of trials in which the target lane was left, and the \ac{TCT}.

While the analysis of the \ac{RTLX} and the Likert questionnaires showed no differences between \seltime{2} and \seltime{3} conditions, both were rated significantly better than the \seltime{1} conditions. The qualitative feedback of participants supported this result: While participants' opinions were mixed for the \enquote{best} condition between \seltime{2} and \seltime{3}, there was a clear agreement that \seltime{1} was too short. 


\subsection{RQ3: Interaction Effects between Number of Lanes and Selection Time}

The analysis of the experiment showed interaction effects between the \factorLanes{} and \factorTime{} for both, the accuracy and the efficiency measurements. This effect can be attributed to a mutual reinforcement of the influences of the individual factors described above: Lower numbers and, thus, wider lanes led to an increased width of the intermediate lanes between the participant and the target lane. With shorter selection times, this resulted in lower accuracy rates, because a larger lateral distance had to be crossed, resulting in more false selections in between. The reverse effect applies for higher numbers of and, thus, thinner lanes: Due to the thinner lanes, it generally became more difficult for the participants to select a lane by walking on it, resulting in lower efficiency and accuracy rates. This effect is further intensified by forcing the user to walk longer on the lane through higher selection times, increasing the chance of accidentally walking out of the lane.

Taken together, this explains the interaction effects found in the experiment on accuracy and efficiency: Wider lanes require longer selection times, thinner lanes require shorter selection times to attain high accuracy and effectiveness.