The last chapter of this thesis explored foot-taps as an input modality for \acp{HMD}, leveraging the \emph{lower limbs} to interact with \emph{body-stabilized} interfaces. In addition to the presented advantages, the chapter also highlighted limitations of the presented interaction style (see section \ref{sec:cheesyfoot:limitations}). Parts of these limitations are rooted in the inherent limitations of foot-based operation: The user has to stand in a fixed position in order to operate such an interface to overcome the \emph{Midas Tap Problem} (see section \ref{sec:cheesyfoot:limitations:midas}).

While this limitation might be bearable for work, e.g., in an industrial context, it becomes a major challenge in genuinely mobile interaction situations: Mobility is strongly related to locomotion, rendering traditional foot-based interaction techniques difficult to use in such situations, as the user must stop the process of locomotion to interact with the system. If, however, the vision of ubiquitous interaction with information in a digitally augmented physical world is to become reality, a substantial part of the interaction with such devices will happen \emph{on the go}. This highlights the necessity for truly mobile interaction techniques for \acp{HMD} that not only support interaction while being at different places but also during the process of getting there - while walking.

As a possible solution, this chapter proposes the use of a \emph{world-stabilized} interface (see section \ref{sec:relatedwork:hmds:implementation}) that does not move together with the user (see figure \ref{fig:walktheline/overview_walktheline}). This fixation in the world can be leveraged to control an interface by changing the position relative to the interface, providing a solution for interaction with \acp{HMD} in \mobilityLower{} situations for \singleuserLower{}. A multitude of such shifts of the relative position to an interface as input dimension is conceivable. As a first step towards such a concept, this chapter focuses on shifts that occur orthogonally to the user's walking path, thus not modifying their original walking direction and, thereby, not interfering with the process of locomotion. While the contribution of this chapter focuses on \discreteLower{}, we are confident that it can also be transferred to \continuousLower{}, as outlined in the future work section of this chapter (see section \ref{sec:walktheline:limitations}).

The contribution of this chapter is two-fold: First, the chapter contributes the methodology and results of a controlled experiment assessing the accuracy and efficiency of such an interfaces. Second, based on the results of the controlled experiment, the chapter presents a prototype implementation of a \interactionStyleBased{} input modality for \acp{HMD} together with three example applications.

The remainder of this chapter is structured as follows: After reviewing the related works (section \ref{sec:walktheline:rw}), section \ref{sec:walktheline:concept} proposes a concept to interact with \acp{HMD} using lateral shifts of the walking path. Afterward, section \ref{sec:walktheline:methodology} describes the methodology and research questions of the controlled experiment. After that, section \ref{sec:walktheline:results} and \ref{sec:walktheline:discussion} present and discuss the results of the controlled experiment. Based on these results, section \ref{sec:walktheline:prototype} describes a prototype implementation of a walking-based input modality for interacting with \acp{HMD} together with three example applications. The chapter concludes with limitations and proposes directions for future work (section \ref{sec:walktheline:limitations}). 

\cptIntroBox{Muller2020}{The student \emph{Daniel Schmitt} implemented the study client application. \emph{Sebastian Günther}, \emph{Martin Schmitz}, \emph{Markus Funk} and \emph{Max Mühlhäuser} supported the conceptual design and contributed to the writing process.}