\documentclass[conference]{IEEEtran}
\pdfoutput=1

\usepackage{times}

% numbers option provides compact numerical references in the text. 
\usepackage[numbers]{natbib}
\usepackage{multicol}
\usepackage{amsmath}
\usepackage{amssymb}
\usepackage{booktabs}
% \usepackage[colorlinks=true]{hyperref}
\usepackage[colorlinks=false,hidelinks]{hyperref}


% Libraries
\usepackage{amsmath,amssymb,amsfonts}

\usepackage{float}
\usepackage{listings}
\usepackage{multirow}
\usepackage{xparse}
\usepackage{optidef}
\usepackage{algorithm}
\usepackage{algpseudocode}
\usepackage{xcolor}
%\usepackage[bookmarks=true]{hyperref}
\usepackage{blindtext}

% Color code for symbolic predicates and objects
\definecolor{flodarkpurple}{rgb}{0.288,0.1196,0.7}
\newcommand{\graytext}[1]{\textcolor{flodarkpurple}{#1}}

% Methods
\newcommand{\gs}{\textsc{generator-scorer}}
\newcommand{\scgs}{\textsc{saycan-gs}}
\newcommand{\imgs}{\textsc{innermono-gs}}
\newcommand{\hm}{\textsc{hierarchical}}
\newcommand{\ttm}{\textsc{text2motion}}

% Tasks
\newcommand{\LH}{\textbf{LH}}
\newcommand{\LG}{\textbf{LG}}
\newcommand{\PAP}{\textbf{PAP}}

% Comment macros
\newcommand{\todo}[1]{\textcolor{red}{\{TODO #1\}}}
\newcommand{\klin}[1]{{\color{blue}{[Kevin: #1]}}}
\newcommand{\chris}[1]{{\color{red}{[Chris: #1]}}}
\DeclareMathSymbol{\shortminus}{\mathbin}{AMSa}{"39}

% Nice monospace font for code.
\usepackage{inconsolata}
\usepackage[T1]{fontenc}

% Color code block boxes with gray border.
\definecolor{light-gray}{rgb}{0.8, 0.8, 0.8}
\definecolor{comment-green}{rgb}{0.435, 0.576, 0.106}
\definecolor{prompt-blue}{HTML}{2596be}
\definecolor{code-function}{HTML}{379fbe}
\definecolor{code-function}{HTML}{693da8}  % brian (maybe remove)
\definecolor{code-syntax}{HTML}{0060b1}
\definecolor{code-constant}{HTML}{d86001}
\definecolor{prompt-gray}{HTML}{a7a7a7}
\definecolor{highlight}{HTML}{f8f9cb}
\definecolor{highlight}{HTML}{e3eeff}  % brian (maybe remove)
\definecolor{code-perception}{HTML}{2ecc71}
\definecolor{code-control}{HTML}{ff9900}
\definecolor{code-undefined}{HTML}{ff0000}
\renewcommand\fbox{\fcolorbox{light-gray}{white}}

\setlength{\fboxsep}{1.1pt} % This makes it so that colorbox (for text highlights) are not too tall.
% \newcommand{\hlcode}[1]{{\sethlcolor{highlight}\hl{#1}}}
\newcommand{\hlcode}[1]{\colorbox{highlight}{\makebox[0.99\linewidth][l]{#1}}}
\newcommand{\hlsummary}[1]{\colorbox{highlight}{\makebox[0.4\linewidth][l]{#1}}}
\newcommand{\hlcoderesp}[1]{\colorbox{highlight}{\makebox[0.96\linewidth][l]{#1}}}
\newcommand{\link}[2]{\textcolor{magenta}{\href{#1}{#2}}}
\newcommand{\authorhref}[2]{\textcolor{black}{\href{#1}{#2}}}

\NewDocumentCommand{\code}{v}{%
\texttt{\small{\textcolor{code-syntax}{#1}}}%
}

\newcommand{\query}[1]{\textcolor{comment-green}{#1}}
\newcommand{\prompt}[1]{\textcolor{prompt-gray}{#1}}
\newcommand{\resp}[1]{#1}

\usepackage{lipsum}

\newcommand\blfootnote[1]{%
  \begingroup
  \renewcommand\thefootnote{}\footnote{#1}%
  \addtocounter{footnote}{-1}%
  \endgroup
}

\begin{document}

%\title{Text2Motion: Verifying Feasibility of Natural Language Plans}
% \title{\LARGE {\bf Text2Motion}: Turning Natural Language Instructions into Feasible Plans}
\title{\LARGE {\bf Text2Motion}: From Natural Language Instructions to Feasible Plans \\ \vspace{7pt} \large{\bf Project page: \link{https://sites.google.com/stanford.edu/text2motion}{sites.google.com/stanford.edu/text2motion}} \vspace{-10pt}}

\author{
    \authorblockN{
        \authorhref{https://kevin-thankyou-lin.github.io/}{Kevin Lin}\authorrefmark{1}\authorrefmark{2},
        \authorhref{https://www.chrisagia.com/}{Christopher Agia}\authorrefmark{1}\authorrefmark{2}\authorrefmark{3},
        \authorhref{https://cs.stanford.edu/~takatoki/}{Toki Migimatsu}\authorrefmark{2}, 
        \authorhref{https://web.stanford.edu/~pavone/}{Marco Pavone}\authorrefmark{3} and
        \authorhref{https://web.stanford.edu/~bohg/}{Jeannette Bohg}\authorrefmark{2}
    }
    \vspace{3pt}
    \authorblockA{\authorrefmark{2}
        Department of Computer Science, Stanford University, California, U.S.A.
    }
    \authorblockA{\authorrefmark{3}
        Department of Aeronautics \& Astronautics, Stanford University, California, U.S.A.
    }
    % \texttt{Email: \{kevinlin0,cagia,takatoki,pavone,bohg\}@stanford.edu}
    \texttt{Email: \{\href{mailto:kevinlin0@stanford.edu}{kevinlin0},\href{mailto:cagia@stanford.edu}{cagia},\href{mailto:takatoki@stanford.edu}{takatoki},\href{mailto:pavone@stanford.edu}{pavone},\href{mailto:bohg@stanford.edu}{bohg}\}@stanford.edu}
    \vspace{-3pt}
}

% \href{mailto:kevinlin0@stanford.edu}{kevinlin0} 
% \href{mailto:cagia@stanford.edu}{cagia}
% \href{mailto:takatoki@stanford.edu}{takatoki}
% \href{mailto:pavone@stanford.edu}{pavone} 
% \href{mailto:bohg@stanford.edu}{bohg} 


\maketitle

\begin{abstract}
We propose Text2Motion, a language-based planning framework enabling robots to solve sequential manipulation tasks that require long-horizon reasoning. 
Given a natural language instruction, our framework constructs both a task- and policy-level plan that is verified to reach inferred symbolic goals.
Text2Motion uses skill feasibility heuristics encoded in learned Q-functions to guide task planning with Large Language Models.
Whereas previous language-based planners only consider the feasibility of individual skills, Text2Motion actively resolves geometric dependencies spanning skill sequences by performing policy sequence optimization during its search.
We evaluate our method on a suite of problems that require long-horizon reasoning, interpretation of abstract goals, and handling of partial affordance perception. 
Our experiments show that Text2Motion can solve these challenging problems with a success rate of 64\%, while prior state-of-the-art language-based planning methods only achieve 13\%. %which is significantly higher than the prior best language-based planning method (13\%). 
Text2Motion thus provides promising generalization characteristics to semantically diverse sequential manipulation tasks with geometric dependencies between skills.
\end{abstract}


\IEEEpeerreviewmaketitle

\section{Introduction}


Recent years have witnessed the rise of human digitization~\cite{habermannDeepCapMonocularHuman2020,alexanderCREATINGPHOTOREALDIGITAL,pengNeuralBodyImplicit2021,alldieckDetailedHumanAvatars2018, rajANRArticulatedNeural2020}. This technology greatly impacts the entertainment, education, design, and engineering industry.
There is a well-developed industry solution for this task.
High-fidelity reconstruction of humans can be achieved either with full-body laser scans~\cite{saitoSCANimateWeaklySupervised2021}, dense synchronized multi-view cameras~\cite{xiangModelingClothingSeparate2021a,xiangDressingAvatarsDeep2022a}, or light stages~\cite{alexanderCREATINGPHOTOREALDIGITAL}.
However, these settings are expensive and tedious to deploy and consist of a complex processing pipeline, preventing the technology's democratization.

Another solution is to view the problem as inverse rendering and learn digital humans directly from custom-collected data.
Traditional approaches directly optimize explicit mesh representation~\cite{loperSMPLSkinnedMultiperson2015, fangRMPERegionalMultiperson2018, pavlakosExpressiveBodyCapture2019} which suffers from the problems of smooth geometry and coarse textures~\cite{prokudinSMPLpixNeuralAvatars2020,alldieckVideoBasedReconstruction2018}. Besides, they require professional artists to design human templates, rigging, and unwrapped UV coordinates.
Recently, with the help of volumetric-based implicit representations~\cite{mildenhallNeRFRepresentingScenes2020, parkDeepSDFLearningContinuous2019, meschederOccupancyNetworksLearning2019} and neural rendering~\cite{laineModularPrimitivesHighPerformance2020, liuSoftRasterizerDifferentiable2019, thiesDeferredNeuralRendering2019}, 
one can easily digitize a quality-plausible human avatar from video footage~\cite{jiangNeuManNeuralHuman2022,wengHumanNeRFFreeviewpointRendering}.
Particularly, volumetric-based implicit representations~\cite{mildenhallNeRFRepresentingScenes2020, pengNeuralBodyImplicit2021} can reconstruct scenes or objects with much higher fidelity against previous neural renderer~\cite{thiesDeferredNeuralRendering2019,prokudinSMPLpixNeuralAvatars2020}, and is more user-friendly as it does not need any human templates, pre-set rigging, or UV coordinates.
Captured visual footage and corresponding skeleton tracking are enough for training.
However, better reconstructions and more friendly usability are at the expense of the following factors.
1) \textbf{Inefficiency:}
They require longer optimization times (typically tens of hours or days) and inference slowly.
Volume rendering~\cite{mildenhallNeRFRepresentingScenes2020,lombardiNeuralVolumesLearning2019} formulates images by querying the densities and colors of millions of spatial coordinates. 
In the training stage, due to memory constraints, only a small fraction of points are sampled which leads to slow convergence speed.
2) \textbf{Entangled representations}:
The geometry, materials, and motion dynamics are entangled in the neural networks. 
Due to the implicit nature of neural nets, one can hardly edit one property without touching the others~\cite{yuanNeRFEditingGeometryEditing2022a,liuEditingConditionalRadiance2021}.
3) \textbf{Graphics incompatibility}:
Volume rendering is incompatible with the current popular graphic pipeline,
which renders triangular/quadrilateral meshes efficiently with the rasterization technique.
Many downstream applications require mesh rasterization in their workflow (\eg, editing~\cite{foundationBlenderOrgHome}, simulation~\cite{benderPositionBasedSimulationMethods2015}, real-time rendering~\cite{akenine2019real}, ray-tracing~\cite{waldRTXRayTracing}).
Although there are approaches~\cite{lorensenMarchingCubesHigh,labelleIsosurfaceStuffingFast2007} can convert volumetric fields into meshes, the gaps from discrete sampling degrade the output quality in terms of both meshes and textures.


To address these issues, we present \textbf{EMA}, a method based on \textbf{E}fficient \textbf{M}eshy neural fields to reconstruct animatable human \textbf{A}vatars.
Our method enjoys flexibility from implicit representations and efficiency from explicit meshes, yet still maintains high-fidelity reconstruction quality.
Given video sequences and the corresponding pose tracking, our method digitizes humans in terms of canonical triangular meshes, physically-based rendering (PBR) materials, and skinning weights \textit{w.r.t.} skeletons.
We jointly learn the above components via inverse rendering~\cite{laineModularPrimitivesHighPerformance2020,chenDIBRLearningPredict2021,chenLearningPredict3D2019} in an end-to-end manner.
Each of them is derived from a separate neural field, which relaxes the requirements of a preset human template, rigging, or UV coordinates.
Specifically, we predict a canonical mesh out of a signed distance field (SDF) by differentiable marching tetrahedra~\cite{shenDeepMarchingTetrahedra2021,gaoGET3DGenerativeModel,gaoLearningDeformableTetrahedral2020,munkbergExtractingTriangular3D2022}, then we extend the marching tetrahedra~\cite{shenDeepMarchingTetrahedra2021} for spatial-varying materials by utilizing a neural field to predict PBR materials \textit{on the mesh surfaces} after rasterization~\cite{munkbergExtractingTriangular3D2022,hasselgrenShapeLightMaterial2022,laineModularPrimitivesHighPerformance2020}.
To make the canonical mesh animatable, we take another neural field to model the forward linear blend skinning for the meshes. 
Given a posed skeleton, the canonical mesh is then transformed into the corresponding poses.
Finally, we shade the mesh with a rasterization-based differentiable renderer~\cite{laineModularPrimitivesHighPerformance2020} and train our models with a photo-metric loss.
After training, we export the mesh with materials and discard the neural fields.

\looseness=-1
There are several merits of our method design.
1) \textbf{Efficiency}:
Powered by efficient mesh rendering, our method can render in real-time.
Besides, the training speed is boosted as well, 
since we compute loss holistically on the whole image and the gradients only flow on the mesh surface. In contrast, volume rendering takes limited pixels for loss computation and back-propagates the gradients in the whole space.
Our method only needs about an hour of training and minutes of optimization are enough for plausible avatar reconstruction.
2) \textbf{Disentangled representations}:
Our shape, materials, and motion modules are disentangled naturally by design, which facilitates editing. 
Besides, Canonical meshes with forward skinning modeling handle the out-of-distribution poses better.
3) \textbf{Graphics compatibility}:
Our derived mesh representation is compatible with 
the prominent graphic pipeline, which leads to instant downstream applications (\eg, the shape and materials can be edited directly in design software~\cite{foundationBlenderOrgHome}).
To further improve reconstruction quality, we additionally optimize image-based environment lights and non-rigid motions.


We conduct extensive experiments on standards benchmarks H36M~\cite{ionescuHuman36MLarge2014b} and ZJU-MoCap~\cite{pengNeuralBodyImplicit2021}.
Our method achieves very competitive performance for novel view synthesis, generalizes better for novel poses, 
and significantly improves both training time and inference speed against previous arts.
Our research-oriented code reaches real-time inference speed (100+ FPS for rendering $512\times512$ images).
We in addition showcase applications including novel pose synthesis, material editing, and relighting.
\section{Related Work}
\label{sec:related-works}

\subsection{Task and motion planning}
\label{subsec:tamp-literature}
Task and motion planning refers to a problem setting in which a robot has to solve long-horizon tasks through symbolic and geometric reasoning~\cite{kaelbling2012integrated, integrated-tamp-2021}.
The hierarchical approach~\cite{5980391} characterizes the most common family of solution methods.
Such methods typically employ a) an AI task planner~\cite{bonet2001planning, helmert2006fast} to deduce candidate plan skeletons and b) a motion planner to obtain motion trajectories subject to robot and environmental constraints; e.g. through sampling-based planning~\cite{garrett2020-pddlstream} or constrained optimization~\cite{toussaint2015-lgp, driess2019-hlgp}.

% A consequence of strictly separating task- from motion-level planning is that task plans are uninformed of their geometric feasibility upon construction.
A consequence of strictly separating task- from motion-level planning is that the geometric feasibility of a task plan is unknown at the time of its construction.
Hence, the process of iterating between a) and b) may take on the order of minutes until the TAMP solver returns a solution to a sufficiently complex task or indicates that none exists.
This is in contrast to Text2Motion, which interweaves task and motion planning by using value functions as a geometric feasibility heuristic to guide task-level planning.
Therefore, Text2Motion spends less time considering infeasible plans, and when a plan is returned, it is geometrically feasible by construction.

Another line of works accelerates TAMP by learning sampling distributions~\cite{wang2018active, xu2021deep}, visual feasibility heuristics~\cite{driess2020-dvr, driess2020-dvh, driess2021-lpr}, low-level controllers~\cite{driess2021-lgr, silver2022learning}, or state sparsifiers~\cite{chitnis2021camps, silver2021-ploi}.
However, these methods learn from solutions computed by classical TAMP solvers, and thus, they also depend on meticulously hand-crafted and task-specific symbolic planning domains.
% However, like the TAMP solvers they build upon, they rely on meticulously hand-crafting symbolic planning domains which govern the transition dynamics of actions and may be non-stationary across environments and tasks.
% However, these methods face similar limitations to the TAMP solvers they build upon: they rely on meticulously hand-crafted, task-specific symbolic planning domains.
% However, these works face a challenge similar to that faced by the TAMP solvers they build upon: they rely on meticulously hand-crafted, task-specific symbolic planning domains.
While learning symbolic representations has been proposed for TAMP~\cite{kroemer2016learning, ames2018learning, konidaris2018skills, silver2021learning, curtis2022discovering, chitnis2022learning, silver2022learning}, these approaches often require task-specific symbolic transition experience.
A core element of our work is the use of LLMs as a task- and environment-agnostic planner, while value functions serve as an alternative to symbolic preconditions when determining what skills are admissible in the current state.


\subsection{Language for robot planning}
\label{subsec:language-literature}
% LLMs have been oriented towards solving long-horizon robotics problems on several fronts.
Language has increasingly been explored as a medium for solving long-horizon robotics problems.
For instance, {\em Language-conditioned policies\/} (LCPs) have been applied to robotic manipulation.
While many methods learn short-horizon skills~\cite{stepputtis2020language, jang2021bczero, concept2robot-2021, cliport-2022, perceiver-2022, vima-2022}, others focus on long-horizon tasks~\cite{mees2022calvin, rt1-2022}.
% {\em Language-conditioned policies\/} (LCPs) have been proposed for solving robot manipulation tasks. 
% While some focus on skill learning or solving shorter horizon manipulation tasks~\cite{stepputtis2020language, jang2021bczero, concept2robot-2021, cliport-2022, perceiver-2022, vima-2022}, others focus on solving long-horizon manipulation problems~\cite{mees2022calvin, rt1-2022}.
However, using LCPs alone to solve long-horizon manipulation problems can require expensive data collection and training procedures if the LCP is to generalize to a wide distribution of long-horizon tasks with diverse instructions.
% and b) diverse instructions enabling the learned policy to either interpolate within the training data distribution or generalize beyond it. 

% \subsubsection{Language-conditioned policies (LCPs)}
% \label{subsec:language-policies}
% Concept2Robot~\cite{} was amongst the first to condition manipulation skill policies with LLM embeddings, trained on large-scale video data via inverse reinforcement learning.
% CLIPort proposed a dual-stream semantic-spatial architecture based on CLIP and learned skill policies in pixel-space (Transporter Networks~\cite{}) from curated demonstrations. 
% Perceiver-Actor extended CLIPort to 6DoF settings with Perceiver, substituting pixel-space with attention-based voxel-space reasoning. 
% VIMA expresses general manipulation concepts with multi-modal prompts.

% Several recent works leveraged the generative qualities of LLMs by prompting it to predict long-horizon plans.
% The prompts commonly contain in-context examples and chain-of-thought explanations to guide the LLMs response to the desired format, along with the new task query.
% While this avenue subsumes the data dependence issues of LCPs, it presents challenges in terms of grounding the LLMs plans in actions that are consistent with both the robots capabilities and the constraints of the environment. 

% \citet{zeroshot-llms-2022} proposes to compute embedding similarity scores between the descriptions of LLM-produced actions and actions known to be admissible in the current state. 
% However, the evaluation of this method is limited to non-geometric settings that assumes perfect symbolic transitions of actions.
% A tangential line of work shifts the planning medium from natural language to code.
% CodeAsPolicies~\cite{code-as-policies-2022} engages low-level primitives through sequential execution of a program, while ProgPrompt~\cite{progprompt-2022} expresses task queries to LLMs as programs that encode robot actions and preconditions, solution samples to elicit chain of thought reasoning~\cite{chain-of-though-2022}, and fallback behaviors should actions fail. 
% Meticulously constructing code-like prompts for each new task raises concerns of reliability and scalability to new tasks (akin to defining symbolic planning domains) and is non-trivial compared to natural language for non-expert users.
% Furthermore, while code is a sufficiently expressive representation for plans, it remains unclear how to verify that such plans adhere to constraints and optimize them for desired properties, such as geometric feasibility, which is of key interest for TAMP.

Several recent works leverage the generative qualities of LLMs by prompting them to predict long-horizon plans.
\cite{zeroshot-llms-2022} grounds an LLM planner to admissible action sets, but limits evaluation to task-level planning. 
%non-geometric settings.
Tangential works shift the planning medium from natural language to code~\cite{code-as-policies-2022, progprompt-2022, zelikman2022parsel} and embeds task queries, robot actions, solution samples, and fallback behaviors as programs in the prompt. 
While programs are an expressive representation for plans, we focus on optimizing plans in the form of skill sequences.

Closest in spirit to our work are SayCan~\cite{saycan-2022} and Inner Monologue (IM)~\cite{innermono-2022} which at each timestep score the \textit{usefulness} and \textit{feasibility} of all possible skills and execute the one with the highest score.
Termination occurs when the score of the $\texttt{stop}$ ``skill'' is larger than any other.
IM provides additional sources of feedback to the LLM in the form of object descriptions, skill successes, and task-progress cues. 

While the generality of SayCan and IM is shown over a diverse range of tasks, there are several drawbacks that impede their performance in the settings we study.
First, by only \textit{greedily} considering the next skill at each timestep, they may fail to account for geometric dependencies that exist over the extent of an skill sequence (Fig.~\ref{fig:teaser}).
% Second, by keeping the plan \textit{implicit} to the LLM (i.e. no explicit multi-step plan is predicted), it cannot be verified against a desired property or outcome prior to execution.
Second, using the LLM to plan \textit{implicitly} (as opposed to \textit{explicitly} predicting a multi-step plan) prevents verification of desired properties or outcomes prior to execution.
Examples of such properties could include whether the final state induced by the plan satisfies symbolic constraints or whether the plan adheres to some notion of safety.
Lastly, these methods ignore the uncertainty of skill feasibility predictions, which \cite{taps-2022} demonstrates is important for using skills to solve geometrically complex TAMP problems.
By addressing these limitations, Text2Motion can outperform SayCan and IM on geometrically complex tasks, as demonstrated in the experiments.
\section{Problem Setup}
\label{sec:problem-setup}

\section{\system Framework Design}
\label{sec:system}
We now explain how \system helps author widgets that support transparent, reusable, and customizable user actions. 

% \danc{here do we want to emphasize a) we revise conventional widget design to a statefule design or b) megneton can convert your existing widgets to improve it with our proposed characteristics? It sounds more like b) to me but not sure if that's the intention. } \saj{it's actually a}
%In this section, we discuss the key components that provide the foundation for \system widgets, which instruments the design goals outlined in Section~\ref{sec:design_goal}. 
%We then explain the system architecture of \system.

\subsection{Widget Frameworks: Design and Limitations}
Widgets are interactive elements, \eg sliders, text boxes, buttons, that have representations both in the kernel, \ie where code is executed, and the front-end, \ie the notebook web interface. However, recent frameworks for authoring widgets~\cite{idomjp} also enable integration of interactive dashboards in the front-end~\cite{wu2020b2,bauerle2022symphony, zhang2023meganno}.  

 \begin{figure}[!htb] 
 \centering
  \includegraphics[width=0.8\linewidth]{figures/stateful-widget-redesign-basic.png}
  \caption{Design of basic, \ie traditional widgets.}
  \label{fig:base_widget} 
  \Description{The basic widget design.}
\end{figure}

 
 As shown in Figure~\ref{fig:base_widget}, Widgets (\eg \emph{ipywidgets}~\cite{IPyWidgets}) maintain their state both at the back-end kernel (called \emph{Widget Base}) and the front-end (called \emph{Widget Model}.) The Widget Base and Widget Model remain in-sync via the communication API called \emph{Comm}. 
 %\danc{The previous statement is bit hard to follow. Break down into defining what are widget base and model, and then explain how they work together?} 
However, only the most recent state is maintained, making the widgets essentially \emph{memoryless}. The \emph{Widget Manager} coordinates the display of the widget in the front-end \emph{Widget View}. The Widget View is a container for rendering interactive components using front-end libraries and web frameworks. The Widget View only registers low-level event listeners corresponding to user interactions on the components. %For example, a \emph{drag} interaction that updates position of slider is registered as an \emph{onChange} listener. 
 For example, a \emph{selection} interaction on the graph node in Figure~\ref{fig:teaser}B that updates the bar charts is registered as an \emph{onClick} listener.
 Therefore, these widgets are \emph{agnostic} of the user's high-level interaction types and additional context, such as where the interaction happened and which components were updated. The \emph{memoryless} and \emph{interaction agnostic} nature of widgets prevent tracking of the user's interaction history and the corresponding widget states.
Moreover, such a design primarily serves to parameterize data operations in the kernel using front-end events --- a widget state variable (\eg current node identifier) impacted by a low-level event (\eg \emph{onClick}) serves as an input parameter to a data operation (\eg distribution computation). Any change in the widget variable triggers a recomputation of the data operation. In the notebook, the users can programmatically access and update the parameters of the data operations. However, the data operations in the kernel, designed by widget developers, are neither accessible nor customizable from the front-end. The lack of affordances to override data operations limit 
the end-user's capability to customize the widgets designed by the developers. We describe enhancement of existing widgets with such features next.

\subsection{Towards Persistent, Interaction-Aware, and Customizable Widgets}

%We augment existing widgets to introduce new features such as interaction history, reusable sates, and on-demand customization of data operations. 
We create a persistent and interaction-aware widget called \emph{stateful widget} by extending the Widget Base with state and interaction history management capabilities (see Figure~\ref{fig:stateful_widget}.) Within a stateful widget, the state manager maintains each state updates corresponding to user interactions within a list called \emph{Data States}. The state manager registers the following in the \emph{action history}: (a) context of each event (\eg the front-end interaction type and the component and element where interaction occurred) and (b) the corresponding state identifier in Data States. Since the default Widget View only registers low-level events, we create a Widget View Wrapper that records each event's context as an action via an Action Wrapper. The action wrapper dispatches an action consisting of the event context mentioned earlier via the Comm API. Users can view the interaction history in a separate notebook cell which shows the details of an interaction and the corresponding data state as shown in Figure~\ref{fig:history}, thereby ensuring transparency. The history view is synchronized with the corresponding widget. Therefore, users can leverage the history to load previous states in the Widget View using the \emph{Restore} button. Moreover, users can also access the widget state as a \code{JSON} object using a declarative command as shown in Figure~\ref{fig:teaser}E, thereby ensuring reusability. Such a design also enables users to employ visualizations as a medium for capturing 
and exporting ``actions interactively
performed in the component''~\cite{batch2017interactive} --- 
the outcomes of these interactions are often utilized in 
subsequent steps of a data science workflow~\cite{rahman2022ie}.

 \begin{figure}[!htb] 
  \centering
  \includegraphics[width=\linewidth]{figures/stateful-widget-redesign-mag.png}
  \caption{Design of \system widgets. The dashed (``- -'') elements, \ie the stateful widget and widget view wrappers, are introduced by \system.}
  \label{fig:stateful_widget} 
  \Description{The stateful widget design.}
\end{figure}

% Therefore, interaction-aware state management \todo{via stateful widget} ensures transparency and reusability of user actions. \todo{elaborate}

\begin{figure*}[!htb] 
  \centering
  \includegraphics[width=\linewidth]{figures/history.png}
  \caption{The history view of a widget (\code{widget.history.show()}). Clicking the \emph{Restore} button loads previous state visualizations. }
  \label{fig:history} 
  \Description{The history view of a widget. Clicking the Restore button loads the previous states and their visualizations.}
\end{figure*}

%\hkc{why is it called 'shared'? also which opreations are customizable and which are not?} all data operations are customizable if they are defined as shared
Since data operations in the kernel correspond to user interactions in the front-end component, we introduce the concept of \emph{shared actions}.
Shared actions are data operations that end-users can override from the notebook. The operation definitions are essentially shared between the kernel and front-end. In the \system framework, developers can define a data operation to be shared. 
 For example, a shared data operation may return a distribution sorted by descending order of frequency. However, the user may prefer viewing the distribution in the alphabetic order of labels. As shown in Figure~\ref{fig:teaser}C and~\ref{fig:teaser}D, a user-defined function (UDF) written in the notebook --- which reflects the updated sort order --- is mapped to these the actions during widget instantiation time. 
In the kernel, the state manager parses the UDFs using custom serializers and overrides the data operation corresponding to the shared action. 
Such a design expands the ``events parameterizing code'' paradigm of widgets to ``operations parameterizing code'' and offers more flexible customization capabilities --- users can keep updating the shared actions to explore different objectives by modifying the function defined in the notebook.
Note that developers may implement data operations such as schema generation and distributions computation using standalone libraries or from scratch. In the case studies described in Section~\ref{sec:study}, we used an in-house graph query library, which was published as a Python package. 
% , among others. The operations are part of an in-house graph query library, which is published as a Python package for internal usage.
%We provide examples of these features in the supplementary material. 


%\todo{ADD CODE BLOCK}
\stitle{Components in \system Widget View.} 
%\danc{this paragraph renamed to technical/implementation details? Or is component a special term in megneton? } 
We use the React web framework~\cite{react} to develop the front-end components and the IDOM-Jupyter package~\cite{idomjp} for component rendering in the Widget View. 
The components are TypeScript~\cite{typescript} modules that enable the rendering of a wide range web-based visualization libraries. For example, we used a custom graph visualization library to render the schema graph~\cite{franz2016cytoscape}, Vega-lite~\cite{satyanarayan2016vega} to render the bar charts, and a JavaScript library to render tables. 
%We provide a detailed list of all the components in the supplementary material.
As TypeScript supports static typing, developers can define application-specific data types and use those across the modules. 
Therefore, using TypeScript ensures a tighter integration between the Widget Base in the kernel and the Widget Model in the front-end. 
Moreover, when customizing data operations defined as shared actions, the pre-defined types provide hints to the user about the expected return type of the customized function. Each of the components rendered in the Widget View is fully interactive. These interactions, derived from existing visualization research~\cite{yi2007toward, amar2005low} are reactively synchronized across \system components, enabling multiple-coordinated visualizations (\eg Figure~\ref{fig:teaser}B.) 

We aim to solve long-horizon sequential manipulation problems that require semantic and geometric reasoning from an instruction $i$ expressed in natural language. 
We are given a library of skills $\mathcal{L} = \{\pi^1, \ldots, \pi^N\}$. 
Each skill has a natural language description and comes with a policy $\pi(a|s)$, a Q-function $Q^\pi (s, a)$, and a dynamics model $T^\pi(s' | s, a)$, all of which can be acquired via off-the-shelf reinforcement learning (RL) or imitation learning (IL) methods.
Actions output by the policy $a\sim\pi(\cdot|s)$ are the parameters of a corresponding manipulation primitive~\cite{felip2013manipulation} $\rho(a)$ which consumes the action and executes a series of motor commands on the robot (see Appx.~\ref{sec:implementation-details} for more details).
We also assume that a method exists for conveying the environment state $s$ to the LLM as natural language. The task planning problem is to find a sequence of skills $[\pi_1, \dots, \pi_H]$ that is likely to satisfy the instruction $i$ (for notational convenience, we will hereafter represent sequences with range subscripts, e.g. $\pi_{1:H}$). The task planning objective is to maximize the language model likelihood of the skill sequence $\pi_{1:H}$ given instruction $i$ and initial state $s_1$:
\begin{equation}
    p(\pi_{1:H} \mid i, s_1). \label{eq:task-score}
\end{equation}

This objective only considers the probability that the sequence of skills will satisfy the goal from a symbolic perspective. 
For example, if the goal is to move a box from the table to the rack, a symbolically correct sequence of actions might be \graytext{Pick(box)}, \graytext{Place(box, rack)}. 
However, we must also consider whether the skill sequence can succeed from a geometric perspective. 
Specifically, for each skill $\pi_h$, we need to consider the geometric feasibility of the underlying continuous parameters $a_h$.
For example, a symbolically correct sequence of skills may fail geometrically due to kinematic constraints of the robot. A geometrically feasible plan is one where each skill $\pi_h$ and its continuous action parameter $a_h$ receives a binary reward $r_h$; if just one action fails, then the entire plan fails. The geometric feasibility is defined to be the probability that all skills $\pi_{1:H}$ achieve rewards $r_{1:H}$:
\begin{equation}
    p(r_{1:H} \mid i, s_1, \pi_{1:H}). \label{eq:motion-score}
\end{equation}
Taking the product of Eqs.~\ref{eq:task-score}~and~\ref{eq:motion-score} results in an objective which represents the probability that a skill sequence $\pi_{1:H}$ is both likely to satisfy instruction $i$ and is geometrically feasible:
\begin{equation}
    \begin{split}
        &p(\pi_{1:H}, r_{1:H} \mid i, s_1) \\
        &\quad\quad= p(\pi_{1:H} \mid i, s_1)\, p(r_{1:H} \mid i, s_1, \pi_{1:H}).
    \end{split} \label{eq:tamp-score}
\end{equation}

% Provided are libraries of primitives $\mathcal{L}^\pi = \{\pi^1, \ldots, \pi^N\}$ and predicate classifiers $\mathcal{L}^P = \{p^1, \ldots, p^M\}$.

% \chris{Since we are the first to perform TAMP from natural language, we need a general and flexible problem formalism that admits solutions in the form of reactive agents (SayCan, InnerMonologue), hierarchical planners, and integrated task and motion planners.}

% \chris{As seen below, I was originally trying to separate the \textit{symbolic} from the \textit{geometric} component of the problem, though I'm not sure that this is the best way of going about it. Because symbolic transition dynamics are unknown, notions of planning in the symbolic space don't make much sense anymore. Symbolic states are more like emissions from a physical state that we are altering through actions, and we are trying to drive these symbolic emissions to one of several symbolic goal states that are consistent with the natural language instruction $T_{ins}$.}

% \subsection{Task planning under partial completeness}
% \label{subsec:task-planning-setup}

% \chris{The idea here is to define the \textit{symbolic} component of the problem; but an augmented one, where states are only represented by simple spatial relations (hence the hats atop the caligraphic sets), actions are known and correspond to our primitives $\mathcal{L}^\pi$, but the \klin{symbolic?} transition dynamics are unknown. Hence graph search cannot be used for planning.}

% A \textit{partially complete} task planning problem $\hat{\Pi}$ is a tuple $\langle \mathcal{O}, \hat{\mathcal{P}}, \hat{\mathcal{S}}, \mathcal{A}, \hat{\mathcal{I}}, \hat{\mathcal{G}} \rangle$ with full observability of objects $\mathcal{O}$ but with unspecified transition dynamics.
% $\hat{\mathcal{P}}$ is a set of predicates that describe basic spatial relations of objects like $\mathrm{On}(a, b)$, but does not include more complex relations commonly used to define transition pre- and post-conditions such as $\mathrm{InHand}(a)$ $\mathrm{CollisionFree}(a, b, \tau)$.
% A state $\hat{s} \in \hat{\mathcal{S}}$ is an assignment of values to predicates $\hat{\mathcal{P}}$ over objects, $\hat{\mathcal{I}}$ is the initial state, and $\hat{\mathcal{G}}$ is a set of goal states.
% A solution is a plan skeleton $\tau=[\pi_1, \pi_2, \ldots, \pi_H]$ the drives the initial state $\hat{\mathcal{I}}$ into a state $s\in\hat{\mathcal{G}}$ that satisfies the instruction $T_{ins}$.


% \subsection{Motion planning with primitives}
% \label{subsec:motion-planning-setup}


% \subsection{Task and motion planning from language}
% \label{subsec:tamp-setup}





% \subsection{Motion planning}
% \label{subsec:motion-planning-setup}

% Summarize the motion planning problem defined in TAPS.


% \subsection{Task and motion planning}
% \label{subsec:tamp-setup}
% Merge above.

% In this section, we formalize the task and motion planning problem.
\section{Text2Motion}
\label{sec:text2motion}

%\subsection{Integrating Language-based Task Planning with Policy Sequence Optimization}

Our goal is to find a geometrically feasible long-horizon plan for a given natural language instruction. We follow a modular approach similar to traditional TAMP methods but replace the commonly used symbolic task planner with an LLM. Doing so allows us to broaden the task planning domain without the need for tedious manual construction. The core idea of this paper is ensure the geometric feasibility of an LLM task plan---and thereby its correctness---by predicting the success probability of learned skills that are sequenced according to the task plan. More specifically, we run an iterative search, where at each iteration we query the LLM for a set of candidate skills given the current symbolic state of the environment. For each of these skills, we optimize their action parameters and compute their success probability. This success probability is multiplied by the language model likelihood for the predicted skill to produce an overall score. The skill with the highest overall score is added to the current plan. A dynamics model is used to predict the geometric state that would result from executing this skill. If this predicted state satisfies a termination condition predicted by the LLM, then the instruction is satisfied and the plan is returned. Otherwise, the predicted geometric state is used to start the next iteration of the search. A visualization is provided in Fig.~\ref{fig:system}.

% To find geometrically feasible long-horizon plans with the LLM, we propose an integrated TAMP method that interweaves LLM task planning with motion planning. As a brief overview, the planning iteration starts by asking the LLM to generate a candidate skill skill to execute given the current environment state. Then, a motion planner finds optimal skill parameters for this skill and computes its predicted probability of success. This success probability is multiplied by the language model likelihood for the predicted skill to produce a TAMP score. This is done for a set of candidate skill skills generated by the LLM, and the one with the highest TAMP score is selected. A dynamics model provided by the motion planner predicts the next state that would result from executing this skill, and this predicted state is used to start the next planning iteration. This loop repeats until a termination condition predicted by the LLM is met.

%This loop repeats until the LLM predicts the ``stop'' token, at which point the long-horizon plan is returned.

This iterative approach can be described as a decomposition of the joint probability in Eq.~\ref{eq:tamp-score} by timestep $h$:
\begin{align}
    p(\pi_{1:H}, r_{1:H} \mid i, s_1)
        &= \prod_{h=1}^H p(\pi_h, r_h \mid i, s_1, \pi_{1:h-1},  r_{1:h-1}) \label{eq:tamp-score-decomp} \\
        &\approx \prod_{h=1}^H p(\pi_h, r_h \mid i, s_1, \pi_{1:h-1}) \label{eq:tamp-score-indep}
\end{align}
Eq.~\ref{eq:tamp-score-decomp} is simply a factorization of Eq.~\ref{eq:tamp-score} using conditional probabilities. 
In Eq.~\ref{eq:tamp-score-indep}, we make the assumption that skill $\pi_h$ and reward $r_h$ can be determined solely from initial state $s_1$ and sequence of prior skills $\pi_{1:h-1}$, and thus is independent of prior rewards $r_{1:h-1}$. 
Specifically, Eq.~\ref{eq:tamp-score-indep} more accurately approximates Eq.~\ref{eq:tamp-score-decomp} when $\pi_{1:h-1}$ is foreknown to receive rewards $r_{1:h-1}$. 
We induce this setting through geometric feasibility planning (elaborated upon in Sec.~\ref{sec:geo_feas}).
% In particular, when $\pi_{1:h-1}$ is foreknown to receive rewards $r_{1:h-1}$ (the setting we induce, elaborated upon in Sec.~\ref{sec:geo_feas}), Eq.~\ref{eq:tamp-score-indep} is expected to sufficiently approximate Eq.~\ref{eq:tamp-score-decomp}.
% We motivate this assumption by foreknowledge of $\pi_{1:h-1}$ receiving rewards $r_{1:h-1}$ (elaborated upon in Sec.~\ref{sec:geo_feas}), and thus, we expect a sufficient approximation of Eq.~\ref{eq:tamp-score-decomp} by Eq.~\ref{eq:tamp-score-indep}.
% This simplification allows us to further decompose Eq.~\ref{eq:tamp-score-indep}. 
This allows us to further decompose Eq.~\ref{eq:tamp-score-indep} into the joint probability of $\pi_h$ and $r_h$, which we define as the skill score $S_{\text{skill}}$:
\begin{equation}
    S_{\text{skill}}(\pi_h)
        = p(\pi_h, r_h \mid i, s_1, \pi_{1:h-1}). \label{eq:tamp-step-score}
\end{equation}

Each planning iteration is responsible for finding the skill $\pi_h$ that maximizes the skill score at timestep $h$. We decompose this score into the conditional probabilities of $\pi_h$ and $r_h$:
\begin{equation*}
    S_{\text{skill}}(\pi_h)
        = p(\pi_h \mid i, s_1, \pi_{1:h-1}) \, p(r_h \mid i, s_1, \pi_{1:h})
        \label{eq:s_skill}
\end{equation*}
We define the first factor in this product to be the language model likelihood score:
\begin{equation}
    S_{\text{llm}}(\pi_h) = p(\pi_h \mid i, s_1, \pi_{1:h-1}). \label{eq:lm-step-score}
\end{equation}
This represents the task planning step, where at timestep $h$, we ask the LLM to predict skill $\pi_h$ given instruction $i$, initial state $s_1$, and sequence of previously planned skills $\pi_{1:h-1}$.

The second factor in Eq.~\ref{eq:s_skill} represents the policy sequence optimization step, where we evaluate the geometric feasibility of the skill sequence $\pi_{1:h}$. 
The reward $r_h$ is conditionally independent of instruction $i$ given the initial state and skill sequence $\pi_{1:h}$. We thus define the geometric feasibility score:
\begin{equation}
S_{\text{geo}}(\pi_h) = p(r_h \mid s_1, \pi_{1:h}). \label{eq:motion-step-score}
\end{equation}
The skill score to be optimized at each iteration of planning is therefore the product of the LLM likelihood and the geometric feasibility of the planned skill sequence:
\begin{equation}\label{eq:tamp-step-score-decomp}
    S_{\text{skill}}(\pi_h) = S_{\text{llm}}(\pi_h) \cdot S_{\text{geo}}(\pi_h).
\end{equation}
Text2Motion alternates between incrementally optimizing $S_{\text{llm}}(\pi_h)$
and $S_{\text{geo}}(\pi_h)$ at each timestep $h$. The computation of these scores as well as the terminating strategy for the algorithm are described in more detail in the following subsections.


\subsection{LLM task planning}\label{sec:llm-task-planning}
At each iteration $h$, we optimize $S_{\text{llm}}(\pi_h)$ (Eq.~\ref{eq:lm-step-score}) by querying an LLM to generate $K$ candidate skills  $\{\pi_h^1, \dots, \pi_h^K\}$, each accompanied by their own language model scores $S_{\text{llm}}(\pi_h^k)$.
These scores represent the likelihood of skill $\pi_h^k$ being the correct skill to execute from a language modeling perspective to satisfy instruction $i$.
These candidates are then checked for geometric feasibility as described in Sec.~\ref{sec:geo_feas}. 



The input prompt contains context $c$ in the form of several long-horizon planning examples to inform the LLM of the available skills in the robot's library and their usage semantics (see Sec.~\ref{subsec:prompt-engineering}). 
The prompt also contains the instruction $i$, history of planned skills $\pi_{1:h-1}$, initial state $s_1$, and predicted states $s_{2:h}$ up to step $h$. Descriptions of the states $s_{1:h}$ come in the form of sets of binary propositions such as \graytext{on(milk, table)} or \graytext{under(yogurt, rack)} that describe spatial relationships between objects. To generate these propositions for initial state $s_1$, we use a heuristic approach described in the Appendix (Appx.~\ref{sec:scene-descr-symbolic}). Our framework is agnostic to the specific approach. Suitable examples could be scene graph generation methods~\cite{rosinol2021kimera, hughes2022hydra} or forms of text-based scene description~\cite{gu2021open, kuo2022findit, zeng2022socratic}. States $s_{2:h}$ are predicted given skills $\pi_{1:h-1}$ and their associated dynamics models. We then automatically generate corresponding language descriptions $s_{2:h}$ from these predicted states. Sec.~\ref{subsec:prompt-engineering} provides an example.


How many skills $H$ are required to satisfy the instruction $i$ is unknown, so it is up to the LLM to decide when to terminate. 
We use the goal proposition prediction method described in Sec.~\ref{sec:goal_pred} to determine when to terminate.


% In each iteration $h$, we query an LLM for a candidate next skill $\pi_h$ to be executed for achieving a task, and compute its language model score $S_{\text{skill}}(\pi_h)$ in Eq.~\ref{eq:lm-step-score}.
% This score represents the likelihood of skill $\pi_h$ being the correct skill to execute from a language modeling perspective to satisfy instruction $i$.
% The input prompt contains context $c$ in the form of several long-horizon planning examples to inform the LLM of the available skills in the robot's library and their usage semantics. 
% The prompt also contains the instruction $i$, history of planned skills $\pi_{1:h-1}$, initial state $s_1$, and predicted states $s_{2:h}$ up to step $h$. Descriptions of the states $s_{1:h}$ come in the form of sets of binary propositions such as \graytext{on(milk, table)} or \graytext{under(yogurt, rack)} that describe spatial relationships between objects. To generate these propositions for initial state $s_1$, we use a heuristic approach described in the supplemental material. Our framework is agnostic to the specific approach. Suitable examples could be scene graph generation methods such as~\cite{rosinol2021kimera, hughes2022hydra}. States $s_{2:h}$ are predicted given skills $\pi_{1:h-1}$ and associated dynamics models. We then automatically generate corresponding language descriptions $s_{2:h}$ from these predicted states. Sec.~\ref{subsec:prompt-engineering} provides an example.

% We query the LLM to output up to $K$ candidate skills (see Sec.~\ref{subsec:prompt-engineering}) $\{\pi_h^1, \dots, \pi_h^K\}$, each with their own language model scores (Eq.~\ref{eq:lm-step-score}). 
% These candidates are then checked for geometric feasibility as described in Section~\ref{sec:geo_feas}. 
% How many skills $H$ are required to satisfy the instruction $i$ is unknown, so it is up to the LLM to decide when to terminate. 
% We use the goal proposition prediction method described in Section~\ref{sec:goal_pred} to decide when to terminate.

%For task planning, the LLM is given several long-horizon planning examples as context to inform it of the available skills in the robot's library and their usage semantics. It is also given a description $\mathfrak{s}$ of the current environment state $s$ in the form of a list of binary propositions such as \graytext{On(milk, table)} or \graytext{Under(yogurt, rack)} that describe spatial relationships between objects. Our approach is agnostic to the specific approach that is used for generating these propositions from image observations. Suitable examples could be scene graph generation methods such as~\cite{rosinol2021kimera, hughes2022hydra}. The details of our approach that is currently based on heuristics are described in the supplemental material. 
%already exists; scene graph generation methods such as~\cite{rosinol2021kimera, hughes2022hydra} have shown the ability to generate these spatial descriptions of images. In this work, we use handcrafted heuristics to provide these spatial descriptions, and leave the integration of learned methods for scene graph generation to future work.

%The LLM is asked to output the next skill $\pi_h$ to execute given the instruction $\mathfrak{i}$, history of planned skills $\pi_{1:h-1}$, initial state $\mathfrak{s}_1$, and predicted states $\mathfrak{s}_{2:h}$ up to step $h$. These states $s_{2:h}$ are predicted by the dynamics model. Corresponding language descriptions $\mathfrak{s}_{2:h}$ are generated from these state predictions. 

% We query the LLM to output up to $K$ candidate skills (see Sec.~\ref{subsec:prompt-engineering}) $\{\pi_h^1, \dots, \pi_h^K\}$ each with their own language model score (Eq.~\ref{eq:lm-step-score}). These candidates are then checked for geometric feasibility as described in Section~\ref{sec:geo_feas}. 

% How many skills $H$ are required to satisfy the instruction $i$ is unknown, so it is up to the LLM to decide when to terminate. We use the goal proposition prediction method described in Section~\ref{sec:goal_pred} to decide when to terminate.

% Therefore, the LLM may also predict ``stop" as the next language token, at which point the instruction is deemed to be satisfied and the current plan is returned.



% For each candidate skill, we perform motion planning to check its geometric feasibility, and then select the one with the highest combined TAMP score (Eq.~\ref{eq:tamp-step-score}). This selected skill $\pi_h$ is added to the current plan and included for planning the next timestep.

% The LLM is asked to output $K$ candidate skills $\{\pi_h^1, \dots, \pi_h^K\}$ for the current planning iteration $h$.


\subsection{Geometric feasibility planning}\label{sec:geo_feas}

% While the LLM may be used to predict the most probable next skill from a language modeling perspective, this skill may either be symbolically incorrect (e.g. grasp an object while already holding something) or may neglect geometric constraints such as the configuration of the environment or the kinematics of the robot. 
% Thus, we propose to select the next action $\pi_h$ by also considering its geometric feasibility $S_{\text{geo}}(\pi_h)$ (Eq.~\ref{eq:motion-step-score}) in the context of the full skill sequence $\pi_{1:h}$.

% We therefore propose selecting the next action $\pi_h$ by also considering its geometric feasibility $S_{\text{geo}}$ (Eq.~\ref{eq:motion-step-score}) in the context of executing the full skill sequence $\pi_{1:h}$.

While the LLM may be used to predict the most probable next skill $\pi_h$ from a language modeling perspective, the skill may either be symbolically incorrect (e.g. grasp an object while already holding something) or may neglect geometric constraints (e.g. robot kinematics) in the current state. 
% that the preceding skill subsequence $\pi_{1:h-1}$ is led the .
% While the LLM may be used to predict the most probable next skill $\pi_h$ from a language modeling perspective, $\pi_h$ may neglect geometric constraints (e.g. environment configuration) when taken in the context of its preceding skill subsequence $\pi_{1:h-1}$.
Thus, we propose to compute the geometric feasibility score $S_{\text{geo}}(\pi_h)$ (Eq.~\ref{eq:motion-step-score}) of skill $\pi_h$ after maximizing the combined success probability of the entire skill sequence $\pi_{1:h}$.
We term this process \textit{policy sequence optimization}.
As a result, the skills $\pi_{1:h-1}$ are expected to receive rewards $r_{1:h-1}$, which supports the independence assumption of Eq.~\ref{eq:tamp-score-indep}.

% To this end, we first resolve geometric dependencies across the skill sequence $\pi_{1:h}$ by maximizing the combined success likelihood of the skills' individual actions $a_{1:h}$ (Eq.~\ref{eq:motion-score}).
% This can be computed as the product of step reward probabilities,
% combined success likelihood of the skills' individual actions $a_{1:h}$ (Eq.~\ref{eq:motion-score}).
% This can be computed as the product of step reward probabilities,

To this end, we first resolve geometric dependencies across the full skill sequence $\pi_{1:h}$ by maximizing the 
product of step reward probabilities of the skills' individual actions $a_{1:h}$:
% \begin{equation}
%     p(r_{1:h} \mid s_1, \pi_{1:h}) = \operatorname{E}_{s_{2:h} \sim T}\left[\prod_{t=1}^h p(r_t \mid s_t, a_t)\right], \label{eq:taps-objective}
% \end{equation}
% \begin{equation}
%     a_{1:h}^* = \arg \max_{a_{1:h}} \, \operatorname{E}_{s_{2:h} \sim T}\left[\prod_{t=1}^h p(r_t \mid s_t, a_t)\right], \label{eq:taps-objective}
% \end{equation}
\begin{equation}
    a_{1:h}^* = \arg \max_{a_{1:h}} \, \prod_{t=1}^h p(r_t \mid s_t, a_t), \label{eq:taps-objective}
\end{equation}
where future states $s_{2:h}$ are predicted by dynamics models $s_{t+1} = T^\pi(s_t, a_t)$. Note that the reward probability $p(r_t = 1\mid s_t, a_t)$ is equivalent to a Q-function $Q^{\pi_t}(s_t, a_t)$ for skill $\pi_t$ in a contextual bandit setting with binary rewards.
% ; for simplicity, we will refer to these reward probabilities as Q-functions. 
Text2Motion is agnostic to the specific approach on how to learn the dynamics $T^\pi$ and Q-functions $Q^\pi$ and on how to optimize skill policy outputs $a_{i} \sim \pi_{i}(a \vert s_i)$.
%It is also possible to consider a stochastic formulation of $T^\pi(s_{t+1} \vert s_t, a_t)$.
In the experiments, we leverage STAP~\cite{taps-2022}.

% To this end, we select the actions $a_{1:h}$ (where $a$ is the output of a particular skill's policy that can be further optimized such as through sampling or optimization using a $Q$ function as the objective) for the entire sequence of skills $\pi_{1:h}$ to maximize their combined probability of success (Eq.~\ref{eq:motion-score}). This can be computed as the product of step reward probabilities,
% \begin{equation}
%     p(r_{1:h} \mid s_1, \pi_{1:h}) = \operatorname{E}_{s_{2:h} \sim T}\left[\prod_{t=1}^h p(r_t \mid s_t, a_t)\right], \label{eq:taps-objective}
% \end{equation}
% where future states $s_{2:h}$ are predicted by dynamics models $s_{t+1} \sim T^\pi(s_t, a_t)$. Note that the reward probability $p(r = 1 \mid s_h, a_h)$ is equivalent to a Q-function $Q^{\pi_h}(s_h, a_h)$ for skill $\pi_h$ in a contextual bandit setting with binary rewards; for simplicity, we will refer to these reward probabilities as Q-functions. How the dynamics $T^\pi$ and Q-functions $Q^\pi$ are learned and how skill policy outputs $a_{1:h}$ are optimized are outside the scope of this paper. Text2Motion is agnostic to the specific choice. In the experiments, we leverage TAPS~\cite{taps-2022}.

We can then estimate the geometric feasibility $S_{\text{geo}}(\pi_h)$ (Eq.~\ref{eq:motion-step-score}) in the context of the full skill sequence $\pi_{1:h}$ by the Q-value of the last action,
% \begin{equation}
%     p(r_h \mid s_{1:h}, \pi_{1:h}) = Q^{\pi_h}(s_h, a_h),
% \end{equation}
\begin{equation}\label{eq:q-function-score}
    p(r_h = 1 \mid s_{1}, \pi_{1:h}) \approx Q^{\pi_h}(s_h, a^*_h),
\end{equation}
% where $a_{1:h}$ are determined by TAPS and $s_{2:h}$ are predicted by the dynamics model. 
where $a^*_{h}$ is determined by STAP and $s_{h}$ is predicted by the dynamics model. 
The Q-value is multiplied by the language model likelihood (Eq.~\ref{eq:lm-step-score}) to produce the combined overall score (Eq.~\ref{eq:tamp-step-score-decomp}) for this skill. 

This process is performed for each of the candidate skills $\{\pi_h^1, \dots, \pi_h^K\}$ generated by the LLM at iteration $h$, and the one with the highest overall score is added to the current running plan. 
This skill should be both geometrically feasible and help make progress towards satisfying the instruction $i$.

% Among the set of $K$ candidate skills $\{\pi_h^1, \dots, \pi_h^K\}$ for timestep $h$ generated by the LLM, the one with the highest overall score is added to the current running plan. This skill should be both geometrically feasible and help make progress towards satisfying the instruction $i$.
During planning, the LLM may propose skills $\pi_h^k$ that are out-of-distribution (OOD) given the current state.
For instance, $\pi_h^k$ may be symbolically incorrect, like \graytext{Pick(table)} or \graytext{Place(yogurt, table)} when the yogurt is not in hand.
In a more subtle case, the state may have drifted beyond the training distribution of $\pi_h^k$ as a result of the preceding subsequence $\pi_{1:h-1}$.
% During planning, the LLM may propose a sequence of skills that produces states which are out-of-distribution (OOD) for the learned skills, such as performing a ``pick" action on an unmovable object like the table, or performing a ``place" action when the robot has not yet grasped any object. 
Such skills may end up being selected if we rely on learned Q-values, since the Q-value for an OOD state-action pair may be spuriously high. 
We therefore reject skills proposed by the LLM that are predicted to be OOD by an ensemble of Q-networks \cite{lakshminarayanan2017simple}.
% For OOD detection, we use ensembles of Q-networks for their ease of calibration. 
We consider a skill to be OOD if the standard deviation for the associated Q-value predicted by the ensemble is greater than a threshold $\epsilon_{\text{OOD}}$.

If, after OOD rejection, a state $s_{h}$ in an optimized policy sequence is classified to satisfy the instruction (details in Sec.~\ref{sec:goal_pred}), we terminate search and return the solution $a_{1:h}$ for execution. 
If we instead hit the maximum planning horizon, we register a planning failure and do not execute any actions. 
% We note that it is also possible to take the first (geometric feasibility optimized) action and replan at the next step, as our imperfect dynamics models may have lead to a imperfect geometric states than were not classified as instruction satisfying.

% If the LLM predicts the ``stop" token as the most likely candidate skill for timestep $H$, then the instruction $\mathfrak{i}$ is deemed to be satisfied, and we can return the solution plan $a_{1:H-1}$ for execution. If the LLM does not predict ``stop", then we may terminate when the maximum planning horizon is reached, or if any of the Q-values along the output plan are below $\epsilon_{OOD}$, indicating a likely failure. In the case of such predicted failures, we simply execute the first action in the plan and re-plan from the updated observation.


\subsection{LLM goal prediction} \label{sec:goal_pred}

Given a library of predicate classifiers $\mathcal{L}^P$ describing simple geometric relationships of objects in the scene (e.g. \{\graytext{on(a, b)}, \graytext{inhand(a)}, \graytext{under(a, b)}\}), a list of objects $O$ in the scene, and an instruction $i$, we use the LLM to predict $j$ goal proposition sets $g_{1:j}$ that would satisfy the instruction. 

For example, given an instruction $i=$``stock all the dairy products onto the rack,'' the set of objects $O=$ \{\graytext{table}, \graytext{milk}, \graytext{keys}, \graytext{yogurt}, \graytext{rack}\}, and a library of predicate classifiers with their natural language descriptions $\mathcal{L}^P=$ \{\graytext{on(a, b)}, \graytext{inhand(a)}, \graytext{under(a, b)}\}, the LLM might predict the goal proposition set $g_1=$ \{\graytext{on(milk, rack)}, \graytext{on(yogurt, rack)}\}. If the instruction were $i=$ ``put one dairy product on the rack'', the LLM might predict $g_1=$\{\graytext{on(milk, rack)}\}, $g_2=$\{\graytext{on(yogurt, rack)}\}. 

We define a satisfaction function $f_{\text{sat}}^{g_{1:j}}(s) \in \{0, 1\}$ that checks whether the current symbolic state $s$---obtained from the predicate classifiers $\mathcal{L}^P$---satisfies any of the goal proposition sets $g_{1:j}$.
We use $f_{\text{sat}}$ in two ways: i) as a success classifier to check whether a given state satisfies the natural language instruction and ii) as a chain-of-thought prompt \cite{wei2022chain} when prompting the LLM for action sequences (see Sec.~\ref{subsec:prompt-engineering}).

% \subsection{@Toki New stuff here}

% \subsubsection{Integrated (beam search w/ beam size 1)} -  \href{https://github.com/kevin-thankyou-lin/text2motion/blob/a65686c9163bca99a0709df3308c975598bbd36a/temporal_policies/task_planners/beam_search.py#L585}{code is here}:

% \begin{enumerate}
%     \item Have current state (+ all history) as the 'root' node.
%     \item get-successors(state) by prompting LM to generate K actions
%     \item score successor state by i) perform TAPS optimization from ground truth env state up till the particular node we're looking at; the affordance V score for the successor state is the TAPS optimized last action and LM score is LM score
%     \item check if any successor state reaches goal; if so check if the Q values along the trajectory are all > 0.45 (w/ stds from Q function <= ~0.15 depedning on the skill)
%     \item kill all but 1 successor states
%     \item continue this loops til max depth of until we've found an end node
%     \item take first action and repeat
% \end{enumerate}

% \subsubsection{3 ways to check for is-end both for planning termination criteria and execution termination criteria}

% \begin{enumerate}
%     \item saycan-style of scoring `stop()', given sequence of observed object relationships (containing on/inhand/under) and executed actions
%     \item predicting `stop()', given sequence of observed object relationships (containing on/inhand/under) and executed actions
%     \item predicting goal propositions first and then taking a raw state and checkig if those goal propositions hold 
% \end{enumerate}

% For the `best' method, I think the predicting `done' method is the cleanest (while 3) is the cheapest option)

% \subsubsection{hierarchical baseline}

% \begin{enumerate}
%     \item ask LLM to shoot out K plans
%     \item test if any plan reaches the goal using the `best' is-end check from above; if so, check if the plan's Q values (post taps optimization) are all > 0.45 w/ std > ~0.15
%     \item if there are 2+ plans left, take the plan with highest Q product (use closed loop motion planning - no more task planning)
%     \item if no plans left, fall back to saycan and take a step
% \end{enumerate}


% \subsection{@Toki Old stuff here}
% Given a natural language instruction, we aim to find a sequence of skills and continuous parameters that satisfy the instruction.
% We solve this problem in two steps. First, we parse the natural language instruction to a proposition that evaluates to true or false. Second, we find a sequence of skills and their continuous parameters that satisfy the  proposition.

% \subsection{Semantic parsing of language instructions}
% \label{subsec:semantic-parsing-method}

% A challenge in performing TAMP from language is that goals $\mathfrak{i}$ are expressed in free-form text instead of as a logical expression of predicates $\mathcal{G}$, which TAMP solvers require.

% % Given a library of predicate classifiers $\mathcal{L}^P$, $\mathcal{G}$ can also be used to validate the final geometric state of a motion plan. $\mathcal{G}$ thereby imposes useful symbolic and geometric constraints on TAMP solvers.

% % In AI task planning, $\mathcal{G}$ defines goal states for which the resulting state of a correct task plan from $\mathcal{I}$ must lie within.

% Therefore, given a prompt $\mathfrak{p}$ that consists of the natural language instruction, the first step of our method parses the natural language instruction $\mathfrak{i}$ into python code that composes the predicates and objects into a single goal proposition $\mathcal{G}$ using an LLM (Eq.~\ref{eq:goal-classification}).
% \begin{equation}
%     \label{eq:goal-classification}
%     \mathcal{G} = \mathrm{LLM}(\mathfrak{i}, \mathfrak{p})
% \end{equation}

% We use the parsed goals in two ways: i) as a variant of a chain-of-thought prompt \cite{wei_chain_2022} when prompting the LLM for action sequences ii) as a success classifier to check whether a given state satisfies the natural language instruction.

% % \klin{The following should be in the next section?}
% % Because task planning cannot be performed in the absence of a symbolic transition model $\mathcal{T}$, the second step of our planner performs open-ended task plan generation with the LLM conditioned on $\mathcal{G}$ (Eq.~\ref{eq:cond-plan-generation}).
% % \begin{equation}
% %     \label{eq:cond-plan-generation}
% %     \tau = \mathrm{LLM}(\mathcal{G}, \mathfrak{i}, \mathfrak{p})
% % \end{equation}

% % Without conditioning on $\mathcal{G}$, plans produced by LLMs in open-ended fashion have been shown to contain inaccuracies.
% % Thus, \textbf{our first hypothesis (H1)} is that conditioning LLM plans via Eq.~\ref{eq:cond-plan-generation} will be more accurate than Eq.~\ref{eq:plan-generation}, shown below.
% % \begin{equation}
% %     \label{eq:plan-generation}
% %     \tau = \mathrm{LLM}(\mathfrak{i}, \mathfrak{p})
% % \end{equation}
% % Furthermore, the intended downstream uses of $\mathcal{G}$ places a high fidelity requirement on its prediction; any errors will cascade to task planning and motion planning, potentially nullifying correct solutions. 
% % Hence, \textbf{our second hypothesis (H2)} is that when prompted equivalently, LLMs are better able to perform semantic goal parsing (Eq.~\ref{eq:goal-classification}) than open-ended plan generation (Eq.~\ref{eq:plan-generation}). 
% % Our \textbf{third hypothesis (H3)} is that by querying the LLM for $K$ task plans instead of a single one via Eqs.~\ref{eq:cond-plan-generation}-~\ref{eq:plan-generation}, we acquire diversity that increases the likelihood of a correct plan for unseen tasks. 

% \subsection{Language TAMP with Text2Motion}
% \label{subsec:language-tamp-method}

% \klin{Two potential `methods', previously, I was more confident the integrated approach would definitely work better, but unclear now because the LM probabilities are kind of wild and the two normalization schemes I've tried both have issues. Will check how well adding extra prompts does. Maybe we can just tune the normalization coefficient until things work okay?}

% \textbf{Hierarchical Text2Motion} 

% \begin{enumerate}
%     \item Query LM for $K$ plans
%     \item TAPS optimize all plans
%     \item Filter out non-goal reaching plans \item Select remaining plans with maximum Q product (normalized)
% \end{enumerate}

% \textbf{Integrated Text2Motion} This approach uses a beam search on the sequence of discrete skill actions. At each iteration of increasing the depth of the search, we re-run TAPS until convergence to get the value of the discrete skill action sequence.

% % In the previous section, we described a process that semantically parses a language instruction $\mathfrak{i}$ (with prompt $\mathfrak{p}$) into a propositional goal $\mathcal{G}$ and a set of $K$ candidate plan skeletons $\{\tau_k\}_{k=1}^K$. 
% We assume the existence of a manipulation skill $\pi$ for every symbolic action $a \in \mathcal{A}$ available to the LLM when constructing the plan skeletons.
% Thus, each candidate plan skeleton defines a sequence of skills $\tau = [\pi_1, \ldots, \pi_H]$ that \textit{might} solve the task if executed by the robot.

% A plan skeleton must satisfy two criterion to solve a task: (a) it must be geometrically feasible; (b) executing the plan skeleton must result in a final geometric state that is consistent with goal predicates $\mathcal{G}$.
% TAPS is a recent work that fits these requirements: it resolves long-horizon action dependencies in plan skeletons, returning an optimized action plan $\xi = [a_1,, \ldots, a_H]$ and success likelihood for the plan $J$.
% \begin{equation}
%     \label{eq:taps-call}
%     \xi, J = \mathrm{TAPS}(\tau, \mathcal{L}^\pi)
% \end{equation}
% TAPS' training framework is used to learn a manipulation skill library $\mathcal{L}^\pi$ which includes skill-specific Q-functions $Q^\pi$ and dynamics models $\mathcal{T}^\pi$.

% We apply Eq.~\ref{eq:taps-call} to all plan skeletons $\{\tau_k\}_{k=1}^K$, and use the dynamics models $\mathcal{T}^\pi$ to rollout each action plan $\xi$ into a sequence of geometric states $\chi = [x_1, \ldots, x_H]$.
% Predicate classifiers $\mathcal{L}^P$ are used to check the final state of each plan $x_H$ against the predicted goal predicates $\mathcal{G}$.
% A set of satisfying plans are computed, excluding any plans that do not satisfy the goal:
% \begin{equation}
%     \label{eq:optimal-plan}
%     \Xi = \{\xi\; |\; p(\chi[-1]) = True, \forall str(p) \in \mathcal{G}\}.
% \end{equation}

% The optimal plan is then the most geometrically feasible plan that satisfies the goal:
% \begin{equation}
%     \label{eq:optimal-plan}
%     \xi^* = \underset{J}{\arg max}\; \Xi.
% \end{equation}

% We highlight that $\xi^*$ need not be followed open-loop.
% Repeating the TAMP procedure after executing the first action $\xi^*[0]$ achieves closed-loop behavior.
% In this way Text2Motion is made more robust to imperfect dynamics and value functions, failed actions, and environmental perturbations. 

% There are two underlying hypotheses to performing TAMP with Text2Motion:
% \begin{itemize}
%     \item Success likelihoods computed by TAPS are robust enough to rank plans of various length and complexity \textbf{(H4)}
%     \item Text2Motion can generalize zero-shot to unseen long-horizon TAMP problems better than baselines \textbf{(H5)}
% \end{itemize}

% \subsection{Cost reduction via Generator-Scorer}
% Generator-Scorer a two step process that drastically reduces the cost of using an LLM to score actions for robotics. It involves adding skill function signatures in the prompt, such as \texttt{available skills: pick(a), place(a, b), push(a, b, c), pull(a, b)}

% Then:

% \begin{enumerate}
%     \item Prompt the LLM to generate a sequence of actions
%     \item Take those actions and score them
% \end{enumerate}

% \klin{Adding as place holder for contribution 3; makes most sense if the integrated approach works. Unclear if/how to include this contribution if the integrated approach is too flakey ...}

% We empirically found that using \texttt{text-davinci-003} to generate action skeletons and \textttt{code-davinci-002} to score actions resulted in the best results, so we use this combination of language models for the experimental results.

% Add a small lemma about how our method doesn't harm you e.g. add BFS or something
\section{Experiments}
\label{sec:experiments}

We conduct experiments to test the following hypotheses:
\begin{description}
    % \item[\textbf{H1}] LLMs, robot skills, and policy sequence optimization must be jointly considered to solve TAMP-like tasks.
    \item[\textbf{H1}] Policy sequence optimization is a necessary ingredient for solving TAMP-like problems specified by natural language with LLMs and robot skills.
    % \item[\textbf{H2}] Text2Motion can solve partially observable problems that require joint semantic and geometric reasoning.
    \item[\textbf{H2}] Integrated semantic and geometric reasoning is better equipped to solve tasks with partial affordance perception (PAP, as defined in Sec.~\ref{subsec:task-suite}) compared to hierarchical planning approaches.
    \item[\textbf{H3}] \textit{A priori} goal proposition prediction is a more reliable plan termination strategy than prior scoring methods.
    % \item[\textbf{H2}] Integrated planning is suited to problems where combinatorial complexity can be reduced by accounting for geometric feasibility
    % \item[\textbf{H3}] Plans produce by Text2Motion are more likely to succeed during closed-loop excution. 
\end{description}
The following subsections describe the baseline methods we compare against, details on LLMs and prompts, the tasks over which planners are evaluated, and performance metrics.

\subsection{Baselines}
\label{subsec:baselines}

We compare Text2Motion with a series of language-based planners.
% These methods share similar mechanistic assumptions in the use LLMs and primitive skills $\mathcal{L}^\pi$ to solve Eq.~\ref{eq:tamp-score} in a few-shot manner (i.e. given a number of in-context prompt examples).
For consistency, we use the same set of independently trained primitive policies $\pi_i$, Q-functions $Q^{\pi_i}$, OOD rejection strategy and, where appropriate, dynamics models $T^\pi(s, a)$ across all methods and tasks.
% Because the baselines perform various forms of action ranking through $Q^{\pi_i}$, they are also susceptible to OOD failures from spuriously high Q-values.
% Furthermore, we also provided the baselines with the same OOD rejection strategy --thresholding standard deviations computed over an ensemble of Q-functions--to equalize performance variation along this axis.
% Where needed, value estimates are computed via Monte-Carlo estimation as $V^\pi(s)=\mathbb{E}_\pi[Q^\pi(s, a)]$. 

\textbf{\scgs{}:}
We implement a cost-considerate variant of SayCan with a module dubbed Generator-Scorer (GS).
At each timestep $h$, vanilla SayCan~\cite{saycan-2022} ranks \textit{all possible} actions by $p(\pi_h \mid i, \pi_{1:h-1}) \times V^{\pi_h}(s_h)$, before executing the top scoring action (Scorer).
However, the cost of ranking actions scales unfavorably with the number of scene objects and skills in library $\mathcal{L}^\pi$.
\scgs{} limits the pool of actions considered in the ranking process by querying the LLM for the $K$ most useful actions $\{\pi_h^1, \dots, \pi_h^K\} \sim p(\pi_h \mid i, \pi_{1:h-1})$ (Generator) before engaging Scorer.
Execution terminates when the score of the \texttt{stop} ``skill'' is larger than the other skills.

\textbf{\imgs{}:}
We implement the \textit{Object + Scene} variant of Inner Monologue~\cite{innermono-2022} by providing task-progress scene context in the form of the environment's symbolic state.
We acquire \imgs{} by equipping~\cite{innermono-2022} with the Generator-Scorer for cost efficiency. LLM action likelihoods are equivalent to those from \scgs{} except they are now also conditioned on the visited state history.
%LLM action likelihoods are now also conditioned on the state history $s_{1:h-1}$ and action history $p(\pi_h \mid i, s_{1:h-1}, \pi_{1:h-1})$ visited by the planner.

\textbf{\hm{}:}
We propose and compare with a third baseline method, which we term the \hm{} method.
Like Text2Motion, \hm{} predicts a set of goal propositions for the given instruction $i$ and then converts the goal propositions into a goal satisfaction checker $f_{\text{sat}}$ (see Sec.~\ref{sec:goal_pred}).
In contrast to Text2Motion, which interleaves task planning with policy sequence optimization, \hm{} uses the LLM to generate entire skill sequences $\pi_{1:H}$ (Eq.~\ref{eq:task-score}), runs policy sequence optimization for each of the candidate sequences, and filters out plans with low predicted Q-values or OOD state-action pairs. 
Then, if a geometric state $s_i$ along the optimized trajectory of a given plan satisfies $f_{\text{sat}}$, we add the plan (up to that geometric state) to a set of final candidate plans. 
Finally, if multiple candidates remain, we select the one with highest success probability (Eq.~\ref{eq:taps-objective}).

% A plan is selected if its final state $\mathfrak{s}_{H+1}$ predicted by the motion planner satisfies the  goal propositions.
% If multiple plans meet this criterion, the plan with the highest success probability (Eq.~\ref{eq:taps-objective}) is chosen.
% Without this check, using the LLM for open-ended plan generation is likely to result in many incorrect plans. 

% This baseline also verifies that the instruction $\mathfrak{i}$ has been satisfied by using the LLM to predict the next primitive from the final state $\mathfrak{s}_{H+1}$ predicted by the motion planner. 
% If the next primitive is not ``stop", then the primitive sequence $\pi_{1:H}$ initially generated by the LLM is incomplete and should be filtered out. 

% If no plans remain after filtering, then like Text2Motion, the \hm{} baseline simply executes the first action for the best plan and re-plans from the updated state observation.

% \begin{enumerate}
    
%     \item \textbf{SayCan-GS} is a financially / and computationally cheaper version of SayCan that uses GeneratorScorer to score the top $K$ actions, instead of all possible actions. 

%     \item \textbf{Single-plan SayCan-GS} Basically, the interleaved method except beam size of $1$. Uses a dynamics model to roll forward actions with interleaved TAPS trajectory optimization.  \klin{should upper bound SayCan-GS, assuming good dynamics}

%     \item \textbf{Hierarchical Text2Motion} 
%     \begin{enumerate}
%         \item Query LM for $K$ plans
%         \item TAPS optimize all plans
%         \item Filter out non-goal reaching plans \item Select remaining plans with maximum Q product (normalized)
%     \end{enumerate}
% \end{enumerate}
\begin{figure}
    \centering    
    \includegraphics[width=0.5\textwidth]{figures/imgs/evaluation_task_suite.pdf}
    \vspace{-10pt}
    \caption{\textbf{TableEnv Manipulation evaluation task suite}. We evaluate the performance of all methods on a task suite based on the above manipulation domain. The tasks considered vary in terms of difficult and each contains a subset of three properties: being long-horizon (Tasks 1, 2, 3, 5, 6), containing lifted goals (Tasks 4, 5, 6), and having partial affordance perception (Tasks 4, 5, 6). During evaluation, we randomize the geometric parameters of each task for each random seed.}
    \label{fig:evaluation_task_suite}
    \vspace{-10pt}
\end{figure}

\subsection{Large language model}
\label{subsec:llm}

We use two language models, both of which were accessed through the OpenAI API: i) \texttt{text-davinci-003}, a variant of the InstructGPT \cite{ouyang2022training} language model family which is finetuned from GPT-3 with human feedback and ii) the Codex model \cite{chen2021evaluating} (specifically, \texttt{code-davinci-002)}. To generate task plans for the \hm{} method, we empirically found \texttt{text-davinci-003} to be the most capable for \hm{}'s task plan generation queries and thus use it for that purpose. For all other queries, we use \texttt{code-davinci-002} in our experiments as they were found to be reliable. We do not train the LLMs in any way and use only few shot prompting.


\subsection{Prompt engineering}
\label{subsec:prompt-engineering}

We use the same set of in-context examples for all methods and for all tasks in the prompts passed to the LLM. 
A single prompt has the following structure (prompt template is in black and LLM output is in \textcolor{orange}{orange}):

\vspace{0.3cm}

\noindent\fbox{\parbox{0.97\linewidth}{\small{\texttt{{
Available scene objects: [`table', `hook', `rack', `yellow box', `blue box', `red box']\\\\
Object relationships: [`inhand(hook)', `on(yellow box, table)', `on(rack, table)', `on(blue box, table)']\\\\
Human instruction: How would you push two of the boxes to be under the rack?\\\\
Goal predicate set: {\color{code-constant}[[`under(yellow box, rack)', `under(blue box, rack)'], [`under(blue box, rack)', `under(red box, rack)'], [`under(yellow box, rack)', `under(red box, rack)']]}\\\\
Top 5 next valid robot actions (python list): {\color{code-constant}['push(yellow box, rack)', 'push(red box, rack)', 'place(hook, table)', 'place(hook, rack)', 'pull(red box, hook)']}
}}}}}
\vspace{0.3cm}

The prompt above chains the output of two queries together: one for goal proposition prediction (Sec.~\ref{sec:goal_pred}), and one for next skill generation (Sec.~\ref{sec:llm-task-planning}). To score skills (Eq.~\ref{eq:lm-step-score}), we replace \texttt{Top 5 next valid robot actions} with \texttt{Executed action: }, append the generated skill (e.g. \graytext{Push(yellow box, rack)}), and then add token log-probabilities. We provide the full set of in-context examples in the Appendix (Appx.~\ref{sec:suppl-incontext-examples}).
%We provide the prompts used for our tasks in Appendix \todo{\ref{}}. 

\subsection{Task suite}
\label{subsec:task-suite}

% \todo{terminology: we construct a task suite, not a suite of domains }
We construct a suite of evaluation tasks (Fig.~\ref{fig:evaluation_task_suite}) in a table-top manipulation domain.
Each task includes a natural language instruction $i$ and initial state distribution $\rho(s)$ from which geometric task instances are sampled. 
For the purpose of experiment evaluation only, tasks also contain a ground-truth goal criterion to evaluate whether a plan has satisfied the corresponding task instruction.
% The task suite features characteristic permutations of tasks that are long-horizon, contain lifted goals, and are partially observable.
Finally, each task contains subsets of the following properties:
\begin{itemize}
    \item \textbf{Long-horizon (LH):} task requires skill sequences $\pi_{1:H}$ of length six or greater to solve. For example, Task 1 in Fig.~\ref{fig:evaluation_task_suite} requires the robot to pick and place three objects for a total of six actions. In our task suite, \textbf{LH} tasks also contain geometric dependencies that span across the sequence of action that are unlikely to be solved by myopically executing skills. For example, Task 2 (Fig.~\ref{fig:evaluation_task_suite}) requires the robot to pick and place obstructing boxes (i.e. blue and yellow) to enable a future hook-push of the cyan box underneath the rack.
    \item \textbf{Lifted goals (LG):} Goals are expressed over object classes rather than object instances. For example, the lifted goal instruction "move three boxes to the rack" specifies an object class (i.e. boxes) rather than an object instance (e.g. the red box). This instruction is used for Task 3 (Fig. ~\ref{fig:evaluation_task_suite}). 
    % Instructions express goals over object classes and properties rather than over object instances. Task 3 in Fig.~\ref{fig:evaluation_task_suite} has the instruction ``How would you move three of the boxes to the rack?'' Instruction satisfaction depends on the number of boxes on the rack, rather than the location of any specific object. 
    Moreover, \LG{} tends to correspond to planning tasks with many possible solutions. For instance, there may only be a single solution to the non-lifted instruction ``fetch me the red box and the blue box,'', but an LLM must contend with more options when asked to, for example, ``fetch any two boxes.''
    \item \textbf{Partial affordance perception (PAP):} Skill affordances cannot be perceived solely from the spatial relationships described in the initial state $s_1$. For instance, Task 5 (Fig.~\ref{fig:evaluation_task_suite}) requires the robot to put two boxes onto the rack. However, the scene description obtained through predicate classifiers $\mathcal{L}^P$ (described in Sec.~\ref{sec:goal_pred}) and the instruction $i$ do not indicate whether it is necessary to use a hook to pull an object closer to the robot first. 
\end{itemize}

% \LH{} tasks contain geometric dependencies than span across the sequence of actions, and are unlikely to be solved by greedily executing skills. 
% Consider Task 2 (Fig.~\ref{fig:evaluation_task_suite}), which requires the robot to pick and place obstructing boxes (i.e. blue and yellow) to enable a hook-push of the cyan box underneath the rack.
% Such dependencies are unlikely to be resolved by greedily executing skills in sequence, and thus highlights the importance of policy sequence optimization.
% \LG{} offers a means to increase the combinatorial complexity of planning problems through logical disjunction. 
% \LG{} tends to correspond to planning tasks with many possible solutions.
% For instance, there may only be a single solution to the non-lifted instruction ``fetch me the red box and the blue box,'', but the LLM must contend with more options when asked to ``fetch any two boxes.''
% Note that it may often be tedious or even impossible to represent \LG{} tasks propositionally, which complicates the use of the symbolic solvers that rely on planning domains.
% \PO{} represents a class of tasks where it is difficult to construct a geometrically feasible plan by solely considering the semantic information in the prompt (Sec.~\ref{subsec:prompt-engineering}).
% Since we wish to decrease the reliance of a complex set of predicates and pre-and post condition e.g. the large set of predicates what PDDLStream uses
% These tasks difficult to solve solely from semantic information included in the prompt (Sec.~\ref{subsec:prompt-engineering}), 
% Such tasks require reasoning about more than the semantic information included in the prompt (e.g. the instruction and scene description),
% (such as the instruction and description of the scene); these tasks require reasoning about the physical geometries of the scene itself.
% We also include combination tasks (e.g. \LG{} + \PO{}) to study the unique challenges that arise when tasks contain characteristics of several problem classes. 

% We refer to Appendix A for additional details on planning domains and instructions.
%We consider a family of language planners that relies on prompt-conditioned LLM likelihoods to deduce long-horizon action sequences $\pi_{1:H}$.
% While LLMs have been evaluated on their ability to solve various classes of planning problems~\cite{llms-cant-plan-2022}, the problem setting we consider differs.
% % While LLMs have been evaluated on their ability to solve various classes of planning problems~\cite{llms-cant-plan-2022} (akin to classical task planners~\cite{jiang2019task, long20033rd}), the settings we consider differ in several ways.
% For solving task and motion planning problems, the success of a plan is a product of both its symbolic correctness and its geometric feasibility.
% Thus in our planning domains, the notion of planning performance is shifted from how well the LLM plans in isolation to how useful the LLM is for task and motion planning. 
%Second, to eliminate the need for detailed symbolic domain specifications required by classical TAMP solvers, the LLM is only provided with the actions it may use $\mathcal{L}^\pi$ and a basic scene description in the form of spatial relationships like \graytext{On(box, rack)}.




% Planning domains are categorized by instruction type. 
% We include three instruction types of increasing complexity, adopting the structured language and long-horizon categories from a previous work [SayCan] and introducing the lifted instruction. 
% Brief descriptions are provided below.
% \begin{itemize}
%     \item \textbf{Structured language (SL)} instructions outline a high-level sequence of grounded actions the planner \textit{could} take to solve the tasks, while other solutions are possible.
%     \item \textbf{Long-horizon (LH)} instructions provide an explicit goal description over the scene configuration without explicitly specifying a long-horizon sequence of actions.
%     \item \textbf{Lifted (LF)} instructions describe an implicit goal description over object or scene properties without explicitly specifying a sequence of actions required to achieve them.
% \end{itemize}

% The instruction types ordered as above offer decreasing levels of task specificity to the planner, and as a result, obtaining solutions to Eq.~\ref{eq:task-score} becomes more difficult. 
% \textbf{SL} domains place the least dependence on LLM task planning by directly providing a correct action sequence, but may still require accounting for geometric action dependencies prevalent in TAMP problems.
% \LH{} instructions admit many solutions to a unique goal grounded over known objects, which places a higher dependence on the LLM to determine one such valid plan.
% Many real-world tasks are described over object properties instead of over objects themselves; for instance, \textit{put all the dairy products in the fridge}.
% \textbf{LF} domains mirror such tasks in an attempt to stress-test the semantic understanding of the LLM in the context of TAMP.
% We highlight that it may often be tedious or even impossible to represent \textbf{LF} tasks propositionally, which complicates the use of symbolic TAMP solvers.
% We refer to Appendix. A for additional details on instruction categories and planning domains.

\subsection{Evaluation and metrics}
\label{subsec:evaluation-metrics}


% We consider \scgs{} and \imgs{} reactive planners in that they rely on the LLM and value functions to ground admissible actions $\pi^{(k)}_h$ at the current timestep only. 
% In contrast, Text2Motion and \hm{} perform TAMP with an explicit plan in attempt to find one that satisfies the inferred goal state.
We evaluate Text2Motion and \hm{} by marking a plan as successful if, upon execution, it reaches a final state $s_H$ that satisfies the instruction $i$ of a given task. 
A plan is executed only if the final state $s_H$ predicted by the dynamics model satisfies the inferred goal propositions (Sec.~\ref{sec:goal_pred}). 

Two failure cases are tracked: i) planning failure: the method does not produce a sequence of skills $\pi_{1:H}$ whose optimized actions $a_{1:H}$ (Eq.~\ref{eq:taps-objective}) results in a state that satisfies $f_{\text{sat}}$ within a maximum plan length of $d_{\text{max}}$; ii) execution failure: the execution of a plan that satisfies $f_{\text{sat}}$ does not achieve the ground-truth goal of the task.

Since both our method and \hm{} use learned dynamics models to optimize actions with respect to (potentially erroneous) future state predictions, we perform the low-level execution of the skill sequence $\pi_{1:H}$ in closed-loop fashion. 
Thus, upon executing the skill $\pi_{h}$ at timestep $h$ and receiving environment feedback $s_{h+1}$, STAP~\cite{taps-2022} is called to perform policy sequence optimization on the remaining planned skills $\pi_{h+1:H}$. We do not perform task-level replanning.

\scgs{} and \imgs{} are reactive agents in that they execute the next best admissible skill $\pi_h$ at each timestep. 
Hence, we evaluate them in a closed-loop manner for a maximum of $d_{max}$ steps.
We mark a run as a success if the method issues the \texttt{stop} action and the current state satisfies the ground-truth goal. 
Note that this comparison is advantageous for these baselines because they are given the opportunity to perform closed-loop replanning at both the task and policy level, whereas task-level replanning does not occur for Text2Motion or \hm{}. 
We find that this advantage does not lead to measurable performance gains on the challenging evaluation domains that we consider.

% Since \scgs{} and \imgs{} are reactive planners in that they rely on the LLM and value functions to ground admissible actions $\pi^{(k)}_h$ at the current timestep only, there are subtle differences in their evaluation strategy.

% If termination is due to reaching $s_{max}$, we mark the run as a failure as the planner has not found a valid plan. Since our dynamics models are imperfect, we execute the predicted task plan open loop (no task plan level replanning) but execute the policy level plan closed loop. Since SayCan and Inner Monologue are not inherently planning methods, we evaluate them in a closed loop manner (on both the task and policy level) and report success if the method issues the `stop()' action and the current state satisfies the ground truth goal. Note that this comparison is not entirely fair because SayCan and Inner Monologue are essentially given the opportunity to perform closed loop policy \textit{and} task level replanning; however, we find that this advantage does not lead to measurable performance gains on the evaluation domain that we consider.

We report success rates and subgoal completion rates for all methods.
Success rates are averaged over ten random seeds per task, where each seed corresponds to a different geometric instantiation of the task (Sec.~\ref{subsec:task-suite}).
Subgoal completion rates are computed over all plans by measuring the number of steps an oracle planner must take to reach the ground-truth goal from the final state $s_H$ of the plan.
To further delineate the planning performance of Text2Motion from \hm{}, we also report the percentages of planning and execution failures. 

% Success rates and subgoal completion \todo{@chris can you elaborate on how this is computed e.g. normalized?} averaged over ten seeds are reported for all domains. Furthermore, for the methods that use dynamics models to perform forwards roll outs (Text2Motion and \hm{}), we report the number of plans reported vs number of times the plans succeed.

% However, solely using plan length as a proxy for optimality could be misleading, as certain \textit{successful} plans may violate constraints imposed over the primitives.
% For instance, we employ a safety constraint over all primitives $\pi$ which considers an action successful ($r_\pi = 1$) if no collisions occur with objects not intended to be acted upon, and unsuccessful ($r_\pi = 0$) otherwise.
% We therefore introduce a \textit{do-no-harm} metric to distinguish successful plans that respect the primitive safety constraints from those that do not.


% In our main experiments, we compare the success rate of our method, (integrated) Text2Motion, with several baselines: 

% \end{enumerate}

% \klin{Previous experiments section structure below. Bullets 3/4 use the `let's improve TAMP via language' perspective. I think we c/should split up our experiments section to have `experiments' and also `ablations' depending on what framing we should use. For example, the phrasing of `Solving unrepresentable TAMP problems` might be better for the `TAMP' perspective.}

% In our experiments, we test the following hypotheses:

% \begin{itemize}
%     \item \textbf{Goal classification} is a more achievable problem than open-ended plan generation for LLMs (Sec.~\ref{subsec:semantic-parsing-exp})
%     \item \textbf{Conditioning open-ended plan generation} on predicted symbolic goal propositions, a variant of `chain of thought' prompting \cite{wei_chain_2022}, promotes plan correctness (Sec.~\ref{subsec:semantic-parsing-exp})
%     \item \textbf{Text2Motion} realized with the \textbf{TAPS} skills motion planner can \textbf{solve unseen TAMP problems} (Sec.~\ref{subsec:language-tamp-exp})
%     \item \textbf{Text2Motion} can \textbf{solve tasks} that are \textbf{challenging to represent} for classical TAMP methods (Sec.~\ref{subsec:unrep-language-tamp-exp})
% \end{itemize}

% \subsection{Semantic parsing of language goals}
% \label{subsec:semantic-parsing-exp}

% \textbf{Planned experiments:}
% We plan to verify \textbf{H1}, \textbf{H2}, and \textbf{H3} in one combined experiment which compares the accuracy (with standard deviation) of Eqs.~\ref{eq:goal-classification}-~\ref{eq:plan-generation} under permutations of prompt size and relevancy.
% Let a prompt of \textit{size} $n$ contain $n$ tasks as chain-of-thought examples for the LLM.
% A \textit{relevant} task represents a chain-of-thought example for which the semantic similarity of its natural language goal $\mathfrak{g}_e$ to the current query $\mathfrak{g}$, as computed in an embedding space, is above a specified threshold: $\mathrm{CosSim}(\mathrm{LLM}(\mathfrak{g}_e), \mathrm{LLM}(\mathfrak{g}) > t_{\mathfrak{g}}$.

% Full symbolic domain specifications will be constructed for the long-horizon planning domains in TAPS.
% An automated procedure will be used to collect prompt and query datasets of PDDL problems.
% The query split will contain strictly longer and more complex problems than the prompt split. 
% The PDDL problem files offer a means to score the correctness of Eqs.~\ref{eq:goal-classification}-~\ref{eq:plan-generation}. 
% Correctness of symbolic goal $\mathcal{G}$ predicted by the LLM (Eq.~\ref{eq:goal-classification}) can be measured by directly comparing it to the ground truth symbolic goal $\mathcal{G}_{\text{GT}}$ in the PDDL problem file. 
% Correctness of a plan produced by the LLM (Eqs.~\ref{eq:cond-plan-generation}-~\ref{eq:plan-generation}) can be computed by checking if the resulting state of applying the plan from the initial state satisfies $\mathcal{G}_{\text{GT}}$.

% \textbf{The ideal results} will achieve the following criterion under permutations of prompt \textit{size} and \textit{relevancy}:
% \begin{enumerate}
%     \item Acc of Eq.~\ref{eq:goal-classification} $>$ Acc of Eq.~\ref{eq:cond-plan-generation} $>$ Acc of Eq.~\ref{eq:plan-generation} 
%     \item Acc of Eq.~\ref{eq:cond-plan-generation} with $\mathcal{G}_{GT}$ $>$ Acc of Eq.~\ref{eq:cond-plan-generation} with $\mathcal{G}$
%     \item Acc. of Eqs.~\ref{eq:cond-plan-generation}-~\ref{eq:plan-generation} increases by querying the LLM for larger sets of task plans; in proportion to $K$
% \end{enumerate}


% \subsection{Language TAMP with Text2Motion}
% \label{subsec:language-tamp-exp}

% \textbf{Planned experiments:}
% The number of tasks in long-horizon planning domains from TAPS (i.e. Hook Reach, Constrained Packing, Rearrangement Push) will be expanded from three to ten.
% New tasks will include variations of objects and colors to support TAMP queries that are combinatorial (e.g., put two red objects or two blue objects on the rack). 

% \textbf{The main results} will verify \textbf{H4} and \textbf{H5} by comparing the average success rates of our method to baselines such as SayCan.
% The performance upper-bound will be established by replacing our LLM task planner with a PDDL-based planner to acquire perfect symbolic plan skeletons.
% As in the previous section, ablations of our method should be run across prompt size and relevancy.
% We should emphasize the importance of each component of our method with a component hold-one-out evaluation; e.g., using Eq.~\ref{eq:plan-generation} instead of Eq.~\ref{eq:cond-plan-generation}, omitting policy sequence optimization with TAPS (effects on ranking plans, not resolving dependencies, etc.).
% It may also be helpful to provide a mini-experiment to demonstrate that the new more uniform learning strategy for the policies and Q-functions are improve success rates compared to TAPS' non-uniform SAC training. 

% Extras: one way to emphasize the generality of the LLM-based planner would be to train independent models for goal classification and plan prediction and evaluate their performance on hold-out validation tasks.


% \subsection{Solving unrepresentable TAMP problems}
% \label{subsec:unrep-language-tamp-exp}

% In this section, we provide some light results on our methods ability to solve tasks that cannot easily be described in PDDL, thereby making such problems incompatible with classical TAMP solvers.
% This wouldn't be an extensive empirical evaluation, and will likely be an good place to record some real-world demonstrations for supplementary material. 
\section{Results}
\label{sec:results}

\begin{figure}
    \centering    
    \includegraphics[width=0.5\textwidth]{figures/imgs/main_plot.pdf}
    \vspace{-15pt}
    \caption{\textbf{Results on the TableEnv manipulation domain} with 10 seeds for each task. {\bf Top:} Our method (Text2Motion) significantly outperforms all baselines on tasks involving partial affordance perception (Task 4, 5, 6). For the tasks without partial affordance perception, the methods that use policy sequence optimization (ours and \hm) both convincingly outperform the methods (\scgs~and \imgs) that do not use policy sequence optimization. We note that \hm~performs well on the tasks without partial affordance perception as it has the advantage of outputting \textit{multiple} goal-reaching candidate task plans and selecting the most geometrically feasible. {\bf Bottom:} Methods without policy sequence optimization tend to have high sub-goal completion rates but very low success rates. This divergence arises because it is possible to make progress on tasks without resolving geometric dependencies in the earlier timesteps; however, failure to account for geometric dependencies results in failure of the overall task. }
    \label{fig:planning-result}
    % \vspace{-10pt}
\end{figure}


\subsection{Planning is required to solve TAMP-like tasks (\textbf{H1})}
\label{subsec:components-tamp}
Our first hypothesis is that performing policy sequence optimization on task plans output by the LLM is essential to task success. To test this hypothesis, we compare Text2Motion and \hm{}, which both perform policy sequence optimization, against \scgs{} and \imgs{}, which are both reactive methods.
% In Fig \ref{fig:planning-result}, we see that, when compared with reactive methods (\scgs{} and \imgs{}) that do not perform policy sequence optimization, the planning methods Text2Motion and \hm{} are able to solve tasks with around 75\% accuracy whereas the reactive methods .

Instructions $i$ provided in the first two planning tasks (\LH{}) allude to skill sequences that, if executed appropriately, would solve the task.
In effect, the LLM plays a lesser role in contributing to plan success, as its probabilities are conditioned to mimic the skill sequences in $i$.
On such tasks, the reactive baselines fail to surpass success rates of 20\%, despite completing between 50\%-80\% of the subgoals (Fig.~\ref{fig:planning-result}).
This result is anticipated as the feasibility of the final skills requires coordination with earlier skills in the sequence, which the reactive baselines fail to consider.
As other tasks in our suite combine aspects of \LH{} with \LG{} and \PAP, it remains difficult for \scgs{} and \imgs{} to find solutions.


% We further highlight that \scgs{} matches the performance of its derivative \imgs{}, suggesting that the techniques designed to contend with symbolic complexity (e.g. task-progress feedback) does fulfil the role of geometric reasoning.
Suprisingly, we see \scgs{} closely matches the performance of its derivative \imgs{}, which has the added benefit of scene descriptions (like Text2Motion).
This result suggests that explicit language feedback does not contribute to success on our tasks when considered in isolation from plan feasibility.
% This result suggests that our task suite does not require explicit language feedback to the LLM and that, instead, scene progress feedback through the value functions as well as executed actions were enough for \scgs{} to match (or surpass, possible due to the stochasticity of the underlying policies) \imgs{}.
In contrast, Text2Motion and \hm{} which employ policy sequence optimization over planned skills can better contend with geometric dependencies prevalent in \LH{} tasks and thereby demonstrate higher success rates. 

\begin{figure}
    \centering    
    \includegraphics[width=0.5\textwidth]{figures/imgs/failure_modes.pdf}
    \vspace{-15pt}
    \caption{\textbf{Failure modes of planning based methods on two types of tasks} 
    % \todo{ format this; better caption + labels}. 
    In this plot, we analyse the various types of failure modes that occur with the integrated planner and the hierarchical planner when evaluated on tasks with partial affordance perception (PAP; see Sec.~\ref{subsec:task-suite} for an explanation) and tasks without partial affordance perception (non-PAP). \textbf{Top}: For the PAP tasks, the hierarchical planner outperforms the integrated planner. We attribute this difference to the hierarchical planner's ability to output multiple task plans while the integrated planner can only output a single plan. \textbf{Bottom}: For the non-PAP tasks, the hierarchical planner is much less likely to output a plan than the integrated approach as the integrated approach is able to use its value functions to prune away geometrically infeasible action sequences.}
    \label{fig:failure}
    % \vspace{-10pt}
\end{figure}


% \subsection{When is integrated planning preferred? (\textbf{H2})}
\subsection{Integrated reasoning is required for PAP tasks
 (\textbf{H2})}
\label{subsec:integrated-planning}
Our second hypothesis is that integrated semantic and geometric reasoning is required to solve the \PAP{} family of tasks (defined in Sec.~\ref{subsec:task-suite}). 
We test this hypothesis by comparing Text2Motion and \hm{}, which represent two fundamentally distinct approaches to combining symbolic and geometric reasoning.
\hm{} uses Q-functions of skills to optimize $k$ skill sequences \textit{after} they are generated by the LLM.
Text2Motion uses primitive Q-functions as a skill feasibility heuristic (Eq.~\ref{eq:q-function-score}) for integrated planning \textit{while} a skill sequence is being constructed.

In the first two tasks (\LH{}, Fig.~\ref{fig:planning-result}), we find that \hm{} achieves slightly higher success rates than Text2Motion, while both methods achieve 100\% success rates in the third task (\LH{} + \LG{}).
While confounding at first, this result indicates a subtle advantage of \hm{} when multiple  feasible plans can be directly inferred from $i$ and $s_1$: it can capitalize on diverse orderings of $k$ generated skill sequences (including the one specified in $i$) and select the one with the highest likelihood of success.
% Concretely, it can capitalize on diverse orderings of $k$ primitive sequences and select the one with the highest likelihood of success, which may not be specified in $i$. 
For example, Task 1 (Fig.~\ref{fig:evaluation_task_suite}) asks the robot to put three boxes onto the rack; \hm{} allows the robot to test multiple different strategies while Text2Motion's integrated approach only outputs one plan.
This advantage is primarily enabled by bias in the Q-functions: Eq.~\ref{eq:taps-objective} may indicate that \graytext{Place(dish, rack)} then \graytext{Place(cup, rack)} is more geometrically complex than \graytext{Place(cup, rack)} then \graytext{Place(dish, rack)}, while they are geometric equivalents.

The plans considered by Text2Motion at planning iteration $h$ share the same sequence of predecessor skills $\pi_{1:h-1}$.
This affords limited diversity for the planner to exploit.
% This diversity is not afforded to Text2Motion as the plans it considers during integrated search at step $h$ share the same sequence of predecessor skills $\pi_{1:h-1}$. 
However, Text2Motion has a significant advantage when solving the \PAP{} family of problems (Fig.~\ref{fig:planning-result}, Tasks 4-6).
Here, geometrically feasible skill sequences are difficult to infer directly from $i$, $s_1$, and the in-context examples provided in the prompt.
As a result, \hm{} incurs an 80\% planning failure rate, while Text2Motion finds plans over 90\% of the time (Fig.~\ref{fig:failure}).
In terms of success, Text2Motion solves 40\%-60\% of the tasks, while \hm{} achieves a 10\% success rate on Task 4 (\LG{} + \PAP{}) and fails to solve any of the latter two tasks (\LH{} + \LG{} + \PAP{}).
Moreover, \hm{} does not meaningfully advance on any subgoals, unlike \scgs{} and \imgs{}, which consider the geometric feasibility of skills at each timestep (albeit, greedily). 
% In contrast, Text2Motion find plans over 90\% of the time and attains 40\% and 60\% success rates, t 10\% of the time on the fourth task (\LG{} + \PO) and failing to solve any of the latter two (\LH{} + \LG{} + \PO{}).


% In contrast, Text2Motion finds plans over 90\% of the time, and achieves 40\% and 60\% success rates on the tasks, while \hm{} succeeds only 10\% of the time on the fourth task (\LG{} + \PO) and fails to solve any of the latter two (\LH{} + \LG{} + \PO{}).
% Moreover, \hm{} does not meaningfully advance on any subgoals.

% Thus, when a task admits a large solution space and a feasible plan cannot be deduced from the instruction (\PO{}), the likelihood of \hm{} inferring a geometrically feasible solution amongst many plausible skill sequences is low. 
% In contrast, our integrated planner uses skill Q-functions to prune the space of possible plans and thereby only considers those which are geometrically feasible at $s_1$ and onwards.
% This result is consistent with \textbf{H2}.

% While the strengths of \hm{} (diversity) and Text2Motion (feasibility) 
% We leave the use of beam search with beam size greater than 1 and other high level search strategies that, for example, combine the strengths of Text2Motion (ability to quickly prune out irrelevant actions) and \hm{}'s (diverse plans) for future work.


\begin{figure}
    \centering        
    \includegraphics[width=0.5\textwidth]{figures/imgs/termination_method.pdf}
    \vspace{-15pt}
    \caption{\textbf{Ablation on termination method: goal proposition prediction vs scoring stop}. We use an integrated search approach and compare the use of two different termination conditions. We present results averaged across all six tasks and ten seeds for each variation (120 experiments in total). We see that using the goal proposition prediction termination method results in a 10\% boost in success rate. }
    % \vspace{-5pt}
    \label{fig:ablation-termination}
\end{figure}

% \subsection{Goal prediction is reliable for plan termination (\textbf{H3})}
\subsection{Plan termination is made reliable via goal prediction (\textbf{H3})}
\label{subsec:plan-termination}
Our final hypothesis is that predicting goals from instructions \textit{a priori} and selecting plans based on their satisfication (Sec.~\ref{sec:goal_pred}) is more reliable than 
scoring plan termination (with a dedicated \texttt{stop} skill) at each timestep.
We test this hypothesis in an ablation experiment (Fig.~\ref{fig:ablation-termination}), comparing our plan termination method to that of SayCan and Inner Monologue's, while keeping all else constant for our integrated planner. 
% We run an ablation experiment to test our goal proposition prediction plan termination method with SayCan and Inner Monologue's \texttt{stop} scoring termination method while keeping all else constant for our integrated planner. 
We run 60 experiments (six tasks and ten seeds each) in total on the TableEnv Manipulation task suite. 
The results in Fig.~\ref{fig:ablation-termination} suggest that, for the tasks we consider, our proposed goal proposition prediction method leads to 10\% higher success rates than the scoring baseline. 

We also note the apparent advantages of both techniques.
First, goal proposition prediction is more efficient than scoring \texttt{stop} as the former requires only one LLM query, whereas the latter needs to be queried at every timestep.
Second, goal proposition prediction offers interpretability over \texttt{stop} scoring, as it is possible to inspect the goal that the planner is aiming towards prior to execution. 
% is potentially more interpretable than \texttt{stop} scoring, as it is possible to inspect the goal the planner is aiming towards. 
Nonetheless, \texttt{stop} scoring does provide benefits in terms of expressiveness, as its predictions are not constrained to any specific output format.
This advantage, however, is not captured in our TableEnv Manipulation tasks, which at most require conjunctive ($\land$) and disjunctive ($\lor$) goals.
% However, the tasks considered in the TableEnv Manipulation domain at most require goals that involve $\land$ or $\lor$ operators.
For instance, ``Stock two boxes onto the rack'' could correspond to (\graytext{on(red box, rack)} $\land$ \graytext{on(blue box, rack)}) $\lor$ (\graytext{on(box A, rack)} $\land$ \graytext{on(box C, rack)}), while mechanically, \texttt{stop} scoring can \textit{handle} more complex instructions.

% \todo{We are awating results for this section.}


% The integrated approach (with beam size larger than 1) requires the LM to reason over task plan sequences. However, from non-quantitative analysis, the LLM is quite sensitive to small changes in the prompt. Hence, we only use beam size 1 to restrict the LLM to only need to reason over the `next' action.

% Another challenge with the integrated approach is that it involves feeding parsed `latent` dynamics states to the language model. This process has a high requirement on the fidelity of the dynamics model, since a slighly flawed dynamics prediction may cause a discrete change in the parsed symbolic state that's fed to the language model. For example, the dynamics model for \texttt{place(a, b)} may lead to a predicted geometry where \texttt{a} is slightly below the surface of \texttt{b} and thus the predicate \texttt{on(a, b)} may not be parsed and fed to the language model.
\section{Limitations and Future Work}
\label{sec:limitations}
The failure analysis conducted over the planning methods in Fig.~\ref{fig:failure} highlights several limitations of our language planning framework.
On Tasks 1-3 (\LH{}, \LG{}), Text2Motion incurs a larger planning and execution failure rate than \hm{} ($\sim$10\%) as a consequence of committing to a single skill sequence in the integrated search algorithm. 
In addition, we observed an undesirable pattern emerge in the planning phase of Text2Motion and the execution phase of \scgs{} and \imgs{}, where \textit{recency bias}~\cite{zhao2021calibrate} in the LLM would induce a cyclic state of repeating feasible skill sequences.

% In Fig.~\ref{fig:failure}, we analyse the several failure modes of Text2Motion: 
% (1) Goal proposition prediction failure (we did not observe this failure mode in our evaluation task suite), 
% (2) No successful plan found,
% (3) Plan found but execution error.

% Aside from the failure modes listed above, we also observed another failure mode that occurred in the planning phase of Text2Motion and execution phase of \scgs{} and \imgs{}: LLM would sometimes enter a \textit{cycle}, repeating sequences of geometrically feasible actions.

As with other methods that use skill libraries, Text2Motion is highly reliant on the fidelity of the low-level skill policies, their value functions, and the ability to accurately predicted future states with learned dynamics models.
This presents non-trivial challenges in scaling Text2Motion to high-dimensional observation spaces, such as images or pointclouds.

% Similar to approaches relying on skill libraries, the overall performance of Text2Motion is highly dependent on the quality of the low-level skill policies, their Q functions and the dynamics models. 
% Additionally, Text2Motion must contend with potentially erroneous latent state representations predicted by a learned dynamics model.

Text2Motion is mainly comprised of learned components, which comes with its associated efficiency benefits.
Nonetheless, runtime complexity was not a core focus of this work, and expensive optimization subroutines~\cite{taps-2022} were frequently called.
LLMs also impose an inference bottleneck as each API query (Sec. \ref{subsec:llm}) requires 2-10 seconds, but we anticipate improvements with advances in LLM inference techniques.

% In this work, we do not focus on runtime optimization and thus run policy sequence optimization from scratch at each planning iteration for Text2Motion and \hm{}.
% We further note that our overall inference time is also bottle-necked by that of the LLM. In this case, each query to the LLM API (see Sec. \ref{subsec:llm}) used in this work can take anywhere between two to ten seconds).
% That said, we expect query times to decrease as LLM inference techniques improve.

% We thus outline several avenues for future work:
% (1) Maintain a more diverse set of task plans through beam search with beam size greater than one. Doing so would allow Text2Motion to handle longer horizon robot manipulation tasks where it helps to maintain diverse plans (along with the proposed iterative integrated planning method). We empirically found that the LLMs we used were not suited to guide search for large beam sizes as the scores generated across skill \textit{sequences} were not calibrated relative to each other. 
% However, as their capabilities improve, LLMs should eventually be able to accurately rank and score skill sequences.
% (2) Decrease inference time by caching policy sequence optimization distributions from earlier planning timesteps.

We outline several avenues for future work based on these observations.
First, we aim to explore algorithmic design choices that unify the strengths of integrated (feasibility) and hierarchical (diversity) planning.
Doing so would enable Text2Motion to solve longer horizon robot manipulation tasks where it helps to maintain diverse plans.
Such strategies may require LLMs to produce calibrated scores of entire skill sequences relative to each other, which we empirically found challenging and warrants further investigation.
Second, there remains an opportunity to increase the plan-time efficiency of our method, for instance, by warm starting policy sequence optimization with distributions cached in earlier planning iterations.
Lastly, we hope to explore the use of multi-modal foundation models~\cite{driess2023palm, openai2023gpt4} that are visually grounded, and as a result, may support scaling our planning system to higher dimensional observation spaces.
Progress on each of these fronts would constitute steps in the direction of reliable and real-time language planning capabilities.


% \section{Discussion}
% \label{sec:discussion}

% \klin{need to format}
% \begin{enumerate}
    % \item One class of failure modes that we observed involved the LLM deciding to repeating certain actions it had previous seen. For example, one of the failure modes in figure \ref{fig:failure} (LH LG PO) requires the robot to use the hook to pick and place two boxes onto the rack, then use the hook to pull an object to be inside the workspace and then pick and place that object onto the rack. However, if the task sequence planed so far has a small deviation from a more `optimal' plan, the LLM tends to place a higher score on actions it had previously executed. ['pick(red-box)', 'place(red-box, rack)', 'pick(cyan-box)', 'place(cyan-box, rack)', 'pick(hook)', 'place(hook, table)', 'pick(hook)', 'pull(yellow-box, hook)', 'place(hook, table)', 'pick(red-box)'].
    % \item LLM goal prediction can sometimes be wrong - if being used as chain of thought, especially for code-davinci, need to specify ``\textit{Predicted} goal predicate set'' rather than ``Goal predicate set''. Text davinci models don't get affected by wording as much.
% \end{enumerate}

\section{Conclusion}

In this paper, we investigated the design space and the recognition method of voice-accompanying hand-to-face (VAHF) gestures to enhance voice interaction with  parallel gesture channels. To design VAHF gestures, we first conducted an elicitation study, resulting in a total proposal of 15 gestures, followed by a hierarchical analysis process to output the most salient 8 gestures with the least ambiguity and physical confusion. Then we proposed a novel cross-device sensing method fusing different sensor channels to recognize para-linguistic hand-to-face gestures, achieving a high recognition accuracy of 97.3\% for 3+1\revision{(empty)} gestures and 91.5\% for 8+1\revision{(empty)} gestures recognition on our cross-device VAHF dataset. The uniqueness of our work is that we explored a broadened and scalable VAHF-gesture-based interaction space, which remains under-researched, to facilitate voice interaction in a more diverse manner (e.g., defining a shortcut or parsing parameters). Compared with prior work \cite{10.1145/3411764.3445687,Yan-UIST-2019} where a specific gesture (e.g., bringing the phone to the mouth\cite{10.1145/3411764.3445687}) was designed and recognized for 1-bit modality control (e.g., activating the voice assistant), our multi-device sensing framework is not only capable for recognizing up to 8 VAHF gestures simultaneously \revision{from the hand-off "empty" gesture}, but also benefits from the scalability (e.g., adding a device or adding a gesture is easy under our framework). As mobile devices and scenarios are becoming prevalent these years, voice input has become an essential modality of pervasive interaction. We envision our work would further enhance the efficiency and capability of current voice interaction and serve an important role in the future voice interaction of various scenarios like AR and IoT.





% not in double-blind version
% \section*{Acknowledgments}

% Toyota Research Institute (“TRI”) provided funds to assist
% the authors with their research but this article solely reflects
% the opinions and conclusions of its authors and not TRI or any
% other Toyota entity.

%Add funding information.

%% Use plainnat to work nicely with natbib. 
\bibliographystyle{plainnat}
\bibliography{references}

\clearpage
\onecolumn
% \appendix
% \renewcommand{\thechapter}{\alph{chapter}}
% \renewcommand{\thesection}{\Roman{section}}
% \renewcommand{\thesubsection}{\fnsymbol{subsection}}


% 
\section{General Identification Result}\label{app:identification}

\begin{theorem}\label{thm:identify_gen}
    Consider known nonlinear and differentiable functions $h_i:\R^{1+L}\to [0,1]$.
    Suppose that for $i\in\{1,\dots, B\}$ observations are generated according to
    \[y_i = h_i(\beta^\top z_i, \theta)\]
    for known variables $z_i$ and unknown parameters $\beta\in\mathcal B$ and $\theta\in\mathcal T$, where $\mathcal B$ and $\mathcal T$ are simply connected and compact subsets of $\mathbb{R}^D$ and $\mathbb{R}^L$ respectively.

    % Suppose that $h_i$ is proper on the restricted domain
    Denote by $h_i'$ the partial derivative of $h_i$ with respect to the first argument and $\nabla_\theta h_i$ the gradient with respect to the latter $L$ arguments.
    Define the matrices 
    \begin{align}
        \label{eq:identification-mx-appendix}
        Z = \begin{bmatrix}
         z_1^\top \\ \vdots \\   z_N^\top \end{bmatrix},
        \quad
        B_{\beta,\theta} = \begin{bmatrix}
        h_1'(\beta^\top z_1, \theta) && \\ &\ddots &\\  &&h_N'(\beta^\top z_N, \theta) \end{bmatrix},
        \quad
        \Theta_{\beta,\theta} = \begin{bmatrix}
            \nabla_\theta h_1(\beta^\top z_1, \theta)\\
            \vdots \\ 
            \nabla_\theta h_N( \beta^\top z_N, \theta)
        \end{bmatrix}
    \end{align}
    Then the parameters $\beta$ and $\theta$ can be uniquely identified from a dataset of $\{z_i, y_i\}_{i=1}^N$ if the following conditions hold for all $\beta\in\mathcal B$ and $\theta\in\mathcal T$:
    \begin{itemize}
        % \item TODO: condition on $g,\mathcal B,\mathcal G$ for proper $F_N$; maybe condition for simply connected M2
        \item Rank condition: $B_{\beta,\theta}Z$ and $\Theta_{\beta,\theta}$ are full rank, i.e. $\rank(B_{\beta,\theta}Z) = D$ and $\rank(\Theta_{\beta,\theta}) = L$ 
        \item Independence condition: the column spaces of $B_{\beta,\theta}X$ and $\Theta_{\beta,\theta}$ are perpendicular, i.e. $\colspace(B_{\beta,\theta}X) \perp \colspace(\Theta_{\beta,\theta})$.
    \end{itemize}
\end{theorem}

\begin{lemma}\label{lem:localinv}
Define the observation map $F_N:\mathcal B\times \mathcal T \to \mathbb R^N$ as
\[F_N(\beta, \theta) = \begin{bmatrix}
h_1( \beta^\top z_1, \theta)\\
\vdots \\ 
h_N( \beta^\top z_N, \theta)
\end{bmatrix}\]
The Jacobian of $F_N$ is invertible at $\beta,\theta$ if and only if the rank and independence conditions hold.
\end{lemma}
\begin{proof}
Denote by $J$ the Jacobian of $F_N$. Then
\[J = \begin{bmatrix}\nabla h_1(\beta^\top z_1,\theta) \\ \vdots \\\nabla g_N(\beta^\top z_N,\theta)  \end{bmatrix}= \begin{bmatrix}B_{\beta,\theta}X & \Theta_{\beta,\theta}\end{bmatrix}\:.\]
% By the rank-nullity theorem, 
The Jacobian $J$ is invertible if and only if the nullspace of $J$ contains only zero.

We first argue that the rank and independence conditions are sufficient. Suppose that $Jv=0$ for some $v$.
Letting $v=[v_1,v_2]$, this is equivalent to $BXv_1 + \Gamma v_2=0$.
Notice these terms are elements of $\colspace(BX)$ and $\colspace(\Gamma)$ respectively. 
By the independence condition, it must be that $BXv_1=0$ and $\Gamma v_2=0$.
By the rank condition and the rank-nullity theorem, it must be that $v_1=0$ and $v_2=0$.
Thus the rank and independence conditions imply that $J$ is invertible.

We now show that the rank and independence conditions are necessary.
If the independence condition does not hold, there is some nonzero $u$ 
% contained in both $\colspace(BX)$ and $\colspace(\Gamma)$, which implies that we can write 
such that $u=BXv_1=\Gamma v_2$.
Then $v=[v_1,-v_2]\neq 0$ is in the nullspace of $J$ so $J$ is not invertible.
If either $DX$ or $\Gamma$ is not full rank, then a nonzero element of their nullspace can be used to construct a nonzero element of the nullspace of then $J$.
This concludes the proof.
\end{proof}

\begin{proof}[Proof of Proposition~\ref{thm:identify_gen}]
Define $\mathcal F\subseteq \mathbb R^N$ as the image of the map $F_N$
defined in Lemma~\ref{lem:localinv}.
With some abuse of notation, we will now consider the function $F_N:\mathcal B \times \mathcal G\to \mathcal F$.
Identifiability of the parameters $\beta$ and $\gamma$ is equivalent to global invertibility of the function $F_N$.
We will use a Theorem due to Hadamard~\cite[Theorem 6.2.8]{krantz2002implicit} 
% [TODO perhaps type out].
% This theorem applies because $\mathcal B\times \mathcal G$ is a smooth and connected manifold by assumption. Since $F$ is continuous, the image is also smooth and connected and has the same dimension.
which states that
$F_N$ is globally invertible if it is proper, if the Jacobian never vanishes, and if $\mathcal F$ is simply connected.

% F proper if X is compact and Y is Hausdorf
$F_N$ is proper because each $h_i$ is proper and $Z$ is full rank.
% , so sequences ${\beta_i,\gamma_i}$ limiting to the boundary of $\mathcal B\times\mathcal G$ also limit to the boundary of $\mathcal F$.
Since $\mathcal F$ is the image of a simply connected space under a continuous mapping, it is also simply connected.
Finally, by Lemma~\ref{lem:localinv}, the Jacobian of $F_N$ is everywhere invertible under the rank and independence conditions.\end{proof}



% \section{Proof for Lemma~\ref{lemma1}}\label{app:pflemma1}
% \begin{proof}
%     Let $u < v$ be two actions induced in equilibrium. Since the algorithm anticipates what actions will be induced with its recommendations, there must exists $m_1, m_2 \in [\delta,1-\delta]$ such that 
%     \begin{align*}
%         U^A(u,m_1) > U_A(v,m_1) \\
%         U^A(v,m_2) > U_A(u,m_2)
%     \end{align*}
%     By continuity of $U^A(u,\cdot)-U^A(v,\cdot)$, there exists an $\overline{m}$ between $m_1$ and $m_2$ such that $ U^A(u,\overline{m}) = U_A(v,\overline{m})$. Since $U^A_{11}(\cdot)<0$, $U^A$ has a unique maximum in $y$ for any given $m$. Therefore, 
%     \begin{align}\label{eq:lemma1}
%         u < y^A(\overline{m}) < v 
%     \end{align}
%     Also, $U^A_{12}(\cdot)>0$, so the algorithm prefers a higher action when the true state of the world is higher. This implies 
%     \begin{align*}
%         \text{$u$ is not induced by any $m > \overline{m}$} \\
%         \text{$v$ is not induced by any $m < \overline{m}$}
%     \end{align*}
%     The above statements and our assumption that $U^J_{12}(\cdot)>0$ imply that 
%     \begin{align}\label{eq:lemma2}
%         u \leq y^J(\overline{m},b) \leq v 
%     \end{align}
%     However, if $y^A(m) \neq y^J(m,b)$ for all $m\in [\delta,1-\delta]$, there exists an $\epsilon>0$ such that $|y^A(m)-y^J(m,b)|\geq \epsilon$ for all $m\in [\delta,1-\delta]$. It follows from \eqref{eq:lemma1} and \eqref{eq:lemma2} that $|u-v|\geq \epsilon$. Since the set of actions is bounded and $U^J_{12}>0$, the set of actions must be finite in equilibrium.
% \end{proof}

% \section{Proof for Proposition \ref{prop:partitioneq}} \label{app:pfeq}


% \appendix
% \renewcommand{\thesection}{\Alph{section}.\arabic{section}}
% \setcounter{section}{0}

\begin{appendices}
    % 
\section{General Identification Result}\label{app:identification}

\begin{theorem}\label{thm:identify_gen}
    Consider known nonlinear and differentiable functions $h_i:\R^{1+L}\to [0,1]$.
    Suppose that for $i\in\{1,\dots, B\}$ observations are generated according to
    \[y_i = h_i(\beta^\top z_i, \theta)\]
    for known variables $z_i$ and unknown parameters $\beta\in\mathcal B$ and $\theta\in\mathcal T$, where $\mathcal B$ and $\mathcal T$ are simply connected and compact subsets of $\mathbb{R}^D$ and $\mathbb{R}^L$ respectively.

    % Suppose that $h_i$ is proper on the restricted domain
    Denote by $h_i'$ the partial derivative of $h_i$ with respect to the first argument and $\nabla_\theta h_i$ the gradient with respect to the latter $L$ arguments.
    Define the matrices 
    \begin{align}
        \label{eq:identification-mx-appendix}
        Z = \begin{bmatrix}
         z_1^\top \\ \vdots \\   z_N^\top \end{bmatrix},
        \quad
        B_{\beta,\theta} = \begin{bmatrix}
        h_1'(\beta^\top z_1, \theta) && \\ &\ddots &\\  &&h_N'(\beta^\top z_N, \theta) \end{bmatrix},
        \quad
        \Theta_{\beta,\theta} = \begin{bmatrix}
            \nabla_\theta h_1(\beta^\top z_1, \theta)\\
            \vdots \\ 
            \nabla_\theta h_N( \beta^\top z_N, \theta)
        \end{bmatrix}
    \end{align}
    Then the parameters $\beta$ and $\theta$ can be uniquely identified from a dataset of $\{z_i, y_i\}_{i=1}^N$ if the following conditions hold for all $\beta\in\mathcal B$ and $\theta\in\mathcal T$:
    \begin{itemize}
        % \item TODO: condition on $g,\mathcal B,\mathcal G$ for proper $F_N$; maybe condition for simply connected M2
        \item Rank condition: $B_{\beta,\theta}Z$ and $\Theta_{\beta,\theta}$ are full rank, i.e. $\rank(B_{\beta,\theta}Z) = D$ and $\rank(\Theta_{\beta,\theta}) = L$ 
        \item Independence condition: the column spaces of $B_{\beta,\theta}X$ and $\Theta_{\beta,\theta}$ are perpendicular, i.e. $\colspace(B_{\beta,\theta}X) \perp \colspace(\Theta_{\beta,\theta})$.
    \end{itemize}
\end{theorem}

\begin{lemma}\label{lem:localinv}
Define the observation map $F_N:\mathcal B\times \mathcal T \to \mathbb R^N$ as
\[F_N(\beta, \theta) = \begin{bmatrix}
h_1( \beta^\top z_1, \theta)\\
\vdots \\ 
h_N( \beta^\top z_N, \theta)
\end{bmatrix}\]
The Jacobian of $F_N$ is invertible at $\beta,\theta$ if and only if the rank and independence conditions hold.
\end{lemma}
\begin{proof}
Denote by $J$ the Jacobian of $F_N$. Then
\[J = \begin{bmatrix}\nabla h_1(\beta^\top z_1,\theta) \\ \vdots \\\nabla g_N(\beta^\top z_N,\theta)  \end{bmatrix}= \begin{bmatrix}B_{\beta,\theta}X & \Theta_{\beta,\theta}\end{bmatrix}\:.\]
% By the rank-nullity theorem, 
The Jacobian $J$ is invertible if and only if the nullspace of $J$ contains only zero.

We first argue that the rank and independence conditions are sufficient. Suppose that $Jv=0$ for some $v$.
Letting $v=[v_1,v_2]$, this is equivalent to $BXv_1 + \Gamma v_2=0$.
Notice these terms are elements of $\colspace(BX)$ and $\colspace(\Gamma)$ respectively. 
By the independence condition, it must be that $BXv_1=0$ and $\Gamma v_2=0$.
By the rank condition and the rank-nullity theorem, it must be that $v_1=0$ and $v_2=0$.
Thus the rank and independence conditions imply that $J$ is invertible.

We now show that the rank and independence conditions are necessary.
If the independence condition does not hold, there is some nonzero $u$ 
% contained in both $\colspace(BX)$ and $\colspace(\Gamma)$, which implies that we can write 
such that $u=BXv_1=\Gamma v_2$.
Then $v=[v_1,-v_2]\neq 0$ is in the nullspace of $J$ so $J$ is not invertible.
If either $DX$ or $\Gamma$ is not full rank, then a nonzero element of their nullspace can be used to construct a nonzero element of the nullspace of then $J$.
This concludes the proof.
\end{proof}

\begin{proof}[Proof of Proposition~\ref{thm:identify_gen}]
Define $\mathcal F\subseteq \mathbb R^N$ as the image of the map $F_N$
defined in Lemma~\ref{lem:localinv}.
With some abuse of notation, we will now consider the function $F_N:\mathcal B \times \mathcal G\to \mathcal F$.
Identifiability of the parameters $\beta$ and $\gamma$ is equivalent to global invertibility of the function $F_N$.
We will use a Theorem due to Hadamard~\cite[Theorem 6.2.8]{krantz2002implicit} 
% [TODO perhaps type out].
% This theorem applies because $\mathcal B\times \mathcal G$ is a smooth and connected manifold by assumption. Since $F$ is continuous, the image is also smooth and connected and has the same dimension.
which states that
$F_N$ is globally invertible if it is proper, if the Jacobian never vanishes, and if $\mathcal F$ is simply connected.

% F proper if X is compact and Y is Hausdorf
$F_N$ is proper because each $h_i$ is proper and $Z$ is full rank.
% , so sequences ${\beta_i,\gamma_i}$ limiting to the boundary of $\mathcal B\times\mathcal G$ also limit to the boundary of $\mathcal F$.
Since $\mathcal F$ is the image of a simply connected space under a continuous mapping, it is also simply connected.
Finally, by Lemma~\ref{lem:localinv}, the Jacobian of $F_N$ is everywhere invertible under the rank and independence conditions.\end{proof}



% \section{Proof for Lemma~\ref{lemma1}}\label{app:pflemma1}
% \begin{proof}
%     Let $u < v$ be two actions induced in equilibrium. Since the algorithm anticipates what actions will be induced with its recommendations, there must exists $m_1, m_2 \in [\delta,1-\delta]$ such that 
%     \begin{align*}
%         U^A(u,m_1) > U_A(v,m_1) \\
%         U^A(v,m_2) > U_A(u,m_2)
%     \end{align*}
%     By continuity of $U^A(u,\cdot)-U^A(v,\cdot)$, there exists an $\overline{m}$ between $m_1$ and $m_2$ such that $ U^A(u,\overline{m}) = U_A(v,\overline{m})$. Since $U^A_{11}(\cdot)<0$, $U^A$ has a unique maximum in $y$ for any given $m$. Therefore, 
%     \begin{align}\label{eq:lemma1}
%         u < y^A(\overline{m}) < v 
%     \end{align}
%     Also, $U^A_{12}(\cdot)>0$, so the algorithm prefers a higher action when the true state of the world is higher. This implies 
%     \begin{align*}
%         \text{$u$ is not induced by any $m > \overline{m}$} \\
%         \text{$v$ is not induced by any $m < \overline{m}$}
%     \end{align*}
%     The above statements and our assumption that $U^J_{12}(\cdot)>0$ imply that 
%     \begin{align}\label{eq:lemma2}
%         u \leq y^J(\overline{m},b) \leq v 
%     \end{align}
%     However, if $y^A(m) \neq y^J(m,b)$ for all $m\in [\delta,1-\delta]$, there exists an $\epsilon>0$ such that $|y^A(m)-y^J(m,b)|\geq \epsilon$ for all $m\in [\delta,1-\delta]$. It follows from \eqref{eq:lemma1} and \eqref{eq:lemma2} that $|u-v|\geq \epsilon$. Since the set of actions is bounded and $U^J_{12}>0$, the set of actions must be finite in equilibrium.
% \end{proof}

% \section{Proof for Proposition \ref{prop:partitioneq}} \label{app:pfeq}

    \section{Appendix Overview}
\label{sec:supp-overview}

The appendix offers additional details with respect to the implementation of Text2Motion and language planning baselines (Appx.~\ref{sec:implementation-details}), the experiments conducted (Appx.~\ref{sec:experiment-details}), and the \textbf{real-world demonstrations} (Appx.~\ref{sec:demos}).
Qualitative results are made available at \link{https://sites.google.com/stanford.edu/text2motion}{sites.google.com/stanford.edu/text2motion}.
    \section{Implementation Details}
\label{sec:implementation-details}

Text2Motion performs an integrated task- and policy-level search with three primary components: learned robot skills, large language models (LLMs), and policy sequence optimization (PSO). 
The baseline \hm{} also uses all of the above in an alternative planning strategy, while the baselines \scgs{} and \imgs{} use all but PSO. 
We make the details of each component explicit in the following subsections and provide the hyper-parameters of each planner.

\subsection{Learning robot skills and dynamics}
\label{sec:learning-models}

We learn four manipulation skill policies to solve tasks in simulation and in the real-world.
Each skill policy $\pi(a \vert s)$ is learned over a single-step action primitive $\rho(a)$ that is parameterized in the coordinate frame of a target object (e.g. \graytext{Pick(box)}). 
Thus, the policy is trained to output action parameters $a\sim\pi$ that maximizes the instantiated primitive's $\rho(a)$ probability of success in a contextual-bandit formulation with binary rewards $r(s, a, s') \in \{0, 1\}$.
Only a single policy per primitive is trained, and thereby the policy must learn to engage the primitive over objects with differing geometries (e.g. $\pi_{\text{Pick}}$ is used for both \graytext{Pick(box)} and \graytext{Pick(hook)}).
The observation space for each policy is defined as the concatenation of geometric state features (e.g. pose, size) of all objects in the scene, where the first $n$ object states correspond to the $n$ primitive arguments and the rest are randomized.
For example, the observation for the action \graytext{Pick(hook)} would have be a vector of all objects' geometric state with the first component of the observation corresponding to the \graytext{hook}.  

\textbf{Parameterized manipulation primitives:}
We describe the action parameters and reward function of each parameterized manipulation primitive below.
Collisions with non-argument objects constitutes an execution failure for all primitives, and hence the policy receives a reward of 0, for instance, if the robot collided with \graytext{box} during the execution of \graytext{Pick(hook)}.
\begin{itemize}
    \item \graytext{Pick(obj)}: $a\sim\pi_{\text{Pick}}(a \vert s)$ denotes the grasp pose of \graytext{obj} \textit{w.r.t} the coordinate frame of \graytext{obj}. A reward of 1 is received if the robot successfully grasps \graytext{obj}.
    \item \graytext{Place(obj, rec)}: $a\sim\pi_{\text{Place}}(a \vert s)$ denotes the placement pose of \graytext{obj} \textit{w.r.t} the coordinate frame of \graytext{rec}. A reward of 1 is received if \graytext{obj} is stably placed on \graytext{rec}.
    \item \graytext{Pull(obj, tool)}: $a\sim\pi_{\text{Pull}}(a \vert s)$ denotes the initial position, direction, and distance of a pull on \graytext{obj} with \graytext{tool} \textit{w.r.t} the coordinate frame of \graytext{obj}. A reward of 1 is received if \graytext{obj} moves toward the robot by a minimum of $d_{\text{Pull}}=0.05m$. 
    \item \graytext{Push(obj, tool, rec)}: $a\sim\pi_{\text{Push}}(a \vert s)$ denotes the initial position, direction, and distance of a push on \graytext{obj} with \graytext{tool} \textit{w.r.t} the coordinate frame of \graytext{obj}. A reward of 1 is received if \graytext{obj} moves away from the robot by a minimum of $d_{\text{Push}}=0.05m$ and if \graytext{obj} ends up underneath \graytext{rec}.
\end{itemize}

\textbf{Dataset generation:} 
All planning methods considered in this work rely on having accurate Q-functions $Q^\pi(s, a)$ to estimate the feasibility of skills proposed by the LLM. 
This places a higher fidelity requirement on the Q-functions than needed to learn a reliable policy, as the Q-functions must characterize both skill success (feasibility) and failure (infeasibility) at a given state.
Because the primitives $\rho(s)$ reduce the horizon of skills $\pi(a\vert s)$ to a single time-step, and the reward functions are $r(s, a, s')=\{0, 1\}$, the Q-functions can be interpreted as binary classifiers of state-action pairs.
Thus, we take a staged approach to learning the Q-functions $Q^\pi$, followed by the policies $\pi$, and lastly the dynamics models $T^\pi$.

Scenes in our simulated environment are instantiated from a symbolic specification of objects and spatial relationships, which together form a symbolic state $s$
% $\mathfrak{s}$.
The goal is to learn a \textit{complete} Q-function that sufficiently covers the state-action space of each primitive.
We generate a dataset that meets this requirement in four steps: a) enumerate all valid symbolic states $\mathfrak{s}$; b) sample geometric scene instances $s$ per symbolic state; c) uniformly sample actions over the action space $a\sim\mathcal{U}^{[0,1]^d}$; (d) simulate the states and actions to acquire next states $s'$ and compute rewards $r(s, a, s')$.
We slightly modify this sampling strategy to maintain a minimum success-failure ratio of 40\%, as uniform sampling for more challenging skills like \graytext{Pull} and \graytext{Push} seldom emits a success ($\sim$3\%).
We collect 1M $(s, a, s', r)$ tuples per skill in a process that takes approximately twelve hours.
Of the 1M samples, 800K of them are used for training ($\mathcal{D}_t$) while the remaining 200K are used for validation ($\mathcal{D}_v$).
We use the same datasets to learn the skill policies and dynamics models.
 
\textbf{Model training:}
We train an ensemble of Q-functions with mini-batch gradient descent and logistics regression loss.
Once the Q-functions have converged, we distill their returns into stochastic policies $\pi$ through the maximum-entropy update~\cite{pmlr-v80-haarnoja18b}:
\begin{equation*}
    \pi^* \leftarrow \arg \max_{\pi} \, \operatorname{E}_{(s,a)\sim \mathcal{D}_t}\left[\min(Q_{1:b}^\pi(s, a)) - \alpha \log\pi(a|s) \right],
\end{equation*}
noting that maintaining some degree of policy stochasticity is beneficial for policy sequence optimization (Sec.~\ref{subsec:taps-geometric-planning}).
Instead of evaluating the policies on $\mathcal{D}_v$ which, contains states for which no feasible action exists, the policies are synchronously evaluated in an environment that exhibits only feasible states. 
This simplifies model selection and standardizes skill capabilities across primitives. 
We train a deterministic dynamics model per primitive using the forward prediction loss:
\begin{equation*}
    L_{dynamics}\left(T^\pi; \mathcal{D}_t \right) = \operatorname{E}_{(s,a,s')\sim \mathcal{D}_t}||T^\pi(s, a) - s'||_2^2
\end{equation*}

All Q-functions achieve precision and recall rates of over 95\%. 
The average success rates of the converged skill policies over 100 evaluation episodes are: $\pi_{\text{Pick}}$ with 99\%, $\pi_{\text{Place}}$ with 90\%, $\pi_{\text{Pull}}$ with 86\%, $\pi_{\text{Push}}$ with 97\%.
The dynamics models converge to within millimeter accuracy on the validation split.


\textbf{Hyperparameters:} The Q-functions, skill policies, and dynamics models are MLPs with hidden dimensions of size [256, 256] and ReLU activations.
We train an ensemble of $b=8$ Q-functions with a batch size of 128 and a learning rate of 1e-4 with a cosine annealing decay~\cite{loshchilov2017sgdr}.
The Q-functions for pick, pull, and push converged on $\mathcal{D}_v$ in 3M iterations, while the Q-function for place required 5M iterations.
We hypothesize that this is because classifying successful placements demands carefully attending to the poses and shapes of all objects in the scene so as to avoid collisions.
The skill policies are trained for 250K iterations with a batch size of 128 and a learning rate of 1e-4, leaving all other parameters the same as \cite{pmlr-v80-haarnoja18b}.
The dynamics models are trained for 750K iterations with a batch size of 512 and a learning rate of 5e-4; only on successful transitions to avoid the noise associated with collisions and truncated episodes.
The parallelized training of all models takes approximately 12 hours on an Nvidia Quadro P5000 GPU and 2 CPUs per job.


\subsection{Out-of-distribution detection}
\label{subsec:ood-calibration}
The datasets described in Sec.~\ref{sec:learning-models} contain both successes and failures for symbolically valid skills like \graytext{Pick(box)}.
However, in interfacing robot skills with LLM task planning, it is often the case that the LLM will propose symbolically invalid actions, such as Pull(box, rack), that neither the skill policies, Q-functions, or dynamics models have observed in training.
We found that a percentage of such out-of-distribution (OOD) queries would result in erroneously high Q-values, causing the skill to be selected. 
Attempting to execute such a skill leads to control exceptions or other undesirable events.

Whilst there are many existing techniques for OOD detection of deep neural networks, we opt to detect OOD queries on the learned Q-functions via deep ensembles due to their ease of calibration~\cite{lakshminarayanan2017simple}.
A state-action pair is classified as OOD if the empirical variance of the predicted Q-values is above a determined threshold:
% \begin{equation*}
%     F_{\text{OOD}}(Q_{1:b}^\pi; s, a) = \mathbbm{1}\left(\operatorname{V}[Q^\pi_{1:b}(s,a)] \geq \epsilon^\pi_{\text{OOD}} \right),
% \end{equation*}
\begin{equation*}
    F_{\text{OOD}}(Q_{1:b}^\pi; s, a) = 1\left(V[Q^\pi_{1:b}(s,a)] \geq \epsilon^\pi_{\text{OOD}} \right),
\end{equation*}
where each threshold $\epsilon^\pi_{\text{OOD}}$ is unique to skill $\pi$.

To determine the threshold value, we generate an a calibration dataset of 100K symbolically invalid states and actions for each primitive.
The process takes less than an hour on a single CPU as the actions are infeasible and need not be simulated in the environment (i.e. rewards are known to be 0).
We compute the empirical variance the Q-ensembles across both the in-distribution and out-of-distribution datasets an bin the variances by predicted Q-value to produce a histogram.
We observe that the histogram of variances produced from OOD queries was uniform across all predicted Q-values and was an order of magnitude large than the ensemble variances computed over in-distribution data.
This simplified the selection of OOD thresholds, which we found to be: $\epsilon^{\text{Pick}}_{\text{OOD}} = 0.10$, $\epsilon^{\text{Place}}_{\text{OOD}} = 0.12$, $\epsilon^{\text{Pull}}_{\text{OOD}} = 0.10$, and $\epsilon^{\text{Push}}_{\text{OOD}} = 0.06$.


\subsection{Task planning with LLMs}
\label{subsec:llm-task-planning}
Text2Motion and the reactive planning baselines \scgs{} and \imgs{} uses \texttt{code-davinci-002} (Codex model \cite{chen2021evaluating}) generate and score skills, while \hm{} queries \texttt{text-davinci-003} (variant of InstructGPT \cite{ouyang2022training}) to directly output full skill sequences.
Surprisingly, a temperature of 0 was empirically found to be sufficient for our tasks. 

To maintain consistency in the evaluation of various planners, we allow Text2Motion, \scgs{}, and \imgs{} to generate $k=5$ skills $\pi^k_h$ at each timestep $h$.
Thus, every search iteration of Text2motion considers five possible extensions to the current running sequence of skills $\pi_{1:h-1}$.
Similarly, \hm{} generates $k=5$ skill sequences.

As described in the methods, skills are selected via a combined usefulness and geometric feasibility score:
\begin{align*}
    S_{\text{skill}}(\pi_h) &= S_{\text{llm}}(\pi_h) \cdot S_{\text{geo}}(\pi_h) \\
    &= p(\pi_h \mid i, s_1, \pi_{1:h-1}) \cdot p(r_h \mid i, s_1, \pi_{1:h}),
\end{align*} 
where Text2Motion uses policy sequence optimization to compute $S_{\text{geo}}$ while \scgs{} and \imgs{} use the current value function estimate $V^{\pi_h}(s_h) = \operatorname{E}_{a_h\sim\pi_h}[Q^{\pi_h}(s_h, a_h)]$.
We find that in both cases, taking $S_{\text{llm}}(\pi_h)$ to be the SoftMax log-probability score produces a \textit{winner-takes-all} effect, causing the planner to omit highly feasible skills simply because their associated log-probability was marginally lower than the LLM-likelihood of another skill.
Thus, we dampen the SoftMax operation with a $\beta$-coefficient to balance the ranking of skills based on both feasibility and usefulness. 
We found $\beta=0.3$ to work well in our setup.

\subsection{Geometric planning with PSO}
\label{subsec:taps-geometric-planning}
Text2Motion relies on a technique to compute the geometric feasibility score of a skill $S_{\text{geo}}(\pi_h) = p(r_h \mid i, s_1, \pi_{1:h})$ in the context of its predecessors.
As detailed in Sec.~\ref{subsec:llm-task-planning}, this probability can be computed without considering geometric action dependencies---particularly if actions have already been executed, as is the case with \scgs{} and \imgs{}.
However, Text2Motion computes $S_{\text{geo}}(\pi_h)$ after performing policy sequence optimization; the process by which long-horizon geometric dependencies are coordinated across skill sequences $\pi_{1:h}$.
For example, this would ensure that when evaluating the selection of \graytext{Pull(box, hook)} during integrated search, the algorithm does so in the context of a suitable \graytext{Pick(hook)} action (e.g. an upper-handle grasp on \graytext{hook}).


Text2Motion is agnostic the method that fulfils this role. 
However, in our experiments we leverage Sequencing Task-Agnostic Policies (STAP)~\cite{taps-2022}.
Specifically, we consider the PolicyCEM variant of STAP, where sampling-based optimization of the skill sequence's $\pi_{1:h}$ success probability is warm started with actions sampled from the policies $a_{1:h}\sim \pi_{1:h}$. 
We perform ten iterations of the Cross-Entropy Method (CEM)~\cite{rubinstein1999-cem}, sampling 10K trajectories at each iteration and selecting 10 elites to update the mean of the sampling distribution for the following iteration. 
The standard deviation of the sampling distribution is held constant at 0.3 for all iterations.
    \section{Experiment Details}
\label{sec:experiment-details}

\begin{table*}[t]
    \centering
    \caption{\textbf{TableEnv manipulation task suite}. We use the following shorthands as defined in the paper: lh: long-horizon, lg: lifted goals, pap: partial affordance perception.}
    % \adjustbox{max width=\linewidth}{
    \begin{tabular}{@{}l|c|l@{}}
    \toprule
    \textbf{Task ID} & \textbf{Properties} & \textbf{Instruction}\\
    \midrule
    \textbf{Task 1} & LH & How would you pick and place all of the boxes onto the rack?” \\
    \textbf{Task 2} & LH + LG & How would you pick and place the yellow box and blue box onto the table, then use the hook to push the \\ & & cyan box under the rack?”

 \\
    \textbf{Task 3} & LH + PAP & How would you move three of the boxes to the rack?”
 \\
    \textbf{Task 4} & LG + PAP & How would you put one box on the rack?”

 \\
    \textbf{Task 5} & LH + LG + PAP & How would you get two boxes onto the rack?”

 \\
    \textbf{Task 6} & LH + LG + PAP & How would you move two primary colored boxes to the rack?”

 \\
    % \textbf{Real Task 1} & LH + LG + PO & Description \\
    % \textbf{Real Task 2} & LH + LG + PO & Description \\
    \bottomrule
    \end{tabular}
    \label{table:text2motion-domains}
\end{table*}

We refer to Table.~\ref{table:text2motion-domains} for an overview of the tasks in the TableEnv Manipulation suite.
Task 6 is demonstrated on a real-robot (details Sec.~\ref{sec:demos}).

\subsection{Scene descriptions as symbolic states}
\label{sec:scene-descr-symbolic}
To provide scene context to Text2Motion and the baselines, we take a heuristic approach to converting a geometric state $s$ into a basic symbolic state $\mathfrak{s}$.
Symbolic states are comprised of three spatial relations: \graytext{on(a, b)}, \graytext{under(a, b)}, and \graytext{inhand(a)}. \graytext{inhand(a) = True} when the object a's z value is above a predefined height threshold. \graytext{on(a, b) = True} when i) a is above b (determined by checking if a's axis-aligned bounding box is greater than b'x axis-aligned bounding box) ii) a's bounding box intersects b's bounding box  iii) a is not inhand iv) the z distance between a and b is below a certain threshold. \graytext{under(a, b) = True} when a is not above b and a's bounding box intersects b's bounding box.

% For two objects a and b, \graytext{on(a, b)} evaluates to true if is_above(a, b) is_intersecting(a, b)
% and not is_inhand(obj_a, sim=sim)
% and abs(obj_a.aabb(sim=sim)[0, 2] - obj_b.aabb(sim=sim)[1, 2]) < on_distance

The proposed goal proposition prediction technique is also constrained to predict within this set of predicates.
We highlight that objects are neither specified as within or beyond the robot workspace, as we leave it to the skill value functions to determine the feasibility of the primitives  (Sec.~\ref{sec:learning-models}). 

We note that planning in high-dimensional observation spaces is not the focus of this work. 
Thus, we assume knowledge of objects in the scene and use hand-crafted heuristics to detect spatial relations between objects. 
There exists several techniques to distill high-dimensional observations into scene descriptions, such as the one used in \cite{zeng2022socratic}.
We leave exploration of these options to future work.

\subsection{In-Context Examples}
\label{sec:suppl-incontext-examples}

For all experiments and methods, we use the following in context examples.

\noindent\fbox{\parbox{0.97\linewidth}{\small{\texttt{{\\Available scene objects: ['table', 'hook', 'rack', 'yellow box', 'blue box', 'red box']\\
Object relationships: ['inhand(hook)', 'on(yellow box, table)', 'on(rack, table)', 'on(blue box, table)']\\
Human instruction: How would you push two of the boxes to be under the rack?\\
Goal predicate set: [['under(yellow box, rack)', 'under(blue box, rack)'], ['under(blue box, rack)', 'under(red box, rack)'], ['under(yellow box, rack)', 'under(red box, rack)']]\\
Top 1 robot action sequences: ['push(yellow box, hook, rack)', 'push(red box, hook, rack)']\\}}}}}

\noindent\fbox{\parbox{0.97\linewidth}{\small{\texttt{{\\Available scene objects: ['table', 'cyan box', 'hook', 'blue box', 'rack', 'red box']\\
Object relationships: ['on(hook, table)', 'on(rack, table)', 'on(blue box, table)', 'on(cyan box, table)', 'on(red box, table)']\\
Human instruction: How would you push all the boxes under the rack?\\
Goal predicate set: [['under(blue box, rack)', 'under(cyan box, rack)', 'under(red box, rack)']]\\
Top 1 robot action sequences: ['pick(blue box)', 'place(blue box, table)', 'pick(hook)', 'push(cyan box, hook, rack)', 'place(hook, table)', 'pick(blue box)', 'place(blue box, table)', 'pick(hook)', 'push(blue box, hook, rack)', 'push(red box, hook, rack)']\\}}}}}

\noindent\fbox{\parbox{0.97\linewidth}{\small{\texttt{{\\Available scene objects: ['table', 'cyan box', 'red box', 'hook', 'rack']\\
Object relationships: ['on(hook, table)', 'on(rack, table)', 'on(cyan box, rack)', 'on(red box, rack)']\\
Human instruction: put the hook on the rack and stack the cyan box above the rack - thanks\\
Goal predicate set: [['on(hook, rack)', 'on(cyan box, rack)']]\\
Top 1 robot action sequences: ['pick(hook)', 'pull(cyan box, hook)', 'place(hook, rack)', 'pick(cyan box)', 'place(cyan box, rack)']\\}}}}}

\noindent\fbox{\parbox{0.97\linewidth}{\small{\texttt{{\\Available scene objects: ['table', 'rack', 'hook', 'cyan box', 'yellow box', 'red box']\\
Object relationships: ['on(yellow box, table)', 'on(rack, table)', 'on(cyan box, table)', 'on(hook, table)', 'on(red box, rack)']\\
Human instruction: Pick up any box.\\
Goal predicate set: [['inhand(yellow box)'], ['inhand(cyan box)']]\\
Top 1 robot action sequences: ['pick(yellow box)']\\}}}}}

\noindent\fbox{\parbox{0.97\linewidth}{\small{\texttt{{\\Available scene objects: ['table', 'blue box', 'cyan box', 'hook', 'rack', 'red box', 'yellow box']\\
Object relationships: ['inhand(hook)', 'on(red box, rack)', 'on(yellow box, table)', 'on(blue box, table)', 'on(cyan box, rack)', 'on(rack, table)']\\
Human instruction: could you move all the boxes onto the rack?\\
Goal predicate set: [['on(yellow box, rack)', 'on(blue box, rack)']]\\
Top 1 robot action sequences: ['pull(yellow box, hook)', 'place(hook, table)', 'pick(yellow box)', 'place(yellow box, rack)', 'pick(blue box)', 'place(blue box, rack)']\\}}}}}

\noindent\fbox{\parbox{0.97\linewidth}{\small{\texttt{{\\Available scene objects: ['table', 'blue box', 'red box', 'hook', 'rack', 'yellow box']\\
Object relationships: ['on(hook, table)', 'on(blue box, table)', 'on(rack, table)', 'on(red box, table)', 'on(yellow box, table)']\\
Human instruction: situate an odd number greater than 1 of the boxes above the rack\\
Goal predicate set: [['on(blue box, rack)', 'on(red box, rack)', 'on(yellow box, rack)']]\\
Top 1 robot action sequences: ['pick(hook)', 'pull(blue box, hook)', 'place(hook, table)', 'pick(blue box)', 'place(blue box, rack)', 'pick(red box)', 'place(red box, rack)', 'pick(yellow box)', 'place(yellow box, rack)']\\}}}}}

\noindent\fbox{\parbox{0.97\linewidth}{\small{\texttt{{\\Available scene objects: ['table', 'cyan box', 'hook', 'red box', 'yellow box', 'rack', 'blue box']\\
Object relationships: ['on(hook, table)', 'on(red box, table)', 'on(blue box, table)', 'on(cyan box, table)', 'on(rack, table)', 'under(yellow box, rack)']\\
Human instruction: How would you get the cyan box under the rack and then ensure the hook is on the table?\\
Goal predicate set: [['under(cyan box, rack)', 'on(hook, table)']]\\
Top 1 robot action sequences: ['pick(blue box)', 'place(blue box, table)', 'pick(red box)', 'place(red box, table)', 'pick(hook)', 'push(cyan box, hook, rack)', 'place(hook, table)']\\}}}}}

\noindent\fbox{\parbox{0.97\linewidth}{\small{\texttt{{\\Available scene objects: ['table', 'cyan box', 'hook', 'yellow box', 'blue box', 'rack']\\
Object relationships: ['on(hook, table)', 'on(yellow box, rack)', 'on(rack, table)', 'on(cyan box, rack)']\\
Human instruction: set the hook on the rack and stack the yellow box onto the table and set the cyan box on the rack\\
Goal predicate set: [['on(hook, rack)', 'on(yellow box, table)', 'on(cyan box, rack)']]\\
Top 1 robot action sequences: ['pick(yellow box)', 'place(yellow box, table)', 'pick(hook)', 'pull(yellow box, hook)', 'place(hook, table)']\\}}}}}

\noindent\fbox{\parbox{0.97\linewidth}{\small{\texttt{{\\Available scene objects: ['table', 'cyan box', 'hook', 'rack', 'red box', 'blue box']\\
Object relationships: ['on(hook, table)', 'on(blue box, rack)', 'on(cyan box, table)', 'on(red box, table)', 'on(rack, table)']\\
Human instruction: Move the warm colored box to be underneath the rack.\\
Goal predicate set: [['under(red box, rack)']]\\
Top 1 robot action sequences: ['pick(blue box)', 'place(blue box, table)', 'pick(red box)', 'place(red box, table)', 'pick(hook)', 'push(red box, hook, rack)']\\}}}}}

\noindent\fbox{\parbox{0.97\linewidth}{\small{\texttt{{\\Available scene objects: ['table', 'blue box', 'hook', 'rack', 'red box', 'yellow box']\\
Object relationships: ['on(hook, table)', 'on(red box, table)', 'on(blue box, table)', 'on(yellow box, rack)', 'on(rack, table)']\\
Human instruction: Move the ocean colored box to be under the rack and ensure the hook ends up on the table.\\
Goal predicate set: [['under(blue box, rack)']]\\
Top 1 robot action sequences: ['pick(red box)', 'place(red box, table)', 'pick(yellow box)', 'place(yellow box, rack)', 'pick(hook)', 'push(blue box, hook, rack)', 'place(hook, table)']\\}}}}}

\noindent\fbox{\parbox{0.97\linewidth}{\small{\texttt{{\\Available scene objects: ['table', 'cyan box', 'hook', 'rack', 'red box', 'blue box']\\
Object relationships: ['on(hook, table)', 'on(cyan box, rack)', 'on(rack, table)', 'on(red box, table)', 'inhand(blue box)']\\
Human instruction: How would you set the red box to be the only box on the rack?\\
Goal predicate set: [['on(red box, rack)', 'on(blue box, table)', 'on(cyan box, table)']]\\
Top 1 robot action sequences: ['place(blue box, table)', 'pick(hook)', 'pull(red box, hook)', 'place(hook, table)', 'pick(red box)', 'place(red box, rack)', 'pick(cyan box)', 'place(cyan box, table)']\\}}}}}

\vspace{10pt}
    % \section{Extended Results}
% \label{sec:results-details}

% We provide the expanded results of (1) the failure modes of Text2Motion and \hm{} (Fig.~\ref{}) and (2) the ablation on plan termination techniques applied to Text2Motion (Fig.~\ref{}) comparing \texttt{stop} scoring versus goal proposition prediction.
% The results are consistent with the aggregate metrics reported in the main draft.
    \section{Real world demonstration}
\label{sec:demos}

\subsubsection{Hardware setup}

We use a kinect v2 camera for RGB-D images and manually adjust the color thresholds to segment the objects. Our segmentations allow us to estimate the object poses using the depth image, which we use to construct environment geometric states. We run robot experiments on a Franka robot arm.

 
\subsubsection{Robot demonstration}

% We provide qualitative demos in the supplementary video.
Please see our \link{https://sites.google.com/stanford.edu/text2motion}{project page} for demonstrations of Text2Motion operating on a real robot.

\end{appendices}

\end{document}


