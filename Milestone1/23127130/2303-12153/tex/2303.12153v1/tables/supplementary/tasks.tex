\begin{table*}[t]
    \centering
    \caption{\textbf{TableEnv manipulation task suite}. We use the following shorthands as defined in the paper: lh: long-horizon, lg: lifted goals, pap: partial affordance perception.}
    % \adjustbox{max width=\linewidth}{
    \begin{tabular}{@{}l|c|l@{}}
    \toprule
    \textbf{Task ID} & \textbf{Properties} & \textbf{Instruction}\\
    \midrule
    \textbf{Task 1} & LH & How would you pick and place all of the boxes onto the rack?” \\
    \textbf{Task 2} & LH + LG & How would you pick and place the yellow box and blue box onto the table, then use the hook to push the \\ & & cyan box under the rack?”

 \\
    \textbf{Task 3} & LH + PAP & How would you move three of the boxes to the rack?”
 \\
    \textbf{Task 4} & LG + PAP & How would you put one box on the rack?”

 \\
    \textbf{Task 5} & LH + LG + PAP & How would you get two boxes onto the rack?”

 \\
    \textbf{Task 6} & LH + LG + PAP & How would you move two primary colored boxes to the rack?”

 \\
    % \textbf{Real Task 1} & LH + LG + PO & Description \\
    % \textbf{Real Task 2} & LH + LG + PO & Description \\
    \bottomrule
    \end{tabular}
    \label{table:text2motion-domains}
\end{table*}