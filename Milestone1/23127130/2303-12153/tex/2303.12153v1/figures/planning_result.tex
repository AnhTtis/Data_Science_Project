\begin{figure}
    \centering    
    \includegraphics[width=0.5\textwidth]{figures/imgs/main_plot.pdf}
    \vspace{-15pt}
    \caption{\textbf{Results on the TableEnv manipulation domain} with 10 seeds for each task. {\bf Top:} Our method (Text2Motion) significantly outperforms all baselines on tasks involving partial affordance perception (Task 4, 5, 6). For the tasks without partial affordance perception, the methods that use policy sequence optimization (ours and \hm) both convincingly outperform the methods (\scgs~and \imgs) that do not use policy sequence optimization. We note that \hm~performs well on the tasks without partial affordance perception as it has the advantage of outputting \textit{multiple} goal-reaching candidate task plans and selecting the most geometrically feasible. {\bf Bottom:} Methods without policy sequence optimization tend to have high sub-goal completion rates but very low success rates. This divergence arises because it is possible to make progress on tasks without resolving geometric dependencies in the earlier timesteps; however, failure to account for geometric dependencies results in failure of the overall task. }
    \label{fig:planning-result}
    % \vspace{-10pt}
\end{figure}