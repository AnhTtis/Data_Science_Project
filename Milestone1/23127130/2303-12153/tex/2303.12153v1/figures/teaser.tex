\begin{figure}
    \centering
    \includegraphics[width=\columnwidth]{figures/imgs/teaser.pdf}
    \vspace{-15pt}
    \caption{Consider the following problem where a robot is asked to ``fetch \textit{any} box from the table'' by sequencing pretrained manipulation skills.
    To solve the task, the robot must apply semantic reasoning over scene description and the natural language instruction to deduce what actions would make progress towards acquiring a box for which \graytext{on(box, table)} holds true (i.e. the red box and blue box).
    It must also apply geometric reasoning to ensure that skills are sequenced and executed in a manner that is likely to succeed. 
    A \textit{reactive agent}~\cite{saycan-2022, innermono-2022} that selects skills based on their semantic usefulness and geometric feasibility at the current timestep may avoid infeasible actions like \graytext{Pick(blue)} or \graytext{Pick(red)} (i.e. under the rack and beyond the robot workspace, respectively).
    However, upon deciding on \graytext{Pick(hook)} and executing the action, the reactive agent may be unable to feasibly \graytext{Pull(red, hook)} due to an insufficient handle grasp, and is thus forced to \graytext{Place(hook, table)}, returning the environment to its original state.
    Text2Motion also selects skills by their usefulness and feasibility.
    However, instead of immediately executing actions, Text2Motion constructs plans of skills and coordinates their geometric dependencies through policy sequence optimization. 
    Therefore, upon planning the skill sequence \graytext{Pick(hook)}, \graytext{Pull(red, hook)}, \graytext{Pick(red)}, our method will compute both a grasp position on the hook that enables pulling the red box, and a pulling distance on the red box that enables its acquisition.}
    \label{fig:teaser}
    \vspace{-15pt}
\end{figure}
