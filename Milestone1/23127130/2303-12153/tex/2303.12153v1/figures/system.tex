\begin{figure*}
    \centering
     \vspace{-30pt}
     \includegraphics[width=\textwidth]{figures/imgs/system.pdf}
    \vspace{-20pt}
    \caption{Text2Motion planning overview. The user provides a natural language instruction for the robot, and then Text2Motion outputs a feasible sequence of skills to solve the given task. First, we use the LLM to predict the set of valid goal state propositions given the instruction and current state (left). This goal prediction will be used during planning to decide when the goal is satisfied and planning can be terminated. In the first planning iteration, Text2Motion uses the LLM to propose $k$ candidate skills with the top LLM scores. The geometric feasibility planner then evaluates the feasibility of each candidate skill, and the one with the highest product of LLM and geometric feasibility scores is selected. The successor state of this skill is predicted by the geometric feasibility planner's dynamics model. If the predicted state does not satisfy the goal propositions, then it is given to the LLM to plan the next skill. If the goal propositions are satisfied, then the planner returns. Text2Motion interleaves LLM planning with policy sequence optimization at each iteration. The \hm{} method, which is used as an experiment baseline, uses the LLM to propose entire plans first and then runs policy sequence optimization afterwards. As shown in the experiments, this approach fails when the space of candidate task plans is large but few skills are geometrically feasible.}
    \label{fig:system}
    \vspace{-10pt}
\end{figure*}