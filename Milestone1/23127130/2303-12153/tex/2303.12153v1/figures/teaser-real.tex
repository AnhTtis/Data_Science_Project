\begin{figure}
    \centering
    % \includegraphics[width=\columnwidth]{figures/imgs/t2m-teaser-real-world.pdf}
    \includegraphics[width=0.97\columnwidth]{figures/imgs/t2m-teaser-real-world.pdf}
    \vspace{-8pt}
    \caption{Consider the following problem where a robot is asked to ``get two primary-colored objects onto the rack'' by sequencing pretrained manipulation skills.
    To solve the task, the robot must apply semantic reasoning over the scene description and natural language instruction to deduce what skills should be executed to acquire a second primary-colored object, after noticing that a red object is already on the rack (i.e. \graytext{on(red, rack)}).
    It must also apply geometric reasoning to ensure that skills are sequenced in a manner that is likely to succeed. 
    %A \textit{reactive agent}~\cite{saycan-2022, innermono-2022} that selects skills based on their semantic usefulness and geometric feasibility at the current timestep may avoid infeasible actions like \graytext{Pick(blue)} (beyond the robot's workspace).
    %However, upon deciding to execute \graytext{Pick(hook)}, the reactive agent may be unable to feasibly \graytext{Pull(blue, hook)} due to an insufficient handle grasp, and is thus forced to \graytext{Place(hook, table)}, returning the environment to its original state.
    Text2Motion selects skills by their usefulness and feasibility.
    Related work~\cite{saycan-2022, innermono-2022} immediately executes skills that are deemed useful and feasible at the current timestep. Instead, Text2Motion constructs plans of skills and coordinates their geometric dependencies through policy sequence optimization~\cite{taps-2022}. 
    Therefore, upon planning the skill sequence \graytext{Pick(hook)}, \graytext{Pull(blue, hook)}, \graytext{Pick(blue)}, \graytext{Place(blue, rack)}, our method will compute a grasp position on the hook that enables pulling the blue object, and a pulling distance on the blue object that enables its acquisition, and a grasp location on the blue object that enables a collision-free placement on the rack.}
    \label{fig:teaser}
    \vspace{-20pt}
\end{figure}
