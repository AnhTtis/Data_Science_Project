\documentclass[conference]{IEEEtran}
\IEEEoverridecommandlockouts
% The preceding line is only needed to identify funding in the first footnote. If that is unneeded, please comment it out.
% \usepackage{cite}
\pdfoutput=1
\usepackage{amsmath,amssymb,amsfonts}
\usepackage{physics}
\usepackage{pdfpages}
\usepackage{algorithmic}
\usepackage{graphicx}
\usepackage{multirow}
\usepackage{adjustbox}
\usepackage{wrapfig}
\usepackage{caption}
\usepackage{subcaption}
\usepackage{booktabs}
\usepackage[maxbibnames=99, sorting=none]{biblatex}
\renewcommand{\bibfont}{\normalfont\small}
\addbibresource{citations.bib}
\graphicspath{ {./images/} }
\usepackage{textcomp}
\usepackage{xcolor}

\DeclareRobustCommand*{\IEEEauthorrefmark}[1]{%
  \raisebox{0pt}[0pt][0pt]{\textsuperscript{\footnotesize #1}}%
}

\def\BibTeX{{\rm B\kern-.05em{\sc i\kern-.025em b}\kern-.08em
    T\kern-.1667em\lower.7ex\hbox{E}\kern-.125emX}}
\begin{document}

\title{Multimodal Adaptive Fusion of Face and Gait Features using Keyless attention based Deep Neural Networks for Human Identification\\
{\footnotesize}
\thanks{}
}
\author{\IEEEauthorblockN{Ashwin Prakash\IEEEauthorrefmark{1},
Thejaswin S\IEEEauthorrefmark{1} and 
Athira Nambiar\IEEEauthorrefmark{1}\textsuperscript{*},
Alexandre Bernardino\IEEEauthorrefmark{2}}
\IEEEauthorblockA{\IEEEauthorrefmark{1}Department of Computational Intelligence, SRM Institute of Science and Technology, Chennai, India.\\
\{ap4471, ts6959, athiram\}@srmist.edu.in}
\IEEEauthorblockA{\IEEEauthorrefmark{2}Department of Electrical and Computer Engineering, Instituto Superior Técnico, Univ. of Lisbon, Portugal.\\
alex@isr.tecnico.ulisboa.pt }}

\maketitle

\begin{abstract}
Biometrics plays a  significant role in vision-based surveillance applications. Soft biometrics such as gait is widely used with face in surveillance tasks like person recognition and re-identification. Nevertheless,  in practical scenarios, classical fusion techniques respond poorly to changes in individual users and in the external environment. To this end, we propose a novel adaptive multi-biometric fusion strategy for the dynamic incorporation of gait and face biometric cues by leveraging keyless attention deep neural networks. Various external factors such as viewpoint and distance to the camera, are investigated in this study. Extensive experiments have shown superior performance of the proposed model compared with the state-of-the-art model.


\end{abstract}

\begin{IEEEkeywords}
Soft-biometrics, surveillance, Gait, Face, Adaptive fusion, person identification, Deep Learning, attention models, multimodal fusion.
\end{IEEEkeywords}
\vspace{-0.3cm}
\section{Introduction}

Human biometrics refers to the unique intrinsic physical or behavioural traits that allow distinguishing between different individuals, e.g.,  face, fingerprint, hand geometry, iris, and gait. The use of biometrics helps in various surveillance applications such as access control, human recognition, and re-identification. Single biometric modalities are often affected by practical challenges such as noisy data, lack of distinctiveness, intra/ inter-class variability, error rate, and spoof attacks. A common method to overcome this issue is to combine multiple biometric modalities, known as multimodal biometric fusion.

A critical constraint that any biometric system confronts is the variation in the environment owing to external conditions. This includes user-induced variability, i.e., inherent distinctiveness, pose, distance, and expression, or environment-induced variability, i.e., lighting condition, background noise, and weather conditions~\cite{NAP12720}. These constraints have not been adequately addressed in the literature on multimodal fusion. For instance, most of the existing works are based on static fusion strategies, wherein the fusion rules are fixed for certain external conditions such as pose/ lighting/ distance or based on manual computations. As a result, when the environment changes, the biometric system performs sub-optimally. To overcome this issue, a novel context-aware adaptive multibiometric fusion strategy, which can dynamically adapt the fusion rules to external conditions, is proposed in this paper. In particular, the adaptive fusion of gait and face at different viewpoints was investigated using an attention-based deep learning technique.


% it is required to have an adaptable multimodal fusion strategy that is capable of adjusting based on the context. To this end, this paper proposes context-aware adaptive multi-biometric fusion, which can dynamically adapt the fusion rules to the external conditions for human recognition in realistic environments. 

 

Face is one of the predominant biometric traits commonly employed in human recognition. Similarly, gait is an important soft biometric commonly used in surveillance applications, because it is unobtrusive and perceivable from a distance~\cite{singh2021survey}. While fusing gait and face, the most influential factors may be the view angle and distance from the subject to the camera. Notably, gait can be clearly captured in the lateral view, whereas the face can be well captured in the frontal view. Based on this rationale, a novel context-aware adaptive fusion mechanism was designed to assign weights to gait and face biometric cues based on the context. The key notion of the proposed model is that when the person is in far/ lateral view, gait features should be gaining more priority than the less visible facial cues, whereas when the person is in near/ frontal view, the face should be getting more importance than the partially occluded gait features. 

To facilitate the aforesaid context-aware adaptive fusion strategy, a keyless attention-based deep learning fusion is leveraged in the multimodal biometric fusion framework. As mentioned in~\cite{long2018multimodal}, keyless attention is a sophisticated and effective technique for better accounting for the sequential character of data without the need for supplementary input, thereby excelling in identifying relationships across modalities. Extensive experiments are conducted via individual biometric-based identification, na\"ive bilinear pooling~\cite{lin2015bilinear} based multimodal fusion and keyless attention-based adaptive fusion mechanism. Results clearly highlight the superior performance of the proposed model.

% Accordingly, the weights of gait and face in the adaptive fusion are adjusted in real-time based on the distance and view-point to the camera.


% . Gait entails the series of limb motions that results in locomotion. 
% Different from hard
% biometrics, gait soft biometry lacks the distinctiveness and time invariance to identify a person with high
% reliability. However, they have certain advantages over hard biometrics, making them best
% suited to deploy in surveillance applications, e.g., unobtrusive, and perceivable from a distance. 


The remainder of this paper is organized as follows. Related works on face and gait-based human recognition are detailed in Section 2. Section 3 presents the framework of the proposed context-aware adaptive multibiometric fusion method. The experiments and results are presented in Sections 4 and 5, respectively. Finally, conclusions and future directions are presented in Section 6.



% In addition, a naive bilinear pooling strategy is also investigated to compare against. (https://arxiv.org/pdf/2111.08910.pdf) 
 % To the best of our knowledge, the only available work in this direction of adaptive fusion of gait and face is reported in \cite{geng2008adaptive}, however, it leverages classical machine learning techniques using handcrafted features and manual computations. In contrast to that work, our work presents a more advanced version by integrating better deep neural models with the help of attention mechanisms.


% Biometrics- importance in surveillance-In this regard, various biometric modalities were used e.g. gait and face- multimodal fusion can improve the results- There have been many works in this regard. Classical ML to deep learning approaches. 

% However, the challenge-fusion strategy has to be adaptive- for instance, in a camera i/p scenario, face and gait fusion depends on the camera, same with pose and image i/p. S/m should capable of learning an adaptive strategy based on the situation, which cues to get more priority. However, most of the existing works make use of static fusion rules.

% In this work- a novel keyless attention-based deep learning fusion strategy is proposed- In particular, gait and face features are fused.- To the best of our knowledge, the only available work in this direction is..., however, it's in classical ML. Our work presents a more advanced version by integrating better DL models.





% An extensive area of computer vision research is automated person recognition based on visual cues. Systems for enhancing safety in public places are significant applications. The most common methods for detection and recognition use information from the iris, face, or fingerprint. Although often impractical, these techniques are effective in many cases. They frequently need the subject's cooperation because they are frequently sensitive to obstruction, great distances, or low-resolution data.

% Access control, video surveillance, security, person re-identification, law enforcement, human-computer interface, and other fields all use automatic facial recognition systems. One of the biometric traits that is most frequently employed for recognition is the face. People are remarkably good at recognizing, extracting, and interpreting data from human faces.

% A series of limb motions that results in locomotion is known as a gait. Its analysis has numerous applications in the realms of forensic science, medicine, and surveillance. Gait analysis can be utilized to identify people in forensic and surveillance situations [14,15,16]. Gait Recognition uses a person's natural walking gait to identify them. The movement and silhouette of a human are coarse traits that can withstand noise and low resolution. Additionally, unlike biometrics like iris and fingerprints that necessitate high-resolution photos and specific tools for identification, gait recognition models do not. Furthermore, unlike other methods that call for collaboration between the studied person and the identification system, gait recognition techniques do not depend on individual interaction. Additionally, gait data may be easily gathered from a distance, which is a huge benefit over other methods, particularly when the studied person is not there to aid with identification.

% In some classic biometric systems, such as facial features, a person cannot be reliably identified using just one biometric trait. Using adaptive fusion and two traits—gait and face—we describe a method to address these issues with existing biometric systems and increase the accuracy of human identification.

\section{Related Work}
% \textcolor{blue}{reduce to two third of the current size!}
% Separate into paragraphs for each module ie, face, gait, and fusion
% Adaptive fusion - no previous work done in regard to gait and face

One of the earliest face recognition systems was discovered in~\cite{mm1966} using the manual marking of various facial landmarks. Recognition of faces in images with objects has gained popularity with~\cite{turk1991eigenfaces}, which introduced the eigenface method. Since then, various other similar techniques, e.g., Linear Discriminant Analysis, to produce
Fisherfaces,  Gabor, LBP, and PCANet were reported in ~\cite{wang2021deep}. Recently, deep learning-based techniques have also gained popularity, e.g. 
DeepFaces, Facenet, and Blazeface approach human-level performance under unconstrained conditions~\cite{wang2021deep} (DeepFace: 97.35\% vs. Human: 97.53\%).

% One of the earliest face recognition systems was discovered in~\cite{mm1966} using the manual marking of various facial landmarks. The recognition of faces in images with objects has gained popularity with~\cite{turk1991eigenfaces}, which introduced the eigenface method. Since then, various other similar techniques, e.g., Linear Discriminant Analysis, to produce 
% Fisherfaces \cite{wechsler2007reliable},  Gabor\cite{liu2002gabor}, LBP\cite{zhang2005local}, PCANet 
% Fisherfaces,  Gabor, LBP, and PCANet were reported~\cite{wang2021deep}. Recently, deep learning-based techniques also gained popularity e.g. 
% DeepFaces \cite{taigman2014deepface}, Facenet \cite{schroff2015facenet}, Blazeface etc. 
% DeepFaces, Facenet, and Blazeface approach human-level performance on the unconstrained conditions~\cite{wang2021deep} (DeepFace: 97.35\% vs. Human: 97.53\%). 
% A detailed survey on the various approaches in face recognition is presented in \cite{wang2021deep}.

Classical gait-based identification approaches use either model-based or appearance-based approaches\cite{singh2021survey}. The former detects joints/body parts using 2D cameras or depth cameras. For example,~\cite{cunado2003automatic} applied Hough transform to detect legs in each frame, whereas~\cite{wang2004fusion} leveraged Procrustes shape analysis to calculate joint angles of body parts. Gait recognition/re-identification using a Kinect camera has also been proposed in some works ~\cite{nambiar2017towards}. 
In contrast to model-based approaches, appearance-based approaches use richer information, such as silhouettes of the human body in gait frames, to recognise gaits, e.g., gait energy image (GEI)~\cite{han2005individual} and GEI-based local multi-scale feature descriptors~\cite{lishani2019human}. Recent deep learning approaches presented advanced techniques, e.g., view-invariant gait recognition using a convolutional neural network GEINet~\cite{shiraga2016geinet}, a comprehensive model with both LSTM and residual attention components for cross-view gait recognition~\cite{li2019attentive}.



% Fusion of gait and face for human identification is also reported in the literature. 

On the fusion of gait and face for human identification, one of the early works~\cite{kale2004fusion} proposed a fusion strategy by combining the results of gait and face recognition algorithms based on sequential importance sampling. A probabilistic combination of facial and gait cues was studied in~\cite{shakhnarovich2002probabilistic}. Yet another work on the adaptive fusion of gait and face is~\cite{geng2008adaptive} via score-level fusion. All the aforementioned studies leverage either classical machine learning techniques using handcrafted features, static fusion rules, or manual computations. On the contrary, in this work, we present a deep learning technique based on a keyless attention-based adaptive fusion mechanism for human identification, one of its first kind to the best of our knowledge. 
% \textcolor{red}{repeat again the advantages of using keyless attention}


% To the best of our knowledge, the only available work in this direction of adaptive fusion of gait and face is reported in \cite{geng2008adaptive}. However, it leverages classical machine-learning techniques using handcrafted features and manual computations. In contrast to that work, our work presents a more advanced version by integrating better deep neural models with the help of attention mechanisms. 

% All of the aforementioned works leverage either classical machine learning techniques using handcrafted features or static fusion rules/ manual computations. On the contrary, in this work, we present a deep learning tech que based on a keyless attention-based adaptive fusion mechanism for human identification, one of its first kind to the best of our krecognise 


% . To the best of our knowledge, such a deep learning approach-based adaptive fusion study of gait and face for human identification was not yet reported in the literature.  




%, static fusion rules, and manual computations were involved since it's based on handcrafted features and classical machine learning techniques. 



\section{Multimodal Adaptive Fusion Methodology }
\label{METHOD}
% In this proposed archiecture... (fig of holistic archi), fusing gait and face. In detail, ..explain the archi.
The proposed keyless attention-based adaptive fusion of face and gait towards human identification is shown in Fig.\ref{fig-archi}, in which all the symbols are introduced in the following subsections. The proposed framework maps spatio-temporal feature sequences corresponding to gait and face to a single label. First, the video sequence's descriptors of gait and face are extracted from each frame via a \textbf{\textit{Feature extractor}} module. Further, the \textit{\textbf{Attention \& Fusion}} block is employed to compute the feature importance and adaptively amalgamate them.
% \textcolor{red}{This is not very clear. Maybe explain separately the role of attention and then the role of the fusion.} 
Finally, the class probabilities are generated by a \textbf{\textit{classifier}} module using a fully connected (FC) layer, followed by a softmax layer.


%, which mainly consists of three important parts: audio and video encoder, attention, fusion, and classifier.


% In this section, the framework of the proposed gait and face adaptive fusion is detailed. Fig. 1 shows the proposed framework used in this study. 

% Face and gait features are extracted from video frames. The features are then combined using the proposed adaptive fusion mechanism that fuses both the features and then trains the model for person recognition. The fusion layer imparts weightage to the face and gait features according to what the layer learned in the training process. Details of each component are explained below.

\hfill
\break

%         \begin{center}
%             \includegraphics[width=10cm, height=8cm]{images/architecture.png}
%             \hfill
%             \caption{Fig.1 Model Architecture}
%         \end{center}

\begin{figure*}[htbp]
\begin{center}
\includegraphics[width=14cm, height=7cm]{images/Softbio.pdf}
\caption{Overall architecture of the proposed keyless attention-based adaptive fusion of face and gait for person recognition. The gait and face features are encoded by the gait and face feature extraction networks \textit{$\mathcal{F}$} and \textit{$\mathcal{G}$}, respectively. The outputs are subsequently weighted using keyless attention. Context-aware adaptive multimodal fusion is then employed to fuse global gait and facial features. Finally, the outputs are passed through the classifier to determine the class (Person ID) of the person.}
\label{fig-archi}
\end{center}
\end{figure*}

\vspace{-1cm}

\subsection{Gait feature extractor}
Gait recognition involves recognizing a person based on their gait features, i.e, movement patterns~\cite{murray1967gait}. 
% Human motion involves the temporal variation of human silhouettes. 
The temporal variation in human silhouettes is considered by calculating the cyclic pattern of movement, commonly referred to as the \textit{gait cycle}. It can be observed that the size of the closed area between the legs and the aspect ratio in the human silhouettes are alternating periodically in a gait sequence (Refer Fig. 2(a)\& 2(b)).
% Dual-ellipse fitting approach computes the gait image's first-order derivative to find peak positions by seeking the positive-to-negative zero-crossing points from background removed silhouette images. This work implements the Dual ellipse fitting approach in \cite{hwang2018} to the gait silhouette images shown in Fig. 3a and 
Based on this notion, a complete gait cycle is determined by the number of frames between three consecutive local minima (two red points in Fig. 2(b)). The corresponding frames are extracted from the RGB images. This technique of gait cycle computation is applied to every person. Accordingly, the video is divided into an adequate number of frames required for gait feature computation.

The images are preprocessed and converted from RGB to grayscale to facilitate computational efficiency. Further, the extracted frames of gait silhouette images of height \textit{H} and width \textit{W} are fed onto a Convolutional LSTM~\cite{shi2015convolutional} architecture as depicted in gait feature extractor network \textit{$\mathcal{G}$} in Fig. \ref{fig-archi} and obtains a gait feature descriptor $G$. Formally, the gait feature sequence of a video can be represented as $G = \{g_1, \cdot \cdot \cdot , g_L\}, g_i \in \mathbb{R}^C$
where $g_i$ denotes the gait feature of frame \textit{i}, \textit{C} denotes the feature dimension, and \textit{L} denotes the number of frames.
 


% Formally, We feed 72 images to the network. After convolution, for each subject, the corresponding feature descriptor of size (1, 588) is generated.

% \textcolor{blue}
% {Formally,.. Let xx be the input images. -after conv, for each image frame corresponding feature descriptor of size xx is generated. }
% architecture which is more suited for the analysis of images with temporal parameters. This yields a 5-dimensioned gait feature corresponding to video frames per person. 

% the mean value of the gaitas a standard value which would be applied to every person and. In particular, the video is first divided into 24 frames based on gaobtainsle computation.

% of the gait cycle from the silhouette images and calculate the number of frames required. 



\begin{figure}[h!]%
        \centering
        \begin{subfigure}[htbp]{0.25\textwidth}
            \centering
            \includegraphics[width=3.5cm]{images/gaitsig.png}%
            \caption{}
        \end{subfigure}%
        \begin{subfigure}[htbp]{0.25\textwidth}
            \centering
            \includegraphics[width=3.5cm]{images/gait-signal.png}%
            \caption{}
        \end{subfigure}\break\vskip 1mm
        \begin{subfigure}[htbp]{0.165\textwidth}
            \centering
            \includegraphics[width=2.4cm]{images/gaitsetA-gait1.jpg}%
            \caption{0\textdegree}
        \end{subfigure}%
        \begin{subfigure}[htbp]{0.165\textwidth}
            \centering
            \includegraphics[width=2.4cm]{images/gaitsetA-gait3.jpg}%
            \caption{45\textdegree}
        \end{subfigure}%
        \begin{subfigure}[htbp]{0.165\textwidth}
            \centering
            \includegraphics[width=2.4cm]{images/gaitsetA-gait2.jpg}%
            \caption{90\textdegree}
        \end{subfigure}
        \caption{(a) Human silhouette taken from a gait video sequence of CASIA-A. (b) Representation of silhouette aspect ratio over the whole video. The marked points in red represent the starting and ending of one gait cycle. (c),(d) \& (e) Glimpses from the CASIA-A dataset at angles 0\textdegree, 45\textdegree, 90\textdegree \ respectively.}
        \label{gait_class}
    \end{figure}
% \vspace{-.35cm}    
% \hfill

% (\textcolor{red}{Refer http://palm.seu.edu.cn/xgeng/files/wacv08.pdf})
% \textcolor{red}{explain here- the functional unit of gait is gait cycle.} 
% Using the above technique, we extract the starting and ending frames of the gait cycle from the silhouette images and calculate the number of frames required. 


\subsection{Face feature extractor}
Face recognition involves recognizing a person by his facial features~\cite{mm1966}. In our case, since the viewpoint and distance of the person vary significantly across the frames, traditional face detection algorithms that rely on the frontal view do not work well. Hence, facial bounding boxes are initially cropped out of the video frames leveraging Google Mediapipe human pose detection framework~\cite{bazarevsky2020blazepose}. The framework employs a two-step detector-tracker setup where the detector locates the pose region-of-interest (ROI) within the frame and the tracker predicts all 33 keypoints from this ROI. In the case of videos, the detector is run only on the first frame and the ROI of the subsequent images is derived from the pose keypoints of the previous frame.
\begin{figure}[h!]%
    \centering
    \includegraphics[trim={0 5.5cm 0 6.5cm},clip,width=9cm]{images/FaceRetriever_2.pdf}
    \caption{Process of obtaining face images from Mediapipe pose estimation. 
    % The keypoint of ears are considered with fixed measurements for cropping out face from the input images(NO NEED.. WRITTEN IN THE CONTENT).
    }
    \vspace{-.75cm}
    % The images are fed onto the Mediapipe framework which produces keypoints onto the subject which we use to crop the face from the image by fixed measurements}
    \label{face_structure}
\end{figure}%
As shown in Fig.~\ref{face_structure}, from the estimated Mediapipe keypoints, the human face is manually cropped out by fixed measurements, with respect to the facial coordinates.

% the height and width of the face are set manually with respect to the ear coordinates (6\% and 3.5\% of the original image) and are then cropped from the original image to obtain the face image.


% Different types of Face recognition strategies and algorithms are given in Section II. 
% In this work, The dataset(mentioned in Sec. IVA) contains 3 types of angles namely 0\textdegree, 45\textdegree and 90\textdegree. Due to the difference in the angle of the dataset, traditional face detection algorithms that rely on the frontal view, as well as the clarity of the face, find it challenging to recognize the faces in the dataset. Therefore, the facial structures are cropped from video frames using Google's Mediapipe Human Pose Detection\cite{bazarevsky2020blazepose} Framework. The pose detection method implemented in \cite{bazarevsky2020blazepose} consists of a face detector with pose alignment, while inferencing employs a detector-tracker setup. The tracker predicts the key-point coordinates(33 points around the human body), the presence of a person in the current frame, and the refined region of interest in the current frame. If the tracker indicates no person in the frame, the detector is re-run on the next frame.

% the facial features are extracted by passing each image to the pose detection network and obtaining the landmarks i.e.  the furthest point in either part of the face is the position of the ears, and their coordinates \(e_1 = (x_1,y_1), e_2 = (x_2, y_2)  are chosen. The height and width of the face are set manually with respect to the ear coordinates(6\% and 3.5\% of the original image) and are then cropped from the original image to obtain the face image. 
The cropped face images are preprocessed, converted from RGB to grayscale, and resized to the dimension of $\textit{H} \times \textit{W}$. Further, the images are fed into ConvolutionalLSTM~\cite{shi2015convolutional} architecture to extract facial feature descriptor $F$ per person, as depicted in the face feature extractor network \textit{$\mathcal{F}$} in Fig.\ref{fig-archi}. Formally, the face feature descriptor corresponding to an \textit{L}-frame video is represented as $F = \{f_1, \cdot \cdot \cdot, f_L\}, f_i \in \mathbb{R}^C$, where $f_i$ represents the facial feature of frame \textit{i}. 
% And C is the feature dimension and L is the number of frames (\(L=72  in this work).


% \textcolor{blue}{mention the next sentence -72 images-in implementation details section}Formally, We feed 72 images to the network. 

% After convolution, for each subject, the corresponding feature descriptor $F$ is generated. % in order to extract the facial features across the video frames, Convolutional LSTM\cite{shi2015convolutional} architecture is utilized in order to extract spatio-temporal feature descriptors corresponding to the video frames per person. Formally, We feed 72 images to the network. After convolution, for each subject, the corresponding feature descriptor of size (1, 588) is generated.
% \vspace{-.5cm}
\subsection{Na\"ive fusion of Face and Gait via Bilinear pooling}\label{naive-fusion}

As an initial fusion technique, we propose the na\"ive bilinear pooling (BLP) method~\cite{lin2015bilinear} to fuse features. The method takes in the 3D tensor outputs from the final max-pooling layers of the face (\textit{$\mathcal{F}$}) and gait  (\textit{$\mathcal{G}$}) feature extraction networks. The outputs are further reshaped into the matrix of dimensions $p\times d$ and are combined  via the bilinear pooling method to obtain the fusion result $Z$, as follows:
\break
\vspace{-0.75cm}

\begin{equation}
    Z = FG^{T}, F \in \mathbb{R}^{p\times d}, G \in \mathbb{R}^{p\times d}
\end{equation}

The matrix $Z$ is then flattened into a vector and then passed onto the softmax activation function, where it computes the probability for class \(k\)  out of \(K\) classes.
% \begin{equation}
%     softmax(w_k^Tz) = \frac{e^{w_k^Tz}}{\sum_{j=1}^{K} e^{w_k^Tz}}, ~w_k \in \mathbb{R}^{pp}
% \end{equation}


\subsection{Keyless attention based Adaptive fusion of Face and Gait}
\label{attention}

Attention mechanisms are widely used in sequence models, enabling the modelling of dependencies regardless of their location in the input or output sequences~\cite {bahdanau2014neural}. In our case, not every frame in a video helps identify a subject in the same way. In order to estimate the importance weights for each frame, we adapt the attention mechanism. An attention function is a process that takes a query vector and a set of key-value pairs and produces an output.
% with all of the \textcolor{red}{query, keys, values}, and output represented as vectors. 
In existing soft attention mechanisms~\cite{bahdanau2014neural}, the weight computation is not limited to the feature vectors but also incorporates an additional input, such as the previously hidden state vector of the LSTM or a vector representing a target entity as in~\cite{wang2016relation}. These additional inputs along with feature vectors referred to as \textit{key vectors}, help to 
find the most related weighted average of feature vectors. However, the weights in our work depend only on the feature vectors and do not require any additional input, thus named as \textbf{\textit{keyless attention}}, synonymous to the work \cite{long2018multimodal}. 
In our case, referring to Fig. \ref{fig-archi}, the gait feature descriptor \textit{G} and face feature descriptor \textit{F} are further fed onto two attention modules \textit{viz.} Gait attention block and Face attention block respectively. Further, multimodal adaptive weights are computed via the fusion mechanism. Detailed explanations are given below.

% Since this attention mechanism doesn't rely on any input key vectors, it is referred to as.

% Since this attention mechanism doesn't rely on any input key vectors, it is referred to as \textbf{\textit{keyless attention}}.
% as soft attention finds the most related weighted average of feature vectors, and different key vectors result in different weighted averages}.

\subsubsection{\textbf{Face Attention}}\label{fatt}

The facial feature is updated by incorporating the attention mechanism to assign weighted visual elements. Formally, face attention is computed as follows:
\vspace{-0.25cm}
\begin{equation}
\Bar{f_i} = \mathbf{W_f} f_i + \mathbf{b_f}
\end{equation}
\vskip -4mm
\begin{equation}
\Bar{e_i} = \Bar{u}^T \tanh(\Bar{f_i})
\end{equation}
\vskip -3mm
\begin{equation}
\Bar{\alpha_i} = \frac{\exp{(\lambda \Bar{e_i})}}{\sum_{k=1}^{L} \exp{(\lambda \Bar{e_k})}}
\end{equation}
Here, $\Bar{f_i}$ is the low-dimension representation of frame \textit{i} and $\mathbf{W_f}$ \& $\mathbf{b_f}$ are the learnable parameters. The importance weight, $\bar{e_i}$, of the element $f_i$ is computed by the inner product between the new representation of $\bar{f_i}$ and a learnable vector $\bar{u}$. The normalized importance weight of facial feature $\Bar{\alpha_i}$ is calculated using the softmax function, as shown in Eq. (5).  $\lambda$ is a scale factor that ensures that the importance weights are evenly distributed, which ranges between 0 and 1. Nevertheless, it is observed in Eq.~(4) that $\tanh{(\cdot)}$ non-linearity may not be effective for learning complicated linkages, since $\tanh{(x)}$ is roughly linear for x $\in $ [$-1$, 1]. Therefore, inspired by the method as in~\cite{DauphinFAG16}, we leverage an effective gated mechanism as shown in Eq.~(6) to formulate a better normalized facial importance weight $\Bar{\alpha_i}$.
% to eliminate the problematic linearity of $\tanh(\cdot)$,
\vspace{0mm}
\begin{equation}
\Bar{\alpha_i} = \frac{\exp{\{\lambda \Bar{u}^T (\tanh(\Bar{f_i}) \odot sigm(\Bar{f_i})\}}}{\sum_{k=1}^{L} \exp{\{\lambda \Bar{u}^T (\tanh(\Bar{f_k}) \odot sigm(\Bar{f_k})\}}}
\end{equation}
\vspace{-2mm}
\begin{equation}
\Bar{\alpha} = \sum_{i=1}^{L} \Bar{\alpha_i}
\label{alpha}
\end{equation}%

where \textit{sigm($\cdot$)} is the sigmoid non-linearity and $\odot$ is an element-wise multiplication. This new $\Bar{\alpha_i}$ is further used to compute global facial attention weight 
$\Bar{\alpha}$ by combining facial importance weight across all \textit{L} frames(Refer Eq.~(7)).  

% is the attention weight for the face modality. This creates a learnable nonlinearity which will be leveraged throughout this work. This results in: further used for deriving the adaptive mechanism.
% The gating technique may eliminate the problematic linearity of $\tanh(\cdot)$, which creates a learnable nonlinearity.




\vskip -3mm



% where $\Bar{f_i}$ is a new low-dimension representation of frame i, and $\Bar{f}$ is the facial feature. A new representation of the element $f_i$ is created by feeding it to a fully connected layer first, then to a $\tanh$. After that, the inner product of the new representation of the element $\tanh$\(f_i  $\odot$ sigm\(f_i  and a learnable vector u is used to calculate the importance weight, and the normalized importance weight $\Bar{\alpha_i}$ is then calculated using the softmax function. With importance weights serving as the face feature, the weighted sum of the elements in the set F is used to calculate vector f. In Eq. 5, $\lambda$ is a scale factor that ensures that the importance weights are evenly distributed. It ranges from 0 to 1. The standard softmax function is the scaled-softmax when $\lambda$ = 1, and the vector v will be an average of the set V if $\lambda$ = 0.

\subsubsection{\textbf{Gait Attention}}\label{gatt}
Analogous to the face modality, the attention mechanism is incorporated in the gait counterpart as well. The global gait attention weight $\Bar{\beta}$  by leveraging the weighted visual elements in the gait stream is computed as follows.
% Dimension reduction is carried out before to the attention block to reduce computing complexity and relieve over-fitting. Analogous to Face Attention (Sec. \ref{fatt}), we compute for gait as: 
\vspace{-0.15cm}
% \begin{equation}
% \Bar{g_i} = \bold{W} g_i + \bold{b}
% \end{equation}
\begin{equation}
\Bar{g_i} = \mathbf{W_g} g_i + \mathbf{b_g}
\end{equation}
% \vskip -5mm
\begin{equation}
\Bar{\beta_i} = \frac{\exp{\{\lambda \Bar{u}^T (\tanh(\Bar{g_i}) \odot sigm(\Bar{g_i})\}}}{\sum_{k=1}^{L} \exp{\{\lambda \Bar{u}^T (\tanh(\Bar{g_k}) \odot sigm(\Bar{g_k})\}}}
\end{equation}
\vskip -2mm
\begin{equation}
\Bar{\beta} = \sum_{i=1}^{L} \Bar{\beta_i}
\label{beta}
\end{equation}
% sigm($\cdot$) is the sigmoid non-linearity, while $\odot$ is an element-wise multiplication and $\Bar{\beta}$ is the combined attention weight for gait. The gating technique may eliminate the problematic linearity of $\tanh(\cdot)$, which creates a learnable nonlinearity.



\subsubsection{\textbf{Context-aware Adaptive Fusion}}\label{adap-fusion}
From Eq.(\ref{alpha}) \& Eq.((\ref{beta}), we obtain the value of $\Bar{\alpha}$  and $\Bar{\beta}$, which are the global individual attention weights of the face and gait features, respectively. The weighted average of the face and gait feature is computed using the adaptive weights:\vspace{-2mm}
\begin{equation}
    \alpha = \frac{\norm{\Bar{\alpha}}}{\norm{\Bar{\alpha}} + \norm{\Bar{\beta}}}
\end{equation}
\begin{equation}
    \beta = \frac{\norm{\Bar{\beta}}}{\norm{\Bar{\alpha}} + \norm{\Bar{\beta}}}
\end{equation}
% where $\alpha$ and $\beta$ are the adaptive weights for face and gait features respectively.
The adaptive fusion is performed by combining the two features multiplied individually by their weighted global attention weights, as follows:
\vspace{-0.4cm}
\begin{equation}
    \textit{$\mathbb{Z}$} = \alpha F + \beta G
\end{equation}
\vskip -2mm
where \noindent{ $\mathbb{Z}$ refers to the context-aware adaptively fused feature. }
% The output posterior probabilities of the subject class can be produced using a fully-connected (FC) layer, followed by a softmax layer, based on the fusion vector. With weights defined by a neural network, we propose using a weighted average of low-dimensional features.
$\mathbb{Z}$ is further passed onto a fully-connected (FC) layer, followed by a softmax function that classifies the feature according to the \textit{K} classes provided. The resultant column vector, $R$ is then used to determine the class identifier (\textit{Person ID}) of the subject in consideration for the fused feature $\mathbb{Z}$ by\\ \vspace{-0.5cm}
\begin{equation}
    ID(\mathbb{Z}) = argmax(R)
\end{equation}
\vspace{-1cm}
\subsection{Objective functions}\label{obj-func}
The model classifier employs categorical cross-entropy loss, also known as Softmax loss. This supervised loss calculates the classification error among \textit{K} classes. The number of nodes in the softmax layer depends on the number of identities in the training set. Considering \(t\) and $w_k$ as the target vector and learnable vector respectively, the loss is computed as:
\vspace{-0.1cm}
\begin{equation}
    Loss = -\sum_{i}^{K} t_i\log(softmax(w_k^T\mathbb{Z})_i)
\end{equation}

% \centerline{\(softmax(w_k^TZ`)\)  = softmax function}
% \centerline{$w_k$ = learnable vector}



\section{Experimental Setup}
% In this section, we evaluate the performance of the Bilinear Pooling layer by plugging it with the output of the face and gait architectures.
\noindent \textbf{Dataset:} In this work, we use CASIA Gait Dataset A~\cite{wang2003silhouette}, which includes 19139 images of 20 subjects. Each person has 12 image sequences, 4 sequences for each of the three directions, i.e. 
0\textdegree, 45\textdegree, 90\textdegree (Refer Fig. 2(c), 2(d) and 2(e)). Among the 4 sequences per angle, 2 sequences are used for training, and the remaining 2 sequences are used for testing.
% for each angle of a person. 


\noindent \textbf{Evaluation protocols:} Standard evaluation metrics like \emph{accuracy} and \emph{log-loss} are employed to validate the performance of our model. \emph{Accuracy} is used to evaluate how well the algorithm is performing for all classes by giving them equal importance, whereas \emph{log loss} is considered to be a crucial metric that is based on probabilities. Mathematically, log-loss is computed by:
% Despite its complexity in interpretation, log loss remains a reliable measure for comparing various models. In particular, a decrease in log loss values indicates an improvement in the model's prediction accuracy for a specific problem. 
\begin{equation}
log\_loss = \frac{-1}{N \sum_{i=1}^{N} [y_i\ln p_i+(1-y_i)\ln(1-p_i)]}
\end{equation}
% \vskip 0.5cm
where \textit{N} is number of person and $y_i$ is the observed value and $p_i$ is predicted probability. 


\noindent \textbf{Implementation details:} The proposed method is implemented using the TensorFlow framework.
% and the result is evaluated using scikit-learn package, which is available online \footnote{Scikit-Learn Website \href{https://scikit-learn.org/stable/modules/generated/sklearn.metrics.log_loss.html}{scikit-learn}}. 
During training, video frames correspond to one gait cycle across three orientations 0\textdegree, 45\textdegree, and 90\textdegree are considered. In this work, the gait cycle corresponds to
\textit{L} = 24 frames, each with height $\textit{H} $= 128, and width $\textit{W}$ = 128 is used. The images are normalized using the RGB mean and standard deviation of ImageNet before passing them to the network. After dimension reduction, the resulting dimension of the gait and face feature descriptor is 588 each. In the experiments, we use Optuna~\cite{akiba2019optuna}, a hyperparameter optimization framework, to obtain the best hyperparameter for our models. 
% Optimizers include Adam, RMSprop, and SGD and activation functions include relu, selu, elu, swish. 
We train the network for approximately 1000 iterations. The implementations are done in a machine with Tesla V100 GPU with 12GB RAM and took around 1 hour to train the model.
% The loss function considered for the model is categorical crossentropy.%
\section{Experimental Results}\vspace{-0.1cm}
To verify the effectiveness of our proposed approach, various experiments using single feature-based and multimodal fusion-based human identification are carried out. The result summary is shown in Tables I and II. Referring to Table I, the first two rows are single-modality based results, whereas the remaining are multi-modality results.\\
\noindent \textbf{(i) Face feature based human recognition:}
Training of facial features separately on each angle 0\textdegree, 45\textdegree and 90\textdegree ~with custom parameters and hyperparameter tuning using Optuna~\cite{akiba2019optuna} produce accuracies up to 65\%, 80\%, and 85\%, respectively (Refer Table II). The overall accuracy of the face model incorporating all orientations is 80\% (Refer Table I).\\
% \break
\textbf{(ii) Gait feature based human recognition:} 
Training of the gait features across three view-points 
% with hyperparameter tuning using Optuna~\cite{akiba2019optuna} 
produces accuracies 75\%, 60\%, and 55\%, respectively (Refer Table II). Referring to Table I, the overall accuracy of the gait model across viewpoints is observed to be 70\%.\\% next para
One noteworthy observation from the aforesaid single-modality based results, is the outperformance of face and gait models in 90\textdegree and in 0\textdegree viewpoints, respectively. This accentuates our initial intuition of the influence viewpoint on feature modalities. To incorporate the best of both modalities in different viewpoints, we facilitate fusion techniques. In particular, four fusion approaches are carried out.\\
% The fusion of both modalities is carried out in the following four ways.\\
\textbf{(iii) Average based fusion:}
Average weight-based fusion incorporates a manual weight input to the face and gait model.
% , thus giving importance to the model according to the scenario. 
For this technique, a weightage of 0.5 is devised on the individual face and gait features, achieving an accuracy of $75\%$ on the test dataset. 
\\
\textbf{(iv) Na\"ive fusion via BLP:}
Bilinear Pooling incorporates the fusion of  both gait and face models, as explained in Section \ref{METHOD} (C). The model achieves 85\%, 75\%, and 85\% viewpoint-wise accuracies, as shown in Table II. The overall fused model results in an accuracy of 80\%. It was observed that compared to the \textit{Average based fusion} model, \textit{Na\"ive fusion with BLP} improves the accuracy by $5\%$.\\
\textbf{(v) Attention Fusion:}
In this model, the keyless attention mechanism (discussed in Sec \ref{attention}) is implemented to obtain global face and gait attention weights \textit{$\Bar{\alpha}$} and \textit{$\Bar{\beta}$}. It is 
further multiplied with the respective features \textit{F} and \textit{G} and is concatenated to obtain a single feature vector. Note that no adaptive fusion strategy is employed in this scheme. This model is able to achieve an overall accuracy of $85\%$ incorporating features over all the viewpoints.
% with a comp existing state-of-the-art technique~\cite{geng2008adaptive}.
\\
\textbf{(vi): Context-aware Adaptive Fusion with attention:}
This strategy incorporates the proposed context-aware adaptive fusion strategy to the attention module, discussed in Sec.\ref{attention}. The viewpoint-wise accuracy attained by this method are 90\%, 80\%, and 90\%, respectively. Referring to Table I, this model outperforms all other models by achieving an overall accuracy of 90\%, highlighting the importance of the context-aware fusion of modalities across the view-points. In terms of the log loss metric, the adaptive fusion strategy has achieved the least value with 0.389 compared to all other models.




% ascribable to the use of attention-based deep neural networks and ad


% On contrary to the hand-crafted features and manual conditions towards imparting relative importance of face and gait algorithm based on their respective level of confidence

% On the fusion of gait and face features in the same dataset, the current benchmarking results are given by~\cite{geng2008adaptive} having an overall average test accuracy of 86.67\% (Refer Table I),  which leveraged hand-crafted features and manual conditions towards imparting relative importance of face and gait algorithm based on their respective level of confidence. 



% which supersedes the existing state-of-the-art model~\cite{geng2008adaptive}.
% The current state-of-the-art technique of Fusion of Gait and Face features are given by~\cite{geng2008adaptive} having an overall cross-validation accuracy of 86.67\% using Machine Learning. It uses a gait recognition algorithm that is adaptable to changes in the angle and distance of the subject, with the addition of a face recognition algorithm that adjusts to variation in the subject's posture. These two algorithms are combined, and the system adjusts the relative importance of each algorithm based on their respective level of confidence.

% \vspace{-2.5mm}
\vskip -3mm

\begin{center}
\centering{\textsc{TABLE I:} Overall result summary of the models}
\end{center}
\newcommand*\pct{\scalebox{.9}{\%}}
\begin{tabular}{l@{\hskip -1.5mm}c@{\hskip -2mm}c@{\hskip 2mm}c}\toprule
\small
    \textbf{Index} & \textbf{Model} & \textbf{Accuracy\((\pct)\)}  & \textbf{Log Loss}\\ \midrule
    (i)&Face feature model & 80 & 0.436\\
    (ii)&Gait feature model & 70 & 1.641\\
     \midrule
     (iii)&Average based fusion  &  75 & 0.779\\
     (iv)&Na\"ive Fusion via BLP  &  80 & 0.519\\
     (v)&Attention Fusion  &  85 & 1.619\\
     (vi)&\textbf{Adaptive Fusion Attention}  &  \textbf{90} & \textbf{0.389}\\
     \midrule
     (vii)&\textbf{\shortstack{Geng, Wang et. al.~\cite{geng2008adaptive}}} & 86.67 & -
     \\\bottomrule
\end{tabular}

 
In order to demonstrate the effectiveness of the proposed \textit{adaptive fusion of gait and face with attention} algorithm, a comparative analysis against the state-of-the-art result on CASIA-A dataset is carried out. The experimental result in Table I (vii) shows that our adaptive fusion result (90\%) outperforms the state-of-the-art performance reported by Geng et al.~\cite{geng2008adaptive}, which had an overall average test accuracy of 86.67\%. Our attention fusion model is also found to be achieving competitive result (85\%) with the state-of-the-art result. These results clearly highlight the potential of our proposed attention-based adaptive multimodal fusion using deep neural networks, in contrary to their
classical hand-crafted features and manual condition-based fusion approach.
% In order to validate the view-point-wise performance of the models, angle-wise analysis is also carried out. The results for the same are shown in Table II; 

\vspace{0.3cm}
\begin{center}
\centering{\textsc{TABLE II:} Result of all models conducted angle-wise $0^{\circ}$, $45^{\circ}$, and $90^{\circ}$ with respect to camera}
\end{center}
    % \vskip 3mm
    % \begin{tabular}{c@{\hskip 3mm}|c@{\hskip 2mm}c@{\hskip 2mm}c@{\hskip 2mm}c}\toprule
    %      \textbf{Angle($^{\circ}$)} &  \multicolumn{4}{c}{\textbf{Accuracy\((\pct)\) }}\\
    %      & Face & Gait & Na\"ive Fusion & Adaptive Fusion\\ \midrule
    %      0\textdegree & 65 & 75 & 75 & \underline{85}\\
    %      45\textdegree & 80 & 60 & 75 & \underline{80}\\
    %      90\textdegree & 85 & 55 & 85 & \underline{90}\\ \bottomrule
    % \end{tabular}
 \vspace{-0.3cm}   
\begin{table}[!h]
\centering
\footnotesize
\renewcommand{\arraystretch}{1.2}
\setlength\tabcolsep{5pt}
 \begin{tabular}{|c|c|c|c|c|}
 \hline
\multirow{2}{*}{\textbf{Angle}($^{\circ}$)} & \multicolumn{4}{c|}{\textbf{Accuracy\((\%)\)}} \\\cline{2-5}
& \textbf{Face} & \textbf{Gait} & \textbf{Na\"ive Fusion} & \textbf{Adaptive Fusion}\\\hline
0\textdegree & 65 & 75 & 85 & 90 \\\hline
45\textdegree & 75 & 60 & 75 & 80 \\\hline
90\textdegree & 85 & 55 & 85 & 90 \\\hline
\hline
\end{tabular}
% \vspace{-0.4cm}
\label{tab:table2}
\end{table}


From the view-point-wise performance of the models in Table II, some key interpretations also can be drawn. For 0\textdegree, the gait model surpasses the face model performance, which aligns with our intuition that the model learns gait features better when the subject walks laterally. Similarly, works well when the subject walks towards the camera at 90\textdegree. However, while incorporating adaptive fusion strategy, the best of both modalities are incorporated adaptively based on the context, resulting in high performance irrespective of the view-point. 


% For 45\textdegree, although both models scored less accuracy, it is evident that gait scored more than face due to gait being more visible than face while the subject is walking diagonally with respect to the camera. For 90\textdegree, the face model scored higher than the gait models due to the face feature being more clearly visible than gait as the subject walks towards the camera. 


\begin{figure}[htbp]
    \centering
   \includegraphics[width=0.27\textwidth]{images/corr_plot.png}
    \caption{Confusion matrix of adaptive fusion
attention model on the test set}
    \label{corr_matrix}
\end{figure}   
% \vspace{-2cm}

A visual interpretation of the attention based adaptive fusion model is depicted in Fig. \ref{corr_matrix} in terms of a confusion matrix. A confusion matrix visualizes and summarizes the performance of a classification algorithm. All the three viewpoints together on the test dataset of 20 subjects is depicted in Fig. \ref{corr_matrix}. We can observe that 18 out of 20 subjects are correctly classified, except subject IDs `1' and `13'. The subject ID `1' is most often classified as 14 and the subject ID `13'  as 5 and 20 in most models. Further, we observe that subject 13 being identified correctly by the face model but not in the gait model, which might be  ascribable to the suboptimal learning of the model from the grayscale features of the image.




% of the adaptive fusion attention model incorporating all orientations together on the test dataset of 20 subjects is depicted. Confusion matrix is a performance measurement used to derive metrics like Accuracy, Precision, Recall and ROC-AUC curves. 



% are not classified correctly. We observe that the subject ID `1' is most often classified as 14 and the subject ID `13'  as 5 and 20 in most models. Further, we observe that subject 13 being identified correctly by the face model but not in the gait model, which might be  ascribable to the suboptimal learning of the model from the grayscale features of the image.

% because of model not able to learn much from grayscale features of the image.

\section{Conclusion and Future Works}
In this work, we proposed a multimodal adaptive fusion of the face and gait toward human identification. In particular,  a keyless attention-based deep neural network for learning the attention in the  gait and face videos and a context-aware adaptive fusion strategy to efficiently extract and fuse the features is presented. Based on the observation that single biometric modality results in suboptimal results, various studies leveraging average based, na\"ive fusion, attention fusion and context-aware adaptive fusion were investigated. Results of the proposed attention-based adaptive fusion strategy show superior performance compared to all the other models as well as the state-of-the-art result. Future Improvements can be made by introducing better attention mechanisms such as dense co-attention and spatial-channel attention, as well as advanced fusion mechanisms like tucker fusion, block fusion etc.
% \bibliographystyle{plainnat}
% \vskip -1cm
% \bibliography{citations}
\printbibliography
\end{document}
