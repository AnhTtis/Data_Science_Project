% CVPR 2023 Paper Template
% based on the CVPR template provided by Ming-Ming Cheng (https://github.com/MCG-NKU/CVPR_Template)
% modified and extended by Stefan Roth (stefan.roth@NOSPAMtu-darmstadt.de)

\documentclass[10pt,twocolumn,letterpaper]{article}

%%%%%%%%% PAPER TYPE  - PLEASE UPDATE FOR FINAL VERSION
% \usepackage[review]{cvpr_supp}      % To produce the REVIEW version
\usepackage{cvpr}              % To produce the CAMERA-READY version
%\usepackage[pagenumbers]{cvpr} % To force page numbers, e.g. for an arXiv version

% Include other packages here, before hyperref.
\usepackage{graphicx}
\usepackage{amsmath}
\usepackage{amssymb}
\usepackage{booktabs}

%------ Ours ------%
\usepackage{array,multirow,graphicx}
\usepackage{float}
\usepackage{amsmath}
\usepackage{mathtools}
\usepackage{bbm}
\usepackage{hhline}
\usepackage{boldline}
\usepackage{tabularx}
\usepackage{comment}
\usepackage{color}
\usepackage{amssymb}
\usepackage{booktabs}
\usepackage{hhline}
\usepackage{multicol}
\usepackage{microtype}      % microtypography
\usepackage{xcolor}         % colors
\usepackage{hhline}%#######
\usepackage{bbm}
\usepackage{boldline}
\usepackage{amssymb}
\usepackage{algorithm}
\usepackage{algpseudocode}
% \usepackage{hyperref}
\usepackage{pifont}% http://ctan.org/pkg/pifont
\usepackage{soul} %\st
\newcommand{\cmark}{\ding{51}}%
\newcommand{\xmark}{\ding{55}}%
\usepackage{subcaption}
\usepackage{caption}


% It is strongly recommended to use hyperref, especially for the review version.
% hyperref with option pagebackref eases the reviewers' job.
% Please disable hyperref *only* if you encounter grave issues, e.g. with the
% file validation for the camera-ready version.
%
% If you comment hyperref and then uncomment it, you should delete
% ReviewTempalte.aux before re-running LaTeX.
% (Or just hit 'q' on the first LaTeX run, let it finish, and you
%  should be clear).
\usepackage[pagebackref,breaklinks,colorlinks]{hyperref}
\usepackage{lipsum}

% Support for easy cross-referencing
\usepackage[capitalize]{cleveref}

\crefname{section}{Sec.}{Secs.}
\Crefname{section}{Section}{Sections}
\Crefname{table}{Table}{Tables}
\crefname{table}{Tab.}{Tabs.}


%%%%%%%%% PAPER ID  - PLEASE UPDATE
\def\cvprPaperID{6761} % *** Enter the CVPR Paper ID here
\def\confName{CVPR}
\def\confYear{2023}

\newcommand{\SE}[1]
{
	\textcolor{red}{\bfseries{{SE\checkmark#1}}}
}

\newcommand{\WJ}[1]
{
	\textcolor{green}{\bfseries{{WJ\checkmark#1}}}
}

\newcommand{\SU}[1]{
        \textcolor{blue}{\bfseries{SU\checkmark#1}}
}

\newcommand{\DC}[1]{
        \textcolor{blue}{\bfseries{DC\checkmark#1}}
}

\newcommand{\Skip}[1]{}

\begin{document}


%%%%%%%%% TITLE - PLEASE UPDATE
% \title{Cross-Modal Conditioning: Query-Dependent DETR \\ for Moment Retrieval and Highlight Detection}
% \title{
% Appendix for 
% ``Query-Dependent Video Representation \\ for Moment Retrieval and Highlight Detection''
% }
\title{
Appendix for 
``\textit{Query-Dependent Video Representation \\ for Moment Retrieval and Highlight Detection}''
}

\author{
WonJun Moon$^{1,\ast}$, Sangeek Hyun$^{1,\ast}$, SangUk Park$^{2}$, Dongchan Park$^{2}$, Jae-Pil Heo$^{1,\star}$\\
$^{1}$Sungkyunkwan University, $^{2}$Pyler\\
{\tt\small \{wjun0830, hsi1032, jaepilheo\}@g.skku.edu,} {\tt\small \{psycoder, cto\}@pyler.tech}
}

\maketitle
\newcommand{\mc}[3]{\multicolumn{#1}{#2}{#3}}




% \begingroup
% \setlength{\tabcolsep}{7pt} % Default value: 6pt
% \renewcommand{\arraystretch}{1} % Default value: 1
% \begin{table*}[h!]
% 	\centering
% 	{\small 
% 	\caption{Performance comparison on QVHighlights \textit{test} split. V and A in the Src column denotes video and audio, respectively, which are the modalities of the source. Methods with 'w/ PT' indicates the inclusion of the pre-training phase with captions from automatic speech recognition. Our experiments are averaged over five runs and `$\pm$' denotes the standard deviation.}
% 	\label{table_QVHighlight_PT}
%         \begin{tabular}{l|c|ccccccc}
%         \hlineB{2.5}
%         \multicolumn{1}{c}{\multirow{3}{*}{Method}} & \multirow{3}{*}{Src} & \multicolumn{5}{c}{MR}                                                             & \multicolumn{2}{c}{HD}                        \\ \cline{3-9} 
%         \multicolumn{1}{c}{} & \multicolumn{1}{c|}{} & \multicolumn{2}{c}{R1}          & \multicolumn{3}{c}{mAP}                          & \multicolumn{2}{c}{\textgreater{}= Very Good} \\ \cline{3-9} 
%         \multicolumn{1}{c}{} & \multicolumn{1}{c|}{} & @0.5           & @0.7           & @0.5           & @0.75          & Avg.           & mAP                   & HIT@1                 \\ \hlineB{2.5}
%         Moment-DETR w/ PT~\cite{momentdetr} & V                          & 59.78$_{\pm{0.3}}$          & 40.33$_{\pm{0.5}}$          & 60.51$_{\pm{0.2}}$          & 35.36$_{\pm{0.4}}$          & 36.14$_{\pm{0.3}}$          & 37.43$_{\pm{0.2}}$                 & 60.17$_{\pm{2.7}}$                 \\
%         QD-DETR (\textbf{Ours})                  & V                  & 62.40$_{\pm_{1.1}}$ & 44.98$_{\pm_{0.8}}$ & 62.52$_{\pm_{0.6}}$ & 39.88$_{\pm_{0.7}}$ & 39.86$_{\pm_{0.6}}$ & \textbf{38.94}$_{\pm_{0.4}}$        & \textbf{62.40}$_{\pm_{1.4}}$    \\
%         QD-DETR (\textbf{Ours}) w/ PT & V & \textbf{63.18$_{\pm_{1.0}}$} & \textbf{45.19$_{\pm_{0.7}}$} & \textbf{63.37$_{\pm_{0.7}}$} & \textbf{40.35$_{\pm_{0.8}}$} & \textbf{39.96}$_{\pm_{0.4}}$ & 38.52$_{\pm_{0.1}}$        & \underline{61.91}$_{\pm_{0.5}}$    \\ \hline
%         UMT w/ PT~\cite{umt}            & V+A                       & 60.83          & 43.26          & 57.33          & 39.12          & 38.08          & \textbf{39.12}                 & 62.39                 \\ 
%         QD-DETR (\textbf{Ours})  & V+A & 63.06$_{\pm_{1.0}}$ & 45.10$_{\pm_{0.7}}$ & 63.04$_{\pm_{0.9}}$ & 40.10$_{\pm_{1.0}}$ & 40.19$_{\pm_{0.6}}$ & 39.04$_{\pm_{0.3}}$       & \textbf{62.87$_{\pm_{0.6}}$}    \\
%         QD-DETR (\textbf{Ours}) w/ PT & V+A & \textbf{64.05$_{\pm_{0.2}}$} & \textbf{46.11$_{\pm_{0.2}}$} & \textbf{64.29$_{\pm_{0.4}}$} & \textbf{40.54$_{\pm_{0.4}}$} & \textbf{40.62}$_{\pm_{0.2}}$ & 38.45$_{\pm_{0.2}}$        & 62.27$_{\pm_{0.8}}$    \\ \hlineB{2.5}
%         \end{tabular}
%         }
% \end{table*}
% \endgroup


%%%%%%%%% ABSTRACT

\section{Training Details}
% \SE{} % tau 값 추가 (\tau = 0.5)
% C3D: https://github.com/IsaacChanghau/VSLNet
% VGG + GloVe text embedding, audio: https://github.com/TencentARC/UMT
% SF+c: https://github.com/jayleicn/moment_detr


% % On QVHighlights, we basically follow the settings from Moment-DETR~\cite{momentdetr}.
% % With the video features extracted from pretrained SlowFast and Clip encoder, audio features from Q... ,and text embeddings from Clip, we train QD-DETR for 200 epochs with the learning rate of 1e-4 and weight decay of 1e-4.
% % For the architectural designs, our encoder is composed of 2 cross-attention layers and 2 self-attention layers whereas there are only 2 decoding layers.
% % With the hidden dimension of 256 for the transformers, we use the batch size to 32 and set each $\lambda_{\text{margin}}$, $\lambda_{\text{cont}}$, $\lambda_{L1}$, $\lambda_{\text{gIoU}}$, $\lambda_{\text{CE}}$ and $\lambda_{neg}$ to (margin, contra, neg = 1, span 10, giou = 1, class = 4). 
% % Lastly, our experiments are mostly conducted with a single NVIDIA TITAN RTX GPU.
% % \textcolor{red}{Training details on Charade, TVSUM not done}
% % % 여기 Loss 다 제대로 쓴다음에 람다값 나열
% % 다른 데이터 셋에 관한 것도 나열

% % We basically follow the settings from Moment-DETR~\cite{momentdetr}.
% % For architectural desings, our encoder composes of 2 cross-attention layers and 2 self-attention layers each
% % For architectural designs, both of our transformer encoders compose of 2 attention layers  whereas there are only 2 decoding layers.
% % For architectural designs, we share identical configurations for the entire dataset we tested.
% For the unified configurations across all experiments, our encoder composes of 4 layers of transformer block~(2 cross-attention layers and 2 self-attention layers) whereas there are only 2 layers in the decoder~(For HD, we only use encoding layers).
% We set the hidden dimension of transformers as 256, and use the Adam optimizer with a weight decay of 1e-4.
% Also, the basic setting for balancing parameters is $\lambda_{\text{margin}}=1$, $\lambda_{\text{cont}}=1$, $\lambda_{L1}=10$, $\lambda_{\text{gIoU}}=1$, $\lambda_{\text{CE}}=4$ and $\lambda_{\text{neg}}=1$, unless otherwise mentioned.

% % % V1
% For other training details on QVHighlight, we use video features extracted from both pretrained SlowFast~\cite{slowfast}~(SF) and CLIP encoder~\cite{CLIP}, and text embeddings from CLIP, following the Moment-DETR.
% We train QD-DETR for 200 epochs with a batch size of 32 and a learning rate of 1e-4.
% For the Charades-STA dataset, we utilize official VGG~\cite{VGG} features with GloVe~\cite{pennington2014glove} text embedding.
% To compare with additional baselines, we also test our model on pretrained C3D~\cite{C3D} and SlowFast for video features with CLIP text embedding.
% We train ours for 100 epochs with a batch size of 8 and a learning rate of 1e-4.
% Besides, we set $\lambda_{\text{margin}}$, $\lambda_{\text{cont}}$ and $\lambda_{\text{neg}}$ as 4 for this dataset.
% Lastly, for the TVSum dataset, we use I3D~\cite{I3D} features pretrained on Kinetics-400~\cite{kay2017kinetics} as a visual one, and CLIP for the text embedding.
% We train our model for 2000 epochs with a batch size of 4 and a learning rate of 1e-3.
% Additionally, we use PANN~\cite{kong2020panns} model trained on AudioSet~\cite{gemmeke2017audio} to extract audio features for experiments with audio modality.

In this section, we elaborate on the implementation details and hyperparameters used for experiments in the main manuscript.
To unify configurations across all experiments, our encoder composes of 4 layers of transformer block~(2 cross-attention layers and 2 self-attention layers) whereas there are only 2 layers in the decoder~(For HD dataset, i.e., TVSum, we only use encoding layers).
We set the hidden dimension of transformers as 256, and use the Adam optimizer with a weight decay of 1e-4.
Besides, we set the temperature of a scaling parameter $\tau$ for contrastive loss as 0.5 for all experiments.
Loss balancing parameters are $\lambda_{\text{margin}}=1$, $\lambda_{\text{cont}}=1$, $\lambda_{L1}=10$, $\lambda_{\text{gIoU}}=1$, $\lambda_{\text{CE}}=4$ and $\lambda_{\text{neg}}=1$, unless otherwise mentioned.
Additionally, we use the PANN~\cite{kong2020panns} model trained on AudioSet~\cite{gemmeke2017audio} to extract audio features\footnotemark[1] for experiments with the audio modality.

Other configurations are described as follows:
% Other dataset dependent configurations are described as follows:

\noindent\textbf{QVHighlight.}
We use video features extracted from both pretrained SlowFast~\cite{slowfast}~(SF) and CLIP encoder~\cite{CLIP}, and text embeddings from CLIP, following the Moment-DETR.
We train QD-DETR for 200 epochs with a batch size of 32 and a learning rate of 1e-4.

\noindent\textbf{Charades-STA.}
We utilize official VGG~\cite{VGG} features with GloVe~\cite{pennington2014glove} text embedding.
To compare with additional baselines, we also test our model on pretrained C3D~\cite{C3D}, SlowFast and CLIP for video features with CLIP text embedding.
Specifically, we utilize pre-extracted features provided by other baselines repositories: UMT\footnote{https://github.com/TencentARC/UMT}, VSLNet\footnote{https://github.com/IsaacChanghau/VSLNet} and Moment-DETR\footnote{https://github.com/jayleicn/moment\_detr}.
We train ours for 100 epochs with a batch size of 8 and a learning rate of 1e-4.

\noindent\textbf{TVSum.}
I3D~\cite{I3D} features pretrained on Kinetics-400~\cite{kay2017kinetics} are utilized as a visual one, and CLIP features are used for the text embedding.
Following the most recent work~\cite{umt}, we train our model for 2000 epochs with a learning rate of 1e-3. The batch size is set to 4.
\section{Further study on model performance on varying lengths of the query.}
% version1
% As discussed in the limitation, QD-DETR may highly depend on the quality of provided ground truth text descriptions.
% And as we think the queries with longer lengths may have a higher chance of including noisy texts, we divide the validation set into 3 groups each with long-, medium-, and short-length queries, and report the query-length-wise performances of QD-DETR in \cref{table_CATE_baselines}.
% As shown, QD-DETR works well regardless of the query length, showing [36.7, 28.0, 26.3\%] and [7.3, 11.8, 11.1\%] improvements in mAP each for MR and HD with [Short, Medium, Long] queries.
% This study implies that while irrelevant text descriptions for video contexts can degrade the effectiveness of QD-DETR, QD-DETR is robust for meaningless words that are commonly present in text queries.

% version2
As discussed in the limitation, the performance of QD-DETR may depend on the quality of provided ground truth text descriptions.
Yet, this does not imply the QD-DETR's vulnerability against commonly used meaningless words in text descriptions.
As we think the queries with longer lengths may have a higher chance of including noisy texts, we divide the validation set into 3 groups each with long-, medium-, and short-length queries, and report the query-length-wise performances of QD-DETR in \cref{table_CATE_baselines}.
As shown, QD-DETR works well regardless of the query length, showing [36.7, 28.0, 26.3\%] and [7.3, 11.8, 11.1\%] improvements in mAP each for MR and HD with [Short, Medium, Long] queries.
This study implies that while irrelevant~(wrong) text descriptions for video contexts can degrade the effectiveness of QD-DETR, QD-DETR is robust against meaningless words that are commonly present in text queries.
% SE: 말하는 내용은 동일한 것 같네요. version2로 가는 것도 좋아보입니다.


% Also, we believe that the quality of query cannot be solely determined by \textbf{query length} as QD-DETR outperforms the baseline in all cases with different query lengths, as shown in Tab.~\ref{table_CATE_baselines} (Bottom).


\begingroup
\setlength{\tabcolsep}{4.3pt} % Default value: 6pt
\renewcommand{\arraystretch}{1} % Default value: 1
\begin{table}[t]
	\centering
	{\scriptsize
	\caption{Experimental results on QVHighlights.}
	\label{table_CATE_baselines}
        \vspace{-0.3cm}
        \begin{tabular}{cc|ccccccc}
        \hlineB{2.5}
        \multicolumn{2}{c|}{\multirow{3}{*}{}} &  \multicolumn{5}{c}{MR} & \multicolumn{2}{c}{HD}                        \\ \cline{3-9} 
        \multicolumn{2}{c|}{} & \multicolumn{2}{c}{R1}          & \multicolumn{3}{c}{mAP}                          & \multicolumn{2}{c}{\textgreater{}= Very Good} \\ \cline{3-9} 
        \multicolumn{2}{c|}{}  & @0.5           & @0.7           & @0.5           & @0.75          & Avg.           & mAP                   & HIT@1                 \\ \hlineB{2.5}
        \multicolumn{9}{c}{Performances with respect to query length} \\
        \multicolumn{9}{c}{S: \# words $\leq$ 8,\quad  M: 8 $<$ \# words $\leq$ 13, \quad  L: 13 $<$ \# words} \\
        \hline
        \multicolumn{1}{c|}{\multirow{2}{*}{S}}  & M-DETR & 51.82 & 34.49 & 51.48 & 29.48 & 29.43 & 37.11 & 59.27    \\ 
         \multicolumn{1}{c|}{}& QD-DETR & 63.95 & 48.18 & 61.18 & 40.93 & 40.23 & 38.67 & 63.60    \\ \hline
        \multicolumn{1}{c|}{\multirow{2}{*}{M}} & M-DETR & 57.47 & 39.22 & 57.41 & 33.43 & 34.73 & 37.49 & 56.26    \\ 
         \multicolumn{1}{c|}{}& QD-DETR & 65.91 & 51.43 & 65.48 & 45.54 & 44.46 & 40.07 & 62.90    \\ \hline
        \multicolumn{1}{c|}{\multirow{2}{*}{L}} & M-DETR & 49.35 & 32.90 & 52.89 & 29.14 & 30.54 & 35.95 & 55.16    \\ 
         \multicolumn{1}{c|}{}& QD-DETR & 57.42 & 40.32 & 61.03 & 37.67 & 38.56 & 39.24 & 61.29    \\ \hlineB{2.5} 
        % \multicolumn{2}{l|}{S: \qquad \# words $\leq$ 8} &  \multicolumn{3}{c}{MR} & \multicolumn{2}{c}{HD}                        \\ \hline
        % \multicolumn{2}{l|}{M: 8 $<$ \# words $\leq$ 13} & \multicolumn{2}{c}{R1}          & mAP        & \multicolumn{2}{c}{\textgreater{}= Very Good} \\ \hline
        % \multicolumn{2}{l|}{L: 13 $<$ \# words}  & @0.5           & @0.7           & Avg.           & mAP                   & HIT@1                 \\ \hline
        \end{tabular}
        }
\end{table}
\endgroup

%%%%%%%%% BODY TEXT


%%%%%%%%% REFERENCES
{\small
\bibliographystyle{ieee_fullname}
\bibliography{egbib}
}

\end{document}
