\section{Action Language and Causality}
\label{sec:Action_language}

    This section introduces the formal aspects of the action language proposed by~\cite{sarmiento_action_2022-1} and then briefly describes what is considered to be an actual cause in this formalism. For more details refer to~\cite{sarmiento_action_2023}.

    \subsection{Syntax and Semantics}
    \label{sec:Modele}
        
        The purpose of the action language introduced by~\cite{sarmiento_action_2022-1} is to determine the evolution of the world given a set of actions corresponding to deliberate choices of the agent. These actions might trigger some chain reaction through external events. Therefore, in order to have a complete knowledge of the evolution of the world, \cite{sarmiento_action_2022-1} keep track of both: the evolution of the states of the world and the occurrence of events. 
        %This formalism works on a decomposition of the world into two sets: 
        Hence, we denote by $\mathbb{F}$ the set of %corresponding to 
        variables describing the state of the world, more precisely \emph{ground fluents} representing time-varying properties, and by $\mathbb{E}$ 
        %representing 
        the set of variables describing transitions, more precisely \emph{ground events} that modify fluents.
        
        A \emph{fluent literal} is either a fluent~$f\in \mathbb{F}$, or its negation~$\neg f$. The set of fluent literals in~$\mathbb{F}$ is denoted by~$Lit_{\mathbb{F}}$, i.e.~$Lit_{\mathbb{F}} = \mathbb{F}\cup\left\{\neg f \mid f\in \mathbb{F} \right\}$. The complement of a fluent literal $l$ is defined as~$\overline{l}=\neg f$ if~$l=f$ or~$\overline{l}=f$ if~$l=\neg f$.
        
        \begin{definition}[state]\label{def:state}
            A set $L\subseteq Lit_{\mathbb{F}}$ is a \emph{state} if it is:
                \begin{itemize}
                    \item Coherent: $\forall l\in L, \overline{l}\not\in L$;
                    \item Complete: $\forall f\in \mathbb{F}, f\in L$ or $\neg f \in L$.
                \end{itemize}
        \end{definition}
        
        A state 
        %is thus a set~$L\subseteq Lit_{\mathbb{F}}$ which 
        thus gives the value of each of the fluents describing the world. Time is modelled linearly and in a discrete way to associate a state~$S(t)$ to each time point~$t$ of a set~$\mathbb{T} = \left\{-1,0,\dots,N\right\}$. $S(0)$ is the \emph{initial state}. Using a bounded past formalisation, all states before $t=0$ are gathered in a state~$S(-1) =\mathbb{F} \setminus S(0)$.
        %$S(-1) = \overline{S(0)}$. \mj{utile de dire le overline~?}\Camu{Je pense que oui car c'est le complémentaire} \Isa{on n'utilise plus la notation ? dans ce cas on peut écrire directement $\mathbb{F} \setminus S(0)$  ?}
        
        An event~$e\in\mathbb{E}$ is an atomic formula. Each event is characterised by three elements: preconditions and  triggering conditions give conditions that must be satisfied by a state~$S$  for the event to be triggered (their difference is detailed later in the section); effects indicate the changes to the state that are expected if the event occurs. Note the deliberate use of the term `expected' as an event may have fewer effects than those formalised.
        
        %give conditions that must be satisfied by a state~$S$ for the event to be triggered; triggering conditions give all the conditions that must be satisfied at time point~$t$ for the event to be triggered (the difference with preconditions is detailed later on); effects indicate the changes to the state that are expected if the event occurs. Note the deliberate use of the term ``expected'' as an event may have fewer effects than those formalised.
        
        The preconditions and effects are represented as formulas of the languages~$\mathcal{P}\Coloneqq l|\psi_1 \wedge \psi_2|\psi_1 \vee \psi_2$ and~$\mathcal{E}\Coloneqq l|\varphi_1 \wedge \varphi_2$, respectively. The functions which associate preconditions, triggering conditions and effects with each event are respectively defined as: $pre: \mathbb{E} \rightarrow \mathcal{P}$, $tri: \mathbb{E} \rightarrow \mathcal{P}$, and~$e\ff: \mathbb{E} \rightarrow \mathcal{E}$. 
        %Two disjoint sets form a partition of~$\mathbb{E}$: 
        $\mathbb{E}$ is partitioned into two  disjoint sets: 
        $\mathbb{A}$ contains the actions carried out by an agent and thus subjected to their volition; $\mathbb{U}$ contains the exogenous events which are triggered as soon as all the~$pre$ conditions are fulfilled, therefore without the need for an agent to perform them. Thus, for exogenous events~$pre$ and~$tri$ are the same. By contrast, for actions, $tri$ conditions necessarily include $pre$ conditions but those are not sufficient: the~$tri$ conditions of an action also include the volition of the agent or some kind of manipulation by another agent.
        
        The set of all events which occur at time point~$t$ is denoted by~$E(t)$. Allowing concurrency of events (meaning that more than one event can occur at each time point) is one of the main advantages of this action language.
                
        %In summary, this is 
        These definitions lead to a classical transition system: $E(t)$ 
        %is what 
        generates the transition between the states~$S(t)$ and~$S(t+1)$. Thus, the states follow one another as events occur, simulating the evolution of the world.
        
        With a bounded past formalisation, events that occurred before~$t=0$ must be represented in order to obtain causal results that are consistent with the philosophical conception of causality.
        %Indeed, it might happen that one of the reasons why a formula of $\mathcal{P}$ is true is that one of its fluent literals was true in the initial state and remains so.
        Thus, for each fluent literal~$l\in S(0)$ an event~$ini_l\in\mathbb{E}$ is introduced, such that~$e\ff(ini_l)=l$. Then, $E(-1)=\left\{ini_l,l\in S(0)\right\}$ which satisfies~${e\ff(E(-1))=S(0)}$.
        
        %The occurrence of events~$(e,t)\in\mathbb{E}\times\mathbb{T}$ and~$(e',t)\in\mathbb{E}\times\mathbb{T}$ in the state~$S(t)$ is said to be \emph{interfering} if the set~$e\ff(e)\cup e\ff(e')$ is not coherent according to Definition~\ref{def:state}. 
        To solve potential conflicts or to prioritise between events, a strict partial order~$\succ_{\mathbb{E}}$ is introduced, which ensures the triggering primacy of one event over another.
        
        \begin{definition}[context $\kappa$]\label{def:context}
            The \emph{context}, denoted as $\kappa$, is the octuple $(\mathbb{E},\mathbb{F},pre, tri,e\ff,S(0),\succ_{\mathbb{E}},\mathbb{T})$, where $\mathbb{E}$, $\mathbb{F}$, $pre$, $tri$, $e\ff$, $S(0)$, $\succ_{\mathbb{E}}$, and $\mathbb{T}$ are as defined above.
        \end{definition}
        
%\Isa{Rq : dans tout ce qui suit, on mélange un peu les formules du langage formel (qui est limité, sans négation autre que sur les litéraux, sans quantificateurs), avec des expressions "en anglais". Ce n'est pas gênant pour la compréhension, mais peut-être faut-il indiquer ce raccourci ? par exemple toutes les def et résultats sont donnés en FOL standard }
%\mj{je ne trouve pas que ce soit vraiment gênant + surtout, problème de place...} 
        
        \begin{definition}[valid execution]\label{def:semantics}
            An \emph{execution} is a sequence $E(-1),S(0),E(0),\dots,E(N),S(N+1)$. Such an execution is \emph{valid} given~$\kappa$ if~$\forall t\in\mathbb{T}$:
            \begin{enumerate}
                \item $S(t)\subseteq Lit_{\mathbb{F}}$ is a state according to Definition~\ref{def:state}.
                \item $E(t)\subseteq\mathbb{E}$ verifies:
                \begin{enumerate}
                    \item $\forall e\in E(t)$, $S(t) \models pre(e)$;
                    \item $\nexists (e,e')\in E(t)^2,~e\succ_{\mathbb{E}}e'$;
                    \item $\forall e\in \mathbb{E}$ such that $S(t) \models tri(e)$,\\
                    \phantom{$\forall e\in \mathbb{E}$} $e \in E(t)$ or ${\exists e'\in E(t),}$ $e'\succ_{\mathbb{E}}e$;
                \end{enumerate}
                \item $S(t+1)=\left\{l\in S(t),\forall e\in E(t),\overline{l}\not\in e\ff(e)\right\}\cup$ \\ \phantom{$S(t+1)=$} $\left\{l\in Lit_{\mathbb{F}},\exists e\in E(t),l\in e\ff(e)\right\}$.
            \end{enumerate}
        \end{definition}
        
        There is potentially more than one valid execution for a given context~$\kappa$. In fact, there is no specification of when actions are performed in~$\kappa$. Adding a set of timed actions~$\sigma\subseteq\mathbb{A}\times\mathbb{T}$ which models volition of agents as an input, called scenario, leads to a unique valid execution. From this unique execution the event trace and state trace we are interested in, denoted by~$\tau_{\sigma,\kappa}^e$ and~$\tau_{\sigma,\kappa}^s$, respectively, can be extracted.
        
        \begin{definition}[traces $\tau_{\sigma,\kappa}^e$ and $\tau_{\sigma,\kappa}^s$]\label{def:traces}
            Given a scenario~$\sigma$ and a context~$\kappa$, the \emph{event trace}~$\tau_{\sigma,\kappa}^e$  is the sequence of events~$E(-1),E(0),\dots,E(N)$ from the execution which is valid given~$\kappa$, such that: $\forall t\in\mathbb{T}, {\forall e\in E(t)}, {e \in \mathbb{A}} \Leftrightarrow (e,t) \in \sigma$.
            The \emph{state trace}~$\tau_{\sigma,\kappa}^s$ is the sequence of states $S(0),S(1),\dots,S(N+1)$ corresponding to~$\tau_{\sigma,\kappa}^e$.
        \end{definition}
    
    
    \subsection{Actual Causality}
    \label{sec:causality}
        
        The actual causation definition proposed by~\cite{sarmiento_action_2022-1} is an action language suitable formalisation of Wright's NESS test. Introduced by~\cite{wright_causation_1985}, this test states that: `A particular condition was a cause of a specific consequence if and only if it was a necessary element of a set of antecedent actual conditions that was sufficient for the occurrence of the consequence.'
        
        % This test, which subordinates necessity to sufficiency, is an approach that deals with the most debated cases of causality, the cases of overdetermination~\cite{baumgartner_regularity_2013,wright_causation_1985,wright_causation_1988}---situations in which two events would have been sufficient to cause an effect in the absence of the other. Additionally, the obtained causal inquiry has the advantage to be factual and independent of subjective questions of responsibility~\cite{wright_causation_1985}.
        A causal relation links a cause to an effect. Since action languages represent the evolution of the world as a succession of states produced by the occurrence of events, states are introduced between events. Therefore, in addition to the actual causality relation that links two occurrences of events, as commonly accepted by philosophers, it is necessary to define causal relations where causes are occurrences of events and effects are formulas of the language~$\mathcal{P}$ that are true at a given time. These intermediate relations are established on the basis of Wright's NESS test of causation. In order to give an actual causality definition suitable for action languages, three causal relations are introduced by~\cite{sarmiento_action_2022-1}: (i)~\emph{Direct NESS-causes} give essential information about causal relations by looking at the effects that the occurrence of an event has actually had, which are not necessarily the same as those expected. Direct NESS-causes relate occurrences of events and formulas of $\mathcal{P}$ being true at a specific time point. However, the set of direct NESS-causes of a formula of $\mathcal{P}$ may include exogenous events that are not necessarily relevant. It is therefore essential to establish a causal chain by going back in time in order to find the set of actions that led to the formula truthfulness. (ii)~\emph{NESS-causes} allow for such a causal chain to be found. If we denote by~$\psi\in\mathcal{P}$ the formula true at $t_\psi$ we are interested in, and~$C$ the set of direct NESS-causes of~$(\psi,t_\psi)$, finding the NESS-causes means finding what causes~$(tri(C),t)$ necessarily, where~$t<t_\psi$. Note that direct NESS-causes are by definition a special case of NESS-causes. (iii)~The occurrence of a first event~$e$ is considered an \emph{actual cause} of the occurrence of a second event~$e'$ if and only if the occurrence of~$e$ is a NESS-cause of the triggering of~$e'$. From this we can deduce that, if the occurrence~$(e',t_2)$ is a direct NESS-cause of~$(\psi,t_3)$ and the occurrence~$(e,t_1)$ is an actual cause of~$(e',t_2)$, with~$t_1<t_2<t_3$, then the occurrence~$(e,t_1)$ is a NESS-cause of~$(\psi,t_3)$. These three causal relations are illustrated using Example~\ref{ex:IRM_ou_radio} in Section~\ref{sec:discu_causalite}.
        