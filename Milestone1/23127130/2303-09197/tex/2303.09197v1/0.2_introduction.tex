\section{Introduction}
    
    The abstract argumentation framework (AAF), first introduced in~\cite{dung_acceptability_1995}, provides a suitable framework for representing and reasoning about contradicting information. % through arguments.
    It makes it possible to find sets of arguments that can be accepted together and provides explanations on why such sets have been accepted or not. Thus, AAF 
%    \Camu{falsely(?) seem to} \mj{, oui, mais un peu trop agressif, je trouve : changer le mot dialogue en fait : dialogue -> debates }
    provides convenient tools to model and reason about debates. 
    % \Camu{Indeed (irait avec modif précèdente)} 
    However, it is a static framework which does not include a notion of temporality that seems crucial for modelling dialogues. To address this issue, considering an AAF for each time step could be an option, but it might be expensive to create and compare AAFs at every time step.
    % This approach also allows modeling quite naturally interactions between agents. In this context, we are particularly interested in two notions: the modeling capacities of the framework and the generation of explanations adapted to humans. This last point has recently been the subject of particular attention in argumentation~\cite{cyras_survey}. Nevertheless, although this formalism is very well adapted to the modeling of interactions, it is a static framework which, by default, does not include a notion of temporality. \Camu{}It seems possible to artificially add it by considering an argumentation graph at each time step, \Isa{or an time-based attributed graph?}. However, this is not the approach we have chosen here \Isa{because...?}.\Camu{}

    On the other hand, action languages offer tools to reason about action and change and have been naturally conceived to include the notion of time. The action language introduced by~\cite{sarmiento_action_2022-1}
    %is one of them. It 
    has been designed to determine the evolution of the world given a set of actions corresponding to deliberate choices of the agent, the occurrence of which can trigger a chain reaction through external events. We choose this action language for three reasons. First, it allows for concurrency of events. Other languages also offering this advantage, such as~$\mathcal{C}$~\cite{giunchiglia_action_1998} or PDDL+~\cite{fox_modelling_2006},
    %also offer this advantage. However, their semantics 
    are adapted to non-deterministic or durative actions, which increases complexity and is not useful in our framework. Secondly, there exists a definition of actual causality that is suitable for this action language. 
    % According to~\cite{miller_explanation_2018}, causality is essential for the generation of explanations. 
    Finally,~\cite{sarmiento_action_2023} propose a  sound and complete translation into ASP. 
    %In this paper, we 
    We propose to take advantage of these properties to  study the causal relations in a dialogue, paving the way for the search of explanations.
    
    This paper is structured as follows. Section \ref{sec:AAF} briefly recalls the principles of  %introduces~\cite{dung_acceptability_1995}'s formalism of
    abstract argumentation. Section~\ref{sec:Action_language} 
    %provides a description of 
    describes the chosen action language and the actual causality definition suitable for it. Section~\ref{sec:translation} proposes the main contributions of this paper: a formalisation of acyclic abstract argumentation graphs into an action language, with its corresponding implementation in ASP.
    Section~\ref{sec:formalProp} establishes its formal properties, 
    %of this transformation such as 
    including its soundness and completeness, as well as the relevance of the temporality inclusion. Section~\ref{sec:discussion} illustrates the exploitation of the proposed formalisation to get enriched information, as graphical representations and causal relations. Section~\ref{sec:conclusion}  concludes the paper. 
    %and opens some perspectives on the generation of explanations. \Isa{on peut ne pas mentionner la section 7 dans le plan s'il faut gagner deux lignes}