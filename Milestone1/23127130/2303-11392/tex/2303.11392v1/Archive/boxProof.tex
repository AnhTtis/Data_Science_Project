\documentclass[10pt]{article}
\usepackage{amsmath}
\usepackage{amsthm}
\usepackage{amsfonts}
\usepackage{dsfont}
\usepackage{amssymb}
\usepackage{latexsym}
\usepackage{tensor}
%\usepackage{epsfig}
\usepackage{graphicx}
\usepackage{tikz}
\usetikzlibrary{cd}
%\usepackage[dvips]{graphicx}
\graphicspath{ {images/} }

\usepackage[matrix,tips,graph,curve]{xy}

\newcommand{\mnote}[1]{${}^*$\marginpar{\footnotesize ${}^*$#1}}
\linespread{1.065}

\makeatletter

\setlength\@tempdima  {5.5in}
\addtolength\@tempdima {-\textwidth}
\addtolength\hoffset{-0.5\@tempdima}
\setlength{\textwidth}{5.5in}
\setlength{\textheight}{8.75in}
\addtolength\voffset{-0.625in}

\makeatother

\makeatletter 
\@addtoreset{equation}{section}
\makeatother


\renewcommand{\theequation}{\thesection.\arabic{equation}}

\theoremstyle{plain}
\newtheorem{theorem}[equation]{Theorem}
\newtheorem{corollary}[equation]{Corollary}
\newtheorem{lemma}[equation]{Lemma}
\newtheorem{proposition}[equation]{Proposition}
\newtheorem{conjecture}[equation]{Conjecture}
\newtheorem{fact}[equation]{Fact}
\newtheorem{facts}[equation]{Facts}
\newtheorem*{theoremA}{Theorem A}
\newtheorem*{theoremB}{Theorem B}
\newtheorem*{theoremC}{Theorem C}
\newtheorem*{theoremD}{Theorem D}
\newtheorem*{theoremE}{Theorem E}
\newtheorem*{theoremF}{Theorem F}
\newtheorem*{theoremG}{Theorem G}
\newtheorem*{theoremH}{Theorem H}

\theoremstyle{definition}
\newtheorem{definition}[equation]{Definition}
\newtheorem{definitions}[equation]{Definitions}
%\theoremstyle{remark}

\newtheorem{remark}[equation]{Remark}
\newtheorem{remarks}[equation]{Remarks}
\newtheorem{exercise}[equation]{Exercise}
\newtheorem{example}[equation]{Example}
\newtheorem{examples}[equation]{Examples}
\newtheorem{notation}[equation]{Notation}
\newtheorem{question}[equation]{Question}
\newtheorem{assumption}[equation]{Assumption}
\newtheorem*{claim}{Claim}
\newtheorem{answer}[equation]{Answer}
%%%%%% letters %%%%

\newcommand{\fA}{\mathfrak{A}}
\newcommand{\fB}{\mathfrak{B}}
\newcommand{\fC}{\mathfrak{C}}
\newcommand{\fD}{\mathfrak{D}}
\newcommand{\fE}{\mathfrak{E}}
\newcommand{\fF}{\mathfrak{F}}
\newcommand{\fG}{\mathfrak{G}}
\newcommand{\fH}{\mathfrak{H}}
\newcommand{\fI}{\mathfrak{I}}
\newcommand{\fJ}{\mathfrak{J}}
\newcommand{\fK}{\mathfrak{K}}
\newcommand{\fL}{\mathfrak{L}}
\newcommand{\fM}{\mathfrak{M}}
\newcommand{\fN}{\mathfrak{N}}
\newcommand{\fO}{\mathfrak{O}}
\newcommand{\fP}{\mathfrak{P}}
\newcommand{\fQ}{\mathfrak{Q}}
\newcommand{\fR}{\mathfrak{R}}
\newcommand{\fS}{\mathfrak{S}}
\newcommand{\fT}{\mathfrak{T}}
\newcommand{\fU}{\mathfrak{U}}
\newcommand{\fV}{\mathfrak{V}}
\newcommand{\fW}{\mathfrak{W}}
\newcommand{\fX}{\mathfrak{X}}
\newcommand{\fY}{\mathfrak{Y}}
\newcommand{\fZ}{\mathfrak{Z}}

\newcommand{\fa}{\mathfrak{a}}
\newcommand{\fb}{\mathfrak{b}}
\newcommand{\fc}{\mathfrak{c}}
\newcommand{\fd}{\mathfrak{d}}
\newcommand{\fe}{\mathfrak{e}}
\newcommand{\ff}{\mathfrak{f}}
\newcommand{\fg}{\mathfrak{g}}
\newcommand{\fh}{\mathfrak{h}}
%\newcommand{\fi}{\mathfrak{i}}

\newcommand{\fj}{\mathfrak{j}}
\newcommand{\fk}{\mathfrak{k}}
\newcommand{\fl}{\mathfrak{l}}
\newcommand{\fm}{\mathfrak{m}}
\newcommand{\fn}{\mathfrak{n}}
\newcommand{\fo}{\mathfrak{o}}
\newcommand{\fp}{\mathfrak{p}}
\newcommand{\fq}{\mathfrak{q}}
\newcommand{\fr}{\mathfrak{r}}
\newcommand{\fs}{\mathfrak{s}}
\newcommand{\ft}{\mathfrak{t}}
\newcommand{\fu}{\mathfrak{u}}
\newcommand{\fv}{\mathfrak{v}}
\newcommand{\fw}{\mathfrak{w}}
\newcommand{\fx}{\mathfrak{x}}
\newcommand{\fy}{\mathfrak{y}}
\newcommand{\fz}{\mathfrak{z}}

\newcommand{\bi}{\textbf{i}}
\newcommand{\bj}{\textbf{j}}
\newcommand{\bk}{\textbf{k}}

\newcommand{\sA}{\mathcal{A}\,}
\newcommand{\sB}{\mathcal{B}\,}
\newcommand{\sC}{\mathcal{C}}
\newcommand{\sD}{\mathcal{D}\,}
\newcommand{\sE}{\mathcal{E}\,}
\newcommand{\sF}{\mathcal{F}\,}
\newcommand{\sG}{\mathcal{G}\,}
\newcommand{\sH}{\mathcal{H}}
\newcommand{\sI}{\mathcal{I}\,}
\newcommand{\sJ}{\mathcal{J}\,}
\newcommand{\sK}{\mathcal{K}\,}
\newcommand{\sL}{\mathcal{L}\,}
\newcommand{\sM}{\mathcal{M}\,}
\newcommand{\sN}{\mathcal{N}}
\newcommand{\sO}{\mathcal{O}}
\newcommand{\sP}{\mathcal{P}\,}
\newcommand{\sQ}{\mathcal{Q}\,}
\newcommand{\sR}{\mathcal{R}}
\newcommand{\sS}{\mathcal{S}}
\newcommand{\sT}{\mathcal{T}\,}
\newcommand{\sU}{\mathcal{U}\,}
\newcommand{\sV}{\mathcal{V}\,}
\newcommand{\sW}{\mathcal{W}\,}
\newcommand{\sX}{\mathcal{X}\,}
\newcommand{\sY}{\mathcal{Y}\,}
\newcommand{\sZ}{\mathcal{Z}\,}

\newcommand{\IA}{\mathbb{A}}
\newcommand{\IB}{\mathbb{B}}
\newcommand{\IC}{\mathbb{C}}
\newcommand{\ID}{\mathbb{D}}
\newcommand{\IE}{\mathbb{E}}
\newcommand{\IF}{\mathbb{F}}
\newcommand{\IG}{\mathbb{G}}
\newcommand{\IH}{\mathbb{H}}
\newcommand{\II}{\mathbb{I}}
\newcommand{\IK}{\mathbb{K}}
\newcommand{\IL}{\mathbb{L}}
\newcommand{\IM}{\mathbb{M}}
\newcommand{\IN}{\mathbb{N}}
\newcommand{\IO}{\mathbb{O}}
\newcommand{\IP}{\mathbb{P}}
\newcommand{\IQ}{\mathbb{Q}}
\newcommand{\IR}{\mathbb{R}}
\newcommand{\IS}{\mathbb{S}}
\newcommand{\IT}{\mathbb{T}}
\newcommand{\IU}{\mathbb{U}}
\newcommand{\IV}{\mathbb{V}}
\newcommand{\IW}{\mathbb{W}}
\newcommand{\IX}{\mathbb{X}}
\newcommand{\IY}{\mathbb{Y}}
\newcommand{\IZ}{\mathbb{Z}}

 \newcommand{\tA}{\mathrm {A}}
 \newcommand{\tB}{\mathrm {B}}
 \newcommand{\tC}{\mathrm {C}}
 \newcommand{\tD}{\mathrm {D}}
 \newcommand{\tE}{\mathrm {E}}
 \newcommand{\tF}{\mathrm {F}}
 \newcommand{\tG}{\mathrm {G}}
 \newcommand{\tH}{\mathrm {H}}
 \newcommand{\tI}{\mathrm {I}}
 \newcommand{\tJ}{\mathrm {J}}
 \newcommand{\tK}{\mathrm {K}}
 \newcommand{\tL}{\mathrm {L}}
 \newcommand{\tM}{\mathrm {M}}
 \newcommand{\tN}{\mathrm {N}}
 \newcommand{\tO}{\mathrm {O}}
 \newcommand{\tP}{\mathrm {P}}
 \newcommand{\tQ}{\mathrm {Q}}
 \newcommand{\tR}{\mathrm {R}}
 \newcommand{\tS}{\mathrm {S}}
 \newcommand{\tT}{\mathrm {T}}
 \newcommand{\tU}{\mathrm {U}}
 \newcommand{\tV}{\mathrm {V}}
 \newcommand{\tW}{\mathrm {W}}
 \newcommand{\tX}{\mathrm {X}}
 \newcommand{\tY}{\mathrm {Y}}
 \newcommand{\tZ}{\mathrm {Z}}
%%%%%%% macros %%%%%

%% my definitions %%%

\newcommand{\End}{\mathrm{End}}
\newcommand{\tr}{\mathrm{tr}}
%\newcommand{\ind}{\mathrm{ind}}

\renewcommand{\index}{\mathrm{index \,}}
\newcommand{\Hom}{\mathrm{Hom}}
\newcommand{\Aut}{\mathrm{Aut}}
\newcommand{\Trace}{\mathrm{Trace}\,}
\newcommand{\Res}{\mathrm{Res}\,}
\newcommand{\rank}{\mathrm{rank}}
%\renewcommand{\dim}{\mathrm{dim}}

\renewcommand{\deg}{\mathrm{deg}}
\newcommand{\spin}{\rm Spin}
\newcommand{\Spin}{\rm Spin}
\newcommand{\erfc}{\rm erfc\,}
\newcommand{\sgn}{{\rm sgn\,}}
\newcommand{\Spec}{\rm Spec\,}
\newcommand{\diag}{\rm diag\,}
\newcommand{\Fix}{\mathrm{Fix}}
\newcommand{\Ker}{\mathrm{Ker \,}}
\newcommand{\Coker}{\mathrm{Coker \,}}
\newcommand{\Sym}{\mathrm{Sym \,}}
\newcommand{\Hess}{\mathrm{Hess \,}}
\newcommand{\grad}{\mathrm{grad \,}}
\newcommand{\Center}{\mathrm{Center}}
\newcommand{\Lie}{\mathrm{Lie}}


\newcommand{\ch}{\rm ch} % Chern Character

\newcommand{\rk}{\rm rk} 
%\renewcommand{\c}{\rm c}  % Chern class

\newcommand{\sign}{\rm sign}
\renewcommand\dim{{\rm dim\,}}
\renewcommand\det{{\rm det\,}}
\newcommand{\dimKrull}{{\rm Krulldim\,}}
\newcommand\Rep{\mathrm{Rep}}
\newcommand\Hilb{\mathrm{Hilb}}
\newcommand\vol{\mathrm{vol}}

\newcommand\QED{\hfill $\Box$} %{\bf QED}} 

\newcommand\Pf{\nonintend{\em Proof. }}


\newcommand\reals{{\mathbb R}} 
\newcommand\complexes{{\mathbb C}}
\renewcommand\Re{\mathrm Re}
\renewcommand\Im{\mathrm Im}
\newcommand\integers{{\mathbb Z}}
\newcommand\quaternions{{\mathbb H}}


\newcommand\iso{{\cong}} 
%\newcommand\tensor{{\otimes}}
\newcommand\Tensor{{\bigotimes}} 
\newcommand\union{\bigcup} 
\newcommand\onehalf{\frac{1}{2}}
%\newcommand\Sym[1]{{Sym^{#1}(\complexes^2)}}

\newcommand\lie[1]{{\mathfrak #1}} 
\renewcommand\fk{\mathfrak{K}}
\newcommand\smooth{\mathcal{C}^{\infty}}
\newcommand\trivial{{\mathbb I}}
\newcommand\widebar{\overline}

%%%%%Delimiters%%%%

\newcommand{\<}{\langle}
\renewcommand{\>}{\rangle}

%\renewcommand{\(}{\left(}
%\renewcommand{\)}{\right)}


%%%% Different kind of derivatives %%%%%

\newcommand{\delbar}{\bar{\partial}}
\newcommand{\pdu}{\frac{\partial}{\partial u}}
%\newcommand{\pd}[1][2]{\frac{\partial #1}{\partial #2}}

%%%%% Arrows %%%%%
%\renewcommand{\ra}{\rightarrow}                   % right arrow
%\newcommand{\lra}{\longrightarrow}              % long right arrow
%\renewcommand{\la}{\leftarrow}                    % left arrow
%\newcommand{\lla}{\longleftarrow}               % long left arrow
%\newcommand{\ua}{\uparrow}                     % long up arrow
%\newcommand{\na}{\nearrow}                      %  NE arrow
%\newcommand{\llra}[1]{\stackrel{#1}{\lra}}      % labeled long right arrow
%\newcommand{\llla}[1]{\stackrel{#1}{\lla}}      % labeled long left arrow
%\newcommand{\lua}[1]{\stackrel{#1}{\ua}}      % labeled  up arrow
%\newcommand{\lna}[1]{\stackrel{#1}{\na}}      % labeled long NE arrow

\newcommand{\into}{\hookrightarrow}
\newcommand{\tto}{\longmapsto}
\def\llra{\longleftrightarrow}

\def\d/{/\mspace{-6.0mu}/}
\newcommand{\git}[3]{#1\d/_{\mspace{-4.0mu}#2}#3}
\newcommand\zetahilb{\zeta_{{\text{Hilb}}}}
\def\Fy{\sF \mspace{-3.0mu} \cdot \mspace{-3.0mu} y}
\def\tv{\tilde{v}}
\def\tw{\tilde{w}}
\def\wt{\widetilde}
\def\wtilde{\widetilde}
\def\what{\widehat}

%%%%%%%%%%%%%%%%%%% Mark's definitions %%%%%%%%%%%%%%%%%%%%

\newcommand{\frakg}{\mbox{\frakturfont g}}
\newcommand{\frakk}{\mbox{\frakturfont k}}
\newcommand{\frakp}{\mbox{\frakturfont p}}
\newcommand{\q}{\mbox{\frakturfont q}}
\newcommand{\frakn}{\mbox{\frakturfont n}}
\newcommand{\frakv}{\mbox{\frakturfont v}}
\newcommand{\fraku}{\mbox{\frakturfont u}}
\newcommand{\frakh}{\mbox{\frakturfont h}}
\newcommand{\frakm}{\mbox{\frakturfont m}}
\newcommand{\frakt}{\mbox{\frakturfont t}}
\newcommand{\G}{\Gamma}
\newcommand{\g}{\gamma}
\newcommand{\fraka}{\mbox{\frakturfont a}}
\newcommand{\db}{\bar{\partial}}
\newcommand{\dbs}{\bar{\partial}^*}
\newcommand{\p}{\partial}

%%%%%%%%%%%%% new definitions for the positive mass paper %%%%%%%%%

\newcommand{\sperp}{{\scriptscriptstyle \perp}}

%%%%%%%%%%%%%%%%%%%%%%%


\begin{document}

\begin{definition}
Assume a column of bubbles in the zero-th column.
\[ (0,w_j) = 1 \; \forall j \in \mathbb{P} \]
These bubbles are labelled \textbf{\textit{addendum bubbles}} and are always appended to a diagram when considering the rules below. However, these bubbles are not counted towards the total number of bubbles in each row.
\end{definition}

\begin{definition}
The \textbf{\textit{final bubble}} is defined as the last bubble in a row, where all cells afterwards are empty. Final bubbles will be counted with the following function. It returns a $1$ if the final bubble is at the $(k, w_j)$ location and a $0$ otherwise.
\[ R_{(k,w_j)} =  \begin{cases} 
1 & (k,w_j) = 1 \mid (n,w_j) = 0 \quad \forall n > k \\
0 & \text{otherwise}
\end{cases} \]
Since the addendum bubble exists, each row must contain a final bubble.
\end{definition}


\begin{definition}
Consider a column $l$ which contains $n$ final bubbles below row $w_j$. The \textbf{\textit{vertical ascension}} rule states that for any row above these final bubbles, there can be no bubbles in the $n$ columns to the right of these final bubbles. Further, if these $n$ columns contain $m_0$ final bubbles, then there can be no bubbles in the $n+m_0$ columns. If there are $m_{z+1}$ final bubbles in the $m_{z}$ columns, then the $n+\sum_{i=1}^{z+1} m_i$ columns must be empty.
\[ \sum_{r=1}^{j-1} R_{(l,w_r)} = n \implies (l+t,w_p) = 0 \quad \quad t,p \in \mathbb{N}, \; 0 < t \leq (n+\sum_{i=1}^{z+1} m_i), \; p \geq j \]
Vertical lines of empty cells ascend because of these final bubbles.
\end{definition}

\begin{proposition}
Permutation diagrams satisfy the vertical ascension condition.
\end{proposition}

\begin{proof}
A final bubble $R_{(i,w_j)}$ is always one cell to the left of the death ray beginning in that row, except when another final bubble is in any row below and in the same column as $R_{i,w_j}$. In that case, $R_{(i,w_j)}$'s death ray begins $n+\sum_{t=1}^{z+1} m_t$ cells to the right, where
\[ \sum_{r=1}^{j-1} R_{(i,w_r)} = n, \; \sum_{q=i+1}^{n} \sum_{r=1}^{j-1} R_{(q,w_r)} = m_0, \; \sum_{q=i+\mu+1}^{m_{z}} \sum_{r=1}^{j-1} R_{(q,w_r)} = m_{z+1} \quad \mu = n+m_0+\dots+m_{z-1} \]
Death rays path through the $w_{j+1}$th row in the $n+\sum_{t=1}^{z+1} m_t$ cells to the right of $(i,w_j)$. The vertical ascension rule is identical in disallowing bubbles from these cells. Each death ray cannot have any bubbles in any rows above it, and the vertical ascension condition also matches this statement. Therefore, the vertical ascension condition is satisfied.
\end{proof}

\begin{definition}
Consider a bubble $(i, w_j) = 1$ that is succeeded by $n$ empty cells 
\[(i+k, w_j) = 0 \quad \forall k \leq n, \quad k,n \in \mathbb{P} \]
and a bubble after these empty cells $(i+(n+1), w_{j}) = 1$. Take a box bounded by the two aforementioned bubbles and the bottom of the diagram. Remove the rightmost column and topmost row.
\[ \mathcal{B}_{(i+(n+1), w_{j})} = [(i, w_{j-1}),(i,1)] \times [(i,w_{j-1}), (i+n, w_{j-1})] \]
The \textbf{\textit{empty cell gap}} rule states there are exactly $n$ final bubbles contained within box $\mathcal{B}$.
\end{definition}

\begin{proposition}
Permutation diagrams satisfy the empty cell gap condition.
\end{proposition}

\begin{proof}
Permutation diagrams satisfy the empty cell gap condition. Assume an empty cell gap extending from $(i_1,w_{k+1})$ to $(i_2+1,w_{k+1})$, where $((i_2+1) - i_1) = n$. The permutation diagram requires exactly $n$ death rays to begin between the $i_1$ and $i_2+1$ columns and below the $w_{k+1}$ row. We know, by definition, that each row must contain a final bubble. Given that the $(i_1,w_{k+1})$ bubble exists, we also know that a death ray is not located in the $i_1$ column below the $w_{k+1}$ row. Therefore, for every death ray beginning between the $i_1$ and $i_2+1$ columns, a final bubble is contained in the $i_1$ to $i_2-1$ column range. So, the empty cell gap condition is satisfied by the permutation diagram.
\end{proof}

\begin{proposition}
Arbitrary bubble diagrams that satisfy the cell gap and vertical ascension conditions exist and are unique, given a number of bubbles for each row.
\end{proposition}

\begin{proof}
In order to prove that these two conditions determine a unique bubble diagram (up to rows), we use induction on those rows, starting with the leftmost bubble and moving right in each row.
\\ \par
\textit{Row $1$:} If there are no bubbles in row one $(i,w_1) = 0 \; \forall j \in \mathbb{P}$, then the row is all empty cells (besides the addendum bubble). If $\sum_{i=1}^{\infty} (i,w_1) = m, \; m > 0$, then the row cannot start with any number of empty cells. The empty cell gap rule (given the addendum bubble) states that the $\digamma$ bubble must exist. Since it cannot exist for row one, $(1, w_1) = 1$. The same argument applies to the rest of the row. Any gaps in the first row of bubbles are prevented by the lack of a $\digamma$ bubble. Therefore, for a given number of bubbles, row one is unique.
\\ \par
\textit{Row $k+1$:} If there are no bubbles in row $k+1$
\[ (i,w_{k+1}) = 0 \; \forall i \]
then the row is full of empty cells (besides the addendum bubble, of course). On the other hand, if $\sum_{i=1}^{\infty} (i,w_{k+1}) = m, \; m > 0$, start with the leftmost column, i.e. the zeroth column. We know by the addendum it is always filled. Take cell $c = (i_2+1,w_{k+1})$. Either it is preceded by a bubble, or it is preceded by an empty cell. Take each of these scenarios in turn.
\\ \par
First, assume that cell $c$ is directly preceded by a bubble $(i_2,w_{k+1}) = 1$. Either the vertical ascension rule prevents a bubble from being placed in $c$, or it doesn't. If the vertical ascension rule applies, the cell must be empty. If the vertical ascension rule doesn't apply, assume for contradiction that cell $c$ is left empty. Then, for cell $d_0 = (i_2+2,w_{k+1})$ to be a filled with a bubble, there must be exactly one final bubble in box $\mathcal{B}_{d_0}$, fulfilling the empty cell gap condition. If $\mathcal{B}_{d_0}$ does have a final bubble, then that final bubble could not be in the $i_2$ column because we assumed that the vertical ascension rule does not apply to cell $c$. So, the final bubble must be in box $\mathcal{B}_{d_0}$'s other column, the $i_2+1$ column. A final bubble in this column, however, means that cell $d_0$ would violate the vertical ascension condition if filled. Therefore, $d_0$ must be an empty cell.
\\ \par
Next, take cell $d_{a+1} = (i_2+(a+3),w_{k+1}), \; a \geq 0$. There must be $a+2$ final bubbles in the $\mathcal{B}_{d_{a+1}}$ box for $d_{a+1}$ to fulfill the empty cell gap condition. For each column between $i_2$ and $i_2+(a+3)$, the final bubble count of box $\mathcal{B}_{d_{a+1}}$ can increase by $n$. However, the vertical ascension rule states that the number of empty cells must correspondingly increase by $n$ for each of these rows. Since the first empty cell did not have a final bubble to justify its existence, when the vertical ascension rule is fulfilled, box $\mathcal{B}_{d_{a+1}}$ will always be at least one final bubble short. Therefore, there are bubbles left to place but all remaining locations in the row do not fulfill one or both of the conditions. So, by contradiction, if the vertical ascension rule doesn't apply when a cell $c$ is preceded by a bubble, then $c$ must be a bubble. Bubble locations for an arbitrary diagram are then uniquely determined when those bubbles are placed directly after other bubbles.
%no bubbles are allowed to be placed in the $n$ rows to the right of these final bubbles. Therefore, if $d_{a+1}$ is not at least $n+1$ rows to the right of a given column with $n$ final bubbles, and no more final bubbles were introduced in these $n$ columns, then $d_{a+1}$ must be empty. If $m$ more final bubbles were introduced in these $n$ columns, then $d_{a+1}$ must be at least another $m$ columns away. On the other hand, if $d_{a+1}$ is at least $n+1$ rows to the right of a given column with $n$ final bubbles, then the number of final bubbles in box $\mathcal{B}_{d_{a+1}}$ has only increased by $n$ for $n$ more gaps. The same holds respectively true in the case where $m$ more final bubbles are introduced. Therefore, either the gap still has an insufficient number of final bubbles, or it has a sufficient number of final bubbles, but at least one of those bubbles is in column $i_2+(a+3)$ which would break the vertical ascension rule. 
\\ \par
Conversely, take a cell $c = (i,w_j)$ preceded by at least one empty cell. Assume without loss of generality that $c$ is preceded by $n$ empty cells.
\[(i-k,w_j) = 0 \quad \forall k \in \mathbb{P}, \; k \leq n \]
Either $c$ fulfills the empty cell gap condition, or it doesn't. If it does, then by the logic presented above, a bubble must be placed in cell $c$. Otherwise, the empty cell gap condition will never again be fulfilled (or fulfilled only in the case where the vertical ascension rule is broken), and the row will not have enough bubbles in it. On the other hand, assume cell $c$ does not satisfy the empty cell gap condition. Note that the gap must have originally been created because of an application of the vertical ascension rule. Then, at least one final bubble is in the box $\mathcal{B}_c$ created by this gap. If there are $m$ final bubbles in the $i-n$ column, the gap must (by the vertical ascension condition) be at least $m$ empty cells long. However, if more final bubbles appeared after the $i-n$ column or if $m<n$, then box $\mathcal{B}_c$ may still have too many final bubbles and not enough gaps. Since the bubbles on a graph are finite, at some point the number of final bubbles will stop increasing. Then, there exists some number of gaps which will equal the number of final bubbles in box $\mathcal{B}$. Therefore, there is always a unique place after a gap for a bubble. Before this point, there will be too many empty cells as compared to final bubbles. After this point, there will always be insufficient final bubbles or a contradiction with the vertical ascension rule. Note that it is impossible for a graph to have too many required gaps from the vertical ascension rule and not enough final bubbles for the empty cell gap rule. So, the two rules can always create a unique bubble diagram.
\end{proof}


\begin{theorem}
A bubble diagram satisfies the vertical ascension and empty cell gap conditions if and only if it is a permutation bubble diagram.
\end{theorem}

\begin{proof}
Given a unique number of bubbles for each row, a unique permutation diagram is created. This permutation diagram satisfies the empty cell gap and vertical ascension conditions. These two conditions also correspond to a unique bubble diagram given a unique number of bubbles in each row. Therefore, these two diagrams must always be identical, and so are bijective.
\end{proof}


\end{document}
