\documentclass[10pt]{article}
\usepackage{amsmath}
\usepackage{amsthm}
\usepackage{amsfonts}
\usepackage{dsfont}
\usepackage{amssymb}
\usepackage{latexsym}
\usepackage{tensor}
%\usepackage{epsfig}
\usepackage{graphicx}
\usepackage{tikz}
\usetikzlibrary{cd}
%\usepackage[dvips]{graphicx}
\graphicspath{ {images/} }

\usepackage[matrix,tips,graph,curve]{xy}

\newcommand{\mnote}[1]{${}^*$\marginpar{\footnotesize ${}^*$#1}}
\linespread{1.065}

\makeatletter

\setlength\@tempdima  {5.5in}
\addtolength\@tempdima {-\textwidth}
\addtolength\hoffset{-0.5\@tempdima}
\setlength{\textwidth}{5.5in}
\setlength{\textheight}{8.75in}
\addtolength\voffset{-0.625in}

\makeatother

\makeatletter 
\@addtoreset{equation}{section}
\makeatother


\renewcommand{\theequation}{\thesection.\arabic{equation}}

\theoremstyle{plain}
\newtheorem{theorem}[equation]{Theorem}
\newtheorem{corollary}[equation]{Corollary}
\newtheorem{lemma}[equation]{Lemma}
\newtheorem{proposition}[equation]{Proposition}
\newtheorem{conjecture}[equation]{Conjecture}
\newtheorem{fact}[equation]{Fact}
\newtheorem{facts}[equation]{Facts}
\newtheorem*{theoremA}{Theorem A}
\newtheorem*{theoremB}{Theorem B}
\newtheorem*{theoremC}{Theorem C}
\newtheorem*{theoremD}{Theorem D}
\newtheorem*{theoremE}{Theorem E}
\newtheorem*{theoremF}{Theorem F}
\newtheorem*{theoremG}{Theorem G}
\newtheorem*{theoremH}{Theorem H}

\theoremstyle{definition}
\newtheorem{definition}[equation]{Definition}
\newtheorem{definitions}[equation]{Definitions}
%\theoremstyle{remark}

\newtheorem{remark}[equation]{Remark}
\newtheorem{remarks}[equation]{Remarks}
\newtheorem{exercise}[equation]{Exercise}
\newtheorem{example}[equation]{Example}
\newtheorem{examples}[equation]{Examples}
\newtheorem{notation}[equation]{Notation}
\newtheorem{question}[equation]{Question}
\newtheorem{assumption}[equation]{Assumption}
\newtheorem*{claim}{Claim}
\newtheorem{answer}[equation]{Answer}
%%%%%% letters %%%%

\newcommand{\fA}{\mathfrak{A}}
\newcommand{\fB}{\mathfrak{B}}
\newcommand{\fC}{\mathfrak{C}}
\newcommand{\fD}{\mathfrak{D}}
\newcommand{\fE}{\mathfrak{E}}
\newcommand{\fF}{\mathfrak{F}}
\newcommand{\fG}{\mathfrak{G}}
\newcommand{\fH}{\mathfrak{H}}
\newcommand{\fI}{\mathfrak{I}}
\newcommand{\fJ}{\mathfrak{J}}
\newcommand{\fK}{\mathfrak{K}}
\newcommand{\fL}{\mathfrak{L}}
\newcommand{\fM}{\mathfrak{M}}
\newcommand{\fN}{\mathfrak{N}}
\newcommand{\fO}{\mathfrak{O}}
\newcommand{\fP}{\mathfrak{P}}
\newcommand{\fQ}{\mathfrak{Q}}
\newcommand{\fR}{\mathfrak{R}}
\newcommand{\fS}{\mathfrak{S}}
\newcommand{\fT}{\mathfrak{T}}
\newcommand{\fU}{\mathfrak{U}}
\newcommand{\fV}{\mathfrak{V}}
\newcommand{\fW}{\mathfrak{W}}
\newcommand{\fX}{\mathfrak{X}}
\newcommand{\fY}{\mathfrak{Y}}
\newcommand{\fZ}{\mathfrak{Z}}

\newcommand{\fa}{\mathfrak{a}}
\newcommand{\fb}{\mathfrak{b}}
\newcommand{\fc}{\mathfrak{c}}
\newcommand{\fd}{\mathfrak{d}}
\newcommand{\fe}{\mathfrak{e}}
\newcommand{\ff}{\mathfrak{f}}
\newcommand{\fg}{\mathfrak{g}}
\newcommand{\fh}{\mathfrak{h}}
%\newcommand{\fi}{\mathfrak{i}}

\newcommand{\fj}{\mathfrak{j}}
\newcommand{\fk}{\mathfrak{k}}
\newcommand{\fl}{\mathfrak{l}}
\newcommand{\fm}{\mathfrak{m}}
\newcommand{\fn}{\mathfrak{n}}
\newcommand{\fo}{\mathfrak{o}}
\newcommand{\fp}{\mathfrak{p}}
\newcommand{\fq}{\mathfrak{q}}
\newcommand{\fr}{\mathfrak{r}}
\newcommand{\fs}{\mathfrak{s}}
\newcommand{\ft}{\mathfrak{t}}
\newcommand{\fu}{\mathfrak{u}}
\newcommand{\fv}{\mathfrak{v}}
\newcommand{\fw}{\mathfrak{w}}
\newcommand{\fx}{\mathfrak{x}}
\newcommand{\fy}{\mathfrak{y}}
\newcommand{\fz}{\mathfrak{z}}

\newcommand{\bi}{\textbf{i}}
\newcommand{\bj}{\textbf{j}}
\newcommand{\bk}{\textbf{k}}

\newcommand{\sA}{\mathcal{A}\,}
\newcommand{\sB}{\mathcal{B}\,}
\newcommand{\sC}{\mathcal{C}}
\newcommand{\sD}{\mathcal{D}\,}
\newcommand{\sE}{\mathcal{E}\,}
\newcommand{\sF}{\mathcal{F}\,}
\newcommand{\sG}{\mathcal{G}\,}
\newcommand{\sH}{\mathcal{H}}
\newcommand{\sI}{\mathcal{I}\,}
\newcommand{\sJ}{\mathcal{J}\,}
\newcommand{\sK}{\mathcal{K}\,}
\newcommand{\sL}{\mathcal{L}\,}
\newcommand{\sM}{\mathcal{M}\,}
\newcommand{\sN}{\mathcal{N}}
\newcommand{\sO}{\mathcal{O}}
\newcommand{\sP}{\mathcal{P}\,}
\newcommand{\sQ}{\mathcal{Q}\,}
\newcommand{\sR}{\mathcal{R}}
\newcommand{\sS}{\mathcal{S}}
\newcommand{\sT}{\mathcal{T}\,}
\newcommand{\sU}{\mathcal{U}\,}
\newcommand{\sV}{\mathcal{V}\,}
\newcommand{\sW}{\mathcal{W}\,}
\newcommand{\sX}{\mathcal{X}\,}
\newcommand{\sY}{\mathcal{Y}\,}
\newcommand{\sZ}{\mathcal{Z}\,}

\newcommand{\IA}{\mathbb{A}}
\newcommand{\IB}{\mathbb{B}}
\newcommand{\IC}{\mathbb{C}}
\newcommand{\ID}{\mathbb{D}}
\newcommand{\IE}{\mathbb{E}}
\newcommand{\IF}{\mathbb{F}}
\newcommand{\IG}{\mathbb{G}}
\newcommand{\IH}{\mathbb{H}}
\newcommand{\II}{\mathbb{I}}
\newcommand{\IK}{\mathbb{K}}
\newcommand{\IL}{\mathbb{L}}
\newcommand{\IM}{\mathbb{M}}
\newcommand{\IN}{\mathbb{N}}
\newcommand{\IO}{\mathbb{O}}
\newcommand{\IP}{\mathbb{P}}
\newcommand{\IQ}{\mathbb{Q}}
\newcommand{\IR}{\mathbb{R}}
\newcommand{\IS}{\mathbb{S}}
\newcommand{\IT}{\mathbb{T}}
\newcommand{\IU}{\mathbb{U}}
\newcommand{\IV}{\mathbb{V}}
\newcommand{\IW}{\mathbb{W}}
\newcommand{\IX}{\mathbb{X}}
\newcommand{\IY}{\mathbb{Y}}
\newcommand{\IZ}{\mathbb{Z}}

 \newcommand{\tA}{\mathrm {A}}
 \newcommand{\tB}{\mathrm {B}}
 \newcommand{\tC}{\mathrm {C}}
 \newcommand{\tD}{\mathrm {D}}
 \newcommand{\tE}{\mathrm {E}}
 \newcommand{\tF}{\mathrm {F}}
 \newcommand{\tG}{\mathrm {G}}
 \newcommand{\tH}{\mathrm {H}}
 \newcommand{\tI}{\mathrm {I}}
 \newcommand{\tJ}{\mathrm {J}}
 \newcommand{\tK}{\mathrm {K}}
 \newcommand{\tL}{\mathrm {L}}
 \newcommand{\tM}{\mathrm {M}}
 \newcommand{\tN}{\mathrm {N}}
 \newcommand{\tO}{\mathrm {O}}
 \newcommand{\tP}{\mathrm {P}}
 \newcommand{\tQ}{\mathrm {Q}}
 \newcommand{\tR}{\mathrm {R}}
 \newcommand{\tS}{\mathrm {S}}
 \newcommand{\tT}{\mathrm {T}}
 \newcommand{\tU}{\mathrm {U}}
 \newcommand{\tV}{\mathrm {V}}
 \newcommand{\tW}{\mathrm {W}}
 \newcommand{\tX}{\mathrm {X}}
 \newcommand{\tY}{\mathrm {Y}}
 \newcommand{\tZ}{\mathrm {Z}}
%%%%%%% macros %%%%%

%% my definitions %%%

\newcommand{\End}{\mathrm{End}}
\newcommand{\tr}{\mathrm{tr}}
%\newcommand{\ind}{\mathrm{ind}}

\renewcommand{\index}{\mathrm{index \,}}
\newcommand{\Hom}{\mathrm{Hom}}
\newcommand{\Aut}{\mathrm{Aut}}
\newcommand{\Trace}{\mathrm{Trace}\,}
\newcommand{\Res}{\mathrm{Res}\,}
\newcommand{\rank}{\mathrm{rank}}
%\renewcommand{\dim}{\mathrm{dim}}

\renewcommand{\deg}{\mathrm{deg}}
\newcommand{\spin}{\rm Spin}
\newcommand{\Spin}{\rm Spin}
\newcommand{\erfc}{\rm erfc\,}
\newcommand{\sgn}{{\rm sgn\,}}
\newcommand{\Spec}{\rm Spec\,}
\newcommand{\diag}{\rm diag\,}
\newcommand{\Fix}{\mathrm{Fix}}
\newcommand{\Ker}{\mathrm{Ker \,}}
\newcommand{\Coker}{\mathrm{Coker \,}}
\newcommand{\Sym}{\mathrm{Sym \,}}
\newcommand{\Hess}{\mathrm{Hess \,}}
\newcommand{\grad}{\mathrm{grad \,}}
\newcommand{\Center}{\mathrm{Center}}
\newcommand{\Lie}{\mathrm{Lie}}


\newcommand{\ch}{\rm ch} % Chern Character

\newcommand{\rk}{\rm rk} 
%\renewcommand{\c}{\rm c}  % Chern class

\newcommand{\sign}{\rm sign}
\renewcommand\dim{{\rm dim\,}}
\renewcommand\det{{\rm det\,}}
\newcommand{\dimKrull}{{\rm Krulldim\,}}
\newcommand\Rep{\mathrm{Rep}}
\newcommand\Hilb{\mathrm{Hilb}}
\newcommand\vol{\mathrm{vol}}

\newcommand\QED{\hfill $\Box$} %{\bf QED}} 

\newcommand\Pf{\nonintend{\em Proof. }}


\newcommand\reals{{\mathbb R}} 
\newcommand\complexes{{\mathbb C}}
\renewcommand\Re{\mathrm Re}
\renewcommand\Im{\mathrm Im}
\newcommand\integers{{\mathbb Z}}
\newcommand\quaternions{{\mathbb H}}


\newcommand\iso{{\cong}} 
%\newcommand\tensor{{\otimes}}
\newcommand\Tensor{{\bigotimes}} 
\newcommand\union{\bigcup} 
\newcommand\onehalf{\frac{1}{2}}
%\newcommand\Sym[1]{{Sym^{#1}(\complexes^2)}}

\newcommand\lie[1]{{\mathfrak #1}} 
\renewcommand\fk{\mathfrak{K}}
\newcommand\smooth{\mathcal{C}^{\infty}}
\newcommand\trivial{{\mathbb I}}
\newcommand\widebar{\overline}

%%%%%Delimiters%%%%

\newcommand{\<}{\langle}
\renewcommand{\>}{\rangle}

%\renewcommand{\(}{\left(}
%\renewcommand{\)}{\right)}


%%%% Different kind of derivatives %%%%%

\newcommand{\delbar}{\bar{\partial}}
\newcommand{\pdu}{\frac{\partial}{\partial u}}
%\newcommand{\pd}[1][2]{\frac{\partial #1}{\partial #2}}

%%%%% Arrows %%%%%
%\renewcommand{\ra}{\rightarrow}                   % right arrow
%\newcommand{\lra}{\longrightarrow}              % long right arrow
%\renewcommand{\la}{\leftarrow}                    % left arrow
%\newcommand{\lla}{\longleftarrow}               % long left arrow
%\newcommand{\ua}{\uparrow}                     % long up arrow
%\newcommand{\na}{\nearrow}                      %  NE arrow
%\newcommand{\llra}[1]{\stackrel{#1}{\lra}}      % labeled long right arrow
%\newcommand{\llla}[1]{\stackrel{#1}{\lla}}      % labeled long left arrow
%\newcommand{\lua}[1]{\stackrel{#1}{\ua}}      % labeled  up arrow
%\newcommand{\lna}[1]{\stackrel{#1}{\na}}      % labeled long NE arrow

\newcommand{\into}{\hookrightarrow}
\newcommand{\tto}{\longmapsto}
\def\llra{\longleftrightarrow}

\def\d/{/\mspace{-6.0mu}/}
\newcommand{\git}[3]{#1\d/_{\mspace{-4.0mu}#2}#3}
\newcommand\zetahilb{\zeta_{{\text{Hilb}}}}
\def\Fy{\sF \mspace{-3.0mu} \cdot \mspace{-3.0mu} y}
\def\tv{\tilde{v}}
\def\tw{\tilde{w}}
\def\wt{\widetilde}
\def\wtilde{\widetilde}
\def\what{\widehat}

%%%%%%%%%%%%%%%%%%% Mark's definitions %%%%%%%%%%%%%%%%%%%%

\newcommand{\frakg}{\mbox{\frakturfont g}}
\newcommand{\frakk}{\mbox{\frakturfont k}}
\newcommand{\frakp}{\mbox{\frakturfont p}}
\newcommand{\q}{\mbox{\frakturfont q}}
\newcommand{\frakn}{\mbox{\frakturfont n}}
\newcommand{\frakv}{\mbox{\frakturfont v}}
\newcommand{\fraku}{\mbox{\frakturfont u}}
\newcommand{\frakh}{\mbox{\frakturfont h}}
\newcommand{\frakm}{\mbox{\frakturfont m}}
\newcommand{\frakt}{\mbox{\frakturfont t}}
\newcommand{\G}{\Gamma}
\newcommand{\g}{\gamma}
\newcommand{\fraka}{\mbox{\frakturfont a}}
\newcommand{\db}{\bar{\partial}}
\newcommand{\dbs}{\bar{\partial}^*}
\newcommand{\p}{\partial}

%%%%%%%%%%%%% new definitions for the positive mass paper %%%%%%%%%

\newcommand{\sperp}{{\scriptscriptstyle \perp}}

%%%%%%%%%%%%%%%%%%%%%%%


\begin{document}

One addendum must be noted for the rules below. A column of bubbles is assumed in the zero-th column $(0,w_j) = 1 \; \forall j \in \mathbb{P}$, and the rules are applied accordingly.

\begin{definition}
The \textbf{\textit{final bubble}} $R$ is defined as the last bubble in a row, where all cells afterwards are empty.
\[ R = \{ (i,w_k) = 1 \mid (i,w_n) = 0 \quad \forall n > k \} \]
Since the addendum bubble exists, each row must have a final bubble.
\end{definition}

\begin{definition}
Consider a column $l$ which contains $n$ final bubbles. The \textbf{\textit{vertical ascension}} rule states that for any row $(l,w_j)$ there can be no bubbles in the $n$ columns to the right of these empty cells.
\[ (l+t,w_j) = 0 \quad \forall t \in \mathbb{P}, \; 0 < t \leq n \]

Vertical lines of empty cells ascend because of these final bubbles.
\end{definition}

\begin{definition}
Consider a bubble $(i, w_j) = 1$ that is succeeded by $n$ empty cells 
\[(i+k, w_j) = 0 \quad \forall k \leq n, \quad k,n \in \mathbb{P} \]
and a bubble after these empty cells $(i+(n+1), w_{j}) = 1$. Find the lowest bubble $\digamma$ beneath the $(i, w_j)$ bubble
\[ \digamma = \{ (i, w_{j-t}) = 1 \mid (i, w_{s}) = 0 \quad \forall s < (j-t), \; t \in \mathbb{P} \} \]
Take the box bounded by the three aforementioned bubbles and remove the rightmost column and topmost row
\[ \mathcal{B} = [(i, w_{j-2}), (i+n, w_{j-2})] \times [(i, w_{j-t}), (i, w_{j-2})] \]
The \textbf{\textit{bubble gap}} rule states that $\digamma$ exists, and there are exactly $n$ final bubbles contained within box $\mathcal{B}$.
\end{definition}



%\begin{definition}
%A \textbf{\textit{solid sequence}} is defined as a portion of a row in which all values of cells are identical.
%\[ (i,w_k) = (i+1,w_k) = \dots = (i+j,w_k) \quad j \in \mathbb{P} \]
%\end{definition}

\begin{theorem}
A bubble diagram satisfies the vertical ascension and bubble gap conditions if and only if it is a permutation bubble diagram.
\end{theorem}

\begin{proof}

Permutation diagrams are isomorphic to diagrams with  a single death ray beginning in each row and column. Moreover, each of these diagrams has uniquely determined rows. The number of bubbles in each row uniquely determines the location of the death ray. A diagram satisfying the vertical ascension and bubble gap conditions is also uniquely determined by the number of bubbles in each row. The location of these bubbles corresponds to those in the permutation diagram. These bubbles are placed row-by-row and left to right with the following conditions. Bubbles are placed in each column unless:
\begin{enumerate}
    \item A death ray is in that column, in any of the rows beneath the current row.
    \item The death ray in the current row begins, i.e. we run out of bubbles to place.
\end{enumerate}
The vertical ascension and bubble gap conditions to fulfill both of these parameters inductively on rows and give a bijection to the permutation diagram.
\\ \\
\textit{Row $1$:} If there are no bubbles in row one $(i,w_1) = 0 \; \forall j \in \mathbb{P}$, then the row is all empty cells (besides the addendum bubble). The death ray begins at $(1,w_1)$. The corresponding permutation sends $1$ to itself. If $\sum_{i=1}^{\infty} (i,w_1) = m, \; m > 0$, then the row cannot start with any number of empty cells. The bubble gap rule (given the addendum bubble) states that the $\digamma$ bubble must exist. Since it cannot exist for row one, $(1, w_1) = 1$. The same argument applies to the rest of the row. Any gaps in the first row of bubbles are prevented by the lack of a $\digamma$ bubble. Therefore, the death ray begins at $(m+1,w_1)$ and is preceded by a full line of bubbles. The corresponding permutation sends $1$ to $m+1$.
\\ \\
\textit{Row $k+1$:} If there are no bubbles in row $k+1$
\[ (i,w_{k+1}) = 0 \; \forall i \]
then the row is full of empty cells (besides the addendum bubble, of course). This row places the death ray in the $(j,w_{k+1})$ spot, where the $j$th column is the smallest column in which a death ray has no yet been placed. This corresponds to the permutation sending $k+1$ to $j$.
\\ \\
On the other hand, if $\sum_{i=1}^{\infty} (i,w_{k+1}) = m, \; m > 0$, then the bubble locations must comply with the vertical ascension and bubble gap criteria. Starting with the leftmost column, i.e. the zeroth column, we know by the addendum it is always filled.
\\ \\
Take the cell $c = (i_2+1,w_{k+1})$. Either the cell is preceded by a bubble or it is preceded by an empty cell. Take each of these scenarios in turn. First, assume that the cell $c$ is directly preceded by a bubble $(i_2,w_{k+1}) = 1$. If there are bubbles left to place, $c$ can be a bubble only if there is not a death ray somewhere directly below it. Since the $(i_2,w_{k+1}) = 1$ bubble exists, we know that there is no bubble in the $i_2$ column (below the $w_{k+1}$ row). Therefore, a death ray that begins in the $i_2+1$ column below to $w_{k+1}$ row must have been directly preceded by a bubble. So, the vertical ascension rule ensures that a bubble is not placed in cell $c$ in the case that it is directly preceded by a bubble, and there exists a death ray passing through cell $c$.
\\ \\
Alternatively, assume a bubble gap extending from $(i_1,w_{k+1})$ to $(i_2+1,w_{k+1})$ is created, where $(i_2+1) - (i_1) = n$. For the diagram to correspond to a permutation diagram, exactly $n$ death rays must begin between the $i_1$ and $i_2+1$ columns and below the $w_{k+1}$ row. Otherwise, if less than $n$ death rays begin, then there will be empty cells between  $(i_i,w_{k+1})$ and $(i_2+1,w_{k+1})$ which are not passed through by any death rays. There cannot be more than $n$ death rays, as this would imply some columns have multiple death rays beginning in them. This is impossible as the rows are proven inductively.
\\ \\
We know, by definition, that each row must contain a final bubble. Given that the $(i_1,w_{k+1})$ bubble exists (and given the rows are proven inductively, left to right), we also know that a death ray is not located in the $i_1$ column below the $w_{k+1}$ row. Therefore, for every death ray beginning between the $i_1$ and $i_2+1$ columns, a final bubble is contained in the $i_1$ to $i_2-1$ column range. Furthermore, we know that each row with a final bubble in the $i_1$ to $i_2-1$ column range has a bubble in the $i_1$ column. The lowest of these bubbles is equal to or is above the $\digamma$ bubble. Moreover, if the $\digamma$ bubble is below the lowest final bubble in this range, there is no possibility for extra final bubbles to be included, as this would imply multiple death rays in the same column. So, the bubble gap rule ensures the precise placement for each bubble after a gap. Since at only one bubble location will there be exactly $n$ vertical death rays in box $\mathcal{B}$, the bubble gap rules ensures a unique placement of the bubbles, bijective to the permutation diagram. %this explanation is insufficient i think

\end{proof}

\begin{proposition}
A bubble diagram satisfies the bubble gap conditions and its transposition if and only if it is a permutation bubble diagram.
\end{proposition}


\end{document}
