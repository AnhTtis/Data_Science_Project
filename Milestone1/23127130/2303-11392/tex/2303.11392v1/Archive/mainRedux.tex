\documentclass[10pt]{article}
\usepackage{amsmath}
\usepackage{amsthm}
\usepackage{amsfonts}
\usepackage{dsfont}
\usepackage{amssymb}
\usepackage{latexsym}
\usepackage{tensor}
%\usepackage{epsfig}
\usepackage{graphicx}
\usepackage{tikz}
\usetikzlibrary{cd}
%\usepackage[dvips]{graphicx}
\graphicspath{ {images/} }

\usepackage[matrix,tips,graph,curve]{xy}

\newcommand{\mnote}[1]{${}^*$\marginpar{\footnotesize ${}^*$#1}}
\linespread{1.065}

\makeatletter

\setlength\@tempdima  {5.5in}
\addtolength\@tempdima {-\textwidth}
\addtolength\hoffset{-0.5\@tempdima}
\setlength{\textwidth}{5.5in}
\setlength{\textheight}{8.75in}
\addtolength\voffset{-0.625in}

\makeatother

\makeatletter 
\@addtoreset{equation}{section}
\makeatother


\renewcommand{\theequation}{\thesection.\arabic{equation}}

\theoremstyle{plain}
\newtheorem{theorem}[equation]{Theorem}
\newtheorem{corollary}[equation]{Corollary}
\newtheorem{lemma}[equation]{Lemma}
\newtheorem{proposition}[equation]{Proposition}
\newtheorem{conjecture}[equation]{Conjecture}
\newtheorem{fact}[equation]{Fact}
\newtheorem{facts}[equation]{Facts}
\newtheorem*{theoremA}{Theorem A}
\newtheorem*{theoremB}{Theorem B}
\newtheorem*{theoremC}{Theorem C}
\newtheorem*{theoremD}{Theorem D}
\newtheorem*{theoremE}{Theorem E}
\newtheorem*{theoremF}{Theorem F}
\newtheorem*{theoremG}{Theorem G}
\newtheorem*{theoremH}{Theorem H}

\theoremstyle{definition}
\newtheorem{definition}[equation]{Definition}
\newtheorem{definitions}[equation]{Definitions}
%\theoremstyle{remark}

\newtheorem{remark}[equation]{Remark}
\newtheorem{remarks}[equation]{Remarks}
\newtheorem{exercise}[equation]{Exercise}
\newtheorem{example}[equation]{Example}
\newtheorem{examples}[equation]{Examples}
\newtheorem{notation}[equation]{Notation}
\newtheorem{question}[equation]{Question}
\newtheorem{assumption}[equation]{Assumption}
\newtheorem*{claim}{Claim}
\newtheorem{answer}[equation]{Answer}
%%%%%% letters %%%%

\newcommand{\fA}{\mathfrak{A}}
\newcommand{\fB}{\mathfrak{B}}
\newcommand{\fC}{\mathfrak{C}}
\newcommand{\fD}{\mathfrak{D}}
\newcommand{\fE}{\mathfrak{E}}
\newcommand{\fF}{\mathfrak{F}}
\newcommand{\fG}{\mathfrak{G}}
\newcommand{\fH}{\mathfrak{H}}
\newcommand{\fI}{\mathfrak{I}}
\newcommand{\fJ}{\mathfrak{J}}
\newcommand{\fK}{\mathfrak{K}}
\newcommand{\fL}{\mathfrak{L}}
\newcommand{\fM}{\mathfrak{M}}
\newcommand{\fN}{\mathfrak{N}}
\newcommand{\fO}{\mathfrak{O}}
\newcommand{\fP}{\mathfrak{P}}
\newcommand{\fQ}{\mathfrak{Q}}
\newcommand{\fR}{\mathfrak{R}}
\newcommand{\fS}{\mathfrak{S}}
\newcommand{\fT}{\mathfrak{T}}
\newcommand{\fU}{\mathfrak{U}}
\newcommand{\fV}{\mathfrak{V}}
\newcommand{\fW}{\mathfrak{W}}
\newcommand{\fX}{\mathfrak{X}}
\newcommand{\fY}{\mathfrak{Y}}
\newcommand{\fZ}{\mathfrak{Z}}

\newcommand{\fa}{\mathfrak{a}}
\newcommand{\fb}{\mathfrak{b}}
\newcommand{\fc}{\mathfrak{c}}
\newcommand{\fd}{\mathfrak{d}}
\newcommand{\fe}{\mathfrak{e}}
\newcommand{\ff}{\mathfrak{f}}
\newcommand{\fg}{\mathfrak{g}}
\newcommand{\fh}{\mathfrak{h}}
%\newcommand{\fi}{\mathfrak{i}}

\newcommand{\fj}{\mathfrak{j}}
\newcommand{\fk}{\mathfrak{k}}
\newcommand{\fl}{\mathfrak{l}}
\newcommand{\fm}{\mathfrak{m}}
\newcommand{\fn}{\mathfrak{n}}
\newcommand{\fo}{\mathfrak{o}}
\newcommand{\fp}{\mathfrak{p}}
\newcommand{\fq}{\mathfrak{q}}
\newcommand{\fr}{\mathfrak{r}}
\newcommand{\fs}{\mathfrak{s}}
\newcommand{\ft}{\mathfrak{t}}
\newcommand{\fu}{\mathfrak{u}}
\newcommand{\fv}{\mathfrak{v}}
\newcommand{\fw}{\mathfrak{w}}
\newcommand{\fx}{\mathfrak{x}}
\newcommand{\fy}{\mathfrak{y}}
\newcommand{\fz}{\mathfrak{z}}

\newcommand{\bi}{\textbf{i}}
\newcommand{\bj}{\textbf{j}}
\newcommand{\bk}{\textbf{k}}

\newcommand{\sA}{\mathcal{A}\,}
\newcommand{\sB}{\mathcal{B}\,}
\newcommand{\sC}{\mathcal{C}}
\newcommand{\sD}{\mathcal{D}\,}
\newcommand{\sE}{\mathcal{E}\,}
\newcommand{\sF}{\mathcal{F}\,}
\newcommand{\sG}{\mathcal{G}\,}
\newcommand{\sH}{\mathcal{H}}
\newcommand{\sI}{\mathcal{I}\,}
\newcommand{\sJ}{\mathcal{J}\,}
\newcommand{\sK}{\mathcal{K}\,}
\newcommand{\sL}{\mathcal{L}\,}
\newcommand{\sM}{\mathcal{M}\,}
\newcommand{\sN}{\mathcal{N}}
\newcommand{\sO}{\mathcal{O}}
\newcommand{\sP}{\mathcal{P}\,}
\newcommand{\sQ}{\mathcal{Q}\,}
\newcommand{\sR}{\mathcal{R}}
\newcommand{\sS}{\mathcal{S}}
\newcommand{\sT}{\mathcal{T}\,}
\newcommand{\sU}{\mathcal{U}\,}
\newcommand{\sV}{\mathcal{V}\,}
\newcommand{\sW}{\mathcal{W}\,}
\newcommand{\sX}{\mathcal{X}\,}
\newcommand{\sY}{\mathcal{Y}\,}
\newcommand{\sZ}{\mathcal{Z}\,}

\newcommand{\IA}{\mathbb{A}}
\newcommand{\IB}{\mathbb{B}}
\newcommand{\IC}{\mathbb{C}}
\newcommand{\ID}{\mathbb{D}}
\newcommand{\IE}{\mathbb{E}}
\newcommand{\IF}{\mathbb{F}}
\newcommand{\IG}{\mathbb{G}}
\newcommand{\IH}{\mathbb{H}}
\newcommand{\II}{\mathbb{I}}
\newcommand{\IK}{\mathbb{K}}
\newcommand{\IL}{\mathbb{L}}
\newcommand{\IM}{\mathbb{M}}
\newcommand{\IN}{\mathbb{N}}
\newcommand{\IO}{\mathbb{O}}
\newcommand{\IP}{\mathbb{P}}
\newcommand{\IQ}{\mathbb{Q}}
\newcommand{\IR}{\mathbb{R}}
\newcommand{\IS}{\mathbb{S}}
\newcommand{\IT}{\mathbb{T}}
\newcommand{\IU}{\mathbb{U}}
\newcommand{\IV}{\mathbb{V}}
\newcommand{\IW}{\mathbb{W}}
\newcommand{\IX}{\mathbb{X}}
\newcommand{\IY}{\mathbb{Y}}
\newcommand{\IZ}{\mathbb{Z}}

 \newcommand{\tA}{\mathrm {A}}
 \newcommand{\tB}{\mathrm {B}}
 \newcommand{\tC}{\mathrm {C}}
 \newcommand{\tD}{\mathrm {D}}
 \newcommand{\tE}{\mathrm {E}}
 \newcommand{\tF}{\mathrm {F}}
 \newcommand{\tG}{\mathrm {G}}
 \newcommand{\tH}{\mathrm {H}}
 \newcommand{\tI}{\mathrm {I}}
 \newcommand{\tJ}{\mathrm {J}}
 \newcommand{\tK}{\mathrm {K}}
 \newcommand{\tL}{\mathrm {L}}
 \newcommand{\tM}{\mathrm {M}}
 \newcommand{\tN}{\mathrm {N}}
 \newcommand{\tO}{\mathrm {O}}
 \newcommand{\tP}{\mathrm {P}}
 \newcommand{\tQ}{\mathrm {Q}}
 \newcommand{\tR}{\mathrm {R}}
 \newcommand{\tS}{\mathrm {S}}
 \newcommand{\tT}{\mathrm {T}}
 \newcommand{\tU}{\mathrm {U}}
 \newcommand{\tV}{\mathrm {V}}
 \newcommand{\tW}{\mathrm {W}}
 \newcommand{\tX}{\mathrm {X}}
 \newcommand{\tY}{\mathrm {Y}}
 \newcommand{\tZ}{\mathrm {Z}}
%%%%%%% macros %%%%%

%% my definitions %%%

\newcommand{\End}{\mathrm{End}}
\newcommand{\tr}{\mathrm{tr}}
%\newcommand{\ind}{\mathrm{ind}}

\renewcommand{\index}{\mathrm{index \,}}
\newcommand{\Hom}{\mathrm{Hom}}
\newcommand{\Aut}{\mathrm{Aut}}
\newcommand{\Trace}{\mathrm{Trace}\,}
\newcommand{\Res}{\mathrm{Res}\,}
\newcommand{\rank}{\mathrm{rank}}
%\renewcommand{\dim}{\mathrm{dim}}

\renewcommand{\deg}{\mathrm{deg}}
\newcommand{\spin}{\rm Spin}
\newcommand{\Spin}{\rm Spin}
\newcommand{\erfc}{\rm erfc\,}
\newcommand{\sgn}{{\rm sgn\,}}
\newcommand{\Spec}{\rm Spec\,}
\newcommand{\diag}{\rm diag\,}
\newcommand{\Fix}{\mathrm{Fix}}
\newcommand{\Ker}{\mathrm{Ker \,}}
\newcommand{\Coker}{\mathrm{Coker \,}}
\newcommand{\Sym}{\mathrm{Sym \,}}
\newcommand{\Hess}{\mathrm{Hess \,}}
\newcommand{\grad}{\mathrm{grad \,}}
\newcommand{\Center}{\mathrm{Center}}
\newcommand{\Lie}{\mathrm{Lie}}


\newcommand{\ch}{\rm ch} % Chern Character

\newcommand{\rk}{\rm rk} 
%\renewcommand{\c}{\rm c}  % Chern class

\newcommand{\sign}{\rm sign}
\renewcommand\dim{{\rm dim\,}}
\renewcommand\det{{\rm det\,}}
\newcommand{\dimKrull}{{\rm Krulldim\,}}
\newcommand\Rep{\mathrm{Rep}}
\newcommand\Hilb{\mathrm{Hilb}}
\newcommand\vol{\mathrm{vol}}

\newcommand\QED{\hfill $\Box$} %{\bf QED}} 

\newcommand\Pf{\nonintend{\em Proof. }}


\newcommand\reals{{\mathbb R}} 
\newcommand\complexes{{\mathbb C}}
\renewcommand\Re{\mathrm Re}
\renewcommand\Im{\mathrm Im}
\newcommand\integers{{\mathbb Z}}
\newcommand\quaternions{{\mathbb H}}


\newcommand\iso{{\cong}} 
%\newcommand\tensor{{\otimes}}
\newcommand\Tensor{{\bigotimes}} 
\newcommand\union{\bigcup} 
\newcommand\onehalf{\frac{1}{2}}
%\newcommand\Sym[1]{{Sym^{#1}(\complexes^2)}}

\newcommand\lie[1]{{\mathfrak #1}} 
\renewcommand\fk{\mathfrak{K}}
\newcommand\smooth{\mathcal{C}^{\infty}}
\newcommand\trivial{{\mathbb I}}
\newcommand\widebar{\overline}

%%%%%Delimiters%%%%

\newcommand{\<}{\langle}
\renewcommand{\>}{\rangle}

%\renewcommand{\(}{\left(}
%\renewcommand{\)}{\right)}


%%%% Different kind of derivatives %%%%%

\newcommand{\delbar}{\bar{\partial}}
\newcommand{\pdu}{\frac{\partial}{\partial u}}
%\newcommand{\pd}[1][2]{\frac{\partial #1}{\partial #2}}

%%%%% Arrows %%%%%
%\renewcommand{\ra}{\rightarrow}                   % right arrow
%\newcommand{\lra}{\longrightarrow}              % long right arrow
%\renewcommand{\la}{\leftarrow}                    % left arrow
%\newcommand{\lla}{\longleftarrow}               % long left arrow
%\newcommand{\ua}{\uparrow}                     % long up arrow
%\newcommand{\na}{\nearrow}                      %  NE arrow
%\newcommand{\llra}[1]{\stackrel{#1}{\lra}}      % labeled long right arrow
%\newcommand{\llla}[1]{\stackrel{#1}{\lla}}      % labeled long left arrow
%\newcommand{\lua}[1]{\stackrel{#1}{\ua}}      % labeled  up arrow
%\newcommand{\lna}[1]{\stackrel{#1}{\na}}      % labeled long NE arrow

\newcommand{\into}{\hookrightarrow}
\newcommand{\tto}{\longmapsto}
\def\llra{\longleftrightarrow}

\def\d/{/\mspace{-6.0mu}/}
\newcommand{\git}[3]{#1\d/_{\mspace{-4.0mu}#2}#3}
\newcommand\zetahilb{\zeta_{{\text{Hilb}}}}
\def\Fy{\sF \mspace{-3.0mu} \cdot \mspace{-3.0mu} y}
\def\tv{\tilde{v}}
\def\tw{\tilde{w}}
\def\wt{\widetilde}
\def\wtilde{\widetilde}
\def\what{\widehat}

%%%%%%%%%%%%%%%%%%% Mark's definitions %%%%%%%%%%%%%%%%%%%%

\newcommand{\frakg}{\mbox{\frakturfont g}}
\newcommand{\frakk}{\mbox{\frakturfont k}}
\newcommand{\frakp}{\mbox{\frakturfont p}}
\newcommand{\q}{\mbox{\frakturfont q}}
\newcommand{\frakn}{\mbox{\frakturfont n}}
\newcommand{\frakv}{\mbox{\frakturfont v}}
\newcommand{\fraku}{\mbox{\frakturfont u}}
\newcommand{\frakh}{\mbox{\frakturfont h}}
\newcommand{\frakm}{\mbox{\frakturfont m}}
\newcommand{\frakt}{\mbox{\frakturfont t}}
\newcommand{\G}{\Gamma}
\newcommand{\g}{\gamma}
\newcommand{\fraka}{\mbox{\frakturfont a}}
\newcommand{\db}{\bar{\partial}}
\newcommand{\dbs}{\bar{\partial}^*}
\newcommand{\p}{\partial}

%%%%%%%%%%%%% new definitions for the positive mass paper %%%%%%%%%

\newcommand{\sperp}{{\scriptscriptstyle \perp}}

%%%%%%%%%%%%%%%%%%%%%%%



%%%%%%%%%%%%%%%%%%%%%%%%%%%%%%%%%%%%%%%%%%%%%



%
\begin{document}
%

\title{Characterizing Permutation Bubble Diagrams}
\author{Ben Gillen and Jonathan Michala}


\date{\today}

\maketitle
\section{Introduction}

\section{Definitions for Permutation Diagrams}

\begin{definition}
Consider a column of bubbles in the zero-th column 
\[ (0,w_j) = 1 \; \forall j \in \mathbb{P} \]
These bubbles are labelled \textbf{\textit{addendum bubbles}} and are always appended to a diagram when using the vertical ascension and empty cell gap rules defined below.
\end{definition}


\begin{definition}
Consider a column $l$ which contains $n$ final bubbles. The \textbf{\textit{vertical ascension}} rule states that for any row $(l,w_j)$ there can be no bubbles in the $n$ columns to the right of these empty cells.
\[ (l+t,w_j) = 0 \quad \forall t \in \mathbb{P}, \; 0 < t \leq n \]
Vertical lines of empty cells ascend because of these final bubbles.
\end{definition}

\begin{definition}
The \textbf{\textit{final bubble}} $R$ is defined as the last bubble in a row, where all cells afterwards are empty.
\[ R = \{ (i,w_k) = 1 \mid (i,w_n) = 0 \quad \forall n > k \} \]
Since the addendum bubble exists, each row must contain a final bubble.
\end{definition}

\begin{definition}
Consider a bubble $(i, w_j) = 1$ that is succeeded by $n$ empty cells 
\[(i+k, w_j) = 0 \quad \forall k \leq n, \quad k,n \in \mathbb{P} \]
and a bubble after these empty cells $(i+(n+1), w_{j}) = 1$. Find the lowest bubble $\digamma$ beneath the $(i, w_j)$ bubble
\[ \digamma = \{ (i, w_{j-t}) = 1 \mid (i, w_{s}) = 0 \quad \forall s < (j-t), \; t \in \mathbb{P} \} \]
Take the box bounded by the three aforementioned bubbles and remove the rightmost column and topmost row
\[ \mathcal{B}_{(i+(n+1), w_{j})} = [(i, w_{j-2}), (i+n, w_{j-2})] \times [(i, w_{j-t}), (i, w_{j-2})] \]
The \textbf{\textit{empty cell gap}} rule states that $\digamma$ exists, and there are exactly $n$ final bubbles contained within box $\mathcal{B}$.
\end{definition}
\begin{proposition}
Permutation diagrams satisfy the vertical ascension and empty cell gap conditions.
\end{proposition}

\begin{proof}
Permutation diagrams satisfy the vertical ascension condition. If a column $i$ contains $l$ final bubbles, then each one of those final bubbles corresponds to a death ray in a unique, adjacent column. Therefore, the $l$ columns after $i$ cannot contain any bubbles because of the death rays in each of these columns. So, the vertical ascension condition is satisfied.
\\ \par
Permutation diagrams satisfy the empty cell gap condition. Assume an empty cell gap extending from $(i_1,w_{k+1})$ to $(i_2+1,w_{k+1})$, where $(i_2+1) - (i_1) = n$. The permutation diagram requires exactly $n$ death rays to begin between the $i_1$ and $i_2+1$ columns and below the $w_{k+1}$ row. We know, by definition, that each row must contain a final bubble. Given that the $(i_1,w_{k+1})$ bubble exists, we also know that a death ray is not located in the $i_1$ column below the $w_{k+1}$ row. Therefore, for every death ray beginning between the $i_1$ and $i_2+1$ columns, a final bubble is contained in the $i_1$ to $i_2-1$ column range. Furthermore, we know that each row with a final bubble in the $i_1$ to $i_2-1$ column range has a bubble in the $i_1$ column. The lowest of these bubbles is equal to or is above the $\digamma$ bubble. Moreover, if the $\digamma$ bubble is below the lowest final bubble in this range, there is no possibility for extra final bubbles to be included, as this would imply multiple death rays in the same column. So, the empty cell gap condition is satisfied by the permutation diagram. Therefore, permutation diagrams satisfy both conditions.
\end{proof}

\begin{definition}
Consider two procedures for labeling the bubbles in a bubble diagram with numbers.
In the first procedure, for the $n$th row, we label the bubbles from left to right $n,n+1,n+2,$ and so on.
In the second procedure, for the $n$th column, we label the bubbles from bottom to top $n, n+1,n+2,$ and so on.
We say a bubble diagram satisfies the \textbf{\textit{numbering condition}} if both of these procedures yield the same label for each bubble.
\end{definition}

\begin{proposition}
Permutation diagrams satisfy the numbering condition.
\end{proposition}

\begin{proof}
Let $(i,w_j)$ be a bubble in a permutation diagram.
In the first procedure, we label the bubble $w_j$ plus the number of bubbles below it, or in other words
$$w_j + \#\{(l,w_j) \colon l < j, w_l > w_j, l < i\}$$
Similarly, in the second procedure, we label the bubble
$$i + \#\{(i,w_k) \colon i < k, w_i > w_k, w_k < w_j\}$$
We need to prove that these two values are equivalent.
Since $(i,w_j)$ is a bubble, $i < j, w_i > w_j$, implying our question is reduced to 
$$i + \#\{(i,w_k) \colon i < k, w_k < w_j\} = w_j + \#\{(l,w_j) \colon l < i, w_l > w_j\}.$$
First, we sum up the number of bubbles and empty cells to the left of $(i,w_j)$ to get $w_j$: 
$$\#\{(i,w_k) \colon i < k, w_k < w_j\} + \#\{(i,w_k) \colon i > k, w_k < w_j\} + 1 = w_j$$
Then our question is reduced to
\begin{equation}
i = \#\{(i,w_k) \colon i > k, w_k < w_j\} + \#\{(l,w_j) \colon l < i, w_l > w_j\} + 1.
\label{eq:i}
\end{equation}
This is true, because for any $m < i$, we either have $w_m < w_j$ or $w_m > w_j$, implying that either $(i,w_m) \in \{(i,w_k) \colon i > k, w_k < w_j\}$ or $(m,w_j) \in \{(l,w_j) \colon l < i, w_l > w_j\}$ respectively.
So each $m < i$ gets counted once by the right hand side of \ref{eq:i}, therefore showing the equality.
\end{proof}

\begin{definition}
Consider the following two procedures to add dots to a bubble diagram.
In one procedure, start from the first row and go up, and in each row add a dot in the first cell to the right of all the bubbles in that row such that there is no dot below the newly placed dot.
In the other procedure, start from the left column and go to the right, and in each column add a dot in the first cell above all of the bubbles in that column such that there is no dot to the left of the newly placed dot.
We say that a bubble diagram satisfies the \textbf{\textit{dot condition}} if these two procedures result in the same placement of dots in the whole diagram
\end{definition}

\begin{proposition}
All permutation diagrams satisfy the dot condition
\end{proposition}

\begin{proof}
This is true, because in both procedures, we place dots exactly in the locations $(i,w_i)$ for all $i \in \IZ_+$.
In the first procedure, in the first row, we place bubbles in all the cells $(1,w_j)$ such that $1 < j$ and $w_1 > w_j$.
We know these two conditions are true for the first cells $(1,1),(1,2),...,(1,w_1 - 2),(1,w_1-1)$ and becomes false for $(1,w_1)$, which is therefore where the dot in this row must be placed by the dot procedure.
For the $k$th row, consider any cell $(k,w_j)$.
If $k > j$, then there is a dot at $(j,w_j)$ below the $k$th row and so there cannot be a dot in $(k,w_j)$ by the rules of the procedure.
If $w_k < w_j$, then $(k,w_k)$ is to the left of $(k,w_j)$ and would get a dot before $(k,w_j)$ according to the procedure, so there is no dot in $(k,w_j)$.
This is because we know there is no dot below $(k,w_k)$ by induction on the previous rows.
And if $k < j$ and $w_k > w_j$, then there is a bubble in this $(k,w_j)$.
Hence, if $j \neq k$, then there is no dot in the $(k,w_j)$th cell, and therefore the only cell in which a dot can be placed by the procedure is $(k,w_k)$.

In the second procedure, in the first column, we place bubbles in all the cells $(i,1)$ such that $i < j$ and $w_i > 1$ for $j$ the index giving $w_j = 1$.
We know these two conditions are true for the first cells $(1,1)(2,1),...,(j-2,1),(j-1,1)$ and becomes false for $(j,1)$ which is therefore where the dot in this column must be placed by this dot procedure.
For the $w_k$th column, consider any cell $(i,w_k)$.
If $i < k$ and $w_i > w_k$, then there is a bubble in this cell.
If $i > k$ then $(k,w_k)$ is below $(i,w_k)$ and would get a dot before $(i,w_k)$ by the procedure, so there is no dot in $(i,w_k)$.
This is because we know there is no dot to the left of $(k,w_k)$ by induction on the previous columns.
If $w_i < w_k$, then there is a dot at $(i,w_i)$ to the left of the $w_k$th column and so there cannot be a dot in $(i,w_k)$ by the rules of the procedure.
Hence, if $i \neq k$, then there is no dot in the $(i,w_k)$th cell, and therefore the only cell in which a dot can be placed by the procedure is $(k,w_k)$.

We have then shown that all dots are placed in the same cells by both procedures, so any permutation bubble diagram satisfies the dot condition.
\end{proof}

\section{Properties of Permutation Diagrams}

\begin{lemma}
For every finite sequence of nonnegative integers $a_1,...,a_n$, there exists a permutation such that its bubble diagram contains $a_i$ bubbles in the $i$th row for all $i = 1,...,n$.
\end{lemma}

\begin{proof}
INCOMPLETE
\end{proof}

\section{Dot Proof}




\begin{theorem}
A bubble diagram satisfies the dot condition and the numbering condition if and only if it is a permutation bubble diagram.
\label{thm:DC+NC}
\end{theorem}

To prove this theorem, we show that both permutation diagrams and diagrams that satisfy the dot and numbering conditions are unique up to the number of bubbles in each row.


\begin{lemma}
If a bubble diagram satisfies the dot condition and the numbering condition, then it is unique up to how many bubbles are in each row.
\label{lem:DC+NC=>unique up to rows}
\end{lemma}

\begin{proof}
We again induct on the rows.
The first row is determined just based on the dot condition or the numbering condition independently.
Now, assume for the first $i-1$ rows, the bubbles have been placed uniquely so as not to violate the dot or the numbering conditions.
We claim that we must place the $a_i$ bubbles in the first $a_i$ columns that do not yet contain a dot.
Note first that we cannot place any bubbles above any dots by the dot condition.
Now, assume we can do some other configuration than the one in our claim.
Then there will be an empty cell $(i,w_j)$ in a column with no dot below it and a bubble to the right in some $(i, w_j + k)$, where we'll denote $w_{j'} := w_j + k$.

If there is an empty cell below the bubble in the $w_{j'}$th column, then there must be a dot to the left of the empty cell.
For if there were no dot to the left, then there would be a dot to the right.
This means there is a bubble to the right as well, because if not, then a dot would be placed in the empty cell.
Since there is a bubble to the right, by the uniqueness of the constructed row, there must be a dot below the empty cell prohibiting a bubble from being placed in the empty cell.
This is a contradiction, so there must be a dot to the left of the empty cell.
Hence, the dots to the left correspond exactly to the empty cells below $(i,w_{j'})$.

Let $d$ be the number of dots to the left of the $w_{j'}$ column and below the $i$th row equals $i - 1$.
Then the number of bubbles below $(i,w_{j'})$ equals $i - 1 - d$.
And therefore, we must label the bubble $(i,w_{j'})$, going from bottom to top, with the number $w_{j'} + i - 1 - d$.
Now, consider the number of bubbles to the left of $(i,w_{j'})$.
Since there can be no bubble above a dot, and since there is at least one extra empty cell with no dot below it, the number of bubbles to the left must be strictly less than $w_{j'} -1 - d$.
Hence, we must label the bubble $(i,w_{j'})$, going left to right, with a number strictly less than $i + w_{j'} - 1 - d$, contradicting our previous label.
Therefore, our assumption that we could do anything different than placing the $a_i$ bubbles in the first $a_i$ un-dotted columns was false.

\end{proof}

\noindent\textit{Proof of Theorem \ref{thm:DC+NC}}.
We have seen that a given permutation diagram satisfies the dot and numbering conditions.
Conversely, consider an arbitrary bubble diagram that satisfies the dot and numbering conditions.
Write $a_1,...,a_n$ for the number of bubbles in each row, with the last row of bubbles being the $n$th.
Then, there exists a permutation diagram with the same number of bubbles in the corresponding rows.
This permutation diagram satisfies the dot and numbering conditions, and by the uniqueness in Lemma \ref{lem:DC+NC=>unique up to rows}, this permutation diagram is exactly the bubble diagram we started with.
In particular, the diagram is a permutation diagram.
\QED{}


\begin{proposition}
A bubble diagram satisfies the dot condition and the southwest condition if and only if it is a permutation bubble diagram.
\end{proposition}

\begin{proof}
First, assume we have a permutation bubble diagram.
We claim this diagram satisfies the southwest condition.
Let $(i,w_j)$ and $(i+k,w_j - l)$ be the location of two bubbles in the diagram, for $k,l \in \IZ_+$.
Let $j'$ be the index such that $w_{j'} = w_j - l$.
This means that $i < j$, $w_i > w_j$, $i+k < j'$, and $w_{i+k} > w_{j'}$.
Then $i < i+k < j'$ and $w_i > w_j > w_j - l = w_{j'}$, implying there is a bubble placed at $(i,w_{j'})$.

And we have already seen that permutation diagrams satisfy the dot condition.
Now, we consider the converse.


Let $a_1,...,a_n$ be a sequence of nonnegative integers.
We claim that there is only one diagram with $a_i$ bubbles in the $i$th row for all $i$ that satisfies the dot and southwest conditions.
To satisfy the dot condition in the first row, we must have $a_1$ bubbles placed in the first $a_1$ positions, with a dot in the $(a_1+1)$th position.
For if not, then we would place a dot to the right of all these bubbles, but in the other procedure, we would place a dot in the first empty position in this row.

Now assume the first $k-1$ rows have been determined with bubbles and dots placed uniquely.
We claim that the $a_k$ bubbles and dot must be placed in the first $a_k + 1$ columns that do not contain a dot.
Indeed, if this is not the case, then there is an empty cell in the $k$th row with bubbles to the right and no dots below it.
For the dot condition to be satisfied, we must have the dot in this column placed above the $k$th row, since there are bubbles to the right and the dot in the $k$th row must be placed to the right of them.
But with respect to the other procedure, we cannot place this column's dot above the $k$th row because the southwest condition prohibits any bubbles from being placed above the $k$th row, and the dot is placed in the first position above all the bubbles in the column such that there is no dot to the left.
Thus, the empty cell satisfies these properties, so no dot could be placed above it.
This is a contradiction, so there is only one way to place the $a_k$ bubbles and dot in the $k$th row.

Hence, diagrams that satisfy the dot and southwest condition are unique up to the number of bubbles in each row. And since the permutation diagrams satisfy the dot and southwest conditions and are also unique in this way, they are the same diagrams.
\end{proof}

\section{Box Proof}




\begin{proposition}
Arbitrary bubble diagrams that satisfy the cell gap and vertical ascension conditions exist and are unique, given a number of bubbles for each row.
\end{proposition}

\begin{proof}
In order to prove that these two conditions determine a unique bubble diagram (up to rows), we use induction on those rows, starting with the leftmost bubble and moving right in each row.
\\ \par
\textit{Row $1$:} If there are no bubbles in row one $(i,w_1) = 0 \; \forall j \in \mathbb{P}$, then the row is all empty cells (besides the addendum bubble). If $\sum_{i=1}^{\infty} (i,w_1) = m, \; m > 0$, then the row cannot start with any number of empty cells. The empty cell gap rule (given the addendum bubble) states that the $\digamma$ bubble must exist. Since it cannot exist for row one, $(1, w_1) = 1$. The same argument applies to the rest of the row. Any gaps in the first row of bubbles are prevented by the lack of a $\digamma$ bubble. Therefore, for a given number of bubbles, row one is unique.
\\ \par
\textit{Row $k+1$:} If there are no bubbles in row $k+1$
\[ (i,w_{k+1}) = 0 \; \forall i \]
then the row is full of empty cells (besides the addendum bubble, of course). On the other hand, if $\sum_{i=1}^{\infty} (i,w_{k+1}) = m, \; m > 0$, start with the leftmost column, i.e. the zeroth column. We know by the addendum it is always filled. Take cell $c = (i_2+1,w_{k+1})$. Either it is preceded by a bubble, or it is preceded by an empty cell. Take each of these scenarios in turn.
\\ \par
First, assume that cell $c$ is directly preceded by a bubble $(i_2,w_{k+1}) = 1$. Either the vertical ascension rule prevents a bubble from being placed in $c$, or it doesn't. If the vertical ascension rule applies, the cell must be empty. If the vertical ascension rule doesn't apply, assume for contradiction that cell $c$ is left empty. Then, for cell $d_0 = (i_2+2,w_{k+1})$ to be a bubble, there must be exactly one final bubble in box $\mathcal{B}_{d_0}$. This final bubble would fulfill the empty cell gap condition. If $\mathcal{B}_{d_0}$ does have a final bubble, then that final bubble could not be in the $i_2$ column because we assumed that the vertical ascension rule does not apply to cell $c$. So, the final bubble must be in box $\mathcal{B}_{d_0}$'s other column, the $i_2+1$ column. By definition, cell $d_0$ would violate the vertical ascension condition if a final bubble was in the $i_2+1$ column. Therefore, $d_0$ must be an empty cell.
\\ \par
Next, take cell $d_{a+1} = (i_2+(a+3),w_{k+1})$. There must be $a+2$ final bubbles in the $\mathcal{B}_{d_{a+1}}$ box for $d_{a+1}$ to fulfill the empty cell gap condition and be a bubble. For each column between $i_2$ and $i_2+(a+3)$, the final bubble count of box $\mathcal{B}_{d_{a+1}}$ can increase by $n$. However, the vertical ascension rule states that no bubbles are allowed to be placed in the $n$ rows to the right of these final bubbles. Therefore, if $d_{a+1}$ is not at least $n+1$ rows to the right of a given column with $n$ final bubbles, it must be empty. On the other hand, if $d_{a+1}$ is at least $n+1$ rows to the right of a given column with $n$ final bubbles, then the number of final bubbles in box $\mathcal{B}_{d_{a+1}}$ has only increased by $n$ for $n$ more gaps. Therefore, either the gap still has an insufficient number of final bubbles, or it has a sufficient number of final bubbles, but at least one of those bubbles is in a column that makes $d_{a+1}$ break the vertical ascension rule. Therefore, there are bubbles left to place but all remaining locations in the row do not fulfill one or both of the conditions. So, by contradiction, if the vertical ascension rule doesn't apply when a cell $c$ is preceded by a bubble, then $c$ must be a bubble. Bubble locations for an arbitrary diagram are then uniquely determined when those bubbles are placed directly after other bubbles.
\\ \par
Conversely, take a cell $c = (i,w_j)$ preceded by at least one empty cell. Assume without loss of generality that $c$ is preceded by $n$ empty cells.
\[(i-k,w_j) = 0 \quad \forall k \in \mathbb{P}, \; k \leq n \]
Either $c$ fulfills the empty cell gap condition, or it doesn't. If it does, then by the logic presented above, a bubble must be placed in cell $c$. Otherwise, the empty cell gap condition will never again be fulfilled, and the row will not have enough bubbles in it. On the other hand, assume cell $c$ does not satisfy the empty cell gap condition. Note that the gap must have originally been created because of an application of the vertical ascension rule. Then, at least one final bubble is in the box $\mathcal{B}_c$ created by this gap. If there are $m$ final bubbles in the $i-n$ column, the gap must (by the vertical ascension condition) be at least $m$ empty cells long. However, if more final bubbles appeared after the $i-n$ column or if $m<n$, then box $\mathcal{B}_c$ may still have too many final bubbles and not enough gaps. Since the bubbles on a graph are finite, at some point the number of final bubbles will stop increasing. Then, there exists some number of gaps which will equal the number of final bubbles in box $\mathcal{B}$. Therefore, there is always a unique place after a gap for a bubble. Before this point, there will be too many empty cells as compared to final bubbles. After this point, there will always be insufficient final bubbles or a contradiction with the vertical ascension rule. Note that it is impossible for a graph to have too many required gaps from the vertical ascension rule and not enough final bubbles for the empty cell gap rule. So, the two rules can always create a unique bubble diagram.
\end{proof}




\begin{theorem}
A bubble diagram satisfies the vertical ascension and empty cell gap conditions if and only if it is a permutation bubble diagram.
\end{theorem}

\begin{proof}
Given a unique number of bubbles for each row, a unique permutation diagram is created. This permutation diagram satisfies the empty cell gap and vertical ascension conditions. These two conditions also correspond to a unique bubble diagram given a unique number of bubbles in each row. Therefore, these two diagrams must always be identical, and so are bijective.
\end{proof}


\end{document}