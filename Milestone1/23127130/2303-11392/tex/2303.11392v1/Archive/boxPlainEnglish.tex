\documentclass[11pt]{article}

\usepackage{amsmath,amssymb}
\usepackage{fullpage}

\begin{document}

\section{Direct Formulation}

This section will provide a set of rules to comprehensively analyze a bubble diagram. If the diagram is in accordance with these rules, then the bubble diagram is isomorphic to a unique permutation. The bubble diagram can be viewed (and will be referred to) as a set of bubbles and crosses. The crosses represent 'death rays' and everything they touch. The bubbles represent all other points.

\subsection{Rules}

The three needed rules are:

\begin{enumerate}

\item The first rule is the 'southwest' rule, as given by Assaf (20??). Plainly, this rule states that if two bubbles are on unique rows and columns, a third bubble must exist at the 'southwest' intersection of the two bubbles' rows/columns.

\item The second rule will be called the 'vertical ascension' rule. In plain terms, this rule states that if a cross is on the right side of a bubble, there can be no bubbles in the same row above this cross. A vertical line ascends from this cross, preventing more bubbles. This concept is similar to a death ray, but does not introduce the same problems of finding the unique origin for each ray.

\item The third and last rule will be called the 'bubble gap' rule \footnote{Note, this rule has no relation to bubble gap dimensions.}. Simply stated, consider a row. If a bubble follows  a bubble and then $n$ crosses, there must exist a third bubble inline with and below the left bubble. There may be multiple such bubbles. Consider the lowest one. Draw a box with this bubble as the bottom left corner. The top right corner is the point down and right one from the top rightmost bubble. Count the number of bubbles in this box that are the final bubbles of their respective rows. There must be exactly $n$ such 'final' bubbles.

\end{enumerate}

\noindent
Addendum: One extra condition must be employed, although I don't consider it to be a 'true' rule. Create an imaginary column of bubbles to the left of the first column. The bubble diagram must comply with the rules above with this appended column.

\subsection{Proof of Completeness}

The proof to show these rules correspond to a unique permutation is done by induction on rows. For a given number of bubbles, the bubble locations on a row are unique. This induction is valid as each row is independently valid. It is possible for any set of given rows to append infinite rows of crosses to the top of it. This proof is left as an exercise for the reader.

\textbf{Proof for row $1$:}

If the bubble number of row one $\mathfrak{B}_1 = 0$, then the row is full of crosses. If $\mathfrak{B}_1 > 0$, then the row cannot start with a cross. If it did, any bubble placed after this cross -- or any number of crosses -- would violate the bubble gap rule (given the extra addendum bubble). No third bubble exists on a lower row. Then, the first symbol must be a bubble. This reasoning identically applies if any crosses are attempted to be placed in this first row before the final bubble.

\textbf{Proof for row $k+1$:}

If the bubble number of row $k+1$ $\mathfrak{B}_{k+1} = 0$, then the row is full of crosses. As stated above, this row is always valid. If $\mathfrak{B}_{k+1} > 0$, then the first place a bubble can be placed is in a column that complies with the vertical ascension and bubble gap rule. If this bubble gap rule is fulfilled, then exactly $n$ death rays begin between the top leftmost and rightmost bubble. This fact is true because a final bubble must be to the left of the beginning of every death ray. If a death ray began without a final bubble to the left of it in the box, then there must be a death ray beginning in the same column but below the top leftmost death ray, which is a contradiction The bubble must specifically be \textbf{in} the box by the southwest rule. If the bubble were below the box, then the southwest rule dictates another bottom leftmost bubble must exist. Therefore, at least $n$ final bubbles are in the box. If more than $n$ final bubbles are in this box, then there are some columns for which no vertical death ray yet exists. The bubble is invalid here. So, exactly $n$ final bubbles must be in the box.
Further, bubbles cannot be placed in a long row as the vertical ascension rule requires break where previous vertical death rays appear. The break must be exactly as long as the number of vertical death rays by the bubble gap rule, as just explained. The bubble location is uniquely determined, then.

$\blacksquare$





\end{document}