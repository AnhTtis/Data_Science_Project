\documentclass[10pt]{article}
\usepackage{amsmath}
\usepackage{amsthm}
\usepackage{amsfonts}
\usepackage{dsfont}
\usepackage{amssymb}
\usepackage{latexsym}
\usepackage{tensor}
%\usepackage{epsfig}
\usepackage{graphicx}
\usepackage{tikz}
\usetikzlibrary{cd}
%\usepackage[dvips]{graphicx}
\graphicspath{ {images/} }

\usepackage[matrix,tips,graph,curve]{xy}

\newcommand{\mnote}[1]{${}^*$\marginpar{\footnotesize ${}^*$#1}}
\linespread{1.065}

\makeatletter

\setlength\@tempdima  {5.5in}
\addtolength\@tempdima {-\textwidth}
\addtolength\hoffset{-0.5\@tempdima}
\setlength{\textwidth}{5.5in}
\setlength{\textheight}{8.75in}
\addtolength\voffset{-0.625in}

\makeatother

\makeatletter 
\@addtoreset{equation}{section}
\makeatother


\renewcommand{\theequation}{\thesection.\arabic{equation}}

\theoremstyle{plain}
\newtheorem{theorem}[equation]{Theorem}
\newtheorem{corollary}[equation]{Corollary}
\newtheorem{lemma}[equation]{Lemma}
\newtheorem{proposition}[equation]{Proposition}
\newtheorem{conjecture}[equation]{Conjecture}
\newtheorem{fact}[equation]{Fact}
\newtheorem{facts}[equation]{Facts}
\newtheorem*{theoremA}{Theorem A}
\newtheorem*{theoremB}{Theorem B}
\newtheorem*{theoremC}{Theorem C}
\newtheorem*{theoremD}{Theorem D}
\newtheorem*{theoremE}{Theorem E}
\newtheorem*{theoremF}{Theorem F}
\newtheorem*{theoremG}{Theorem G}
\newtheorem*{theoremH}{Theorem H}

\theoremstyle{definition}
\newtheorem{definition}[equation]{Definition}
\newtheorem{definitions}[equation]{Definitions}
%\theoremstyle{remark}

\newtheorem{remark}[equation]{Remark}
\newtheorem{remarks}[equation]{Remarks}
\newtheorem{exercise}[equation]{Exercise}
\newtheorem{example}[equation]{Example}
\newtheorem{examples}[equation]{Examples}
\newtheorem{notation}[equation]{Notation}
\newtheorem{question}[equation]{Question}
\newtheorem{assumption}[equation]{Assumption}
\newtheorem*{claim}{Claim}
\newtheorem{answer}[equation]{Answer}
%%%%%% letters %%%%

\newcommand{\fA}{\mathfrak{A}}
\newcommand{\fB}{\mathfrak{B}}
\newcommand{\fC}{\mathfrak{C}}
\newcommand{\fD}{\mathfrak{D}}
\newcommand{\fE}{\mathfrak{E}}
\newcommand{\fF}{\mathfrak{F}}
\newcommand{\fG}{\mathfrak{G}}
\newcommand{\fH}{\mathfrak{H}}
\newcommand{\fI}{\mathfrak{I}}
\newcommand{\fJ}{\mathfrak{J}}
\newcommand{\fK}{\mathfrak{K}}
\newcommand{\fL}{\mathfrak{L}}
\newcommand{\fM}{\mathfrak{M}}
\newcommand{\fN}{\mathfrak{N}}
\newcommand{\fO}{\mathfrak{O}}
\newcommand{\fP}{\mathfrak{P}}
\newcommand{\fQ}{\mathfrak{Q}}
\newcommand{\fR}{\mathfrak{R}}
\newcommand{\fS}{\mathfrak{S}}
\newcommand{\fT}{\mathfrak{T}}
\newcommand{\fU}{\mathfrak{U}}
\newcommand{\fV}{\mathfrak{V}}
\newcommand{\fW}{\mathfrak{W}}
\newcommand{\fX}{\mathfrak{X}}
\newcommand{\fY}{\mathfrak{Y}}
\newcommand{\fZ}{\mathfrak{Z}}

\newcommand{\fa}{\mathfrak{a}}
\newcommand{\fb}{\mathfrak{b}}
\newcommand{\fc}{\mathfrak{c}}
\newcommand{\fd}{\mathfrak{d}}
\newcommand{\fe}{\mathfrak{e}}
\newcommand{\ff}{\mathfrak{f}}
\newcommand{\fg}{\mathfrak{g}}
\newcommand{\fh}{\mathfrak{h}}
%\newcommand{\fi}{\mathfrak{i}}

\newcommand{\fj}{\mathfrak{j}}
\newcommand{\fk}{\mathfrak{k}}
\newcommand{\fl}{\mathfrak{l}}
\newcommand{\fm}{\mathfrak{m}}
\newcommand{\fn}{\mathfrak{n}}
\newcommand{\fo}{\mathfrak{o}}
\newcommand{\fp}{\mathfrak{p}}
\newcommand{\fq}{\mathfrak{q}}
\newcommand{\fr}{\mathfrak{r}}
\newcommand{\fs}{\mathfrak{s}}
\newcommand{\ft}{\mathfrak{t}}
\newcommand{\fu}{\mathfrak{u}}
\newcommand{\fv}{\mathfrak{v}}
\newcommand{\fw}{\mathfrak{w}}
\newcommand{\fx}{\mathfrak{x}}
\newcommand{\fy}{\mathfrak{y}}
\newcommand{\fz}{\mathfrak{z}}

\newcommand{\bi}{\textbf{i}}
\newcommand{\bj}{\textbf{j}}
\newcommand{\bk}{\textbf{k}}

\newcommand{\sA}{\mathcal{A}\,}
\newcommand{\sB}{\mathcal{B}\,}
\newcommand{\sC}{\mathcal{C}}
\newcommand{\sD}{\mathcal{D}\,}
\newcommand{\sE}{\mathcal{E}\,}
\newcommand{\sF}{\mathcal{F}\,}
\newcommand{\sG}{\mathcal{G}\,}
\newcommand{\sH}{\mathcal{H}}
\newcommand{\sI}{\mathcal{I}\,}
\newcommand{\sJ}{\mathcal{J}\,}
\newcommand{\sK}{\mathcal{K}\,}
\newcommand{\sL}{\mathcal{L}\,}
\newcommand{\sM}{\mathcal{M}\,}
\newcommand{\sN}{\mathcal{N}}
\newcommand{\sO}{\mathcal{O}}
\newcommand{\sP}{\mathcal{P}\,}
\newcommand{\sQ}{\mathcal{Q}\,}
\newcommand{\sR}{\mathcal{R}}
\newcommand{\sS}{\mathcal{S}}
\newcommand{\sT}{\mathcal{T}\,}
\newcommand{\sU}{\mathcal{U}\,}
\newcommand{\sV}{\mathcal{V}\,}
\newcommand{\sW}{\mathcal{W}\,}
\newcommand{\sX}{\mathcal{X}\,}
\newcommand{\sY}{\mathcal{Y}\,}
\newcommand{\sZ}{\mathcal{Z}\,}

\newcommand{\IA}{\mathbb{A}}
\newcommand{\IB}{\mathbb{B}}
\newcommand{\IC}{\mathbb{C}}
\newcommand{\ID}{\mathbb{D}}
\newcommand{\IE}{\mathbb{E}}
\newcommand{\IF}{\mathbb{F}}
\newcommand{\IG}{\mathbb{G}}
\newcommand{\IH}{\mathbb{H}}
\newcommand{\II}{\mathbb{I}}
\newcommand{\IK}{\mathbb{K}}
\newcommand{\IL}{\mathbb{L}}
\newcommand{\IM}{\mathbb{M}}
\newcommand{\IN}{\mathbb{N}}
\newcommand{\IO}{\mathbb{O}}
\newcommand{\IP}{\mathbb{P}}
\newcommand{\IQ}{\mathbb{Q}}
\newcommand{\IR}{\mathbb{R}}
\newcommand{\IS}{\mathbb{S}}
\newcommand{\IT}{\mathbb{T}}
\newcommand{\IU}{\mathbb{U}}
\newcommand{\IV}{\mathbb{V}}
\newcommand{\IW}{\mathbb{W}}
\newcommand{\IX}{\mathbb{X}}
\newcommand{\IY}{\mathbb{Y}}
\newcommand{\IZ}{\mathbb{Z}}

 \newcommand{\tA}{\mathrm {A}}
 \newcommand{\tB}{\mathrm {B}}
 \newcommand{\tC}{\mathrm {C}}
 \newcommand{\tD}{\mathrm {D}}
 \newcommand{\tE}{\mathrm {E}}
 \newcommand{\tF}{\mathrm {F}}
 \newcommand{\tG}{\mathrm {G}}
 \newcommand{\tH}{\mathrm {H}}
 \newcommand{\tI}{\mathrm {I}}
 \newcommand{\tJ}{\mathrm {J}}
 \newcommand{\tK}{\mathrm {K}}
 \newcommand{\tL}{\mathrm {L}}
 \newcommand{\tM}{\mathrm {M}}
 \newcommand{\tN}{\mathrm {N}}
 \newcommand{\tO}{\mathrm {O}}
 \newcommand{\tP}{\mathrm {P}}
 \newcommand{\tQ}{\mathrm {Q}}
 \newcommand{\tR}{\mathrm {R}}
 \newcommand{\tS}{\mathrm {S}}
 \newcommand{\tT}{\mathrm {T}}
 \newcommand{\tU}{\mathrm {U}}
 \newcommand{\tV}{\mathrm {V}}
 \newcommand{\tW}{\mathrm {W}}
 \newcommand{\tX}{\mathrm {X}}
 \newcommand{\tY}{\mathrm {Y}}
 \newcommand{\tZ}{\mathrm {Z}}
%%%%%%% macros %%%%%

%% my definitions %%%

\newcommand{\End}{\mathrm{End}}
\newcommand{\tr}{\mathrm{tr}}
%\newcommand{\ind}{\mathrm{ind}}

\renewcommand{\index}{\mathrm{index \,}}
\newcommand{\Hom}{\mathrm{Hom}}
\newcommand{\Aut}{\mathrm{Aut}}
\newcommand{\Trace}{\mathrm{Trace}\,}
\newcommand{\Res}{\mathrm{Res}\,}
\newcommand{\rank}{\mathrm{rank}}
%\renewcommand{\dim}{\mathrm{dim}}

\renewcommand{\deg}{\mathrm{deg}}
\newcommand{\spin}{\rm Spin}
\newcommand{\Spin}{\rm Spin}
\newcommand{\erfc}{\rm erfc\,}
\newcommand{\sgn}{{\rm sgn\,}}
\newcommand{\Spec}{\rm Spec\,}
\newcommand{\diag}{\rm diag\,}
\newcommand{\Fix}{\mathrm{Fix}}
\newcommand{\Ker}{\mathrm{Ker \,}}
\newcommand{\Coker}{\mathrm{Coker \,}}
\newcommand{\Sym}{\mathrm{Sym \,}}
\newcommand{\Hess}{\mathrm{Hess \,}}
\newcommand{\grad}{\mathrm{grad \,}}
\newcommand{\Center}{\mathrm{Center}}
\newcommand{\Lie}{\mathrm{Lie}}


\newcommand{\ch}{\rm ch} % Chern Character

\newcommand{\rk}{\rm rk} 
%\renewcommand{\c}{\rm c}  % Chern class

\newcommand{\sign}{\rm sign}
\renewcommand\dim{{\rm dim\,}}
\renewcommand\det{{\rm det\,}}
\newcommand{\dimKrull}{{\rm Krulldim\,}}
\newcommand\Rep{\mathrm{Rep}}
\newcommand\Hilb{\mathrm{Hilb}}
\newcommand\vol{\mathrm{vol}}

\newcommand\QED{\hfill $\Box$} %{\bf QED}} 

\newcommand\Pf{\nonintend{\em Proof. }}


\newcommand\reals{{\mathbb R}} 
\newcommand\complexes{{\mathbb C}}
\renewcommand\Re{\mathrm Re}
\renewcommand\Im{\mathrm Im}
\newcommand\integers{{\mathbb Z}}
\newcommand\quaternions{{\mathbb H}}


\newcommand\iso{{\cong}} 
%\newcommand\tensor{{\otimes}}
\newcommand\Tensor{{\bigotimes}} 
\newcommand\union{\bigcup} 
\newcommand\onehalf{\frac{1}{2}}
%\newcommand\Sym[1]{{Sym^{#1}(\complexes^2)}}

\newcommand\lie[1]{{\mathfrak #1}} 
\renewcommand\fk{\mathfrak{K}}
\newcommand\smooth{\mathcal{C}^{\infty}}
\newcommand\trivial{{\mathbb I}}
\newcommand\widebar{\overline}

%%%%%Delimiters%%%%

\newcommand{\<}{\langle}
\renewcommand{\>}{\rangle}

%\renewcommand{\(}{\left(}
%\renewcommand{\)}{\right)}


%%%% Different kind of derivatives %%%%%

\newcommand{\delbar}{\bar{\partial}}
\newcommand{\pdu}{\frac{\partial}{\partial u}}
%\newcommand{\pd}[1][2]{\frac{\partial #1}{\partial #2}}

%%%%% Arrows %%%%%
%\renewcommand{\ra}{\rightarrow}                   % right arrow
%\newcommand{\lra}{\longrightarrow}              % long right arrow
%\renewcommand{\la}{\leftarrow}                    % left arrow
%\newcommand{\lla}{\longleftarrow}               % long left arrow
%\newcommand{\ua}{\uparrow}                     % long up arrow
%\newcommand{\na}{\nearrow}                      %  NE arrow
%\newcommand{\llra}[1]{\stackrel{#1}{\lra}}      % labeled long right arrow
%\newcommand{\llla}[1]{\stackrel{#1}{\lla}}      % labeled long left arrow
%\newcommand{\lua}[1]{\stackrel{#1}{\ua}}      % labeled  up arrow
%\newcommand{\lna}[1]{\stackrel{#1}{\na}}      % labeled long NE arrow

\newcommand{\into}{\hookrightarrow}
\newcommand{\tto}{\longmapsto}
\def\llra{\longleftrightarrow}

\def\d/{/\mspace{-6.0mu}/}
\newcommand{\git}[3]{#1\d/_{\mspace{-4.0mu}#2}#3}
\newcommand\zetahilb{\zeta_{{\text{Hilb}}}}
\def\Fy{\sF \mspace{-3.0mu} \cdot \mspace{-3.0mu} y}
\def\tv{\tilde{v}}
\def\tw{\tilde{w}}
\def\wt{\widetilde}
\def\wtilde{\widetilde}
\def\what{\widehat}

%%%%%%%%%%%%%%%%%%% Mark's definitions %%%%%%%%%%%%%%%%%%%%

\newcommand{\frakg}{\mbox{\frakturfont g}}
\newcommand{\frakk}{\mbox{\frakturfont k}}
\newcommand{\frakp}{\mbox{\frakturfont p}}
\newcommand{\q}{\mbox{\frakturfont q}}
\newcommand{\frakn}{\mbox{\frakturfont n}}
\newcommand{\frakv}{\mbox{\frakturfont v}}
\newcommand{\fraku}{\mbox{\frakturfont u}}
\newcommand{\frakh}{\mbox{\frakturfont h}}
\newcommand{\frakm}{\mbox{\frakturfont m}}
\newcommand{\frakt}{\mbox{\frakturfont t}}
\newcommand{\G}{\Gamma}
\newcommand{\g}{\gamma}
\newcommand{\fraka}{\mbox{\frakturfont a}}
\newcommand{\db}{\bar{\partial}}
\newcommand{\dbs}{\bar{\partial}^*}
\newcommand{\p}{\partial}

%%%%%%%%%%%%% new definitions for the positive mass paper %%%%%%%%%

\newcommand{\sperp}{{\scriptscriptstyle \perp}}

%%%%%%%%%%%%%%%%%%%%%%%



%%%%%%%%%%%%%%%%%%%%%%%%%%%%%%%%%%%%%%%%%%%%%



%
\begin{document}
%

\title{Characterizing Permutation Bubble Diagrams}
\author{Ben Gillen and Jonathan Michala}


\date{\today}

\maketitle
\section{Introduction}

\section{Properties of Permutation Diagrams}

\begin{definition}
Assume a column of bubbles in the zero-th column.
\[ (0,w_j) = 1 \; \forall j \in \IZ_+ \]
These bubbles are labelled \textbf{\textit{addendum bubbles}} and are always appended to a diagram when considering the rules below. However, these bubbles are not counted towards the total number of bubbles in each row.
\end{definition}

\textbf{TODO: Show main example with this addition}

\begin{definition}
The \textbf{\textit{final bubble}} is defined as the last bubble in a row, where all cells afterwards are empty. Final bubbles will be counted with the following function. It returns a $1$ if the final bubble is at the $(i, w_j)$ location and a $0$ otherwise.
\[ R_{(i,w_j)} =  \begin{cases} 
1 & (i,w_j) = 1 \mid (n,w_j) = 0 \quad \forall n > i \\
0 & \text{otherwise}
\end{cases} \]
Since the addendum bubble exists, each row must contain a final bubble.
\end{definition}

\textbf{TODO: Point out all final bubbles in main example.}

\begin{definition}
Consider a column $l$ which contains $m_0$ final bubbles below row $w_j$. The \textbf{\textit{vertical ascension}} rule states that for row $w_j$ and above, there can be no bubbles in the $m_0$ columns to the right of column $l$. Further, if these $m_0$ columns contain $m_1$ final bubbles below row $w_j$, then there can be no bubbles in the $m_0+m_1$ columns in row $w_j$ and above. This property is recursively defined. If there are $m_{z+1}$ final bubbles in the $m_{z}$ columns below row $w_j$, then the $\sum_{s=1}^{z+1} m_s$ columns must be empty. %At some point there will be no final bubbles added because the permutations are finite.
\[
\sum_{r=1}^{j-1} R_{(l,w_r)} = m_0, \;  \sum_{q=l}^{\mu}\sum_{r=1}^{j-1} R_{(q,w_r)} = \sum_{s=0}^{z+1} m_{s} \quad (\mu = l+m_0+\dots+m_{z})
\]
\[
\implies (l+t,w_p) = 0 \quad \quad t,p \in \mathbb{N}, 0 < t \leq \sum_{s=1}^{z+1} m_s, \; p \geq j
\]

Final bubbles are added to create an empty region, and as more final bubbles fall into this region, the region is proportionally extended. Vertical lines of empty cells ascend because of these final bubbles.
\end{definition}

\textbf{TODO: Show main example satisfies this rule}

\begin{proposition}
Permutation diagrams satisfy the vertical ascension rule.
\end{proposition}

\begin{proof}
Take column $l$ with $m_0$ final bubbles below row $w_j$. Each of these final bubbles corresponds to a death ray. There must be a death ray in the succeeding $\sum_{s=1}^{z+1} m_s$ columns after column $l$ where
\[ \sum_{r=1}^{j-1} R_{(l,w_r)} = m_0, \;  \sum_{q=l}^{\mu}\sum_{r=1}^{j-1} R_{(q,w_r)} = \sum_{s=0}^{z+1} m_{s} \quad (\mu = l+m_0+\dots+m_{z}) \]
because multiple death rays are not allowed to be placed in the same column. Each death ray extends infinitely upwards, preventing bubbles from being placed. The vertical ascension rule is identical in disallowing bubbles from these exact same cells. Therefore, the vertical ascension rule is satisfied.
\end{proof}

\textbf{TODO: Show example diagram satisfying the VA rule but is not a permutation diagram.}

\begin{definition}
Consider a bubble $(i, w_j) = 1$ that is succeeded by $n$ empty cells 
\[(i+k, w_j) = 0 \quad \forall k \leq n, \quad k,n \in \IZ_+ \]
and a bubble after these empty cells $(i+(n+1), w_{j}) = 1$. Take a box bounded by the two aforementioned bubbles and the bottom of the diagram. Remove the rightmost column and topmost row.
\[ \mathcal{B}_{(i+(n+1), w_{j})} = [(i, w_{j-1}),(i,1)] \times [(i,w_{j-1}), (i+n, w_{j-1})] \]
The \textbf{\textit{empty cell gap}} rule states there are exactly $n$ final bubbles contained within box $\mathcal{B}$.
\end{definition}

\textbf{TODO: Show main example satisfies this rule}

\begin{proposition}
Permutation diagrams satisfy the empty cell gap rule.
\end{proposition}

\begin{proof}
Assume a bubble in the cell $(i_1,w_{j})$ and $(i_2+1,w_{j})$, where the $((i_2+1) - i_1) = n$ cells in between are all empty. The permutation diagram requires exactly $n$ death rays to begin between the $i_1$ and $i_2+1$ columns and below the $w_{j}$ row. Given that the $(i_1,w_{j})$ bubble exists,  a death ray cannot be located in the $i_1$ column below the $w_{j}$ row. Therefore, for every death ray beginning between the $i_1$ and $i_2+1$ columns, a final bubble is contained in the $i_1$ to $i_2-1$ column range. So, the empty cell gap rule is satisfied by the permutation diagram.
\end{proof}

\textbf{TODO: Show example diagram satisfying the ECG rule but is not a permutation diagram. Also, possibly show examples why the final bubble must be in the $i_1$ to $i_2-1$ range?}

\begin{definition}
Consider two procedures for labeling the bubbles in a bubble diagram with numbers.
In the first procedure, for the $n$th row, we label the bubbles from left to right $n,n+1,n+2,$ and so on.
In the second procedure, for the $n$th column, we label the bubbles from bottom to top $n, n+1,n+2,$ and so on.
We say a bubble diagram satisfies the \textbf{\textit{numbering condition}} if both of these procedures yield the same label for each bubble.
\end{definition}

\textbf{TODO: Show main example satisfies this condition}

\begin{proposition}
Permutation diagrams satisfy the numbering condition.
\end{proposition}

\begin{proof}
Let $(i,w_j)$ be a bubble in a permutation diagram.
In the first procedure, we label the bubble $w_j$ plus the number of bubbles below it, or in other words
$$w_j + \#\{(l,w_j) \colon l < j, w_l > w_j, l < i\}$$
Similarly, in the second procedure, we label the bubble
$$i + \#\{(i,w_k) \colon i < k, w_i > w_k, w_k < w_j\}$$
We need to prove that these two values are equivalent.
Since $(i,w_j)$ is a bubble, $i < j, w_i > w_j$, implying our question is reduced to 
$$i + \#\{(i,w_k) \colon i < k, w_k < w_j\} = w_j + \#\{(l,w_j) \colon l < i, w_l > w_j\}.$$
First, we sum up the number of bubbles and empty cells to the left of $(i,w_j)$ to get $w_j$: 
$$\#\{(i,w_k) \colon i < k, w_k < w_j\} + \#\{(i,w_k) \colon i > k, w_k < w_j\} + 1 = w_j$$
Then our question is reduced to
\begin{equation}
i = \#\{(i,w_k) \colon i > k, w_k < w_j\} + \#\{(l,w_j) \colon l < i, w_l > w_j\} + 1.
\label{eq:i}
\end{equation}
This is true, because for any $m < i$, we either have $w_m < w_j$ or $w_m > w_j$, implying that either $(i,w_m) \in \{(i,w_k) \colon i > k, w_k < w_j\}$ or $(m,w_j) \in \{(l,w_j) \colon l < i, w_l > w_j\}$ respectively.
So each $m < i$ gets counted once by the right hand side of \ref{eq:i}, therefore showing the equality.
\end{proof}

\textbf{TODO: Show example of a diagram that satisfies the numbering condition but is not a permutation diagram}

\begin{definition}
Consider the following two procedures to add dots to a bubble diagram.
In one procedure, start from the first row and go up, and in each row add a dot in the first cell to the right of all the bubbles in that row such that there is no dot below the newly placed dot.
In the other procedure, start from the left column and go to the right, and in each column add a dot in the first cell above all of the bubbles in that column such that there is no dot to the left of the newly placed dot.
We say that a bubble diagram satisfies the \textbf{\textit{dot condition}} if these two procedures result in the same placement of dots in the whole diagram
\end{definition}

\textbf{TODO: Show main example satisfies this condition}

\begin{proposition}
Permutation diagrams satisfy the dot condition
\end{proposition}

\begin{proof}
This is true, because in both procedures, we place dots exactly in the locations $(i,w_i)$ for all $i \in \IZ_+$.
In the first procedure, in the first row, we place bubbles in all the cells $(1,w_j)$ such that $1 < j$ and $w_1 > w_j$.
We know these two conditions are true for the first cells $(1,1),(1,2),...,(1,w_1 - 2),(1,w_1-1)$ and becomes false for $(1,w_1)$, which is therefore where the dot in this row must be placed by the dot procedure.
For the $k$th row, consider any cell $(k,w_j)$.
If $k > j$, then there is a dot at $(j,w_j)$ below the $k$th row and so there cannot be a dot in $(k,w_j)$ by the rules of the procedure.
If $w_k < w_j$, then $(k,w_k)$ is to the left of $(k,w_j)$ and would get a dot before $(k,w_j)$ according to the procedure, so there is no dot in $(k,w_j)$.
This is because we know there is no dot below $(k,w_k)$ by induction on the previous rows.
And if $k < j$ and $w_k > w_j$, then there is a bubble in this $(k,w_j)$.
Hence, if $j \neq k$, then there is no dot in the $(k,w_j)$th cell, and therefore the only cell in which a dot can be placed by the procedure is $(k,w_k)$.

In the second procedure, in the first column, we place bubbles in all the cells $(i,1)$ such that $i < j$ and $w_i > 1$ for $j$ the index giving $w_j = 1$.
We know these two conditions are true for the first cells $(1,1)(2,1),...,(j-2,1),(j-1,1)$ and becomes false for $(j,1)$ which is therefore where the dot in this column must be placed by this dot procedure.
For the $w_k$th column, consider any cell $(i,w_k)$.
If $i < k$ and $w_i > w_k$, then there is a bubble in this cell.
If $i > k$ then $(k,w_k)$ is below $(i,w_k)$ and would get a dot before $(i,w_k)$ by the procedure, so there is no dot in $(i,w_k)$.
This is because we know there is no dot to the left of $(k,w_k)$ by induction on the previous columns.
If $w_i < w_k$, then there is a dot at $(i,w_i)$ to the left of the $w_k$th column and so there cannot be a dot in $(i,w_k)$ by the rules of the procedure.
Hence, if $i \neq k$, then there is no dot in the $(i,w_k)$th cell, and therefore the only cell in which a dot can be placed by the procedure is $(k,w_k)$.

We have then shown that all dots are placed in the same cells by both procedures, so any permutation bubble diagram satisfies the dot condition.
\end{proof}

\textbf{TODO: Show example of a diagram that satisfies the dot condition but is not a permutation diagram}

\section{Conditions for Building Permutation Diagrams}

\begin{theorem}
A bubble diagram satisfies the dot condition and the numbering condition if and only if it is a permutation bubble diagram.
\label{thm:DC+NC}
\end{theorem}

\begin{proof}
Consider an arbitrary bubble diagram that satisfies the dot and numbering conditions.
Place the dots on the diagram and let $(1,w_1),(2,w_2),...$ denote their positions.
In fact, since the dots are never in the same column, we are able to refer to the $w_j$th column and therefore the $(i,w_j)$th position without confusion.
We claim that a bubble is placed in the position $(i,w_j)$ exactly when $i < j$ and $w_i > w_j$, which shows that this diagram is a permutation bubble diagram corresponding to the permutation $w_1,w_2,...$

Assume there is a bubble in position $(i,w_j)$.
By the dot condition there must be a dot above and a dot to the right of this position.
The dot above is in position $(j,w_j)$ meaning $i < j$, and the dot to the right is in position $(i,w_i)$ meaning $w_i > w_j$.

Conversely, consider a position $(i,w_j)$ such that $i < j$ and $w_i > w_j$.
We claim there must be a bubble in any such position, and we prove it by induction on $i$.
In the base case, the positions to the left of the dot at $(1,w_1)$ are the only ones to satisfy the assumption.
And indeed we must place a bubble in each of these positions.
For if not, then using the vertical dot-placing method, we would place a dot in a position to the left of $(1,w_1)$, violating the dot condition.

Now, let's assume we've shown that there is a bubble in every position $(i,w_j)$ such that $i < j$ and $w_i > w_j$ for all $i < I$, for some $I > 1$.
Consider a position $(I,w_j)$ with $I < j$ and $w_I > w_j$, and assume there is no bubble in this position.
Then, there must be some bubble $A$ in position $(I,w_{j'})$ for some $w_j < w_{j'} < w_{I}$, for if not then the horizontal method would place a dot in a different position than $(I,w_I)$ contradicting the dot condition.
There must be a dot above $A$ by the dot condition.
Then by induction, we see that the number of bubbles below $A$ is equal to the number of dots to the right of the $w_{j'}$th column and below the $I$th row, call this value $d$.
Then by the numbering condition, we would label $A$ with the number $w_{j'} + d$ when counting vertically.

However, we get something different if we count horizontally.
There are $I - 1 - d$ dots to the left of the $w_{j'}$th column and below the $I$th row, none of which are in the $w_j$th column.
That means there are at least $I - d$ empty cells to the left of $A$, since $(I,w_j)$ is empty.
If we count horizontally, then we would label $A$ with the number
$$I + \#\{\text{positions to the left of }A\} - \#\{\text{empty positions to the left of }A\}.$$
This is at most
$$I + (w_{j'} - 1) - (I - d) = w_{j'} + d - 1,$$ 
which is strictly less than our previous label.
This contradicts the numbering condition and our assumption that $(I,w_j)$ contains no bubble must be false.
Hence, this diagram is in fact the permutation bubble diagram for the permutation $w_1,w_2,...$, showing that any diagram satisfying the dot and numbering conditions is a permutation diagram.
\end{proof}


\begin{proposition}
A bubble diagram satisfies the dot condition and the southwest condition if and only if it is a permutation bubble diagram.
\end{proposition}

\begin{proof}
\textbf{(TODO: need to reword)} First, assume we have a permutation bubble diagram.
We claim this diagram satisfies the southwest condition.
Let $(i,w_j)$ and $(i+k,w_j - l)$ be the location of two bubbles in the diagram, for $k,l \in \IZ_+$.
Let $j'$ be the index such that $w_{j'} = w_j - l$.
This means that $i < j$, $w_i > w_j$, $i+k < j'$, and $w_{i+k} > w_{j'}$.
Then $i < i+k < j'$ and $w_i > w_j > w_j - l = w_{j'}$, implying there is a bubble placed at $(i,w_{j'})$.

And we have already seen that permutation diagrams satisfy the dot condition.
Now, we consider the converse.


Let $a_1,...,a_n$ be a sequence of nonnegative integers.
We claim that there is only one diagram with $a_i$ bubbles in the $i$th row for all $i$ that satisfies the dot and southwest conditions.
To satisfy the dot condition in the first row, we must have $a_1$ bubbles placed in the first $a_1$ positions, with a dot in the $(a_1+1)$th position.
For if not, then we would place a dot to the right of all these bubbles, but in the other procedure, we would place a dot in the first empty position in this row.

Now assume the first $k-1$ rows have been determined with bubbles and dots placed uniquely.
We claim that the $a_k$ bubbles and dot must be placed in the first $a_k + 1$ columns that do not contain a dot.
Indeed, if this is not the case, then there is an empty cell in the $k$th row with bubbles to the right and no dots below it.
For the dot condition to be satisfied, we must have the dot in this column placed above the $k$th row, since there are bubbles to the right and the dot in the $k$th row must be placed to the right of them.
But with respect to the other procedure, we cannot place this column's dot above the $k$th row because the southwest condition prohibits any bubbles from being placed above the $k$th row, and the dot is placed in the first position above all the bubbles in the column such that there is no dot to the left.
Thus, the empty cell satisfies these properties, so no dot could be placed above it.
This is a contradiction, so there is only one way to place the $a_k$ bubbles and dot in the $k$th row.

Hence, diagrams that satisfy the dot and southwest condition are unique up to the number of bubbles in each row. And since the permutation diagrams satisfy the dot and southwest conditions and are also unique in this way, they are the same diagrams.
\end{proof}

\begin{proposition}
Arbitrary bubble diagrams that satisfy the cell gap and vertical ascension rules are unique, given a number of bubbles for each row.
\end{proposition}

\begin{proof}
In order to prove that these two rules determine a unique bubble diagram (up to rows), we use induction on those rows, starting with the leftmost bubble in each row and moving right.
\\ \par
\textit{Row $1$:} If there are no bubbles in row one $(i,w_1) = 0 \; \forall j \in \IZ_+$, then the row is all empty cells (besides the addendum bubble). If $\sum_{i=1}^{\infty} (i,w_1) = m, \; m > 0$, then the row cannot start with any number of empty cells. These empty cells would violate the empty cell gap rule (given the addendum bubble). There are no rows below the first row, so any gap will necessary have an insufficient number of bubbles. The same argument applies to the rest of the row. Any gaps in the first row of bubbles are prevented by the lack of lower final bubbles. Therefore, for a given number of bubbles, row one is unique.
\\ \par
\textit{Row $k+1$:} If there are no bubbles in row $k+1$
\[ (i,w_{k+1}) = 0 \; \forall i \]
then the row is full of empty cells (besides the addendum bubble, of course). On the other hand, if $\sum_{i=1}^{\infty} (i,w_{k+1}) = m, \; m > 0$, start with the leftmost column, i.e. the zeroth column. We know by the addendum it is always filled. Take cell $c = (i_2+1,w_{k+1})$. Either it is preceded by a bubble, or it is preceded by an empty cell. Take each of these scenarios in turn.
\\ \par
First, assume that cell $c$ is directly preceded by a bubble $(i_2,w_{k+1}) = 1$. Either the vertical ascension rule prevents a bubble from being placed in $c$, or it doesn't. If the vertical ascension rule applies, the cell must be empty. If the vertical ascension rule doesn't apply, assume for contradiction that cell $c$ is left empty. Then, for cell $d_0 = (i_2+2,w_{k+1})$ to be a filled with a bubble, there must be exactly one final bubble in box $\mathcal{B}_{d_0}$, fulfilling the empty cell gap rule. If $\mathcal{B}_{d_0}$ does have a final bubble, then that final bubble could not be in the $i_2$ column because we assumed that the vertical ascension rule does not apply to cell $c$. So, the final bubble must be in box $\mathcal{B}_{d_0}$'s other column, the $i_2+1$ column. A final bubble in this column, however, means that cell $d_0$ would violate the vertical ascension rule if filled. Therefore, $d_0$ must be an empty cell.
\\ \par
Next, take cell $d_{a+1} = (i_2+(a+3),w_{k+1}), \; a \geq 0$. There must be $a+2$ final bubbles in the $\mathcal{B}_{d_{a+1}}$ box for $d_{a+1}$ to fulfill the empty cell gap rule. For each column between $i_2$ and $i_2+(a+3)$, the final bubble count of box $\mathcal{B}_{d_{a+1}}$ can increase by $n$. However, the vertical ascension rule states that the number of empty cells must correspondingly increase by $n$ for each of these rows. Since the first empty cell did not have a final bubble to justify its existence, when the vertical ascension rule is fulfilled, box $\mathcal{B}_{d_{a+1}}$ will always be at least one final bubble short. Therefore, there are bubbles left to place but all remaining locations in the row do not fulfill one or both of the rules. So, by contradiction, if the vertical ascension rule doesn't apply when a cell $c$ is preceded by a bubble, then $c$ must be a bubble. Bubble locations for an arbitrary diagram are then uniquely determined when those bubbles are placed directly after other bubbles.
%no bubbles are allowed to be placed in the $n$ rows to the right of these final bubbles. Therefore, if $d_{a+1}$ is not at least $n+1$ rows to the right of a given column with $n$ final bubbles, and no more final bubbles were introduced in these $n$ columns, then $d_{a+1}$ must be empty. If $m$ more final bubbles were introduced in these $n$ columns, then $d_{a+1}$ must be at least another $m$ columns away. On the other hand, if $d_{a+1}$ is at least $n+1$ rows to the right of a given column with $n$ final bubbles, then the number of final bubbles in box $\mathcal{B}_{d_{a+1}}$ has only increased by $n$ for $n$ more gaps. The same holds respectively true in the case where $m$ more final bubbles are introduced. Therefore, either the gap still has an insufficient number of final bubbles, or it has a sufficient number of final bubbles, but at least one of those bubbles is in column $i_2+(a+3)$ which would break the vertical ascension rule. 
\\ \par
Conversely, take a cell $c = (i,w_j)$ preceded by at least one empty cell. Assume without loss of generality that $c$ is preceded by $n$ empty cells.
\[(i-k,w_j) = 0 \quad \forall k \in \IZ_+, \; k \leq n \]
Either $c$ fulfills the empty cell gap rule, or it doesn't. If it does, then by the logic presented above, a bubble must be placed in cell $c$. Otherwise, the empty cell gap rule will never again be fulfilled (or fulfilled only in the case where the vertical ascension rule is broken), and the row will not have enough bubbles in it. On the other hand, assume cell $c$ does not satisfy the empty cell gap rule. Note that the gap must have originally been created because of an application of the vertical ascension rule. Then, at least one final bubble is in the box $\mathcal{B}_c$ created by this gap. If there are $m$ final bubbles in the $i-n$ column, the gap must (by the vertical ascension rule) be at least $m$ empty cells long. However, if more final bubbles appeared after the $i-n$ column or if $m<n$, then box $\mathcal{B}_c$ may still have too many final bubbles and not enough gaps. Since the bubbles on a graph are finite, at some point the number of final bubbles will stop increasing. Then, there exists some number of gaps which will equal the number of final bubbles in box $\mathcal{B}$. Therefore, there is always a unique place after a gap for a bubble. Before this point, there will be too many empty cells as compared to final bubbles. After this point, there will always be insufficient final bubbles or a contradiction with the vertical ascension rule. Note that it is impossible for a graph to have too many required gaps from the vertical ascension rule and not enough final bubbles for the empty cell gap rule. So, the two rules can always create a unique bubble diagram.
\end{proof}


\begin{theorem}
A bubble diagram satisfies the vertical ascension and empty cell gap rules if and only if it is a permutation bubble diagram.
\end{theorem}

\begin{proof}
Given a unique number of bubbles for each row, a unique permutation diagram is created. This permutation diagram satisfies the empty cell gap and vertical ascension rules. These two rules also correspond to a unique bubble diagram given a unique number of bubbles in each row. Therefore, these two diagrams must always be identical, and so are bijective.
\end{proof}



%Old version: The \textbf{\textit{vertical ascension}} rule states that for row $w_j$ and above, there can be no bubbles in the $m_0$ columns to the right of column $l$. Further, if these $m_0$ columns contain $m_1$ final bubbles below row $w_j$, then there can be no bubbles in the $m_0+m_1$ columns in row $w_j$ and above. This property is recursively defined. If there are $m_{z+1}$ final bubbles in the $m_{z}$ columns below row $w_j$, then the $\sum_{s=1}^{z+1} m_s$ columns must be empty. %At some point there will be no final bubbles added because the permutations are finite.


\[
\sum_{r=1}^{j-1} R_{(l,w_r)} = m_0, \;  \sum_{q=l}^{\mu}\sum_{r=1}^{j-1} R_{(q,w_r)} = \sum_{s=0}^{z+1} m_{s} \quad (\mu = l+m_0+\dots+m_{z})
\]
\[
\implies (l+t,w_p) = 0 \quad \quad t,p \in \mathbb{N}, 0 < t \leq \sum_{s=1}^{z+1} m_s, \; p \geq j
\]

\[ \sum_{r=1}^{j-1} R_{(l,w_r)} = m_0, \;  \sum_{q=l}^{\mu}\sum_{r=1}^{j-1} R_{(q,w_r)} = \sum_{s=0}^{z+1} m_{s} \quad (\mu = l+m_0+\dots+m_{z}) \]

\end{document}