\documentclass[10pt]{article}
\usepackage{amsmath}
\usepackage{amsthm}
\usepackage{amsfonts}
\usepackage{dsfont}
\usepackage{amssymb}
\usepackage{latexsym}
\usepackage{tensor}
%\usepackage{epsfig}
\usepackage{graphicx}
\usepackage{tikz}
\usetikzlibrary{cd}
%\usepackage[dvips]{graphicx}
\graphicspath{ {images/} }

\usepackage[matrix,tips,graph,curve]{xy}

\newcommand{\mnote}[1]{${}^*$\marginpar{\footnotesize ${}^*$#1}}
\linespread{1.065}

\makeatletter

\setlength\@tempdima  {5.5in}
\addtolength\@tempdima {-\textwidth}
\addtolength\hoffset{-0.5\@tempdima}
\setlength{\textwidth}{5.5in}
\setlength{\textheight}{8.75in}
\addtolength\voffset{-0.625in}

\makeatother

\makeatletter 
\@addtoreset{equation}{section}
\makeatother


\renewcommand{\theequation}{\thesection.\arabic{equation}}

\theoremstyle{plain}
\newtheorem{theorem}[equation]{Theorem}
\newtheorem{corollary}[equation]{Corollary}
\newtheorem{lemma}[equation]{Lemma}
\newtheorem{proposition}[equation]{Proposition}
\newtheorem{conjecture}[equation]{Conjecture}
\newtheorem{fact}[equation]{Fact}
\newtheorem{facts}[equation]{Facts}
\newtheorem*{theoremA}{Theorem A}
\newtheorem*{theoremB}{Theorem B}
\newtheorem*{theoremC}{Theorem C}
\newtheorem*{theoremD}{Theorem D}
\newtheorem*{theoremE}{Theorem E}
\newtheorem*{theoremF}{Theorem F}
\newtheorem*{theoremG}{Theorem G}
\newtheorem*{theoremH}{Theorem H}

\theoremstyle{definition}
\newtheorem{definition}[equation]{Definition}
\newtheorem{definitions}[equation]{Definitions}
%\theoremstyle{remark}

\newtheorem{remark}[equation]{Remark}
\newtheorem{remarks}[equation]{Remarks}
\newtheorem{exercise}[equation]{Exercise}
\newtheorem{example}[equation]{Example}
\newtheorem{examples}[equation]{Examples}
\newtheorem{notation}[equation]{Notation}
\newtheorem{question}[equation]{Question}
\newtheorem{assumption}[equation]{Assumption}
\newtheorem*{claim}{Claim}
\newtheorem{answer}[equation]{Answer}
%%%%%% letters %%%%

\newcommand{\fA}{\mathfrak{A}}
\newcommand{\fB}{\mathfrak{B}}
\newcommand{\fC}{\mathfrak{C}}
\newcommand{\fD}{\mathfrak{D}}
\newcommand{\fE}{\mathfrak{E}}
\newcommand{\fF}{\mathfrak{F}}
\newcommand{\fG}{\mathfrak{G}}
\newcommand{\fH}{\mathfrak{H}}
\newcommand{\fI}{\mathfrak{I}}
\newcommand{\fJ}{\mathfrak{J}}
\newcommand{\fK}{\mathfrak{K}}
\newcommand{\fL}{\mathfrak{L}}
\newcommand{\fM}{\mathfrak{M}}
\newcommand{\fN}{\mathfrak{N}}
\newcommand{\fO}{\mathfrak{O}}
\newcommand{\fP}{\mathfrak{P}}
\newcommand{\fQ}{\mathfrak{Q}}
\newcommand{\fR}{\mathfrak{R}}
\newcommand{\fS}{\mathfrak{S}}
\newcommand{\fT}{\mathfrak{T}}
\newcommand{\fU}{\mathfrak{U}}
\newcommand{\fV}{\mathfrak{V}}
\newcommand{\fW}{\mathfrak{W}}
\newcommand{\fX}{\mathfrak{X}}
\newcommand{\fY}{\mathfrak{Y}}
\newcommand{\fZ}{\mathfrak{Z}}

\newcommand{\fa}{\mathfrak{a}}
\newcommand{\fb}{\mathfrak{b}}
\newcommand{\fc}{\mathfrak{c}}
\newcommand{\fd}{\mathfrak{d}}
\newcommand{\fe}{\mathfrak{e}}
\newcommand{\ff}{\mathfrak{f}}
\newcommand{\fg}{\mathfrak{g}}
\newcommand{\fh}{\mathfrak{h}}
%\newcommand{\fi}{\mathfrak{i}}

\newcommand{\fj}{\mathfrak{j}}
\newcommand{\fk}{\mathfrak{k}}
\newcommand{\fl}{\mathfrak{l}}
\newcommand{\fm}{\mathfrak{m}}
\newcommand{\fn}{\mathfrak{n}}
\newcommand{\fo}{\mathfrak{o}}
\newcommand{\fp}{\mathfrak{p}}
\newcommand{\fq}{\mathfrak{q}}
\newcommand{\fr}{\mathfrak{r}}
\newcommand{\fs}{\mathfrak{s}}
\newcommand{\ft}{\mathfrak{t}}
\newcommand{\fu}{\mathfrak{u}}
\newcommand{\fv}{\mathfrak{v}}
\newcommand{\fw}{\mathfrak{w}}
\newcommand{\fx}{\mathfrak{x}}
\newcommand{\fy}{\mathfrak{y}}
\newcommand{\fz}{\mathfrak{z}}

\newcommand{\bi}{\textbf{i}}
\newcommand{\bj}{\textbf{j}}
\newcommand{\bk}{\textbf{k}}

\newcommand{\sA}{\mathcal{A}\,}
\newcommand{\sB}{\mathcal{B}\,}
\newcommand{\sC}{\mathcal{C}}
\newcommand{\sD}{\mathcal{D}\,}
\newcommand{\sE}{\mathcal{E}\,}
\newcommand{\sF}{\mathcal{F}\,}
\newcommand{\sG}{\mathcal{G}\,}
\newcommand{\sH}{\mathcal{H}}
\newcommand{\sI}{\mathcal{I}\,}
\newcommand{\sJ}{\mathcal{J}\,}
\newcommand{\sK}{\mathcal{K}\,}
\newcommand{\sL}{\mathcal{L}\,}
\newcommand{\sM}{\mathcal{M}\,}
\newcommand{\sN}{\mathcal{N}}
\newcommand{\sO}{\mathcal{O}}
\newcommand{\sP}{\mathcal{P}\,}
\newcommand{\sQ}{\mathcal{Q}\,}
\newcommand{\sR}{\mathcal{R}}
\newcommand{\sS}{\mathcal{S}}
\newcommand{\sT}{\mathcal{T}\,}
\newcommand{\sU}{\mathcal{U}\,}
\newcommand{\sV}{\mathcal{V}\,}
\newcommand{\sW}{\mathcal{W}\,}
\newcommand{\sX}{\mathcal{X}\,}
\newcommand{\sY}{\mathcal{Y}\,}
\newcommand{\sZ}{\mathcal{Z}\,}

\newcommand{\IA}{\mathbb{A}}
\newcommand{\IB}{\mathbb{B}}
\newcommand{\IC}{\mathbb{C}}
\newcommand{\ID}{\mathbb{D}}
\newcommand{\IE}{\mathbb{E}}
\newcommand{\IF}{\mathbb{F}}
\newcommand{\IG}{\mathbb{G}}
\newcommand{\IH}{\mathbb{H}}
\newcommand{\II}{\mathbb{I}}
\newcommand{\IK}{\mathbb{K}}
\newcommand{\IL}{\mathbb{L}}
\newcommand{\IM}{\mathbb{M}}
\newcommand{\IN}{\mathbb{N}}
\newcommand{\IO}{\mathbb{O}}
\newcommand{\IP}{\mathbb{P}}
\newcommand{\IQ}{\mathbb{Q}}
\newcommand{\IR}{\mathbb{R}}
\newcommand{\IS}{\mathbb{S}}
\newcommand{\IT}{\mathbb{T}}
\newcommand{\IU}{\mathbb{U}}
\newcommand{\IV}{\mathbb{V}}
\newcommand{\IW}{\mathbb{W}}
\newcommand{\IX}{\mathbb{X}}
\newcommand{\IY}{\mathbb{Y}}
\newcommand{\IZ}{\mathbb{Z}}

 \newcommand{\tA}{\mathrm {A}}
 \newcommand{\tB}{\mathrm {B}}
 \newcommand{\tC}{\mathrm {C}}
 \newcommand{\tD}{\mathrm {D}}
 \newcommand{\tE}{\mathrm {E}}
 \newcommand{\tF}{\mathrm {F}}
 \newcommand{\tG}{\mathrm {G}}
 \newcommand{\tH}{\mathrm {H}}
 \newcommand{\tI}{\mathrm {I}}
 \newcommand{\tJ}{\mathrm {J}}
 \newcommand{\tK}{\mathrm {K}}
 \newcommand{\tL}{\mathrm {L}}
 \newcommand{\tM}{\mathrm {M}}
 \newcommand{\tN}{\mathrm {N}}
 \newcommand{\tO}{\mathrm {O}}
 \newcommand{\tP}{\mathrm {P}}
 \newcommand{\tQ}{\mathrm {Q}}
 \newcommand{\tR}{\mathrm {R}}
 \newcommand{\tS}{\mathrm {S}}
 \newcommand{\tT}{\mathrm {T}}
 \newcommand{\tU}{\mathrm {U}}
 \newcommand{\tV}{\mathrm {V}}
 \newcommand{\tW}{\mathrm {W}}
 \newcommand{\tX}{\mathrm {X}}
 \newcommand{\tY}{\mathrm {Y}}
 \newcommand{\tZ}{\mathrm {Z}}
%%%%%%% macros %%%%%

%% my definitions %%%

\newcommand{\End}{\mathrm{End}}
\newcommand{\tr}{\mathrm{tr}}
%\newcommand{\ind}{\mathrm{ind}}

\renewcommand{\index}{\mathrm{index \,}}
\newcommand{\Hom}{\mathrm{Hom}}
\newcommand{\Aut}{\mathrm{Aut}}
\newcommand{\Trace}{\mathrm{Trace}\,}
\newcommand{\Res}{\mathrm{Res}\,}
\newcommand{\rank}{\mathrm{rank}}
%\renewcommand{\dim}{\mathrm{dim}}

\renewcommand{\deg}{\mathrm{deg}}
\newcommand{\spin}{\rm Spin}
\newcommand{\Spin}{\rm Spin}
\newcommand{\erfc}{\rm erfc\,}
\newcommand{\sgn}{{\rm sgn\,}}
\newcommand{\Spec}{\rm Spec\,}
\newcommand{\diag}{\rm diag\,}
\newcommand{\Fix}{\mathrm{Fix}}
\newcommand{\Ker}{\mathrm{Ker \,}}
\newcommand{\Coker}{\mathrm{Coker \,}}
\newcommand{\Sym}{\mathrm{Sym \,}}
\newcommand{\Hess}{\mathrm{Hess \,}}
\newcommand{\grad}{\mathrm{grad \,}}
\newcommand{\Center}{\mathrm{Center}}
\newcommand{\Lie}{\mathrm{Lie}}


\newcommand{\ch}{\rm ch} % Chern Character

\newcommand{\rk}{\rm rk} 
%\renewcommand{\c}{\rm c}  % Chern class

\newcommand{\sign}{\rm sign}
\renewcommand\dim{{\rm dim\,}}
\renewcommand\det{{\rm det\,}}
\newcommand{\dimKrull}{{\rm Krulldim\,}}
\newcommand\Rep{\mathrm{Rep}}
\newcommand\Hilb{\mathrm{Hilb}}
\newcommand\vol{\mathrm{vol}}

\newcommand\QED{\hfill $\Box$} %{\bf QED}} 

\newcommand\Pf{\nonintend{\em Proof. }}


\newcommand\reals{{\mathbb R}} 
\newcommand\complexes{{\mathbb C}}
\renewcommand\Re{\mathrm Re}
\renewcommand\Im{\mathrm Im}
\newcommand\integers{{\mathbb Z}}
\newcommand\quaternions{{\mathbb H}}


\newcommand\iso{{\cong}} 
%\newcommand\tensor{{\otimes}}
\newcommand\Tensor{{\bigotimes}} 
\newcommand\union{\bigcup} 
\newcommand\onehalf{\frac{1}{2}}
%\newcommand\Sym[1]{{Sym^{#1}(\complexes^2)}}

\newcommand\lie[1]{{\mathfrak #1}} 
\renewcommand\fk{\mathfrak{K}}
\newcommand\smooth{\mathcal{C}^{\infty}}
\newcommand\trivial{{\mathbb I}}
\newcommand\widebar{\overline}

%%%%%Delimiters%%%%

\newcommand{\<}{\langle}
\renewcommand{\>}{\rangle}

%\renewcommand{\(}{\left(}
%\renewcommand{\)}{\right)}


%%%% Different kind of derivatives %%%%%

\newcommand{\delbar}{\bar{\partial}}
\newcommand{\pdu}{\frac{\partial}{\partial u}}
%\newcommand{\pd}[1][2]{\frac{\partial #1}{\partial #2}}

%%%%% Arrows %%%%%
%\renewcommand{\ra}{\rightarrow}                   % right arrow
%\newcommand{\lra}{\longrightarrow}              % long right arrow
%\renewcommand{\la}{\leftarrow}                    % left arrow
%\newcommand{\lla}{\longleftarrow}               % long left arrow
%\newcommand{\ua}{\uparrow}                     % long up arrow
%\newcommand{\na}{\nearrow}                      %  NE arrow
%\newcommand{\llra}[1]{\stackrel{#1}{\lra}}      % labeled long right arrow
%\newcommand{\llla}[1]{\stackrel{#1}{\lla}}      % labeled long left arrow
%\newcommand{\lua}[1]{\stackrel{#1}{\ua}}      % labeled  up arrow
%\newcommand{\lna}[1]{\stackrel{#1}{\na}}      % labeled long NE arrow

\newcommand{\into}{\hookrightarrow}
\newcommand{\tto}{\longmapsto}
\def\llra{\longleftrightarrow}

\def\d/{/\mspace{-6.0mu}/}
\newcommand{\git}[3]{#1\d/_{\mspace{-4.0mu}#2}#3}
\newcommand\zetahilb{\zeta_{{\text{Hilb}}}}
\def\Fy{\sF \mspace{-3.0mu} \cdot \mspace{-3.0mu} y}
\def\tv{\tilde{v}}
\def\tw{\tilde{w}}
\def\wt{\widetilde}
\def\wtilde{\widetilde}
\def\what{\widehat}

%%%%%%%%%%%%%%%%%%% Mark's definitions %%%%%%%%%%%%%%%%%%%%

\newcommand{\frakg}{\mbox{\frakturfont g}}
\newcommand{\frakk}{\mbox{\frakturfont k}}
\newcommand{\frakp}{\mbox{\frakturfont p}}
\newcommand{\q}{\mbox{\frakturfont q}}
\newcommand{\frakn}{\mbox{\frakturfont n}}
\newcommand{\frakv}{\mbox{\frakturfont v}}
\newcommand{\fraku}{\mbox{\frakturfont u}}
\newcommand{\frakh}{\mbox{\frakturfont h}}
\newcommand{\frakm}{\mbox{\frakturfont m}}
\newcommand{\frakt}{\mbox{\frakturfont t}}
\newcommand{\G}{\Gamma}
\newcommand{\g}{\gamma}
\newcommand{\fraka}{\mbox{\frakturfont a}}
\newcommand{\db}{\bar{\partial}}
\newcommand{\dbs}{\bar{\partial}^*}
\newcommand{\p}{\partial}

%%%%%%%%%%%%% new definitions for the positive mass paper %%%%%%%%%

\newcommand{\sperp}{{\scriptscriptstyle \perp}}

%%%%%%%%%%%%%%%%%%%%%%%



%%%%%%%%%%%%%%%%%%%%%%%%%%%%%%%%%%%%%%%%%%%%%



%
\begin{document}
%

\title{Characterizing Permutation Bubble Diagrams}
\author{Ben Gillen and Jonathan Michala}


\date{\today}

\maketitle

\section{Dot Proof}

\begin{theorem}
A bubble diagram satisfies the dot condition and the numbering condition if and only if it is a permutation bubble diagram.
\label{thm:DC+NC}
\end{theorem}

To prove this theorem, we show that both permutation diagrams and diagrams that satisfy the dot and numbering conditions are unique up to the number of bubbles in each row.

\begin{lemma}
For every finite sequence of nonnegative integers $a_1,...,a_n$, there exists a permutation such that its bubble diagram contains $a_i$ bubbles in the $i$th row for all $i = 1,...,n$.
\end{lemma}

\begin{proof}
INCOMPLETE
\end{proof}

\begin{lemma}
If a bubble diagram satisfies the dot condition and the numbering condition, then it is unique up to how many bubbles are in each row.
\label{lem:DC+NC=>unique up to rows}
\end{lemma}

\begin{proof}
We again induct on the rows.
The first row is determined just based on the dot condition or the numbering condition independently.
Now, assume for the first $i-1$ rows, the bubbles have been placed uniquely so as not to violate the dot or the numbering conditions.
We claim that we must place the $a_i$ bubbles in the first $a_i$ columns that do not yet contain a dot.
Note first that we cannot place any bubbles above any dots by the dot condition.
Now, assume we can do some other configuration than the one in our claim.
Then there will be an empty cell $(i,w_j)$ in a column with no dot below it and a bubble to the right in some $(i, w_j + k)$, where we'll denote $w_{j'} := w_j + k$.

If there is an empty cell below the bubble in the $w_{j'}$th column, then there must be a dot to the left of the empty cell.
For if there were no dot to the left, then there would be a dot to the right.
This means there is a bubble to the right as well, because if not, then a dot would be placed in the empty cell.
Since there is a bubble to the right, by the uniqueness of the constructed row, there must be a dot below the empty cell prohibiting a bubble from being placed in the empty cell.
This is a contradiction, so there must be a dot to the left of the empty cell.
Hence, the dots to the left correspond exactly to the empty cells below $(i,w_{j'})$.

Let $d$ be the number of dots to the left of the $w_{j'}$ column and below the $i$th row equals $i - 1$.
Then the number of bubbles below $(i,w_{j'})$ equals $i - 1 - d$.
And therefore, we must label the bubble $(i,w_{j'})$, going from bottom to top, with the number $w_{j'} + i - 1 - d$.
Now, consider the number of bubbles to the left of $(i,w_{j'})$.
Since there can be no bubble above a dot, and since there is at least one extra empty cell with no dot below it, the number of bubbles to the left must be strictly less than $w_{j'} -1 - d$.
Hence, we must label the bubble $(i,w_{j'})$, going left to right, with a number strictly less than $i + w_{j'} - 1 - d$, contradicting our previous label.
Therefore, our assumption that we could do anything different than placing the $a_i$ bubbles in the first $a_i$ un-dotted columns was false.

\end{proof}

\noindent\textit{Proof of Theorem \ref{thm:DC+NC}}.

Consider an arbitrary bubble diagram that satisfies the dot and numbering conditions.
Let $(x_1,1),...,(x_m,m)$ be the positions of the dots such that after the $m$th row, all the dots are placed on the $y = x$ diagonal.
We claim that the permutation $x_1,...,x_m$ corresponds to this bubble diagram.

First, we show that a given position in this diagram contains a bubble if and only if there is a dot to the right and a dot above the position.
Immediately by the dot condition, we see that if a bubble is in a position, then there must be a dot to the right and a dot above. 

On the other hand, consider a position $(x_0,y_0)$ with a dot to the right and a dot above.
Assume $(x_0,y_0)$ is not filled with a bubble.
In this case, for the diagram to satisfy the dot condition, there must be a bubble to the right of $(x_0,y_0)$ before the dot.
Say the position of this bubble is $(x_0+k,y_0)$.
There must be a dot above $(x_0 + k,y_0)$ by the dot condition.
Then by induction, we see that the number of bubbles below $(x_0+k,y_0)$ equals the number of dots to the right of the $(x_0 + k)$th column below the $y_0$th row, call this value $d$.
Then by the numbering condition, we would label the bubble in position $(x_0+k,y_0)$ with $x_0+k+d$, when counting vertically.

However, we get something different if we count horizontally.
There are $y_0-1-d$ dots to the left of the $(x_0 +k)$th column below the $y_0$th row, none of which are in the $x_0$th column.
That means there are at least $y_0-d$ empty cells to the left of $(x_0+k,y_0)$, since $(x_0,y_0)$ is empty.
By the numbering condition, if we count horizontally, the label we give $(x_0+k,y_0)$ must be at most $y_0 + (x_0 + k - 1) - (y_0 - d) = x_0 + k + d -1$, which is strictly less than our previous label.
Hence, the contradiction shows there must be a bubble in position $(x_0,y_0)$.\vspace{1em}


Claim: Consider a permutation $w_1,...,w_m$ and its corresponding permutation bubble diagram, whose positions are written in coordinates $(i,w_j)$ with $i$ the vertical axis and $w_j$ the horizontal axis.
Place dots in all positions of the form $(i,w_i)$ (which does not contain a bubble because $i \nless i$).
We claim that the position $(i,w_j)$ contains a bubble if and only if there is a dot the the right and a dot above the position.\vspace{1em}

If a bubble is in position $(i,x_j)$ then $i < j$ and $x_i > x_j$ by definition, implying that the dot at $(j,x_j)$ is above $(i,x_j)$ and the dot at $(i,x_i)$ is to the right of $(i,x_j)$, respectively.
Conversely, consider a position with a dot to the right and a dot above.
Write $(i,x_i)$ for the dot to the right, and $(i+k,x_{i+k})$, $k > 0$, for the dot above.
Then the position in question is $(i,x_{i+k})$, and we have $i < i+k$ and $x_i > x_{i+k}$, implying a bubble must be in this position.

\QED{}


Claim: consider a diagram with dots in positions $(1,x_1),...,(m,x_m),(m+1,m+1),(m+2,m+2),...$, for $m\geq 0$ and distinct $x_i$ such that $0 < x_i \leq m$ for all $i$.
Then the permutation $x_1,...,x_m$ corresponds to the bubble diagram constructed by placing bubbles in all the positions with a dot to the right and a dot above.\vspace{1em}


\textbf{2nd construction of the permutation bubble diagram.}
Consider a permutation $w_1,w_2,...,w_m$, and an empty diagram with coordinates $(i,w)$ with vertical $i$-axis and horizontal $w$-axis.
Place dots in all positions of the form $(i,w_i)$.
Then place bubbles in all positions which have a dot to the right of it and a dot above it.
This is equivalent to placing bubbles in a position if.\vspace{1em}

\textbf{Another Try From Scratch}

Consider an arbitrary bubble diagram that satisfies the dot and numbering conditions.
Place the dots on the diagram and let $(1,w_1),(2,w_2),...$ denote their positions.
In fact, since the dots are never in the same column, we are able to refer to the $w_j$th column and therefore the $(i,w_j)$th position without confusion.
We claim that a bubble is placed in the position $(i,w_j)$ exactly when $i < j$ and $w_i > w_j$, which shows that this diagram is a permutation bubble diagram corresponding to the permutation $w_1,w_2,...$

Assume there is a bubble in position $(i,w_j)$.
By the dot condition there must be a dot above and a dot to the right of this position.
The dot above is in position $(j,w_j)$ meaning $i < j$, and the dot to the right is in position $(i,w_i)$ meaning $w_i > w_j$.

Conversely, consider a position $(i,w_j)$ such that $i < j$ and $w_i > w_j$.
We claim there must be a bubble in any such position, and we prove it by induction on $i$.
In the base case, the positions to the left of the dot at $(1,w_1)$ are the only ones to satisfy the assumption.
And indeed we must place a bubble in each of these positions.
For if not, then using the vertical dot-placing method, we would place a dot in a position to the left of $(1,w_1)$, violating the dot condition.

Now, let's assume we've shown that there is a bubble in every position $(i,w_j)$ such that $i < j$ and $w_i > w_j$ for all $i < I$, for some $I > 1$.
Consider a position $(I,w_j)$ with $I < j$ and $w_I > w_j$, and assume there is no bubble in this position.
Then, there must be some bubble $A$ in position $(I,w_{j'})$ for some $w_j < w_{j'} < w_{I}$, for if not then the horizontal method would place a dot in a different position than $(I,w_I)$ contradicting the dot condition.
There must be a dot above $A$ by the dot condition.
Then by induction, we see that the number of bubbles below $A$ is equal to the number of dots to the right of the $w_{j'}$th column and below the $I$th row, call this value $d$.
Then by the numbering condition, we would label $A$ with the number $w_{j'} + d$ when counting vertically.

However, we get something different if we count horizontally.
There are $I - 1 - d$ dots to the left of the $w_{j'}$th column and below the $I$th row, none of which are in the $w_j$th column.
That means there are at least $I - d$ empty cells to the left of $A$, since $(I,w_j)$ is empty.
If we count horizontally, then we would label $A$ with the number
$$I + \#\{\text{positions to the left of }A\} - \#\{\text{empty positions to the left of }A\}.$$
This is at most
$$I + (w_{j'} - 1) - (I - d) = w_{j'} + d - 1,$$ 
which is strictly less than our previous label.
This contradicts the numbering condition and our assumption that $(I,w_j)$ contains no bubble must be false.
Hence, this diagram is in fact the permutation bubble diagram for the permutation $w_1,w_2,...$, showing that any diagram satisfying the dot and numbering conditions is a permutation diagram.\vspace{1em}


\textbf{Old Stuff}

\textbf{dot and numbering iff permutation diagram}

To prove this theorem, we show that both permutation diagrams and diagrams that satisfy the dot and numbering conditions are unique up to the number of bubbles in each row.

\begin{lemma}
For every finite sequence of nonnegative integers $a_1,...,a_n$, there exists a permutation such that its bubble diagram contains $a_i$ bubbles in the $i$th row for all $i = 1,...,n$.
\end{lemma}

\begin{proof}
INCOMPLETE
\end{proof}

\begin{lemma}
If a bubble diagram satisfies the dot condition and the numbering condition, then it is unique up to how many bubbles are in each row.
\label{lem:DC+NC=>unique up to rows}
\end{lemma}

\begin{proof}
We again induct on the rows.
The first row is determined just based on the dot condition or the numbering condition independently.
Now, assume for the first $i-1$ rows, the bubbles have been placed uniquely so as not to violate the dot or the numbering conditions.
We claim that we must place the $a_i$ bubbles in the first $a_i$ columns that do not yet contain a dot.
Note first that we cannot place any bubbles above any dots by the dot condition.
Now, assume we can do some other configuration than the one in our claim.
Then there will be an empty cell $(i,w_j)$ in a column with no dot below it and a bubble to the right in some $(i, w_j + k)$, where we'll denote $w_{j'} := w_j + k$.

If there is an empty cell below the bubble in the $w_{j'}$th column, then there must be a dot to the left of the empty cell.
For if there were no dot to the left, then there would be a dot to the right.
This means there is a bubble to the right as well, because if not, then a dot would be placed in the empty cell.
Since there is a bubble to the right, by the uniqueness of the constructed row, there must be a dot below the empty cell prohibiting a bubble from being placed in the empty cell.
This is a contradiction, so there must be a dot to the left of the empty cell.
Hence, the dots to the left correspond exactly to the empty cells below $(i,w_{j'})$.

Let $d$ be the number of dots to the left of the $w_{j'}$ column and below the $i$th row equals $i - 1$.
Then the number of bubbles below $(i,w_{j'})$ equals $i - 1 - d$.
And therefore, we must label the bubble $(i,w_{j'})$, going from bottom to top, with the number $w_{j'} + i - 1 - d$.
Now, consider the number of bubbles to the left of $(i,w_{j'})$.
Since there can be no bubble above a dot, and since there is at least one extra empty cell with no dot below it, the number of bubbles to the left must be strictly less than $w_{j'} -1 - d$.
Hence, we must label the bubble $(i,w_{j'})$, going left to right, with a number strictly less than $i + w_{j'} - 1 - d$, contradicting our previous label.
Therefore, our assumption that we could do anything different than placing the $a_i$ bubbles in the first $a_i$ un-dotted columns was false.

\end{proof}

\noindent\textit{Proof of Theorem}.
We have seen that a given permutation diagram satisfies the dot and numbering conditions.
Conversely, consider an arbitrary bubble diagram that satisfies the dot and numbering conditions.
Write $a_1,...,a_n$ for the number of bubbles in each row, with the last row of bubbles being the $n$th.
Then, there exists a permutation diagram with the same number of bubbles in the corresponding rows.
This permutation diagram satisfies the dot and numbering conditions, and by the uniqueness in Lemma \ref{lem:DC+NC=>unique up to rows}, this permutation diagram is exactly the bubble diagram we started with.
In particular, the diagram is a permutation diagram.
\QED{}


dot plus southwest
\begin{proof}
\textbf{(TODO: need to reword)} First, assume we have a permutation bubble diagram.
We claim this diagram satisfies the southwest condition.
Let $(i,w_j)$ and $(i+k,w_j - l)$ be the location of two bubbles in the diagram, for $k,l \in \IZ_+$.
Let $j'$ be the index such that $w_{j'} = w_j - l$.
This means that $i < j$, $w_i > w_j$, $i+k < j'$, and $w_{i+k} > w_{j'}$.
Then $i < i+k < j'$ and $w_i > w_j > w_j - l = w_{j'}$, implying there is a bubble placed at $(i,w_{j'})$.

And we have already seen that permutation diagrams satisfy the dot condition.
Now, we consider the converse.


Let $a_1,...,a_n$ be a sequence of nonnegative integers.
We claim that there is only one diagram with $a_i$ bubbles in the $i$th row for all $i$ that satisfies the dot and southwest conditions.
To satisfy the dot condition in the first row, we must have $a_1$ bubbles placed in the first $a_1$ positions, with a dot in the $(a_1+1)$th position.
For if not, then we would place a dot to the right of all these bubbles, but in the other procedure, we would place a dot in the first empty position in this row.

Now assume the first $k-1$ rows have been determined with bubbles and dots placed uniquely.
We claim that the $a_k$ bubbles and dot must be placed in the first $a_k + 1$ columns that do not contain a dot.
Indeed, if this is not the case, then there is an empty cell in the $k$th row with bubbles to the right and no dots below it.
For the dot condition to be satisfied, we must have the dot in this column placed above the $k$th row, since there are bubbles to the right and the dot in the $k$th row must be placed to the right of them.
But with respect to the other procedure, we cannot place this column's dot above the $k$th row because the southwest condition prohibits any bubbles from being placed above the $k$th row, and the dot is placed in the first position above all the bubbles in the column such that there is no dot to the left.
Thus, the empty cell satisfies these properties, so no dot could be placed above it.
This is a contradiction, so there is only one way to place the $a_k$ bubbles and dot in the $k$th row.

Hence, diagrams that satisfy the dot and southwest condition are unique up to the number of bubbles in each row.
And since the permutation diagrams satisfy the dot and southwest conditions and are also unique in this way, they are the same diagrams.
\end{proof}


\end{document}