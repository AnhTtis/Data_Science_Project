
%%%%%%%%%%%%%%%%%%%%%%
% macreau for tableaux
%%%%%%%%%%%%%%%%%%%%%%
\newlength\cellsize \setlength\cellsize{8\unitlength}

\savebox2{% BOX
\begin{picture}(8,8)
\put(0,0){\line(1,0){8}}
\put(0,0){\line(0,1){8}}
\put(8,0){\line(0,1){8}}
\put(0,8){\line(1,0){8}}
\end{picture}}

\newcommand\boxify[1]{\def\thearg{#1}\def\nothing{}%
\ifx\thearg\nothing\vrule width0pt height\cellsize depth0pt%
  \else\hbox to 0pt{\usebox2\hss}\fi%
  \vbox to \cellsize{\vss\hbox to \cellsize{\hss$_{#1}$\hss}\vss}}

\savebox3{% CIRCLE
\begin{picture}(8,8)
\put(4,4){\circle{8}}
\end{picture}}

\newcommand{\circify}[1]{\def\thearg{#1}\def\nothing{}%
\ifx\thearg\nothing\vrule width0pt height\cellsize depth0pt%
  \else\hbox to 0pt{\usebox3\hss}\fi%
  \vbox to \cellsize{\vss\hbox to \cellsize{\hss$_{#1}$\hss}\vss}}

\newcommand\nullify[1]{\def\thearg{#1}\def\nothing{}%
\ifx\thearg\nothing\vrule width0pt height\cellsize depth0pt%
  \else\hbox to 0pt{\hss}\fi%
  \vbox to \cellsize{\vss\hbox to \cellsize{\hss$_{#1}$\hss}\vss}}


\newcommand\tableau[1]{\vtop{\let\\=\cr
\setlength\baselineskip{-8000pt}
\setlength\lineskiplimit{8000pt}
\setlength\lineskip{0pt}
\halign{&\boxify{##}\cr#1\crcr}}}

\newcommand\cirtab[1]{\vline\vtop{\let\\=\cr
\setlength\baselineskip{-8000pt}
\setlength\lineskiplimit{8000pt}
\setlength\lineskip{0pt}
\halign{&\circify{##}\cr#1\crcr}}}

\newcommand\nulltab[1]{\vtop{\let\\=\cr
\setlength\baselineskip{-8000pt}
\setlength\lineskiplimit{8000pt}
\setlength\lineskip{0pt}
\halign{&\nullify{##}\cr#1\crcr}}}

\savebox2{% BOX
\begin{picture}(8,8)
\put(0,0){\line(1,0){8}}
\put(0,0){\line(0,1){8}}
\put(8,0){\line(0,1){8}}
\put(0,8){\line(1,0){8}}
\end{picture}}

\newcommand{\csix}{red}
\newcommand{\cfiv}{orange}
\newcommand{\cfou}{yellow}
\newcommand{\cthr}{green}
\newcommand{\ctwo}{blue}
\newcommand{\cone}{violet}


\newcommand{\cball}[2]{%
  \begin{tikzpicture}
    \filldraw[fill=#1!35,draw=black] (0,0) circle (4pt) node {$\scriptstyle #2$};
  \end{tikzpicture}
}

\newcommand{\cbox}[2]{%
  \begin{tikzpicture}
    \draw +(-4pt,-4pt) rectangle +(4pt,4pt);
    \node (0,0) {$\scriptstyle \mathbf{#2}$};
  \end{tikzpicture}
}

\newcommand{\leftball}[2]{\makebox[0pt]{\raisebox{1.5pt}{$\leftarrow$}}\cball{#1}{#2}}

\begin{document}

\begin{figure}[ht]
  \begin{center}
    \begin{tikzpicture}[xscale=1.2,yscale=0.8]
      \node at (0,5) (A) {$\cirtab{ ~ \\ ~ & ~ \\ \\\hline}$};
      \node at (1,4) (B) {$\cirtab{ ~ \\ ~ \\ & ~ \\\hline}$};
      \node at (2,3) (C) {$\cirtab{ ~ \\ \\ ~ & ~ \\\hline}$};
      \node at (1,2) (E) {$\cirtab{ \\ ~ \\ ~ & ~ \\\hline}$};
      \node at (0,3) (H) {$\cirtab{ \\ ~ & ~ \\ ~ \\\hline}$};
      \draw[thin] (A) -- (H) ;
      \draw[thin] (A) -- (B) ;
      \draw[thin] (H) -- (E) ;
      \draw[thin] (B) -- (E) ;
      \draw[thin] (B) -- (C) ;
      \draw[thin] (C) -- (E) ;
      \node at (4,5) (A2) {$\vline\tableau{3 & 2 \\ 2 \\ & \\\hline}$};
      \node at (5,4) (B2) {$\vline\tableau{3 & 1 \\ 2 \\ & \\\hline}$};
      \node at (6,3) (C2) {$\vline\tableau{3 & 1 \\ 1 \\ & \\\hline}$};
      \node at (5,2) (E2) {$\vline\tableau{2 & 1 \\ 1 \\ & \\\hline}$};
      \node at (4,3) (H2) {$\vline\tableau{2 & 2 \\ 1 \\ & \\\hline}$};
      \draw[thin] (A2) -- (H2) ;
      \draw[thin] (A2) -- (B2) ;
      \draw[thin] (H2) -- (E2) ;
      \draw[thin] (B2) -- (E2) ;
      \draw[thin] (B2) -- (C2) ;
      \draw[thin] (C2) -- (E2) ;
    \end{tikzpicture}
    \caption{\label{fig:kohnert}The set $\KD(0,2,1)$ of Kohnert diagrams for $(0,2,1)$ and their images under the injection with reverse tableaux.}
  \end{center}
\end{figure}

\end{document}