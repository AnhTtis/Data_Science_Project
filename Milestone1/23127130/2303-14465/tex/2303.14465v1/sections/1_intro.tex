\vspace{-0.1in}
\section{Introduction}
Vision-language (VL) training is all about learning ``good'' features for each modality, 
such that the features should faithfully represent the underlying semantics. Thanks to the large-scale image-text pairs on the Web, we have abundant multimodal supervision for the two features with the same semantic meaning~\cite{tan-bansal-2019-lxmert,li2021align,radford2021learning,jia2021scaling}---each matched image-text pair should have ``similar'' visual and textual features, and each unmatched pair should have ``dissimilar'' ones. Thus, the image-text similarity plays a crucial role to define the feature quality in training VL foundation models (VLMs)~\cite{wang2022image,yu2022coca,li2021align,radford2021learning,jia2021scaling,dou2021empirical,dou2022coarse,li2022blip}. 


\begin{figure}[t]
    \centering
    \includegraphics[scale=0.265]{images/intro_bear.PNG}
    \caption{\textbf{How does language context impact region-word alignment and downstream detection?} In captions, objects (\eg bear) are often described with rich contextual information (\eg attributes like large, furry, brown) which we hypothesize can impact the effectiveness of object grounding and therefore downstream detection. We show that prior work in detection with region-word alignment does not fully benefit from context such as attributes. We thus provide strategies to better leverage important object context. 
     }
    \label{intro_fig}
\end{figure}

% Such strategy can enable better understanding of fine-grained categories (\eg tuk tuk).

% Multimodal pretraining for object-level tasks involves learning to ground visual regions to caption word embeddings. 


Has the prevailing ``matched \textit{vs.} unmatched'' similarity fulfilled its duty? Yes and no. On the one hand, recent VLMs~\cite{rombach2022high,wang2022image,dou2022coarse,radford2021learning,yu2022coca,ramesh2022hierarchical} have demonstrated impressive results in various downstream VL tasks such as image-text retrieval.   
However, on the other hand, it is acknowledged by the community that the VLMs still fall short in \textit{nuanced and complex semantic compositions}~\cite{ramesh2022hierarchical,cho2022dall,parcalabescu2021valse,thrush2022winoground}.
In this regard, we present a text-to-image retrieval example on LAION400M~\cite{schuhmann2021laion} with the most recent SOTA VLM FIBER~\cite{dou2022coarse}.
As shown in Figure~\ref{fig:intro_fig}(a), given the query text ``the house on the \emph{right} side of the road'', we first invite 5 graduate students to rank 25 candidate images from most similar to least similar. The continuously decreasing ranking from human judges (\inlineimg{images-inline/black-line}) is served as the oracle semantic similarity measure.
We then compared this ranking with the ones from FIBER~\cite{dou2022coarse} (\inlineimg{images-inline/green-line}). 
Although FIBER correctly retrieved the top-$1$ image (image\#1, ranks 1), some semantically incorrect images (\eg, image\#25, ranks 17) are falsely ranked higher than the correct ones (\eg, image\#2, ranks 20).
Furthermore, when modifying the query text with a slight semantic change (``\underline{right}'' $\rightarrow$ ``\underline{left}''), the rankings remain almost the same.
Clearly, the similarity changes in FIBER do not faithfully reflect the semantic changes in images (\#1 $\rightarrow$ \#25) or text queries (``\underline{right}'' $\rightarrow$ ``\underline{left}'').








To quantitatively measure the above inconsistency between semantic and similarity score changes, we consider two matched image-text pairs $\{I_1, T_1\}$ and $\{I_2, T_2\}$ that are semantically similar but only different in the number of clocks in Figure~\ref{fig:intro_fig} (b). 
With a slight change of clock counts in caption (``2''$\rightarrow$``3''), FIBER mistakenly assigns a higher similarity score to $\{I_1,T_2\}$ rather than $\{I_1,T_1\}$ ($3.83$ v.s. $3.79$). Furthermore, the changes in similarity scores guided by the semantic change (``2''$\leftrightarrow$``3'') 
are highly inconsistent ($+0.04$ v.s. $-1.81$). 
Ideally, an \textbf{\emph{equivariant}} image-text similarity measure should faithfully reflect the semantic change, \ie, the same semantic changes should lead to a similar amount of similarity changes (\eg, $-0.22$ v.s. $-0.17$ of ours in Figure~\ref{fig:intro_fig}(b)).







\begin{figure}[t]
    \centering
    \footnotesize
    \includegraphics[width=.47\textwidth]{images/intro_semantic_file.pdf}
    \caption{The illustration of the core idea in \algname. Besides the two matched pairs $\{I_1, T_1\}$ and $\{I_2, T_2\}$, we don't need extra annotation such as the middle pair. 
    }
    \vspace{-3mm}
    \label{fig:intro_semantic}
\end{figure}



\noindent\textbf{Equivariance Loss}. To address this non-equivariance issue, we propose Equivariant Similarity Learning (\algname), which imposes additional equivariance regularization on image-text pairs for VLM learning without additional supervision. Figure~\ref{fig:intro_semantic} illustrates the underlying semantics perceived by human, where each matched pair demonstrates the image and text corresponding to the underlying semantic. 
Given two matched image-text pairs $\{I_1,T_1\}$ as semantic 1 and $\{I_2,T_2\}$ as semantic 2, we can obtain four similarity scores $s_{11}$, $s_{12}$, $s_{22}$, and $s_{21}$. We define \textbf{\emph{Equivariant Similarity}} to be an image-text similarity function, whose output value should correspond to the underlying semantic change, which can be measured by text or image change. 

\vspace{-4pt}
\begin{definition} (Equivariant Similarity)
The similarity $s$ between image and text is equivariant if and only if the following equations hold:

\vspace{-4mm}
{\small{
\begin{equation}
%\begin{align}
s_{11}-s_{12} =  \underbrace{\sum\nolimits_{T_1}^{T_2} \mu(T)},
~~~s_{22}-s_{21} =  \underbrace{\sum\nolimits_{T_2}^{T_1} \mu(T)},
%\end{align}
\vspace{-0.2cm}
\label{eq:intro_eq1}
\end{equation}
}
\hspace{20mm} Semantic Change Measured by Text Change
}

\vspace{-0.4cm}
% \noindent Semantic change by image change:
{\small{
\begin{equation}
s_{11}-s_{21} =  \underbrace{\sum\nolimits_{I_1}^{I_2} \mu(I)},
~~~s_{22}-s_{12} =  \underbrace{\sum\nolimits_{I_2}^{I_1} \mu(I)},
\vspace{-0.2cm}
\label{eq:intro_eq2}
\end{equation}
\hspace{20mm} Semantic Change Measured by Image Change
}}\label{def:eqsim}
\end{definition}
where $\mu(I)$ ($\mu(T)$) denotes the measure~\cite{royden1988real} in image (text) space, \ie, an infinitesimal unit of visual (textual) change. 
Based on Definition~\ref{def:eqsim}, we formally derive \algname, an equivariance loss for a hybrid learning strategy on both semantically close and distant training pairs (Section~\ref{sec:method}). Specifically, \algname directly enforces $s_{11}-s_{12}=s_{22}-s_{21}$ and $s_{11}-s_{21}=s_{22}-s_{12}$ for semantically close samples; while for semantically distant samples, we derive a simplified formulation of $s_{12}=s_{21}$.
We show that adding \algname as a regularization term improves existing similarity training objectives significantly on challenging datasets (\eg, over $4\%$ on Winoground~\cite{thrush2022winoground}) and tricky tasks (\eg, around $30\%$ on VALSE~\cite{parcalabescu2021valse}). \algname can also retain or even improve retrieval performance on Flickr30K~\cite{plummer2015flickr30k} dataset. 


\noindent\textbf{Equivariance Benchmark}. To further facilitate the proper evaluation of equivariance in VL community, we present a novel evaluation benchmark dubbed \benchname (Section \ref{sec:dataset}). 
Motivated by the examples in Figure~\ref{fig:intro_fig}(b), \benchname features ``slightly'' mis-matched pairs with a \emph{minimal semantic drift} from the matched pairs, as opposed to ``very different'' matched and unmatched pairs that are easily distinguishable by both non-equivariant and equivariant similarities.  
Unlike recent efforts~\cite{parcalabescu2021valse,thrush2022winoground} focusing on minimal semantic changes in captions, \benchname pivots on diverse \textit{visual}-minimal changes, automatically curated from time-varying visual contents in natural videos and synthetic engines with more precise control. 
We benchmark a full spectrum of VLMs on \benchname, and reveal that the non-equivariant similarity in existing VLMs fails easily. On this new test bed, \algname can serve as a remedy and bring a large performance gain of $\sim$3\% on average. 











 





    
% conclude motivation
Our contributions are summarized as follows: \textbf{(1)} We comprehensively study the problem of similarity equivariance in VLMs. We propose \algname for equivariant training and \benchname for diagnostic evaluation; \textbf{(2)} \algname is not only theoretically grounded but also simple, effective and easily pluggable; and \textbf{(3)} \benchname clearly diagnoses that conventional evaluation is not responsive to equivariance. Furthermore, \algname can significantly improve VLMs on \benchname, as well as other challenging benchmarks. 
% in three-fold:
% \vspace{-2pt}
% \begin{itemize}
% \setlength\itemsep{-2pt}
%     \item We comprehensively study the problem of similarity equivariance in VLMs. We propose \algname for equivariant training and \benchname for diagnostic evaluation.
%     \item \algname is not only theoretically grounded but also simple, effective and easily pluggable.
%     \item \benchname clearly diagnoses that conventional evaluation is not responsive to equivariance. Furthermore, \algname can significantly improve VLMs on \benchname, as well as other challenging benchmarks. 
    
    
%     % \keli{\item We introduce similarity equivariance}
%     % \keli{\item \algname is theoretically grounded, and effectively improves existing VMLs on multiple diagnosis benchmarks including Winoground and VALSE.}
%     % \keli{\item We present a new challenging benchmark \benchname, and benckmark a full spectrum of VLMs, revealing ...}
    
%     % \linjie{The last sentence does not read correct.}
% \end{itemize}
% \vspace{-2pt} 
    
