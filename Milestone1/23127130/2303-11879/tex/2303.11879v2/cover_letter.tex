
\documentclass[11pt]{letter} 

\makeatletter
\newenvironment{thebibliography}[1]
     {\list{\@biblabel{\@arabic\c@enumiv}}%
           {\settowidth\labelwidth{\@biblabel{#1}}%
            \leftmargin\labelwidth
            \advance\leftmargin\labelsep
            \usecounter{enumiv}%
            \let\p@enumiv\@empty
            \renewcommand\theenumiv{\@arabic\c@enumiv}}%
      \sloppy
      \clubpenalty4000
      \@clubpenalty \clubpenalty
      \widowpenalty4000%
      \sfcode`\.\@m}
     {\def\@noitemerr
       {\@latex@warning{Empty `thebibliography' environment}}%
      \endlist}
\newcommand\newblock{\hskip .11em\@plus.33em\@minus.07em}


\usepackage[numbers]{natbib}
\let\bibsection\relax
\usepackage{graphicx}

\usepackage{fancyhdr} % Required for custom headers
%\usepackage[margin=1in]{geometry} % Required to make the margins smaller to fit more content on each page
\usepackage[linkcolor=blue]{hyperref} % Required to create hyperlinks to questions from elsewhere in the document
\hypersetup{pdfborder={0 0 0}, colorlinks=true, urlcolor=blue} % Specify a color for hyperlinks
\usepackage{todonotes} % Required for the boxes that questions appear in

\usepackage{microtype} % Slightly tweak font spacing for aesthetics
\usepackage{palatino} % Use the Palatino font
\usepackage{epstopdf}
\usepackage{tocloft} % Required to give customize the table of contents to display questions

%tocloft I don't recognize any sectional divisions so I'll do nothing
% Create and define the list of questions
%\newcommand{\listquestions}{List of Questions}
%\newlistof{questions}{faq}{\listquestions}
%\setlength\cftbeforefaqtitleskip{.3em} % Adjusts the vertical space between the title and subtitle
%\setlength\cftafterfaqtitleskip{.3em} % Adjusts the vertical space between the subtitle and the first question
%\setlength\cftparskip{.3em} % Adjusts the vertical space between questions in the list of questions

% Create the command used for questions
\newcommand{\question}[1] % This is what you will use to create a new question
{
%\refstepcounter{questions} % Increases the questions counter, this can be referenced anywhere with \thequestions
\par\noindent % Creates a new unindented paragraph
\phantomsection % Needed for hyperref compatibility with the \addcontensline command
%\addcontentsline{faq}{questions}{#1} % Adds the question to the list of questions
\todo[inline, color=aliceblue]{\textbf{#1}} % Uses the todonotes package to create a fancy box to put the question
\vspace{1em} % White space after the question before the start of the answer
}
\makeatother

\definecolor{aliceblue}{rgb}{0.94, 0.97, 1.0}
\fancyhf{}
%\chead{\hmwkClass\ (\hmwkClassInstructor\ \hmwkDueDate\ \hmwkClassTime) \hmwkTitle} % Top center head
%\rfoot{-\thepage-} % Bottom right footer
\pagestyle{fancy}

\setlength\parindent{0pt} % Removes all indentation from paragraphs



\usepackage{newcent} % Default font is the New Century Schoolbook PostScript font
%\usepackage{helvet} % Uncomment this (while commenting the above line) to use the Helvetica font
\usepackage[export]{adjustbox}

\usepackage{amssymb}
\usepackage{amsmath}

\usepackage{xspace}
\usepackage{subcaption}
\usepackage{multirow}
\usepackage{xcolor}

\usepackage{textcomp}

\usepackage{caption}
\usepackage{floatrow}
\usepackage{amsmath,amssymb,amsfonts}
%\usepackage{algorithmic}
\usepackage{graphicx}

\usepackage{amssymb}
\usepackage{enumitem}  
\usepackage{algorithm, algorithmicx, algpseudocode}

\newfloatcommand{capbtabbox}{table}[][\FBwidth]
%\usepackage[numbers]{natbib}
%\usepackage[linkcolor=blue]{hyperref} % Required to create hyperlinks to questions from elsewhere in the document
%\hypersetup{pdfborder={0 0 0}, colorlinks=true, urlcolor=blue} % Specify a color for hyperlinks
%\let\bibsection\relax

% Margins
\topmargin=-0.7in % Moves the top of the document 1 inch above the default
\textheight=9.6in % Total height of the text on the page before text goes on to the next page, this can be increased in a longer letter
\oddsidemargin=-10pt % Position of the left margin, can be negative or positive if you want more or less room
\textwidth=6.5in % Total width of the text, increase this if the left margin was decreased and vice-versa

%\let\raggedleft\raggedright % Pushes the date (at the top) to the left, comment this line to have the date on the right



\begin{document}
%----------------------------------------------------------------------------------------
%	ADDRESSEE SECTION
%----------------------------------------------------------------------------------------

\begin{letter}{\large \bf Editors of ACM Transactions on Recommender Systems (TORS)}

%----------------------------------------------------------------------------------------
%	YOUR NAME & ADDRESS SECTION
%----------------------------------------------------------------------------------------

\begin{flushright}
\large Lingzi Zhang, Xin Zhou, Zhiwei Zeng, and Zhiqi Shen. \\ % Your name
Nanyang Technological University \\
%\vspace{20pt} \hrule height 1pt % If you would like a horizontal line separating the name from the address, uncomment the line to the left of this text
50 Nanyang Ave, 639798, Singapore
\end{flushright}
\date{}
%\vfill

\signature{ Lingzi Zhang, Xin Zhou, Zhiwei Zeng, and Zhiqi Shen.} % Your name for the signature at the bottom

%----------------------------------------------------------------------------------------
%	LETTER CONTENT SECTION
%----------------------------------------------------------------------------------------

\opening{Dear Editors,}


We are pleased to submit an original research article entitled ``Multimodal Pre-training for Sequential Recommendation via Contrastive Learning'' for consideration by ACM Transactions on Recommender Systems. 


In the manuscript, we study how to effectively utilize both users’ sequential behavior and items’ multimodal content to alleviate the data sparsity of the user behavior data and further enhance the sequential recommendation performance. Specifically, we propose to tokenize images of every item into multiple textual keywords and uses the pre-trained BERT model to obtain initial textual and visual features of items, for eliminating the discrepancy between the text and image modalities. A novel backbone network optimized by two designed contrastive learning tasks is proposed to bridge the gap between the item multimodal content and the user behavior, using a complementary sequence mixup strategy. In order to demonstrate the effectiveness of the proposed method, we have performed extensive experiments on real-world datasets for challenging tasks, including sequential and cold-start recommendations. 

We confirm that the submitted manuscript is original, unpublished, and not currently under review with another journal. 
In addition, all authors have confirmed this submission and agreed to its review of TORS.

We deeply appreciate your consideration of our manuscript, and we look forward to receiving comments from the reviewers. Please address all correspondence concerning this manuscript to lingzi001@e.ntu.edu.sg.

\vfill
\closing{Sincerely yours,}
%\encl{References} % List your enclosed documents here, comment this out to get rid of the "encl:"

%----------------------------------------------------------------------------------------

%\bibliographystyle{unsrtnat}
%\bibliography{refer-main}
\end{letter}

\end{document}
