\subsection{Hyperparameters}

\Cref{tab:app_hyperparameters} shows the hyper parameters used for the training of the Topograph models presented in this paper.
The models were trained using \textsc{Tensorflow} v2.10~\cite{tensorflow2015-whitepaper}.

\begin{table}[htbp]
    \caption{Hyper parameters used for the training of Topographs}
    \label{tab:app_hyperparameters}
    \begin{tabular}{cc}
        \toprule
        Hyper parameter & Value \\
        \midrule
        Optimizer & \textit{AdamW} \\
        Epochs & 100 \\
        n original message passing & 2 \\
        n topograph updates & 4 \\
        lr schedule & cosine annealing \\
        initial learning rate & 0.001 \\
        decay steps & 2 epochs \\
        batch size & 256 \\
        pooling & attention \\
        W initialisation & attention pooling \\
        Top initialisation & attention pooling \\
        attention units & [32, 32, 1] \\
        activation & gelu \\
        normalisation & layer norm \\
        regression units & [64, 64, 3] \\
        edge classification units & [128, 128, 128, 1] \\
        graph processing units & [256, 256, 64] \\
        persistent edges & true \\
        classification loss & weighted binary crossentropy \\
        regression loss & mean absolute error \\
        \bottomrule
    \end{tabular}
\end{table}


\subsection{Example Topograph}

A complete Topograph network is shown in \cref{fig:topottW:embed-noprior} for reconstruction of $t\bar{t}W$ in events containing exactly one lepton, one neutrino and multiple jets.
For the production of a top quark pair in association with a $W$ boson ($t\bar{t}W$), two top quark blocks and one $W$ block are connected to the input particles, but not to one another. The Topograph network is trained to identify the edges of the true daughter particles of each particle in the process, and predicts the kinematics of the two top quarks and the three $W$ bosons.
As Topographs are defined to represent an underlying physics process, it is also a choice whether to predefine whether the lepton and neutrino originate from a $W$ boson coming from a top decay, or the additional $W$ boson.
Here we show two options, in \cref{fig:topottW:embed-prior}, where the lepton is required to come from a top decay, and \cref{fig:topottW:embed-noprior}, where no assumption is made.
% Topographs can be used to test both hypotheses simultaneously, or be used without predefined edges to predict the likeliest assignment itself.

\begin{figure}[htbp]
%   \centering
%   \begin{subfigure}[b]{0.4\textwidth}
    \centering
    \resizebox*{0.4\textwidth}{!}{%}
    \includegraphics{figures/topotttW-interact-embed}%
    }
    \caption{Topograph network for the $t\bar{t}W$ process comprising two $t$ blocks and a $W$ block.
    The input particles are first passed through an embedding network $\phi$ unique to each type of particle; either a jet $j$, lepton $\ell$ or neutrino $\nu$.
    Embedded particles are then passed through an (optional) information exchange layer.
    All particles are connected to all possible mother particles, as shown by the dashed edges.}
    % , whereas in \subref{fig:topottW:embed-noprior} no prior connections are enforced. They could equally be defined to originate from the rightmost $W$ block}
    \label{fig:topottW:embed-noprior}
%   \end{subfigure}
\end{figure}

\begin{figure}[htbp]
%   \begin{subfigure}[b]{0.4\textwidth}
    \centering
    \resizebox*{0.4\textwidth}{!}{%}
    \includegraphics{figures/topotttW-interact-embed-prior}%
    }
    \label{fig:topottW:embed-prior}
%   \end{subfigure}
  \caption{Topograph network for the $t\bar{t}W$ process comprising two $t$ blocks and a $W$ block
%   The input particles are first passed through an embedding network $\phi$ unique to each type of particle; either a jet $j$, lepton $\ell$ or neutrino $\nu$.
%   Embedded particles are then passed through an (optional) information exchange layer.
%   All particles are connected to all possible mother particles, as shown by the dashed edges.
  where the $\ell$ and $\nu$ are predefined to connect to the $W$ in one top block.}
  % In \subref{fig:topottW:embed-prior} $\ell$ and $\nu$ are predefined to connect to the $W$ in one top block, whereas in \subref{fig:topottW:embed-noprior} no prior connections are enforced.}

\end{figure}



\subsection{Comparisons with partial event trainings}
\label{app:partialspa}

Instead of training using only complete events, both the Topograph and \spanet models can be trained including partial events, that is, events where some partons are not able to be matched to reconstructed jets.
In the training events, for Topographs at least one $W$ boson, and for \spanet at least one top quark are required to be fully reconstructable.
For these models the efficiencies of fully reconstructable events change slightly, with Topographs having a slight reduction in efficiencies and \spanet a slight increase.
However, for most selections, the efficiencies are still within the uncertainties of models trained only on fully reconstructable events.
\Cref{tab:app_efficiencies_partial} shows the parton matching efficiencies for events categorised based on how many partons can be matched to reconstructed jets.
Here the benefit of training on partial events can be seen, with both Topographs and \spanet having higher efficiencies of correctly matching jets to the all available partons.

% On complete events, Topographs still have a higher overall reconstruction efficiency.

\begin{table*}[tb]
    \centering
    \caption{Event reconstruction efficiencies for the $\chi^2$ method, the \spanet model and our Topograph model in different jet and $b$-jet multiplicities.
    Both models were trained on complete and partial events.
    % Five models are trained with different random initialisations for both \spanet and Topographs.
    The highest efficiency is highlighted in bold.}
    \label{tab:app_efficiencies_partial}
    \begin{tabularx}{0.95\textwidth}{Y|Y|YYY} 
        \toprule
         $N_{jets}$ & $N_{b-\mathrm{jets}}$ & \spanet & Topograph & $\chi^2$ \\ 
         \midrule
         $6$ & $2$ & $\mathbf{81.03\pm0.32}$ & $80.77\pm0.32$ & $72.73\pm0.36$ \\
         $6$ & $\geq2$ & $\mathbf{79.11\pm0.30}$ & $78.93\pm0.30$ & $70.94\pm0.33$ \\
         $7$ & $2$ & $\mathbf{65.35\pm0.44}$ & $65.18\pm0.44$ & $54.28\pm0.46$ \\
         $7$ & $\geq2$ & $\mathbf{63.40\pm0.40}$ & $63.13\pm0.40$ & $52.11\pm0.41$ \\
         $\geq6$ & $2$ & $\mathbf{68.93\pm0.25}$ & $68.75\pm0.25$ & $58.57\pm0.26$ \\
         $\geq6$ & $\geq2$ & $\mathbf{66.33\pm0.23}$ & $66.21\pm0.23$ & $55.90\pm0.24$ \\
         \bottomrule
    \end{tabularx}
\end{table*}


\begin{table*}[tb]
    \centering
    \caption{Percentage of events with at most $N$ incorrectly matched partons using the $\chi^2$ method, \spanet and Topographs.
    The \spanet model was trained on complete and partial events.
    Events are categorised based on the number of partons matched to jets at truth level.
    The highest efficiency is highlighted in bold.}
    \label{tab:app_correct_jets}
    \begin{tabularx}{0.95\textwidth}{YY|YYYYYY}
        \toprule
        $N_{partons}$ &  & \multicolumn{6}{c}{Incorrectly matched partons [\%]} \\
        matched & Model & 0 & $\leq1$ & $\leq2$ & $\leq3$ & $\leq4$ & $\leq5$  \\
         \midrule
         \multirow{3}{*}{3} & \spanet & $36.68\pm0.66$ & $76.94\pm0.57$ & $98.55\pm0.16$ & $100.00\pm0.00$ & - & - \\
         & Topograph & $\mathbf{37.27\pm0.66}$ & $77.89\pm0.57$ & $98.77\pm0.15$ & $100.00\pm0.00$ & - & - \\
         & $\chi^2$ & $34.30\pm0.65$ & $\mathbf{80.12\pm0.54}$ & $\mathbf{99.05\pm0.13}$ & $100.00\pm0.00$ & - & - \\
         \midrule
         \multirow{3}{*}{4} & \spanet & $37.78\pm0.28$ & $66.29\pm0.28$ & $91.78\pm0.16$ & $99.46\pm0.04$ & $100.00\pm0.00$ & -  \\
         & Topograph & $\mathbf{38.08\pm0.28}$ & $\mathbf{67.53\pm0.27}$ & $\mathbf{92.57\pm0.15}$ & $99.62\pm0.04$ & $100.00\pm0.00$ & -  \\
         & $\chi^2$ & $31.01\pm0.27$ & $63.84\pm0.28$ & $92.19\pm0.16$ & $\mathbf{99.73\pm0.03}$ & $100.00\pm0.00$ & -  \\
         \midrule
         \multirow{3}{*}{5} & \spanet & $48.14\pm0.19$ & $67.24\pm0.18$ & $88.18\pm0.12$ & $98.14\pm0.05$ & $99.95\pm0.01$ & $100.00\pm0.00$  \\
         & Topograph & $\mathbf{48.22\pm0.19}$ & $\mathbf{69.27\pm0.18}$ & $\mathbf{89.52\pm0.12}$ & $\mathbf{98.68\pm0.04}$ & $\mathbf{99.96\pm0.01}$ & $100.00\pm0.00$  \\
         & $\chi^2$ & $32.36\pm0.18$ & $58.27\pm0.19$ & $84.70\pm0.14$ & $98.21\pm0.05$ & $99.94\pm0.01$ & $100.00\pm0.00$  \\
         \midrule
         \multirow{3}{*}{6} & \spanet & $\mathbf{66.33\pm0.23}$ & $74.38\pm0.21$ & $89.86\pm0.14$ & $96.18\pm0.09$ & $99.54\pm0.03$ & $100.00\pm0.00$ \\
         & Topograph & $66.20\pm0.23$ & $\mathbf{74.87\pm0.21}$ & $\mathbf{91.11\pm0.14}$ & $\mathbf{97.04\pm0.08}$ & $\mathbf{99.73\pm0.02}$ & $100.00\pm0.00$ \\
         & $\chi^2$ & $55.90\pm0.24$ & $64.93\pm0.23$ & $84.14\pm0.17$ & $93.97\pm0.11$ & $99.43\pm0.04$ & $99.98\pm0.01$ \\
        \bottomrule
    \end{tabularx}
\end{table*}


\subsection{Impact of systematic variations}

For applications in high energy physics, it is crucial that any new approach is not sensitive to changes under systematic variations.
In particular with machine learning approaches, it would be problematic if methods were sensitive to underlying and non-physical effects arising from the simulated samples on which they were trained.
Other sources of variation come from differences in the calibration or reconstruction of physics objects between simulation and data.

To test the dependence on the simulated samples used to train the Topograph, we evaluate the best performing model trained on the nominal MadGraph data on an alternative independent dataset.
This alternative sample consists of all-hadronic \ttbar events simulates both the hard interactions and parton shower are with Pythia8~(v8.307), using the Monash tuned set of parameters~\cite{Monash} at leading order accuracy.

To test the dependence on reconstruction effects, we apply a shift or gaussian smearing to the energy of reconstructed jets in the events.

The absolute change in performance arising from the systematic variations is summarised in \cref{tab:systematic_uncertainties}.
The impact is compared for Topographs, \spanet, and $\chi^2$.
Evaluating on the alternative sample results in a slightly reduced overall efficiency for all three approaches.
This effects Topographs and \spanet slightly more than $\chi^2$,
however, the overall gain in performance remains similar.
Both Topographs and \spanet are robust under systematic shifts or reduced jet energy resolution, whereas the $\chi^2$ method suffers from a substantial drop in efficiency, especially at higher jet multiplicities.

\begin{table*}[tb]
    \centering
    \caption{Difference in reconstruction efficiencies when evaluating the different methods on systematic variations.
    The difference is taken with respect to \cref{tab:efficiencies}.
    The systematic variations include: evaluating on a dataset produced with a Pythia for the matrix element generation and the parton shower, evaluating on the nominal data set scaling all jet energies by 2.5\%, and evaluating on the nominal data set smearing all jet energies by 5\%.}
    \label{tab:systematic_uncertainties}
    \begin{tabularx}{0.95\textwidth}{Y|Y|YYY|YYY|YYY} 
        \toprule
         & & \multicolumn{3}{c}{Pythia} & \multicolumn{3}{c}{Jet Scale} & \multicolumn{3}{c}{Jet Resolution} \\
         $N_{jets}$ & $N_{b-\mathrm{jets}}$ & \spanet & Topograph & $\chi^2$ & \spanet & Topograph & $\chi^2$ & \spanet & Topograph & $\chi^2$ \\ 
         \midrule
         $6$ & $2$ & $-2.13$ & $-2.00$ & $-0.94$ & $+0.04$ & $\pm0.00$ & $\vphantom{1}-2.82$ & $+0.08$ & $\pm0.00$ & $-12.79$ \\
         $6$ & $\geq2$ & $-2.28$ & $-2.13$ & $-1.33$ & $+0.07$ & $+0.03$ & $\vphantom{1}-3.63$ & $+0.04$ & $+0.01$ & $-14.30$ \\
         $7$ & $2$ & $-2.04$ & $-2.14$ & $-1.66$ & $\pm0.00$ & $\pm0.00$ & $-12.30$ & $-0.05$ & $-0.06$ & $-21.52$ \\
         $7$ & $\geq2$ & $-2.36$ & $-2.37$ & $-1.62$ & $-0.02$ & $+0.04$ & $-12.69$ & $-0.10$ & $-0.07$ & $-21.63$ \\
         $\geq6$ & $2$ & $-1.64$ & $-1.53$ & $-0.72$ & $+0.04$ & $+0.01$ & $\vphantom{1}-8.23$ & $+0.02$ & $+0.02$ & $-17.18$ \\
         $\geq6$ & $\geq2$ & $-1.86$ & $-1.89$ & $-0.94$ & $+0.06$ & $+0.05$ & $\vphantom{1}-9.02$ & $-0.01$ & $+0.03$ & $-17.96$ \\
         \bottomrule
    \end{tabularx}
\end{table*}


\subsection{Studying edge scores}

\label{app:studies}

\cref{fig:app_scores_wplus} shows the distribution of edge scores of the jets to the $W$ helper node which is decided to be the $W^+$ based on the loss.
\cref{fig:app_scores_wplus:all} includes all jets in the event.
The distribution for the jets which originate from the $W^+$ peak at one, whereas the distributions of all other jet types peak at zero.
A small peak at one can be seen for the $b$-jets originating from the top quark.
To investigate this peak, the scores of the $b$-jets from both the top and the anti-top quark are shown in \cref{fig:app_scores_wplus:btag}.
They are further split based on their $b$-tagging score.
The small peak at one originates from $b$-jets from the top quark which are not $b$-tagged.
Furthermore, the impact of the $b$-tagging on the score can be seen.
The $b$-jets from the anti-top quark which are not $b$-tagged have on average higher scores than the $b$-jets from the top quark which are $b$-tagged.
This could point to the $b$-tagging result being more important for the association to the $W$ nodes than kinematics.

\cref{fig:app_scores_top} shows the same distributions but to the top helper node which is decided to be the top quark.
Again, good separation of the true edges from the false edges can be seen.
Considering only $b$-jets and splitting them based on their $b$-tagging score, it can be seen, that the $b$-tagged jets originating from the anti-top quark have on average lower scores than the non $b$-tagged jets originating from the top quark.
So, the $b$-tagging decision is not as important for the association to the top helper nodes.

This reliance on the $b$-tagging result is not unexpected.
With a $b$-tagging efficiency of $70\%$, around $30\%$ of the $b$-jets will not be tagged as a $b$-jet, leaving a large contribution of true edges to the top helper nodes which are not $b$-tagged.
The observed mis-tag efficiency is around $5\%$. 
Therefore, only a small fraction of the true edges to the $W$ helper nodes is $b$-tagged.

\cref{fig:app_scores_wminus,fig:app_scores_antitop} show the same plots for the $W^-$ and anti-top. 
No qualitative differences can be observed to the plots for the $W^+$ and the top.

\begin{figure*}[htbp]
    \centering
    \begin{subfigure}[b]{0.45\textwidth}
        \includegraphics[width=\textwidth]{figures/results/firstWEdgeScores}
        \caption{ }
        \label{fig:app_scores_wplus:all}
    \end{subfigure}    
    \begin{subfigure}[b]{0.45\textwidth}
        \includegraphics[width=\textwidth]{figures/results/firstWEdgeScoresBTagged}
        \caption{ }
        \label{fig:app_scores_wplus:btag}
    \end{subfigure}    
    \caption{Edge scores of jets to the helper node representing the $W^+$.
    The decision which $W$ node is the $W^+$ is taken by choosing the minimum of the loss under both hypotheses.
    \subref{fig:app_scores_wplus:all} shows all jets, while \subref{fig:app_scores_wplus:btag} only shows the $b$-jets from the two tops.
    They are further split into whether the jet was tagged as a $b$-jet or not.
    No requirement is placed on the number of $b$-tags.}
    \label{fig:app_scores_wplus}
\end{figure*}

\begin{figure*}[htbp]
    \centering
    \begin{subfigure}[b]{0.45\textwidth}
        \includegraphics[width=\textwidth]{figures/results/firstWEdgeScoresTop}
        \caption{ }
        \label{fig:app_scores_top:all}
    \end{subfigure}        
    \begin{subfigure}[b]{0.45\textwidth}
        \includegraphics[width=\textwidth]{figures/results/firstWEdgeScoresBTaggedTop}
        \caption{ }
        \label{fig:app_scores_top:btag}
    \end{subfigure}       
    \caption{Edge scores of jets to the helper node representing the $t$.
    The decision which $t$ node is the $t$ is taken by choosing the minimum of the loss under both hypotheses.
    \subref{fig:app_scores_top:all} shows all jets, while \subref{fig:app_scores_top:btag} only shows the $b$-jets from the two tops.
    They are further split into whether the jet was tagged as a $b$-jet or not.
    No requirement is placed on the number of $b$-tags.}
    \label{fig:app_scores_top}
\end{figure*}


\begin{figure*}[htbp]
    \centering
    \begin{subfigure}[b]{0.45\textwidth}
        \includegraphics[width=\textwidth]{figures/results/secondWEdgeScores}
        \caption{ }
        \label{fig:app_scores_wminus:all}
    \end{subfigure}    
    \begin{subfigure}[b]{0.45\textwidth}
        \includegraphics[width=\textwidth]{figures/results/secondWEdgeScoresBTagged}
        \caption{ }
        \label{fig:app_scores_wminus:btag}
    \end{subfigure}        
    \caption{Edge scores of jets to the helper node representing the $W^-$.
    The decision which $W$ node is the $W^-$ is taken by choosing the minimum of the loss under both hypotheses.
    \subref{fig:app_scores_wminus:all} shows all jets, while \subref{fig:app_scores_wminus:btag} only shows the $b$-jets from the two tops.
    They are further split into whether the jet was tagged as a $b$-jet or not.
    No requirement is placed on the number of $b$-tags.}
    \label{fig:app_scores_wminus}
\end{figure*}

\begin{figure*}[htbp]
    \centering
    \begin{subfigure}[b]{0.45\textwidth}
        \includegraphics[width=\textwidth]{figures/results/secondWEdgeScoresTop}
        \caption{ }
        \label{fig:app_scores_antitop:all}
    \end{subfigure}        
    \begin{subfigure}[b]{0.45\textwidth}
        \includegraphics[width=\textwidth]{figures/results/secondWEdgeScoresBTaggedTop}
        \caption{ }
        \label{fig:app_scores_antitop:btag}
    \end{subfigure}           
    \caption{Edge scores of jets to the helper node representing the $\bar{t}$.
    The decision which $t$ node is the $\bar{t}$ is taken by choosing the minimum of the loss under both hypotheses.
    \subref{fig:app_scores_antitop:all} shows all jets, while \subref{fig:app_scores_antitop:btag} only shows the $b$-jets from the two tops.
    They are further split into whether the jet was tagged as a $b$-jet or not.
    No requirement is placed on the number of $b$-tags.}
    \label{fig:app_scores_antitop}
\end{figure*}


\subsection{Additional figures}
\label{app:plots}

\Cref{fig:app_regression_w,fig:app_regression_top} show the regression results for the $\eta$ and $\phi$ coordinate. 
The difference between the value of the predicted $\eta$ or $\phi$ and the true parton property is shown.
For the predicted properties, the parton is either reconstructed from the jets that the model associates to the parton or the regression result is taken.
`Correct' and `incorrect' events are shown in separate distributions. 
For `incorrect' events an additional prediction is shown by taking the true jets to reconstruct the parton properties.
For `correct' events, the regression has for both quantities a wider distribution than the model association. 
For `incorrect' events, the regression and the model association perform worse, however using the true jets also has a worse resolution for these events.
Especially for the $W$ boson, the regression and the model association have a worse performance.
However, for the top quark `incorrect' labelled events can still contain the correct three jets but with a wrong assignment by switching one of the $W$-jets with the $b$-jet.


\begin{figure*}[hbt]
    \centering
    \begin{subfigure}[b]{0.45\textwidth}
        \includegraphics[width=\textwidth]{figures/results/regression_w_eta}
        \caption{ }
        \label{fig:app_regression_w:eta}
    \end{subfigure}
    \begin{subfigure}[b]{0.45\textwidth}
        \includegraphics[width=\textwidth]{figures/results/regression_w_phi}
        \caption{ }
        \label{fig:app_regression_w:phi}
    \end{subfigure}                         
    \caption{Resolution of the reconstructed $W$ boson $\eta$ and $\phi$ coordinates from the Topograph.
    % Reconstructed values use either the jets assigned to the $W$ boson, or the prediction from the Topograph.
    Comparing the prediction from the invariant system of the assigned jets (solid line) and the Topograph regression network (dashed line) for correct assigned events (green) and incorrect assigned events (orange).}
    \label{fig:app_regression_w}
\end{figure*}


\begin{figure*}[hbt]
    \centering
    \begin{subfigure}[b]{0.45\textwidth}
        \includegraphics[width=\textwidth]{figures/results/regression_top_eta}
        \caption{ }
        \label{fig:app_regression_top:eta}
    \end{subfigure}
    \begin{subfigure}[b]{0.45\textwidth}
        \includegraphics[width=\textwidth]{figures/results/regression_top_phi}
        \caption{ }
        \label{fig:app_regression_top:phi}
    \end{subfigure}      
    \caption{Resolution of the reconstructed top quark $\eta$ and $\phi$ coordinates from the Topograph.
    % Reconstructed values use either the jets assigned to the top quark, or the prediction from the Topograph.
    Comparing the prediction from the invariant system of the assigned jets (solid line) and the Topograph regression network (dashed line) for correct assigned events (green) and incorrect assigned events (orange).}
    \label{fig:app_regression_top}
\end{figure*}


\Cref{fig:app_stacked_reco_mass_w} shows the distribution of the reconstructed $W$ mass split into `correct', `incorrect', and `impossible' events as stacked histograms as an alternative representation compared to \cref{fig:reco_mass_comparison_w}. Similarly, \cref{fig:app_stacked_reco_mass_top} is an alternative representation of \cref{fig:reco_mass_comparison_top} showing the reconstructed top quark mass.


\begin{figure*}[hbt]
    \centering
    \begin{subfigure}[b]{0.32\textwidth}
        \includegraphics[width=\textwidth]{figures/results/w_masses_stacked_topo}
        \caption{ }
        \label{fig:app_stacked_reco_mass_w:topo}
    \end{subfigure}
    \begin{subfigure}[b]{0.32\textwidth}
        \includegraphics[width=\textwidth]{figures/results/w_masses_stacked_spa}
        \caption{ }
        \label{fig:app_stacked_reco_mass_w:spa}
    \end{subfigure} 
    \begin{subfigure}[b]{0.32\textwidth}
        \includegraphics[width=\textwidth]{figures/results/w_masses_stacked_chi2}
        \caption{ }
        \label{fig:app_stacked_reco_mass_w:chi2}
    \end{subfigure}             
    \caption{Stacked distributions of the reconstructed $m_W$ using \subref{fig:app_stacked_reco_mass_w:topo} Topographs, \subref{fig:app_stacked_reco_mass_w:spa}\spanet, and \subref{fig:app_stacked_reco_mass_w:chi2} $\chi^2$.
    The single histograms split the candidates into `correct', `incorrect', and `impossible' candidates.}
    \label{fig:app_stacked_reco_mass_w}
\end{figure*}


\begin{figure*}[hbt]
    \centering
    \begin{subfigure}[b]{0.32\textwidth}
        \includegraphics[width=\textwidth]{figures/results/top_masses_stacked_topo}
        \caption{ }
        \label{fig:app_stacked_reco_mass_top:topo}
    \end{subfigure}
    \begin{subfigure}[b]{0.32\textwidth}
        \includegraphics[width=\textwidth]{figures/results/top_masses_stacked_spa}
        \caption{ }
        \label{fig:app_stacked_reco_mass_top:spa}
    \end{subfigure}
    \begin{subfigure}[b]{0.32\textwidth}
        \includegraphics[width=\textwidth]{figures/results/top_masses_stacked_chi2}
        \caption{ }
        \label{fig:app_stacked_reco_mass_top:chi2}
    \end{subfigure}            
    \caption{Stacked distributions of the reconstructed $m_{top}$ using \subref{fig:app_stacked_reco_mass_top:topo} Topographs, \subref{fig:app_stacked_reco_mass_top:spa}\spanet, and \subref{fig:app_stacked_reco_mass_top:chi2} $\chi^2$.
    The single histograms split the candidates into `correct', `incorrect', and `impossible' candidates.}
    \label{fig:app_stacked_reco_mass_top}
\end{figure*}


\Cref{fig:app_reco_mass_w,fig:app_reco_mass_top} show the same distributions for the reconstructed $W$ and top quark mass, respectively, but every distribution is normalised to unity.


\begin{figure*}[hbt]
    \centering
    \begin{subfigure}[b]{0.32\textwidth}
        \includegraphics[width=\textwidth]{figures/results/w_masses_topo}
        \caption{ }
        \label{fig:app_reco_mass_w:topo}
    \end{subfigure}
    \begin{subfigure}[b]{0.32\textwidth}
        \includegraphics[width=\textwidth]{figures/results/w_masses_spa}
        \caption{ }
        \label{fig:app_reco_mass_w:spa}
    \end{subfigure}
    \begin{subfigure}[b]{0.32\textwidth}
        \includegraphics[width=\textwidth]{figures/results/w_masses_chi2}
        \caption{ }
        \label{fig:app_reco_mass_w:chi2}
    \end{subfigure} 
    \caption{Normalised distributions of the reconstructed $m_W$ using \subref{fig:app_reco_mass_w:topo} Topographs, \subref{fig:app_reco_mass_w:spa}\spanet, and \subref{fig:app_reco_mass_w:chi2} $\chi^2$.
    The single histograms split the candidates into `correct', `incorrect', and `impossible' candidates.}
    \label{fig:app_reco_mass_w}
\end{figure*}


\begin{figure*}[hbt]
    \centering
    \begin{subfigure}[b]{0.32\textwidth}
        \includegraphics[width=\textwidth]{figures/results/top_masses_topo}
        \caption{ }
        \label{fig:app_reco_mass_top:topo}
    \end{subfigure}
    \begin{subfigure}[b]{0.32\textwidth}
        \includegraphics[width=\textwidth]{figures/results/top_masses_spa}
        \caption{ }
        \label{fig:app_reco_mass_top:spa}
    \end{subfigure}
    \begin{subfigure}[b]{0.32\textwidth}
        \includegraphics[width=\textwidth]{figures/results/top_masses_chi2}
        \caption{ }
        \label{fig:app_reco_mass_top:chi2}
    \end{subfigure} 
    \caption{Normalised distributions of the reconstructed $m_{top}$ using \subref{fig:app_reco_mass_top:topo} Topographs, \subref{fig:app_reco_mass_top:spa}\spanet, and \subref{fig:app_reco_mass_top:chi2} $\chi^2$.
    The single histograms split the candidates into `correct', `incorrect', and `impossible' candidates.}
    \label{fig:app_reco_mass_top}
\end{figure*}

\Cref{fig:app_reco_mass_comparison_normalised} shows the same distributions but with the different models combined in single plots.

\begin{figure*}[hbt]
    \centering
    \begin{subfigure}[b]{0.45\textwidth}
        \includegraphics[width=\textwidth]{figures/results/w_masses_components_normalised}
        \caption{ }
        \label{fig:app_reco_mass_comparison_normalised:w}
    \end{subfigure}
    \begin{subfigure}[b]{0.45\textwidth}
        \includegraphics[width=\textwidth]{figures/results/top_masses_components_normalised}
        \caption{ }
        \label{fig:app_reco_mass_comparison_normalised:top}
    \end{subfigure}    
    \caption{Distributions of the reconstructed \subref{fig:app_reco_mass_comparison_normalised:w} $m_W$ and \subref{fig:app_reco_mass_comparison_normalised:top} $m_{top}$. 
    Each histogram is normalised to an area of one.
    The solid lines show the distributions for the Topograph, the dashed lines show the distributions for the \spanet, and the dashed-dotted lines show the distributions for the $\chi^2$ method.
    The different colours show the different types of events based on the assignment of the model: correct, incorrect, impossible.}
    \label{fig:app_reco_mass_comparison_normalised}
\end{figure*}