% \subsubsection*{The Topograph}

% Nevertheless, the edges in a graph in particle physics can be used to identify associations between objects, for example whether they originate from the same particle.
% In Graph Neural Networks~\cite{battaglia2018relational,wu2020comprehensive}, operations are applied to a permutation invariant graph which comprises nodes, which have attributes, and edges, which connect the nodes; edges, too, can have attributes.
The use of GNNs in high energy physics applications~\cite{Qu:2019gqs,Moreno:2019bmu,shlomi2020graph,Bernreuther:2020vhm,Ju:2020tbo,Guo:2020vvt,Dreyer:2020brq,Hariri:2021clz,DeZoort:2021rbj,Atkinson:2021nlt,Thais:2022iok,Gong:2022lye,CMS-DP-2020-002,ATL-PHYS-PUB-2022-027} is a recent development which is gaining in popularity.
However, graphs have also long been used to describe underlying processes occurring in particle physics in the form of Feynman diagrams. 

In a Feynman diagram, the vertices (nodes) represent the interactions between particles and the edges represent the particles themselves. This representation can equally be converted into a node-and-edge graph by representing the particles as nodes, and defining edges based on the vertices. An example of the Feynman digram and one such graph for top quark pair production is shown in \cref{fig:topgraphscomp}.
In current graph based approaches this physical inspired graph representation
% seen in \cref{fig:topgraphscomp:ttbargraph} 
is not exploited, and instead fully connected graphs are constructed from all objects recorded by the detector.
% Topographs use this structure to define neural networks which can be used to predict the correct edges from final state particles and the properties of intermediate particles.

\begin{figure}[h]
    \centering
    \begin{subfigure}[b]{0.45\textwidth}
        \centering
        \resizebox*{0.75\textwidth}{!}{%
        \includegraphics{figures/ttbarfeyn}}
        \caption{Feynman diagram}
        \label{fig:topgraphscomp:ttbarfeyn}
    \end{subfigure}
    \hfill
    \begin{subfigure}[b]{0.45\textwidth}
        \centering
        \resizebox*{0.4\textwidth}{!}{%
        \includegraphics{figures/ttbargraph}}
        \caption{Node and edge graph}
        \label{fig:topgraphscomp:ttbargraph}
    \end{subfigure}
    \caption{Two graph representations of top pair production, with one top decaying semi-leptonically and the other hadronically. The production mechanism in \subref{fig:topgraphscomp:ttbarfeyn} is shown to be gluon fusion however in \subref{fig:topgraphscomp:ttbargraph} it is represented by the hashed circle. Time flows from left to right in both graphs.}
    \label{fig:topgraphscomp}
\end{figure}

% In a Topograph, additional nodes are injected into the collection of final state objects for each intermediate particle in the decay process.
% All final state objects are connected to each relevant injected node by a single edge each, and during training the kinematic properties of the injected nodes are predicted as auxiliary tasks with dedicated regression networks for each particle type.

% The novel development with Topographs that makes them more suitable to particle physics than other approaches is how they use graphs to represent the objects.
% In current graph based approaches the natural graph structure seen in \cref{fig:topgraphscomp:ttbargraph} is not exploited, and instead fully connected graphs are constructed from all objects recorded by the detector.
We introduce a new method, called the Topograph, which builds upon the structure seen in \cref{fig:topgraphscomp:ttbargraph} to define neural networks which can be used to predict the correct edges from final state particles and the properties of intermediate particles.
% Conversely, the Topograph recreates this graph to extract more information from the process, leveraging our understanding of particles decay.
% Unlike other approaches which identify doublets and triplets from a fully connected graph of recorded objects, Topographs introduce the intermediary particles of a process. All objects are then connected to their potential mother particles by edges.
% The Topograph instead seeks to address this flaw by taking inspiration from another type of graph used in particle physics to represent physics processes and the decays of particles, \emph{Feynman diagrams}.
% Instead of trying to identify triplets, quadruplets or pairs of objects from a fully connected graph, the Topograph can be seen to reconstruct the Feynman diagram.

Firstly, the intermediate particles in a chosen physics process are injected into the graph. Secondly, instead of connecting all objects to one another, they are instead connected to all of their potential mother particles.
%Then, instead of predicting the edges based on whether objects share a common origin, the exact connect of mother to daughter particle is predicted.
% Then, unlike other approaches which trying to identify pairs or triplets of objects from a fully connected graph, based on a shared origin, in a Topograph the edges between mother and daughter particles are identified.
From this new graph, instead of identifying the edges based on a shared origin, true edges are identified as being between a mother and daughter particle, and the result is the reconstruction of the Feynman diagram as depicted in \cref{fig:topgraphscomp}.

Secondly, as the intermediate particles are now represented by their own nodes,  during training the kinematic properties of the injected nodes are predicted as auxiliary tasks with dedicated regression networks for each particle type.
% and not just calculated afterwards from the matched objects.
% , instead of calculated afterwards using the matched reconstructed objects. 
Edges in GNNs are not only used to identify connections but also to propagate information. By utilising the edges of the Topograph as message passing layers to update the properties of the injected nodes, properties of the intermediary particles are extracted from the graph  and regression networks can predict their kinematic properties. 
This is advantageous both through improvements in the edge classification, but also from the additional extracted information.
% This advantage provides additional benefits in b

One leading advantage of Topographs over fully connected GNNs can be seen in Fig.~\ref{fig:topedges}, showing the simple case of identifying the jets from the decay of a single top quark.
In comparison to the fully connected GNN, which has $N\left(N-1\right)$ edges, the Topograph only requires $\mathcal{O}\left(N\right)$, which scales linearly with the number of intermediary particles; in this case there are $2N$ edges. %; a further reduction in comparison to combinatoric approaches.  %, which in this example is $2N$.
For cases where $N>M+1$, where $M$ is the number of intermediary particles, there are always fewer edges associated to reconstructed objects in a Topograph than a fully connected GNN.
% Complex underlying physics processes can be injected as priors simply by changing the injected particles and their potential connections. This is a major advantage over standard approaches, and enables additional information to be included when training the networks.
% In addition to the classification of edges, the properties of the injected intermediary particles are predicted by the Topograph model. This helps with the performance of the edge classification task but also provides substantial extra benefits in event .
% Additionally, in edge classification in GNNs, the same network is applied to all edges to predict whether it is a true edge or not. In the case of a top decay, for the case of a fully connected GNN, this network should treat the edge between each jet from the $W$ boson and the jet from the $b$-quark from the top decay the same as the edge between the two jets from the $W$ boson. In the Topograph, in comparison, a true edge is always a connection between a daughter particle and its mother particle, where some of these edges can be predefined as is the case between the $W$ and $t$ nodes in Fig.~\ref{fig:topedges:topograph}.


% An additional advantage of the Topograph method is that not only are the topology and combinatorics of a process reconstructed, but the kinematics and properties of intermediary particles are predicted. This introduces greater flexibility and precision in analyses and could be used to test the likelihood of a hypothesised process similar to the matrix element method~(MEM)~\cite{Fiedler_2010}. 
% Topographs capture the benefits of distinct approaches used to solve combinatorics and predict the kinematics of intermediate particles into a single method without the computational limitations.

\begin{figure}[h]
    \centering
    \begin{subfigure}{0.45\textwidth} 
        \centering
        \includegraphics{figures/FCGraph}
        \caption{Fully connected graph}
        \label{fig:topedges:fcg}
    \end{subfigure}
    \begin{subfigure}{0.45\textwidth} 
        \centering
        \includegraphics{figures/Topograph}
          \caption{Topograph}
          \label{fig:topedges:topograph}
    \end{subfigure}
\caption{Comparison of the edges in \subref{fig:topedges:fcg} fully connected GNNs %, as used for edge classification in machine learning tasks,
and \subref{fig:topedges:topograph} Topographs, when identifying the three jets $j$ which originate from a top quark decay.
% The nodes, labelled $j$, represent the jets at arbitrary positions in the detector $\eta-\phi$ plane.
The blue nodes in \subref{fig:topedges:topograph} represent the injected nodes, a key component of the Topograph, for both the top quark ($t$) and $W$ boson ($W$). The true edges, which identify jets which \subref{fig:topedges:fcg} originate from the same top quark or \subref{fig:topedges:topograph} reconstruct the decay chain, are green. All false edges in the graphs are dashed red lines. The predefined connection between $t$ and $W$ boson is blue.}
\label{fig:topedges}
\end{figure}

% With Topographs, complex underlying physics processes can be injected as priors by changing the injected particles and their potential connections. This enables additional information to be included when designing and training the networks over standard approaches.

% A second improvement with Topographs is that the true kinematics of injected particles are predicted by the network, 
% and not just calculated afterwards from the matched objects.
% , instead of calculated afterwards using the matched reconstructed objects. 
%Edges in GNNs are not only used to identify connections but also propagate information. By utilising message passing GNN layers, properties of the intermediary particles are extracted from the graph, and regression networks can predict their kinematic properties. 
%This is advantageous both through improvements in the edge classification, but also from the additional extracted information.
%% This advantage provides additional benefits in both physics analyses and the application in flavour tagging. %, bringing it closer to the Matrix Element Method, but will also help improve the edge classification performance.
%% Preliminary studies show this task improves the edge classification performance in the Topograph.
%% Furthermore, it provides an additional benefit of the approach in both physics analyses and the application in flavour tagging.

Furthermore, having a handle on the underlying kinematics of the intermediary particles with auxiliary tasks should improve the resolution and accuracy of current differential measurements, which are often a function of intermediate particle kinematics.
These properties could also be used in a likelihood test of the process, as is done in the MEM, instead of only using objects recorded by the detector.
% These properties could also be used in a full likelihood test of the process, as is done in the MEM, instead of being restricted to the objects recorded by the detector.

With Topographs, complex underlying physics processes can be injected as priors by changing the injected particles and their potential connections. This enables additional information to be included when designing and training the networks over standard approaches.

\subsection*{Building blocks}

Although the Topograph could be visualised as one large graph with injected nodes and edges, we break them down into a simple set of building blocks.
The core building block of the Topograph is the particle block.
It includes the edge definitions between an input set of particles, or nodes, and the target mother particle. In addition, it contains the regression network used to predict the kinematic properties of the injected mother particle.
If a Topograph is visualised as a Feynman diagram of a process, a particle block is the subcomponent which determines the correct connections to recreate a single vertex alongside the properties of the incoming particle.
% The particle block can be interpreted as a subcomponent in a Topograph, which represents the Feynman diagram of a process, trying to determine the correct connections to recreate a single vertex and the properties of the incoming particle.
The basic representation of a particle block for a mother particle $M$ is depicted in \cref{fig:pblock:one} alongside a representative Feynman diagram vertex.
% The complexity of the Topograph is restricted by enforcing the underlying priors from particle physics, but also in a machine learning sense the networks are not complex. As in standard graphs, which are described by nodes and edges, the same network and layers are used for all edges, and equivalently for all nodes. This results in the same edge classifier being used to predict the true edges between for example $W$ bosons and the input particles, and $top$ quarks and all input particles.
The injected mother particle $M$ can be initialised with random values, or using information extracted from all potential daughter particles.
Its properties are learned from message passing layers between itself and the input particles.

\begin{figure}[h]
    \centering
    \begin{subfigure}{0.5\textwidth} 
      \centering
      \includegraphics{figures/pblock}
    \end{subfigure}
    \begin{subfigure}{0.4\textwidth} 
      \centering
      \includegraphics{figures/Mvertex}
    \end{subfigure}
    \caption{The particle block for a given mother particle $M$. %, the building block of the Topograph.
    All input particles are connected to the mother particle by an edge; for illustrative purposes the true edges are shown in green and the false edges are represented by dashed red lines.
    The trapezoid $M_\mathit{reg}$ is the regression network which predicts the kinematic properties of $M$. % from its learned properties. %The output of the block is the node representing $M$ which behaves like all other nodes.
    The inset is the notation used to represent the whole particle block.
    Shown alongside the Feynman diagram representing the process, with time running from bottom to top.}
    % \caption{The composition of a basic particle block for a given mother particle $M$.
    % All input particles are connected to the mother particle by an edge, for illustrative purposes the true edges are shown in green and the false edges are represented by dashed red lines.
    % The trapezoid $M_{reg}$ is the regression network which learns to predict the kinematic properties of $M$ from its learned properties. The output of the block is the node representing $M$ which behaves like all other nodes.
    % The inset is the notation used to represent the whole particle block.
    % Shown alongside the Feynman diagram, to demonstrate the prior information being exploited, with time running from bottom to top.}
    \label{fig:pblock:one}
\end{figure}
    


Any hypothesised process can be described by combining multiple particle blocks into a single network, connecting them to the input particles and to one another.
% Multiple particle blocks are incorporated into the same network to accommodate any hypothesised process
% and particle blocks can be connected to one another.
Edges between objects and particle blocks can also be predefined, for example between the particle blocks for a $W$ boson and a top quark, as shown in \cref{fig:pblock:top}.
% Similarly, multiple particle blocks can be used together to describe the hypothesised process, for example in Fig.~\ref{fig:topottW:simple} the $t\bar{t}W$ process is described by two $t$ blocks and a $W$ block, connected to all particles.
% Full processes can be described by combining multiple particle blocks into a single network and connecting them to the input particles.
% For the production of a top quark pair in association with a $W$ boson ($t\bar{t}W$), two top quark blocks and one $W$ block are connected to the input particles, but not to one another. The Topograph network is trained to identify the edges of the true daughter particles of each particle in the process, and predicts the kinematics of the two top quarks and the three $W$ bosons.
% Starting from the particle blocks, processes can be described by combining multiple into a single network and connected them to the input particles. For the production of a top quark pair in association with a $W$ boson ($t\bar{t}W$), two top quark blocks and one $W$ block would be connected to the input particles, but not to one another. The Topograph then identifies the edges of the true daughter particles and predicts the kinematics of the two top quarks and the three $W$ bosons in the process.

% In building a Topograph network, multiple particle blocks can be incorporated into the same network to accommodate any prior physics process. The injected particles can be connected to other injected particles, either with connections which need to be predicted or with predefined connections.
% A particle block for the top quark, which contains a particle block for the $W$ boson with a prior connection is depicted in Fig.~\ref{fig:pblock:top}. The combination of particle blocks here matches the Topograph depicted in Fig.~\ref{fig:topedges:topograph}, with the inclusion of the regression networks to predict the properties of the $W$ boson and top quark.
% Likewise, multiple particle blocks can be used together to describe the hypothesised process, for example in Fig.~\ref{fig:topottW:simple} the $t\bar{t}W$ process is described by two $t$ blocks and a $W$ block, connected to all particles.

% One area for additional study in the development of these models and the Topograph toolkit in this project will be how to extract the most information from the set of inputs to improve the edge classification accuracy, and the mother particle regression. One approach is to include an information exchange layer, as shown in Fig.~\ref{fig:topottW:interact}. This layer could be a message passing GNN layer, or make use of attention transformers~\cite{vaswani2017attention}.


\begin{figure}[h]
  \centering
  \begin{subfigure}[b]{0.5\textwidth} 
    \centering
    \includegraphics{figures/topblock}
  \end{subfigure}
  \begin{subfigure}[b]{0.4\textwidth} 
    \centering
  %   \feynmandiagram[vertical=b to a]{
  %     i1 [particle=\(p_1\)] -- [Green,very thick,fermion] c -- [Green,very thick,fermion] i2 [particle=\(p_3\)],
  %     i3 [particle=\(p_N\)] -- [Green, very thick, fermion] b -- [blue, very thick, photon] c,% [particle=\(W^\pm\)],
  %     a -- [blue,fermion] b,% [blue,particle=\(t\)],
  % };
  % Using the layered layout
  \includegraphics{figures/tdecay}

  \end{subfigure}
  \caption{Particle block of a top quark $t$. A particle block for the $W$ is nested within the $t$ block. %, with all input particles connected to both it and the $t$ node.
  A connection between the $W$ boson and the top quark is predefined (shown in blue).
  % In the $t$ block there is only one true edge to the $t$ node to be predicted, as the other true edge is defined by the physics prior, here depicted in blue.
  Shown alongside the corresponding Feynman diagram.}
  \label{fig:pblock:top}
\end{figure}

\subsection*{Assembling a neural network with Topographs}

For many processes, there will be more than one reconstructed object type or final state particle in the chosen process. To address this a set of neural networks $\phi_p$, one for each particle type $p$, can be incorporated into the Topograph model as a series of particle embedding networks. %This is analogous to using tokenisation for attention networks~\cite{vaswani2017attention}.
Furthermore, in order to maximise initial information exchange before the Topograph it may be beneficial to include a normal message passing layer before the Topograph.
% This will be in the form of a different neural network or transformer $\phi_p$ for each type of particle $p$.
% This will ensure that all particles are handled consistently in the network.
% An additional area of study will be how to extract the most information from the set of inputs to improve the Topograph predictions. One potential development is to include an information exchange layer between all initial particles. %, as shown in \cref{fig:topottW:interact}.
Options for this layer include attention transformers and standard message passing GNN layers.
% This layer could be a message passing GNN layer, or make use of attention transformers~\cite{vaswani2017attention}.
Furthermore, in complex processes it is possible to define which edges are to be predicted and which are fixed. For example, two leptons in the production of a $Z$ boson in association with two top quarks could be set to originate from the $Z$ boson, or one from each $W$ boson.

% A complete Topograph network is shown in \cref{fig:topottW:embed} for reconstruction of $t\bar{t}W$ in events containing exactly one lepton, one neutrino and multiple jets.
% As Topographs are defined to represent an underlying physics process, it is also a choice whether to predefine whether the lepton and neutrino originate from a $W$ boson coming from a top decay, or the additional $W$ boson.
% % Topographs can be used to test both hypotheses simultaneously, or be used without predefined edges to predict the likeliest assignment itself.

% \begin{figure}[ht]
%   \centering
%   \begin{subfigure}[b]{0.48\textwidth}
%     \centering
%     \resizebox*{\textwidth}{!}{%}
%     \includegraphics{figures/topotttW-interact-embed}%
%     }
%     \caption{}
%     \label{fig:topottW:embed-noprior}
%   \end{subfigure}
%   \begin{subfigure}[b]{0.48\textwidth}
%     \centering
%     \resizebox*{\textwidth}{!}{%}
%     \includegraphics{figures/topotttW-interact-embed-prior}%
%     }
%     \caption{}
%     \label{fig:topottW:embed-prior}
%   \end{subfigure}
%   \caption{Topograph network for the $t\bar{t}W$ process comprising two $t$ blocks and a $W$ block.
%   The input particles are first passed through an embedding network $\phi$ unique to each type of particle; either a jet $j$, lepton $\ell$ or neutrino $\nu$.
%   Embedded particles are then passed through an (optional) information exchange layer.
%   All particles are connected to all possible mother particles, as shown by the dashed edges.
%   The $\ell$ and $\nu$ in \subref{fig:topottW:embed-prior} are predefined to connect to the $W$ in one top block, whereas in \subref{fig:topottW:embed-noprior} no prior connections are enforced. They could equally be defined to originate from the rightmost $W$ block.}
%   % In \subref{fig:topottW:embed-prior} $\ell$ and $\nu$ are predefined to connect to the $W$ in one top block, whereas in \subref{fig:topottW:embed-noprior} no prior connections are enforced.}
%   \label{fig:topottW:embed}

% \end{figure}


% \begin{figure}[t]
%   \centering
%     \includegraphics{figures/topotttallhad-interact-embed}%
%   % \caption{Topograph network for the $t\bar{t}W$ process comprising two $t$ blocks and a $W$ block.
%   % The input particles are first passed through an embedding network $\phi$ unique to each type of particle; either a jet $j$, lepton $\ell$ or neutrino $\nu$.
%   % Embedded particles are then passed through an (optional) information exchange layer.
%   % All particles are connected to all possible mother particles, as shown by the dashed edges.
%   % The $\ell$ and $\nu$ in \subref{fig:topottW:embed-prior} are predefined to connect to the $W$ in one top block, whereas in \subref{fig:topottW:embed-noprior} no prior connections are enforced. They could equally be defined to originate from the rightmost $W$ block.}
%   % % In \subref{fig:topottW:embed-prior} $\ell$ and $\nu$ are predefined to connect to the $W$ in one top block, whereas in \subref{fig:topottW:embed-noprior} no prior connections are enforced.}
%   % \label{fig:topottW:embed}
% \end{figure}