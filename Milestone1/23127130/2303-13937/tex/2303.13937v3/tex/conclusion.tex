In this work we have introduced Topographs, a novel approach for solving the combinatorics and reconstructing the topology of a particle physics process from final state objects reconstructed by a detector.
The performance matches the current state-of-the-art technique using symmetry preserving attention transformers, and surpasses the standard approach commonly used in analyses, with a computational complexity which scales only linearly with increasing final state object multiplicity.
% Although not studied in this work, training Topographs on partial events is could increase the overall performance. % in line with other approaches.

The edge scores from Topographs can be combined into discriminants to assign a confidence to the jet-parton assignments, which could be useful in downstream applications.
Furthermore, the additional regression tasks included in Topographs demonstrate good predictive power with similar accuracy but reduced bias compared to using only the jets assigned to intermediate particles.
However, in both cases there remains room for improvement.

There are several other areas open for further optimisation.
Due to the message passing layers used to define the Topograph, it was found that fully connected graph layers between all jets for information exchange lead to faster convergence whilst training, and also resulted in requiring fewer learnable parameters in the model.
This causes the complexity of the network in this work to scale quadratically with the number of jets, and not linearly as can be achieved by only using the particle blocks in Topographs.
By moving to other architectures such as Transformer Encoders with cross-attention, the need for information exchange layers could be mitigated.
Alternative approaches for assigning jets to partons based on the edge scores could also improve the overall performance.

% In Ref.~\cite{shmakov2021spanet} it was found that training the network using only partial events improved the performance on all events. This is also possible with Topographs but has not yet been studied in great detail.
Applications of Topographs are not limited to the case study presented in this paper, and due to their modular nature Topographs can be generalised to almost all particle physics processes.
Their applications are also not limited to matching final state objects to an underlying physics process, but could also find use in jet identification, reconstructing displaced vertices from heavy flavour hadron decays or using the constituents in large radius jets in boosted topologies.