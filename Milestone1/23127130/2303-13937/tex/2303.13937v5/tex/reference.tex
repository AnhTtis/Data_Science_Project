
\subsection{Reference methods}

\subsubsection*{$\chi^2$ implementation}
The $\chi^2$ score for \ttbar decays used in this work is given by
\begin{align*}
    \chi^2 = \frac{(m_{b_1q_1q_2} - m_t)^2}{\sigma_t^2} + \frac{(m_{b_2q_3q_4} - m_t)^2}{\sigma_t^2}\\ + \frac{(m_{q_1q_2} - m_W)^2}{\sigma_W^2} + \frac{(m_{q_3q_4} - m_W)^2}{\sigma_W^2},
\end{align*}
where $m_{b_iq_jq_k}$ is the invariant mass of the jets in the permutation, $m_t$ and $m_W$ are the masses of the top quark and W boson, and $\sigma_t$ and $\sigma_W$ are the widths of the top quark and W boson in the dataset.
The values for $m_t$ and $m_W$ are obtained by taking the mean of the reconstructed invariant masses in our dataset, while $\sigma_t$ and $\sigma_W$ are set to the standard deviation.
The calculated values are $m_t=169.8\ \mathrm{GeV}$, $m_W=81.0\ \mathrm{GeV}$, $\sigma_t=29.0\ \mathrm{GeV}$, and $\sigma_W=18.5\ \mathrm{GeV}$.
Jets associated to the $b$-quark of the top decays are required to be $b$-tagged, while no requirement is placed on the jets associated to the $W$ boson decays.

\subsubsection*{\spanet implementation}
% For the comparison with \spanet, models are trained on the same data as the Topograph.
The implementation is taken from the github repository\footnote{\url{https://github.com/Alexanders101/SPANet/tree/v1.0}}.
The hyperparameters of the model are optimised for the dataset using fully matched events, with the values in Ref.~\cite{fenton2021permutationless} using \emph{v1.0} resulting in the highest overall efficiencies.
%  with those in the tag labelled \emph{v1.0} with the hyperparameters for the full training.
% For these comparisons only fully reconstructable events are used in training. Using these events was found in Ref.~\cite{shmakov2021spanet} to increase the performance, however for fairer comparison with the current Topograph implementation this is not performed.
% Results for the \spanet models trained with partial events are presented in \cref{app:partialspa}.
All \spanet models are trained for 100 epochs, taking the weights after the epoch with the lowest loss on the validation set.

\subsection{Partial event trainings}

It is also possible to train Topographs on events in which not all partons from the $t\bar{t}$ decay are matched to jets.
For these events, the same network is used but the edge classification and regression loss terms are not considered for the $W$ boson or top quarks which have partons not able to be matched to jets. % (top quark or $W$ boson). % are not considered.
In the case of the $b$-quark from the top quark decay not being matched to a jet, the $\mathcal{L}(W_i,W^+)$ term is still considered.
Where a quark from the $W$ boson decay is missing, both the top quark and $W$ boson loss terms are not considered.
At least one $W$ boson is required to be fully reconstructable, with both quarks matched to jets in the event.

As introduced in Ref.~\cite{shmakov2021spanet}, \spanet can also be trained on non-fully reconstructable events, however in comparison to Topographs this is only at the level of each top quark.
When a $b$-quark from the top quark is missing, the corresponding $W$ boson is also not considered.
For a fairer comparison with both models trained on the same number of events, we compare models trained only on fully reconstructable events.
Results for the Topograph and \spanet models trained with partial events are presented in \cref{app:partialspa}.

% Topographs and \spanet can both be trained using events in which not all partons from the t\bar{t} decay are matched to jets.
% For these events, the 