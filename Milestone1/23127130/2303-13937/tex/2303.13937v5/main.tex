\documentclass[aps,prd,reprint,longbibliography,nofootinbib,superscriptaddress]{revtex4-2}
% \usepackage[latin1]{inputenc}
\usepackage[british]{babel}
\usepackage[all]{xy}
\usepackage{amscd}
\usepackage{amssymb}
\usepackage{amsthm}
\usepackage{enumitem}
\usepackage{mathrsfs,bbm}
\usepackage{xcolor,graphicx}
\usepackage{graphics}
\usepackage{soul}
\usepackage{comment}
\usepackage[all]{xy}
\usepackage{amscd}
\usepackage{amssymb,amsmath,latexsym}
\usepackage{amsthm}
\usepackage{enumitem}
\usepackage{mathrsfs,bbm}
\usepackage{dsfont}
\usepackage{tikz-cd}
\usepackage[T1]{fontenc}
\usepackage[utf8]{inputenc}  
 %
%%%%%%%%%%%%%%%%%%%%%%%%%%%%%%%%%%
%pagestyle
%%%%%%%%%%%%%%%%%%%%%%%%%%%%%%%%%%
%\pagestyle{plain}
\textwidth=430pt
\headsep=.7cm
\evensidemargin=15pt
\oddsidemargin=15pt
\leftmargin=0cm
\rightmargin=0cm
%%
%%%%%%%%%%%%%%%%%%%%%%%
\newcommand*\fixitem {\item[]%
  \refstepcounter{enumi}\hskip-\leftmargin\labelenumi\hskip\labelsep}
\newtheorem*{mainthm}{Main Theorem}
\newtheorem*{mainthm1}{Theorem}
\newtheorem*{maincor}{Corollary}
\usepackage[colorlinks=true]{hyperref}
\DeclareMathOperator{\Forall}{\forall}
\DeclareMathOperator{\Exists}{\exists}
\DeclareMathOperator{\ord}{ord}
\newcommand{\phiD}{\varphi_D}
\newcommand{\phiDI}{\varphi_{\mathbf{D}_I}}
\newcommand{\phiDIj}{\varphi_{\mathbf{D}_I (j)}}
\newcommand{\phiH}{\varphi_H}
\newcommand{\phiTimes}{\phiD \otimes \phiH}
\newcommand{\phiTimesDI}{\varphi_{\mathbf{D}_I} \otimes \phiH}
\newcommand{\R}{\mathscr{A}}
\newcommand{\X}{\mathscr{X}}
\newcommand{\Xf}{\mathscr{X}_{(k_0 ,i)}[r_0]}
\newcommand{\Xfr}{\mathscr{X}_{(k_0,i)}[r]}
\newcommand{\hotimes}{\widehat{\otimes}}
\newcommand{\C}{\mathbb{C}_p}
\newcommand{\V}{\mathscr{V}}
\newcommand{\B}{\mathscr{B}}
\newcommand{\dualD}{\mathfrak{D}}
\newcommand{\Dg}{\mathbf{D}}
\newcommand{\DD}{\mathcal{D}^0}
\newcommand{\DDg}{\mathcal{D}}
\newcommand{\DV}{\mathcal{D}}
\newcommand{\W}{\mathscr{W}_N}
\newcommand{\Ao}{\mathbf{A}^\circ}
\newcommand{\AoK}{\mathbf{A}^\circ_{\K}}
\newcommand{\AK}{\mathbf{A}_{/\K}}
\newcommand{\OOO}{\mathscr{A}^\circ}
\newcommand{\K}{\mathcal{K}} 
\newcommand{\OK}{\mathcal{O}_{\K}}
\newcommand{\varprojlog}[1]{\underleftarrow{\log\!^{#1}}}
\newcommand{\T}{\mathscr{T}}
\newcommand{\TT}{\mathbf{T}}
\newcommand{\VV}{\mathbf{V}}
\newcommand{\HH}{\mathcal{H}}
\newcommand{\hh}{\mathcal{H}^+}
\newcommand{\HG}[2]{\mathcal{H}_{#1}(#2)}
\newcommand{\hhl}{\mathcal{H}^{+,[l]}}
\newcommand{\hhj}{\mathcal{H}^{+,[j]}}
\newcommand{\hhjj}{\mathcal{H}^{+,[l,l']}}
\newcommand{\GS}{G_{\mathbb{Q},S}}
\newcommand{\Rf}{R_{(k_0 ,i)}[r_0]}
\newcommand{\Rfr}{R_{(k_0 ,i)}[r]}
\newcommand{\parT}{\langle T\rangle}
\newcommand{\Zf}{Z_{(k_0 ,i)}[r_0]}
\newcommand{\Zfr}{\mathscr{Z}_{(k_0 ,i)}[r]}
\newcommand{\ZFf}{\mathscr{Z}_{(k_0 ,i)}[r_0]}
\newcommand{\ZFfr}{\mathscr{Z}_{(k_0 ,i)}[r]}
\newcommand{\ZF}{\mathscr{Z}}

% \usepackage[nonatbib,preprint]{neurips}

% \author{%
%   Lukas Ehrke \\
%   University of Geneva\\
%   \texttt{lukas.ehrke@unige.ch} \\
%   \And
%   John Andrew Raine \\
%   University of Geneva\\
%   \texttt{john.raine@unige.ch} \\
%   \AND
%   Manuel Guth \\
%   University of Geneva\\
%   \texttt{manuel.guth@unige.ch} \\
%   \And
%   Knut Zoch \\
%   University of Geneva\\
%   \texttt{knut.zoch@unige.ch} \\
%   \And
%   Tobias Golling \\
%   University of Geneva\\
%   \texttt{tobias.golling@unige.ch} \\
% }

\begin{document}
  \title{Topological Reconstruction of Particle Physics Processes using Graph Neural Networks}
  \author{Lukas Ehrke$^a$}
  \email{lukas.ehrke@unige.ch}
  \affiliation{Département de physique nucléaire et corpusculaire, University of Geneva, Switzerland}
  \author{John Andrew Raine$^a$}
  \email{john.raine@unige.ch}
  \affiliation{Département de physique nucléaire et corpusculaire, University of Geneva, Switzerland}
  \author{Knut Zoch}
  \affiliation{Département de physique nucléaire et corpusculaire, University of Geneva, Switzerland}
  \affiliation{Laboratory for Particle Physics and Cosmology, Harvard University, Cambridge, 02138 MA, USA\\[1ex]}
  \author{Manuel Guth}
  \affiliation{Département de physique nucléaire et corpusculaire, University of Geneva, Switzerland}
  \author{Tobias Golling}
  \affiliation{Département de physique nucléaire et corpusculaire, University of Geneva, Switzerland}
  
  
  \begin{abstract}
    % We present a new approach to topologically reconstruct particle physics processes from observed final state objects at collider experiments and solve combinatoric assignment of reconstructed objects to their initial particles.
    We present a new approach, the Topograph, which reconstructs underlying physics processes, including the intermediary particles, by leveraging underlying priors from the nature of particle physics decays and the flexibility of message passing graph neural networks.
    The Topograph not only solves the combinatoric assignment of observed final state objects, associating them to their original mother particles, but directly predicts the properties of intermediate particles in hard scatter processes and their subsequent decays.
    In comparison to standard combinatoric approaches or modern approaches using graph neural networks, which scale exponentially or quadratically, the complexity of Topographs scales linearly with the number of reconstructed objects. % for any given underlying process.
    
    We apply Topographs to top quark pair production in the all hadronic decay channel, where we outperform the standard approach and match the performance of the state-of-the-art machine learning technique.
  \end{abstract}
  
  \maketitle
  \renewcommand\thefootnote{\alph{footnote}}\footnotetext{These authors contributed equally to this work}\setcounter{footnote}{0}\renewcommand\thefootnote{\arabic{footnote}}

  \section{Introduction}
  \section{Introduction}

The increasing complexity of source code poses a key challenge to the reliability of large-scale software systems. Software bugs in these systems can lead to safety issues~\cite{bug_safety} for users around the world as well as cause non-negligible financial losses~\cite{bug_loss}. As such, developers have to spend a large amount of time and effort on bug fixing. Consequently, \aprfull (\apr), designed to automatically generate patches to fix software bugs, has attracted wide attention from both academia and industry~\cite{long2016prophet, legoues2012genprog, long2015spr, lou2020can, tufano2018empstudy}. 


To achieve \apr, one popular approach is known as Generate-and-Validate (G\&V)~\cite{qi2015gv, ghanbari2019prapr, lou2020can, le2016hdrepair, legoues2012genprog, wen2018capgen, hua2018sketchfix, martinez2016astor, koyuncu2020fixminder, liu2019tbar, liu2019avatar}, which is typically based on the following pipeline: First, fault localization techniques~\cite{wong2016fl, abreu2007ochiai, zhang2013injecting, papadakis2015metallaxis, li2019deepfl, li2017transforming} are applied to determine the suspicious locations in programs where bugs are likely to exist. Then, the buggy locations are used by the \apr tools to generate a list of patches that replace buggy lines with correct lines. Afterward, each patch is validated against the original test suite to identify any \emph{plausible patches} (i.e., passing all tests in the test suite). Finally, to determine the \emph{correct patches}, developers examine the list of plausible patches to see if any of them can correctly fix the bug. 

Traditional \apr tools can mainly be categorized into heuristic-based~\cite{legoues2012genprog, le2016hdrepair, wen2018capgen}, constraint-based~\cite{mechtaev2016angelix, le2017s3, demacro2014nopol, long2015spr} and \template~\cite{ghanbari2019prapr, hua2018sketchfix, martinez2016astor, liu2019tbar, liu2019avatar}. Among these traditional tools, \template \apr tools~\cite{ghanbari2019prapr, liu2019tbar, benton2020effectiveness} have been able to achieve state-of-the-art results. \Template \apr tools typically leverage pre-defined templates (e.g., adding a nullness check) for bug fixing. However, since these fix templates are typically handcrafted, the number and types of bugs they are able to fix can be limited. 



To address the limitations of traditional \apr, researchers have proposed various \learning \apr tools~\cite{li2020dlfix, chen2018sequencer, jiang2021cure, lutellier2020coconut, zhu2021recoder, ye2022rewardrepair} based on the \nmtfull (\nmt) architecture~\cite{sutskever2014mt} where the input is the buggy code snippets and the goal is to translate the buggy code snippets into a fixed version. To accomplish this, \learning \apr tools require supervised training datasets with pairs of both buggy and fixed code snippets in order to learn how to perform this translation step. These training data are usually obtained by mining historical bug fixes using heuristics/keywords~\cite{dallmeier2007benchmark}, which can be imprecise for identifying bug-fixing commits; even the actual bug-fixing commits can include irrelevant code changes, leading to further pollution in the dataset~\cite{xia2022alpharepair}.
% 
Moreover, it can be hard for such \apr tools to generalize and fix bug types unseen during training. 



To better leverage recent advances in \plmfull{s} (\plm{s}), researchers~\cite{xia2022alpharepair, xia2023repairstudy, kolak2022patch, prenner2021codexws} have directly applied \plm{s} to generate patches without bug-fixing datasets. These \llm-based \apr tools work by either directly generating a complete code function~\cite{prenner2021codexws, xia2023repairstudy} or predict/infill the correct code snippet given its surrounding context~\cite{xia2022alpharepair, xia2023repairstudy}. By directly using \llm{s} that are pre-trained on billions of open-source code snippets, \llm-based \apr tools can achieve state-of-the-art performance on many repair datasets~\cite{xia2022alpharepair}. 


% 
%
%

Traditional \apr tools have long used the insight of the \emph{plastic surgery hypothesis}~\cite{barr2014plastic} where it states that the code ingredients to fix a bug already exist within the same project. Traditional \apr tools have manually designed pattern-~\cite{ghanbari2019prapr, saha2017elixir} or heuristic-based~\cite{jiang2018simfix, legoues2012genprog} approaches to finding and using such relevant code ingredients to generate fixes for bugs. However, the plastic surgery hypothesis has been largely ignored in \llm-based \apr. In fact, \llm provides a unique opportunity to fully automate the plastic surgery hypothesis idea via fine-tuning (learning project-specific information via model updates from the buggy project) and prompting (directly providing relevant code ingredients to the model), and make it directly applicable to different languages (since the \llm{s} are typically multi-lingual).%
Moreover, despite the intensive manual efforts involved, traditional \apr tools still cannot fully leverage project-specific information due to large search space for leveraging/composing existing code ingredients. In contrast, the project-specific information can effectively leveraged by \llm{s} due to their power in code understanding/vectorization, e.g., even partial/imprecise information may still guide \llm{s} in correct patch generation!
 To this end, we ask the question: \emph{How useful is the plastic surgery hypothesis in the era of \plm{s}}?








\mypara{Our Work.} To answer the question, we present \ourtech{\xspace} -- a \llm-based approach that automatically utilizes the plastic surgery hypothesis by systematically combining multiple fine-tuning and prompting strategies for \apr. \ourtech fine-tunes \plm{s} using two novel domain-specific training strategies: \textbf{\epfinetune} -- we fine-tune using the original buggy project by aggressively masking out a high percentage of tokens, which allows \plm to learn project-specific code tokens and programming styles; and \textbf{\rofinetune} -- which only masks out a single continuous code sequence per training sample, allowing the model to get used to the final \csapr task of predicting a single continuous code sequence. Furthermore, we directly leverage the ability for \plm{s} to understand natural language instructions and introduce a novel prompting strategy, \textbf{\idprompting}, which uses information retrieval and static analysis to obtain a list of relevant identifiers for the buggy lines. While such relevant identifiers are critical for fixing some difficult bugs, they may not be seen by the \llm during inference due to limited context window size. Through the use of prompting, we directly tell the model to use these extracted identifiers (relevant code ingredients) to generate the correct code. Finally, to perform repair, we combine all four model variants (including the base model, both fine-tuned models and the base model with prompting) for the final repair.





While our insight of leveraging the plastic surgery hypothesis for \llm-based \apr is generalizable across different types of \plm{s}, to implement \ourtech, we choose a recent \plm{\xspace}, \ctfive~\cite{wang2021codet5}, which is pre-trained on millions of open-source code snippets. \ctfive is an encoder-decoder model trained using \mspfull (\msp) objective where a percentage of tokens are masked out and each continuous masked token sequence is referred to as a masked span. Also, although we only extract relevant identifiers from the current buggy project (since this paper focuses on the plastic surgery hypothesis), our work can be easily extended to obtain other code information (such as relevant statements or functions) from other sources, such as  the massive pre-training corpora~\cite{husain2020codesearchnet} or historical bug-fixing datasets~\cite{jiang2019infer}, which can provide more coding knowledge for \llm{s}. Besides, although we mainly focus on using traditional string comparison algorithms for information retrieval in this paper, these techniques can be easily replaced by other frequency-based retrieval~\cite{robertson2009probabilistic} and neural search (or embedding-based search)~\cite{reimers2019sentence}.
  In summary, this paper makes the following contributions:


%


\begin{itemize}[noitemsep, leftmargin=*, topsep=0pt]
    \item \textbf{Dimension.} This paper is the first to revisit the important plastic surgery hypothesis in the era of \llm{s}. It opens up a new dimension for \llm-based \apr to incorporate previously neglected information from the buggy project itself to boost \apr performance. Furthermore, it demonstrates the promising future of retrieval-based prompting for modern \llm-based \apr.
    \item \textbf{Implementation.} We implement \ourtech based on the recent \ctfive model. We augment the model using two novel fine-tuning strategies: \epfinetune and \rofinetune, along with a novel prompting strategy based on information retrieval and static analysis: \idprompting. We combine the patches generated by all four models together and perform patch ranking to speed up \apr.% 
    \item \textbf{Evaluation Study.} We conduct an extensive evaluation against state-of-the-art \apr tools. On the widely studied \dfj 1.2 and 2.0 datasets~\cite{just2014dfj}, \ourtech is able to achieve the new state-of-the-art results of 89 and 44 correct bug fixes (15 and 8 more than best baseline) respectively.  Furthermore, we perform a broad ablation study to justify our design. \ourtech demonstrates for the first time that the plastic surgery hypothesis can substantially boost \llm-based \apr and advance state-of-the-art \apr, while being fully automated and general. Moreover, even partial/imprecise code ingredients may still effectively guide \llm{s} for \apr!
\end{itemize}


  
  \section{Motivation}
  \section{Threat Model and Advantages of Our Hardware-based Adversarial Detector} \label{sec: motivation}
\ry{In this part, I want to highlight the comparison between hardware and software attacks}
%Normally, software-based adversarial detectors are easier to implement, cheaper to develop and more well-studied than those based on hardware computational signals.
% We would like to stress that our goal for investigating hardware-based adversarial detectors is not to achieve better performance in detection than the conventional white-box software based methods.  
\subsection{Threat Model} \label{sec: threat model}
\ry{This section is threat model: attack is `white-box', detector is `black-box'}
The victim is a DNN classifier, which is pre-trained with a public dataset. The testing dataset may be kept private.
We assume the strongest `white-box' attack model, where the attacker has full knowledge of the victim model and training dataset in order to generate adversarial samples with minimum perturbations. 
On the contrary, the detection system assumes the most limited scenario, under a `black-box' view of the victim, without access to the victim's inputs, parameters, and intermediate outputs or execution details. 
The only information available to the detector to distinguish adversarial samples is the EM side-channel measurement and the victim model's prediction class.
For training the adversarial detector with EM traces, a public benign dataset is used. 

\if false 
\ry{In this part, we discuss more settings of the detector especially the data used in two phases.}
In general, the detecting process can be summed up into two phases, training phase and detecting phase.
To begin with, we train an Out-of-Distribution(OOD) detector on a public benign dataset of the same classification task, which should be distinct from the victim's training dataset.
For each query, the detector will obtain the classification result and an EM trace along with the model execution to fit its EM classifiers and anomaly detectors.  
During the detection phase, the victim model is in operation and under attack when the pre-trained detector decides whether the current input is adversarial or not, only based on the victim model output and its EM trace.
\fi 

\subsection{Advantages}
Compared to software-based adversarial detection methods, our hardware-based detector, EMShepherd, has three distinct advantages: privacy-preserving, portability, and robustness.

\begin{itemize}[leftmargin=*]
    \item \ry{Add a new motivation here. The motivation is that using \name can help the user protect their privacy.} 
    \name protects the DNN model user's data privacy as it is agnostic to the model's inputs, which instead are always required by prior reconstruction-based detection methods~\cite{meng2017magnet, yang2022you}. 
    %Most model users are benign whose inputs may be sensitive and should not be shared with \textit{third-party detectors}. 
    The sensitive inputs should not be shared with \textit{third-party detectors}. 
    Our design only requires the output class labels and the EM signals, which are passively leaked to common acquisition equipment. 
    %    Our design is suitable for such cases as it only requires the EM signals and the inference outputs during the model execution. Generally speaking, EM signals and labels have less private information leakage.
    \item \ry{The second motivation is still related to privacy. This time we consider model privacy when the model structure or parameters should be kept private.}
   \name also protects the model confidentiality.  No model information, including %Using hardware-based detectors can prevent the third-party defender from accessing some confidential model information such as  
   hyper-parameters, parameters, and logits, is needed, in stark contrast to the previous software-based detection methods~\cite{ma2019nic,feinman2017detecting}.
    %Our \name only acquires the EM traces during model inference in a passive and noninvasive manner, 
    The EM data processing and the adversarial detector training process are both victim model-agnostic. 
    Therefore, our method has more general usage, applicable to closed-source DNN applications, which are pervasive in edge devices where the user only queries the models for the final prediction output. 
    \item \ry{The third motivation is portability.}  
    Owing to the model-agnostic feature, EMShepherd can be easily ported for wide-range hardware devices with different DNN implementations for diverse applications. It can be used as a `plug and play' (PnP) device, aside from the target system, to work automatically without user intervention or contact with the victim system. 
    \item \ry{The last motivation is about adaptive attacks, we should propose that EM signal is hard to imitate, so it is hard for adaptive attacks to generate sample fraud both detector and victim.} 
    Adaptive attack~\cite{adaptive} is a threat to most software defense methods where the attacker adjusts the adversarial perturbations to mislead both the victim models and defense systems.
   %  The hardware-based detection method can provide a double protection on top of most software defense methods such as adversarial training.
   %  Although the adptive adversarial example fools the robust model, its computation patterns during the DNN model execution are still well kept in the EM traces and our EMShepherd framework still works well for detecting the new type of adversarial examples.  
   %  Meanwhile, due to the high complexity of EM signals and non-explicit dependency of the EM signals on computations, it is extremely hard to have an adaptive attack on our detection method, i.e., adversarial examples whose EM signals are deliberately controlled to evade the EM-based detector.
   However, due to the high complexity and non-explicit dependency of the EM signals on computations and data, 
   it is extremely hard to have an adaptive attack on our detection method, 
   i.e., adversarial examples whose EM signals are deliberately controlled to evade the EM-based detector. 
\end{itemize}





  % \section{}
  \section{Current Approaches}
  In top quark physics kinematic event reconstruction forms a key part of many measurements. The $\chi^2$ method~\cite{TOPQ-2015-03} and the Kinematic Likelihood fitter (KLFitter)~\cite{Erdmann_2014} have been employed in a large number of analyses~\cite{TOPQ-2012-08, CMS-TOP-14-022, ATLAS:2015pfy,  TOPQ-2015-03, TOPQ-2016-01,  CMS:2016oae, CMS-TOP-17-008, CMS:2018htd,  CMS:2018quc, ATLAS:2018fwq, ATLAS:2019guf, ATLAS:2019hxz, CMS:2021vhb, ATLAS:2022waa}.
In both approaches, all combinatorics of jet matching to final state quarks and gluons (partons) in the $t\bar{t}$ final state are tested with kinematic constraints based on the masses of the reconstructed $W$ bosons and top quarks as minimisation criteria.
In the case of KLFitter these are used in conjuction with  with transfer functions and the particle decay widths. %With these approaches, reasonably good performance can be achieved. However, there are two main drawbacks to this approach.

Although good performance can be achieved with such an approach, as the number of jets in an event increases, as well as the multiplicity of final state objects to be reconstructed, the number of combinations increases exponentially.
For example, in an all hadronic top quark pair event with 6 jets, there are 720 potential combinations. This is reduced to 90 by exploiting underlying symmetries. However for events with 7 jets the combinations increase to 630, and for 8 jets they increase further to 2520. %This level of computational cost is unmanageable.
Furthermore, as the exact values of the mass of the top quark and $W$ boson are used to test the likelihood of a combination, this leads to a biased estimator which focusses on assigning jets which together are closest to the hypothesised particles mass, rather than exploiting all the information about the pairs or triplets of objects.
It also assumes that in all cases the top quarks and $W$ bosons are on-shell. %As the multiplicities of objects in the final state increase, the likelier it becomes that an incorrect pair or triplet will have an invariant mass closer to the target mass than the correct assignment.

Building on the previous combinatoric approaches, simple approaches using machine learning (ML) have been developed. Instead of finding the most probable assignment using just the masses of intermediary particles, machine learning discriminants are used to identify correct assignments, exploiting more information from the event~\cite{HIGG-2017-03}.
Nonetheless, these approaches still suffer from the same problems as the KLFitter and $\chi^2$ methods, with each combination needing to be tested to identify the most likely.% combination.

Another approach which uses more information from the event is the Matrix Element Method~(MEM)~\cite{Fiedler_2010,CMS-SMP-13-004,TOPQ-2015-01,HIGG-2017-03}.
The MEM not only attempts to match objects to the final state objects in an event, but directly assesses the likelihood of observing an event given the matrix element for a process. This can be evaluated for each potential combination with the highest resulting probability chosen as the correct assignment.
However, it is extremely slow and computationally intensive.
To calculate the likelihood of an event, an integral over the whole phase space of possible final state particle momenta must be performed.
It is also reliant on a transfer function, which is used to convert the jets, charged leptons and missing transverse momentum recorded by the detector to the partons, charged leptons and neutrinos before any hadronisation and detector effects. As there is no accurate function to model this, it is at best an approximation optimised by hand.
Normalising flows present a solution to the computational challenge and approximate functions~\cite{Butter:2022vkj}, however do not yet address the combinatoric solving.

% Graph Neural Networks~(GNNs)~\cite{battaglia2018relational,wu2020comprehensive} and attention transformers~\cite{vaswani2017attention} are .
% % In GNNs, operations are applied to a permutation invariant graph which comprises nodes, which have attributes, and edges, which connect the nodes; edges, too, can have attributes. %These machine learning models are now being used more frequently in high energy physics applications~\cite{Qu:2019gqs,Moreno:2019bmu,shlomi2020graph,Bernreuther:2020vhm,Ju:2020tbo,Guo:2020vvt,Dreyer:2020brq,Hariri:2021clz,DeZoort:2021rbj,Atkinson:2021nlt,Thais:2022iok,Gong:2022lye,CMS-DP-2020-002,ATL-PHYS-PUB-2022-027}.
% GNN based approaches for combinatoric solving in secondary use edge classification techniques to identify nodes coming from a common parent~\cite{Shlomi_2021}. % commonly used in the field of machine learning. %The natural graph structure in \cref{fig:topgraphscomp:ttbargraph} is not exploited, and instead a fully connected graph is constructed from all objects recorded by the detector.
% % From the fully connected graph, where all nodes are connected via edges, correct edges are defined as those which connect objects originating from the same origin of interest. In the case of top quark pair production, this is all objects which originate from the same top quark.
% % This necessitates $N(N-1)$ edges in the graph, where $N$ is the number of objects. As a result, the complexity grows quadratically, which is more manageable than the combinatoric approaches. However, this does not exploit all information available from the prior knowledge of the physics process.
% Typically, fully connected graphs are used, where all objects are connected to all other objects via edges, with message passing layers used to update the edge features and nodes within the graph. %, resulting in $N\left(N-1\right)$ edges.
% From the fully connected graph, correct edges are defined as those which connect objects originating from the same origin of interest. In the case of top quark pair production, this represents all objects which originate from the same top quark.
% This necessitates $N(N-1)$ edges in the graph, where $N$ is the number of objects. As a result, the complexity grows quadratically, which is more manageable than the combinatoric approaches. However, this does not exploit all information available from the prior knowledge of the physics process.

\subsection*{State of the art}

The state of the art machine learning approach uses attention transformers~\cite{SAJANet,fenton2021permutationless,shmakov2021spanet} to identify the indices of final state objects coming from intermediate particles. In this approach no graph structure is used and only the permutation invariant collection of objects are considered. %The goal of these approaches is to identify the most probable combinations of objects using tensors without having to cycle over all possible combinatorics. 
The complexity of the approach can be reduced by taking into account the symmetries, as performed in Refs.~\cite{fenton2021permutationless,shmakov2021spanet} (SPA-Net), corresponding to removing potential solutions in the combinatoric approaches, which leads to an overall complexity of $\mathcal{O}\left(N^2\right)$. %With a clever design these approaches also have a complexity of $\mathcal{O}\left(N^2\right)$.

Graph Neural Networks~\cite{battaglia2018relational,wu2020comprehensive} are also employed in HEP to associate objects to a common origin, for example in secondary vertex reconstruction~\cite{Shlomi_2021} and could similarly be applied to combinatoric solving at the event level. These approaches have fully connected graphs with $N(N-1)$ edges.

In addition to their reduced computational complexity in comparison to traditional approaches, both attention and GNN approaches also demonstrate reduction in biases towards particle masses, as often seen in the combinatoric approaches.
However, in both GNNs and \spanet the target is to identify the two triplets of objects which correspond to the decay of each top quark, neglecting the structure of the decay, and the properties of the intermediary particles.
% As a consequence, the approaches need to be redesigned for each application, and blocks are not reused within the same model even where there are multiple objects of the same type.
%The reduction in bias from the method in Ref.~\cite{fenton2021permutationless} can be seen in Fig.~\ref{fig:spatter_mass} for the case of reconstructing the $W$ boson in all-hadronic top quark pair production events. %The $\chi^2$ method selects $W$ candidates with a much narrower spread around the true $W$ boson mass, but with a higher rate of incorrect matches than in the attention based approach.
% \begin{figure}[h]
%     \centering
%     \begin{subfigure}{0.49\textwidth}
%         \includegraphics[width=\textwidth]{figures/spatter_w_chi2}
%         \caption{\ }
%         \label{fig:spattw:chi2}
%         % \phantomcaption
%     \end{subfigure}
%     \begin{subfigure}{0.49\textwidth}
%         \includegraphics[width=\textwidth]{figures/spatter_w}
%         % \phantomcaption
%         \caption{\ }
%         \label{fig:spattw:spa}
%     \end{subfigure}
%     \caption{The mass of the candidate $W$ bosons calculated from the invariant mass of the two jets associated to the $W$ boson using the \subref{fig:spattw:chi2} $\chi^2$ approach and \subref{fig:spattw:spa} the attention transformer in Ref.~\cite{fenton2021permutationless}.
%     The correct assignments are represented by the red component with incorrect solutions in blue. A higher accuracy is present in \subref{fig:spattw:spa} alongside a much reduced sculpting of the mass distribution towards the $W$ peak at 80.4~GeV.}
%     \label{fig:spatter_mass}
% \end{figure}
% Nevertheless, just as is the case for current graph applications, in the attention approaches presented in Ref.~\cite{fenton2021permutationless}, the target is to identify the two triplets of objects originating from the same top quark, neglecting the structure of the decay.
% As a consequence, the approaches need to be redesigned for each application, and blocks are not reused within the same model even where there are multiple objects of the same type. %Furthermore, the same trained network cannot be easily transferred to another application using transfer learning, which could reduce the computing resources required for each new setting.
% Efforts to improve these approaches and provide a new toolkit using modern machine learning are not currently undertaken by Swiss institutes in the ATLAS or CMS collaborations, though any solutions and improvements will be relevant to the physics programmes in all groups.

  \section{The Topograph}
  \section{Method}
\label{sec:method}

% \ml{``Inconsistent'' to ``large variation''}

% In this section, we propose our methods based on the observations in Section \ref{sec:motivation}.
In this section, we propose two techniques to further enhance the strong baseline to capture the variation of activation distributions better.
We first introduce spatial re-scaling to adapt the network to pixel-to-pixel variation.
We then propose channel-wise shifting and re-scaling to better capture the channel-to-channel variation.
Meanwhile, as both of the two methods are image-dependent, the image-to-image variation can be captured naturally.
By combining the two methods with our strong baseline, we build our enhanced BNN for SR, named EBSR.

% Because the activation distributions among pixels, channels and images have large variations \red{**are highly inconsistent} in SR networks, we introduce spatial re-scaling to adapt to pixel-wise variations and channel shift and re-scaling to adapt to channel-wise variations. And both of them are image-dependent to adapt to image-wise variations, which means during inference our network re-scales and shifts the distributions of activations flexibly for different input images. Based on these methods, we build an enhanced binary neural network for image super-resolution (EBSR).

% According to [3], the difference of activation magnitudes indicates different scaling factors are needed for each pixel.

\subsection{Spatial Re-scaling}
% It is better to use different scaling factors for different pixels to reduce the quantization error and retain more detailed information for image super-resolution. 

% \ml{In the main method, we do not need to introduce the previous works but can focus on introducing our own method. Channel rescaling in Real-to-binary Net is not relevant in this context.}

% Re-scaling the output of binary convolutions was proposed at the birth of BNN in XNOR-Net \cite{rastegari2016xnor} to reduce quantization error and improve accuracy for image classification tasks.
% It is computed as below:
% \begin{equation}
% \mathcal{A} * \mathcal{W} \approx(\operatorname{sign}(\mathcal{A}) \circledast \operatorname{sign}(\mathcal{W})) \odot \mathcal{K} \alpha
% \label{eq:xnor-net rescale}
% \end{equation}
% where $\circledast$ denotes the binary convolution and $\odot$ denotes the element-wise multiplication.
% $\mathcal{A}$, $\mathcal{W}$, $\alpha$, and $\mathcal{K}$ denote the activation, weight, weight scaling factor, and activation scaling factor, respectively.
%  Later in XNOR-Net++ \cite{bulat2019xnor}, Bulat et al. fuse the activation and weight scaling factors into a single one that is learned end-to-end based on gradients and this improves the classification accuracy on ImageNet dataset.

% % It is computed as Eq.~\ref{eq:xnor-net rescale}, where $\circledast$ denotes 
% %  the binary convolution and $\odot$ denotes the element-wise multiplication. The binary convolution of $\mathcal{A}$ and $\mathcal{W}$ is rescaled by the weight scaling factor $\alpha$ and the activation scaling factor $\mathcal{K}$, both of which are calculated analytically.


% \zc{Similarly, you should explain the meaning of A, W and the operators $\circledast$ in the formula}
% Then in Real-to-binary Net \cite{martinez2020training}, Martinez et al. used a data-driven channel re-scaling module that takes the pre-convolution activations as input to predict the activation scaling factor. Unlike that in XNOR-Net++ \cite{bulat2019xnor}, these scaling factors are not fixed during inference but rather inferred from data. By doing this, they further improved the classification accuracy on ImageNet over XNOR-Net++. 
As is shown in Figure \ref{fig:pixel}, activation distributions have large pixel-to-pixel variation in SR networks
and the difference of activation magnitudes indicates different scaling factors are preferred for different pixels.
Inspired by \cite{martinez2020training}, we propose spatial re-scaling to better adapt the network to the spatial variation
of activation distributions in SR networks.
% fit the various pixel-wise distributions in SR networks.
We take the real-valued activations $A$ before convolution as input and predict pixel-wise scaling factors $S(A)$, which re-scale the binary convolution output. Spatial re-scaling process can be formulated as follows:
\begin{equation}
A * W \approx(\operatorname{sign}(A) \circledast \operatorname{sign}(W)) \odot \alpha \odot S(A)
\label{eq:spatial rescale}
\end{equation}
where $\circledast$ denotes 
the binary convolution and $\odot$ denotes the element-wise multiplication. $A$, $W$, $\alpha$, and $S\left(A\right)$ denote real-valued activations, weights, the scaling factor of weights, and the spatial-wise scaling factor of activations respectively. $S\left(A\right) \in \mathbb{R}^{1\times H\times W}$ can be calculated with a convolution and a sigmoid function.
% as $\sigma\left( CONV\left(A\right)\right)$. 
As shown in Figure \ref{fig:method}(a), real-valued activations first go through a convolution layer,
which has an input channel of $C$ and an output channel of 1, 
and then pass through a sigmoid function to produce the scaling factors $S(A)$ along the spatial dimension.
During inference, the scaling factor will change dynamically according to different input feature maps.
By re-scaling binary convolution output using $S(A)$, we can reduce the quantization error and the original pixel-wise information in FP activation
will be preserved much better.
Spatial re-scaling leads to a large PSNR improvement of 0.24 dB (from 30.30 dB to 31.54 dB) on Set5 and 0.22 dB (from 25.09 dB to 25.31 dB)
on Urban100 compared with our strong baseline. 

\subsection{Channel-wise Shifting and Re-scaling}

\begin{table}[!tb]
\centering
\caption{Comparison between whether to fuse channel-wise shifting and re-scaling or not based on our baseline with spatial re-scaling. }
\label{tab:fusing}

\scalebox{0.65}{
\begin{tabular}{c|cc|cc|cc}
\hline
\multirow{2}{*}{Method}     & \multirow{2}{*}{OPs} & \multirow{2}{*}{Params} & \multicolumn{2}{c|}{Set5} & \multicolumn{2}{c}{Urban100} \\ \cline{4-7} 
                            &                      &                         & PSNR        & SSIM        & PSNR          & SSIM         \\ \hline
Baseline + spatial re-scale & 2.16G                & 0.05M                   & 31.54       & 0.883       & 25.31         & 0.759        \\
+ channel-wise shift and re-scale             & 2.34G                & 0.09M                   & 31.61       & 0.885       & 25.35         & 0.761        \\
+ w/ fusing                   & 2.27G                & 0.08M                   & \textbf{31.64}       & \textbf{0.885}       & \textbf{25.36}         & \textbf{0.761}        \\ \hline
\end{tabular}
}
\end{table}

In SR networks, activation distributions exhibit larger channel-to-channel variation (Figure \ref{fig:chl}).
Both the mean and magnitude of the activation distributions vary significantly across channels.
% Thus we use channel-wise shifting and re-scaling to adapt to various channel-wise distributions. 
\cite{martinez2020training} has proposed the data-driven channel re-scaling, 
but our method differs from them in further introducing data-driven thresholds to handle the channel-wise variation of both mean and magnitude.
Since the blocks to generate the scaling factors and thresholds are very similar, we further propose to fuse them into one module.
% and fusing channel-wise shifting and re-scaling into one module.
We evaluate the effect of fusing the two blocks in Table \ref{tab:fusing}.
With channel-wise shifting and re-scaling fused, our models have fewer operations and parameters overhead and slightly higher performance.

For the specific process, we take the real-valued activations as input and predict different thresholds and scaling factors for each channel. They are also image dependent, e.g., $\beta_{i}$ in Eq.\ref{eq:act_binarize} is no longer fixed during inference but generated according to different input feature maps. Channel-wise shifting and re-scaling can be formulated as follows:
\begin{equation}
A * W \approx(\operatorname{sign}(A-C_s(A)) \circledast \operatorname{sign}(W)) \odot \alpha \odot C_r(A)
\label{eq:channel-wise_shift_and_rescale}
\end{equation}
where $\circledast$ denotes 
the binary convolution and $\odot$ denotes the element-wise multiplication. $C_s(A), C_r(A) \in \mathbb{R}^{C\times1\times1}$ denote the channel-wise threshold and scaling factor, respectively. 
We show the block diagram in Figure \ref{fig:method}(b).
The real-valued input feature map is first squeezed to a ${C\times1\times1}$ vector by a global average pooling (GAP) layer.
The subsequent fully connected layers and ReLU learn the channel-wise information and output a ${2C\times1\times1}$ vector.
Then the ${2C\times1\times1}$ vector is split into two ${C\times1\times1}$ vectors.
We use the first $C$ channels as the channel-wise bias and pass the last $C$ channels through a sigmoid layer 
as the channel-wise scaling factor, which are used to shift the real-valued activations and re-scale the binary convolution output, respectively. 


% \ml{We can mention previously, channel-wise re-scale has been proposed. We propose to fuse them. Add the comparison between fuse v.s. no fuse.}

\begin{figure}[!tbp]%
  \centering
    \includegraphics[width=0.4\textwidth]{fig/methods.png}
  
% \subfloat[channel-wise shifting\&re-scale]{
%     \label{subfig:channel-wise shifting and re-scale}
%     \includegraphics[width=0.2\textwidth]{fig/chl shift and rescale.png}
%   }

  \caption{Block diagram for spatial re-scaling, and channel-wise shifting and re-scaling.} 
  % Input A is the real-valued activation tensor and C, H, and W denote its dimension. GAP stands for global average pooling. The reduction ratio r is set to 16 for a better trade-off between the performance and the number of operations and parameters.}
  \label{fig:method}
\end{figure}


\subsection{Network Structure}

Combining the spatial re-scaling and the channel-wise shifting and re-scaling methods, we construct the enhanced convolution layer (E-Conv).
Then we build our EBSR model based on E-Conv.
In Figure \ref{fig:E-conv}, we compare the binary convolution layer used in the baseline network and our proposed E-Conv.
We use spatial and channel-wise scaling factors to re-scale the binary convolution output,
and use channel-wise shifting to learn appropriate thresholds for each channel before binarization.
The scaling factors and threshold used in E-Conv are learnable and depend on the real-valued input activations.
In this way, our proposed EBSR can adapt to pixel-to-pixel, channel-to-channel, and image-to-image variations
to reduce the large binarization error and preserve more details.
% In this way, our proposed E-Conv reduces the large quantization error caused by binarization and keeps the original information of input feature maps to a large extent.


\begin{figure}[!tb]%
  \centering

    \includegraphics[width=0.5\textwidth]{fig/E-conv.png}

  \caption{Comparison of (a) the binary convolution layer with a skip connection used in our baseline network and (b) the proposed E-Conv.}
  \label{fig:E-conv}
\end{figure}


Figure \ref{fig:network} shows the basic block based on the E-Conv and our EBSR composed of the basic blocks. Following existing works, the convolution layers in the head and tail modules are not binarized. We choose the lightweight EDSR which has 16 basic blocks and 64 channels, and EDSR which has 32 basic blocks and 256 channels as our backbones, which correspond to EBSR-light and EBSR, respectively.

\begin{figure}[!tb]%
  \centering
  {
    \includegraphics[width=0.35\textwidth]{fig/network.png}
  }
  
  \caption{The structure of our proposed EBSR.  Convolution layers in purple are real-valued vanilla 3x3 convolutions.}
  \label{fig:network}
\end{figure}
  
  % \FloatBarrier
  \section{Solving combinatorics in $\mathrm{t\bar{t}}$ events}
  % For an initial application and for direct comparison with other state-of-the-art methods we apply Topographs to top quark pair production with both tops decaying hadronically. 
For an initial application and for direct comparison with other state-of-the-art methods we apply Topographs to top quark pair production with both tops decaying hadronically. 
We compare the performance of our method to a benchmark non-ML approach used in many top quark analyses, the $\chi^2$ method, and the state-of-the-art ML approach, \spanet~\cite{shmakov2021spanet}.
All models are trained and evaluated on the same dataset.

20 million $t\bar{t}$ events with a centre of mass energy $\sqrt{s}=13$~TeV are simulated using MadGraph5\_aMC@NLO~\cite{MadGraph}~(v3.1.0), with decays of top quarks and $W$~bosons modelled with MadSpin \cite{MadSpin}, with both $W$~bosons decaying to two quarks.%
\footnote{Dataset available at \url{https://zenodo.org/record/7737248} \cite{zoch_knut_2023_7737248}}
The parton shower and hadronisation is performed with Pythia~\cite{Pythia} (v8.243).
The detector response is simulated using Delphes~\cite{Delphes} (v3.4.2) with a parametrisation similar to the response of the ATLAS detector.
Jet clustering is performed using the anti-$k_{t}$ algorithm~\cite{AntiKt} with a radius parameter $R=0.4$ using the FastJet~\cite{FastJet} package.
Jets originating from $b$-quarks ($b$-jets) are identified with a simple binary discriminant corresponding to an inclusive 70\% signal efficiency.

For training, events with at least six jets are selected, keeping up to 16 jets per event as ordered by their transverse momentum.
Truth matching of jets to the partons in the hard scatter is performed using a cone of $\Delta R < 0.4$.
Events with partons matched to multiple jets, or jets matched to multiple partons are discarded.
Events are further required to have zero reconstructed leptons, though no requirement on the number of $b$-jets. 
Finally, events are required to be fully reconstructable, where jets are matched to all six partons of the \ttbar decay.
After these requirements there are 1,340,000 training events and 71,000 validation events.

An additional 298,000 events are reserved for evaluating the final performance.
These also contain events where not all partons have a jet matched to them.
In 76,000 of these events it is possible to associate a jet to all six partons.
% Requiring all partons to be reconstructable reduces the size of this test dataset to 76,000 events.
After requiring at least 2 $b$-jets in the event, there are 147,000 events of which 44,000 are fully reconstructable.

As inputs for both ML models the four momentum $(p_x, p_y, p_z, E)$ together with a boolean flag showing whether the jet was $b$-tagged or not are used.
The three momenta are normalised subtracting the mean and dividing by the standard deviation. 
The energy is normalised in the same way after applying the logarithm.
No normalisation or transformation is applied to the $b$-tagging flag.
As both models represent the single jets as nodes, additional information per jet can easily be added for both models as additional node features.

\subsection{Topograph implementation}

Before the particle blocks for the two top quarks, two fully connected message-passing graph layers are used to provide an information exchange between the jets in the event and update the jet features. 
% As a first step a graph network with connections between all jets is implemented.
From these updated jets, the $W$ nodes are initialised using attention weighted pooling.
Two different networks are used to obtain two sets of attention weights, one for each of the $W$ nodes.
The top nodes are initialised in the same way from the jets but with the corresponding $W$ node concatenated to the pooled jet information.
% This enforces a fixed connection between the W and top nodes.
The regression targets of the $W$ and top nodes are the truth level properties of the particles.
In both cases the three momentum $(p_x, p_y, p_z)$ is chosen as the regression target.
In our dataset the truth mass is fixed to the Monte Carlo mass values. % and is always a constant value.
The network structure is shown in \cref{fig:topott}.


\begin{figure}[ht]
    \centering
    \includegraphics{figures/topotttallhad-interact.pdf}
    \caption{Topograph network for the $t\bar{t}$ process comprising two $t$ blocks.
    The jets are passed through an information exchange layer, using a fully connected graph message passible layer.
    All jets are connected to all possible mother particles, as shown by the dashed edges.
    The information exchange comprises multiple message passing graph layers and the node updates with the Topograph are performed $N$ times before the edge values are used for parton assignment and the properties for the top quarks and $W$ bosons are extracted.}
    % In \subref{fig:topottW:embed-prior} $\ell$ and $\nu$ are predefined to connect to the $W$ in one top block, whereas in \subref{fig:topottW:embed-noprior} no prior connections are enforced.}
    \label{fig:topott}
\end{figure}

% Including the W and top nodes, the overall graph has now three different node types.
% All jets are connected between each other, all jets are connected to both the W and top nodes, the W nodes only have their fixed connection to one top node.
Within both the information exchange and Topograph blocks, message passing is bidirectional with a separate edge for each direction. Shared weights are used for calculating the attention pooling weights for each category of edge, defined by the sending and receiving particle type.
% All different types of edges, based on the different types of nodes are passed through a separate graph network.
% All pooling operations are based on attention weighted pooling.
Edge features are formed by concatenating the features of the sending node and the receiving node.
Edge features are persistent between updates, and after the first iteration the current values are concatenated to the new features.
% After each iteration, there are additional networks to regress towards the true parton properties and edges between jets and Ws and jets and tops are classified.
% The regression and classification results are taken from the networks after the last iteration.

Four message passing steps in the Topograph update the jets, edges and injected $W$ and top nodes.
After the final message passing step, the properties of the $W$ bosons and top quarks and particle matching scores are extracted.
There are four matching scores for each jet, from which six jets are assigned to the six partons.
The scores are calculated by passing the edge properties of the jet to $W$ and top node through a classification network to determine if it is a true edge.


\subsubsection*{Loss function}

The loss function is calculated from both the edge classification task and the regression tasks.
Edges are classified as true if a jet originated from the $b$-quark, in the case of the top nodes, or one of the two quarks from the decay of the $W$ boson, for the $W$ nodes.
Here, binary cross-entropy is used for the loss function,
and it is weighted using the number of true and false edges across the dataset in order to improve convergence during training.
For each regression task, the mean absolute error (MAE) function is used between the predicted values of the node and the true values of the top quark or $W$ boson.
% Target labels defined by whether a jet originated from the $b$-quark in the decay of the top-quark, 

Since it is not possible for the network to determine which top node corresponds to the top quark or the anti-top quark, a symmetrised version of the losses is calculated in which the loss is calculated for both possible cases.
% The first top node corresponding to the actual top and the second to the anti-top, and the first top node corresponding to the anti-top with the second one corresponding to the top.
The loss is then defined as the minimum of the two.
Since the $W$ boson nodes are directly connected to the top nodes, the loss terms for the $W$ bosons are included in the loss calculation without need for additional symmetrisation.
The full loss is given by
\begin{align*}
    \mathcal{L}_\mathrm{symm} = \min(&\mathcal{L}(t_1, t) + \mathcal{L}(W_1, W^+) \\&+ \mathcal{L}(t_2, \bar{t}) + \mathcal{L}(W_2, W^-), \\
    &\mathcal{L}(t_1, \bar{t}) + \mathcal{L}(W_1, W^-) \\&+ \mathcal{L}(t_2, t) + \mathcal{L}(W_2, W^+)),
\end{align*}
where $\mathcal{L}(p_i, p)$ corresponds to the combined edge classification loss and regression loss for each of the injected particles $p_i \in t_1$, $t_2$, $W_1$ and $W_2$,
with respect to the truth particle $p \in t$, $\bar{t}$, $W^+$, $W^-$.


% where $\mathcal{L}$ corresponds to the combined edge classification loss $\mathcal{L}_C$ and regression loss $\mathcal{L}_R$ for each of the injected particles $p_i \in t_1$, $t_2$, $W_1$ and $W_2$,
% \begin{equation}
%     \mathcal{L}(p_i, p) = \mathcal{L}_C (p_i, p) + \mathcal{L}_R (p_i, p).
% \end{equation}
% $\mathcal{L}_C$ is the edge classification loss for each jet connected to $p_i$ for whether it originates from a quark in the decay of the particle $p$.
% $\mathcal{L}_R$ is the MAE between the predicted properties of the injected particle $p_i$ and the properties of the truth particle $p$.
% Each injected node is compared to the true (anti-)top $t$ ($\bar{t}$) or $W$ boson.

% In the classification task, the target values are whether a jet originates from the corresponding $b$-quark or a quark from the $W$ boson decay,
% in the case of $t_i$ and $W_i$ respectively.
% In the regression tasks, the target values are the momenta of the top quark or $W$ boson.
% Here, $t_i$ and $W_i$ signify the quantity connected to the i-th top node or $W_i$ node respectively.
% For the target values, $t$, $\bar{t}$, and $W$ represent the quantities of the true particles in the regression task, or whether a jet originates from a decay product of this particle for the edge classification.

\subsubsection*{Parton assignment} 
Several options could be tested for assigning jets to the partons for the edge score.
For this initial study a simple iterative approach is chosen.
First, the edge with the highest score is labelled as a true edge, with the jet assigned to this parton.
Next all edges connected to the corresponding jet and parton are removed, and the next highest edge is chosen.
In the all hadronic \ttbar case, there is one parton per top quark, corresponding to the $b$-quarks in the decay, and two per $W$ boson.
This assignment procedure is repeated until all six partons are assigned, and results in a solution where neither a jet or parton can be assigned to multiple targets.

% To avoid assigning multiple jets to one parton or a jet to multiple partons, all edges of this jet and if the parton has enough jets assigned to it - one for top quarks and two for $W$ bosons - are removed.


  
\subsection{Reference methods}

\subsubsection*{$\chi^2$ implementation}
The $\chi^2$ score for \ttbar decays used in this work is given by
\begin{align*}
    \chi^2 = \frac{(m_{b_1q_1q_2} - m_t)^2}{\sigma_t^2} + \frac{(m_{b_2q_3q_4} - m_t)^2}{\sigma_t^2}\\ + \frac{(m_{q_1q_2} - m_W)^2}{\sigma_W^2} + \frac{(m_{q_3q_4} - m_W)^2}{\sigma_W^2},
\end{align*}
where $m_{b_iq_jq_k}$ is the invariant mass of the jets in the permutation, $m_t$ and $m_W$ are the masses of the top quark and W boson, and $\sigma_t$ and $\sigma_W$ are the widths of the top quark and W boson in the dataset.
The values for $m_t$ and $m_W$ are obtained by taking the mean of the reconstructed invariant masses in our dataset, while $\sigma_t$ and $\sigma_W$ are set to the standard deviation.
The calculated values are $m_t=169.8\ \mathrm{GeV}$, $m_W=81.0\ \mathrm{GeV}$, $\sigma_t=29.0\ \mathrm{GeV}$, and $\sigma_W=18.5\ \mathrm{GeV}$.
Jets associated to the $b$-quark of the top decays are required to be $b$-tagged, while no requirement is placed on the jets associated to the $W$ boson decays.

\subsubsection*{\spanet implementation}
% For the comparison with \spanet, models are trained on the same data as the Topograph.
The implementation is taken from the github repository\footnote{\url{https://github.com/Alexanders101/SPANet/tree/v1.0}}.
The hyperparameters of the model are optimised for the dataset using fully matched events, with the values in Ref.~\cite{fenton2021permutationless} using \emph{v1.0} resulting in the highest overall efficiencies.
%  with those in the tag labelled \emph{v1.0} with the hyperparameters for the full training.
% For these comparisons only fully reconstructable events are used in training. Using these events was found in Ref.~\cite{shmakov2021spanet} to increase the performance, however for fairer comparison with the current Topograph implementation this is not performed.
% Results for the \spanet models trained with partial events are presented in \cref{app:partialspa}.
All \spanet models are trained for 100 epochs, taking the weights after the epoch with the lowest loss on the validation set.

\subsection{Partial event trainings}

It is also possible to train Topographs on events in which not all partons from the $t\bar{t}$ decay are matched to jets.
For these events, the same network is used but the edge classification and regression loss terms are not considered for the $W$ boson or top quarks which have partons not able to be matched to jets. % (top quark or $W$ boson). % are not considered.
In the case of the $b$-quark from the top quark decay not being matched to a jet, the $\mathcal{L}(W_i,W^+)$ term is still considered.
Where a quark from the $W$ boson decay is missing, both the top quark and $W$ boson loss terms are not considered.
At least one $W$ boson is required to be fully reconstructable, with both quarks matched to jets in the event.

As introduced in Ref.~\cite{shmakov2021spanet}, \spanet can also be trained on non-fully reconstructable events, however in comparison to Topographs this is only at the level of each top quark.
When a $b$-quark from the top quark is missing, the corresponding $W$ boson is also not considered.
For a fairer comparison with both models trained on the same number of events, we compare models trained only on fully reconstructable events.
Results for the Topograph and \spanet models trained with partial events are presented in \cref{app:partialspa}.

% Topographs and \spanet can both be trained using events in which not all partons from the t\bar{t} decay are matched to jets.
% For these events, the 
  
  \section{Results}
  \begin{table*}[t]
% \begin{small}
\centering
\begin{tabular}{l|c|cc|c|ccc}
% \begin{tabular}{l|>{\columncolor{greenish}}c>{\columncolor{greenish}}c>{\columncolor{redish}}c|c}
\toprule
\textbf{Method} & \textbf{Type} & \textbf{Storage} & \textbf{Memory} & \textbf{Overhead} & \textbf{$<$1B} & \textbf{$<$20B} & \textbf{$>$20B} \\
\midrule
Adapters          \cite{adapters}                & \ad     & \yes & \yes & FFN        & \yes & \yes & \yes \\
AdaMix            \cite{adamix}                  & \ad     & \yes & \yes & FFN        & \yes & \no  & \no \\
SparseAdapter     \cite{sparse_adapter}          & \ad \se & \yes & \yes & FFN        & \yes & \no  & \no \\
Cross-Attn tuning \cite{cross_attention_tuning}  & \se     & \yes & \yes & \gren{no}  & \yes & \no  & \no \\
BitFit            \cite{bitfit}                  & \se     & \yes & \yes & \gren{no}  & \yes & \yes & \no \\
DiffPruning       \cite{diff_pruning}            & \se     & \yes & \no  & \gren{no}  & \yes & \no  & \no \\
Fish-Mask         \cite{fish_mask}               & \se     & \yes & \orange{maybe}\footnotemark[5] & \gren{no}  & \yes & \yes & \no \\
LT-SFT            \cite{lottery_ticket_tuning}   & \se     & \yes & \orange{maybe}\footnotemark[5] & \gren{no} & \yes & \no & \no \\
Prompt Tuning     \cite{prompt_tuning}           & \ad     & \yes & \yes & Seq.       & \yes & \yes & \yes \\
Prefix-Tuning     \cite{prefix_tuning}           & \ad     & \yes & \yes & Seq.       & \yes & \yes & \yes \\
Spot              \cite{spot}                    & \ad     & \yes & \yes & Seq.       & \yes & \yes & \no \\
IPT               \cite{ipt}                     & \ad     & \yes & \yes & FFN, seq.  & \yes & \no  & \no \\
MAM Adapter       \cite{parallel_adapter}        & \ad     & \yes & \yes & FFN, seq.  & \yes & \no  & \no \\
Parallel Adapter  \cite{parallel_adapter}        & \ad     & \yes & \yes & FFN        & \yes & \no  & \no \\
\footnotesize{Intrinsinc SAID} \cite{intrinsic_said} & \rp & \no  & \no  & \gren{no}  & \yes & \yes & \no \\
LoRa              \cite{lora}                    & \rp     & \yes & \yes & \gren{no}  & \yes & \yes & \yes \\
% DyLoRa            \cite{dylora}                  & \rp     & \yes & \yes & \no  & \gren{No}         \\
UniPELT           \cite{unipelt}                 & \ad \rp & \yes & \yes & FFN, seq.  & \yes & \no  & \no \\
\footnotesize{Compacter}   \cite{compacter}      & \ad \rp & \yes & \yes & FFN & \yes & \yes  & \no \\
\footnotesize{PHM Adapter} \cite{compacter}      & \ad \rp & \yes & \yes & FFN        & \yes & \no  & \no \\
KronA             \cite{krona}                   &     \rp & \yes & \yes & \gren{no}  & \yes & \no  & \no \\
KronA$^B_{res}$   \cite{krona}                   & \ad \rp & \yes & \yes & FFN        & \yes & \no  & \no \\
(IA)$^3$          \cite{t_few}                   & \ad     & \yes & \yes & $l_{ff}$      & \no  & \yes  & \no \\
% \footnotesize{WARP \cite{Hambardzumyan2021WARPWA}} & \ad   & \yes & \yes & \no  & Extra input                \\
% IPT               \cite{ipt}                     & \ad     & \yes & \yes & \no  & Extra input                  \\
Attention Fusion  \cite{attention_fusion}        & \ad     & \yes & \yes & Decoder    & \yes & \no  & \no \\
LeTS              \cite{lets}                    & \ad     & \yes & \yes & FFN        & \yes & \no  & \no \\
Ladder Side-Tuning \cite{ladder_side_tuning}     & \ad     & \yes & \yes & Decoder    & \yes & \yes & \no \\
FAR               \cite{far_edge}                & \se     & \yes & \orange{maybe}\footnotemark[6] & \gren{no} & \yes & \no  & \no \\
S4          \cite{design_spaces}                 &\ad \rp \se& \yes & \yes & FFN, seq.& \yes & \yes  & \no \\
\bottomrule
% Meta-Adapters     \cite{meta_adapters}           & A      & \yes & \yes & \no  & Extra FFN                  \\
% Priming           \cite{}                        & A      & \yes & \yes & \no  &                              \\
% Prompt Mapping    \cite{prompt_mapping}          & A      & \yes & \yes & \no  & Extra input                \\
% Polyhistor        \cite{polyhistor}              & A      & \yes & \yes & \no  &                              \\
% PETT              \cite{pett}                    & \rped{?}& \yes & \yes & \no  &                              \\
\end{tabular}
\vspace{2pt}
% \end{small}
\caption{Comparing \peft{} methods across storage, memory, and computational efficiency in terms of reducing backpropagation costs and inference overhead. \textbf{\ad} -- additive, \textbf{\se} -- selective, \textbf{\rp} -- reparametrization-based. Inference overhead: FFN -- extra FFN layer, seq. -- longer attention sequence, decoder -- extra transformer decoder.
% Note that S4-model has smaller inference overhead than other methods that combine Adapters and Prefix-tuning, because it uses differnt combinations of \peft{} methods on different layer (Section \rpef{sec:s4}). FishMask and FAR memory efficiency depend on weight pruning method. FishMask can significantly benefint from sparse operations hardware support (Section \rpef{sec:fishmask}).
% ^* Kronerker layer is a linear layer where weight matrix is constructed as a kroneker product of
}
\label{tab:efficiency_2}
\end{table*}

  \section{Results}
\label{results}

\begin{figure*}[ht]
    \centering
    \includegraphics[scale=0.15,trim={0 2.5cm 0 5cm},clip]{images/aoi-single_burst}
    \caption{The time average peak Age of Information with burst and \gls{soa} loss values against the dynamic reliability logic for different network topologies.}
    \label{fig:aoi_burst}\vspace{-0.4cm}
\end{figure*}


This paper focuses on both transport layer and application layer metrics to determine the feasibility of dynamic reliability. For this, we have selected the session packet volume, as transmitted, retransmitted, lost and backlogged packets as \glspl{kpi} for the transport layer; while focusing on the \gls{aoi} for the application layer. The \gls{aoi} was chosen as a crucial indicator for the freshness of packets in real-time applications. More specifically, this work adopts the time average peak \gls{aoi} equation \cite{aoi_equation} depicted in Eq. \ref{aoi}, where $\Delta(r_{i+1})$ is the $i$th update at the time it was received at the server, for a session time period of $\tau$.

\begin{equation}
    \label{aoi}
    \gls{aoi}_\tau = \frac{1}{n-1}\sum_{i=1}^{n-1} \Delta(r_{i+1})
\end{equation}

We include a comparison between the vanilla QUIC implementation which does not enjoy the dynamic reliability extension, with a number of dynamic reliability policies. The tests were run a number of times for statistical significance, with the mean value of vanilla implementation used as a baseline for comparison. The topology utilised both random loss and bursty loss to explore the bounds of dynamic reliability. The \gls{soa} loss in the figures correspond to the loss values presented in Table. \ref{tab:path_char}, for ease of comparison between bursty and random loss scenarios.

\subsection{Transport-Layer KPIs}

To analyse the performance gain at the transport layer due to dynamic reliability, the volume of transmitted and backlogged packets is examined. The figures are in the form of boxplots, which take the vanilla implementation as a benchmark, depicted as the red dashed line.

As seen in Fig. \ref{fig:sent_burst}, the loss plays a crucial role in the performance of the reliability policies. The policies under random loss did incredibly well for the networks with a larger capacity, namely \gls{mmwave} and Sub-6~GHz, whereas for burst loss, the lower network capacities had a larger packet reduction. With the increase in burst loss, the behaviour of the set split reliable policies became unpredictable, if a reliable assignment happened to coincide with a burst loss, the number of transmitted packets increases, and vice versa. On the other hand, in smarter policies, such as Loss-Aware, the performance lightly matched the vanilla baseline, as the reliable assignment dominated the session to compensate for a higher burst loss. Not only that but, the burst loss also impacted the variance of the transmitted packets for the policies.

Unsurprisingly, the unreliable focused policy, 80-20 split, outperformed other policies for all topologies in random and bursty loss scenarios, with an approximate reduction of 80\%. That being said, the majority of the policies reduced the transmitted packets on the link by approximately 70\% for random loss, while the reduction started at $\approx 15\%$ and decreased as the loss increased for the burst loss scenario.

The retransmitted and lost packets, not shown due to space limitations, followed the same trend as the transmitted packets for the random loss scenarios. However, for the burst loss scenarios, the larger capacity networks had a lower reduction in the retransmitted and lost packets. This can be seen as a favorable outcome since the lower capacity networks are scarce on resources. It is important to note that the Loss-Aware policy mimicked the vanilla approach as the burst loss increased, signifying the overwhelming appointment of reliable packets in adapting to the harsh burst loss conditions.
 
Alternatively, Fig. \ref{fig:backlog_burst} clearly shows a stark comparison between the policies and loss scenario in the reduction of the backlogged packets. The Loss-Aware policy for random loss scenario reduced the backlogged packets by up to 50\%, beating all other policies by approximately 30\%. Furthermore, it is clear that the unreliability focused policies resulted in the lowest backlog for the session. In comparison, we notice that the burst loss and the backlogged frequency have a positive correlation, where the maximum reduction of the backlogged packets for the policies is at most 20\%. Much like the transmitted packets, the probability of a burst loss occurrence plays a vital role in the number of retransmissions sent and by extension the number of backlogged packets. Thus, we can conclude that the stress placed on the buffer is a result of the reliable packets which is tightly coupled with the congestion on the session. Whereas, unreliable focused policies did not encounter such a phenomenon regardless if it was experiencing a burst loss.


\subsection{Application-Layer KPIs}

The feasibility of dynamic reliability for real-time applications can be determined by the \gls{aoi}, with comparison across different topologies and policies. If we take a strict approach and consider anything below $10$~ms is real-time \cite{real-time}, then all the reliability policies passed that requirement, which is attractive for real-time applications, as shown in Fig. \ref{fig:aoi_burst}. Utilising the median as an estimate of the runs, the policies in the WLAN and Sub-6~GHz topology with random loss floated around $4-5$~ms with negligible difference, while the \gls{aoi} for \gls{mmwave} was $\approx 2-3$~ms. It is clear that the \gls{aoi} and the network capacity have a negative correlation, as the network capacity decreases, the \gls{aoi} increases. The same correlation is extended to the bursty loss scenarios, where \gls{mmwave} dominated the other topologies. That being said, it is crucial to note that the \gls{aoi} for the reliability policies is often slightly better than or equal to the \gls{aoi} of the vanilla implementation, proving that dynamic reliability reduces the congestion of the session at no cost to the \gls{aoi}.


  % \FloatBarrier
  % \section{Extensions}
  % \change{The presented performance analysis framework paves the way for the control signaling design and quantification of more sophisticated \gls{ris}-empowered wireless systems. It can be applied, for example, to multi-user wideband/\gls{ofdm} communications~\cite{Huang2019,risTUTORIAL2020}, by accounting for the subcarrier allocation of the different control and payload messages. For this system setup, the Algorithmic phase needs to also consider the resource allocation process, whose output should be signaled to the \glspl{ue} through a specific design of the Setup phase. In addition, the current framework, by omitting, merging, or repeating some of its general phases, can set the basis for the control design in \gls{ris}-assisted networks with a multi-antenna \gls{bs} and multi-antenna \glspl{ue}, and smart wireless environments with multiple, possibly machine learning orchestrated, \glspl{ris}~\cite{RIS_pervasive_ML}, as well as shared \glspl{ris} among multiple communications pairs~\cite{EURASIP_RIS}. Of late interest are also multi-functional \glspl{ris}~\cite{Tsinghua_RIS_Tutorial}, and especially those possessing sensing capabilities~\cite{HRIS_VTM}, which may provide higher flexibility for efficient control signaling designs~\cite{croisfelt2023orchestration}.}

\change{We next elaborate in more detail on the case where the \gls{bs} is equipped with multiple antennas and there exists the possibility of a weak direct link between itself and the \gls{ue}. For the \gls{oce} communication paradigm, the \gls{ris} configuration and the \gls{bs} combiner can be jointly optimized~\cite{risTUTORIAL2020}, at the cost, of course, of a larger \gls{ce} overhead and complexity, as well as larger algorithmic complexity. It is noted, however, that the increased beamforming gain from the multiple \gls{bs} antennas might lead to cases where the \gls{bs}-\gls{ue} link is satisfactory, implying that the \gls{ris} deployment can be avoided, reducing the control overhead. For this mode to be realized, the operation protocol needs to enable, for example, the separate estimation of the \gls{ue}-\gls{ris}-\gls{bs} and \gls{ue}-\gls{bs} channels, via activation/deactivation of the \gls{ris} panel, as well as a relevant action during the Initialization phase.}   
\change{There exist various modes of operation when the \gls{bsw} paradigm is adopted. One is to perform \gls{BSW} at the \gls{bs} during the Initialization phase, together with \gls{BSW} at the \gls{ris}, again at a cost of a larger overhead for both the fixed and flexible frame structures. Alternatively, to reduce the control signaling overhead, the \gls{bs} combiner can be designed to solely match its channel with the \gls{ris}, or that with the \gls{ue} if the \gls{ris} can be avoided, as discussed in the \gls{oce} paradigm. One way to achieve the former is to capitalize on the common assumption that the \gls{ris} is placed such that there exists a strong line-of-sight with the BS~\cite{RIS_security_mal}. In this way, the \gls{bs} may adopt the reception configuration closest to maximal ratio combining. When the \gls{ris} can be avoided, the \gls{bs} combiner can be similarly designed, now for the channel towards the \gls{ue} - this operational mode can be decided similarly to the respective \gls{oce} case.}

\change{It is finally noted that the two presented communication paradigms, namely \gls{oce} and \gls{bsw}, can be combined to devise additional signaling schemes. One example is presented in~\cite{Jamali2022}, where the \gls{bsw} is performed to set the \gls{ris} configuration when the link budget falls below a certain threshold, and then \gls{oce} follows to set the BS combiner and/or the \gls{ue} beamformer, treating the configured \gls{ris} as an unknown scatterer.}




  
  \section{Conclusion}
  \section{Conclusion}\label{sec:conclusion}
In this work, we focus on addressing the fundamental challenge of OOD detection tasks, which is how to fully understand the semantic discrepancy between the ID/OOD samples. We reveal that the key to success in the realistic SCOOD task is to allocate as many ID samples in the unlabeled set correctly as possible. To this end, we propose a novel uncertainty-aware optimal transport scheme that introduces class-specific energy scores as guidance for effective label assignment. Experimental results show that our method achieves better performance than previous state-of-the-art methods on SCOOD benchmarks.

\textbf{Limitations.} In addition to temperature scaling, other techniques such as feature clipping applied in ReAct~\cite{sun2021react} also enhance the performance of energy score, so how to obtain an OOD score that best fits the SCOOD task can be further explored. Moreover, a setting highly related to SCOOD has been proposed in \cite{katz2022training} and formulated as a constrained optimization problem. We will also theoretically analyze these practical OOD settings in our feature work.

% \section*{Acknowledgments}
\textbf{Acknowledgments.} 
This work is supported by National Key R\&D Program of China under Grant 2020AAA0105701, National Natural Science Foundation of China (NSFC) under Grants 61872327, Major Special Science and Technology Project of Anhui, National Natural Science Foundation of China (62033012) and Ant Group through Ant Research Intern Program.



  \section*{Acknowledgements}
  The authors would like to acknowledge funding through the SNSF Sinergia grant called "Robust Deep Density Models for High-Energy Particle Physics and Solar Flare Analysis (RODEM)" with funding number CRSII$5\_193716$, the SNSF project grant 200020\_212127 called "At the two upgrade frontiers: machine learning and the ITk Pixel detector", and the Alexander von Humboldt foundation Feodor Lynen fellowship programme.

  % \phantomsection
  % \addcontentsline{toc}{chapter}{References}
  % \printbibliography[title=References]
  % \bibliographystyle{apsrev4-2} % Tell bibtex which bibliography style to use
  \bibliography{bib/ATLAS.bib, bib/CMS.bib, bib/PubNotes.bib, bib/myrefs.bib}

  % \clearpage
  \appendix
  \section{Appendix}
  \section{Appendix for Proofs}

\paragraph{Proof of Theorem \ref{thm:main}.}

\begin{proof}
\label{proof:main}
Our proof has two steps. In Step 1, we will show that SimCLR is equivalent to minimizing the cross entropy loss defined in Eqn.~(\ref{eqn:cross-entropy}). 
In Step 2, we will show  that minimizing the cross-entropy loss 
is equivalent to spectral clustering on $\bfpi$. 
Combining the two steps together, we have proved our theorem. 

\textbf{Step 1: } SimCLR is equivalent to minimizing the cross entropy loss.

The cross-entropy loss takes expectation over 
$\bfW_\bfX\sim \mathbb{P}(\cdot ; \bfpi)$, 
which means $\bfW_\bfX$ has exactly one non-zero entry in each row $i$. By Lemma~\ref{lem:multinomial}, we know every row $i$ of $\bfW_\bfX$ is independent of other rows. Moreover, 
$\bfW_{\bfX,i}\sim \mathcal{M}(1, \bfpi_i/\sum_j \bfpi_{i,j})=\mathcal{M}(1, \bfpi_i)$, because $\bfpi_i$ itself is a probability distribution.
Similarly, we know $\bfW_\bfZ$ also has the row-independent property by sampling over $\mathbb{P}(\cdot;\bfK_\bfZ)$.
Therefore, by Lemma~\ref{lem:cross_split}, we know Eqn.~(\ref{eqn:cross-entropy}) is equivalent to:
\[
 -\sum_{i=1}^n \mathbb{E}_{\bfW_{\bfX,i}}[\log \mathbb{P}(\bfW_{\bfZ,i}=\bfW_{\bfX,i};\bfK_\bfZ)],
\]

This expression takes expectation over $\bfW_{\bfX,i}$ for the given row $i$. Notice that 
$\bfW_{\bfX,i}$ has exactly one non-zero entry, which equals $1$ (same for $\bfW_{\bfZ,i}$). 
As a result
we expand the above expression to be:
\begin{equation}
 -\sum_{i=1}^n \sum_{j\neq i} \Pr(\bfW_{\bfX,i,j}=1)\log \Pr(\bfW_{\bfZ,i,j}=1).
\label{eqn:detailed-expansion}    
\end{equation}


By Lemma~\ref{lem:multinomial}, $\Pr(\bfW_{\bfZ,i,j}=1)=\bfK_{\bfZ,i,j}/\|\bfK_{\bfZ,i}\|_1$ for $j\neq i$. Recall that $\bfK_\bfZ=(k(\bfZ_i-\bfZ_j))_{(i,j)\in[n]^2}$, which means 
$\bfK_{\bfZ,i,j}/\|\bfK_{\bfZ,i}\|_1=\frac{\exp(-\|\bfZ_i-\bfZ_j\|^2/{2\tau})}{\sum_{k\neq i}
\exp(-\|\bfZ_i-\bfZ_k\|^2/{2\tau})
}$ for $j\neq i$, when $k$ is the Gaussian kernel with variance $\tau$. 

Notice that $\bfZ_i=f(\bfX_i)$, so we know
\begin{equation}
-\log \Pr(\bfW_{\bfZ,i,j}=1)=
-\log \frac{\exp(-\|f(\bfX_i)-f(\bfX_j)\|^2/{2\tau})}{\sum_{k\neq i}
\exp(-\|f(\bfX_i)-f(\bfX_k)\|^2/{2\tau}),
}
\label{eqn:infonce-equivalence}    
\end{equation}


The right hand side is exactly the InfoNCE loss defined in Eqn.~(\ref{eqn:infonce}).
Inserting Eqn.~(\ref{eqn:infonce-equivalence}) into Eqn.~(\ref{eqn:detailed-expansion}), we get the SimCLR algorithm, which first samples augmentation pairs $(i,j)$ with $\Pr(\bfW_{\bfX,i,j}=1)$ for each row $i$, and then optimize the InfoNCE loss. 

\textbf{Step 2: } minimizing the cross entropy loss 
is equivalent to spectral clustering on $\bfpi$.


By Lemma~\ref{lem:convert_to_spectral}, we may further convert the loss to 
\begin{equation}
\label{eqn:main-theorem-repul-attr}
\min_{\bfZ}
-\sum_{(i,j)\in [n]^2} \mathbf{P}_{i,j}
\log k (\bfZ_i-\bfZ_j)+\log \mathbf{R}(\bfZ).
\end{equation}
Since $k$ is the Gaussian kernel, this reduces to \[
\min_\bfZ \mathrm{tr}(\bfZ^\top \mathbf{L}(\bfpi) \bfZ)
+\log \mathbf{R}(\bfZ),
\]

where we use the fact that $\mathbb{E}_{\bfW_\bfX\sim \mathbb{P}(\cdot; \bfpi)}[\mathbf{L}(\bfW_\bfX)]
=\mathbf{L}(\bfpi)
$, because the Laplacian operator is linear and $
\mathbb{E}_{\bfW_\bfX\sim \mathbb{P}(\cdot; \bfpi)}(\bfW_\bfX)=\bfpi
$.
\end{proof}

\paragraph{Proof of Theorem \ref{thm:clip}.}
\begin{proof}
Since $\bfW_\bfX\sim \mathbb{P}(\cdot;\bfpi_{\mathbf{A}, \mathbf{B}})$, we know 
$\bfW_\bfX$ has exactly one non-zero entry in each row, denoting the pair that got sampled. 
A notable difference compared to the previous proof is we now have $n_\mathcal{A}+n_\mathcal{B}$ objects in our graph. CLIP deals with this by taking a mini-batch of size $2N$, 
such that $n_\mathcal{A}=n_\mathcal{B}=N$, and adding the $2N$ InfoNCE losses together. We label the objects in $\mathcal{A}$ as $[n_\mathcal{A}]$, and the objects in $\mathcal{B}$ as $\{n_\mathcal{A}+1, \cdots, n_\mathcal{A}+n_\mathcal{B}\}$. 

Notice that $\bfpi_{\mathbf{A}, \mathbf{B}}$ is a bipartite graph, so the edges of objects in $\mathcal{A}$ will only connect to object in $\mathcal{B}$ and vice versa. We can define the similarity matrix in $\cZ$ as $\bfK_\bfZ$, 
where $\bfK_\bfZ(i, j+n_\mathcal{A})=\bfK_\bfZ(j+n_\mathcal{A},i)= k(\bfZ_i-\bfZ_j)$ for $i\in [n_\mathcal{A}], j\in [n_\mathcal{B}]$, and otherwise we set $\bfK_\bfZ(i,j)=0$. 
The rest is same as the previous proof. 
\end{proof}

\paragraph{Proof of Theorem \ref{thm:exponential}.}

\begin{proof}
\label{proof:exponential}
Since the objective function consists of a linear term combined with an entropy regularization, which is a strongly concave function, the maximization problem is a convex optimization problem. Owing to the implicit constraints provided by the entropy function, the problem is equivalent to having only the equality constraint. We then introduce the Lagrangian multiplier $\lambda$ and obtain the following relaxed problem:

$$
\widetilde{E}(\boldsymbol{\alpha})=\psi_{1}-\sum_{i=1}^n \alpha_{i} \psi_{i}+\tau \sum_{i=1}^n \alpha_{i}\log \alpha_{i}+\lambda\left(\boldsymbol{\alpha}^{\top} \mathbf{1}_n-1\right).
$$

As the relaxed problem is unconstrained, taking the derivative with respect to $\alpha_{i}$ yields

$$
\frac{\partial \widetilde{E}(\boldsymbol{\alpha})}{\partial \alpha_{i}}=-\psi_{i}+\tau\left(\log \alpha_{i}+\alpha_{i} \frac{1}{\alpha_{i}}\right)+\lambda=0.
$$

Solving the above equation implies that $\alpha_{i}$ takes the form
$
\alpha_{i}=\exp \left(\frac{1}{\tau} \psi_{i}\right) \exp \left(\frac{-\lambda}{\tau}-1\right).
$ Since $\alpha_{i}$ lies on the probability simplex, the optimal $\alpha_{i}$ is explicitly given by
$
\alpha^{*}_{i}=\frac{\exp \left(\frac{1}{\tau} \psi_{i}\right)}{\sum_{i^{\prime}=1}^n \exp \left(\frac{1}{\tau} \psi_{i^{\prime}}\right)} .
$ Substituting the optimal point into the objective function, we obtain
$$
\begin{aligned}
E\left(\boldsymbol{\alpha}^*\right)  &=\psi_1-\sum_{i=1}^n \frac{\exp \left(\frac{1}{\tau} \psi_{i}\right)}{\sum_{i^{\prime}=1}^n \exp \left(\frac{1}{\tau} \psi_{i^{\prime}}\right)} \psi_{i}+\tau \sum_{i=1}^n \frac{\exp \left(\frac{1}{\tau} \psi_{i}\right)}{\sum_{i^{\prime}=1}^n \exp \left(\frac{1}{\tau} \psi_{i^{\prime}}\right)}\log \frac{\exp \left(\frac{1}{\tau} \psi_{i}\right)}{\sum_{i^{\prime}=1}^n \exp \left(\frac{1}{\tau} \psi_{i^{\prime}}\right)} \\
& =\psi_1 - \tau \log \left(\sum_{i=1}^n \exp \left(\frac{1}{\tau} \psi_{i}\right)\right).
\end{aligned}
$$
Thus, the Lagrangian dual function is given by
\begin{equation*}
-E\left(\boldsymbol{\alpha}^*\right)= -\tau \log \frac{\exp \left(\frac{1}{\tau} \psi_{1}\right)}{\sum_{i=1}^n \exp \left(\frac{1}{\tau} \psi_{i}\right)}.\qedhere
\end{equation*}
\end{proof}



\section{More on Experiments} \label{section: experiment_details}

\paragraph{CIFAR-10 and CIFAR-100} CIFAR-10 ~\citep{krizhevsky2009learning} and CIFAR-100 ~\citep{krizhevsky2009learning} are well-known classic image classification datasets. Both CIFAR-10 and CIFAR-100 contain a total of 60k $32 \times 32$ labeled images of different classes, with 50k for training and 10k for testing. CIFAR-10 is similar to CIFAR-100, except there are 10 different classes in CIFAR-10 and 100 classes in CIFAR-100.

\paragraph{TinyImageNet} TinyImageNet ~\citep{le2015tiny} is a subset of ImageNet ~\citep{deng2009imagenet}. There are 200 different object classes in TinyImageNet, with 500 training images, 50 validation images, and 50 test images for each class. All the images in TinyImageNet are colored and labeled with a size of $64 \times 64$.

\textbf{Pseudo-code.} Algorithm \ref{alg:Training Procedure} presents the pseudo-code for our empirical training procedure.

\begin{algorithm}[!htbp]
\caption{Training Procedure}
\label{alg:Training Procedure}
\begin{algorithmic}[1]
\REQUIRE trainable encoder network $f$, batch size $N$, augmentation strategy \textit{aug}, loss function $L$ with hyperparameters \textit{args}
\FOR {sampled minibatch ${x_i}_{i=1}^N$}
\FORALL{$i \in { 1, ..., N }$}
\STATE draw two augmentations $t_i = \textit{aug}\left(x_i\right) $, $t_i' = \textit{aug}\left(x_i\right) $
\STATE $z_i = f\left(t_i\right)$, $z_i' = f\left(t_i'\right)$
\ENDFOR
\STATE compute loss $\mathcal{L} = L(N, z, z', \textit{args})$
\STATE update encoder network $f$ to minimize $\mathcal{L}$
\ENDFOR
\STATE \textbf{Return} encoder network $f$
\end{algorithmic}
\end{algorithm}

We also provide the pseudo-code for our core loss function used in the training procedure in Algorithm \ref{alg:Core loss}. The pseudo-code is almost identical to SimCLR's loss function, with the exception of an extra parameter $\gamma$.

\begin{algorithm}[!htbp]
\caption{Core loss function $\mathcal{C}$}
\label{alg:Core loss}
\begin{algorithmic}[1]
\REQUIRE batch size $N$, two encoded minibatches $z_1, z_2$, $\gamma$, temperature $\tau$
\STATE $z = \textit{concat}\left(z_1, z_2\right)$
\FOR {$i \in {1, ..., 2N }, j \in {1, ..., 2N}$ }
\STATE $s_{i,j} = \Vert z_i - z_j \Vert_2^{\gamma}$
\ENDFOR
\STATE \textbf{define} $l(i, j)$ \textbf{as} $l(i, j) = - \log \frac{exp\left(s_{i,j}/\tau \right)}{\sum_{k=1}^{2N} \mathbf{1}{[k \ne i]} exp\left(s{i, j} / \tau \right)} $
\STATE \textbf{Return} $\frac{1}{2N} \sum_{k=1}^N\left[l(i, i+N) + l(i+N, i)\right]$
\end{algorithmic}
\end{algorithm}

Utilizing the core loss function $\mathcal{C}$, we can define all kernel loss functions used in our experiments in Table \ref{table: loss definition}. For all $z_i \in z$ with even dimensions $n$, we define $z_{L_i} = z_i\left[0:n/2\right]$ and $z_{R_i} = z_i\left[n/2:n\right]$.

\begin{table}[ht]
\centering
\begin{tabular}{{@{}l|l@{}}}
Kernel  &  Loss function \\ \midrule
Laplacian & $\mathcal{C}\left(N, z, z', \gamma=1, \tau\right)$\\ \midrule
Sum       & $\lambda * \mathcal{C}\left(N, z, z', \gamma=1, \tau_1\right) + (1-\lambda) * \mathcal{C}\left(N, z, z', \gamma=2, \tau_2\right)$  \\ \midrule
Concatenation Sum&$\lambda * \mathcal{C}\left(N, z_L, z'_L, \gamma=1, \tau_1\right) + (1-\lambda) * \mathcal{C}\left(N, z_R, z'_R, \gamma=2, \tau_2\right)$\\ \midrule
$\gamma = 0.5$ & $\mathcal{C}\left(N, z, z', \gamma=0.5, \tau\right)$          \\ 

\end{tabular}

\caption{Definition of kernel loss functions in our experiments}
\label {table: loss definition}
\end{table}

\textbf{Baselines.} We reproduce the SimCLR algorithm using PyTorch Lightning~\citep{PytorchLightning}.

\textbf{Encoder details.}
The encoder $f$ consists of a backbone network and a projection network. We employ ResNet50~\citep{ResNet} as the backbone and a 2-layer MLP (connected by a batch normalization~\citep{ioffe2015batch} layer and a ReLU \cite{nair2010rectified} layer) with hidden dimensions 2048 and output dimensions 128 (or 256 in the concatenation kernel case).

\textbf{Encoder hyperparameter tuning.}
For each encoder training case, we randomly sample 500 hyperparameter groups (sample details are shown in Table \ref{table: Hyperparameter sample}) and train these samples simultaneously using Ray Tune ~\citep{RayTune}, with the ASHA scheduler~\citep{li2018massively}. Ultimately, the hyperparameter group that maximizes the online validation accuracy (integrated in PyTorch Lightning) within 5000 validation steps is chosen for the given encoder training case.

\begin{table}[ht]
\centering

\begin{tabular}{@{}l|l|l@{}}
\midrule
Hyperparameter  & Sample Range & Sample Strategy \\ \midrule
start learning rate & $\left[10^{-2}, 10\right]$ & log uniform \\ \midrule
$\lambda$       & $\left[0, 1\right]$ & uniform \\ \midrule
$\tau$, $\tau_1$, $\tau_2$ & $\left[0, 1\right]$ & log uniform \\ \midrule
\end{tabular}

\caption{Hyperparameters sample strategy}
\label {table: Hyperparameter sample}
\end{table}

\textbf{Encoder training.} 
We train each encoder using the LARS optimizer~\citep{LARSOptimizer}, LambdaLR Scheduler in PyTorch, momentum 0.9, weight decay $10^{-6}$, batch size 256, and the aforementioned hyperparameters for 400 epochs on a single A-100 GPU.

\textbf{Image transformation.} The image transformation strategy, including augmentation, is identical to the default transformation strategy provided by PyTorch Lightning.

\textbf{Linear evaluation.}
The linear head is trained using the SGD optimizer with a cosine learning rate scheduler, batch size 64, and weight decay $10^{-6}$ for 100 epochs. The learning rate starts at $0.3$ and ends at $0$.

\textbf{Moco Experiments.} We also tested our method based on MoCo~\citep{he2019moco}. The results are summarized in Table \ref{tab:results-moco}. Here we choose ResNet18~\citep{ResNet} as the backbone and set a temperature of $0.1$ as default. For our simple sum kernel, we set $\lambda=0.8$. The results show that our method outperforms the original MoCo method.

\begin{table}[thb]
\centering
\caption{MoCo Experiment Results on CIFAR-10 and CIFAR-100.}
\label{tab:results-moco}
\resizebox{\textwidth}{!}{%
\begin{tabular}{@{}c|ccc|ccc@{}}
\toprule
\multirow{3}{*}{Method} & \multicolumn{3}{c|}{CIFAR-10} & \multicolumn{3}{c}{CIFAR-100} \\ \cmidrule(lr){2-4} \cmidrule(lr){5-7} 
                        & 200 epochs & 400 epochs    & 1000 epochs   & 200 epochs & 400 epochs & 1000 epochs         \\ \midrule
MoCo (repro.)         & $76.41 \pm 0.12$    & $80.01 \pm 0.15$          & $84.45 \pm 0.08$    & $\mathbf{47.02 \pm 0.11}$ & $52.50 \pm 0.07$ & $57.62 \pm 0.15$            \\
\midrule
Laplacian Kernel        & ${78.09 \pm 0.10}$    & $\mathbf{83.85 \pm 0.09}$          & $\mathbf{88.34 \pm 0.16}$    & $46.12 \pm 0.22$   & $53.44 \pm 0.17$ & $59.10 \pm 0.14$        \\
Simple Sum Kernel & $\mathbf{78.12 \pm 0.15}$   & $83.23 \pm 0.18$ & $87.50 \pm 0.20$ & $46.65 \pm 0.06$ & $\mathbf{53.62 \pm 0.19}$ & $\mathbf{59.83 \pm 0.12}$\\
\bottomrule
\end{tabular}
}
\end{table}



\section{More Experiments on Synthetic Data}


Consider a scenario with $n$ clusters, each containing $k$ vertices. Let the probability of vertices $u$ and $v$ from the same cluster belonging to $\bfpi$ be $p$. Conversely, for vertices $u$ and $v$ from different clusters, let the probability of belonging to $\pi$ be $q$. We generate the graph $\bfpi$ randomly, based on $p$ and $q$. We experiment with values of $k=100$ and $n=6$ for ease of visualization, embedding all points in a two-dimensional space. Each vertex's initial position originates from a normal distribution. In each iteration, we sample a subgraph of $\bfpi$ uniformly, ensuring each vertex has an out-degree of $1$. We then optimize the corresponding vectors using InfoNCE loss with an SGD optimizer and iterate until convergence. Our experimental setup consists of an SGD learning rate of $1$, an InfoNCE loss temperature of $0.5$, and a batch size of $50$. We evaluate two scenarios with different $p$ and $q$ values: $p=1$, $q=0$, and $p=0.75$, $q=0.2$. The results of these experiments are visualized in Figure \ref{fig:vis-spectral-cluster}. The obtained embeddings exhibit the hallmark pattern of spectral clustering of graph $\bfpi$.

\begin{figure}[!tb]
\centering
\subfigure{
\includegraphics[width=1\textwidth]{Figures/cluster_pi.png}
\label{fig:vis-cluster}
}
\subfigure{
\includegraphics[width=1\textwidth]{Figures/noised_cluster_pi.png}
\label{fig:vis-noised-cluster}
}
\caption{Visualizations of the optimization process using InfoNCE Loss on the vectors corresponding to $\bfpi$. Points of identical color belong to the same cluster within $\bfpi$. To showcase the internal structure of $\bfpi$, we randomly select 10 vertices from each cluster to display the edge distribution of $\bfpi$.}
\label{fig:vis-spectral-cluster}
\end{figure}



\end{document}
