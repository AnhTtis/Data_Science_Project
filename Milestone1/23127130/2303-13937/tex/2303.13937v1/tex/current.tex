In top quark physics kinematic event reconstruction forms a key part of many measurements. The $\chi^2$ method~\cite{TOPQ-2015-03} and the Kinematic Likelihood fitter (KLFitter)~\cite{Erdmann_2014} have been employed in a large number of analyses~\cite{TOPQ-2012-08, CMS-TOP-14-022, ATLAS:2015pfy,  TOPQ-2015-03, TOPQ-2016-01,  CMS:2016oae, CMS-TOP-17-008, CMS:2018htd,  CMS:2018quc, ATLAS:2018fwq, ATLAS:2019guf, ATLAS:2019hxz, CMS:2021vhb, ATLAS:2022waa}.
In both approaches, all combinatorics of jet matching to final state quarks and gluons (partons) in the $t\bar{t}$ final state are tested with kinematic constraints based on the masses of the reconstructed $W$ bosons and top quarks as minimisation criteria.
In the case of KLFitter these are used in conjuction with  with transfer functions and the particle decay widths. %With these approaches, reasonably good performance can be achieved. However, there are two main drawbacks to this approach.

Although good performance can be achieved with such an approach, as the number of jets in an event increases, as well as the multiplicity of final state objects to be reconstructed, the number of combinations increases exponentially.
For example, in an all hadronic top quark pair event with 6 jets, there are 720 potential combinations. This is reduced to 90 by exploiting underlying symmetries. However for events with 7 jets the combinations increase to 630, and for 8 jets they increase further to 2520. %This level of computational cost is unmanageable.
Furthermore, as the exact values of the mass of the top quark and $W$ boson are used to test the likelihood of a combination, this leads to a biased estimator which focusses on assigning jets which together are closest to the hypothesised particles mass, rather than exploiting all the information about the pairs or triplets of objects.
It also assumes that in all cases the top quarks and $W$ bosons are on-shell. %As the multiplicities of objects in the final state increase, the likelier it becomes that an incorrect pair or triplet will have an invariant mass closer to the target mass than the correct assignment.

Building on the previous combinatoric approaches, simple approaches using machine learning (ML) have been developed. Instead of finding the most probable assignment using just the masses of intermediary particles, machine learning discriminants are used to identify correct assignments, exploiting more information from the event~\cite{HIGG-2017-03}.
Nonetheless, these approaches still suffer from the same problems as the KLFitter and $\chi^2$ methods, with each combination needing to be tested to identify the most likely.% combination.

Another approach which uses more information from the event is the Matrix Element Method~(MEM)~\cite{Fiedler_2010,CMS-SMP-13-004,TOPQ-2015-01,HIGG-2017-03}.
The MEM not only attempts to match objects to the final state objects in an event, but directly assesses the likelihood of observing an event given the matrix element for a process. This can be evaluated for each potential combination with the highest resulting probability chosen as the correct assignment.
However, it is extremely slow and computationally intensive.
To calculate the likelihood of an event, an integral over the whole phase space of possible final state particle momenta must be performed.
It is also reliant on a transfer function, which is used to convert the jets, charged leptons and missing transverse momentum recorded by the detector to the partons, charged leptons and neutrinos before any hadronisation and detector effects. As there is no accurate function to model this, it is at best an approximation optimised by hand.
Normalising flows present a solution to the computational challenge and approximate functions~\cite{Butter:2022vkj}, however do not yet address the combinatoric solving.

% Graph Neural Networks~(GNNs)~\cite{battaglia2018relational,wu2020comprehensive} and attention transformers~\cite{vaswani2017attention} are .
% % In GNNs, operations are applied to a permutation invariant graph which comprises nodes, which have attributes, and edges, which connect the nodes; edges, too, can have attributes. %These machine learning models are now being used more frequently in high energy physics applications~\cite{Qu:2019gqs,Moreno:2019bmu,shlomi2020graph,Bernreuther:2020vhm,Ju:2020tbo,Guo:2020vvt,Dreyer:2020brq,Hariri:2021clz,DeZoort:2021rbj,Atkinson:2021nlt,Thais:2022iok,Gong:2022lye,CMS-DP-2020-002,ATL-PHYS-PUB-2022-027}.
% GNN based approaches for combinatoric solving in secondary use edge classification techniques to identify nodes coming from a common parent~\cite{Shlomi_2021}. % commonly used in the field of machine learning. %The natural graph structure in \cref{fig:topgraphscomp:ttbargraph} is not exploited, and instead a fully connected graph is constructed from all objects recorded by the detector.
% % From the fully connected graph, where all nodes are connected via edges, correct edges are defined as those which connect objects originating from the same origin of interest. In the case of top quark pair production, this is all objects which originate from the same top quark.
% % This necessitates $N(N-1)$ edges in the graph, where $N$ is the number of objects. As a result, the complexity grows quadratically, which is more manageable than the combinatoric approaches. However, this does not exploit all information available from the prior knowledge of the physics process.
% Typically, fully connected graphs are used, where all objects are connected to all other objects via edges, with message passing layers used to update the edge features and nodes within the graph. %, resulting in $N\left(N-1\right)$ edges.
% From the fully connected graph, correct edges are defined as those which connect objects originating from the same origin of interest. In the case of top quark pair production, this represents all objects which originate from the same top quark.
% This necessitates $N(N-1)$ edges in the graph, where $N$ is the number of objects. As a result, the complexity grows quadratically, which is more manageable than the combinatoric approaches. However, this does not exploit all information available from the prior knowledge of the physics process.

\subsection*{State of the art}

The state of the art machine learning approach uses attention transformers~\cite{SAJANet,fenton2021permutationless,shmakov2021spanet} to identify the indices of final state objects coming from intermediate particles. In this approach no graph structure is used and only the permutation invariant collection of objects are considered. %The goal of these approaches is to identify the most probable combinations of objects using tensors without having to cycle over all possible combinatorics. 
The complexity of the approach can be reduced by taking into account the symmetries, as performed in Refs.~\cite{fenton2021permutationless,shmakov2021spanet} (SPA-Net), corresponding to removing potential solutions in the combinatoric approaches, which leads to an overall complexity of $\mathcal{O}\left(N^2\right)$. %With a clever design these approaches also have a complexity of $\mathcal{O}\left(N^2\right)$.

Graph Neural Networks~\cite{battaglia2018relational,wu2020comprehensive} are also employed in HEP to associate objects to a common origin, for example in secondary vertex reconstruction~\cite{Shlomi_2021} and could similarly be applied to combinatoric solving at the event level. These approaches have fully connected graphs with $N(N-1)$ edges.

In addition to their reduced computational complexity in comparison to traditional approaches, both attention and GNN approaches also demonstrate reduction in biases towards particle masses, as often seen in the combinatoric approaches.
However, in both GNNs and \spanet the target is to identify the two triplets of objects which correspond to the decay of each top quark, neglecting the structure of the decay, and the properties of the intermediary particles.
% As a consequence, the approaches need to be redesigned for each application, and blocks are not reused within the same model even where there are multiple objects of the same type.
%The reduction in bias from the method in Ref.~\cite{fenton2021permutationless} can be seen in Fig.~\ref{fig:spatter_mass} for the case of reconstructing the $W$ boson in all-hadronic top quark pair production events. %The $\chi^2$ method selects $W$ candidates with a much narrower spread around the true $W$ boson mass, but with a higher rate of incorrect matches than in the attention based approach.
% \begin{figure}[h]
%     \centering
%     \begin{subfigure}{0.49\textwidth}
%         \includegraphics[width=\textwidth]{figures/spatter_w_chi2}
%         \caption{\ }
%         \label{fig:spattw:chi2}
%         % \phantomcaption
%     \end{subfigure}
%     \begin{subfigure}{0.49\textwidth}
%         \includegraphics[width=\textwidth]{figures/spatter_w}
%         % \phantomcaption
%         \caption{\ }
%         \label{fig:spattw:spa}
%     \end{subfigure}
%     \caption{The mass of the candidate $W$ bosons calculated from the invariant mass of the two jets associated to the $W$ boson using the \subref{fig:spattw:chi2} $\chi^2$ approach and \subref{fig:spattw:spa} the attention transformer in Ref.~\cite{fenton2021permutationless}.
%     The correct assignments are represented by the red component with incorrect solutions in blue. A higher accuracy is present in \subref{fig:spattw:spa} alongside a much reduced sculpting of the mass distribution towards the $W$ peak at 80.4~GeV.}
%     \label{fig:spatter_mass}
% \end{figure}
% Nevertheless, just as is the case for current graph applications, in the attention approaches presented in Ref.~\cite{fenton2021permutationless}, the target is to identify the two triplets of objects originating from the same top quark, neglecting the structure of the decay.
% As a consequence, the approaches need to be redesigned for each application, and blocks are not reused within the same model even where there are multiple objects of the same type. %Furthermore, the same trained network cannot be easily transferred to another application using transfer learning, which could reduce the computing resources required for each new setting.
% Efforts to improve these approaches and provide a new toolkit using modern machine learning are not currently undertaken by Swiss institutes in the ATLAS or CMS collaborations, though any solutions and improvements will be relevant to the physics programmes in all groups.