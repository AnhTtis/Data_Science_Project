% For an initial application and for direct comparison with other state-of-the-art methods we apply Topographs to top quark pair production with both tops decaying hadronically. 
For an initial application and for direct comparison with other state-of-the-art methods we apply Topographs to top quark pair production with both tops decaying hadronically. 
We compare the performance of our method to a benchmark non-ML approach used in many top quark analyses, the $\chi^2$ method, and the state-of-the-art ML approach, \spanet~\cite{shmakov2021spanet}.
All models are trained and evaluated on the same dataset.

20 million $t\bar{t}$ events with a centre of mass energy $\sqrt{s}=13$~TeV are simulated using MadGraph5\_aMC@NLO~\cite{MadGraph}~(v3.1.0), with decays of top quarks and $W$~bosons modelled with MadSpin \cite{MadSpin}, with both $W$~bosons decaying to two quarks.%
\footnote{Dataset available at \url{https://zenodo.org/record/7737248} \cite{zoch_knut_2023_7737248}}
The parton shower and hadronisation is performed with Pythia~\cite{Pythia} (v8.243).
The detector response is simulated using Delphes~\cite{Delphes} (v3.4.2) with a parametrisation similar to the response of the ATLAS detector.
Jet clustering is performed using the anti-$k_{t}$ algorithm~\cite{AntiKt} with a radius parameter $R=0.4$ using the FastJet~\cite{FastJet} package.
Jets originating from $b$-quarks ($b$-jets) are identified with a simple binary discriminant corresponding to an inclusive 70\% signal efficiency.

For training, events with at least six jets are selected, keeping up to 16 jets per event as ordered by their transverse momentum.
Truth matching of jets to the partons in the hard scatter is performed using a cone of $\Delta R < 0.4$.
Events with partons matched to multiple jets, or jets matched to multiple partons are discarded.
Events are further required to have zero reconstructed leptons, though no requirement on the number of $b$-jets. 
Finally, events are required to be fully reconstructable, where jets are matched to all six partons of the \ttbar decay.
After these requirements there are 1,340,000 training events and 71,000 validation events.

An additional 298,000 events are reserved for evaluating the final performance.
These also contain events where not all partons have a jet matched to them.
In 76,000 of these events it is possible to associate a jet to all six partons.
% Requiring all partons to be reconstructable reduces the size of this test dataset to 76,000 events.
After requiring at least 2 $b$-jets in the event, there are 147,000 events of which 44,000 are fully reconstructable.

As inputs for both ML models the four momentum $(p_x, p_y, p_z, E)$ together with a boolean flag showing whether the jet was $b$-tagged or not are used.
The three momenta are normalised subtracting the mean and dividing by the standard deviation. 
The energy is normalised in the same way after applying the logarithm.
No normalisation or transformation is applied to the $b$-tagging flag.
As both models represent the single jets as nodes, additional information per jet can easily be added for both models as additional node features.

\subsection{Topograph implementation}

Before the particle blocks for the two top quarks, two fully connected message-passing graph layers are used to provide an information exchange between the jets in the event and update the jet features. 
% As a first step a graph network with connections between all jets is implemented.
From these updated jets, the $W$ nodes are initialised using attention weighted pooling.
Two different networks are used to obtain two sets of attention weights, one for each of the $W$ nodes.
The top nodes are initialised in the same way from the jets but with the corresponding $W$ node concatenated to the pooled jet information.
% This enforces a fixed connection between the W and top nodes.
The regression targets of the $W$ and top nodes are the truth level properties of the particles.
In both cases the three momentum $(p_x, p_y, p_z)$ is chosen as the regression target.
In our dataset the truth mass is fixed to the Monte Carlo mass values. % and is always a constant value.
The network structure is shown in \cref{fig:topott}.


\begin{figure}[ht]
    \centering
    \includegraphics{figures/topotttallhad-interact.pdf}
    \caption{Topograph network for the $t\bar{t}$ process comprising two $t$ blocks.
    The jets are passed through an information exchange layer, using a fully connected graph message passible layer.
    All jets are connected to all possible mother particles, as shown by the dashed edges.
    The information exchange comprises multiple message passing graph layers and the node updates with the Topograph are performed $N$ times before the edge values are used for parton assignment and the properties for the top quarks and $W$ bosons are extracted.}
    % In \subref{fig:topottW:embed-prior} $\ell$ and $\nu$ are predefined to connect to the $W$ in one top block, whereas in \subref{fig:topottW:embed-noprior} no prior connections are enforced.}
    \label{fig:topott}
\end{figure}

% Including the W and top nodes, the overall graph has now three different node types.
% All jets are connected between each other, all jets are connected to both the W and top nodes, the W nodes only have their fixed connection to one top node.
Within both the information exchange and Topograph blocks, message passing is bidirectional with a separate edge for each direction. Shared weights are used for calculating the attention pooling weights for each category of edge, defined by the sending and receiving particle type.
% All different types of edges, based on the different types of nodes are passed through a separate graph network.
% All pooling operations are based on attention weighted pooling.
Edge features are formed by concatenating the features of the sending node and the receiving node.
Edge features are persistent between updates, and after the first iteration the current values are concatenated to the new features.
% After each iteration, there are additional networks to regress towards the true parton properties and edges between jets and Ws and jets and tops are classified.
% The regression and classification results are taken from the networks after the last iteration.

Four message passing steps in the Topograph update the jets, edges and injected $W$ and top nodes.
After the final message passing step, the properties of the $W$ bosons and top quarks and particle matching scores are extracted.
There are four matching scores for each jet, from which six jets are assigned to the six partons.
The scores are calculated by passing the edge properties of the jet to $W$ and top node through a classification network to determine if it is a true edge.


\subsubsection*{Loss function}

The loss function is calculated from both the edge classification task and the regression tasks.
Edges are classified as true if a jet originated from the $b$-quark, in the case of the top nodes, or one of the two quarks from the decay of the $W$ boson, for the $W$ nodes.
Here, binary cross-entropy is used for the loss function,
and it is weighted using the number of true and false edges across the dataset in order to improve convergence during training.
For each regression task, the mean absolute error (MAE) function is used between the predicted values of the node and the true values of the top quark or $W$ boson.
% Target labels defined by whether a jet originated from the $b$-quark in the decay of the top-quark, 

Since it is not possible for the network to determine which top node corresponds to the top quark or the anti-top quark, a symmetrised version of the losses is calculated in which the loss is calculated for both possible cases.
% The first top node corresponding to the actual top and the second to the anti-top, and the first top node corresponding to the anti-top with the second one corresponding to the top.
The loss is then defined as the minimum of the two.
Since the $W$ boson nodes are directly connected to the top nodes, the loss terms for the $W$ bosons are included in the loss calculation without need for additional symmetrisation.
The full loss is given by
\begin{align*}
    \mathcal{L}_\mathrm{symm} = \min(&\mathcal{L}(t_1, t) + \mathcal{L}(W_1, W^+) \\&+ \mathcal{L}(t_2, \bar{t}) + \mathcal{L}(W_2, W^-), \\
    &\mathcal{L}(t_1, \bar{t}) + \mathcal{L}(W_1, W^-) \\&+ \mathcal{L}(t_2, t) + \mathcal{L}(W_2, W^+)),
\end{align*}
where $\mathcal{L}(p_i, p)$ corresponds to the combined edge classification loss and regression loss for each of the injected particles $p_i \in t_1$, $t_2$, $W_1$ and $W_2$,
with respect to the truth particle $p \in t$, $\bar{t}$, $W^+$, $W^-$.


% where $\mathcal{L}$ corresponds to the combined edge classification loss $\mathcal{L}_C$ and regression loss $\mathcal{L}_R$ for each of the injected particles $p_i \in t_1$, $t_2$, $W_1$ and $W_2$,
% \begin{equation}
%     \mathcal{L}(p_i, p) = \mathcal{L}_C (p_i, p) + \mathcal{L}_R (p_i, p).
% \end{equation}
% $\mathcal{L}_C$ is the edge classification loss for each jet connected to $p_i$ for whether it originates from a quark in the decay of the particle $p$.
% $\mathcal{L}_R$ is the MAE between the predicted properties of the injected particle $p_i$ and the properties of the truth particle $p$.
% Each injected node is compared to the true (anti-)top $t$ ($\bar{t}$) or $W$ boson.

% In the classification task, the target values are whether a jet originates from the corresponding $b$-quark or a quark from the $W$ boson decay,
% in the case of $t_i$ and $W_i$ respectively.
% In the regression tasks, the target values are the momenta of the top quark or $W$ boson.
% Here, $t_i$ and $W_i$ signify the quantity connected to the i-th top node or $W_i$ node respectively.
% For the target values, $t$, $\bar{t}$, and $W$ represent the quantities of the true particles in the regression task, or whether a jet originates from a decay product of this particle for the edge classification.

\subsubsection*{Parton assignment} 
Several options could be tested for assigning jets to the partons for the edge score.
For this initial study a simple iterative approach is chosen.
First, the edge with the highest score is labelled as a true edge, with the jet assigned to this parton.
Next all edges connected to the corresponding jet and parton are removed, and the next highest edge is chosen.
In the all hadronic \ttbar case, there is one parton per top quark, corresponding to the $b$-quarks in the decay, and two per $W$ boson.
This assignment procedure is repeated until all six partons are assigned, and results in a solution where neither a jet or parton can be assigned to multiple targets.

% To avoid assigning multiple jets to one parton or a jet to multiple partons, all edges of this jet and if the parton has enough jets assigned to it - one for top quarks and two for $W$ bosons - are removed.

