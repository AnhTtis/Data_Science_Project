To obtain accurate and scene-plausible human motion $\bm{M^W}$ in the world coordinate, we use scene geometry \bm{${S}$} with several physics terms to perform joint optimizations. The following terms are used: 
the smoothness term $\mathcal{L}_{smt}$,
the scene-aware contact term $\mathcal{L}_{sc}$.
the pose prior term $\mathcal{L}_{pri}$, 
and the mesh to point term $\mathcal{L}_{m2p}$. 
The optimization is to find the $\theta$ and $T$ to minimize $\mathcal{L}$:
\begin{equation}
	\begin{split}   
    & \bm{M} = 
    \arg \min _{\bm{\theta, T}}\mathcal{L} (\theta, T, \bm{S}),\\
    & \mathcal{L}_{} = 
        {\mathcal{L}}_{smt} + 
        \lambda_{sc}\mathcal{L}_{sc} + 
        {\lambda_{pri}\mathcal{L}}_{pri}+
        {\lambda_{m2p}\mathcal{L}}_{m2p}, \\
    & \mathcal{L}_{smt} = 
        \lambda_{trans} \mathcal{L}_{trans} + 
        \lambda_{orit} \mathcal{L}_{orit} + 
        \lambda_{jts} \mathcal{L}_{jts}.
    \end{split}
\end{equation}
\noindent
where $\lambda_{sc}$, $\lambda_{pri}$, $\lambda_{trans}$, $\lambda_{orit}$, $\lambda_{jts}$, and $\lambda_{m2p}$ are  loss terms' coefficients. $\mathcal{L}$ is minimized with a gradient descent algorithm.























\PAR{smoothness term.} This term includes the translation loss $\mathcal{L}_{trans}$, the orientation loss $\mathcal{L}_{orit}$, and the joints loss $\mathcal{L}_{jts}$. The position of the pelvis joint is used to define the human's translation, while the other joints subtract it to define the root-relative joints. The objective of all sub-terms is to minimize acceleration to smooth human movements. 


























\PAR{Scene-aware contact term.} we compare the movement of every foot vertices in IMU motions $\bm{{M}_k^I}$ and label the foot as stable if its velocity is less than 0.1 $m/s^2$. Finally, the Chamfer Distance (CD) between this foot and its closest surface is expressed as the scene contact loss $\mathcal{L}_{sc}$.

\PAR{Pose prior term.} The poses estimated by IMUs cause some misalignments to the end of the body limb due to the drifting but are relatively accurate in a short period. Hence, we use the $\mathcal{L}_{prior}$ to constrain the root-relative body pose $\theta$, encouraging an optimized pose as close to it as possible. 

\PAR{Mesh-to-points term.}
The point cloud $\mathcal{P}$ from the moving LiDAR provides strong prior depth information, and we use them to optimize $\theta$ and $T$. 
Although the SMPL mesh is watertight and complete, the human points are sparse and partial, which makes the registration methods such as ICP, not ideal as expected.
To address this issue, we propose a viewpoint-based mesh-to-point loss function. First, we remove the hidden SMPL mesh faces from the LiDAR's viewpoint. Then we sample points, denoted as $P'\,\!$, from the remaining faces by LiDAR resolution. Finally, we define the $\mathcal{L}_{m2p}$ as minimizing the Chamfer Distance from $P'\,\!_{i}$ to $\mathcal{P}$. 

\begin{equation}
    \vspace{-2mm}
	\begin{split} 
    & \mathcal{L}_{trans} = 
        \frac{1}{l-2}\sum_{j=k}^{k+l-2}\|T_{j+2} - 2·T_{j+1} + T_{j}\|_2^2,\\ 
    & \mathcal{L}_{jts} = 
        \frac{1}{l-2} \sum_{j=k}^{k+l-2}\|J_{j+2} - 2·J_{j+1} + J_{j}\|_2^2,\\ 
    & \mathcal{L}_{\text {orit}} = 
        \frac{1}{l-1}\sum_{j=k}^{k+l-1} 
        \|R_{j+1} - R_{j}\|_{2}, \\
    & \mathcal{L}_{\text {pri}} = 
        \frac{1}{l}\sum_{j=k}^{k+l} 
        \|\theta_{j} - \theta_{j}^{I}\|_{2}, \\
    & \mathcal{L}_{m2p} = 
        \frac{1}{l}\sum_{j=k}^{k+l} \sum_{p'\,\!\in \mathcal{P}'\,\!_{j}}\min_{p \in \mathcal{P}_{j}}\|p - p'\,\!\|_{2}^{2}.
    \end{split}
    \vspace{-4mm}
\end{equation}

\PAR{Camera extrinsic optimization.}
We utilize two types of 2D-3D correspondences losses to optimize the $K_{ex}$ parameters. One loss $\mathcal{L}_{kpt}$ measures the mean square error (MSE)  between the 2D human keypoints $kpt_{2d}$ in the image and the 3D human keypoints of the optimized SMPL model $kpt_{3d}$ projected to image space; 
the another loss, denoted as $\mathcal{L}_{jts}$, measures the Intersection over Union(IoU) between the 2D human bounding box in the image and the 3D human bounding box projected to image space. 
The objective of the optimization is to minimize the $\mathcal{L}_{cam}$ loss function, and we use gradient descent to iteratively optimize the extrinsic.

\begin{equation}
    \vspace{-3mm}
	\begin{split}   
    K_{ex} = &
    \arg \min _{K_{ex}} \mathcal{L}_{cam}(K_{ex}; kpt_{3d}, kpt_{2d}, Box_{3d}, Box_{2d}), \\
    \mathcal{L}_{cam} = &
    \lambda_{kpt} \mathcal{L}_{kpt} + \lambda_{box} \mathcal{L}_{box}, \\
    \mathcal{L}_{jts} = &
     MSE(K_{ex}kpt_{3d} - kpt_{2d}), \\
    \mathcal{L}_{box} = &
    IoU(K_{ex}Box_{3d} - Box_{2d})
    \end{split}
    \vspace{-3mm}
\end{equation}
\noindent
where $\lambda_{kpt}$ and $\lambda_{box}$ are constant coefficients. 

