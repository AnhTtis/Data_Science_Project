



\documentclass[10pt,twocolumn,letterpaper]{article}



\usepackage[pagenumbers]{cvpr} % To force page numbers, e.g. for an arXiv version

\usepackage{graphicx}
\usepackage{amsmath}
\usepackage{amssymb}
\usepackage{booktabs}
\usepackage{stfloats}
\usepackage{cite}
\usepackage{color}
\usepackage{times}
\usepackage{overpic}
\usepackage{bm}
\usepackage{tabu}
\usepackage{bbding}
\usepackage{multicol}

\usepackage{diagbox}
\usepackage{multirow}
\usepackage{scalerel,stackengine}
\usepackage[table ]{xcolor}
\usepackage[accsupp]{axessibility}
\newcommand{\PAR}[1]{\vskip3pt \noindent {{\bf #1~}}}
\newcommand{\TITLE}{SLOPER4D}
\newcommand{\framevideo}{300k~}
\newcommand{\framelidar}{100k~}
\newcommand{\framemocap}{500k~}
\newcommand{\numberscene}{10~}
\newcommand{\numberseq}{15~}
\newcommand{\numberperson}{12~}










\usepackage[pagebackref,breaklinks,colorlinks]{hyperref}
\usepackage[capitalize]{cleveref}
\crefname{section}{Sec.}{Secs.}
\Crefname{section}{Section}{Sections}
\Crefname{table}{Table}{Tables}
\crefname{table}{Tab.}{Tabs.}

\def\cvprPaperID{5047} % *** Enter the CVPR Paper ID here
\def\confName{CVPR}
\def\confYear{2023}


\begin{document}

\title{SLOPER4D: A Scene-Aware Dataset for Global 4D Human Pose Estimation in Urban Environments

}

\author{Yudi~Dai\textsuperscript{1}
\and Yitai~Lin\textsuperscript{1}
\and Xiping~Lin\textsuperscript{1}
\and Chenglu~Wen\textsuperscript{1}\thanks{Corresponding author.}
\and Lan~Xu\textsuperscript{2}
\and Hongwei~Yi\textsuperscript{3} 
\and Siqi~Shen\textsuperscript{1} 
\and Yuexin~Ma\textsuperscript{2}
\and Cheng~Wang\textsuperscript{1}
\and $^{1}$Xiamen University, China 
\hspace{20mm}$^{2}$ShanghaiTech University, China
\and
$^{3}$Max Planck Institute for Intelligent Systems, Germany
} 
















\makeatletter
\let\@oldmaketitle\@maketitle% Store \@maketitle
\renewcommand{\@maketitle}{
   \@oldmaketitle% Update \@maketitle to insert...
	\begin{center}
      \vspace{-6mm}
      \includegraphics[width=0.96\linewidth]{figures/teaser.pdf}
	\end{center}
   \vspace{-2mm}

  \refstepcounter{figure}\normalfont Figure~\thefigure. 
  Using the head-mounted LiDAR and camera to scan the IMUs wearer, we construct \TITLE, a large scene-aware dataset for global 4D human pose estimation in urban environments, including LiDAR point clouds, the RGB videos with 2D/3D annotations, accurate global human pose annotations, and the reconstructed scene.
  \label{fig:teaser}
  \newline
  }

\makeatother

\maketitle

\begin{abstract}













   We present SLOPER4D, a novel scene-aware dataset collected in large urban environments to facilitate the research of global human pose estimation (GHPE) with human-scene interaction in the wild. Employing a head-mounted device integrated with a LiDAR and camera, we record 12 human subjects' activities over 10 diverse urban scenes from an egocentric view. Frame-wise annotations for 2D key points, 3D pose parameters, and global translations are provided, together with reconstructed scene point clouds. To obtain accurate 3D ground truth in such large dynamic scenes, we propose a joint optimization method to fit local SMPL meshes to the scene and fine-tune the camera calibration during dynamic motions frame by frame, resulting in plausible and scene-natural 3D human poses. Eventually, SLOPER4D consists of 15 sequences of human motions, each of which has a trajectory length of more than 200 meters (up to 1,300 meters) and covers an area of more than 2,000 $m^2$ (up to 13,000 $m^2$), including more than 100K LiDAR frames, 300k video frames, and 500K IMU-based motion frames. With SLOPER4D, we provide a detailed and thorough analysis of two critical tasks, including camera-based 3D HPE and LiDAR-based 3D HPE in urban environments, and benchmark a new task, GHPE. The in-depth analysis demonstrates SLOPER4D poses significant challenges to existing methods and produces great research opportunities. The dataset and code are released at \url{http://www.lidarhumanmotion.net/sloper4d/}.
\end{abstract}

\vspace{-2ex}

\section{Introduction}
\label{sec:intro}
\section{Introduction}
\label{sec:introduction}
% \begin{itemize}
%     % Diffusion of FL
%     \item {\st{Diffusion of FL}}
%     % Security threats to FL
%     \item {\st{Security threats to FL with particular focus on model poisoning}}
%     % Limitations of existing countermeasures
%     \item {\st{Current countermeasures (e.g., KRUM) and their limitations}}
%     % Proposed method and its advantages
%     \item {\st{Intuitive description of the proposed method and its difference (i.e., advantages) w.r.t. state of the art}}
%     % Main contributions
%     \item {\st{Summary of the main contributions of this work}}
%     % Paper's structure and organization
%     \item {\st{Paper's structure and organization}}
% \end{itemize}

% Diffusion of FL
Recently, {\em federated learning} (FL) has emerged as the leading paradigm for training distributed, large-scale, and privacy-preserving machine learning (ML) systems~\cite{mcmahan2017googleai,mcmahan2017aistats}. 
The core idea of FL is to allow multiple edge clients to collaboratively train a shared, global model without disclosing their local private training data.
%Specifically, an FL system consists of a central server and many edge clients; 
A typical FL round involves the following steps: {\em(i)} the server randomly picks some clients and sends them the current, global model; {\em(ii)} each selected client locally trains its model with its own private data; then, it sends the resulting local model to the server;\footnote{Whenever we refer to global/local model, we mean global/local model {\em parameters}.} {\em(iii)} the server updates the global model by computing an \emph{aggregation function}, usually the average (FedAvg), on the local models received from clients.
% \begin{enumerate}
%     \item[{\em(i)}] the server sends the current, global model to the clients and appoints some of them for training;
%     \item[{\em(ii)}] each selected client locally trains its copy of the global model with its own private data; then, it sends the resulting local model back to the server;\footnote{Whenever we refer to global/local model, we mean global/local model {\em parameters}.}
%     \item[{\em(iii)}] the server updates the global model by computing an \emph{aggregation function} on the local models received from clients (by default, the average, also referred to as FedAvg~\cite{mcmahan2017aistats}).
% \end{enumerate}
This process goes on until the global model converges. %(e.g., after a certain number of rounds or other similar stopping criteria).
%\\
% The advantages of FL over the traditional, centralized learning paradigm are undoubtedly clear in terms of flexibility/scalability (clients can join/disconnect from the FL network dynamically), network communications (only model weights\footnote{We will use \textit{parameters} and \textit{weights} interchangeably.} are exchanged between clients and server), and privacy (each client's private training data is kept local at the client's end and not uploaded to the server).
\\
% Security threats to FL
%However, the growing adoption of FL also raises security concerns~\cite{costa2022covert}, particularly about its confidentiality, integrity, and availability.
Although its advantages over standard ML, FL also raises security concerns~\cite{costa2022covert}. %, particularly about its confidentiality, integrity, and availability~\cite{costa2022covert}.
% OLD, LONG VERSION
% Indeed, some work deals with privacy leakage that may expose the local data of some clients~\cite{melis2019sp}. 
% A large body of work, instead, investigates attacks that usually aim to detriment the predictive accuracy of the learned global model. For instance, \emph{data poisoning} attacks achieve this goal by letting an adversary pollute the training set of some corrupt FL clients with maliciously crafted examples~\cite{jagielski2018sp}.
% Similarly, in \emph{model poisoning} the attacker attempts to tweak the global model weights~\cite{bhagoji2019pmlr} by directly perturbing the local model's weights of some infected FL clients before these are sent to the central server for aggregation, usually via so-called Byzantine attacks. 
% It turns out that Byzantine model poisoning attacks severely impact standard FedAvg; therefore, more robust aggregation functions must be designed to make FL systems secure.
Here, we focus on \emph{untargeted model poisoning} attacks~\cite{bhagoji2019pmlr}, where an adversary attempts to tweak the global model weights %\footnote{We will use the terms \textit{parameters} and \textit{weights} interchangeably.} 
by directly perturbing the local model's parameters of some infected clients before these are sent to the central server for aggregation.
In doing so, the adversary aims to jeopardize the global model \textit{indiscriminately} at inference time.
Such model poisoning attacks severely impact standard FedAvg; therefore, more robust aggregation functions must be designed to secure FL systems.
\\
% In this paper, we focus on designing a novel robust aggregation scheme at the server's end to contrast the effect of Byzantine model poisoning attacks.
%
% Current countermeasures and their limitations
%Several countermeasures have been proposed in the literature to combat model poisoning attacks on FL systems.
% Some methods use simple statistics more robust than plain average to smooth the impact of malicious updates (e.g., Trimmed Mean and FedMedian~\cite{yin2018icml}). 
% Other defenses implement outlier detection techniques to discard malicious updates from the aggregation performed at the server's end. Those are either based on heuristics (e.g., Krum/Multi-Krum~\cite{blanchard2017nips} and Bulyan~\cite{mhamdi2018pmlr}) or data-driven approaches (e.g., K-means clustering~\cite{shen2016acm} or DnC via spectral analysis~\cite{shejwalkar2021ndss}). 
% Finally, some strategies rely on a centralized ``source of trust'' to spot potential malicious updates (e.g., FLTrust~\cite{cao2020fltrust}).
% Several countermeasures have been proposed in the literature to combat model poisoning attacks on FL systems, i.e., to discard possible malicious local updates from the aggregation performed at the server's end. 
% These techniques range from simple statistics more robust than plain average (e.g., Trimmed Mean and FedMedian~\cite{yin2018icml}) to outlier detection heuristics (e.g., Krum/Multi-Krum~\cite{blanchard2017nips} and Bulyan~\cite{mhamdi2018pmlr}) or data-driven approaches (e.g., spectral analysis via K-means clustering~\cite{shen2016acm} or spectral analysis), or methods based on ``source of trust'' (e.g., FLTrust~\cite{cao2020fltrust}).
% OLD, LONG VERSION
%Several countermeasures have been proposed in the literature to combat Byzantine model poisoning attacks on FL systems.
% Descriptive statistics
% For example, Trimmed Mean and FedMedian aggregate local model updates using more robust statistics than standard average~\cite{yin2018icml}.
%
% % Heuristics for outlier detection
% Many existing Byzantine-resilient strategies implement some outlier detection heuristics to discard the model updates sent by potentially malicious clients from the input of the aggregation function.
% One of the most popular heuristics is Krum~\cite{blanchard2017nips}.
% This strategy tries to mitigate the impact of Byzantine attacks by selecting as a global model the local model with the smallest sum of Euclidean distances to {\em all} the other local models.
% Although powerful, Krum requires the server to know (or, at least, estimate) the number of malicious FL clients upfront, which is generally impossible in a realistic attack scenario. %
% Moreover, Krum may become ineffective for complex, high-dimensional model parameter spaces due to the curse of dimensionality.
% Bulyan~\cite{mhamdi2018pmlr} tries to overcome this issue by combining Krum with a variant of Trimmed Mean.
% % Data-driven outlier detection
% Other strategies use data-driven outlier detection techniques -- e.g., via K-means clustering~\cite{shen2016acm} -- to spot potential malicious local model updates. 
% %For instance, Shen et al. propose to cluster local model updates with K-means and thus identify outliers.
%
% % Other techniques
% As far as the server is concerned, any local model received can be from a potential malicious client. 
% FLTrust~\cite{cao2020fltrust} assumes the server acts as a client, i.e., trains a local model on an additional {\em trustworthy} dataset at the server's end and compares it against all the local models from other clients. 
% This way, the server can rely on some ``source of trust'' when discarding potentially malicious clients.
%\\
% Limitations of existing Byzantine-resilient strategies
Unfortunately, existing defense mechanisms either rely on simple heuristics (e.g., Trimmed Mean and FedMedian by~\cite{yin2018icml}) or need strong and unrealistic assumptions to work effectively (e.g., foreknowledge or estimation of the number of malicious clients in the FL system, as for Krum/Multi-Krum~\cite{blanchard2017nips} and Bulyan~\cite{mhamdi2018pmlr}, which, however, cannot exceed a fixed threshold).
Furthermore, outlier detection methods using K-means clustering~\cite{shen2016acm} or spectral analysis like DnC~\cite{shejwalkar2021ndss} do not directly consider the temporal evolution of local model updates received.
Finally, strategies like FLTrust~\cite{cao2020fltrust} require the server to collect its own dataset and act as a proper client, thereby altering the standard FL protocol.
\\
% OLD, LONG VERSION
% Overall, existing Byzantine-resilient strategies are either simple heuristics (e.g., FedMedian) or, if they are more complex, they rely on strong and unrealistic assumptions to work effectively (e.g., knowing the number of malicious clients in the FL system in advance, as for Krum and alike).
% Furthermore, data-driven outlier detection methods do not consider the temporary evolution of local model updates received (e.g., K-means clustering). 
% Finally, strategies like FLTrust requires the server to collect its own dataset and act as a proper client, thereby altering the standard FL protocol.
%
% Description of the proposed method
This work introduces a novel pre-aggregation \textit{filter} robust to untargeted model poisoning attacks. Notably, this filter $(i)$ operates without requiring prior knowledge or constraints on the number of malicious clients and $(ii)$ inherently integrates temporal dependencies. 
The FL server can employ this filter as a preprocessing step before applying \textit{any} aggregation function, be it standard like FedAvg or robust like Krum or Bulyan.
Specifically, we formulate the problem of identifying corrupted updates as a multidimensional (i.e., matrix-valued) time series anomaly detection task. 
The key idea is that legitimate local updates, resulting from well-calibrated iterative procedures like stochastic gradient descent (SGD) with an appropriate learning rate, show \textit{higher predictability} compared to malicious updates. This hypothesis stems from the fact that the sequence of gradients (thus, model parameters) observed during legitimate training exhibit regular patterns, as validated in Section~\ref{subsec:intuition}. %until convergence. 
%This regularity may be more pronounced for smooth convex loss functions, but it can still be captured within an appropriate time window, even for more complex and convoluted loss surfaces. 
%We provide evidence of this claim in Appendix~B, where we show that the average mutual information (i.e., ``predictability''), calculated over pairs of legitimate model updates sent at different FL rounds, is significantly higher than the corresponding computation for a malicious client.
\\
Inspired by the matrix autoregressive (MAR) framework for multidimensional time series forecasting~\cite{chen2021je}, we propose the FLANDERS ({\em \textbf{F}ederated \textbf{L}earning meets \textbf{AN}omaly \textbf{DE}tection for a \textbf{R}obust and \textbf{S}ecure}) filter.
The main advantages of FLANDERS over existing strategies like FLDetector~\cite{zhao2020multivariate} are its resilience to large-scale attacks, where $50\%$ or more FL participants are hostile, and the capability of working under realistic non-iid scenarios.
We attribute such a capability to two key factors: $(i)$ FLANDERS works without knowing a priori the ratio of corrupted clients, and $(ii)$ it embodies temporal dependencies between intra- and inter-client updates, quickly recognizing local model drifts caused by evil players. Below, we summarize our main contributions:

\begin{itemize}
\item[{\em(i)}]
We provide empirical evidence that the sequence of models sent by legitimate clients is more predictable than those of malicious participants performing untargeted model poisoning attacks.
\\
\item[{\em(ii)}] 
We introduce FLANDERS, the first pre-aggregation filter for FL robust to untargeted model poisoning based on multidimensional time series anomaly detection.
\\
\item[{\em(iii)}] 
We integrate FLANDERS into Flower,\footnote{\scriptsize{\url{https://flower.dev/}}} a popular FL simulation framework for reproducibility.
\\
\item[{\em(iv)}] 
We show that FLANDERS improves the robustness of the existing aggregation methods under multiple settings: different datasets, client's data distribution (non-iid), models, and attack scenarios.
\\
\item[{\em(v)}] 
We publicly release all the implementation code of FLANDERS along with our experiments.\footnote{\scriptsize{\url{https://anonymous.4open.science/r/flanders_exp-7EEB}}}
\end{itemize}

% Paper's structure and organization
The remainder of the paper is structured as follows. %some related work and the current state-of-the-art solutions to security issues that FL entails. 
Section~\ref{sec:background} covers background and preliminaries. 
In Section~\ref{sec:related}, we discuss related work.
Section~\ref{sec:problem} and Section~\ref{sec:method} describe the problem formulation and the method proposed. % to tackle it. 
Section~\ref{sec:experiments} gathers experimental results. %, and Section~\ref{sec:limitations} discusses some limitations of this work.
Finally, we conclude in Section~\ref{sec:conclusion}.
 %discusses the limitations of this work and draws future research directions.
%reports conclusions and draws perspectives for future research directions.

%%%%%%% OLD %%%%%%%
%to overcome the resilience of Byzantine failures in distributed Stochastic Gradient Descent computations. 
% The strength of Krum is its time complexity, which is linear in the gradient dimension. 
% However, the robustness of the approach is guaranteed for gradient-based learning applications only when the majority of the clients are not compromised. 
% Besides, the aggregation mechanism of Krum, as well as that of similar methods, is robust from a coarse-grained perspective and does not provide solutions to errors and perturbations that may occur at inference time.
%A related approach to~\cite{blanchard2017nips} is the work of Su et al.~\cite{su2016dc}. Here, the authors propose an iterated approximate agreement to tackle a multi-layer scenario attacked by Byzantine agents. 
%However, the method works efficiently on the sole discrete context and it is inapplicable to continuous state environments.
%\gabri{Maybe, we should just talk about the main limitations of existing countermeasures without digging into their details (or, we can just mention Krum as this is the most popular one). I will move the description of all these methods to the Related Work section.}

\section{Related Work}
\label{sec:Related work}
\section{Related Work}
\label{sec:relatedwork}

%%%%%%%%%%%%%%%%%%%%%%%%%% Outline %%%%%%%%%%%%%%%%%%%%%%%%%%%%%%%%%%%%%
%(1) Evasion Attacks
%(1.1) Surveys on evasion attacks and their relation to data properties - Michael
%(1.2) Individual papers that study non-data related reasons behind evasion attacks - Michael
%(1.3) Techniques related to evasion attacks and defenses (new) - Gabby
%(2) Non-Evasion Attacks (new), and - ???
%(3) Effects of training data on standard generalization - done 
%
%
%
%(1) Evasion Attacks
%(1.1) A number of surveys review literature on evasion attacks. - Michael
%Most of them do not focus specifically on properties of data but also discuss attack and defense mechanisms, non-data-related reasons for adversarial vulnarability, and  more. ~\jr{cite 4}.
%Yet, they these surveys mention data and its relation to evasion attacks. Specifically \jr{what they say about data.}
%The most close to ours is concurrent work by XXX + concrete facts that we have and they don't.
%
%(1.2) individual papers that study non-data related reasons behind evasion attacks, - Michael
%Literature identifies multiple reasons for adversarial vulnerability, in particular, for evasion attacks. 
%These include data-related properties extensively discussed in this survey, as well as reasons related to the models 		   themselves, computations resources, and feature representations. We discuss these below. 
%
%\jr{the rest is from the paper (non-data related reasons for adversarial vulnerability), with sections potentially renamed.}
%
%{\bf Model.}
%
%{\bf Computational Resources.}
%
%{\bf Robustness of Features.}
%
%(1.3) Techniques Related to Evasion Attacks and Defenses (new) - Gabby
%A number of works focus on techniques for generating evasion attacks, countermeasures against these attacks, 
%and defining the notion of the attack itself.   
%
%{\bf Attacks and Defense.}
%Here are the 5 remaining surveys + 1 additional paper for the reviewer.
%
%{\bf Adversarial Examples.}
%2 surveys lines 13 and 14 + 1 additional paper for the reviewer.
%
%(2) Non-Evasion Attacks (new) 
%Need to say that there are other type of attacks, define them, cite surveys (Bo's survey, maybe something else). 
%Only one work explicitly focus on effects of data. 
%
%
%(3) Effects of training data on standard generalization (done)

%%%%%%%%%%%%%%%%%%%%%%%%% Outline %%%%%%%%%%%%%%%%%%%%%%%%%%%%%%%%%%%%%


\revreplace{
We divide related work into three categories:
(1) surveys on adversarial robustness and its relation to data properties,
(2) surveys that discuss the influence of data properties on standard generalization, and
(3) individual papers that study non-data-related reasons for adversarial vulnerability.\\
}
{
This survey investigates properties of training data in the context of model robustness under evasion attacks. 
We start the discussion of related work by reviewing other surveys that focus on evasion attacks and 
include some discussion about data (Section~\ref{sec:relatedwork-surveys-data}).  
We then discuss non-data related reasons behind evasion attacks (Section~\ref{sec:relatedwork-not-data}),
as well as techniques related to evasion attacks and defenses (Section~\ref{sec:relatedwork-attacks}). 
Finally, we discuss data-related concerns for non-evasion attacks (Section~\ref{sec:relatedwork-poisoning}) and
the effects of training data on standard generalization (Section~\ref{sec:relatedwork-standard}).
}

%\vspace{-0.1in}
\subsection{Surveys on Evasion Attacks that Discuss Data}
\label{sec:relatedwork-surveys-data}
Numerous existing surveys 
\revreplace{focus on attack and defense techniques for adversarial robustness. 
%~\cite{Biggio:Roli:PR:2018,
%Rosenberg:Shabtai:Elovici:Rokach:CSUR:2021,
%Li:Li:Ye:Xu:CSUR:2021,
%Maiorca:Biggio:Giorgio:CSUR:2019,
%Demetrio:Coull:Biggio:Lagorio:Armando:Roli:ACMTPS:2021,
%Liu:Tantithamthavorn:Li:Liu:CSUR:2022,
%Liu:Nogueria:Fernandes:Kantarci:IEEECST:2022,
%Akhtar:Mian:IEEEAccess:2018,
%Akhtar:Mian:Kardan:Shah:IEEEAccess:2021,
%Serban:Poll:Visser:CSUR:2020,
%Machado:Silva:Goldschmidt:CSUR:2021,
%Zhang:Sheng:Alhazmi:Li:ACMTIST:2020}.
Only a few of these works mention the relationship between adversarial robustness and properties of the underlying data.} 
{review the literature on evasion attacks.
Most of these works do not focus specifically on properties of data but discuss attack and defense mechanisms, non-data-related reasons for adversarial vulnerability, 
and the different threat models. 
Only a few of these works mention data-related reasons for the existence of adversarial examples~\cite{Serban:Poll:Visser:CSUR:2020, Machado:Silva:Goldschmidt:CSUR:2021, Akhtar:Mian:Kardan:Shah:IEEEAccess:2021, Akhtar:Mian:IEEEAccess:2018}.
}
Specifically, Serban et al.~\cite{Serban:Poll:Visser:CSUR:2020} observe that adversarial vulnerability can be caused by an insufficient training sample size %~\cite{Schmidt:Santurkar:Tsipras:Talwar:Madry:NeurIPS:2018}
and high data dimensionality. %~\cite{Gilmer:Metz:Faghri:Schoenholz:Raghu:Wattenberg:Goodfellow:ICLR:2018}.
Similarly, Machado et al.~\cite{Machado:Silva:Goldschmidt:CSUR:2021} mention that the lack of sufficient training data, high dimensionality, 
and high concentration contribute to adversarial vulnerability.
\revadd{
Akhtar et al.~\cite{Akhtar:Mian:IEEEAccess:2018, Akhtar:Mian:Kardan:Shah:IEEEAccess:2021} also mention high dimensionality, along with other non-data-related reasons, 
as a source of adversarial examples.}

\revadd{A concurrent work by Han et al.~\cite{Han:Lin:Shen:Wang:Guan:CSUR:2023} (published at the end of April 2023) 
studies the origins of adversarial vulnerability in deep learning w.r.t. the model, data, and other perspectives.
The authors mention high dimensionality, distributions with high concentration, a small number of output classes, data imbalance, and the perceptual difference in image frequencies as potential sources of adversarial examples.
However, as (a) the focus of that survey is not on data-related properties in particular, 
(b) its paper search was conducted in 2021, and 
(c) it focuses on deep learning models only, 
our work was able to identify more than 50 additional relevant papers which focus on other types of models, 
e.g., non-parametric and linear classifiers, 
and/or discuss additional types of data-related properties, 
such as, types of distribution, class density, separation, and label quality.}
\revreplace{Yet, none of these surveys explicitly collect and analyze work that focuses on the effects of data properties
on adversarial robustness.}
{In summary, by explicitly focusing on the effects of data properties on evasion attacks in our survey, 
we are able to provide a more complete and detailed discussion on this topic, not covered in prior surveys.}

\vspace{-0.05in}
\subsection{Non-data-related Reasons Behind Evasion Attacks}
\label{sec:relatedwork-not-data}

%\vspace{-0.1in}
%\subsection{Non-data Related Reasons for Adversarial Vulnerability}

There has been a variety of hypotheses regarding the reasons behind adversarial vulnerability of ML systems, particularly for evasion attacks.
%\revreplace{
%In addition to the data used for training,  adversarial robustness could also depend on the choice of the model architecture,
%the training procedure, and the interplay between data and the learning algorithm, i.e., correspondence between the complexity of a model to that of the data.
%This section summarizes the key hypotheses regarding these aspects.
%%The hypotheses reviewed in this section are complementary to the potential influence from the data.
%}
These include data-related properties extensively discussed in this survey, as well as reasons related to the models themselves, 
computational resources, and feature learning procedures. We discuss these below.

%\jr{there is a lot of undefined terminology and jargon in this section.}

\vspace{0.02in}
\noindent
\textbf{Model.}
When Szegedy et al.~\cite{Szegedy:Zaremba:Sutskever:Bruna:Erhan:Goodfellow:Fergus:ICLR:2014} first discovered adversarial examples for visual models, they suspected that the high non-linearity of DNNs resulted in low probability `pockets' of adversarial examples in the learned representation manifold.
They hypothesize that while these pockets can be found through attack algorithms, the samples residing in these pockets have different distributions compared to normal samples and are thus subsequently harder to find when randomly sampling from the input space.
Instead, Goodfellow et al.~\cite{Goodfellow:Shlens:Szegedy:ICLR:2015} hypothesize that
the linearity from activation functions, like ReLU and sigmoid found in high-dimensional neural networks, induce vulnerability towards adversarial perturbations.
To support their claim, they present the attack method FGSM that exploits the linearity of the target classifier.
Fawzi et al.~\cite{Fawzi:Fawzi:Frossard:ICMLWorkshop:2015} also argue against the hypothesis of high non-linearity as the cause for adversarial examples.
They show that all classifiers are susceptible to adversarial attacks and claim that it is the low flexibility of the classifier compared to the complexity of the classification task that results in vulnerability.
The lack of consensus on the primary causes of model vulnerability invites more studies on this topic.

Singla et al.~\cite{Singla:Ge:Basri:Jacobs:NeurIPS:2021} show that enforcing invariance to circular shifts (e.g., rotation) in neural networks induces decision boundaries with a smaller margin than normal, fully connected networks,
which, in turn, reduces the adversarial robustness of the model.
Moosavi{-}Dezfooli et al.~\cite{Moosavi-Dezfooli:Fawzi:Fawzi:Frossard:Soatto:ICLR:2018} introduce universal,
input-agnostic perturbations to mislead the classifier and hypothesize that the vulnerability of a multi-class classifier to such perturbations is related to the shape of its decision boundaries, e.g.,
linear classifiers with decision boundaries that are parallel to each other and
nonlinear classifier with decision boundaries that are curved in a similar way
tend to be less robust as
perturbations in one direction can change the prediction label for a different class.

Tanay and Griffin~\cite{Tanay:Griffin:ArXiv:2016} conjecture that the decision boundary learned by the classifier being too close to (or `tilted towards') the data manifold instead of being perpendicular to it,
results in small perturbations being sufficient to move samples across the decision boundary for misclassification.
%data manifold refers to the underlying structure that the data exhibit

\vspace{0.02in}
\noindent
\textbf{Computational Resources.}
Bubeck et al.~\cite{Bubeck:Lee:Price:Razenshteyn:ICML:2019} use computational hardness theory to show that the time complexity for learning a robust model is exponential to the size of input data and thus is computationally intractable.
Hence, they attribute adversarial vulnerability to computational limitations of current learning algorithms.
Degwekar et al.~\cite{Degwekar:Nakkiran:Vaikuntanathan:COLT:2019} further extend this work and also show the impossibility of efficiently training robust classifiers.

%\subsubsection{Ineffective Learning Perspective}
\vspace{0.02in}
\noindent
\textbf{Feature Learning.}
Ilyas et al.~\cite{Ilyas:Santurkar:Tsipras:Engstrom:Tran:Madry:NeurIPS:2019} show that adversarial vulnerability can be a consequence of a model exploiting well-generalizing but non-robust features,
i.e., features that are spurious and sometimes incomprehensible to humans;
when constraining the model to use robust features, the adversarial robustness increases together with the
interpretability of the learned features.
However, Tsipras et al.~\cite{Tsipras:Santurkar:Engstrom:Turner:Madry:ICLR:2019} note that, as the features for achieving high accuracy may be different from the ones for achieving high robustness, robustness may be at odds with standard accuracy.
%
%\jr{why is it called Ineffective learning when it is about features.}\gx{I put it under ineffective learning as in this case, the model learns/decides the features for generalization, and when given the correct objective, the model in fact, can learn more robust features, so I think the underlying reason is objective we gave for the model didn't guide the model to learn the right features}
%
Instead of seeing adversarial vulnerability as a product of classifiers being overly sensitive to changes in spurious features, Jacobsen et al.~\cite{Jacobsen:Behrmann:Zemel:Bethge:ICLR:2019} hypothesize that classifiers can rather be
overly insensitive to relevant semantic information, e.g., images with drastically different content can share similar latent representations.
The authors introduce a new type of adversarial examples that exploit such insensitivity, where the content of images is altered without changing the resulting prediction label.
%As both insensitivity to semantic content and sensitivity to spurious changes can simultaneously exist in models,
%more investigation into how to define proper objectives for models to effectively distinguish the relevant information is needed.

While all these works propose possible reasons for adversarial vulnerabilities, they are orthogonal to our survey, which focuses particularly on the influence of training data.

\vspace{-0.05in}
\revadd{
\subsection{Evasion Attacks and Defenses}
\label{sec:relatedwork-attacks}
A number of works focus on techniques for generating evasion attacks, countermeasures against these attacks, 
and defining the notion of the attack itself.

%\jr{need to include~\cite{Biggio:Roli:PR:2018,
%Rosenberg:Shabtai:Elovici:Rokach:CSUR:2021,
%Li:Li:Ye:Xu:CSUR:2021,
%Maiorca:Biggio:Giorgio:CSUR:2019,
%Demetrio:Coull:Biggio:Lagorio:Armando:Roli:ACMTPS:2021,
%Liu:Tantithamthavorn:Li:Liu:CSUR:2022,
%Liu:Nogueria:Fernandes:Kantarci:IEEECST:2022,
%Zhang:Sheng:Alhazmi:Li:ACMTIST:2020} x and one more survey.}
%\js{\cite{Biggio:Roli:PR:2018, Rosenberg:Shabtai:Elovici:Rokach:CSUR:2021} moved to Adversarial Examples.
%\cite{Rosenberg:Shabtai:Elovici:Rokach:CSUR:2021,
%Li:Li:Ye:Xu:CSUR:2021,
%Maiorca:Biggio:Giorgio:CSUR:2019, Liu:Tantithamthavorn:Li:Liu:CSUR:2022,
%Liu:Nogueria:Fernandes:Kantarci:IEEECST:2022,
%Zhang:Sheng:Alhazmi:Li:ACMTIST:2020, Demetrio:Coull:Biggio:Lagorio:Armando:Roli:ACMTPS:2021} in Attacks and Defense. \cite{Sun:Dou:Yang:Zhang:Wang:Philip:He:Li:TKDE:2022} was the "one more survey" and is also in Attacks and Defenses.}

\vspace{0.02in}
\noindent
{\bf Attacks and Defense.}
Several works~\cite{Liu:Tantithamthavorn:Li:Liu:CSUR:2022,Liu:Nogueria:Fernandes:Kantarci:IEEECST:2022,Sun:Dou:Yang:Zhang:Wang:Philip:He:Li:TKDE:2022, Demetrio:Coull:Biggio:Lagorio:Armando:Roli:ACMTPS:2021} survey adversarial attacks and defenses, observing that most work focuses on computer vision and NLP domains. 
Zhang et al.~\cite{Zhang:Sheng:Alhazmi:Li:ACMTIST:2020}, 
Rosenberg et al.~\cite{Rosenberg:Shabtai:Elovici:Rokach:CSUR:2021},
Li et al.~\cite{Li:Li:Ye:Xu:CSUR:2021}, and 
Maiorca et al.~\cite{Maiorca:Biggio:Giorgio:CSUR:2019}, 
survey attacks and defenses in the NLP domain, cybersecurity domain for networks, Android malware, and PDF malware, respectively. 
These works identify a similar trend of new attacks constantly bypassing defenses, which gives rise to new defenses being proposed, only to be broken again (a.k.a. the `cat and mouse race' or the `arms race'). 
They also observe that research in this field studies attacks / defenses at a feature-level, which restricts 
the practicality of the developed techniques by the feasibility of perturbing the corresponding features in real life. 

%practical attacks are quite difficult and require some basic knowledge about the model or training data such as the feature set or model architecture. 
%Zhang et al.~\cite{Zhang:Sheng:Alhazmi:Li:ACMTIST:2020}, who study adversarial attacks and defenses in the NLP domain,  
%also find that there are obstacles to generating attacks in real-time. 
%For instance, methods that iteratively use gradients to create adversarial examples can be time-consuming, while one-time approaches may fail to produce potent adversarial examples.
%Several works~\cite{Liu:Tantithamthavorn:Li:Liu:CSUR:2022,Liu:Nogueria:Fernandes:Kantarci:IEEECST:2022,Sun:Dou:Yang:Zhang:Wang:Philip:He:Li:TKDE:2022, Demetrio:Coull:Biggio:Lagorio:Armando:Roli:ACMTPS:2021} 
%discuss how most new attacks and defenses are explored in computer vision and NLP, prior to other fields.


%our survey finds the state of the art w.r.t. data properties
%our survey finds that dimensionality is bad ...
%
%%%Here are the 5 remaining surveys + 1 additional paper for the reviewer.
%Numerous surveys have explored the landscape of adversarial evasion attacks and defenses. 
%For instance, Akhtar et al.~\cite{Akhtar:Mian:IEEEAccess:2018, Akhtar:Mian:Kardan:Shah:IEEEAccess:2021} survey the literature on adversarial robustness of deep learning models from Computer Vision field.
%They review popular attacks on visual models, and provided a categorization of existing defense techniques based on the components it modify in the visual model system \gx{Check}.
%
%Rosenberg et al.~\cite{Rosenberg:Shabtai:Elovici:Rokach:ACMComputingSurvey:2021}, Li et al. ~\cite{Li:Li:Ye:Xu:ACMComputingSurvey:2021} and Demetrio et al.~\cite{Demetrio:Coull:Biggio:Lagorio:Armando:Roli:ACMTPS:2021} review the literature on evasion attacks for cyber-security fields. 
%Li et al. proposed a partial order scheme to compare key attacks and defenses techniques for malware detection in Windows, Android, and PDF domains. 
%
%Zhang et al.~\cite{Zhang:Sheng:Alhazmi:Li:ACMTIST:2020} review the literature on adversarial attacks on deep-learning models for textual classification.
%They pointed out the intrinsic differences between Computer Vision and Natural Language Processing fields that pose challenges to directly apply attacks proposed for Visual models to NLP models and identified the strategies proposed that overcomes the barriers.
%The challenges they identified for creating realistic attacks in NLP fields are from a domain characteristics perspective (e.g., definition of imperceptible perturbations, measurement of the semantic changes),  we differ from them by trying to understand the adversarial robustness of machine learning from the characteristics of underlying data. 
%
%Attack and Defenses for wireless and Mobile systems~\cite{Liu:Nogueria:Fernandes:Kantarci:IEEECST:2022}
%
%

More recent research, not included in the surveys above, has also started investigating the 
susceptibility of newer models to adversarial evasion attacks. 
For example, several studies~\cite{Wang:Pan:Hu:Duan:Pan:IJSWIS:2022,Yin:Lin:Sun:Wei:Chen:TIFS:2023, 
Shi:Han:Tan:Kuang:NeurIPS:2022, Wang:Xie:Microsoft:ChatGPT:ArXiv:2023} proposed attack techniques against contemporary models, 
such as Graph Neural Networks, Generative Pre-training Transformers (GPT), and Vision Transformers. 
These studies showed that adversarial examples persist even for the newer models, some of which are 
trained with large volumes of data. 
As all these works focus on attack and defense mechanisms rather than 
the effects of data on adversarial robustness, our work extends and complements this research.
}

\revadd{
\vspace{0.02in}
\noindent
{\bf Adversarial Examples.}
%2 surveys lines 13 and 14 + 1 additional paper for the reviewer.
Adversarial examples are inputs constructed by perturbing a correctly classified sample in a way that makes the change imperceptible to a human. % but causes the model to misclassify the sample.
However, as `imperceptible to a human' is hard to define, existing research on adversarial examples approximates imperceptibility with a small perturbation measured through $L_p$ norms.
A line of research~\cite{Gilmer:Adams:Goodfellow:Anderson:Dahl:ArXiv:2018,Sharif:Bauer:Reiter:CVPRW:2018,Fezza:Bakhti:Hamidouche:Deforges:QoMEX:2019, Mezher:Deng:Karam:EUVIP:2022} 
investigates the validity of this assumption. 
This work shows that perturbations generated by $L_p$ norms do not entirely align with human perceptions, 
i.e., some changes with a small $L_p$ norm can be apparent to humans. 
In addition, adversarial examples with the minimum $L_p$ perturbation may be less effective and transferable than 
higher perturbation~\cite{Biggio:Roli:PR:2018,Rosenberg:Shabtai:Elovici:Rokach:CSUR:2021}. 
Hence, a number of approaches explore metrics for imperceptibility 
in computer vision and NLP domains~\cite{Fezza:Bakhti:Hamidouche:Deforges:QoMEX:2019,Mezher:Deng:Karam:EUVIP:2022, Zhang:Sheng:Alhazmi:Li:ACMTIST:2020}. 
Yet another issue with $L_p$ norms is that they cannot be used reliably in domains other than images. 
For example, in the case of software/malware, simply generating adversarial examples with $L_p$ norms 
may result in feature representations that are not possible in 
the problem space~\cite{Rosenberg:Shabtai:Elovici:Rokach:CSUR:2021,Pierazzi:Pendlebury:Cortellazz:Cavallaro:2020}. 

While all these works focus on the properties of adversarial examples, 
they are orthogonal to the topic of our survey, as we rather focus on how properties of the training data 
affect the success of adversarial examples.
}

%Gilmer et al.~\cite{Gilmer:Adams:Goodfellow:Anderson:Dahl:ArXiv:2018} argue that, while constraining the perturbations by sufficiently small $L_p$ norms can generate indistinguishable samples for most inputs, the actual imperceptibility of the changes depends on the input sample. 
%Several individual studies~\cite{Sharif:Bauer:Reiter:CVPRW:2018,Fezza:Bakhti:Hamidouche:Deforges:QoMEX:2019, Mezher:Deng:Karam:EUVIP:2022} find faults with using $L_p$ norms to generate adversarial examples. They show that the changes measured by $L_p$ norm, does not entirely align with human perceptions, i.e., some changes with a small $L_p$ norm appear apparent to humans. 
%In some domains adversarial examples do not need to be imperceptible but rather semantically preserving. 
%For example, in the case of Android malware~\cite{Rosenberg:Shabtai:Elovici:Rokach:CSUR:2021}, adversarial examples are small perturbations which fool a model while preserving the semantics of the sample, 
%i.e., a malware stays malicious even after the perturbation. 
%This highlights another problem with $L_p$ norm based adversarial examples as Dong et al.~\cite{Dong:Liu:Shang:NeurIPS:2022} show that the semantics of a sample change during adversarial training. 
%Hence, there is a need for metrics to measure the size of perturbations that is imperceptible or semantically preserving.
%Fezza et al.~\cite{Fezza:Bakhti:Hamidouche:Deforges:QoMEX:2019} and Mezher et al.~\cite{Mezher:Deng:Karam:EUVIP:2022} propose to use objective metrics for image quality to approximate the imperceptibility in the computer vision domain.
%Zhang et al.~\cite{Zhang:Sheng:Alhazmi:Li:ACMTIST:2020}, focusing on providing such a metric for Natural Language Processing.
%Vadillo et al.~\cite{Vadillo:Santana:CS:2022} also highlight conducted subject studies to evaluate the noticeability of audio adversarial examples.

%Even in computer vision, adversarial examples are not always imperceptible. For example, Machado et al.~\cite{Machado:Silva:Goldschmidt:CSUR:2021} find that visible perturbations such as adversarial patch~\cite{Brown:Mane:Roy:Abadi:Gilmer:ArXiv:2017}, and graffiti on stop signs~\cite{Eykholt:Evtimov:Fernandes:Li:Rahmati:Xiao:Prakash:Kohno:Song:CVPR:2018} are also considered adversarial examples in research.

%The aforementioned research examines the work on defining and creating adversarial examples, demonstrating the insufficiency of using conventional $L_p$ norms to evaluate the imperceptibility and semantics between clean and adversarial examples. 

\vspace{-0.1in}
\revadd{
\subsection{Non-Evasion Attacks}
\label{sec:relatedwork-poisoning}
Similar to evasion attacks, data poisoning and backdoor attacks aim to compromise model accuracy. 
However, they achieve it by tampering the training data to create deceptive model decision boundaries. 
%Data poisoning attacks involve modifying the training data to create deceptive decision boundaries, either to manipulate the prediction outcomes of a specific input or the entire model.
%Meanwhile, Backdoor attacks are a form of poisoning attacks where the attacker inject tempered training data with triggers 
% and then activates the attack by showing the trigger pattern at inference time.
In addition, backdoor attacks also require perturbing the test instance to result in a misclassification. 
This is achieved by introducing manipulated training data with triggers that can be activated during the testing phase.

Goldblum et al.~\cite{Goldblum:Tsipras:Xie:Chen:Schwarzchild:song:Madry:Li:Goldstein:TPAMI:2022} and Cinà et al.~\cite{Cina:Grosse:Demontis:Sebastiano:Zellinger:Moser:Oprea:Biggio:Pelillo:Roli:CSUR:2023} 
review recent literature on attack methodologies and countermeasures for both poisoning and backdoor attacks.
Both of these surveys found that existing research made overly-optimistic assumptions when designing / validating attack techniques, e.g., assuming the knowledge of a large portion of training data. 
They advocate for researchers to test proposed methods in more realistic situations to better assess the potential threats. 
Furthermore, they encourage exploration of the relationship between poisoning attacks and evasion attacks. 
This could lead to the creation of attacks that produce less noticeable poisoning examples, 
or defensive strategies that can safeguard models against both backdoor and evasion attacks.
%Their survey catalogs and systematizes the threats in the dataset creation process, and discuss the open problems that benefits the understanding of dataset security. 

In addition to undermining model accuracy, 
adversarial attacks also aim at breaching the privacy and confidentiality of training data. 
In particular, membership inference attacks~\cite{Shokri:Stronati:Song:Shmatikov:SP:2017} attempt to determine whether a specific data point was part of the training set used to train the model.
Hu et al.~\cite{Hu:Salcic:Sun:Dobbie:Yu:Zhang:CSUR:2022} present a comprehensive survey of existing research efforts on membership inference attacks. 
They find that, similar to evasion attacks, the membership inference attack success rate decreases as 
%the training data better represents the whole data distribution, i.e., 
the number of training samples increases.
%and model stealing attacks~\cite{Oliynyk:Mayer:Rauber:CSUR:2023} are designed to breach the privacy of training data and machine learning models. 
However, all these attacks are orthogonal to our survey, as we focus on adversarial evasion attacks.

%Li et al. ~\cite{Li:Jiang:Li:Xia:TNNLS:2022} 
%provide the first survey that focuses on backdoor attacks and identified common scenarios in which backdoor attack happen in real life. 
%Furthermore, they proposed a systematic taxonomy for backdoor attacks and defenses for researchers and practitioners to identify the characteristics and limitations of each method. 

%Wang et al.~\cite{Wang:Ma:Wang:Hu:Qin:Ren:CSUR:2022} and Tian et al.~\cite{Tian:Cui:Liang:Yu:CSUR:2022} argue federated learning~\cite{McMahan:Moore:Ramage:Hampson:Arcas:AISTATS:2017} 
%creates new venue for poisoning attack, and survey recent literature on poisoning attacks for both standard and federated learning scenarios. 
%They present a unified framework to categorize both data poisoning and model poisoning attacks, and compared the defense techniques proposed for each of the learning framework, analyzed their advantages and disadvantages.
}

\vspace{-0.1in}
\subsection{Effects of Training Data on Standard Generalization}
\label{sec:relatedwork-standard}
A number of surveys investigate the influence of data properties on standard
rather than robust generalization.
One of the earliest is probably the work of Raudys and Jain~\cite{Raudys:Jain:TPAMI:1991},
who review studies related to the influence of sample size on binary classifiers, showing that
a limited sample size usually leads to sub-optimal generalization.
%With the development of deep learning and the ever-increasing need for larger training datasets,
%a variety of data augmentation techniques have been proposed.
Bansal et al.~\cite{Bansal:Sharma:Kathuria:CSUR:2021} and
Bayer et al.~\cite{Bayer:Kaufhold:Reuter:CSUR:2022} also survey papers addressing the data scarcity problem,
focusing in particular on the recent advancements in data augmentation techniques in the fields of computer vision, security, and text classification.
Their results show that augmentation techniques %exist for various application domain and
can help improve a model's generalization by reducing the problem of model overfitting.
%They evaluate the effectiveness of such techniques in improving the accuracy of machine learning models.

%Limited sample size is also one of the culprit behind poor robust generalization~\cite{Schmidt:Santurkar:Tsipras:Talwar:Madry:NeurIPS:2018}, we collected a number of researches characterize the sample complexity for robust generalization or propose data augmentation techniques to fill in the sample complexity gap.

Label noise is another aspect of data that influences both standard and robust generalization.
Most works on this topic find that the presence of noisy labels increases the need for a greater number of training samples and may result in unnecessarily complex decision boundaries~\cite{Frenay:Verleysen:TNNLS:2014,Song:Kim:Park:Shin:Lee:TNNLS:2022}.
For example, Fr\'{e}nay and Verleysen~\cite{Frenay:Verleysen:TNNLS:2014} show
that overfitting to label noise greatly degrades a model's standard generalization;
the same effect has been observed in the case of robust generalization~\cite{Sanyal:Dokania:Kanade:Torr:ICLR:2021}.
Song et al.~\cite{Song:Kim:Park:Shin:Lee:TNNLS:2022} survey the impact of label noise in deep learning, arguing
that the presence of noisy labels is a more serious concern for deep models as they contain a larger number of parameters which makes them prone to overfitting to the noise in training data.
%They also point out the connection between adversarial poisoning attacks and noisy labels as
%the countermeasures for both share the goal of learning noise-resilient representations.
They mention that adversarial defense techniques, e.g., adversarial training, are effective against label noise~\cite{Zhu:Zhang:Han:Liu:Niu:Yang:Kankanhalli:Sugiyama:ArXiv:2021, Fatras:Damodaran:Lobry:Flamary:Tuia:Courty:TPAMI:2022}
but do not discuss how label noise influences a deep learning model's robustness under attacks.

Lorena et al.~\cite{Lorena:Garcia:Lehmann:Souto:Ho:CSUR:2020} identify a collection of 26 quantitative metrics that measure data complexity with respect to
(1) ambiguity of classes, i.e., whether the classes can be clearly distinguished with the given features,
(2) sparsity and dimensionality of data, 
%i.e., whether enough information are provided to learn confident decision boundaries, and
(3) complexity of boundary separating the classes, i.e., whether more intricate functions are required to describe the decision boundaries.
The authors also discuss how these metrics help estimate the difficulty of performing classification on a given dataset.
Similar to our survey, the authors show that high dimensionality and small separation between classes hinder standard generalization.
However, the relationship of some of the metrics reviewed by these authors, e.g.,
%faction of borderline points (i.e., a measure for the complexity of the required decision boundary) and
%the fraction of hyperspheres covering data (i.e.,
the number of non-intersecting spheres needed to enclose all data points of a class,
to robust generalization is not studied, according to our survey.

%Moreover, the effect of XXX on standard generalization needs future investigation as well (that is if we found something they do not have).

%Knowing the characteristics of a dataset according to these perspectives can assist researchers and practitioners to select optimal learning algorithms~\cite{Ho:Basu:TPAMI:2002}.

He and Garcia~\cite{He:Garcia:TKDE:2009} focus on the imbalance learning problem. %~--
%the disproportion in the number of samples belonging to each class in a given dataset.
The authors found that most standard algorithms %are designed with the assumption of a balanced class distribution.
%These algorithms
fail to reliably represent the characteristics of the imbalanced data and result in unfavorable performance across classes.
Furthermore, L\'{o}pez et al.~\cite{Lopez:Fernandez:Garcia:Palade:Herrera:InfSci:2013} discuss six intrinsic data characteristics that potentially complicate learning from imbalanced data:
low density, sample overlap between classes, noisy data, borderline instances,
dataset shift between training and testing distributions, and
small disjuncts, i.e., disperse small clusters of samples from a single class.
Their analysis concludes that while all these ``unfavorable'' data characteristics further complicate the data imbalance
issues, data overlap between classes is probably one of the most harmful.
To follow up on this point, Santos et al.~\cite{Santos:Henriques:Pedro:Japkowicz:Fernandez:Soares:Wilk:Santos:AIR:2022}
focus on the joint effect of data imbalance and class overlap on model generalization.
The negative impact of data imbalance, low separation, and noisy data on robust generalization was also discussed in our survey.
Yet, the compounding effect of these factors, as well as the effect of other properties,
on robust generalization needs future investigation.

Recently, Yang et al.~\cite{Yang:Jiang:Song:Guo:IJCV:2022} summarized relevant studies focusing on
long-tailed distributions in the field of Computer Vision.
% and categorize the main methods for alleviating the issues caused by long-tailed distribution.
%They present quantitative metrics for measuring data imbalance and .
This survey also includes work on the influence of long-tail distributions on a model's adversarial robustness~\cite{Wu:Liu:Huang:Wang:Lin:CVPR:2021}, which is covered in our survey.
%which is included in our survey,
The authors advocate for more research on adapting long-tailed-based approaches for standard generalization to improve robust generalization.

Finally, Moreno-Torres et al.~\cite{MorenoTorres:Raeder:Rodrigues:Chawla:Herrera:PR:2012} present a unifying framework to categorize existing definitions of dataset shift~-- the case where the joint distribution of inputs and outputs differs between training and testing data.
While ML models are normally trained under the premise that testing data has a similar distribution to the training data,
in reality, the observed data distribution may be different from the historical data that the model is trained on.
Such difference can substantially compromise the quality of model predictions.
The authors analyze the possible causes for dataset shift, e.g., malicious software that evolves over time, and
review the techniques dealing with dataset shift.
They characterize adversarial attacks as one form of dataset shift, where adversaries adaptively
change test instances to create a distribution that differs from training data.
%All works discussed in our survey assumed similar distribution on training and testing data, treating adversarial attacks as the only dataset shift in the problem setup.
%However, in real applications, the underlying data distribution itself can be non-stationary, and the characterize the influence of the dataset shift between training and testing data on the adversarial robustness is yet to be investigated.

\revadd{Overall, despite the similarities with our work, literature discussed in this section focuses on standard generalization while our survey discusses 
the effect of data on robust generalization.}

%More works use the connection between adversarial attacks and distributional shift to analyze the effect of adversaries on generalization performance~\cite{Tu:Zhang:Tao:NeurIPS:2019}.
%However, we do not discuss them in detail, as they focus more on models instead of data.
%\jr{How is that relevant to data properties section?} \gx{This can be removed, as it an individual work we filtered}

\vspace{-0.1in}
\subsection{Summary}
\revadd{
Our survey is the first to explicitly focus on properties of training data in the context of model robustness under evasion attacks.
Numerous other surveys on evasion attacks discuss attack and defense mechanisms, non-data-related reasons for adversarial vulnerability, and the different threat models. 
We identified only five surveys that considered data-related reasons for evasion attacks. 
However, as these surveys are older and do not focus on data in particular, our work provides a more extensive
and comprehensive view on this topic. 
By including more than 50 papers not covered in prior work, we were able to 
identify additional relevant properties, practical suggestions, and future research directions in this area. 

Additional work studies non-data-related reasons for evasion attacks, as well as non-evasion attacks, 
such as poisoning and backdoor. 
Yet another body of literature examines how data properties affect standard generalization. These works show that 
some of the properties discussed in our survey, such as 
the number of samples, dimensionality, and label quality, also affect clean accuracy. 
There are also additional data properties that are covered exclusively by these or by our work. 
Studying the interplay between data properties for clean and robust accuracy is an interesting research direction, 
which could be facilitated by our work. 
However, all these current works are orthogonal and complementary to ours.
}

%\ad{
%The related work of our survey can be categorized into four key topics: 
%The first topic examines data for other adversarial attacks, this include the research that investigates the link between the data characteristics and model's resilience against poisoning attacks as well as the studies that explore data poisoning and backdoor attacks and their countermeasures. \jr{same issues as before: this is meta-summary, we need a concrete summary.}
%These studies complement our survey as they highlight the threats directly aimed at data, thus emphasizing the importance of secure data collection. 
%The second topic focuses on the relationship between various properties of training data and model's standard generalization ability. 
%This body of work suggests that data traits such as number of samples, dimensionality, label quality also influence model's ability to generalize in standard classification. \jr{this looks more concrete!}
%
%The third strand of research concerns adversarial evasion attacks. 
%The work in this area encompasses the research frontier in evasion attacks and the countermeasures. 
%Due to the large volume of work in this area, there are numerous surveys that gives more detail on the advancement. 
%\jr{meta-summary again}
%In addition to attacks and defenses, one relevant line of work investigates the alignment of the conventional similarity metrics used for adversarial examples and human perception, showing the need for supplementary metrics. \jr{why important?}
%These studies \jr{which "these studies"?} collectively present an extensive overview of other types of work conducted on adversarial robustness.
%The last category of work proposes alternative explanations for model vulnerability to adversarial examples.
%These studies presented hypothesis showing the characteristics of machine learning models, e.g., nonlinearity, invariance to rotational shift etc, induces susceptibility to attacks, as well as limited computational resources and non-robust feature representations. \jr{all text based on previous related work looks somewhat concrete; the new additions should be at least at the same level, or better.}
%These studies supplement our work, offering a broader perspective of potential factors affecting model's robust generalization ability. }
%



\section{\TITLE~Dataset}
\label{sec:dataset_constructing}
\TITLE~collects scene-aware 4D human data with our body-worn capturing system in urban scenes. In this section, we first introduce the data acquisition in \cref{subsec:data_collecting}, second, we detail the data construction and annotation process in \cref{subsec:data_anno}, then we introduce the global optimization-based \cref{subsec:optimization} method to obtain high quality both 3D/2D data, finally, we compare our dataset in \cref{subsec:data_comapre} with the existing datasets and highlight our novelty.

\subsection{Data Acquisition}
\label{subsec:data_collecting}

\begin{figure}[!htb]
    \centering
     \includegraphics[width=0.98\linewidth]{figures/hardware.pdf}
     \vspace{-1mm}
     \caption{\textbf{Our capturing system's hardware details.} The sensor module includes a LiDAR, a camera, and 17 body-attached IMU sensors. The storage module consists of a NUC11, a receiver, and a battery in the backpack.}
     \label{fig:system_design}
    %  \vspace{-\baselineskip}
    \vspace{-5mm}
\end{figure}

\PAR{Hardware setup.}
As shown in \cref{fig:teaser}, during the data collection procedure, the scanning person follows the performer (IMUs wearer) and scans him with a LiDAR and a camera on the helmet. Additionally,  \cref{fig:system_design} shows the hardware details of our capturing system. Regarding the sensor module, the 128-beams Ouster-os1 LiDAR and the DJI-Action2 wide-range camera are rigidly installed on the helmet. To capture raw human motions, we use Noitom's inertial MoCap product, PN Studio, to attach 17 wireless IMUs to the IMU wearer's body limbs, torso, and head. 
The camera's field of view (FOV) is 116°$\times$84° and the LiDAR's FOV is 360°$\times$45°. To make the performer within the LiDAR's FOV as much as possible, we tilt the LiDAR down around 45°. 
Regarding the storage module, the scanning person's backpack places a wireless IMU data receiver, a 24V battery, and an Intel NUC11. The mini-computer NUC11 stores IMU data from the wireless receiver and point clouds from LiDAR in real time. Videos are stored locally in the camera. The LiDAR and NUC11 are both powered by the battery.

\PAR{Coordinate systems.}
Let's define three coordinate systems: 1) IMU coordinate system \{$I$\}: the origin is at LiDAR wearer's spine base at the starting time, and the $X/Y/Z$ axis is pointing left/upward/forward of the human. 2) LiDAR Coordinate system \{$L$\}: the origin is at the center of the LiDAR, and the $X/Y/Z$ axis is pointing right/forward/upward of the LiDAR. 3) Global/World coordinate system \{$W$\}: the origin is on the floor of the LiDAR wearer's starting position, and the $X/Y/Z$ axis is pointing right/forward/upward of the LiDAR wearer.

\PAR{Calibration.} 
Following the setup in \cite{wen2019toward}, we use a chessboard to calibrate the camera intrinsic $K_{in}$ and introduces a terrestrial laser scanner (TLS) to obtain accurate camera extrinsic parameter, $K_{ex}$. Due to the LiDAR point cloud being too sparse, we manually choose the corresponding points both on the 2D image and the TLS map registered to the point cloud, and then we solve the perspective-n-point (PnP) problem to obtain $K_{in}$.
For every 3D scene, the calibration $R_{WL}$, which transforms \{$L$\} to \{$W$\} is manually set to make the ground's $z$-axis upward and height to zero for the starting position.
By using singular value decomposition, the calibration $R_{WI}$, which transforms \{$I$\} to \{$W$\}, is calculated through the similarity between IMU trajectory and LiDAR trajectory on the XY plane.
\PAR{Synchronization.}
The synchronization of data from multiple sensors in human subject data is achieved through peak detection.
Before and after the capture, the subject is asked to perform jumps. Then the peak height time in IMU is automatically detected and the peak times in the LiDAR and camera data are manually identified. Finally, all modalities are aligned by the peaks and downsampled to match the LiDAR frame rate of 20 Hz.  

\subsection{Data Processing}
\label{subsec:data_anno}
\PAR{2D pose detection.} 
We use Detectron~\cite{wu2019detectron2} to detect and Deepsort~\cite{wojke2017simple} to track humans in videos. However, the tracking often fails due to the IMUs wearer entering/exiting the field of view or occlusions. To solve this problem, we manually assign the same ID for the tracked person in a video sequence. As for 3D point cloud reference, we project them on images according to the $K_{ex}$. However, due to the jitter brought by dynamic motions, the camera and the LiDAR are not perfectly rigidly connected. 
Thus, $K_{ex}$ will be further optimized in \cref{subsec:optimization}. 

\PAR{LiDAR-inertial localization and mapping.}
The LiDAR-only method often fails in mapping because of the dynamic head rotation and crowded urban environments.
Incorporating an IMU can compensate for motion distortion in a LiDAR scan $p^{L}$ and provide an accurate initial pose. 
Using a LiDAR with an integrated IMU, and by combining Kalman filter-based lidar-inertial odometry\cite{Xu2022FASTLIO2FD} with factor graph-based loop closure optimization\cite{gtsam}\cite{Kim2018ScanCE}, we successfully estimate the ego-motion of LiDAR and build the global consistency 3D scene map with n frame point clouds $P_{1:n}^L=\{p_{1}^{L},\ldots, p_{n}^{L}\}$. 
To provide accurate scene constrain in \cref{subsec:optimization}, we utilize the VDB-Fusion~\cite{vizzo2022sensors} to generate a clean scene mesh $\bm{S}$ that excludes moving objects.

\PAR{IMUs pose estimation.}
We use SMPL~\cite{smpl2015loper} to represent the human body motion ${M^I} = \varPhi (\theta^I, t^I, \beta) \in \mathbb{R}^{6890}$ in IMU coordinate space\{$I$\}, 
where pose parameter $\Theta_{1:n}^I = \{\theta_{1}^{I},\ldots, \theta_{n}^{I}\} \in \mathbb{R}^{72 \times n}$ is composed of pelvis joint's orientation $R_{1:n}^I = \{r_{1}^{I},\ldots, r_{n}^{I}\} \in \mathbb{R}^{3 \times n}$ and the other 23 joints' rotation relative to their parent joint. 
The $T_{1:n}^I = \{t_{1}^{I},\ldots, r_{n}^{I}\} \in \mathbb{R}^{3 \times n}$ is the pelvis joint's translation and $\beta \in \mathbb{R}^{10}$ is a constant value representing a person's body shape. 
$T$ and $\Theta$ are estimated by the commercial MoCap product, while $\beta$ is obtained by using IPNet~\cite{bhatnagar2020ipnet} to fit the scanned model captured by an iPhone13 Promax.
Since the IMU are accurate locally but drift globally, $T^I$ is used for raw calibration of the \{$I$\} to \{$W$\}, and the initial global motion $M = M^W = R^{WI}M^I$ will be further optimized.

\begin{figure*}[!htb]
    \centering
     \includegraphics[width=0.98\linewidth]{figures/method.pdf}
     \vspace{-1mm}
     \caption{\textbf{The pipeline of the dataset construction.} The capturing system simultaneously collects multimodal data, including LiDAR, camera, and IMU data. Then they are further processed. A joint optimization approach with multiple loss terms is then employed to optimize motion locally and globally. As a result, we obtain rich 2D/3D annotations with accurate global motion and scene information.
     }
     \vspace{-3mm}

     \label{fig:method}
\end{figure*}

\subsection{Data Optimization}
\label{subsec:optimization}
To obtain accurate and scene-plausible human motion $\bm{M^W}$ in the world coordinate, we use scene geometry \bm{${S}$} with several physics terms to perform joint optimizations. The following terms are used: 
the smoothness term $\mathcal{L}_{smt}$,
the scene-aware contact term $\mathcal{L}_{sc}$.
the pose prior term $\mathcal{L}_{pri}$, 
and the mesh to point term $\mathcal{L}_{m2p}$. 
The optimization is to find the $\theta$ and $T$ to minimize $\mathcal{L}$:
\begin{equation}
	\begin{split}   
    & \bm{M} = 
    \arg \min _{\bm{\theta, T}}\mathcal{L} (\theta, T, \bm{S}),\\
    & \mathcal{L}_{} = 
        {\mathcal{L}}_{smt} + 
        \lambda_{sc}\mathcal{L}_{sc} + 
        {\lambda_{pri}\mathcal{L}}_{pri}+
        {\lambda_{m2p}\mathcal{L}}_{m2p}, \\
    & \mathcal{L}_{smt} = 
        \lambda_{trans} \mathcal{L}_{trans} + 
        \lambda_{orit} \mathcal{L}_{orit} + 
        \lambda_{jts} \mathcal{L}_{jts}.
    \end{split}
\end{equation}
\noindent
where $\lambda_{sc}$, $\lambda_{pri}$, $\lambda_{trans}$, $\lambda_{orit}$, $\lambda_{jts}$, and $\lambda_{m2p}$ are  loss terms' coefficients. $\mathcal{L}$ is minimized with a gradient descent algorithm.























\PAR{smoothness term.} This term includes the translation loss $\mathcal{L}_{trans}$, the orientation loss $\mathcal{L}_{orit}$, and the joints loss $\mathcal{L}_{jts}$. The position of the pelvis joint is used to define the human's translation, while the other joints subtract it to define the root-relative joints. The objective of all sub-terms is to minimize acceleration to smooth human movements. 


























\PAR{Scene-aware contact term.} we compare the movement of every foot vertices in IMU motions $\bm{{M}_k^I}$ and label the foot as stable if its velocity is less than 0.1 $m/s^2$. Finally, the Chamfer Distance (CD) between this foot and its closest surface is expressed as the scene contact loss $\mathcal{L}_{sc}$.

\PAR{Pose prior term.} The poses estimated by IMUs cause some misalignments to the end of the body limb due to the drifting but are relatively accurate in a short period. Hence, we use the $\mathcal{L}_{prior}$ to constrain the root-relative body pose $\theta$, encouraging an optimized pose as close to it as possible. 

\PAR{Mesh-to-points term.}
The point cloud $\mathcal{P}$ from the moving LiDAR provides strong prior depth information, and we use them to optimize $\theta$ and $T$. 
Although the SMPL mesh is watertight and complete, the human points are sparse and partial, which makes the registration methods such as ICP, not ideal as expected.
To address this issue, we propose a viewpoint-based mesh-to-point loss function. First, we remove the hidden SMPL mesh faces from the LiDAR's viewpoint. Then we sample points, denoted as $P'\,\!$, from the remaining faces by LiDAR resolution. Finally, we define the $\mathcal{L}_{m2p}$ as minimizing the Chamfer Distance from $P'\,\!_{i}$ to $\mathcal{P}$. 

\begin{equation}
    \vspace{-2mm}
	\begin{split} 
    & \mathcal{L}_{trans} = 
        \frac{1}{l-2}\sum_{j=k}^{k+l-2}\|T_{j+2} - 2·T_{j+1} + T_{j}\|_2^2,\\ 
    & \mathcal{L}_{jts} = 
        \frac{1}{l-2} \sum_{j=k}^{k+l-2}\|J_{j+2} - 2·J_{j+1} + J_{j}\|_2^2,\\ 
    & \mathcal{L}_{\text {orit}} = 
        \frac{1}{l-1}\sum_{j=k}^{k+l-1} 
        \|R_{j+1} - R_{j}\|_{2}, \\
    & \mathcal{L}_{\text {pri}} = 
        \frac{1}{l}\sum_{j=k}^{k+l} 
        \|\theta_{j} - \theta_{j}^{I}\|_{2}, \\
    & \mathcal{L}_{m2p} = 
        \frac{1}{l}\sum_{j=k}^{k+l} \sum_{p'\,\!\in \mathcal{P}'\,\!_{j}}\min_{p \in \mathcal{P}_{j}}\|p - p'\,\!\|_{2}^{2}.
    \end{split}
    \vspace{-4mm}
\end{equation}

\PAR{Camera extrinsic optimization.}
We utilize two types of 2D-3D correspondences losses to optimize the $K_{ex}$ parameters. One loss $\mathcal{L}_{kpt}$ measures the mean square error (MSE)  between the 2D human keypoints $kpt_{2d}$ in the image and the 3D human keypoints of the optimized SMPL model $kpt_{3d}$ projected to image space; 
the another loss, denoted as $\mathcal{L}_{jts}$, measures the Intersection over Union(IoU) between the 2D human bounding box in the image and the 3D human bounding box projected to image space. 
The objective of the optimization is to minimize the $\mathcal{L}_{cam}$ loss function, and we use gradient descent to iteratively optimize the extrinsic.

\begin{equation}
    \vspace{-3mm}
	\begin{split}   
    K_{ex} = &
    \arg \min _{K_{ex}} \mathcal{L}_{cam}(K_{ex}; kpt_{3d}, kpt_{2d}, Box_{3d}, Box_{2d}), \\
    \mathcal{L}_{cam} = &
    \lambda_{kpt} \mathcal{L}_{kpt} + \lambda_{box} \mathcal{L}_{box}, \\
    \mathcal{L}_{jts} = &
     MSE(K_{ex}kpt_{3d} - kpt_{2d}), \\
    \mathcal{L}_{box} = &
    IoU(K_{ex}Box_{3d} - Box_{2d})
    \end{split}
    \vspace{-3mm}
\end{equation}
\noindent
where $\lambda_{kpt}$ and $\lambda_{box}$ are constant coefficients. 




The ARMBench dataset presents: 1) a collection of sensor data acquired by a robotic manipulation workcell performing pick-and-place operation, 2) metadata and reference images for objects in containers, 3) a set of annotations acquired either automatically, by virtue of the system design, or via manual labeling, and 4) tasks and metrics to benchmark perception algorithms for robotic manipulation. Fig.\ \ref{fig:contributions} illustrates the benchmark tasks and variety of objects captured in the dataset. The dataset captures diversity in objects with respect to Amazon product categories as well as physical characteristics such as size, shape, material, deformability, appearance, fragility, etc. 

The data collection platform is a robotic manipulation workcell performing pick-and-place operation in a warehouse \cite{Sparrow2022}. The workcell contains a robotic arm mounted with a vacuum-based end-effector. It is presented with a heterogeneous collection of objects placed in unstructured configurations within a container (storage tote). The robotic arm is tasked with picking one object at a time (singulation) and place it on moving trays until the container is empty. The empty container ejects the workcell and is replaced by a new container. While the operation is completely autonomous, it includes a human-in-the-loop to monitor the status of each pick-and-place activity, annotate, and resolve any defects during manipulation. Multiple imaging sensors are placed in the workcell to facilitate and validate the pick-and-place operation. Following is a list of sensor data (Fig.\ \ref{fig:intro}) associated with each pick activity:
\begin{itemize}
\item Pick-image: A 5\,MP camera is used to capture a top-down image of the container.
% \item Pick-3D: Two Ensenso sensors capture the 3D point cloud of the source container.
\item Transfer-images: Multiple 5\,MP cameras are placed on different sides in the workcell to capture the moving object from different viewpoints.
% \item Transfer-Barcode: Multiple Cognex barcode sensors are used to scan the barcode of the object during transfer.
\item Place-image: A top-down view of the object is captured once it is placed on the tray.
\item Video: A camera is mounted to capture 720p videos of pick-and-place manipulation processes at 30\,FPS
\end{itemize}
Additionally, the following metadata (Fig.\ \ref{fig:contributions} (b)) is available by virtue of a warehouse tracking system:
\begin{itemize}
\item Container-manifest: A list of objects present in the container along with data such as product description, coarse dimensions, and weight.
\item Reference images: One or more images of objects from previous operations within the warehouse.
\end{itemize}
The sensor data and metadata were consumed by perception algorithms required to autonomously operate the robotic workcell. Benchmarking against these algorithms would not only optimize a manipulation task such as the one used for data collection but also enable more complex and intentional manipulation. This work considers a subset of such perception tasks namely object segmentation, object identification, and defect detection. These are critical not only to make informed grasping and motion decisions but also to track the state of the objects and containers within the warehouse. The following sections will describe these tasks and present the challenges using annotations, baseline algorithms, and evaluation metrics.


\section{Experiments}
\label{sec:experiments}
We present in section~\ref{ssec:faces} an application of PnP-HVAE on face images, using a pretrained state-of-the-art hierarchical VAE. 
Next, we study the application of our framework to natural images. To that end, we introduce  in section~\ref{ssec:patchVDVAE}  a patch hierachical VAE architecture, that is able to model natural images of different resolutions. In section~\ref{ssec:app_nat}, we provide deblurring, super-resolution and inpainting experiments to demonstrate the relevance of the proposed method.

Additional results are presented in Appendix~\ref{app:add}. All experiments can be reproduced using the code available at \url{https://github.com/jprost76/PnP-HVAE}.



\subsection{Face Image restoration (FFHQ)}\label{ssec:faces}
We first demonstrate the effectiveness of PnP-HVAE on highly structured data, by performing face image restoration.
Latent variable generative models can accurately model structured images such as face images \cite{karras2019style,vahdat2020nvae,child2021very,kingma2018glow}, and then be used to produce high quality restoration of such data. 
In our experiments, we use the VDVAE model of~\cite{child2021very}, pre-trained on the FFHQ dataset~\cite{karras2019style}, as our hierarchical VAE prior.
VDVAE has $L=66$ latent variable groups in its hierarchy and generates images at resolution $256\times256$.

We compare PnP-HVAE with the intermediate layer optimization algorithm (ILO)~\cite{daras2021intermediate} that is based on a different class of generative models than HVAE. ILO is a GAN inversion method which optimizes the image latent code along with the intermediate layer representation of a StyleGAN to generate an image consistent with a degraded observation.
We use the official implementation of ILO, along with a StyleGAN2 model~\cite{karras2020analyzing, stylegan2pytorch}, that was trained for 550k iterations on images of resolution $256\times256$ from FFHQ.  
As VDVAE and StyleGAN models are not trained on the same train-test split of FFHQ, we chose to evaluate the methods on a subset of 100 images from the CelebA dataset~\cite{liu2018large}. 
For super-resolution, the degradation model corresponds to the application of a gaussian low-pass filter followed by a $\times 4$ sub-sampling, and the addition of a gaussian white noise with $\sigma=3$.
For the deblurring, we considered motion blur and  gaussian kernels, both with a noise level $\sigma=8$. %

We provide quantitative comparisons in table~\ref{table:comp_ILO}, along with a visual comparison of the results in figure~\ref{fig:face_restoration}.
PnP-HVAE has the best  PSNR and SSIM results for all the considered restoration tasks, while ILO provides better results  for the perceptual distance.
By jointly optimizing the image and its latent variable, PnP-HVAE provides  results that are both realistic and consistent with the degraded observation.
On the other hand,  ILO  only optimizes on an extended latent space. This method generates  sharp and realistic images with better LPIPS scores,   
but the results lack  of consistency with respect to the observation, which explains the overall lower PSNR performance. 






\subsection{PatchVDVAE: a HVAE for natural images}\label{ssec:patchVDVAE}
Available generative models in the literature operate on images of  fixed resolutions and
are either restrained to datasets of limited diversity, or even to registered face images~\cite{kingma2018glow,child2021very, vahdat2020nvae, karras2019style}, or requiring additional class information~\cite{brock2018large, dhariwal2021diffusion, song2020score, luhman2022optimizing}.
Fitting an unconditional model on natural images appears to be a more difficult task, as their resolution can change, and their content is highly diverse.
The complexity of the problem can be reduced by learning a prior model on patches of reduced dimension. 
For image restoration problems, the patch model can be reused on images of higher dimensions~\cite{zoran2011learning,prost2021learning,altekruger2022patchnr}. When the model is a full CNN, the prior on the set of the  patches can  be computed efficiently by applying the network on the full image~\cite{prost2021learning}.

We thus introduce  patchVDVAE, a fully convolutional hierarchical VAE.
Contrary to existing HVAE models whose resolution is constrained by the constant tensor at the input of the top-down block, patchVDVAE can generate images of different resolutions by controlling the dimension of the input latent. 
This amounts to defining a prior on patches whose dimension corresponds to the receptive field of the VAE. A similar model is used for image denoising in~\cite{prakash2021interpretable}.

 
For PatchVDVAE architecture, we use the same bottom-up and top-down blocks as VDVAE~\cite{child2021very}, and replace the constant trainable input in the first top-down block by a latent variable, to make the model fully convolutional (details on the  architecture are given in Appendix~\ref{app:details}). 
The training dataset is composed of $128\times 128$ patches extracted from a combination of DIV2K~\cite{agustsson2017ntire} and Flickr2K~\cite{Lim_2017_CVPR_workshops} datasets.
We perform data augmentation by extracting  patches at $3$ resolutions: HR-images and $\times 2$ and $\times 4$ downscaled images. 
The model is trained for $7.10^5$ iterations with a batch size of $64$. Following the recommendation of~\cite{hazami2022efficient}, we use Adamax optimizer with an exponential moving average and gradient smoothing of the variance.
We set the decoder model to be a gaussian with diagonal covariance, as in~\cite{luhman2022optimizing}.
PatchVDVAE is fully convolutional and can generate images of dimension that are multiples of $64$ as illustrated by
figure~\ref{fig:vdvae}.

\newlength{\patchwidth}
\setlength{\patchwidth}{0.135\columnwidth}
\begin{figure}[!ht]
    \centering
    \begin{subfigure}[t]{.34\columnwidth}\hspace{0.1cm}
        \setlength{\tabcolsep}{0.02pt}
\renewcommand{\arraystretch}{0}
        \begin{tabular}{*{2}{p{1.03\patchwidth}}}
            \includegraphics[width=\patchwidth]{figures_arxiv/patchVDVAE/samples/generated/64x64/setup-5-image-0018.png} &
            \includegraphics[width=\patchwidth]{figures_arxiv/patchVDVAE/samples/generated/64x64/setup-5-image-0016.png} \\
            \includegraphics[width=\patchwidth]{figures_arxiv/patchVDVAE/samples/generated/64x64/setup-5-image-0008.png} &
            \includegraphics[width=\patchwidth]{figures_arxiv/patchVDVAE/samples/generated/64x64/setup-5-image-0019.png}   
        \end{tabular}
    \end{subfigure}\hspace{-0.15cm}
    \begin{subfigure}[t]{.64\columnwidth}
\begin{tabular}{cc}\vspace{-0.1cm}
\includegraphics[width=2\patchwidth]{figures_arxiv/patchVDVAE/samples/generated/256x256/setup-2-image-0009.png}&
        \includegraphics[width=2\patchwidth]{figures_arxiv/patchVDVAE/samples/generated/256x256/setup-2-image-0002.png}\end{tabular}

    \end{subfigure}
    \caption{\label{fig:vdvae} Left: $64\times64$ patches samples from our patchVDVAE model trained on patches from natural images.
    Right: PatchVDVAE is fully convolutional and it can generate images of higher resolution (here: $128\times128$).\vspace{-0.2cm}}
\end{figure}

\subsection{Natural images restoration}\label{ssec:app_nat}
We  evaluate PnP-HVAE on natural image restoration.
For each task, we report the average value of the PSNR, the SSIM, and the LPIPS metrics on $20$ images from the test set of the BSD dataset~\cite{MartinFTM01}.\\


\noindent
{\bf Image deblurring.}
In the experiments, we consider $2$ gaussian kernels and $2$ motion blur kernels from~\cite{levin2009understanding}, with $3$ different noise levels 
$\sigma \in \{2.55, 7.65, 12.75\}$.
As a baseline we consider  EPLL~\cite{zoran2011learning}, which learns a prior on image patches with a gaussian mixture model.
We also compare PnP-HVAE  with PnP-MMO and GS-PnP, $2$ competing convergent Plug-and-Play methods based on CNN denoisers.
PnP-MMO~\cite{pesquet2021learning} restricts the denoiser to be contraction in order to guarantee the convergence of the PnP forward-backard algorithm. GS-PnP~\cite{hurault2022gradient} considers a gradient step denoiser and reaches state-of-the-art performances of non converging methods~\cite{zhang2021plug}.
We set the temperature $\tau$  in our method as $0.95$, $0.8$ and $0.6$ for noise levels $2.55$, $7.65$ and $12.75$ respectively, and we let it run for a maximum of $50$ iterations. 
For the three compared methods we use the official implementations and pre-trained models provided by the respective authors. 
Details on the choice of hyperparameters for the concurrent methods are provided in the Appendix~\ref{app:details}
Figure~\ref{fig:deblurring_bsd} illustrates that our method provides correct deblurring results. 

According to table~\ref{tab:deb}, the performance of PnP-HVAE is between those of EPLL and GS-PnP and it outperforms PnP-MMO for large noise levels.\\

\begin{table}
\begin{center}\footnotesize
    \begin{tabular}{>{\centering}m{.3cm}*{5}{c}}
    $\sigma$ &Method & PSNR$\uparrow$ & SSIM$\uparrow$ & LPIPS$\downarrow$  \\ 
    \hline
    \multirow{4}{*}{\vcell{$2.55$}}
    & PnP-HVAE & $27.75$ & $0.79$ & $0.31$\\
    & GS-PNP \cite{hurault2022gradient} & $\mathbf{29.59}$ & $\mathbf{0.84}$ & $\mathbf{0.22}$\\
    & EPLL \cite{zoran2011learning} & $26.49$ & $0.71$ & $0.36$\\ 
    & PnP-MMO \cite{pesquet2021learning} & $\underbar{29.50}$ & $\underbar{0.83}$ & $\underbar{0.20}$ \\ \hline
    \multirow{4}{*}{\vcell{$7.65$}}
    & PnP-HVAE & $\underbar{26.36}$ & $\underbar{0.72}$ & $\underbar{0.40}$\\
    & GS-PNP \cite{hurault2022gradient} & $\mathbf{27.33}$ & $\mathbf{0.77}$ & $\mathbf{0.31}$\\
    & EPLL \cite{zoran2011learning} & $24.04$ & $0.66$ & $0.45$ \\ 
    & PnP-MMO \cite{pesquet2021learning} & $25.34$ & $0.69$ & $0.34$\\
    \hline
    \multirow{4}{*}{\vcell{$12.75$}}
    & PnP-HVAE & $\underbar{25.12}$ & $\mathbf{0.73}$ & $\underbar{0.47}$\\
    & GS-PNP \cite{hurault2022gradient} & $\mathbf{26.32}$ & $\mathbf{0.73}$ & $\mathbf{0.37}$\\
    & EPLL \cite{zoran2011learning} & $23.28$ & $0.61$ & $0.51$ \\ 
    & PnP-MMO \cite{pesquet2021learning} & $22.42$ & $0.53$& $0.54$ \\
    \hline
    &\vspace*{-.3cm}\\
            \multicolumn{2}{c}{Blur and motion kernels}& \multicolumn{3}{c}{
        \includegraphics*[scale=1]{figures_arxiv/kernels/4.png}\;\includegraphics*[scale=1]{figures_arxiv/kernels/7.png}\;\includegraphics*[scale=1]{figures_arxiv/kernels/9.png}\;\includegraphics*[scale=1]{figures_arxiv/kernels/11.png}} 
    \end{tabular}
        \caption{\label{tab:deb}Comparison  of PnP-HVAE  and other restoration methods on deblurring. Results are averaged on $4$ kernels.\vspace{-0.2cm}}% on image deblurring.}
    \end{center}
\end{table}

\begin{figure}
    
    \begin{subfigure}[h]{\linewidth}
        \centering
        \includegraphics*[width=\columnwidth]{figures_arxiv/deb_s255_k7.pdf}\vspace{-0.1cm}
        \caption{Gaussian blur, $\sigma=2.55$}
    \end{subfigure}
    \begin{subfigure}[h]{\linewidth}
        \centering
        \includegraphics*[width=\columnwidth]{figures_arxiv/deb_s765_k11.pdf}\vspace{-0.1cm}
        \caption{Motion blur, $\sigma=7.65$}
    \end{subfigure}\vspace*{-0.1cm}
    \caption{\label{fig:deblurring_bsd} Natural image deblurring\vspace{-0.1cm}}
\end{figure}

\noindent {\bf Effect of the temperature.}
PnP-HVAE gives control on the temperature of the prior over the latent space.
In figure~\ref{fig:temp_effect}, we illustrate that reducing the temperature increases the strength of the regularization prior. In this example the tuning $\tau=0.7$ produces the best performance.\\
\begin{figure}[!ht]
   
    \includegraphics[width=\columnwidth]{figures_arxiv/demo_temp.pdf}\vspace{-0.15cm}
    \caption{ \label{fig:temp_effect} Effect of the temperature in PnP-VAE on a deblurring problem, with $\sigma=7.65$.\vspace{-0.15cm}}
\end{figure}


\noindent
{\bf Image inpainting.}
Next we consider the task of noisy image inpainting. 
We compose a test-set of 10 images from the validation set of BSD~\cite{MartinFTM01} and we create masks
  by occluding diverse objects of small size in the images. 
A gaussian white noise with $\sigma=3$ is added to the images.
As a comparaison, we still consider GS-PnP and EPLL.
For PnP-HVAE, the temperature is set to $\tau=0.6$, and the algorithm is run for a maximum of $200$ iterations, unless the residual $||\x_{k+1}-\x_k||$ is on a plateau.
We provide on Table~\ref{tab:inpainting_bsd} the distortion metrics with the ground truth, as well as a visual
\begin{table}



\begin{center}
    \begin{tabular}{cccc}
        & PSNR$\uparrow$ & SSIM$\uparrow$ &LPIPS$\downarrow$ \\\hline
        PnP-HVAE  & $\mathbf{29.54}$ & $\mathbf{0.93}$ & $\mathbf{0.06}$\\
        GS-PNP & $28.52$ & $\mathbf{0.93}$ & $0.09$\\
        EPLL & $\underline{29.16}$ & $\mathbf{0.93}$ & $\mathbf{0.06}$\\
    \end{tabular}
    \caption{\label{tab:inpainting_bsd}Quantitative evaluation for inpainting on BSD.}
    \end{center}
\end{table}
comparison on figure~\ref{fig:inpainting_bsd}. 
With its hierarchical structure,  PnP-HVAE outperforms the compared methods. \vspace{0.05cm}



\begin{figure}[!h]
    \includegraphics[width=\columnwidth]{figures_arxiv/demo_inp_bsd2.pdf}\vspace{-0.1cm}
    \caption{\label{fig:inpainting_bsd}Natural image inpainting\vspace{-0.3cm}}
\end{figure}












\subsection{Cross-Dataset Evaluation}

\begin{table}[!htb]
    \centering
    \footnotesize

\begin{tabular}{l|rr|rr}
    \hline
    \multirow{2}{*}{\begin{tabular}[c]{@{}l@{}}\diagbox{Train\quad}{Test}\end{tabular}} & \multicolumn{2}{c|}{LH26M } & \multicolumn{2}{c}{Ours} \\ 

        & MPJPE  & PA-MPJPE & MPJPE  & PA-MPJPE \\ 
        \hline
    LH26M      & \textbf{79.3} & \textbf{67.0} & 228.7 & 149.9 \\
    Ours       & 212.3 & 128.3  & 86.1  & 65.1 \\
    LH26M + Ours & 85.5  & 72.0   & \textbf{79.2}  & \textbf{60.1} \\
    \hline

    \multicolumn{5}{c}{(a) LiDAR-based 3D pose estimation with LiDARCap~\cite{li2022lidarcap}.} \\
    \multicolumn{5}{c}{} \\

    \hline
    \multirow{2}{*}{\begin{tabular}[c]{@{}l@{}}\diagbox{Train\quad}{Test}\end{tabular}} & \multicolumn{2}{c|}{3DPW} & \multicolumn{2}{c}{Ours} \\ 

        & MPJPE & PA-MPJPE & MPJPE  & PA-MPJPE \\ \hline
    VIBE~\cite{kocabas2020vibe} & 93.5  & 56.5 & 102.5 & 66.2 \\
    Ours + AMASS & 124.3 & 66.8 & \textbf{86.6}  & \textbf{52.4} \\
     w. 3DPW     & \textbf{83.0} & \textbf{52.0} & 90.0  & 58.3 \\

    \hline
    HbryIK~\cite{li2021hybrik}  & 88.7  & 49.3  & 104.9 & 57.0          \\
    w. Ours   & 87.3 & 49.2 & 67.6 & 44.2    \\
    w. 3DPW   & \textbf{71.3} & \textbf{41.8} & 75.8 & 50.0 \\
    w. 3DPW + Ours & 76.4 & 46.7 & \textbf{66.2} & \textbf{42.8}  \\


    \hline
    \multicolumn{5}{c}{(b)  Camera-based 3D pose estimation.} \\
    \end{tabular}%
    \vspace{-1mm}
    \caption{Cross-dataset evaluation results with different modalities. The LH26M in (a) refers to LiDARHuman26M dataset from LiDARcap. VIBE is pre-trained on AMASS~\cite{mahmood2019amass}, MPI-INF-3DHP~\cite{mono-3dhp2017} InstaVariety~\cite{humanMotionKZFM19}, PoseTrack~\cite{andriluka2018posetrack}, PennAction~\cite{zhang2013actemes}. HbryIK is pre-trained on H36M, MPI-INF-3DHP, MSCOCO~\cite{lin2014microsoft}}
    \label{tab:cross}
    \vspace{-3mm}
 \end{table}


 
We evaluate root-relative 3D human pose estimation with different modalities, namely the LiDAR and the camera. 3DPW is an in-the-wild human motion dataset that is most related to us. With VIBE, we cross-evaluated our dataset's camera modal by using 3DPW.
LiDARHuman26M is a lidar-based dataset for long-range human pose estimation. We can cross-evaluate our dataset's LiDAR modal with it.
\cref{tab:cross}(a) shows the evaluation results on LiDAR-based 3D pose estimation task and \cref{tab:cross}(b) shows the results on camera-based 3D pose estimation. Taking the results from \cref{tab:cross}(a), for example, when the model is trained from another dataset only,  the errors are the largest.
But the error will be further reduced by around 60\% when training on LiDARHuman26M and our dataset together. 
It suggests a domain gap exists between different LiDAR sensors, and both datasets complement each other.
The results of another task show that the pre-trained VIBE model generalizes better on 3DPW than on our dataset. But the error on 3DPW increases when finetuned on our dataset, while the error decreases on our dataset. This suggests that the pre-trained model complements \TITLE~better than the opposite. Comparing the results across different modalities, the error on our dataset from the method trained on mixed LiDAR point cloud datasets is 13\% lower than the method trained on the images.

\subsection{Benchmark on Global Human Pose Estimation}

\begin{table}[htb]
    \centering
    \footnotesize
    \vspace{-3mm}
\begin{tabular}{llrrrr}
    \toprule
    Scene & Metric & \multicolumn{1}{l}{RMSE $\downarrow$} & \multicolumn{1}{l}{$mean$} & \multicolumn{1}{l}{$std.$} & \multicolumn{1}{l}{$max$} \\
    \midrule
    Football & ATE & 3.26 & 2.85 & 1.58 & 11.83 \\
    Running001 & ATE & 29.48 & 25.55 & 14.72 & 56.07 \\
    Garden001 & ATE & \textbf{2.86} & 2.57 & 1.26 & 6.55 \\
    \midrule
    Football & RPE & 0.08 & 0.06 & 0.05 & 1.34 \\
    Running001 & RPE & 0.40 & 0.35 & 0.19 & 1.04 \\
    Garden001 & RPE & \textbf{0.06} & 0.04 & 0.04 & 0.71 \\
    \bottomrule
    \end{tabular}%
    \vspace{-2mm}

    \caption{Global trajectory evaluation of GLAMR. Unit: $m$.}
    \vspace{-3mm}

    \label{tab:ghpe}
\end{table}

\begin{table}[htb]
    \centering
    \footnotesize
    \vspace{-3mm}
\begin{tabular}{lrrrr}
    \toprule
    Scene & \multicolumn{1}{l}{Scale} & \multicolumn{1}{l}{MPJPE $\downarrow$} & \multicolumn{1}{l}{PA-MPJPE $\downarrow$} & \multicolumn{1}{l}{G-MPJPE $\downarrow$} \\
    \midrule
    Football & 11.83 & 264.6 & 118.5 & 5268.7 \\
    Running001 & 56.07 & 652.1 & 119.6 & 32329.3 \\
    Garden001 & 6.55 & \textbf{139.4} & \textbf{86.3} & \textbf{4407.0} \\
    \bottomrule
    \end{tabular}%
    \vspace{-2mm}
    
    \caption{GHPE results from GLAMR. Unit: $mm$.}
    \vspace{-3mm}
    \label{tab:ghpe2}
\end{table}


\begin{figure}[!htb]
    \centering
     \includegraphics[width=0.98\linewidth]{figures/glamr.pdf}
     \vspace{-2mm}
     \caption{The ATE error mapped on the GT trajectory. The color represents the error according to the color bar.}
     \label{fig:glamr}

    \vspace{-4mm}
\end{figure}












In this subsection, we benchmark the GHPE task of GLAMR~\cite{yuan2022glamr} on \TITLE. GLAMR is a global occlusion-aware method for 3D global human mesh recovery from dynamic monocular cameras. For the scale uncertainty of the monocular camera, we compute the affine matrix from the estimated trajectory to the ground truth trajectory and rotate, translate and scale the estimated trajectory before error computation.

\cref{tab:ghpe} reports the global trajectory error with ATE and RPE, \cref{tab:ghpe2} reports the global human pose metric, and \cref{fig:glamr} shows the ATE error mapped on GT trajectory.
Comparing the results on the three scenes, the \textit{football} and \textit{Garden001} have a significantly lower RPE in the global scene.
In comparison, GLAMR performs the worst on the running scene, with an ATE's RMSE of 29.48 m. This scene has the largest area size and the highest human pace. GLAMR achieves a low PA-MPJPE of 86.3mm on \textit{Garden001}, a sequence with daily walking and visiting motions. It's the first time that we have tested the GPHE on such large outdoor scenes. GLAMR achieves relatively better results on daily human motion while performing worse on high-dynamic activities in the wild. The interesting point is that the trajectory tendency is pretty similar to the reference, even in dynamic football training motions, which demonstrates the ability of GLAMR to be a baseline. It is expected that more research will focus on GHPE in real-world interactive scenarios, and the experiments show our \TITLE's potential to promote urban-level GHPE research.



















\section{Discussions}
\label{sec:discussion}

\PAR{Limitations.} 
Firstly, SLOPER4D is limited to single-person capture though it perceives multiple-person data. Secondly, the camera and LiDAR are not synchronized online, causing tedious offline work if the camera loses frames even with a low time offset (\textless50 $ms$). Finally, texture information from the camera is not fully exploited for color and texture reconstruction of scenes and humans.
In our future work, we will propose an online synchronization algorithm and extend our work to multiple-person capturing. 



\PAR{Conclusions.} 
We propose the first large-scale urban-level human pose dataset with multi-modal capture data and rich human-scene annotations. Based on our proposed new dataset, we benchmark two critical tasks, camera-based 3D HPE and LiDAR-based 3D HPE. SLOPER4D also benchmarks the GHPE task. The results demonstrate the potential of SLOPER4D in boosting the development of these areas. 

Our work contributes to extending motion capture to large global scenes based on the current methods and datasets. We hope this work will foster future creation and interaction in urban environments. 

{Acknowledgements.} We thank Zhiyong Wang for helping us incorporate FAST-LIO2 into our mapping system.
This work was supported in part by the National Natural Science Foundation of China (No.62171393, No.62206173), 
the Fundamental Research Funds for the Central Universities (No.20720220064), 
the open fund of PDL (WDZC20215250113, 2022-KJWPDL-12),
and FuXiaQuan National Independent Innovation Demonstration Zone Collaborative Innovation Platform (No.3502ZCQXT2021003).
 We also acknowledge support from Shanghai Frontiers Science Center of Human-centered Artificial Intelligence (ShangHAI). 

{\small
\bibliographystyle{ieee_fullname}
\bibliography{egbib}
}


\end{document}