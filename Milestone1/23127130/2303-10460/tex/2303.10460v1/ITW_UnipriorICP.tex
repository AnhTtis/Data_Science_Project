\documentclass[conference]{IEEEtran}


\usepackage{amsmath}
\usepackage{amsthm}
\usepackage{amssymb}
\usepackage{graphicx}
\usepackage{subcaption}
\usepackage{url}
\usepackage{epstopdf}
\usepackage{placeins}
\usepackage{epsfig}
\usepackage{bbm}
%\usepackage{calrsfs}
\usepackage{amsfonts}
\usepackage{balance}
\usepackage{graphicx}
\usepackage{xfrac}
\usepackage{mathrsfs}
\usepackage[shortlabels]{enumitem}
\usepackage{algorithm} 
\usepackage{algpseudocode}

%\usepackage{algpseudocode}
\usepackage{footnote}
\usepackage[table]{xcolor}
\usepackage{mathtools,amssymb,lipsum, nccmath}
\usepackage{cuted}
\newcommand{\algorithmicbreak}{\textbf{break}}
\newcommand{\BREAK}{\STATE \algorithmicbreak}
\makeatother
\renewcommand{\algorithmicrequire}{\textbf{Input:}}
\renewcommand{\algorithmicensure}{\textbf{Output:}}
%\usepackage{showframe}
%\usepackage{tikz}
%\usepackage[para,online,flushleft]{threeparttable}
%\usepackage[ruled,vlined]{algorithm2e}
\DeclareMathOperator*{\argmin}{arg\,min}
\DeclareMathOperator*{\argmax}{arg\,max}

\newtheorem{theorem}{Theorem}
\newtheorem{prop}[theorem]{Proposition}
\newtheorem{definition}{Definition}
\newtheorem{lemma}[theorem]{Lemma}
\newtheorem{example}{Example}
\newtheorem{remark}{Remark}
\newtheorem{corollary}[theorem]{Corollary}
\newcommand{\mytilde}{\raise.17ex\hbox{$\scriptstyle\mathtt{‌​\sim}$}}
\setlength{\textfloatsep}{5pt}
\usepackage{multirow}
\DeclareMathOperator{\rank}{rank}
%\renewcommand{\arraystretch}{1.2}

\begin{document}
	\title{Average Probability of Error for Single Uniprior Index Coding  over Rayleigh Fading Channel}
	
	\author{%
		\IEEEauthorblockN{Anjana A Mahesh, Charul Rajput, Bobbadi Rupa, and B. Sundar Rajan\\}
		\IEEEauthorblockA{ Department of Electrical Communication Engineering, Indian Institute of Science, Bengaluru 560012, KA, India \\
			E-mail: \{anjanamahesh,charulrajput,bobbadirupa,bsrajan\}@iisc.ac.in}
	}
	
	
	
	
	%	\markboth{IEEE Transactions on Vehicular Technology,~Vol.~XX, No.~XX, XXX~2019}
	{}
	
	
	
	\maketitle
	
	\begin{abstract} 
		Ong and Ho developed optimal linear index codes for single uniprior index coding problems (ICPs) by finding a spanning tree for each of the strongly connected components of the corresponding information-flow graphs, following which Thomas et al. considered the same class of ICPs over Rayleigh fading channel. They developed the min-max probability of error criterion for choosing an index code which minimized the probability of error at the receivers and showed that there always exist optimal linear index codes for which any receiver takes at most two transmissions to decode a requested message. Motivated by the above works, this paper considers single uniprior ICPs over Rayleigh fading channels for which minimizing average probability of error is shown to be a criterion for further selection of index codes. The optimal index  code w.r.t this criterion  is shown to be  one that minimizes the total number of transmissions used for decoding the message requests at all the receivers. An algorithm that generates a spanning tree which has a lower value of this metric as compared to the optimal star graph is also presented. For a given set of parameters of  single uniprior ICPs, a  lower bound for the total number of transmissions used by any optimal index code is derived, and a class of ICPs for which this bound is tight is identified.  
	\end{abstract}
	
%	\begin{IEEEkeywords}
%		Uniprior Index Coding, Spanning tree, Star graph, Rayleigh fading
%	\end{IEEEkeywords}
	
	\IEEEpeerreviewmaketitle
	
	%%%%%%%%%%%%%%%%%%%%%%%%%%%%%%%%%%%%%%%%%%%%%%%%%%%%%%%%%%%%%%%%%%%%%%%%%
	\section{INTRODUCTION}
	\label{sec:Intro}
	
	\IEEEPARstart{I}ntroduced by Birk and Kol in \cite{BiK}, an index coding problem (ICP) has a server with access to a library of messages broadcasting to a set of caching receivers. Each  receiver has a subset of these messages available in its cache and  requests a non-intersecting subset of messages from the server. Index coding aims to satisfy the message requests of all the receivers with a minimum number of server transmissions by utilizing the information of receivers' cached contents. A solution of an ICP, which is a set of transmissions from the server that satisfies all the receivers' message requests, is called an index code, and the number of transmissions in it is called its length. An index code is said to be linear if all the transmissions  are linear combinations of the messages and optimal if it has the minimum possible number of transmissions.  
	
	
	Ong and Ho in \cite{OngHo} introduced a class of single uniprior ICPs, where each receiver knows a unique message and demands a subset of the messages available at the server. This class of ICPs was represented graphically using directed graphs called information-flow graphs in \cite{OngHo}. An algorithm which takes this graph as input and generates its strongly connected components (SCCs), called the pruning algorithm was given, and linear index codes for each of these SCCs were also developed based on their spanning trees. Further, it was shown in \cite{OngHo} that linear index codes are optimal for this class of ICPs.  
	
    A noisy version of single uniprior ICPs over Rayleigh fading channels was explored in \cite{TRAR}.  In this setting, different bandwidth-optimal index codes were found to give different probability of error performances at the receivers, and  \cite{TRAR} introduced the min-max probability of error criterion, which minimized the maximum number of transmissions used by any receiver to decode a single requested message. Similar to \cite{TRAR}, we also consider single uniprior ICPs over Rayleigh fading channels in this paper, for which we make the following technical contributions. 	
	\begin{itemize}
		\item Minimizing the average probability of error over all the message requests is introduced as a criterion to choose an index code from the set of bandwidth-optimal index codes satisfying the min-max probability of error criterion.
		\item We prove that minimizing the average probability of error is equivalent to minimizing the total number of transmissions used for decoding the message requests across all the receivers.
		\item For a given SCC of an information-flow graph, a condition for choosing an optimal spanning tree of diameter two is derived. 
		\item An algorithm to generate a spanning tree which improves upon the optimal star graph w.r.t average probability of error while keeping the maximum number of transmissions used for decoding any requested message as two, called Algorithm 1, is presented. 
		\item A lower bound for the total number of transmissions required for decoding the message requests is derived. 
		\item A class of information-flow graphs for which Algorithm 1 gives optimal spanning trees is  identified.
		
	\end{itemize}
	
	
	The rest of this paper is organized as follows. In Section \ref{sec:Sys}, the single uniprior index coding setting considered in this paper as well as  relevant existing results, are explained. A criterion to choose an index code from the class of bandwidth optimal index codes satisfying min-max probability of error criterion in \cite{TRAR} is developed in the following Section \ref{sec:AvgPe}. For a chosen SCC of the information-flow graph, how to choose a spanning tree which gives an index code that minimizes the total number of transmissions used in decoding is discussed in Section \ref{sec:Algo}. A lower bound is  derived for the minimum value of the total number of transmissions used in decoding that can be attained by an optimal index code, and a family of information-flow graphs is identified  for which the lower bound is tight in Section \ref{sec:Results}.  Concluding remarks are given in Section \ref{sec:Conc}. 
	
	
	\emph{Notations}: The binary field consisting of the elements $0$ and $1$ is denoted as $\mathbb{F}_2$. For a positive integer $n$, $[n]$ denotes the set $\{1,2,\cdots,n\}$. The set of positive integers is denoted by $\mathbb{Z}^+$. 
	For a graph $\mathcal{G}$, $\mathcal{V}(\mathcal{G})$  denotes the set of vertices in $\mathcal{G}$ and $\mathcal{E}(\mathcal{G})$ denotes the set of arcs(edges) in a directed(undirected) graph $\mathcal{G}$. For a vertex $v \in  \mathcal{V}(\mathcal{G})$, its degree and the set of its neighbors in $\mathcal{G}$ are denoted as $\deg_{\mathcal{G}}(v)$ and $N_{\mathcal{G}}(v)$,  respectively. In a directed graph $\mathcal{G}$, for a pair of vertices $u$ and $v \in  \mathcal{V}(\mathcal{G})$, a double arc is said to exist between $u$ and $v$, if both the arcs $(u,v)$ and $(v,u)$ are present in $\mathcal{E}(\mathcal{G})$. For a given graph $\mathcal{G}$,  $\Delta(\mathcal{G})$ is used to denote the maximum degree of a vertex in $\mathcal{V}(\mathcal{G})$, i.e., $\Delta(\mathcal{G}) = \max\limits_{v \in \mathcal{V}(\mathcal{G})}\deg_{\mathcal{G}}(v)$. 

	%%%%%%%%%%%%%%%%%%%%%%%%%%%%%%%%%%%%%%%%%%%%%%%%%%%%%%%%%%%%%%%%%%%%%%%%%%%%%%%%%%%%%%%%%%%%%%%%%%%%
	
	\section{System Model \& Preliminaries}
	\label{sec:Sys}
	
	We consider single uniprior ICPs with the central server accessing a library  $\mathcal{X} = \{x_1,x_2,\cdots, x_n\}$  of $n$ messages, $x_i \in \mathbb{F}_2$, and transmitting to a set of $n$ receivers $\mathcal{R} = \{R_1,R_2,\cdots, R_n\}$. Each of the receivers knows a unique message as side information. Without loss of generality, let us assume that the receiver $R_i$ knows the message $x_i$ \emph{a priori} and demands a subset $\mathcal{W}_i$ of $\mathcal{X} \setminus \{x_i\}$. This single uniprior ICP is represented as $\mathcal{I}(n,\mathcal{W})$, where, $\mathcal{W} \triangleq \{\mathcal{W}_1,\mathcal{W}_2,\cdots,\mathcal{W}_n\}$,. 
	
	
	\begin{definition}\cite{OngHo}
		A single uniprior ICP $\mathcal{I}(n,\mathcal{W})$ is represented using a directed graph called an information-flow graph $\mathcal{G} = (\mathcal{V},\mathcal{E})$, where the vertex set represents the set of receivers, $\mathcal{V} = \{1,2,\cdots,n\}$ and there is an edge from $i$ to $j$ if $R_j$ demands $x_i$, i.e., $\mathcal{E} = \{(i,j) : x_i \in \mathcal{W}_j\}$. 
	\end{definition}
	

	
	The encoding scheme for a linear index code of length $N$ can be represented using an $n \times N$ matrix, $\mathbf{L}$, over $\mathbb{F}_2$. Let the index coded vector be represented as $\vec{c} = (c_1,c_2,\cdots,c_N) \in \mathbb{F}_2^N$ such that for a message realization $\vec{x} \in \mathbb{F}_2^n$, $\vec{c} = \vec{x}\mathbf{L}$. We assume that the server sends each encoded bit separately after binary modulation over a Rayleigh fading channel and that each of the receivers independently estimates the binary-modulated transmitted symbols and performs index decoding to retrieve their requested message bits. Let the binary modulated symbols corresponding to the index codeword $(c_1,c_2,\cdots,c_N)$ be represented as $(s_1,s_2,\cdots,s_N)$. 


At a receiver $R_j$, each binary symbol can be assumed to be transmitted over a binary symmetric channel with probability of error, $p_j$, where $p_j$ is determined by the fade characteristics of the channel between the source and receiver $R_j$. Since the channels from the source to the different receivers are assumed to be independently and identically distributed, their fade characteristics are identical, and hence it is assumed that $p_1 = p_2 = \cdots = p_n = p$. For a Rayleigh fading channel for which the channel coefficient $h$ is distributed as $\mathcal{CN}(0,1)$, $p = \mathbb{E}_{|h|^2}\left[Q\left(\sqrt{2|h|^2SNR}\right)\right]$, where $SNR$ is the signal to noise ratio measured at a receiver, $Q(x)$ is the tail probability of standard normal distribution, and the notation $\mathbb{E}_{y}(x)$  denotes the expected value of $x$ w.r.t $y$


The min-max probability of error criterion of choosing a bandwidth-optimal index code was derived in \cite{TRAR} which minimizes the maximum number of transmissions used in decoding a requested message at any receiver. The paper \cite{TRAR} also gave an algorithm (Algorithm 2 in \cite{TRAR}) to generate bandwidth-optimal index codes satisfying min-max probability of error criterion  for every strongly connected component (SCC) of the information-flow graph, $\mathcal{G}$. For an SCC  $\mathcal{G}_{\text{sub},i}$ on $n_i$ vertices, this optimal index code was obtained by coding along the edges of a spanning tree of the complete graph $K_{n_i}$ which minimized the maximum distance between any two vertices connected by an arc in $\mathcal{G}_{\text{sub},i}$. 
%	
	For a given ICP $\mathcal{I}(n,\mathcal{W})$ and a chosen index code $\mathcal{C}$, let the maximum number of transmissions used at any receiver to decode a single requested message be denoted as  $l_{max}(\mathcal{C})$. In \cite{TRAR}, it was shown that for every single uniprior ICP, it is always possible to find an optimal index code $\mathcal{C}$ such that  $l_{max}(\mathcal{C})$ is two. 
	
	
	
	\section{Average Probability of Error}
	\label{sec:AvgPe}
	
	For a spanning tree $\mathscr{T}$, an index code $\mathcal{C}_{\mathscr{T}}$ is said to be obtained by coding along the edges of $\mathscr{T}$, if $\mathcal{C}_{\mathscr{T}} =  \{x_i+x_j, \ \forall (i,j) \in \mathcal{E}(\mathscr{T})\}$. Cayley's formula \cite{Cay} gives the number of spanning trees for a labeled complete graph $K_m$ on $m$ nodes to be $m^{m-2}$. Among these spanning trees, consider the $m$ star graphs with vertex  $i$ as head for each $i \in [m]$.  Since any spanning tree on $m$ nodes has $m-1$ edges, the index code obtained by coding along its edges is of length $m-1$, and they are all optimal in terms of bandwidth occupied. Further, for any spanning tree with diameter two, since the maximum number of transmissions used by any receiver is two, they are also equivalent with respect to the min-max probability of error criterion. Hence, to further select a code from among the bandwidth-optimal index codes satisfying the min-max probability of error criterion, we choose the average probability of error across all message demands at all the receivers as a new criterion.
	
	For a single uniprior ICP, $\mathcal{I}(n,\mathcal{W})$ represented by its information-flow graph $\mathcal{G} = (\mathcal{V},\mathcal{E})$ and for a chosen index code $\mathcal{C}$, the average probability of error is defined as
	$$P_{\text{avg},\mathcal{C}} = \frac{1}{E}\sum\limits_{i \in [n]}\sum\limits_{x_j \in \mathcal{W}_i}Pr(x_j^i \neq x_j),$$ where $E = |\mathcal{E}|$, is the total number of arcs in  $\mathcal{G}$ which is same as the total number of demands in the problem, and $x_j^i$ is the estimate of the requested message bit $x_j$ at the receiver $R_i$. Since we are trying to optimize the performance from among the bandwidth-optimal index codes that require at most two transmissions to decode a message at any receiver,  the probability of error in estimating the requested message bit $x_j$ at the receiver $R_i$ is $Pr(x_j^i \neq x_j) = l_j^i \ p(1-p)^{l_j^i-1}$, where $l_j^i$ is the number of transmissions used to estimate $x_j$ at $R_i$ which takes a value in $\{1,2\}$. Therefore, 
	
	\begin{align*}
		P_{\text{avg},\mathcal{C}} &= \frac{1}{E}\sum\limits_{i \in [n]}\sum\limits_{x_j \in \mathcal{W}_i}\left(l_j^i \ p(1-p)^{l_j^i-1}\right) \\
		&= \frac{1}{E}\left(\sum\limits_{i \in [n]}\sum\limits_{x_j \in \mathcal{W}_i, \  l_j^i = 1 }p + \sum\limits_{i \in [n]}\sum\limits_{x_j \in \mathcal{W}_i ,\ l_j^i = 2}2p(1-p)\right)	 	 
	\end{align*}
	
	Let the number of demands in $\mathcal{I}(n,\mathcal{W})$, which require one transmission each to decode, be denoted as $t$, which implies that there are $(E-t)$ demands each of which requires two transmissions. With this, $P_{\text{avg},\mathcal{C}}$ further reduces to 
	
	\begin{align*}
		P_{\text{avg},\mathcal{C}} &= tp + 2(E-t)p(1-p) \\
		&= 2Ep(1-p) - \underbrace{tp\big(2(1-p) - 1\big)}_{\text{Term 2}}
	\end{align*}
	
	For all channels over which information exchange is possible, $p <0.5$, and hence $(1-2p)$ is strictly greater than zero, and hence to reduce the average probability of error, $t$ needs to be increased so as to increase term 2 above. Since the total number of transmissions used to decode message requests across all receivers is given as $T= 2E-t$, increasing $t$ is equivalent to reducing the total number of transmissions used $T$. Hence, in this paper, we give index codes which minimize the total number of transmissions used from among the class of bandwidth-optimal index codes satisfying the min-max probability of error criterion. 
	
	%%%%%%%%%%%%%%%%%%%%%%%%%%%%%%%%%%%%%%%%%%%%%%%%%%%%%%%%%%%%%%%%%%%%%%%%%%%%%%%%%%%%%%%%%%%%%%%%%%%%%%%%%%%%%%%%%%%%	 
	\section{Choosing a spanning tree which minimizes the total number of transmissions used}
	\label{sec:Algo}
	
	For a single uniprior ICP represented by the information-flow graph $\mathcal{G}$, since the bandwidth-optimal index code in \cite{OngHo} gives a separate code for each strongly connected component of $\mathcal{G}$, in the rest of this paper, we consider all information-flow graphs to be strongly connected. Further, for an index code $\mathcal{C}_{\mathscr{T}}$ obtained from a spanning tree $\mathscr{T}$, both the notations $l_{max}(\mathcal{C}_{\mathscr{T}})$ as well as $l_{max}(\mathscr{T})$ are used interchangeably to mean the maximum number of transmissions used to decode a single requested message at any receiver while using the index code $\mathcal{C}_{\mathscr{T}}$.
	\begin{example}
		\label{ex:star}
		Consider the single uniprior ICP represented by the information-flow graph, $\mathcal{G}$ shown in Fig. \ref{fig:Ex_star}. For this graph, Algorithm 2 in \cite{TRAR} forms the connected graph on $4$ labeled nodes and finds a spanning tree of diameter two. There are four possible spanning trees of diameter two for a labeled $K_4$, which are shown in Fig. \ref{fig:Ex1_ST}. Corresponding to each of these spanning trees, the bandwidth-optimal index codes satisfying the min-max probability of error criterion are given in Table \ref{Tab:Codes_Ex_star}. The decoding at each of the $4$ receivers and the total number of transmissions used, $T$ in each of these codes, is shown in Table \ref{Tab:Txns_Ex_star}. From Table \ref{Tab:Txns_Ex_star}, it can be seen that even though all four codes are bandwidth-optimal as well as satisfy the min-max probability of error criterion with $l_{max} = 2$, code $\mathcal{C}_2 $ gives the best average probability of error performance.
		\begin{figure} 		
			\centering
			\scalebox{0.45}{\includegraphics{Figures/Ex1.eps}}
			\caption{Information-flow graph in Example \ref{ex:star}.}
			\label{fig:Ex_star}
		\end{figure}
		\begin{figure} 		
		\centering
		\scalebox{0.45}{\includegraphics{Figures/Ex1_ST.eps}}
		\caption{Spanning trees of diameter two for Example \ref{ex:star}.}
		\label{fig:Ex1_ST}
	\end{figure}
			\begin{table}
			
			\centering
			\begin{tabular}{|c|c|c|c|c|}
				\hline 
				& Code $\mathcal{C}_1$ & Code $\mathcal{C}_2$ & Code $\mathcal{C}_3$ & Code $\mathcal{C}_4$ \\ 
				\hline 
				$c_1$& $x_1+x_2$ & $x_2+x_1$ &$x_3+x_1$  &$x_4+x_1$  \\ 
				\hline 
				$c_2$& $x_1+x_3$ & $x_2+x_3$ &$x_3+x_2$  &$x_4+x_2$  \\ 
				\hline 
				$c_3$& $x_1+x_4$ & $x_2+x_4$ &$x_3+x_4$  &$x_4+x_3$  \\ 
				\hline 
			\end{tabular} 
			\caption{Optimal Index Codes from  \cite{TRAR} for Example \ref{ex:star}.}
			
			\label{Tab:Codes_Ex_star}
		\end{table}
		
		\begin{table}
			\centering
			\tiny
			\begin{tabular}{|c|c|c|c|c|c|}
				
				\hline 
				Rx& $\mathcal{W}_i$ & Code $\mathcal{C}_1$ & Code $\mathcal{C}_2$ & Code $\mathcal{C}_3$ & Code $\mathcal{C}_4$ \\ 
				\hline 
				$R_1$& $x_4$ & $x_1 +c_3$ & $x_1 +c_1+c_3$  & $x_1 +c_1+c_3$ & $x_1 +c_1$ \\ 
				\hline 
				$R_1$& $x_2$ & $x_1+c_1$ & $x_1 +c_1$ & $x_1 +c_1+c_2$ &  $x_1 +c_1+c_2$\\ 
				\hline 
				$R_2$& $x_1$ & $x_2+c_1$ & $x_2 +c_1$ & $x_2 +c_1+c_2$ & $x_2 +c_1+c_2$  \\ 
				\hline 
				$R_2$& $x_3$ & $x_2+c_1+c_2$ & $x_2 +c_2$ & $x_2 +c_2$ & $x_2 +c_2 +c_3$ \\ 
				\hline 
				$R_3$& $x_2$ & $x_3+c_1+c_2$ & $x_3 +c_2$ & $x_3 +c_2$ & $x_3 +c_2+ c_3$ \\ 
				\hline 
				$R_4$& $x_3$ & $x_4+c_2+c_3$ & $x_4+c_2+c_3$  & $x_4 +c_3$ & $x_4 +c_3$ \\ 
				\hline 
				$T$&  & $9$ &  $8$ &  $9$ &  $10$ \\ 
				\hline 
			\end{tabular} 
			\caption{Total number of transmissions required to decode for each code in Example \ref{ex:star}.}
			
			\label{Tab:Txns_Ex_star}
		\end{table}
	\end{example}
	
	\subsection{Optimal Star Graph}
	For a complete graph $K_m$, the only spanning trees with diameter two are the $m$ star graphs, each with a different vertex as the head. Since, from the example above, we saw that not all star graphs perform equally in terms of the average probability of error, we give a criterion for choosing the best star graph.  Let the star graph with vertex $j$ as the head be denoted as $\mathcal{G}^*_{j}$.
	
	\begin{prop}
		\label{Prop_star}
		Consider a single uniprior ICP represented by its information-flow graph $\mathcal{G}$ on $n$ vertices.  For an index code based on the star graph $\mathcal{G}^*_{j}$, the total number of transmissions used is $T = 2|\mathcal{E}(\mathcal{G})| - \deg_{\mathcal{G}}(j)$.
		
		\begin{proof}
			For the single ICP represented by $\mathcal{G}$, the number of arcs in $\mathcal{G}$ given by  $|\mathcal{E}(\mathcal{G})|$ is equal to the number of demands in the ICP. The transmissions in the index code obtained by coding along the edges of a star graph $\mathcal{G}^*_{j}$ are of the form $x_j + x_k$, for all $x_k \in \mathcal{V}(\mathcal{G}) \setminus \{x_j\}$. Hence the demands that take a single transmission to decode are either when $x_j$ is demanded, represented by outgoing arcs from vertex $j$, or those demanded by the receiver $R_j$, which are represented by incoming arcs to vertex $j$. Hence the total number of demands taking a single transmission to decode is equal to the sum of in-degree$(j)$ and out-degree$(j)$, which is equal to the degree of $j$ in  $\mathcal{G}$. Every other demand requires two transmissions to decode. 
		\end{proof}
		
	\end{prop}
	
	\begin{corollary}
		For a single uniprior ICP represented by its information-flow graph $\mathcal{G}$, the star graph which minimizes the average probability of error is the one with vertex $j$ as head where $j \in \argmax\limits_{ v \in \mathcal{V}(\mathcal{G})}(\deg_{\mathcal{G}}(v))$.
	\end{corollary}
	
	For the information-flow graph in Fig. \ref{fig:Ex_star}, vertex $2$ has the maximum degree, and hence $\mathcal{G}_2^*$ will use the minimum number of transmissions for decoding, which can be verified from Table \ref{Tab:Txns_Ex_star}. Hence $\mathcal{G}_2^*$ will give the minimum average probability of error. In the following subsection, we give an algorithm to generate a spanning tree which improves upon the star graph in terms of the total number of transmissions used. 

	
	\subsection{Improving the Optimal Star Graph}
	\begin{example}
		\label{ex:Alg1}
		For the information-flow graph, $\mathcal{G}$ in Fig. \ref{fig:Ex2_Alg1}(a), vertex $3$ has the maximum degree, and hence among the $5$ star graphs, $\mathcal{G}_3^*$, shown in Fig. \ref{fig:Ex2_Alg1}(b) will give the best average probability of error. The total number of transmissions used in decoding the requested messages when the index code based on $\mathcal{G}_3^*$ is transmitted is $T = 13$. Now consider the tree  $\mathscr{T}$ shown in Fig. \ref{fig:Ex2_Alg1}(c). For this tree, the total number of transmissions used is $T=10$. Since there are $9$ arcs in $\mathcal{G}$ is $9$, $T$ is at least $9$. However, it can be verified that there is no index code of length $N=4$ for which $T=9$. Hence the tree in Fig. \ref{fig:Ex2_Alg1}(c) is optimal w.r.t average probability of error.
	\end{example}
	
	
	\begin{figure} [H]		
		\centering
		\scalebox{0.45}{\includegraphics{Figures/Ex2.eps}}
		\caption{(a) Information-flow graph $\mathcal{G}$, (b) Optimal Star Graph $\mathcal{G}^*_{3}$, and (c) Optimal Spanning Tree $\mathscr{T}$ for Example \ref{ex:Alg1}.}
		\label{fig:Ex2_Alg1}
	\end{figure}
	
	The following two modifications were done on  $\mathcal{G}^*_{3}$ in Fig. \ref{fig:Ex2_Alg1}(b) to obtain the optimal spanning tree in Fig. \ref{fig:Ex2_Alg1}(c).
	\begin{itemize}
		\item Removal of the edge $(3,2)$  and addition of $(1,2)$ in $\mathcal{G}^*_{3}$. \label{op1}
		\item Removal of the edge $(3,5)$  and addition of  $(4,5)$ in $\mathcal{G}^*_{3}$. \label{op2}
	\end{itemize}
	
	In $\mathcal{G}^*_{3}$, $(3,2)$ is an edge of the tree and hence the demand $(3,2) \in \mathcal{E}(\mathcal{G})$  takes one transmission to decode in $\mathcal{C}_{\mathcal{G}^*_{3}}$ whereas, $(3,2)$ is no longer an edge of  $\mathscr{T}$ and hence the demand $(3,2) \in \mathcal{E}(\mathcal{G})$ requires two transmissions to decode. But $(1,2)$ is now an edge in the tree, and hence each of the two demands $(1,2)$ and $(2,1) \in \mathcal{E}(\mathcal{G})$ takes one transmission each as opposed to taking two each in $\mathcal{C}_{\mathcal{G}^*_{3}}$. Hence, the modification $\big\{\mathcal{E}(\mathcal{G}^*_3) \setminus \{(3,2)\} \big\}\cup \{(1,2)\}$ gives a net reduction of 1 in the total number of transmissions used. Now consider the  operation  $\big\{\mathcal{E}(\mathcal{G}^*_3) \setminus \{(3,5)\} \big\}\cup \{(4,5)\}$.  Here, the removal of the edge $(3,5)$ from $\mathcal{G}^*_{3}$ does not affect the total number of transmissions used as neither $(3,5)$ nor $(5,3)$ is an edge in $\mathcal{G}$ but the addition of the edge $(4,5)$ decreases the number of transmissions needed to decode each of the demands $(4,5)$ and $(5,4)$ in $\mathcal{E}(\mathcal{G})$ from two to one. Hence, this operation gives an overall reduction of two to the total number of transmissions used. 
	
	In a directed graph $\mathcal{G} = (\mathcal{V}, \mathcal{E})$, for a pair of vertices $u, \ v \in \mathcal{V}$, the parameter $conn(u,v)$ which is used to denote the number of arcs between $u$ and $v$ is defined as 
	$$conn(u,v) = \begin{cases}
		1, \text{ if  } (u,v) \in \mathcal{E} \text{ or } (v,u) \in \mathcal{E} \text{ but not both}, \\
		2, \text{ if both  }(u,v) \in \mathcal{E} \text{ and } (v,u) \in \mathcal{E}, \\
		0, \text{ otherwise. }
	\end{cases}$$
	
	From Example \ref{ex:Alg1}, it can be seen that for a given information-flow graph, $\mathcal{G}$, there is scope for improving $\mathcal{G}^*_{j}$ if there exists at least one vertex $k \in \mathcal{V}(\mathcal{G})$ such that, either
	\begin{enumerate}
		\item $k \notin N_{\mathcal{G}}(j)$ and $N_{\mathcal{G}}(k) = \{l\}$, or \label{cond1}
		\item $k \in N_{\mathcal{G}}(j)$ with $conn(k,j) = 1$ and $N_{\mathcal{G}}(k) \setminus \{j\} = \{l\}$ with $conn(k,l)  = 2$. \label{cond2}
	\end{enumerate} 
	
	The improvement is obtained by removing the edge $(j,k)$ from $\mathcal{G}^*_{j}$ and adding the edge $(l,k)$.
	As we saw from Proposition \ref{Prop_star}, for an index code obtained from  $\mathcal{G}^*_{j}$, the number of demands that require one transmission is $\deg_{\mathcal{G}}{j}$. For a vertex $j \in \mathcal{V}(\mathcal{G})$, the parameter $adv(j)$ is defined as the advantage it gives, i.e., the number of demands which takes one transmission to decode when the transmitted index code is based on a tree obtained by modifying the star graph $\mathcal{G}^*_{j}$ as $\mathcal{G}^*_{j} \cup \{(l,k)\} \setminus \{(j,k)\} $ for each vertex $k$ satisfying condition \ref{cond1}) or \ref{cond2}) above.
	
	\begin{definition}
		\label{defn:adv}
		Given an information-flow graph $\mathcal{G}$, for a vertex $j \in \mathcal{V}(\mathcal{G})$, its advantage $adv(j)$ is defined as $adv(j) = \deg_{\mathcal{G}}(j) + p_j + 2o_j$, where, \begin{itemize}
			\item $\mathcal{P}_{j}  = \{k \in N_{\mathcal{G}}(j) \text{ s.t } conn(j,k) = 1, \ N_{\mathcal{G}}(k) = \{j,l\} \text{ and } conn(k,l) = 2\}$, 
			\item $p_j = |\mathcal{P}_{j}|- \frac{1}{2}|\{(k,l) \text{ s.t } k,l \in \mathcal{P}_j \}$, and
			\item $o_j = |\{k  \notin N_{\mathcal{G}}(j) \text{ s.t } N_{\mathcal{G}}(k) = \{l\} \}|$. 
			
		\end{itemize}
	\end{definition}
	
	For a demand corresponding to an arc $(i,j) \in \mathcal{E}(\mathcal{G})$, to use one transmission to decode in an index code based on a tree $\mathscr{T}$, it should be an edge in $\mathscr{T}$. Hence the total number of demands which take a single transmission each to decode is equal to the number of arcs $(i,j)$ in $\mathcal{G}$ such that the edge $(i,j)$ is present in $\mathscr{T}$. We define the set $\mathcal{S}_{\mathscr{T}}$ as $\mathcal{S}_{\mathscr{T}} \triangleq \{(i,j) \in \mathcal{E}(\mathcal{G}) \text{ s.t } (i,j) \in \mathcal{E}(\mathscr{T}) \}$. Hence the number of demands which take one transmission each to decode for the index code based on $\mathscr{T}$ is  $|\mathcal{S}_{\mathscr{T}}|$ which implies that the total number of transmissions used for decoding is $T = 2|\mathcal{E}(\mathcal{G})| -  |\mathcal{S}_{\mathscr{T}}|$. With the notations in place, we propose the following algorithm. 
	

	
	\begin{algorithm}
		\caption{Generate a spanning tree which improves upon the optimal star graph.}
		
		\begin{algorithmic}[1]
			
			\Require Information flow graph, $\mathcal{G} = (\mathcal{V},\mathcal{E})$
			\Ensure Tree $\mathscr{T}$
			\State Determine the set $\mathcal{A} := \argmax\limits_{v \in \mathcal{V}}(adv(v))$. 
			\State Compute  $\mathcal{A}_{\Delta} := \argmax\limits_{v \in \mathcal{A}}(\deg_{\mathcal{G}}(v))$. 
			\State Pick an $i$ from $\mathcal{A}_{\Delta}$.
			\State $\mathscr{T} \leftarrow \mathcal{G}^*_{i}$. 
			\If{$\Delta(\mathcal{G}) < adv(i)$}
			
			\State $\mathcal{V}^{'}= \mathcal{V} \setminus \{i\}$.
			\While {  $\exists \ j \in \mathcal{V}^{'} $ s.t $N_{\mathcal{G}}(j) \setminus \{i\}= \{k\}$ and $conn(j,k) > conn(i,j)$} 
			\State $\mathcal{E}(\mathscr{T}) = \big\{\mathcal{E}(\mathscr{T}) \setminus \{(i,j)\} \big\}  \cup \{(k,j)\} $.
			\State $\mathcal{V}^{'} = \mathcal{V}^{'} \setminus \{j,k\}$.
			\EndWhile
			\EndIf	
		\end{algorithmic}
		\label{alg:1}
	\end{algorithm}
	
	\begin{figure} 		
		\centering
		\scalebox{0.45}{\includegraphics{Figures/Alg1.eps}}
		\caption{Proposed modification to star graph in Algorithm \ref{alg:1}.}
		\label{fig:Alg1}
	\end{figure}
	
	\begin{lemma}
		\label{lem_diam}
		For the index code obtained from the tree $\mathscr{T}$ returned by Algorithm \ref{alg:1}, the maximum number of transmissions used by any receiver to decode a single demand is at most 2.
		\begin{proof}
			For an information-flow graph $\mathcal{G} = (\mathcal{V},\mathcal{E})$, Algorithm \ref{alg:1} starts with $\mathscr{T} = \mathcal{G}^*_{i}$, with $i$ being a vertex which gives maximum advantage. For the index code based on $\mathcal{G}^*_{i}$, the maximum number of transmissions used by any receiver to decode a requested message, $l_{max}(\mathcal{G}^*_{i})$ is two. It needs to be proved that $l_{max}$ is not increased by the modifications to  $\mathscr{T}$. 
			
			Every modification is of the form $\mathscr{T} \cup (k,j) \setminus (i,j)$, for a vertex $j \in \mathcal{V} \setminus \{i\}$ which satisfies the condition that $N_{\mathcal{G}}(j) \setminus \{i\} = \{k\}$ and $conn(j,k) > conn(i,j)$. The operation $\mathscr{T} \cup (k,j) \setminus (i,j)$ moves the vertex $j$ from Level 1 to Level 2, as shown in Fig. \ref{fig:Alg1}, due to which the demands in $\mathcal{G}$ that can require more than two transmissions to decode  are of the form $(l,j)$ or  $(j,l) \in \mathcal{E}$ for $l \in \mathcal{V} \setminus \{i,k\}$. Since $N_{\mathcal{G}}(j) \setminus \{i\} = \{k\}$, $N_{\mathcal{G}}(j) \subseteq \{i,k\}$. Hence no such demand exists in $\mathcal{G}$, which can require more than two transmissions to decode. 
			
			
		\end{proof}
	\end{lemma}
	
	
	\begin{lemma}
		\label{lem_T}
		For the index code obtained from the tree $\mathscr{T}$ returned by Algorithm \ref{alg:1}, the total number of transmissions used in decoding is $T = 2|\mathcal{E}(\mathcal{G})| - adv(i)$.
%		\begin{proof}
%			Given in \cite{arX}.
%		\end{proof}
		
		\begin{proof}
			From Lemma \ref{lem_diam}, we know that each demand needs a maximum of two transmissions to decode. The demand corresponding to an arc $(j, k) \in \mathcal{E}(\mathcal{G})$ needs only one transmission to decode if and only if $(j, k) \in \mathcal{E}(\mathscr{T})$.  Therefore, we have $T = 2|\mathcal{E}(\mathcal{G})|- |\mathcal{S}_{\mathscr{T}}|$.  Now we need to prove that $|\mathcal{S}_{\mathscr{T}}|=adv(i)$ for the tree $\mathscr{T}$ obtained from Algorithm \ref{alg:1}. It starts with a star graph $\mathcal{G}^*_i$, for which $|\mathcal{S}_{\mathcal{G}^*_i }| = \deg_{\mathcal{G}}(i)$. All the operations performed by Algorithm \ref{alg:1} are of the form $\big\{\mathscr{T}  \setminus (i, j)\big\} \cup (k, j)$,  $\forall j \in V \setminus \{i\}$ which satisfy the condition that $N_{\mathcal{G}}(j)\setminus \{i\} = \{k\}$ and $conn(j, k) > conn(i, j)$. This condition is equivalent to the following two cases. 
			
			\emph{Case 1}: $j \notin N_{\mathcal{G}}(i)$ and $N_{\mathcal{G}}(j) = \{k\}$ - Since $j$ has only one neighbor $k$ in $\mathcal{G}$, removing  the edge $(i,j)$ from $\mathscr{T}$ will not affect the number of transmissions used, whereas, by adding  the edge $(k, j)$ to $\mathscr{T}$, $|\mathcal{S}_{\mathscr{T}}|$ will increase by $2$ as both $(j, k)$ as well as $(k, j) \in \mathcal{E}(\mathcal{G})$ (since $\mathcal{G}$ is an SCG). The number of vertices satisfying this condition is represented as $o_i$ in Definition 2.
			
			\emph{Case 2}: $N_{\mathcal{G}}(j)=\{i,k\},\ conn(i,j)= 1 \text{ and } conn(j, k) =2$ - Clearly, by removing edge $(i,j)$ from $\mathscr{T}$,  $|\mathcal{S}_{\mathscr{T}}|$ will decrease by $1$ as there is an arc between the vertices $i$ and $j$. However, the addition of the edge $(k,j)$ to $\mathscr{T}$ will increase $|\mathcal{S}_{\mathscr{T}}|$ by $2$ as there are two arcs between $k$ and $j$. Hence, a vertex $j$ satisfying the condition in this case  gives a net increment of $1$ in $|\mathcal{S}_{\mathscr{T}}|$. Again, in Definition 2, the number of vertices satisfying this case is represented by $p_i$. 
			
			The while loop in Algorithm \ref{alg:1} finds all vertices satisfying either of the two cases above and hence gives $|\mathcal{S}_{\mathscr{T}}|= deg_{\mathcal{G}}(i) + 2 o_i + p_i$ which is equal to  $adv(i)$.
		\end{proof}
	\end{lemma}

	
	
	%%%%%%%%%%%%%%%%%%%%%%%%%%%%%%%%%%%%%%%%%%%%%%%%%%%%%%%%%%%%%%%%%%%%%%%%%%%%%%%%%%%%%%%%%%%%%%%%%%%%%
	
	
	\section{Lower Bound and Optimality Results}
	\label{sec:Results}
	
	Let a single uniprior ICP be represented by its information-flow graph $\mathcal{G}$. For this directed graph, let $\mathcal{G}_U$ denote the simplified undirected graph, where the simplification involves removing multiple edges between a pair of vertices. In the graph $\mathcal{G}$, let $\mathcal{D}(\mathcal{G})$ denote the vertex pairs such that there are two arcs between them, i.e., for some ordering on the vertices in $\mathcal{V}(\mathcal{G}), \ \mathcal{D}(\mathcal{G}) = \{(u,v) : u < v, \ (u,v), \ (v,u) \in \mathcal{E}(\mathcal{G})\}$ and let $d(\mathcal{G})$ be defined as $d(\mathcal{G}) = |\mathcal{D}(\mathcal{G})|$.
	

	
	\begin{theorem}
		\label{Thm:LB1}
		For an information-flow graph $\mathcal{G}$ on $n$ vertices, the total number of transmissions used by the receivers to decode their demands is lower bounded as $ T \geq |\mathcal{E}(\mathcal{G})|+|\mathcal{E}(\mathcal{G}_U)|-(n-1)$.
		\begin{proof}
			The number of demands in the single uniprior problem represented by $\mathcal{G}$ is equal to $|\mathcal{E}(\mathcal{G})|$, each of which takes at least one transmission to decode. An index code of length $n-1$ for this problem corresponds to a tree $\mathscr{T}$ on $n$ vertices. For such an index code with $l_{max}(\mathscr{T}) =2$, the number of demands which could take one transmission to decode  correspond to at most $n-1$ edges  in $\mathcal{G}_U$. Hence, there exists at least $|\mathcal{E}(\mathcal{G}_U)|-(n-1)$ demands, each of which takes one transmission extra over the one already counted in   $|\mathcal{E}(\mathcal{G})|$. Hence the lower bound. 
		\end{proof}
	\end{theorem}
	
%%%%%%%%%%%%%%%%%%%%%%%%%%%%%%%%%%%%%%%%%%%%%%%%%%%%%%%%%%
	

	\begin{theorem}
		\label{Thm:Opt1}
		For an information-flow graph, $\mathcal{G}$, with $d(\mathcal{G})  \leq n-1$, let $i \in \mathcal{V}(\mathcal{G})$ be a vertex with maximum advantage. If $\mathcal{G}$ satisfies the following conditions,  then the tree obtained from Algorithm 1 is optimal. 
		\begin{enumerate}
			\item For each $(j,k) \in \mathcal{D}(\mathcal{G})$, $j,k \in N_{\mathcal{G}}(i)$, either $N_{\mathcal{G}}(j) = \{i,k\}$ and $conn(i,j) = 1$ or  $N_{\mathcal{G}}(k) = \{i,j\}$ and $conn(i,k) = 1$.
			\item For each $j \in \mathcal{V}(\mathcal{G}) \setminus  (N_{\mathcal{G}}(i) \cup \{i\}$, $N_{\mathcal{G}}(j) =\{k\}$.
		\end{enumerate}
		\begin{figure} 		
			\centering
			\scalebox{0.45}{\includegraphics{Figures/Thm1.eps}}
			\caption{Possible double-arcs that can exist in $\mathcal{G}$ in Theorem \ref{Thm:Opt1}.}
			\label{fig:Thm1}
		\end{figure}
		\begin{proof}
			In the tree $\mathscr{T}$ obtained from Algorithm \ref{alg:1}, every edge that does not correspond to a double-edge $(j,k)$ in $\mathcal{G}$ is of the form $(i,l)$ for some $l \in \mathcal{V}(\mathcal{G})$. From condition 2) of the theorem, if a vertex  $j \notin N_{\mathcal{G}}(i)$, then it can appear in $\mathcal{G}$ only as shown in Fig. \ref{fig:Thm1}(c) and hence such a vertex $j$ will be present in Level 2 of the tree $\mathscr{T}$ as shown in Fig. \ref{fig:Alg1}. Hence , $ \forall (i,l) \in \mathcal{E}(\mathscr{T})$ , $l \in N_{\mathcal{G}}(i)$. 
			
			\emph{Claim}: For every $(j,k) \in \mathcal{D}(\mathcal{G})$, the edge $(j,k) \in \mathscr{T}$.
			
			The fact that $\forall (i,l) \in \mathscr{T}$, $l \in N_{\mathcal{G}}(i)$, along with the claim, would imply that every edge in the tree $\mathscr{T}$ corresponds to an arc in $\mathcal{G}$. Further since  $d(\mathcal{G})  \leq n-1$, out of the $(n-1)$ edges in the tree $\mathscr{T}$, $d(\mathcal{G})$ edges correspond to double-arcs in $\mathcal{G}$ and the remaining $(n-1-d(\mathcal{G}))$ correspond to single-arcs in $\mathcal{G}$. Thus, the total number of arcs in $\mathcal{G}$ for which a corresponding edge is present in $\mathscr{T}$ is $ 2d(\mathcal{G}) + (n-1-d(\mathcal{G})) = d(\mathcal{G}) + (n-1)$ which implies that the total number of transmissions used in decoding is $T = 2|\mathcal{E}(\mathcal{G})| - d(\mathcal{G}) - (n-1) = |\mathcal{E}(\mathcal{G})|+|\mathcal{E}(\mathcal{G}_U)|-(n-1)$ which is equal to the lower bound in Theorem \ref{Thm:LB1}. 
			
		\end{proof}
		
	\end{theorem}
	
	\textbf{Proof of Claim}: Satisfying the conditions of the theorem, a double edge can occur in $\mathcal{G}$ only in the three ways shown in Fig. \ref{fig:Thm1}(a), (b) and (c), where a dashed arc between the vertices $i$ and $j$ indicates that the corresponding arc may or may not be present in $\mathcal{E}(\mathcal{G})$.  
	\begin{enumerate}[(a)]
		\item Since Algorithm \ref{alg:1} starts with $\mathscr{T} = \mathcal{G}^*_i$ and doesn't remove the edge $(i,j)$, this type of double-edge is present in $\mathscr{T}$.
		\item From condition 1) of the theorem, either $N_{\mathcal{G}}(j) = \{i,k\}$ and $conn(i,j) = 1$ or $N_{\mathcal{G}}(k) = \{i,j\}$ and $conn(i,k) = 1$. If $N_{\mathcal{G}}(j) = \{i,k\}$ and $conn(i,j) = 1$, the tree $\mathscr{T}$ will be modified as $\big\{\mathscr{T}  \setminus (i, j)\big\} \cup (k, j)$. Similarly if $N_{\mathcal{G}}(k) = \{i,j\}$ and $conn(i,k) = 1$, the modification to the tree done by Algorithm \ref{alg:1} is $\big\{\mathscr{T}  \setminus (i, k)\big\} \cup (j, k)$. In either of these two cases, the edge $(j,k)$ will be added to $\mathscr{T}$, and if an end vertex of the double-edge $(j,k)$ has a double-edge with $i$, it is retained in $\mathscr{T}$.  
		\item In this case, the vertex $j$ satisfies the condition of the while loop in Algorithm \ref{alg:1} and hence the edge $(k,j)$ will be added to the tree $\mathscr{T}$. 
	\end{enumerate}

	%%%%%%%%%%%%%%%%%%%%%%%%%%%%%%%%%%%%%%%%%%%%%%%%%%%%%%%%%%%%%%%%%%%%%%%%%%%%%%%%%%%%%%%%%%%%%%%%%%%%%
	\section{Conclusion}
	\label{sec:Conc}
	In this paper, we considered single uniprior ICPs over Rayleigh fading channels and showed that minimizing the total number of transmissions used in decoding the requested messages minimized the average probability of message errors. An algorithm to generate a spanning tree which could result in an index code with fewer transmissions being used in decoding as compared to the optimal star graph was presented, and a condition under which this algorithm generates the optimal spanning tree was derived.
	
	%%%%%%%%%%%%%%%%%%%%%%%%%%%%%%%%%%%%%%%%%%%%%%%%%%%%%%%%%%%%%%%%%%%%%%%%%%%%%%%%%%%%%%%%%%%%%%%%%%%%%
	
	\section{Appendix}
	\textbf{Graph Theoretic Preliminaries} : 
	The following is a list of some basic graph theoretic definitions and notations  \cite{RD,DBW} that are used in this paper.
	A graph $G$ is a triple consisting of a vertex set $V(G)$, an edge set $E(G)$, and a relation that associates with each edge two vertices called its endpoints. A \emph{directed} graph is a graph with a direction associated with each edge in it. This implies that an edge $(u,v)$  directed from the vertex $u$ to the vertex $v$ and an edge $(v,u)$ from $v$ to $u$ are two different edges in a directed graph, whereas in an undirected graph, $(u,v)$ and $(v,u)$ mean the same edge between the endpoints $u$ and $v$.	The edges in a directed graph are also called arcs. Two vertices $u$ and $v$ are \emph{adjacent} or \emph{neighbors} in an undirected graph $G$ if there exists an edge $(u,v)$ in $G$.  In a directed graph $G$, the out-neighborhood of a vertex $u$, denoted by $N_G^+(u)$, is the set of vertices $\{v:(u,v) \in E(G)\}$ and its in-neighborhood, denoted as $N_G^-(u)$, is the set of vertices $\{v:(v,u) \in E(G)\}$. The set of neighbors of a vertex $v \in V(G)$ is denoted as $N_G(v)$ and for a directed graph $N_G(v) = N_G^+(u) \cup N_G^-(u)$.  A graph $G'$ is called a sub-graph of the graph $G$, written as $G' \subseteq G$, if $V(G') \subseteq V(G)$ and $E(G') \subseteq E(G)$. The \emph{degree} of a vertex $v$, $\deg_{G}(v)$, is the number of edges incident at it, which is equal to the number of neighbors of the vertex $v$ in the graph $G$.  For a vertex $v$ in a directed graph $G$, its in-degree, denoted as $in-\deg_{G}(v)$ is the number of vertices in its in-neighborhood and its out-degree $out-\deg_{G}(v)$ is the number of vertices in its out-neighborhood and hence its degree, $\deg_{G}(v)$ is equal to the sum of its in-degree and out-degree in $G$. A \emph{path} is a non-empty graph $P = (V,E)$ of the form $V = \{x_0,x_1,\cdots,x_{k-1},x_k\}$ and $E = \{(x_0,x_1), (x_1,x_2),\cdots,(x_{k-1},x_k)\}$, where all the $x_i$s are all distinct. A \emph{cycle} is a path with the first and last vertices being the same. An undirected graph $G$ is called \emph{connected} if it is non-empty and any two of its vertices are linked by a path in $G$.  A directed graph $G$ is said to be \emph{strongly connected} if there exists a path from $u$ to $v$ and another path from $v$ to $u$ for every pair of vertices $u, v \in V(G)$. A \emph{strongly connected component} of a directed graph $G$ is a sub-graph of $G$ that is strongly connected and is maximal with this property.  An undirected graph is said to be a \emph{tree} if it is connected and does not have any cycles. A tree on $n$ vertices has $n-1$ edges. A \emph{rooted tree} is a tree in which one vertex, called the \emph{root vertex}, is distinguishable from the others. A rooted tree with root vertex $v$ is said to be a \emph{star graph} with vertex $v$ as \emph{head} if every edge in the tree has $v$ as one of its endpoints. For an undirected graph $G$, a \emph{spanning tree} is a sub-graph which is a tree and includes all the vertices in $G$. For a pair of vertices $u,v$ in a connected graph $G$, the \emph{distance} between $u$ and $v$, denoted $dist(u,v)$, is the length of the shortest path between $u$ and $v$. The \emph{diameter} of a graph is the length of the shortest path between a pair of vertices which are at maximum distance from each other in $G$. A \emph{complete graph} on $n$ vertices, denoted $K_n$, is the undirected graph on $n$ vertices, with every pair of vertices being connected by an edge. 
%	%	 A \emph{Hamiltonian} cycle in a graph is a cycle which visits each of the vertices in the graph exactly once.
%	%	%	\item A directed graph $G$ is said to be \emph{strongly connected} if for every pair of vertices $(u,v)$ in $G$, there exists a directed path from $u$ to $v$ as well as a directed path from $v$ to $u$. 
%	%	
%	%	%%%%%%%%%%%%%%%%%%%%%%%%%%%%%%%%%%%%%%%%%%%%%%%%%%%%%%%%%%%%%%%%%%%%%%%%%%%%%%%%%%%%%%%%%%%%%%%%%%%%%
%	%	
	\section*{Acknowledgment}
	This work was supported partly by the Science and Engineering Research Board (SERB) of Department of Science and Technology (DST), Government of India, through J.C. Bose National Fellowship to Prof. B. Sundar Rajan and IISc-IoE postdoctoral fellowship awarded to Dr. Charul Rajput.
%	
	%%%%%%%%%%%%%%%%%%%%%%%%%%%%%%%%%%%%%%%%%%%%%%%%%%%%%%%%%%%%%%%%%%%%%%%%%%%%%%%%%%%%%%%%%%%%%%%%%%%%%%%%%%%%%
	\begin{thebibliography}{1}
		
		
		
		\bibitem{BiK}
		Y. Birk and T. Kol, "Informed-source coding-on-demand (ISCOD) over broadcast channels," Proceedings. IEEE INFOCOM '98, pp. 1257-1264. 
		
		
%		\bibitem{BBJK}
%		------------, ``Index Coding With Side Information,'' in \emph{IEEE Transactions on Information Theory}, vol. 57, no. 3, pp. 1479-1494, March 2011.
%		
%		\bibitem{LS}
%		E. Lubetzky and U. Stav, ``Nonlinear Index Coding Outperforming the Linear Optimum," in \emph{IEEE Tran.  Info. Theory}, vol. 55, no. 8, pp. 3544-3551, Aug. 2009.
		
		\bibitem{OngHo}
		L. Ong and C. K. Ho, ``Optimal index codes for a class of multicast networks with receiver side information," in \emph{Proc. 2012 IEEE ICC}, Ottawa, ON, Canada, 2012, pp. 2213-2218.
		
		\bibitem{TRAR}
		A. Thomas, K. Radhakumar, C. Attada and B. S. Rajan, ``Single Uniprior Index Coding With Min–Max Probability of Error Over Fading Channels," in IEEE Tran. on Vehicular Tech., vol. 66, no. 7, pp. 6050-6059, July 2017.
		
		\bibitem{Cay}
		Cayley, A. (1889) A Theorem on Trees. The Quarterly Journal of Mathematics, 23, 376-378.
		
	
		
		\bibitem{RD}
		R. Diestel, ``Graph Theory (Graduate Texts in Mathematics)", Springer, 2005.
		
		\bibitem{DBW}
		D. B. West, ``Introduction to Graph Theory", 2nd ed. Prentice Hall, 2000.

	\end{thebibliography}
	%%%%%%%%%%%%%%%%%%%%%%%%%%%%%%%%%%%%%%%%%%%%%%%%%%%%%%%%%%%%%%%%%%%%%
	
	
	
	
\end{document}
