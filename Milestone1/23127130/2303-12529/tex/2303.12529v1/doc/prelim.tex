\section{Preliminaries}
\label{sec:prelim}
In this section, we will introduce some concepts and background related to this work.
Following the traditions, we denote $\mathbf{Z}_\mathrm{t}$, $\mathbf{M}$, and $\mathbf{I}$, as the target layout, mask image, and intensity of aerial image respectively.
$\mathbf{Z}$, $\mathbf{Z}_{in}$, and $\mathbf{Z}_{out}$ are the wafer image under the nominal/min/max process conditions.
We use $\mathbf{H}$ for lithography kernels and $\phi$ for the level set function (LSF).


\subsection{Level Set-Based ILT Algorithms}

Level set as a mathematical technique is pioneered by Osher \textit{et al.}~\cite{osher1988fronts}.
Recently, it has been actively explored as a feasible alternative to tackle the ILT problem.
% When applying the level set methods for mask optimization in $2D$ space $\Omega$,
Let $\mathcal{C} : \Omega \rightarrow \mathbb{R}^2$ denote a parametric curve in $2$D space $\Omega$,
%  in a binary image.
%  where $s \in \Omega_P = [0, 1]$ is the parameterization interval.
the level set method implicitly represents the boundary of the mask using zero crossing of a LSF $\phi(x,y): \Omega \rightarrow \mathbb{R}$:
\begin{equation}
  \mathcal{C}=\{(x, y) \mid \phi(x, y)=0\} .
\end{equation}
% \begin{equation}
%   \left\{\begin{array}{r}
%   C=\{(x, y) \mid \phi(x, y)=0\} \\
%   \text{inside}(C)=\{(x, y) \mid \phi(x, y) < 0\} \\
%   \text{outside}(C)=\{(x, y) \mid \phi(x, y) > 0\}
%   \end{array}\right.
% \end{equation}
As depicted in \Cref{fig:ls_vis_evo}, the mask optimization process can be viewed as the evolution along the descent of the LSF $\phi$.
The commonly used LSF $\phi$ is the signed distance function (SDF),
\begin{equation}
  \label{eq:sdf}
  \phi_{\mathrm{SDF}}(x, y)=\left\{\begin{array}{ll}
  -d(x, y), & \mathrm{if} (x, y) \in \mathrm{inside}(\mathcal{C}), \\
  0, & \mathrm{if} (x, y) \in \mathcal{C}, \\
  d(x, y), & \mathrm {if} (x, y) \in \mathrm{outside}(\mathcal{C}),
  \end{array}\right.
\end{equation}
where $d(x, y)$ is the minimum Euclidean distance from point $(x, y)$ to the parametric curve $\mathcal{C}$.
As illustrated in \Cref{fig:test8_mask_ls}, the contours are labeled with its SDF values,
and the $\mathcal{C}$ is the contour labeled by 0.
Now the mask image $\mathbf{M}$ can be represented by $\phi$ as
\begin{equation}
  \label{eq:phi2mask}
  \mathbf{M}(x, y)=\left\{\begin{array}{ll}
  1, & \mathrm{ if }\ \phi(x, y) \leq 0 \\
  0, & \mathrm{ if }\ \phi(x, y) > 0 .
  \end{array}\right.
\end{equation}
During the evolution, the mask boundary $\mathcal{C}(t)$ changes over time $t \in \mathbb{R}$, the curve evolution then can be formally defined as
\begin{equation}
  \label{eq:partial_c_t}
  \frac{\partial \mathcal{C}(t)}{\partial t}=v \vec{n},
\end{equation}
where $\vec{n} = \frac{\nabla \phi}{\vert\nabla\phi\vert}$ is the unit vector in the outward normal direction of the curve $\mathcal{C}$
and $v$ indicates the velocity along the normal direction.
% Assume the point $p = (x,y)$ belongs to the evolving front $\mathcal{C}(t)$ so that $p(t)$ is its position over time $t$.
We use the zero level set to implicitly represent the mask boundary, thus: $\phi(\mathcal{C}(t),t) = 0$.
The chain rule gives us,
% \begin{equation}
%   \label{eq:partial_chain_rule}
%   \begin{aligned}
%     \frac{\partial \phi(\mathcal{C}(t), t)}{\partial t} &=0 \\
%     \frac{\partial \phi}{\partial \mathcal{C}(t)} \frac{\partial \mathcal{C}(t)}{\partial t}+\frac{\partial \phi }{\partial t} &= 0
%   \end{aligned}
% \end{equation}
\begin{equation}
  \label{eq:partial_chain_rule}
  \begin{aligned}
    \frac{\partial \phi(\mathcal{C}(t), t)}{\partial t} =0 \rightarrow
    \frac{\partial \phi}{\partial \mathcal{C}(t)} \frac{\partial \mathcal{C}(t)}{\partial t}+\frac{\partial \phi }{\partial t} = 0 .
  \end{aligned}
\end{equation}
Consider all the points on the evolving front $\mathcal{C}(t)$, $\frac{\partial \phi}{\partial \mathcal{C}} = \nabla\phi$, combining the \Cref{eq:partial_c_t} and \Cref{eq:partial_chain_rule},
the motion equation of LSF $\frac{\partial\phi}{\partial t}$ can be formally expressed by
\begin{equation}
  \label{eq:phi_v}
  \frac{\partial \phi}{\partial t}= -v\vert\nabla\phi\vert.
\end{equation}

\Cref{eq:phi_v} is a partial differential equation (PDE), once the level set $\phi$ and velocity $v$ are defined,
the first-order derivative in space and time of \Cref{eq:phi_v} can be approximated using finite difference techniques.
% Given an initial $\phi$ at $t = 0$, it will be possible to get $\phi$ at anytime with the motion equation $\frac{\partial\phi}{\partial t}$.
Evolution of LSF $\phi(x, y, t)$ can be performed iteratively.
We use $\phi_i(x, y)$ to denote $\phi(x, y, t_i)$ for simplicity.
For $i \in \{0, 1, 2 \dots T-1\}$, the $i^{th}$-step update is
\begin{equation}
  \label{eq:levelset_evolution}
  \phi_{i+1}(x, y)=\phi_{i}(x, y)+\Delta t \frac{\partial \phi_{i}}{\partial t} ,
\end{equation}
where $\Delta t$ is the time step, $\phi_0(x,y)$ is the initial LSF,
and the $\phi_T(x, y)$ is the corresponding output LSF after $T$ evolution steps.
As shown in \Cref{fig:ls_vis_evo}, we can obtain the optimized mask after $T$ steps by applying the \Cref{eq:phi2mask}.

\subsection{The Lithography Simulation Model}
\label{sec:lithomodel}

During the conventional lithography process, the input mask $\mathbf{M}$ is transformed through an optical projection system into the aerial image. 
% The photoresist material on the wafer plane is then exposed under the incident aerial light.
The distribution of aerial light intensity $\mathbf{I}$ floating on the wafer forms the printed image $\mathbf{Z}$.
The optical projection system can be expressed mathematically using Hopskin's diffraction model~\cite{OPC-RSL1951-Hopkins}.
The sum of coherent systems (SOCS) can roughly estimate Hopskin's diffraction model by performing singular value decomposition, the optical projection process is then replaced by a set of coherent kernels. 
% The eigenfunction and eigenvalue works as a low pass filters, the eigenvalue works as its weight and the corresponding eigenfunction represents the filtering behavior.
% For a mask $\mathbf{M}$, the intensity of aerial image $\mathbf{I}$ could be represented using $N_k$ optical kernels
The intensity of aerial image $\mathbf{I}$ can be represented by convolving the mask $\mathbf{M}$ and a set of optical kernels $\mathbf{H}$,
\begin{equation}
  \mathbf{I}(x, y) = \sum_{i=1}^{N^2} \sigma_i | \mathbf{M}(x, y) \otimes h_i(x, y)|^2 .
\end{equation}
Here $\otimes$ denotes the convolution operation,
and $h_i$ is the $i^{th}$ kernel of the optical kernel set $\mathbf{H}$
and $\sigma_i$ is the corresponding weight of the coherent system.
The $N_k^{th}$ order approximation to the partially coherent system can be obtained by,
\begin{equation}
  \mathbf{I}(x, y) \approx \sum_{i=1}^{N_{k}} \sigma_{i}\left|\mathbf{M}(x, y) \otimes h_{i}(x, y)\right|^{2},
\end{equation}
where $N_k = 24$ in our implementation.
% The aerial image is then cast on the wafer plane and undergoes a photoresist model.
After optical simulation, the aerial image undergoes a resist model
to estimate the final printed  shape on wafer.
% Although in real situations the resist effect is far more complex,
For methodology verification and also for simplicity,
we adopt the constant threshold resist model which is consistent with the ICCAD 2013 contest settings~\cite{OPC-ICCAD2013-Banerjee}.
As depicted in \Cref{fig:test8_mask_pixel}, given the print threshold $I_{th}$, the printed wafer image can be expressed as:
% \begin{equation}
  \begin{align}
    \label{eq:binary_intensity}
    \begin{split}
      \mathbf{Z}= \left \{
        \begin{array}{lr}
          1,  & \mathrm{if} \ \mathbf{I} \ge I_{th}, \\
          0,  & \mathrm{if} \ \mathbf{I} < I_{th}.
        \end{array}
        \right.
      \end{split}
    \end{align}
% \end{equation}




% The detailed algorithm of CUDA-based high performance lithography model is inspired by,

\subsection{Mask Printability and Mask Manufacturability}
Mask printability represents the quality of the printed patterns generated from the optimized mask.
In this paper, we use squared $L_2$ error and process variation band (PVBand) as two typical metrics to evaluate mask printability.
Moreover, the mask fracturing shot count proposed in Neural-ILT~\cite{NEURAL-ILT-ICCAD2020-Jiang} is also applied in this work to evaluate mask complexity and manufacturability.
% In this paper, we use edge placement error (EPE) and process variation band (PVBand) as two typical metrics to evaluate mask printability.

% \subsubsection{EPE} Edge placement error (EPE) is used to evaluate the pattern fidelity under nominal condition.
% In our framework, EPE is statically measured among the pre-defined probe points along the boundary.
% Given the $D_{epe}$ as the maximum EPE tolerance distance,
% if the printed patterns at a probe point is inside or outside the target pattern, and the perpendicular distance to its edge is larger than $D_{epe}$,
% then an EPE violations appears at the probe point.
% The total EPE violations are the sum of the violations at the probe points set.
% Pattern fidelity of the Mask is evaluated to be inversely propotional to the number of total EPE violations.
\subsubsection{Squared $L_2$ error} Given the wafer image $\mathbf{Z}$ and target image $\mathbf{Z}_\mathrm{t}$, the squared $L_2$ error is calculated by:
$\left\|\mathbf{Z}-\mathbf{Z}_{\mathrm{t}}\right\|_{2}^{2}$.

\subsubsection{PVBand} Process variation band (PVBand) is the bitwise-XOR region among all the printed patterns under different process conditions.
In our work, for simplicity, we calculate the PVBand under two extreme conditions, one at nominal condition with $+2\%$ dose and the other one at defocus and $-2\%$ dose.
A mask is more robust if its PVBand area is smaller.


\subsubsection{Mask Fracturing Shot Count} Many conventional pixel-based ILT methods tend to optimize the mask only to improve mask printability.
However, most of these optimized masks contain plenty of tiny irregular sub-features, which increase the difficulty for mask manufacture.
% In this wo, the mask manufacturability is also considered as an optimization target.
In this work, we use shot count to evaluate the mask manufacturability.
% \textbf{Shot count}:
An evaluated mask $\mathbf{M}$ can be fractured into a set of small rectangles which could replicate exactly the original mask.
Mask fracturing shot count stands for the number of fractured rectangles.
% The scripts for shot count evaluation are obtained from the authors of Neural-ILT~\cite{NEURAL-ILT-ICCAD2020-Jiang} to guarantee the results are comparable.

% \todo{Add problem formulation.}
