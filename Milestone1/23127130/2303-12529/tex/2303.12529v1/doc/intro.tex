\section{Introduction}

As minimum feature size continues to shrink,
the optical diffraction and proximity effects in lithography become not negligible,
which could seriously degrade the yield of integrated circuits.
To compensate for pattern distortion and improve process window in the lithography process,
optical proximity correction (OPC) is used to ensure pattern transfer fidelity.
Typical OPC approaches encompass rule-based methods~\cite{OPC-ISQED2000-Park},
model-based methods~\cite{OPC-ICCAD2014-Awad, OPC-TCAD2016-Su},
inverse lithography techniques~\cite{OPC-DAC2014-Gao, OPC-ICCAD2017-Ma},
and DNN-based methods\cite{OPC-TCAD2020-Yang, chen2020damo, NEURAL-ILT-ICCAD2020-Jiang}.

In model-based OPC procedure, the edges of the initial mask are fragmented into segments,
which are moved iteratively under the guidance of lithography simulation.
% Simple yet efficient, model-based methods dominate in early design automation software like Calibre\cite{TOOL-calibre}.
Inverse lithography techniques (ILT) also leverage rigorous simulation to perform mask printability enhancement.
Moreover, ILT can achieve pixel-level optimization and thus find a better solution
in a larger solution space by seeing mask optimization as an inverse imaging problem.
Gao \textit{et al.}~\cite{OPC-DAC2014-Gao} derived a closed-form gradients descent algorithm through direct edge placement error and process window optimization.
% Although Model-based and ILT-based methods have been widely adopted in industry,
% they inevitably suffer from heavy computational overhead because multiple rounds of lithography simulation are essential during the optimization process.
% This issue gets more acute with decreasing technology node and more complicated layout shapes.
In recent years, DNN-based methods have drawn great attention as they can attain significant speedup
while preserve comparable mask printability by incorporating previous experience.
% Choi \textit{et al.}~\cite{choi2018neural} construct a classifier-based mask bias model to improve previous regression models.
Yang \textit{et al.}~\cite{OPC-TCAD2020-Yang} proposed a generative model to produce an initial mask solution,
which greatly lowers the number of iterations required in traditional ILT methods.
% Chen \textit{et al.}~\cite{chen2020damo} firstly enable full chip OPC by incorporating a DNN-based lithography simulator
% and mask generator simultaneously and outperform industry toolkit on ISPD 2019 benchmark.
\ifshowfig
\begin{figure}[tb!]
    \vspace{-.3in}
    \centering
    \subfloat[]{ \includegraphics[width=.76\linewidth]{M1_test1_plasma} \label{fig:ls_vis_phi}}
    \subfloat[]{ \includegraphics[width=.2\linewidth]{phi_evo} \label{fig:ls_vis_evo}}
    \caption{
      (a) The 3D illustration of level set function $\phi$.
      The mask is shaped as the cross-section of the level set continuum with the zero plane.
      The contours on the x-y plane are the projected level set.
      (b) The level set evolution process.
    }
    \label{fig:ls_vis}
\end{figure}
\fi



% \ifshowfig
% \begin{figure}[tb!]
%   \centering
%   \includegraphics[width=.88\linewidth]{ls3d/M1_test1_plasma.png}
%   \caption{}
%   \label{fig:ls_vis}
% \end{figure}
% \fi

In the past decades, level set-based ILT methods have been actively explored as a feasible alternative to pixel-based ILT methods in OPC tasks.
As illustrated in \Cref{fig:ls_vis}, the implicit representation of level set method is naturally more effective in dealing with complex topology changes and lithography development~\cite{sethian1997overview}.
Shen \textit{et al.}~\cite{OPC-OPEX2009-Shen} solved the inverse lithography problem using a level set time-dependent model with finite difference schemes.
% They further considered defocus and aberration to enhance robustness against process variations~\cite{shen2011robust}.
% Lv \textit{et al.}~\cite{OPC-JVSTB2013-Lv} improved the pattern fidelity with fast convergence by employing the conjugate gradient method and optimized time step.
% Geng \textit{et al.}~\cite{geng2015fast} adopted process variation band cost function and reduced the runtime by leveraging the hybird conjugate gradient (CG) method.
Yu \textit{et al.}~\cite{OPC-TR2020-Yu} proposed a momentum-based conjugate gradient (CG) method and accelerated the level set evolution with GPU-enabled Fast Fourier Transform (FFT) algorithm.



Briefly, there are two main approaches for the inverse lithography techniques: parametric and implicit.
% The parametric methods~\cite{OPC-DAC2014-Gao, OPC-ICCAD2017-Ma, OPC-TCAD2020-Yang, chen2020damo, NEURAL-ILT-ICCAD2020-Jiang} use pixel-wise tensorsor neural networks as the parameterization functions to generate the mask (\Cref{fig:test8_pixel}).
The parametric methods~\cite{OPC-DAC2014-Gao, OPC-ICCAD2017-Ma, OPC-TCAD2020-Yang, chen2020damo, NEURAL-ILT-ICCAD2020-Jiang} use pixel-wise tensors to generate the mask (\Cref{fig:test8_pixel}).
While the implicit approaches represent the mask as a zero level set cross-section~\cite{OPC-OPEX2009-Shen, shen2011robust, OPC-JVSTB2013-Lv, geng2015fast, OPC-TR2020-Yu} (\Cref{fig:test8_ls}).
So far, due to the simplicity and flexibility of pixel-based gradient descent methods, the parametric methods have been thoroughly researched through the perspectives of objective function, optimization method, and the DNN acceleration, achieving state-of-the-art (SOTA) runtime performance and mask print fidelity.
However, as depicted in \Cref{fig:test8_mask_pixel}, the parametric methods inevitably generate unnecessary isolated stains or edge glitches with zigzagging and tortuous complex mask boundaries
while the level set implicit representation is accomplished in mask boundary continuity and curvature control (\Cref{fig:test8_mask_ls}).
Unfortunately, due to the extra computational overhead introduced by the level set evolution, the application of level set-based ILT method has been greatly underestimated.


With the rapid development of GPU and deep learning, the progressive potential of level set-based ILT methods should be reconsidered.
Motivated by these issues, we present the DevelSet framework, which contains two parts.
The GPU accelerated level set optimizer (DevelSet-Optimizer) and the deep level set neural network (DevelSet-Net).
Following the improvements proposed by the previous work, DevelSet-Optimizer (DSO) incorporates the curvature term into level set-based ILT to reduce mask complexity
and develops a set of GPU friendly algorithm to overcome the computational overhead.
% and leverage GPU to overcome the computational overhead.
DevelSet-Net (DSN) is designed to provide better initial solutions by leveraging the fast inference ability of neural network and compensate DSO for the curvature cost by applying a novel modulation branch.
The DevelSet framework benefits from end-to-end joint optimization of DSN and DSO, achieving SOTA fast convergence and mask printability.
Our main contributions are:
\begin{itemize}
  \item We propose DevelSet, an improved level set-based ILT framework with CUDA and DNN acceleration.
  \item We firstly introduce curvature term into level set-based ILT methods to reduce mask complexity and leverage GPU to perform all the calculations.
  \item We are the first to integrate level set into deep neural network for an end-to-end joint mask optimization flow.
  \item We design a novel multi-branch neural network architecture with level set embeddings to further boost the performance and improve mask printability.
  \item Experimental results show that DevelSet achieves SOTA mask printability with predominant runtime advantage for instant mask optimization \ie around 1 second.
\end{itemize}
% \item
% \item We implement our improved level set ILT method with fully CUDA utilization, achieving SOTA runtime performance
% \item we develop a set of GPU-friendly
% \item We build a unified CUDA framework that allows the improved level set ILT algorithm to achieve fast convergence.
% \item We implement our improved level set ILT method with fully CUDA utilization, achieving SOTA runtime performance

The rest of the paper is organized as follows: \Cref{sec:prelim} lists some preliminaries about level set algorithms and mask optimization methods.
\Cref{sec:algo} details the DevelSet algorithm. \Cref{sec:exp_results} presents our experimental results, followed by a conclusion in \Cref{sec:conclu}.
% In this work, we improve the level set-based ILT method by

\ifshowfig
\begin{figure}[tb!]
    \centering
    \subfloat[]{ \includegraphics[height=.416\linewidth]{M1_test8_pixel} \label{fig:test8_pixel}}
    \subfloat[]{ \includegraphics[height=.416\linewidth]{M1_test8_plasma} \label{fig:test8_ls}} \\
    \subfloat[]{ \includegraphics[height=.426\linewidth]{pixel_mask} \label{fig:test8_mask_pixel}}
    \subfloat[]{ \includegraphics[height=.426\linewidth]{ls_mask} \label{fig:test8_mask_ls}}
    \caption{
        Comparison of pixel-based ILT and level set-based ILT.
        (a) Intensity matrix of pixel-based ILT;
        (b) Level set-based ILT;
        (c) Mask generated by pixel-wise intensity threshold;
        (d) Mask generated by zero level set.
    }
    \label{fig:pixel_vs_ls}
\end{figure}
\fi

% most task in computer vision can be completed in less than a second.
% A quite straight forward question is that can we finish the mask optimization process in seconds?
% Unfortunately, all current approaches can not get an optimized mask within 10 seconds.
% Previous level set-based ILT approaches \cite{shen2011robust} introduce more computational efforts and thus are more time-consuming in practice.

