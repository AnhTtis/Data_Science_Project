%\title{dinbrief package example}
% User guide: http://texdoc.net/texmf-dist/doc/latex/dinbrief/dinbrief.pdf
% Note: you can also use ä ö ü and ß directly instead of \"a \"o \"u and \ss
\documentclass[10pt]{dinbrief}%
\usepackage[utf8x]{inputenc}%
% \usepackage{german}%
\usepackage{graphicx}%
%
%
%
%
\address{Dr. Jonas Oppenlaender\\

Seminaarinkatu 15\\
40014 Jyv\"askyl\"a\\
Finland\\[\medskipamount]
% oppenlaenderj@acm.org\\
joppenlu@jyu.fi\\
https://www.jonaso.de
% Mobil: +358 46 52 19726
}%
%
\nowindowtics
%
% \backaddress{J. Oppenl\"ander, Peltokatu 19 B 62, 90120 Oulu}%
% \signature{Jonas Oppenl\"ander}%
% \place{Oulu}%
%
\begin{document}
\phone{+358}{46 52 19726}
% \email{jonas.oppenlaender@oulu.fi}

\begin{letter}{%
Editors of the \\
International Journal of \\
Human-Computer Studies
% \\[\medskipamount]
}%
% \yourmail{01.04.93}
% \sign{FI76268768\_E5}

% \subject{\textbf{Assistant Professor of Human Computer Interaction (Vacancy number \VACANCYCODE)}}

\opening{Dear Editors,}

%%%%%%%%%%%%%%%%%%%%%%%%%%%%%%%%%%%%%%%%%%%
% Please tell us briefly why you are interested in this particular job and why you believe you would be a good match.
%%%%%%%%%%%%%%%%%%%%%%%%%%%%%%%%%%%%%%%%%%%


% \begin{minipage}{\textwidth}
% \begin{samepage}


in the past two years, we have seen the rise of text-to-image generative models.
These generative models use zero-shot prompting to synthesize images from text written in natural language.
Within the community of art practitioners, this iterative interaction with the system with the purpose of synthesizing artworks is often referred to as prompt engineering.
To date, only few studies in HCI have investigated the skill of ``prompt engineering'' from a human-centered perspective.

Our paper provides a timely and an in-depth investigation of the skill of prompt engineering in three experimental studies conducted with participants recruited from a crowdsourcing platform.
Based on the findings of our three studies, we provide a set of clear guidelines for conducting experiments on prompt engineering with participants recruited from paid crowdsourcing platforms.
All data pertaining to our work in this paper is made publicly available for the benefit of the research community and in the spirit of open science.\footnote{https://osf.io/bjwf4/?view\_only=caf73282354643e9bfb34b3b05ef4b62}


% Crowdsourcing has become part of the standard toolbox of researchers in many disciplines.
% Crowdsourcing is often the top choice for conducting experiments online and this development has accelerated even more during the COVID-19 pandemic.
% Typically, crowdsourcing campaigns are prefaced with smaller studies conducted in an ad-hoc manner. These \textit{pilot studies} are conducted to estimate important parameters of the crowdsourcing campaign, such as the price of the task.
% Many mentions of such pilot studies can be found in the scholarly literature in the fields of CSCW and HCI.

% Despite the important role that crowd pilot studies play in configuring and thereby shaping crowdsourcing studies, authors typically mention pilot studies only in passing.
% Details about the pilot studies are seldom reported. %UG will do this: Add a couple of sentences about why this is problematic! (1) replication, reproduction of studies, findings.. fundamental tenets of the scientific method... (2) open science ... (3) defining standard practice as a research community ... transparency)
% This is undesirable, since a lack of detail hinders future reproduction and replication. % ~\citep{echtler2018open}.
% This also lies in stark contrast to one of the fundamental tenets of open science -- to make knowledge transparent and accessible. For instance, readers can glean little from reading that authors `\textit{iterated extensively in pilot studies with crowd workers to strike a balance between simplicity (avoid complex or numerous instructions) and effectiveness (make the layout better)}'~--- a quote from literature reviewed in our work. 
% Researchers or practitioners who may want to replicate such a process for their own crowdsourcing study based on such a pilot study description will arguably be left guessing. It is important for the crowdsourcing research community to set the right precedents and establish good practices.
% It is worth noting that pilot studies are just as likely to be flawed as any other (main) studies which are expected to withstand the scrutiny of peer-review as a means to ensure quality, reliability, good practice, and a sound scientific method. Such flaws in pilot studies can go unnoticed if they are not reported in sufficient detail.
% While there are guidelines and checklists for running and reporting crowdsourcing experiments, 
% there is a gap in the  scholarly literature on pilot studies in crowdsourcing research.


% Our systematic literature review is the first to provide a detailed investigation on the current state of practice of pilot study reporting in the crowdsourcing and HCI literature. 
% %
% Based on the findings of our literature review and survey study, we provide a set of clear guidelines for reporting pilot studies.
% We reflect on the trade-offs around running pilot studies and discuss implications for the design of crowdsourcing platforms. All data % and code
% pertaining to our work in this paper are made publicly available for the benefit of the research community and in the spirit of open science.\footnote{\url{https://osf.io/46fxj/?view_only=0eac3aaf2c734a6096e33f9734f62902}}



\noindent
\textbf{The submission contains original work which has not been published previously. The work is not under consideration for publication elsewhere.}



Best regards,
\\[.4em]
    \includegraphics[width=4cm]{figures/signature.jpg}%

% \end{samepage}
% \end{minipage}


% \closing{Mit freundlichen Grüßen,
% \\%[.2em]
%     \includegraphics[width=5cm]{figures/signature.jpg}%
% }




% \ps{Wir bitten um schnelle Erledigung.}

% \cc{Deutsche Bundespost\\
% Karlsruher Privatfunk\\
% S\"uddeutscher Rundfunk}

% \encl{Abschrift der Urkunde}

\end{letter}
\end{document}