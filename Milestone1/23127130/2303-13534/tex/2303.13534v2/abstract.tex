%% This is file `sample-acmsmall-submission.tex',
%
%% Commands for TeXCount
%TC:macro \cite [option:text,text]
%TC:macro \citep [option:text,text]
%TC:macro \citet [option:text,text]
%TC:envir table 0 1
%TC:envir table* 0 1
%TC:envir tabular [ignore] word
%TC:envir displaymath 0 word
%TC:envir math 0 word
%TC:envir comment 0 0
%%
%%
%% The first command in your LaTeX source must be the \documentclass command.
\documentclass[acmsmall,screen,authorversion,nonacm,anonymous]{acmart}%
% \documentclass[acmsmall]{acmart}%
%%% sigchi
%
%%%%%%%%%%%%%%%%%%%%%%%%%%%%
%TC:ignore
\usepackage{afterpage}%
\usepackage{tabularx}%
\usepackage{color}%
\usepackage{soul}%
%     \definecolor{HLColor}{rgb}{1,0.94,0.72}
%     \sethlcolor{HLColor}%
\newcommand{\todo}[1]{{\leavevmode\color{red}#1}}
\newcommand{\ok}[1]{{\leavevmode\color{green}#1}}
\usepackage{multirow}%
\usepackage{makecell}%
\usepackage[export]{adjustbox}% 
\usepackage{graphicx}
\usepackage{subcaption}
\usepackage{booktabs}
%%%
% \usepackage{sparklines}
% \def\sparkrectangleh #1 #2 {%
%   \ifdim #1pt > #2pt
%         \errmessage{The left corner #1 of rectangle cannot be lower than #2}%
%   \fi
%   {\pgfmoveto{\pgforigin}\color{sparkrectanglecolor}%
%   \pgfrect[fill]{\pgfxy(#1, 0)}{\pgfxy(#2-#1,1)}}}%
% \setlength\sparklinethickness{0.5pt}
% \def\sparklineheight ex{7pt}
%TC:endignore
%%%%%%%%%%%%%%%%%%%%%%%%%%%%

\usepackage{setspace}%
\setstretch{1.5}%

%TC:ignore
\setcopyright{acmcopyright}%
\copyrightyear{2022}%
\acmYear{2022}%
\acmDOI{XXXXXXX.XXXXXXX}%
\acmJournal{JACM}%
\acmVolume{37}%
\acmNumber{4}%
\acmPrice{}%
\acmArticle{}%
\acmMonth{7}%
%TC:endignore

%TC:ignore
% remove acm reference format for review - add again later!
\settopmatter{printacmref=false}
\setcopyright{none} 
%TC:endignore

% \usepackage{lscape}%

%% Submission ID.
%% Use this when submitting an article to a sponsored event. You'll
%% receive a unique submission ID from the organizers
%% of the event, and this ID should be used as the parameter to this command.
%%\acmSubmissionID{123-A56-BU3}

%%
%% For managing citations, it is recommended to use bibliography
%% files in BibTeX format.
%%
%% You can then either use BibTeX with the ACM-Reference-Format style,
%% or BibLaTeX with the acmnumeric or acmauthoryear sytles, that include
%% support for advanced citation of software artefact from the
%% biblatex-software package, also separately available on CTAN.
%%
%% Look at the sample-*-biblatex.tex files for templates showcasing
%% the biblatex styles.
%%

%%
%% The majority of ACM publications use numbered citations and
%% references.  The command \citestyle{authoryear} switches to the
%% "author year" style.
%%
%% If you are preparing content for an event
%% sponsored by ACM SIGGRAPH, you must use the "author year" style of
%% citations and references.
%% Uncommenting
%% the next command will enable that style.
%%\citestyle{acmauthoryear}

% ====================
%TC:ignore
\begin{table*}[ht]
\centering
\begin{tabular}{@{} l p{11cm} ccc @{}}
\toprule[.1em]
No.    &  Instruction prompts & $\tau$=0.1 & $\tau$=0.5 & $\tau$=0.9 \\ \midrule[.1em]
1 & Make this sound more fluent:~\bn \bn~\sent & 0.314 & 0.301 & 0.266 \\ \midrule
2 & Update to fix all grammatical and spelling errors:~\bn \bn~\sent & 0.368 & 0.355 & 0.330 \\ \midrule
3 & Improve the grammar of this text:~\bn \bn~\sent & 0.494 & 0.486 & 0.459 \\ \midrule
4 & Correct this to standard English:~\bn \bn~"\sent" & 0.503 & 0.500 & 0.486 \\ \midrule
5 & Act as an editor and fix the issues with this text:~\bn \bn~\sent & 0.516 & 0.505 & 0.494 \\ \midrule
6 & Original sentence:~\sent~\bn~Corrected sentence: & 0.552 & 0.547 & 0.533 \\ \midrule
7 & Correct this to standard English:~\bn \bn~\sent & 0.559 & 0.554 & 0.542 \\ \midrule
8 & Correct the following to standard English:~\bn \bn~Sentence:~\sent~\bn~Correction: & 0.569 & 0.564 & 0.551 \\ \midrule
9 & Fix the errors in this sentence:~\bn \bn~\sent & 0.569 & 0.566 & 0.554 \\ \midrule
10 & Reply with a corrected version of the input sentence with all grammatical and spelling errors fixed. If there are no errors, reply with a copy of the original sentence.~\bn \bn~Input sentence:~\sent~\bn~Corrected sentence: & \textbf{0.582} & 0.581 & 0.577 \\
\bottomrule
\end{tabular}
\caption{Performance of different GPT-3 prompts and temperature parameter combinations in a zero-shot GEC setting. All scores are GLEU scores on the JFLEG development set. \sent~represents a source sentence. \bn~ represents a line break. The bold number indicates the best-performing combination.}
\vspace{-0.0cm}
\label{tab:prompts}
\end{table*}

%TC:endignore
% ====================

%% end of the preamble, start of the body of the document source.
\begin{document}

%% The "title" command has an optional parameter,
%% allowing the author to define a "short title" to be used in page headers.
%TC:ignore
\title[%
    % Prompting for Text-to-Image Generation: An Investigation into a Novel Creative Skill
]{%
    Prompting AI Art: An Investigation into the Creative Skill of Prompt Engineering
}%
%TC:endignore

%TC:ignore
%% The "author" command and its associated commands are used to define
%% the authors and their affiliations.
%% Of note is the shared affiliation of the first two authors, and the
%% "authornote" and "authornotemark" commands
%% used to denote shared contribution to the research.
% \author{Jonas Oppenlaender (\href{https://orcid.org/0000-0002-2342-1540}{0000-0002-2342-1540})}
% % \authornote{Both authors contributed equally to this research.}
% \email{jonas.x1.oppenlander@jyu.fi}
% \orcid{0000-0002-2342-1540}
% % \authornotemark[1]
% \affiliation{%
%   \institution{University of Jyväskylä}
%   \city{Jyväskylä}
%   \country{Finland}
%   % \streetaddress{Seminaarinkatu 15}%
%   % \postcode{40014}
% }

% \author{Rhema Linder (\href{https://orcid.org/0000-0003-4720-6818}{0000-0003-4720-6818})}
% \email{rlinder@utk.edu}
% \orcid{0000-0003-4720-6818}
% \affiliation{%
%   \institution{University of Tennessee}
%   \city{Knoxville}
%   \state{Tennessee}
%   \country{United States}
%   % \streetaddress{Seminaarinkatu 15}%
%   % \postcode{40014}
% }

% \author{Johanna Silvennoinen (\href{https://orcid.org/0000-0002-0763-0297}{0000-0002-0763-0297})}
% \email{johanna.silvennoinen@jyu.fi}
% \orcid{0000-0002-0763-0297}
% \affiliation{%
%   \institution{University of Jyväskylä}
%   \city{Jyväskylä}
%   \country{Finland}
%   % \streetaddress{Seminaarinkatu 15}%
%   % \postcode{40014}
% }
%TC:endignore


% \author{Lars Th{\o}rv{\"a}ld}
% \affiliation{%
%   \institution{The Th{\o}rv{\"a}ld Group}
%   \streetaddress{1 Th{\o}rv{\"a}ld Circle}
%   \city{Hekla}
%   \country{Iceland}}
% \email{larst@affiliation.org}

%% By default, the full list of authors will be used in the page
%% headers. Often, this list is too long, and will overlap
%% other information printed in the page headers. This command allows
%% the author to define a more concise list
%% of authors' names for this purpose.
% \renewcommand{\shortauthors}{Trovato et al.}

%%
%% The abstract is a short summary of the work to be presented in the
%% article.
% \begin{abstract}%
% \end{abstract}%

%TC:ignore
%% http://dl.acm.org/ccs.cfm
% \begin{CCSXML}
% <ccs2012>
%   <concept>
%       <concept_id>10010405.10010469.10010470</concept_id>
%       <concept_desc>Applied computing~Fine arts</concept_desc>
%       <concept_significance>300</concept_significance>
%       </concept>
%   <concept>
%       <concept_id>10003120.10003121.10003124</concept_id>
%       <concept_desc>Human-centered computing~Interaction paradigms</concept_desc>
%       <concept_significance>500</concept_significance>
%       </concept>
%  </ccs2012>
% \end{CCSXML}
% \ccsdesc[300]{Applied computing~Fine arts}
% \ccsdesc[500]{Human-centered computing~Interaction paradigms}
%
%% Keywords. The author(s) should pick words that accurately describe
%% the work being presented. Separate the keywords with commas.
% \keywords{prompting, prompt engineering, text-to-image generation, visual art, AI art, creativity}%
%TC:endignore
%
\maketitle%
%
%%%%%%%%%%%%%%%%%%%%%%%%%%%%%%%%%%%%%%%%%%%%
%%%%%%%%%%%%%%%%%%%%%%%%%%%%%%%%%%%%%%%%%%%%

%%% Front page

% ====================
\section*{Abstract}%
% ====================
%
Humankind is entering a novel era of creativity --- an era in which anybody can synthesize digital content.
The % interactional
paradigm under which this revolution takes place is prompt-based learning (or in-context learning).
This paradigm was first applied to adapt large language models (LLMs) to diverse downstream tasks without the need for retraining the language model. % in the field of Natural Language Processing (NLP).
The paradigm also found fruitful application in text-to-image generation where it is being used to synthesize digital images from zero-shot text prompts in natural language for the purpose of creating AI art.
% Practitioners of prompting for text-to-image generation sometimes call themselves `prompt engineers.'
% In this context,
This activity is referred to as prompt engineering --- the practice of iteratively crafting prompts
% using specific keywords
to generate and improve images.
%
In this paper, we investigate prompt engineering as a novel creative skill for creating prompt-based art.
In three studies with participants recruited from a crowdsourcing platform, we explore whether untrained participants could 1) recognize the quality of prompts, 2) write prompts, and 3) improve their prompts.
% We investigate whether prompt engineering is a skill that participants apply intuitively or whether it is a learned skill that requires expertise.
% We investigate whether laypeople apply the skill intuitively or whether it is a learned skill that is acquired from experimentation and practice.
% We investigate this research question in four studies with participants ($n=355$) recruited from a crowdsourcing platform.
% In three studies, we investigated whether participants could recognize the quality of prompts (Study 1), write prompts (Study 2), and improve prompts (Study 3).
Our results indicate that
    % Our first study shed light on whether people have an understanding of what makes a ``good'' prompt.
    % Interestingly, participants were better able to identify the quality of prompts than images.
    % This is surprising, given the difficulty of the task of imagining the visual outcome of a prompt.
    % % The presence of style modifiers may have been an indicator for workers to rate these prompts higher than prompts without prompt modifiers.
    % %%%ChatGPT:
    % The presence of style modifiers may have influenced participants to rate these prompts higher than prompts without style modifiers.
    % %
    % Our findings nevertheless indicate that
participants could assess the quality of prompts and respective images. This ability increased with the participants' experience and interest in art.
%
Participants further were able to write prompts in rich descriptive language.
% , but did not apply the expert vocabulary needed for producing images of a certain style of quality.
    % and some participants had a good grasp of what makes a prompt successful.
However, even though participants were specifically instructed to generate artworks, participants' prompts were missing the specific vocabulary needed to
% turn images into artworks of
apply a certain style to the generated images.
Our results suggest that prompt engineering is a % non-intuitive
learned skill that requires expertise and practice.
Based on our findings and experience with running our studies with participants recruited from a crowdsourcing platform, we provide ten recommendations for conducting experimental research on text-to-image generation and prompt engineering with a paid crowd.
Our studies offer a deeper understanding of prompt engineering thereby opening up avenues for research on the future of prompt engineering.
We conclude by speculating on four possible futures of prompt engineering.
% and how crowd workers would be affected.
% In this paper, we present an exploration of the potential for crowd workers to contribute to this novel creative economy.
% In four studies with 355 crowd workers, we explore workers' knowledge of Fine Art, understanding of prompt quality, and practical ability to write and improve prompts for text-to-image systems with a specific focus on generating digital artworks.
% Our results indicate that crowd workers can write prompts in rich descriptive language and some workers have a good grasp of what makes a prompt successful.
% We also investigate whether prompt engineering is a skill that we humans apply intuitively or whether it is a learned skill that requires expertise. We conclude by discussing four possible futures of prompt engineering and how crowd workers would be affected.
%
\\[2\baselineskip]
\textbf{Keywords:} prompt engineering, prompting, text-to-image generation, AI art, creativity

\end{document}
\endinput
%% End of file `sample-acmsmall-submission.tex'.
