%TC:ignore
\noindent%
\begin{figure}[h]%
\centering%
  % \includegraphics[width=\linewidth]{figures/boxplots/study34-comparison-boxplots.pdf}
  % \includegraphics[width=\linewidth]{figures/study34/changes.pdf}
%   \\
   % \includegraphics[width=\linewidth]{figures/boxplots/study34-boxplots.pdf}
\begin{subfigure}[b]{0.42\textwidth}
    % \includegraphics[width=\linewidth]{figures/study34/changes.pdf}
    \includegraphics[width=.905\linewidth]{figures/study34/tokens_added_removed.pdf}%
    \subcaption{%
        \centering
        Histogram of changes in tokens
    }%
    \label{fig:study4:changes:histograms}%
\end{subfigure}%
\begin{subfigure}[b]{0.42\textwidth}%
    \includegraphics[width=\linewidth]{figures/study34/study34-co-occurrence-matrix.pdf}%
    \subcaption{%
        \centering
        Co-occurrence matrix of changes
    }%
    \label{fig:study4:changes:cooccurence}%
\end{subfigure}%
\caption{The crowd workers added more tokens than they removed in Study 3 (\autoref{fig:study4:changes:histograms}).
\autoref{fig:study4:changes:cooccurence} depicts the changes that often co-occurred with one another. For instance, a change (addition or removal) to an adjective often co-occurred with changes to other adjectives in the prompt.}
\Description{Histograms of changes in tokens and co-occurence matrix of changes made by crowd workers to their prompts in Study 3. The graphs show that more tokens were added than removed and most workers made changes to adjectives.}
\label{fig:study34:changes}%
\end{figure}%
%TC:endignore
