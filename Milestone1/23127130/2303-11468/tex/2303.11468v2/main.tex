\documentclass{compositio}
\usepackage [T1]{fontenc}
\usepackage [utf8]{inputenc}
\usepackage [english]{babel}
\usepackage {amsfonts}
\usepackage {amssymb}
\usepackage {mathbbol}

\usepackage {amsthm}
\usepackage {amsopn}
\usepackage {mathtools}
\usepackage {tikz-cd}
\usepackage{csquotes}
\usepackage{xfrac}
\usepackage{faktor}
\usepackage{dsfont}
\usepackage{bm}

\usepackage{smartdiagram}

  \newtheorem{innercustomprop}{Proposition}
\newenvironment{customprop}[1]
  {\renewcommand\theinnercustomprop{#1}\innercustomprop}
  {\endinnercustomprop}
  
\newcommand{\catname}[1]{{\normalfont\textbf{#1}}}
\newcommand{\C}{{\mathbb{C}_\infty}}
\newcommand{\CC}{{\mathbb{C}_\F}}
\newcommand{\K}{{K_\infty}}
\newcommand{\KK}{{K_v}}
\newcommand{\FF}{{\F_q^{ac}}}
\newcommand{\F}{\mathbb{F}}
\newcommand{\T}{\mathbb{T}}
\newcommand{\N}{\mathbb{N}}
\newcommand{\Q}{\mathbb{Q}}
\newcommand{\Z}{\mathbb{Z}}
\renewcommand{\P}{\mathbb{P}}
\newcommand{\m}{\mathfrak{m}}
\newcommand{\mm}{\tilde{\mathfrak{m}}}
\newcommand{\p}{\mathfrak{p}}
\newcommand{\q}{\mathfrak{q}}
\newcommand{\A}{\mathcal{A}}
\renewcommand{\O}{\mathcal{O}}
\newcommand{\OL}{\widehat{\Z_p[p^{\frac{1}{p^\infty}}]}}
\renewcommand{\L}{\mathcal{L}}
\newcommand{\I}{\mathcal{I}}
\newcommand{\G}{\text{Gal}}
\newcommand{\red}{\text{red}}
\newcommand{\Div}{\text{Div}}
\newcommand{\Pic}{\text{Pic}}
\newcommand{\Spec}{\text{Spec}}
\newcommand{\Frob}{\text{Frob}}
\DeclareMathOperator{\Frac}{Frac}
\newcommand{\ch}{\text{char}}
\DeclareMathOperator{\Hom}{Hom}
\DeclareMathOperator{\res}{res}
\DeclareMathOperator{\Lie}{Lie}
\DeclareMathOperator{\Span}{Span}
\DeclareMathOperator{\End}{End}
\DeclareMathOperator{\Aut}{Aut}
\DeclareMathOperator{\Tr}{Tr}
\DeclareMathOperator{\Ext}{Ext}
\DeclareMathOperator{\Tor}{Tor}
\DeclareMathOperator{\coker}{coker}
\DeclareMathOperator{\im}{im}
\DeclareMathOperator{\rk}{rk}
\DeclareMathOperator{\Mat}{Mat}
\DeclareMathOperator{\GL}{GL}
\newcommand{\Mod}{\catname{Mod}}
\newcommand{\into}{\hookrightarrow}
\newcommand{\onto}{\twoheadrightarrow}
\renewcommand{\baselinestretch}{1.0}
\newtheorem{teo}{Theorem}[section]
\newtheorem{Teo}{Theorem}
\newtheorem{lemma}[teo]{Lemma}
\newtheorem{fact}[teo]{Fact}
\newtheorem{prop}[teo]{Proposition}
\newtheorem{cor}[teo]{Corollary}
\newtheorem{conj}[teo]{Conjecture}
\theoremstyle{definition}
\newtheorem{Def}[teo]{Definition}
%\theoremstyle{remark}
\newtheorem{con}[teo]{Construction}
\theoremstyle{remark}
\newtheorem{oss}[teo]{Remark}
\theoremstyle{remark}
\newtheorem{ex}[teo]{Example}



\newcommand\blankpage{%
    \null
    \thispagestyle{empty}%
    \addtocounter{page}{-1}%
    \newpage}
\usepackage[
backend=biber,
style=alphabetic,
sorting=nyt
]{biblatex}
\addbibresource{main.bib}

\begin{document}
\title{Special functions and dual special functions in Drinfeld modules of arbitrary rank}
\author{Giacomo Hermes Ferraro}

\email{giacomohermes.ferraro@uniroma1.it}
\address{Sapienza Università di Roma\\Piazzale Aldo Moro, 5\\00183, Rome, Italy}

\classification{11G09}
\keywords{Drinfeld modules, Pellarin $L$-series, shtuka functions, special functions}
\thanks{This work has been produced as part of the PhD thesis of the author at the Department of Mathematics Guido Castelnuovo, under the supervision of Federico Pellarin.}


\begin{abstract}
    In the context of Drinfeld-Hayes modules over any curve $X/\F_q$ with a point $\infty\in X(\F_q)$, the author proved in a previous paper that the product between a so-called "special function" and a zeta-like function à la Pellarin is rational over $X_\C$, and that the latter can be interpreted as a dual special function. Both results are generalized in this paper to Drinfeld modules of arbitrary rank.

    We use a very simple functorial point of view to interpret special functions and dual special functions in a Drinfeld module $\phi$ of arbitrary rank, allowing us to define a universal special function $\omega_\phi$ and a universal dual special function $\zeta_\phi$. In analogy to the Drinfeld-Hayes case, we prove that the latter can be expressed as an Eisenstein-like series over the period lattice, and that the scalar product $\omega_\phi\cdot\zeta_\phi$, an element of $\C\hat\otimes\Omega$, is rational; although - unlike in the case of rank 1 - we are not able to fully recover its divisor.
    
    We also describe the module of special functions in the generality of Anderson modules, as already done by Gazda and Maurischat, and answer a question they posed about the invertibility of special functions in the context of Drinfeld-Hayes modules.
\end{abstract}
\maketitle

\section{Introduction}
Let $\F_q$ be the finite field with $q$ elements, and let $(X,\O_X)$ be a projective, geometrically irreducible, smooth curve of genus $g$ over $\F_q$, with a point $\infty\in X(\F_q)$. We define $A:=\O_X(X\setminus\{\infty\})$, with the degree function $\deg:A\to\mathbb{N}$ being the opposite of the valuation at $\infty$; $\Omega$ is the module of K\"ahler differentials of $A$ over $\F_q$, $K$ is the fraction field of $A$, $\K$ is the completion of $K$ with respect to the $\infty$-adic norm $\|\cdot\|$, and $\C$ is the completion of an algebraic closure $K_\infty^{ac}$. By convention, for all $a\in A$, $\|a\|=q^{\deg(a)}$.

Let $\C[\tau]\subseteq\End_{\F_q}(\C)$ be the sub-$\C$-algebra generated by the Frobenius endomorphism. A \textit{Drinfeld module} $\phi$ of rank $r$ is a ring homomorphism $A\to\C[\tau]$ such that, for all $a\in A$, $\phi_a$ has degree $r\cdot\deg(a)$ in $\tau$, and its constant term is $a$. There is a unique $A$-module $\Lambda\subseteq\C$ of rank $r$, with an associated (surjective) exponential map $\exp_\Lambda:=x\prod_{\lambda\in\Lambda\setminus\{0\}}\left(1-\frac{x}{\lambda}\right)\in\C[[x]]$, such that $\phi_a\circ\exp_\Lambda(x)=\exp_\Lambda(ax)$ in $\C[[x]]$.

The special case of the \textit{Carlitz module}, with $A=\F_q[\theta]$ and $\phi_\theta=\theta+\tau$, is the easiest to study. The Anderson-Thakur special function $\omega$, introduced in \cite{AndersonThakur}, is defined as the unique series $\sum_{i\geq0}c_it^i$ in the Tate algebra $\C\langle t\rangle$ such that $c_0=1$ and
\[\sum_{i\geq0}\phi_\theta(c_i)t^i=\sum_{i\geq0}c_i t^{i+1}.\]
It holds a lot of information: for example, as shown in \cite{AP14}, $\omega$ is connected to the explicit class field theory of $\F_q(\theta)$, and its $\F_q^{ac}$-rational values interpolate Gauss-Thakur sums. In \cite{Pellarin2011}, Pellarin proved the following identity in $\C\langle t\rangle$, relating $\omega$ to a zeta-like function:
\[\frac{\tilde{\pi}}{(t-\theta)\omega}=\sum_{h\in\F_q[T]\setminus\{0\}}\frac{h(t)}{h(\theta)},\] where $\tilde{\pi}$ is the \textit{fundamental period} of the Carlitz module.

The module of "special functions" (as defined in \cite{ANDTR}) generalizes the Anderson-Thakur function to any Drinfeld module $\phi$: first, we endow $\C\otimes A$ with the sup norm induced by $\|\cdot\|$ (for any $\F_q$-basis $(a_i)_{i\in I}$ of $A$, we can set $\|\sum_i c_i\otimes a_i\|:=\sup(\{\|c_i\|\}_i)$, which is actually independent of the chosen basis), and denote by $\C\hat\otimes A$ its completion; the module of special functions $Sf_\phi(A)$ is defined as the following submodule:
\[\{\omega\in\C\hat\otimes A|\phi_a(\omega)=(1\otimes a)\omega\;\forall a\in A\},\]
where $\phi_a$ sends an infinite sum $\sum_i c_i\otimes a_i$ to $\sum_i\phi_a(c_i)\otimes a_i$. The Pellarin zeta function has an obvious generalization as an object of $\C\hat\otimes A$: $\zeta_A:=\sum_{a\in A\setminus\{0\}}a^{-1}\otimes a$; it was conjectured that the relation between Pellarin zeta functions and special functions could be generalized at least to Drinfeld modules of rank $1$.

Until recently, the only result in this direction had been a paper by Green and Papanikolas (\cite{Green}), who studied the case of genus $g(X)=1$ with some assumptions on the Drinfeld divisor, exploiting the group structure of elliptic curves. In a previous paper (\cite{Ferraro}), the author proved for all Drinfeld modules of rank $1$ that the product between a special function and the Pellarin zeta function is in the fraction field of $\C\otimes A$, and explicitly described its divisor and the scalar proportionality constant (\cite{Ferraro}[Thm. 6.3]). In the same paper, the following identities are also proved for all $a\in A$ : $\phi^*_a(\zeta_\phi)=(1\otimes a)\zeta_\phi$, where $\phi^*:A\to\C[\tau^{-1}]$ is the dual Drinfeld module and $\zeta_\phi$ is a slight modification of $\zeta_A$ (see \cite{Ferraro}[Prop. 7.7]); this allows us to interpret $\zeta_\phi$ as a dual special function.

This paper generalizes the main results of \cite{Ferraro} to a Drinfeld module of arbitrary rank by defining higher rank analogues of the special and dual special functions, and proving the rationality of their scalar product, which is an element of $\C\hat\otimes\Omega$. It turns out that the algebra $\C\hat\otimes A$ is not the natural environment for (dual) special functions, and a functorial point of view is needed both to define a "canonical" (dual) special function and to formulate the correct generalization of \cite{Ferraro}[Thm 6.3]. What this approach gains in naturality and generality it loses in precision: contrary to \cite{Ferraro}, in this paper we are not able to describe explicitly the divisor of the resulting form.

The starting observation in this paper is that (dual) special functions - for a Drinfeld module $\phi$ with period lattice $\Lambda$ - can be easily defined as elements of $\C\hat\otimes M$ for any $A$-module $M$, and that this definition is functorial in $M$. In Section \ref{section Pontryagin duality}, we provide a crucial alternative interpretation for the "completed tensor product" $\hat\otimes$: we prove that $\C\hat\otimes M$ is canonically isomorphic to the set of continuous $\F_q$-linear functions from the Pontryagin dual of $M$, $\hat{M}$, to $\C$ (Lemma \ref{duality tensor hom 2}). The submodule of special functions is then $\Hom_A^{cont}(\hat{M},\C^\phi)$, which we prove in Section \ref{section special functions} to be representable: it turns out that the universal object $\omega_\phi$, which we call \textit{universal special function}, lies in $\Hom_A^{cont}\left(\faktor{\K\Lambda}{\Lambda},\C^\phi\right)\subseteq\C\hat\otimes\Hom_A(\Lambda,\Omega)$, and is precisely the exponential function.

Similarly, in Section \ref{section zeta functions}, we prove that the functor of dual special functions is represented by $\Lambda$, and that the \textit{universal dual special function} $\zeta_\phi\in\C\hat\otimes\Lambda$ is the Eisenstein-like series $\sum_{\lambda\in\Lambda\setminus\{0\}}\lambda^{-1}\otimes\lambda$, which is the correct generalization of the Pellarin zeta function. 

Finally, in Section \ref{section pairing}, we define the canonical pairing $ev:(\C\hat\otimes\Lambda)\otimes(\C\hat\otimes\Hom_A(\Lambda,\Omega))\to\C\hat\otimes\Omega$ and prove that $ev(\zeta_\phi\otimes\omega_\phi)$ is in fact a rational differential form. Moreover, in the case $A=\F_q[\theta]$, we also manage to prove that this form is $d\theta\in\Omega$ for any Drinfeld module $\phi$.

Since Gazda and Maurischat were able to describe the module of special functions in the full generality of Anderson modules in \cite{Gazda}, in Section \ref{section special functions} we work in the same generality and come to some of the same conclusions. Moreover, in Section \ref{section question}, we answer affirmatively to a question they posed in their paper about the invertibility of special functions in the context of Drinfeld modules of rank $1$. Both of those results, while not necessary for the proof of the generalization of \cite{Ferraro}[Thm. 6.3], are included to show the explanatory power of the language used in this paper.


\section{Pontryagin duality of $A$-modules}\label{section Pontryagin duality}

\subsection{Basic statements about Pontryagin duality}

In this paper, compact and locally compact spaces are always assumed to be Hausdorff.

\begin{Def}[(Pontryagin duality)]
    Call $\mathbb{S}^1\subseteq\mathbb{C}^\times$ the unit circle. The \textit{Pontryagin duality} is a contravariant functor from the category of topological abelian groups to itself, sending $M$ to the set of continuous morphisms of $\Z$-modules $\hat{M}:=\Hom_\Z^{cont}(M,\mathbb{S}^1)$, endowed with the compact open topology.
\end{Def}
\begin{oss}
    If $M$ is an $A$-module, we can endow $\hat{M}$ with a natural structure of $A$-module. Moreover, since $M$ is an $\F_q$-vector space, we have the following natural isomorphisms of topological $A$-modules:
    \[\hat{M}:=\Hom^{cont}_\Z(M,S^1)=\Hom^{cont}_{\F_p}(M,\F_p)\cong\Hom^{cont}_{\F_q}(M,\Hom_{\F_p}(\F_q,\F_p))\cong\Hom^{cont}_{\F_q}(M,\F_q).\]
\end{oss}

The following well known result, adapted to the case of $A$-modules and which we do not prove, justifies the terminology "duality".
\begin{prop}\label{Pontryagin duality}
    For any topological $A$-module $M$, there is a natural morphism $M\to\hat{\hat M}$; if $M$ is locally compact, $\hat{M}$ is locally compact, and the previous morphism is an isomorphism.

    Moreover, if $M$ is compact (resp. discrete) $\hat{M}$ is discrete (resp. compact).
\end{prop}

Let's fix some notation.

\begin{Def}\label{complete tensor product}
Let $M$ and $N$ be topological $\F_q$-vector spaces with $N$ locally compact. We denote $M\hat\otimes N:=\Hom_{\F_q}^{cont}(\hat{N},M)$.
\end{Def}

\begin{lemma}\label{duality tensor hom 1}
    For any pair of locally compact $A$-modules $M,N$, there is a natural isomorphism of $A\otimes A$-modules between $M\hat\otimes N$ and $N\hat\otimes M$.
\end{lemma}
\begin{proof}
    By Proposition \ref{Pontryagin duality}, the Pontryagin duality induces an antiequivalence of the category of locally compact $A$-modules with itself, hence:
    \[\Hom_{\F_q}^{cont}(\hat N,M)\cong\Hom_{\F_q}^{cont}(\hat{M},\hat{\hat N})\cong\Hom_{\F_q}^{cont}(\hat M,N);\]
    the $A\otimes A$-linearity is a simple check.
\end{proof}

We introduce some other useful terminology.

\begin{oss}
    For any discrete $\F_q$-vector space $M$, an isomorphism $\F_q^{\oplus I}\cong M$, i.e. an $\F_q$-basis $(m_i)_{i\in I}$, induces an isomorphism of topological vector spaces between $\F_q^I=\widehat{\F_q^{\oplus{I}}}$ and $\hat{M}$.
\end{oss}
\begin{Def}
    If $M$ is a discrete $\F_q$-vector space with basis $(m_i)_{i\in I}$, we denote for all $i\in I$ $m_i^*$ the image of $(\delta_{i,j})_{j\in I}\in\F_q^I$ via the previous isomorphism, so that for all $j\in I$ $m_i^*(m_j)=\delta_{i,j}$. We call $(m_i^*)_{i\in I}$ the \textit{dual basis} of $\hat{M}$ relative to $(m_i)_{i\in I}$.
\end{Def}

\begin{oss}
    In the previous definition, a generic element $f\in\hat{M}$ corresponds to $(f(m_i))_i\in\F_q^I$. It's immediate to check that, for all $m\in M$, $f(m)=\sum_{i\in I}f(m_i)m_i^*(m)$, which is actually a finite sum, hence we are justified to use the following formal notation: $f=\sum_{i\in I}f(m_i)m_i^*$. The existence and uniqueness of this expression for all $f\in\hat{M}$ explains the terminology "dual basis" for $(m_i^*)_i$.
\end{oss}

\subsection{Application to $A$-modules}

The following (see \cite[Theorem 8]{Poonen}) is a notable result about the Pontryagin duality of $A$-modules.

\begin{teo}[(Poonen)]
    The computation of the residue at $\infty$ induces a perfect pairing between $\Omega\otimes_A\K$ and $\K$, which restricts to a perfect pairing between the discrete $A$-module $\Omega$ and the compact $A$-module $\K/A$. In other words, $\widehat{\Omega\otimes_A\K}\cong\K$ and $\hat\Omega\cong\faktor{\K}{A}$.   
\end{teo}

\begin{oss}\label{oss dual}
    For any discrete projective $A$-module $\Lambda$ of finite rank $r$ with $\Lambda^*:=\Hom_A(\Lambda,A)$, we have the following isomorphisms of topological $A$-modules:
    \[\widehat{\Lambda^*\otimes_A\Omega}=\Hom_{\F_q}(\Lambda^*\otimes_A\Omega,\F_q)\cong\Hom_A(\Lambda^*,\Hom_{\F_q}(\Omega,\F_q))\cong\Lambda\otimes_A(\K/A).\]
    Retracing the isomorphisms, it's easy to check that the pairing map between $\Lambda^*\otimes_A\Omega$ and $\Lambda\otimes_A\K/A$ sends $(\lambda^*\otimes\omega,\lambda\otimes b)$ to the residue of $\omega\lambda^*(\lambda)b$ at $\infty$.
\end{oss}

%\begin{lemma}\label{residue equivariance}Denote $res_A$ the pairing between $\K/A$ and $\Omega$ and $res_\Lambda$ the pairing between $\Lambda\otimes_A\K/A$ and $\Lambda^*\otimes_A\Omega$. For all $\mu\in\Lambda^*\otimes_A\Omega$, for all $b\in\K/A$, for all $\lambda\in\Lambda$, we have:\[res_A(b,\mu(\lambda))=res_\Lambda(\lambda\otimes b,\mu).\]\end{lemma}


%\begin{proof}For any map $f:N\to \hat{M}$ define $\Phi(f):M\to \hat{N}$ as follows: for all $m\in M$ and $n\in N$, $\Phi(f)(m)$ sends $n$ to $f(n)(m)$.We have:\begin{align*}\Phi((a\otimes b)f)(m)(n)&=((a\otimes b)f)(n)(m)=(a(f(bn)))(m)=f(bn)(am)\\&=\Phi(f)(am)(bn)=((b\otimes a)\Phi(f))(m)(n),\end{align*}which proves $A\otimes A$-linearity. Suppose that $f$ is continuous and fix an open neighborhood of $0$ of $\hat{N}=\Hom_{\F_q}^{cont}(N,\F_q)$ of the form $V(K):=\{f|f(K)=0\}$ for some $K\subseteq N$ compact: it suffices to prove that its preimage is open. We have:\[\Phi(f)^{-1}(V(K))=\{m\in M|\Phi(f)(m)\in V(K)\}=\{m\in M|\forall n\in K \Phi(f)(m)(n)=0\}=\{m\in M|\forall n\in K f(n)(m)=0\}\]For all $g:M\to\hat{N}$, define $\Psi(g):N\to\hat{M}$ such that, for all $m\in M$ and $n\in N$, $\Psi(g)(n)$ sends $m$ to $g(m)(n)$; we have:\[\Phi(\Psi(g))(m)(n)=\Psi(g)(n)(m)=g(m)(n),\]hence $\Phi(\Psi(g))=g$. Similarly, for all $f:N\to\hat{M}$, $\Psi(\Phi(f))=f$.\end{proof}$M\hat\otimes N\to\hat{M}\hat\otimes\hat{N}$ sending to the function $m\mapsto\sum_i m_i^*(m)n_i$. The map $\Phi$ is obviously $A\otimes A$-linear, and it is well defined because for all $m\in M$ there is only a finite number of $i$'s such that $m_i^*(m)\neq0$.If $\Phi(x)\equiv0$ we have $n_i=\Phi(x)(m_i)=0$ for all $i$, hence $x=0$. Viceversa, for all $f\in\Hom_{\F_q}(M,N)$, if we define $x:=\sum_i m_i^*\otimes f(m_i)$, we have $\Phi(x)(m_i)=f(m_i)$ for all $i$, hence $\Phi(x)=f$. We deduce that $\Phi$ is bijective. Finally, if $M=N$, $\Phi$ sends the element $\sum_i m_i^*\otimes m_i$ to the identity.
We now show an alternative way to think about the tensor product $\hat\otimes$, which makes our notation agree with the usual notation for the Tate algebra used in \cite{Gazda}, \cite{Ferraro}, and others.

\begin{Def}\label{Def complete space}
    Let $C$ be a complete topological $\F_q$-vector space with open subspaces $(U_j)_{j\in J}$ as a fundamental system of open neighborhoods of $0$, and let $M$ be a discrete $\F_q$-vector space. We denote $C\tilde{\otimes}M$ the completion of $C\otimes M$ with respect to $(U_j\otimes M)_{j\in J}$.
\end{Def}
\begin{oss}\label{alternative condition on C}
    For all $j\in J$, since $U_j$ is an $\F_q$-vector space, for all $y\in C\setminus U_j$, $(y+U_j)\cap U_j=\emptyset$, hence $U_j$ is also closed and $C/U_j$ is discrete. This means that an $\F_q$-vector space satisfies the conditions of Definition \ref{Def complete space} for $C$ if and only if it is the limit of discrete $\F_q$-vector spaces; for example, any compact $\F_q$-vector space does.
    
    Moreover, being a nonarchimedean complete normed vector space over $\F_q$, $\C$ satisfies the conditions on $C$, as we can set $J=\Z$ and $U_n$ the ball of radius $\frac{1}{q^n}$ for all integers $n$. 
\end{oss}


\begin{lemma}\label{duality tensor hom 2}
    Let $C$ and $M$ be as in Definition \ref{Def complete space}. There is a natural isomorphism of $\F_q$-vector spaces $\Phi:C\tilde\otimes M\to C\hat\otimes M$. Moreover, if $C$ and $M$ are $A$-modules, it is an isomorphism of $A\otimes A$-modules.
\end{lemma}
\begin{proof}
    Fix an $\F_q$-basis $(m_i)_{i\in I}$ of $M$; any $x\in C\tilde\otimes M$ can be expressed in a unique way as $\sum_{i\in I}c_i\otimes m_i$, where $c_i\in C$ for all $i\in I$, and for all $j\in J$ the set $I_j:=\{i\in I|c_i\not\in U_j\}$ is finite. We define $\Phi(x):\hat{M}\to C$ as follows:
    \[\forall f\in\hat{M},\;\Phi(x)(f):=\lim_{\substack{J\subseteq I\\\#J<\infty}}\sum_{i\in J} f(m_i)c_i,\]
    which is well defined because $C$ is complete and, for all $j\in J$, $I_j$ is finite and $U_j$ is an $\F_q$-vector space. The map $\Phi(x)$ is obviously $\F_q$-linear; for all $j\in J$, $V_j:=\{f\in\hat{M}|f(m_i)=0\;\forall i\in I_j\}\subseteq\hat{M}$ is a neighborhood of $0$, and is contained in $\Phi(x)^{-1}(U_j)$, hence $\Phi(x)$ is also continuous.
    The map $\Phi$ is manifestly $\F_q$-linear, and, if $C$ and $M$ are $A$-modules, $A\otimes A$-linear, so we just need to prove bijectivity.

    On one hand, if $\Phi(x)\equiv0$, we have $0=\Phi(x)(m_i^*)=c_i$ for all $i\in I$, hence $x=0$. On the other hand, if $g:\hat{M}\to C$ is a continuous function, for all $j\in J$ the set $\{i\in I|g(m_i^*)\not\in U_j\}$ is finite because $\hat{M}$ is compact, hence $y:=\sum_i g(m_i^*)\otimes m_i$ is an element of $C\tilde\otimes M$; since $\Phi(y)(m_j^*)=g(m_j^*)$ for all $j$, $\Phi(y)=g$.
\end{proof}

\section{Special functions}\label{section special functions}

\begin{Def}
    An \textit{Anderson $A$-module} $(E,\phi)$ over $\C$ of dimension $d$ consists of a $\C$-group scheme $E\cong\mathbb{G}_{a,\C}^d$ and an $\F_q$-linear action $\phi$ of $A$ over $E$ such that $\Lie\phi_a-a:\Lie(E)\to\Lie(E)$ is nilpotent for all $a\in A$.
\end{Def}

Fix an Anderson $A$-module $(E,\phi)$. The following Proposition sums up various basic results about $(E,\phi)$ (see \cite{Goss}[Thm. 5.9.6] and \cite{Goss}[Lemma 5.1.9] for proofs in the case $A=\F_q[\theta]$).

\begin{prop}
    There is an associated $\F_q$-linear exponential function $\exp_E:\Lie(E)\to E(\C)$ such that $\exp_E\circ\Lie\phi_a=\phi_a\circ\exp_E$ for all $a\in A$; its kernel $\Lambda_E\subseteq\Lie(E)$ is an $A$-module of finite rank (with respect to the $A$-module structure induced by $\Lie\phi$ on $\Lie(E)$).

    Moreover, if we fix an isomorphism $E\cong\mathbb{G}_{a,\C}^d$, the exponential function is an element of $\C^{d\times d}[[\tau]]$, where $\tau$ is the Frobenius endomorphism, and its leading term is the identity matrix.
\end{prop} 

\begin{Def}
    An Anderson $A$-module $(E,\phi)$ is said to be \textit{uniformizable} if $\exp_E$ is surjective.
\end{Def}

\begin{oss}
    Since $E$ and $\mathbb{G}_{a,\C}^d$ are isomorphic $\C$-group schemes, and the group of automorphisms of $\mathbb{G}_{a,\C}^d$ as a $\C$-group scheme is $\GL_{n,\C}$, we can identify the set $E(\C)$ with $\mathbb{G}_{a,\C}^d(\C)=\C^d$ up to an element of $\GL_{n,\C}(\C)$. In particular, $E(\C)$ has a natural structure of finite $\C$-vector space, hence a natural topology coming from $\C$.

    Similarly, $\Lie(E)$ also has a natural structure of finite $\C$-vector space; moreover, by the inverse function theorem applied to $\exp_E$, we get that $\Lambda_E\subseteq\Lie(E)$ is a discrete subset. In light of this remark, $\exp_E$ is a morphism of topological $A$-modules.
\end{oss}

\begin{lemma}\label{K-action on Lie(E)}
    The $A$-module structure of $\Lie(E)$ induced by $\Lie\phi$ can be extended to a a structure of topological vector space over $\K$.
\end{lemma}
\begin{proof}
    Since the endomorphisms $(\Lie\phi_a)_{a\in A\setminus\{0\}}$ commute and are invertible, the ring homomorphism $\Psi:=\Lie\phi$ can be extended uniquely to $K$, and we can fix a basis $\Lie(E)\cong\C^d$ in which, for all $c\in K$, $\Psi_c$ is a triangular matrix with $N_c:=c^{-1}\Psi_c-\mathds{1}_n$ nilpotent - precisely, $N_c^d=0$. We endow any matrix $N\in\End_n(\C)$, with the sub-multiplicative norm $\|N\|:=\max_{i,j}\{\|N_{ij}\|\}$; to extend continuously $\Psi$ to $\K$, it suffices to prove that $\|\Psi_c\|$ tends to $0$ as $\|c\|$ tends to $0$.
    
    Set $I:=\{i\in\N|\exists a\in A \deg(a)=i\}$. There is a finite set $a_1,\cdots,a_n\in A$ such that for all $i\in I$ there is a finite product of $a_j$'s such that its degree is $i$; in particular, if we call it $b_i$, the sequence $(b_i)_{i\in I}$ is an $\F_q$-basis of $A$. If we call $M:=\max\{1,\|N_{a_1}\|,\dots,\|N_{a_n}\|\}$, for all $i$ we have:
    \[\|N_{b_i}\|\leq\max_{e_1,\dots,e_n\in\N}\left\{\prod_{j=1}^n\|N_{a_j}\|^{e_j}\right\}=\max_{e_1,\dots,e_n<d}\left\{\prod_{j=1}^n\|N_{a_j}\|^{e_j}\right\}<M^{nd}.\]
    For any $a\in A\setminus\{0\}$ of degree $i$ we have:
    \[\|N_a\|\leq\max_{j\leq i}\|a^{-1}b_jN_{b_j}\|\leq\max_{j\leq i}\|a\|^{-1}\|b_j\|M^{nd}\leq M^{nd}.\]
    For all $c\in K^\times$, if we write $c=ab^{-1}$ with $a,b\in A\setminus\{0\}$, we have:
    \[\|c^{-1}\Psi_c\|=\|a^{-1}\Psi_a(b^{-1}\Psi_b)^{-1}\|=\left\|(\mathds{1}_n+N_a)\left(\sum_{i=0}^{d-1}N_b^i\right)\right\|\leq M^{nd^2},\]
    which proves the thesis.
    %and $N_t:=\Psi_t-t\cdot id_{\Lie(E)}$ is nilpotent for all $t\in K$. Fix a uniformizer $t$ of $\K$ contained in $K$ and fix a norm $\|\cdot\|$ for $\Lie(E)$; for all positive integers $n$ and for all $c\in\Lie(E)$ we have:\[\left\|\Psi_{t^n}(c)\right\|=\left\|(N_t+t)^n(c)\right\|\leq\max\{q^{k-n}\|N_t^k(c)\||0\leq k<d\},\]which tends to $0$ as $n$ tends to infinity. For any $b\in\K$ we write $b=\sum_{i\in\mathbb{Z}}\lambda_i t^i$ with $\lambda_i\in\F_q$ for all $i$, and set $\Psi'_b:=\sum_{i\in\mathbb{Z}}\lambda_i \Phi_{t^i}$; we just need to prove that, if $b\in K$, $\Psi_b=\Psi'_b$.
\end{proof}

\begin{Def}
    The \textit{functor of special functions} $Sf_\phi:A-\Mod\to A-\Mod$ is defined by sending a discrete $A$-module $M$ to $Sf_\phi(M):=\Hom_A^{cont}(\hat{M},E(\C))\subseteq E(\C)\hat\otimes M$.
\end{Def}

\begin{oss}
    By Lemma \ref{duality tensor hom 2}, our definition of $E(\C)\hat\otimes A$ coincides with the one given in \cite{Gazda}. The $A$-module $Sf_\phi(A)$ is the subset of $E(\C)\hat\otimes A$ of the elements on which the left and right $A$-actions coincide, hence it's the same as the module of special functions defined in \cite{Gazda}.
\end{oss}

The following proposition examines the functor $Sf_\phi$.

\begin{prop}\label{main result}
    Suppose that $E$ is uniformizable. The functor $Sf_\phi$ is naturally isomorphic to $\Hom_A(\Lambda_E^*\otimes_A\Omega,\_)$, and the universal object in $\Hom_A^{cont}\left(\faktor{\K\Lambda_E}{\Lambda_E},E(\C)\right)$ sends $c\in\K\Lambda_E$ to $\exp_E(c)$.
\end{prop}

\begin{proof}
    Since $E$ is uniformizable, $E(\C)$ is isomorphic to $\faktor{\Lie(E)}{\Lambda_E}$ as a topological $A$-module. Endow $\Lie(E)$ with the structure of topological $\K$-vector space described in Lemma \ref{K-action on Lie(E)}; the finite $\K$-vector subspace $\K\Lambda_E\subseteq\Lie(E)$ admits a topological complement $C$, which induces an isomorphism of $A$-modules $E(\C)\cong\faktor{\K\Lambda_E}{\Lambda_E}\bigoplus C$. For any discrete $A$-module $M$, for any $\omega\in Sf_\phi(M)$, its projection $\overline{\omega}$ onto $C\hat\otimes M$ is in $\Hom_A^{cont}(\hat{M},C)$; since $\hat{M}$ is compact, the image of $\overline{\omega}$ must be a compact sub-$A$-module of $C$, but since $C\subseteq\Lie(E)$, for any $c\in C\setminus\{0\}$ the set $A\cdot c$ is unbounded, hence the only compact sub-$A$-module of $C$ is $\{0\}$ and $\overline\omega=0$. We deduce the following natural isomorphisms:
    \[Sf_\phi(M)=\Hom_A^{cont}(\hat{M},E(\C))\cong\Hom_A^{cont}\left(\hat{M},\faktor{\K\Lambda_E}{\Lambda_E}\right)\cong\Hom_A(\Lambda_E^*\otimes_A\Omega,M),\]
    where we used Lemma \ref{duality tensor hom 1} for the second isomorphism.
    
    Setting $M:=\Lambda_E^*\otimes_A\Omega$, and following the identity along the chain of isomorphisms, we deduce that the universal object in $\Hom_A^{cont}\left(\faktor{\K\Lambda_E}{\Lambda_E},E(\C)\right)$ is the continuous $A$-linear map sending $c\in\K\Lambda_E$ to $\exp_E(c)$.
\end{proof}

For the sake of completeness, let's prove a statement which does not assume  uniformizability.

\begin{prop}
    If we restrict the functor $Sf_\phi$ to the subcategory of torsionless $A$-modules, it is naturally isomorphic to $\Hom_A(\Lambda_E^*\otimes_A\Omega,\_)$.
\end{prop}
\begin{proof}
    The map $\exp_E$ is open because its Jacobian at all points is the identity; call $C$ its image. Since $C$ is an $\F_q$-vector space, for all $y\in E(\C)\setminus C$, the open set $y+C$ does not intersect $C$, hence $C$ is also closed. In particular, the quotient $\faktor{E(\C)}{C}$ is a discrete $A$-module.
    
    A discrete $A$-module $M$ is torsionless if and only if it has no nontrivial compact submodules; in this case, $\hat{M}$ is a compact $A$-module with no nontrivial discrete quotients. In particular, any function $f\in Sf_\phi(M)=\Hom_A^{cont}(\hat{M},E(\C))$, projected onto $\faktor{E(\C)}{C}$, is trivial, hence the image of $f$ must be contained in $C$. The rest of the proof is the same as Proposition \ref{main result} up to substituting $E(\C)$ with $C$.
\end{proof}
As a Corollary, we have obtained a variation of \cite{Gazda}[Thm. 3.11].
\begin{cor}
    The following isomorphism of $A$-modules holds:
    \[Sf_\phi(A)=\{\omega\in E(\C)\hat\otimes A|\phi_a(\omega)=(1\otimes a)\omega\;\forall a\in A\}\cong\Omega^*\otimes_A\Lambda_E.\]
\end{cor}

\begin{oss}\label{oss universal special function}
    If we fix an $\F_q$-basis $(\mu_i)_i$ of $\Hom_A(\Lambda_E,\Omega)$, with $\{\mu_i^*\}$ dual basis of $\faktor{\K\Lambda_E}{\Lambda_E}$, by Remark \ref{alternative condition on C} and Lemma \ref{duality tensor hom 2}, we can express the universal object in the following alternative way as an element of $E(\C)\hat\otimes\Hom_A(\Lambda_E,\Omega)$:
    \[\exp_E\left(\sum_i\mu_i^*\otimes\mu_i\right)=\sum_i\exp_E(\mu_i^*)\otimes\mu_i.\]
\end{oss}

%\section{Special functions}\label{section special functions}Fix a Drinfeld-Hayes module $\phi$ with lattice $\Lambda\subseteq\C$. \begin{Def}Denote $\C^\phi$ and $\C^{\phi^*}$ the topological $\F_q$-vector space $\C$ endowed with the $A$-module structure induced by $\phi$ and $\phi^*$, respectively.For any discrete $A$-module $M$, call $Sf_{\phi}(M)$ and $Sf_{\phi^*}(M)$ the sub-$A\otimes A$-modules respectively of $\C^{\phi}\hat\otimes M$ and $\C^{\phi^*}\hat\otimes M$, consisting of the elements on which the left $A$-action coincides with the right $A$-action.\end{Def}The following proposition examines the function $Sf_\phi$. To prove an analogous result about $Sf_{\phi^*}$ (see Proposition \ref{zeta function functor}) we first need some technical lemmas, expressed in Section \ref{section technical lemmas}.\begin{prop}\label{main result}The map associating the discrete $A$-module $M$ to $Sf_\phi(M)$ can be extended to a subfunctor $Sf_\phi$ of the functor $\C^\phi\hat\otimes\_:A-\Mod\to A\otimes A-\Mod$.The functor $Sf_\phi$ is representable by $\Lambda^*\otimes_A\Omega$, and the universal object, considered as an element of $\Hom_{\F_q}^{cont}\left(\faktor{\K\Lambda}{\Lambda},\C^\phi\right)$, sends $c$ to $\exp_\Lambda(c)$.\end{prop}\begin{proof}For the first part, it suffices to prove that, for all $A$-modules $M,N$, for any $A$-linear map $f:M\to N$ and any $\omega\in Sf_\phi(M)$, the element $(1\otimes f)(\omega)$ is contained in $Sf_N(\phi)$; we have, for all $a\in A\setminus\{0\}$:\[(\phi_a\otimes1)\left((1\otimes f)(\omega)\right)=(1\otimes f)\left((\phi_a\otimes1)(\omega)\right)=(1\otimes f)\left((1\otimes a)\omega\right)=(1\otimes a)\left((1\otimes f)(\omega)\right).\]For all $M$ the natural morphism of $A\otimes A$-modules $\exp_\Lambda\otimes id_M:\C\hat\otimes M\to\C^\phi\hat\otimes M$ is surjective with kernel $\Lambda\otimes M$, hence it induces a natural isomorphism between $\faktor{\C}{\Lambda}\hat\otimes M$ and $\C^\phi\hat\otimes M$. By Lemma \ref{duality tensor hom 2}, this is equivalent to the fact that the composition with $\exp_\Lambda$ induces a natural isomorphism of $A\otimes A$-modules between $\Hom^{cont}_{\F_q}\left(\hat{M},\faktor{\C}{\Lambda}\right)$ and $\Hom^{cont}_{\F_q}\left(\hat{M},\C^\phi\right)$.Its inverse induces a natural isomorphism between the functor $Sf_\phi$, considered as a subfunctor of $\Hom_{\F_q}(\widehat{\_},\C^\phi)$, and:\[F:M\mapsto\left\{f\in\Hom^{cont}_{\F_q}\left(\widehat{M},\faktor{\C}{\Lambda}\right)\bigg|(a\otimes1)x=(1\otimes a)x\;\forall a\in A\right\}=\Hom^{cont}_A\left(\widehat{M},\faktor{\C}{\Lambda}\right).\]Note that, if we call $C\subseteq \C$ a complementary $\K$-vector space to $\K\Lambda$, we have a decomposition of $A$-modules $\faktor{\C}{\Lambda}\cong C\oplus \faktor{\K\Lambda}{\Lambda}$, and for all $M$ we have a natural decomposition of $F(M)$ as an $A\otimes A$-module as follows:\[F(M)=\Hom^{cont}_A\left(\widehat{M},\faktor{\K\Lambda}{\Lambda}\right)\oplus\Hom^{cont}_A(\widehat{M},C).\]The second summand is actually a $\K\hat\otimes A$-module, hence, if $y$ is in the second summand, for any $a\in A\setminus\F_q$, we have:\[y=1\cdot y=\left(\left(\sum_{i\geq0}a^{-i}\otimes a^i\right)(1-a^{-1}\otimes a)\right)y=\left(\sum_{i\geq0}a^{-i}\otimes a^i\right)(1-a^{-1}\otimes a)y=0;\]we deduce that $Sf_\phi(M)\cong F(M)=\Hom^{cont}_A\left(\widehat{M},\faktor{\K\Lambda}{\Lambda}\right)$. Since $\faktor{\K\Lambda}{\Lambda}\cong\widehat{\Lambda^*\otimes_A\Omega}$ by Remark \ref{oss dual}, there is a natural isomorphism between $Sf_\phi(M)$ and $\Hom_A(\Lambda^*\otimes_A\Omega,M)$ by Remark \ref{lemma1+lemma2}.Consider the following chain of morphisms of $A\otimes A$-modules:\[\begin{tikzcd}[column sep=scriptsize]\Hom_A(\Lambda^*\otimes_A\Omega,\Lambda^*\otimes_A\Omega)\arrow[r,phantom,"\cong"] & \Hom_A^{cont}\left(\faktor{\K\Lambda}{\Lambda},\faktor{\K\Lambda}{\Lambda}\right)\arrow[r,phantom,"\cong"]\arrow[d, phantom, sloped, "\subseteq"]&Sf_\phi(\Lambda^*\otimes_A\Omega)\arrow[d, phantom, sloped, "\subseteq"]\\&\Hom_{\F_q}^{cont}\left(\faktor{\K\Lambda}{\Lambda},\faktor{\K\Lambda}{\Lambda}\right)\arrow[r,"\exp_\Lambda"] & \Hom_{\F_q}^{cont}\left(\faktor{\K\Lambda}{\Lambda},\C^\phi\right).\end{tikzcd}\]Since the universal object is the image of the identity on $\Lambda^*\otimes_A\Omega$, it corresponds, as an element of $\Hom_{\F_q}^{cont}\left(\faktor{\K\Lambda}{\Lambda},\C^\phi\right)$, to the continuous morphism of $A$-modules $\exp_\Lambda:\faktor{\K\Lambda}{\Lambda}\to\C^\phi$.\end{proof}\begin{oss}If we fix an $\F_q$-basis $\{\mu_i\}_i$ of $\Hom_A(\Lambda,\Omega)$, with $\{\mu_i^*\}$ dual basis of $\faktor{\K\Lambda}{\Lambda}$, by Lemma \ref{duality tensor hom 1}, we can express the universal object in the following alternative way as an element of $\C\hat\otimes\Hom_A(\Lambda,\Omega)$:\[\exp_\Lambda\left(\sum_i\mu_i^*\otimes\mu_i\right)=\sum_i\exp_\Lambda(\mu_i^*)\otimes\mu_i.\]\end{oss}\begin{cor}For all discrete $A$-modules $M$, $Sf_\phi(M)$ is isomorphic to $\Hom_A(\Omega,\Lambda\otimes_A M)$ as an $A\otimes A$-module. \end{cor}


%Fix an ideal $I\unlhd A$, $a_I\in I$ a nonzero element of least degree, $f_*$ the dual shtuka function, $\phi^*:A\to H^{perf}\{\tau\}$ the relative dual Drinfeld module of rank $1$, $\exp$ the relative exponential, with kernel $\frac{\tilde{\pi}_I}{a_I} I$, $\zeta_I:=\sum_{c\in^* I}c^{-1}\otimes c$ the zeta function, $\zeta:=(\frac{a_I}{\tilde{\pi}}\otimes 1)\zeta_I$ so that $\frac{\zeta^{(-1)}}{\zeta}=f_*$ i.e. $\phi_a^*(\zeta)=(1\otimes a)\zeta$ for all $a\in A$.Fix $a\in A$ and consider $b_1,\dots,b_n\in I$ of strictly increasing degree whose projections to $I/aI$ are an $\F_q$-basis; for $i>n$ we define $b_i:=a b_{i-n}$, so that $\{b_i\}_i$ is an $\F_q$-basis of $I$. For $i\geq1$, call $\lambda_i: I\to\F_q$ the maps sending an element to the coefficient of $b_i$. We have:\begin{align*}\phi_a^*(\zeta)&=\sum_{c\in I}\phi_a^*\left(\frac{a_I}{\tilde{\pi} c}\right)\otimes c=\sum_{c\in I}\sum_i \phi_a^*\left(\frac{\lambda_i(c)a_I}{\tilde{\pi} c}\right)\otimes b_i=\sum_i\left(\sum_{c\in I}\phi_a^*\left(\frac{\lambda_i(c)a_I}{\tilde{\pi} c}\right)\right)\otimes b_i;\\(1\otimes a)\zeta&=\sum_{c\in I}\frac{a_I}{\tilde{\pi} c}\otimes ac=\sum_i\left(\sum_{c\in I}\frac{a_I\lambda_i(ac)}{\tilde{\pi} c}\right)\otimes b_i.\end{align*}Since $\phi_a^*(\zeta)=(1\otimes a)\zeta$, for all $i\geq1$ we have:\[\phi_a^*\left(\frac{a_I}{\tilde{\pi}}\sum_{c\in I}\lambda_i(c)c^{-1}\right)=\frac{a_I}{\tilde{\pi}}\sum_{c\in I}\lambda_i(ac)c^{-1}.\]Define $\Lambda:=\frac{\tilde{\pi}}{a_I}I$, i.e. the lattice relative to $\phi$. Define:\[S_a:=\left\{\sum_{c\in\Lambda}\frac{l(c)}{c}\bigg|l\in\Hom_{\F_q}(\Lambda,\F_q)\text{ s.t. for some $n$ }l|_{a^n \Lambda}\equiv 0\right\}.\]The closure of $S_a$ in $\C$ is:\[S:=\left\{\sum_{c\in\Lambda}\frac{l(c)}{c}\bigg|l\in\Hom_{\F_q}(\Lambda,\F_q)\right\}.\]For all linear functions $l:\Lambda\to\F_q$, for all $a\in A$, by continuity of $\phi_a^*$ we have that:\[\phi_a^*\left(\sum_{c\in\Lambda}\frac{l(c)}{c}\right)=\sum_{c\in\Lambda}\frac{l(ac)}{c}.\]\begin{oss}For all $a$, the $q^{\deg(a)}$ zeroes of $\phi_a^*(t)t^{q^{\deg(a)}}\in H^\text{perf}[t]$ are exactly:\[\left\{\sum_{c\in \Lambda}\frac{l(c)}{c}\bigg|l\in\Hom_{\F_q}(\Lambda,\F_q)\lambda|_{aI}\equiv0\right\}.\]\end{oss}This induces a natural claim, by removing the rank condition on $\Lambda$. It implies the following equality in $\C\hat\otimes\Lambda$: \[\sum_{c\in\Lambda}\phi^*_a(c^{-1})\otimes c=\sum_{c\in\Lambda} c^{-1}\otimes ac.\]

\section{Dual special functions}\label{section zeta functions}

\subsection{Poonen's duality}

Fix a finitely generated projective $A$-submodule $\Lambda\subseteq\C$ of rank $r$.
\begin{oss}
    Since $A$ is a Dedekind domain, every finitely generated and torsion-free $A$-module is projective, so the latter condition is superfluous.
\end{oss}

Since the exponential relative to $\Lambda$ is $\F_q$-linear, we can write $\exp_\Lambda=\sum_{i\geq0} e_{\Lambda,i}\tau^i\in\C[[\tau]]$, where $\tau$ is the Frobenius endomorphism of $\C$. Call $\phi$ the Drinfeld module associated to $\exp_\Lambda$.
\begin{Def}
    For all $a\in A$, if $\phi_a=\sum_i a_i\tau^i\in\C[\tau]$, we set:
    \[\phi^*_a:=\sum_i \tau^{-i}a_i=\sum_i a_i^{q^{-i}}\tau^{-i}\in\C[\tau^{-1}];\]
    we define the \textit{dual Drinfeld module} of $\phi$ as the ring homomorphism $\phi^*:A\to\C[\tau^{-1}]$ sending $a$ to $\phi^*_a$.
    
    Similarly, the \textit{dual exponential function} is defined as follows:
    \[\exp_\Lambda^*:=\sum_{i\geq0} \tau^{-i}e_{\Lambda,i}=\sum_{i\geq0} e_{\Lambda,i}^{q^{-i}}\tau^{-i}\in\C[[\tau^{-1}]],\]
    so that for all $a\in A$, $\exp_\Lambda^*\circ\phi^*_a=a\exp_\Lambda^*$.
    
\end{Def}

We follow a construction due to Poonen, who proved a duality result of central importance to this section (\cite{Poonen}[Theorem 10]).
\begin{Def}
    For all $\beta\in\ker(\exp_\Lambda^*)\setminus\{0\}$, call $g_\beta\in\C[[\tau]]$ the unique function such that $\exp_\Lambda^*\circ\beta=g_\beta^*\circ(1-\tau^{-1})$. 
\end{Def}
\begin{oss}
    This definition is well posed for all $\beta\in\ker(\exp_\Lambda^*)\setminus\{0\}$, because $\F_q\subseteq\ker(\exp_\Lambda^*\circ\beta)$. 
    Moreover, since $(1-\tau)\circ g_\beta=\beta\exp_\Lambda$, $g_\beta|_{\Lambda}$ has image in $\F_q$.
\end{oss}

\begin{teo}[(Poonen)]\label{Poonen}
The function $\ker(\exp_\Lambda^*)\to\widehat{\ker(\exp_\Lambda)}$ sending $0$ to $0$ and $\beta\neq0$ to $g_\beta$ is an isomorphism of topological $A$-modules, where $A$ acts via $\phi^*$ on the LHS and via multiplication on the RHS.
\end{teo}

The following proposition, which is proven in Section \ref{section technical lemmas}, can be viewed as an explicit formula for the inverse of Poonen's isomorphism in Theorem \ref{Poonen}.

\begin{prop}\label{identity 1}
    For all $\beta\in\ker(\exp_\Lambda^*)\setminus\{0\}$, the following identity holds in $\C$:
    \[\beta=-\sum_{\lambda\in\Lambda\setminus\{0\}}\frac{g_\beta(\lambda)}{\lambda}.\]
\end{prop}

\subsection{Functor of dual special functions}

We have the following result from \cite{Ferraro}[Prop. 7.7,Prop. 7.17]):
\begin{prop}
    Suppose that $\phi$ is Drinfeld-Hayes module (i.e. a normalized Drinfeld module of rank $r=1$), and let $\zeta_\phi:=-\sum_{\lambda\in\Lambda\setminus\{0\}}\lambda^{-1}\otimes\lambda\in\C\hat\otimes\Lambda$. For all $a\in A\setminus\{0\}$:
    \[\phi^*_a(\zeta_\phi)=(1\otimes a)\zeta_\phi.\]
\end{prop}

In this section, we prove a generalization to Drinfeld modules of arbitrary rank.

\begin{Def}
    Denote $\C^{\phi^*}$ the topological $\F_q$-vector space $\C$ endowed with the $A$-module structure induced by $\phi^*$.

    The \textit{functor of dual special functions} $Sf_{\phi^*}:A-\Mod\to A-\Mod$ is defined by sending a discrete $A$-module $M$ to $Sf_{\phi^*}(M):=\Hom_A^{cont}(\hat{M},\C^{\phi^*})\subseteq \C^{\phi^*}\hat\otimes M$.
\end{Def}

\begin{prop}\label{zeta function functor}
    The functor $Sf_{\phi^*}$ is naturally isomorphic to $\Hom_A(\Lambda,\_)$, and the universal object in $\C^{\phi^*}\hat\otimes\Lambda$ is $-\sum_{\lambda\in\Lambda\setminus\{0\}}\lambda^{-1}\otimes\lambda$.
\end{prop}

\begin{proof}
    The map $\exp_\Lambda^*:\C^{\phi^*}\to\C$ is a continuous $A$-linear morphism; for any $A$-module $M$, it induces a morphism $Sf_{\phi^*}(M)\to\Hom^{cont}_A(\hat{M},\C)$. Fix some $\zeta\in Sf_{\phi^*}(M)$, with image $\overline{\zeta}$: since $\hat{M}$ is compact, the image of $\hat{\zeta}$ must be a compact sub-$A$-module of $\C$, but for any $c\in\C\setminus\{0\}$ the set $A\cdot c$ is unbounded, hence $\overline{\zeta}\equiv0$. We deduce that the image of $\zeta:\hat{M}\to\C^{\phi^*}$ must be contained in $\ker\exp_\Lambda^*$, which by Proposition \ref{Poonen} is isomorphic as a topological $A$-module to $\hat{\Lambda}$; we have the following natural isomorphisms:
    \[Sf_{\phi^*}(M)=\Hom_A^{cont}(\hat{M},\ker\exp_\Lambda^*)\cong\Hom_A^{cont}\left(\widehat{\ker\exp_\Lambda^*},M\right)\cong\Hom_A(\Lambda,M),\]
    where we used Lemma \ref{duality tensor hom 1} for the second isomorphism.

    The universal object $\zeta_\phi\in\C^{\phi^*}\hat\otimes\Lambda$ is given by the natural morphism of Proposition \ref{Poonen} $\psi:\hat{\Lambda}\cong\ker{\exp_\Lambda^*}\subseteq\C^{\phi^*}$, which by Proposition \ref{identity 1} sends $g\in\hat\Lambda$ to $-\sum_{\lambda\in\Lambda\setminus\{0\}}\frac{g(\lambda)}{\lambda}$.

    If we fix an $\F_q$-basis $(\lambda_i)_i$ of $\Lambda$, with $(\lambda_i^*)_i$ dual basis of $\hat\Lambda$, by Remark \ref{alternative condition on C} and Lemma \ref{duality tensor hom 2} we can write:
    \[\zeta_\phi=\sum_i\psi(\lambda_i^*)\otimes\lambda_i
    =\sum_i\left(-\sum_{\lambda\in\Lambda\setminus\{0\}}\frac{\lambda_i^*(\lambda)}{\lambda}\right)\otimes\lambda_i=-\sum_{\lambda\in\Lambda\setminus\{0\},i}\lambda^{-1}\otimes \lambda_i^*(\lambda)\lambda_i=-\sum_{\lambda\in\Lambda\setminus\{0\}}\lambda^{-1}\otimes\lambda.\]
\end{proof}

\begin{cor}
   For all discrete $A$-modules $M$, $Sf_{\phi^*}(M)$ is isomorphic to $\Hom_A(\Lambda,M)$ as an $A\otimes A$-module. In particular, for any $M$ we have the following equality between subsets of $\C\hat\otimes M$:
   \[Sf_{\phi^*}(M)=\left\{\sum_{\lambda\in\Lambda}\lambda^{-1}\otimes l(\lambda)\bigg|l\in\Hom_A(\Lambda,M)\right\}.\]
\end{cor}


\section{Pairing between special functions and dual special functions}\label{section pairing}

\begin{Def}\label{def main result}
    For any Drinfeld module $\phi$ with lattice $\Lambda\subseteq\C$, we define the \textit{universal special function} $\omega_\phi\in\C\hat\otimes\Hom_A(\Lambda,\Omega)$ and the \textit{universal zeta function} $\zeta_\phi\in\C\hat\otimes\Lambda$   as the universal objects of the functors $Sf_\phi$ and $Sf_{\phi^*}$, respectively.
\end{Def}

The following rationality result (a weak version of \cite{Ferraro}[Thm. 6.3]) links zeta functions and special functions in the context of Drinfeld-Hayes modules.

\begin{teo}[(Ferraro)]
Let $\phi$ be a Drinfeld-Hayes module. The product of an element in $Sf_{\phi^*}(A)$ and an element in $Sf_{\phi}(A)$ is rational.
\end{teo}

To generalize this result to Drinfeld modules of arbitrary rank, we need a preliminary lemma.

\begin{lemma}\label{ev}
The following morphism is well defined:
\[\begin{tikzcd}[column sep=scriptsize, row sep=small]
&\C\hat\otimes\Lambda\arrow[r,phantom,"\otimes"] & \C\hat\otimes(\Lambda^*\otimes_A\Omega)\arrow[r,"ev"]&\C\hat\otimes\Omega\\
&\sum_i c_i\otimes\lambda_i\arrow[r,phantom,"\otimes"] 
&\sum_j d_j\otimes(\lambda^*_j\otimes\omega_j)\arrow[r,phantom,"\mapsto"] 
&\sum_{i,j}(c_i d_j)\otimes(\lambda^*_j(\lambda_i)\omega_j)\\
&&f\arrow[u,phantom,sloped,":="]&g\arrow[u,phantom,sloped,":="]
\end{tikzcd}\]
Moreover, considering $g$ and $f$ as continuous functions respectively from $\faktor{\K}{A}$ and $\Lambda\otimes_A\faktor{\K}{A}$ to $\C$, for all $b\in\faktor{\K}{A}$ we have:
\[g(b)=\sum_i c_i f(\lambda_i\otimes b).\]
\end{lemma}
\begin{proof}
The morphism is well defined because for all $\varepsilon>0$ there are finitely many pairs of indices $(i,j)$ such that $\|c_id_j\|>\varepsilon$. Call $\res:\Omega\otimes_{\F_q}\K/A\to\F_q$ and $\res_\Lambda:(\Lambda^*\otimes_A\Omega)\otimes_{\F_q}\left(\Lambda\otimes_A\faktor{\K}{A}\right)\to\F_q$ the two perfect pairings. By Remark \ref{oss dual} we have:
\[g(b)=\sum_{i,j}c_id_j\res(\lambda^*_j(\lambda_i)\omega_j,b)=\sum_i c_i\sum_j d_j\res_\Lambda(\lambda^*_j\otimes\omega_j,\lambda_i\otimes b)=\sum_i c_i f(\lambda_i\otimes b).\]
\end{proof}

\begin{teo}\label{teo}
Fix a Drinfeld module $\phi$ of rank $r$ and period lattice $\Lambda$, and let the morphism $ev$ be defined like in the previous lemma. Then, for all integers $0\leq j<r$, $ev(\zeta_\phi\otimes\omega_\phi^{(j)})$ is a rational differential function over $X_\C$. Moreover, if we denote $J\unlhd \C\otimes A$ the ideal generated by $\{a\otimes1-1\otimes a\}_{a\in A}$, $J\cdot ev(\zeta_\phi\otimes\omega_\phi)$ and $\C\otimes A\cdot ev(\zeta_\phi\otimes\omega_\phi^{(j)})$ are contained in $\C\otimes\Omega$ for $1\leq j\leq r-1$.
\end{teo}
%Qui in realtà vorrei dire che k_1

We assume the following results, which are proven in Section \ref{section technical lemmas}.

\begin{prop}\label{identity 2}
    For all $c\in\K\setminus\{0\}$ with $\|c\|<1$, the following identity holds in $\C$:
    \[c=-\sum_{\lambda\in\Lambda\setminus\{0\}}\frac{\exp_\Lambda(c\lambda)}{\lambda}.\]
\end{prop}

\begin{prop}\label{identity 3}
    Fix an integer $j\geq1$ and suppose $r=rk(\Lambda)\geq2.$
    For all $c\in\K\setminus\{0\}$ with $\|c\|\leq q^{-j}$, the following identity holds in $\C$:
    \[\sum_{\lambda\in\Lambda\setminus\{0\}}\frac{\exp_\Lambda(c\lambda)^{q^j}}{\lambda}=0.\]
\end{prop}

\begin{proof}
    As an element of $\Hom_{\F_q}^{cont}\left(\faktor{\K\Lambda}{\Lambda},\C\right)$, $\omega_\phi$ sends $c\in\K\Lambda$ to $\exp_\Lambda(c)$. By Lemma \ref{ev}, since $\zeta_\phi=-\sum_{\lambda\in\Lambda\setminus\{0\}}\lambda^{-1}\otimes\lambda$, for all $b\in\K$ and for all integers $j\geq0$ we have:
    \[ev(\zeta_\phi\otimes\omega_\phi^{(j)})(b)=-\sum_{\lambda\in\Lambda\setminus\{0\}}\frac{\exp_\Lambda(b\lambda)^{q^j}}{\lambda}.\]
    By Proposition \ref{identity 2}, if $\|b\|<1$, $ev(\zeta_\phi\otimes\omega_\phi)(b)=b.$
    Fix some $a\in A\setminus\{0\}$; for all $b\in\K$ with $\|ab\|<1$, we have:
    \[\left((a\otimes1-1\otimes a)ev(\zeta_\phi\otimes\omega_\phi)\right)(b)=a\cdot ev(\zeta_\phi\otimes\omega_\phi)(b)-ev(\zeta_\phi\otimes\omega_\phi)(ab)=ab-ab=0.\]
    Fix an ordered basis $(\mu_i)_{i\geq1}$ of $\Omega$, with $(\mu_i^*)_{i\geq1}$ dual basis of $\faktor{\K}{A}$, and write:
    \[(a\otimes1-1\otimes a)ev(\zeta_\phi\otimes\omega_\phi)=\sum_i c_i\otimes\mu_i\in\C\hat\otimes\Omega\]
    for some $c_i$'s in $\C$. Since $\lim_i \mu_i^*=0$ in $\faktor{\K}{A}$, there is some $k$ such that, for all $j>k$, there is a lifting $b_j\in\K$ of $\mu_j^*$ with $\|b_j\|<\|a\|^{-1}$; in particular, for all $j>k$:
    \[0=(a\otimes1-1\otimes a)ev(\zeta_\phi\otimes\omega_\phi)(b_j)=(a\otimes1-1\otimes a)ev(\zeta_\phi\otimes\omega_\phi)(\mu_j^*)=\left(\sum_i c_i\otimes\mu_i\right)(\mu_j^*)=c_j,\]
    hence $(a\otimes1-1\otimes a)ev(\zeta_\phi\otimes\omega_\phi)=\sum_{i\leq k}c_i\otimes\mu_i\in\C\otimes\Omega$..

    On the other hand, by Proposition \ref{identity 3}, for all integers $1\leq j\leq r-1$, if $b\in\K$ has norm at most $q^{-j}$ we have $ev(\zeta_\phi\otimes\omega_\phi^{(j)})(b)=0$; for the same reason as above, $ev(\zeta_\phi\otimes\omega_\phi^{(j)})\in\C\otimes\Omega$.
\end{proof}

%Aggiungerò che sono in H^1(-j\infty), or something like that.
\begin{oss}
    For general $A$, if we call $K_{<1}:=\{b\in K|\|b\|<1\}$, by Riemann-Roch we have $\dim_{\F_q}\left(\faktor{K}{(A\oplus K_{<1})}\right)=g$. In particular, Theorem \ref{teo} determines $ev(\zeta_\phi\otimes\omega_\phi)$ and $ev(\zeta_\phi\otimes\omega_\phi^{(i)})$ for $1\leq i\leq r$ up to a $\C$-linear subspace of $\C\hat\otimes\Omega$ of dimension respectively $g$ and $g+i-1$.
\end{oss}

\subsection{Application to the case $A=\F_q[\theta]$}

By the previous remark, when the genus $g=0$ and $i=0,1$, we can compute explicitly the forms $ev(\zeta_\phi\otimes\omega_\phi^{(i)})\in\C\hat\otimes\Omega$. Therefore, in this subsection we fix $A=\F_q[\theta]$, with $\K=\F_q[[\theta^{-1}]]$ and $\Omega=\F_q[\theta]d\theta$, where $d\theta:\K\to\F_q$ sends $\theta^n$ to $\delta_{-1,n}$.

\begin{prop}
    We have the following identities in $\C\hat\otimes\Omega$:
    \begin{align*}
     ev(\zeta_\phi\otimes\omega_\phi)&=\frac{d\theta}{(\theta\otimes1-1\otimes \theta)};\\
     ev(\zeta_\phi\otimes\omega_\phi^{(1)})&=0.
    \end{align*}
\end{prop}
\begin{proof}
    As functions from $\faktor{\K}{A}$ to $\C$, $ev(\zeta_\phi\otimes\omega_\phi)$ and $ev(\zeta_\phi\otimes\omega_\phi^{(1)})$ send $\theta^n$ respectively to $\theta^n$ and $0$ for all $n<0$ by Theorem \ref{teo}, since $\|\theta^n\|<1$.
    Since they both send $\theta^n$ to $0$ for $n\geq0$, we have the following identities for all $n$:
    \begin{align*}
    \left((\theta\otimes1-1\otimes \theta)ev(\zeta_\phi\otimes\omega_\phi)\right)(\theta^n)&=\theta\left(ev(\zeta_\phi\otimes\omega_\phi)(\theta^n)\right)-ev(\zeta_\phi\otimes\omega_\phi)(\theta^{n+1})=\delta_{-1,n}=d\theta(\theta^n),\\
    ev(\zeta_\phi\otimes\omega_\phi)(\theta^n)&=0;
    \end{align*}
    this proves the proposition.
\end{proof}

\begin{oss}
    The previous proposition can be thought of as the translation of \cite{Ferraro}[Thm. 6.3] to the context of Drinfeld modules of rank $2$ over $\F_q[\theta]$.
\end{oss}

We now relate the traditional definition of Anderson generating functions to the universal special function, by giving a basis-dependent description of the latter.

\begin{lemma}
    Fix the $A$-linear bases $\{\pi_1,\dots,\pi_r\}$ of $\Lambda$ and $\{\pi_1^*,\dots,\pi_r^*\}$ of $\Lambda^*$, with $\pi_i^*(\pi_j)=\delta_{i,j}$. Then, we have:
    \begin{align*}
    &\omega_\phi= \sum_{i=1}^r\sum_{j\geq0} \exp_\Lambda\left(\frac{\pi_i}{\theta^{j+1}}\right) \otimes\theta^j\pi_i^*d\theta,
    &\zeta_\phi=\sum_{i=1}^r\sum_{j\geq0}\left(\sum_{\lambda\in\Lambda\setminus\{0\}}\frac{d\theta\pi_i^*}{\theta^{j+1}}(\lambda)\lambda^{-1}\right)\otimes\theta^j\pi_i.
\end{align*}
\end{lemma}

\begin{proof}
The chosen bases induce an isomorphism $\Hom_A(\Lambda,\Omega)\cong\bigoplus_{i=1}^r A d\theta\pi_i^*$. The $\F_q$-linear basis $\{\theta^j d\theta\pi_i^*\}_{j\geq0,1\leq i\leq r}$ of $\Hom_A(\Lambda,\Omega)$ induces a dual basis $\{\theta^{-j-1}\pi_i\}_{j\geq 0,1\leq i\leq r}$ of $\widehat{\Hom_A(\Lambda,\Omega)}$. Similarly, the $\F_q$-lineas basis $\{\theta^j\pi_i\}_{j\geq0,1\leq i\leq r}$ of $\Lambda$: induces the dual basis $\{\theta^{-j-1}d\theta\pi_i^*\}_{j\geq 0,1\leq i\leq r}$ of $\widehat{\Lambda}$. This proves the lemma, by virtue of Remark \ref{oss universal special function} and the proof of Proposition \ref{zeta function functor}.
\end{proof}

\begin{Def}
For $i=1,\dots,r$ we define the $i$-th Anderson generating function as:
\[\omega_{\phi,i}:=\sum_{j\geq0}\exp_\Lambda\left(\frac{\pi_i}{\theta^{j+1}}\right)\otimes\theta^j\in\C\hat\otimes A.\]

Similarly, for $i=1,\dots,r$ we define the $i$-th dual generating function as:
    \[\zeta_{\phi,i}= \sum_{j\geq0} \left(\sum_{\lambda\in\Lambda\setminus\{0\}} \frac{d\theta\pi_i^*}{\theta^{j+1}}(\lambda)\lambda^{-1}\right)\otimes\theta^j\in\C\hat\otimes A.\]
\end{Def}
\begin{oss}
For all integers $1\leq i\leq r$, $\omega_{\phi,i}$ and $\zeta_{\phi,i}$ are the unique elements in $\C\hat\otimes A$ such that the equalities $\pi_i(\omega_\phi)=\omega_{\phi,i} d\theta$ and $\pi^*_i(\zeta_\phi)=\zeta_{\phi,i}$ hold (in $\C\hat\otimes\Omega$ and $\C\hat\otimes A$, respectively).
\end{oss}

\begin{Def}
    Let's denote $\boldsymbol{\omega}_\phi:=(\omega_{\phi,i}^{(j-1)})_{i,j}\in\Mat_{r\times r}(\C\hat\otimes A)$. We call it the \textit{rigid analitic trivialization} of the $t$-motive attached to $\phi$.
\end{Def}

The previous matrix has been studied in various articles (see for example \cite{Pellarin2007}[Section 4.2], \cite{Khaochim}, \cite{Gezmis}). We can use it to write the following corollary to Theorem \ref{teo}.
\begin{prop}
    Let's interpret $\zeta_\phi$ as $(\zeta_{\phi,i})_i\in \Mat_{1\times r}(\C\hat\otimes A)$. Then, the product $\zeta_\phi\cdot\boldsymbol{\omega}_\phi$ belongs to $\Mat_{1\times r}(\Frac(\C\otimes A))$.
\end{prop}
\begin{proof}
    If we multiply by $d\theta\in\Omega$ the $j$-th coordinate of the product, we get:
    \[\left(\sum_{i=1}^r\omega_{\phi,i}^{(j-1)} \zeta_{\phi,i}\right)d\theta= \left(\sum_{i=1}^r\omega_{\phi,i}\pi_i^*d\theta\right)^{(j-1)}\cdot\left(\sum_{i=1}^r\zeta_{\phi,i}\pi_i\right)=\omega_\phi^{(j-1)}\cdot\zeta_\phi,\]
    which is a rational differential form by Theorem \ref{teo}.
\end{proof}

\begin{oss}
    We get the same result by multiplying the vector $(\omega_{\phi,i})_i\in\Mat_{1\times r}(\C\hat\otimes A)$ and the matrix $(\zeta_{\phi,i}^{(1-j)})_{i,j}\in\Mat_{r\times r}(\C\hat\otimes A)$.  
\end{oss}

\begin{oss}
    It's a well known result 
    that the determinant of the matrix $\boldsymbol{\omega}_\phi$ is nonzero (see for example \cite{Gezmis}[Prop. 6.2.4]), so by the previous corollary we can recover $\zeta_\phi$ from $\boldsymbol{\omega}_\phi$ and the products $\{\omega_\phi^{(k)}\cdot\zeta_\phi\}_{k=0,\dots,r-1}$. 
    %Similarly, if we know these products and the matrix $(\zeta_{\phi,i}^{(1-j)})_{i,j}$, we can compute the special function $\omega_\phi$.
\end{oss}

%Specifically, for $k=0$ the rational function is $\frac{1}{\theta\otimes1-1\otimes\theta}$, while for $k\geq1$ it only has a pole at $\infty$, of degree at most $k-1$.


\section{Some notable identities in $\C$}\label{section technical lemmas}

In this section we provide a proof for the identities in Propositions \ref{identity 1}, \ref{identity 2}, and \ref{identity 3}.

\begin{customprop}{\ref{identity 1}}
    For all $\beta\in\ker(\exp_\Lambda^*)\setminus\{0\}$, the following identity holds in $\C$:
    \[\beta=-\sum_{\lambda\in\Lambda\setminus\{0\}}\frac{g_\beta(\lambda)}{\lambda}.\]
\end{customprop}

\begin{customprop}{\ref{identity 2}}
    For all $c\in\K\setminus\{0\}$ with $\|c\|<1$, the following identity holds in $\C$:
    \[c=-\sum_{\lambda\in\Lambda\setminus\{0\}}\frac{\exp_\Lambda(c\lambda)}{\lambda}.\]
\end{customprop}

\begin{customprop}{\ref{identity 3}}
    Fix an integer $j\geq1$ and suppose $r=rk(\Lambda)\geq2.$
    For all $c\in\K\setminus\{0\}$ with $\|c\|\leq q^{-j}$, the following identity holds in $\C$:
    \[\sum_{\lambda\in\Lambda\setminus\{0\}}\frac{\exp_\Lambda(c\lambda)^{q^j}}{\lambda}=0.\]
\end{customprop}

\subsection{Locally finite subspaces}

\begin{Def}\label{def Lambda_m}
    A \textit{locally finite subspace} $V\subseteq\C$ is an $\F_q$-vector space such that for any positive real number $r$ there are finitely many elements of $V$ of norm at most $r$.
    
    An \textit{ordered basis} of $V$ is a sequence $(v_i)_{i\geq1}$ with the following property: for all $m\geq1$, $v_m$ is an element of $V\setminus\Span(\{v_i\}_{i<m})$ of least norm.

    We call the sequence of real numbers $(\|v_i\|)_{i\geq1}$ the \textit{norm sequence} of $V$.
\end{Def}

The next two results aim to justify the nomenclature "ordered basis" and the well-posedness of the norm sequence.

\begin{lemma}
    If $(v_i)_{i\geq1}$ is an ordered basis of a locally finite subspace $V\subseteq\C$, it is a basis of $V$ as an $\F_q$-vector space.
\end{lemma}
\begin{proof}
    For all $m\geq1$ $v_m\not\in \Span(\{v_i\}_{i<m})$, hence the $v_i$'s are $\F_q$-linearly independent. Since for all $r\in\mathbb{R}$ there is a finite number of elements of $V$ with norm at most $r$, the norm sequence $(\|v_i\|)_{i\geq1}$ tends to infinity; in particular, for all $v\in V$ there is an integer $m$ such that $\|v_m\|>\|v\|$, so $v\in \Span(\{v_i\}_{i<m})$ by definition of $v_m$.
\end{proof}
\begin{prop}\label{norm sequence}
    If $(v_i)_{i\geq1}$ is an ordered basis of a locally finite subspace $V\subseteq\C$, and $(v'_i)_{i\geq1}$ is a sequence of elements in $V$ that are $\F_q$-linearly independent and with weakly increasing norm, then $\|v'_i\|\geq\|v_i\|$ for all $i$. In particular, the norm sequence of $V$ does not depend on the chosen ordered basis of $V$.
%Moreover, if $\|v'_i\|=\|v_i\|$ eventually for all $i$, then $V=\Span_{\F_q}(\{v'_i\}_i)$.
\end{prop}
\begin{proof}
    If $\|v'_m\|<\|v_m\|$ for some $m$, then for all $i\leq m$ we have $\|v'_i\|\leq\|v'_m\|<\|v_m\|$, so $v'_i\in V_{m-1}$; since $\{v'_i\}_{i\leq m}$ are $\F_q$-linear independent and $V_m$ has dimension $m-1$ as an $\F_q$-vector space, we reach a contradiction. If we take $(v'_i)_i$ to be another ordered basis, by this reasoning we get both $\|v'_m\|\geq\|v_m\|$ and $\|v_m\|\geq\|v'_m\|$, hence the norm sequence is independent from the choice of the ordered basis.
%Set $W:=\Span_{\F_q}(\{v'_i\}_i)$ and suppose that $W\subsetneq V$. For all positive integers $n$, call $V(n)$ and $W(n)$ the $\F_q$-subspaces of $V$ and $W$ respectively consisting of the elements with norm at most $n$, and denote $k_n$ the least integer such that $\|v_{k_n+1}\|>n$. Obviously $V_{k_n}\subseteq V(n)$; vice versa, for all $v\in V(n)$, since $\|v\|<\|v_{k_n+1}\|$, $v\in V_{k_n}$ by the definition of $v_{k_n+1}$. Since $\bigcup_n V(n)=V$ and $\bigcup W(n)=W$, eventually $W(n)\subsetneq V(n)$; for all such $n$, $\|v'_{k_n}\|>\|v_{k_n}\|$: if it were not true, we would have $\|v'_i\|\leq\|v_{k_n}\|\leq n$ for all $i\leq k_n$, hence $\dim_{\F_q}(W(n))\geq k_n=\dim_{\F_q}(V(n))$, which is a contradiction.
\end{proof}

Finally, we show that the norm sequence is reasonably well behaved with regard to subspaces.

\begin{lemma}\label{norm sequence codim 1}
    Take two locally finite subspaces $W\subseteq V\subseteq\C$ with $\dim_{\F_q}\left(\faktor{V}{W}\right)=1$, with norm sequences respectively $(s_i)_{i\geq1}$ and $(r_i)_{i\geq1}$. Then, there is a positive integer $m$ such that for all $i<m$ $s_i=r_i$ and for all $i\geq m$ $s_i=r_{i+1}$.
\end{lemma}
\begin{proof}
    Let's fix an ordered basis $(w_i)_{i\geq1}$ of $W$ and an element $u\in V\setminus W$ of least norm. Let $m$ be the least positive integer such that $\|w_m\|>\|u\|$, and define the following vectors $v_i$ for all positive integers $i$:
    \[v_i=\begin{cases}w_i\text{ if }i<m\\u\text{ if }i=m\\w_{i-1}\text{ if }i>m\end{cases}.\]
    We want to prove that the sequence $(v_i)_{i\geq1}$ is an ordered basis of $V$, i.e. that, for all $k$, $v_k$ is an element of least norm not contained in $\Span(\{v_i\}_{i<k})$. 
    %For $k<m$, it's true because $v_k=w_k$ is an element of $W$ of least norm not contained in $\Span(\{w_i\}_{i<k})$, and the elements of $V\setminus W$ have norm at least $\|u\|\geq\|w_k\|$. For $k=m$, it's true because $w_m$ is an element of $W$ of least norm not contained in $\Span(\{w_i\}_{i<m})$, and the elements of $V\setminus W$ have norm at least $\|u\|<\|w_m\|$. 
    For $k\leq m$ it's obvious by the definition of $u$ and $(w_i)_{i\geq0}$. For $k>m$, $v_{k+1}=w_k$ is an element of $W$ of least norm not contained in $\Span(\{w_i\}_{i<k})$, and is also not contained in $\Span(\{v_i\}_{i<k+1})=\Span(\{w_i\}_{i<k}\cup\{u\})$. We need to prove that any element of $V\setminus W$ not contained in $\Span(\{v_i\}_{i<k+1})$ has norm at least $\|w_k\|$; since $\dim_{\F_q}\left(\faktor{V}{W}\right)=1$, it can be written as $u+w$ with $w\in W$ and $w\not\in\Span(\{v_i\}_{i<k+1})$, so $\|w\|\geq\|w_k\|>\|u\|$ and $\|u+w\|=\|w\|\geq\|w_k\|$.

    As a consequence, $s_i=\|w_i\|=\|v_i\|=r_i$ for all $i<m$, and $s_i=\|w_i\|=\|v_{i+1}\|=r_{i+1}$ for all $i\geq m$.
\end{proof}

\subsection{Estimation of the coefficients of $g_\beta$ and $\exp_\Lambda$}

The following result is similar to the well known "vanishing lemma" (see \cite{Goss}[Lemma 8.8.1]).
\begin{lemma}\label{S_n,k}
    Call $S_{n,d}(x_1,\dots,x_n)\in\F_q[x_1,\dots,x_n]$ the sum of the $d$-th powers of all the homogeneous linear polynomials. Suppose that the coefficient of monomial $x_1^{d_1}\cdots x_n^{d_n}$ in the expansion of $S_{n,d}(x_1,\dots,x_n)$ is nonzero: then, for all $1\leq j\leq n$, $\sum_{i=1}^j d_i\geq q^j-1$. In particular, if $d<q^n-1$, $S_{n,d}=0$.
\end{lemma}
\begin{proof}
    The coefficient $c_{d_1,\dots,d_n}$ of the monomial $x_1^{d_1}\cdots x_n^{d_n}$ is:
    \[\frac{d!}{d_1!\cdots d_n!}\sum_{a_1,\dots,a_n\in\F_q}a_1^{d_1}\cdots a_n^{d_n}=\frac{d!}{d_1!\cdots d_n!}\prod_{i=1}^n\left(\sum_{a_i\in\F_q}a_i^{d_i}\right),\]
    where by convention we set $0^0=1$.  On one hand, if the multinomial coefficient $\frac{d!}{d_1!\cdots d_n!}$ is nonzero in $\F_q$, $C(d)=C(d_1)+\cdots+C(d_n)$, where we denote by $C(m)$ the sum of the digits in base $q$ of the nonnegative integer $m$; in particular, for $1\leq j\leq n$ this implies $C(d_1+\cdots+d_j)=C(d_1)+\cdots+C(d_j)$. On the other hand, $\sum_{a_i\in\F_q}a_i^{d_i}\neq0$ if and only if $d_i>0$ and $q-1|d_i$; in particular, this implies $C(d_i)\geq q-1$ for all $i$. 
    
    If $c_{d_1,\dots,d_n}\neq0$, for $1\leq j\leq n$ we have:
    \[C\left(\sum_{i=1}^j d_i\right)=\sum_{i=1}^j C(d_i)\geq(q-1)j,\]
    hence $\sum_{i=1}^j d_i\geq q^j-1$. Applying this to $j=n$ we get the condition $d\geq q^n-1$, therefore $S_{n,d}=0$ for all $d<q^n-1$.
\end{proof}

\begin{Def}
    For a locally finite subspace $V\subseteq \C$, for all integers $i\geq0$ we define:
    \[e_{V,i}:=\sum_{\substack{I\subseteq V\setminus{0}\\|I|=q^i-1}}\prod_{v\in I}v^{-1}\]
    (by convention, $e_{V,0}=1$).
\end{Def}
\begin{oss}\label{remark e_V}
    For all $c\in\C$, since $V\subseteq\C$ is locally finite, the infinite product $c\prod_{v\in V}\left(1-\frac{c}{v}\right)$ converges, and is equal to $\sum_{n\geq0} e_{V,n}c^{q^n}$. In particular, $\sum_{n\geq0}e_{V,n}x^{q^n}\in\C[[x]]$ is the only analytic function such that: it converges everywhere, its zeroes are simple and coincide with $V$, and its leading coefficient is $1$.
\end{oss}

\begin{lemma}\label{coefficient bounds}
    Fix a locally finite subspace $V\subseteq\C$, with norm sequence $(r_i)_{i\geq1}$. Fix an ordered basis $(v_i)_{i\geq1}$ and call $V_m:=\Span(\{v_i\}_{i\leq m})$ for all $m\geq0$. We have:
    \begin{itemize}
        \item for all $k\geq0$:
        \[\|e_{V,k}\|\leq\prod_{i=1}^k r_i^{q^{i-1}-q^i};\]
        \item for all $m>0$, for all $k>0$:
        \[\left\|\sum_{v\in V_m}v^{q^k-1}\right\|\begin{cases}=0\text{ if }k<m\\
        \leq r_m^{q^k-q^m}\prod_{i=1}^m r_i^{q^i-q^{i-1}}\text{ if }k\geq m\end{cases}.\]
    \end{itemize}
%    In particular, for all $m>0$, for all $k>0$, the product $e_{V,k}\sum_{v\in V_m}v^{q^k-1}$ has norm at most $1$.
\end{lemma}
\begin{proof}
    For the first part, if $k=0$ then $e_{V,k}=1$, so there is nothing to prove. If $k>0$, we have:
    \[\|e_{V,k}\|=\left\|\sum_{\substack{I\subseteq V\setminus\{0\}\\|I|=q^k-1}}\prod_{v\in I}v^{-1}\right\|\leq\max_{\substack{I\subseteq V\setminus\{0\}\\|I|=q^k-1}}\left\|\prod_{v\in I}v^{-1}\right\|=\left\|\prod_{v\in V_k}v^{-1}\right\|=\prod_{i=1}^k r_i^{q^{i-1}-q^i}.\]
    For the second part, note that the element whose norm we are trying to estimate is equal to $S_{m,q^k-1}(v_1,\dots,v_m)$, in the notation of Lemma \ref{S_n,k}. By that lemma, if $k<m$, the element is zero, otherwise we have the following inequality:
    \[\|S_{m,q^k-1}(v_1,\dots,v_m)\|\leq\max_{\substack{d_1,\dots,d_m\\d_1+\cdots+d_m=q^k-1\\ \forall j\;d_1+\cdots+d_j\geq q^j-1}}\left\|v_1^{d_1}\cdots v_m^{d_m}\right\|.\]
    It's easy to see that the maximum norm of the product $v_1^{d_1}\cdots v_m^{d_m}$ under the specified conditions is obtained when we set $d_i=q^i-q^{i-1}$ for $i<m$ and $d_m=q^k-q^{m-1}$, therefore we get the desired inequality.
%    Finally, for all $m>0$, for all $k>0$, we get:\[\|e_{V,k}S_{m,k}(v_1,\dots,v_m)\|\leq\left(\prod_{i=1}^k r_i^{q^{i-1}-q^i}\right)\left(r_m^{q^k-q^{m-1}}\prod_{i=1}^{m-1}r_i^{q^i-q^{i-1}}\right)=\prod_{i=m+1}^k \left(\frac{r_m}{r_i}\right)^{q^i-q^{i-1}}\leq1.\]
\end{proof}
%\begin{lemma}If $W\subsetneq V\subseteq\C$ are locally finite subspaces, $\lim_{i\to\infty}\frac{e_{W,i}}{e_{V,i}}=0$ \end{lemma}\begin{proof}Call $(r_i)_{i\geq1}$ and $(s_i)_{i\geq1}$ the norm sequences of $V$ and $W$, respectively. By Proposition \ref{norm sequence}, for all $i$'s $r_i\geq s_i$, and there is at least an integer $j$ such that $r_j>s_j$. By Lemma \ref{coefficient bounds}, for $m\gg j$ we have:\[\frac{e_{W,i}}{e_{V,i}}=\end{proof}



\begin{oss}
    Since $\Lambda\subseteq\K\Lambda$ is discrete and $\K\Lambda\cong\K^r$ is locally compact, $\Lambda$ is a locally finite subspace of $\C$. Moreover, $e_{\Lambda,n}$ is exactly the coefficient in degree $q^n$ of the exponential function $\exp_\Lambda$.
\end{oss}



\begin{lemma}\label{V_beta}
    For all $\beta\in\ker(\exp_\Lambda)\setminus\{0\}$, $\ker(g_\beta)$ is an $\F_q$-vector subspace of $\Lambda$ of codimension $1$. In particular, $g_\beta=\beta\sum_{n\geq0}e_{\ker(g_\beta),n}\tau^n$.
\end{lemma}
\begin{proof}
    Let's denote $V_\beta:=\ker(g_\beta)$. If $c\in V_\beta$ then $\exp_\Lambda(c)=\beta^{-1}(1-\tau)(g_\beta(c))=0$, hence $c\in\Lambda$. Moreover, $g_\beta|_\Lambda$ is an $\F_q$-linear function with image in $\F_q$, hence its kernel $V_\beta$ has codimension at most $1$ in $\Lambda$. It is exactly $1$ because $g_\beta|_\Lambda$ is not identically zero by Proposition \ref{Poonen}.

    From the identity $(1-\tau)\circ g_\beta=\beta\exp_\Lambda$, since the zeroes of $\exp_\Lambda$ are simple, we deduce the same for the zeroes of $g_\beta$, therefore $g_\beta=c_\beta \sum_{n\geq0}e_{V_\beta,n}\tau^n$ for some constant $c_\beta\in\C$ by Remark \ref{remark e_V}. Finally, from the same identity we deduce that the coefficient of $\tau$ in $g_\beta$ is $\beta$, hence $c_\beta=\beta$.
\end{proof}

\subsection{Proof of the identities}
We can finally prove the main propositions of this section.

\begin{customprop}{\ref{identity 1}}
    For all $\beta\in\ker(\exp_\Lambda^*)\setminus\{0\}$, the following identity holds in $\C$:
    \[\beta=-\sum_{\lambda\in\Lambda\setminus\{0\}}\frac{g_\beta(\lambda)}{\lambda}.\]
\end{customprop}
%\begin{lemma}\label{lemma 1 beta}There is a real constant $C$ such that $v(b_i)\geq(i-C)q^i$ for all $i\geq0$.\end{lemma}\begin{proof}If we write $\exp_\Lambda:=\sum_{i\geq0}c_i\tau^i$, we know (see for example \cite{Ferraro}[Rmk 7.13]) that there is a real constant $C'$ such that $v(c_i)\geq(i-C')q^i$ for all $i\geq0$. Since $g_\beta$ has degree $r\deg(a)-1$ in $\tau$, if we call $M$ the minimum valuation of its coefficients, for all $i\geq0$ we get:\[v(b_i)\geq\min_{0\leq j<r\deg(a)}\{M+v(c_{i-j}^{q^j})\}\geq M+\min_{0\leq j<r\deg(a)}\{(i-j-C')q^i\}\geq(i+1+M-r\deg(a)-C')q^i.\qedhere\]\end{proof}\begin{lemma}\label{lemma 2 beta}There is a real constant $k$ such that, for $m\gg0$, for all $\lambda\in\Lambda_m$, $v(\lambda)\geq -\frac{m}{r}-k$.\end{lemma}\begin{proof}Fix an $\F_q$-basis $\{a_i\}_{i\geq0}$ of $A$ of decreasing valuation, and $r$ $A$-linearly independent elements $\mu_1,\dots,\mu_r\in\Lambda$, with $M:=\min_{1\leq i\leq r}\{v(\mu_i)\}$. The $A$-span $\Lambda'\subseteq\Lambda$ of $\{\mu_1,\dots,\mu_r\}$, as an $\F_q$-vector space, has basis $\{a_{j_1}\mu_1+\cdots+a_{j_r}\mu_r\}_{j_1,\dots,j_r}$; since the latter are $\F_q$-linearly independent, we can order them in the sequence $(\lambda'_i)_{i\geq0}$ of decreasing valuation, and by the previous Remark, for all $m\geq0$, $v(\lambda_m)\geq v(\lambda'_m)$. On the other hand, for any real number $\alpha\geq g-M$:\begin{align*}\#\{\lambda\in\Lambda|v(\lambda)\geq -\alpha\}&\geq\#\{\lambda\in\Lambda'|v(\lambda)\geq -\alpha\}\geq\prod_{i=1}^r \#\{a\in A|v(a)+v(\mu_i)\geq -\alpha\}\\&\geq\left(\#\{a\in A|v(a)\geq -\alpha-M\}\right)^r\geq q^{(\lfloor\alpha+M\rfloor-g)r};\end{align*}in particular, $v\left(\lambda'_{(\lfloor\alpha+M\rfloor-g)r}\right)\geq-\alpha$. We deduce that for all $m\geq0$:\[v(\lambda_m)\geq v(\lambda'_m)\geq v(\lambda'_{r\lceil\frac{m}{r}\rceil})\geq -\left\lceil\frac{m}{r}\right\rceil-g+M\geq-\frac{m}{r}-g+M-1.\]\end{proof}

\begin{proof}

First of all, the series converges because the denominators belong to the locally finite subspace $\Lambda$ and the numerators to $\F_q$.
Fix an ordered basis $(\lambda_i)_{i\geq1}$ of $\Lambda$ and define $\Lambda_m:=\Span(\{\lambda_i\}_{i\leq m})$ for all $m\geq0$. Call $V_\beta:=\ker(g_\beta)$; by Lemma \ref{V_beta}, $V_\beta\subseteq\Lambda$ has codimension $1$, hence by Lemma \ref{norm sequence codim 1}, if we denote $(r_i)_{i\geq1}$ and $(s_i)_{i\geq1}$ the norm sequences respectively of $\Lambda$ and $V_\beta$, there is a positive integer $N$ such that for all $i<N$ $s_i=r_i$, and for all $i\geq N$ $s_i=r_{i+1}$.
For all $m\geq N$, we define:
    \[S_m:=\beta+\sum_{\lambda\in\Lambda_m\setminus\{0\}}\frac{g_\beta(\lambda)}{\lambda}=\beta\sum_{k\geq1}e_{V_\beta,k}\sum_{\lambda\in\Lambda_m}\lambda^{q^k-1}.\]
    By Lemma \ref{coefficient bounds}, we have:
    \begin{align*}
        &\|\beta^{-1}S_m\|=\left\|\sum_{k\geq1}e_{V_\beta,k}\sum_{\lambda\in\Lambda_m}\lambda^{q^k-1}\right\|\leq\max_{k\geq m}\left\{\|e_{V_\beta,k}\|\left\|\sum_{\lambda\in\Lambda_m}\lambda^{q^k-1}\right\|\right\}\\
        \leq&\max_{k\geq m}\left\{\left(\prod_{i=1}^k s_i^{q^{i-1}-q^i}\right)\left(r_m^{q^k-q^m}\prod_{i=1}^m r_i^{q^i-q^{i-1}}\right)\right\}\\
        =&\max_{k\geq m}\left\{\left(\prod_{i=N}^k r_{i+1}^{q^{i-1}-q^i}\right)\left(r_m^{q^k-q^m}\prod_{i=N}^m r_i^{q^i-q^{i-1}}\right)\right\}\\
        =&\max_{k\geq m}\left\{\left(\prod_{i=N}^{m} \left(\frac{r_i}{r_{i+1}}\right)^{q^{i}-q^{i-1}}\right)\left(\prod_{i=m+1}^k\left(\frac{r_m}{r_i}\right)^{q^i-q^{i-1}}\right)\right\}\\
        =&\prod_{i=N}^{m} \left(\frac{r_i}{r_{i+1}}\right)^{q^{i}-q^{i-1}}=\left(\frac{r_N}{r_{m+1}}\right)^{q^N-q^{N-1}}\prod_{i=N+1}^{m}\left(\frac{r_i}{r_{m+1}}\right)^{q^i-2q^{i-1}+q^{i-2}}\leq\left(\frac{r_N}{r_{m+1}}\right)^{q^N-q^{N-1}}.
    \end{align*}
Since this number tends to zero as $m$ tends to infinity, we have the following identities in $\C$:
    \[0=\lim_m S_m=\lim_m\left(\beta+\sum_{\lambda\in\Lambda_m\setminus\{0\}}\frac{g_\beta(\lambda)}{\lambda}\right)=\beta+\sum_{\lambda\in\Lambda\setminus\{0\}}\frac{g_\beta(\lambda)}{\lambda}.\]
\end{proof}


\begin{customprop}{\ref{identity 2}}
    For all $c\in\K\setminus\{0\}$ with $\|c\|<1$, the following identity holds in $\C$:
    \[c=-\sum_{\lambda\in\Lambda\setminus\{0\}}\frac{\exp_\Lambda(c\lambda)}{\lambda}.\]
\end{customprop}
\begin{proof}
    First of all, the series converges because the denominators belong to the locally finite subspace $\Lambda$ and the numerators to the compact subspace $\exp_\Lambda(\K\Lambda)\cong\faktor{\K\Lambda}{\Lambda}$. Fix an ordered basis $(\lambda_i)_{i\geq1}$ of $\Lambda$ and define $\Lambda_m:=\Span(\{\lambda_i\}_{i\leq m})$ for all $m\geq0$; denote $(r_i)_{i\geq1}$ the norm sequence of $\Lambda$. For all $m\geq0$, define:
    \[S_m:=c+\sum_{\lambda\in\Lambda_m\setminus\{0\}}\frac{\exp_\Lambda(c\lambda)}{\lambda}=\sum_{k\geq1}e_{\Lambda,k}c^{q^k}\sum_{\lambda\in\Lambda_m}\lambda^{q^k-1}.\]
    By Lemma \ref{coefficient bounds}, if $\|c\|<1$ we have:
    \begin{align*}
        \|S_m\|=&\left\|\sum_{k\geq1}e_{\Lambda,k}c^{q^k}\sum_{\lambda\in\Lambda_m}\lambda^{q^k-1}\right\|\leq\max_{k\geq m}\left\{\left\|e_{\Lambda,k}c^{q^k}\right\|\left\|\sum_{\lambda\in\Lambda_m}\lambda^{q^k-1}\right\|\right\}\\
        \leq&\max_{k\geq m}\left\{\|c\|^{q^k}\left(\prod_{i=1}^k r_i^{q^{i-1}-q^i}\right)\left(r_m^{q^k-q^m}\prod_{i=1}^m r_i^{q^i-q^{i-1}}\right)\right\}\\
        =&\max_{k\geq m}\left\{\|c\|^{q^k}\left(\prod_{i=m+1}^k \left(\frac{r_m}{r_i}\right)^{q^i-q^{i-1}}\right)\right\}\leq\|c\|^{q^m}.
    \end{align*}
This number tends to zero as $m$ tends to infinity, hence we have the following identities in $\C$:
    \[0=\lim_m S_m=\lim_m\left(c+\sum_{\lambda\in\Lambda_m\setminus\{0\}}\frac{\exp_\Lambda(c\lambda)}{\lambda}\right)=c+\sum_{\lambda\in\Lambda\setminus\{0\}}\frac{\exp_\Lambda(c\lambda)}{\lambda}.\]
\end{proof}
\begin{oss}
    The previous result is optimal, in the sense that it cannot be meaningfully extended to elements in $\K$ of higher norm. For $c=1$ the RHS is $0$, and the LHS $1$; moreover, for all $x\in\K$ with $\|x\|>1$, setting $c=x$ and $c=x+1$ yields the same RHS and different LHS.
\end{oss}



%\begin{cor}\label{cor1}For any $a\in A\setminus\{0\}$, the isomorphism $\widehat{\Lambda/a\Lambda}\to\ker(\phi^*_a)$ sends $l$ to $\sum_{\lambda\in\Lambda}\frac{l(\lambda)}{\lambda}$.\end{cor}\begin{proof}The natural isomorphism of $A$-modules $\Lambda/a\Lambda\to\ker(\phi_a)$ sending $\lambda$ to $\exp_\Lambda\left(\frac{\lambda}{a}\right)$ induces the dual isomorphism $\widehat{\ker(\phi_a)}\to\widehat{\Lambda/a\Lambda}$. Composing with the isomorphism from Proposition \ref{Poonen}, we get an isomorphism of $A$-modules $\ker(\phi_a^*)\to\widehat{\Lambda/a\Lambda}$ sending $\beta$ to $g_\beta\circ\exp_\Lambda\circ a^{-1}$. By Proposition \ref{identity 1}, its inverse sends $l\in\widehat{\Lambda/a\Lambda}$ to $-\sum_{\lambda\in\Lambda\setminus\{0\}}\frac{l(\lambda)}{\lambda}$.\end{proof}

%We use the following result (\cite{Poonen}[Prop. 20])\begin{prop}\label{Poonen2}Consider the subset $\bigcup_{a\in A\setminus\{0\}}\ker(\phi^*_a)\subseteq\C$. Its closure coincides with $\ker(\exp_\Lambda^*)$.\end{prop}\begin{cor}\label{cor2}There map $\widehat{\Lambda}\to\ker(\exp^*_\Lambda)$ sending $l$ to $\sum_{\lambda\in\Lambda}\frac{l(\lambda)}{\lambda}$ is an isomorphism of locally compact $A$-modules.\end{cor}\begin{proof}As a map from $\widehat{\Lambda}$ to $\C$, we can show that the described map is an isometry with respect to the natural norm induced on $\widehat{\Lambda}$ by the inclusion $\Lambda\subseteq\C$. If $\lambda'\in\Lambda$ is the element of least norm such that $l(\lambda)\neq0$, then $|l|=\frac{1}{|\lambda'|}$, and:\[\left|\sum_{\lambda\in\Lambda}\frac{l(\lambda)}{\lambda}\right|=\left|\frac{l(\lambda')}{\lambda'}\right|=\frac{1}{|\lambda'|}.\]Restricted to the dense subset $\bigcup_{a\in A\setminus\{0\}}\widehat{\Lambda/a\Lambda}\subseteq\widehat{\Lambda}$, the map is $A$-linear and its image is $\bigcup_{a\in A\setminus\{0\}}\ker(\phi^*_a)$, by Corollary \ref{cor1}. Since the map is an isometry, and the closure of the right hand side in $\C$ is $\ker(\exp_\Lambda^*)$ by Proposition \ref{Poonen2}, we get the desired isomorphism.\end{proof}

\begin{customprop}{\ref{identity 3}}
    Fix an integer $j\geq1$ and suppose $r=rk(\Lambda)\geq2.$
    For all $c\in\K\setminus\{0\}$ with $\|c\|\leq q^{-j}$, the following identity holds in $\C$:
    \[\sum_{\lambda\in\Lambda\setminus\{0\}}\frac{\exp_\Lambda(c\lambda)^{q^j}}{\lambda}=0.\]
\end{customprop}

\begin{proof}
    First of all, the series converges because the denominators belong to the locally finite subspace $\Lambda$ and the numerators to the image of the compact subspace $\exp_\Lambda(\K\Lambda)\cong\faktor{\K\Lambda}{\Lambda}$ under the $j$-th frobenius map. Fix an ordered basis $(\lambda_i)_{i\geq1}$ of $\Lambda$ and define $\Lambda_m:=\Span(\{\lambda_i\}_{i\leq m})$ for all $m\geq0$; denote $(r_i)_{i\geq1}$ the norm sequence of $\Lambda$. 

Let's start with an observation: we can fix $\pi_1,\dots,\pi_r\in\C$ such that $\Lambda\subseteq \bigoplus_{i=1}^r\pi_i A=\Lambda'$ and $\Lambda'/\Lambda$ is finite; if $(r'_i)_i$ is the norm sequence of $\Lambda'$, there is an integer $k$ such that for large enough $i$ $r'_i=r_{i-k}$. For any norm $n$, the dimension of the subspace of the elements $\Lambda'$ with norm at most $N$ is at least $\sum_{i=1}^r\dim_{\F_q}A\left(\leq\log_q\left(\frac{N}{\|\pi_i\|}\right)\right)>r\log_q(N)-rg-\log_q(\|\pi_1\cdots\pi_r\|)$; in particular, if $r'_i\geq N$ we have $i>r\log_q(N)-rg-\log_q(\|\pi_1\cdots\pi_r\|)$, hence for all $i\in\N$ we have $\log_q(r'_i)<\frac{i+\log_q(\|\pi_1\cdots\pi_r\|)}{r}+g$. We deduce that $\limsup_i \frac{\log_q(r'_i)}{i}=\limsup_i\frac{\log_q(r_i)}{i}\leq\frac{1}{r}$.


For all $m\geq0$, define:
    \[S_m:=\sum_{\lambda\in\Lambda_m\setminus\{0\}}\frac{\exp_\Lambda(c\lambda)^{q^j}}{\lambda}=\sum_{k\geq1}e_{\Lambda,k}^{q^j}c^{q^{k+j}}\sum_{\lambda\in\Lambda_m}\lambda^{q^{k+j}-1}.\]
    By Lemma \ref{coefficient bounds}, if $\|c\|<1$ we have:
    \begin{align*}
        \|S_m\|=&\left\|\sum_{k\geq1}e_{\Lambda,k}^{q^j}c^{q^{k+j}}\sum_{\lambda\in\Lambda_m}\lambda^{q^{k+j}-1}\right\|\leq\max_{k\geq m-j}\left\{\left\|e_{\Lambda,k}c^{q^k}\right\|^{q^j}\left\|\sum_{\lambda\in\Lambda_m}\lambda^{q^{k+j}-1}\right\|\right\}\\
        \leq&\max_{k\geq m}\left\{\|c\|^{q^k}\left(\prod_{i=j+1}^k r_{i-j}^{q^{i-1}-q^i}\right)\left(r_m^{q^k-q^m}\prod_{i=1}^m r_i^{q^i-q^{i-1}}\right)\right\}\\
        \leq&\prod_{i=1}^j (\|c\|r_i)^{q^i-q^{i-1}}\max_{k\geq m}\left\{\|c\|^{q^k}\left(\prod_{i=j+1}^k \left(\frac{\|c\|r_i}{r_{i-j}}\right)^{q^i-q^{i-1}}\right)\right\};\\
        \limsup_m\|S_m\|\leq&\prod_{i=1}^j (\|c\|r_i)^{q^i-q^{i-1}}\limsup_k \prod_{i=j+1}^k \left(\frac{\|c\|r_i}{r_{i-j}}\right)^{q^i-q^{i-1}}.
    \end{align*}
If we go from $k$ to $k+1$, the element in the curly brackets gets multiplied by:
\[\left(\frac{\|c\|r_m}{r_{k+1-j}}\right)^{q^{k+1}-q^k}\leq \left(\frac{\|c\|r_m}{r_{m+1-j}}\right)^{q^k-q^{k-1}}\leq(\|c\|q^{j-1});\]
hence for every $c$ with $\|c\|\leq q^{-j}$, for $m\gg0$, we get:
    \begin{align*}
        \|S_m\|\leq\|c\|^{q^m}\left(\prod_{i=j+1}^m r_{i-j}^{q^{i-1}-q^i}\right)\left(\prod_{i=1}^m r_i^{q^i-q^{i-1}}\right)=\|c\|\prod_{i=1}^j(\|c\|r_i)^{q^i-q^{i-1}}\prod_{i=j+1}^m\left(\frac{\|c\|r_i}{r_{i-j}}\right)^{q^i-q^{i-1}}.
    \end{align*}

Since for all $i\in\N$ $\frac{r_{i+1}}{r_i}\leq q$, we can bound the right hand side in the following way:
\begin{align*}
    &\limsup_m\frac{1}{m}\log_q\left(\|c\|\prod_{i=1}^j(\|c\|r_i)^{q^i-q^{i-1}}\prod_{i=j+1}^m\left(\frac{\|c\|r_i}{r_{i-j}}\right)^{q^i-q^{i-1}}\right)\\
    =&\limsup_m\frac{1}{m}\log_q\left(\prod_{i=j+1}^m\left(\frac{\|c\|r_i}{r_{i-j}}\right)^{q^i-q^{i-1}}\right)
    \leq\limsup_m\frac{1}{m}\log_q\left(\prod_{i=j+1}^m\frac{\|c\|r_i}{r_{i-j}}\right)\\
    =&\limsup_m\frac{1}{m}\left(\log_q\left(\prod_{i=1}^j\frac{1}{r_i}\right)+(m-j)\log_q(\|c\|)+\sum_{i=m-j+1}^m \log_q(r_i)\right)\leq-j+\frac{j}{r}<0,
\end{align*}
hence the sequence $\|S_m\|$ converges to $0$.
We have the following identities in $\C$:
    \[0=\lim_m S_m=\lim_m\sum_{\lambda\in\Lambda_m\setminus\{0\}}\frac{\exp_\Lambda(c\lambda)^{q^j}}{\lambda}=\sum_{\lambda\in\Lambda\setminus\{0\}}\frac{\exp_\Lambda(c\lambda)^{q^j}}{\lambda}.\]
\end{proof}

\section{Invertible special functions in Drinfeld modules of rank $1$}\label{section question}

We use the considerations of Section \ref{section special functions}, applied to the context of a Drinfeld-Hayes module $\phi$ with lattice $\Lambda$, to answer a question posed by Gazda and Maurischat in \cite{Gazda}.

We know that there is a some $f\in\Frac(\C\otimes A)$, called \textit{shtuka function}, such that, for all $\omega\in\C\hat\otimes A$, $\omega\in Sf_\phi(A)$ if and only if $\omega^{(1)}=f\omega$. In particular, if there is some $\omega\in Sf_\phi(A)$ which is an invertible element of the ring $\C\hat\otimes A$, for all $\omega'\in Sf_\phi(A)$ we have $\left(\frac{\omega'}{\omega}\right)^{(1)}=\frac{\omega'}{\omega}$, i.e. $\frac{\omega'}{\omega}\in\F_q\otimes A$, hence $Sf_\phi(A)=A\cdot\omega$. The conjecture of Gazda and Maurischat in \cite{Gazda} was that the converse is also true. 

First, we prove two results to show that duality is well-behaved with respect to norms. For starters, we endow the space $\widehat{\K}\cong\Omega\otimes_A\K$ with a norm $|\cdot|$ such that it is a normed vector space over $(\K,\|\cdot\|)$, and we use the same notation for the induced norm on the quotient $\hat{A}$; note that $|\cdot|$ is unique up to a scalar factor in $\mathbb{R}^+$.

\begin{prop}\label{norm pairing}
    Up to a scalar factor in $\mathbb{R}^+$, for all $f\in\widehat{\K}\setminus\{0\}$, we have
    \[|f|^{-1}=\min\{\|\lambda\|\text{ s.t. }\lambda\in \K,f(\lambda)\neq0\}.\]
\end{prop}
\begin{proof}
    We can identify $\K$ with $\F_q((t))$, and $\widehat{\K}$ with $\F_q((t))dt$, so that for all $p(t)=\sum_{i\in\Z}\lambda_i t^i\in\F_q((t))$ with leading term $\lambda_k t^k$, $\|p(t)\|=q^{-k}$, and $dt(p)=\lambda_{-1}$. Up to a scalar factor in $\mathbb{R}^+$, we can assume $|dt|=q^{-1}$.
    
    Take $\mu\in\F_q((t))dt$ with leading term $b_k t^k dt$, so that $|\mu|=q^{-k-1}$: if $p\in\F_q((t))$ has $\|p\|< q^{k+1}$, its leading term has degree at least $-k$, hence $\mu(p)=0$; on the other hand $\|t^{-k-1}\|=q^{k+1}$ and $\mu(t^{-k-1})=b_k\neq0$. In particular:
    \[|\mu|^{-1}=\min\{\|p\|\text{ s.t. }p\in \F_q((t)),\mu(p)\neq0\}.\]
\end{proof}
\begin{prop}\label{monotony}
Fix an $\F_q$-basis $(a_i)_{i\in I}$ of $A$ strictly ordered by degree, with $(a_i^*)_{i\in I}$ dual basis of $\hat{A}\cong\faktor{\Omega\otimes_A\K}{\Omega}$. The sequence $(|a_i^*|)_{i\in I}$ is strictly decreasing.
\end{prop}
\begin{proof}
We can assume $I\subseteq\Z$ to be the set of degrees of elements in $A$, and that $a_i$ has degree $i$ for all $i\in I$. For all $i\in I$ set $b_i:=a_i$, while for all $i\in\Z\setminus I$ choose some $b_i\in\K$ with valuation $-i$: every $c\in\K$ can be expressed in a unique way as $\sum_{i\in\Z}\lambda_i b_i$ where $\lambda_i\in\F_q$ for all $i\in\Z$ and $\lambda_i=0$ for $i\gg0$. Denote $(b_i^*)_{i\in\Z}$ the sequence in $\widehat{\K}$ determined by the property $b_i^*(b_j)=\delta_{i,j}$ for all $i,j\in\Z$: every $c\in\widehat{\K}$ can be expressed in a unique way as $\sum_{i\in\Z}\lambda_i b_i^*$ where $\lambda_i\in\F_q$ for all $i\in\Z$ and $\lambda_i=0$ for $i\gg0$. By Proposition \ref{norm pairing}, up to rescaling $|\cdot|$, we have for all $i\in\Z$:
\[|b_i^*|^{-1}=\min\{\|c\|\text{ s.t. }c\in\K\text{ and }b_i^*(c)\neq0\}=\min\left\{\left\|\sum_{j\in\Z}\lambda_j b_j\right\|\text{ s.t. }\lambda_i\neq0\right\}=\|b_i\|.\]
For any $c\in\widehat{\K}$, call $\overline{c}$ its projection onto $\hat{A}$. Since $(b_i)_{i\in I}=(a_i)_{i\in I}$ is an $\F_q$-basis of $A$, $\overline{b_i^*}$ is $a_i^*$ if $i\in I$, and $0$ otherwise. For all $i\in I$, we have:
\[|a_i^*|=\min\{|c|\text{ s.t. }c\in\widehat{K},\overline{c}=a_i^*\}=\min\left\{\left|\sum_{j\in\Z}\lambda_j b_j^*\right|\text{ s.t. }\lambda_j=\delta_{i,j}\forall j\in I\right\}=|b_i^*|=\|a_i\|^{-1}.\] 
\end{proof}

\begin{prop}
    Suppose $Sf_\phi(A)\cong A$. Then, there is a special function in $Sf_\phi(A)$ which is invertible as an element of $\C\hat\otimes A$.
\end{prop}
\begin{proof}
    Fix an $\F_q$-basis $(a_i)_{i\in I}$ of $A$ like in the proof of Proposition \ref{monotony}, with $a_0=1$, and let $(a_i^*)_{i\in I}$ be the dual basis of $\hat{A}\cong\faktor{\Omega\otimes_A\K}{\Omega}\cong\faktor{\K\Lambda}{\Lambda}$.
    
    By Remark \ref{oss universal special function}, the universal special function $\omega_\phi\in\C\hat\otimes A$ can be written as $\sum_i\exp_\Lambda(a_i^*)\otimes a_i$. To prove it is invertible, it suffices to show that for all $i>0$ $\|\exp_\Lambda(a_0^*)\|>\|\exp_\Lambda(a_i^*)\|$: if this is the case, and we call $\omega:=(\exp_\Lambda(a_0^*)^{-1}\otimes1)\omega_\phi$, the series $\sum_n(1-\omega)^n$ converges in $\C\hat\otimes A$, and is an inverse to $\omega$.
    
    For all indices $i$, choose a lifting $c_i\in\K\Lambda\subseteq\C$ of $a_i^*\in\faktor{\K\Lambda}{\Lambda}$ with the least norm; in particular, there are no $\lambda\in\Lambda$ such that $\|\lambda\|=\|c_i\|$, so we have:
    \[\|\exp_\Lambda(a_i^*)\|=\|c_i\|\prod_{\lambda\in\Lambda\setminus\{0\}}\left\|1-\frac{c_i}{\lambda}\right\|=\|c_i\|\prod_{\substack{\lambda\in\Lambda\setminus\{0\}\\\|\lambda\|\leq\|c_i\|}}\left\|1-\frac{c_i}{\lambda}\right\|=\|c_i\|\prod_{\substack{\lambda\in\Lambda\setminus\{0\}\\\|\lambda\|<\|c_i\|}}\left\|\frac{c_i}{\lambda}\right\|.\]
    By Proposition \ref{monotony}, the sequence $(\|c_i\|)_i$ is strictly decreasing, hence the sequence $(\|\exp_\Lambda(a_i^*)\|)_i$ is strictly decreasing.
\end{proof}



\printbibliography
\end{document}



Now we can answer affirmatively to a question posed by Quentin Gazda and Andreas Maurischat in \cite{Gazda}.
\begin{cor}
    If $\Lambda\cong \Omega$, there is an invertible special function in $Sf_A(\phi)$.
\end{cor}
\begin{proof}
    Fix an isomorphism of $A$-modules $l:L\to A$: the special function $\omega:=(1\otimes l)(\omega_\phi)\in\C\hat\otimes A$ generates $Sf_A(\phi)$. Fix an ordered basis $\{b_i\}_{i\geq0}$ of $L$ with $\lambda(b_0)=1$: we have the identity $\omega=\sum_i\exp_\Lambda(b_i^*)\otimes l(b_i)$. If we fix a uniformizer $u\in\K$, by Proposition \ref{monotony} we have:
    \[\red_u(\omega)=\red_u\left(\sum_i\exp_\Lambda(b_i^*)\otimes l(b_i)\right)=\red_u\left(\exp_\Lambda(b_0^*)\otimes l(b_0)\right)=l(b_0)=1.\]
\end{proof}
We can deduce a more general result for $\rk(L)=1$.
\begin{cor}\label{general corollary}
    Fix a uniformizer $u\in\K$ and fix an immersion $L\into K$. We have $\red_u(Sf_A(\phi))=a_L L^{-1}$, where $a_L\in L$ is an element of least degree.
\end{cor}
\begin{proof}
    By Proposition \ref{main result} and Definition \ref{def main result}, a generic special function is written as $(1\otimes l)\omega_\phi$, where $l\in \Hom(L,A)$. If we fix an ordered basis $\{b_i\}_i$ of $L$, with dual basis $\{b_i^*\}_i$ and $b_0=a_L$, we have:
    $\red_u\left((1\otimes l)\omega_\Lambda\right)=\red_u\left(\sum_i\exp_\Lambda(b_i^*)\otimes l(b_i)\right)=\red_u\left(\exp_\Lambda(b_0^*)\otimes l(b_0)\right)=l(b_0)=l(a_L).$
    Since $l$ varies among $L^{-1}$, we have:
    \[\red_u(Sf_\Lambda)=\left\{\red_u\left((1\otimes l)\omega_\Lambda\right)\right\}_{l\in L^{-1}}=\{l(a_L)\}_{l\in L^{-1}}=a_L L^{-1}.\]
\end{proof}
\begin{prop}
    Call $V_{\bar{I}}$ the Drinfeld divisor relative to $\phi$, coming from some ideal $I\unlhd A$. We have $IL\cong A$, and:
    \[\red_\K(V_{\bar{I}})=\Div(1\otimes a_{I^{-1}})+I-(g+\deg(I))\infty.\]
\end{prop}
\begin{proof}
    From \cite{Ferraro}[Thm. 6.3] we know that:
    \[Sf_A(\phi)=\frac{\delta_{\bar{I}}}{(\gamma_I\otimes1)\zeta_I^{(-1)}}(\F_q\otimes I),\]
    for some factor $\gamma_I\in\C$, where $\delta_{\bar{I}}$ is the unique rational function with sign $1$ and divisor $V_{\bar{I}}+V_{\bar{I},*}-2g\infty$. If we apply $\red_u$, by Corollary \ref{general corollary} we get the following:
    \[a_L L^{-1}=\red_u(Sf_\Lambda)=\red_u\left(\frac{\delta_{\bar{I}}}{(\gamma_I\otimes1)\zeta_I^{(-1)}}(\F_q\otimes I)\right)=\frac{\red_u(\delta_{\bar{I}})}{\red_u(\zeta_I^{(-1)})}I=\red_u(\delta_{\bar{I}})a_I^{-1}I.\]
    In particular, we get that $L^{-1}\cong I$ as $A$-modules, hence WLOG we can set $L=I^{-1}$. We deduce that
    \[\red_u(\delta_{\bar{I}})=a_{I^{-1}}a_I,\]
    therefore:
    \[\Div(1\otimes a_{I^{-1}})+\Div(1\otimes a_I)=\Div(\red_u(\delta_{\bar{I}}))=\red_\K(\Div(\delta_{\bar{I}}))=\red_\K(V_{\bar{I}})+\red_\K(V_{\bar{I},*})-2g\infty.\]
    On the other hand, we already know that $\red_\K(V_{\bar{I}.*})=\Div(1\otimes a_I)-I+(g+\deg(I))\infty$, hence we deduce:
    \[\red_\K(V_{\bar{I}})=\Div(1\otimes a_{I^{-1}})+I-(g+\deg(I))\infty.\]
\end{proof}
\begin{oss}
    If we call $J$ the unique fractional ideal in $K$ isomorphic to $I^{-1}$ and with $J(\leq0)=\F_q^\times$, with obvious notation we have $\red_\K(V_{\bar{I}})=\Div(J)$. Similarly, if we call $J'$ the unique fractional ideal in $K$ isomorphic to $I$ and with $J'(\leq0)=\F_q^\times$, we have $\red_\K(V_{\bar{I},*})=\Div(J')$. In particular, their sum is $\Div(JJ')$.
\end{oss}
\section{Shtuka matrix and dual shtuka matrix}
We want to find an analogue to the shtuka and dual shtuka functions for Drinfeld modules of arbitrary rank. Fix a Drinfeld module $\phi$ with lattice $\Lambda$ of rank $r$ and call $L:=\Hom_A(\Lambda,\Omega)$.
\begin{prop}
    There is a rational function $f$ over $X_\K$ and an element $M\in\K\otimes\End_A(L)$ such that:
    \[\omega_\phi^{(1)}=f M\omega_\phi.\]

    There is a rational function $f_*$ over $X_\K$ and an element $M_*\in\K\otimes\End_A(\Lambda)$ such that:
    \[\zeta_\phi^{(-1)}=f_* M_*\zeta_\phi.\]
\end{prop}
\begin{proof}
    For simplicity, we just prove the first statement. Let's prove that the $\K\otimes A$-module spanned by the elements of $Sf_\Lambda$ and all of their twists is projective of rank $r^2$. 
\end{proof}
\begin{prop}
    The element $\theta_L:=\sum_i b_i^*\otimes b_i\in \hat{L}\hat\otimes L$ does not depend on the choice of the basis $\{b_i\}_i$.
\end{prop}
\begin{proof}
    Let's pick a different standard basis $\{b_i'\}_i$: for all $i$ we can write in a unique way $b_i'=\sum_j \alpha_{i,j}b_j$ with $\alpha_{i,j}\in\F_q$. As observed in the previous proof, we have:
    \[b_j^*=\sum_i b_j^*(b'_i){b'_i}^*=\sum_i\alpha_{i,j}{b'_i}^*,\]
    where the sum is possibly infinite. Now we can prove the statement:
    \begin{align*}
        \sum_i {b'_i}^*\otimes b'_i
        =\sum_i \left({b'_i}^*\otimes\sum_j \alpha_{i,j}b_j\right)
        =\sum_i\sum_j {b'_i}^*\otimes\alpha_{i,j} b_j
        =\sum_j\sum_i \alpha_{i,j}{b'_i}^*\otimes b_j
        =\sum_j b_j^*\otimes b_j.
    \end{align*}
\end{proof}
\begin{teo}\label{teo}
    For all $a\in A$, we have $(a\otimes 1)\theta_L=(1\otimes a)\theta_L$.
\end{teo}
\begin{proof}
    First of all, note that we can give $L$ a discrete norm (for example by choosing an immersion $L\into\C$ as a discrete subset), with the property that $|ax|=|a|\cdot |x|$ for all $x\in L$ and $a\in A$, where on $A$ we consider the canonical norm $|a|=q^{\deg(a)}$.

    Fix $a\in A$: for all positive integers $n$ we $L/a^n L$ is an $\F_q$-vector space of dimension $n\deg(a)r$. We pick recursively a sequence $\{b_i\}_i$ in $L$ in the following way: for $i<\deg(a)r$, $b_i$ is an element of least norm in $L$ such that its projection to $L/aL$ is not spanned by the projections of $\{b_j\}_{j<i}$; for $i\geq\deg(a)r$, $b_i:=a b_{i-\deg(a)r}$.
    
    Let's prove by induction that for all $n>0$ the projections of $\{b_i\}_{i<n\deg(a)r}$ onto $L/a^nL$ are an $\F_q$-basis. The case $n=1$ is true by construction. Suppose by inductive hypothesis that the statement is true for $m>0$. Then, $\{b_i\}_{\deg(a)r\leq i<(m+1)\deg(a)r}=\{a b_i\}_{i<m\deg(a)r}$, projected onto $L/a^{m+1}L$, is an $\F_q$-basis of $aL/a^{m+1}L$; they are $\F_q$-linearly independent from the projections of $\{b_i\}_{i<\deg(a)r}$, and together they span the entire $L/a^{m+1}L$.

    We claim that $\{b_i\}_i$, is an $\F_q$-basis of $L$. On one hand, they are independent because for all $n>0$ the projections of the first $n\deg(a)r$ onto $L/a^n L$ are $\F_q$-independent. If by contradiction they do not generate $L$, we can pick an element $x\in L\setminus\Span_{\F_q}\left(\{b_i\}_i\right)$ with the least norm. There is a unique $\F_q$-linear combination $\tilde{x}:=\sum_{i<\deg(a)r} \alpha_i b_i$ such that $x-\tilde{x}=ay$ for some $y\in L$. On one hand, if $|x|<|\tilde{x}|$, there is some $b_i$ with $\alpha_i\neq0$ and $|x|<|b_i|$; this is a contradiction, because both $x$ and $b_i$ are elements such that their projection to $L/aL$ are not spanned by the projections of $\{b_j\}_{j<i}$, so by the construction of $b_i$ we should have $|b_i|\leq|x|$. Since $|x|\geq|\tilde{x}|$, we have $|y|<|ay|\leq|x|$, so, by minimality of $x$, there is an $\F_q$-linear combination $y=\sum_j \beta_j b_j$; therefore, we can write:
    \[x=\tilde{x}+ay=\sum_{i<\deg(a)r} \alpha_i b_i+a\sum_j \beta_j b_j=\sum_{i<\deg(a)r} \alpha_i b_i+\sum_j \beta_j b_{j+\deg(a)r}\in\Span_{\F_q}\left(\{b_i\}_i\right),\]
    which is a contradiction.

    We define $\{b_i^*\}_i$ in the usual way, and we can write $\theta_L=\sum_{i\in\Z}b_i\otimes b_i^*$ - where by convention, for $i<0$, we set $b_i^*=0$ and $b_i=0$.
    Note that the action of $A$ on $\hat{L}$ is such that, for all $f\in\hat{L},c\in L$, $(af)(c):=f(ac)$, so we have the following:
    \[a b_j^*=\sum_i (a b_j^*)(b_i)b_i^*=\sum_i b_j^*(ab_i)b_i^*=\sum_i b_j^*(b_{i+\deg(a)r})b_i^*=b_{j-\deg(a)r}^*,\]
    with the same convention on indices as before. Now we can easily prove the thesis:
    \[(1\otimes a)\theta_L=\sum_{i\in\Z}b_i^*\otimes a b_i=\sum_{i\in\Z}b_i^*\otimes b_{i+\deg(a)r}=\sum_{i\in\Z} b_{i-\deg(a)r}^*\otimes b_i=\sum_{i\in\Z}a b_i^*\otimes b_i=(a\otimes 1)\theta_L.\]
\end{proof}
Fix an immersion $\Lambda\into\C$ as a discrete submodule and extend it to the injection $\K\otimes_A\Lambda\into\C$. Consider $\exp_\Lambda:\C\to\C$ and restrict it to $\K\otimes_A\Lambda$: it factors through $\K\otimes_A\Lambda/\Lambda\cong\hat{L}$, so we can apply the exponential to the first coordinate of $\theta_L$, obtaining the following function in $\C\otimes L$ (expressed with respect to an $\F_q$-basis $\{b_i\}_i$ of $L$):
\[\omega_\Lambda:=\exp_\Lambda(\theta_L)=\sum_i \exp_\Lambda(b_i^*)\otimes b_i.\]
Call $\phi:A\to\C\{\tau\}$ the Drinfeld module relative to $\exp_\Lambda$, consider the action of $\C\{\tau\}$ on $\C\otimes L$ on the first coordinate. We have the following proposition.
\begin{cor}\label{prop}
    For all $a\in A$, $\phi_a(\omega_\Lambda)=(1\otimes a)\omega_\Lambda$.
\end{cor}
\begin{proof}
    For all $a\in A$ we have:
    \[\phi_a(\omega_\Lambda)=\phi_a\circ\exp_\Lambda(\theta_L)=\exp_\Lambda((a\otimes1)\theta_L)=\exp_\Lambda((1\otimes a)\theta_L)=(1\otimes a)\exp_\Lambda(\theta_L)=(1\otimes a)\omega_\Lambda.\]
    %Fix $a\in A$ and pick the $\F_q$-basis $\{b_i\}_i$ of $L$ relative to $A$ as in Theorem \ref{teo}. We have:\begin{align*}\phi_a(\omega_\Lambda)&=\sum_{i\in\Z}\phi_a\circ\exp_\Lambda(b_i^*)\otimes b_i=\sum_{i\in\Z}\exp_\Lambda(a b_i^*)\otimes b_i=\sum_{i\in\Z}\exp_\Lambda(b_{i-\deg(a)r}^*)\otimes b_i\\&=\sum_{i\in\Z}\exp_\Lambda(b_i^*)\otimes b_{i+\deg(a)r}=\sum_{i\in\Z}\exp_\Lambda(b_i^*)\otimes a b_i=(1\otimes a)\omega_\Lambda.\end{align*}
\end{proof}
For a function $\lambda\in\Hom_A(L,K)\cong K^r$, define $1\otimes\lambda:\C\hat\otimes L\to\C\hat\otimes K$ by acting on the second coordinate. We have the following result.
\begin{prop}\label{images of omega_Lambda}
    The $1\otimes K$-linear function $\Hom_A(L,K)\to\C\hat\otimes K$ sending $\lambda$ to $(1\otimes\lambda)(\omega_\Lambda)$ is injective. Moreover, for every element $\omega$ in the image, for all $a\in A$, $\phi_a(\omega)=(1\otimes a)\omega$.
\end{prop}
\begin{proof}
    Let's first prove the second statement. For all $a\in A$, for all $\lambda\in\Hom_A(L,K)$, using Corollary \ref{prop} and the $A$-linearity of $\lambda$, we have:
    \[\phi_a\left((1\otimes\lambda)(\omega_\Lambda)\right)=(1\otimes\lambda)\left(\phi_a(\omega_\Lambda)\right)=(1\otimes\lambda)\left((1\otimes a)\omega_\Lambda\right)=(1\otimes a)\left((1\otimes\lambda)(\omega_\Lambda)\right).\]

    Pick a nonzero $\lambda\in\Hom_A(L,K)$ with kernel $L'\subsetneq L$, and suppose $(1\otimes\lambda)\omega_\Lambda=0$. Since $\lambda(L)\subseteq K$ is a projective $A$-module, there is a section $\lambda(L)\into L$ which induces an isomorphism $L\cong L'\oplus\lambda(L)$. Fix $\F_q$ bases $\{b_i\}_i$ of $\lambda(L)\subseteq L$ and $\{b_i'\}_i$ of $L'$, with dual basis $\{b_i^*\}_i\cup\{{b'_i}^*\}_i$ of $\hat{L}$, so that we have:
    \[0=(1\otimes\lambda)\omega_\Lambda=\sum_i\exp_\Lambda(b_i^*) \otimes \lambda(b_i)+\sum_i\exp_\Lambda({b'_i}^*) \otimes \lambda(b'_i)=\sum_i\exp_\Lambda(b_i^*) \otimes \lambda(b_i).\]
    Since $\{\lambda(b_i)\}_i$ are $\F_q$-linearly independent, we get that $\exp_\Lambda(b_i^*)=0$ for all $i$, which is a contradiction because, for all $i$, $b_i^*$ is a nonzero element of $\K\otimes\Lambda/\Lambda$.
\end{proof}
