\documentclass[pdflatex,sn-mathphys-num]{sn-jnl}
\usepackage [T1]{fontenc}
\usepackage [utf8]{inputenc}
\usepackage [english]{babel}
\usepackage {amsfonts}
\usepackage {amssymb}
\usepackage {mathbbol}

\usepackage {amsthm}
\usepackage {amsopn}
\usepackage {mathtools}
\usepackage {tikz-cd}
\usepackage{csquotes}
\usepackage{xfrac}
\usepackage{faktor}
\usepackage{dsfont}
\usepackage{bm}

\usepackage{smartdiagram}

  \newtheorem{innercustomprop}{Proposition}
\newenvironment{customprop}[1]
  {\renewcommand\theinnercustomprop{#1}\innercustomprop}
  {\endinnercustomprop}

\newcommand{\ols}[1]{\mskip.5\thinmuskip\overline{\mskip-.5\thinmuskip {#1} \mskip-.5\thinmuskip}\mskip.5\thinmuskip} % overline short
\newcommand{\olsi}[1]{\,\overline{\!{#1}}} % overline short italic
\makeatletter
\newcommand\closure[1]{
  \tctestifnum{\count@stringtoks{#1}>1} %checks if number of chars in arg > 1 (including '\')
  {\ols{#1}} %if arg is longer than just one char, e.g. \mathbb{Q}, \mathbb{F},...
  {\olsi{#1}} %if arg is just one char, e.g. K, L,...
}
\long\def\count@stringtoks#1{\tc@earg\count@toks{\string#1}}
\long\def\count@toks#1{\the\numexpr-1\count@@toks#1.\tc@endcnt}
\long\def\count@@toks#1#2\tc@endcnt{+1\tc@ifempty{#2}{\relax}{\count@@toks#2\tc@endcnt}}
\def\tc@ifempty#1{\tc@testxifx{\expandafter\relax\detokenize{#1}\relax}}
\long\def\tc@earg#1#2{\expandafter#1\expandafter{#2}}
\long\def\tctestifnum#1{\tctestifcon{\ifnum#1\relax}}
\long\def\tctestifcon#1{#1\expandafter\tc@exfirst\else\expandafter\tc@exsecond\fi}
\long\def\tc@testxifx{\tc@earg\tctestifx}
\long\def\tctestifx#1{\tctestifcon{\ifx#1}}
\long\def\tc@exfirst#1#2{#1}
\long\def\tc@exsecond#1#2{#2}
\makeatother

\usepackage{yhmath}

  
\newcommand{\catname}[1]{{\normalfont\textbf{#1}}}
\newcommand{\C}{{\mathbb{C}_\infty}}
\newcommand{\CC}{{\mathbb{C}_\F}}
\newcommand{\K}{{K_\infty}}
\newcommand{\KK}{{K_v}}
\newcommand{\FF}{{\closure{\mathbb{F}_q}}}
\newcommand{\F}{\mathbb{F}}
\newcommand{\T}{\mathbb{T}}
\newcommand{\N}{\mathbb{N}}
\newcommand{\Q}{\mathbb{Q}}
\newcommand{\R}{\mathbb{R}}
\newcommand{\Z}{\mathbb{Z}}
\renewcommand{\P}{\mathcal{P}}
\newcommand{\m}{\mathfrak{m}}
\newcommand{\mm}{\tilde{\mathfrak{m}}}
\newcommand{\p}{\mathfrak{p}}
\newcommand{\q}{\mathfrak{q}}
\newcommand{\A}{\mathcal{A}}
\renewcommand{\O}{\mathcal{O}}
\newcommand{\U}{\mathcal{U}}
\newcommand{\OL}{\widehat{\Z_p[p^{\frac{1}{p^\infty}}]}}
\renewcommand{\L}{\mathcal{L}}
\newcommand{\I}{\mathcal{I}}
\newcommand{\G}{\text{Gal}}
\newcommand{\red}{\text{red}}
\newcommand{\Div}{\text{Div}}
\newcommand{\Pic}{\text{Pic}}
\newcommand{\Spec}{\text{Spec}}
\newcommand{\Frob}{\text{Frob}}
\DeclareMathOperator{\Frac}{Frac}
\newcommand{\ch}{\text{char}}
\DeclareMathOperator{\Hom}{Hom}
\DeclareMathOperator{\res}{res}
\DeclareMathOperator{\Lie}{Lie}
\DeclareMathOperator{\Span}{Span}
\DeclareMathOperator{\End}{End}
\DeclareMathOperator{\Aut}{Aut}
\DeclareMathOperator{\Tr}{Tr}
\DeclareMathOperator{\Ext}{Ext}
\DeclareMathOperator{\Tor}{Tor}
\DeclareMathOperator{\coker}{coker}
\DeclareMathOperator{\im}{im}
\DeclareMathOperator{\rk}{rk}
\DeclareMathOperator{\Mat}{Mat}
\DeclareMathOperator{\GL}{GL}
\DeclareMathOperator{\Sf}{Sf}
\newcommand{\Mod}{\catname{Mod}}
\newcommand{\into}{\hookrightarrow}
\newcommand{\onto}{\twoheadrightarrow}
\renewcommand{\baselinestretch}{1.0}
\newtheorem{teo}{Theorem}[section]
\newtheorem{Teo}{Theorem}
\newtheorem{lemma}[teo]{Lemma}
\newtheorem{fact}[teo]{Fact}
\newtheorem{prop}[teo]{Proposition}
\newtheorem{Prop}[Teo]{Proposition}
\newtheorem{cor}[teo]{Corollary}
\newtheorem{conj}[teo]{Conjecture}
\theoremstyle{definition}
\newtheorem{Def}[teo]{Definition}
\newtheorem{con}[teo]{Construction}
\newtheorem{oss}[teo]{Remark}
\newtheorem{Oss}{Remark}
\newtheorem{ex}[teo]{Example}

\begin{document}
\title{A duality result about special functions in Drinfeld modules of arbitrary rank}
\author*[]{\fnm{Giacomo Hermes} \sur{Ferraro}}\email{giacomohermes.ferraro@gmail.com}

\affil*{\orgdiv{Dipartimento di Matematica Guido Castelnuovo}, \orgname{Sapienza Università di Roma}, \orgaddress{\street{Piazzale Aldo Moro, 5}, \city{Rome}, \postcode{00183}, \state{Lazio}, \country{Italy}}}

\keywords{Drinfeld modules, Anderson modules, Pellarin $L$-series, shtuka functions, special functions}
\abstract{
In the setting of a Drinfeld module $\phi$ over a curve $X/\F_q$ with an $\F_q$-rational point $\infty$, we use a functorial point of view to define \textit{Drinfeld eigenvectors}, a generalization of the so called "special functions" introduced by Anglès, Ngo Dac and Tavares Ribeiro. We adopt an analogous approach with the dual Drinfeld module $\phi^*$ to define \textit{dual Drinfeld eigenvectors}, and we prove that both functors are representable with universal objects $\omega_\phi$ and $\zeta_\phi$, respectively. We prove that $\zeta_\phi$ can be expressed as an Eisenstein-like series over the period lattice; moreover, for all integers $i$ we define dot products $\zeta_\phi\cdot\omega_\phi^{(i)}$, which are meromorphic differential forms over a base change of the curve $X\setminus\infty$, and prove they are actually rational. These two results complete the investigation started by the author in a previous work on Drinfeld modules of rank $1$, albeit with very different methods. In this paper, although we are not able to compute the divisors of those differential forms for arbitrary curves, we give a working description of the generating series $\sum_{i\in\Z}\zeta_\phi\cdot\omega_\phi^{(i)}\tau^i$, and we apply it to compute the forms $\zeta_\phi\cdot\omega_\phi^{(i)}$ in the case of genus $0$.
    
We also describe the module of special functions for Anderson modules, as already done by Gazda and Maurischat, and prove their conjecture about the invertibility of special functions for Drinfeld modules of rank $1$.
}
\maketitle


\section{Introduction}
Let $\F_q$ be the finite field with $q$ elements, and let $(X,\O_X)$ be a projective, geometrically irreducible, smooth curve of genus $g(X)$ over $\F_q$, with a point $\infty\in X(\F_q)$. We define $A\coloneqq\O_X(X\setminus\infty)$, with the degree function $\deg:A\to\mathbb{N}$ being the opposite of the valuation at $\infty$. We denote by $\Omega$ the module of K\"ahler differentials of $A$ over $\F_q$, by $K$ the fraction field of $A$, by $\K$ the completion of $K$ with respect to the $\infty$-adic norm $\|\cdot\|$, and by $\C$ the completion of an algebraic closure $\closure{K_\infty}$. By convention, for all $a\in A$, we set $\|a\|=q^{\deg(a)}$. Tensor products without pedices are implied to be over $\F_q$.

Let $\C[\tau]\subseteq\End_{\F_q}(\C)$ be the sub-$\C$-algebra generated by the Frobenius endomorphism. A \textit{Drinfeld module} $\phi$ of rank $r$ is a ring homomorphism $A\to\C[\tau]$ such that, for all $a\in A$, $\phi_a=\sum_i\phi_{a,i}\tau^i$ has degree $r\cdot\deg(a)$ in $\tau$, and its constant term is $a$. There is a unique power series $\exp_\phi\in x+O(x^2)\subseteq\C[[x]]$ such that $\phi_a\circ\exp_\phi(x)=\exp_\phi(ax)$ in $\C[[x]]$ for all $a\in A$; moreover, $\exp_\phi:\C\to\C$ is $\F_q$-linear, surjective, and everywhere converging. If we call $\Lambda_\phi\coloneqq\ker(\exp_\phi)$, $\Lambda_\phi\subseteq\C$ is a discrete projective $A$-module of rank $r$, and $\exp_\phi(x)=x\prod_{\lambda\in\Lambda_\phi\setminus\{0\}}\left(1-\frac{x}{\lambda}\right)$ (see \cite{Goss} for details).

The special case of the \textit{Carlitz module} is the easiest to study: we suppose $X=\mathbb{P}^1_{\F_q}$, so that $A=\F_q[\theta]$ for some rational function $\theta$, and set $\phi_\theta\coloneqq\theta+\tau$; in this case, $\Lambda_\phi=\tilde{\pi}A$ for some $\tilde{\pi}\in\mathbb{C}_\infty^\times$. Working in the Tate algebra $\C\langle t\rangle$, i.e. the set of converging power series in $t$ on the unit disc of $\C$, Anderson and Thakur introduced in \cite{AndersonThakur} the function $\omega$, defined as the unique series $\sum_{i\geq0}c_it^i$ such that $c_0=1$ and
\[\sum_{i\geq0}\phi_\theta(c_i)t^i=\sum_{i\geq0}c_i t^{i+1}.\]
This series has various uses: for example, as shown in \cite{AP14} by Anglès and Pellarin, $\omega$ is connected to the explicit class field theory of $\F_q(\theta)$; moreover, its $\FF$-rational values interpolate Gauss-Thakur sums. In \cite{Pellarin2011}, Pellarin proved the following identity in $\C\langle t\rangle$, relating $\omega$ to the so-called Pellarin zeta function $\zeta$ attached to $\mathbb{P}^1_{\F_q}$:
\begin{equation}\label{eq. Pellarin}
    \frac{\tilde{\pi}}{(\theta-t)\omega}=-\sum_{h\in\F_q[t]\setminus\{0\}}\frac{h}{h(\theta)}=:\zeta.
\end{equation}

The analogue to $\C\langle t\rangle$ for a general $A$ is the completion $\C\closure\otimes A$ of the ring $\C\otimes A$ with respect to the sup norm induced by $\|\cdot\|$.
The module of "special functions" (as defined in \cite{ANDTR} by Anglès, Ngo Dac, and Tavares Ribeiro) generalizes the Anderson-Thakur function to any Drinfeld module $\phi$ as follows:
\[\Sf(\phi)\coloneqq\{\omega\in\C\closure\otimes A|\phi_a(\omega)=(1\otimes a)\omega\;\forall a\in A\},\]
where $\phi_a$ acts on the first coordinate, i.e. it sends an infinite series $\sum_i c_i\otimes a_i\in\C\closure\otimes A$ to $\sum_i\phi_a(c_i)\otimes a_i$. The Pellarin zeta function also has an obvious generalization as an object of $\C\closure\otimes A$: $\zeta_A\coloneqq-\sum_{a\in A\setminus\{0\}}a^{-1}\otimes a$. In a letter to the author in 2021, Pellarin conjectured that the product of a Pellarin zeta function and a special function belongs to the fraction field of $\C\otimes A$ for any Drinfeld module of rank $1$: Green and Papanikolas had already proven this statement in the case $g(X)=1$ (\cite{Green}[Thm. 7.1]). 

In the paper \cite{Ferraro}, the author proved the conjecture for any genus $g(X)\geq1$ (\cite{Ferraro}[Thm. 6.3]), and also showed the following identities for all $a\in A$: 
\[\phi^*_a(\zeta)=(1\otimes a)\zeta,\] where $\phi^*:A\to\C[\tau^{-1}]$ is the \textit{dual Drinfeld module}, i.e. the ring homomorphism that sends $a\in A$ to $\phi^*_a=\sum_i\tau^{-i}\phi_{a,i}$, and $\zeta$ is a modified version of $\zeta_A$ (see \cite{Ferraro}[Prop. 7.8] and \cite{Ferraro}[Prop. 7.20]); this suggests that $\zeta$ ought to be interpreted as a "dual special function".

In this paper, we generalize the previous results to Drinfeld modules of arbitrary rank. It turns out that the algebra $\C\closure\otimes A$ is not the natural environment for "special functions", so for any $A$-module $M$ we define the $A$-module of \textit{Drinfeld eigenvectors} as follows:
\[\Sf_\phi(M)\coloneqq\{\omega\in\C\closure\otimes M|\phi_a(\omega)=(1\otimes a)\omega\;\;\forall a\in A\};\]
this definition is functorial, and generalizes both the "special functions" (obtained by setting $M=A$) and the Gauss-Thakur sums, as they have been intepreted in \cite{Gazda} by Gazda and Maurischat (obtained by setting $M=\faktor{A}{\mathfrak{p}}$ for some prime ideal $\mathfrak{p}<A$). 

We show that $\C\closure\otimes M$ is canonically isomorphic to the set of continuous $\F_q$-linear functions from the Pontryagin dual $\hat{M}$ of $M$ to $\C$ (Proposition \ref{duality tensor hom 2}), so that:
\[\Sf_\phi(M)=\{f\in\Hom_{\F_q}^{cont}(\hat{M},\C)| f(a\;\cdot\,)=\phi_a(f(\cdot))\;\;\forall a\in A\}.\] 
Since the Pontryagin dual of $\Hom_A(\Lambda_\phi,\Omega)$ is isomorphic to $\faktor{\K\Lambda_\phi}{\Lambda_\phi}$, as proven by Poonen in \cite[Thm. 8]{Poonen}, we are able to prove the following theorem.
\begin{Teo}[Thm. \ref{main result}]
    The functor $\Sf_\phi$ is naturally isomorphic to $\Hom_A(\Lambda_\phi^*\otimes_A\Omega,\_)$, and the universal object in $\Hom_{\F_q}^{cont}\left(\faktor{\K\Lambda_\phi}{\Lambda_\phi},\C\right)$ sends the projection of $c\in\K\Lambda_\phi$ to $\exp_\phi(c)$.
\end{Teo}
We call the universal object $\omega_\phi\in\C\closure\otimes\Hom_A(\Lambda_\phi,\Omega)$ the \textit{universal Drinfeld eigenvector}.
In section \ref{section special functions} the previous proposition is actually stated in the generality of Anderson modules, to recover the state-of-the-art result obtained in \cite{Gazda} by Gazda and Maurischat, describing the isomorphism class of the module of special functions.

We apply a similar reasoning to the dual Drinfeld module $\phi^*$: we define, for any $A$-module $M$, $\Sf_{\phi^*}(M)\subseteq\C\closure\otimes M$ as the $A$-module of \textit{dual Drinfeld eigenvectors}, i.e. the set of the elements $\zeta\in\C\closure\otimes M$ such that $\phi^*_a(\zeta)=(1\otimes a)\zeta$ for all $a\in A$.
\begin{Teo}[Thm.\ref{zeta function functor}]
    The functor $\Sf_{\phi^*}$ is naturally isomorphic to $\Hom_A(\Lambda_\phi,\_)$, and the universal object in $\C\closure\otimes\Lambda_\phi$ is $-\sum_{\lambda\in\Lambda_\phi\setminus\{0\}}\lambda^{-1}\otimes\lambda$.
\end{Teo}
We call the universal object $\zeta_\phi\in\C\closure\otimes\Lambda_\phi$ the \textit{universal dual Drinfeld eigenvector}. The expression of $\zeta_\phi$ as an Eisenstein-like series, as expected, makes it strongly related to Pellarin zeta functions when $\phi$ has rank $1$. 
The element $\zeta_\phi$ is also naturally connected to the Eisenstein series constructed in \cite{Pellarin2024}[Chapter 5, 7] (we deal with the case of general $X$ while the author of ibid. only consider the case of the projective line allowing at once several variables).

The canonical $\C\closure\otimes A$-linear pairing $(\C\closure\otimes\Lambda_\phi)\otimes(\C\closure\otimes\Hom_A(\Lambda_\phi,\Omega))\to\C\closure\otimes\Omega$ (see Lemma \ref{ev}), which we simply call \textit{dot product}, allows us to formulate the following analogue to \cite{Ferraro}[Thm 6.3], where we use the notation $\omega_\phi^{(k)}\coloneqq(\tau^k\otimes1)(\omega_\phi)$ for all integers $k$.
\begin{Teo}[Thm. \ref{Teo rationality}]
    For any Drinfeld module $\phi$, for all integers $k$, the dot product $\zeta_\phi\cdot\omega_\phi^{(k)}$ in $\C\closure\otimes\Omega$ is a rational differential form over the base-changed curve $X_{\C}$. Moreover, for all positive integers $k$, $\zeta_\phi\cdot\omega_\phi^{(k)}\in \C\otimes\Omega$.
\end{Teo}

In the proof of Theorem \ref{Teo rationality} we make the following remark.

\begin{Oss}\label{oss1}
    Since $\omega_\phi$ is a Drinfeld eigenvector, if we fix $a\in A\setminus\F_q$ with $\phi_a=\sum_i a_i\tau^i$ we get:
    \[(a^{q^k}\otimes1-1\otimes a)\zeta_\phi\cdot\omega_\phi^{(k)}=-\sum_{i=0}^{r\deg(a)}a_i^{q^k}\zeta_\phi\cdot\omega_\phi^{(k+i)}.\]
    In particular, if we already know $\zeta_\phi\cdot\omega_\phi^{(k)}$ for $0\leq k<r\deg(a)$, we can compute recursively the differential forms for all integer values of $k$.
\end{Oss}


In this paper, working with a Drinfeld module $\phi$ of arbitrary rank $r$, we lose the knowledge of the divisors of the differential forms $\zeta_\phi\cdot\omega_\phi^{(k)}$ for a generic curve, which are instead explicitly described in \cite{Ferraro}. On the other hand, we are able to prove the following result about their generating series, where we identify $\C\closure\otimes\Omega$ with the set of continuous $\F_q$-linear homomorphisms from $\faktor{\K}{A}$ to $\C$, as per Proposition \ref{complete tensor product} and \cite[Thm. 8]{Poonen}.

\begin{Teo}[Thm. \ref{main teo}]
Let $\Phi,\hat{\Phi}:\K\to\C[[\tau]][\tau^{-1}]$ be the unique ring homomorphisms which extend respectively $\phi,\phi^*:A\to\C[[\tau]][\tau^{-1}]$ and such that their $k$-th coefficient is a continuous function from $\K$ to $\C$ for all $k\in\Z$. The following identity holds in the $\C[\tau,\tau^{-1}]$-module $\C[[\tau,\tau^{-1}]]$ for all $c\in\K$:
    \[\sum_{k\in\mathbb\Z} \left(\zeta_\phi\cdot\omega_\phi^{(k)}\right)(c)\tau^k=\Phi_c^*-\hat{\Phi}_c.\]
\end{Teo}

In the case $A=\F_q[\theta]$, so that $\Omega=Ad\theta$, the previous theorem allows us to prove the following result.
\begin{Teo}[Thm. \ref{g=0 i=0}]
    Suppose $A=\F_q[\theta]$ and let $\phi$ be a Drinfeld module of rank $r$. We have the following identities in $\C\closure\otimes\Omega$:
    \begin{align*}
        \zeta_\phi\cdot\omega_\phi&=\frac{d\theta}{\theta\otimes1-1\otimes\theta};\\
        \zeta_\phi\cdot\omega_\phi^{(k)}&=0\text{ if }1\leq k\leq r-1.
    \end{align*}
\end{Teo}
The previous identities, when $r=1$, imply the original identity (\ref{eq. Pellarin}) proved by Pellarin in \cite{Pellarin2011} in the context of the Carlitz module.
By Remark \ref{oss1}, knowing the coefficients of $\phi_\theta$, we can compute recursively $\zeta_\phi\cdot\omega_\phi^{(k)}$ for all $k$. 

Hartl and Juschka proved in \cite{Hartl}[Thm. 5.13] for any abelian $A$-module $E$ that there is a perfect pairing of $\C\otimes A$-modules $\check{M}(E)\otimes_{\C\otimes A}\tau^*M(E)\to\C\otimes\Omega$ between the dual $A$-motive and the twist of the $A$-motive. Using the explicit computations of Theorem \ref{g=0 i=0}, we prove the following:
\begin{Teo}[Prop. \ref{prop motive immersion}, Thm. \ref{teo restricted pairing}]
    Let $\phi$ be a Drinfeld module. There are a canonical immersion of left $\C\otimes A[\tau]$-modules $\tau^*M(\phi)\subseteq\C\overline\otimes\Hom_A(\Lambda_\phi,\Omega)$ and a canonical immersion of left $\C\otimes A[\tau^{-1}]$-modules $\check{M}(\phi)\subseteq\C\overline\otimes\Lambda_\phi$. If we suppose $A=\F_q[\theta]$, the restriction of the dot product to these submodules coincides with the Hartl-Juschka pairing.
\end{Teo}

Theorem \ref{main teo} also allows us to outline an algorithm to compute the differential forms for any given Drinfeld module on any given curve.
As an example, we apply this algorithm to the simple case of a normalized Drinfeld module $\phi$ of rank $1$ on an elliptic curve $X$, so that $\Omega=A\nu$ for a certain $\nu\in\Omega$, to recover the same result originally found by Green and Papanikolas in \cite{Green}.
\begin{Prop}[Prop. \ref{prop genus 1}]
    Suppose $A=\faktor{\F_q[x,y]}{y^2-P(x)}$ for some separable monic polynomial $P$ of degree $3$. Let $\phi:A\to\C[\tau]$ be a normalized Drinfeld module of rank $1$, with $\phi_x=\tau^2+x_1\tau+x$ and $\phi_y=\tau^3+y_2\tau^2+y_1\tau+y$. We have the following identity in $\C\closure\otimes\Omega$:
    \[\zeta_\phi\cdot\omega_\phi=\frac{(y-x y_2+x x_1^q)\otimes1+(y_2-x_1^q)\otimes x+1\otimes y}{x\otimes1-1\otimes x}\nu.\]
\end{Prop}




 To conclude, in Section \ref{section question}, we answer affirmatively to a question that Gazda and Maurischat posed in the paper \cite{Gazda} about the invertibility of special functions in the context of Drinfeld modules of rank $1$.

\begin{Teo}[Thm. \ref{teo Gazda}]
    Suppose $\Sf_\phi(A)\cong A$. Then, there is a special function in $\Sf_\phi(A)$ which is invertible as an element of $\C\closure\otimes A$.
\end{Teo}

\subsection*{Acknowledgements}
The author is thankful to Matt Papanikolas for the interesting and enlightening discussions relating to the contents of the present paper, and to Federico Pellarin for his guidance and suggestions.

The author thanks the Department of Mathematics Guido Castelnuovo, where he has carried out his Ph.D. studies. This work has been produced as part of the Ph.D. thesis of the author.

\section{Pontryagin duality of \texorpdfstring{$A$}{A}-modules}\label{section Pontryagin duality}

\subsection{Basic statements about Pontryagin duality}

In this paper, compact and locally compact spaces are always assumed to be Hausdorff.

\begin{Def}[Pontryagin duality]
    Call $\mathbb{S}^1\subseteq\mathbb{C}^\times$ the complex unit circle. For any commutative ring with unity $B$, the \textit{Pontryagin duality} is a contravariant functor from the category of topological $B$-modules to itself, sending $M$ to the set of continuous group homomorphism $\hat{M}\coloneqq\Hom_\Z^{cont}(M,\mathbb{S}^1)$, endowed with the compact open topology and with the natural $B$-module structure.
\end{Def}
The following well known result, which we do not prove, justifies the terminology "duality".
\begin{prop}\label{Pontryagin duality}
    For any ring $B$ and any topological $B$-module $M$, consider the group homomorphism $i_M:M\to\hat{\hat M}$ sending $m\in M$ to $(f\mapsto f(m))$. The map $i_M$ is actually a continuous $B$-linear homomorphism; if $M$ is locally compact, $\hat{M}$ is locally compact, and $i_M$ is an isomorphism. Moreover, if $M$ is compact (resp. discrete) $\hat{M}$ is discrete (resp. compact).
\end{prop}

\begin{oss}
    If $M$ is an $A$-module, since $M$ is also an $\F_q$-vector space, we have the following natural isomorphisms of topological $A$-modules:
    \[\hat{M}\coloneqq\Hom^{cont}_\Z(M,\mathbb{S}^1)\cong\Hom^{cont}_{\F_q}(M,\Hom_{\F_q}(\F_q,\mathbb{S}^1))=\Hom^{cont}_{\F_q}(M,\hat\F_q).\]
\end{oss}

We fix an isomorphism $\F_q\cong\hat{\F}_q$ so that from now on, to ease notation, we can write $\hat{M}=\Hom_{\F_q}^{cont}(M,\F_q)$ for any $\F_q$-vector space $M$. Let's fix some additional notation.

\begin{Def}\label{complete tensor product}
Let $M$ and $N$ be topological $\F_q$-vector spaces with $N$ locally compact. We define the topological tensor product of $M$ and $N$, and denote by $M\hat\otimes N$, the space $\Hom_{\F_q}^{cont}(\hat{N},M)$ of continuous $\F_q$-linear homomorphisms from $\hat{N}$ to $M$.
\end{Def}

\begin{oss}
    The topological tensor product can be endowed with the compact open topology, but we will only need to use the definition of the underlying set.
\end{oss}

\begin{lemma}\label{duality tensor hom 1}
    For any pair of locally compact $A$-modules $M,N$, there is a natural isomorphism of $A\otimes A$-modules between $M\hat\otimes N$ and $N\hat\otimes M$.
\end{lemma}
\begin{proof}
    By Proposition \ref{Pontryagin duality}, the Pontryagin duality induces an antiequivalence of the category of locally compact $\F_q$-vector spaces with itself, hence we have the following natural bijections:
    \[\Hom_{\F_q}^{cont}(\hat N,M)\cong\Hom_{\F_q}^{cont}(\hat{M},\hat{\hat N})\cong\Hom_{\F_q}^{cont}(\hat M,N);\]
    the $A\otimes A$-linearity is a simple check.
\end{proof}

We introduce some other useful terminology.

\begin{oss}
    For any set $I$, the Pontryagin dual of $\F_q^{\oplus I}$ can be identified with $\F_q^I$. In particular, for any discrete $\F_q$-vector space $M$, an isomorphism $\F_q^{\oplus I}\cong M$, i.e. an $\F_q$-basis $(m_i)_{i\in I}$, induces an isomorphism of topological vector spaces between $\F_q^I=\widehat{\F_q^{\oplus{I}}}$ and $\hat{M}$.
\end{oss}
\begin{Def}
    If $M$ is a discrete $\F_q$-vector space with basis $(m_i)_{i\in I}$, for all $i\in I$ we denote by $m_i^*$ the image of $(\delta_{i,j})_{j\in I}\in \F_q^I$ via the isomorphism with $\hat{M}$, so that for all $j\in I$ $m_i^*(m_j)=\delta_{i,j}$. We call $(m_i^*)_{i\in I}$ the \textit{dual basis} of $\hat{M}$ relative to $(m_i)_{i\in I}$.
\end{Def}

\begin{oss}
    In the previous definition, a generic element $f\in\hat{M}$ corresponds to $(f(m_i))_i\in\F_q^I$. It's immediate to check that, for all $m\in M$, $f(m)=\sum_{i\in I}f(m_i)m_i^*(m)$, which is actually a finite sum, hence we are justified in the use the following formal notation: $f=\sum_{i\in I}f(m_i)m_i^*$. The existence and uniqueness of this expression for all $f\in\hat{M}$ explains the terminology "dual basis" for $(m_i^*)_i$.
\end{oss}

\subsection{Application to \texorpdfstring{$A$}{A}-modules}

The following (see \cite[Thm. 8]{Poonen}) is a fundamental result about the Pontryagin duality of $A$-modules.

\begin{teo}[Poonen]
    The computation of the residue at $\infty$ induces a perfect pairing between $\Omega\otimes_A\K$ and $\K$, which restricts to a perfect pairing between the discrete $A$-module $\Omega$ and the compact $A$-module $\faktor{\K}{A}$. In other words, $\widehat{\Omega\otimes_A\K}\cong\K$ and $\hat\Omega\cong\faktor{\K}{A}$.   
\end{teo}

\begin{oss}\label{oss dual}
    For any discrete projective $A$-module $\Lambda$ of finite rank $r$, we have the following natural isomorphisms of topological $A$-modules, where $\Lambda^*\coloneqq\Hom_A(\Lambda,A)$:
    \[\widehat{\Lambda^*\otimes_A\Omega}=\Hom_{\F_q}(\Lambda^*\otimes_A\Omega,\F_q)\cong\Hom_A(\Lambda^*,\Hom_{\F_q}(\Omega,\F_q))\cong\Lambda\otimes_A(\K/A).\]
    Retracing the isomorphisms, it's easy to check that the pairing $(\Lambda^*\otimes_A\Omega)\otimes(\Lambda\otimes_A\K/A)\to \F_q$ sends the element $(\lambda^*\otimes\omega)\otimes(\lambda\otimes b)$ to the pairing of $\lambda^*(\lambda)b\otimes\omega\in\faktor{\K}{A}\otimes_A\Omega$.
\end{oss}



We now show that in some cases the topological tensor product of two spaces is naturally isomorphic to a completion of their tensor product. This makes our notation agree with the usual notation $\C\hat\otimes A$ employed for the Tate algebra in works like \cite{Gazda}, \cite{Ferraro}, and others.

\begin{Def}\label{Def complete space}
    Let $C$ be a topological vector spaces which is the projective limit of a diagram of discrete $\F_q$-vector spaces $\{C_i\}_{i\in I}$: we call such a space a \textit{prodiscrete} $\F_q$-vector space; we call its \textit{associated filter} the collection $\U\coloneqq\{\ker(C\to C_i)\}_{i\in I}$, which is a neighborhood filter of $0$ comprised of open (and closed) subspaces of $C$. 
    
    For any discrete $F_q$-vector space $M$ and any prodiscrete $\F_q$-vector space $C$, we denote by $C\closure\otimes M$ the completion of $C\otimes M$ with respect to the neighborhood filter of $0$ given by $\{U\otimes M\}_{U\in\U}$.
\end{Def}

\begin{oss}\label{oss C_infty is prodiscrete}
    The open ball $B_r\subseteq\C$ of radius $r\in\R_{>0}$ is an $\F_q$-vector space, because the norm on $\C$ is nonarchimedean. Since $\C$ is complete, $\C$ is a prodiscrete $\F_q$-vector space, with associated filter $\{B_r\}_{r\in\R_{>0}}$.
\end{oss}


\begin{prop}\label{duality tensor hom 2}
    Let $C$ be a prodiscrete $\F_q$-vector space and $M$ be a discrete $\F_q$-vector space. There is a natural $\F_q$-linear bijection  $\Phi:C\closure\otimes M\to C\hat\otimes M$. Moreover, if $C$ and $M$ are $A$-modules, $\Phi$ is $A\otimes A$-linear.
\end{prop}
\begin{proof}
    Fix an $\F_q$-basis $(m_i)_{i\in I}$ of $M$ and let $\U$ be an associated filter of $C$. Any $x\in C\closure\otimes M$ can be expressed in a unique way as $\sum_{i\in I}x_i\otimes m_i$, where $x_i\in C$ for all $i\in I$, and for all $U\in \U$ the set $I_U\coloneqq\{i\in I|x_i\not\in U\}$ is finite. We define $\Phi(x):\hat{M}\to C$ as follows:
    \[\forall f\in\hat{M},\;\Phi(x)(f)\coloneqq\lim_{J\in\P^{fin}(I)}\sum_{i\in J} f(m_i)x_i.\]
    
    Since $C$ is complete with respect to the neighborhood filter $\U$, and for all $U\in\U$ the set $\{i\in I|f(m_i)x_i\not\in U\}\subseteq I_U$ is finite, the map $\Phi(x)$ is well defined.
    
    For all $U\in\U$, the set $\{f\in\hat{M}|f(m_i)=0\;\forall i\in I_U\}$ is a neighborhood of $0$ in $\hat{M}$, and is contained in $\Phi(x)^{-1}(U)$, hence $\Phi(x)$ is continuous. Since $\Phi(x)$ is also obviously $\F_q$-linear, $\Phi(x)\in C\hat\otimes M$ for all $x\in C\closure\otimes M$.
    
    The map $\Phi$ is manifestly $\F_q$-linear, and, if $C$ and $M$ are $A$-modules, $A\otimes A$-linear, so we just need to prove bijectivity.
    On one hand, if $\Phi(x)\equiv0$, we have $0=\Phi(x)(m_i^*)=x_i$ for all $i\in I$, hence $x=0$. On the other hand, if $g:\hat{M}\to C$ is a continuous function, for all $U\in\U$ the set $\{i\in I|g(m_i^*)\not\in U\}$ is finite because $\hat{M}$ is compact, hence $y\coloneqq\sum_i g(m_i^*)\otimes m_i$ is an element of $C\closure\otimes M$; since $\Phi(y)(m_i^*)=g(m_i^*)$ for all $i\in I$, $\Phi(y)=g$.
\end{proof}

\begin{oss}\label{oss identity series}
    If $M$ is a discrete $\F_q$-vector space, it is a colimit of finite subspaces, hence $\hat{M}$ is a profinite $\F_q$-vector space, and the bijection outlined in the previous proposition holds. In particular, if we fix a basis $(m_i)_{i\in I}$ of $M$, with $(m_i^*)$ dual basis of $\hat{M}$, the identity map in $\hat{M}\hat\otimes M$ sends $m_i^*$ to $m_i^*$ for all $i$, hence it can be expressed as the following infinite series: $\sum_{i\in I}m_i^*\otimes m_i$.
\end{oss}

\section{Universal Drinfeld eigenvector}\label{section special functions}

In this section, we will define the functor of \textit{Drinfeld eigenvectors} relative to an Anderson module $(E,\phi)$, which generalizes the concept of special functions and Gauss-Thakur sums (see Definition \ref{Def special functor}), and prove that under some conditions it is representable (see Theorem \ref{main result} and Theorem \ref{non uniformizable case}). As a corollary, we get a variant of the result \cite{Gazda}[Thm. 3.11], in which Gazda and Maurischat described the module of special functions for any Anderson module $(E,\phi)$.

\subsection{Anderson modules}

\begin{Def}
    An \textit{Anderson $A$-module} $(E,\phi)$ over $\C$ of dimension $d$ consists of a $\C$-group scheme $E\cong\mathbb{G}_{a,\C}^d$ and an $\F_q$-linear action $\phi$ of $A$ on $E$ such that $\Lie\phi_a-a:\Lie(E)\to\Lie(E)$ is nilpotent for all $a\in A$.
\end{Def}

Fix an Anderson $A$-module $(E,\phi)$. The following proposition sums up various basic results about $(E,\phi)$ (see \cite{Goss}[Thm. 5.9.6] and \cite{Goss}[Lemma 5.1.9] for proofs).

\begin{prop}
    There is an $\F_q$-linear function $\exp_E:\Lie(E)\to E(\C)$, called \textit{exponential}, such that $\exp_E\circ\Lie\phi_a=\phi_a\circ\exp_E$ for all $a\in A$; its kernel $\Lambda_E\subseteq\Lie(E)$ is an $A$-module of finite rank (with respect to the $A$-module structure induced by $\Lie\phi$ on $\Lie(E)$).

    Moreover, if we fix an isomorphism $E\cong\mathbb{G}_{a,\C}^d$, the exponential function can be identified with a series in $\mathbb{C}_\infty^{d\times d}[[\tau]]$ - where $\tau$ is the Frobenius endomorphism - whose leading term is the identity matrix.
\end{prop} 

\begin{Def}
    An Anderson $A$-module $(E,\phi)$ is said to be \textit{uniformizable} if $\exp_E$ is surjective.
\end{Def}

\begin{oss}
    Since $E$ and $\mathbb{G}_{a,\C}^d$ are isomorphic $\C$-group schemes, and the group of automorphisms of $\mathbb{G}_{a,\C}^d$ as a $\C$-group scheme is $\GL_{n,\C}$, we can identify the set $E(\C)$ with $\mathbb{G}_{a,\C}^d(\C)=\mathbb{C}_\infty^d$ up to an element of $\GL_{n,\C}(\C)$. In particular, $E(\C)$ has a natural structure of finite $\C$-vector space, hence, since $\C$ is a complete normed field, a natural structure of complete topological vector space over $\C$.

    Similarly, $\Lie(E)$ also has a natural structure of complete topological vector space over $\C$; moreover, by the inverse function theorem applied to $\exp_E$, we get that $\Lambda_E\subseteq\Lie(E)$ is a discrete subset. In light of this remark, and since for all $a\in A$ $\exp_E\circ\Lie\phi_a=\phi_a\circ\exp_E$, $\exp_E$ is a morphism of topological $A$-modules.
\end{oss}

The following is a known lemma, so we just give an outline of the proof.

\begin{lemma}\label{K-action on Lie(E)}
    The $A$-module structure of $\Lie(E)$ induced by $\Lie\phi$ can be extended to a structure of topological vector space over $\K$.
\end{lemma}
\begin{proof}
    Since the endomorphisms $(\Lie\phi_a)_{a\in A\setminus\{0\}}$ commute and are invertible, the ring homomorphism $\Psi:A\to \End_\C(\Lie(E))$ sending $a$ to $\Lie\phi_a$ can be extended uniquely to $K$, and we can fix a basis $\Lie(E)\cong\mathbb{C}_\infty^d$ in which, for all $c\in K$, $\Psi_c$ is a triangular matrix with $N_c\coloneqq c^{-1}\Psi_c-\mathds{1}_d$ nilpotent - precisely, $N_c^d=0$. We can endow any matrix $N\in\End_d(\C)$ with a sub-multiplicative norm $\|\cdot\|$; to extend continuously $\Psi$ to $\K$, it suffices to prove that the set $\{\|c^{-1}\Psi_c\|\}_{c\in K\setminus\{0\}}$ is bounded, so that $\|\Psi_c\|$ tends to $0$ as $\|c\|$ tends to $0$.
    
    We can fix a finite set $a_1,\cdots,a_n\in A$ whose finite products generate $A$ as an $\F_q$-vector space. If we call $M\coloneqq\max\{1,\|N_{a_1}\|,\dots,\|N_{a_n}\|\}$, it's easy to prove that, for all $b\in A$, $\|N_b\|\leq M^{nd}$.
    For all $c\in K^\times$, if we write $c=ab^{-1}$ with $a,b\in A\setminus\{0\}$, we have:
    \[\|c^{-1}\Psi_c\|=\|a^{-1}\Psi_a(b^{-1}\Psi_b)^{-1}\|=\left\|(\mathds{1}_n+N_a)\left(\sum_{i=0}^{d-1}N_b^i\right)\right\|\leq M^{nd^2},\]
    which proves the thesis.
\end{proof}

\subsection{Functor of Drinfeld eigenvectors}

\begin{Def}\label{Def special functor}
    For any discrete $A$-module $M$, its set of \textit{Drinfeld eigenvectors} is defined as the $A$-module of continuous $A$-linear homomorphisms $\Hom_A^{cont}(\hat{M},E(\C))\subseteq E(\C)\hat\otimes M$, where the $A$-module structure on $E(\C)$ is the one induced by $\phi$.
    We denote by $\Sf_\phi:A-\Mod\to A-\Mod$ the natural functor that extends this map.
\end{Def}

\begin{oss}
    By Proposition \ref{duality tensor hom 2}, our definition of $E(\C)\hat\otimes A$ coincides with the one given in \cite{Gazda}. The $A$-module $\Sf_\phi(A)$ is the subset of $E(\C)\hat\otimes A$ comprised of the elements on which the left and right $A$-actions coincide, hence $\Sf_\phi(A)$ is the module of special functions as defined in \cite{Gazda}.
\end{oss}

With the following theorem, we describe the functor $\Sf_\phi$.

\begin{teo}\label{main result}
    Suppose that $E$ is uniformizable. The functor $\Sf_\phi$ is naturally isomorphic to $\Hom_A(\Lambda_E^*\otimes_A\Omega,\_)$, and the universal object in $\Hom_A^{cont}\left(\faktor{\K\Lambda_E}{\Lambda_E},E(\C)\right)$ sends the projection of $c\in\K\Lambda_E$ to $\exp_E(c)$.
\end{teo}

\begin{proof}
    Since $E$ is uniformizable, $E(\C)$ is isomorphic to $\faktor{\Lie(E)}{\Lambda_E}$ as a topological $A$-module. Endow $\Lie(E)$ with the structure of topological $\K$-vector space described in Lemma \ref{K-action on Lie(E)}; the finite $\K$-vector subspace $\K\Lambda_E\subseteq\Lie(E)$ admits a topological complement $V$, which induces an isomorphism of topological $A$-modules $E(\C)\cong\faktor{\K\Lambda_E}{\Lambda_E}\bigoplus V$. 
    
    For any discrete $A$-module $M$, for any $\omega\in \Sf_\phi(M)$, its projection $\overline{\omega}$ onto $V\hat\otimes M$ belongs to $\Hom_A^{cont}(\hat{M},V)$. Since $\hat{M}$ is compact, the image of $\overline{\omega}$ must be a compact sub-$A$-module of $V$; on the other hand, since $V$ is a topological $\K$-vector space, for any $v\in V\setminus\{0\}$ the set $A\cdot v$ is unbounded, so the only compact sub-$A$-module of $V$ is $\{0\}$. We deduce that, for any $\omega\in \Sf_\phi(M)$, $\overline\omega=0$, therefore we have the following natural isomorphisms:
    \[\Sf_\phi(M)\cong\Hom_A^{cont}\left(\hat{M},\faktor{\K\Lambda_E}{\Lambda_E}\oplus V\right)=\Hom_A^{cont}\left(\hat{M},\faktor{\K\Lambda_E}{\Lambda_E}\right);\]
    by Lemma \ref{duality tensor hom 1}, the right hand side is naturally isomorphic to $\Hom_A(\Lambda_E^*\otimes_A\Omega,M)$.
    
    Setting $M\coloneqq\Lambda_E^*\otimes_A\Omega$, and following the identity along the chain of isomorphisms, we deduce that the universal object in $\omega_\phi\in\Hom_A^{cont}\left(\faktor{\K\Lambda_E}{\Lambda_E},E(\C)\right)$ is the continuous $A$-linear map sending the projection of $c\in\K\Lambda_E$ to $\exp_E(c)$.
\end{proof}

For the sake of completeness, let's prove a statement which does not assume uniformizability.

\begin{teo}\label{non uniformizable case}
    If we restrict the functor $\Sf_\phi$ to the subcategory of torsionless $A$-modules, it is naturally isomorphic to $\Hom_A(\Lambda_E^*\otimes_A\Omega,\_)$.
\end{teo}
\begin{proof}
    The map $\exp_E$ is open because its Jacobian at all points is the identity; call $C$ its image. Since $C$ is an open $\F_q$-vector space, the quotient $\faktor{E(\C)}{C}$ is a discrete $A$-module.
    
    A discrete $A$-module $M$ is torsionless if and only if it has no nontrivial compact submodules; in this case, $\hat{M}$ is a compact $A$-module with no nontrivial discrete quotients. In particular, for any function $f\in \Sf_\phi(M)=\Hom_A^{cont}(\hat{M},E(\C))$, its projection onto $\faktor{E(\C)}{C}$ is trivial, hence the image of $f$ must be contained in $C$. The rest of the proof is the same as Theorem \ref{main result} up to substituting $E(\C)$ with $C$.
\end{proof}
As a Corollary, we can describe the isomorphism class of the module of special functions $\Sf_\phi(A)$ for any Anderson module $E$, as already done by Gazda and Maurischat (\cite{Gazda}[Thm. 3.11]).
\begin{cor}\label{cor Gazda}
    The following isomorphism of $A$-modules holds:
    \[\Sf_\phi(A)=\{\omega\in E(\C)\hat\otimes A|\phi_a(\omega)=(1\otimes a)\omega\;\forall a\in A\}\cong\Omega^*\otimes_A\Lambda_E.\]
\end{cor}

\begin{oss}\label{oss universal special function}
    Fix an $\F_q$-basis $(\mu_i)_i$ of the discrete $A$-module $\Hom_A(\Lambda_E,\Omega)$, with $(\mu_i^*)_i$ dual basis of $\faktor{\K\Lambda_E}{\Lambda_E}$. By Remark \ref{oss identity series} we can express the universal object in the following alternative way as an element of $E(\C)\hat\otimes\Hom_A(\Lambda_E,\Omega)$:
    \[\omega_\phi=\exp_E\left(\sum_i\mu_i^*\otimes\mu_i\right)=\sum_i\exp_E(\mu_i^*)\otimes\mu_i,\]
    where by slight abuse of notation we considered $\exp_E$ as a map from $\faktor{\K\Lambda_E}{\Lambda_E}$ to $E(\C)$.
\end{oss}







\section{Universal dual Drinfeld eigenvector}\label{section zeta functions}

In this section, we focus on a Drinfeld module $\phi$ of arbitrary rank. We want to adapt the approach of the previous section to dual Drinfeld modules; to do so, we rely on a useful result by Poonen (\cite{Poonen}[Thm. 10]).

\subsection{Poonen duality}

Denote by $\exp_\phi\in\C[[\tau]]$ the exponential relative to $\phi$, where $\tau$ is the Frobenius endomorphism of $\C$. We denote by $\Lambda_\phi$ the kernel of the exponential, which we know is a finitely generated discrete sub-$A$-module of $\C$.
\begin{Def}
    Let $\C[[\tau,\tau^{-1}]]$ denote the $\C[\tau,\tau^{-1}]$-bimodule of unbounded formal series in $\tau$.
    For any formal series $s=\sum_i s_i\tau^i\in\C[[\tau,\tau^{-1}]]$, we define its dual as $s^*\coloneqq\sum_i s_i^{q^{-i}}\tau^{-i}\in\C[[\tau,\tau^{-1}]]$.
\end{Def}

\begin{oss}
    It's easy to show that, since $\exp_\phi$ has an infinite radius of convergence, the dual exponential $\exp_\phi^*\in\C[[\tau^{-1}]]$ also converges everywhere on $\C$.
\end{oss}

We follow a construction due to Poonen, who proved a duality result of central importance to this section (\cite{Poonen}[Thm. 10]).
\begin{lemma}\label{lemma g_beta}
    For all $\beta\in\ker(\exp_\phi^*)\setminus\{0\}$, there is a unique element $g_\beta\in\C[[\tau]]$ such that $(1-\tau)g_\beta=\beta\exp_\phi$. Moreover, $g_\beta$ has infinite radius of convergence.
\end{lemma}
\begin{proof}
    Let's set $h:=\sum_{i\geq0}\tau^i$; since $h(1-\tau)=1$, the defining property of $g_\beta$ is equivalent to the identity $g_\beta=h\beta\exp_\phi$: this proves both existence and uniqueness. If we call $e_i$ the $i$-th coefficient of $\exp_\phi$ and $c_i$ the $i$-th coefficient of $g_\beta$, from the identity $g_\beta=h\beta\exp_\phi$ we get the following:
    \[c_k=\sum_{i=0}^k\beta^{q^i}e_{k-i}^{q^i}\;\forall k\in\Z_{\geq0}\Longrightarrow\lim_k c_k^\frac{1}{q^k}=\lim_k\sum_{i=0}^k e_i^{q^{-i}}\beta^{q^{-i}}=\exp_\phi^*(\beta)=0,\]
    hence the radius of convergence of $g_\beta$ is infinite.
\end{proof}
\begin{oss}
    Since $\ker((1-\tau)g_\beta)=\ker(\beta\exp_\phi)=\Lambda_\phi$, $g_\beta|_{\Lambda_\phi}$ has image in $\F_q$.
\end{oss}

By convention, we set $g_0=0$.

\begin{teo}[Poonen]\label{Poonen}
The function $\ker(\exp_\phi^*)\to\hat\Lambda_\phi$ sending $\beta$ to $g_\beta|_{\Lambda_\phi}$ is an $A$-linear homeomorphism, where $A$ acts via $\phi^*$ on the left hand side.
\end{teo}

The following proposition, which is proven in Section \ref{section technical lemmas}, can be viewed as an explicit algebraic formula for the inverse of the isomorphism in Theorem \ref{Poonen}.

\begin{prop}\label{identity 1}
    For all $\beta\in\ker(\exp_\phi^*)\setminus\{0\}$, the following identity holds in $\C$:
    \[\beta=-\sum_{\lambda\in\Lambda_\phi\setminus\{0\}}\frac{g_\beta(\lambda)}{\lambda}.\]
\end{prop}

\subsection{Functor of dual Drinfeld eigenvectors}

In \cite{Ferraro}, the author proved that, if $\phi$ is a normalized Drinfeld module of rank $1$ and $\Lambda_\phi=\tilde{\pi}I$ for some $\tilde{\pi}\in\C$ and some ideal $I<A$, the following result holds (see \cite{Ferraro}[Prop. 7.8, Prop. 7.21]).
\begin{prop}
    Let $\zeta_I\coloneqq\sum_{a\in I\setminus\{0\}}a^{-1}\otimes a\in\C\hat\otimes A$. For all $a\in A\setminus\{0\}$:
    \[\phi^*_a\left((\tilde{\pi}^{-1}\otimes1)\zeta_I\right)=(\tilde{\pi}^{-1}\otimes a)\zeta_I.\]
\end{prop}

In this section, we prove a generalization to Drinfeld modules of arbitrary rank. 

\begin{Def}
Denote by $\mathbb{C}_\infty^{\phi^*}$ the topological $\F_q$-vector space $\C$ endowed with the $A$-module structure induced by $\phi^*$. For any discrete $A$-module $M$, its set of \textit{dual Drinfeld eigenvectors} is defined as the $A$-module of continuous $A$-linear homomorphisms $\Hom_A^{cont}(\hat{M},\mathbb{C}_\infty^{\phi^*})\subseteq \mathbb{C}_\infty^{\phi^*}\hat\otimes M$.
We denote by $\Sf_{\phi^*}:A-\Mod\to A-\Mod$ the natural functor that extends this map.
\end{Def}

\begin{teo}\label{zeta function functor}
    The functor $\Sf_{\phi^*}$ is naturally isomorphic to $\Hom_A(\Lambda_\phi,\_)$, and the universal object in $\mathbb{C}_\infty^{\phi^*}\hat\otimes\Lambda_\phi$ is $-\sum_{\lambda\in\Lambda_\phi\setminus\{0\}}\lambda^{-1}\otimes\lambda$.
\end{teo}

\begin{proof}
    The map $\exp_\phi^*:\mathbb{C}_\infty^{\phi^*}\to\C$ is a continuous $A$-linear morphism; for any $A$-module $M$, it induces a morphism $\Sf_{\phi^*}(M)\to\Hom^{cont}_A(\hat{M},\C)$. Fix some $\zeta\in \Sf_{\phi^*}(M)$, with image $\overline{\zeta}$: since $\hat{M}$ is compact, the image of $\overline{\zeta}$ must be a compact sub-$A$-module of $\C$, but for any $c\in\C\setminus\{0\}$ the set $A\cdot c$ is unbounded, hence $\overline{\zeta}\equiv0$. We deduce that the image of $\zeta:\hat{M}\to\mathbb{C}_\infty^{\phi^*}$ must be contained in $\ker\exp_\phi^*$, which by Theorem \ref{Poonen} is isomorphic as a topological $A$-module to $\hat\Lambda_\phi$; we have the following natural isomorphisms:
    \[\Sf_{\phi^*}(M)=\Hom_A^{cont}(\hat{M},\ker\exp_\phi^*)\cong\Hom_A^{cont}\left(\widehat{\ker\exp_\phi^*},M\right)\cong\Hom_A(\Lambda_\phi,M),\]
    where we used Lemma \ref{duality tensor hom 1} for the second isomorphism.

    The universal object $\zeta_\phi\in\mathbb{C}_\infty^{\phi^*}\hat\otimes\Lambda_\phi$ is given by the natural morphism $\psi:\hat\Lambda_\phi\cong\ker{\exp_\phi^*}\subseteq\mathbb{C}_\infty^{\phi^*}$ of Theorem \ref{Poonen}, which by Proposition \ref{identity 1} sends $g\in\hat\Lambda_\phi$ to $-\sum_{\lambda\in\Lambda_\phi\setminus\{0\}}\frac{g(\lambda)}{\lambda}$.

    If we fix an $\F_q$-basis $(\lambda_i)_i$ of $\Lambda_\phi$, with $(\lambda_i^*)_i$ dual basis of $\hat\Lambda_\phi$, by Remark \ref{oss identity series} we can write $\zeta_\phi=\sum_i\psi(\lambda_i^*)\otimes\lambda_i$, hence:
    \[\zeta_\phi
    =\sum_i\left(-\sum_{\lambda\in\Lambda_\phi\setminus\{0\}}\frac{\lambda_i^*(\lambda)}{\lambda}\right)\otimes\lambda_i=-\sum_{\lambda\in\Lambda_\phi\setminus\{0\},i}\lambda^{-1}\otimes \lambda_i^*(\lambda)\lambda_i=-\sum_{\lambda\in\Lambda_\phi\setminus\{0\}}\lambda^{-1}\otimes\lambda.\]
\end{proof}

\begin{cor}
   For all discrete $A$-modules $M$, $\Sf_{\phi^*}(M)$ is isomorphic to $\Hom_A(\Lambda_\phi,M)$ as an $A\otimes A$-module. In particular, for any $M$ we have the following equality between subsets of $\C\hat\otimes M$:
   \[\Sf_{\phi^*}(M)=\left\{\sum_{\lambda\in\Lambda_\phi\setminus\{0\}}\lambda^{-1}\otimes l(\lambda)\bigg|l\in\Hom_A(\Lambda_\phi,M)\right\}.\]
\end{cor}


\section{Pairing between Drinfeld eigenvectors and dual Drinfeld eigenvectors}\label{section pairing}

Let's fix a Drinfeld module $\phi$ with exponential $\exp_\phi=\sum_{i\geq0}e_i\tau^i$ and suppose that $\Lambda_\phi:=\ker\exp_\phi\subseteq\C$ has rank $r$ as an $A$-module; let $\log_\phi=\sum_{i\geq0}l_i\tau^i$ be the inverse of $\exp_\phi$ as an element of $\C[[\tau]]$.

\subsection{Definition of the dot product}
\begin{Def}\label{def main result}
    We define the \textit{universal Drinfeld eigenvector} $\omega_\phi\in\C\hat\otimes\Hom_A(\Lambda_\phi,\Omega)$ and the \textit{universal dual Drinfeld eigenvector} $\zeta_\phi\in\C\hat\otimes\Lambda_\phi$ as the universal objects of the functors $\Sf_\phi$ and $\Sf_{\phi^*}$, respectively.
\end{Def}

The following rationality result (a weak version of \cite{Ferraro}[Thm. 6.3]) links the two objects in the rank 1 case.

\begin{teo}[Ferraro]\label{Ferraro}
Suppose that $\phi$ is a normalized Drinfeld module of rank $1$. The product of an element in $\Sf_{\phi^*}(A)$ and an element in $\Sf_{\phi}(A)$ is a rational function over $X_\C$.
\end{teo}

To generalize this statement to Drinfeld modules of arbitrary rank, we need a proper way of "multiplying" $\zeta_\phi$ and $\omega_\phi$, established in the following lemma.

\begin{lemma}\label{ev}
The following $\C\otimes A$-linear pairing is well defined:
\[\begin{tikzcd}[column sep=scriptsize, row sep=small]
&\C\hat\otimes\Lambda_\phi\arrow[r,phantom,"\otimes"] & \C\hat\otimes(\Lambda_\phi^*\otimes_A\Omega)\arrow[r]&\C\hat\otimes\Omega\\
&\sum_i c_i\otimes\lambda_i\arrow[r,phantom,"\otimes"] 
&\sum_j d_j\otimes(\lambda^*_j\otimes\omega_j)\arrow[r,phantom,"\mapsto"] 
&\sum_{i,j}(c_i d_j)\otimes(\lambda^*_j(\lambda_i)\omega_j)\\
&&f\arrow[u,phantom,sloped,"\coloneqq"]&g\arrow[u,phantom,sloped,"\coloneqq"]
\end{tikzcd}\]
Moreover, considering $g$ and $f$ as continuous functions respectively from $\faktor{\K}{A}$ and $\Lambda_\phi\otimes_A\faktor{\K}{A}$ to $\C$, for all $b\in\faktor{\K}{A}$ we have:
\[g(b)=\sum_i c_i f(\lambda_i\otimes b).\]
\end{lemma}
\begin{proof}
The morphism is well defined because for all $\varepsilon>0$ there are finitely many pairs of indices $(i,j)$ such that $\|c_id_j\|>\varepsilon$; the $\C\otimes A$-linearity is also obvious from the definition. Call $\res:\Omega\otimes\K/A\to\F_q$ and $\res_{\Lambda_\phi}:(\Lambda_\phi^*\otimes_A\Omega)\otimes\left(\Lambda_\phi\otimes_A\faktor{\K}{A}\right)\to\F_q$ the two perfect pairings. By Remark \ref{oss dual} we have:
\[g(b)=\sum_{i,j}c_id_j\res(\lambda^*_j(\lambda_i)\omega_j,b)=\sum_i c_i\sum_j d_j\res_\Lambda(\lambda^*_j\otimes\omega_j,\lambda_i\otimes b)=\sum_i c_i f(\lambda_i\otimes b).\tag*{\qedhere}\]
\end{proof}

\subsection{Rationality of the dot products \texorpdfstring{$\zeta_\phi\cdot\omega_\phi^{(k)}$}{ζ · ω}}

The pairing defined in Lemma \ref{ev} will be denoted by a dot product. For any element $h\in\C\hat\otimes\Omega=\Hom^{cont}_{\F_q}\left(\faktor{\K}{A},\C\right)$ and for any $b\in\K$ with projection $\closure{b}\in\faktor{\K}{A}$, to simplify notation we will write $h(b)$ to denote $h(\closure{b})$. We can now state the generalization of Theorem \ref{Ferraro}.

\begin{teo}\label{Teo rationality}
For any Drinfeld module $\phi$, for all integers $k$, the dot product $\zeta_\phi\cdot\omega_\phi^{(k)}$ in $\C\hat\otimes\Omega$ is a rational differential form over the base-changed curve $X_{\C}$. Moreover, for all positive integers $k$, $\zeta_\phi\cdot\omega_\phi^{(k)}\in \C\otimes\Omega$.
\end{teo}

We assume the following result, which is proven in Section \ref{section technical lemmas}.
\begin{prop}\label{identity 2}
    For all integers $k$, for all $c\in\K\setminus\{0\}$ with $\|c\|\leq q^{\frac{k-1}{r}}$, the following identity holds in $\C$:
    \[\sum_{\lambda\in\Lambda_\phi\setminus\{0\}}\frac{\exp(c\lambda)}{\lambda^{q^k}}=-\sum_{j=0}^k e_jl_{k-j}^{q^j}c^{q^j},\]
    where by convention the summation on the right hand side is $0$ if $k<0$.
\end{prop}

\begin{proof}
    As an element of $\Hom_{\F_q}^{cont}\left(\faktor{\K\Lambda_\phi}{\Lambda_\phi},\C\right)$, $\omega_\phi$ sends the projection of any $c\in\K\Lambda_\phi$ to $\exp_\phi(c)$. By Lemma \ref{ev}, since $\zeta_\phi=-\sum_{\lambda\in\Lambda_\phi\setminus\{0\}}\lambda^{-1}\otimes\lambda$, for all $b\in\K$ and for all integers $k$ we have:
    \[\zeta_\phi\cdot\omega_\phi^{(k)}(b)=-\sum_{\lambda\in\Lambda_\phi\setminus\{0\}}\frac{\exp(b\lambda)^{q^k}}{\lambda}=\left(-\sum_{\lambda\in\Lambda_\phi\setminus\{0\}}\frac{\exp(b\lambda)}{\lambda^{q^{-k}}}\right)^{q^k}.\]
    By Proposition \ref{identity 2}, for all positive integers $k$, if $b\in\K$ has norm at most $q^{-\frac{k+1}{r}}$, $\zeta_\phi\cdot\omega_\phi^{(k)}(b)=0$. Let's denote by $C\subseteq\faktor{\K}{A}$ the subspace generated by the projections of elements in $\K$ with norm at most $q^h$, and denote by $Q$ the quotient. Since $Q$ is a finite $\F_q$-vector space, we get the following:
    \[\Hom_{\F_q}^{cont}\left(\faktor{\K}{A},\C\right)\supseteq\Hom_{\F_q}^{cont}(Q,\C)=\Hom_{\F_q}(Q,\C)=\C\otimes\hat{Q}.\]
    Since $\zeta_\phi\cdot\omega_\phi^{(k)}$ restricted to $C$ is identically $0$, it's contained in $\C\otimes\hat{Q}$, therefore it can be expressed as a finite sum:
    \[\zeta_\phi\cdot\omega_\phi^{(k)}=\sum_i c_i\otimes\mu_i\in\C\otimes\hat{Q}\subseteq\C\otimes\widehat{\faktor{\K}{A}}=\C\otimes\Omega.\]

    To prove the theorem for all integers $k$ we proceed by induction. Suppose that the result holds for all integers bigger than $k$, and fix some $a\in A\setminus\F_q$. From the definition of special functions we have:
    \begin{align*}
        &(1\otimes a-a\otimes1)\omega_\phi=\sum_{i=1}^{r\deg(a)}(\phi_a)_i\omega_\phi^{(i)}\\
        \Longrightarrow&\zeta_\phi\cdot\omega_\phi^{(k)}=\frac{1}{1\otimes a-a^{q^k}\otimes 1}\sum_{i=1}^{r\deg(a)}(\phi_a)_i^{q^k}\zeta_\phi\cdot\omega_\phi^{(k+i)},
    \end{align*}
    hence $\zeta_\phi\cdot\omega_\phi^{(k)}$ is a rational differential form over $X_\C$.
\end{proof}

\begin{oss}\label{oss computation}
    From the previous proof we deduce that, if we can compute the dot product $\zeta_\phi\cdot\omega_\phi^{(k)}$ for $r\deg(a)$ consecutive integers $k$, then we can compute it for any value of $k$.
\end{oss}

\subsection{Computation of the dot products \texorpdfstring{$\zeta_\phi\cdot\omega_\phi^{(k)}$  for $k\ll0$}{ω · ζ}}

We can expand on the previous theorem. In fact, we are able to describe explicitly the differential form $\zeta_\phi^{(k)}\cdot\omega_\phi$ for $k$ large enough by using once again Proposition \ref{identity 2}.

\begin{teo}\label{Teo pairing for large k}
    For all $b\in\K$ denote by $s(b)\in\K$ an element of smallest norm such that $b-s(b)\in A$. For all integers $k>r(2g-1)$, we have the following identity for all $b\in\K$:
    \[\zeta_\phi^{(k)}\cdot\omega_\phi(b)=\sum_{j=0}^k e_jl_{k-j}^{q^j}s(b)^{q^j}.\]
\end{teo}
\begin{proof}
    Fix any $b\in\K$; there is no nonzero element $a\in A$ with the same norm as $s(b)$, otherwise there would be some $\gamma\in\F_q^*$ such that $\|s(b)-\gamma a\|<\|s(b)\|$, contradicting the minimality condition on $s(b)$. By Riemann-Roch, for all $d\geq2g$ there is an element $a\in A$ of degree $d$, hence $\|s(b)\|\leq q^{2g-1}$. Since $\frac{k-1}{r}\geq2g-1$, by Proposition \ref{identity 2} we have:
    \[\zeta_\phi^{(k)}\cdot\omega_\phi(b)=\zeta_\phi^{(k)}\cdot\omega_\phi(s(b))=-\sum_{\lambda\in\Lambda_\phi\setminus\{0\}}\frac{\exp_\phi(s(b)\lambda)}{\lambda^{q^k}}=\sum_{j=0}^k e_jl_{k-j}^{q^j}s(b)^{q^j}.\tag*{\qedhere}\]
\end{proof}
\begin{oss}\label{oss pairing for large k}
    Equivalently, for all integers $k>r(2g-1)$ and for all $b\in\K$:
    \[\zeta_\phi\cdot\omega_\phi^{(-k)}(b)=\left(\sum_{j=0}^k e_jl_{k-j}^{q^j}s(b)^{q^j}\right)^{q^{-k}}.\]
    In principle, we can use this result to compute $\zeta_\phi\cdot\omega_\phi^{(i)}(b)$ for all $i$ and all $b$, in the same way we proved rationality in Theorem \ref{Teo rationality}, as we observed in Remark \ref{oss computation}.
\end{oss}

\subsection{The generating series of the dot products \texorpdfstring{$\zeta_\phi\cdot\omega_\phi^{(k)}$}{ζ · ω}}

Using Theorem \ref{Teo pairing for large k} and Remark \ref{oss pairing for large k}, we can in principle compute the dot product $\zeta_\phi\cdot\omega_\phi^{(k)}$ for any $k\geq-r(2g-1)$, but since the sketched algorithm is recursive, it's necessary to compute all the intermediate dot products $\zeta_\phi\cdot\omega_\phi^{(i)}$ for $-r(2g-1)\leq i\leq k$.

The objective of this subsection is to streamline this computation by studying the generating series $\sum_{k\in\Z}\zeta_\phi\cdot\omega_\phi^{(k)}\tau^k$.

\begin{Def}
Denote by $\C\langle\tau\rangle$ the subset of $\C[[\tau]][\tau^{-1}]$ given by the series with a nonzero radius of convergence on $\C$.
\end{Def}
\begin{oss}
    The set $\C\langle\tau\rangle$ is closed under addition and composition, hence it is a subring of $\C[[\tau]][\tau^{-1}]$.
\end{oss}
\begin{oss}\label{oss C<t>}
    Since the radius of convergence of $h=\sum_i h_i\tau^i\in\C[[\tau]][\tau^{-1}]$ is the inverse of $\limsup_{i\to\infty}\|h_i\|^{q^{-i}}$, we have that $h\in\C\langle\tau\rangle$ if and only if $\limsup_{i\to\infty}\|h_i\|^{q^{-i}}<\infty$.
\end{oss}

\begin{lemma}
     Every nonzero element $h\in\C[\tau,\tau^{-1}]$ admits a (unique) bilateral inverse in $\C\langle\tau\rangle$.    
\end{lemma}
\begin{proof}
    Since $\tau:\C\to\C$ is an isomorphism, up to multiplication we can assume $h=\sum_{i\geq0}h_i\tau^i$ with $h_0=1$. If we call $h_+\coloneqq\sum_{i\geq1}h_i\tau^i$, the series $\sum_{i\geq0}h_+^i$ is a well defined bilateral inverse of $h$ in $\C[[\tau]]$. Since $h$ has finitely many nonzero coefficients, it's easy to see that there is some $R\in\R_{>0}$ and some positive real constant $C<1$ such that, for all $x\in\C$ with norm less than $R$, $\|h_ix^{q^i}\|\leq C\|x\|$ for all $i\geq1$. In particular, for all $x\in\C$ with norm less than $R$, each of the finitely many summands in the expansion of $h_+^i(x)$ has norm at most $C^i\|x\|$, hence the series $\sum_{i\geq0}h_+^i(x)$ converges. We deduce that the series $\sum_{i\geq0}h_+^i$ has a nonzero radius of convergence, hence it belongs to $\C\langle\tau\rangle$.
\end{proof}


\begin{Def}
    For all $c\in\K$ we define $\Phi_c\in\C\langle\tau\rangle$ as $\exp_\phi\circ c\circ\log_\phi$.
\end{Def}

\begin{oss}
    For all $a\in A$, $\Phi_a=\phi_a$. The map $\Phi:\K\to\C\langle\tau\rangle$ sending $c$ to $\Phi_c$ is the unique ring homomorphism which extends $\phi:A\to\C\langle\tau\rangle$ such that each coefficient is a continuous function.
\end{oss}

\begin{prop} 
    Let $\mu:\K\to\C[[\tau,\tau^{-1}]]$ be a function with the following properties:
    \begin{enumerate}
        \item $\forall k\in\Z$ the function sending $c$ to $(\mu_c)_k$ is $\F_q$-linear and continuous;
        \item $\forall a\in A,c\in\K$, $\mu_{ac}=\mu_c\phi_a$;
        \item $\forall a\in A$, $\mu_a=0$;
        \item $\forall R\in\R\exists n_0\in\Z$ such that for all $n\geq n_0$, for all $c\in\K$ with $\|c\|\leq R$, $(\mu_c)_n=(\Phi_c)_n$.
    \end{enumerate}
    Then, $\mu$ is uniquely determined; in particular, for any $c\in\K$, we have:
    \[\mu_c=\sum_{k\in\mathbb\Z} \left(\zeta_\phi^{(k)}\cdot\omega_\phi\right)(c)\tau^k.\]%, and $\forall a\in A,\forall c\in\K$ $\phi_a\mu_c=\mu_c\phi_a$
\end{prop}
\begin{proof}
    To prove uniqueness, let's take two such functions $\mu$ and $\mu'$, and define $\lambda:=\mu-\mu'$. For each element $c\in\K$ let $s(c)$ be an element of least norm such that $c-s(c)\in A$. As we already said in the proof of Theorem \ref{Teo pairing for large k}, $\|s(c)\|<q^{2g}$ for all $c\in\K$; using properties (i),(iii), and (iv) with $R=q^{2g}$, we deduce that there is some integer $n_0$ such that, for all $n\geq n_0$, for all $c\in\K$:
    \[(\lambda_c)_n=(\lambda_{s(c)})_n+(\lambda_{c-s(c)})_n=(\lambda_{s(c)})_n=(\mu_{s(c)})_n-(\mu'_{s(c)})_n=(\Phi_{s(c)})_n-(\Phi_{s(c)})_n=0.\]
    If by contradiction $\lambda\not\equiv0$, there is an element $c\in\K$ such that $\lambda_c$ has the highest degree; by property (ii), for any $a\in A\setminus\F_q$, $\lambda_{ac}=\lambda_c\phi_a$, which has a greater degree than $\lambda_c$, reaching a contradiction.

    Let's check that $\mu_c\coloneqq\sum_{k\in\mathbb\Z} \left(\zeta_\phi^{(k)}\cdot\omega_\phi\right)(c)\tau^k$ satisfies all conditions. The properties (i) and (iii) are obvious. For property (iv), note that for all $c\in\K$
    \[(\Phi_c)_k=(\exp_\phi\circ c\circ\log_\phi)_k=\sum_{i+j=k}e_ic^{q^i}l_j^{q^i},\]
    which is equal to $\left(\zeta_\phi^{(k)}\cdot\omega_\phi\right)(c)$ for all $k\geq r\cdot\log_q(\|c\|)+1$ by Proposition \ref{identity 2}. Finally, for property (ii), since $\zeta_\phi$ is a Drinfeld eigenvector, for all $a\in A$ we have:
    \[(1\otimes a)\zeta_\phi=\sum_{i=0}^{r\deg(a)}(\phi_a)_i^{q^{-i}}\zeta_\phi^{(-i)}\Longrightarrow\forall k\in\Z:\;(1\otimes a)\zeta_\phi^{(k)}=\sum_{i=0}^{r\deg(a)}(\phi_a)_i^{q^{k-i}}\zeta_\phi^{(k-i)}.\]
    We deduce that, for all $c\in\K$:
    \begin{align*}
        \mu_c\phi_a&= \left(\sum_{k\in\Z}\left(\zeta_\phi^{(k)}\cdot\omega_\phi\right)(c)\tau^k\right)\left(\sum_{i=0}^{r\deg(a)}(\phi_a)_i\tau^i\right)\\
        &=\sum_{k\in\Z}\left(\sum_{i=0}^{r\deg(a)}(\phi_a)_i^{q^{k-i}}\left(\zeta_\phi^{(k-i)}\cdot\omega_\phi\right)(c)\right)\tau^k\\
        &=\sum_{k\in\Z}\left((1\otimes a)\zeta_\phi^{(k)}\cdot\omega_\phi\right)(c)\tau^k=\sum_{k\in\Z}\left(\zeta_\phi^{(k)}\cdot\omega_\phi\right)(ac)\tau^k=\mu_{ac}.\tag*{\qedhere}
    \end{align*}
\end{proof}
\begin{oss}
    As a function, $\mu_c$ never converges if $c\not\in A$. For example, for all $c\in K\setminus A$ we can choose $a\in A$ so that $ca\in A$, and we get that $\mu_c\phi_a=\mu_{ac}=0$: since $\mu_c\neq0$, this implies that its radius of convergence is $0$.
\end{oss}

\begin{Def}
    For all $c\in\K$ we define $\hat{\Phi}_c\coloneqq(\Phi_c-\mu_c)^*\in\C[[\tau,\tau^{-1}]]$.
\end{Def}

\begin{prop}
    For all $c\in\K$, the series $\hat{\Phi}_c$ has a nonzero radius of convergence. Moreover, the map $\hat{\Phi}:\K\to\C\langle\tau\rangle$ sending $c$ to $\hat{\Phi}_c$ is the unique ring homomorphism which extends $\phi^*:A\to\C\langle\tau\rangle$ such that each coefficient is a continuous function.
\end{prop}
\begin{proof}
    Uniqueness is obvious: by multiplicativity there is at most one way to extend $\phi^*$ to the fraction field $K$, and by continuity there is at most one way to extend it to the completion $\K$. By definition of $\Phi$ and $\mu$, each coefficient of $\hat{\Phi}_c$ is a continuous function of $c$. 
    
    For all $c\in\K$, by Proposition \ref{identity 2} we have $(\hat{\Phi}_c)_k=\left((\Phi_c-\mu_c)^*\right)_k=0$ for $k\ll0$, hence $\hat{\Phi}_c\in\C[[\tau]][\tau^{-1}]$.
    On the other hand, for $k\gg0$:
    \[((\hat{\Phi}_c)_k)^{q^{-k}}=((-\mu_c^*)_k)^{q^{-k}}=-(\mu_c)_{-k}=-\left(\zeta_\phi^{(-k)}\cdot\omega_\phi\right)(c)=\sum_{\lambda\in\Lambda_\phi\setminus\{0\}}\frac{\exp(c\lambda)}{\lambda^{q^{-k}}};\]
    all the numerators of the series belong to the compact space $\exp(\K\Lambda_\phi)\cong\faktor{\K\Lambda_\phi}{\Lambda_\phi}$, and since $\Lambda_\phi\subseteq\C$ is discrete all the denominators are bounded from below: this means that the set $\{((\hat{\Phi}_c)_k)^{q^{-k}}\}_{k\gg0}$ is bounded, hence $\hat{\Phi}_c\in\C\langle\tau\rangle$ by Remark \ref{oss C<t>}.
    For all $a\in A$, for all $c\in \K$:
    \begin{align*}
    \hat{\Phi}_a&=(\Phi_a-\mu_a)^*=\phi^*_a\\
    \phi^*_a\circ\hat{\Phi}_c&=(\Phi_c\circ\phi_a-\mu_c\circ\phi_a)^*=(\Phi_{ac}-\mu_{ac})^*=\hat{\Phi}_{ac},
    \end{align*}
    which proves that $\hat{\Phi}$ extends $\phi^*$ multiplicatively.
\end{proof}

\begin{oss}
    For all $c\in\K$ we have:
    \[\mu_c^*=\sum_{k\in\Z}\left(\zeta_\phi\cdot\omega_\phi^{(k)}\right)(c)\tau^k.\]
\end{oss}
A fortiori, we can repackage the results of this subsection under the following theorem.

\begin{teo}\label{main teo}
    Let $\Phi,\hat{\Phi}:\K\to\C\langle\tau\rangle$ be the unique ring homomorphisms which extend respectively $\phi,\phi^*:A\to\C\langle\tau\rangle$ and such that their $k$-th coefficient is a continuous function from $\K$ to $\C$ for all $k\in\Z$. The following identity holds in the $\C[\tau,\tau^{-1}]$-module $\C[[\tau,\tau^{-1}]]$ for all $c\in\K$:
    \[\sum_{k\in\mathbb\Z} \left(\zeta_\phi\cdot\omega_\phi^{(k)}\right)(c)\tau^k=\Phi_c^*-\hat{\Phi}_c.\]
\end{teo}

This Theorem allows us to partially carry out the computation of the dot products $\zeta_\phi\cdot\omega_\phi^{(k)}$, such as in the following Proposition.

\begin{prop}\label{cor 0 forms}
    For all $c\in\K$ with norm less than $1$:
    \[\left(\zeta_\phi\cdot\omega_\phi^{(k)}\right)(c)=\begin{cases}
        c\text{ if }k=0;\\
        0\text{ if }1\leq k\leq r-1.
    \end{cases}\]
\end{prop}
\begin{proof}
    For all $c\in\K$ the lowest degree of $\hat\Phi_c$ is $-r\deg(c)$, while the highest degree of $\Phi_c^*$ is $0$. In particular, if $\|c\|<1$, i.e. $\deg(c)\leq-1$, we have:
    \[\left(\zeta_\phi\cdot\omega_\phi^{(k)}\right)(c)=(\Phi^*_c-\hat{\Phi}_c)_k=\begin{cases}
        (\Phi^*_c-\hat{\Phi}_c)_0=(\Phi^*_c)_0=c&\text{ if }k=0;\\
        (\Phi^*_c-\hat{\Phi}_c)_k=0&\text{ if }1\leq k\leq r-1.
    \end{cases}\]
    \end{proof}

\subsection{Application to the case of genus \texorpdfstring{$0$}{0} and arbitrary rank}

Thanks to Theorem \ref{main teo}, we can compute efficiently the dot products $\zeta_\phi\cdot\omega_\phi^{(k)}$ in the case of genus $0$. In this subsection we suppose $X=\mathbb{P}^1_{\F_q}$, and we fix a rational function $\theta$ over $X$ with a simple pole at $\infty$. In this case we can write $A=\F_q[\theta]$, $\K=\F_q((\theta^{-1}))$ and $\Omega=\F_q[\theta]d\theta$, where $d\theta:\faktor{\K}{A}\to\F_q$ sends $\theta^n$ to $\delta_{-1,n}$ for all $n\in\Z$.

\begin{prop}\label{g=0 i=0}
Let $\phi:\F_q[\theta]\to\C[\tau]$ be a Drinfeld module of rank $r$. We have the following identities in $\C\hat\otimes\Omega$:
    \begin{align*}
        \zeta_\phi\cdot\omega_\phi&=\frac{d\theta}{\theta\otimes1-1\otimes\theta};\\
        \zeta_\phi\cdot\omega_\phi^{(k)}&=0\;\;\;\forall1\leq k\leq r-1.
    \end{align*}
\end{prop}

\begin{proof}
By Proposition \ref{cor 0 forms}, for all $n>0$ we have:
\[\left(\zeta_\phi\cdot\omega_\phi^{(k)}\right)(\theta^{-n})=\begin{cases}
    \theta^{-n}\text{ if }k=0\\
    0\text{ if }1\leq k\leq r-1
\end{cases}.\]
Since $\theta^n\in A$ for all $n\geq0$, we also have $\left(\zeta_\phi\cdot\omega_\phi^{(k)}\right)(\theta^n)=0$ for all $n\geq0$ and for all $k$ so, if $1\leq k\leq r-1$, $\zeta_\phi\cdot\omega_\phi^{(k)}$ is identically zero. If instead $k=0$ we have the following identity for all integers $n$:
\[\left((\theta\otimes1-1\otimes \theta)\zeta_\phi\cdot\omega_\phi\right)(\theta^n)=\theta\left(\zeta_\phi\cdot\omega_\phi(\theta^n)\right)-\zeta_\phi\cdot\omega_\phi(\theta^{n+1})=\delta_{-1,n}=d\theta(\theta^n),\]
hence $\zeta_\phi\cdot\omega_\phi=\frac{d\theta}{(\theta\otimes1-1\otimes \theta)}$.
\end{proof}

We now relate the usual definition of Anderson generating functions to the universal Drinfeld eigenvector, by giving a basis-dependent description of the latter.

\begin{lemma}
    Fix the $A$-linear bases $\{\pi_1,\dots,\pi_r\}$ of $\Lambda_\phi$ and $\{\pi_1^*,\dots,\pi_r^*\}$ of $\Lambda_\phi^*$, where $\pi_i^*(\pi_k)=\delta_{i,k}$. Then, we have:
    \begin{align*}
    &\omega_\phi= \sum_{i=1}^r\sum_{j\geq0} \exp_\phi\left(\frac{\pi_i}{\theta^{j+1}}\right) \otimes\theta^j\pi_i^*d\theta,
    &\zeta_\phi=\sum_{i=1}^r\sum_{j\geq0}\left(\sum_{\lambda\in\Lambda\setminus\{0\}}\frac{d\theta\pi_i^*}{\theta^{j+1}}(\lambda)\lambda^{-1}\right)\otimes\theta^j\pi_i.
\end{align*}
\end{lemma}

\begin{proof}
When used as indices, we imply $i$ to vary among the integers between $1$ and $r$, extremes included, and $j$ to vary among the nonnegative integers.
The chosen bases induce an isomorphism $\Hom_A(\Lambda_\phi,\Omega)\cong\bigoplus_i A d\theta\pi_i^*$. The $\F_q$-linear basis $\{\theta^j d\theta\pi_i^*\}_{i,j}$ of $\Hom_A(\Lambda_\phi,\Omega)$ induces a dual basis $\{\theta^{-j-1}\pi_i\}_{i,j}$ of $\widehat{\Hom_A(\Lambda_\phi,\Omega)}\cong\faktor{\K\Lambda_\phi}{\Lambda_\phi}$. Similarly, the $\F_q$-linear basis $\{\theta^j\pi_i\}_{i,j}$ of $\Lambda_\phi$: induces the dual basis $\{\theta^{-j-1}d\theta\pi_i^*\}_{i,j}$ of $\hat\Lambda_\phi\cong\faktor{\K\Hom_A(\Lambda_\phi,\Omega)}{\Hom_A(\Lambda_\phi,\Omega)}$. This proves the lemma, by virtue of Remark \ref{oss universal special function} and the proof of Theorem \ref{zeta function functor}.
\end{proof}

\begin{Def}
For $i=1,\dots,r$ we define the $i$-th Anderson generating function as:
\[\omega_{\phi,i}\coloneqq\sum_{j\geq0}\exp_\phi\left(\frac{\pi_i}{\theta^{j+1}}\right)\otimes\theta^j\in\C\hat\otimes A.\]

Similarly, for $i=1,\dots,r$ we define the $i$-th dual Anderson generating function as:
    \[\zeta_{\phi,i}= \sum_{j\geq0} \left(\sum_{\lambda\in\Lambda\setminus\{0\}} \frac{d\theta\pi_i^*}{\theta^{j+1}}(\lambda)\lambda^{-1}\right)\otimes\theta^j\in\C\hat\otimes A.\]
\end{Def}
\begin{oss}
For all integers $1\leq i\leq r$, $\omega_{\phi,i}$ and $\zeta_{\phi,i}$ are the unique elements in $\C\hat\otimes A$ such that the identities $(1\otimes\pi_i)(\omega_\phi)=\omega_{\phi,i} d\theta$ and $(1\otimes\pi^*_i)(\zeta_\phi)=\zeta_{\phi,i}$ hold (in $\C\hat\otimes\Omega$ and $\C\hat\otimes A$, respectively).
\end{oss}

\begin{Def}
    Let's define $\boldsymbol{\omega}_\phi\coloneqq(\omega_{\phi,i}^{(j-1)})_{i,j}\in\Mat_{r\times r}(\C\hat\otimes A)$. We call it the \textit{rigid analitic trivialization} of the $t$-motive attached to $\phi$.
\end{Def}

The previous matrix has been studied in various articles (see for example \cite{Pellarin2007}[Section 4.2], \cite{Khaochim}, \cite{Gezmis}). We can use it to state the following Theorem.
\begin{teo}
    The product of $\zeta_\phi\in \Mat_{1\times r}(\C\hat\otimes A)$ and $\boldsymbol{\omega}_\phi\in\Mat_{r\times r}(\C\hat\otimes A)$ is the vector $\frac{1}{(\theta\otimes1-1\otimes \theta)}\cdot(1,0,\dots,0)\in\Mat_{1\times r}(\C\hat\otimes A)$.
\end{teo}
\begin{proof}
    Note that we have interpreted $\zeta_\phi$ as $(\zeta_{\phi,i})_i\in \Mat_{1\times r}(\C\hat\otimes A)$. If we multiply by $d\theta\in\Omega$ the $j$-th coordinate of the product, we get:
    \[\sum_{i=1}^r\omega_{\phi,i}^{(j-1)} \zeta_{\phi,i}d\theta= \left(\sum_{i=1}^r\omega_{\phi,i}\pi_i^*d\theta\right)^{(j-1)}\cdot\left(\sum_{i=1}^r\zeta_{\phi,i}\pi_i\right)=\omega_\phi^{(j-1)}\cdot\zeta_\phi,\]
    which is $\frac{d\theta}{(\theta\otimes1-1\otimes \theta)}$ if $j=1$ by Proposition \ref{g=0 i=0}, and $0$ otherwise by Proposition \ref{cor 0 forms}.
\end{proof}


\begin{oss}
    It's a well known result 
    that the determinant of the matrix $\boldsymbol{\omega}_\phi$ is nonzero (see for example \cite{Gezmis}[Prop. 6.2.4]), so by the previous theorem we can recover $\zeta_\phi$ from $\boldsymbol{\omega}_\phi$.
\end{oss}


\subsection{Application to the case of genus \texorpdfstring{$1$}{1} and rank \texorpdfstring{$1$}{1}}

In the case of rank $1$ normalized Drinfeld modules, the result \cite{Ferraro}[Thm. 7.26] can be used to express the rational form $\zeta_\phi\cdot\omega_\phi$ in terms of the Drinfeld divisor. While Theorem $\ref{main teo}$, in principle, completely describes the form $\zeta_\phi\cdot\omega_\phi$, it's not as explicit a result for arbitrary curves.

In this subsection we show how, restricting ourselves to the case of genus $g(X)=1$ (and rank $1$), we are able to easily recover an expression for $\zeta_\phi\cdot\omega_\phi$ in terms of the coefficients of the Drinfeld module $\phi$. It can also be directly compared to the results of Green and Papanikolas, who tackled this specific case in \cite{Green}.

In this subsection, we suppose that $X$ is an elliptic curve: we can assume that $A=\faktor{\F_q[x,y]}{y^2-ay-P(x)}$ for some monic separable polynomial $P(x)$ of degree $3$ and some constant $a\in\F_q$. We call $\nu\in\Omega$ the differential form that, under the prefect pairing $\Omega\otimes \faktor{\K}{A}\to\F_q$, sends $\frac{y}{x}$ to $1$ and all elements of norm less than $1$ to $0$. It's not difficult to show that $\Omega$ is a free $A$-module generated by $\nu$.
\begin{lemma}\label{lemma 1 g=1}
    Denote by $\left(\faktor{\K}{A}\right)_{<q^{-2}}\subseteq\faktor{\K}{A}$ the subspace of the elements with norm less than $q^{-2}$, and call $C$ the cokernel of this inclusion. Then, the images of $\frac{y}{x},\frac{y}{x^2},\frac{1}{x}$ are a basis of $C$, and the set $\{\nu,x\nu,(y-a)\nu\}$ is the corresponding dual basis of $\hat{C}\subseteq\Omega$.
\end{lemma}
\begin{proof}
    Since $\frac{y}{x},\frac{y}{x^2},\frac{1}{x}$ have degrees respectively $1,-1,-2$, their images are $\F_q$-linearly independent in $C$. Since the set of degrees of the elements in $A$ is $\Z_{\geq0}\setminus\{1\}$, each element in $\faktor{\K}{A}$ can be represented by an element of $\K$ with degree either negative or equal to $1$. In particular, $C$ is spanned by the images of $\frac{y}{x},\frac{y}{x^2},\frac{1}{x}$. 
    On one hand, for all $c\in\left(\faktor{\K}{A}\right)_{<q^{-2}}$:
    \[\begin{rcases*}\nu(c)\\(x\nu)(c)=\nu(xc)\\((y-a)\nu)(c)=\nu((y-a)c)\end{rcases*}=0\text{ because the arguments have norm less than }q,
    \]
    so $\nu,x\nu,(y-a)\nu\in\hat{C}$. On the other hand, we have the following identities, where we set $y':=y-a$:
    \begin{align*}
        &\nu\left(\frac{y}{x}\right)=1;
        &&(x\nu)\left(\frac{y}{x}\right)=\nu(y)=0;
        &&(y'\nu)\left(\frac{y}{x}\right)=\nu\left(\frac{P(x)}{x}\right)=0;\\
        &\nu\left(\frac{y}{x^2}\right)=0;
        &&(x\nu)\left(\frac{y}{x^2}\right)=\nu\left(\frac{y}{x}\right)=1;
        &&(y'\nu)\left(\frac{y}{x^2}\right)=\nu\left(\frac{P(x)}{x^2}\right)=0;\\
        &\nu\left(\frac{1}{x}\right)=0;
        &&(x\nu)\left(\frac{1}{x}\right)=\nu(1)=0;
        &&(y'\nu)\left(\frac{1}{x}\right)=\nu\left(\frac{y}{x}\right)-\nu\left(\frac{a}{x}\right)=1.
    \end{align*}
    This proves the result.
\end{proof}

Let's pick a Drinfeld module $\phi:A\to\C[\tau]$ of rank $1$. We can assume $\phi$ to be normalized without changing the dot product $\zeta_\phi\cdot\omega_\phi$, so we set:
\[\phi_x=x+x_1\tau+\tau^2;\;\;\phi_y=y+y_1\tau+y_2\tau^2+\tau^3.\]
\begin{lemma}\label{lemma 2 g=1}
    We have the following identity:
    \[\zeta_\phi\cdot\omega_\phi\left(\frac{y}{x}\right)=\frac{y}{x}-y_2+x_1^q.\]
\end{lemma}
\begin{proof}
    By Theorem \ref{main teo}, we have:
    \[\zeta_\phi\cdot\omega_\phi\left(\frac{y}{x}\right)=\left(\Phi^*_{\frac{y}{x}}\right)_0-\left(\hat{\Phi}_{\frac{y}{x}}\right)_0=\frac{y}{x}-\left((\phi^*_x)^{-1}\circ\phi^*_y\right)_0.\]
     By direct computation we get:
\[((\phi^*_x)^{-1}\circ\phi^*_y)_0=\left((\tau^2-x_1^q\tau^3+O(\tau^4))(\tau^{-3}+\tau^{-2}y_2+\tau^{-1}y_1+y)\right)_0=y_2-x_1^q.\tag*{\qedhere}\]
\end{proof}   
\begin{prop}\label{prop genus 1}
    We have the following identity of rational differential forms:
    \[\zeta_\phi\cdot\omega_\phi=\frac{(y-x y_2+x x_1^q)\otimes1+(y_2-x_1^q)\otimes x+1\otimes(y-a)}{x\otimes1-1\otimes x}\nu.\]
\end{prop}
\begin{proof}
    By Proposition \ref{cor 0 forms}, for all $c\in\K$ of norm less than $1$, $(\zeta_\phi\cdot\omega_\phi)(c)=0$. In particular, for all $c\in\left(\faktor{\K}{A}\right)_{<q^2}$ we have:
    \[(x\otimes1-1\otimes x)(\zeta_\phi\cdot\omega_\phi)(c)=x(\zeta_\phi\cdot\omega_\phi)(c)-(\zeta_\phi\cdot\omega_\phi)(xc)=0.\]
In particular, by Lemma \ref{lemma 1 g=1} $(x\otimes1-1\otimes x)(\zeta_\phi\cdot\omega_\phi)$ is completely determined by its evaluation at $\frac{y}{x},\frac{y}{x^2},\frac{1}{x}$ as a function from $\faktor{\K}{A}$ to $\C$. Since by Lemma \ref{lemma 2 g=1} $(\zeta_\phi\cdot\omega_\phi)\left(\frac{y}{x}\right)=\frac{y}{x}-y_2+x_1^q$, we can compute the following evaluations:
\begin{align*}
    &(x\otimes1-1\otimes x)(\zeta_\phi\cdot\omega_\phi)\left(\frac{y}{x}\right)=x(\zeta_\phi\cdot\omega_\phi)\left(\frac{y}{x}\right)-(\zeta_\phi\cdot\omega_\phi)(y)=y-x(y_2-x_1^q);\\
    &(x\otimes1-1\otimes x)(\zeta_\phi\cdot\omega_\phi)\left(\frac{y}{x^2}\right)=x(\zeta_\phi\cdot\omega_\phi)\left(\frac{y}{x^2}\right)-(\zeta_\phi\cdot\omega_\phi)\left(\frac{y}{x}\right)=y_2-x_1^q;\\
    &(x\otimes1-1\otimes x)(\zeta_\phi\cdot\omega_\phi)\left(\frac{1}{x}\right)=x(\zeta_\phi\cdot\omega_\phi)\left(\frac{1}{x}\right)-(\zeta_\phi\cdot\omega_\phi)(1)=1.
\end{align*}
By Lemma \ref{lemma 1 g=1} we deduce the following identity in $\C\otimes\Omega$:
\[(x\otimes1-1\otimes x)(\zeta_\phi\cdot\omega_\phi)=\left((y-x y_2+x x_1^q)\otimes1+(y_2-x_1^q)\otimes x+1\otimes(y-a)\right)\nu,\]
which proves the thesis.
\end{proof}

\subsection{Link with the Hartl-Juschka pairing for Drinfeld modules in genus \texorpdfstring{$0$}{0}}

In this subsection, we compare our theory with that of Hartl and Juschka in \cite{Hartl}. In that article, given an abelian Anderson $A$-module $E$, Hartl and Juschka define a perfect pairing of $\C\otimes A$-modules between the dual $A$-motive $\check{M}(E)$ and the twist of the $A$-motive $M(E)$ (i.e. $\tau^*M(E):=M(E)\otimes_{\C\otimes A,\tau}\C\otimes A$). We prove that, in the special case where $E$ comes from a Drinfeld module $\phi$ and $A=\F_q[\theta]$, the Hartl-Juschka pairing is induced by the dot product defined in Lemma \ref{ev}.

\begin{teo}[{\cite{Hartl}[Thm. 5.13]}]
Let $E$ be an abelian, $A$-finite Anderson $A$-module. There is a canonical $\C\otimes A$-linear perfect pairing $HJ:\check{M}(E)\otimes_{\C\otimes A}\tau^* M(E)\to\C\otimes\Omega$.
\end{teo}

If $\phi$ is a Drinfeld module, we can identify the twist of the $A$-motive $M(\phi)$ with the left $\C\otimes A[\tau]$-module $\tau\C[\tau]$, where $\C[\tau]$ acts by composition on the left  and $a\in A$ acts by composition on the right with $\phi_a$. On the other hand, if we denote by $\sigma$ the inverse of the Frobenius endomorphism on $\C$, the dual $A$-motive $\check{M}(\phi)$ can be identified with the left $\C\otimes A[\sigma]$-module $\C[\sigma]$, where $\C[\sigma]$ acts by composition on the left and $a\in A$ acts by composition on the right with $\phi^*_a$.

Hartl and Juschka leave as an open question the computation of the pairing $HJ$ in the general case, but they carry it out in the case $A=\F_q[\theta]$. In particular, they show the following.

\begin{prop}[{\cite{Hartl}[Ex. 5.16]}]\label{Hartl computation}
    Suppose $A=\F_q[\theta]$ and let $\phi:A\to\C[\tau]$ be a Drinfeld module of rank $r$, with $\phi_\theta=\sum_i t_i\tau^i$. Let $\{\alpha_{i,j}\}_{0\leq i,j<r}\in \mathbb{C}_\infty^{r\times r}$ be the matrix with entries $\alpha_{i,j}:=-t_{i+j+1}^{q^{-i}}$, and let $\{\beta_{i,j}\}_{0\leq i,j<r}\in \mathbb{C}_\infty^{r\times r}$ be its inverse. Then for all $0\leq i,j<r$, the following identity holds: $HJ(\sigma^j\otimes\tau^{i+1})=\beta_{i,j}d\theta$.
\end{prop}

In the following proposition, we consider $\C\hat\otimes\Lambda_\phi$ as a left $\C\otimes A[\sigma]$-module, where $\sigma$ sends $x\in\C\hat\otimes\Lambda_\phi$ to $x^{(-1)}$, and we consider $\C\hat\otimes\Hom_A(\Lambda_\phi,\Omega)$ as a left $\C\otimes A[\tau]$-module where $\tau$ sends $x\in\C\hat\otimes\Hom_A(\Lambda_\phi,\Omega)$ to $x^{(1)}$

\begin{prop}\label{prop motive immersion}
    The $\C[\sigma]$-linear morphism $F:\check{M}(\phi)\cong\C[\sigma]\to\C\hat\otimes\Lambda_\phi$ sending $1$ to $\zeta_\phi$ is an injective morphism of $\C\otimes A[\sigma]$-modules.

    The $\C[\tau]$-linear morphism $G:M(\phi)\cong\tau\C[\tau]\to\C\hat\otimes\Hom_A(\Lambda_\phi,\Omega)$ sending $\tau$ to $\omega_\phi^{(1)}$ is an injective morphism of $\C\otimes A[\tau]$-modules.
\end{prop}
\begin{proof}
    Let's prove the $A$-linearity of $F$. Let $x:=\sum_i c_i\sigma^i\in\check{M}(\phi)$, so that its image is $F(x)=\sum_i (c_i\otimes 1)\zeta_\phi^{(-i)}$; for all $a\in A$, since $\zeta_\phi$ is a dual Drinfeld eigenvector we have:
    \[F(x\circ\phi_a^*)=(x\circ\phi_a^*)(F(1))=x(\phi_a^*(\zeta_\phi))=x(\zeta_\phi\cdot(1\otimes a))=(1\otimes a)\cdot x(\zeta_\phi)=(1\otimes a)F(x).\]
    To prove injectivity, let's fix $\sum_i c_i\sigma^i\in\check{M}(\phi)\setminus\{0\}$ and prove that $\sum c_i\zeta_\phi^{(-i)}$ is a nonzero element of $\C\hat\otimes\Lambda_\phi$. Let $N$ be the smallest index such that $c_N\neq0$: we prove that $\left(\sum c_i\zeta_\phi^{(-i)}\right)\cdot\omega_\phi^{(-N)}\in\C\hat\otimes\Omega\setminus\C\otimes\Omega$. On one hand, for all $i>N$, $\zeta_\phi^{(-i)}\cdot\omega_\phi^{(-N)}=\left(\zeta_\phi\cdot\omega_\phi^{(i-N)}\right)\in\C\otimes\Omega$ by Theorem \ref{Teo rationality}. On the other hand, by Proposition \ref{cor 0 forms}, $\zeta_\phi^{(-N)}\cdot\omega_\phi^{(-N)}$ sends any $c\in\K$ with norm less than $1$ to $c^\frac{1}{q^n}$, as a function from $\faktor{\K}{A}$ to $\C$; since any form in $\C\otimes\Omega\subseteq\C\hat\otimes\Omega\cong\Hom_{\F_q}^{cont}\left(\faktor{\K}{A},\C\right)$ has finite support, we deduce that $\zeta_\phi^{(-N)}\cdot\omega_\phi^{(-N)}\not\in\C\otimes\Omega$, in particular $\left(\sum c_i\zeta_\phi^{(-i)}\right)\cdot\omega_\phi^{(-N)}$ is not zero.

    Let's prove the $A$-linearity of $G$. Let $x:=\sum_i c_i\tau^i$ and $x\tau\in M(\phi)$, so that its image is $G(x)=\sum_i (c_i\otimes 1)\omega_\phi^{(i+1)}$; for all $a\in A$, since $\omega_\phi$ is a Drinfeld eigenvector we have:
    \[G(x\tau\phi_a)=(x\tau\phi_a\tau^{-1})G(\tau)=(x\tau)(\phi_a(\omega_\phi))=(1\otimes a)x(\omega_\phi^{(1)})=(1\otimes a)G(x).\]
    To prove injectivity, let's fix $\sum_i c_i\tau^i\in\tau^*M(\phi)\setminus\{0\}$ and prove that $\sum c_i\omega_\phi^{(i)}$ is a nonzero element of $\C\hat\otimes\Hom_A(\Lambda_\phi,\Omega)$. Let $N$ be the smallest index such that $c_N\neq0$: as shown above, for all $i>N$, $\zeta_\phi^{(N)}\cdot\omega_\phi^{(i)}\in\C\otimes\Omega$, while $\zeta_\phi^{(N)}\cdot\omega_\phi^{(N)}\not\in\C\otimes\Omega$, so $\sum c_i\omega^{(i)}\neq0$.
\end{proof}

\begin{oss}
    Under the anti-isomorphism of rings $\C[\sigma]\to\C[\tau]$ sending $\sum_i c_i\sigma^i$ to $\sum_i \tau^i c_i$, the dual $A$-motive $\Span_\C\left(\{\zeta_\phi^{(-i)}\}_{i\in\Z_{\geq0}}\right)\subseteq\C\hat\otimes\Lambda_\phi$ can also be viewed as a right $\C\otimes A[\tau]$ module, where $(c\otimes a)\tau^k$ sends $\sum_i c_i\zeta_\phi^{(-i)}$ to $\sum_i (c_i c)^{q^{-k}}\zeta_\phi^{(-i-k)}(1\otimes a)$.
\end{oss}

We can finally prove the following proposition about the dot product defined in Lemma \ref{ev}.

\begin{teo}\label{teo restricted pairing}
    Suppose $A=\F_q[\theta]$ and let $\phi:A\to\C[\tau]$ be a Drinfeld module. Under the isomorphisms of $\C\otimes A$-modules $\tau^* M(\phi)\cong\C[\tau]\omega_\phi^{(1)}\subseteq\C\hat\otimes\Hom_A(\Lambda_\phi,\Omega)$ and $\check{M}(\phi)\cong\C[\sigma]\zeta_\phi\subseteq\C\hat\otimes\Lambda_\phi$, the $\C\otimes A$-linear pairing $HJ:\check{M}(\phi)\otimes_{\C\otimes A}\tau^* M(\phi)\to\C\otimes\Omega$ coincides with the one induced by the dot product.
\end{teo}
\begin{proof}
    Let $r$ be the rank of $\phi$, and let $\phi_\theta:=\sum_i t_i\tau^i$. Since $\deg_\tau(\phi_\theta)=r$, the sets $\{\zeta_\phi^{(-j)}\}_{0\leq i<r}$ and $\{\omega_\phi^{(i+1)}\}_{0\leq j<r}$ generate respectively $\check{M}(\phi)$ and $\tau^* M(\phi)$ as $\C\otimes A$-modules, hence it suffices to prove that for all $0\leq i,j<r$ $\zeta_\phi^{(-j)}\cdot\omega_\phi^{(i+1)}=HJ\left(\zeta_\phi^{(-j)}\otimes\omega_\phi^{(i+1)}\right)$. By Proposition \ref{Hartl computation}, the thesis is equivalent to the identity of matrices $(t_{i+j+i}^{q^{-i}})_{i,j}\left(\zeta_\phi^{(-k)}\cdot\omega_\phi^{(j+1)}\right)_{j,k}=-\delta_{i,k}d\theta$. 

    If $i>k$, we have:
    \[\sum_{j=0}^{r-1}t_{i+j+1}^{q^{-i}}\zeta_\phi^{(-k)}\cdot\omega_\phi^{(j+1)}=\sum_{j=r-i}^{r-1}t_{i+j+1}^{q^{-i}}\zeta_\phi^{(-k)}\cdot\omega_\phi^{(j+1)}+\sum_{j=0}^{r-1-i}t_{i+j+1}^{q^{-i}}\zeta_\phi^{(-k)}\cdot\omega_\phi^{(j+1)}=0,\]
    where the first sum is $0$ because $t_l=0$ if $l>r$, and the second sum is $0$ because, by Proposition \ref{g=0 i=0}, $\zeta_\phi^{(a)}\cdot\omega_\phi^{(b)}=\left(\zeta_\phi\cdot\omega_\phi^{(b-a)}\right)^{(a)}=0$ if $0<b-a<r$.

    Since $\omega_\phi$ is a Drinfeld eigenvector, the identity $\sum_l t_l\omega_\phi^{(l)}=\omega_\phi(1\otimes\theta)$, hence if $i\leq k$ we have:
    \begin{align*}
        &\left(\sum_{j=0}^{r-1}t_{i+j+1}^{q^{-i}}\zeta_\phi^{(-k)}\cdot\omega_\phi^{(j+1)}\right)^{(i)}=\zeta_\phi^{(i-k)}\cdot\sum_{j=i+1}^{r+i}t_j\omega_\phi^{(j)}\\
        =&-\zeta_\phi^{(i-k)}\cdot\left(\sum_{j=1}^i t_j\omega_\phi^{(j)}\right)+(1\otimes\theta-\theta)\zeta_\phi^{(i-k)}\cdot\omega_\phi.
    \end{align*}
    By Proposition \ref{g=0 i=0}, the first sum is $0$, while $(1\otimes\theta-\theta)\zeta_\phi^{(i-k)}\cdot\omega_\phi$ is $0$ if $i<k$ and $-d\theta$ if $i=k$.
\end{proof}

\section{Some remarkable identities\texorpdfstring{ in $\C$}{}}\label{section technical lemmas}

In this section we provide proofs for the identities in Propositions \ref{identity 1} and \ref{identity 2}. Given a discrete, finite, projective $A$-module $\Lambda\subseteq\C$ of rank $l$, we call $\exp_\Lambda\in\C[[\tau]]$ the relative exponential, which as a function from $\C$ to itself sends $x$ to $x\prod_{\lambda\in\Lambda\setminus\{0\}}\left(1-\frac{x}{\lambda}\right)$. For all $\beta\in\ker(\exp_\Lambda^*)\setminus\{0\}$, define $g_\beta\in\C[[\tau]]$ as the unique, everywhere converging function such that $(1-\tau)g_\beta=\beta\exp_\Lambda$ (see Lemma \ref{lemma g_beta}), and set $g_0=0$.
\begin{customprop}{\ref{identity 1}}
    For all $\beta\in\ker(\exp_\Lambda^*)$, the following identity holds in $\C$:
    \[\beta=-\sum_{\lambda\in\Lambda\setminus\{0\}}\frac{g_\beta(\lambda)}{\lambda}.\]
\end{customprop}

Write $\exp_\Lambda=\sum_{i\geq0}e_i\tau^i$, and denote by $\log_\Lambda=\sum_{i\geq0}l_i\tau^i$ its inverse in $\C[[\tau]]$.

\begin{customprop}{\ref{identity 2}}
    For all integers $k$, for all $c\in\K\setminus\{0\}$ with $\|c\|\leq q^{\frac{k-1}{l}}$, the following identity holds in $\C$:
    \[\sum_{\lambda\in\Lambda\setminus\{0\}}\frac{\exp_\Lambda(c\lambda)}{\lambda^{q^k}}=-\sum_{j=0}^k e_jl_{k-j}^{q^j}c^{q^j},\]
    where by convention the summation on the right hand side is $0$ if $k<0$.
\end{customprop}

\subsection{Lattices}

Throughout this subsection, $C$ will be a complete normed $\K$-vector space (with nonarchimedean norm).

\begin{Def}\label{def Lambda_m}
    An $\F_q$-linear subspace $V\subseteq C$ is a \textit{lattice} if for any positive real number $r$ there are finitely many elements of $V$ of norm at most $r$.
    
    An \textit{ordered basis} of $V$ is a sequence $(v_i)_{i\geq1}$ with the following property: for all $m\geq1$, $v_m$ is an element of $V\setminus\Span_{\F_q}(\{v_i\}_{i<m})$ of least norm.

    We call the sequence of real numbers $(\|v_i\|)_{i\geq1}$ the \textit{norm sequence} of $V$.
\end{Def}
\begin{oss}
    If $V\subseteq C$ is a lattice, every subset $S\subseteq V$ has an element of least norm. In particular, we can construct an ordered basis of $V$ by recursion.
\end{oss}

The next two results aim to justify the nomenclature "ordered basis" and to prove that the norm sequence is well defined.

\begin{lemma}
    If $(v_i)_{i\geq1}$ is an ordered basis of a lattice $V\subseteq C$, it is a basis of $V$ as an $\F_q$-vector space.
\end{lemma}
\begin{proof}
    For all $m\geq1$ $v_m\not\in \Span_{\F_q}(\{v_i\}_{i<m})$, hence the $v_i$'s are $\F_q$-linearly independent. Since for all $r\in\R$ there is a finite number of elements of $V$ with norm at most $r$, the norm sequence $(\|v_i\|)_{i\geq1}$ tends to infinity; in particular, for all $v\in V$ there is an integer $m$ such that $\|v_m\|>\|v\|$, so $v\in \Span_{\F_q}(\{v_i\}_{i<m})$ by construction of $v_m$.
\end{proof}
\begin{prop}\label{norm sequence}
    If $(v_i)_{i\geq1}$ is an ordered basis of a lattice $V\subseteq C$, and $(v'_i)_{i\geq1}$ is a sequence of elements in $V$ that are $\F_q$-linearly independent and with weakly increasing norm, then $\|v'_i\|\geq\|v_i\|$ for all $i$. In particular, the norm sequence of $V$ does not depend on the chosen ordered basis of $V$.
\end{prop}
\begin{proof}
    By contradiction, suppose $\|v'_m\|<\|v_m\|$ for some $m$. Then for all $i\leq m$ we have $\|v'_i\|\leq\|v'_m\|<\|v_m\|$, so $v'_i\in \Span_{\F_q}(\{v_j\}_{j<m})$ by construction of $v_m$; since $\{v'_i\}_{i\leq m}$ is a set of $m$ $\F_q$-linearly independent vectors and $\dim_{\F_q}\Span_{\F_q}(\{v_j\}_{j<m})=m-1$, we have reached a contradiction. If we take $(v'_i)_i$ to be another ordered basis, by this reasoning we get both $\|v'_m\|\geq\|v_m\|$ and $\|v_m\|\geq\|v'_m\|$, hence the norm sequence is independent from the choice of the ordered basis.
\end{proof}

Finally, we show that the norm sequence is reasonably well behaved with regard to subspaces.

\begin{lemma}\label{subspace ordered basis}
Let $W\subseteq V\subseteq C$ be lattices. The norm sequence $(s_i)_{i\geq1}$ of $W$ is a subsequence of the norm sequence $(r_i)_{i\geq1}$ $V$. 

Moreover, if $\dim_{\F_q}\faktor{V}{W}=n<\infty$, for $i\gg0$ we have $r_i=s_{i+n}$.
\end{lemma}
\begin{proof}
Let $(w_i)_{i\geq1}$ be an ordered basis of $W$. Let's construct an ordered basis $(v_i)_{i\geq1}$ of $V$ recursively in the following way. For all $k\geq1$ let $f(k)$ be the least integer such that $w_{f(k)}\not\in\Span_{\F_q}(\{v_i\}_{i<k})$, and let $v_k'$ be an element of least norm in $V\setminus\Span_{\F_q}(\{v_i\}_{i<k})$. If $\|v_k'\|<\|w_{f(k)}\|$, we set $v_k\coloneqq v_k'$, otherwise we set $v_k\coloneqq w_{f(k)}$. 

By construction $(v_k)_{k\geq1}$ is an ordered basis of $V$, so we only need to show that for all $j\geq1$ there is some $k\geq1$ such that $v_k=w_j$. By contradiction, let $j$ be the first integer such that this does not happen, and let $k$ be the greatest integer such that $w_j\not\in\Span_{\F_q}(\{v_i\}_{i<k})$, which exists because $(v_i)_{i\geq1}$ is a basis of $V$. This means that $w_j=\alpha v_k+v$ for some $v\in\Span_{\F_q}(\{v_i\}_{i<k})$ and some constant $\alpha\in\F_q^\times$, and since $v_k\neq w_j$, by our algorithm we must have $\|v_k\|<\|w_j\|$; as a consequence $\|v\|=\|w_j-\alpha v_k\|=\|w_j\|>\|v_k\|$, which is a contradiction because, since $(v_i)_{i\geq1}$ is an ordered basis, $\|v_k\|\geq\|v_i\|$ for all $i<k$, hence $\|v_k\|\geq\|v\|$.

If $\dim_{\F_q}\faktor{V}{W}=n<\infty$, since the basis $\{v_i\}_{i\geq1}$ of $V$ extends the basis $\{w_i\}_{i\geq1}$ of $W$, there are exactly $n$ elements of the former which are not contained in the latter. Since, taking the order into account, $(w_i)_{i\geq1}$ is a subsequence of $(v_i)_{i\geq1}$, for $i\gg0$ we have $v_i=w_{i+n}$, hence $r_i=s_{i+n}$.
\end{proof}


\begin{lemma}\label{norm bound}
    Let $V\subseteq\C$ be a lattice which is also a (projective) $A$-submodule of finite rank $r$, and let $(s_i)_{i\geq1}$ be its norm sequence. Then:
    \begin{itemize}
        \item for all $i\gg0$, $s_{i+r}=q\cdot s_i$;
        \item for all $k\in\Z$, for all $i\gg0$, $\frac{s_{i+k}}{s_i}\leq q^{\left\lceil\frac{k}{r}\right\rceil}$;
        \item for all $k\in\Z$, for infinitely many $i$, $\frac{s_{i+k}}{s_i}\leq q^\frac{k}{r}$.
    \end{itemize}
\end{lemma}
\begin{proof}
    We can choose $a,b\in A\setminus\{0\}$ such that $\deg(b)=\deg(a)+1$. Fix an ordered basis $(v_i)_{i\geq1}$ of $V$: obviously, $(av_i)_{i\geq1}$ and $(bv_i)_{i\geq1}$ are ordered bases respectively of $aV$ and $bV$. Since $V$ has rank $r$, $\dim_{\F_q}\faktor{V}{aV}=r\deg(a)$ and $\dim_{\F_q}\faktor{V}{bV}=r\deg(b)$, therefore by Lemma \ref{subspace ordered basis} we have $\|v_i\|=\|av_{i-r\deg(a)}\|=\|bv_{i-r\deg(b)}\|$ for $i\gg0$. Rearranging the terms, we get that, for $i\gg0$:
    \[\|v_{i-r\deg(a)}\|=\|a\|^{-1}\|b\|\|v_{i-r\deg(b)}\|=q\|v_{i-r\deg(b)}\|.\]
    Shifting the pedices we get $\|v_i\|=q\|v_{i-r(\deg(b)-\deg(a))}\|=q\|v_{i-r}\|$ for $i\gg0$, which is the first statement.

    For all $k\in\Z$, since the norm sequence is weakly increasing, we have the following inequality for $i\gg0$:
    \[\frac{s_{i+k}}{s_i}\leq\frac{s_{i+r\left\lceil\frac{k}{r}\right\rceil}}{s_i}= q^{\left\lceil\frac{k}{r}\right\rceil}.\]
    Moreover, for all $i\gg0$:
\[\prod_{j=0}^{r-1}\frac{s_{i+k(j+1)}}{s_{i+kj}}=\frac{s_{i+kr}}{s_i}=\begin{cases}
    \prod_{j=0}^{k-1}\frac{s_{i+(j+1)r}}{s_{i+jr}}=q^k\text{ if }k\geq0\\
    \prod_{j=k}^{-1}\frac{s_{i+(j+1)r}}{s_{i+jr}}=q^k\text{ if }k<0
\end{cases},\]
hence at least one of the factors on the left hand side has norm at most $q^{\frac{k}{r}}$; this implies that the inequality $\frac{s_{i+k}}{s_i}\leq q^\frac{k}{r}$ holds for infinitely many values of $i$.
    


    
\end{proof}

\subsection{Estimation of the coefficients of \texorpdfstring{$g_\beta$ and $\exp_\Lambda$}{gβ and exp}}

The following result is similar to the well known "vanishing lemma" (see \cite{Goss}[Lemma 8.8.1]).
\begin{lemma}\label{S_n,k}
    Call $S_{n,d}(x_1,\dots,x_n)\in\F_q[x_1,\dots,x_n]$ the sum of the $d$-th powers of all the homogeneous linear polynomials. Suppose that the coefficient of monomial $x_1^{d_1}\cdots x_n^{d_n}$ in the expansion of $S_{n,d}(x_1,\dots,x_n)$ is nonzero: then, for all $1\leq j\leq n$, $\sum_{i=1}^j d_i\geq q^j-1$. In particular, if $d<q^n-1$, $S_{n,d}=0$.
\end{lemma}
\begin{proof}
    The coefficient $c_{d_1,\dots,d_n}$ of the monomial $x_1^{d_1}\cdots x_n^{d_n}$ is:
    \[\frac{d!}{d_1!\cdots d_n!}\sum_{a_1,\dots,a_n\in\F_q}a_1^{d_1}\cdots a_n^{d_n}=\frac{d!}{d_1!\cdots d_n!}\prod_{i=1}^n\left(\sum_{a_i\in\F_q}a_i^{d_i}\right),\]
    where by convention we set $0^0=1$.  On one hand, if the multinomial coefficient $\frac{d!}{d_1!\cdots d_n!}$ is nonzero in $\F_q$, we must have $C(d)=C(d_1)+\cdots+C(d_n)$, where we denote by $C(m)$ the sum of the digits in base $q$ of the nonnegative integer $m$; in particular, for $1\leq j\leq n$ this implies $C(d_1+\cdots+d_j)=C(d_1)+\cdots+C(d_j)$. On the other hand, $\sum_{a_i\in\F_q}a_i^{d_i}\neq0$ if and only if $d_i>0$ and $q-1|d_i$; in particular, this implies $C(d_i)\geq q-1$ for all $i$. 
    
    If $c_{d_1,\dots,d_n}\neq0$, for $1\leq j\leq n$ we have:
    \[C\left(\sum_{i=1}^j d_i\right)=\sum_{i=1}^j C(d_i)\geq(q-1)j,\]
    hence $\sum_{i=1}^j d_i\geq q^j-1$. Applying this to $j=n$ we get the condition $d\geq q^n-1$, therefore $S_{n,d}=0$ for all $d<q^n-1$.
\end{proof}

\begin{Def}
    For a lattice $V\subseteq \C$, for all integers $i\geq0$ we define:
    \[e_{V,i}\coloneqq\sum_{\substack{I\subseteq V\setminus{0}\\|I|=q^i-1}}\prod_{v\in I}v^{-1}\]
    (by convention, $e_{V,0}=1$).
\end{Def}
\begin{oss}\label{remark e_V}
    For all $c\in\C$, since $V\subseteq\C$ is a lattice, the infinite product $c\prod_{v\in V}\left(1-\frac{c}{v}\right)$ converges, and is equal to $\sum_{n\geq0} e_{V,n}c^{q^n}$. In particular, $\sum_{n\geq0}e_{V,n}x^{q^n}\in\C[[x]]$ is the only power series with infinite radius of convergence and with leading coefficient $1$ such that its zeroes are simple and coincide with $V$.
\end{oss}

\begin{lemma}\label{coefficient bounds}
    Fix a lattice $V\subseteq\C$, with norm sequence $(r_i)_{i\geq1}$. Fix an ordered basis $(v_i)_{i\geq1}$ and call $V_m\coloneqq\Span_{\F_q}(\{v_i\}_{i\leq m})$ for all $m\geq0$. We have:
    \begin{itemize}
        \item for all $k\geq0$:
        \[\|e_{V,k}\|\leq\prod_{i=1}^k r_i^{q^{i-1}-q^i};\]
        \item for all $m>0$, for all $k>0$:
        \[\left\|\sum_{v\in V_m}v^{q^k-1}\right\|\begin{cases}=0&\text{ if }k<m\\
        \leq r_m^{q^k-q^m}\prod_{i=1}^m r_i^{q^i-q^{i-1}}&\text{ if }k\geq m\end{cases}.\]
    \end{itemize}
\end{lemma}
\begin{proof}
    For the first part, if $k=0$ then $e_{V,k}=1$, so there is nothing to prove. If $k>0$, we have:
    \[\|e_{V,k}\|=\left\|\sum_{\substack{I\subseteq V\setminus\{0\}\\|I|=q^k-1}}\prod_{v\in I}v^{-1}\right\|\leq\max_{\substack{I\subseteq V\setminus\{0\}\\|I|=q^k-1}}\left\|\prod_{v\in I}v^{-1}\right\|=\left\|\prod_{v\in V_k}v^{-1}\right\|=\prod_{i=1}^k r_i^{q^{i-1}-q^i}.\]
    For the second part, note that the element whose norm we are trying to estimate is equal to $S_{m,q^k-1}(v_1,\dots,v_m)$, in the notation of Lemma \ref{S_n,k}. By that lemma, if $k<m$, the element is zero, otherwise we have the following inequality:
    \[\|S_{m,q^k-1}(v_1,\dots,v_m)\|\leq\max_{\substack{d_1,\dots,d_m\\d_1+\cdots+d_m=q^k-1\\ \forall j\;d_1+\cdots+d_j\geq q^j-1}}\left\|v_1^{d_1}\cdots v_m^{d_m}\right\|.\]
    It's easy to see that the maximum norm of the product $v_1^{d_1}\cdots v_m^{d_m}$ under the specified conditions is obtained when we set $d_i=q^i-q^{i-1}$ for $i<m$ and $d_m=q^k-q^{m-1}$, therefore we get the desired inequality.
\end{proof}



\begin{oss}
    Since $\Lambda\subseteq\K\Lambda$ is discrete and $\K\Lambda\cong K_\infty^r$ is locally compact, $\Lambda$ is a lattice of $\C$. Moreover, $e_{\Lambda,n}$ is exactly the coefficient of $\tau^n$ of the exponential function $\exp_\Lambda\in\C[[\tau]]$.
\end{oss}

Recall the definition of $g_\beta$ given at the start of the section for all $\beta\in\ker(\exp_\Lambda^*)$.

\begin{lemma}\label{V_beta}
    For all $\beta\in\ker(\exp_\Lambda^*)\setminus\{0\}$, $\ker(g_\beta)$ is an $\F_q$-vector subspace of $\Lambda$ of codimension $1$. In particular, $g_\beta=\beta\sum_{n\geq0}e_{\ker(g_\beta),n}\tau^n$.
\end{lemma}
\begin{proof}
    Let's denote by $V_\beta\coloneqq\ker(g_\beta)$. If $c\in V_\beta$ then $\exp_\Lambda(c)=\beta^{-1}(1-\tau)(g_\beta(c))=0$, hence $c\in\Lambda$. Moreover, $g_\beta|_\Lambda$ is an $\F_q$-linear function with image in $\F_q$, hence its kernel $V_\beta$ has codimension at most $1$ in $\Lambda$. It is exactly $1$ because $g_\beta|_\Lambda$ is not identically zero by Theorem \ref{Poonen}.

    From the identity $(1-\tau)\circ g_\beta=\beta\exp_\Lambda$, since the zeroes of $\exp_\Lambda$ are simple, we deduce the same for the zeroes of $g_\beta$, therefore $g_\beta=c_\beta \sum_{n\geq0}e_{V_\beta,n}\tau^n$ for some constant $c_\beta\in\C$ by Remark \ref{remark e_V}. Finally, from the same identity we deduce that the coefficient of $\tau$ in the expansion of $g_\beta$ is $\beta$, hence $c_\beta=\beta$.
\end{proof}

\subsection{Proof of the identities}
We can now prove the main propositions of this section.



\begin{customprop}{\ref{identity 1}}
    For all $\beta\in\ker(\exp_\Lambda^*)$, the following identity holds in $\C$:
    \[\beta=-\sum_{\lambda\in\Lambda\setminus\{0\}}\frac{g_\beta(\lambda)}{\lambda}.\]
\end{customprop}

\begin{proof}

The series converges for all $\beta\in\ker(\exp_\Lambda^*)$ because the denominators belong to the lattice $\Lambda$ and the numerators to $\F_q$. For $\beta=0$ the identity is obvious, hence we can suppose $\beta\neq0$.
Fix an ordered basis $(\lambda_i)_{i\geq1}$ of $\Lambda$ and define $\Lambda_m\coloneqq\Span_{\F_q}(\{\lambda_i\}_{i\leq m})$ for all $m\geq0$. Call $V_\beta\coloneqq\ker(g_\beta)$; by Lemma \ref{V_beta}, $V_\beta\subseteq\Lambda$ has codimension $1$, hence by Lemma \ref{subspace ordered basis}, if we denote by $(r_i)_{i\geq1}$ and $(s_i)_{i\geq1}$ the norm sequences respectively of $\Lambda$ and $V_\beta$, there is a positive integer $N$ such that for all $i<N$ $s_i=r_i$, and for all $i\geq N$ $s_i=r_{i+1}$.
For all $m\geq N$, we define:
    \[S_m\coloneqq\beta+\sum_{\lambda\in\Lambda_m\setminus\{0\}}\frac{g_\beta(\lambda)}{\lambda}=\beta\sum_{k\geq1}e_{V_\beta,k}\sum_{\lambda\in\Lambda_m}\lambda^{q^k-1}.\]
    By Lemma \ref{coefficient bounds}, we have:
    \begin{align*}
        \|\beta^{-1}S_m\|=&\left\|\sum_{k\geq1}e_{V_\beta,k}\sum_{\lambda\in\Lambda_m}\lambda^{q^k-1}\right\|\leq\max_{k\geq m}\left\{\|e_{V_\beta,k}\|\left\|\sum_{\lambda\in\Lambda_m}\lambda^{q^k-1}\right\|\right\}\\
        \leq&\max_{k\geq m}\left\{\left(\prod_{i=1}^k s_i^{q^{i-1}-q^i}\right)\left(r_m^{q^k-q^m}\prod_{i=1}^m r_i^{q^i-q^{i-1}}\right)\right\}\\
        =&\max_{k\geq m}\left\{\left(\prod_{i=N}^k r_{i+1}^{q^{i-1}-q^i}\right)\left(r_m^{q^k-q^m}\prod_{i=N}^m r_i^{q^i-q^{i-1}}\right)\right\}\\
        =&\max_{k\geq m}\left\{\left(\prod_{i=N}^{m} \left(\frac{r_i}{r_{i+1}}\right)^{q^{i}-q^{i-1}}\right)\left(\prod_{i=m+1}^k\left(\frac{r_m}{r_i}\right)^{q^i-q^{i-1}}\right)\right\}\\
        =&\prod_{i=N}^{m} \left(\frac{r_i}{r_{i+1}}\right)^{q^{i}-q^{i-1}}\\
        =&\left(\frac{r_N}{r_{m+1}}\right)^{q^N-q^{N-1}}\prod_{i=N+1}^{m}\left(\frac{r_i}{r_{m+1}}\right)^{q^i-2q^{i-1}+q^{i-2}}\\
        \leq&\left(\frac{r_N}{r_{m+1}}\right)^{q^N-q^{N-1}}.
    \end{align*}
Since this number tends to zero as $m$ tends to infinity, we have the following identity in $\C$:
    \[0=\lim_m S_m=\lim_m\left(\beta+\sum_{\lambda\in\Lambda_m\setminus\{0\}}\frac{g_\beta(\lambda)}{\lambda}\right)=\beta+\sum_{\lambda\in\Lambda\setminus\{0\}}\frac{g_\beta(\lambda)}{\lambda}.\tag*{\qedhere}\]
\end{proof}









Before the last Proposition, let's recall the following well-known result about the coefficients of the logarithm $\log_\Lambda=\sum_i l_i\tau^i$ (see for example the proof of \cite{Ferraro}[Lemma 7.1.8], which holds for arbitrary rank).

\begin{lemma}\label{lemma log coeff}
    For all $i\geq1$, $l_i=-\sum_{\lambda\in\Lambda\setminus\{0\}}\lambda^{1-q^i}$.
\end{lemma}

\begin{customprop}{\ref{identity 2}}
    For all integers $k$, for all $c\in\K\setminus\{0\}$ with $\|c\|\leq q^{\frac{k-1}{l}}$, the following identity holds in $\C$:
    \[\sum_{\lambda\in\Lambda\setminus\{0\}}\frac{\exp_\Lambda(c\lambda)}{\lambda^{q^k}}=-\sum_{j=0}^k e_jl_{k-j}^{q^j}c^{q^j},\]
    where by convention the summation on the right hand side is $0$ if $k<0$.
\end{customprop}

\begin{proof}
    First of all, let's show that all summands on the left hand side are bounded by the same constant, so that the series converges. Let's call $(r_i)_{i\geq1}$ the norm sequence of $\Lambda$. Since $\exp_\Lambda(\K\Lambda)$ is homeomorphic to the compact space $\faktor{\K\Lambda}{\Lambda}$, it's bounded in norm by some positive real constant $C$, therefore for all $\lambda\neq0$ the summand $\frac{\exp_\Lambda(c\lambda)}{\lambda^{q^k}}$ is bounded in norm by $\frac{C}{r_1^{q^k}}$.

Fix an ordered basis $(\lambda_i)_{i\geq1}$ of $\Lambda$ and define $\Lambda_m\coloneqq\Span_{\F_q}(\{\lambda_i\}_{i\leq m})$ for all $m\geq1$; define:
    \[S_m\coloneqq\sum_{\lambda\in\Lambda_m\setminus\{0\}}\frac{\exp_\Lambda(c\lambda)}{\lambda^{q^k}}-\sum_{\substack{0\leq j\leq k\\\lambda\in\Lambda_m\setminus\{0\}}}e_jc^{q^j}\lambda^{q^j-q^k}=\sum_{j\geq1}e_{k+j}c^{q^{k+j}}\left(\sum_{\lambda\in\Lambda_m}\lambda^{q^j-1}\right)^{q^k},\]
    where by convention we set $e_j=0$ for all $j<0$.
By Lemma \ref{coefficient bounds}, for all $m\gg0$ we have:
    \begin{align*}
        \|S_m\|=&\left\|\sum_{j\geq1}e_{k+j}c^{q^{k+j}}\left(\sum_{\lambda\in\Lambda_m}\lambda^{q^j-1}\right)^{q^k}\right\|\leq\max_{j\geq m}\left\{\|e_{k+j}\|\|c\|^{q^{k+j}}\left\|\sum_{\lambda\in\Lambda_m}\lambda^{q^j-1}\right\|^{q^k}\right\}\\
        \leq&\max_{j\geq m}\left\{\|c\|^{q^{k+j}}\left(\prod_{i=1}^{k+j} r_i^{q^{i-1}-q^i}\right)\left(r_m^{q^j-q^m}\prod_{i=1}^m r_i^{q^i-q^{i-1}}\right)^{q^k}\right\}\\
        \leq&\max_{j\geq m}\left\{\|c\|^{q^k}\left(\prod_{i=1-k}^{j} r_{i+k}^{q^{i+k-1}-q^{i+k}}\right)\left(\prod_{i=1}^j (\|c\|r_i)^{q^{i+k}-q^{i+k-1}}\right)\right\}\\
        =& C_k\cdot\max_{j\geq m}\left\{\prod_{i=m+1}^{j}\left(\frac{\|c\|r_i}{r_{i+k}}\right)^{q^{i+k}-q^{i+k-1}}\right\}\\
        &\Longrightarrow\limsup_m\|S_m\|\leq C_k\cdot\limsup_j \prod_{i=m+1}^j \left(\frac{\|c\|r_i}{r_{i+k}}\right)^{q^{i+k}-q^{i+k-1}},
    \end{align*}
where $C_k$ is a nonzero constant which depends on $k$.
Since the norms of nonzero elements of $\K$ are integer powers of $q$, we actually have the inequality $\|c\|\leq q^{\left\lfloor\frac{k-1}{l}\right\rfloor}$. By Lemma \ref{norm bound}, we have:
\begin{align*}
    \|c\|\frac{r_i}{r_{i+k}}&\leq q^{\left\lfloor\frac{k-1}{l}\right\rfloor}\cdot q^{\left\lceil-\frac{k}{l}\right\rceil}=q^{\left\lfloor\frac{k-1}{l}\right\rfloor}\cdot q^{-\left\lfloor\frac{k}{l}\right\rfloor}\leq 1\text{ for all $i$ large enough;}\\
    \|c\|\frac{r_i}{r_{i+k}}&\leq q^{\left\lfloor\frac{k-1}{l}\right\rfloor}\cdot q^{-\frac{k}{l}}\leq q^{\frac{k-1}{l}}\cdot q^{-\frac{k}{l}}=q^{-\frac{1}{l}}<1\text{ for infinitely many values of $i$}.
\end{align*}
The first inequality implies that the limit superior on the right hand side is finite, the second inequality that it is zero.
We deduce that the sequence $\|S_m\|$ converges to $0$. If $k<0$, we get the following identity in $\C$:
\[\sum_{\lambda\in\Lambda\setminus\{0\}}\frac{\exp_\Lambda(c\lambda)}{\lambda^{q^k}}=\lim_m S_m=0.\]
If instead $k\geq0$, we get the following identity in $\C$:
\begin{align*}
    \sum_{\lambda\in\Lambda\setminus\{0\}}\frac{\exp_\Lambda(c\lambda)}{\lambda^{q^k}}&=\lim_m\left(S_m+\sum_{j=0}^k e_jc^{q^j}\sum_{\lambda\in\Lambda_m\setminus\{0\}}\lambda^{q^j-q^k}\right)\\
    &=\sum_{j=0}^{k-1} \left(e_jc^{q^j}\sum_{\lambda\in\Lambda\setminus\{0\}}\lambda^{q^j-q^k}\right)-e_kc^{q^k}\\
    &=-\sum_{j=0}^k e_jl_{k-j}^{q^j}c^{q^j}
\end{align*}
where the last equality follows from Lemma \ref{lemma log coeff}.
\end{proof}



\section{Proof of a conjecture of Gazda and Maurischat}\label{section question}

We apply the results of Section \ref{section special functions} in the context of a Drinfeld module $\phi$ of rank $1$, to answer a question posed by Gazda and Maurischat in \cite{Gazda}.

We know that there is an $f\in\Frac(\C\otimes A)$, called \textit{shtuka function}, such that, for all $\omega\in\C\hat\otimes A$, $\omega\in \Sf_\phi(A)$ if and only if $\omega^{(1)}=f\omega$. In particular, if there is some $\omega\in \Sf_\phi(A)$ which is an invertible element of the ring $\C\hat\otimes A$, for all $\omega'\in \Sf_\phi(A)$ we have $\left(\frac{\omega'}{\omega}\right)^{(1)}=\frac{\omega'}{\omega}$, i.e. $\frac{\omega'}{\omega}\in\F_q\otimes A$, hence $\Sf_\phi(A)=A\cdot\omega$. The conjecture of Gazda and Maurischat in \cite{Gazda} is that the converse is also true, i.e. that if $\Sf_\phi(A)=A$, there is some $\omega\in\Sf_\phi(A)$ which is invertible as an element of $\C\hat\otimes A$.

First, we prove two results to show that Pontryagin duality is well-behaved with respect to norms. For starters, we endow the space $\widehat{\K}\cong\Omega\otimes_A\K$ with a norm $|\cdot|$ such that it is a normed vector space over $(\K,\|\cdot\|)$, and for any ideal $J<A$ we use the same notation for the induced norm on the quotient $\hat{J}$; note that, since $\widehat{\K}$ has dimension $1$ as a $\K$-vector space, $|\cdot|$ is unique up to a scalar factor in $\R_{>0}$.

\begin{prop}\label{norm pairing}
    Up to a scalar factor in $\R_{>0}$, for all $f\in\widehat{\K}\setminus\{0\}$, we have
    \[|f|^{-1}=\min\{\|\lambda\|\text{ s.t. }\lambda\in \K\text{ and }f(\lambda)\neq0\}.\]
\end{prop}
\begin{proof}
    Let $t\in\K$ be a uniformizer: we can identify $\K$ with $\F_q((t))$, where if the series $p(t)=\sum_{i\in\Z}\lambda_i t^i\in\F_q((t))$ has leading term $\lambda_k t^k$, its norm is $q^{-k}$. Consider the function $dt\in\widehat{\F_q((t))}$ which sends $p(t)$ as defined above to $\lambda_{-1}$: under the identification $\K=\F_q((t))$, we have $\widehat{\K}=\F_q((t))dt$, and up to a scalar factor in $\R_{>0}$ we can assume $|dt|=q^{-1}$.
    
    Take $\mu\in\F_q((t))dt\setminus\{0\}$ with leading term $b_k t^k dt$, so that $|\mu|=q^{-k-1}$: if $p\in\F_q((t))$ has $\|p\|< q^{k+1}$, its leading term has degree at least $-k$, hence $\mu(p)=0$; on the other hand $\|t^{-k-1}\|=q^{k+1}$ and $\mu(t^{-k-1})=b_k\neq0$. In particular:
    \[|\mu|^{-1}=q^{k+1}=\min\{\|p\|\text{ s.t. }p\in \F_q((t))\text{ and }\mu(p)\neq0\}.\tag*{\qedhere}\]
\end{proof}
\begin{prop}\label{monotony}
Let $J<A$ be a nonzero ideal and fix an $\F_q$-basis $(a_i)_{i\in I}$ of $J$ strictly ordered by degree, with $(a_i^*)_{i\in I}$ dual basis of $\hat{J}$. The sequence $(|a_i^*|)_{i\in I}$ is strictly decreasing.
\end{prop}
\begin{proof}
We can assume $I\subseteq\Z$ to be the set of degrees of elements in $J$, and that $a_i$ has degree $i$ for all $i\in I$. For all $i\in I$ set $b_i\coloneqq a_i$, while for all $i\in\Z\setminus I$ choose some $b_i\in\K$ with valuation $-i$: since all nonzero elements of $\K$ have integer valuation, it's easy to check that every $c\in\K$ can be expressed in a unique way as $\sum_{i\in\Z}\lambda_i b_i$ where $\lambda_i\in\F_q$ for all $i\in\Z$ and $\lambda_i=0$ for $i\gg0$. Denote by $(b_i^*)_{i\in\Z}$ the sequence in $\widehat{\K}$ determined by the property $b_i^*(b_j)=\delta_{i,j}$ for all $i,j\in\Z$. By Proposition \ref{norm pairing}, up to rescaling $|\cdot|$ by some positive real factor, we have for all $i\in\Z$:
\begin{align*}
    |b_i^*|^{-1}=\min\{\|c\|\text{ s.t. }c\in\K\text{ and }b_i^*(c)\neq0\}=&\min\left\{\left\|\sum_{j\in\Z}\lambda_j b_j\right\|\text{ s.t. }\lambda_i\neq0\right\}=\|b_i\|.
\end{align*}

Let's prove that any $c\in\widehat{\K}$ can be expressed in a unique way as a series $\sum_{i\in\Z}\lambda_i b_i^*$ with $\lambda_i\in\F_q$ for all $i$ and $\lambda_i=0$ for $i\ll0$. We have:
\[c=\sum_{i\in\Z}\lambda_i b_i^*\Leftrightarrow c(b_j)=\left(\sum_{i\in\Z}\lambda_i b_i^*\right)(b_j)\forall j\in\Z\Leftrightarrow c(b_j)=\lambda_j \forall j\in\Z,\]
which proves uniqueness. Viceversa, since $c$ is continuous, $c(b_j)=0$ for $j\ll0$, and since the sequence $(|b_j^*|)_{j\in\Z}=(\|b_j\|^{-1})_{j\in\Z}$ is strictly decreasing and tends to $0$, the series $\sum_{i\in\Z}c(b_i) b_i^*$ converges in $\widehat{\K}$.

For any $c\in\widehat{\K}$, call $\overline{c}$ its projection onto $\hat{J}$. Since $(b_i)_{i\in I}=(a_i)_{i\in I}$ is an $\F_q$-basis of $J$, $\overline{b_i^*}=a_i^*$ if $i\in I$, and $\overline{b_i^*}=0$ otherwise. For all $i\in I$, we have:
\[|a_i^*|=\min\{|c|\text{ s.t. }\overline{c}=a_i^*\}=\min\left\{\left|\sum_{j\in\Z}\lambda_j b_j^*\right|\text{ s.t. }\lambda_j=\delta_{i,j}\forall j\in I\right\}=|b_i^*|=\|a_i\|^{-1}.\tag*{\qedhere}\] 
\end{proof}

\begin{teo}\label{teo Gazda}
    Suppose $\Sf_\phi(A)\cong A$. Then, there is a special function in $\Sf_\phi(A)$ which is invertible as an element of $\C\hat\otimes A$.
\end{teo}
\begin{proof}
    As shown in Corollary \ref{cor Gazda}, $\Lambda_\phi\cong\Omega$.
    Fix an $\F_q$-basis $(a_i)_{i\in I}$ of $A$ like in the proof of Proposition \ref{monotony}, with $a_0=1$, and let $(a_i^*)_{i\in I}$ be the dual basis of its Pontryagin dual $\hat{A}\cong\faktor{\Omega\otimes_A\K}{\Omega}\cong\faktor{\K\Lambda_\phi}{\Lambda_\phi}$.
    
    By Remark \ref{oss universal special function}, we can write the universal Drinfeld eigenvector as an infinite series $\omega_\phi=\sum_i\exp_\phi(a_i^*)\otimes a_i\in\C\hat\otimes A$ (where by slight abuse of notation we considered $\exp_\phi$ as a map from $\faktor{\K\Lambda_\phi}{\Lambda_\phi}$ to $\C$). To prove it is invertible, it suffices to show that, for all $i\geq1$, $\|\exp_\phi(a_0^*)\|>\|\exp_\phi(a_i^*)\|$: indeed, if this is the case, and we set $\omega\coloneqq(\exp_\phi(a_0^*)^{-1}\otimes1)\omega_\phi$, the element $1-\omega\in\C\hat\otimes A$ has norm less than $1$, hence the series $\sum_{n\geq0}(1-\omega)^n$ converges in $\C\hat\otimes A$, and is an inverse to $1-(1-\omega)=\omega$.
    
    For all indices $i$, choose a lifting $c_i\in\K\Lambda_\phi\subseteq\C$ of $a_i^*\in\faktor{\K\Lambda_\phi}{\Lambda_\phi}$ with the least norm, so that $\|c_i\|=|a_i^*|$; in particular, since $\Lambda_\phi$ has rank $1$, there are no $\lambda\in\Lambda_\phi$ such that $\|\lambda\|=\|c_i\|$, so we have:
    \[\|\exp_\phi(a_i^*)\|=\|c_i\|\prod_{\lambda\in\Lambda_\phi\setminus\{0\}}\left\|1-\frac{c_i}{\lambda}\right\|=\|c_i\|\prod_{\substack{\lambda\in\Lambda_\phi\setminus\{0\}\\\|\lambda\|\leq\|c_i\|}}\left\|1-\frac{c_i}{\lambda}\right\|=\|c_i\|\prod_{\substack{\lambda\in\Lambda_\phi\setminus\{0\}\\\|\lambda\|<\|c_i\|}}\left\|\frac{c_i}{\lambda}\right\|.\]
    Since by Proposition \ref{monotony} the sequence $(\|c_i\|)_i$ is strictly decreasing, from the previous equality we deduce that the sequence $(\|\exp_\phi(a_i^*)\|)_i$ is also strictly decreasing. In particular, $\|\exp_\phi(a_0^*)\|>\|\exp_\phi(a_i^*)\|$ for all $i\geq1$.
\end{proof}

\bibliography{main}
\end{document}