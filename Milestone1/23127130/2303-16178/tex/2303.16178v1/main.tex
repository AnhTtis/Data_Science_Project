\documentclass[10pt,conference]{IEEEtran}
%\IEEEoverridecommandlockouts
% The preceding line is only needed to identify funding in the first footnote. If that is unneeded, please comment it out.
%\documentclass[sigconf,review, anonymous]{acmart}
%\acmConference[ASE 2022]{The37th IEEE/ACM International Conference on Automated Software Engineering}{Sept. 26 -- Oct. 1, 2022}{Ann Arbor, MI, USA}
%\acmConference[ESEM 2022]{The 16th ACM/IEEE International Symposium on Empirical Software Engineering and Measurement}{Sept. 19 -- Sept. 23, 2022}{Helsinki, Finland}
%\usepackage{cite}
%\usepackage{amsmath,amssymb,amsfonts}
\usepackage{amsmath,amsfonts}
\usepackage{algorithmic}\usepackage[export]{adjustbox}
\usepackage{graphicx}
\usepackage{textcomp}
\usepackage{rotating}
\usepackage{array}
\usepackage{xcolor}
\usepackage{tikz}
\usepackage{makecell}
\usepackage{multirow}
\usepackage{dblfloatfix}
\usepackage[export]{adjustbox}
%\usepackage{floatrow}
\usepackage{etoolbox}
\usepackage{url}
\usepackage{caption,setspace}
\def\BibTeX{{\rm B\kern-.05em{\sc i\kern-.025em b}\kern-.08em
    T\kern-.1667em\lower.7ex\hbox{E}\kern-.125emX}}
\begin{document}

\newcommand*\circled[1]{\tikz[baseline=(char.base)]{
		\node[shape=circle,draw,inner sep=0.8pt] (char) {#1};}}
	
%\setcopyright{acmcopyright}

\title{Label Smoothing Improves Neural\\Source Code Summarization
}


\author{\IEEEauthorblockN{Sakib Haque,
		Aakash Bansal, and
		Collin McMillan\\
		\IEEEauthorblockA{Department of Computer Science\\
			University of Notre Dame, Notre Dame, IN, USA\\
			Email: \{shaque, abansal1, cmc\}@nd.edu}}}

\maketitle

%\IEEEtitleabstractindextext{
\begin{abstract}
Label smoothing is a regularization technique for neural networks.  Normally neural models are trained to an output distribution that is a vector with a single 1 for the correct prediction, and 0 for all other elements.  Label smoothing converts the correct prediction location to something slightly less than 1, then distributes the remainder to the other elements such that they are slightly greater than 0.  A conceptual explanation behind label smoothing is that it helps prevent a neural model from becoming ``overconfident'' by forcing it to consider alternatives, even if only slightly.  Label smoothing has been shown to help several areas of language generation, yet typically requires considerable tuning and testing to achieve the optimal results.  This tuning and testing has not been reported for neural source code summarization -- a growing research area in software engineering that seeks to generate natural language descriptions of source code behavior.  In this paper, we demonstrate the effect of label smoothing on several baselines in neural code summarization, and conduct an experiment to find good parameters for label smoothing and make recommendations for its use.
\end{abstract}

\begin{IEEEkeywords}
	Source code summarization, automatic documentation generation, label smoothing, optimization
\end{IEEEkeywords}
%}

%\keywords{source code summarization, automatic documentation generation, label smoothing, optimization}

%\maketitle

\input intro
\input background
\input experiment
\input results
\input discussion

\section*{Acknowledgment}

This work is supported in part by NSF CCF-1452959 and CCF-1717607. Any opinions, findings, and conclusions expressed herein are the authors and do not necessarily reflect those of the sponsors.

\bibliographystyle{IEEEtran}
\bibliography{main}

\end{document}
