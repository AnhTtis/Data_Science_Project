%% bare_jrnl.tex
%% V1.4b
%% 2015/08/26
%% by Michael Shell
%% see http://www.michaelshell.org/
%% for current contact information.
%%
%% This is a skeleton file demonstrating the use of IEEEtran.cls
%% (requires IEEEtran.cls version 1.8b or later) with an IEEE
%% journal paper.
%%
%% Support sites:
%% http://www.michaelshell.org/tex/ieeetran/
%% http://www.ctan.org/pkg/ieeetran
%% and
%% http://www.ieee.org/

%%*************************************************************************
%% Legal Notice:
%% This code is offered as-is without any warranty either expressed or
%% implied; without even the implied warranty of MERCHANTABILITY or
%% FITNESS FOR A PARTICULAR PURPOSE!
%% User assumes all risk.
%% In no event shall the IEEE or any contributor to this code be liable for
%% any damages or losses, including, but not limited to, incidental,
%% consequential, or any other damages, resulting from the use or misuse
%% of any information contained here.
%%
%% All comments are the opinions of their respective authors and are not
%% necessarily endorsed by the IEEE.
%%
%% This work is distributed under the LaTeX Project Public License (LPPL)
%% ( http://www.latex-project.org/ ) version 1.3, and may be freely used,
%% distributed and modified. A copy of the LPPL, version 1.3, is included
%% in the base LaTeX documentation of all distributions of LaTeX released
%% 2003/12/01 or later.
%% Retain all contribution notices and credits.
%% ** Modified files should be clearly indicated as such, including  **
%% ** renaming them and changing author support contact information. **
%%*************************************************************************


% *** Authors should verify (and, if needed, correct) their LaTeX system  ***
% *** with the testflow diagnostic prior to trusting their LaTeX platform ***
% *** with production work. The IEEE's font choices and paper sizes can   ***
% *** trigger bugs that do not appear when using other class files.       ***                          ***
% The testflow support page is at:
% http://www.michaelshell.org/tex/testflow/



\documentclass[journal]{IEEEtran}
%
% If IEEEtran.cls has not been installed into the LaTeX system files,
% manually specify the path to it like:
% \documentclass[journal]{../sty/IEEEtran}





% Some very useful LaTeX packages include:
% (uncomment the ones you want to load)


% *** MISC UTILITY PACKAGES ***
%
%\usepackage{ifpdf}
% Heiko Oberdiek's ifpdf.sty is very useful if you need conditional
% compilation based on whether the output is pdf or dvi.
% usage:
% \ifpdf
%   % pdf code
% \else
%   % dvi code
% \fi
% The latest version of ifpdf.sty can be obtained from:
% http://www.ctan.org/pkg/ifpdf
% Also, note that IEEEtran.cls V1.7 and later provides a builtin
% \ifCLASSINFOpdf conditional that works the same way.
% When switching from latex to pdflatex and vice-versa, the compiler may
% have to be run twice to clear warning/error messages.

% *** PACKAGES ***
\usepackage{amsfonts}
\usepackage{color}
\usepackage{amsmath}
\DeclareMathOperator*{\argmax}{arg\,max}
\interdisplaylinepenalty=2500
\usepackage{graphicx}
\newtheorem{Proposition}{Proposition}[section]
\newtheorem{Corollary}{Corollary}[section]
\newtheorem{Lemma}{Lemma}[section]
\newtheorem{Remark}{Remark}
% \usepackage{algorithm}
% \usepackage{algorithmic}
\usepackage[linesnumbered,lined,ruled,commentsnumbered]{algorithm2e}%
\usepackage{bm}
\usepackage{multirow}
\newcommand{\minitab}[2][l]{\begin{tabular}{#1}#2\end{tabular}}

\usepackage{float}

%\usepackage[caption=false,font=normalsize,labelfont=sf,textfont=sf]{subfig}
\usepackage[caption=false,font=normalsize]{subfig}
\captionsetup[subfigure]{position=top,font={ rm,md,up },justification=raggedright,
captionskip=-1pt,singlelinecheck=false}%%
% \captionsetup[subtable]{position=top,font={ rm,md,up },justification=raggedright,
% captionskip= 2pt,singlelinecheck=false,farskip= 1pt}%
\captionsetup[subtable]{position=top,font={ rm,md,up },justification=raggedright,
captionskip= 1pt,farskip= 0pt}%

%\usepackage[numbers,sort&compress]{natbib}


% *** CITATION PACKAGES ***
%
\usepackage{cite}
% cite.sty was written by Donald Arseneau
% V1.6 and later of IEEEtran pre-defines the format of the cite.sty package
% \cite{} output to follow that of the IEEE. Loading the cite package will
% result in citation numbers being automatically sorted and properly
% "compressed/ranged". e.g., [1], [9], [2], [7], [5], [6] without using
% cite.sty will become [1], [2], [5]--[7], [9] using cite.sty. cite.sty's
% \cite will automatically add leading space, if needed. Use cite.sty's
% noadjust option (cite.sty V3.8 and later) if you want to turn this off
% such as if a citation ever needs to be enclosed in parenthesis.
% cite.sty is already installed on most LaTeX systems. Be sure and use
% version 5.0 (2009-03-20) and later if using hyperref.sty.
% The latest version can be obtained at:
% http://www.ctan.org/pkg/cite
% The documentation is contained in the cite.sty file itself.






% *** GRAPHICS RELATED PACKAGES ***
%
\ifCLASSINFOpdf
  % \usepackage[pdftex]{graphicx}
  % declare the path(s) where your graphic files are
  % \graphicspath{{../pdf/}{../jpeg/}}
  % and their extensions so you won't have to specify these with
  % every instance of \includegraphics
  % \DeclareGraphicsExtensions{.pdf,.jpeg,.png}
\else
  % or other class option (dvipsone, dvipdf, if not using dvips). graphicx
  % will default to the driver specified in the system graphics.cfg if no
  % driver is specified.
  % \usepackage[dvips]{graphicx}
  % declare the path(s) where your graphic files are
  % \graphicspath{{../eps/}}
  % and their extensions so you won't have to specify these with
  % every instance of \includegraphics
  % \DeclareGraphicsExtensions{.eps}
\fi
% graphicx was written by David Carlisle and Sebastian Rahtz. It is
% required if you want graphics, photos, etc. graphicx.sty is already
% installed on most LaTeX systems. The latest version and documentation
% can be obtained at:
% http://www.ctan.org/pkg/graphicx
% Another good source of documentation is "Using Imported Graphics in
% LaTeX2e" by Keith Reckdahl which can be found at:
% http://www.ctan.org/pkg/epslatex
%
% latex, and pdflatex in dvi mode, support graphics in encapsulated
% postscript (.eps) format. pdflatex in pdf mode supports graphics
% in .pdf, .jpeg, .png and .mps (metapost) formats. Users should ensure
% that all non-photo figures use a vector format (.eps, .pdf, .mps) and
% not a bitmapped formats (.jpeg, .png). The IEEE frowns on bitmapped formats
% which can result in "jaggedy"/blurry rendering of lines and letters as
% well as large increases in file sizes.
%
% You can find documentation about the pdfTeX application at:
% http://www.tug.org/applications/pdftex





% *** MATH PACKAGES ***
%
%\usepackage{amsmath}
% A popular package from the American Mathematical Society that provides
% many useful and powerful commands for dealing with mathematics.
%
% Note that the amsmath package sets \interdisplaylinepenalty to 10000
% thus preventing page breaks from occurring within multiline equations. Use:
%\interdisplaylinepenalty=2500
% after loading amsmath to restore such page breaks as IEEEtran.cls normally
% does. amsmath.sty is already installed on most LaTeX systems. The latest
% version and documentation can be obtained at:
% http://www.ctan.org/pkg/amsmath





% *** SPECIALIZED LIST PACKAGES ***
%
%\usepackage{algorithmic}
% algorithmic.sty was written by Peter Williams and Rogerio Brito.
% This package provides an algorithmic environment fo describing algorithms.
% You can use the algorithmic environment in-text or within a figure
% environment to provide for a floating algorithm. Do NOT use the algorithm
% floating environment provided by algorithm.sty (by the same authors) or
% algorithm2e.sty (by Christophe Fiorio) as the IEEE does not use dedicated
% algorithm float types and packages that provide these will not provide
% correct IEEE style captions. The latest version and documentation of
% algorithmic.sty can be obtained at:
% http://www.ctan.org/pkg/algorithms
% Also of interest may be the (relatively newer and more customizable)
% algorithmicx.sty package by Szasz Janos:
% http://www.ctan.org/pkg/algorithmicx




% *** ALIGNMENT PACKAGES ***
%
%\usepackage{array}
% Frank Mittelbach's and David Carlisle's array.sty patches and improves
% the standard LaTeX2e array and tabular environments to provide better
% appearance and additional user controls. As the default LaTeX2e table
% generation code is lacking to the point of almost being broken with
% respect to the quality of the end results, all users are strongly
% advised to use an enhanced (at the very least that provided by array.sty)
% set of table tools. array.sty is already installed on most systems. The
% latest version and documentation can be obtained at:
% http://www.ctan.org/pkg/array


% IEEEtran contains the IEEEeqnarray family of commands that can be used to
% generate multiline equations as well as matrices, tables, etc., of high
% quality.




% *** SUBFIGURE PACKAGES ***
%\ifCLASSOPTIONcompsoc
%  \usepackage[caption=false,font=normalsize,labelfont=sf,textfont=sf]{subfig}
%\else
%  \usepackage[caption=false,font=footnotesize]{subfig}
%\fi
% subfig.sty, written by Steven Douglas Cochran, is the modern replacement
% for subfigure.sty, the latter of which is no longer maintained and is
% incompatible with some LaTeX packages including fixltx2e. However,
% subfig.sty requires and automatically loads Axel Sommerfeldt's caption.sty
% which will override IEEEtran.cls' handling of captions and this will result
% in non-IEEE style figure/table captions. To prevent this problem, be sure
% and invoke subfig.sty's "caption=false" package option (available since
% subfig.sty version 1.3, 2005/06/28) as this is will preserve IEEEtran.cls
% handling of captions.
% Note that the Computer Society format requires a larger sans serif font
% than the serif footnote size font used in traditional IEEE formatting
% and thus the need to invoke different subfig.sty package options depending
% on whether compsoc mode has been enabled.
%
% The latest version and documentation of subfig.sty can be obtained at:
% http://www.ctan.org/pkg/subfig




% *** FLOAT PACKAGES ***
%
%\usepackage{fixltx2e}
% fixltx2e, the successor to the earlier fix2col.sty, was written by
% Frank Mittelbach and David Carlisle. This package corrects a few problems
% in the LaTeX2e kernel, the most notable of which is that in current
% LaTeX2e releases, the ordering of single and double column floats is not
% guaranteed to be preserved. Thus, an unpatched LaTeX2e can allow a
% single column figure to be placed prior to an earlier double column
% figure.
% Be aware that LaTeX2e kernels dated 2015 and later have fixltx2e.sty's
% corrections already built into the system in which case a warning will
% be issued if an attempt is made to load fixltx2e.sty as it is no longer
% needed.
% The latest version and documentation can be found at:
% http://www.ctan.org/pkg/fixltx2e


%\usepackage{stfloats}
% stfloats.sty was written by Sigitas Tolusis. This package gives LaTeX2e
% the ability to do double column floats at the bottom of the page as well
% as the top. (e.g., "\begin{figure*}[!b]" is not normally possible in
% LaTeX2e). It also provides a command:
%\fnbelowfloat
% to enable the placement of footnotes below bottom floats (the standard
% LaTeX2e kernel puts them above bottom floats). This is an invasive package
% which rewrites many portions of the LaTeX2e float routines. It may not work
% with other packages that modify the LaTeX2e float routines. The latest
% version and documentation can be obtained at:
% http://www.ctan.org/pkg/stfloats
% Do not use the stfloats baselinefloat ability as the IEEE does not allow
% \baselineskip to stretch. Authors submitting work to the IEEE should note
% that the IEEE rarely uses double column equations and that authors should try
% to avoid such use. Do not be tempted to use the cuted.sty or midfloat.sty
% packages (also by Sigitas Tolusis) as the IEEE does not format its papers in
% such ways.
% Do not attempt to use stfloats with fixltx2e as they are incompatible.
% Instead, use Morten Hogholm'a dblfloatfix which combines the features
% of both fixltx2e and stfloats:
%
% \usepackage{dblfloatfix}
% The latest version can be found at:
% http://www.ctan.org/pkg/dblfloatfix




%\ifCLASSOPTIONcaptionsoff
%  \usepackage[nomarkers]{endfloat}
% \let\MYoriglatexcaption\caption
% \renewcommand{\caption}[2][\relax]{\MYoriglatexcaption[#2]{#2}}
%\fi
% endfloat.sty was written by James Darrell McCauley, Jeff Goldberg and
% Axel Sommerfeldt. This package may be useful when used in conjunction with
% IEEEtran.cls'  captionsoff option. Some IEEE journals/societies require that
% submissions have lists of figures/tables at the end of the paper and that
% figures/tables without any captions are placed on a page by themselves at
% the end of the document. If needed, the draftcls IEEEtran class option or
% \CLASSINPUTbaselinestretch interface can be used to increase the line
% spacing as well. Be sure and use the nomarkers option of endfloat to
% prevent endfloat from "marking" where the figures would have been placed
% in the text. The two hack lines of code above are a slight modification of
% that suggested by in the endfloat docs (section 8.4.1) to ensure that
% the full captions always appear in the list of figures/tables - even if
% the user used the short optional argument of \caption[]{}.
% IEEE papers do not typically make use of \caption[]'s optional argument,
% so this should not be an issue. A similar trick can be used to disable
% captions of packages such as subfig.sty that lack options to turn off
% the subcaptions:
% For subfig.sty:
% \let\MYorigsubfloat\subfloat
% \renewcommand{\subfloat}[2][\relax]{\MYorigsubfloat[]{#2}}
% However, the above trick will not work if both optional arguments of
% the \subfloat command are used. Furthermore, there needs to be a
% description of each subfigure *somewhere* and endfloat does not add
% subfigure captions to its list of figures. Thus, the best approach is to
% avoid the use of subfigure captions (many IEEE journals avoid them anyway)
% and instead reference/explain all the subfigures within the main caption.
% The latest version of endfloat.sty and its documentation can obtained at:
% http://www.ctan.org/pkg/endfloat
%
% The IEEEtran \ifCLASSOPTIONcaptionsoff conditional can also be used
% later in the document, say, to conditionally put the References on a
% page by themselves.




% *** PDF, URL AND HYPERLINK PACKAGES ***
%
%\usepackage{url}
% url.sty was written by Donald Arseneau. It provides better support for
% handling and breaking URLs. url.sty is already installed on most LaTeX
% systems. The latest version and documentation can be obtained at:
% http://www.ctan.org/pkg/url
% Basically, \url{my_url_here}.




% *** Do not adjust lengths that control margins, column widths, etc. ***
% *** Do not use packages that alter fonts (such as pslatex).         ***
% There should be no need to do such things with IEEEtran.cls V1.6 and later.
% (Unless specifically asked to do so by the journal or conference you plan
% to submit to, of course. )


% correct bad hyphenation here
\hyphenation{op-tical net-works semi-conduc-tor}


\begin{document}
%
% paper title
% Titles are generally capitalized except for words such as a, an, and, as,
% at, but, by, for, in, nor, of, on, or, the, to and up, which are usually
% not capitalized unless they are the first or last word of the title.
% Linebreaks \\ can be used within to get better formatting as desired.
% Do not put math or special symbols in the title.
%\title{Optimal Distributed Data Association and Estimation Fusion with Random Coefficient Matrices Kalman Filter}

%\title{On Equivalence of Measurement Transformation for Multisensor Track-to-Track Fusion}

%\title{On Equivalence of Measurement Transformation for Multisensor Track Association}

%\title{On Equivalency of Track Fusion with Raw Measurements and  That with Transformed Measurements}
\title{On Communication-Efficient Multisensor Track Association via Measurement Transformation (Extended Version)}
%\title{On Equivalency between Track Association with Raw Measurements and that with Transformed Measurements}

%Lossless Linear Transformation of Track Measurements for Distributed Data Association
%Lossless Linear Transformation of Track Measurements on Distributed Fusion for Multi-target Tracking
%Lossless Linear Transformation of Track Measurements for Multi-target Track Fusion
%Lossless Linear Transformation of Track Measurements for Track-to-track Fusion

%
% author names and IEEE memberships
% note positions of commas and nonbreaking spaces ( ~ ) LaTeX will not break
% a structure at a ~ so this keeps an author's name from being broken across
% two lines.
% use \thanks{} to gain access to the first footnote area
% a separate \thanks must be used for each paragraph as LaTeX2e's \thanks
% was not built to handle multiple paragraphs
%\thanks{This work was supported in part by the NSFC No. 61673282, U1836103 and the PCSIRT16R53.}
\author{Haiqi~Liu,
		Jiajie~Sun,
		Xuqi~Zhang,
		Fanqin~Meng,
		Xiaojing~Shen,
		and~Pramod~K.~Varshney% <-this % stops a space
\thanks{This work was supported in part by Sichuan Youth Science and Technology Innovation Team under Grant 2022JDTD0014, Grant 2021JDJQ0036, and Grant SQ2020YFA070244. \textit{(Corresponding author: Jiajie Sun.)}}
\thanks{Haiqi Liu, Jiajie Sun, Xuqi Zhang and Xiaojing Shen are with School of Mathematics, Sichuan University, Chengdu, Sichuan 610064, China (e-mail: haiqiliu0330@163.com, sunjiajie369@126.com, zxqcc@stu.scu.edu.cn, shenxj@scu.edu.cn).}%
\thanks{Fanqin Meng is with School of Aeronautics and Astronautics, Sichuan University, Chengdu, Sichuan 610064, China, also with School of Automation and Information Engineering, Sichuan University of Science and Engineering, Yibin, Sichuan 644000, China (e-mail: mengfanqin2008@163.com).}%
\thanks{P. K. Varshney is with the Department of Electrical Engineering and Computer Science, Syracuse University, Syracuse, NY 13244 USA (e-mail: varshney@syr.edu).}
}

% note the % following the last \IEEEmembership and also \thanks -
% these prevent an unwanted space from occurring between the last author name
% and the end of the author line. i.e., if you had this:
%
% \author{....lastname \thanks{...} \thanks{...} }
%                     ^------------^------------^----Do not want these spaces!
%
% a space would be appended to the last name and could cause every name on that
% line to be shifted left slightly. This is one of those "LaTeX things". For
% instance, "\textbf{A} \textbf{B}" will typeset as "A B" not "AB". To get
% "AB" then you have to do: "\textbf{A}\textbf{B}"
% \thanks is no different in this regard, so shield the last } of each \thanks
% that ends a line with a % and do not let a space in before the next \thanks.
% Spaces after \IEEEmembership other than the last one are OK (and needed) as
% you are supposed to have spaces between the names. For what it is worth,
% this is a minor point as most people would not even notice if the said evil
% space somehow managed to creep in.



% The paper headers
% \markboth{Journal of \LaTeX\ Class Files,~Vol.~14, No.~8, August~2015}%
% {Shell \MakeLowercase{\textit{et al.}}: Bare Demo of IEEEtran.cls for IEEE Journals}
% The only time the second header will appear is for the odd numbered pages
% after the title page when using the twoside option.
%
% *** Note that you probably will NOT want to include the author's ***
% *** name in the headers of peer review papers.                   ***
% You can use \ifCLASSOPTIONpeerreview for conditional compilation here if
% you desire.




% If you want to put a publisher's ID mark on the page you can do it like
% this:
%\IEEEpubid{0000--0000/00\$00.00~\copyright~2015 IEEE}
% Remember, if you use this you must call \IEEEpubidadjcol in the second
% column for its text to clear the IEEEpubid mark.



% use for special paper notices
%\IEEEspecialpapernotice{(Invited Paper)}




% make the title area
\maketitle

% As a general rule, do not put math, special symbols or citations
% in the abstract or keywords.
{\color{blue}\begin{abstract}
%Multisensor track-to-track fusion for target tracking involves two main operations: estimation fusion and track association. Lossless measurement transformation of sensor measurements has been proposed for estimation fusion, where each sensor sends the transformed data to a fusion center. In this paper, we investigate the track association problem in terms of track-to-track association likelihoods. We show that the track association algorithm with linear transformations of measurements is equivalent to the track association algorithm with raw measurements. Moreover, by using suitable transformations, the communication requirements from each sensor to the fusion center can be or less than those of fusion with raw measurements. An extension to the case of track association without prior track information, and a belief propagation track association approach to reduce the computational complexity is also discussed. Numerical examples  for tracking an unknown number of targets using limited field-of-view sensors verify that the performance of fusion with track measurement transformation is the same as that of fusion with raw measurements.
Multisensor track-to-track fusion for target tracking involves two primary operations: track association and estimation fusion. For estimation fusion, lossless measurement transformation of sensor measurements has been proposed for single target tracking. In this paper, we investigate track association which is a fundamental and important problem for multitarget tracking. First, since the optimal track association problem is a multi-dimensional assignment (MDA) problem, we demonstrate that MDA-based data association (with and without prior track information) using linear transformations of track measurements is lossless, and is equivalent to that using raw track measurements. Second, recent superior scalability and performance of belief propagation (BP) algorithms enable new real-time applications of multitarget tracking with resource-limited devices. Thus, we present a BP-based multisensor track association method with transformed measurements and show that it is equivalent to that with raw measurements. Third, considering communication constraints, it is more beneficial for local sensors to send in compressed data. Two analytical lossless transformations for track association are provided, and it is shown that their communication requirements from each sensor to the fusion center are less than those of fusion with raw track measurements. Numerical examples for tracking an unknown number of targets verify that track association with transformed track measurements has the same performance as that with raw measurements and requires fewer communication bandwidths.
\end{abstract}}

% Note that keywords are not normally used for peerreview papers.
\begin{IEEEkeywords}
  Multitarget tracking,  track association, estimation fusion, measurement transformation.
\end{IEEEkeywords}






% For peer review papers, you can put extra information on the cover
% page as needed:
% \ifCLASSOPTIONpeerreview
% \begin{center} \bfseries EDICS Category: 3-BBND \end{center}
% \fi
%
% For peerreview papers, this IEEEtran command inserts a page break and
% creates the second title. It will be ignored for other modes.
\IEEEpeerreviewmaketitle



\section{Introduction}
\IEEEPARstart{M}{ultisensor} multitarget tracking (MSMTT) is the problem of estimating the states of targets based on information provided by multiple sensors \cite{bar1990multitarget}. It originated in the military field \cite{Liggins2008_Handbook}, and now it is being applied to many nonmilitary fields. The applications of MSMTT include: surveillance \cite{blackman1999design}, autonomous driving \cite{urmson2008autonomous,patole2017automotive}, indoor
localization \cite{bartoletti2014sensor,shen2014estimating}, biomedical analytics \cite{mavska2014benchmark}, computer vision \cite{hoseinnezhad2012visual}, and robotics \cite{mullane2011random} etc.
There are three classical and popular frameworks for multitarget tracking \cite{Vo2015Multitarget}: joint
probabilistic data association (JPDA) \cite{BarShalom2009Theprobabilistic}, multiple hypotheses tracking (MHT) \cite{Reid1979An}, and random finite sets (RFS) \cite{mahler2007statistical}.

In this paper, we consider multisensor track-to-track fusion \cite{chong2000architectures}, where there is a fusion center. The sensor data are processed locally to
form sensor tracks, which are sent to the fusion center where they are fused to form system tracks. Track fusion is needed to associate the sensor tracks and generate an improved target state estimate. Therefore, multisensor track-to-track fusion for target tracking involves two main operations: estimation fusion and track association \cite{Liggins1997Distributed}.


Estimation fusion, or data fusion for estimation, is the problem of how to best utilize the information contained in multiple sets of data for the purpose of estimating a quantity--a parameter or process \cite{Li2003_Optimal}. Estimation fusion has been researched extensively and numerous results are available. Two general approaches have been used. One is the estimation approach that converts an estimation fusion problem to an estimation problem by treating available data from sensors as measurements \cite{Li2003_Optimal,bar1986effect,chang1997optimal,chen2003performance,chang2004performance,tian2009exact,chen2018new}. In \cite{bar1986effect,chang1997optimal,chen2003performance}, and the authors have proposed track-to-track fusion algorithms based on the maximum likelihood estimation (MLE)  and weighted least-squares (WLS) methods, respectively. In \cite{chang2004performance}, the authors proposed an optimal fusion algorithm based on the maximum a posteriori (MAP) formalism. In \cite{tian2009exact}, an estimation fusion algorithm was proposed, which is optimal in the sense of minimum mean-squared error (MMSE) for the Gaussian case. In \cite{Li2003_Optimal}, unified fusion rules were proposed in the sense of best linear unbiased estimate (BLUE) and WLS for all fusion architectures with arbitrary correlation. In \cite{chen2018new}, the authors proposed a  new approach to the nonlinear  fusion estimation problem. The other  estimation fusion approach is based on showing the equivalence between estimation fusion and centralized fusion \cite{Chong1979_Hierarchical,hashemipour1988decentralized,zhu2001optimality,zhu2012networked,sun2020distributed,2011Li_distributed}. The estimation fusion approaches are optimal in the sense that they are equivalent to the optimal centralized fusion. An optimal information filter fusion was proposed and discussed in \cite{Chong1979_Hierarchical,hashemipour1988decentralized}. The performance analysis for their fusion algorithm with feedback was given \cite{zhu2001optimality,zhu2012networked}. In \cite{sun2020distributed}, the author proposed  optimal linear fusion predictors and filters for systems with random parameter matrices and correlated noises. In \cite{2011Li_distributed}, by taking linear transformation of raw measurements of each sensor, two optimal fusion algorithms are proposed. Compared with existing fusion algorithms, their communication requirements from each sensor to the fusion center are equal to or less than those of the centralized and most existing fusion algorithms.
%However, these algorithms are based on the assumption that the tracks from the different sensors have been well associated.

Track association refers to finding multiple tracks for the same target using different systems. Before the track state estimates can be fused, the sensor tracks have to be associated either with each other (sensor track to sensor track association) or with the system tracks (sensor track to system track association) \cite{chong2000architectures}. Track association consists of the two key steps: computing a table of association metrics and selecting the best association hypothesis, usually by some statistical algorithm or assignment algorithm. In \cite{singer1971computer}, the authors proposed the statistical algorithm by using the weighted distance test method \cite{kanyuck1970correlation}, which used the Chi-squared distribution to detect whether the two estimates belong to the same target. Bar-Shalom \cite{bar1995multitarget} provided a method to test for the weighted distance under relevant conditions that work by introducing two estimated covariance matrix cross terms. Later, in \cite{BarShalom1997Ageneralized}, the authors proposed a generalized S-D assignment algorithm for multisensor-multitarget state estimation. In \cite{Sathyan2011MDA}, the authors introduced a multi-dimensional assignment (MDA)-based data association approach with prior track information for passive multitarget tracking.
%Pranay Sharma et al. [10] introduced a decentralized fusion method for an unknown and time-varying number of targets under association uncertainty [20].
The optimal track association problem is an MDA problem. The MDA problem is NP-hard, and the optimal solution can only be obtained by a global search. The classical methods use the Lagrangian relaxation method \cite{BarShalom1992_Anewrelaxation,Poore97_LRA}, linear relaxation approach \cite{Storms2003An,Coraluppi2000_Allsource} and the sequential m-best algorithm to find the suboptimal solution of the problem, which achieve excellent results while limiting the computational cost \cite{popp2001m}. Recently, in \cite{williams2014approximate}, the authors proposed a novel scalable method for solving data association problems using belief propagation (BP) on a particular graphical model formulation. In \cite{meyer2017scalable}, the authors proposed a scalable algorithm for tracking an unknown number of targets using multiple sensors. In \cite{Win2018Message}, the authors summarized a recently proposed paradigm for scalable multitarget tracking based on the BP algorithm. Additionally, recent MSMTT algorithms based on RFS theory can be seen in \cite{fantacci2016scalable, Yi2017ditributedfusion,Vo2019Multisensor,gao2020multiobject,gostar2020centralized,van2021distributed} etc. %li2020arithmetic
%Mahler2015distributed,nannuru2016multisensor,saucan2017multisensor
%To the best of our knowledge, these works do not consider track measurement transformation for reducing communication. Simultaneously, verify that the performance of the fusion with track measurement transformation is the same as that of the centralized fusion.
%In this paper, inspired by \cite{2011Li_distributed}, we discuss equivalency of linear transformation of track measurement for multisensor track association in terms of track-to-track association likelihoods.

To the best of our knowledge, there are many estimation fusion results for {\color{blue}single target tracking} in different scenarios (see, e.g., \cite{Chong1979_Hierarchical,hashemipour1988decentralized,zhu2001optimality,zhu2012networked,sun2020distributed}), which are optimal in the sense that they are equivalent to optimal centralized fusion with raw measurements. However, {\color{blue}the other main operation of multisensor track-to-track fusion, namely track association, which is a fundamental and important problem for multitarget tracking, the equivalence of measurement transformations has not been explored to the best of our knowledge. To solve this problem of track association, one usually needs to deal with difficult optimization problems, and it is unclear whether some measurement transformations can be selected that reduce communication bandwidths. In this paper, our goal is to analyze their equivalency in terms of track-to-track association likelihoods and reduce communication requirements by using suitable measurement transformations.
Our main contributions are as follows. 
\begin{itemize}
	\item {\it We demonstrate that the MDA-based track association method with linear transformations of track measurements is equivalent to that with raw track measurements.} The fundamental problem of multisensor track-to-track fusion for multitarget tracking is the data association problem of assigning the sensor track measurements to targets or clutter. If one knows which measurement originates from which target, then estimation techniques such as optimal information filter fusion or transformed measurement-based estimation \cite{2011Li_distributed} can be utilized to determine accurate state estimates. Since the optimal track association problem is an MDA problem, we demonstrate that MDA-based data association (with and without prior track information) using linear transformations of track measurements is lossless, i.e., it is equivalent to that using raw track measurements.
	\item {\it We derive a BP-based track association method via lossless linear transformations of track measurements.} Superior scalability and performance of BP algorithms enable new real-time applications of multitarget tracking with resource-limited devices \cite{Win2018Message}. This approach has advantages with respect to estimation accuracy, computational complexity, and implementation flexibility. Thus, we present a BP-based multisensor track association method using measurement transformations and show that they are equivalent to that using raw measurements. 
	\item {\it We provide two analytical lossless transformations for track association and analyze their communication requirements.} Considering communication constraints, it is more beneficial for local sensors to send compressed data with the goal of reducing the communication requirements. Two analytical lossless transformations for track association are provided, so that communication requirements from each sensor to the fusion center are less than that of fusion with raw track measurements. Therefore, communication-efficient transformations are suggested for different dynamic systems.
\end{itemize}
}

% for the other main operation of multisensor track-to-track fusion, namely track association, it is still theoretically unclear whether the  track association algorithm based on transformed measurements is equivalent to the track association algorithm based on raw measurements.
% %Note that \cite{2011Li_distributed} provided two lossless linear transformations of sensor measurement for estimation fusion, where their communication requirements from each sensor to the fusion center are equal to or less than those of the centralized.
% %However, when both the track association and estimation fusion problems are solved, it is still theoretically unclear whether the distributed track-to-track fusion  is equivalent to the centralized track-to-track fusion. In this paper,
% %inspired by \cite{2011Li_distributed},
% In this paper, our goal is to analyze their equivalency in terms of track-to-track association likelihoods and reduce communication requirements by using suitable measurement transformations.

% {\color{blue}
% Our main contributions in this paper are summarized as follows:
% \begin{itemize}
% \item Without the knowledge of the transformation matrix, we prove that the MDA track association with linear transformations of the track measurements is lossless, equivalent to that with raw track measurements. An extension to the case of MDA track association without prior track information is also analyzed.

% \item A BP track association approach is also analyzed to improve the scalability, which reduces the computational complexity to linearly in the number of sensors.   

% \item Three lossless transformations for track association are provided, and communication requirements from each sensor to the fusion center can be less than those of fusion with raw track measurements.
% \end{itemize}}

Numerical examples for tracking an unknown number of targets verify that the performance of fusion with transformed measurements is the same as that of fusion with raw measurements.

The rest of this paper is organized as follows. In Section \ref{sec_PF}, we formulate the problem of multisensor multitarget tracking and give preliminaries. In Sections \ref{sec_MDA} and \ref{sec_BF}, we derive the equivalence of measurement transformation and the use of raw measurements for
multisensor track association, where MDA and BP-based track association are analyzed respectively. Communication requirements are discussed in Section \ref{sec_CR}.  In Section \ref{sec_SR}, numerical examples are given. In Section \ref{sec_cl}, concluding remarks are provided.

\section{Problem Formulation}\label{sec_PF}
Let us consider sensor measurement transformations for multisensor track-to-track data association approaches:
1) MDA-based data association with or without prior track information \cite{Sathyan2011MDA,BarShalom1997Ageneralized}; 2)  a scalable BP-based data association \cite{meyer2017scalable,Win2018Message}.
We give some preliminaries as follows:
\subsection{Dynamic system}\label{Sec_2}
The linear dynamic model is given by
\begin{align}
\mathbf{x}_{k} = F_{k-1} \mathbf{x}_{k-1} + v_{k-1},
\end{align}
where $\mathbf{x}_{k} \in \mathbb{R}^{n}$ is the target state, $F_{k-1}\in \mathbb{R}^{n \times n}$ is the state transition matrix, and $v_{k-1}\in \mathbb{R}^{n}$ is the process noise, which is assumed Gaussian with zero mean and the covariance is $Q_{k-1} \in \mathbb{R}^{n \times n}$.

The measurement model is
\begin{align}\label{measure_model_1}
\mathbf{z}_{k} = H_{k}\mathbf{x}_{k} + w_{k},
\end{align}
where $H_k \in \mathbb{R}^{m\times n}$ is the measurement matrix, $\mathbf{z}_{k} \in \mathbb{R}^{m}$ and $w_{k}\in \mathbb{R}^{m}$ are the measurement vector and the measurement noise, respectively. Usually, the measurement noise is assumed Gaussian, where the covariance is $R_k \in \mathbb{R}^{m \times m}$.


\subsection{Sensor Measurement Transformation}
Let us consider a linear transformation of measurements for multisensor fusion. Compared with sending the raw measurement, each sensor sending the transformed data to the fusion center may potentially reduce communication requirements and  obtain lossless estimation \cite{2011Li_distributed}. Let $\breve{\mathbf{z}}_k := A_{k}\mathbf{z}_{k}$, 
% \begin{align}\label{measure_trans2_0}
% &\breve{\mathbf{z}}_k := A_{k}\mathbf{z}_{k},
% \end{align}
where $A_{k}$ is a linear transformation matrix. By using (\ref{measure_model_1}), we have
%fully column rank
\begin{align}\label{measure_type2}
\breve{\mathbf{z}}_k = \breve{H}_k\mathbf{x}_k + \breve{w}_k,
\end{align}
where
\begin{align}
&\breve{H}_k = A_{k}H_k, ~\breve{w}_{k}=A_{k}w_{k}, \\
& \breve{R}_{k}:=\mathnormal{Cov}(\breve{w}_k) = A_{k}R_k(A_{k})^{\text{T}}, \label{tran1_R_0}
\end{align}
and where $(\cdot)^{\text{T}}$ is the transpose of a matrix.

% For convenience, we use a single unified observation model for multisensor fusion. Let $\mathbf{z}_{k}$ denote the data that the local sensor sends to the fusion center,
% \begin{equation*}
% \setlength{\nulldelimiterspace}{0pt}
% \mathbf{z}_{k} = \left\{
% \begin{IEEEeqnarraybox}[][c]{ls}
% \mathbf{z}_{k},\quad &raw measurement, \\
% \breve{\mathbf{z}}_{k},\quad  &transformed measurement.%
% \end{IEEEeqnarraybox}\right.
% \end{equation*}
% The corresponding observation matrix and the noise term are
% \begin{equation*}
% \setlength{\nulldelimiterspace}{0pt}
% h_{k} = \left\{
% \begin{IEEEeqnarraybox}[][c]{ls}
% H_{k},\quad  &raw measurement, \\
% \breve{H}_{k},\quad  &transformed measurement,%
% \end{IEEEeqnarraybox}\right.
% \end{equation*}
% and
% \begin{equation*}
% \setlength{\nulldelimiterspace}{0pt}
% w_{k} = \left\{
% \begin{IEEEeqnarraybox}[][c]{ls}
% \eta_{k},\quad  &raw measurement, \\
% \breve{\eta}_{k},\quad  &transformed measurement,%
% \end{IEEEeqnarraybox}\right.
% \end{equation*}
% respectively. Therefore, the unified observation model is
% \begin{align*}
% \mathbf{z}_{k} = h_{k} \mathbf{x}_{k} + w_{k}.
% \end{align*}

\subsection{Multisensor Multitarget Tracking}
In this subsection, we present the track association problem of multisensor multitarget tracking. {\color{blue}Let us consider a scenario, where $L$ heterogeneous sensors monitor a surveillance region of interest and send measurements at each time step to a fusion center. We assume that each sensor is equipped with a local computing unit that can compute local multitarget state estimates, and a transceiver that can transmit data to the fusion center. In this paper, we consider the case where each sensor tracks an unknown and time-varying number of targets by performing data association and track management, and transmits the data of updated tracks to the fusion center. The data of updated tracks sent by the local sensor can be raw track measurements or the estimates of targets, or the transformed track measurements. 

At time $k$, let $\mathbf{Z}_{k,l}= [(\mathbf{z}_{k,l}^{(1)})^{\text{T}},\cdots,(\mathbf{z}_{k,l}^{(M_{k,l})})^{\text{T}}]^{\text{T}}$ denote the track measurements sent by the $l$-th sensor, where $l=1,\cdots,L$ and $M_{k,l}$ is the number of measurements. On the other hand, let $\breve{\mathbf{Z}}_{k,l}= [(\breve{\mathbf{z}}_{k,l}^{(1)})^{\text{T}},\cdots,(\breve{\mathbf{z}}_{k,l}^{(M_{k,l})})^{\text{T}}]^{\text{T}}$ denote the transformed track measurements sent by the $l$-th sensor at time $k$. We define the data at the fusion center at time $k$ as $\mathbf{Z}_{k}=[\mathbf{Z}_{k,1}^{\text{T}},\cdots,\mathbf{Z}_{k,L}^{\text{T}}]^{\text{T}}$ or $\breve{\mathbf{Z}}_{k}=[\breve{\mathbf{Z}}_{k,1}^{\text{T}},\cdots,\breve{\mathbf{Z}}_{k,L}^{\text{T}}]^{\text{T}}$. Let $N_k$ denote the number of targets (possibly unknown) presented at time $k$. We use the target state to identify a target, i.e., the target $\tau$ at time $k$ is identified by its state $\mathbf{x}_{k}^{(\tau)} \in \mathbb{R}^{n}$. The stacked state at time $k$ is then denoted by $\mathbf{X}_k = [(\mathbf{x}_{k}^{(1)})^{\text{T}},\cdots,(\mathbf{x}_{k}^{(N_k)})^{\text{T}} ]^{\text{T}}$. 

Let the $(L+1)$-tuple $(\tau,i_1,\cdots,i_L)$ denote a track hypothesis formed by target $\tau$, and measurement $\mathbf{z}_{k,l}^{(i_l)}$ of the $l$-th sensor ($l=1,\cdots,L$), i.e., measurements $\mathbf{z}_{k,l}^{(i_l)}$ ($l=1,\cdots,L$) are originated from the same target $\tau$. Note that $\tau \in \{0,1,\cdots,N_{k}\}$ and $i_l \in \{0,1,\cdots,M_{k,l}\}$ in the $(L+1)$-tuple $(\tau,i_1,\cdots,i_L)$. If $\tau= 0$, the $(L+1)$-tuple means that measurements $\mathbf{z}_{k,l}^{(i_l)}$ ($l=1,\cdots,L$) are originated from clutter. If $\tau \neq 0$ and $i_l = 0$, the $(L+1)$-tuple means that target $\tau$ generates no measurements at the $l$-th sensor. Moreover, with the commonly used assumption that a target can generate at most one measurement, and a measurement can originate from at most one target, the track association problem can be formulated as an MDA problem \cite{bar1990multitarget,Liggins2008_Handbook,blackman1999design}. Before we formulate the MDA problem, the score function of the track hypothesis is given in the following subsection.}

% \subsection{Score Function}\label{sec_scorefn}
% For each track hypothesis $(\tau,i_{1},\cdots,i_{L})$, its score function $L_{(\tau,i_{1},\cdots,i_{L})}$ can be defined as a likelihood ratio \cite{BarShalom07_dimen_TAES}. Let $\mathbf{Z}_{i_1 \cdots i_L}$ denote the $L$-tuple data at time $k$ ($k$ is omitted here for simplicity) that resulted in the track hypothesis $(\tau,i_{1},\cdots,i_{L})$, the score function defined by the likelihood ratio is expressed as
% \begin{align}\label{cost_local}
% 	L_{(\tau,i_{1},\cdots,i_{L})} = \frac{\Lambda(\mathbf{Z}_{i_1 \cdots i_L}|(\tau,i_{1},\cdots,i_{L}))}{\Lambda(\mathbf{Z}_{i_1 \cdots i_L}|\emptyset)},
% \end{align}
% where the term in the numerator is the likelihood of $\mathbf{Z}_{i_1 \cdots i_L}$ given the track hypothesis $(\tau,i_{1},\cdots,i_{L})$, and the term in the denominator is the likelihood that the data $\mathbf{Z}_{i_1 \cdots i_L}$ is spurious data. When $\tau \neq 0$, with the known $\mathbf{x}_{k}^{(\tau)}$, the likelihood in the numerator is (cf., \cite{BarShalom1997Ageneralized})
% \begin{align}
% 	&\Lambda(\mathbf{Z}_{i_1 \cdots i_L}|(\tau,i_1,\cdots,i_L)) \nonumber\\
% 	& \quad = \prod_{l=1}^L\left\{
% 	[1-P_d^{(l)}]^{1-u(i_l)}[P_d^{(l)} p(\mathbf{z}_{k,l}^{(i_l)}|\mathbf{x}_{k}^{(\tau)})]^{u(i_l)}
% 	\right\},
% \end{align}
% where $u(i_l) = 0$ if $i_l=0$, or $u(i_l) = 1$ otherwise, and $P_d^{(l)}$ is the probability of detection of the $l$-th sensor. The likelihood in the denominator of (\ref{cost_local}) is
% {\color{blue}\begin{align}
% \Lambda(\mathbf{Z}_{i_1 \cdots i_L}|\emptyset) =  \prod_{l=1}^L [\lambda_{f_l} p_{f_{l}}(\mathbf{z}_{k,l}^{(i_l)})]^{u(i_l)}. 
% \end{align}}
% where $\lambda_{f_l}$ is the mean number of clutter, and $p_{f_{l}}(\mathbf{z}_{k,l}^{(i_l)})$ is the {\it{probability density function}} (pdf) of the clutter measurement.
% Thus, the score function (\ref{cost_local}) is defined as
% \begin{align}\label{score_function}
% 	L_{(\tau,i_{1},\cdots,i_{L})} = \prod_{l=1}^L \frac{[1-P_d^{(l)}]^{1-u(i_l)}[P_d^{(l)} p(\mathbf{z}_{k,l}^{(i_l)}|\mathbf{x}_{k}^{(\tau)})]^{u(i_l)}}{[\lambda_{f_l} p_{f_{l}}(\mathbf{z}_{k,l}^{(i_l)})]^{u(i_{l})}}.
% \end{align}
% {\color{blue}On the other hand, when $\tau= 0$, the track hypothesis $(0, i_1,\cdots,i_L)$ represents the fact that the measurements with indices $(i_1,\cdots,i_L)$ are clutter measurements. Then, 
% \begin{align}
% 	\Lambda(\mathbf{Z}_{i_1 \cdots i_L}|(0,i_1,\cdots,i_L))
% 	=  \prod_{l=1}^L [\lambda_{f_l} p_{f_{l}}(\mathbf{z}_{k,l}^{(i_l)})]^{u(i_l)},
% \end{align}
% and we have
% \begin{align}
% 	L_{(0,i_{1},\cdots,i_{L})} = \frac{\Lambda(\mathbf{Z}_{i_1 \cdots i_L}|(0,i_1,\cdots,i_L))}{\Lambda(\mathbf{Z}_{i_1 \cdots i_L}|\emptyset)}= 1.
% \end{align}}

In this paper, our goal is to analyze the equivalency  between track association with raw measurements and track association with transformed measurements in terms of the association score function. Moreover, an extension to the case of track association without prior track information, and a BP-based track association way to reduce the computational complexity are also discussed.



\section{Establishment of Equivalence for MDA Track Association}\label{sec_MDA}
In this section, we present the main results on equivalence between the  track association with raw measurements and track association with transformed measurements based on MDA with and without prior track information, respectively.

\subsection{MDA Track Association with Prior Track Information }\label{sec_2_A}
At time $k$, assuming that the fusion center has maintained $N_k$ tracks and received data $\mathbf{Z}_k$ from the $L$ sensors. Let $\hat{\mathbf{x}}_{k-1|k-1}^{(\tau)}$ and $P_{k-1|k-1}^{(\tau)}$ denote the mean and the covariance of the state estimate of the track $\tau$ at time $k-1$, respectively, the prior track information at time $k$ can be represented by $\hat{\mathbf{x}}_{k|k-1}^{(\tau)}$ and $P_{k|k-1}^{(\tau)}$, where 
\begin{align}
	&\hat{\mathbf{x}}_{k|k-1}^{(\tau)} = F_{k-1} \hat{\mathbf{x}}_{k-1|k-1}^{(\tau)}, \label{MDA_pred_x} \\
	&P_{k|k-1}^{(\tau)} = F_{k-1} P_{k-1|k-1}^{(\tau)} F_{k-1}^{\text{T}} + Q_{k-1}. \label{MDA_pred_p}
\end{align}
For each track hypothesis $(\tau,i_{1},\cdots,i_{L})$, when $\tau \neq 0$, the score function $L_{(\tau,i_{1},\cdots,i_{L})}$ can be defined as a likelihood ratio~\cite{Sathyan2011MDA}, i.e., 
\begin{align}\label{score_function_prior}
	L_{(\tau,i_{1},\cdots,i_{L})} = \prod_{l=1}^L \frac{[1-P_d^{(l)}]^{1-u(i_l)}[P_d^{(l)} p(\mathbf{z}_{k,l}^{(i_l)}|\hat{\mathbf{x}}_{k|k-1}^{(\tau)})]^{u(i_l)}}{[\lambda_{f_l} p_{f_{l}}(\mathbf{z}_{k,l}^{(i_l)})]^{u(i_{l})}},
\end{align}
where $u(i_l) = 0$ if $i_l=0$, or $u(i_l) = 1$ otherwise. 
In the numerator of (\ref{score_function_prior}), $P_d^{(l)}$ is the probability of detection of the $l$-th sensor, and $p(\mathbf{z}_{k,l}^{(i_l)}|\hat{\mathbf{x}}_{k|k-1}^{(\tau)})$ is the likelihood that $\mathbf{z}_{k,l}^{(i_l)}$ originates from target $\tau$; in the denominator of (\ref{score_function_prior}), $\lambda_{f_l}$ is the mean number of clutter, and $p_{f_{l}}(\mathbf{z}_{k,l}^{(i_l)})$ is the {\it{probability density function}} (pdf) of the clutter measurement. {\color{blue}On the other hand, when $\tau= 0$, the track hypothesis $(0, i_1,\cdots,i_L)$ represents the fact that the measurements with indices $(i_1,\cdots,i_L)$ are clutter measurements, i.e., 
\begin{align}
	L_{(0,i_{1},\cdots,i_{L})} = \frac{\prod_{l=1}^L [\lambda_{f_l} p_{f_{l}}(\mathbf{z}_{k,l}^{(i_l)})]^{u(i_l)}}{\prod_{l=1}^L [\lambda_{f_l} p_{f_{l}}(\mathbf{z}_{k,l}^{(i_l)})]^{u(i_l)}}= 1.
\end{align}}

For fusion with raw measurements, the corresponding likelihood is
\begin{align}\label{likeli_centralized}
p(\mathbf{z}_{k,l}^{(i_l)} | \hat{\mathbf{x}}_{k|k-1}^{(\tau)}) =& \frac{(2\pi)^{-m/2}}{(|S_{k,l}^{(\tau)}|)^{1/2}} \exp \Bigl( -\frac{1}{2} (\mathbf{z}_{k,l}^{(i_l)}- \hat{\mathbf{z}}_{k|k-1,l}^{(\tau)})^{\text{T}} \nonumber \\
& \times  (S_{k,l}^{(\tau)})^{-1}(\mathbf{z}_{k,l}^{(i_l)}- \hat{\mathbf{z}}_{k|k-1,l}^{(\tau)}) \Bigr),
\end{align}
where $m$ is the dimension of the measurement, and the predicted measurement $\hat{\mathbf{z}}_{k|k-1,l}^{(\tau)}$ and the innovation covariance matrix $S_{k,l}^{(\tau)}$ are obtained as follows,
\begin{align}
&\hat{\mathbf{z}}_{k|k-1,l}^{(\tau)} = H_{k,l} \hat{\mathbf{x}}_{k|k-1}^{(\tau)}, \label{MDA_pred_z}\\
&S_{k,l}^{(\tau)} = H_{k,l} P_{k|k-1}^{(\tau)} H_{k,l}^{\text{T}} + R_{k,l}. \label{MDA_pred_s}
\end{align}
On the other hand, for fusion with transformed measurements, $\mathbf{z}_{k,l}^{(i_l)}$ in (\ref{score_function_prior}) is replaced by $\breve{\mathbf{z}}_{k,l}^{(i_l)}$, and a generalized likelihood \cite{Rao1973Linear} is applied:
\begin{align}\label{likeli_distributed}
p(\breve{\mathbf{z}}_{k,l}^{(i_l)} | \hat{\mathbf{x}}_{k|k-1}^{(\tau)}) =& \frac{(2\pi)^{-m/2}}{(\prod_{i=1}^{m}e_i)^{1/2} } \exp \Bigl(-\frac{1}{2} (\breve{\mathbf{z}}_{k,l}^{(i_l)}- \breve{\mathbf{z}}_{k|k-1,l}^{(\tau)})^{\text{T}} \nonumber\\
& \times  (\breve{S}_{k,l}^{(\tau)})^{\dagger} (\breve{\mathbf{z}}_{k,l}^{(i_l)}- \breve{\mathbf{z}}_{k|k-1,l}^{(\tau)}) \Bigr),%
\end{align}
where $\breve{\mathbf{z}}_{k|k-1,l}^{(\tau)} = A_{k,l}\hat{\mathbf{z}}_{k|k-1,l}^{(\tau)}$ and $\breve{S}_{k,l}^{(\tau)} =A_{k,l} S_{k,l}^{(\tau)} A_{k,l}^{\text{T}}$; $A_{k,l}$ represents the linear transformation of the $l$-th sensor at time $k$; $(\cdot)^{\dagger}$ is the Moore-Penrose pseudo inverse; $e_{i}$, $ i=1,\cdots,m$ are the nonzero eigenvalues of $\breve{S}_{k,l}^{(\tau)} $. 

By defining the cost of each track hypothesis $(\tau,i_1,\cdots,i_L)$: $C_{(\tau,i_1,\cdots,i_L)}=-\log L_{(\tau,i_1,\cdots,i_L)}$, 
% \begin{align*}
% C_{(\tau,i_1,\cdots,i_L)}=-\log L_{(\tau,i_1,\cdots,i_L)},
% \end{align*}
the track-measurement association problem between the existing tracks and the measurements of the $L$ sensors is formulated as an MDA problem~\cite{Sathyan2011MDA}:
\begin{align}\label{LIP}
&\min_{\delta_{(\tau,i_1,\cdots,i_L)}}~\sum_{\tau}\sum_{\bm{i}}C_{(\tau,i_1,\cdots,i_L)}\delta_{(\tau,i_1,\cdots,i_L)} \notag
\\
&\text{s.t.} \nonumber \\
&\sum_{\bm{i}}\delta_{(\tau,i_1,\cdots,i_L)}=1,~\tau=1,\dots,N_{k}, \notag
\\
&\sum_{\tau}\sum_{\bm{i}\setminus \{i_l\}}\delta_{(\tau,i_1,\cdots,i_L)}=1,\notag
\\
&\text{for }~i_{l}=1,\dots,M_{k,l} ~\text{ and }~ l=1,\dots,L, \notag
\\
&\delta_{(\tau,i_1,\cdots,i_L)}\in \{0,1\}~ \text{ for all } ~(\tau,i_1,\cdots,i_L).
\end{align}
Note that $\delta_{(\tau,i_1,\cdots,i_L)} = 1$ means that the measurements with indices $(i_1,\cdots,i_L)$ originate from the track $\tau$. The constraints in (\ref{LIP}) mean that each measurement from the local sensors can only correspond to one track, and a track can only match with one measurement at the same sensor.

Here, we derive the relationship between data association with transformed measurements and that with raw measurements under some regularity conditions, which is summarized in the following Proposition \ref{equivalence_score}.

\begin{Proposition}\label{equivalence_score}
	Let $A_{k,l}$ be a full column rank matrix, the processed data $\breve{\mathbf{z}}_{k,l}^{(i_{l})}$ be a linear transformation of $\mathbf{z}_{k,l}^{(i_{l})}$, i.e., $\breve{\mathbf{z}}_{k,l}^{(i_{l})} = A_{k,l} \mathbf{z}_{k,l}^{(i_{l})}$, and the clutter is uniform in the region of interest. Then, the score function of the track association problem with raw measurements is equal to that of the problem with transformed measurements, i.e., $L_{(\tau,i_1,\cdots,i_L)}^c = L_{(\tau,i_1,\cdots,i_L)}^d$, where
	\begin{align}\label{score_centralized}
	L_{(\tau,i_1,\cdots,i_L)}^c = \prod_{l=1}^L \frac{[1-P_d^{(l)}]^{1-u(i_l)}[P_d^{(l)} p(\mathbf{z}_{k,l}^{(i_{l})}|\hat{\mathbf{x}}_{k|k-1}^{(\tau)})]^{u(i_l)}}{[\lambda_{f_l} p_{f_{l}}(\mathbf{z}_{k,l}^{(i_{l})})]^{u(i_{l})}},
	\end{align}
	and
	\begin{align}\label{score_distributed}
	L_{(\tau,i_1,\cdots,i_L)}^d = \prod_{l=1}^L \frac{[1-P_d^{(l)}]^{1-u(i_l)}[P_d^{(l)} p(\breve{\mathbf{z}}_{k,l}^{(i_{l})}|\hat{\mathbf{x}}_{k|k-1}^{(\tau)})]^{u(i_l)}}{[\lambda_{f_l} p_{f_{l}}(\breve{\mathbf{z}}_{k,l}^{(i_{l})})]^{u(i_{l})}}.
	\end{align}
	Moreover, the MDA  problem (\ref{LIP}) for track association with raw measurements is equivalent to that with transformed measurements.		
\end{Proposition}

The proof of Proposition \ref{equivalence_score} is in \cite[app A]{Liu2023Oncommunication}. The assumption that $A_{k,l}$ is full column rank is not too stringent. We will provide some transformation matrices with full column rank which can reduce communication requirements in Section \ref{sec_CR}.


\subsection{MDA Track Association without Prior Track Information} \label{sen_2_b}

In this subsection, we formulate MDA track association without prior track information. There are some scenarios where the prior track information at the fusion center is not available, such as sensor track to sensor track fusion (but not to system track fusion), track initialization \cite{BarShalom1997Ageneralized,chong2000architectures}, etc. In this case, the score function for each hypothesis $(i_1,\cdots,i_L)$ is defined as the following generalized likelihood ratio \cite{BarShalom1997Ageneralized}, i.e., 
\begin{align}\label{score_function}
	L_{(i_{1},\cdots,i_{L})} = \prod_{l=1}^L \frac{[1-P_d^{(l)}]^{1-u(i_l)}[P_d^{(l)} p(\mathbf{z}_{k,l}^{(i_l)}|\hat{\mathbf{x}}_{k,\text{ML}}^{(\tau)})]^{u(i_l)}}{[\lambda_{f_l} p_{f_{l}}(\mathbf{z}_{k,l}^{(i_l)})]^{u(i_{l})}},
\end{align}
where $\hat{\mathbf{x}}_{k,\text{ML}}^{(\tau)}$ is the MLE of the target state $\mathbf{x}_{k}^{(\tau)}$: 
\begin{align}\label{Ml_EST}
	\hat{\mathbf{x}}_{k,\text{ML}}^{(\tau)}=\argmax_{\mathbf{x}_{k}^{(\tau)}} \prod_{l \in \{l|u(i_{l})=1\}}p(\mathbf{z}_{k,l}^{(i_{l})}|\mathbf{x}_{k}^{(\tau)}).
\end{align} 

For fusion with raw measurements, the conditional pdf $p(\mathbf{z}_{k,l}^{(i_l)}|\hat{\mathbf{x}}_{k,\text{ML}}^{(\tau)})$ has the form as follows,
\begin{align}\label{likeli_meas_init}
&p(\mathbf{z}_{k,l}^{(i_{l})}|\hat{\mathbf{x}}_{k,\text{ML}}^{(\tau)}) = \frac{(2\pi)^{-m/2}}{(|R_{k,l}|)^{1/2}}\notag\\
& \times \exp \Bigl(-\frac{1}{2}(\mathbf{z}_{k,l}^{(i_{l})}- H_{k,l}\hat{\mathbf{x}}_{k,\text{ML}}^{(\tau)})^{\text{T}} R_{k,l}^{-1}(\mathbf{z}_{k,l}^{(i_{l})}- H_{k,l}\hat{\mathbf{x}}_{k,\text{ML}}^{(\tau)}) \Bigr).
\end{align}
On the other hand, for fusion with transformed measurements, the conditional pdf of the likelihood of transformed measurements is,
\begin{align}
&p(\breve{\mathbf{z}}_{k,l}^{(i_{l})}|\hat{\mathbf{x}}_{k,\text{ML}}^{(\tau)}) = \frac{(2\pi)^{-m/2}}{(\prod_{i=1}^{m}e_i)^{1/2} } \nonumber\\
& \times \exp \Bigl(-\frac{1}{2} [\breve{\mathbf{z}}_{k,l}^{(i_{l})}-f_l(\hat{\mathbf{x}}_{k,\text{ML}}^{(\tau)}) ]^{\text{T}} [g_l(R_{k,l}) ]^{\dagger} [\breve{\mathbf{z}}_{k,l}^{(i_{l})}-f_l(\hat{\mathbf{x}}_{k,\text{ML}}^{(\tau)}) ] \Bigr),%
\end{align}
where the linear transformation $f_l(\hat{\mathbf{x}}_{k,\text{ML}}^{(\tau)}) =A_{k,l}H_{k,l}\hat{\mathbf{x}}_{k,\text{ML}}^{(\tau)}$, and the transformed covariance $g_l(R_{k,l}) =A_{k,l}R_{k,l}^{-1} A_{k,l}^{\text{T}}$.
% \begin{align*}
% 	f_l(\hat{\mathbf{x}}_{k,\text{ML}}^{(\tau)}) &=A_{k,l}H_{k,l}\hat{\mathbf{x}}_{k,\text{ML}}^{(\tau)}\\
% 	g_l(R_{k,l}) &=A_{k,l}R_{k,l}^{-1} A_{k,l}^{\text{T}}.
% \end{align*}	

Define the cost of the candidate association $(i_1,\cdots,i_L)$ as $C_{(i_1,\cdots,i_L)} = -\log L_{(i_{1},\cdots,i_{L})}$, the problem of track association without prior track information can be formulated as an $L$-D assignment problem:
\begin{align}\label{LIP_noinit}
	&\min_{\delta_{(i_1,\cdots,i_L)}}~\sum_{\bm{i}}C_{(i_1,\cdots,i_L)}\delta_{(i_1,\cdots,i_L)} \notag
	\\
	&\text{s.t.}\notag
	\\
	&\sum_{\bm{i} \setminus \{i_l\}}\delta_{(i_1,\cdots,i_L)}=1,\notag
	\\
	&\text{for }~i_{l}=1,\dots,M_{k,l} ~\text{ and } l=1,\dots,L, \notag
	\\
	&\delta_{(i_1,\cdots,i_L)}\in \{0,1\}~ \text{ for all } (i_1,\cdots,i_L).
\end{align}
Note that $\delta_{(i_1,\cdots,i_L)} = 1$ means that the measurements with indices $(i_1,\cdots,i_L)$ originate from the same target, i.e., they can be used to initialized a new track. The constraints in (\ref{LIP_noinit}) mean that each measurement from the local sensors can only correspond to one target.
	
For MDA track association without prior track information, we have the following equivalency result.
\begin{Corollary}\label{corro_nopiror}
Under the conditions of Proposition \ref{equivalence_score}, the score function of fusion with raw measurements is equal to that of fusion with transformed measurements, i.e., $L_{(i_{1},\cdots,i_{L})}^{c}=L_{(i_{1},\cdots,i_{L})}^{d}$, 
% \begin{align*}
% 	L_{(i_{1},\cdots,i_{L})}^{c}=L_{(i_{1},\cdots,i_{L})}^{d},
% \end{align*}
where
\begin{align}\label{score_centralized_nopior}
	L_{(i_1,\cdots,i_L)}^c = \prod_{l=1}^L \frac{[1-P_d^{(l)}]^{1-u(i_l)}[P_d^{(l)} p(\mathbf{z}_{k,l}^{(i_{l})}|\hat{\mathbf{x}}_{k,\text{ML}}^{(\tau),c})]^{u(i_l)}}{[\lambda_{f_l} p_{f_{l}}(\mathbf{z}_{k,l}^{(i_{l})})]^{u(i_{l})}},
\end{align}
and
	\begin{align}\label{score_distributed_nopior}
	L_{(i_1,\cdots,i_L)}^d = \prod_{l=1}^L \frac{[1-P_d^{(l)}]^{1-u(i_l)}[P_d^{(l)} p(\breve{\mathbf{z}}_{k,l}^{(i_{l})}|\hat{\mathbf{x}}_{k,\text{ML}}^{(\tau),d})]^{u(i_l)}}{[\lambda_{f_l} p_{f_{l}}(\breve{\mathbf{z}}_{k,l}^{(i_{l})})]^{u(i_{l})}}.
	\end{align}
Moreover, the MDA problem (\ref{LIP_noinit})  for track association  with raw measurements is equivalent to that for track association  with transformed measurements.
\end{Corollary}

The proof of Corollary \ref{corro_nopiror} can be found in \cite[app B]{Liu2023Oncommunication}.


\subsection{Summary of MDA Track Association}\label{sec_c}
Combining MDA Track association method with prior information with the estimation fusion method in \cite{2011Li_distributed}, the complete multisensor track-to-track fusion algorithm with transformed measurements is summarized in \cite[Algorithm 1]{Liu2023Oncommunication}, which is equivalent to that with raw measurements.

\begin{algorithm}
	\caption{MDA Association and Fusion}\label{alg:MDA}
	\SetKwBlock{uBegin}{}{}
	\SetKwData{Left}{left}\SetKwData{This}{this}\SetKwData{Up}{up} 
	\SetKwFunction{Union}{Union}\SetKwFunction{FindCompress}{FindCompress} 
	\SetKwInOut{Input}{input}\SetKwInOut{Output}{output}

	\Input{$\{\hat{\mathbf{x}}_{k-1|k-1}^{(\tau)}, P_{k-1|k-1}^{(\tau)} \}_{\tau=1}^{N_{k-1}}$, $\breve{\mathbf{Z}}_{k,l}$, $\{H_{k,l}\}_{l=1}^L$, or $\{R_{k,l}\}_{l=1}^L$;}
	\Output{$\{\hat{\mathbf{x}}_{k|k}^{(\tau)}, P_{k|k}^{(\tau)}\}_{\tau=1}^{N_k}$;}
	%\BlankLine
	\emph{Prediction: }
	\uBegin{
		\emph{Calculate $\{\hat{\mathbf{x}}_{k|k-1}^{(\tau)}, P_{k|k-1}^{(\tau)} \}_{\tau=1}^{N_{k-1}}$ using (\ref{MDA_pred_x})--(\ref{MDA_pred_p})}\;
	}
	\emph{Track maintenance: }
	\uBegin{
		\emph{Establish the MDA problem (\ref{LIP})}\;
		\emph{Solve the MDA problem (\ref{LIP})}\;
		\emph{Track update $\{\hat{\mathbf{x}}_{k|k}^{(\tau)}, P_{k|k}^{(\tau)}\}_{\tau=1}^{N_{k-1}}$: Each track $\tau$ is updated by $\{\breve{\mathbf{z}}_{k,l}^{(i_l)}\}_{l=1}^{L}$ for $\delta_{(\tau,i_1,\cdots,i_L)}=1$ using the method in \cite{2011Li_distributed}}\;
		\emph{Delete $\{\breve{\mathbf{z}}_{k,l}^{(i_l)}\}_{l=1}^{L}$ in $\breve{\mathbf{Z}}_{k}$ for $\delta_{(\tau,i_1,\cdots,i_L)}=1$}\;
	}
	\emph{Newborn track initialization: }
	\uBegin{
		\emph{Establish the MDA problem (\ref{LIP_noinit})}\;
		\emph{Solve the MDA problem (\ref{LIP_noinit})}\;
		\emph{Initialize tracks $\{\hat{\mathbf{x}}_{k|k}^{(\tau')}, P_{k|k}^{(\tau')} \}_{\tau'=N_{k-1}+ 1}^{N_{k-1}+N'_k}$: Each newborn track $\tau'$ is initialized by $\{\breve{\mathbf{z}}_{k,l}^{(i_l)}\}_{l=1}^{L}$ for $\delta_{(i_1,\cdots,i_L)}=1$}\;
	}
	\emph{Output: }
	\uBegin{
		\emph{$N_k = N_{k-1} + N'_k$}\;
		\emph{$\{\hat{\mathbf{x}}_{k|k}^{(\tau)}\}_{\tau=1}^{N_{k}} = \left\{\{\hat{\mathbf{x}}_{k|k}^{(\tau)}\}_{\tau=1}^{N_{k-1}}, \{\hat{\mathbf{x}}_{k|k}^{(\tau')}\}_{\tau'=N_{k-1}+ 1}^{N_{k-1}+ N'_k}\right\}$}\;
		\emph{$\{P_{k|k}^{(\tau)}\}_{\tau=1}^{N_{k}} = \left\{\{P_{k|k}^{(\tau)}\}_{\tau=1}^{N_{k-1}}, \{P_{k|k}^{(\tau')}\}_{\tau'=N_{k-1}+ 1}^{N_{k-1}+ N'_k}\right\}$}\;
		\emph{Set $k=k+1$}.
	}
\end{algorithm}

The MDA track association with or without prior information requires the solution of an $(L+1)$-D or $L$-D assignment problem, respectively. The $L$-D assignment problem is an NP-hard problem for $L>2$. Although it is NP-hard, the original or primal $L$-D assignment problem can be relaxed, via successive constraint relaxation, to a two-dimensional (2-D) subproblem, which is optimally solvable at each iteration in polynomial time $\mathcal{O}(C N^{3})$, where $N$ is the number of tracks or sensor measurements, and $C$ is the range of the of the cost coefficient. Moreover, the worst case complexity of the relaxed $L$-D assignment algorithm in \cite{BarShalom1997Ageneralized} is  $\mathcal{O}((L-1)C N^{3})$.
% The solution to the $L$-D problem is equivalent to solve a series of $2$-D problem\cite{kiruba2011mda_based}. The computational complexity of the algorithm to solve the $2$-D assignment problem is in $\mathcal{O}((\sum_{l=1}^LN_{l})^{3})$ where $N_{l}$ is the number of measurements of $l$-th local sensor \cite{BarShalom1997Ageneralized}.
Furthermore, to reduce computation complexity, we consider the scalable track association using the BP method \cite{williams2014approximate,Win2018Message}.
%\textcolor[rgb]{1.00,0.00,0.00}{The computational complexity of the assignment problem scales exponentially in the number of sensors and number of targets.}



\section{Equivalence of BP-Based Track Association and Fusion}\label{sec_BF}
In this section, we analyze the equivalency between BP-based track association  with raw measurements and that with transformed measurements for tracking an unknown, time-varying number of targets \cite{williams2014approximate,Win2018Message}.

\subsection{The BP-based MSMTT algorithm}\label{facot_graph}
At time $k$, the fusion center receives data from $L$ sensors, which are sequentially processed for the $l$-th sensor, where $l=1,\cdots,L$. A target is either a newborn one or a target established in the past and survived to the present. The states of the potential targets for the $l$-th sensor are represented by
$\overline{\mathbf{X}}_{k,l} = [(\overline{\mathbf{x}}_{k,l}^{(1)})^{\text{T}},\cdots,(\overline{\mathbf{x}}_{k,l}^{(M_{k,l})})^{\text{T}}]^{\text{T}}$, where $M_{k,l}$ is the number of measurements. The states of the survived targets up to receiving the data of the $l$-th sensor at the fusion center are $\underline{\mathbf{X}}_{k,l} = [(\underline{\mathbf{x}}_{k,l}^{(1)})^{\text{T}},\cdots,(\underline{\mathbf{x}}_{k,l}^{(N_{k,l})})^{\text{T}}]^{\text{T}}$, where the number $N_{k,l}$ of the survived targets that are updated by using data of the $1$-st sensor to the $(l-1)$-th sensor. Meanwhile, the 0-1 variable $\overline{r}_{k,l}^{(i_{l})}=1$ represents that the measurement $i_{l}$ generated by a new target and $\underline{r}_{k,l}^{(\tau)}=1$ means that the survived target $\tau$ exists up to the $l$-th sensor at time $k$. We define that $\overline{r}_{k,l} = [\overline{r}_{k,l}^{(1)},\cdots,\overline{r}_{k,l}^{(M_{k,l})}]^{\text{T}}$ for the new targets, and $\underline{r}_{k,l} = [\underline{r}_{k,l}^{(1)},\cdots,\underline{r}_{k,l}^{(N_{k,l})}]^{\text{T}}$ for the survived targets. After $L$ iterations, the states of the survived targets at the fusion center are also denoted as $\tilde{f}(\mathbf{x}_{k}^{(\tau)},r_{k}^{(\tau)})$, $\tau= 1,\cdots,N_k$, where
\begin{equation}\label{bp_prior}
	\setlength{\nulldelimiterspace}{0pt}
	\tilde{f}(\mathbf{x}_{k}^{(\tau)},r_{k}^{(\tau)}) = \left\{
	\begin{IEEEeqnarraybox}[][c]{ls}
		\tilde{f}(\underline{\mathbf{x}}_{k,L}^{(\tau)},\underline{r}_{k,L}^{(\tau)}) ,\quad & $\tau \leq N_{k,L}$  \\
		\\
		\tilde{f}(\overline{\mathbf{x}}_{k,L}^{(i_L)},\overline{r}_{k,L}^{(i_L)}), & $i_{L}= \tau - N_{k,L} $,%
	\end{IEEEeqnarraybox}\right.
\end{equation}
and $N_k = N_{k,L}+M_{k,L}$.

Let $a_{k,l} = [a_{k,l}^{(1)},\cdots,a_{k,l}^{(N_{k,l})}]^{\text{T}}$ denote the (unknown) data association variable vector at time $k$, where $a_{k,l}^{(\tau)} = i_l \in \{ 1,\cdots,M_{k,l} \}$ if target $\tau$ generates a measurement $i_l$ at the $l$-th sensor and $a_{k,l}^{(\tau)} = 0 $ if target $\tau$ does not generate a measurement at the $l$-th sensor. On the other hand, an alternative association vector $b_{k,l} = [b_{k,l}^{(1)},\cdots,b_{k,l}^{(M_{k,l})}]^{\text{T}}$ is introduced for the $l$-th sensor, where $b_{k,l}^{(i_l)} = \tau \in \{ 1,\cdots,N_{k,l} \} $ if measurement $i_l$ originates from target $\tau$ and $b_{k,l}^{(i_l)} = 0 $ if measurement $i_l$ is a clutter measurement. 
% whose components have the following form, 
% \begin{equation*}
% 	\setlength{\nulldelimiterspace}{0pt}
% 	a_{k,l}^{(\tau)} := \left\{
% 	\begin{IEEEeqnarraybox}[][c]{ls}
% 	i_l \in \{ 1,\cdots,M_{k,l} \},\quad & if target $\tau$ generates a\\
% 	&\ measurement $i_l$ at sensor $l$;\\
% 	0, & if target $\tau$ does not generate \\
% 	&\ a measurement at sensor $l$.%
% 	\end{IEEEeqnarraybox}\right.
% \end{equation*}
% On the other hand, we introduce an alternative association vector $b_{k,l} = [b_{k,l}^{(1)},\cdots,b_{k,l}^{(M_{k,l})}]^{\text{T}}$ for the $l$-th sensor, where
% \begin{equation*}
% 	\setlength{\nulldelimiterspace}{0pt}
% 	b_{k,l}^{(i_l)} := \left\{
% 	\begin{IEEEeqnarraybox}[][c]{ls}
% 	\tau \in \{ 1,\cdots,N_{k,l} \},\quad & if measurement $i_l$ originates\\
% 	&\ from target $\tau$;\\
% 	0, & if measurement $i_l$ is a clutter.%
% 	\end{IEEEeqnarraybox}\right.
% \end{equation*}
The constraints for data association, i.e., at time $k$, one target can only generate at most one measurement at each sensor, and one measurement can only originate from one target or clutter, can be represented by an indicator function:
\begin{align}
	\psi(a_{k,l},b_{k,l}) = \prod_{\tau = 1}^{N_{k,l}} \prod_{i_l = 1}^{M_{k,l}} \psi(a_{k,l}^{(\tau)},b_{k,l}^{(i_l)}),
\end{align}
where $\psi(a_{k,l}^{(\tau)},b_{k,l}^{(i_l)}) = 0$ if $a_{k,l}^{(\tau)} = i_l$ and $b_{k,l}^{(i_l)}\neq \tau$ or $b_{k,l}^{(i_l)} = \tau$ and $a_{k,l}^{(\tau)}\neq i_l$, and $\psi(a_{k,l}^{(\tau)},b_{k,l}^{(i_l)}) = 1$ otherwise. 
% \begin{equation*}
% 	\setlength{\nulldelimiterspace}{0pt}
% 	\psi(a_{k,l}^{(\tau)},b_{k,l}^{(i_l)}) :=  \left\{
% 	\begin{IEEEeqnarraybox}[][c]{ls}
% 	0,\quad & $a_{k,l}^{(\tau)} = i_l$, $b_{k,l}^{(i_l)}\neq \tau$\\
% 	&\ or $b_{k,l}^{(i_l)} = \tau$, $a_{k,l}^{(\tau)}\neq i_l$\\
% 	1, & otherwise.%
% 	\end{IEEEeqnarraybox}\right.
% \end{equation*}

At time $k$, when processing the $l$-th sensor, we assume that the beliefs $\tilde{f}_{l-1}(\underline{\mathbf{x}}_{k,l-1}^{(\tau)},\underline{r}_{k,l-1}^{(\tau)})$ and $\tilde{f}_{l-1}(\overline{\mathbf{x}}_{k,l-1}^{(i_{l-1})},\overline{r}_{k,l-1}^{(i_{l-1})})$ for both survived targets and new targets are calculated up to the $(l-1)$-th sensor. Main steps of the BP data association and fusion update algorithm for the measurements of the $l$-th sensor are as follows \cite{Win2018Message}:
\subsubsection{Initialization}
For $l=1$, the beliefs $\tilde{f}_0(\underline{\mathbf{x}}_{k,1}^{(\tau)},\underline{r}_{k,1}^{(\tau)})$ are initialized by $\alpha(\underline{\mathbf{x}}_{k}^{(\tau)},\underline{r}_{k}^{(\tau)})$, i.e., $\tilde{f}_0(\underline{\mathbf{x}}_{k,1}^{(\tau)},\underline{r}_{k,1}^{(\tau)}) = \alpha(\underline{\mathbf{x}}_{k}^{(\tau)},\underline{r}_{k}^{(\tau)})$, 
% \begin{align*}
% 	\tilde{f}_0(\underline{\mathbf{x}}_{k,1}^{(\tau)},\underline{r}_{k,1}^{(\tau)}) = \alpha(\underline{\mathbf{x}}_{k}^{(\tau)},\underline{r}_{k}^{(\tau)}),
% \end{align*}
where
\begin{align}\label{bp_init_1}
	\nonumber
	\alpha(\underline{\mathbf{x}}_{k}^{(\tau)},\underline{r}_{k}^{(\tau)}) =\sum_{\underline{r}_{k}^{(\tau)}\in \{0,1\}} \int & f(\underline{\mathbf{x}}_{k}^{(\tau)},\underline{r}_{k}^{(\tau)}| \mathbf{x}_{k-1}^{(\tau)},r_{k-1}^{(\tau)})\\
	&\times \tilde{f}(\mathbf{x}_{k-1}^{(\tau)},r_{k-1}^{(\tau)}) d \mathbf{x}_{k-1}^{(\tau)}.
\end{align}
Here, $\alpha(\underline{\mathbf{x}}_{k}^{(\tau)},\underline{r}_{k}^{(\tau)})$ is obtained by performing prediction from time $k-1$ to time $k$, $f(\underline{\mathbf{x}}_{k}^{(\tau)},\underline{r}_{k}^{(\tau)}|\mathbf{x}_{k-1}^{(\tau)},r_{k-1}^{(\tau)})$ is the single-target augmented state-transition pdf \cite{Win2018Message}, and $\tilde{f}(\mathbf{x}_{k-1}^{(\tau)},r_{k-1}^{(\tau)})$ is the belief calculated at time $k-1$.

For $l > 1$, the beliefs $\tilde{f}_{l-1}(\underline{\mathbf{x}}_{k,l}^{(\tau)},\underline{r}_{k,l}^{(\tau)})$ are initialized by,
\begin{align}\label{bp_init_2}
	\setlength{\nulldelimiterspace}{0pt}
	\tilde{f}_{l-1} & (\underline{\mathbf{x}}_{k,l}^{(\tau)},\underline{r}_{k,l}^{(\tau)}) \nonumber\\
	&= \left\{
	\begin{IEEEeqnarraybox}[][c]{ls}
		\tilde{f}_{l-1}(\underline{\mathbf{x}}_{k,l-1}^{(\tau)},\underline{r}_{k,l-1}^{(\tau)}) ,\quad & $\tau \leq N_{k,l-1}$  \\
		\\
		\tilde{f}_{l-1}(\overline{\mathbf{x}}_{k,l-1}^{(i_{l-1})},\overline{r}_{k,l-1}^{(i_{l-1})}), & $i_{l-1}= \tau - N_{k,l-1} $,%
	\end{IEEEeqnarraybox}\right.
\end{align}
where $\tau = 1,\cdots,N_{k,l}$, and $N_{k,l}= N_{k,l-1} + M_{k,l-1}$.

\subsubsection{Measurement evaluation}
For the survived targets,
\begin{align}\label{bp_2_meas}
	\beta(a_{k,l}^{(\tau)}) = \sum_{\underline{r}_k^{\tau}\in \{0,1\}} \int & q(\underline{\mathbf{x}}_{k,l}^{(\tau)},\underline{r}_{k,l}^{(\tau)}, a_{k,l}^{(\tau)} ; \mathbf{Z}_{k,l}) \nonumber \\
	&\qquad \times \tilde{f}_{l-1}(\underline{\mathbf{x}}_{k,l}^{(\tau)},\underline{r}_{k,l}^{(\tau)}) d \underline{\mathbf{x}}_{k,l}^{(\tau)},
\end{align}
where $q(\underline{\mathbf{x}}_{k,l}^{(\tau)},\underline{r}_{k,l}^{(\tau)}, a_{k,l}^{(\tau)} ; \mathbf{Z}_{k,l})$ is defined as follows,
\begin{align}
	\setlength{\nulldelimiterspace}{0pt}
	q(&\underline{\mathbf{x}}_{k,l}^{(\tau)},1, a_{k,l}^{(\tau)} ; \mathbf{Z}_{k,l})
	\nonumber \\
	&=\left\{
	\begin{IEEEeqnarraybox}[][c]{ls}
	\frac{P_d^{(l)}(\underline{\mathbf{x}}_{k,l}^{(\tau)}) p(\mathbf{z}_{k,l}^{(i_{l})}|\underline{\mathbf{x}}_{k,l}^{(\tau)})}{\lambda_{f_l} p_{f_l}(\mathbf{z}_{k,l}^{(i_{l})})},\quad & if $a_{k,l}^{(\tau)}\in \{1,\cdots,M_{k,l}\}$\\
	1-P_{d}^{(l)}(\underline{\mathbf{x}}_{k,l}^{(\tau)}),\quad & if $a_{k,l}^{(\tau)}= 0$.
	\end{IEEEeqnarraybox}\right. \label{bp_factor_q_1}
	\\
	q(&\underline{\mathbf{x}}_{k,l}^{(\tau)},0, a_{k,l}^{(\tau)} ; \mathbf{Z}_{k,l})
		= \mathbf{1}(a_{k,l}^{(\tau)}). \label{bp_q_x_r0}
\end{align}
Here, $P_{d}^{(l)}(\underline{\mathbf{x}}_{k,l}^{(\tau)})$ is the detection probability that the target $\tau$ is detected by the $l$-th sensor. $\mathbf{1}(a_{k,l}^{(\tau)}) = 0$ when $a_{k,l}^{(\tau)}\in \{1,\cdots,M_{k,l}\}$, and $\mathbf{1}(a_{k,l}^{(\tau)}) = 1$ when $a_{k,l}^{(\tau)}=0$. {\color{blue}Note that in the settings of limited field-of-view sensors, we assume that the fusion center knows the field-of-view information. When the target $\underline{\mathbf{x}}_{k,l}^{(\tau)}$ is not within the observation area of the sensor, we modify the probability so that $P_{d}^{(l)}(\underline{\mathbf{x}}_{k,l}^{(\tau)}) = 0$. }

For the new targets,
\begin{align}\label{bp_mess_xi}
	\xi(b_{k}^{(i_l)}) = \sum_{\overline{r}_{k,l}^{(i_l)} \in \{0,1\}} \int v(\overline{\mathbf{x}}_{k,l}^{(i_{l})},\overline{r}_{k,l}^{(i_{l})},b_{k,l}^{(i_{l})};\mathbf{z}_{k,l}^{(i_{l})}) d \overline{\mathbf{x}}_{k,l}^{(i_{l})},
\end{align}
where $v(\overline{\mathbf{x}}_{k,l}^{(i_{l})},\overline{r}_{k,l}^{(i_{l})},b_{k,l}^{(i_{l})};\mathbf{z}_{k,l}^{(i_{l})})$ is defined as follows,
\begin{align}
	\setlength{\nulldelimiterspace}{0pt}
	v(&\overline{\mathbf{x}}_{k,l}^{(i_{l})},1,b_{k,l}^{(i_{l})};\mathbf{z}_{k,l}^{(i_{l})})
	\notag\\
	&=\left\{
	\begin{IEEEeqnarraybox}[][c]{ls}
	\frac{\lambda_{n_l} f_{n}(\overline{\mathbf{x}}_{k,l}^{(i_{l})})p(\mathbf{z}_{k,l}^{(i_{l})}|\overline{\mathbf{x}}_{k,l}^{(i_{l})})}{\lambda_{f_l} p_{f_l}(\mathbf{z}_{k,l}^{(i_{l})})},\quad & if $b_{k,l}^{(i_{l})}=0$\\
	0,\quad & if $b_{k,l}^{(i_{l})}\in \{1,\cdots,N_{k,l}\}$,
	\end{IEEEeqnarraybox}\right. \label{bp_factor_v_1}
	\\
	v(&\overline{\mathbf{x}}_{k,l}^{(i_{l})},0,b_{k,l}^{(i_{l})};\mathbf{z}_{k,l}^{(i_{l})})=f_{D}(\overline{\mathbf{x}}_{k,l}^{(i_{l})}). \label{bp_factor_v_2}
\end{align}
Here, $\lambda_{n_l}$ is the mean number of new targets, and $f_{D}(\overline{\mathbf{x}}_{k,l}^{(i_{l})})$ represents a dummy pdf, which means that the target $i_l$ does not exist.

\subsubsection{Iterative data association}
For the data of the $l$-th sensor, at each iteration $p\in \{1,\cdots,P\}$, the following recursions are excuted for all measurements $i_l\in I_l$:
\begin{align}\label{ite_1_v}
	\nu_{i_l \rightarrow \tau}^{(p)}(a_{k,l}^{(\tau)}) = \sum_{b_{k,l}^{(i_l)}=0}^{N_{k,l}} \xi(b_{k,l}^{(i_l)}) \psi(a_{k,l}^{(\tau)},b_{k,l}^{(i_l)}) \prod_{ \genfrac{}{}{0pt}{2}{\tau' = 1}{\tau'\neq \tau} }^{N_{k,l}} \varphi_{\tau' \rightarrow i_l}^{(p-1)}(b_{k,l}^{(i_l)}),
\end{align}
and (when $p\neq P$)
\begin{align}\label{ite_2_kesai}
	\varphi_{\tau \rightarrow i_l}^{(p)}(b_{k,l}^{(i_l)})= \sum_{a_{k,l}^{(\tau)}=0}^{M_{k,l}} \beta(a_{k,l}^{(\tau)}) \psi(a_{k,l}^{(\tau)},b_{k,l}^{(i_l)}) \prod_{ \genfrac{}{}{0pt}{2}{i'_l= 1}{i'_l \neq i_l} }^{M_{k,l}} \nu_{i'_l \rightarrow \tau}^{(p)}(a_{k,l}^{(\tau)}).
\end{align}
For the initialization, i.e., $p = 0$,
\begin{align}\label{ite_init}
	\zeta_{\tau \rightarrow i_l}^{(0)}(b_{k,l}^{(i_l)}) = \sum_{a_{k,l}^{(\tau)}=0}^{M_{k,l}} \beta(a_{k,l}^{(\tau)}) \psi(a_{k,l}^{(\tau)},b_{k,l}^{(i_l)}).
\end{align}
After the last iteration $p=P$ is executed, we multiply the messages $ \nu_{i_l \rightarrow \tau}^{(p)}(a_{k,l}^{(\tau)})$ for $i_l = 1,\cdots,M_{k,l}$,
\begin{align}\label{ite_end}
	\kappa(a_{k,l}^{(\tau)}) = \prod_{i_l=1}^{M_{k,l}} \nu_{i_l \rightarrow \tau}^{(P)}(a_{k,l}^{(\tau)}),
\end{align}
and multiply the messages $\varphi_{\tau \rightarrow i_l}^{(p)}(b_{k,l}^{(i_l)})$ for $\tau = 1,\cdots,N_{k,l}$,
\begin{align}\label{bp_prob_b}
	\iota (b_{k,l}^{(i_l)}) = \prod_{\tau=1}^{N_{k,l}} \varphi_{\tau \rightarrow i_l}^{(P)}(b_{k,l}^{(i_l)}).
\end{align}

\subsubsection{Measurement update}
For the survived targets,
\begin{align}\label{bp_meas_up}
	\gamma(\underline{\mathbf{x}}_{k,l}^{(\tau)},1) &= \sum_{a_{k,l}^{(\tau)} = 0}^{M_{k,l}} q(\underline{\mathbf{x}}_{k,l}^{(\tau)},1, a_{k,l}^{(\tau)} ; \mathbf{Z}_{k,l})  \kappa(a_{k,l}^{(\tau)}),\\
	\gamma(\underline{\mathbf{x}}_{k,l}^{(\tau)},0) &= \kappa(a_k^{(\tau)}=0).\label{bp_meas_up_2}
\end{align}
For the new targets,
\begin{align}
	\varsigma  (\overline{\mathbf{x}}_{k,l}^{(i_{l})},1) &= v(\overline{\mathbf{x}}_{k,l}^{(i_{l})},1,b_{k,l}^{(i_{l})} = 0;\mathbf{z}_{k,l}^{(i_{l})}) \iota (0), \label{bp_meas_up_3} \\
	\varsigma  (\overline{\mathbf{x}}_{k,l}^{(i_{l})},0) &= \sum_{b_{k,l}^{(i_l)}=0}^{N_{k,l}} \iota (b_{k,l}^{(i_l)}) f_D(\overline{\mathbf{x}}_{k,l}^{(i_{l})}). \label{bp_meas_up_4}
\end{align}

\subsubsection{Belief Calculation}
For the survived targets, the fusion beliefs are calculated by 
\begin{align}\label{bp_appx}
	\tilde{f}_{l}(\underline{\mathbf{x}}_{k,l}^{(\tau)},1) &= \frac{1}{\underline{C}_k^{\tau}} \tilde{f}_{l-1}(\underline{\mathbf{x}}_{k,l}^{(\tau)},1) \gamma^{(\tau)}(\underline{\mathbf{x}}_{k,l}^{(\tau)},1),\\
	\label{bp_appx_2}
	\tilde{f}_{l}(\underline{\mathbf{x}}_{k,l}^{(\tau)},0) &= \frac{1}{\underline{C}_k^{\tau}} \tilde{f}_{l-1}(\underline{\mathbf{x}}_{k,l}^{(\tau)},0) \gamma^{(\tau)}(\underline{\mathbf{x}}_{k,l}^{(\tau)},0),
\end{align}
where the constant $\underline{C}_k^{\tau} = \int \tilde{f}_{l-1}(\underline{\mathbf{x}}_{k,l}^{(\tau)},1) \gamma^{(\tau)}(\underline{\mathbf{x}}_{k,l}^{(\tau)},1) d \underline{\mathbf{x}}_{k,l}^{(\tau)} + \tilde{f}_{l-1}(\underline{\mathbf{x}}_{k,l}^{(\tau)},0) \gamma^{(\tau)}(\underline{\mathbf{x}}_{k,l}^{(\tau)},0)$. 
% \begin{align*}
% 	\underline{C}_k^{\tau} =& \int \tilde{f}_{l-1}(\underline{\mathbf{x}}_{k,l}^{(\tau)},1) \gamma^{(\tau)}(\underline{\mathbf{x}}_{k,l}^{(\tau)},1) d \underline{\mathbf{x}}_{k,l}^{(\tau)} \nonumber \\
% 	& \qquad + \tilde{f}_{l-1}(\underline{\mathbf{x}}_{k,l}^{(\tau)},0) \gamma^{(\tau)}(\underline{\mathbf{x}}_{k,l}^{(\tau)},0).
% \end{align*}
For the new targets, the beliefs are calculated by
\begin{align}
	\label{bp_belief_new_1}
	\tilde{f}_l (\overline{\mathbf{x}}_{k,l}^{(i_{l})},1) &= \frac{1}{\overline{C}_k^{i_l}} \varsigma  (\overline{\mathbf{x}}_{k,l}^{(i_{l})},1)\\
	\label{bp_belief_new_2}
	\tilde{f}_l (\overline{\mathbf{x}}_{k,l}^{(i_{l})},0) &= \frac{1}{\overline{C}_k^{i_l}} \varsigma  (\overline{\mathbf{x}}_{k,l}^{(i_{l})},0),
\end{align}
where $\overline{C}_k^{i_l} := \int \varsigma  (\overline{\mathbf{x}}_{k,l}^{(i_{l})},1) d \overline{\mathbf{x}}_{k,l}^{(i_{l})} + \varsigma  (\overline{\mathbf{x}}_{k,l}^{(i_{l})},0)$.

\subsubsection{Target Declaration, State Estimation, and Pruning}
For survived targets and new targets, we use their existence beliefs $\tilde{p}(\underline{r}_{k,l}^{(\tau)}=1)$ and $\tilde{p}(\overline{r}_{k,l}^{(i_l)}=1)$ to declare whether the targets exist. {\color{blue}Here, the existence beliefs are calculated as follows. 
\begin{align}
	\tilde{p}(\underline{r}_{k,l}^{\tau}= 1)= \int \tilde{f}_{l}(\underline{\mathbf{x}}_{k,l}^{\tau},1)  d \underline{\mathbf{x}}_{k,l}^{\tau},\\
	\tilde{p}(\overline{r}_{k,l}^{i_l}= 1)= \int \tilde{f}_{l}(\overline{\mathbf{x}}_{k,l}^{i_l},1)  d \overline{\mathbf{x}}_{k,l}^{i_l}.
\end{align}
}
Given an appropriate threshold $P_{th}$ \cite{Win2018Message}, the survived target or new target is declared to exist if $\tilde{p}(\underline{r}_{k,l}^{(\tau)}= 1) > P_{th}$ or $\tilde{p}(\overline{r}_{k,l}^{(i_l)}= 1) > P_{th}$.

State estimation is performed by
\begin{align}\label{bp_est_1}
	\hat{\underline{\mathbf{x}}}_{k,l}^{\tau,\text{MMSE}} = \int \underline{\mathbf{x}}_{k,l}^{(\tau)} \tilde{f}_{l}(\underline{\mathbf{x}}_{k,l}^{(\tau)},1)/ \tilde{p}(\underline{r}_{k,l}^{(\tau)} = 1) d \underline{\mathbf{x}}_{k,l}^{(\tau)},
\end{align}
for survived targets, and
\begin{align}\label{bp_est_2}
	\hat{\overline{\mathbf{x}}}_{k,l}^{i_l,\text{MMSE}} = \int \overline{\mathbf{x}}_{k,l}^{(i_l)} \tilde{f}_{l}(\overline{\mathbf{x}}_{k,l}^{(i_l)},1)/ \tilde{p}(\overline{r}_{k,l}^{(i_l)} = 1) d \overline{\mathbf{x}}_{k,l}^{(i_l)},
\end{align}
for new targets.

Finally, similar to the target declaration, given appropriate thresholds $P_{pr}$ and $N_{pr}$, survived and new targets are removed when their existence beliefs $\tilde{p}(\underline{r}_{k,l}^{(\tau)}=1)$ and $\tilde{p}(\overline{r}_{k,l}^{(i_l)}=1)$ are below~$P_{pr}$ or they lose measurements more than $N_{pr}$ scans.

\subsection{Equivalency}\label{equ_bp}
Here, we derive the relationship between BP track association with transformed measurements and that with raw measurements in the following Proposition \ref{bp_Prop1}. 
	%Based on Lemmas \ref{Lemma_1}--\ref{Lemma_2}, we have the following equivalency results.

\begin{Proposition}\label{bp_Prop1}
	Under the conditions of Proposition \ref{equivalence_score}, in the measurement evaluation step, the calculation of the factor nodes $q$ and $v$ with raw measurements are equal to those with transformed measurements, respectively, i.e.,
	% Under the conditions of Lemmas \ref{Lemma_1}--\ref{Lemma_2}, in the measurement evaluation step, the factor nodes $q(\underline{\mathbf{x}}_{k,l}^{(\tau)},\underline{r}_{k,l}^{(\tau)}, a_{k,l}^{(\tau)} ; \mathbf{Z}_{k,l})$ and $v(\overline{\mathbf{x}}_{k,l}^{(i_{l})},\overline{r}_{k,l}^{(i_{l})},b_{k,l}^{(i_{l})};\mathbf{z}_{k,l}^{(i_{l})})$ with raw measurements are equal to the factor nodes $q(\underline{\mathbf{x}}_{k,l}^{(\tau)},\underline{r}_{k,l}^{(\tau)}, a_{k,l}^{(\tau)} ; \breve{\mathbf{Z}}_{k,l})$ and $v(\overline{\mathbf{x}}_{k,l}^{(i_{l})},\overline{r}_{k,l}^{(i_{l})},b_{k,l}^{(i_{l})};\breve{\mathbf{z}}_{k,l}^{(i_{l})})$ with transformed measurements, respectively, i.e.,
	\begin{align}
		q(\underline{\mathbf{x}}_{k,l}^{(\tau)},\underline{r}_{k,l}^{(\tau)}, a_{k,l}^{(\tau)} ; \mathbf{Z}_{k,l})&=q(\underline{\mathbf{x}}_{k,l}^{(\tau)},\underline{r}_{k,l}^{(\tau)}, a_{k,l}^{(\tau)} ; \breve{\mathbf{Z}}_{k,l}), \notag\\
		v(\overline{\mathbf{x}}_{k,l}^{(i_{l})},\overline{r}_{k,l}^{(i_{l})},b_{k,l}^{(i_{l})};\mathbf{z}_{k,l}^{(i_{l})})&=v(\overline{\mathbf{x}}_{k,l}^{(i_{l})},\overline{r}_{k,l}^{(i_{l})},b_{k,l}^{(i_{l})};\breve{\mathbf{z}}_{k,l}^{(i_{l})}), \notag
	\end{align}
	where $q(\cdot ; \breve{\mathbf{Z}}_{k,l})$ and $v(\cdot ; \breve{\mathbf{z}}_{k,l}^{(i_{l})})$ are defined by (\ref{bp_factor_q_1})--(\ref{bp_q_x_r0}) and (\ref{bp_factor_v_1})--(\ref{bp_factor_v_2}) by replacing $\mathbf{Z}_{k,l}$ and $\mathbf{z}_{k,l}^{(i_{l})}$ with transformed measurements $\breve{\mathbf{Z}}_{k,l}$ and $\breve{\mathbf{z}}_{k,l}^{(i_{l})}$, respectively. 
	Moreover, for survived targets $\tau= 1,\cdots,N_{k,l-1}$, the fusion beliefs $\tilde{f}_{l}^c(\underline{\mathbf{x}}_{k,l}^{(\tau)},\underline{r}_{k,l}^{(\tau)})$ with raw measurements are equal to the fusion beliefs $\tilde{f}_{l}^d(\underline{\mathbf{x}}_{k,l}^{(\tau)},\underline{r}_{k,l}^{(\tau)})$ with transformed measurements; for new targets $i_l= 1,\cdots,M_{k,l}$, the beliefs $\tilde{f}_{l}^c(\overline{\mathbf{x}}_{k,l}^{(i_l)},\overline{r}_{k,l}^{(i_l)})$ with raw measurements are equal to the beliefs $\tilde{f}_{l}^d(\overline{\mathbf{x}}_{k,l}^{(i_l)},\overline{r}_{k,l}^{(i_l)})$ with transformed measurements.
\end{Proposition}

The proof of Proposition \ref{bp_Prop1} is given in \cite[app C]{Liu2023Oncommunication}. Proposition \ref{bp_Prop1} shows that, under some regularity conditions, the BP track association and fusion with transformed measurements are equivalent to that with raw measurements. Moreover, the complete algorithm of BP track association and fusion with transformed measurements is summarized in \cite[Algorithm~2]{Liu2023Oncommunication}.

\begin{algorithm}
	\caption{BP-Based Association and Fusion}\label{alg:BP}
	\SetKwBlock{uBegin}{}{}
	\SetKwData{Left}{left}\SetKwData{This}{this}\SetKwData{Up}{up} 
	\SetKwFunction{Union}{Union}\SetKwFunction{FindCompress}{FindCompress} 
	\SetKwInOut{Input}{input}\SetKwInOut{Output}{output}

	\Input{$\{\tilde{f}(\mathbf{x}_{k-1}^{(\tau)},r_{k-1}^{(\tau)})\}_{\tau=1}^{N_{k-1}}$, $\breve{\mathbf{Z}}_k$, $\{H_{k,l}\}_{l=1}^L$, or $\{R_{k,l}\}_{l=1}^L$;}
	\Output{$\{\tilde{f}(\mathbf{x}_{k}^{(\tau)},r_{k}^{(\tau)})\}_{\tau=1}^{N_k} $;}
	%\BlankLine
	\emph{Prediction: }
	\uBegin{
		\emph{$\tilde{f}_0(\underline{\mathbf{x}}_{k,1}^{(\tau)},\underline{r}_{k,1}^{(\tau)}) = \alpha(\underline{\mathbf{x}}_{k}^{(\tau)},\underline{r}_{k}^{(\tau)})$ via (\ref{bp_init_1})}\;
		\emph{$N_{k,1} = N_{k-1}$}\;
	}
	\emph{Sequential processing: }\\
	\For{$l=1,\cdots,L$}{
		\emph{Measurement evaluation: (\ref{bp_2_meas})}\;
		\emph{Iterative data association: (\ref{ite_1_v})--(\ref{bp_prob_b})}\;
		\emph{Measurement update: (\ref{bp_meas_up})}\;
		\emph{Belief calculation for survived targets: $\tilde{f}_{l}(\underline{\mathbf{x}}_{k,l}^{(\tau)},\underline{r}_{k,l}^{(\tau)})$ using (\ref{bp_appx})--(\ref{bp_appx_2})}\;
		\emph{Belief calculation for new targets: $\tilde{f}_{l}(\overline{\mathbf{x}}_{k,l}^{(i_l)},\overline{r}_{k,l}^{(i_l)})$ using (\ref{bp_belief_new_1})--(\ref{bp_belief_new_2})}\;
		\emph{State Estimation: (\ref{bp_est_1})--(\ref{bp_est_2})}\;
		\emph{$\tilde{f}_{l}(\underline{\mathbf{x}}_{k,l+1}^{\tau},\underline{r}_{k,l+1}^{\tau}) = \tilde{f}_{l}(\underline{\mathbf{x}}_{k,l}^{(\tau)},\underline{r}_{k,l}^{(\tau)})$, for $\tau \leq N_{k,l}$}\;
		\emph{$\tilde{f}_{l}(\underline{\mathbf{x}}_{k,l+1}^{\tau},\underline{r}_{k,l+1}^{\tau}) = \tilde{f}_{l}(\overline{\mathbf{x}}_{k,l}^{\tau - N_{k,l}},\overline{r}_{k,l}^{\tau - N_{k,l}})$,
		for $N_{k,l} < \tau \leq N_{k,l} + M_{k,l}$}\;
	}
	\emph{Output: }
	\uBegin{
		\emph{$\tilde{f}(\mathbf{x}_{k}^{(\tau)},r_{k}^{(\tau)}) = \tilde{f}_{L}(\underline{\mathbf{x}}_{k,L}^{(\tau)},\underline{r}_{k,L}^{(\tau)})$, for $\tau \leq N_{k,L}$}\;
		\emph{$\tilde{f}(\mathbf{x}_{k}^{(\tau)},r_{k}^{(\tau)}) = \tilde{f}_{L}(\overline{\mathbf{x}}_{k,L}^{(\tau - N_{k,L})},\overline{r}_{k,L}^{(\tau - N_{k,L})})$}\;
		\emph{$N_{k,L} < \tau \leq N_{k,L} + M_{k,L}$}\;
		\emph{$N_{k} = N_{k,L} + M_{k,L}$}\;
		\emph{Set $k=k+1$}\;
	}
\end{algorithm}

\begin{Remark}
{\color{blue}The main advantage of the BP method is its scalability. For a fixed number of iterations $P$ of message passing, the computation complexity of calculating the marginal posterior pdfs of all the target states is only linear in the number of sensors $L$. The complexity of an iteration of the scalable scheme (\ref{ite_1_v})--(\ref{ite_init}) performed for the $l$-th sensor scales as $\mathcal{O}(N_{k,l} M_{k,l})$. If the number of measurements $M_{k,l}$ increases linearly with the number of targets $N_{k,l}$, then the overall complexity of the method scales linearly in the number of sensors and quadratically in the number of targets \cite{meyer2017scalable}. Moreover, if the maximum number of targets is $N_{\text{max}}$, then the worst case computational complexity is $\mathcal{O}(L (P N_{\text{max}}^2))$. } 
%		\item Communication Requirement: In this article, we consider the system have a fusion center. The iteration of BP method can be operated in the fusion center once it receive data from local sensors. Therefore, there has no extra communication burden for the scalable scheme, compared with the MDA track association.
\end{Remark}

\section{Communication Requirements of MDA and BP-Based Track Association}\label{sec_CR}
In this section, the communication requirements of different types of transformation matrices $A_k$ are discussed, and a comparison of communication requirements between sending transformed measurements, raw measurements, and filter information is provided. 

\subsection{Two Types of Transformations}
Let us consider two linear transformations, which have lower communication requirements without loss of information, compared with centralized fusion with raw measurements and information filter fuison.

\begin{table*}[tbp]
	\captionsetup[subtable]{position=top,font={ rm,md,up },justification=raggedright, captionskip= 1pt,farskip= 2pt}%
	\centering
	\caption{Summary of communication requirements}\label{table_1}
	\resizebox{.99\textwidth}{!}{
		\begin{tabular}{l|c|c|c|c}
			\hline
			\multirow{2}{*}{Fusion type}
			&\multirow{2}{*}{Fusion with raw measurements}
			&\multirow{2}{*}{Information filter fusion}
			& \multicolumn{2}{c}{ Fusion with transformed measurements}
			\\ \cline{4-5}
			&\multicolumn{1}{c|}{} &\multicolumn{1}{c|}{}
			&Type 1 &Type 2    \\ \hline
			\multicolumn{1}{l|}{Communication variables}
			&  $\mathbf{z}_{k},H_{k}$, $R_{k}$
			&$\hat{\mathbf{x}}_{k|k},~P_{k|k},~$$ \hat{\mathbf{x}}_{k|k-1},~P_{k|k-1}$
			&~~~~$\breve{\mathbf{z}}_{k}^{(1)},\breve{H}_{k}^{(1)}$~~~~
			&$\breve{\mathbf{z}}_{k}^{(2)},\breve{R}_{k}^{(2)}$  \\ \hline
			%\multicolumn{1}{l|}{Communication requirements (8-byte for one dimension)} & $m+mn+\frac{m(m+1)}{2}$ &$2n+n(n+1)$  &$m+mn$ &$m+\frac{m(m+1)}{2}$ \\ \hline
			\multirow{2}{3.5cm}{Communication requirements (8-byte for one dimension)}
			& \multirow{2}{*}{$8 (m+ mn + \frac{m(m+1)}{2}) N_{\text{max}}$} & \multirow{2}{*}{$8 (2n+ n(n+1))N_{\text{max}}$}  
			& \multirow{2}{*}{$8 (m+ mn)N_{\text{max}}$} & \multirow{2}{*}{$8 (m+ \frac{m(m+1)}{2})N_{\text{max}}$} \\ 
			& & & & \\ \hline
			\multirow{3}{3.5cm}{Communication requirements in Kilobytes (KB) for $N_{\text{max}}= 100$, $m=2$, $n=4$} & \multirow{3}{*}{10.16 KB} & \multirow{3}{*}{21.88 KB} & \multirow{3}{*}{7.81 KB} & \multirow{3}{*}{3.91 KB} \\ 
			& & & & \\
			& & & & \\ \hline
		\end{tabular}}
	% \subfloat[Communication requirements]{
	% 	\label{table_1_1}
	% 	\resizebox{.98\textwidth}{!}{
	% 	\begin{tabular}{l|c|c|c|c}
	% 		\hline
	% 		\multirow{2}{*}{Fusion type}
	% 		&\multirow{2}{*}{Fusion with raw measurements}
	% 		&\multirow{2}{*}{Information filter fusion}
	% 		& \multicolumn{2}{c}{ Fusion with transformed measurements}
	% 		\\ \cline{4-5}
	% 		&\multicolumn{1}{c|}{} &\multicolumn{1}{c|}{}
	% 		&Type 1 &Type 2    \\ \hline
	% 		\multicolumn{1}{l|}{Communication variables}
	% 		&  $\mathbf{z}_{k},H_{k}$, $R_{k}$
	% 		&$\hat{\mathbf{x}}_{k|k},~P_{k|k},~$$ \hat{\mathbf{x}}_{k|k-1},~P_{k|k-1}$
	% 		&~~~~$\breve{\mathbf{z}}_{k}^{(1)},\breve{H}_{k}^{(1)}$~~~~
	% 		&$\breve{\mathbf{z}}_{k}^{(2)},\breve{R}_{k}^{(2)}$  \\ \hline
	% 		%\multicolumn{1}{l|}{Communication requirements (8-byte for one dimension)} & $m+mn+\frac{m(m+1)}{2}$ &$2n+n(n+1)$  &$m+mn$ &$m+\frac{m(m+1)}{2}$ \\ \hline
	% 		\multirow{2}{3.5cm}{Communication requirements (8-byte for one dimension)}
	% 		& \multirow{2}{*}{$8 (m+ mn + \frac{m(m+1)}{2}) N_{\text{max}}$} & \multirow{2}{*}{$8 (2n+ n(n+1))N_{\text{max}}$}  
	% 		& \multirow{2}{*}{$8 (m+ mn)N_{\text{max}}$} & \multirow{2}{*}{$8 (m+ \frac{m(m+1)}{2})N_{\text{max}}$} \\ 
	% 		& & & & \\ \hline
	% 		$N_{\text{max}}= 10$, $m=2$, $n=4$ & 10.16 KB & 21.88 KB & 7.81 KB & 3.91 KB \\ \hline
	% 	\end{tabular}
	% 	}
	% }
	% \\
	% \subfloat[Bandwidth requirements of the sensor and fusion center.]{
	% 	\label{table_2}
	% 	\resizebox{.98\textwidth}{!}{
	% 	\begin{tabular}{l|c|c|c|c|c}
	% 		\hline
	% 		\multicolumn{2}{l}{\multirow{2}{*}{Fusion type}}  &\multicolumn{2}{|c}{ 8-byte for one dimension } & \multicolumn{2}{|l}{ $N_{max}= 10$, $L=10$, $m=2$, $n=4$ }\\
	% 		\cline{3-6}
	% 		\multicolumn{2}{c|}{} &  Sensor & Fusion center & ~~~~~~Sensor~~~~~~ & Fusion center\\
	% 				\hline
	% 		\multicolumn{2}{l|}{Fusion with raw measurements} & $8 (m+ mn + \frac{m(m+1)}{2}) N_{\text{max}}$  &  $8 (m+ mn + \frac{m(m+1)}{2}) N_{\text{max}} L$ & 10.16 KB & 101.56 KB
	% 		\\
	% 		\hline
	% 		\multicolumn{2}{l|}{Information filter fusion} & $8 (2n+ n(n+1))N_{\text{max}}$ & $8 (2n+ n(n+1))N_{\text{max}}L$  &  21.88 KB & 218.75 KB
	% 		\\ 
	% 		\hline
	% 		\multirow{2}{2.8cm}{Fusion with transformed measurements} & Type 1 & $8 (m+ mn)N_{\text{max}}$ & $8 (m+ mn)N_{\text{max}}L$ & 7.81 KB & 78.13 KB
	% 		\\
	% 		\cline{2-6} 
	% 		& Type 2 & $8 (m+ \frac{m(m+1)}{2})N_{\text{max}}$ & $8 (m+ \frac{m(m+1)}{2})N_{\text{max}}L$  & 3.91  KB & 39.06 KB
	% 		\\
	% 		\hline 
	% 	\end{tabular}
	% 	}
	% }
\end{table*}

% \subsubsection{Type 1 Transformation}
% The first type is based on  the information filter \cite{Chong1979_Hierarchical} as follows:
% \begin{align}\label{measure_trans2}
% &\breve{\mathbf{z}}_k^{(1)} := H_k^{\text{T}} R_k^{-1}\mathbf{z}_k = P_{k|k}^{-1}\hat{\mathbf{x}}_{k|k} - P_{k|k-1}^{-1}\hat{\mathbf{x}}_{k|k-1},\\
% &\breve{H}_k^{(1)} := H_k^{\text{T}} R_k^{-1} H_k, \;\breve{\eta}_k^{(1)}:= H_k^{\text{T}} R_k^{-1} \eta_k, \\
% &\breve{R}_k^{(1)}:= \mathnormal{Cov}(\breve{\eta}_k^{(1)}) = H_k^{\text{T}} R_k^{-1} \mathnormal{Cov}(\eta_k) ( H_k^{\text{T}} R_k^{-1})^{\text{T}} = \breve{H}_{k}^{(1)}. \label{tran1_R}
% \end{align}
% where $P_{k|k-1}$ and $P_{k|k}$ are the covariance matrices of the estimation error of estimates $\hat{\mathbf{x}}_{k|k-1}$ and $\hat{\mathbf{x}}_{k|k}$, respectively. $H_k$ and $R_k$ are the full row rank measurement matrix and the covariance of noise, respectively. Here, $A_k= H_k^{\text{T}}R_k^{-1}$ is full column rank matrix, which performs track association without loss of performance via Proposition \ref{equivalence_score} and Proposition \ref{bp_Prop1}.

% Note that, to compute state estimation and association score, each sensor needs to transmit the measurement $\breve{\mathbf{z}}_{k}^{(1)}$, the measurement matrix $\breve{H}_{k}^{(1)}$ and the covariance of the measurement noise $\breve{R}_{k}^{(1)}$ to the fusion center. Equation (\ref{tran1_R}) shows that  $\breve{R}_{k}^{(1)} = \breve{H}_{k}^{(1)}$. Therefore, due to the symmetry of $\breve{H}_{k}^{(1)}$,  the total communication requirement of Type 1 transformation is $(n+n(n+1)/2)$ \cite{2011Li_distributed}.

%{\color{blue}Note that, to compute state estimation and association score, each sensor needs to transmit measurement $\breve{\mathbf{z}}_{k}^{(1)}$, measurement matrix $\breve{H}_{k}^{(1)}$ and the covariance of the measurement noise $\breve{\eta}_{k}^{(1)}$ to the fusion center. With (\ref{tran1_R}), we notice that the covariance of measurement noise $\breve{\eta}_{k}^{(1)}$ is equal to measurement matrix $\breve{H}_{k}^{(1)}$. Therefore, the corresponding communication requirement of type 1 transformation is $(n+n(n+1)/2)$. However, in distributed information  filter fusion, each sensor needs to send data $P_{k|k}$, $P_{k|k-1}$, $\hat{\mathbf{x}}_{k|k}$ and $\hat{\mathbf{x}}_{k|k-1}$ to the fusion center. Thus, the communication requirement for information matrix filtering is ($2n+n(n+1)$) \cite{2011Li_distributed}.}

\subsubsection{Type 1 Transformation}
The first type defined in \cite{2011Li_distributed} is as follows:
\begin{align}\label{measure_trans3}
&\breve{\mathbf{z}}_{k}^{(1)}:=C_{k}\mathbf{z}_{k},\\
&\breve{H}_{k}^{(1)}:=\left[ B_{k}^{\text{T}}R_{k}^{-1}B_{k}\right]^{\frac{1}{2}}D_{k},~\breve{\eta}_{k}^{(1)}:=C_{k}\eta_{k},\\
&\breve{R}_{k}^{(1)}:=\mathnormal{Cov}(\breve{\eta}_k^{(2)})=C_{k}\mathnormal{Cov}(\eta_{k}) C_{k}^{\text{T}}=\mathbf{I},\label{tran2_R}
\end{align}
where $C_{k}=\left[ B_{k}^{\text{T}}R_{k}^{-1}B_{k}\right]^{-\frac{1}{2}} B_{k}^{\text{T}}R_{k}^{-1}$ and $B_{k}$ satisfies $H_{k}=B_{k}D_{k}$ which is the full rank decomposition of $H_{k}$. The rank of $H_{k}$ satisfies, $rank(H_{k})=r_{H}\leq  min(n,m)$ and $\mathbf{I}$ is an identity matrix with dimension $r_{H}$. The transformation matrix for Type 1 is $A_k = C_k= \left[ B_{k}^{\text{T}}R_{k}^{-1}B_{k}\right]^{-\frac{1}{2}} B_{k}^{\text{T}}R_{k}^{-1}$. Note that, for the Type 1 transformation,  $\breve{R}_{k}^{(1)}$ equals an identity matrix $\mathbf{I}$. Thus, each local sensor only needs to send $\breve{\mathbf{z}}_{k}^{(1)}$ and $\breve{H}_{k}^{(1)}$ to the fusion center and the corresponding communication  requirement is $(r_{H}+r_{H}\times n)$ \cite{2011Li_distributed}. Specifically, if $H_k$ is full row matrix, then $C_{k}$ is full column rank matrix and the communication requirement is $m+mn$.


\subsubsection{Type 2 Transformation}The second type is based on a special case of measurement matrix $H_{k}=[E_{k},\mathbf{O}]$, which is defined as follows:
\begin{align}
&\breve{\mathbf{z}}_{k}^{(2)}:= E_{k}^{-1}\mathbf{z}_{k},\\
&\breve{H}_{k}^{(2)}:=\left[\mathbf{I},\mathbf{O}\right],\label{tran_R3} \ \breve{\eta}_{k}^{(2)}:=E_{k}^{-1}\eta_{k},\\
&\breve{R}_k^{(2)}:=\mathnormal{cov}(\breve{\eta}_{k}^{(2)})=E_{k}^{-1}\mathnormal{cov}(\eta_{k})(E_{k}^{-1})^{\text{T}}.
\end{align}
where $E_{k}$ and $\mathbf{O}$ are a full rank matrix and zero matrix, respectively. The transformation matrix for Type 2 is $A_k = E_{k}^{-1}$. Since $\breve{H}_{k}^{(2)}:=\left[\mathbf{I},\mathbf{O}\right]$ is time invariant,  each local sensor only needs send $\breve{\mathbf{z}}_{k}^{(2)}$ and $\breve{R}_k^{(2)}$ to the fusion center and the corresponding transformation requirement  is $m+m(m+1)/2$.


\subsection{Comparison of Communication Requirements between Transformed Measurements and Other Data}
For fusion with raw measurements, each sensor needs to transmit measurement $\mathbf{z}_{k}$, measurement matrix $H_{k}$ and the covariance of the measurement noise $R_{k}$ to the fusion center. Thus, the communication requirements for each sensor is $m+mn+\frac{m(m+1)}{2}$.

In multisensor information  filter fusion \cite{Chong1979_Hierarchical}, each sensor needs to send data $P_{k|k}$, $P_{k|k-1}$, $\hat{\mathbf{x}}_{k|k}$ and $\hat{\mathbf{x}}_{k|k-1}$ to the fusion center. Thus, the communication requirement for information matrix filtering is ($2n+n(n+1)$) \cite{2011Li_distributed}.



The communication requirements for lossless measurement transformations are summarized  in Table \ref{table_1}. Note that, for time-invariant systems, only the measurements are sent to the fusion center. It is not necessary  to send $H_{k}$ and $R_{k}$  to the fusion center at each time $k$. In this case, the communication requirements of fusion with raw measurements are the same as those of Type 1 and Type 2. The communication dimension equals $m$. However, for time-varying systems, Table \uppercase\expandafter{\romannumeral1} shows that the communication requirements of  fusion with lossless transformation Types 1--2 are less than those of fusion with raw measurements or the information filter fusion, where the transformation matrix $A_k \in \mathbb{R}^{n\times m}$ is full column rank. {\color{blue}Furthermore, if the local sensor sends transformed measurements and the corresponding transformation matrix $A_k$ to the fusion center, then the fusion center can reconstruct the raw measurements and raw measurement model. However, the fusion center is sometimes unaware about the information of the transformation matrix (i.e., the local sensor does not share the transformation matrix to the fusion center). The analysis for the track association and fusion algorithms with transformed measurements in Section \ref{sec_MDA} and Section \ref{sec_BF} shows that they are equivalent to the track association and fusion algorithms with raw measurements, which does not require knowledge of the transformation matrix. }

{\color{blue}
\begin{Remark}
	Suppose that $N_{\text{max}}$ is the maximum number of targets seen by the sensor network, i.e.,  $N_{\text{max}}= \max (N_{1}, \cdots, N_{L})$, where $L$ is the number of sensors, then the order of magnitude of data that a sensor needs to transmit to the fusion center is upper bounded by $N_{\text{max}}$. Suppose that each dimension is represented by an 8-byte floating-point value. Then the communication bandwidth requirements of the sensor are summarized in Row 3 of Table \ref{table_1}. Specifically, let $n= 4$, $m= 2$, and $N_{\text{max}}= 100$. The corresponding communication bandwidth requirements of the sensor in Kilobytes are given in Row 4 of Table \ref{table_1}. It also shows that the communication bandwidth requirements of fusion with lossless transformation Types 1--2 are less than those of fusion with raw measurements and information filter fusion, where the transformation matrix $A_k \in \mathbb{R}^{n \times m}$ is full column rank.
\end{Remark}}

{\color{blue}
\begin{Remark}
	The communication rate is one of the important factors affecting the performance of multisensor fusion. Under the assumption of full-rate communication, multisensor estimation fusion is usually equivalent to centralized measurement fusion \cite{Li2003_Optimal,chang1997optimal,Chong1979_Hierarchical,2011Li_distributed}. However, if full-rate communication is not available, the performance of estimation fusion is not optimal in general since information sent to the fusion center is reduced. A track-to-track fusion method at arbitrary communication rates can be seen proposed in \cite{Koch2008On,Koch2009Exact,Govaers2012An}. Our paper discusses the lossless track measurement transformations under full-rate communication. We show that the transformed track measurements reduce the communication requirements (see Table \ref{table_1}), and are equivalent to the raw track measurements. On the other hand, in the case of reduced-rate communication, track association with transformed track measurements is still equivalent to that with raw track measurements, since the proof of equivalence does not depend on the communication rate. But the performance is worse than that of the full-rate communication. 
\end{Remark}}

%{\color{blue} For type 1 transformation, any local sensor needs send measurement $\breve{y}_{k}^{(1)}$, measurement matrix $\breve{H}_{k}^{(1)}$ and the covariance of the measurement noise $\breve{\eta}_{k}^{(1)}$ to the fusion center. With (\ref{tran1_R}), we notice that the covariance of measurement noise $\breve{\eta}_{k}^{(1)}$ equals to measurement matrix $\breve{H}_{k}^{(1)}$. Therefore, the communication requirement can be reduced to $(n+n(n+1)/2)$\cite{2011Li_distributed}. Similarly, with (\ref{tran2_R}), the covariance of measurement noise $\breve{\eta}_{k}^{(2)}$ equals to an unity matrix and the corresponding communication requirement of type 2 transformation is $(r_{H}+r_{H}\times n)$. With (\ref{tran_R3}), the measurement matrices $\breve{H}_{k}^{(3)}$ do not change over time. Therefore, for any local sensor, it only need sent $\breve{\mathbf{z}}_{k}^{(3)}$ and $\breve{\eta}_{k}^{(3)}$ to the fusion center and the corresponding communication requirement is $(m+m(m+1)/2)$.}


\section{Simulation Results}\label{sec_SR}
In this section, we consider two scenarios with different numbers of sensors and targets, where the first one is a simple case with two sensors and three targets, and the second one contains ten sensors and ten targets. They are used to verify the main equivalency results of the MDA track association and BP-based track association, respectively. The performance of the proposed algorithms is evaluated by OSPA distance (with cutoff $c=50$ m, order $p=2$) \cite{Schuhmacher2008Aconsistent} as well as OSPA$^{(2)}$ distance (with the same cutoff $c$ and order $p$, and window length $w=10$) \cite{Beard2017OSPA}, and the estimated number of targets. Similar scenarios can be seen in \cite{van2021distributed}.
\subsection{Simulation Setting}
Let us consider the scenario where the targets are moving in the 2-D plane. The state of each target is modeled as 2-D position and velocity, i.e., the state of target $\tau$ is denoted by $\mathbf{x}_{k}^{(\tau)} = [x_{k}^{(\tau),1}, x_{k}^{(\tau),2},\dot{x}_{k}^{(\tau),1},\dot{x}_{k}^{(\tau),2}]^{\text{T}}$. Each target follows a nearly constant velocity model: $\mathbf{x}_{k}^{(\tau)} = F \mathbf{x}_{k-1}^{(\tau)} + \Gamma v_{k-1}^{(\tau)}$. Here, $F= [1, \Delta T; 0, 1] \otimes \mathbf{I}_2$ and $\Gamma = [\Delta T^2/2; \Delta T] \otimes \mathbf{I}_2$, where $\otimes$ denotes for the Kronecker tensor product, $\mathbf{I}_2$ is the 2-D identity matrix, and $\Delta T$ is the sampling period;  
% \begin{eqnarray}
% % \nonumber to remove numbering (before each equation)
%   F = \left[
%   \begin{array}{cccc}
%     1 & 0 & \Delta T & 0 \\
%     0 & 1& 0 & \Delta T \\
%     0 & 0 & 1 & 0 \\
%     0 & 0 & 0& 1\\
%   \end{array}
% \right], 
% \end{eqnarray}
% and 
% \begin{eqnarray}
% 	\Gamma = \left[
% 	  \begin{array}{cc}
% 		\Delta T^2/2 & 0  \\
% 		0 & \Delta T^2/2 \\
% 		\Delta T & 0  \\
% 		0 & \Delta T  \\
% 	  \end{array}
% 	\right],
% 	\end{eqnarray}
the process noise $v_{k-1}^{(\tau)} \sim \mathcal{N}(\mathbf{0},q^2 \mathbf{I}_2) $ is a zero-mean Gaussian process noise, where $q$ characterizes the average increment of target speed in $\Delta T$. 
% \begin{eqnarray}
% % \nonumber to remove numbering (before each equation)
%   Q = q^2 \left[
%            \begin{array}{cccc}
%             \Delta T^4/3  & 0 & \Delta T^3/2& 0 \\
%              0 & \Delta T^4/3  & 0& \Delta T^3/2\\
%              \Delta T^3/2 & 0 &\Delta T^2& 0 \\
%             0& \Delta T^3/2 & 0& \Delta T^2\\
%            \end{array}
%          \right]
% \end{eqnarray}
The raw measurement $\mathbf{z}_{k}^{(i_l)}$ originates from target $\tau$ at the $l$-th sensor is modeled according to $\mathbf{z}_{k}^{(i_l)}=H_{k,l} \mathbf{x}_{k}^{(\tau)} + w_{k,l}^{(i_l)}$,
% \begin{align}\label{sim_meas_fun}
% 	\mathbf{z}_{k}^{(i_l)}=H_{k,l} \mathbf{x}_{k}^{(\tau)} + w_{k,l}^{(i_l)},
% \end{align}
where $H_{k,l}= [\text{diag}(1+ \theta_{k,l}^{(1)},1+ \theta_{k,l}^{(2)}), \mathbf{O}]$,  
% \begin{eqnarray}
% % \nonumber to remove numbering (before each equation)
%   H_{k,l} = \left[
%             \begin{array}{cccc}
%               1+ \theta_{k,l}^{(1)} & 0 & 0 & 0 \\
%               0 & 1+ \theta_{k,l}^{(2)} & 0 & 0 \\
%             \end{array}
%           \right],
% \end{eqnarray}
$\theta_{k,l}^{(1)}$ and $\theta_{k,l}^{(2)}$ are time-varying uncertain parameters which may be sensor bias estimates. Here, they are known and are uniformly generated from [-0.02, 0.02]. The measurement noise $w_{k,l}^{(i_l)} \sim \mathcal{N}(\mathbf{0}, R_{k,l})$ is a zero-mean Gaussian noise with covariance $R_{k,l} = \text{diag}(\sigma^2+ \vartheta_{k,l}^{(1)},\sigma^2+ \vartheta_{k,l}^{(2)})$, where $\sigma$ is the reference standard deviation, $\vartheta_{k,l}^{(1)}$ and $\vartheta_{k,l}^{(2)}$ are time-varying uncertain parameters which may be the covariance of the sensor bias estimates. $\vartheta_{k,l}^{(1)}$ and $\vartheta_{k,l}^{(2)}$ are known and are uniformly generated from  [0, 1]. Each sensor can only detect targets within its field-of-view (angle of $[-45^\circ,45^\circ]$, range of $1200$ m) with probability $P_d^{(l)}$, $l=1,\cdots,L$. The clutter pdf is assumed uniform in the field-of-view of each sensor, and the number of clutter is assumed Poisson distributed with a mean number of $\lambda_{f_l}$, i.e., the clutter rate is $\lambda_{f_l}$. For this measurement equation, Type 1 transformation is $A_{k,l}= C_{k,l}$, where $C_{k,l}= \text{diag}(\sigma^2 + \vartheta_{k,l}^{(1)}, \sigma^2 + \vartheta_{k,l}^{(2)})^{-\frac{1}{2}}$; Type 2 transformation is $A_{k,l} =E_{k,l}^{-1}$, where $E_{k,l}= \text{diag}(1+ \theta_{k,l}^{(1)},1+ \theta_{k,l}^{(2)})$. Since the communication requirement of Type~2 transformation is less than that of Type~1 transformation in the scenarios, we use Type~2 transformation to verify the equivalency results throughout this section.

At each local sensor, initialization and tracking of the local tracks are obtained by the global nearest neighbor tracker (other trackers can be used, such as JPDA and MHT, etc.). The track management settings for the local sensor are as follows: a track is confirmed if it has been associated with at least four measurements and deleted if it loses measurements over $N_{pr}$ consecutive scans, where $N_{pr}= 3$. At each scan, once the tracks at the local sensor are updated, the local sensor sends the confirmed track measurements to the fusion center. 

{\color{blue}
We describe the two scenarios and present the corresponding simulation results in the following two subsections. The parameters common to both scenarios are set as follows. The sampling period is $\Delta T = 1$~s; the standard deviation of the process noise is $q = 0.1$~m/s$^2$; the reference standard deviation of the measurement noise is $\sigma = 5$~m.
}

%In the following, we consider two scenarios with different number of sensors, where the first one is a simple case with 2 sensors and 3 targets, and the second one contains 10 sensors and 10 targets. They are used to verify the main equivalency results of MDA track association and  BP track association, respectively. The performance of the proposed algorithms is measured by OSPA distance, OSPA$^{(2)}$ distance, as well as the number of targets.
\subsection{Scenario 1: Two Nodes with Three Targets}
\begin{figure}[H]% 
	\vspace{-5mm}
	\centering 
	\subfloat[][]{% 
	\label{fig_scen_1_tracks_1}% 
	\includegraphics[width= .24\textwidth]{fig/scen_1_truth.eps}}%  
	\subfloat[][]{% 
	\label{fig_scen_1_tracks_4}% 
	\includegraphics[width= .24\textwidth]{fig/tracks_pd_7.eps}}%
	\\
	\subfloat[][]{% 
	\label{fig_scen_1_tracks_5}% 
	\includegraphics[width= .24\textwidth]{fig/tracks_lambda_10.eps}}% 
	\subfloat[][]{% 
	\label{fig_scen_1_tracks_8}% 
	\includegraphics[width= .24\textwidth]{fig/tracks_lambda_40.eps}}%
	\caption[Scenario 1 results.]{Scenario 1 ground truth and estimated trajectories under different probabilities of detection and clutter rates: \subref{fig_scen_1_tracks_1} groud truth, \subref{fig_scen_1_tracks_4} estimated trajectories with $P_d^{(l)}= 0.9$, $\lambda_{f_l}= 10$, \subref{fig_scen_1_tracks_5} estimated trajectories (blue lines) with $P_d^{(l)}= 0.9$, $\lambda_{f_l}= 40$, \subref{fig_scen_1_tracks_8} estimated trajectories with $P_d^{(l)}= 0.7$, $\lambda_{f_l}= 10$. Starting and stopping positions are denoted by $\circ $ and $\square $, respectively. }% 
	\label{fig_scen_1_tracks}% 
	\vspace{-2mm}
\end{figure}
As shown in Fig. \ref{fig_scen_1_tracks}\subref{fig_scen_1_tracks_1}, we placed three targets moving in the 2-D plane $[-800 \text{ m}, 800 \text{ m}]\times[-800\text{ m},400\text{ m}]$ and used two sensors with limited field-of-view to monitor the targets. The whole period from the first target's birth to the last target's death is 100~s. Specifically, Targets 1 and 2 are born at time 1 s and die at time 100~s, and Target~3 is born at time 10~s and dies at time 80~s. The two sensors are located at $[-600,-800]^{\text{T}}$ m and $[600,-800]^{\text{T}}$ m, respectively. {\color{blue}To test the performance of the MDA track fusion algorithm, we evaluate it under different values of the clutter rate $\lambda_{f_l}= 10, 20, 30$, and $40$ when $P_d^{(l)}= 0.9$, and under different values of probability of detection $P_d^{(l)}= 0.7, 0.8, 0.9$, and $0.99$ when $\lambda_{f_l}= 10$.} We applied the LP relaxation-based algorithm \cite{Storms2003An} to solve the MDA track association problem. The track management settings for the fusion center are as follows: a track is confirmed if it is associated with at least two measurements and deleted if it loses measurements over $N_{pr}$ consecutive scans, where $N_{pr}= 3$.

{\color{blue}
Figs. \ref{fig_scen_1_tracks}\subref{fig_scen_1_tracks_4}--\ref{fig_scen_1_tracks}\subref{fig_scen_1_tracks_8} show the estimated trajectories under different probabilities of detection and clutter rates, the common parameter of Figs. \ref{fig_scen_1_tracks}\subref{fig_scen_1_tracks_4}--\ref{fig_scen_1_tracks}\subref{fig_scen_1_tracks_5} is the clutter rate $\lambda_{f_l}= 10$, and that of Figs. \ref{fig_scen_1_tracks}\subref{fig_scen_1_tracks_5}--\ref{fig_scen_1_tracks}\subref{fig_scen_1_tracks_8} is the probability of detection $P_d^{(l)}= 0.9$. Furthermore, as expected, Figs. \ref{fig_scen_1_tracks}\subref{fig_scen_1_tracks_4}--\ref{fig_scen_1_tracks}\subref{fig_scen_1_tracks_5} indicate that the lower the probability of detection, the slower the track initialization, and Figs. \ref{fig_scen_1_tracks}\subref{fig_scen_1_tracks_5}--\ref{fig_scen_1_tracks}\subref{fig_scen_1_tracks_8} show that the higher the clutter rate, the more the false tracks. 
}

\begin{figure}% 
	\centering \subfloat[][]{% 
	\label{fig_2_1}% 
	\includegraphics[width= .48\textwidth]{fig/scen_1_ospa_lambda.eps}}% 
	\hspace{8pt}%
	\subfloat[][]{% 
	\label{fig_2_2}% 
	\includegraphics[width= .48\textwidth]{fig/scen_1_ospa2_lambda.eps}} 
	\hspace{8pt}%
	\subfloat[][]{% 
	\label{fig_2_3}% 
	\includegraphics[width= .48\textwidth]{fig/scen_1_num_lambda.eps}}% 
	\caption[Scenario 1 results.]{Comparison results under different clutter rates: \subref{fig_2_1} OSPA distance, \subref{fig_2_2} OSPA$^{(2)}$ distance, \subref{fig_2_3} estimated number of targets (the black line represents the true number of targets). In the legends, ``Raw'' and ``Tran'' are the abbreviations of ``Raw measurements'' and ``Transformed measurements'', respectively.}% 
	\label{fig_2_all}% 
\end{figure}

{\color{blue}Figs. \ref{fig_2_all}--\ref{fig_3_all} illustrate the performance results of the MDA association and fusion algorithm with raw measurements and transformed measurements, where the OSPA distance, the OSPA$^{(2)}$ distance, and the estimated number of targets are plotted as a function of time steps, respectively. Figs. \ref{fig_2_all}--\ref{fig_3_all} confirm that the performance of the MDA association and fusion algorithm with raw measurements is equivalent to that with transformed measurements, which corroborates Proposition~\ref{equivalence_score}. Moreover, Figs. \ref{fig_2_all}\subref{fig_2_1}--\ref{fig_2_all}\subref{fig_2_3} show the performance results under $\lambda_{f_l}= 10,20,30$, and $40$ when $P_d^{(l)}= 0.9$, respectively. The curves of the OSPA distances shown in Fig. \ref{fig_2_all}\subref{fig_2_1} exhibit peaks at time $k = 10$ and $80$ s, respectively. The reason is that Target 3 is born at time $k=10$ s and dies at time $k=80$ s. Since a local track is confirmed when it has at least four measurements, the duration of the first two peaks of OSPA is about 4 s. Since a track is deleted if it loses measurements over three consecutive scans, the durations of last peak of OSPA are about 2 s. As expected, the OSPA distance, the OSPA$^{(2)}$ distance, and the estimated number of targets increase as the clutter rate $\lambda_{f_l}$ increases, which is consistent with the phenomenon in Figs. \ref{fig_scen_1_tracks}\subref{fig_scen_1_tracks_5}--\ref{fig_scen_1_tracks}\subref{fig_scen_1_tracks_8}. The reason is that the number of false tracks at the local sensors increases as $\lambda_{f_l}$ increases, and the number of false track measurements sent to the fusion center also increases, which results in an increase in the number of false tracks at the fusion center. Besides, a high clutter rate results in the curve of estimated number of targets in Fig. \ref{fig_2_all}\subref{fig_2_3} above that of the true number of targets.


\begin{figure}% 
	\centering \subfloat[][]{% 
	\label{fig_3_1}% 
	\includegraphics[width= .48\textwidth]{fig/scen_1_ospa_Pd.eps}}% 
	\hspace{8pt}%
	\subfloat[][]{% 
	\label{fig_3_2}% 
	\includegraphics[width= .48\textwidth]{fig/scen_1_ospa2_Pd.eps}} 
	\hspace{8pt}%
	\subfloat[][]{% 
	\label{fig_3_3}% 
	\includegraphics[width= .48\textwidth]{fig/scen_1_num_Pd.eps}}% 
	\caption[Scenario 1 results.]{Comparison results under different probabilities of detection: \subref{fig_3_1} OSPA distance, \subref{fig_3_2} OSPA$^{(2)}$ distance, \subref{fig_3_3} estimated number of targets. }% 
	\label{fig_3_all}% 
\end{figure}

Figs. \ref{fig_3_all}\subref{fig_3_1}--\ref{fig_3_all}\subref{fig_3_3} show the performance results under $P_d^{(l)}= 0.7,0.8,0.9$, and $0.99$ when $\lambda_{f_l}= 10$, respectively. For the same reason as before, the curves of the OSPA distances shown in Fig. \ref{fig_3_all}\subref{fig_3_1} exhibit two peaks. As expected, the OSPA and OSPA$^{(2)}$ distances decrease as the probability $P_d^{(l)}$ of detection increases, and the estimated number of targets increases as $P_d^{(l)}$ increases, which is consistent with the phenomenon in Figs. \ref{fig_scen_1_tracks}\subref{fig_scen_1_tracks_4}--\ref{fig_scen_1_tracks}\subref{fig_scen_1_tracks_5}. The reason is that a low $P_d^{(l)}$ results in a late track initialization at the local sensor, which is the reason that the curve of the estimated number of targets in Fig.~\ref{fig_3_all}\subref{fig_3_3} is below that of the true number of targets. 

Finally, Table \ref{table_3} summarizes the averaged communication requirements over 100 scans and 100 Monte Carlo runs at the fusion center. Under different values of $\lambda_{f_l}$ and $P_d^{(l)}$, the communication requirements for sending transformed measurements are less than those for sending raw measurements, which corroborates the communication requirement analysis in Section \ref{sec_CR}.





\begin{table}[H]
	\centering
	\caption{Communication Requirements in Bytes (B) of Scenario 1}\label{table_3}
	\subfloat[Different values of clutter rate $\lambda_{f_l}$]{
		\resizebox{.49\textwidth}{!}{
			\begin{tabular}{|l|c|c|c|c|c|}
				\hline
				\multicolumn{2}{|l}{\multirow{2}{*}{Fusion type}} & \multicolumn{4}{|c|}{Clutter rate $\lambda_{f_l}$} \\
				\cline{3-6}
				\multicolumn{2}{|c|}{} & 10 & 20 & 30 & 40\\
				\hline
				\multicolumn{2}{|l|}{Fusion with raw measurements} & 499.4 B & 506.3 B & 526.6 B & 579.9 B \\
				\hline 
				\multirow{2}{2.8cm}{Fusion with transformed measurements} &Type 1 & 307.4 B & 311.5 B & 324.2 B & 356.8 B\\ 
				\cline{2-6}
				& Type 2 & 192.1 B & 194.7 B & 202.6 B & 223.0 B\\ \hline
			\end{tabular}
		}
	}\\
	\subfloat[Different values of probability of detection $P_d^{(l)}$]{
		\resizebox{.49\textwidth}{!}{
			\begin{tabular}{|l|c|c|c|c|c|}
				\hline
				\multicolumn{2}{|l}{\multirow{2}{*}{Fusion type}} & \multicolumn{4}{|c|}{Probability of detection $P_d^{(l)}$} \\
				\cline{3-6}
				\multicolumn{2}{|c|}{} & 0.7 & 0.8 & 0.9 & 0.99\\
				\hline
				\multicolumn{2}{|l|}{Fusion with raw measurements} & 363.6 B & 444.2 B & 499.4 B & 533.1 B \\
				\hline 
				\multirow{2}{2.8cm}{Fusion with transformed measurements} & Type 1 & 223.7 B & 273.3 B & 307.4 B & 328.0 B\\ \cline{2-6}
				& Type 2 & 139.8 B & 170.8 B & 192.1 B & 205.0 B\\ \hline
			\end{tabular}
		}
	}
\end{table}
}

\subsection{Scenario 2: Ten Nodes with Ten Targets}
\begin{figure}[H]% 
	\vspace{-3mm}
	\captionsetup[subfigure]{position=top,font={ rm,md,up 	},justification=raggedright, captionskip=-3pt,farskip= -3pt,singlelinecheck=false}%%
	\centering 
	\subfloat[][]{% 
	\label{fig_scen_2_truth}% 
	\includegraphics[width= .24\textwidth]{fig/scen_2_truth.eps}}% 
	\subfloat[][]{% 
	\label{fig_scen_2_tracks_1}% 
	\includegraphics[width= .24\textwidth]{fig/scen_2_tracks_lambda_10_pd_7.eps}}%  
	\\
	\subfloat[][]{% 
	\label{fig_scen_2_tracks_4}% 
	\includegraphics[width= .24\textwidth]{fig/scen_2_tracks_lambda_10_pd_9.eps}}%
	\subfloat[][]{% 
	\label{fig_scen_2_tracks_5}% 
	\includegraphics[width= .24\textwidth]{fig/scen_2_tracks_lambda_40_pd_9.eps}}% 
	\caption[Scenario 2 results.]{Scenario 2 ground truth and estimated trajectories under different probabilities of detection and clutter rates: \subref{fig_scen_2_truth} groud truth, \subref{fig_scen_2_tracks_1} $P_d^{(l)}= 0.7$, $\lambda_{f_l}= 10$, \subref{fig_scen_2_tracks_4} $P_d^{(l)}= 0.9$, $\lambda_{f_l}= 10$, \subref{fig_scen_2_tracks_5} $P_d^{(l)}= 0.9$, $\lambda_{f_l}= 40$.}% 
	\label{fig_scen_2_tracks}% 
\end{figure}
% \begin{figure}[H]
% 	\centering
% 	\includegraphics[width= 0.32\textwidth]{fig/scen_2_truth.eps}
% 	\caption{Scenario 2 ground truth (starting and stopping positions are denoted by $\circ $ and $\Box $, respectively) and sensor positions (and the corresponding limited field-of-views). }
% 	\label{fig_5}
% \end{figure}
In this scenario, we use ten sensors to track ten targets and validate that the BP-based track association and fusion algorithm with transformed measurements is equivalent to that with raw measurements. The ten targets move in the 2-D plane $[-1000 \text{ m}, 1000 \text{ m}]\times[-1000\text{ m},1000\text{ m}]$, where the whole period from the first target's birth to the last target's death is 100~s. Specifically, Targets 1--3 are born at time 1~s and die at time 100 s, Targets 4--6 are born at time 20~s and die at time 60~s, and Targets 7--10 are born at time 40~s and die at time 80~s. We set up ten sensors located on a circle with a radius of $1000$ m, where all the sensors are equidistant from each other. The sensors and the targets of Scenario~2 are shown in Fig.~\ref{fig_scen_2_tracks}\subref{fig_scen_2_truth}. {\color{blue}To test the performance of the BP track association and fusion algorithm, we evaluate it under different $\lambda_{f_l}= 10, 20, 30$, and $40$ when $P_d^{(l)}= 0.9$, and under different $P_d^{(l)}= 0.7, 0.8, 0.9$, and $0.99$ when $\lambda_{f_l}= 10$.} We used a particle-based implementation of the BP-based track association algorithm, where for each target, a set of 1000 particles is applied to approximate its belief. The number of iterations in the step of iterative data association is set to $P=10$. For target declaration, a target is declared if its existence belief $\tilde{p}(\underline{r}_{k,l}^{(\tau)}= 1) > P_{th}$ or $\tilde{p}(\overline{r}_{k,l}^{(i_l)}= 1) > P_{th}$, where $P_{th} = 0.7$. For target pruning, a target is removed if its existence belief $\tilde{p}(\underline{r}_{k,l}^{(\tau)}= 1) < P_{pr}$ or $\tilde{p}(\overline{r}_{k,l}^{(i_l)}= 1) < P_{pr}$, or if it loses measurements over $N_{pr}$ consecutive scans, where $P_{pr} = 1\text{e}^{-6}$ and $N_{pr}= 3$.

% \begin{figure}% 
% 	\captionsetup[subfigure]{position=top,font={ rm,md,up 	},justification=raggedright, captionskip=-3pt,farskip= -3pt,singlelinecheck=false}%%
% 	\centering 
% 	\subfloat[][]{% 
% 	\label{fig_scen_2_tracks_1}% 
% 	\includegraphics[width= .32\textwidth]{fig/scen_2_tracks_lambda_10_pd_7.eps}}%  
% 	\\
% 	\subfloat[][]{% 
% 	\label{fig_scen_2_tracks_4}% 
% 	\includegraphics[width= .32\textwidth]{fig/scen_2_tracks_lambda_10_pd_9.eps}}%
% 	\\
% 	\subfloat[][]{% 
% 	\label{fig_scen_2_tracks_5}% 
% 	\includegraphics[width= .32\textwidth]{fig/scen_2_tracks_lambda_40_pd_9.eps}}% 
% 	\caption[Scenario 2 results.]{Estimated trajectories under different probabilities of detection and clutter rates: \subref{fig_scen_2_tracks_1} $P_d^{(l)}= 0.7$, $\lambda_{f_l}= 10$, \subref{fig_scen_2_tracks_4} $P_d^{(l)}= 0.9$, $\lambda_{f_l}= 10$, \subref{fig_scen_2_tracks_5} $P_d^{(l)}= 0.9$, $\lambda_{f_l}= 40$.}% 
% 	\label{fig_scen_2_tracks}% 
% \end{figure}

{\color{blue}
Figs. \ref{fig_scen_2_tracks}\subref{fig_scen_2_tracks_1}--\ref{fig_scen_2_tracks}\subref{fig_scen_2_tracks_5} shows the estimated trajectories under different probabilities of detection and clutter rates. As expected, Figs. \ref{fig_scen_2_tracks}\subref{fig_scen_2_tracks_1}--\ref{fig_scen_2_tracks}\subref{fig_scen_2_tracks_4} indicate that the lower the probability of detection, the slower the track initialization, and Figs. \ref{fig_scen_2_tracks}\subref{fig_scen_2_tracks_4}--\ref{fig_scen_2_tracks}\subref{fig_scen_2_tracks_5} show that the higher the clutter rate, the more the false tracks. Besides, when compared with Fig. \ref{fig_scen_1_tracks} in Scenario 1, Fig. \ref{fig_scen_2_tracks} shows fewer false tracks at high clutter rates and faster track initialization at low detection rates. The reason is that in this scenario, the fusion center utilizes complementary information from more sensors. 
%, the common parameter of Figs. \ref{fig_scen_2_tracks}\subref{fig_scen_2_tracks_1}--\ref{fig_scen_2_tracks}\subref{fig_scen_2_tracks_4} is the clutter rate $\lambda_{f_l}= 10$, and that of Figs. \ref{fig_scen_2_tracks}\subref{fig_scen_2_tracks_4}--\ref{fig_scen_2_tracks}\subref{fig_scen_2_tracks_5} is the probability of detection $P_d^{(l)}= 0.9$. Moreover, as expected, Figs. \ref{fig_scen_2_tracks}\subref{fig_scen_2_tracks_1}--\ref{fig_scen_2_tracks}\subref{fig_scen_2_tracks_4} indicate that the lower probability of detection, the slower the track initialization, and Figs. \ref{fig_scen_2_tracks}\subref{fig_scen_2_tracks_4}--\ref{fig_scen_2_tracks}\subref{fig_scen_2_tracks_5} show that the higher the clutter rate, the more false tracks. Besides, when compared with Fig. \ref{fig_scen_1_tracks} in Scenario 1, Fig. \ref{fig_scen_2_tracks} shows fewer false tracks at high clutter rates and faster track initialization at low detection rates. The reason is that in this scenario, the fusion center utilizes complementary information from more sensors. 

Figs. \ref{fig_5_all}--\ref{fig_6_all} show that the performance results of the BP-based association and fusion algorithm with raw measurements and transformed measurements, where the OSPA distance, the OSPA$^{(2)}$ distance, and the estimated number of targets are plotted as a function of time steps, respectively. The results shown in Figs.~\ref{fig_5_all}--\ref{fig_6_all} confirm that the performance of the BP-based association and fusion algorithm with raw measurements is equivalent to that with transformed measurements, which corroborates Proposition \ref{bp_Prop1}. Besides, the curves of Figs. \ref{fig_5_all}--\ref{fig_6_all} are closer to each other than those of Figs. \ref{fig_2_all}--\ref{fig_3_all} in Scenario 1 since the fusion center in Scenario 2 utilizes more complementary information from more sensors. Moreover, Figs.~\ref{fig_5_all}\subref{fig_5_1}--\ref{fig_5_all}\subref{fig_5_3} show the performance results under $\lambda_{f_l}= 10,20,30$, and $40$ when $P_d^{(l)}= 0.9$, respectively. The curves of the OSPA distances shown in Fig. \ref{fig_5_all}\subref{fig_5_1} exhibit peaks at time $k = 1$, $20$, $40$, $60$, and $80$ s, due to the targets being born and dying. For the same reason as discussed in Scenario 1, the duration of the first three peaks (due to the targets being born) of OSPA is about 4 s. Since a track is deleted if its existence belief $\tilde{p}(\underline{r}_{k,l}^{(\tau)}= 1) < P_{pr}= 1\text{e}^{-6}$ or it loses measurements over $N_{pr}=3$ consecutive scans, it yields that the duration of last two peaks of OSPA is about 1--2 s. As expected, the OSPA distance, the OSPA$^{(2)}$ distance, and the estimated number of targets increase as the clutter rate $\lambda_{f_l}$ increases, which is consistent with the phenomenon in Figs. \ref{fig_scen_2_tracks}\subref{fig_scen_2_tracks_4}--\ref{fig_scen_2_tracks}\subref{fig_scen_2_tracks_5}. The reason is the same as that in Scenario 1.

\begin{figure}% 
	\centering \subfloat[][]{% 
	\label{fig_5_1}% 
	\includegraphics[width=.48\textwidth]{fig/scen_2_ospa_lambda.eps}}% 
	\hspace{8pt}%
	\subfloat[][]{% 
	\label{fig_5_2}% 
	\includegraphics[width=.48\textwidth]{fig/scen_2_ospa2_lambda.eps}} 
	\hspace{8pt}%
	\subfloat[][]{% 
	\label{fig_5_3}% 
	\includegraphics[width=.48\textwidth]{fig/scen_2_num_lambda.eps}}% 
	\caption[Scenario 2 results.]{Comparison results under different clutter rates: \subref{fig_5_1} OSPA distance, \subref{fig_5_2} OSPA$^{(2)}$ distance, \subref{fig_5_3} estimated number of targets.}% 
	\label{fig_5_all}% 
\end{figure}

\begin{figure}% 
	\centering \subfloat[][]{% 
	\label{fig_6_1}% 
	\includegraphics[width=.48\textwidth]{fig/scen_2_ospa_Pd.eps}}% 
	\hspace{8pt}%
	\subfloat[][]{% 
	\label{fig_6_2}% 
	\includegraphics[width=.48\textwidth]{fig/scen_2_ospa2_Pd.eps}} 
	\hspace{8pt}%
	\subfloat[][]{% 
	\label{fig_6_3}% 
	\includegraphics[width=.48\textwidth]{fig/scen_2_num_Pd.eps}}% 
	\caption[Scenario 2 results.]{Comparison results under different probabilities of detection: \subref{fig_5_1} OSPA distance, \subref{fig_5_2} OSPA$^{(2)}$ distance, \subref{fig_5_3} estimated number of targets.}% 
	\label{fig_6_all}% 
\end{figure}

Figs. \ref{fig_6_all}\subref{fig_6_1}--\ref{fig_6_all}\subref{fig_6_3} show the performance results under $P_d^{(l)}= 0.7,0.8,0.9$, and $0.99$ when $\lambda_{f_l}= 10$, respectively. Similarly, the curves of the OSPA distances shown in Fig. \ref{fig_6_all}\subref{fig_6_1} also exhibit five peaks. As expected, the OSPA and OSPA$^{(2)}$ distances decrease as the probability $P_d^{(l)}$ of detection increases, and the estimated number of targets increases as $P_d^{(l)}$ increases, which is consistent with the phenomenon in Figs. \ref{fig_scen_2_tracks}\subref{fig_scen_2_tracks_1}--\ref{fig_scen_2_tracks}\subref{fig_scen_2_tracks_4}. The reason is also the same as that in Scenario 1. 

Finally, Table \ref{table_4} summarizes the averaged communication requirements over 100 scans and 100 Monte Carlo runs at the fusion center. Under different values of $\lambda_{f_l}$ and $P_d^{(l)}$, the communication requirements for sending transformed measurements are less than those for sending raw measurements, which corroborates the analysis of communication requirement in Section \ref{sec_CR}. Compared to Table \ref{table_3} in Scenario 1, the communication requirements increase with the number of targets and the number of sensors. Moreover, the amount of communication bandwidths reduced by transmitting transformed measurements in Scenario 2 is more than that in Scenario 1.

% Furthermore, Fig. \ref{fig_7} illustrates the averaged OSPA distance, OSPA$^{(2)}$ distance, and runtime over 100 time steps and 200 Monte Carlo runs as a function of the number of sensors, where the number of sensors increases from 2 to 10. As expected, the performance of the BP track association and fusion algorithm increases as the number of sensors increases. Moreover, the runtime of the algorithm increases linearly with the number of sensors, i.e., the computational complexity of the algorithm scales linearly in the number of sensors.  

\begin{table}
	\centering
	\caption{Communication Requirements in Kilobytes (KB) of Scenario~2}\label{table_4}
	\subfloat[Different values of clutter rate $\lambda_{f_l}$]{
		\resizebox{.49\textwidth}{!}{
			\begin{tabular}{|l|c|c|c|c|c|}
				\hline
				\multicolumn{2}{|l}{\multirow{2}{*}{Fusion type}} & \multicolumn{4}{|c|}{Clutter rate $\lambda_{f_l}$} \\
				\cline{3-6}
				\multicolumn{2}{|c|}{} & 10 & 20 & 30 & 40\\
				\hline
				\multicolumn{2}{|l|}{Fusion with raw measurements} & 4.64 KB & 4.70 KB  & 4.93 KB & 5.24 KB \\
				\hline 
				\multirow{2}{2.8cm}{Fusion with transformed measurements} &Type 1 & 2.87 KB & 2.90 KB & 3.02 KB & 3.23 KB\\ 
				\cline{2-6}
				& Type 2 & 1.79 KB & 1.81 KB & 1.89 KB & 2.02 KB\\ \hline
			\end{tabular}
		% \begin{tabular}{|l|c|c|c|c|}
		% 	\hline
		% 	\multirow{2}{*}{Type of data} & \multicolumn{4}{c|}{Clutter rate $\lambda_{f_l}$} \\
		% 	\cline{2-5}
		% 	 & 10 & 20 & 30 & 40\\
		% 	\hline
		% 	Raw measurement & 4.64 KB & 4.70 KB  & 4.93 KB & 5.24 KB \\
		% 	\hline 
		% 	Type 1 Transformation & 2.87 KB & 2.90 KB & 3.02 KB & 3.23 KB\\ \hline
		% 	Type 2 Transformation & 1.79 KB & 1.81 KB & 1.89 KB & 2.02 KB\\ \hline
		% \end{tabular}
		}
	}\\
	\subfloat[Different values of probability of detection $P_d^{(l)}$]{
		\resizebox{.49\textwidth}{!}{
			\begin{tabular}{|l|c|c|c|c|c|}
				\hline
				\multicolumn{2}{|l}{\multirow{2}{*}{Fusion type}} & \multicolumn{4}{|c|}{Probability of detection $P_d^{(l)}$} \\
				\cline{3-6}
				\multicolumn{2}{|c|}{} & 0.7 & 0.8 & 0.9 & 0.99\\
				\hline
				\multicolumn{2}{|l|}{Fusion with raw measurements} & 3.57 KB & 4.22 KB & 4.64 KB & 4.93 KB \\
				\hline 
				\multirow{2}{2.8cm}{Fusion with transformed measurements} & Type 1 & 2.19 KB & 2.59 KB & 2.87 KB & 3.04 KB\\ \cline{2-6}
				& Type 2 & 1.37 KB & 1.62 KB & 1.79 KB & 1.90 KB\\ \hline
			\end{tabular}
		% \begin{tabular}{|l|c|c|c|c|}
		% 	\hline
		% 	\multirow{2}{*}{Type of data} & \multicolumn{4}{c|}{Probability of detection $P_d^{(l)}$} \\
		% 	\cline{2-5}
		% 	 & 0.7 & 0.8 & 0.9 & 0.99\\
		% 	\hline
		% 	Raw measurement & 3.57 KB & 4.22 KB & 4.64 KB & 4.93 KB \\
		% 	\hline 
		% 	Type 1 Transformation & 2.19 KB & 2.59 KB & 2.87 KB & 3.04 KB\\ \hline
		% 	Type 2 Transformation & 1.37 KB & 1.62 KB & 1.79 KB & 1.90 KB\\ \hline
		% \end{tabular}
		}
	}
\end{table}

% \begin{figure}
% 	\centering \subfloat[][]{% 
% 	\label{fig_7_1}% 
% 	\includegraphics[width=.48\textwidth]{fig/scen_2_scale_ospa.eps}}% 
% 	\hspace{8pt}%
% 	\subfloat[][]{% 
% 	\label{fig_7_2}% 
% 	\includegraphics[width=.48\textwidth]{fig/scen_2_scale_time.eps}} 
% 	\caption[Scenario 2 results.]{Scenario 2 results: \subref{fig_7_1} OSPA and OSPA$^{(2)}$ distances under different number of sensors; \subref{fig_7_2} Runtime under different number of sensors.}
% 	\label{fig_7}
% \end{figure}
}

\section{Conclusion}\label{sec_cl}
{\color{blue}In this paper, for the fundamental problem of multisensor track-to-track fusion for multitarget tracking, we demonstrated the MDA-based data association (with and without prior track information) using linear transformations of track measurements is lossless, and is equivalent in terms of performance to that based on raw track measurements. Next, we presented a BP-based multisensor track association method based on measurement transformations and showed that it is equivalent to that with raw measurements. Finally, considering communication efficiency, two analytical lossless transformations for track association were provided, and communication requirements from each sensor to the fusion center were shown to be less than those of fusion with raw track measurements. Numerical examples for tracking an unknown number of targets using limited field-of-view sensors verified that the performance of fusion with transformed measurements is the same as that of fusion with raw measurements. Future works may include analyzing set-type track association methods \cite{van2021distributed}, distributed consensus fusion systems with or without feedback \cite{zhu2001optimality,zhu2012networked}, and non-linear dynamic systems \cite{Xiao2022Multisensor,Tan2023Nolinear}.}

%In this paper, for track association performed for multisensor track-to-track fusion, we have shown that the MDA and BP track associations based on transformed measurements are equivalent to the track associations based on raw measurements. Computation complexity and communication requirements were discussed. We described two types of transformations that reduce the communication requirements from each sensor to the fusion center compared to those of the  fusion with raw measurements. Numerical examples for tracking an unknown number of targets using limited field-of-view sensors verified that the performance of fusion with transformed measurements is the same as that of fusion with raw measurements. Future works may include analyzing set-type track association methods, distributed consensus fusion system with or without feedback, and non-linear dynamic systems etc.


% if have a single appendix:
%\appendix[Proof of the Zonklar Equations]
% or
%\appendix  % for no appendix heading
% do not use \section anymore after \appendix, only \section*
% is possibly needed

% use appendices with more than one appendix
% then use \section to start each appendix
% you must declare a \section before using any
% \subsection or using \label (\appendices by itself
% starts a section numbered zero.)
%




% use section* for acknowledgment
\section*{Acknowledgment}
The authors would like to thank Yunmin Zhu for helpful suggestions.


% Can use something like this to put references on a page
% by themselves when using endfloat and the captionsoff option.
\ifCLASSOPTIONcaptionsoff
  \newpage
\fi



% trigger a \newpage just before the given reference
% number - used to balance the columns on the last page
% adjust value as needed - may need to be readjusted if
% the document is modified later
%\IEEEtriggeratref{8}
% The "triggered" command can be changed if desired:
%\IEEEtriggercmd{\enlargethispage{-5in}}

% references section

% can use a bibliography generated by BibTeX as a .bbl file
% BibTeX documentation can be easily obtained at:
% http://mirror.ctan.org/biblio/bibtex/contrib/doc/
% The IEEEtran BibTeX style support page is at:
% http://www.michaelshell.org/tex/ieeetran/bibtex/
\bibliographystyle{IEEEtran}
% argument is your BibTeX string definitions and bibliography database(s)
\bibliography{IEEEabrv,reference1(5)}

\appendices

\section{Proof of Proposition \ref{equivalence_score}}\label{proof_prop1}
In this section, we prove Proposition \ref{equivalence_score}, which demonstrates that the MDA-based track association with prior track information with transformed measurements is equivalent to that with raw track measurements. Before we begin, we provide the preliminary results in Lemmas \ref{Lemma_1} and \ref{Lemma_2}.  
\begin{Lemma}\label{Lemma_1}
	Let $A_{k,l}$ be a full column rank matrix, and the processed data $\breve{\mathbf{z}}_{k,l}^{(i_{l})}$ be a linear transformation of $\mathbf{z}_{k,l}^{(i_{l})}$, i.e.,
	$\breve{\mathbf{z}}_{k,l}^{(i_{l})} = A_{k,l} \mathbf{z}_{k,l}^{(i_{l})}.$
	Then, the ratio of the likelihood $p(\mathbf{z}_{k,l}^{(i_{l})}|\hat{\mathbf{x}}_{k|k-1}^{(\tau)})$ for fusion with raw measurements (\ref{likeli_centralized}) and the likelihood $p(\breve{\mathbf{z}}_{k,l}^{(i_{l})}|\hat{\mathbf{x}}_{k|k-1}^{(\tau)})$ for fusion with transformed measurements (\ref{likeli_distributed}) is given by
	\begin{align}
		\frac{p(\mathbf{z}_{k,l}^{(i_{l})}|\hat{\mathbf{x}}_{k|k-1}^{(\tau)})}{p(\breve{\mathbf{z}}_{k,l}^{(i_{l})}|\hat{\mathbf{x}}_{k|k-1}^{(\tau)})} = \frac{(\prod_{i=1}^m e_i)^{1/2}}{(|S_{k,l}^{(\tau)}|)^{1/2}},
	\end{align}
	where $S_{k,l}^{(\tau)}$ is the innovation matrix and $e_i$ is the nonzero eigenvalue of $A_{k,l} S_{k,l}^{(\tau)} A_{k,l}^{\text{T}}$.
	\end{Lemma}
	\begin{IEEEproof}
		Without loss of generality, we omit the time index $k$, the sensor index $l$, and the target index $\tau$ to simplify writing. The likelihood function with raw measurements (\ref{likeli_centralized}) is
	\begin{equation}\label{likelihoodcenter}
		p\left(\mathbf{z} \mid \mathbf{x}\right)=\frac{(2\pi)^{-m/2}}{(|S|)^{1/2}}  e^{-\frac{1}{2}\left(\mathbf{z}-\hat{\mathbf{z}}\right)^{\text{T}}S^{-1}\left(\mathbf{z}-\hat{\mathbf{z}}\right)},
	\end{equation}
	and the likelihood function with transformed measurements (\ref{likeli_distributed}) is
	%With the linear transformation of $\breve{\mathbf{z}}=A\mathbf{z}$, the likelihood function of distributed fusion (\ref{likeli_distributed}) is
	\begin{equation}\label{likelihooddistri}
		p\left(\breve{\mathbf{z}} \mid \mathbf{x}\right)=\frac{(2\pi)^{-m/2}}{(\prod_{i=1}^{m}e_i)^{1/2} } e^{-\frac{1}{2} (\mathbf{z}-\hat{\mathbf{z}})^{\text{T}} A^{\text{T}} (AS A^{\text{T}})^{\dagger}A (\mathbf{z}-\hat{\mathbf{z}}) }.
	\end{equation}
	The ratio of the above two pdfs $p\left(\mathbf{z} \mid \mathbf{x}\right)$ and $p\left(\breve{\mathbf{z}} \mid \mathbf{x}\right)$ is
	%The ratio of the pdf $p\left(\mathbf{z} \mid \mathbf{x}_{\tau},s\right)$ for centralized fusion and the pdf $p\left(\breve{\mathbf{z}} \mid \mathbf{x}_{\tau},s\right)$ for distributed fusion is
	\begin{equation}\label{pf_likeli_ratio}
		\frac{p\left(\mathbf{z} \mid \mathbf{x}\right)}{p\left(\breve{\mathbf{z}} \mid \mathbf{x}\right)}=\frac{\sqrt{\prod_{i=1}^{m}e_{i}}}{\sqrt{|S|}}e^{-\frac{1}{2} (\mathbf{z}-\hat{\mathbf{z}})^{\text{T}}[S^{-1}- A^{\text{T}} (AS A^{\text{T}})^{\dagger}A] (\mathbf{z}-\hat{\mathbf{z}}) }.
	\end{equation}
	If the equation 
	\begin{align}\label{pf_eq}
		A^{\text{T}}(AS A^{\text{T}})^{\dagger}A=S^{-1}
	\end{align}
	holds, then Lemma \ref{Lemma_1} is proved. 

	Next, we give the proof of Equation (\ref{pf_eq}). The range spaces of matrices $A^{\text{T}}AS A^{\text{T}}$ and $SA^{\text{T}}$ can be written as follows:
	\begin{align}
		&\mathbf{W}_{1}=\{y|y= A^{\text{T}}ASA^{\text{T}}x, x \in \mathbb{R}^{n}\},\\
		\label{A16}
		&\mathbf{W}_{2}=\{y^{*}|y^{*}=S A^{\text{T}}x, x \in \mathbb{R}^{n}\},
	\end{align}
	and $\mathbf{W}_{1},\mathbf{W}_{2} \subseteq  \mathbb{R}^{m}$.
	Since $A$ and $S$ are full column rank matrix and fully rank matrix, respectively, the dimension of $\mathbf{W}_{2}$ is $m$ and $\mathbf{W}_{2} = \mathbb{R}^{m}$.
	Through (\ref{A16}), the range space $\mathbf{W}_{1}$ can be represented as follows:
	\begin{align*}
		\mathbf{W}_{1}=\{y|y= A^{\text{T}}Ay^{*}, y^{*} \in \mathbf{W}_{2}\},
	\end{align*}
	Since $A^{\text{T}} A$ is a fully rank matrix, the projection between $y$ and $y^{*}$ is a bijection, i.e., $\mathbf{W}_{1}=\mathbf{W}_{2}=\mathbb{R}^{m}$. Thus, we have
	\begin{align}\label{A(16)}
		\mathbb{R}(A^{\text{T}}AS A^{\text{T}})=\mathbb{R}(S A^{\text{T}}).
	\end{align}
	Similarly, we have the following relationships:
	\begin{align}\label{A(17)}
		&\mathbb{R}(S A^{\text{T}}AS A^{\text{T}})=\mathbb{R}(A^{\text{T}}),\\
		\label{mpinverse_3}
		&\mathbb{R}( S^{\text{T}}S A^{\text{T}})=\mathbb{R}( A^{\text{T}}),\\
		\label{mpinverse_4}
		&\mathbb{R}( A^{\text{T}}A S^{\text{T}})=\mathbb{R}( S^{\text{T}}).
	\end{align}
	Through (\ref{A(16)})--(\ref{A(17)}) and the reverse order laws for Moore-Penrose inverse, the following equation is established:
	\begin{align}\label{inverse_1}
		A^{\text{T}}(AS A^{\text{T}})^{\dagger}A= A^{\text{T}}(S A^{\text{T}})^{\dagger}.
	\end{align}
	Similarly, through (\ref{mpinverse_3})--(\ref{mpinverse_4}) and the reverse order laws for Moore-Penrose inverse, the equation
	\begin{align}\label{inverse_2}
		A^{\text{T}}(S A^{\text{T}})^{\dagger}=S^{-1},
	\end{align}
	holds. By (\ref{inverse_1}) and (\ref{inverse_2}), we have
	\begin{align}\label{dagger_lemma}
		A^{\text{T}}(AS A^{\text{T}})^{\dagger}A=S^{-1},
	\end{align}
	thus, the ratio of the two pdfs is
	\begin{align}
		\frac{p\left(\mathbf{z}\mid \mathbf{x}\right)}{p\left(\breve{\mathbf{z}} \mid \mathbf{x}\right)}=\frac{\sqrt{\prod_{i=1}^{m}e_{i}}}{\sqrt{|S|}},
	\end{align}
	and Lemma \ref{Lemma_1} is proved.
\end{IEEEproof}

\begin{Lemma}\label{Lemma_2}
	If $A_{k,l}$ is a full column rank matrix and the clutter is uniform in the region of interest, then the ratio of the clutter pdf $p_{f_{l}}(\mathbf{z}_{k,l}^{(i_{l})})$ of fusion with raw measurements and the clutter pdf $p_{f_{l}}(\breve{\mathbf{z}}_{k,l}^{(i_{l})})$ of fusion with transformed measurements is
	\begin{align}
		\frac{p_{f_{l}}(\mathbf{z}_{k,l}^{(i_{l})})}{p_{f_{l}}(\breve{\mathbf{z}}_{k,l}^{(i_{l})})} = \sqrt{|A_{k,l}^{\text{T}}A_{k,l}|}.
		\end{align}
	\end{Lemma}
	\begin{IEEEproof}
		Without loss of generality, we omit the time index $k$, the sensor index $l$, and the target index $\tau$ to simplify writing. Let $\mathbf{z}\in \mathbb{R}^m$ and $S_{\mathbf{z}}\subseteq \mathbb{R}^m$ denote the measurement and surveillance region, respectively. If the clutter is assumed uniform, the pdf of a clutter is
	\begin{align}
		p_{f_{l}}\left(\mathbf{z}\mid \gamma_{0}\right)=\frac{\lambda_{f_l}}{V_{\mathbf{z}}},
	\end{align}
	where $V_{\mathbf{z}}$ is the volume of the surveillance region $S_{\mathbf{z}}$, i.e., $V_{\mathbf{z}} = |S_{\mathbf{z}}|$.
	Since $\breve{\mathbf{z}}=A\mathbf{z}$, where $A$ is an $m_1 \times m$ matrix and $m_1 > m$, we have the transformed vector $\breve{\mathbf{z}} \in \mathbb{R}^{m_1}$ and the transformed region $S_{\breve{\mathbf{z}}}\subseteq \mathbb{R}^{m_1}$. Let $V_{\breve{\mathbf{z}}}$ denote the volume of $S_{\breve{\mathbf{z}}}$, the pdf of $\breve{\mathbf{z}}$ is
	\begin{align}
		p_{f_{l}}\left(\breve{\mathbf{z}}\mid \gamma_{0}\right)=\frac{\lambda_{f_l}}{V_{\breve{\mathbf{z}}}}.
	\end{align}
	Then, the ratio of $p_{f_{l}}(\mathbf{z}\mid \gamma_{0})$ and $p_{f_{l}}(\breve{\mathbf{z}}\mid \gamma_{0})$ is
	\begin{align}\label{eq_lem_1} 
		\frac{p_{f_{l}}\left(\mathbf{z}\mid \gamma_{0}\right)}{p_{f_{l}}\left(\breve{\mathbf{z}}\mid \gamma_{0}\right)} = \frac{V_{\breve{\mathbf{z}}}}{V_{\mathbf{z}}}.
	\end{align}
		Since $\breve{\mathbf{z}}=A\mathbf{z}$, where $A$ is an $m_1 \times m$ matrix ($m_1 > m$), by singular value decomposition (SVD) of $A$, we obtain
		\begin{align}\label{svd}
		\breve{\mathbf{z}}=U_{1}\Sigma_{A} V_{1}^{\text{T}}\mathbf{z},
		\end{align}
	where $A = U_{1}\Sigma_{A} V_{1}^{\text{T}}$, and $U_{1}$ and $ V_{1}^{\text{T}}$ are unitary matrices and $\Sigma_{A}$ is an $m_1 \times m$ matrix which nonzero elements are the singular values of $A$. Left multiply $ U_1^{\text{T}}$ by the both sides of (\ref{svd}), i.e.,
	\begin{align}\label{svd_1}
		U_1^{\text{T}}\breve{\mathbf{z}}&= U_1^{\text{T}}U_{1}\Sigma_{A} V_{1}^{\text{T}}\mathbf{z}=\Sigma_{A} V_{1}^{\text{T}}\mathbf{z}.
	\end{align}
	From (\ref{svd_1}), the first $m$ components of $(U_1)^{\text{T}}\breve{\mathbf{z}}$ are represented as
	\begin{align}
		( U_1^{\text{T}}\breve{\mathbf{z}})_{(i)} = \sigma_i ( V_{1}^{\text{T}}\mathbf{z})_{(i)},\quad i= 1,\cdots,m,
	\end{align}
	where $\sigma_i$ is the $i$-th singular value. The last $(m_1- m)$ components of $ U_1^{\text{T}}\breve{\mathbf{z}}$ are all zeros.
	Since $ U_1^{\text{T}}$ and $ V_{1}^{\text{T}}$ are the unitary matrices, which do not change the length of $\breve{\mathbf{z}}$ and $\mathbf{z}$, respectively, the ratio of the volumes $V_{\breve{\mathbf{z}}}$ and $V_{\mathbf{z}}$ is
	\begin{align}\label{eq_lem_2}
		\frac{V_{\breve{\mathbf{z}}}}{V_{\mathbf{z}}} = \prod_{i=1}^m \sigma_i.
	\end{align}
	Note that the singular values of $A$ are the square roots of the eigenvalues of $A^{\text{T}}A$, i.e.,
	\begin{align}\label{eq_lem_3}
		\prod_{i=1}^m \sigma_i = \sqrt{| A^{\text{T}} A|}.
	\end{align}
	Thus, based on (\ref{eq_lem_1}), (\ref{eq_lem_2}) and (\ref{eq_lem_3}), we conclude that
	\begin{align}
		\frac{p_{f_{l}}\left(\mathbf{z}\mid \gamma_{0}^{k}\right)}{p_{f_{l}}\left(\breve{\mathbf{z}}\mid \gamma_{0}^{k}\right)} = \sqrt{| A^{\text{T}}A|},
	\end{align}
	and Lemma \ref{Lemma_2} is proved.
\end{IEEEproof}

Lemma \ref{Lemma_1} provides the likelihood ratio for a measurement originating from a target, between raw measurements and transformed measurements. Lemma \ref{Lemma_2} provides the likelihood ratio for a clutter measurement, also between raw measurements and transformed measurements. In the following, we prove Proposition \ref{equivalence_score} using the results of Lemmas \ref{Lemma_1} and \ref{Lemma_2}. 

Through (\ref{score_centralized})--(\ref{score_distributed}), the ratio of the two score functions is rewritten as follows,
\begin{align}\label{A30}
&\frac{L_{(\tau,i_1,\cdots,i_L)}^{c}}{L_{(\tau,i_1,\cdots,i_L)}^{d}}=\prod_{l \in \{l|u(i_{l})=1\}}
\left[\frac{p(\mathbf{z}_{k,l}^{(i_{l})}|\hat{\mathbf{x}}_{k|k-1}^{(\tau)})}{p(\breve{\mathbf{z}}_{k,l}^{(i_{l})}|\hat{\mathbf{x}}_{k|k-1}^{(\tau)})} \frac{p_{f_{l}}(\breve{\mathbf{z}}_{k,l}^{(i_{l})}|\gamma^k_0)}{p_{f_{l}}(\mathbf{z}_{k,l}^{(i_{l})}|\gamma^k_0)}\right].
\end{align}
By Lemma {\ref{Lemma_1}} and Lemma {\ref{Lemma_2}}, the ratio (\ref{A30}) can be simplified as follows
\begin{align}\label{proof_ratio}
	\frac{L_{(\tau,i_1,\cdots,i_L)}^{c}}{L_{(\tau,i_1,\cdots,i_L)}^{d}}=\prod_{l \in \{l|u(i_{l})=1\}}\frac{(\prod_{i=1}^m e_{i,l})^{1/2}}{(|S_{k,l}^{(\tau)}| )^{1/2}(|A_{k,l}^{\text{T}} A_{k,l}|)^{1/2}},
\end{align}
where $e_{i,l}$ is the $i$-th eigenvalue of $A_{k,l} S_{k,l}^{(\tau)} A_{k,l}^{\text{T}}$. Next, we prove that $(|S_{k,l}^{(\tau)}| )^{1/2}(| A_{k,l}^{\text{T}}A_{k,l}|)^{1/2} = (\prod_{i=1}^m e_{i,l})^{1/2}$, then the conclusion of Proposition $\ref{equivalence_score}$  holds. Without loss of generality, we omit the time index $k$, the sensor index $l$, and the target index $\tau$ to simplify writing.
The SVD of $AS A^{\text{T}}$ and $A$ are obtained as follows,
\begin{align}
\label{A29}
AS A^{\text{T}}=U\Sigma U^{\text{T}},\quad A=U_{1} \Sigma_{A} V_{1}^{\text{T}},
\end{align}
where $U,U_{1},V_{1}$ are identity matrices, and
\begin{align}
  &\Sigma=\left[\begin{array}{ccccc}
    \varLambda & \mathbf{0}  \\
    \mathbf{0} & \mathbf{0}  \\
    \end{array}\right], \quad
    \varLambda = \text{diag}(e_{1},\cdots,e_{m}),\\
  &\Sigma_{A} = \left[\begin{array}{ccccc}
    \Sigma_{A}^{(1)}  \\
    \mathbf{0}   \\
    \end{array}\right], \quad
  \Sigma_{A}^{(1)} = \text{diag}(\sigma_1,\cdots,\sigma_m),
\end{align}
where $\sigma_i$ is the $i$-th singular value of $A$, and $e_i$ is the $i$-th eigenvalue of $A S A^{\text{T}}$.
Through (\ref{A29}), we obtain
\begin{IEEEeqnarray}{rCl}
	U \Sigma U^{\text{T}} = AS A^{\text{T}} = U_{1}\Sigma_{A} V_{1}^{\text{T}} S V_{1} \Sigma_{A}^{\text{T}} U_{1}^{\text{T}}.
\end{IEEEeqnarray}
Thus, we have $\Sigma\sim \Sigma_{A} V_{1}^{\text{T}} S V_{1} \Sigma_{A}^{\text{T}}$, 
% \begin{align}\label{A32}
% \Sigma\sim \Sigma_{A} V_{1}^{\text{T}} S V_{1} \Sigma_{A}^{\text{T}},
% \end{align}
where
\begin{align}
  \Sigma_{A} V_{1}^{\text{T}} S V_{1} \Sigma_{A}^{\text{T}} = \left[\begin{array}{cc}
    \Sigma_{A}^{(1)} V_{1}^{\text{T}} S V_{1} (\Sigma_{A}^{(1)})^{\text{T}} & \mathbf{0}\\
    \mathbf{0} & \mathbf{0}
  \end{array}\right].
\end{align}
Due to the similarity of $\Sigma$ and $\Sigma_{A} V_{1}^{\text{T}}S V_{1} \Sigma_{A}^{\text{T}}$, the eigenvalues of $\varLambda$  and $\Sigma_{A}^{(1)} V_{1}^{\text{T}}S V_{1} (\Sigma_{A}^{(1)})^{\text{T}}$ are the same. Based on the relationship between eigenvalue and determinant, we obtain $|\varLambda|=|\Sigma_{A}^{(1)} V_{1}^{\text{T}} S V_{1} (\Sigma_{A}^{(1)})^{\text{T}}|$, 
% \begin{align*}
% |\varLambda|=|\Sigma_{A}^{(1)} V_{1}^{\text{T}} S V_{1} (\Sigma_{A}^{(1)})^{\text{T}}|,
% \end{align*}
i.e.,
\begin{align}\label{A24}
\prod_{i=1}^m e_{i}=|S| |(\Sigma_A^{(1)})^{\text{T}}\Sigma_A^{(1)}| = |S | | A^{\text{T}} A |,
\end{align}
where the last equation follows from (\ref{eq_lem_3}).
Thus, by (\ref{proof_ratio}) and (\ref{A24}), the score function with raw measurements is equal to that with transformed measurements, i.e.,
\begin{align}
	\frac{L_{(\tau,i_1,\cdots,i_L)}^{c}}{L_{(\tau,i_1,\cdots,i_L)}^{d}}=1,
\end{align}
and Proposition {\ref{equivalence_score}} is proved.

{\section{Proof of Corollary \ref{corro_nopiror}}\label{proof_coro_1}
In this section, we prove Proposition \ref{corro_nopiror}, which states that the MDA-based track association without prior track information with transformed measurements is equivalent to that with raw track measurements. We begin by presenting the preliminary result in the following Lemma \ref{Lemma_3}. 
\begin{Lemma}\label{Lemma_3}
	Let $A_{k,l}$ be a full column rank matrix, and the processed data $\breve{\mathbf{z}}_{k,l}^{(i_{l})}$ be a linear transformation of $\mathbf{z}_{k,l}^{(i_{l})}$. The MLE solution based on raw measurements is the same as that based on linearly transformed measurements, i.e., $\hat{\mathbf{x}}_{k,\text{ML}}^{(\tau),c}=\hat{\mathbf{x}}_{k,\text{ML}}^{(\tau),d}$, 
		% \begin{align*}
		% \hat{\mathbf{x}}_{k,\text{ML}}^{(\tau),c}=\hat{\mathbf{x}}_{k,\text{ML}}^{(\tau),d},
		% \end{align*}
	where
	\begin{align}\label{cen_ML}
	\hat{\mathbf{x}}_{k,\text{ML}}^{(\tau),c}=\argmax_{\mathbf{x}_{k}^{(\tau)}} \prod_{l \in \{l|u(i_{l})=1\}}p(\mathbf{z}_{k,l}^{(i_{l})}|\mathbf{x}_{k}^{(\tau)}),
		\end{align}
	and
	\begin{align}\label{dis_ML}
	\hat{\mathbf{x}}_{k,\text{ML}}^{(\tau),d}=\argmax_{\mathbf{x}_{k}^{(\tau)}} \prod_{l \in \{l|u(i_{l})=1\}}p(\breve{\mathbf{z}}_{k,l}^{(i_{l})}|\mathbf{x}_{k}^{(\tau)}).
	\end{align}
\end{Lemma}
\begin{IEEEproof}
	For a given track hypothesis $(i_{1},\cdots,i_{L})$ and $\mathbf{x}_{k}^{(\tau)}$, the problem (\ref{cen_ML}) is equivalent to the following problem,
	\begin{align}\label{equ_1_change}
	\hat{\mathbf{x}}_{k,\text{ML}}^{(\tau),c}=\argmax_{\mathbf{x}_{k}^{(\tau)}} \prod_{l \in \{l|u(i_{l})=1\}}p(\mathbf{z}_{k,l}^{(i_{l})}|\mathbf{x}_{k}^{(\tau)}).
	\end{align}
	By taking the logarithm of the objective function in problem (\ref{equ_1_change}), the original problem is equivalent to the problem
	\begin{align}
	\hat{\mathbf{x}}_{k,\text{ML}}^{(\tau),c}=\argmax_{\mathbf{x}_{k}^{(\tau)}}\sum_{l \in \{l|u(i_{l})=1\}}f_{k,l}^c(\mathbf{x}_{k}^{(\tau)}),
	\end{align}
	where
	\begin{align}
	f_{k,l}^c(\mathbf{x}_k^{\tau})=-\frac{1}{2}(\mathbf{z}_{k,l}^{(i_{l})}- H_{k,l}\mathbf{x}_{k}^{(\tau)})^{\text{T}} R_{k,l}^{-1}(\mathbf{z}_{k,l}^{(i_{l})}- H_{k,l}\mathbf{x}_{k}^{(\tau)}).
	\end{align}
	Similarly, the problem (\ref{dis_ML}) is equivalent to the following problem,
	\begin{align}
	\hat{\mathbf{x}}_{k,\text{ML}}^{(\tau),d}=\argmax_{\mathbf{x}_{k}^{(\tau)}}\sum_{l \in \{l|u(i_{l})=1\}}f_{k,l}^d(\mathbf{x}_{k}^{(\tau)}),
	\end{align}
	where
	\begin{align}
	\nonumber	f_{k,l}^d(\mathbf{x}_{k}^{(\tau)})&=-\frac{1}{2}(\mathbf{\breve{y}}_{k,l}^{(i_l)}- \breve{H}_{k,l}\mathbf{x}_{k}^{(\tau)})^{\text{T}} \check{R}_{k,l}^{\dagger}(\mathbf{\breve{y}}_{k,l}^{(i_l)}- \breve{H}_{k,l}\mathbf{x}_{k}^{(\tau)})\\
	\nonumber
	&=-\frac{1}{2}(\mathbf{z}_{k,l}^{(i_{l})}- H_{k,l}\mathbf{x}_{k}^{(\tau)})^{\text{T}} A_{k,l}^{\text{T}}(A_{k,l}R_{k,l} A_{k,l}^{\text{T}})^{\dagger}\\
	&\quad \times A_{k,l}(\mathbf{z}_{k,l}^{(i_{l})}- H_{k,l}\mathbf{x}_{k}^{(\tau)}).
	\end{align}
	By (\ref{dagger_lemma}), we have $f_{k,l}^c(\mathbf{x}_{k}^{(\tau)})=f_{k,l}^d(\mathbf{x}_{k}^{(\tau)})$. Moreover, the solution of problem (\ref{cen_ML}) is same as that of problem (\ref{dis_ML}). Lemma \ref{Lemma_3} is proved.
\end{IEEEproof}

Lemma \ref{Lemma_3} shows that the MLE solution based on transformed measurements is equivalent to that based on raw measurements. Based on the results of Lemma \ref{Lemma_3} and the proof of Proposition \ref{equivalence_score}, Corollary \ref{corro_nopiror} is proved by replacing $S$ with $R$ in Appendix \ref{proof_prop1}.


\section{Proof of Proposition \ref{bp_Prop1}}\label{appe_bp_lemma1}
In this section, we show that the BP track association method with raw measurements is equivalent to that with transformed measurements. We first prove that $q(\cdot ; \mathbf{Z}_{k,l}) = q(\cdot ; \breve{\mathbf{Z}}_{k,l})$. When $\underline{r}_{k,l}^{(\tau)} = 0$, from (\ref{bp_q_x_r0}), $q(\underline{\mathbf{x}}_{k,l}^{(\tau)}, 0, a_{k,l}^{(\tau)} ; \mathbf{Z}_{k,l}) = q(\underline{\mathbf{x}}_{k,l}^{(\tau)}, 0, a_{k,l}^{(\tau)} ; \breve{\mathbf{Z}} _{k,l})$. When $\underline{r}_{k,l}^{(\tau)} = 1$,
from (\ref{bp_factor_q_1}), the equation holds when $a_{k,l}^{(\tau)} = 0$. Thus, we just need to consider the case of $\underline{r}_{k,l}^{(\tau)} = 1$ and $a_{k,l}^{(\tau)} \neq 0$, where the ratio of $q(\cdot ; \mathbf{Z}_{k,l})$ and $q(\cdot ; \breve{\mathbf{Z}}_{k,l})$ is 
\begin{align}
	\frac{q(\underline{\mathbf{x}}_{k,l}^{(\tau)}, 1, a_{k,l}^{(\tau)} ; \mathbf{Z}_{k,l})}{q(\underline{\mathbf{x}}_{k,l}^{(\tau)}, 1, a_{k,l}^{(\tau)} ; \breve{\mathbf{Z}} _{k,l})}
	= \frac{p(\mathbf{z}_{k,l}^{(a_{k,l}^{(\tau)})}|\underline{\mathbf{x}}_{k,l}^{(\tau)})}{p(\breve{\mathbf{z}}_{k,l}^{(a_{k,l}^{(\tau)})}|\underline{\mathbf{x}}_{k,l}^{(\tau)})} \frac{p_{f_{l}}(\breve{\mathbf{z}}_{k,l}^{(a_{k,l}^{(\tau)})})}{p_{f_{l}}(\mathbf{z}_{k,l}^{(a_{k,l}^{(\tau)})})}.
\end{align}
By using Lemma \ref{Lemma_1} and Lemma \ref{Lemma_2}, we have
\begin{align}
	\frac{q(\underline{\mathbf{x}}_{k,l}^{(\tau)}, 1, a_{k,l}^{(\tau)} ; \mathbf{Z}_{k,l})}{q(\underline{\mathbf{x}}_{k,l}^{(\tau)}, 1, a_{k,l}^{(\tau)} ; \breve{\mathbf{Z}} _{k,l})}  = \frac{\sqrt{\prod_{i=1}^{m}e_{i}}}{\sqrt{|R|}\sqrt{| A^{\text{T}}A|}}  = 1,
\end{align}
where $e_i$, $i=1,\cdots,m$ are the nonzero eigenvalues of $AR A^{\text{T}}$, and the last equation holds due to (\ref{A24}) in Appendix~\ref{proof_prop1} by replacing $S$ with $R$. Then, the following equation holds, 
\begin{align}\label{pf_bp_q}
	q(\underline{\mathbf{x}}_{k,l}^{(\tau)}, \underline{r}_{k,l}^{(\tau)}, a_{k,l}^{(\tau)} ; \mathbf{Z}_{k,l}) = q(\underline{\mathbf{x}}_{k,l}^{(\tau)}, \underline{r}_{k,l}^{(\tau)}, a_{k,l}^{(\tau)} ; \breve{\mathbf{Z}} _{k,l}).
\end{align}
Let $\beta^c(a_{k,l}^{(\tau)})$ denote the message (\ref{bp_2_meas}) with raw measurements and $\beta^d(a_{k,l}^{(\tau)})$ denote the message (\ref{bp_2_meas}) with transformed measurements.
We have
\begin{align}
	\nonumber
	\beta^c&(a_{k,l}^{(\tau)}) - \beta^d(a_{k,l}^{(\tau)}) \\
	\nonumber
	=& \sum_{r_{k,l}^{\tau} \in \{0,1\}} \int (q(\underline{\mathbf{x}}_{k,l}^{(\tau)}, \underline{r}_{k,l}^{(\tau)}, a_{k,l}^{(\tau)} ; \mathbf{Z}_{k,l})  \\
	\nonumber
	&\qquad\qquad - q(\underline{\mathbf{x}}_{k,l}^{(\tau)}, \underline{r}_{k,l}^{(\tau)}, a_{k,l}^{(\tau)} ; \breve{\mathbf{Z}} _{k,l}))  \tilde{f}_{l-1}(\underline{\mathbf{x}}_{k,l}^{(\tau)},\underline{r}_{k,l}^{(\tau)}) d \underline{\mathbf{x}}_{k,l}^{(\tau)}\\
	=& 0,
\end{align}
i.e., $\beta^c(a_{k,l}^{(\tau)}) = \beta^d(a_{k,l}^{(\tau)})$. Similarly, we can prove that
\begin{align}
	v(\overline{\mathbf{x}}_{k,l}^{(i_{l})},\overline{r}_{k,l}^{(i_{l})},b_{k,l}^{(i_{l})};\mathbf{z}_{k,l}^{(i_{l})}) = v(\overline{\mathbf{x}}_{k,l}^{(i_{l})},\overline{r}_{k,l}^{(i_{l})},b_{k,l}^{(i_{l})};\breve{\mathbf{z}}_{k,l}^{(i_{l})}),
\end{align}
and the message (\ref{bp_mess_xi}) with raw measurements is equal to the message (\ref{bp_mess_xi}) with transformed measurements, i.e., $\xi^{c} (b_{k,l}^{(i_l)}) = \xi^{d} (b_{k,l}^{(i_l)})$. From (\ref{ite_1_v})--(\ref{ite_init}), we know that the inputs of the step of iterative data association are $\beta(a_{k,l}^{(\tau)})$ and $\xi(b_{k,l}^{(i_l)})$. Since the inputs $\beta^c(a_{k,l}^{(\tau)})$ and $\xi^{c} (b_{k,l}^{(i_l)})$ are equal to the inputs $\beta^d(a_{k,l}^{(\tau)})$ and $\xi^{d} (b_{k,l}^{(i_l)})$, the outputs of iterative data association
\begin{align}
	\label{pf_bp_kappa}
	\kappa^{c}(a_{k,l}^{(\tau)})&=\kappa^{d}(a_{k,l}^{(\tau)}),\\
	\label{pf_bp_iota}
	\iota^c (b_{k,l}^{(i_l)})&=\iota^d (b_{k,l}^{(i_l)}),
\end{align}
where $\{\kappa^{c}(a_{k,l}^{(\tau)}),\iota^c (b_{k,l}^{(i_l)})\}$ and $\{\kappa^{d}(a_{k,l}^{(\tau)}),\iota^d (b_{k,l}^{(i_l)})\}$ are the outputs of the step of iterative data association with raw measurements and transformed measurements, respectively. 

Next, we prove that in the step of measurement update, the messages $\gamma^{c}(\underline{\mathbf{x}}_{k,l}^{(\tau)},\underline{r}_{k,l}^{(\tau)})$ and $\varsigma^{c} (\overline{\mathbf{x}}_{k,l}^{(i_{l})},\overline{r}_{k,l}^{(i_{l})}) $ with raw measurements are equal to the messages $\gamma^{d}(\underline{\mathbf{x}}_{k,l}^{(\tau)},\underline{r}_{k,l}^{(\tau)})$ and $\varsigma^{d} (\overline{\mathbf{x}}_{k,l}^{(i_{l})},\overline{r}_{k,l}^{(i_{l})})$ with transformed measurements, respectively,  i.e.,
\begin{align}
\gamma^{c}(\underline{\mathbf{x}}_{k,l}^{(\tau)},\underline{r}_{k,l}^{(\tau)})&=\gamma^{d}(\underline{\mathbf{x}}_{k,l}^{(\tau)},\underline{r}_{k,l}^{(\tau)}), \\
\varsigma^{c}  (\overline{\mathbf{x}}_{k,l}^{(i_{l})},\overline{r}_{k,l}^{(i_{l})})&=\varsigma^{d}  (\overline{\mathbf{x}}_{k,l}^{(i_{l})},\overline{r}_{k,l}^{(i_{l})}).
\end{align}
For survived targets, the following equations hold from Equations (\ref{pf_bp_q}) and (\ref{pf_bp_kappa}):
\begin{align}
	q(\underline{\mathbf{x}}_{k,l}^{(\tau)}, \underline{r}_{k,l}^{(\tau)}, a_{k,l}^{(\tau)} ; \mathbf{Z}_{k,l}) &= q(\underline{\mathbf{x}}_{k,l}^{(\tau)}, \underline{r}_{k,l}^{(\tau)}, a_{k,l}^{(\tau)} ; \breve{\mathbf{Z}} _{k,l}),\\
	\kappa^{c}(a_{k,l}^{(\tau)})&=\kappa^{d}(a_{k,l}^{(\tau)}).
\end{align}
From (\ref{bp_meas_up}) and (\ref{bp_meas_up_2}), we conclude that $\gamma^{c}(\underline{\mathbf{x}}_{k,l}^{(\tau)},\underline{r}_{k,l}^{(\tau)})=\gamma^{d}(\underline{\mathbf{x}}_{k,l}^{(\tau)},\underline{r}_{k,l}^{(\tau)})$.
Similarly, for new targets, 
\begin{align}
	v(\overline{\mathbf{x}}_{k,l}^{(i_{l})},\overline{r}_{k,l}^{(i_{l})},b_{k,l}^{(i_{l})};\mathbf{z}_{k,l}^{(i_{l})}) &= v(\overline{\mathbf{x}}_{k,l}^{(i_{l})},\overline{r}_{k,l}^{(i_{l})},b_{k,l}^{(i_{l})};\breve{\mathbf{z}}_{k,l}^{(i_{l})}),\\
	\iota^{c} (b_{k,l}^{(i_l)})&=\iota^{d} (b_{k,l}^{(i_l)}).
\end{align}
From (\ref{bp_meas_up_3}) and (\ref{bp_meas_up_4}), we have $\varsigma^{c}  (\overline{\mathbf{x}}_{k,l}^{(i_{l})},\overline{r}_{k,l}^{(i_{l})})=\varsigma^{d}  (\overline{\mathbf{x}}_{k,l}^{(i_{l})},\overline{r}_{k,l}^{(i_{l})})$.

Finally, the beliefs of survived targets $\tilde{f}_l^c(\underline{\mathbf{x}}_{k,l}^{(\tau)},\underline{r}_{k,l}^{(\tau)}) = \tilde{f}_l^d(\underline{\mathbf{x}}_{k,l}^{(\tau)},\underline{r}_{k,l}^{(\tau)})$ hold from (\ref{bp_appx}) and (\ref{bp_appx_2}), and the beliefs of new targets $\tilde{f}_l^c(\overline{\mathbf{x}}_{k,l}^{(i_l)},\overline{r}_{k,l}^{(i_{l})}) = \tilde{f}_l^d(\overline{\mathbf{x}}_{k,l}^{(i_l)},\overline{r}_{k,l}^{(i_{l})})$ hold from (\ref{bp_belief_new_1}) and (\ref{bp_belief_new_2}). Proposition \ref{bp_Prop1} is proved.

% \section{Algorithm}
% \begin{figure}[H]
% 	\centering
% 	\includegraphics[width= 0.49\textwidth]{fig/algorithm_1.pdf}
% 	\caption{}
% 	\label{appd_fig_1}
% \end{figure}

% \section{Algorithm}
% \begin{figure}[H]
% 	\centering
% 	\includegraphics[width= 0.49\textwidth]{fig/algorithm_1.pdf}
% 	\caption{}
% 	\label{appd_fig_2}
% \end{figure}
%
% <OR> manually copy in the resultant .bbl file
% set second argument of \begin to the number of references
% (used to reserve space for the reference number labels box)
% \begin{thebibliography}{1}

% \bibitem{IEEEhowto:kopka}
% H.~Kopka and P.~W. Daly, \emph{A Guide to \LaTeX}, 3rd~ed.\hskip 1em plus
%   0.5em minus 0.4em\relax Harlow, England: Addison-Wesley, 1999.

% \end{thebibliography}

% biography section
%
% If you have an EPS/PDF photo (graphicx package needed) extra braces are
% needed around the contents of the optional argument to biography to prevent
% the LaTeX parser from getting confused when it sees the complicated
% \includegraphics command within an optional argument. (You could create
% your own custom macro containing the \includegraphics command to make things
% simpler here.)
%\begin{IEEEbiography}[{\includegraphics[width=1in,height=1.25in,clip,keepaspectratio]{mshell}}]{Michael Shell}
% or if you just want to reserve a space for a photo:

% \begin{IEEEbiography}{Michael Shell}
% Biography text here.
% \end{IEEEbiography}

% if you will not have a photo at all:
% \begin{IEEEbiographynophoto}{John Doe}
% Biography text here.
% \end{IEEEbiographynophoto}

% insert where needed to balance the two columns on the last page with
% biographies
%\newpage

% \begin{IEEEbiographynophoto}{Jane Doe}
% Biography text here.
% \end{IEEEbiographynophoto}

% You can push biographies down or up by placing
% a \vfill before or after them. The appropriate
% use of \vfill depends on what kind of text is
% on the last page and whether or not the columns
% are being equalized.

%\vfill

% Can be used to pull up biographies so that the bottom of the last one
% is flush with the other column.
%\enlargethispage{-5in}



% that's all folks
\end{document}


