
\documentclass[10pt,onecolumn]{article}



%%%% Standard Packages
\usepackage{chemfig}
\usepackage{float}
\usepackage{xr} 
\usepackage{hyperref}
\usepackage{gensymb}
\usepackage{lscape}
\usepackage{array,booktabs,longtable}
\newcolumntype{C}[1]{>{\centering\arraybackslash}m{#1}}
\usepackage{chemnum}
\usepackage{tabularx}
\usepackage{graphicx}
\usepackage{authblk}
\newcommand{\coo}{CO\textsubscript{2}}
\newcommand{\degc}{$^{\circ}\textrm{C}$}
\newcommand{\ndatafull}{130}
\newcommand{\ndata}{128}
\newcommand{\ntrain}{99}
\newcommand{\nval}{11}
\newcommand{\ntest}{20}
\newcommand{\ntesttotal}{8}
\newcommand{\nzinc}{17000}
\newcommand{\codedataurl}{\url{https://github.com/IBM/Carbon-capture-fingerprint-generation}}
\newcommand{\ac}{absorption capacity}
\newcommand{\oir}{observed initial rate}
\newcommand{\si}{Supporting Information}
\newcommand{\AC}{Absorption capacity}
\newcommand{\OIR}{Observed initial rate}
\newcommand{\SI}{Supporting information}
\newcommand{\jm}[1]{{\color{red}JmcD: #1}}
\newcommand{\fsc}[1]{{\color{orange}FSC: #1}}
\newcommand{\gp}{Gaussian Process}
\newcommand{\ab}{Adaboost}
\newcommand{\svm}{Support Vector Machine}
\newcommand{\dt}{Decision Tree}
\newcommand{\et}{Extra Trees}

\raggedbottom

\title{Machine Guided Discovery of Novel Carbon Capture Solvents}

\author[1,2]{James L. McDonagh}
\author[4]{Benjamin H. Wunsch}
\author[3]{Stamatia Zavitsanou}
\author[1]{Alexander Harrison}
\author[4]{Bruce Elmegreen}
\author[4]{Stacey Gifford}
\author[4]{Theodore van Kessel}
\author[1]{Flaviu Cipcigan}


\affil[1]{IBM Research Europe - UK, Hartree Centre, SciTech Daresbury, Warrington, Cheshire WA4 4AD, UK}
\affil[2]{Current address: Ladder Therapeutics doing business as Serna Bio, Lab F37, Stevenage Bioscience Catalyst, Gunnels Wood Road, Stevenage, Hertfordshire, SG1 2FX, UK}
\affil[3]{University of Oxford, Physical and Theoretical Chemistry Laboratory, Oxford, 610101, Oxfordshire, UK}
\affil[4]{IBM Research, IBM T.J. Watson Research Center, Yorktown Heights 10598, New York, USA}

\begin{document}

\maketitle





J.M. led the computational work and B.W. led the experimental work; these two authors contributed equally to this work \\

To whom correspondence should be addressed. E-mail: J.M. james.mcdonagh@serna.bio, B.W. bhwunsch@us.ibm.com

\section{\textbf{Abstract}}
The increasing importance of carbon capture technologies for deployment in remediating \coo{} emissions, and thus the necessity to improve capture materials to allow scalabiltiy and efficiency, faces the challenge of materials development, which can require substantial costs and time.  Machine learning offers a promising method for reducing the time and resource burdens of materials development through efficient correlation of structure-property relationships to allow down-selection and focusing on promising candidates.  Towards demonstrating this, we have developed an end-to-end "discovery cycle" to select new aqueous amines compatible with the commercially viable acid gas scrubbing carbon capture. We combine a simple, rapid laboratory assay for \coo{} absorption with a machine learning based molecular fingerprinting model approach.  The prediction process shows 60\% accuracy against experiment for both material parameters and 80\% for a single parameter on an external test set. The discovery cycle determined several promising amines that were verified experimentally, and which had not been applied to carbon capture previously.  In the process we have compiled a large, single-source data set for carbon capture amines and produced an open source machine learning tool for the identification of amine molecule candidates (https://github.com/IBM/Carbon-capture-fingerprint-generation).   

\section{\textbf{Introduction}}
Anthropogenic climate change will be a major world-wide concern of this century.\cite{wu2020solvent} Greenhouse gases such as \coo{}, methane and nitrous oxides (NO$_{x}$) are major contributing species to the climate emergency of global warming. Of these gases \coo{} is rightly receiving major attention, as it is the largest fraction of green house gases emitted and long lasting in the atmosphere.\cite{olivier2017trends} Recent reports\cite{Carrington2022Revealed} show that major oil and gas reserves continue to be exploited, with new fossil fuel burning infrastructure still being built.\cite{wu2020solvent} These activities add to the committed emissions from existing infrastructure that endanger climate targets.\cite{Tong2019} 

Carbon Capture, Utilization and Storage (CCUS) technologies, specifically those that target \coo{} emissions, have the potential for negative emissions and are likely to be necessary in meeting the Paris climate accord.\cite{wu2020solvent,bruhn2016separating} Although within CCUS there is growing interest in direct air capture projects to remove \coo{} from the air, this technology and its commercial deployment remains in its infancy, leaving point-source capture from post-combustion or heavy-emission industries the primary target for practical remediation \cite{Realmont2019}. CCUS technology can allow decarbonization of existing infrastructure and hard-to-abate emissions.\cite{iea2020} Recent publications have estimated there are 87 planned CCUS plants between 2020–2030\cite{iogp2020}.

Current CCUS technologies span carbon capture solvents, polymers or other membranes, and solid state materials. Of these technologies carbon capture solvents are the most mature, with on-going commercial usage and planned developments.\cite{carbonclean2023solvent, chao2020post, bui2018carbon} Carbon capture solvent technology is dominated by amine based solvents. Amines are common organic bases derived from ammonia by replacement of hydrogen with larger organic groups. Primary and secondary amines react with \coo{} through nucleophilic addition to form a carbamic acid (zwitterion intermediate) which is then further deprotenated by an additional amine (or base) to the carbmate; a process which leads to a 2:1 amine-to-\coo{} capture ratio (Figure \ref{scheme:prim_sec}). Tertiary amines cannot form the carbamic intermediate; instead they act on dissolved \coo{} as Bronsted bases to generate carbonates, allowing one amine to capture one \coo{} molecule (Figure \ref{scheme:tert}).\cite{said2020unified,puxty2009carbon, kenarsari2013review, yang2017computational} 

\begin{figure}[H]
\centering
\schemestart
 \chemname{\chemfig{HNR_1R_2}}{}
 \+{0pt,1em}
 \chemname{\chemfig{CO_2}}{}
 \arrow{<=>}
 \chemname{\chemfig{R_1R_2\chemabove{N}{\scriptstyle\oplus}HCO\chemabove{O}{\scriptstyle\ominus}}}{}
\schemestop
\label{scheme:prim_sec}
\end{figure}

\begin{figure}[H]
\centering
\schemestart
 \arrow{<=>[$\text{HNR}_{1}\text{R}_{2}$]}
  \chemname{ [\chemfig{H_{2}\chemabove{N}{\scriptstyle\oplus}R_1R_2}] }{}
  \chemname{[\chemfig{R_1R_2NCO\chemabove{O}{\scriptstyle\ominus}}]}{}
 \schemestop
 \caption{Primary and secondary amine general reaction scheme.}
 \label{scheme:prim_sec}
\end{figure}

\begin{figure}[H]
\centering
\schemestart
 \chemname{\chemfig{NR_1R_2R_3}}{}
 \+
 \chemname{\chemfig{CO_2}}{}
 \+
  \chemname{\chemfig{H_{2}O}}{}
 \arrow{<=>}
  \chemname{[\chemfig{H\chemabove{N}{\scriptstyle\oplus}R_1R_2R_3}]}{}
 \chemname{[\chemfig{HOOC\chemabove{O}{\scriptstyle\ominus}}]}{}
\schemestop
 \caption{Tertiary amine general reaction scheme.}
\label{scheme:tert}
\end{figure}


Post-combustion point-source capture is not widely adopted in any major carbon intensive industry (e.g. power generation, cement, heavy industry). Expanded adoption hinges on lower cost-per-ton-\coo{}, which can be improved through the amine solvent chemistry.  There is still a need for solvent development.\cite{Bernhardsen2017, raksajati2018comparison, raksajati2018solvent, orlov2022computational} Commonly used  amines for carbon capture include monoethanolamine (MEA), diethanolamine (DEA), piperazine (PZ), and methyldiethanolamine (MDEA), with state-of-the-art formulations often including these species. However, these solvents have drawbacks that include (but are not limited to) capture efficiency, thermal degradation, corrosion and, significantly, heat of regeneration.\cite{chao2020post} Substantial work has been made on screening and developing new carbon capture amine systems.\cite{orlov2022computational, mcdonagh2022chemical}. Experimental screening has provided diverse data sets of organic amines and alkaline solvents for carbon capture \cite{Puxty2009, Porcheron2011, Singh2007, Singh2009, Hussain2011, Conway2012, Conway2014, Chowdhury2013, Xiao2016, Yang2016, Bernhardsen2017, Kessler2018, Kessler2021}.  In addition to this, amines are also being explored as functionalization systems in solid state carbon capture for applications including DAC.\cite{hamdy2021application} Machine learning has been used to reconcile discrepancies in published carbon capture data to improve accuracy of modeling \cite{Suleman2017}.  In general, having a large data set analyzed with the same technique allows for more consistent cross-comparison of materials, enabling evaluation of structure-property relationships and down-selection for superior performance and further testing.

In this work, we detail a combined machine learning and  experimental loop for the discovery of several new promising amine candidate molecules for carbon capture, using an in-house generated data set of \coo{} absorption by various amine and organic nitrogen compounds.  The candidate molecules have been identified through computational screening followed by rapid laboratory testing. We detail here our screening methodology, predictive methods and experimental procedures. This process has also allowed us to create the largest set of single amine carbon capture performance metrics from a common experimental source available in the open literature, to the best of the authors' knowledge. This set comprises \ndatafull{} molecules. Some of the molecules in this data set overlap with other existing publications and find generally positive agreement in the property determinations. The majority of the data set however, appears to be novel. 

\section{Results}

The process to survey amines for \coo{} capture reported here combines computational and experimental methods in an end-to-end discovery cycle and is shown schematically in Figure \ref{fig:hl_methods}.  We have previously developed a chemical fingerprinting technique that is successful at classifying and correlating molecules based on molecular fragments, showing improvement over other classification methods in the carbon capture space.\cite{mcdonagh2022chemical}  Based on this technique, we set out to develop the discovery cycle to train, validate and predict amine performance using in-house testing for all data generation and validation. 

\begin{figure}[H]
\centering
\includegraphics[width=\linewidth]{images/methods/high_level_discovery.png}
\caption{Outline of molecular discovery cycle for carbon capture amines.}\label{fig:hl_methods}
\end{figure}


\subsection{Validation of Classifier Models}\label{ss:val}

To establish the predictive capabilities of the system required first training and validating the machine learning models (classifiers). We used our fingerprinting technique to train binary classifiers on the molecule's performance based upon two key metrics of carbon capture: \textit{\ac{}} and \textit{\oir{}}.   The \ac{} is the molar ratio of \coo{} absorbed to amine moieties present (note: all capacities are \textit{per amine}, allowing comparison between mono- and polyamine molecules).  The \oir{} is the fitted pseudo-first order rate constant.  These metrics are central material properties for assessing how well a given amine captures \coo{}. The binary classifier outputs a 1 if a molecule's property is higher than a threshold and a 0 otherwise; thresholds are discussed in Methods and Materials. An initial training data set of \ntrain{} amines was generated using an in-house, rapid analysis instrument (see Methods and Materials) to assess the material properties.  A further validation set of \nval{} amines was used to assess the classifier performance.

We trained 10 classifiers for each metric and assessed their performance. Table \ref{tab:classifiers} presents the validation results for the top performing classifiers.  Results for all classifiers are given in \si{} section 4.  It is seen that for both \ac{} and \oir{} the classifiers achieve high accuracy ($>70\%$), with the \oir{} being the most accurately predicted of the two properties.  For \ac{} the sensitivity (rate of correct predictions for the positive class) and specificity (rate of correct predictions for the negative class) vary between the three top classifiers. For both \gp{} and \svm{} the classifiers are stronger in the prediction of the positive class, whilst for \ab{} it is stronger in the prediction of the negative class. Overall there is a marginally improved performance from \ab{} which is shown by the higher Receiver Operating Characteristic Area Under the Curve (ROC AUC) and Matthews Correction Coefficent (MCC) scores compared to the \gp{} and \svm{} models, and for this reason we selected the \ab{} model as our top capacity classifier.  For the \oir{} prediction we find that all three classifiers provide the same predictive performance to two decimal places, and have marginally improved metrics compared to the \ac{} metrics. We found from application of all three models to the external test set that the Gaussian Process model provides the best performance and was therefore selected as the presented model in this work. Results for the other models are provided in the \si{} section 4. Of note, all models show an accurate prediction of the negative class predicting all negative class molecules (5 of 11) to be in the negative class. The positive class prediction performance is less effective with the classifiers achieving 67\% correct positive class predictions.

\begin{table}[h!]
    \caption{Validation metrics from the op three models for predicting amine carbon capture performance. The full set of classifier performance metrics can be found in the \si{} section 4.  GP is Gaussian Process, AB is AdaBoost, SP is Support Vector Machine, DT is Decision Tree, XT is Extra Trees.  The training metrics are AC = \ac{}, OIR = \oir{}. ROC AUC is receiver operator curve, area under the curve, MCC is Matthews correlation constant.}
    \begin{tabular}{lrrrrr}
    \toprule
    Classifier &  Accuracy &  Sensitivity &  Specificity &  ROC AUC &       MCC \\
    \midrule
    GP\textunderscore{}AC &  0.73 &     0.83 &          0.60 & 0.72 &  0.45 \\
    AB\textunderscore{}AC &  0.73 &     0.67 &          0.80 & 0.73 &  0.47 \\
    SP\textunderscore{}AC &  0.73 &     0.83 &          0.60 & 0.72 &  0.45 \\
    GP\textunderscore{}OIR &  0.82 &     0.67 &          1.00 & 0.83 & 0.69 \\
    DT\textunderscore{}OIR &  0.82 &     0.67 &          1.00 & 0.83 & 0.69 \\
    XT\textunderscore{}OIR &  0.82 &     0.67 &          1.00 & 0.83 & 0.69 \\
    \bottomrule
    \end{tabular}
    \label{tab:classifiers}
\end{table}


\subsection{Prediction and Experimental Testing of Amine \coo{} Capture}

To assess the predictions of the amines, we used a validation set of \nval{} molecules to select the best performing models following training. These results are presented in table \ref{tab:classifiers} for the top three classifiers for each property. Having selected the top classifers, we next applied these to a test set of \ntest{} molecules. These results are shown in table \ref{tab:test_data}. These molecules were selected from our \ndatafull{} data set pseudo-randomly in that we required the sets to be approximately evenly split between the positive and negative classes for both properties. These compounds were tested experimentally with the rapid \coo{} absorption instrument and the data used to (1) compare the predictive performance of the classifiers and, (2) highlight molecules with high \ac{} and \oir{}.

Figure \ref{fig:computational_predictions_probability} shows the  prediction results of \ac{} and \oir{}, and their assessment against the experimental values.  These axes are discretized based on the probability of a molecule being in the positive class, P(active). This gives a natural zero to one axes with 0.5 marking the boundary between the classes i.e. $< 0.5$ negative class and $\geq 0.5$ positive class.  The data are color coded based on if the predicted performance matches the experimental result. Panels A-D in Figure \ref{fig:computational_predictions_probability} show the chemical structures of the molecules predicted to be in each quadrant of the 2D plane.  Molecules in panel C are predicted to be unpromising for both \ac{} and \oir{} (panel c). Panels A and D show molecules predicted to be promising respectively for \oir{} or \ac{}, while panel B shows those molecules predicted to perform well for both metrics.  The prediction comparisons of the molecules for each quadrant, as well as the molecules in each of these categories based upon the experimental values of the properties, are shown in the \si{} Figures 4 and 5, respectively. Table \ref{tab:test_data} shows the predicted metrics for the displayed classifiers.

Considering the overall performance when combining these two classifiers we see that 60\% (4 of 7) of the molecules in the positive class for both properties are correctly predicted within the same quadrant as the experimental data. The remaining three molecules experimentally determined to be in this class are correctly predicted as promising for one of the two properties. We see a particularly strong performance in identifying the poor performing molecules; approximately 85\% (6 of 7) are found in the correct quadrant for both properties. In terms of screening for materials development this allows one to rapidly deprioritise molecules for testing with reasonable confidence. Of note is the promising performance obtained for these computationally efficient  statistical models with a limited training data set. We did not benchmark deep learning embeddings here largely due to the data set size for model fine tuning, and the context of this work being focused upon developing a discovery cycle. However, a recent benchmark has shown that molecular fingerprints often achieve better performance than deep learning embeddings for QSAR tasks such as the ones performed here \cite{Sabando2021UsingME}. The majority of top candidates predicted successfully are amino alcohols, either primary or secondary.  Although several amines have a beta hydroxyl, it seems the separation of the alcohol and amine groups is not a strong constraint given the examples of proply and pentyl derivatives. 

\begin{figure*}[h]
\centering
\includegraphics[width=0.7\textwidth]{images/results/prediction_plot_2.png}
\caption{Plots of the probability, from our classifier models. The \ac{} is predicted using the Adaboost classifier and \oir{} is predicted using the Gaussian Process. Panel A shows molecules with good probability for a promising \oir{} but unpromising \ac{}. Panel B shows molecules predicted well for both metrics. Panel C shows molecules with a low probability of either property being promising. Panel D molecules with good probability for a promising \ac{} but unpromising \oir{}.}\label{fig:computational_predictions_probability}
\end{figure*}

\begin{table}[h!]
    \tiny
    \centering
    \caption{Table of the predictions from the selected models for the external test set data. AC is \ac{} and OIR is \oir{}. Results for all models are given in the \si{}.}
    
    \label{tab:test_data}
    \begin{tabular}{llrrrrr}
    \toprule
    Property &                Model &  Accuracy &  Sensitivity &  Specificity &   ROC AUC &    MCC \\
    \midrule
    AC &        AdaBoost &  0.73 &     0.67 &          0.80 & 0.73 &  0.47 \\
     OIR & GaussianProcess &      0.75 &         0.58 &         1.00 &     0.79 & 0.60 \\
    \bottomrule
    \end{tabular}
\end{table}



\subsection{Data Set Exploration}

Figure \ref{fig:experimental_dataset_graphs} collects the \ndatafull{} experimentally measured  material properties for the amines used at the different computational stages of training the machine learning model (blue), validating (green) and testing  performance predictions (red). The molecules used in training and validation where selected as a sub-set from a wider computational screening and identification as outlined in sub-section \textcolor{red}{\ref{subsec:identification}}. The molecules in Figure \ref{fig:experimental_dataset_graphs} A and B are labelled by index from Table 1 in the \si{} which also gives the molecules IUPAC names and InChIkey's as identifiers. We provide all molecules SMILES, InChI, InChIkeys, IUPAC names and measure properties in the form of a csv file together with the fingerprint generation method at \codedataurl{} and a summary of the data used in the \si{} Table 1.  The data represents a diverse set of amines and organic nitrogen species, as seen in \si{} Figure 1.  Structurally, the data contains primary, secondary and tertiary amines, as well as polyamine combinations and other carbon-nitrogen moieties (\si{}, Figure 2).  Predominately the amines are aliphatic, but there is a small sub-set of aromatic compounds, all of which have primary amines.

Figure \ref{fig:experimental_dataset_graphs} (panels A and B) shows there is a wide range of values for the \ac{} and \oir{}, showing amines cover a wide parameter space in which to explore candidates.  Focusing on \ac{}, four clusters of values are observed (see \si{} Figure 3): species with \ac{} above 0.9 mol(\coo{})/mol(amine), between 0.45 and 0.7 mol(\coo{})/mol(amine), between 0.2 and 0.45 mol(\coo{})/mol(amine) and those below 0.2 mol(\coo{})/mol(amine). Based on the amine types, the two higher-value clusters correspond to tertiary/frustrated secondary amines (carbonate reaction mechanism) and primary/secondary amines (carbamate reaction mechanism).  The low \ac{} cluster represents poor or non-reaction species.  The 0.2 - 0.4 mol(\coo{})/mol(amine) is interesting in that it represents predominately primary and secondary amines that have reduced reactivity.

\begin{figure*}[h]
\centering
\includegraphics[width=0.7\textwidth]{images/results/figure_discovery_cycle_compiled2.png}
\caption{Overview of the experimental data. Panel A: highlights the range of amine \ac{} explored in this data set. Panel B: Expounds upon the range of \oir{} explored in this data set. Panel C: Explores the relation between the two properties \ac{} and \oir{}; the ideal molecule would be positioned in the top right corner. P
anel D: Highlights the known trend of \ac{} with pKb which our data follows. Here we have predicted the pKb with the OPERA tools from the EPA.} \label{fig:experimental_dataset_graphs}
\end{figure*}

Panel C in Figure \ref{fig:experimental_dataset_graphs}  displays the concurrent behaviour of the molecules over both properties. It is clear in several cases that it is possible to achieve a high \ac{} but the \oir{} remains low. On this plane the ideal molecule would be at the top right corner. The molecule monoethanolamine (MEA) (a.k.a. 2-aminoethanol), a common standard for solvent based carbon capture studies, is highlighted using a star in Figure \ref{fig:experimental_dataset_graphs} C \cite{romeo2020comparative}. We see that in this space there are several molecules which improve on MEA for both properties concurrently. Finally, we turn to pKb. The pKa of the conjugate acid has been used as a indicator for carbon capture performance for many years. Here we have used the OPERA toolkit from the EPA to predict the molecule's basicity in terms of pKb. We see the expected trend of higher basicity correlating with higher capacity as has been seen previously \cite{Puxty2009}. 

Some of the most commonly used solvents, either independently or as part of a formulation, such as MEA and DEA \cite{romeo2020comparative} can also be used here as a benchmark. Doing this, we find 25 and 9 molecules in this data set which supersede both the \ac{} and \oir{} of these commonly used solvents, respectively (see \si{}). Of particular note within this set is (2R,3R,4R,5S)-6-(methylamino)hexane-1,2,3,4,5-pentol (N-methyl-D-glucamine) (Figure \ref{fig:promising_molecules}, image 4), which is a known additive for pharmaceuticals.  As an aminated sugar derivative, it suggests exploring other saccharides scaffolds for structure-function relationships in \coo{} capture. This molecule was also predicted to be the least toxic by the OPERA toolkit based on the CATMoS LD50 predictions. Potentially, such a molecule can provide a route to low carbon production of carbon capture agents. Of note also is 1-aminocyclohexan-1-ol (as a racemic mixture) which was found to have the highest relative \ac{} and \oir{} out of the data set, making it the optimal compromise molecule within in the data set.  That the MEA substructure (a well-known motif in carbon capture amines) is part of this compound, but with more constraint on its bond configurations due to the aliphatic cycle, it is of interest to consider if this preconditions the amine to a more favorable conformation for reacting with \coo{}; something a future study of all stereoisomers could resolve.

Finally, it is observed that, from the outset, the screening suggested some molecules with unusually high \oir{}. These three molecules are (1S,2R)-cyclohexane-1,2-diamine, (1R,2R)-cyclohexane-1,2-diamine and 2,3,4,6,7,8-hexahydropyrrolo[1,2-a]pyrimidine (1,5-diazabicyclo[4.3.0]non-5-ene, a.k.a. DBN) going from largest to smallest rate. Figure \ref{fig:promising_molecules}, images 1 - 3 shows the chemical structures of these molecules.  DBN is a well established organic base extensively used in synthesis, and is known to react with \coo{}.  That the diaminocyclohexane stereoisomers both show higher \oir{} compared to the majority of compounds could suggest a cooperatively effect due to the close proximity of the amines, though the differences in the stereochemistry may have a subtle effect on the coordination of \coo{} and thus the rate, given that the \textit{cis} isomer is moderately faster.

\begin{figure}[H]
\centering
\includegraphics[width=\linewidth]{images/results/molecules_notable_validation.png}
\caption{Molecules 1 - 3 represent the highest reaction rate molecules we have encountered. Molecule 4 is a sugar derivative which shows promising carbon capture capabilities. Molecule 5 (which was tested as a commercially available racemic mixture) represents the best compromise between \ac{} and \oir{} in the current data set.} \label{fig:promising_molecules}
\end{figure}



\section{Discussion}
New small molecule amines for carbon capture have been identified through a process of machine-learning driven computational screening coupled with laboratory scale evaluation of \ac{} and \oir{}. The results highlight the potential of data driven computational techniques in carbon capture chemistry. We have shown the use of a rapid-analysis instrument to assess \coo{} absorption in molecules and generate a sufficient data set for training successful, predictive models.  For a given test, only 200$\mu$L of sample is used, requiring 5-30 min, allowing a 100+ data set to be built up in a few weeks on a single instrument. The data set itself forms the largest single-source set of amines tested for carbon capture to date. 

We have shown that molecules in our training set, such as (2R,3R,4R,5S)-6-(methylamino)hexane-1,2,3,4,5-pentol, cyclohexane-1,2-diamine and 2-aminocyclohexan-1-ol have potential applications in carbon capture solvents. It is especially interesting to see molecules such as (2R,3R,4R,5S)-6-(methylamino)hexane-1,2,3,4,5-pentol which could be produced from waste biomass materials, as this is a cheap and likely low carbon feed stock material. Of note is the that cyclohexane-1,2-diamine  shows remarkable \oir{}. Although it exhibited an average \ac{} molecules such as this could be very useful in formulated blends to enhance the rate of \coo{} reactions.

The current work shows the value of data in the CCUS field. Using robust, widely deployable and computationally efficient models trained on fairly small data sets, has enabled promising screening of small amine based carbon capture solvents. This field remains relatively data poor, in part from the fact that much of the current formulation development has been done proprietorially for commercial use. The current work shows the potential value in opening \coo{} capture chemistry and formulation data and allowing computational modelling to be more extensively applied to these vital chemical technologies.  A centralized and curated source of carbon capture molecules could greatly improve machine learning and AI assisted development.

The structural space of organic amines is vast, and the results shown here reinforce the notion that there is still potential to locate new  molecules which can be utilized for carbon capture.  Of necessity for future work is to expand the training metrics to include material properties that become relevant at the industrial-scale deployment, including viscosity as a function of \coo{} loading, surface tension, corrosion, and oxidation and other degradation mechanisms, so that the large structure space can be probed more thoroughly.  Importantly is the inclusion of regeneration energy as an optimization parameter. In this work we have focused on using these molecules as independent, active capture systems i.e. a carbon capture solvent made up of amines diluted in water. However, there is scope to consider the utility of these molecules in formulated blends and in solid state materials such as MOFs and polymers, where these molecules may be able to improve upon the plain MOF or polymer skeleton.

The computational-experimental process and data set herein provide a baseline upon which further data can be generated and new models built and trialled. Ideally, this can form the basis for data standards and sharing to be discussed and applied within this field to enable the comparison of computational models. In particular, in this study we have chosen to not combine our dataset with existing ones to train machine learning models in order to use data from a single consistent experimental procedure. Further studies could investigate combining multiple data sources to expand on the screening models.

Carbon capture as a technology is advancing quickly. However, with many plants already under construction, more advanced and efficient carbon capture solvents are required urgently to help to mitigate and deliver negative emissions technology. Rapid experimental screening and computational modelling will enable researchers to advance this field at an enhanced rate in order to meet this challenge.

\section{Materials and Methods}

\subsection{Computational methods}
Figure \ref{fig:computer_meths} outlines the computational methods. The process is broken into two discrete stages: candidate molecule identification and high throughput screening of molecules. The authors have previously carried out detailed analysis of chemical representations and models for predicting \ac{} which inform the process used in the bottom row.\cite{mcdonagh2022chemical}

\begin{figure}[H]
\centering
\includegraphics[width=\linewidth]{images/methods/computational_workflow.png}
\caption{Computational methods scheme outlining at a high level the process of molecule identification and high throughput virtual screening. }\label{fig:computer_meths}
\end{figure}

\subsubsection{Candidate Molecule Identification and Screening}\label{subsec:identification}
The identification stage is shown in the top row of Figure \ref{fig:computer_meths}. Here we apply tools more commonly associated with the pharmaceutical industry to identify molecules suitable for carbon capture. 

We have identified 164 unique amine molecules which have been reported in the literature\cite{puxty2009carbon, singh2007structure, singh2009structure, kim2015comparison, chowdhury2013co2, evjen2019aminoalkyl, hartono2017screening, rezaei2020molecular, yang2016toward} in relation to a range of carbon capture performance metrics such as absorption capacity, cyclic capacity and initial reaction rate. With this we created a data set by extracting string representations of these molecules from public databases PubChem\cite{kim2021pubchem} and ChemSpider\cite{pence2010chemspider}. We applied Matched Molecular Pairs (MMP) analysis to these molecules through MMPDB.\cite{dalke2018mmpdb} This analysis identifies molecular graph transformations up to three changes and generates transformation rules in the SMIRKS language.\cite{bone1999smiles} Transformations are rooted to an atom type based upon a molecular fingerprint which encodes the environment around an atom. Having identified these transformation rules and the atom type, the rules can then be applied to other molecules. The MMP method can then generate new molecular graphs by applying the rules to existing chemical graphs. This means that only atoms with very similar environments, as defined by the fingerprint, are considered equivalent and hence valid points of modification by applying the relevant SMIRKS. This allows us to deploy a data driven generative engine with only modest amounts of data related to chemical structures. This analysis was supplemented by molecular similarity searches on the PubChem data base using the generated structures.

Using this method we generated over 10,000 possible chemical structures. We filtered this set down in a step wise manner. Initially, we removed any duplicate molecules and any molecules with invalid string representations. We then predict important physical and chemical properties, such as water solubility, pKb and LD50 using the OPERA toolkit from the EPA\cite{mansouri2018opera} (please see \si{}). The current commercial offerings for carbon capture are predominately aqueous amines, hence water solubility is key. We additionally took note of the toxicity, via LD50, and pKb of the molecules from OPERA as highly toxic molecules are undesirable and very weak bases tend be poor for carbon capture. We also considered where possible the probability of a molecules possessing a promising \ac{} using model built on data from\cite{Puxty2009}. These factors were all considered and a manual down selection based upon these predictions and expert input was performed. This led to 287 molecules which formed our refined MMP set. Finally, we manually screened the compounds for (1) commercial availability at > 1 g scale and (2) compounds that represented unique structures/moieties. This later criterion was used to ensure a wider range of different organic amines (and organic nitrogen derivatives) was surveyed.  This down-selection yielded \ndatafull{} candidates.  

\subsubsection{High Throughput Screening}\label{subsec:hvts}
In our previous work we used 98 of these molecules to investigate classification model training and featurization \cite{mcdonagh2022chemical}. In the present work we extend the set of molecules and train new classifier models in over two properties \ac{} and \oir{}. We have trained our classifier models on a new random sample of \ntrain{} molecules, validated on \nval{} molecules ($\approx 90\%$ train $\approx 10\%$ validate) over both properties. The sets used are the same for both properties. We use CCS fingerprints as input features\cite{mcdonagh2022chemical} and apply Principle Component Analysis (PCA) to reduce the dimensionality, such that the PCA components explain 95\% of the variance. This leads to 21 components as features which we use to train our models to classify molecules. All models and training were carried out using scikit-learn version 1.0.2.\cite{scikit-learn}

Molecules were classified for each property separately. For \ac{} we followed the classification procedure defined in our previous work.\cite{mcdonagh2022chemical} This procedure classifies molecules based upon the amine types (primary, secondary, tertiary and poly) which make up the molecule and the likely reaction routes as defined in schemes \ref{scheme:prim_sec} and \ref{scheme:tert} for each amine group type. For the \oir{} we classified the molecules using MEA as a standard. We used our own measured value of MEA's \oir{} ($0.0868 \frac{\text{mol}_{\text{\coo{}}}}{\text{mol}_{\text{amine}}.\text{s}}$), which is in good agreement with the literature\cite{puxty2009carbon}. We therefore classified molecules whose \oir{} was $< 0.0868$ as the negative class and those whose \oir{} was $\geq 0.0868$ as the positive class.

We have applied these models to an external test set of \ntest{} molecules, which had not previously been measured to our knowledge, to fully test the screening capability. We have chosen these properties as they provide proxies for the thermodynamic capture (\ac{}) and the kinetic reaction rate (\oir{}). Considering a molecule in this 2D space allows us to seek the optimal balance between these often completing processes.  Given the limited data we have chosen a relatively small validation and test set to try to make the best use of the data in training our models, however this limits our ability to judge the generalizablity of the models following training. 

The total data set presented in this work is \ndatafull{} (only \ndata{} were used in model building as the data set contains some polymeric molecules which the current featureization is not suitable for). This set includes our previous set of molecules\cite{mcdonagh2022chemical} and adds new experimental data points. For each amine molecule we provide the measured properties of \ac{} and \oir{} in the \si{} Section 1. This is the largest single source data set for carbon capture amines in the open literature to the authors knowledge. 

We applied the CCS fingerprint representation \cite{mcdonagh2022chemical} as features for our models. We trained ten classifiers for each property and selected the best classifiers for each property using the validation set. 

The models were trained in a train test split fashion, performing a 10 fold cross validation to determine the optimal hyper-parameters from a limited grid search for each model over the training data (details of the hyper parameters are provided in the \si{} section 2). In each case the best parameters from this grid search were applied to each model and the validation set was applied to determine the models' performance.

A new unseen set of \ntest{} molecules was passed through the top performing classifiers and ranked based on the probability of each property being in the positive class.  These are the molecules tested by the classifiers and presented in the \textbf{Results} (Fig \ref{fig:computational_predictions_probability}). We additionally, supplemented these predictions with predictions of pKa and LD50 using the OPERA tools from the EPA.\cite{mansouri2018opera} These properties were not used for ranking, but provided further information on the molecules suitability for carbon capture. All predictions and properties are provided with the dataset (https://github.com/IBM/Carbon-capture-fingerprint-generation).

\subsection{Experimental methods}\label{sec:exp_method}
Aqueous amine solutions were tested for \coo{} absorption using a simple, in-house testing apparatus based on infrared absorbance (see Figure \ref{fig:expfig01}\textbf{a} and  \si{}). A gas stream of \coo{} / N$_{2}$ was bubbled into nominally 200 $\mu$L amine solution and the exhaust gas analyzed for \coo{} at the 4.3 $\mu$m absorption band (Figure \ref{fig:expfig01}\textbf{b}) A 3.9 $\mu$m reference band was used to account for slight attenuation due to humidity and signal drift.  The absorption signal was calibrated against 3 sources: atmosphere (taken to be 414 ppm), 9.96\%v/v \coo{} / balance nitrogen (hereafter referred to as 10\%v/v), and pure \coo{} as a function of flow rate, $q$ = 10 sccm.  Amine solutions were held at 40$\degree$C, chosen to fit typical industrial absorption operating temperature \cite{Notz2007}.  Signals were transformed from optical transmission into volume fraction \coo{} absorbed using a calibrated Modified Beer-Lambert equation (see \si{}). The measurement principle is that \coo{} lost in the exhaust stream must be absorbed in the amine solution; quantification of the gas content as a function of time and integration affords the total \coo{} absorbed and capture capacity, $\alpha$ (mol \coo{} $\cdot$ mol amine$^{-1}$).  

\begin{figure}[H]
\centering
\includegraphics[width=0.9\textwidth]{images/methods/expfig01v4.png}
\caption{Carbon dioxide absorption testing. \textbf{a}) Schematic representation of testing apparatus \textbf{b}) principle of analysis; gas is injected at a continuous rate and volume fraction $f_{o}$ of \coo{} into the aqueous amine solution.  The change in the \coo{} content of the exhaust stream is measured by infrared absorption at 4.3 $\mu$m.  \coo{} that reacts and absorbed by the solution is lost in the output gas stream, causing the measured fraction $f_{CO2}$ to be less than the supply fraction $f_{o}$. \textbf{c}) Representative signal of MEA absorption run, showing saturation region quantified by the breakthrough time, $t_{b}$ and roll-off to baseline ($f_{CO2}$ = $f_{o}$). \textbf{d}) Control signal of pure water absorption versus MEA. \textbf{e}) Absorption signals for I. MEA, II. L-leucinol and III. tert-butylaminoethanol.  Signals I and II exemplify fast reacting amines, showing either an exponential or sigmoidal type roll-off behavior (respectively).  \textbf{f}) Correlation of measured capture capacity $\alpha$ to that reported by Puxty \textit{et al} \cite{Puxty2009}.  \textbf{g}) Ranked capture capacity of tested samples labeled by amine type.
} \label{fig:expfig01}
\end{figure}

Monoethanolamine (MEA), 30\% w/w aqueous, was used as a calibrant as it has a well-established capture capacity of $\alpha$ = 0.50 mol CO2 / mol amine \cite{Puxty2009} (Figure \ref{fig:expfig01}\textbf{c}).  The estimated apparatus delay time is 0.16 min, and control experiments with pure water show a background absorption of $\sim$ 20 $\mu$mol \coo{} (Figure \ref{fig:expfig01}\textbf{d}). Analysis of the time progression of the absorbed \coo{} signal affords information on the relative speed of reaction (Figure \ref{fig:expfig01}\textbf{e}).

Benchmarking was performed with 23 reagents selected from the work of Puxty and co-workers, which had previously established a large dataset of 76 amines for carbon dioxide capture using a combination of microscale (100 $\mu$l) gravimetric analysis and macroscale (20 or 300 mL) liquid absorption analysis \cite{Puxty2009}.  The assay’s capture capacity shows a moderate correlation with the reference (Figure \ref{fig:expfig01}\textbf{f}) at $r^{2}$ = 0.67.  The current work recapitulates the capacity seen for fast reacting amines, as exemplified by MEA  ( $\alpha_{current}$ = 0.55 , $\alpha_{ref}$ = 0.49 / 0.56 ), as well as for slower tertiary / frustrated amines such as N,N-dimethylaminoethanol ($\alpha_{current}$ = 0.93 , $\alpha_{ref}$ = 0.92 ). 

130 aqueous amine solutions were prepared and tested for CO2 absorption (Figure \ref{fig:expfig01} $\textbf{g}$). Species include 1$\degree$, 2$\degree$ and 3$\degree$ amines, as well as polyamines, including both aliphatic and aromatic derivatives, and other organic nitrogen species (\si{} Figure 2 and 12). All solutions were prepared at 30$\%$ w/w (unless noted) as this concentration is typical in currently deployed industrial offerings.  Run times varied from 5 – 120 min to ensure a substantial saturated (i.e. no \coo{} absorbed) baseline was observed (analysis time can be shorted by 2x-4x if this criterion is removed).  Capture capacity (Figure \ref{fig:expfig01}\textbf{g}) shows the expected trend of $\alpha$ $\sim$ 1.0 for 3$\degree$ amines and sterically hindered 2$\degree$ / 1$\degree$ amines, and $\alpha$ $\sim$ 0.5 for 2$\degree$ / 1$\degree$ amines.  As seen in literature, there are middling cases where amine capacities are higher than expected, e.g. piperidine (2$\degree$) $\alpha$ = 0.93 ± 0.072, 2-amino-2-methyl-1-propanol (1$\degree$) $\alpha$ = 0.90 ± 0.056, and 2-aminocyclohexanol (1$\degree$) $\alpha$ = 0.75 ± 0.007, as well as a population of amines with poor performance, tailing to effectively non-reactive $\alpha$ = 0.006.  The distribution of capture capacity is multi-modal (\si{} Figure 6), with populations centered at $\alpha$ = 0.0, 0.27 and 0.49, as well as a low-sampled population at $\alpha$ $\sim$ 1.0.  The populations at 0.49 and $\sim$ 1.0 can be assigned to the expected carbamate and carbonate reaction paths.
\\
\\
\\
\textbf{Acknowledgement} The authors thank Mathias Steiner, Binquan Luan, James Hendrick and Nathaniel Park for insightful conversations. Data and required materials for this work can be found in the supporting information and at the following URL \codedataurl{}.

% Bibliography
\bibliography{arXiv_2023newmolcCC_IBM_refs}
\bibliographystyle{plain}

\end{document}
