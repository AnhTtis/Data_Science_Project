\documentclass{article}
\usepackage[utf8]{inputenc}
\usepackage{amsmath,amsfonts, amssymb, amsthm, amscd}
%\usepackage{MnSymbol}
\usepackage{tikz,tikz-cd}
\usepackage{xcolor,url}
\usepackage{titlesec, eucal}
\usepackage{lineno}
%\linenumbers
\titleformat{\subsubsection}[runin]{\normalfont\bfseries}{\thesubsubsection.}{3pt}{}
\setcounter{tocdepth}{2}
\usepackage{multicol}

\newcommand*{\DashedArrow}[1][]{\mathbin{\tikz [baseline=-0.25ex,-latex, dashed,#1] \draw [#1] (0pt,0.5ex) -- (1.3em,0.5ex);}}%

\addtolength{\oddsidemargin}{-10mm}
\addtolength{\textwidth}{20mm}

\newtheorem{proposition}[subsubsection]{Proposition}
\newtheorem{theorem}[subsubsection]{Theorem}
\newtheorem{lemma}[subsubsection]{Lemma}
\newtheorem{corollary}[subsubsection]{Corollary}
\theoremstyle{definition}
\newtheorem{definition}[subsubsection]{Definition}
\newtheorem{example}[subsubsection]{Example}
\newtheorem{remark}[subsubsection]{Remark}
\newtheorem{conjecture}{Conjecture}
\newtheorem{claim}[subsubsection]{Claim}
\newtheorem{question}{Question}
\newtheorem{problem}{Problem}
\theoremstyle{plain}
\newtheorem{thm}{Theorem}
\renewcommand{\thethm}{\Alph{thm}} 


\renewcommand{\proofname}{\textnormal{\textbf{Proof:  }}}
\newcommand{\C}{\mathbb C}
\renewcommand{\refname}{Bibliography}
\newcommand{\Z}{\mathbb Z}
\newcommand{\Q}{\mathbb Q}
\renewcommand{\o}{\otimes}
\newcommand{\R}{\mathbb R}
\renewcommand{\O}{\mathcal O}
\newcommand{\g}{\mathfrak g}
\newcommand{\D}{\mathcal D}
\newcommand{\di}{\partial}
\newcommand{\dibar}{\overline{\partial}}
\newcommand{\arr}{\xrightarrow}
\newcommand{\im}{\operatorname{im}}
\renewcommand{\ker}{\operatorname{ker}}
\newcommand{\Hom}{\operatorname{Hom}}
\newcommand{\Ext}{\operatorname{Ext}}
\newcommand{\Alb}{\operatorname{Alb}}
\newcommand{\Hilb}{\operatorname{Hilb}}
\renewcommand{\P}{\mathbb{P}}
\newcommand{\F}{\mathbb F}
\newcommand{\acts}{\lefttorightarrow}
\newcommand{\codim}{\operatorname{codim}}
\newcommand{\rk}{\operatorname{rk}}
\newcommand{\hdot}{{\:\raisebox{3pt}{\text{\circle*{1.5}}}}}
\newcommand{\Tot}{\operatorname{Tot}}
\newcommand{\Pic}{\operatorname{Pic}}
\newcommand{\Bl}{\operatorname{Bl}}
\newcommand{\GL}{\operatorname{GL}}
\newcommand{\Sym}{\operatorname{Sym}}
\newcommand{\diag}{\operatorname{diag}}
\renewcommand{\phi}{\varphi}
\newcommand{\V}{\mathrm{Var}}
\binoppenalty = 10000
\relpenalty = 10000

%%%%%%%% Timeline
%%% 2022
%%% Dec 10 Started the file (Anna)
%%% Dec 12 Partial classification of rational and ruled surfaces with a square-zero elliptic curve (Anna)
%%% Dec 13 Section on Hopf surfaces (Rodion)
%%% Dec ? Something on Hopf surfaces in the section about classification (Rodion)
%%% Dec 17 No examples in kappa = one. Hopf transform of a Hopf surface is Hopf or ruled. (Anna)
%%% Dec 23 reference to Enoki paper: any class VII_0 surface with a smooth elliptic curve is a Hopf surface; corrections on Hopf surfaces (Rodion)
%%% Dec 25 Hopf transforms are well defined (Anna)
%%% Dec 28 Переставила главы и переделала в стиль под SGA (Аня)
%%% Dec 31 Enoki's classification, simplification of 3.3.3 (Rodion)
%%% Jan 3 a page on Hopf conjugacy (can be discarded afterwards), started a section on G-equivariant Hopf transforms (Rodion)
%%% Jan 9 a section on Hopf transforms in singular curves started (Rodion)
%%% Jan 10 desctiption of non-diagonal Hopfs (Rodion)
%%% Jan 12 2.2.9: all analytic compactifications are Hopf transforms; 4.0.2. different curves have different classes in K-theory (Anna)
%%% Jan 27 cosmetical changes in the section on rational surfaces (Rodion)
%%% ? Dissenting version (=version 0.1) created (Rodion)
%%% ? Changes in Section 1 (Rodion)
%%% Feb 12 Version 0.2 created. Rewrote subsections on analytic compactifications and Hopf surfaces. (Anna)
%%% Feb 14  Added a subsection on secondary Hopf surfaces (Anna)
%%% Feb 18 Slightly rewrote the subsection on Hopf duals and analytic cobordance (Anna)
%%% Feb 25 introduction, slight redressing of 'Surfaces of class VII' section (Rodion)
%%% Feb 25 Rewrote subsections 2.2 and 2.3 (Hopf surfaces, Secondary Hopf surfaces). Divided the proof that Hopf transforms are well defined in two statements. Also, wrote section 5 about topology of Hopf transforms (Anna)
%%% Feb 26 Edited section 2 (anna)
%%% Feb 27 edited sections 3 and 4 (Rodion)
%%% Feb 27 removed the section on topology of hopf transforms. it's saved in appendix.tex (anna)
%%% Feb 28 Intro expanded, definitions in K_0-section rewritten (rodion)
%%% Mar 1 убрала излишне безумные куски введения чтоб глаза мои их не видели (аня)
%%% Mar 1 rewrote the intro (rodion)
%%% Mar 1 rewrote the intro (anna)
%%% Mar 2 нормальные куски были, чего бухтеть (родя)
%%% Mar 2 some complements in the intro and Section 4 (K_0 of non-algebraic varieties) (rodion)
%%% Mar 3 revised the abstract, the intro and section 4 (rodion)
%%% Mar 3 rewrote the abstract (anna)
%%% Mar 4 revised everything (rodion)
%%% Mar 5 the first draft is ready!

\title{Complex surfaces with many algebraic structures}
\author{Anna Abasheva, Rodion D\'eev}
\date{}

\begin{document}

\maketitle

\begin{abstract}

We find new examples of complex surfaces with countably many non-isomorphic algebraic structures. Here is one such example: take an elliptic curve $E$ in $\mathbb P^2$ and blow up nine general points on $E$. Then the complement $M$ of the strict transform of $E$ in the blow-up has countably many algebraic structures. Moreover, each algebraic structure comes from an embedding of $M$ into a blow-up of $\mathbb P^2$ in nine points lying on an elliptic curve $F\not\simeq E$. We classify algebraic structures on $M$ using a {\bf Hopf transform}: a way of constructing a new surface by cutting out an elliptic curve and pasting a different one. Next, we introduce the notion of an {\bf analytic K-theory of varieties}. Manipulations with the example above lead us to prove that classes of all elliptic curves in this K-theory coincide. To put in another way, all motivic measures on complex algebraic varieties that take equal values on biholomorphic varieties do not distinguish elliptic curves.

\end{abstract}

\tableofcontents

\newpage

\section{Introduction}

A compact complex analytic variety $X^{an}$ has at most one algebraic structure; that is to say, there is at most one algebraic variety $Y$ such that $Y^{an} \cong X^{an}$. However, this is false in the non-compact world. We do not have general techniques to describe the set of algebraic structures on a non-compact complex variety. Available results show that a variety can have infinitely many of algebraic structures. We briefly review examples in dimension two.

\begin{itemize}
    \item Let $C$ be an affine curve of positive genus. Then $C^{an} \times \C$ admits uncountably many non-isomorphic affine structures \cite{Jelonek}. Indeed, every line bundle on $C$ is analytically trivial, hence its total space is biholomorphic to $C^{an}\times \C$. %This example partially generalises to curves over $p$-adic fields \cite{Petrov}.
    \item The surface $\C^* \times \C^*$ is biholomorphic to the unique non-trivial affine bundle of rank one and degree zero over any elliptic curve \cite[Thm.\:6.12]{Neeman}. Hence, it has uncountably many non-isomorphic algebraic structures. 
    \item Let $C$ be a smooth projective curve and $p\colon X\to C$ an algebraic elliptic surface. Choose a smooth fiber $E = p^{-1}(x)$. For every $n>0$, we can find an elliptic fibration $p_n\colon X_n\to C$, called a {\bf log-transform} of $p$, that has a fiber of multiplicity $n$ over $x$ and is biholomorphic to $p$ over $C-\{x\}$. The new surfaces $X_n$ are algebraic unless $X\simeq E\times C$ \cite[Ch.\:I, Thm.\:6.12]{Friedman_Morgan}. Hence the surface $X-E$ has at least countably many algebraic structures.
    \item We can construct even more algebraic structures on the surface $X-E$ from the previous example using Shafarevich--Tate twists. We refer the reader to \cite[Ch.\:I, Sections 1.5.1, 1.5.3]{Friedman_Morgan} for more details.
\end{itemize}

We complement this list by the following beautiful example.

\begin{thm}[Theorem \ref{many-algebraic-structures}, Corollary \ref{Hopf-transform-preserves-rationality}]\begin{enumerate}
\label{intro_theorem_A}
    \item Pick nine points in $\P^2$ in sufficiently general position. Let $X$ be the blow-up of $\P^2$ in these points and $E$ the strict transform of the unique elliptic curve passing through them. Then the complement of $E$ in $X$ admits countably many algebraic structures.
    \item More generally, let $X$ be a compact rational surface containing a square-zero elliptic curve $E$. Assume that the normal bundle to $E$ in $X$ is sufficiently general. Then the surface $X-E$ admits countably many algebraic structures.

\end{enumerate}

\end{thm}

In addition, in the Case 1 every algebraic structure on $X-E$ arises from a holomorphic embedding into a blow-up $Y$ of $\P^2$ in some other nine points. The complement of $X-E$ in $Y$ is an elliptic curve $F$ which is remarkably {\em not isomorphic and not even isogenous} to $E$. In the algebraic realm, this is only possible for singular curves (\cite{Blanc_Poloni_van_Santen} and references therein). Indeed, if $S$ and $S'$ are smooth algebraic surfaces containing smooth curves $C$ and $C'$ respectively, then an algebraic isomorphism between $S-C$ and $S'-C'$ implies an isomorphism between $C$ and $C'$.

\hfill

Our construction in Theorem \ref{intro_theorem_A} is fundamentaly different from those in the list above. We developed a method to manufacture new surfaces from a surface with a square-zero elliptic curve that we call a {\bf Hopf transform} (Definition \ref{Hopf_transform}). It serves as a counterpart to a log-transform in the case of non-elliptic surfaces.


\hfill


We can summarize a Hopf transform as follows: take an elliptic curve on a surface, cut it out, and paste another elliptic curve instead. Thanks to a theorem of Arnold and Ueda (Theorem \ref{Arnold}), a square-zero elliptic curve $E \subset S$ whose normal bundle $L$ is sufficiently general has a holomorphic tubular neighbourhood. That is to say, $S$ is locally biholomorphic to $\Tot(L)$ along $E$. We will clarify the meaning of ``sufficiently general'' in Definition \ref{diophantine-definition}; for now, it is enough to say that all ``sufficiently general'' line bundles are non-torsion. Using Hopf surfaces, i.e., non-K\"ahler surfaces of the form $\C^2-\{0\}/\diag(\lambda,\mu)$ and their finite quotients, we show that there exist countably many degree zero line bundles $L'\to F$ over different elliptic curves $F$ such that $\Tot(L)^{an} - E \cong \Tot(L')^{an} - F$. We call such line bundles $L$ and $L'$ {\bf analytically cobordant} (Definition \ref{def of duals and cobordant}). A Hopf transform, which can be interpreted as a gluing of a surface with an appropriate Hopf surface, replaces an elliptic curve $E\subset S$ with $F$. Moreover, we are free to choose from a countable set of candidates for $F$. We apply this surgery to a blow-up of nine points on $\P^2$ and show that the result is again a blow-up of $\P^2$ along some other nine points. It yields Theorem \ref{many-algebraic-structures}.

\hfill

Our notion of a Hopf transform was inspired by Koike and Uehara's construction in \cite{Koike_Uehara_non_proj}. They use the theorem of Arnold and Ueda mentioned above to glue two open subsets of two different blow-ups of $\P^2$ in nine points into a K3 surface.

\hfill

We apply Theorem \ref{intro_theorem_A} to study the {\bf analytic Grothendieck group} $K_0^{an}$ of varieties (Definition \ref{analytic-grothendieck-definition}). It is the quotient of the Grothendieck group $K_0(\V_\C)$ of complex algebraic varieties\footnote{$K_0(\V_\C)$ is the abelian group generated by classes of algebraic varieties over $\C$ modulo scissor relations: $[X]-[Y] = [X-Y]$, where $Y$ is a closed subvariety of $X$.} by additional relations of the form
$$
[X] = [Y] \:\:\:\text{     if     }\:\:\: X^{an}\simeq Y^{an}.
$$
While $K_0(\V_\C)$ has attracted a considerable amount of attention (see \cite{Bittner,Kontsevich_Tschinkel_specialization,Nicaise_Shinder} among many others), the notion of $K_0^{an}$ is, to our knowledge, new. The group $K_0^{an}$ is strikingly different from its algebraic counterpart as the theorem below illustrates.


\begin{thm}[Theorem \ref{analytic K zero}]\label{intro_theorem_B}
    All elliptic curves have the same classes in $K_0^{an}$. %The natural map $K_0(\V_{\C}) \to K_0^{an}$ is not injective.
\end{thm}

\noindent In contrast, non-isomorphic curves have different classes in $K_0(\V_\C)$ \ref{Bittner-theorem}.

\hfill

This paper is organized as follows. In Section \ref{hopf-transforms-section}, we prove that the total space of a non-torsion degree-zero line bundle on an elliptic curve has countably many compactifications by different elliptic curves (Theorem \ref{embeddings into all hopfs}). Then we introduce the notions of Hopf duality and analytic cobordance (Defintion \ref{def of duals and cobordant}), and give the definition of Hopf transforms (Definition \ref{Hopf_transform}). In Section \ref{square-zero-section}, we partially classify surfaces to which Hopf transforms apply, that is, surfaces with square-zero elliptic curves with non-torsion normal bundle. With this classification in mind, we prove Theorem \ref{intro_theorem_A}, the main result of our paper. Finally, in Section \ref{grothendieck-section}, we introduce the analytic Grothendieck group and prove Theorem \ref{intro_theorem_B}. Towards the end, we pose several intriguing questions about the analytic Grothendieck group.

%%%%%%%%%%%%%%%%%%%%
%%%%%%%%%%%%%%%%%%%%
%%%%%%%%%%%%%%%%%%%%

\section{Hopf transforms}\label{hopf-transforms-section}

%%%%%%%%%%%%%%%%%%
%%%%%%%%%%%%%%%%%%%%%
\subsection{Analytic compactifications}

%%%%%%%%%%%%%%%%%%%

\begin{definition}
    Let $M$ be a non-compact complex surface. A pair $(S, C)$ of a smooth compact complex surface $S$ and (possibly singular) curve $C \subset S$ is called an {\it analytic compactification} or just a {\it compactification of $M$} if $S-C$ is biholomorphic to $M$. A compactification is called {\em minimal} if none of the curves in the support of $C$ are exceptional.
\end{definition}

The following classification theorem is due to Enoki \cite[Theorem C-I$_0$]{Enoki82}.

%%%%%%%%%%%%%%%%%%%%%%%%%%%%%%

\begin{theorem}\label{enoki-classification}
    Let $\Tot(L)$ be the total space of a line bundle $L$ of degree zero over an elliptic curve $E$. Then every minimal compactification $(S,C)$ of $M$ is of the following form:
    \begin{enumerate}
        \item $S\cong \P({\cal O}\oplus L)$ is a $\P^1$-bundle over an elliptic curve; $C$ is its section, and $(C)^2=0$.
        \item $S$ is a Hopf surface (Definition \ref{Hopf_surface_def}), and $C$ is an elliptic curve.
    \end{enumerate}
\end{theorem}

If $L$ is non-torsion, the set of Hopf surfaces compactifying $\Tot(L)$ is countable, as we will see in Theorem \ref{embeddings into all hopfs}. For now, let us note that all such compactifications contain at least two elliptic curves.

%%%%%%%%%%%%%%%%%%%%%%%%%%%%%%
%%%%%%%%%%%%%%%%%%%%%

\subsection{Hopf surfaces}


\begin{definition}
\label{Hopf_surface_def}\cite[Sect.\:10]{KodairaII}
    A {\it Hopf surface} is a compact complex surface whose universal cover is $\C^2-\{0\}$. A Hopf surface is called {\em primary} if its fundamental group is $\Z$, otherwise it is called {\em secondary}.
\end{definition}

%%%%%%%%%%%%%%%%%%%%%%%%%%%%%%%%%%%%

\subsubsection{}\label{subsubsection primary Hopfs non linear}
All Hopf surfaces are non-K\"ahler. By \cite[Sect.\:10, p.\:695]{KodairaII} every primary Hopf surface $X$ is biholomorphic to a quotient $\C^2-\{0\}/ \Gamma$ where $\Gamma$ is either a linear diagonal operator or of the form
$$
\Gamma(x,y) = (\alpha^nx + y^n, \alpha y).
$$
In the second case, a Hopf surface has only one irreducible curve: the image of the $x$-axis. Therefore, such surfaces do not arise as compactifications of $\Tot(L)$. 

%%%%%%%%%%%%%%%%%%%%

\subsubsection{}
\label{subsubsection primary Hopfs linear}
Let us study the case when $\Gamma = \diag(\lambda,\mu)$. In order for $\C^2-\{0\}/\Gamma$ to be compact, the absolute values of the eigenvalues must be either both greater than one or both less than one. By replacing $\Gamma$ with $\Gamma^{-1}$ if necessary, we may assume the latter. 

\begin{definition}
    A {\em diagonal Hopf surface} is $\C^2-\{0\}/ \diag(\lambda, \mu)$ for $|\lambda|<1$, $|\mu|<1$. We will denote it by $H(\lambda,\mu)$
\end{definition}

%%%%%%%%%%%%%%%%%%%%%%

\subsubsection{}
One can show that two Hopf surfaces $H(\lambda, \mu)$ and $H(\lambda',\mu')$ are isomorphic if and only if $\lambda = \lambda'$ and $\mu = \mu'$ (up to a permutation). All automorphisms of $H(\lambda,\mu)$ are induced by a linear diagonal automorphism of $\C^2$.

%%%%%%%%%%%%%%%%%%%%%%%%%%%%%%%%%%%

\subsubsection{}\label{generalities on primary hopfs}
If $\lambda^n = \mu^m$ for some non-zero integers $n$ and $m$ then $H(\lambda,\mu)$ is an elliptic surface with smooth but possibly non-reduced fibers \cite[Ch.\:V, Prop.\:18.2]{Barth_Hulek_Peters_Van_de_Ven}. All elliptic curves in $X$ are fibers, hence their normal bundles are torsion. Such a surface cannot be a compactification of $\Tot(L)$ for a non-torsion line bundle $L$.

When $\lambda^n\ne\mu^m$ for any pair of non-zero integers the surface $H(\lambda,\mu)$ contains exactly two irreducible curves \cite[Ch.\:V, Prop.\:18.2]{Barth_Hulek_Peters_Van_de_Ven}. They are the images of the axes under the quotient map and are isomorphic to $E_\lambda:=\C^*/\lambda$ and $E_\mu:=\C^*/\mu$. They have both square zero. 

The complement to $E_\mu$ in $H(\lambda,\mu)$ is $\C^*\times \C/\diag(\lambda,\mu)$. It is the total space of a line bundle over $E_\lambda$, and the map to $E_\lambda$ is the projection to the first coordinate.

%%%%%%%%%%%%%%%%%%%%%%%%%%%%%%%%

\subsubsection{} Every elliptic curve can be written as $\C/\Z + \Z\tau$ for $\operatorname{Im}(\tau)>0$. The parameter $\tau$ is defined modulo the $\operatorname{SL}_2(\Z)$-action. The map $x\mapsto e^{2\pi ix}$ induces an isomorphism
$$
\C/\Z + \Z\tau \stackrel{\sim}{\to} \C^*/q^\tau.
$$
Here we use the shorthand $q^\tau$ for $e^{2\pi i\tau}$. The total space of every line bundle of degree zero over $\C/\Z + \Z\tau$ is of the form
\begin{equation}\label{total_of_L}
    \mathrm{L}(\tau,A,B) = \mathbb C^2/\sim:\:\:\:\:\:\: (x,y) \sim (x+1, Ay);\: (x,y)\sim (x+\tau, By),
\end{equation}
where $A,B\in U(1)$. These numbers are uniquely defined once we fix $\tau$. Similarly, the total space of a line bundle of degree zero over $E_\lambda = \C^*/\lambda$ can be written as
\begin{equation}
    \mathcal{L}(\lambda,\mu) = \C^*\times \C/\diag(\lambda,\mu),
\end{equation}
where $\lambda,\mu\in \C^*$. 

%%%%%%%%%%%%%%%%%%%%%%%%%%%%%%%%%

\subsubsection{}\label{when L(lambda,mu) are isomorphic}The class of $\mathcal{L}(\lambda,\mu)$ in $\operatorname{Pic}^0(E_\lambda)\simeq E_\lambda$ is $\mu\pmod{\lambda^\Z}$ \cite[Sect.\:10, p.\:696]{KodairaII}. In particular, line bundles $\mathcal{L}(\lambda,\mu)$ and $\mathcal{L}(\lambda,\mu')$ are isomorphic if and only if $\mu' = \mu\lambda^n$.

%%%%%%%%%%%%%%%%%%%%%%%%%%

\subsubsection{}
\label{main_biholomorphism}
A line bundle on an elliptic curve can be represented as $\mathrm{L}(\tau,A,B)$ or $\mathcal{L}(\lambda,\mu)$. Let us figure out how to switch between the two descriptions. Write $A = q^u$ for a number $u\in \R$. Then we have a linear isomorphism of line bundles

$$
\alpha_{\tau, u, B}\colon \mathrm{L}(\tau, q^u,B) \stackrel{\sim}\to \mathcal L(q^\tau, q^{-u\tau}B)
$$
given by 
\begin{equation}
\label{from tau to hopf}
        \alpha_{\tau, u, B}(x,y) = (q^{x}, q^{-ux}y).
\end{equation}

Note that $|q^\tau|<1$ and $|q^{-u\tau}B|<1$ if $u$ is chosen to be negative. Given a line bundle $\mathcal L(q^\tau, \mu)$, we can find corresponding $u$ and $B$. Indeed, write $\mu = q^\sigma$ and $B = q^v$. The equation
$$
\sigma = u\tau + v
$$
determines $u$ and $v$ uniquely since $u$ and $v$ are real numbers. Different choice of $\sigma$ does not change $B = q^v$.



%%%%%%%%%%%%%%%%%%%%%%%%%

\subsubsection{}\label{subsection on sl2 action} The isomorphism (\ref{from tau to hopf}) depends only on the choice of $\tau$ and $u$ such that $q^u=A$. Let us see what happens if we change $\tau$. There is a linear isomorphism of line bundles
$$
\beta_\Gamma \colon \mathrm{L}(\tau, q^u,q^v)\stackrel{\sim}\to\mathrm{L}\left(\frac{k\tau + l}{m\tau +n}, q^{mv+nu}, q^{kv+lu}\right)
$$
for $\begin{pmatrix}k & l\\m & n\end{pmatrix}\in\operatorname{SL}_2(\Z)$. The map (\ref{from tau to hopf}) identifies $\mathrm{L}(\tau, q^u,q^v)$ with $\mathcal L(\lambda', \mu')$, where
\begin{gather*}
   \lambda' = q^{\frac{k\tau+l}{m\tau+n}},\\ 
   \mu' = q^{-(mv+nu+r)\frac{k\tau+l}{m\tau+n} + kv+lu},
\end{gather*}
and $r\in \Z$. Using that $nk-ml = 1$, we can rewrite the exponent in $\mu'$ as
$$
-(mv+nu+r)\frac{k\tau+l}{m\tau+n} + kv+lu = \frac{v-u\tau + r(k\tau + l)}{m\tau + n}.
$$
The elliptic curve in the complement to $\Tot(L)$ in $H(\lambda',\mu')$ is 
$$
E_{\mu'} \simeq \C/\Z + \Z\cdot \left(\frac{v-u\tau + r(k\tau + l)}{m\tau + n}\right)
$$
If $L$ is generic, different choices of $\begin{pmatrix}k & l\\m & n\end{pmatrix}$ and $r$ lead to different elliptic curves $E_{\mu'}$.

%%%%%%%%%%%%%%%%%%%%%%%%

\subsubsection{}\label{translations} Consider a translation $t_s\colon x\mapsto x+s$ on the curve $\C/\Z+\Z\tau$. The line bundle $\mathrm{L}(\tau, A,B)$ has degree zero, hence it is isomorphic to its pullback along the translation. The isomorphism can be written explicitly as
\begin{gather*}
    t_s^*\colon \mathrm{L}(\tau, A, B)\to \mathrm L(\tau, A, B)\\ (x,y)\mapsto (x+s,y).
\end{gather*}
When we write $\mathrm{L}(\tau, A,B)$ as $\mathcal L(q^{\tau}, q^{-u\tau}B)$ as in \ref{main_biholomorphism}, the translation turns into the map
\begin{gather*}
 t_s^*\colon \mathcal L(q^{\tau}, q^{-u\tau}B) \to \mathcal L(q^{\tau}, q^{-u\tau}B)\\
 (x,y)\mapsto (q^s, q^{-us}).   
\end{gather*}
This map extends to the automorphism $\diag(q^s,q^{-us})$ of the Hopf surface $H(q^\tau, q^{-u\tau}B)$.


%%%%%%%%%%%%%%%%%%%%%%%%%%%%%

\subsubsection{}\label{automorphisms fixing zero} Consider the automorphism $f$ of $E$ sending $x$ to $-x$. The line bundle $f^*\mathrm{L}(\tau, A, B)$ is isomorphic to $\mathrm{L}(\tau, A^{-1}, B^{-1})$. In general, an automorphism $f$ of $E$ fixing $0$ induces a linear biholomorphism 
$$
f^*\colon \mathrm{L}(\tau, A, B)\stackrel{\sim}\to \mathrm{L}(\tau, A'', B''),
$$
where the pair of numbers $(A'', B'')$ does not necessarily coincide with $(A,B)$.


%%%%%%%%%%%%%%%%%%%%%%%%%%%%

\subsubsection{}\label{multiplication by a constant} Every automorphism of $\mathcal L(\lambda,\mu)$ as a line bundle is the fiberwise multiplication by a constant $c$. It extends to the automorphism $\diag(1,c)$ of the Hopf surface $H(\lambda, \mu)$.

%%%%%%%%%%%%%%%%%%%%%



\begin{proposition}
\label{which hopfs arise}
Let $L$ be a degree-zero non-torsion line bundle on an elliptic curve $E$. Then the set of equivalence classes of open embeddings $\Tot(L)\stackrel{\iota}\to H$ into a primary Hopf surface is countable. Here, we call two embeddings $\iota\colon \Tot(L)\to H$ and $\iota'\colon \Tot(L)\to H'$ equivalent if there is an automorphism $g$ of $\Tot(L)$ and a biholomorphism $h\colon H\to H'$ such that the following diagram commutes:

\begin{center}
\begin{tikzcd}
\operatorname{Tot}(L) \arrow[r, "\iota", hook] \arrow[d, "g"] & H \arrow[d, "h"] \\
\operatorname{Tot}(L) \arrow[r, "\iota'", hook]               & H'              
\end{tikzcd}
\end{center}

\end{proposition}
\begin{proof}
Suppose $E=X/\Z+\Z\tau$ and $L$ is such that $\Tot(L)=\mathrm{L}(\tau,A,B)$. Consider an open embedding $\iota\colon \mathrm{L}(\tau, A, V)\to H$. By \ref{subsubsection primary Hopfs non linear} and \ref{generalities on primary hopfs}, the surface $H$ must be of the form $H(\lambda,\mu)$, and $\mathrm{L}(\tau,A,B)$ is mapped onto $H(\lambda,\mu)-E_\mu\simeq \mathcal L(\lambda,\mu)$. We saw in \ref{main_biholomorphism} that there exists a biholomorphism $$\alpha_{\tau',u',B'}\colon \mathrm{L}(\tau',A',B')\stackrel{\sim}\to \mathcal L(\lambda,\mu)$$
for some $\tau'\in\C$ such that $\mathrm{Im}~\tau'>0$ and $A',B'\in U(1)$.

The elliptic curves $E = \C/\Z+\Z\tau$ and $\C/\Z+\Z\tau'$ must be isomorphic, hence $\tau = \Gamma\cdot\tau'$ for a matrix $\Gamma\in\operatorname{SL}_2(\Z)$. In particular, the set of possible $\tau'$ is countable. The line bundle $\mathrm{L}(\tau', A', B')$ is linearly isomorphic to $\mathrm{L}(\tau, A'',B'')$ for some $A'', B''\in U(1)$ (\ref{subsection on sl2 action}). Therefore, the embedding $\mathrm{L}(\tau, A,B)\stackrel{\iota}\to H(\lambda,\mu)$ is the composition of the following maps:

\begin{center}\begin{tikzcd}
{\mathrm{L}(\tau, A,B)} \arrow[r, "\varphi"] & {\mathrm{L}(\tau, A'',B'')} \arrow[r, "\beta_\Gamma"] & {\mathrm L(\tau, A', B')} \arrow[r, "{\alpha_{\tau',u',B'}}"] & {\mathcal L(\lambda,\mu)} \arrow[r, hook] & {H(\lambda,\mu)},
\end{tikzcd}\end{center}

where $\mathrm{L}(\tau, A,B)\stackrel{\phi}\to \mathrm{L}(\tau, A'',B'')$ is some biholomorphism. 

\hfill

It remains to classify biholomorphisms $\mathrm{L}(\tau, A,B)\stackrel{\phi}\to \mathrm{L}(\tau, A'',B'')$ modulo equivalence. Let $f$ be the restriction of $\phi$ to the zero section $E\subset \mathrm{L}(\tau, A,B)$. It is a composition of a translation and an automorphism preserving zero. Two biholomorphisms $\mathrm{L}(\tau, A,B)\to\mathrm{L}(\tau, A'',B'')$ that differ by a translation are equivalent because a translation extends to an automorphism of $H(\lambda,\mu)$ (\ref{translations}). Hence, we may assume that $f\colon E\to E$ preserves zero. We have an isomorphism of line bundles $\mathrm{L}(\tau,A,B)\simeq f^*\mathrm{L}(\tau,A'',B'')$ as in \ref{automorphisms fixing zero}. The set of automorphisms fixing zero is finite, hence the number of pairs $(A'',B'')$ such that $\mathrm{L}(\tau, A,B)$ is isomorphic to $\mathrm{L}(\tau, A'',B'')$ is also finite. We have shown that every embedding $\mathrm{L}(\tau, A,B)\stackrel{\iota}\to H(\lambda,\mu)$ is equivalent to the composition 

\begin{center}
\begin{tikzcd}{L(\tau, A,B)} \arrow[r, "\psi"] & {\mathrm{L}(\tau, A,B)} \arrow[r, "f^*"] & {\mathrm{L}(\tau, A'',B'')} \arrow[r, "\beta_\Gamma"] & {\mathrm L(\tau, A', B')} \arrow[r, "{\alpha_{\tau',u',B'}}", hook] & {H(\lambda,\mu)},\end{tikzcd}\end{center}

where $\psi$ is a biholomorphism of $\mathrm{L}(\tau, A,B)$ identical on $E$. We will see later in Lemma \ref{open embeddings} that every such biholomorphism is multiplication by a constant. It extends to an automorphism of $H(\lambda,\mu)$ by \ref{multiplication by a constant}. Therefore, every embedding is equivalent to the composition of morphisms
\begin{center}
\begin{tikzcd}
{\mathrm{L}(\tau, A,B)} \arrow[r, "f^*"] & {\mathrm{L}(\tau, A'',B'')} \arrow[r, "\beta_\Gamma"] & {\mathrm L(\tau, A', B')} \arrow[r, "{\alpha_{\tau',u',B'}}", hook] & {H(\lambda,\mu)}
\end{tikzcd}
\end{center}
Such embedding depends only on the choice of an automorphism $f\colon E\to E$ fixing zero (finite number of choices), $\tau'$ in the $\operatorname{SL}_2(\Z)$-orbit of $\tau$ (countable number of choices) and $u'<0$ such that $A'=q^{u'}$ (countable number of choices). Our claim follows.
\end{proof}







%%%%%%%%%%%%%%%%%%%%%

%%%%%%%%%%%%%%%%%%%%%%%%%

\subsection{Secondary Hopf surfaces}

%%%%%%%%%%%%%%%%

\subsubsection{}\label{secondary-with-non-torsion}
We have just classified primary Hopf surfaces that arise as compactifications of $\Tot(L)$. We will focus in this subsection on compactifications that are secondary Hopf surfaces. Every secondary Hopf surface is a quotient of a primary Hopf surface by a free action of a finite group \cite[Sect.10, p.695]{KodairaII}. Assume that $S$ is a secondary Hopf surface containing an elliptic curve with non-torsion normal bundle. Then $S$ is a quotient of a diagonal Hopf surface $H(\lambda,\mu)$ by an operator $\diag(q^{1/n},q^{r/n})$, where $q^{1/n}$ and $q^{r/n}$ are primitive $n$-roots of unity \cite[Sect.\:10, Thm.\:32]{KodairaII}.



\begin{theorem}
\label{embeddings into all hopfs}
    Let $L$ be a degree-zero non-torsion line bundle on an elliptic curve $E$. Then the set of equivalence classes of open embeddings $\Tot(L)\stackrel{\iota}\to H$ of $\Tot(L)$ into a Hopf surface (primary or secondary) is countable.
\end{theorem}
\begin{proof}


We know the claim for embeddings into primary Hopf surfaces (Proposition \ref{which hopfs arise}), hence it is enough to assume that $H$ is a secondary Hopf surface. By \ref{secondary-with-non-torsion}, every secondary Hopf surface compactifying $\Tot(L)$ must be of the form $H/\diag(q^{1/n},q^{r/n})$ for $(r,n)=1$, where $H$ is a diagonal Hopf surface. Given an embedding of $\Tot(L)$ into $H/\diag(q^{1/n},q^{r/n})$, we can construct an embedding of $\Tot(f^*L)$ to $H$, where $f\colon E'\to E$ is a quotient by an $n$-torsion element of $E'$. The set of possible $f$ is countable, and so is the set of equivalence classes of embeddings of $\Tot(f^*L)$ into a primary Hopf surface, hence the claim.
\end{proof}
%%%%%%%%%%%%%%%%%%%%%%%%%%%%%%%%%%
%%%%%%%%%%%%%%%%%%%%%%%%%%%%%%%%%%%%

\subsection{Hopf duality and analytic cobordance}

\subsubsection{} Let $L \to E$ be a non-trivial degree-zero line bundle. The ruled surface $\P(\O_E \oplus L)$ contains two sections: $\P(\O)$ and $\P(L)$. The map $\Tot(L) \to \P(\O\oplus L)$, $v \mapsto [1 : v]$ defines a biholomorphism between $\Tot(L)$ and $\P(\O\oplus L) - \P(L)$. Similarly, $\P(\O\oplus L) - \P(\O)$ is biholomorphic to $\Tot(L^{-1})$. Hence the surface $\P(\O_E\oplus L)$ is a compactification of total spaces of dual line bundles: $\Tot(L)$ and $\Tot(L^{-1})$. The analogous property of the diagonal primary Hopf surfaces motivates the following definition.

\begin{definition}
\label{def of duals and cobordant}
    Let $L \to E$ and $L' \to E'$ be two degree-zero line bundles on elliptic curves. They are called 
    \begin{itemize}
    \item {\bf Hopf dual} if there exists a Hopf surface $H$ with two elliptic curves $E_\lambda$ and $E_\mu$ such that $H - E_\mu\simeq \Tot(L)$ and $H - E_\lambda\simeq \Tot(L')$;
    \item {\bf analytically cobordant} if $L^{-1}\to E$ and $L'\to E'$ are either Hopf dual, or $E=E'$ and $L=L'$.
    \end{itemize}
\end{definition}

By Proposition \ref{which hopfs arise}, any non-torsion line bundle admits a countable number of Hopf dual and analytically cobordant line bundles. The base curves of analytically cobordant line bundles are in general non-isomorphic (\ref{subsection on sl2 action}). 

\subsubsection{}\label{Hopf-duals}
    Let $L \to E$ and $L' \to E'$ be Hopf dual line bundles. Then the spaces $\Tot(L) - 0_L$ and $\Tot(L') - 0_{L'}$ are biholomorphic, where $0_{\Xi} \subset \Tot(\Xi)$ denotes the zero section. Indeed, both spaces are biholomorphic to $H(\lambda,\mu) - \left(E_\lambda\cup E_\mu\right)$. The biholomorphism sends the neighborhood of the zero section in one to the neighborhood of the infinity section in the other and vice versa. Similarly, let $L \to E$ and $L' \to E'$ be analytically cobordant line bundles. Then $\Tot(L) - 0_L$ and $\Tot(L') - 0_{L'}$ are biholomorphic so that a neighbourhood of the zero section in one maps to the neighbourhood of the zero section in the other.

\subsubsection{} Of course, if $E \not\cong E'$, the biholomorphism $\Tot(L) - 0_L \simeq \Tot(L') - 0_{L'}$ cannot be extended to the zero section. To internalize it, let us understand the behavior of fibers of the bundle under this biholomorphism. Let $H = H(\lambda,\mu)$ be a Hopf surface. The fibers of $H - E_\mu = \mathcal L(\lambda,\mu) \to E_\lambda$ are vertical lines. When we project $\C^2-\{0\}$ to $H(\mu,\lambda)$, we identify vertical lines $x = c, x = ac, x = a^2c$ etc. Hence, the image of a vertical line in $H(\mu,\lambda)$ is a non-closed horn-like subspace with the curve $E_\mu$ in its closure.

\begin{proposition}
\label{isomorphism and cobordance}
    Let $L\to E$ and $L'\to E'$ be two line bundles on elliptic curves. Then
    \begin{enumerate}
        \item $L$ and $L'$ are Hopf dual if and only if there exists a biholomorphism $\Tot(L) - 0_L$ and $\Tot(L') - 0_{L'}$ sending a neighborhood of $0_L$ to a neighborhood of infinity in $\Tot(L')$.
        \item $L$ and $L'$ are analytically cobordant if and only if there exists a biholomorphism $\Tot(L) - 0_L$ and $\Tot(L') - 0_{L'}$ sending a neighborhood of $0_L$ to a neighborhood of $0_{L'}$.
    \end{enumerate}
\end{proposition}

\begin{proof}
    The second statement follows immediately from the first, so we will only prove the first one. We saw in (\ref{Hopf-duals}) that Hopf dual line bundles satisfy the condition of the theorem. Conversely, suppose there is a biholomorphism $\Tot(L) - 0_L$ and $\Tot(L') - 0_{L'}$ as in the proposition. Let us glue $\Tot(L)$ and $\Tot(L')$ by this isomorphism. The result is a compact Hausdorff surface compactifying $\Tot(L)$. By Enoki's theorem, it is either a ruled or a Hopf surface, hence the claim.
\end{proof}



%%%%%%%%%%%%%%%%%%%%
\subsubsection{}
By Proposition \ref{isomorphism and cobordance}, analytic cobordance is an equivalence relation on the set of pairs $(E,L)$ where $E$ is an elliptic curve and $L$ is a line bundle on $E$. However, if $L$ is analytically cobordant to $L'$ and $L''$ through primary Hopf surfaces, then $L'$ and $L''$ can be analytically cobordant through a secondary Hopf surface.

\subsubsection{} The surface $M = \Tot(L)$ carries a holomorphic symplectic form $\sigma$ with a simple pole at the zero section. Indeed, $$K_M = \pi^*K_E \otimes K_{M/E} = K_{M/E} = \pi^*L^*.$$ 
The second isomorphism holds because $K_E$ is trivial. The pullback $\pi^*L$ has the tautological section, which vanishes on $0_L$; thus $K_M = \pi^*L^*$ has a nowhere zero section with a simple pole along $0_L$.



\subsubsection{} \label{cobord-symplect} Consider non-torsion analytically cobordant line bundles $L$ and $L'$. Let $\sigma$ and $\sigma'$ be the holomorphic symplectic forms on $\Tot(L) - 0_L$ and $\Tot(L') - 0_{L'}$ The pullback of $\sigma'$ under the holomophic isomorphism $f\colon \Tot(L) - 0_L \to \Tot(L') - 0_{L'}$ equals $c\sigma$ for a non-zero constant $c$. Indeed, $f^*\sigma' = g\sigma$ for a holomorphic function $g$ on $\Tot(L) - 0_L$. By \cite[Lemma 2.2]{Koike_Uehara_non_proj}, all holomorphic functions on $\Tot(L) - 0_L$ are constant.

%%%%%%%%%%%%%%%%%%%%%%%

\subsection{Hopf transforms}

\subsubsection{} A tubular neighborhood of a complex submanifold is in general not biholomorphic to a neighborhood of the zero section of its normal bundle. For example, a smooth cubic $C \subset \P^2$ can be deformed to a non-isomorphic curve, whereas all deformations of the zero section of $\Tot(\nu_{C/\P^2})$ are isomorphic to $C$. Therefore, neighborhoods of $C$ in $\P^2$ and in $\Tot(\nu_{C/\P^2})$ cannot be biholomorphic. However, analogues of the tubular neighbourhood theorem are known in certain situations.

\begin{definition}\label{diophantine-definition}
    Let $L \to E$ be a degree-zero line bundle on an elliptic curve, and $d$ an translation-invariant metric on the Picard variety $\mathrm{Pic}^0(E)$. Suppose that $$-\log d(\O_E, L^{\otimes n}) = O(\log n).$$ Then $L$ is called {\bf Diophantine}.
\end{definition}

The Diophantine property does not depend on the choice of the metric $d$. The set of Diophantine line bundles on $E$ is the complement to a measure-zero set in $\mathrm{Pic^0}(E)$.

\begin{theorem}[Arnold--Ueda theorem {\cite[4.3]{Arnold}} {\cite[Th. 3]{Ueda}}]
\label{Arnold}
    Let $S$ be a complex surface and $E \subset S$ an elliptic curve. If the normal bundle $\nu_{E/S}$ is non-torsion and of degree zero, then the formal neighbourhoods of $E$ in $S$ and in $\Tot(\nu_{E/S})$ are isomorphic. If $\nu_{E/S}$ is Diophantine, then this isomorphism extends to a biholomorphism of an analytic neighborhood of $E$ in $S$ with a neighborhood of $E$ in $\Tot(\nu_{E/S})$.
\end{theorem}

\subsubsection{} The explicit description in \cite[4.3]{Arnold} implies that if $L\to E$ and $L'\to E'$ are analytically cobordant through a primary Hopf surface and $L$ is Diophantine, then so is $L'$. It follows easily from this and \ref{secondary-with-non-torsion} that the Diophantine property is preserved by any analytic cobordism.



%%%%%%%%%%%%%%%%%%%%%%%%%%%

\subsubsection{} Arnold--Ueda theorem fails in the case of torsion line bundles. A counterexample is a fiber $F$ of a non-isotrivial elliptic fibraton. It has a holomorphically trivial normal bundle, yet, its neighbors are not isomorphic to $F$.


\begin{definition}
    \label{Hopf_transform}
    Let $E \subset S$ be an elliptic curve on a surface $S$ and $L$ its normal bundle. Assume that $\deg_EL = 0$ and $L$ satisfies the Diophantine condition. Let $L' \to E'$ be a bundle analytically cobordant to $L \to E$ (Definition \ref{def of duals and cobordant}). Pick a tubular neigbourhood of $E \subset S$ as in Theorem \ref{Arnold}, throw away $E$, and glue the tubular neighbourhood of the zero section of $L' \to F$ through a holomorphic isomorphism between $\Tot(L)-0_L$ and $\Tot(L')-0_{L'}$. The resulting surface $S'$ is called the {\bf Hopf transform} of $S$ in $E$ by $L'\to E'$. The image of the curve $E'$ in the Hopf transform is called the {\bf graft}.
\end{definition}

%%%%%%%%%%%%%%%%%%%%%%

\subsubsection{} The new surface $S'$ may not be algebraic or even K\"ahler even if $S$ is. For example, take $S = \P(\O_E\oplus L)$ and embed $E$ as one of the sections. Then $S'$ is a Hopf surface.

%%%%%%%%%%%%%%%%%%%%%%

\subsubsection{} The Hopf transform of a given surface in a given curve can be made in a countable number of non-isomorphic ways due to Theorem \ref{embeddings into all hopfs}. Nevertheless, the result of a Hopf transform depends only on the choice of an analytically cobordant line bundle and not on an isomorphism between a neighbourhood of $E$ in $S$ and in $\Tot(L)$. The proof relies on the following Lemma.

%%%%%%%%%%%%%%%%%%%%

\begin{lemma}
\label{open embeddings}
    Let $L$ be a non-torsion degree-zero line bundle on an elliptic curve $E$. For $r>0$, denote by $W_r$ the subset of vectors in $\Tot(L)$ of length less than $r$. Consider an open embedding $j\colon W_r \to W_R$ identical on $E$, where $r,R\in \R_{>0}\sqcup\infty$. Then $j$ is the fiberwise multiplication by a constant $c\in \C^*$ such that $|c|\le R/r$.
\end{lemma}
\begin{proof}
{\bf Step 1.} The space $W_r$ (resp.\! $W_R$) is isomorphic to $\C\times B_r/\Lambda$ (resp. $\C\times B_R/\Lambda$) where $\Lambda:= \Z + \Z\cdot\tau$ acts as follows: 
$$
\gamma\cdot(x,y) = (x+\gamma, \rho(\gamma)y).
$$
Here $B_r$ (resp. $B_R$) is the open disk of radius $r$ (resp. $R$) and $\rho\colon \Lambda\to U(1)$ is the monodromy representation. By the universal property of universal covers, we can lift $j$ to an open embedding $J\colon \C\times B_r\to \C\times B_{R}$. The map $J$ descends to quotients, hence for every $\gamma\in \Lambda$ there exists $\gamma'\in \Lambda$ such that $J(\gamma\cdot(x,y)) = \gamma'\cdot J(x,y)$. When $y=0$ the map $J$ is the identity, hence $\gamma' = \gamma$. We conclude that $J$ is $\Lambda$-equivariant.

\hfill

%%%%%%%%%%%%%%%%%

{\bf Step 2.} Write $J$ as $(f,g)$, where $f\colon \C\times B_r \to \C$ and $g\colon \C\times B_{r}\to B_{R}$. We will see in this step that $g(x,y) = cy$ for a constant $c\in\C^*$. For a fixed $y\in B_r$, the map $g$ is a holomorphic function from $\C$ to $B_{R}$, hence constant. We conclude that $g = g(y)$. The $\Lambda$-equivariance of $J$ implies that $g(\rho(\gamma)y) = \rho(\gamma)g(y)$. The image of $\Lambda$ in $U(1)$ is dense because $L$ is non-torsion. Therefore, $g$ is $U(1)$-equivariant, and our claim follows.

\hfill

%%%%%%%%%%%%%%%%%%%

{\bf Step 3.} By $\Lambda$-equivariance of $J$ we have that 
\begin{equation}
\label{aaa}
f(x+\gamma, \rho(\gamma)y) = f(x,y) + \gamma.
\end{equation} 
Let as differentiate the equation (\ref{aaa}) $k$ times with respect to $y$. We obtain
$$
\rho^k(\gamma)\frac{\di^k f}{\di y^k}(x+\gamma,\rho(\gamma)y) = \frac{\di^k f}{\di y^k}(x,y)
$$
Set $y=0$. The function $\left|\frac{\di^k f}{\di y^k}(x,0)\right|$ is invariant under shifts by $\gamma\in \Lambda$, hence bounded. We conclude that $\frac{\di^k f}{\di y^k}(x,0)$ is constant. The representation $\rho^k$ is non-trivial for every $k$, hence $\frac{\di^k f}{\di y^k}(x,0)$ vanishes. We conclude that $f$ does not depend on $y$. Hence $f(x,y) = f(x,0) = x$. We showed that $J(x,y) = (x,cy)$, hence the claim.
\end{proof}

%%%%%%%%%%%%%%%%%%%%

\begin{corollary}
Let $E\subset S$ be as in Definition \ref{Hopf_transform}. Fix a line bundle $L'$ analytically cobordant to $L\to E$ (Definition \ref{def of duals and cobordant}). Then the Hopf transform of $S$ in $E$ by $L'$ is well-defined. Namely, consider two biholomorphisms $\phi_1\colon U_1\stackrel{\sim}\to W_1$ and $\phi_2\colon U_2\stackrel{\sim}\to W_2$  between neighborhoods $U_1$ and $U_2$ of $E$ in $S$ and neighborhoods $W_1$ and $W_2$ of $E$ in $\Tot(L)$. Assume that $\phi_1|_E = \phi_2|_E$. Then the Hopf transforms $S_1$ and $S_2$ of $S$ obtained through $\phi_1$ and $\phi_2$ respectively are biholomorphic.   
\end{corollary}



%%%%%%%%%%%%%%%%%%%%%%%%%

\begin{proof}The Hopf transforms $S_i$, $i=1,2$, will not change if we shrink $U_i$. Hence we may assume that $U_1\subset U_2$ and $\phi_1(U_1) = W_r$, $\phi_2(U_2)=W_{R}$ for some $r,R\in\R_{>0}$. The map $j:= \phi_2|_{U_2}\circ\phi_1^{-1}$ is an open embedding $W_r\to W_R$. Lemma \ref{open embeddings} implies that $j$ is the multiplication by a constant $c\in\C^\times $. Embed $\Tot(L^{-1})$ into a Hopf surface $H(\lambda,\mu)$ such that $H(\lambda,\mu)-E_\lambda\simeq \Tot(L^{-1})$ and $H(\lambda,\mu)-E_\mu\simeq \Tot(L')$. The biholomorphism $j$ extends to the automorphism $\diag(1,c)$ of $H(\lambda,\mu)$ by \ref{multiplication by a constant}. This automorphism of $H(\lambda,\mu)$ induces a biholomorphism $S_1\to S_2$.
\end{proof}


%%%%%%%%%%%%%%%%%%%%%%%%%%%%%%







\subsubsection{} We only define the Hopf transform 
in elliptic curves with Diophantine normal bundle. The same definition 
works for any square-zero elliptic curve with
a holomorphic tubular neighbourhood. 

\hfill

An equivalent formulation of Enoki's theorem (Theorem \ref{enoki-classification}) 
is that every compactification of $\Tot(L)$ is a Hopf 
transform of $\mathbb P(\O\oplus L)$ in the infinity section. 
This statement generalizes partially 
to other complex surfaces.

%%%%%%%%%%%%%%%%%

\begin{theorem}
\label{all compactifications}
Suppose $M$ is a complex surface realizable as the complement to a
square-zero elliptic curve $E$ with a Diophantine normal bundle $L$ in a 
compact surface $S$. Then every minimal analytic compactification of $M$ is 
a Hopf transform of $S$ in $E$.
\end{theorem}

\begin{proof}
The Diophantine condition implies that the neighborhood of infinity in $X$ 
is biholomorphic to a neighborhood of the zero section in $\Tot(L)$ with 
the zero section removed (Theorem \ref{Arnold}). Hence, every partial compactification of $\Tot(L) - 0_L$ 
near the zero section produces a compactification of $M$ and vice versa. 
By Theorem \ref{embeddings into all hopfs} and Enoki's theorem, every partial compactification of $\Tot(L)-0_L$ is a Hopf transform at $0_L$.
\end{proof}

%%%%%%%%%%%%%%%%%%%%%%%%%%%%%

\begin{corollary}
    A Hopf transform of a Hopf transform in its graft is either a Hopf transform, or the initial surface. Inverse of a Hopf transform is a Hopf transform.
\end{corollary}
\begin{proof}
The composition of Hopf transforms of $S$ in $E$ is a compactification of $S - E$, hence a Hopf transform (Theorem \ref{all compactifications}).
\end{proof}

\begin{corollary}
\label{algebraic_structures}
    Let $S$ be a projective surface with a square-zero elliptic curve $E$ with a Diophantine normal bundle. Suppose that each Hopf transform of $S$ in $E$ is projective. Then the set of algebraic structures on $M:=S-E$ is countable.
\end{corollary}
\begin{proof}
    By Theorem \ref{all compactifications}, every compactification of $M$ is a Hopf transform of $S$ in $E$. Thus every compactification induces an algebraic structure on $M$. Conversely, every algebraic structure is induced from a compactification. Therefore, the set of compactifications $(S',E')$ of $M$ is countable (Theorem \ref{embeddings into all hopfs}). Moreover, the set of possible $E'$ is also countable. So, it is enough to prove that two compactifications $(S',E')$ and $(S'',E'')$ such that $E'\not\simeq E''$ induce distinct algebraic structures on $M$. An algebraic isomorphism between $\phi\colon S'-E'\to S''-E''$ would induce a birational morphism $\phi\colon S' \DashedArrow[->,densely dashed    ]  S''$. A birational map of smooth surfaces cannot contract a non-rational curve, hence $E'\simeq \phi(E')\simeq E''$, contradiction.
\end{proof}







%%%%%%%%%%%%%%%%%%%%%%
%%%%%%%%%%%%%%%%%%%%%%
%%%%%%%%%%%%%%%%%%%%%%





\section{Surfaces with square-zero elliptic curves}\label{square-zero-section}

%%%%%%%%%%%%%%%%
%%%%%%%%%%%%%%%%%%%%

\subsection{Kodaira dimension}

%%%%%%%%%%%%%%%%%%

\subsubsection{} 
\label{kodaira_not_two}
Let $S$ be a complex surface (not necessarily projective) containing a square-zero elliptic curve $E$. Then its Kodaira dimension $\kappa(S)$ is at most one. Indeed, suppose $\kappa(S) = 2$. Then the map $\phi\colon S \to \P^N$ induced by the linear system $|nK_S|$ is birational onto its image for sufficiently large $n$. Every square-zero elliptic curve $E$ satisfies $K_S\cdot E = 0$. Hence the map $\phi$ must contract $E$. Yet, a birational map of surfaces cannot contract a square-zero curve.

%%%%%%%%%%%%%%%%%

\begin{proposition}
    Let $S$ be a surface with square-zero elliptic curve $E\subset S$ whose normal bundle $\nu_{E/S}$ is non-torsion. Then $\kappa(S)=-\infty$.
\end{proposition}
\begin{proof}
    Observe that $\nu_{E/S}\simeq \O_E(-K_S|_E)$. We consider two cases. 
    
    {\bf Case 1: $\kappa=0$.} Every minimal surface with $\kappa = 0$ has torsion canonical bundle, hence the statement is trivial in this case. If $S$ is non-minimal, then $K_S = \sum E_i +D$, where $E_i$'s are exceptional curves and $D$ is a torsion divisor. We get that $K_S|_E$ is either torsion or has positive degree for every curve $E\subset S$. 
    
    {\bf Case 2: $\kappa=1$.} The linear system $|nK_S|$ for $n\gg0$ induces a morphism $\phi \colon S\to \P^N$ whose image is a curve. It is an elliptic fibration. Elliptic curves in the fibers of $\phi$ have torsion normal bundle, so we can consider only horizontal curves. For every horizontal curve $C$, we have $n K_S|_C = (\phi|_C)^*H$, where $H$ is the hyperplane section. Therefore, $K_S\cdot C$ is positive. 
\end{proof}

%%%%%%%%%%%%%%%%%

\subsubsection{} \label{trichotomy} A surface of negative Kodaira dimension is one of the following:

\begin{itemize}
    \item rational;
    \item birational to a ruled surface;
    \item of class VII, i.e., a non-K\"ahler surface with $\kappa=-\infty$ and $b_1=1$.
\end{itemize}

If a blow-up of a ruled surface $S$ contains an elliptic curve, then $S$ is ruled over an elliptic curve.

\begin{proposition}\label{anti-canon-means-P2}
    Let $E \subset S$ be an anti-canonical elliptic curve with a degree-zero non-torsion normal bundle $\nu_{E/S}$. Assume  $b_1(S)=0$. Then $S$ is a blow-up of a length nine subscheme in $\P^2$.
\end{proposition}
\begin{proof}
    The canonical class of $S$ is anti-effective, hence all plurigenera $p_n := h^0(K_S^n)$ of $S$ vanish. Since $b_1(S) = 0$, the irregularity $q = h^1(\O_Y)$ vanishes as well. Castelnuovo theorem implies that $S$ is rational, in particular, projective \cite[VI (3.4)]{Barth_Hulek_Peters_Van_de_Ven}. Its minimal model is either $\P^2$ or a Hirzebruch surface $\F_n = \P\left(\O_{\P^1}\oplus\O_{\P^1}(-n)\right)$. The image of $E$ in a minimal model is an irreducible anti-canonical curve: indeed, $E$ is anti-canonical, so it intersects each $(-1)$-curve transversely at one point. Since $K_S^2=0$, $S$ is either a blow-up of nine points in $\P^2$ or of eight points in $\F_n$. 

\hfill

    The Hirzebruch surface $\F_n$ contains a section $C_n$ of square $-n$. Projection formula yields $-K_{\F_n}\cdot C = 2-n$, thus for $n>2$ an anti-canonical curve contains $C$ and is reducible. Therefore, any minimal model of $S$ is $\P^2$, $\F_0 = \P^1\times\P^1$, or $\F_2$. An anti-canonical curve in $\F_2$ does not intersect $C_2$. A blow-up of a point away from $C_2$ is isomorphic to a blow-up of a length-two subscheme in $\P^2$. A blowup of any point on $\F_0 = \P^1\times\P^1$ is isomorphic to a blowup of two points on $\P^2$. Thus $S$ is a blow-up of $\P^2$.
\end{proof}




%%%%%%%%%%%%%%%%%%%%%%%%%%%
%%%%%%%%%%%%%%%%%%%%%%%%%%%

\subsection{Rational surfaces}

%%%%%%%%%%%%%%%%%%%%%%%%%%

Let us start with the following lemma.% we shall make use of almost instantly

\begin{lemma}\label{excision-lemma}Let $S$ be a surface, and $C \subset S$ a smooth curve. Then the natural map $H_1(S-C,\Q) \to H_1(S,\Q)$ is surjective. If $S$ is K\"ahler, then it is also injective.
\end{lemma}
\begin{proof}
    Look at the long exact sequence of cohomology associated with the decomposition $S = C \sqcup (S-C)$:
        $$\dots \to H^2(S,\Q) \to H^2(C,\Q) \to H^3_c(S-C,\Q) \to H^3(S,\Q) \to H^3(C,\Q) = 0.$$
    If $S$ is K\"ahler, the class of $C$ in $H_2(S)$ is non-trivial. Therefore the map $H^2(S,\Q)\to H^2(C,\Q)$ is surjective, and $H_1(S-C,\Q) \simeq H^3_c(S-C,\Q)\simeq H^3(S,\Q)\simeq H_1(S,\Q)$.
\end{proof}

The next statement follows easily from Lemma \ref{excision-lemma}, and we omit its proof.

\begin{corollary}
    \begin{enumerate}
        \item Let $S$ be a rational surface, $C \subset S$ a smooth curve. Then $h_1(S-C)=0$.
        \item Let $S$ be a blow-up of a ruled surface over an elliptic curve, $C \subset S$ a smooth curve. Then $h_1(S-C)=2$. 
    \end{enumerate}
\end{corollary}

\begin{corollary}\label{Hopf-transform-preserves-rationality}
    \begin{enumerate}
        \item A Hopf transform of a rational surface is rational.
        \item A Hopf transform of a surface birational to a ruled surface is birational either to a ruled surface or a Hopf surface.
    \end{enumerate}
\end{corollary}
\begin{proof}
    Let $S$ be a rational surface. Lemma \ref{excision-lemma} implies that $h_1(S-C) =0$. Consider a Hopf transform $(S',C')$ of $S$ in $C$. The space $S'-C'$ is biholomorphic to $S-C$, hence $h_1(S'-C') = h_1(S-C)$. Lemma \ref{excision-lemma} implies that $h_1(S'-C') \geqslant h_1(S')$. Thus a Hopf transform of a rational surface satisfies $h_1(S') = 0$. The classification in \ref{trichotomy} implies that $S'$ is rational. 
    
    A Hopf transform of a blow-up of a ruled surface cannot be rational by the previous statement. By \ref{trichotomy} its Hopf transform is either a blow-up of a ruled surface or a Hopf surface.
\end{proof}


\subsubsection{} Choose an elliptic curve $E$ in $\P^2$ and nine points $p_1,...p_9\in E$ (points are allowed to collide). Denote the hyperplane section of $\P^2$ as $H$. Assume that the line bundle $\O_E(3H-p_1 -...-p_9)$ satisfies the Diophantine condition (Definition \ref{diophantine-definition}). This line bundle is isomorphic to the normal bundle to the strict transform of $E$ in the blow-up $X$ in $p_1,...p_9$. Therefore, a Hopf transform $Y$ of $X$ in $E$ is well-defined.

\begin{theorem}
\label{hopf_of_blow_up9}
    Any Hopf transform $Y$ of $X$ is the blow-up of $\P^2$ in a length nine subscheme $\gamma$. The graft is the strict preimage in $Y$ of the elliptic curve $F \subset \P^2$ passing through $\gamma$.
\end{theorem}
\begin{proof} 
{\bf Step 1.} The surface $(X,E)$ is log Calabi--Yau, that is, $K_X+E=0$. This implies existence of a meromorphic symplectic form on $X$ with a simple pole along $E \subset X$. We know from \ref{cobord-symplect} that a Hopf transform is a gluing along a holomorphic symplectomorphism, hence $Y$ carries a symplectic form with a simple pole along $F \subset Y$. Thus $K_Y+F=0$.



\hfill



{\bf Step 2.} By Corollary \ref{Hopf-transform-preserves-rationality}, $Y$ is rational. Moreover, its anti-canonical divisor is effective. By Proposition \ref{anti-canon-means-P2}, $Y$ is a blowup of a length-nine subscheme in $\P^2$, and $F \subset Y$ is the strict transform of the only plane cubic passing through it. 
\end{proof}

The following question remains open:

\begin{problem}
   Determine the nine points $q_1,...q_9\in \P^2$ such that $Y \simeq \Bl_{q_1,..q_9}(\P^2)$. 
\end{problem}

%%%%%%%%%%%%%%%%%%%%%%%%%%%%%%%

\begin{theorem}\label{many-algebraic-structures}
    Let $X$ be the blowup of $\P^2$ in a very generic length-nine subscheme $\gamma$, and $E \subset X$ the strict transform of the plane cubic passing through $\gamma$. Then the analytification of $X-E$ admits countably many algebraic structures.
\end{theorem}
\begin{proof}
The claim follows from Corollary \ref{algebraic_structures} because every Hopf transform of $X$ is projective.

\end{proof} 



%%%%%%%%%%%%%%%%%%%%%%%%%%%

\subsubsection{} There are rational surfaces containing a not anti-canonical elliptic curve of square zero. Let $Q \subset \P^2$ be a plane quartic with two nodes. Blow them up; the strict transform $\tilde{Q}$ has square $8$. By blowing up eight more points on $\tilde{Q}$, one gets a square-zero elliptic curve on $\P^2$ blown up in ten points. Notice that `having a node at a given point' is a codimension three condition, thus plane quartics with two nodes at $p,q\in\P^2$ form an eight-dimensional space. Thus, a generic tuple of ten points with two distinguished ones determines a unique plane quartic with two nodes through it.

\subsubsection{} By \cite[Lemma 2.2]{Koike_Uehara_non_proj}, punctured neighbourhood of a square-zero elliptic curve with non-torsion normal bundle carries no nonconstant functions. Hence none of the surfaces $X-E$ from Theorem \ref{many-algebraic-structures} is Stein.




%%%%%%%%%%%%%%%%%%%%%%%%
%%%%%%%%%%%%%%%%%%%%%%%%%%%%%%%



\subsection{Surfaces of class VII}

%%%%%%%%%%%%%%%%%%


All surfaces of class VII that contain an elliptic curve with a non-torsion degree-zero normal bundle are Hopf surfaces, as we prove below.

    \begin{proposition}\label{classification-for-class-vii}
        Let $S$ be a surface of class VII and $E \subset S$ a smooth elliptic curve with non-torsion normal bundle $\nu_{E/S}$ of degree zero. Then $S$ is a primary Hopf surface $H(\lambda,\mu)$ with $\lambda^n\neq\mu^m$ for any $n,m\in\Z$, or a secondary Hopf surface $H(\lambda,\mu)/\mu_n$ from \ref{secondary-with-non-torsion}, or a blow-up of such a surface away from $E$.
    \end{proposition}
    \begin{proof}
        Every curve on a smooth non-K\"ahler surface has a non-positive square \cite[Ch.\:IV, Thm.\:2.14]{Barth_Hulek_Peters_Van_de_Ven}. Applying this fact to a minimal model of $S$, we obtain that a curve of square zero does not intersect exceptional curves.

        If $E \subset S$ is a nonsingular square-zero curve on a minimal surface $S$ of class VII, then $S$ is a Hopf surface \cite[Proposition 4.12]{Enoki}. Hopf surfaces that contain an elliptic curve with non-torsion normal bundle were classified in \ref{generalities on primary hopfs} and \ref{secondary-with-non-torsion}.
    \end{proof}



%%%%%%%%%%%%%%%%%%%%%%%%%%
%%%%%%%%%%%%%%%%%%%%%%%%%%%
%%%%%%%%%%%%%%%%%%%%%%%%%%%%%%

\section{Analytic Grothendieck group}\label{grothendieck-section}

\stepcounter{subsection}


\begin{definition}
    The {\bf Grothendieck group $K_0(\V_\C)$ of complex algebraic varieties} is the abelian group generated by classes of $\C$-varieties modulo scissor relations:
    $$[X - Y] = [X]-[Y],$$
    where $Y \subset X$ is a closed algebraic subvariety.\footnote{It is actually a ring with multiplication $[U]\cdot[V] = \left[{U}\times_k{V}\right]$, but we do not use it.}
\end{definition}

We introduce an interesting quotient of this group.

\begin{definition}\label{analytic-grothendieck-definition}
    The {\bf analytic Grothendieck group of varieties} $K_0^{an}$ is the quotient of $K_0(\V_\C)$ by differences $[U]-[V]$ where $U$ and $V$ are biholomorphic varieties.
\end{definition}


\begin{proposition}
\label{analytic K zero}
    The classes of any two elliptic curves in $K_0^{an}$ are equal.
\end{proposition}
\begin{proof}
    {\bf Step 1.} Suppose that $E, F$ are two elliptic curves and $L\to E$, $L' \to F$ are two analytically cobordant Diophantine line bundles. Then $[E] = [F] \in K_0^{an}$. Indeed, let $X$ be a blow-up of nine points in $\P^2$ such that the cubic passing through them is isomorphic to $E$, and the normal bundle of its strict transform is isomorphic to $L$. Even in $K_0(\V_{\C})$, one has $[X] = \mathbb{L}^2 + 10\mathbb{L} + [\mathrm{pt}]$, where $\mathbb{L}$ is the class of the affine line. By Theorem \ref{hopf_of_blow_up9}, the Hopf transform $Y$ of $X$ in $E$ by $L'\to F$ is also a blow-up of $\P^2$ in a length nine subscheme. Thus $[Y] = \mathbb{L}^2 + 10\mathbb{L} + 1 = [X]$. It follows from the relations $[X] = [E] + [X-E]$ and $[Y] = [F] + [Y-F]$ that
    $$
    [E]^{an} = [X]^{an} - [X-E]^{an} = [Y]^{an} - [Y-F]^{an} = [F]^{an} \in K_0^{an}.
    $$

    {\bf Step 2.} For any pair of elliptic curves $E$, $F$, there exists a primary Hopf surface $H = H_{E,F}$ containing $E$ and $F$. For a fixed $E$, the set of curves $F$ for which the normal bundle $\nu_{E/H_{E,F}}$ is not Diophantine is meagre in the moduli space of elliptic curves. Thus for arbitrary two curves $E$, $E'$, the complements of the corresponding meagre sets have nonempty intersection. Let $F$ be any curve from this intersection. Step 1 implies that $[E]^{an} = [F]^{an}$ and $[F]^{an} = [E']^{an}$, hence the claim.
\end{proof}


\subsubsection{} \label{Bittner-theorem} Although the classes of elliptic curves in $K_0^{an}$ 
coincide, the classes of different elliptic curves in $K_0(\V_\C)$ are distinct, as 
follows from Franziska Bittner's description of $K_0(\V_\C)$ in \cite{Bittner}. She proved 
that $K_0(\V_\C)$ is generated by classes of smooth projective varieties modulo relations 
\begin{equation}\label{blow-up}
[X] - [Y] = [Bl_YX] - [E],
\end{equation}
where $Y$ is a smooth subvariety of a smooth projective variety $X$ 
and $E\subset Bl_YX$ 
is the exceptional divisor of the blow-up of $X$ in $Y$. This result is very powerful as it 
enables us to construct motivic measures (i.e., maps from $K_0(\V_\C)$ to an abelian group) 
by checking only the relation (\ref{blow-up}). For example, 
consider the functional on smooth projective varieties sending a variety to its stable 
birationality class. This functional descends to a motivic measure
$$
K_0(\V_\C) \to \Z[\mathrm{SB}]
$$
with values in the free abelian group generated by stable birationality classes. Indeed, 
$Bl_YX$ is birational to $X$ and $E$ is stably birational to $Y$, so the relation (\ref{blow-up}) 
is preserved. We see that non-stably birational varieties have different classes in $K_0(\V_\C)$. 
Two curves are stably birational if and only if they are isomorphic \cite[V Ex.\! 2.1]{Hartshorne}. In particular, 
non-isomorphic elliptic curves have different classes in $K_0(\V_\C)$. 

\subsubsection{} Topological Euler characteristic is a homomorphism $K_0^{an} \to \Z$. However, we were not able to find more interesting analytic motivic measures. This motivates the following

\begin{problem}
    Is the class of an elliptic curve in $K_0^{an}$ trivial? If so, is $K_0^{an}$ isomorphic to $\Z$?
\end{problem}



\paragraph{Acknowledgements.} We are grateful to Piotr Achinger, to whom we owe the idea of Section \ref{grothendieck-section}. Many thanks to Robert Friedman, Micha\l~Kapustka, Takayuki Koike, Aleksandr Petrov, and Giulia Sacc\`a for insightful discussions, and to Nathan Chen, Andr\'es Fern\'andez Herrero, and Morena Porzio for reading an early draft of the paper.


%\begin{problem}
%    Do Hopf transforms exist for rigid analytic spaces? If so, how similar are the sets of algebraic structures on a given analytic surface in complex and $p$-adic world?
%\end{problem}

%\begin{problem}
%    The last problem is rather archaeological than mathematical. The~analytic cobordism classes, viewed as equivalence classes on $\mathrm{Pic}^0$ of the~universal elliptic curve, is very similar to the orbits of an action of a group similar to an absolute Galois group. Historically the Arnold--Ueda theorem has been called (e.~g. by Arnold himself) the ``theorem on small denominators,'' reflecting a number-theoretic understanding of objects from complex geometry which has since been lost. Thus, the construction of formal tubular neighbourhood for a non-torsion line bundle is an analogue of the Diophantine approximation by the means of continued fraction expansion, and convergence of the power series parallels the Liouville's theorem on algebraic numbers. 
%
%    This kind of interplay between algebraic geometry and theory of algebraic numbers, which circumvents by-now usual realm of arithmetic geometry and passes through the domain of analysis instead, used to be very classical. 19th century mathematicians, describing Abel's work in what we call today the Galois theory, had praised him for ``expansion of radicals of algebraic expressions into continued fractions'' (cf. e.~g. \cite{Tchebychef}). Maybe it would be too na\"\i ve to expect that a theory exists that unifies Ueda's theory with Diophantine approximations, as well as algebraic and analytic Tate--Shafarevich twists, yet this is an interesting topic to explore. 
%\end{problem}
%
%This last problem seems to imply that the authors' purpose is overtly Ar\-nold\-i\-an. In fact, it's quite the opposite. To draw the possible accusations away, let us explain ourselves. 
%
%An important idea in magic is that our world---broadly speaking, not just the physical one,---is a covering space, and performing operations localised in one of its points $x$ can produce a consequence elswhere in the point related to $x$ by the Galois group action. This kind of action at a distance is radically different from the physical one, which is local in its nature and governed by the differential equations. The Moon is related to the water of the seas by the means of fields and forces, the sympathetic relation between a wound and the cold iron it inflicted is established not via Riemann's and Faraday's lines of force, but rather an incidence, belonging to the same group orbit. In reality the quotient by this group action may not be a manifold, its orbits being dense---and astrologists of the past understood this chaotic behaviour well. The tragedy of astrology, and at the same time the triumph of celestial mechanics was the reduction of the apparent randomness of the flux of heavenly bodies to mere integrating of differential equations. To find the group acting on $\mathrm{Pic}^0$ of the universal elliptic curve permuting the analytically cobordant line bundles would mean an amendment of the one-sided Newtonian turn and upholding the natural rights of the natural magic.

%%%%%%%%%%%%%%%%%%%%%%%%%%%%
%%%%%%%%%%%%%%%%%%%%%%%%%%%%%%%%
%%%%%%%%%%%%%%%%%%%%%%%%%%%%%%%%



\bibliographystyle{alpha} 
\bibliography{mybib.bib} 

\begin{multicols}{2}
%\footnotesize
\noindent {\sc {Anna Abasheva} \\
Columbia University,\\
Department of Mathematics, \\
2990 Broadway, \\
New York, NY, USA}\\
%also:\\
%{\sc Independent University of Moscow\\
%Bolshoy Vlasievskiy per., 11, \\
%Moscow, Russia}\\
{\tt anabasheva(at)math.columbia.edu}

\columnbreak

\noindent {\sc {Rodion D\'eev}\\
Institute of Mathematics, \\ Polish Academy of Sciences,\\
\'Sniadeckich 8, \\ 
Warsaw, Poland}\\
{\tt rdeev(at)impan.pl}

\end{multicols}

\end{document}
