\documentclass[prx,letterpaper,nobalancelastpage,twocolumn,superscriptaddress,nofootinbib]{revtex4-2}

\usepackage{graphicx}
\usepackage{amsmath}
\usepackage{bbold}
\usepackage{amssymb}
\usepackage[english]{babel}
\usepackage{color}
\usepackage[version=4]{mhchem}
\usepackage[hidelinks]{hyperref}
\usepackage{dsfont}
\usepackage{placeins}
\usepackage{mathtools}
\usepackage[normalem]{ulem}
\usepackage{lipsum}

\setcitestyle{super}

\usepackage{soul}
\setlength{\parindent}{8pt}
\setlength{\parskip}{0pt}

\frenchspacing

\newcommand{\HRule}{\rule{\linewidth}{0.5mm}}
\newcommand{\of}[1]{\left( #1 \right)}
\newcommand{\sqof}[1]{\left[ #1 \right]}
\newcommand{\abs}[1]{\left| #1 \right|}
\newcommand{\avg}[1]{\left< #1 \right>}

\newcommand{\cuof}[1]{\left \{ #1 \right \} }
\newcommand{\bx}{\mathbf{x}}
\newcommand{\by}{\mathbf{y}}
\newcommand{\bk}{\mathbf{k}}
\newcommand{\bp}{\mathbf{p}}
\newcommand{\bl}{\mathbf{l}}
\newcommand{\bq}{\mathbf{q}}
\newcommand{\br}{\mathbf{r}}
\newcommand{\mc}[1]{\mathcal{#1}}
\newcommand{\Uhat}{\widehat{U}}
\newcommand{\up}{\uparrow}
\newcommand{\down}{\downarrow}
\newcommand{\Rb}{$^{87}$Rb }
\newcommand{\ket}[1]{| #1 \rangle}
\newcommand{\bra}[1]{\langle #1 |}
\newcommand{\braket}[1]{\langle #1 | #1 \rangle}
\newcommand{\ketbra}[1]{|  #1 \rangle \langle #1 |}
\newcommand{\s}[1]{\substack{#1}}
\newcommand{\pro}[2]{\langle #1| #2 \rangle}
\newcommand{\defs}{:=}
\newcommand{\tr}[1]{\mathrm{tr}\left\{#1\right\}}
\newcommand{\lr}[1]{\left( #1 \right)}
\newcommand{\za}{Z_{A}}
\newcommand{\zb}{Z_{B}}
\newcommand{\pza}{p(\za)}
\newcommand{\pzazb}{p(\za|\zb)}

\newcommand{\Caltech}{California Institute of Technology, Pasadena, CA 91125, USA}
\newcommand{\MIT}{Center for Theoretical Physics, Massachusetts Institute of Technology, Cambridge, MA 02139, USA}
\newcommand{\Stanford}{Department of Electrical Engineering, Stanford University, Stanford, CA, USA}
\newcommand{\redst}[1]{\red{\st{#1}}}
\newcommand{\blue}[1]{{\color[rgb]{0,0,0.6}{#1}}}
\newcommand{\red}[1]{{\color[rgb]{0.8,0,0}{#1}}}
\newcommand{\green}[1]{{\color[rgb]{0,0.8,0}{#1}}}
\newcommand{\ALS}[1]{{\color[rgb]{0.5,0,0}{ALS: #1}}}
\newcommand{\ME}[1]{\blue{#1}}

\usepackage{caption}

\DeclareCaptionLabelSeparator{bar}{ \textbf{\textbar}~}
\captionsetup[figure]{labelfont={bf},name={Fig.},labelsep=bar,justification=raggedright,font=small}
\captionsetup[table]{labelfont={bf},name={Table},labelsep=bar,justification=raggedright,font=small}

\usepackage{enumitem}
\setlist{nolistsep}

\begin{document}

\title{Multi-ensemble metrology by programming local rotations with atom movements}
% \author{Us}\thanks{These authors contributed equally to this work}
% \affiliation{\Caltech}
% \affiliation{\MIT}
\author{Adam L. Shaw}\thanks{These authors contributed equally to this work}
\author{Ran Finkelstein}\thanks{These authors contributed equally to this work}
\author{Richard Bing-Shiun Tsai}
\author{Pascal Scholl}
\affiliation{\Caltech}
\author{\\Tai Hyun Yoon}\thanks{Permanent address: Department of Physics, Korea University, Seoul 02841, Republic of Korea}
\affiliation{\Caltech}
\author{Joonhee Choi}
\affiliation{\Caltech}
\affiliation{\Stanford}
\author{Manuel Endres}\email{mendres@caltech.edu}
\affiliation{\Caltech}

\maketitle

\textbf{Current optical atomic clocks do not utilize their resources optimally. In particular, an exponential gain could be achieved if multiple atomic ensembles were to be controlled or read-out individually, even without entanglement. However, controlling optical transitions locally remains an outstanding challenge for neutral atom based clocks and quantum computing platforms. Here we show arbitrary, single-site addressing for an optical transition via sub-wavelength controlled moves of tweezer-trapped atoms, which we perform with $99.84(5)\%$ fidelity and with $0.1(2)\%$ crosstalk to non-addressed atoms. The scheme is highly robust as it relies only on relative position changes of tweezers and requires no additional addressing beams. Using this technique, we implement single-shot, dual-quadrature readout of Ramsey interferometry using two atomic ensembles simultaneously, and show an enhancement of the usable interrogation time at a given phase-slip error probability, yielding a 2.55(9) dB gain over standard, single-ensemble methods. Finally, we program a sequence which performs local dynamical decoupling during Ramsey evolution to evolve three ensembles with variable phase sensitivities, a key ingredient of optimal clock interrogation. Our results demonstrate the potential of fully programmable quantum optical clocks even without entanglement and could be combined with metrologically useful entangled states in the future.}

Sensors based on quantum probes provide some of the most precise measurements in science~\cite{Ludlow2015,Safronova2018,Andreev2018,Roussy2022,Degen2017}. For many such systems, fundamental sensitivity limits can be improved through entanglement~\cite{Pezze2018,Macieszczak2014,Kaubruegger2021}, but in the presence of noise, a practical advantage of such schemes is not guaranteed~\cite{Huelga1997,Schulte2020}. A complementary approach studies optimal metrology with entanglement-free quantum control and readout methods, for which an important figure of merit is not just the sensitivity to a given observable, but also the dynamic range over which that observable can unambiguously be estimated~\cite{Rosenband2013,Borregaard2013}.

In the particular case of optical atomic clocks~\cite{Ludlow2015}, the observable of interest is the stochastically evolving phase of a laser acting as a local oscillator, which is mapped into population imbalance of an ultra-narrow optical transition. The clock stability improves with the interrogation time, but the phase can only be unambiguously mapped when it is in the range $[-\pi/2, \pi/2]$; phases outside of this range lead to phase-slip errors, which limits the attainable interrogation time at a given phase-slip error probability, a major limitation to the stability of state-of-the-art neutral atom atomic clocks. Optimal readout schemes~\cite{Rosenband2013,Borregaard2013,Li2022} could improve the attainable interrogation time exponentially but require local rotational control over sub-ensembles during the sensing protocol or local mid-circuit readout and reset, both of which have not been demonstrated to date.

Here we show local control of optical transitions in a tweezer array clock~\cite{Madjarov2019,Norcia2019,Young2020} by using rearrangement techniques~\cite{Endres2016,Barredo2016,Bluvstein2022,Dordevic2021,Lengwenus2010,Beugnon2007} on atoms in superposition states to precisely control the position-dependent phase imprinted by light-matter interaction. The scheme~\cite{Schaetz2004,Chen2022} is experimentally simple and highly robust as it relies solely on the relative stability of tweezer positions and does not involve any auxiliary addressing beams. Using this technique, we demonstrate arbitrary, parallel, single-site-resolved optical qubit rotations with high fidelity.

We utilize such rotations to double the dynamic range of optical Ramsey spectroscopy by performing simultaneous evolution on two separate atomic ensembles within one tweezer array, each of which measures a different phase quadrature~\cite{Li2022}; we achieve a metrological gain of 2.55(9) dB over standard, single ensemble sequences. Finally, we realize a proof-of-principle protocol for programming local dynamical decoupling sequences during Ramsey interrogation such that different ensembles within a single atom array have different sensitivities to phase variation, and discuss its implementation as part of a general protocol for improving clock stability~\cite{Rosenband2013,Borregaard2013}.

Aside from clocks, our technique for implementing local, parallel rotations about arbitrary axes might also find use in neutral atom quantum computing platforms utilizing optical transitions~\cite{Chen2022,Wu2022}, where local coherent control of optical qubits has not been demonstrated before. More generally, our results point to a future of fully programmable neutral atom optical clocks that incorporate features of quantum computers.


\begin{figure*}[ht!]
	\centering
	\includegraphics[width=\textwidth]{MT/fig1.png}
	\caption{\textbf{Single-site addressing with movement-induced phase shifts.} \textbf{(a)} We consider two atoms individually trapped in optical tweezers, both initially in the electronic ground state. Traveling light emitted from a global laser beam applies a $\pi$/2 rotation to both atoms, and is disabled, but remains phase coherent with the atomic transition.  One of the atoms is then moved by half the laser wavelength, $\lambda$, from its initial position, rotating the effective local laser frame by an angle $\phi=\pi$. When the laser drive is restarted to apply another $\pi$/2 pulse, the moved atom now rotates back to the ground state, while the static atom rotates to the excited state. \textbf{(b)} Control over the atom displacement, $\Delta x$, is equivalent to arbitrary local rotations of the laser drive by $\phi=k\Delta x$ about the $\hat{Z}$-axis. \textbf{(c)} We implement this protocol with an array of $^{88}$Sr atoms utilizing the ultra-narrow $^1S_0{\leftrightarrow}^3P_0$ transition with $\lambda=698.4$ nm for global driving. \textbf{(d)} Top: with an array of 39 tweezers in one dimension, we apply the protocol in \textbf{b}, shifting every odd site (purple markers) in the array while leaving all even sites static (blue markers) during the dynamics. Bottom: a sinusoidal oscillation emerges in the excited state population of the shifted sites, with a period of 699(1) nm. \textbf{(e)} Focusing on the region around $\Delta x=\lambda$ (grey shaded region in \textbf{d}), we find the shifted atom shows no measurable loss in fidelity compared to the unshifted atoms. Correcting for the bare fidelity for performing a global $\hat{X}(\pi)$ rotation (red dashed line, 0.9956(1)), we find the shift operation is performed with a fidelity of 0.9984(5). The ratio of the shifted to unshifted fidelities is 0.9998(5), suggesting that the dominant source of error comes from global laser phase noise during the finite wait time required to perform the shift, rather than the movement itself. From data in \textbf{d}, we find the crosstalk to the static atoms is $0.1(2)\%$, consistent with 0. \textbf{(f)} The shift to apply a $\hat{Z}(2\pi)$ rotation can be performed without noticeable loss of fidelity down to shift times of ${\sim}20$ $\mu$s; data in $\textbf{e}$ are taken with a shift time of 32 $\mu$s, in addition to an extra wait time of 34 $\mu$s to account for finite jitter in the control timings.
 }  
	\vspace{-0.5cm}
	\label{Fig1}
\end{figure*}

The basic principle of our scheme is illustrated in Fig.~\ref{Fig1}a. We consider two atoms both initially in the ground electronic state, $|0\rangle$, interacting with a global laser beam characterized by wavevector $k=2\pi/\lambda$, and wavelength $\lambda$, propagating along the array axis. With the globally applied laser, we create an equal superposition state of $|0\rangle$ and the excited state, $|1\rangle$; in a Bloch sphere picture, this corresponds to a $\pi/2$ rotation around the $x$-axis ($\hat{X}(\pi/2$)). The laser beam is then extinguished with an optical modulator, but remains phase coherent with the atomic transition. Using atom rearrangement techniques~\cite{Endres2016,Barredo2016,Bluvstein2022,Dordevic2021,Lengwenus2010,Beugnon2007}, one of the atoms is shifted from its original position by $\Delta x$, applying an effective phase shift of $\phi=k\Delta x$ (Methods). In Fig.~\ref{Fig1}a, we first consider the special case of $\Delta x=\lambda/2$, or equivalently a $\pi$ rotation around the $z$-axis ($\hat{Z}(\pi$)) for the shifted atom (Fig.~\ref{Fig1}a). Subsequently, we apply a second global $\hat{X}(\pi/2)$ rotation with the same laser as before; the shifted atom now rotates back to $|0\rangle$ because of the movement-induced phase shift, while the unmoved atom completes its rotation to $|1\rangle$. 

The main principle behind this scheme is a locally controlled change of the relative phase between the atomic dipole-oscillation and the phase of the laser while the atom is in a superposition state; in essence, our scheme realizes a locally controlled Ramsey sequence with global driving (Methods). Similar techniques have been used in the context of ion trap experiments with two ions~\cite{Schaetz2004}, but not in a scalable fashion, as is possible with tweezer arrays~\cite{Chen2022}.

\begin{figure*}[t!]
	\centering
	\includegraphics[width=\textwidth]{MT/fig2.png}
	\caption{\textbf{Arbitrary, parallel, local rotations.} \textbf{(a)} We implement site-resolved phase shifts, $\phi_j$, during the dark time, $t$, of standard Ramsey interrogation by inserting arbitrary and parallel shifts of various distances to the array of atoms. \textbf{(b)} Results of this operation as a function of Ramsey time ($x$-axis) for different tweezers in the array ($y$-axis). The corresponding programmed phase-shift pattern is shown on the right of each subfigure. \textbf{(c)} By applying multiple global $\hat{X}(\pi/2)$ pulses (grey blocks), in tandem with local movement shifts (same color scale as in $\textbf{a}$), arbitrary local rotations can be performed. We show a demonstration by rotating an array of six atoms, initially in the $|0\rangle=|{-}Z\rangle$ state, in parallel to the six cardinal states ($|{-}Z\rangle, |{+}Z\rangle, |{-}Y\rangle, |{+}Y\rangle, |{-}X\rangle, |{+}X\rangle$), achieving an average fidelity of 0.984(2) (blue bars), and 0.987(2) SPAM-corrected (tan bars), limited by global $\hat{X}(\pi/2)$ fidelity and decoherence during the time needed for movement (Methods). \textbf{(d)} Bloch sphere representations of the states measured with quantum state tomography in \textbf{c}.
	}
	\vspace{-0.5cm}
	\label{Fig2}
\end{figure*}


We show an experimental demonstration with our $^{88}$Sr optical tweezer array experiment~\cite{Cooper2018,Madjarov2019,Choi2023}. We employ a one-dimensional array of 39 optical tweezers generated via an acousto-optic deflector (AOD) driven by an arbitrary waveform generator (AWG). This allows for precise control over the relative tweezer positions at the nanometer level, enabling arbitrary $\hat{Z}(\phi)$ rotations (Fig.~\ref{Fig1}b). Global driving is performed on the ultra-narrow $^1S_0{\leftrightarrow}^3P_0$ optical clock transition with a transition wavelength of $\lambda=698.4$ nm (Fig.~\ref{Fig1}c).

In a first experiment, we apply an $\hat{X}(\pi/2)$ operation globally to the entire array, then shift every odd site by the same distance, $\Delta x$, apply another global $\hat{X}(\pi/2)$ rotation, and finally measure the excited state population in both the shifted and unshifted sub-arrays. The excited state population of shifted atoms, $P_s$, shows sinusoidal oscillations with a period of 699(1) nm as a function of $\Delta x$, consistent with $\phi/2\pi=\Delta x/\lambda$, where $\lambda$ is the transition wavelength. The quoted error on this measurement is purely statistical, and ignores potential systematic error arising from the independent distance calibration performed with an optical resolution test target. We note that the present measurement is likely a far more precise and accurate distance calibration tool, and could find use as an effective \textit{in-situ} laser-based ruler with applications for precision determination of distance-dependent inter-atom effects, such as Rydberg interactions~\cite{Beguin2013}.

To quantify the phase shift fidelity, we focus on a narrow region around $\Delta x= \lambda$, corresponding to a $\hat{Z}(2\pi)$ rotation (Fig.~\ref{Fig1}e). A quadratic fit to $P_s(\Delta x)$ shows a maximum value of $P_s=0.9940(5)$, consistent with the mean excited state population of unshifted atoms, $P_u=0.9942(2)$, in the same range. Correcting for the bare $\hat{X}(\pi)$ fidelity (red dashed line) of 0.9956(1) shows the shift operation is performed with a fidelity of 0.9984(5). The ratio of the shifted to unshifted fidelities is 0.9998(5), suggesting that the dominant source of error comes from global laser phase noise during the finite wait time required to perform the shift, rather than the movement itself. We study the fidelity to perform the $\hat{Z}(2\pi)$ rotation as a function of the shift time (Fig.~\ref{Fig1}f), and find that the fidelity remains constant down to shift times of $t_s=20\ \mu$s; data in Fig.~\ref{Fig1}e were taken with $t_s=32\ \mu$s, plus an additional 34 $\mu$s of wait time to account for jitter in the subsequent control timings. Importantly, for all shift distances in Fig.~\ref{Fig1}d, the excited state population of the neighboring unshifted atoms is nearly constant, showing crosstalk of only $0.1(2)\%$ (Methods).

\begin{figure*}[ht!]
	\centering
	\includegraphics[width=\textwidth]{MT/fig3.png}
	\caption{\textbf{Enhanced sensing with dual-quadrature measurement.} \textbf{(a)} For a given phase angle, $\theta$, population measurement in only a single basis, e.g. $Y$, can only be inverted within a dynamic range of $-\pi/2<\theta<\pi/2$. Phase-slips occur when the phase angle exceeds this dynamic range, meaning the phase can no longer be unambiguously determined (right). However, by measuring both quadratures, $X$ and $Y$, this dynamic range can be doubled to $-\pi<\theta<\pi$, effectively allowing for interrogating larger spreads in phase, such as when measuring for longer times. \textbf{(b)} We implement dual-quadrature readout of time-resolved Ramsey interrogation by applying local $\pi/2$ phase shifts to all odd sites in the array, with approximately 10 atoms measuring each quadrature. \textbf{(c)} With single-quadrature readout, the interrogation time is limited due to phase-slips, visible by the separation between a decay envelope reconstructed from the single-quadrature phase spread (orange dashed line) and the averaged Ramsey signal (blue and red markers and lines). The equivalent reconstruction with dual-quadrature readout (green dashed line) is an accurate estimator of the signal up to longer times. \textbf{(d)} To perform this reconstruction, we measure time-resolved probability distributions of the estimated phase relative to the mean from dual-quadrature measurement. As the standard deviation, $\sigma$, of the phase distribution grows (inset), the estimated phase begins exceeding the $-\pi/2<\theta<\pi/2$ range for normal spectroscopy (black dashed lines), but is still resolvable via dual-quadrature measurement. Note that the time-independent contribution from quantum projection noise to the standard deviation has been subtracted off in the inset (Methods).  \textbf{(e)} We estimate the phase-slip probability, $\epsilon$, for single- (orange markers) and dual-quadrature (green markers) measurements by fitting a Gaussian to the time-resolved estimated phases in \textbf{e}. The fit is folded over at the boundaries of the dynamic range to account for the behavior of phase-slips, as in \textbf{a}. This fit is then used to estimate decay envelopes for the Ramsey interrogation signal in \textbf{c}. \textbf{(f)} For a given allowable phase-slip probability, the enhanced dynamic range of the dual-quadrature readout improves the maximum possible interrogation time. For our particular phase growth profile (inset of \textbf{d}), the maximum interrogation time is increased by a factor of ${\sim}3.24$, corresponding to 2.55(9) dB of metrological gain.
	} 
	\vspace{-0.5cm}
	\label{Fig3}
\end{figure*}


Arbitrary rotation patterns can be imprinted on the array by shifting all of the atoms by varying distances such that rotations about the $z$-axis with tweezer-resolved phase, $\phi_j$, are applied (Fig.~\ref{Fig2}a). We show the results of time-resolved Ramsey spectroscopy for four different choices of single-site addressing patterns, demonstrating arbitrary, site-revolved, and parallel $\hat{Z}$ rotations (Fig.~\ref{Fig2}b). Such addressing patterns could be used to negate variations in the transition frequency across the array, for instance due to gradients in magnetic field or from the finite differences in tweezer wavelengths as generated by an AOD~\cite{Madjarov2019}. Combining these single-site $\hat{Z}(\phi_j)$ rotations with a series of global $\hat{X}(\pi/2)$ pulses allows for rotations about \textit{any axes}, not just the $z$-axis. As a demonstration (Fig.~\ref{Fig2}c,d), we choose a set of 6 contiguous atoms, initially in the ground state (denoted here as $|{-}Z\rangle$), and rotate them each in parallel into the six states $|{-}Z\rangle, |{+}Z\rangle, |{-}Y\rangle, |{+}Y\rangle, |{-}X\rangle, |{+}X\rangle$, with an average fidelity of 0.984(2) (0.987(2) SPAM-corrected), as determined by state tomography (Methods). The dominant limitations to this value are likely from global drive infidelity and dephasing during the finite shift times.

We note that while here we have demonstrated our protocol on a one-photon optical transition, it could be used to induce a similar effect for two-photon Raman transitions, for instance between hyperfine states~\cite{Levine2022}, assuming the two beams are counter-propagating. Further, the movement-induced phase-shifts employed here rely solely on a relative change in tweezer position, in contrast to alternative techniques that apply additional addressing beams~\cite{Weitenberg2011,Levine2018,Graham2022,Wang2016}, where the phase shift is proportional to a local addressing beam's intensity. While the addressing beam intensity and alignment are prone to drifts on experimental time scales, relative atom movements are ultimately derived from the radiofrequency electronic output of an AWG, which is precise, consistent, and robust. We emphasize our results did not utilize noise-compensating composite pulse sequences and that all data were taken without any system realignments or recalibrations of the atom movements. 

We now demonstrate that access to such robust, high-fidelity, single-site operations can enable enhanced sensing protocols for entanglement-free metrology. In particular, several protocols relying on local control have been proposed for improving the stability of phase-estimation~\cite{Buzek1999,Rosenband2013,Borregaard2013,Li2022} by increasing the dynamic range in which the stochastically evolving laser phase, $\theta$, can be estimated. 

Here we show one such proposal~\cite{Li2022} experimentally, by splitting the array into two sub-ensembles using local addressing to perform Ramsey interferometry simultaneously in two orthogonal bases, $X$ and $Y$, yielding populations $P^{(x)}$ and $P^{(y)}$. While readout in a single basis limits the invertible phase range to $\theta\in[-\pi/2,\pi/2]$, readout in both bases allows this range to be extended unambiguously to $[-\pi,\pi]$ (Fig.~\ref{Fig3}a). Consequently, we can afford a longer Ramsey interrogation time before $\theta$ drifts outside of the invertible range, which would cause a phase-slip error. Note that while the atom number in each quadrature has been halved, there is no added quantum projection noise~\cite{Itano1993} from the dual-quadrature measurement compared to a single-basis measurement~\cite{Li2022,Rosenband2013}.

To implement this dual-quadrature readout, we perform Ramsey inteferometry with the addition of a $\hat{Z}(\pi/2)$ rotation to all odd sites in the array before readout (Fig.~\ref{Fig3}b). The resultant oscillations in $P^{(x)}$ and $P^{(y)}$ show a $\pi/2$ phase shift between the even ($X$) and odd ($Y$) sites in the array (Fig.~\ref{Fig3}c). For every repeated measurement (indexed by $j$) at time $t$ we estimate the phase as~\cite{Rosenband2013} 
\begin{align}
\theta_j(t) = \textrm{arg}(z^{(x)}_{j}(t) + i z^{(y)}_{j}(t)),
\label{eq:phaseinversion}
\end{align}
where \mbox{$z^{(x,y)}_{j}(t) = (2P^{(x,y)}_{j}(t)-1)$} and $\textrm{arg}$ is the argument function. We then calculate the difference, $\delta_j(t)$, of $\theta_j(t)$ from its mean phase (Methods).

We plot the probability distribution $\mathcal{P}(\delta_j(t))$ in Fig.~\ref{Fig3}d, and observe a continuous growth of its standard deviation $\sigma$ (inset).  We stress that we are interested in the distribution of the laser phase itself, which determines the phase-slip error probability. Hence, we have subtracted off the contribution from quantum projection noise~\cite{Itano1993} to our experimental data shown in the inset of Fig.~\ref{Fig3}d (Methods). If this standard deviation of the laser phase itself becomes too large compared to the dynamic range, frequent phase-slip errors occur. In Fig.~\ref{Fig3}e we evaluate the phase-slip probability, $\epsilon$, that the phase has exceeded the bounds $[-\pi/2,\pi/2]$ (in emulation of a theoretical single-basis measurement, black dashed lines in Fig.~\ref{Fig3}d), or $[-\pi,\pi]$ (for the dual-quadrature readout); we find that the error probability for the single-basis case quickly becomes substantially larger at shorter interrogation times (Methods).

We further characterize the maximum interrogation time, $T_\textrm{max}(\epsilon)$, for which the phase-slip error probability is still below a threshold $\epsilon$ (Methods). We find that $T_\textrm{max}(\epsilon)$ is significantly increased for the dual-quadrature case (Fig.~\ref{Fig3}f). From the square-root of the ratio of $T_\textrm{max}(\epsilon)$ for the single- and dual-quadrature cases, we find a metrological gain in sensitivity of $2.55(9)$ dB for our particular noise profile (Fig.~\ref{Fig3}f, inset). This constitutes a demonstrable and practical improvement in phase estimation without increasing the probability for phase-slip errors, a common problem for entanglement enhanced metrology schemes~\cite{Kessler2014,Schulte2020}. 

\begin{figure}[t!]
	\centering
	\includegraphics[width=\columnwidth]{MT/fig4.png}
	\caption{\textbf{Local dynamical decoupling towards optimal metrology.} \textbf{(a)} We split the array into three ensembles, and perform a local dynamical decoupling (DD) sequence such that even though the total Ramsey dark time is $T$, individual ensembles experience different effective evolution times of $T/4, T/2,$ and $T$, respectively. The phase of each ensemble is then measured using dual-quadrature readout. \textbf{(b)} Slower evolving ensembles (those which experience less evolution time) can be used to detect phase-slips in faster evolving ensembles, extending the effective interrogation time of optical clocks. Following the sequence in \textbf{a}, we find the three ensembles evolve at relative rates of 1:1.99(1):4.10(4) with respect to the total evolution time, $T$. The demonstrated scheme in \textbf{a-b} is effective for the case of slow frequency noise where the corresponding noise correlation time is longer than the total evolution time. \textbf{(c)} To handle generic time-dependent noise with shorter correlation times, we envision breaking the total evolution time into $k$ kernels of length $\tau$, each of which is composed of local dynamical decoupling and free evolution. In this way, as long as $\tau$ is shorter than the correlation time of any time-dependent noise affecting the system, the different $M$ ensembles (indexed by $m=0,\cdots, M-1$) can accumulate phase in a correlated manner over the interleaved Ramsey interrogation periods.
 }

	\vspace{-0.5cm}
	\label{Fig4}
\end{figure}

Even greater enhancements in dynamic range, and hence clock stability, could be possible through the use of multiple ensembles explicitly programmed to have different sensitivities to the global laser phase~\cite{Rosenband2013,Borregaard2013}. In these protocols, the total number of atoms is evenly divided into $M$ ensembles, which are each further subdivided into two sub-ensembles for dual-quadrature measurement. One ensemble is used for normal phase measurement, while for the rest the free evolution time is reduced by factors of $2^{-1},\cdots,2^{1-M}$, or equivalently their effective phase accumulation is reduced by the same amount. If this procedure is performed correctly, the effective ensemble coherence times will then be extended by factors of $2,\cdots,2^{M-1}$, meaning slower evolving ensembles can be used to probe for phase-slips in the fastest ensembles. This then allows for phase estimation over a wider dynamic range beyond $[-\pi,\pi]$, and potentially allows for an improved scaling of the clock stability with atom number~\cite{Rosenband2013} at fixed phase-slip probability (Fig.~\ref{Fig4}a).

As an outlook, we demonstrate a proof-of-principle of local control techniques towards such protocols by performing local dynamical decoupling such that three ensembles experience different effective Ramsey evolution times of $T$, $T/2$, and $T/4$. This is accomplished by inserting local $\hat{X}(\pi)$ pulses (using techniques from Fig.~\ref{Fig2}c) during the evolution at time $T/4$ for the second-fastest ensemble, and time $3T/8$ for the slowest ensemble. Each ensemble is then subdivided further into two sub-ensembles for dual-quadrature readout (Fig.~\ref{Fig4}b). Resultant Ramsey oscillations versus the total evolution time, $T$, show a frequency ratio of 1:1.99(1):4.10(4), very close to the desired 1:2:4 ratio. 

Following this experimental demonstration, we now discuss two limitations (and possible solutions) of this scheme, specifically related to the frequency noise profile and the atom number per ensemble. First, for the simplest case of shot-to-shot noise of laser frequencies that are otherwise constant during the interrogation, our scheme would allow the clock stability to be improved exponentially~\cite{Rosenband2013} by a factor of $\sqrt{2^{M-1}/M}$; however, for more general time-dependent frequency noise, the situation is more complex, requiring a higher-order pulse sequence~\cite{Rosenband2013}. We propose one such pulse sequence in Fig.~\ref{Fig4}c, by breaking the total evolution time, $T$, into multiple kernels of length $\tau$. Within each kernel, each ensemble experiences a combination of local dynamical decoupling and free evolution, such that the net phase evolution time is $T, T/2,\cdots, T/2^{M-1}$. This scheme could handle noise profiles where the local phase accumulation period, $\tau$, is shorter than the correlation time of the noise.

Second, multi-ensemble estimation schemes in general require sufficient atom number per ensemble to be useful~\cite{Rosenband2013}. When the number of atoms per ensemble is limited, quantum projection noise can negate any advantage. For the present experimental demonstration with $N\approx6$ per ensemble we do not expect a metrological gain, but we note that a generalization of our addressing scheme to two dimensional tweezer clock systems~\cite{Young2020} is straightforward. For example, we imagine a realistic scenario of a $10\times20$ atom array with column-by-column control of tweezer positions, such as could be generated with crossed AODs or an AOD combined with a spatial light modulator. In this case, each pair of columns could realize one ensemble with dual-quadrature readout. 

In summary, we have demonstrated arbitrary local rotations for optical transitions through robust phase-sensitive position control in neutral atom arrays, with sub-diffraction limited precision. We have used such rotations to interrogate two atomic ensembles simultaneously for dual-quadrature readout of a Ramsey interferometry signal with demonstrable metrological gain, and have shown a proof-of-principle for controlling many ensembles with variable sensitivity during Ramsey evolution, a key ingredient of proposals for optimal clocks. Further, these methods could be naturally combined with metrologically useful entangled states~\cite{Li2022,Kessler2014,Marciniak2022}. More generally, our results are an important step towards a fully programmable quantum optical clock based on neutral atoms, which would incorporate quantum computing techniques towards metrological gains, similar to work done with ion trap devices~\cite{Marciniak2022,Schmidt2005} but likely in a more scalable fashion. Such a universal neutral atom clock system would ideally combine arbitrary local rotations, as shown here, with two-qubit entangling operations for optical transitions~\cite{Schine2022}, and mid-circuit readout and reset, which has not been demonstrated so far.

\textit{Note---}During completion of this work we became aware of related work performing local $\hat{Z}$ rotations and studying entanglement-enhanced metrology in an optical tweezer array clock experiment~\cite{Eckner2023}.

\begin{acknowledgements}
We acknowledge useful conversations with Kon Leung, Hannah Manetsch, Su Direkci, and Tuvia Gefen. Further, we thank Jacob Covey for a careful evaluation of our manuscript. This material is based upon work supported by the U.S. %Department of Energy, Office of Science, National Quantum Information Science Research Centers, Quantum Systems Accelerator. 
We acknowledge support from the Army Research Office MURI program (W911NF2010136), from the Institute for Quantum Information and Matter, an NSF Physics Frontiers Center (NSF Grant PHY-1733907), the NSF CAREER award (1753386), the AFOSR YIP (FA9550-19-1-0044), the DARPA ONISQ program (W911NF2010021), and the NSF QLCI program (2016245). ALS acknowledges support from the Eddleman Quantum Graduate Fellowship. RF acknowledges support from the Troesh postdoctoral fellowship. RBST acknowledges support from the Taiwan-Caltech Fellowship. THY acknowledges support from the IQIM Visiting Fellowship and in part by the NRF (2022M3K4A1094781).
\end{acknowledgements}


\section{Methods}
\subsection*{Light-matter interaction Hamiltonian}
After applying the rotating-wave-approximation, the light-matter interaction Hamiltonian considered in this work is given by~\cite{Steck2007}
\begin{align}
\hat{H}=\frac{\hbar |\Omega|}{2}(|0\rangle\langle 1|e^{i k x} + |1\rangle\langle 0|e^{-i k x}),
\label{eq:lightmatter}
\end{align}
where $\Omega$ is the Rabi frequency, $k$ is the global laser wavevector, $x$ is the atomic position along the beam propagation axis, $|0\rangle$ is the ground state and $|1\rangle$ is the excited state. Atom displacements by $\Delta x$ correspond to phase shifts of $\phi=k\Delta x$, as described in the main text. 

Note the choice of the phase for the initial $\hat{X}(\pi/2)$ rotation is a local gauge freedom, and thus all atoms in the array can be said to experience the rotation about the same local axis, e.g. the $x$-axis, despite the spacing between atoms generically not being perfectly commensurate with the driving wavelength. When the atom is shifted, it can be thought of as changing $\hat{X} \rightarrow \hat{X}\cos(\phi)+i \hat{Y}\sin(\phi)$. For the case of only a single global $\hat{X}$ rotation after the movement, this is equivalent to an effective $\hat{Z}(\phi)$ rotation of the quantum state; however, in general if multiple global $\hat{X}$ operations are performed, the equivalence with an effective $\hat{Z}$ rotation breaks down.

\subsection*{Data analysis}
Discrimination between $|0\rangle=|^1S_0\rangle$ and $|1\rangle=|^3P_0\rangle$ is performed by strongly driving the $^1S_0{\leftrightarrow}^1P_1$ transition for $10\ \mu$s, which heats and ejects all atoms in $^1S_0$. (For details on the strontium level structure see Refs.~\cite{Cooper2018,Stellmer2014}). Atoms in $^3P_0$ are then pumped back into $^1S_0$, and imaged with lower power on the $^1S_0{\leftrightarrow}^1P_1$ transition; imaging is performed in 120 ms. 

For all data in the main text, we interleave~\cite{Choi2023} data-taking with feedback to the global $|0\rangle{\leftrightarrow}|1\rangle$ drive frequency every ${\sim}4\ $minutes to counteract slow (${\sim}40\ $minute period) oscillations arising from environmental drifts. $1\sigma$ error bars in the main text are typically smaller than the marker size in all figures.

\subsection*{Operation fidelities}
Error modelling suggests that the global $X(\pi)$ fidelity of 0.9956(1) is primarily limited by: measurement errors (see below), finite temperature (the atom temperature is $\bar{n}{\sim}0.2$, leading to a $2\times10^{-3}$ infidelity), and frequency noise on our laser; the latter of these we believe is also the dominant limitation to our Ramsey coherence time (Fig.~\ref{Fig3}c). The Rabi frequency is ${\sim}2.5$ kHz for all measurements in this work, which allows for fast operations compared to the timescale of decay from $^3P_0$ (${\sim}$550 ms~\cite{Shaw2023}). Measurement errors are dominated by the vacuum-limited atom survival during imaging (0.9995(4)), the imaging fidelity for detecting the presence of an atom (0.9997(2)), and the likelihood of ejecting atoms in $^1S_0$ from the tweezers to perform state discrimination (0.9967(5)) before the final imaging. We note that not all of these measurement errors contribute equally, and their relative importance generically depends on the amount of excited state population in the measured state.

The wavelength of the oscillation, $\lambda_\textrm{osc}$ in Fig.~\ref{Fig1}d is found from fitting the excited state population of shifted atoms with a sinusoid of the form $A\sin(2\pi \Delta x/\lambda_\textrm{osc})+B$, from which we determine the quoted value of $\lambda_\textrm{osc}=699(1)$ nm. Per our independent calibration of distances within the array, we expect there is an additional ${\sim}5$ nm of systematic uncertainty on this measurement. Further systematic uncertainty could arise if the beam propagation is not perfectly coaxial with the array. The crosstalk fidelity of $0.1(2)\%$ is found by fitting the unshifted atom excited state populations in Fig.~\ref{Fig1}d with a sinusoid of the same period as was determined for the shifted atoms; the quoted crosstalk is then the amplitude, $A$, of this sinusoid. 

To determine the fidelity of arbitrary local rotations (Fig.~\ref{Fig2}c), we perform quantum state tomography by reading out the produced states in the $x$-, $y$- and $z$- bases by rotating the state with a global $\pi/2$ pulse of a given global laser phase as necessary. The fidelity is then estimated with $F = \langle\psi_{\textrm{target}}|\rho|\psi_{\textrm{target}}\rangle$, where $\rho$ is the experimental state determined by quantum tomography. Due to the choice of the six arbitrary rotations, the fidelity estimation coincides with the population of the excited state or the ground state along the corresponding axis. For instance, the fidelity of $|{+}Y\rangle$ is determined by the population of the excited state when the prepared state is rotated and measured in the $y$-basis. In order to access the intrinsic infidelity induced by arbitrary local rotations, we extract SPAM errors and correct them. SPAM sources are dominated by the same detection infidelities as for the global $\hat{X}(\pi/2)$ fidelity (see above), and the finite SPAM-corrected readout $\pi/2$ pulse fidelity of 0.9982(4). 


\subsection*{Fitting the phase-slip probability}
To find the phase deviation from the mean phase for a given shot, we fit the Ramsey oscillations in Fig.~\ref{Fig3}c with a decaying sinusoid, and at each time define the fitted mean populations as $\bar{P}^{(x)}(t)$ and $\bar{P}^{(y)}(t)$. These populations are inverted via Eq.~\ref{eq:phaseinversion} into the mean phase, $\bar{\theta}(t)$, and finally we calculate the phase deviation from the mean as \mbox{$\delta_j(t)=\textrm{mod}(\theta_j(t)-\bar{\theta}(t),\pi)$}. 

For the phase-slip probability (Fig.~\ref{Fig3}e), we fit the probability densities $\mathcal{P}(\delta_j(t))$ with $G(\mathcal{P})$, where $G$ is a Gaussian distribution folded into the range $[-\pi,\pi]$. This fit provides an estimation of the true standard deviation, $\sigma$,  of the $\delta_j(t)$ distribution. With this in hand, for a given half-dynamic range, $B$, we then find the phase-slip probability, $\epsilon$, as 
\begin{align}
\epsilon=2\int_B^\infty G(\mathcal{P})d\mathcal{P}=\textrm{erfc}\Big(\frac{B}{\sqrt{2}\sigma}\Big),
\label{eq:phaseslipprobability}
\end{align}
where $\textrm{erfc}$ is the complementary error function. Note that here $B=\pi/2$ corresponds to a single-basis measurement while $B=\pi$ corresponds to a dual-quadrature measurement.

In order to calculate the metrological gain as a function of time, we further fit the time-resolved profile of $\sigma(t)$, as $\sigma(t)^2=(\beta t^\alpha)^2+\sigma_\textrm{QPN}^2$, where $\sigma_\textrm{QPN}$ is a fixed parameter which depends on the number of atoms interrogated~\cite{Itano1993}. Given that the number of atoms per ensemble varies around a mean of $N{\approx}9.5$, here we take $\sigma_\textrm{QPN}=\pi\times0.0915$, the average of the values \mbox{$\sigma_\textrm{QPN}(N=9)=\pi\times0.0934$} and \mbox{$\sigma_\textrm{QPN}(N=10)=\pi\times0.0897$}. We find $\beta=\pi\times0.117(5)$ and $\alpha=0.59(2)$. The growth of $\sigma$ over time can then be used to predict the Ramsey decay envelope, $C$, for different choices of $B$, as $C=e^{-\sigma^2/2}$. In Fig.~\ref{Fig3}c, we show this envelope estimation for the cases of $B=\pi/2$ and $B=\pi$ (orange and green dashed lines, respectively).

We then analytically calculate the maximum interrogation time at a fixed phase-slip error probability, $\epsilon$, from Eq.~\ref{eq:phaseslipprobability} as
\begin{align}
T_{\textrm{max}}(\epsilon)=(B/(\sqrt{2}\beta \textrm{erfc}^{-1}(\epsilon)))^{1/\alpha}.
\end{align}
The relative gain of the dual-quadrature readout, as defined in the main text, is then simply given as $2^{1/(2\alpha)}$.

\FloatBarrier

\newpage

\setcounter{figure}{0}
\captionsetup[figure]{labelfont={bf},name={Ext. Data Fig.},labelsep=bar,justification=raggedright,font=small}
\clearpage
\FloatBarrier

\FloatBarrier

\begin{thebibliography}{10}

\bibitem{Ludlow2015}
A.~D. Ludlow, M.~M. Boyd, J.~Ye, E.~Peik, and P.~O. Schmidt,
\newblock Reviews of Modern Physics {\bf 87}, 637 (2015).

\bibitem{Safronova2018}
M.~Safronova, D.~Budker, D.~DeMille, D.~F.~J. Kimball, A.~Derevianko, and C.~W.
  Clark,
\newblock Reviews of Modern Physics {\bf 90}, 25008 (2018).

\bibitem{Andreev2018}
ACME~Collaboration,
\newblock Nature {\bf 562}, 355 (2018).

\bibitem{Roussy2022}
T.~S. Roussy, L.~Caldwell, T.~Wright, W.~B. Cairncross, Y.~Shagam, K.~B. Ng,
  N.~Schlossberger, S.~Y. Park, A.~Wang, J.~Ye, and E.~A. Cornell,
\newblock arXiv:2212.11841 (2022).

\bibitem{Degen2017}
C.~Degen, F.~Reinhard, and P.~Cappellaro,
\newblock Reviews of Modern Physics {\bf 89}, 35002 (2017).

\bibitem{Pezze2018}
L.~Pezzè, A.~Smerzi, M.~K. Oberthaler, R.~Schmied, and P.~Treutlein,
\newblock Reviews of Modern Physics {\bf 90}, 35005 (2018).

\bibitem{Macieszczak2014}
K.~Macieszczak, M.~Fraas, and R.~Demkowicz-Dobrzański,
\newblock New Journal of Physics {\bf 16}, 113002 (2014).

\bibitem{Kaubruegger2021}
R.~Kaubruegger, D.~V. Vasilyev, M.~Schulte, K.~Hammerer, and P.~Zoller,
\newblock Physical Review X {\bf 11}, 41045 (2021).

\bibitem{Huelga1997}
S.~F. Huelga, C.~Macchiavello, T.~Pellizzari, A.~K. Ekert, M.~B. Plenio, and
  J.~I. Cirac,
\newblock Phys. Rev. Lett. {\bf 79}, 3865 (1997).

\bibitem{Schulte2020}
M.~Schulte, C.~Lisdat, P.~O. Schmidt, U.~Sterr, and K.~Hammerer,
\newblock Nature Communications 2020 11:1 {\bf 11}, 1 (2020).

\bibitem{Rosenband2013}
T.~Rosenband and D.~R. Leibrandt,
\newblock arXiv:1303.6357 (2013).

\bibitem{Borregaard2013}
J.~Borregaard and A.~S. Sørensen,
\newblock Physical Review Letters {\bf 111}, 90802 (2013).

\bibitem{Li2022}
W.~Li, S.~Wu, A.~Smerzi, and L.~Pezzè,
\newblock Physical Review A {\bf 105}, 53116 (2022).

\bibitem{Madjarov2019}
I.~S. Madjarov, A.~Cooper, A.~L. Shaw, J.~P. Covey, V.~Schkolnik, T.~H. Yoon,
  J.~R. Williams, and M.~Endres,
\newblock Physical Review X {\bf 9}, 41052 (2019).

\bibitem{Norcia2019}
M.~A. Norcia, A.~W. Young, W.~J. Eckner, E.~Oelker, J.~Ye, and A.~M. Kaufman,
\newblock Science {\bf 366}, 93 (2019).

\bibitem{Young2020}
A.~W. Young, W.~J. Eckner, W.~R. Milner, D.~Kedar, M.~A. Norcia, E.~Oelker,
  N.~Schine, J.~Ye, and A.~M. Kaufman,
\newblock Nature {\bf 588}, 408 (2020).

\bibitem{Endres2016}
M.~Endres, H.~Bernien, A.~Keesling, H.~Levine, E.~R. Anschuetz, A.~Krajenbrink,
  C.~Senko, V.~Vuletic, M.~Greiner, and M.~D. Lukin,
\newblock Science {\bf 354}, 1024 (2016).

\bibitem{Barredo2016}
D.~Barredo, S.~de~Leseleuc, V.~Lienhard, T.~Lahaye, and A.~Browaeys,
\newblock Science {\bf 354}, 1021 (2016).

\bibitem{Bluvstein2022}
D.~Bluvstein, H.~Levine, G.~Semeghini, T.~T. Wang, S.~Ebadi, M.~Kalinowski,
  A.~Keesling, N.~Maskara, H.~Pichler, M.~Greiner, V.~Vuletić, and M.~D.
  Lukin,
\newblock Nature {\bf 604}, 451 (2022).

\bibitem{Dordevic2021}
T.~\DJ orđević, P.~Samutpraphoot, P.~L. Ocola, H.~Bernien, B.~Grinkemeyer,
  I.~Dimitrova, V.~Vuletić, and M.~D. Lukin,
\newblock Science {\bf 373}, 1511 (2021)

\bibitem{Lengwenus2010}
A.~Lengwenus, J.~Kruse, M.~Schlosser, S.~Tichelmann, and G.~Birkl,
\newblock Physical Review Letters {\bf 105}, 170502 (2010).

\bibitem{Beugnon2007}
J.~Beugnon, C.~Tuchendler, H.~Marion, A.~Gaëtan, Y.~Miroshnychenko, Y.~R.~P.
  Sortais, A.~M. Lance, M.~P.~A. Jones, G.~Messin, A.~Browaeys, and
  P.~Grangier,
\newblock Nature Physics {\bf 3}, 696 (2007).

\bibitem{Schaetz2004}
T.~Schaetz, M.~D. Barrett, D.~Leibfried, J.~Chiaverini, J.~Britton, W.~M.
  Itano, J.~D. Jost, C.~Langer, and D.~J. Wineland,
\newblock Phys. Rev. Lett. {\bf 93}, 40505 (2004).

\bibitem{Chen2022}
N.~Chen, L.~Li, W.~Huie, M.~Zhao, I.~Vetter, C.~H. Greene, and J.~P. Covey,
\newblock Physical Review A {\bf 105}, 52438 (2022).

\bibitem{Zheng2022}
X.~Zheng, J.~Dolde, V.~Lochab, B.~N. Merriman, H.~Li, and S.~Kolkowitz,
\newblock Nature {\bf 602}, 425 (2022).

\bibitem{Wu2022}
Y.~Wu, S.~Kolkowitz, S.~Puri, and J.~D. Thompson,
\newblock Nature Communications {\bf 13}, 4657 (2022).

\bibitem{Cooper2018}
A.~Cooper, J.~P. Covey, I.~S. Madjarov, S.~G. Porsev, M.~S. Safronova, and
  M.~Endres,
\newblock Physical Review X {\bf 8}, 41055 (2018).

\bibitem{Choi2023}
J.~Choi, A.~L. Shaw, I.~S. Madjarov, X.~Xie, R.~Finkelstein, J.~P. Covey, J.~S.
  Cotler, D.~K. Mark, H.-Y. Huang, A.~Kale, H.~Pichler, F.~G. S.~L. Brandão,
  S.~Choi, and M.~Endres,
\newblock Nature {\bf 613}, 468 (2023).

\bibitem{Beguin2013}
L.~B\'eguin, A.~Vernier, R.~Chicireanu, T.~Lahaye, A.~Browaeys,
\newblock Physical Review Letters {\bf 110} 263201 (2013)

\bibitem{Levine2022}
H.~Levine, D.~Bluvstein, A.~Keesling, T.~T. Wang, S.~Ebadi, G.~Semeghini, A.~Omran, M.~Greiner, V.~Vuletić, and M.~D. Lukin,
\newblock Physical Review A {\bf 105}, 032618 (2022)

\bibitem{Weitenberg2011}
C.~Weitenberg, M.~Endres, J.~F. Sherson, M.~Cheneau, P.~Schauß, T.~Fukuhara,
  I.~Bloch, and S.~Kuhr,
\newblock Nature {\bf 471}, 319 (2011).

\bibitem{Levine2018}
H.~Levine, A.~Keesling, A.~Omran, H.~Bernien, S.~Schwartz, A.~S. Zibrov,
  M.~Endres, M.~Greiner, V.~Vuletić, and M.~D. Lukin,
\newblock Physical Review Letters {\bf 121}, 123603 (2018).

\bibitem{Graham2022}
T.~M. Graham, Y.~Song, J.~Scott, C.~Poole, L.~Phuttitarn, K.~Jooya, P.~Eichler,
  X.~Jiang, A.~Marra, B.~Grinkemeyer, M.~Kwon, M.~Ebert, J.~Cherek, M.~T.
  Lichtman, M.~Gillette, J.~Gilbert, D.~Bowman, T.~Ballance, C.~Campbell, E.~D.
  Dahl, O.~Crawford, N.~S. Blunt, B.~Rogers, T.~Noel, and M.~Saffman,
\newblock Nature {\bf 604}, 457 (2022).

\bibitem{Wang2016}
Y.~Wang, A.~Kumar, T.-Y. Wu, and D.~S. Weiss,
\newblock Science {\bf 352}, 1562 (2016).

\bibitem{Buzek1999}
V.~Bužek, R.~Derka, and S.~Massar,
\newblock Physical Review Letters {\bf 82}, 2207 (1999).

\bibitem{Itano1993}
W.~M. Itano, J.~C. Bergquist, J.~J. Bollinger, J.~M. Gilligan, D.~J. Heinzen,
  F.~L. Moore, M.~G. Raizen, and D.~J. Wineland,
\newblock Physical Review A {\bf 47}, 3554 (1993).

\bibitem{Kessler2014}
E.~M. Kessler, I.~Lovchinsky, A.~O. Sushkov, and M.~D. Lukin,
\newblock Physical Review Letters {\bf 112}, 150802 (2014).

\bibitem{Marciniak2022}
C.~D. Marciniak, T.~Feldker, I.~Pogorelov, R.~Kaubruegger, D.~V. Vasilyev,
  R.~van Bijnen, P.~Schindler, P.~Zoller, R.~Blatt, and T.~Monz,
\newblock Nature {\bf 603}, 604 (2022).

\bibitem{Schmidt2005}
P.~O. Schmidt, T.~Rosenband, C.~Langer, W.~M. Itano, J.~C. Bergquist, and D.~J. Wineland,
\newblock Science {\bf 309}, 5735 (2005).

\bibitem{Schine2022}
N.~Schine, A.~W. Young, W.~J. Eckner, M.~J. Martin, and A.~M. Kaufman,
\newblock Nature Physics {\bf 18}, 1067 (2022).

\bibitem{Eckner2023}
W.~J. Eckner, N.~D. Oppong, A.~Cao, A.~W. Young, W.~R. Milner, J.~M. Robinson,
  J.~Ye, and A.~M. Kaufman,
\newblock arXiv:2303.08078 (2023).

\bibitem{Steck2007}
Steck, D.A.
\newblock {\em Quantum and Atom Optics} (2007)

\bibitem{Stellmer2014}
S.~Stellmer, F.~Schreck, and T.~C. Killian,
\newblock Annual Review of Cold Atoms and Molecules {\bf 1}, 1 (2014).

\bibitem{Shaw2023}
A.~L. Shaw, P.~Scholl, R.~Finklestein, I.~S. Madjarov, B.~Grinkemeyer, and
  M.~Endres,
\newblock arXiv:2302.10855 (2023).

\end{thebibliography}


\end{document}


















