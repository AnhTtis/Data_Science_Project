\documentclass[prx,letterpaper,nobalancelastpage,twocolumn,superscriptaddress,nofootinbib,longbibliography]{revtex4-2}

\usepackage{graphicx}
\usepackage{amsmath}
\usepackage{bbold}
\usepackage{amssymb}
\usepackage[english]{babel}
\usepackage{color}
\usepackage[version=4]{mhchem}
\usepackage[hidelinks]{hyperref}
\usepackage{dsfont}
\usepackage{placeins}
\usepackage{mathtools}
\usepackage[normalem]{ulem}
\usepackage{lipsum}

\setcitestyle{super}

\usepackage{soul}
\setlength{\parindent}{8pt}
\setlength{\parskip}{0pt}
 
\frenchspacing

\newcommand{\HRule}{\rule{\linewidth}{0.5mm}}
\newcommand{\of}[1]{\left( #1 \right)}
\newcommand{\sqof}[1]{\left[ #1 \right]}
\newcommand{\abs}[1]{\left| #1 \right|}
\newcommand{\avg}[1]{\left< #1 \right>}

\newcommand{\cuof}[1]{\left \{ #1 \right \} }
\newcommand{\bx}{\mathbf{x}}
\newcommand{\by}{\mathbf{y}}
\newcommand{\bk}{\mathbf{k}}
\newcommand{\bp}{\mathbf{p}}
\newcommand{\bl}{\mathbf{l}}
\newcommand{\bq}{\mathbf{q}}
\newcommand{\br}{\mathbf{r}}
\newcommand{\mc}[1]{\mathcal{#1}}
\newcommand{\Uhat}{\widehat{U}}
\newcommand{\up}{\uparrow}
\newcommand{\down}{\downarrow}
\newcommand{\Rb}{$^{87}$Rb }
\newcommand{\ket}[1]{| #1 \rangle}
\newcommand{\bra}[1]{\langle #1 |}
\newcommand{\braket}[1]{\langle #1 | #1 \rangle}
\newcommand{\ketbra}[1]{|  #1 \rangle \langle #1 |}
\newcommand{\s}[1]{\substack{#1}}
\newcommand{\pro}[2]{\langle #1| #2 \rangle}
\newcommand{\defs}{:=}
\newcommand{\tr}[1]{\mathrm{tr}\left\{#1\right\}}
\newcommand{\lr}[1]{\left( #1 \right)}
\newcommand{\za}{Z_{A}}
\newcommand{\zb}{Z_{B}}
\newcommand{\pza}{p(\za)}
\newcommand{\pzazb}{p(\za|\zb)}

\newcommand{\Caltech}{California Institute of Technology, Pasadena, CA 91125, USA}
\newcommand{\MIT}{Center for Theoretical Physics, Massachusetts Institute of Technology, Cambridge, MA 02139, USA}
\newcommand{\Stanford}{Department of Electrical Engineering, Stanford University, Stanford, CA, USA}
\newcommand{\redst}[1]{\red{\st{#1}}}
\newcommand{\blue}[1]{{\color[rgb]{0,0,0.6}{#1}}}
\newcommand{\red}[1]{{\color[rgb]{0.8,0,0}{#1}}}
\newcommand{\green}[1]{{\color[rgb]{0,0.8,0}{#1}}}
\newcommand{\ALS}[1]{{\color[rgb]{0.5,0,0}{ALS: #1}}}
\newcommand{\ME}[1]{\blue{#1}}

\usepackage{caption}

\DeclareCaptionLabelSeparator{bar}{ \textbf{\textbar}~}
\captionsetup[figure]{labelfont={bf},name={Fig.},labelsep=bar,justification=raggedright,font=small}
\captionsetup[table]{labelfont={bf},name={Table},labelsep=bar,justification=raggedright,font=small}

\usepackage{enumitem}
\setlist{nolistsep}

\begin{document}

\title{Multi-ensemble metrology by programming local rotations with atom movements}
% \author{Us}\thanks{These authors contributed equally to this work}
% \affiliation{\Caltech}
% \affiliation{\MIT}
\author{Adam L. Shaw}\thanks{These authors contributed equally to this work}
\author{Ran Finkelstein}\thanks{These authors contributed equally to this work}
\author{Richard Bing-Shiun Tsai}
\author{Pascal Scholl}
\affiliation{\Caltech}
\author{\\Tai Hyun Yoon}\thanks{Permanent address: Department of Physics, Korea University, Seoul 02841, Republic of Korea}
\affiliation{\Caltech}
\author{Joonhee Choi}
\affiliation{\Caltech}
\affiliation{\Stanford}
\author{Manuel Endres}\email{mendres@caltech.edu}
\affiliation{\Caltech}

\maketitle
\textbf{Current optical atomic clocks do not utilize their resources optimally. In particular, an exponential gain in sensitivity could be achieved if multiple atomic ensembles were to be controlled or read-out individually, even without entanglement. However, controlling optical transitions locally remains an outstanding challenge for neutral atom based clocks and quantum computing platforms. Here we show arbitrary, single-site addressing for an optical transition via sub-wavelength controlled moves of tweezer-trapped atoms, which we perform with $99.84(5)\%$ fidelity and with $0.1(2)\%$ crosstalk to non-addressed atoms. The scheme is highly robust as it relies only on relative position changes of tweezers and requires no additional addressing beams. Using this technique, we implement single-shot, dual-quadrature readout of Ramsey interferometry using two atomic ensembles simultaneously, and show an enhancement of the usable interrogation time at a given phase-slip error probability. Finally, we program a sequence which performs local dynamical decoupling during Ramsey evolution to evolve three ensembles with variable phase sensitivities, a key ingredient of optimal clock interrogation. Our results demonstrate the potential of fully programmable quantum optical clocks even without entanglement and could be combined with metrologically useful entangled states in the future.}

Sensors based on quantum probes provide some of the most precise measurements in science~\cite{Ludlow2015,Safronova2018,Andreev2018,Roussy2022,Degen2017}. For many such systems, fundamental sensitivity limits can be improved through entanglement~\cite{Pezze2018,Pedrozo2020,Macieszczak2014,Kaubruegger2021}, but in the presence of noise, a practical advantage of such schemes is not guaranteed~\cite{Huelga1997,Schulte2020}. A complementary approach studies optimal metrology with entanglement-free quantum control and readout methods. For both approaches, an important figure of merit is not just the sensitivity to a given observable, but also the dynamic range over which that observable can unambiguously be estimated~\cite{Rosenband2013,Borregaard2013,Colombo2022a,Demkowicz2012}.

In the particular case of optical atomic clocks~\cite{Ludlow2015}, the observable of interest is the stochastically evolving phase of a laser acting as a local oscillator, which is mapped into population imbalance of an ultra-narrow optical transition. The clock stability improves with the interrogation time, but the phase can only be unambiguously mapped when it is in the range $[-\pi/2, \pi/2]$; phases outside of this range lead to phase-slip errors, which limits the attainable interrogation time at a given phase-slip error probability in the case of local oscillator limited clocks. Optimal readout schemes~\cite{Rosenband2013,Borregaard2013,Li2022} could improve the attainable interrogation time exponentially but require local rotational control over sub-ensembles during the sensing protocol or local mid-circuit readout and reset, both of which have not been demonstrated to date.

\begin{figure*}[ht!]
	\centering
	\includegraphics[width=\textwidth]{MT/fig1.png}
	\caption{\textbf{Single-site addressing with movement-induced phase shifts.} \textbf{(a)} We consider two atoms individually trapped in optical tweezers, both initially in the electronic ground state. Traveling light emitted from a global laser beam applies a $\pi$/2 rotation to both atoms, and is disabled, but remains phase coherent with the atomic transition.  One of the atoms is then moved by half the laser wavelength, $\lambda$, from its initial position, rotating the effective local laser frame by an angle $\phi=\pi$. When the laser drive is restarted to apply another $\pi$/2 pulse, the moved atom now rotates back to the ground state, while the static atom rotates to the excited state. \textbf{(b)} Control over the atom displacement, $\Delta x$, is equivalent to arbitrary local rotations of the laser drive by $\phi=k\Delta x$ about the $\hat{Z}$-axis. \textbf{(c)} We implement this protocol with an array of $^{88}$Sr atoms utilizing the ultra-narrow $^1S_0{\leftrightarrow}^3P_0$ transition with $\lambda=698.4$ nm for global driving. \textbf{(d)} Top: with an array of 39 tweezers in one dimension, we apply the protocol in \textbf{b}, shifting every odd site (purple markers) in the array while leaving all even sites static (blue markers) during the dynamics. Bottom: a sinusoidal oscillation emerges in the excited state population of the shifted sites, with a period of 699(1) nm. \textbf{(e)} Focusing on the region around $\Delta x=\lambda$ (grey shaded region in \textbf{d}), we find the shifted atom shows no measurable loss in fidelity compared to the unshifted atoms. Correcting for the bare fidelity for performing a global $\hat{X}(\pi)$ rotation (red dashed line, 0.9956(1)), we find the shift operation is performed with a fidelity of 0.9984(5). The ratio of the shifted to unshifted fidelities is 0.9998(5), suggesting that the dominant source of error comes from global laser phase noise during the finite wait time required to perform the shift, rather than the movement itself. From data in \textbf{d}, we find the crosstalk to the static atoms is $0.1(2)\%$, consistent with 0. \textbf{(f)} The shift to apply a $\hat{Z}(2\pi)$ rotation can be performed without noticeable loss of fidelity down to shift times of ${\sim}20$ $\mu$s; data in $\textbf{e}$ are taken with a shift time of 32 $\mu$s, in addition to an extra wait time of 34 $\mu$s to account for finite jitter in the control timings.
 }  
	\vspace{-0.5cm}
	\label{Fig1}
\end{figure*}

Here we show local control of optical transitions in a tweezer array clock~\cite{Madjarov2019,Norcia2019a,Young2020} by using rearrangement techniques~\cite{Endres2016,Barredo2016,Bluvstein2022,Dordevic2021,Lengwenus2010,Beugnon2007} on atoms in superposition states to precisely control the position-dependent phase imprinted by light-matter interaction. The scheme~\cite{Schaetz2004,Chen2022} is experimentally simple and highly robust as it relies solely on the relative stability of tweezer positions and does not involve any auxiliary addressing beams. Using this technique, we demonstrate arbitrary, parallel, single-site-resolved optical qubit rotations with high fidelity.

We utilize such rotations to double the dynamic range of optical Ramsey spectroscopy by performing simultaneous evolution on two separate atomic ensembles within one tweezer array, each of which measures a different phase quadrature~\cite{Li2022}; we extend the coherent interrogation time by a factor of 3.43(13) relative to the standard, single ensemble sequence. Finally, we realize a proof-of-principle protocol for programming local dynamical decoupling sequences during Ramsey interrogation such that different ensembles within a single atom array have different sensitivities to phase variation, and discuss its implementation as part of a general protocol for improving clock stability~\cite{Rosenband2013,Borregaard2013}.

Aside from clocks, our technique for implementing local, parallel rotations about arbitrary axes might also find use in neutral atom quantum computing platforms utilizing optical transitions~\cite{Chen2022,Wu2022}, where local coherent control of optical qubits has not been demonstrated before. More generally, our results point to a future of fully programmable neutral atom optical clocks that incorporate features of quantum computers.

The basic principle of our scheme is illustrated in Fig.~\ref{Fig1}a. We consider two atoms both initially in the ground electronic state, $|0\rangle$, interacting with a global laser beam characterized by wavevector $k=2\pi/\lambda$, and wavelength $\lambda$, propagating along the array axis. With the globally applied laser, we create an equal superposition state of $|0\rangle$ and the excited state, $|1\rangle$; in a Bloch sphere picture, this corresponds to a $\pi/2$ rotation around the $x$-axis ($\hat{X}(\pi/2$)). The laser beam is then extinguished with an optical modulator, but remains phase coherent with the atomic transition. Using atom rearrangement techniques~\cite{Endres2016,Barredo2016,Bluvstein2022,Dordevic2021,Lengwenus2010,Beugnon2007}, one of the atoms is shifted from its original position by $\Delta x$, applying an effective phase shift of $\phi=k\Delta x$ (Methods). In Fig.~\ref{Fig1}a, we first consider the special case of $\Delta x=\lambda/2$, or equivalently a $\pi$ rotation around the $z$-axis ($\hat{Z}(\pi$)) for the shifted atom (Fig.~\ref{Fig1}a). Subsequently, we apply a second global $\hat{X}(\pi/2)$ rotation with the same laser as before; the shifted atom now rotates back to $|0\rangle$ because of the movement-induced phase shift, while the unmoved atom completes its rotation to $|1\rangle$. 

The main principle behind this scheme is a locally controlled change of the relative phase between the atomic dipole-oscillation and the phase of the laser while the atom is in a superposition state; in essence, our scheme realizes a locally controlled Ramsey sequence with global driving (Methods). Similar techniques have been used in the context of ion trap experiments with two ions~\cite{Schaetz2004}, but not in a scalable fashion, as is possible with tweezer arrays~\cite{Chen2022}.

\begin{figure*}[t!]
	\centering
	\includegraphics[width=\textwidth]{MT/fig2.png}
	\caption{\textbf{Arbitrary, parallel, local rotations.} \textbf{(a)} We implement site-resolved phase shifts, $\phi_j$, during the dark time, $t$, of standard Ramsey interrogation by inserting arbitrary and parallel shifts of various distances to the array of atoms. \textbf{(b)} Results of this operation as a function of Ramsey time ($x$-axis) for different tweezers in the array ($y$-axis). The corresponding programmed phase-shift pattern is shown on the right of each subfigure. \textbf{(c)} By applying multiple global $\hat{X}(\pi/2)$ pulses (grey blocks), in tandem with local movement shifts (same color scale as in $\textbf{a}$), arbitrary local rotations can be performed. We show a demonstration by rotating an array of six atoms, initially in the $|0\rangle=|{-}Z\rangle$ state, in parallel to the six cardinal states ($|{-}Z\rangle, |{+}Z\rangle, |{-}Y\rangle, |{+}Y\rangle, |{-}X\rangle, |{+}X\rangle$), achieving an average fidelity of 0.984(2) (blue bars), and 0.987(2) after state preparation and measurement (SPAM) correction (tan bars), limited by global $\hat{X}(\pi/2)$ fidelity and decoherence during the time needed for movement (Methods). \textbf{(d)} Bloch sphere visualizations of the states measured with quantum state tomography in \textbf{c}.
	}
	\vspace{-0.5cm}
	\label{Fig2}
\end{figure*}

We show an experimental demonstration with our $^{88}$Sr optical tweezer array experiment~\cite{Cooper2018,Madjarov2019,Choi2023}. We employ a one-dimensional array of 39 optical tweezers generated via an acousto-optic deflector (AOD) driven by an arbitrary waveform generator (AWG). This allows for precise control over the relative tweezer positions at the nanometer level, enabling arbitrary $\hat{Z}(\phi)$ rotations (Fig.~\ref{Fig1}b). Global driving is performed on the ultra-narrow $^1S_0{\leftrightarrow}^3P_0$ optical clock transition with a transition wavelength of $\lambda=698.4$ nm (Fig.~\ref{Fig1}c).

In a first experiment, we apply an $\hat{X}(\pi/2)$ operation globally to the entire array, then shift every odd site by the same distance, $\Delta x$, apply another global $\hat{X}(\pi/2)$ rotation, and finally measure the excited state population in both the shifted and unshifted sub-arrays. The excited state population of shifted atoms, $P_s$, shows sinusoidal oscillations with a period of 699(1) nm as a function of $\Delta x$, consistent with $\phi/2\pi=\Delta x/\lambda$, where $\lambda$ is the transition wavelength. The quoted error on this measurement is purely statistical, and ignores potential systematic error arising from the independent distance calibration performed with an optical resolution test target. We note that the present measurement is likely a far more precise and accurate distance calibration tool, and could find use as an effective \textit{in-situ} laser-based ruler with applications for precision determination of distance-dependent inter-atom effects, such as Rydberg interactions~\cite{Beguin2013}.

To quantify the phase shift fidelity, we focus on a narrow region around $\Delta x= \lambda$, corresponding to a $\hat{Z}(2\pi)$ rotation (Fig.~\ref{Fig1}e). A quadratic fit to $P_s(\Delta x)$ shows a maximum value of $P_s=0.9940(5)$ (not corrected for state preparation and measurement (SPAM) errors), consistent with the mean excited state population of unshifted atoms, $P_u=0.9942(2)$, in the same range. Correcting for the bare $\hat{X}(\pi)$ fidelity (red dashed line) of 0.9956(1)  shows the shift operation is performed with a fidelity of 0.9984(5). We note that applying SPAM correction on the bare fidelities maintains the shift fidelity largely unchanged as it is calculated from the ratio of the two.  The ratio of the shifted to unshifted fidelities is 0.9998(5), suggesting that the dominant source of error comes from global laser phase noise during the finite wait time required to perform the shift, rather than the movement itself. We study the fidelity to perform the $\hat{Z}(2\pi)$ rotation as a function of the shift time (Fig.~\ref{Fig1}f), and find that the fidelity remains constant down to shift times of $t_s=20\ \mu$s; data in Fig.~\ref{Fig1}e were taken with $t_s=32\ \mu$s, plus an additional 34 $\mu$s of wait time to account for jitter in the subsequent control timings. Importantly, for all shift distances in Fig.~\ref{Fig1}d, the excited state population of the neighboring unshifted atoms is nearly constant, showing crosstalk of only $0.1(2)\%$ (Methods).

\begin{figure*}[ht!]
	\centering
	\includegraphics[width=\textwidth]{MT/fig3.png}
	\caption{\textbf{Enhanced sensing with dual-quadrature measurement.} \textbf{(a)} For a given phase angle, $\theta$, population measurement in only a single basis, e.g. $Y$, can only be inverted within a dynamic range of $-\pi/2<\theta<\pi/2$.  By measuring both quadratures, $X$ and $Y$, this dynamic range can be doubled to $-\pi<\theta<\pi$, allowing for interrogating larger spreads in phase, such as when measuring for longer times. \textbf{(b)} We implement dual-quadrature readout of Ramsey interrogation by applying local $\pi/2$ phase shifts to all odd sites in the array. \textbf{(c)} With single-quadrature readout, the interrogation time is limited due to phase-slips, visible by the separation between a decay envelope reconstructed from the single-quadrature phase spread (orange dashed line) and the averaged Ramsey signal (blue and red markers and lines). The equivalent reconstruction with dual-quadrature readout (green dashed line) is accurate up to longer times. \textbf{(d)} To perform this reconstruction, we measure time-resolved probability distributions of the estimated phase relative to the mean from dual-quadrature measurement. As the standard deviation, $\sigma$, of the phase distribution grows (inset), the estimated phase begins exceeding the $-\pi/2<\theta<\pi/2$ range for normal spectroscopy (black dashed lines), but is still resolvable via dual-quadrature measurement. Note that the time-dependent contribution from quantum projection noise to the standard deviation has been subtracted off in the inset (Methods).  \textbf{(e)} We estimate the phase-slip probability, $\epsilon$, for single- (orange circles) and dual-quadrature (green circles) measurements by fitting a folded Gaussian to the time-resolved estimated phases in \textbf{d}. The fit is folded over at the boundaries of the dynamic range to account for the behavior of phase-slips, as in \textbf{a}. For the single-quadrature case, we also estimate the probability directly from the underlying data (squares), which is in good agreement with the estimate from fit. Solid lines are the predicted phase slip probabilities from the fit in the inset of \textbf{d}. This fit is used to estimate decay envelopes in \textbf{c}. \textbf{(f)} For a given allowable phase-slip probability, the enhanced dynamic range of the dual-quadrature readout improves the maximum possible interrogation time. For our particular phase growth profile (inset of \textbf{d}), the improvement is a factor of ${\sim}3.43$.
	} 
	\vspace{-0.5cm}
	\label{Fig3}
\end{figure*}

Arbitrary rotation patterns can be imprinted on the array by shifting all of the atoms by varying distances such that rotations about the $z$-axis with tweezer-resolved phase, $\phi_j$, are applied (Fig.~\ref{Fig2}a). We show the results of time-resolved Ramsey spectroscopy for four different choices of single-site addressing patterns, demonstrating arbitrary, site-revolved, and parallel $\hat{Z}$ rotations (Fig.~\ref{Fig2}b). Such addressing patterns could be used to negate variations in the transition frequency across the array, for instance due to gradients in magnetic field or from the finite differences in tweezer wavelengths as generated by an AOD~\cite{Madjarov2019}. Combining these single-site $\hat{Z}(\phi_j)$ rotations with a series of global $\hat{X}(\pi/2)$ pulses allows for rotations about \textit{any axes}, not just the $z$-axis. As a demonstration (Fig.~\ref{Fig2}c,d), we choose a set of 6 contiguous atoms, initially in the ground state (denoted here as $|{-}Z\rangle$), and rotate them each in parallel into the six states $|{-}Z\rangle, |{+}Z\rangle, |{-}Y\rangle, |{+}Y\rangle, |{-}X\rangle, |{+}X\rangle$, with an average fidelity of 0.984(2) (0.987(2) SPAM-corrected), as determined by state tomography (Methods). The dominant limitations to this value are likely from global drive infidelity and dephasing during the finite shift times.

We note that while here we have demonstrated our protocol on a one-photon optical transition, it could be used to induce a similar effect for two-photon Raman transitions, for instance between hyperfine states~\cite{Levine2022}, assuming the two beams are counter-propagating. Further, the movement-induced phase-shifts employed here rely solely on a relative change in tweezer position, in contrast to alternative techniques that apply additional addressing beams~\cite{Weitenberg2011,Levine2018,Graham2022,Wang2016}, where the phase shift is proportional to a local addressing beam's intensity. While the addressing beam intensity and alignment are prone to drifts on experimental time scales, relative atom movements are ultimately derived from the radiofrequency electronic output of an AWG, which is precise, consistent, and robust. We emphasize our results did not utilize noise-compensating composite pulse sequences and that all data were taken without any system realignments or recalibrations of the atom movements. 

We now demonstrate that access to such robust, high-fidelity, single-site operations can enable enhanced sensing protocols for entanglement-free metrology. In particular, several protocols relying on local control have been proposed for improving the stability of phase-estimation~\cite{Buzek1999,Rosenband2013,Borregaard2013,Li2022} by increasing the dynamic range in which the stochastically evolving laser phase, $\theta$, can be estimated. 

Here we show one such proposal~\cite{Li2022} experimentally, by splitting the array into two sub-ensembles using local addressing to perform Ramsey interferometry simultaneously in two orthogonal bases, $X$ and $Y$, yielding populations $P^{(x)}$ and $P^{(y)}$. While readout in a single basis limits the invertible phase range to $\theta\in[-\pi/2,\pi/2]$, readout in both bases allows this range to be extended unambiguously to $[-\pi,\pi]$ (Fig.~\ref{Fig3}a). Consequently, we can afford a longer Ramsey interrogation time before $\theta$ drifts outside of the invertible range, which would cause a phase-slip error. Note that while the atom number in each quadrature has been halved, this typically does not increase the quantum projection noise (QPN)~\cite{Itano1993} from the dual-quadrature measurement compared to a single-basis measurement (Methods)~\cite{Li2022,Rosenband2013}.

To implement this dual-quadrature readout, we perform Ramsey inteferometry with the addition of a $\hat{Z}(\pi/2)$ rotation to all odd sites in the array before readout (Fig.~\ref{Fig3}b). The resultant oscillations in $P^{(x)}$ and $P^{(y)}$ show a $\pi/2$ phase shift between the even ($X$) and odd ($Y$) sites in the array (Fig.~\ref{Fig3}c). For every repeated measurement (indexed by $j$) at time $t$ we estimate the phase as~\cite{Rosenband2013} 
\begin{align}
\theta_j(t) = \textrm{arg}(z^{(x)}_{j}(t) + i z^{(y)}_{j}(t)),
\label{eq:phaseinversion}
\end{align} 
where \mbox{$z^{(x,y)}_{j}(t) = (2P^{(x,y)}_{j}(t)-1)$} and $\textrm{arg}$ is the argument function. We then calculate the difference, $\delta_j(t)$, of $\theta_j(t)$ from its mean phase (Methods).

We plot the probability distribution $\mathcal{P}(\delta_j(t))$ in Fig.~\ref{Fig3}d, and observe a continuous growth of its standard deviation (STD) $\sigma$ (inset).  We stress that we are interested in the distribution of the laser phase itself, which determines the phase-slip error probability. Hence, we have subtracted off the contribution from QPN to our experimental data shown in the inset of Fig.~\ref{Fig3}d (Methods). We find that the laser phase STD grows with time as a power law $\sigma=\beta t^\alpha$, with $\alpha=0.56(2)$ which we attribute to a power spectral density composed of $1/f$ and white frequency noise. If this standard deviation of the laser phase itself becomes too large compared to the dynamic range, frequent phase-slip errors occur. In Fig.~\ref{Fig3}e we evaluate the phase-slip probability, $\epsilon$, that the phase has exceeded the bounds $[-\pi/2,\pi/2]$ (in emulation of a theoretical single-basis measurement, black dashed lines in Fig.~\ref{Fig3}d), or $[-\pi,\pi]$ (for the dual-quadrature readout); we find that the error probability for the single-basis case quickly becomes substantially larger at shorter interrogation times (Methods).

We further characterize the maximum interrogation time, $T_\textrm{max}(\epsilon)$, for which the phase-slip error probability is still below a threshold $\epsilon$ (Methods). We find that $T_\textrm{max}(\epsilon)$ is significantly increased for the dual-quadrature case (Fig.~\ref{Fig3}f) by a factor of $3.43(13)$, the exact numerical value of which is determined by the phase STD growth rate observed experimentally and is related to the laser noise spectrum (Methods). Such elongation in the attainable interrogation time can be translated directly to enhanced stability in a metrological setting. For example, in a zero dead-time optical clock the stability is proportional to the square-root of $T_\textrm{max}(\epsilon)$, such that we can project an increase in stability by a factor of $\sqrt{3.43}\sim1.8$ for our particular noise profile. This would constitute a practical improvement in phase estimation without increasing the probability of phase-slip errors, a common problem for entanglement enhanced metrology schemes~\cite{Kessler2014,Schulte2020}.

\begin{figure}[t!]
	\centering
	\includegraphics[width=\columnwidth]{MT/fig4.png}
	\caption{\textbf{Local dynamical decoupling towards optimal metrology.} \textbf{(a)} We split the array into three ensembles, and perform a local dynamical decoupling (DD) sequence such that even though the total Ramsey dark time is $T$, individual ensembles experience different effective evolution times of $T/4, T/2,$ and $T$, respectively. The phase of each ensemble is then measured using dual-quadrature readout. \textbf{(b)} Slower evolving ensembles (those which experience less evolution time) can be used to detect phase-slips in faster evolving ensembles, extending the effective interrogation time of optical clocks. Following the sequence in \textbf{a}, we find the three ensembles evolve at relative rates of 1:1.99(1):4.10(4) with respect to the total evolution time, $T$. The demonstrated scheme in \textbf{a-b} is effective for the case of slow frequency noise where the corresponding noise correlation time is longer than the total evolution time. \textbf{(c)} To handle generic time-dependent noise with shorter correlation times, we envision breaking the total evolution time into $k$ kernels of length $\tau$, each of which is composed of local dynamical decoupling and free evolution. In this way, as long as $\tau$ is shorter than the correlation time of any time-dependent noise affecting the system, the different $M$ ensembles (indexed by $m=0,\cdots, M-1$) can accumulate phase in a correlated manner over the interleaved Ramsey interrogation periods.
 }

	\vspace{-0.5cm}
	\label{Fig4}
\end{figure}


Even greater enhancements in dynamic range, and hence clock stability, could be possible through the use of multiple ensembles with different interrogation times by utilizing fast QND measurements \cite{Bowden2020,Kohlhaas2015} or by
explicitly programming ensembles with different sensitivities to the global laser phase~\cite{Rosenband2013,Borregaard2013}. In the latter of these protocols, the total number of atoms is evenly divided into $M$ ensembles, which are each further subdivided into two sub-ensembles for dual-quadrature measurement. One ensemble is used for normal phase measurement, while for the rest the free evolution time is reduced by factors of $2^{-1},\cdots,2^{1-M}$, or equivalently their effective phase accumulation is reduced by the same amount. If this procedure is performed correctly, the effective ensemble coherence times will then be extended by factors of $2,\cdots,2^{M-1}$, meaning slower evolving ensembles can be used to probe for phase-slips in the fastest ensembles. This then allows for phase estimation over a wider dynamic range beyond $[-\pi,\pi]$, and potentially allows for an improved scaling of the clock stability with atom number~\cite{Rosenband2013} at fixed phase-slip probability (Fig.~\ref{Fig4}a).

As an outlook, we demonstrate a proof-of-principle of local control techniques towards such protocols by performing local dynamical decoupling such that three ensembles experience different effective Ramsey evolution times of $T$, $T/2$, and $T/4$. This is accomplished by inserting local $\hat{X}(\pi)$ pulses (using techniques from Fig.~\ref{Fig2}c) during the evolution at time $T/4$ for the second-fastest ensemble, and time $3T/8$ for the slowest ensemble. Each ensemble is then subdivided further into two sub-ensembles for dual-quadrature readout (Fig.~\ref{Fig4}b). Resultant Ramsey oscillations versus the total evolution time, $T$, show a frequency ratio of 1:1.99(1):4.10(4), very close to the desired 1:2:4 ratio. 

Following this experimental demonstration, we now discuss two limitations (and possible solutions) of this scheme, specifically related to the frequency noise profile and the atom number per ensemble. First, for the simplest case of shot-to-shot noise of laser frequencies that are otherwise constant during the interrogation, our scheme would allow the clock stability to be improved exponentially~\cite{Rosenband2013} by a factor of $\sqrt{2^{M-1}/M}$; the factor of  $\sqrt{1/M}$ stems from increased QPN in the ensemble used for phase estimation and assumes the total number of atoms is distributed uniformly across the $M$ ensembles.
However, for more general time-dependent frequency noise, the situation is more complex, requiring a higher-order pulse sequence~\cite{Rosenband2013}. We propose one such pulse sequence in Fig.~\ref{Fig4}c, by breaking the total evolution time, $T$, into multiple kernels of length $\tau$. Within each kernel, each ensemble experiences a combination of local dynamical decoupling and free evolution, such that the net phase evolution time is $T, T/2,\cdots, T/2^{M-1}$. This scheme could handle noise profiles where the local phase accumulation period, $\tau$, is shorter than the correlation time of the noise. We numerically find that exponential scaling of the maximal interrogation time is then possible up to a saturation point set by the effective decoupling bandwidth (Ext. Data Fig.~\ref{EFig_MultiEns}).

Second, multi-ensemble estimation schemes in general require sufficient atom number per ensemble to be useful~\cite{Rosenband2013}. When the number of atoms per ensemble is limited, quantum projection noise can negate any advantage, i.e. when the error probability in estimating a phase slip by using a slower evolving ensemble exceeds the actual phase slip probability in the fastest evolving ensemble. For the present experimental demonstration with $N\approx6$ per ensemble we do not expect a metrological gain (Ext. Data Fig. \ref{EFig_MultiEns}), but we note that a generalization of our addressing scheme to two dimensional tweezer clock systems~\cite{Young2020} is straightforward. For example, we imagine a realistic scenario of a $10\times20$ atom array with column-by-column control of tweezer positions, such as could be generated with crossed AODs or an AOD combined with a spatial light modulator. In this case, each pair of columns could realize one ensemble with dual-quadrature readout. Finally, we note that such exponential scaling is possible only up to a time-scale where decoherence is dominated by local oscillator noise. Beyond that, interrogation time will be limited by atomic coherence and ultimately by atomic state lifetime~\cite{Bothwell2022}. 

In summary, we have demonstrated arbitrary local rotations for optical transitions through robust phase-sensitive position control in neutral atom arrays, with sub-diffraction limited precision. We have used such rotations to interrogate two atomic ensembles simultaneously for dual-quadrature readout of a Ramsey interferometry signal with demonstrable metrological gain, and have shown a proof-of-principle for controlling many ensembles with variable sensitivity during Ramsey evolution, a key ingredient of proposals for optimal clocks. Further, these methods could be naturally combined with metrologically useful entangled states~\cite{Li2022,Kessler2014,Marciniak2022} to enable simultaneously high sensitivity with a large dynamic range. More generally, our results are an important step towards a fully programmable quantum optical clock based on neutral atoms, which would incorporate quantum computing techniques towards metrological gains, similar to work done with ion trap devices~\cite{Marciniak2022,Schmidt2005} but likely in a more scalable fashion. Such a universal neutral atom clock system would ideally combine arbitrary local rotations, as shown here, with two-qubit entangling operations for optical transitions~\cite{Schine2022}, and mid-circuit readout and reset, which has not been demonstrated so far.

\textit{Note---}During completion of this work we became aware of related work performing local $\hat{Z}$ rotations and studying entanglement-enhanced metrology in an optical tweezer array clock experiment~\cite{Eckner2023}.

\begin{acknowledgements}
We acknowledge useful conversations with Kon Leung, Hannah Manetsch, Su Direkci, and Tuvia Gefen. Further, we thank Jacob Covey for a careful evaluation of our manuscript. We acknowledge support from the Army Research Office MURI program (W911NF2010136), from the Institute for Quantum Information and Matter, an NSF Physics Frontiers Center (NSF Grant PHY-1733907), the NSF CAREER award (1753386), the AFOSR YIP (FA9550-19-1-0044), the DARPA ONISQ program (W911NF2010021), and the NSF QLCI program (2016245). ALS acknowledges support from the Eddleman Quantum Graduate Fellowship. RF acknowledges support from the Troesh postdoctoral fellowship. RBST acknowledges support from the Taiwan-Caltech Fellowship. THY acknowledges support from the IQIM Visiting Fellowship and in part by the NRF (2022M3K4A1094781).
\end{acknowledgements}

% \section{Author Contributions}
% A.L.S., R.F., and M.E. conceived the idea and experiment. A.L.S., R.F., R.B.T., and J.C. performed the experiments, data analysis, and numerical simulations. A.L.S., R.F., R.B.T., P.S., T.H.Y., and J.C. contributed to the experimental set-up. A.L.S., R.F., and M.E. wrote the manuscript with input from all authors. T.H.Y. and M.E. supervised this project.

% \section*{Competing interests}
% The authors declare no competing interests.

 \section*{Data availability}
The data and codes that support the findings of this study are available from the corresponding author upon reasonable request.

% \newpage
\FloatBarrier

\newpage
\FloatBarrier

\bibliography{library_endreslab.bib}
\bibliographystyle{adamref}

% % This must be in the first 5 lines to tell arXiv to use pdfLaTeX, which is strongly recommended.
\pdfoutput=1
% In particular, the hyperref package requires pdfLaTeX in order to break URLs across lines.

\documentclass[11pt]{article}

% Remove the "review" option to generate the final version.
%\usepackage[review]{ACL2023}
\usepackage{ACL2023}

% Standard package includes
\usepackage{times}
\usepackage{latexsym}

% For proper rendering and hyphenation of words containing Latin characters (including in bib files)
\usepackage[T1]{fontenc}
% For Vietnamese characters
% \usepackage[T5]{fontenc}
% See https://www.latex-project.org/help/documentation/encguide.pdf for other character sets

% This assumes your files are encoded as UTF8
\usepackage[utf8]{inputenc}

% This is not strictly necessary, and may be commented out.
% However, it will improve the layout of the manuscript,
% and will typically save some space.
\usepackage{microtype}

% This is also not strictly necessary, and may be commented out.
% However, it will improve the aesthetics of text in
% the typewriter font.
\usepackage{inconsolata}


% If the title and author information does not fit in the area allocated, uncomment the following
%
%\setlength\titlebox{10cm}
%
% and set <dim> to something 5cm or larger.

%%%%%%%%%%%%%%%%%%%%%%%%%%%%%%%%%%
\usepackage{graphicx}
\usepackage{amsfonts}
\usepackage{amsmath}
\usepackage{bigdelim}
\usepackage{diagbox}
\usepackage{amsthm}
\usepackage{makecell}
\usepackage{mathtools}
\usepackage{booktabs}
\usepackage[shortlabels]{enumitem}
\graphicspath{ {figs/} }

\theoremstyle{remark}
\newtheorem*{question}{Question}

\newcommand{\tk}[1]{\textcolor{blue}{{#1}}}
\newcommand{\sy}[1]{\textcolor{red}{{#1}}}
\newcommand{\mg}[1]{\textcolor{purple}{{#1}}}
\newcommand{\lh}[1]{\textcolor{green}{{#1}}}
\newcommand{\lc}[1]{\textcolor{green}{{#1}}}

% Rounded color box
\definecolor{light_blue}{HTML}{cfdfff}
\usepackage[most]{tcolorbox}
\tcbset{on line, 
        boxsep=1pt, left=0pt,right=0pt,top=0pt,bottom=0pt,
        colframe=white,colback=light_blue,  
        highlight math style={enhanced}
        }

\newcommand{\quash}[1]{}  %Anything in \quash is ignored
\newcommand{\gpt}{\textsc{GPT-2}}
\newcommand{\bert}{\textsc{BERT}}
\newcommand{\bertlarge}{\textsc{BERT-large}}
\newcommand{\mask}{\texttt{[MASK]}}
\newcommand{\cls}{\texttt{[CLS]}}
\newcommand{\sep}{\texttt{[SEP]}}
\newcommand{\mat}{\texttt{mat}}
\newcommand{\id}{\texttt{id}}
\newcommand{\matl}{\texttt{mat}_{\ell \rightarrow \ell'}}
\newcommand{\matattnl}{\texttt{mat\_attn}_{\ell \rightarrow \ell'}}
\newcommand{\matffl}{\texttt{mat\_ffn}_{\ell \rightarrow \ell'}}
\newcommand{\matlnl}{\texttt{mat\_ln1\_ln2}_{\ell \rightarrow \ell'}}
\newcommand{\idl}{\texttt{id}_{\ell \rightarrow \ell'}}
\newcommand{\matlL}{\texttt{mat}_{\ell \rightarrow L}}
\newcommand{\matattnlL}{\texttt{mat\_attn}_{\ell \rightarrow L}}
\newcommand{\matfflL}{\texttt{mat\_ffn}_{\ell \rightarrow L}}
\newcommand{\matlnlL}{\texttt{mat\_ln1\_ln2}_{\ell \rightarrow L}}
\newcommand{\idlL}{\texttt{id}_{\ell \rightarrow L}}

\definecolor{blue(munsell)}{rgb}{0.0, 0.5, 0.69}
%%%%%%%%%%%%%%%%%%%%%%%%%%%%%%%%%%

\title{Jump to Conclusions: Short-Cutting Transformers\\With Linear Transformations}

% Author information can be set in various styles:
% For several authors from the same institution:
% \author{Author 1 \and ... \and Author n \\
%         Address line \\ ... \\ Address line}
% if the names do not fit well on one line use
%         Author 1 \\ {\bf Author 2} \\ ... \\ {\bf Author n} \\
% For authors from different institutions:
% \author{Author 1 \\ Address line \\  ... \\ Address line
%         \And  ... \And
%         Author n \\ Address line \\ ... \\ Address line}
% To start a seperate ``row'' of authors use \AND, as in
% \author{Author 1 \\ Address line \\  ... \\ Address line
%         \AND
%         Author 2 \\ Address line \\ ... \\ Address line \And
%         Author 3 \\ Address line \\ ... \\ Address line}

\author{Alexander Yom Din$^{1}$ ~~~~~ Taelin Karidi$^{1}$ ~~~~~ Leshem Choshen$^{1}$ ~~~~~
Mor Geva$^{2}$ 
\vspace{0.2cm} \\
$^1$Hebrew University of Jerusalem ~~~ $^2$Google Research \\
\small{\texttt{\{alexander.yomdin, taelin.karidi, leshem.choshen\}@mail.huji.ac.il}}, \small{\texttt{pipek@google.com}}}

\quash{
\author{Alexander Yom Din \\
  Hebrew University of Jerusalem \\ \texttt{alexander.yomdin@mail.huji.ac.il} \\\And
  Taelin Karidi \\
  Hebrew University of Jerusalem \\
  \texttt{taelin.karidi@mail.huji.ac.il} \\\And
  Leshem Choshen \\
  Hebrew University of Jerusalem \\ \texttt{leshem.choshen@mail.huji.ac.il} \\\And
  Mor Geva \\
  Google Research \\
  \texttt{pipek@google.com} \\}
}

\begin{document}
\maketitle



\begin{abstract}
% \vspace{-1em}
The diffusion-based generative models have achieved remarkable success in text-based image generation. However, since it contains enormous randomness in generation progress, it is still challenging to apply such models for real-world visual content editing, especially in videos. 
In this paper, we propose \texttt{FateZero}, a zero-shot text-based editing method on real-world videos without per-prompt training or use-specific mask. 
\RM{Specifically, different from a pipeline of two independent inversion and then generation stages, we find the intermediate attention maps during inversions store better structure and motion information. We thus reform them to temporally casual attention and replace them in the generation progress. To further reduce the unnecessary semantic leakage of source video and enhance the editing quality, we then remix the temporally casual attentions via the cross-attention features of the source prompt as the mask.}
To edit videos consistently, we propose several techniques based on the pre-trained models. Firstly, in contrast to the straightforward DDIM inversion technique, our approach captures intermediate attention maps during inversion, which effectively retain both structural and motion information. These maps are directly fused in the editing process rather than generated during denoising. To further minimize semantic leakage of the source video, we then fuse self-attentions with a blending mask obtained by cross-attention features from the source prompt. Furthermore, we have implemented a reform of the self-attention mechanism in denoising UNet by introducing spatial-temporal attention to ensure frame consistency.
Yet succinct, our method is the first one to show the ability of zero-shot text-driven video style and local attribute editing from the trained text-to-image model. We also have a better zero-shot shape-aware editing ability based on the text-to-video model~\cite{tuneavideo}. \RM{Besides video, our unified method also achieves state-of-the-art performance in zero-shot image editing.\chenyang{Need exp or remove the zero-shot image}} Extensive experiments demonstrate our superior temporal consistency and editing capability than previous works.
% The code will be released.
% \chenyang{emphasize: our observation at inversion time} \xiaodong{replacing the bold part to the actual pipeline: \textbf{Specifically, we work on replacing and mixing the attention maps between the inversion and generation since the self-attention map keeps the structure of the original natural image and the cross-attention is semantic-related, after remixing, we replace them in the corresponding generation steps for denoising.}}
% \footnote{Since there is no general video diffusion model is publicly available, we use one-shot video generation method~(Tune-A-Video~\cite{tuneavideo}) as the pretrained video diffusion model for zero-shot video editing\xiaodong{can be removed if we actually zero-shot on video}.}.
\end{abstract}
\section{Introduction}

The ability to reason about plans is critical for performing long-horizon tasks \citep{erol1996hierarchical, sohn2018hierarchical, sharma-etal-2022-skill}, compositional generalization \citep{corona-etal-2021-modular} and generalization to unseen tasks and environments \citep{shridhar2020alfred}.
Consider a simple long-horizon planning scenario where a robot is tasked with preparing a meal and serving it on the table. 
This presents a non-trivial planning problem since the agent needs to understand the sequence of operations required to perform the task and search for the relevant objects in the unfamiliar environment by interacting with various objects. %



Large language models have been recently shown to possess commonsense knowledge about the world such as object affordances and physical dynamics \citep{ouyang2022training,chowdhery2022palm}.
Early approaches considered text based environments and fine-tuned PLMs to predict actions given the history of past observations and actions \citep{jansen-2020-visually,micheli-fleuret-2021-language,yao-etal-2020-keep}.
Recent work has used this ability to reason about plans from text instructions in simulated household environments with simplifying assumptions such as text-only environment observations or feedback \citep{huang2022language,ahn2022can,li2022pre,logeswaran-etal-2022-shot}.


We focus on \emph{visually grounded planning} with PLMs --- the ability to adapt plans based on interaction and visual feedback from the environment.
While PLMs have strong planning commonsense priors, predictions from a PLM may not be directly realizable in the environment since the observation and action spaces are unknown.
This requires \emph{grounding} the PLM in the environment and adapting it to observe visual feedback, which is highly non-trivial.
Some prior works assume the availability of a pre-trained affordance function \citep{ahn2022can} or a success detector \citep{mirchandani2021ella}.
Notably, SayCan \citep{ahn2022can} completely decouples the PLM from observation information by selecting actions that have both high affordability (through a pre-trained affordance model) and high PLM likelihood.
Although this partially addresses the grounding problem, the use of visual feedback for action affordance alone is limited.
Often an agent must choose one of many affordable actions using information from observations.
For example, a driving agent should re-navigate and possibly turn around when encountering a ``road closed'' sign, but both turning around and driving forward are indistinguishable to SayCan because they are both affordable and the PLM is blind to observations.

Another workaround explored in prior work is translating the information in the visual observations to text using a pre-trained captioning system \citep{shridhar2021alfworld,huang2022language}.
However, it can be difficult to faithfully describe an image in words and information is lost in this inherently noisy process, which limits the information available to the planner.



Recent work shows that PLMs can be adapted for various natural language tasks by inserting tunable embeddings or soft prompts at the input of the PLM (also called prompt tuning or prefix tuning)~\citep{li-liang-2021-prefix,lester-etal-2021-power}.
This approach also extends to multi-modal understanding tasks such as image captioning \citep{mokady2021clipcap} and VQA \citep{tsimpoukelli2021multimodal} where images are encoded as soft prompts and finetuned for the target task.
Transformer based architectures have also been successfully applied to offline Reinforcement Learning in recent work \citep{chen2021decision,janner2021offline,li2022pre,reid2022can}.

Taking inspiration from these works, we propose the simple approach of embedding visual observations (`visual prompts') and \textit{directly inserting them as PLM input embeddings}.
The visual encoder and PLM are jointly trained for the target task, an approach we call \textbf{\oursfull}~(\ours).
By teaching the PLM to use observations for planning in an end to end manner, we remove the dependency on external data such as captions and affordability information that was used in prior work.
We show that this simple approach performs better than prior PLM-based planning approaches on two embodied planning benchmarks based on ALFWorld~\citep{shridhar2021alfworld} and Virtualhome~\cite{puig2018virtualhome}.



\section{Related Work}

%Here we summarize prior work on transfer learning and property inference.

%\shortsection{Transfer Learning}
%%Transfer learning reuses features learned by pre-trained models for new tasks, with the pretext that inherent similarities in the generic features will be useful for the downstream tasks and hence reducing their cost of downstream training. Specifically, the downstream model trainer will use a pre-trained upstream model as the starting point for the downstream training, with inclusion of (or replacement with) the task-specific classification layer/module. The downstream model is then trained by either updating all layers of the model (including ones reused from upstream model) or freezing some earlier layers of the reused parts as the ``feature extractor'' and only updating the rest. The latter approach is more popular as the reused feature extractors can already learn useful feature representations and the training cost is also much lower and affordable for individuals with limited computational resources. We study the vulnerability of the latter transfer learning approach in this paper. 


%\shortsection{Transfer Learning} 
Several works have demonstrated risks associated with transfer learning across a variety of attack goals. Wang et al.~\cite{wang2018great} and Yao et al.~\cite{yao2019latent} consider manipulating the upstream model such that the fine-tuned downstream models contain backdoors, misclassifying test inputs that contain predefined backdoor triggers. These transfer manipulations are tailored to their particular attack goals and cannot be applied for the property inference goal considered in this paper. Zou et al.~\cite{zou2020privacy} study the threat of membership inference attacks on transfer learning, but with normally trained upstream models.  
%\dnote{its clear that the goals are different for these attacks, but how similar are the methods?} \ynote{similarity of the methods? more details about the methods? do not know what is expected here}
%In contrast, we investigate the possibility of boosting the effectiveness of property inference by manipulating the upstream model training. % Schuster et al.~\cite{schuster2020humpty} show that the attacker can modify the corpus on which the word embedding is trained such that the downstream NLP models which use that embedding will behave abnormally.

%\shortsection{Property Inference}
The risk of property inference was introduced by Ateniese et al.~\cite{ateniese2015hacking}, % introduces the threat of inferring properties of the training data from pre-trained models, 
and several subsequent works have developed property inference (also known as distribution inference) attacks~\cite{Wang2022GroupPI, suri2022formalizing, Jurez2022BlackBoxAF, Hartmann2022DistributionIR}.
% Ganju et al.~\cite{ganju2018property} and Suri and Evans~\cite{suri2022formalizing} 
These works study property inference against normally trained models, and they launch attacks using a variety of black-box and white-box attacks. All the white-box attacks use meta-classifiers, which take the permutation-invariant representation~\cite{ganju2018property} of the model parameters as the features. We use the state-of-the-art white-box attack~\cite{suri2022formalizing} in our experiments.
%We will use the state-of-the-art white-box method proposed by Ganju et al.~\cite{ganju2018property} and later extended by suri et al.~\cite{suri2022formalizing} in this paper.
%\dnote{do we use these attacks?} 
Melis et al.~\cite{melis2019exploiting} and Zhang et al.~\cite{zhang2021leakage} focus on property inference in distributed training scenarios. In their settings, the attacker is a participant in the global model training and conducts property inference using meta-classifiers that are trained on model outputs or gradients. Similarly, Suri et al.~\cite{suri2022subject} focus on federated learning settings where the attacker is a participant (or the central server) that utilizes black-box attacks for inferring membership of data from particular subjects. %\dnote{if we use black-box attacks, explain which ones, or how ours are related to previous ones} 
For our experiments, We improve the black-box meta-classifier proposed by Zhang et al.~\cite{zhang2021leakage} using the ``query tuning'' technique in Xu et al.~\cite{xu2019detecting}. 

The closest works to ours are Chase et al.~\cite{saeed} and Chaudhari et al.~\cite{Chaudhari2022SNAPEE}, which both consider a scenario where the attacker can manipulate some of the training data of the model to induce a model that significantly increases property inference risk.
% \dnote{it enables precise property inference attacks?}.
These works assume an adversary with the ability to poison the victim's training data, while the adversary in our scenario has no access to the victim's training data, and therefore, their methods are not applicable.
% \dnote{example how different from ours, and why the methods are not applicable}
%Thus, their methods are not applicable to our transfer learning scenario.
%Their methods rely on inducing certain behavior correlated with the properties to be inferred, and thus are not applicable to our transfer learning scenario. \anote{Still a bit unclear why that is the case.}
%
There are also works similar to ours that leverage ``adversarial initializations'' for attack purposes.
% \cite{grosse2019adversarial, boenisch2021curious, wen2022fishing, fowl2021robbing}.
Grosse et al.~\cite{grosse2019adversarial} focus on scenarios where the attacker can control the parameter initialization of a model, and demonstrate that the attacker can use special initializations to damage the performance of the trained model. %This attack is orthogonal to ours.
Other works \cite{boenisch2021curious, wen2022fishing, fowl2021robbing} show that the malicious central server in a federated learning protocol can reconstruct some training samples via falsifying the global model in some training rounds and then analyzing the submitted gradients. These kinds of attacks do not apply to our transfer-learning scenario since the attacker cannot access the downstream gradients, and can only manipulate the upstream training.

\iffalse %%%%%%%%%%%%%%%%%%%%%%%%%%%%%%%%

In this section, we provide the background and also the summary of prior attacks on transfer learning (Section~\ref{sec:transfer_learning}) and property inference (Section~\ref{sec:property_inference}). Then, we introduce the closely related manipulation attacks against machine learning models to boost different privacy risks in Section~\ref{sec:active_inference_attacks}.

%\anote{Do we really need a dedicated section for this? It's barely 2 paragraphs right now.}

%\dnote{the most closely related work to ours are works that attempt to amplify inference attacks by poisoning models, the two most relevant I know of are \url{https://www.computer.org/csdl/proceedings-article/sp/2022/131600b569/1CIO8nmuota} and \url{https://arxiv.org/abs/2204.00032}, but need to look thoroughly for others. We should definitely be describing this and relating it to our work, probably in the introduction. Most of what is here is Background, but should be clear what this section is for (not muddling background and related work)}

\subsection{Transfer Learning} \label{sec:transfer_learning}
Transfer learning reuses features learned by pre-trained models for new tasks, with the pretext that inherent similarities in generic features can be useful for downstream tasks, thus reducing the cost of downstream training. Specifically, the downstream model trainer uses a pre-trained upstream model as the starting point for downstream training, with the inclusion (or replacement) of task-specific classification layers/modules. The downstream model is then trained by either updating all layers of the model (including ones reused from the upstream model) or freezing some earlier layers of the reused parts as the ``feature extractor'' and only updating the rest. The latter approach is more popular as the reused feature extractors can already learn useful feature representations and the training cost is also much lower and affordable for individuals with limited computational resources. We study the vulnerability of the latter transfer learning approach in this paper. 
%mainly in two ways:  1) all the layers (including ones reused from ) and tune the full model; the other one is to freeze some earlier layers of the model as the feature extractor and only tune the rest later layers. The second update strategy could achieve better efficiency since the frozen layers can already produce meaningful feature representations~\cite{wang2018great,yao2019latent}, and we will study the transfer learning using this strategy. 

Recently, various attacks have been proposed for the transfer learning setting, but with different attack goals from ours. Wang et al.~\cite{wang2018great} generate adversarial examples against black-box student models that transfer knowledge from publicly available teacher models without repeated queries. Yao et al.~\cite{yao2019latent} propose to manipulate the upstream model such that the downstream models derived from the upstream model contain backdoors, which would misclassify test inputs that contain some predefined backdoor triggers. Zou et al.~\cite{zou2020privacy} study the threat of membership inference attacks on transfer learning and the upstream models are trained normally. In contrast, we investigate the possibility of boosting the effectiveness of property inference by manipulating the upstream model training. Schuster et al.~\cite{schuster2020humpty} show that the attacker can modify the corpus on which the word embedding is trained such that the downstream NLP models which use that embedding will behave abnormally.

%This additionally allows model trainers to achieve satisfactory performance with limited training samples, leading to reduced computational costs. The most common approach reuses parameters in the earlier layers of the pre-trained model, either by fixing them as the feature extractor or just using them for initialization, to conduct downstream training.

\subsection{Property Inference} \label{sec:property_inference}

\shortsection{Property Inference Attacks} In property inference attacks, the adversary aims to infer some sensitive properties of some data, given a model trained on it. For example, the adversary may be interested in sensitive properties like the presence of people of a specific race in the dataset~\cite{ateniese2015hacking, melis2019exploiting}), or even be curious about the 
the statistics of the training set (e.g, the ratio of people with a specific gender~\cite{saeed, ganju2018property, suri2022formalizing, zhang2021leakage}).


Ateniese et al.~\cite{ateniese2015hacking} were the first to identify the threat of inferring properties of the training data from pre-trained models. Ganju et al.~\cite{ganju2018property} and Suri and Evans~\cite{suri2022formalizing} 
study property inference against normally trained models, and they launch attacks using white-box meta-classifiers, which utilize the permutation-invariance representation~\cite{ganju2018property} of the model parameters, while other works focus on distributed training~\cite{zhang2021leakage} where the attacker is a participant in the global model training and conducts property inference using meta-classifiers trained on model outputs. Similarly, Suri et al.~\cite{suri2022subject} focus on federated learning, where the attacker is a participant (or the central server) that utilizes black-box attacks for inferring membership of data from particular subjects. Chase et al.~\cite{saeed} propose an active property inference attack for data poisoning scenarios, which we will cover and compare to in Section~\ref{sec:active_inference_attacks}.

%The closest work to ours are by Chase et al.~\cite{saeed} and Tramer et al.~\cite{tramer2022truth}. In their work, the attacker can manipulate some of the training data of the model such that a model trained (from scratch) on the poisoned data has an increased inference risk. However, their methods are not applicable to the transfer learning scenario. 
%In this work, we will focus on the property inference in transfer learning scenarios in which the attacker releases the upstream model and infer sensitive properties of the downstream models tuned from that upstream model.
% 

\shortsection{Defenses}
Defending against property inference attacks is an open problem. There are no studies in the current literature on active adversaries, and only a couple on passive ones. Ma et. al.~\cite{ma2021nosnoop} propose a defense against property inference attacks on data batches in the  collaborative learning setting. However, adversaries in the transfer-learning setting do not have access to batch-wise gradients of the downstream trainer. Chen and Ohrimenko~\cite{chen2022protecting} utilize mechanisms that add carefully-crafted noise to features to provide theoretical guarantees against inference adversaries, but focus on query-based access to the underlying dataset, not a machine learning model trained on it. These existing defenses thus do not apply to our threat model.

%propose a framework that reduces property inference to Boolean functions of individual members, posing the ratio of members satisfying the given function in a dataset as the property. These property inference attacks have since then been proposed as distribution inference attacks~\cite{suri2022formalizing}, presenting such attacks as inferring properties of the distributions used to sample datasets, differentiating them from exact inference attacks like dataset inference~\cite{maini2021dataset}. Nearly all property inference attacks use meta-classifiers to perform inference: training models on versions of datasets with and without the target property, followed by training a meta-classifier on top of these classifiers's model representations. These representations can take several forms: using model weights themselves with permutation-invariance~\cite{ganju2018property}, or model activations or logits for a generated set of query points~\cite{xu2019detecting}. However, the capability of such approaches is limited: the most that these attacks have been shown to work is medium-sized convolutional networks on the CelebA dataset~\cite{suri2022formalizing}.


\subsection{Active Privacy Attacks} \label{sec:active_inference_attacks}
% Perhaps the closely related works to ours as ones that proactively enhance the effectiveness of privacy attacks by manipulating the model training process in certain ways~\cite{saeed, melis2019exploiting, nasr2019comprehensive, tramer2022truth}. 
%shown that the adversary can, by using proactive ways, achieve stronger attacks that infer private information from deep learning systems~\cite{nasr2019comprehensive, melis2019exploiting, tramer2022truth, saeed}. In this section, we introduce the ones that are close to ours.

In the decentralized federated learning training, by submitting specially crafted gradients to the central server, malicious agents can increase membership inference risk~\cite{nasr2019comprehensive} and property inference risks~\cite{melis2019exploiting} of other benign agents' training data. However, these attacks do not apply to transfer learning scenario, as the attacker cannot control model gradients of downstream training. In the centralized setting, researchers propose attacks to poison the victim's training data such that the impacts of attribute inference and membership inference~\cite{tramer2022truth} and property inference~\cite{saeed} attacks are amplified on the poisoned model.
The ability to poison the victim's data is a threat model orthogonal to ours, since we have no access to the victim's downstream data. While there is scope to combine such approaches for stronger attacks (albeit with stronger access assumptions), we choose to focus on the scenario with no read/write access to the victim's data.

\fi %%%%%%%%%%%%%%%%%%%%%%%%%%%%%%%%

\section{Linear Shortcut Across Blocks}
\label{sec:layer_jump}

To use a hidden representation from layer $\ell<L$ as a final representation, we propose to cast it using linear regression, while skipping the computation in-between these layers. More generally, this approach can be applied to cast any $\ell$-th hidden representation to any subsequent layer $\ell'>\ell$.


\subsection{Method}
\label{subsec:methodology_linear_shortcut}

Given a source layer $\ell$ and a target layer $\ell'$ such that $0 \leq \ell < \ell' \leq L$, our goal is to learn a mapping
%$A_{\ell', \ell} \in \mathbb{R}^{d_h \times d_h}$
from hidden representations at layer $\ell$ to those at layer $\ell'$. To this end, we first collect a set of corresponding hidden representation pairs $(h^\ell, h^{\ell'})$. Concretely, we run a set $\mathcal{T}$ of input sequences through the model, and for each input $s$, we extract the hidden representations $h_{i_s}^{\ell}, h_{i_s}^{\ell'}$, where $i_s$ is a random position in $s$.
Next, we learn a matrix $A_{\ell', \ell} \in \mathbb{R}^{d_h \times d_h}$ by fitting linear regression over $\mathcal{T}$, i.e., $A_{\ell', \ell}$ is a numerical minimizer for:
$$ A \mapsto \sum_{s \in \mathcal{T}} || A \cdot h_{i_s}^\ell - h_{i_s}^{\ell'} ||^2,$$ 
and define the mapping of a representation $h$ from layer $\ell$ to layer $\ell'$ as:
\begin{equation}
\label{eq:linear_jump}
    \matl{} (h) \coloneqq A_{\ell', \ell} \cdot h.
\end{equation}


\subsection{Baseline}
\label{subsec:baseline}

We evaluate 
% our method against 
the prevalent approach of ``reading'' hidden representations directly, without any transformation. 
Namely, the propagation of a hidden representation from layer $\ell$ to layer $\ell'$ is given by the identity function, dubbed \id{}:

$$ \idl{} (h) \coloneqq h.$$

% Notably, 
This baseline 
assumes that representations at different layers operate in the same linear space.

\subsection{Quality of Fit}
\label{subsec:experiments_r2}

We first evaluate our method by measuring how well the learned linear mappings approximate the representations at the target layer. To this end, we calculate the (coordinate-averaged) $r^2$-score of our mapping's outputs with respect to the representations obtained from a full inference pass, and compare to the same for the \id{} baseline.


\paragraph{Models.}

We use \gpt{} \cite{radford2019language}, a decoder-only auto-regressive LM, with $L = 48$, $d_h = 1600$, and \bert{} \cite{devlin-etal-2019-bert}, an encoder-only model trained with masked language modeling, with $L=24$, $d_h=1024$.
% \footnote{\label{footnote:hf}We use models and data from Huggingface \cite{wolf-etal-2020-transformers,lhoest-etal-2021-datasets}.}
%For masked token prediction, we use a masked LM head pre-trained for our \bert{} model.

% \footnote{Specifically, we use the Huggingface Transformers \cite{wolf-etal-2020-transformers} implementations of all these models.}

%\sy{We use \gpt{} \cite{radford2019language}, a decoder-only auto-regressive LM, coming in four scales; $\texttt{gpt2}$ ($L = 12$, $d_h = 768$), $\texttt{gpt2-medium}$ ($L = 24$, $d_h = 1024$), $\texttt{gpt2-large}$ ($L = 36$, $d_h = 1280$) and $\texttt{gpt2-xl}$ ($L = 48$, $d_h = 1600$). Also, we use \bert{} \cite{devlin-etal-2019-bert}, an encoder-only model trained with masked language modeling, coming in two scales;  \texttt{bert-base-uncased} ($L=12$, $d_h=768$) and \texttt{bert-large-uncased} ($L=24$, $d_h=1024$). For masked token prediction, we use masked LM heads pre-trained for our models. Specifically, we use the Huggingface Transformers \cite{wolf-etal-2020-transformers} implementations of all these models. The plots presented in this section are for $48$-layered \gpt{} and $24$-layered \bert{}.}

%\sy{We use \gpt{} \cite{radford2019language}, a decoder-only auto-regressive LM, in the Huggingface \cite{wolf-etal-2020-transformers} implementation\footnote{\url{https://huggingface.co/gpt2}}, coming in four scales; $\texttt{gpt2}$ ($L = 12$, $d_h = 768$), $\texttt{gpt2-medium}$ ($L = 24$, $d_h = 1024$), $\texttt{gpt2-large}$ ($L = 36$, $d_h = 1280$) and $\texttt{gpt2-xl}$ ($L = 48$, $d_h = 1600$). Also, we use \bert{} \cite{devlin-etal-2019-bert}, an encoder-only model trained with masked language modeling, in the Hugginface implementation, coming in two scales;  \texttt{bert-base-uncased}\footnote{\url{https://huggingface.co/bert-base-uncased}} ($L=12$, $d_h=768$) and \texttt{bert-large-uncased}\footnote{\url{https://huggingface.co/bert-large-uncased}} ($L=24$, $d_h=1024$). For masked token prediction, we use the \texttt{BertForMaskedLM} heads from Huggingface, pretrained for these models. The plots presented in this section are for $48$-layered \gpt{} and $24$-layered \bert{}.}

\paragraph{Data.}
We sample random sentences from Wikipedia,
% \footref{footnote:hf} 
collecting 9,000 (resp. 3,000) sentences for the training set $\mathcal{T}$ (resp. validation set $\mathcal{V}$).\footnote{We use sentences rather than full documents to simplify the analysis.}
%\sy{We use two data sources to evaluate our method. One is Wikiepdia \cite{lhoest-etal-2021-datasets}\footnote{\url{https://huggingface.co/datasets/wikipedia}}; we use \texttt{spaCy}\footnote{\url{https://spacy.io/}} to divide documents into sentences\footnote{We use sentences rather than full documents to simplify the analysis.}\footnote{We pick randomly a Wikipedia document and then pick randomly a sentence ending in a newline character in it. \sy{[maybe this footnote is not needed?]}}, collecting 9,000 (resp. 3,000) random sentences for the training set $\mathcal{T}$ (resp. validation set $\mathcal{V}$). The second is a news article sentences dataset, the 10K English 2020 news sentences corpus
% \footnote{\url{https://downloads.wortschatz-leipzig.de/corpora/eng_news_2020_10K.tar.gz}} from the Leipzig Corpora Collection \cite{goldhahn-etal-2012-building}, which we randomly divide into a training set $\mathcal{T}$ consisting of 9,000 examples and a validation set $\mathcal{V}$ consisting of 1,000 examples.
% We truncate sentences to the maximal token length allowed by the model \mg{do we ever need to truncate? a sentence has about 10 words and the max. input len is thousands} \sy{[I surely did not need to in Leipzig, but discovered (via a transformers runtime warning) that I do need to for some (probably a minority) of the Wikipedia sentences. This probably has to do with that it is not really ``sentences" necessarily, for example, I noticed that it has some listings or something like that (bulleted items)... So some minority might get very long I guess...]}.
For each example $s$, we select a random position $i_s$ and extract the hidden representations $h_{i_s}^{\ell}$ at that position from all the layers.
For \bert{}, we first replace the input token at position $i_s$ with a \mask{} token, as our motivation is interpreting predictions, which are obtained via masked tokens in \bert{} (see \S\ref{subsec:BERT}).
Thus, in this case, the hidden representations we consider
%in the case of \bert{}
are of \mask{} tokens only.
%As we observed highly similar results for the two data sources across all our experiments, throughout the paper we will mainly report results for Wikipedia (except for \S\ref{sec:robustness}, where we cross-validate).


\begin{figure}[t]
\includegraphics[scale=0.2]{figs/r2_scores_48.pdf}
% \includegraphics[width=\columnwidth]{figs/r2_scores_48.pdf}
\caption{The coordinate-averaged $r^2$-score of $\matl{}$ (left) and $\idl{}$ (right) (\gpt{}).}
\label{fig:r2_scores}
\end{figure}


\begin{figure}[t]
\setlength{\belowcaptionskip}{-10pt}
\includegraphics[scale=0.2]{figs/bertmask_r2_scores_24.pdf}
% \includegraphics[width=\columnwidth]{figs/bertmask_r2_scores_24.pdf}
\caption{The coordinate-averaged $r^2$-score of $\matl{}$ (left) and $\idl{}$ (right) (\bert{}).}
\label{fig:bertmask_r2_scores}
\end{figure}



\paragraph{Evaluation.}
For every pair of layers $\ell, \ell'$, such that $0 \leq \ell < \ell' \leq L$, we use the training set $\mathcal{T}$ to fit linear regression as described in \S\ref{subsec:methodology_linear_shortcut}, and obtain a mapping $\matl{}$. 
Next, we evaluate the quality of $\matl{}$ as well as of $\idl{}$ using the $r^2$-coefficient, uniformly averaged over all coordinates. Concretely, we compute the $r^2$-coefficient of each of the predicted representations $\matl{} (h_{i_s}^{\ell})$ and $\idl{} (h_{i_s}^{\ell})$ versus the true representations $h_{i_s}^{\ell'}$
over all $s \in \mathcal{V}$.
%as we vary $s \in \mathcal{V}$.
%for every $s \in \mathcal{V}$.



\paragraph{Results.}
Results for \gpt{} and \bert{} are presented in Figs.~\ref{fig:r2_scores} and~\ref{fig:bertmask_r2_scores}, respectively.
In both models, \mat{} consistently yields better approximations than \id{}, as it obtains higher $r^2$-scores (in blue) across the network. 
This gap between \mat{} and \id{} is especially evident in \bert{}, where \id{} completely fails to map the representations between most layers, suggesting that hidden representations are modified  substantially by every transformer block.
Overall, this highlights the shortcoming of existing practices to inspect representations in the same linear space, and the gains from using our method to approximate future layers.
% in the network.
\section{Linear Shortcut for Language Modeling}
\label{sec:prediction}

We saw that our method approximates future hidden representations substantially better than a naive propagation. 
In this section, we will show that this improvement also translates to better predictive abilities from earlier layers. Specifically, we will use our method to estimate how often intermediate representations encode the final prediction, in the context of two fundamental LM tasks; next token prediction and masked token prediction.

\paragraph{Evaluation Metrics.}
Let $h, h' \in \mathbb{R}^{d_h}$ be a final representation and a substitute final representation obtained by some mapping, and denote by $\delta (h), \delta (h') \in \mathbb{R}^{d_v}$ their corresponding output probability distributions (obtained through projection to the output vocabulary -- see details below). 
We measure the prediction quality of $h'$ with respect to $h$ using two metrics:
\begin{itemize}
[leftmargin=*,topsep=1pt,parsep=1pt]
    \item \textbf{Precision@$k$} ($\uparrow$ is better): This checks whether the token with the highest probability according to $\delta(h')$ appears in the top-$k$ tokens according to $\delta(h)$. Namely, we sort $\delta(h)$ and assign a score of $1$ if $\arg\max(\delta(h'))$ appears in the top-$k$ tokens by $\delta(h)$, and $0$ otherwise.
    
    \item \textbf{Surprisal} ($\downarrow$ is better): We measure the minus log-probability according to $\delta(h)$, of the highest-probability token according to $\delta(h')$. Intuitively, low values mean that the model sees the substitute result as probable and hence not surprising.
\end{itemize}

\noindent We report the average Precision@$k$ and Surprisal over the validation set $\mathcal{V}$.



\subsection{Next Token Prediction}
\label{subsec:next_token_prediction_task}

Auto-regressive LMs output for every position a probability distribution over the vocabulary for the next token. Specifically, the output distribution for every position $i$ is given by $\delta (h_i^L)$, where:
\begin{equation}\label{eq:output_distribution}
    \delta (h) = \texttt{softmax} ( E^\top \cdot h) \in \mathbb{R}^{d_v}
\end{equation}
For some LMs, including \gpt{}, a layer normalization $\texttt{ln\_f}$ is applied to the final layer representation before this conversion (i.e., computing $\delta (\texttt{ln\_f}(h))$ rather than $\delta (h)$).

Recall that our goal is to measure how well this distribution can be estimated from intermediate representations, i.e. estimating $\delta (h_i^L)$ from $\delta (h_i^\ell)$ where $\ell<L$. To this end, we first run examples from the validation set through the model, while extracting for each example $s$ the hidden representation of a random position $i_s$ at every layer. Next, we apply our mappings $\matlL{}$ and the $\idlL{}$ baseline to cast the hidden representations of every layer $\ell$ to final layer substitutes (see \S\ref{sec:layer_jump}). Last, for each layer, we convert its corresponding final-layer substitute to an output distribution (Eq.~\ref{eq:output_distribution}) and compute the average Precision@$k$ (for $k=1,5,10$) and Surprisal scores with respect to the final output distribution, over the validation set.

\paragraph{Results.}
Figs.~\ref{fig:pre} and~\ref{fig:surp} show the average Precision@$k$ and Surprisal scores per layer in $48$-layered \gpt{}, respectively (the plots for the other \gpt{} models are presented in \S\ref{sec:app_scale}). Across all layers, \mat{} outperforms \id{} in terms of both scores, often by a large margin (e.g. till layer $44$ the Precision@$1$ achieved by \mat{} is bigger than that of $\id{}$ by more than $0.2$). 
This shows that linear mappings enable not just better estimation of final layer representations, but also of the predictions they induce. Moreover, the relatively high Precision@$k$ scores of \mat{} in early layers ($0.62$-$0.82$ for $k=10$, $0.52$-$0.74$ for $k=5$, and $0.28$-$0.45$ for $k=1$) suggest that early representations already encode a good estimation of the final prediction. Also, the substantially lower Surprisal scores of \mat{} compared to \id{} imply that our method allows for a more representative reading into the layer-wise prediction-formation of the model than allowed through direct projection to the vocabulary.

\begin{figure}[t]
\centering
\includegraphics[scale=0.4]{figs/pre_48.pdf}
\caption{Precision@$k$ ($k = 1,5, 10$) of $\matlL{}$ and $\idlL{}$ for next token prediction in $48$-layered \gpt{}.}
\label{fig:pre}
\end{figure}

\begin{figure}[t]
\centering
\includegraphics[scale=0.35]{figs/surp_48.pdf}
\caption{Surprisal for $\matlL$ and the baseline $\idlL{}$ ($48$-layered \gpt{} next token prediction task). A 95\% confidence interval surrounds the lines.}
\label{fig:surp}
\end{figure}

\subsection{Masked Token Prediction}
\label{subsec:BERT}

We now conduct the same experiment for the task of masked language modeling, where the model predicts a probability distribution of a masked token in the input rather than the token that follows the input. Unlike next token prediction, where the output distribution is computed from representations of varying input tokens, in masked token prediction the output is always obtained from representations of the same input token (i.e. \texttt{[MASK]}).

For this experiment, we use \bert{}, on top of which we use a pretrained masked language model head $\delta$; given a token sequence $s$, a \mask{} token inside it and its final representation $h$, $\delta (h) \in \mathbb{R}^{d_v}$
 is a probability distribution over tokens giving the model's assessment
 of the likelihood of tokens to be fitting in place of the \mask{} token in $s$.


\begin{figure}[t]
\centering
\includegraphics[scale=0.4]{figs/bertmask_pre_24.pdf}
\caption{Precision@$k$ ($k = 1,5, 10$) for  $\matlL{}$ and the baseline $\idlL{}$ ($24$-layered \bert{} masked token prediction task).}
\label{fig:bertmask_pre}
\end{figure}

\begin{figure}[t]
\centering
\includegraphics[scale=0.35]{figs/bertmask_surp_24.pdf}
\caption{Surprisal for $\matlL{}$ and the baseline $\idlL{}$ ($24$-layered \bert{} masked token prediction task). A 95\% confidence interval surrounds the lines.}
\label{fig:bertmask_surp}
\end{figure}

\paragraph{Results.}
Figs.~\ref{fig:bertmask_pre} and~\ref{fig:bertmask_surp} present the average Precision@$k$ and Surprisal scores per layer in $24$-layered \bert{} (the plots for the $12$-layered \bert{} model are presented in \S\ref{sec:app_scale}), overall showing trends similar to those observed for next token prediction in \gpt{} (\S\ref{subsec:next_token_prediction_task}). This is despite the differences between the two tasks and the considerable architectural differences between \bert{} and \gpt{}.
Notably, the superiority of \mat{} over \id{} in this setting is even more prominent; 
while \mat{}'s precision is between $0.2-0.6$ in the first ten layers (Fig.~\ref{fig:bertmask_pre}), \id{}'s precision for all values of $k$ is close to zero, again strongly indicating that our method allows for better reading into early layer hidden representations. 
More generally, \mat{} improves the Precision@$1$ of \id{} by more than $17\%$ at most layers, and unveils that a substantial amount of predictions ($>25\%$ starting from layer $3$) appear already in the very first layers.
Interestingly, the (rough) divide between the first half of layers and last half of layers for $\id{}$ in Figs.~\ref{fig:bertmask_pre},~\ref{fig:bertmask_surp} seems to align with the two-hump shape of the blue region for $\mat{}$ in Fig.~\ref{fig:bertmask_r2_scores}.

\paragraph{Analysis.}
We manually compare the predictions of our mapping $\matlL{}$ with $\idlL{}$, for a $24$-layered \bert{} model.  Concretely, we select 50 random sentences from the Leipzig dataset. Next, for each layer $\ell$, we manually analyze how many of the top-$5$ tokens according to $\matlL{}$ and $\idlL{}$ fit into context. We consider a token to fit into context if it is grammatically plausible within the sentence (see Tab.~\ref{tab:manual} for concrete examples).
In the resulting $1250$ instances (i.e. $50$ sentences $\times$ $25$ representations), we observe a substantially higher plausibility rate of $85.36\%$ for \mat{} compared to $52.8\%$ for \id{}. In fact, only in less than $4.3\%$ of the instances there are more plausible tokens among the top-$5$ tokens according to \id{} than among the top-$5$ tokens according to \mat{}, further supporting the Surprisal results above.

\begin{table*}
\footnotesize
\setlength{\belowcaptionskip}{-15pt}
\begin{tabular}{p{0.3\linewidth}ccccc}
& $\texttt{id}_{4 \rightarrow 24}$ & $\texttt{mat}_{4 \rightarrow 24}$ & $\texttt{id}_{12 \rightarrow 24}$ & $\texttt{mat}_{12 \rightarrow 24}$ & $\texttt{id}_{24 \rightarrow 24}$ \\ \midrule
\multirow{5}{=}{aldridge had shoulder surgery in \mask{}.} & fellowship & \tcbox{time} & cyclist & \tcbox{2009} & \tcbox{september} \\
& employment & \tcbox{it} & emergencies & \tcbox{2008} & \tcbox{november} \\
& agreement & her & seniors & \tcbox{2010} & \tcbox{december} \\
& \#\#ostal & them & cycling & \tcbox{2006} & \tcbox{august} \\
& \#\#com & work & \tcbox{pennsylvania} & \tcbox{2007} & \tcbox{july} \\ \midrule
\multirow{5}{=}{on your next view you will be asked to \mask{} continue reading.} & \#\#com & be & be & be & \tcbox{please} \\
& accreditation & get & undergo & \tcbox{please} & \tcbox{simply} \\ 
& $	\copyright$ & go & spartans & help & \tcbox{also} \\ 
& fellowship & \tcbox{help} & seniors & \tcbox{simply} & \tcbox{again} \\ 
& summer & have & * & say & \tcbox{immediately} \\ \bottomrule
\end{tabular}
\caption{Examples of top-$5$ predictions at layers $4$, $12$ and $24$, under the mappings $\matlL{}$ and $\idlL{}$, for a $24$-layered \bert{} model. Grammatically plausible predictions (according to a human annotator) are marked in \tcbox{blue}. Note that at layer $24$ the predictions of $\matlL{}$ and $\idlL{}$ are the same (by definition).} 
\label{tab:manual}
\end{table*}

\section{Implication to Early Exiting}
\label{sec:applications}

%The fact that it is often possible to approximate
The possibility of approximating
the final prediction already in the early layers has important implications for efficiency; applying our linear mapping instead of executing transformer blocks of quadratic time complexity, could save a substantial portion of the computation. In this section, we demonstrate this in the context of early exiting.

When 
% performing transformer model inference under 
using an early exit strategy \cite{schwartz-etal-2020-right, xin-etal-2020-deebert, schuster2022confident}, one aims at deciding dynamically at which layer to stop the computation and ``read'' the prediction from the hidden representation of that layer.
More precisely, under a confidence measure paradigm, one decides to stop the computation for a position $i$ at layer $\ell$ based on a confidence criterion, that is derived from casting the hidden representation $h_i^\ell$ as a final-layer representation and converting it to an output probability distribution. Specifically, following \citet{schuster2022confident}, a decision to exit is made if the difference between the highest and the second highest probabilities is bigger than $$ 0.9 \cdot \lambda + 0.1 \cdot {\rm exp} (-4 i / N),$$
where $N$ is the average length of the input until position $i_s$ for $s \in \mathcal{V}$, and $\lambda$ is a hyper-parameter.

\begin{figure}[t]
\setlength{\belowcaptionskip}{-10pt}
\centering
\includegraphics[width=\columnwidth]{figs/ee_gpt2bert.pdf}
\caption{Precision@$1$ with early exit and ``fixed exit'', applied to the $24$-layer \gpt{} for next token prediction (left) and the $24$-layer \bert{} for masked token prediction (right). Varying the confidence parameter $\lambda$, the $x$-coordinate is the average number of layers processed before an early exit decision is reached.}
\label{fig:ee_gpt2bert}
\end{figure}

\quash{
\begin{figure}[t]
\setlength{\belowcaptionskip}{-10pt}
\centering
\includegraphics[scale=0.35]{figs/ee_pre1_24.pdf}
\caption{Precision@$1$ for the various early exit methods, and previous ``fixed exit'' methods for comparison ($24$-layer \gpt{} next token prediction task). Varying the confidence parameter $\lambda$, the $x$-coordinate is the average number of layers processed before an early exit decision is reached.}
\label{fig:ee_pre1}
\end{figure}
}

\paragraph{Experiment.}
We assess the utility of our mapping $\matlL{}$ for early exit as a plug-and-play replacement for $\idlL{}$, through which intermediate representations are cast into final-layer representations.
We use \gpt{} for the next token prediction and \bert{} for masked token prediction (both with 24 layers).
We run each of the models over the validation set examples, while varying the confidence parameter $\lambda$ and using either $\idlL{}$ or $\matlL{}$ for casting intermediate representations.
Furthermore, we compare these early exit variants to the ``fixed exit'' strategy from \S\ref{sec:prediction}, where the computation is stopped after a pre-defined number of layers rather than relying on a dynamic decision.
We evaluate each variant in terms of both prediction's accuracy, using the Precision@$1$ metric (see \S\ref{sec:prediction}), and efficiency, measured as the average number of transformer layers processed during inference.


\paragraph{Results.}
%Figs.~\ref{fig:ee_pre1} and~\ref{fig:bertmask_ee_pre1}
Fig.~\ref{fig:ee_gpt2bert}
plots the average Precision@$1$ score against the average number of layers processed, for $24$-layer \gpt{} and $24$-layer \bert{}. For both models, under an early exit strategy our mapping \mat{} again provides a substantial improvement over \id{}.
For example, aiming at $95\%$ average precision, \mat{} saves $\sim3.3$ ($13.8$\%) layers in \gpt{} compared to only $\sim1.4$ ($5.9$\%) layers by \id{}, and $\sim4.8$ ($20$\%) layers in \bert{} versus $\sim3.5$ ($14.6$\%) layers by \id{}.
These results highlight the potential gains prominent early exit methods can obtain by using our method.
Notably, in both models and for each of the mapping methods, early exit obtains better results than fixed layer exit, as expected. 

\quash{
\begin{figure}[t]
\setlength{\belowcaptionskip}{-10pt}
\centering
\includegraphics[scale=0.35]{figs/bertmask_ee_pre1_24.pdf}
\caption{Precision@$1$ for the various early exit methods, and previous ``fixed exit'' methods for comparison ($24$-layer \bert{} masked token prediction task). Varying the confidence parameter $\lambda$, the $x$-coordinate is the average number of layers processed before an early exit decision is reached.}
\label{fig:bertmask_ee_pre1}
\end{figure}
}
\section{Linear Shortcut Across Sub-Modules}
\label{sec:submodules}

% Our experiments show that
% , despite the commonly-applied simplification by interpretability works, transformer layers do not operate in the same linear space and 
% there is a major gap in approximating future representations using an identity mapping (\S\ref{sec:layer_jump}, \S\ref{sec:prediction}).
% Here, 
In this section, we investigate whether discrepancies across layers result from specific sub-modules or are a general behaviour of all sub-modules in the network.  
This is done by extending our approach to test how well particular components in transformer blocks can be linearly approximated. 


\paragraph{Method.}

Consider \gpt{} for definiteness, then:
% we have 
$$ \texttt{b}_{\ell} = \texttt{b}_{\ell}^{\texttt{ffn}} \circ \texttt{b}_{\ell}^{\texttt{attn}}$$ 
% with
\begin{equation}\label{eq:attn} \texttt{b}^{\texttt{attn}}_{\ell} (H) = \texttt{attn}_{\ell} (\texttt{ln1}_{\ell} (H)) + H,\end{equation} 
where $\texttt{attn}_{\ell}$ is
%a multi-head self-attention
a MHSA
layer and \texttt{ln1} is a layer normalization (LN), and 
$$ \texttt{b}^{\texttt{ffn}}_{\ell} (H) = \texttt{ffn}_{\ell} (\texttt{ln2}_{\ell} (H)) + H,$$  
where $\texttt{ffn}_{\ell}$ is
%a feed-forward network
an FFN
layer and $\texttt{ln2}$ is a LN.
\quash{
Given a block $\texttt{b}_\ell$ and one of its sub-modules $\texttt{ln1}_\ell, \ \texttt{attn}_\ell, \ \texttt{ln2}_\ell$, or $\texttt{ffn}_\ell$, we fit linear regression approximating the output of the sub-module given its input and then use it in order to define mappings, as we now describe.
}
Given a block $\texttt{b}_\ell$ and one of its sub-modules $\texttt{ln1}_\ell, \ \texttt{attn}_\ell, \ \texttt{ln2}_\ell$, or $\texttt{ffn}_\ell$, we fit linear regression approximating the output of the sub-module given its input, and then use it to define mappings $\matattnl{}$, $\matlnl{}$ and $\matffl{}$.
%We provide the definition of $\matattnl{}$ below, and that of the other two in App. \ref{sec:app_submodule_skip_description}.
We provide the formal definitions of these mappings in App. \ref{sec:app_submodule_skip_description}.
\iffalse
\paragraph{$\matattnl{}$.}
%Illustrating this on $\texttt{attn}_\ell$ for definiteness,
For an input $s$, let $v^\ell_{i_s}$ be the vector at position $i_s$ in the output of $\texttt{attn}_\ell (\texttt{ln1}_\ell (H^{\ell - 1}))$. We denote by $A_\ell^{\texttt{attn}} \in \mathbb{R}^{d_h \times d_h}$ the matrix numerically minimizing 
$$ A \mapsto \sum_{s \in \mathcal{T}} || A \cdot \texttt{ln1}_\ell (h^{\ell-1}_{i_s}) - v^\ell_{i_s}||^2,$$
and define an attention sub-module replacement (Eq.~\ref{eq:attn}) by $$
\texttt{b}^{\overline{\texttt{attn}}}_\ell (h) \coloneqq A_{\ell}^{\texttt{attn}} \cdot \texttt{ln1}_\ell (h) + h. $$
We then define a mapping between two layers ${\ell \rightarrow \ell'}$ by:
$$ \matattnl{} (h) \coloneqq $$
$$ \texttt{b}^{\texttt{ffn}}_{\ell'} ( \texttt{b}^{\overline{\texttt{attn}}}_{\ell'} ( \ldots (\texttt{b}^{\texttt{ffn}}_{\ell+1} ( \texttt{b}^{\overline{\texttt{attn}}}_{\ell+1} (h)))\ldots)).$$ 
Namely, when applying each $\ell''$-th block, $\ell < \ell'' \leq \ell'$, we replace its attention sub-module $\texttt{attn}_{\ell''}$ by its linear approximation.
%In an analogous way, we consider the mappings $\matffl{}$ and $\matlnl{}$, where in the latter we perform the linear shortcut both for \texttt{ln1} and for \texttt{ln2} (see~\S\ref{sec:app_submodule_skip_description} for precise descriptions).
Importantly, unlike the original attention module, the approximation $\texttt{b}^{\overline{\texttt{attn}}}_\ell$ operates on each position independently, and therefore applying $\matattnl{}$ disables any contextualization between the layers $\ell$ and $\ell'$. Note that this is not the case for $\matffl{}$ and $\matlnl{}$, which retain the self-attention sub-modules and operate contextually.
\fi

\paragraph{Evaluation.}


We analyze the $24$-layered \gpt{}, and proceed completely analogously to \S\ref{subsec:next_token_prediction_task}, evaluating the Precision@$1$ and Surprisal metrics for the mappings $\matattnlL{}$, $\matfflL{}$ and $\matlnlL{}$.

\begin{figure}[t]
\setlength{\belowcaptionskip}{-0pt}
\centering
%\includegraphics[scale=0.2]
\includegraphics[width=\columnwidth]{figs/parts_presurp_24.pdf}
\caption{Precision@$1$ and Surprisal for the various sub-module linear mappings, and $\matlL{}$ for comparison ($24$-layer \gpt{} next token prediction task). A 95\% confidence interval surrounds the Surprisal lines.}
\label{fig:parts_presurp}
\end{figure}

\quash{
\begin{figure}[t]
\centering
\includegraphics[scale=0.4]{figs/parts_pre1_24.pdf}
\caption{Precision@$1$ for the various sub-module linear shortcut mappings, and the mapping $\matlL{}$ for comparison (\gpt{} next token prediction task).}
\label{fig:parts_pre1}
\end{figure}

\begin{figure}[t]
\centering
\includegraphics[scale=0.35]{figs/parts_surp_24.pdf}
\caption{Surprisal for the various sub-module linear shortcut mappings, and the mapping $\matlL{}$ for comparison (\gpt{} next token prediction task). A 95\% confidence interval surrounds the lines.}
\label{fig:parts_surp}
\end{figure}
}

\paragraph{Results.}
Fig.~\ref{fig:parts_presurp} shows the average Precision@$1$ and Surprisal scores per layer.
From a certain layer (\textasciitilde$7$), all sub-module mappings achieve better results than the full-block mapping $\matlL{}$. Thus, it is not just the cumulative effect of all the sub-modules in the transformer block that is amenable to linear approximation, but also individual sub-modules can be linearly approximated. 
Furthermore, the linear approximation of attention sub-modules is less harmful than that of the FFN or LN sub-modules. 
% Hypothetically, 
A possible reason is that the linear replacement of FFN or LN ``erodes'' the self-attention computation after a few layers. 
Moreover, the good performance of $\matattnlL{}$ suggests that contextualization often exhausts itself in early layers; speculatively, it is only in more delicate cases that the self-attention of late layers adds important information. Last, remark the sharp ascent of the scores for layer normalization in layers $5$-$8$, for which we do not currently see a particular reason. To conclude, we see that the possibility of linear approximation permeates
%the various
transformer components.


\section{Related Work}

Recently, there was a lot of interest in utilizing intermediate representations in transformer-based LMs, both for interpretability and for efficiency.

In the direction of interpretability, one seeks to understand the prediction construction process of the model \cite{tenney-etal-2019-bert, voita-etal-2019-bottom}.

More recent works use mechanistic interpretability and view the inference pass as a residual stream of information \cite{dar2022analyzing,geva-etal-2022-transformer}. Additionally, there are works on probing, attempting to understand what features are stored in the hidden representations \cite{adi2017finegrained, conneau-etal-2018-cram,liu-etal-2019-linguistic}. Our work is different in that it attempts to convert intermediate representations into a final-layer form, which is interpretable by design.

In the direction of efficiency, there is the thread of work on early exit, where computation is cut at a dynamically-decided earlier stage \cite{schwartz-etal-2020-right,xin-etal-2020-deebert,schuster2022confident}. Other works utilize a fixed early stage network to parallelize inference \citep{leviathan2022fast, chen2023accelerating}. However, intermediate representations are directly propagated in these works, which we show is substantially worse than our approach. Moreover, our method requires training considerably less parameters than methods such as \citet{schuster-etal-2021-consistent}, that learn a different output softmax for each intermediate layer.  

More broadly, skipping transformer layers and analyzing the linearity properties of transformer components have been discussed in prior works \cite{Zhao2021of,mickus-etal-2022-dissect,wang-etal-2022-skipbert,lamparth2023analyzing}.


\section{Conclusion and Future Work}

We present a simple and effective method for enhancing utilization of hidden representations in transformer-based LMs, that uses 
pre-fitted context-free and token-uniform linear mappings.
Through a series of experiments on different data sources, model architectures and scales, we show that our method consistently outperforms the prevalent practice of interpreting representations in the final-layer space of the model, yielding better approximations of succeeding representations and the predictions they induce, thus allowing a more faithful interpretation of the model's prediction-formation.
We demonstrate the practicality of our method for improving computation efficiency, saving a substantial amount of compute on top of prominent early exiting approaches. 
Also, by extending our method to sub-modules, 
% more specifically the attention sub-modules, 
we observe that replacing a part of the transformer inference by a non-contextual linear computation often results in a small deterioration of the prediction.
This opens new research directions for improving model efficiency,
% and parallelizability.
% including breaking the computation into several parallelizable tasks.
including breaking the computation into parallel tasks.

\section*{Limitations}

Although we see in this work that there is more linear structure to transformer inference than could be explained solely by the residual connection, we do not elucidate a reason for that. We also do not try to formulate formal criteria according to which to judge, in principle, the quality of ways of short-cutting transformer inference in-between layers. In addition, our experiments cover only English data.


%\section*{Ethics Statement}
%Scientific work published at ACL 2023 must comply with the ACL Ethics Policy.\footnote{\url{https://www.aclweb.org/portal/content/acl-code-ethics}} We encourage all authors to include an explicit ethics statement on the broader impact of the work, or other ethical considerations after the conclusion but before the references. The ethics statement will not count toward the page limit (8 pages for long, 4 pages for short papers).

\section*{Acknowledgements}

We thank Tal Schuster for constructive comments.

% Entries for the entire Anthology, followed by custom entries
\bibliography{anthology,custom}
\bibliographystyle{acl_natbib}

\appendix

\section{Descriptions of $\matattn{}$, $\matff{}$ and $\matln{}$}
\label{sec:app_submodule_skip_description}

Here we detail the definitions of the mappings $\matattnl{}$, $\matffl{}$ and $\matlnl{}$ utilized in \S\ref{sec:submodules}.

\paragraph{Description of $\matattnl{}$.}
%Illustrating this on $\texttt{attn}_\ell$ for definiteness,
For an input $s$, let $v^\ell_{i_s}$ be the vector at position $i_s$ in the output of $\texttt{attn}_\ell (\texttt{ln1}_\ell (H^{\ell - 1}))$. We denote by $A_\ell^{\texttt{attn}} \in \mathbb{R}^{d_h \times d_h}$ the matrix numerically minimizing 
$$ A \mapsto \sum_{s \in \mathcal{T}} || A \cdot \texttt{ln1}_\ell (h^{\ell-1}_{i_s}) - v^\ell_{i_s}||^2,$$
and define an attention sub-module replacement (Eq.~\ref{eq:attn}) by $$
\texttt{b}^{\overline{\texttt{attn}}}_\ell (h) \coloneqq A_{\ell}^{\texttt{attn}} \cdot \texttt{ln1}_\ell (h) + h. $$
We then define a mapping between two layers ${\ell \rightarrow \ell'}$ by:
$$ \matattnl{} (h) \coloneqq $$
$$ \texttt{b}^{\texttt{ffn}}_{\ell'} ( \texttt{b}^{\overline{\texttt{attn}}}_{\ell'} ( \ldots (\texttt{b}^{\texttt{ffn}}_{\ell+1} ( \texttt{b}^{\overline{\texttt{attn}}}_{\ell+1} (h)))\ldots)).$$ 
Namely, when applying each $\ell''$-th block, $\ell < \ell'' \leq \ell'$, we replace its attention sub-module $\texttt{attn}_{\ell''}$ by its linear approximation.
%In an analogous way, we consider the mappings $\matffl{}$ and $\matlnl{}$, where in the latter we perform the linear shortcut both for \texttt{ln1} and for \texttt{ln2} (see~\S\ref{sec:app_submodule_skip_description} for precise descriptions).
Importantly, unlike the original attention module, the approximation $\texttt{b}^{\overline{\texttt{attn}}}_\ell$ operates on each position independently, and therefore applying $\matattnl{}$ disables any contextualization between the layers $\ell$ and $\ell'$. Note that this is not the case for $\matffl{}$ and $\matlnl{}$, which retain the self-attention sub-modules and operate contextually.

\paragraph{Description of $\matffl{}$.}
Let $v^\ell_{i_s}$ be the vector at position $i_s$ in the output of $\texttt{ln2}_{\ell} (\texttt{b}_\ell^{\texttt{attn}} (H^{\ell - 1}))$, for a given input $s$. We denote by $A_\ell^{\texttt{ffn}} \in \mathbb{R}^{d_h \times d_h}$ the matrix numerically minimizing 
$$ A \mapsto \sum_{s \in \mathcal{T}} || A \cdot v^{\ell}_{i_s} - \texttt{ffn}_{\ell} (v^\ell_{i_s})||^2,$$
and define a replacement of the feed-forward sub-module $\texttt{b}_{\ell}^{\texttt{ffn}}$ by $$ \texttt{b}^{\overline{\texttt{ffn}}}_\ell (H) \coloneqq A_{\ell}^{\texttt{ffn}} \cdot \texttt{ln2}_\ell (H) + H.$$
We then define a mapping between two layers ${\ell \rightarrow \ell'}$ by:
$$ \matffl{} (H) \coloneqq $$
$$ \texttt{b}^{\overline{\texttt{ffn}}}_{\ell'} ( \texttt{b}^{\texttt{attn}}_{\ell'} ( \ldots (\texttt{b}^{\overline{\texttt{ffn}}}_{\ell+1} ( \texttt{b}^{\texttt{attn}}_{\ell+1} (H))\ldots)).$$

\paragraph{Description of $\matlnl{}$.}
Let $v^\ell_{i_s}$ be the vector at position $i_s$ in the output of $\texttt{b}^{\texttt{attn}}_{\ell} (H^{\ell - 1})$, for a given input $s$. We denote by $A_\ell^{\texttt{ln1}} \in \mathbb{R}^{d_h \times d_h}$ the matrix numerically minimizing 
$$ A \mapsto \sum_{s \in \mathcal{T}} || A \cdot h^{\ell}_{i_s} - \texttt{ln1}_{\ell} (h^\ell_{i_s})||^2$$ and we denote by $A_\ell^{\texttt{ln2}} \in \mathbb{R}^{d_h \times d_h}$ the matrix numerically minimizing $$ A \mapsto \sum_{s \in \mathcal{T}} || A \cdot v^{\ell}_{i_s} - \texttt{ln2}_{\ell} (v^\ell_{i_s})||^2.$$ We define a replacement of the block $\texttt{b}^{\texttt{attn}}_{\ell}$ by \begin{equation} \texttt{b}^{\overline{\texttt{ln1}}}_\ell (H) \coloneqq \texttt{attn}_{\ell} (A_{\ell}^{\texttt{ln1}} \cdot H) + H\end{equation} and we define a replacement of the block $\texttt{b}^{\texttt{ffn}}_{\ell}$ by \begin{equation} \texttt{b}^{\overline{\texttt{ln2}}}_\ell (H) \coloneqq \texttt{ffn}_{\ell} (A_{\ell}^{\texttt{ln2}} \cdot H) + H.\end{equation}
We then define a mapping between two layers ${\ell \rightarrow \ell'}$ by:
$$ \matlnl{} (H) \coloneqq $$
$$ \texttt{b}^{\overline{\texttt{ln2}}}_{\ell'} ( \texttt{b}^{\overline{\texttt{ln1}}}_{\ell'} ( \ldots (\texttt{b}^{\overline{\texttt{ln2}}}_{\ell+1} ( \texttt{b}^{\overline{\texttt{ln1}}}_{\ell+1} (H))\ldots)).$$


\end{document}




\newpage
\clearpage
\section{Methods}
\subsection*{Light-matter interaction Hamiltonian}
After applying the rotating-wave-approximation, the light-matter interaction Hamiltonian considered in this work is given by~\cite{Steck2007}
\begin{align}
\hat{H}=\frac{\hbar |\Omega|}{2}(|0\rangle\langle 1|e^{i k x} + |1\rangle\langle 0|e^{-i k x}),
\label{eq:lightmatter}
\end{align}
where $\Omega$ is the Rabi frequency, $k$ is the global laser wavevector, $x$ is the atomic position along the beam propagation axis, $|0\rangle$ is the ground state and $|1\rangle$ is the excited state. Atom displacements by $\Delta x$ correspond to phase shifts of $\phi=k\Delta x$, as described in the main text. 

Note the choice of the phase for the initial $\hat{X}(\pi/2)$ rotation is a local gauge freedom, and thus all atoms in the array can be said to experience the rotation about the same local axis, e.g. the $x$-axis, despite the spacing between atoms generically not being perfectly commensurate with the driving wavelength. When the atom is shifted, it can be thought of as changing $\hat{X} \rightarrow \hat{X}\cos(\phi)+i \hat{Y}\sin(\phi)$. For the case of only a single global $\hat{X}$ rotation after the movement, this is equivalent to an effective $\hat{Z}(\phi)$ rotation of the quantum state; however, in general if multiple global $\hat{X}$ operations are performed, the equivalence with an effective $\hat{Z}$ rotation breaks down.

\subsection*{Data analysis}
Discrimination between $|0\rangle=|^1S_0\rangle$ and $|1\rangle=|^3P_0\rangle$ is performed by strongly driving the $^1S_0{\leftrightarrow}^1P_1$ transition for $10\ \mu$s, which heats and ejects all atoms in $^1S_0$. (For details on the strontium level structure see Refs.~\cite{Cooper2018,Stellmer2014}). Atoms in $^3P_0$ are then pumped back into $^1S_0$, and imaged with lower power on the $^1S_0{\leftrightarrow}^1P_1$ transition; imaging is performed in 120 ms. 

For all data in the main text, we interleave~\cite{Choi2023} data-taking with feedback to the global $|0\rangle{\leftrightarrow}|1\rangle$ drive frequency every ${\sim}4\ $minutes to counteract slow (${\sim}40\ $minute period) oscillations arising from environmental drifts. $1\sigma$ error bars in the main text are typically smaller than the marker size in all figures; this includes Fig~\ref{Fig1}def, Fig~\ref{Fig3}c, and Fig.~\ref{Fig4}b.

\subsection*{Operation fidelities}
Error modeling suggests that the global $X(\pi)$ fidelity of 0.9956(1) is primarily limited by: measurement errors (see below), finite temperature (the average motional occupation of the atoms along the radial axis is $\bar{n}{\sim}0.2$, leading to a $2\times10^{-3}$ infidelity)), and frequency noise on our laser; the latter of these we believe is also the dominant limitation to our Ramsey coherence time (Fig.~\ref{Fig3}c). The Rabi frequency is ${\sim}2.5$ kHz for all measurements in this work, which allows for fast operations compared to the timescale of decay from $^3P_0$ (${\sim}$550 ms~\cite{Shaw2023}). Measurement errors are dominated by the vacuum-limited atom survival during imaging (0.9995(4)), the imaging fidelity for detecting the presence of an atom (0.9997(2)), and the likelihood of ejecting atoms in $^1S_0$ from the tweezers to perform state discrimination (0.9967(5)) before the final imaging. We note that not all of these measurement errors contribute equally, and their relative importance generically depends on the amount of excited state population in the measured state.

The wavelength of the oscillation, $\lambda_\textrm{osc}$ in Fig.~\ref{Fig1}d is found from fitting the excited state population of shifted atoms with a sinusoid of the form $A\sin(2\pi \Delta x/\lambda_\textrm{osc})+B$, from which we determine the quoted value of $\lambda_\textrm{osc}=699(1)$ nm. Per our independent calibration of distances within the array, we expect there is an additional ${\sim}5$ nm of systematic uncertainty on this measurement. Further systematic uncertainty could arise if the beam propagation is not perfectly coaxial with the array. The crosstalk fidelity of $0.1(2)\%$ is found by fitting the unshifted atom excited state populations in Fig.~\ref{Fig1}d with a sinusoid of the same period as was determined for the shifted atoms; the quoted crosstalk is then the amplitude, $A$, of this sinusoid. We note that we have also checked  for any residual linear phase shifts by repeating this experiment with a global phase shift of $\pi/2$ on the final Ramsey pulse which yields linear sensitivity to small phase shifts. We find the unshifted atoms phase to still be consistent with zero over a range of more than one wavelength.  

To determine the fidelity of arbitrary local rotations (Fig.~\ref{Fig2}c), we perform quantum state tomography by reading out the produced states in the $x$-, $y$- and $z$- bases by rotating the state with a global $\pi/2$ pulse of a given global laser phase as necessary. The fidelity is then estimated with $F = \langle\psi_{\textrm{target}}|\rho|\psi_{\textrm{target}}\rangle$, where $\rho$ is the experimental state determined by quantum tomography. Due to the choice of the six arbitrary rotations, the fidelity estimation coincides with the population of the excited state or the ground state along the corresponding axis. For instance, the fidelity of $|{+}Y\rangle$ is determined by the population of the excited state when the prepared state is rotated and measured in the $y$-basis. In order to access the intrinsic infidelity induced by arbitrary local rotations, we extract SPAM errors and correct them. SPAM sources are dominated by the same detection infidelities as for the global $\hat{X}(\pi/2)$ fidelity (see above), and the finite SPAM-corrected readout $\pi/2$ pulse fidelity of 0.9982(4). 

\subsection*{Estimating laser phase prior width}
Here, we detail how we find and isolate the prior width of the laser phase distribution from the experimentally measured values. We note that while the phase slip probability depends only on this prior width, the measured distribution is further affected by quantum projection noise (QPN). 
The QPN itself is a function of the fraction of excited atoms measured in the Ramsey sequence.  
One thus finds that the relative contribution of the QPN term varies with the central phase of the laser and the prior width of the phase distribution. Assuming a system of $N=20$ atoms, we plot the calculated phase distribution width as a function of central laser phase $\bar{\theta}$, but with the underlying prior phase distribution having a width of zero (Ext. Data Fig.~\ref{EFig_QPN}(a)). We repeat this calculation for both single and dual quadrature phase estimation.
We note that these estimators are affected slightly differently by QPN. Specifically, while the optimal single quadrature working point in terms of minimal QPN is around a mean phase of 0 (which corresponds to measurement with an excitation fraction of 0.5), we find that the optimal working point for a dual-quadrature estimation is around a mean phase of $\pi/4$ (corresponding to the two quadratures having excitation fractions of 0.85), though we note the dual-quadrature value is relatively flat over the entire bandwidth. For longer interrogation times, the prior width grows as a power law which depends on the laser noise spectrum. The contribution from projection noise is thus in general time-dependent.

To isolate this effect and learn the true laser phase distribution as a function of interrogation time, we first calculate the total observable width including QPN, $\sigma_{tot}$, for a range of prior widths $\sigma_{\delta}$ at a given laser central phase $\bar{\theta}$. This is done by sampling random phases from a normal distribution, followed by sampling the observed phase from a binomial random process representing the projection uncertainty. Repeating this process over a million draws we obtain the observed distribution as a function of the prior width (see Extended Data Fig. \ref{EFig_QPN}(b)). We then invert the function to obtain $\sigma_{\delta}$ ($\sigma_{tot}$) and interpolate the latter to find the prior width at the given measured $\sigma_{tot}$.

%While the details of added noise are important for the attainable stability, we note that quantum projection noise itself can not cause a phase slip and thus does not limit the interrogation time.

\subsection*{Fitting the phase-slip probability}
To find the phase deviation from the mean phase for a given shot, we fit the Ramsey oscillations in Fig.~\ref{Fig3}c with a decaying sinusoid, and at each time define the fitted mean populations as $\bar{P}^{(x)}(t)$ and $\bar{P}^{(y)}(t)$. These populations are inverted via Eq.~\ref{eq:phaseinversion} into the mean phase, $\bar{\theta}(t)$, and finally we calculate the phase deviation from the mean as \mbox{$\delta_j(t)=\textrm{mod}(\theta_j(t)-\bar{\theta}(t),\pi)$}. 

For the phase-slip probability (Fig.~\ref{Fig3}e), we fit the probability densities $\mathcal{P}(\delta_j(t))$ with $G(\mathcal{P})$, where $G$ is a Gaussian distribution folded into the range $[-\pi,\pi]$. This fit provides an estimation of the true standard deviation, $\sigma$,  of the $\delta_j(t)$ distribution. With this in hand, for a given half-dynamic range, $B$, we then find the phase-slip probability, $\epsilon$, as 
\begin{align}
\epsilon=2\int_B^\infty G(\mathcal{P})d\mathcal{P}=\textrm{erfc}\Big(\frac{B}{\sqrt{2}\sigma}\Big),
\label{eq:phaseslipprobability}
\end{align}
where $\textrm{erfc}$ is the complementary error function. Note that here $B=\pi/2$ corresponds to a single-basis measurement while $B=\pi$ corresponds to a dual-quadrature measurement.

In order to calculate the maximal interrogation time, we first find the laser phase prior width (as described in the previous section) and then fit the time-resolved profile of $\sigma(t)$, as $\sigma(t)=\beta t^\alpha$. We find $\beta=\pi\times0.119(6)$ and $\alpha=0.56(2)$. The growth of $\sigma$ over time can then be used to predict the Ramsey decay envelope, $C$, for different choices of $B$, as $C=e^{-\sigma^2/2}$. In Fig.~\ref{Fig3}c, we show this envelope estimation for the cases of $B=\pi/2$ and $B=\pi$ (orange and green dashed lines, respectively).

We then analytically calculate the maximum interrogation time at a fixed phase-slip error probability, $\epsilon$, from Eq.~\ref{eq:phaseslipprobability} as
\begin{align}
T_{\textrm{max}}(\epsilon)=(B/(\sqrt{2}\beta \textrm{erfc}^{-1}(\epsilon)))^{1/\alpha}.
\end{align} 

\subsection*{Limits in multi-ensemble metrology}

To study the possible limitations of the multi-ensemble scheme we simulate stochastic phase evolution of a local oscillator with $1/f$ frequency noise, whose overall power sets a characteristic single ensemble $1/e$ Ramsey coherence time $\mathrm{T}_{\mathrm{LO}}$. We numerically find the maximal interrogation time at a fixed phase slip probability $\epsilon$ (here we use $\epsilon=5\cdot10^{-3}$) with increasing ensemble number $M$ by iteratively correcting for phase slips as described in Ref. \cite{Rosenband2013}. We repeat this calculation for different dynamical decoupling block lengths $\tau$.
In Ext. Data Fig.~\ref{EFig_MultiEns}a we plot the results for up to $M=9$ ensembles, assuming infinite atom number. We find that adding more ensembles indeed enables exponential scaling of the interrogation time up to a saturation point set by the effective dynamical decoupling bandwidth (expressed in terms of $\tau/\mathrm{T_{LO}}$).

We further study the effect of quantum projection noise on the efficacy of the scheme in the case of low atom number per ensemble. For the optimal dynamical decoupling sequence found previously, we vary the number of atoms per ensemble $N$ and repeat the calculation, which is now affected by quantum projection noise. Specifically, the use of slower evolving ensembles with limited atom number for the iterative correction of phase slips in the fastest evolving ensembles is prone to errors due to the increased variance in such estimation.  
For a small number of atoms per ensemble, this negates any advantage. However, we note that $N\simeq20$ atoms per ensemble suffice for efficient operation of the scheme, with the interrogation times and noise strength tested here.    
 

\FloatBarrier
\clearpage
\newpage
\setcounter{figure}{0}
%\renewcommand\thefigure{S\arabic{figure}}

\captionsetup[figure]{labelfont={bf},name={Ext. Data Fig.},labelsep=bar,justification=raggedright,font=small}

\section*{Extended Data Figures}

\begin{figure}[ht!]
	\centering
	\includegraphics[width=\columnwidth]{ED/FigS1.pdf}
\caption{\textbf{Quantum projection noise in a dual-quadrature measurement.} 
\textbf{a,} Added standard deviation due to quantum projection noise (QPN) for phase estimation around different average phases $\bar\theta$, plotted for $N=20$ atoms utilized in a single-quadrature (orange) or dual-quadrature (green, 10 atoms per quadrature) measurement. The added QPN varies with the phase the measurement is taken at; thus as the prior width of the phase distribution grows over time, and a broader range of phases is sampled, the QPN will vary. \textbf{b,} To learn the prior width from the measured width we sample random phases from a normal distribution, followed by sampling the observed phase from a binomial random process representing the projection uncertainty (inset). We use the sampled distributions for the dual-quadrature estimator to calculate the width including QPN for a range of prior laser widths. We then invert this function and interpolate if needed, to find the prior width for any measured width.}  

	\label{EFig_QPN}

 \end{figure}

\begin{figure}[ht!]
	\centering
     \includegraphics[width=\columnwidth]{ED/FigS2.pdf}
\caption{\textbf{Limits in multi-ensemble metrology.}  \textbf{a,} Asymptotic scaling of the extended interrogation time factor as a function of the number of ensembles M employed. We numerically calculate the maximal interrogation time at a fixed phase slip probability for different dynamical decoupling block lengths $\tau$, in terms of the local-oscillator coherence time $\mathrm{T}_\mathrm{LO}$, assuming $1/f$ frequency noise and infinite atom number. The addition of more ensembles enables exponential scaling of the maximal interrogation time (solid line marks $2^{M-1}$) up to a saturation point set by the effective decoupling bandwidth. The latter can be extended by reducing the block length while maintaining a sufficiently high Rabi frequency with respect to the fast noise frequency. \textbf{b,} For the optimal decoupling sequence found in \textbf{a}, we plot the extended interrogation time as a function of the number of atoms per ensemble $\mathrm{N}$. For a small number of atoms per ensemble, quantum projection noise results in an enhanced rate of false positive indication of a phase slip, negating any advantage. We find that $\mathrm{N}\simeq20$ atoms per ensemble suffice for efficient operation.}
	\label{EFig_MultiEns}

 \end{figure}




\FloatBarrier
\newpage


\end{document}


















