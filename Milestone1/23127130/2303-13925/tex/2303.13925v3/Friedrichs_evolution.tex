\def\sCl{-1.5} %x-shift of classical terms
\def\sQ{6.7} %x-shift of classical terms
\def\r{0.55} %radius of the 1st internal vertex
\def\rB{0.8} %radius of the 2nd internal vertex
\def\rC{0.5} %radius of the 3rd internal vertex
\def\extl(#1, #2){\fill (#1, #2) circle (0.05); \fill[opacity = 0.3, blue] (#1, #2) circle (0.1);  } %external vertex to the left (x,y)
\def\extr(#1, #2){\fill (#1, #2) circle (0.05); \fill[opacity = 0.3, green!50!black] (#1, #2) circle (0.1);  } %external vertex to the right (x,y)
\def\conn(#1, #2, #3, #4){({#4*cos(#1) + #2},{#4*sin(#1) + #3}) --  ({(#4+0.2)*cos(#1) + #2},{(#4+0.2)*sin(#1) + #3})} %connector (phi, x, y, r)
\def\connt(#1, #2, #3, #4, #5){({#4*cos(#1) + #2},{#4*sin(#1) + #3}) .. controls  ({(#4+#5)*cos(#1) + #2},{(#4+#5)*sin(#1) + #3}) and} %connector trace point (phi, x, y, r, delta r)
\def\connte(#1, #2, #3, #4, #5){ ({(#4+#5)*cos(#1) + #2},{(#4+#5)*sin(#1) + #3}) .. ({#4*cos(#1) + #2},{#4*sin(#1) + #3}) } %entering connector trace point (phi, x, y, r, delta r)
\begin{tikzpicture}
\useasboundingbox (-2,-1) rectangle (4.5,1);


%external legs

\draw[thick] \connt(150, \sCl-2, 2, (\r+0.2), 0.8) ++(0.2,0) .. (\sCl-5.5, 3); \extl(\sCl-5.5, 3)
\draw[thick] \connt(150, \sCl-4.2, 1, (\r+0.2), 0.2) ++(0.2,0) .. (\sCl-5.5, 1+0.5); \extl(\sCl-5.5, 1+0.5)
\draw[thick] \connt(210, \sCl-4.2, 1, (\r+0.2), 0.2) ++(0.2,0) .. (\sCl-5.5, 1-0.5); \extl(\sCl-5.5, 1-0.5)
\draw[thick] \connt(250, \sCl+0.2, 1, (\r+0.2), 1.7) ++(0.2,0) .. (\sCl-5.5, -0.8); \extl(\sCl-5.5, -0.8)

\draw[thick] \connt(30, \sCl+0.2, 1, (\r+0.2), 0.2) ++(0.2,0) .. (\sCl+1.5, 1+0.5); \extr(\sCl+1.5, 1+0.5)
\draw[thick] \connt(-30, \sCl+0.2, 1, (\r+0.2), 0.2) ++(0.2,0) .. (\sCl+1.5, 1-0.5); \extr(\sCl+1.5, 1-0.5)
\draw[thick] \connt(30, \sCl-2, 2, (\r+0.2), 0.8) ++(0.2,0) .. (\sCl+1.5, 3); \extr(\sCl+1.5, 3)
\draw[thick] \connt(-30, \sCl-2, 2, (\r+0.2), 0.7) ++(0.2,0) .. (\sCl+1.5, 2.2); \extr(\sCl+1.5, 2.2)



\draw[thick] \connt(150, \sQ-2, 2, (\r+0.2), 0.8) ++(0.2,0) .. (\sQ-5.5, 3); \extl(\sQ-5.5, 3)
\draw[thick] \connt(150, \sQ-4.2, 1, (\r+0.2), 0.2) ++(0.2,0) .. (\sQ-5.5, 1+0.5); \extl(\sQ-5.5, 1+0.5)
\draw[thick] \connt(210, \sQ-4.2, 1, (\r+0.2), 0.2) ++(0.2,0) .. (\sQ-5.5, 1-0.5); \extl(\sQ-5.5, 1-0.5)

\draw[thick] \connt(30, \sQ+0.2, 1, (\r+0.2), 0.2) ++(0.2,0) .. (\sQ+1.5, 1+0.5); \extr(\sQ+1.5, 1+0.5)
\draw[thick] \connt(-30, \sQ+0.2, 1, (\r+0.2), 0.2) ++(0.2,0) .. (\sQ+1.5, 1-0.5); \extr(\sQ+1.5, 1-0.5)
\draw[thick] \connt(30, \sQ-2, 2, (\r+0.2), 0.8) ++(0.2,0) .. (\sQ+1.5, 3); \extr(\sQ+1.5, 3)



%contractions
\draw[red, opacity = .8, line width = 1] \connt(0, \sCl-2, 0, (\r+0.2), 0.5) \connte(150, \sCl+0.2, 1, (\r+0.2), 0.5);
\draw[red, opacity = .8, line width = 1] \connt(30, \sCl-4.2, 1, (\r+0.2), 0.5) \connte(210, \sCl-2, 2, (\r+0.2), 0.5);
\draw[red, opacity = .8, line width = 1] \connt(-30, \sCl-4.2, 1, (\r+0.2), 0.5) \connte(180, \sCl-2, 0, (\r+0.2), 0.5);



\draw[red, opacity = .8, line width = 1] \connt(0, \sQ-2, 0, (\r+0.2), 0.5) \connte(150, \sQ+0.2, 1, (\r+0.2), 0.5);
\draw[red, opacity = .8, line width = 1] \connt(30, \sQ-4.2, 1, (\r+0.2), 0.5) \connte(210, \sQ-2, 2, (\r+0.2), 0.5);
\draw[red, opacity = .8, line width = 1] \connt(-30, \sQ-4.2, 1, (\r+0.2), 0.5) \connte(180, \sQ-2, 0, (\r+0.2), 0.5);
\draw[red, opacity = .8, line width = 1] \connt(-30, \sQ-2, 2, (\r+0.2), 1) \connte(250, \sQ+0.2, 1, (\r+0.2), 0.5);



%internal vertex 0 (Classical)
\filldraw[fill = yellow!50!white, thick] (\sCl-2,0) circle (\r) node{$A$} ;

\draw[line width = 2, red!50!blue] \conn(0, \sCl-2, 0, \r);

\draw[line width = 2, red!50!blue] \conn(180, \sCl-2, 0, \r);

%internal vertex 0 (Quantum)
\filldraw[fill = yellow!50!white, thick] (\sQ-2,0) circle (\r) node{$A$} ;

\draw[line width = 2, red!50!blue] \conn(0, \sQ-2, 0, \r);

\draw[line width = 2, red!50!blue] \conn(180, \sQ-2, 0, \r);


%internal vertex 1 (Classical)
\filldraw[fill = yellow!50!white, thick] (\sCl-4.2,1) circle (\r) node{$V^{t_1}$} ;
\draw[line width = 2, red!50!blue] \conn(-30, \sCl-4.2, 1, \r);
\draw[line width = 2, red!50!blue] \conn(30, \sCl-4.2, 1, \r);

\draw[line width = 2, red!50!blue] \conn(150, \sCl-4.2, 1, \r);
\draw[line width = 2, red!50!blue] \conn(210, \sCl-4.2, 1, \r);

%internal vertex 1 (Quantum)
\filldraw[fill = yellow!50!white, thick] (\sQ-4.2,1) circle (\r) node{$V^{t_1}$} ;
\draw[line width = 2, red!50!blue] \conn(-30, \sQ-4.2, 1, \r);
\draw[line width = 2, red!50!blue] \conn(30, \sQ-4.2, 1, \r);

\draw[line width = 2, red!50!blue] \conn(150, \sQ-4.2, 1, \r);
\draw[line width = 2, red!50!blue] \conn(210, \sQ-4.2, 1, \r);


%internal vertex 2 (Classical)
\filldraw[fill = yellow!50!white, thick] (\sCl-2,2) circle (\r) node{$V^{t_2}$} ;
\draw[line width = 2, red!50!blue] \conn(-30, \sCl-2, 2, \r);
\draw[line width = 2, red!50!blue] \conn(30, \sCl-2, 2, \r);

\draw[line width = 2, red!50!blue] \conn(150, \sCl-2, 2, \r);
\draw[line width = 2, red!50!blue] \conn(210, \sCl-2, 2, \r);

%internal vertex 2 (Quantum)
\filldraw[fill = yellow!50!white, thick] (\sQ-2,2) circle (\r) node{$V^{t_2}$} ;
\draw[line width = 2, red!50!blue] \conn(-30, \sQ-2, 2, \r);
\draw[line width = 2, red!50!blue] \conn(30, \sQ-2, 2, \r);

\draw[line width = 2, red!50!blue] \conn(150, \sQ-2, 2, \r);
\draw[line width = 2, red!50!blue] \conn(210, \sQ-2, 2, \r);


%internal vertex 3 (Classical)
\filldraw[fill = yellow!50!white, thick] (\sCl+0.2,1) circle (\r) node{$V^{t_3}$} ;
\draw[line width = 2, red!50!blue] \conn(-30, \sCl+0.2, 1, \r);
\draw[line width = 2, red!50!blue] \conn(30, \sCl+0.2, 1, \r);

\draw[line width = 2, red!50!blue] \conn(150, \sCl+0.2, 1, \r);
\draw[line width = 2, red!50!blue] \conn(250, \sCl+0.2, 1, \r);

%internal vertex 3 (Quantum)
\filldraw[fill = yellow!50!white, thick] (\sQ+0.2,1) circle (\r) node{$V^{t_3}$} ;
\draw[line width = 2, red!50!blue] \conn(-30, \sQ+0.2, 1, \r);
\draw[line width = 2, red!50!blue] \conn(30, \sQ+0.2, 1, \r);

\draw[line width = 2, red!50!blue] \conn(150, \sQ+0.2, 1, \r);
\draw[line width = 2, red!50!blue] \conn(250, \sQ+0.2, 1, \r);










\end{tikzpicture}