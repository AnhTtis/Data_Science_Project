\documentclass[10pt,twocolumn,letterpaper]{article}

\usepackage{iccv}
\usepackage{times}
\usepackage{epsfig}
\usepackage{graphicx}
\usepackage{amsmath}
\usepackage{amssymb}
\usepackage{booktabs}
\usepackage{color}
\usepackage{subfigure}
\usepackage{multirow}
\usepackage{makecell}
\usepackage{bbding}
\usepackage{xcolor}
\usepackage{bm}

\newcommand{\jj}[1]{\textcolor{red}{\small{\bf [JJ: #1 ]}}}
\newcommand{\yan}[1]{\textcolor{blue}{\small{\bf [yan: #1 ]}}}
\newcommand{\wz}[1]{\textcolor{orange}{\small{\bf [wz: #1 ]}}}
\newcommand{\siqi}[1]{\textcolor{cyan}{\small{\bf [siqi: #1 ]}}}

% Include other packages here, before hyperref.

% If you comment hyperref and then uncomment it, you should delete
% egpaper.aux before re-running latex.  (Or just hit 'q' on the first latex
% run, let it finish, and you should be clear).
\usepackage[pagebackref=true,breaklinks=true,letterpaper=true,colorlinks,bookmarks=false]{hyperref}
\iccvfinalcopy % *** Uncomment this line for the final submission

\def\iccvPaperID{8867} % *** Enter the ICCV Paper ID here
\def\httilde{\mbox{\tt\raisebox{-.5ex}{\symbol{126}}}}

% Pages are numbered in submission mode, and unnumbered in camera-ready
\ificcvfinal\pagestyle{empty}\fi

\begin{document}

%%%%%%%%% TITLE
\title{VIMI: Vehicle-Infrastructure Multi-view Intermediate Fusion \\for Camera-based 3D Object Detection}


\author{Zhe Wang$^{1}$, Siqi Fan$^{1}$, Xiaoliang Huo$^{1,2}$, Tongda Xu$^{1}$, Yan Wang$^{1}$\thanks{Corresponding authors.} , Jingjing Liu$^{1}$, Yilun Chen$^{1}$, Ya-Qin Zhang$^{1}$\footnotemark[1]\\
$^{1}$ Institute for AI Industry Research (AIR), Tsinghua University, Beijing, China \\
$^{2}$ Beihang University, Beijing, China \\
{\tt\small \{wangzhe, fansiqi, wangyan\}@air.tsinghua.edu.cn}
}

\maketitle
% Remove page # from the first page of camera-ready.
\ificcvfinal\thispagestyle{empty}\fi

\renewcommand{\thefootnote}{\fnsymbol{footnote}}
\footnotetext[2]{Code will be made publicly available at the \href{https://github.com/Bosszhe/VIMI}{link}.}
%%%%%%%%% ABSTRACT
\begin{abstract}


In autonomous driving, Vehicle-Infrastructure Cooperative 3D Object Detection (VIC3D) makes use of multi-view cameras from both vehicles and traffic infrastructure, providing a global vantage point with rich semantic context of road conditions beyond a single vehicle viewpoint. 
Two major challenges prevail in VIC3D: $1)$ inherent calibration noise when fusing multi-view images, caused by time asynchrony across cameras;
$2)$ information loss when projecting 2D features into 3D space.
To address these issues, We propose a novel 3D object detection framework, Vehicles-Infrastructure Multi-view Intermediate fusion (VIMI).
First, to fully exploit the holistic perspectives from both vehicles and infrastructure, we propose a Multi-scale Cross Attention (MCA) module that fuses infrastructure and vehicle features on selective multi-scales to correct the calibration noise introduced by camera asynchrony. Then, we design a  Camera-aware Channel Masking (CCM) module that uses camera parameters as priors to augment the fused features. We further introduce a Feature Compression (FC) module with channel and spatial compression blocks to reduce the size of transmitted features for enhanced efficiency.
Experiments show that VIMI achieves 15.61\% overall $AP_{\text{3D}}$ and 21.44\% $AP_{\text{BEV}}$ on the new VIC3D dataset, DAIR-V2X-C, significantly outperforming state-of-the-art early fusion and late fusion methods with comparable transmission cost.

\end{abstract}
\section{Introduction}


3D object detection is one of the most important environmental perception tasks for autonomous driving (AD). 
Subject to sensor limitations, autonomous vehicles lack a global perception capability for monitoring holistic road conditions and accurately detecting surrounding objects, which bears great safety risks. Vehicle-to-everything (V2X) aims to build a communication system between vehicles and other devices in a complex traffic environment. V2X can further enlarge the perception range of a single vehicle and enables detection for blind areas.



% Most existing research on VIC3D has focused on LiDAR-based methods due to the fusion convenience and the performance advantage. While cameras are widely used in autonomous driving due to their ability to provide dense semantic information about the surrounding environment, we turn to explore camera-based VIC3D in this paper, considering the deployment and transmission cost in practical applications.



% Due to the gap between 2D image plane and 3D space, image fusion can not be as direct as point clouds. Few approaches focus on IF methods for cameras, especially in real scenarios, while cameras can provide extra and complementary visual information compared to point clouds, making it essential for further V2X studies.


\begin{figure}[t]
	\centering  
	\includegraphics[width=\linewidth]{gt_veh_inf.png} 
	\caption{Labels (3D bounding boxes) projected from 3D space to vehicle (a)  and infrastructure (b) image planes often suffer from misalignment between the ground truth and the projection position in 2D images (as illustrated by the misaligned green bounding boxes), because calibration noise inherently exists in the joint-labeling of different views in VIC3D datasets.}  
	\label{fig:calib_noise}   
\end{figure}


% \begin{figure}[t]
% {
% \centering
% \subfigure[Vehicle view] {
%  \label{gt_veh}
%  \includegraphics[width=0.45\columnwidth]{Figures/019940_gt_veh.png}
% }
% \hspace{0.1in}
% \subfigure[Infrastructure view] {
% \label{gt_inf}
% \includegraphics[width=0.45\columnwidth]{Figures/004105_gt_inf.png}
% }
% \caption{Labels (3D bounding boxes) projected from 3D space to vehicle (a)  and infrastructure (b) image planes often suffer from misalignment between the ground truth and the projection position in 2D images (as illustrated by the misaligned green bounding boxes), because calibration noise inherently exists in the joint-labeling process of VIC3D datasets from different views.
% %\jj{It's hard to tell where the misalignment is from the two images. Need to explain the relative views of the two images, and better illustrate/highlight the contrast and misalignment}
% % \jj{Make the green boxes lines thicker}
% % \jj{Brighter and bolder green lines in (a)}
% }}
% \label{calib_noise}
% \end{figure}




% In contrast to single-vehicle 3D object detection tasks, VIC3D leverages images captured from both vehicle and infrastructure, naturally resulting in cross-agent perception challenges, like temporal asynchrony, inaccurate relative position (calibration noise), and transmission cost. Therefore, fusion methods to tackle such cross-agent perception challenges are the key to VIC3D.

% However, multi-view 2D images can hardly be fused at the early stage like point clouds because of the lack of precise depth.

% but this result-level fusion highly relies on accurate extrinsic and intrinsic parameters, which is not guaranteed in VIC3D task due to time asynchrony and calibration noise (as shown in Figure~\ref{gt_inf}). 

%LF method is sensitive to calibration so that even when prediction from the infrastructure side is perfect, the vehicle will receive biased 3D detection.
% Compared with raw data or output predictions, features are more flexible for transmission, since it is high-level representations that can be further compressed to reduce transmission cost and dynamically enhanced to tackle noises. 



Existing public V2X datasets are mostly simulated, such as OPV2V~\cite{xu2022opv2v}, V2X-Sim~\cite{li2022v2xsim} and V2XSet~\cite{xu2022v2xvit}. Most existing research on V2X has focused on LiDAR-assisted methods, including \textit{early fusion} (EF) of raw signals~\cite{yu2022dairv2x,hu2022where2comm,chen2022co3}, \textit{intermediate fusion} (IF) of features~\cite{mehr2019disconet,xu2022opv2v,wang2020v2vnet}, and \textit{late fusion} (LF) of prediction outputs~\cite{yu2022dairv2x,chen2022model-agnostic}. Most recent research~\cite{yu2022dairv2x,chen2022model-agnostic} adopts a late-fusion method based on 3D predictions of each monocular detector (e.g., 3D bounding boxes from the camera and LiDAR). When considering intermediate fusion, prior methods~\cite{wang2020v2vnet,mehr2019disconet,hu2022where2comm,xu2022v2xvit} have mainly focused on additional features extracted from simulated point clouds in Vehicle-to-Vehicle (V2V) scenarios. 
As LiDAR is highly expensive and difficult to deploy in each vehicle in practical applications, an alternative solution is Vehicle-to-Infrastructure (V2I), in which case standard cameras are installed in shared traffic environment providing a holistic view of road conditions. Due to the lack of real V2I infrastructure and publicly available data, few studies have investigated such a vehicle-infrastructure camera fusion problem.
Recently, DAIR-V2X~\cite{yu2022dairv2x} proposed Vehicle-infrastructure cooperative 3D object detection (VIC3D) task and released new benchmarks using point clouds and camera images from real scenarios. These datasets contain real data with roadside cameras complimenting single vehicle viewpoint, which provides a broader perception range that better captures vehicle blind spots. The baseline method in~\cite{yu2022dairv2x} relies on late fusion by combining prediction outputs from each camera. 
%, where the features from each view are similar. \jj{What does it mean features are similar? Features from different vehicles are not similar.} 
%Cameras can provide dense semantic information and are less expensive and easier to deploy in practical applications compared with LiDAR, while few research has studied the intermediate fusion in vehicle-to-infrastructure (V2I) scenarios because of the huge view gap between the two sides.  
% \jj{Explain why you don't consider LiDAR in this work.} 







In this paper, we propose a novel framework for this new VIC3D task, \textit{Vehicle-Infrastructure Multi-view Intermediate fusion} (VIMI).
We choose intermediate instead of late fusion, as the latter highly relies on accurate values of extrinsic and intrinsic camera parameters. This is not guaranteed in VIC3D task, as there exists an inherent temporal asynchrony caused by transmission delay and calibration noises between the vehicle and infrastructure. As shown in Figure~\ref{fig:calib_noise}, this time asynchrony and calibration error can result in inaccurate relative position detection. 
%, and high transmission cost further aggravates this problem. 
By focusing on feature-level fusion between vehicle and infrastructure cameras, high-dimensional features extracted from raw data can be compressed and transmitted, which can be used 
%to save transmission costs and mitigate time asynchrony. Latent representations can also be adjusted and augmented dynamically 
to alleviate the negative effect of calibration noises.
%\wz{ IF methods transmit features extracted from raw data, which can be easily compressed to save transmission costs,features are latent representations of information and can be adjusted and augmented dynamically to alleviate the negative effect of calibration noises.}
%\jj{Doesn't this time asynchrony also appears between cameras and LiDAR in general V2X?}
%\jj{This is the key paragraph to explain your motivation. Need to articulate why your approach can solve asynchrony problem while existing methods can't, and why you didn' use LiDAR as other works did.}

% When considering only multi-view camera-images, 2D images cannot be easily fused at early stage either, due to the lack of precise depth (unlike point clouds). 
%Whereas intermediate fusion, compared with using raw data (EF) or output predictions (LF), utilizes intermediate features that are more flexible for transmission, which as high-dimensional representations can be further compressed to reduce transmission cost and dynamically enhanced to tackle noises. 

Specifically, VIMI includes a Feature Compression (FC) module which compresses 2D features transmitted from the infrastructure to vehicle to alleviate transmission delay. Then, considering the same object can be captured by sensors from both vehicle and infrastructure at different distances, we introduce a Multi-scale Cross Attention (MCA) module to attentively fuse multi-scale features according to feature scale correlations between vehicle and infrastructure. To correct calibration errors born from multiple cameras, features from both infrastructure and vehicle are further enhanced by a  Camera-aware Channel Masking (CCM) module via a learned channel-wise mask following guidance of camera priors (intrinsic and extrinsic parameters). Then, the refined features are transformed into voxel features leveraging calibration parameters and projected into 3D space. Finally splatted into BEV space, the fused feature is fed into detection heads for object detection. For evaluation purposes, we have built a new multi-view camera fusion benchmark on the latest DAIR-V2X dataset. Experiments demonstrate the effectiveness of each VIMI module in reducing calibration error and achieving better prediction accuracy than existing EF and LF methods. 

Our contributions can be summarized as follows:



% IF method has two advantages compared with EF and LF. First, IF methods transmit features extracted from raw data, which can be easily compressed to save transmission costs. Meanwhile, features are latent representations of information and can be adjusted and augmented dynamically to alleviate the negative effect of calibration noises. In the VIC system, the same object can be observed at different distances by cameras on vehicle and infrastructure, which indicates that different scale features contain information of different importance for fusion. For feature fusion, camera parameters can be regarded as priors to help the model augment features. The design of IF method needs to consider letting the network aware of differences among multi-scale features and camera parameters.


% In the proposed method, VIMI compresses the infrastructure feature with FC module to reduce the transmission cost, which is helpful to decrease time asynchrony in practical applications. MCA module refines multi-scale image features with MS Block considering calibration noises and applies cross-attention to the correlation between multi-scale infrastructure features and vehicle features. In addition, CCM module reweights image features for better detection performance with a camera-aware channel mask generated from camera parameters. The augmented image features are then transformed into 3D space with calibration and are used to generate voxel features, which are friendly to fusion. Finally, the fused voxel feature can be splatted into BEV feature so that 2D detection head can be applied for prediction. 


\begin{itemize}
    \item We propose VIMI, a novel framework for multi-view 3D object detection, the first intermediate-fusion method for camera-based VIC3D task. %which can fuse information from different views at feature level.

    \item We design MCA and CCM modules to dynamically augment image features for better detection performance, with an additional FC module to reduce transmission costs in VIC3D system.

    \item %We compared \wz{remove benchmark}
  %  We construct a multi-view camera-image fusion benchmark on DAIR-V2X-C dataset and 
  We achieve state-of-the-art results on DAIR-V2X-C dataset, the latest VIC3D benchmark with real data, where VIMI outperforms existing LF and EF methods with comparable transmission costs. 


\end{itemize}

% DRAFT
% Due to the low cost for deployment and rich semantic information, cameras own desirable advantages to 3D object detection. While camera-based 3D detection is still an ill-posed task because RGB images don't contain geometry information about the scene. 
% RGB images with a lack of precise depth can't be fused directly in raw data domain, which is the way of Early Fusion (EF) for point clouds. Several Intermediate Fusion (IF) works have fused features obtained from multi-view images. However, these methods have not been verified in Vehicle-Infrastructure Cooperative (VIC) scenarios. The most direct way is Late Fusion (LF), which fuses predictions like 3D bounding boxes obtained from each monocular camera independently. 

% There are joint labeling errors in vehicle-infrastructure cooperative annotations caused by calibration noises and time asynchrony (more detail can be observed in Figure~\ref{gt_inf}. The 3D bounding boxes of cars have been projected to the image plane of the infrastructure camera but have a displacement relative to their true positions in the image). 


% According to these assumptions, three questions about the design of IF-based multi-view camera fusion are needed to answer: 1) How to alleviate the negative effect of calibration noise? 2) Which feature level should be used to fuse? 3) How to use camera parameters as priors to augment extracted features? 

% \begin{figure}[htbp]
% {
% \centering
% \subfigure[Vehicle image] {
%  \label{fig6_4}
%  \includegraphics[width=0.48\columnwidth]{Figures/015372_veh_015372.png}
% }
% \hspace{0.1in}
% \subfigure[Infrastructure image] {
% \label{fig6_5}
% \includegraphics[width=0.48\columnwidth]{Figures/015372_inf_000016.png}
% }
% \caption{Infrastructure \& Vehicle image in DAIR-V2X Dataset.}}
% \label{fig6_45}
% \end{figure}




\section{Related Work}

\subsection{V2X Cooperative Perception}

Current research on V2X cooperative perception mainly focuses on simulated datasets, such as OPV2V~\cite{xu2022opv2v}, V2X-Sim~\cite{li2022v2xsim} and V2XSet~\cite{xu2022v2xvit}. Existing intermediate-fusion methods focused on simulated point clouds, such as V2VNet~\cite{wang2020v2vnet} transmitted compressed features to nearby vehicles and generated joint perception/prediction. DiscoNet~\cite{mehr2019disconet} introduced graphs into feature fusion and proposed edge weights to highlight different informative regions during feature propagation. Recent Where2comm~\cite{hu2022where2comm} considered the spatial confidence of features and selected features with high confidence and complementary to others, which effectively saves transmission costs. 
Different from point clouds, images from vehicle and infrastructure have a huge view gap, thus features need to be transformed into unified space for fusion. One direct way for fusing multi-view images is late fusion, such as DAIR-V2X~\cite{yu2022dairv2x}, which proposed a result-level fusion model for cameras with separate detectors~\cite{rukhovich2022imvoxelnet}. Few approaches have focused on IF methods for cameras, especially in real scenarios. 
%, while cameras can provide extra and complementary visual information compared to point clouds, making it essential for further V2X studies.

\begin{figure*}[ht]
	\centering  
	\includegraphics[width=\linewidth]{VIMI_architecture.png} 
	\caption{The general framework of VIMI. Separate image backbone and neck extract multi-scale image feature from vehicle and infrastructure images. FC module compresses source infrastructure feature $f^{S}_{inf}$ and decompresses it to multi-scale ones $f^{M}_{inf}$. MCA module augments features $f^{M}_{veh/inf}$ by seeking the correlation between the two sides, and CCM takes camera parameters $(R,t,K)$ as input to reweight features $f_{veh/inf}$ with channel relationship. Finally, Point-Sampling Voxel Fusion projects image features $f^{\prime}_{veh/inf}$ into 3D space to generate a unified voxel feature $V_{vic}$, which can be applied to 3D neck and head in turn for detection prediction. 
 % \jj{Need prettier figure}\wz{what can be improved? color needs to be changed? make icon bigger?} \jj{Module names don't match (FSC, MSCA, etc.) Symbols are not explained (R, t ,K, N, C, I, etc.) Arrows are not explained (what each process is for, what's the input/output). Legends (CA, MP, CCM, etc.) are cramped together with the process figure. }
 % \jj{Add () to dimensions (N * N ...)}
 }  
	\label{fig:framework}   
\end{figure*}

\subsection{Multi-View Camera Fusion}

%Muti-view camera fusion methods can be summarized into three main categories: direct prediction, lift-based, and projection-based methods.

\textbf{Direct Prediction} methods extract image features with object query~\cite{wang2022detr3d,chen2022futr3d,liu2022petr,liu2022petrv2} or directly on front-view image~\cite{wang2021fcos3d}. DETR3D~\cite{wang2022detr3d} used a sparse set of 3D object queries to sample 2D multi-view image features and predicted 3D bounding boxes with set-to-set loss. PETR~\cite{liu2022petr,liu2022petrv2} transformed image features into 3D position-aware representation by encoding 3D coordinates into position embedding. FCOS3D~\cite{wang2021fcos3d} transformed 3D labels to front-view images and directly predicted 3D information by extending FCOS~\cite{tian2019fcos} to 3D detection.

\textbf{Lift-based} methods project features from image plane to BEV  (bird's eye view) plane through depth estimation. Most methods~\cite{huang2021bevdet,huang2022bevdet4d,xie2022m2bev,zhang2022beverse,reading2021caddn} applied 2D-to-3D transformation following LSS~\cite{philion2020lss}, which predicted a depth distribution for each pixel and lifted image features into frustum features with camera parameters, then splatted all frustums into a rasterized BEV feature. BEVDepth~\cite{li2022bevdepth} claimed the quality of intermediate depth estimation is the key to improving multi-view 3D object detection and added explicit depth supervision with groundtruth depth generated from point clouds. PON~\cite{roddick2020pon} learned the transformation leveraging geometry relationship between image locations and BEV locations in the horizontal direction. 

\textbf{Projection-based} methods generate dense voxel or BEV representation from image features through 3D-to-2D projection~\cite{ma2022bevsurvey}. ImVoxelNet~\cite{rukhovich2022imvoxelnet} aggregated the projected features from several images via a simple element-wise averaging, where spatial information might not be exploited sufficiently.
Transformer-based methods~\cite{li2022bevformer,peng2022bevsegformer} mapped perspective view to BEV with designed BEV queries and leveraged cross- and self-attention to aggregate spatial and temporal information into BEV queries. Since global attention needs huge memory with high time cost, deformable attention was adopted in BEVFormer~\cite{li2022bevformer}. 




\section{VIMI Framework}

VIMI aims to fuse vehicle and infrastructure features by utilizing V2X communication. It includes four main modules: Feature Compression (FC), Multi-Scale Cross Attention (MCA), Camera-aware Channel Masking (CCM), and Point-Sampling Voxel Fusion, as illustrated in Figure~\ref{fig:framework}. 

System input is a pair of RGB images from both vehicle and infrastructure cameras. First, features are extracted separately from backbone and 2D neck on each side. Then, the infrastructure feature is sent to FC module, which compresses the feature, transmits it to vehicle side, and decompresses the feature. Multi-scale features are generated from decompression output and sent to MCA module for augmentation. Then, image features are integrated with camera parameters through CCM module. The augmented features are projected into a 3D voxel volume, which aggregates the features via element-wise averaging. Next, the fused voxel feature is transformed into BEV feature via 3D neck. Finally, 2D detection heads composed of several CNN blocks take the BEV feature as input and predict 3D bounding boxes. 

Prediction results are in ego-vehicle coordinate system, which is shown in Figure~\ref{fig:Coord} and can be parameterized as $(x, y, z, w, h, l, \theta)$, where $(x, y, z)$ are the coordinates of box's center, $w, h, l$ refer to object's width, height and length, and $\theta$ is rotation angle around $z-$axis. 

\begin{figure}[t]
	\centering  
	\includegraphics[width=0.8\linewidth]{Coord.png} 
	\caption{Illustration of the coordinate system in VIMI from BEV. The vehicle (in yellow) communicates with infrastructure (in green), and the two cameras have different view fields. The vehicle coordinate system takes LiDAR as the origin with $x$- and $y$- axis parallel to the ground and $z$-axis upward vertically. Image features need to be transformed into a voxel range (in purple rectangle), as detailed in section 3.4.
 % \jj{Make the font bigger. Move }
}
	\label{fig:Coord}   
\end{figure}

% and features or bounding boxes need to be transformed into the same vehicle coordinate system for training and evaluation

\subsection{Feature Compression}


\begin{figure}[ht]
	\centering  \includegraphics[width=0.9\linewidth]{FC.png} 
	\caption{Illustration of FC module. Feature $f_{inf}^{S}$ is compressed into $f_{inf}^{T}$ through the channel and spatial compressors, which is transmitted to vehicle and is decoded into $f_{inf}^{S\prime}$ through the channel and spatial decompressors. Finally, multi-scale infrastructure features $f_{inf}^{M}$ can be recovered from $f_{inf}^{S\prime}$ with several Conv Blocks with stride 2. 
 % \jj{Move to Appendix? Add a simplified version to Fig. 2?}
 }
	\label{fig:FC}   
\end{figure}


The images from vehicle and infrastructure are denoted as $I_{veh}$ and $I_{inf}$, respectively, and the shape of both images are $ \left[ H \times W \times 3 \right]$. We use separate pre-trained backbones and necks on the vehicle and infrastructure respectively to extract multi-scale image features. The output multi-scale features $f^{M}_{s}, s = veh/inf$ can be denoted as:

% The scales of features are $\left[\frac{H}{4} \times \frac{W}{4} \times C \right]$,$\left[\frac{H}{8} \times \frac{W}{8} \times C \right]$,$\left[\frac{H}{16} \times \frac{W}{16} \times C \right]$, and $\left[\frac{H}{32} \times \frac{W}{32} \times C \right]$. 


\begin{equation}
    \begin{split}
        &f^{M}_{s} =\{f^{m}_{s} \in \left[h_m \times w_m \times C \right],\; m=0,1,2,3\} \\
        &h_0=\frac{H}{4},\; w_0=\frac{W}{4},\; h_m=\frac{h_{m-1}}{2},\; w_m=\frac{w_{m-1}}{2}
    \end{split}
\end{equation}


VIMI transmits image features and camera parameters instead of voxel feature after projection because voxel feature is too large to be transmitted efficiently. The Feature Compression (FC) module (shown in Figure~\ref{fig:FC}) compresses the largest infrastructure feature $f^{0}_{inf}$ (noted as $f^{S}_{inf}$), transmits it to vehicle and regenerate multi-scale features$f^{M}_{inf}$ through decompression.

% The Feature Compression (FC) module (shown in Figure~\ref{fig:FC}) compresses the largest infrastructure feature $f^{0}_{inf}$ (noted as $f^{S}_{inf}$), making it easier to transmit. This compressed feature $f^{T}_{inf}$ can then be recovered through a symmetric decompression. Finally, the recovered feature $f^{S\prime}_{inf}$ can be used to regenerate multi-scale features $f^{M}_{inf}$ using Conv Blocks with stride 2. 






The compression and decompression process in FC module is an Encoder-Decoder with four components: Channel Compressor (CC), Spatial Compressor (SC), Spatial Decompressor (SD), and Channel Decompressor (CD). CC and CD are composed of several convolutional layers. SC comprises several Conv Blocks with stride 2 so that feature scales are reduced to half after each one. SD only replaces convolution with transposed convolution. The Compression Rate (CR) is determined by Channel Compression Rate (CCR) and Spatial Compression Rate (SCR). The number of SC's layers is calculated by $\alpha = \log_{4}{\text{SCR}}$. 





\subsection{Multi-scale Cross Attention}

MCA module (Figure ~\ref{fig:MCA}) applies cross-attention between multi-scale infrastructure and vehicle features to select useful multi-scale features, and contains a Multi-Scale (MS) Block to alleviate the negative effect of calibration noise. Multi-scale features get spatial information surrounding each pixel and are scaled to the same size through MS Block (Figure~\ref{fig:MSB}).

% The MS block (Figure~\ref{fig:MSB}) is first applied to multi-scale features. Since calibration noises can cause a displacement between the projected and groundtruth positions on 2D plane, we apply deformable convolutional networks (DCN) for each scale feature, so that every pixel can get spatial information surrounding it. Then, features at different scales are upsampled to the same size through UpConv blocks.

 MCA applies MeanPooling operation to obtain the representation of different scales of infrastructure features, while vehicle features at different scales are first fused by mean operation and then refined by MeanPooling. To find the correlation between vehicle features and infrastructure features at different scales, cross attention is applied to infrastructure representations as Key and vehicle representation as Query, which generates attention weights $\omega^{m}_{inf}$ for each scale $m$. We calculate inter-product between features $\hat{f}^{M}_{inf}$ and weights $\omega^{m}_{inf}$. The final outputs of MCA are augmented infrastructure image feature $f_{inf}$ and vehicle image feature $f_{veh}$.

\begin{figure}[htbp]
	\centering  
	\includegraphics[width=0.9\linewidth]{MCA.png} 
	\caption{Schema of MCA module.  In the lower branch, vehicle feature $f_{veh}$ is generated from $f^{M}_{veh}$ through MS Block and Mean. In the upper branch, $f^{M}_{inf}$ is refined into `key' through MS Block and MeanPooling, and queries are generated from $f_{veh}$ through MeanPooling. The output weights $\omega_{inf}^{m}$ of cross-attention are applied to $\hat{f}^{M}_{inf}$ with inner product to form infrastructure feature$f_{inf}$.} 
	\label{fig:MCA}   
\end{figure}




 \begin{figure}[ht]
	\centering  
	\includegraphics[width=0.9\linewidth]{MSB.png} 
	\caption{Details of MS Block. Every pixel-wise feature is integrated with the spatial information of surrounding pixels via DCN, and multi-scale features are scaled to the same size through UpConv blocks.}  
	\label{fig:MSB}   
\end{figure}

\subsection{Camera-aware Channel Masking}

% Depth information is important for 3D detection and LSS~\cite{philion2020lss} learns a latent depth prediction in pixel-wise and multiplies depth with image features for more precise splat operation. 

Considering the assumption that the nearer to the camera, the more valuable information can be obtained. And given that camera's extrinsic and intrinsic parameters can instill image features with camera distance information, it is intuitive to take camera parameters as priors to augment image features. %Taking camera parameters as priors has already been applied in some multi-view 3D perception tasks.

 Inspired by the decoupled nature of SENet~\cite{hu2018SE} and  LSS~\cite{philion2020lss}, we generate a channel mask to let each feature be aware of camera parameters (Figure~\ref{fig:CCM}). First, camera intrinsic and extrinsic are stretched into one dimension and concatenated together. Then, they are scaled up to the feature’s dimension $C$ using MLP to generate a channel mask $M_{veh/inf}$. Finally, $M_{veh/inf}$ is used to re-weight the image features $f_{veh/inf}$ in channel-wise and obtain results $f^{\prime}_{veh/inf}$. The overall CCM module can be written as:

\begin{equation}
    \begin{split}
        f^{\prime}_{s}  &= M_{s} \odot f_{s} , s= veh, inf \\
        m_{s} &=  \text{MLP} \left(\xi\left(R_s\right) \oplus \xi\left(t_s\right) \oplus \xi\left(K_s\right)\right)
    \end{split}
\end{equation}
% hxl edited
% \begin{equation}
%     \begin{split}
%         \Tilde{f}^{i}_{s}  &=\text{Conv}\left(SE\left(\hat{f}^{i}_{s} \mid m^{i}_{s} \right)\right),\; s= veh,\; inf \\
%         m^{i}_{s} &= \text{MLP} \left(\xi\left(R_s\right) \oplus \xi\left(t_s\right) \oplus \xi\left(K_s\right)\right)
%     \end{split}
% \end{equation}
% CCM_s &= \text{Sigmoid} (\text{Relu} (\text{FC} (\text{Relu} (\text{FC}(m_{s}))))) \\ 
$\xi$ denotes the flat operation and $\oplus$ means concatenation. The input of MLP is the combination of camera rotation matrix $R_s\in \mathbb{R}^{3\times3}$, translation $t_{s}$ and camera intrinsics $K_{s}$. 

% $M_{s}$ can be obtained from MLP's output $m_{s}$ through several Fully Connected (FC) layers and Activation (Relu, Sigmoid) layers. 

\begin{figure}[htbp]
	\centering  
	\includegraphics[width=0.9\linewidth]{CCM.png} 
	\caption{The schema of CCM module. Camera intrinsic and extrinsic are encoded into the channel mask, then image feature is integrated with it through inner-product operation.}  
	\label{fig:CCM}   
\end{figure}



\subsection{Point-Sampling Voxel Fusion}
\label{section:voxel_fusion}
The augmented vehicle feature $f^{\prime}_{veh}$ and infrastructure feature $f^{\prime}_{inf}$ are projected into 3D space for fusion and generate two voxel features, denoted as $V_{veh}$ and $V_{inf}$, respectively. As seen in Figure~\ref{fig:Coord}, we follow the vehicle coordinate system in DAIR-V2X-C dataset to set $x,y,z$-axis of voxel volume. Each voxel has the same spatial limits in all three axes, which can be denoted as $\left[ x_{\text{min}},y_{\text{min}},z_{\text{min}},x_{\text{max}},x_{\text{max}},x_{\text{max}}\right]$ and every voxel element has the same size $\delta=(\delta_x,\delta_y,\delta_z)$. Therefore, the number of voxels along each axis can be formulated as:
\begin{equation}
N_p = \frac{p_{\text{max}}-p_{\text{min}}}{\delta_p}, p=x,y,z 
\end{equation}

We calculate the 2D coordinates $(u,v)$ in feature map $f^{\prime}_{veh/inf}$ from 3D coordinates $(x,y,z)$ in voxel volume $V_{veh/inf}$ with camera parameters. The depth $d$ at coordinates $(u,v)$ can also be calculated through the transformation.

\begin{equation}
d\left[\begin{array}{l}
u \\
v\\
1
\end{array}\right]=K_s \left[\begin{array}{ll}
R_s & t_s \\
\overrightarrow{0} & 1
\end{array}\right] \left[\begin{array}{c}
x \\
y \\
z \\
1
\end{array}\right] ,s=veh,inf
\end{equation}

All voxel elements along a camera ray are filled with the same feature following the projection principle. A voxel mask $M_s$ with the same shape as $V_s$ is defined  to indicate whether 2D coordinates are within the range of the feature map. So the Point-Sampling can be formulated as:

\begin{equation}
M_s(x, y, z)= \begin{cases}1, &  0 \leq u \leq w_0 \text{ and } 0 \leq v \leq h_0 \\ 
0, & \text { ow. }\end{cases}
\end{equation}

\begin{equation}
V_s(x, y, z)= \begin{cases}f^{\prime}_{s}(u, v), & M_s(x, y, z)=1 \\ 0, & \text { ow. }\end{cases}
\end{equation}

We obtain the final voxel feature $V_{vic} \in N_x\times N_y\times N_z \times C_1$ by averaging sampled voxel features $V_{veh}$ and $V_{inf}$. Then,  the same 3D neck as~\cite{rukhovich2022imvoxelnet}, which is composed of 3D CNN and downsampling layers, transforms voxel feature $V_{vic}$ into BEV feature $B_{vic} \in N_X \times N_y \times C_2$. BEV feature can be used as input of common 2D detection heads to predict 3D detection results.

%\jj{Move this to method section}
The loss of detection heads is similar to SECOND~\cite{yan2018second}, which consists of smooth L1 Loss for bounding box $L_{\text{bbox}}$, focal loss for classification $L_{\text{cls}}$, and cross-entropy loss for direction $L_{\text{dir}}$. The final loss function can be formulated as:

\begin{equation}
L=\frac{1}{n}\left(\lambda_{\text {bbox}} L_{\text {bbox}}+\lambda_{\text {cls}} L_{\text {cls}}+\lambda_{\text {dir}} L_{\text {dir}}\right)
\end{equation}
where $n$ is the number of positive anchors.
\section{Experiments}
\subsection{Settings}

\begin{figure*}[htbp]
	\centering  
	\includegraphics[width=\linewidth]{VIS_Figure2.png} 
	\caption{Visualization results of ImVoxelNet (LF) (left column), ImVoxelNet\_M (EF) (middle column), and VIMI (IF) (right column). Bounding boxes in BEV (bottom row) are projected to vehicle and infrastructure image planes (top two rows). Groundtruth are in green and predictions in red. 
 From BEV, it is clear that red and green bounding boxes from VIMI are better aligned than LF and EF methods. This shows that ImVoxelNet (LF) and  ImVoxelNet\_M (EF) have detected more false positive objects and fewer true positive objects than VIMI (IF).
 % \jj{What's the purple circle?}
 % \jj{Use stronger colors for the ranges. It's hard to see right now.} 
 % \jj{Where does this come from? From the overlapping of green and red bounding boxes in BEV? }.
 % \wz{previous version can be seen in appendix, which is better? need to adjust caption}
 % \jj{Move the Infra images below the Vehicle images. Enlarge the images. This is an important illustration demonstrating the key idea of the paper and deserves more space and better clarity. To save space, perhaps move some earlier figures to Appendix (Fig. 4 or 6?}
 }
	\label{fig:vis_results}   
\end{figure*}

\textbf{Datasets.}
Many studies on Vehicle-to-Infrastructure (V2I)  and Vehicle-to-Vehicle (V2V)  are based on simulated datasets, in which V2X communication systems are relatively idealistic compared with real scenarios. We conduct our experiments on multi-view 3D object detection over a vehicle-infrastructure-cooperation dataset called DAIR-V2X~\cite{yu2022dairv2x}, in which all frames are captured from real scenarios. The component dataset DAIR-V2X-C contains 38,845 frames of images and point clouds for cooperative detection tasks. We utilize the VIC-Sync portion of DAIR-V2X-C dataset for training and evaluation, which is composed of 9,311 pairs of infrastructure and vehicle frames captured at the same time. Annotations of each pair frame are in world coordinate and need to be translated into vehicle coordinate system for training and evaluation (detailed in appendix). We use the translated labels as the benchmark for our experiments. The resolution of images collected by RGB cameras is $1920\times1080$. DAIR-V2X-C dataset is split into train/val/test sets of 4822/1795/2694 frames. 

% \jj{Why choose this component and not others? Explain}

% \jj{Why choose this portion and not others? Explain}
\textbf{Evaluation Metrics.}
DAIR-V2X~\cite{yu2022dairv2x} proposed to use the evaluation metrics of Average Precision (AP)~\cite{Geiger2013KITTI} and Average Byte (AB) to measure detection performance and transmission cost. AP is used to evaluate 3D detection performance compared with DAIR-V2X-C labels. Since the evaluation of VIC3D only focuses on vehicle-egocentric surroundings, we first need to filter the objects outside the cubic area $ x\in \left[x_{\text{min}},x_{\text{max}}\right]\text{m}, y \in \left[y_{\text{min}},y_{\text{max}}\right]\text{m}, z\in\left[z_{\text{min}},z_{\text{max}}\right]\text{m} $ in vehicle coordinate system. The evaluation metrics are based on the detection range surrounding the vehicle, including Overall (0-100m), 0-30m, 30-50m, and 50-100m. All metrics are calculated with $\text{IoU}=0.5$ and can be divided into 2 parts:  $AP_{\text{3D}}$ and $AP_{\text{BEV}}$. AB means the average size of transmitted data, which is the feature map $f_{inf}^{\text{T}}$ in our method. For EF and LF baselines, images and detection results are transmitted respectively.




\textbf{Baselines.} For evaluation, we compare VIMI with several baselines: $1)$ For Late Fusion, we choose ImVoxelNet~\cite{rukhovich2022imvoxelnet} as the monocular detector on each side, same as ~\cite{yu2022dairv2x}. Another LF baseline is a modified version of VIMI, named \textit{VIMI\_Veh} and \textit{VIMI\_Inf} (detailed in~\ref{detection_results}). $2)$ For Early Fusion, we consider typical methods for multi-view camera fusion in nuScenes~\cite{nuscenes2019}, such as the projection-based method BEVFormer~\cite{li2022bevformer} and the lift-based BEVDepth~\cite{li2022bevdepth}. For fair comparison, we implement BEVFormer\_S without temporal attention module. Also for compatibility with DAIR-V2X-C dataset format, we remove depth supervision from point clouds in BEVDepth to take into account camera images only. We also compare with the multi-view fusion version of ImVoxelNet~\cite{rukhovich2022imvoxelnet} (noted as \textit{ImVoxelNet\_M}). All EF methods concatenate images together and send them to backbones with shared weights.


% \jj{List all baselines used in experiments and explain why choose them}
% \wz{list LF(ImVoxelNet,BEVFormer,BEVDepth}


\subsection{Implementation Details}



\textbf{Voxel Feature Construction.}  We use ResNet-50~\cite{he2016resnet} as backbone and Feature Pyramid Networks (FPN) as 2D neck to extract image features. The channel number $C$ of the neck's output is 64. We set the channel of 3D voxel feature $C_1$ to 64 and the channel of BEV feature $C_2$ to 256 following~\cite{yan2018second,lang2019pointpillars}.

We determine the spatial range of the voxel feature by considering the perception range of the camera in DAIR-V2X-C dataset. We use the same size as the Late Fusion baseline used in DAIR-V2X-C. Concretely, $\left[ x_{\text{min}},y_{\text{min}},z_{\text{min}},x_{\text{max}},y_{\text{max}},z_{\text{max}}\right]$ are set to $[0, -39.68, -3, 92.16, 39.68, 1]$ m. We set voxel resolution as $\left[0.32\times0.32\times0.33\right]$ m, so the shape of the voxel volume is $\left[288\times248\times12\right]$.




\textbf{Training.} The original image size is $\left[1080 \times 1920 \right]$ for both infrastructure and vehicle on DAIR-V2X-C dataset. In the training process, we adopt data augmentations including random scaling with the range of $\left[\text{H}\in(912,1008),\text{W}\in(513,567)\right]$ and random horizontal flipping with the probability of 50\% following~\cite{rukhovich2022imvoxelnet}.

We use AdamW optimizer with an initial learning rate of 0.0001 and weight decay of 0.0001. We use Nvidia Tesla A30 GPUs for training with batch size set to 4. The implementation is based on MMDetection3D framework~\cite{mmdet3d2020,mmdetection}, with its default training settings. VIMI is trained for 12 epochs, and learning rate is reduced by ten times after the 8-th and 11-th epoch. we use RepeatDataset in MMDetection3D involving each scene three times in one training epoch. The outputs of anchor-based detection head are filtered with Rotated NMS~\cite{mmdet3d2020} to get the final prediction.
For hyper-parameters, $\lambda_{\text {bbox}}=2.0$, $\lambda_{\text {cls}}=1.0$, $\lambda_{\text {dir}}=0.2$.

For ablation study, all experiments are trained for 12 epochs with batch size 2. As baselines of early fusion, BEVDepth~\cite{li2022bevdepth} and BEVformer\_S~\cite{li2022bevformer} are trained on DAIR-V2X-C for 20 epochs without CBGS strategy~\cite{zhu2019CGBS}.


\subsection{Object Detection Results}
\label{detection_results}
We compare the performance of baseline Late Fusion (LF) methods with ImVoxelNet and our proposed single-side model \textit{VIMI\_Veh/Inf} on DAIR-V2X-C dataset. We also implement several multi-view camera-based methods that have been applied to nuScenes dataset~\cite{nuscenes2019} in VIC3D scenario. The evaluation results on VIC-Sync portion of DAIR-V2X-C dataset are shown in Table~\ref{tab:vimi_SOTA} and Figure~\ref{fig:vis_results}. From the table, Intermediate Fusion (IF) method VIMI has achieved state-of-the-art performance on the multi-view camera fusion benchmark, compared with other methods of Late Fusion (LF) and Early Fusion (EF). VIMI obtains 15.61 $AP_{\text{3D}}$ and 21.44 $AP_{\text{BEV}}$ in overall setting.


\textit{VIMI\_Veh} and \textit{VIMI\_Inf} remove the MCA module but preserve CCM and FC modules so that models can be applied to the vehicle side and infrastructure side respectively without interaction between them, and predictions can be used for Late Fusion. VIMI achieves higher $AP_{\text{3D}}$ and $AP_{\text{BEV}}$ compared with ImVoxelNet~\cite{rukhovich2022imvoxelnet} under the setting of Only-Veh, Only-Inf, and LF. This indicates that VIMI's  single-side model has a stronger feature extraction ability.



% \begin{figure}[htbp]
% 	\centering  
% 	\includegraphics[width=0.8\linewidth]{Figures/vis_results.png} 
% 	\caption{Top row shows the Visualization results of Late-Fusion(ImxovelNet) and the bottom row is the results of VIMI\_C.}  
% 	\label{vis_results}   
% \end{figure}

What is interesting is that Only-Inf methods achieve the best scores in 50-100m $AP_{\text{3D}}$ and $AP_{\text{BEV}}$. As mentioned before, these metrics are related to detecting objects far from the ego vehicle. We count 16,934 objects within the distance range of 50-100m from vehicle, which are used to calculate  the metric of 50-100m $AP_{\text{3D}}$. Among these objects, almost three quarters (12,651) objects are closer to infrastructure camera, which are easier to be detected by Only-Inf models. %brings more benefits for leveraging the infrastructure viewpoint.

% \jj{Is there evidence for this assumption?} 
% \wz{add a figure to show statistics of distance from v and i}


\begin{table*}
  \footnotesize
  \centering
  % \resizebox{\textwidth}{!}{
    \begin{tabular}{ccccccccccc}
    \hline
    \multirow{2}{*}{\textbf{Fusion}} & \multirow{2}{*}{\textbf{Model}} & \multicolumn{4}{c}{\bm{$AP_{\textbf{3D (IoU=0.5)}}$}} & \multicolumn{4}{c}{\bm{$AP_{\textbf{BEV (IoU=0.5)}}$}} & 
    \multirow{2}{*}{\makecell[c]{AB\\(Byte)}} \\ \cline{3-10}
                                &               & Overall   & 0-30m    & 30-50m    & 50-100m   & Overall   & 0-30m     & 30-50m    & 50-100m   & \\ \hline
    \multirow{2}{*}{Only-Veh}   & ImvoxelNet~\cite{rukhovich2022imvoxelnet}    & 7.29      & 16.98     & 2.35      & 0.13      & 8.85      & 19.89     & 3.44      & 0.28      & 
                                \multirow{2}{*}{\textbackslash{}}       \\
                                & VIMI\_Veh  & 8.65     & 19.11     & 4.33      & 0.20      & 10.46     & 22.42     & 5.57      & 0.42      & \\ \hline
    \multirow{2}{*}{Only-Inf}   & ImvoxelNet~\cite{rukhovich2022imvoxelnet}    & 8.66      & 13.05     & 5.79      & \underline{5.50}      & 14.41     & 17.98     & 10.34     & \underline{11.19}     &    
                                \multirow{2}{*}{\textbackslash{}}        \\ 
                                & VIMI\_Inf  & 9.76     & 13.59     & 6.90      & \textbf{6.63} & 14.81 & 18.78     & \underline{11.50}     & \textbf{11.43} &  \\ \hline
    \multirow{2}{*}{Late-Fusion (LF)} & ImVoxelNet~\cite{yu2022dairv2x} & 11.08  & 22.27     & 4.40      & 2.33      & 14.76     & 27.02     & 7.13      & 4.73      & 0.28K \\
                                & VIMI\_Veh/Inf & 11.99 & \underline{24.79}     & 6.08      & 2.30      & 15.79     & 30.39     & 8.50      & 4.84      & 0.28K \\ \hline
    \multirow{3}{*}{Early-Fusion (EF)} & BEVDepth~\cite{li2022bevdepth} & 7.36    & 16.23     & 1.79      & 0.18      & 13.17     & 26.42     & 5.00      & 4.82      &
                                \multirow{3}{*}{550.84K}            \\
                                & BEVFormer\_S~\cite{li2022bevformer}     & 8.80      & 18.07     & 3.71      & 1.76      & 13.45     & 24.76     & 6.46      & 4.63      &       \\
                                & ImVoxelNet\_M~\cite{rukhovich2022imvoxelnet}    & \underline{12.72}     & 23.63     & \underline{7.38}      & 3.11      & \underline{18.17}     & \underline{30.54}     & 11.39     & 7.00      &        \\ \hline
    Intermediate-Fusion (IF)    & VIMI   & \textbf{15.61} & \textbf{29.12} & \textbf{9.07} & 4.01    & \textbf{21.44} & \textbf{36.24}   & \textbf{13.51}     & 8.28      & 32.64K  \\ \hline
    \end{tabular}
    \caption{Quantitative evaluation on DAIR-V2X-C. Best values are marked by bold, and the second best is underlined. All scores in $\%$. 
    % \jj{Update model names in all tables and figures}
    % \jj{Why does ImVoxelNet show up in both LF and EF? If compare with different settings, use different model names (e.g., ImVoxelNet\_LF)}
    }
    \label{tab:vimi_SOTA}
\end{table*}



\subsection{Ablation Study}

We remove MCA, CCM, and FC modules in VIMI and regard it as baseline in the ablation study. We also conduct experiments to investigate when to fuse information from vehicle and infrastructure (details of compared architectures are provided in appendix).

\begin{table}[htbp]
  \footnotesize
  \centering
    \begin{tabular}{ccccc}
    \hline
    \textbf{MCA} & \textbf{CCM} & \textbf{FC} & \bm{$AP_{\textbf{3D}}$} & \bm{$AP_{\textbf{BEV}}$} \\ \hline 
            &               &               & 13.60 & 20.05   \\ 
                & \Checkmark    &               & 13.98 & 20.23   \\ 
     \Checkmark  &               &               & 14.65 & 20.64   \\
     \Checkmark  &  \Checkmark    &               & \underline{15.27} & \underline{21.03}  \\
     \Checkmark  &  \Checkmark    &  \Checkmark    & \textbf{15.61} & \textbf{21.44}   \\ \hline
     
    \end{tabular}
    \caption{Ablation study on each component of VIMI.  
    % \jj{Why are AP-3D and AP-BEV stretched?}
     % \jj{'Check' marks are not aligned with numbers  horizontally. Not fixed}
    }
    \label{TAB:AB}
\end{table}

\textbf{Effect of Each Component.} The ablation results on MCA, CCM, and FC modules are summarized in Table~\ref{TAB:AB}. Comparing the 2nd and 3rd rows with the 1st row, both MCA and CCM can improve performance over baseline, and MCA has increased $AP_{\text{3D}}$ and $AP_{\text{BEV}}$ by 1.05 and 0.59, better than 0.38 and 0.18 increase induced by CCM module. These results demonstrate the validity of MCA, which selects more useful infrastructure features at different scales based on vehicle features with a cross-attention mechanism. FC is designed to eliminate redundant information included in features, while it also improves detection performance. This is because FC module increases the depth of the whole network and introduces extra computation, which can be regarded as feature-refinement.




\textbf{Early or Intermediate Fusion?} The main difference between EF and IF is the type of information transmitted. EF transmits images while IF transmits features. For the former, features can be extracted from the shared-weights backbone, which is a common method in multi-view works~\cite{li2022bevdepth,li2022bevformer,huang2022bevdet4d}. For the latter, since the views of vehicle and infrastructure cameras are different, it is reasonable that images are processed by the backbone and neck on each side separately. The output feature can be transmitted and then fused, which is a common pipeline of IF method. Meanwhile, vehicles and infrastructure can change different backbones for training. Table~\ref{TAB:VoxelBEV} shows that VIMI fusing feature at voxel level achieves better performance compared with EF.

\textbf{Voxel or BEV Fusion?} To investigate when to fuse features in IF method (at voxel or BEV level), we compare the performance of VIMI with \textit{VIMI\_BEV}.
The former belongs to the IF-Voxel pipeline while the latter belongs to the IF-BEV fusion pipeline, which condenses voxel features $V_{veh}$ and $V_{inf}$ into BEV feature respectively with two 3D necks, and then two BEV features are averaged for fusion. Results (Table~\ref{TAB:VoxelBEV}) show that fusion at the voxel level has better performance, which indicates that the transformation from voxel to BEV feature can cause higher information loss.


\begin{table}[htb]
  \footnotesize
  \centering
    \begin{tabular}{cccc}
    \hline
    \textbf{Fusion} & \textbf{Model} & \bm{$AP_{\textbf{3D}}$} & \bm{$AP_{\textbf{BEV}}$}  \\ \hline
            LF       & ImVoxelNet & 11.08  & 14.76   \\ 
            EF       & ImVoxelNet\_M & \underline{12.72}  & \underline{18.17}   \\ 
            IF (BEV)  & VIMI\_BEV & 11.50  & 16.23   \\ 
            IF (Voxel) & VIMI & \textbf{13.37}  & \textbf{19.66}   \\ \hline
    \end{tabular}
    \caption{Analysis on the choice of feature fusion. 
    % \jj{Why are AP-3D and AP-BEV stretched?}
    }
    \label{TAB:VoxelBEV}
\end{table}


\subsection{Impact of Feature Compression}
%\jj{Move to after Ablation Study}

As seen in Figure~\ref{fig:comp_3d}, We investigate the effect of Channel Compressor and Spatial Compressor. First, we change Channel Compression Rate (CCR) from $\times1$ to $\times64$, and the model performance is almost stable at low compression rates, which indicates that channel compression can extract more useful information and remove redundancy. 

% with increasing depth of the network.

% proving that it is better to press the channel first

After CCR reaches the maximum, we continue to compress features with Spatial Compressor. The compression rate ranges from $\times64$ to $\times 16384$. With compressed feature shapes getting smaller, the $AP_{\text{3D}}$ declines from 15.33 to 12.63 but is still higher than LF, and the transmission cost has fallen to 0.51KB which is comparable to LF's cost. 

%This demonstrates spatial compression can bring information loss the same as other detection tasks. There is a trade-off between performance and transmission cost.

\begin{figure}[ht]
	\centering  
	\includegraphics[width=0.9\linewidth]{CM_3D.png} 
	\caption{$AP_{\text{3D (IoU=0.5)}}$ with respect to Compression Rate (shown as number $\times$). CCR is changed from $\times 1$ to $\times64$ and SCR is set from $\times 1$ to $\times 256$ with CCR set to $\times64$. 
 % \jj{Add captions to two figures (a) and (b)}
 }  
	\label{fig:comp_3d}   
\end{figure}

% \begin{figure}[ht]
% {
%     \centering
%     \subfigure [Results of $AP_{\text{3D}}$] {
%      \includegraphics[width=0.45\columnwidth]{Figures/CM_3D.png}
%     }
%     \hspace{0.05in}
%     \subfigure [Results of $AP_{\text{BEV}}$] {
%     \includegraphics[width=0.45\columnwidth]{Figures/CM_BEV.png}
%     }
% \caption{Impact of Compression Rate (CR). CCR is changed from $\times 1$ to $\times64$. SCR is set from $\times 1$ to $\times 256$ with CCR set to $\times64$.}
% \label{fig:comp}  
% }
% \end{figure}

% \begin{table}[htb]
%   \footnotesize
%   \centering
%     \begin{tabular}{ccccc}
%     \hline
%       Model & CCR &$AP_{\text{3D}}$ & $AP_{\text{BEV}}$ & \makecell[c]{AB\\(Byte)} \\ \hline
%             \multirow{4}{*}{VIMI\_C}       
%                   & $\times1$ & 15.28  & 21.45 &   8.356M\\ 
%                   & $\times4$ & 15.24  & 20.99 &   2.088M\\ 
%                   & $\times16$ & 15.07  & 20.48 &  0.522M\\ 
%                   & $\times64$ & 14.96  & 20.50 &  0.130M\\ 
%     \hline
%     \end{tabular}
%     \caption{Quantitative analysis on channel compression rate of FC.}
%     \label{TAB:COMP}
% \end{table}

\subsection{Analysis of Translation Noise}

In order to evaluate model robustness, we conduct several experiments to test the model's ability to overcome transmission noise.  Object position deviation exists commonly in V2X systems due to calibration noise, time asynchrony of sensor triggering, and transmission latency. Since objects like vehicles only move on the ground plane in most scenarios, this deviation can be simplified and formulated as additional noise on $x,y$ axis according to the groundtruth position. Stochastic noises $(\Delta t_x,\Delta t_y)$ simulated through Gaussian distribution with different noise amplitudes $T$ are respectively added to the part of translation $(t_x,t_y)$ of camera parameters. The standard deviation of Gaussian distribution is set to $\frac{1}{3}$ according to 3-sigma rule. 


\begin{table}[htbp]
  \footnotesize
  \centering
      \resizebox{\linewidth}{!}{
    \begin{tabular}{ccccccc}
    \hline
    \multirow{2}{*}{\textbf{Metric}} & \multirow{2}{*}{\textbf{Model}} & \multicolumn{5}{c}{\textbf{Translation Noise Amplitude $T$ (m)}}
     \\ \cline{3-7} & & 0.0 & 0.1 & 0.2 & 0.5 & 1.0  \\ \hline
                            \multirow{3}{*}{$AP_{\text{3D}}$}& ImVoxelnet (LF) & 11.07 & 11.06 &11.00 & 10.76 & 9.92  \\ 
                            & VIMI\_B& 13.60 & 13.62 & 13.71 & 12.90 & 11.04   \\ 
                           & VIMI & \textbf{15.61} & \textbf{15.53} & \textbf{15.38} & \textbf{14.51} & \textbf{12.36}  \\ \hline

                          \multirow{3}{*}{$AP_{\text{BEV}}$} & ImVoxelnet (LF) & 14.89 & 14.85 & 14.84 & 14.41 & 13.32  \\ 
                            & VIMI\_B&20.05 & 20.04 & 19.90 & 18.68 & 15.87  \\ 
                           & VIMI & \textbf{21.44} & \textbf{21.43} & \textbf{21.33} & \textbf{20.46} & \textbf{17.46}  \\ \hline
    \end{tabular}}
    \caption{Results of $AP_{\text{3D (IoU=0.5)}}$ and $AP_{\text{BEV (IoU=0.5)}}$ considering transmission noise.}
    \label{TAB:T_NOISE}
\end{table}

We compare the impact of transmission noise on three models: ImVoxelNet (LF), VIMI, and \textit{VIMI\_B}, which removes MCA, FC, and CCM modules and only keeps the fusion methodology at feature level. As shown in Table~\ref{TAB:T_NOISE}, the performance of three models have a similar decline trend with noise amplitude increasing from 0.1m to 1m. However, even with noise amplitude at 1m, IF-based method VIMI achieves 12.36 $AP_{\text{3D}}$ and 17.46 $AP_{\text{BEV}}$, which are better compared to LF's results without noise (11.07 $AP_{\text{3D}}$ and 14.89 $AP_{\text{BEV}}$). The results also indicate that transmission noise has a negative impact on detection performance, and further study is needed to tackle this practical challenge.



\section{Conclusion}

VIMI is a novel multi-view intermediate-fusion framework for camera-based VIC3D task. To correct the negative effect of calibration noises and time asynchrony, we design a Multi-scale Cross-Attention module and Camera-aware Channel Masking module to fuse and augment multi-view features. VIMI also effectively reduces transmission cost via Feature Compression, and has achieved state-of-the-art results on DAIR-V2X-C benchmark, significantly outperforming previous EF and LF methods. Future study points to extension of the framework to more data modalities.
%Experimental results demonstrate that our method achieves better trade-off between performance and transmitting costs. Moreover, we further show that voxel level in 3D space is optimal for feature fusion, which can facilitate further studies on intermediate fusion with other data modalities.


\clearpage


{\small
\bibliographystyle{ieee_fullname}
\bibliography{egbib}
}

\end{document}
