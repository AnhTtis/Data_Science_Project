\section{}\label{sec:appendixC}


% \begin{lemma}\label{lem: tight}
% Suppose that the sequences $X_{\cdot}^n$ and $Y_{\cdot}^n$ are tight processes in $C[0, T]$ for a $T > 0$. Then we have that the sequence $ (X_{\cdot}^n + Y_{\cdot}^n)$ is a tight process in $C[0, T]$.
% \end{lemma}

% \begin{proof}
% Let us denote the probability measure $P_n$ on $C[0, T]$ as the distribution of $ (X_{\cdot}^n + Y_{\cdot}^n)$. First we will prove that for each positive $\eta$, there exist an $a$ and an $n_0$ such that
% \begin{equation}\label{eq: ap0}
% P_n[f \in C[0, T]: |f(0)| \ge a] \le \eta \quad \text{for all } n \ge n_0 \,. 
% \end{equation}
% We denote the set $G_0 (a):=\{f \in C[0, T]: |f(0)| \ge a\}$. 

% Since $X_{\cdot}^n$ and $Y_{\cdot}^n$ are tight processes in $C[0, T]$, by Theorem 7.3 in~\cite{billingsley1999probability} there exist $a_x$, $a_y$, $n_0^x$ and $n_0^y$ such that
% \begin{equation}\label{eq: ap1}
% \begin{split}
% & \PP[|X_0^n| \ge a_x] \le \frac{\eta}{2} \quad \text{for all } n \ge n_0^x \quad \text{and}
% \\
% & \PP[|Y_0^n| \ge a_y] \le \frac{\eta}{2} \quad \text{for all } n \ge n_0^y \,.
% \end{split}    
% \end{equation}

% Now let us choose $a \ge a_x + a_y$. By triangle inequality and union bound  we have
% \begin{equation}\label{eq: ap2}
% \begin{split}
% P_n & [G_0(a)]  = \PP[|X_0^n + Y_0^n| \ge a] \le \PP[|X_0^n| + |Y_0^n| \ge a] 
% \\
% & \le \PP[\{|X_0^n| \ge a_x\} \cup \{|Y_0^n| \ge a_y\}] \le \PP[|X_0^n| \ge a_x] + \PP[|Y_0^n| \ge a_y] \,.
% \end{split}    
% \end{equation}

% Thus for a $n_0 = \max \{n_0^x, n_0^y\}$, by~\eqref{eq: ap1}, ~\eqref{eq: ap2}, we obtain~\eqref{eq: ap0}.

% We set $\omega_f (\delta)$ as the \textit{modulus of continuity} of an arbitrary function $f(\cdot)$ on $[0, T]$. Now we shall prove that for each positive $\varepsilon$ and $\eta$, there exist a $\delta \in (0, 1)$ and an $m_0$ such that
% \begin{equation}\label{eq: ap31}
% P_n[f \in C[0,T]: \omega_f (\delta) \ge \varepsilon] \le \eta \quad \text{for all } n \ge m_0 \,.    
% \end{equation}
% We denote the set $H_t(\varepsilon, \delta):= \{f \in C[0,T]: \omega_f (\delta) \ge \varepsilon \}$.


% Since $X_{\cdot}^n$ and $Y_{\cdot}^n$ are tight processes in $C[0, T]$, by Theorem 7.3 in~\cite{billingsley1999probability} there exist $\delta_x \in (0,1)$, $\delta_y \in (0,1)$, $m_0^x$ and $m_0^y$ such that
% \begin{equation}\label{eq: ap3}
% \begin{split}
% & \PP\left[\sup_{|s-t| \ge \delta_x } |X_s^n - X_t^n| \ge \frac{\varepsilon}{2} \right] \le \frac{\eta}{2} \quad \text{for all } n \ge m_0^x \quad \text{and}
% \\
% & \PP\left[\sup_{|s-t| \ge \delta_y } |Y_s^n - Y_t^n| \ge \frac{\varepsilon}{2}\right] \le \frac{\eta}{2} \quad \text{for all } n \ge m_0^y \,.
% \end{split}    
% \end{equation}

% Then we obtain by triangle inequality and union bound the following
% \begin{equation}\label{eq: ap4}
% \begin{split}
% P_n[H_t(\varepsilon,\delta)] & = \PP\left[\sup_{|s-t| \ge \delta} |X_s^n + Y_s^n - X_t^n -Y_t^n| \ge \varepsilon \right]
% \\
% & \le \PP\left[ \sup_{|s-t|\le \delta}|X_s^n - X_t^n| + \sup_{|s-t|\le \delta}|Y_s^n - Y_t^n| \ge \varepsilon \right]
% \\
% & \le \PP\left[ \left\{ \sup_{|s-t|\le \delta}|X_s^n - X_t^n| \le \frac{\varepsilon}{2} \right\} \bigcup  \left\{\sup_{|s-t|\le \delta}|Y_s^n - Y_t^n| \le \frac{\varepsilon}{2} \right\} \right]
% \\
% & \le \PP\left[ \sup_{|s-t|\le \delta}|X_s^n - X_t^n| \le \frac{\varepsilon}{2} \right] + \PP\left[\sup_{|s-t|\le \delta}|Y_s^n - Y_t^n| \le \frac{\varepsilon}{2} \right] \,.
% \end{split}    
% \end{equation}


% We choose now a $\delta= \min\{\delta_x, \delta_y\}$ and $m_0 = \max\{m_0^x , m_0^y \}$, thus by~\eqref{eq: ap3} and~\eqref{eq: ap4} we obtain~\eqref{eq: ap31}. Hence by Theorem 7.3 in~\cite{billingsley1999probability} we finish the proof.


 
% %Let us denote $A_X$, $A_Y$ and $A_{X+Y}$ as sets in $C[0,T]$ which contains the processes $X_{\cdot}^n$, $Y_{\cdot}^n$ and $(X_{\cdot}^n + Y_{\cdot}^n)$ respectively for all $n \ge 0$. 

% %Since the processes $X_{\cdot}^n$ and $Y_{\cdot}^n$ are tight by Prohorov's Theorem (see Theorem 5.1 in~\cite{billingsley1999probability}) the sets $A_X$ and $A_Y$ are relatively compacts in $C[0, T]$.

% %If we prove the set $A_{X+Y}$ is relatively compact in $C[0, T]$ by Prohorov's Theorem (see Theorem 5.2 in~\cite{billingsley1999probability}) we have the desired result. For that we will use the Arzelà-Ascoli Theorem (see Theorem 7.2 in~\cite{billingsley1999probability}).

% %By triangle inequality we have
% %\begin{equation}\label{eq: A_x+y}
% %\sup_{f \in A_{X+Y}} |f(0)| \le \sup_{f \in A_{X}} |f(0)| + \sup_{f \in A_{Y}} |f(0)| < \infty \,.  
% %\end{equation}
% %Since $A_X$ and $A_Y$ are relative compact in $C[0, T]$ we have the last inequality in~\eqref{eq: A_x+y}.

% %Let us denote $\omega_f (\delta)$ as the \textit{modulus of continuity} of an arbitrary function $f(\cdot)$ on $[0, T]$. Now one can see that for all $n \ge 0$
% %\begin{equation}\label{eq: modcon}
% %\begin{split}
% %\omega_{X_{\cdot}^n + Y_{\cdot}^n}(\delta) & = \sup_{|s-t|\le \delta} |X_s^n + Y_s^n - X_t^n -Y_t^n| 
% %\\
% %& = \sup_{|s-t|\le \delta} |(X_s^n - X_t^n) + (Y_s^n -Y_t^n)|
% %\\
% %& \le \sup_{|s-t|\le \delta}(|X_s^n - X_t^n|) + \sup_{|s-t|\le \delta}(|Y_s^n - Y_t^n|)
% %\\
% %& \le \omega_{X_{\cdot}^n}(\delta) + \omega_{ Y_{\cdot}^n}(\delta) \,.
% %\end{split}    
% %\end{equation}
% %The second inequality in~\eqref{eq: modcon} we apply triangle inequality.

% %By~\eqref{eq: modcon} and the fact that $A_X$ and $A_Y$ are relatively compacts in $C[0, T]$ we obtain
% %\begin{equation}\label{eq: A_x+y2}
% %\lim_{\delta \to 0} \sup_{f \in A_{X+Y}} \omega_f (\delta) \le \lim_{\delta \to 0} \left(\sup_{f \in A_X}\omega_f (\delta) + \sup_{f \in A_Y}\omega_f (\delta)\right) = 0 \,.     
% %\end{equation}

% %Since we have~\eqref{eq: A_x+y} and~\eqref{eq: A_x+y2} by the Arzelà-Ascoli Theorem (see Theorem 7.2 in~\cite{billingsley1999probability}), $A_{X+Y}$ is relatively compact in $C[0,T]$ and we finish the proof.
% \end{proof}

% \begin{lemma}\label{lem: convprob}
% Suppose that we have a process $Y_{\lfloor n \cdot \rfloor}$ that converges in probability to zero in the space $C_{\Rs^d}[0, T]$ with the uniform metric for all $T >0$. Then $Y_{\lfloor n \cdot \rfloor}$ converges in probability to zero in the space $C_{\Rs^d}[0, \infty)$ equipped with the following metric
% \begin{equation*}
%  \rho(f, g) := \sum_{k=1}^{\infty}\frac{1}{2^k} \sup_{0 \le t \le k}(||f(t) - g(t)|| \wedge 1) \,.   
% \end{equation*}
% \end{lemma}

% \begin{proof}
% Since $Y_{\lfloor n \cdot \rfloor}$ converges in probability to zero in $C_{\Rs^d}[0, T]$, we have
% \begin{equation}\label{eq: convprob}
%     \PP\left[ \sup_{0 \le t \le T}|| Y_{\lfloor nt \rfloor}|| > \delta \right] \to 0 \quad \text{as } n \to \infty \,,
% \end{equation}
% for any $\delta > 0$. 

% Now let $\varepsilon > 0$ and we choose a positive integer $N$ such that $\sum_{l \ge N} 2^{-l} < \varepsilon/2$. Thus we have the following
% \begin{equation}\label{eq: convprob2}
% \begin{split}
% & \PP\left[ \sum_{k=1}^{\infty} 2^{-k} \sup_{0 \le t \le k}(||Y_{\lfloor nt \rfloor}|| \wedge 1) > \varepsilon \right] =  
% \\
% & = \PP\left[ \sum_{k=1}^{N} 2^{-k} \sup_{0 \le t \le k}(||Y_{\lfloor nt \rfloor}|| \wedge 1) + \sum_{k=N+1}^{\infty} 2^{-k} \sup_{0 \le t \le k}(||Y_{\lfloor nt \rfloor}|| \wedge 1) > \varepsilon \right]
% \\
% & \le \PP\left[ \sum_{k=1}^{N} \sup_{0 \le t \le k}||Y_{\lfloor nt \rfloor}|| + \sum_{k=N+1}^{\infty} 2^{-k}  > \varepsilon \right] \,.
% \end{split} 
% \end{equation}

% Since we choose $N$ large enough by~\eqref{eq: convprob2} we obtain that
% \begin{equation}\label{eq: convprob3}
% \begin{split}
% & \PP\left[ \sum_{k=1}^{\infty}  \sup_{0 \le t \le k}(||Y_{\lfloor nt \rfloor}|| \wedge 1) > \varepsilon \right] 
% \\
% & \le  \PP\left[ \left\{\sum_{k=1}^{N}  \sup_{0 \le t \le k}||Y_{\lfloor nt \rfloor}|| > \frac{\varepsilon}{2} \right\} \bigcup \left\{ \sum_{k=N+1}^{\infty} 2^{-k}  > \frac{\varepsilon}{2} \right\}\right] 
% \\
% & \le \PP\left[ \sum_{k=1}^{N}  \sup_{0 \le t \le k}||Y_{\lfloor nt \rfloor}|| > \frac{\varepsilon}{2} \right] + \PP\left[ \sum_{k=N+1}^{\infty} 2^{-k}  > \frac{\varepsilon}{2} \right] 
% \\
% & \le \PP\left[ \bigcup_{k=1}^{N}  \sup_{0 \le t \le k}||Y_{\lfloor nt \rfloor}|| > \frac{\varepsilon}{2N} \right] \le  \sum_{k=1}^{N} \PP\left[ \sup_{0 \le t \le k}||Y_{\lfloor nt \rfloor}|| > \frac{\varepsilon}{2N} \right] \,.
% \end{split}    
% \end{equation}
% We have the third and last inequalities in~\eqref{eq: convprob3} by union bound. 

% Now one can see that all sum portions in the last inequality in~\eqref{eq: convprob3} go to zero as $n$ tends to infinity by~\eqref{eq: convprob}, ergo we have
% \begin{equation*}
% \PP\left[ \sum_{k=1}^{\infty} 2^{-k} \sup_{0 \le t \le k}(||Y_{\lfloor nt \rfloor}|| \wedge 1) > \varepsilon \right] \to 0 \quad \text{as } n \to \infty \,,    
% \end{equation*}
% for any $\varepsilon > 0$. Hence we obtain the desired result.
% \end{proof}

\begin{lemma}\label{lem: iid}
Let $\{\phi_n\}_{n \ge 1}$ be a sequence of i.i.d. random vectors on $\ZZ^d$, with $d \ge 2$ and $\{\kappa_n\}_{n \ge 1}$ an increasing sequence of $\FF_n$-predictable times defined on 
a probability space $(\Omega, \FF, \PP)$, where $\{\FF_n\}_{n\ge 1}$ is the filtration $\FF_0 = \{\emptyset, \Omega\}$ and $\FF_n=\sigma(\phi_1, \dots, \phi_n)$, $n\ge 1$. Then we have that $\{\phi_{\kappa_n}\}_{n \ge 1}$ is i.i.d. and moreover $\phi_{\kappa_n}$ has the same distribution of $\phi_1$ for every $n \ge 1$. 
\end{lemma}
\begin{proof}
We begin showing that for any $n \ge 1$, $\phi_{\kappa_n}$ has the same distribution of $\phi_1$.
%
Let $A$ be a subset of   $\ZZ^d$ and fix a $j \ge 1$. Then we have that
\begin{equation}\label{eq: iid1}
\begin{split}
& \PP[\phi_{\kappa_j} \in A | \FF_{\kappa_{j}-1}] = \sum_{n=1}^{\infty} \PP[\{ \phi_{\kappa_j} \in A \} \cap \{\kappa_j = n\}| \FF_{\kappa_{j}-1}]
\\
& = \sum_{n=1}^{\infty} \PP[\phi_{\kappa_j} \in A |\{\kappa_j = n\}, \FF_{\kappa_{j}-1}] \PP[\kappa_j = n| \FF_{\kappa_{j}-1}] \\
& = \sum_{n=1}^{\infty} \PP[\phi_n \in A ]\PP[\kappa_j = n| \FF_{\kappa_{j}-1}] = \PP[\phi_1 \in A ] \underbrace{\sum_{n=1}^{\infty} \PP[\kappa_j = n| \FF_{\kappa_{j}-1}]}_{=1}\,.
\end{split}    
\end{equation}
The third equality in~\eqref{eq: iid1} follows from  the fact that $\{\kappa_n\}_{n \ge 1}$ is an increasing sequence of $\FF_n$-predictable times, $\{\phi_n\}_{n \ge 1}$ is i.i.d. and so $\phi_n$ is independent of $\FF_{n-1}$ for all $n \ge 1$.
%
It remains to prove independence. We will only prove pairwise independence, but it is straightforward to generalize the proof by induction and we leave the details to the reader.  For $B$ and $D$ subsets of $\ZZ^d$ and $j, i \in \mathbb{N}$ such that $j > i$ we have that 
\begin{eqnarray}\label{eq: iid2}
\lefteqn{\PP[\{\phi_{\kappa_i} \in B\} \cap \{\phi_{\kappa_j} \in D\}] =} \nonumber
\\ 
& & = \sum_{n=1}^{\infty} \sum_{m > n} \PP[\{\{\phi_{\kappa_i} \in B\} \cap \{\kappa_i = n\}\} \cap \{\{\phi_{\kappa_j} \in D\} \cap \{\kappa_j = m\}\}] \nonumber
\\
& & = \sum_{n=1}^{\infty} \sum_{m > n} \PP[\phi_{\kappa_j} \in D | \kappa_i = n, \kappa_j = m, \phi_{n} \in B]\PP[\kappa_i = n, \kappa_j = m, \phi_{\kappa_i} \in B] \nonumber
\\
& & = \sum_{n=1}^{\infty} \sum_{m > n} \PP[\phi_m \in D]\PP[\kappa_i = n, \kappa_j = m, \phi_{\kappa_i} \in B]\,.
\end{eqnarray}
The last equality in~\eqref{eq: iid2} follows from the fact that the event $\{\kappa_i = n, \kappa_j = m, \phi_{n} \in B\}$, for $m>n$, is $\FF_{m-1}$ measurable, since $\{\kappa_n\}_{n \ge 1}$ is an increasing sequence of $\FF_n$-predictable times.

Since the sequence $\{ \phi_n\}_{n \ge 1}$ is i.i.d., we can continue the computation in~\eqref{eq: iid2} and obtain that
\begin{equation*}
\begin{split}
\PP[\{\phi_{\kappa_i} \in B\} \cap \{\phi_{\kappa_j} \in D\}] & = \PP[\phi_1 \in D] \sum_{n=1}^{\infty} \sum_{m \ge n} \PP[\kappa_j = m,\kappa_i = n, \phi_{\kappa_i} \in B]
\\
& = \PP[\phi_1 \in D] \sum_{n=1}^{\infty}  \PP[\kappa_i = n, \phi_{\kappa_i} \in B] 
\\
& = \PP[\phi_1 \in D] \sum_{n=1}^{\infty} \PP[\phi_{n} \in B]\PP[\kappa_i = n]
\\
& = \PP[\phi_1 \in D] \PP[\phi_1 \in B] = \PP[\phi_{\kappa_i} \in D]\PP[\phi_{\kappa_j} \in B]\,,
\end{split}    
\end{equation*}
concluding the proof.
\end{proof}










