\section{}\label{sec:appendix-mainTheorem}

\com{I do not know if makes anymore sense to include this appendix .....}\comu{concordo. está na outra versão, então podemos remover logo.}

In this Appendix we show that, by imposing a restriction on the direction of the drift $\ell$, it is possible to provide a ``weaker''  version of Theorem~\ref{pn-ERW-d=>4} which holds in any dimension $d\geq 4$. 
%
  The restriction is due to the technique we used to prove a weaker version of Proposition~\ref{prop: bound_Kn} by avoiding using Proposition~\ref{prop:RangeERW_lower} (which requires $d\geq 22$). The main idea to circumvent the usage of Proposition~\ref{prop:RangeERW_lower} consists in lower bounding the range of the $p_n$-\Nametwo{} with the range of a lazy random walk and then using Theorem~\ref{teo: RnZ>} (LLN). For this to work properly the lazy random walk should be at least $3$ dimensional, and we achieve that by restricting the direction of the drift to a subspace which is at most $d-3$ dimensional. 

Let us formally introduce the dimensional restriction. Let us define the set $\bD \subset \{e_1, \dots, e_d\}$, where $d \ge 4$ and $1 \leq k:= |\bD| \leq d-3$. Now set $\ell_{\bD}$ as a direction in the unit sphere in dimension $d$  such that $\ell_{\bD} = \sum_{i=1}^k \alpha_i x_i$, where $\alpha_i \in [0,1]$ and $x_i \in \bD$, both for all $1 \leq i \leq k$. In essence,  $\ell_{\bD}$ is a direction in the unit sphere in dimension $d$ determined by the canonical directions of the set $\bD$. 

As before, we  set $\pi_d$ as the probability that the random walk with increments $\{\xi_i\}_{i\geq 0}$ 
($\ZZ^d$-valued i.i.d. random variables with zero-mean vector and finite variance), never returns to the origin. Moreover, if  $X$ is a $p_n$-\Nametwo{} in direction $\ell_{\bD}$ and  $\mathcal{P}_{\bD^c}$ denotes the projection on $\bD^c$, then $\pi_{d-k}$ denotes the probability that the $(d-k)$-dimensional lazy random walk with increments $\{\mathcal{P}_{\bD^c}(\xi_i)\}_{i\geq 0}$  never returns to the origin. Note that $\pi_{d-k} \le \pi_{d}$.



\begin{theorem}\label{th:}
Let $X$ be a $p_n$-\Nametwo{} in direction $\ell_{\bD}$, on $\ZZ^d$ with $d \ge 4$,  $p_n= \mathcal{C} n^{-1/2} \wedge 1$. Then $\{\Hat{B}_{\cdot}^n\}_{n\ge 1}$ is tight in $C_{\Rs^d}[0, \infty)$  and there exists a Brownian Motion $W_{\cdot}$ such that for every limit point $\mathcal{Y}_{\cdot}$ of $\{\Hat{B}_{\cdot}^n\}_{n\ge 1}$ 
\begin{align*}
\left\{W_t \cdot \ell_{\bD} + 2 \hat{c}_1 \sqrt{t}\right\}_{t\ge 0} \preceq \{\mathcal{Y}_t \cdot \ell_{\bD}\}_{t\ge 0} \preceq \left\{W_t \cdot \ell_{\bD} + 2 {c}_2 \sqrt{t}\right\}_{t\ge 0} \,,   
\end{align*}
where $\hat{c}_1 = \hat{\mu}_\gamma ( 1-\sqrt{1 -  \pi_{d-k}} )$, ${c}_2 =\hat{\mu}_\gamma \sqrt{\pi_d}$ with $\hat{\mu}_{\gamma} := \EE[\gamma_i \cdot \ell_{\bD}]$ and $\preceq$ means ``stochastically less or equal to''. 
\end{theorem}
%
Note that the constant appearing in the lower bound in Theorem \ref{th:} is different from that in $b)$ of Theorem~\ref{pn-ERW-d=>4}. Clearly, in order to fairly compare the two result we must consider Theorem~\ref{pn-ERW-d=>4} for a drift direction which satisfies the above mentioned restriction. Assuming the latter, since $\pi_{d-k} \le \pi_d$ we obtain that  $\hat c_1 = \hat{\mu}_\gamma ( 1-\sqrt{1 -  \pi_{d-k}} ) \le \hat{\mu}_\gamma ( 1-\sqrt{1 -  \pi_{d}} ) = c_1$. Therefore, for $d\ge 22$, the statement of Theorem \ref{th:} is weaker than that of Theorem~\ref{pn-ERW-d=>4}. Nevertheless   Theorem~\ref{th:} holds for  any $d\geq 4$. 


\subsection{Proof of Theorem~\ref{th:}}
\hfill

 We only need to prove the lower bound.
We begin  by defining a useful coupling between the $p_n$-\Nametwo{} and a  lazy random walk. Again we assume $\CC = 1$, i.e., $p_n=n^{-1/2}$.

Let $\{X_i\}_{i \ge 0}$ a $p_n$-\Nametwo{} on $\ZZ^d$ with drift direction $\ell_{\bD}$ and let $\{Y_i\}_{i \geq 0}$ be the sequence of random vectors on $\ZZ^d$ defined as $Y_0=0$ and recursively for $n \ge 1$ 
\begin{equation*}
Y_{n+1} = Y_n + \mathcal{P}_{\bD^c} \big(\um_{E_n \cap \{U_{n+1} \leq n^{-1/2}\}}\gamma_{n+1} + \um_{E_n \cap \{U_{n+1} > n^{-1/2}\}} \xi_{n+1} + \um_{E_n^c}\xi_{n+1} \big) \,.     
\end{equation*}
The following properties stem directly from the definition:
\begin{itemize}
    %\item $\{Y_i\}_{i \geq 0}$ is a $p_n$-\Nametwo{} on $\ZZ^d$ with drift direction $\ell_{\bD}$.
    \item $\{Y_i\}_{i \geq 0}$ is a lazy random walk on $\ZZ^d$, evolving in a sublattice isomorphic to $\ZZ^{d-k}$.
    \item For all $e_j \in \bD^c$ and $i \geq 0$, we have $Y_i \cdot e_j = X_i \cdot e_j$. 
\end{itemize}

% Recall that the $p_n$-\Nametwo{} which we consider in this section has a restriction on the drift direction $\ell$. Specifically, $\ell$ could be any direction of the unitary sphere that is in the span of at most $d-3$ canonical directions.
% %
% This dimensional constraint   is crucial in our proof technique and it is due to how we use Theorem~\ref{teo: RnZ>} (SLLN) \com{in the proof of Proposition~\ref{prop: bound_Kn} (see, Remark~\ref{rem:restriction})}.

% Recall that in this section $\bD \subset \{e_1, \dots, e_d\}$, where $1 \leq k :=|\bD| \leq d-3$, $\ell_{\bD} \in \mathbb{S}^{d-1}$ is spanned by the canonical directions in $\bD$, $\mathcal{P}_{\bD^c}$ denotes the projection on $\bD^c$ and $\{X_i\}_{i \ge 0}$ in a $p_n$-\Nametwo{} on $\ZZ^d$ with drift direction $\ell_{\bD}$. Here $\{Y_i\}_{i \geq 0}$ is the sequence of random vectors on $\ZZ^d$ defined as $Y_0=0$ and recursively for $n \ge 1$ 
% \begin{equation*}
% Y_{n+1} = Y_n + \mathcal{P}_{\bD^c} \big(\um_{E_n \cap \{U_{n+1} \leq n^{-1/2}\}}\gamma_{n+1} + \um_{E_n \cap \{U_{n+1} > n^{-1/2}\}} \xi_{n+1} + \um_{E_n^c}\xi_{n+1} \big) \,.     
% \end{equation*}
% The following properties stem directly from the definition:
% \begin{itemize}
%     %\item $\{Y_i\}_{i \geq 0}$ is a $p_n$-\Nametwo{} on $\ZZ^d$ with drift direction $\ell_{\bD}$.
%     \item $\{Y_i\}_{i \geq 0}$ is a lazy random walk on $\ZZ^d$, evolving in a sublattice isomorphic to $\ZZ^{d-k}$.
%     \item For all $e_j \in \bD^c$ and $i \geq 0$, we have $Y_i \cdot e_j = X_i \cdot e_j$. 
% \end{itemize}

% Henceforth we suppose that all the following definitions are in the same common probability space. Let $\{U'_i\}_{i\geq 1}$ be a sequence of i.i.d. random variables uniformly distributed in [0,1]. We set $\{ \xi_i \}_{i \ge 1}$ and $\{ \gamma_i \}_{i \ge 1}$ as before (see Section~\ref{sec:p_n-ERW}) such that the drift direction of $\{\gamma_i\}_{i \ge 1}$ is $\ell_{\bD}$; both independent of the sequence $\{U'_i\}_{i\geq 1}$. We also define $\mathcal{P}_{\bD}$ and $\mathcal{P}_{\bD^c}$ as the projections respectively on $\bD$ and $\bD^c$. As one last but important restriction, we suppose that $\xi_i, \gamma_i \in \mathcal{P}_{\bD}(\mathbb{Z}^d) \cup \mathcal{P}_{\bD^c}(\mathbb{Z}^d)$, and moreover $\mathcal{P}_{\bD^c}(\xi_i)$ and $\mathcal{P}_{\bD^c}(\gamma_i)$ are identically distributed. \comu{Não vejo motivo para essas definições}
%Let $\{\phi_i\}_{i\geq 1}$ be a sequence of i.i.d uniform distribution on the set $\{1, 2, \dots, d\}$ and $\mathcal{D}_k$ be a set which is a subset of $\{1, 2, \dots, d\}$ and $|\mathcal{D}_k|=k$ where $1 \leq k \leq d-3$. Also, the sequences of random variables $\{U_i\}_{i \geq 1}$ and $\{U'_i\}_{i\geq 1}$ are i.i.d uniforms on $[0, 1]$ and independent of  $\{\phi_i\}_{i\geq 1}$. 

% We define the sequences $\{Y_i\}_{i \geq 0}$ and $\{Z_i\}_{i \geq 0}$ of random vectors on $\ZZ^d$ by setting $Y_0 = Z_0 =0$ and recursively for $n \geq 1$  
% \begin{equation*}
% Y_{n+1} = Y_n + \um_{\mathbb{B}_n\cap \{U'_{n+1} \leq p_{n+1}\}}\gamma_{n+1} + \um_{\mathbb{B}_n \cap \{U'_{n+1} > p_{n+1}\}} \xi_{n+1} + \um_{\mathbb{B}_n^c}\xi_{n+1} \, ,    
% \end{equation*}
% and 
% \begin{equation*}
% Z_{n+1} = Z_n + \mathcal{P}_{\bD^c} \big(\um_{\mathbb{B}_n\cap \{U'_{n+1} \leq p_{n+1}\}}\gamma_{n+1} + \um_{\mathbb{B}_n \cap \{U'_{n+1} > p_{n+1}\}} \xi_{n+1} + \um_{\mathbb{B}_n^c}\xi_{n+1} \big) \, ,    
% \end{equation*}
% where $\mathbb{B}_n:=\{Y_n \notin \Rr_{n-1}^Y\}$. The following properties stem directly from the definitions:
% \begin{itemize}
%     \item $\{Y_i\}_{i \geq 0}$ is a $p_n$-\Nametwo{} on $\ZZ^d$ with drift direction $\ell_{\bD}$.
%     \item $\{Z_i\}_{i \geq 0}$ is an aperiodic random walk (lazy random walk) on $\ZZ^d$. Besides that the process $Z$ behaves like a lazy random walk in $\ZZ^{d-k}$.
%     \item For all $e_j \in \bD^c$ and $i \geq 0$, we have $Y_i \cdot e_j = Z_i \cdot e_j$. 
% \end{itemize}
The next lemma states another important property of the coupling.
%Hence we will prove that if the process $\{Z_i\}_{i \geq 0}$ visits a new site then the process $\{Y_i\}_{i \geq 0}$ will reach a new site too. 

\begin{lemma}\label{lemaZY}
Whenever $\{Y_i\}_{i \geq 0}$ visits a new site, $\{X_i\}_{i \geq 0}$ also visits a new one, which in turn implies that $|\Rr_n^X| \geq |\Rr_n^Y|$ for all $n \geq 0$. 
\end{lemma}

\begin{proof}[Proof of Lemma~\ref{lemaZY}.]  
Suppose that at time $n$ $\{Y_i\}_{i \geq 0}$ reaches a new site ($Y_n \notin \Rr_{n-1}^Y$) and $\{X_i\}_{i \geq 0}$ does not ($X_n \in \Rr_{n-1}^X$). By the definition of $\{Y_i\}_{i \geq 0}$ the direction of its displacement to attain the new site is spanned by $\bD^c$. Moreover we have $Y_n \cdot e_i = X_n \cdot e_i$ for every $e_i \in \bD^c$. Since $X_n \in \Rr_{n-1}^Y$, there exist a $0 \leq j \leq n-1$ such that $X_j = X_n$. Hence $Y_n \cdot e_i = Y_j \cdot e_i$ for every $e_i \in \bD^c$ and again by the definition of the coupling $Y_j \cdot e_i = X_j \cdot e_i$ for every $e_i \in \bD^c$, which means that $Y_n = Y_j$. Thus we have a contradiction. 
\end{proof}

% Let $\pi_d$ denotes the probability that the random walk with i.i.d. (with zero mean and finite variance) increments $\{\xi_i\}_{i\geq 0}$ on $\ZZ^d$ never returns to the origin, and $\pi_{d-k}$ denotes the probability of a random walk with i.i.d. increments (with zero mean and finite variance) given by the corresponding lazy random walk of the coupling never returning to the origin. Recall  that $k=|\bD|$ is the number of canonical directions spanned by the drift direction $\ell_{\bD}$  and $k\leq d-3$.

The proof of Theorem~\ref{th:} follows the same lines of the  proof of Theorem~\ref{pn-ERW-d=>4} with the only difference that we need a ``version'' of $(b)$ in Proposition~\ref{prop: bound_Kn} for $d\geq 4$. This is the content of the next proposition. 
%
\begin{proposition}\label{prop:}  
If $H_{\cdot}$ is a limit point of $\{\Hat{K}^n_{\cdot }/n^{1/2}\}_{n\ge 1}$, then for $d\geq 4$ it holds that
\begin{equation*}
\PP \left[\forall t \in [0,\infty): H_t \ge 2t^{1/2}(1-(1 -  \pi_{d-k})^{1/2}) \ \right] = 1 \, . 
\end{equation*}
\end{proposition} 

\begin{proof}[Proof of Proposition~\ref{prop:}.] We just have to follow the proof of $(b)$ in Proposition~\ref{prop: bound_Kn} until we get to \eqref{eq: B_t^cM>}. At that point we will need to show that 
\begin{equation}\label{lb-AP}
\lim_{n\to \infty} \PP \big[\forall 
 t \in [c,M]:|\Rr_{\lfloor nt \rfloor}^X| \ge \delta'' \lfloor nt \rfloor\big] = 1\,,
\end{equation}
without Proposition \ref{prop:RangeERW_lower}. After this we can finish the proof in the same way.  

Using the coupling with the lazy random walk, we have that 
$$
\PP\big[ \forall t \in [c,M]: |\Rr_{\lfloor nt \rfloor}^X| \ge \delta'' \lfloor nt \rfloor  \big]
 \ge \PP \big[\forall 
 t \in [c,M]:|\Rr_{\lfloor nt \rfloor}^Y| \ge \delta'' \lfloor nt \rfloor\big] \,.    
$$ 
Since $d\geq 4$ and $d-k\geq 3$, we have that $\pi_{d-k}>0$ and by Theorem~\ref{teo: RnZ>} (LLN) there exists an integer random variable $N_{\delta''}$ such that $\PP [|\Rr_{m}^Y| \ge \delta''m ,\ \forall m \ge N_{\delta''} ] = 1$, thus 
$$\lim_{n\to \infty} \PP \big[\forall 
 t \in [c,M]:|\Rr_{\lfloor nt \rfloor}^Y| \ge \delta'' \lfloor nt \rfloor\big] \ge \lim_{n\to \infty} \PP [N_{\delta''} \le cn ] = 1,
 $$ 
for every $\delta'' \in (0,\pi_{d-k})$. Therefore \eqref{lb-AP} holds.
\end{proof}

\begin{comment}
\begin{proof}[Proof of Proposition~\ref{prop:}.] Note that the statement follows if
\begin{equation}\label{oldstatement-AP}
\PP \left[\forall t \in [0,\infty): 2t^{1/2}(1-(1 -  \delta')^{1/2}) \le H_t \le 2(t \delta)^{1/2} \right] = 1 \, ,  
\end{equation}
for every $\delta' \in (0, \pi_{d-k})$ and $\delta \in (\pi_d, 1)$. Indeed taking sequences $\delta'_n \uparrow \pi_{d-k}$ and $\delta_n \downarrow \pi_d$ as $n\to \infty$, we have that 
$$
\Big\{ \forall t \in [0,\infty): 2t^{1/2}(1-(1 -  \pi_{d-k})^{1/2}) \le H_t \le 2(t \pi_d)^{1/2} \Big\}\,,
$$
is given by
$$
\bigcap_{n\ge 1} \Big\{ \forall t \in [0,\infty): 2t^{1/2}(1-(1 - \delta'_n)^{1/2}) \le H_t \le 2(t \delta_n)^{1/2} \Big\}\, .
$$
Let us begin with some instrumental facts: recall (from page \pageref{eq: def Kn}) that $\varphi_i := \psi_i + 1$, where $\{\psi_i\}_{i \ge 1}$ denotes  the sequence of $\FF$-stopping times  corresponding to the  times the $p_n$-\Nametwo{} visits a new site, and let us define: 
\begin{align*}
J'_n(\delta) := 
\sum_{i=1}^{\delta n} \um_{\{U_{\varphi_i} \le \varphi_i^{-1/2} \}}     %\label{eq: domJn<J'}
\qquad \text{ and } \qquad 
V'_n(\delta') :=\sum_{i = 1}^{\delta' n} \um_{\{U_{\varphi_i} \leq (\varphi_i \wedge (n-i) )^{-1/2} \}}\,, %\label{eq: F'n_domsto}
\end{align*}
with $\delta \in (\pi_d, 1)$ and $\delta' \in (0, \pi_{d-k})$. 
%
Using Lemma \ref{lem: iid} and since $\{U_i\}_{i \ge 1}$ is i.i.d. we have that 
\begin{equation}\label{eq:dominance-AP}
J'_n(\delta)  \preceq J_n(\delta) \qquad  \text{ and }  \qquad  V'_n(\delta')  \succeq  V_n(\delta')\,, 
\end{equation} 
where, $J_n$ and $V_n$ are defined  in~\eqref{Bn'} and \eqref{Fn'}, respectively.  


We divide the proof of \eqref{oldstatement} into two parts: the first is concerned with the upper bound and the second with the lower bound. 
Consider the event 
\begin{equation*}
A_n^{{\delta}, c, M}:= \Big\{\forall t \in [c, M]: \frac{|K_{\lfloor nt \rfloor}|}{n^{1/2}} \le 2({\delta} t)^{1/2}  \Big\} \,,   
\end{equation*}
where $c$, $M$ and ${\delta}$ are positive constants such that $M > c$ and ${\delta} \in (\pi_d, 1)$, and recall that $|K_{\lfloor nt \rfloor}|  = \sum_{j=1}^{|\Rr_{\lfloor nt \rfloor}^X|} \um_{\{U_{\varphi_j} \leq \varphi_j^{-1/2} \}}$ (see,  \eqref{eq: def Kn}). For every $\Hat{\delta} \in (\pi_d,\delta)$ we have that
\begin{align}\label{eq: A_t^cM<-AP}
\begin{split}
& \PP[A_n^{{\delta}, c,M}]  \ge \PP[A_n^{\delta, c,M} \cap \{ \forall t \in [c,M]: |\Rr_{\lfloor nt \rfloor}^X| \le \Hat \delta \lfloor nt \rfloor\}]
\\
& \ge \PP\Big[ \Big\{\forall t \in [c,M]: \frac{J'_{\lfloor nt \rfloor}(\delta)}{n^{1/2}} \le 2(\delta t)^{1/2} \Big\} \cap \big\{ \forall t \in [c,M]: |\Rr_{\lfloor nt \rfloor}^X| \le \Hat \delta \lfloor nt \rfloor \big\}  \Big] \, .
\end{split}    
\end{align}
Considering the second event on the right-hand side of~\eqref{eq: A_t^cM<}, by Proposition~\ref{prop:RangeERW}, for every $\Hat{\delta} \in (\pi_d, \delta)$ there exists an integer random variable $N_{\Hat{\delta}}$ such that $\PP[|\Rr_m^X| \le \Hat{\delta} m, \, \forall m \ge N_{\Hat{\delta}}] = 1$, hence $\lim_{n \to \infty} \PP[\forall t \in [c,M]: |\Rr_{\lfloor nt \rfloor}^X| \le \Hat \delta \lfloor nt \rfloor] \ge \lim_{n \to \infty} \PP[N_{\Hat{\delta}} \le cn]=1$.
Now concerning the first event on the right-hand side of~\eqref{eq: A_t^cM<}, by~\eqref{eq:dominance}, specifically since $J'_n(\delta)  \preceq J_n(\delta)$, we obtain that 
\begin{equation*} 
\PP\Big[ \forall t \in [c,M]: \frac{J'_{\lfloor nt \rfloor}(\delta)}{n^{1/2}} \le 2(\delta t)^{1/2} \Big] \ge
\PP\Big[ \forall t \in [c,M]: \frac{J_{\lfloor nt \rfloor}(\delta)}{n^{1/2}} \le 2(\delta t)^{1/2} \Big] \, ,
\end{equation*}
%
where the right-hand side converges to one by Lemma~\ref{lem: tightaux} part $i)$ (convergence in distribution to a deterministic function implies convergence in probability, see \cite[page 27]{billingsley1999probability}).
%it holds that \com{notation of convergence of processes}
%\begin{equation}\label{eq:Jninprob}
%\frac{J_{\lfloor n \cdot \rfloor}}{n^{1/2}} \to 2(\delta \cdot)^{1/2}  \text{ as } n \to \infty  \,, 
%\end{equation}
%in probability, since it converges in distribution to a deterministic function in $C_{\mathbb{R}}[0,\infty)$ (see \cite[page 27]{billingsley1999probability}).
%
Hence on the right-hand side of~\eqref{eq: A_t^cM<} we have an intersection of two events whose probability converges to 1 as $n$ goes to infinity. Thus, for every $M > c >0$ and $\delta \in (\pi_d, 1]$ 
\begin{equation*}
\lim_{n\to \infty}  \PP[A_n^{\delta, c,M}]  = 1\,.
%\PP \Big[ \forall t \in [c, M]: \frac{|K_{\lfloor nt \rfloor}|}{n^{1/2}} \le 2(\Hat{\delta} t)^{1/2} \Big] \to 1 \text{ as } n \to \infty \,,   
\end{equation*}
%
%Since  $\hat{\delta} > \delta$ is arbitrary \com{I ?????}, we  obtain that 
%\begin{equation*}
%\PP \Big[ \forall t \in [c, M]: \frac{|K_{\lfloor nt \rfloor}|}{n^{1/2}} \le 2(\delta t)^{1/2}  \Big] \to 1 \text{ as } n \to \infty\,,   
%\end{equation*}
%
Now suppose that we have monotone decreasing and increasing  sequences $\{c_j\}_{j \ge 1}$ and $\{M_j\}_{j \ge 1}$ respectively, such that  $c_j \to 0$ and $M_j \to \infty$ as $j$ goes to infinity. Let $\{
H_t\}_{t\ge 0}$ be a limit point in distribution of a subsequence of $\{\Hat{K}^n_{\cdot}/n^{1/2}\}_{n\ge 1}$, which is  tight  by  
Lemma~\ref{Jntight} (item $i)$), and define
\begin{equation*}
A^{\delta}:= \left\{\forall t \in [0, \infty): H_t \le 2(\delta t)^{1/2}  \right\} \,,   
\end{equation*}
and 
$$
A^{\delta, c_i, M_i} = \left\{\forall t \in [c_i, M_i]: H_t \le 2(\delta t)^{1/2}  \right\}\,,
$$
and $A^{\delta} = \cap_{i=1}^{\infty} A^{ \delta ,c_i, M_i}$ (since $H_0=0$).
%
Then,  by Portmanteau Theorem we have that for every $i \in \ZZ^+$ %\com{on the rhs we should specify the dependence on $\delta$ and add that for all $\delta \in (\pi_d,1]$...}
\begin{equation*}
\PP[A^{\delta ,c_i, M_i}] \ge \limsup_{n \to \infty} \PP[A_n^{\delta, c_i, M_i}] =1 \,.    
\end{equation*}
%
Hence, $\PP[A^{\delta, c_i, M_i}] = 1$ for all $i \in \ZZ^+$ and we obtain the desired result, namely
\begin{equation}\label{eq:At_1-AP}
\PP[A^{\delta}] = \PP\Big[ \bigcap_{j=1}^{\infty} A^{\delta, c_j, M_j}\Big] = 1 \,. 
\end{equation}
%
\com{Here is where the main difference with $d\geq 22$ is ....}As far as the lower bound is concerned, 
let us define the following events 
\begin{equation*}
\begin{split}
& B_n^{\delta',c, M}:= \Big\{\forall t \in [c, M]:2t^{1/2}(1-(1-\delta')^{1/2}) \le \frac{|K_{\lfloor nt \rfloor}|}{n^{1/2}} \Big\} \,,
\\
& H_n^{\delta',c,M}:= \Big\{\forall t \in [c, M]: 2t^{\frac{1}{2}}(1-(1-\delta')^{\frac{1}{2}}) \le \frac{V'_{\lfloor nt \rfloor}(\delta')}{n^{1/2}} \Big\}\,,
% \\
% & R_{\lfloor nt \rfloor}:= \big\{ \forall t \in [c,M]: |\Rr_{\lfloor nt \rfloor}^X| \ge \delta'' \lfloor nt \rfloor \big\} \,,
\end{split}
\end{equation*}
where $c$, $M$ and  $\delta'$  are positive constants such that $M > c$ and $\delta'\in (0,\pi_{d-k})$. 
%
Given $\delta'' \in (\delta', \pi_{d-k})$,  we have that
\begin{equation}\label{eq: B_t^cM>-AP}
\begin{split}
\PP[B_n^{\delta',c,M}]  & \ge \PP\big[B_n^{\delta',c,M} \cap \big\{ \forall t \in [c,M]: |\Rr_{\lfloor nt \rfloor}^X| \ge \delta'' \lfloor nt \rfloor \big\}\big] 
\\
&\ge \PP\big[ H_n^{\delta',c,M} \cap \big\{ \forall t \in [c,M]: |\Rr_{\lfloor nt \rfloor}^X| \ge \delta'' \lfloor nt \rfloor \big\} \big]\,.
\end{split}
\end{equation}
%
By~\eqref{eq:dominance}, specifically by $V'_n(\delta')  \succeq V_n(\delta')$, it holds that 
$$
\PP\big[ H_n^{\delta',c,M} \big]
\ge \PP \Big[ \forall t \in [c, M]: 2t^{1/2}(1-(1-\delta')^{1/2}) \le \frac{V_{\lfloor nt \rfloor}(\delta')}{n^{1/2}} \Big]\,,
$$
and by the coupling with the lazy random walk 
$$
\PP\big[ \forall t \in [c,M]: |\Rr_{\lfloor nt \rfloor}^X| \ge \delta'' \lfloor nt \rfloor  \big]
 \ge \PP \big[\forall 
 t \in [c,M]:|\Rr_{\lfloor nt \rfloor}^Y| \ge \delta'' \lfloor nt \rfloor\big] \,.    
$$ 
Now by Lemma~\ref{lem: tightaux} part $ii)$  (convergence in distribution to a deterministic function implies convergence in probability, see \cite[page 27]{billingsley1999probability}) 
we obtain that $\lim_{n \to \infty}\PP\big[ H_n^{\delta',c,M} \big]=1$, for every $\delta' \in (0,\pi_{d-k})$. Moreover, since $d\geq 4$ and $d-k\geq 3$, we have that $\pi_{d-k}>0$ and by Theorem~\ref{teo: RnZ>} (LLN) there exists an integer random variable $N_{\delta''}$ such that $\PP [|\Rr_{m}^Y| \ge \delta''m ,\ \forall m \ge N_{\delta''} ] = 1$, thus $\lim_{n\to \infty} \PP \big[\forall 
 t \in [c,M]:|\Rr_{\lfloor nt \rfloor}^Y| \ge \delta'' \lfloor nt \rfloor\big] \ge \lim_{n\to \infty} \PP [N_{\delta''} \le cn ] = 1$, for every $\delta'' \in (0,\pi_{d-k})$. 
%we obtain that
%\begin{equation}\label{eq:Fninprob}
%\frac{\sum_{j=1}^{\lfloor n\cdot \rfloor} \um_{\{ U_j \le j^{-1/2} \}} - F_{\lfloor n\cdot \rfloor}}{n^{\frac{1}{2}}} \to 2(\cdot)^{1/2}(1-(1-\delta')^{1/2}) \text{ as } n \to \infty  \,, 
%\end{equation}
%in probability \com{notation of convergence of processes!}, since it converges in distribution to a  continuous function in $t$ (see \cite[page 27]{billingsley1999probability}).
%
Since in~\eqref{eq: B_t^cM>} we have an intersection of two events whose probability converges to 1 as $n$ goes to infinity, we may conclude that for every $M > c >0$ 
\begin{equation*}
\PP \Big[ \forall t \in [c, M]:2t^{1/2}(1-(1-\delta')^{1/2}) \le \frac{|K_{\lfloor nt \rfloor}|}{n^{1/2}}  \Big] \xrightarrow[n \to \infty]{} 1 \,.
\end{equation*} 
%
Finally, we finish the proof by the same argument used to obtain~\eqref{eq:At_1}.
\end{proof}
\end{comment}


%%%%%%% OLD STUFF
\begin{comment}
\subsection{OLD STUFF.....}


\subsubsection{Case $\beta = 1/2$ and $d \ge 4$; proof of Theorem~\ref{pn-ERW-d=>4}}\label{sec: d>4}

\hfill \\



\comu{essa seção toda precisa de revisão. Minha sugestão é criar um apêndice para o resultado que usa o acoplamento em $4\le d \le 21$. }
We begin this section by defining a useful coupling between the $p_n$-\Nametwo{} and a  lazy random walk. Again we assume $\CC = 1$, i.e., $p_n=n^{-1/2}$.

The $p_n$-\Nametwo{} which we consider  in this section has a restriction on the drift direction $\ell$. Specifically, $\ell$ could be any direction of the unitary sphere that is in the span of at most $d-3$ canonical directions.
%
This dimensional constraint   is crucial in our proof technique and it is due to how we use Theorem~\ref{teo: RnZ>} (SLLN) in the proof of Proposition~\ref{prop: bound_Kn} (see, Remark~\ref{rem:restriction}).

Recall that in this section $\bD \subset \{e_1, \dots, e_d\}$, where $1 \leq k :=|\bD| \leq d-3$, $\ell_{\bD} \in \mathbb{S}^{d-1}$ is spanned by the canonical directions in $\bD$, $\mathcal{P}_{\bD^c}$ denotes the projection on $\bD^c$ and $\{X_i\}_{i \ge 0}$ in a $p_n$-\Nametwo{} on $\ZZ^d$ with drift direction $\ell_{\bD}$. Here $\{Y_i\}_{i \geq 0}$ is the sequence of random vectors on $\ZZ^d$ defined as $Y_0=0$ and recursively for $n \ge 1$ 
\begin{equation*}
Y_{n+1} = Y_n + \mathcal{P}_{\bD^c} \big(\um_{E_n \cap \{U_{n+1} \leq n^{-1/2}\}}\gamma_{n+1} + \um_{E_n \cap \{U_{n+1} > n^{-1/2}\}} \xi_{n+1} + \um_{E_n^c}\xi_{n+1} \big) \,.     
\end{equation*}
The following properties stem directly from the definition:
\begin{itemize}
    %\item $\{Y_i\}_{i \geq 0}$ is a $p_n$-\Nametwo{} on $\ZZ^d$ with drift direction $\ell_{\bD}$.
    \item $\{Y_i\}_{i \geq 0}$ is a lazy random walk on $\ZZ^d$, evolving in a sublattice isomorphic to $\ZZ^{d-k}$.
    \item For all $e_j \in \bD^c$ and $i \geq 0$, we have $Y_i \cdot e_j = X_i \cdot e_j$. 
\end{itemize}

% Henceforth we suppose that all the following definitions are in the same common probability space. Let $\{U'_i\}_{i\geq 1}$ be a sequence of i.i.d. random variables uniformly distributed in [0,1]. We set $\{ \xi_i \}_{i \ge 1}$ and $\{ \gamma_i \}_{i \ge 1}$ as before (see Section~\ref{sec:p_n-ERW}) such that the drift direction of $\{\gamma_i\}_{i \ge 1}$ is $\ell_{\bD}$; both independent of the sequence $\{U'_i\}_{i\geq 1}$. We also define $\mathcal{P}_{\bD}$ and $\mathcal{P}_{\bD^c}$ as the projections respectively on $\bD$ and $\bD^c$. As one last but important restriction, we suppose that $\xi_i, \gamma_i \in \mathcal{P}_{\bD}(\mathbb{Z}^d) \cup \mathcal{P}_{\bD^c}(\mathbb{Z}^d)$, and moreover $\mathcal{P}_{\bD^c}(\xi_i)$ and $\mathcal{P}_{\bD^c}(\gamma_i)$ are identically distributed. \comu{Não vejo motivo para essas definições}
%Let $\{\phi_i\}_{i\geq 1}$ be a sequence of i.i.d uniform distribution on the set $\{1, 2, \dots, d\}$ and $\mathcal{D}_k$ be a set which is a subset of $\{1, 2, \dots, d\}$ and $|\mathcal{D}_k|=k$ where $1 \leq k \leq d-3$. Also, the sequences of random variables $\{U_i\}_{i \geq 1}$ and $\{U'_i\}_{i\geq 1}$ are i.i.d uniforms on $[0, 1]$ and independent of  $\{\phi_i\}_{i\geq 1}$. 

% We define the sequences $\{Y_i\}_{i \geq 0}$ and $\{Z_i\}_{i \geq 0}$ of random vectors on $\ZZ^d$ by setting $Y_0 = Z_0 =0$ and recursively for $n \geq 1$  
% \begin{equation*}
% Y_{n+1} = Y_n + \um_{\mathbb{B}_n\cap \{U'_{n+1} \leq p_{n+1}\}}\gamma_{n+1} + \um_{\mathbb{B}_n \cap \{U'_{n+1} > p_{n+1}\}} \xi_{n+1} + \um_{\mathbb{B}_n^c}\xi_{n+1} \, ,    
% \end{equation*}
% and 
% \begin{equation*}
% Z_{n+1} = Z_n + \mathcal{P}_{\bD^c} \big(\um_{\mathbb{B}_n\cap \{U'_{n+1} \leq p_{n+1}\}}\gamma_{n+1} + \um_{\mathbb{B}_n \cap \{U'_{n+1} > p_{n+1}\}} \xi_{n+1} + \um_{\mathbb{B}_n^c}\xi_{n+1} \big) \, ,    
% \end{equation*}
% where $\mathbb{B}_n:=\{Y_n \notin \Rr_{n-1}^Y\}$. The following properties stem directly from the definitions:
% \begin{itemize}
%     \item $\{Y_i\}_{i \geq 0}$ is a $p_n$-\Nametwo{} on $\ZZ^d$ with drift direction $\ell_{\bD}$.
%     \item $\{Z_i\}_{i \geq 0}$ is an aperiodic random walk (lazy random walk) on $\ZZ^d$. Besides that the process $Z$ behaves like a lazy random walk in $\ZZ^{d-k}$.
%     \item For all $e_j \in \bD^c$ and $i \geq 0$, we have $Y_i \cdot e_j = Z_i \cdot e_j$. 
% \end{itemize}
The next lemma states another important property of the coupling.
%Hence we will prove that if the process $\{Z_i\}_{i \geq 0}$ visits a new site then the process $\{Y_i\}_{i \geq 0}$ will reach a new site too. 

\begin{lemma}\label{lemaZY}
Whenever $\{Y_i\}_{i \geq 0}$ visits a new site, $\{X_i\}_{i \geq 0}$ also visits a new one, which in turn implies that $|\Rr_n^X| \geq |\Rr_n^Y|$ for all $n \geq 0$. 
\end{lemma}
%
The proof of Lemma~\ref{lemaZY} will be postponed to the end of this section.
% Before we present the proof of Theorem~\ref{pn-ERW-d=>4} we need to present some useful auxiliary results. We set $\pi_d$ as the probability of a random walk in $\ZZ^d$ never returns to the origin. From Theorem 1 in~\cite{hamana2001large} we have the following result.
%
% \begin{theorem}[see~\cite{hamana2001large}, Theorem 1]\label{teo: RnZ>}
% Let $Z$ be an aperiodic random walk in $\ZZ^d$, then for every $\theta < \pi_d$ and $\theta' \in (\pi_d, 1]$ we have respectively 
% \begin{equation*}
% \lim_{n \to \infty} \PP[|\Rr_{n}^Z| \geq \theta n]= 1 \quad \text{ and } \quad  \PP[|\Rr_{n}^Z| \geq \theta' n] \leq e^{-c_{\theta'}n}\;,    
% \end{equation*}
% where $c_{\theta'}$ is a positive constant that depends of $\theta'$. 
% \end{theorem}

% Now we will analyze the behavior of the range of the $p_n$-\Nametwo{} in $\ZZ^d$, with $d \geq 3$ and $\beta =1/2$. The proof of this Proposition uses the same techniques from the case in $\ZZ^2$.
% \begin{proposition}\label{prop: RnX<d3}
% Let $\{X_n\}_{n \geq 0}$ be a $p_n$-\Nametwo{} in $\ZZ^d$, with $d \geq 3$ and $\beta = 1/2$ then we have that
% \begin{equation*}
%     \PP\left[ |\Rr_n ^X| \leq \delta n \right] \to 1 \quad \text{as } n \to \infty \,,
% \end{equation*}
% for any $\delta \in (\pi_{d}, 1] $, \cm{where $\pi_{d}$ }.
% \end{proposition}

% The proof of Proposition~\ref{prop: RnX<d3} will be postponed to end of this section. 
%Henceforth, without loss of generality, we shall assume $\CC = 1$, i.e., $p_n=n^{-1/2}$. 




\medskip 
Let us define the following random variable:
\begin{equation}
\label{Bn'}
J_n(\delta) := \sum_{i=1}^{\delta n} \um_{\{U_i \leq i^{-1/2} \}}\,,
\end{equation}
%As we saw in~\eqref{Kn<Jn}, we have $|K_n| \preceq |J_n|$. 
%
where $\delta \in (\pi_d, 1)$ and $\pi_d$ denotes the probability that the random walk with i.i.d. (with zero mean and finite variance) increments $\{\xi_i\}_{i\geq 0}$  never returns to the origin, and 
%
\begin{equation}\label{Fn'}
V_n(\delta'):=\sum_{i=n-\delta' n+1}^{n} \um_{\{U_i \leq i^{-1/2} \}}
% = \sum_{i=1}^n  \um_{\{U_i \leq i^{-1/2} \}} - \underbrace{\sum_{i=1}^{n-\delta' n} \um_{\{U_i \leq i^{-1/2} \}}}_{:= F'_n}
\,, 
\end{equation}
where $\delta' \in (0, \pi_{d-k})$ and $\pi_{d-k}$ denotes the probability of a random walk with i.i.d. increments (with zero mean and finite variance) given by the corresponding lazy random walk of the coupling never returning to the origin (recall that $k=|\bD|$ is the number of canonical directions spanned by the drift direction $\ell_{\bD}$  and $k\leq d-3$).
%
% and, by the same reasons we have~\eqref{Kn<Jn}, we obtain \comu{como antes a troca não é direta}
% \begin{equation}
% \sum_{i=n-|\Rr_n^X|+1}^{n} 1_{\{U_i \leq i^{-1/2} \}} \preceq \sum_{j=1}^{|\Rr_n^X|} 1_{\{U_{\tau_j} \leq \tau_j^{-1/2} \}} = |K_n| \,.    
% \end{equation}
% \com{\texttt{I would remove this and place it in the proof of Proposition~\ref{prop: bound_Kn}}
% Using Lemma \ref{lem: iid} and since $\{U_i\}_{i \ge 1}$ is i.i.d. we have that 
% \begin{equation}\label{eq: domJn<J'}
% J'_n := 
% \sum_{i=1}^{\delta n} \um_{\{U_{\varphi_i} \le \varphi_i^{-1/2} \}} \preceq J_n \, .
% \end{equation} 
% }
% %since $\{\varphi_i\}_{i \ge 1}$ is a sequence of of $\FF$-stopping times and $\{U_i\}_{i \ge 1}$ is i.i.d..
% % \sout{Now we define  the following random variable}
% % \com{No need to define $F_n$ and $F_n'$!}
% % \begin{equation}\label{Fn'}
% % F_n:=\sum_{i=1}^{n-\delta' n} \um_{\{U_i \leq i^{-1/2} \}}  
% %  = \sum_{i=1}^n  \um_{\{U_i \leq i^{-1/2} \}} - \sum_{i=n-\delta' n+1}^{n} \um_{\{U_i \leq i^{-1/2} \}}\,, 
% % \end{equation}
% %Since $\{\varphi_i\}_{i \ge 1}$ is a sequence of $\FF$-stopping times and the sequence $\{U_i\}_{i \ge 1}$ is i.i.d.,  one can note that
% \com{\texttt{I would remove this and place it in the proof of Proposition~\ref{prop: bound_Kn}}
% Using again Lemma \ref{lem: iid} and the fact that $\{U_i\}_{i \ge 1}$ is i.i.d., we obtain that
% \begin{equation}\label{eq: F'n_domsto}
% V_n = \sum_{i=n-\delta' n+1}^{n} \um_{\{U_i \leq i^{-1/2} \}} \preceq \sum_{i = 1}^{\delta' n} \um_{\{U_{\varphi_i} \leq (\varphi_i \wedge (n-i) )^{-1/2} \}} =: V'_n
% \,.
% \end{equation}
% }
% Let $\pi_d$ denote the probability of a random walk with i.i.d. increments (with zero mean and finite variance) given by the corresponding $\{\xi_i\}_{i\geq 0}$ never returning to the origin. We set the following random variable
% \begin{equation}\label{Bn'}
% |J'_n| := \sum_{i=1}^{\delta n}  1_{\{U_i \leq i^{-1/2} \}}  \quad \text{where } \delta \in (\pi_d,1] \,.   
% \end{equation}
% One can notice that by Proposition~\ref{prop:RangeERW} we obtain
% \begin{equation}\label{Bn<=Bn'} {\color{blue}
% \PP[|J_n| \le |J'_n| ] \rightarrow 1 \quad \text{as } n \to \infty \,.   } 
% \end{equation}

% An important  consequence of the  coupling described at the begging of this section between  a random walk  $Z$ and a process $X$ which is a $p_n$-\Nametwo{} in direction $\ell_{D_k}$ in $\ZZ^d$, where $d \ge 4$ (note that  $Z$ is a lazy random walk in $\ZZ^{d-k}$) is Lemma~\ref{lemaZY} which implies that $|\Rr_n^X| \ge |\Rr_n^Z|$ for all $n \ge 1$. Hence we have the following
% \begin{equation}\label{Fn<=}
% |F_n| \le \sum_{i=1}^{n - |\Rr_n^Z|}  1_{\{U_i \leq i^{-1/2} \}} \quad \text{for all } n \ge 1  \,.
% \end{equation}

% Let $\pi_{d-k}$ the probability of a random walk with i.i.d. increments (with zero mean and finite variance) given by the corresponding lazy random walk of the coupling never returning to the origin.  We now define the following random variable
% \begin{equation}\label{Fn'}
% |F'_n| := \sum_{i=1}^{n - \delta' n}  1_{\{U_i \leq i^{-1/2} \}}  \quad \text{where } \delta' \in (0,\pi_{d-k}) \,.
% \end{equation}
% Thus by Theorem~\ref{teo: RnZ>} (part (U)) and~\eqref{Fn<=} we obtain {\color{blue}
% \begin{equation}\label{Fn<=Fn'}
% \PP[|F_n| \le |F'_n| ] \rightarrow 1 \quad \text{as } n \to \infty \,.   
% \end{equation}}

The random variables $J_n$ and $V_n$ will be important to compute the constants $c_1$ and $c_2$ in the statement of Theorem~\ref{pn-ERW-d=>4} and below we state a few results about them. The proofs of these results are deferred to the end of this section. 
%Informally speaking, in Theorem~\ref{pn-ERW-d=>4} we obtain  that every limit point of the $p_n$-\Nametwo{} in direction $\ell_{\bD}$ suitably rescaled will be confined within a sort of ``cone'' region, with high probability (see the dashed region in  Figure~\ref{fig:cone}).


% \begin{figure}[h]
%     \centering
% \tikzset{every picture/.style={line width=0.45pt}} %set default line width to 0.75pt        

% \begin{tikzpicture}[x=0.45pt,y=0.45pt,yscale=-1,xscale=1]
% %uncomment if require: \path (0,313); %set diagram left start at 0, and has height of 313

% %Straight Lines [id:da8304803087608328] 
% \draw    (88,166) -- (182.5,283.5) ;
% %Straight Lines [id:da2794399174681552] 
% \draw    (131.5,219) -- (240.47,127.29) ;
% \draw [shift={(242,126)}, rotate = 139.92] [color={rgb, 255:red, 0; green, 0; blue, 0 }  ][line width=0.75]    (10.93,-3.29) .. controls (6.95,-1.4) and (3.31,-0.3) .. (0,0) .. controls (3.31,0.3) and (6.95,1.4) .. (10.93,3.29)   ;
% %Shape: Free Drawing [id:dp08864986808181619] 
% \draw  [color={rgb, 255:red, 0; green, 0; blue, 0 }  ,draw opacity=1 ][line width=0.75] [line join = round][line cap = round] (133,221) .. controls (132.67,221) and (132.33,221) .. (132,221) ;
% %Shape: Free Drawing [id:dp030738142652891653] 
% \draw  [color={rgb, 255:red, 0; green, 0; blue, 0 }  ,draw opacity=1 ][line width=0.75] [line join = round][line cap = round] (131,218) .. controls (131,215.81) and (131.39,207.39) .. (129,205) .. controls (127.73,203.73) and (131,211.8) .. (131,210) .. controls (131,208.74) and (128,194) .. (128,194) .. controls (128,194) and (128.47,194.73) .. (129,195) .. controls (130.07,195.54) and (131.15,193.85) .. (132,193) .. controls (134.46,190.54) and (133.07,201.11) .. (139,194) .. controls (140.93,191.69) and (137.13,188.43) .. (140,187) .. controls (141.7,186.15) and (143.38,191.16) .. (145,189) .. controls (147.51,185.65) and (146.57,175.4) .. (147,172) .. controls (147.18,170.57) and (148.8,174.43) .. (149,173) .. controls (149.33,170.69) and (148.67,168.31) .. (149,166) .. controls (149.11,165.23) and (152.37,171.78) .. (153,168) .. controls (153.7,163.8) and (152.52,155.69) .. (154,152) .. controls (154.35,151.12) and (155.48,153.22) .. (156,154) .. controls (157,155.49) and (159.24,156.35) .. (161,156) .. controls (163.76,155.45) and (164.39,141.83) .. (165,140) .. controls (165.86,137.42) and (174.34,137.91) .. (175,138) .. controls (175.47,138.07) and (175.85,139.45) .. (176,139) .. controls (177.02,135.94) and (173.72,132.56) .. (175,130) .. controls (178.22,123.57) and (178.1,119.01) .. (179,110) .. controls (179.07,109.34) and (183.98,113.09) .. (184,113) .. controls (184.32,111.71) and (183.68,110.29) .. (184,109) .. controls (184.34,107.63) and (185.74,111.37) .. (187,112) .. controls (189.11,113.05) and (191.76,110.25) .. (194,111) .. controls (196,111.67) and (198.74,114.69) .. (200,113) .. controls (203.84,107.88) and (200.1,93.9) .. (205,89) .. controls (205.6,88.4) and (215.1,94.52) .. (217,95) .. controls (221.16,96.04) and (217.77,93.07) .. (219,90) .. controls (219.49,88.77) and (225,85.56) .. (225,83) ;
% %Shape: Free Drawing [id:dp1632566533166797] 
% \draw  [color={rgb, 255:red, 0; green, 0; blue, 0 }  ,draw opacity=1 ][line width=0.75] [line join = round][line cap = round] (136,222) .. controls (134.95,222) and (134.05,223) .. (133,223) ;
% %Shape: Free Drawing [id:dp052118055244864125] 
% \draw  [color={rgb, 255:red, 0; green, 0; blue, 0 }  ,draw opacity=1 ][line width=0.75] [line join = round][line cap = round] (132,219) .. controls (131.67,218.67) and (130.85,219.55) .. (131,220) .. controls (131.09,220.28) and (138.31,224) .. (140,224) .. controls (141.99,224) and (141.41,217.8) .. (143,217) .. controls (143.99,216.51) and (147.08,221.92) .. (148,221) .. controls (148.6,220.4) and (148.68,214.95) .. (150,216) .. controls (151.67,217.33) and (152.09,220.05) .. (154,221) .. controls (156.62,222.31) and (155.9,218.9) .. (157,220) .. controls (157.99,220.99) and (157.02,221.62) .. (158,223) .. controls (160.06,225.88) and (163.49,227.49) .. (166,230) .. controls (166.24,230.24) and (166.93,230.33) .. (167,230) .. controls (167.33,228.37) and (166.73,226.64) .. (167,225) .. controls (167.36,222.83) and (168.24,225.88) .. (170,225) .. controls (170.75,224.62) and (171.07,217.38) .. (172,218) .. controls (173,218.67) and (173.15,220.15) .. (174,221) .. controls (174.79,221.79) and (177.08,218.23) .. (178,218) .. controls (179.65,217.59) and (181.48,219.76) .. (183,219) .. controls (183.84,218.58) and (184.09,217.23) .. (185,217) .. controls (186.96,216.51) and (190.55,225.07) .. (193,222) .. controls (194.51,220.11) and (195.71,214.32) .. (196,212) .. controls (196.17,210.64) and (196.14,206.93) .. (197,208) .. controls (199.06,210.58) and (201.85,217.51) .. (204,215) .. controls (206.23,212.39) and (206.14,204.1) .. (209,206) .. controls (212.53,208.35) and (214.34,212.86) .. (218,215) .. controls (221.62,217.11) and (222.14,213.09) .. (225,215) .. controls (226.28,215.85) and (231.94,222.53) .. (233,222) .. controls (234.79,221.11) and (233.25,217.98) .. (233,216) .. controls (232.91,215.26) and (232,213.25) .. (232,214) .. controls (232,218.16) and (233.37,217.73) .. (235,221) .. controls (235.21,221.42) and (235.96,222.47) .. (236,222) .. controls (236.33,218.01) and (235.5,213.97) .. (236,210) .. controls (236.17,208.64) and (236.43,212.75) .. (237,214) .. controls (238.11,216.45) and (239.44,218.81) .. (241,221) .. controls (241.43,221.61) and (242.95,222.74) .. (243,222) .. controls (243.35,216.68) and (242.52,211.31) .. (243,206) .. controls (243.15,204.31) and (243.19,209.51) .. (244,211) .. controls (245.25,213.29) and (247.49,214.88) .. (249,217) .. controls (249.27,217.38) and (249.97,218.47) .. (250,218) .. controls (250.33,213.01) and (249.67,207.99) .. (250,203) .. controls (250.08,201.8) and (252.46,202.07) .. (253,201) .. controls (254.46,198.08) and (254.03,191.19) .. (258,190) .. controls (260.24,189.33) and (262.77,190.67) .. (265,190) .. controls (270.02,188.49) and (267.92,178.51) .. (272,178) .. controls (274.32,177.71) and (276.69,177.67) .. (279,178) .. controls (281.25,178.32) and (281.47,183) .. (286,183) .. controls (286.81,183) and (287.81,180) .. (288,180) .. controls (289.16,180) and (289.79,187.62) .. (293,182) .. controls (294.16,179.97) and (292.62,177.3) .. (293,175) .. controls (293.12,174.26) and (293.27,176.85) .. (294,177) .. controls (296.88,177.58) and (301,176.75) .. (301,180) ;
% %Straight Lines [id:da19972782215377438] 
% \draw  [dash pattern={on 4.5pt off 4.5pt}]  (130,204) -- (143.5,218.75) ;
% %Straight Lines [id:da1175829093412124] 
% \draw  [dash pattern={on 4.5pt off 4.5pt}]  (142,189) -- (170,223.5) ;
% %Straight Lines [id:da006638707249905007] 
% \draw  [dash pattern={on 4.5pt off 4.5pt}]  (149,175.5) -- (192,222.5) ;
% %Straight Lines [id:da7126297398825152] 
% \draw  [dash pattern={on 4.5pt off 4.5pt}]  (152,154) -- (201,215.5) ;
% %Straight Lines [id:da45421712013082516] 
% \draw  [dash pattern={on 4.5pt off 4.5pt}]  (164,141) -- (221,215.5) ;
% %Straight Lines [id:da09458219023136327] 
% \draw  [dash pattern={on 4.5pt off 4.5pt}]  (175,131) -- (241,215.5) ;
% %Straight Lines [id:da9986396235508677] 
% \draw  [dash pattern={on 4.5pt off 4.5pt}]  (183,114.5) -- (252,200.5) ;
% %Straight Lines [id:da8110942219871895] 
% \draw  [dash pattern={on 4.5pt off 4.5pt}]  (203,109) -- (268,191.5) ;
% %Straight Lines [id:da06174397419817579] 
% \draw  [dash pattern={on 4.5pt off 4.5pt}]  (207,93) -- (280,178.5) ;
% %Straight Lines [id:da9135651086247871] 
% \draw  [dash pattern={on 4.5pt off 4.5pt}]  (132.5,193.5) -- (154,219.75) ;

% % Text Node
% \draw (250,106.4) node [anchor=north west][inner sep=0.75pt]    {$\ell_{\bD}{}$};
% % Text Node
% \draw (202,56.4) node [anchor=north west][inner sep=0.75pt]    {$W_{t} \cdot \ell _{\bD}{} +\ g( t)$};
% % Text Node
% \draw (270,198.4) node [anchor=north west][inner sep=0.75pt]    {$W_{t} \cdot \ell _{\bD}{} +\ f( t)$};

% \end{tikzpicture}
% \caption{ ``Cone'' region representation around the  direction $\ell_{\bD}$.}
% \label{fig:cone}

% \end{figure}



%\medskip 
%The first two results concern the asymptotic behavior of $J'_n/n^{1/2}$ and $F'_n/n^{1/2}$.  
%
% \begin{lemma}\label{B'_n}
% Let $\{J_n\}_{n \geq 1}$ be defined as in~\eqref{Bn'} with $\delta \in (\pi_d, 1)$ and  $\{V_n\}_{n \geq 1}$ be defined as in~\eqref{Fn'} with $\delta' \in (0, \pi_{d-k})$. Then, it holds that:
% \begin{itemize}
%     \item [(i)] 
%     \[ 
%     \lim_{n \to \infty} \frac{\EE[J_n]}{n^{1/2}} = 2\delta^{1/2} ;
%     \]
    
%     \item[(ii)] For any $\varepsilon > 0$, 
%     \begin{equation*}
%     \lim_{n \to \infty} \PP[|J_n - \EE[J_n]| > \varepsilon n^{1/2}] = 0 ;   
%     \end{equation*}
    
%      \item [(iii)]
%     \begin{equation*}
%     \lim_{n \to \infty} \frac{\EE[F_n]}{n^{1/2}} =  2(1 -\delta')^{1/2} ; 
%     \end{equation*}
    
%     \item[(iv)] For any $\varepsilon > 0$, 
%     \begin{equation*}
%     \lim_{n \to \infty} \PP[|F_n - \EE[F_n]| > \varepsilon n^{1/2}] = 0 ; 
%     \end{equation*}
    
%      \item [(v)]
%     \begin{equation*}
%     \lim_{n \to \infty} \frac{\EE[\sum_{i=1}^n  \um_{\{U_i \leq i^{-1/2} \}}]}{n^{1/2}} =  2 .
%     \end{equation*}
    
%     \item[(vi)] For any $\varepsilon > 0$, 
%     \begin{equation*}
%     \lim_{n \to \infty} \PP\Big[\Big|\sum_{i=1}^n  \um_{\{U_i \leq i^{-1/2} \}} - \EE\Big[\sum_{i=1}^n  \um_{\{U_i \leq i^{-1/2} \}}\Big]\Big| > \varepsilon n^{1/2}\Big] = 0  \,. 
%     \end{equation*}
% \end{itemize}
% \end{lemma}


\begin{lemma}\label{B'_n} \com{to be adjusted and should hold for any $d\geq 3$! }
Let $\{J_n(\delta)\}_{n \geq 1}$ be defined as in~\eqref{Bn'} with $\delta \in (\pi_d, 1)$ and  $\{V_n(\delta')\}_{n \geq 1}$ be defined as in~\eqref{Fn'} with $\delta' \in (0, \pi_{d-k})$. Then, it holds that:

\begin{align*}
i) &\quad \lim_{n \to \infty} \frac{\EE[J_n(\delta)]}{n^{1/2}} = 2\delta^{1/2} \quad \text{ and }\quad  \lim_{n \to \infty} \frac{\EE[V_n(\delta')]}{n^{1/2}} = 2-  2(1 -\delta')^{1/2}\,,
\\
   ii) &\quad \text{For any $\varepsilon > 0$,} 
   \\
   &\qquad \lim_{n \to \infty} \PP[|J_n(\delta) - \EE[J_n(\delta)]| > \varepsilon n^{1/2}] = 0\,, 
   % \text{ and } \lim_{n \to \infty} \PP[|V_n(\delta') - \EE[V_n(\delta')]| > \varepsilon n^{1/2}] = 0 \,.  
   \\
   &\quad \text{  and the same holds for $V_n(\delta')$. }
    \end{align*}
   
    %  \item [(iii)]
    % \begin{equation*}
    % \lim_{n \to \infty} \frac{\EE[V_n]}{n^{1/2}} = 2-  2(1 -\delta')^{1/2} ; 
    % \end{equation*}
    
    % \item[(iv)] For any $\varepsilon > 0$, 
    % \begin{equation*}
    % \lim_{n \to \infty} \PP[|V_n - \EE[V_n]| > \varepsilon n^{1/2}] = 0 ; 
    % \end{equation*}
\end{lemma}
\medskip 
%
From Lemma~\ref{B'_n} we obtain the following corollary. 

% \begin{corollary}\label{B'n->p}
% Let $\{J_n\}_{n \geq 1}$ be defined in~\eqref{Bn'} with $\delta \in (\pi_d, 1)$ and  $\{F_n\}_{n \geq 1}$ be defined in~\eqref{Fn'} with $\delta' \in (0, \pi_{d-k})$. Then, it holds that: 
% \begin{itemize}
%     \item [i)] 
%     \begin{equation*}
%         \frac{J_n}{n^{1/2}} \xrightarrow[n \to \infty]{} 2\delta^{1/2} \ \text{ in probability;}
%     \end{equation*}
    
%     \item [ii)]
%     \begin{equation*}
%         \frac{F_n}{n^{1/2}} \xrightarrow[n \to \infty]{} 2(1- \delta')^{1/2} \  \text{ in probability;}
%     \end{equation*}
    
%     \item [iii)]
%     \begin{equation*}
%          \frac{\sum_{i=1}^{n} \um_{\{U_i \le i^{1/2}\}}}{n^{1/2}}  \xrightarrow[n \to \infty]{} 2 \ \text{ in probability.}
%     \end{equation*}
% \end{itemize}
% \end{corollary}

 
\begin{corollary}\label{B'n->p}
\com{to be adjusted and should hold for any $d\geq 3$.....}Let $\{J_n(\delta)\}_{n \geq 1}$ be defined in~\eqref{Bn'} with $\delta \in (\pi_d, 1)$ and  $\{V_n(\delta')\}_{n \geq 1}$ be defined in~\eqref{Fn'} with $\delta' \in (0, \pi_{d-k})$. Then, it holds that: 
    \begin{equation*}
        \frac{J_n(\delta)}{n^{1/2}} \xrightarrow[n \to \infty]{} 2\delta^{1/2} \quad \text{ and } \quad  \frac{V_n(\delta')}{n^{1/2}} \xrightarrow[n \to \infty]{} 2- 2(1- \delta')^{1/2} \quad   \text{ in probability}\,.
    \end{equation*}
    % \begin{equation*}
    %  \tag{ii}   \frac{V_n}{n^{1/2}} \xrightarrow[n \to \infty]{} 2- 2(1- \delta')^{1/2} \  \text{ in probability}\,.
    % \end{equation*}
\end{corollary}

\medskip 
%
Relying on  Corollary~\ref{B'n->p}, we are able to prove the following result: 

\begin{lemma}\label{lem: tightaux} \com{to be adjusted and should hold for any $d\geq 3$.....}
Let $\{\Hat{J}^{n}_{\cdot}(\delta)\}_{n\geq 1}$ and $\{\Hat{V}^n_{\cdot}(\delta')\}_{n\geq 1}$ be respectively the sequences of processes in $C_{\Rs}[0, \infty)$ corresponding to  
$J_n(\delta)$ with $\delta \in (\pi_d, 1)$ and  $V_n(\delta')$ with $\delta' \in (0, \pi_{d-k})$  defined in~\eqref{Bn'} and \eqref{Fn'}, respectively.   Then, it holds that 
%
\begin{itemize}
    \item [i)]
    $\{\Hat{J}^n_{\cdot}(\delta)/n^{1/2}\}_{n\geq 1}$ converges in distribution as random elements of $C_{\Rs}[0, \infty)$ to the deterministic function $t \mapsto 2(\delta t)^{1/2}$, $t\ge 0$.  
    \item[ii)] $\{\Hat{V}^n_{\cdot}(\delta')/n^{1/2}\}_{n\geq 1}$
%$$
%\left\{\frac{\sum_{i=1}^{\lfloor n \cdot \rfloor}  \um_{\{U_i \leq i^{-1/2} \}} - F_{\lfloor n \cdot \rfloor}}{n^{1/2}}\right\}_{t\geq 0} 
%$$
converges in distribution as random elements of $C_{\Rs}[0, \infty)$ to the deterministic function $t \mapsto 2t^{1/2}(1\!-\!(1\!-\!\delta')^{1/2})$, $t\ge 0$.  
\end{itemize}
\end{lemma}

% The proof of Lemma~\ref{lem: tightaux} will be postponed to the end of this section.

% The next result states that $(|K_{\lfloor n\cdot \rfloor}|/n^{1/2})_{n\ge 1}$ is tight in $C_{\Rs}[0, \infty)$. %In the proof of the Theorem~\ref{pn-ERW-d=>4}, it will be more clear the importance of this result.


% \begin{lemma}\label{lem: Knt/n1/2tig}
% The sequence of processes $\{\Hat{K}^n_{\cdot }/n^{1/2}\}_{n\ge 1}$ is tight  in  $C_{\Rs}[0, \infty)$.
% \end{lemma}

By Lemma~\ref{Jntight} (item $i)$), for $d\geq 4$, every subsequence of $\{\Hat{K}^n_{\cdot }/n^{1/2}\}_{n\ge 1}$ has a limit point. The next result states that those limits points are concentrated on paths confined between the curves  $t \mapsto 2 (1-\sqrt{1 -  \pi_{d-k}}) \sqrt{t}$ and $t \mapsto 2  \sqrt{ \pi_d \, t}$.
% The proof of Lemma~\ref{lem: Knt/n1/2tig} will be postponed at the end of this section.

% \begin{proposition}\label{prop: bound_Kn}
% %Let $|K_{\lfloor n \cdot \rfloor}|/n^{1/2}$ be a sequence of processes in $C_{\Rs}[0, \infty)$. 
% If $H_{\cdot}$ is a limit point of $\{\Hat{K}^n_{\cdot }/n^{1/2}\}_{n\ge 1}$, then
% \begin{equation*}
% \PP \left[\forall t \in [0,\infty): 2t^{1/2}(1-(1 -  \delta')^{1/2}) \le H_t \le 2(t \delta)^{1/2} \right] = 1 \, ,  
% \end{equation*}
% where $\delta'$ and $\delta$ are positive constants such that $\delta' \in (0, \pi_{d-k})$ and $\delta \in (\pi_d, 1)$. 
% \end{proposition}

\begin{proposition}\label{prop: bound_Kn}
\com{This proposition should be restated separating upper and lower bound, the former is proved using Proposition~\ref{prop:RangeERW} and the second using Proposition~\ref{prop:RangeERW_lower}.....}
%Let $|K_{\lfloor n \cdot \rfloor}|/n^{1/2}$ be a sequence of processes in $C_{\Rs}[0, \infty)$. 
If $H_{\cdot}$ is a limit point of $\{\Hat{K}^n_{\cdot }/n^{1/2}\}_{n\ge 1}$, then
\begin{equation*}
\PP \left[\forall t \in [0,\infty): 2t^{1/2}(1-(1 -  \pi_{d-k})^{1/2}) \le H_t \le 2(t \pi_d)^{1/2} \right] = 1 \, . 
\end{equation*}
\end{proposition} 

\begin{remark}\label{rem:conjecture}
If Conjecture~\ref{conj_range} were true, we would be able to strengthen the claim of Proposition~\ref{prop: bound_Kn} and obtain that \com{for any $d\geq 3$.....}
\begin{equation*}
\PP \left[\forall t \in [0,\infty): 2t^{1/2}(1-(1 -  \pi_{d})^{1/2}) \le H_t \le 2(t \pi_d)^{1/2} \right] = 1 \, . 
\end{equation*}
The proof would then avoid the use of the coupling and consequently, there will be no dimensional constraint (this is why the lower bound would be given in terms of $\pi_d$ rather than the current $\pi_{d-k}$).
\end{remark}


% The proof of Proposition~\ref{prop: bound_Kn} will be postponed at the end of this section.


We now have all the auxiliaries results to prove Theorem~\ref{pn-ERW-d=>4}. The main idea is to use~\eqref{xn-incremento2} and then analyze separately the rescaled sum terms. We show  that the sequences corresponding to both terms are tight in $C_{\Rs^d}[0, \infty)$, consequently we obtain that $\{\Hat{B}_{\cdot}^n\}_{n\geq 1}$ is also tight. Finally, we  describe the processes which stochastically dominate the limit points of $\{\Hat{B}_{\cdot}^n\}_{n\geq 1}$. The strategy is similar to that used in the proof of Theorem~\ref{pn-ERW-d=2}, but here the term representing the drift direction does not go to zero. The random variables $J_n$ and $V_n$ play an important role in controlling this non-vanishing term.  
%
\begin{proof}[Proof of Theorem~\ref{pn-ERW-d=>4}]
We begin showing that 
$\{{\Hat{B}}_{\cdot}^n\}_{n\ge 1}$ is tight in $C_{\Rs^d}[0, \infty)$. By Remark~\ref{rem:conver}-$b)$ it suffices  to show that $\{{\Hat{B}}_{\cdot}^n\}_{n\ge 1}$ is tight in $C_{\Rs^d}[0, T)$, for all $T>0$.

%
Using ~\eqref{xn-incremento2} we can rewrite the process $B_t^n$ as
\begin{align}\label{p_n-ERW_incrementos_d=>4}
\begin{split} 
& \frac{1}{n^{1/2}}\sum_{i=1}^{\lfloor nt \rfloor} \xi_i + \frac{1}{n^{1/2}} \sum_{i=1}^{\lfloor nt \rfloor} \um_{\{E_{i-1}^c \cap \{ U_i \leq i^{-1/2}\}\}} (\gamma_i - \xi_i) \, , \ \forall \, t \ge 0\,.
% \\
% & = \frac{1}{n^{1/2}}\sum_{i=1}^{\lfloor nt \rfloor} \xi_i + \frac{1}{n^{1/2}} \sum_{i \in K_{\lfloor nt \rfloor}}  (\gamma_i - \xi_i) 
% \\
% & 
% = \frac{1}{n^{1/2}}\sum_{i=1}^{\lfloor nt \rfloor} \xi_i + \frac{|K_{\lfloor nt \rfloor}|}{n^{1/2}} \sum_{i \in K_{\lfloor nt \rfloor}} \frac{ (\gamma_i - \xi_i)}{|K_{\lfloor nt \rfloor}|}\,.
\end{split}
\end{align}
By Donsker's Theorem the first term of~\eqref{p_n-ERW_incrementos_d=>4} converges in distribution as random elements of $C_{\Rs^d}[0, \infty)$ to a Brownian Motion, i.e., 
\begin{equation}\label{xi_i->W2_d=>4}
 \Big\{ \frac{1}{n^{1/2}}\sum_{i=1}^{\lfloor nt \rfloor} \xi_i \Big\}_{t\ge 0} \xrightarrow[n \to \infty]{\mathcal{D}} \{ W_{t} \}_{t\ge 0} \,,
\end{equation}
where $W_{\cdot}$ is a Brownian Motion in dimension $d$ with zero-mean vector and covariance matrix $\EE[\xi_1 \xi_1^T]$.
Then,  to show that $\{{\Hat{B}}_{\cdot}^n\}_{n\ge 1}$ is tight in $C_{\Rs^d}[0, T]$ for all $T>0$, it is enough to prove that the second term in~\eqref{p_n-ERW_incrementos_d=>4} is tight in $C_{\Rs^d}[0, T]$. 
Indeed, $\{\Hat{B}_{\cdot}^n\}_{n\geq 1}$ 
 would be the sum of two tight sequences of processes, thus also tight. %(see Lemma~\ref{lem: tight}) \cm{colocar uma referencia ou deixar?}.
%
% Then by~\cite[Theorem 4.10 in Chapter 2]{karatzas2012brownian}, since $B_{\cdot}^n$ is  tight  in $C_{\Rs^d}[0, T]$ for all $T>0$ with the topology of uniform convergence in the compacts, $B_{\cdot}^n$ is  tight  in $C_{\Rs^d}[0, \infty)$.

In order to show that the second term in~\eqref{p_n-ERW_incrementos_d=>4} is tight in $C_{\Rs^d}[0, T]$ for all $T>0$,  we use Remark~\ref{rem:conver}-$c)$.
% \cite[Theorem 7.3]{billingsley1999probability} which provides two sufficient conditions for tightness.
Recall the definition of $D_{\lfloor n t \rfloor}$ from the statement of Lemma \ref{lem: tgD}, remembering that here we have set $\mathcal{C} = 1$ and that we are under distinct hypotheses from those of Section \ref{prova-pn-WGERW}. We will show that $\{\Hat{D}^n_\cdot\}_{n\ge 1}$ is tight.
%$$
%D_{\lfloor n t \rfloor}:= \frac{1}{n^{1/2}} \sum_{i=1}^{\lfloor nt \rfloor} \um_{\{E_{i-1}^c \cap \{ U_i \leq i^{-1/2}\}\}} (\gamma_i - \xi_i) \,.$$
The first condition in Remark~\ref{rem:conver}-$c)$ is satisfied, since  $\Hat{D}^n_0 \equiv 0$, for all $n\geq 1$. To prove that $\{\Hat{D}^n_\cdot\}_{n\ge 1}$ satisfies the second condition in Remark~\ref{rem:conver}-$c)$  we  use~\cite[Corollary on page 83]{billingsley1999probability} which states that the second condition of ~\cite[Theorem 7.3]{billingsley1999probability} holds if, for every positive $\varepsilon$ and $\eta$, there exists a $\phi \in (0,1)$, and an integer $n_0$ such that
\begin{equation}\label{eq: PnA}
\frac{1}{\phi} \, \PP \Big[ \sup_{t \le s \le t + \phi} \big\|\Hat{D}^n_{s} - \Hat{D}^n_{t} \big\|  \ge \varepsilon \Big]  \le \eta \quad \forall n \ge n_0\,.
\end{equation}
%\begin{equation}\label{eq: PnA}
%\frac{1}{\phi} P_n \Big[f \in C_{\Rs^d}[0,T] : \sup_{t \le s \le t+\phi} |f(s) - f(t)| \ge \varepsilon \Big] \le \eta \quad \forall n \ge n_0\,,
%\end{equation}
%where the probability measure $P_n$ on $C_{\Rs^d}[0,T]$  is the distribution of $D_{\lfloor n \cdot \rfloor}$. 
%In order to show that \eqref{eq: PnA} actually holds, note  that, if we define 
%\begin{equation}\label{eq:Atphi}
%A_t(\varepsilon, \phi):= \Big\{f \in C_{\Rs}[0,T] : \sup_{t \le s \le t+\phi} |f(s) - f(t)| \ge \varepsilon \Big\}\,,
%\end{equation} 
%the left-hand side of \eqref{eq: PnA} reduces to
% In order to show that actually holds,  let us define 
% $A_t(\varepsilon, \phi):= \{f \in C_{\Rs^d}[0,T] : \sup_{t \le s \le t+\phi} |f(s) - f(t)| \ge \varepsilon \}$. 
% Hence we obtain the following 
Note that the probability in \eqref{eq: PnA} is bounded from above by
\begin{equation}
\label{eq: PnPP}
%\PP\left[ D_{\lfloor n \cdot \rfloor} \in A_t(\varepsilon, \phi) \right] = 
%\PP \Big[ \sup_{t \le s \le t + \phi} \big\|\Hat{D}^n_{s} - \Hat{D}^n_{t} \big\|  \ge \varepsilon \Big] 
%& & = \PP \left[ \sup_{t \le s \le t+ \phi} \left\|\frac{\sum_{i=\lfloor nt \rfloor + 1}^{\lfloor ns \rfloor} 1_{\{E_{i-1}^c \cap \{ U_i \leq i^{-1/2}\}\}} (\gamma_i - \xi_i)}{n^{\frac{1}{2}}}  \right\| \ge \varepsilon \right] 
\PP\Big[ \sup_{t \le s \le t+ \phi} \Big\| \sum_{i=\lfloor nt \rfloor}^{\lfloor ns \rfloor + 1} \um_{E_{i-1}^c \cap \{ U_i \leq i^{-1/2}\}} (\gamma_i - \xi_i)  \Big\| \ge \varepsilon n^{\frac{1}{2}} \Big] \, ,
%\\
%& = \PP \left[ \sum_{i = \lfloor nt \rfloor + 1}^{\lfloor n(t + \phi) \rfloor} 1_{\{E_{i-1}^c \cap \{U_i \le i^{-\frac{1}{2}}\}\}} \ge \varepsilon n^{\frac{1}{2}} \right] \le \PP \left[ \sum_{i = \lfloor nt \rfloor + 1}^{\lfloor n(t + \phi) \rfloor} 1_{ \{U_i \le i^{-\frac{1}{2}}\}} \ge \varepsilon n^{\frac{1}{2}} \right]   
\end{equation}
%
% Now we only analyze the process inside the probability measure in~\eqref{eq: PnPP}. We will find an upper bound for this process for all the trajectory. 
for all $s \in [t, t+ \phi]$ and 
\begin{eqnarray}
\label{eq: trajetoria}
\Big\| \sum_{i=\lfloor nt \rfloor}^{\lfloor ns \rfloor + 1} \um_{E_{i-1}^c \cap \{ U_i \leq i^{-1/2}\}} (\gamma_i - \xi_i)  \Big\| &\le & \sum_{i=\lfloor nt \rfloor}^{\lfloor ns \rfloor + 1} \big\|  \um_{ \{ U_i \leq i^{-1/2}\}} (\gamma_i - \xi_i)  \big\| \nonumber
 \\
% \le  \sum_{i=\lfloor nt \rfloor + 1}^{\lfloor ns \rfloor} \big\|  \um_{ \{ U_i \leq i^{-1/2}\}} (\gamma_i - \xi_i)  \big\| 
& \le & \sum_{i=\lfloor nt \rfloor}^{\lfloor ns \rfloor + 1}  \um_{ \{ U_i \leq i^{-1/2}\}} 2K  \,, 
\end{eqnarray}
where the second inequality follows from  triangle inequality and the last from Condition~\ref{condition_A}. 
%
Then from~\eqref{eq: PnPP} and ~\eqref{eq: trajetoria} we obtain that
\begin{eqnarray}\label{eq: PnA<1}
\lefteqn{\!\!\!\!\!\!\!\! \PP \Big[ \sup_{t \le s \le t + \phi} \big\|\Hat{D}^n_{s} - \Hat{D}^n_{t} \big\|  \ge \varepsilon \Big] \le 
\PP \Big[ \sum_{i = \lfloor nt \rfloor}^{\lfloor n(t + \phi) \rfloor +1} \um_{ \{U_i \le i^{-\frac{1}{2}}\}} 2K \ge \varepsilon n^{\frac{1}{2}} \Big] } \nonumber
\\
& & \le \exp\Big(\frac{-\varepsilon n^{\frac{1}{2}}}{2K}\Big)\EE\Big[\exp\Big( \sum_{i = \lfloor nt \rfloor}^{\lfloor n(t + \phi)\rfloor+1} \um_{ \{U_i \le i^{-\frac{1}{2}}\}} \Big) \Big] 
%& & \le \exp\Big(\frac{-\varepsilon n^{\frac{1}{2}}}{2K}\Big) \prod_{i = \lfloor nt \rfloor + 1}^{\lfloor n(t + \phi)\rfloor} \EE\left[\exp\left(  \um_{ \{U_i \le i^{-\frac{1}{2}}\}} \right) \right] \,,
\end{eqnarray}
where in the last inequality we have used exponential Markov's inequality.
%
Setting  $c = \varepsilon/(2K)$ an  continuing the computation in~\eqref{eq: PnA<1} we obtain that
\begin{eqnarray}\label{eq: PnA<}
\lefteqn{\frac{1}{\phi} \PP \Big[ \sup_{t \le s \le t + \phi} \big\|\Hat{D}^n_{s} - \Hat{D}^n_{t} \big\|  \ge \varepsilon \Big] \le \frac{1}{\phi} e^{-c n^{\frac{1}{2}}} \prod_{i = \lfloor nt \rfloor}^{\lfloor n(t + \phi)\rfloor + 1} \EE\left[\exp\left(  \um_{ \{U_i \le i^{-\frac{1}{2}}\}} \right) \right]} \nonumber \\
& & = \frac{1}{\phi} e^{-c n^{\frac{1}{2}}} \prod_{i = \lfloor nt \rfloor}^{\lfloor n(t + \phi)\rfloor + 1} \left(1+ \frac{e-1}{i^{\frac{1}{2}}} \right) \le \frac{1}{\phi} e^{-c n^{\frac{1}{2}}} \prod_{i = \lfloor nt \rfloor}^{\lfloor n(t + \phi)\rfloor + 1} \exp\left(\frac{e-1}{i^{\frac{1}{2}}} \right)  
% \\
% & \le \frac{1}{\phi} e^{-c n^{\frac{1}{2}}}  \exp\left(\sum_{i = \lfloor nt \rfloor + 1}^{\lfloor n(t + \phi)\rfloor}\frac{e-1}{i^{\frac{1}{2}}} \right) 
\nonumber \\
& & \le \frac{1}{\phi} \exp(-c n^{\frac{1}{2}})\exp\left(2(e-1)(\sqrt{n(t + \phi)} - \sqrt{nt} + 2) \right) \,,  
\end{eqnarray}
where %the second inequality follows by the moment generating function of a Bernoulli  and the third by the fact that $1+x<e^x$ for all $x$.  
the last inequality above follows from noticing that 
\begin{align*}
\sum_{i = \lfloor nt \rfloor}^{\lfloor n(t + \phi)\rfloor + 1}\frac{1}{i^{\frac{1}{2}}} & \le \int_{nt - 1}^{n(t+\phi) + 1} x^{-1/2}dx \le 2\left(\sqrt{n(t + \phi)} - \sqrt{nt} + 2\right) \,.
\end{align*}
Therefore, in order to show that \eqref{eq: PnA}  holds, it remains to show that for every positive $\varepsilon$ (recall that $c=\varepsilon/(2K)$) and $\eta$,  there exists a $\phi \in (0,1)$, and an integer $n_0$ such that
\begin{equation}\label{exp<eta}
\frac{1}{\phi} \exp(-c n^{\frac{1}{2}})\exp\bigg(2(e-1)\left(\sqrt{n}\left(\sqrt{t + \phi} - \sqrt{t} \right) + 2 \right) \bigg) \le \eta \quad \forall n \ge n_0 \,.
\end{equation}
We accomplish this choosing $\phi \in (0,1)$ sufficiently small such that $\sqrt{t+\phi} - \sqrt{t}< c/4(e-1)$. Then, for every $\eta>0$, choosing $n$ sufficiently large, we obtain that~\eqref{exp<eta} is satisfied. 
% One can see that, since we have $\phi \in (0,1)$,  for all $\hat{\varepsilon} > 0$, there exists a $\phi' > \phi$ such that $|\sqrt{t+\phi'} - \sqrt{t}|< \hat{\varepsilon}$. Now we can choose $\hat{\varepsilon} = c/4(e-1)$ and we obtain, for a large enough $n$, that~\eqref{exp<eta} is fulfilled for all $\eta$.
Consequently, we have that~\eqref{eq: PnA} is satisfied and thus $\{\Hat{D}^n_{\cdot}\}_{n \ge 1}$ is tight in $C_{\Rs^d}[0,T]$. 
 
% Since the process $B_{\cdot}^n$  is the sum of two tight processes in $C_{\Rs^d}[0, T]$,  we obtain that $B_{\cdot}^n$ is a tight process in $C_{\Rs^d}[0, T]$ for all $T>0$ as a simple exercise. %(see Lemma~\ref{lem: tight}) \cm{colocar uma referencia ou deixar?}.

% Now by~\cite[Theorem 4.10 in Chapter 2]{karatzas2012brownian} one can see that since $B_{\cdot}^n$ is  tight  in $C_{\Rs^d}[0, T]$ for all $T>0$ with the topology of uniform convergence in the compacts then  $B_{\cdot}^n$ is  tight  in $C_{\Rs^d}[0, \infty)$.

We now prove the second part of the theorem, namely the stochastic domination. Let us begin rewriting $B_t^n$ again in the slightly different form
\begin{equation}
 \label{p_n-ERW_incrementos_d=>4-bis}
% & = \frac{1}{n^{1/2}}\sum_{i=1}^{\lfloor nt \rfloor} \xi_i + \frac{1}{n^{1/2}} \sum_{i=1}^{\lfloor nt \rfloor} \um_{\{E_{i-1}^c \cap \{ U_i \leq i^{-1/2}\}\}} (\gamma_i - \xi_i) 
% % \\
% % & = \frac{1}{n^{1/2}}\sum_{i=1}^{\lfloor nt \rfloor} \xi_i + \frac{1}{n^{1/2}} \sum_{i \in K_{\lfloor nt \rfloor}}  (\gamma_i - \xi_i) 
% \\
% & 
\frac{1}{n^{1/2}}\sum_{i=1}^{\lfloor nt \rfloor} \xi_i + \frac{|K_{\lfloor nt \rfloor}|}{n^{1/2}} \sum_{i \in K_{\lfloor nt \rfloor}} \frac{ (\gamma_i - \xi_i)}{|K_{\lfloor nt \rfloor}|}\,.
\end{equation}
As already mentioned the first term converges to a Brownian Motion (see \eqref{xi_i->W2_d=>4}).  We now analyze the second term in~\eqref{p_n-ERW_incrementos_d=>4-bis}. By Proposition~\ref{prop: bound_Kn} we have  that $|K_{\lfloor nt \rfloor}| \to \infty$ as $n \to \infty$ almost surely. 
% \begin{align}\label{sup_
% inf_Kn}
% \PP \left[\forall t \in [0,\infty): 2t^{1/2}(1-(1 -  \pi_{d-k})^{1/2}) \le H_t \le 2(t \pi_d)^{1/2}  \right]  = 1 \,,
% \end{align}
% where $\{H_t\}_{t\ge 0}$ is a limit point of a subsequence of $\{ \Hat{K}^n_{\cdot}/n^{1/2}\}_{n\geq 1}$.
%\com{Above $\delta --> \delta' $ and  $\hat\delta --> \dekta $ right?}
%$\delta''$, $\Hat{\delta}$, $\delta'$ and $\delta$ are positive constants such that $\delta'' \in (0, \delta')$, $\Hat{\delta} \in (\delta, 1]$, $\delta \in (\pi_d, 1]$ and $\delta' \in (0, \pi_{d-k})$.
% \begin{equation}\label{Fn<Kn<Bn}
% \frac{1}{n^{1/2}} \sum_{i=1}^{\lfloor nt \rfloor}  1_{\{U_i \leq i^{-1/2} \}} - \frac{|F_{\lfloor nt \rfloor}|}{n^{1/2}} \preceq \frac{|K_{\lfloor nt \rfloor}|}{n^{1/2}} \preceq \frac{|J_{\lfloor nt \rfloor}|}{n^{1/2}}\,.
% \end{equation}
% Thus, by~\eqref{Bn<=Bn'} and~\eqref{Fn<Kn<Bn}, one can see that
% \begin{align}\label{eq:Kn<B'n}
% \begin{split}
% & \PP \left[ \forall t \in [0,\infty) : \frac{|K_{\lfloor nt \rfloor}|}{n^{1/2}} \le \frac{|J'_{\lfloor nt \rfloor}|}{n^{1/2}} \right] \ge \PP\left[ \forall t \in [0,\infty]: \frac{|J_{\lfloor nt \rfloor}|}{n^{1/2}} \le \frac{|J'_{\lfloor nt \rfloor}|}{n^{1/2}} \right] 
% \\
% & \to 1 \quad \text{as } n \to \infty\,.
% \end{split}
% \end{align}
% Now we will obtain, in the same sense of~\eqref{eq:Kn<B'n}, a lower bound. Hence by~\eqref{Fn<=Fn'} and~\eqref{Fn<Kn<Bn} we have
% \begin{equation}\label{eq: Kn>s-F'n}
% \begin{split}
% & \PP \left[\forall t \in [0,\infty) : \frac{\sum_{i=1}^{\lfloor nt \rfloor}  1_{\{U_i \leq i^{-1/2} \}}}{n^{1/2}} - \frac{|F'_{\lfloor nt \rfloor}|}{n^{1/2}} \le \frac{|K_{\lfloor nt \rfloor}|}{n^{1/2}}  \right] 
% \\
% & \ge \PP\left[\forall t \in [0,\infty): \frac{\sum_{i=1}^{\lfloor nt \rfloor}  1_{\{U_i \leq i^{-1/2} \}} - |F'_{\lfloor nt \rfloor}|}{n^{1/2}} \le \frac{\sum_{i=1}^{\lfloor nt \rfloor}  1_{\{U_i \leq i^{-1/2} \}} - |F_{\lfloor nt \rfloor}|}{n^{1/2}} \right] 
% \\
% & \to 1 \quad \text{as } n \to \infty \,.
% \end{split}
% \end{equation}
% Hence by Lemma~\ref{lem: tightaux}, Corollary~\ref{B'n->p}, ~\eqref{eq:Kn<B'n} and~\eqref{eq: Kn>s-F'n} we obtain that for all $t_0>0$
% \begin{align}\label{sup_
% inf_Kn}
% \PP \left[\forall t \in [t_0,\infty): 2t^{1/2}(1-(1 -  \delta')^{1/2}) \le \frac{|K_{\lfloor nt \rfloor}|}{n^{1/2}} \le 2(t \delta)^{1/2} \right] \to 1 \,,
% \end{align}
% as $n$ goes to infinity where $\delta$ and $\delta'$ are positive constants such that $\delta \in (\pi_d, 1]$ and $\delta' \in (0, \pi_{d-k})$.
%
Furthermore,  the sequence of random vectors $\{\gamma_{\varphi_i} -\xi_{\varphi_i}\}_{i \geq 1}$ is i.i.d. and has the same distribution of $\{\gamma_{i} -\xi_{i}\}_{i \geq 1}$, which is i.i.d. too (see Lemma~\ref{lem: iid}). Thus, we can use~\cite[Theorem 8.2 item (iii)]{gut2005probability} to conclude that 
\begin{equation}\label{eq:prob02_d=>4}
   \sum_{i \in K_{\lfloor nt \rfloor}} \frac{ (\gamma_i - \xi_i)}{|K_{\lfloor nt \rfloor}|} \xrightarrow[n \to \infty]{} \EE[\gamma_1 -\xi_1]  = \EE[\gamma_1] \, \text{ a.s..}
\end{equation}
Recall that $\{\gamma_n\}_{n \ge 1}$ is an i.i.d. sequence of random vectors and  $\lambda  \le \EE[\gamma_i \cdot \ell_{\bD}] \le K$ for all $i \ge 1$, we set $\mu_{\gamma} := \EE[\gamma_i \cdot \ell_{\bD}]$ for all $i \ge 1$. 
%\cm{Now we will prove that the second sum portion in~\eqref{p_n-ERW_incrementos_d=>4} is a tight process in $C[0,T]$. For that we will use again Theorem 7.3 in~\cite{billingsley1999probability}
%\begin{equation*}
%\begin{split}
%& \PP\left[ \sup_{t \le s \le t+ \phi} \left\| \sum_{i=\lfloor nt \rfloor + 1}^{\lfloor ns \rfloor} 1_{ \{ U_i \leq i^{-1/2}\}} (\gamma_i - \xi_i)  \right\| \ge \varepsilon n^{\frac{1}{2}} \right] \le
%\\
%& \PP\left[ \sup_{t \le s \le t+ \phi} \left( \sum_{i=\lfloor nt \rfloor + 1}^{\lfloor ns \rfloor} \left\|  1_{ \{ U_i \leq i^{-1/2}\}} (\gamma_i - \xi_i)  \right\| \right) \ge \varepsilon n^{\frac{1}{2}} \right] \le
%\\
%& \PP\left[ \sup_{t \le s \le t+ \phi} \left( \sum_{i=\lfloor nt \rfloor + 1}^{\lfloor ns \rfloor}   1_{ \{ U_i \leq i^{-1/2}\}} 2K \right) \ge \varepsilon n^{\frac{1}{2}} \right] \le
%\\
%& \PP\left[  \sum_{i=\lfloor nt \rfloor + 1}^{\lfloor n(t+\phi) \rfloor}   1_{ \{ U_i \leq i^{-1/2}\}} 2K  \ge \varepsilon n^{\frac{1}{2}} \right] \,.
%\end{split}    
%\end{equation*}}
%
From Proposition~\ref{prop: bound_Kn} and~\eqref{eq:prob02_d=>4} we obtain that the linearly interpolated version of the sequence 
$$
\left\{ \frac{|K_{\lfloor n \cdot \rfloor}|}{n^{1/2}} \sum_{i \in K_{\lfloor nt \rfloor}} \frac{ (\gamma_i - \xi_i) \cdot \ell_{\bD}}{|K_{\lfloor n \cdot \rfloor}|} \right\}_{n\ge 1}\,, 
$$
is tight in $C_{\Rs}[0, \infty)$ and any of its limit points $\widetilde H_t$ satisfies
\begin{equation}\label{eq: c1c2}
\PP\left[\forall t \in [0,\infty): 
2\mu_{\gamma} t^{\frac{1}{2}}(1 - (1 - \pi_{d-k})^{\frac{1}{2}})
\le \widetilde H_t \le 2\mu_{\gamma}(t \pi_d)^{1/2} \right] = 1\,.
\end{equation}

%\begin{equation}\label{eq: c1c2}
%\begin{split}
%& \PP\left[\forall t \in [0,\infty): \frac{|K_{\lfloor nt \rfloor}|}{n^{1/2}} \sum_{i \in K_{\lfloor nt \rfloor}} \frac{ (\gamma_i - \xi_i) \cdot \ell_{D_k}}{|K_{\lfloor nt \rfloor}|}  \le 2\mu_{\gamma}(t\delta)^{1/2} \right] \to 1 \quad \text{and}
%\\
%& \PP\left[\forall t \in [0,\infty): \frac{|K_{\lfloor nt \rfloor}|}{n^{1/2}} \sum_{i \in K_{\lfloor nt \rfloor}} \frac{ (\gamma_i - \xi_i) \cdot \ell_{D_k}}{|K_{\lfloor nt \rfloor}|} \ge 2t^{\frac{1}{2}}(1 - (1 - \delta')^{\frac{1}{2}})\mu_{\gamma}\right] \to 1 \,,
%\end{split}   
%\end{equation}
%as $n$ goes to infinity.

Since $\{\Hat{B}_{\cdot}^n\}_{n\geq 1}$ is  tight in $C_{\Rs^d}[0, \infty)$, thus relatively compact by Prohorov's Theorem (see, e.g., ~\cite[Theorem 5.1]{billingsley1999probability}) and consequently every subsequence has a limit point.
%
By~\eqref{xi_i->W2_d=>4} and~\eqref{eq: c1c2} for any of those limit points $\{Y_{t}\}_{t\ge 0}$ we have that for all $t \in [0, \infty)$
\begin{equation*}
\{W_t \cdot \ell_{D_k} + 2 c_1 \sqrt{t}\}_{t\ge 0} \preceq \{Y_t \cdot \ell_{D_k}\}_{t\ge 0} \preceq \{W_t \cdot \ell_{D_k} + 2 c_2 \sqrt{t} \}_{t\ge 0} \,,
\end{equation*}
where $c_1 = (1 - \sqrt{1 - \pi_{d-k}})\mu_{\gamma}$ and $c_2 = \sqrt{\pi_d} \, \mu_{\gamma}$ (and $0 < c_1 \le c_2$). 
%\com{here $\delta'' --> \delta $ and $\hat \delta --> \delta $?}
%
\end{proof}



\begin{remark}\label{rem:restriction}
Theorem~\ref{pn-ERW-d=>4} has some restrictions on the dimension and in the drift direction.  Those restrictions are due to the technique we used to prove Proposition~\ref{prop: bound_Kn}, which essentially consists in lower bounding the range of the $p_n$-\Nametwo{} with the range of a lazy random walk and then using Theorem~\ref{teo: RnZ>} (LLN). For this to work properly the lazy random walk should be at least $3$ dimensional, thus the dimensional restrictions. 
% ( The coupling of the $p_n$-\Nametwo{} and a lazy random walk had a fundamental part in the proof of Theorem~\ref{pn-ERW-d=>4}. Indeed, with this technique, we could find a lower bound for the range of the $p_n$-\Nametwo{} and describe the functions $c_1 t^{1/2}$ and $c_2t^{1/2}$. 
% However, this method breaks down when, for example, we have  a $p_n$-\Nametwo{} in direction $\ell \in \mathbb{S}^{d-1}$ on $\ZZ^d$, and $\ell$ has to be written with more than $d-3$ canonical directions of $\ZZ^d$. This is due to the fact that for the proof of Theorem~\ref{pn-ERW-d=>4} to work, the lazy random walk, which is used to lower bound  the range of the $p_n$-\Nametwo{}, should be at least  $3$ dimensional.  
% We will now propose a conjecture about the range of the $p_n$-\Nametwo{} in $\ZZ^d$, with $d \ge 3$ and in direction $\ell \in \mathbb{S}^{d-1}$, which would be  helpful to extend the Theorem~\ref{pn-ERW-d=>4}.

% \com{Rodrigo essa conjetura á para $\beta=1/2$ ne? O que conjeturamos em $d=2$ e $\beta=1/2$? Talvez seja mesmo melhor colocar essa conjetura no main result! e so explicar aqui como essa conjetura melhoraria o Teorema 1.6 }
% \begin{conjecture}\label{conj_range}
% Let $X$ be a $p_n$-\Nametwo{} in $\ZZ^d$, with $d \ge 3$ and in direction $\ell \in \mathbb{S}^{d-1}$ which can be written as in~\eqref{xn-incremnto1}. Then we have
% \begin{equation*}
%     \frac{|\Rr_n^X|}{n} \to \pi_d \quad \text{as } n \to \infty \text{ a.s.,}
% \end{equation*}
% where $\pi_d$ is the probability of $\{\xi_i\}_{i \ge 0}$ never returning to the origin. 
% \end{conjecture}
%
%An idea to prove the Conjecture~\ref{conj_range} is to find the same upper and lower bound. For the upper bound we can use the same arguments that we use in the proof of Proposition~\ref{prop: RnX<d3}. \cm{Now the lower bound is the actual problem. Seems to us that it could be found using similar technique from Spitzer for range of a random walk with i.i.d increments.}    
%
% \com{I would remove the following paragraph....}
% If Conjecture~\ref{conj_range} (about the range of the $p_n$-\Nametwo{}) holds true (see, Section~\ref{main_pn}),  it would be possible to prove Theorem~\ref{pn-ERW-d=>4} for a $p_n$-\Nametwo{} on $\ZZ^d$, with $d \ge 3$ and any direction $\ell \in \mathbb{S}^{d-1}$,   avoiding the use of the coupling that we described at the beginning of this section. 
% %
% Specifically,  it would be possible to prove a stronger version of Proposition~\ref{prop: bound_Kn} (see, Remark~\ref{rem:conjecture}) and consequently extend  Theorem~\ref{pn-ERW-d=>4} to  dimension $3$.  
\end{remark}


\begin{proof}[Proof of Lemma~\ref{Jntight}] 
Item $ii)$: 
 By   Remark~\ref{rem:conver}-$a)$,  it is enough to prove that the sequence of processes $\{\Hat{K}^n_{\cdot}/n^{1/2}\}_{n\geq 1}$ converges in probability as random elements of $C_{\Rs}[0, T)$ to the  zero function, for all $T > 0$. For the latter it suffices to prove  
 convergence in probability of $\sup_{0 \le t \le T} |K_{\lfloor nt \rfloor}|/n^{1/2}$ to zero for all $T > 0$.
% , since convergence in $C_{\Rs^d}[0, T]$ for all $T >0$ implies convergence in $C_{\Rs^d}[0, \infty)$ under the metric $\rho$ 
% \com{...HERE!!}
% . 
To this aim, let us define  
$$G_n:=\Big\{\sup_{0 \le t \le T} |K_{\lfloor nt \rfloor}| > \varepsilon \sqrt{n}\Big\} = \left\{ |K_{\lfloor nT \rfloor}| > \varepsilon \sqrt{n} \right\}\, ,
$$
For every $\varepsilon > 0$ and $\delta>0$, %consider the following event $G = \{|K_{\lfloor nT \rfloor}| > \varepsilon \sqrt{n}\}$. 
by Markov's inequality, we have that 
\begin{align*}\label{eq: Jnprob}
\begin{split}
& \PP[ G_n ] = \PP\big[
G_n \cap \{|\Rr_{\lfloor nT \rfloor} ^X| > \delta \lfloor nT \rfloor\}\big] +  \PP\big[ G_n \cap \{|\Rr_{\lfloor nT \rfloor}^X| \leq \delta \lfloor nT \rfloor\}\big]
\\
& \leq \PP\big[|\Rr_{\lfloor nT \rfloor} ^X| > \delta \lfloor nT \rfloor\big] + \PP\Big[ \sum_{i=1}^{\lceil \delta n T\rceil} \um_{\{U_i \leq i^{-1/2} \}} > \varepsilon\sqrt{n} \Big] 
\\
& \leq \PP\big[|\Rr_{\lfloor nT \rfloor} ^X| > \delta \lfloor nT \rfloor\big] + \frac{1}{\varepsilon \sqrt{n}} \sum_{i=1}^{\lceil \delta nT\rceil} \frac{1}{i^{1/2}}\,. \end{split}
\end{align*}
Using  Proposition~\ref{prop:RangeERW}, we have that, for all sufficiently large $n$,  $
    \PP[ |\Rr_n ^X| \leq \delta n ] = 1$, for every  $\delta > \pi_d$. Since for $d=2$, $\pi_d =0$,  we obtain that   $\lim_{n \to \infty}\PP\big[|\Rr_{\lfloor nT \rfloor} ^X| > \delta \lfloor nT \rfloor\big]=0$, for every $\delta>0$. Moreover, noticing that   $\sum_{i=1}^{\lceil \delta n\rceil} \frac{1}{i^{1/2}} = \Theta(\lceil \delta n\rceil^{1/2})$, we conclude that
\begin{equation}\label{eq: Jnprob2}
\begin{split}
& \limsup_{n \to \infty}  \PP[ G_n ]  
%\leq \limsup_{n \to \infty} \Big( \PP\big[|\Rr_{\lfloor nT \rfloor}^X| > \delta \lfloor nT \rfloor\big] + \frac{1}{\varepsilon \sqrt{n}} \sum_{i=1}^{\lceil \delta nT \rceil} \frac{1}{i^{1/2}} \Big) \\ & \leq \limsup_{n \to \infty} \PP\big[|\Rr_{\lfloor nT \rfloor}^X| > \delta \lfloor nT \rfloor\big] + \limsup_{n \to \infty} \Big( \frac{1}{\varepsilon \sqrt{n}} \sum_{i=1}^{\lceil \delta nT \rceil} \frac{1}{i^{1/2}} \Big) 
\leq \frac{c' (\delta T)^{1/2}}{\varepsilon}\,.
\end{split}
\end{equation}
Since $\delta>0$ is arbitrary,  $\limsup_{n \to \infty} \PP[ G_n ] = 0$ for every $\varepsilon$ fixed. Therefore, for all $T > 0$ the sequence of processes $\left\{\Hat{K}^n_{\cdot}/n^{1/2}\right\}_{n\geq 1}$ converges in probability, as random elements of $C_{\Rs}[0,T]$, to the  zero function.  %(see Lemma~\ref{lem: convprob}) \cm{pensar em uma referencia ou não ha necessidade?}. 

\medskip 
Item $i)$: 
% The proof follows the very same lines of that of Theorem~~\ref{pn-ERW-d=>4}, and we omit it. 
%
By  Remark~\ref{rem:conver}-$b)$  it suffices to show tightness in $C_{\Rs}[0, T]$ for all $T > 0$ and this is equivalent to show the sequence of processes $\left\{\Hat{K}^n_{\cdot}/n^{1/2}\right\}_{n\geq 1}$  satisfies  the two conditions in  Remark~\ref{rem:conver}-$c)$. 
Note that $\Hat{K}^n_{0}/n^{1/2}\equiv 0$ for all $n \ge 1$ and therefore the first condition in Remark~\ref{rem:conver}-$c)$ is satisfied.
%
To prove that $\{\Hat{K}^n_\cdot /n^{1/2}\}_{n\ge 1}$ satisfies the second condition in Remark~\ref{rem:conver}-$c)$  we  use~\cite[Corollary on page 83]{billingsley1999probability} which states that the second condition of ~\cite[Theorem 7.3]{billingsley1999probability} holds if, for every positive $\varepsilon$ and $\eta$, there exists a $\phi \in (0,1)$, and an integer $n_0$ such that
\begin{equation*}
\frac{1}{\phi} \, \PP \Big[ \sup_{t \le s \le t + \phi} \big\|\Hat{K}^n_{s} - \Hat{K}^n_{t} \big\|  \ge \varepsilon n^{1/2} \Big]  \le \eta \quad \forall n \ge n_0\,.
\end{equation*}
From this point on, using that 
\[
\PP \Big[ \sup_{t \le s \le t + \phi} \big\|\Hat{K}^n_{s} - \Hat{K}^n_{t} \big\|  \ge \varepsilon n^{1/2} \Big]  \le \PP \Big[ \sum_{i = \lfloor nt \rfloor }^{\lfloor n(t + \phi) \rfloor +1} \um_{\{U_i \le i^{-\frac{1}{2}}\}} \ge \varepsilon n^{\frac{1}{2}} \Big]\,,
\]
the computations are very similar to those used in the proof of Theorem~\ref{pn-ERW-d=>4}, and we omit it. 
\end{proof}





\begin{proof}[Proof of Lemma~\ref{B'_n}.] To avoid clutter in the notation, we write $J_n$ and $V_n$, thus omitting the dependence on $\delta$ and $\delta'$.  
As far as $J_n$ is concerned, we have 
\begin{equation*}
\frac{\EE[J_n]}{n^{1/2}} = \frac{1}{n^{1/2}}\EE\Big[\sum_{i=1}^{\delta n} \um_{\{U_i \leq i^{-1/2} \}}\Big] = \frac{1}{n^{1/2}}\sum_{i=1}^{\delta n} i^{-1/2} \xrightarrow[n \to \infty]{} 2\delta^{1/2}  \,. 
\end{equation*}
Also,  using Chebyshev's inequality and the independence of the random variables $\{U_i\}_{i \geq 1}$ we have that 
\begin{align*}
\PP \big[ |J_n - \EE[J_n]| > \varepsilon n^{1/2} \big] &\leq \frac{1}{\varepsilon^2 n} \text{Var}\Big[ \sum_{i=1}^{\delta n} \um_{\{U_i \leq i^{-1/2} \}} \Big]
\\
&=\frac{1}{\varepsilon^2 n} \sum_{i=1}^{\delta n} \frac{1}{i^{1/2}}\left(1-\frac{1}{i^{1/2}}\right) \xrightarrow[n \to \infty]{} 0  \,. 
\end{align*}

The proofs for $V_n$ are similar once we write
\[
V_n = \underbrace{\sum_{i=1}^n  \um_{\{U_i \leq i^{-1/2} \}}}_{=:I_n} - \underbrace{\sum_{i=1}^{n -\delta' n} \um_{\{U_i \leq i^{-1/2} \}}}_{=:F_n'}\,,
\]
and observe that  
\begin{align*}
&\frac{1}{n^{1/2}}\EE[I_n] = \frac{1}{n^{1/2}}\sum_{i=1}^{n} i^{-1/2} \xrightarrow[n \to \infty]{} 2\,,
\\
&\frac{1}{n^{1/2}}\EE[F'_n] = \frac{1}{n^{1/2}}\sum_{i=1}^{n - \delta' n} i^{-1/2} \xrightarrow[n \to \infty]{} 2(1-\delta')^{1/2} \,, 
\end{align*}
and that, by Chebyshev's inequality and the independence of the random variables $\{U_i\}_{i \geq 1}$, it holds that 
\begin{align*}
\PP & [|I_n - \EE[I_n]| > \varepsilon n^{1/2}] 
 \leq \frac{1}{\varepsilon^2 n} \text{Var}\Big[ \sum_{i=1}^{n} \um_{\{U_i \leq i^{-1/2} \}} \Big] 
% = 
% \frac{1}{\varepsilon^2 n} \sum_{i=1}^{n} i^{-1/2}(1-i^{-1/2})
\xrightarrow[n \to \infty]{} 0  
\,,
\\
\PP &\big[ |F'_n - \EE[F'_n]| > \varepsilon n^{1/2} \big] \leq \frac{1}{\varepsilon^2 n} \text{Var}\Big[ \sum_{i=1}^{n -\delta' n} \um_{\{U_i \leq i^{-1/2} \}} \Big]
% \\
% &
% =  \frac{1}{\varepsilon^2 n} \sum_{i=1}^{n - \delta' n} i^{-1/2}(1-i^{-1/2}) 
\xrightarrow[n \to \infty]{} 0 
\,.
\end{align*}
% The proof of point $v)$ is also straightforward: 
% \begin{equation*}
% \frac{\EE[\sum_{i=1}^n  \um_{\{U_i \leq i^{-1/2} \}}]}{n^{1/2}} = \frac{1}{n^{1/2}}\sum_{i=1}^{n} i^{-1/2} \xrightarrow[n \to \infty]{} 2 
% \,.  
% \end{equation*}
% For the proof of point $vi)$ we use Chebyshev's inequality and the independence of the random variables $\{U_i\}_{i \geq 1}$ and we obtain
% \begin{align*}
% % \label{eq: P}
% % \begin{split}
% \PP & \Big[\Big|\sum_{i=1}^n  \um_{\{U_i \leq i^{-1/2} \}} - \EE\Big[\sum_{i=1}^n  \um_{\{U_i \leq i^{-1/2} \}}\Big]\Big| > \varepsilon n^{1/2}\Big] 
%  \leq \frac{1}{\varepsilon^2 n} \text{Var}\Big[ \sum_{i=1}^{n} \um_{\{U_i \leq i^{-1/2} \}} \Big] 
%  \\
% &= \frac{1}{\varepsilon^2 n} \sum_{i=1}^{n} i^{-1/2}(1-i^{-1/2}) \to 0  \quad \text{as } n \to \infty 
% \,.  
% \end{align*}
\end{proof}

\begin{proof}[Proof of Lemma~\ref{lem: tightaux}.] To avoid clutter in the notation we omit the dependence on $\delta$ and $\delta'$. By Corollary~\ref{B'n->p}  we already have the convergence of the finite-dimensional distributions. Therefore,  by~\cite[Theorem 7.1]{billingsley1999probability},  to prove points $i)$ and $ii)$ it only remains to prove that both sequences of processes are tight in $C_{\Rs}[0, \infty)$. By Remark~\ref{rem:conver}-$b)$ we only need to prove tightness in $C_{\Rs}[0, T]$ for all $T>0$. The proof strategy is analogous to the one used in the proof of Theorem~\ref{pn-ERW-d=>4} which relies on Remark~\ref{rem:conver}-$c)$.  %
% when we show the second sum portion in~\eqref{p_n-ERW_incrementos_d=>4} is tight in $C_{\Rs}[0, T]$, for all $T>0$.

Item $i)$:  $\{\Hat{J}^n_{\cdot }/n^{1/2}\}_{n\geq 1}$ satisfies the first condition in Remark~\ref{rem:conver}-$c)$, since $\Hat{J}^n_0 = 0$ for all $n\geq 1$.  To prove the second condition in Remark~\ref{rem:conver}-$c)$, as in \eqref{eq: PnA} we need to show that for every positive $\varepsilon$ and $\eta$, there exists a $\phi \in (0,1)$, and an integer $n_0$ such that
\begin{equation}\label{eq: PnA2}
\frac{1}{\phi} \, \PP \Big[ \sup_{t \le s \le t + \phi} \big\|\Hat{J}^n_{s} - \Hat{J}^n_{t} \big\| \ge \varepsilon \Big]  \le \eta \quad \forall n \ge n_0\,.
\end{equation}
Following basically same steps as in the proof of Theorem~\ref{pn-ERW-d=>4}, we have that
\begin{equation*}
\begin{split}
&\frac{1}{\phi} \PP \Big[ \sup_{t \le s \le t + \phi} \big\|\Hat{J}^n_{s} - \Hat{J}^n_{t} \big\| \ge \varepsilon \Big] \le 
 \frac{1}{\phi} \PP \Big[ \sum_{i = \delta\lfloor nt \rfloor}^{\delta\lfloor n(t + \phi) \rfloor +1} \um_{ \{U_i \le i^{-\frac{1}{2}}\}} \ge \varepsilon n^{\frac{1}{2}} \Big] 
% \\
 %& \leq  \frac{1}{\phi} e^{-\varepsilon n^{\frac{1}{2}}}\EE\Big[\exp\Big( \sum_{i = \delta\lfloor nt \rfloor + 1}^{\delta\lfloor n(t + \phi)\rfloor} \um_{ \{U_i \le i^{-\frac{1}{2}}\}} \Big) \Big] 
% \\
% & = \frac{1}{\phi} e^{-\varepsilon n^{\frac{1}{2}}} \prod_{i = \delta\lfloor nt \rfloor + 1}^{\delta\lfloor n(t + \phi)\rfloor} \left(1+ \frac{e-1}{i^{\frac{1}{2}}} \right) \le \frac{1}{\phi} e^{-\varepsilon n^{\frac{1}{2}}} \prod_{i = \delta\lfloor nt \rfloor + 1}^{\delta\lfloor n(t + \phi)\rfloor} \exp\left(\frac{e-1}{i^{\frac{1}{2}}} \right)  
% \\
% & \le \frac{1}{\phi} e^{-c n^{\frac{1}{2}}}  \exp\left(\sum_{i = \lfloor nt \rfloor + 1}^{\lfloor n(t + \phi)\rfloor}\frac{e-1}{i^{\frac{1}{2}}} \right) 
\\
& \le \frac{1}{\phi} \exp(-\varepsilon n^{\frac{1}{2}})\exp\left(2(e-1)(\sqrt{\delta n(t + \phi)} - \sqrt{\delta nt} + 2) \right) \,.
\end{split}    
\end{equation*}
and we obtain \eqref{eq: PnA2} choosing $\phi \in (0,1)$ sufficiently small such that $\sqrt{t+\phi} - \sqrt{t}< \varepsilon/4\sqrt{\delta}(e-1)$ for all $t \in [0, T]$. 
% \begin{equation*}
% \frac{1}{\phi} \exp(-\varepsilon n^{\frac{1}{2}})\exp\bigg(2(e-1)\sqrt{n\delta}\left(\sqrt{t + \phi} - \sqrt{t}\right) \bigg) \le \eta \quad \forall n \ge n_0 \,.
% \end{equation*}
% The latter can be easily verified as it was done in the proof of Theorem~\ref{pn-ERW-d=>4}. 

% % From now on the proof follows exactly  as in Theorem~\ref{pn-ERW-d=>4}, when we prove the second sum portion in~\eqref{p_n-ERW_incrementos_d=>4} fulfills the second condition of~\cite[Theorem 7.3] {billingsley1999probability}. \comu{precisa ser mais específico aqui} Then we have that for each positive $\varepsilon$ and $\eta$, there exists a $\phi \in (0,1)$, and an integer $n_0$ such that
% % \begin{equation*}
% % \frac{1}{\phi} \PP [|J'_{\lfloor n \cdot \rfloor}|/n^{1/2} \in A_t(\varepsilon, \phi)] \le \eta \quad \forall n \ge n_0\,.
% % \end{equation*}
% % Ergo by~\cite[Theorem 7.3]{billingsley1999probability} we obtain that the sequence $|J_{\lfloor n \cdot \rfloor}|/n^{1/2}$ is a tight in $C_{\Rs}[0, T]$ for all $T > 0$ with the topology of uniform convergence in compacts and moreover by~\cite[Theorem 2.4.10]{karatzas2012brownian} \com{here we cite a different result than the one cited in the remark!} is a tight sequence of processes in $C_{\Rs}[0, \infty)$. 
The proof of item $ii)$ is similar with the only difference that we analyze separately $\sum_{i=1}^{\lfloor n \cdot \rfloor} \um_{ \{U_i \le i^{-\frac{1}{2}}\}}/n^{1/2}$ and    $\sum_{i=1}^{\lfloor (n -\delta' n) \cdot \rfloor} \um_{\{U_i \leq i^{-1/2} \}}/n^{1/2}$.
%
Using the very same computation of item $i)$, we conclude that the linearly interpolated version of $\left\{\sum_{i=1}^{\lfloor n \cdot \rfloor} \um_{ \{U_i \le i^{-\frac{1}{2}}\}}/n^{1/2}\right\}_{n\geq 1}$ and $\left\{ \sum_{i=1}^{\lfloor (n -\delta' n) \cdot \rfloor} \um_{\{U_i \leq i^{-1/2} \}}/n^{1/2}\right\}_{n\geq 1}$ 
% $\{\Hat{F}^n_{\cdot}/n^{1/2}\}_{n\geq 1}$ 
are tight sequences in $C_{\Rs}[0, T]$ for all $T>0$. Thus, the same holds for their difference.
%
\end{proof}


% \begin{proof}[Proof of Lemma~\ref{lem: Knt/n1/2tig}.] The proof follows the very same lines of that of Theorem~~\ref{pn-ERW-d=>4}, and we omit it. 
% {\color{red} 
% By  Remark~\ref{rem:conver}-$b)$  it suffices to show tightness in $C_{\Rs}[0, T]$ for all $T > 0$ and this is equivalent to show the the sequence of processes $\{|K_{\lfloor n \cdot \rfloor}|/n^{1/2}\}_{n\geq 1}$  satisfies  the two conditions in  Remark~\ref{rem:conver}-$c)$. 
% Note that $|K_{\lfloor n \cdot 0 \rfloor}|/n^{1/2}\equiv 0$ for all $n \ge 1$ and therefore the first condition in Remark~\ref{rem:conver}-$c)$ is satisfied.
% %
% To prove the second condition in Remark~\ref{rem:conver}-$c)$, 
% set 
% $
% A_t(\varepsilon, \phi):= \{f \in C_{\Rs}[0,T] : \sup_{t \le s \le t+\phi} |f(s) - f(t)| \ge \varepsilon \}$. 
% Then, following the same steps as in the proof of Theorem~\ref{pn-ERW-d=>4}, we have that  
% \begin{equation*}
% \begin{split}
% &\frac{1}{\phi} \PP [ |K_{\lfloor n \cdot \rfloor}|/n^{1/2} \in A_t(\varepsilon, \phi)]  = \frac{1}{\phi} \PP\Big[\sup_{t \le s \le t+\phi } | |K_{\lfloor ns \rfloor}| - |K_{\lfloor n t \rfloor}|| \ge \varepsilon n^{\frac{1}{2}} \Big]
% \\
% & = \frac{1}{\phi} \PP \Big[ \sum_{i = \lfloor nt \rfloor + 1}^{\lfloor n(t + \phi) \rfloor} \um_{E_i^c \cap \{U_i \le i^{-\frac{1}{2}}\}} \ge \varepsilon n^{\frac{1}{2}} \Big]
%  \le \frac{1}{\phi} \PP \Big[ \sum_{i = \lfloor nt \rfloor + 1}^{\lfloor n(t + \phi) \rfloor} \um_{\{U_i \le i^{-\frac{1}{2}}\}} \ge \varepsilon n^{\frac{1}{2}} \Big]\,.
% \end{split}    
% \end{equation*}
% From this point on, the computations are exactly the same used in the proof of Theorem~\ref{pn-ERW-d=>4}, and we omit it.

% when we show the second sum portion in~\eqref{p_n-ERW_incrementos_d=>4} fulfills the second condition in~\cite[Theorem 7.3]{billingsley1999probability}.
% Then we have that for each positive $\varepsilon$ and $\eta$, there exists a $\phi \in (0,1)$, and an integer $n_0$ such that
% \begin{equation*}
% \frac{1}{\phi} P_n \Big[f \in C_{\Rs}[0,T] : \sup_{t \le s \le t+\phi} |f(s) - f(t)| \ge \varepsilon \Big] \le \eta \, , \quad \forall n \ge n_0\,.
% \end{equation*}
% Ergo by~\cite[Theorem 7.3]{billingsley1999probability} we obtain that $|K_{\lfloor n \cdot \rfloor}|/n^{1/2}$ is tight in $C_{\Rs}[0, T]$ for all $T > 0$. Thus by~\cite[Theorem 2.4.10]{karatzas2012brownian} it is also a tight sequence of processes in $(C_{\Rs}[0, \infty),\rho)$
% } 
% \end{proof}
\end{comment}


