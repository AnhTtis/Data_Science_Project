\documentclass[reqno, 11pt]{amsart} 
\usepackage[utf8]{inputenc}
\usepackage{amssymb}
\usepackage{amsthm}
\usepackage{amsfonts}
\usepackage{amsmath}
\usepackage{dsfont}
\usepackage{hyperref}
\hypersetup{
   colorlinks=true,
   linkcolor=blue,
}
\usepackage{IEEEtrantools}
\usepackage{enumerate}
\usepackage{mathtools}
%%%%%%para a figura%%%%%
\usepackage{physics}
\usepackage{tikz}
\usepackage{mathdots}
\usepackage{yhmath}
\usepackage{cancel}
\usepackage{color}
%\usepackage{natbib} % citations
\usepackage{siunitx}
\usepackage{array}
\usepackage{multirow}
\usepackage{amssymb}
\usepackage{tabularx}
\usepackage{booktabs}
\usepackage{graphicx}
\graphicspath{ {./simulacoes/} }
\usepackage[normalem]{ulem} 
\usepackage{soul}

%\usetikzlibrary{fadings}
\usetikzlibrary{patterns}
%\usetikzlibrary{shadows.blur}
\usetikzlibrary{shapes}
%%%%%%%%%%%%%%%%%%%%%%%%%%

\usepackage{enumerate}
\usepackage{pdfpages}
\usepackage{refcount}
\usepackage{comment}


\usetikzlibrary{automata}
\usetikzlibrary{shapes,calc}
\usetikzlibrary{arrows.meta,arrows}
\usetikzlibrary{positioning}
\usepackage{caption}
\usepackage{subcaption}


%%%%%%%%%%%%%%%%%%%

\setcounter{tocdepth}{2}
\makeatletter
\def\l@subsection{\@tocline{2}{0pt}{2.5pc}{5pc}{}}
%Make Chapter disapear in ToC
\renewcommand\tocchapter[3]{%
  \indentlabel{\@ifnotempty{#2}{\ignorespaces#2.\quad}}#3%
}
\newcommand\@dotsep{4.5}
\def\@tocline#1#2#3#4#5#6#7{\relax
  \ifnum #1>\c@tocdepth % then omit
  \else
    \par \addpenalty\@secpenalty\addvspace{#2}%
    \begingroup \hyphenpenalty\@M
    \@ifempty{#4}{%
      \@tempdima\csname r@tocindent\number#1\endcsname\relax
    }{%
      \@tempdima#4\relax
    }%
    \parindent\z@ \leftskip#3\relax \advance\leftskip\@tempdima\relax
    \rightskip\@pnumwidth plus1em \parfillskip-\@pnumwidth
    #5\leavevmode\hskip-\@tempdima{#6}\nobreak
    \leaders\hbox{$\m@th\mkern \@dotsep mu\hbox{.}\mkern \@dotsep mu$}\hfill
    \nobreak
    \hbox to\@pnumwidth{\@tocpagenum{#7}}\par
    \nobreak
    \endgroup
  \fi}
\makeatother
\AtBeginDocument{%
\makeatletter
\expandafter\renewcommand\csname r@tocindent0\endcsname{0pt}
\makeatother
}
\def\l@subsection{\@tocline{2}{0pt}{2.5pc}{5pc}{}}
\newcommand\atotoc[1]{\addtocontents{toc}{#1\par}}



%%%%%%%%%%%%%%%%%%%%%%%%%
\newtheorem{lemma}{Lemma}[section]
\newtheorem{proposition}{Proposition}[section]
\newtheorem{claim}{Claim}[section]
\newtheorem{condition}{Condition}[section]
\newtheorem{conditionbis}{Condition}[section]
\newtheorem{corollary}{Corollary}[section]
\renewcommand\thecondition{\Roman{condition}}
\renewcommand\theconditionbis{\Roman{conditionbis}}

\newtheorem{theorem}{Theorem}[section]

\newtheorem{remark}{Remark}[section]

\newtheorem{conjecture}{Conjecture}[section]
%%%%%%%%%%%%%%%%%%%%%%%%%
%%%%%%% NEW COMMAND %%%%%
%%%%%%%%%%%%%%%%%%%%%%%%%

\newcommand{\PP}{\mathbb{P}}
\newcommand{\EE}{\mathbb{E}}
\newcommand{\ZZ}{\mathbb{Z}}
\newcommand{\FF}{\mathcal{F}}
\newcommand{\Rr}{\mathcal{R}}
\newcommand{\GG}{\mathcal{G}}
\newcommand{\Norm}[1]{{\Vert #1 \Vert}} % Norm
\newcommand{\modu}[1]{{\left| #1 \right|}} %modulo
\newcommand{\CC}{\mathcal{C}}
\newcommand{\Rs}{\mathbb{R}}
\newcommand{\um}{\mathds{1}}
\newcommand{\bD}{\mathbb{D}}

%%%% Process name 
\newcommand{\Name}{{\rm GERW}}
\newcommand{\Nameone}{{\rm ESRW}}
\newcommand{\Nametwo}{{\rm ERW}}
\newcommand{\Names}{{\rm  SGERW}}

% comments 
\newcommand{\com}[1]{\textcolor{red}{\texttt{giulio:} #1}}
\newcommand{\comu}[1]{\textcolor{cyan}{\texttt{Glauco:} #1}}
\newcommand{\cm}[1]{\textcolor{purple}{\texttt{Rodrigo:} #1}}
\newcommand{\br}[1]{\textcolor{brown}{#1}}

\usepackage{chngcntr}
\counterwithin{figure}{section}

\makeatletter
\@addtoreset{equation}{section}
\makeatother

\renewcommand\theequation{\thesection.\arabic{equation}}
%\renewcommand\thefigure{\thesection.\@arabic\c@figure}
\renewcommand\thetable{\thesection.\@arabic\c@table}

%%%%%%%%%%%%%%%%%%%%%%%%%%%%%%%%%%%%%%%PAGINA%%%%%%%%%%%%%%%%%%%%%


%%%%%%%%%%%%%%%%%%%%%%%%%%%%%%%%%%%%%%%%%%%%%%%%%%%%%%%%%%%%%
%%%%%%%%%%%%%%%%%%%%%%%%%%%%%%%%%%%%%%%%%%%%%%%%%%%%%%%%%%%%%


%\begin{center}
 %Projeto do segundo exame de Qualificação para o Doutorado em Estatística no PPGE-UFRJ   
%\end{center}

\title[excited random walk in Bernoulli environment]{
% Generalized excited random walk in nonhomogeneous Bernoulli environment
% \\
Limit Theorems for Generalized Excited Random Walks in time-inhomogeneous Bernoulli environment}

\author[R. Alves, G. Iacobelli, G. Valle, L. Zuaznabár]{Rodrigo B. Alves$^1$, Giulio Iacobelli$^2$,\\ Glauco Valle$^3$, Leonel Zuazn\'abar$^4$ }
\thanks{1. Supported by CAPES}
\thanks{2. Supported by }
\thanks{3. Supported by CNPq grant 307938/2022-0 and FAPERJ grant E-26/202.636/2019.}

%\date{\today}

\address{
\newline
\newline
Escola de matem\'atica aplicada, Fundação Get\'ulio Vargas.
\newline Caixa postal 22250-900, Rio de Janeiro, Brasil
\newline
$^1$ e-mail: {\rm \texttt{rodrigo.alves@fgv.br}}
}


\address{
\newline
\newline
Instituto de Matem\'atica, Universidade Federal do Rio de Janeiro.
\newline  Caixa Postal 68530, Rio de Janeiro, RJ, Brasil
\newline
$^2$ e-mail: {\rm \texttt{giulio@im.ufrj.br}}
\newline
$^3$ e-mail: {\rm \texttt{glauco.valle@im.ufrj.br}}
}


\address{
\newline
\newline
Centro de Matem\'atica, Computa\c{c}\~ao e Cogni\c{c}\~ao, Universidade Federal do ABC.
\newline Santo André, SP, Brasil.
\newline
$^4$ e-mail: {\rm \texttt{l.zuaznabar@ufabc.edu.br}}
}


\subjclass[2020]{60K37}
\keywords{excited random walks, non-Markovian processes, random environment, ballisticity, law of large numbers,  central limit theorem}


\begin{document}

\maketitle

\begin{abstract}

We study a variant of the Generalized Excited Random Walk (\Name) on $\mathbb{Z}^d$ introduced by Menshikov et al. in \cite{menshikov2012general}.  It consists in a particular version of the model studied in \cite{alves2022note} where excitation may or may not occur according to a time-dependent probability.  Specifically, given a sequence $\{p_n\}_{n \ge 1}$, $p_n \in (0, 1]$ for all $n \ge 1$, if at time $n$ the process visits a site for the first time, then with probability $p_n$ it gains a drift in a determined direction. Otherwise, it behaves as a $d$-dimensional martingale with zero-mean vector. We refer to the model as a \Name{} in time-inhomogeneous Bernoulli environment, in short, $p_n$-\Name{}. 

%Otherwise, with probability $1-p_n$, it  behaves as a $d$-martingale with zero-mean vector.  Whenever the process visits an already-visited site, the process acts again as a $d$-martingale with zero-mean vector. We refer to the model as a \Name{} in Bernoulli environment, in short $p_n$-\Name{}.

Under bounded jumps hypothesis and with %a sequence $\{p_n\}_{n \ge 1}$ which decays polynomially, namelly $p_n = \mathcal{C}n^{-\beta} \wedge 1$ with $\beta > 0$ and $\mathcal{C}$ is a positive constant, 
$p_n \approx n^{-\beta}$,  
we show a series of results for the $p_n$-\Name{} under diffusive scaling depending on the value of $\beta$ and on the dimension. Specifically, for $\beta > 1/2$ and $d\geq 2$, or  $\beta=1/2$ and $d=2$, we obtain a Functional Central Limit Theorem. For $\beta=1/2$ and $d \ge 22$ we obtain that the diffusively rescaled $p_n$-\Name{} is tight, and  every limit point $\mathcal{Y}$ satisfies $W_t \cdot \ell + c_1 \sqrt{t} \preceq \mathcal{Y}_t \cdot \ell \preceq W_t \cdot \ell + c_2 \sqrt{t}$,  where $c_1$ and $c_2$ are known positive constants, $W_{\cdot}$ is a Brownian motion and $\ell$ is the direction of the drift. The upper bound also holds for $d\ge 3$. The proof of these results relies on a Law of Large Numbers for the range of the $p_n$-\Name{}, which we believe to be of independent interest.

%Among these results, which depend on the dimension and how fast the sequence $\{p_n\}_{n \ge 1}$ decays, we obtain a lower bound to the probability of the $p_n$-\Name{} never return to the origin in the drift direction and with a suitably rescaled a Functional Central Limit Theorem.     

\end{abstract}



% If you want the subsection to appear in the table of content use 
%\setcounter{tocdepth}{3}

\tableofcontents

%\pagenumbering{arabic}

\setcounter{tocdepth}{2}
\section{Introduction}

\section{Introduction} \label{sec.intro}

We consider the problem of minimizing a function over a network. In this setting, each node of the network has a portion of the global objective function and the edges represent neighbor nodes that can exchange information, i.e., communicate. The goal is to collectively minimize a finite sum of functions where each component is only known to one of the $n$ nodes (or agents) of the network. Such problems arise in many application areas such as machine learning \cite{forero2010consensus,tsianos2012consensus}, sensor networks \cite{baingana2014proximal, predd2007distributed}, multi-agent coordination \cite{cao2012overview, zhou2011multirobot} and signal processing \cite{combettes2011proximal}. The problem, known as a \emph{decentralized optimization} problem, can be represented as follows:
% We consider the problem of minimizing a function over a network. Each node of the network has a portion of the global objective function and the edges represent \asb{neighbor} nodes that can \asb{exchange information, i.e., communicate}. Such problems arise in many application areas such as machine learning \cite{forero2010consensus,tsianos2012consensus}, sensor networks \cite{baingana2014proximal, predd2007distributed}, multi-agent coordination \cite{cao2012overview, zhou2011multirobot} and signal processing \cite{combettes2011proximal}. The goal is to collectively minimize a finite sum of functions where each component is only known to one of the $n$ nodes (or agents) of the network. This problem, known as a \emph{decentralized optimization} problem, can be represented as follows:
\begin{align}		\label{eq:prob}
	\min_{x\in \mathbb{R}^d}\quad f(x) = \frac{1}{n} \sum_{i=1}^n f_i(x),
\end{align}
where $f: \mathbb{R}^d \rightarrow \mathbb{R}$ is the global objective function, $f_i: \mathbb{R}^d \rightarrow \mathbb{R}$ for each $i\in \{1,2,...,n \}$ is the local objective function known only to node $i$ and $x\in \mathbb{R}^d$ is the decision variable.

To decouple the computation across different nodes, \eqref{eq:prob} is often reformulated as (see e.g., \cite{bertsekas2015parallel,nedic2009distributed})
\begin{equation}\label{eq:cons_prob}
\begin{aligned}	
	\min_{x_i \in \mathbb{R}^d}&\quad \frac{1}{n} \sum_{i=1}^n f_i(x_i)\\
	 \text{s.t.} &\quad  x_i = x_j, \quad \forall \,\, (i, j) \in \mathcal{E},
\end{aligned}
\end{equation}
where $x_i \in \mathbb{R}^d$ for each node $i\in \{1,2,...,n \}$ is a local copy of the decision variable, and  $\mathcal{E}$ denotes the set of edges of the network. If the underlying network is connected, the \emph{consensus} constraint ensures that all local copies are equal, and, thus, problems \eqref{eq:prob} and \eqref{eq:cons_prob} are equivalent. For compactness, we express problem \eqref{eq:cons_prob} as
\begin{equation}\label{eq:cons_prob1}
\begin{aligned}		
	\min_{x_i \in \mathbb{R}^d}&\quad \textbf{f} (\textbf{x}) = \frac{1}{n} \sum_{i=1}^n f_i(x_i)\\
	\text{s.t.} & \quad (\textbf{W}\otimes I_d)\textbf{x} = \textbf{x}, 
\end{aligned}
\end{equation}
where $\textbf{x} \in \mathbb{R}^{nd}$ is a concatenation of local copies $x_i$, $\textbf{W} \in \mathbb{R}^{n \times n}$ is a matrix that captures the connectivity of the underlying network, $I_d \in \mathbb{R}^{d \times d}$ is the identity matrix of dimension $d$, and the operator $\otimes$ denotes the Kronecker product,  $\textbf{W}\otimes I_d \in \mathbb{R}^{nd \times nd}$. The matrix $\textbf{W}$, known as the \emph{mixing} matrix, is a symmetric, doubly-stochastic matrix with $w_{ii}>0$ and $w_{ij}>0$ ($i\neq j$) if and only if $(i, j) \in \mathcal{E}$ in the underlying network. This matrix ensures that $(\textbf{W}\otimes I_d) \textbf{x}=\textbf{x}$ if and only if $x_i=x_j \,\, \forall \,\, (i, j) \in \mathcal{E}$ in the connected network, thus, % making 
\eqref{eq:cons_prob} and \eqref{eq:cons_prob1} are equivalent.% problems.

In this paper, we focus on gradient tracking methods. These first-order methods update and communicate the local decision variables, and also maintain, update and communicate an additional auxiliary variable that estimates (tracks) the gradient of the global objective function.
%\rb{These first-order methods update and communicate the local decision variables, and an additional auxiliary variable that estimates (tracks) the gradient of the global objective function.  }
%average gradient across all the nodes.} 
%These methods maintain an auxiliary variable ($y_i$) that estimates the average gradient across the network in addition to the decision variable ($x_i$) at each agent (node) $i$. 
We refer to the information shared by the methods as the communication strategy. When applied to the same decentralized setting, the theoretical convergence guarantees and practical implementations of gradient tracking methods with different communication strategies can vary significantly. %When applied to the same decentralized setting, the theoretical convergence guarantees and practical implementations of the methods vary with respect to the communication strategy. 
We propose an algorithmic framework that unifies communication strategies in gradient tracking methods and that allows for a direct theoretical and empirical comparison. The framework recovers popular gradient tracking methods as special cases.

The update form of gradient tracking methods can be generalized and decomposed as: $(1)$ one \emph{computation step} of calculating the local gradients, and $(2)$ one \emph{communication step} of sharing information based on the communication strategy. In practice, the complexity of these two steps can vary significantly across applications. For example, a large-scale machine learning problem solved on a cluster of computers with shared memory access has a higher cost of computation than communication \cite{tsianos2012consensus}. On the other hand, optimally allocating channels over a wireless sensor network requires economic usage of communications due to limited battery power \cite{magnusson2017bandwidth}.
% The same being solved for data stored over distant data servers would have a higher cost of communication than computation. 
The subject of developing algorithms (and convergence guarantees) that balance these costs has received significant attention in recent years; see e.g.,~\cite{chen2012fast,berahas2018balancing,9479747,berahas2019nested,sayed2014diffusion,zhang2018communication} and the references therein. In this paper, we follow the approach used in~\cite{berahas2018balancing} and %. That is, we 
explicitly decompose the two steps. %, and so o
As a result, our algorithms are endowed with flexibility in terms of the number of communication and computation steps performed at each iteration. We show the benefits of this flexibility theoretically and empirically.

\subsection{Literature Review} \label{sec.lit}

Decentralized Gradient Descent (DGD) \cite{bertsekas2015parallel, nedic2009distributed}, a primal first-order method, is considered the prototypical method for solving~\eqref{eq:prob}. %The DGD method 
At each iteration nodes perform local computations and communicate local decision variable to neighbors. 
%that employs a CTA \sg{(Combine-then-Adapt \cite{sayed2014diffusion})} communication strategy. 
% Under reasonable assumptions, with a constant step size, it converges to a neighbourhood of the solution \cite{yuan2016convergence}. 
Gradient tracking methods, e.g., EXTRA \cite{shi2015extra}, SONATA \cite{sun2022distributed}, NEXT \cite{di2016next}, DIGing \cite{nedic2017achieving}, Aug-DGM \cite{xu2015augmented}, have emerged as popular alternatives due to their superior theoretical guarantees and empirical performance. %as they converge to the solution with a constant step size.}
% have emerged as popular alternatives as they avoid the aforementioned shortcomings. 
They maintain, update and communicate an additional auxiliary variable that tracks the average gradient (additional communication cost compared to DGD). These methods are usually applied to smooth convex functions over undirected networks; however, they are also applicable to various other settings such as time varying networks \cite{nedic2017achieving}, uncoordinated stepsizes \cite{nedic2017geometrically, xu2015augmented}, directed networks \cite{nedic2017achieving, pu2020push}, nonconvex functions \cite{di2016next, sun2022distributed} %, sun2016distributed} 
and stochastic gradients \cite{pu2021distributed}. Our algorithmic framework generalizes and extends current gradient tracking methodologies, allowing for a unified analysis and direct comparison of popular methods. In \cite{sundararajan2017robust,zhang2019computational}, semi-definite programming is used to unify communication strategies in gradient tracking methods. 
% old version
% the authors presented a method to unify CTA and ATC \sg{(Adapt-then-Combine \cite{sayed2014diffusion})} communication strategies in gradient tracking \sg{methods} using semi-definite programming. 
Our framework is simpler and allows for more general communication strategies than those in \cite{sundararajan2017robust,zhang2019computational}.

% also encompassesheterogeneous communication strategies \rb{which allow for asymmetric sharing of information across different subsets of the edges in the network.}

Another class of popular methods is %that \sg{can converge to the solution with a constant stepsize are}
% mitigate the shortcomings of DGD are 
primal-dual methods \cite{arjevani2020ideal, jakovetic2014linear, ling2015dlm, shi2014linear, wei20131, mansoori2021flexpd, mancino2021decentralized}. Of these methods, Flex-PD \cite{mansoori2021flexpd} and ADAPD \cite{mancino2021decentralized} allow for flexibility with respect to the number of communication and computation steps. That said, Flex-PD \cite{mansoori2021flexpd} does not show improved performance with the employment of the flexibility and ADAPD \cite{mancino2021decentralized} does not allow for a balance between communication and computation. 
%, however, \rb{these methods either fail to show improved performance with the employment of the flexibility or do not provide the precision to specify the exact balance between communication and computation as does our framework.}
% these methods fail to simultaneously provide improvement with the employment of the flexibility and the exact specification of communication and computation budget balance in iterations like our algorithm. 
Finally, algorithms that consider the consensus constraint as a proximal operator have been proposed. These algorithms aim to reduce communication load on distributed systems via a randomization scheme but are primarily designed for fully connected networks (all pairs of nodes are connected). %networks where all pairs of nodes are connected (fully connected networks). 
Examples of such methods include, but are not limited to, Scaffnew \cite{mishchenko2022proxskip}, FedAvg \cite{li2019convergence}, Scaffold \cite{karimireddy2020scaffold}, Local-SGD \cite{gorbunov2021local} and FedLin \cite{mitra2021linear}.


% Old version
% Another class of methods that achieve linear convergence to the solution are \asb{primal-dual} methods that work with the Lagrangian function \cite{arjevani2020ideal, jakovetic2014linear, ling2015dlm, shi2014linear, wei20131, mansoori2021flexpd, mancino2021decentralized}. Flex-PD \cite{mansoori2021flexpd} and ADAPD \cite{mancino2021decentralized}  build upon the idea of augmented lagrangian and decentralised ADMM to provide the flexibility in number of communication and computation steps. These methods fail to simultaneously provide improvement with the employment of the flexibility and the exact specification of communication and computation budget balance in iterations like our algorithm. 

% Another approach to this problem is looking at the consensus constraint as a proximal operator. These algorithms aim to reduce communication load on distributed systems but are designed for networks where all pairs of nodes are connected. Algorithms such as Scaffnew \cite{mishchenko2022proxskip}, FedAvg \cite{li2019convergence}, Scaffold \cite{karimireddy2020scaffold}, Local-SGD \cite{gorbunov2021local}, FedLin \cite{mitra2021linear} achieve this by performing communications less often decided via a randomized scheme while continuously performing local updates.

% This paper is part of the growing literature in decentralized optimization. One of the first algorithms to solve problem \eqref{eq:cons_prob1} is Distributed Gradient Descent (DGD,  \cite{nedic2009distributed, yuan2016convergence}), a first order primal iterative method with a CTA communication structure. It has been shown to converge to a neighbourhood of the solution under a constant stepsize when the problem is smooth and convex. In \cite{berahas2018balancing} and \cite{9479747}, the authors introduced variants of DGD with an ATC communication strategy. These variants show convergence to the solution every iteration under a constant step size when communications are increased every iteration, not possible with DGD. Algorithms with similar neighbourhood convergence results can be found in \cite{mokhtari2016network, sayed2014diffusion}.

% Gradient tracking methods such as EXTRA \cite{shi2015extra}, SONATA \cite{sun2022distributed}, NEXT \cite{di2016next}, DIGing \cite{nedic2017achieving}, Aug-DGM \cite{xu2015augmented} also collectively referred to as push-pull algorithms \cite{pu2020push} are first order primal methods. They use an auxiliary variable to track the average gradient across the network. These algorithms achieve a linear rate of convergence to the solution with a constant stepsize. Usually applied to smooth convex functions over undirected networks, they are also applicable to various other settings like time varying networks \cite{nedic2017achieving}, uncoordinated stepsizes \cite{nedic2017geometrically, pu2020push, xu2015augmented}, directed networks \cite{nedic2017achieving, pu2020push}, non convex functions \cite{di2016next, sun2022distributed, sun2016distributed} and stochastic gradients \cite{pu2021distributed} and randomized communication \cite{pu2021distributed}. In this paper we unify and further generalise the communication strategies of gradient tracking algorithms present in literature while restricting ourselves to smooth convex functions over undirected graphs with deterministic gradients. We also introduce flexibility in terms of number of computation and communication steps in iterations in the general framework. In \cite{sundararajan2017robust} and \cite{zhang2019computational}, the authors presented a method to unify all CTA and ATC diffusion strategies in gradient tracking algorithms using semi-definite programming. Our framework is simpler and also able to encompass heterogeneous communication strategies where all information is not communicated among the same neighbors.

% Another class of methods that achieve linear convergence to the solution are primal dual methods based on the idea of lagrangian dualitly and alternating method of multipliers (ADMM) \cite{arjevani2020ideal, jakovetic2014linear, ling2015dlm, shi2014linear, wei20131}. Algorithms such as Flex-PD \cite{mansoori2021flexpd} and ADAPD \cite{mancino2021decentralized}  have built upon the idea of augmented lagrangian and decentralised ADMM to provide the flexibility in number of communication and computation steps. 

% While Flex-PD works for smooth strongly convex functions, it does not provide the flexibility to manipulate both communications and computations at the same time while our framework does. Flex-PD shows an exponential decrease in stepsize requirement with increase in both communications and computation in an iteration. Our algorithm shows an improved stepsize and convergence rate with increase in communication. Our algorithm shows a polynomial decrease in the stepsize with increased computation. ADAPD is a framework for non convex problems. It shows improvement in performance with multiple communications. Multiple computations are performed within ADAPD in terms of solving a local problem to a certain accuracy. Our method allows to specify the exact number of computations being performed at each node. This allows greater control over the balance between computation and communication to the user based on the systems properties. ADAPD framework also does not empirically show any improvements in terms of iterations when multiple computations are performed while ours does.

% Another approach to this problem is looking at the consensus constraint as a proximal operator. These algorithms aim to reduce communication load on distributed systems but are designed for fully connected networks where all pair of nodes are connected. Algorithms such as Scaffnew \cite{mishchenko2022proxskip}, FedAvg \cite{li2019convergence}, Scaffold \cite{karimireddy2020scaffold}, Local-SGD \cite{gorbunov2021local}, FedLin \cite{mitra2021linear} achieve this by performing communications less often while continuously performing local updates. The decision to whether or not communicate in an iteration is determined using either a deterministic or randomized sequence. The iterate update is usually accompanied by a correction term to correct for the bias from local updates. In this paper, we propose a deterministic method to introduce flexibility in number of communication and computation steps for any connectivity of the network.

\subsection{Contributions} \label{sec.contri}
We summarize our main contributions as follows:
\begin{enumerate}
	\item We propose a gradient tracking algorithmic framework (\texttt{GTA}) that unifies communication strategies in gradient tracking methods and provides flexibility in the number of communication and computation steps performed at each iteration. The framework recovers as special cases popular gradient tracking methods, i.e., ~\texttt{GTA-1} \cite{shi2015extra, nedic2017achieving}, \texttt{GTA-2} \cite{di2016next, sun2022distributed} and \texttt{GTA-3} \cite{nedic2017geometrically, xu2015augmented}; see \cref{tab: Algorithm Def}.
 % The framework recovers as special cases popular methods, i.e., ~\texttt{GTA-1} \cite[EXTRA]{shi2015extra}, \cite[DIGing]{nedic2017achieving}, \texttt{GTA-2} \cite[NEXT]{di2016next}, \cite[SONATA]{sun2022distributed} and \texttt{GTA-3} \cite[ATC-Digging]{nedic2017geometrically}, \cite[Aug-DGM]{xu2015augmented}; see \cref{tab: Algorithm Def}.
    \item We establish the conditions required, on the communication strategy and the step size parameter, that %to
    guarantee a global linear rate of convergence for \texttt{GTA} with multiple communication and multiple computation steps. 
    %multiple communication and single computation steps (\cref{th. general g=1 step cond}), and multiple communication and multiple computation steps (\cref{th. alpha bound g > 1}). 
    We also compare the relative performance of the special case gradient tracking algorithms, and illustrate the theoretical advantages of \texttt{GTA-3} over \texttt{GTA-2} (and \texttt{GTA-2} over \texttt{GTA-1}), a direct comparison not established in prior literature. %show that the rate constant of \texttt{GTA-3} is better than \texttt{GTA-2} which is better than \texttt{GTA-1}, a comparison that has not been established in prior literature.
    \item We show that the rate of convergence improves %rate constants improve 
    with increasing the number of communication steps, and the extent of improvement depends on the communication strategy. % (\cref{th.incr rates g > 1}). 
    %We show that t
    The improvements are much more profound in \texttt{GTA-3} as compared to \texttt{GTA-2} and \texttt{GTA-1}. %(Corollary~\ref{col. g=1 rate bound}) in explicit form when a single computation step is performed in each iteration.
    \item We illustrate the empirical performance of the proposed \texttt{GTA} framework on quadratic and binary classification logistic regression problems. We show the effect and benefits of %performing 
    multiple communication and/or computation steps per iteration on the performance of the special case algorithms.
\end{enumerate}

\subsection{Notation} \label{sec.notation}
Our proposed algorithmic framework is iterative and works with inner and outer loops. The variables $x_{i, k, j} \in \mathbb{R}^d$ and $y_{i, k, j} \in \mathbb{R}^d$ denote the local copies %copy 
of the decision variable and the auxiliary variable, respectively, of node $i$, in outer iteration $k$ and inner iteration $j$. The average of all local decision variables and local auxiliary variables are denoted by $\bar{x}_{k, j} = \frac{1}{n} \sum_{i=1}^n x_{i, k, j}$ and $\bar{y}_{k, j} = \frac{1}{n} \sum_{i=1}^n y_{i, k, j}$, respectively. Boldface lowercase letters represent concatenated vectors of local copies
\begin{align*}
    \xmbf_{k, j} = 
    \begin{bmatrix}
        x_{1, k, j}\\
        x_{2, k, j}\\
        \vdots \\
        x_{n, k, j}
    \end{bmatrix} \in \mathbb{R}^{nd}\mbox{,} \quad
    \ymbf_{k, j} = 
    \begin{bmatrix}
        y_{1, k, j}\\
        y_{2, k, j}\\
        \vdots \\
        y_{n, k, j}
    \end{bmatrix} \in \mathbb{R}^{nd}
    \mbox{,} \quad
      \nabla \fmbf(\xmbf_{k, j}) = 
    \begin{bmatrix}
        \nabla f_1(x_{1, k, j})\\
        \nabla f_2(x_{2, k, j})\\
        \vdots \\
        \nabla f_n(x_{n, k, j})
    \end{bmatrix} \in \mathbb{R}^{nd}.%,
\end{align*}
%in outer iteration $k$ and inner iteration $j$. 
The concatenated vector of the average of decision variables ($\Bar{x}_{k, j}$) and auxiliary variables ($\Bar{y}_{k, j}$) repeated $n$ times is denoted by $\xbb_{k, j}$ and $\ybb_{k, j}$, respectively. 
% The concatenated vector of the average of decision variables and auxiliary variables in outer iteration $k$ and inner iteration $j$ repeated $n$ times is denoted by $\xbb_{k, j}$ and $\ybb_{k, j}$ respectively. 
The $n$ dimensional vector of all ones is denoted by $1_n$ and the identity matrix of dimension $n$ is denoted by $I_n$. The spectral radius of square matrix $A$ is $\rho(A)$. Matrix inequalities are defined component wise. %, i.e., $A \geq B$ implies every element in matrix A is greater than or equal to the corresponding element in matrix $B$. 
The Kronecker product of any two matrices $A \in \mathbb{R}^{n \times n}$ and $B \in \mathbb{R}^{d \times d}$ is represented using the operator $\otimes$ and denoted as $A \otimes B \in \mathbb{R}^{nd \times nd}$.

\subsection{Paper Organization} In \cref{sec.methods}, we describe our proposed gradient tracking algorithmic framework (\texttt{GTA}). In \cref{sec.theory}, we provide theoretical convergence guarantees for the proposed algorithmic framework for multiple communication steps and a single computation step at each iteration (\cref{sec.mult comms}) and multiple communication and computation steps at each iteration (\cref{sec.mult grads}). In \cref{sec.full graph res}, we consider the special case %(not captured in \cref{sec.mult comms,sec.mult grads}) 
of fully connected networks. Numerical experiments on quadratic and binary classification logistic regression problems %that illustrate the advantages of the flexibility 
are presented in \cref{sec.num_exp}. Finally, we provide concluding remarks in \cref{sec.conc}.

% Old version
%In \cref{sec.methods}, we describe our proposed algorithm to unify communication strategies in gradient tracking methods with the flexibility in number of communication and computation steps. In \cref{sec.theory}, we provide theoretical convergence guarantees for the proposed algorithm. We first analyse the algorithm with multiple communication steps and a single computation step being performed in each iteration in \cref{sec.mult comms}. We extend the analysis to multiple communication and computation steps in \cref{sec.mult grads}. In \cref{sec.full graph res}, we analyse the algorithm under a fully connected network as it is a special case not covered by \cref{sec.mult comms} and \cref{sec.mult grads}. In \cref{sec.num_exp}, we illustrate the performance of the algorithm over  quadratic functions and binary logistic regression. Finally, we provide concluding remarks and future directions in \cref{sec.conc}.



%We focus now on the case in which the random environment  $\omega(x)_{x \in \ZZ^d} \in [0, 1]^{\ZZ^d}$ is i.i.d. and given a fixed $p \in (0,1]$ we set $p_n=p$, for every $n$. 

%In this text we focus on the homogeneous case, i.e., for every $n$ we set $p_n=p$ \com{no need to say this if we already from the beginning focus on the constant case},  with $p \in (0,1]$ and the marginals of the random element $\pi$ are independent. 

% \subsection{Main results for the $p$-\Name}\label{LGNCLT}\label{sec: res_p-gerw}
% As mentioned in the Introduction the homogeneous case can be reduced to the \Name{}. Specifically, the $p$-\Name{} with a given $\lambda$ in Condition~\ref{condição2} reduces to a \Name{} with $p\lambda$ in Condition $C^+$ in \cite{menshikov2012general}. As a matter of fact, if we denote by $\widetilde{\FF}_n=\sigma(X_1, \ldots, X_n)$, by integrating 
% $\mathbb{E} [ X_{n+1} - X_n | \mathcal{F}_n] \cdot \ell$ with respect to $\pi(X_1), \ldots, \pi(X_n)$ we obtain $\mathbb{E} [ X_{n+1} - X_n | \widetilde{\FF}_n] \cdot \ell\geq p\lambda$, which is Condition $C^+$ in \cite{menshikov2012general} with a $\lambda'=p\lambda$. 
% %
% Despite the close connection with \Name{}, some of the techniques developed to prove results for the  $p$-\Name{} will be useful for the time dependent case. 


% Our first result  is that for every $p \in (0,1]$, the $p$-\Name{} is ballistic in $\ell$ direction. 

% \medskip

% \begin{theorem}[Ballisticity of $p$-\Name{}]\label{teo11}
% Let $X$ be a $p$-\Name{} in direction $\ell \in \mathbb{S}^{d-1}$. %(i.e., satisfies Conditions~\ref{condição1},\ref{condição2} and \ref{condição3}). 
% Then
% \begin{equation*}
% \liminf_{n \to \infty} \frac{X_n \cdot \ell}{n} > 0\;, \quad \text{a.s..}    
% \end{equation*}
% \end{theorem}


% \medskip

% % We could prove that the $p$-\Name{} is transient for any $p > 0$ and find a good lower bound for the probability of the process never return to the initial position. Besides, we are able to see how this bound behaves according to the choice of $p$. 
% %To obtain the ballisticity result we  first upper bound the probability that  the $p$-\Name{} visits less than $n^{1/2+\alpha}$ distinct sites until time $n$ for all $\alpha \in (0, 1/4)$ (see, Proposition~\ref{prop41}) and then, using the latter, we show that the $p$-\Name{} moves ballistically in direction $\ell$ (see, Proposition~\ref{prop42}). The proof is similar to the proof in~\cite{menshikov2012general}. The main difference is that  in~\cite{menshikov2012general}  the set of sites with cookies is deterministic, indeed every site has a cookie, whereas in  our model the set of sites with cookies is random and must be properly controlled. \comu{Acho que podemos remover este parágrafo e fazer a observação antes da prova da Proposição 3. Lembrando que \cite{menshikov2012general} já foi mencionado como a principal referência.}

% \medskip 

% The next two results are the Law of Large Numbers and the Central Limit Theorem which hold for a special case of  $p$-\Name{}. Specifically,   we  need to introduce a fourth condition (see, Condition~\ref{t_k+1 - t_k ind} in Section~\ref{renewalstruc}) which is related to the distribution of the increments of the regeneration times associated to  the $p$-\Name{}. A $p$-\Name{} satisfying this fourth condition will be called \textit{p-Strong General Excited Random Walk} ($p$-\Names).  
% %
% It can be shown (see, Corollary~\ref{pERWcond4})  that the $p$-\Nametwo{} introduced in the  Example in Section~\ref{sec:model} is an example of  $p$-\Names{}. 

% \medskip

% \begin{theorem}[Law of Large Numbers]\label{LLN}
% Assume the process $X$ is a $p$-\Names{} in direction $\ell$ (i.e., satisfies Conditions~\ref{condição1},\ref{condição2}, \ref{condição3} and \ref{t_k+1 - t_k ind}), then there exists  $v \in \mathbb{R}^d$ such that $v \cdot \ell > 0$ and
% \begin{equation}
% \label{Xn/n->v}
% \lim_{n \to \infty} \frac{X_n}{n} = v\;, \quad \text{a.s..}    
% \end{equation}
% \end{theorem}

% \medskip

% Let $X$ be a $p$-\Names{} in direction $\ell$ and $v \in \mathbb{R}^d$   from~\eqref{Xn/n->v}.
% Let us define the process
% \begin{equation}\label{B}
% B_t^n = \frac{X_{\lfloor nt \rfloor} - \lfloor nt \rfloor v}{n^{1/2}}\;,  \ t \ge 0\;.     
% \end{equation}
% %Then, regarding $B_{\cdot}^n$ the following theorem holds. 

% \smallskip

% \begin{theorem}[Central Limit Theorem]\label{CLT} The process $B_{\cdot}^n$ converges in distribution, as $n \to \infty$, to a $d$-dimensional Brownian Motion with a non-degenerate covariance matrix.
% \end{theorem}


\subsection{Main results}\label{main_pn}

Let us recall that we shall focus on the case in which the sequence $\{p_n\}_{n \ge 1}$ is of the form $p_n =\mathcal{C} n^{-\beta} \wedge 1$ for some $\mathcal{C} > 0$. Before we state the main results, let us introduce some notation. Define 
\begin{equation}\label{eq:B}
\Hat{B}_t^n := \frac{X_{\lfloor nt \rfloor}}{n^{1/2}} + (nt - \lfloor nt \rfloor)\frac{(X_{\lfloor nt \rfloor + 1} - X_{\lfloor nt \rfloor})}{n^{1/2}} \,, \ t\ge 0 \,,
\end{equation}
where the process $X$ will make reference to a specified process ($p_n$-\Name, $p_n$-\Name*{} or $p_n$-\Nametwo{}) in each result; we shall refer to  $\Hat{B}_t^n$ as the corresponding rescaled process.

Let $C_{\Rs^d}[0,T]$ be the space of continuous functions from $[0,T]$ to $\mathbb{R}^d$ for every $T > 0$. We consider $C_{\Rs^d}[0,T]$ endowed with the uniform topology. Denote by $C_{\Rs^d}[0, \infty)$  the space of continuous functions from $[0,\infty)$ to $\mathbb{R}^d$  endowed  with the metric
\begin{equation}\label{def:rho}
 \rho(f, g) := \sum_{k=1}^{\infty}\frac{1}{2^k} \sup_{0 \le t \le k}(||f(t) - g(t)|| \wedge 1) \,,  \ f, \, g \in  C_{\Rs^d}[0, \infty) \,.
\end{equation}
% It is well-known that if a sequence of random functions in $C_{\Rs^d}[0, \infty)$ converges in probability under the uniform metric in $C_{\Rs^d}[0, T]$ for all $T >0$, then it also converges in probability under the metric $\rho$ \com{this last paragraph  could be removed once we fix the remark 3.1}.

Our first result is a Central Limit Theorem for the $p_n$-\Name* when  $\beta>1/2$ and $d\geq 2$. Henceforth,  for a column vector $a \in \mathbb{R}^d$, we denote by $a^T$ its transpose.
%Here we will set $p_n = K'n^{-1/2}$ \com{$p_n = K'n^{-\beta}$?} where $\beta > 1/2$ and $K'$ is a positive constant such that $K' \in (0,1]$.

\begin{proposition}\label{pn-WGERW-Gauss} 
Let $X$ be a $p_n$-\Name* in direction $\ell$ on $\ZZ^d$, with $d \ge 2$,  $p_n= \mathcal{C}n^{-\beta} \wedge 1$, with $\beta > 1/2$. 
%We define
%\begin{equation*}
    %B_t^{n} = \frac{X_{\lfloor nt \rfloor}}{n^{1/2}} \quad \text{for } t \in \mathbb{R}^{+}\;.
%\end{equation*}
Suppose that 
    \begin{equation*}
    \lim_{k \to \infty} k^{-1/2} \EE\Big[ \sup_{1 \leq i \leq k} \| \xi_i \| \Big]  = 0 \,, \quad %\text{ and}
    \end{equation*}
    and that there exists $C=((c_{i,j}))$ a continuous $d \times d$ matrix-valued function on $[0,\infty)$ satisfying $C(0) = 0$  and 
\begin{equation*}
\sum_{i,j = 1}^d (c_{i,j}(t) - c_{i,j}(s))\alpha_i \alpha_j \geq 0 \quad \text{for any } \alpha \in \mathbb{R}^d, \quad t > s \geq 0\,,
\end{equation*}
such that
\begin{equation}\label{condGaussiano}
    \frac{1}{n}\sum_{i=1}^{\lfloor nt \rfloor} \xi_i \xi_i^T \xrightarrow[n \to \infty]{} C(t)  \ \textrm{ in probability}\,.
    \end{equation}
% where $C=((c_{i,j}))$ is a continuous $d \times d$ matrix-valued function on $[0,\infty)$ satisfying $C(0) = 0$  and 
% \begin{equation*}
% \sum_{i,j = 1}^d (c_{i,j}(t) - c_{i,j}(s))\alpha_i \alpha_j \geq 0 \quad \text{for any } \alpha \in \mathbb{R}^d, \quad t > s \geq 0\,.
% \end{equation*}
Then $\{\Hat{B}_{\cdot}^n\}_{n\geq 1}$ converges in distribution to a process with independent Gaussian increments with sample paths in $C_{\mathbb{R}^d}[0, \infty)$.
\end{proposition}

From Proposition~\ref{pn-WGERW-Gauss} we obtain the following corollary.

\begin{corollary}\label{pnESRW->BM}
Let $X$ be a $p_n$-\Nametwo{} in direction $\ell$ on $\ZZ^d$, with $d \ge 2$, $p_n= Cn^{-\beta} \wedge 1$, with $\beta > 1/2$. %We define
%\begin{equation*}
 %   B_t^{n} = \frac{X_{\lfloor nt \rfloor}}{n^{1/2}} \quad \text{for } t \in \mathbb{R}^{+}\;.
%\end{equation*}
Then $\{\Hat{B}_{\cdot}^n\}_{n\geq 1}$ converges in distribution to a $d$-dimensional Brownian Motion in $C_{\Rs^d}[0,\infty)$.
\end{corollary}

\begin{remark}
Note that  the $p_n$-\Name* when $\beta>1$ eventually behaves as a $d$-dimensional martingale and, in this case, the above proposition (essentially) reduces to  \cite[Theorem 7.1.4]{ethier2009markov}. In light of that, Proposition~\ref{pn-WGERW-Gauss} is a simple and not too surprising result which we state and prove for completeness. The main and more technical results in this paper are related to the case $\beta=1/2$.
\end{remark}

% \begin{remark}\label{rem_cond_antes}
% Let us point out that Theorem~\ref{pn-WGERW-Gauss} holds true under slightly weaker conditions. Specifically the Condition~\ref{condiçao I*} and the sequence $\{p_n\}_{n \ge 1}$ can be more general. As it emerges from the proof,  for the statement of Theorem~\ref{pn-WGERW-Gauss} be true we only need that $\sum_{i=1}^{\lfloor nt \rfloor} p_i \EE[||\gamma_i||] = o(\sqrt{n})$ (for more details, see Remark~\ref{rem_cond_frac} ). 
% %
% % \cm{then it will be possible to prove that the second sum portion in~\eqref{p_n-WGERW_incrementos} goes to zero in probability. Hence we finish the poof with Slutsky's Theorem (Theorem 11.4 from~\cite{gut2005probability})}.
% \end{remark}

%\comu{Podemos colocar um remark. Bastaria ter $p_n \sim o(\sqrt{n})$ e $\sup \sqrt{n}p_n E[\|\gamma_n\|] < \infty$} \cm{Coloquei um remark após a prova, naõ exatamente isso.}

%\com{Eu sugiro colocar o remark aqui!}



We now consider $\beta = 1/2$. In this case, our results are for the  $p_n$-\Nametwo{} and they depend on the dimension $d$. First, we  present a Central Limit Theorem in $d=2$.

\begin{theorem}\label{pn-ERW-d=2} 
Let $X$ be a two dimensional $p_n$-\Nametwo{} in direction $\ell$ on $\ZZ^2$ with $p_n= \mathcal{C} n^{-1/2} \wedge 1$. %We define,
%\begin{equation*}
 %   B_t^{n} = \frac{X_{\lfloor nt \rfloor}}{n^{1/2}} \quad \text{for } t \in \mathbb{R}^{+}\;.
%\end{equation*}
Then $\{\Hat{B}_{\cdot}^n\}_{n\geq 1}$ converges in distribution to a $2$-dimensional Brownian Motion in $C_{\Rs^2}[0, \infty)$.
\end{theorem}

\begin{remark}
Note that in Corollary~\ref{pnESRW->BM} and Theorem~\ref{pn-ERW-d=2} the $p_n$-\Nametwo{} under  diffusive scaling 
% \sout{a suitable rescaling} 
converges in distribution to a Brownian Motion and has no ballisticity, differently from what happens in the \Nametwo{} (see~\cite{benjamini2003excited}, \cite{kozma2003excited} and \cite{kozma2005excited}). 
% \cm{For simplicity, when we refer to the \textit{rescaled $p_n$-\Nametwo{}}, we mean the $p_n$-\Nametwo{} under a suitable rescaling.}
% \com{As far as I understand, the reviewer is suggesting to replace (everywhere in the paper),  "$p_n$-\Nametwo{} under a suitable rescaling" with  "rescaled $p_n$-\Nametwo{}" after we state once that we only work under diffusive scaling; I do agree!}
\end{remark}

We now state our result for the $p_n$-\Nametwo{} in higher dimensions. Here we obtain that the rescaled $p_n$-\Nametwo{} is tight and every limit point is stochastically dominated in the drift direction $\ell$ from above and below by a Brownian Motion plus a continuous function in $[0, \infty)$. 

% Let us define the set $\bD \subset \{e_1, \dots, e_d\}$, where $d \ge 4$ and $1 \leq k:= |\bD| \leq d-3$. Now set $\ell_{\bD}$ as a direction in the unit sphere in dimension $d$, that is, $\ell_{\bD} \in \mathbb{S}^{d-1}$, such that $\ell_{\bD} = \sum_{i=1}^k \alpha_i x_i$, where $\alpha_i \in [0,1]$ and $x_i \in \bD$, both for all $1 \leq i \leq k$. In essence,  $\ell_{\bD}$ is a direction in the unit sphere in dimension $d$ determined by the canonical directions of the set $\bD$. \comu{esses dois parágrafos vão para um remark?}

% We also set $\pi_d$ as the probability that the random walk with increments $\{\xi_i\}_{i\geq 0}$ 
% ($\ZZ^d$-valued i.i.d. random variables with zero-mean vector and finite variance), never returns to the origin. Moreover, if  $X$ is a $p_n$-\Nametwo{} in direction $\ell_{\bD}$ and  $\mathcal{P}_{\bD^c}$ denotes the projection on $\bD^c$, then $\pi_{d-k}$ denotes the probability that the $(d-k)$-dimensional lazy random walk with increments $\{\mathcal{P}_{\bD^c}(\xi_i)\}_{i\geq 0}$  never returns to the origin. Note that $\pi_{d-k} \le \pi_{d}$.

Set $\pi_d$ as the probability that the random walk with increments $\{\xi_i\}_{i\geq 0}$ never returns to the origin. We also use the symbol $\preceq$ to denote stochastic domination between two process, i.e., if $\mathcal{Y}_\cdot$ and $\mathcal{Z}_\cdot$ are random elements of $C_\mathbb{R}[0,\infty)$, then $\mathcal{Y}_\cdot \preceq \mathcal{Z}_\cdot$ if we can couple both processes such that (using the same notation for the ``possibly'' distinct versions of the processes)
$$
\PP \big[ \forall t \in [0,\infty) : \mathcal{Y}_t \le \mathcal{Z}_t \big]=1\,.
$$


\begin{theorem}\label{pn-ERW-d=>4} 
Let $X$ be a $p_n$-\Nametwo{} in direction $\ell$ on $\ZZ^d$ with   $p_n= \mathcal{C} n^{-1/2} \wedge 1$.  Then, for any $d\geq 3$,  the process $\{\Hat{B}_{\cdot}^n\}_{n\ge 1}$ is tight in $C_{\Rs^d}[0, \infty)$.
 Moreover, setting $\mu_{\gamma} := \EE[\gamma_i \cdot \ell]$, we have that
 \begin{enumerate}[a)]
    \item For any $d\geq 3$, there exists a Brownian Motion $W_{\cdot}$ such that for every limit point $\mathcal{Y}_{\cdot}$ of $\{\Hat{B}_{\cdot}^n\}_{n\ge 1}$ it holds that
\begin{align*}
\{\mathcal{Y}_t \cdot \ell\}_{t\ge 0} \preceq \left\{W_t \cdot \ell + 2 c_2 \sqrt{t}\right\}_{t\ge 0} \,,   \quad \text{ with $c_2 =\mu_\gamma \sqrt{\pi_d}$\,; }
\end{align*}
 \item For any $d\geq 22$, there exists a Brownian Motion $W_{\cdot}$ such that for every limit point $\mathcal{Y}_{\cdot}$ of $\{\Hat{B}_{\cdot}^n\}_{n\ge 1}$ it holds that
\begin{align*}
\left\{W_t \cdot \ell + 2 c_1 \sqrt{t}\right\}_{t\ge 0} \preceq \{\mathcal{Y}_t \cdot \ell\}_{t\ge 0} \,,  \quad \text{ with $c_1 = \mu_\gamma ( 1-\sqrt{1 -  \pi_{d}} )$\,. } 
\end{align*}
In $a),\, b)$ above, $\preceq$ means ``stochastically less or equal to''. 
\end{enumerate}
\end{theorem}

\begin{corollary}\label{cor:cone}
Let $X$ be a $p_n$-\Nametwo{} in direction $\ell$ on $\ZZ^d$ with   $p_n= \mathcal{C} n^{-1/2} \wedge 1$.  Then, for any $d\geq 22$, there exists a Brownian Motion $W_{\cdot}$ such that for every limit point $\mathcal{Y}_{\cdot}$ of $\{\Hat{B}_{\cdot}^n\}_{n\ge 1}$ we have that 
\begin{align*}
\left\{W_t \cdot \ell + 2 c_1 \sqrt{t}\right\}_{t\ge 0} \preceq \{\mathcal{Y}_t \cdot \ell\}_{t\ge 0} \preceq \left\{W_t \cdot \ell + 2 c_2 \sqrt{t}\right\}_{t\ge 0} \,,
\end{align*}
where $c_1, c_2$ are as in Theorem~\ref{pn-ERW-d=>4}.
\end{corollary}

Informally speaking, in Corollary~\ref{cor:cone} we obtain that every limit point of the rescaled $p_n$-\Nametwo{} in direction $\ell$ will be confined within a sort of time dependent ``cone'' region, with high probability (see, Figure~\ref{fig:cone}).

\begin{remark}\label{rem:missing-cases}
The reason for the difference in the dimension appearing in $a)$ and $b)$ in Theorem~\ref{pn-ERW-d=>4} is due to the fact that the proof technique relies on a control on the range of a $p_n$-\Nametwo{}. While we have a tight upper bound on the range which works in any dimension (see, Proposition~\ref{prop:RangeERW}), the  technique used to obtain the lower bound does not work for $3\leq d\leq 21$ (see, Proposition~\ref{prop:RangeERW_lower}).  
However, by imposing some restriction to the drift direction $\ell$, it is possible to obtain a weaker version of Corollary~\ref{cor:cone}, which holds on any dimension $d\geq 4$. Since we consider it  a minor result as compared to Theorem~\ref{pn-ERW-d=>4},   we opted to relegate its precise statement to Appendix~\ref{sec:appendix-mainTheorem}.  
\end{remark}


% \begin{figure}[h]
%     \centering
% \tikzset{every picture/.style={line width=0.75pt}} %set default line width to 0.75pt        

% \begin{tikzpicture}[x=0.75pt,y=0.75pt,yscale=-1,xscale=1]
% %uncomment if require: \path (0,300); %set diagram left start at 0, and has height of 300

% %Straight Lines [id:da7123294838448495] 
% \draw    (112,121.25) -- (219,218.75) ;
% %Straight Lines [id:da030786395375731024] 
% \draw    (162.75,167.38) -- (242.57,89.15) ;
% \draw [shift={(244,87.75)}, rotate = 135.58] [color={rgb, 255:red, 0; green, 0; blue, 0 }  ][line width=0.75]    (10.93,-3.29) .. controls (6.95,-1.4) and (3.31,-0.3) .. (0,0) .. controls (3.31,0.3) and (6.95,1.4) .. (10.93,3.29)   ;
% %Straight Lines [id:da9825301152582373] 
% \draw  [dash pattern={on 0.75pt off 0.75pt}]  (162.75,167.38) .. controls (161.48,165.39) and (161.83,163.76) .. (163.82,162.49) .. controls (165.8,161.21) and (166.15,159.58) .. (164.88,157.6) .. controls (163.61,155.61) and (163.96,153.99) .. (165.95,152.72) .. controls (167.93,151.44) and (168.28,149.81) .. (167.01,147.83) .. controls (165.74,145.84) and (166.09,144.22) .. (168.08,142.95) .. controls (170.06,141.67) and (170.41,140.04) .. (169.14,138.06) .. controls (167.87,136.07) and (168.22,134.45) .. (170.21,133.18) .. controls (172.19,131.9) and (172.54,130.27) .. (171.27,128.29) .. controls (170,126.3) and (170.35,124.68) .. (172.34,123.41) .. controls (174.32,122.13) and (174.67,120.5) .. (173.4,118.52) .. controls (172.13,116.53) and (172.48,114.91) .. (174.47,113.64) .. controls (176.45,112.36) and (176.8,110.73) .. (175.53,108.75) .. controls (174.26,106.76) and (174.61,105.14) .. (176.6,103.87) .. controls (178.58,102.59) and (178.93,100.96) .. (177.66,98.98) .. controls (176.39,96.99) and (176.74,95.37) .. (178.73,94.1) .. controls (180.71,92.82) and (181.06,91.19) .. (179.79,89.21) .. controls (178.52,87.22) and (178.87,85.6) .. (180.86,84.33) .. controls (182.84,83.05) and (183.19,81.42) .. (181.92,79.44) .. controls (180.65,77.45) and (181,75.83) .. (182.99,74.56) .. controls (184.97,73.28) and (185.32,71.65) .. (184.05,69.67) .. controls (182.78,67.68) and (183.13,66.06) .. (185.12,64.79) -- (186,60.75) -- (186,60.75) ;
% %Straight Lines [id:da5469873734744604] 
% \draw  [dash pattern={on 0.75pt off 0.75pt}]  (165.5,170) .. controls (166.67,167.95) and (168.27,167.51) .. (170.32,168.68) .. controls (172.37,169.85) and (173.98,169.41) .. (175.15,167.36) .. controls (176.32,165.31) and (177.92,164.88) .. (179.97,166.05) .. controls (182.02,167.22) and (183.62,166.78) .. (184.79,164.73) .. controls (185.96,162.68) and (187.57,162.24) .. (189.62,163.41) .. controls (191.67,164.58) and (193.27,164.14) .. (194.44,162.09) .. controls (195.61,160.04) and (197.21,159.6) .. (199.26,160.77) .. controls (201.31,161.94) and (202.91,161.5) .. (204.08,159.45) .. controls (205.25,157.4) and (206.86,156.97) .. (208.91,158.14) .. controls (210.96,159.31) and (212.56,158.87) .. (213.73,156.82) .. controls (214.9,154.77) and (216.5,154.33) .. (218.55,155.5) .. controls (220.6,156.67) and (222.21,156.23) .. (223.38,154.18) .. controls (224.55,152.13) and (226.15,151.69) .. (228.2,152.86) .. controls (230.25,154.03) and (231.85,153.59) .. (233.02,151.54) .. controls (234.19,149.49) and (235.8,149.06) .. (237.85,150.23) .. controls (239.9,151.4) and (241.5,150.96) .. (242.67,148.91) .. controls (243.84,146.86) and (245.44,146.42) .. (247.49,147.59) .. controls (249.54,148.76) and (251.15,148.32) .. (252.32,146.27) .. controls (253.49,144.22) and (255.09,143.78) .. (257.14,144.95) .. controls (259.19,146.12) and (260.79,145.68) .. (261.96,143.63) .. controls (263.13,141.58) and (264.73,141.15) .. (266.78,142.32) .. controls (268.83,143.49) and (270.44,143.05) .. (271.61,141) .. controls (272.78,138.95) and (274.38,138.51) .. (276.43,139.68) -- (278,139.25) -- (278,139.25) ;
% %Straight Lines [id:da11795329726550197] 
% \draw  [dash pattern={on 4.5pt off 4.5pt}]  (167.5,146.5) -- (186.5,165.25) ;
% %Straight Lines [id:da300875256242809] 
% \draw  [dash pattern={on 4.5pt off 4.5pt}]  (172,135) -- (196,160.25) ;
% %Straight Lines [id:da8860639268549979] 
% \draw  [dash pattern={on 4.5pt off 4.5pt}]  (176.5,123) -- (207.5,157.75) ;
% %Straight Lines [id:da6796902558092259] 
% \draw  [dash pattern={on 4.5pt off 4.5pt}]  (180,107.5) -- (221.75,154.63) ;
% %Straight Lines [id:da7178424539547166] 
% \draw  [dash pattern={on 4.5pt off 4.5pt}]  (181,92.5) -- (233,149.25) ;
% %Straight Lines [id:da10145270887795221] 
% \draw  [dash pattern={on 4.5pt off 4.5pt}]  (183,77) -- (247.5,148.75) ;

% % Text Node
% \draw (246,91.15) node [anchor=north west][inner sep=0.75pt]  [font=\tiny]  {$t$};
% % Text Node
% \draw (117,110.4) node [anchor=north west][inner sep=0.75pt]  [font=\tiny]  {$\ell $};
% % Text Node
% \draw (187,49.4) node [anchor=north west][inner sep=0.75pt]  [font=\tiny]  {$W_{t} \ \cdot \ell \ +\ c_{2}\sqrt{t}$};
% % Text Node
% \draw (255,120.4) node [anchor=north west][inner sep=0.75pt]  [font=\tiny]  {$W_{t} \ \cdot \ell \ +\ c_{1}\sqrt{t}$};


% \end{tikzpicture}
        


% \caption{\color{cyan}Representation of the space-time region in direction $\ell$ between $W_t \cdot \ell + 2 c_1 \sqrt{t}$ and $W_t \cdot \ell + 2 c_1 \sqrt{t}$, $t\ge 0$. }
% \label{fig:cone}

% \end{figure}


\begin{comment}
\begin{figure}[h]
    \centering
    %%%%
    \newcommand{\Emmett}[5]{% points, advance, rand factor, options, end label, random seed
    \pgfmathsetseed{12}%
\draw[#4] (0,0)
\foreach \x in {1,...,#1}
{ 
-- ++(#2,rand*#3) 
% -- ++(#2,{rand*#3 + 0.03*sqrt(\x)*#3}) 
}
node[right] (a) {#5};
}

    
\begin{tikzpicture}
\draw[gray, ->, -{Stealth[length=2mm, width=1.5mm]}] (1,-2.5) -- (-1,2.5);
\node[help lines] at (1.2, -2.4) {$\ell$};
\draw[gray,->,-{Stealth[length=2mm, width=1.5mm]}] (0,0) -- (6,0);
\node[help lines] at (6, -0.3) {$t$};

\Emmett{400}{0.014}{0.15}{black}{$W_{t}$}



\draw[help lines, dashed] ($(0,0)!(a)!(1,-2.5)$) -- (a);
\draw ($(0,0)!(a)!(1,-2.5)$) node (b) {}; 

\draw ($(b)! 0.75!(-1,2.5)$) node (d) {};
\draw ($(d)! 0.55!(b)$) node (e) {};

\draw[black, very thick,|-|] (d) -- (e);

\draw[black] (d) node[left]  {$W_{t} \cdot \ell+2c_2\sqrt{t}\quad$}; 
\draw[black] (e) node[left]  {$W_{t} \cdot \ell+2c_1\sqrt{t}\quad$}; 
\draw[black] ($(0,0)!(a)!(1,-2.5)$) node[left]  {$W_{t} \cdot \ell$}; 
\end{tikzpicture}
\caption{\color{cyan}Representation of the width at time $t>0$ in direction $\ell$ between $W_{t}\cdot \ell + 2 c_1 \sqrt{t}$ and $W_{t} \cdot \ell + 2 c_1 \sqrt{t}$.\comu{Ainda não entendo bem a figura.}}
\label{fig:cone}
\end{figure}



\begin{figure}[h]
    \centering
    %%%%
    \newcommand{\Emmett}[6]{% points, advance, rand factor, options, end label, random seed
    % \pgfmathsetseed{12}%
\draw[#4] (0,0)
\foreach \x in {1,...,#1}
{ 
-- ++(#2,{rand*#3 + #6*sqrt(\x)*#3}) 
% -- ++(#2,{rand*#3 + 0.03*sqrt(\x)*#3}) 
}
node[right] (a) {#5};
}

    
\begin{tikzpicture}
\draw[gray, ->, -{Stealth[length=2mm, width=1.5mm]}] (0,0) -- (0,4);
\draw[gray,->,-{Stealth[length=2mm, width=1.5mm]}] (0,0) -- (6,0);
\node[help lines] at (6, -0.3) {$t$};

 \pgfmathsetseed{12}
\Emmett{250}{0.015}{0.15}{black, thick}{$W_{t}\cdot \ell + 2c_2 \sqrt{t}$}{0.005}

 \pgfmathsetseed{4}
\Emmett{250}{0.015}{0.15}{black, thick}{$W_{t}\cdot \ell + 2c_1 \sqrt{t}$}{0.0005}

 \pgfmathsetseed{16}
\Emmett{250}{0.015}{0.15}{black,dotted,thick}{$\mathcal{Y}_t \cdot \ell$}{0.004}

\end{tikzpicture}
\caption{\color{cyan}Representation of the width at time $t_0>0$ in direction $\ell$ between $W_{t_0}\cdot \ell + 2 c_1 \sqrt{t_0}$ and $W_{t_0} \cdot \ell + 2 c_1 \sqrt{t_0}$.\comu{Ainda não entendo bem a figura.}}
\label{fig:cone}
\end{figure}
\end{comment}


\begin{figure}
\centering
\begin{subfigure}[b]{0.6\textwidth}
  \centering
    \newcommand{\Emmett}[5]{% points, advance, rand factor, options, end label, random seed
    \pgfmathsetseed{12}%
\draw[#4] (0,0)
\foreach \x in {1,...,#1}
{ 
-- ++(#2,rand*#3) 
% -- ++(#2,{rand*#3 + 0.03*sqrt(\x)*#3}) 
}
node[right] (a) {#5};
}
    
\begin{tikzpicture}[scale=0.65]
\draw[gray, ->, -{Stealth[length=2mm, width=1.5mm]}] (0.5,-1.25) -- (-1,2.5);
\node[help lines] at (-0.5, 2.4) {$\ell$};
\draw[gray,->,-{Stealth[length=2mm, width=1.5mm]}] (0,0) -- (5,0);
\node[help lines] at (5, -0.3) {$t$};

\Emmett{250}{0.014}{0.15}{black}{$W_{t}$}

\draw[help lines, dashed] ($(0,0)!(a)!(1,-2.5)$) -- (a);
\draw ($(0,0)!(a)!(1,-2.5)$) node (b) {}; 

\draw ($(b)! 0.75!(-1,2.5)$) node (d) {};
\draw ($(d)! 0.55!(b)$) node (e) {};

\draw[black, very thick,|-|] (d) -- (e);

\draw[black] (d) node[left]  {\small $W_{t}\! \cdot \! \ell+2c_2\sqrt{t}\quad$}; 
\draw[black] (e) node[left]  {\small $W_{t} \!\cdot\! \ell+2c_1\sqrt{t}\quad$}; 
\draw[black] ($(0,0)!(a)!(1,-2.5)$) node[left]  {\small $W_{t}\! \cdot \! \ell$}; 
\end{tikzpicture}
  \caption{}
  % \label{fig:sub1}
\end{subfigure}%
\begin{subfigure}[b]{.4\textwidth}
  \centering
    \newcommand{\Emmett}[6]{% points, advance, rand factor, options, end label, random seed
    % \pgfmathsetseed{12}%
\draw[#4] (0,0)
\foreach \x in {1,...,#1}
{ 
-- ++(#2,{rand*#3 + #6*sqrt(\x)*#3}) 
% -- ++(#2,{rand*#3 + 0.03*sqrt(\x)*#3}) 
}
node[right] (a) {#5};
}
\begin{tikzpicture}[scale=0.56]
\draw[gray, ->, -{Stealth[length=2mm, width=1.5mm]}] (0,0) -- (0,3.5);
\draw[gray,->,-{Stealth[length=2mm, width=1.5mm]}] (0,0) -- (6,0);
\node[help lines] at (6, -0.3) {$t$};

%\pgfmathsetseed{12}
\pgfmathsetseed{19}
\Emmett{200}{0.015}{0.15}{black, thick}{\small $W_{t}\!\cdot \!\ell + 2c_2 \sqrt{t}$}{0.015}

%\pgfmathsetseed{4}
\pgfmathsetseed{19}
\Emmett{200}{0.015}{0.15}{black, thick}{\small $W_{t}\!\cdot \!\ell + 2c_1 \sqrt{t}$}{0.000005}

 \pgfmathsetseed{19}
\Emmett{210}{0.015}{0.15}{black,dotted}{\small $\mathcal{Y}_t \! \cdot \! \ell$}{0.007}
\end{tikzpicture}
  \caption{}
\end{subfigure}
\caption{(A) Region width at time $t$ along direction $\ell$ to which $\mathcal{Y}_t \cdot \ell$ belongs a.s.; (B) Cone region between time $0$ and time $t$, where the whole trajectory of $\mathcal{Y}_t \cdot \ell$ (dotted line) is confined a.s..}
\label{fig:cone}
\end{figure}



Now we propose a conjecture for the $p_n$-\Nametwo{}.
% in direction $\ell \in \mathbb{S}^{d-1}$, on $\ZZ^d$, with $d \ge 3$ and $p_n = \mathcal{C}n^{-1/2} \wedge 1$.

\begin{conjecture}\label{conj_dist}
Let  $X$ be a $p_n$-\Nametwo{} in direction $\ell \in \mathbb{S}^{d-1}$, on $\ZZ^d$ with $d \ge 3$,  $p_n= \mathcal{C} n^{-1/2} \wedge 1$. %We define,
%\begin{equation*}
 %   B_t^{n} = \frac{X_{\lfloor nt \rfloor}}{n^{1/2}} \quad \text{for } t \in \mathbb{R}^{+}\;.
%\end{equation*}
Then there exists a constant $c \in (c_1,c_2)$ such that $\{\Hat{B}_{\cdot}^n\}_{n\geq 1}$ converges in distribution to 
$\{W_{t} \cdot \ell + 2 c \sqrt{t}\}_{t>0}$,
where $W_{\cdot}$ is a Brownian Motion. 
\end{conjecture}

%\cm{The positive constant $c$ in Conjecture~\ref{conj_dist}, we believe that will be such as $c_1 < c < c_2$, where $c_1$ and $c_2$ are from Theorem~\ref{pn-ERW-d=>4}.}

The proof of Theorem~\ref{pn-ERW-d=2} and Theorem~\ref{pn-ERW-d=>4} rely on a control on the range of a $p_n$-\Nametwo{}. Given $X$  a $p_n$-\Nametwo{},  its range  $\Rr^X_{n}$ is defined as the random set
\[
\Rr^X_{n}:=\{X_0, X_1, \ldots, X_n\} \subset \mathbb{Z}^d\,,
\]
 i.e.,  the set of sites visited by  $X$ up to time $n$.
Below we state two results which combined provide a Strong Law of Large Numbers (SLLN) for $\Rr^X_{n}$ when the dimension is either $d=2$ or $d\geq 22$. We conjecture that the SLLN  holds in any dimension. We provide a tight upper bound that works in any dimension, but the technique employed to prove the lower bound does not work for $3\leq d\leq 21$. In passing, we mention that the absence of the Markov property in our model makes the control of the range a hard problem to tackle.

% Considering the representation in \eqref{xn-incremnto1} for $X$ a $p_n$-\Nametwo{} (see, Section~\ref{sec:p_n-ERW}), 
% Let us denote by $\pi_d$ the probability of a random walk with i.i.d. increments (with zero mean and finite variance) given by the corresponding $\{\xi_i\}_{i\geq 0}$ never returning to the origin. 
%
\begin{proposition}\label{prop:RangeERW} 
Let  $X$ be a $p_n$-\Nametwo{} in direction $\ell$ on $\ZZ^d$ with $d\geq 2$, $p_n= \mathcal{C} n^{-\beta} \wedge 1$ and $\beta \ge 1/2$. Then, it holds that
\begin{align*}
    \limsup_{n \to \infty} \frac{|\Rr_n^X|}{n} \le \pi_d \ \text{ a.s..}
\end{align*}
\end{proposition}

%
Note that for $d=2$, we have that $\pi_d=0$, whereas for $d\geq 3 $, $\pi_d\in (0,1]$.  %Below, we propose a conjecture about the range of the $p_n$-\Nametwo{} on $\ZZ^d$, in direction $\ell \in \mathbb{S}^{d-1}$,  with $p_n= \mathcal{C}n^{-\beta} \wedge 1$, with $\beta\geq 1/2$ and  $d \ge 2$.
%
\begin{proposition}\label{prop:RangeERW_lower} 
Let  $X$ be a $p_n$-\Nametwo{} in direction $\ell$ on $\ZZ^d$ with $d\geq 22$, $p_n= \mathcal{C} n^{-\beta} \wedge 1$ and $\beta \ge 1/2$. Then, it holds that
\begin{align*}
\liminf_{n \to \infty} \frac{|\Rr_n^X|}{n} \ge \pi_d \ \text{ a.s..}
\end{align*}
\end{proposition}

By combining the results obtained in Proposition~\ref{prop:RangeERW} and Proposition~\ref{prop:RangeERW_lower}, we derived a SLLN for the range of the $p_n$-\Nametwo{} in any direction $\ell \in \mathbb{S}^{d-1}$ on $\ZZ^d$ with $d \ge 22$. The proofs of both propositions are given in Section~\ref{sec:rangeERW}. 

{\begin{corollary}[SLLN]\label{col:LLN_range}
Let $X$ be a $p_n$-\Nametwo{} in direction $\ell \in \mathbb{S}^{d-1}$ on $\ZZ^d$ with either $d=2$ or $d \geq 22$, $p_n= \mathcal{C}n^{-\beta} \wedge 1$ and $\beta \ge 1/2$. Then, it holds that 
\begin{equation*}
    \frac{|\Rr_n^X|}{n} \xrightarrow[n \to \infty]{} \pi_d \ \text{ a.s..}
\end{equation*}    
\end{corollary}

% \textcolor{red}{Below, we propose a conjecture about the range of the $p_n$-\Nametwo{} on $\ZZ^d$, in direction $\ell \in \mathbb{S}^{d-1}$,  with $p_n= \mathcal{C}n^{-\beta} \wedge 1$, with $\beta\geq 1/2$ and  $2 \le d \le 21$.}\comu{Acho que podemos remover a frase e deixar a conjectura vir direto.}

\begin{conjecture}\label{conj_range}
Let $X$ be a $p_n$-\Nametwo{} in direction $\ell\in \mathbb{S}^{d-1}$ on $\ZZ^d$ with $3 \le d \le 21$, $p_n= \mathcal{C}n^{-\beta} \wedge 1$, and $\beta\geq 1/2$. Then, we have that
\begin{equation*}
    \frac{|\Rr_n^X|}{n} \xrightarrow[n \to \infty]{} \pi_d \ \text{ a.s..}
\end{equation*}
\end{conjecture}

\begin{remark} If  Conjecture~\ref{conj_range} holds true, 
we would be able to extend the result in Corollary~\ref{cor:cone} to any dimension $d\geq 3$.
Note, however, that this is not yet enough to imply Conjecture~\ref{conj_dist}, since $c_1<c_2$ for all $d\geq 3$.
%%%%%OLD CONJECTURE %%%%%%%%
% If  Conjecture~\ref{conj_range} holds true, 
% %we could not  prove yet the Conjecture~\ref{conj_dist}. However, 
% we would be able to extend the result in Theorem~\ref{pn-ERW-d=>4} to $d=3$ and to any direction in the unit sphere.
% % \sout{(see Remark~\ref{rem:restriction} in  Section~\ref{sec: d>4})}. 
% Note, however, that this is not yet enough to imply Conjecture~\ref{conj_dist} {\color{cyan} as in the case of dimension $d \ge 22$}. It is worth mentioning that, despite our attempts,  proving  Conjecture~\ref{conj_range} via a coupling with the simple symmetric random walk  turned out to be quite hard, and the dimension should play an important role in the proof argument.
\end{remark}


Table~\ref{table:1} provides a summary of the main results concerning the $p_n$-\Name{}, with $p_n=\mathcal{C}n^{-\beta} \wedge 1$ for different values of $\beta$ and dimension $d$. 

\renewcommand\thetable{\thesection.\arabic{table}}

\begin{table}[!h]    
\caption{Results for the rescaled $p_n$-\Name{}.\\ }
\label{table:1}
\begin{tabular}{
|p{0.40\textwidth}
|p{0.54\textwidth}|}
% \hline 
%  $\displaystyle p$-GERW \hfill ($\displaystyle d  \ge 2$, $p\in(0,1]$) &  \ ballisticity in the drift direction for every $\displaystyle p\  >\ 0$. \\
% \hline 
%  $\displaystyle p$-SGERW \hfill ($\displaystyle d  \ge 2$, $p\in(0,1]$) &  \ besides ballisticity, LLN and CLT. \\
% \hline 
%  $\displaystyle p_{n}$-GERW \ \hfill ($\displaystyle \beta <  1/6$, $\displaystyle d \geq 2$) &  \ positive probability of never returning to the origin (in the direction $\ell$) \\
%\hline 
 %$\displaystyle p_{n}$-\Nameone{} \ \ \hfill ($\displaystyle \beta  >1/2$, $\displaystyle d\geq 2$) &  \ convergence in distribution to standard Brownian Motion. \\
\hline 
 $\displaystyle p_{n}$-GERW* \hfill ($\displaystyle \beta   > 1/2$, $\displaystyle d\geq 2$) &   {\it \small Convergence in distribution to a Gaussian Process.} \\
\hline 
 $\displaystyle p_{n}$-\Nametwo{} \hfill ($\displaystyle \beta =1/2$, $\displaystyle d=2$) &  {\it \small Convergence in distribution to a Brownian Motion.} \\
\hline 
 $\displaystyle p_{n}$-ERW \hfill ($\displaystyle \beta =1/2$, $\displaystyle  d\geq 3$) &   {\it \small  Tightness  and any limit point is stochastically dominated in the drift direction from above by a Brownian Motion plus a multiple of square root of time.}
 % \sout{All sub-sequences converge, in distribution,  to a process which is stochastically dominated in the drift direction below and above by a Brownian Motion plus a continuous function.}
 \\
\hline 
$\displaystyle p_{n}$-ERW \hfill ($\displaystyle \beta =1/2$, $\displaystyle d\geq 22$) &   {\it \small  Any limit point is also stochastically dominated in the drift direction from below by a Brownian Motion plus a multiple of square root of time.}
 \\
 \hline
  $\displaystyle p_{n}$-GERW \hfill ($\displaystyle \beta$ small, $\displaystyle d\geq 2$) &   {\it \small Directional transience (see  \cite{alves2022note}).}\\
 \hline
\end{tabular}
\end{table}
%\cm{Hence with the Conjecture~\ref{conj_range} we are able to prove the following result, which is similar with Theorem~\ref{pn-ERW-d=>4}. However we can expand what we obtain for dimension 3 and every direction in the unit sphere.
%\begin{conjecture}\label{conj_dist}
%Let the process $X$ be a $p_n$-\Nametwo{} in direction $\ell \in \mathbb{S}^{d-1}$, in $\ZZ^d$ with $d \ge 3$,  $p_n= C n^{-1/2} \wedge 1$. %We define,
%\begin{equation*}
 %   B_t^{n} = \frac{X_{\lfloor nt \rfloor}}{n^{1/2}} \quad \text{for } t \in \mathbb{R}^{+}\;.
%\end{equation*}
%Then the process $\Hat{B}_{\cdot}^n$ is tight in $C_{\Rs^d}[0, \infty)$ and there exists a Brownian Motion $W_{\cdot}$ such that for every limit point $Y_{\cdot}$ from the process $B_{\cdot}^n$ it holds that
%\begin{align*}
%W_t \cdot \ell + f(t) \preceq Y_t \cdot \ell \preceq W_t \cdot \ell + g(t) \,,   
%\end{align*}
%where $f$ and $g$ are continuous function on $[0, \infty)$, such that $f(t) = c^\prime_1 \sqrt{t}$ and $g(t) = c_2 \sqrt{t}$ with $c_2 > c^\prime_1 >0$. \end{conjecture}}

%\cm{It is important to notice that $c_2$ in Conjecture~\ref{conj_dist} is the same of Theorem~\ref{pn-ERW-d=>4}, however $c^\prime_1 \le c_1$.}



% We now provide our last result, which states that, for $\beta<1/6$ and $d\geq 2$ the $p_n$-\Name{} in direction $\ell$, in $\ZZ^d$ has a positive probability of never returning to the origin in the $\ell$ direction. 
% . Now we set $\beta < \alpha < 1/6$ and $d \ge 2$, where $\alpha$ is establish in Proposition~\ref{prop41}. We will obtain that the $p_n$-\Name{} in direction $\ell$, in $\ZZ^d$ has a positive probability that never returns to the origin in the $\ell$ direction.}



% Before stating the theorem, we  need some notation. 
% for our next result  \cm{which provides a uniform lower bound on the probability of the $p_n$-\Name{} in direction $\ell$ never return to the origin in this direction.} 
% For every $\ell \in \mathbb{S}^{d-1}$, let $\mathbb{M}_{\ell}$ denote the positive half-space in direction $\ell$, that is, $\mathbb{M}_{\ell} = \{ x \in \ZZ^d : x \cdot \ell > 0 \}$. 
% We define $A$ as the \textit{excitation-allowing set}, which means the set of sites where there is the possibility of having cookies (see Condition~\ref{condição2A}). We set the event $\{\eta(X_0) = \infty\}$ as the event in which  the process $X$ never returns to the origin in the drift direction.


% \begin{theorem}\label{prop43_pnn0}
% Let $X$ be a $p_n$-\Name~in direction $\ell$, in $\ZZ^d$ with $d \ge 2$, where  $p_n = (q_0 +n)^{-\beta}$, with $\beta<1/6$,  $q_0$ is a non negative integer and excitation-allowing set $A \subset \mathbb{Z}^d$ such that $\mathbb{M}_{\ell} \subset A$. There exists $\psi > 0$ depending on the parameters of the model  such that
% \begin{equation*}
% \PP\left[ \eta(X_0) = \infty \right] \geq \PP\left[ X_n \cdot \ell > 0 \text{  for all  } n\geq 1\right] \geq \psi\, . 
% \end{equation*}
% % where $\psi = h^{\lceil r^{-1} \rceil C \left(\frac{3}{\lambda} \right)^{\frac{1}{\delta -1}}} c$,  $c \in (0, 1)$, $\delta = (2-\alpha+\beta)(1/2 + \alpha-\beta)$, 
% % \begin{align*}
% % C & = K^{\frac{1}{\delta-1}} \left( \eta + \lceil r^{-1} \rceil^{\frac{1}{\delta-1}}\right) + q_0\;, 
% % \\
% % \eta & = \left( \frac{ 2-\alpha+\beta}{\vartheta_1 \varphi_1}\right)^{\frac{1}{\varphi_1}}  \, , \quad \varphi_1  = \min \left\{ \alpha-\beta, (2-\alpha+\beta)\vartheta_2 \right\}\, , \nonumber
% % \end{align*}
% % and $\vartheta_1$, $\vartheta_2$ are as in Proposition~\ref{prop42_pnn0}. \comu{Fica um pouco estranho esta referência a proposição 3.3, porque ela só aparece no final do texto. Acho melhor definir $\vartheta_1$, $\vartheta_2$ e fazer menção aos valores aqui no enunciado da 3.3 (mesmo sabendo que usamos a 3.3 para provar o teorema.}
% \end{theorem}


% \br{
% \begin{theorem}\label{prop43_pnn0}
% Let $X$ be a $p_n$-\Name~in direction $\ell$ with $p_n = (q_0 +n)^{-\beta}$, where $q_0$ is a non negative integer and excitation-allowing set $A \subset \mathbb{Z}^d$ such that $\mathbb{M}_{\ell} \subset A$. There exists $\psi > 0$ depending on $d$, $K$, $h$, $r$, $\lambda$, $\alpha$ and $\beta$ such that
% \begin{equation*}
% \PP\left[ \eta(X_0) = \infty \right] \geq \PP\left[ X_n \cdot \ell > 0 \text{  for all  } n\geq 1\right] \geq \psi\, ,
% \end{equation*}
% where $\psi = h^{\lceil r^{-1} \rceil C \left(\frac{3}{\lambda} \right)^{\frac{1}{\delta -1}}} c$,  $c \in (0, 1)$, $\delta = (2-\alpha+\beta)(1/2 + \alpha-\beta)$, 
% \begin{align*}
% C & = K^{\frac{1}{\delta-1}} \left( \eta + \lceil r^{-1} \rceil^{\frac{1}{\delta-1}}\right) + q_0\;, 
% \\
% \eta & = \left( \frac{ 2-\alpha+\beta}{\vartheta_1 \varphi_1}\right)^{\frac{1}{\varphi_1}}  \, , \quad \varphi_1  = \min \left\{ \alpha-\beta, (2-\alpha+\beta)\vartheta_2 \right\}\, , \nonumber
% \end{align*}
% and $\vartheta_1$, $\vartheta_2$ are as in Proposition~\ref{prop42_pnn0}. \comu{Fica um pouco estranho esta referência a proposição 3.3, porque ela só aparece no final do texto. Acho melhor definir $\vartheta_1$, $\vartheta_2$ e fazer menção aos valores aqui no enunciado da 3.3 (mesmo sabendo que usamos a 3.3 para provar o teorema.}
% \end{theorem} }



% Figure~\ref{table:1} provides a summary of the main results concerning $p$-\Name{} and the $p_n$-\Name{}, with $p_n=\mathcal{C}n^{-\beta} \wedge 1$ for different values of $\beta$ and dimension $d$.

% \begin{figure}[!h]
%         \centering
        
% \begin{tabular}{|p{0.40\textwidth}|p{0.54\textwidth}|}
% % \hline 
% %  $\displaystyle p$-GERW \hfill ($\displaystyle d  \ge 2$, $p\in(0,1]$) &  \ ballisticity in the drift direction for every $\displaystyle p\  >\ 0$. \\
% % \hline 
% %  $\displaystyle p$-SGERW \hfill ($\displaystyle d  \ge 2$, $p\in(0,1]$) &  \ besides ballisticity, LLN and CLT. \\
% % \hline 
% %  $\displaystyle p_{n}$-GERW \ \hfill ($\displaystyle \beta <  1/6$, $\displaystyle d \geq 2$) &  \ positive probability of never returning to the origin (in the direction $\ell$) \\
% %\hline 
%  %$\displaystyle p_{n}$-\Nameone{} \ \ \hfill ($\displaystyle \beta  >1/2$, $\displaystyle d\geq 2$) &  \ convergence in distribution to standard Brownian Motion. \\
% \hline 
%  $\displaystyle p_{n}$-GERW* \hfill ($\displaystyle \beta   > 1/2$, $\displaystyle d\geq 2$) &  \ convergence in distribution to a Guassian Process \\
% \hline 
%  $\displaystyle p_{n}$-\Nametwo{} \hfill ($\displaystyle \beta =1/2$, $\displaystyle d=2$) &  \ convergence in distribution to a Brownian Motion \\
% \hline 
%  $\displaystyle p_{n}$-ERW \hfill ($\displaystyle \beta =1/2$, $\displaystyle d\geq 4$) &  \ all sub-sequences converge, in distribution,  to a process which is stochastically dominated in the drift direction below and above by a Brownian Motion plus a continuous function.\\
%  \hline
% \end{tabular}
% \caption{Summary of the results for \Name{}.}
% \label{table:1}
% \end{figure}



% \cm{We now present the $p_n$-\Name{}, a special type of the $\lambda_n$-\Name{} as we will see. Set $\{U_i\}_{i \geq 1}$ as a sequence of i.i.d. random variables with uniform distribution in $[0,1]$ and a sequence $\{p_n\}_{n \ge 1}$ such that $p_n \in (0, 1]$ for all $n \ge 1$. For $i\ge 1$ we define $E_i$ as the event that the process $\{X_n\}_{n \geq 0}$ is, at time $i$,  in an already visited site, i.e.,  $E_i:= \{ \exists\;  k < i \; \text{ such that }\;  X_k = X_i \}$ and $E_0:= \emptyset$. We write $\{X_n\}_{n \geq 0}$ as
% \begin{align}\label{xn-incremnto1}
% \begin{split}
% X_n & = \sum_{i=1}^n  (X_i - X_{i-1})
% \\
% & = \sum_{i=1}^n \big(1_{\{E_{i-1}\}} \xi_i + 1_{\{E_{i-1}^c\}} 1_{\{U_{i} > p_{i} \}} \xi_i + 1_{\{E_{i-1}^c\}}1_{\{ U_{i} \leq p_{i}\}} \gamma_i \big) \,. 
% \end{split}
% \end{align}
%where $\{\xi_i, \FF_i\}_{i \geq 1}$ is an increment of a $d$-martingale with zero mean and $\{\gamma_i, \FF_i\}_{i\geq 1}$ is a random vector such that $\EE[\gamma_i \cdot \ell | \FF_{i-1}] \ge \lambda$ for all $i \ge 1$.  
% We also have that $|| \xi_i||<K$ as well as $||\gamma_i||<K$ for all $i \ge 1$. We can explain the model as the following, given the sequence  $\{p_n\}_{n \ge 1}$, if at time $n$ the process visits a site for the first time, it finds a cookie with probability $p_n$ (thus gaining a drift). Otherwise, with probability $1-p_n$, it finds no cookie (no drift) and behaves as a $d$-martingale with zero-mean vector. Otherwise, if the process has already visited the site, there is no cookie and the process acts again as a $d$-martingale with zero-mean vector.}

% \cm{The $p_n$-\Name{} can be reduced to the $\lambda_n$-\Name{}. Specifically, the $p_n$-\Name{} with a given $\lambda$ reduces to a $\lambda_n$-\Name{} with $p_n\lambda$ in Condition~\ref{condição2}. As a matter of fact, if we denote by $\widetilde{\FF}_n=\sigma(X_1, \ldots, X_n)$, by integrating 
% $\mathbb{E} [ X_{n+1} - X_n | \mathcal{F}_n] \cdot \ell$ with respect to $U_1, \ldots, U_n$ we obtain $\mathbb{E} [ X_{n+1} - X_n | \widetilde{\FF}_n] \cdot \ell\geq p_n\lambda$, which is Condition~\ref{condição2} in with a $\lambda_n = p_n\lambda$.}

%\comu{Se vamos fixar $\lambda_n = \lambda (n_0+n)^{-\beta}\,$, $n\ge 0$, faremos aqui.}

%%%%%%%%%%%%%%%%%%%%%%%%%%%%%%%%%%%%%%%%%
%%%%%%%%%%%%%%%%%%%%%%%%%%%%%%%%%%%%%%%%%
%%%%%%%%%%%%%%%%%%%%%%%%%%%%%%%%%%%%%%%%%
%\subsection{Main results for the $\lambda_n$-\Name}\label{main_pn}

%\cm{We provide our main result, which states that, there exists a $\beta > 0$ which the $\lambda_n$-\Name{} in direction $\ell$, in $\ZZ^d$ has a positive probability of never returning to the origin in the $\ell$ direction.} 
% . Now we set $\beta < \alpha < 1/6$ and $d \ge 2$, where $\alpha$ is establish in Proposition~\ref{prop41}. We will obtain that the $p_n$-\Name{} in direction $\ell$, in $\ZZ^d$ has a positive probability that never returns to the origin in the $\ell$ direction.}

% For every $\ell \in \mathbb{S}^{d-1}$, let $\mathbb{M}_{\ell}$ denote the positive half-space in direction $\ell$, that is, $\mathbb{M}_{\ell} = \{ x \in \ZZ^d : x \cdot \ell > 0 \}$.
% Our main result is stated below:

% \begin{theorem}\label{prop43_pnn0}
% Let $X$ be a  $\lambda_n$-\Name{} in direction $\ell$ with excitation set $A \supset \mathbb{M}_{\ell}$.  There exists $\beta_0 < 1/6$ such that if for some $n_0 \in \mathbb{N}$, $\lambda >0$ and $\beta<\beta_0$, we have $\lambda_n \ge \lambda (n_0+n)^{-\beta}$ for every $n\ge 1$, then
% $$
% \PP \big( \lim_{n \rightarrow \infty} X_n \cdot \ell = \infty \big) > 0\,.
% $$
% \end{theorem}

% \begin{remark}\label{rem:sub-balistic}
% 1. If $\lambda_n$ is $O(n^{-\beta})$, then the  $\lambda_n$-\Name{} is not ballistic since the total mean drift accumulated by time $n$ is bounded by  $n^{1-\beta}$. We conjecture that $\lim_{n \rightarrow \infty} X_n \cdot \ell = \infty$ holds almost surely, and even that $\liminf_{n \rightarrow \infty} n^{\beta - 1} X_n \cdot \ell > 0$ almost surely. 2. The condition $\beta < 1/6$ in the statement of Theorem \ref{prop43_pnn0} follows from limitations in our proof. We also conjecture that the result holds for $\beta < 1/2$. For a discussion on the case $\beta \ge 1/2$ see \cite{AIV}.
% \end{remark}

%This text is organized as follows: The renewal structure and the proof of the main results for the $p$-\Name{} are presented in Section \ref{renewalstruc}. Our main contributions are given in Section \ref{resultados_pn}
%and  \ref{proofpropR_n},where we study the $p_n$-\Name{}, with $p_n=n^{-\beta}$. We prove several asymptotic results depending on the value of $\beta$ and on the dimension $d$ (see Section~\ref{prova-pn-WGERW},~\ref{prova-pn-ERW_d=2},~\ref{sec: d>4} and~\ref{sec:d>2_b<1/6}). In Section~\ref{sec:rangeERW} we provide the proof of Proposition~\ref{prop:RangeERW}, in which we analyze the asymptotic behavior of the range of the $p_n$-\Nametwo{} in $d \ge 2$ and $\beta = 1/2$. Finally,  Appendix~\ref{sec:appendixA}, \ref{sec:appendixB} contain some proofs which were omitted in the main text and Appendix~\ref{sec:appendixC} some auxiliary results.

%%%%%%%%%%%%%%%%%%%%%%%%%%%%%%%%%%%%%%%%%
%%%%%%%%%%%%%%%%%%%%%%%%%%%%%%%%%%%%%%%%%
%%%%%%%%%%%%%%%%%%%%%%%%%%%%%%%%%%%%%%%%%


%\subsection{The model}\label{sec:model}
% We now formally introduce the $p_n$-\Name{}. Recall that $d \ge 2$ is the fixed dimension and let $\{p_n\}_{n \ge 1}$ be a sequence of parameters with $p_n \in (0, 1]$ for all $n \ge 1$.
% In a broader sense our process is a random element $(X,\pi)$ of $(\mathbb{Z}^d)^{\mathbb{Z}_+} \times [0, 1]^{\ZZ^d}$ endowed with the product Borel $\sigma$-algebra. The second coordinate $\pi = \{\pi(x)\}_{x \in \ZZ^d} \in [0, 1]^{\ZZ^d}$ is a random element whose marginals have uniform distribution in $[0,1]$ \and independents. We denote by $Q$ the probability law of $\pi$.  
% The first coordinate $X = \{ X_n \}_{n \geq 0}$ is a $\ZZ^d$ valued process with $X_0=0$ which is adapted to a filtration  $\mathcal{F} = \{ \mathcal{F}_n \}_{n \geq 0}$, where $\FF_n = \sigma(X_1,\dots, X_n, \pi(X_1), \dots,$ $\pi(X_n))$ and $\sigma(Y)$ represents the smallest $\sigma$-algebra generated by a random vector $Y$. We denote the law of $(X,\pi)$ by $\PP$ and by $\mathbb{E}$ its expectation, we can think of $\PP$ as the semi-direct measure $Q \otimes P_{\hat \pi}$, where $P_{\hat \pi}$ is the quenched measure for $X$, i.e., the conditional probability law of $X$ given a realization $\hat \pi$ of $\pi$. Now fix $\ell \in \mathbb{S}^{d-1}$, where $\mathbb{S}^{d-1}$ is the unit sphere of $\mathbb{R}^d$, and let $||\cdot||$ be the euclidean norm in $\mathbb{R}^d$. The process $X$ is called a $p_n$-\Name{} in direction $\ell$, if it satisfies the following conditions:

% \begin{condition}[Bounded increments]\label{condição1}
% There exists a constant $K > 0$ such that $ \sup_{n \geq 0} || X_{n+1} - X_{n} || < K$ on every realization. 
% \end{condition}

% \begin{condition}\label{condição2}
%  There exists $\lambda > 0$ such that: 

% \begin{itemize}
%     \item almost surely on the event $\{ X_k \neq X_n \text{ for all } \; k < n \}$, either
% $$
%     \mathbb{E} [ X_{n+1} - X_n | \mathcal{F}_n] \cdot \ell  \geq \lambda\;, \  \text{if $\pi(X_n) \leq p_n$}\, ,
% $$
% or
% $$
%     \mathbb{E} [ X_{n+1} - X_n | \mathcal{F}_n] =0 \;, \ \text{if $\pi(X_n) >p_n$}\, .
% $$
%     \item almost surely on the event $\{ \exists\,  k < n  \text{ such that }  X_k = X_n \}$,
%     \[
%      \mathbb{E} [ X_{n+1} - X_n | \mathcal{F}_n] = 0\, .
%     \]
% \end{itemize}
% \end{condition}


% \begin{condition}\label{condição3}
%  There exist $h, r > 0$ such that

% \begin{itemize}
%     \item {\rm Uniformly elliptic in direction $\ell$:}  for all $n$
% \begin{equation} \label{3 1.4}
% \tag{UE1}\PP \left[ \left( X_{n+1} - X_n \right) \cdot \ell > r | \mathcal{F}_n \right] \geq h\;, \; {a.s..}
% \end{equation}
% \item {\rm Uniformly elliptic on the event $\{\mathbb{E} [ X_{n+1} - X_n | \mathcal{F}_n] = 0\}$:} on the event $ \{ \mathbb{E} [ X_{n+1} - X_n | \mathcal{F}_n] = 0 \}$,  for all $\ell' \in \mathbb{S}^{d-1}$,  with $|| \ell '|| = 1$
% \begin{equation} \label{3 1.5}
% \tag{UE2}\PP \left[ \left( X_{n+1} - X_n \right) \cdot \ell ' > r | \mathcal{F}_n \right] \geq h\;, \; {a.s..}
% \end{equation}
% \end{itemize}
% \end{condition}

% We now present a helpful way to write the $p_n$-\Name{}, especially in the proofs. Let $\{X_n\}_{n \ge 0}$ be a $p_n$-\Name{} in direction $\ell$. Increasing the probability space the following representation for the process holds: Set $\{U_i\}_{i \geq 1}$ as a sequence of i.i.d. random variables with uniform distribution in $[0,1]$~\footnote{Note that this sequence is part of the marginals of  $\pi$. To avoid clutter, we set this i.i.d. uniform distribution as $\{U_i\}_{i \geq 1}$ from now on.}. For $i\ge 1$ we define $E_i$ as the event that the $p_n$-\Name{} is, at time $i$,  in an already visited site, i.e.,  $E_i := \{ \exists\;  k < i \; \text{ such that }\;  X_k = X_i \}$ and $E_0 := \emptyset$. 
% %
% We can write $\{X_n\}_{n \geq 0}$ as

% \begin{align}\label{xn-incremnto1}
% \begin{split}
% X_n & = \sum_{i=1}^n  (X_i - X_{i-1})
% \\
% & = \sum_{i=1}^n \big(1_{\{E_{i-1}\}} \xi_i + 1_{\{E_{i-1}^c\}} 1_{\{U_{i} > p_{i} \}} \xi_i + 1_{\{E_{i-1}^c\}}1_{\{ U_{i} \leq p_{i}\}} \gamma_i \big) \,, 
% \end{split}
% \end{align}
% where $\{\xi_i, \FF_i\}_{i \geq 1}$ is an increment of a $d$-martingale with zero mean and $\{\gamma_i, \FF_i\}_{i\geq 1}$ is a random vector such that $\EE[\gamma_i \cdot \ell | \FF_{i-1}] \ge \lambda$ for all $i \ge 1$.  Note that due to Condition~\ref{condição1} (bounded jumps), we also have that $|| \xi_i||<K$ as well as $||\gamma_i||<K$.
%When we assume the process is a submartingale in the direction $\ell$, we mean
%\begin{itemize}
   % \item $\EE|\gamma_n \cdot \ell| < \infty $ for all $n$. 
%    \item $\gamma_n \cdot \ell$ is adapted to $\FF_n$.
 %   \item $\EE[\gamma_{n+1} \cdot \ell| \FF_n] \geq \gamma_n$.
%\end{itemize}

















% \section{Proofs of the main theorems for the $p$-\Name{}}\label{renewalstruc}

% \input{renewalstructure}

%\section{Proof of main Theorems}\label{proofLGN}

%\input{proofmaintheorems}

% \subsection{The behavior of the increments of the regeneration times}\label{prooftkfinito}

% \input{proof_proptaukfinito}


\section{Proof of the main results for the $p_n$-\Name}\label{resultados_pn}

% We start this section remembering that if $\{X_n\}_{n \ge 0}$ is a $p_n$-\Name{} in direction $\ell$ we can write it as in~\eqref{xn-incremnto1}, hence we have
% $$
% X_n = \sum_{i=1}^n \big(1_{\{E_{i-1}\}} \xi_i + 1_{\{E_{i-1}^c\}} 1_{\{U_i > p_i \}} \xi_i + 1_{\{E_{i-1}^c\}}1_{\{ U_i \leq p_i\}} \gamma_i \big) \,,
% $$
% where $\{U_i\}_{i \geq 1}$ is a sequence  of i.i.d. random variables with uniform distribution in [0, 1], $E_i = \{ \exists\;  k < i \; \text{ such that }\;  X_k = X_i \}$ for all $i \ge 1$, $\{\xi_i, \FF_i\}_{i \geq 1}$ is an increment of a $d$-martingale with zero mean and $\{\gamma_i, \FF_i\}_{i\geq 1}$ is random vector such that $\EE[\gamma_i \cdot \ell |\FF_{i-1}] \ge \lambda$ .

Let $\{X_n\}_{n \ge 0}$ be a $p_n$-\Name{} then we can rewrite~\eqref{xn-incremnto1} as 
\begin{align}\label{xn-incremento2}
\begin{split}
X_n  & = \sum_{i=1}^n \big( \um_{ E_{i-1} \cup \{ U_i > p_i\}} \xi_i + \um_{E_{i-1}^c \cap \{ U_i \leq p_i\}} \gamma_i \big)
\\
& = \sum_{i=1}^n \big( \xi_i + \um_{E_{i-1}^c \cap \{ U_i \leq p_i\}} (\gamma_i - \xi_i) \big) \,.
\end{split}
\end{align}
%Before we provide the proofs of the main results for the $p_n$-\Name{}, let us point out that, 
Also for the sake of simplicity, we will henceforth work with the \[B_{\cdot}^n := \frac{X_{\lfloor n \cdot \rfloor}}{n^{1/2}}\;.\] instead of its interpolated version $\Hat{B}_{t}^n$ in \eqref{eq:B} which is continuous in $[0, \infty)$.  
More generally, 
in order to simplify writing and notation, if we have a sequence of c\`adl\`ag processes with values in $\mathbb{R}^m$, for $m \ge 2$, of the form $\Sigma^{n}_t = \Sigma_{\lfloor n t \rfloor}$, we denote by $\Hat{\Sigma}^{n}_t$ its linearly interpolated version, i.e., 
\begin{equation*}
\Hat{\Sigma}^{n}_t := \Sigma_{\lfloor n t \rfloor} + (nt - \lfloor nt \rfloor)(\Sigma_{\lfloor nt \rfloor + 1} - \Sigma_{\lfloor nt \rfloor}) \,, \ t\ge 0 \,,
\end{equation*}
which is a random element of $C_{\Rs^m}[0, \infty)$. Moreover by ~\cite[Proposition 10.4, Chapter 3]{ethier2009markov} if we have convergence in distribution in the Skorohod space of $\{\Sigma_{\cdot}^n\}_{n\ge 1}$ to a continuous process, then we also have convergence in distribution in $C_{\Rs^m}[0, \infty)$ of $\{\Hat{\Sigma}_{\cdot}^n\}_{n\geq 1}$ to the same limit. %Still about notation, we can also have $\Sigma^{n}_t = \Sigma_{\lfloor n t \rfloor}$ as a set value function and coherently we set
% \begin{equation*}
% \Hat{\Sigma}^{n}_t := |\Sigma_{\lfloor n t \rfloor}| + (nt - \lfloor nt \rfloor)(|\Sigma_{\lfloor nt \rfloor + 1}| - |\Sigma_{\lfloor nt \rfloor}|) \,, \ t\ge 0 \, .
% \end{equation*}




\begin{remark}\label{rem:conver}
 In the proofs, we will often make use of the following facts:
 \begin{itemize}
     \item[a)] If a sequence of processes converges in probability with respect to the uniform norm in $C_{\Rs^m}[0, T]$ for all $T >0$, then it converges in probability in $C_{\Rs^m}[0, \infty)$ under the metric $\rho$ defined in \eqref{def:rho} (this is a well-known result which can be easily proved).
     \item[b)] If a sequence of processes is tight in $C_{\Rs^m}[0, T]$ for all $T>0$ with the topology of uniform convergence in the compacts, then the sequence is tight in $C_{\Rs^m}[0, \infty)$ (see, e.g., \cite[Theorem 4.10,  Chapter 2]{karatzas2012brownian}). 
     \item[c)] \cite[Theorem 7.3]{billingsley1999probability}: A sequence of processes $\{\Hat{\Sigma}_{\cdot}^n\}_{n\geq 1}$ is tight in $C_{\Rs^m}[0, T]$  if and only if the following two conditions are satisfied:
     \begin{enumerate}[1.]
         \item For each positive $\eta$ there exist  $a$ and $n_0$ such that 
         \[ \PP\big[|\Hat{\Sigma}_{0}^n|\geq a\big]\leq \eta\,, \quad \text{ for all $n\geq n_0$}\,.
         \]
         \item For each positive $\varepsilon$ and $\eta$ there exist $\delta\in (0,1)$ and $n_0$ such that
         \[
         \PP\Big[\sup_{|s-t|\leq\delta}|\Hat{\Sigma}_{s}^n-\Hat{\Sigma}_{t}^n|\geq \varepsilon\Big]\leq \eta\,,  \quad \text{ for all $n\geq n_0$}\,.
         \]
     \end{enumerate}
 \end{itemize}
\end{remark}

%However for the computations we will do in the next sections, its clear that by Markov's inequality the second sum portion of $B_{t}^n$ converges to zero in probability. Hence, for our purposes, we only need to analyze the asymptotic behave of the first sum portion of $B_{t}^n$. 

\subsection{Case $\beta>1/2$; proof of Proposition~\ref{pn-WGERW-Gauss}}\label{prova-pn-WGERW}

\hfill \\
%\subsection{Proof of the convergence in distribution of the $p_n$-\Name*  with $\beta>1/2$}\label{prova-pn-WGERW}


%The $p_n$-\Name* can be written as in~\eqref{xn-incremnto1}. 
Recall that for the $p_n$-\Name*, the variables $\{U_i\}_{i \geq 1}$ are independent of  both sequences $\{\gamma_i\}_{ i\geq 1}$ and $\{\xi_i\}_{i \geq 1}$. 

%\cm{In this section we will consider a weaker condition. The sequence $\{U_i\}_{i \geq 1}$ will be uncorrelated, rather than independent, with both sequences $\{\gamma_i\}_{ i\geq 1}$ and $\{\xi_i\}_{i \geq 1}$.}  %\cm{Moreover we remember that the $p_n$-W\Name{} will be the process which satisfies the Conditions~\ref{condiçao I*},~\ref{condição2} and~\ref{condição3}.}

%Besides that we will also impose in this section a weaker condition that will replace Condition~\ref{condição1} and define the $p_n$-W\Name. Before, let us
%denote  $C=((c_{i,j}))$ as a continuous, $d \times d$ matrix-valued function, defined in $[0, \infty)$, satisfying $C(0) = 0$ and 
%\begin{equation*}
%\sum_{i,j = 1}^d (c_{i,j}(t) - c_{i,j}(s))\alpha_i \alpha_j \geq 0 \quad \text{for any } \alpha \in \mathbb{R}^d, \quad t > s \geq 0\;.  
%\end{equation*}
%By Theorem 7.1.1 from~\cite{ethier2009markov} there exists an unique process $Z$, in distribution, with sample paths in $C_{\mathbb{R}^d}[0, \infty)$ such that $Z_i$ and $Z_i Z_j - c_{i,j}$, for $i, j \in \{1,2, \dots, d\}$, are local martingales with respect to $\sigma$-algebra generated by historical of the process $Z$. Furthermore the process $Z$ has independent Gaussian increments.

%Now we can enunciate the new condition

%%%%%%%%%%%%
%\setcounter{condition}{0}
%\renewcommand{\thecondition}{\Roman{condition}*}


%%%%%%%%%%%%%%

%\begin{condition}\label{condiçao I*}
%\begin{itemize}
 %   \item [i)] For all $k \geq 1$ and $\theta < \beta - 1/2$, we have
  %  \[\sup_{k \geq 1} \frac{\EE[\| \gamma_k \|]}{k^{\theta}} < \infty \;. \]
   % \item [ii)] When the process behaves like a $d$-martingale with zero mean, it holds the following,
  %  \begin{equation}\label{condGaussiano}
   % \frac{1}{n}\sum_{i=1}^{\lfloor nt \rfloor} \xi_i \xi_i^T \to C(t) \quad \text{as } n \to \infty \;,
  %  \end{equation}
  %  in probability and
  %  \begin{equation*}
   % \lim_{k \to \infty} k^{-1/2} \EE\left[ \sup_{1 \leq i \leq k} \| \xi_i \| \right]  = 0\;. 
  %  \end{equation*}
    
    %\item [ii)] For a large $n$ we have,
    %\[ \sum_{k = 1}^{\infty} \PP[\| X_k - X_{k-1} \| > k^{\frac{\theta}{2}}] < \infty \;, \]
    %where $\theta$ is a positive constant such that $\theta/2 < \beta - 1/2$.
%\end{itemize}
%\end{condition}

%Let $\{X_n\}_{n \ge 0}$ be written as~\eqref{xn-incremnto1} where the sequence $\{U_i\}_{i \geq 1}$ is uncorrelated with both sequences $\{\gamma_i\}_{ i\geq 1}$ and $\{\xi_i\}_{i \geq 1}$. Additionally, if it satisfies Condition~\ref{condiçao I*}, \ref{condição2} and~\ref{condição3} \marginpar{\tiny\cm{acho que nem precisamos de III}} the process $X$ will be called a $p_n$-W\Name{} in direction $\ell$. 

% We rewrite a version of the Theorem 7.1.4 from~\cite{ethier2009markov} where the authors state a convergence in distribution of a $d$-martingale to a process with independents Gaussian increments. We will denote $\xrightarrow{\mathcal{D}}$ as convergence in distribution.

% \begin{theorem}[see~\cite{ethier2009markov}, Theorem 7.1.4]\label{kurtz}
% Let $(\phi_k, k \geq 1)$ be a sequence of $\mathbb{R}^d$-valued random vectors such that $\EE[\phi_k|\FF_{k-1}]=0$ where $\FF_k = \sigma(\phi_l, l \leq k)$. Define,
% \begin{align*}
% M_{\lfloor nt \rfloor} = \sum_{i=1}^{\lfloor nt \rfloor} \phi_i \quad \text{ and} \quad
% A_{\lfloor nt \rfloor} = \frac{1}{n} \sum_{i=1}^{\lfloor nt \rfloor} \phi_i \phi_i^T \;.
% \end{align*}
% Assume that the following conditions hold:
% \begin{itemize}
%     \item [i)]\begin{equation*}
%     \lim_{n \to \infty} n^{-1/2} \EE\left[ \sup_{1 \leq k \leq n} |M_{k-1} - M_k| \right]  = 0\;. \end{equation*}
%     \item[ii)] For each $t \geq 0$,
%     \begin{equation*}
%         A_{\lfloor nt \rfloor} \to C(t) \quad \text{as } n \to \infty\;,
%     \end{equation*}
%     in probability.
% \end{itemize}
% Then $M_{\lfloor n \cdot \rfloor}/n^{1/2} \xrightarrow{\mathcal{D}} Z_{\cdot}$ , where $Z$ is a process with independent Gaussian increments.
% \end{theorem}

We start stating a result that will be used in the proof of Proposition~\ref{pn-WGERW-Gauss}. %\cm{Due to this following Lemma it will be possible to see that the drift push through time will not make a difference for the process.}

\begin{lemma}\label{lem: tgD}
Let $X$ be a $p_n$-\Name* 
in direction $\ell\in \mathbb{S}^{d-1}$, on $\ZZ^d$ with $d\geq 2$, $p_n= \mathcal{C}n^{-\beta} \wedge 1$ with $\beta > 1/2$. Define  
% Considering the representation~\eqref{xn-incremento2} set 
\begin{equation*}
D_{\lfloor nt \rfloor} := \frac{1}{n^{1/2}} \sum_{i=1}^{\lfloor nt \rfloor} \um_{E_{i-1}^c \cap \{ U_i \leq  \mathcal{C} i^{-\beta}\}} (\gamma_i - \xi_i), \ t\ge 0 \,.
\end{equation*}
Then $\{\Hat{D}^n_\cdot\}_{n\ge 1}$ as a sequence of random elements of $C_{\Rs^d}[0, \infty)$,  converges in probability to the zero function.
\end{lemma}

The proof of Lemma~\ref{lem: tgD} will be postponed at the end of this section.
% \comu{Acho que este parágrafo pode ser removido. A prova é curta e bem claro o que será feito} The main idea behind the proof of Theorem~\ref{pn-WGERW-Gauss} is to use the decomposition~\eqref{xn-incremento2} and to analyze separately the sum portions suitably rescaled. We will see that the sum portion corresponding to the part of  $d$-martingale will converge in distribution and the other will converge to zero in probability. Thus, using Slutsky's Theorem we obtain the desired result. 

\begin{proof}[Proof of Proposition~\ref{pn-WGERW-Gauss}]
Using~\eqref{xn-incremento2} we write 
\begin{equation}\label{p_n-WGERW_incrementos}
    B_t^n = \frac{1}{n^{1/2}}\sum_{i=1}^{\lfloor nt \rfloor} \xi_i + \frac{1}{n^{1/2}} \sum_{i=1}^{\lfloor nt \rfloor} \um_{E_{i-1}^c \cap \{ U_i \leq {\mathcal{C}} i^{-\beta}\}} (\gamma_i - \xi_i) \,.
\end{equation}
%Now we will analyze separately the two portion sums of~\eqref{p_n-WGERW_incrementos}.
By Lemma~\ref{lem: tgD} the linear interpolation of the second term in~\eqref{p_n-WGERW_incrementos}  
%\begin{equation}\label{gamma_i-xi_i_wgerw}\frac{1}{n^{1/2}}\sum_{i=1}^{\lfloor nt \rfloor} 1_{\{E_{i-1}^c\}}1_{\{ U_i \leq i^{-\beta}\}} \gamma_i - 1_{\{E_{i-1}^c\}}1_{\{ U_i \leq i^{-\beta}\}} \xi_i  \to  0 \quad \text{as } n \to \infty\,, %\PP-\text{probability}\;.\end{equation}
converges in probability to the zero function (as a sequence of random elements of $C_{\Rs^d}[0, \infty)$). For the first term in~\eqref{p_n-WGERW_incrementos} we  use~\cite[Theorem 7.1.4, 7.1.1]{ethier2009markov} to obtain that
\begin{equation}\label{xi_i->Z}
    \Big\{ \frac{1}{n^{1/2}}\sum_{i=1}^{\lfloor nt \rfloor} \xi_i \Big\}_{t\ge 0} \xrightarrow[n \to \infty]{\mathcal{D}} \{ Z_{t} \}_{t\ge 0} \,,
\end{equation}
where $Z_{\cdot}$ denotes a Gaussian process with independent increments. Using Slutsky's Theorem (see~\cite[Theorem 11.4]{gut2005probability}) we finish the proof. 
\end{proof}

% \cm{Acho queue este remark poderia ficar depois da prova do lema.}
% \begin{remark}\label{rem_cond_frac}
% As we already point out in Remark~\ref{rem_cond_antes}  the Condition~\ref{condiçao I*} and the sequence $\{p_n\}_{n \ge 1}$ can be more general. Specifically, if we had $\sum_{i=1}^{\lfloor nt \rfloor} p_i \EE[||\gamma_i||] = o(\sqrt{n})$, then it will be possible to prove that the second sum portion in~\eqref{p_n-WGERW_incrementos} goes to zero in probability. Hence we finish the proof with Slutsky's Theorem (Theorem 11.4 from~\cite{gut2005probability}).
% \end{remark}

The proof of Corollary~\ref{pnESRW->BM}  (stating that the rescaled $p_n$-\Nametwo{} converges in distribution to a standard Brownian Motion) follows the proof of Proposition~\ref{pn-WGERW-Gauss} line by line.  The main difference is  in~\eqref{xi_i->Z} where instead of using ~\cite[Theorem 7.1.4]{ethier2009markov}, we can apply Donsker's Theorem, since the $\{\xi_i\}_{i\geq 1}$ corresponding to  $p_n$-\Nametwo{} are i.i.d. with zero-mean vector and finite covariance matrix (see, e.g., \cite[Theorem 8.2]{billingsley1999probability} or~\cite[Theorem 5.1.2]{ethier2009markov}).


\medskip
\begin{proof}[Proof of Lemma~\ref{lem: tgD}.] Without loss of generality, we shall assume $\CC = 1$. In light of Remark~\ref{rem:conver}-$a)$ it suffices to show convergence in $C_{\Rs^d}[0, T]$ for all $T>0$.   Let us then define 
\begin{equation*}
D_{\lfloor n t \rfloor}^{\gamma}:=\frac{1}{n^{1/2}}\sum_{i=1}^{\lfloor n t \rfloor} \um_{E_{i-1}^c}\um_{\{ U_i \leq i^{-\beta}\}} \gamma_i \,, \ t \ge 0,
%\ \text{and} \
% D_{\lfloor n\cdot \rfloor}^{\xi} := \frac{1}{n^{1/2}}\sum_{i=1}^{\lfloor nt \rfloor} 1_{\{E_{i-1}^c\}}1_{\{ U_i \leq i^{-\beta}\}} \xi_i \,. 
\end{equation*}
and analogously $D_{\lfloor n\cdot \rfloor}^{\xi}$ replacing $\gamma_i$ by $\xi_i$ in each term of the sum.
We begin showing  that $\{\Hat{D}_\cdot ^{n,\gamma}\}_{n\ge 1}$ converges in probability to the  zero function in $C_{\Rs^d}[0, T]$ for all $T > 0$ (recall that $\Hat{D}_t ^{n,\gamma}$ is the linearly interpolated version of   $D_{\lfloor nt \rfloor}^{\gamma}$). Note that
\begin{eqnarray}\label{eq: Dgamma}
& & \PP \Big( \sup_{0 \le t \le T} \big\| \Hat{D}^{n,\gamma}_{t} \big\| > \varepsilon  \Big)  \le \PP \Big( \sup_{0 \le t \le T} \sum_{i=1}^{\lfloor nt \rfloor+1}  \big\| \um_{E_{i-1}^c \cap \{ U_i \leq i^{-\beta}\}} \gamma_i \big\| > \varepsilon n^{\frac{1}{2}} \Big) \nonumber
\\
& & \le \PP \Big( \sum_{i=1}^{\lfloor nT \rfloor +1} \big\| \um_{E_{i-1}^c \cap \{ U_i \leq i^{-\beta}\}} \gamma_i \big\| > \varepsilon n^{\frac{1}{2}} \Big) \le  \PP \Big( \sum_{i=1}^{\lfloor nT \rfloor +1} \big\| \um_{ \{ U_i \leq i^{-\beta}\}} \gamma_i \big\| > \varepsilon n^{\frac{1}{2}} \Big) \nonumber
\\
& & \le \frac{1}{n^{1/2}\varepsilon} \sum_{i=1}^{\lfloor nT \rfloor +1} \frac{1}{i^{\beta}} \EE\left[\left\| \gamma_i \right\| \right]  \leq \frac{1}{n^{1/2}\varepsilon}  \sum_{i=1}^{\lfloor nT \rfloor +1} \frac{\EE\left[\left\| \gamma_i \right\| \right]}{i^{\theta}} \times   \frac{1}{i^{\beta-\theta}}  \,.    
\end{eqnarray}  
By Condition~\ref{condiçaoI*}, we know that  for all $i \geq 1$ and all $\theta < \beta - 1/2$ there exists a positive constant $L$ such that 
$\EE\left[\left\| \gamma_i \right\| \right] \le i^{\theta}   L$. 
Going back to~\eqref{eq: Dgamma}, we get
\begin{align*}%\label{eq: Dgamma3}
\begin{split}
\PP & \Big( \sup_{0 \le t \le T} \big\| \Hat{D}^{n,\gamma}_{t} \big\| > \varepsilon  \Big) %\le \frac{1}{n^{1/2}\varepsilon}  \sum_{i=1}^{\lfloor nT \rfloor} \frac{\EE\left[\left\| \gamma_i \right\| \right]}{i^{\theta}} \times   \frac{1}{i^{\beta-\theta}}  \\
\leq \frac{L}{n^{1/2}\varepsilon} \sum_{i=1}^{\lfloor nT \rfloor + 1} \frac{1}{i^{\beta-\theta}} 
\\ 
& \leq \frac{L}{n^{1/2}\varepsilon} \times c' (\lfloor nT \rfloor +1)^{1-\beta+\theta}
 \leq \frac{Lc'}{\varepsilon} \times \frac{\lfloor nT \rfloor + 1}{n} \times \frac{n^{1/2}}{(\lfloor nT \rfloor+1)^{\beta-\theta}}
\end{split}
\end{align*}
for some $c' > 0$, since $\sum_{i=1}^{k} \frac{1}{i^{\beta}} = O(k^{1-\beta})$. Given that $\beta >1/2+\theta$, we obtain that $\{\Hat{D}^{n,\gamma}_{\cdot}\}_{n\ge 1}$ converges uniformly in probability to the  zero function for all $T >0$.
%\begin{equation*}
%\frac{\sum_{i=1}^{\lfloor nt \rfloor} 1_{\{E_{i-1}^c\}}1_{\{ U_i \leq i^{-\beta}\}} \gamma_i}{n^{1/2}} \to  0 \quad \text{as } n \to \infty\,, %\PP-\text{probability}\;.    
%\end{equation*}

Using again Condition~\ref{condiçaoI*} and applying the same 
computations as above, we show that $\{\Hat{D}^{n,\xi}_{\cdot}\}_{n\ge 1}$ also converges uniformly in probability to the  zero function for all $T >0$.
%\begin{equation*}\frac{\sum_{i=1}^{\lfloor nt \rfloor} 1_{\{E_{i-1}^c\}}1_{\{ U_i \leq i^{-\beta}\}} \xi_i}{n^{1/2}}  \to  0 \quad \text{as } n \to \infty\,, %\PP-\text{probability}\;.    \end{equation*}
%in probability in the space $C_{\Rs^d}[0, T]$ for all $T >0$.
Therefore the same convergence also holds for $\{\Hat{D}^{n}_{\cdot} = \Hat{D}^{n,\gamma}_{\cdot} - \Hat{D}^{n,\xi}_{\cdot}\}_{n\ge 1}$. 
%
% Since convergence in $C_{\Rs^d}[0, T]$ for all $T >0$ implies convergence in $C_{\Rs^d}[0, \infty)$ under the metric $\rho$, then $D_{\lfloor n\cdot \rfloor}$ converges in probability to zero as a sequence of random elements of $C_{\Rs^d}[0, \infty)$ 
% \com{...HERE!!}
% .
\end{proof}

%\subsection{Proof of the convergence in distribution of the $p_n$-\Nametwo{} with $\beta = 1/2$ and $d=2$.}\label{prova-pn-ERW_d=2}

\subsection{Case $\beta = 1/2$; proof of Theorem~\ref{pn-ERW-d=2} and Theorem~\ref{pn-ERW-d=>4}}\label{sec:main-theorem}

\hfill \\

We begin introducing some auxiliary results.
% required to analyze the asymptotic behavior of the $p_n$-\Nametwo{} on $\ZZ^d$, with $\beta = 1/2$ and $d\geq 2$.
%
Let $\{\xi_i\}_{i \ge 1}$ be i.i.d. $\ZZ^d$-valued random variables with zero-mean vector and finite variance. Let $\{Y_n\}_{n \ge 0}$ be the random walk on $\ZZ^d$ with increments $\{\xi_i\}_{i \ge 1}$ starting at $Y_0 = 0$, thus $Y_n = \sum_{i=1}^{n} \xi_i$, $n\ge 1$.  For $m \leq n$ define 
\[ \Rr_{[m,n]} ^Y := \{Y_m, Y_{m+1}, \ldots, Y_n\}\;,\]
and denote by $\Rr_{n}^{Y} = \Rr^{Y}_{[0,n]}$,  the range of the random walk $\{Y_n\}_{n \geq 0}$.

We now state two known results about the range of a random walk on $\ZZ^d$ with i.i.d. increments which are instrumental in our proofs. Henceforth, we denote by $\pi_d$ the probability that $\{Y_n\}_{n \geq 0}$ never returns to the origin (as in the statement of Proposition~\ref{prop:RangeERW} and Proposition~\ref{prop:RangeERW_lower}).
\begin{theorem}[{\cite[pages 38-40]{spitzer2001principles}}\&{\cite[Theorem 1]{hamana2001large}}]\label{teo: RnZ>}
Let $\{Y_n\}_{n \geq 0}$ be an aperiodic random walk with i.i.d increments on $\ZZ^d$ for $d\geq 2$. It holds that 
\begin{itemize}
    \item  {\rm Law of Large Numbers:}
\begin{align*}
\tag{LLN}\frac{|\Rr_n^Y|}{n} \xrightarrow[n \to \infty]{} \pi_d \ \text{ a.s.}\,.
\end{align*}
\item {\rm Large Deviation:}  for every  $\theta' > \pi_d$ and $n$ sufficiently large
\begin{align*}
\tag{LD}\PP[|\Rr_{n}^Y| \geq \theta' n] \leq e^{-c_{\theta'}n}\,,
\end{align*}
where $c_{\theta'}$ is a positive constant that depends on $\theta'$.
\end{itemize}
 Note that for $d=2$, we have that $\pi_d=0$, whereas for $d\geq 3 $, $\pi_d\in (0,1]$. 
\end{theorem}

\begin{remark}
The meaning of aperiodicity used above is the one given in \cite[Chapter II, section 7]{spitzer2001principles} and it is different from the usual one (see, e.g., \cite[page 7]{levin2017markov}).  
Aperiodicity in Theorem~\ref{teo: RnZ>} refers to the following property: 
a  process on $\ZZ^d$ is aperiodic if the group generated by the support of the process is all of $\ZZ^d$.  However, as explained in~\cite[page 188-beginning of Section 2]{hamana2001large}, this condition does not entail a loss of generality. %However, as explained in~\cite[page 188-beginning of Section 2]{hamana2001large}, this condition can be relaxed to \com{include??} periodic random walks. \comu{incluir comentário sobre definição de aperiódico}
\end{remark}

Let us denote by $K_n$ the set of times from $0$ to $n-1$ in which $X$ visits a site for the first time and becomes excited. Henceforth, without loss of generality, we  assume $\CC = 1$, i.e., $p_n=n^{-1/2}$. We can write $K_n$ as
\begin{equation*}
K_n = \big\{ i \in \{1, 2, \dots, n\} : \um_{E_{i-1}^c \cap \{ U_i \leq i^{-1/2}\}} =1 \big\}\,.  
\end{equation*}
If $\{\psi_i\}_{i \ge 1}$ denotes  the sequence of $\FF$-stopping times  corresponding to the  times the $p_n$-\Nametwo{} visits a new site, then setting $\varphi_i = \psi_i + 1$, we have that
\begin{equation}\label{eq: def Kn}
|K_n| = \sum_{i=1}^{n} \um_{E_{i-1}^c \cap \{ U_i \leq i^{-1/2}\}} = \sum_{j=1}^{|\Rr_{n-1}^X|} \um_{\{U_{\varphi_j} \leq \varphi_j^{-1/2} \}}\,. 
\end{equation}

%If we further define  the following random variable
%\begin{equation}\label{eq: def Jn}
 %   |J_n| := \sum_{i=1}^{|\Rr_n^X|} 1_{\{U_i \leq i^{-1/2} \}}\,,
%\end{equation}
%then we obtain
%\begin{equation}\label{Kn<Jn}
%|K_n| =  \sum_{j=1}^{|\Rr_n^X|} 1_{\{U_{\tau_j} \leq \tau_j^{-1/2} \}} \preceq \sum_{i=1}^{|\Rr_n^X|} 1_{\{U_i \leq i^{-1/2} \}} = |J_n| \,,   
%\end{equation}
%by the fact that the realizations of the sequence $\{U_i\}_{i \ge 1}$ is i.i.d. and the $\tau_j$'s are stopping times.

Below we present an important auxiliary result that will be useful in the proof of Theorem~\ref{pn-ERW-d=2} and Theorem~\ref{pn-ERW-d=>4}.

\begin{lemma}\label{Jntight}
Let $|K_n|$ be defined as in~\eqref{eq: def Kn} and consider the corresponding sequence of continuous  processes $\{\Hat{K}^n_{\cdot}/n^{1/2}\}_{n\geq 1}$. It holds that
\begin{enumerate}[i)]
    \item For $d\geq 2$,  $\{\Hat{K}^n_{\cdot }/n^{1/2}\}_{n\ge 1}$ is tight  in  $C_{\Rs}[0, \infty)$;
    \item For $d= 2$, $\{\Hat{K}^n_{\cdot}/n^{1/2}\}_{n\geq 1}$ converges in probability as random elements of $C_{\Rs}[0, \infty)$ to the zero function.
\end{enumerate}

\end{lemma}
%
The proof of Lemma~\ref{Jntight} will be postponed at the end of this section.


\subsubsection{Case $\beta = 1/2$ and $d=2$; proof of Theorem~\ref{pn-ERW-d=2}} \label{prova-pn-ERW_d=2}

\begin{proof}[Proof of Theorem~\ref{pn-ERW-d=2}] 
% The idea of the proof of Theorem~\ref{pn-ERW-d=2} is similar to the one used in Theorem~\ref{pn-WGERW-Gauss}, however to obtain a version of Lemma~\ref{lem: tgD} in the case $d=2$ and $\beta=1/2$  we will use Proposition~\ref{prop:RangeERW} (see, proof of Lemma~\ref{Jntight}). 

By~\eqref{xn-incremento2} and the definition of $K_n$, we can rewrite the the process $B_t^n$ as
\begin{align}\label{p_n-ERW_incrementos_d=2}
\begin{split}
B_t^n & 
%= \frac{1}{n^{1/2}}\sum_{i=1}^{\lfloor nt \rfloor} \xi_i + \frac{1}{n^{1/2}} \sum_{i=1}^{\lfloor nt \rfloor} \um_{E_{i-1}^c \cap \{ U_i \leq i^{-1/2}\}} (\gamma_i - \xi_i)\\ & 
= \frac{1}{n^{1/2}}\sum_{i=1}^{\lfloor nt \rfloor} \xi_i + \frac{1}{n^{1/2}} \sum_{i \in K_{\lfloor nt \rfloor}}  (\gamma_i - \xi_i) 
\\
& = 
\frac{1}{n^{1/2}}\sum_{i=1}^{\lfloor nt \rfloor} \xi_i + \frac{|K_{\lfloor nt \rfloor}|}{n^{1/2}} \sum_{i \in K_{\lfloor nt \rfloor}} \frac{ (\gamma_i - \xi_i)}{|K_{\lfloor nt \rfloor}|}\, , \  t\ge 0\,.
\end{split}
\end{align}

For the first term in \eqref{p_n-ERW_incrementos_d=2} apply Donsker's Theorem (see, e.g., \cite[Theorem 5.1.2 $(c)$]{ethier2009markov}) we obtain that %the first sum portion of~\eqref{p_n-ERW_incrementos_d=2} converges in distribution in $C_{\Rs^2}[0, \infty)$ to a Brownian Motion, i.e.,  
\begin{equation}\label{xi_i->W2}
    \Big\{ \frac{1}{n^{1/2}}\sum_{i=1}^{\lfloor nt \rfloor} \xi_i \Big\}_{t\ge 0} \xrightarrow[n \to \infty]{\mathcal{D}} \{ W_{t} \}_{t\ge 0} \,,
\end{equation}
where $W_{\cdot}$ is a Brownian Motion in dimension 2 with zero-mean vector and covariance matrix $\EE[\xi_1 \xi_1^T]$. 
%
%Let us remember that $|K_{\lfloor n \cdot \rfloor}|/n^{1/2} \preceq |J_{\lfloor n \cdot \rfloor}|/n^{1/2}$ (see~\eqref{Kn<Jn}). 
%By Lemma~\ref{Jntight} $\{\Hat{K}^n_{\cdot}/n^{1/2}\}_{n\geq 1}$ converges in probability as random elements of $C_{\Rs}[0, \infty)$ to the identically zero function.
%\begin{equation}\label{eq: Kntprob}
%\left\{\frac{|K_{\lfloor nt \rfloor}|}{n^{1/2}}\right\}_{t\geq 0} \xrightarrow[n \to \infty]{} 0 \,,    
%\end{equation}
%in probability \com{we could use the notation of convergence in probability} as random elements of 
%
%\cm{Before we analyze the second sum portion in~\eqref{p_n-ERW_incrementos_d=2} let us give a little script how the proof will continuing. First we will show that the random variable $|J_n|/\sqrt{n}$ converges in probability to 0. Then with this computation it will be possible to see that the finite dimensional of the process $|J_{\lfloor nt \rfloor}|/\sqrt{n}$ converges in probability to 0 for any $t \ge 0$.}
%\cm{Finally since the process $|J_{\lfloor nt \rfloor}|/\sqrt{n}$ is tight in the space $C[0, \infty)$ by Lemma~\ref{Jntight}, we will see that converges in distribution to zero. Hence with the stochastic domination in~\eqref{Kn<Jn} and~\eqref{xi_i->W2} we will be able to obtain the desired result.}
%Let us denote the following event, for any $\varepsilon > 0$, set $G = \{|J_{n}| > \varepsilon \sqrt{n}\}$.
%Now, from Markov's inequality, for a $\delta>0$, one can see that 
%\begin{align}\label{eq:Bn}
%\end{align}
%From~\eqref{eq:Bn} we obtain 
%\begin{align}\label{eq:Bn2}
%\end{align}
%The last equality in~\eqref{eq:Bn2}, we obtain from Proposition~\ref{Rn<dn} and the fact that $\sum_{i=1}^{\lceil \delta n\rceil} \frac{1}{i^{1/2}} = \Theta(\lceil \delta n\rceil^{1/2})$. Since we can choose $\delta$ as small as we want. One can conclude 
%\begin{equation}\label{eq:Bn3}
 %   \lim_{n \to \infty} \PP[|J_n| > \varepsilon \sqrt{n}] = 0 \;.
%\end{equation}
%\cm{Now if we set $\{|J_{\lfloor nt \rfloor}| > \varepsilon \sqrt{n}\}$ for any $t \ge 0$, we have, by same computations we did above to obtain~\eqref{eq:Bn3}}
%\cm{
%\begin{equation}\label{eq:Bnt3}
%    \lim_{n \to \infty} \PP[|J_{\lfloor nt \rfloor}| > \varepsilon \sqrt{n}] = 0 \;.
%\end{equation}}
%\cm{Then by~\eqref{eq:Bnt3} we can conclude that the the finite dimensional of the process $|J_{\lfloor nt \rfloor}|/\sqrt{n}$ converges in probability to zero.} %and the process $|J_{\lfloor n \cdot \rfloor}|/\sqrt{n}$ is tight for $t=0$ in $C[0, T]$ for all $T > 0$.}
%\cm{Now, following our little script, we will show that the process $|J_{\lfloor nt \rfloor}|/\sqrt{n}$ converges in probability to 0 in the space $C[0, T]$. Notes that for any $\delta'>0$ and $t >0$, we have
%\begin{equation*}
%\PP \left[\sup_{|s-t| \le \phi} \left| \frac{|J_{\lfloor ns \rfloor}| - |J_{\lfloor nt \rfloor}|}{n^{1/2}} \right| \ge \varepsilon  \right] = \PP \left[ ||J_{\lfloor n(t+ \phi) \rfloor}| - |J_{\lfloor nt \rfloor}|| \ge \varepsilon n^{1/2}  \right]\,,    
%\end{equation*}
%by the definition of $|J_{\lfloor nt \rfloor}|$.}
%\cm{Then by the same techniques to obtain~\eqref{eq:Bn3} we have
%\begin{equation}\label{eq: Bntight}
%\lim_{\delta' \to 0} \limsup_{n \to \infty} \frac{1}{\delta'} \PP \left[\sup_{t \le s \le t + \delta'} \left| \frac{|J_{\lfloor ns \rfloor}| - |J_{\lfloor nt \rfloor}|}{n^{1/2}} \right| \ge \varepsilon  \right] = 0 \,.    
%\end{equation}}
%\cm{Since we have that the the finite dimensional of the process $|J_{\lfloor n\cdot \rfloor}|/\sqrt{n}$ converges in probability for 0 and by Lemma~\ref{Jntight} this process is tight in $C[0, \infty)$, we obtain by Theorem 2.4.15 in~\cite{karatzas2012brownian} the following
%\begin{equation}\label{eq: Bntprob}
%\frac{|J_{\lfloor nt \rfloor}|}{n^{1/2}} \to 0 \quad \text{as } n \to \infty \,,    
%\end{equation}
%in distribution in $C[0, \infty)$ and consequently in probability.}
%Hence from the stochastic domination  and~\eqref{eq: Bntprob} we obtain 
%\begin{equation}\label{eq:prob01}
%    \frac{|K_{\lfloor nt \rfloor}|}{n^{1/2}} \to 0 \quad \text{as } n \to \infty \,,
%\end{equation}
%in probability in $C[0, \infty)$. 
%

Now we consider the second term in \eqref{p_n-ERW_incrementos_d=2}. Clearly with probability one either $\lim_{n \to \infty} |K_{\lfloor nt \rfloor}|< \infty$ or $\lim_{n \to \infty} |K_{\lfloor nt \rfloor}|=+ \infty$.
%
% We shall now consider two cases: $\lim_{n \to \infty} |K_{\lfloor nt \rfloor}|< \infty$ almost surely and $\lim_{n \to \infty} |K_{\lfloor nt \rfloor}|=+ \infty$ almost surely. \comu{São só esses os casos possíveis? Acho que temos que fixar a realização.}
%
If $\lim_{n \to \infty} |K_{\lfloor nt \rfloor}|< \infty$,  then 
% there exists a positive constant $L$ \com{this $L$ should be 1?} such that 
\begin{equation}\label{eq:prob02finito}
\lim_{n \to \infty} \sum_{i \in K_{\lfloor nt \rfloor}} \frac{ \Vert \gamma_i - \xi_i \Vert}{|K_{\lfloor nt \rfloor}|} \leq 2K\,,
\end{equation}
where $K$ is from Condition~\ref{condition_A}.
%
%Hence from~\eqref{eq: Kntprob} and~\eqref{eq:prob02finito} we have \begin{equation}\label{eq:prob03} \frac{|K_{\lfloor nt \rfloor}|}{n^{1/2}} \sum_{i \in K_{\lfloor nt \rfloor}} \frac{ (\gamma_i - \xi_i)}{|K_{\lfloor nt \rfloor}|} \to 0 \quad \text{as } n \to \infty \,,\end{equation}in the space $C_{\Rs^2}[0, \infty)$ in probability. Then from~\eqref{xi_i->W2},~\eqref{eq:prob03} and Slutsky's Theorem (see~\cite[Theorem 11.4]{gut2005probability}) we have our result.
%
%\cm{Now we will show that $|K_{\lfloor nt \rfloor}| \to \infty$, almost surely as $n \to \infty$. Let us set a $\delta' \in (0,1)$. Hence we have}
%\cm{\begin{align*}
%|K_n| & = |K_n|1_{\{|\Rr_n^X| \le \delta' n \}} + |K_n|1_{\{|\Rr_n^X| > \delta' n \}}
%\\
%& \ge |K_n|1_{\{|\Rr_n^X| > \delta' n \}} \ge \sum_{i= \delta' n +1}^n 1_{\{U_{j_i} \le i^{-1/2} \}} \,.
%\end{align*}}
%
%\cm{We set the random variable $Y_i = 1_{\{U_i \le (\delta'n +i)^{-1/2} \}}$, thus we have $\sum_{i=1}^{n} Y_i = \sum_{i= \delta' n +1}^n 1_{\{U_i \ge i^{-1/2} \}}$. Let $\varepsilon' > 0$ and we obtain
%\begin{align*}
%\begin{split}
%\sum_{i=1}^n \PP[|Y_i| > \varepsilon'] & = \sum_{i=1}^n \PP[ Y_i \neq 0] =  \sum_{i=1}^n \frac{1}{(\delta' n +i)^{\frac{1}{2}}}
%\\
%& = \sum_{i= \delta' n +1}^n \frac{1}{i^{\frac{1}{2}}} \to \infty \quad \text{as } n \to \infty \,.
%\end{split}
%\end{align*}}
%\cm{Hence by the Second Lemma of Borel-Cantelli we have that $Y_i > \varepsilon'$ happens infinitely often and we can conclude $\sum_{i=1}^{n} Y_i \to \infty$ almost surely as $n \to \infty$.} 

If $\lim_{n \to \infty}|K_{\lfloor nt \rfloor}| = + \infty$, since  the sequence of random vectors $\{\gamma_{\varphi_i} -\xi_{\varphi_i}\}_{i \geq 1}$ is i.i.d. having  the same distribution as $\{\gamma_{i} -\xi_{i}\}_{i \geq 1}$, which is also i.i.d. (see Lemma~\ref{lem: iid}), we can use~\cite[Theorem 8.2 item (iii)]{gut2005probability} and obtain 
\begin{equation}\label{eq:prob02}
   \sum_{i \in K_{\lfloor nt \rfloor}} \frac{ (\gamma_i - \xi_i)}{|K_{\lfloor nt \rfloor}|} = \sum_{i=1}^{|K_{\lfloor nt \rfloor}|} \frac{ (\gamma_{\varphi_i} - \xi_{\varphi_i})}{|K_{\lfloor nt \rfloor}|} \xrightarrow[n \to \infty]{} \EE[\gamma_1 -\xi_1] \, \text{ a.s..}
\end{equation}

Thus, from Lemma~\ref{Jntight} (item $ii)$), \eqref{eq:prob02finito} and~\eqref{eq:prob02} we have
\begin{equation}\label{eq:prob03}
    \sup_{0\le t \le T} \frac{|K_{\lfloor nt \rfloor}|}{n^{1/2}} \sum_{i \in K_{\lfloor nt \rfloor}} \frac{ (\gamma_i - \xi_i)}{|K_{\lfloor nt \rfloor}|} \xrightarrow[n \to \infty]{} 0 \,,
\end{equation}
in probability.
%
Hence, from~\eqref{xi_i->W2},~\eqref{eq:prob03} and Slutsky's Theorem (see, e.g., \cite[Theorem 11.4]{gut2005probability}) we obtain our result. 
\end{proof}


% \subsubsection{Case $\beta = 1/2$ and $d=2$; proof of Theorem~\ref{pn-ERW-d=2}} \label{prova-pn-ERW_d=2}

% Let us denote by $K_n$ the set of times until time $n$ in which $X$ visits a site for the first time and becomes excited. Henceforth, without loss of generality, we  assume $\CC = 1$, i.e., $p_n=n^{-1/2}$. We can write $K_n$ as
% \begin{equation*}
% K_n = \{ i \in \{1, 2, \dots, n\} : \um_{E_{i-1}^c \cap \{ U_i \leq i^{-1/2}\}} =1 \}\,.  
% \end{equation*}
% Now we set the sequence of $\FF$-stopping times $\{\varphi_i\}_{i \ge 1}$ corresponding to the times the $p_n$-\Nametwo{} visits a new site. One can easily check that
% \begin{equation}\label{eq: def Kn}
% |K_n| = \sum_{i=1}^{n} \um_{E_{i-1}^c \cap \{ U_i \leq i^{-1/2}\}} = \sum_{j=1}^{|\Rr_n^X|} \um_{\{U_{\varphi_j} \leq \varphi_j^{-1/2} \}}\,. 
% \end{equation}
% %If we further define  the following random variable
% %\begin{equation}\label{eq: def Jn}
%  %   |J_n| := \sum_{i=1}^{|\Rr_n^X|} 1_{\{U_i \leq i^{-1/2} \}}\,,
% %\end{equation}
% %then we obtain
% %\begin{equation}\label{Kn<Jn}
% %|K_n| =  \sum_{j=1}^{|\Rr_n^X|} 1_{\{U_{\tau_j} \leq \tau_j^{-1/2} \}} \preceq \sum_{i=1}^{|\Rr_n^X|} 1_{\{U_i \leq i^{-1/2} \}} = |J_n| \,,   
% %\end{equation}
% %by the fact that the realizations of the sequence $\{U_i\}_{i \ge 1}$ is i.i.d. and the $\tau_j$'s are stopping times.

% Below we present an important auxiliary result that will be useful in the proof of Theorem~\ref{pn-ERW-d=2}.

% \begin{lemma}\label{Jntight}
% Let $|K_n|$ be defined as in~\eqref{eq: def Kn}.  We have that the sequence of processes $\{\Hat{K}^n_{\cdot}/n^{1/2}\}_{n\geq 1}$ converges in probability as random elements of $C_{\Rs}[0, \infty)$ to the identically zero function.
% \end{lemma}
% %
% The proof of Lemma~\ref{Jntight} will be postponed at the end of this section.







%Thereunto we will see that this process fulfills the condition on Theorem 7.3 in~\cite{billingsley1999probability}.  

%One can notice that the process $J_{\lfloor n \cdot \rfloor}/n^{1/2}$ is tight for $t=0$ for all $n \ge 1$. Thus the first condition on Theorem 7.3 in~\cite{billingsley1999probability} is satisfied.

%Then let $P_n$ be a probability measure on $C[0, T]$ and the distribution of $J_{\lfloor n\cdot \rfloor}/n^{1/2}$. We denote the set $A_t(\varepsilon, \phi):= \{f \in C[0,T] : \sup_{t \le s \le t+\phi} |f(s) - f(t)| \ge \varepsilon \}$.
%To prove that the process satisfies the second condition on Theorem 7.3 in~\cite{billingsley1999probability} we will use the Corollary on page 83 in~\cite{billingsley1999probability}.

%This Corollary states that Condition $(ii)$ of Theorem 7.3 in~\cite{billingsley1999probability} holds if, for each positive $\varepsilon$ and $\eta$, there exists a $\phi \in (0,1)$, and an integer $n_0$ such that
%\begin{equation}\label{eq: pnaJn}
%\frac{1}{\phi}  P_n[A_t(\varepsilon, \phi)] \le \eta \quad \forall n \ge n_0\,.    
%\end{equation}

%Hence we have
%\begin{equation*}
%\begin{split}
%\frac{1}{\phi}  P_n[A_t(\varepsilon, \phi)] & = \frac{1}{\phi} \PP\left[\sup_{t \le s \le t+\phi } | |J_{\lfloor ns \rfloor}| - |J_{\lfloor n t \rfloor}|| \ge \varepsilon n^{\frac{1}{2}}  \right]
%\\
%& \le \frac{1}{\phi} \PP\left[|J_{\lfloor n(t+\phi) \rfloor}|  \ge \varepsilon n^{\frac{1}{2}}\right] \,.
%\end{split}    
%\end{equation*}
%Now with the same techniques we use to achieve~\eqref{eq:Bn3} we obtain


%\subsection{Proof of the convergence in distribution of the $p_n$-\Nametwo{} with $\beta = 1/2$ and $d \ge 4$.}\label{sec: d>4}

\subsubsection{Case $\beta = 1/2$ and $d \ge 3$; proof of Theorem~\ref{pn-ERW-d=>4}}\label{sec: d>4}

\hfill \\


Recall that $\pi_d$ denotes the probability that the random walk with i.i.d. (with zero mean and finite variance) increments $\{\xi_i\}_{i\geq 0}$ on $\ZZ^d$ never returns to the origin. 
%
Given $\delta \in (0, 1)$, let us define the following random variables:
\begin{align}
\label{Bn'}
J_n(\delta) &:= \sum_{i=1}^{\delta n} \um_{\{U_i \leq i^{-1/2} \}}\,,
\\
\label{Fn'}
V_n(\delta')&:=\sum_{i=n-\delta' n+1}^{n} \um_{\{U_i \leq i^{-1/2} \}}\,.
\end{align}

The random variables $J_n$ and $V_n$ will be important to compute the constants $c_1$ and $c_2$ in the statement of Theorem~\ref{pn-ERW-d=>4}. Below we state a few  simple results about them; we defer the proofs to the end of this section. 


\begin{lemma}\label{B'_n} Fix $d\geq 3$ and
let $\{J_n(\delta)\}_{n \geq 1}$ be defined as in~\eqref{Bn'} with $\delta \in (\pi_d, 1)$ and  $\{V_n(\delta')\}_{n \geq 1}$ be defined as in~\eqref{Fn'} with $\delta' \in (0, \pi_{d})$. Then, it holds that
%
\begin{align*}
i) &\quad \lim_{n \to \infty} \frac{\EE[J_n(\delta)]}{n^{1/2}} = 2\delta^{1/2} \quad \text{ and }\quad  \lim_{n \to \infty} \frac{\EE[V_n(\delta')]}{n^{1/2}} = 2-  2(1 -\delta')^{1/2}\,,
\\
   ii) &\quad \text{For any $\varepsilon > 0$,} 
   \\
   &\qquad \lim_{n \to \infty} \PP[|J_n(\delta) - \EE[J_n(\delta)]| > \varepsilon n^{1/2}] = 0\,, 
   % \text{ and } \lim_{n \to \infty} \PP[|V_n(\delta') - \EE[V_n(\delta')]| > \varepsilon n^{1/2}] = 0 \,.  
   \\
   &\quad \text{  and the same holds for $V_n(\delta')$. }
    \end{align*}
   
    %  \item [(iii)]
    % \begin{equation*}
    % \lim_{n \to \infty} \frac{\EE[V_n]}{n^{1/2}} = 2-  2(1 -\delta')^{1/2} ; 
    % \end{equation*}
    
    % \item[(iv)] For any $\varepsilon > 0$, 
    % \begin{equation*}
    % \lim_{n \to \infty} \PP[|V_n - \EE[V_n]| > \varepsilon n^{1/2}] = 0 ; 
    % \end{equation*}
\end{lemma}
\medskip 
%
From Lemma~\ref{B'_n} we obtain the following corollary. 

% \begin{corollary}\label{B'n->p}
% Let $\{J_n\}_{n \geq 1}$ be defined in~\eqref{Bn'} with $\delta \in (\pi_d, 1)$ and  $\{F_n\}_{n \geq 1}$ be defined in~\eqref{Fn'} with $\delta' \in (0, \pi_{d-k})$. Then, it holds that: 
% \begin{itemize}
%     \item [i)] 
%     \begin{equation*}
%         \frac{J_n}{n^{1/2}} \xrightarrow[n \to \infty]{} 2\delta^{1/2} \ \text{ in probability;}
%     \end{equation*}
    
%     \item [ii)]
%     \begin{equation*}
%         \frac{F_n}{n^{1/2}} \xrightarrow[n \to \infty]{} 2(1- \delta')^{1/2} \  \text{ in probability;}
%     \end{equation*}
    
%     \item [iii)]
%     \begin{equation*}
%          \frac{\sum_{i=1}^{n} \um_{\{U_i \le i^{1/2}\}}}{n^{1/2}}  \xrightarrow[n \to \infty]{} 2 \ \text{ in probability.}
%     \end{equation*}
% \end{itemize}
% \end{corollary}

 
\begin{corollary}\label{B'n->p}
Fix $d\geq 3$ and let $\{J_n(\delta)\}_{n \geq 1}$ be defined in~\eqref{Bn'} with $\delta \in (\pi_d, 1)$ and  $\{V_n(\delta')\}_{n \geq 1}$ be defined in~\eqref{Fn'} with $\delta' \in (0, \pi_{d})$. Then, it holds that
    \begin{equation*}
        \frac{J_n(\delta)}{n^{1/2}} \xrightarrow[n \to \infty]{} 2\delta^{1/2} \quad \text{ and } \quad  \frac{V_n(\delta')}{n^{1/2}} \xrightarrow[n \to \infty]{} 2- 2(1- \delta')^{1/2} \quad   \text{ in probability}\,.
    \end{equation*}
    % \begin{equation*}
    %  \tag{ii}   \frac{V_n}{n^{1/2}} \xrightarrow[n \to \infty]{} 2- 2(1- \delta')^{1/2} \  \text{ in probability}\,.
    % \end{equation*}
\end{corollary}

\medskip 
%
Relying on  Corollary~\ref{B'n->p}, we are able to prove the following result: 

\begin{lemma}\label{lem: tightaux} Fix $d\geq 3$ and
let $\{\Hat{J}^{n}_{\cdot}(\delta)\}_{n\geq 1}$ and $\{\Hat{V}^n_{\cdot}(\delta')\}_{n\geq 1}$ be respectively the sequences of processes in $C_{\Rs}[0, \infty)$ corresponding to  
$J_n(\delta)$ with $\delta \in (\pi_d, 1)$ and  $V_n(\delta')$ with $\delta' \in (0, \pi_{d})$  defined in~\eqref{Bn'} and \eqref{Fn'}, respectively.   Then, it holds that:
%
\begin{itemize}
    \item [i)]
    $\{\Hat{J}^n_{\cdot}(\delta)/n^{1/2}\}_{n\geq 1}$ converges in distribution as random elements of $C_{\Rs}[0, \infty)$ to the deterministic function $t \mapsto 2(\delta t)^{1/2}$, $t\ge 0$.  
    \item[ii)] $\{\Hat{V}^n_{\cdot}(\delta')/n^{1/2}\}_{n\geq 1}$
%$$
%\left\{\frac{\sum_{i=1}^{\lfloor n \cdot \rfloor}  \um_{\{U_i \leq i^{-1/2} \}} - F_{\lfloor n \cdot \rfloor}}{n^{1/2}}\right\}_{t\geq 0} 
%$$
converges in distribution as random elements of $C_{\Rs}[0, \infty)$ to the deterministic function $t \mapsto 2t^{1/2}(1\!-\!(1\!-\!\delta')^{1/2})$, $t\ge 0$.  
\end{itemize}
\end{lemma}

% The proof of Lemma~\ref{lem: tightaux} will be postponed to the end of this section.

% The next result states that $(|K_{\lfloor n\cdot \rfloor}|/n^{1/2})_{n\ge 1}$ is tight in $C_{\Rs}[0, \infty)$. %In the proof of the Theorem~\ref{pn-ERW-d=>4}, it will be more clear the importance of this result.


% \begin{lemma}\label{lem: Knt/n1/2tig}
% The sequence of processes $\{\Hat{K}^n_{\cdot }/n^{1/2}\}_{n\ge 1}$ is tight  in  $C_{\Rs}[0, \infty)$.
% \end{lemma}

By Lemma~\ref{Jntight} (item $i)$), for $d\geq 3$, every subsequence of $\{\Hat{K}^n_{\cdot }/n^{1/2}\}_{n\ge 1}$ has a limit point. The next result states that those limits points are concentrated on paths confined between the curves  $t \mapsto 2 (1-\sqrt{1 -  \pi_{d}}) \sqrt{t}$ and $t \mapsto 2  \sqrt{ \pi_d \, t}$.
% The proof of Lemma~\ref{lem: Knt/n1/2tig} will be postponed at the end of this section.

% \begin{proposition}\label{prop: bound_Kn}
% %Let $|K_{\lfloor n \cdot \rfloor}|/n^{1/2}$ be a sequence of processes in $C_{\Rs}[0, \infty)$. 
% If $H_{\cdot}$ is a limit point of $\{\Hat{K}^n_{\cdot }/n^{1/2}\}_{n\ge 1}$, then
% \begin{equation*}
% \PP \left[\forall t \in [0,\infty): 2t^{1/2}(1-(1 -  \delta')^{1/2}) \le H_t \le 2(t \delta)^{1/2} \right] = 1 \, ,  
% \end{equation*}
% where $\delta'$ and $\delta$ are positive constants such that $\delta' \in (0, \pi_{d-k})$ and $\delta \in (\pi_d, 1)$. 
% \end{proposition}

\begin{proposition}\label{prop: bound_Kn}
%Let $|K_{\lfloor n \cdot \rfloor}|/n^{1/2}$ be a sequence of processes in $C_{\Rs}[0, \infty)$. 
If $\mathcal{H}_{\cdot}$ is a limit point of $\{\Hat{K}^n_{\cdot }/n^{1/2}\}_{n\ge 1}$, then
\begin{align}
\tag{a}&\text{ For $d\geq 3$, }\, \quad 
\PP \left[\forall t \in [0,\infty):  \mathcal{H}_t \le 2(t \pi_d)^{1/2} \right] = 1 \, ; 
\\
\tag{b}&\text{ For $d\geq 22$,} \quad 
\PP \left[\forall t \in [0,\infty): \mathcal{H}_t \ge 2t^{1/2}(1-(1 -  \pi_{d})^{1/2}) \right] = 1 \, . 
\end{align}
\end{proposition} 

\begin{remark}\label{rem:conjecture}
If Conjecture~\ref{conj_range} were true, we would be able to strengthen the claim of Proposition~\ref{prop: bound_Kn} and obtain that for any $d\geq 3$ it holds that
\begin{equation*}
\PP \left[\forall t \in [0,\infty): 2t^{1/2}(1-(1 -  \pi_{d})^{1/2}) \le \mathcal{H}_t \le 2(t \pi_d)^{1/2} \right] = 1 \, . 
\end{equation*}
\end{remark}


% The proof of Proposition~\ref{prop: bound_Kn} will be postponed at the end of this section.


We now have all the auxiliaries results to prove Theorem~\ref{pn-ERW-d=>4}. The main idea is to use~\eqref{xn-incremento2} and then analyze separately the rescaled sum terms. We show  that the sequences corresponding to both terms are tight in $C_{\Rs^d}[0, \infty)$, consequently we obtain that $\{\Hat{B}_{\cdot}^n\}_{n\geq 1}$ is also tight. Finally, we  describe the processes which stochastically dominate the limit points of $\{\Hat{B}_{\cdot}^n\}_{n\geq 1}$. The strategy is similar to that used in the proof of Theorem~\ref{pn-ERW-d=2}, but here the term representing the drift direction does not go to zero. The random variables $J_n$ and $V_n$ play an important role in controlling this non-vanishing term.  
%
\begin{proof}[Proof of Theorem~\ref{pn-ERW-d=>4}]
We begin showing that 
$\{{\Hat{B}}_{\cdot}^n\}_{n\ge 1}$ is tight in $C_{\Rs^d}[0, \infty)$ for any $d\geq 3$. By Remark~\ref{rem:conver}-$b)$ it suffices  to show that $\{{\Hat{B}}_{\cdot}^n\}_{n\ge 1}$ is tight in $C_{\Rs^d}[0, T)$, for all $T>0$.

%
Using ~\eqref{xn-incremento2} we can rewrite the process $B_t^n$ as
\begin{align}\label{p_n-ERW_incrementos_d=>4}
\begin{split} 
& \frac{1}{n^{1/2}}\sum_{i=1}^{\lfloor nt \rfloor} \xi_i + \frac{1}{n^{1/2}} \sum_{i=1}^{\lfloor nt \rfloor} \um_{\{E_{i-1}^c \cap \{ U_i \leq i^{-1/2}\}\}} (\gamma_i - \xi_i) \, , \ \forall \, t \ge 0\,.
% \\
% & = \frac{1}{n^{1/2}}\sum_{i=1}^{\lfloor nt \rfloor} \xi_i + \frac{1}{n^{1/2}} \sum_{i \in K_{\lfloor nt \rfloor}}  (\gamma_i - \xi_i) 
% \\
% & 
% = \frac{1}{n^{1/2}}\sum_{i=1}^{\lfloor nt \rfloor} \xi_i + \frac{|K_{\lfloor nt \rfloor}|}{n^{1/2}} \sum_{i \in K_{\lfloor nt \rfloor}} \frac{ (\gamma_i - \xi_i)}{|K_{\lfloor nt \rfloor}|}\,.
\end{split}
\end{align}
By Donsker's Theorem the first term of~\eqref{p_n-ERW_incrementos_d=>4} converges in distribution as random elements of $C_{\Rs^d}[0, \infty)$ to a Brownian Motion, i.e., 
\begin{equation}\label{xi_i->W2_d=>4}
 \Big\{ \frac{1}{n^{1/2}}\sum_{i=1}^{\lfloor nt \rfloor} \xi_i \Big\}_{t\ge 0} \xrightarrow[n \to \infty]{\mathcal{D}} \{ W_{t} \}_{t\ge 0} \,,
\end{equation}
where $W_{\cdot}$ is a Brownian Motion in dimension $d$ with zero-mean vector and covariance matrix $\EE[\xi_1 \xi_1^T]$.
Then,  to show that $\{{\Hat{B}}_{\cdot}^n\}_{n\ge 1}$ is tight in $C_{\Rs^d}[0, T]$ for all $T>0$, it is enough to prove that the second term in~\eqref{p_n-ERW_incrementos_d=>4} is tight in $C_{\Rs^d}[0, T]$. 
Indeed, $\{\Hat{B}_{\cdot}^n\}_{n\geq 1}$ 
 would be the sum of two tight sequences of processes, thus also tight. %(see Lemma~\ref{lem: tight}) \cm{colocar uma referencia ou deixar?}.
%
% Then by~\cite[Theorem 4.10 in Chapter 2]{karatzas2012brownian}, since $B_{\cdot}^n$ is  tight  in $C_{\Rs^d}[0, T]$ for all $T>0$ with the topology of uniform convergence in the compacts, $B_{\cdot}^n$ is  tight  in $C_{\Rs^d}[0, \infty)$.

In order to show that the second term in~\eqref{p_n-ERW_incrementos_d=>4} is tight in $C_{\Rs^d}[0, T]$ for all $T>0$,  we use Remark~\ref{rem:conver}-$c)$.
% \cite[Theorem 7.3]{billingsley1999probability} which provides two sufficient conditions for tightness.
Recall the definition of $D_{\lfloor n t \rfloor}$ from the statement of Lemma \ref{lem: tgD}, remembering that here we have set $\mathcal{C} = 1$ and that we are under distinct hypotheses from those of Section \ref{prova-pn-WGERW}. We will show that $\{\Hat{D}^n_\cdot\}_{n\ge 1}$ is tight.
%$$
%D_{\lfloor n t \rfloor}:= \frac{1}{n^{1/2}} \sum_{i=1}^{\lfloor nt \rfloor} \um_{\{E_{i-1}^c \cap \{ U_i \leq i^{-1/2}\}\}} (\gamma_i - \xi_i) \,.$$
The first condition in Remark~\ref{rem:conver}-$c)$ is satisfied, since  $\Hat{D}^n_0 \equiv 0$, for all $n\geq 1$. To prove that $\{\Hat{D}^n_\cdot\}_{n\ge 1}$ satisfies the second condition in Remark~\ref{rem:conver}-$c)$  we  use~\cite[Corollary on page 83]{billingsley1999probability} which states that the second condition of ~\cite[Theorem 7.3]{billingsley1999probability} holds if, for every positive $\varepsilon$ and $\eta$, there exists a $\phi \in (0,1)$, and an integer $n_0$ such that
\begin{equation}\label{eq: PnA}
\frac{1}{\phi} \, \PP \Big[ \sup_{t \le s \le t + \phi} \big\|\Hat{D}^n_{s} - \Hat{D}^n_{t} \big\|  \ge \varepsilon \Big]  \le \eta \quad \forall n \ge n_0\,.
\end{equation}
%\begin{equation}\label{eq: PnA}
%\frac{1}{\phi} P_n \Big[f \in C_{\Rs^d}[0,T] : \sup_{t \le s \le t+\phi} |f(s) - f(t)| \ge \varepsilon \Big] \le \eta \quad \forall n \ge n_0\,,
%\end{equation}
%where the probability measure $P_n$ on $C_{\Rs^d}[0,T]$  is the distribution of $D_{\lfloor n \cdot \rfloor}$. 
%In order to show that \eqref{eq: PnA} actually holds, note  that, if we define 
%\begin{equation}\label{eq:Atphi}
%A_t(\varepsilon, \phi):= \Big\{f \in C_{\Rs}[0,T] : \sup_{t \le s \le t+\phi} |f(s) - f(t)| \ge \varepsilon \Big\}\,,
%\end{equation} 
%the left-hand side of \eqref{eq: PnA} reduces to
% In order to show that actually holds,  let us define 
% $A_t(\varepsilon, \phi):= \{f \in C_{\Rs^d}[0,T] : \sup_{t \le s \le t+\phi} |f(s) - f(t)| \ge \varepsilon \}$. 
% Hence we obtain the following 
Note that the probability in \eqref{eq: PnA} is bounded from above by
\begin{equation}
\label{eq: PnPP}
%\PP\left[ D_{\lfloor n \cdot \rfloor} \in A_t(\varepsilon, \phi) \right] = 
%\PP \Big[ \sup_{t \le s \le t + \phi} \big\|\Hat{D}^n_{s} - \Hat{D}^n_{t} \big\|  \ge \varepsilon \Big] 
%& & = \PP \left[ \sup_{t \le s \le t+ \phi} \left\|\frac{\sum_{i=\lfloor nt \rfloor + 1}^{\lfloor ns \rfloor} 1_{\{E_{i-1}^c \cap \{ U_i \leq i^{-1/2}\}\}} (\gamma_i - \xi_i)}{n^{\frac{1}{2}}}  \right\| \ge \varepsilon \right] 
\PP\Big[ \sup_{t \le s \le t+ \phi} \Big\| \sum_{i=\lfloor nt \rfloor}^{\lfloor ns \rfloor + 1} \um_{E_{i-1}^c \cap \{ U_i \leq i^{-1/2}\}} (\gamma_i - \xi_i)  \Big\| \ge \varepsilon n^{\frac{1}{2}} \Big] \, ,
%\\
%& = \PP \left[ \sum_{i = \lfloor nt \rfloor + 1}^{\lfloor n(t + \phi) \rfloor} 1_{\{E_{i-1}^c \cap \{U_i \le i^{-\frac{1}{2}}\}\}} \ge \varepsilon n^{\frac{1}{2}} \right] \le \PP \left[ \sum_{i = \lfloor nt \rfloor + 1}^{\lfloor n(t + \phi) \rfloor} 1_{ \{U_i \le i^{-\frac{1}{2}}\}} \ge \varepsilon n^{\frac{1}{2}} \right]   
\end{equation}
%
% Now we only analyze the process inside the probability measure in~\eqref{eq: PnPP}. We will find an upper bound for this process for all the trajectory. 
for all $s \in [t, t+ \phi]$ and 
\begin{eqnarray}
\label{eq: trajetoria}
\Big\| \sum_{i=\lfloor nt \rfloor}^{\lfloor ns \rfloor + 1} \um_{E_{i-1}^c \cap \{ U_i \leq i^{-1/2}\}} (\gamma_i - \xi_i)  \Big\| &\le & \sum_{i=\lfloor nt \rfloor}^{\lfloor ns \rfloor + 1} \big\|  \um_{ \{ U_i \leq i^{-1/2}\}} (\gamma_i - \xi_i)  \big\| \nonumber
 \\
% \le  \sum_{i=\lfloor nt \rfloor + 1}^{\lfloor ns \rfloor} \big\|  \um_{ \{ U_i \leq i^{-1/2}\}} (\gamma_i - \xi_i)  \big\| 
& \le & \sum_{i=\lfloor nt \rfloor}^{\lfloor ns \rfloor + 1}  \um_{ \{ U_i \leq i^{-1/2}\}} 2K  \,, 
\end{eqnarray}
where the second inequality follows from  triangle inequality and the last from Condition~\ref{condition_A}. 
%
Then from~\eqref{eq: PnPP} and ~\eqref{eq: trajetoria} we obtain that
\begin{eqnarray}\label{eq: PnA<1}
\lefteqn{\!\!\!\!\!\!\!\! \PP \Big[ \sup_{t \le s \le t + \phi} \big\|\Hat{D}^n_{s} - \Hat{D}^n_{t} \big\|  \ge \varepsilon \Big] \le 
\PP \Big[ \sum_{i = \lfloor nt \rfloor}^{\lfloor n(t + \phi) \rfloor +1} \um_{ \{U_i \le i^{-\frac{1}{2}}\}} 2K \ge \varepsilon n^{\frac{1}{2}} \Big] } \nonumber
\\
& & \le \exp\Big(\frac{-\varepsilon n^{\frac{1}{2}}}{2K}\Big)\EE\Big[\exp\Big( \sum_{i = \lfloor nt \rfloor}^{\lfloor n(t + \phi)\rfloor+1} \um_{ \{U_i \le i^{-\frac{1}{2}}\}} \Big) \Big] 
%& & \le \exp\Big(\frac{-\varepsilon n^{\frac{1}{2}}}{2K}\Big) \prod_{i = \lfloor nt \rfloor + 1}^{\lfloor n(t + \phi)\rfloor} \EE\left[\exp\left(  \um_{ \{U_i \le i^{-\frac{1}{2}}\}} \right) \right] \,,
\end{eqnarray}
where in the last inequality we have used exponential Markov's inequality.
%
Setting  $c = \varepsilon/(2K)$ an  continuing the computation in~\eqref{eq: PnA<1} we obtain that
\begin{eqnarray}\label{eq: PnA<}
\lefteqn{\frac{1}{\phi} \PP \Big[ \sup_{t \le s \le t + \phi} \big\|\Hat{D}^n_{s} - \Hat{D}^n_{t} \big\|  \ge \varepsilon \Big] \le \frac{1}{\phi} e^{-c n^{\frac{1}{2}}} \prod_{i = \lfloor nt \rfloor}^{\lfloor n(t + \phi)\rfloor + 1} \EE\left[\exp\left(  \um_{ \{U_i \le i^{-\frac{1}{2}}\}} \right) \right]} \nonumber \\
& & = \frac{1}{\phi} e^{-c n^{\frac{1}{2}}} \prod_{i = \lfloor nt \rfloor}^{\lfloor n(t + \phi)\rfloor + 1} \left(1+ \frac{e-1}{i^{\frac{1}{2}}} \right) \le \frac{1}{\phi} e^{-c n^{\frac{1}{2}}} \prod_{i = \lfloor nt \rfloor}^{\lfloor n(t + \phi)\rfloor + 1} \exp\left(\frac{e-1}{i^{\frac{1}{2}}} \right)  
% \\
% & \le \frac{1}{\phi} e^{-c n^{\frac{1}{2}}}  \exp\left(\sum_{i = \lfloor nt \rfloor + 1}^{\lfloor n(t + \phi)\rfloor}\frac{e-1}{i^{\frac{1}{2}}} \right) 
\nonumber \\
& & \le \frac{1}{\phi} \exp(-c n^{\frac{1}{2}})\exp\left(2(e-1)(\sqrt{n(t + \phi)} - \sqrt{nt} + 2) \right) \,,  
\end{eqnarray}
where %the second inequality follows by the moment generating function of a Bernoulli  and the third by the fact that $1+x<e^x$ for all $x$.  
the last inequality above follows from noticing that 
\begin{align*}
\sum_{i = \lfloor nt \rfloor}^{\lfloor n(t + \phi)\rfloor + 1}\frac{1}{i^{\frac{1}{2}}} & \le \int_{nt - 1}^{n(t+\phi) + 1} x^{-1/2}dx \le 2\left(\sqrt{n(t + \phi)} - \sqrt{nt} + 2\right) \,.
\end{align*}
Therefore, in order to show that \eqref{eq: PnA}  holds, it remains to show that for every positive $\varepsilon$ (recall that $c=\varepsilon/(2K)$) and $\eta$,  there exists a $\phi \in (0,1)$, and an integer $n_0$ such that
\begin{equation}\label{exp<eta}
\frac{1}{\phi} \exp(-c n^{\frac{1}{2}})\exp\bigg(2(e-1)\left(\sqrt{n}\left(\sqrt{t + \phi} - \sqrt{t} \right) + 2 \right) \bigg) \le \eta \quad \forall n \ge n_0 \,.
\end{equation}
We accomplish this choosing $\phi \in (0,1)$ sufficiently small such that $\sqrt{t+\phi} - \sqrt{t}< c/4(e-1)$. Then, for every $\eta>0$, choosing $n$ sufficiently large, we obtain that~\eqref{exp<eta} is satisfied. 
% One can see that, since we have $\phi \in (0,1)$,  for all $\hat{\varepsilon} > 0$, there exists a $\phi' > \phi$ such that $|\sqrt{t+\phi'} - \sqrt{t}|< \hat{\varepsilon}$. Now we can choose $\hat{\varepsilon} = c/4(e-1)$ and we obtain, for a large enough $n$, that~\eqref{exp<eta} is fulfilled for all $\eta$.
Consequently, we have that~\eqref{eq: PnA} is satisfied and thus $\{\Hat{D}^n_{\cdot}\}_{n \ge 1}$ is tight in $C_{\Rs^d}[0,T]$. 
 
% Since the process $B_{\cdot}^n$  is the sum of two tight processes in $C_{\Rs^d}[0, T]$,  we obtain that $B_{\cdot}^n$ is a tight process in $C_{\Rs^d}[0, T]$ for all $T>0$ as a simple exercise. %(see Lemma~\ref{lem: tight}) \cm{colocar uma referencia ou deixar?}.

% Now by~\cite[Theorem 4.10 in Chapter 2]{karatzas2012brownian} one can see that since $B_{\cdot}^n$ is  tight  in $C_{\Rs^d}[0, T]$ for all $T>0$ with the topology of uniform convergence in the compacts then  $B_{\cdot}^n$ is  tight  in $C_{\Rs^d}[0, \infty)$.

We now prove the second part of the theorem, namely the stochastic domination in $a)$ and $b)$. Let us begin rewriting $B_t^n$ again in the slightly different form
\begin{equation}
 \label{p_n-ERW_incrementos_d=>4-bis}
% & = \frac{1}{n^{1/2}}\sum_{i=1}^{\lfloor nt \rfloor} \xi_i + \frac{1}{n^{1/2}} \sum_{i=1}^{\lfloor nt \rfloor} \um_{\{E_{i-1}^c \cap \{ U_i \leq i^{-1/2}\}\}} (\gamma_i - \xi_i) 
% % \\
% % & = \frac{1}{n^{1/2}}\sum_{i=1}^{\lfloor nt \rfloor} \xi_i + \frac{1}{n^{1/2}} \sum_{i \in K_{\lfloor nt \rfloor}}  (\gamma_i - \xi_i) 
% \\
% & 
\frac{1}{n^{1/2}}\sum_{i=1}^{\lfloor nt \rfloor} \xi_i + \frac{|K_{\lfloor nt \rfloor}|}{n^{1/2}} \sum_{i \in K_{\lfloor nt \rfloor}} \frac{ (\gamma_i - \xi_i)}{|K_{\lfloor nt \rfloor}|}\,.
\end{equation}
As already mentioned the first term converges to a Brownian Motion (see \eqref{xi_i->W2_d=>4}).  We now analyze the second term in~\eqref{p_n-ERW_incrementos_d=>4-bis}. 
%
For $d\geq 22$, by Proposition~\ref{prop: bound_Kn} part $(b)$,  we have  that $|K_{\lfloor nt \rfloor}| \to \infty$ as $n \to \infty$ almost surely. Moreover, the sequence of random vectors $\{\gamma_{\varphi_i} -\xi_{\varphi_i}\}_{i \geq 1}$ is i.i.d. and has the same distribution of $\{\gamma_{i} -\xi_{i}\}_{i \geq 1}$, which is i.i.d. too (see Lemma~\ref{lem: iid}). Thus, we can use~\cite[Theorem 8.2 item (iii)]{gut2005probability} to conclude that 
\begin{equation}\label{eq:prob02_d=>4}
   \sum_{i \in K_{\lfloor nt \rfloor}} \frac{ (\gamma_i - \xi_i)}{|K_{\lfloor nt \rfloor}|} \xrightarrow[n \to \infty]{} \EE[\gamma_1 -\xi_1]  = \EE[\gamma_1] \, \text{ a.s.}\,.
\end{equation}
Recall that $\{\gamma_n\}_{n \ge 1}$ is an i.i.d. sequence of random vectors and  $\lambda  \le \EE[\gamma_i \cdot \ell] \le K$ for all $i \ge 1$, we set $\mu_{\gamma} := \EE[\gamma_i \cdot \ell]$ for all $i \ge 1$. By Lemma~\ref{Jntight} and and~\eqref{eq:prob02_d=>4} we obtain that the linearly interpolated version of the sequence 
\begin{equation}\label{eq:seq}
\left\{ \frac{|K_{\lfloor n \cdot \rfloor}|}{n^{1/2}} \sum_{i \in K_{\lfloor nt \rfloor}} \frac{ (\gamma_i - \xi_i) \cdot \ell}{|K_{\lfloor n \cdot \rfloor}|} \right\}_{n\ge 1}\,, 
\end{equation}
is tight in $C_{\Rs}[0, \infty)$ and, from   Proposition~\ref{prop: bound_Kn} part $(b)$, for $d\geq 22$  any of its limit points $\mathcal{H}_t$ satisfies
\begin{equation}\label{eq:c1}
\PP\left[\forall t \in [0,\infty): 
2\mu_{\gamma} t^{\frac{1}{2}}(1 - (1 - \pi_{d})^{\frac{1}{2}})
\le \mathcal{H}_t \right] = 1\,.
\end{equation}
For $d\geq 3$, with probability one either $\lim_{n \to \infty} |K_{\lfloor nt \rfloor}|< \infty$ or $\lim_{n \to \infty} |K_{\lfloor nt \rfloor}|=+ \infty$.
%
On the event $\lim_{n \to \infty} |K_{\lfloor nt \rfloor}|< \infty$,  we have that 
$\sum_{i \in K_{\lfloor nt \rfloor}} \frac{ \Vert \gamma_i - \xi_i \Vert}{|K_{\lfloor nt \rfloor}|} \leq 2K$ for all $n$ ($K$ is from Condition~\ref{condition_A}) and $\lim_{n \to \infty}\frac{|K_{\lfloor n t \rfloor}|}{n^{1/2}}=0$. Thus, on this event, the linearly interpolated version of the sequence in \eqref{eq:seq} converges to the zero function. On the event $\lim_{n \to \infty} |K_{\lfloor nt \rfloor}|=+ \infty$ \eqref{eq:prob02_d=>4} holds and, similarly to the case $d\geq 22$, by Lemma~\ref{Jntight} together with  Proposition~\ref{prop: bound_Kn} part $(a)$, any limit point $\mathcal{H}_t$ of the linearly interpolated version of the sequence in \eqref{eq:seq} satisfies $\mathcal{H}_t \le 2\mu_{\gamma}(t \pi_d)^{1/2}$. Since $2\mu_{\gamma}(t \pi_d)^{1/2}\geq 0$,  overall we then obtain that 
\begin{equation}\label{eq:c2}
\PP\left[\forall t \in [0,\infty): 
\mathcal{H}_t \le 2\mu_{\gamma}(t \pi_d)^{1/2} \right] = 1\,.
\end{equation}

% \begin{align}\label{sup_
% inf_Kn}
% \PP \left[\forall t \in [0,\infty): 2t^{1/2}(1-(1 -  \pi_{d-k})^{1/2}) \le H_t \le 2(t \pi_d)^{1/2}  \right]  = 1 \,,
% \end{align}
% where $\{H_t\}_{t\ge 0}$ is a limit point of a subsequence of $\{ \Hat{K}^n_{\cdot}/n^{1/2}\}_{n\geq 1}$.
%\com{Above $\delta --> \delta' $ and  $\hat\delta --> \dekta $ right?}
%$\delta''$, $\Hat{\delta}$, $\delta'$ and $\delta$ are positive constants such that $\delta'' \in (0, \delta')$, $\Hat{\delta} \in (\delta, 1]$, $\delta \in (\pi_d, 1]$ and $\delta' \in (0, \pi_{d-k})$.
% \begin{equation}\label{Fn<Kn<Bn}
% \frac{1}{n^{1/2}} \sum_{i=1}^{\lfloor nt \rfloor}  1_{\{U_i \leq i^{-1/2} \}} - \frac{|F_{\lfloor nt \rfloor}|}{n^{1/2}} \preceq \frac{|K_{\lfloor nt \rfloor}|}{n^{1/2}} \preceq \frac{|J_{\lfloor nt \rfloor}|}{n^{1/2}}\,.
% \end{equation}
% Thus, by~\eqref{Bn<=Bn'} and~\eqref{Fn<Kn<Bn}, one can see that
% \begin{align}\label{eq:Kn<B'n}
% \begin{split}
% & \PP \left[ \forall t \in [0,\infty) : \frac{|K_{\lfloor nt \rfloor}|}{n^{1/2}} \le \frac{|J'_{\lfloor nt \rfloor}|}{n^{1/2}} \right] \ge \PP\left[ \forall t \in [0,\infty]: \frac{|J_{\lfloor nt \rfloor}|}{n^{1/2}} \le \frac{|J'_{\lfloor nt \rfloor}|}{n^{1/2}} \right] 
% \\
% & \to 1 \quad \text{as } n \to \infty\,.
% \end{split}
% \end{align}
% Now we will obtain, in the same sense of~\eqref{eq:Kn<B'n}, a lower bound. Hence by~\eqref{Fn<=Fn'} and~\eqref{Fn<Kn<Bn} we have
% \begin{equation}\label{eq: Kn>s-F'n}
% \begin{split}
% & \PP \left[\forall t \in [0,\infty) : \frac{\sum_{i=1}^{\lfloor nt \rfloor}  1_{\{U_i \leq i^{-1/2} \}}}{n^{1/2}} - \frac{|F'_{\lfloor nt \rfloor}|}{n^{1/2}} \le \frac{|K_{\lfloor nt \rfloor}|}{n^{1/2}}  \right] 
% \\
% & \ge \PP\left[\forall t \in [0,\infty): \frac{\sum_{i=1}^{\lfloor nt \rfloor}  1_{\{U_i \leq i^{-1/2} \}} - |F'_{\lfloor nt \rfloor}|}{n^{1/2}} \le \frac{\sum_{i=1}^{\lfloor nt \rfloor}  1_{\{U_i \leq i^{-1/2} \}} - |F_{\lfloor nt \rfloor}|}{n^{1/2}} \right] 
% \\
% & \to 1 \quad \text{as } n \to \infty \,.
% \end{split}
% \end{equation}
% Hence by Lemma~\ref{lem: tightaux}, Corollary~\ref{B'n->p}, ~\eqref{eq:Kn<B'n} and~\eqref{eq: Kn>s-F'n} we obtain that for all $t_0>0$
% \begin{align}\label{sup_
% inf_Kn}
% \PP \left[\forall t \in [t_0,\infty): 2t^{1/2}(1-(1 -  \delta')^{1/2}) \le \frac{|K_{\lfloor nt \rfloor}|}{n^{1/2}} \le 2(t \delta)^{1/2} \right] \to 1 \,,
% \end{align}
% as $n$ goes to infinity where $\delta$ and $\delta'$ are positive constants such that $\delta \in (\pi_d, 1]$ and $\delta' \in (0, \pi_{d-k})$.
%
\begin{comment}
Furthermore,  the sequence of random vectors $\{\gamma_{\varphi_i} -\xi_{\varphi_i}\}_{i \geq 1}$ is i.i.d. and has the same distribution of $\{\gamma_{i} -\xi_{i}\}_{i \geq 1}$, which is i.i.d. too (see Lemma~\ref{lem: iid}). Thus, we can use~\cite[Theorem 8.2 item (iii)]{gut2005probability} to conclude that 
\begin{equation}\label{eq:prob02_d=>4}
   \sum_{i \in K_{\lfloor nt \rfloor}} \frac{ (\gamma_i - \xi_i)}{|K_{\lfloor nt \rfloor}|} \xrightarrow[n \to \infty]{} \EE[\gamma_1 -\xi_1]  = \EE[\gamma_1] \, \text{ a.s..}
\end{equation}
Recall that $\{\gamma_n\}_{n \ge 1}$ is an i.i.d. sequence of random vectors and  $\lambda  \le \EE[\gamma_i \cdot \ell] \le K$ for all $i \ge 1$, we set $\mu_{\gamma} := \EE[\gamma_i \cdot \ell]$ for all $i \ge 1$. 
%\cm{Now we will prove that the second sum portion in~\eqref{p_n-ERW_incrementos_d=>4} is a tight process in $C[0,T]$. For that we will use again Theorem 7.3 in~\cite{billingsley1999probability}
%\begin{equation*}
%\begin{split}
%& \PP\left[ \sup_{t \le s \le t+ \phi} \left\| \sum_{i=\lfloor nt \rfloor + 1}^{\lfloor ns \rfloor} 1_{ \{ U_i \leq i^{-1/2}\}} (\gamma_i - \xi_i)  \right\| \ge \varepsilon n^{\frac{1}{2}} \right] \le
%\\
%& \PP\left[ \sup_{t \le s \le t+ \phi} \left( \sum_{i=\lfloor nt \rfloor + 1}^{\lfloor ns \rfloor} \left\|  1_{ \{ U_i \leq i^{-1/2}\}} (\gamma_i - \xi_i)  \right\| \right) \ge \varepsilon n^{\frac{1}{2}} \right] \le
%\\
%& \PP\left[ \sup_{t \le s \le t+ \phi} \left( \sum_{i=\lfloor nt \rfloor + 1}^{\lfloor ns \rfloor}   1_{ \{ U_i \leq i^{-1/2}\}} 2K \right) \ge \varepsilon n^{\frac{1}{2}} \right] \le
%\\
%& \PP\left[  \sum_{i=\lfloor nt \rfloor + 1}^{\lfloor n(t+\phi) \rfloor}   1_{ \{ U_i \leq i^{-1/2}\}} 2K  \ge \varepsilon n^{\frac{1}{2}} \right] \,.
%\end{split}    
%\end{equation*}}
%
From Proposition~\ref{prop: bound_Kn} \com{here the reference to Proposition seems off...} and~\eqref{eq:prob02_d=>4} we obtain that the linearly interpolated version of the sequence 
$$
\left\{ \frac{|K_{\lfloor n \cdot \rfloor}|}{n^{1/2}} \sum_{i \in K_{\lfloor nt \rfloor}} \frac{ (\gamma_i - \xi_i) \cdot \ell}{|K_{\lfloor n \cdot \rfloor}|} \right\}_{n\ge 1}\,, 
$$
is tight in $C_{\Rs}[0, \infty)$ and, from  {\color{red} Proposition~\ref{prop: bound_Kn} part $(a)$, for $d\geq 3$ } any of its limit points $\widetilde H_t$ satisfies
{\color{red} 
\begin{equation}\label{eq:c2}
\PP\left[\forall t \in [0,\infty): 
\widetilde H_t \le 2\mu_{\gamma}(t \pi_d)^{1/2} \right] = 1\,,
\end{equation}
whereas, from Proposition~\ref{prop: bound_Kn} part $(b)$, for $d\geq 22$  any of its limit points $\widetilde H_t$ satisfies 
\begin{equation}\label{eq:c1}
\PP\left[\forall t \in [0,\infty): 
2\mu_{\gamma} t^{\frac{1}{2}}(1 - (1 - \pi_{d})^{\frac{1}{2}})
\le \widetilde H_t \right] = 1\,.
\end{equation}

%\begin{equation}\label{eq: c1c2}
%\begin{split}
%& \PP\left[\forall t \in [0,\infty): \frac{|K_{\lfloor nt \rfloor}|}{n^{1/2}} \sum_{i \in K_{\lfloor nt \rfloor}} \frac{ (\gamma_i - \xi_i) \cdot \ell_{D_k}}{|K_{\lfloor nt \rfloor}|}  \le 2\mu_{\gamma}(t\delta)^{1/2} \right] \to 1 \quad \text{and}
%\\
%& \PP\left[\forall t \in [0,\infty): \frac{|K_{\lfloor nt \rfloor}|}{n^{1/2}} \sum_{i \in K_{\lfloor nt \rfloor}} \frac{ (\gamma_i - \xi_i) \cdot \ell_{D_k}}{|K_{\lfloor nt \rfloor}|} \ge 2t^{\frac{1}{2}}(1 - (1 - \delta')^{\frac{1}{2}})\mu_{\gamma}\right] \to 1 \,,
%\end{split}   
%\end{equation}
%as $n$ goes to infinity.
\end{comment}
Since $\{\Hat{B}_{\cdot}^n\}_{n\geq 1}$ is  tight in $C_{\Rs^d}[0, \infty)$, thus relatively compact by Prohorov's Theorem (see, e.g., ~\cite[Theorem 5.1]{billingsley1999probability}),  every subsequence has a limit point.
%
By~\eqref{xi_i->W2_d=>4}, \eqref{eq:c2} and \eqref{eq:c1},  for any of those limit points $\{\mathcal{Y}_{t}\}_{t\ge 0}$ we have that for all $t \in [0, \infty)$
\begin{align*}
 \{\mathcal{Y}_t \cdot \ell\}_{t\ge 0} &\preceq \{W_t \cdot \ell + 2 c_2 \sqrt{t} \}_{t\ge 0} \,, && \text{ for $d\geq 3$}\,,\\
\{W_t \cdot \ell + 2 c_1 \sqrt{t}\}_{t\ge 0} &\preceq \{\mathcal{Y}_t \cdot \ell\}_{t\ge 0}\,, && \text{ for $d\geq 22$}\,,  
\end{align*}
where $c_1 = (1 - \sqrt{1 - \pi_{d}})\mu_{\gamma}$ and $c_2 = \sqrt{\pi_d} \, \mu_{\gamma}$ (and $0 < c_1 \le c_2$).  
\end{proof}



\begin{proof}[Proof of Lemma~\ref{Jntight}] 
Item $ii)$: 
 By   Remark~\ref{rem:conver}-$a)$,  it is enough to prove that the sequence of processes $\{\Hat{K}^n_{\cdot}/n^{1/2}\}_{n\geq 1}$ converges in probability as random elements of $C_{\Rs}[0, T)$ to the  zero function, for all $T > 0$. For the latter it suffices to prove  
 convergence in probability of $\sup_{0 \le t \le T} |K_{\lfloor nt \rfloor}|/n^{1/2}$ to zero for all $T > 0$.
% , since convergence in $C_{\Rs^d}[0, T]$ for all $T >0$ implies convergence in $C_{\Rs^d}[0, \infty)$ under the metric $\rho$ 
% \com{...HERE!!}
% . 
To this aim, let us define  
$$G_n:=\Big\{\sup_{0 \le t \le T} |K_{\lfloor nt \rfloor}| > \varepsilon \sqrt{n}\Big\} = \left\{ |K_{\lfloor nT \rfloor}| > \varepsilon \sqrt{n} \right\}\, ,
$$
For every $\varepsilon > 0$ and $\delta>0$, %consider the following event $G = \{|K_{\lfloor nT \rfloor}| > \varepsilon \sqrt{n}\}$. 
by Markov's inequality, we have that 
\begin{align*}\label{eq: Jnprob}
\begin{split}
& \PP[ G_n ] = \PP\big[
G_n \cap \{|\Rr_{\lfloor nT \rfloor} ^X| > \delta \lfloor nT \rfloor\}\big] +  \PP\big[ G_n \cap \{|\Rr_{\lfloor nT \rfloor}^X| \leq \delta \lfloor nT \rfloor\}\big]
\\
& \leq \PP\big[|\Rr_{\lfloor nT \rfloor} ^X| > \delta \lfloor nT \rfloor\big] + \PP\Big[ \sum_{i=1}^{\lceil \delta n T\rceil} \um_{\{U_i \leq i^{-1/2} \}} > \varepsilon\sqrt{n} \Big] 
\\
& \leq \PP\big[|\Rr_{\lfloor nT \rfloor} ^X| > \delta \lfloor nT \rfloor\big] + \frac{1}{\varepsilon \sqrt{n}} \sum_{i=1}^{\lceil \delta nT\rceil} \frac{1}{i^{1/2}}\,. \end{split}
\end{align*}
Note that we have used $\Rr_{\lfloor nT \rfloor}^X$ instead of $\Rr_{\lfloor nT \rfloor -1}^X$, the reader can check that indeed this does not change the inequality and simplifies notation. This will happen in other inequalities in this section.
Using  Proposition~\ref{prop:RangeERW}, we have that, for all sufficiently large $n$,  $
    \PP[ |\Rr_n ^X| \leq \delta n ] = 1$, for every  $\delta > \pi_d$. Since for $d=2$, $\pi_d =0$,  we obtain that   $\lim_{n \to \infty}\PP\big[|\Rr_{\lfloor nT \rfloor} ^X| > \delta \lfloor nT \rfloor\big]=0$, for every $\delta>0$. Moreover, noticing that   $\sum_{i=1}^{\lceil \delta n\rceil} \frac{1}{i^{1/2}} = \Theta(\lceil \delta n\rceil^{1/2})$, we conclude that
\begin{equation}\label{eq: Jnprob2}
\begin{split}
& \limsup_{n \to \infty}  \PP[ G_n ]  
%\leq \limsup_{n \to \infty} \Big( \PP\big[|\Rr_{\lfloor nT \rfloor}^X| > \delta \lfloor nT \rfloor\big] + \frac{1}{\varepsilon \sqrt{n}} \sum_{i=1}^{\lceil \delta nT \rceil} \frac{1}{i^{1/2}} \Big) \\ & \leq \limsup_{n \to \infty} \PP\big[|\Rr_{\lfloor nT \rfloor}^X| > \delta \lfloor nT \rfloor\big] + \limsup_{n \to \infty} \Big( \frac{1}{\varepsilon \sqrt{n}} \sum_{i=1}^{\lceil \delta nT \rceil} \frac{1}{i^{1/2}} \Big) 
\leq \frac{c' (\delta T)^{1/2}}{\varepsilon}\,.
\end{split}
\end{equation}
Since $\delta>0$ is arbitrary,  $\limsup_{n \to \infty} \PP[ G_n ] = 0$ for every $\varepsilon$ fixed. Therefore, for all $T > 0$ the sequence of processes $\left\{\Hat{K}^n_{\cdot}/n^{1/2}\right\}_{n\geq 1}$ converges in probability, as random elements of $C_{\Rs}[0,T]$, to the  zero function.  %(see Lemma~\ref{lem: convprob}) \cm{pensar em uma referencia ou não ha necessidade?}. 

\medskip 
Item $i)$: 
% The proof follows the very same lines of that of Theorem~~\ref{pn-ERW-d=>4}, and we omit it. 
%
By  Remark~\ref{rem:conver}-$b)$  it suffices to show tightness in $C_{\Rs}[0, T]$ for all $T > 0$ and this is equivalent to show the sequence of processes $\left\{\Hat{K}^n_{\cdot}/n^{1/2}\right\}_{n\geq 1}$  satisfies  the two conditions in  Remark~\ref{rem:conver}-$c)$. 
Note that $\Hat{K}^n_{0}/n^{1/2}\equiv 0$ for all $n \ge 1$ and therefore the first condition in Remark~\ref{rem:conver}-$c)$ is satisfied.
%
To prove that $\{\Hat{K}^n_\cdot /n^{1/2}\}_{n\ge 1}$ satisfies the second condition in Remark~\ref{rem:conver}-$c)$  we  use~\cite[Corollary on page 83]{billingsley1999probability} which states that the second condition of ~\cite[Theorem 7.3]{billingsley1999probability} holds if, for every positive $\varepsilon$ and $\eta$, there exists a $\phi \in (0,1)$, and an integer $n_0$ such that
\begin{equation*}
\frac{1}{\phi} \, \PP \Big[ \sup_{t \le s \le t + \phi} \big\|\Hat{K}^n_{s} - \Hat{K}^n_{t} \big\|  \ge \varepsilon n^{1/2} \Big]  \le \eta \quad \forall n \ge n_0\,.
\end{equation*}
From this point on, using that 
\[
\PP \Big[ \sup_{t \le s \le t + \phi} \big\|\Hat{K}^n_{s} - \Hat{K}^n_{t} \big\|  \ge \varepsilon n^{1/2} \Big]  \le \PP \Big[ \sum_{i = \lfloor nt \rfloor }^{\lfloor n(t + \phi) \rfloor +1} \um_{\{U_i \le i^{-\frac{1}{2}}\}} \ge \varepsilon n^{\frac{1}{2}} \Big]\,,
\]
the computations are very similar to those used in the proof of Theorem~\ref{pn-ERW-d=>4}, and we omit it. 
\end{proof}





\begin{proof}[Proof of Lemma~\ref{B'_n}.] To avoid clutter in the notation, we write $J_n$ and $V_n$, thus omitting the dependence on $\delta$ and $\delta'$.  
As far as $J_n$ is concerned, we have 
\begin{equation*}
\frac{\EE[J_n]}{n^{1/2}} = \frac{1}{n^{1/2}}\EE\Big[\sum_{i=1}^{\delta n} \um_{\{U_i \leq i^{-1/2} \}}\Big] = \frac{1}{n^{1/2}}\sum_{i=1}^{\delta n} i^{-1/2} \xrightarrow[n \to \infty]{} 2\delta^{1/2}  \,. 
\end{equation*}
Also,  using Chebyshev's inequality and the independence of the random variables $\{U_i\}_{i \geq 1}$ we have that 
\begin{align*}
\PP \big[ |J_n - \EE[J_n]| > \varepsilon n^{1/2} \big] &\leq \frac{1}{\varepsilon^2 n} \text{Var}\Big[ \sum_{i=1}^{\delta n} \um_{\{U_i \leq i^{-1/2} \}} \Big]
\\
&=\frac{1}{\varepsilon^2 n} \sum_{i=1}^{\delta n} \frac{1}{i^{1/2}}\left(1-\frac{1}{i^{1/2}}\right) \xrightarrow[n \to \infty]{} 0  \,. 
\end{align*}

The proofs for $V_n$ are similar once we write
\[
V_n = \underbrace{\sum_{i=1}^n  \um_{\{U_i \leq i^{-1/2} \}}}_{=:I_n} - \underbrace{\sum_{i=1}^{n -\delta' n} \um_{\{U_i \leq i^{-1/2} \}}}_{=:F_n'}\,,
\]
and observe that  
\begin{align*}
&\frac{1}{n^{1/2}}\EE[I_n] = \frac{1}{n^{1/2}}\sum_{i=1}^{n} i^{-1/2} \xrightarrow[n \to \infty]{} 2\,,
\\
&\frac{1}{n^{1/2}}\EE[F'_n] = \frac{1}{n^{1/2}}\sum_{i=1}^{n - \delta' n} i^{-1/2} \xrightarrow[n \to \infty]{} 2(1-\delta')^{1/2} \,, 
\end{align*}
and that, by Chebyshev's inequality and the independence of the random variables $\{U_i\}_{i \geq 1}$, it holds that 
\begin{align*}
\PP & [|I_n - \EE[I_n]| > \varepsilon n^{1/2}] 
 \leq \frac{1}{\varepsilon^2 n} \text{Var}\Big[ \sum_{i=1}^{n} \um_{\{U_i \leq i^{-1/2} \}} \Big] 
% = 
% \frac{1}{\varepsilon^2 n} \sum_{i=1}^{n} i^{-1/2}(1-i^{-1/2})
\xrightarrow[n \to \infty]{} 0  
\,,
\\
\PP &\big[ |F'_n - \EE[F'_n]| > \varepsilon n^{1/2} \big] \leq \frac{1}{\varepsilon^2 n} \text{Var}\Big[ \sum_{i=1}^{n -\delta' n} \um_{\{U_i \leq i^{-1/2} \}} \Big]
% \\
% &
% =  \frac{1}{\varepsilon^2 n} \sum_{i=1}^{n - \delta' n} i^{-1/2}(1-i^{-1/2}) 
\xrightarrow[n \to \infty]{} 0 
\,.
\end{align*}
% The proof of point $v)$ is also straightforward: 
% \begin{equation*}
% \frac{\EE[\sum_{i=1}^n  \um_{\{U_i \leq i^{-1/2} \}}]}{n^{1/2}} = \frac{1}{n^{1/2}}\sum_{i=1}^{n} i^{-1/2} \xrightarrow[n \to \infty]{} 2 
% \,.  
% \end{equation*}
% For the proof of point $vi)$ we use Chebyshev's inequality and the independence of the random variables $\{U_i\}_{i \geq 1}$ and we obtain
% \begin{align*}
% % \label{eq: P}
% % \begin{split}
% \PP & \Big[\Big|\sum_{i=1}^n  \um_{\{U_i \leq i^{-1/2} \}} - \EE\Big[\sum_{i=1}^n  \um_{\{U_i \leq i^{-1/2} \}}\Big]\Big| > \varepsilon n^{1/2}\Big] 
%  \leq \frac{1}{\varepsilon^2 n} \text{Var}\Big[ \sum_{i=1}^{n} \um_{\{U_i \leq i^{-1/2} \}} \Big] 
%  \\
% &= \frac{1}{\varepsilon^2 n} \sum_{i=1}^{n} i^{-1/2}(1-i^{-1/2}) \to 0  \quad \text{as } n \to \infty 
% \,.  
% \end{align*}
\end{proof}

\begin{proof}[Proof of Lemma~\ref{lem: tightaux}.] To avoid clutter in the notation we omit the dependence on $\delta$ and $\delta'$. By Corollary~\ref{B'n->p}, for all $t> 0$ we have that 
\begin{align*}
\frac{J_{\lfloor n t\rfloor}(\delta)}{\sqrt{n}} & \xrightarrow[n \to \infty]{} 2\delta^{1/2} \sqrt{t}\,, && \forall \delta \in (\pi_d,1)\,;
\\
\frac{V_{\lfloor n t\rfloor}(\delta')}{\sqrt{n}} &\xrightarrow[n \to \infty]{} \left(2 - 2(1-\delta')^{1/2}\right)\sqrt{t}\,, && \forall  \delta' \in (0,\pi_d)\,,  
\end{align*}
in probability, i.e., the one-dimensional distributions  converge in probability to a constant. Therefore the joint distributions corresponding to times $t_1, \ldots, t_m$ also converge in probability and we obtain the convergence in the sense of the finite-dimensional distributions for both processes.
% By Corollary 3.1 we already have the convergence of the
% finite-dimensional distributions. 
% we have the convergence of the one-dimensional distributions in probability to a constant. Since each one-dimensional random variable converges in probability to a constant, the joint convergence in probability follows naturally, which in turn implies convergence in distribution.

By~\cite[Theorem 7.1]{billingsley1999probability},  to prove points $i)$ and $ii)$ it only remains to prove that both sequences of processes are tight in $C_{\Rs}[0, \infty)$. By Remark~\ref{rem:conver}-$b)$ we only need to prove tightness in $C_{\Rs}[0, T]$ for all $T>0$. The proof strategy is analogous to the one used in the proof of Theorem~\ref{pn-ERW-d=>4} which relies on Remark~\ref{rem:conver}-$c)$.  %
% when we show the second sum portion in~\eqref{p_n-ERW_incrementos_d=>4} is tight in $C_{\Rs}[0, T]$, for all $T>0$.

Item $i)$:  $\{\Hat{J}^n_{\cdot }/n^{1/2}\}_{n\geq 1}$ satisfies the first condition in Remark~\ref{rem:conver}-$c)$, since $\Hat{J}^n_0 = 0$ for all $n\geq 1$.  To prove the second condition in Remark~\ref{rem:conver}-$c)$, as in \eqref{eq: PnA} we need to show that for every positive $\varepsilon$ and $\eta$, there exists a $\phi \in (0,1)$, and an integer $n_0$ such that
\begin{equation}\label{eq: PnA2}
\frac{1}{\phi} \, \PP \Big[ \sup_{t \le s \le t + \phi} \big\|\Hat{J}^n_{s} - \Hat{J}^n_{t} \big\| \ge \varepsilon \Big]  \le \eta \quad \forall n \ge n_0\,.
\end{equation}
Following the same steps as in the proof of Theorem~\ref{pn-ERW-d=>4}, we have that
\begin{equation*}
\begin{split}
&\frac{1}{\phi} \PP \Big[ \sup_{t \le s \le t + \phi} \big\|\Hat{J}^n_{s} - \Hat{J}^n_{t} \big\| \ge \varepsilon \Big] \le 
 \frac{1}{\phi} \PP \Big[ \sum_{i = \delta\lfloor nt \rfloor}^{\delta\lfloor n(t + \phi) \rfloor +1} \um_{ \{U_i \le i^{-\frac{1}{2}}\}} \ge \varepsilon n^{\frac{1}{2}} \Big] 
% \\
 %& \leq  \frac{1}{\phi} e^{-\varepsilon n^{\frac{1}{2}}}\EE\Big[\exp\Big( \sum_{i = \delta\lfloor nt \rfloor + 1}^{\delta\lfloor n(t + \phi)\rfloor} \um_{ \{U_i \le i^{-\frac{1}{2}}\}} \Big) \Big] 
% \\
% & = \frac{1}{\phi} e^{-\varepsilon n^{\frac{1}{2}}} \prod_{i = \delta\lfloor nt \rfloor + 1}^{\delta\lfloor n(t + \phi)\rfloor} \left(1+ \frac{e-1}{i^{\frac{1}{2}}} \right) \le \frac{1}{\phi} e^{-\varepsilon n^{\frac{1}{2}}} \prod_{i = \delta\lfloor nt \rfloor + 1}^{\delta\lfloor n(t + \phi)\rfloor} \exp\left(\frac{e-1}{i^{\frac{1}{2}}} \right)  
% \\
% & \le \frac{1}{\phi} e^{-c n^{\frac{1}{2}}}  \exp\left(\sum_{i = \lfloor nt \rfloor + 1}^{\lfloor n(t + \phi)\rfloor}\frac{e-1}{i^{\frac{1}{2}}} \right) 
\\
& \le \frac{1}{\phi} \exp(-\varepsilon n^{\frac{1}{2}})\exp\left(2(e-1)(\sqrt{\delta n(t + \phi)} - \sqrt{\delta nt} + 2) \right) \,.
\end{split}    
\end{equation*}
and we obtain \eqref{eq: PnA2} choosing $\phi \in (0,1)$ sufficiently small such that $\sqrt{t+\phi} - \sqrt{t}< \varepsilon/4\sqrt{\delta}(e-1)$ for all $t \in [0, T]$. 
% \begin{equation*}
% \frac{1}{\phi} \exp(-\varepsilon n^{\frac{1}{2}})\exp\bigg(2(e-1)\sqrt{n\delta}\left(\sqrt{t + \phi} - \sqrt{t}\right) \bigg) \le \eta \quad \forall n \ge n_0 \,.
% \end{equation*}
% The latter can be easily verified as it was done in the proof of Theorem~\ref{pn-ERW-d=>4}. 

% % From now on the proof follows exactly  as in Theorem~\ref{pn-ERW-d=>4}, when we prove the second sum portion in~\eqref{p_n-ERW_incrementos_d=>4} fulfills the second condition of~\cite[Theorem 7.3] {billingsley1999probability}. \comu{precisa ser mais específico aqui} Then we have that for each positive $\varepsilon$ and $\eta$, there exists a $\phi \in (0,1)$, and an integer $n_0$ such that
% % \begin{equation*}
% % \frac{1}{\phi} \PP [|J'_{\lfloor n \cdot \rfloor}|/n^{1/2} \in A_t(\varepsilon, \phi)] \le \eta \quad \forall n \ge n_0\,.
% % \end{equation*}
% % Ergo by~\cite[Theorem 7.3]{billingsley1999probability} we obtain that the sequence $|J_{\lfloor n \cdot \rfloor}|/n^{1/2}$ is a tight in $C_{\Rs}[0, T]$ for all $T > 0$ with the topology of uniform convergence in compacts and moreover by~\cite[Theorem 2.4.10]{karatzas2012brownian} \com{here we cite a different result than the one cited in the remark!} is a tight sequence of processes in $C_{\Rs}[0, \infty)$. 
The proof of item $ii)$ is similar with the only difference that we analyze separately $\sum_{i=1}^{\lfloor n \cdot \rfloor} \um_{ \{U_i \le i^{-\frac{1}{2}}\}}/n^{1/2}$ and    $\sum_{i=1}^{\lfloor (n -\delta' n) \cdot \rfloor} \um_{\{U_i \leq i^{-1/2} \}}/n^{1/2}$.
%
Using the very same computation as in  item $i)$, we conclude that the linearly interpolated version of 
$$
    \Big\{\sum_{i=1}^{\lfloor n \cdot \rfloor} \um_{ \{U_i \le i^{-\frac{1}{2}}\}}/n^{1/2}\Big\}_{n\geq 1}\,  \text{ and } \Big\{ \sum_{i=1}^{\lfloor (n -\delta' n) \cdot \rfloor} \um_{\{U_i \leq i^{-1/2} \}}/n^{1/2}\Big\}_{n\geq 1}\,,
$$
% $\{\Hat{F}^n_{\cdot}/n^{1/2}\}_{n\geq 1}$ 
are tight sequences in $C_{\Rs}[0, T]$ for all $T>0$. Thus, the same holds for their difference.
%
\end{proof}


% \begin{proof}[Proof of Lemma~\ref{lem: Knt/n1/2tig}.] The proof follows the very same lines of that of Theorem~~\ref{pn-ERW-d=>4}, and we omit it. 
% {\color{red} 
% By  Remark~\ref{rem:conver}-$b)$  it suffices to show tightness in $C_{\Rs}[0, T]$ for all $T > 0$ and this is equivalent to show the the sequence of processes $\{|K_{\lfloor n \cdot \rfloor}|/n^{1/2}\}_{n\geq 1}$  satisfies  the two conditions in  Remark~\ref{rem:conver}-$c)$. 
% Note that $|K_{\lfloor n \cdot 0 \rfloor}|/n^{1/2}\equiv 0$ for all $n \ge 1$ and therefore the first condition in Remark~\ref{rem:conver}-$c)$ is satisfied.
% %
% To prove the second condition in Remark~\ref{rem:conver}-$c)$, 
% set 
% $
% A_t(\varepsilon, \phi):= \{f \in C_{\Rs}[0,T] : \sup_{t \le s \le t+\phi} |f(s) - f(t)| \ge \varepsilon \}$. 
% Then, following the same steps as in the proof of Theorem~\ref{pn-ERW-d=>4}, we have that  
% \begin{equation*}
% \begin{split}
% &\frac{1}{\phi} \PP [ |K_{\lfloor n \cdot \rfloor}|/n^{1/2} \in A_t(\varepsilon, \phi)]  = \frac{1}{\phi} \PP\Big[\sup_{t \le s \le t+\phi } | |K_{\lfloor ns \rfloor}| - |K_{\lfloor n t \rfloor}|| \ge \varepsilon n^{\frac{1}{2}} \Big]
% \\
% & = \frac{1}{\phi} \PP \Big[ \sum_{i = \lfloor nt \rfloor + 1}^{\lfloor n(t + \phi) \rfloor} \um_{E_i^c \cap \{U_i \le i^{-\frac{1}{2}}\}} \ge \varepsilon n^{\frac{1}{2}} \Big]
%  \le \frac{1}{\phi} \PP \Big[ \sum_{i = \lfloor nt \rfloor + 1}^{\lfloor n(t + \phi) \rfloor} \um_{\{U_i \le i^{-\frac{1}{2}}\}} \ge \varepsilon n^{\frac{1}{2}} \Big]\,.
% \end{split}    
% \end{equation*}
% From this point on, the computations are exactly the same used in the proof of Theorem~\ref{pn-ERW-d=>4}, and we omit it.

% when we show the second sum portion in~\eqref{p_n-ERW_incrementos_d=>4} fulfills the second condition in~\cite[Theorem 7.3]{billingsley1999probability}.
% Then we have that for each positive $\varepsilon$ and $\eta$, there exists a $\phi \in (0,1)$, and an integer $n_0$ such that
% \begin{equation*}
% \frac{1}{\phi} P_n \Big[f \in C_{\Rs}[0,T] : \sup_{t \le s \le t+\phi} |f(s) - f(t)| \ge \varepsilon \Big] \le \eta \, , \quad \forall n \ge n_0\,.
% \end{equation*}
% Ergo by~\cite[Theorem 7.3]{billingsley1999probability} we obtain that $|K_{\lfloor n \cdot \rfloor}|/n^{1/2}$ is tight in $C_{\Rs}[0, T]$ for all $T > 0$. Thus by~\cite[Theorem 2.4.10]{karatzas2012brownian} it is also a tight sequence of processes in $(C_{\Rs}[0, \infty),\rho)$
% } 
% \end{proof}

\begin{proof}[Proof of Proposition~\ref{prop: bound_Kn}.] 
We first prove $(a)$. 
Note that the statement follows if we show that
\begin{equation}\label{oldstatement}
\PP \left[\forall t \in [0,\infty): \mathcal{H}_t \le 2(t \delta)^{1/2} \right] = 1 \, ,  
\end{equation}
for every $\delta \in (\pi_d, 1)$. Indeed taking a sequence $\delta_n \downarrow \pi_d$ as $n\to \infty$, we have that 
$$
\Big\{ \forall t \in [0,\infty): \mathcal{H}_t \le 2(t \pi_d)^{1/2} \Big\} = \bigcap_{n\ge 1} \Big\{ \forall t \in [0,\infty):  \mathcal{H}_t \le 2(t \delta_n)^{1/2} \Big\}\, .
$$

Let us begin with some instrumental fact: recall (from page \pageref{eq: def Kn}) that $\varphi_i := \psi_i + 1$, where $\{\psi_i\}_{i \ge 1}$ denotes  the sequence of $\FF$-stopping times  corresponding to the  times the $p_n$-\Nametwo{} visits a new site, and let us define: 
\begin{align*}
J'_n(\delta) := 
\sum_{i=1}^{\delta n} \um_{\{U_{\varphi_i} \le \varphi_i^{-1/2} \}}\,,  
\end{align*} 
with $\delta \in (\pi_d, 1)$. 
%
Using Lemma \ref{lem: iid} and since $\{U_i\}_{i \ge 1}$ is i.i.d. we have that 
\begin{equation*}\label{eq:dominance}
J'_n(\delta)  \preceq J_n(\delta)\,, 
\end{equation*}
for all $\delta \in (\pi_d, 1)$ where $J_n(\delta)$ is defined  in~\eqref{Bn'}.

Consider the event 
\begin{equation*}
A_n^{{\delta}, c, M}:= \Big\{\forall t \in [c, M]: \frac{|K_{\lfloor nt \rfloor}|}{n^{1/2}} \le 2({\delta} t)^{1/2}  \Big\} \,,   
\end{equation*}
where $c$, $M$ and ${\delta}$ are positive constants such that $M > c$ and ${\delta} \in (\pi_d, 1)$, and recall that $|K_{\lfloor nt \rfloor}|  = \sum_{j=1}^{|\Rr_{\lfloor nt \rfloor -1}^X|} \um_{\{U_{\varphi_j} \leq \varphi_j^{-1/2} \}}$ (see,  \eqref{eq: def Kn}. For every $\Hat{\delta} \in (\pi_d,\delta)$ we have that
\begin{align}\label{eq: A_t^cM<}
\begin{split}
& \PP[A_n^{{\delta}, c,M}]  \ge \PP[A_n^{\delta, c,M} \cap \{ \forall t \in [c,M]: |\Rr_{\lfloor nt \rfloor}^X| \le \Hat \delta \lfloor nt \rfloor\}]
\\
& \ge \PP\Big[ \Big\{\forall t \in [c,M]: \frac{J'_{\lfloor nt \rfloor}(\delta)}{n^{1/2}} \le 2(\delta t)^{1/2} \Big\} \cap \big\{ \forall t \in [c,M]: |\Rr_{\lfloor nt \rfloor}^X| \le \Hat \delta \lfloor nt \rfloor \big\}  \Big] \, .
\end{split}    
\end{align}
Considering the second event on the right-hand side of~\eqref{eq: A_t^cM<}, by Proposition~\ref{prop:RangeERW}, for every $\Hat{\delta} \in (\pi_d, \delta)$ there exists an integer random variable $N_{\Hat{\delta}}$ such that $\PP[|\Rr_m^X| \le \Hat{\delta} m, \, \forall m \ge N_{\Hat{\delta}}] = 1$, hence 
$$\lim_{n \to \infty} \PP[\forall t \in [c,M]: |\Rr_{\lfloor nt \rfloor}^X| \le \Hat \delta \lfloor nt \rfloor] \ge \lim_{n \to \infty} \PP[N_{\Hat{\delta}} \le cn]=1.
$$
Now concerning the first event on the right-hand side of~\eqref{eq: A_t^cM<}, since $J'_n(\delta)  \preceq J_n(\delta)$, we obtain that 
\begin{equation*} 
\PP\Big[ \forall t \in [c,M]: \frac{J'_{\lfloor nt \rfloor}(\delta)}{n^{1/2}} \le 2(\delta t)^{1/2} \Big] \ge
\PP\Big[ \forall t \in [c,M]: \frac{J_{\lfloor nt \rfloor}(\delta)}{n^{1/2}} \le 2(\delta t)^{1/2} \Big] \, ,
\end{equation*}
%
where the right-hand side converges to one by Lemma~\ref{lem: tightaux} part $i)$ (convergence in distribution to a deterministic function implies convergence in probability, see \cite[page 27]{billingsley1999probability}).
%it holds that \com{notation of convergence of processes}
%\begin{equation}\label{eq:Jninprob}
%\frac{J_{\lfloor n \cdot \rfloor}}{n^{1/2}} \to 2(\delta \cdot)^{1/2}  \text{ as } n \to \infty  \,, 
%\end{equation}
%in probability, since it converges in distribution to a deterministic function in $C_{\mathbb{R}}[0,\infty)$ (see \cite[page 27]{billingsley1999probability}).
%
Hence on the right-hand side of~\eqref{eq: A_t^cM<} we have an intersection of two events whose probability converges to 1 as $n$ goes to infinity. Thus, for every $M > c >0$ and $\delta \in (\pi_d, 1]$ 
\begin{equation*}
\lim_{n\to \infty}  \PP[A_n^{\delta, c,M}]  = 1\,.
%\PP \Big[ \forall t \in [c, M]: \frac{|K_{\lfloor nt \rfloor}|}{n^{1/2}} \le 2(\Hat{\delta} t)^{1/2} \Big] \to 1 \text{ as } n \to \infty \,,   
\end{equation*}
%
%Since  $\hat{\delta} > \delta$ is arbitrary \com{I ?????}, we  obtain that 
%\begin{equation*}
%\PP \Big[ \forall t \in [c, M]: \frac{|K_{\lfloor nt \rfloor}|}{n^{1/2}} \le 2(\delta t)^{1/2}  \Big] \to 1 \text{ as } n \to \infty\,,   
%\end{equation*}
%
Now suppose that we have monotone decreasing and increasing  sequences $\{c_j\}_{j \ge 1}$ and $\{M_j\}_{j \ge 1}$ respectively, such that  $c_j \to 0$ and $M_j \to \infty$ as $j$ goes to infinity. Let $\{
\mathcal{H}_t\}_{t\ge 0}$ be a limit point in distribution of a subsequence of $\{\Hat{K}^n_{\cdot}/n^{1/2}\}_{n\ge 1}$, which is  tight  by  
Lemma~\ref{Jntight} (item $i)$), and define
\begin{equation*}
A^{\delta}:= \left\{\forall t \in [0, \infty): \mathcal{H}_t \le 2(\delta t)^{1/2}  \right\} \,,   
\end{equation*}
and 
$$
A^{\delta, c_i, M_i} = \left\{\forall t \in [c_i, M_i]: \mathcal{H}_t \le 2(\delta t)^{1/2}  \right\}\,,
$$
and $A^{\delta} = \cap_{i=1}^{\infty} A^{ \delta ,c_i, M_i}$ (since $H_0=0$).
%
Then,  by Portmanteau Theorem we have that for every $i \geq 1$ %\com{on the rhs we should specify the dependence on $\delta$ and add that for all $\delta \in (\pi_d,1]$...}
\begin{equation*}
\PP[A^{\delta ,c_i, M_i}] \ge \limsup_{n \to \infty} \PP[A_n^{\delta, c_i, M_i}] =1 \,.    
\end{equation*}
%
Hence, for all $i\geq 1$ we obtain that $\PP[A^{\delta, c_i, M_i}] = 1$ which, in turn, implies that 
\begin{equation}\label{eq:At_1}
\PP[A^{\delta}] = \PP\Big[ \bigcap_{j=1}^{\infty} A^{\delta, c_j, M_j}\Big] = 1 \,. 
\end{equation}
%
We now prove $(b)$. Analogously to the beginning of the proof of $(a)$, it is enough to prove that 
\begin{equation}\label{oldstatement}
\PP \left[\forall t \in [0,\infty): \mathcal{H}_t \ge  2t^{1/2}(1-(1 -  \delta')^{1/2})  \right] = 1 \, ,  
\end{equation}
for every  $\delta' \in (0, \pi_{d})$.
Indeed, if $\delta'_n \uparrow \pi_{d}$  as $n\to \infty$, we have that 
$$
\Big\{ \forall t \in [0,\infty): \mathcal{H}_t \ge 2t^{1/2}(1-(1 -  \pi_{d})^{1/2})  \Big\}\,,
$$
is equal to
$$
\bigcap_{n\ge 1} \Big\{ \forall t \in [0,\infty): \mathcal{H}_t \ge 2t^{1/2}(1-(1 - \delta'_n)^{1/2})  \Big\}\, .
$$
 Moreover, defining
\begin{align*}
V'_n(\delta') :=\sum_{i = 1}^{\delta' n} \um_{\{U_{\varphi_i} \leq (\varphi_i \wedge (n-i) )^{-1/2} \}}\,, %\label{eq: F'n_domsto}
\end{align*}
and using again Lemma \ref{lem: iid} and the i.i.d. property of $\{U_i\}_{i \ge 1}$, we have that 
\begin{equation*}
V'_n(\delta')  \succeq  V_n(\delta')\,, 
\end{equation*}for all $\delta' \in (0, \pi_{d})$
where $V_n(\delta')$ is defined in \eqref{Fn'}. 


We also define the following events 
\begin{equation*}
\begin{split}
& B_n^{\delta',c, M}:= \Big\{\forall t \in [c, M]: \frac{|K_{\lfloor nt \rfloor}|}{n^{1/2}} \ge 2t^{1/2}(1-(1-\delta')^{1/2}) \Big\} \,,
\\
& H_n^{\delta',c,M}:= \Big\{\forall t \in [c, M]: \frac{V'_{\lfloor nt \rfloor}(\delta')}{n^{1/2}} \ge 2t^{\frac{1}{2}}(1-(1-\delta')^{\frac{1}{2}})  \Big\}\,,
% \\
% & R_{\lfloor nt \rfloor}:= \big\{ \forall t \in [c,M]: |\Rr_{\lfloor nt \rfloor}^X| \ge \delta'' \lfloor nt \rfloor \big\} \,,
\end{split}
\end{equation*}
where $c$, $M$ and  $\delta'$  are positive constants such that $M > c$ and $\delta'\in (0,\pi_{d})$. 
%
Given $\delta'' \in (\delta', \pi_{d})$,  we have that
\begin{equation}\label{eq: B_t^cM>}
\begin{split}
\PP[B_n^{\delta',c,M}]  & \ge \PP\big[B_n^{\delta',c,M} \cap \big\{ \forall t \in [c,M]: |\Rr_{\lfloor nt \rfloor}^X| \ge \delta'' \lfloor nt \rfloor \big\}\big] 
\\
&\ge \PP\big[ H_n^{\delta',c,M} \cap \big\{ \forall t \in [c,M]: |\Rr_{\lfloor nt \rfloor}^X| \ge \delta'' \lfloor nt \rfloor \big\} \big]\,.
\end{split}
\end{equation}
%
Since $V'_n(\delta')  \succeq V_n(\delta')$, it holds that 
$$
\PP\big[ H_n^{\delta',c,M} \big]
\ge \PP \Big[ \forall t \in [c, M]: \frac{V_{\lfloor nt \rfloor}(\delta')}{n^{1/2}} \ge 2t^{1/2}(1-(1-\delta')^{1/2}) \Big]\, .
$$
%and by the coupling with the lazy random walk 
%$$
%\PP\big[ \forall t \in [c,M]: |\Rr_{\lfloor nt \rfloor}^X| \ge \delta'' \lfloor nt \rfloor  \big]
% \ge \PP \big[\forall 
% t \in [c,M]:|\Rr_{\lfloor nt \rfloor}^Y| \ge \delta'' \lfloor nt \rfloor\big] \,.    
%$$ 
Now by Lemma~\ref{lem: tightaux} part $ii)$  (convergence in distribution to a deterministic function implies convergence in probability, see \cite[page 27]{billingsley1999probability}) 
we obtain that $\lim_{n \to \infty}\PP\big[ H_n^{\delta',c,M} \big]=1$, for every $\delta' \in (0,\pi_{d})$. Moreover, since $d\geq 22$, we have that $\pi_{d}>0$ and by Proposition~\ref{prop:RangeERW_lower} there exists an integer random variable $N_{\delta''}$ such that $\PP [|\Rr_{m}^X| \ge \delta''m ,\ \forall m \ge N_{\delta''} ] = 1$, thus $\lim_{n\to \infty} \PP \big[\forall 
 t \in [c,M]:|\Rr_{\lfloor nt \rfloor}^X| \ge \delta'' \lfloor nt \rfloor\big] \ge \lim_{n\to \infty} \PP [N_{\delta''} \le cn ] = 1$, for every $\delta'' \in (0,\pi_{d})$.
%we obtain that
%\begin{equation}\label{eq:Fninprob}
%\frac{\sum_{j=1}^{\lfloor n\cdot \rfloor} \um_{\{ U_j \le j^{-1/2} \}} - F_{\lfloor n\cdot \rfloor}}{n^{\frac{1}{2}}} \to 2(\cdot)^{1/2}(1-(1-\delta')^{1/2}) \text{ as } n \to \infty  \,, 
%\end{equation}
%in probability \com{notation of convergence of processes!}, since it converges in distribution to a  continuous function in $t$ (see \cite[page 27]{billingsley1999probability}).
%
Since in~\eqref{eq: B_t^cM>} we have an intersection of two events whose probability converges to 1 as $n$ goes to infinity, we may conclude that for every $M > c >0$ 
\begin{equation*}
\PP \Big[ \forall t \in [c, M]: \frac{|K_{\lfloor nt \rfloor}|}{n^{1/2}} \ge 2t^{1/2}(1-(1-\delta')^{1/2}) \Big] \xrightarrow[n \to \infty]{} 1 \,.
\end{equation*} 
%
Finally, we finish the proof by the same argument used to obtain~\eqref{eq:At_1}.
\end{proof}

\section{On the range of $p_n$-\Nametwo{} for $\beta=1/2$ and $d\geq 2$}\label{sec:rangeERW}

\subsection{Upper bound for the range (proof of  Proposition~\ref{prop:RangeERW})}

\hfill \\

  
% Before we start, we introduce some auxiliary results required to analyze the asymptotic behavior of the $p_n$-\Nametwo{} on $\ZZ^d$, with $d\geq 2$ and $\beta = 1/2$.
%
% Let $\{\xi_i\}_{i \ge 1}$ be i.i.d. $\ZZ^d$-valued random variables with zero-mean vector and finite variance. Let $\{Y_n\}_{n \ge 0}$ be the random walk on $\ZZ^d$ with increments $\{\xi_i\}_{i \ge 1}$ starting at $Y_0 = 0$, thus $Y_n = \sum_{i=1}^{n} \xi_i$, $n\ge 1$.  For $m \leq n$ define 
% \[ \Rr_{[m,n]} ^Y := \{Y_m, Y_{m+1}, \ldots, Y_n\}\;,\]
% and denote by $\Rr_{n} = \Rr_{[0,n]}$,  the range of the random walk $\{Y_n\}_{n \geq 0}$.
%
% We now state a known result about the range of a random walk on $\ZZ^d$ with i.i.d. increments which is instrumental in our proof. Recall from the statement of Proposition~\ref{prop:RangeERW} that $\pi_d$ denotes the probability that $\{Y_n\}_{n \geq 0}$ never returns to the origin. 
%
% \begin{theorem}[{\cite[Theorem 1]{hamana2001large}}]\label{teo: RnZ>}
% Let $\{Y_n\}_{n \geq 0}$ be an aperiodic random walk on $\ZZ^d$, with $d\geq 2$. It holds that 
% \begin{align*}
% \tag{L}&\lim_{n \to \infty} \PP[|\Rr_{n}^Y| \geq \theta n]= 1\,, &&  \text{for every $\theta < \pi_d$}\,, 
% \\
% \tag{U}&\PP[|\Rr_{n}^Y| \geq \theta' n] \leq e^{-c_{\theta'}n}\,,    && \text{for every  $\theta' > \pi_d$ and $n$ sufficiently large}\,, 
% \end{align*}
% where $c_{\theta'}$ is a positive constant that depends of $\theta'$ (note that for $d=2$, we have that $\pi_d=0$, whereas for $d\geq 3 $, $\pi_d\in (0,1]$. 
% \end{theorem}
%
% \begin{remark}
% It is important to notice that~\cite[Theorem 1]{hamana2001large} is stated for aperiodic random walks. \cm{ The definition of aperiodicity applied in this Theorem is different than the usual one (see Yuval). here the definition used comes from Spitzer. There to a process be aperiodic means (...).  However, as explained in~\cite[page 188-beginning of Section 2]{hamana2001large}, this condition does not entail a loss of generality.} %However, as explained in~\cite[page 188-beginning of Section 2]{hamana2001large}, this condition can be relaxed to \com{include??} periodic random walks. \comu{incluir comentário sobre definição de aperiódico}
% \end{remark}
The heuristic of the proof of Proposition~~\ref{prop:RangeERW} is the following: 
suppose that the $p_n$-\Nametwo{} at time $j$ visits a new site and gets excited. Then, after $k$ further steps it visits another new site and gets excited again and  no excitation occurs between time $j+1$ and $j+k-1$. Thus,  we know that between time $j+2$ and $j+k$ the process evolves as a random walk with i.i.d. increments. Specifically, in each of these time windows between two consecutive excitations, we can use the range of the random walk with i.i.d. increments to  upper bound the range of the $p_n$-\Nametwo{}.

%The main idea of this proof is we know that the $p_n$-\Nametwo{} behaves like random walk biased in direction $\ell$ \comu{"drift random walk" não é um bom termo} when it eats a cookie. Then between two cookies we have that the process behaves like an i.i.d. random walk. Hence in this lengths, we think as independent i.i.d. random walks \comu{"i.i.d. random walk" não é um bom termo, talvez "random walk with i.i.d. increments. Recorrente.} and use the ranges of those process to upper bound the range of the $p_n$-\Nametwo{}. By Lemma~\ref{RnY_upperb} we can control the range of each independent random walk, thus we obtain the desired result.  \comu{Só neste parágrafos há vários problemas de inglês}

\begin{proof}[Proof of Proposition~\ref{prop:RangeERW}]
%\textcolor{red}{Let us denote $(N_i, i \geq 0)$  as the sequence of times that the process $X$ is allowed to eat a cookie, this happens if the position is being visited for the first time and the Bernoulli trial in the site goes in favor of the process with a drift

%\texttt{The above paragraph is confusing: $N_i$ only looks at the Bernoulli variables regardless if the sites visited at the corresponding time has already been visited before or not!}}. %, that is, it behaves like process with a drift to the right. 
%\cm{Let us denote $\{N_i\}_{i \geq 0}$ as the sequence of times that the Bernoulli trial goes in favor of the $p_n$-\Nametwo{} behaves like a random walk biased in direction $\ell$.}


%Let us denote $\{N_i\}_{i \geq 0}$ as the sequence of times that the Bernoulli trial in the site goes in favor of the process with a drift, independently of the process has visited the site or not \comu{É muito difícil entender esta última frase}.
We provide the proof for $\beta = 1/2$, which represents the most  challenging scenario. The proof for $\beta > 1/2$ can be derived by the same technique presented here.

Denote by $\{N_i\}_{i \geq 0}$ the sequence of stopping times 
\[ N_0 \equiv 0 \ \ \mathrm{and} \ \ N_i := \inf\{ k > N_{i-1}: Z_k = 1 \} \, , \ i\ge 1 \,, \]
where $Z_k = \um_{\{U_k \le k^{-1/2} \}}$, $k \geq 1$, are independent random variables with Bernoulli distribution of parameter respectively $k^{-1/2}$ for each $k$.
%
Set $\Delta N_i = N_i - N_{i-1} $ and define 
\[ M_n := \inf \Big\{ i \geq 1 : \sum_{j=1}^i \Delta N_j \geq n \Big\} \,. \]

Note that $|\Rr_n ^X| - 1  = \sum_{t=1}^n \um_{\{X_t \neq X_l, \forall l < t \}}$ is bounded from above by 
\begin{align*}
 &  \sum_{t=1}^{N_1} \um_{\{X_t \neq X_l, \forall l < t \}} + \sum_{t=N_1 +1}^{N_2} \um_{\{X_t \neq X_l, \forall l < t \}} + \dots + \sum_{t=N_{M_n -1} +1}^{N_{M_n}} \um_{\{X_t \neq X_l, \forall l < t \}}
 \\
 & \leq M_n + \sum_{j = 1}^{M_n} \sum_{t = N_{j-1} + 2}^{N_j} \um_{\{X_t \neq X_l, \forall l < t \}}\\
 &\leq M_n + \sum_{j = 1}^{M_n} \sum_{t = N_{j-1} + 2}^{N_j} \um_{\{X_t \neq X_l, \forall l \in [N_{j-1} + 1, t) \}} \,.
\end{align*}
In each time interval $[N_{j-1} + 2, N_j]$ the process $X$ behaves like a random walk with i.i.d. increments. In order to have some control on the length of these intervals, or equivalently on $\{\Delta N_j\}_{j \geq 1}$, we proceed as follows: Let $\varepsilon \in (0, 1)$ and note that 
\begin{equation}\label{eq: Rn<}
|\Rr_n ^X| \leq |\Rr_{[0,n^{\varepsilon}]} ^X| + |\Rr_{[n^{\varepsilon}, n]} ^X| \leq n^{\varepsilon} + |\Rr_{[n^{\varepsilon}, n]} ^X|\; , 
\end{equation}
this last step will be necessary to guarantee that after time $n^\varepsilon$ the time intervals $\Delta N_j$ are (in distribution) sufficiently large. Thus, we may just redefine $N_0 \equiv n^{\varepsilon}$ and apply the very same decomposition as before to obtain %\comu{Qual o efeito da escolha do $N_0$ do lado direito?} {\color{blue} (note que não podemos usar $\Rr^Y_{\Delta N_j}$ na soma abaixo, porque teríamos ranges sobre intervalos que não são disjuntos.)}
\begin{align} \label{pR_DNj}
|\Rr_{[n^{\varepsilon}, n]} ^X| -1 &\leq
 M_n + \sum_{j = 1}^{M_n} \sum_{t = N_{j-1} + 2}^{N_j} \um_{\{X_t \neq X_l, \forall l \in [N_{j-1}+1, t) \}} \nonumber
 \\ 
 & \leq  M_n + \sum_{j = 1}^{M_n} |\Rr_{[N_{j-1} + 2, N_j]} ^Y| \,. 
\end{align}
%where $\{Y_n\}_{n \geq 0}$ denotes a random walk whose i.i.d.~increments in $\ZZ^2$ are $\{\xi_i\}_{i \ge 1}$. 
%\cm{ Important to notice that $Y$ is independent of $N$, since the increments of $Y$ are determined by the sequence of $\{ \xi_i \}_{i \ge 1}$ which occur in the length $[N_{j-1}+ 2, N_j]$. ??} \comu{importante mencionar que Y é independente dos N's pois seus incrementos são deteminados pelos $\xi$'s.}
%
We pointed out that to deal properly with the rightmost sum in~\eqref{pR_DNj}, we need to keep in mind that $Y$ is independent of  $\{N_j\}_{j\ge 1}$.
Now, for any $k \in \{1, 2, \dots, n \}$ fixed, we define  the  random set 
\[
A_{n,k} \coloneqq \{ j \in \{1, 2, \dots, M_n \} : \Delta N_j \leq k \}\,,
\] 
and we write 
\begin{align*}
 \sum_{j = 1}^{M_n} |\Rr_{[N_{j-1} + 2, N_j]}^Y| & = \sum_{j \in A_{n,k}} |\Rr_{[N_{j-1} + 2, N_j]}^Y| + \sum_{j \in A_{n,k}^{c}} |\Rr_{[N_{j-1} + 2, N_j]}^Y| 
\\
& \leq (k-1)|A_{n,k}| + \sum_{j \in A_{n,k}^c} |\Rr_{[N_{j-1} + 2, N_j]}^Y| \;. 
\end{align*}
Using the simple inequality
$$
M_n = |A_{n,k}| + |A_{n,k}^c| \le |A_{n,k}| + \frac{M_n}{k} \le |A_{n,k}| + \frac{n}{k}\,,
$$
and \eqref{pR_DNj}, we obtain that 
\begin{equation}\label{R_DNj}
|\Rr_{[n^{\varepsilon}, n]} ^X| \le 1 +  \frac{n}{k} + k |A_{n,k}| + \sum_{j \in A_{n,k}^c} |\Rr_{[N_{j-1} + 2, N_j]}^Y|\,.
\end{equation}
To control the right-hand side of \eqref{R_DNj}, we begin with estimates on $|A_{n,k}|$. By coupling arguments, if $G$ denotes a Geometric random variable with parameter $n^{-\varepsilon/2}$, we have that  $G \preceq \Delta N_j$ for all $j \in \{1, 2, \dots, M_n \}$, where $\preceq$ means stochastic dominance. Thus, it holds that
\begin{equation}\label{eq: DNj<k}
\PP[\Delta N_j \leq k] \leq 1 - \left( 1 - \frac{1}{n^{\varepsilon/2}} \right)^k \,.    
\end{equation}

Since $|A_{n,k}| = \sum_{j=1}^{M_n} \um_{\{\Delta N_j \leq k \}}$ and $\{\Delta N_j\}_{j=1}^n$ are independet, using \eqref{eq: DNj<k},   for every $a>0$ and $n$ sufficiently large, it holds that
\begin{align*}
&\PP\Big[  \sum_{j=1}^{M_n} \um_{\{\Delta N_j \leq k \}}  > a \Big]  \leq  \PP\Big[  \sum_{j=1}^{n} \um_{\{\Delta N_j \leq k \}}  > a \Big]
%& \leq \PP\left[  \sum_{j=1}^{n} 1_{\{\Delta N_j \leq k \}}  > a \right]
%\\
 \leq \binom{n}{\lceil a \rceil} \PP[G \leq k]^{\lceil a \rceil} \,  
\\
& \leq \Big( \frac{ne}{\lceil a \rceil} \Big)^a \Big( 1 - \Big( 1 - \frac{1}{n^{\frac{\varepsilon}{2}}} \Big)^k \Big)^{\lceil a \rceil}
 \leq \Big( \frac{ne}{a} \Big)^{\lceil a \rceil} \Big( 1 - \exp{-\frac{3}{2}\frac{k} {n^{\frac{\varepsilon}{2}}}} \Big)^{\lceil a \rceil} 
 \\&\leq \Big( \frac{ne}{a} \times \frac{3k}{2n^{\frac{\varepsilon}{2}}} \Big)^{\lceil a \rceil} \,,
\end{align*}
where in the last inequalities we used that 
$\left(1-\frac{1}{x}\right)^x\geq e^{-3/2}$, $\forall x\geq 2$ (with $n$ sufficiently large) and that $1-e^{-x}\leq x$. 
Setting  $a= n^{1-\varepsilon/4}$ we obtain that 
\begin{equation}\label{eq:1_DNj<k}
\PP\Big[  \sum_{j=1}^{M_n} \um_{\{\Delta N_j \leq k \}}  > n^{1-\varepsilon/4} \Big] \leq \Big( \frac{3ek}{2n^{\frac{\varepsilon}{4}}} \Big)^{n^{1-\varepsilon/4}} \,.  
\end{equation}


We now set $k = \lceil \log^2 (n) \rceil$. With this choice, the deterministic first term in \eqref{R_DNj} divided by $n$ converges to zero.  Moreover, the sum in $n$ of the probabilities of the events $\{ |A_{n, \lceil \log^2 (n) \rceil}|  > n^{1-\varepsilon/4} \}$ is finite by \eqref{eq:1_DNj<k}. Thus, by Borel-Cantelli,  the second term in \eqref{R_DNj} divided by $n$ converges to zero almost surely.

Now we are left with the  analysis of the third term in~\eqref{R_DNj}. 
% First we will obtain an upper bound on the probability of the event that there exist at least one $j \in A_{n, \lceil \log^2 (n) \rceil}^c$, such that $|\Rr_{[N_{j-1} + 2, N_j]}^Y| > \gamma \Delta N_j$, where $\gamma \in (\pi_d, 1]$.
%\begin{align}\label{eq: R_DNj>y}
%\PP[\exists j \in A_{n, \lceil \log^2 (n) \rceil}^c : |\Rr_{[N_{j-1} + 2, N_j]}^Y| > \gamma \Delta N_j] & = \sum_{j=1}^{n} 1_{\{j \in A_{n, \lceil \log^2 (n) \rceil}^c \}} \PP[\Rr_{[N_{j-1} + 2, N_j]}^Y| > \gamma \Delta N_j]
%\\
%& \leq \sum_{j \in A_{n, \lceil \log^2 (n) \rceil}^c} \PP[\Rr_{[N_{j-1} + 2, N_j]}^Y| > \gamma (\Delta N_j-2)] \nonumber
%\\
%& \leq |A_{n, \lceil \log^2 (n) \rceil}^c| \exp \left(-c_{\gamma} \lceil \log^2 (n) -2\rceil \right) \nonumber
%\\
%& \leq \frac{n}{\lceil \log^2 (n) \rceil} \exp\left(-c_{\gamma} \lceil \log^2 (n) -2\rceil \right) \nonumber
%\\
%& \leq \exp\left( \log\left( \frac{n}{\lceil \log^2 (n) \rceil} \right) - c_{\gamma} \lceil \log^2 (n)-2 \rceil \right) \;. \nonumber
%\end{align}
%In the third inequality in~\eqref{eq: R_DNj>y} we use Lemma~\ref{RnY_upperb}. 
%
%Let us now analyze the probability term in~\eqref{eq: R_DNj>y}, remembering that $j \in A_{n, \lceil \log^2 (n) \rceil}^c$, then we have $\Delta N_j > \lceil \log^2 (n) \rceil$, for all $j$ 
%\begin{align}\label{eq: R_DNj>y2}
%\PP[\Rr_{[N_{j-1} + 2, N_j]}^Y| > \gamma \Delta N_j] & = \sum_{i=\lceil \log^2 (n) \rceil +1}^{\infty} 1_{\{\Delta N_j = i \}} \PP[\Rr_{[N_{j-1} + 2, N_j]}^Y| > \gamma(i-2)]
%\\
%& \leq \sum_{i=\lceil \log^2 (n) \rceil +1}^{\infty} 1_{\{\Delta N_j = i \}} \exp \left(-c_{\gamma} ( i -2)\right) \nonumber
%\\
%& \leq \exp \left(-c_{\gamma} \lceil \log^2 (n)\rceil -2 \right) \;. \nonumber
%\end{align}
%In the second inequality in~\eqref{eq: R_DNj>y2} we use Lemma~\ref{RnY_upperb}.
%
%Since we have
%\begin{equation}\label{eq: R_DNj>y3}
%|A_{n, \lceil \log^2 (n) \rceil}^c| \leq  \frac{n}{\lceil \log^2 (n) \rceil} \;,    
%\end{equation}
%for all $n$. We obtain in~\eqref{eq: R_DNj>y} from~\eqref{eq: R_DNj>y2} and~\eqref{eq: R_DNj>y3}
%\begin{align}\label{eq: gDNj}
%\PP[\exists j \in A_{n, \lceil \log^2 (n) \rceil}^c : |\Rr_{[N_{j-1} + 2, N_j]}^Y| > \gamma \Delta N_j] & \leq \frac{n}{\lceil \log^2 (n) \rceil} \exp\left(-c_{\gamma} \lceil \log^2 (n) -2\rceil \right)
%\\
%& \leq \exp\left( \log\left( \frac{n}{\lceil \log^2 (n) \rceil} \right) - c_{\gamma} \lceil \log^2 (n)-2 \rceil \right) \;. \nonumber
%\end{align}
%
First,  we observe that for all $n\geq 2 $
\begin{equation}\label{eq: R_DNj>y3}
|A_{n, \lceil \log^2 (n) \rceil}^c| \leq  \frac{n}{\lceil \log^2 (n) \rceil} \,.    
\end{equation}

By Theorem~\ref{teo: RnZ>} (LD),  for all $i> \lceil \log^2(n) \rceil$ (with   $n$ sufficiently large) and for all $\gamma \in (\pi_d,1]$, it holds that 
\begin{align}\label{eq: RYi}
\begin{split}
\PP[|\Rr_{i-2}^Y| > \gamma i] &\leq  \PP[|\Rr_{i-2}^Y| > \gamma (i-2)]
 \leq \exp(-c_{\gamma}(i-2)) 
\\
& \leq \exp(-c_{\gamma}(\lceil \log^2(n) \rceil-2)) \,.
\end{split}
\end{align}
Recalling that for all $j \in A_{n, \lceil \log^2 (n) \rceil}^c$ it holds that $\Delta N_j > \lceil \log^2 (n) \rceil$,   by~\eqref{eq: R_DNj>y3} and~\eqref{eq: RYi} we obtain that
\begin{align}\label{eq: gDNj}
\begin{split}
\PP & \left[\exists j \in A_{n, \lceil \log^2 (n) \rceil}^c : |\Rr_{[N_{j-1} + 2, N_j]}^Y| > \gamma \Delta N_j\right] 
 \leq \frac{n \exp\left(-c_{\gamma} \lceil \log^2 (n) -2\rceil \right)}{\lceil \log^2 (n) \rceil}  
\\
& \leq \exp\left( \log\left( \frac{n}{\lceil \log^2 (n) \rceil} \right) - c_{\gamma} \lceil \log^2 (n)-2 \rceil \right) \,.
\end{split}
\end{align}
%\textcolor{purple}{Remember that, for all $j \in A_{n, \lceil \log^2 (n) \rceil}^c$, then we have $\Delta N_j \in ( \lceil \log^2 (n) \rceil$, n]. Hence we obtain
%\begin{align*}
%\PP[\Rr_{[N_{j-1} + 2, N_j]}^Y| > \gamma \Delta N_j] & \leq \sum_{i=\lceil \log^2 (n) \rceil +1}^{n} \PP[\Rr_{i-2}^Y| > \gamma i| \Delta N_j = i]
%\\
%& \leq \sum_{i=\lceil \log^2 (n) \rceil +1}^{n} \PP[\Rr_{i-2}^Y| > \gamma (i-2)| \Delta N_j = i]
%\\
%& \leq \sum_{i=\lceil \log^2 (n) \rceil +1}^{n} \exp(-c_{\gamma}(i-2))
%\\
%& \leq (n-\lceil \log^2 (n) \rceil)\exp\left(-c_{\gamma} \lceil \log^2 (n) -2\rceil \right)\;. 
%end{align*}}
Since it holds that
% $\big\{ \exists j \in A_{n, \lceil \log^2 (n) \rceil}^c : |\Rr_{[N_{j-1} + 2, N_j]}^Y| > \gamma \Delta N_j\big\}$ contains 
\begin{align*}
&\left\{ \exists j \in A_{n, \lceil \log^2 (n) \rceil}^c : |\Rr_{[N_{j-1} + 2, N_j]}^Y| > \gamma \Delta N_j\right\} \\
&\supseteq
\Big\{\sum_{j \in A_{n,\lceil \log^2 (n) \rceil}^c} |\Rr_{[N_{j-1} + 2, N_j]}^Y| > \gamma n\Big\}\,,
\end{align*}
then this last event has probability bounded above by the rightmost term in \eqref{eq: gDNj}
%\begin{equation}\label{eq: sumRy}
%\begin{split}    
%    \PP\Big[ \sum_{j \in A_{n,k}^c} |\Rr_{[N_{j-1} + 2, N_j]}^Y| > \gamma n \Big] & \leq \exp\Big( \log\Big( \frac{n}{\lceil \log^2 (n) \rceil} \Big) - c_{\gamma} \lceil \log^2 (n)-2 \rceil \Big) \,.
    %\\
    %& \to 0 \quad \text{as } n \to \infty \,.
%\end{split}    
%\end{equation}
which is summable.
%$$\Big\{ \sum_{j \in A_{n,k}^c} |\Rr_{[N_{j-1} + 2, N_j]}^Y| > \gamma n \Big\}$$ 
Thus by Borell-Cantelli Lemma, the third term in~\eqref{R_DNj} divided by $n$ is bigger than $\gamma$ only finitely many times almost surely.
%
Hence,  letting $\gamma\downarrow \pi_d$ completes the proof. 
\end{proof}

\subsection{Lower bound for the range (proof of Proposition~\ref{prop:RangeERW_lower})}
\hfill \\

Given  $\beta\geq 1/2$ and  $\ell \geq 1$ set $Z_\ell^\beta=\um_{\{U_\ell \leq  \ell^{-\beta/2}\}}$, i.e., $Z_\ell^\beta$ is a random variable with Bernoulli distribution with parameter $\ell^{-\beta/2}$. 
Let us define the following set:
\begin{align}\label{eq:set_A}
A_{k,\delta, \beta} := \left\{\sum_{\ell = k}^{k + \lfloor k^\delta \rfloor}Z_\ell^\beta=0\right\}\,,
\end{align}
i.e.,  the event that in the first $\lfloor k^\delta \rfloor$ steps after time $k$ none of the corresponding Bernoulli are successful. In particular, the occurrence of the  event $A_{k,\delta, \beta}$ implies that the random walk does not get excited in the time window from $k$ to $k + \lfloor k^\delta \rfloor$. 

Before proving Proposition~\ref{prop:RangeERW_lower} we state a couple of auxiliary results. 
%
\begin{lemma}\label{lemma_1}  
Given $\beta\geq 1/2$ and  $\delta \in (0, 1/2)$  consider the sequence of events $\{A_{k, \delta,\beta}\}_{k\geq 1}$ defined in \eqref{eq:set_A}. Then, it holds that 
\begin{equation*}			
\lim_{n \to \infty} \frac{1}{n} \sum_{k = 1}^n \um_{A_{k,\delta, \beta}^c} = 0\,, \, \text{ a.s.}.
\end{equation*}
\end{lemma}

%%%%%% OLD LEMA %%%%%%%%%
% \begin{lemma}\label{lemma_1}  \com{maybe we should state this lemma for $\beta \geq 1/2$ to be more coherent with the rest....} 
% Let $\delta$ be a positive real number such that $\delta \in (0, 1/2)$ and we consider the event $A_{k, \delta}$ for $k \ge 1$. Then we have
% \begin{equation*}			
% \lim_{n \to \infty} \frac{1}{n} \sum_{k = 1}^n \um_{A_{k,\delta}^c} = 0\,, \, \text{ a.s.}.
% \end{equation*}
% \end{lemma}

The proof of Lemma~\ref{lemma_1} is given in  Appendix~\ref{sec:appendixC}.

\medskip

Given a $\mathbb{Z}^d$-valued process $\{S_n\}_{n\ge 1}$, we set
$$
e_k^S := \um_{\{ S_m \neq S_k \text{ for all } m > k \}} \,.
$$
Let $\mathcal{D}_n^S$ denote  the set of sites visited by the process $S$ up to time $n$  which are never revisited later. Then,  
\begin{equation*}
  |\mathcal{D}_n^S|= \sum_{k = 0}^{n} e_k^S \,.
\end{equation*}

\begin{lemma}\label{compXY}
Let  $X$ be a $p_n$-\Nametwo{} in direction $\ell$, on $\ZZ^d$ with $d\geq 22$, $p_n= \mathcal{C} n^{-\beta} \wedge 1$ and $\beta \ge 1/2$. For $k\ge 1$, let $Y^k =\{Y^k_i\}_{i\ge 0}$ denote a random walk on $\ZZ^d$ defined by $Y_0^k = X_k$ and for $n\geq 1$
\begin{equation*}
     Y_n^k = X_k + \sum_{i = 1}^{n}\xi_{k + i}\,.
\end{equation*}
Then, for any $\delta \in (0, 1/2)$ it holds that  
\begin{equation*}
\sum_{k = 1}^{\infty} \mathbb{P} \left( e_k^{X} \um_{A_{k,\delta,\beta}} < e_0^{Y^k} \um_{A_{k, \delta,\beta}} \right) < \infty \,.   
\end{equation*}
\end{lemma}

Before proving Lemma~\ref{compXY}, we show how the proof of Proposition~\ref{prop:RangeERW_lower} follows from it.
\medskip

\begin{proof}[Proof of Proposition~\ref{prop:RangeERW_lower}]
Note that $|\Rr_n^X| \ge |\mathcal{D}_n^X|$, since if $x \in \mathcal{D}_n^X$ then $x \in \Rr_n^X$. Therefore, it holds that 
\begin{align}\label{eq_4.10}
\liminf_{n \to \infty} \frac{|\Rr_n^X|}{n} &\ge \liminf_{n \to \infty} \frac{|\mathcal{D}_n^X|}{n}\nonumber \\
&= \liminf_{n \to \infty} \frac{1}{n}\sum_{k=1}^n e_k^X \um_{A_{k,\delta,\beta}} + \liminf_{n \to \infty} \frac{1}{n}\sum_{k=1}^n e_k^X \um_{A_{k,\delta,\beta}^c}\,.  
\end{align}
For $\delta \in (0, 1/2)$, Lemma~\ref{lemma_1} implies that 
\begin{equation}\label{eq_4.11}
    \liminf_{n \to \infty} \frac{1}{n}\sum_{k=1}^n e_k^X \um_{A_{k,\delta,\beta}^c}=0, \; \text{ a.s.}\,,
\end{equation}
whereas, from Lemma~\ref{compXY} together with Borel-Cantelli's Lemma, we conclude that 
\begin{equation}\label{eq_4.12}
\liminf_{n \to \infty} \frac{1}{n} \sum_{k = 1}^n e_k^X \um_{A_{k, \delta,\beta}} \ge \liminf_{n \to \infty} \frac{1}{n} \sum_{k = 1}^n e_0^{Y^k} \um_{A_{k, \delta,\beta}} \;, \text{ a.s.}\,.
\end{equation}
Note that
\begin{align*}
    e_0^{Y^k} = \um_{\{\sum_{i = k+1}^m\xi_{i} \neq 0, \, \forall \, m\ge k + 1\}} = \um_{\{\sum_{i = 1}^m\xi_{i} + Y_0\neq \sum_{i = 1}^k\xi_{i} + Y_0, \, \forall \, m\ge k + 1\}} =e^Y_k\,,
\end{align*}
where $Y=Y^0$ on the RHS above denotes a random walk with i.i.d. increments $\{\xi_i\}_{i\geq 1}$. 
Hence
\begin{equation}\label{eq_4.13}
    \liminf_{n \to \infty} \frac{1}{n} \sum_{k = 1}^n e_0^{Y^k} \um_{A_{k, \delta,\beta}} = \liminf_{n \to \infty} \frac{1}{n} \sum_{k = 1}^n e_k^{Y} \um_{A_{k, \delta,\beta}}\,. 
\end{equation}
Using again Lemma~\ref{lemma_1} and the known fact that $\lim_{n \to \infty} \frac{|\mathcal{D}_n^Y|}{n} = \pi_d$ (see e.g.~\cite{spitzer2001principles} page 39) we obtain that 
\begin{equation}\label{eq_4.14}
\lim_{n \to \infty} \frac{1}{n} \sum_{k = 1}^n e_k^Y \um_{A_{k, \delta,\beta}} = \pi_d \;, \text{ a.s.}\,.
\end{equation}
Then,  by equations \eqref{eq_4.10},\eqref{eq_4.11},\eqref{eq_4.12},\eqref{eq_4.13} and \eqref{eq_4.14}, we obtain that
$$
    \liminf_{n \to \infty} \frac{|\Rr_n^X|}{n} \ge \pi_d \; , \text{ a.s.}\,.
$$
\end{proof}

\begin{proof}[Proof of Lemma~\ref{compXY}] 
We conduct the proof for $\beta = 1/2$, which represents the most  challenging scenario. The case  $\beta > 1/2$ follows using the same computations and techniques presented here. To avoid clutter  we will denote $A_{k,\delta, 1/2}$ just by $A_{k,\delta}$.

Let $\nu_k^X$ be the first time the process $X$ returns to the site it visited at time $k$.  We also introduce the sequence $\{ \tau_i^k \}_{i \ge 0}$ where $\tau_i^k$ for $i \ge 1$ represents the $i$-th time  the  random walk gets excited  after time $k$. Specifically, 
\begin{align*}
& \nu^{X}_k:= \inf \{n \geq 1: X_{k+n}= X_{k}\}\,,
\\
& \tau^k_0\equiv 0, \text{ and }  \tau_i^k:=\inf\{n> \tau^k_{i-1}: X_{k+n} \text{ gets excited}\}, \text{ for $i\geq 1$}\,.
\end{align*}

For $k \geq 1$, let us also define a sequence of independent random variables $\{H_j^k\}_{j \ge 1}$, such that each $H_j^k$ has a geometric distribution with parameter $(k + j)^{-1/2}$ for all $j \ge 1$. Additionally, we introduce the following definitions   
\begin{align*}
& \mathcal{G}_m^k = \sum_{j = 1}^m H_j^k \,,
\\
& G^k_0\equiv 0 \text{ and }  G^k_i:= \inf\{\ell > G_{i-1}^k: Z_{\ell + k} =1\},  \text{ for $i\geq 1$}\,,
\end{align*}
where, $Z_\ell=\um_{\{U_\ell \leq  \ell^{-1/2}\}}$.
%\textcolor{red}{For each fixed $k \ge 1$ we define a coupling $\hat{\mathbb{P}}^k$ between the excited random walk $X$ and the random walk $Y$ with i.i.d. increments, defined such that $Y_0 = X_k$ and $Y_n := X_k + \sum_{i = k + 1}^{k + n} \xi_i$ for $n \ge 1$.} Suppose we have $\delta \in (0, 1/2)$. The first step in our proof is to establish the following
% \begin{equation*}
% \sum_{k = 1}^{\infty} \hat{\mathbb{P}} \left( e_k^{X} \um_{A_{k, \delta}} < e_k^{Y} \um_{A_{k, \delta}} \right) < \infty \,.   
% \end{equation*}
%
What we are after is to provide an upper bound for $\PP ( e_k^{X} \um_{A_{k, \delta}} < e_0^{Y^k} \um_{A_{k, \delta}} )$ for all $k \ge 1$. Recall that the process $X$ can be rewritten as in~\eqref{xn-incremento2}, i.e.
$$
X_0 = 0, \text{ and }\; X_n  = \sum_{i=1}^n \big( \um_{B_i^c} \xi_i + \um_{B_i} \gamma_i \big) \, , \ n\ge 1\,,
$$
where $B_i:= E_{i-1}^c \cap \{U_i \le i^{-1/2}\}$ for any $i \ge 1$. Then, we have that
\begin{equation*}
\begin{split}
& \mathbb{P}( e^{X}_k \um_{A_{k,\delta}} < e^{Y^k}_0 \um_{A_{k, \delta}})= \mathbb{P}(e^{X}_k=0, e^{Y^k}_0 =1, A_{k, \delta}) 
\\
& = \mathbb{P}(\nu^{X}_k < +\infty, e^{X}_k=0, e^{Y^k}_0 = 1, A_{k,\delta}) 
\\
& = \sum_{m=1}^{\infty} \mathbb{P}\left( \tau_{m}^k < \nu^{X}_k \leq \tau_{m+1}^k, e^{X}_k=0, e^{Y^k}_0 =1, A_{k, \delta} \right)
\\
& \le \sum_{m=1}^{\infty} \mathbb{P}\Big(\tau_{m}^k < \nu^{X}_k \leq \tau_{m+1}^k, \sum_{j = k+1}^{k+\nu^{X}_k} \left( \um_{B^{c}_j} \xi_j + \um_{B_j} \gamma_j \right) = 0, \sum_{j = k+1}^{k+\nu^{X}_k} \xi_j\neq 0, A_{k, \delta} \Big)\,.
\end{split}
\end{equation*}

It is important to notice that $\tau^k_i\geq G^k_i$ for all $i\geq 0$. Thus, we have that

% \textcolor{brown}{leonel: acho que o evento $A_{k,\delta}$ sobra nas fórmulas a seguir, uma vez que temos $G_m^k = l  + \lfloor k^{\delta}\rfloor$}\comu{concordo.}

\begin{equation*}
\begin{split}
& \mathbb{P}  ( e^{X}_k \um_{A_{k,\delta}} < e^{Y^k}_0 \um_{A_{k, \delta}})
\\
& \le \sum_{m=1}^{\infty} \sum_{\ell = m}^{\infty} \mathbb{P}\Big( G_m^k < \nu^{X}_k, \sum_{j=k+1}^{k+\nu^{X}_k} \xi_j = \sum_{j=1}^{m} \big( \xi_{\tau_j^k} - \gamma_{\tau_j^k} \big), G_m^k = \ell + \lfloor k^{\delta} \rfloor, A_{k, \delta} \Big)
\\
& \leq \sum_{m=1}^{\infty} \sum_{\ell=m}^\infty \sum_{n=\ell+\lfloor k^\delta\rfloor + 1}^{\infty} \mathbb{P}\Big( \nu^{X}_k=n, \sum_{j=k+1}^{k+n} \xi_j = \sum_{j=1}^{m} \big( \xi_{\tau_j^k} - \gamma_{\tau_j^k} \big), G_m^k = \ell + \lfloor k^\delta\rfloor \Big) 
\\
& \leq \sum_{m=1}^{\infty} \sum_{\ell=m}^\infty \sum_{n=\ell+\lfloor k^\delta\rfloor + 1}^{\infty} \mathbb{P}\Big(\sum_{j=k+1}^{k+n}\xi_j = \sum_{j=1}^{m}\big(\xi_{\tau_j^k} - \gamma_{\tau_j^k}\big), G_m^k = \ell + \lfloor k^\delta\rfloor \Big) \,.
\end{split}    
\end{equation*}

Let $\alpha \in (1/2,1)$ be a parameter to be determined later and let $\mu$ denote the mean vector of $\gamma$. For $m\geq 1$,  let us define 
\[
D_m:= \left\{x+y: x \in B(0, 2Km^\alpha) \text{ and } y \in B(m\mu, 2Km^\alpha)\right\}\,,
\]
where   $B(a, R):=\{x\in \mathbb{Z}^d: \Vert x-a\Vert^2\leq R\}$ with  $a \in \mathbb{Z}^d$ and $R \in [0,+\infty)$. 
Then we have that
\begin{equation}\label{eq:decomposition}
\begin{split}
& \mathbb{P}\Big(\sum_{j=k+1}^{k+n}\xi_j = \sum_{j=1}^{m} \big( \xi_{\tau_j^k} - \gamma_{\tau_j^k}\big), G_m^k = \ell + \lfloor k^\delta\rfloor \Big) 
=  
\\
& \sum_{z \in D_m} \mathbb{P}\Big(
%\nu^{\rm ERW}_k=n,
\sum_{j=k+1}^{k+n}\xi_j =z, \sum_{j=1}^{m} \big( \xi_{\tau_j^k} - \gamma_{\tau_j^k} \big) = z, G_m^k = \ell + \lfloor k^\delta\rfloor  \Big)
\\
&+ \sum_{z \in D^\complement_m} \mathbb{P} \Big(
%\nu^{\rm ERW}_k=n, 
\sum_{j=k+1}^{k+n}\xi_j = z, \sum_{j=1}^{m} \big( \xi_{\tau_j^k} - \gamma_{\tau_j^k} \big) = z, G_m^k = \ell + \lfloor k^\delta\rfloor \Big)\,.    
\end{split}
\end{equation}

We will analyze the two terms on the RHS of ~\eqref{eq:decomposition} separately. For the  first sum portion it holds that
\begin{equation}\label{eq_sum1<1}
\begin{split}
\sum_{z \in D_m} & \mathbb{P} \Big(
\sum_{j=k+1}^{k+n} \xi_j = z, \sum_{j=1}^{m} \big( \xi_{\tau_j^k} - \gamma_{\tau_j^k} \big) = z, G_m^k = \ell + \lfloor k^\delta\rfloor \Big)
\\
& \leq \sum_{z \in D_m} \mathbb{P}\Big(
\sum_{j=k+1}^{k+n}\xi_j = z \Big) \mathbb{P} ( G_m^k = \ell + \lfloor k^\delta\rfloor) 
% \\
% & \leq |D_m| \frac{C_d}{n^{d/2}} \hat{\mathbb{P}}(A_k, G_m^k = \ell + \lfloor k^\delta\rfloor) 
\\
& \leq (4Km^\alpha)^d \frac{C_d}{n^{d/2}} \mathbb{P}( G_m^k = \ell + \lfloor k^\delta\rfloor)\,,    
\end{split}    
\end{equation}
where, in the second inequality in~\eqref{eq_sum1<1} we use the fact that $|D_m| \le (4Km^{\alpha})^d$ and the Local Central Limit Theorem 
% (see inequality (2.4) in \cite[Theorem 2.1.1]{lawler2010random} and commentary in page 24)  
see, e.g., Inequality (2.8) in \cite[Theorem 2.1.3]{lawler2010random}  and the fact that $\Bar{p}_n(x) \le C_d n^{-d/2}$, with $C_d$ a positive constant. For the second term in~\eqref{eq:decomposition} we have
\begin{equation}\label{eq_sum<2}
\begin{split}
&\sum_{z \in D^c_m} \mathbb{P}\Big(
%\nu^{\rm ERW}_k=n, 
\sum_{j=k+1}^{k+n} \xi_j = z, \sum_{j=1}^{m} \big( \xi_{\tau_j^k} - \gamma_{\tau_j^k} \big) = z, G_m^k = \ell + \lfloor k^\delta\rfloor \Big)
\\
& \le \sum_{z \in D^c_m \cap B(0,2Km)} \mathbb{P}\Big(
%\nu^{\rm ERW}_k=n, 
\sum_{j=k+1}^{k+n} \xi_j = z,\sum_{j=1}^{m}\big( \xi_{\tau_j^k} - \gamma_{\tau_j^k} \big) = z \Big) %, A_k, G_m^k = \ell + \lfloor k^\delta\rfloor \right)
\\
&\leq \mathbb{P}\Big(
%\nu^{\rm ERW}_k=n, 
\sum_{j=k+1}^{k+n} \xi_j \in B(0,2Km) \setminus D_m, \sum_{j=1}^{m} \big( \xi_{\tau_j^k} - \gamma_{\tau_j^k} \big) \in B(0,2Km) \setminus D_m \Big) %, A_k, G^k_m = \ell + \lfloor k^\delta\rfloor \right) 
\\
&\leq \mathbb{P}\Big(
%\nu^{\rm ERW}_k=n, 
\sum_{j=k+1}^{k+n} \xi_j \in B(0,2Km) \setminus D_m \Big)^{1/2} \mathbb{P}\Big( \sum_{j=1}^{m} \big( \xi_{\tau_j^k} - \gamma_{\tau_j^k} \big) \in B(0,2Km) \setminus D_m \Big)^{1/2} \,, \end{split}    
\end{equation}
where, in the last  inequality above, we used Cauchy-Schwarz inequality. Regarding the first term in~\eqref{eq_sum<2}, by using again the Local Central Limit Theorem, it holds that 
\begin{equation}\label{eq_sum<2.1}
\begin{split}
\mathbb{P}\Big(
%\nu^{\rm ERW}_k=n, 
\sum_{j=k+1}^{k+n}\xi_j \in B(0,2Km) \setminus D_m \Big) & = \sum_{z \in D^c_m \cap B(0,2Km)} \mathbb{P}\Big(
%\nu^{\rm ERW}_k=n, 
\sum_{j=k+1}^{k+n} \xi_j = z\Big)
\\
& \leq \frac{C_d}{n^{d/2}} |B(0,2Km)|\,.
\end{split}
\end{equation}
For the second term in~\eqref{eq_sum<2}, let us define $F := B(0,2Km) \setminus D_m$, $G := B(0,2Km) \setminus B(0,2Km^{\alpha})$ and $A := B(0, 2Km) \setminus B(m\mu, 2Km^{\alpha})$. We then  obtain
\begin{equation}\label{eq_sum<2.2}
\begin{split}
& \mathbb{P}\Big(\sum_{j=1}^{m} \big(\xi_{\tau_j^k} - \gamma_{\tau_j^k} \big) \in F \Big) \leq \mathbb{P}\Big( \sum_{j=1}^{m} \xi_{\tau^k_j} \in G \Big)  + \mathbb{P}\Big(\sum_{j=1}^{m} \gamma_{\tau^k_j} \in A \Big) %\Big| A_k, G^k_m = \ell + \lfloor k^\delta\rfloor \right)
\\
& = \mathbb{P}\Big(\sum_{j=1}^{m}\xi_{j} \in G \Big) + \hat{\mathbb{P}}\Big(\sum_{j=1}^{m}\gamma_{j} \in A \Big) 
\\
& = \mathbb{P}\Big(\sum_{j=1}^{m}\xi_{j} \in G \Big) + \mathbb{P}\Big( \sum_{j=1}^{m}(\gamma_{j}-\mu) \in B(-m\mu,2Km) \setminus  B(0,2Km^\alpha)\Big) 
% \\
% & \leq 2|B(0,2Km)| \Hat{C}_{d,K} \Big[ e^{-2dK^2m^{2\alpha-1}}\Big(\frac{1}{m^{\frac{d}{2}}} + m^{\frac{7d-2}{2}} + \frac{1}{m^{\frac{d+2}{2}}} \Big)  + \frac{1}{m^{\frac{9d-1}{2}}} \Big]
% \\
% & \leq 2|B(0,2Km)| \Hat{C}_{d,K} \Big(m^{4d} e^{-2dK^2m^{2\alpha-1}} + \frac{1}{m^{\frac{9d-1}{2}}} \Big)\,. 
\\
& 
\leq 2|B(0,2Km)| \Hat{C}_{d,K} \Big[ e^{-2d\delta K^2m^{2\alpha-1}}\Big(\frac{1}{m^{\frac{d}{2}}} + \frac{(2Km)^{8d}}{m^{4d}m^{\frac{d+2}{2}}} + \frac{1}{m^{\frac{d+2}{2}}} \Big)  + \frac{1}{m^{\frac{9d-1}{2}}} \Big]
\\
& 
\leq 2|B(0,2Km)| \widetilde{C}_{d,K} \Big(m^{4d} e^{-2d \delta K^2m^{2\alpha-1}} + \frac{1}{m^{\frac{9d-1}{2}}} \Big)\,. 
\end{split}    
\end{equation}
In~\eqref{eq_sum<2.2}, we applied Lemma~\ref{lem: iid} to achieve the first equality. For the first inequality, we utilized the Local Central Limit Theorem see, e.g., Inequality (2.8) in \cite[Theorem 2.1.3]{lawler2010random} with $k=8d$
 and the fact that $\Bar{p}_n(x) \le C_d n^{-d/2}\exp(-||x||^2 \delta d/2n)$), with $\delta >0$ and the constant $C_d$ depending on the covariance matrix of $\xi$ (see, e.g., Equations~2.2 and 1.1 and Proposition~1.1.1 (c) in \cite{lawler2010random}).  
 %along with the observation that $(2Km^{\alpha})^2 > m$, which follows since, by definition, $\alpha > 1/2$. The final inequality is derived from the fact that $\alpha < 1$. 
%
Thus, using~\eqref{eq_sum<2.1} and~\eqref{eq_sum<2.2} we obtain that the rightmost term in~\eqref{eq_sum<2} is bounded above by 
\begin{equation}\label{eq:sum<2.3}
\begin{split}
%\sum_{z \in D^\complement_m} &\hat{\mathbb{P}}\left(
%\nu^{\rm ERW}_k=n, 
%\sum_{j=k+1}^{k+n}\xi_j =z, \sum_{j=1}^{m}\left(\xi_{\tau_j^k} - \gamma_{\tau_j^k}\right)=z \right) %, A_k, G_m^k = \ell + \lfloor k^\delta\rfloor  \right)
%\\
\Big[ 2 & |B(0,2Km)| \widetilde{C}_{d,K} \Big(m^{4d} e^{-2d\delta K^2m^{2\alpha-1}} + m^{-\frac{9d-1}{2}} \Big) \Big]^{1/2} \Big[ \frac{C_d}{n^{d/2}} |B(0,2Km)|\Big]^{1/2}
%\Big(4 & |B(0,2Km)|  \frac{C_d^2}{n^{d/2}}\Big)^{1/2} \Big( 2m^{3d+9} e^{-\frac{d(2Km^\alpha)^2}{2m}} + m^{-(d+8)}\Big)^{1/2} %\hat{\mathbb{P}}\left(A_k, G_m^k = \ell + \lfloor k^\delta\rfloor  \right)
\\
& \le \frac{m^d\Tilde{C}_{d,K}}{n^{d/4}} \Big(m^{4d} e^{-2d\delta K^2m^{2\alpha-1}} + m^{-\frac{9d-1}{2}} \Big)^{1/2} 
\\
& \le \frac{m^d\Tilde{C}_{d,K}}{n^{d/4}} \Big(m^{2d} e^{-d\delta K^2 m^{2\alpha-1}} + m^{-\frac{9d-1}{4}} \Big)\,.
%&\leq 2(2Km)^{d/2} \frac{C_d}{n^{d/4}} \Big( 2^{\frac{1}{2}}m^{\frac{3}{2}(d+3)} e^{-\frac{d(2Km^\alpha)^2}{4m}} + m^{-\frac{1}{2}(d+8)}\Big) \,.
% \hat{\mathbb{P}}\left(A_k, G_m^k = \ell + \lfloor k^\delta\rfloor  \right) 
% \\
% &=2^{1/2} (2Km)^d \left(\frac{C_d}{m^{d/2}}\right)^{1/2}  e^{-\frac{d(2Km^\alpha)^2}{4m}} \left(\frac{C_d}{n^{d/2}}\right)^{1/2} 
% %\hat{\mathbb{P}}\left(A_k, G_m^k = \ell + \lfloor k^\delta\rfloor  \right) 
% \,.
\end{split}
\end{equation}
Overall, by~\eqref{eq_sum1<1} and~\eqref{eq:sum<2.3} we obtain that

\begin{equation*}
\begin{split}
& \hat{\mathbb{P}}  ( e^{X}_k \um_{A_{k,\delta}} < e^{Y}_k \um_{A_{k, \delta}})  \leq\underbrace{\sum_{m=1}^{\infty} \sum_{\ell=m}^\infty  \sum_{n=\ell+ \lfloor k^\delta\rfloor + 1}^{\infty} (4Km^\alpha)^d \frac{C_d}{n^{d/2}} \mathbb{P}(G_m^k = \ell + \lfloor k^\delta\rfloor) }_{SUM_1}
\\
&+\underbrace{\sum_{m=1}^{\infty} \sum_{\ell = m}^\infty \sum_{n = \ell + \lfloor k^\delta\rfloor + 1}^{\infty} \frac{m^d\Tilde{C}_{d,K}}{n^{d/4}} \Big(m^{2d} e^{-d\delta K^2 m^{2\alpha-1}} + m^{-\frac{9d-1}{4}} \Big) }_{SUM_2}\,. %\hat{\mathbb{P}}\left(A_k, G_m^k = \ell + \lfloor k^\delta\rfloor \right)}_{SUM_2} \,. 
\end{split}
\end{equation*}

We will prove that $ \sum_{k \ge 1}\mathbb{P} ( e^{X}_k \um_{A_{k,\delta}} < e^{Y^k}_0 \um_{A_{k, \delta}}) < \infty$ by demonstrating that both $SUM_1$ and $SUM_2$ are summable over $k$. We begin with $SUM_1$  for which we obtain that 
\begin{equation*}
\begin{split}
& SUM_1 = \sum_{m=1}^{\infty} (4Km^\alpha)^d \sum_{\ell=m}^\infty \mathbb{P}(G_m^k = \ell + \lfloor k^\delta\rfloor) \sum_{n=\ell+ \lfloor k^\delta\rfloor + 1}^{\infty}  \frac{C_d}{n^{d/2}} 
\\
& \le C_d \sum_{m=1}^{\infty} (4Km^\alpha)^d \sum_{\ell = m}^\infty \mathbb{P}( G_m^k = \ell + \lfloor k^\delta\rfloor) \int_{\ell + \lfloor k^\delta\rfloor}^\infty  \frac{1}{x^{d/2}}dx
\\
& \overset{d>2}{=} \frac{C_d (4K)^d}{d/2-1}\sum_{m=1}^{\infty}    m^{\alpha d}   \sum_{\ell=m}^\infty \mathbb{P} (G_m^k = \ell + \lfloor k^\delta\rfloor) \frac{1}{(\ell + \lfloor k^\delta\rfloor)^{d/2-1}}
\\
& \leq C_{d,K} \sum_{m=1}^{\infty}    m^{\alpha d}   \sum_{\ell=m}^\infty \mathbb{P}(G_m^k = \ell) \frac{1}{(\ell + \lfloor k^\delta\rfloor)^{d/2-1}} 
\\
& \leq C_{d,K} \sum_{m=1}^{\infty}    m^{\alpha d} \,\mathbb{E}\Big[ \frac{1}{( G_m^k + \lfloor k^\delta\rfloor)^{d/2-1}} \Big] \,.
\end{split}    
\end{equation*}

We first note that $G^k_m \ge m$ and stochastically dominates $\mathcal{G}_m^k$, and recall that $ \mathcal{G}_m^k = \sum_{j=1}^m H^k_j$, where $\{H^k_j\}_{j\geq 1}$ are independent random variables with geometric distribution with parameter $(k+j)^{-1/2}$ for each $j \ge 1$. Now, we will compute an upper bound for $\mathbb{E}[(G_m^k + \lfloor k^\delta \rfloor)^{1-d/2}]$. 
\begin{equation*}
\begin{split}
&\mathbb{E}[(G_m^k + \lfloor k^\delta \rfloor)^{1-d/2}] = 
\\
& \mathbb{E}\Big[ \frac{1}{( G_m^k + \lfloor k^\delta\rfloor)^{d/2-1}} ; G_m^k < \frac{m^{\frac{3}{2}}}{6} \Big] +  \mathbb{E}\Big[ \frac{1}{( G_m^k + \lfloor k^\delta\rfloor)^{d/2-1}} ; G_m^k \ge \frac{m^{\frac{3}{2}}}{6} \Big]
\\
& \leq \frac{1}{( m + \lfloor k^\delta\rfloor)^{d/2-1}} \mathbb{P} \Big[ G_m^k < \frac{m^{\frac{3}{2}}}{6} \Big] + \frac{6^{d/2-1}}{(m^{3/2} + 6 \lfloor k^\delta\rfloor)^{d/2-1}}
\\
& \leq \frac{1}{( m + \lfloor k^\delta\rfloor)^{d/2-1}} \mathbb{P} \Big[ \mathcal{G}_m^k < \frac{m^{\frac{3}{2}}}{6} \Big] + \frac{6^{d/2-1}}{(m^{3/2} + 6 \lfloor k^\delta\rfloor)^{d/2-1} }\,.
\end{split}
\end{equation*}
%
Then, we use Lemma~\ref{lem:geo_sum_bound} for $\PP[\mathcal{G}_m^k < 6^{-1}m^{\frac{3}{2}}]$ with $\theta = 5$ and choose $\alpha=1/2 + 1/(cd)>1/2$, where $c$ is a sufficiently large positive constant, to obtain  the following 
\begin{equation}\label{eq:sum1_2}
\begin{split}
& SUM_1  \leq C_{d,K} \Big(\sum_{m=1}^{\infty}    \frac{m^{d/2 + 1/c - 5}}{(m+\lfloor k^\delta\rfloor)^{d/2-1}} + 6^{d/2-1} \sum_{m = 1}^{\infty} \frac{m^{d/2 + 1/c} }{(m^{3/2} + 6 \lfloor k^\delta\rfloor)^{d/2-1}}\Big)
\\
& \leq C_{d,K}' \Big(\sum_{m=1+\lfloor k \rfloor^\delta}^{\infty}    \frac{m^{d/2 + 1/c - 5}}{m^{d/2-1}} +  \int_0^{\infty} \frac{(u^{2/3})^{d/2 + 1/c}}{(u + 6 \lfloor k^{\delta} \rfloor)^{d/2 - 1}} \times \frac{2}{3} u^{-1/2} du \Big)
\\
& \leq C_{d,K}' \Big(\int_{\lfloor k^\delta \rfloor}^{\infty}\frac{1}{x^{4-1/c}}dx + 
\int_0^{\infty} \frac{(u + 6\lfloor k^{\delta} \rfloor)^{2/3(d/2 + 1/c) - 1/2}}{(u + 6 \lfloor k^{\delta} \rfloor)^{d/2 - 1}} du \Big)
\\
& \le C_{d,K,c} \Big( \frac{1}{(\lfloor k^\delta \rfloor)^{(3 -1/c)}} + 
\int_{6\lfloor k^{\delta} \rfloor}^{\infty} \frac{w^{2/3(d/2 + 1/c) - 1/2}}{w^{d/2 - 1}} dw \Big)
\\
& \le C_{d,K,c}'' \Big( \frac{1}{(\lfloor k^\delta \rfloor)^{(3 -1/c)}} + \frac{1}{\lfloor k^{\delta} \rfloor^{d/6 - 3/2 -2/3c}} \Big)\,,
\end{split}    
\end{equation}
where, in the second inequality of~\eqref{eq:sum1_2}, we used the change of variable  $u = m^{2/3}$ and, in the fifth inequality, we again changed variable with $w = u + 6\lfloor k^{\delta} \rfloor$. 

Now we will compute an upper bound for $SUM_2$.
\begin{equation}\label{eq:sum2_1}
\begin{split}
& SUM_2 = \Tilde{C}_{d,K} \sum_{m=1}^{\infty} m^d \Big( m^{2d} e^{-d\delta K^2 m^{2\alpha-1}} + \frac{1}{m^{\frac{9d-1}{4}}} \Big) \sum_{\ell = m}^\infty \sum_{n = \ell + \lfloor k^\delta\rfloor + 1}^{\infty} n^{-d/4}
\\
& \le \Tilde{C}_{d,K} \sum_{m=1}^{\infty} m^d \Big( m^{2d} e^{-d\delta K^2 m^{2\alpha-1}} + \frac{1}{m^{\frac{9d-1}{4}}} \Big) 
\sum_{\ell = m}^\infty \frac{1}{(\lfloor k^\delta\rfloor + \ell)^{\frac{d}{4}-1}}
\\
& \le \Tilde{C}_{d,K} \sum_{m=1}^{\infty} m^d \Big( m^{2d} e^{-d\delta K^2 m^{2\alpha-1}} + \frac{1}{m^{\frac{9d-1}{4}}} \Big) 
\sum_{\ell = \lfloor k^\delta\rfloor + m}^\infty \frac{1}{\ell^{\frac{d}{4}-1}}
\\
& \le \Tilde{C}_{d,K} \sum_{m=1}^{\infty} m^d \Big( m^{2d} e^{-d\delta K^2 m^{2\alpha-1}} + \frac{1}{m^{\frac{9d-1}{4}}} \Big)  \frac{1}{(m+ \lfloor k^\delta\rfloor -1)^{\frac{d}{4}-2}}
\\
& \le \frac{C_{d,K}'}{\lfloor k^\delta\rfloor^{d/4-2}} \sum_{m=1}^{\infty} m^d \Big( m^{2d} e^{-d\delta K^2 m^{2\alpha-1}} + \frac{1}{m^{\frac{9d-1}{4}}} \Big) \le \frac{C_{d,K}''}{\lfloor k^\delta\rfloor^{d/4-2}} \,,
\end{split}    
\end{equation}
where, in the last inequality of~\eqref{eq:sum2_1} we used that $\alpha > 1/2$. 
%
Now we use the bounds in~\eqref{eq:sum1_2} and~\eqref{eq:sum2_1} and, since $d \ge 22$, we finally obtain that
\begin{equation*}
\begin{split}
& \sum_{k=1}^\infty \mathbb{P}  ( e^{X}_k \um_{A_{k,\delta}} < e^{Y^k}_0 \um_{A_{k, \delta}}) 
\\
& \le \sum_{k = 1}^\infty \left(C_{d,K,c}'' \Big( \frac{1}{(\lfloor k^\delta \rfloor)^{(3 -1/c)}} + \frac{1}{\lfloor k^{\delta} \rfloor^{d/6 - 3/2 -2/3c}} \Big) +  \frac{C_{d,K}''}{\lfloor k^\delta\rfloor^{d/4-2}} \right)< \infty \,.
\end{split}
\end{equation*}
\end{proof}



%{\color{red} I will state the  result as a lemma.... something like
%\begin{lemma}\label{compXY}
%Let  $X$ be a $p_n$-\Nametwo{} in direction $\ell$, on $\ZZ^d$ with $d\geq 22$, $p_n= \mathcal{C} n^{-\beta} \wedge 1$ and $\beta \ge 1/2$ and let $Y$ denote a random walk on $\ZZ^d$ with i.i.d. increments given by  $\{\xi_i\}_i$. For $\delta \in (0, 1/2)$ and for every fixed $k\geq 1$ there exists a coupling  $\hat{\mathbb{P}}_k$ between $X$ and  $Y$ such that 
%\begin{equation*}
%\sum_{k = 1}^{\infty} \hat{\mathbb{P}}_k \left( e_k^{X} \um_{A_{k, \delta}} < e_k^{Y} \um_{A_{k, \delta}} \right) < \infty \,.   
%\end{equation*}
%\end{lemma}

%\begin{proof}[Proof of Proposition~\ref{prop:RangeERW_lower}]
%Note that $|\Rr_n^X| \ge |\mathcal{D}_n^X|$, since if $x \in \mathcal{D}_n^X$ then $x \in \Rr_n^X$ (and it is possible for a $x \in \Rr_n^X$ to be  revisited later on, thus implying $x \notin \mathcal{D}_n^X $). Therefore, it holds that 
%\begin{align*}
%\liminf_{n \to \infty} \frac{|\Rr_n^X|}{n} \ge \liminf_{n \to \infty} \frac{|\mathcal{D}_n^X|}{n} = \liminf_{n \to \infty} \frac{1}{n}\sum_{k=1}^n e_k^X \um_{A_{k,\delta}} + \liminf_{n \to \infty} \frac{1}{n}\sum_{k=1}^n e_k^X \um_{A_{k,\delta}^c}\,.  
%\end{align*}
%For $\delta \in (0, 1/2)$, Lemma~\ref{lemma_1} implies that $\liminf_{n \to \infty} \frac{1}{n}\sum_{k=1}^n e_k^X \um_{A_{k,\delta}^c}=0$ a.s., whereas from Lemma~\ref{compXY} together with Borel-Cantelli's Lemma we conclude that 
%\begin{equation*}
%\liminf_{n \to \infty} \frac{1}{n} \sum_{k = 1}^n e_k^X \um_{A_{k, \delta}} \ge \liminf_{n \to \infty} \frac{1}{n} \sum_{k = 1}^n e_k^Y \um_{A_{k, \delta}} \,, 
%\end{equation*}
%almost surely \texttt{with respect to which measure....}. Then using again Lemma~\ref{lemma_1} and the known fact that $\lim_{n \to \infty} \frac{|\mathcal{D}_n^Y|}{n} = \pi_d$ (see e.g.~\cite{spitzer2001principles} page 39) we obtain that 
%\[
%\liminf_{n \to \infty} \frac{1}{n} \sum_{k = 1}^n e_k^Y \um_{A_{k, \delta}} = \pi_d \,, \text{a.s..}
%\]
%\end{proof}
%}

\medskip 
\paragraph{\bf Acknowledgements:} We would like to thank Augusto Quadros Teixeira, Christophe Gallesco, Guilherme Ost, Luiz Renato Fontes and Maria Eulalia Vares for useful comments and suggestions. G.I.~was supported by FAPERJ (grant E-26/210.516/2024).

%\medskip

%\paragraph{\bf Conflict of interest:} The authors have no competing interests to declare that are relevant to the content of this article.

%\medskip

%\paragraph{\bf Data Availability:} Data sharing is not applicable to this article as no datasets were generated or analyzed during the current study.

%\medskip

\appendix

\setcounter{tocdepth}{\value{tocdepth}}%
  \addtocontents{toc}{\protect\setcounter{tocdepth}{1}}% 
  
\section{}\label{sec:appendixC}


% \begin{lemma}\label{lem: tight}
% Suppose that the sequences $X_{\cdot}^n$ and $Y_{\cdot}^n$ are tight processes in $C[0, T]$ for a $T > 0$. Then we have that the sequence $ (X_{\cdot}^n + Y_{\cdot}^n)$ is a tight process in $C[0, T]$.
% \end{lemma}

% \begin{proof}
% Let us denote the probability measure $P_n$ on $C[0, T]$ as the distribution of $ (X_{\cdot}^n + Y_{\cdot}^n)$. First we will prove that for each positive $\eta$, there exist an $a$ and an $n_0$ such that
% \begin{equation}\label{eq: ap0}
% P_n[f \in C[0, T]: |f(0)| \ge a] \le \eta \quad \text{for all } n \ge n_0 \,. 
% \end{equation}
% We denote the set $G_0 (a):=\{f \in C[0, T]: |f(0)| \ge a\}$. 

% Since $X_{\cdot}^n$ and $Y_{\cdot}^n$ are tight processes in $C[0, T]$, by Theorem 7.3 in~\cite{billingsley1999probability} there exist $a_x$, $a_y$, $n_0^x$ and $n_0^y$ such that
% \begin{equation}\label{eq: ap1}
% \begin{split}
% & \PP[|X_0^n| \ge a_x] \le \frac{\eta}{2} \quad \text{for all } n \ge n_0^x \quad \text{and}
% \\
% & \PP[|Y_0^n| \ge a_y] \le \frac{\eta}{2} \quad \text{for all } n \ge n_0^y \,.
% \end{split}    
% \end{equation}

% Now let us choose $a \ge a_x + a_y$. By triangle inequality and union bound  we have
% \begin{equation}\label{eq: ap2}
% \begin{split}
% P_n & [G_0(a)]  = \PP[|X_0^n + Y_0^n| \ge a] \le \PP[|X_0^n| + |Y_0^n| \ge a] 
% \\
% & \le \PP[\{|X_0^n| \ge a_x\} \cup \{|Y_0^n| \ge a_y\}] \le \PP[|X_0^n| \ge a_x] + \PP[|Y_0^n| \ge a_y] \,.
% \end{split}    
% \end{equation}

% Thus for a $n_0 = \max \{n_0^x, n_0^y\}$, by~\eqref{eq: ap1}, ~\eqref{eq: ap2}, we obtain~\eqref{eq: ap0}.

% We set $\omega_f (\delta)$ as the \textit{modulus of continuity} of an arbitrary function $f(\cdot)$ on $[0, T]$. Now we shall prove that for each positive $\varepsilon$ and $\eta$, there exist a $\delta \in (0, 1)$ and an $m_0$ such that
% \begin{equation}\label{eq: ap31}
% P_n[f \in C[0,T]: \omega_f (\delta) \ge \varepsilon] \le \eta \quad \text{for all } n \ge m_0 \,.    
% \end{equation}
% We denote the set $H_t(\varepsilon, \delta):= \{f \in C[0,T]: \omega_f (\delta) \ge \varepsilon \}$.


% Since $X_{\cdot}^n$ and $Y_{\cdot}^n$ are tight processes in $C[0, T]$, by Theorem 7.3 in~\cite{billingsley1999probability} there exist $\delta_x \in (0,1)$, $\delta_y \in (0,1)$, $m_0^x$ and $m_0^y$ such that
% \begin{equation}\label{eq: ap3}
% \begin{split}
% & \PP\left[\sup_{|s-t| \ge \delta_x } |X_s^n - X_t^n| \ge \frac{\varepsilon}{2} \right] \le \frac{\eta}{2} \quad \text{for all } n \ge m_0^x \quad \text{and}
% \\
% & \PP\left[\sup_{|s-t| \ge \delta_y } |Y_s^n - Y_t^n| \ge \frac{\varepsilon}{2}\right] \le \frac{\eta}{2} \quad \text{for all } n \ge m_0^y \,.
% \end{split}    
% \end{equation}

% Then we obtain by triangle inequality and union bound the following
% \begin{equation}\label{eq: ap4}
% \begin{split}
% P_n[H_t(\varepsilon,\delta)] & = \PP\left[\sup_{|s-t| \ge \delta} |X_s^n + Y_s^n - X_t^n -Y_t^n| \ge \varepsilon \right]
% \\
% & \le \PP\left[ \sup_{|s-t|\le \delta}|X_s^n - X_t^n| + \sup_{|s-t|\le \delta}|Y_s^n - Y_t^n| \ge \varepsilon \right]
% \\
% & \le \PP\left[ \left\{ \sup_{|s-t|\le \delta}|X_s^n - X_t^n| \le \frac{\varepsilon}{2} \right\} \bigcup  \left\{\sup_{|s-t|\le \delta}|Y_s^n - Y_t^n| \le \frac{\varepsilon}{2} \right\} \right]
% \\
% & \le \PP\left[ \sup_{|s-t|\le \delta}|X_s^n - X_t^n| \le \frac{\varepsilon}{2} \right] + \PP\left[\sup_{|s-t|\le \delta}|Y_s^n - Y_t^n| \le \frac{\varepsilon}{2} \right] \,.
% \end{split}    
% \end{equation}


% We choose now a $\delta= \min\{\delta_x, \delta_y\}$ and $m_0 = \max\{m_0^x , m_0^y \}$, thus by~\eqref{eq: ap3} and~\eqref{eq: ap4} we obtain~\eqref{eq: ap31}. Hence by Theorem 7.3 in~\cite{billingsley1999probability} we finish the proof.


 
% %Let us denote $A_X$, $A_Y$ and $A_{X+Y}$ as sets in $C[0,T]$ which contains the processes $X_{\cdot}^n$, $Y_{\cdot}^n$ and $(X_{\cdot}^n + Y_{\cdot}^n)$ respectively for all $n \ge 0$. 

% %Since the processes $X_{\cdot}^n$ and $Y_{\cdot}^n$ are tight by Prohorov's Theorem (see Theorem 5.1 in~\cite{billingsley1999probability}) the sets $A_X$ and $A_Y$ are relatively compacts in $C[0, T]$.

% %If we prove the set $A_{X+Y}$ is relatively compact in $C[0, T]$ by Prohorov's Theorem (see Theorem 5.2 in~\cite{billingsley1999probability}) we have the desired result. For that we will use the Arzelà-Ascoli Theorem (see Theorem 7.2 in~\cite{billingsley1999probability}).

% %By triangle inequality we have
% %\begin{equation}\label{eq: A_x+y}
% %\sup_{f \in A_{X+Y}} |f(0)| \le \sup_{f \in A_{X}} |f(0)| + \sup_{f \in A_{Y}} |f(0)| < \infty \,.  
% %\end{equation}
% %Since $A_X$ and $A_Y$ are relative compact in $C[0, T]$ we have the last inequality in~\eqref{eq: A_x+y}.

% %Let us denote $\omega_f (\delta)$ as the \textit{modulus of continuity} of an arbitrary function $f(\cdot)$ on $[0, T]$. Now one can see that for all $n \ge 0$
% %\begin{equation}\label{eq: modcon}
% %\begin{split}
% %\omega_{X_{\cdot}^n + Y_{\cdot}^n}(\delta) & = \sup_{|s-t|\le \delta} |X_s^n + Y_s^n - X_t^n -Y_t^n| 
% %\\
% %& = \sup_{|s-t|\le \delta} |(X_s^n - X_t^n) + (Y_s^n -Y_t^n)|
% %\\
% %& \le \sup_{|s-t|\le \delta}(|X_s^n - X_t^n|) + \sup_{|s-t|\le \delta}(|Y_s^n - Y_t^n|)
% %\\
% %& \le \omega_{X_{\cdot}^n}(\delta) + \omega_{ Y_{\cdot}^n}(\delta) \,.
% %\end{split}    
% %\end{equation}
% %The second inequality in~\eqref{eq: modcon} we apply triangle inequality.

% %By~\eqref{eq: modcon} and the fact that $A_X$ and $A_Y$ are relatively compacts in $C[0, T]$ we obtain
% %\begin{equation}\label{eq: A_x+y2}
% %\lim_{\delta \to 0} \sup_{f \in A_{X+Y}} \omega_f (\delta) \le \lim_{\delta \to 0} \left(\sup_{f \in A_X}\omega_f (\delta) + \sup_{f \in A_Y}\omega_f (\delta)\right) = 0 \,.     
% %\end{equation}

% %Since we have~\eqref{eq: A_x+y} and~\eqref{eq: A_x+y2} by the Arzelà-Ascoli Theorem (see Theorem 7.2 in~\cite{billingsley1999probability}), $A_{X+Y}$ is relatively compact in $C[0,T]$ and we finish the proof.
% \end{proof}

% \begin{lemma}\label{lem: convprob}
% Suppose that we have a process $Y_{\lfloor n \cdot \rfloor}$ that converges in probability to zero in the space $C_{\Rs^d}[0, T]$ with the uniform metric for all $T >0$. Then $Y_{\lfloor n \cdot \rfloor}$ converges in probability to zero in the space $C_{\Rs^d}[0, \infty)$ equipped with the following metric
% \begin{equation*}
%  \rho(f, g) := \sum_{k=1}^{\infty}\frac{1}{2^k} \sup_{0 \le t \le k}(||f(t) - g(t)|| \wedge 1) \,.   
% \end{equation*}
% \end{lemma}

% \begin{proof}
% Since $Y_{\lfloor n \cdot \rfloor}$ converges in probability to zero in $C_{\Rs^d}[0, T]$, we have
% \begin{equation}\label{eq: convprob}
%     \PP\left[ \sup_{0 \le t \le T}|| Y_{\lfloor nt \rfloor}|| > \delta \right] \to 0 \quad \text{as } n \to \infty \,,
% \end{equation}
% for any $\delta > 0$. 

% Now let $\varepsilon > 0$ and we choose a positive integer $N$ such that $\sum_{l \ge N} 2^{-l} < \varepsilon/2$. Thus we have the following
% \begin{equation}\label{eq: convprob2}
% \begin{split}
% & \PP\left[ \sum_{k=1}^{\infty} 2^{-k} \sup_{0 \le t \le k}(||Y_{\lfloor nt \rfloor}|| \wedge 1) > \varepsilon \right] =  
% \\
% & = \PP\left[ \sum_{k=1}^{N} 2^{-k} \sup_{0 \le t \le k}(||Y_{\lfloor nt \rfloor}|| \wedge 1) + \sum_{k=N+1}^{\infty} 2^{-k} \sup_{0 \le t \le k}(||Y_{\lfloor nt \rfloor}|| \wedge 1) > \varepsilon \right]
% \\
% & \le \PP\left[ \sum_{k=1}^{N} \sup_{0 \le t \le k}||Y_{\lfloor nt \rfloor}|| + \sum_{k=N+1}^{\infty} 2^{-k}  > \varepsilon \right] \,.
% \end{split} 
% \end{equation}

% Since we choose $N$ large enough by~\eqref{eq: convprob2} we obtain that
% \begin{equation}\label{eq: convprob3}
% \begin{split}
% & \PP\left[ \sum_{k=1}^{\infty}  \sup_{0 \le t \le k}(||Y_{\lfloor nt \rfloor}|| \wedge 1) > \varepsilon \right] 
% \\
% & \le  \PP\left[ \left\{\sum_{k=1}^{N}  \sup_{0 \le t \le k}||Y_{\lfloor nt \rfloor}|| > \frac{\varepsilon}{2} \right\} \bigcup \left\{ \sum_{k=N+1}^{\infty} 2^{-k}  > \frac{\varepsilon}{2} \right\}\right] 
% \\
% & \le \PP\left[ \sum_{k=1}^{N}  \sup_{0 \le t \le k}||Y_{\lfloor nt \rfloor}|| > \frac{\varepsilon}{2} \right] + \PP\left[ \sum_{k=N+1}^{\infty} 2^{-k}  > \frac{\varepsilon}{2} \right] 
% \\
% & \le \PP\left[ \bigcup_{k=1}^{N}  \sup_{0 \le t \le k}||Y_{\lfloor nt \rfloor}|| > \frac{\varepsilon}{2N} \right] \le  \sum_{k=1}^{N} \PP\left[ \sup_{0 \le t \le k}||Y_{\lfloor nt \rfloor}|| > \frac{\varepsilon}{2N} \right] \,.
% \end{split}    
% \end{equation}
% We have the third and last inequalities in~\eqref{eq: convprob3} by union bound. 

% Now one can see that all sum portions in the last inequality in~\eqref{eq: convprob3} go to zero as $n$ tends to infinity by~\eqref{eq: convprob}, ergo we have
% \begin{equation*}
% \PP\left[ \sum_{k=1}^{\infty} 2^{-k} \sup_{0 \le t \le k}(||Y_{\lfloor nt \rfloor}|| \wedge 1) > \varepsilon \right] \to 0 \quad \text{as } n \to \infty \,,    
% \end{equation*}
% for any $\varepsilon > 0$. Hence we obtain the desired result.
% \end{proof}

\begin{lemma}\label{lem: iid}
Let $\{\phi_n\}_{n \ge 1}$ be a sequence of i.i.d. random vectors on $\ZZ^d$, with $d \ge 2$ and $\{\kappa_n\}_{n \ge 1}$ an increasing sequence of $\FF_n$-predictable times defined on 
a probability space $(\Omega, \FF, \PP)$, where $\{\FF_n\}_{n\ge 1}$ is the filtration $\FF_0 = \{\emptyset, \Omega\}$, $\FF_n=\sigma(\phi_1, \dots, \phi_n)$, $n\ge 1$ and $\kappa_n < \infty$ for all $n$. Then we have that $\{\phi_{\kappa_n}\}_{n \ge 1}$ is i.i.d. and moreover $\phi_{\kappa_n}$ has the same distribution of $\phi_1$ for every $n \ge 1$. 
\end{lemma}
\begin{proof}
We begin showing that for any $n \ge 1$, $\phi_{\kappa_n}$ has the same distribution of $\phi_1$.
%
Let $A$ be a subset of   $\ZZ^d$ and fix a $j \ge 1$. Then we have that
\begin{equation}\label{eq: iid1}
\begin{split}
& \PP[\phi_{\kappa_j} \in A | \FF_{\kappa_{j}-1}] = \sum_{n=1}^{\infty} \PP[\{ \phi_{\kappa_j} \in A \} \cap \{\kappa_j = n\}| \FF_{\kappa_{j}-1}]
\\
& = \sum_{n=1}^{\infty} \PP[\phi_{\kappa_j} \in A |\{\kappa_j = n\}, \FF_{\kappa_{j}-1}] \PP[\kappa_j = n| \FF_{\kappa_{j}-1}] \\
& = \sum_{n=1}^{\infty} \PP[\phi_n \in A ]\PP[\kappa_j = n| \FF_{\kappa_{j}-1}] = \PP[\phi_1 \in A ] \underbrace{\sum_{n=1}^{\infty} \PP[\kappa_j = n| \FF_{\kappa_{j}-1}]}_{=1}\,.
\end{split}    
\end{equation}
The third equality in~\eqref{eq: iid1} follows from  the fact that $\{\kappa_n\}_{n \ge 1}$ is an increasing sequence of $\FF_n$-predictable times, $\{\phi_n\}_{n \ge 1}$ is i.i.d. and so $\phi_n$ is independent of $\FF_{n-1}$ for all $n \ge 1$.
%
It remains to prove independence. We will only prove pairwise independence, but it is straightforward to generalize the proof by induction and we leave the details to the reader.  For $B$ and $D$ subsets of $\ZZ^d$ and $j, i \in \mathbb{N}$ such that $j > i$ we have that 
\begin{equation}
\label{eq: iid2}
\begin{split}
\PP[&\{\phi_{\kappa_i} \in B\} \cap \{\phi_{\kappa_j} \in D\}] = 
\\ 
&= \sum_{n=1}^{\infty} \sum_{m > n} \PP[\{\{\phi_{\kappa_i} \in B\} \cap \{\kappa_i = n\}\} \cap \{\{\phi_{\kappa_j} \in D\} \cap \{\kappa_j = m\}\}] 
\\
&= \sum_{n=1}^{\infty} \sum_{m > n} \PP[\phi_{\kappa_j} \in D | \kappa_i = n, \kappa_j = m, \phi_{n} \in B]\PP[\kappa_i = n, \kappa_j = m, \phi_{\kappa_i} \in B] 
\\
&= \sum_{n=1}^{\infty} \sum_{m > n} \PP[\phi_m \in D]\PP[\kappa_i = n, \kappa_j = m, \phi_{\kappa_i} \in B]\,.
\end{split}
\end{equation}
The last equality in~\eqref{eq: iid2} follows from the fact that the event $\{\kappa_i = n, \kappa_j = m, \phi_{n} \in B\}$, for $m>n$, is $\FF_{m-1}$ measurable, since $\{\kappa_n\}_{n \ge 1}$ is an increasing sequence of $\FF_n$-predictable times.

Since the sequence $\{ \phi_n\}_{n \ge 1}$ is i.i.d., we can continue the computation in~\eqref{eq: iid2} and obtain that
\begin{equation*}
\begin{split}
\PP[\{\phi_{\kappa_i} \in B\} \cap \{\phi_{\kappa_j} \in D\}] & = \PP[\phi_1 \in D] \sum_{n=1}^{\infty} \sum_{m \ge n} \PP[\kappa_j = m,\kappa_i = n, \phi_{\kappa_i} \in B]
\\
& = \PP[\phi_1 \in D] \sum_{n=1}^{\infty}  \PP[\kappa_i = n, \phi_{\kappa_i} \in B] 
\\
& = \PP[\phi_1 \in D] \sum_{n=1}^{\infty} \PP[\phi_{n} \in B]\PP[\kappa_i = n]
\\
& = \PP[\phi_1 \in D] \PP[\phi_1 \in B] = \PP[\phi_{\kappa_i} \in D]\PP[\phi_{\kappa_j} \in B]\,,
\end{split}    
\end{equation*}
concluding the proof.
\end{proof}




Now we present the proof of Lemma~\ref{lemma_1}.
\begin{proof}[Proof of Lemma~\ref{lemma_1}] 
For simplicity, we denote $b_k = \lfloor k^{\delta}\rfloor$. From the definition of the set $A_{k, \delta, \beta}$ in \eqref{eq:set_A} we have that  there exists a positive constant $c_1$ such that
\begin{equation}\label{eq:n1}
\mathbb{P} (A_{k, \delta, \beta}^c) \leq 1 - \Big(1 - \frac{1}{\sqrt{k}}\Big)^{b_k} \le c_1 \frac{b_k}{\sqrt{k}} \,.
\end{equation}
Hence,  $\lim_{n\to\infty}\frac{1}{n}\sum_{k = 1}^{n}\mathbb{P}(A_{k,\delta, \beta}^c) = 0$ for all $\delta \in (0,1/2)$ and all $\beta \geq 1/2$.
%
Then, to prove the result it is enough to show that 
\begin{equation*}%\label{eq_1}
\lim_{n\to\infty} \frac{1}{n} \sum_{k = 1}^n \Big(\um_{A_{k, \delta, \beta}^c} - \mathbb{P}(A_{k, \delta, \beta}^c)\Big) = 0, \, \text{ a.s..}
\end{equation*}
By Borel-Cantelli's Lemma the latter holds if can show that 
\begin{equation}\label{eq_2}
\sum_{n\ge 1}\mathbb{E}\left[\left(\frac{1}{n}\sum_{k = 1}^n\Big(\um_{A_{k, \delta,\beta}^c} - \mathbb{P}\left(A_{k, \delta, \beta}^c\right)\Big)\right)^2\right] < \infty \,,
\end{equation}
holds for any $\delta \in (0,1/2)$ and $\beta \geq 1/2$. 
%
To avoid clutter, henceforth we will denote $A_k := A_{k, \delta, \beta}$.  Note that
\begin{equation}\label{eq:divide}
\begin{split}
\mathbb{E}\Big[\Big(\frac{1}{n}\sum_{k = 1}^{n}[\um_{A_k^c} - \mathbb{P}(A_k^c)]\Big)^2\Big] &=
 \frac{1}{n^2}\sum_{k=1}^{n}\mathbb{E}\left[(\um_{A_k^c} - \mathbb{P}(A_k^c))^2\right]
 \\
+& \frac{2}{n^2}\sum_{k = 1}^{n - 1}\sum_{m = k + 1}^{n}\mathbb{E}\left[(\um_{A_k^c} - \mathbb{P}(A_k^c))(\um_{A_m^c} - \mathbb{P}(A_m^c))\right] \,.
\end{split}
\end{equation}
For the first on the RHS of \eqref{eq:divide}, it holds that 
\begin{equation}\label{eq:n2}
\begin{split}
& \frac{1}{n^2}\sum_{k=1}^{n}\mathbb{E}\left[\left(\um_{A_k^c} - \mathbb{P}(A_k^c)\right)^2\right]  \le \frac{1}{n^2}\sum_{k = 1}^{n}\mathbb{P}(A_k^c) 
% =\frac{1}{n^2}\sum_{k = 1}^n\left[1 - \left(1 - \frac{1}{\sqrt{k}}\right)^{b_k}\right] 
% \\
% & 
\le \frac{c_1}{n^2}\sum_{k = 1}^n\frac{k^{\delta}}{\sqrt{k}} \le \frac{c_2}{n^{1 + 1/2 - \delta}}\,,
\end{split}
\end{equation}
where,  $c_2$ is a positive constant and in the second inequality we used~\eqref{eq:n1}. 
%
Since $\delta \in (0, 1/2)$, by~\eqref{eq:n2} we obtain
\begin{equation*}
\frac{1}{n^2}\sum_{n = 1}^{\infty} \sum_{k=1}^{n}\mathbb{E}\left[\left(\um_{A_k^c} - \mathbb{P}(A_k^c)\right)^2\right] < \infty\,.
\end{equation*}

For the second sum on the RHS of \eqref{eq:divide}, since the $\{U_i\}_{i \ge 1}$ are independent,   we have that  $\mathbb{E}[(\um_{A_k^c} - \mathbb{P}(A_k^c))(\um_{A_m^c} - \mathbb{P}(A_m^c))] = 0$ for $m > k + b_k$. Therefore,  we obtain that 
\begin{align*}
& \frac{1}{n^2}\sum_{k = 1}^{n - 1}\sum_{m = k + 1}^{n}\mathbb{E}\left[\left(\um_{A_k^c} - \mathbb{P}(A_k^c)\right)\left(\um_{A_m^c} - \mathbb{P}(A_m^c)\right)\right] = 
\\
& \frac{1}{n^2}\sum_{k = 1}^{n - 1}\;\;\sum_{m = k + 1}^{n\wedge(k + b_k)}\Big(\mathbb{P}\left(A_k^c\cap A_m^c\right) - \mathbb{P}(A_k^c)\mathbb{P}(A_m^c)\Big) \,.
\end{align*}

Recalling the definition of $A_k$ in \eqref{eq:set_A}, for $m\leq k +b_k$ we have that 
% \begin{align*}
% A_k\cap A_m^c & = \left(\bigcap_{n = 1}^{b_k}\left\{U_{n + k} > \frac{1}{\sqrt{k}}\right\}\right)\bigcap\left(\bigcup_{l = 1}^{b_m}\left\{U_{l + m} \le \frac{1}{\sqrt{m}}\right\}\right)
% \\
% & = \left(\bigcap_{n = 1}^{b_k}\left\{U_{n + k} > \frac{1}{\sqrt{k}}\right\}\right)\bigcap\left(\bigcup_{l =  b_k + k - m + 1}^{b_m}\left\{U_{l + m} \le \frac{1}{\sqrt{m}}\right\}\right)\,.
% \end{align*}
%
\begin{align*}
A_k\cap A_m^c & = \bigcap_{n = 0}^{b_k}\left\{U_{n + k} > \frac{1}{(k+n)^{\beta/2}}\right\}\cap \;\bigcup_{l = 0}^{b_m}\left\{U_{l + m} \le \frac{1}{(m+l)^{\beta/2}}\right\}
\\
& = \bigcap_{n = 0}^{b_k}\left\{U_{n + k} > \frac{1}{(k+n)^{\beta/2}}\right\}\cap \;\bigcup_{l = b_k + k - m + 1}^{b_m}\left\{U_{l + m} \le \frac{1}{(m+l)^{\beta/2}}\right\}\,.
\end{align*}
%
Setting  $E_{k,m, \beta} := \displaystyle\bigcup_{l =  b_k + k - m + 1}^{b_m}\left\{U_{l + m} \le \frac{1}{(m+l)^{\beta/2}}\right\}$, and noticing that $$E_{k,m, \beta} \subseteq A_{m}^c \, ,$$ together with the fact that 
\begin{equation*}
\mathbb{P}\left(A_k^c\cap A_m^c\right) = \mathbb{P}\left(A_m^c\right) - \mathbb{P}\left(A_k\cap A_m^c\right) = \mathbb{P}\left(A_m^c\right) - \mathbb{P}\left(A_k\right)\mathbb{P}\left(E_{k,m, \beta}\right)\,,
\end{equation*}
we obtain that 
\begin{align*}
& \mathbb{P}\left(A_k^c\cap A_m^c\right) - \mathbb{P}\left(A_k^c\right)\mathbb{P}\left(A_m^c\right)\\
&= \mathbb{P}\left(A_m^c\right) - \mathbb{P}\left(A_k\right)\mathbb{P}\left(E_{k,m,\beta}\right) - \mathbb{P}\left(A_k^c\right)\mathbb{P}\left(A_m^c\right) 
\\
& 
% = \mathbb{P}(A_m^c)\mathbb{P}\left(A_k\right) - \mathbb{P}\left(A_k\right)\mathbb{P}\left(E_{k,m}\right)
= \mathbb{P}\left(A_k\right)\left[\mathbb{P}\left(A_m^c\right) - \mathbb{P}\left(E_{k,m,\beta}\right)\right] 
= \mathbb{P}\left(A_k\right)\mathbb{P}\left(A_m^c\setminus E_{k,m,\beta}\right)
\\
& = \mathbb{P}\left(A_k\right) \mathbb{P}\Big(\bigcup_{l = 0}^{b_k + k - m}\Big\{U_{l + m} \le \frac{1}{(m+l)^{\beta/2}}\Big\}\Big)
\\
& \leq \mathbb{P}\left(A_k\right)\Big[ 1 - \left(1 - \frac{1}{\sqrt{m}}\right)^{b_k + k - m} \Big] \le 1 - \left(1 - \frac{1}{\sqrt{m}}\right)^{b_k + k - m} \,.
\end{align*}

Since $x \ge 1 - e^{-x}$ for all $x \ge 0$, the RHS above can be bounded  by \begin{align*}
 1 - \left(1 - \frac{1}{\sqrt{m}}\right)^{b_k + k - m} &= 1 - e^{(b_m + k - m)\log(1 - \frac{1}{\sqrt{m}})}
\\
& \le - (b_m + k - m)\log(1 - \frac{1}{\sqrt{m}}) \\
&=-\frac{b_m + k - m}{\sqrt{m}}\log(1 - \frac{1}{\sqrt{m}})^{\sqrt{m}}
\\
& \le c_3 \frac{b_m + k - m}{\sqrt{m}}\,, 
\end{align*}
for some constant $c_3 > 0$. Then
\begin{align*}
& \sum_{m = k + 1}^{n\wedge(k + b_k)}\Big(\mathbb{P}\left(A_k^c\cap A_m^c\right) - \mathbb{P}(A_k^c)\mathbb{P}(A_m^c)\Big) \le c_3 \sum_{m = k + 1}^{k + b_k}\frac{b_k + k - m}{\sqrt{m}} 
\\
& =  c_3 \sum_{m =1}^{b_k}\frac{b_k - m}{\sqrt{m + k}} 
=  \frac{c_3}{\sqrt{b_k}}\sum_{m =1}^{b_k}\frac{1 - \frac{m}{b_k}}{\sqrt{\frac{m}{b_k} + \frac{k}{b_k}}}\frac{1}{b_k} \le  \frac{c_3}{\sqrt{b_k}}\sum_{m =1}^{b_k}\frac{1 - \frac{m}{b_k}}{\sqrt{\frac{m}{b_k} + 1}}\frac{1}{b_k}\,.
\end{align*}
Using that 
\begin{equation*}
\lim_{k\to\infty}\sum_{m =1}^{b_k}\frac{1 - \frac{m}{b_k}}{\sqrt{\frac{m}{b_k} + 1}}\frac{1}{b_k} = \int_{0}^{1}\frac{1 - x}{\sqrt{1 + x}}dx\,, 
\end{equation*}
we conclude that  there exists some positive constant $c_4$ such that 
\begin{equation*}
\sum_{m = k + 1}^{n\wedge(k + b_k)}\Big(\mathbb{P}\left(A_k^c\cap A_m^c\right) - \mathbb{P}(A_k^c)\mathbb{P}(A_m^c)\Big) \le c_4\frac{1}{\sqrt{b_k}}\,,
\end{equation*}
which, in turn,  implies that 
\begin{equation*}
\begin{split}
& \sum_{n = 1}^{\infty}\frac{1}{n^2}\sum_{k = 1}^{n - 1}\sum_{m = k + 1}^{n}\mathbb{E}\left[\left(\um_{A_k^c} - \mathbb{P}(A_k^c)\right)\left(\um_{A_m^c} - \mathbb{P}(A_m^c)\right)\right]
\\
&  \le c_4 \sum_{n = 1}^{\infty} \frac{1}{n^2}\sum_{k = 1}^{n}\frac{1}{\sqrt{b_k}}  \le c_5 \sum_{n = 1}^{\infty} \frac{1}{n^{1 + \delta/2}} < \infty \,.
\end{split}
\end{equation*}
\end{proof}

The next result is a general result for a sum of geometric random variables which will be important in the proof of Proposition~\ref{prop:RangeERW_lower}.

\begin{lemma}\label{lem:geo_sum_bound}
Let $\{H_j\}_{j \ge 1}$ be a sequence of independent random variables such that for each $ j \ge 1$, $H_j\sim {\rm Geo} (1/\sqrt{k+j})$. Then, for any integer $\theta\geq 1$ there exists a constant $\tilde{C}_\theta$ depending on $\theta$ only such that for any $m\geq 1$ it holds that
\begin{equation*}
  P\left(\sum_{j=1}^m H_j\leq \frac{m^{3/2}}{6} \right)\leq \frac{\Tilde{C}_\theta}{m^{\theta}}\,.  
\end{equation*}
\end{lemma}


\begin{proof}
Let us being noticing that for any $m,k\geq 1$ it holds that 
\[
\mathbb{E}\Big[\sum_{j=1}^m H_j\Big]\geq \int_{0}^m (k+x)^{1/2}dx\geq \frac{2}{3}m(k+m)^{1/2}\,,
\]
which ensures that $\mathbb{E}\Big[\sum_{j=1}^m H_j\Big] - \frac{m(m+k)^{1/2}}{2}\geq \frac{m(m+k)^{1/2}}{6} $. 
Therefore, we have that 
% $ Y_j - \mathbb{E}[\sum_{j = 1}^m Y_j]$ and $X_{j} = Y_j - \mathbb{E}[Y_j]$. 
% Then we have that the sequence $(X_{j})_{j \ge 1}$ is independent and $\mathbb{E}[X_{j}] = 0$. 
%
\begin{align*}
\mathbb{P}\left( \sum_{j=1}^m H_j < \frac{m^{3/2}}{6} \right) &\le \mathbb{P}\left( \sum_{j=1}^m H_j < \frac{m(m+k)^{1/2}}{6} \right)
\\
&\leq \mathbb{P}\left( \sum_{j=1}^m H_j < \mathbb{E}\Big[\sum_{j=1}^m H_j\Big] -\frac{m(m+k)^{1/2}}{2} \right)\\
&\leq 
\mathbb{P}\left( \Big| \sum_{j=1}^m X_j \Big| > \frac{ m \sqrt{k+m} }{2} \right) \,,   
\end{align*}
where $X_{j} = H_j - \mathbb{E}[H_j]$. Then, for  $\theta\geq 1$  a positive integer we have that 
\begin{equation}\label{eq:P(calG)<}
\begin{split}   
% & \mathbb{P}\left( \sum_{j=1}^m Y_j < \frac{m^{3/2}}{6} \right) \le
\mathbb{P}\left( \Big| \sum_{j=1}^m X_j \Big| > \frac{ m \sqrt{k+m} }{2} \right) 
 \leq 
 % \frac{C}{m^{2\theta} (k+m)^{\theta}} \mathbb{E}[H_{m,k}^{2 \theta}] 
 % = 
 \frac{4^\theta}{m^{2 \theta} (k+m)^{\theta}} \mathbb{E}\Big[ \Big( \sum_{j = 1}^m X_{j} \Big)^{2 \theta} \Big] %= \frac{C}{i^6 (k+i)^3} \mathbb{E} \Big[ \sum_{j,h,l,s,n,m = 1}^i X_{j,k} X_{h,l} X_{l,k} X_{s,k} X_{n,k} X_{m,k} \Big]
 \,. 
\end{split}
\end{equation}
Since $\{X_j\}_{j\geq 1}$ is a sequence of independent random variables of mean value zero, when expanding $\mathbb{E}\Big[ \Big( \sum_{j = 1}^m X_{j} \Big)^{2 \theta} \Big]$ the only terms that do not vanish are those of the form $\mathbb{E}[X_{j_1}^{\alpha_1} ... X_{j_r}^{\alpha_r}]$ where $1\leq r\leq m$, $j_1 \neq j_2 \neq \dots \neq j_r$, $\alpha_1 + \alpha_2 + \dots +\alpha_r = 2 \theta$ and $\alpha_1, \dots, \alpha_r$  are integers greater or equal than 2. Note that  we should have $r \le \theta$, since $r > \theta$ implies that $\alpha_i = 1$ for some $i$. 
%%%%
% {\color{cyan} From here we do not try to obtain a sharp upper bound on the number of terms, so we fix $r=\theta$ and we allow $\alpha_1, \dots, \alpha_\theta \ge 0$. Thus we have at most $m (m-1) ... (m-\theta +1) \le m^\theta$ choices for the indexes $j_1,...,j_\theta$ and ${{3 \theta - 1}\choose{\theta -1}}$ ways to choose the integers $\alpha_1, \dots, \alpha_\theta \ge 0$ such that 
% $\alpha_1 + \alpha2 + \dots +\alpha_r = 2 \theta$. Therefore
% $$
% \mathbb{E}\Big[ \Big( \sum_{j = 1}^m X_{j} \Big)^{2 \theta} \Big] \le {{3 \theta - 1}\choose{\theta -1}} m^\theta \max_{j_1\neq ... \neq j_\theta \atop \alpha_1 + ... +\alpha_r = 2\theta} \prod_{i=1}^{\theta} \mathbb{E}[X_{j_i}^{\alpha_i}]
% $$
% Let us recall that for $W$  a random variable with Geometric distribution with parameter $p$ it holds that $\mathbb{E}[W^n] \le n!/p^n$ for all $n \ge 1$ 
% (see, e.g., Example $8f$ in Chapter 4 in [ref Ross] \texttt{check the constant!!!!...}). Using the latter, we obtain that
% \begin{align*}
% \mathbb{E}[|W - \mathbb{E}[W]|^n] &\le \mathbb{E}[(W + \mathbb{E}[W])^n]= \sum_{\ell = 0}^n \binom{n}{\ell}\mathbb{E}[W^\ell]\mathbb{E}[W]^{n-\ell}
% \\
% &\leq  \frac{1}{p^n}\sum_{\ell = 0}^n \binom{n}{\ell}\ell! \leq \frac{n!}{p^n}(1+ e)\,.
% \end{align*}
% Thus 
% $$
% \max_{j_1\neq ... \neq j_\theta \atop \alpha_1 + ... +\alpha_r = 2\theta} \prod_{i=1}^{\theta} \mathbb{E}[X_{j_i}^{\alpha_i}] \le (2\theta)! (1+e)^\theta (k+m)^\theta
% $$
% which implies that
% $$
% \mathbb{E}\Big[ \Big( \sum_{j = 1}^m X_{j} \Big)^{2 \theta} \Big] \le {{3 \theta - 1}\choose{\theta -1}} (2\theta)! (1+e)^\theta m^\theta (k+m)^\theta.
% $$
% Going back to \eqref{eq:P(calG)<}, we obtain that
% $$
% \mathbb{P}\left( \sum_{j=1}^m Y_j < \frac{m^{3/2}}{6} \right) \le \frac{C_\theta}{m^\theta}
% $$
% where $C_\theta$ is a constant depending only on $\theta$. 
% }
%%%%%%%%%%%%%%%%%%%%%%%
Therefore, 
\begin{align}\label{eq:sum_sum}
\begin{split}
&\mathbb{P}\left( \sum_{j=1}^m H_j < \frac{m^{3/2}}{6} \right)\\ & \le  \frac{4^\theta}{m^{2 \theta} (k+m)^{\theta}}
\sum_{r = 1}^{\theta \wedge m}
\sum_{\substack{I\subseteq \{1, \ldots, m\}:\\ |I|=r}}\!\!\sum_{\substack{(n_i)_{i \in I}:\\n_i\geq 2, \forall i \\
\sum_{i \in I}n_i=2\theta}}\!\!\!\frac{(2\theta)! \mathbb{E}\big[\prod_{i \in I}X_i^{n_i}\big]}{\prod_{i \in I}n_i!} \,.
\end{split}
\end{align}


Let us recall that for $W$  a random variable with Geometric distribution with parameter $p$ it holds that $\mathbb{E}[W^n] \le n!/p^n$ for all $n \ge 1$ 
% where $C(n)$ is a positive constant that depends on $n$ only 
(see, e.g., \cite[Example $8f$-Chapter 4]{ross2010}).  Using the latter, we obtain that
\begin{align*}
\mathbb{E}[|W - \mathbb{E}[W]|^n] &\le \mathbb{E}[(W + \mathbb{E}[W])^n]= \sum_{\ell = 0}^n \binom{n}{\ell}\mathbb{E}[W^\ell]\mathbb{E}[W]^{n-\ell}
\\
&\leq  \frac{1}{p^n}\sum_{\ell = 0}^n \binom{n}{\ell}\ell! \leq \frac{n!}{p^n}(1+ e)\,.
\end{align*}
 Using the above bound in ~\eqref{eq:sum_sum} and using the independence of $\{X_{j}\}_{j \ge 1}$ we obtain that 
\begin{equation*}
\begin{split}
\mathbb{P}&\left( \sum_{j=1}^m H_j < \frac{m^{3/2}}{6} \right)\\
&\le  \frac{4^\theta}{m^{2 \theta} (k+m)^{\theta}}
\sum_{r = 1}^{\theta\wedge m} 
\sum_{\substack{I\subseteq \{1, \ldots, m\}:\\ |I|=r}}\sum_{\substack{(n_i)_{i \in I}:\\n_i\geq 2, \forall i \\
\sum_{i \in I}n_i=2\theta}}\!\!\!\frac{(2\theta)!}{\prod_{i \in I}n_i!} \prod_{i \in I}\mathbb{E}\big[X_i^{n_i}\big]
\\
&
\le  \frac{4^\theta}{m^{2 \theta} (k+m)^{\theta}}
\sum_{r = 1}^{\theta\wedge m} 
\sum_{\substack{I\subseteq \{1, \ldots, m\}:\\ |I|=r}}\sum_{\substack{(n_i)_{i \in I}:\\n_i\geq 2, \forall i \\
\sum_{i \in I}n_i=2\theta}}\frac{(2\theta)!}{\prod_{i \in I}n_i!} (k+m)^\theta (1+e)^r \prod_{i \in I}n_i!  
\\
&
\le  \frac{4^\theta}{m^{2 \theta} (k+m)^{\theta}}
\sum_{r = 1}^{\theta\wedge m} 
\sum_{\substack{I\subseteq \{1, \ldots, m\}:\\ |I|=r}}\binom{2\theta - \theta -1}{\theta-1} (2\theta)!  (k+m)^\theta (1+e)^r 
\\
&
\le  \frac{4^\theta}{m^{2 \theta} (k+m)^{\theta}}
\binom{3\theta -1}{\theta-1} (2\theta)!  (k+m)^\theta \sum_{r = 1}^{\theta\wedge m} 
\binom{m}{r}(1+e)^r 
\\
&
\le  \frac{4^\theta}{m^{2 \theta} (k+m)^{\theta}}
\theta m^\theta
\binom{3\theta -1}{\theta-1} (2\theta)!  (k+m)^\theta (1+e)^\theta = \frac{C_\theta}{m^\theta}\,,  
\end{split}
\end{equation*}
where $C_\theta$ is a constant depending on $\theta$ only. 
\end{proof}



\section{}\label{sec:appendix-mainTheorem}

\com{I do not know if makes anymore sense to include this appendix .....}\comu{concordo. está na outra versão, então podemos remover logo.}

In this Appendix we show that, by imposing a restriction on the direction of the drift $\ell$, it is possible to provide a ``weaker''  version of Theorem~\ref{pn-ERW-d=>4} which holds in any dimension $d\geq 4$. 
%
  The restriction is due to the technique we used to prove a weaker version of Proposition~\ref{prop: bound_Kn} by avoiding using Proposition~\ref{prop:RangeERW_lower} (which requires $d\geq 22$). The main idea to circumvent the usage of Proposition~\ref{prop:RangeERW_lower} consists in lower bounding the range of the $p_n$-\Nametwo{} with the range of a lazy random walk and then using Theorem~\ref{teo: RnZ>} (LLN). For this to work properly the lazy random walk should be at least $3$ dimensional, and we achieve that by restricting the direction of the drift to a subspace which is at most $d-3$ dimensional. 

Let us formally introduce the dimensional restriction. Let us define the set $\bD \subset \{e_1, \dots, e_d\}$, where $d \ge 4$ and $1 \leq k:= |\bD| \leq d-3$. Now set $\ell_{\bD}$ as a direction in the unit sphere in dimension $d$  such that $\ell_{\bD} = \sum_{i=1}^k \alpha_i x_i$, where $\alpha_i \in [0,1]$ and $x_i \in \bD$, both for all $1 \leq i \leq k$. In essence,  $\ell_{\bD}$ is a direction in the unit sphere in dimension $d$ determined by the canonical directions of the set $\bD$. 

As before, we  set $\pi_d$ as the probability that the random walk with increments $\{\xi_i\}_{i\geq 0}$ 
($\ZZ^d$-valued i.i.d. random variables with zero-mean vector and finite variance), never returns to the origin. Moreover, if  $X$ is a $p_n$-\Nametwo{} in direction $\ell_{\bD}$ and  $\mathcal{P}_{\bD^c}$ denotes the projection on $\bD^c$, then $\pi_{d-k}$ denotes the probability that the $(d-k)$-dimensional lazy random walk with increments $\{\mathcal{P}_{\bD^c}(\xi_i)\}_{i\geq 0}$  never returns to the origin. Note that $\pi_{d-k} \le \pi_{d}$.



\begin{theorem}\label{th:}
Let $X$ be a $p_n$-\Nametwo{} in direction $\ell_{\bD}$, on $\ZZ^d$ with $d \ge 4$,  $p_n= \mathcal{C} n^{-1/2} \wedge 1$. Then $\{\Hat{B}_{\cdot}^n\}_{n\ge 1}$ is tight in $C_{\Rs^d}[0, \infty)$  and there exists a Brownian Motion $W_{\cdot}$ such that for every limit point $\mathcal{Y}_{\cdot}$ of $\{\Hat{B}_{\cdot}^n\}_{n\ge 1}$ 
\begin{align*}
\left\{W_t \cdot \ell_{\bD} + 2 \hat{c}_1 \sqrt{t}\right\}_{t\ge 0} \preceq \{\mathcal{Y}_t \cdot \ell_{\bD}\}_{t\ge 0} \preceq \left\{W_t \cdot \ell_{\bD} + 2 {c}_2 \sqrt{t}\right\}_{t\ge 0} \,,   
\end{align*}
where $\hat{c}_1 = \hat{\mu}_\gamma ( 1-\sqrt{1 -  \pi_{d-k}} )$, ${c}_2 =\hat{\mu}_\gamma \sqrt{\pi_d}$ with $\hat{\mu}_{\gamma} := \EE[\gamma_i \cdot \ell_{\bD}]$ and $\preceq$ means ``stochastically less or equal to''. 
\end{theorem}
%
Note that the constant appearing in the lower bound in Theorem \ref{th:} is different from that in $b)$ of Theorem~\ref{pn-ERW-d=>4}. Clearly, in order to fairly compare the two result we must consider Theorem~\ref{pn-ERW-d=>4} for a drift direction which satisfies the above mentioned restriction. Assuming the latter, since $\pi_{d-k} \le \pi_d$ we obtain that  $\hat c_1 = \hat{\mu}_\gamma ( 1-\sqrt{1 -  \pi_{d-k}} ) \le \hat{\mu}_\gamma ( 1-\sqrt{1 -  \pi_{d}} ) = c_1$. Therefore, for $d\ge 22$, the statement of Theorem \ref{th:} is weaker than that of Theorem~\ref{pn-ERW-d=>4}. Nevertheless   Theorem~\ref{th:} holds for  any $d\geq 4$. 


\subsection{Proof of Theorem~\ref{th:}}
\hfill

 We only need to prove the lower bound.
We begin  by defining a useful coupling between the $p_n$-\Nametwo{} and a  lazy random walk. Again we assume $\CC = 1$, i.e., $p_n=n^{-1/2}$.

Let $\{X_i\}_{i \ge 0}$ a $p_n$-\Nametwo{} on $\ZZ^d$ with drift direction $\ell_{\bD}$ and let $\{Y_i\}_{i \geq 0}$ be the sequence of random vectors on $\ZZ^d$ defined as $Y_0=0$ and recursively for $n \ge 1$ 
\begin{equation*}
Y_{n+1} = Y_n + \mathcal{P}_{\bD^c} \big(\um_{E_n \cap \{U_{n+1} \leq n^{-1/2}\}}\gamma_{n+1} + \um_{E_n \cap \{U_{n+1} > n^{-1/2}\}} \xi_{n+1} + \um_{E_n^c}\xi_{n+1} \big) \,.     
\end{equation*}
The following properties stem directly from the definition:
\begin{itemize}
    %\item $\{Y_i\}_{i \geq 0}$ is a $p_n$-\Nametwo{} on $\ZZ^d$ with drift direction $\ell_{\bD}$.
    \item $\{Y_i\}_{i \geq 0}$ is a lazy random walk on $\ZZ^d$, evolving in a sublattice isomorphic to $\ZZ^{d-k}$.
    \item For all $e_j \in \bD^c$ and $i \geq 0$, we have $Y_i \cdot e_j = X_i \cdot e_j$. 
\end{itemize}

% Recall that the $p_n$-\Nametwo{} which we consider in this section has a restriction on the drift direction $\ell$. Specifically, $\ell$ could be any direction of the unitary sphere that is in the span of at most $d-3$ canonical directions.
% %
% This dimensional constraint   is crucial in our proof technique and it is due to how we use Theorem~\ref{teo: RnZ>} (SLLN) \com{in the proof of Proposition~\ref{prop: bound_Kn} (see, Remark~\ref{rem:restriction})}.

% Recall that in this section $\bD \subset \{e_1, \dots, e_d\}$, where $1 \leq k :=|\bD| \leq d-3$, $\ell_{\bD} \in \mathbb{S}^{d-1}$ is spanned by the canonical directions in $\bD$, $\mathcal{P}_{\bD^c}$ denotes the projection on $\bD^c$ and $\{X_i\}_{i \ge 0}$ in a $p_n$-\Nametwo{} on $\ZZ^d$ with drift direction $\ell_{\bD}$. Here $\{Y_i\}_{i \geq 0}$ is the sequence of random vectors on $\ZZ^d$ defined as $Y_0=0$ and recursively for $n \ge 1$ 
% \begin{equation*}
% Y_{n+1} = Y_n + \mathcal{P}_{\bD^c} \big(\um_{E_n \cap \{U_{n+1} \leq n^{-1/2}\}}\gamma_{n+1} + \um_{E_n \cap \{U_{n+1} > n^{-1/2}\}} \xi_{n+1} + \um_{E_n^c}\xi_{n+1} \big) \,.     
% \end{equation*}
% The following properties stem directly from the definition:
% \begin{itemize}
%     %\item $\{Y_i\}_{i \geq 0}$ is a $p_n$-\Nametwo{} on $\ZZ^d$ with drift direction $\ell_{\bD}$.
%     \item $\{Y_i\}_{i \geq 0}$ is a lazy random walk on $\ZZ^d$, evolving in a sublattice isomorphic to $\ZZ^{d-k}$.
%     \item For all $e_j \in \bD^c$ and $i \geq 0$, we have $Y_i \cdot e_j = X_i \cdot e_j$. 
% \end{itemize}

% Henceforth we suppose that all the following definitions are in the same common probability space. Let $\{U'_i\}_{i\geq 1}$ be a sequence of i.i.d. random variables uniformly distributed in [0,1]. We set $\{ \xi_i \}_{i \ge 1}$ and $\{ \gamma_i \}_{i \ge 1}$ as before (see Section~\ref{sec:p_n-ERW}) such that the drift direction of $\{\gamma_i\}_{i \ge 1}$ is $\ell_{\bD}$; both independent of the sequence $\{U'_i\}_{i\geq 1}$. We also define $\mathcal{P}_{\bD}$ and $\mathcal{P}_{\bD^c}$ as the projections respectively on $\bD$ and $\bD^c$. As one last but important restriction, we suppose that $\xi_i, \gamma_i \in \mathcal{P}_{\bD}(\mathbb{Z}^d) \cup \mathcal{P}_{\bD^c}(\mathbb{Z}^d)$, and moreover $\mathcal{P}_{\bD^c}(\xi_i)$ and $\mathcal{P}_{\bD^c}(\gamma_i)$ are identically distributed. \comu{Não vejo motivo para essas definições}
%Let $\{\phi_i\}_{i\geq 1}$ be a sequence of i.i.d uniform distribution on the set $\{1, 2, \dots, d\}$ and $\mathcal{D}_k$ be a set which is a subset of $\{1, 2, \dots, d\}$ and $|\mathcal{D}_k|=k$ where $1 \leq k \leq d-3$. Also, the sequences of random variables $\{U_i\}_{i \geq 1}$ and $\{U'_i\}_{i\geq 1}$ are i.i.d uniforms on $[0, 1]$ and independent of  $\{\phi_i\}_{i\geq 1}$. 

% We define the sequences $\{Y_i\}_{i \geq 0}$ and $\{Z_i\}_{i \geq 0}$ of random vectors on $\ZZ^d$ by setting $Y_0 = Z_0 =0$ and recursively for $n \geq 1$  
% \begin{equation*}
% Y_{n+1} = Y_n + \um_{\mathbb{B}_n\cap \{U'_{n+1} \leq p_{n+1}\}}\gamma_{n+1} + \um_{\mathbb{B}_n \cap \{U'_{n+1} > p_{n+1}\}} \xi_{n+1} + \um_{\mathbb{B}_n^c}\xi_{n+1} \, ,    
% \end{equation*}
% and 
% \begin{equation*}
% Z_{n+1} = Z_n + \mathcal{P}_{\bD^c} \big(\um_{\mathbb{B}_n\cap \{U'_{n+1} \leq p_{n+1}\}}\gamma_{n+1} + \um_{\mathbb{B}_n \cap \{U'_{n+1} > p_{n+1}\}} \xi_{n+1} + \um_{\mathbb{B}_n^c}\xi_{n+1} \big) \, ,    
% \end{equation*}
% where $\mathbb{B}_n:=\{Y_n \notin \Rr_{n-1}^Y\}$. The following properties stem directly from the definitions:
% \begin{itemize}
%     \item $\{Y_i\}_{i \geq 0}$ is a $p_n$-\Nametwo{} on $\ZZ^d$ with drift direction $\ell_{\bD}$.
%     \item $\{Z_i\}_{i \geq 0}$ is an aperiodic random walk (lazy random walk) on $\ZZ^d$. Besides that the process $Z$ behaves like a lazy random walk in $\ZZ^{d-k}$.
%     \item For all $e_j \in \bD^c$ and $i \geq 0$, we have $Y_i \cdot e_j = Z_i \cdot e_j$. 
% \end{itemize}
The next lemma states another important property of the coupling.
%Hence we will prove that if the process $\{Z_i\}_{i \geq 0}$ visits a new site then the process $\{Y_i\}_{i \geq 0}$ will reach a new site too. 

\begin{lemma}\label{lemaZY}
Whenever $\{Y_i\}_{i \geq 0}$ visits a new site, $\{X_i\}_{i \geq 0}$ also visits a new one, which in turn implies that $|\Rr_n^X| \geq |\Rr_n^Y|$ for all $n \geq 0$. 
\end{lemma}

\begin{proof}[Proof of Lemma~\ref{lemaZY}.]  
Suppose that at time $n$ $\{Y_i\}_{i \geq 0}$ reaches a new site ($Y_n \notin \Rr_{n-1}^Y$) and $\{X_i\}_{i \geq 0}$ does not ($X_n \in \Rr_{n-1}^X$). By the definition of $\{Y_i\}_{i \geq 0}$ the direction of its displacement to attain the new site is spanned by $\bD^c$. Moreover we have $Y_n \cdot e_i = X_n \cdot e_i$ for every $e_i \in \bD^c$. Since $X_n \in \Rr_{n-1}^Y$, there exist a $0 \leq j \leq n-1$ such that $X_j = X_n$. Hence $Y_n \cdot e_i = Y_j \cdot e_i$ for every $e_i \in \bD^c$ and again by the definition of the coupling $Y_j \cdot e_i = X_j \cdot e_i$ for every $e_i \in \bD^c$, which means that $Y_n = Y_j$. Thus we have a contradiction. 
\end{proof}

% Let $\pi_d$ denotes the probability that the random walk with i.i.d. (with zero mean and finite variance) increments $\{\xi_i\}_{i\geq 0}$ on $\ZZ^d$ never returns to the origin, and $\pi_{d-k}$ denotes the probability of a random walk with i.i.d. increments (with zero mean and finite variance) given by the corresponding lazy random walk of the coupling never returning to the origin. Recall  that $k=|\bD|$ is the number of canonical directions spanned by the drift direction $\ell_{\bD}$  and $k\leq d-3$.

The proof of Theorem~\ref{th:} follows the same lines of the  proof of Theorem~\ref{pn-ERW-d=>4} with the only difference that we need a ``version'' of $(b)$ in Proposition~\ref{prop: bound_Kn} for $d\geq 4$. This is the content of the next proposition. 
%
\begin{proposition}\label{prop:}  
If $H_{\cdot}$ is a limit point of $\{\Hat{K}^n_{\cdot }/n^{1/2}\}_{n\ge 1}$, then for $d\geq 4$ it holds that
\begin{equation*}
\PP \left[\forall t \in [0,\infty): H_t \ge 2t^{1/2}(1-(1 -  \pi_{d-k})^{1/2}) \ \right] = 1 \, . 
\end{equation*}
\end{proposition} 

\begin{proof}[Proof of Proposition~\ref{prop:}.] We just have to follow the proof of $(b)$ in Proposition~\ref{prop: bound_Kn} until we get to \eqref{eq: B_t^cM>}. At that point we will need to show that 
\begin{equation}\label{lb-AP}
\lim_{n\to \infty} \PP \big[\forall 
 t \in [c,M]:|\Rr_{\lfloor nt \rfloor}^X| \ge \delta'' \lfloor nt \rfloor\big] = 1\,,
\end{equation}
without Proposition \ref{prop:RangeERW_lower}. After this we can finish the proof in the same way.  

Using the coupling with the lazy random walk, we have that 
$$
\PP\big[ \forall t \in [c,M]: |\Rr_{\lfloor nt \rfloor}^X| \ge \delta'' \lfloor nt \rfloor  \big]
 \ge \PP \big[\forall 
 t \in [c,M]:|\Rr_{\lfloor nt \rfloor}^Y| \ge \delta'' \lfloor nt \rfloor\big] \,.    
$$ 
Since $d\geq 4$ and $d-k\geq 3$, we have that $\pi_{d-k}>0$ and by Theorem~\ref{teo: RnZ>} (LLN) there exists an integer random variable $N_{\delta''}$ such that $\PP [|\Rr_{m}^Y| \ge \delta''m ,\ \forall m \ge N_{\delta''} ] = 1$, thus 
$$\lim_{n\to \infty} \PP \big[\forall 
 t \in [c,M]:|\Rr_{\lfloor nt \rfloor}^Y| \ge \delta'' \lfloor nt \rfloor\big] \ge \lim_{n\to \infty} \PP [N_{\delta''} \le cn ] = 1,
 $$ 
for every $\delta'' \in (0,\pi_{d-k})$. Therefore \eqref{lb-AP} holds.
\end{proof}

\begin{comment}
\begin{proof}[Proof of Proposition~\ref{prop:}.] Note that the statement follows if
\begin{equation}\label{oldstatement-AP}
\PP \left[\forall t \in [0,\infty): 2t^{1/2}(1-(1 -  \delta')^{1/2}) \le H_t \le 2(t \delta)^{1/2} \right] = 1 \, ,  
\end{equation}
for every $\delta' \in (0, \pi_{d-k})$ and $\delta \in (\pi_d, 1)$. Indeed taking sequences $\delta'_n \uparrow \pi_{d-k}$ and $\delta_n \downarrow \pi_d$ as $n\to \infty$, we have that 
$$
\Big\{ \forall t \in [0,\infty): 2t^{1/2}(1-(1 -  \pi_{d-k})^{1/2}) \le H_t \le 2(t \pi_d)^{1/2} \Big\}\,,
$$
is given by
$$
\bigcap_{n\ge 1} \Big\{ \forall t \in [0,\infty): 2t^{1/2}(1-(1 - \delta'_n)^{1/2}) \le H_t \le 2(t \delta_n)^{1/2} \Big\}\, .
$$
Let us begin with some instrumental facts: recall (from page \pageref{eq: def Kn}) that $\varphi_i := \psi_i + 1$, where $\{\psi_i\}_{i \ge 1}$ denotes  the sequence of $\FF$-stopping times  corresponding to the  times the $p_n$-\Nametwo{} visits a new site, and let us define: 
\begin{align*}
J'_n(\delta) := 
\sum_{i=1}^{\delta n} \um_{\{U_{\varphi_i} \le \varphi_i^{-1/2} \}}     %\label{eq: domJn<J'}
\qquad \text{ and } \qquad 
V'_n(\delta') :=\sum_{i = 1}^{\delta' n} \um_{\{U_{\varphi_i} \leq (\varphi_i \wedge (n-i) )^{-1/2} \}}\,, %\label{eq: F'n_domsto}
\end{align*}
with $\delta \in (\pi_d, 1)$ and $\delta' \in (0, \pi_{d-k})$. 
%
Using Lemma \ref{lem: iid} and since $\{U_i\}_{i \ge 1}$ is i.i.d. we have that 
\begin{equation}\label{eq:dominance-AP}
J'_n(\delta)  \preceq J_n(\delta) \qquad  \text{ and }  \qquad  V'_n(\delta')  \succeq  V_n(\delta')\,, 
\end{equation} 
where, $J_n$ and $V_n$ are defined  in~\eqref{Bn'} and \eqref{Fn'}, respectively.  


We divide the proof of \eqref{oldstatement} into two parts: the first is concerned with the upper bound and the second with the lower bound. 
Consider the event 
\begin{equation*}
A_n^{{\delta}, c, M}:= \Big\{\forall t \in [c, M]: \frac{|K_{\lfloor nt \rfloor}|}{n^{1/2}} \le 2({\delta} t)^{1/2}  \Big\} \,,   
\end{equation*}
where $c$, $M$ and ${\delta}$ are positive constants such that $M > c$ and ${\delta} \in (\pi_d, 1)$, and recall that $|K_{\lfloor nt \rfloor}|  = \sum_{j=1}^{|\Rr_{\lfloor nt \rfloor}^X|} \um_{\{U_{\varphi_j} \leq \varphi_j^{-1/2} \}}$ (see,  \eqref{eq: def Kn}). For every $\Hat{\delta} \in (\pi_d,\delta)$ we have that
\begin{align}\label{eq: A_t^cM<-AP}
\begin{split}
& \PP[A_n^{{\delta}, c,M}]  \ge \PP[A_n^{\delta, c,M} \cap \{ \forall t \in [c,M]: |\Rr_{\lfloor nt \rfloor}^X| \le \Hat \delta \lfloor nt \rfloor\}]
\\
& \ge \PP\Big[ \Big\{\forall t \in [c,M]: \frac{J'_{\lfloor nt \rfloor}(\delta)}{n^{1/2}} \le 2(\delta t)^{1/2} \Big\} \cap \big\{ \forall t \in [c,M]: |\Rr_{\lfloor nt \rfloor}^X| \le \Hat \delta \lfloor nt \rfloor \big\}  \Big] \, .
\end{split}    
\end{align}
Considering the second event on the right-hand side of~\eqref{eq: A_t^cM<}, by Proposition~\ref{prop:RangeERW}, for every $\Hat{\delta} \in (\pi_d, \delta)$ there exists an integer random variable $N_{\Hat{\delta}}$ such that $\PP[|\Rr_m^X| \le \Hat{\delta} m, \, \forall m \ge N_{\Hat{\delta}}] = 1$, hence $\lim_{n \to \infty} \PP[\forall t \in [c,M]: |\Rr_{\lfloor nt \rfloor}^X| \le \Hat \delta \lfloor nt \rfloor] \ge \lim_{n \to \infty} \PP[N_{\Hat{\delta}} \le cn]=1$.
Now concerning the first event on the right-hand side of~\eqref{eq: A_t^cM<}, by~\eqref{eq:dominance}, specifically since $J'_n(\delta)  \preceq J_n(\delta)$, we obtain that 
\begin{equation*} 
\PP\Big[ \forall t \in [c,M]: \frac{J'_{\lfloor nt \rfloor}(\delta)}{n^{1/2}} \le 2(\delta t)^{1/2} \Big] \ge
\PP\Big[ \forall t \in [c,M]: \frac{J_{\lfloor nt \rfloor}(\delta)}{n^{1/2}} \le 2(\delta t)^{1/2} \Big] \, ,
\end{equation*}
%
where the right-hand side converges to one by Lemma~\ref{lem: tightaux} part $i)$ (convergence in distribution to a deterministic function implies convergence in probability, see \cite[page 27]{billingsley1999probability}).
%it holds that \com{notation of convergence of processes}
%\begin{equation}\label{eq:Jninprob}
%\frac{J_{\lfloor n \cdot \rfloor}}{n^{1/2}} \to 2(\delta \cdot)^{1/2}  \text{ as } n \to \infty  \,, 
%\end{equation}
%in probability, since it converges in distribution to a deterministic function in $C_{\mathbb{R}}[0,\infty)$ (see \cite[page 27]{billingsley1999probability}).
%
Hence on the right-hand side of~\eqref{eq: A_t^cM<} we have an intersection of two events whose probability converges to 1 as $n$ goes to infinity. Thus, for every $M > c >0$ and $\delta \in (\pi_d, 1]$ 
\begin{equation*}
\lim_{n\to \infty}  \PP[A_n^{\delta, c,M}]  = 1\,.
%\PP \Big[ \forall t \in [c, M]: \frac{|K_{\lfloor nt \rfloor}|}{n^{1/2}} \le 2(\Hat{\delta} t)^{1/2} \Big] \to 1 \text{ as } n \to \infty \,,   
\end{equation*}
%
%Since  $\hat{\delta} > \delta$ is arbitrary \com{I ?????}, we  obtain that 
%\begin{equation*}
%\PP \Big[ \forall t \in [c, M]: \frac{|K_{\lfloor nt \rfloor}|}{n^{1/2}} \le 2(\delta t)^{1/2}  \Big] \to 1 \text{ as } n \to \infty\,,   
%\end{equation*}
%
Now suppose that we have monotone decreasing and increasing  sequences $\{c_j\}_{j \ge 1}$ and $\{M_j\}_{j \ge 1}$ respectively, such that  $c_j \to 0$ and $M_j \to \infty$ as $j$ goes to infinity. Let $\{
H_t\}_{t\ge 0}$ be a limit point in distribution of a subsequence of $\{\Hat{K}^n_{\cdot}/n^{1/2}\}_{n\ge 1}$, which is  tight  by  
Lemma~\ref{Jntight} (item $i)$), and define
\begin{equation*}
A^{\delta}:= \left\{\forall t \in [0, \infty): H_t \le 2(\delta t)^{1/2}  \right\} \,,   
\end{equation*}
and 
$$
A^{\delta, c_i, M_i} = \left\{\forall t \in [c_i, M_i]: H_t \le 2(\delta t)^{1/2}  \right\}\,,
$$
and $A^{\delta} = \cap_{i=1}^{\infty} A^{ \delta ,c_i, M_i}$ (since $H_0=0$).
%
Then,  by Portmanteau Theorem we have that for every $i \in \ZZ^+$ %\com{on the rhs we should specify the dependence on $\delta$ and add that for all $\delta \in (\pi_d,1]$...}
\begin{equation*}
\PP[A^{\delta ,c_i, M_i}] \ge \limsup_{n \to \infty} \PP[A_n^{\delta, c_i, M_i}] =1 \,.    
\end{equation*}
%
Hence, $\PP[A^{\delta, c_i, M_i}] = 1$ for all $i \in \ZZ^+$ and we obtain the desired result, namely
\begin{equation}\label{eq:At_1-AP}
\PP[A^{\delta}] = \PP\Big[ \bigcap_{j=1}^{\infty} A^{\delta, c_j, M_j}\Big] = 1 \,. 
\end{equation}
%
\com{Here is where the main difference with $d\geq 22$ is ....}As far as the lower bound is concerned, 
let us define the following events 
\begin{equation*}
\begin{split}
& B_n^{\delta',c, M}:= \Big\{\forall t \in [c, M]:2t^{1/2}(1-(1-\delta')^{1/2}) \le \frac{|K_{\lfloor nt \rfloor}|}{n^{1/2}} \Big\} \,,
\\
& H_n^{\delta',c,M}:= \Big\{\forall t \in [c, M]: 2t^{\frac{1}{2}}(1-(1-\delta')^{\frac{1}{2}}) \le \frac{V'_{\lfloor nt \rfloor}(\delta')}{n^{1/2}} \Big\}\,,
% \\
% & R_{\lfloor nt \rfloor}:= \big\{ \forall t \in [c,M]: |\Rr_{\lfloor nt \rfloor}^X| \ge \delta'' \lfloor nt \rfloor \big\} \,,
\end{split}
\end{equation*}
where $c$, $M$ and  $\delta'$  are positive constants such that $M > c$ and $\delta'\in (0,\pi_{d-k})$. 
%
Given $\delta'' \in (\delta', \pi_{d-k})$,  we have that
\begin{equation}\label{eq: B_t^cM>-AP}
\begin{split}
\PP[B_n^{\delta',c,M}]  & \ge \PP\big[B_n^{\delta',c,M} \cap \big\{ \forall t \in [c,M]: |\Rr_{\lfloor nt \rfloor}^X| \ge \delta'' \lfloor nt \rfloor \big\}\big] 
\\
&\ge \PP\big[ H_n^{\delta',c,M} \cap \big\{ \forall t \in [c,M]: |\Rr_{\lfloor nt \rfloor}^X| \ge \delta'' \lfloor nt \rfloor \big\} \big]\,.
\end{split}
\end{equation}
%
By~\eqref{eq:dominance}, specifically by $V'_n(\delta')  \succeq V_n(\delta')$, it holds that 
$$
\PP\big[ H_n^{\delta',c,M} \big]
\ge \PP \Big[ \forall t \in [c, M]: 2t^{1/2}(1-(1-\delta')^{1/2}) \le \frac{V_{\lfloor nt \rfloor}(\delta')}{n^{1/2}} \Big]\,,
$$
and by the coupling with the lazy random walk 
$$
\PP\big[ \forall t \in [c,M]: |\Rr_{\lfloor nt \rfloor}^X| \ge \delta'' \lfloor nt \rfloor  \big]
 \ge \PP \big[\forall 
 t \in [c,M]:|\Rr_{\lfloor nt \rfloor}^Y| \ge \delta'' \lfloor nt \rfloor\big] \,.    
$$ 
Now by Lemma~\ref{lem: tightaux} part $ii)$  (convergence in distribution to a deterministic function implies convergence in probability, see \cite[page 27]{billingsley1999probability}) 
we obtain that $\lim_{n \to \infty}\PP\big[ H_n^{\delta',c,M} \big]=1$, for every $\delta' \in (0,\pi_{d-k})$. Moreover, since $d\geq 4$ and $d-k\geq 3$, we have that $\pi_{d-k}>0$ and by Theorem~\ref{teo: RnZ>} (LLN) there exists an integer random variable $N_{\delta''}$ such that $\PP [|\Rr_{m}^Y| \ge \delta''m ,\ \forall m \ge N_{\delta''} ] = 1$, thus $\lim_{n\to \infty} \PP \big[\forall 
 t \in [c,M]:|\Rr_{\lfloor nt \rfloor}^Y| \ge \delta'' \lfloor nt \rfloor\big] \ge \lim_{n\to \infty} \PP [N_{\delta''} \le cn ] = 1$, for every $\delta'' \in (0,\pi_{d-k})$. 
%we obtain that
%\begin{equation}\label{eq:Fninprob}
%\frac{\sum_{j=1}^{\lfloor n\cdot \rfloor} \um_{\{ U_j \le j^{-1/2} \}} - F_{\lfloor n\cdot \rfloor}}{n^{\frac{1}{2}}} \to 2(\cdot)^{1/2}(1-(1-\delta')^{1/2}) \text{ as } n \to \infty  \,, 
%\end{equation}
%in probability \com{notation of convergence of processes!}, since it converges in distribution to a  continuous function in $t$ (see \cite[page 27]{billingsley1999probability}).
%
Since in~\eqref{eq: B_t^cM>} we have an intersection of two events whose probability converges to 1 as $n$ goes to infinity, we may conclude that for every $M > c >0$ 
\begin{equation*}
\PP \Big[ \forall t \in [c, M]:2t^{1/2}(1-(1-\delta')^{1/2}) \le \frac{|K_{\lfloor nt \rfloor}|}{n^{1/2}}  \Big] \xrightarrow[n \to \infty]{} 1 \,.
\end{equation*} 
%
Finally, we finish the proof by the same argument used to obtain~\eqref{eq:At_1}.
\end{proof}
\end{comment}


%%%%%%% OLD STUFF
\begin{comment}
\subsection{OLD STUFF.....}


\subsubsection{Case $\beta = 1/2$ and $d \ge 4$; proof of Theorem~\ref{pn-ERW-d=>4}}\label{sec: d>4}

\hfill \\



\comu{essa seção toda precisa de revisão. Minha sugestão é criar um apêndice para o resultado que usa o acoplamento em $4\le d \le 21$. }
We begin this section by defining a useful coupling between the $p_n$-\Nametwo{} and a  lazy random walk. Again we assume $\CC = 1$, i.e., $p_n=n^{-1/2}$.

The $p_n$-\Nametwo{} which we consider  in this section has a restriction on the drift direction $\ell$. Specifically, $\ell$ could be any direction of the unitary sphere that is in the span of at most $d-3$ canonical directions.
%
This dimensional constraint   is crucial in our proof technique and it is due to how we use Theorem~\ref{teo: RnZ>} (SLLN) in the proof of Proposition~\ref{prop: bound_Kn} (see, Remark~\ref{rem:restriction}).

Recall that in this section $\bD \subset \{e_1, \dots, e_d\}$, where $1 \leq k :=|\bD| \leq d-3$, $\ell_{\bD} \in \mathbb{S}^{d-1}$ is spanned by the canonical directions in $\bD$, $\mathcal{P}_{\bD^c}$ denotes the projection on $\bD^c$ and $\{X_i\}_{i \ge 0}$ in a $p_n$-\Nametwo{} on $\ZZ^d$ with drift direction $\ell_{\bD}$. Here $\{Y_i\}_{i \geq 0}$ is the sequence of random vectors on $\ZZ^d$ defined as $Y_0=0$ and recursively for $n \ge 1$ 
\begin{equation*}
Y_{n+1} = Y_n + \mathcal{P}_{\bD^c} \big(\um_{E_n \cap \{U_{n+1} \leq n^{-1/2}\}}\gamma_{n+1} + \um_{E_n \cap \{U_{n+1} > n^{-1/2}\}} \xi_{n+1} + \um_{E_n^c}\xi_{n+1} \big) \,.     
\end{equation*}
The following properties stem directly from the definition:
\begin{itemize}
    %\item $\{Y_i\}_{i \geq 0}$ is a $p_n$-\Nametwo{} on $\ZZ^d$ with drift direction $\ell_{\bD}$.
    \item $\{Y_i\}_{i \geq 0}$ is a lazy random walk on $\ZZ^d$, evolving in a sublattice isomorphic to $\ZZ^{d-k}$.
    \item For all $e_j \in \bD^c$ and $i \geq 0$, we have $Y_i \cdot e_j = X_i \cdot e_j$. 
\end{itemize}

% Henceforth we suppose that all the following definitions are in the same common probability space. Let $\{U'_i\}_{i\geq 1}$ be a sequence of i.i.d. random variables uniformly distributed in [0,1]. We set $\{ \xi_i \}_{i \ge 1}$ and $\{ \gamma_i \}_{i \ge 1}$ as before (see Section~\ref{sec:p_n-ERW}) such that the drift direction of $\{\gamma_i\}_{i \ge 1}$ is $\ell_{\bD}$; both independent of the sequence $\{U'_i\}_{i\geq 1}$. We also define $\mathcal{P}_{\bD}$ and $\mathcal{P}_{\bD^c}$ as the projections respectively on $\bD$ and $\bD^c$. As one last but important restriction, we suppose that $\xi_i, \gamma_i \in \mathcal{P}_{\bD}(\mathbb{Z}^d) \cup \mathcal{P}_{\bD^c}(\mathbb{Z}^d)$, and moreover $\mathcal{P}_{\bD^c}(\xi_i)$ and $\mathcal{P}_{\bD^c}(\gamma_i)$ are identically distributed. \comu{Não vejo motivo para essas definições}
%Let $\{\phi_i\}_{i\geq 1}$ be a sequence of i.i.d uniform distribution on the set $\{1, 2, \dots, d\}$ and $\mathcal{D}_k$ be a set which is a subset of $\{1, 2, \dots, d\}$ and $|\mathcal{D}_k|=k$ where $1 \leq k \leq d-3$. Also, the sequences of random variables $\{U_i\}_{i \geq 1}$ and $\{U'_i\}_{i\geq 1}$ are i.i.d uniforms on $[0, 1]$ and independent of  $\{\phi_i\}_{i\geq 1}$. 

% We define the sequences $\{Y_i\}_{i \geq 0}$ and $\{Z_i\}_{i \geq 0}$ of random vectors on $\ZZ^d$ by setting $Y_0 = Z_0 =0$ and recursively for $n \geq 1$  
% \begin{equation*}
% Y_{n+1} = Y_n + \um_{\mathbb{B}_n\cap \{U'_{n+1} \leq p_{n+1}\}}\gamma_{n+1} + \um_{\mathbb{B}_n \cap \{U'_{n+1} > p_{n+1}\}} \xi_{n+1} + \um_{\mathbb{B}_n^c}\xi_{n+1} \, ,    
% \end{equation*}
% and 
% \begin{equation*}
% Z_{n+1} = Z_n + \mathcal{P}_{\bD^c} \big(\um_{\mathbb{B}_n\cap \{U'_{n+1} \leq p_{n+1}\}}\gamma_{n+1} + \um_{\mathbb{B}_n \cap \{U'_{n+1} > p_{n+1}\}} \xi_{n+1} + \um_{\mathbb{B}_n^c}\xi_{n+1} \big) \, ,    
% \end{equation*}
% where $\mathbb{B}_n:=\{Y_n \notin \Rr_{n-1}^Y\}$. The following properties stem directly from the definitions:
% \begin{itemize}
%     \item $\{Y_i\}_{i \geq 0}$ is a $p_n$-\Nametwo{} on $\ZZ^d$ with drift direction $\ell_{\bD}$.
%     \item $\{Z_i\}_{i \geq 0}$ is an aperiodic random walk (lazy random walk) on $\ZZ^d$. Besides that the process $Z$ behaves like a lazy random walk in $\ZZ^{d-k}$.
%     \item For all $e_j \in \bD^c$ and $i \geq 0$, we have $Y_i \cdot e_j = Z_i \cdot e_j$. 
% \end{itemize}
The next lemma states another important property of the coupling.
%Hence we will prove that if the process $\{Z_i\}_{i \geq 0}$ visits a new site then the process $\{Y_i\}_{i \geq 0}$ will reach a new site too. 

\begin{lemma}\label{lemaZY}
Whenever $\{Y_i\}_{i \geq 0}$ visits a new site, $\{X_i\}_{i \geq 0}$ also visits a new one, which in turn implies that $|\Rr_n^X| \geq |\Rr_n^Y|$ for all $n \geq 0$. 
\end{lemma}
%
The proof of Lemma~\ref{lemaZY} will be postponed to the end of this section.
% Before we present the proof of Theorem~\ref{pn-ERW-d=>4} we need to present some useful auxiliary results. We set $\pi_d$ as the probability of a random walk in $\ZZ^d$ never returns to the origin. From Theorem 1 in~\cite{hamana2001large} we have the following result.
%
% \begin{theorem}[see~\cite{hamana2001large}, Theorem 1]\label{teo: RnZ>}
% Let $Z$ be an aperiodic random walk in $\ZZ^d$, then for every $\theta < \pi_d$ and $\theta' \in (\pi_d, 1]$ we have respectively 
% \begin{equation*}
% \lim_{n \to \infty} \PP[|\Rr_{n}^Z| \geq \theta n]= 1 \quad \text{ and } \quad  \PP[|\Rr_{n}^Z| \geq \theta' n] \leq e^{-c_{\theta'}n}\;,    
% \end{equation*}
% where $c_{\theta'}$ is a positive constant that depends of $\theta'$. 
% \end{theorem}

% Now we will analyze the behavior of the range of the $p_n$-\Nametwo{} in $\ZZ^d$, with $d \geq 3$ and $\beta =1/2$. The proof of this Proposition uses the same techniques from the case in $\ZZ^2$.
% \begin{proposition}\label{prop: RnX<d3}
% Let $\{X_n\}_{n \geq 0}$ be a $p_n$-\Nametwo{} in $\ZZ^d$, with $d \geq 3$ and $\beta = 1/2$ then we have that
% \begin{equation*}
%     \PP\left[ |\Rr_n ^X| \leq \delta n \right] \to 1 \quad \text{as } n \to \infty \,,
% \end{equation*}
% for any $\delta \in (\pi_{d}, 1] $, \cm{where $\pi_{d}$ }.
% \end{proposition}

% The proof of Proposition~\ref{prop: RnX<d3} will be postponed to end of this section. 
%Henceforth, without loss of generality, we shall assume $\CC = 1$, i.e., $p_n=n^{-1/2}$. 




\medskip 
Let us define the following random variable:
\begin{equation}
\label{Bn'}
J_n(\delta) := \sum_{i=1}^{\delta n} \um_{\{U_i \leq i^{-1/2} \}}\,,
\end{equation}
%As we saw in~\eqref{Kn<Jn}, we have $|K_n| \preceq |J_n|$. 
%
where $\delta \in (\pi_d, 1)$ and $\pi_d$ denotes the probability that the random walk with i.i.d. (with zero mean and finite variance) increments $\{\xi_i\}_{i\geq 0}$  never returns to the origin, and 
%
\begin{equation}\label{Fn'}
V_n(\delta'):=\sum_{i=n-\delta' n+1}^{n} \um_{\{U_i \leq i^{-1/2} \}}
% = \sum_{i=1}^n  \um_{\{U_i \leq i^{-1/2} \}} - \underbrace{\sum_{i=1}^{n-\delta' n} \um_{\{U_i \leq i^{-1/2} \}}}_{:= F'_n}
\,, 
\end{equation}
where $\delta' \in (0, \pi_{d-k})$ and $\pi_{d-k}$ denotes the probability of a random walk with i.i.d. increments (with zero mean and finite variance) given by the corresponding lazy random walk of the coupling never returning to the origin (recall that $k=|\bD|$ is the number of canonical directions spanned by the drift direction $\ell_{\bD}$  and $k\leq d-3$).
%
% and, by the same reasons we have~\eqref{Kn<Jn}, we obtain \comu{como antes a troca não é direta}
% \begin{equation}
% \sum_{i=n-|\Rr_n^X|+1}^{n} 1_{\{U_i \leq i^{-1/2} \}} \preceq \sum_{j=1}^{|\Rr_n^X|} 1_{\{U_{\tau_j} \leq \tau_j^{-1/2} \}} = |K_n| \,.    
% \end{equation}
% \com{\texttt{I would remove this and place it in the proof of Proposition~\ref{prop: bound_Kn}}
% Using Lemma \ref{lem: iid} and since $\{U_i\}_{i \ge 1}$ is i.i.d. we have that 
% \begin{equation}\label{eq: domJn<J'}
% J'_n := 
% \sum_{i=1}^{\delta n} \um_{\{U_{\varphi_i} \le \varphi_i^{-1/2} \}} \preceq J_n \, .
% \end{equation} 
% }
% %since $\{\varphi_i\}_{i \ge 1}$ is a sequence of of $\FF$-stopping times and $\{U_i\}_{i \ge 1}$ is i.i.d..
% % \sout{Now we define  the following random variable}
% % \com{No need to define $F_n$ and $F_n'$!}
% % \begin{equation}\label{Fn'}
% % F_n:=\sum_{i=1}^{n-\delta' n} \um_{\{U_i \leq i^{-1/2} \}}  
% %  = \sum_{i=1}^n  \um_{\{U_i \leq i^{-1/2} \}} - \sum_{i=n-\delta' n+1}^{n} \um_{\{U_i \leq i^{-1/2} \}}\,, 
% % \end{equation}
% %Since $\{\varphi_i\}_{i \ge 1}$ is a sequence of $\FF$-stopping times and the sequence $\{U_i\}_{i \ge 1}$ is i.i.d.,  one can note that
% \com{\texttt{I would remove this and place it in the proof of Proposition~\ref{prop: bound_Kn}}
% Using again Lemma \ref{lem: iid} and the fact that $\{U_i\}_{i \ge 1}$ is i.i.d., we obtain that
% \begin{equation}\label{eq: F'n_domsto}
% V_n = \sum_{i=n-\delta' n+1}^{n} \um_{\{U_i \leq i^{-1/2} \}} \preceq \sum_{i = 1}^{\delta' n} \um_{\{U_{\varphi_i} \leq (\varphi_i \wedge (n-i) )^{-1/2} \}} =: V'_n
% \,.
% \end{equation}
% }
% Let $\pi_d$ denote the probability of a random walk with i.i.d. increments (with zero mean and finite variance) given by the corresponding $\{\xi_i\}_{i\geq 0}$ never returning to the origin. We set the following random variable
% \begin{equation}\label{Bn'}
% |J'_n| := \sum_{i=1}^{\delta n}  1_{\{U_i \leq i^{-1/2} \}}  \quad \text{where } \delta \in (\pi_d,1] \,.   
% \end{equation}
% One can notice that by Proposition~\ref{prop:RangeERW} we obtain
% \begin{equation}\label{Bn<=Bn'} {\color{blue}
% \PP[|J_n| \le |J'_n| ] \rightarrow 1 \quad \text{as } n \to \infty \,.   } 
% \end{equation}

% An important  consequence of the  coupling described at the begging of this section between  a random walk  $Z$ and a process $X$ which is a $p_n$-\Nametwo{} in direction $\ell_{D_k}$ in $\ZZ^d$, where $d \ge 4$ (note that  $Z$ is a lazy random walk in $\ZZ^{d-k}$) is Lemma~\ref{lemaZY} which implies that $|\Rr_n^X| \ge |\Rr_n^Z|$ for all $n \ge 1$. Hence we have the following
% \begin{equation}\label{Fn<=}
% |F_n| \le \sum_{i=1}^{n - |\Rr_n^Z|}  1_{\{U_i \leq i^{-1/2} \}} \quad \text{for all } n \ge 1  \,.
% \end{equation}

% Let $\pi_{d-k}$ the probability of a random walk with i.i.d. increments (with zero mean and finite variance) given by the corresponding lazy random walk of the coupling never returning to the origin.  We now define the following random variable
% \begin{equation}\label{Fn'}
% |F'_n| := \sum_{i=1}^{n - \delta' n}  1_{\{U_i \leq i^{-1/2} \}}  \quad \text{where } \delta' \in (0,\pi_{d-k}) \,.
% \end{equation}
% Thus by Theorem~\ref{teo: RnZ>} (part (U)) and~\eqref{Fn<=} we obtain {\color{blue}
% \begin{equation}\label{Fn<=Fn'}
% \PP[|F_n| \le |F'_n| ] \rightarrow 1 \quad \text{as } n \to \infty \,.   
% \end{equation}}

The random variables $J_n$ and $V_n$ will be important to compute the constants $c_1$ and $c_2$ in the statement of Theorem~\ref{pn-ERW-d=>4} and below we state a few results about them. The proofs of these results are deferred to the end of this section. 
%Informally speaking, in Theorem~\ref{pn-ERW-d=>4} we obtain  that every limit point of the $p_n$-\Nametwo{} in direction $\ell_{\bD}$ suitably rescaled will be confined within a sort of ``cone'' region, with high probability (see the dashed region in  Figure~\ref{fig:cone}).


% \begin{figure}[h]
%     \centering
% \tikzset{every picture/.style={line width=0.45pt}} %set default line width to 0.75pt        

% \begin{tikzpicture}[x=0.45pt,y=0.45pt,yscale=-1,xscale=1]
% %uncomment if require: \path (0,313); %set diagram left start at 0, and has height of 313

% %Straight Lines [id:da8304803087608328] 
% \draw    (88,166) -- (182.5,283.5) ;
% %Straight Lines [id:da2794399174681552] 
% \draw    (131.5,219) -- (240.47,127.29) ;
% \draw [shift={(242,126)}, rotate = 139.92] [color={rgb, 255:red, 0; green, 0; blue, 0 }  ][line width=0.75]    (10.93,-3.29) .. controls (6.95,-1.4) and (3.31,-0.3) .. (0,0) .. controls (3.31,0.3) and (6.95,1.4) .. (10.93,3.29)   ;
% %Shape: Free Drawing [id:dp08864986808181619] 
% \draw  [color={rgb, 255:red, 0; green, 0; blue, 0 }  ,draw opacity=1 ][line width=0.75] [line join = round][line cap = round] (133,221) .. controls (132.67,221) and (132.33,221) .. (132,221) ;
% %Shape: Free Drawing [id:dp030738142652891653] 
% \draw  [color={rgb, 255:red, 0; green, 0; blue, 0 }  ,draw opacity=1 ][line width=0.75] [line join = round][line cap = round] (131,218) .. controls (131,215.81) and (131.39,207.39) .. (129,205) .. controls (127.73,203.73) and (131,211.8) .. (131,210) .. controls (131,208.74) and (128,194) .. (128,194) .. controls (128,194) and (128.47,194.73) .. (129,195) .. controls (130.07,195.54) and (131.15,193.85) .. (132,193) .. controls (134.46,190.54) and (133.07,201.11) .. (139,194) .. controls (140.93,191.69) and (137.13,188.43) .. (140,187) .. controls (141.7,186.15) and (143.38,191.16) .. (145,189) .. controls (147.51,185.65) and (146.57,175.4) .. (147,172) .. controls (147.18,170.57) and (148.8,174.43) .. (149,173) .. controls (149.33,170.69) and (148.67,168.31) .. (149,166) .. controls (149.11,165.23) and (152.37,171.78) .. (153,168) .. controls (153.7,163.8) and (152.52,155.69) .. (154,152) .. controls (154.35,151.12) and (155.48,153.22) .. (156,154) .. controls (157,155.49) and (159.24,156.35) .. (161,156) .. controls (163.76,155.45) and (164.39,141.83) .. (165,140) .. controls (165.86,137.42) and (174.34,137.91) .. (175,138) .. controls (175.47,138.07) and (175.85,139.45) .. (176,139) .. controls (177.02,135.94) and (173.72,132.56) .. (175,130) .. controls (178.22,123.57) and (178.1,119.01) .. (179,110) .. controls (179.07,109.34) and (183.98,113.09) .. (184,113) .. controls (184.32,111.71) and (183.68,110.29) .. (184,109) .. controls (184.34,107.63) and (185.74,111.37) .. (187,112) .. controls (189.11,113.05) and (191.76,110.25) .. (194,111) .. controls (196,111.67) and (198.74,114.69) .. (200,113) .. controls (203.84,107.88) and (200.1,93.9) .. (205,89) .. controls (205.6,88.4) and (215.1,94.52) .. (217,95) .. controls (221.16,96.04) and (217.77,93.07) .. (219,90) .. controls (219.49,88.77) and (225,85.56) .. (225,83) ;
% %Shape: Free Drawing [id:dp1632566533166797] 
% \draw  [color={rgb, 255:red, 0; green, 0; blue, 0 }  ,draw opacity=1 ][line width=0.75] [line join = round][line cap = round] (136,222) .. controls (134.95,222) and (134.05,223) .. (133,223) ;
% %Shape: Free Drawing [id:dp052118055244864125] 
% \draw  [color={rgb, 255:red, 0; green, 0; blue, 0 }  ,draw opacity=1 ][line width=0.75] [line join = round][line cap = round] (132,219) .. controls (131.67,218.67) and (130.85,219.55) .. (131,220) .. controls (131.09,220.28) and (138.31,224) .. (140,224) .. controls (141.99,224) and (141.41,217.8) .. (143,217) .. controls (143.99,216.51) and (147.08,221.92) .. (148,221) .. controls (148.6,220.4) and (148.68,214.95) .. (150,216) .. controls (151.67,217.33) and (152.09,220.05) .. (154,221) .. controls (156.62,222.31) and (155.9,218.9) .. (157,220) .. controls (157.99,220.99) and (157.02,221.62) .. (158,223) .. controls (160.06,225.88) and (163.49,227.49) .. (166,230) .. controls (166.24,230.24) and (166.93,230.33) .. (167,230) .. controls (167.33,228.37) and (166.73,226.64) .. (167,225) .. controls (167.36,222.83) and (168.24,225.88) .. (170,225) .. controls (170.75,224.62) and (171.07,217.38) .. (172,218) .. controls (173,218.67) and (173.15,220.15) .. (174,221) .. controls (174.79,221.79) and (177.08,218.23) .. (178,218) .. controls (179.65,217.59) and (181.48,219.76) .. (183,219) .. controls (183.84,218.58) and (184.09,217.23) .. (185,217) .. controls (186.96,216.51) and (190.55,225.07) .. (193,222) .. controls (194.51,220.11) and (195.71,214.32) .. (196,212) .. controls (196.17,210.64) and (196.14,206.93) .. (197,208) .. controls (199.06,210.58) and (201.85,217.51) .. (204,215) .. controls (206.23,212.39) and (206.14,204.1) .. (209,206) .. controls (212.53,208.35) and (214.34,212.86) .. (218,215) .. controls (221.62,217.11) and (222.14,213.09) .. (225,215) .. controls (226.28,215.85) and (231.94,222.53) .. (233,222) .. controls (234.79,221.11) and (233.25,217.98) .. (233,216) .. controls (232.91,215.26) and (232,213.25) .. (232,214) .. controls (232,218.16) and (233.37,217.73) .. (235,221) .. controls (235.21,221.42) and (235.96,222.47) .. (236,222) .. controls (236.33,218.01) and (235.5,213.97) .. (236,210) .. controls (236.17,208.64) and (236.43,212.75) .. (237,214) .. controls (238.11,216.45) and (239.44,218.81) .. (241,221) .. controls (241.43,221.61) and (242.95,222.74) .. (243,222) .. controls (243.35,216.68) and (242.52,211.31) .. (243,206) .. controls (243.15,204.31) and (243.19,209.51) .. (244,211) .. controls (245.25,213.29) and (247.49,214.88) .. (249,217) .. controls (249.27,217.38) and (249.97,218.47) .. (250,218) .. controls (250.33,213.01) and (249.67,207.99) .. (250,203) .. controls (250.08,201.8) and (252.46,202.07) .. (253,201) .. controls (254.46,198.08) and (254.03,191.19) .. (258,190) .. controls (260.24,189.33) and (262.77,190.67) .. (265,190) .. controls (270.02,188.49) and (267.92,178.51) .. (272,178) .. controls (274.32,177.71) and (276.69,177.67) .. (279,178) .. controls (281.25,178.32) and (281.47,183) .. (286,183) .. controls (286.81,183) and (287.81,180) .. (288,180) .. controls (289.16,180) and (289.79,187.62) .. (293,182) .. controls (294.16,179.97) and (292.62,177.3) .. (293,175) .. controls (293.12,174.26) and (293.27,176.85) .. (294,177) .. controls (296.88,177.58) and (301,176.75) .. (301,180) ;
% %Straight Lines [id:da19972782215377438] 
% \draw  [dash pattern={on 4.5pt off 4.5pt}]  (130,204) -- (143.5,218.75) ;
% %Straight Lines [id:da1175829093412124] 
% \draw  [dash pattern={on 4.5pt off 4.5pt}]  (142,189) -- (170,223.5) ;
% %Straight Lines [id:da006638707249905007] 
% \draw  [dash pattern={on 4.5pt off 4.5pt}]  (149,175.5) -- (192,222.5) ;
% %Straight Lines [id:da7126297398825152] 
% \draw  [dash pattern={on 4.5pt off 4.5pt}]  (152,154) -- (201,215.5) ;
% %Straight Lines [id:da45421712013082516] 
% \draw  [dash pattern={on 4.5pt off 4.5pt}]  (164,141) -- (221,215.5) ;
% %Straight Lines [id:da09458219023136327] 
% \draw  [dash pattern={on 4.5pt off 4.5pt}]  (175,131) -- (241,215.5) ;
% %Straight Lines [id:da9986396235508677] 
% \draw  [dash pattern={on 4.5pt off 4.5pt}]  (183,114.5) -- (252,200.5) ;
% %Straight Lines [id:da8110942219871895] 
% \draw  [dash pattern={on 4.5pt off 4.5pt}]  (203,109) -- (268,191.5) ;
% %Straight Lines [id:da06174397419817579] 
% \draw  [dash pattern={on 4.5pt off 4.5pt}]  (207,93) -- (280,178.5) ;
% %Straight Lines [id:da9135651086247871] 
% \draw  [dash pattern={on 4.5pt off 4.5pt}]  (132.5,193.5) -- (154,219.75) ;

% % Text Node
% \draw (250,106.4) node [anchor=north west][inner sep=0.75pt]    {$\ell_{\bD}{}$};
% % Text Node
% \draw (202,56.4) node [anchor=north west][inner sep=0.75pt]    {$W_{t} \cdot \ell _{\bD}{} +\ g( t)$};
% % Text Node
% \draw (270,198.4) node [anchor=north west][inner sep=0.75pt]    {$W_{t} \cdot \ell _{\bD}{} +\ f( t)$};

% \end{tikzpicture}
% \caption{ ``Cone'' region representation around the  direction $\ell_{\bD}$.}
% \label{fig:cone}

% \end{figure}



%\medskip 
%The first two results concern the asymptotic behavior of $J'_n/n^{1/2}$ and $F'_n/n^{1/2}$.  
%
% \begin{lemma}\label{B'_n}
% Let $\{J_n\}_{n \geq 1}$ be defined as in~\eqref{Bn'} with $\delta \in (\pi_d, 1)$ and  $\{V_n\}_{n \geq 1}$ be defined as in~\eqref{Fn'} with $\delta' \in (0, \pi_{d-k})$. Then, it holds that:
% \begin{itemize}
%     \item [(i)] 
%     \[ 
%     \lim_{n \to \infty} \frac{\EE[J_n]}{n^{1/2}} = 2\delta^{1/2} ;
%     \]
    
%     \item[(ii)] For any $\varepsilon > 0$, 
%     \begin{equation*}
%     \lim_{n \to \infty} \PP[|J_n - \EE[J_n]| > \varepsilon n^{1/2}] = 0 ;   
%     \end{equation*}
    
%      \item [(iii)]
%     \begin{equation*}
%     \lim_{n \to \infty} \frac{\EE[F_n]}{n^{1/2}} =  2(1 -\delta')^{1/2} ; 
%     \end{equation*}
    
%     \item[(iv)] For any $\varepsilon > 0$, 
%     \begin{equation*}
%     \lim_{n \to \infty} \PP[|F_n - \EE[F_n]| > \varepsilon n^{1/2}] = 0 ; 
%     \end{equation*}
    
%      \item [(v)]
%     \begin{equation*}
%     \lim_{n \to \infty} \frac{\EE[\sum_{i=1}^n  \um_{\{U_i \leq i^{-1/2} \}}]}{n^{1/2}} =  2 .
%     \end{equation*}
    
%     \item[(vi)] For any $\varepsilon > 0$, 
%     \begin{equation*}
%     \lim_{n \to \infty} \PP\Big[\Big|\sum_{i=1}^n  \um_{\{U_i \leq i^{-1/2} \}} - \EE\Big[\sum_{i=1}^n  \um_{\{U_i \leq i^{-1/2} \}}\Big]\Big| > \varepsilon n^{1/2}\Big] = 0  \,. 
%     \end{equation*}
% \end{itemize}
% \end{lemma}


\begin{lemma}\label{B'_n} \com{to be adjusted and should hold for any $d\geq 3$! }
Let $\{J_n(\delta)\}_{n \geq 1}$ be defined as in~\eqref{Bn'} with $\delta \in (\pi_d, 1)$ and  $\{V_n(\delta')\}_{n \geq 1}$ be defined as in~\eqref{Fn'} with $\delta' \in (0, \pi_{d-k})$. Then, it holds that:

\begin{align*}
i) &\quad \lim_{n \to \infty} \frac{\EE[J_n(\delta)]}{n^{1/2}} = 2\delta^{1/2} \quad \text{ and }\quad  \lim_{n \to \infty} \frac{\EE[V_n(\delta')]}{n^{1/2}} = 2-  2(1 -\delta')^{1/2}\,,
\\
   ii) &\quad \text{For any $\varepsilon > 0$,} 
   \\
   &\qquad \lim_{n \to \infty} \PP[|J_n(\delta) - \EE[J_n(\delta)]| > \varepsilon n^{1/2}] = 0\,, 
   % \text{ and } \lim_{n \to \infty} \PP[|V_n(\delta') - \EE[V_n(\delta')]| > \varepsilon n^{1/2}] = 0 \,.  
   \\
   &\quad \text{  and the same holds for $V_n(\delta')$. }
    \end{align*}
   
    %  \item [(iii)]
    % \begin{equation*}
    % \lim_{n \to \infty} \frac{\EE[V_n]}{n^{1/2}} = 2-  2(1 -\delta')^{1/2} ; 
    % \end{equation*}
    
    % \item[(iv)] For any $\varepsilon > 0$, 
    % \begin{equation*}
    % \lim_{n \to \infty} \PP[|V_n - \EE[V_n]| > \varepsilon n^{1/2}] = 0 ; 
    % \end{equation*}
\end{lemma}
\medskip 
%
From Lemma~\ref{B'_n} we obtain the following corollary. 

% \begin{corollary}\label{B'n->p}
% Let $\{J_n\}_{n \geq 1}$ be defined in~\eqref{Bn'} with $\delta \in (\pi_d, 1)$ and  $\{F_n\}_{n \geq 1}$ be defined in~\eqref{Fn'} with $\delta' \in (0, \pi_{d-k})$. Then, it holds that: 
% \begin{itemize}
%     \item [i)] 
%     \begin{equation*}
%         \frac{J_n}{n^{1/2}} \xrightarrow[n \to \infty]{} 2\delta^{1/2} \ \text{ in probability;}
%     \end{equation*}
    
%     \item [ii)]
%     \begin{equation*}
%         \frac{F_n}{n^{1/2}} \xrightarrow[n \to \infty]{} 2(1- \delta')^{1/2} \  \text{ in probability;}
%     \end{equation*}
    
%     \item [iii)]
%     \begin{equation*}
%          \frac{\sum_{i=1}^{n} \um_{\{U_i \le i^{1/2}\}}}{n^{1/2}}  \xrightarrow[n \to \infty]{} 2 \ \text{ in probability.}
%     \end{equation*}
% \end{itemize}
% \end{corollary}

 
\begin{corollary}\label{B'n->p}
\com{to be adjusted and should hold for any $d\geq 3$.....}Let $\{J_n(\delta)\}_{n \geq 1}$ be defined in~\eqref{Bn'} with $\delta \in (\pi_d, 1)$ and  $\{V_n(\delta')\}_{n \geq 1}$ be defined in~\eqref{Fn'} with $\delta' \in (0, \pi_{d-k})$. Then, it holds that: 
    \begin{equation*}
        \frac{J_n(\delta)}{n^{1/2}} \xrightarrow[n \to \infty]{} 2\delta^{1/2} \quad \text{ and } \quad  \frac{V_n(\delta')}{n^{1/2}} \xrightarrow[n \to \infty]{} 2- 2(1- \delta')^{1/2} \quad   \text{ in probability}\,.
    \end{equation*}
    % \begin{equation*}
    %  \tag{ii}   \frac{V_n}{n^{1/2}} \xrightarrow[n \to \infty]{} 2- 2(1- \delta')^{1/2} \  \text{ in probability}\,.
    % \end{equation*}
\end{corollary}

\medskip 
%
Relying on  Corollary~\ref{B'n->p}, we are able to prove the following result: 

\begin{lemma}\label{lem: tightaux} \com{to be adjusted and should hold for any $d\geq 3$.....}
Let $\{\Hat{J}^{n}_{\cdot}(\delta)\}_{n\geq 1}$ and $\{\Hat{V}^n_{\cdot}(\delta')\}_{n\geq 1}$ be respectively the sequences of processes in $C_{\Rs}[0, \infty)$ corresponding to  
$J_n(\delta)$ with $\delta \in (\pi_d, 1)$ and  $V_n(\delta')$ with $\delta' \in (0, \pi_{d-k})$  defined in~\eqref{Bn'} and \eqref{Fn'}, respectively.   Then, it holds that 
%
\begin{itemize}
    \item [i)]
    $\{\Hat{J}^n_{\cdot}(\delta)/n^{1/2}\}_{n\geq 1}$ converges in distribution as random elements of $C_{\Rs}[0, \infty)$ to the deterministic function $t \mapsto 2(\delta t)^{1/2}$, $t\ge 0$.  
    \item[ii)] $\{\Hat{V}^n_{\cdot}(\delta')/n^{1/2}\}_{n\geq 1}$
%$$
%\left\{\frac{\sum_{i=1}^{\lfloor n \cdot \rfloor}  \um_{\{U_i \leq i^{-1/2} \}} - F_{\lfloor n \cdot \rfloor}}{n^{1/2}}\right\}_{t\geq 0} 
%$$
converges in distribution as random elements of $C_{\Rs}[0, \infty)$ to the deterministic function $t \mapsto 2t^{1/2}(1\!-\!(1\!-\!\delta')^{1/2})$, $t\ge 0$.  
\end{itemize}
\end{lemma}

% The proof of Lemma~\ref{lem: tightaux} will be postponed to the end of this section.

% The next result states that $(|K_{\lfloor n\cdot \rfloor}|/n^{1/2})_{n\ge 1}$ is tight in $C_{\Rs}[0, \infty)$. %In the proof of the Theorem~\ref{pn-ERW-d=>4}, it will be more clear the importance of this result.


% \begin{lemma}\label{lem: Knt/n1/2tig}
% The sequence of processes $\{\Hat{K}^n_{\cdot }/n^{1/2}\}_{n\ge 1}$ is tight  in  $C_{\Rs}[0, \infty)$.
% \end{lemma}

By Lemma~\ref{Jntight} (item $i)$), for $d\geq 4$, every subsequence of $\{\Hat{K}^n_{\cdot }/n^{1/2}\}_{n\ge 1}$ has a limit point. The next result states that those limits points are concentrated on paths confined between the curves  $t \mapsto 2 (1-\sqrt{1 -  \pi_{d-k}}) \sqrt{t}$ and $t \mapsto 2  \sqrt{ \pi_d \, t}$.
% The proof of Lemma~\ref{lem: Knt/n1/2tig} will be postponed at the end of this section.

% \begin{proposition}\label{prop: bound_Kn}
% %Let $|K_{\lfloor n \cdot \rfloor}|/n^{1/2}$ be a sequence of processes in $C_{\Rs}[0, \infty)$. 
% If $H_{\cdot}$ is a limit point of $\{\Hat{K}^n_{\cdot }/n^{1/2}\}_{n\ge 1}$, then
% \begin{equation*}
% \PP \left[\forall t \in [0,\infty): 2t^{1/2}(1-(1 -  \delta')^{1/2}) \le H_t \le 2(t \delta)^{1/2} \right] = 1 \, ,  
% \end{equation*}
% where $\delta'$ and $\delta$ are positive constants such that $\delta' \in (0, \pi_{d-k})$ and $\delta \in (\pi_d, 1)$. 
% \end{proposition}

\begin{proposition}\label{prop: bound_Kn}
\com{This proposition should be restated separating upper and lower bound, the former is proved using Proposition~\ref{prop:RangeERW} and the second using Proposition~\ref{prop:RangeERW_lower}.....}
%Let $|K_{\lfloor n \cdot \rfloor}|/n^{1/2}$ be a sequence of processes in $C_{\Rs}[0, \infty)$. 
If $H_{\cdot}$ is a limit point of $\{\Hat{K}^n_{\cdot }/n^{1/2}\}_{n\ge 1}$, then
\begin{equation*}
\PP \left[\forall t \in [0,\infty): 2t^{1/2}(1-(1 -  \pi_{d-k})^{1/2}) \le H_t \le 2(t \pi_d)^{1/2} \right] = 1 \, . 
\end{equation*}
\end{proposition} 

\begin{remark}\label{rem:conjecture}
If Conjecture~\ref{conj_range} were true, we would be able to strengthen the claim of Proposition~\ref{prop: bound_Kn} and obtain that \com{for any $d\geq 3$.....}
\begin{equation*}
\PP \left[\forall t \in [0,\infty): 2t^{1/2}(1-(1 -  \pi_{d})^{1/2}) \le H_t \le 2(t \pi_d)^{1/2} \right] = 1 \, . 
\end{equation*}
The proof would then avoid the use of the coupling and consequently, there will be no dimensional constraint (this is why the lower bound would be given in terms of $\pi_d$ rather than the current $\pi_{d-k}$).
\end{remark}


% The proof of Proposition~\ref{prop: bound_Kn} will be postponed at the end of this section.


We now have all the auxiliaries results to prove Theorem~\ref{pn-ERW-d=>4}. The main idea is to use~\eqref{xn-incremento2} and then analyze separately the rescaled sum terms. We show  that the sequences corresponding to both terms are tight in $C_{\Rs^d}[0, \infty)$, consequently we obtain that $\{\Hat{B}_{\cdot}^n\}_{n\geq 1}$ is also tight. Finally, we  describe the processes which stochastically dominate the limit points of $\{\Hat{B}_{\cdot}^n\}_{n\geq 1}$. The strategy is similar to that used in the proof of Theorem~\ref{pn-ERW-d=2}, but here the term representing the drift direction does not go to zero. The random variables $J_n$ and $V_n$ play an important role in controlling this non-vanishing term.  
%
\begin{proof}[Proof of Theorem~\ref{pn-ERW-d=>4}]
We begin showing that 
$\{{\Hat{B}}_{\cdot}^n\}_{n\ge 1}$ is tight in $C_{\Rs^d}[0, \infty)$. By Remark~\ref{rem:conver}-$b)$ it suffices  to show that $\{{\Hat{B}}_{\cdot}^n\}_{n\ge 1}$ is tight in $C_{\Rs^d}[0, T)$, for all $T>0$.

%
Using ~\eqref{xn-incremento2} we can rewrite the process $B_t^n$ as
\begin{align}\label{p_n-ERW_incrementos_d=>4}
\begin{split} 
& \frac{1}{n^{1/2}}\sum_{i=1}^{\lfloor nt \rfloor} \xi_i + \frac{1}{n^{1/2}} \sum_{i=1}^{\lfloor nt \rfloor} \um_{\{E_{i-1}^c \cap \{ U_i \leq i^{-1/2}\}\}} (\gamma_i - \xi_i) \, , \ \forall \, t \ge 0\,.
% \\
% & = \frac{1}{n^{1/2}}\sum_{i=1}^{\lfloor nt \rfloor} \xi_i + \frac{1}{n^{1/2}} \sum_{i \in K_{\lfloor nt \rfloor}}  (\gamma_i - \xi_i) 
% \\
% & 
% = \frac{1}{n^{1/2}}\sum_{i=1}^{\lfloor nt \rfloor} \xi_i + \frac{|K_{\lfloor nt \rfloor}|}{n^{1/2}} \sum_{i \in K_{\lfloor nt \rfloor}} \frac{ (\gamma_i - \xi_i)}{|K_{\lfloor nt \rfloor}|}\,.
\end{split}
\end{align}
By Donsker's Theorem the first term of~\eqref{p_n-ERW_incrementos_d=>4} converges in distribution as random elements of $C_{\Rs^d}[0, \infty)$ to a Brownian Motion, i.e., 
\begin{equation}\label{xi_i->W2_d=>4}
 \Big\{ \frac{1}{n^{1/2}}\sum_{i=1}^{\lfloor nt \rfloor} \xi_i \Big\}_{t\ge 0} \xrightarrow[n \to \infty]{\mathcal{D}} \{ W_{t} \}_{t\ge 0} \,,
\end{equation}
where $W_{\cdot}$ is a Brownian Motion in dimension $d$ with zero-mean vector and covariance matrix $\EE[\xi_1 \xi_1^T]$.
Then,  to show that $\{{\Hat{B}}_{\cdot}^n\}_{n\ge 1}$ is tight in $C_{\Rs^d}[0, T]$ for all $T>0$, it is enough to prove that the second term in~\eqref{p_n-ERW_incrementos_d=>4} is tight in $C_{\Rs^d}[0, T]$. 
Indeed, $\{\Hat{B}_{\cdot}^n\}_{n\geq 1}$ 
 would be the sum of two tight sequences of processes, thus also tight. %(see Lemma~\ref{lem: tight}) \cm{colocar uma referencia ou deixar?}.
%
% Then by~\cite[Theorem 4.10 in Chapter 2]{karatzas2012brownian}, since $B_{\cdot}^n$ is  tight  in $C_{\Rs^d}[0, T]$ for all $T>0$ with the topology of uniform convergence in the compacts, $B_{\cdot}^n$ is  tight  in $C_{\Rs^d}[0, \infty)$.

In order to show that the second term in~\eqref{p_n-ERW_incrementos_d=>4} is tight in $C_{\Rs^d}[0, T]$ for all $T>0$,  we use Remark~\ref{rem:conver}-$c)$.
% \cite[Theorem 7.3]{billingsley1999probability} which provides two sufficient conditions for tightness.
Recall the definition of $D_{\lfloor n t \rfloor}$ from the statement of Lemma \ref{lem: tgD}, remembering that here we have set $\mathcal{C} = 1$ and that we are under distinct hypotheses from those of Section \ref{prova-pn-WGERW}. We will show that $\{\Hat{D}^n_\cdot\}_{n\ge 1}$ is tight.
%$$
%D_{\lfloor n t \rfloor}:= \frac{1}{n^{1/2}} \sum_{i=1}^{\lfloor nt \rfloor} \um_{\{E_{i-1}^c \cap \{ U_i \leq i^{-1/2}\}\}} (\gamma_i - \xi_i) \,.$$
The first condition in Remark~\ref{rem:conver}-$c)$ is satisfied, since  $\Hat{D}^n_0 \equiv 0$, for all $n\geq 1$. To prove that $\{\Hat{D}^n_\cdot\}_{n\ge 1}$ satisfies the second condition in Remark~\ref{rem:conver}-$c)$  we  use~\cite[Corollary on page 83]{billingsley1999probability} which states that the second condition of ~\cite[Theorem 7.3]{billingsley1999probability} holds if, for every positive $\varepsilon$ and $\eta$, there exists a $\phi \in (0,1)$, and an integer $n_0$ such that
\begin{equation}\label{eq: PnA}
\frac{1}{\phi} \, \PP \Big[ \sup_{t \le s \le t + \phi} \big\|\Hat{D}^n_{s} - \Hat{D}^n_{t} \big\|  \ge \varepsilon \Big]  \le \eta \quad \forall n \ge n_0\,.
\end{equation}
%\begin{equation}\label{eq: PnA}
%\frac{1}{\phi} P_n \Big[f \in C_{\Rs^d}[0,T] : \sup_{t \le s \le t+\phi} |f(s) - f(t)| \ge \varepsilon \Big] \le \eta \quad \forall n \ge n_0\,,
%\end{equation}
%where the probability measure $P_n$ on $C_{\Rs^d}[0,T]$  is the distribution of $D_{\lfloor n \cdot \rfloor}$. 
%In order to show that \eqref{eq: PnA} actually holds, note  that, if we define 
%\begin{equation}\label{eq:Atphi}
%A_t(\varepsilon, \phi):= \Big\{f \in C_{\Rs}[0,T] : \sup_{t \le s \le t+\phi} |f(s) - f(t)| \ge \varepsilon \Big\}\,,
%\end{equation} 
%the left-hand side of \eqref{eq: PnA} reduces to
% In order to show that actually holds,  let us define 
% $A_t(\varepsilon, \phi):= \{f \in C_{\Rs^d}[0,T] : \sup_{t \le s \le t+\phi} |f(s) - f(t)| \ge \varepsilon \}$. 
% Hence we obtain the following 
Note that the probability in \eqref{eq: PnA} is bounded from above by
\begin{equation}
\label{eq: PnPP}
%\PP\left[ D_{\lfloor n \cdot \rfloor} \in A_t(\varepsilon, \phi) \right] = 
%\PP \Big[ \sup_{t \le s \le t + \phi} \big\|\Hat{D}^n_{s} - \Hat{D}^n_{t} \big\|  \ge \varepsilon \Big] 
%& & = \PP \left[ \sup_{t \le s \le t+ \phi} \left\|\frac{\sum_{i=\lfloor nt \rfloor + 1}^{\lfloor ns \rfloor} 1_{\{E_{i-1}^c \cap \{ U_i \leq i^{-1/2}\}\}} (\gamma_i - \xi_i)}{n^{\frac{1}{2}}}  \right\| \ge \varepsilon \right] 
\PP\Big[ \sup_{t \le s \le t+ \phi} \Big\| \sum_{i=\lfloor nt \rfloor}^{\lfloor ns \rfloor + 1} \um_{E_{i-1}^c \cap \{ U_i \leq i^{-1/2}\}} (\gamma_i - \xi_i)  \Big\| \ge \varepsilon n^{\frac{1}{2}} \Big] \, ,
%\\
%& = \PP \left[ \sum_{i = \lfloor nt \rfloor + 1}^{\lfloor n(t + \phi) \rfloor} 1_{\{E_{i-1}^c \cap \{U_i \le i^{-\frac{1}{2}}\}\}} \ge \varepsilon n^{\frac{1}{2}} \right] \le \PP \left[ \sum_{i = \lfloor nt \rfloor + 1}^{\lfloor n(t + \phi) \rfloor} 1_{ \{U_i \le i^{-\frac{1}{2}}\}} \ge \varepsilon n^{\frac{1}{2}} \right]   
\end{equation}
%
% Now we only analyze the process inside the probability measure in~\eqref{eq: PnPP}. We will find an upper bound for this process for all the trajectory. 
for all $s \in [t, t+ \phi]$ and 
\begin{eqnarray}
\label{eq: trajetoria}
\Big\| \sum_{i=\lfloor nt \rfloor}^{\lfloor ns \rfloor + 1} \um_{E_{i-1}^c \cap \{ U_i \leq i^{-1/2}\}} (\gamma_i - \xi_i)  \Big\| &\le & \sum_{i=\lfloor nt \rfloor}^{\lfloor ns \rfloor + 1} \big\|  \um_{ \{ U_i \leq i^{-1/2}\}} (\gamma_i - \xi_i)  \big\| \nonumber
 \\
% \le  \sum_{i=\lfloor nt \rfloor + 1}^{\lfloor ns \rfloor} \big\|  \um_{ \{ U_i \leq i^{-1/2}\}} (\gamma_i - \xi_i)  \big\| 
& \le & \sum_{i=\lfloor nt \rfloor}^{\lfloor ns \rfloor + 1}  \um_{ \{ U_i \leq i^{-1/2}\}} 2K  \,, 
\end{eqnarray}
where the second inequality follows from  triangle inequality and the last from Condition~\ref{condition_A}. 
%
Then from~\eqref{eq: PnPP} and ~\eqref{eq: trajetoria} we obtain that
\begin{eqnarray}\label{eq: PnA<1}
\lefteqn{\!\!\!\!\!\!\!\! \PP \Big[ \sup_{t \le s \le t + \phi} \big\|\Hat{D}^n_{s} - \Hat{D}^n_{t} \big\|  \ge \varepsilon \Big] \le 
\PP \Big[ \sum_{i = \lfloor nt \rfloor}^{\lfloor n(t + \phi) \rfloor +1} \um_{ \{U_i \le i^{-\frac{1}{2}}\}} 2K \ge \varepsilon n^{\frac{1}{2}} \Big] } \nonumber
\\
& & \le \exp\Big(\frac{-\varepsilon n^{\frac{1}{2}}}{2K}\Big)\EE\Big[\exp\Big( \sum_{i = \lfloor nt \rfloor}^{\lfloor n(t + \phi)\rfloor+1} \um_{ \{U_i \le i^{-\frac{1}{2}}\}} \Big) \Big] 
%& & \le \exp\Big(\frac{-\varepsilon n^{\frac{1}{2}}}{2K}\Big) \prod_{i = \lfloor nt \rfloor + 1}^{\lfloor n(t + \phi)\rfloor} \EE\left[\exp\left(  \um_{ \{U_i \le i^{-\frac{1}{2}}\}} \right) \right] \,,
\end{eqnarray}
where in the last inequality we have used exponential Markov's inequality.
%
Setting  $c = \varepsilon/(2K)$ an  continuing the computation in~\eqref{eq: PnA<1} we obtain that
\begin{eqnarray}\label{eq: PnA<}
\lefteqn{\frac{1}{\phi} \PP \Big[ \sup_{t \le s \le t + \phi} \big\|\Hat{D}^n_{s} - \Hat{D}^n_{t} \big\|  \ge \varepsilon \Big] \le \frac{1}{\phi} e^{-c n^{\frac{1}{2}}} \prod_{i = \lfloor nt \rfloor}^{\lfloor n(t + \phi)\rfloor + 1} \EE\left[\exp\left(  \um_{ \{U_i \le i^{-\frac{1}{2}}\}} \right) \right]} \nonumber \\
& & = \frac{1}{\phi} e^{-c n^{\frac{1}{2}}} \prod_{i = \lfloor nt \rfloor}^{\lfloor n(t + \phi)\rfloor + 1} \left(1+ \frac{e-1}{i^{\frac{1}{2}}} \right) \le \frac{1}{\phi} e^{-c n^{\frac{1}{2}}} \prod_{i = \lfloor nt \rfloor}^{\lfloor n(t + \phi)\rfloor + 1} \exp\left(\frac{e-1}{i^{\frac{1}{2}}} \right)  
% \\
% & \le \frac{1}{\phi} e^{-c n^{\frac{1}{2}}}  \exp\left(\sum_{i = \lfloor nt \rfloor + 1}^{\lfloor n(t + \phi)\rfloor}\frac{e-1}{i^{\frac{1}{2}}} \right) 
\nonumber \\
& & \le \frac{1}{\phi} \exp(-c n^{\frac{1}{2}})\exp\left(2(e-1)(\sqrt{n(t + \phi)} - \sqrt{nt} + 2) \right) \,,  
\end{eqnarray}
where %the second inequality follows by the moment generating function of a Bernoulli  and the third by the fact that $1+x<e^x$ for all $x$.  
the last inequality above follows from noticing that 
\begin{align*}
\sum_{i = \lfloor nt \rfloor}^{\lfloor n(t + \phi)\rfloor + 1}\frac{1}{i^{\frac{1}{2}}} & \le \int_{nt - 1}^{n(t+\phi) + 1} x^{-1/2}dx \le 2\left(\sqrt{n(t + \phi)} - \sqrt{nt} + 2\right) \,.
\end{align*}
Therefore, in order to show that \eqref{eq: PnA}  holds, it remains to show that for every positive $\varepsilon$ (recall that $c=\varepsilon/(2K)$) and $\eta$,  there exists a $\phi \in (0,1)$, and an integer $n_0$ such that
\begin{equation}\label{exp<eta}
\frac{1}{\phi} \exp(-c n^{\frac{1}{2}})\exp\bigg(2(e-1)\left(\sqrt{n}\left(\sqrt{t + \phi} - \sqrt{t} \right) + 2 \right) \bigg) \le \eta \quad \forall n \ge n_0 \,.
\end{equation}
We accomplish this choosing $\phi \in (0,1)$ sufficiently small such that $\sqrt{t+\phi} - \sqrt{t}< c/4(e-1)$. Then, for every $\eta>0$, choosing $n$ sufficiently large, we obtain that~\eqref{exp<eta} is satisfied. 
% One can see that, since we have $\phi \in (0,1)$,  for all $\hat{\varepsilon} > 0$, there exists a $\phi' > \phi$ such that $|\sqrt{t+\phi'} - \sqrt{t}|< \hat{\varepsilon}$. Now we can choose $\hat{\varepsilon} = c/4(e-1)$ and we obtain, for a large enough $n$, that~\eqref{exp<eta} is fulfilled for all $\eta$.
Consequently, we have that~\eqref{eq: PnA} is satisfied and thus $\{\Hat{D}^n_{\cdot}\}_{n \ge 1}$ is tight in $C_{\Rs^d}[0,T]$. 
 
% Since the process $B_{\cdot}^n$  is the sum of two tight processes in $C_{\Rs^d}[0, T]$,  we obtain that $B_{\cdot}^n$ is a tight process in $C_{\Rs^d}[0, T]$ for all $T>0$ as a simple exercise. %(see Lemma~\ref{lem: tight}) \cm{colocar uma referencia ou deixar?}.

% Now by~\cite[Theorem 4.10 in Chapter 2]{karatzas2012brownian} one can see that since $B_{\cdot}^n$ is  tight  in $C_{\Rs^d}[0, T]$ for all $T>0$ with the topology of uniform convergence in the compacts then  $B_{\cdot}^n$ is  tight  in $C_{\Rs^d}[0, \infty)$.

We now prove the second part of the theorem, namely the stochastic domination. Let us begin rewriting $B_t^n$ again in the slightly different form
\begin{equation}
 \label{p_n-ERW_incrementos_d=>4-bis}
% & = \frac{1}{n^{1/2}}\sum_{i=1}^{\lfloor nt \rfloor} \xi_i + \frac{1}{n^{1/2}} \sum_{i=1}^{\lfloor nt \rfloor} \um_{\{E_{i-1}^c \cap \{ U_i \leq i^{-1/2}\}\}} (\gamma_i - \xi_i) 
% % \\
% % & = \frac{1}{n^{1/2}}\sum_{i=1}^{\lfloor nt \rfloor} \xi_i + \frac{1}{n^{1/2}} \sum_{i \in K_{\lfloor nt \rfloor}}  (\gamma_i - \xi_i) 
% \\
% & 
\frac{1}{n^{1/2}}\sum_{i=1}^{\lfloor nt \rfloor} \xi_i + \frac{|K_{\lfloor nt \rfloor}|}{n^{1/2}} \sum_{i \in K_{\lfloor nt \rfloor}} \frac{ (\gamma_i - \xi_i)}{|K_{\lfloor nt \rfloor}|}\,.
\end{equation}
As already mentioned the first term converges to a Brownian Motion (see \eqref{xi_i->W2_d=>4}).  We now analyze the second term in~\eqref{p_n-ERW_incrementos_d=>4-bis}. By Proposition~\ref{prop: bound_Kn} we have  that $|K_{\lfloor nt \rfloor}| \to \infty$ as $n \to \infty$ almost surely. 
% \begin{align}\label{sup_
% inf_Kn}
% \PP \left[\forall t \in [0,\infty): 2t^{1/2}(1-(1 -  \pi_{d-k})^{1/2}) \le H_t \le 2(t \pi_d)^{1/2}  \right]  = 1 \,,
% \end{align}
% where $\{H_t\}_{t\ge 0}$ is a limit point of a subsequence of $\{ \Hat{K}^n_{\cdot}/n^{1/2}\}_{n\geq 1}$.
%\com{Above $\delta --> \delta' $ and  $\hat\delta --> \dekta $ right?}
%$\delta''$, $\Hat{\delta}$, $\delta'$ and $\delta$ are positive constants such that $\delta'' \in (0, \delta')$, $\Hat{\delta} \in (\delta, 1]$, $\delta \in (\pi_d, 1]$ and $\delta' \in (0, \pi_{d-k})$.
% \begin{equation}\label{Fn<Kn<Bn}
% \frac{1}{n^{1/2}} \sum_{i=1}^{\lfloor nt \rfloor}  1_{\{U_i \leq i^{-1/2} \}} - \frac{|F_{\lfloor nt \rfloor}|}{n^{1/2}} \preceq \frac{|K_{\lfloor nt \rfloor}|}{n^{1/2}} \preceq \frac{|J_{\lfloor nt \rfloor}|}{n^{1/2}}\,.
% \end{equation}
% Thus, by~\eqref{Bn<=Bn'} and~\eqref{Fn<Kn<Bn}, one can see that
% \begin{align}\label{eq:Kn<B'n}
% \begin{split}
% & \PP \left[ \forall t \in [0,\infty) : \frac{|K_{\lfloor nt \rfloor}|}{n^{1/2}} \le \frac{|J'_{\lfloor nt \rfloor}|}{n^{1/2}} \right] \ge \PP\left[ \forall t \in [0,\infty]: \frac{|J_{\lfloor nt \rfloor}|}{n^{1/2}} \le \frac{|J'_{\lfloor nt \rfloor}|}{n^{1/2}} \right] 
% \\
% & \to 1 \quad \text{as } n \to \infty\,.
% \end{split}
% \end{align}
% Now we will obtain, in the same sense of~\eqref{eq:Kn<B'n}, a lower bound. Hence by~\eqref{Fn<=Fn'} and~\eqref{Fn<Kn<Bn} we have
% \begin{equation}\label{eq: Kn>s-F'n}
% \begin{split}
% & \PP \left[\forall t \in [0,\infty) : \frac{\sum_{i=1}^{\lfloor nt \rfloor}  1_{\{U_i \leq i^{-1/2} \}}}{n^{1/2}} - \frac{|F'_{\lfloor nt \rfloor}|}{n^{1/2}} \le \frac{|K_{\lfloor nt \rfloor}|}{n^{1/2}}  \right] 
% \\
% & \ge \PP\left[\forall t \in [0,\infty): \frac{\sum_{i=1}^{\lfloor nt \rfloor}  1_{\{U_i \leq i^{-1/2} \}} - |F'_{\lfloor nt \rfloor}|}{n^{1/2}} \le \frac{\sum_{i=1}^{\lfloor nt \rfloor}  1_{\{U_i \leq i^{-1/2} \}} - |F_{\lfloor nt \rfloor}|}{n^{1/2}} \right] 
% \\
% & \to 1 \quad \text{as } n \to \infty \,.
% \end{split}
% \end{equation}
% Hence by Lemma~\ref{lem: tightaux}, Corollary~\ref{B'n->p}, ~\eqref{eq:Kn<B'n} and~\eqref{eq: Kn>s-F'n} we obtain that for all $t_0>0$
% \begin{align}\label{sup_
% inf_Kn}
% \PP \left[\forall t \in [t_0,\infty): 2t^{1/2}(1-(1 -  \delta')^{1/2}) \le \frac{|K_{\lfloor nt \rfloor}|}{n^{1/2}} \le 2(t \delta)^{1/2} \right] \to 1 \,,
% \end{align}
% as $n$ goes to infinity where $\delta$ and $\delta'$ are positive constants such that $\delta \in (\pi_d, 1]$ and $\delta' \in (0, \pi_{d-k})$.
%
Furthermore,  the sequence of random vectors $\{\gamma_{\varphi_i} -\xi_{\varphi_i}\}_{i \geq 1}$ is i.i.d. and has the same distribution of $\{\gamma_{i} -\xi_{i}\}_{i \geq 1}$, which is i.i.d. too (see Lemma~\ref{lem: iid}). Thus, we can use~\cite[Theorem 8.2 item (iii)]{gut2005probability} to conclude that 
\begin{equation}\label{eq:prob02_d=>4}
   \sum_{i \in K_{\lfloor nt \rfloor}} \frac{ (\gamma_i - \xi_i)}{|K_{\lfloor nt \rfloor}|} \xrightarrow[n \to \infty]{} \EE[\gamma_1 -\xi_1]  = \EE[\gamma_1] \, \text{ a.s..}
\end{equation}
Recall that $\{\gamma_n\}_{n \ge 1}$ is an i.i.d. sequence of random vectors and  $\lambda  \le \EE[\gamma_i \cdot \ell_{\bD}] \le K$ for all $i \ge 1$, we set $\mu_{\gamma} := \EE[\gamma_i \cdot \ell_{\bD}]$ for all $i \ge 1$. 
%\cm{Now we will prove that the second sum portion in~\eqref{p_n-ERW_incrementos_d=>4} is a tight process in $C[0,T]$. For that we will use again Theorem 7.3 in~\cite{billingsley1999probability}
%\begin{equation*}
%\begin{split}
%& \PP\left[ \sup_{t \le s \le t+ \phi} \left\| \sum_{i=\lfloor nt \rfloor + 1}^{\lfloor ns \rfloor} 1_{ \{ U_i \leq i^{-1/2}\}} (\gamma_i - \xi_i)  \right\| \ge \varepsilon n^{\frac{1}{2}} \right] \le
%\\
%& \PP\left[ \sup_{t \le s \le t+ \phi} \left( \sum_{i=\lfloor nt \rfloor + 1}^{\lfloor ns \rfloor} \left\|  1_{ \{ U_i \leq i^{-1/2}\}} (\gamma_i - \xi_i)  \right\| \right) \ge \varepsilon n^{\frac{1}{2}} \right] \le
%\\
%& \PP\left[ \sup_{t \le s \le t+ \phi} \left( \sum_{i=\lfloor nt \rfloor + 1}^{\lfloor ns \rfloor}   1_{ \{ U_i \leq i^{-1/2}\}} 2K \right) \ge \varepsilon n^{\frac{1}{2}} \right] \le
%\\
%& \PP\left[  \sum_{i=\lfloor nt \rfloor + 1}^{\lfloor n(t+\phi) \rfloor}   1_{ \{ U_i \leq i^{-1/2}\}} 2K  \ge \varepsilon n^{\frac{1}{2}} \right] \,.
%\end{split}    
%\end{equation*}}
%
From Proposition~\ref{prop: bound_Kn} and~\eqref{eq:prob02_d=>4} we obtain that the linearly interpolated version of the sequence 
$$
\left\{ \frac{|K_{\lfloor n \cdot \rfloor}|}{n^{1/2}} \sum_{i \in K_{\lfloor nt \rfloor}} \frac{ (\gamma_i - \xi_i) \cdot \ell_{\bD}}{|K_{\lfloor n \cdot \rfloor}|} \right\}_{n\ge 1}\,, 
$$
is tight in $C_{\Rs}[0, \infty)$ and any of its limit points $\widetilde H_t$ satisfies
\begin{equation}\label{eq: c1c2}
\PP\left[\forall t \in [0,\infty): 
2\mu_{\gamma} t^{\frac{1}{2}}(1 - (1 - \pi_{d-k})^{\frac{1}{2}})
\le \widetilde H_t \le 2\mu_{\gamma}(t \pi_d)^{1/2} \right] = 1\,.
\end{equation}

%\begin{equation}\label{eq: c1c2}
%\begin{split}
%& \PP\left[\forall t \in [0,\infty): \frac{|K_{\lfloor nt \rfloor}|}{n^{1/2}} \sum_{i \in K_{\lfloor nt \rfloor}} \frac{ (\gamma_i - \xi_i) \cdot \ell_{D_k}}{|K_{\lfloor nt \rfloor}|}  \le 2\mu_{\gamma}(t\delta)^{1/2} \right] \to 1 \quad \text{and}
%\\
%& \PP\left[\forall t \in [0,\infty): \frac{|K_{\lfloor nt \rfloor}|}{n^{1/2}} \sum_{i \in K_{\lfloor nt \rfloor}} \frac{ (\gamma_i - \xi_i) \cdot \ell_{D_k}}{|K_{\lfloor nt \rfloor}|} \ge 2t^{\frac{1}{2}}(1 - (1 - \delta')^{\frac{1}{2}})\mu_{\gamma}\right] \to 1 \,,
%\end{split}   
%\end{equation}
%as $n$ goes to infinity.

Since $\{\Hat{B}_{\cdot}^n\}_{n\geq 1}$ is  tight in $C_{\Rs^d}[0, \infty)$, thus relatively compact by Prohorov's Theorem (see, e.g., ~\cite[Theorem 5.1]{billingsley1999probability}) and consequently every subsequence has a limit point.
%
By~\eqref{xi_i->W2_d=>4} and~\eqref{eq: c1c2} for any of those limit points $\{Y_{t}\}_{t\ge 0}$ we have that for all $t \in [0, \infty)$
\begin{equation*}
\{W_t \cdot \ell_{D_k} + 2 c_1 \sqrt{t}\}_{t\ge 0} \preceq \{Y_t \cdot \ell_{D_k}\}_{t\ge 0} \preceq \{W_t \cdot \ell_{D_k} + 2 c_2 \sqrt{t} \}_{t\ge 0} \,,
\end{equation*}
where $c_1 = (1 - \sqrt{1 - \pi_{d-k}})\mu_{\gamma}$ and $c_2 = \sqrt{\pi_d} \, \mu_{\gamma}$ (and $0 < c_1 \le c_2$). 
%\com{here $\delta'' --> \delta $ and $\hat \delta --> \delta $?}
%
\end{proof}



\begin{remark}\label{rem:restriction}
Theorem~\ref{pn-ERW-d=>4} has some restrictions on the dimension and in the drift direction.  Those restrictions are due to the technique we used to prove Proposition~\ref{prop: bound_Kn}, which essentially consists in lower bounding the range of the $p_n$-\Nametwo{} with the range of a lazy random walk and then using Theorem~\ref{teo: RnZ>} (LLN). For this to work properly the lazy random walk should be at least $3$ dimensional, thus the dimensional restrictions. 
% ( The coupling of the $p_n$-\Nametwo{} and a lazy random walk had a fundamental part in the proof of Theorem~\ref{pn-ERW-d=>4}. Indeed, with this technique, we could find a lower bound for the range of the $p_n$-\Nametwo{} and describe the functions $c_1 t^{1/2}$ and $c_2t^{1/2}$. 
% However, this method breaks down when, for example, we have  a $p_n$-\Nametwo{} in direction $\ell \in \mathbb{S}^{d-1}$ on $\ZZ^d$, and $\ell$ has to be written with more than $d-3$ canonical directions of $\ZZ^d$. This is due to the fact that for the proof of Theorem~\ref{pn-ERW-d=>4} to work, the lazy random walk, which is used to lower bound  the range of the $p_n$-\Nametwo{}, should be at least  $3$ dimensional.  
% We will now propose a conjecture about the range of the $p_n$-\Nametwo{} in $\ZZ^d$, with $d \ge 3$ and in direction $\ell \in \mathbb{S}^{d-1}$, which would be  helpful to extend the Theorem~\ref{pn-ERW-d=>4}.

% \com{Rodrigo essa conjetura á para $\beta=1/2$ ne? O que conjeturamos em $d=2$ e $\beta=1/2$? Talvez seja mesmo melhor colocar essa conjetura no main result! e so explicar aqui como essa conjetura melhoraria o Teorema 1.6 }
% \begin{conjecture}\label{conj_range}
% Let $X$ be a $p_n$-\Nametwo{} in $\ZZ^d$, with $d \ge 3$ and in direction $\ell \in \mathbb{S}^{d-1}$ which can be written as in~\eqref{xn-incremnto1}. Then we have
% \begin{equation*}
%     \frac{|\Rr_n^X|}{n} \to \pi_d \quad \text{as } n \to \infty \text{ a.s.,}
% \end{equation*}
% where $\pi_d$ is the probability of $\{\xi_i\}_{i \ge 0}$ never returning to the origin. 
% \end{conjecture}
%
%An idea to prove the Conjecture~\ref{conj_range} is to find the same upper and lower bound. For the upper bound we can use the same arguments that we use in the proof of Proposition~\ref{prop: RnX<d3}. \cm{Now the lower bound is the actual problem. Seems to us that it could be found using similar technique from Spitzer for range of a random walk with i.i.d increments.}    
%
% \com{I would remove the following paragraph....}
% If Conjecture~\ref{conj_range} (about the range of the $p_n$-\Nametwo{}) holds true (see, Section~\ref{main_pn}),  it would be possible to prove Theorem~\ref{pn-ERW-d=>4} for a $p_n$-\Nametwo{} on $\ZZ^d$, with $d \ge 3$ and any direction $\ell \in \mathbb{S}^{d-1}$,   avoiding the use of the coupling that we described at the beginning of this section. 
% %
% Specifically,  it would be possible to prove a stronger version of Proposition~\ref{prop: bound_Kn} (see, Remark~\ref{rem:conjecture}) and consequently extend  Theorem~\ref{pn-ERW-d=>4} to  dimension $3$.  
\end{remark}


\begin{proof}[Proof of Lemma~\ref{Jntight}] 
Item $ii)$: 
 By   Remark~\ref{rem:conver}-$a)$,  it is enough to prove that the sequence of processes $\{\Hat{K}^n_{\cdot}/n^{1/2}\}_{n\geq 1}$ converges in probability as random elements of $C_{\Rs}[0, T)$ to the  zero function, for all $T > 0$. For the latter it suffices to prove  
 convergence in probability of $\sup_{0 \le t \le T} |K_{\lfloor nt \rfloor}|/n^{1/2}$ to zero for all $T > 0$.
% , since convergence in $C_{\Rs^d}[0, T]$ for all $T >0$ implies convergence in $C_{\Rs^d}[0, \infty)$ under the metric $\rho$ 
% \com{...HERE!!}
% . 
To this aim, let us define  
$$G_n:=\Big\{\sup_{0 \le t \le T} |K_{\lfloor nt \rfloor}| > \varepsilon \sqrt{n}\Big\} = \left\{ |K_{\lfloor nT \rfloor}| > \varepsilon \sqrt{n} \right\}\, ,
$$
For every $\varepsilon > 0$ and $\delta>0$, %consider the following event $G = \{|K_{\lfloor nT \rfloor}| > \varepsilon \sqrt{n}\}$. 
by Markov's inequality, we have that 
\begin{align*}\label{eq: Jnprob}
\begin{split}
& \PP[ G_n ] = \PP\big[
G_n \cap \{|\Rr_{\lfloor nT \rfloor} ^X| > \delta \lfloor nT \rfloor\}\big] +  \PP\big[ G_n \cap \{|\Rr_{\lfloor nT \rfloor}^X| \leq \delta \lfloor nT \rfloor\}\big]
\\
& \leq \PP\big[|\Rr_{\lfloor nT \rfloor} ^X| > \delta \lfloor nT \rfloor\big] + \PP\Big[ \sum_{i=1}^{\lceil \delta n T\rceil} \um_{\{U_i \leq i^{-1/2} \}} > \varepsilon\sqrt{n} \Big] 
\\
& \leq \PP\big[|\Rr_{\lfloor nT \rfloor} ^X| > \delta \lfloor nT \rfloor\big] + \frac{1}{\varepsilon \sqrt{n}} \sum_{i=1}^{\lceil \delta nT\rceil} \frac{1}{i^{1/2}}\,. \end{split}
\end{align*}
Using  Proposition~\ref{prop:RangeERW}, we have that, for all sufficiently large $n$,  $
    \PP[ |\Rr_n ^X| \leq \delta n ] = 1$, for every  $\delta > \pi_d$. Since for $d=2$, $\pi_d =0$,  we obtain that   $\lim_{n \to \infty}\PP\big[|\Rr_{\lfloor nT \rfloor} ^X| > \delta \lfloor nT \rfloor\big]=0$, for every $\delta>0$. Moreover, noticing that   $\sum_{i=1}^{\lceil \delta n\rceil} \frac{1}{i^{1/2}} = \Theta(\lceil \delta n\rceil^{1/2})$, we conclude that
\begin{equation}\label{eq: Jnprob2}
\begin{split}
& \limsup_{n \to \infty}  \PP[ G_n ]  
%\leq \limsup_{n \to \infty} \Big( \PP\big[|\Rr_{\lfloor nT \rfloor}^X| > \delta \lfloor nT \rfloor\big] + \frac{1}{\varepsilon \sqrt{n}} \sum_{i=1}^{\lceil \delta nT \rceil} \frac{1}{i^{1/2}} \Big) \\ & \leq \limsup_{n \to \infty} \PP\big[|\Rr_{\lfloor nT \rfloor}^X| > \delta \lfloor nT \rfloor\big] + \limsup_{n \to \infty} \Big( \frac{1}{\varepsilon \sqrt{n}} \sum_{i=1}^{\lceil \delta nT \rceil} \frac{1}{i^{1/2}} \Big) 
\leq \frac{c' (\delta T)^{1/2}}{\varepsilon}\,.
\end{split}
\end{equation}
Since $\delta>0$ is arbitrary,  $\limsup_{n \to \infty} \PP[ G_n ] = 0$ for every $\varepsilon$ fixed. Therefore, for all $T > 0$ the sequence of processes $\left\{\Hat{K}^n_{\cdot}/n^{1/2}\right\}_{n\geq 1}$ converges in probability, as random elements of $C_{\Rs}[0,T]$, to the  zero function.  %(see Lemma~\ref{lem: convprob}) \cm{pensar em uma referencia ou não ha necessidade?}. 

\medskip 
Item $i)$: 
% The proof follows the very same lines of that of Theorem~~\ref{pn-ERW-d=>4}, and we omit it. 
%
By  Remark~\ref{rem:conver}-$b)$  it suffices to show tightness in $C_{\Rs}[0, T]$ for all $T > 0$ and this is equivalent to show the sequence of processes $\left\{\Hat{K}^n_{\cdot}/n^{1/2}\right\}_{n\geq 1}$  satisfies  the two conditions in  Remark~\ref{rem:conver}-$c)$. 
Note that $\Hat{K}^n_{0}/n^{1/2}\equiv 0$ for all $n \ge 1$ and therefore the first condition in Remark~\ref{rem:conver}-$c)$ is satisfied.
%
To prove that $\{\Hat{K}^n_\cdot /n^{1/2}\}_{n\ge 1}$ satisfies the second condition in Remark~\ref{rem:conver}-$c)$  we  use~\cite[Corollary on page 83]{billingsley1999probability} which states that the second condition of ~\cite[Theorem 7.3]{billingsley1999probability} holds if, for every positive $\varepsilon$ and $\eta$, there exists a $\phi \in (0,1)$, and an integer $n_0$ such that
\begin{equation*}
\frac{1}{\phi} \, \PP \Big[ \sup_{t \le s \le t + \phi} \big\|\Hat{K}^n_{s} - \Hat{K}^n_{t} \big\|  \ge \varepsilon n^{1/2} \Big]  \le \eta \quad \forall n \ge n_0\,.
\end{equation*}
From this point on, using that 
\[
\PP \Big[ \sup_{t \le s \le t + \phi} \big\|\Hat{K}^n_{s} - \Hat{K}^n_{t} \big\|  \ge \varepsilon n^{1/2} \Big]  \le \PP \Big[ \sum_{i = \lfloor nt \rfloor }^{\lfloor n(t + \phi) \rfloor +1} \um_{\{U_i \le i^{-\frac{1}{2}}\}} \ge \varepsilon n^{\frac{1}{2}} \Big]\,,
\]
the computations are very similar to those used in the proof of Theorem~\ref{pn-ERW-d=>4}, and we omit it. 
\end{proof}





\begin{proof}[Proof of Lemma~\ref{B'_n}.] To avoid clutter in the notation, we write $J_n$ and $V_n$, thus omitting the dependence on $\delta$ and $\delta'$.  
As far as $J_n$ is concerned, we have 
\begin{equation*}
\frac{\EE[J_n]}{n^{1/2}} = \frac{1}{n^{1/2}}\EE\Big[\sum_{i=1}^{\delta n} \um_{\{U_i \leq i^{-1/2} \}}\Big] = \frac{1}{n^{1/2}}\sum_{i=1}^{\delta n} i^{-1/2} \xrightarrow[n \to \infty]{} 2\delta^{1/2}  \,. 
\end{equation*}
Also,  using Chebyshev's inequality and the independence of the random variables $\{U_i\}_{i \geq 1}$ we have that 
\begin{align*}
\PP \big[ |J_n - \EE[J_n]| > \varepsilon n^{1/2} \big] &\leq \frac{1}{\varepsilon^2 n} \text{Var}\Big[ \sum_{i=1}^{\delta n} \um_{\{U_i \leq i^{-1/2} \}} \Big]
\\
&=\frac{1}{\varepsilon^2 n} \sum_{i=1}^{\delta n} \frac{1}{i^{1/2}}\left(1-\frac{1}{i^{1/2}}\right) \xrightarrow[n \to \infty]{} 0  \,. 
\end{align*}

The proofs for $V_n$ are similar once we write
\[
V_n = \underbrace{\sum_{i=1}^n  \um_{\{U_i \leq i^{-1/2} \}}}_{=:I_n} - \underbrace{\sum_{i=1}^{n -\delta' n} \um_{\{U_i \leq i^{-1/2} \}}}_{=:F_n'}\,,
\]
and observe that  
\begin{align*}
&\frac{1}{n^{1/2}}\EE[I_n] = \frac{1}{n^{1/2}}\sum_{i=1}^{n} i^{-1/2} \xrightarrow[n \to \infty]{} 2\,,
\\
&\frac{1}{n^{1/2}}\EE[F'_n] = \frac{1}{n^{1/2}}\sum_{i=1}^{n - \delta' n} i^{-1/2} \xrightarrow[n \to \infty]{} 2(1-\delta')^{1/2} \,, 
\end{align*}
and that, by Chebyshev's inequality and the independence of the random variables $\{U_i\}_{i \geq 1}$, it holds that 
\begin{align*}
\PP & [|I_n - \EE[I_n]| > \varepsilon n^{1/2}] 
 \leq \frac{1}{\varepsilon^2 n} \text{Var}\Big[ \sum_{i=1}^{n} \um_{\{U_i \leq i^{-1/2} \}} \Big] 
% = 
% \frac{1}{\varepsilon^2 n} \sum_{i=1}^{n} i^{-1/2}(1-i^{-1/2})
\xrightarrow[n \to \infty]{} 0  
\,,
\\
\PP &\big[ |F'_n - \EE[F'_n]| > \varepsilon n^{1/2} \big] \leq \frac{1}{\varepsilon^2 n} \text{Var}\Big[ \sum_{i=1}^{n -\delta' n} \um_{\{U_i \leq i^{-1/2} \}} \Big]
% \\
% &
% =  \frac{1}{\varepsilon^2 n} \sum_{i=1}^{n - \delta' n} i^{-1/2}(1-i^{-1/2}) 
\xrightarrow[n \to \infty]{} 0 
\,.
\end{align*}
% The proof of point $v)$ is also straightforward: 
% \begin{equation*}
% \frac{\EE[\sum_{i=1}^n  \um_{\{U_i \leq i^{-1/2} \}}]}{n^{1/2}} = \frac{1}{n^{1/2}}\sum_{i=1}^{n} i^{-1/2} \xrightarrow[n \to \infty]{} 2 
% \,.  
% \end{equation*}
% For the proof of point $vi)$ we use Chebyshev's inequality and the independence of the random variables $\{U_i\}_{i \geq 1}$ and we obtain
% \begin{align*}
% % \label{eq: P}
% % \begin{split}
% \PP & \Big[\Big|\sum_{i=1}^n  \um_{\{U_i \leq i^{-1/2} \}} - \EE\Big[\sum_{i=1}^n  \um_{\{U_i \leq i^{-1/2} \}}\Big]\Big| > \varepsilon n^{1/2}\Big] 
%  \leq \frac{1}{\varepsilon^2 n} \text{Var}\Big[ \sum_{i=1}^{n} \um_{\{U_i \leq i^{-1/2} \}} \Big] 
%  \\
% &= \frac{1}{\varepsilon^2 n} \sum_{i=1}^{n} i^{-1/2}(1-i^{-1/2}) \to 0  \quad \text{as } n \to \infty 
% \,.  
% \end{align*}
\end{proof}

\begin{proof}[Proof of Lemma~\ref{lem: tightaux}.] To avoid clutter in the notation we omit the dependence on $\delta$ and $\delta'$. By Corollary~\ref{B'n->p}  we already have the convergence of the finite-dimensional distributions. Therefore,  by~\cite[Theorem 7.1]{billingsley1999probability},  to prove points $i)$ and $ii)$ it only remains to prove that both sequences of processes are tight in $C_{\Rs}[0, \infty)$. By Remark~\ref{rem:conver}-$b)$ we only need to prove tightness in $C_{\Rs}[0, T]$ for all $T>0$. The proof strategy is analogous to the one used in the proof of Theorem~\ref{pn-ERW-d=>4} which relies on Remark~\ref{rem:conver}-$c)$.  %
% when we show the second sum portion in~\eqref{p_n-ERW_incrementos_d=>4} is tight in $C_{\Rs}[0, T]$, for all $T>0$.

Item $i)$:  $\{\Hat{J}^n_{\cdot }/n^{1/2}\}_{n\geq 1}$ satisfies the first condition in Remark~\ref{rem:conver}-$c)$, since $\Hat{J}^n_0 = 0$ for all $n\geq 1$.  To prove the second condition in Remark~\ref{rem:conver}-$c)$, as in \eqref{eq: PnA} we need to show that for every positive $\varepsilon$ and $\eta$, there exists a $\phi \in (0,1)$, and an integer $n_0$ such that
\begin{equation}\label{eq: PnA2}
\frac{1}{\phi} \, \PP \Big[ \sup_{t \le s \le t + \phi} \big\|\Hat{J}^n_{s} - \Hat{J}^n_{t} \big\| \ge \varepsilon \Big]  \le \eta \quad \forall n \ge n_0\,.
\end{equation}
Following basically same steps as in the proof of Theorem~\ref{pn-ERW-d=>4}, we have that
\begin{equation*}
\begin{split}
&\frac{1}{\phi} \PP \Big[ \sup_{t \le s \le t + \phi} \big\|\Hat{J}^n_{s} - \Hat{J}^n_{t} \big\| \ge \varepsilon \Big] \le 
 \frac{1}{\phi} \PP \Big[ \sum_{i = \delta\lfloor nt \rfloor}^{\delta\lfloor n(t + \phi) \rfloor +1} \um_{ \{U_i \le i^{-\frac{1}{2}}\}} \ge \varepsilon n^{\frac{1}{2}} \Big] 
% \\
 %& \leq  \frac{1}{\phi} e^{-\varepsilon n^{\frac{1}{2}}}\EE\Big[\exp\Big( \sum_{i = \delta\lfloor nt \rfloor + 1}^{\delta\lfloor n(t + \phi)\rfloor} \um_{ \{U_i \le i^{-\frac{1}{2}}\}} \Big) \Big] 
% \\
% & = \frac{1}{\phi} e^{-\varepsilon n^{\frac{1}{2}}} \prod_{i = \delta\lfloor nt \rfloor + 1}^{\delta\lfloor n(t + \phi)\rfloor} \left(1+ \frac{e-1}{i^{\frac{1}{2}}} \right) \le \frac{1}{\phi} e^{-\varepsilon n^{\frac{1}{2}}} \prod_{i = \delta\lfloor nt \rfloor + 1}^{\delta\lfloor n(t + \phi)\rfloor} \exp\left(\frac{e-1}{i^{\frac{1}{2}}} \right)  
% \\
% & \le \frac{1}{\phi} e^{-c n^{\frac{1}{2}}}  \exp\left(\sum_{i = \lfloor nt \rfloor + 1}^{\lfloor n(t + \phi)\rfloor}\frac{e-1}{i^{\frac{1}{2}}} \right) 
\\
& \le \frac{1}{\phi} \exp(-\varepsilon n^{\frac{1}{2}})\exp\left(2(e-1)(\sqrt{\delta n(t + \phi)} - \sqrt{\delta nt} + 2) \right) \,.
\end{split}    
\end{equation*}
and we obtain \eqref{eq: PnA2} choosing $\phi \in (0,1)$ sufficiently small such that $\sqrt{t+\phi} - \sqrt{t}< \varepsilon/4\sqrt{\delta}(e-1)$ for all $t \in [0, T]$. 
% \begin{equation*}
% \frac{1}{\phi} \exp(-\varepsilon n^{\frac{1}{2}})\exp\bigg(2(e-1)\sqrt{n\delta}\left(\sqrt{t + \phi} - \sqrt{t}\right) \bigg) \le \eta \quad \forall n \ge n_0 \,.
% \end{equation*}
% The latter can be easily verified as it was done in the proof of Theorem~\ref{pn-ERW-d=>4}. 

% % From now on the proof follows exactly  as in Theorem~\ref{pn-ERW-d=>4}, when we prove the second sum portion in~\eqref{p_n-ERW_incrementos_d=>4} fulfills the second condition of~\cite[Theorem 7.3] {billingsley1999probability}. \comu{precisa ser mais específico aqui} Then we have that for each positive $\varepsilon$ and $\eta$, there exists a $\phi \in (0,1)$, and an integer $n_0$ such that
% % \begin{equation*}
% % \frac{1}{\phi} \PP [|J'_{\lfloor n \cdot \rfloor}|/n^{1/2} \in A_t(\varepsilon, \phi)] \le \eta \quad \forall n \ge n_0\,.
% % \end{equation*}
% % Ergo by~\cite[Theorem 7.3]{billingsley1999probability} we obtain that the sequence $|J_{\lfloor n \cdot \rfloor}|/n^{1/2}$ is a tight in $C_{\Rs}[0, T]$ for all $T > 0$ with the topology of uniform convergence in compacts and moreover by~\cite[Theorem 2.4.10]{karatzas2012brownian} \com{here we cite a different result than the one cited in the remark!} is a tight sequence of processes in $C_{\Rs}[0, \infty)$. 
The proof of item $ii)$ is similar with the only difference that we analyze separately $\sum_{i=1}^{\lfloor n \cdot \rfloor} \um_{ \{U_i \le i^{-\frac{1}{2}}\}}/n^{1/2}$ and    $\sum_{i=1}^{\lfloor (n -\delta' n) \cdot \rfloor} \um_{\{U_i \leq i^{-1/2} \}}/n^{1/2}$.
%
Using the very same computation of item $i)$, we conclude that the linearly interpolated version of $\left\{\sum_{i=1}^{\lfloor n \cdot \rfloor} \um_{ \{U_i \le i^{-\frac{1}{2}}\}}/n^{1/2}\right\}_{n\geq 1}$ and $\left\{ \sum_{i=1}^{\lfloor (n -\delta' n) \cdot \rfloor} \um_{\{U_i \leq i^{-1/2} \}}/n^{1/2}\right\}_{n\geq 1}$ 
% $\{\Hat{F}^n_{\cdot}/n^{1/2}\}_{n\geq 1}$ 
are tight sequences in $C_{\Rs}[0, T]$ for all $T>0$. Thus, the same holds for their difference.
%
\end{proof}


% \begin{proof}[Proof of Lemma~\ref{lem: Knt/n1/2tig}.] The proof follows the very same lines of that of Theorem~~\ref{pn-ERW-d=>4}, and we omit it. 
% {\color{red} 
% By  Remark~\ref{rem:conver}-$b)$  it suffices to show tightness in $C_{\Rs}[0, T]$ for all $T > 0$ and this is equivalent to show the the sequence of processes $\{|K_{\lfloor n \cdot \rfloor}|/n^{1/2}\}_{n\geq 1}$  satisfies  the two conditions in  Remark~\ref{rem:conver}-$c)$. 
% Note that $|K_{\lfloor n \cdot 0 \rfloor}|/n^{1/2}\equiv 0$ for all $n \ge 1$ and therefore the first condition in Remark~\ref{rem:conver}-$c)$ is satisfied.
% %
% To prove the second condition in Remark~\ref{rem:conver}-$c)$, 
% set 
% $
% A_t(\varepsilon, \phi):= \{f \in C_{\Rs}[0,T] : \sup_{t \le s \le t+\phi} |f(s) - f(t)| \ge \varepsilon \}$. 
% Then, following the same steps as in the proof of Theorem~\ref{pn-ERW-d=>4}, we have that  
% \begin{equation*}
% \begin{split}
% &\frac{1}{\phi} \PP [ |K_{\lfloor n \cdot \rfloor}|/n^{1/2} \in A_t(\varepsilon, \phi)]  = \frac{1}{\phi} \PP\Big[\sup_{t \le s \le t+\phi } | |K_{\lfloor ns \rfloor}| - |K_{\lfloor n t \rfloor}|| \ge \varepsilon n^{\frac{1}{2}} \Big]
% \\
% & = \frac{1}{\phi} \PP \Big[ \sum_{i = \lfloor nt \rfloor + 1}^{\lfloor n(t + \phi) \rfloor} \um_{E_i^c \cap \{U_i \le i^{-\frac{1}{2}}\}} \ge \varepsilon n^{\frac{1}{2}} \Big]
%  \le \frac{1}{\phi} \PP \Big[ \sum_{i = \lfloor nt \rfloor + 1}^{\lfloor n(t + \phi) \rfloor} \um_{\{U_i \le i^{-\frac{1}{2}}\}} \ge \varepsilon n^{\frac{1}{2}} \Big]\,.
% \end{split}    
% \end{equation*}
% From this point on, the computations are exactly the same used in the proof of Theorem~\ref{pn-ERW-d=>4}, and we omit it.

% when we show the second sum portion in~\eqref{p_n-ERW_incrementos_d=>4} fulfills the second condition in~\cite[Theorem 7.3]{billingsley1999probability}.
% Then we have that for each positive $\varepsilon$ and $\eta$, there exists a $\phi \in (0,1)$, and an integer $n_0$ such that
% \begin{equation*}
% \frac{1}{\phi} P_n \Big[f \in C_{\Rs}[0,T] : \sup_{t \le s \le t+\phi} |f(s) - f(t)| \ge \varepsilon \Big] \le \eta \, , \quad \forall n \ge n_0\,.
% \end{equation*}
% Ergo by~\cite[Theorem 7.3]{billingsley1999probability} we obtain that $|K_{\lfloor n \cdot \rfloor}|/n^{1/2}$ is tight in $C_{\Rs}[0, T]$ for all $T > 0$. Thus by~\cite[Theorem 2.4.10]{karatzas2012brownian} it is also a tight sequence of processes in $(C_{\Rs}[0, \infty),\rho)$
% } 
% \end{proof}
\end{comment}




\bibliographystyle{abbrv}
\bibliography{referencias}
 
 \end{document}
