The multi-dimensional Generalized Excited Random Walk (\Name) was introduced by Menshikov et al. in~\cite{menshikov2012general} following a series of works on multi-dimensional excited random walk \cite{benjamini2003excited,kozma2003excited,berard2007central}. The model considered in ~\cite{menshikov2012general} is a  uniformly elliptic, self-interacting,  random walk with bounded jumps in dimension $d\ge 2$ which behaves as follows:   on already visited sites it behaves as a $d$-dimensional martingale with bounded jumps and zero-mean vector, whereas whenever a site is visited for the first time its increment has a drift in a fixed direction $\ell$ of the unit sphere in $\mathbb{R}^d$. They show that the GERW with a drift condition in direction $\ell$ is ballistic in that direction. Besides that, they proved a Law of Large Numbers and a Central Limit Theorem (both for dimensions $d \geq 2$) under stronger hypothesis on the definition of GERW; these particular models were called \textit{excited random walk in random environment}. 

 The GERW is an important class of toy models of non-Markovian random walks used to understand how weakly, or rarely, the random walk needs to be pushed in a fixed direction to still exhibit a ballistic behavior. In \cite{alves2022note} we have studied a variation of the GERW  considering that the strength of the drift on first visits decreases with time. Specifically, we assume that the drift in a fixed direction $\ell$ at time $n$, say $\lambda_n$, is of order $n^{-\beta}$ (if at this time a site is visited for the first time). We called this variation  $\lambda_n$-\Name{}. If $\{X_n\}_{n\geq 0}$ denotes this $\lambda_n$-\Name{}, we show in \cite{alves2022note} that $\lim_{n \rightarrow \infty} X_n \cdot \ell = \infty$ (directional transience) with positive probability if $\beta$ is sufficiently small. This shows that the $\lambda_n$-\Name{}  has an intermediary behavior between a zero-mean random walk (non-excited) and the \Name{} considered in \cite{menshikov2012general}. Indeed, for  $\lambda_n$ of order  $n^{-\beta}$, the  $\lambda_n$-\Name{} is not ballistic, since the total mean drift accumulated by time $n$ is bounded by  $n^{1-\beta}$.
 The $\lambda_n$-\Name{} is well motivated since for the multi-dimensional excited random walk a relevant question is if it is still possible to guarantee properties such as directional transience and ballisticity by reducing the number of times it has a drift in a given direction.
 
In this paper, we discuss limit theorems 
% for $\beta \ge 1/2$ 
for a particular class of $\lambda_n$-\Name{}, in the same spirit of the {\em excited random walk in random environment} discussed in 
\cite{menshikov2012general} (and mentioned above). Let us point out that the model considered in this paper is neither a generalization nor a particular case of the excited random walk in random environment. Specifically, given a sequence of parameters $\{p_n\}_{n \geq 1}$ with $p_n \in (0,1]$ for all $n \geq 1$, if at time $n$ the process visits a site for the first time, it becomes excited with probability $p_n$ (thus gaining a drift). Otherwise, with probability $1-p_n$, it gains no drift and behaves as a $d$-dimensional martingale with zero-mean vector. If instead  the process has already visited the site, there is no excitation and the process acts again as a $d$-dimensional martingale with zero-mean vector. We call this model $p_n$-\Name{}. %We will focus on two specific cases $\{p_n\}_{n \geq 1}$: 
We will consider $p_n = \mathcal{C}n^{-\beta} \wedge 1$ with $\beta >0$ and $\mathcal{C}$ a positive constant.  Depending on the value of $\beta$ and on the dimension $d$, we obtain the following results:   
\begin{itemize}
    \item For $d\geq 2$ and $\beta$ sufficiently small the $p_n$-\Name{} has a positive probability of never returning to the origin in a fixed  direction $\ell$. This is proved in \cite{alves2022note}.
    \item For $d\geq 2$ and $\beta > 1/2$ the $p_n$-\Name{} suitably rescaled converges in distribution to a Gaussian Process (see, Proposition~\ref{pn-WGERW-Gauss}). Note that, for $\beta>1$ the $p_n$-\Name{} only steps with a drift finitely many times almost surely and therefore the process eventually behaves as a $d$-dimensional martingale; in this case, the convergence in distribution follows directly from \cite[Theorem 7.1.4]{ethier2009markov}.
    \item For $\beta = 1/2$, the dimension $d$ plays an important role: 
    \begin{itemize}
        \item[$\ast$] If $d=2$ the $p_n$-\Nametwo{}, which is a special type of $p_n$-\Name{} (see,  Section~\ref{sec:p_n-ERW} for its definition), converges in distribution,  under diffusive scaling,  to a Brownian Motion (see, Theorem~\ref{pn-ERW-d=2}).
        %
        \item[$\ast$] If $d \ge 3$,  the $p_n$-\Nametwo{} under  diffusive scaling  is  tight and  every limit point is stochastically dominated, in direction $\ell$, by a Brownian Motion plus a  multiple of square root of time  (see, $(a)$ in Theorem~\ref{pn-ERW-d=>4}).
        \item[$\ast$] If $d \ge 22$, the $p_n$-\Nametwo{}  under diffusive scaling   is  tight and  every limit point stochastically dominates, in direction $\ell$, a Brownian Motion plus a  multiple of square root of time (see, $(b)$ in Theorem~\ref{pn-ERW-d=>4}). 
    \end{itemize}
\end{itemize}



%\begin{itemize}
%    \item[1)] {\sc homogeneous:}  the sequence $\{p_n\}_{n \geq 1}$ will be constant, i.e., for a $p \in (0,1]$ we have $p_n = p$ for all $n \geq 1$. This case will be called $p$-\Name. As it turns out, the homogeneous case bears no novelty as it can be reduced to the \Name{} with certain adjustments (see discussion at the beginning of  Section~\ref{sec: res_p-gerw}). However, the analysis of this simple case, in particular the understanding of how $p$ will affect the results, turns out to  be important for the time dependent case.
%    \item[2)] {\sc polynomial  decay:}  $p_n = \mathcal{C}n^{-\beta} \wedge 1$ with $\beta > 0$ and $\mathcal{C}$ is a positive constant. %\com{Note that for $\beta>1$, the model will eventually behave as a $d$-martingale almost surely...., therefore the interesting case is for $\beta \in (0,1]$.}
%\end{itemize}

%Specifically,  in dimension one for the nearest neighbor ERW with an independent cookie environment, the mean number of cookies per site should be greater than one for the system to be ballistic, see \cite{zk2008}.

%Specifically, given a sequence of probabilities $\{p_n\}_{n \geq 0}$, with $p_n\in (0,1]$ for all $n \geq 1$, if at time $n$ the process visits a site for the first time, it finds a cookie with probability $p_n$ (thus gaining a drift). Otherwise, with probability $1-p_n$, it finds no cookie (no drift) and behaves as a $d$-martingale with zero-mean vector. If instead  the process has already visited the site, there is no cookie and the process acts again as a $d$-martingale with zero-mean vector. We call this model $p_n$-\Name{}. Here, we study the specific case in which  the sequence $\{p_n\}_{n \geq 0}$ is constant, i.e., for a $p \in (0,1]$ we have $p_n = p$ for all $n \geq 0$. It will be labeled as $p$-\Name. Note that, if $p=1$ than the $p$-\Name{} reduces to GERW. Our model is well motivated since for the many-dimensional excited random walk a relevant question is if it is still possible to reduce the number of cookies in the system to still guarantee properties as directional transience and ballisticity. This goes in the same direction of well-known results in dimension one \com{at the beginning we say $d\geq 2$, maybe we should mention immediately there that the model and related variants have been studied in dimension one as well}. For instance in dimension one for the nearest neighbor ERW with an independent cookie environment, the mean number of cookies per site should greater than one for the system to ballistic, see \cite{zk2008}.
%As regards to the $p$-\Name{},  under the same hypothesis of \cite{menshikov2012general}\footnote{Bounded jumps, an uniformly elliptic condition and a drift condition in an arbitrary direction of the unit sphere.}, we show that the $p$-\Name{}  is ballistic for all $p\in (0,1]$ (see, Theorem~\ref{teo11}) and under the same stronger assumptions, that is, the increments of the regeneration times be i.i.d., we obtain a Law of Large Numbers and a Central Limit Theorem (see, Theorem~\ref{LLN} and Theorem~\ref{CLT}, respectively). 

Let us briefly discuss the missing cases. 
For $\beta=1/2$ and $3\leq d\leq 21$, we conjecture that the behavior of the $p_n$-\Nametwo{} would be analogous to that for $d\geq22$ (i.e., stochastically dominated from above and below, in direction $\ell$, by a Brownian Motion plus a sublinear continuous function in $[0, \infty)$). However, due to some technical difficulties, we were not able to prove it, see Remark~\ref{rem:missing-cases}. Fop $\beta<1/2$ and $d\geq 2$, it is also conjectured that the $p_n$-\Name{} has a positive probability of never returning to the origin, see discussion in \cite{alves2022note}. 
%%%%%%% OLD %%%%%%%%
% For $d=3$ and $\beta=1/2$, we conjecture that the behavior of the $p_n$-\Nametwo{} would be analogous to that for $d\geq4$ and $\beta=1/2$. However, due to some technical difficulties, we were not able to prove it, \textcolor{brown}{see Remark~\ref{rem:restriction}}. 
% We also conjecture that in $d \ge 2$  the $p_n$-\Name{} has a positive probability to never return to the origin for any $\beta < 1/2$, see discussion in \cite{alves2022note}. 
%
% \br{For $\beta < 1/6$ and $d\geq 2$, we have that the $p_n$-\Name{} is transient in the drift direction. For $\beta > 1/2$ and $d\geq 2$ we obtain that, under certain conditions, the $p_n$-\Name{} converges in distribution to a Gaussian Process. Now for $\beta = 1/2$, the dimension $d$ makes a difference. In dimension $d=2$ the $p_n$-\Nameone{}, which is a specific type of $p_n$-\Name{} that will be presented later, converges in distribution to a standard Brownian Motion. However in $d \ge 4$, we will obtain \cm{a tight process which every limit point satisfies the following property, it  is stochastically dominated from below in direction $\ell$ by Brownian Motions with a drift and from above by other one with a different drift.}} 
%
%
%that every sub-sequence will converge to a distribution, which is stochastically dominated from below in direction $\ell$ by Brownian Motions with a drift and from above by other one with a different drift.  
%For $\beta < 1/6$ we have that the $p_n$-\Name{} is transient in the drift direction. For $\beta > 1/2$ we obtain that the $p_n$-\Name{} converges in distribution for a Gaussian Process. In dimension $d=2$ and $\beta = 1/2$, we obtain that for a specific type of $p_n-\Name{}$} 

\medskip 
The paper is organized as follows: In Section~\ref{sec:model} we formally define the $p_n$-\Name{},  as well as the $p_n$-\Nametwo{},  and state the main theorems concerning the different asymptotic behaviors  depending on the model ($p_n$-\Name{} or $p_n$-\Nametwo{}), the value of $\beta$ and on the dimension $d$. Section~\ref{resultados_pn} is devoted to the proofs of the main results. In Section~\ref{sec:rangeERW} we provide the proof of an auxiliary result, which may be of independent interest,  regarding the asymptotic behavior of the range of the $p_n$-\Nametwo{} in $d \ge 2$ and $\beta = 1/2$ (see,  Proposition~\ref{prop:RangeERW} and Proposition~\ref{prop:RangeERW_lower}). 
% ; these two results are used in the proof of Theorem~\ref{pn-ERW-d=>4}. 

\section{Formal definitions and main results}\label{sec:model}
Let $d \ge 2$ be the fixed dimension, $\{\xi_i, \FF_i\}_{i \geq 1}$ be the increments of a $d$-martingale  
(i.e., a $d$ dimensional process where each component is a martingale) 
with zero-mean vector and $\{\gamma_i, \FF_i\}_{i\geq 1}$ be a $\ZZ^d$ random vector,  both adapted processes on a stochastic basis $(\Omega,\mathcal{F},\PP,\{ \mathcal{F}_n \}_{n \geq 0})$ where $\FF_0$ contains all the $\PP$-null sets of $\FF$. %satisfying the usual conditions \cm{i.e., the filtration is complete and right-continuous, ou devemos definir a filtração?} \com{list the conditions and give a reference}.
We denote by $\mathbb{E}$ the expectation with respect to $\PP$ and by $||\cdot||$  the Euclidean norm in $\mathbb{R}^d$. Let  $\ell \in \mathbb{S}^{d-1}$ be a direction, where $\mathbb{S}^{d-1}$ is the unit sphere of $\mathbb{R}^d$. We assume the following conditions:

\begin{condition}\label{condition_A}
There exists a positive constant $K$ such that
\[
\sup_{n \geq 1} || \xi_n || \le K \quad \text{  and } \quad  \sup_{n \ge 1}||\gamma_n|| \le K\;,
\]
on every realization.    
\end{condition}

\begin{condition}\label{condition_B}
For every $n \ge 1$,  we have that 
\[
\EE[\xi_n| \FF_{n-1}] = 0 \quad  \text{ and } \quad \EE[\gamma_n \cdot \ell | \FF_{n-1}] \ge \lambda\;,
\]
 where $\lambda$ is a positive constant.    
\end{condition}
 
Let $\{U_i\}_{i \geq 1}$ be a sequence of i.i.d. random variables with uniform distribution in $[0,1]$ independent of the $\{\xi_i\}_{i\geq 1}$ and $\{\gamma_i\}_{i\geq 1}$, and $\{p_n\}_{n \ge 1}$ be a sequence  such that $p_n \in (0, 1]$ for all $n \ge 1$. 
The $p_n$-\Name{} is a $\ZZ^d$ valued process $X=\{X_n\}_{n \ge 0}$.  Setting   $E_0:= \emptyset$ and, 
for $i\ge 1$,  $E_i:= \{ \exists\;  k < i \; \text{ such that }\;  X_k = X_i \}$ (i.e.,  $E_i$ corresponds to the event that the process $\{X_n\}_{n \geq 0}$ is, at time $i$,  in an already visited site),   we define  $\{X_n\}_{n \geq 0}$ recursively as: $X_0 = 0$ and 
%
% Let $\{X_n\}_{n \ge 0}$ be a $\ZZ^d$ valued process with $X_0 = 0$. Set $\{U_i\}_{i \geq 1}$ as a sequence of i.i.d. random variables with uniform distribution in $[0,1]$ and a sequence $\{p_n\}_{n \ge 1}$ such that $p_n \in (0, 1]$ for all $n \ge 1$. For $i\ge 1$ we define $E_i$ as the event that the process $\{X_n\}_{n \geq 0}$ is, at time $i$,  in an already visited site, i.e.,  $E_i:= \{ \exists\;  k < i \; \text{ such that }\;  X_k = X_i \}$ and $E_0:= \emptyset$. We write $\{X_n\}_{n \geq 0}$ as
%
\begin{align}\label{xn-incremnto1}
X_n := \sum_{i=1}^n \big( \um_{E_{i-1}} \xi_i + \um_{E_{i-1}^c \cap \{U_{i} > p_{i} \}} \xi_i + \um_{E_{i-1}^c \cap \{ U_{i} \leq p_{i}\}} \gamma_i \big) , \ n\ge 1 \,, 
\end{align}
where $\um_E$ denotes the indicator function of set $E$. 
The $p_n$-\Name{} can be interpreted as follows: if at time $n$ the process visits a site for the first time, it becomes excited with probability $p_n$ (thus gaining a drift). Otherwise, with probability $1-p_n$, it does not become excited (no drift) and behaves as a $d$-martingale with zero-mean vector. If the process has already visited the site, there is no excitation and the process acts again as a $d$-martingale with zero-mean vector.

\subsubsection*{\bf A special case of $p_n$-\Name{} ($p_n$-\Nametwo{})}\label{sec:p_n-ERW}  
A particular type of $p_n$-\Name{}, which we call $p_n$-\Nametwo{} in the direction $\ell$, is obtained if we further assume that the sequence $\{\xi_i\}_{i \ge 1}$ is i.i.d. on $\ZZ^d$ with zero-mean vector and finite covariance matrix. Additionally, we assume $\PP[ \xi_i \cdot e_k = 0 ] < 1$ for all $i \ge 0$ and for each $k \in \{1,2, \dots, d\}$ (i.e., the increments $\xi_i$ are truly $d$-dimensional). Moreover, we assume that the sequence $\{\gamma_i\}_{i \ge 1}$ is also i.i.d. on $\ZZ^d$  (recall that $\gamma_i$ satisfies  $\EE[\gamma_i \cdot \ell |\FF_{i-1}] \ge \lambda$), 
and that the sequences $\{\xi_i\}_{i \ge 1}$ and $\{\gamma_i\}_{i \ge 1}$ are independent.
 
\medskip
\noindent
{\it Example of $p_n$-\Nametwo{}:}  
Here we provide a concrete example of a $p_n$-\Nametwo{} on $\ZZ^d$ which may be thought of as a generalization of the classical ERW~\cite{benjamini2003excited}. Specifically, it evolves as the classical ERW but when a site is visited for the first time, it  gains a drift with probability $p_n$.   
Let $\ell = e_1$ (drift direction) and let $\{\xi_i\}_{i \ge 1}$ and $\{\gamma_i\}_{i \ge 1}$ be i.i.d. sequences of discrete random variables taking values on $\{\pm e_1, \ldots, \pm e_d\}$ (canonical directions).  We denote by $q_{\xi}$ and $q_{\gamma}$, the probability distributions associated   with $\{\xi_i\}_{i \ge 1}$ and $\{\gamma_i\}_{i \ge 1}$, respectively. 
We assume that $q_{\xi}(e_j)=q_{\xi}(-e_j)=\frac{1}{2d}$, for all $j \in \{1, \ldots, d\}$. As regards to $q_{\gamma}$, fixing $\delta \in (\frac{1}{2},1]$, we assume that  $q_{\gamma}(e_1)= \frac{\delta}{d}$, $q_{\gamma}(-e_1)=\frac{1-\delta}{d}$ and  $q_{\gamma}(e_j)=q_{\gamma}(-e_j)=\frac{1}{2d}$, for all $j \in \{2, \ldots, d\}$. Note that, with the above choice, we obtain that $\EE[\gamma_i \cdot e_1 |\FF_{i-1}]= \EE[\gamma_1 \cdot e_1] = 
% \frac{\delta}{d} - \frac{1-\delta}{d}=
\frac{2\delta -1}{d}=:\lambda >0$ (the first equality follows from the i.i.d. hypothesis and the fact that the sequences $\{\gamma_i\}_{i \ge 1}$ and $\{\xi_i\}_{i \ge 1}$ are independent). 
%for all $n \geq 1$.   
% Next we will give an example of a process which respect the conditions to be a $p_n$-\Name{}. 
%Here we suppose that $\pi$ have iid marginals. 
% Fix $\delta \in (1/2,1]$ and let $q^{(0)}(x, e_i)$, $x \in \ZZ^d$, $i=1,...,d$, be defined as
% \begin{equation*}\label{prob-ERW}
% \begin{split}
% q^{(0)}(x, e_1) & = \delta/d , \ \,
% q^{(0)}(x, -e_1)  = (1-\delta)/d\;, 
% \\
% q^{(0)}(x, \pm e_i) & = 1/2d \quad \text{for all } i = 2,\dots, d\;,
% \end{split}
% \end{equation*}
% and $q^{(1)}(x, e_i)$, $x \in \ZZ^d$, $i=1,...,d$, be the transition probabilities of a SRW in $\ZZ^d$.
The corresponding  $p_n$-\Nametwo{} (see, \eqref{xn-incremnto1}) is a process $\{X_n\}_{n \geq 0}$ on $\ZZ^d$ with transition probabilities 
\begin{equation*}
\begin{split}
\PP & \Big[X_{n+1} = x + e_i  \Big\vert X_n = x, \sum_{j=0}^{n-1} \um_{\{X_
j = x\}} = 0 \Big] = 
\\
& = \um_{\{U_n \leq p_n\}} q_{\gamma}(e_i) +   \um_{\{U_n > p_n\}} q_{\xi}(e_i) \,, 
\end{split}
\end{equation*}
and for every $m \in \{1,2,\dots, n-1\}$ we have 
\begin{equation*}
\PP \Big[X_{n+1} = x + e_i  \Big\vert X_n = x, \sum_{j=0}^{n-1}\um_{\{X_
j = x\}} = m \Big] = q_{\xi}(e_i) \,, 
\end{equation*}
Note that, when $p_n =1$ for all $n \ge 1$, we recover the classical ERW.
\hfill $\diamond$


 
\medskip
\noindent
{\it Relation between  $p_n$-\Name{} and the $\lambda_n$-\Name{} introduced in~ \cite{alves2022note}.}  
We show here that a $p_n$-\Name{} can be thought of as a special type of a $\lambda_n$-\Name{} if we further impose a uniform elliptic condition.
As can be seen in \cite{alves2022note}, the $\lambda_n$-\Name{} is also defined in terms of three conditions (along the lines of \cite{menshikov2012general}), which are summarised below: 
\begin{enumerate}[$i)$]
    \item $\exists K>0$ such that $\sup_{n \geq 0} || X_{n+1} - X_{n} || < K$ on every realization; 
    \item almost surely, on $\{ X_k \neq X_n \, \forall \; k < n \}$, $\mathbb{E} [ X_{n+1} - X_n | \mathcal{F}_n] \cdot \ell  \geq \lambda_n$, and 
    \\
    on $\{ \exists\,  k < n  \text{ such that }  X_k = X_n \}$,
 $\mathbb{E} [ X_{n+1} - X_n | \mathcal{F}_n] = 0$;
 \item  There exist $h, r > 0$ such that: 
 for all $n$, 
 \\ $\PP \left[ \left( X_{n+1} - X_n \right) \cdot \ell > r | \mathcal{F}_n \right] \geq h$, a.s., and 
 \\
 on the event $ \{ \mathbb{E} [ X_{n+1} - X_n | \mathcal{F}_n] = 0 \}$,  for all $\ell' \in \mathbb{S}^{d-1}$,  with $|| \ell '|| = 1$,
$\PP \left[ \left( X_{n+1} - X_n \right) \cdot \ell ' > r | \mathcal{F}_n \right] \geq h$, a.s..
\end{enumerate}
Now if $\{X_n\}_{n \ge 0}$ is a $p_n$-\Name{} (see, \eqref{xn-incremnto1}), Condition~\ref{condition_A}  implies $i)$. As far as $ii)$ is concerned, if we denote by $\widetilde{\FF}_n=\sigma(X_1, \ldots, X_n)$, by integrating 
$\mathbb{E} [ X_{n+1} - X_n | \mathcal{F}_n] \cdot \ell$ with respect to $U_1, \ldots, U_{n+1}$ and using Condition~\ref{condition_B}, we obtain $\mathbb{E} [ X_{n+1} - X_n | \widetilde{\FF}_n] \cdot \ell\geq p_{n+1}\lambda$.  Thus $ii)$  is satisfied with $\lambda_n = p_{n+1}\lambda$.
Condition $iii)$ is a uniform elliptic condition which we do not assume for $p_n$-\Name{}; thus in order to compare a $p_n$-\Name{} with a $\lambda_n$-\Name{}, this condition must be further imposed. 
%
\begin{remark}
Note that if $\{X_n\}_{n \ge 0}$ is a $p_n$-\Name{} with $p_{n+1} = p$ where $p \in (0, 1]$ for all $n \ge 1$ (homogeneous) which also satisfies the uniform elliptic condition $iii)$, we obtain the~\Name 
~defined in~\cite{menshikov2012general}. Thus, from the result in~\cite{menshikov2012general}, we obtain that the homogeneous $p_n$-\Name{} satisfying $iii)$ is ballistic  in the direction of the drift. 
\end{remark} 

\medskip 


In this paper we study the $p_n$-\Name{} when the sequence $\{p_n\}_{n \ge 1}$ is of the form $p_n =\mathcal{C} n^{-\beta} \wedge 1$, with $\beta\geq  1/2$ and $\mathcal{C}$ is a positive constant.
 When $\beta>1/2$, we may relax the bounded jump assumption of  Condition~\ref{condition_A};   namely, we may just assume the following condition: 
%
 \renewcommand{\theconditionbis}{\Roman{conditionbis}*}
\setcounter{conditionbis}{0}
\begin{conditionbis}\label{condiçaoI*}
For all $k \geq 1$ and $\theta < \beta - 1/2$, where $\beta> 1/2$, we have
    \[\sup_{k \geq 1} \frac{\EE[\| \xi_k \|]}{k^{\theta}} < \infty  \quad \text{ and } \quad
    \sup_{k \geq 1} \frac{\EE[\| \gamma_k \|]}{k^{\theta}} < \infty \;. \]
    % \item [ii)]  It holds that
    % \begin{equation}\label{condGaussiano}
    % \frac{1}{n}\sum_{i=1}^{\lfloor nt \rfloor} \xi_i \xi_i^T \to C(t) \quad \text{as } n \to \infty \;,
    % \end{equation}
    % in probability and
    % \begin{equation*}
    % \lim_{k \to \infty} k^{-1/2} \EE\left[ \sup_{1 \leq i \leq k} \| \xi_i \| \right]  = 0\,. 
%\end{equation*}     %\item [ii)] For a large $n$ we have,
    %\[ \sum_{k = 1}^{\infty} \PP[\| X_k - X_{k-1} \| > k^{\frac{\theta}{2}}] < \infty \;, \]
    %where $\theta$ is a positive constant such that $\theta/2 < \beta - 1/2$.
%\end{itemize}
\end{conditionbis}
If a process $X$ (defined as in~\eqref{xn-incremnto1}) has associated sequences $\{\gamma_i\}_{ i\geq 1}$ and $\{\xi_i\}_{i \geq 1}$ which  satisfy  Condition~\ref{condiçaoI*}, and~\ref{condition_B},  we call $X$ a $p_n$-\Name*.
 


% %%%%%%%%%% OLD STUFF ABOUT CONNECTION BETWEEN $p_n$-GERW and $\lambda_n$-GERW

% \cm{Now let a process $\{X_n\}_{n \ge 0}$ be a $p_n$-\Name{}. It is possible to make a connection with the $\lambda_n$-\Name{} described in~\cite{alves2022note}. For this, we fix a sequence $\{\lambda_n\}_{n \ge 0}$ of positive real numbers. We will see that a $p_n$-\Name{} can be a special type of a $\lambda_n$-\Name{} if we impose a uniform elliptic condition.}




% %%%%%%%%%%%%%%%%%%Texto antigo%%%%%%%%%%%%%%

% % \cm{Let $d \ge 2$ be the fixed dimension, $\{\xi_i, \FF_i\}_{i \geq 1}$ is an increment of a $d$-martingale with zero mean and $\{\gamma_i, \FF_i\}_{i\geq 1}$ is a $\ZZ^d$ random vector such that $\EE[\gamma_i \cdot \ell | \FF_{i-1}] \ge \lambda$ for all $i \ge 1$, where $\lambda >0$  and  $X = \{ X_n \}_{n \geq 0}$ be a $\ZZ^d$ valued adapted process on a stochastic basis $(\Omega,\mathcal{F},\PP,\{ \mathcal{F}_n \}_{n \geq 0})$ satisfying the usual conditions. Forward, it will be clear the connections between the processes $\{X_n\}_{n \ge 0}$, $\{\xi_n\}_{n \ge 1}$ and $\{\gamma_n\}_{n \ge 1}$.  We denote by $\mathbb{E}$ the expectation with respect to $\PP$ and by $||\cdot||$  the euclidean norm in $\mathbb{R}^d$. Now fix $\{\lambda_n\}_{n\ge 0}$ a sequence of positive real numbers, $\ell \in \mathbb{S}^{d-1}$, where $\mathbb{S}^{d-1}$ is the unit sphere of $\mathbb{R}^d$ and a nonempty set $A \subset \ZZ^d$. We assume that $X_0=0$ and we call $X$ a $\lambda_n$-\Name{} in direction $\ell$ with excitation set $A$, if it satisfies the following conditions:}

% %We now formally introduce the \cm{$\lambda_n$-\Name{}}.
% % \cm{\textbf{Cond I:} for all $i \ge 1$ we have $||\xi_i|| < K$ and $||\gamma_i|| < K$. \\
% % \textbf{Cond II:} $\EE[\gamma_i \cdot \ell| \mathcal{G}_{i-1}] \ge \lambda_n$ and $\EE[\xi_i | \mathcal{G}_{i-1}] = 0$ \\
% % \textbf{Cond III:} $\PP[\gamma_i \cdot \ell \ge r | \mathcal{G}_{i-1}] \ge h$ and $\PP[\xi_i \cdot \ell' \ge r | \mathcal{G}_{i-1}] \ge r$. \\
% % Define the sequences $\{U_i\}$ and $\{p_n\}$. Write $X_n$ as in (1). Enlarge the sigma-algebra $\FF_n = \sigma(\xi_n, \gamma_n, U_n)$.}


% % \cm{Let $d \ge 2$ be the fixed dimension and  $X = \{ X_n \}_{n \geq 0}$ be a $\ZZ^d$ valued adapted process on a stochastic basis $(\Omega,\mathcal{F},\PP,\{ \mathcal{F}_n \}_{n \geq 0})$ satisfying the usual conditions. We denote by $\mathbb{E}$ the expectation with respect to $\PP$ and by $||\cdot||$  the euclidean norm in $\mathbb{R}^d$.
% %where $\FF_n = \sigma(X_1,\dots, X_n, \pi(X_1), \dots,$ $\pi(X_n))$ and $\sigma(Y)$ represents the smallest $\sigma$-algebra generated by a random vector $Y$. we can think of $\PP$ as the semi-direct measure $Q \otimes P_{\hat \pi}$, where $P_{\hat \pi}$ is the quenched measure for $X$, i.e., the conditional probability law of $X$ given a realization $\hat \pi$ of $\pi$. 
% % Now fix $\{\lambda_n\}_{n\ge 0}$ a sequence of positive real numbers, $\ell \in \mathbb{S}^{d-1}$, where $\mathbb{S}^{d-1}$ is the unit sphere of $\mathbb{R}^d$ and a nonempty set $A \subset \ZZ^d$. We assume that $X_0=0$ and we call $X$ a $\lambda_n$-\Name{} in direction $\ell$ with excitation set $A$, if it satisfies the following conditions:}

% %%%%%%%%%%%%%%%%%%%%%%%%%%%%%%%%%%%%

% \begin{condition}[Bounded increments]\label{condição1}
% There exists a positive constant $K$ such that $\sup_{n \geq 0} || X_{n+1} - X_{n} || < K$ on every realization. 
% \end{condition}

% \cm{It is easily verified that $\{X_n\}_{n \ge 0}$ satisfied the Condition~\ref{condição1} since its increments, independent if it behaviors as $d$-martingale or a process with a drift, are bounded.}


% \begin{condition}\label{condição2} Almost surely
% \begin{itemize}
%     \item on $\{ X_k \neq X_n \, \forall \; k < n \}$, 
% $$
%     \mathbb{E} [ X_{n+1} - X_n | \mathcal{F}_n] \cdot \ell  \geq \lambda_n \, .
% $$
%     \item on $\{ \exists\,  k < n  \text{ such that }  X_k = X_n \}$,
%     \[
%      \mathbb{E} [ X_{n+1} - X_n | \mathcal{F}_n] = 0\, .
%     \]
% \end{itemize}
% \end{condition}

% \cm{For Condition~\ref{condição2}, if we denote by $\widetilde{\FF}_n=\sigma(X_1, \ldots, X_n)$, by integrating 
% $\mathbb{E} [ X_{n+1} - X_n | \mathcal{F}_n] \cdot \ell$ with respect to $U_1, \ldots, U_{n+1}$ we obtain $\mathbb{E} [ X_{n+1} - X_n | \widetilde{\FF}_n] \cdot \ell\geq p_{n+1}\lambda$.  Ergo Condition~\ref{condição2} is satisfied with $\lambda_n = p_{n+1}\lambda$.} \textcolor{red}{verificar esses indices, pois na soma de (1) fizemos $X_i - X_{i-1}$.}

% \cm{If we impose a uniformly elliptic condition for $\{X_n\}_{n \ge 0}$, as the following, we obtain that $\{X_n\}_{n \ge 0}$ is a $\lambda_n$-\Name{}.}

% \begin{condition}\label{condição3}
%  There exist $h, r > 0$ such that

% \begin{itemize}
%     \item $X$ is {\rm uniformly elliptic in direction $\ell$}, i.e.,  for all $n$
% \begin{equation} \label{3 1.4}
% \tag{UE1}\PP \left[ \left( X_{n+1} - X_n \right) \cdot \ell > r | \mathcal{F}_n \right] \geq h\,, \; {a.s..}
% \end{equation}
% \item $X$ is {\rm uniformly elliptic on the event $\{\mathbb{E} [ X_{n+1} - X_n | \mathcal{F}_n] = 0\}$:} on the event $ \{ \mathbb{E} [ X_{n+1} - X_n | \mathcal{F}_n] = 0 \}$,  for all $\ell' \in \mathbb{S}^{d-1}$,  with $|| \ell '|| = 1$
% \begin{equation} \label{3 1.5}
% \tag{UE2}\PP \left[ \left( X_{n+1} - X_n \right) \cdot \ell ' > r | \mathcal{F}_n \right] \geq h\,, \; {a.s..}
% \end{equation}
% \end{itemize}
% \end{condition}

% %When $A = \ZZ^d$, we call $X$ simply a $\lambda_n$-\Name{}.

% \cm{\begin{remark}
% One can see that if we make the sequence $\{\lambda_n\}_{n \ge 1}$ homogeneous, i.e., $p_{n+1} = p$ where $p \in (0, 1]$ for all $n \ge 1$ and impose the Condition~\ref{condição3} for the $p_n$-\Name{} we obtain the~\Name 
% ~defined in~\cite{menshikov2012general}. Thus it is possible to see that the $p_n$-\Name{} is  in the direction of the drift if the sequence $\{p_n\}_{n \ge 1}$ is constant. 
% \end{remark}}



% %%%%%%%%%%%%%%


  %\footnote{ Note that, the $p_n$-W\Name{} is weaker also due to the fact that in  the  $p_n$-\Name{} the sequences $\{U_i\}_{i \geq 1}$, $\{\gamma_i\}_{ i\geq 1}$ and $\{\xi_i\}_{i \geq 1}$ are independents.} 
 
% \medskip 
% \subsubsection*{A specific case of $p_n$-\Name{} ($p_n$-\Nametwo{})}\label{sec:p_n-ERW}  
% If we further assume that the sequence $\{\xi_i\}_{i \ge 1}$ is i.i.d.  with zero mean vector and finite variance and  the sequence $\{\gamma_i\}_{i \ge 1}$ is also i.i.d. with  finite variance (recall that $\gamma_i$ satisfies  $\EE[\gamma_i \cdot \ell |\FF_{i-1}] \ge \lambda$),  then we obtain a specific type of the $p_n$-\Name{}, which we call $p_n$-\Nametwo{} in the direction $\ell$. 
% In the Example below, we provide a concrete example of a $p_n$-\Nametwo{}. 
 
% \medskip
% \noindent
% {\bf Example:}  
% Here we provide a concrete example of a $p_n$-\Name{} which may be thought of as a generalization of the classical ERW. Specifically it evolves as the classical ERW but when a site is visited for the first time, it finds a cookie (thus gaining a drift) with probability $p_n$. This generalization reduces to the  classical ERW when $p_n =1$ for all $n \ge 1$. %for all $n \geq 1$.   
% % Next we will give an example of a process which respect the conditions to be a $p_n$-\Name{}. 
% %Here we suppose that $\pi$ have iid marginals. 
% Fix $\delta \in (1/2,1]$ and let $q^{(0)}(x, e_i)$, $x \in \ZZ^d$, $i=1,...,d$, be defined as
% \begin{equation*}\label{prob-ERW}
% \begin{split}
% q^{(0)}(x, e_1) & = \delta/d , \ \,
% q^{(0)}(x, -e_1)  = (1-\delta)/d\;, 
% \\
% q^{(0)}(x, \pm e_i) & = 1/2d \quad \text{for all } i = 2,\dots, d\;,
% \end{split}
% \end{equation*}
% and $q^{(1)}(x, e_i)$, $x \in \ZZ^d$, $i=1,...,d$, be the transition probabilities of a SRW in $\ZZ^d$.
% Let $\{X_n\}_{n \geq 0}$ be a process  in $\ZZ^d$ with transition probabilities
% \begin{align*}
% P & \Big[X_{n+1} = x + e_i  \Big\vert X_n = x, \sum_{j=0}^{n-1}1_{\{X_
% j = x\}} = 0 \Big] = 
% \\
% & \qquad = 1_{\{\pi(x) \leq p_n\}} q^{(0)}(x,e_i) +   1_{\{\pi(x) > p_n\}} q^{(1)}(x, e_i) \;, 
% \end{align*}
% and for every $m \in \{1,2,\dots, n-1\}$ we have
% \begin{equation*}
% P \Big[X_{n+1} = x + e_i  \Big\vert X_n = x, \sum_{j=0}^{n-1}1_{\{X_
% j = x\}} = m \Big] = q^{(1)}(x, e_i) \;. 
% \end{equation*}
% The $\{X_n\}_{n \geq 0}$ is clearly a  $p_n$-\Nametwo{}  when $p_n = p$ for all $n \ge 1$. %See Figure \ref{fig:2} for a simulation  of this  $p$-\Nametwo{} in $\ZZ^2$. The  simulation suggests  that the $p$-\Nametwo{} is ballistic in direction $e_1$. Indeed, we will prove that the $p$-\Nametwo{} satisfies a ballistic Law of Large Numbers and a Central Limit Theorem. 

%In Figure~\ref{fig:1} we set $p=0.03$ and the simulation runs for $20000$ steps. The direction of the drift is $e_1$ and we have the following transitions probabilities when a cookie in consumed, $q^{(0)}(x,  e_1)  = 0.375$, $q^{(0)}(x, -e_1)  = 0.125$ and $q^{(0)}(x, \pm e_2)  = 0.25$. The red (resp., yellow) mark denotes the initial (resp., final) position of the random walk ($X_0 =(0,0)$ and $X_{20000} = (52, -43)$). To evaluate $v \cdot e_1$, we ran 1000 independent rounds and made the empirical average of $X_{20000}/20000$. We obtain as a result $\overline{v \cdot e_1} = 0.002087$. To estimate the standard deviation $\sigma$, we used the same rounds and made the standard deviation of $X_{20000}/20000$. We have as a result $\sigma  = 0.005029$.    

%In Figure~\ref{fig:2} the simulation is run with the same parameters as before  but for  $p=0.25$. We obtain that the final position is $X_{20000} = (404, -43)$, $\overline{v \cdot e_1} = 0.019856$ and $\sigma = 0.006217$. As we expected the process advances more in the direction $e_1$ as compared to the simulation in Figure~\ref{fig:1} and the empirical average $\overline{v \cdot e_1}$ is bigger. %\com{In this case it would be better to do various rounds and estimate the velocity by the empirical average}


% In the Figure~\eqref{fig:1} we mark red for the initial position of the process ($X_0 =(0,0)$) and yellow for its final position ($X_{20000} = (140, 245)$). Then we have, for this simulation, $v\cdot e_1 = 0.007$, where $v \cdot e_1 = x_{20000}/20000$,   i.e., an estimation of the velocity of the process in direction $e_1$.
% \newpage
%\begin{figure}[h]
%    \centering
%    \includegraphics[scale=0.50]{simulacoes/n20000_p003_2.png}
%    \caption{20000 steps simulation of $p$-\Nameone{} for $d=2$, $p=0.03$, $q^{(0)}(x,  e_1)  = 0.375$, $q^{(0)}(x, -e_1)  = 0.125$ and $q^{(0)}(x, \pm e_2)  = 0.25$. The red (resp., yellow) mark denotes the initial (resp., final) position of the random walk ($X_0 =(0,0)$ and $X_{20000} = (52, -43)$).}
%    \label{fig:1}
%\end{figure}

% \begin{figure}[h]
%     \centering
%     \includegraphics[scale=0.36]{simulacoes/path_simulation_p_025.png}
%     \caption{20000 steps simulation of $p$-\Nametwo{} for $d=2$, $p=0.25$, $q^{(0)}(x,  e_1)  = 0.375$, $q^{(0)}(x, -e_1)  = 0.125$ and $q^{(0)}(x, \pm e_2)  = 0.25$. The initial position of the random walk is $X_0 =(0,0)$ and the final $X_{20000} = (397, -20)$.}
%     \label{fig:2}
% \end{figure}

%\com{The legend on the x/y axis in both figures should be bigger. The title of boths figure  should be bigger and I suggest to write "path simulation of $p$-ERW"}

% Here in Figure~\eqref{fig:2}, we obtain the final position $X_{20000} = (382, 21)$ and $v \cdot e_1 = 0.0191$. As we expected the process advances more in the direction $e_1$ from the first simulation. 

%\textcolor{brown}{Glauco. Ficou ótima a figura. Seria bom colocar a estimativa da velocidade na direção $e_1$. O seu algoritmo permite simular o caso $p_n$ variável? Por exemplo o caso $p_n = n^{-\beta}$?}

% the case which we have a fixed $p \in (0, 1]$ and set $p_n = p$ for every $n$.




%If the process visits a site $x \in \ZZ^d$ for the first time, we say that there is a cookie on $x$ with probability $p_n$, that is, it has a $p_n$ probability of gaining  a drift in direction $\ell \in \mathbb{S}^{d-1}$. Otherwise, with probability $1-p_n$, the process behaves as a $d$-martingale. If instead  the process  has already visited the site, there is no cookie at the site and the process behaves as  a $d$-martingale. 
% 
% Given
% $\ell \in \mathbb{S}^{d-1}$, a stochastic process $X = (X_n, n \geq 0)$ adapted to a filtration $\mathcal{F} = \{ \mathcal{F}_n , n \geq 0 \}$ generated by $X$,  satisfying  Conditions~\ref{condição1},~\ref{condição2} and~\ref{condição3} with respect to $\ell$ and $X_0 = 0$ will be called  a \textit{$p_n$-\Name} in direction $\ell$.

%This present work is organized in the following way in Section~\ref{LGNCLT} we will present ours mains results, ballisticity of the process, LLN and CLT for $p_n$-\Name, in the case that $p_n=p$ for all $n$, that is, fixed. In Section~\ref{renewalstruc} we will introduce the renewal structure and the Proposition~\ref{tkfinito}, which gave us that $\tau_k < \infty$ for all $k$. Section~\ref{proofLGN} we will proof the main Theorems of Section~\ref{LGNCLT}.   

 %Similar works worth mentioning along these lines are~\cite{angel2021balanced,benjamini2011balanced,peres2016martingale}. What makes the GERW  an interesting model is the self-interaction encoded in the different behavior the process has on sites visited for the first time as compared to sites already visited.

%The many-dimensional excited random walk (ERW) is a model introduced in 2003 by Benjamini and Wilson~\cite{benjamini2003excited}. It is a discrete time non Markovian random walk in $\ZZ^d$, with $d \geq 2$. It jumps as a simple random walk biased  in direction $e_1$ (with bias $\delta$) every time it visits a site for the first time,  where $\{ e_i : 1 \leq i \leq d \}$ denotes the canonical base of $\ZZ^d$, otherwise it jumps as a simple symmetric random walk.
%depends on a fixed parameter $\delta \in (1/2, 1]$. The random walk starts from the origin and, every time it visits a site \emph{for the first time}, it will jump to the nearest neighbor along direction $e_1$ with probability $\delta/d$, to the nearest neighbor along direction $-e_1$ with probability $(1- \delta)/d$, and to any of the others nearest neighbors with probability $1/2d$. Parameter $\delta$ encodes the tendency the random walk has to choose to jump along direction $e_1$ on the first visit to a site, where $\{ e_i : 1 \leq i \leq d \}$ denote the canonical base of $\ZZ^d$. On the contrary, if the random walk visits an \emph{already visited} site, it will jump to any nearest neighbor with uniform probability $1/2d$.  

%In~\cite{benjamini2003excited}, Benjamini and Wilson proved that ERW is transient in direction $e_1$, i.e., $\lim_{n \to \infty} X_n \cdot e_1 = \infty$ almost surely. Furthermore, they also show that, if $d \geq 4$, ERW  is ballistic to the right, i.e., 
%\begin{equation*}\liminf_{n \to \infty} \frac{X_n \cdot e_1}{n} > 0\;, \qquad a.s..\end{equation*}
%Later on, Kozma extended the proof of ballisticity to $d=3$ in~\cite{kozma2003excited}, and $d=2$ in~\cite{kozma2005excited}. In 2007,  Bernard and Ramirez ~\cite{berard2007central} proved a Law of Large Numbers (LLN) and a Central Limit Theorem (CLT) for $d \geq 2$. Specifically, they prove that   
%\begin{equation*}\lim_{n \to \infty} \frac{X_n \cdot e_1}{n} = v\;, \qquad a.s.,   \end{equation*}for some $v = v(\delta, d) \in \mathbb{R}^+$,and that\begin{equation*}\Big\{ \frac{X_{\lfloor nt \rfloor } \cdot e_1 - \lfloor nt \rfloor v}{\sqrt{n}}\Big\}_{t\ge 0}\;,\end{equation*}converges in distribution as $n \to \infty$ (with respect to the Skorohod topologyon the space of c{\`a}dl{\`a}g functions) to a Brownian Motion with a finite variance depending on $\delta$ and $d$. Their proof relies on the introduction of an appropriate regeneration structure that was first used in the context of random walks in random environments, see for instance~\cite{sznitman1999law}.   

%In order to provide a few insights on the proofs of the results mentioned above, let us introduce a few concepts.As described in the first paragraph, the ERW behaves like a symmetric simple random walk (SRW) in $\ZZ^d$ on already visited sites and on first visited sites like a biased simple random walk with a fixed parameter $\delta$, i.e., 
%\begin{equation}\label{prob-ERW}
%\begin{split}
%p(x, e_1) & = \delta/d \;,
%\\
%p(x, -e_1) & = (1-\delta)/d \;, 
%\\
%p(x, \pm e_i) & = 1/(2d) \quad \text{for all } i = 2,\dots, d \;,
%\end{split}
%\end{equation}
%where $x \in \ZZ^d$ and $p(x, e_i)$ is the transition probability of the walk from site $x$  to $x+e_i$.The proofs of directional transience in~\cite{benjamini2003excited}, the LLG and the CLT  in~\cite{berard2007central}, rest upon two important ingredients. A coupling between the ERW and the simple symmetric random walk (SSRW)  which implies that the distance between the ERW and the SSRW at time $n$, in the direction $e_1$, is non decreasing in $n$, while for the others directions it is zero. Using this coupling, the authors provide  a lower bound on the cardinality of the set of visited sites by the ERW up to time $n$ (the range of ERW) in terms of \textit{tan points} for the SSRW, i.e., those sites $x \in \ZZ^d$ such that $x$ is the first site visited in the set $\{ x + ke_1 : k \geq 0 \}$. A direct consequence of the coupling is  that when the SSRW reaches a tan point, the ERW  visits a new site and thus it is pushed in  direction $e_1$ by a positive drift. Then in~\cite{berard2007central}, using this coupling, the authors proved that the range of the ERW up to time $n$ in dimension $d \geq 2$  is of order at least $n^{3/4}$ with large probability. This fact alone is not enough to provide a direct proof for a linear speed of the process, however it is instrumentalto guarantee the existence of a renewal structure for the process which leads to the limit theorems.  

%A drawback of the technique based on tan points is that it is tailored to the basic model of ERW and it is not robust, i.e., the coupling with the SSRW would not work if for example we consider a  random walk with bounded jumps, rather than nearest neighbor jumps, or even if we suppose a drift not parallel to any canonical direction. A more robust technique was developed by  Menshikov, Popov, Ramirez and Vachkovskaia in~\cite{menshikov2012general}. %\textcolor{red}{They proved a Law of Large Numbers and a Central Limit Theorem, both for dimensions $d \geq 2$ and for a more general model than ERW. Retirar?}
%
%The model they considered is a generalization of the ERW  and is  as follows:  on already visited sites the process  behaves like a $d$-dimensional martingale with bounded jumps and zero mean vector (rather than a SSRW) and whenever the process visits a site for the  first time it behaves as follows: it  has bounded jumps, satisfies an uniformly elliptic condition and a drift condition in an arbitrary direction $\ell$ of the unit sphere in $\mathbb{R}^d$. They call this model \textit{generalized excited random walk} (GERW). They show that the GERW with a drift condition in direction $\ell$ is ballistic in that direction. Besides that, they proved a LLG and a CLT (both for dimensions $d \geq 2$) for a special case of the GERW, which they called \textit{excited random walk in random environment}. This special model consists in an excited random walk in an i.i.d. random environment, which means that the process still has a mean drift in direction $\ell $ when it visits a site for the first time and whenever it hits an already visited site it has a zero mean drift (for more details see page 2110 in~\cite{menshikov2012general}). Along with that, the probability transitions for nearest neighbors of the process are explicit.
%
%Similarly to what was done for ERW,  the first step in their proof consists  in controlling the range of the process. Proposition 4.1 in~\cite{menshikov2012general} states that the range of the GERW is smaller than $n^{1/2 + \alpha}$ with probability that decays as a stretched exponential, where $\alpha > 0$ does not depend on the parameters of the model. A similar result can be found in~\cite{menshikov2014range} (see, Theorem 1.3). However, the range considered therein is for, what the authors call, a directed submartigale in direction $\ell$.  The proof of Proposition 4.1 completely avoids the use of the coupling and the tan points. Again, this  control on the range of the process allows the construction of a regeneration structure for the GERW. If $\{X_n\}_{n\geq 0}$ denotes the GERW, the regeneration structure  consists in a properly defined sequence of finite regeneration times $\tau_k$, $k \geq 1$, that correspond to those times when the process $\{X_n \cdot \ell\}_{n \geq 0}$ reaches for the first time the level $X_{\tau_k} \cdot \ell$ and never comes back below $X_{\tau_k} \cdot \ell$ after time $\tau_k$.  The renewal structure considered in~\cite{menshikov2012general} follows the standard approach and notation presented in~\cite{berard2007central} and~\cite{sznitman1999law}.
%  Furthermore, with the techniques developed for excited random they shown a Law of Large Numbers and a Central Limit Theorem.          What makes the GERW (and the ERW) an interesting model is the self-interaction encoded in the different behavior the process has on sites visited for the first time as compared to sites already visited. Similar works worth mentioning along these lines are~\cite{benjamini2011balanced},~\cite{peres2016martingale},  and~\cite{angel2021balanced}.   It is customary to think that initially all sites  have a {\em cookie}.   Whenever the process visits a site for the first time, it eats the cookie and gains a drift in a given direction. On subsequent visits to a site, since there is no cookie left, the process has no drift (for this reason ERW are also referred to as cookie random walks).  A natural question is  what happens to GERW when on the first visit to a site the random walk may or may not find/eat a cookie, with a probability that may possibly depend on the time of the first visit. Would the process still be ballistic in the direction of the drift? What about LLN and CLT?In order to address this question,   we introduce and study a model  which is a variation of the GERW. 
%\com{I would not mentioned the more general model with $p_n$, since in this paper  we consider only the constant case!} \comu{Neste caso, acho que também devemos supor desde o início a independência no ambiente de cookies.}


