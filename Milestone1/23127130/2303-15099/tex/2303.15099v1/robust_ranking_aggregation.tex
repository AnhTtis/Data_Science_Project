%% LyX 2.3.7 created this file.  For more info, see http://www.lyx.org/.
%% Do not edit unless you really know what you are doing.
\documentclass[twoside,english]{elsarticle}
\usepackage[T1]{fontenc}
\usepackage[latin9]{inputenc}
\pagestyle{headings}
\usepackage{float}
\usepackage{amsmath}
\usepackage{amsthm}
\usepackage{amssymb}
\usepackage{graphicx}

\makeatletter
%%%%%%%%%%%%%%%%%%%%%%%%%%%%%% Textclass specific LaTeX commands.
\theoremstyle{plain}
\newtheorem{thm}{\protect\theoremname}
\theoremstyle{definition}
\newtheorem{defn}[thm]{\protect\definitionname}

%%%%%%%%%%%%%%%%%%%%%%%%%%%%%% User specified LaTeX commands.
% specify here the journal
\journal{Example: Nuclear Physics B}

% use this if you need line numbers
%\usepackage{lineno}

\makeatother

\usepackage{babel}
\providecommand{\definitionname}{Definition}
\providecommand{\theoremname}{Theorem}

\begin{document}

\begin{frontmatter}{}

\title{Towards secure judgments aggregation in AHP}

\author[kis]{Konrad Ku\l akowski\corref{cor1}}

\ead{konrad.kulakowski@agh.edu.pl}

\author[wms]{Jacek~Szybowski}

\ead{jacek.szybowski@agh.edu.pl}

\author[opf]{Jiri~Mazurek}

\ead{mazurek@opf.slu.cz}

\author[kis]{Sebastian~Ernst}

\ead{sebastian.ernst@agh.edu.pl}

\cortext[cor1]{Corresponding author}

\address[kis]{AGH, University of Science and Technology, Department of Applied
Computer Science }

\address[opf]{Department of Informatics and Mathematics, Silesian University in
Opava, School of Business Administration in Karvina, Univerzitni namesti
1934/3, Czech Republic}

\address[wms]{AGH, University of Science and Technology, Faculty of Applied Mathematics}
\begin{abstract}
In the decision making methods the common assumption is the honesty
and professionalism of experts. However, this is not the case when
one or more experts in the group decision making framework, such as
the group analytic hierarchy process (GAHP), try to manipulate results
in their favor. The aim of this paper is to introduce two heuristics
in the GAHP setting allowing to detect the manipulators and minimize
their effect on the group consensus by diminishing their weights.
The first heuristic is based on the assumption that manipulators will
provide judgments which can be considered outliers with respect to
judgments of the rest of the experts in the group. Second heuristic
assumes that dishonest judgments are less consistent than average
consistency of the group. Both approaches are illustrated with numerical
examples and simulations. 
\end{abstract}
\begin{keyword}
pairwise comparisons \sep manipulation \sep group decision-making
\sep AHP \sep Analytic Hierarchy Process 
\end{keyword}

\end{frontmatter}{}


\section{Introduction\label{sec:Introduction}}

Group decision making refers to the situations where the problem of
a selection of the best alternative (option, solution, etc.) is handled
collectively by a set of individuals, preferably experts in the field.
Usually, such situations involve complex problems too difficult to
be handled by an individual, or they deal with actions required to
be made by a collective by the law. These situations include government
meetings, various board negotiations, policy making, business dealings,
complex laboratory experiments, elections, jury trials, reaching consensus
in social networks, and many others. The fundamentals of group decision
making can be found e.g. in \citep{Castelan1993iagd,Dong2016cbig,GarciaZamora2022lsgd,Hwang1987gdmu,Kilgour2021hogd,Lootsma1999gdm,Lootsma1999mcdm,Saaty1989gdma}.

One of the problems associated with group decision making (GDM) is
that it is susceptible to a manipulation if one or more experts try
to influence the group outcome in their favor by, for example, providing
dishonest judgments. The manipulation, especially in the political
context, can be traced back at least to the ancient societies of Greece
or Rome, see e.g. \citep{Connor1987tfap}. More recently, a description
of political manipulation in a GDM setting can be found for example
in the work of Maoz \citep{Maoz1990ftni}, who investigated examples
of U.S. and Israeli foreign policy choices under crisis conditions,
or Hoyt \citep{Hoyt1997tpmo}, who examined American decision process
during Iranian revolution. Analysis of manipulation in selected voting
methods can be found e.g. in \citep{Brandt2016hocs,Gardenfors1976mosc,Gibbard1973movs,Macintyre1993mumd,Smith1999maoc,Taylor2005scat}.
Further on, Faliszewski et al. \citep{Faliszewski2009lacv,Faliszewski2010uctp}
proposed different approaches of manipulation protection in the context
of elections. 

Another area of group decision making vulnerable to manipulation are
social networks. An approach to prevent weight manipulation by minimum
adjustment and maximum entropy method in social network group decision
making can be found in \citep{Sun2022aatp}. Similarly, Wu et al.
\citep{Wu2021aofm} introduced a novel framework to prevent manipulation
behavior in consensus reaching process under social network group
decision making. They considered two means of manipulation: individual
manipulation, where each expert manipulates his/her own behavior to
achieve higher importance (weight); and group manipulation, where
a group of experts forces inconsistent experts to adopt specific recommendation
advices, and investigated models to counteract both kinds of manipulation.
Manipulation in multiple-criteria group decision making attracted
attention of several recent studies. Dong et al. \citep{Dong2021cras}
presented a new strategic manipulation called trust relationship manipulation
and discussed clique-based strategies to manipulate trust relationships
to obtain the desired ranking of the alternatives. Hnatiienko \citep{Hnatiienko2019cmim}
studied the problem of manipulating the choice of decision options
in situations of peer review process and proposed a classification
of selection manipulation problems in experts\textquoteright{} evaluation.
Lev and Lewenberg \citep{Lev2019rgmi} investigated cases when agents
may wish to redraw organizational chart of a company, or markets (which
is called \textquoteleft reverse gerrymandering\textquoteright ) to
maximize their influence across the company\textquoteright s sub-units,
or to allocate resources to the desired areas. Yager \citep{Yager2001pspm,Yager2002dasm}
studied methods of strategic manipulation of preferential data. He
proposed modification of the preference aggregation function in such
a way that the attempts of individual agents to manipulate the data
are penalized. Dong et al. \citep{Dong2018swmi} defined the concept
of the ranking range of an alternative in the multiple attribute decision
making and proposed a series of mixed binary linear programming models
to show the process of designing a strategic attribute weight vector.
Moreover, the authors studied the conditions to manipulate a strategic
attribute weight based on the ranking range and the proposed model.
Sasaki \citep{Sasaki2023smig} discussed the issue of strategic manipulation
in the context of group decision-making with pairwise comparisons.
The author considered a scenario of group decision-making situations
formulated as strategic games and his theoretical results show truthful
judgments (pairwise comparisons) can be a dominant strategy only in
very limited situations. 

Apart from the last study, the problem of manipulation in pairwise
comparisons methods has not been studied thoroughly as of yet. Therefore,
this paper fills the aforementioned gap and focuses on the group decision
making in the analytic hierarchy process (GAHP) and a problem of a
possible manipulation of its outcome. In the GAHP setting, a group
of experts provides pairwise comparisons of alternatives under consideration
with the aim of selecting the best alternative, see e.g. Dong and
Saaty \citep{Dong2014aahp}, Ramanathan and Ganesh \citep{Ramanathan1994gpam},
or Saaty \citep{Saaty1989gdma}. The aim of the paper is to introduce
two heuristics in the GAHP setting allowing to detect the manipulators
and minimize their effect on the group consensus by minimizing their
weights. The first heuristic is based on the assumption that manipulators
will provide judgments which can be considered outliers with respect
to judgments of the rest of the experts in the group. Second heuristic
assumes that dishonest judgments are less consistent than average
consistency of the group. Both approaches are illustrated with numerical
examples and simulations. 

The paper is composed of five sections where \emph{Introduction} (Sec.
\ref{sec:Introduction}) and \emph{Preliminaries} (Sec. \ref{sec:Preliminaries})
aim to introduce the reader to the literature on the subject and recall
the necessary concepts and definitions of the quantitative and qualitative
pairwise comparisons method. The next section (Sec. \ref{sec:Inconsistency-Driven-Pairwise})
\emph{Inconsistency Driven Pairwise Ranking Aggregation} identifies
the problem of ranking manipulation and introduces the proposed robust
methods for aggregating results coming from various experts. The last
but one (Sec. \ref{sec:Montecarlo-experiments}) contains two Montecarlo
experiments allowing to assess effectiveness of the proposed methods.
The presented�work�ends with (Sec. \ref{sec:Summary}) containing
a short summary of the results achieved.

\noindent 

\section{Preliminaries\label{sec:Preliminaries}}

\subsection{Pairwise comparisons\label{subsec:Pairwise-comparisons}}

Comparing alternatives in pairs underlies many decision-making methods
including AHP, BWM, HRE, MACBETH and others \citep{Saaty1977asmf,Rezaei2015bwmc,Kulakowski2014hrea,BanaECosta2016otmf}.
In these methods, the results of the comparisons constitute decision-making
data that are subject to further processing. Let $A=\{a_{1},\ldots,a_{n}\}$
be a finite set of alternatives (available options that each expert
can choose) and $E=\{e_{1},\ldots,e_{k}\}$ be the set of experts
involved in the decision-making process. Similarly, let $C_{q}=\left\{ c_{ijq}\in\mathbb{R}_{+}\,:\,i,j=1,\ldots,n\right\} $
be a set of pairwise judgments provided by the q-th expert so that
$c_{ijq}$ is the relative importance of $a_{i}$ with respect to
$a_{j}$ according to the opinion of $e_{r}$. It is convenient to
represent set of judgments as a pairwise comparisons (PC) matrix $C_{q}=(c_{ijq})$.
For the sake of readability, however, we will try to leave the additional
index $q$ wherever it is not necessary i.e. when the expert's number
will be irrelevant. In such a case the PC matrix takes the form $C=(c_{ij})$.
PC matrix entries can be interpreted as a ratio of individual priorities.
Thus, when for some PC matrix $C$ holds $c_{ij}=x$ we mean that
our expert decided that $a_{i}$ is $x$ times more important than
$a_{j}$. For the same reason $c_{ij}=1$ means that both compared
alternatives are equally preferred. The diagonal of $C$ contains
the results of comparisons of alternatives with themselves, i.e. it
is filled by $1$'s. Similarly, in most of the cases we may expect
that $c_{ij}=c_{ji}^{-1}$. This allows us to formally define this
property. 
\begin{defn}
A PC matrix $C=(c_{ij})$ is said to be reciprocal if for every $c_{ij}$
holds $c_{ij}=c_{ji}^{-1}$. 
\end{defn}

The purpose of the decision-making methods is to prepare recommendations.
It usually takes the form of a numerical ranking that assigns some
real values to the alternatives. 
\begin{defn}
Let $A$ be a set of alternatives. The numerical ranking function
for $A$ is a mapping $w:A\rightarrow\mathbb{R}_{+}$ assigning a
real and positive number to each alternative. 

The numerical ranking takes the form of a weight (priority) vector
$w$:
\[
w=\left[w(a_{1}),\ldots,w(a_{n})\right]^{T}.
\]

In the literature we may find more than a dozen methods allowing us
to determine the priority vector \citep{Kulakowski2020utahp,Mazurek2023aipc}.
The most popular one are EVM (Eigenvalue Method) and GMM (Geometric
Mean Method) \citep{Saaty1977asmf,Crawford1985anot}. According to
the first of these methods the ranking vector is calculated as the
normalized principal eigenvector. Thus, having the solution of equation

\[
C\widehat{w}=\lambda_{\textit{max}}\widehat{w}
\]
where $\lambda_{\textit{max}}$ is a principal eigenvalue of $C$,
entries of priority vector $w_{\textit{ev}}=\left[w_{\textit{ev}}(a_{1}),\ldots,w_{\textit{ev}}(a_{n})\right]^{T}$
are given as 
\[
w_{\textit{ev}}(a_{i})=\frac{\widehat{w}(a_{i})}{\sum_{j=1}^{n}\widehat{w}(a_{j})}.
\]
In the case of the GMM method, although the assumptions of the procedure
are similar \citep{Kulakowski2016srot}, the calculations are simpler.
In this approach the entries of a priority vector: $w_{\textit{gm}}=\left[w_{\textit{gm}}(a_{1}),\ldots,w_{\textit{gm}}(a_{n})\right]^{T}$
have the form:
\[
w_{\textit{gm}}(a_{i})=\frac{s_{i}}{\sum_{j=1}^{n}s_{j}},
\]
where 
\[
s_{i}=\left(\prod_{j=1}^{n}c_{ij}\right)^{1/n}.
\]
Thus, the individual priorities of alternatives are just geometric
means of rows of a PC matrix. Both above methods have their incomplete
versions \citep{Harker1987amoq,kulakowski2020otgm}, i.e. procedures
that allow to calculate the priority vector even if not all entries
of $C$ are known. 
\end{defn}


\subsection{Inconsistency}

Comparing alternatives pairwise is easier than comparing more alternatives
at the same time. However, if one makes comparisons independently,
it may lead (and usually does) to inconsistency. 
\begin{defn}
A PCM $C$ is \textit{consistent} \citep{Kulakowski2020utahp} if
for every $i,j,k=1,\ldots,n$ holds that
\[
c_{ik}c_{kj}=c_{ij}.
\]
 It is fairly easy to prove that it is equivalent to existence a positive
vector $w$ such that for every $i,j=1,\ldots,n$ 
\begin{equation}
c_{ij}=\frac{w(a_{i})}{w(a_{j})}.\label{cons}
\end{equation}
\end{defn}

In real applications inconsistent PCMs appear naturally. Nonetheless,
the level of inconsistency should not be too high as it may lead to
several problems including its impact on sensitivity of data \citep{Kulakowski2019tqoi}
or be a reason for questioning the competence of experts, and thus
be considered as unreliable. Thus, in the literature plenty of inconsistency
indicators have been defined. One of the most popular is the Consistency
Index introduced by Saaty in \citet{Saaty1977asmf}:
\begin{defn}
The Consistency Index of a $n\times n$ PC matrix $C=[c_{ij}]$ is
given by 
\[
\textit{CI}(C)=\frac{\lambda_{max}-n}{n-1},
\]
where $\lambda_{\textit{max}}$ is the principal right eigenvalue
of $C$ (i.e. the maximum one according to the absolute value). 
\end{defn}

The index $\textit{CI}(C)$ is zero if $C$ is consistent. Otherwise,
it is positive, and more specifically $0\leq\textit{CI}(C)\leq(q-1)^{2}/2q$
where $1/q\leq c_{ij}\leq q$, for $i,j=1,\ldots,n$ \citep[p. 103]{Aupetit1993osup,Kulakowski2020utahp}.

Another interesting inconsistency indicator has been proposed by Koczkodaj
\citep{Koczkodaj1993ando}. 
\begin{defn}
The Koczkodaj's inconsistency index of PC matrix $C$ is defined as
\[
K(C)=\max\left\{ K_{ijk}:1\leq i<j<k\leq n\right\} ,
\]

where 
\[
K_{ijk}=\min\left\{ \left|1-\frac{c_{ik}c_{kj}}{c_{ij}}\right|,\left|1-\frac{c_{ij}}{c_{ik}c_{kj}}\right|\right\} .
\]
\end{defn}

The difference between the two indices is that the latter is not related
to the priority deriving method, while the first one contains a reference
to the principal eigenvector of $C$. Both indices have their versions
for incomplete matrices \citep{Kulakowski2020iifi}. Very often, $K(C)$
is considered a \emph{local} inconsistency indicator, while $\textit{CI(C)}$
is called \emph{global} consistency index \citep{Kulakowski2020utahp}.

Ranking vectors can be compared in many ways. Depending on whether
the comparison is quantitative or qualitative, the appropriate metric
is used. A convenient way to compare two ordinal ranking vectors is
the Kendall Tau distance \citep{Kendall1938anmo,Kulakowski2019tqoi}. 
\begin{defn}
The Kendall Tau distance for two ordinal ranking vectors $u$ and
$v$ is defined as 
\begin{equation}
K_{\textit{rd}}(u,v)=\frac{2K_{d}(u,v)}{n(n-1)},\label{eq:norm-kendall-dist}
\end{equation}

where 
\begin{equation}
K_{d}(u,v)=\left|\left\{ (i,j)\,\text{s.t.}\,i<j\,\wedge\,\textit{sign}\left(u(a_{i})-u(a_{j})\right)\neq\textit{sign}\left(v(a_{i})-v(a_{j})\right)\right\} \right|.\label{eq:kendall-dist}
\end{equation}
\end{defn}

$K_{d}(u,v)$ counts the number of pairwise swaps that distinguish
two vectors. For example if $u=(1,2,3)$ and $v=(2,1,3)$ then as
the only one swap between $1$ and $2$ is needed to get vector $v$
from $u$ then in this case $K_{d}(u,v)=1$. Due to the similarity
of this idea to the algorithm of the so-called bubble sort, this value
is sometimes called bubble sort distance. 

It is easy to see that for the most distant two vectors $u$ and $v$
where $\left|u\right|=\left|v\right|=n$, the value of $K_{d}(u,v)=n(n-1)/2$
(in such a case, $u$ contains the elements of $v$ in reverse order).
Thus, the normalized Kendall distance of two ordinal vectors takes
its final form (\ref{eq:norm-kendall-dist}), where for the most distant
vectors $u,v$ the value $K_{\textit{rd}}(u,v)=1$. It is worth noting
that $K_{\textit{rd}}$ does not depend on the number of alternatives
in the ranking.

For quantitative rankings, any measure of vector distance can be used
to determine their distance. For the purpose of this article we use
Manhattan distance, however, one can also meet in the literature Chebyshev
distance \citep{Harker1987ipci}.
\begin{defn}
The Manhattan distance between two cardinal ranking vectors $u$ and
$v$ is defined as
\[
M_{d}(u,v)=\sum_{i=1}^{n}\left|u(a_{i})-v(a_{i})\right|.
\]
\end{defn}

Providing that all the entries of both $u$ and $v$ sum up to $1$
the result $M_{d}(u,v)\leq2$.

\subsection{Group Decision Making\label{subsec:Group-Decision-Making}}

In a situation where many experts work on a recommendation and each
of them presents their own PC matrix, these data must be aggregated.
Typically, arithmetic or geometric weighted averages are used for
aggregation, although there are strong axiomatic arguments for using
geometric mean for this purpose \citep{Aczel1983pfsr}. Hence, in
our further considerations, we will focus on this very method. 

We can aggregate either entire PC matrices or priority vectors resulting
from these matrices. The first approach is called AIJ (Aggregation
of Individual Judgments) while the second AIP (Aggregation of Individual
Priorities \citep{Forman1998aija}). Let us consider a group of experts
$E=\{e^{1},\ldots,e^{k}\}$ whose task is to compare a set $A=\{a_{1},\ldots,a_{n}\}$
of alternatives pairwise. Each of them provides a PC matrix $C_{q}=[c_{ijq}]$
containing its personal judgements on elements of $A$. In the AIJ
approach first we create the aggregated matrix

\[
C=\left[\begin{array}{cccc}
1 & \prod_{q=1}^{k}c_{12q}^{r_{q}} & \cdots & \prod_{q=1}^{k}c_{1nq}^{r_{q}}\\
\vdots & 1 & \vdots & \vdots\\
\vdots & \vdots & \ddots & \vdots\\
\prod_{q=1}^{k}c_{n1q}^{r_{q}} & \cdots & \cdots & 1
\end{array}\right],
\]
where $r_{1},\ldots,r_{k}\in[0,1]$ and $\sum_{q=1}^{k}r_{q}=1$.
Then, adopting $C$ as input, we calculate the final priority vector
using the method we prefer. The values $r_{1},\ldots,r_{k}$ mean
the priorities assigned to individual experts. They correspond to
the strength of the influence of individual experts' opinions on the
final result. In a situation where the opinion of each of the experts
counts the same (this is most often the case) $r_{q}=1/k$ for $q=1,\ldots,k$. 

In the AIP approach, first for each $C_{q}$ we calculate a priority
vector 
\[
w_{q}=\left[w_{q}(a_{1}),\ldots,w_{q}(a_{n})\right]^{T}.
\]
Then, we aggregate vectors so that resulting ranking is given as 
\[
w=\left[w(a_{1}),\ldots,w(a_{n})\right]^{T},
\]
where
\begin{equation}
w(a_{i})=\frac{\prod_{q=1}^{k}w_{q}^{r_{q}}(a_{i})}{\sum_{i=1}^{n}\prod_{q=1}^{k}w_{q}^{r_{q}}(a_{i})}.\label{eq:aij-final-weight}
\end{equation}
 Similarly as before, the higher value of $r_{q}$ the stronger impact
of q-th expert to the final recommendation. 

\section{Inconsistency Driven Pairwise Ranking Aggregation\label{sec:Inconsistency-Driven-Pairwise}}

\subsection{Problem statement\label{subsec:Problem-statement}}

In the decision making method, the common assumption is the honesty
and professionalism of experts. According to the first of these assumptions,
each of the experts will try to express opinions that are as close
to the actual state as possible. In other words, they will not express
an opinion that is contrary to their own knowledge and inner conviction.
Their alleged professionalism, on the other hand, allows us to believe
that the assessment made by experts will be reliable and will be based
on a possibly objective comparison of various considered options.
Both of these assumptions allow us to hope that the judgments of different
but honest and professional experts should rather coincide. Therefore,
outliers are likely either dishonest or unprofessional. In either
case, there is a good reason to reduce the impact of such opinions
on the final result.

Describing the facts and fiction about AHP \citep[p. 22]{Forman1993fafa},
Forman notes that \textquotedbl It is possible to be perfectly consistent
but consistently wrong''. Similarly, to paraphrase Forman, one might
say that it is possible to be perfectly consistent and completely
dishonest. Nevertheless, in practice, when there are many alternatives
and little time to decide, the questions are asked in a random order
and the expert does not have the opportunity to learn about the set
of alternatives beforehand, the chance of giving consistent but insincere
answers does not seem very high. Hence, we can expect the insincere
expert will give less consistent answers than the average. This leads
to the formulation of a second possible heuristic that may point to
dishonest or incompetent experts. Too high inconsistency in an expert's
response may be a good reason for reducing their impact on the final
ranking values. 

The above observations allow us to propose two procedures for prioritizing
experts, so that potentially dishonest experts receive a lower priority
than others. The first procedure will be based on a deviation from
the average (i.e. it will detect outliers in terms of the preferences
presented) (Sec. \ref{subsec:Distance-driven-expert-prioritiz}).
The second will take into account the inconsistency in the context
of a certain average inconsistency (Sec. \ref{subsec:Inconsistency-driven-expert-prio}).
We are also considering combination both of the above heuristics (Sec.
\ref{subsec:Inconsistency-driven-expert-prio-1}).

\subsubsection*{Example\label{aggr}}

One of the common variants of manipulation in the social choice theory
is control \citep{Brandt2016hocs}. It is usually carried out by the
election organizer. Paradigmatic examples of control are adding or
deleting voters. A similar effect can be seen with the pairwise comparison
method. Hence, by adding or removing experts, one can try to affect
the ranking results\footnote{In contrast to the rank reversal problem \citep{Dyer1990rota}, in
this case we are dealing with deliberately crafted experts answers
.}. Let us consider the example where the opinion of six experts $e_{1},\ldots,e_{6}$
was taken into account in order to develop recommendations on the
four alternatives considered. The experts' opinions were in the form
of $4\times4$ matrices: 

\noindent 
\[
C_{1}=\left(\begin{array}{cccc}
1. & 1.322 & 1.926 & 2.0784\\
0.7566 & 1. & 1.63 & 2.6057\\
0.5192 & 0.6136 & 1. & 1.011\\
0.4811 & 0.3838 & 0.9888 & 1.
\end{array}\right),
\]

\[
C_{2}=\left(\begin{array}{cccc}
1. & 1.335 & 2.408 & 3.2864\\
0.749 & 1. & 2.0619 & 1.619\\
0.4153 & 0.485 & 1. & 1.599\\
0.3043 & 0.6178 & 0.6255 & 1.
\end{array}\right),
\]

\[
C_{3}=\left(\begin{array}{cccc}
1. & 1.034 & 2.0437 & 2.6168\\
0.9668 & 1. & 1.956 & 1.987\\
0.4893 & 0.5113 & 1. & 1.32\\
0.3821 & 0.5034 & 0.7574 & 1.
\end{array}\right),
\]

\[
C_{4}=\left(\begin{array}{cccc}
1. & 1.3 & 2.1679 & 2.0552\\
0.7691 & 1. & 1.603 & 2.5181\\
0.4613 & 0.624 & 1. & 1.005\\
0.4866 & 0.3971 & 0.9949 & 1.
\end{array}\right),
\]
\[
C_{5}=\left(\begin{array}{cccc}
1. & 1.052 & 2.1005 & 2.5668\\
0.9501 & 1. & 2.025 & 1.646\\
0.4761 & 0.4938 & 1. & 1.564\\
0.3896 & 0.6074 & 0.6394 & 1.
\end{array}\right),
\]
\[
C_{6}=\left(\begin{array}{cccc}
1. & 1.153 & 2.2591 & 2.4132\\
0.8669 & 1. & 1.832 & 1.629\\
0.4426 & 0.5459 & 1. & 1.284\\
0.4144 & 0.6137 & 0.7785 & 1.
\end{array}\right).
\]

\noindent The normalized weight vectors obtained by GMM are as follows:
\[
w_{1}=\left[0.355802,0.314085,0.176758,0.153356\right]^{T},
\]

\noindent 
\[
w_{2}=\left[0.409827,0.285829,0.171231,0.133114\right]^{T},
\]
\[
w_{3}=\left[0.356504,0.323638,0.176235,0.143623\right]^{T},
\]
\[
w_{4}=\left[0.362966,0.31053,0.171584,0.154919\right]^{T},
\]
\[
w_{5}=\left[0.360622,0.311708,0.181946,0.145724\right]^{T},
\]
\[
w_{6}=\left[0.371263,0.297355,0.174992,0.156391\right]^{T},
\]

\noindent which means that the six experts indicate $a_{1}$ as the
best alternative. Their aggregated ranking is 
\[
w_{1-6}=\left[0.369045,0.306942,0.175421,0.147627\right]^{T}.
\]

\noindent However, a dishonest organizer (process facilitator) added
two more experts $e_{7}$ and $e_{8}$ who lobby for $a_{2}$. As
they know that $a_{1}$ is its main competitor, they proposed opinions
(matrices $C_{7}$ and $C_{8}$) i.e.: 

\noindent 
\[
C_{7}=\left(\begin{array}{cccc}
1. & 0.1752 & 0.5622 & 0.3212\\
5.7078 & 1. & 2.584 & 1.604\\
1.779 & 0.387 & 1. & 1.785\\
3.1135 & 0.6236 & 0.5602 & 1.
\end{array}\right),
\]
\[
C_{8}=\left(\begin{array}{cccc}
1. & 0.4551 & 0.2515 & 0.5273\\
2.1971 & 1. & 2.0847 & 2.2497\\
3.9758 & 0.4797 & 1. & 1.47\\
1.896 & 0.4445 & 0.6803 & 1.
\end{array}\right),
\]

\noindent resulting the ranking lowering $a_{1}$ and boosting $a_{2}$:

\noindent 
\[
w_{7}=\left[0.0897127,0.469102,0.223955,0.21723\right]^{T},
\]

\noindent 
\[
w_{8}=\left[0.0897127,0.469102,0.223955,0.21723\right]^{T}.
\]
Then applying the standard aggregation process \citep{Forman1998aija,Groselj2015cosa},
the final ranking is as follows: 
\[
w_{1-8}=\left[0.266227,0.334807,0.192645,0.160465\right]{}^{T}.
\]
This determines the order of alternatives: $a_{2},a_{1},a_{3},a_{4}$
which is in line with the expectations of experts $e_{7}$ and $e_{8}$. 

\subsection{Preferential distance-driven expert prioritization\label{subsec:Distance-driven-expert-prioritiz}}

Let $\widehat{w}_{i}=\left[\widehat{w}_{i}(a_{1}),\ldots,\widehat{w}_{i}(a_{k})\right]^{T}$
be the ranking vector calculated using either EVM or GMM (Sec. \ref{subsec:Pairwise-comparisons})
based on the matrix $C_{i}$ provided by the expert $e_{i}$ for $i=1,\ldots,k$.
Similarly, let $\widehat{w}=\left[\widehat{w}(a_{1}),\ldots,\widehat{w}(a_{n})\right]^{T}$
be a ranking vector calculated using AIP (Sec. \ref{subsec:Group-Decision-Making})
based on $\widehat{w}_{1},\ldots,\widehat{w}_{k}$. According to the
adopted heuristics, the more the opinion of the i-th expert differs
from that of the team of experts, the greater the risk of manipulation.
Thus, let 
\begin{equation}
\textit{d}_{i}=\left|\widehat{w}-\widehat{w}_{i}\right|\label{eq:indiv-distances}
\end{equation}
be a quantitative distance\footnote{For example Manhattan distance, Euclidean distance or Chebyshev distance.}
between vectors $\widehat{w}$ and $\widehat{w}_{i}$. Then we need
to map the individual distances $d_{i}$ (\ref{eq:indiv-distances})
to priority values over a certain numerical scale. For this purpose,
let us denote the minimum and maximum of the values $D=\{d_{1},\ldots,d_{k}\}$
as $d_{\textit{min}}$ and $d_{\textit{max}}$ correspondingly. As
$d_{\textit{min}}$ corresponds to the most preferred expert and $d_{\textit{min}}$
corresponds to the least preferred expert, we assign the value $h\in\mathbb{R}_{+}$
to $d_{\textit{min}}$ and $l\in\mathbb{R}_{+}$ to $d_{\textit{max}}$,
where of course $h>l$. The ratio $h/l$ should correspond to the
comparison of the reliability of the expert corresponding to $d_{\textit{min}}$
to the reliability of the expert corresponding to $d_{\textit{max}}$.
Values $l$ and $h$ form the scale on which all the distances $d_{1},\ldots,d_{k}$
will be transformed. 

Let $f:\mathbb{R}_{+}\rightarrow\mathbb{R}$ be a mapping transforming
distances $d_{i}$ to priorities, which after normalization can be
used in the weighted AIP procedure. The function $f$ should pass
through two points $X=(d_{\textit{min}},h)$ and $Y=(d_{\textit{max}},l)$,
so that the highest priority value $h$ is assigned to the expert
whose opinion was closest to the mean, and the lowest priority value
$l$ is assigned to the expert whose opinion was the farthest from
the mean. 

As the mapping $f$ let us use a linear function passing through two
points $X=(x_{1},x_{2})$ and $Y=(y_{1},y_{2})$ in the form 
\begin{equation}
f_{XY}(x)=\frac{x_{2}-y_{2}}{x_{1}-y_{1}}x+\left(x_{2}-\frac{x_{2}-y_{2}}{x_{1}-y_{1}}x_{1}\right).
\end{equation}
This allow us to calculate the values\footnote{As long as points $X$ and $Y$ are well defined, we will write $f(x)$
instead of $f_{XY}(x)$.} $f(d_{1}),\ldots,f(d_{k})$ which determines the priorities of experts
$e_{1},\ldots,e_{k}$. In order to satisfy the form of the weighted
geometric mean one would need to rescale the priority values so that
they sum up to one. Hence, the final experts' priorities take the
form: 

\begin{equation}
r_{i}=\frac{f(d_{i})}{\sum_{i=1}^{k}f(d_{i})},\,\,\,i=1,\ldots,k.\label{eq:prior_resc}
\end{equation}
After computing $r_{1},\ldots,r_{k}$ we calculate the final priority
vector $w=\left[w(a_{1}),\right.$ $\ldots$ $\left.,w(a_{n})\right]^{T}$
using the weighted version of AIP (\ref{eq:aij-final-weight}). 

For the purpose of the example (end of the Section \ref{subsec:Problem-statement}),
let us assume that the expert with the lowest value $d_{\textit{min}}$
is strongly more credible than the expert with the highest value $d_{\textit{max}}$.
Following the fundamental scale the ratio $h/l=5$, so we may assign
$h=5$ and $l=1$. In result we get the linear mapping function $f(x)=-15.0607x+7.0075$
determining the expert weights (Fig. \ref{fig:linear-mappin-example}). 

\begin{figure}
\begin{centering}
\includegraphics[width=0.9\textwidth]{figlinmap_tex}
\par\end{centering}
\caption{Linear function mapping distances to priorities passing points $(min,h)$
and $(max,l)$. }
\label{fig:linear-mappin-example}
\end{figure}
The weights are: 
\[
\left[f(d_{1}),\ldots,f(d_{8})\right]^{T}=\left[5.,3.37,4.97,4.78,4.85,4.53,1.,1.65\right]^{T}.
\]
Rescaling produces the values that can be used as input to AIP procedure:
\[
\left[r_{1},\ldots,r_{8}\right]^{T}=\left[0.165,0.11,0.16,0.15,0.16,0.15,0.0331,0.054\right]^{T}.
\]
After re-aggregating the results, we get a priority vector:
\[
w_{1-8}=\left[0.327,0.317,0.182,0.152\right]^{T}.
\]
As we can see the alternative $a_{1}$ with the priority $w_{1-8}(a_{1})=0.327$,
which is preferred by the majority of voters returned to the winner's
position. 

\subsection{Inconsistency-driven expert prioritization\label{subsec:Inconsistency-driven-expert-prio}}

Instead of distance between individual judgments vectors and the mean
we may use distance between inconsistencies of individual experts
and the average inconsistency of their judgments. Thus, let 
\[
d_{i}=\left(I(C_{i})-\frac{1}{k}\sum_{j=1}^{k}I(C_{j})\right),
\]

where $I$ is some selected inconsistency index \citep{Brunelli2018aaoi,Kulakowski2020iifi}.
Contrary to the previous case, in which the distance of the individual
ranking from the average value was important, here we have to distinguish
whether the inconsistency of a given expert is below or above the
average. If it is above average, it may mean either an attempt (perhaps
naive) of manipulation or the deficiencies of the expert himself (lack
of experience, lack of firmness, distraction, time pressure, etc.).
However, if the individual ranking is below average, i.e., the expert's
consistency is higher than the average, this may, to some extent,
be in favor of the expert. On the other hand, a certain degree of
inconsistency is considered desirable \citep[p. 265]{Saaty2008rmai}
or \citep[p. 172]{Saaty1987tahp}. In other words, both too high and
too small a degree of inconsistency may indicate a strategic decision-making,
although it is quite difficult to \textquotedbl punish\textquotedbl{}
too much consistency.

These two observations lead to the conclusion that the mapping of
distances to priorities should differ depending on the sign of $d_{i}$.
Providing that there exists at least two matrices $C_{i}$ and $C_{j}$
such that $I(C_{i})\neq I(C_{j})$ then there must exists two matrices
such that $I(C_{p})<1/k\sum_{j=1}^{k}I(C_{j})<I(C_{\textit{q}})$.
Let the inconsistency values for the most consistent $e_{\textit{min}}$
and inconsistent $e_{\textit{max}}$ expert be: $I_{\textit{min}}=\min_{j=1,\ldots,k}I(C_{j})$
and $I_{\textit{max}}=\max_{j=1,\ldots,k}I(C_{j})$. Additionally
let $I_{\textit{mid}}$ is the value of the inconsistency of the expert
whose opinions' consistency is closest to the average, i.e. $I_{\textit{mid}}=I(C_{i})$
such that $\left|I(C_{i})-1/k\sum_{j=1}^{k}I(C_{j})\right|$ is minimal.

In the next step, we should set the weights high, middle and low i.e.
$h,m$ and $l$ corresponding to the values $I_{\textit{min}},I_{\textit{mid}}$
and $I_{\textit{max}}$. Thus, we perform three comparisons of experts'
credibility, which results in the following matrix: 
\begin{equation}
\textit{C}_{\textit{ex}}=\left(\begin{array}{ccc}
1 & \textit{c}_{\textit{min},\textit{mid}} & \textit{c}_{\textit{min},\textit{max}}\\
1/\textit{c}_{\textit{min},\textit{mid}} & 1 & \textit{c}_{\textit{mid},\textit{max}}\\
1/\textit{c}_{\textit{min},\textit{max}} & 1/\textit{c}_{\textit{mid},\textit{max}} & 1
\end{array}\right).\label{eq:tree-point-credibility-matrix}
\end{equation}
It is worth noting that the heuristic the smaller the inconsistency,
the better results in the constraint according to which $\textit{c}_{\textit{min},\textit{mid}}$,
$\textit{c}_{\textit{mid},\textit{max}}$ and $\textit{c}_{\textit{min},\textit{max}}$
cannot be greater than $1$. 

Calculating the ranking based on (\ref{eq:tree-point-credibility-matrix})
brings us the desired values $h,m$ and $l$. These allow us to formulate
three points through which the function mapping the distance $d_{i}$'s
to the weight values of individual experts should pass. These points
are: $A=(d_{\textit{min}},h)$, $B=(d_{\textit{mid}},m)$ and $C=(d_{\textit{max}},l)$.

As mapping $f:\mathbb{R}_{+}\rightarrow\mathbb{R}$ transforming distances
$I_{i}$ to priorities let us use a piecewise linear function including
two segments: $A-B$ and $B-C$. Thus, 
\[
f_{XYZ}(x)=\begin{cases}
f_{XY}(x) & x<0\\
f_{YZ}(x) & x\geq0
\end{cases}.
\]
Similarly as before $f_{ABC}=f$ allows us to calculate values $f(d_{1}),\ldots,f(d_{k})$
which after appropriate rescaling (\ref{eq:prior_resc}) form experts'
priorities $r_{1},\ldots,r_{k}$. Finally, the ranking is calculated
taking into account the priorities of individual experts.

In the case of Example \ref{aggr} experts $e_{1},\ldots,e_{8}$ achieved
the following consistency values:
\[
w_{\textit{cexp}}=\{0.0101,0.0152,0.0026,0.0088,0.013,0.0049,0.0528,0.0519\}
\]
It is easy to see that $I_{\textit{min}}=0.0026$ (expert $e_{3}$)
and $I_{\textit{max}}=0.0528$ (expert $e_{7}$). The average inconsistency
is $0.02$, thus the nearest inconsistency result was achieved by
the expert $e_{2}$ with $I_{\textit{mid}}=0.0152$. In the next step
we need to compare credibility of $e_{3},e_{2}$ and $e_{7}$. Let
\[
C_{\textit{ex}}=\left(\begin{array}{ccc}
1 & 2 & 7\\
\frac{1}{2} & 1 & 4\\
\frac{1}{7} & \frac{1}{4} & 1
\end{array}\right),
\]
what results the following credibility values: $0.603,0.315$ and
$0.082$. This allow us to construct the mapping $f_{XYZ}$ (Fig.
\ref{fig:linear-mappin-example-2}) where $X=(0.0026,0.603)$, $Y=(0.02,0.315)$
and $Z=(0.082,0.082)$. 

\begin{figure}
\begin{centering}
\includegraphics[width=0.9\columnwidth]{figpiecewiselinmap_tex}
\par\end{centering}
\caption{Piecewise linear function mapping distances to priorities passing
points $A,B$ and $C$.}
\label{fig:linear-mappin-example-2}
\end{figure}
According to the function $f_{XYZ}=f$ experts $e_{1},\ldots,e_{8}$
get the following weights: 
\[
\left[f(d_{1}),\ldots,f(d_{8})\right]^{T}=[0.432,0.315,0.602,0.462,0.354,0.551,0.082,0.088]^{T}.
\]
After rescaling so that all weights sum up to $1$ we obtain the values
that can be used as input to AIP procedure: 

\[
\left[r_{1},\ldots,r_{8}\right]^{T}=\left[0.149,0.109,0.208,0.159,0.122,0.19,\ 0.028,0.03\right]^{T}.
\]
Then re-aggregating of expert results leads to a final priority vector:
\[
w_{1-8}=\left[0.339,0.314,0.1793,0.151\right]^{T}.
\]

The alternative $a_{1}$ with the priority $w_{1-8}(a_{1})=0.329$
has the highest rank, whilst $a_{2}$ takes a proper second place.
Thanks to the introduction of priorities, it was once again possible
to avoid manipulation.

\subsection{Mixed expert prioritization\label{subsec:Inconsistency-driven-expert-prio-1}}

It is possible to use both of the above heuristics simultaneously.
So let $r_{i}^{1}$ be the weight of the i-th expert calculated on
the basis of the heuristic distance from the mean judgment (Sec. \ref{subsec:Distance-driven-expert-prioritiz}),
while $r_{i}^{2}$ be the weight of the expert calculated from the
difference in inconsistencies (Sec. \ref{subsec:Inconsistency-driven-expert-prio}).
Thus, the mixed expert weight is the linear combination of both: 
\[
r_{i}=\beta r_{i}^{1}+(1-\beta)r_{i}^{2},\,\,\text{for}\,\,i=1,\ldots,k,
\]
where $0\leq\beta\leq1$ is the coefficient determining the impact
of both heuristics. Since $\sum_{i=1}^{k}r_{i}^{1}=\sum_{i=1}^{k}r_{i}^{2}=1$
then also $\sum_{i=1}^{k}r_{i}=1$. Thus, the obtained weights $r_{1},\ldots,r_{k}$
fit the definition of the weighted geometric mean.

In the case of the Example \ref{aggr} and assuming that both heuristics
contribute equally to the weights of the experts i.e. $\beta=0.5$,
we get 

\[
\left[r_{1},\ldots,r_{8}\right]^{T}=\left[0.157,0.11,0.186,0.159,0.141,0.17,\ 0.03,0.042\right]^{T},
\]
which results in the following priority vector: 
\[
w_{1-8}=\left[0.333,0.316,0.18,0.151\right]^{T}.
\]


\subsection{The degree of expert's credibility }

In each of the two heuristics described above (Section \ref{subsec:Distance-driven-expert-prioritiz}
and \ref{subsec:Inconsistency-driven-expert-prio}), two or three
key experts are first selected for credibility assessment. Then, based
on this result, a mapping is proposed to prioritize all experts. The
credibility evaluation must be made in accordance with the adopted
heuristics, i.e. experts less consistent in their judgments or more
distant from the average than the competitor have to get the smaller
score. This may bring�a certain inconvenience for those assessing
the credibility of experts. They may focus�on their attitude towards
individual experts, and not on the quality of their expertise. As
the result a person who is liked and popular in the society may get
a better assessment than a reliable but not sociable expert. 

The way to avoid this trap is to provide a procedure that allow us
to prioritize key experts without their explicit comparisons. So we
can assume a priori that the ratio of the best to the worst expert
(the first heuristic, Section \ref{subsec:Distance-driven-expert-prioritiz})
is e.g. $5:1$, or ratios of the best, average and worst expert (the
second heuristic, Section \ref{subsec:Inconsistency-driven-expert-prio})
is e.g. $9:4:1$. 

It is also possible to determine these relations in a procedural /
functional way. For instance in the case of our example and the second
heuristics we may assume that the expert priority should be linearly
correlated with inconsistency. Thus, as $d_{\textit{min}}=0.0026$,
$d_{\textit{mid}}=0.0152$ and $d_{\textit{max}}=0.0528$ (Section
\ref{subsec:Inconsistency-driven-expert-prio}) the priority may take
the values: $h=\alpha\cdot d_{\textit{max}}/d_{\textit{min}}$, $m=\alpha\cdot d_{\textit{mid}}/d_{\textit{min}}$
and $l=1$, where $\alpha\geq1$ is a gain factor. 

\section{Montecarlo experiments\label{sec:Montecarlo-experiments}}

\subsection{Data preparation}

For the purpose of both experiments (Sections \ref{subsec:Defense-against-manipulation}
and \ref{subsec:Vulnerability-to-original}) we prepared $4,000$
sets of $20$-matrix sets corresponding to different decision scenarios.
For this purpose, we first drew $34$ priority vectors for $5$ alternatives,
$33$ vectors for $6$ alternatives, and another $33$ vectors for
$7$ alternatives. Then, as every priority vector corresponds to exactly
one consistent PC matrix \citep{Kulakowski2020utahp}, we created
$100$ consistent PC matrices of sizes $5\times5,6\times6$ and $7\times7$.
Thus, if 
\[
w=[w(a_{1}),\ldots,w(a_{5})]^{T}
\]
is a priority vector corresponding to a certain five alternatives
then the consistent matrix corresponding to $w$ is: 
\[
C_{w}=\left(\begin{array}{ccccc}
1 & \frac{w(a_{1})}{w(a_{2})} & \frac{w(a_{1})}{w(a_{3})} & \frac{w(a_{1})}{w(a_{4})} & \frac{w(a_{1})}{w(a_{5})}\\
\frac{w(a_{2})}{w(a_{1})} & 1 & \frac{w(a_{2})}{w(a_{3})} & \frac{a_{2})}{w(a_{4})} & \frac{w(a_{2})}{a_{5})}\\
\frac{w(a_{3})}{w(a_{1})} & \frac{w(a_{3})}{w(a_{2})} & 1 & \frac{w(a_{3})}{w(a_{4})} & \frac{w(a_{3})}{w(a_{5})}\\
\frac{w(a_{4})}{w(a_{1})} & \frac{w(a_{4})}{w(a_{2})} & \frac{w(a_{4})}{w(a_{3})} & 1 & \frac{w(a_{4})}{w(a_{5})}\\
\frac{w(a_{5})}{w(a_{1})} & \frac{w(a_{5})}{w(a_{2})} & \frac{w(a_{5})}{w(a_{3})} & \frac{w(a_{5})}{w(a_{4})} & 1
\end{array}\right).
\]

In the next step, we disturbed the elements of these matrices by multiplying
them by a random factor $\epsilon\in[1/\alpha,\alpha],$ where $\alpha=1.1,1.2,\ldots,5$.
Thus, the disturbed version of $C_{w}$ takes the form: 

\[
\widetilde{C}_{w,\alpha}=\left(\begin{array}{ccccc}
1 & \epsilon_{12}\frac{w(a_{1})}{w(a_{2})} & \epsilon_{13}\frac{w(a_{1})}{w(a_{3})} & \epsilon_{14}\frac{w(a_{1})}{w(a_{4})} & \epsilon_{15}\frac{w(a_{1})}{w(a_{5})}\\
\epsilon_{21}\frac{w(a_{2})}{w(a_{1})} & 1 & \epsilon_{23}\frac{w(a_{2})}{w(a_{3})} & \epsilon_{24}\frac{a_{2})}{w(a_{4})} & \epsilon_{25}\frac{w(a_{2})}{a_{5})}\\
\epsilon_{31}\frac{w(a_{3})}{w(a_{1})} & \epsilon_{32}\frac{w(a_{3})}{w(a_{2})} & 1 & \epsilon_{34}\frac{w(a_{3})}{w(a_{4})} & \epsilon_{35}\frac{w(a_{3})}{w(a_{5})}\\
\epsilon_{41}\frac{w(a_{4})}{w(a_{1})} & \epsilon_{42}\frac{w(a_{4})}{w(a_{2})} & \epsilon_{43}\frac{w(a_{4})}{w(a_{3})} & 1 & \epsilon_{45}\frac{w(a_{4})}{w(a_{5})}\\
\epsilon_{51}\frac{w(a_{5})}{w(a_{1})} & \epsilon_{52}\frac{w(a_{5})}{w(a_{2})} & \epsilon_{53}\frac{w(a_{5})}{w(a_{3})} & \epsilon_{54}\frac{w(a_{5})}{w(a_{4})} & 1
\end{array}\right),
\]

where $\epsilon_{ij}\in[1/\alpha,\alpha]$ and $\epsilon_{ij}=1/\epsilon_{ji}$
for $i,j=1,\ldots,5$. For every consistent PC matrix $C_{w_{x}}$
(where $x=1,\ldots,100$) and $\alpha_{k}\in\{1.1,\ldots,5\}$ we
randomly selected $20$ matrices $\widetilde{C}_{w,\alpha_{k}}^{(q)}$
where $q=1,\ldots,20$. Thus, every set $S_{w_{x}\alpha_{k}}=\left\{ \widetilde{C}_{w_{x},\alpha_{k}}^{(1)},\ldots,\widetilde{C}_{w_{x},\alpha_{k}}^{(q)}\right\} $
corresponds to a single group decision scenario in which $q=20$ experts
make decisions as to the priorities of $5,6$ or $7$ alternatives
convergent (to some extent) with the vector $w_{x}$. The input to
each experiment were $4,000$ such $S_{w_{x}\alpha_{k}}$ corresponding
to different sets of alternatives ($100$ vectors $w_{x}$) and different
average inconsistency of experts ($40$ different values of $\alpha_{k}$). 

For every $S_{w_{x}\alpha_{k}}$ we determined the average level of
inconsistency as the arithmetic mean of its components. I.e. 
\[
I(S_{w_{x}\alpha_{k}})=\frac{I\left(\widetilde{C}_{w_{x},\alpha_{k}}^{(1)}\right)+\ldots+I\left(\widetilde{C}_{w_{x},\alpha_{k}}^{(q)}\right)}{\left|S_{w_{x}\alpha_{k}}\right|},
\]
where $I$ denotes the inconsistency indicator (for the purpose of
the experiments we used Saaty's consistency index $\textit{CI}$)
and $q=\left|S_{w_{x}\alpha_{k}}\right|$ is the number of experts
involved in the decision-making process (for the purposes of the experiments,
we adopted the number 20). As a general rule, an increase in $\alpha_{k}$
causes an increase in $I(S_{w_{x}\alpha_{k}})$. 

For the purpose of the experiments we used GMM (Section \ref{subsec:Pairwise-comparisons})
for priorities calculation. In such a case, it is easy to show that
AIP and AIJ (Section \ref{subsec:Group-Decision-Making}) lead to
the same results, we do not need to consider both aggregation methods
separately. 

\subsection{Defense against manipulation\label{subsec:Defense-against-manipulation}}

\subsubsection{Model of manipulation\label{subsec:Model-of-manipulation}}

In the first experiment, we will assume that a certain number of experts
are bribed to submit manipulated matrices. For the purposes of the
experiment, we assume that the grafter's goal is to make the original
second alternative the winner of the ranking. For this purpose, he
bribes several experts who are most in favor of the original winner
(bribing experts who do not support the current winner seems to be
a less effective strategy). In exchange for a bribe, the experts undertake
to indicate that the comparison of the second best alternative so
far with any other is $9$ (the largest value of the fundamental scale),
and the comparison of the current leader with any other alternative
is $1/9$. 

Let us see how this somewhat simple group decision-making manipulation
scheme works on the simple example. In order to evaluate the five
alternatives, four different experts would prepare four pairwise comparison
matrices:
\[
C_{1}=\left(\begin{array}{ccccc}
1. & 0.246 & 3.731 & 2.75 & 0.65\\
4.069 & 1. & 10.83 & 9.54 & 1.25\\
0.268 & 0.092 & 1. & 0.39 & 0.311\\
0.363 & 0.105 & 2.566 & 1. & 0.729\\
1.54 & 0.799 & 3.219 & 1.37 & 1.
\end{array}\right),
\]

\[
C_{2}=\left(\begin{array}{ccccc}
1. & 0.347 & 1.03 & 0.668 & 0.629\\
2.881 & 1. & 7.614 & 5.696 & 0.835\\
0.969 & 0.131 & 1. & 1.98 & 0.335\\
1.5 & 0.175 & 0.506 & 1. & 0.734\\
1.59 & 1.2 & 2.982 & 1.36 & 1.
\end{array}\right),
\]

\[
C_{3}=\left(\begin{array}{ccccc}
1. & 0.798 & 2.497 & 0.61 & 0.359\\
1.25 & 1. & 4.799 & 1.57 & 1.29\\
0.4 & 0.208 & 1. & 0.374 & 0.544\\
1.64 & 0.637 & 2.672 & 1. & 1.01\\
2.787 & 0.774 & 1.84 & 0.99 & 1.
\end{array}\right),
\]

\[
C_{4}=\left(\begin{array}{ccccc}
1. & 0.83 & 1.28 & 1.91 & 0.535\\
1.2 & 1. & 8.778 & 7.832 & 3.302\\
0.781 & 0.114 & 1. & 1.09 & 0.244\\
0.524 & 0.128 & 0.921 & 1. & 0.228\\
1.87 & 0.303 & 4.097 & 4.376 & 1.
\end{array}\right).
\]
which would result in the following four priority vectors: 
\[
w_{1}=[0.16,0.507,0.045,0.085,0.203]^{T},
\]
\[
w_{2}=[0.115,0.425,0.102,0.105,0.252]^{T},
\]

\[
w_{3}=[0.154,0.301,0.081,0.224,0.24]^{T},
\]

\[
w_{4}=[0.154,0.467,0.072,0.065,0.242]^{T}.
\]
After aggregation via the AIJ method the ranking vector would look
like: 
\[
w=[0.145,0.417,0.072,0.107,0.233]^{T},
\]
i.e. the winner is the second alternative with the score $w(a_{2})=0.417$.
Therefore, in order to push through the candidature of the vice-leader
of the current ranking, i.e. alternative $a_{5}$ with the score $w(a_{5})=0.233$,
grafter bribes $a_{2}$'s the strongest supporter, i.e. the expert
no. 1. Hence, in fact, the first expert submits a manipulated matrix:
\[
\widetilde{C}_{1}=\left(\begin{array}{ccccc}
1 & 9 & 3.731 & 2.75 & 1/9\\
1/9 & 1 & 1/9 & 1/9 & 1/9\\
0.268 & 9 & 1 & 0.39 & 1/9\\
0.363 & 9 & 2.566 & 1 & 1/9\\
9 & 9 & 9 & 9 & 1
\end{array}\right).
\]
After aggregation $\widetilde{C}_{1},C_{2},C_{3}$ and $C_{4}$ it
turns out that the final priorities looks like follows:
\[
\widetilde{w}=[0.148,0.183,0.08,0.113,0.31]^{T},
\]
which means that the manipulation was successful. The new winner was
alternative $a_{5}$ with the score $w(a_{5})=0.31$, which without
manipulation would have taken the second position. If bribing one
expert was not enough, the grafter would try to bribe next by one
strongest supporter of $a_{2}$ etc.

In the above procedure, we assume that grafter knows who the winner's
strongest supporter is (and bribery of whom could potentially be most
disadvantageous to the winner and beneficial to the preferred alternative).
In practice, grafter usually does not have such knowledge and must
rely on his intuition and knowledge of expert preferences. So one
can hope that in practice the real grafter will work less efficiently
than the one from our experiment.

\subsubsection{Experiment results}

The input data for the experiment was $4,000$ sets $S_{w_{x}\alpha_{k}}$
composed of twenty PC matrices, each corresponding to the given initial
priority vector $w_{x}$ and the range of disturbance factor $\alpha_{k}$.
For each set $S_{w_{x}\alpha_{k}}$ first we calculate the aggregated
priorities\footnote{For the sake of simplicity we assume that the best alternative in
the non-manipulated ranking is indexed as $a_{1}$, the second best
alternative is $a_{2}$ and so on.} 

\begin{equation}
\bar{w}_{x}=\left[\bar{w}_{x}(a_{1}),\bar{w}_{x}(a_{2}),\ldots,\bar{w}_{x}(a_{n})\right]^{T},\label{eq:ranking-with-standard-aip}
\end{equation}
and based on it we carry out a simulated manipulation attack in accordance
with the method described in (Section \ref{subsec:Model-of-manipulation}).
We also determine the average inconsistency of experts' responses
$I\left(S_{w_{x}\alpha_{k}}\right)$. As expected, along the increase
of the $\alpha_{k}$ coefficient, the average inconsistency also gets
higher. 

In most cases, it is enough to \textquotedbl bribe\textquotedbl{}
one to three experts, i.e. manipulate from one to three matrices from
the entire $S_{w_{x}\alpha_{k}}$ set. In each of the analyzed cases,
the attack method is effective. This means that in each considered
case it is possible to \textquotedbl improve\textquotedbl{} the experts'
answers to achieve the intended goal, i.e. to promote the second best
alternative in the original ranking to the leader. Let us denote the
manipulated set of expert answers as $\widetilde{S}_{w_{x}\alpha_{k}}$,
and the manipulated priority vector for $\widetilde{S}_{w_{x}\alpha_{k}}$
aggregated using the AIP method as: 
\[
\widetilde{w}_{x}=\left[\widetilde{w}_{x}(a_{2}),\widetilde{w}_{x}(a_{i_{1}}),\ldots,\widetilde{w}_{x}(a_{i_{n}})\right]^{T},
\]
where $i_{1},i_{2},\ldots,i_{n}$ is some permutation of indices from
the set $\{1,3,4,\ldots,n\}$. Of course, according to the assumption
of the manipulation (Section \ref{subsec:Model-of-manipulation}),
even though $w_{x}(a_{1})>w_{x}(a_{2})$, in the manipulated ranking
$\widetilde{w}(a_{1})<\widetilde{w}(a_{2})$. Then, we estimated the
priority vector with the help of APDD (Aggregation of Preferential
Distance-Driven expert prioritization , Section \ref{subsec:Distance-driven-expert-prioritiz}),
AID (Aggregation of Inconsistency-Driven expert prioritization, Section
\ref{subsec:Inconsistency-driven-expert-prio}) and (MX) the mixed
method based on a linear combination of priorities of APDD and AID
(Section \ref{subsec:Inconsistency-driven-expert-prio-1}). As a result
for each $\widetilde{S}_{w_{x}\alpha_{k}}$ we obtained the following
vectors: 
\[
\widehat{w}_{x}^{\textit{APDD}}=\left[\widehat{w}_{x}^{\textit{APDD}}(a_{p_{1}}),\ldots,\widehat{w}_{x}^{\textit{APDD}}(a_{p_{n}})\right]^{T},
\]

\[
\widehat{w}_{x}^{\textit{AID}}=\left[\widehat{w}_{x}^{\textit{AID}}(a_{q_{1}}),\ldots,\widehat{w}_{x}^{\textit{AID}}(a_{q_{n}})\right]^{T},
\]

\[
\widehat{w}_{x}^{\textit{MX}}=\left[\widehat{w}_{x}^{\textit{MX}}(a_{r_{1}}),\ldots,\widehat{w}_{x}^{\textit{MX}}(a_{r_{n}})\right]^{T},
\]
where $(p_{1},\ldots,p_{n})$, $(q_{1},\ldots,q_{n})$ and $(r_{1},\ldots,r_{n})$
are some permutations of elements from $\{1,2,\ldots,n\}$. We consider
$\widehat{w}_{x}^{\textit{APDD}}$ as a WR (winner restoration case)
if the order of the first two alternatives has been restored, i.e.
$p_{1}=1$ and $p_{2}=2$, and as a RR (ranking restoration case),
if the order of the alternatives is the same as before manipulation
i.e. $p_{1}=1,p_{2}=2,\ldots,p_{n}=n$. The cases of $\widehat{w}_{x}^{\textit{APDD}}$
when $p_{1}\neq1$ or $p_{2}\neq2$ are considered as a ``failure''.
We similarly denoted results in the case of $\widehat{w}_{x}^{\textit{AID}}$
and $\widehat{w}_{x}^{\textit{MX}}$. 

\begin{figure}
\begin{centering}
\includegraphics[width=0.9\textwidth]{fig_3_apdd_success_ratio_tex}
\par\end{centering}
\caption{The ratio of number of cases where APDD managed to restore the winner
(circles) or restore the entire ranking (squares) to the number of
all considered cases.}

\label{fig:fig_3_apdd_success_ratio}
\end{figure}

In Figure \ref{fig:fig_3_apdd_success_ratio} we can see how the ratios
of WR and RR to the number of considered sets $S_{w_{x}\alpha_{k}}$
changes with the increase in the average inconsistency of $S_{w_{x}\alpha_{k}}$
in the case of the APDD method. In particular, we can observe that
for the average consistency $\textit{CI}\leq0.1$ there are $89\%$
of cases (value $0.89$ on the plot) in which APDD was able to restore
the correct winner. Similarly, there are $86\%$ of cases where APDD
reconstructed a complete ranking. 

The AID and MX methods results are presented in Figures \ref{fig:fig_4_aid_success_ratio}
and \ref{fig:fig_5_mx_success_ratio}.

\begin{figure}
\begin{centering}
\includegraphics[width=0.9\textwidth]{fig_4_aid_success_ratio_tex}
\par\end{centering}
\caption{The ratio of number of cases where AID managed to restore the winner
(circles) or restore the entire ranking (squares) to the number of
all considered cases.}

\label{fig:fig_4_aid_success_ratio}
\end{figure}

For $\textit{CI}\leq0.1$ AID restored winner in $85\%$ and the whole
ranking in $83\%$. The combined MX method reconstructed the winer
in $88\%$ and the complete ranking in $86\%$. In all cases, the
number of decision models in which the winner was restored is slightly
larger than the number of those where the entire ranking was restored.

\begin{figure}
\begin{centering}
\includegraphics[width=0.9\textwidth]{fig_5_mx_success_ratio_tex}
\par\end{centering}
\caption{The ratio of number of cases where MX managed to restore the winner
(circles) or restore the entire ranking (squares) to the number of
all considered cases.}

\label{fig:fig_5_mx_success_ratio}
\end{figure}
Despite the good efficiency in restoring the order of alternatives,
the APDD, AID and MX methods are not able to ensure that the resulting
ranking will be exactly the same as before the manipulation. In this
case, the quality of the obtained result also depends on the average
inconsistency of $S_{w_{x}\alpha_{k}}$. In Figures \ref{fig:fig_6_apdd_distance},
\ref{fig:fig_7_aid_distance} and \ref{fig:fig_8_mx_distance} we
see how the restored results differ from non-manipulated rankings
expressed in the form of the average values of the Manhattan distances\footnote{$M_{d}(u,v)=1/n\sum_{i=1}^{n}\left|u(a_{i})-v(a_{i})\right|$}
depending on $M_{d}(\bar{w}_{x},\widehat{w}_{x}^{*})$ where $*$
denotes APDD, AID or MX method. 

\begin{figure}
\begin{centering}
\includegraphics[width=0.7\textwidth]{fig_6_apdd_distances_tex}
\par\end{centering}
\caption{The average Manhattan distance $M_{d}$ between the un-manipulated
ranking and its restored by the APDD method after manipulation counterpart.}

\label{fig:fig_6_apdd_distance}
\end{figure}

\begin{figure}
\begin{centering}
\includegraphics[width=0.7\textwidth]{fig_7_aid_distances_tex}
\par\end{centering}
\caption{The average Manhattan distance $M_{d}$ between the un-manipulated
ranking and its restored by the AID method after manipulation counterpart.}

\label{fig:fig_7_aid_distance}
\end{figure}

\begin{figure}
\begin{centering}
\includegraphics[width=0.7\textwidth]{fig_8_mx_distances_tex}
\par\end{centering}
\caption{The average Manhattan distance $M_{d}$ between the un-manipulated
ranking and its restored by the MX method after manipulation counterpart.}

\label{fig:fig_8_mx_distance}
\end{figure}

It can be seen that with a relatively small average inconsistency
of experts (let say\footnote{In \citep{Saaty1977asmf} the assessment of the inconsistency level
was made dependent on the average inconsistency of the random matrix.
Therefore, to decide on the acceptability of the level of inconsistency,
it is more convenient to use \emph{CR} (consistency ratio) \citep{Saaty1977asmf}.
Since the assessment of inconsistency is not the aim of the work,
we abandoned the use of \emph{CR} in favor of the \emph{CI} (inconsistency
index).} around $0.1$), the difference between the original and the reconstructed
ranking is also reasonably small. In the case of APDD it is: $M_{d}(u,v)=0.0336$,
AID: $M_{d}(u,v)=0.047$ and MX $M_{d}(u,v)=0.0327$. Therefore, in
a large number of cases, such a result can be considered acceptable.

\subsection{Vulnerability to original ranking perturbation\label{subsec:Vulnerability-to-original}}

In the second experiment, we assumed that all experts acted honestly.
Hence, methods of aggregating expert opinions are defined to minimize
the effects of manipulation one can perceive as disturbances. Therefore
our goal this time is to check to what extent the proposed methods
can \textquotedbl disturb\textquotedbl{} the actual ranking if the
manipulation did not occur. For this purpose, we took the same dataset
as in the previous experiment, but this time we did not manipulate
individual data sets $S_{w_{x}\alpha_{k}}$ to modify the original
result. Similarly as before $\bar{w}_{x}$ denotes the aggregated
ranking (\ref{eq:ranking-with-standard-aip}) calculated using standard
AIP procedure (see Sec. \ref{subsec:Group-Decision-Making}). The
results aggregated using the modified APDD, AID and MX aggregation
procedures will be denoted as $\breve{w}_{x}$ with appropriate superscripted
acronym. However, this time both ranking vectors $\bar{w}_{x}$ and
$\breve{w}_{x}$ are calculated based on the same data set $S_{w_{x}\alpha_{k}}$.
Therefore, the distance between them can be understood as an indicator
of the disturbance of the original ranking by unnecessary use APDD,
AID and MX aggregation methods. 

As expected, (Fig. \ref{fig:fig_9_apdd_manh_dist_vs_inconsist}, \ref{fig:fig_10_aid_manh_dist_vs_inconsist}
and \ref{fig:fig_11_mx_manh_dist_vs_inconsist}) the size of the ranking
disturbances depends on the degree of inconsistency. 

\begin{figure}[h]
\begin{centering}
\includegraphics[width=0.7\textwidth]{fig_9_apdd_manh_dist_vs_inconsist_tex}
\par\end{centering}
\caption{Distances between non-manipulated rankings calculated in a standard
way $\bar{w}_{x}$ and with the help of APDD method. }

\label{fig:fig_9_apdd_manh_dist_vs_inconsist}
\end{figure}

Basically, the greater the inconsistency, the greater the differences
between the two vectors $\bar{w}_{x}$ and $\breve{w}_{x}$. Interestingly,
the best results are achieved by the mixed method (Fig. \ref{fig:fig_11_mx_manh_dist_vs_inconsist}),
which is a combination of both other strategies (Fig. \ref{fig:fig_8_mx_distance},
\ref{fig:fig_9_apdd_manh_dist_vs_inconsist}). The advantage of the
MX method can be seen not only in the drawings. Indeed, the average
distance $M_{d}\left(\bar{w}_{x},\breve{w}_{x}\right)$ between rankings
aggregated using standard and modified procedures are as follows:
\begin{equation}
\frac{1}{4000}\sum_{\bar{w}_{x}}M_{d}\left(\bar{w}_{x},\breve{w}_{x}^{\text{APDD}}\right)=0.017\label{eq:apdd-quantitative-diff-2nd-mc-ex}
\end{equation}

\begin{equation}
\frac{1}{4000}\sum_{\bar{w}_{x}}M_{d}\left(\bar{w}_{x},\breve{w}_{x}^{\text{AID}}\right)=0.011\label{eq:aid-quantitative-diff-2nd-mc-ex}
\end{equation}

\begin{equation}
\frac{1}{4000}\sum_{\bar{w}_{x}}M_{d}\left(\bar{w}_{x},\breve{w}_{x}^{\text{MX}}\right)=0.009\label{eq:mx-quantitative-diff-2nd-mc-ex}
\end{equation}
From the above, it is easy to see that, on average, the results of
the MX method (value $0.009$) are the least distant from the unmodified
aggregation method. 

\begin{figure}[h]
\begin{centering}
\includegraphics[width=0.7\textwidth]{fig_10_aid_manh_dist_vs_inconsist_tex}
\par\end{centering}
\caption{Distances between non-manipulated rankings calculated in a standard
way $\bar{w}_{x}$ and with the help of AID method.}

\label{fig:fig_10_aid_manh_dist_vs_inconsist}
\end{figure}

\begin{figure}[h]
\begin{centering}
\includegraphics[width=0.7\textwidth]{fig_11_mx_manh_dist_vs_inconsist_tex}
\par\end{centering}
\caption{Distances between non-manipulated rankings calculated in a standard
way $\bar{w}_{x}$ and with the help of MX method.}

\label{fig:fig_11_mx_manh_dist_vs_inconsist}
\end{figure}

The disruption of the ranking may be not only quantitative but also
qualitative. This means that the modified method may propose a ranking
that will differ from the \textquotedbl original\textquotedbl{} in
the order of the alternatives. As the standard�method for determining
the ordinal difference between rankings is the Kendall Tau distance
(\ref{eq:kendall-dist}), we calculated\footnote{In the experiment, the compared rankings did not have ties. Hence,
the expression (\ref{eq:kendall-dist}) could be used to calculate
the Kendall Tau distance $K_{d}$. } $K_{d}(\bar{w}_{x},\breve{w}_{x})$ for every $S_{w_{x}\alpha_{k}}$
for which $I(S_{w_{x}\alpha_{k}})\leq0.1$. The obtained values (vertical
axis) can be interpreted as the expected probability that with the
assumed not too high inconsistency of the group of experts ($I(S_{w_{x}\alpha_{k}})\leq0.1$),
both rankings will differ by a given number of transpositions (horizontal
axis). For example, in the case of APDD method we can see (Fig. \ref{fig:fig_12_apdd_kendall_prob})
that if the average inconsistency in the group of experts is not to
high $I(S_{w_{x}\alpha_{k}})\leq0.1$ then there is a $92\%$ chance
that both rankings remains unchanged. Going forward, there is a $2.5\%$
chance that they will differ in one transposition, etc. 

\begin{figure}[H]
\begin{centering}
\includegraphics[width=0.7\textwidth]{fig_12_apdd_kendall_prob_tex}
\par\end{centering}
\caption{Estimated probability that for $I(S_{w_{x}\alpha_{k}})\protect\leq0.1$
the Kendall Tau distance�$K_{d}(\bar{w}_{x},\breve{w}_{x}^{\text{APDD}})$
will be: $0,1,\ldots,12$.}

\label{fig:fig_12_apdd_kendall_prob}
\end{figure}

Similarly, for the AID grouping method, the chance that the ranking
will not change (i.e. $K_{d}(\bar{w}_{x},\breve{w}_{x}^{\text{AID}})=0$)
is $94.4\%$ (Fig. \ref{fig:fig_13_aid_kendall_prob}). The difference
of one transposition (i.e. $K_{d}(\bar{w}_{x},\breve{w}_{x}^{\text{AID}})=1$)
occurred in $1.5\%$ of cases, etc.

\begin{figure}[H]
\begin{centering}
\includegraphics[width=0.7\textwidth]{fig_13_aid_kendall_prob_tex}
\par\end{centering}
\caption{Estimated probability that for $I(S_{w_{x}\alpha_{k}})\protect\leq0.1$
the Kendall Tau distance�$K_{d}(\bar{w}_{x},\breve{w}_{x}^{\text{AID}})$
will be: $0,1,\ldots,12$.}

\label{fig:fig_13_aid_kendall_prob}
\end{figure}

Finally, in the case of a mixed method the chance that the ranking
remains unchanged (i.e. $K_{d}(\bar{w}_{x},\breve{w}_{x}^{\text{MX}})=0$)
is $95.1\%$ (Fig. \ref{fig:fig_14_mx_kendall_prob}). Analogously,
the difference of one transposition (i.e. $K_{d}(\bar{w}_{x},\breve{w}_{x}^{\text{MX}})=1$)
occurred in $1.4\%$ of cases, three transpositions are needed to
transform $\bar{w}_{x}$ into $\breve{w}_{x}^{\text{MX}}$ in $1.2\%$
of cases and so on. 

\begin{figure}[H]
\begin{centering}
\includegraphics[width=0.7\textwidth]{fig_14_mx_kendall_prob_tex}
\par\end{centering}
\caption{Estimated probability that for $I(S_{w_{x}\alpha_{k}})\protect\leq0.1$
the Kendall Tau distance�$K_{d}(\bar{w}_{x},\breve{w}_{x}^{\text{MX}})$
will be: $0,1,\ldots,12$.}

\label{fig:fig_14_mx_kendall_prob}
\end{figure}


\subsection{Experiments summary and discussion}

The conducted experiments used $4000$ sets containing $20$ matrices
each simulating the opinions of experts in the group decision-making
process. The simulated scenarios contained from $5$ to $7$ alternatives.
The first of the Montecarlo experiments consisted in simulating a
simple manipulation and checking the robustness of the experts' ranking
aggregation. Three proposed methods were tested: APDD (Aggregation
of Preferential Distance-Driven expert prioritization), AID (Aggregation
of Inconsistency-Driven expert prioritization) and MX (the mixed method)
being a combination of the previous two. For all methods, their efficiency
depended on the average inconsistency of the matrix in the tested
20-element set. The higher the inconsistency, the lower the effectiveness.
In the case of small inconsistencies (close to 0), the effectiveness
of the APDD method was close to $95\%$. i.e. in $95\%$ of cases
out of a hundred, this method was able to mitigate the negative effects
of the attack (Fig. \ref{fig:fig_3_apdd_success_ratio}). The AID
method fared slightly worse with effectiveness around $88\%$ (Fig.
\ref{fig:fig_4_aid_success_ratio}). The mixed approach seem to be
in between APDD and AID with effectiveness around $90\%$ (Fig. \ref{fig:fig_5_mx_success_ratio}).
For inconsistency around $0.1$ the effectiveness of APDD drops to
$89\%$ (WR - winner restoration) and $86\%$ (RR - ranking restoration).
The results for AID and MX are respectively: $85\%$ (WR), $83\%$
(RR), $88\%$ (WR) and $86\%$ (RR). Quantitative differences between
the original (non-manipulated) ranking and the \textquotedbl fixed\textquotedbl{}
ranking were not large (Fig. \ref{fig:fig_6_apdd_distance}, \ref{fig:fig_7_aid_distance}
and \ref{fig:fig_8_mx_distance}) and varied from $0.0327$ to $0.047$.
Interestingly, the mixed method performed as well as the first APDD
method.

The purpose of the second experiment was to test the proposed aggregation
methods on non-manipulated data. In this case, these methods seem
unnecessary and superfluous. Hence, changes in the differences between
the ranking obtained using these methods and the ranking obtained
using the classical method can be treated as an unnecessary disturbance.
We examined the generated data for the quantitative and qualitative
difference between the two ranking vectors. We achieved the best quantitative
result for the MX method. The average Manhattan distance for MX was
$0.009$ (\ref{eq:mx-quantitative-diff-2nd-mc-ex}). The AID method
fared slightly worse with the Manhattan distance $0.011$ (\ref{eq:aid-quantitative-diff-2nd-mc-ex}),
and at the end APDD with the result $0.017$ (\ref{eq:apdd-quantitative-diff-2nd-mc-ex}).
In all the cases, it can be seen that this distance depends on the
average inconsistency of experts and increases along with increasing
inconsistency.

To examine the qualitative difference between the classic aggregation
method and the modified methods, we used Kendall's tau distance measure
and the subset of data for which the average inconsistency is not
too high (less than $0.1$). With these assumptions, the MX method
turned out to be the best again, as in the $95\%$ of cases, the ranking
did not change (Fig. \ref{fig:fig_14_mx_kendall_prob}). The AID method
fared slightly worse since the 94 of cases, the order remained unchanged.
The last position took the APDD method with $92\%$ untouched rankings. 

The proposed methods are not perfect. However, almost $90\%$ of effectiveness
related to eliminating the negative effects of manipulation is paid
for, with a $5-8\%$ risk of introducing disturbances when the manipulation
occurs. The negative impact in the second case is much less severe
when the ranking result is interpreted quantitatively, i.e., when
the rank value is more important than the position on the list. Then
the chance to eliminate or mitigate the potentially significant changes
introduced by the manipulation is paid for with relatively small quantitative
changes in the non-manipulated ranking. However, is it worth using
modified aggregation methods when the ranking result is ultimately
given an ordinal meaning? It depends on the subjective assessment
of the people responsible for the decision process. In other words,
if the risk of manipulation is not negligible, then it may be worth
using the presented aggregation methods. In the article, we proposed
three heuristic ranking aggregation methods, the third of which combined
the other two. Based on the conducted experiments, the third one is
the most effective in practice. However, this observation suggests
that adding more heuristics to identify possible manipulations could
improve the results. In particular, effective methods of mitigating
the effects of manipulation should simultaneously be based on many
mutually complementary approaches.

\section{Summary\label{sec:Summary}}

The article presents three modified procedures for aggregating expert
opinions that can be used in group decision-making in the AHP method.
They allow for mitigating (or eliminating) the adverse effects of
manipulation with a small risk of distortion of the ranking. The first
two methods are based on heuristics that make the weight of a given
expert dependent on the level of its inconsistency and the group's
average opinion. The third method, perhaps the most promising, combines
the other two. Developing more secure and tamper-resistant methods
based on comparing alternatives in pairs will require further study
of attack and defense methods against manipulation. Thus, the presented
solution is a step closer to this goal.

\section*{Acknowledgements }

The research was supported by the National Science Centre, Poland,
as a part of the project SODA no. 2021/41/B/HS4/03475. 

\bibliographystyle{elsarticle-harv}
\addcontentsline{toc}{section}{\refname}\bibliography{papers_biblio_reviewed}

\end{document}
