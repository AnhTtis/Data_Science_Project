\section{Experiment}



\subsection{Datasets}
In \Cref{tab:data}, our model is evaluated on both metal and via layer designs.
The t-SNE distribution of datasets is shown in \Cref{fig:general}(a).
\subsubsection{ICCAD-2013}
ICCAD 2013 CAD contest~\cite{OPC-ICCAD2013-Banerjee} provides ten 4$\mu m^2$ tiles for testing.
We obtain the training set from GAN-OPC~\cite{OPC-TCAD2020-Yang},
which contains 4K 4$\mu m^2$ tiles generated following the same design rules of the contest.
To test the robustness of the model, we also generate OPC'ed mask using MOSAIC~\cite{OPC-DAC2014-Gao}.
We apply the lithography simulator from~\cite{OPC-ICCAD2013-Banerjee} to obtain the golden aerial and resist images.
Different from previous art DOINN~\cite{DAC22-DOINN-Yang}, we directly use the highest resolution,
\ie 4$\mu m^2$ tiles are converted to $2000 \times 2000$-pixels images.

\subsubsection{ISPD-2019}
ISPD 2019 initial detailed routing contest~\cite{ispd2019-benchmark} provides designs synthesized with commercial placement and routing tools.
We randomly choose 4$\mu m^2$ tiles from ISPD-2019 designs for a fair comparison.
Unlike DOINN only tests its model on ISPD via layers, we choose both via layers and metal layers.
\begin{itemize}
  \item ISPD-2019 via layers: we apply the high-resolution settings from DOINN, 10K training set and 10K testing set with  4$\mu m^2$ tiles converted to
  $2000 \times 2000$ images.
  \item ISPD-2019 metal layers: we randomly select 1K metal patterns for training and 300 testing patterns.
  All metal patterns will be cropped into 4$\mu m^2$ tiles and converted to $2000\times 2000$ images.
\end{itemize}
The ground truth for ISPD metal and via layers are generated by commercial tool Mentor Calibre~\cite{TOOL-calibre} with $\lambda = 193 nm, N\!A=1.35$.

\begin{table}[tb!]
	\centering
	\caption{Details of the Dataset.}
	\label{tab:data}
	\setlength{\tabcolsep}{3pt}
	\renewcommand{\arraystretch}{.9}
	\begin{tabular}{l|l|cccc}
		\toprule
		\multicolumn{1}{c|}{Dataset}     & Alias &Train  & Test  & Tile Size  & Litho Engine \\ \midrule
		ICCAD-2013  & B1      & 4875  & 10    & 4$\mu m^2$ & Lithosim \cite{OPC-ICCAD2013-Banerjee}  \\
	  ICCAD-2013 (OPC) & B1opc & -     & 10    & 4$\mu m^2$ & Lithosim \cite{OPC-ICCAD2013-Banerjee}  \\
		ISPD-2019-metal & B2m     & 1000  & 300 & 4$\mu m^2$ & Calibre \cite{TOOL-calibre}  \\
		ISPD-2019-via   & B2v     & 10000 & 10000 & 4$\mu m^2$ & Calibre \cite{TOOL-calibre}  \\ \bottomrule
	\end{tabular}
\end{table}




\subsection{Results Comparison with State-of-the-Art}
\subsubsection{Model performance}

\begin{table}[tb!]
  \centering
  \setlength{\tabcolsep}{.5pt}
  \renewcommand{\arraystretch}{.5}
  \begin{tabular}{lcccccc}
      \midrule
      (a) B1   & \includegraphics[width=.135\linewidth,valign=m]{ibm_mask}   & \includegraphics[width=.135\linewidth,valign=m]{ibm_resist_gt}  & \includegraphics[width=.135\linewidth,valign=m]{ibm_resist_unet} & \includegraphics[width=.135\linewidth,valign=m]{ibm_resist_fno}  & \includegraphics[width=.135\linewidth,valign=m]{ibm_resist_nitho} & \includegraphics[width=.135\linewidth,valign=m]{ibm_aerial_nitho} \\ \midrule
      (b) B2m  & \includegraphics[width=.135\linewidth,valign=m]{b2m_mask}   & \includegraphics[width=.135\linewidth,valign=m]{b2m_resist_gt}  & \includegraphics[width=.135\linewidth,valign=m]{b2m_resist_unet} & \includegraphics[width=.135\linewidth,valign=m]{b2m_resist_fno}  & \includegraphics[width=.135\linewidth,valign=m]{b2m_resist_nitho} & \includegraphics[width=.135\linewidth,valign=m]{b2m_aerial_nitho}\\ \midrule
      (c) B2v  & \includegraphics[width=.135\linewidth,valign=m]{b2v_mask}   & \includegraphics[width=.135\linewidth,valign=m]{b2v_resist_gt}  & \includegraphics[width=.135\linewidth,valign=m]{b2v_resist_unet} & \includegraphics[width=.135\linewidth,valign=m]{b2v_resist_fno}  & \includegraphics[width=.135\linewidth,valign=m]{b2v_resist_nitho}  & \includegraphics[width=.135\linewidth,valign=m]{b2v_aerial_nitho}\\ \midrule
      & Mask   & Resist G.T. &  TEMPO  &  DOINN & Ours &{\footnotesize Our Aerial} \\
  \end{tabular}
  \captionof{figure}{Visualization of the results of Nitho in aerial and resist stage.}
  \label{fig:vis_masks}
\end{table}
\begin{table*}[htbp]
  \centering
  \caption{Result Comparison with State-of-the-Art.}
  \setlength{\tabcolsep}{5pt}
	\renewcommand{\arraystretch}{1}
  \makebox[\textwidth][c]{
    \begin{tabular}{c|ccc|ccc|ccc|cc|cc|cc}
      \toprule
      \multirow{4}{*}{Bench} & \multicolumn{9}{c|}{Aerial Image}                                                                                            & \multicolumn{6}{c}{Resist Image}                                                   \\
      \cmidrule(lr){2-10} \cmidrule(lr){11-16}
                             & \multicolumn{3}{c|}{TEMPO~\cite{ISPD-2020-TEMPO}}                & \multicolumn{3}{c|}{DOINN~\cite{DAC22-DOINN-Yang}}               & \multicolumn{3}{c|}{Nitho}             & \multicolumn{2}{c|}{$\text{TEMPO}^{*}$~\cite{ISPD-2020-TEMPO}} & \multicolumn{2}{c|}{$\text{DOINN}^{*}$~\cite{DAC22-DOINN-Yang}} & \multicolumn{2}{c}{Nitho} \\
                             & MSE           & ME               & PSNR & MSE           & ME               & PSNR & MSE           & ME               & PSNR & mPA         & mIOU       & mPA         & mIOU        & mPA          & mIOU         \\
                             & $\times 10^{-5}$ & $\times 10^{-2}$ & dB   & $\times 10^{-5}$ & $\times 10^{-2}$ & dB   & $\times 10^{-5}$ & $\times 10^{-2}$ & dB   & (\%)        & (\%)       & (\%)        & (\%)        & (\%)         & (\%)         \\ \midrule
      B1                     & 108.29        &10.49           &32.01 &5.55         &1.94 	            &47.10 &\textbf{1.32}          &\textbf{0.51}   &\textbf{50.75} &94.60       &88.70     &99.19	      &98.32	      &\textbf{99.45}	       &\textbf{99.21}         \\
      B2m                    & 1899.04 	     &13.96 	        &30.77 &1202.39 	   &6.11            	&31.64 &\textbf{25.48}         &\textbf{0.82}   &\textbf{49.06} &98.24	      &96.55	   &98.79	      &97.10        &\textbf{99.15}        &\textbf{99.02}         \\
      B2v                    & 6.54          &3.86 	          &42.76 &2.26 	       &2.75 	            &46.37 &\textbf{2.01}          &\textbf{0.68}   &\textbf{48.06} &99.06    	  &93.28  	 &99.21	      &98.41	      &\textbf{99.59}        &\textbf{99.34}         \\
      B2m + B2v                & 4352.25       &15.21           &27.10 &3114.24 	   &12.35             &29.92 &\textbf{33.13}          &\textbf{0.78}   &\textbf{47.88} &98.63    	&95.84  	 &98.71	      &96.68	      &\textbf{99.61}        &\textbf{99.36}         \\ \midrule
      Average                & 1591.53 	     &10.88          	&33.16 &1081.11    	 &5.79            	&39.26 &\textbf{15.49}        	&\textbf{0.70} 	 &\textbf{48.94} &97.63	      &93.59	   &98.98	      &97.63	      &\textbf{99.45}	       &\textbf{99.23}         \\
      Ratio                  & 102.77 	     &15.55         	&0.68  &69.81        &8.27            	&0.80  &\textbf{1.00}          &\textbf{1.00} 	 &\textbf{1.00} &0.98  	    &0.94 	   &0.99 	      &0.98 	      &\textbf{1.00}	       &\textbf{1.00}         \\ \bottomrule
      \multicolumn{16}{l}{\footnotesize{* Models are re-trained using resist image dataset with an amendment to the final activation layer.}} \\
    \end{tabular}}
    \label{tab:results}
\end{table*}


% \begin{table*}[]
% 	\centering
% 	% \vspace{.3cm}
% 	\caption{Result Comparison with State-of-the-Art.}
% 	\label{tab:result}
% 	% \setlength{\tabcolsep}{2pt}
% 	\renewcommand{\arraystretch}{1.3}
% 	\begin{tabular}{c|cc|cc|cc}
% 		\toprule
% 		\multirow{4}{*}{Benchmark} & \multicolumn{6}{c}{Resist Image} \\ \cline{2-7}
%     & \multicolumn{2}{c|}{UNet \cite{UNet}}  & \multicolumn{2}{c|}{DOINN \cite{DAC22-DOINN-Yang}} & \multicolumn{2}{c}{Ours} \\
% 		& mPA   & mIOU     & mPA      & mIOU   & mPA   & mIOU   \\ 
% 		& (\%)   & (\%)     & (\%)      & (\%)   & (\%)   & (\%)   \\ \hline\hline
% 		B1 &   99.40    &    98.03   &   99.25       &    98.11     &   \textbf{99.43}    &  \textbf{98.27}     \\
% 		B2m             &   99.08    &    97.97   &   -           &     -        &   \textbf{99.21}    &  \textbf{98.45}      \\
% 		B2v            &   97.30    &    95.38   &   98.94       &    96.97     &   \textbf{98.98}    &  \textbf{97.79}      \\ \bottomrule
% 	\end{tabular}
% \end{table*}


% \begin{table*}[]
%   \centering
%   \caption{Comparison of the CCSHR estimators based on beta and logit-normal distributions}
% \makebox[\textwidth][c]
% {
%     \begin{tabular}{cccccccccccccc}
% \toprule
%       &       & \multicolumn{6}{c}{Aerial Image} & \multicolumn{6}{c}{Resist Image} \\
%                                   \cmidrule(lr){3-9}\cmidrule(lr){10-12}
% mu & sigma & Beta  & Logit-normal & TRUE  & Beta  & Logit-normal & & & & & & TRUE \\ \midrule
% 0 & 0.33 & 2.099 (0.134) & 2.106 (0.170) & 2.053 & 1.918 (0.129) & & & & & & 1.915 (0.132) & 1.947 \\
% 0 & 1.00 & 2.359 (0.170) & 2.348 (0.187) & 2.347 & 1.665 (0.134) & & & & & & 1.677 (0.143) & 1.653 \\
% \bottomrule
% \end{tabular}
% }
% \label{tab:addlabel}
% \end{table*}



\begin{figure}[tb!]
  \centering
  \includegraphics[width=.8\linewidth]{bar-graph}
  \caption{
    Runtime comparison with SOTA.
  }
  \label{fig:runtime}
\end{figure}

We first compare Nitho with TEMPO~\cite{ISPD-2020-TEMPO}, DOINN~\cite{DAC22-DOINN-Yang}, which are SOTA models in aerial and resist stage.
Details are in \Cref{tab:results}, where ``MSE'', ``ME'', ``PSNR'', ``mPA'' and ``mIOU'' are introduced in \Cref{sec:prelim}.
We also merge the B2m and B2v datasets to be ``B2m$+$B2v'' dataset to evaluate our models' performance on larger distribution case.

As can be seen from \Cref{tab:results}, Nitho outperforms SOTA image learning-based models in both aerial and resist stage.
In aerial stage, we achieve 69$\times$ and 102$\times$ smaller MSE than DOINN and TEMPO, with 8$\times$, 15$\times$ smaller ME.
We achieve an average of 48.94dB PSNR compared with 39.26dB of DOINN and 27.10dB of TEMPO.
The results demonstrate the superior advantage of Nitho on high-resolution aerial image generation over previous SOTA.
In resist stage, we achieve above 99\% mPA and mIOU in all datasets with 99.45\%, 99.23\% average mPA and mIOU: 1\% better mPA and 2\% better mIOU than SOTA DOINN.
The improvement can be attributed to the accurate optical kernel regression, which generates high precision aerial images with richer information.
From the results on ``B2m+B2v'' dataset, we find that Nitho's high accuracy can still be maintained, while the performance of DOINN and TEMPO degrades badly in both aerial and resist stage.
It indicates that the image learning-based models have difficulty learning on more complex distributions, while Nitho can still accurately extract optical kernel information.
Result samples are visualized in \Cref{fig:vis_masks}.

\subsubsection{Generalization capability and robustness}
\begin{table}[htbp]
  \centering
  \caption{Comparison with SOTA on out-of-distribution dataset.}
  \setlength{\tabcolsep}{3pt}
	\renewcommand{\arraystretch}{1}
  \begin{tabular}{cc|cc|cc|cc}
  \toprule
  \multicolumn{2}{c|}{Benchmark}                        & \multicolumn{2}{c|}{TEMPO~\cite{ISPD-2020-TEMPO}} & \multicolumn{2}{c|}{DOINN~\cite{DAC22-DOINN-Yang}} & \multicolumn{2}{c}{Nitho} \\
  Train & Test & mPA                     & mIOU        & mPA         & mIOU        & mPA          & mIOU         \\
  on    &  on  & \%                      & \%          & \%          & \%          & \%           & \%           \\ \midrule
  B1   & B1opc & 90.25                   & 86.15       & 98.03       & 94.76       & 99.43        & 99.17             \\
  \multicolumn{2}{c|}{Drop}  & $\downarrow$ 4.35       & $\downarrow$ 2.55  & $\downarrow$ 1.16  & $\downarrow$ 3.56 & $\downarrow$ 0.02 & $\downarrow$ 0.04  \\ \midrule
  B2m  & B2v   & 99.40                   & 71.86       & 99.64       & 78.31            &   99.58           &  97.33            \\
  \multicolumn{2}{c|}{Drop} & $\uparrow$ 0.34       & $\downarrow$ 21.42  & $\uparrow$ 0.43  & $\downarrow$ 20.10 & $\downarrow$ 0.01 & $\downarrow$ 2.01  \\ \midrule
  B2v  & B2m   & 66.06                   & 55.82       & 76.43       & 68.73            &  98.08             &  97.18      \\
  \multicolumn{2}{c|}{Drop} & $\downarrow$ 32.18       & $\downarrow$ 40.73  & $\downarrow$ 22.36  & $\downarrow$ 28.37 & $\downarrow$ 1.07 & $\downarrow$ 1.84  \\ \midrule

  \multicolumn{2}{c|}{Average}   & 85.24  & 71.28  & 91.36   &  80.60           &   \textbf{99.03}   & \textbf{97.90}               \\
  \multicolumn{2}{c|}{Avg. Drop} & $\downarrow$ 12.06  & $\downarrow$ 21.57  & $\downarrow$ 7.70  & $\downarrow$ 17.34 & $\downarrow$ \textbf{0.37} & $\downarrow$ \textbf{1.29} \\
  \bottomrule
  \end{tabular}
  \label{tab:unseen}
\end{table}
In \Cref{tab:unseen} and \Cref{fig:general}(b), we compare three models on out-of-distribution (OOD) datasets to verify the generalization performance.
Column ``train on'' means the model is trained on the referenced dataset but is tested on another dataset with a different mask shape and image distribution from the column ``test on''.
Row ``Drop'' denotes the difference of results between test results on the same distribution and OOD datasets,
with the direction of changes indicated by the up-down arrows $\uparrow, \downarrow$.
The results demonstrate that Nitho can still achieve remarkable accuracy when tested on OOD datasets,
while TEMPO and DOINN suffer strong performance degradation.

\Cref{fig:tnum} demonstrates that Nitho can use less training data to achieve better accuracy than previous art.
We list the average PSNR of B1, B2m, and B2v test sets in y-axis of \Cref{fig:tnum}, where the x-axis represents the training set percentage.
It can be concluded that when Nitho only uses 10\% of the training data, it is already more accurate than TEMPO and DOINN with 100\% of the training data.



\subsubsection{Model size and runtime comparison}
As shown in \Cref{tab:diff}, Nitho can use 31\% and 1\% parameters of DOINN and TEMPO to achieve better performance.
\Cref{fig:runtime} shows the runtime comparison of three models and traditional lithography simulators in terms of throughput ($\mu m^2/s$).
With a smaller model size and hierarchical GPU acceleration,
Nitho has 1.3$\times$ and 1.6$\times$ higher throughput than DOINN and TEMPO since no network inference is required.
Compared with traditional lithography simulators,
from which we obtain ground truth aerial and resist images,
Nitho achieves $\sim$90$\times$ speed up with less than 1\% accuracy loss.



\subsection{Ablation Study}
We also conduct experiments to verify our design on kernel dimensions and effectiveness of position encoding.
\subsubsection{Kernel dimensions with resolution limit}
In \Cref{fig:kw}, the x-axis is width/height of the kernel, \ie $(m, n)$ of $\mathcal{K} \in \mathbb{C}^{r \times n \times m}$, and $m = n$ in our settings.
The y-axis is PSNR on the corresponding dataset.
We can observe that as the kernel width/height increase,
the curve flattens out and stops growing.
This indicates that there is an optimal dimension,
and the network can not learn more information after the optimal dimension due to the \textit{resolution limit}.
The experimental results are consistent with the given optimal dimension in \Cref{eq:kernel_mn}.
On the other hand, if $\lambda$ and $N\!A$ of the lithography system are unknown, the optimal dimension can also be obtained through experiments by hyperparameter search.

\subsubsection{Positional encoding}
Starting from an estimate of the \gls{gwb} spectrum $\hat{\Omega}_{{\rm GW}, f}$, with variance $\sigma^{2}_{{\rm GW}, f}$, it is possible to place stringent constraints on the \gls{gwb} amplitude using a hybrid frequentist-Bayesian approach. %
We consider the general case where we have a set of \gls{gwb} measurements $\hat{\Omega}^{IJ}_{{\rm GW}, f}$ from different detector pairs, or {\it baselines}, $IJ$. %
We define a Gaussian likelihood for $B$ pairs of detectors, 
\begin{equation}
\label{eq:likelihood}
p\qty(\hat{\Omega}^{IJ}_{{\rm GW}, f} | \mathbf{\Theta}) \propto\exp\left[  -\frac{1}{2} \sum_{IJ}^B \sum_f \left(\frac{\hat{\Omega}^{IJ}_{{\rm GW}, f}  - \Omega_{\rm M}(f|\mathbf{\Theta})}{\sigma^{IJ}_{{\rm GW}, f}}\right)^2  \right],
\end{equation}
where $\Omega_{\rm M}(f|\mathbf{\Theta})$ is the \gls{gwb} model and $\mathbf{\Theta}$ are its parameters. %
Bayes' theorem is used to obtain posterior distributions on the model parameters, %given the likelihood defined in Eq.~(\ref{eq:likelihood}) and the priors on those parameters:
\begin{equation}\label{eq:likelihood_params}
    p\qty(\mathbf{\Theta}|\hat{\Omega}^{IJ}_{{\rm GW}, f}) \propto p\qty(\hat{\Omega}^{IJ}_{{\rm GW}, f}| \mathbf{\Theta})\,p(\mathbf{\Theta})\,,
\end{equation}
where the priors $p(\mathbf{\Theta})$ are employed. %
In practice, when performing parameter estimation on a large dataset, we take the post-processed, {\it unweighted} (i.e., $\alpha=0$) estimate $\hat{\Omega}^{0, IJ}_{{\rm ref}, f}$ to be the measured \gls{gwb} spectrum in each frequency bin, and plug it into Eq.~\eqref{eq:likelihood}. %
Note that it is necessary for the input spectra used in parameter estimation to be unweighted as any other value would constitute a model choice and bias results. %

Within the {\tt pygwb} package, we include the {\tt pe} module to perform parameter estimation as an integral part of the analysis, which naturally follows the computation of the optimal estimate of the \gls{gwb}. %
This is a notable improvement compared to previous LVK analyses, where data products and parameter estimation were handled independently by packages in different programming languages. % %$\hat{C}^{IJ}(f_k)$ and $\sigma_{IJ}(f_k)$, after which a separate parameter estimation analysis followed.
Furthermore, the {\tt pe} module is a simple and user-friendly toolkit for any model builder to constrain their physical models with \gls{gw} data. %

The {\tt pe} module is built on class inheritance, with {\tt GWBModel} as the parent class. %
The methods of the parent class are functions shared between different \gls{gwb} models, e.g., the likelihood formulation in Eq.~(\ref{eq:likelihood}), as well as the noise likelihood, given by Eq.~(\ref{eq:likelihood}) with $\Omega_{\rm M}(f|\mathbf{\Theta})\equiv0$. %
It is possible to include calibration uncertainty by modifying the {\tt calibration\_epsilon} parameter, which defaults to 0. %
For details on the marginalization over calibration uncertainty, see App.~\ref{sec:app_calibration} and \cite{Whelan:2012ur}. %
The \gls{gw} polarization used for analysis is user-defined, and defaults to standard \gls{gr} polarization (i.e., tensor). %, still giving the user the flexibility to change the GW polarisation and explore models that predict a scalar or vector polarisation of the \gls{gwb}. %
More details on possible polarization choices can be found in Sec.~\ref{sec:baseline}. %
In our implementation of {\tt pe}, we rely on the {\tt Bilby} package~\cite{Ashton:2018jfp} to perform parameter space exploration, and employ the sampler {\tt dynesty} by default \cite{dynesty}. %
The user has flexibility in choosing the sampler as well as the sampler settings. %

Child classes in the {\tt pe} module inherit attributes and methods from the {\tt GWBModel} class. %
Each child class represents a single \gls{gwb} model, and combined they form a catalog of available \gls{gwb} models that may be probed with \gls{gw} data. %
The inheritance structure of the module makes it straightforward to expand the catalog, allowing users of the {\tt pygwb} package to add their own $\Omega_{\rm M}(f|\mathbf{\Theta})$ models. %
The flexibility of the {\tt pe} module allows the user to combine several \gls{gwb} models defined within the module. %
A particularly useful application of this is the modelling of a \gls{gwb} in the presence of correlated magnetic noise, as discussed in \cite{Meyers_2020}, or the simultaneous estimation of astrophysical and cosmological \gls{gwb}s \cite{PhysRevD.103.043023}. %
The {\tt pygwb} documentation~\cite{docs} contains information on the existing models in the catalog, with a description of the GWB models and their parameters. 
% \subsubsection{Unseen data}
\Cref{tab:pe} illustrates that positional encoding is extremely critical to Nitho.
We first remove the positional encoding layer by using a simple Gaussian matrix.
As shown in the first line of \Cref{tab:pe}, PSNR in test set B1 goes down to 25.33 with worse MSE and ME than TEMPO and DOINN.
Then we apply NeRF's PE in \Cref{eq:nerf_pe} and our RFF PE in \Cref{eq:cplx_gaussian_pe},
we get $\sim$2$\times$ PSNR of 48.83 and 50.75 compared with Nitho without PE.




\begin{figure}[tb!]
  \centering
  \subfloat[]{\includegraphics[width=.47\linewidth]{tnum_wrapper} \label{fig:tnum}}
  \subfloat[]{\includegraphics[width=.47\linewidth]{kwh} \label{fig:kw}}
  \caption{
    (a) Comparison with SOTA on smaller training sets.
    (b) Ablation study on kernel size on different datasets.
  }
  \label{fig:kw_tnum}
\end{figure}


