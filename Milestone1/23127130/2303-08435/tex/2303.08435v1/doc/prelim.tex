\section{Preliminaries}
\label{sec:prelim}
This section is about basic terminology related to \fname~framework and problem formulation.
Throughout this paper, we use lowercase letters for scalars, bold lowercase for vectors, and bold uppercase for matrices,
$\odot$ for element-wise production, $^{*}$ for complex conjugate,
$\vec{I}$, $\vec{Z}$, $\vec{M}$ for aerial, resist and mask image,
$\mathcal{F}$ and $\mathcal{F}^{-1}$ for fast Fourier transform (FFT) and inverse FFT, respectively

\begin{figure}
  \centering
  \includegraphics[width=.9\linewidth]{general}
  \caption{
    (a) t-SNE distribution of datasets listed in \Cref{tab:data}.
    (b) Comparison of generalization capability on out-of-distribution datasets.}
  \label{fig:general}
\end{figure}

\subsection{Hopkins Model and Transmission Cross-Coefficient (TCC)}
In 1953, Hopkins~\cite{hopkins1953diffraction} developed a formulation that has been wildly used afterward,
whose purpose is a separation of the influence of the mask and the imaging system, including the pupil function and the illumination.
For numerical convenience, Hopkins method is often stated in terms of the spatial spectrum of the aerial intensity $\vec{I}$:
% in the image plane:
\begin{equation}
  \begin{aligned}
  \mathcal{F}(\vec{I})(f, g)=\iint_{-\infty}^{\infty} \mathcal{T}\left(\left(f^{\prime}+f, g^{\prime}+g\right),\left(f^{\prime}, g^{\prime}\right)\right) \\
  \mathcal{F}(\vec{M})\left(f^{\prime}+f, g^{\prime}+g\right) \mathcal{F}(\vec{M})^*\left(f^{\prime}, g^{\prime}\right) \mathrm{d} f^{\prime} \mathrm{d} g^{\prime},
  \end{aligned}
  \label{eq:hopkins}
\end{equation}
where $\vec{M}$ is mask, $(f, g)$ is its frequencies. $\mathcal{T}$ is TCC given by:
\begin{equation}
  \label{eq:tcc}
  % \small{
    \begin{aligned}
      \mathcal{T}\left(\left(f^{\prime}, g^{\prime}\right),\left(f^{\prime \prime}, g^{\prime \prime}\right)\right):=\iint_{-\infty}^{\infty} \mathcal{F}(J)(f, g) \\
      \mathcal{F}(H)\left(f+f^{\prime}, g+g^{\prime}\right) \mathcal{F}(H)^*\left(f+f^{\prime \prime}, g+g^{\prime \prime}\right) \mathrm{d} f \mathrm{~d} g,
    \end{aligned}
  % }
\end{equation}
where the weight factor $J$ solely depends on effective source, $H$ is projector transfer function.
Observing \Cref{eq:hopkins,eq:tcc}, $\mathcal{T}$ is independent of mask transmission functions $\mathcal{F}(\vec{M})$.
For applications where source and pupil are fixed, but images must be calculated for a wide range of different masks,
the TCC matrix can be pre-calculated and stored to accelerate imaging calculations.





\subsubsection{Sum of Coherent Sources Approach (SOCS)}
\label{subsubsec:socs}
The TCC is Hermitian and positive definite and can be decomposed into a set of eigenvalues with corresponding eigenvectors.
An approximation solution for the Hopkins imaging equations called \textit{Sum of Coherent Source} (SOCS)~\cite{cobb1995sum} using SVD algorithm which provides
relatively fast computation times and is thus widely used in some inverse imaging calculation tasks such as mask optimization.
The TCC spectrum matrix can be written as
\begin{equation}
  \mathcal{F}(\mathcal{T})=\sum_i \alpha_i \mathbf{h}_i \mathbf{h}_i^* ,
  \label{eq:socs_svd}
\end{equation}
where $\mathbf{h}_i$ are the eigenvectors and $\alpha_i$ are eigenvalues of TCC spectrum $\mathcal{F}(\mathcal{T})$.
By Fourier-back-transfoming we then obtain the SOCS formula:
\begin{equation}
  \label{eq:socs}
  \vec{I}=\sum_{i} \alpha_i\left|\mathcal{F}^{-1}\left(\mathcal{F}\left(\boldsymbol{h}_i\right) \odot \mathcal{F}(\boldsymbol{M})\right)\right|^2 .
\end{equation}



\subsection{Problem Formulation}
In aerial image generation stage, the lithography modeling can be viewed as a pixel-wise regression task,
therefore we adopt Mean of Squared Error (MSE), Peak Signal-to-Noise Ratio (PSNR), Max Error (ME) to evaluate the performance.
In resist image generation stage, it can be viewed as a pixel-wise classification problem,
we adopt the same matrices including Mean Intersection Over Union (mIOU) and Mean Pixel Accuracy (mPA) as used in SOTA~\cite{DAC22-DOINN-Yang}.

\subsubsection{MSE, PSNR.}
Given aerial image $\vec{I}$ and its prediction $\hat{\vec{I}}$, the MSE can be calculated as
\begin{equation}
  \operatorname{MSE}(\vec{I}, \hat{\vec{I}}) = \frac{1}{N}\sum_{i}^{N}(\vec{I}_i - \hat{\vec{I}}_i)^2,
\end{equation}
where $N$ is the total number of pixels in $\vec{I}$ and $\hat{\vec{I}}$. The smaller MSE, the better performance.
PSNR is a quality measurement between the original and the predicted image, in decibels.
The higher PSNR, the better quality of the predicted image. PSNR is given as
\begin{equation}
  \operatorname{PSNR}(\vec{I}, \hat{\vec{I}}) = 10 * \log_{10} ({\operatorname{max}(\vec{I})^2} / {\operatorname{MSE}(\vec{I}, \hat{\vec{I}})}).
\end{equation}


\subsubsection{mIOU, mPA}
Given k classes of predicted resist patterns $\hat{\vec{Z}}_i$ and their ground truth $\vec{Z}_i$, $i = 1, 2 \ldots k$, the mIOU and mPA are:
\begin{equation}
  \small
  \begin{aligned}
    \operatorname{mIOU}(\vec{Z}, \hat{\vec{Z}})=\frac{1}{k} \sum_{i=1}^k \frac{\vec{Z}_i \cap \hat{\vec{Z}}_i}{\vec{Z}_i \cup \hat{\vec{Z}}_i},~
    \operatorname{mPA}(\vec{Z}, \hat{\vec{Z}})=\frac{1}{k} \sum_{i=1}^k \frac{\hat{\vec{Z}}_i \cap \vec{Z}_i}{\vec{Z}_i} .
  \end{aligned}
\end{equation}

\subsubsection{ME}
Max error (ME) is  measured by the maximum difference between the predicted aerial image $\hat{\vec{I}}$ and its ground truth $\vec{I}$:
\begin{equation}
  \operatorname{ME} = \operatorname{Max}(|\vec{I} - \hat{\vec{I}}|) .
\end{equation}



\subsubsection{Problem definition}
Given a set of mask images $\mathcal{M}_{tr} = \{\vec{M}_1, \ldots, \vec{M}_n\}$
and their corresponding aerial images $\mathcal{I}_{tr} = \{\vec{I}_1, \ldots, \vec{I}_n\}$.
Resist images $\mathcal{Z}_{tr} = \{\vec{Z}_1, \ldots, \vec{Z}_n\}$ can be obtained by a constant threshold $I_{thres}$.
Our target is to design a machine learning model that can
reconstruct the TCC optical kernels, then for new designs $\mathcal{M}_{te}$, $\mathcal{I}_{te}$, $\mathcal{Z}_{te}$
using the SOCS formula in \Cref{eq:socs} to get the predicted aerial images $\hat{\mathcal{I}} = \{\hat{\vec{I}}_1, \ldots, \hat{\vec{I}}_n\}$,
and predicted resist images $\hat{\mathcal{Z}}$.
The $\operatorname{MSE}(\mathcal{I}_{te}, \hat{\mathcal{I}})$, $\operatorname{ME}(\mathcal{I}_{te}, \hat{\mathcal{I}})$ can be minimized
and $\operatorname{PSNR}(\mathcal{I}_{te}, \hat{\mathcal{I}})$,
$\operatorname{mIOU}(\mathcal{Z}_{te}, \hat{\mathcal{Z}})$, $\operatorname{mPA}(\mathcal{Z}_{te}, \hat{\mathcal{Z}})$ can be maximized.




