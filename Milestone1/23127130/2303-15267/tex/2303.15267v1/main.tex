% mnras_template.tex 
%
% LaTeX template for creating an MNRAS paper
%
% v3.0 released 14 May 2015
% (version numbers match those of mnras.cls)
%
% Copyright (C) Royal Astronomical Society 2015
% Authors:
% Keith T. Smith (Royal Astronomical Society)

% Change log
%
% v3.0 May 2015
%    Renamed to match the new package name
%    Version number matches mnras.cls
%    A few minor tweaks to wording
% v1.0 September 2013
%    Beta testing only - never publicly released
%    First version: a simple (ish) template for creating an MNRAS paper

%%%%%%%%%%%%%%%%%%%%%%%%%%%%%%%%%%%%%%%%%%%%%%%%%%
% Basic setup. Most papers should leave these options alone.
\documentclass[fleqn,usenatbib]{mnras}

% MNRAS is set in Times font. If you don't have this installed (most LaTeX
% installations will be fine) or prefer the old Computer Modern fonts, comment
% out the following line
\usepackage{newtxtext,newtxmath,array}
% Depending on your LaTeX fonts installation, you might get better results with one of these:
%\usepackage{mathptmx}
%\usepackage{txfonts}
\usepackage{multirow}


% Use vector fonts, so it zooms properly in on-screen viewing software
% Don't change these lines unless you know what you are doing
\usepackage[T1]{fontenc}
\usepackage{soul}
% Allow "Thomas van Noord" and "Simon de Laguarde" and alike to be sorted by "N" and "L" etc. in the bibliography.
% Write the name in the bibliography as "\VAN{Noord}{Van}{van} Noord, Thomas"
\DeclareRobustCommand{\VAN}[3]{#2}
\let\VANthebibliography\thebibliography
\def\thebibliography{\DeclareRobustCommand{\VAN}[3]{##3}\VANthebibliography}


%%%%% AUTHORS - PLACE YOUR OWN PACKAGES HERE %%%%%

% Only include extra packages if you really need them. Common packages are:
\usepackage{graphicx}	% Including figure files
\usepackage{amsmath}	% Advanced maths commands
\let\Bbbk\relax
\usepackage{amssymb}	% Extra maths symbols
\usepackage{tabularx}


%%%%%%%%%%%%%%%%%%%%%%%%%%%%%%%%%%%%%%%%%%%%%%%%%%

%%%%% AUTHORS - PLACE YOUR OWN COMMANDS HERE %%%%%

% Please keep new commands to a minimum, and use \newcommand not \def to avoid
% overwriting existing commands. Example:
%\newcommand{\pcm}{\,cm$^{-2}$}	% per cm-squared
\def\slopep{\Delta\mathrm{HR}/ \Delta\mathrm{age}}
\def\magGyr{\,\mathrm{mag/Gyr}}
%%%%%%%%%%%%%%%%%%%%%%%%%%%%%%%%%%%%%%%%%%%%%%%%%%

%%%%%%%%%%%%%%%%%%% TITLE PAGE %%%%%%%%%%%%%%%%%%%

% Title of the paper, and the short title which is used in the headers.
% Keep the title short and informative.
\title[age dependence of supernova luminosity]{Revisiting progenitor-age dependence of type Ia supernova luminosity standardization process}

% The list of authors, and the short list which is used in the headers.
% If you need two or more lines of authors, add an extra line using \newauthor
%\author[K. T. Smith et al.]{
%Keith T. Smith,$^{1}$\thanks{E-mail: mn@ras.org.uk (KTS)}
%A. N. Other,$^{2}$
%Third Author$^{2,3}$
%and Fourth Author$^{3}$

\author[Junchao Wang et al.]{
Junchao Wang$^1$,
Zhiqi Huang$^{1,2}$\thanks{E-mail: huangzhq25@mail.sysu.edu.cn},
Lu Huang$^1$
\\
% List of institutions
${}^{1}$School of Physics and Astronomy, Sun Yat-sen University, 2 Daxue Road, Tangjia, Zhuhai, 519082, P.R.China\\
${}^{2}$ CSST Science Center for the Guangdong-Hongkong-Macau Greater Bay Area, Sun Yat-sen University, Zhuhai, 519082, China 
%$^{1}$Royal Astronomical Society, Burlington House, Piccadilly, London W1J 0BQ, UK\\
%$^{2}$Department, Institution, Street Address, City Postal Code, Country\\
%$^{3}$Another Department, Different Institution, Street Address, City Postal Code, Country
}

% These dates will be filled out by the publisher
\date{Accepted XXX. Received YYY; in original form ZZZ}

% Enter the current year, for the copyright statements etc.
\pubyear{2023}

% Don't change these lines
\begin{document}
\label{firstpage}
\pagerange{\pageref{firstpage}--\pageref{lastpage}}
\maketitle

% Abstract of the paper
\begin{abstract}


Much of the research in supernova cosmology is based on an assumption that the peak luminosity of type Ia supernovae (SNe Ia), after a standardization process, is independent of the galactic environment. A series of recent studies suggested that there is a significant correlation between the standardized luminosity and the progenitor age of SNe Ia. The correlation found in the most recent work by Lee et al. is strong enough to explain the extra dimming of distant SNe Ia, and therefore casts doubts on the direct evidence of cosmic acceleration. The present work improves the previous work by incorporating the uncertainties of progenitor ages, which were ignored in Lee et al., into a fully Bayesian inference framework. We find a weaker dependence of supernova standardized luminosity on the progenitor age,  but the detection of correlation remains significant (3.3$\sigma$). 
Assuming that such correlation can be extended to high redshift and applying it to the Pantheon SN Ia data set, we find that the correlation cannot be the primary cause of the observed extra dimming of distant SNe Ia. Further more, we use the PAge formalism, which is a good approximation to many dark energy and modified gravity models, to do a model comparison. The concordance Lambda cold dark matter model remains a good fit when the progenitor-age dependence of SN Ia luminosity is corrected. The best-fit parameters, however, are in $\sim 2\sigma$ tension with the standard values inferred from cosmic microwave background observations.
\end{abstract}

% Select between one and six entries from the list of approved keywords.
% Don't make up new ones.
\begin{keywords}
distance scale -- cosmological parameters -- dark energy
\end{keywords}
%%%%%%%%%%%%%%%%%%%%%%%%%%%%%%%%%%%%%%%%%%%%%%%%%%
%%%%%%%%%%%%%%%%% BODY OF PAPER %%%%%%%%%%%%%%%%%%
\section{Introduction}\label{intro}

Supernova cosmology has been viewed as offering the clearest proof of an accelerating universe for many years. The empirical standardization of its peak luminosity is obtained by calibrating light-curve width and color~\citep{tripp1998two,phillips1999reddening,guy2007salt2}. The generality of this standardization procedure is based on an assumption that type Ia supernova (SN Ia) outbursts are triggered by a critical condition that has little to do with the galactic environment~\citep{jha2019observational}. But this assumption has been challenged. In the past decade, a series of studies have shown that the luminosity of SNe Ia may also depend on their host galaxy properties~\citep{hicken2009improved,sullivan2010dependence,rigault2013evidence,rigault2015confirmation,uddin2020carnegie,10.5303/JKAS.2019.52.5.181,briday2022accuracy}. In particular, the correlation between progenitor age and Hubble residual (HR) has been debated in recent studies~\citep{kang2016early,jha2019observational,rose2019think,rose2020evidence,kang2020early,lee2020further,rigault2020strong,zhang2021improving,lee2022evidence}. This correlation is manifested by a decrease in SN Ia magnitude with increasing progenitor age. Since the mean age of SN Ia progenitor evolves with redshift, the systematic bias may contribute to the observed extra dimming of distant SNe Ia, and should be properly subtracted for precision cosmology that aims to measure the property of dark energy.
\defcitealias{rose2019think}{R19}\defcitealias{campbell2013cosmology}{C13}\defcitealias{kang2020early}{K20}\defcitealias{rose2020evidence}{R20}\defcitealias{lee2020further}{L20}\defcitealias{lee2022evidence}{L22}

\citet[][hereafter \citetalias{rose2019think}]{rose2019think} selected 102 SNe Ia at redshift of $0.05 < z < 0.2$ with a median $z = 0.14$ from the supernova sample of~\citet[][hereafter~\citetalias{campbell2013cosmology}]{campbell2013cosmology} to obtain ages of the local stellar population and host galaxy, respectively. The authors found a step between the HRs of younger galaxies (age $\leq 8\,\mathrm{Gyr}$) and older galaxies (age $>8\,\mathrm{Gyr}$), but there is no correlation between the age and HR of each subgroup.
Another team,~\citet[][hereafter~\citetalias{kang2020early}]{kang2020early}, analyzed 51 nearby early-type host galaxies and found a correlation between age and HR with a greater slope of $\slopep= -0.051 \pm 0.022\magGyr$. If extrapolated to higher redshift, this correlation would lead to a redshift-dependent luminosity evolution. This conclusion was immediately refuted by~\citet{rose2020evidence}, who argued that the analysis of~\citetalias{kang2020early} was biased by several outliers (bad-quality data points). Further more,~\citet{huang2020supernova} pointed out that the cosmological scenario suggested by~\citetalias{kang2020early} leads to a universe that is younger than the observed oldest stars. \citet{huang2020supernova} proposed a Parameterization based on cosmic Age (PAge), a unified parameterization that can well approximate many dark energy and modified gravity models, to study the cosmology with supernova magnitude evolution.~\citet{huang2020supernova} found that a decelerating universe is consistent with the supernova data when the $\slopep$ parameter is allowed to vary in the range $\lvert \slopep\rvert \lesssim 0.057\magGyr$, as suggested by~\citetalias{kang2020early}. More recently,~\citet{lee2020further} recalculated the age-HR correlation of SN Ia data set from~\citetalias{rose2019think}  based on Gaussian mixture models, and the outcome was consistent with the result of~\citetalias{kang2020early} from early-type host galaxies.

\citet[][hereafter~\citetalias{lee2022evidence}]{lee2022evidence} divided the data of~\citetalias{rose2019think} into young and old groups, with a gray area in between. The width of the gray area is taken to be greater than the typical age measurement error to avoid sample contamination between two subgroups. \citetalias{lee2022evidence} utilized the light-curve parameters provided in~\citetalias{campbell2013cosmology} to determine $\Delta$HR. The dependence of the luminosity on the progenitor age is estimated by taking the ratio of $\Delta$HR to $\Delta\mathrm{age}$, the difference in the median local (environmental stellar population typically within a few $\mathrm{kpc}$) ages of these two groups. Combining the measured slope, which is $\slopep = -0.040\magGyr$, with mass-weighted mean age evolution of stellar population obtained from cosmic star formation history, the authors derived the redshift evolution of $\Delta$HR and claimed that there is almost no evidence of an accelerating universe from supernova data alone. 

The analysis in~\citetalias{lee2022evidence} is straightforward and intuitively easy to understand, but it may still be challenged by the same argument that the data set is contaminated by some bad-quality samples. The present work aims to advance the analysis in \citetalias{lee2022evidence}  by using a fully Bayesian approach without artificial selection or grouping of data. Our catalog includes all the 102 supernovae in the redshift range $0.05<z<0.2$ with progenitor age measurements~\citepalias{rose2019think} and the light-curve data from~\citetalias{campbell2013cosmology}. The Bayesian approach makes use of the age uncertainty information of each supernova, therefore naturally suppresses the statistical contribution from bad-quality samples. 

This paper is organized as follows. In Section~\ref{corre} we introduce our statistical method and apply it to the supernova catalog to update the constraint on $\slopep$. In Section~\ref{COSMO}, we use the measured $\slopep$ to correct the cosmological likelihood of type Ia supernovae~\citep{scolnic2018complete} and discuss its cosmological implication. Section~\ref{discu} concludes.

\section{Correlation between Progenitor Age and Hubble Residual}\label{corre}

For the luminosity standardization of SNe Ia, the distance modulus is given by SALT2~\citep{guy2007salt2,guy2010supernova},
\begin{equation}\label{1}
\mu_{\rm SN}=m-M+\alpha x_1-\beta c,
\end{equation}
where $m$ is the apparent magnitude, i.e., $-2.5\log_{10}(x_0)$ in~\citetalias{campbell2013cosmology}, $M$ is the absolute magnitude, and $\alpha x_1$ and $\beta c$ are correction terms depending on the light-curve width ($x_1$) and color ($c$). The relative luminosity is then represented by the Hubble residual
\begin{equation}\label{2}
\rm{HR}=\mu_{\rm SN}-\mu_{\rm model},
\end{equation}
where $\mu_{\rm model}$ is the theoretical distance modulus, given by
\begin{equation}
    \mu_{\rm model} = 25 + 5\log_{10}\frac{d_L}{\mathrm{Mpc}}.
\end{equation} 
The luminosity distance $d_L$ as a function of cosmological redshift $z$ is specified by the cosmological model, for instance in flat Lambda cold dark matter ($\Lambda$CDM) model
\begin{equation}
    \left.d_L(z)\right\vert_{\rm \Lambda CDM} = \frac{c}{H_0}\int_0^z \frac{\,dz'}{\sqrt{\Omega_m(1+z')^3 + 1 - \Omega_m}},
\end{equation} 
where $H_0$ is the Hubble constant and $\Omega_m$ is the matter abundance parameter at the present epoch.

We consider a linear regression between HR and progenitor age
\begin{equation}
    \mathrm{HR} = \frac{\Delta \mathrm{HR}}{\Delta \mathrm{age}} \times \mathrm{age} + \mathrm{constant}.
\end{equation}
The constant term on the right hand side can be absorbed into the definition of $M$ parameter and will be ignored hereafter. 

To properly account for the measurement error and intrinsic scatter in the regression analysis, we write the likelihood function as
\begin{equation}\label{3}
\mathcal{L} \propto \exp\left\{-\frac{1}{2}\sum_{i}\left[\frac{\left(\rm{HR}_{\emph{i}}- \frac{\Delta \mathrm{HR}}{\Delta \mathrm{age}} \times \rm{age}_{\emph{i}}\right)^2}{\sigma_i^2}+\ln{\left(2\pi\sigma_i^2\right)}\right]\right\},
\end{equation}
where ${\rm HR}_{i}=-2.5\log_{10}{x_{0i}}+\alpha x_{1i}-\beta c_i-M-\mu_{\rm model}\left(z_i\right)$ and the index $i$ runs over the 102 low-redshift SN Ia samples used here. For each SN Ia, the total uncertainty is given by 
\begin{equation}
    \sigma^2=\left(\frac{2.5\sigma_{x_0}}{x_0\ln{1}0}\right)^2+\left(\alpha\sigma_{x_1}\right)^2+\left(\beta\sigma_c\right)^2+\left(\frac{\Delta \mathrm{HR}}{\Delta \mathrm{age}}\sigma_{\rm age}\right)^2+\sigma_{\mathrm{intrisic}}^2,
\end{equation}
where $\sigma_{\rm intrinsic}$  is the intrinsic scatter, representing the unaccounted for additional variability. The uncertainty of the local peculiar motion is not considered because it can be neglected on scales larger than 100 Mpc ($z\sim 0.02$). For flat $\Lambda$CDM model, the likelihood depends on the parameters $\{\alpha,\beta,\slopep,\Omega_{m},\sigma_{\rm intrinsic},M_{\rm eff} \}$, where $M_{\rm eff} = M - 5\log_{10}\frac{H_0}{70\mathrm{km/s/Mpc}}$. 

We employ the local age of the stellar population near each SN Ia from~\citetalias{rose2019think} and corresponding light curve data ($x_0, x_1$ and $c$) from~\citetalias{campbell2013cosmology}. Then we run Monte Carlo Markov Chain (MCMC) simulations with flat priors $\alpha\in[0.001,5]$, $\beta\in[0.5,5]$, $\slopep\in[-1,1]$, $\Omega_m\in[0,1]$, $\sigma_{\rm intrinsic} \in [0,3] $, $M_{\rm eff}\in[-40,-10]$. The 68.3\% confidence-level bounds of parameters are listed in Table~\ref{iii}. Notably, we find a $3.3\sigma$ hint of nonzero $\slopep$, which indicates a correlation between the progenitor age and the Hubble residual that is directly visualized in Fig.~\ref{slope}. We further test the robustness of the constraint on $\slopep$ by randomly removing 30 supernova samples in our catalog. With the randomly selected subgroup of data, we still find $\slopep<0$ at $\gtrsim 2\sigma$ confidence level. 



\begin{table}
\centering
\caption{The constraints on parameters. Data: 102 low-$z$ supernova samples with progenitor age information.}
\label{iii}
\begin{tabular}{cc}
\hline
Parameter& Mean value with 1$\sigma$ error\\
\hline
$\alpha$&0.16 $\pm$ 0.02\\
$\beta$&2.88 $\pm$ 0.18\\
$\slopep$&$-0.020 \pm$ 0.006 \,$\mathrm{mag/Gyr}$\\
$M_{\rm eff}$&$-29.59\pm0.06$\\
$\sigma_{\rm intrisic}$ &0.08$\pm$0.02\\
$\Omega_m$ & $0.46^{+0.19}_{-0.23}$\\
\hline
\end{tabular}
\end{table}

\begin{figure}
\centering
   \includegraphics[width=0.46\textwidth]{slope.pdf}  
  \caption{\label{slope}{The age–luminosity relation. The data points are 102 SNe Ia in \citetalias{rose2019think}. The solid orange line is the regression fitting with the mean value of $\slopep$ given in Table~\ref{iii}. The shaded area represents the 1$\sigma$ level region.}}
\end{figure}

As shown in Fig.~\ref{slope-Om}, there is no noticeable degeneracy between $\slopep$ and $\Omega_m$. The preference of a nonzero $\slopep$ is then, in the $\Lambda$CDM framework, cosmology-independent. This is somewhat expected because the Hubble diagram of low-redshift ($z\le 0.2$ here) SNe Ia is insensitive to cosmology. For the same reason, the analysis here has almost no constraining power on cosmological parameters.  In the next section we study cosmology with the Pantheon supernova catalog that covers a much broader redshift range $0<z<1.5$~\citep{scolnic2018complete}, based on the assumption that the HR-age relation also applies to high-redshift SNe Ia. 

\begin{figure}
\centering
   \includegraphics[width=0.46\textwidth]{slope-Om.pdf}	  
  \caption{\label{slope-Om}{The marginalized distributions of $\slopep$ and $\Omega_m$. Inner and outer contours correspond to 68.3\% and 95.4\% confidence levels, respectively.}}
\end{figure}



\section{Cosmology Analysis}\label{COSMO}


%because as the redshift increases, the age of the universe and the possible maximum age of the stellar population become younger. Consequently, the average age of the stellar population at a given redshift also becomes younger with redshift, which is well supported by observations of different redshift galaxies \citep{gupta2011improved,schiavon2006deep2,choi2014assembly,fumagalli2016ages}.

The age information of individual samples in Pantheon supernova catalog is not available. We instead use  the mean age of stellar population as a function of redshift, derived from the theory of cosmic star formation history and tabulated in Table 1 of~\citetalias{lee2022evidence}. We removed six supernovae with redshifts $z>1.5$ that are out of the tabulated redshift range. We revise the standard supernova likelihood by adding a $\slopep \times \mathrm{age}$ term to the distance modulus, where $\mathrm{age}$ is the mean age of stellar population, and apply a flat prior on $\slopep\in [-0.1, 0.1]$.

\begin{figure}
  \centering
   \includegraphics[width=0.46\textwidth]{LCDM.pdf}
  \caption{\label{lcdm} The marginalized $68.3\%$ (inner blue contour) and $95.4\%$ (outer blue contour) confidence level constraints with Pantheon SN Ia data set and a flat prior on $\slopep$. The horizontal black band stands for Planck constraint on matter abundance ($\Omega_m=0.315 \pm 0.007$). The vertical bands are three cases of supernova magnitude evolution: (i) the standard assumption of no HR-age dependence ($\slopep = 0$, green); (ii) the \citetalias{lee2022evidence} result $\slopep=-0.04\magGyr$ (magenta); (iii) $\slopep = -0.020\pm 0.006\magGyr$ (yellow) that was found in this work.  }
\end{figure} 

%\begin{table}
%    \centering
%\caption{1$\sigma$ uncertainty results for the $\Lambda$CDM model in a flat universe. 'This work' means using a Gaussian prior of $\slopep = -0.020 \pm 0.006\magGyr$, and 'L22' is for the fix value $\slopep = -0.04\magGyr$.}
%\label{table:LCDM}
%    \begin{tabular}{ccc}
%    \hline
%        ~ & $\Omega_m$ & $\chi^2_{\rm min}/\rm{d.o.f}$ \\ \hline
%        This work & $0.392\pm 0.037$ & 0.989 \\ 
%        L22 & $0.502^{+0.024}_{-0.0028}$ & 0.992 \\ \hline
%    \end{tabular}
%\end{table}


Fig.~\ref{lcdm} shows the result of MCMC calculations. The marginalized constraints on $\Omega_m$ and $\slopep$ are compared with three cases of supernova magnitude evolution: (i) the standard assumption of no HR-age dependence ($\slopep = 0$); (ii) $\slopep=-0.04\magGyr$ that was found in~\citetalias{lee2022evidence}; (iii) $\slopep = -0.020\pm 0.006\magGyr$ (Gaussian error) that was found in this work with a Bayesian method.  We label the three priors on $\slopep$ as ``standard'', ``\citetalias{lee2022evidence}'' and ``this work'', respectively. 

Compared to the standard ($\slopep=0$) case, the HR-age dependence found in this work ($\slopep = -0.020\pm 0.006\magGyr$) drives the cosmology towards a larger $\Omega_m$ that is in mild ($\sim 2\sigma$) tension with the concordance model ($\Omega_m = 0.315\pm 0.007$) based on the Planck satellite cosmic microwave background data~\citep{Planck18}. If we accept the assumption that the HR-age dependence can be extended to high redshift, the $\sim 2\sigma$ tension with the Planck standard cosmology could be a hint of new physics beyond the flat $\Lambda$CDM model. To test whether a non-standard cosmology can fit the data much better, we use the PAge approximation, which provides a simple and almost model-independent approximation to $\Lambda$CDM and many beyond-$\Lambda$CDM models~\citep{huang2020supernova, luo2020reaffirming, MAPAge, GRB2021PAge, Huang2022PAge, cai2022no2, cai2022no1, li2022redshift}. Model comparison can be easily done in the PAge parameter space with one single MCMC run.


The PAge approximation is formulated as 
\begin{equation}\label{4}
\frac{H}{H_0}=1+\frac{2}{3}\left (1-\eta\frac{H_0t}{p_{\mathrm{age}}}\right )\left (\frac{1}{H_0t}-\frac{1}{p_{\mathrm{age}}}\right ),
\end{equation}
where $t$ is the cosmological time. $p_{\rm age}=H_0t_0$ is the product of the Hubble constant $H_0$ and the current age of the universe $t_0$. The phenomenological parameter $\eta$ can be regarded as a fitting parameter that represents the deviation from an Einstein-de Sitter Universe. The Planck concordance cosmology ($\Omega_m\approx 0.315$) is well approximated by the PAge model with $p_{\rm age} = 0.951$ and $\eta= 0.359$.

Because PAge is a phenomenological parameterization for late universe, we are not supposed to apply PAge to cosmic microwave background data, whose anisotropy involves the evolution of cosmological perturbations from the early universe. For comparison with external data, we instead use the baryon acoustic oscillations (BAOs). The BAO data set includes the latest measurements from the Sloan Digital Sky Survey (SDSS)-IV\footnote{https://svn.sdss.org/public/data/eboss/DR16cosmo/tags/v1\_0\_0/likelihoods
/BAO-only/} \citep{alam2021completed}, the Six-degree Field Galaxy Survey (6dFGS) \citep{beutler20116df}, and the Dark Energy Survey Year 3 (DES Y3) final release \citep{abbott2022dark}. 

%It is an approximation to apply the mean ages in Table 1 of ~\citetalias{lee2022evidence}, which are computed with a reference $\Lambda$CDM model, to models that are beyond, but in the proximity of $\Lambda$CDM. 
The mean ages in  Table 1 of ~\citetalias{lee2022evidence} are computed with a reference  $\Lambda$CDM model. Applying these ages to models that are beyond, but in the  proximity of $\Lambda$CDM is an approximate method.
To avoid running into parameter space that is too far away from the concordance model and hence the approximation may fail, for all cases we apply a lower limit of $12\,\mathrm{Gyr}$ for the age of the universe~\citep{vandenberg2014three,catelan2017ages,sahlholdt2019benchmark} and a Gaussian prior for the Hubble constant $H_0 = 70 \pm 2\,\mathrm{km/s/Mpc}$ to restrict the parameter space.

%\begin{table*}
%\renewcommand\arraystretch{1.5} 
%\centering
%\caption{1$\sigma$ uncertainty results for the PAge approximation in a flat universe.}
%\label{i}
%\begin{tabular}{cccccc}
%\hline
%\multicolumn{2}{c}~& $p_{\rm age}$ & $\eta$ & $q_0$ & $\chi^2_{\rm min}/\rm{d.o.f}$ \\ \hline
%\multirow{3}{*}{Pantheon} & Standard & $1.00^{+0.04}_{-0.09}$ & $0.29^{+0.30}_{-0.15}$ & $-0.54\pm 0.10$ & 0.991 \\
% & This work & $0.99^{+0.04}_{-0.11}$ & $-0.10^{+0.45}_{-0.17}$ & $-0.27^{+0.14}_{-0.15}$ & 0.990 \\
% & L22 & $0.93^{+0.03}_{-0.08}$ & $-0.33^{+0.39}_{-0.14}$ & $0.01^{+0.10}_{-0.13}$ & 0.989\\ \hline
%\multicolumn{2}{c}{BAO} & $0.94\pm0.02$ & $0.12^{+0.16}_{-0.14}$ & $-0.32^{+0.12}_{-0.16}$& 1.288\\ \hline
%	\end{tabular}
%	\end{table*}


\begin{figure}
  \centering
   \includegraphics[width=0.46\textwidth]{Page.pdf}
  \caption{\label{page}{The $1\sigma$ and $2\sigma$ confidence level contours of the flat PAge approximation for the supernova data with three priors on $\slopep$ and for the BAO data.  The black dot $(\eta, p_{\rm age}) = (0.359, 0.951)$ is a good approximation to the Planck best-fit $\Lambda$CDM model ($\Omega_{m} \approx 0.315$). The dashed blue $q_0=0$ line is the critical line between cosmic acceleration (upper right region) and deceleration (lower left region).}}
\end{figure} 

We then run three MCMC calculations by replacing the flat prior on $\slopep$ with the three cases of supernova magnitude evolution, the ``standard'' case with fixed $\slopep=0$, the ``\citetalias{lee2022evidence}'' case with fixed $\slopep=-0.04$, and ``this work'' case $\slopep=-0.020\pm 0.006$. The marginalized constraints on PAge parameters $p_{\rm age}$ and $\eta$ are shown in Fig~\ref{page}. Compared to the ``\citetalias{lee2022evidence}'' case where a decelerating universe is more favored, the updated constraint on $\slopep$ in this work reconciles cosmic acceleration at $\sim 2\sigma$ level. The Planck best-fit $\Lambda$CDM model, well approximated by the $(\eta, p_{\rm age}) = (0.359, 0.951)$ point, is consistent with the result in this work. 

For ``this work'' case, we run an additional MCMC calculation for $\Lambda$CDM model and compare the result with that of PAge. The best-fit $\chi^2$ (defined as $-2\ln \mathcal{L}$) from $\Lambda$CDM to PAge is only improved by $1029.6-1027.3 = 2.3$. Given that PAge contains one more degree of freedom than $\Lambda$CDM, the results do not indicate any strong evidence for a beyond-$\Lambda$CDM  model.

%From Table~\ref{i} and Fig.~\ref{page}, we can see that our correction results do not contradict BAOs, and favour the cosmic acceleration at the confidence levels of 1.9$\sigma$, while \citetalias{lee2022evidence} support the deceleration factor $q_0 = 0$ at $\ll 1\sigma$ level. The mapping of $\Lambda$CDM on PAge, represented by the black dots shown in Fig.~\ref{page}, are $\sim 1.4\sigma$ of our results, which is lower than one in the $\Lambda$CDM case, because the posterior distribution under the page approximation is non-Gaussian as can be seen from Fig.~\ref{page}. Compare the $\chi_{\rm min}^2$/d.o.f in Table~\ref{table:LCDM} and  Table~\ref{i}, it can be found that the PAge approximat is not much better than $\Lambda$CDM model. So, even if age dependence is admitted, there is no evidence of beyond $\Lambda$CDM yet. 



\section{Discussion and Conclusion}\label{discu}

The usage of standardized peak luminosity of type Ia supernova as a standard candle has been recently challenged by a series of works that claim detection of correlations between the standardized luminosity and properties of the host galaxies. In particular,  for low-redshift supernovae whose environmental stellar population can be resolved,~\citetalias{lee2022evidence} found a dependence of the Hubble residual on the age of its environmental stellar population. The analysis in~\citetalias{lee2022evidence} does not make use of the age uncertainties, therefore may be contaminated by bad-quality samples. The present work advances the analysis in~\citetalias{lee2022evidence} by including the age uncertainties in a fully Bayesian inference framework, and finds a significantly weaker dependence of the Hubble residual on the progenitor age,  $\slopep=-0.020\pm 0.006$. 

With a bold assumption that the progenitor age dependence can be extended to high redshift supernovae, we studied the impact of the supernova luminosity evolution on cosmology. We find $\Lambda$CDM model remains a good fit for the supernova data, but a high $\Omega_m$ value is preferred ($\sim 2\sigma$ larger than the Planck results). In contrast, the~\citetalias{lee2022evidence} result ($\slopep=-0.04\magGyr$) almost excludes (in $\sim 4\sigma$ tension with) the Planck concordance model.

Another aspect is that the Pantheon data set and the low-redshift catalog with age information are two different catalogs and may follow different statistics. While this might be true, we argue that the luminosity evolution problem remains a possible systematic error and is worthy of in-depth study for future high precision supernova cosmology.

%We also attempt to use the global (host galaxy) ages from \citetalias{rose2019think} and get $\Delta$HR/$\Delta$age = $-0.016 \pm 0.006\magGyr$ that is consistent with the result from local (stellar population) ages. 

%Thus, we posit that the adoption of either global age or local age is unlikely to elicit a substantial deviation in outcomes.

%Previous investigations, such as \citetalias{rose2019think}, \citetalias{lee2020further}, and \citetalias{lee2022evidence}, have also utilized data from the same sources. Our analysis is as follows: 
%\begin{itemize}
%\item From Fig.~\ref{slope}, we can see that the error of progenitor ages is very considerable. However, \citetalias{lee2022evidence} wiped out the progenitor age error of each supernova in the subgroup, resulting in an overcorrection of the age-luminosity ($\slopep = -0.04\magGyr$). Therefore, improving the calculation method of \citetalias{lee2022evidence} was the motivation of our study. After accounting for the progenitor age errors, $\lvert \Delta$HR/$\Delta$age$\rvert $ is reduced by $50\%$.
%\item The $\Delta$HR/$\Delta$age of \citetalias{rose2019think} is shallower than ours (see their Fig.~6). We also do not agree with \citetalias{rose2019think} that there is almost no correlation between progenitor age and HR. Our results give the correlation at a 3.3$\sigma$ level. \citetalias{lee2020further} have pointed out that \citetalias{rose2019think} produced a regression dilution bias, seriously underestimates the slope and significance of the age-HR correlation.
%\item \citetalias{lee2020further} presented a relatively steep value of $\slopep = -0.057\pm0.016\magGyr$. We find that the HR values used by \citetalias{lee2020further} were based on $\alpha$ = 0.22, which was obtained from the full photometric sample by \citetalias{campbell2013cosmology}. However, \citetalias{campbell2013cosmology} indicated that this $\alpha$ was larger than the typical value. Therefore, this large $\alpha$ amplified the results of \citetalias{lee2020further}. By changing the $\alpha$ value to 0.16 which is suggested by \citetalias{campbell2013cosmology}, and replicating the work of \citetalias{lee2020further} using the LINMIX package, we obtain $\slopep = -0.032\pm 0.017\magGyr$, which differs from our results by only a 0.7$\sigma$ level.
%\end{itemize}
%\st{As redshift increases, the maximum possible age of the stellar population decreases, resulting in a corresponding decrease in the average age of the stellar population. The age of the population is correlated with HR, leading to redshift evolution in the peak luminosity of supernovae. This deviates from conventional perspectives and may no longer provide evidence for an accelerating universe. } 
%Then we assum that the HR-age relationalso applies to high redshift SNe and study cosmology with the Pantheon supernova catalog~\citep{scolnic2018complete} under flat $\Lambda$CDM model. For the Gaussian prior $\slopep = -0.020\pm 0.006\magGyr$ , we obtain $\Omega_m=0.392\pm 0.037$, which is in $\sim 2\sigma$ tension with the measurement from cosmic microwave background (CMB)~\citep{Planck18}. 
%Fig.~\ref{lcdm} indicates that after considering age dependence, our result is a little contradiction with planck standard cosmology, but the contradiction is not as exaggerated as \citetalias{lee2022evidence}. 

%And for non-flat case, we acquire $\Omega_{m} = 0.344 \pm 0.093$ and $\Omega_{k} = 0.14^{+0.22}_{-0.24}$, which shows that even considering the evolution of supernova brightness, it is not necessary, as in \citetalias{kang2020early} and \citetalias{lee2022evidence}, to use a large spatial curvature ($\Omega_{ k}\sim 0.7$) to fit the Hubble diagram. 

%Subsequently, we employ the PAge approximation to see if age dependence imply other beyond $\Lambda$CDM models. The PAge approximation involves the same number of parameters as $w$CDM but covers a more comprehensive range of scenarios beyond the conventional dark energy concept. We use the supernova data with and without HR-age dependence and BAO data to constrain the approximation of PAge, respectively. As we can see from Fig.~\ref{page}, we yield a support for cosmic acceleration at the 1.9$\sigma$ confidence levels. The Planck bestfit $\Lambda$CDM model is a good fit with $\sim 1.4\sigma$ confidence level, which is lower than one in the $\Lambda$CDM case, because the posterior distribution under the page approximation is non-Gaussian as can be seen from Fig.~\ref{page}. Compare with $\Lambda$CDM, the $\chi_{\rm min}^2$/d.o.f of PAge approximation is not improved. So even if age dependence is admitted, there is no evidence of beyond $\Lambda$CDM yet. 
%In addition, our corrected results are also consistent with BAOs. 
%While the consequences for \citetalias{lee2022evidence} fully support $q_0$ = 0 at $\ll 1\sigma$ level and exclude the Planck bestfit $\Lambda$CDM model.


%Therefore, we argue that the non-accelerated universe claimed by \citetalias{lee2022evidence} does not match other observations (e.g.  the Planck results for base $\Lambda$CDM) due to their neglect of the progenitor age errors and exaggeration of the age-HR steepness. We show that even considering the age-luminosity evolution, the results are still consistent with BAOs and the Planck results for base $\Lambda$CDM. The luminosity evolution of supernovae due to the correlation between progenitor age and HR stands up to scrutiny. This systematic bias should be considered before delving into the details of the dark energy.

\section*{Acknowledgement}

This work is supported by National SKA Program of China No. 2020SKA0110402, the National Natural Science Foundation of China (NSFC) under Grant No. 12073088, National key R\&D  Program of China (Grant No. 2020YFC2201600), and Guangdong Major Project of Basic and Applied Basic Research (Grant No. 2019B030302001).


%%%%%%%%%%%%%%%%%%%%%%%%%%%%%%%%%%%%%%%%%%%%%%%%%%
\section*{Data Availability}
The BAO data underlying this article are publicly available in~\citet{beutler20116df,alam2021completed,abbott2022dark}.
The supernova data underlying this article are publicly available in~\citet{campbell2013cosmology,scolnic2018complete,rose2019think}. 

%%%%%%%%%%%%%%%%%%%% REFERENCES %%%%%%%%%%%%%%%%%%

% The best way to enter references is to use BibTeX:

\bibliographystyle{mnras}
\bibliography{example} % if your bibtex file is called example.bib


% Alternatively you could enter them by hand, like this:
% This method is tedious and prone to error if you have lots of references
%\begin{thebibliography}{99}
%\bibitem[\protect\citeauthoryear{Author}{2012}]{Author2012}
%Author A.~N., 2013, Journal of Improbable Astronomy, 1, 1
%\bibitem[\protect\citeauthoryear{Others}{2013}]{Others2013}
%Others S., 2012, Journal of Interesting Stuff, 17, 198
%\end{thebibliography}

%%%%%%%%%%%%%%%%%%%%%%%%%%%%%%%%%%%%%%%%%%%%%%%%%%

%%%%%%%%%%%%%%%%% APPENDICES %%%%%%%%%%%%%%%%%%%%%

%\appendix


%%%%%%%%%%%%%%%%%%%%%%%%%%%%%%%%%%%%%%%%%%%%%%%%%%


% Don't change these lines
\bsp	% typesetting comment
\label{lastpage}
\end{document}

% End of mnras_template.tex
