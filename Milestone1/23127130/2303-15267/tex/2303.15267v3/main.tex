% ****** Start of file apssamp.tex ******
%
%   This file is part of the APS files in the REVTeX 4.2 distribution.
%   Version 4.2a of REVTeX, December 2014
%
%   Copyright (c) 2014 The American Physical Society.
%
%   See the REVTeX 4 README file for restrictions and more information.
%
% TeX'ing this file requires that you have AMS-LaTeX 2.0 installed
% as well as the rest of the prerequisites for REVTeX 4.2
%
% See the REVTeX 4 README file
% It also requires running BibTeX. The commands are as follows:
%
%  1)  latex apssamp.tex
%  2)  bibtex apssamp
%  3)  latex apssamp.tex
%  4)  latex apssamp.tex
%
\documentclass[%
 reprint,
%superscriptaddress,
%groupedaddress,
%unsortedaddress,
%runinaddress,
%frontmatterverbose, 
%preprint,
%preprintnumbers,
%nofootinbib,
%nobibnotes,
%bibnotes,
 amsmath,amssymb,
 aps,
%pra,
%prb,
%rmp,
%prstab,
%prstper,
%floatfix,
]{revtex4-2}

\usepackage{graphicx}% Include figure files
\usepackage{dcolumn}% Align table columns on decimal point
\usepackage{bm}% bold math
\usepackage{multirow} 
%\usepackage{hyperref}% add hypertext capabilities
%\usepackage[mathlines]{lineno}% Enable numbering of text and display math
%\linenumbers\relax % Commence numbering lines

%\usepackage[showframe,%Uncomment any one of the following lines to test 
%%scale=0.7, marginratio={1:1, 2:3}, ignoreall,% default settings
%%text={7in,10in},centering,
%%margin=1.5in,
%%total={6.5in,8.75in}, top=1.2in, left=0.9in, includefoot,
%%height=10in,a5paper,hmargin={3cm,0.8in},
%]{geometry}

\begin{document}
%\preprint{}
\def\slopep{\Delta\mathrm{HR}/ \Delta\mathrm{age}}
\def\magGyr{\,\mathrm{mag/Gyr}}
\title{\boldmath Revisiting progenitor-age dependence of type Ia supernova luminosity standardization process}% Force line breaks with \\
%\thanks{A footnote to the article title}%

\author{Junchao Wang}
 \affiliation{School of Physics and Astronomy, Sun Yat-sen University,\\
Zhuhai, 519082, P.R.China}
 \affiliation{CSST Science Center for the Guangdong-Hongkong-Macau Greater Bay Area, Sun Yat-sen University,\\
Zhuhai, 519082, P.R.China}
\author{Zhiqi Huang}
 \email{huangzhq25@mail.sysu.edu.cn}
 \affiliation{School of Physics and Astronomy, Sun Yat-sen University,\\
Zhuhai, 519082, P.R.China}
 \affiliation{CSST Science Center for the Guangdong-Hongkong-Macau Greater Bay Area, Sun Yat-sen University,\\
Zhuhai, 519082, P.R.China}
\author{Lu Huang}
 \affiliation{School of Physics and Astronomy, Sun Yat-sen University,\\
Zhuhai, 519082, P.R.China}
 \affiliation{CSST Science Center for the Guangdong-Hongkong-Macau Greater Bay Area, Sun Yat-sen University,\\
Zhuhai, 519082, P.R.China}
%\date{\today}% It is always \today, today,
             %  but any date may be explicitly specified

\begin{abstract}
Much of the research in supernova cosmology is based on an assumption that the peak luminosity of type Ia supernovae (SNe Ia), after a standardization process, is independent of the galactic environment. A series of recent studies suggested that there is a significant correlation between the standardized luminosity and the progenitor age of SNe Ia. The correlation found in the most recent work by Lee et al. is strong enough to explain the extra dimming of distant SNe Ia and therefore casts doubts on the direct evidence of cosmic acceleration. The present work incorporates the uncertainties of progenitor ages, which were ignored in Lee et al., into a fully Bayesian inference framework. We find a weaker dependence of supernova standardized luminosity on the progenitor age,  but the detection of correlation remains significant (3.5$\sigma$). Assuming that such correlation can be extended to high redshift and applying it to the Pantheon SN Ia data set, we confirm that when the Hubble residual does not include intrinsic scatter, the age-bias could be the primary cause of the observed extra dimming of distant SNe Ia. Furthermore, we use the PAge formalism, which is a good approximation to many dark energy and modified gravity models, to do a model comparison. We find that if intrinsic scatter is included in the Hubble residual, the Lambda cold dark matter model remains a good fit. However, in a scenario without intrinsic scatter, the Lambda cold dark matter model faces a challenge.
\end{abstract}

%\keywords{Suggested keywords}%Use showkeys class option if keyword
                              %display desired
\maketitle

%\tableofcontents

\section{Introduction}\label{intro}
Supernova cosmology has been viewed as offering the clearest proof of an accelerating universe for many years. The empirical standardization of its peak luminosity is obtained by calibrating light-curve width and color~\citep{tripp1998two,phillips1999reddening,guy2007salt2}. The generality of this standardization procedure is based on an assumption that type Ia supernova (SN Ia) outbursts are triggered by a critical condition that has little to do with the galactic environment~\citep{jha2019observational}. But this assumption has been challenged. In the past decade, a series of studies have shown that the luminosity of SNe Ia may also depend on their host galaxy properties~\citep{schiavon2006deep2,hicken2009improved,sullivan2010dependence,rigault2013evidence,rigault2015confirmation,choi2014assembly,fumagalli2016ages,uddin2020carnegie,10.5303/JKAS.2019.52.5.181,briday2022accuracy}. In particular, the correlation between progenitor age and Hubble residual (HR) has been debated in recent studies~\citep{gupta2011improved,kang2016early,jha2019observational,rose2019think,rose2020evidence,kang2020early,lee2020further,rigault2020strong,zhang2021improving,lee2022evidence,wiseman2022galaxy,wiseman2023further}. This correlation is manifested by an increase in SN Ia brightness with increasing progenitor age. Since the mean age of SN Ia progenitor evolves with redshift, the systematic bias may contribute to the observed extra dimming of distant SNe Ia and should be properly subtracted for precision cosmology that aims to measure the property of dark energy.

\defcitealias{rose2019think}{R19}\defcitealias{campbell2013cosmology}{C13}\defcitealias{kang2020early}{K20}\defcitealias{huang2020supernova}{H20}\defcitealias{lee2020further}{L20}\defcitealias{lee2022evidence}{L22}\defcitealias{zhang2021improving}{Z21}

Rose et al. (ref.~\citep{rose2019think}, hereafter~\citetalias{rose2019think}) selected 102 SNe Ia at redshift of $0.05 < z < 0.2$ with a median $z = 0.14$ from the supernova sample of Campbell et al. (ref.~\citep{campbell2013cosmology}, hereafter~\citetalias{campbell2013cosmology}) to obtain ages of the local stellar population and host galaxy, respectively. The authors found a step between the HRs of younger galaxies (age $\leq 8\,\mathrm{Gyr}$) and older galaxies (age $>8\,\mathrm{Gyr}$), but there is no correlation between the age and HR of each subgroup.
Another team Kang et al. (ref.~\citep{kang2020early}, hereafter~\citetalias{kang2020early}) analyzed 51 nearby early-type host galaxies and found a correlation between age and HR with a greater slope of $\slopep= -0.051 \pm 0.022\magGyr$. If extrapolated to higher redshift, this correlation would lead to a redshift-dependent luminosity evolution. This conclusion was immediately refuted by Rose et al. (ref.~\citep{rose2020evidence}), who argued that the analysis of~\citetalias{kang2020early} was biased by several outliers (bad-quality data points). Furthermore, Huang (ref.~\citep{huang2020supernova}, hereafter~\citetalias{huang2020supernova}) pointed out that the cosmological scenario suggested by~\citetalias{kang2020early} leads to a universe that is younger than the observed oldest stars. \citetalias{huang2020supernova} proposed a Parameterization based on cosmic Age (PAge), a unified parameterization that can well approximate many dark energy and modified gravity models, to study the cosmology with supernova magnitude evolution. \citetalias{huang2020supernova} found that a decelerating universe is consistent with the supernova data when the $\slopep$ parameter is allowed to vary in the range $\lvert \slopep\rvert \lesssim 0.057\magGyr$, as suggested by~\citetalias{kang2020early}. More recently, Lee et al. (ref.~\citep{lee2020further}, hereafter~\citetalias{lee2020further}), recalculated the age-HR correlation of SN Ia data set from~\citetalias{rose2019think}  based on Gaussian mixture models, and the outcome was consistent with the result of~\citetalias{kang2020early} from early-type host galaxies.

Lee et al. (ref.~\citep{lee2022evidence}, hereafter~\citetalias{lee2022evidence}) divided the data of~\citetalias{rose2019think} into young and old groups, with a gray area in between. The width of the gray area is taken to be greater than the typical age measurement error to avoid sample contamination between two subgroups. ~\citetalias{lee2022evidence} utilized the light-curve parameters provided in~\citetalias{campbell2013cosmology} to determine $\Delta$HR. The dependence of the luminosity on the progenitor age is estimated by taking the ratio of $\Delta$HR to $\Delta\mathrm{age}$, the difference in the median local (environmental stellar population typically within a few $\mathrm{kpc}$) ages of these two groups. Combining the measured slope, which is $\slopep = -0.040\magGyr$, with mass-weighted mean age evolution of stellar population obtained from cosmic star formation history, the authors derived the redshift evolution of $\Delta$HR and claimed that there is almost no evidence of an accelerating universe from supernova data alone. 

The analysis in~\citetalias{lee2022evidence} is straightforward and intuitively easy to understand, but it may still be challenged by the same argument that the data set is contaminated by some bad-quality samples. The present work aims to advance the analysis in \citetalias{lee2022evidence}  by using a fully Bayesian approach without artificial selection or grouping of data. Our catalog includes all the 102 supernovae in the redshift range $0.05<z<0.2$ with progenitor age measurements~\citetalias{rose2019think} and the light-curve data from~\citetalias{campbell2013cosmology}. The Bayesian approach makes use of the age uncertainty information of each supernova, and therefore naturally suppresses the statistical contribution from bad-quality samples. 

This paper is organized as follows. In Section~\ref{corre} we introduce our statistical method and apply it to the supernova catalog to update the constraint on $\slopep$. In Section~\ref{COSMO}, we use the measured $\slopep$ to correct the cosmological likelihood of type Ia supernovae~\citep{scolnic2018complete} and discuss its cosmological implication. Section~\ref{discu} concludes.
\section{Correlation between Progenitor Age and Hubble Residual}\label{corre}

For the luminosity standardization of SNe Ia, the distance modulus is given by SALT2~\citep{guy2007salt2,guy2010supernova},
\begin{equation}\label{1}
\mu_{\rm SN}=m-M+\alpha x_1-\beta c,
\end{equation}
where $m$ is the apparent magnitude, i.e., $-2.5\log_{10}(x_0)$ in~\citetalias{campbell2013cosmology}, $M$ is the absolute magnitude, and $\alpha x_1$ and $\beta c$ are correction terms depending on the light-curve width ($x_1$) and color ($c$). The relative luminosity is then represented by the Hubble residual
\begin{equation}\label{2}
\rm{HR}=\mu_{\rm SN}-\mu_{\rm model},
\end{equation}
where $\mu_{\rm model}$ is the theoretical distance modulus, given by
\begin{equation}
    \mu_\mathrm {model} = 25 + 5\log_{10}\frac{d_L}{\mathrm{Mpc}}.
\end{equation} 
The luminosity distance $d_L$ as a function of cosmological redshift $z$ is specified by the cosmological model, for instance in the flat Lambda cold dark matter ($\Lambda$CDM) model
\begin{equation}
    \left.d_L(z)\right\vert_{\rm \Lambda CDM} = (1+z)\chi(z).
\end{equation} 
The comoving angular diameter distance $\chi(z)$ is given by
\begin{equation}
    \chi(z) = \frac{c}{H_0}\int_0^z \frac{\mathrm{ d}z'}{\sqrt{\Omega_m(1+z')^3 + 1 - \Omega_m}},
\end{equation}
where $H_0$ is the Hubble constant and $\Omega_m$ is the matter abundance parameter at the present epoch.

We consider a linear regression between HR and progenitor age
\begin{equation}
    \mathrm{HR} = \frac{\Delta \mathrm{HR}}{\Delta \mathrm{age}} \times \mathrm{age} + \mathrm{intercept}.
\end{equation}
The intercept term on the right-hand side can be absorbed into the definition of $M$ parameter and will be ignored hereafter. 

To properly account for the measurement error and intrinsic scatter in the regression analysis, we write the likelihood function as
\begin{equation}\label{3}
\mathcal{L} \propto \exp\left\{-\frac{1}{2}\sum_{i}\left[\frac{\left(\rm{HR}_{\emph{i}}- \frac{\Delta \mathrm{HR}}{\Delta \mathrm{age}} \times \rm{age}_{\emph{i}}\right)^2}{\sigma_i^2}+\ln{\left(2\pi\sigma_i^2\right)}\right]\right\},
\end{equation}
where ${\rm HR}_{i}=-2.5\log_{10}{x_{0i}}+\alpha x_{1i}-\beta c_i-M-\mu_{\rm model}\left(z_i\right)$ and the index $i$ runs over the 102 low-redshift SN Ia samples used here. For each SN Ia, the total uncertainty is given by 
\begin{equation}
    \begin{aligned}
        \sigma^2 &= \left(\frac{2.5\sigma_{x_0}}{x_0\ln{1}0}\right)^2+\left(\alpha\sigma_{x_1}\right)^2+\left(\beta\sigma_c\right)^2\\
        &+2\left(-\frac{2.5\alpha}{x_0 \ln{10}}\sigma_{x_0,x_1}+\frac{2.5\beta}{x_0 \ln{10}}\sigma_{x_0,c}-\alpha\beta \sigma_{x_1,c}\right)\\
        &+\left(\frac{\Delta \mathrm{HR}}{\Delta \mathrm{age}}\sigma_{\rm age}\right)^2+\sigma_{\mathrm{pec}}^2+\sigma_{\mathrm{intrinsic}}^2,
    \end{aligned}
\end{equation}
where $\sigma_{\rm intrinsic}$  is the intrinsic scatter due to other unknow factors.  The uncertainty from the peculiar velocity uncertainty $\sigma_{\rm pec} = \frac{\partial \mu_\mathrm {model} }{ \partial z}  \frac{v_\mathrm{ pec} }{ c} $ , where $v_{\rm pec}$ is set to $500{\rm km/s}$. For flat $\Lambda$CDM model, the likelihood depends on the parameters $\{\alpha,\beta,\slopep,\Omega_{m},\sigma_{\rm intrinsic},M_{\rm eff} \}$, where $M_{\rm eff} = M - 5\log_{10}\frac{H_0}{70\mathrm{km/s/Mpc}}$. 

We employ the local age of the stellar population near each SN Ia from~\citetalias{rose2019think} and corresponding light curve data ($x_0, x_1$ and $c$) from~\citetalias{campbell2013cosmology}. Then we run Monte Carlo Markov Chain (MCMC) simulations with flat priors $\alpha\in[0.001,5]$, $\beta\in[0.5,5]$, $\slopep\in[-1,1]\magGyr$, $\Omega_m\in[0,1]$, $\sigma_{\rm intrinsic} \in [0,3] $, $M_{\rm eff}\in[-40,-10]$. 

The intrinsic scatter has been included in the analyses of \citetalias{lee2020further,lee2022evidence} and Zhang et al. (ref.~\citep{zhang2021improving}, hereafter \citetalias{zhang2021improving}). However, \citetalias{lee2022evidence} noted that HR scatter of the progenitor age at a given redshift is the most likely source. The intrinsic scatter tends to weaken the strength of the detected correlations. Here we also evaluate the scenario where $\sigma_{\rm intrinsic}$=0. Table~\ref{iii} lists the 68.3\% confidence-level bounds of parameters, covering both scenarios with intrinsic scatter and without intrinsic scatter. Notably, we find at least a $3.5\sigma$ hint of nonzero $\slopep$, which indicates a correlation between the progenitor age. We further test the robustness of the constraint on $\slopep$ by randomly removing 30 supernova samples in our catalog. Repeated tests show that with the randomly selected subgroup of data, we still find $\slopep<0$ at $\gtrsim 2\sigma$ confidence level. 
\begin{table*}[h!t]
\centering
\caption{Mean value with 1$\sigma$ error. Data: 102 low-$z$ supernova samples with progenitor age information from~\citetalias{rose2019think} and corresponding light curve data from~\citetalias{campbell2013cosmology}.}
\label{iii}
\begin{tabular}{ccc}
\hline
Parameter& With intrinsic scatter& Without intrinsic scatter\\
\hline
$\alpha$&0.16 $\pm$ 0.02&0.18 $\pm$ 0.01\\
$\beta$&2.88 $\pm$ 0.18&3.20 $\pm$ 0.14\\
$\slopep$&$-0.021 \pm$ 0.006 \,$\mathrm{mag/Gyr}$&$-0.032^{+0.004}_{-0.005}$ \,$\mathrm{mag/Gyr}$\\
$M_{\rm eff}$&$-29.58\pm0.06$&$-29.51\pm0.05$\\
$\sigma_{\rm intrinsic}$ &0.108 $\pm$ 0.015&-\\
$\Omega_m$ & $0.46^{+0.19}_{-0.25}$& $0.56\pm 0.19$\\
\hline
\end{tabular}
\end{table*}
\begin{figure}[b]
\centering
\includegraphics[width=0.46\textwidth]{slope-Om.pdf}	  
\caption{The marginalized distributions of $\slopep$ and $\Omega_m$. Inner and outer contours correspond to 68.3\% and 95.4\% confidence levels, respectively. The orange and green regions represent situations involving intrinsic scatter and no intrinsic scatter, respectively.}
\label{slope-Om}
\end{figure}
\begin{figure}[b]
\centering
\includegraphics[width=0.46\textwidth]{slope.pdf}  
\caption{The age–luminosity relation. The data points are 102 SNe Ia in \citetalias{rose2019think}. The orange and green lines, corresponding to cases with intrinsic scatter and without intrinsic scatter respectively, represent regression fittings with the mean value of $\slopep$ given in Table~\ref{iiii}. The shaded area represents the 1$\sigma$ level region.}
\label{slope}
\end{figure}
As shown in Figure~\ref{slope-Om}, there is no noticeable degeneracy between $\slopep$ and $\Omega_m$. The preference of a nonzero $\slopep$ is then, in the $\Lambda$CDM framework, cosmology-independent. This is somewhat expected because the calibration of $\slopep$ only involves low-redshift ($z\le 0.2$ here) supernova data, which is insensitive to cosmology. For the same reason, the analysis here has almost no constraining power on cosmological parameters. To validate this claim, we directly constrained the parameters using the \citetalias{rose2019think} HR dataset. The HR values in the data set are ascertained through \citetalias{campbell2013cosmology}, with the stretch parameter $\alpha$ being adjusted to $0.16$. The constrained results are shown in Table~\ref{iiii}. The Hubble residual is directly visualized in Figure~\ref{slope}. We find that $\slopep$ almost remains unaffected.
%we note that the value of beta is relatively smaller compared to \citetalias{campbell2013cosmology}. Hence, we adopt Gaussian priors $\Omega_m = 0.27 \pm 0.155$ and $\beta = 3.12 \pm 0.12$ as suggested by \citetalias{campbell2013cosmology}. The constrained results are shown in Table~\ref{iiii}. We find that $\slopep$ remains unaffected.
\begin{table*}[h!t]
\centering
\caption{Mean value with 1$\sigma$ error. Data: 102 low-$z$ supernova samples with progenitor age information and corresponding HR value from~\citetalias{rose2019think}.}
\label{iiii}
\begin{tabular}{ccc}
\hline
Parameter& With intrinsic scatter& Without intrinsic scatter\\
\hline
$\slopep$&$-0.021 \pm$ 0.006 \,$\mathrm{mag/Gyr}$&$-0.033^{+0.004}_{-0.005}$ \,$\mathrm{mag/Gyr}$\\
$\mathrm{Intercept}$&$0.092\pm0.034$&$0.152^{+0.02}_{-0.03}$\\
$\sigma_{\rm intrinsic}$ &$0.117^{+0.013}_{-0.015}$&- \\
\hline
\end{tabular}
\end{table*}
In the next section we study cosmology with the Pantheon supernova catalog that covers a much broader redshift range $0<z<1.5$~\citep{scolnic2018complete}, based on the assumption that the HR-age relation also applies to high-redshift SNe Ia. 
\section{Cosmology Analysis}\label{COSMO}

The age information of individual samples in the Pantheon supernova catalog is not available. We instead use the mean age of the stellar population as a function of redshift, derived from the theory of cosmic star formation history and tabulated in Table 1 of~\citetalias{lee2022evidence}. 
We utilize the heliocentric redshift $z_{\rm hel}$, the CMB restframe redshift $z_{\rm cmb}$ and the covariance matrix of distance modulus provided by Pantheon supernova catalog. The luminosity distance of SN Ia can be expressed as $d_L=(1+z_\mathrm{hel})\chi(z_\mathrm{ cmb})$.  The nuisance parameter $M$ is marginalized using the method of ref.~\citep{conley2010supernova}. Additionally, we removed six supernovae with redshifts $z>1.5$ that are out of the tabulated redshift range and revised the standard supernova likelihood by adding a $\slopep \times \mathrm{age}$ term to the distance modulus, where $\mathrm{age}$ is the mean age of the stellar population, and apply a flat prior on $\slopep\in [-0.1, 0.1]\,\mathrm{mag/Gyr}$.
\begin{figure}[b]
\centering
\includegraphics[width=0.46\textwidth]{LCDM.pdf}
\caption{ The marginalized $68.3\%$ (inner blue contour) and $95.4\%$ (outer blue contour) confidence level constraints with Pantheon SN Ia data set and a flat prior on $\slopep$. The horizontal black band stands for Planck constraint on matter abundance ($\Omega_m=0.315 \pm 0.007$). The vertical bands are four cases of supernova magnitude evolution: (i) the standard assumption of no HR-age dependence ($\slopep = 0$, purple); (ii) the \citetalias{lee2022evidence} result $\slopep=-0.04\magGyr$ (red); (iii) $\slopep = -0.021\pm 0.006\magGyr$ (orange), which was identified in this research including intrinsic scatter; (iv) $\slopep = -0.032^{+0.004}_{-0.005}$ (green), as found in this study without intrinsic scatter. We label these four cases on $\slopep$ as ``Standard'', ``\citetalias{lee2022evidence}'', ``With intrinsic scatter'' and ``Without intrinsic scatter'', respectively. }
\label{lcdm}
\end{figure} 

\begin{figure*}[htbp]
\centering
\includegraphics[width=0.46\textwidth]{Page_10.pdf}
\includegraphics[width=0.46\textwidth]{Page.pdf}
\caption{The $1\sigma$ and $2\sigma$ confidence level contours of the flat PAge approximation for the supernova data with four priors on $\slopep$ and for the BAO data. Figure (a) and Figure (b) respectively depict the scenarios with a lower age limit for the universe of 10 Gyr and 12 Gyr. The black dot $(\eta, p_{\rm age}) = (0.359, 0.951)$ is a good approximation to the Planck best-fit $\Lambda$CDM model ($\Omega_{m} \approx 0.315$).  The dashed blue $q_0=0$ line is the critical line between cosmic acceleration (upper right region) and deceleration (lower left region).}
\label{page}
\end{figure*} 
\begin{figure*}[htbp]
\centering
\includegraphics[width=0.23\textwidth]{Page_random_10.pdf}
\includegraphics[width=0.23\textwidth]{Page_random_noint_10.pdf}
\includegraphics[width=0.23\textwidth]{Page_random.pdf}
\includegraphics[width=0.23\textwidth]{Page_random_noint.pdf}
\caption{The $1\sigma$ and $2\sigma$ confidence level contours of the flat PAge approximation for the supernova data shown in Figure (a) and (c) with the prior $\slopep = -0.021\pm 0.006\magGyr$, and in Figure (b) and (d) with the prior $\slopep = -0.032\pm 0.005\magGyr$. Figure (a) and (b) depict scenarios with a lower age limit for the universe of 10 Gyr, while Figure (c) and (d) represent scenarios with a lower age limit of 12 Gyr. The inner blue contours are the cases where $-1 \sim 1$ Gyr age fluctuation is randomly added to each supernova. The black dot $(\eta, p_{\rm age}) = (0.359, 0.951)$ is a good approximation to the Planck best-fit $\Lambda$CDM model ($\Omega_{m} \approx 0.315$).}
\label{page_random}
\end{figure*} 

The marginalized constraints on $\Omega_m$ and $\slopep$ are compared with four cases of supernova magnitude evolution: (i) the standard assumption of no HR-age dependence ($\slopep = 0$); (ii) $\slopep=-0.04\magGyr$ that was found in~\citetalias{lee2022evidence}; (iii) $\slopep = -0.021\pm 0.006\magGyr$ (Gaussian error) that was found in this work with intrinsic scatter, and (iv) $\slopep = -0.032\pm 0.005\magGyr$ (Gaussian error) for the case without intrinsic scatter. The result is visually shown in Figure~\ref{lcdm}, which is a contour with Pantheon SN Ia data set and a flat prior on $\slopep$. Compared to the standard ($\slopep=0$) case, the HR-age dependence in the case with intrinsic scatter ($\slopep = -0.021\pm 0.006\magGyr$) leads to a larger $\Omega_m = 0.392 \pm 0.037$ that is in mild ($\sim 2\sigma$) tension with the concordance model ($\Omega_m = 0.315\pm 0.007$) ~\citep{Planck18}. For the case without intrinsic scatter ($\slopep = -0.032\pm 0.005\magGyr$), the matter abundance further increases to $\Omega_m = 0.480^{+0.040}_{-0.047}$, which is in $\sim 3.5\sigma$  tension with the concordance model\footnote{The matter abundance $\Omega_m$ here is constrained by the Pantheon SN Ia data set, while $\Omega_m$ in Table~\ref{iii} is constrained by the 102 low-redshift SN Ia samples, so they are different.}. If we accept the assumption that the HR-age dependence can be extended to high redshift, the tension with the Planck standard cosmology could be a hint of new physics beyond the flat $\Lambda$CDM model. 

To test whether a non-standard cosmology can fit the data much better, we use the PAge approximation, which provides a simple and almost model-independent approximation to $\Lambda$CDM and many beyond-$\Lambda$CDM models~\citep{huang2020supernova, luo2020reaffirming, MAPAge, GRB2021PAge,huang2022s, Huang2022PAge, cai2022no2, cai2022no1, li2022redshift}. Model comparison can be easily done in the PAge parameter space with one single MCMC run. The PAge approximation is formulated as 
\begin{equation}\label{4}
\frac{H}{H_0}=1+\frac{2}{3}\left (1-\eta\frac{H_0t}{p_{\mathrm{age}}}\right )\left (\frac{1}{H_0t}-\frac{1}{p_{\mathrm{age}}}\right ),
\end{equation}
where $t$ is the cosmological time. $p_{\rm age}=H_0t_0$ is the product of the Hubble constant $H_0$ and the current age of the universe $t_0$. The phenomenological parameter $\eta$ can be regarded as a fitting parameter that represents the deviation from an Einstein-de Sitter Universe. Studying the representation of a specific model in the PAge parameter space requires (i) the product of the current Hubble constant ($H_0$) and cosmic age ($t_0$), and (ii) the present-day deceleration parameter ($q_0$). Subsequently, we can obtain the PAge parameter $p_{age} = H_0 t_0$ and $\eta = 1 - \frac{3}{2} p_{age}^2(1 + q_0)$. The Planck concordance cosmology ($\Omega_m\approx 0.315$) is well approximated by the PAge model with $p_{\rm age} = 0.951$ and $\eta= 0.359$.

Because PAge is a phenomenological parameterization for the late universe, we are not supposed to apply PAge to cosmic microwave background data, whose anisotropy involves the evolution of cosmological perturbations from the early universe. For comparison with external data, we instead use the baryon acoustic oscillations (BAOs). The BAO data set includes the latest measurements from the Sloan Digital Sky Survey (SDSS)-IV\footnote{https://svn.sdss.org/public/data/eboss/DR16cosmo/\\
tags/v1\_0\_0/likelihoods
/BAO-only/}~\citep{alam2021completed}, the Six-degree Field Galaxy Survey (6dFGS)~\citep{beutler20116df}, and the Dark Energy Survey Year 3 (DES Y3) final release~\citep{abbott2022dark}. 
%It is an approximation to apply the mean ages in Table 1 of ~\citetalias{lee2022evidence}, which are computed with a reference $\Lambda$CDM model, to models that are beyond, but in the proximity of $\Lambda$CDM. 

The mean ages in Table 1 of ~\citetalias{lee2022evidence} are computed with a reference $\Lambda$CDM model. Applying these ages to models that are beyond, but in the proximity of $\Lambda$CDM is an approximate method. To avoid running into parameter space that is too far away from the concordance model and hence the approximation may fail, for all cases we apply a lower limit for the age of the universe and a flat prior for the Hubble constant $H_0 \in [65,75]\,\mathrm{km/s/Mpc}$ to restrict the parameter space. For the lower limit age of the universe, we consider the 12 Gyr age provided by metal-poor globular clusters (GC)~\citep{vandenberg2014three,catelan2017ages,sahlholdt2019benchmark}. However, due to the uncertainty in the absolute age determination of GCs, we also investigate the impact of a lower age (10 Gyr) of metal-poor GCs. 

We then run MCMC calculations by replacing the flat prior on $\slopep$ with the four cases of supernova magnitude evolution, the ``standard'' case with fixed $\slopep=0$, the ``\citetalias{lee2022evidence}'' case with fixed $\slopep=-0.04\magGyr$, ``with intrinsic scatter'' case $\slopep=-0.021\pm 0.006\magGyr$, and ``without intrinsic scatter'' case $\slopep=-0.032\pm 0.005\magGyr$. The marginalized constraints on PAge parameters $p_{\rm age}$ and $\eta$ are shown in Figure~\ref{page}. Compared to the ``\citetalias{lee2022evidence}'' case where a decelerating universe is more favored, the updated constraint on $\slopep$ including intrinsic scatter in this work reconciles cosmic acceleration at $\sim 2\sigma$ level. The Planck best-fit $\Lambda$CDM model, well approximated by the $(\eta, p_{\rm age}) = (0.359, 0.951)$ point, is in $1.5\sigma$ tension with the best-fit parameters when considering the intrinsic scatter. While for the case without intrinsic scatter, we find $\sim 1\sigma$ hint of cosmic acceleration, and $\sim 3 \sigma$ tension to the Planck best-fit $\Lambda$CDM model. For a given redshift, the age uncertainties for different SNe Ia may have implications. Figure~\ref{page_random} shows that after adding a random age fluctuation of $-1 \sim 1$Gyr to each supernova, the tension between this work and CMB is further reduced. 

In both cases of the Hubble residual including and excluding intrinsic scatter, we conduct a comparison of the difference in the best-fit $\chi^2$ values, which are defined as $-2\ln \mathcal{L}$, between the MCMC results for the $\Lambda$CDM model and the PAge model. When we examine the case with intrinsic scatter, the difference in the best-fit $\chi^2$ is less than 0.3, suggesting no significant evidence supporting a model beyond $\Lambda$CDM model. However, for the case without intrinsic scatter, the best-fit $\chi^2$ for the Page model is 6.2 lower than that for the $\Lambda$CDM model, suggesting a challenge to the $\Lambda$CDM model.

\section{Discussion and Conclusion}\label{discu}
%From Table~\ref{i} and Fig.~\ref{page}, we can see that our correction results do not contradict BAOs, and favor the cosmic acceleration at the confidence levels of 1.9$\sigma$, while \citetalias{lee2022evidence} support the deceleration factor $q_0 = 0$ at $\ll 1\sigma$ level. The mapping of $\Lambda$CDM on PAge, represented by the black dots shown in Fig.~\ref{page}, are $\sim 1.4\sigma$ of our results, which is lower than one in the $\Lambda$CDM case, because the posterior distribution under the page approximation is non-Gaussian as can be seen from Fig.~\ref{page}. Compare the $\chi_{\rm min}^2$/d.o.f in Table~\ref{table:LCDM} and  Table~\ref{i}, it can be found that the PAge approximat is not much better than $\Lambda$CDM model. So, even if age dependence is admitted, there is no evidence of beyond $\Lambda$CDM yet. 
The usage of standardized peak luminosity of type Ia supernova as a standard candle has been recently challenged by a series of works that claim the detection of correlations between the standardized luminosity and properties of the host galaxies. In particular,  for low-redshift supernovae whose environmental stellar population can be resolved,~\citetalias{lee2022evidence} found a dependence of the Hubble residual on the age of its environmental stellar population. The analysis in~\citetalias{lee2022evidence} does not make use of the age uncertainties, and therefore may be contaminated by bad-quality samples. The present work advances the analysis in~\citetalias{lee2022evidence} by including the age uncertainties in a fully Bayesian inference framework and finds a significantly weaker dependence of the Hubble residual on the progenitor age, $\slopep=-0.021\pm 0.006\magGyr$ with intrinsic scatter around 0.11.

With a bold assumption that the progenitor age dependence can be extended to high redshift supernovae, we studied the impact of the supernova luminosity evolution on cosmology. We find that if Hubble residual includes the intrinsic scatter, $\Lambda$CDM model remains a good fit for the supernova data, but a high $\Omega_m$ value is preferred ($\sim 2\sigma$ larger than the Planck results). In contrast, the~\citetalias{lee2022evidence} result ($\slopep=-0.04\magGyr$) almost excludes (in $\sim 4\sigma$ tension with) the Planck concordance model. In addition, under the non-flat $\Lambda$CDM model, both CMB and BAO are consistent with the result in this work if we consider the intrinsic scatter (Figure~\ref{ocdm}).

\citetalias{lee2022evidence} is based upon previous research, such as \citetalias{kang2020early}, \citetalias{lee2020further}, and \citetalias{zhang2021improving}, which all thoroughly considered the uncertainty of progenitor ages. \citetalias{kang2020early} claimed to have discovered a correlation between the ages of 34 early-type host galaxies and the peak luminosity of Type Ia supernovae, with a slope $\slopep = -0.051 \pm 0.022 \magGyr$. However, \citetalias{rose2019think} pointed out that due to poor sampling of the light curves of Type Ia supernovae, \citetalias{kang2020early} overestimated the slope. \citetalias{lee2020further} and \citetalias{zhang2021improving} conducted their studies using the same data source as in this paper, namely the \citetalias{rose2019think} HR dataset. \citetalias{lee2020further} presented a relatively steep value of $\slopep =  -0.057 \pm 0.016 \magGyr$. We have found that the HR values used by \citetalias{lee2020further} were based on an $\alpha$ value of $0.22$. However, \citetalias{campbell2013cosmology} and \citetalias{rose2019think} indicated that this $\alpha$ value was larger than the typical value. Consequently, this larger $\alpha$ amplified the results of \citetalias{lee2020further}. By changing the $\alpha$ value to $0.16$ and replicating the work of \citetalias{lee2020further} using the LINMIX package, we obtained $\slopep =  -0.032 \pm 0.017 \magGyr$, which differs from our results by only a $0.7\sigma$ level. This value is close to the one obtained by \citetalias{zhang2021improving}, $\slopep =  -0.035 \pm 0.007 \magGyr$, whose posterior sampling method adopts the full posterior for the age error instead of the Gaussian error. The tension between the results of \citetalias{zhang2021improving} and our findings is at a $1.5\sigma$ level.
\begin{figure}[htbp]
\centering
\includegraphics[width=0.46\textwidth]{ocdm.pdf}
\caption{ The marginalized $68.3\%$ and $95.4\%$ 
confidence level constraints for the $\Omega_m$ and $\Omega_{\Lambda}$ cosmological parameters under the non-flat $\Lambda$CDM model. For this diagram, SN Ia and BAO data sets are the same as Figure~\ref{page}. The CMB data are from the base\_omegak\_plikHM\_TTTEEE\_low\_lowE\_lensing chain provided by Planck~\citep{Planck18}.}
\label{ocdm}
\end{figure} 

Another possible interpretation of the luminosity evolution is that the SNe Ia survey strategy favors large-mass galaxies, and the median age of the host galaxies of Pantheon catalog may be biased by the selection effect.  Moreover, Pantheon catalog and the low-redshift catalog with age information are two different catalogs and may follow different statistics. A more detailed calculation of the age distribution of the host galaxies of the Pantheon catalog is beyond the scope of the present work. We leave exploration along this direction as our future work. While the impact of the selection effect remains uncertain, we argue that the luminosity evolution problem remains a possible systematic error and is worthy of in-depth study for future high-precision supernova cosmology and more detailed comparison with other cosmological probes in the near future~\cite{ade2019simons,liu2022forecasts,miao2023cosmological}. 

%We also attempt to use the global (host galaxy) ages from \citetalias{rose2019think} and get $\Delta$HR/$\Delta$age = $-0.016 \pm 0.006\magGyr$ that is consistent with the result from local (stellar population) ages. 

%Thus, we posit that the adoption of either global age or local age is unlikely to elicit a substantial deviation in outcomes.

%Previous investigations, such as \citetalias{rose2019think}, \citetalias{lee2020further}, and \citetalias{lee2022evidence}, have also utilized data from the same sources. Our analysis is as follows: 
%\begin{itemize}
%\item From Fig.~\ref{slope}, we can see that the error of progenitor ages is very considerable. However, \citetalias{lee2022evidence} wiped out the progenitor age error of each supernova in the subgroup, resulting in an overcorrection of the age-luminosity ($\slopep = -0.04\magGyr$). Therefore, improving the calculation method of \citetalias{lee2022evidence} was the motivation of our study. After accounting for the progenitor age errors, $\lvert \Delta$HR/$\Delta$age$\rvert $ is reduced by $50\%$.
%\item The $\Delta$HR/$\Delta$age of \citetalias{rose2019think} is shallower than ours (see their Fig.~6). We also do not agree with \citetalias{rose2019think} that there is almost no correlation between progenitor age and HR. Our results give the correlation at a 3.3$\sigma$ level. \citetalias{lee2020further} have pointed out that \citetalias{rose2019think} produced a regression dilution bias, seriously underestimates the slope and significance of the age-HR correlation.
%\item \citetalias{lee2020further} presented a relatively steep value of $\slopep = -0.057\pm0.016\magGyr$. We find that the HR values used by \citetalias{lee2020further} were based on $\alpha$ = 0.22, which was obtained from the full photometric sample by \citetalias{campbell2013cosmology}. However, \citetalias{campbell2013cosmology} indicated that this $\alpha$ was larger than the typical value. Therefore, this large $\alpha$ amplified the results of \citetalias{lee2020further}. By changing the $\alpha$ value to 0.16 which is suggested by \citetalias{campbell2013cosmology}, and replicating the work of \citetalias{lee2020further} using the LINMIX package, we obtain $\slopep = -0.032\pm 0.017\magGyr$, which differs from our results by only a 0.7$\sigma$ level.
%\end{itemize}
%\st{As redshift increases, the maximum possible age of the stellar population decreases, resulting in a corresponding decrease in the average age of the stellar population. The age of the population is correlated with HR, leading to redshift evolution in the peak luminosity of supernovae. This deviates from conventional perspectives and may no longer provide evidence for an accelerating universe. } 
%Then we assum that the HR-age relationalso applies to high redshift SNe and study cosmology with the Pantheon supernova catalog~\citep{scolnic2018complete} under flat $\Lambda$CDM model. For the Gaussian prior $\slopep = -0.020\pm 0.006\magGyr$ , we obtain $\Omega_m=0.392\pm 0.037$, which is in $\sim 2\sigma$ tension with the measurement from cosmic microwave background (CMB)~\citep{Planck18}. 
%Fig.~\ref{lcdm} indicates that after considering age dependence, our result is a little contradiction with planck standard cosmology, but the contradiction is not as exaggerated as \citetalias{lee2022evidence}. 

%And for non-flat case, we acquire $\Omega_{m} = 0.344 \pm 0.093$ and $\Omega_{k} = 0.14^{+0.22}_{-0.24}$, which shows that even considering the evolution of supernova brightness, it is not necessary, as in \citetalias{kang2020early} and \citetalias{lee2022evidence}, to use a large spatial curvature ($\Omega_{ k}\sim 0.7$) to fit the Hubble diagram. 

%Subsequently, we employ the PAge approximation to see if age dependence imply other beyond $\Lambda$CDM models. The PAge approximation involves the same number of parameters as $w$CDM but covers a more comprehensive range of scenarios beyond the conventional dark energy concept. We use the supernova data with and without HR-age dependence and BAO data to constrain the approximation of PAge, respectively. As we can see from Fig.~\ref{page}, we yield a support for cosmic acceleration at the 1.9$\sigma$ confidence levels. The Planck bestfit $\Lambda$CDM model is a good fit with $\sim 1.4\sigma$ confidence level, which is lower than one in the $\Lambda$CDM case, because the posterior distribution under the page approximation is non-Gaussian as can be seen from Fig.~\ref{page}. Compare with $\Lambda$CDM, the $\chi_{\rm min}^2$/d.o.f of PAge approximation is not improved. So even if age dependence is admitted, there is no evidence of beyond $\Lambda$CDM yet. 
%In addition, our corrected results are also consistent with BAOs. 
%While the consequences for \citetalias{lee2022evidence} fully support $q_0$ = 0 at $\ll 1\sigma$ level and exclude the Planck bestfit $\Lambda$CDM model.


%Therefore, we argue that the non-accelerated universe claimed by \citetalias{lee2022evidence} does not match other observations (e.g.  the Planck results for base $\Lambda$CDM) due to their neglect of the progenitor age errors and exaggeration of the age-HR steepness. We show that even considering the age-luminosity evolution, the results are still consistent with BAOs and the Planck results for base $\Lambda$CDM. The luminosity evolution of supernovae due to the correlation between progenitor age and HR stands up to scrutiny. This systematic bias should be considered before delving into the details of the dark energy.

%%%%%%%%%%%%%%%%%%%%%%%%%%%%%%%%%%%%%%%%%%%%%%%%%%%%%%%
%%% Acknowledgements. ??§Ý
%%%%%%%%%%%%%%%%%%%%%%%%%%%%%%%%%%%%%%%%%%%%%%%%%%%%%%%
\section*{Acknowledgements}
This work was supported by the National SKA Program of China (Grant No. 2020SKA0110402), the National Natural Science Foundation of China (NSFC) (Grant No. 12073088), the National Key R\&D Program of China (Grant No. 2020YFC2201600), and the Guangdong Major Project of Basic and Applied Basic Research (Grant No. 2019B030302001).

\bibliography{biblio} 
\end{document}
%
% ****** End of file apssamp.tex ******
