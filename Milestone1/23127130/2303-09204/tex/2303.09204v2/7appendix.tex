








\section{Proofs for Section~\ref{sec-polyseq}}
\label{sec:app-lowerbownun}

\begin{proof}[Proof of \autoref{claim:sizefunnyQ}]
First note that $A=\frac{b-a}{c}$.
The claim states that
\[ \log(F_n)	\geq \frac{2(b-a)}{c} (n\log n - n) \, .\]

The proof is as follows.
Given $y\in \N$, we first observe that 
\[\prod_{cx \leq y} (cx)^2 \geq   c^{2y}\bigg(\bigg\lfloor\frac{y}{c} \bigg\rfloor!\bigg)^2.\]
By Stirling's formula, the logarithm of the  quantity above is at least 
\begin{equation}
\label{eq:logQI}
\frac{2y}{c}\log c+ \frac{2y}{c} \log y -\frac{2y}{c}.
\end{equation}

Now \(F_n = \prod_{k\in I(n)} (k^2+\beta)\) is bounded from below by
\begin{equation*}
	F_n \geq \prod_{k\in I(n)} k^2
	\geq \prod_{ an\leq cx \leq bn} (cx+d)^2 
	\geq \prod_{ an\leq cx \leq bn} (cx)^2.
	\end{equation*}
By the above, and Equation~\eqref{eq:logQI} we conclude that \(\log(F_n)\) is bounded from below by
\begin{equation*}
	\log \prod_{cx \leq bn} c^2x^2  - \log \prod_{cx \leq an} c^2x^2
	\geq  \frac{2(b-a)}{c}( n \log n -n),
\end{equation*}
as required.  
\end{proof}

We now prove the inequality~\eqref{eq:BOUND} from the proof
of~\Cref{theo:bigprimespolyT}.
Noting that $A=\frac{b-a}{c}$, the inequality states
that
\begin{equation}
  e_p \leq  \frac{2An}{p-1}+  \frac{\varepsilon_2 \log n}{\log p}
\tag{\ref{eq:BOUND}}
\end{equation}

\begin{proof}[Proof of Inequality \eqref{eq:BOUND}]

If $e_p=0$ then the bound trivially holds.
Suppose $e_p>0$.  Then the function $f$ has two roots in $\Z/p\Z$.
Define $g\in\Z[x]$ by $g(x):=f(cx+d)$.
In case $p>c$ then $g$ also has two roots in $\Z/p\Z$.
For all $n\in\N$ define the products
\begin{equation*}
    G_n:=\prod_{k=1}^{\left\lfloor \frac{bn-d}{c} \right\rfloor} g(k)
    \quad \text{and} \quad
    H_n:=\prod_{k=1}^{\left\lceil \frac{an-d}{c} \right\rceil - 1} g(k)
\end{equation*}
    Then $F_n=\frac{G_n}{H_n}$ and hence
    $e_p=v_p(F_n)=v_p(G_n)-v_p(H_n)$.
Applying \Cref{prop:moll}, we get, for some constant $\varepsilon>0$,
\begin{align*}
v_p(G_n) \leq & \frac{2(bn-d)}{c(p-1)} + \frac{\varepsilon\log 
n}{\log p} \qquad {\text{ and }  }\\
v_p(H_n) \geq & \frac{2(an-d-c)}{c(p-1)} -
\frac{\varepsilon\log 
n}{\log p}.
\end{align*}
The upper bound in~\eqref{eq:BOUND} follows, for a suitable
choice of the constant $\varepsilon_2$, by subtracting the
upper bound for $v_p(G_n)$ from the lower bound for $v_p(H_n)$.
\end{proof}

\section{Second Case in the proof of Theorem~\ref{theo-decide-quad}}
\label{app-sec-quad}

Let $\beta \not\equiv 1 \pmod{4}$ be a square-free integer and
$\K=\Q(\sqrt{\beta})$ a quadratic field over which the polynomials $f$
and $g$ in~\eqref{eq:rel} split completely.  By Theorem~\ref{thm:quadratic}, the ring of integers of the field 
$\K$ is $\Z[\sqrt{\beta}]$. We define $\theta:=\sqrt{\beta}$,
so that $m_{\theta}:=x^2-\beta$ is the minimal polynomial of~$\theta$. 

Exactly as in Subsection~\ref{subsec-partition-roots}, we partition the set $\mathcal R$ of roots of $fg$ into classes, define the balanced and unbalanced classes, define the linear ordering~$\prec$ on $\mathcal R$, and consider the \emph{least unbalanced class} $C_0$.  
Let $a_0\theta+b_0$ be the greatest element in $C_0$ and note that $a_0\geq 1$ as before.


\subsection{Threshold conditions}
Next we exhibit a threshold $B$ (defined in terms of the
recurrence~\eqref{eq:rel}) such that for all $n>B$ there are
rational integers $\theta'$ and $p$, with $p>n$ prime, satisfying the three
conditions (P1)--(P3) as stated in Subsection~\ref{subsec-thre}.



The definitions for $\theta'$ and $p$ are as follows.
Consider
the interval
\[ I(n):= \left\{ k\in\N : \frac{3n}{3a_0+2}\leq k \leq
    \frac{3n}{3a_0+1} \right\} \] 
    and let
    $M$ be an upper bound on $\{|a|,|b|: a \theta+b\in 
\mathcal{R}\}$, and the height of the minimal polynomials of the elements of $\mathcal{R}$.
By
\autoref{theo:bigprimespolyT}, there is an effective threshold $B$, which we may assume to be greater than $3M(M+1)$,
such that for all $n>B$ there exists a prime $p > 3Mn$ that divides the
product
\[ \prod_{k\in I(n)} k^2-\beta.\]
Further, since \(p>3Mn\) is prime, we deduce that for \(n>B\) there exists 
$\theta' \in I(n)$ such that $(\theta')^2 \equiv \beta \pmod{p}$. 

We will show that $\theta'$ and $p$ satisfy Conditions (P1)--(P3).  Now
\begin{equation*}
    m_\theta(\theta') \equiv (\theta')^2-\beta \equiv 0 \pmod{p}.
\end{equation*}
Thus $\theta'$ satisfies
Condition (P1).

We turn next to establishing Condition (P2).
Since $\theta'\in I(n)$, it is straightforward to show that
\begin{equation}
(a_0+\textstyle\frac{1}{3})\theta' \leq n \leq
(a_0+\textstyle\frac{2}{3})\theta' \, .
  \label{app-eq:INEQ}
\end{equation}
These 
  bounds are identical to those in~\eqref{eq:INEQ}.
  In this case, Conditions (P2) and (P3) follow by an analogous argument to that given in Subsection~\ref{subsec-thre}.

\bigskip
\bigskip

The remaining part of the proof for the case $\beta\equiv  1 \pmod{4}$, as given  in Subsection~\ref{subsec-primediv} and Subsection~~\ref{subsec-conclud}, carries over to the present case without change. 