In this section, we study  
hypergeometric sequences~$\langle u_n \rangle_{n=0}^\infty$, satisfying  first-order recurrences of the special form
\begin{equation}\label{eq:poly}
u_n=f(n) u_{n-1} \quad \text{ and } \quad u_0=1,
\end{equation}
where $f\in \Z[x]$ has no non-negative integer roots.  We call such a
recurrence \emph{monic}.  We analyse the prime divisors of
sequences~$\langle u_n \rangle_{n=0}^\infty$ that satisfy
such a monic
recurrence.  In particular, we recall two results that will serve as
stepping stones toward our main decidability theorems in the
subsequent sections.  
Following~\cite{Moll09}, for a fixed prime \(p\), the first result establishes an asymptotic estimate 
for the $p$-adic valuation $v_p(u_n)$
as $n$ tends to infinity.  Next, following~\cite{everest07}, when
$f$ is a quadratic polynomial we prove a result that yields asymptotic
estimates on the size of the largest prime divisors of $u_n$ as $n$
tends to infinity.  The restriction on the degree is necessary given the
state of the art: estimates on large prime divisors constitute hard
open problems in the theory of
polynomials~\cite{hinz1996multiplicative,heathbrown2001largest}.

\subsection{Asymptotic growth of valuations}
\label{subsec:asymval}

Let $p\in \N$ be  prime.
Consider a hypergeometric sequence $\langle u_n \rangle_{n=0}^\infty$, satisfying a monic recurrence~\eqref{eq:poly}. 
Since $u_n=\prod_{k=1}^n f(k)$, we have 
\begin{equation*}
	v_p(u_n) = \sum_{k=1}^n v_p(f(k)). 
\end{equation*}
In this section we recall the result of~\cite{Moll09} that
characterises the asymptotic growth of~$v_p(u_n)$ in terms of the
number of roots of $f$ in $\Z/p\Z$.  The key tool in this argument is
Hensel's Lemma.

\begin{theorem}[Hensel's Lemma {\cite[Theorem 4.7.2]{gouvea2020padic}}]
Let \(f(x)\in\Z[x]\)  and assume that there exist polynomials 
\(g(x)\) and \(h(x)\) such that: 
i) \(g(x)\) is monic, 
ii) \(g(x)\) and \(h(x)\) are relatively prime modulo \(p\), and 
iii) \(f(x) = g(x)h(x) \pmod{p}\).

Then for all $e>0$ there exist polynomials \(g_1(x),h_1(x)\in\Z[x]\) such that: i) \(g_1(x)\) is monic, ii) \(g_1(x)\equiv g(x) \pmod{p}\) and \(h_1(x)\equiv h(x) \pmod{p}\), and \(f(x)=g_1(x)h_1(x) \pmod{p^e}\).
\end{theorem}

Define 
a \emph{Hensel prime} for $f\in \Z[x]$ to be a prime that does not divide the
discriminant of any irreducible factor of $f$.  Since the discriminant
of an irreducible polynomial is non-zero, all but finitely many primes
are Hensel primes for a given polynomial.

Given a prime $p$, suppose that $f \in \Z[x]$ has $m$ roots in
$\Z/p\Z$,
i.e., suppose that~$f$ factors as
\[ f=(x-\alpha_1)^{m_1} \cdots (x-\alpha_\ell)^{m_\ell} g(x) \pmod{p},\]
%
where $\alpha_1,\ldots,\alpha_\ell\in \Z$, $g\in \Z[x]$ has no root  modulo $p$, and $m=m_1+\cdots+m_\ell$.
In this case, if 
$p$ is a Hensel prime for $f$ then for all $e>0$ we can apply Hensel's Lemma to obtain a factorisation
\[f(x)=(x-\beta_1)^{m_1} \cdots (x-\beta_\ell)^{m_\ell} h(x) \pmod{p^e}\]
where $\beta_1,\ldots,\beta_\ell \in \Z$, and $h\in \Z[x]$ has no root
modulo $p$.  
In other words, $f$ has exactly $m$ roots in the ring $\Z/p^e\Z$.

The following result is a reformulation
of~\cite[Corollary 3.1]{Moll09}.  For later use, we formulate the result so as to make explicit the 
dependence of the bounds for $v_p(u_n)$ on the prime $p$.  The proof remains the same.
\begin{proposition}[{\cite[Corollary~3.1]{Moll09}}]
\label{prop:moll}
Suppose that $\langle u_n \rangle_{n=0}^\infty$ satisfies the monic recurrence in Equation~\eqref{eq:poly} with polynomial coefficient
$f \in \Z[x]$.  
Let $p$ be a Hensel prime of $f$ such that $f$ has $m$
roots modulo $p$.
Then there exist  effectively computable constants~$\varepsilon, n_0>0$ such that if $n>n_0$, 
\[\frac{mn}{p-1}- \frac{\varepsilon \log n}{\log p} \leq v_p(u_n) \leq \frac{mn}{p-1}+ \frac{\varepsilon \log n}{\log p}\]
and where $\varepsilon$ depends only on~$f$.
\end{proposition}
\begin{proof}
There exists an effective constant $\varepsilon_0$,
independent of $p$, such that for all $n\geq 2$
and all $1\leq k \leq n$ we have 
\[ |f(k)|\leq n^{\varepsilon_0} = p^{\varepsilon_0 \log n / \log p}.\]
Fix $n\geq 2$ and define $e_{\max}:=\max\{ v_p(f(k)) : 1\leq k\leq n\}$. 
Then
\begin{equation}
 \label{eq:emax}
e_{\max} \leq \frac{\varepsilon_0 \log n}{\log p}. 
\end{equation}
Since $p$ is a Hensel prime, by Hensel's Lemma, there is a factorisation 
\[  f(x)=(x-\beta_1)^{m_1} \cdots (x-\beta_\ell)^{m_\ell} h(x) 
\pmod{p^{e_{\max}}} \, . \]
where $m=m_1+\cdots+m_\ell$ and $h$ has no zero modulo~$p$.  Then 
\begin{align}
v_p(u_n) =& \sum_{k=1}^n v_p(f(k))\notag \\
           =& \sum_{k=1}^n \sum_{i=1}^\ell m_i \, v_p(k-\beta_i) \notag \\
        =& \sum_{k=1}^n \sum_{i=1}^\ell \sum_{e=1}^{e_{\max}} m_i \, \mathbb{I}\{p^e \mid k - \beta_i\} \notag\\
        =& \sum_{e=1}^{e_{\max}}\sum_{i=1}^\ell
        \sum_{k=1}^n
        m_i \mathbb{I}\{p^e \mid k - \beta_i\}. \label{eq:TAG}
\end{align}

Now for all $1\leq e\leq e^{\max}$ the set $\{ k \in \N : p^e \mid k-\beta_i\}$
is an arithmetic progression with period $p^e$ and so
\begin{align}
\label{eq-m-AP}
		\ \frac{n}{p^e} -1  \leq  \sum_{k=1}^n \mathbb{I}\{p^e \mid k-\beta_i \} \leq \frac{n}{p^e} +1 ,
	\end{align} 
Combining inequality~\eqref{eq-m-AP} with Equation~\eqref{eq:TAG} we obtain
\begin{equation}
 	\label{eq-sum-III}
	m \sum_{e=1}^{e_{\max}} \Big( \frac{n}{p^e} -1 \Big)  \leq  v_p(u_n)  \leq m \sum_{e=1}^{e_{\max}} \Big( \frac{n}{p^e} +1 \Big).
\end{equation}
Let $\varepsilon>\varepsilon_0$ be a positive constant.
The desired result follows by sandwiching the term $\sum_{e=1}^{e_{\max}} \frac{1}{p^e}$ in~\eqref{eq-sum-III} by
\[\frac{1-\frac{p}{f(n)}}{p-1}\leq \frac{1-p^{-e_{\max}}}{p-1}  =\sum_{e=1}^{e_{\max}}  \frac{1}{p^e} \leq \, \frac{1}{p-1}\]
in combination with the upper bound on $e_{\max}$ in~\eqref{eq:emax}.

\end{proof}

\subsection{Asymptotic estimate for the largest prime divisor}
Fix a polynomial $f(x):=x^2+\beta \in\Z[x]$.  We assume that $-\beta$
is not a perfect square, which is equivalent to assuming that $f$
is irreducible.  Let $a,b\in \Q$ be such that $0 \leq a <b$.  Let
$c,d \in \N$.  For all $n\in \N$ we define
\[I(n):=\{k \in \N : an  \leq  k\leq bn\} \cap (c\N+d) \] and
\[F_n:= \prod_{k\in I(n)} f(k).\]

Informally speaking, the following theorem gives effective
super-linear lower bounds on the growth of the function that maps $n$
to the greatest prime divisor of~$F_n$.  The result itself and the
proof are a slight generalisation of~\cite[Theorem 5.1]{everest07}.  The
main difference is that we permit $I(n)$ to be the intersection of an
interval and an arithmetic progression, whereas 
the work cited above considers unrefined intervals~$I(n)=\{1,\ldots,n\}$.


\begin{restatable}{theorem}{theobigprimespolyT} 
\label{theo:bigprimespolyT}
Let $M \in \N$. There exists an effectively computable bound~$B\in \N$
such that for all \(n>B\)  there exists a prime~$p>Mn$ that divides $F_n$.
\end{restatable}
\begin{proof}
Given $n\in \N$, we have the prime factorisation $F_n = \prod_{p} p^{e_p}$ 
where $e_p:= v_p(F_n)$ for each prime $p$.  Note that $e_p=0$ for
all but finitely many $p$.
Taking logarithms, we get
\begin{equation*}
	\log(F_n) =\sum_{p} e_p \log p . 
\end{equation*}
Partitioning the above sum into a sub-sum over primes at most $Mn$ and a sub-sum over primes greater than $Mn$, we obtain
\begin{equation}
\label{eq:logunn}
	\sum_{p >Mn}  e_p \log p = \log(F_n) - 	\sum_{p \leq Mn}   e_p
        \log p .
\end{equation} 

The theorem at hand follows from a lower bound on the sum
$\sum_{p >Mn} e_p \log p$ on the left-hand side of~\eqref{eq:logunn}.
To this end we have two sub-goals: give a lower bound on $\log(F_n)$
and an upper bound on $\sum_{p \leq Mn} e_p \log p$.

Write $A:=\frac{b-a}{c}$.
The following lower bound on $\log(F_n)$ is a consequence of Stirling's formula.
The proof is in Appendix~\ref{sec:app-lowerbownun}.
\begin{claim}
  \label{claim:sizefunnyQ}
  $\log(F_n) \geq 2A  (n\log n - n)$.
\end{claim}

The next task is give an upper bound on $\sum_{p \leq Mn}   e_p \log p$.
Here we follow the approach in~\cite{everest07} and further partition the sum into 
those primes $p<n$ (treated in~\Cref{claim-small-prime-F}) and those primes 
\(n\le p\le Mn\) (treated in~\Cref{claim-average-prime}).

\begin{claim}
\label{claim-small-prime-F}
There exist positive constants \(\varepsilon, n_0>0\) such that if \(n>n_0\), then
\begin{equation*}
  \sum_{p <n}  e_p \log p \leq  An\log n  + \varepsilon n. 
\end{equation*}
\end{claim}
\begin{proof}
Let $S_n$ be the set of primes $p<n$ such that 
$p$ divides $F_n$ and $p$ is a Hensel prime for $f$.
Observe that 
\[ \sum_{p < n}  e_p \log p - \sum_{p\in S_n}
e_p \log p \leq \varepsilon_0 \log n\]
for an effective constant $\varepsilon_0$.
Indeed, if $p<n$ is a prime divisor of ${F_n}$ 
that does not lie $S_n$ then $p$ divides the discriminant 
of $f$---and there are finitely many such primes.
Thus to prove the claim it will suffice to show the following bound for some effective constant $\varepsilon_1$:
\begin{equation} \sum_{p \in S_n}  e_p \log p \leq
  An\log n  + \varepsilon_1  n. 
\label{eq:SUM2}
\end{equation}

For \(p\in S_n\), we establish an upper bound on $e_p$ which follows from
\Cref{prop:moll}:
\begin{equation}
  e_p \leq  \frac{2An}{p-1}+  \frac{\varepsilon_2 \log n}{\log p}.
\label{eq:BOUND}
  \end{equation}
  Here the constant $\varepsilon_2$ is effective
and independent of the prime $p$.
The justification is given in~\Cref{sec:app-lowerbownun}.

We next argue that there exist 
effective constants $\varepsilon_3,\varepsilon_4,n_1>0$  such that the following chain of inequalities is valid 
 for all $n\geq n_1$.  
 We have that
 \begin{eqnarray*}
 	\sum_{p\in S_n} e_p \log p & \leq & \sum_{p \in S_n} \left( \frac{2An }{p-1}+\varepsilon_2 \, \frac{\log n}{\log p} \right) \log p
  \qquad\mbox{(by~\eqref{eq:BOUND})}\\
 	& \leq & \, 2An \sum_{p \in S_n} \frac{\log p }{p-1} + \varepsilon_2 \pi(n) \log n\\
 	&\leq &   2An \sum_{p \in S_n} \frac{\log p }{p-1} + \varepsilon_3 n
                \qquad \mbox{(by \autoref{thm:pnt})}\\
   &=& 2An \sum_{p \in S_n} \frac{\log p }{p} \left(1+\frac{1}{p-1}\right)+ \varepsilon_3 n\\
 	& \leq  & 2An \sum_{p \in S_n} \frac{\log p }{p}+ \varepsilon_4.
  \end{eqnarray*} 

  No prime in $S_n$ divides the discriminant of $f$.  Since the latter
  is equal to $-4\beta$, no prime in $S_n$ divides $\beta$.
  In addition, every prime in $S_n$ is a divisor of $F_n$; i.e., a
  divisor of $k^2+\beta$ for some $k\in I(n)$, we have that $\beta$ is
  a quadratic residue modulo $p$ for every prime $p \in S_n$.
 Thus, for sufficiently large \(n\), we have that
\[ \sum_{p \in S_n} \frac{\log p}{p} \leq \frac{1}{2} \log n + \varepsilon_5\]
(by \Cref{prop:apostol_primes_beta})
for some effective constant $\varepsilon_5$.

The desired bound~\eqref{eq:SUM2} follows by combining the previous two inequalities.
 \end{proof}


\begin{claim}
\label{claim-average-prime}
	There exist positive constants \(n_0,\varepsilon>0\) 
	such that if \(n>n_0\),
then
\begin{equation*} 
\sum_{n\leq p \leq Mn}  e_p \log p \leq \varepsilon n.
\end{equation*}
\end{claim}

\begin{proof}
  Let $n \in \N$.  Suppose that \(p> (b-a)n\) is a prime divisor of
  \(F_n\).  For such primes, we shall first show that
  $e_p := v_p(F_n) \le 2$.  Assume, for a contradiction, that there
  are distinct integers $k_1 < k_2 < k_3$ in $I(n)$ such that $p$
  divides $k_1^2+\beta$, $k_2^2+\beta$, and $k_3^2+\beta$.  Then
  $p \mid k_1^2 - k_2^2$.  
  Since \(p\) is prime, either
  $p \mid k_1-k_2$ or \(p \mid k_1+k_2\).  Since
  \(0< k_2-k_1 < (b-a)n \le p\), we deduce that $p \mid k_1+k_2$.  By
  symmetric reasoning we have that $p \mid k_2+k_3$.  Thus $p$ must
  also divide $(k_2+k_3)-(k_1+k_2) = k_3 - k_1$.  However, this leads
  to a contradiction since $p \geq (b-a)n \geq k_3-k_1$.  Hence for
  each prime divisor \(p\mid F_n\) with \(p\ge (b-a)n\), we find that
  $e_p = v_p(F_n) \le 2$.

Thus we bound the summation in the statement of the claim by
\begin{equation*} 
    \sum_{n< p \leq Mn}  e_p \log p
    \le \sum_{p\le Mn} 2 \log p
    \le 2 \log(Mn) \pi(Mn).
\end{equation*}
The desired result follows from the estimate on \(\pi(x)\) given by the Prime Number Theorem  (\autoref{thm:pnt}). 
\end{proof}


We return to the proof of \autoref{theo:bigprimespolyT}.  
From
Equation~\eqref{eq:logunn}, \cref{claim-small-prime-F}, and \cref{claim-average-prime}, there exist positive constants
\(\varepsilon,n_0>0\) such that if \(n>n_0\) then
\begin{equation*}
\sum_{p > Mn}  e_p \log p \geq An\log n - \varepsilon n.
\end{equation*}

In turn, the above lower bound entails that for sufficiently
large~\(n\), there exist prime divisors \(p \mid F_n\) such that \(p>
Mn\).  This concludes the proof.
\end{proof}





  





