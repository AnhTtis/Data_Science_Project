\paragraph{Hypergeometric Sequences}

A hypergeometric sequence $\langle u_n\rangle_{n=0}^\infty$ is a sequence of rational numbers that satisfies 
a recurrence of the form \eqref{eq:rel}
where $f,g \in \Z[x]$ are polynomials, and  $f(x)$
has no non-negative integer zeros. 
By the latter requirement on~$f(x)$, the  recurrence~\eqref{eq:rel} 
uniquely defines an infinite sequence of rational numbers once the initial element $u_0$ 
is specified.

An instance of the Membership Problem for hypergeometric sequences consists of a recurrence~\eqref{eq:rel}, an initial value 
$u_0 \in \Q$, and a target $t \in \Q$.  
The problem asks to decide whether there exists \(n\in\N\) such that \(u_n = t\).
We say that such an instance is in \emph{standard form} if~(S1) the initial condition is $u_0=1$; (S2)~the polynomial $g(x)$ has no positive integer root; (S3)~the target $t$ is non-zero;
(S4)~the polynomials $f$ and $g$ have the same degree and leading coefficient.

For the purposes of deciding the Membership Problem, we can assume without loss of generality that all instances are in standard form.  An arbitrary instance can be transformed into one satisfying Condition~(S1) by 
multiplying the sequence and target by a suitable constant.  Instances of the  Membership Problem that fail to satisfy 
Conditions~(S2) and (S3) are trivially solvable.  The positive integer roots of~$g$ can be computed and for any such root $n_0$, we have $u_n=0$ for all $n\geq n_0$.
%
Finally,
for recurrences that fail Condition~(S4) we have that \[ \frac{u_n}{u_{n-1}}=\frac{g(n)}{f(n)} \] either converges to $0$ or diverges in absolute value.  Under the assumption that~$t\neq 0$, in each case we can compute an effective threshold $n_0$ such that $u_n\neq t$ for all $n\geq n_0$.

\paragraph{The $p$-adic valuation}
Let $p\in \N$ be a  prime.
 Denote by~$v_p:\Q \to \Z \cup\{\infty\}$
the $p$-adic valuation on~$\Q$. 
Recall that for a  non-zero  number~$x\in \Q$, 
$v_p(x)$ is the unique integer such that~$x$ can be written in the form
\[x=p^{v_p(x)}\; \frac{a}{b}\]
where $a,b\in \Z$ and $p$ divides neither $a$ nor $b$.
The value $v_p(0)$ is defined to be $\infty$. 
The valuation possesses two important properties:
\begin{enumerate}
	\item[-]$v_p(x+y)\geq \min\{v_p(x),v_p(y)\}$ \, (\emph{strong triangle inequality}),
	\item[-]$v_p(xy)=v_p(x)+v_p(y)$ \, (\emph{multiplicative property}).
\end{enumerate}



\paragraph{Asymptotic estimates for series over primes}
Given  ${\sim} \in {\{<,=,> \}}$ and $x\in \Q$,
we denote sums over primes \(p\in\N\) such that 
\(p \sim x\) by \(\sum_{p \sim x}\).
Let \(\pi(x) := \sum_{p\le x} 1\) count the number of primes of size at most~\(x\).
The following result is a consequence of the Prime Number Theorem.
    \begin{theorem}\label{thm:pnt} Let \(\pi(x)\) count the number of primes of size at most \(x\), then  
            \begin{equation*}
               \pi(x) = \frac{x}{\log x} + O\Bigl(\frac{x}{\log^2 x}\Bigr).
            \end{equation*}
    \end{theorem}

As an aside, an element \(a\in\Z\) is a \emph{square} modulo a  prime \(p\in \N\) if there exists an \(x\in\Z\) such that  \(x^2 \equiv a \pmod{p}\). 
An element \(a\in\Z\) is a \emph{quadratic residue} modulo  \(p\) if \(a\) is both a square modulo \(p\), and furthermore \(a\) and \(p\) are co-prime.
We denote by \(\mathcal{L}_p\) the set of quadratic residues modulo \(p\).



Recall the first of Mertens' three theorems \cite{mertens1874zahlentheorie}  (see also \cite[Theorem~4.10]{apostol1998introduction}),
\[
 \sum_{p \leq x } \frac{\log p}{p} =  \log x + O(1) \,.
 \] 
In the sequel we shall make use of the following refinement of Mertens' theorem.
\begin{proposition} \label{prop:apostol_primes_beta}
Suppose that \(a \in \Z\) is not a  perfect square. 
Then
\begin{equation*}
\sum_{p\leq x, \, a \in \mathcal{L}_p}  \frac{\log p}{p} =  \frac{1}{2}\log(x)+O(1).
\end{equation*}
\end{proposition}
\autoref{prop:apostol_primes_beta} appears in work by Selberg  \cite[Equation (3.3)]{selberg1950pnt-ap}
on an elementary proof of Dirichlet's theorem in arithmetic progressions.

