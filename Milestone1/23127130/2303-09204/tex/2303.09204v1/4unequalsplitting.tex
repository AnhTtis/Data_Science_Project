In this section we show decidability of the Membership Problem for recurrence sequences that satisfy a first-order relation of the form \eqref{eq:rel} subject to the condition that the polynomial coefficients 
$f,g\in \Z[x]$ have different splitting fields.
To this end, it is useful to introduce the following terminology.
Let $p$ be a Hensel prime for~$fg$.
We say that the recurrence~\eqref{eq:rel} is 
\emph{$p$-symmetric} if 
the two polynomials $f$ and~$g$ have the same number of roots 
in $\Z/p\Z$.  Otherwise we say that the recurrence is 
\emph{$p$-asymmetric}.

We first show decidability of the Membership Problem in the case
of $p$-asymmetric recurrences and then we apply the Chebotarev Density Theorem to show that every recurrence in which $f$ and $g$ 
have different splitting fields is $p$-asymmetric for infinitely many primes $p$.

\begin{lemma}\label{lem-p-assym}
There is a procedure to decide the Membership Problem 
for the class of hypergeometric sequences whose defining recurrences 
are $p$-asymmetric for some prime $p$.
\end{lemma}
\begin{proof}
Suppose that the hypergeometric sequence 
$\langle u_n \rangle_{n=0}^\infty$ satisfies 
the recurrence~\eqref{eq:rel} and moreover that
there is a prime 
$p$ with respect to which the recurrence is $p$-asymmetric.
We want to decide whether such a sequence reaches a given target value $t$.

Consider the sequences $\langle x_n \rangle_{n=0}^\infty$ and
$\langle y_n \rangle_{n=0}^\infty$ respectively defined by the monic
recurrences $x_n=f(n)x_{n-1}$, $y_n=g(n)y_{n-1}$, with $x_0=y_0=1$.
Then $u_n = \frac{x_n}{y_n}$ and hence, for the aforementioned prime \(p\),
\[v_p(u_n)=v_p(x_n)-v_p(y_n) = \sum_{\ell=1}^n (v_p(f(\ell)) - v_p(g(\ell)))\]
by the multiplicative property.

Recall that $p$ is, by definition, a Hensel prime for both $f$ and $g$.
Here,
by \Cref{prop:moll}, we obtain an asymptotic estimate of the form
\[|v_p(x_n)-v_p(y_n)| = \frac{|m_f-m_g|n}{p-1}  + O(\log n)\] 
where \(m_f\) is the number of roots of \(f\) modulo \(p\) and \(m_g\) is defined similarly.
Here the implied constant depends on \(fg\) and \(p\).
The proof concludes by noting that $v_p(t)$ is a constant, whereas \(v_p(u_n)\) is bounded away from \(v_p(t)\) for sufficiently large \(n\) (note this threshold is computable).
We deduce that \(u_n\neq t\), again, for sufficiently large \(n\), from which the desired result follows.
\end{proof}

We now give a sufficient condition for a recurrence to be 
$p$-asymmetric.  We use the following 
consequence of the  Chebotarev Density Theorem.  
Let $\K$ be a Galois field of degree $d$ over $\Q$, and denote by $\OO$ its ring of integers. 
Let  $\Spl(\K)$ be the set of rational primes~$p$
such that  the ideal $p\OO$ 
totally splits in $\OO$, i.e., such that 
\(p\OO = \mathfrak{p}_1 \cdots \mathfrak{p}_d\)
where the $\mathfrak{p}_i$ are distinct prime ideals.
The following result appears as \cite[Corollary 8.39]{milneANT} and \cite[Corollary 13.10]{neukirch1999algebraic}.
	The latter reference attributes the result to Bauer.
\begin{theorem}
\label{theo-split-prime}
Let $\K$ and $\LL$ be  Galois extensions of~$\Q$ such that
$\K \neq \LL$. Then $\Spl(\K)$ and $\Spl(\LL)$ differ in infinitely many primes. 
\end{theorem}

We state the main theorem of this section.
\begin{restatable}{theorem}{theodistinctsplit} 
\label{theo-distinctsplit}
	There is a procedure to decide the Membership Problem 
for the class of hypergeometric recurrences~\eqref{eq:rel} whose polynomial coefficients have different splitting fields.
\end{restatable}
\begin{proof}[Proof of \autoref{theo-distinctsplit}]
Let $\langle u_n\rangle_{n=0}^\infty$ satisfy a recurrence~\eqref{eq:rel} for which the coefficients~$f$ and $g$ have respective splitting fields $\K$ and $\LL$, with $\K\neq \LL$.
Recall that there are only finitely many primes that are not Hensel
primes for~$fg$. 
By Theorem~\ref{theo-split-prime}, there exists a Hensel prime for $fg$ that lies in exacly one of the two sets 
$\Spl(\K)$ and $\Spl(\LL)$.
For such a prime $p$, the recurrence~\eqref{eq:rel} is $p$-asymmetric.  Hence the result follows 
from~\Cref{lem-p-assym}.
\end{proof}

We note that the recurrence~\eqref{eq:rel} can 
can be $p$-asymmetric even when $f$ and $g$ have the same splitting field.
We demonstrate this phenomenon with the following example.
\begin{example}
Consider following choice of coefficients $f$ and $g$ in~\eqref{eq:rel}:
	\begin{equation*}
 f(x) :=  (x^2+1)(x^2-2) \quad \text{ and } \quad
 g(x) :=  x^4-2x^2+9.
 \end{equation*}
     Note that $\Q(\sqrt{2},\iu)$ is the splitting field of both 
     $f$ and $g$.
	   It is straightforward to verify $7$ is a Hensel prime for~$fg$ by checking that $7$ does not divide  the discriminants of the respective irreducible factors of $f$ and $g$.
	
	We now show that the recurrence~\eqref{eq:rel} is $7$-asymmetric in this case. 
	 Indeed $x^2-2$ factors as~$(x+4)(x-4) \pmod{7}$  over $\Z/7\Z$, whereas $x^2+1$ remains irreducible. 
By comparison, polynomial \(g\), the minimal polynomial of $\sqrt{2}+\iu$, has no roots over $\Z/7\Z$.
  Indeed, \(g\) factors into a pair of irreducible quadratics over $\Z/7\Z$.
\end{example}   

