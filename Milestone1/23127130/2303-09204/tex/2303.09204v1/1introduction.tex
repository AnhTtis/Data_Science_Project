
\paragraph{Background and Motivation}
Recursively defined sequences are ubiquitous in mathematics and computer science.  A fundamental open problem in this context is the decidability of the \emph{Membership Problem}, which asks to determine whether a given value is an element of a given sequence.  The Skolem Problem for \emph{C-finite} sequences (those sequences that satisfy a linear recurrence relation with constant coefficients) is the best known variant of the Membership Problem.  The Skolem Problem asks to determine whether a given C-finite sequence vanishes at some index~\cite{everest2003recurrence}.  Decidability of this problem is known for recurrences of order at most four \cite{mignotte1984distance,vereshchagin1985occurence} but is open in general. Proving decidability of the Skolem Problem would be equivalent to giving an effective proof of the celebrated Skolem--Mahler--Lech Theorem, which states that every non-degenerate C-finite sequence that is not identically zero has a finite set of zeros.

In this paper we consider the most basic case of the Membership Problem for a class of P-finite sequences (those sequences that satisfy a linear recurrence with polynomial coefficients).  Specifically, we consider the Membership Problem for the class of hypergeometric sequences.  A rational-valued sequence $\langle u_n \rangle_{n=0}^\infty$ is \emph{hypergeometric} if it satisfies a recurrence relation of the form
 \begin{equation}\label{eq:rel}
 f(n)u_{n} - g(n)u_{n-1} = 0 \, ,
\end{equation} 
 where $f,g \in \Z[x]$ are polynomials, and  $f(x)$
 has no non-negative integer zeros.   
 By the latter assumption on~$f(x)$, the  recurrence relation~\eqref{eq:rel} 
 uniquely defines an infinite sequence of rational numbers once the initial value $u_0\in\Q$ is specified.


Formally, the Membership Problem asks, given a recurrence~\eqref{eq:rel}, initial value $u_0\in \Q$, and target $t\in \Q$, whether $t$ lies in the sequence $\langle u_n \rangle_{n=0}^\infty$.  At first glance, this problem may seem easy to decide.  Without loss of generality we can assume that the sequence $\langle u_n \rangle_{n=0}^\infty$ either diverges to infinity or converges ultimately monotonously to a finite limit.  
If the sequence does not converge to~$t$
then one can compute a bound $B$ such that $u_n \neq t$ for all $n>B$.  Such a bound can also be computed in case one knows that $\langle u_n \rangle_{n=0}^\infty$ converges to $t$ (by straightforward arguments about the monotonicity of the convergence).  The challenge is to distinguish the two cases above.  The ability to do this is related to conjectures concerning the gamma function (see the discussion below).

As an aside,
the term 
 \emph{hypergeometric} was introduced by John Wallis in the 17th century~\cite{wallis1655arithmetica}. 
 Hypergeometric sequences and their 
 associated generating functions, the hypergeometric series, 
 have a long and illustrious history in the mathematics literature.  In particular, hypergeometric series encompass many of the common mathematical functions
 and have numerous applications in analytic combinatorics and the theory of generating functions~\cite{Concrete11,FS09}.

\paragraph{Contributions}
In this paper we approach the Membership Problem by considering the prime divisors of the values of a hypergeometric sequence
$\langle u_n \rangle_{n=0}^\infty$.  The overall strategy is to exhibit an effective threshold $B$ such that for all $n>B$ there a prime divisor of $u_n$ that is not a divisor of the target $t$.  Our two main contributions are as follows:
%
\begin{itemize}
    \item The Membership Problem for hypergeometric sequences whose polynomial coefficients (as in \eqref{eq:rel}) have distinct splitting fields is decidable (\autoref{theo-distinctsplit}).
    \item The Membership Problem for hypergeometric sequences whose polynomial coefficients are monic and split over a quadratic field is decidable (\autoref{theo-decide-quad}).
\end{itemize}

The proofs of our main results involve two different implementations of our general strategy.  The proof of \Cref{theo-distinctsplit} applies the Chebotarev density theorem to find a single prime $p\in \Z$ that does not divide the target $t$ but divides all members of an infinite tail of the sequence.  Meanwhile, the proof of \Cref{theo-decide-quad} shows that for all sufficiently large $n$ there exists a prime $p$, that is allowed to depend on $n$, such that $p$ divides $u_n$ but not $t$.  To find such a prime we rely on (a mild generalisation of) a result of~\cite{everest07} concerning prime divisors of the values of a quadratic polynomial.

\Cref{theo-distinctsplit} expands the class of sequences for which the Membership Problem is ``generically easy'' and, further, isolates its hard instances.  The paper~\cite{NPSW022} handles perhaps the easiest sub-case of the Membership Problem that does not fall under~\Cref{theo-distinctsplit}, namely when the polynomial coefficients both split over~$\Q$.  The second main result of the present paper handles another naturally occurring sub-case: when the polynomial coefficients split over the ring of integers of a quadratic field~$\K$.  A common refinement of these two cases---that the polynomial coefficients split over $\K$---is the subject of current research.  Generalisations of the  results of~\cite{everest07} to higher-degree polynomials are a subject of ongoing research in number theory and potentially would allow us to extend our approach beyond the quadratic case.
 
\paragraph{Related Work}
There is a growing body of work that addresses membership and threshold problems for sequences satisfying low-order polynomial recurrences.  Here the \emph{Threshold Problem} asks to determine whether every term in a sequence lies above a given threshold, for example, whether every term is non-negative.  

The recent preprint~\cite{kenison2022applications} 
establishes decidability results (some conditional on Schanuel's Conjecture) for both the Membership and Threshold Problems for hypergeometric sequences.
The approach of~\cite{kenison2022applications}
relies on transcendence theory for the gamma function (as well as underlying properties of modular functions established by Nesterenko~\cite{nesterenko1996modular}).
By contrast, the algebraic techniques of the present paper seem appropriate only for the Membership Problem.  
We note that the approach of~\cite{kenison2022applications} requires certain restrictions, e.g., decidability is only unconditional when the parameters are drawn from imaginary quadratic fields.



The problem of deciding positivity of order-two P-finite sequences and of deciding the existence of zeros in such sequences is considered in~\cite{KauersP10,KenisonKLLMOW021,NeumannO021,PillweinS15}. These works all place syntactic restrictions on the degrees of the polynomial coefficients involved in the recurrences, and all four give algorithms that are not guaranteed to terminate for all initial values of a given recurrence.  For example, in~\cite{KauersP10}
the termination proof of the algorithm for determining positivity of order-two sequences requires 
that the characteristic roots of the recurrence be distinct and that one is working with a generic solution of the recurrence (in which the asymptotic rate of growth corresponds to the dominant characteristic root of the recurrence).   
Simple manipulations show that the Membership Problem considered in this paper is equivalent to the problem of finding a zero term in an order-two P-finite sequence~$\langle u_n \rangle_{n=0}^\infty$ arising as a sum of two hypergeometric sequences.

Links between the Membership and Threshold Problems and the Rohrlich--Lang Conjecture appear in previous works \cite{NPSW022,kenison2020positivity}.  Here the Rohrlich--Lang Conjecture concerns multiplicative relations for the gamma function evaluated at rational points.


The $p$-adic techniques used in the present paper bear many similarities with 
work on developing criteria for hypergeometric sequences to be integer valued.
For example, work by Landau in 1900 \cite{landau1900factorielles} uses \(p\)-adic analysis to establish a necessary and sufficient condition for integrality in the so-called class of \emph{factorial} hypergeometric sequences.
In more recent work, Hong and Wang \cite{hongarxiv2016} establish a criterion for the integrality of hypergeometric series with parameters from quadratic fields.
We observe that some of the intermediate asymptotic results in Hong and Wang's note are close to \cite[Corollary~3.1]{Moll09}
(\autoref{prop:moll} herein).

\paragraph{Structure}

The remainder of this paper is structured as follows.
We briefly review preliminary material in \cref{sec-pre}, including some standard assumptions about instances of the Membership Problem that can be made without loss of generality.
In \cref{sec-polyseq}, we recall useful technical results on the prime divisors of hypergeometric sequences that satisfy monic recurrence relations (see \eqref{eq:poly}).
In \cref{sec-unequalsplitting}, we prove \cref{theo-distinctsplit}.
The proof of \cref{theo-decide-quad} is given in \cref{sec-quad}.
We discuss ideas for future research in \cref{sec:discussion}.
The remaining appendices prove technical results omitted from the main text.
