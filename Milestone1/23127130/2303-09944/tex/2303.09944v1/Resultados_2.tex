
\section{Results}
\label{Results}




Our model consists of a $S=1/2$ impurity that is weakly connected to two electrodes under a finite DC bias and a CW drive.  
Our aim is to explore the behavior of the ESR signal as the DC voltage is varied for a set of parameters intended to mimic conditions found in ESR-STM experiments. 
\subsection{Model parameters}
\label{parameters}

The model parameters are chosen under the proviso of obtaining a strong ESR signal of a $S=1/2$ system weakly connected to two electrodes under electrical driving. To achieve this, we need:

\begin{enumerate}
    \item an imbalance in the transport-electron spin in order to make the main rates different from zero. This is achieved by having different spin-polarization of the electrodes.
    
    \item a predominant long-time average population of one electron in the impurity, otherwise the system does not behave like a $S=1/2$. 
    
    \item an electronic level, $\varepsilon$, within the DC-bias range. 
    
    \item to flip the transport spin using a magnetic field transversal to the electron spin polarization.
    
    \item a modulation of the tunneling hopping with the spin-polarized electrode by the oscillating electric field. 

    \item low temperature. We take 1 K for both electrodes.
\end{enumerate}

In our calculations, we achieve the above conditions with the following parameters: 1. The  left electrode has a polarization of $P_L= 0.45$ in Eq. (\ref{Pol}). Increasing the polarization up to 100\%  will increase the ESR signal amplitude. 2. To stabilize the charge state, we apply different couplings with $\gamma_{R}=20\times\gamma_{L}=5\ \mu$eV. This coupling asymmetry is often found in experiments, where the impurity couples more strongly to the substrate than the STM tip. The DC bias drop is $eV_{DC}=\mu_L-\mu_R$. We use the model of a double-barrier tunnel unction \cite{Tu_X_double-barrier} and assume an asymmetric DC bias drop where $\mu_L=(1-\eta)eV_{DC}$ and $\mu_R=-\eta eV_{DC}$ with the factor $\eta=\gamma_L/(\gamma_L+\gamma_R)=1/21$. This means that the bias drop takes places mostly on the left electrode. 3. The energy of our model is set by $\varepsilon=-10$ meV. In addition, the electronic states are assumed to have an intrinsic width of $\hbar/\tau_c=10 \;\mu$eV. 
%which helps with the convergence of the impurity's free Green's function.
In order to explore the interplay of the many-body states in ESR processes, we take a fixed charging energy close to the electronic level energy, of $U=3|\varepsilon|/2=15$ meV. 4. In order to flip the spin, defined along the $z$-axis of the spin polarization, $P_L$, we apply a $B$-field component along the $x$-axis perpendicular to the $z$-axis component. The magnetic field is taken as $\mathbf{B}=(0.6, 0, 0.1)$ T, which gives a Larmor frequency of approximately 17 GHz. The largest ESR signal takes place for a magnetic field completely aligned with the $x$-axis in good agreement with experiments~\cite{Rotation}. 5. The modulation of the tunneling matrix element is $A_L=50\%$, Eq. (\ref{hopping}) and applied only to the left electrode, which is the polarized one. Since the right electrode is not spin polarized, $A_R$ does not contribute to the resonance, but only to the background current.



%The impurity energy scheme is shown in Fig. \ref{Energy_Sch} (b). In this figure, we assume that the $|p\rangle$ basis set in Eq. \ref{eq:himp} is similar to the eigenstate basis set, $|l\rangle$ in Eq. \ref{l-basis}, although the magnetic field mixes the spin-up and down states. 

\subsection{Non-zero rates: the opening of transport channels with applied bias}

A transport channel opens when the corresponding rates, Eq. (\ref{rate}) are different from zero. Inspection of Eq. (\ref{rateT}) shows that this occurs when two conditions are met: The first one is energy conservation, largely controlled by the Fermi factors. The energy conservation implies that the change of state has to be compensated by the applied bias. Under our present conditions, the bias drop takes place largely at the left electrode, then $\Delta_{v,l}=E_v-E_l$ has to be larger than $\mu_L=eV_L=(1-\eta)eV_{DC}$. This is due to the appearance of a term $f(\Delta_{v,l})$ in Eq. (\ref{rateT}) when $1/\tau_c\rightarrow 0^+$. 
The second condition is that the sequential transport process leads to a change in the charge state of the impurity such that  $\lambda_{vl\sigma}\neq 0$ when $v$ and $l$ differ in one electron of spin $\sigma$. Then, the difference in energy $\Delta_{v,l}$ in the rates Eq. (\ref{rate}) always addresses states differing by one electron.



\subsection{DC-bias dependence of the ESR signal}

The DC-bias will determine when the transport channels of the system opens. But the occurrence of ESR further depends on the possibility of a spin-flip process. For this, the transport channel must be compatible with spin-flip processes. 



First, we study the dependence of the magnitude and sign of the ESR signal $\Delta I$ as function of the  magnitude and sign of the applied DC bias. Figure~\ref{IvsV} (a) and (b) show two representative spectra taken at opposite signs of the DC bias. The difference between both spectra is more than a change of sign. 
%Interestingly, our model exhibits a strong asymmetry of the ESR signal when changing the sign of the bias.
To better understand this behavior, Fig.~\ref{IvsV} (c) shows the ESR peak intensity as function of $V_{\rm{DC}}$. Take, for example, a positive bias where we obtain a large negative value of the ESR signal. This correlates with a large contribution of the coherence-term $\rho_{\uparrow \downarrow}$ between spin up and down (Fig.~\ref{IvsV} (d)). We emphasize that this occurs in the long-time limit under substantial decoherence of the system as long as the drive sustains the coherences. The connection between ESR signal and coherences of the density matrix can be understood by studying the behavior of the electronic current, Eq. (\ref{current_t}). 


When the applied bias is positive ($\mu_L-\mu_R>0$), spin-polarized electrons flow from the left electrode into the impurity. A negative ion is formed if $\mu_L> \Delta_{2,\downarrow}=E_2-E_\downarrow\approx U+\epsilon$ (we have neglected the Zeeman energy), that corresponds to a transition from a singly-occupied level (with spin down, $u=\downarrow$) to a doubly charged level ($v=2$). At the same time, we need that $\mu_R<\Delta_{2,\downarrow}$, as is the case at positive bias. Similarly, the formation of the positively-charge ion is energetically possible. However, there is an important asymmetry due to the very different couplings between impurity and electrodes ($\gamma_L\ll\gamma_R$) as well as in the bias drop. As a consequence the formation of the negative ion is favored over the positive one for this present case.



\begin{figure}
\centering 
\includegraphics[width=1.1\linewidth]{I_and_rho12.png}
\caption{ a) and b) ESR signal  $\Delta I(f)=I(f)-I_{BG}$ as function of relative frequency $\delta=f-f_0$ for two different signs of the DC bias. in a) the DC bias is negative and in b) the DC bias is positive which inverts the ESR amplitude. For this system the Larmor frequency is $f_0=17.025$ GHz which is the natural resonance frequency of the Hamiltonian plus the re-normalization imposed by the Lamb shift. c) ESR signal and d) real part of the coherence $\rho_{\uparrow \downarrow}$ between spin up and down as a function of DC bias when on resonance ($\delta=0$). The transport channels are closed for $V_{DC}\lesssim U+\varepsilon$ (neglecting the Zeeman energy) and $V_{DC}\geq \varepsilon$. In this work we took $U=3 |\varepsilon|/2$,
so the ESR signal is zero between $V_{DC}/|\varepsilon|\lesssim 0.5$ and $V_{DC}/|\varepsilon|\geq-1$. The behavior of the ESR signal reflects the behavior of the coherences except for a sign.} 
\label{IvsV}
\end{figure}
Then, we can simplify the expression for the electron current, Eq. (\ref{current_t}), by neglecting the involvement of the positive ion, and only considering the negative ion as the intermediate step in the electron transfer between electrodes through the impurity:
\begin{eqnarray}
I(\omega)&=&
\frac{2e}{\hbar} \mbox{Re}  \bigg\{\rho_{\downarrow}(\omega) \Gamma_{\downarrow 2,2\downarrow,L;0}^{-} + \rho_{\uparrow}(\omega) \Gamma_{\uparrow 2,2\uparrow,L;0}^{-} + \nonumber \\
%\cancelto{0}{\rho_{\emptyset}(\omega)} \sum_{j=\uparrow,\downarrow}\Gamma_{\emptyset j,j\emptyset,L;0}^{-}
&&  \rho_{\downarrow,\uparrow}(\omega) \Gamma_{\downarrow 2,2\uparrow,L;-1}^{-}+\rho_{\uparrow,\downarrow}(\omega) \Gamma_{\uparrow 2,2\downarrow,L;1}^{-}
 \bigg\},
 \label{current_posV}
\end{eqnarray}
where, for instance, $\Gamma_{\downarrow 2,2\downarrow,L;0}^{-}$ is the electron rate for a process that involves a non-spin-flip transition (spin-up state) through the doubly-occupied one by exchanging an electron with the left electrode, Floquet index $n=0$. At the same time, $\rho_{\downarrow}=\rho_{\downarrow\downarrow,0}$ while $\rho_{\downarrow\uparrow}=\rho_{\downarrow\uparrow,1}$ and $\rho_{\uparrow\downarrow}=\rho_{\uparrow\downarrow,-1}$ where $-1,0,1$ are Floquet indices.


At a large-enough bias, all channels are open giving a background current, $I_{BG}$:
\[
I_{BG}=
\frac{2e}{\hbar} \mbox{Re}  \bigg\{\rho_{\downarrow}(\omega) \Gamma_{\downarrow 2,2\downarrow,L;0}^{-} + \rho_{\uparrow}(\omega) \Gamma_{\uparrow 2,2\uparrow,L;0}^{-} \bigg\},
\]
which recovers the usual expression for the current for very asymmetrical couplings~\cite{meir_wingreen_prl_1992}. The background current shows a small frequency dependence as it is largely given by the rates with Floquet index $n=0$. Indeed, there is no coherence in the density matrix when the driving frequency is different from the Larmor frequency (off resonance) and $I(\omega)=I_{BG}$. 

Only on resonance, is the coherence $\rho_{\downarrow,\uparrow}(\omega)$ different from zero. Then, there is a clear frequency-dependent contribution to the current at the Larmor frequency that originates in the coherences of the density matrix. Accordingly, the coherences contribution to the DC current depends on the Floquet indices $n=\pm 1$.

Increasing the value of the charging energy, $U$, moves the doubly-occupied state energy ($E_{2}=2\epsilon+U$). For $U\rightarrow+\infty$, it becomes impossible to open the channel connecting the single-electron states with the doubly-occupied one. As a consequence, the ESR signal completely disappears for positive bias.






At negative bias, $\mu_L<\Delta_{\downarrow,\emptyset}=-10$ meV marks the threshold for having a current, where $v=\emptyset$ corresponds to the positively charged impurity. As in the discussion above, we have neglected the Zeeman energy. The ESR signal also follows the behavior of $-\rho_{\uparrow \downarrow}$ as above, Fig. \ref{IvsV}. 

The intermediate state mediating the transport process at negative bias is the one corresponding to the positive ion, $v=\emptyset$. Then Eq. (\ref{current_t}) can be simplified by taking the positive ion contribution:
\begin{eqnarray}
I(\omega)&=&
-\frac{2e}{\hbar} \mbox{Re}  \bigg\{\rho_{\downarrow}(\omega) \Gamma_{\downarrow \emptyset,\emptyset\downarrow,L;0}^{+} + \rho_{\uparrow}(\omega) \Gamma_{\uparrow \emptyset,\emptyset\uparrow,L;0}^{+} + \nonumber \\
&&  \rho_{\downarrow,\uparrow}(\omega) \Gamma_{\downarrow \emptyset,\emptyset\uparrow,L;-1}^{+}+\rho_{\uparrow,\downarrow}(\omega) \Gamma_{\uparrow \emptyset,\emptyset\downarrow,L;1}^{+}
 \bigg\},
 \label{current_negV}
\end{eqnarray}
where again, the ESR signal originates in the coherences of the density matrix. Contrary to the positive-bias case, the limit $U\rightarrow\infty$ does not alter the results since the doubly-occupied level is not involved.




The presence of a finite charging energy then leads to breaking the electron-hole symmetry. At $U\rightarrow\infty$ the electron-hole asymmetry becomes the largest, with no ESR signal for positive bias and a large signal for negative bias at the bias threshold marked by the impurity level.


\subsection{ESR-STM linewidths}


Figure~\ref{pos} shows four characteristic CW ESR-STM signals as a function of the frequency of the drive, $f=\omega/2\pi$, for positive DC bias. At threshold, $V_{DC} \approx U+\varepsilon\approx 5$ mV, a strongly asymmetric Fano profile is obtained. This behavior can be traced back to the interference between the on-resonance scattering with the background. As the bias is further reduced, the transmission channel is increasingly closed, leading to a smaller background current and a smaller signal. In this regime, the ESR signal also depends on the change of the populations, in stark contrast to the open channel case, where the ESR signal is basically determined by the coherences.
%This close-channel region is of great interest because it leads to an enhanced coherence, and will be of physical interest in so far as higher-order transport processes such as cotunneling, are not dominant.

This closed-channel region is of practical importance because here the system exhibits an enhanced coherence time.
The present treatment of this regime is valid as long as higher-order transport processes such as cotunneling are not dominating.

\begin{figure}
\centering 
\includegraphics[width=1.05\linewidth]{I_vs_f_posV.png}
\caption{DC current as a function of the driving frequency$\delta=f-f_0$, for the four different positive voltages a) $V_{DC}=0$, b) $V_{DC}=2.5$ mV, c) $V_{DC}=5$ mV and d) $V_{DC}=7.5$ mV. The background current was not removed. The current changes in a small interval about the resonance frequency. For DC bias below the threshold (at 5 mV here) the DC current drops dramatically as the channel closes and the line shape as a function of frequency becomes increasingly asymmetric. Moreover, the width of the resonance also increases with the DC bias, leading to smaller $T_2$ times as the decoherence is enhanced. The more asymmetric Fano profiles are found near the transport-channel thresholds.} 
\label{pos}
\end{figure}