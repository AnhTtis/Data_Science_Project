\documentclass[conference]{IEEEtran}
\IEEEoverridecommandlockouts
\usepackage{cite}
\usepackage{amsmath,amssymb,amsfonts}
\usepackage{algorithmic}
\usepackage{graphicx}
\usepackage{textcomp}
\usepackage{xcolor}
\usepackage{hyperref}
\usepackage{amsthm}
\usepackage{adjustbox}
\usepackage{array, multirow}
\usepackage{booktabs}
\usepackage{bbm}
\usepackage[]{todonotes}
\usepackage{soul}
\usepackage{cite}
\usepackage{balance}
\usepackage{textcomp}
\ifCLASSINFOpdf
\else
\usepackage[dvips]{graphicx}
\fi
\usepackage{algorithm}      
\usepackage{algorithmic} 
\let\oldemptyset\emptyset
\let\emptyset\varnothing
\usepackage{varwidth}
\usepackage{amsthm}

\theoremstyle{definition}
\newtheorem{rem}{Remark}

\begin{document}
\title{Control-aware Communication for Cooperative Adaptive Cruise Control}
\author{Mahdi Razzaghpour, Rodolfo Valiente, Mahdi Zaman, Yaser P. Fallah
\thanks{ Connected \& Autonomous Vehicle Research Lab (CAVREL), University of Central Florida, Orlando, FL, USA. \tt\small {razzaghpour.mahdi@knights.ucf.edu}}
\thanks{This research was supported by the National Science Foundation under grant number CNS-1932037.}}
\maketitle

% %%%%%%%%%%%%%%%%%%%%%%%%%%%%%%%%%%%%%%%%%%%%%%%%%%%%%%%%%%%%%%%%%%%%%%%%%%%%%%%%%%%%%%%%%%%%%%%%%%%
% %%%%%%%%%%%%%%%%%%%%%%%%%%%%%%%%%%%%%%%%%%%%%%%%%%%%%%%%%%%%%%%%%%%%%%%%%%%%%%%%%%%%%%%%%%%%%%%%%%%
\begin{abstract}
% %%%%%%%%%%%%%%%%%%%%%%%%%%%%%%%%%%%%%%%%%%%%%%%%%%%%%%%%%%%%%%%%%%%%%%%%%%%%%%%%%%%%%%%%%%%%%%%%%%%
% %%%%%%%%%%%%%%%%%%%%%%%%%%%%%%%%%%%%%%%%%%%%%%%%%%%%%%%%%%%%%%%%%%%%%%%%%%%%%%%%%%%%%%%%%%%%%%%%%%%
Utilizing vehicle-to-everything (V2X) communication technologies, vehicle platooning systems are expected to realize a new paradigm of cooperative driving with higher levels of traffic safety and efficiency. Connected and Autonomous Vehicles (CAVs) need to have proper awareness of the traffic context. However, as the quantity of interconnected entities grows, the expense of communication will become a significant factor. As a result, the cooperative platoon's performance will be influenced by the communication strategy. While maintaining desired levels of performance, periodic communication can be relaxed to more flexible aperiodic or event-triggered implementations. In this paper, we propose a control-aware communication solution for vehicle platoons. The method uses a fully distributed control-aware communication strategy, attempting to decrease the usage of communication resources while still preserving the desired closed-loop performance characteristics. We then leverage Model-Based Communication (MBC) to improve cooperative vehicle perception in non-ideal communication and propose a solution that combines control-aware communication with MBC for cooperative control of vehicle platoons. Our approach achieves a significant reduction in the average communication rate ($47\%$) while only slightly reducing control performance (e.g., less than $1\%$ speed deviation). Through extensive simulations, we demonstrate the benefits of combined control-aware communication with MBC for cooperative control of vehicle platoons.
\end{abstract}
\begin{IEEEkeywords}
Cooperative Driving, Distributed Event-triggered Communication, Model-based Communication, Multi-Agent Systems, Platooning
\end{IEEEkeywords}

% %%%%%%%%%%%%%%%%%%%%%%%%%%%%%%%%%%%%%%%%%%%%%%%%%%%%%%%%%%%%%%%%%%%%%%%%%%%%%%%%%%%%%%%%%%%%%%%%%%%
% %%%%%%%%%%%%%%%%%%%%%%%%%%%%%%%%%%%%%%%%%%%%%%%%%%%%%%%%%%%%%%%%%%%%%%%%%%%%%%%%%%%%%%%%%%%%%%%%%%%
\section{Introduction} \label{sec::intro}
% %%%%%%%%%%%%%%%%%%%%%%%%%%%%%%%%%%%%%%%%%%%%%%%%%%%%%%%%%%%%%%%%%%%%%%%%%%%%%%%%%%%%%%%%%%%%%%%%%%%
% %%%%%%%%%%%%%%%%%%%%%%%%%%%%%%%%%%%%%%%%%%%%%%%%%%%%%%%%%%%%%%%%%%%%%%%%%%%%%%%%%%%%%%%%%%%%%%%%%%%
\noindent Cooperative Adaptive Cruise Control (CACC), which relies on Vehicle to Vehicle (V2V) communication, offers the potential to enhance traffic flow dynamics by ensuring string stability and allowing for closer inter-vehicle distances\cite{double_throughput,Safety_ECC}. The backbone of distributed Multi-Agent Systems (MASs) is information exchange to create situational awareness. Overuse of communication resources may lead to communication congestion, resulting in long latency, increased packet loss, and reduced throughput, all of which will inevitably degrade system stability, performance, and reliability\cite{Information_Dissemination,MJLS_Syscon}. Therefore, a crucial consideration when developing an appropriate distributed control system for MASs is to ensure not only the desired control performance but also the ability to conserve limited communication and computation resources.

%%%%%%%%%%%%%%%%%%%%%%%%%
\begin{figure}
    \centering
    \includegraphics[width=\linewidth,trim={0mm 5mm 0mm 0mm},clip]{Figures/Picture2.jpg}
    \vspace{-0.2in}
    \caption{A description of the communication topology, with the dashed lines indicating the flow of information among vehicles. The distance between the $n^{th}$ vehicle and its preceding vehicle is denoted as $d_i$.}
    \label{fig:diagram}
    \vspace{-0.2in}
\end{figure}
%%%%%%%%%%%%%%%%%%%%%%%%%

Model-Based Communication (MBC) is a relatively new communication scalability solution that has shown promise in reducing channel congestion\cite{model_based_communication}. The main objective of the MBC scheme is to utilize a content structure that is more adaptable for transmitting packets that contain the parameters of the joint vehicle dynamic/driver behavioral models than the Basic Safety Message (BSM) content structure defined by the J2735 standard\cite{saej2735}. To represent the vehicle's dynamic when using the MBC scheme, various modeling methods can be considered. Non-parametric Bayesian inference methods, such as Gaussian Processes (GP), hold great potential for modeling the joint vehicle's dynamic/driver's behavior.

It should be noted that the majority of existing vehicle platoon control research uses Time-Triggered Communication (TTC), in which data exchange between two successive vehicles is performed periodically according to a fixed communication rate. In this case, data communications are activated on a regular basis regardless of measurement changes, even if the difference between two successively transmitted values is very small. Because the scheduling of transmission instants is purely based on time and not on the actual status of the vehicle, TTC frequently results in inefficient use of communication resources, which is undesirable in the context of CACC. Therefore, it appears more logical to utilize communication techniques that are sensitive to control needs, determining the timing of transmissions based on output measurements in order to achieve a better balance between communication efficiency and control performance.

Because there is a tradeoff between platoon control performance and communication resource utilization, designing an efficient Event-Triggered Communication (ETC) for the vehicle platoon is critical. We describe the design of such a system in this paper and provide a new perspective on how the interaction between different components of a platoon can be modeled to improve performance. The contributions of this paper are as follows:
\begin{itemize}
    \item We proposed a control-aware communication solution that combines ETC with MBC for cooperative control of vehicle platoons.
    \item We describe a fully distributed ETC design for vehicle platoons that achieves a significant reduction in the average communication rate while only mildly reducing control performance. 
\end{itemize}

% %%%%%%%%%%%%%%%%%%%%%%%%%%%%%%%%%%%%%%%%%%%%%%%%%%%%%%%%%%%%%%%%%%%%%%%%%%%%%%%%%%%%%%%%%%%%%%%%%%%
% %%%%%%%%%%%%%%%%%%%%%%%%%%%%%%%%%%%%%%%%%%%%%%%%%%%%%%%%%%%%%%%%%%%%%%%%%%%%%%%%%%%%%%%%%%%%%%%%%%%
\section{Related Work} \label{sec::Related}
% %%%%%%%%%%%%%%%%%%%%%%%%%%%%%%%%%%%%%%%%%%%%%%%%%%%%%%%%%%%%%%%%%%%%%%%%%%%%%%%%%%%%%%%%%%%%%%%%%%%
% %%%%%%%%%%%%%%%%%%%%%%%%%%%%%%%%%%%%%%%%%%%%%%%%%%%%%%%%%%%%%%%%%%%%%%%%%%%%%%%%%%%%%%%%%%%%%%%%%%%
\noindent The control of multiple CAVs' collective behavior is based on the vehicles' mutual awareness of their states (e.g., inter-vehicle distance and vehicle speed), which is accomplished through inter-vehicle sensing and communication. This section will go over one of the cooperative driving applications, CACC, and V2V communication.
% %%%%%%%%%%%%%%%%%%%%%%%%%%%%%%%%%%%%%%%%%%%%%%%%%%%%%%%%%%%%%%%%%%%%%%%%%%%%%%%%%%%%%%%%%%%%%%%%%%%
% %%%%%%%%%%%%%%%%%%%%%%%%%%%%%%%%%%%%%%%%%%%%%%%%%%%%%%%%%%%%%%%%%%%%%%%%%%%%%%%%%%%%%%%%%%%%%%%%%%%
\subsection{\textbf{Cooperative Adaptive Cruise Control (CACC)}}
% %%%%%%%%%%%%%%%%%%%%%%%%%%%%%%%%%%%%%%%%%%%%%%%%%%%%%%%%%%%%%%%%%%%%%%%%%%%%%%%%%%%%%%%%%%%%%%%%%%%
% %%%%%%%%%%%%%%%%%%%%%%%%%%%%%%%%%%%%%%%%%%%%%%%%%%%%%%%%%%%%%%%%%%%%%%%%%%%%%%%%%%%%%%%%%%%%%%%%%%%
\noindent CACC systems must be constructed to endure any special maneuvers, such as interfering vehicles cutting into CACC platoons or hard braking by leading vehicles\cite{8370701,D2cav}. The faster and more precise information about the motion of preceding vehicles provided by V2V allows the CACC vehicle to more accurately follow the preceding cooperating vehicle, even at a considerably shorter distance. This not only enhances user acceptance but also has the potential to significantly improve traffic flow dynamics and lane throughput capacity. Researchers demonstrated that vehicle platooning has enormous potential to address a wide range of transportation issues\cite{Impact_CACC_Shladover,Effects_CACC_stability}. A well-designed CACC must keep the deviation from the desired gap, known as spacing error, as small as possible in order to reduce the risk of collision and enjoy the advantages of platoon formation, such as lower fuel consumption and higher traffic throughput\cite{fuel}.

Communication imperfections can significantly affect the performance of CACC systems, as excessively long communication delays or low transmission rates can compromise string stability and other performance properties for a given time gap\cite{CACC_VTC}. As a result, the number of transmissions in time must be large enough and communication delays must be short enough to achieve the desired platooning behavior\cite{8025403}.

% %%%%%%%%%%%%%%%%%%%%%%%%%%%%%%%%%%%%%%%%%%%%%%%%%%%%%%%%%%%%%%%%%%%%%%%%%%%%%%%%%%%%%%%%%%%%%%%%%%%
% %%%%%%%%%%%%%%%%%%%%%%%%%%%%%%%%%%%%%%%%%%%%%%%%%%%%%%%%%%%%%%%%%%%%%%%%%%%%%%%%%%%%%%%%%%%%%%%%%%%
\subsection{\textbf{Vehicle to Vehicle (V2V) Communication}}
% %%%%%%%%%%%%%%%%%%%%%%%%%%%%%%%%%%%%%%%%%%%%%%%%%%%%%%%%%%%%%%%%%%%%%%%%%%%%%%%%%%%%%%%%%%%%%%%%%%%
% %%%%%%%%%%%%%%%%%%%%%%%%%%%%%%%%%%%%%%%%%%%%%%%%%%%%%%%%%%%%%%%%%%%%%%%%%%%%%%%%%%%%%%%%%%%%%%%%%%%
\noindent The exchange of information is critical for platoon deployment because it allows control actions to be taken utilizing the latest information regarding road and traffic conditions. Many studies have been conducted to determine the impact of the communication network on platoon performance\cite{7997746,V2V_platooning}. One of TTC's major flaws is its lack of flexibility and scalability. This section discusses ETC and MBC as a flexible and scalable solution. The Cellular Vehicle to Everything (C-V2X) standard establishes a lower bound for the Minimum Inter-Event Time (MIET), which is the minimum amount of time that must pass between two consecutive transmissions\cite{saej3161}. This inter-packet duration is limited to $100 ms$ in the lower bound and $600 ms$ in the upper bound. This strictly positive lower bound is required to avoid Zeno behavior (an infinite number of events in finite time) and to make the ETC system practical to implement.

% %%%%%%%%%%%%%%%%%%%%%%%%%%%%%%%%%%%%%%%%%%%%%%%%%%%%%%%%%%%%%%%%%%%%%%%%%%%%%%%%%%%%%%%%%%%%%%%%%%%
% %%%%%%%%%%%%%%%%%%%%%%%%%%%%%%%%%%%%%%%%%%%%%%%%%%%%%%%%%%%%%%%%%%%%%%%%%%%%%%%%%%%%%%%%%%%%%%%%%%%
\subsubsection{\textbf{Event-Triggered Communication}}
% %%%%%%%%%%%%%%%%%%%%%%%%%%%%%%%%%%%%%%%%%%%%%%%%%%%%%%%%%%%%%%%%%%%%%%%%%%%%%%%%%%%%%%%%%%%%%%%%%%%
% %%%%%%%%%%%%%%%%%%%%%%%%%%%%%%%%%%%%%%%%%%%%%%%%%%%%%%%%%%%%%%%%%%%%%%%%%%%%%%%%%%%%%%%%%%%%%%%%%%%
Event-based strategies are a popular way to ensure that communication resources are used efficiently in MASs\cite{Lemmon2010,SEYBOTH2013245}. In contrast to traditional TTC, event-based approaches transmit data only when necessary to meet a control system specification. It was found that event-triggered systems exhibit better real-time performance than time-triggered systems. In\cite{WEN2018341}, an approach is proposed to reduce the communication burden by using a flexible event-triggering strategy based on tunable parameters for each platoon member.

Each agent sends its current state to neighboring agents only when the difference between the current state and the previously transmitted state surpasses a time-varying threshold, or when it reaches the maximum value of the inter-event interval. As a more feasible limitation for event-triggered approaches, researchers have explored imposing a minimum time interval between events\cite{NOWZARI20191}. Each vehicle's event-triggered scheme can be defined as follows:
\begin{equation}
t_{k+1} = t_k + \min \left(\tau_k, \tau\right) \text {, }
\end{equation}
where $\tau$ is a positive constant that denotes the upper bound of the inter-event interval. $\tau_k$ is determined by the following equation,
\begin{equation}
\tau_k = \inf _{t>t_k}\left\{t-t_k \mid C\left(S(t) \:,\tilde S\left(t\right)\right)>0\right\}, \quad \text { for } t \geq 0 .
\label{ETC_form}
\end{equation}

When new information is received, each agent will update its control input and use the received model for prediction. It should be noted that these trigger instants are not synchronized among agents. In such techniques, each vehicle runs a local duplicate of its neighbors' dynamics. In the proposed configuration, each vehicle uses its own transmitted model to determine when to send data. If the prediction of the kinematic model since the last broadcast is still accurate, the vehicle will not transmit a message.
% %%%%%%%%%%%%%%%%%%%%%%%%%%%%%%%%%%%%%%%%%%%%%%%%%%%%%%%%%%%%%%%%%%%%%%%%%%%%%%%%%%%%%%%%%%%%%%%%%%%
% %%%%%%%%%%%%%%%%%%%%%%%%%%%%%%%%%%%%%%%%%%%%%%%%%%%%%%%%%%%%%%%%%%%%%%%%%%%%%%%%%%%%%%%%%%%%%%%%%%%
\subsubsection{\textbf{Model-based Communication}}
% %%%%%%%%%%%%%%%%%%%%%%%%%%%%%%%%%%%%%%%%%%%%%%%%%%%%%%%%%%%%%%%%%%%%%%%%%%%%%%%%%%%%%%%%%%%%%%%%%%%
% %%%%%%%%%%%%%%%%%%%%%%%%%%%%%%%%%%%%%%%%%%%%%%%%%%%%%%%%%%%%%%%%%%%%%%%%%%%%%%%%%%%%%%%%%%%%%%%%%%%
When designing CACC systems, we must explicitly account for the uncertainty in vehicle state, behavior, and communication\cite{Gp_VNC}. Because the information from the neighboring vehicles is not continuously available, each agent must run an estimator. In this scheme, each agent uses a model to predict the measurements of the other agents when they do not receive a packet either due to packet loss or because the transmission was not triggered.

In this paper, we consider the velocity time series of each cooperative vehicle, $v_{n}(t)$, to be a GP defined by the mean function $m_{n}(t)$ and the covariance kernel function $\kappa_{n}(t, t^{\prime})$ as
\begin{equation}
v_{n}(\mathbf{t}) \sim \mathcal{G} \mathcal{P}\left(m_{n}(\mathbf{t}), \kappa_{n}\left(\mathbf{t}, \mathbf{t}^{\prime}\right)\right).
\end{equation}
We are interested in incorporating the knowledge that the observed velocity data provide about the underlying function, $v_{n}(t)$, and its future values. Assuming that for each cooperative vehicle, the mean of the process is zero, $m_{n}(t)=0$, the covariance kernel is a Radial Basis Function (RBF), and the measurement noises are independent and identically distributed ($i.i.d.$) with the Gaussian distribution $\mathcal{N}(0,\,\gamma_{n, noise}^{2})$, the covariance matrix of the observed velocity of the $n^{th}$ cooperative vehicle is
%%%%%%%%%%%%%%%%%%%%%%%%%%%%%%%%%%%%%%%%%%%%%%%%%%%%%%%%%%%%%%%%%%%%%%%%%%%
\begin{equation}
\begin{aligned}
\label{kernel matrix}
 K_{n}(\boldsymbol{t},\boldsymbol{t^{\prime}})= \kappa_{n}(t,t^{\prime}) + \gamma_{n,noise}^{2}I
\end{aligned}
\end{equation}
%%%%%%%%%%%%%%%%%%%%%%%%%%%%%%%%%%%%%%%%%%%%%%%%%%%%%%%%%%%%%%%%%%%%%%%%%%
where $I$ denotes the identity matrix of dimension equal to the size of the training (measured) data and $\kappa_{n}(t,t^{\prime})$ can be calculated using the RBF definition as
\begin{equation}
 \kappa_{n}(t,t^{\prime})=\exp(-\frac{||t-t^{\prime}||^2}{2\gamma_{n}^{2}}).
\end{equation}
Using the aforementioned assumptions, the joint distribution of the past observed values, $\mathcal{V}_{n}^{obs}$, and the future values $\mathcal{V}_{n}^{\ast}$, can be represented as
%%%%%%%%%%%%%%%%%%%%%%%%%%%%%%%%%%%%%%%%%%%%%%%%%%%%%%%%%%%%%%%%
\begin{equation}
\left[\begin{array}{l}
\mathbf{\mathcal{V}_{n}^{obs}} \\
\mathbf{\mathcal{V}_{n}}^{*}
\end{array}\right] \sim \mathcal{N}\left(\mathbf{0},\left[\begin{array}{ll}
K_{n}(\boldsymbol{t}, \boldsymbol{t}) & K_{n}\left(\boldsymbol{t}, \boldsymbol{t^{*}}\right) \\
K_{n}\left(\boldsymbol{t^{*}}, \boldsymbol{t}\right) & K_{n}\left(\boldsymbol{t^{*}}, \boldsymbol{t^{*}}\right)
\end{array}\right]\right),
\end{equation}
%%%%%%%%%%%%%%%%%%%%%%%%%%%%%%%%%%%%%%%%%%%%%%%%%%%%%%%%%%%%%%%%
where $\boldsymbol{t}$ and $\boldsymbol{t^{*}}$ denote the sets of observation and future value time stamps, respectively, and $K_{n}(.,.)$ can be obtained from \eqref{kernel matrix}.
%%%%%%%%%%%%%%%%%%%%%%%%%%%%%%%%%%%%%%%%%%%%%%%%%%%%%%%%%%%%%%%%%%%%%

% %%%%%%%%%%%%%%%%%%%%%%%%%%%%%%%%%%%%%%%%%%%%%%%%%%%%%%%%%%%%%%%%%%%%%%%%%%%%%%%%%%%%%%%%%%%%%%%%%%%
% %%%%%%%%%%%%%%%%%%%%%%%%%%%%%%%%%%%%%%%%%%%%%%%%%%%%%%%%%%%%%%%%%%%%%%%%%%%%%%%%%%%%%%%%%%%%%%%%%%%
\section{Preliminaries and Problem Formulation}
% %%%%%%%%%%%%%%%%%%%%%%%%%%%%%%%%%%%%%%%%%%%%%%%%%%%%%%%%%%%%%%%%%%%%%%%%%%%%%%%%%%%%%%%%%%%%%%%%%%%
% %%%%%%%%%%%%%%%%%%%%%%%%%%%%%%%%%%%%%%%%%%%%%%%%%%%%%%%%%%%%%%%%%%%%%%%%%%%%%%%%%%%%%%%%%%%%%%%%%%%
\noindent It is challenging to achieve a substantial reduction in V2V communication while preserving the desired performance of the vehicular platoon. Therefore, a crucial matter to tackle is how to devise suitable control strategies that can sustain acceptable MAS control performance while considerably decreasing the overuse of communication and computation resources. In our control formulation, local information such as spacing error and velocity error is used in a relative sense, that is, in comparison to the agent's own state, to adjust the control input of each subsequent vehicle to match the velocity of the leading vehicles while sustaining a consistent time gap between any two successive vehicles. During V2V communication outages, communication losses for CACC control were mitigated by using a GP to estimate the speed of the preceding vehicles.

% %%%%%%%%%%%%%%%%%%%%%%%%%%%%%%%%%%%%%%%%%%%%%%%%%%%%%%%%%%%%%%%%%%%%%%%%%%%%%%%%%%%%%%%%%%%%%%%%%%%
% %%%%%%%%%%%%%%%%%%%%%%%%%%%%%%%%%%%%%%%%%%%%%%%%%%%%%%%%%%%%%%%%%%%%%%%%%%%%%%%%%%%%%%%%%%%%%%%%%%%
\subsection{\textbf{Vehicle Model and Model Predictive Control Design Approach}}
% %%%%%%%%%%%%%%%%%%%%%%%%%%%%%%%%%%%%%%%%%%%%%%%%%%%%%%%%%%%%%%%%%%%%%%%%%%%%%%%%%%%%%%%%%%%%%%%%%%%
% %%%%%%%%%%%%%%%%%%%%%%%%%%%%%%%%%%%%%%%%%%%%%%%%%%%%%%%%%%%%%%%%%%%%%%%%%%%%%%%%%%%%%%%%%%%%%%%%%%%
\noindent In this study, we consider a platoon of $N_v$ vehicles, where $n\in\{0,1,\hdots,N_v-1\}$ denotes the $n^{th}$ vehicle in the platoon, and $n=0$ represents the platoon leader as shown in Figure \ref{fig:diagram}. $d_n$ denotes the gap between $n^{th}$ and $(n-1)^{th}$ vehicles and is defined as
\begin{equation}
    d_n = x_{n-1}-x_{n}-l^v_n,
\end{equation}
where $x_n$ and $l^v_n$ are the longitudinal location of the $n^{th}$ vehicle rear bumper and the vehicle length, respectively. The desired spacing policy can be defined as follows:
\begin{equation} \label{d*}
    \begin{aligned}
        d^{*}_{n}(t) = \delta_n\,v_{n}(t)+d^{s}_{n}.
    \end{aligned}
\end{equation}
In \eqref{d*}, $v_{n}$ is the velocity of the $n^{th}$ vehicle, $\delta_n$ is the time gap, and $d^s_n$ represents the standstill distance. The difference between the gap and its desired value is defined as $\Delta d_{n}(t) = d_{n}(t)-d^{*}_{n}(t)$, and the velocity difference between $n^{th}$ vehicle and its predecessor is defined as
$\Delta v_{n}(t) = v_{n-1}(t)-v_{n}(t)$. Hence, $\Delta \dot{d}_n$ turns into $\Delta \dot{d}_n(t)=\Delta v_n(t)-\delta_n\,a_n(t)$ and $\Delta \dot{v}_n=a_{n-1}-a_n$, where $a_n$ denotes the acceleration of the $n^{th}$ vehicle. By taking the driveline dynamics $f_n$ into account, the derivative of the acceleration of vehicle $n$ is $\dot{a_{n}}(t) = -\mathnormal{f}_{n} a_{n}(t) + \mathnormal{f}_{n}u_{n}(t)$, where $u_{n}(t)$ acts as vehicle input. By considering $S_n=[\Delta d_n\,\,\,\Delta v_n\,\,\, a_n]^T$ as the vector of states for $n^{th}$ vehicle, the state-space representation for each vehicle is
\begin{multline}  \label{css}
        \dot{S}_n(t)=A_n\,S_n(t)+B_n\,u_n(t)+D\,a_{n-1}(t)\\ \\=
        \begin{bmatrix}
            0&1&-\delta_n \\0&0&-1\\0&0&-\mathnormal{f}_{n}
        \end{bmatrix}S_n(t)+
        \begin{bmatrix}
            0\\0\\\mathnormal{f}_{n}
        \end{bmatrix}u_{n}(t)+
        \begin{bmatrix}
            0\\1\\0
        \end{bmatrix}a_{n-1}(t).
\end{multline}
For $n=0$ (leader), $a_{n-1}(t)$ is replaced by zero. The following equation describes the discrete-time state-space model when the first-order forward time approximation is employed;
\begin{multline}\label{fdss}
        S_n(k+1) =\\ (I+t_s\,A_n)\,S_n(k)+t_s\,B_n\,u_n(k)+t_s\,D\,a_{n-1}(k),
\end{multline}
where $t_s$ is the sampling time.

Some constraints on the system states and input are also considered, including bounds on acceleration and input, road speed limit, and distance between vehicles (note that a negative distance implies collision and therefore should not occur). The following inequalities (hard constraints) should always hold true
\begin{subequations} \label{bounds}
    \begin{gather}
        \label{accbound}
        a_n^{min}\leq a_n(k)\leq a_n^{max},\\ 
        \label{inputbound}
        u_n^{min}\leq u_n(k)\leq u_n^{max},\\
        \label{speedlim}
        v_n(k)\leq v^{max},\\
        d_n(k)>0.
    \end{gather}
\end{subequations}
Besides, for passenger comfort, system input changes are bounded as
\begin{equation}
    \begin{gathered}
        t_s\,u_n^{min} \leq u_n(k+1)-u_n(k) \leq t_s\,u_n^{max}.
    \end{gathered}
    \label{passenger_comfort}
\end{equation}
The MPC design problem for each vehicle is
\begin{multline} \label{multi_cost}
    \sum_{k=0}^{N-1}\Bigg[(\mathbf{S}_n(k)-R_n)^T\, Q_n \,(\mathbf{S}_n(k)-R_n)\\
    +\hspace{-5pt}\sum_{i=n-r}^{n-1}\hspace{-5pt}\Big[ c^d_i\,\Big(x_i(k)-x_n(k)-\hspace{-5pt}\sum_{j=i+1}^{n}\hspace{-3pt}(d^*_j(k)+l^v_j)\Big)^2\\
     \qquad \qquad \qquad  \quad  +c^v_i\,\Big(v_i(k)-v_n(k)\Big)^2\Big]\Bigg],\\
        \text{subject to: System Constrains,} \qquad \qquad
\end{multline}
where $\textbf{u}_n$ is the system inputs from $k=0$ to $k=N-1$, $c^d_i$ and $c^v_i$ are positive coefficients, and $r$ denotes the number of predecessors sharing information with the $n^{th}$ vehicle.

% %%%%%%%%%%%%%%%%%%%%%%%%%%%%%%%%%%%%%%%%%%%%%%%%%%%%%%%%%%%%%%%%%%%%%%%%%%%%%%%%%%%%%%%%%%%%%%%%%%%
% %%%%%%%%%%%%%%%%%%%%%%%%%%%%%%%%%%%%%%%%%%%%%%%%%%%%%%%%%%%%%%%%%%%%%%%%%%%%%%%%%%%%%%%%%%%%%%%%%%%
\subsection{\textbf{Event-triggering conditions}}
% %%%%%%%%%%%%%%%%%%%%%%%%%%%%%%%%%%%%%%%%%%%%%%%%%%%%%%%%%%%%%%%%%%%%%%%%%%%%%%%%%%%%%%%%%%%%%%%%%%%
% %%%%%%%%%%%%%%%%%%%%%%%%%%%%%%%%%%%%%%%%%%%%%%%%%%%%%%%%%%%%%%%%%%%%%%%%%%%%%%%%%%%%%%%%%%%%%%%%%%%
\noindent Transmission instants in ETC schemes are determined online by a "smart" triggering condition that depends on, for example, system output measurements, so that transmission is only scheduled when necessary to guarantee some performance properties. It is only necessary to verify and execute the event-triggered condition periodically at each communication moment.

The event-triggered condition is "fully distributed" in the sense that it does not rely on global communication topology information. In these systems, if each agent decides when to broadcast its state information on its own, not only the control effort but also the network load will be reduced.

% %%%%%%%%%%%%%%%%%%%%%%%%%%%%%%%%%%%%%%%%%%%%%%%%%%%%%%%%%%%%%%%%%%%%%%%%%%%%%%%%%%%%%%%%%%%%%%%%%%%
% %%%%%%%%%%%%%%%%%%%%%%%%%%%%%%%%%%%%%%%%%%%%%%%%%%%%%%%%%%%%%%%%%%%%%%%%%%%%%%%%%%%%%%%%%%%%%%%%%%%
\subsubsection{\textbf{Control-aware Triggering}}
% %%%%%%%%%%%%%%%%%%%%%%%%%%%%%%%%%%%%%%%%%%%%%%%%%%%%%%%%%%%%%%%%%%%%%%%%%%%%%%%%%%%%%%%%%%%%%%%%%%%
% %%%%%%%%%%%%%%%%%%%%%%%%%%%%%%%%%%%%%%%%%%%%%%%%%%%%%%%%%%%%%%%%%%%%%%%%%%%%%%%%%%%%%%%%%%%%%%%%%%%
Making explicit decisions based on current control system states results in a control-aware approach. In desirable states, control systems perform modestly, whereas, in undesirable states, systems change their states and are given higher priority for data transmission. Transmission thresholds are set so that each vehicle can meet stability goals and make transmission decisions to meet performance targets while reducing the total transmission rate. As a result, $\tau_k$ in \ref{ETC_form} for control-aware triggering will take the form of
\begin{equation}
\tau_k=\inf _{t>t_k}\left\{t - t_k \mid \|\mathcal{C}_i\| \geq \beta \right\}, \quad \text { for } t \geq 0 .
\end{equation}
where $\mathcal{C}_i$ is the cost function in \ref{multi_cost}.

% %%%%%%%%%%%%%%%%%%%%%%%%%%%%%%%%%%%%%%%%%%%%%%%%%%%%%%%%%%%%%%%%%%%%%%%%%%%%%%%%%%%%%%%%%%%%%%%%%%%
% %%%%%%%%%%%%%%%%%%%%%%%%%%%%%%%%%%%%%%%%%%%%%%%%%%%%%%%%%%%%%%%%%%%%%%%%%%%%%%%%%%%%%%%%%%%%%%%%%%%
\section{Experimental Results}
% %%%%%%%%%%%%%%%%%%%%%%%%%%%%%%%%%%%%%%%%%%%%%%%%%%%%%%%%%%%%%%%%%%%%%%%%%%%%%%%%%%%%%%%%%%%%%%%%%%%
% %%%%%%%%%%%%%%%%%%%%%%%%%%%%%%%%%%%%%%%%%%%%%%%%%%%%%%%%%%%%%%%%%%%%%%%%%%%%%%%%%%%%%%%%%%%%%%%%%%%
\noindent In our experiments, we considered the Packet Error Rate (PER) to be an i.i.d. random variable with two values of $0$ (ideal communication) and $0.6$ (randomly losing 60\% of packets) to study the effect of communication loss on the CACC performance. In our studies, the simulation step is $100\,ms$ which is the communication periodicity. To validate the proposed strategy and demonstrate its technical feasibility, the ETC policy has been simulated on a platoon of $N_v = 10$ vehicles. CVXPY package in Python is used for implementing the optimization problem and Gurobi optimization package is used as the solver\cite{cvxpy,gurobi}.

To check the event-triggering condition, each vehicle only needs to record the state of its most recent event-triggered instant and continuously monitor its own state. We used an All-Predecessor-Leader-Following (APLF) topology\cite{CACC_VTC}.Our method is to establish a correlation between communication behaviors and platoon performance, which directly relies on the quantity of accurately transmitted vehicle information and, of course, the number of nodes that receive the information\cite{Information_Dissemination}. We investigate the behavior of a platoon as the threshold values of the proposed triggering condition is varied over a selected range in terms of vehicle efficiency and safety.

\vspace{-0.05in}
% %%%%%%%%%%%%%%%%%%%%%%%%%%%%%%%%%%%%%%%%%%%%%%%%%%%%%%%%%%%%%%%%%%%%%%%%%%%%%%%%%%%%%%%%%%%%%%%%%%%
% %%%%%%%%%%%%%%%%%%%%%%%%%%%%%%%%%%%%%%%%%%%%%%%%%%%%%%%%%%%%%%%%%%%%%%%%%%%%%%%%%%%%%%%%%%%%%%%%%%%
\subsection{\textbf{Implementation Details}}
% %%%%%%%%%%%%%%%%%%%%%%%%%%%%%%%%%%%%%%%%%%%%%%%%%%%%%%%%%%%%%%%%%%%%%%%%%%%%%%%%%%%%%%%%%%%%%%%%%%%
% %%%%%%%%%%%%%%%%%%%%%%%%%%%%%%%%%%%%%%%%%%%%%%%%%%%%%%%%%%%%%%%%%%%%%%%%%%%%%%%%%%%%%%%%%%%%%%%%%%%
\noindent The parameters used in the simulations can be found in Table \ref{table1}. Each scenario takes $60\,s$, in which the objective of the platoon is to maintain the desired gap time of $0.6\,s$ with the preceding vehicle. Upon each transmission opportunity, each cooperative vehicle uses its $5$ most recent velocity observations, measured at equally-distanced $100\,ms$ time intervals, to train a GP model and obtain the set of parameters $\Theta_{n}=\{\gamma_{n},\gamma_{n, noise}\}$. After the GP parameters are learned, the transmitting vehicle shares the model parameters along with its history of the $5$ most recent velocity measurements and the current position and acceleration with their time stamps.
In addition, the $10$ future velocity values (parameter $N$ in Table \ref{table1}) predicted by the vehicle's MPC are included in the transmitting packet. The cooperative vehicles update the preceding vehicles' information either based on the newly received information from them or based on the GP predictive model every $100\,ms$. This information is fed into the MPC for updating the control action. Furthermore, the control module provides the optimal predicted states' values of the ego vehicle. Finally, if a triggering condition is detected, the control module will send current states and predicted future velocity trajectory values to the networking module for broadcasting.

\vspace{-0.05in}
% %%%%%%%%%%%%%%%%%%%%%%%%%%%%%%%%%%%%%%%%%%%%%%%%%%%%%%%%%%%%%%%%%%%%%%%%%%%%%%%%%%%%%%%%%%%%%%%%%%%
% %%%%%%%%%%%%%%%%%%%%%%%%%%%%%%%%%%%%%%%%%%%%%%%%%%%%%%%%%%%%%%%%%%%%%%%%%%%%%%%%%%%%%%%%%%%%%%%%%%%
\subsection{\textbf{Analysis and Results}}
% %%%%%%%%%%%%%%%%%%%%%%%%%%%%%%%%%%%%%%%%%%%%%%%%%%%%%%%%%%%%%%%%%%%%%%%%%%%%%%%%%%%%%%%%%%%%%%%%%%%
% %%%%%%%%%%%%%%%%%%%%%%%%%%%%%%%%%%%%%%%%%%%%%%%%%%%%%%%%%%%%%%%%%%%%%%%%%%%%%%%%%%%%%%%%%%%%%%%%%%%
\noindent We use the experimental results from a TTC scheme, which determines transmission instants based on a fixed transmission rate of $10\,Hz$, as a benchmark to assess the performance of the proposed ETC approach and communication resource utilization. To compare the performance of control-aware triggering, the average transmission rate is used as a function of the network resource usage pattern. We defined the distance error as the absolute value of the difference between the actual distance gap and desired distance gap in meters. Also, the difference between maximum and minimum speed and acceleration considering all platoon members at each time step is a good measure for traffic flow and CACC performance. We define these metrics as speed difference and acceleration difference. The mean of the absolute value of spacing error, speed difference, and acceleration difference for an ideal $10\,Hz$ TTC scheme are $\textbf{0.302}\,m$, $\textbf{4.693}\,m/s$, and $\textbf{1.257}\,m/s^{2}$, respectively (see Figure \ref{fig:TTC_0}). These are the smallest errors achieved using our method.

%%%%%%%%%%%%%%%%%%%%%%%%%%%%%%%%%%%
\vspace{-0.15in}
\begin{table}[h]
\caption{Model and optimization parameters used in the simulations.}
\centering
\renewcommand{\arraystretch}{1.1}
\begin{tabular}{|p{1.3cm}|p{1.2cm}||p{1.3cm}|p{1.2cm}|} 
 \hline
 Parameter & Value & Parameter & Value\\ [0.5ex] 
 \hline\
 $N$ &  $10$ & $t_s$ & $0.1\,s$
 \\\
 $l^v_n$ & $5\,m$ & $d^s_n$ & $2\,m$ \\\
 $a_n^{max}$ & $ 3\,m/s^2$  & $a_n^{min}$ & $ -4\,m/s^2$\\\
 $u_n^{max}$ & $ 3\,m/s^2$  & $u_n^{min}$ & $ -4\,m/s^2$\\ 
 $f_n$ & $10\,s^{-1}$ &  & \\ 
 \hline
\end{tabular}
\label{table1}
\vspace{-0.2in}
\end{table}
%%%%%%%%%%%%%%%%%%%%%%%%%%%%%%%%%%%

%%%%%%%%%%%%%%%%%%%%%%%%%%%%%%%%%%%
\vspace{-0.15in}
\begin{table}[h]
\caption{Statistics for Control-aware Triggering with varying threshold levels}
\begin{center}
\renewcommand{\arraystretch}{1}
\begin{tabular}{|m{1.35cm}|m{1.37cm}|m{0.94cm}|m{1.14cm}|m{1.15cm}|} 
 \hline
 Thresholding level & Mean communication rate [$Hz$] & Mean Spacing error [$m$] & Mean Speed difference [$m/s$] & Mean Acceleration difference [$m/s^{2}$]\\ [0.6ex]
\hline $200$ & $7.52$ & $0.304$ & $4.697$ & $1.252$ \\
\hline $300$ & $6.82$ & $0.306$ & $4.698$ & $1.252$ \\
\hline $400$ & $6.72$ & $0.306$ & $4.698$ & $1.253$ \\
\hline $500$ & $6.04$ & $0.308$ & $4.699$ & $1.251$ \\
\hline $600$ & $5.99$ & $0.308$ & $4.700$ & $1.251$ \\
\hline $700$ & $5.28$ & $0.309$ & $4.700$ & $1.250$ \\
\hline
\end{tabular}
\label{tab:results}
\end{center}
\vspace{-0.2in}
\end{table}
%%%%%%%%%%%%%%%%%%%%%%%%%%%%%%%%%%%

%%%%%%%%%%%%%%%%%%%%%%%%%%%%%%%%%%%%%%%%%%%%%%%%%
\begin{figure}[H]
  \centering
    \includegraphics[width=1\linewidth]{Figures/Picture12.jpg}
    \vspace{-0.2in}
  \caption{Performance of the CACC with TTC, PER=0, and fixed communication rate of 10 Hz.}
  \label{fig:TTC_0}
  \vspace{-0.15in}
\end{figure}
%%%%%%%%%%%%%%%%%%%%%%%%%%%%%%%%%%%%%%%%%%%%%%%%%

%%%%%%%%%%%%%%%%%%%%%%%%%%%%%%%%%%%%%%%%%%%%%%%%%
\begin{figure}[H]
  \centering
    \includegraphics[width=1\linewidth]{Figures/Picture5.jpg}
    \vspace{-0.2in}
  \caption{Performance of the CACC with control-aware triggering ETC, PER=0, level 6 threshold, and average communication rate of 5.28 Hz.}
  \label{fig:control_ideal}
  \vspace{-0.1in}
\end{figure}
%%%%%%%%%%%%%%%%%%%%%%%%%%%%%%%%%%%%%%%%%%%%%%%%%%%

% %%%%%%%%%%%%%%%%%%%%%%%%%%%%%%%%%%%%%%%%%%%%%%%%%
% \begin{figure}[H]
%   \centering
%     \includegraphics[width=1\linewidth]{Figures/Picture8.jpg}
%   \caption{Performance of the CACC with TTC, and PER=0.6, and fixed communication rate of 10 Hz.}
%   \label{fig:TTC_0.6}
%   \vspace{-0.2in}
% \end{figure}
% %%%%%%%%%%%%%%%%%%%%%%%%%%%%%%%%%%%%%%%%%%%%%%%%%%%

In Figures \ref{fig:TTC_0}, \ref{fig:control_ideal}, and \ref{fig:control_aware_0.6}, the first subplot (first row) shows the distance of each vehicle from its predecessor ($d_{n}(t)$) while the second subplot shows the velocity of each vehicle ($v_{n}(t)$). The third subplot depicts the acceleration information for each vehicle, and in the last subplot, every time instance ($\xi_{n}(t)$) when a vehicle sends information to its following vehicles is marked. We want to note that, $d_{0}(t)$ cannot be defined for the leader because there is no platoon member in front of it, therefore, $d_{0}(t)$ is not included in the first subplots of the mentioned figures. Despite the significant reduction in communication achieved by the ETC scheme, the responses to the ETC scheme look similar to the responses to the TTC scheme. These findings demonstrate that the frequency of communication can be significantly reduced while maintaining the desired control performance.
Figure \ref{fig:control_ideal} represents the simulation's best-case scenario for the level 6 threshold. Because we use an ideal communication ($PER=0$) for this figure, we only considered the effect of the ETC scheme. We considered the combined effect of the ETC and PER in figure \ref{fig:control_aware_0.6} which represents the worst-case scenario of the simulation under consideration in this paper (the highest level of threshold and the highest level of PER). Despite the smooth acceleration shown in Figure \ref{fig:TTC_0}, the acceleration profile in all other figures fluctuates due to a lack of precise information (either because the packets were lost or because the transmissions were not triggered). Vehicles using the proposed communication paradigm can safely follow the vehicle in front of them. The threshold levels for control-aware triggering are six equally spaced steps from $200$ to $700$ as shown in Table \ref{tab:results}.

%%%%%%%%%%%%%%%%%%%%%%%%%%%%%%%%%%%%%%%%%%%%%%%%%
\begin{figure}[H]
  \centering
    \includegraphics[width=1\linewidth]{Figures/Picture10.jpg}
    \vspace{-0.2in}
  \caption{Performance of the CACC with control-aware triggering ETC, PER=0.6, level 6 threshold, and average communication rate of 5.28 Hz.}
  \label{fig:control_aware_0.6}
  \vspace{-0.1in}
\end{figure}
%%%%%%%%%%%%%%%%%%%%%%%%%%%%%%%%%%%%%%%%%%%%%%%%%%%

In Figure \ref{fig:control_ideal}, communication triggering is based on the current state of the control system. Because the platoon's last members will have a relatively worst control situation (they must compensate for the errors of preceding vehicles in order to provide a string stable platoon), they will transmit more frequently. For instance, communication events for $\xi_{8}(t)$ and $\xi_{9}(t)$ are always one.

Table \ref{tab:results} demonstrates how the trigger level can be selected to trade performance against the data transmission rate. Each data point is an average of 70 simulation rounds to provide a better sense of performance. Table \ref{tab:results} shows the mean of spacing error ($m$), speed difference ($m/s$), and acceleration difference ($m/s^2$), respectively. The ETC policy induces a tradeoff between control performance and communication frequency. As can be seen, lowering the trigger level leads to a lower error, but a higher data transmission rate. Long inter-event times result in significant performance errors. As a result, raising the threshold will degrade the control performance.

% \vspace{-0.1in}
% %%%%%%%%%%%%%%%%%%%%%%%%%%%%%%%%%%%%%%%%%%%%%%%%%%%%%%%%%%%%%%%%%%%%%%%%%%%%%%%%%%%%%%%%%%%%%%%%%%%
% %%%%%%%%%%%%%%%%%%%%%%%%%%%%%%%%%%%%%%%%%%%%%%%%%%%%%%%%%%%%%%%%%%%%%%%%%%%%%%%%%%%%%%%%%%%%%%%%%%%
\section{Conclusion}
% %%%%%%%%%%%%%%%%%%%%%%%%%%%%%%%%%%%%%%%%%%%%%%%%%%%%%%%%%%%%%%%%%%%%%%%%%%%%%%%%%%%%%%%%%%%%%%%%%%%
% %%%%%%%%%%%%%%%%%%%%%%%%%%%%%%%%%%%%%%%%%%%%%%%%%%%%%%%%%%%%%%%%%%%%%%%%%%%%%%%%%%%%%%%%%%%%%%%%%%%
\noindent As the number of devices connected to a shared network increases beyond the previous limit, distributed time-triggered coordination techniques become less scalable. Such constraints necessitate a shift away from the periodic communication paradigm toward opportunistic schemes, such as the one discussed in this paper. Excessive use of communication resources, on the other hand, can have a negative impact on their reliability. As a result, in this article, a resource-sensitive CACC communication strategy is proposed, with the goal of reducing the utilization of communication resources compared to conventional TTC methods while maintaining system performance. Alternatively, the performance of the system can be improved given a fixed communication rate. Furthermore, the minimum inter-event times are guaranteed to have a positive lower-bound by design to avoid Zeno behavior.

In addition, we combined MBC with ETC to propose a communication strategy for distributed multi-agent coordination. Each agent decides based on local information, mainly on the difference between its current state/model and its latest broadcast state/model, when a new measurement has to be transmitted over the network. Only the event-triggered condition needs to be periodically checked and executed at each communication time instant. The simulation results show that it is possible to achieve an ETC with good performance that reduces network load ($47\%$) when compared to a TTC while only slightly reducing control performance (e.g., less than $1\%$ speed deviation).
% %%%%%%%%%%%%%%%%%%%%%%%%%%%%%%%%%%%%%%%%%%%%%%%%%%%%%%%%%%%%%%%%%%%%%%%%%%%%%%%%%%%%%%%%%%%%%%%%%%%
% %%%%%%%%%%%%%%%%%%%%%%%%%%%%%%%%%%%%%%%%%%%%%%%%%%%%%%%%%%%%%%%%%%%%%%%%%%%%%%%%%%%%%%%%%%%%%%%%%%%
\balance
\bibliography{main.bib}{}
\bibliographystyle{unsrt}
\end{document}