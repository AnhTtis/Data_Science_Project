% CVPR 2023 Paper Template
% based on the CVPR template provided by Ming-Ming Cheng (https://github.com/MCG-NKU/CVPR_Template)
% modified and extended by Stefan Roth (stefan.roth@NOSPAMtu-darmstadt.de)

\documentclass[10pt,twocolumn,letterpaper]{article}

%%%%%%%%% PAPER TYPE  - PLEASE UPDATE FOR FINAL VERSION
% \usepackage[review]{cvpr}      % To produce the REVIEW version
\usepackage{cvpr}              % To produce the CAMERA-READY version
%\usepackage[pagenumbers]{cvpr} % To force page numbers, e.g. for an arXiv version

% Include other packages here, before hyperref.
\usepackage{graphicx}
\usepackage{amsmath}
\usepackage{amssymb}
\usepackage{amsthm}
\usepackage{booktabs}
\usepackage{adjustbox}
\usepackage{multicol}
\usepackage{algpseudocode}


% It is strongly recommended to use hyperref, especially for the review version.
% hyperref with option pagebackref eases the reviewers' job.
% Please disable hyperref *only* if you encounter grave issues, e.g. with the
% file validation for the camera-ready version.
%
% If you comment hyperref and then uncomment it, you should delete
% ReviewTempalte.aux before re-running LaTeX.
% (Or just hit 'q' on the first LaTeX run, let it finish, and you
%  should be clear).
\usepackage[pagebackref,breaklinks,colorlinks]{hyperref}


% Support for easy cross-referencing
\usepackage[capitalize]{cleveref}
\crefname{section}{Sec.}{Secs.}
\Crefname{section}{Section}{Sections}
\Crefname{table}{Table}{Tables}
\crefname{table}{Tab.}{Tabs.}

\usepackage{caption}

%%%%%%%%% PAPER ID  - PLEASE UPDATE
\def\cvprPaperID{76} % *** Enter the CVPR Paper ID here
\def\confName{CVPR}
\def\confYear{2023}

\input{preamble_packages}
\input{preamble_symbols}
%!TEX root = main.tex
% \usepackage{tikz}
% \usetikzlibrary{shapes}
% \newcommand{\mysymbol}[1][]{
% \begin{tikzpicture}[#1]
% \node[draw,ellipse,minimum height=5pt,minimum width=10pt](e){};
% \end{tikzpicture}
% }


\newcommand{\Sn}{\mathbb{S}^n}
\newcommand{\R}{\mathbb{R}}
\newcommand{\cA}{\mathcal{A}}
\newcommand{\cB}{\mathcal{B}}
\newcommand{\cL}{\mathcal{L}}
\renewcommand{\norm}[1]{\Vert #1 \Vert}
\newcommand{\inprod}[2]{\left\langle #1, #2 \right\rangle}
\newcommand{\vectorize}[1]{\mathrm{vec}\parentheses{#1}}
\newcommand{\Fnorm}[1]{\Vert #1 \Vert_{\mathrm{F}}}

\newcommand{\nnReal}[1]{\mathbb{R}_{+}^{#1}}
\newcommand{\vcat}{\ ;\ }
\newcommand{\mymid}{\ \middle\vert\ }
\newcommand{\cbrace}[1]{\left\{#1\right\}}
\newcommand{\sym}[1]{\mathbb{S}^{#1}}
\newcommand{\calAadj}{\calA^{*}}
\newcommand{\barq}{\bar{q}}
\newcommand{\barW}{\bar{W}}
\newcommand{\bary}{\bar{y}}
\newcommand{\barC}{\bar{C}}
\newcommand{\barcalA}{\bar{\calA}}
\newcommand{\barcalAadj}{\bar{\calA}^{*}}
\newcommand{\bmat}{\left[ \begin{array}}
\newcommand{\emat}{\end{array}\right]}
\newcommand{\psd}[1]{\sym{#1}_{+}}
\newcommand{\parentheses}[1]{\left(#1\right)}
\newcommand{\half}{\frac{1}{2}}

\newcommand{\usphere}[1]{\calS^{#1}}
\newcommand{\hpartial}{\hat{\partial}}
\newcommand{\baralpha}{\bar{\alpha}}
\newcommand{\tldW}{\widetilde{\MW}}
\newcommand{\bareta}{\bar{\eta}}
\newcommand{\tWnu}{\tilde{W}_{\nu}}
\newcommand{\tWzero}{\tilde{W}_{0}}
\newcommand{\tWone}{\tilde{W}_{1}}
\newcommand{\tWhalf}{\tilde{W}_{1/2}}
\newcommand{\Qa}{Q_{\alpha}}
\newcommand{\Qb}{Q_{\baralpha}}
\newcommand{\tV}{\tilde{V}}
\newcommand{\tVzero}{\tV_{0}}
\newcommand{\tVhalf}{\tV_{1/2}}
\newcommand{\tVone}{\tV_{1}}
\newcommand{\tVnu}{\tV_{\nu}}
\newcommand{\Mnu}{M_{\nu}}
\newcommand{\tMnu}{\tilde{M}_{\nu}}
\newcommand{\tM}{\tilde{M}}
\newcommand{\hV}{\hat{V}}
\newcommand{\hM}{\hat{M}}
\newcommand{\hMnu}{\hat{M}_{\nu}}
\newcommand{\Omegahalf}{\Omega^{.5}}
\newcommand{\bracket}[1]{\left[#1\right]}
\newcommand{\abs}[1]{\left|#1\right|}
\newcommand{\tldr}{\tilde{r}}
\newcommand{\tldQ}{\tilde{Q}}
\newcommand{\haty}{\hat{y}}

\renewcommand{\prob}{\mathbb{P}}
\newcommand{\probof}[1]{\mathbb{P}\left[#1\right]}
\newcommand{\Feps}{F^{\epsilon}}
\newcommand{\Fepsball}{\Feps_{\mathrm{ball}}}
\newcommand{\Fepsellipse}{\Feps_{\mathrm{ellipse}}}
\newcommand{\tcalY}{\tilde{\calY}}
\newcommand{\floor}[1]{\lfloor #1 \rfloor}
\newcommand{\ceil}[1]{\lceil #1 \rceil}

\newcommand{\lmo}{\scenario{LM-O}}
\newcommand{\heatmapball}{\scenario{heatmap-ball}}
\newcommand{\heatmapellipse}{\scenario{heatmap-ellipse}}
\newcommand{\pvnetball}{\scenario{PVNet-ball}}
\newcommand{\pvnetellipse}{\scenario{PVNet-ellipse}}
\newcommand{\gtball}{\scenario{gt-ball}}
\newcommand{\gtellipse}{\scenario{gt-ellipse}}
\newcommand{\frcnnball}{\scenario{frcnn-ball}}
\newcommand{\frcnnellipse}{\scenario{frcnn-ellipse}}
\newcommand{\purse}{\scenario{PURSE}}
\newcommand{\ransag}{\scenario{RANSAG}}
\newcommand{\pthreep}{\scenario{P3P}}
\newcommand{\phipeak}{\phi_{\mathrm{peak}}}
\newcommand{\phicov}{\phi_{\mathrm{cov}}}
\newcommand{\pnp}{\scenario{PnP}}
\newcommand{\Seps}{S^{\epsilon}}
\newcommand{\ransac}{\scenario{RANSAC}}
\newcommand{\Rgt}{R_{\mathrm{gt}}}
\newcommand{\tgt}{t_{\mathrm{gt}}}
\newcommand{\sgt}{s_{\mathrm{gt}}}

\newcommand{\gray}[1]{{\color{gray}#1}}

\begin{document}

%%%%%%%%% TITLE - PLEASE UPDATE
% \title{Conformal Semantic Keypoint Detection with Statistical Guarantees}

% \title{Your Pose is in My PURSE: Probabilistically Correct Object Pose Estimation with Conformal Prediction}
\title{\vspace{-18mm} Object Pose Estimation with Statistical Guarantees: \\ Conformal Keypoint Detection and Geometric Uncertainty Propagation \vspace{-6mm}}

\author{Heng Yang and Marco Pavone\\
NVIDIA Research
% \\
% {\tt\small \{hengy,mpavone\}@nvidia.com} 
}
% For a paper whose authors are all at the same institution,
% omit the following lines up until the closing ``}''.
% Additional authors and addresses can be added with ``\and'',
% just like the second author.
% To save space, use either the email address or home page, not both
% \and
% Second Author\\
% Institution2\\
% First line of institution2 address\\
% {\tt\small secondauthor@i2.org}
% }
% \maketitle

\twocolumn[{%
\renewcommand\twocolumn[1][]{#1}%
\maketitle
% \vspace{-mm}
\vspace{-8mm}
% !TEX root =  main.tex
%%%%%%%%%%%%%%%%%%%%%%%%%%%%%%%%%%%%%%%%%%%%%%%%%%%%%%%%%%
\begin{figure*}[!t]
\centering
\begin{tikzpicture}[
        box/.style={rectangle,draw=black, minimum size=0.25cm},
        ]
        \foreach \x in {-9.2,-8.95,...,-6.95}{
            \foreach \y in {2,2.25,...,4.25}
                \node[box, fill=gray!40] at (\x,\y){};
        }
       
        \foreach \y in {2.5,2.75,3}{
                \node[box, fill=white] at (-8.45,\y){};
        }
        % \foreach \x in {-8.75,-8.5, -8.25}{
        %         \node[box, fill=white] at (\x,3){};
        % }
        \node[box, fill=white] at (-7.95,2.75){};
        \node[box, fill=white] at (-7.7,2.75){};
          \foreach \x in {-8.45,-8.2,-7.95}{
                \node[box, fill=white] at (\x,3){};
        }
        \foreach \y in {2.75,3,3.25, 3.5}{
                \node[box, fill=white] at (-7.45,\y){};
        }
        \foreach \x in {-7.45,-7.7,-7.95,-8.2}{
                \node[box, fill=white] at (\x,3.5){};
        }
        \node[box, fill=white] at (-8.2,3.75){};
         %%% bounding box
        \draw[draw=black, thick] (-9.3,1.9) rectangle (-6.85,4.35);
        \node[draw, fill=red, star, star points=5,inner sep=0pt,minimum size=5pt] at (-8.2,3.75){};
        \node[draw, fill=blue!50, dart, rotate=90, inner sep=0.2pt,minimum size=4pt] at (-8.45,2.5){};
        %%%% code
        \node[text width=0.5cm, anchor=west, right] at (-6.45, 3.5)
    {\begin{boxcode}{3.68cm}{0.80}{0.975}
				\textcode{def }\DSLRun\textcode{()\{}\\
                \quad \DSLMove\\
                \quad \DSLMove\\
                \quad \DSLTurnRight\\
                \quad \ldots\\
                \quad \text{21 more action blocks}\\
				\textcode{\}}
				% \vspace{0.5em}
			\end{boxcode}};
	%%%%% bounding box around the task-code pair
	\draw[draw=black, thick] (-9.95,1.1) rectangle (-2.1, 4.9);
	%%%%%%%%% solution code
	\node[text width=0.5cm, anchor=east, left] at (-12.9, 3.1)
    {\begin{boxcode}{3.68cm}{0.70}{0.55}
				\textcode{def }\DSLRun\textcode{()\{}\\
				\quad \DSLRepeatUntil\textcode{(}\DSLBoolGoal\textcode{)\{}\\
                \quad \quad
              \DSLIf\textcode{(}\DSLBoolPathAhead\textcode{)\{}\\
				\quad \quad \quad \DSLMove\\ 
				% \textcode{\}}\\
				\quad \quad \textcode{\}}\\
				\quad \quad 
				\DSLElse\textcode{\{}\\
				\quad \quad \quad \DSLIf\textcode{(}\DSLBoolPathRight\textcode{)\{}\\
				\quad \quad \quad \quad \DSLTurnRight\\
				% \textcode{\}}\\
				\quad \quad \quad \textcode{\}}\\
				\quad \quad \quad
				\DSLElse\textcode{\{}\\
				\quad \quad \quad \quad \DSLTurnLeft\\
				% \textcode{\}}\\
				\quad \quad \quad \textcode{\}}\\
			     \vspace{-3.5mm}
				\quad \quad \textcode{\}}\\
		     	\vspace{-4.5mm}				
				\quad \textcode{\}}\\
				\vspace{-5.5mm}				
				\textcode{\}}
            %\vspace{-40mm}				
			\end{boxcode}
		    %\vspace{-6mm}
			};
	%%%%% bounding box around the solution-code pair
	\draw[draw=black, thick, pattern=north east lines, pattern color=gray!30] (-13.5,1.1) rectangle (-10.3, 4.9);
	%%%% captions for the task-codes
	\node[text width=0.5cm] at (-7.975,4.7) {$T^\text{in}$};
	\node[text width=0.5cm] at (-4.46,4.7) {\studentcode};
	\node[text width=0.5cm] at (-11.8,4.7) {\solutioncode};
	%%%%%%%%%%%% Intervention boxes
% 	\draw[draw=black, thick] (-1.2,0.5) rectangle (6.9, 3.8);
	\draw[draw=black, thick] (-13.5,-2.2) rectangle (-2.1, 0.9);
	%%%%%% T-out: intervention task
	[
        box/.style={rectangle,draw=black, minimum size=0.25cm},
        ]
    
        \foreach \x in {-13,-12.75,...,-10.75}{
            \foreach \y in {-2,-1.75,...,0.25}
                \node[box, fill=gray!40] at (\x,\y){};
            
        }
    	
     \foreach \y in {-2,-1.75,...,-0.25, 0}{
    		\node[box, fill=white] at (-13,\y){};
    		
    	}
    	
    \foreach \x in {-13,-12.75}{
    	\node[box, fill=white] at (\x,0){};
    		
    }

        \node[box, fill=white] at (-13,-2){};
        %%% bounding box
        \draw[draw=black, thick] (-13.1,-2.1) rectangle (-10.65,0.35);
         \node[draw, fill=red, star, star points=5,inner sep=0pt,minimum size=5pt] at (-13,-2){};
         \node[draw, fill=blue!50, dart, rotate=180, inner sep=0.2pt,minimum size=4pt] at (-12.75,0){};
    %%%%%%%% C-out: intervention code
    \node[text width=0.5cm, anchor=west, right] at (-10.3, -0.6)
    {\begin{boxcode}{3.68cm}{0.80}{1}
				\textcode{def }\DSLRun\textcode{()\{}\\
				\quad \DSLMove\\
			   \quad \DSLTurnLeft\\
			   \quad
		 \DSLRepeatUntil\textcode{(}\DSLBoolGoal\textcode{)\{}\\
				\quad \quad \framebox[8.0\width]{?}\\
				\quad \textcode{\}}\\
				% \vspace{-5mm}
				\textcode{\}}
				% \vspace{-10mm}
			\end{boxcode}};
	%%%%%%%%% Multiple Choice Question
	 \node[text width=0.5cm, anchor=west, right] at (-6.4, -0.6)
	 {\begin{boxcode}{3.98cm}{0.8}{1.15}
				\textcode{Q.} \text{Fill in the blank from: }\\
				\quad \quad \tikz\draw[black,fill=none] (0,0) circle (.5ex); \DSLMove\\
				\quad \quad \tikz\draw[black,fill=none] (0,0) circle (.5ex); \DSLTurnLeft\\
				\quad \quad \tikz\draw[black,fill=none] (0,0) circle (.5ex); \DSLTurnRight\\
				\\
			\end{boxcode}
	 };
	 %%%% captions for the task-codes
	\node[text width=0.5cm] at (-11.7,0.7) {$T^\text{quiz}$};
	\node[text width=2.6cm] at (-8.25,0.7) {$C^\text{quiz}\text{ with }1\text{ blank}$};
	\node[text width=1cm] at (-4.1,0.7) {Quiz};
% 	 %%%%%%%% connector arrows between task and intervention --- OLD
% 	\draw[draw=black,solid,line width=1mm,
% preaction={-triangle 90,thin,draw,shorten >=-1mm}] (-2.10, 2.4) to [out=60,in=120](-1.3,2.4);
% 	\draw[draw=black,solid,line width=1mm,
% preaction={-triangle 90,thin,draw,shorten >=-1mm}] (-1.25, 1.4) to [out=-120,in=-60] (-2.05, 1.4);
 %%%%%%%% connector arrows between task and intervention --- new
	\tikzstyle{doublearr}=[latex-latex, black, line width=1.5pt]
    \draw [doublearr, bend left]    (-2.1, 2.8) to (-2.1,-1);
    %%%%%%%%%%\draw (-7.5,2.7) rectangle (-6.5,3.2) 
    \node[text width=1cm] at (-14.5,2.8){\textbf{Task}};
    %%%%%%%%%%%%%%\draw (1.9,2.7) rectangle (2.9,3.2) 
    \node[text width=1.7cm] at (-14.25,-1){\textbf{Pop Quiz}};
    \end{tikzpicture}
    %\setlength{\belowcaptionskip}{-16pt} 

%\vspace{-1.5mm}
\caption{Illustration of our pop quiz based framework. The ``Task'' panel shows an input task $T^\text{in}$ from HOC~\cite{hourofcode_maze}, the student's current attempt \studentcode, and the solution code \solutioncode~(not revealed to the student). The student is currently unsuccessful in solving the task: the current attempt \studentcode~does not solve the visual puzzle within the maximal number of permitted blocks ($7$ blocks) and does not use any of the required constructs  (\DSLRepeatUntil and \DSLIfElse constructs). The ``Pop Quiz'' panel shows a pop quiz generated by our algorithm in the form of task-code pair ($T^\text{quiz}, C^\text{quiz}$) along with a multiple choice question, introducing the \DSLRepeatUntil~construct. After the student solves the pop quiz, they resume working on the input task. The framework would be invoked when a student needs help; importantly, the pop quizzes presented to the student are adaptive w.r.t. the student's current attempt \studentcode. Moreover, our algorithm generates pop quizzes that are easy to comprehend and solve, and $C^\text{quiz}$ sufficiently conceals \solutioncode.
}
% 
\label{fig:intro}
%\vspace{-4.2mm}
\end{figure*}






























%%%%%%%%%%%%%%%%%%%%%%%%%%%%%%%% OLD
% \begin{figure*}[!t]
% \centering
% \begin{tikzpicture}[
%         box/.style={rectangle,draw=black, minimum size=0.2cm},
%         ]
%         \foreach \x in {-7.1,-6.9,...,-5.1}{
%             \foreach \y in {1.5,1.7,...,3.3}
%                 \node[box, fill=gray!40] at (\x,\y){};
%         }
       
%         \foreach \y in {1.9,2.1,2.3}{
%                 \node[box, fill=white] at (-6.5,\y){};
%         }
%         \foreach \x in {-6.5,-6.3, -6.1}{
%                 \node[box, fill=white] at (\x,2.3){};
%         }
%         \node[box, fill=white] at (-6.1,2.1){};
%           \foreach \x in {-6.1,-5.9,-5.7}{
%                 \node[box, fill=white] at (\x,2.1){};
%         }
%         \foreach \y in {2.1,2.3,2.5, 2.7}{
%                 \node[box, fill=white] at (-5.7,\y){};
%         }
%         \foreach \x in {-5.7,-5.9,-6.1,-6.3}{
%                 \node[box, fill=white] at (\x,2.7){};
%         }
%         \node[box, fill=white] at (-6.3,2.9){};
%          %%% bounding box
%         \draw[draw=black, thick] (-7.2,1.4) rectangle (-5.2,3.4);
%         \node[draw, fill=red, star, star points=5,inner sep=0pt,minimum size=5pt] at (-6.3,2.9){};
%         \node[draw, fill=blue!50, dart, rotate=90, inner sep=0.2pt,minimum size=4pt] at (-6.5,1.9){};
%         %%%% code
%         \node[text width=0.5cm, anchor=west, right] at (-5.35, 2.6)
%     {\begin{boxcode}{3.68cm}{0.70}{1.0}
% 				\textcode{def }\DSLRun\textcode{()\{}\\
%                 \quad \DSLMove\\
%                 \quad \DSLMove\\
%                 \quad \DSLTurnRight\\
%                 \quad \ldots\\
%                 \quad \textcode{21 more action blocks}\\
% 				\textcode{\}}
% 				% \vspace{0.5em}
% 			\end{boxcode}};
% 	%%%%% bounding box around the task-code pair
% 	\draw[draw=black, thick] (-7.35,0.5) rectangle (-2.1, 3.8);
% 	%%%%%%%%% solution code
% 	\node[text width=0.5cm, anchor=east, left] at (-9.88, 2.25)
%     {\begin{boxcode}{3.68cm}{0.62}{0.65}
% 				\textcode{def }\DSLRun\textcode{()\{}\\
% 				\quad \DSLRepeatUntil\textcode{(}\DSLBoolGoal\textcode{)\{}\\
%                 \quad \quad
%               \DSLIf\textcode{(}\DSLBoolPathAhead\textcode{)\{}\\
% 				\quad \quad \quad \DSLMove\\ 
% 				% \textcode{\}}\\
% 				\quad \quad \textcode{\}}\\
% 				\quad \quad 
% 				\DSLElse\textcode{\{}\\
% 				\quad \quad \quad \DSLIf\textcode{(}\DSLBoolPathRight\textcode{)\{}\\
% 				\quad \quad \quad \quad \DSLTurnRight\\
% 				% \textcode{\}}\\
% 				\quad \quad \quad \textcode{\}}\\
% 				\quad \quad \quad
% 				\DSLElse\textcode{\{}\\
% 				\quad \quad \quad \quad \DSLTurnLeft\\
% 				% \textcode{\}}\\
% 				\quad \quad \quad \textcode{\}}\\
% 				%\vspace{-4.5mm}
% 				\quad \quad \textcode{\}}\\
% 				%\vspace{-6mm}				
% 				\quad \textcode{\}}\\
% 				%\vspace{-8mm}				
% 				\textcode{\}}
%             %\vspace{-6mm}				
% 			\end{boxcode}
% 			%\vspace{-6mm}
% 			};
% 	%%%%% bounding box around the solution-code pair
% 	\draw[draw=black, thick, pattern=north east lines, pattern color=gray!30] (-10.5,0.5) rectangle (-7.5, 3.8);
% 	%%%% captions for the task-codes
% 	\node[text width=0.5cm] at (-5.975,3.6) {$T^\text{in}$};
% 	\node[text width=0.5cm] at (-3.86,3.6) {\studentcode};
% 	\node[text width=0.5cm] at (-9.1,3.6) {\solutioncode};
% 	%%%%%%%%%%%% Intervention boxes
% 	\draw[draw=black, thick] (-1.2,0.5) rectangle (6.9, 3.8);
% 	%%%%%% T-out: intervention task
% 	[
%         box/.style={rectangle,draw=black, minimum size=0.2cm},
%         ]
%         \foreach \x in {-1,-0.8,...,0.8}{
%             \foreach \y in {1.5,1.7,...,3.3}
%                 \node[box, fill=gray!40] at (\x,\y){};
            
%         }
    	
%     	 \foreach \y in {3.1,2.9,...,1.5}{
%     		\node[box, fill=white] at (-1,\y){};
    		
%     	}
    	
%     	 \foreach \x in {-1,-0.8}{
%     		\node[box, fill=white] at (\x,3.1){};
    		
%     	}
%         \node[box, fill=white] at (-1,1.5){};
%         %%% bounding box
%         \draw[draw=black, thick] (-1.1,1.4) rectangle (0.9,3.4);
%          \node[draw, fill=red, star, star points=5,inner sep=0pt,minimum size=5pt] at (-1,1.5){};
%          \node[draw, fill=blue!50, dart, rotate=180, inner sep=0.2pt,minimum size=4pt] at (-0.8,3.1){};
%     %%%%%%%% C-out: intervention code
%     \node[text width=0.5cm, anchor=west, right] at (0.7, 2.65)
%     {\begin{boxcode}{3.68cm}{0.65}{1.0}
% 				\textcode{def }\DSLRun\textcode{()\{}\\
% 				\quad \DSLMove\\
% 			   \quad \DSLTurnLeft\\
% 			   \quad
% 			   \DSLRepeatUntil\textcode{(}\DSLBoolGoal\textcode{)\{}\\
% 				\quad \quad \framebox[8.0\width]{?}\\
% 				\quad \textcode{\}}\\
% 				\textcode{\}}
% 			\end{boxcode}};
% 	%%%%%%%%% Multiple Choice Question
% 	 \node[text width=0.5cm, anchor=west, right] at (3.45, 2.7)
% 	 {\begin{boxcode}{3.88cm}{0.70}{1.0}
% 				\quad \textcode{Q.} \text{Fill in the blank from: }\\
% 				\quad \quad \tikz\draw[black,fill=none] (0,0) circle (.5ex); \DSLMove\\
% 				\quad \quad \tikz\draw[black,fill=none] (0,0) circle (.5ex); \DSLTurnLeft\\
% 				\quad \quad \tikz\draw[black,fill=none] (0,0) circle (.5ex); \DSLTurnRight\\
% 			\\
% 			\end{boxcode}
% 	 };
% 	 %%%% captions for the task-codes
% 	\node[text width=0.5cm] at (0.125,3.6) {$T^\text{quiz}$};
% 	\node[text width=2.6cm] at (2.5,3.6) {$C^\text{quiz}\text{ with }1\text{ blank}$};
% 	\node[text width=1cm] at (5.44,3.6) {Quiz};
% % 	 %%%%%%%% connector arrows between task and intervention 
% 	\draw[draw=black,solid,line width=1mm,
% preaction={-triangle 90,thin,draw,shorten >=-1mm}] (-2.10, 2.4) to [out=60,in=120](-1.3,2.4);
% 	\draw[draw=black,solid,line width=1mm,
% preaction={-triangle 90,thin,draw,shorten >=-1mm}] (-1.25, 1.4) to [out=-120,in=-60] (-2.05, 1.4);
% 	%%%%% intervention box
%     %%%%%%%%%%%\draw (-7.5,2.7) rectangle (-6.5,3.2) 
%     \node[text width=2cm] at (-7,4.2){\textbf{Current Task}};
%     %%%%%%%%%%%%%%\draw (1.9,2.7) rectangle (2.9,3.2) 
%     \node[text width=1.4cm] at (2.7,4.2){\textbf{Pop Quiz}};
%     \end{tikzpicture}
%     %\setlength{\belowcaptionskip}{-16pt} 
%     \vspace{-3.2mm}
% \caption{Illustration of our pop quiz based framework. The ``Current Task'' panel shows the input task $T^\text{in}$ from HOC~\cite{hourofcode_maze}, the student's current attempt \studentcode, and the solution code \solutioncode~(not revealed to the student). The student is currently unsuccessful in solving the task with the maximal number of permitted blocks ($7$ blocks) as \studentcode~does not use any of the constructs \DSLRepeatUntil, \DSLIf~and \DSLElse~needed to solve the task correctly. The ``Pop Quiz'' panel shows a pop quiz generated by our algorithm in the form of task-code pair ($T^\text{quiz}, C^\text{quiz}$) along with a multiple choice question, introducing the \DSLRepeatUntil~construct. After the student solves the pop quiz, they resume working on the input task. The framework would be invoked when a student seeks help on their attempt; importantly, the pop quizzes presented to the student are adaptive w.r.t. the student's current attempt \studentcode. }
% %successfully
% %\vspace{-3mm}
% %Throughout this process, the solution code~\solutioncode~is never exposed to the student. 
% \label{fig:intro}
% \vspace{-2.8mm}
% \end{figure*}


}]

%%%%%%%%% ABSTRACT


Over the past few years, there has been a significant amount of research focused on studying the ReLU activation function, with the aim of achieving neural network convergence through over-parametrization. However, recent developments in the field of Large Language Models (LLMs) have sparked interest in the use of exponential activation functions, specifically in the attention mechanism.

Mathematically, we define the neural function $F: \R^{d \times m} \times  \mathbb{R}^d \rightarrow \mathbb{R}$ using an exponential activation function. Given a set of data points with labels $\{(x_1, y_1), (x_2, y_2), \dots, (x_n, y_n)\} \subset \mathbb{R}^d \times \mathbb{R}$ where $n$ denotes the number of the data. Here $F(W(t),x)$ can be expressed as $F(W(t),x) := \sum_{r=1}^m a_r \exp(\langle w_r, x \rangle)$, where $m$ represents the number of neurons, and $w_r(t)$ are weights at time $t$. It's standard in literature that $a_r$ are the fixed weights and it's never changed during the training. We initialize the weights $W(0) \in \mathbb{R}^{d \times m}$ with random Gaussian distributions, such that $w_r(0) \sim \mathcal{N}(0, I_d)$ and initialize $a_r$ from random sign distribution for each $r \in [m]$.

Using the gradient descent algorithm, we can find a weight $W(T)$ such that $\| F(W(T), X) - y \|_2 \leq \epsilon$ holds with probability $1-\delta$, where $\epsilon \in (0,0.1)$ and $m = \Omega(n^{2+o(1)}\log(n/\delta))$. To optimize the over-parametrization bound $m$, we employ several tight analysis techniques from previous studies [Song and Yang arXiv 2019, Munteanu, Omlor, Song and Woodruff ICML 2022]. 

 

\section{Introduction}
\label{sec:introduction}
% \begin{itemize}
%     % Diffusion of FL
%     \item {\st{Diffusion of FL}}
%     % Security threats to FL
%     \item {\st{Security threats to FL with particular focus on model poisoning}}
%     % Limitations of existing countermeasures
%     \item {\st{Current countermeasures (e.g., KRUM) and their limitations}}
%     % Proposed method and its advantages
%     \item {\st{Intuitive description of the proposed method and its difference (i.e., advantages) w.r.t. state of the art}}
%     % Main contributions
%     \item {\st{Summary of the main contributions of this work}}
%     % Paper's structure and organization
%     \item {\st{Paper's structure and organization}}
% \end{itemize}

% Diffusion of FL
Recently, {\em federated learning} (FL) has emerged as the leading paradigm for training distributed, large-scale, and privacy-preserving machine learning (ML) systems~\cite{mcmahan2017googleai,mcmahan2017aistats}. 
The core idea of FL is to allow multiple edge clients to collaboratively train a shared, global model without disclosing their local private training data.
%Specifically, an FL system consists of a central server and many edge clients; 
A typical FL round involves the following steps: {\em(i)} the server randomly picks some clients and sends them the current, global model; {\em(ii)} each selected client locally trains its model with its own private data; then, it sends the resulting local model to the server;\footnote{Whenever we refer to global/local model, we mean global/local model {\em parameters}.} {\em(iii)} the server updates the global model by computing an \emph{aggregation function}, usually the average (FedAvg), on the local models received from clients.
% \begin{enumerate}
%     \item[{\em(i)}] the server sends the current, global model to the clients and appoints some of them for training;
%     \item[{\em(ii)}] each selected client locally trains its copy of the global model with its own private data; then, it sends the resulting local model back to the server;\footnote{Whenever we refer to global/local model, we mean global/local model {\em parameters}.}
%     \item[{\em(iii)}] the server updates the global model by computing an \emph{aggregation function} on the local models received from clients (by default, the average, also referred to as FedAvg~\cite{mcmahan2017aistats}).
% \end{enumerate}
This process goes on until the global model converges. %(e.g., after a certain number of rounds or other similar stopping criteria).
%\\
% The advantages of FL over the traditional, centralized learning paradigm are undoubtedly clear in terms of flexibility/scalability (clients can join/disconnect from the FL network dynamically), network communications (only model weights\footnote{We will use \textit{parameters} and \textit{weights} interchangeably.} are exchanged between clients and server), and privacy (each client's private training data is kept local at the client's end and not uploaded to the server).
\\
% Security threats to FL
%However, the growing adoption of FL also raises security concerns~\cite{costa2022covert}, particularly about its confidentiality, integrity, and availability.
Although its advantages over standard ML, FL also raises security concerns~\cite{costa2022covert}. %, particularly about its confidentiality, integrity, and availability~\cite{costa2022covert}.
% OLD, LONG VERSION
% Indeed, some work deals with privacy leakage that may expose the local data of some clients~\cite{melis2019sp}. 
% A large body of work, instead, investigates attacks that usually aim to detriment the predictive accuracy of the learned global model. For instance, \emph{data poisoning} attacks achieve this goal by letting an adversary pollute the training set of some corrupt FL clients with maliciously crafted examples~\cite{jagielski2018sp}.
% Similarly, in \emph{model poisoning} the attacker attempts to tweak the global model weights~\cite{bhagoji2019pmlr} by directly perturbing the local model's weights of some infected FL clients before these are sent to the central server for aggregation, usually via so-called Byzantine attacks. 
% It turns out that Byzantine model poisoning attacks severely impact standard FedAvg; therefore, more robust aggregation functions must be designed to make FL systems secure.
Here, we focus on \emph{untargeted model poisoning} attacks~\cite{bhagoji2019pmlr}, where an adversary attempts to tweak the global model weights %\footnote{We will use the terms \textit{parameters} and \textit{weights} interchangeably.} 
by directly perturbing the local model's parameters of some infected clients before these are sent to the central server for aggregation.
In doing so, the adversary aims to jeopardize the global model \textit{indiscriminately} at inference time.
Such model poisoning attacks severely impact standard FedAvg; therefore, more robust aggregation functions must be designed to secure FL systems.
\\
% In this paper, we focus on designing a novel robust aggregation scheme at the server's end to contrast the effect of Byzantine model poisoning attacks.
%
% Current countermeasures and their limitations
%Several countermeasures have been proposed in the literature to combat model poisoning attacks on FL systems.
% Some methods use simple statistics more robust than plain average to smooth the impact of malicious updates (e.g., Trimmed Mean and FedMedian~\cite{yin2018icml}). 
% Other defenses implement outlier detection techniques to discard malicious updates from the aggregation performed at the server's end. Those are either based on heuristics (e.g., Krum/Multi-Krum~\cite{blanchard2017nips} and Bulyan~\cite{mhamdi2018pmlr}) or data-driven approaches (e.g., K-means clustering~\cite{shen2016acm} or DnC via spectral analysis~\cite{shejwalkar2021ndss}). 
% Finally, some strategies rely on a centralized ``source of trust'' to spot potential malicious updates (e.g., FLTrust~\cite{cao2020fltrust}).
% Several countermeasures have been proposed in the literature to combat model poisoning attacks on FL systems, i.e., to discard possible malicious local updates from the aggregation performed at the server's end. 
% These techniques range from simple statistics more robust than plain average (e.g., Trimmed Mean and FedMedian~\cite{yin2018icml}) to outlier detection heuristics (e.g., Krum/Multi-Krum~\cite{blanchard2017nips} and Bulyan~\cite{mhamdi2018pmlr}) or data-driven approaches (e.g., spectral analysis via K-means clustering~\cite{shen2016acm} or spectral analysis), or methods based on ``source of trust'' (e.g., FLTrust~\cite{cao2020fltrust}).
% OLD, LONG VERSION
%Several countermeasures have been proposed in the literature to combat Byzantine model poisoning attacks on FL systems.
% Descriptive statistics
% For example, Trimmed Mean and FedMedian aggregate local model updates using more robust statistics than standard average~\cite{yin2018icml}.
%
% % Heuristics for outlier detection
% Many existing Byzantine-resilient strategies implement some outlier detection heuristics to discard the model updates sent by potentially malicious clients from the input of the aggregation function.
% One of the most popular heuristics is Krum~\cite{blanchard2017nips}.
% This strategy tries to mitigate the impact of Byzantine attacks by selecting as a global model the local model with the smallest sum of Euclidean distances to {\em all} the other local models.
% Although powerful, Krum requires the server to know (or, at least, estimate) the number of malicious FL clients upfront, which is generally impossible in a realistic attack scenario. %
% Moreover, Krum may become ineffective for complex, high-dimensional model parameter spaces due to the curse of dimensionality.
% Bulyan~\cite{mhamdi2018pmlr} tries to overcome this issue by combining Krum with a variant of Trimmed Mean.
% % Data-driven outlier detection
% Other strategies use data-driven outlier detection techniques -- e.g., via K-means clustering~\cite{shen2016acm} -- to spot potential malicious local model updates. 
% %For instance, Shen et al. propose to cluster local model updates with K-means and thus identify outliers.
%
% % Other techniques
% As far as the server is concerned, any local model received can be from a potential malicious client. 
% FLTrust~\cite{cao2020fltrust} assumes the server acts as a client, i.e., trains a local model on an additional {\em trustworthy} dataset at the server's end and compares it against all the local models from other clients. 
% This way, the server can rely on some ``source of trust'' when discarding potentially malicious clients.
%\\
% Limitations of existing Byzantine-resilient strategies
Unfortunately, existing defense mechanisms either rely on simple heuristics (e.g., Trimmed Mean and FedMedian by~\cite{yin2018icml}) or need strong and unrealistic assumptions to work effectively (e.g., foreknowledge or estimation of the number of malicious clients in the FL system, as for Krum/Multi-Krum~\cite{blanchard2017nips} and Bulyan~\cite{mhamdi2018pmlr}, which, however, cannot exceed a fixed threshold).
Furthermore, outlier detection methods using K-means clustering~\cite{shen2016acm} or spectral analysis like DnC~\cite{shejwalkar2021ndss} do not directly consider the temporal evolution of local model updates received.
Finally, strategies like FLTrust~\cite{cao2020fltrust} require the server to collect its own dataset and act as a proper client, thereby altering the standard FL protocol.
\\
% OLD, LONG VERSION
% Overall, existing Byzantine-resilient strategies are either simple heuristics (e.g., FedMedian) or, if they are more complex, they rely on strong and unrealistic assumptions to work effectively (e.g., knowing the number of malicious clients in the FL system in advance, as for Krum and alike).
% Furthermore, data-driven outlier detection methods do not consider the temporary evolution of local model updates received (e.g., K-means clustering). 
% Finally, strategies like FLTrust requires the server to collect its own dataset and act as a proper client, thereby altering the standard FL protocol.
%
% Description of the proposed method
This work introduces a novel pre-aggregation \textit{filter} robust to untargeted model poisoning attacks. Notably, this filter $(i)$ operates without requiring prior knowledge or constraints on the number of malicious clients and $(ii)$ inherently integrates temporal dependencies. 
The FL server can employ this filter as a preprocessing step before applying \textit{any} aggregation function, be it standard like FedAvg or robust like Krum or Bulyan.
Specifically, we formulate the problem of identifying corrupted updates as a multidimensional (i.e., matrix-valued) time series anomaly detection task. 
The key idea is that legitimate local updates, resulting from well-calibrated iterative procedures like stochastic gradient descent (SGD) with an appropriate learning rate, show \textit{higher predictability} compared to malicious updates. This hypothesis stems from the fact that the sequence of gradients (thus, model parameters) observed during legitimate training exhibit regular patterns, as validated in Section~\ref{subsec:intuition}. %until convergence. 
%This regularity may be more pronounced for smooth convex loss functions, but it can still be captured within an appropriate time window, even for more complex and convoluted loss surfaces. 
%We provide evidence of this claim in Appendix~B, where we show that the average mutual information (i.e., ``predictability''), calculated over pairs of legitimate model updates sent at different FL rounds, is significantly higher than the corresponding computation for a malicious client.
\\
Inspired by the matrix autoregressive (MAR) framework for multidimensional time series forecasting~\cite{chen2021je}, we propose the FLANDERS ({\em \textbf{F}ederated \textbf{L}earning meets \textbf{AN}omaly \textbf{DE}tection for a \textbf{R}obust and \textbf{S}ecure}) filter.
The main advantages of FLANDERS over existing strategies like FLDetector~\cite{zhao2020multivariate} are its resilience to large-scale attacks, where $50\%$ or more FL participants are hostile, and the capability of working under realistic non-iid scenarios.
We attribute such a capability to two key factors: $(i)$ FLANDERS works without knowing a priori the ratio of corrupted clients, and $(ii)$ it embodies temporal dependencies between intra- and inter-client updates, quickly recognizing local model drifts caused by evil players. Below, we summarize our main contributions:

\begin{itemize}
\item[{\em(i)}]
We provide empirical evidence that the sequence of models sent by legitimate clients is more predictable than those of malicious participants performing untargeted model poisoning attacks.
\\
\item[{\em(ii)}] 
We introduce FLANDERS, the first pre-aggregation filter for FL robust to untargeted model poisoning based on multidimensional time series anomaly detection.
\\
\item[{\em(iii)}] 
We integrate FLANDERS into Flower,\footnote{\scriptsize{\url{https://flower.dev/}}} a popular FL simulation framework for reproducibility.
\\
\item[{\em(iv)}] 
We show that FLANDERS improves the robustness of the existing aggregation methods under multiple settings: different datasets, client's data distribution (non-iid), models, and attack scenarios.
\\
\item[{\em(v)}] 
We publicly release all the implementation code of FLANDERS along with our experiments.\footnote{\scriptsize{\url{https://anonymous.4open.science/r/flanders_exp-7EEB}}}
\end{itemize}

% Paper's structure and organization
The remainder of the paper is structured as follows. %some related work and the current state-of-the-art solutions to security issues that FL entails. 
Section~\ref{sec:background} covers background and preliminaries. 
In Section~\ref{sec:related}, we discuss related work.
Section~\ref{sec:problem} and Section~\ref{sec:method} describe the problem formulation and the method proposed. % to tackle it. 
Section~\ref{sec:experiments} gathers experimental results. %, and Section~\ref{sec:limitations} discusses some limitations of this work.
Finally, we conclude in Section~\ref{sec:conclusion}.
 %discusses the limitations of this work and draws future research directions.
%reports conclusions and draws perspectives for future research directions.

%%%%%%% OLD %%%%%%%
%to overcome the resilience of Byzantine failures in distributed Stochastic Gradient Descent computations. 
% The strength of Krum is its time complexity, which is linear in the gradient dimension. 
% However, the robustness of the approach is guaranteed for gradient-based learning applications only when the majority of the clients are not compromised. 
% Besides, the aggregation mechanism of Krum, as well as that of similar methods, is robust from a coarse-grained perspective and does not provide solutions to errors and perturbations that may occur at inference time.
%A related approach to~\cite{blanchard2017nips} is the work of Su et al.~\cite{su2016dc}. Here, the authors propose an iterated approximate agreement to tackle a multi-layer scenario attacked by Byzantine agents. 
%However, the method works efficiently on the sole discrete context and it is inapplicable to continuous state environments.
%\gabri{Maybe, we should just talk about the main limitations of existing countermeasures without digging into their details (or, we can just mention Krum as this is the most popular one). I will move the description of all these methods to the Related Work section.}

\section{Related Work}\label{sec:related-work}



Over the last few years, several benchmarks for stream processing frameworks have been proposed and stream processing benchmarking studies have been conducted. The differentiation between benchmarks and experimental studies applying them is sometimes blurry. Many publications that present benchmarks perform also an experimental study with them. On the other hand, many experimental studies utilize existing benchmarks, but modify them.
Nevertheless, we structure this section into two parts: First, we give an overview of stream processing benchmarks to justify our benchmark selection for this study. Second, we discuss related stream processing benchmarking studies.

\subsection{Related Work on Stream Processing Benchmarks}

Besides the Theodolite benchmarks for event-driven microservices used in this study, several other benchmarks for stream processing frameworks have been proposed.
\cref{tab:related-benchmarks} summarizes characteristics of the discussed benchmarks. 


\begin{table*}
	\begin{threeparttable}[b]
		\caption{Overview of the characteristics and implementations of stream processing benchmarks.}
		\label{tab:related-benchmarks}
		\footnotesize
		\newcommand{\cmark}{\ding{51}}%
		\newcommand{\xmark}{\ding{55}}%
		\newcommand{\qmark}{\makebox[0pt][l]{\textbf{\textit{?}}}\phantom{\cmark}}%
		
		\newcommand{\txnote}[1]{\makebox[0pt][l]{\tnote{#1}}}
		
		\newcommand\undefcolumntype[1]{\expandafter\let\csname NC@find@#1\endcsname\relax}
		\newcommand\forcenewcolumntype[1]{\undefcolumntype{#1}\newcolumntype{#1}}
		
		\newcommand*\rot{\rotatebox{90}}
		\newcolumntype{L}{>{\raggedright\arraybackslash}X}
		\newcolumntype{R}{>{\raggedleft\arraybackslash}X}
		\newcolumntype{C}{>{\centering\arraybackslash}X}
		\newcolumntype{o}{p{0pt}}
		\renewcommand{\arraystretch}{1.2}
		\newcommand{\fnoptional}{a}
		\newcommand{\fnbeam}{b}
		\newcommand{\fnriottasksamples}{d}
		\newcommand{\fnbeamnexmark}{c}
		\begin{tabularx}{\textwidth}{ll o C o C o CCC o CCCCCCC o C o CCC}
			\toprule
			&&& && && \multicolumn{3}{c}{Messaging} && \multicolumn{7}{c}{Stream processing framework} && && \multicolumn{3}{c}{Cloud-native} \\
			\cmidrule{8-10} \cmidrule{12-18} \cmidrule{22-24}
			Benchmark & Published && \rot{Task samples} && \rot{Open source} && \rot{Kafka} & \rot{Others} & \rot{None} && \rot{Flink} & \rot{Spark} & \rot{Storm} & \rot{Samza} & \rot{Kafka Streams} & \rot{Hazelcast Jet} & \rot{Others} && \rot{Database} && \rot{Containers} & \rot{Kubernetes} & \rot{Others} \\
			\midrule
			Theodolite \cite{BDR2021} & \citeyear{BDR2021}
			& %
			& 4
			& %
			& \cmark %
			& %
			& \cmark %
			& %
			& %
			& %
			& \cmark %
			& %
			& %
			& \cmark\txnote{\fnbeam} %
			& \cmark %
			& \cmark %
			& \cmark\txnote{\fnbeam} %
			& %
			& \phantom{\cmark}\txnote{\fnoptional} %
			& %
			& \cmark %
			& \cmark %
			& %
			\\
			Beam Nexmark \cite{BeamNexmark2022} & \citeyear{BeamNexmark2022}\txnote{\fnbeamnexmark}
			& %
			& 13
			& %
			& \cmark %
			& %
			& \cmark %
			& \cmark %
			& %
			& %
			& \cmark\txnote{\fnbeam} %
			& \cmark\txnote{\fnbeam} %
			& %
			& \qmark\txnote{\fnbeam} %
			& %
			& \qmark\txnote{\fnbeam} %
			& \cmark\txnote{\fnbeam} %
			& %
			& %
			& %
			& %
			& %
			& %
			\\
			ESPBench \cite{Hesse2021} & \citeyear{Hesse2021}
			& %
			& 5
			& %
			& \cmark %
			& %
			& \cmark %
			& %
			& %
			& %
			& \cmark\txnote{\fnbeam} %
			& \cmark\txnote{\fnbeam} %
			& %
			& \qmark\txnote{\fnbeam} %
			& %
			& \cmark\txnote{\fnbeam} %
			& \qmark\txnote{\fnbeam} %
			& %
			& \cmark %
			& %
			& %
			& %
			& %
			\\
			OSPBench \cite{vanDongen2020} & \citeyear{vanDongen2020}
			& %
			& 5
			& %
			& \cmark %
			& %
			& \cmark %
			& %
			& %
			& %
			& \cmark %
			& \cmark %
			& %
			& %
			& \cmark %
			& %
			& %
			& %
			& %
			& %
			& \cmark %
			& %
			& \cmark %
			\\
			DSPBench \cite{Bordin2020} & \citeyear{Bordin2020}
			& %
			& 5
			& %
			& \cmark %
			& %
			& \cmark %
			& %
			& %
			& %
			& %
			& \cmark %
			& \cmark %
			& %
			& %
			& %
			& %
			& %
			& \cmark %
			& %
			& %
			& %
			& %
			\\
			\citet{Shahverdi2019} & \citeyear{Shahverdi2019}
			& %
			& 1
			& %
			& \cmark %
			& %
			& \cmark %
			& %
			& %
			& %
			& \cmark %
			& \cmark %
			& \cmark %
			& %
			& \cmark %
			& \cmark %
			& %
			& %
			& \cmark %
			& %
			& %
			& %
			& %
			\\
			\citet{Karimov2018} & \citeyear{Karimov2018}
			& %
			& 2
			& %
			& %
			& %
			& %
			& %
			& \cmark %
			& %
			& \cmark %
			& \cmark %
			& \cmark %
			& %
			& %
			& %
			& %
			& %
			& %
			& %
			& %
			& %
			& %
			\\
			RIoTBench \cite{Shukla2017} & \citeyear{Shukla2017}
			& %
			& 4\txnote{\fnriottasksamples} %
			& %
			& \cmark %
			& %
			& %
			& \cmark %
			& %
			& %
			& %
			& %
			& \cmark %
			& %
			& %
			& %
			& %
			& %
			& \cmark %
			& %
			& %
			& %
			& %
			\\
			YSB \cite{Chintapalli2016} & \citeyear{Chintapalli2016}
			& %
			& 1
			& %
			& \cmark %
			& %
			& \cmark %
			& %
			& %
			& %
			& \cmark %
			& \cmark %
			& \cmark %
			& %
			& %
			& %
			& %
			& %
			& \cmark %
			& %
			& %
			& %
			& %
			\\
			SparkBench \cite{Li2015} & \citeyear{Li2015}
			& %
			& 10
			& %
			& \cmark %
			& %
			& %
			& %
			& \cmark %
			& %
			& %
			& \cmark %
			& %
			& %
			& %
			& %
			& %
			& %
			& %
			& %
			& %
			& %
			& %
			\\
			StreamBench \cite{Lu2014} & \citeyear{Lu2014}
			& %
			& 7
			& %
			& %
			& %
			& \cmark %
			& %
			& %
			& %
			& %
			& \cmark %
			& \cmark %
			& %
			& %
			& %
			& %
			& %
			& %
			& %
			& %
			& %
			& %
			\\
			Linear Road \cite{Arasu2004} & \citeyear{Arasu2004}
			& %
			& 5
			& %
			& %
			& %
			& %
			& %
			& \cmark %
			& %
			& %
			& %
			& %
			& %
			& %
			& %
			& \cmark %
			& %
			& %
			& %
			& %
			& %
			& %
			\\
			\bottomrule
		\end{tabularx}
		\begin{tablenotes}\footnotesize
			\item[\fnoptional] optional
			\item[\fnbeam] using Apache Beam
			\item[\fnbeamnexmark] the Beam Nexmark benchmarks are based on the Nexmark paper \cite{Tucker2010} published in \citeyear{Tucker2010}
			\item[\fnriottasksamples] RIoTBench's 4 application benchmarks are composed of 27 microbenchmarks
		\end{tablenotes}
	\end{threeparttable}
\end{table*}



StreamBench~\cite{Lu2014} is one of the earliest benchmarks for modern stream processing frameworks. While originally only implemented for Spark and Storm, it has later been used to benchmark Apache Apex, Beam, Flink, and Samza as well \cite{Hesse2019, Qian2016}.
As its name suggests, SparkBench~\cite{Li2015} is a benchmark tailored to Apache Spark.
The Yahoo Streaming Benchmark (YSB) \cite{Chintapalli2016} is frequently used and adapted in research \cite{Lopez2016, Yang2017, Karakaya2017, Nasiri2019, Zeuch2019, Chu2020, vanDongen2020}.
Worth highlighting is the work of \citet{Shahverdi2019}, who extend YSB with implementations for the frameworks Kafka Streams and Hazelcast Jet. As discussed in \cref{sec:frameworks}, these frameworks are particularly suited for building event-driven microservices.
RIoTBench \cite{Shukla2017} provides four application benchmarks for Storm composed of 27~small task samples. \citet{Nasiri2019} adopt RIoTBench for Flink and Spark.
\citet{Karimov2018} present a benchmark with two task samples, derived from a real industrial context, yet without providing open-source implementations.

More recently, DSPBench \cite{Bordin2020}, OSPBench~\cite{vanDongen2020}, and ESPBench \cite{Hesse2021} have been proposed.
DSPBench contains 15~benchmarks, which resample typical stream processing applications, derived from reviewing the literature.
OSPBench provides benchmarks for analyzing traffic sensor data. Besides evaluations of latency, throughput, and resource usage, \citeauthor{vanDongen2020} used OSPBench to also evaluate scalability~\cite{vanDongen2021b} and fault recovery~\cite{vanDongen2021a}.
In contrast to most other benchmarks, OSPBench provides implementations for the rather new framework Kafka Streams, which is also evaluated in this study.
The Enterprise Stream Processing Benchmark (ESPBench) builds upon the Senska benchmark \cite{Hesse2018}.
It is special in the sense that it integrates a relational database management system.
In contrast to most other benchmarks, ESPBench's task samples are implemented with Apache Beam. While \citet{Hesse2021} only perform evaluations with Spark, Flink, and Hazelcast Jet, we expect that also other Beam runners can be used to run the benchmark.

The Nexmark benchmark \cite{Tucker2010} has originally been proposed as the \textit{Niagara Extension to the XMark benchmark} addressed to first-generation stream processing systems (see the survey of \citet{Fragkoulis2023} for a discussion of first and second-generation stream processing systems).
The Apache Beam community adapted and extended Nexmark with implementations for Beam to benchmark the performance of different runners~\cite{BeamNexmark2022}.
Documentation and benchmark results are provided for the direct runner as well as for the Flink, the Spark, and the Google Cloud Dataflow runners.
However, running the benchmark with other runners should be possible as well.
Recently, there seems to be an effort to implement the Nexmark task samples with other frameworks in an open-source project.\footnote{\url{https://github.com/nexmark/nexmark}}
However, currently this project only provides implementations for Apache Flink.
Moreover, \citet{Gencer2021} implemented the Nexmark benchmark for their performance evaluation of Hazelcast Jet.

Worth mentioning is also the Linear Road benchmark presented by \citet{Arasu2004}. Although published years before all modern stream processing frameworks considered in this work have been released, it is still used in research \cite{Zhang2017,Zeuch2019,Sax2020} and compared to newer benchmarks \cite{Bordin2020,Hesse2021}.
\citet{Pagliari2020} and \citet{Garcia2022a, Garcia2022b} present approaches to generate benchmarks.






From \cref{tab:related-benchmarks}, we can see that a lot of open-source benchmarks have been proposed. Apart from the Theodolite benchmarks, none of these benchmarks is particularly addressed to scalability.
Often originating in data management research, many benchmarks are defined as ``queries'' over data streams~\cite{Tucker2010,Karimov2018,Hesse2021}.
Most benchmarks include a messaging system as a middleware component between workload generation and stream processing framework. In the vast majority of cases, this is Apache Kafka.
\citet{Karimov2018} exclude such a system to not let it become the benchmark's bottleneck. Our Theodolite benchmarks purposely include Kafka to represent more realistic event-driven microservice deployments~\cite{BDR2021}.
Flink, Spark, and Storm are by far the most supported frameworks. Only a few benchmarks exist for Samza, Kafka Streams, and Hazelcast Jet, which are frameworks particularly suited for implementing event-driven microservice. Our Theodolite benchmarks are the only ones providing implementations for all of them.
While some benchmarks include an interaction with a database in their setup, others do not.
With the Theodolite benchmarks, a database can optionally be used as we did in a previous study~\cite{IC2E2022FaaSStreaming}.
Besides our Theodolite benchmarks, there is only one other benchmark (OSPBench) that is provided as container images to be used in a cloud-native setting. No other benchmark provides Kubernetes manifests.





\subsection{Related Work on Stream Processing Benchmarking}


\begin{table*}
	\begin{threeparttable}[b]
		\caption{Overview of employed benchmarks, frameworks, and experimental setup of stream processing benchmarking studies.}
		\label{tab:related-experiments}
		\footnotesize
		\newcommand{\cmark}{\ding{51}}%
		\newcommand{\xmark}{\ding{55}}%
		\newcommand{\qmark}{\makebox[0pt][l]{\textbf{\textit{?}}}\phantom{\cmark}}%
		
		\newcommand{\txnote}[1]{\makebox[0pt][l]{\tnote{#1}}}
		
		\newcommand\undefcolumntype[1]{\expandafter\let\csname NC@find@#1\endcsname\relax}
		\newcommand\forcenewcolumntype[1]{\undefcolumntype{#1}\newcolumntype{#1}}
		
		\newcommand*\rot{\rotatebox{90}}
		\newcolumntype{L}{>{\raggedright\arraybackslash}X}
		\newcolumntype{R}{>{\raggedleft\arraybackslash}X}
		\newcolumntype{C}{>{\centering\arraybackslash}X}
		\newcolumntype{o}{p{0pt}}
		\renewcommand{\arraystretch}{1.2}
		\newcommand{\fnvandenpoel}{a}
		\newcommand{\fnbeam}{b}
		\begin{tabularx}{\textwidth}{ll o CCCCCCCCCCCCC o CCCCCCC o CCCCCC}
			\toprule
			&&& \multicolumn{13}{c}{Benchmark} && \multicolumn{7}{c}{Framework} && \multicolumn{6}{c}{Execution} \\
			\cmidrule{4-16} \cmidrule{18-24} \cmidrule{26-31}
			Publication & Year &&
			\rot{Theodolite \cite{BDR2021}} &
			\rot{Beam Nexmark \cite{BeamNexmark2022}} &
			\rot{ESPBench \cite{Hesse2021}} &
			\rot{OSPBench \cite{vanDongen2020}} &
			\rot{DSPBench \cite{Bordin2020}} &
			\rot{\citet{Shahverdi2019}} &
			\rot{\citet{Karimov2018}} &
			\rot{RIoTBench \cite{Shukla2017}} &
			\rot{YSB \cite{Chintapalli2016}} &
			\rot{SparkBench \cite{Li2015}} &
			\rot{StreamBench \cite{Lu2014}} &
			\rot{Linear Road \cite{Arasu2004}} &
			\rot{Others}
			&&
			\rot{Flink} &
			\rot{Spark} &
			\rot{Storm} &
			\rot{Samza} &
			\rot{Kafka Streams} &
			\rot{Hazelcast Jet} &
			\rot{Others}
			&&
			\rot{Cloud environment} &
			\rot{Distributed} &
			\rot{Different resource amounts} &
			\rot{\dots in isolated experiments} &
			\rot{Different load intensities} &
			\rot{\dots in isolated experiments}
			\\
			\midrule
			This work &
				& %
				& \cmark %
				& %
				& %
				& %
				& %
				& %
				& %
				& %
				& %
				& %
				& %
				& %
				& %
				& %
				& \cmark %
				& %
				& %
				& \cmark\txnote{\fnbeam} %
				& \cmark %
				& \cmark %
				& %
				& %
				& \cmark %
				& \cmark %
				& \cmark %
				& \cmark %
				& \cmark %
				& \cmark %
			\\
			\citet{IC2E2022FaaSStreaming} & \citeyear{IC2E2022FaaSStreaming}
				& %
				& \cmark %
				& %
				& %
				& %
				& %
				& %
				& %
				& %
				& %
				& %
				& %
				& %
				& %
				& %
				& \cmark\txnote{\fnbeam} %
				& %
				& %
				& \cmark\txnote{\fnbeam} %
				& %
				& %
				& \cmark\txnote{\fnbeam} %
				& %
				& \cmark %
				& \cmark %
				& \cmark %
				& \cmark %
				& \cmark %
				& \cmark %
			\\
			\citet{Hesse2021} & \citeyear{Hesse2021}
				& %
				& %
				& %
				& \cmark %
				& %
				& %
				& %
				& %
				& %
				& %
				& %
				& %
				& %
				& %
				& %
				& \cmark\txnote{\fnbeam} %
				& \cmark\txnote{\fnbeam} %
				& %
				& %
				& %
				& \cmark\txnote{\fnbeam} %
				& %
				& %
				& %
				& \cmark %
				& \cmark %
				& \cmark %
				& %
				& %
			\\
			van Dongen\tnote{\fnvandenpoel} \cite{vanDongen2021b} & \citeyear{vanDongen2021b}
				& %
				& %
				& %
				& %
				& \cmark %
				& %
				& %
				& %
				& %
				& %
				& %
				& %
				& %
				& %
				& %
				& \cmark %
				& \cmark %
				& %
				& %
				& \cmark %
				& %
				& %
				& %
				& \cmark %
				& \cmark %
				& \cmark %
				& %
				& \cmark %
				& \cmark %
			\\
			van Dongen\tnote{\fnvandenpoel} \cite{vanDongen2021a} & \citeyear{vanDongen2021a}
				& %
				& %
				& %
				& %
				& \cmark %
				& %
				& %
				& %
				& %
				& %
				& %
				& %
				& %
				& %
				& %
				& \cmark %
				& \cmark %
				& %
				& %
				& \cmark %
				& %
				& %
				& %
				& \cmark %
				& \cmark %
				& %
				& %
				& \cmark %
				& %
			\\
			\citet{Bordin2020} & \citeyear{Bordin2020}
				& %
				& %
				& %
				& %
				& %
				& \cmark %
				& %
				& %
				& %
				& %
				& %
				& %
				& %
				& %
				& %
				& %
				& \cmark %
				& \cmark %
				& %
				& %
				& %
				& %
				& %
				& \cmark %
				& \cmark %
				& %
				& %
				& \cmark %
				& \cmark %
			\\
			\citet{Chu2020} & \citeyear{Chu2020}
				& %
				& %
				& %
				& %
				& %
				& %
				& %
				& %
				& %
				& \cmark %
				& %
				& %
				& %
				& \cmark %
				& %
				& \cmark %
				& %
				& \cmark %
				& %
				& %
				& %
				& \cmark %
				& %
				& %
				& \cmark %
				& \cmark %
				& %
				& %
				& %
			\\
			\citet{Vikash2020} & \citeyear{Vikash2020}
				& %
				& %
				& %
				& %
				& %
				& %
				& %
				& %
				& %
				& %
				& %
				& %
				& %
				& \cmark %
				& %
				& \cmark %
				& \cmark %
				& \cmark %
				& %
				& %
				& %
				& \cmark %
				& %
				& %
				& \cmark %
				& %
				& %
				& \cmark %
				& \cmark %
			\\
			van Dongen\tnote{\fnvandenpoel} \cite{vanDongen2020} & \citeyear{vanDongen2020}
				& %
				& %
				& %
				& %
				& \cmark %
				& %
				& %
				& %
				& %
				& %
				& %
				& %
				& %
				& %
				& %
				& \cmark %
				& \cmark %
				& %
				& %
				& \cmark %
				& %
				& %
				& %
				& \cmark %
				& \cmark %
				& \cmark %
				& %
				& %
				& %
			\\
			\citet{Nasiri2019} & \citeyear{Nasiri2019}
				& %
				& %
				& %
				& %
				& %
				& %
				& %
				& %
				& \cmark %
				& \cmark %
				& %
				& %
				& %
				& %
				& %
				& \cmark %
				& \cmark %
				& \cmark %
				& %
				& %
				& %
				& %
				& %
				& %
				& \cmark %
				& \cmark %
				& \cmark %
				& \cmark %
				& \cmark %
			\\
			\citet{Shahverdi2019} & \citeyear{Shahverdi2019}
				& %
				& %
				& %
				& %
				& %
				& %
				& \cmark %
				& %
				& %
				& %
				& %
				& %
				& %
				& %
				& %
				& \cmark %
				& \cmark %
				& \cmark %
				& %
				& \cmark %
				& \cmark %
				& %
				& %
				& \cmark %
				& \cmark %
				& \cmark %
				& \cmark %
				& %
				& %
			\\
			\citet{Zeuch2019} & \citeyear{Zeuch2019}
				& %
				& %
				& %
				& %
				& %
				& %
				& %
				& %
				& %
				& \cmark %
				& %
				& %
				& \cmark %
				& \cmark %
				& %
				& \cmark %
				& \cmark %
				& \cmark %
				& %
				& %
				& %
				& \cmark %
				& %
				& %
				& \cmark %
				& %
				& %
				& \cmark %
				& \cmark %
			\\
			\citet{Karimov2018} & \citeyear{Karimov2018}
				& %
				& %
				& %
				& %
				& %
				& %
				& %
				& \cmark %
				& %
				& %
				& %
				& %
				& %
				& %
				& %
				& \cmark %
				& \cmark %
				& \cmark %
				& %
				& %
				& %
				& %
				& %
				& %
				& \cmark %
				& \cmark %
				& %
				& \cmark %
				& \cmark %
			\\
			\citet{Truong2018} & \citeyear{Truong2018}
				& %
				& %
				& %
				& %
				& %
				& %
				& %
				& %
				& %
				& %
				& %
				& %
				& %
				& \cmark %
				& %
				& %
				& %
				& %
				& %
				& %
				& %
				& \cmark %
				& %
				& \cmark %
				& \cmark %
				& %
				& %
				& \cmark %
				& \cmark %
			\\
			\citet{Karakaya2017} & \citeyear{Karakaya2017}
				& %
				& %
				& %
				& %
				& %
				& %
				& %
				& %
				& %
				& \cmark %
				& %
				& %
				& %
				& %
				& %
				& \cmark %
				& \cmark %
				& \cmark %
				& %
				& %
				& %
				& %
				& %
				& %
				& \cmark %
				& %
				& %
				& \cmark %
				& \cmark %
			\\
			\citet{Shukla2017} & \citeyear{Shukla2017}
				& %
				& %
				& %
				& %
				& %
				& %
				& %
				& %
				& \cmark %
				& %
				& %
				& %
				& %
				& %
				& %
				& %
				& %
				& \cmark %
				& %
				& %
				& %
				& %
				& %
				& \cmark %
				& \cmark %
				& \cmark %
				& %
				& %
				& %
			\\
			\citet{Yang2017} & \citeyear{Yang2017}
				& %
				& %
				& %
				& %
				& %
				& %
				& %
				& %
				& %
				& \cmark %
				& %
				& %
				& %
				& \cmark %
				& %
				& \cmark %
				& \cmark %
				& \cmark %
				& %
				& %
				& %
				& %
				& %
				& \cmark %
				& \cmark %
				& %
				& %
				& %
				& %
			\\
			\citet{Chintapalli2016} & \citeyear{Chintapalli2016}
				& %
				& %
				& %
				& %
				& %
				& %
				& %
				& %
				& %
				& \cmark %
				& %
				& %
				& %
				& %
				& %
				& \cmark %
				& \cmark %
				& \cmark %
				& %
				& %
				& %
				& %
				& %
				& %
				& \cmark %
				& \cmark %
				& \cmark %
				& %
				& %
			\\
			\citet{Lopez2016} & \citeyear{Lopez2016}
				& %
				& %
				& %
				& %
				& %
				& %
				& %
				& %
				& %
				& %
				& %
				& %
				& %
				& \cmark %
				& %
				& \cmark %
				& \cmark %
				& \cmark %
				& %
				& %
				& %
				& %
				& %
				& %
				& \cmark %
				& %
				& %
				& \cmark %
				& \cmark %
			\\
			\citet{Qian2016} & \citeyear{Qian2016}
				& %
				& %
				& %
				& %
				& %
				& %
				& %
				& %
				& %
				& %
				& %
				& \cmark %
				& %
				& %
				& %
				& %
				& \cmark %
				& \cmark %
				& \cmark %
				& %
				& %
				& %
				& %
				& %
				& \cmark %
				& \cmark %
				& \cmark %
				& %
				& %
			\\
			\citet{Lu2014} & \citeyear{Lu2014}
				& %
				& %
				& %
				& %
				& %
				& %
				& %
				& %
				& %
				& %
				& %
				& \cmark %
				& %
				& %
				& %
				& %
				& \cmark %
				& \cmark %
				& %
				& %
				& %
				& %
				& %
				& %
				& \cmark %
				& \cmark %
				& \cmark %
				& %
				& %
			\\
			\bottomrule
		\end{tabularx}
		\begin{tablenotes}\footnotesize
			\item[\fnvandenpoel] and van den Poel
			\item[\fnbeam] using Apache Beam
		\end{tablenotes}
	\end{threeparttable}
\end{table*}

\cref{tab:related-experiments} provides an overview of experimental performance evaluation and benchmarking studies. It indicates the applied benchmark, the evaluated stream processing, and information regarding the experiment setup and method. The latter includes whether the respective study was performed in a cloud environment, in a distributed fashion with multiple instances of the framework deployed. Moreover, it shows whether the benchmarks have been executed with different resource amounts and different load intensities and whether different resource amounts and load intensities are evaluated in isolated experiments. In previous work, we argued that scalability should be evaluated with isolated experiments for different combinations of load and resources~\cite{LTB2021,EMSE2022}.

We can observe that there is no established stream processing benchmark. Only YSB is used in several studies. However, YSB can be considered a micro-benchmark~\cite{Bermbach2017} and, hence, is less suited to benchmark entire microservices.
Except for the preliminary evaluation of our Theodolite benchmarks~\cite{BDR2021}, there is no benchmarking study addressed to stream processing frameworks employed within microservice architectures.

Flink, Spark, and Strom are by far the most frequently benchmarked frameworks. Kafka Stream, Hazelcast Jet, and Samza, which are particularly suited for implementing event-driven microservices, are only benchmarked in a few studies and there is no study benchmarking all of them.

9 out of 20 studies report on experiments in public or private clouds.
Except for this and our previous study~\cite{IC2E2022FaaSStreaming}, there are no evaluations in Kubernetes.
Likewise, there are no further studies evaluating scalability with a systematic approach as we do in this study. \citet{Vikash2020}, \citet{Nasiri2019}, \citet{Karakaya2017}, and \citet{vanDongen2021b} explicitly evaluate scalability, however, without testing different load intensities against different resource amounts in isolated experiments. \citet{Nasiri2019} conduct independent evaluations of scaling load and computing resources and, thus, address another aspect than our study.
Our previous study~\cite{IC2E2022FaaSStreaming} applies our Theodolite method as well, but benchmarks scalability with respect to costs and is addressed to comparing stream processing deployments against Function-as-a-Service offerings.



%!TEX root = ../main.tex

\section{Inductive Conformal Prediction}
\label{sec:pre:icp}
Given a set $\{ z_i = ( x_i, y_i ) \}_{i=1}^l$ with observation $x_i \in \calX$ and label $y_i \in \calY$ such that each $z_i \in \calZ := \calX \times \calY $ is drawn i.i.d. from an \emph{unknown} distribution on $\calZ$, inductive conformal prediction (ICP) provides 
% a simple yet powerful framework to learn 
a \emph{set prediction} $\Feps(x) \subseteq \calY$, parameterized by an error rate $0 < \epsilon <1$, such that given a new sample $z_{l+1} = (x_{l+1},y_{l+1})$ satisfying an \emph{exchangeability} condition (elaborated in Theorem~\ref{thm:icp-validity}), we have
\bea\label{eq:icpmiscoverage}
\probof{ y_{l+1} \in \Feps(x_{l+1}) } \geq 1-\epsilon, 
\eea
\ie, the prediction set $\Feps$ guarantees to contain the true label $y_{l+1}$ with probability at least $1-\epsilon$. 

% In order to achieve the probabilistic coverage in~\eqref{eq:icpmiscoverage}, ICP performs the following three steps.

{\bf Training}. We start by dividing the dataset into a \emph{proper training set} $\{ z_1,\dots,z_m \}$ and a \emph{calibration set} $\{ z_{m+1},\dots,z_{l} \}$. We shorthand $n = l - m$ as the size of the calibration set.
We learn a prediction function $f: \calX \rightarrow \tcalY$ from the proper training set using \emph{any} architecture, which allows us to fully exploit the power of modern deep learning. The prediction space $\tcalY$ can be the same as the label space $\calY$, or can contain auxiliary information such as a heuristic notion of uncertainty (\eg, softmax scores in classification or a heatmap in the case of keypoint detection). 

{\bf Conformal calibration}. 
% Leveraging the learned $f$, 
We define a \emph{nonconformity} function $S: \calZ^{m} \times \calZ \rightarrow \Real{}$ to measure how well a given sample $z = (x,y)$ \emph{conforms} to the proper training set. A popular instance of $S$ leverages the learned prediction $f$:
\bea \label{eq:nonconformity}
S\parentheses{\cbrace{z_1,\dots,z_m},(x,y)} \stackrel{\eg}{=} r(y,f(x)),
\eea
where $r: \calY \times \tcalY \rightarrow \Real{}$ is a measure of disagreement between the label $y$ and the prediction $f(x)$. For example, consider $\calY = \tcalY = \Real{}$, one can design $r(y,f(x)) = \abs{y - f(x)}$: if $(x,y)$ poorly conforms to the training set, $f$ will incur large errors.   
While the function $S$ can be arbitrary (\eg, a learnable neural network~\cite{stutz22iclr-learnconformal}), \eqref{eq:nonconformity} is a convenient definition since $f$ is implicitly dependent on $\{z_i\}_{i=1}^m$ and $r$ can incorporate domain-specific knowledge.
We then compute the nonconformity scores on the calibration set as $\alpha_i = r(y_i,f(x_i)), i = m+1,\dots,l$,
and sort them in \emph{nonincreasing} order $\alpha_{\pi(1)}\geq\dots \geq \alpha_{\pi(n)}$, where $\pi(i) \in \{m+1,\dots,l\}$ is an index permutation.
 % (offset by $m$).

{\bf Conformal prediction}. Given a new observation $x_{l+1}$ (with an unknown $y_{l+1}$) and a user-specified $\epsilon \in (0,1)$, we compute the inductive conformal prediction (ICP) set as
\bea\label{eq:icpcompute}
\Feps \parentheses{x_{l+1}} = \cbrace{y \in \calY \mid \alpha^y \leq \alpha_{\pi(\floor{(n+1)\epsilon})}},
\eea
where $\alpha^y = r(y,f(x_{l+1}))$
is the nonconformity score of the new sample when fixing the true label to be $y$. In other words, the ICP set~\eqref{eq:icpcompute} outputs the set of all labels that make the nonconformity score of the new sample no greater than $\alpha_{\pi(\floor{(n+1)\epsilon})}$ -- the $\floor{(n+1)\epsilon}$-th largest nonconformity score in the calibration set. 
% By doing so, ICP ensures that there are at least $\floor{(n+1)\epsilon}$ samples in the calibration set that are less conforming than the new sample. 
We have the following result stating the probabilistic coverage of the ICP set~\eqref{eq:icpcompute}.
% provides a valid statistical coverage of the true label $y_{l+1}$.

\begin{theorem}[Validity of ICP Coverage {\cite{vovk05book-conformal,lei18jasa-conformal,vovk12acml-icpconditional}}] \label{thm:icp-validity}
If $z_{m+1},\dots,z_l$, $z_{l+1} = (x_{l+1},y_{l+1})$ are exchangeable, \ie, their distribution is invariant under permutation, then
\bea\label{eq:icpvalidity}
1 - \epsilon \leq \probof{y_{l+1} \in \Feps(x_{l+1})} \leq 1 - \epsilon + 1/(n+1)
\eea
for any $\epsilon \in (0,1)$. Furthermore, when conditioned on the calibration set, calling $h = \floor{(n+1)\epsilon}$, we have
\begin{equation}\label{eq:beta}
\hspace{-4mm}\probof{y_{l+1}\!\in\!\Feps(x_{l+1})\!\mid\!\{z_{m+1},\dots,z_l\}}\!\sim\!\mathrm{Beta}(n+1\!-\!h,h).
\end{equation}
\end{theorem}
A few remarks are in order about Theorem~\ref{thm:icp-validity}.
First, asking $z_{m+1},\dots,z_l,z_{l+1}$ to be exchangeable is weaker than asking them to be independent. However, this assumption typically fails when the calibration set is a single video sequence, where the image frames $\{z_{m+1},\dots,z_l\}$ are temporally correlated~\cite{luo21arxiv-conformalsafety}. Fortunately, as we detail in Section~\ref{sec:experiments}, the way the LineMOD Occlusion dataset~\cite{brachmann14eccv-linemodocc} was collected makes the exchangeability condition easily satisfied, which also suggests best practices to make the exchangeability condition hold in computer vision. 
Second, the lower bound in~\eqref{eq:icpvalidity} can be intuitively proved because under exchangeability, $\alpha_{l+1} := r(y_{l+1},f(x_{l+1}))$ --the nonconformity score of the new sample with the true label-- is \emph{exchangeable} with the nonconformity scores of the calibration samples, and hence \emph{equally likely} to fall in anywhere between the scores $\{ \alpha_{\pi(i)}\}_{i=1}^n$. Consequently, $\probof{y_{l+1} \in \Feps(x_{l+1})} = \probof{\alpha_{l+1} \leq \alpha_{\pi(\floor{(n+1)\epsilon})}} = 1 - \floor{(n+1)\epsilon}/(n+1) \geq 1 - \epsilon$. The upper bound in \eqref{eq:icpvalidity} states that $1-\epsilon$ is not overly conservative (indeed tight if $n$ is large). 
Lastly, the probabilistic guarantee in \eqref{eq:icpvalidity} is \emph{marginal} over the randomness of the calibration set, meaning if one chooses an infinite number of calibration sets,  the \emph{average} empirical coverage will converge to $1-\epsilon$. This, however, implies that the empirical coverage given one calibration set is a random variable that fluctuates as the Beta distribution~\eqref{eq:beta}. Fig.~\ref{fig:beta-distribution} plots the Beta distribution at $\epsilon=0.1$ with different sizes of the calibration set. We observe that as $n$ increases the empirical coverage becomes more concentrated at $1-\epsilon$. Our experiments show that even with a small ($n=200$) calibration set, the empirical coverage is close to, and mostly higher than, $1-\epsilon$.

% \begin{proposition}[Conditional Validity of ICP {\cite{vovk12acml-icpconditional}}] \label{prop:icp-conditional-validity}
% \red{To be filled out}
% \end{proposition}


% Proposition~\ref{prop:icp-validity} states that, if the new observation $z_{l+1}$ is exchangeable with the calibration set (which is a weaker condition than requiring $z_{l+1}$ is jointly i.i.d. with the calibration set), then no matter which prediction function $f$ has been learned from the proper training set, and which function $A$ has been chosen to compute the nonconformity score, we have at least $1-\epsilon$ confidence that the ICP $\Feps$ defined in \eqref{eq:icp} contains the true label. Of course, the caveat here is that the quality of the learned prediction function $f$ and the nonconformity function $A$ will decide the conservativeness of the ICP $\Feps$. For example, if $f$ has poor predictive power, then the set $\Feps$ may be arbitrarily large so that it tells little information about the true label $y$. \red{Fortunately, as we will show in experiments, with modern deep learning architectures for learning $f$, we can obtain ICPs that are both confident and tight.}

%!TEX root = ../main.tex
% \begin{figure}
% \hspace{-4mm}\includegraphics[width=1.1\columnwidth]{icp-overview-half.pdf}
% \caption{Given a learned prediction function and a calibration set of $n$ samples, conformal calibration uses a nonconformity function~\eqref{eq:nonconformity} to compute and sort nonconformity scores $\{ \alpha_{\pi(i)}\}_{i=1}^n$. Given a new observation and an error rate $\epsilon$, conformal prediction~\eqref{eq:icpcompute} outputs a prediction set of all labels under which the nonconformity score of the new sample is no larger than $\alpha_{\pi(\floor{(n+1)\epsilon})}$.
% \label{fig:icp-overview}}
% \end{figure}
\begin{figure}
\vspace{-4mm}
\begin{center}
\includegraphics[width=0.6\columnwidth]{beta.pdf}
\end{center}
\vspace{-6mm}
\caption{Beta distribution of the conditional coverage in~\eqref{eq:beta} with $\epsilon=0.1$ and different $n$. Notice how the conditional probability becomes more concentrated around $1-\epsilon$ when $n$ increases.
\label{fig:beta-distribution}}
\vspace{-7mm}
\end{figure}
%!TEX root = ../main.tex

\section{Conformal Keypoint Detection}
\label{sec:keypoint:conformal}
In this section, we apply the ICP framework in Section~\ref{sec:pre:icp} to the problem of semantic keypoint detection. 

{\bf Setup}. Denote by $x \in \Real{H\times W\times 3}$ an RGB image picturing an object, by $\vy = (y_1,\dots,y_K) \in \Real{2} \times \dots \times \Real{2} := \calY$ the groundtruth locations of $K$ semantic keypoints of the object. We partition a given dataset $\{z_{i}:=(x_{i},\vy_{i})\}_{i=1}^l$ 
% containing pairs of images and keypoint annotations. We partition the dataset 
into a proper training set (of size $m$) and a calibration set (of size $n$).
We follow the three steps in Section~\ref{sec:pre:icp} to perform ICP.

{\bf Training}. We choose 
% one of the first and most popular approaches for training a network that can predict semantic keypoints: 
the heatmap approach in~\cite{pavlakos17icra-semantic,schmeckpeper22jfr-semantic} as the prediction function: given an image $x$, \cite{schmeckpeper22jfr-semantic} outputs a set of heatmaps $\vf(x) = (f(x)_1,\dots,f(x)_K)$, where each $f(x)_{k} \in \Delta^{HW}:=\{v \in \mathbb{R}^{HW}_{+}\mid \sum_{i}^{HW} v_i = 1\}$ predicts the probability distribution of the $k$-th keypoint lying on each pixel of the image.\footnote{The heatmap in the original paper~\cite{pavlakos17icra-semantic} is not a valid probability distribution as it contains negative values and do not sum up to $1$. We remove the negative values and normalize it to be a valid probability distribution.} For convenience, we use $q^j \in \Real{2}$ to denote the $j$-th pixel location in $x$ and $f(x)_k^j \in \mathbb{R}_{+}$ to denote the probability of the $k$-th keypoint lying on $q^j$. Let $\sigma_k$ be the index permutation that sorts $f(x)_k$ in nonincreasing order, \ie, $f(x)_k^{\sigma_k(1)} \geq \dots \geq f(x)_k^{\sigma_k(HW)}$. As we will soon show, choosing the heatmap approach leads to simple and intuitive designs of the nonconformity function.

{\bf Conformal calibration}. 
% We conformalize the heatmap by designing a nonconformity function. 
We design the following nonconformity function 
\bea\label{eq:maxnonconformity}
r(\vy,\vf(x)) = \max\{ \phi(y_k,f(x)_k) \}_{k=1}^K
\eea
that uses $\phi$ to score each keypoint and then selects the maximum score. This design considers the worst keypoint detection performance of $\vf$. We provide two designs of $\phi$ below.

{\emph{(a) Peak}}. Shorthand $p_k = f(x)_k^{\sigma_k(1)}$ as the peak probability in the $k$-th heatmap and $q_k = q^{\sigma_k(1)}$ as the pixel location attaining the peak probability, we design
\begin{equation}\label{eq:peak}
\phipeak(y_k, f(x)_k) = p_k \norm{y_k - q_k} \tag{peak}
\end{equation}
which computes the error between the true keypoint location $y_k$ and the most probable keypoint location $q_k$ and scales the error by the peak probability $p_k$. $\phipeak$ describes nonconformity because it becomes larger when the network $\vf$ is \emph{confidently wrong} (both $\norm{y_k - q_k}$ and $p_k$ are large), implying the sample is highly nonconforming. 

{\emph{(b) Covariance}}. Let $\barq_k = \sum_{j=1}^J f(x)_k^{\sigma_k(j)} q^{\sigma_k(j)}$ be the expected location of the top-$J$ most likely detections for the $k$-th keypoint, and $\Sigma_k = \sum_{j=1}^J f(x)_k^{\sigma_k(j)} \cdot (q^{\sigma_k(j)} - \barq_k)(q^{\sigma_k(j)} - \barq_k)\tran$ as the covariance, we design
\begin{equation}\label{eq:cov}
\phicov(y_k, f(x)_k) = (y_k - \barq_k)\tran \Sigma_k\inv (y_k - \barq_k) \tag{cov}
\end{equation}
which computes the squared Mahalanobis distance~\cite{mahalanobis36nisi-mahalanobis} from the groundtruth $y_k$ to the top-$J$ keypoint detections (represented by the mean $\barq_k$ and covariance $\Sigma_k$).\footnote{We only choose the top-$J$ ($J=100$) most likely detections on the heatmap because the heatmap can be quite noisy in practice.} A larger Mahalanobis distance indicates more abnormality of the heatmap $f(x)_k$ (compared to the groundtruth $y_k$)~\cite{geun00cs-multivariate}, and hence implies higher nonconformity.

Using the nonconformity function \eqref{eq:maxnonconformity} with \eqref{eq:peak} or \eqref{eq:cov}, we compute the nonconformity scores of the calibration set and sort them as: $\alpha_{\pi(1)} \geq \dots \geq \alpha_{\pi(n)}$.

{\bf Conformal prediction}. Given an error rate $\epsilon \in (0,1)$, we first find $\alpha_{\pi(\floor{(n+1)\epsilon})}$. Then, according to the ICP set definition~\eqref{eq:icpcompute} and our nonconformity function~\eqref{eq:maxnonconformity}, we output the ICP set for a new $x_{l+1}$ as
\bea\label{eq:generalkpticp}
 & \Feps(x_{l+1}) \nonumber \\
\hspace{-4mm} = & \hspace{-3mm} \{\vy \in \calY \mid \max\{ \phi(y_k,f(x_{l+1})) \}_{k=1}^K \leq \alpha_{\pi(\floor{(n+1)\epsilon})} \}  \nonumber \\
\hspace{-4mm} = & \{\vy \in \calY \mid \phi(y_k,f(x_{l+1})) \leq \alpha_{\pi(\floor{(n+1)\epsilon})}, \forall k \},
\eea
where we used $\max\{\phi_1,\dots,\phi_K\} \leq \alpha$ if and only if $\phi_k \leq \alpha$ for any $k$. Insert~\eqref{eq:peak} into~\eqref{eq:generalkpticp}, we have $\Fepsball(x_{l+1})$ as
\begin{equation}\label{eq:icp-ball} \tag{ball}
\cbrace{ \vy \in \calY \mid \norm{y_k - q_{l+1,k}} \leq \frac{ \alpha_{\pi(\floor{(n+1)\epsilon})} }{p_{l+1,k}}, \forall k},
\end{equation}
which defines --for the $k$-th keypoint-- a ball centered at $q_{l+1,k}$ (the most likely detection) with a radius inversely proportional to $p_{l+1,k}$ and proportional to $\alpha_{\pi(\floor{(n+1)\epsilon})}$. Similarly, insert~\eqref{eq:cov} into~\eqref{eq:generalkpticp}, we have $\Fepsellipse(x_{l+1})$ as
\begin{equation}\label{eq:icp-ellipse} \tag{ellipse}
\hspace{-4mm}\cbrace{ \vy \in \calY \mid (y_k - \barq_{l+1,k})\tran \frac{\Sigma_{l+1,k}\inv }{\alpha_{\pi(\floor{(n+1)\epsilon})}} (y_k - \barq_{l+1,k}) \leq 1, \forall k},
\end{equation}
which defines --for the $k$-th keypoint-- an ellipse centered at $\barq_{l+1,k}$ (the expected location of the top-$J$ detections) with an area proportional to $\det(\Sigma_{l+1,k})$ and $\alpha_{\pi(\floor{(n+1)\epsilon})}$.\footnote{The area of  $(x-\mu)\tran A (x-\mu) \leq 1$ is proportional to $\det(A\inv)$.} From \eqref{eq:icp-ball} and \eqref{eq:icp-ellipse}, we observe that the prediction sets become larger when (i) the heatmaps are uncertain, \ie, the peak probability is low or the covariance matrix has large determinant; and (ii) the heatmaps perform poorly on the calibration set, leading to a large $\alpha_{\pi(\floor{(n+1)\epsilon})}$.

{\bf Connections to geometric vision}. Our nonconformity function bears similarity to the \emph{residual} function in geometric vision~\cite{hartley03book-geometry,antonante21tro-outlier,chin18eccv-robust}. For example, the \eqref{eq:peak} and \eqref{eq:cov} functions are similar to the (weighted) reprojection error~\cite{hartley03book-geometry}, and the ``$\max$'' in~\eqref{eq:maxnonconformity} can be connected to seminal work on optimizing the $\ell_{\infty}$ norm~\cite{kahl08tpami-multiple}.

{\bf Outlier-robust nonconformity}? One potential issue of the nonconformity function~\eqref{eq:maxnonconformity} is that a \emph{single} {outlier} can inflate the score and the calibration quantile $\alpha_{\pi(\floor{(n+1)\epsilon})}$ and lead to conservative prediction sets (\eg, when $\vf$ predicts $K-1$ keypoints perfectly but misses one keypoint). A potential remedy in geometric vision is to use robust cost functions~\cite{black96ijcv-unification,yang20ral-gnc,barron19cvpr-general}. Therefore, a natural question is whether ``robustifying'' the nonconformity function \eqref{eq:maxnonconformity} can lead to better prediction sets. Here we focus on only robustifying $\phi$ in \eqref{eq:maxnonconformity} and provide a negative answer.

\begin{proposition}[Invariance of ICP]\label{prop:invariance}
Let $\rho: \mathbb{R}_+ \mapsto \mathbb{R}_+$ be any monotonically increasing function. Fixing the calibration set and error rate $\epsilon$, the nonconformity function
\bea\label{eq:robustnonconformity}
r_{\rho}(\vy, \vf(x)) = \max\{ \rho(\phi(y_k,f(x)_k)) \}_{k=1}^K
\eea
leads to the same ICP set as \eqref{eq:maxnonconformity}.
\end{proposition}
% \begin{proof} Let $\{ \alpha_{i}  \}_{i=1}^n$ be the calibration scores obtained by applying $r$ in \eqref{eq:maxnonconformity} to the calibration set, and $\{ \alpha^{\rho}_i \}_{i=1}^{n}$ be the scores obtained by applying $r_{\rho}$ in \eqref{eq:robustnonconformity}. Observe that $\alpha^{\rho}_i = \rho(\alpha_i)$ because $\rho$ being monotonically increasing implies $\max \circ \rho = \rho \circ \max$ (``$\circ$'' describes function composition). As a result, it follows that $\alpha^{\rho}_{\pi(\floor{(n+1)\epsilon})} = \rho(\alpha_{\pi(\floor{(n+1)\epsilon})})$.
% Let $\Feps_{\rho}$ be the ICP set due to $r_{\rho}$ for a given $\epsilon$, we have
% \bea
% \Feps_{\rho} = \{ \vy \in \calY \mid  \max\{ \rho(\phi(y_k,f(x)_k)) \}_{k=1}^K \leq \alpha^{\rho}_{\pi(\floor{(n+1)\epsilon})}  \} \nonumber \\
% = \{ \vy \in \calY \mid  \rho( \max\{ \phi(y_k,f(x)_k) \}_{k=1}^K ) \leq \rho(\alpha_{\pi(\floor{(n+1)\epsilon})} ) \} \nonumber\\
% = \{ \vy \in \calY \mid \max\{ \phi(y_k,f(x)_k) \}_{k=1}^K \leq \alpha_{\pi(\floor{(n+1)\epsilon})} \}, \nonumber 
% \eea
% where the last set is $\Feps$, the ICP set induced by $r$.
% \end{proof}
The proof of Proposition~\ref{prop:invariance} is presented in Supplementary Material. We conclude that common robust costs, such as $\ell_1$, Huber, Geman-McClure, and Barron's adaptive kernel~\cite{black96ijcv-unification,barron19cvpr-general} (which are monotonically increasing on $[0,+\infty]$) cannot change the ICP sets by robustifying the individual score $\phi$. However, it remains an open question whether changing the ``$\max$'' operation in \eqref{eq:maxnonconformity} can give rise to better ICP sets. For instance, replacing ``$\max$'' with ``$\sum$'' in \eqref{eq:maxnonconformity} and using the Geman-McClure robust cost $\rho(\phi) = \frac{\phi^2}{1 + \phi^2}$ with $\phi = \phipeak$ results in the following ICP set 
\bea
\cbrace{ \vy \in \calY \mid \sum_{k=1}^K \frac{p_k^2 \norm{y_k - q_k}^2}{1 + p_k^2 \norm{y_k - q_k}^2} \leq \alpha_{\pi(\floor{(n+1)\epsilon})} }
\eea
that does not admit a geometric interpretation that is as simple and intuitive as the \eqref{eq:icp-ball} and \eqref{eq:icp-ellipse} sets introduced before. In fact, it is indeed the simplicity of \eqref{eq:icp-ball} and \eqref{eq:icp-ellipse} that enables us to propagate the uncertainty in keypoints to the object pose, as we will show in the next section.




% {\bf Heatmap-based keypoint detector}. Given an RGB image $x \in \Real{H \times W \times 3}$, heatmap-based methods~\cite{pavlakos17icra-semantic,schmeckpeper22jfr-semantic,lin22icra-single} output $f(x) \in \Delta^{HW} := \{ v \in \Real{HW} \mid v_{i} \geq 0, \sum_{i}^{HW} v_{i} = 1 \}$ where $f(x)_{i} \geq 0$ indicates the probability of the keypoint on the $i$-th pixel (we vertically concatenate all pixels). Let $\pi(\cdot)$ denote an index permutation that sorts $f(x)$ as $f(x)_{\pi(1)} \geq \dots \geq f(x)_{\pi(HW)}$.
% % \footnote{In practice, heatmaps directly coming out of neural networks are often not constrained to be nonnegative and sum up to one~\cite{schmeckpeper22jfr-semantic}. In this case, we offset and normalize the heatmaps.} 

% {\bf Nonconformity function}. It is tempting to treat keypoint detection as a classification problem and adopt popular nonconformity functions designed for classification (\eg the one in~\cite{romano20neurips-classification,angelopoulos21iclr-uncertainty}). However, in the Supplementary Material we show the resulting prediction sets are loose and hard to interpret. This motivates us to design the following three nonconformity functions.

% {\bf (I) Peak}. Let $q^\star \in \Real{2}$ be the pixel location with maximum probability $p^\star := f(x)_{\pi(1)}$. We design
% \begin{equation}\label{eq:score-peak}
% r(y, f(x)) = p^\star \norm{y - q^\star}. \tag{Peak}
% \end{equation}
% According to \eqref{eq:icpcompute}, we have the following inductive conformal prediction set
% \begin{equation}\label{eq:icp-heatmap-peak}
% \Feps(x_{l+1}) = \cbrace{y \in \calY \ \middle\vert\ \norm{y - q^\star_{l+1}} \leq {\alpha_{\pi(\floor{(n+1)\epsilon})}}/{p^\star_{l+1}} }. \tag{ICP-Peak}
% \end{equation}
% This ICP set is a ball centered at $q^\star_{l+1}$ with a radius \emph{inversely} proportional to $p^\star_{l+1}$. Intuitively, when $p^\star_{l+1}$ is small, \ie $f(x)$ is uncertain, the ball is enlarged to account for higher uncertainty.
    
% {\bf (II) Variance}. Let $q_i \in \Real{2}$ be the $i$-th pixel location. Compute $\bar{q} = \sum_{i=1}^{K} f(x)_{\pi(i)} \cdot q_{\pi(i)}$ as the expected location of the top-$K$ keypoints, and $\gamma^2 = \sum_{i=1}^{K} f(x)_{\pi(i)} \cdot \Vert q_{\pi(i)} - \bar{q} \Vert^2$ as the ``variance'' (we use $K=100$ pixels because the heatmap can be quite noisy). We design
% \begin{equation}\label{eq:score-variance}
% r(y,f(x)) = {\norm{y - \bar{q}}}/{\gamma}. \tag{Var}
% \end{equation}
% According to \eqref{eq:icpcompute}, we have the inductive conformal prediction set
% \begin{equation}\label{eq:icp-heatmap-var}
% \Feps(x_{l+1}) = \cbrace{ y \in \calY \ \middle\vert\ \norm{y - \bar{q}_{l+1}} \leq \gamma_{l+1} \alpha_{\pi(\floor{(n+1)\epsilon})}  }. \tag{ICP-Var}
% \end{equation}
% This ICP set is a ball centered at $\bar{q}_{l+1}$ with a radius proportional to the ``standard deviation'' $\gamma_{l+1}$. Intuitively, when the heatmap is spread out and $f(x)$ has higher uncertainty, the ball becomes larger. 
    
% {\bf (III) Covariance}. Compute the expected top-$K$ keypoint location $\bar{q}$ as before. Then compute the \emph{covariance matrix} $\Sigma = \sum_{i=1}^{K} f(x)_{\pi(i)} \cdot (q_{\pi(i)} - \bar{q})(q_{\pi(i)} - \bar{q})\tran$.
% We design
% \begin{equation}\label{eq:score-cov}
% r(y,f(x)) = (y - \bar{q})\tran \Sigma\inv (y-\bar{q}) \tag{Cov}.
% \end{equation}
% According to \eqref{eq:icpcompute}, we have the inductive conformal prediction set
% \begin{equation}\label{eq:icp-heatmap-cov}
% \Feps(x_{l+1}) = \cbrace{y \in \calY \ \middle\vert\ (y-\bar{q}_{l+1})\tran {\Sigma_{l+1}\inv} (y-\bar{q}_{l+1}) \leq \alpha_{\pi(\floor{(n+1)\epsilon})} }.\tag{ICP-Cov}
% \end{equation}
% This ICP set is an ellipse centered at $\bar{q}_{l+1}$. The eigenvectors of ${\Lambda}_{l+1} := \Sigma\inv_{l+1} / \alpha_{\pi(\floor{(n+1)\epsilon})}$ point in the directions of the principal axes, while the eigenvalues are $1/a^2$ and $1/b^2$, with $a\leq b$ the lengths of the semi-axes. Similar to~\eqref{eq:icp-heatmap-var}, the ellipse gets larger if the heatmap has higher uncertainty. Different from~\eqref{eq:icp-heatmap-var}, the ellipse better captures nonuniform uncertainty, as we show in Section~\ref{sec:experiments}. 




%!TEX root = ../main.tex

\section{Geometric Uncertainty Propagation}
\label{sec:uncertainty:propagation}
Conformalizing the heatmaps gives us prediction sets that guarantee probabilistic coverage of the true keypoints. We unify the prediction sets \eqref{eq:icp-ball} and \eqref{eq:icp-ellipse} as
\bea\label{eq:icpunify}
\Feps(x) = \cbrace{\vy \in \calY \mid (y_k - \mu_k)\tran \Lambda_k (y_k - \mu_k) \leq 1,\forall k},
\eea
where $\mu_k = q_{l+1,k}, \Lambda_k = \frac{p_{l+1,k}^2}{\alpha^2_{\pi(\floor{(n+1)\epsilon})}} \eye_2$ for~\eqref{eq:icp-ball}, $\mu_k = \barq_{l+1,k},\Lambda_k = \frac{\Sigma_{l+1,k}\inv}{\alpha_{\pi(\floor{(n+1)\epsilon})}}$ for~\eqref{eq:icp-ellipse}, and we omit the subscript $l+1$ for simplicity.

{\bf Why not uncertainty-aware \pnp}? A popular way to estimate pose from~\eqref{eq:icpunify} is to solve an uncertainty-aware \pnp
\bea \label{eq:uncertainpnp}
\min_{(R,t) \in \SEthree} & \displaystyle \sum_{k=1}^K (y_k - \mu_k)\tran \Lambda_k (y_k - \mu_k) \nonumber \\
\subject & y_k = \Pi(RY_k + t), k=1,\dots,K
\eea
where $Y_k \in \Real{3},k=1,\dots,K$ are the 3D object keypoints and $\Pi(\cdot)$ denotes the camera projection. We challenge this approach and point out its two drawbacks. First, it is difficult to solve~\eqref{eq:uncertainpnp} to global optimality due to (i) the nonconvex $\SEthree$ constraint and (ii) the rational polynomial appearing in $\Pi(\cdot)$. The best known approach to solve \eqref{eq:uncertainpnp} relies on either branch-and-bound~\cite{olsson06icpr-optimal} or local optimization. Second, solving~\eqref{eq:uncertainpnp} typically outputs a \emph{single} optimal pose without uncertainty quantification. Are there other poses that attain similar costs as the optimal pose? How close is the optimal pose to the groundtruth pose? These questions remain not answered in the literature.
% . Lastly, to the best of our knowledge there is no provable reason to believe that the optimal solution of~\eqref{eq:uncertainpnp} is close to the groundtruth. 
% Bounding the error from the optimal pose to the groundtruth pose is critical for downstream decision making, \eg, robust control~\cite{khalil96book-robustcontrol}. 

{\bf Pose UnceRtainty SEt (\purse)}. We propose to, instead of solving a {\pnp} problem similar to~\eqref{eq:uncertainpnp}, directly propagate the uncertainty in the ICP sets to the object pose. 
\begin{proposition}[\purse]\label{prop:purse}
Let $\sgt = [\vectorize{\Rgt}\tran; \tgt\tran]\tran$ be the groundtruth object pose (that lies in front of the camera). Then, the groundtruth keypoints $\vy = (y_1,\dots,y_K)$ belong to the ICP set $\Feps(x)$ in~\eqref{eq:icpunify} if and only if $\sgt$ belongs to the following pose uncertainty set
\begin{equation}\label{eq:purse}
\hspace{-3mm} \Seps = \cbrace{s \in \SEthree \ \middle\vert\ \substack{ \displaystyle s\tran A_k s \leq 0,k=1,\dots,K \\ \displaystyle b_k\tran s > 0, k=1,\dots,K} }, \tag{\purse}
\end{equation} 
where $A_k\in \sym{12}, b_k \in \Real{12},k=1,\dots,K$ are constant matrices dependent on $\mu_k,\Lambda_k,Y_k$ and camera intrinsics.
\end{proposition}
The detailed proof for Proposition~\ref{prop:purse} is algebraically involved and postponed to the Supplementary Material. The high-level intuition is, however, straightforward: we plug in $y_k = \Pi(RY_k + t)$ into \eqref{eq:icpunify} and obtain $K$ quadratic inequalities of the form $s\tran A_k s \leq 0$. The linear inequalities $b_k\tran s > 0$ are added to enforce the (transformed) 3D keypoints lie in front of the camera. Proposition~\ref{prop:purse} implies, if we are $1-\epsilon$ confident the groundtruth keypoints can be anywhere inside $\Feps(x)$, then we should also be confident any pose in~\eqref{eq:purse} can be the groundtruth. Viewing pose estimation as a set estimation with guaranteed probabilistic coverage of the groundtruth is fundamentally different from viewing it as computing a single pose from~\eqref{eq:uncertainpnp} that is (hopefully) close enough to the groundtruth.

{\bf RANdom SAmple averaGing (\ransag)}. Verifying if a given pose belongs to the \purse is straightforward via checking the inequalities in~\eqref{eq:purse}. However, the {\purse} does not directly give us estimated poses. Therefore, we propose an efficient sampling algorithm called \emph{RANdom SAmple averaGing} (\ransag) that is analogous to {\ransac}~\cite{fischler81acm-ransac} and leverages the minimal solver \pthreep~\cite{gao03pami-p3p}, presented in Algorithm~\ref{alg:ransag}.
% \footnote{A naive way to sample from {\purse} is to randomly sample in $\SEthree$ and reject samples that fail the inequalities. This approach has very low success rate because $\SEthree$ is a high-dimensional continuous space (\eg, it is common to sample $1000$ times and get zero valid samples).}
The intuition is that, though it is difficult to sample directly in \purse due to the (nonconvex) constraints, it is easy to sample from the keypoint prediction set~\eqref{eq:icpunify} due to its simple geometry (balls and ellipses). Thus, at each iteration (line~\ref{line:p3piter}) {\ransag} samples three keypoints (line~\ref{line:samplek}-\ref{line:samplekeypoints}), solves the \pthreep inverse problem, and accept the poses that belong to the \eqref{eq:purse} (line~\ref{line:solvep3p}). \ransag typically returns around $100$ valid samples with $T=1000$ trials. However, in difficult cases (\eg, when $\Seps$ is small or even empty) it is possible to obtain zero samples ($S = \emptyset$). In this situation, {\ransag} samples $\floor{T/20}$ (default $50$) poses without checking if they belong to the \purse, via sampling $K$ keypoints and solving \pnp (line~\ref{line:emptyset}-\ref{line:solvepnp}).\footnote{Here we switch from \pthreep to \pnp because \pnp uses all $K$ keypoints and there is less ambiguity in its solution.} After obtaining a set of poses, {\ransag} performs rotation averaging (line~\ref{line:rotavg}) and translation averaging (line~\ref{line:transavg}) to obtain an average pose $\bar{s}$.\footnote{In Algorithm~\ref{alg:ransag} we use rotation averaging with the Chordal distance metric. The user is free to choose other single rotation averaging algorithms with different distance metrics~\cite{hartley13ijcv-rotation}.} Note that {\ransag} does not check if $\bar{s}$ lies in the \purse.

%!TEX root = ../main.tex
\setlength{\textfloatsep}{5pt}%
\begin{algorithm}[t]
\SetAlgoLined
{\bf Input:} an ICP set $\Feps(x)$~\eqref{eq:icpunify} and its corresponding \eqref{eq:purse} $\Seps$; maximum trials $T$; initial $\hat{S} = \emptyset$; \\
{\bf Output:} sample poses ${S} \subset \SEthree$ in \purse, and an average pose $\bar{s} \in \SEthree$; \\
\For{ $\tau \gets 1$ to $T$ \label{line:p3piter}}{
	Sample $\{k_1,k_2,k_3\}$ from $[K]$ ($k_1 \neq k_2 \neq k_3$); \label{line:samplek}\\
	Sample $\haty_{k_i}, i=1,2,3$ from \label{line:samplekeypoints}
	\begin{equation}
	\{ y \in \Real{2} \mid (y - \mu_{k_i})\tran \Lambda_{k_i} (y - \mu_{k_i}) \leq 1 \}; \nonumber
	\end{equation}\\
	$\hat{S} \leftarrow \hat{S} \cup ( \Seps \cap \text{\pthreep}(\{\haty_{k_i} \leftrightarrow Y_{k_i}\}_{i=1}^3 ) )$; \label{line:solvep3p}
}
$S = \hat{S}$;\\
\If{$\hat{S} = \emptyset$ \label{line:emptyset} }{
	\For{$\tau \gets 1$ to $\floor{T/20}$}{
	Sample $\haty_{k}, k=1,\dots,K$ from $\Feps(x)$;\\
	$\hat{S} \leftarrow \hat{S} \cup \text{\pnp}(\{\haty_{k} \leftrightarrow Y_{k}\}_{k=1}^K)$;\label{line:solvepnp}
	}
}
\label{line:rotavg} $\bar{R} = \text{proj}_{\SOthree}( \sum_{(R_j,*) \in \hat{S}} R_j )$; \\
\label{line:transavg} $\bar{t} = \frac{1}{\vert \hat{S} \vert } \sum_{(*,t_j) \in \hat{S}} t_j $; \\
{\bf return:} $S$, $\bar{s} = (\bar{R},\bar{t})$ 
\caption{RANdom SAmple averaGing \label{alg:ransag}}
\end{algorithm}


{\bf Worst-case error bounds}. To upper bound the errors between the average pose $\bar{s}$ and the groundtruth $(\Rgt,\tgt)$, we maximize the squared \emph{pose-to-\purse} distance:
\begin{equation}\label{eq:pose2purse}
d_{\epsilon,\lambda}^2 = \max_{(R,t) \in \Seps} \lambda \Fnorm{R - \bar{R}}^2 + (1-\lambda) \norm{t - \bar{t}}^2
\end{equation} 
given $\lambda \in [0,1]$. Particularly, we compute two cases $\lambda = 1$ (the maximum rotation distance) and $\lambda = 0$ (the maximum translation distance). Proposition~\ref{prop:purse} states the groundtruth $(\Rgt,\tgt)$ lies in $\Seps$ with $1-\epsilon$ probability, hence
\bea\label{eq:boundRt}
\Fnorm{\bar{R} - \Rgt} \leq d_{\epsilon,1}, \quad \norm{\bar{t} - \tgt} \leq d_{\epsilon,0}
\eea
holds with probability $1-\epsilon$.

{\bf Computing the bounds}. Problem~\eqref{eq:pose2purse} is nonconvex due to the constraints of the \eqref{eq:purse} $\Seps$. 
% Fortunately, thanks to the recent success in applying semidefinite relaxations to solve nonconvex optimization problems in computer vision~\cite{yang22pami-certifiably,briales18cvpr-certifiably,kahl07ijcv-globally}, 
We relax the nonconvex problem~\eqref{eq:pose2purse} into a convex semidefinite program (SDP) and employ off-the-shelf solvers to optimize the SDP~\cite{yang22pami-certifiably,briales18cvpr-certifiably,kahl07ijcv-globally}.\footnote{We omit the technical details and refer the interested reader to \cite[Section 2]{yang22pami-certifiably} for a pragmatic introduction to SDP relaxations. In practice, we use the code provided by~\cite{yang22pami-certifiably} in \url{https://github.com/MIT-SPARK/CertifiablyRobustPerception}, apply a second-order SDP relaxation to~\eqref{eq:pose2purse}, and use MOSEK~\cite{mosek} to solve the SDP (in about 8 seconds). Solving a first-order SDP relaxation of~\eqref{eq:pose2purse} takes about $0.1$ second but yields looser bounds.} Two possible outcomes can happen: (i) the optimal SDP value coincides with the optimal value of~\eqref{eq:pose2purse}. The relaxation is said to be \emph{exact} and one can extract an optimal solution of~\eqref{eq:pose2purse} from the SDP, or (ii) the relaxation is not exact, but the optimal SDP value still provides an \emph{upper bound} for the optimal value of~\eqref{eq:pose2purse}. Therefore, we either exactly compute $d^2_{\epsilon,\lambda}$ or find an upper bound, both can bound the worst-case error (\cf~\eqref{eq:boundRt}).\footnote{The \purse can potentially be empty, leading to infeasibility of problem~\eqref{eq:pose2purse}. In such cases, empirically the SDP solver returns ``\texttt{PRIMAL\_INFEASIBLE}'' (red squares lying on the $y$-axis of Fig.~\ref{fig:coverage-and-bound}).}

% {\bf Summary}. Given the ICP set $\Feps$~\eqref{eq:icpunify}, we first obtain the \eqref{eq:purse} representation $\Seps$. We then use \ransag to compute an average pose $(\bar{R},\bar{t})$, followed by solving the SDP relaxation of the maximum pose-to-\purse distance~\eqref{eq:pose2purse} to bound the worst-case rotation error and translation error as in~\eqref{eq:boundRt}, which holds with a marginal probability $1-\epsilon$.

We end with a remark about computing tighter bounds.
% \begin{remark}[Unknown-but-bounded Noise Estimation]
% Our {\purse} estimation methodology can be connected to early work in control theory on \emph{unknown-but-bounded noise estimation} which pointed out that bounded noise defines a feasible set for the unknown model (\cf \cite[eqs.~(28)-(29)]{milanese91automatica-optimal}). Despite being conceptually elegant, the framework was impractical due to (i) the difficulty to justify the unknown-but-bounded noise assumption, and (ii) the nonconvexity of the feasible set. In this paper we make the framework practical for object pose estimation by (i) using conformal prediction to obtain provably correct unknown-but-bounded noise~\eqref{eq:icpunify}, and (ii) developing {\ransag} to sample from {\purse} and SDP relaxations to compute estimation bounds.
% \end{remark}

\begin{remark}[Best Worst-case Error Bounds]
\label{rmk:bestbound}
\eqref{eq:pose2purse} can be used to bound errors for all possible pose estimators (\eg, from \pnp~\eqref{eq:uncertainpnp}). What is the best estimator that attains the smallest error bounds? This boils down to solving
\bea\label{eq:bestbound}
\min_{(\bar{R},\bar{t}) \in \SEthree} \left[ \max_{(R,t)\in \Seps} \lambda \Fnorm{R - \bar{R}}^2 + (1-\lambda)\norm{t - \bar{t}}^2\right]
\eea
whose solution is known as the \emph{Chebyshev center}~\cite{milanese91automatica-optimal,eldar08sp-minimax} of the \purse $\Seps$. Unfortunately, problem~\eqref{eq:bestbound} is more challenging than~\eqref{eq:pose2purse} and there is no efficient algorithm to solve it to global optimality. In the Supplementary Material, we evaluate the worst-case error bounds for multiple $(\bar{R},\bar{t})$ samples, select the smallest bounds, and compare them with those of the average pose. An interesting future research direction is to explore differentiable optimization~\cite{pineda22neurips-theseus} or bilevel polynomial optimization~\cite{nie17siopt-bilevel} to solve~\eqref{eq:bestbound}.
\end{remark}
\vspace{-4mm}


We present in section~\ref{ssec:faces} an application of PnP-HVAE on face images, using a pretrained state-of-the-art hierarchical VAE. 
Next, we study the application of our framework to natural images. To that end, we introduce  in section~\ref{ssec:patchVDVAE}  a patch hierachical VAE architecture, that is able to model natural images of different resolutions. In section~\ref{ssec:app_nat}, we provide deblurring, super-resolution and inpainting experiments to demonstrate the relevance of the proposed method.

Additional results are presented in Appendix~\ref{app:add}. All experiments can be reproduced using the code available at \url{https://github.com/jprost76/PnP-HVAE}.



\subsection{Face Image restoration (FFHQ)}\label{ssec:faces}
We first demonstrate the effectiveness of PnP-HVAE on highly structured data, by performing face image restoration.
Latent variable generative models can accurately model structured images such as face images \cite{karras2019style,vahdat2020nvae,child2021very,kingma2018glow}, and then be used to produce high quality restoration of such data. 
In our experiments, we use the VDVAE model of~\cite{child2021very}, pre-trained on the FFHQ dataset~\cite{karras2019style}, as our hierarchical VAE prior.
VDVAE has $L=66$ latent variable groups in its hierarchy and generates images at resolution $256\times256$.

We compare PnP-HVAE with the intermediate layer optimization algorithm (ILO)~\cite{daras2021intermediate} that is based on a different class of generative models than HVAE. ILO is a GAN inversion method which optimizes the image latent code along with the intermediate layer representation of a StyleGAN to generate an image consistent with a degraded observation.
We use the official implementation of ILO, along with a StyleGAN2 model~\cite{karras2020analyzing, stylegan2pytorch}, that was trained for 550k iterations on images of resolution $256\times256$ from FFHQ.  
As VDVAE and StyleGAN models are not trained on the same train-test split of FFHQ, we chose to evaluate the methods on a subset of 100 images from the CelebA dataset~\cite{liu2018large}. 
For super-resolution, the degradation model corresponds to the application of a gaussian low-pass filter followed by a $\times 4$ sub-sampling, and the addition of a gaussian white noise with $\sigma=3$.
For the deblurring, we considered motion blur and  gaussian kernels, both with a noise level $\sigma=8$. %

We provide quantitative comparisons in table~\ref{table:comp_ILO}, along with a visual comparison of the results in figure~\ref{fig:face_restoration}.
PnP-HVAE has the best  PSNR and SSIM results for all the considered restoration tasks, while ILO provides better results  for the perceptual distance.
By jointly optimizing the image and its latent variable, PnP-HVAE provides  results that are both realistic and consistent with the degraded observation.
On the other hand,  ILO  only optimizes on an extended latent space. This method generates  sharp and realistic images with better LPIPS scores,   
but the results lack  of consistency with respect to the observation, which explains the overall lower PSNR performance. 






\subsection{PatchVDVAE: a HVAE for natural images}\label{ssec:patchVDVAE}
Available generative models in the literature operate on images of  fixed resolutions and
are either restrained to datasets of limited diversity, or even to registered face images~\cite{kingma2018glow,child2021very, vahdat2020nvae, karras2019style}, or requiring additional class information~\cite{brock2018large, dhariwal2021diffusion, song2020score, luhman2022optimizing}.
Fitting an unconditional model on natural images appears to be a more difficult task, as their resolution can change, and their content is highly diverse.
The complexity of the problem can be reduced by learning a prior model on patches of reduced dimension. 
For image restoration problems, the patch model can be reused on images of higher dimensions~\cite{zoran2011learning,prost2021learning,altekruger2022patchnr}. When the model is a full CNN, the prior on the set of the  patches can  be computed efficiently by applying the network on the full image~\cite{prost2021learning}.

We thus introduce  patchVDVAE, a fully convolutional hierarchical VAE.
Contrary to existing HVAE models whose resolution is constrained by the constant tensor at the input of the top-down block, patchVDVAE can generate images of different resolutions by controlling the dimension of the input latent. 
This amounts to defining a prior on patches whose dimension corresponds to the receptive field of the VAE. A similar model is used for image denoising in~\cite{prakash2021interpretable}.

 
For PatchVDVAE architecture, we use the same bottom-up and top-down blocks as VDVAE~\cite{child2021very}, and replace the constant trainable input in the first top-down block by a latent variable, to make the model fully convolutional (details on the  architecture are given in Appendix~\ref{app:details}). 
The training dataset is composed of $128\times 128$ patches extracted from a combination of DIV2K~\cite{agustsson2017ntire} and Flickr2K~\cite{Lim_2017_CVPR_workshops} datasets.
We perform data augmentation by extracting  patches at $3$ resolutions: HR-images and $\times 2$ and $\times 4$ downscaled images. 
The model is trained for $7.10^5$ iterations with a batch size of $64$. Following the recommendation of~\cite{hazami2022efficient}, we use Adamax optimizer with an exponential moving average and gradient smoothing of the variance.
We set the decoder model to be a gaussian with diagonal covariance, as in~\cite{luhman2022optimizing}.
PatchVDVAE is fully convolutional and can generate images of dimension that are multiples of $64$ as illustrated by
figure~\ref{fig:vdvae}.

\newlength{\patchwidth}
\setlength{\patchwidth}{0.135\columnwidth}
\begin{figure}[!ht]
    \centering
    \begin{subfigure}[t]{.34\columnwidth}\hspace{0.1cm}
        \setlength{\tabcolsep}{0.02pt}
\renewcommand{\arraystretch}{0}
        \begin{tabular}{*{2}{p{1.03\patchwidth}}}
            \includegraphics[width=\patchwidth]{figures_arxiv/patchVDVAE/samples/generated/64x64/setup-5-image-0018.png} &
            \includegraphics[width=\patchwidth]{figures_arxiv/patchVDVAE/samples/generated/64x64/setup-5-image-0016.png} \\
            \includegraphics[width=\patchwidth]{figures_arxiv/patchVDVAE/samples/generated/64x64/setup-5-image-0008.png} &
            \includegraphics[width=\patchwidth]{figures_arxiv/patchVDVAE/samples/generated/64x64/setup-5-image-0019.png}   
        \end{tabular}
    \end{subfigure}\hspace{-0.15cm}
    \begin{subfigure}[t]{.64\columnwidth}
\begin{tabular}{cc}\vspace{-0.1cm}
\includegraphics[width=2\patchwidth]{figures_arxiv/patchVDVAE/samples/generated/256x256/setup-2-image-0009.png}&
        \includegraphics[width=2\patchwidth]{figures_arxiv/patchVDVAE/samples/generated/256x256/setup-2-image-0002.png}\end{tabular}

    \end{subfigure}
    \caption{\label{fig:vdvae} Left: $64\times64$ patches samples from our patchVDVAE model trained on patches from natural images.
    Right: PatchVDVAE is fully convolutional and it can generate images of higher resolution (here: $128\times128$).\vspace{-0.2cm}}
\end{figure}

\subsection{Natural images restoration}\label{ssec:app_nat}
We  evaluate PnP-HVAE on natural image restoration.
For each task, we report the average value of the PSNR, the SSIM, and the LPIPS metrics on $20$ images from the test set of the BSD dataset~\cite{MartinFTM01}.\\


\noindent
{\bf Image deblurring.}
In the experiments, we consider $2$ gaussian kernels and $2$ motion blur kernels from~\cite{levin2009understanding}, with $3$ different noise levels 
$\sigma \in \{2.55, 7.65, 12.75\}$.
As a baseline we consider  EPLL~\cite{zoran2011learning}, which learns a prior on image patches with a gaussian mixture model.
We also compare PnP-HVAE  with PnP-MMO and GS-PnP, $2$ competing convergent Plug-and-Play methods based on CNN denoisers.
PnP-MMO~\cite{pesquet2021learning} restricts the denoiser to be contraction in order to guarantee the convergence of the PnP forward-backard algorithm. GS-PnP~\cite{hurault2022gradient} considers a gradient step denoiser and reaches state-of-the-art performances of non converging methods~\cite{zhang2021plug}.
We set the temperature $\tau$  in our method as $0.95$, $0.8$ and $0.6$ for noise levels $2.55$, $7.65$ and $12.75$ respectively, and we let it run for a maximum of $50$ iterations. 
For the three compared methods we use the official implementations and pre-trained models provided by the respective authors. 
Details on the choice of hyperparameters for the concurrent methods are provided in the Appendix~\ref{app:details}
Figure~\ref{fig:deblurring_bsd} illustrates that our method provides correct deblurring results. 

According to table~\ref{tab:deb}, the performance of PnP-HVAE is between those of EPLL and GS-PnP and it outperforms PnP-MMO for large noise levels.\\

\begin{table}
\begin{center}\footnotesize
    \begin{tabular}{>{\centering}m{.3cm}*{5}{c}}
    $\sigma$ &Method & PSNR$\uparrow$ & SSIM$\uparrow$ & LPIPS$\downarrow$  \\ 
    \hline
    \multirow{4}{*}{\vcell{$2.55$}}
    & PnP-HVAE & $27.75$ & $0.79$ & $0.31$\\
    & GS-PNP \cite{hurault2022gradient} & $\mathbf{29.59}$ & $\mathbf{0.84}$ & $\mathbf{0.22}$\\
    & EPLL \cite{zoran2011learning} & $26.49$ & $0.71$ & $0.36$\\ 
    & PnP-MMO \cite{pesquet2021learning} & $\underbar{29.50}$ & $\underbar{0.83}$ & $\underbar{0.20}$ \\ \hline
    \multirow{4}{*}{\vcell{$7.65$}}
    & PnP-HVAE & $\underbar{26.36}$ & $\underbar{0.72}$ & $\underbar{0.40}$\\
    & GS-PNP \cite{hurault2022gradient} & $\mathbf{27.33}$ & $\mathbf{0.77}$ & $\mathbf{0.31}$\\
    & EPLL \cite{zoran2011learning} & $24.04$ & $0.66$ & $0.45$ \\ 
    & PnP-MMO \cite{pesquet2021learning} & $25.34$ & $0.69$ & $0.34$\\
    \hline
    \multirow{4}{*}{\vcell{$12.75$}}
    & PnP-HVAE & $\underbar{25.12}$ & $\mathbf{0.73}$ & $\underbar{0.47}$\\
    & GS-PNP \cite{hurault2022gradient} & $\mathbf{26.32}$ & $\mathbf{0.73}$ & $\mathbf{0.37}$\\
    & EPLL \cite{zoran2011learning} & $23.28$ & $0.61$ & $0.51$ \\ 
    & PnP-MMO \cite{pesquet2021learning} & $22.42$ & $0.53$& $0.54$ \\
    \hline
    &\vspace*{-.3cm}\\
            \multicolumn{2}{c}{Blur and motion kernels}& \multicolumn{3}{c}{
        \includegraphics*[scale=1]{figures_arxiv/kernels/4.png}\;\includegraphics*[scale=1]{figures_arxiv/kernels/7.png}\;\includegraphics*[scale=1]{figures_arxiv/kernels/9.png}\;\includegraphics*[scale=1]{figures_arxiv/kernels/11.png}} 
    \end{tabular}
        \caption{\label{tab:deb}Comparison  of PnP-HVAE  and other restoration methods on deblurring. Results are averaged on $4$ kernels.\vspace{-0.2cm}}% on image deblurring.}
    \end{center}
\end{table}

\begin{figure}
    
    \begin{subfigure}[h]{\linewidth}
        \centering
        \includegraphics*[width=\columnwidth]{figures_arxiv/deb_s255_k7.pdf}\vspace{-0.1cm}
        \caption{Gaussian blur, $\sigma=2.55$}
    \end{subfigure}
    \begin{subfigure}[h]{\linewidth}
        \centering
        \includegraphics*[width=\columnwidth]{figures_arxiv/deb_s765_k11.pdf}\vspace{-0.1cm}
        \caption{Motion blur, $\sigma=7.65$}
    \end{subfigure}\vspace*{-0.1cm}
    \caption{\label{fig:deblurring_bsd} Natural image deblurring\vspace{-0.1cm}}
\end{figure}

\noindent {\bf Effect of the temperature.}
PnP-HVAE gives control on the temperature of the prior over the latent space.
In figure~\ref{fig:temp_effect}, we illustrate that reducing the temperature increases the strength of the regularization prior. In this example the tuning $\tau=0.7$ produces the best performance.\\
\begin{figure}[!ht]
   
    \includegraphics[width=\columnwidth]{figures_arxiv/demo_temp.pdf}\vspace{-0.15cm}
    \caption{ \label{fig:temp_effect} Effect of the temperature in PnP-VAE on a deblurring problem, with $\sigma=7.65$.\vspace{-0.15cm}}
\end{figure}


\noindent
{\bf Image inpainting.}
Next we consider the task of noisy image inpainting. 
We compose a test-set of 10 images from the validation set of BSD~\cite{MartinFTM01} and we create masks
  by occluding diverse objects of small size in the images. 
A gaussian white noise with $\sigma=3$ is added to the images.
As a comparaison, we still consider GS-PnP and EPLL.
For PnP-HVAE, the temperature is set to $\tau=0.6$, and the algorithm is run for a maximum of $200$ iterations, unless the residual $||\x_{k+1}-\x_k||$ is on a plateau.
We provide on Table~\ref{tab:inpainting_bsd} the distortion metrics with the ground truth, as well as a visual
\begin{table}



\begin{center}
    \begin{tabular}{cccc}
        & PSNR$\uparrow$ & SSIM$\uparrow$ &LPIPS$\downarrow$ \\\hline
        PnP-HVAE  & $\mathbf{29.54}$ & $\mathbf{0.93}$ & $\mathbf{0.06}$\\
        GS-PNP & $28.52$ & $\mathbf{0.93}$ & $0.09$\\
        EPLL & $\underline{29.16}$ & $\mathbf{0.93}$ & $\mathbf{0.06}$\\
    \end{tabular}
    \caption{\label{tab:inpainting_bsd}Quantitative evaluation for inpainting on BSD.}
    \end{center}
\end{table}
comparison on figure~\ref{fig:inpainting_bsd}. 
With its hierarchical structure,  PnP-HVAE outperforms the compared methods. \vspace{0.05cm}



\begin{figure}[!h]
    \includegraphics[width=\columnwidth]{figures_arxiv/demo_inp_bsd2.pdf}\vspace{-0.1cm}
    \caption{\label{fig:inpainting_bsd}Natural image inpainting\vspace{-0.3cm}}
\end{figure}











\section{Conclusion}\label{sec:conclusion}
In this work, we focus on addressing the fundamental challenge of OOD detection tasks, which is how to fully understand the semantic discrepancy between the ID/OOD samples. We reveal that the key to success in the realistic SCOOD task is to allocate as many ID samples in the unlabeled set correctly as possible. To this end, we propose a novel uncertainty-aware optimal transport scheme that introduces class-specific energy scores as guidance for effective label assignment. Experimental results show that our method achieves better performance than previous state-of-the-art methods on SCOOD benchmarks.

\textbf{Limitations.} In addition to temperature scaling, other techniques such as feature clipping applied in ReAct~\cite{sun2021react} also enhance the performance of energy score, so how to obtain an OOD score that best fits the SCOOD task can be further explored. Moreover, a setting highly related to SCOOD has been proposed in \cite{katz2022training} and formulated as a constrained optimization problem. We will also theoretically analyze these practical OOD settings in our feature work.

% \section*{Acknowledgments}
\textbf{Acknowledgments.} 
This work is supported by National Key R\&D Program of China under Grant 2020AAA0105701, National Natural Science Foundation of China (NSFC) under Grants 61872327, Major Special Science and Technology Project of Anhui, National Natural Science Foundation of China (62033012) and Ant Group through Ant Research Intern Program.

The authors would like to acknowledge funding through the SNSF Sinergia grant called "Robust Deep Density Models for High-Energy Particle Physics and Solar Flare Analysis (RODEM)" with funding number CRSII$5\_193716$, the SNSF project grant 200020\_212127 called "At the two upgrade frontiers: machine learning and the ITk Pixel detector", and the Alexander von Humboldt foundation Feodor Lynen fellowship programme.

\clearpage
\onecolumn

\begin{center}
    {\Large \bf Supplementary Material}
\end{center}

\setcounter{section}{0}
\renewcommand{\thesection}{A\arabic{section}}
\renewcommand{\theequation}{A\arabic{equation}}
% \renewcommand{\theproposition}{A\arabic{proposition}}
\renewcommand{\thetheorem}{A\arabic{theorem}}
\renewcommand{\thefigure}{A\arabic{figure}}
\renewcommand{\thetable}{A\arabic{table}}


%!TEX root = ../main.tex

\section{Proof of Proposition~\ref{prop:invariance}}

\begin{proof} Let $\{ \alpha_{i}  \}_{i=1}^n$ be the calibration scores obtained by applying $r$ in \eqref{eq:maxnonconformity} to the calibration set, and $\{ \alpha^{\rho}_i \}_{i=1}^{n}$ be the scores obtained by applying $r_{\rho}$ in \eqref{eq:robustnonconformity}. Observe that $\alpha^{\rho}_i = \rho(\alpha_i)$ because $\rho$ being monotonically increasing implies $\max \circ \rho = \rho \circ \max$ (``$\circ$'' describes function composition). As a result, it follows that $\alpha^{\rho}_{\pi(\floor{(n+1)\epsilon})} = \rho(\alpha_{\pi(\floor{(n+1)\epsilon})})$.
Let $\Feps_{\rho}$ be the ICP set due to $r_{\rho}$ for a given $\epsilon$, we have
\bea
\Feps_{\rho} = \{ \vy \in \calY \mid  \max\{ \rho(\phi(y_k,f(x)_k)) \}_{k=1}^K \leq \alpha^{\rho}_{\pi(\floor{(n+1)\epsilon})}  \} \nonumber \\
= \{ \vy \in \calY \mid  \rho( \max\{ \phi(y_k,f(x)_k) \}_{k=1}^K ) \leq \rho(\alpha_{\pi(\floor{(n+1)\epsilon})} ) \} \nonumber\\
= \{ \vy \in \calY \mid \max\{ \phi(y_k,f(x)_k) \}_{k=1}^K \leq \alpha_{\pi(\floor{(n+1)\epsilon})} \}, \nonumber 
\eea
where the last set is precisely $\Feps$, the ICP set induced by $r$.
\end{proof}





%!TEX root = ../main.tex

\section{Proof of Proposition~\ref{prop:purse}}
\begin{proof} Recall the ICP set in~\eqref{eq:icpunify}
\bea\label{eq:icpunifyrestate}
\Feps(x) = \cbrace{\vy \in \calY \mid (y_k - \mu_k)\tran \Lambda_k (y_k - \mu_k) \leq 1,\forall k}
\eea
that defines either a \eqref{eq:icp-ball} or an \eqref{eq:icp-ellipse}. From the pinhole camera projection model, we know that the groundtruth keypoints $\vy = (y_1,\dots,y_K)$ satisfy
\bea\label{eq:projection}
y_k = \Pi (\Rgt Y_k + \tgt) = \frac{[P(\Rgt Y_k + \tgt)]_{1:2}}{[P(\Rgt Y_k + \tgt)]_{3}}, k=1,\dots,K 
\eea
where $P \in \Real{3\times 3}$ denotes the camera intrinsics, $Y_k \in \Real{3}$ is location of the $k$-th 3D keypoint in the object's coordinate frame, $[\vv]_{1:2}$ (resp. $[\vv]_3$) denotes the first two (resp. third) entries of a 3D vector $\vv$. To simplify our notation, we develop~\eqref{eq:projection} as
\bea
P\Rgt Y_k + P \tgt = (Y_k\tran \kron P) \vectorize{\Rgt} + P \tgt 
= \underbrace{\bmat{cc} Y_k\tran \kron P & P \emat}_{:= U_{k} \in \Real{3 \times 12}}
\bmat{c} \vectorize{\Rgt} \\ \tgt \emat = \bmat{c} u_{k,1}\tran \\ u_{k,2}\tran \\ u_{k,3}\tran \emat \sgt \\
\Longrightarrow y_k = \parentheses{\bmat{c} u_{k,1}\tran \\ u_{k,2}\tran \emat \sgt} \bigg/ (u_{k,3}\tran \sgt), \label{eq:simpleyk}
\eea
where $u_{k,j}\tran \in \Real{1 \times 12}$ denotes the $j$-th row of matrix $U_k$. Notice that $u_{k,3}\tran \sgt$ is the depth of the $k$-th 3D keypoint in the camera coordinate frame (after rigid transformation $(\Rgt,\tgt)$).
% Similarly, we write the transformed $k$-th 3D keypoint as
% \bea
% \Rgt Y_k + \tgt = (Y_k \kron \eye_3) \vectorize{\Rgt} + \tgt = \underbrace{\bmat{cc} Y_k\tran \kron \eye_3 & \eye_3 \emat}_{:=B_k \in \Real{3 \times 12}} \bmat{c} \vectorize{\Rgt} \\ \tgt \emat = \bmat{c} b_{k,1}\tran \\ b_{k,2}\tran \\ b_{k,3}\tran \emat \sgt.
% \eea

{\bf In front of the camera}. Since the camera observes the object, the groundtruth pose $\sgt$ must transform the object to lie in front of the camera. Therefore, the keypoints must have positive depth values:
\bea
u_{k,3}\tran \sgt > 0, k=1,\dots,K.
\eea

{\bf Within ICP sets}. We now insert~\eqref{eq:simpleyk} back to the constraint defined by the ICP set~\eqref{eq:icpunifyrestate}, leading to
\bea
(y_k - \mu_k)\tran \Lambda_k (y_k - \mu_k) \leq 1 \Longleftrightarrow \\ 
\frac{1}{(u_{k,3}\tran \sgt)^2} \sgt\tran \bmat{cc} u_{k,1} - \mu_{k,1} u_{k,3} & u_{k,2} - \mu_{k,2} u_{k,3} \emat \Lambda_k  \bmat{c} u_{k,1}\tran - \mu_{k,1} u_{k,3}\tran \\ u_{k,2}\tran - \mu_{k,2} u_{k,3} \emat \sgt \leq 1 \Longleftrightarrow \\
\sgt\tran \bmat{cc} u_{k,1} - \mu_{k,1} u_{k,3} & u_{k,2} - \mu_{k,2} u_{k,3} \emat \Lambda_k  \bmat{c} u_{k,1}\tran - \mu_{k,1} u_{k,3}\tran \\ u_{k,2}\tran - \mu_{k,2} u_{k,3} \emat \sgt \leq \sgt\tran \parentheses{u_{k,3} u_{k,3}\tran} \sgt \Longleftrightarrow \\
\sgt\tran \underbrace{\parentheses{ \bmat{cc} u_{k,1} - \mu_{k,1} u_{k,3} & u_{k,2} - \mu_{k,2} u_{k,3} \emat \Lambda_k  \bmat{c} u_{k,1}\tran - \mu_{k,1} u_{k,3}\tran \\ u_{k,2}\tran - \mu_{k,2} u_{k,3} \emat - u_{k,3} u_{k,3}\tran }}_{: = A_k \in \sym{12}} \sgt \leq 0, \label{eq:Ak}
\eea
which indicates that the groundtruth pose $\sgt$ must satisfy $K$ quadratic constraints, one for each keypoint.
In summary, the groundtruth pose $\sgt$ must lie in the \eqref{eq:purse} with $b_k = u_{k,3}$ and $A_k$ as in~\eqref{eq:Ak}.
\end{proof}
%!TEX root = ../main.tex

\section{Supplementary Experiments}

%!TEX root = ../main.tex
\begin{figure}[h]
\begin{center}
\begin{minipage}{\textwidth}
\centering
\begin{tabular}{cc}%
	    \begin{minipage}{5cm}%
		\centering%
		\includegraphics[width=\columnwidth]{R_err_bound_radius-maxp_0.10_relax-1.pdf}
		\end{minipage}
	&  	
	    \begin{minipage}{5cm}%
		\centering%
		\includegraphics[width=\columnwidth]{t_err_bound_radius-maxp_0.10_relax-1.pdf}
		\end{minipage}
	\\
	\multicolumn{2}{c}{(a) $\epsilon=0.1$}
	\\
	    \begin{minipage}{5cm}%
		\centering%
		\includegraphics[width=\columnwidth]{R_err_bound_radius-maxp_0.40_relax-1.pdf}
		\end{minipage}
	&  
	    \begin{minipage}{5cm}%
		\centering%
		\includegraphics[width=\columnwidth]{t_err_bound_radius-maxp_0.40_relax-1.pdf}
		\end{minipage}
	\\
	\multicolumn{2}{c}{(b) $\epsilon=0.4$}
\end{tabular}
\end{minipage}
\caption{Looser (albeit faster) worst-case error bounds computed from solving first-order relaxation of~\eqref{eq:pose2purse}, compared to worst-case error bounds computed from solving second-order relaxation shown in Fig.~\ref{fig:coverage-and-bound} middle and right columns. \label{fig:relax-order}} 
\end{center}
\vspace{-7mm}
\end{figure}

\subsection{Ablation: Relaxation Order}
In the main document, we briefly described that we applied second-order semidefinite relaxations to compute the worst-case error bounds in~\eqref{eq:pose2purse} and reported that the average runtime is around $8$ seconds on an ordinary workstation. Here we justify the choice of second-order relaxations by showing that first-order relaxations, although much faster (average runtime is about $0.1$ seconds), lead to much looser upper bounds for the optimization~\eqref{eq:pose2purse}. 

To help the reader better understand the approach, we first give a very short introduction to semidefinite relaxations for polynomial optimization problems (POPs). We refer the reader to~\cite[Section 2]{yang22pami-certifiably} for a detailed introduction.  

{\bf Polynomial optimization problems} (POPs) are problems of the following general formulation
\bea
\min_{x \in \Real{n}} & p(x) \\
\subject & h_i(x) = 0, i=1,\dots,l_h,\\
& g_j(x) \geq 0, j=1,\dots,l_g
\eea 
where $p,\{h_i\}_{i=1}^{l_h}, \{ g_j\}_{j=1}^{l_g}$ are all polynomial functions in $x \in \Real{n}$. Notice that if we denote $s = [\vectorize{R}\tran,t\tran]\tran \in \Real{12}$, it is clear that the cost function of~\eqref{eq:pose2purse} is a polynomial in $s$ when fixing a particular $\lambda$ (we can add a minus sign to the cost of~\eqref{eq:pose2purse} so that we convert ``$\max$'' to ``$\min$''). The constraints for~\eqref{eq:pose2purse} is $(R,t) \in \Seps$ where $\Seps$ has the form in~\eqref{eq:purse}. We claim that the~\eqref{eq:purse} can be described by a set of polynomial equalities and inequalities. This is because (i) the rotation constraint $R \in \SOthree$ can be described by $15$ quadratic equality constraints~\cite{yang22pami-certifiably}; (ii) the quadratic constraints in~\eqref{eq:purse} are already polynomial constraints; and (iii) the linear inequalities $b_k\tran s > 0$ can be equivalently written as $b_k\tran s \geq \epsilon$ for a small $\epsilon > 0$ (note that $b_k\tran s$ is the depth of the 3D keypoints, so it makes sense to enforce they are larger than, say $\epsilon = 0.001$). We conclude that computing the worst-case error bounds~\eqref{eq:pose2purse} is a POP.

{\bf Semidefinite relaxations} are a powerful tool to approximate (or even exactly compute) the \emph{global optimal} solutions for (generally nonconvex) POPs. In particular, Lasserre's hierarchy of moment and sums-of-squares relaxations~\cite{lasserre01global} provides a systematic approach to design such semidefinite relaxations. In particular, Lasserre's hierarchy relaxes a POP into a hierarchy of convex semidefinite programs (SDPs) of increasing size. Each relaxation, at a so-called \emph{relaxation order}, in this hierarchy can be solved in polynomial time and provides a valid lower bound for the POP (if the POP aims to maximize, as in~\eqref{eq:pose2purse}, then a valid upper bound is provided). Moreover, under mild technical conditions, the lower (upper) bounds of these relaxations coincide with the global optimum of the original POP, in which case we say the relaxation is \emph{exact}, or \emph{tight}.

{\bf First-order vs. second-order relaxations}. The minimum relaxation order for the POP~\eqref{eq:pose2purse} is $1$, since all the polynomials in~\eqref{eq:pose2purse} have degree at most $2$ (in general, the minimum relaxation order for a POP is $\lceil d/2 \rceil$, where $d$ is the maximum degree of the polynomials defining a POP). In practice we choose a second-order relaxation instead of a first-order relaxation because first-order relaxations give loose upper bounds for problem~\eqref{eq:pose2purse}. Fig.~\ref{fig:relax-order} plots the worst-case error bounds computed by solving the first-order relaxation of~\eqref{eq:pose2purse} under the same {\gtball} setup. Compared to Fig.~\ref{fig:coverage-and-bound} middle and right columns, we clearly see that solving the first-order relaxation produces overly conservation upper bounds for~\eqref{eq:pose2purse}. For example, when $\epsilon=0.1$, solving the first-order relaxation never produces a rotation error bound that is below $100^\circ$, while in Fig.~\ref{fig:coverage-and-bound} we see a cluster of blue circles near the bottom left corner indicating tight bounds.

One nice property of applying semidefinite relaxations is that we get a certificate of global optimality when the relaxation is indeed exact. Such certificates typically come in the form of a rank-one optimal SDP solution, or a relative suboptimality gap (\cf \cite[eq. (24)]{yang22pami-certifiably}), which indicates exactness of the relaxation when the value is numerically zero (loosely speaking, a relative suboptimality gap of $\epsilon\%$ means that the global optimum of the SDP is at most $\epsilon$ percentage away from the global optimum of the POP). When we solve second-order relaxations of problem~\eqref{eq:pose2purse} under the {\gtball} setup with $\lambda = 1$, we obtain a relative suboptimality gap that is below $10^{-3}$ (resp. $10^{-6}$) for $99.02\%$ (resp. $72.51\%$) of the $8784$ test problems, indicating that the second-order relaxation is sufficient to obtain (approximately) globally optimal solutions for problem~\eqref{eq:pose2purse}.

\subsection{Qualitative ICP Sets}

Fig.~\ref{fig:methodoverview}(b) shows circular and elliptical examples of the ICP sets. Fig.~\ref{fig:heatmap-qualitative} provides more examples of the ICP sets with $\epsilon=0.1$ and $\epsilon=0.4$. Notice how the ICP sets become smaller when $\epsilon$ increases.

%!TEX root = ../main.tex
\newcommand{\myimgwidth}{0.85\textwidth}
\begin{figure*}[h]
\centering
\vspace{-8mm}
\includegraphics[width=\myimgwidth]{0.10/radius-maxp/000051_000051_08.pdf}\\
\vspace{-1mm}
\includegraphics[width=\myimgwidth]{0.10/radius-cov-topk/000051_000051_08.pdf}\\
\vspace{-1mm}
\includegraphics[width=\myimgwidth]{0.10/radius-maxp/000051_000051_08_frcnn.pdf}\\
\vspace{-1mm}
\includegraphics[width=\myimgwidth]{0.10/radius-cov-topk/000051_000051_08_frcnn.pdf}\\
\vspace{-2mm}
{\small (a) $\epsilon=0.1$, object: \emph{driller}, top to bottom: \gtball, \gtellipse, \frcnnball, \frcnnellipse  }\\
\includegraphics[width=\myimgwidth]{0.10/radius-maxp/000083_000083_09.pdf}\\
\vspace{-1mm}
\includegraphics[width=\myimgwidth]{0.10/radius-cov-topk/000083_000083_09.pdf}\\
\vspace{-1mm}
\includegraphics[width=\myimgwidth]{0.10/radius-maxp/000083_000083_09_frcnn.pdf}\\
\vspace{-1mm}
\includegraphics[width=\myimgwidth]{0.10/radius-cov-topk/000083_000083_09_frcnn.pdf}\\
\vspace{-2mm}
{\small (b) $\epsilon=0.1$, object: \emph{duck}, top to bottom: \gtball, \gtellipse, \frcnnball, \frcnnellipse  }\\
\includegraphics[width=\myimgwidth]{0.40/radius-maxp/000051_000051_08.pdf}\\
\vspace{-1mm}
\includegraphics[width=\myimgwidth]{0.40/radius-cov-topk/000051_000051_08.pdf}\\
\vspace{-1mm}
\includegraphics[width=\myimgwidth]{0.40/radius-maxp/000051_000051_08_frcnn.pdf}\\
\vspace{-1mm}
\includegraphics[width=\myimgwidth]{0.40/radius-cov-topk/000051_000051_08_frcnn.pdf}\\
\vspace{-2mm}
{\small (c) $\epsilon=0.4$, object: driller, top to bottom: \gtball, \gtellipse, \frcnnball, \frcnnellipse}\\
\includegraphics[width=\myimgwidth]{0.40/radius-maxp/000083_000083_09.pdf}\\
\vspace{-1mm}
\includegraphics[width=\myimgwidth]{0.40/radius-cov-topk/000083_000083_09.pdf}\\
\vspace{-1mm}
\includegraphics[width=\myimgwidth]{0.40/radius-maxp/000083_000083_09_frcnn.pdf}\\
\vspace{-1mm}
\includegraphics[width=\myimgwidth]{0.40/radius-cov-topk/000083_000083_09_frcnn.pdf}\\
\vspace{-2mm}
{\small (d) $\epsilon=0.4$, object: \emph{duck}, top to bottom: \gtball, \gtellipse, \frcnnball, \frcnnellipse  }\\
\vspace{-3mm}
\caption{ICP sets on {\lmo}~\cite{brachmann14eccv-linemodocc}. Last image of each row overlays \emph{all} groundtruth keypoints (squares) and ICP sets (balls \& ellipses) on the original image. Other images overlay the heatmap, groundtruth location, and ICP set of a \emph{single} keypoint on the original image.
\label{fig:heatmap-qualitative}} 
\end{figure*}


\subsection{Worst-case Error Bounds under {\gtellipse}, {\frcnnball}, and {\frcnnellipse} setups}
Fig.~\ref{fig:coverage-and-bound} middle and right columns (from the main document) show the worst-case error bounds (computed from~\eqref{eq:pose2purse}) of the average pose under the {\gtball} setup. Fig.~\ref{fig:other-error-bounds} shows the worst-case error bounds under the {\gtellipse}, {\frcnnball}, and {\frcnnellipse} setups, which are qualitatively similar to Fig.~\ref{fig:coverage-and-bound}. Notice that the blue circles never cross the $y=x$ diagonal, indicating our bounds are always valid when the \purse contains the groundtruth pose.

%!TEX root = ../main.tex
\begin{figure}[h]
\begin{center}
\begin{minipage}{\textwidth}
\centering
\begin{tabular}{cccc}%
	    \begin{minipage}{4cm}%
		\centering%
		\includegraphics[width=\columnwidth]{R_err_bound_radius-cov-topk_0.10.pdf}
		\end{minipage}
	&  	
	    \begin{minipage}{4cm}%
		\centering%
		\includegraphics[width=\columnwidth]{t_err_bound_radius-cov-topk_0.10.pdf}
		\end{minipage}
	&
	\begin{minipage}{4cm}%
		\centering%
		\includegraphics[width=\columnwidth]{R_err_bound_radius-cov-topk_0.40.pdf}
		\end{minipage}
	&  	
	    \begin{minipage}{4cm}%
		\centering%
		\includegraphics[width=\columnwidth]{t_err_bound_radius-cov-topk_0.40.pdf}
		\end{minipage}
	\\
	\multicolumn{4}{c}{(a) \gtellipse. Left two columns: $\epsilon=0.1$; right two columns: $\epsilon=0.4$. }
	\\
	    \begin{minipage}{4cm}%
		\centering%
		\includegraphics[width=\columnwidth]{R_err_bound_radius-maxp_0.10_frcnn.pdf}
		\end{minipage}
	&  	
	    \begin{minipage}{4cm}%
		\centering%
		\includegraphics[width=\columnwidth]{t_err_bound_radius-maxp_0.10_frcnn.pdf}
		\end{minipage}
	&
	\begin{minipage}{4cm}%
		\centering%
		\includegraphics[width=\columnwidth]{R_err_bound_radius-maxp_0.40_frcnn.pdf}
		\end{minipage}
	&  	
	    \begin{minipage}{4cm}%
		\centering%
		\includegraphics[width=\columnwidth]{t_err_bound_radius-maxp_0.40_frcnn.pdf}
		\end{minipage}
	\\
	\multicolumn{4}{c}{(b) \frcnnball. Left two columns: $\epsilon=0.1$; right two columns: $\epsilon=0.4$. }
	\\
	\begin{minipage}{4cm}%
		\centering%
		\includegraphics[width=\columnwidth]{R_err_bound_radius-cov-topk_0.10_frcnn.pdf}
		\end{minipage}
	&  	
	    \begin{minipage}{4cm}%
		\centering%
		\includegraphics[width=\columnwidth]{t_err_bound_radius-cov-topk_0.10_frcnn.pdf}
		\end{minipage}
	&
	\begin{minipage}{4cm}%
		\centering%
		\includegraphics[width=\columnwidth]{R_err_bound_radius-cov-topk_0.40_frcnn.pdf}
		\end{minipage}
	&  	
	    \begin{minipage}{4cm}%
		\centering%
		\includegraphics[width=\columnwidth]{t_err_bound_radius-cov-topk_0.40_frcnn.pdf}
		\end{minipage}
	\\
	\multicolumn{4}{c}{(c) \frcnnellipse. Left two columns: $\epsilon=0.1$; right two columns: $\epsilon=0.4$. }
\end{tabular}
\end{minipage}
\vspace{-4mm}
\caption{ Worst-case error bounds under (a) \gtellipse, (b) \frcnnball, and (c) \frcnnellipse setups. $x$-axis represents the worst-case error bounds computed from~\eqref{eq:pose2purse}, $y$-axis represents the actual error between average pose and groundtruth pose. The area below the diagonal $y=x$ indicates correctness of the bounds (\ie, bound $\geq$ error), and points that are closer to the diagonal from below indicate \emph{tighter} bounds (perfect if precisely lie on the diagonal). Blue circles plot cases where the {\purse} covers the groundtruth pose and red squares plot cases were the {\purse} does not cover the groundtruth. Notice that blue circles never cross the diagonal and our bounds are correct when the \purse contains the pose (which holds with $1-\epsilon$ marginal probability).\label{fig:other-error-bounds}} 
\end{center}
\vspace{-7mm}
\end{figure}

\subsection{A Closer Look at the Conservative Error Bounds}
The reader may have noticed two unusual results in the experiments on \lmo. First, the success rate on eggbox is consistently lower than other categories in our methods and other baselines (\eg, PVNet achieves $8.43\%$ success rate on eggbox, while the second lowest success rate is $55.37\%$). Second, the worst-case error bounds can be overly conservative, \eg, having $180^\circ$ rotation error bounds. It turns out both unusual results can be explained by the same reason: a labelling discrepancy in the {\lmo} dataset about eggbox.

We noticed the low success rate on eggbox across all baseline methods and contacted the authors of~\cite{schmeckpeper22jfr-semantic}, who encountered the same problem. One author told us ``\emph{I think this is a mistake or discrepancy in the 6DoF annotations of the dataset itself. As it [the eggbox] is considered a symmetric object, annotators for LMOD might not have consistently annotate it}''. Though it is possible to revise the nonconformity score for symmetric objects, the manually chosen keypoints by~\cite{schmeckpeper22jfr-semantic} break the symmetry. Therefore we decided to leave this discrepancy as is because it does not affect our probabilistic guarantees.
% even though the eggbox is actually not completely symmetric. 

This labeling discrepancy, however, does translate to \emph{ conservative} prediction sets for the eggbox, in order to contain the (wrong) groundtruth at the desired probability. Fig.~\ref{supp:fig:eggbox} shows the eggbox prediction sets are \emph{one order of magnitude} larger than the other categories, leading to worst-case rotation error bounds being mostly $180^\circ$ (because the \purse is large enough to cover the entire $\SOthree$). {This indeed shows the advantage of our framework}: \emph{the user will see the large uncertainty produced by our algorithm and be alerted}!

Finally, because the \purse is too large, \ransag essentially 
returns a random sample in $\SOthree$, which has zero probability being close to the (wrong) groundtruth. Hence, a $0\%$ success rate makes sense. 

\begin{figure}
    \includegraphics*[width=0.49\linewidth]{ball_size.pdf}
    \hfill
    \includegraphics*[width=0.4\linewidth]{bound.pdf}
    \vspace{-3mm}
    \caption{Left: cumulative distribution (CDF) of the average radius of prediction sets (under \gtball).
    Right: CDF of the worst-case rotation error bounds; \emph{eggbox} error bounds are mostly $180^\circ$. \label{supp:fig:eggbox}}
    \vspace{-4mm}
 \end{figure}

\subsection{Best Worst-case Error Bounds from Samples (Remark~\ref{rmk:bestbound})}

In Remark~\ref{rmk:bestbound}, we discussed that since solving~\eqref{eq:pose2purse} can provide worst-case error bounds for any pose estimator, the natural question is to ask if we can find better pose estimators (than the average pose computed from {\ransag}) with tighter worst-case error bounds, which boils down to solving the minimax problem in~\eqref{eq:bestbound}. However, problem~\eqref{eq:bestbound} is much more challenging to solve than~\eqref{eq:pose2purse}, and to the best of our knowledge, there is no efficient way to obtain a globally optimal solution. We think a good future research direction may be to explore methods in~\cite{pineda22neurips-theseus} or~\cite{nie17siopt-bilevel} for solving~\eqref{eq:bestbound}.

In this section, we provide a very preliminary study to explore if~\eqref{eq:bestbound} can indeed offer us tighter error bounds. Towards this goal, we randomly select $M=5$ pose samples $\{(R_i,t_i)\}_{i=1}^M$ from the results of {\ransag} (recall \ransag not only returns an average pose, but also returns a set of poses), and compute
\bea\label{eq:sampleworstbound}
\underline{d}^2_{\epsilon,\lambda} = \min \cbrace{  {d}^2_{i, \epsilon,\lambda} = \max_{(R,t) \in \Seps} \lambda \Fnorm{R - R_i}^2 + (1-\lambda) \norm{ t - t_i}^2  }_{i=1}^M,
\eea
which first solves~\eqref{eq:pose2purse} (inner ``$\max$'' in~\eqref{eq:sampleworstbound}) for each $(R_i,t_i)$ and then selects the minimum (tightest) error bounds. Note that we still apply a second-order SDP relaxation when computing the error bounds for each $(R_i,t_i)$ since~\eqref{eq:pose2purse} is nonconvex. 

Fig.~\ref{fig:avg-vs-smp} plots the cumulative distribution functions (CDF) of the error bounds under the {\gtball} setup with $\epsilon=0.1$. The blue curves plot the CDF of the error bounds computed for the average pose, while the red curves plot the CDF of the error bounds computed from solving~\eqref{eq:sampleworstbound}. We can see that solving~\eqref{eq:sampleworstbound} does slightly improve the tightness of the translation bounds (while the rotation bounds are very close). Considering that we only select the minimum error bounds from $M=5$ samples, we conjecture solving the minimax problem can give us much tighter error bounds, and we leave this as an exciting future research.

%!TEX root = ../main.tex
\begin{figure}[h]
\begin{center}
\begin{minipage}{\textwidth}
\centering
\begin{tabular}{cc}%
	    \begin{minipage}{5cm}%
		\centering%
		\includegraphics[width=\columnwidth]{R_bound_avg_vs_smp.pdf}
		\end{minipage}
	&  	
	    \begin{minipage}{5cm}%
		\centering%
		\includegraphics[width=\columnwidth]{t_bound_avg_vs_smp.pdf}
		\end{minipage}
\end{tabular}
\end{minipage}
\vspace{-4mm}
\caption{Cumulative distribution function (CDF) of the worst-case error bounds under the {\gtball} setup with $\epsilon=0.1$. Blue curve plots the CDF of the error bounds of the average pose, and red curve plots the CDF of the minimum error bounds of the pose samples (\ie solving~\eqref{eq:sampleworstbound}). We can see that the translation error bounds are slightly tightened by selecting the minimum error bounds for multiple pose samples. \label{fig:avg-vs-smp}} 
\end{center}
\vspace{-7mm}
\end{figure}


%%%%%%%%% REFERENCES
{\small
\bibliographystyle{ieee_fullname}
\bibliography{refs}
}

\end{document}
