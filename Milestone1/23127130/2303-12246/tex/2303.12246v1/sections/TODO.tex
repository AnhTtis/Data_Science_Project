%!TEX root = ../main.tex

\section{To be added in a full conference submission}

% \red{This is a nonconformity function that has been shown to be quite effective for classification problems. However, in the Appendix we show that this nonconformity function leads to prediction sets that are larger than necessary.}
% We design the \emph{nonconformity} function as
% \bea 
% r(y,f(x)) = \sum \cbrace{ f(x)_{i} \mid f(x)_{i} \geq f(x)_y  }
% \eea
% where $f(x)_y$ is the probability of the semantic keypoint lying on the pixel that is closest to $y$. Intuitively, the nonconformity function sums up all the probability masses at pixel locations where the true label is more likely to lie than $y$. Then, according to \eqref{eq:icpcompute}, we have the prediction set being
% \bea\label{eq:icp-heatmap-topk}
% \Feps (x_{l+1}) = \{\pi(1),\dots,\pi(k^\star) \}, \text{where } k^\star = \max \cbrace{ k \ \middle\vert\  \sum_{i=1}^{k} f(x_{l+1})_{\pi(k)} \leq \alpha_{\pi(\floor{(n+1)\epsilon}) }  }
% \eea
% which outputs the top-$k$ most likely pixel locations whose probability masses sum up to no greater than $\alpha_{\pi(\floor{(n+1)\epsilon}) }$ (in eq.~\eqref{eq:icp-heatmap-topk} $f(x_{l+1})_{\pi(k)}$ is the $k$-th highest probability).


{\bf Pixel-wise Voting} (PVNet)~\cite{peng19cvpr-pvnet}. A drawback of heatmap-based methods is that they cannot predict the location of the semantic keypoint when the object is partially cropped from the image (since the heatmap cannot cross the boundaries of the image). PVNet addresses this issue by instead learning pixel-wise directional vectors that point to the semantic keypoint. Formally, we have $f(x) = \{ (p_i,v_i) \}_{i=1}^t$ where $p_i \in \Real{2}$ is the pixel location, $v_i \in \usphere{2}$ is a unit vector pointing from the pixel to the semantic keypoint, and $t \leq HW$ is the number of pixel-wise directions. To design a nonconformity function, we take inspiration from outlier-robust geometric estimation~\cite{yang22pami-certifiably,antonante21tro-outlier}. We first generate a set of candidate keypoint locations
\bea
\calQ = \cbrace{ q_k \in \Real{2} \mid q_k = \calH(p_i,v_i) \cap \calH(p_j,v_j), \forall i \neq j \in [t]}
\eea
by intersecting any pair of half-lines generated by PVNet, where $\calH(p_i,v_i) := \{p \in \Real{2} \mid p = p_i + \tau v_i, \tau \geq 0 \}$ is the half-line starting at $p_i$ with direction $v_i$ (we skip pairs of half-lines that do not intersect). We assume the total number of candidates is $K = \abs{\calQ}$. We then solve the following \emph{truncated least squares} estimation problem
\bea
q^\star = \argmin_{q \in \Real{2}} \sum_{k=1}^K \min \cbrace{ \frac{\norm{q - q_k}^2}{\beta^2}, 1 }
\eea
using the graduated non-convexity algorithm~\cite{yang20ral-gnc} to obtain a single estimation of the semantic keypoint location that is robust against the potential outliers in the set $\calQ$. We let $\calQ_{\mathrm{I}} = \{q_k \in \calQ \mid \norm{q^\star - q_k}^2 / \beta^2 \leq 1 \}$ be the set of estimated inliers with cardinality $K_{\mathrm{I}}$, and compute
\bea
\Sigma = \frac{1}{K_{\mathrm{I}}} \sum_{q_k \in \calQ_{\mathrm{I}}} (q_k - q^\star)(q_k - q^\star)\tran \succ 0
\eea
as the covariance matrix describing the uncertainty of the semantic keypoint computed from only the inliers (we assume $K_{\mathrm{I}} \geq 2$ noncollinear inliers such that the covariance matrix is full-rank). With $(q^\star, \Sigma)$, the nonconformity function is
\bea
r(y,f(x)) = (y - q^\star)\tran \Sigma\inv (y - q^\star).
\eea
According to eq.~\eqref{eq:icpcompute}, we have the prediction set for a new sample $x_{l+1}$ be the following ellipsoid
\bea
\Feps(x_{l+1}) = \{ y \in \Real{2} \mid (y-q^\star_{l+1})\tran \Sigma\inv_{l+1} (y-q^\star_{l+1}) \leq  \alpha_{\pi(\floor{(n+1)}\epsilon)} \}.
\eea