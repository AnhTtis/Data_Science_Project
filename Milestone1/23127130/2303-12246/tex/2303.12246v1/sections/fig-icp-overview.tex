%!TEX root = ../main.tex
% \begin{figure}
% \hspace{-4mm}\includegraphics[width=1.1\columnwidth]{icp-overview-half.pdf}
% \caption{Given a learned prediction function and a calibration set of $n$ samples, conformal calibration uses a nonconformity function~\eqref{eq:nonconformity} to compute and sort nonconformity scores $\{ \alpha_{\pi(i)}\}_{i=1}^n$. Given a new observation and an error rate $\epsilon$, conformal prediction~\eqref{eq:icpcompute} outputs a prediction set of all labels under which the nonconformity score of the new sample is no larger than $\alpha_{\pi(\floor{(n+1)\epsilon})}$.
% \label{fig:icp-overview}}
% \end{figure}
\begin{figure}
\vspace{-4mm}
\begin{center}
\includegraphics[width=0.6\columnwidth]{beta.pdf}
\end{center}
\vspace{-6mm}
\caption{Beta distribution of the conditional coverage in~\eqref{eq:beta} with $\epsilon=0.1$ and different $n$. Notice how the conditional probability becomes more concentrated around $1-\epsilon$ when $n$ increases.
\label{fig:beta-distribution}}
\vspace{-7mm}
\end{figure}