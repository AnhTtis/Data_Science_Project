%!TEX root = ../main.tex
\vspace{-6mm}
\section{Introduction}
Estimating object poses from images is a fundamental problem in computer vision and finds extensive applications in augmented reality~\cite{klein07mar-ptam}, autonomous driving~\cite{shi21rss-optimal}, robotic manipulation~\cite{manuelli19isrr-kpam}, and space robotics~\cite{chen19iccvw-satellite}. One of the most popular paradigms for object pose estimation is a \emph{two-stage} pipeline~\cite{peng19cvpr-pvnet,pavlakos17icra-semantic,schmeckpeper22jfr-semantic,tekin18cvpr-yolo,zakharov19iccv-dpod,sun22cvpr-onepose,shi22arxiv-optimal,chen20cvpr-backproppnp}, where the first stage detects (semantic) {keypoints} of the objects on the image, and the second stage computes the object pose by solving an optimization known as \emph{Perspective-$n$-Points} (\pnp) that minimizes {reprojection errors} of the detected keypoints.

\emph{Safety-critical} applications call for \emph{provably correct} computer vision algorithms. Existing algorithms in the two-stage paradigm (reviewed in Section~\ref{sec:related-work}), however, provide few performance guarantees on the quality of the estimated poses, due to three challenges. (C1) It is difficult to ensure the detected keypoints (typically from neural networks) are close to the groundtruth keypoints. In practice, the first stage often outputs keypoints that are arbitrarily wrong, known as \emph{outliers}. (C2) Robust estimation is employed in the second stage to reject outliers, leading to nonconvex optimizations. Fast heuristics such as \ransac~\cite{fischler81acm-ransac} are widely adopted to find an approximate solution but they cannot guarantee global optimality and often fail without notice. (C3) There is no provably correct \emph{uncertainty quantification} of the estimation, notably, a \emph{formal worst-case error bound} between the estimation and the groundtruth. Though recent work~\cite{yang22pami-certifiably} proposed convex relaxations to certify global optimality of~\ransac and addressed (C2), it cannot ensure correct estimation as the optimal pose may be far away from the correct pose when the keypoints are unreliable.


{\bf Contributions}. We propose a two-stage object pose estimation framework with \emph{statistical guarantees}, illustrated in Fig.~\ref{fig:methodoverview}. Given an input image, we assume a neural network~\cite{pavlakos17icra-semantic} is available to generate \emph{heatmap} predictions of the object keypoints (Fig.~\ref{fig:methodoverview}(a)). Our framework then proceeds in two stages, namely \emph{conformal keypoint detection} (Section~\ref{sec:keypoint:conformal}) and \emph{geometric uncertainty propagation} (Section~\ref{sec:uncertainty:propagation}). We first apply the statistical machinery of inductive conformal prediction (introduced in Section~\ref{sec:pre:icp}), with \emph{nonconformity} functions inspired by the design of residual functions in classical geometric vision~\cite{kahl08tpami-multiple}, to conformalize the heatmaps into circular or elliptical prediction sets --one for each keypoint-- that guarantee coverage of the groundtruth keypoints with a user-specified \emph{marginal} probability (Fig.~\ref{fig:methodoverview}(b)). This provides a simple and general methodology to bound the keypoint prediction errors (\ie, addressing (C1)). 
Given the keypoint prediction sets,
% , instead of solving an (uncertainty-aware) {\pnp} problem whose solution would come with no formal error bounds, 
we reformulate the constraints (enforced by the prediction sets) on the keypoints as constraints on the object pose, leading to a \emph{Pose UnceRtainty SEt} (\purse) that guarantees coverage of the groundtruth pose with the same probability. Fig.~\ref{fig:methodoverview}(c) plots the boundary of an example {\purse} (roll, pitch, raw angles for the rotation, and Euclidean coordinates for the translation). The {\purse}, however, is an abstract nonconvex set that does not directly admit estimated poses and uncertainty. Therefore, we develop \emph{RANdom SAmple averaGing} (\ransag) to compute an average pose (Fig.~\ref{fig:methodoverview}(d)) and employ semidefinite relaxations to upper bound the worst-case rotation and translation errors between the average pose and the groundtruth (Fig.~\ref{fig:methodoverview}(e)). This gives rise to the first kind of \emph{computable} worst-case probabilistic error bounds for object pose estimation (\ie, addressing (C3)). Our {\purse} methodology has connections to the framework of \emph{unknown-but-bounded} noise estimation in control theory~\cite{milanese91automatica-optimal}, with special provisions to derive the bounds in a statistically principled way and enable efficient computation.

We test our framework on the LineMOD Occlusion (\lmo) dataset~\cite{brachmann14eccv-linemodocc} to verify the correctness of the theory (Section~\ref{sec:experiments}). First, we empirically show that the {\purse} indeed contains the groundtruth pose according to the user-specified probability. Second, we demonstrate the correctness of the worst-case error bounds: when the {\purse} contains the groundtruth, our bounds are always larger than, and in many cases close to, the actual errors between the average pose and the groundtruth pose. Third, we benchmark the accuracy of the average pose (coming from \ransag) with representative two-stage pipelines based on sparse keypoints (\eg, PVNet~\cite{peng19cvpr-pvnet}) and show that the average pose achieves better or similar accuracy.

{\bf Limitations}. A drawback of our approach, and conformal prediction in general, is that the size of the prediction sets depends on the nonconformity function (whose design can be an art) and may be conservative. 
Our experiments suggest the bounds are loose when the keypoint prediction sets are large (\eg, giving $180^\circ$ rotation bound). We discuss challenges and opportunities in tightening the bounds.


% {\bf Paper organization}. Section~\ref{sec:related-work} briefly reviews related work. Section~\ref{sec:pre:icp} gives a self-contained introduction of the inductive conformal prediction (ICP) machinery. Section~\ref{sec:keypoint:conformal} applies ICP to keypoint detection and discusses the connections between the nonconformity function and classical geometric vision. Section~\ref{sec:uncertainty:propagation} derives the pose uncertainty set (\purse) and algorithms for computing the average pose and bounding worst-case errors. Section~\ref{sec:experiments} demonstrates experimental results and Section~\ref{sec:conclusion} provides concluding remarks.