%!TEX root = ../main.tex

%!TEX root = ../main.tex
% \newcommand{\mpwthree}{4.8cm}
\newcommand{\coverwidth}{5.4cm}
\newcommand{\boundwidth}{4.2cm}
\begin{figure*}
\vspace{-8mm}
\begin{center}
\begin{minipage}{\textwidth}
\centering
\begin{tabular}{ccc}%
	    \begin{minipage}{\coverwidth}%
		\centering%
		\includegraphics[width=\columnwidth]{coverage_heatmap_0.10.pdf}
		\end{minipage}
	&  
	    \begin{minipage}{\boundwidth}%
		\centering%
		\includegraphics[width=\columnwidth]{R_err_bound_radius-maxp_0.10.pdf}
		\end{minipage}
	&  	
	    \begin{minipage}{\boundwidth}%
		\centering%
		\includegraphics[width=\columnwidth]{t_err_bound_radius-maxp_0.10.pdf}
		\end{minipage}
	\\
		\begin{minipage}{\coverwidth}%
		\centering%
		\includegraphics[width=\columnwidth]{coverage_heatmap_0.40.pdf}
		\end{minipage}
	& 
	    \begin{minipage}{\boundwidth}%
		\centering%
		\includegraphics[width=\columnwidth]{R_err_bound_radius-maxp_0.40.pdf}
		\end{minipage}
	&  
	    \begin{minipage}{\boundwidth}%
		\centering%
		\includegraphics[width=\columnwidth]{t_err_bound_radius-maxp_0.40.pdf}
		\end{minipage}
\end{tabular}
\vspace{-4mm}
\end{minipage}
\caption{Empirical coverage (left) and worst-case error bounds (middle: rotation, right: translation).  Top: $\epsilon=0.1$, bottom: $\epsilon=0.4$. For middle and right columns, $x$-axis represents the worst-case error bounds computed from~\eqref{eq:pose2purse}, $y$-axis represents the actual error between average pose and groundtruth pose. The area below the diagonal $y=x$ indicates correctness of the bounds (\ie, bound $\geq$ error), and points that are closer to the diagonal from below indicate \emph{tighter} bounds (perfect if precisely lie on the diagonal). Blue circles plot cases where the {\purse} covers the groundtruth pose and red squares plot cases were the {\purse} does not cover the groundtruth. Notice that blue circles never cross the diagonal and our bounds are correct when the \purse contains the pose (which holds with $1-\epsilon$ marginal probability).  
\label{fig:coverage-and-bound}} 
\end{center}
\vspace{-7mm}
\end{figure*}



% \includegraphics[width=0.9\columnwidth]{coverage_heatmap_0.10.pdf}\\
% {(a) $\epsilon=0.1$}\\
% \includegraphics[width=0.9\columnwidth]{coverage_heatmap_0.40.pdf}\\
% {(b) $\epsilon=0.4$}
% \caption{Empirical coverage rate of the groundtruth pose for $8$ different objects in the LineMOD Occlusion (\lmo) dataset with two user-specified error rates: (a) $\epsilon=0.1$ and (b) $\epsilon=0.4$.
%!TEX root = ../main.tex

\begin{table*}
\centering
\begin{adjustbox}{width=0.9\textwidth}
\begin{tabular}{c|cccc|cc|cc|cc|cc}
\hline
 & \multicolumn{4}{c}{Baselines (results adapted from~\cite{peng19cvpr-pvnet})} & \multicolumn{8}{|c}{Conformalized heatmap} \\
\cline{2-13}
  & Tekin & PoseCNN & Oberweger & PVNet & \multicolumn{2}{c|}{\gtball} & \multicolumn{2}{c|}{\gtellipse} & \multicolumn{2}{c|}{\frcnnball} & \multicolumn{2}{c}{\frcnnellipse}\\
 objects& \cite{tekin18cvpr-yolo} & \cite{xiang18rss-posecnn} & \cite{oberweger18eccv-heatmap} & \cite{peng19cvpr-pvnet} & $\epsilon=0.1$ & $\epsilon=0.4$ & $\epsilon=0.1$ & $\epsilon=0.4$ & $\epsilon=0.1$ & $\epsilon=0.4$ & $\epsilon=0.1$ & $\epsilon=0.4$ \\
\hline
ape & $7.01$ & $34.6$ & $69.6$ & $69.14$ & $77.70$ & $79.52$ & $79.26$  & $79.88$  & $70.20$ & $71.01$ & $68.84$ & $69.11$\\
can & $11.20$ & $15.10$ & $82.60$ & $86.09$ & $73.41$ & $75.97$ & $75.81$ & $78.13$ & $67.52$ & $69.81$ & $67.69$ & $69.56$ \\
cat & $3.62$ & $10.40$ & $65.10$ & $65.12$ & $87.36$ & $90.59$ & $89.54$  & $90.11$ & $74.95$ & $80.23$ & $68.98$ & $78.57$ \\
duck & $5.07$ & $31.80$ & $61.40$ & $61.44$ & $82.71$ & $83.08$ & $84.02$  & $83.55$ & $79.30$ & $80.62$ & $80.06$ & $80.53$ \\
driller & $1.40$ & $7.40$ & $73.80$ & $73.06$ & $79.32$ & $82.54$ & $81.22$  & $82.04$  & $58.48$ & $65.92$ & $58.06$ & $65.67$ \\
eggbox & - & $1.90$ & $13.10$ & $8.43$ & $0$ & $0$ & $0.09$ & $0.18$  & $0$ & $0$ & $0$ & $0.14$ \\
glue & $4.70$ & $13.80$ & $54.90$ & $55.37$ & $56.49$ & $71.08$ & $71.69$ & $72.93$ & $30.03$ & $47.18$ & $41.96$ & $48.26$ \\
holepuncher & $8.26$ & $23.10$ & $66.40$ & $69.84$ & $81.65$ & $82.89$ & $83.22$ & $84.30$ & $74.96$ & $77.85$ & $76.28$ & $78.18$ \\
\hline
average & $6.16$ & $17.20$ & $60.90$ & $61.06$ & $67.33$ & $70.71$ & $70.61$ & $71.39$ & $56.93$ & $61.58$ & $57.73$ & $61.25$\\
\hline
\end{tabular}
\end{adjustbox}
\vspace{-3mm}
\caption{Success rates of baseline methods and our conformalized heatmap (using the average pose) based on the 2D projection metric (\ie, a pose estimation is considered successful if the average 2D reprojection error is below $5$ pixels).\label{table:accuracy}}
\vspace{-6mm}
\end{table*}

\section{Experiments}
\label{sec:experiments}

We test our approach on the LineMOD Occlusion (\lmo) dataset~\cite{brachmann14eccv-linemodocc} to (i) justify the exchangeability assumption (Theorem~\ref{thm:icp-validity}) and suggest best practices for applying conformal prediction; (ii) evaluate the empirical coverage of the {\purse} and verify the correctness of Theorem~\ref{thm:icp-validity}, and (iii) compute the worst-case error bounds and demonstrate tightness or looseness. We also (iv) show that the average pose achieves better or similar accuracy as other approaches.

{\bf Implementation and runtime}. We set $T=1000$ in {\ransag}; use OpenGV~\cite{kneip14icra-opengv} for \pthreep and \pnp; and add a redundant $\norm{t} \leq 5$ in~\eqref{eq:purse} to ensure bounded translation. All procedures are implemented in Python except SDP relaxations are implemented in Matlab. The runtime of {\ransag} is comparable to {\ransac} and below one second. The runtime of computing~\eqref{eq:pose2purse} via SDPs is around $8$ seconds on a workstation with 2.2GHz AMD CPUs. The (second-order) SDP relaxations are almost always exact. 
% It is possible to compute a faster (in $0.1$ second) but looser upper bound for~\eqref{eq:pose2purse} by using a first-order SDP relaxation.

{\bf Dataset and exchangeability}. The \lmo dataset contains $1214$ test images capturing $8$ different objects on a table, of which $200$ images were chosen by BOP19'20~\cite{hodan18eccv-bop}. We use the $200$ images for calibration and the entire $1214$ images for testing. As mentioned in Section~\ref{sec:pre:icp}, if the dataset was collected as a single video sequence under natural motion (\eg, a straight line), then the exchangeability assumption would fail. However, \cite{hinterstoisser12accv-linemod} described the data collection:
\begin{quote}
\vspace{-2mm}  
In order to guarantee a well distributed pose space sampling of the dataset pictures, we \emph{uniformly} divided the upper hemisphere of the objects into \emph{equally distant} pieces and took \emph{at most one image per piece}. As a result, our sequences provide \emph{uniformly distributed views} ...
\end{quote}
\vspace{-2mm}  
which indicates the $1214$ images are independent (\cf~\cite[Figs.~5-6]{hinterstoisser12accv-linemod}) and therefore exchangeable. This demonstrates a good example for data collection --to equally divide the parameter space and collect one observation per division-- so the guarantees offered by conformal prediction are valid. 


{\bf Empirical coverage}. Our approach conformalizes the heatmaps~\cite{schmeckpeper22jfr-semantic} as~\eqref{eq:icp-ball} or~\eqref{eq:icp-ellipse}. The implementation\footnote{\url{https://github.com/yufu-wang/6D_Pose}} of~\cite{schmeckpeper22jfr-semantic} uses either groundtruth or Faster RCNN~\cite{ren15neurips-frcnn} bounding boxes, giving four variants of our approach: groundtruth box plus~\eqref{eq:icp-ball} or~\eqref{eq:icp-ellipse} (labels: \gtball, \gtellipse), and Faster RCNN box plus~\eqref{eq:icp-ball} or~\eqref{eq:icp-ellipse} (labels: \frcnnball, \frcnnellipse). Fig.~\ref{fig:coverage-and-bound} left column shows the empirical coverage (\ie, the percentage of images whose groundtruth poses lie in~\eqref{eq:purse}) of all four variants with $\epsilon=0.1$ and $\epsilon=0.4$. We see the empirical coverage is around $90\%$ when $\epsilon=0.1$ and around $60\%$ when $\epsilon=0.4$, for all $8$ objects. Though the empirical coverage can deviate from $1-\epsilon$, it generally stays within $\pm 5\%$ and mostly goes above $1-\epsilon$, which is encouraging given that our calibration set only has size $n=200$. Fig.~\ref{fig:methodoverview} (b) plots examples of the prediction sets. More examples are shown in the Supplementary Material. 


{\bf Worst-case error bounds}. Fig.~\ref{fig:coverage-and-bound} middle column plots the worst-case rotation error bound ($x$-axis) vs. the actual rotation error between the average pose and the groundtruth ($y$-axis) for our approach using the {\gtball} setup (results for {\gtellipse}, {\frcnnball} and {\frcnnellipse} are similar and provided in the Supplementary Material). First, when the {\purse} covers the groundtruth (blue circles), the rotation error bound is always larger than the actual error (\ie, the blue circles never cross the $y=x$ diagonal). Second, when the error rate is increased from $\epsilon=0.1$ to $\epsilon=0.4$, we observe a shift of the blue circles towards $y=x$, indicating the error bounds get tightened. Third, our bounds are reasonably tight for most test images (\ie, the bottom-left cluster of blue circles) especially when $\epsilon=0.4$. However, they can become overly conservative (\ie, the line of blue circles on the right-side boundary) due to the keypoint prediction sets become too large. Fig.~\ref{fig:coverage-and-bound} right column plots similar results for the translation. The Supplementary Material gives a more detailed analysis of this conservatism, wherein we also solve~\eqref{eq:pose2purse} for multiple samples computed by {\ransag}, choose the minimum bound, and compare them with those obtained for the average pose (\cf Remark~\ref{rmk:bestbound}). 

{\bf Accuracy of the average pose}. We compare the accuracy of our average pose with other methods according to the 2D projection metric (an estimation is correct if the mean reprojection error is below $5$ pixels). Table~\ref{table:accuracy} shows: (i) our average pose achieves significantly better success rates when using groundtruth bounding boxes, and similar success rates when using Faster RCNN; (ii) the accuracy of the average pose increases when $\epsilon$ increases. 