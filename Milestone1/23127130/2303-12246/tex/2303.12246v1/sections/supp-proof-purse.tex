%!TEX root = ../main.tex

\section{Proof of Proposition~\ref{prop:purse}}
\begin{proof} Recall the ICP set in~\eqref{eq:icpunify}
\bea\label{eq:icpunifyrestate}
\Feps(x) = \cbrace{\vy \in \calY \mid (y_k - \mu_k)\tran \Lambda_k (y_k - \mu_k) \leq 1,\forall k}
\eea
that defines either a \eqref{eq:icp-ball} or an \eqref{eq:icp-ellipse}. From the pinhole camera projection model, we know that the groundtruth keypoints $\vy = (y_1,\dots,y_K)$ satisfy
\bea\label{eq:projection}
y_k = \Pi (\Rgt Y_k + \tgt) = \frac{[P(\Rgt Y_k + \tgt)]_{1:2}}{[P(\Rgt Y_k + \tgt)]_{3}}, k=1,\dots,K 
\eea
where $P \in \Real{3\times 3}$ denotes the camera intrinsics, $Y_k \in \Real{3}$ is location of the $k$-th 3D keypoint in the object's coordinate frame, $[\vv]_{1:2}$ (resp. $[\vv]_3$) denotes the first two (resp. third) entries of a 3D vector $\vv$. To simplify our notation, we develop~\eqref{eq:projection} as
\bea
P\Rgt Y_k + P \tgt = (Y_k\tran \kron P) \vectorize{\Rgt} + P \tgt 
= \underbrace{\bmat{cc} Y_k\tran \kron P & P \emat}_{:= U_{k} \in \Real{3 \times 12}}
\bmat{c} \vectorize{\Rgt} \\ \tgt \emat = \bmat{c} u_{k,1}\tran \\ u_{k,2}\tran \\ u_{k,3}\tran \emat \sgt \\
\Longrightarrow y_k = \parentheses{\bmat{c} u_{k,1}\tran \\ u_{k,2}\tran \emat \sgt} \bigg/ (u_{k,3}\tran \sgt), \label{eq:simpleyk}
\eea
where $u_{k,j}\tran \in \Real{1 \times 12}$ denotes the $j$-th row of matrix $U_k$. Notice that $u_{k,3}\tran \sgt$ is the depth of the $k$-th 3D keypoint in the camera coordinate frame (after rigid transformation $(\Rgt,\tgt)$).
% Similarly, we write the transformed $k$-th 3D keypoint as
% \bea
% \Rgt Y_k + \tgt = (Y_k \kron \eye_3) \vectorize{\Rgt} + \tgt = \underbrace{\bmat{cc} Y_k\tran \kron \eye_3 & \eye_3 \emat}_{:=B_k \in \Real{3 \times 12}} \bmat{c} \vectorize{\Rgt} \\ \tgt \emat = \bmat{c} b_{k,1}\tran \\ b_{k,2}\tran \\ b_{k,3}\tran \emat \sgt.
% \eea

{\bf In front of the camera}. Since the camera observes the object, the groundtruth pose $\sgt$ must transform the object to lie in front of the camera. Therefore, the keypoints must have positive depth values:
\bea
u_{k,3}\tran \sgt > 0, k=1,\dots,K.
\eea

{\bf Within ICP sets}. We now insert~\eqref{eq:simpleyk} back to the constraint defined by the ICP set~\eqref{eq:icpunifyrestate}, leading to
\bea
(y_k - \mu_k)\tran \Lambda_k (y_k - \mu_k) \leq 1 \Longleftrightarrow \\ 
\frac{1}{(u_{k,3}\tran \sgt)^2} \sgt\tran \bmat{cc} u_{k,1} - \mu_{k,1} u_{k,3} & u_{k,2} - \mu_{k,2} u_{k,3} \emat \Lambda_k  \bmat{c} u_{k,1}\tran - \mu_{k,1} u_{k,3}\tran \\ u_{k,2}\tran - \mu_{k,2} u_{k,3} \emat \sgt \leq 1 \Longleftrightarrow \\
\sgt\tran \bmat{cc} u_{k,1} - \mu_{k,1} u_{k,3} & u_{k,2} - \mu_{k,2} u_{k,3} \emat \Lambda_k  \bmat{c} u_{k,1}\tran - \mu_{k,1} u_{k,3}\tran \\ u_{k,2}\tran - \mu_{k,2} u_{k,3} \emat \sgt \leq \sgt\tran \parentheses{u_{k,3} u_{k,3}\tran} \sgt \Longleftrightarrow \\
\sgt\tran \underbrace{\parentheses{ \bmat{cc} u_{k,1} - \mu_{k,1} u_{k,3} & u_{k,2} - \mu_{k,2} u_{k,3} \emat \Lambda_k  \bmat{c} u_{k,1}\tran - \mu_{k,1} u_{k,3}\tran \\ u_{k,2}\tran - \mu_{k,2} u_{k,3} \emat - u_{k,3} u_{k,3}\tran }}_{: = A_k \in \sym{12}} \sgt \leq 0, \label{eq:Ak}
\eea
which indicates that the groundtruth pose $\sgt$ must satisfy $K$ quadratic constraints, one for each keypoint.
In summary, the groundtruth pose $\sgt$ must lie in the \eqref{eq:purse} with $b_k = u_{k,3}$ and $A_k$ as in~\eqref{eq:Ak}.
\end{proof}