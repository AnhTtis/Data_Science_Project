%!TEX root = ../main.tex

\section{Proof of Proposition~\ref{prop:invariance}}

\begin{proof} Let $\{ \alpha_{i}  \}_{i=1}^n$ be the calibration scores obtained by applying $r$ in \eqref{eq:maxnonconformity} to the calibration set, and $\{ \alpha^{\rho}_i \}_{i=1}^{n}$ be the scores obtained by applying $r_{\rho}$ in \eqref{eq:robustnonconformity}. Observe that $\alpha^{\rho}_i = \rho(\alpha_i)$ because $\rho$ being monotonically increasing implies $\max \circ \rho = \rho \circ \max$ (``$\circ$'' describes function composition). As a result, it follows that $\alpha^{\rho}_{\pi(\floor{(n+1)\epsilon})} = \rho(\alpha_{\pi(\floor{(n+1)\epsilon})})$.
Let $\Feps_{\rho}$ be the ICP set due to $r_{\rho}$ for a given $\epsilon$, we have
\bea
\Feps_{\rho} = \{ \vy \in \calY \mid  \max\{ \rho(\phi(y_k,f(x)_k)) \}_{k=1}^K \leq \alpha^{\rho}_{\pi(\floor{(n+1)\epsilon})}  \} \nonumber \\
= \{ \vy \in \calY \mid  \rho( \max\{ \phi(y_k,f(x)_k) \}_{k=1}^K ) \leq \rho(\alpha_{\pi(\floor{(n+1)\epsilon})} ) \} \nonumber\\
= \{ \vy \in \calY \mid \max\{ \phi(y_k,f(x)_k) \}_{k=1}^K \leq \alpha_{\pi(\floor{(n+1)\epsilon})} \}, \nonumber 
\eea
where the last set is precisely $\Feps$, the ICP set induced by $r$.
\end{proof}




