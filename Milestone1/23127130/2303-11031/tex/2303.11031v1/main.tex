\documentclass{aa}  
\usepackage{graphicx}
\usepackage{txfonts}
\usepackage[T1]{fontenc}
\usepackage{graphicx}
\usepackage{multirow}
\usepackage{blindtext}
\usepackage{amsmath}
\usepackage[usenames, dvipsnames]{color}
\usepackage{booktabs}
\usepackage{scrextend}
\usepackage{mathtools}
\usepackage{amsmath}
\usepackage[shortlabels]{enumitem}
\usepackage{lipsum}
\usepackage[normalem]{ulem}
\usepackage{newtxtext,newtxmath}
\usepackage{diagbox}
\newcommand{\Hb}{H$\beta$\xspace}
\newcommand{\Ha}{H$\alpha$\xspace}
\newcommand{\FHa}{$ F_{\rm H\alpha}$\xspace}
\newcommand{\FOII}{$ F_{\rm [O\sc{II}]}$\xspace}
\newcommand{\Hd}{H$\delta$\xspace}
\newcommand{\LHa}{$ L_{\rm H\alpha}$\xspace}
\newcommand{\LHacont}{$L_{\rm{H\alpha}}^{\rm{cont}}$\xspace}
\newcommand{\LOII}{$L_{\rm{\left[\mathrm{O\,\textrm{\sc{ii}}}\right]}}$\xspace}
\newcommand{\LOIII}{$L_{\rm{\left[\mathrm{O\,\textrm{\sc{iii}}}\right]}}$\xspace}
\newcommand{\OI}{$\left[\mathrm{O\,\textrm{\sc{i}}}\right]$\xspace}
\newcommand{\SII}{$\left[\mathrm{S\,\textrm{\sc{ii}}}\right]$\xspace}
\newcommand{\OIII}{$\left[\mathrm{O\,\textrm{\sc{iii}}}\right]$\xspace}
\newcommand{\OII}{$\left[\mathrm{O\,\textrm{\sc{ii}}}\right]$\xspace}
\newcommand{\NII}{$\left[\mathrm{N\,\textrm{\sc{ii}}}\right]$\xspace}
\newcommand{\logaLHa}{log($L_{\rm{H\alpha}}/\rm{erg\,s^{-1}}$)\xspace}
\newcommand{\logaLOII}{log(L${\left[\mathrm{O\,\textrm{\textsc{ii}}}\right]/\rm{erg\,s^{-1}}}$)\xspace}
\newcommand{\fluxunit}{\,s$^{-1}$cm$^{-2}$erg}
\newcommand{\lumunit}{\,s$^{-1}$erg}

\newcommand{\MD}{\textsc{MultiDark-Galaxies}\xspace}
\newcommand{\GE}{\textsc{GET\_\,EMLINES}\xspace}
\newcommand{\parent}{{\it Parent-ELG}\xspace}
\newcommand{\mainsel}{{\it Main-ELG}\xspace}

% Comments
\newcommand{\vgp}[1]{ \textcolor{green}{(VGP: {#1})} }
\newcommand{\ya}[2][]{ \textcolor{PineGreen}{ \emph{#1} (YA: {#2})} }
\newcommand{\pablo}[2][]{ \textcolor{magenta}{ \emph{#1} (P: {#2})} }
\newcommand{\gi}[2][]{ \textcolor{red}{ \emph{#1} (G: {#2})} }
\DeclarePairedDelimiter\abs{\lvert}{\rvert}%
\DeclarePairedDelimiter\norm{\lVert}{\rVert}%
\definecolor{mygray}{gray}{0.5}

\begin{document} 
   \title{Characterizing the ELG luminosity functions in the nearby Universe}
   \titlerunning{\em ELG LFs at $z\sim0.1$}
   \author{G. Favole,\thanks{E-mail: gfavole@iac.es}
          \inst{1,2}
          V. Gonzalez-Perez,\inst{3,4} Y. Ascasibar,\inst{3,4}
 P. Corcho-Caballero,\inst{3,5} A. D. Montero-Dorta,\inst{6} A. J. Benson,\inst{7} J. Comparat,\inst{8} S. A. Cora,\inst{9,10} D. Croton\inst{11} H. Guo,\inst{12} D. Izquierdo-Villalba,\inst{13,14} A. Knebe,\inst{3,4,15} \'A. Orsi,\inst{16} \\
 D. Stoppacher,\inst{3,17,18} C. A. Vega-Mart\'inez\,\inst{19,20}
          } %A. R. H. Stevens,\inst{17} 
          \authorrunning{\em Favole et al. 2022}

\institute{Instituto de Astrof\'{\i}sica de Canarias, s/n, E-38205, La Laguna, Tenerife, Spain
\and
Departamento de Astrof\'{\i}sica, Universidad de La Laguna, E-38206, La Laguna, Tenerife, Spain
\and
%Institute of Physics, Laboratory of Astrophysics, Ecole Polytechnique F\'ed\'erale de Lausanne, Obs. de Sauverny, 1290 Versoix, Switzerland
%\and
Departamento de F\'isica Te\'orica, Facultad de Ciencias, Universidad Aut\'onoma de Madrid, E-28049, Spain
\and
Centro de Investigaci\'on Avanzada en F\'isica Fundamental, Facultad de Ciencias, Universidad Aut\'onoma de Madrid, E-28049 Madrid, Spain
\and
Australian Astronomical Optics, Macquarie University, 105 Delhi Rd, North Ryde, NSW 2113, Australia
\and
Departamento de F\'isica, Universidad T\'ecnica Federico Santa Mar\'ia, Casilla 110-V, Avda. Espa\~na 1680, Valpara\'iso, Chile
\and
Carnegie Observatories, 813 Santa Barbara Street, Pasadena, CA 91101, USA
\and
Max-Planck-Institut f\"{u}r extraterrestrische Physik (MPE), Giessenbachstrasse 1, D-85748 Garching bei M\"{u}nchen, Germany
\and
Instituto de Astrof\'isica de La Plata (CCT La Plata, CONICET, UNLP), Paseo del Bosque s/n, B1900FWA, La Plata, Argentina
\and
Facultad de Ciencias Astron\'omicas y Geof\'isicas, UNLP, Paseo del Bosque s/n, B1900FWA, La Plata, Argentina
\and
Centre for Astrophysics \& Supercomputing, Swinburne University of Technology, P.O. Box 218, Hawthorn, Victoria 3122, Australia
\and
Key Laboratory for Research in Galaxies and Cosmology, Shanghai Astronomical Observatory, Shanghai 200030, China
\and
Dipartimento di Fisica ``G. Occhialini'', Universit\`{a} degli Studi di Milano-Bicocca, Piazza della Scienza 3, I-20126 Milano, Italy
\and
INFN, Sezione di Milano-Bicocca, Piazza della Scienza 3, 20126 Milano, Italy
\and
International Centre for Radio Astronomy Research, The University of Western Australia, Crawley, WA 6009, Australia
\and
PlantTech Research Institute Limited. South British House, 4th Floor, 35 Grey Street, Tauranga 3110, New Zealand
\and
Instituto de Astrof\'isica, Pontificia Universidad Cat\'olica de Chile, Campus San Joaqu\'in, Avda. Vicu\~na Mackenna 4860, Santiago, Chile
\and
Facultad de F\'isicas, Universidad de Sevilla, Avda.\ Reina Mercedes s/n, Campus de Reina Mercedes, 41012 Sevilla, Spain
\and
Instituto de Investigaci\'on Multidisciplinar en Ciencia y Tecnolog\'ia, Universidad de La Serena, Ra\'ul Bitr\'an 1305, La Serena, Chile
\and
Departamento de Astronom\'ia, Universidad de La Serena, Av. Juan Cisternas 1200 Norte, La Serena, Chile
             }

   \date{Received xxx, 2022; accepted xxx}

 
  \abstract
  % context heading (optional)
  % {} leave it empty if necessary  
   {Nebular emission lines are powerful diagnostics for the physical processes at play in galaxy formation and evolution. Moreover, emission-line galaxies (ELGs) are one of the main targets of current and forthcoming spectroscopic cosmological surveys.}
  % aims heading (mandatory)
   {We investigate the contributions to the line luminosity functions (LFs) of different galaxy populations in the local Universe, providing a benchmark for future surveys of earlier cosmic epochs.}
  % methods heading (mandatory)
   {The large statistics of the observations from the SDSS DR7 Main galaxy sample and the MPA-JHU spectral catalogue enables us to precisely measure the \Ha, \Hb, \OII, \OIII, \NII, and \SII emission-line LFs over $\sim2.4$ Gyrs in the low-$z$ Universe, $0.02<z<0.22$. We present a generalised $1/V_{\rm max}$ LF estimator capable of simultaneously correcting for spectroscopic, $r-$band magnitude, and emission-line incompleteness. We study the contribution to the LF of different types of ELGs classified using two methods: (i) the value of the specific star formation rate (sSFR), and (ii) the line ratios on the Baldwin--Phillips--Terlevich (BPT) and the WHAN (i.e. \Ha equivalent width, EW$_{\rm H\alpha}$, versus the \NII/\Ha line ratio) diagrams.}
  % results heading (mandatory)
   {
   The ELGs in our sample are mostly star forming, with $83.6$ per cent having ${\rm sSFR}>10^{-11}{\rm yr^{-1}}$.
   When classifying ELGs using the BPT+WHAN diagrams, we find that $63$ per cent are star forming, $1.5$ are passively evolving, and $2.9$ have nuclear activity (Seyfert). The rest are LINERs and Composite ELGs.
   We find that a Saunders function is the most appropriate to describe all the emission-line LFs.
   They are dominated by star-forming regions, except for the bright end of the \OIII and \NII LFs (i.e. $L_{[\rm NII]}>10^{42}{\rm s^{-1}erg}$, $L_{[\rm OIII]}>10^{43}{\rm s^{-1}erg}$), where the contribution of Seyfert galaxies is not negligible.
   Besides the star-forming population, Composite galaxies and LINERs are the ones that contribute the most to the ELG numbers at $L < 10^{41}\,{\rm s^{-1}erg}$.
   All our results, including data points and analytical fits, are publicly available.
   }
   {}
   \keywords{Galaxies: luminosity function, distances and redshifts, star formation, stellar content, starburst, statistics, Seyfert %distances and redshifts --
                %galaxies: haloes -- galaxies: statistics --
                %cosmology: observations -- cosmology: theory -- large-scale structure of Universe
               }

   \maketitle
%
%-------------------------------------------------------------------

\section{Introduction}
Current and upcoming spectroscopic cosmological surveys, such as DESI\,\citep{desi} and Euclid\,\citep{euclid} rely on galaxies with strong spectral emission, or emission-line galaxies (ELGs), to build accurate and deep 3-D cosmic maps and infer the cosmological composition and evolution of the Universe. 
According to the origin of their spectral lines, different types of ELGs might trace different regions of the cosmic web, or might be the result of a different evolution: for instance, quasars (QSOs) are more strongly clustered than star-forming (SF) galaxies\,\citep{Zhao2021}.
Until now, the statistical errors of cosmological surveys have been larger than the uncertainties due to our lack of understanding of the galaxy formation and evolution processes~\citep{Avila2020,Raichoor2021}. However, this might change with the new generation of Stage-IV cosmological surveys, such as DESI \citep{desi}, Euclid \citep{euclid}, 4MOST \citep{dejong2012}, PSF\,\citep{psf} or SphereX \citep{spherex}.


ELGs are also interesting as they enable us to reconstruct the cosmic star formation history (SFH) out to $z\sim 2$ \citep[e.g.][]{1998ARA&A..36..189K, 1998ApJ...498..106M, 2002A&A...387..396A, 2002AJ....124.3135K, Kewley2004, 2003ApJ...599..971H, 2007ApJ...666..870C, 2010ApJ...714.1256C, 2006ApJ...642..775M,2007ApJS..173..267S, 2007ApJ...671..333K, 2009ApJ...703.1672K, 2009ApJ...692..556R, 2010ApJ...719.1191T}. 
The study of star formation from emission lines has been possible thanks to an immense observational effort over the course of the last decades.
In the past, high-sensitivity infra-red (IR) space telescopes, such as Spitzer\footnote{\url{http://irsa.ipac.caltech.edu/data/SPITZER/docs/}} or Herschel\footnote{\url{http://sci.esa.int/herschel/}}, enabled the calibration of monochromatic star formation rate (SFR) indicators in nearby galaxies \citep[]{2013seg..book.....F, 2013seg..book..419C}, complementing the efforts in the UV and optical channels to map the SFR evolution of galaxies out to $z\sim9$ \citep[]{2004ApJ...600L.103G, 2009ApJ...705..936B, 2010ApJ...709L.133B}. 



While star-forming regions constitute the main origin of the spectral emission lines for ELGs \citep[e.g.][]{Kennicut1992, Sobral2013, Pirzkal2018, Xiao2018, Kewley2019}, other origins are also possible, such as active galactic nuclei\,\citep[AGN; e.g.][]{Marziani2017,Lin2022}, shocks~\citep[e.g.][]{hirschmann2022} and old stellar populations\,\citep[e.g.][]{Kennicut1992,Sansom2015,Byler2019,Nersesian2019,Clarke2021}.
Emission-line diagnostic ratios, such as the BPT diagram \citep{Baldwin1981} or the $\rm D_n(4000)$ break index \citep{Bruzual1983, Balogh1999}, have been used to separate SF ELGs from AGN, as well as older and younger stellar population contributions \citep[e.g.][]{Kewley2001, Kewley2006, Kauffmann2003, Kauffman2003D4000, Gallazzi2005, Belfiore2016, Wu2018, Angthopo2020}.
The WHAN diagram, relating the equivalent width of the \Ha line and the \NII/\Ha ratio \citep[e.g.][]{Stasi2006, Cid2011} provides additional information to discriminate between SF and active galaxies, and the relation between EW$_{\rm H\alpha}$ and the D4000 index (the so-called ageing diagram) has been proposed to identify sudden changes in the recent star formation activity \citep{Casado15, CorchoCaballero20, CorchoCaballero21b, CorchoCaballero23}.



Over the years, several studies have combined ELG observations both from spectroscopic and imaging surveys in order to constrain the emission-line luminosity functions.
The \Ha \citep{1995ApJ...455L...1G, Tresse2002} and the \OII \citep{Gallego2002} LFs were among the first ones to be characterised in the local Universe. 
\citet[]{Fujita2003} at $z=0.24$ and \citet{Ly2007} at $0.07<z<1.47$ used broad-band galaxy colours to discriminate \Ha from other lines, finding that the \Ha LF evolution is stronger in the faint end than in the bright one.
\citet[]{Gilbank2010} explored  the \OII, \Ha and $u$-band luminosities as SFR indicators at $z<0.2$, finding that, in the high-mass end (i.e. $\rm M_\star>10^{10}\,M_\odot$), \OII needs a larger correction to compensate for the effects of metallicity dependence and dust extinction. 
\citet[]{2013MNRAS.433.2764G} studied the \Ha LF and SFR density at $z<0.35$, observing an increasing number of SF galaxies in the faint end.
\citet[]{Sobral2013} studied the SFH and \Ha LF evolution at $0.40<z<2.2$, finding that the \Ha line traces the bulk of star formation over the last 11\,Gyr.
In this period, the SF activity has produced $\sim95$ per cent of the total stellar mass density observed locally, half of which was assembled within 2\,Gyr between $1.2< z< 2.2$.
\citet{Mehta2015} studied the bivariate \Ha-\OIII LF at $z\sim1$ using galaxies from the WFC3 Infrared Spectroscopic Parallel \citep[WISP;][]{Atek2010} Survey, They showed that the \Ha LF can be determined by exclusively fitting \OIII data.
\citet{Zhu2009} and \citet{Comparat2015} respectively studied the \OII LF evolution at $0.75<z<1.45$ and $0.1 <z< 1.65$. \citet{Comparat2016} measured the \OII, \OIII and, for the first time, the \Hb LFs over the last nine billions years. They found that both the characteristic luminosity and the density of all LFs increase with redshift.
\cite{Saito2020} used photometric data to model galaxy spectral energy distributions (SED) and emission-line fluxes and use them to derive accurate predictions for the \Ha and \OII LF up to $z=2.5$. 

All the studies above show that, so far, the focus has been mainly on \Ha, \OII and \OIII lines. Here we propose a novel analysis aimed at exploring also other lines, namely \Hb, \NII and \SII. We want to split the different galaxy contributions to the ELG production to understand the impact of each one on the line LF. This work will be directly relevant to future high-redshift studies \citep[see e.g.][]{Violeta2020, Zhai2019}. 
In particular, the aim of our work is twofold: (i) to measure the \Ha, \Hb, \OII, \OIII, \NII, and \SII luminosity functions in the nearby Universe with high accuracy, using a uniform procedure to select our galaxy sample and account for statistical incompleteness; (ii) to establish the contribution of different ELG types to the total LF.


For this study we use a subsample of the SDSS DR7 Main galaxy sample \citep{Strauss2002} at $0.02<z<0.22$, with spectral properties from the MPA-JHU\footnote{\url{https://www.sdss.org/dr12/spectro/galaxy_mpajhu/}\label{mpanote}} release. We classify the selected ELGs based on their specific star formation rate (sSFR, star formation rate divided by the stellar mass), and their position in the BPT and WHAN diagrams.
These diagnostics allow us to classify galaxies beyond the star-forming and passive split, to distinguish composite galaxies from those with spectral emission lines produced in jets or shocks, which, in many cases, host active galactic nuclei, i.e. Seyfert galaxies.



The paper is organised as follows. In Sec.\,\ref{sec:data} we describe the SDSS Main galaxy sample, its MPA-JHU spectral properties, the sample selections performed, and their incompleteness effects. In Sec.\,\ref{sec:LFcalculation} we present a generalised $1/V_{\rm max}$ LF estimator capable of simultaneously correcting from spectroscopic, $r-$band magnitude, and emission-line incompleteness.
In Sec.\,\ref{sec:classification} we explain the methods adopted to classify 
ELGs. In Sec.\,\ref{sec:results}, we present the measured LFs and our findings are summarised in Sec.\,\ref{sec:disc}.

Throughout the paper we adopt the MultiDark Planck 2 cosmology consistent with \cite{Planck2016}. Our parameters are: $\Omega_{\rm m} = 0.3071$,  $\Omega_{\rm b} = 0.0482$, $\Omega_\Lambda = 0.6928$, $h=0.6777$, $\sigma_8 = 0.8228$ and $n_s = 0.96$.


\begin{figure}
\centering 
   \includegraphics[width=0.95\linewidth]{MainELGHamegacortelineflux6lines.png}\hfill
\caption{\parent signal-to-noise as a function of the \Ha line flux, color-coded by sSFR. The black-dashed lines in the main panel represent the flux and S/N cuts we impose on the \parent sample to obtain our \mainsel sample: $F>2\times 10^{-16}{\rm cm^{-2}s^{-1}erg}$ and ${\rm S/N>2}$ (see Sec.\,\ref{sec:selections}). The top and right panels show the flux and S/N histograms of the \parent (grey-dashed lines) and the \mainsel (green-solid lines) samples. The marginal distributions displayed both here and in Fig.\,\ref{fig:megacortealllines} for the other lines motivate the flux and S/N cuts chosen to select a complete ELG sample.}
  \label{fig:megacorte}
\end{figure}

%%%%%%%%%%%%%%%%%%%%%%%%%%%%%%%%%%
%2.  SDSS MAIN PARENT ELG SAMPLE
%%%%%%%%%%%%%%%%%%%%%%%%%%%%%%%%%%
\section{Observational data}
\label{sec:data}

In this work we aim at characterising the luminosity functions for a range of spectral emission lines in the local Universe. In particular, we study the following lines: \Ha${\lambda\,6563}$\,\AA,  \Hb${\lambda\,4861}$\,\AA, \OII${\lambda\,3727,3729}$\,\AA, \OIII${\lambda\,5007}$\,\AA, \NII${\lambda\,6584}$\,\AA, \SII${\lambda\,6717,6731}$\,\AA. Here we describe how we generate a sample of ELGs with adequate fluxes and signal-to-noise ratios (S/N) to then study their completeness and measure their LFs.

%%%%%%%%%%%%%%%%%%%%%%%%%%%%%%%%%
% 2.1 \parent selection
%%%%%%%%%%%%%%%%%%%%%%%%%%%%%%%%%
\subsection{The Parent-ELG sample}
\label{sec:parentELGsample}

We select galaxies with good spectra, (i.e. with \texttt{ZWARNING=0}) from the SDSS DR7 Main sample \citep[]{Strauss2002} using the NYU-Value Added Galaxy Catalogue\footnote{\url{http://cosmo.nyu.edu/blanton/vagc/}} \citep[]{2005AJ....129.2562B}. We spectroscopically match these galaxies to the MPA-JHU DR7\footref{mpanote} spectral release to obtain further properties, such as star formation rates, stellar masses, spectral emission-line fluxes and equivalent widths \citep{Brinchmann2004, Tremonti2004}. 

The SDSS Main galaxy sample covers an effective area of 7300\,deg$^2$ and is limited in $r-$band petrosian magnitude at $r_{ \rm p}<17.77$. The SDSS spectra span wavelengths of 3800--9200\,\AA, with a resolution that varies from $R=1500$ at $\lambda=3800\,$\AA, to $R=2500$ at $\lambda=9000\,$\AA\, \citep{Stoughton2002}.
We limit our sample to the redshift range $0.02<z<0.22$. The lower redshift cut ensures that we are studying galaxies beyond the local group, reducing the cosmic variance in our sample. The upper limit is chosen to mimic the SDSS Main selection in \cite{Favole2017} and \cite{Guo2015}, minimising the effect of k-corrections and cosmic evolution.
This matched sample, hereafter `\parent', is composed of 426625 galaxies.

We calculate the luminosities of the \parent sample from the observed (i.e. dust attenuated) fluxes $F$ provided in the MPA-JHU catalogue in units of erg~s$^{-1}$~cm$^{-2}$. These were measured by fitting the spectra using \cite{Bruzual2003} stellar population synthesis models, accounting for stellar absorption. Note that, in the case of the \OII${\lambda\,3727,3729}$\,\AA\, and \SII${\lambda\,6717,6731}$\,\AA\, doublets, the flux is the sum of the individual line fluxes.  

The luminosities are obtained as \citep[see e.g.][]{2003ApJ...599..971H, Favole2017}:
\begin{equation}
L[{\rm s^{-1}erg}]=4\pi D_{\rm L}^2(z) 10^{-0.4(m_{\rm p}-m_{\rm{fib}})}F,
\label{eq:lum}
\end{equation}
where $D_{\rm L}(z)$ is the luminosity distance as a function of redshift and cosmology. 
The exponent $(m_{\rm p}-m_{\rm{fib}})$ represents the fiber aperture correction given as a function of the SDSS petrosian ($m_{\rm p}$) and fiber ($m_{\rm fib}$) magnitudes. The fiber aperture correction takes into account that only the portion of the flux within the fiber, that has a diameter of $\sim 3^{\prime\prime}$, is detected by the spectrograph \citep{Strauss2002}.
Thus, we implicitly assume that the emission measured through the fiber is characteristic of the whole galaxy and that the line equivalent width does not significantly vary across the galaxy.
In the SDSS $ugriz$ photometric system \citep[]{Gunn1998, 1996AJ....111.1748F}, the \Ha, \NII and \SII lines fall in the $r-$band filter, the \OII doublet in the $u-$band, while the \OIII and \Hb fall in the $g-$band. We therefore adopt the corresponding Petrosian ($r_{\rm p}$, $u_{\rm p}$, $g_{\rm p}$) and fiber ($r_{\rm{fib}}$, $u_{\rm{fib}}$, $g_{\rm{fib}}$) magnitudes to implement the fiber aperture correction for each line.


%%%%%%%%%%%%%%%%%%%%%%%%%%%%%%%%%
% 2.2 \main selection
%%%%%%%%%%%%%%%%%%%%%%%%%%%%%%%%%
\subsection{The Main-ELG selection}
\label{sec:selections}

We aim at selecting a complete population of bright ELGs with well measured fluxes in all of the following six emission lines: \Ha${\lambda\,6563}$\,\AA, \Hb${\lambda\,4861}$\,\AA, \OII${\lambda\,3727,3729}$\,\AA, \OIII${\lambda\,5007}$\,\AA, \NII${\lambda\,6584}$\,\AA, and \SII${\lambda\,6717,6731}$\,\AA. To achieve this, we extract a sub-sample of the \parent sample above, and then we impose a combination of cuts in emission-line flux and signal-to-noise (S/N) in all the six lines of interest. 
We define the signal-to-noise as the ratio between the observed flux and its error, $\sigma_F$, as given by the MPA-JHU DR7 catalogues: ${\rm S/N}=F/\sigma_F$.

\begin{figure*}
\centering 
   \includegraphics[width=0.45\linewidth]{MainELGHacharlineflux6lines.png}\hfill
   \includegraphics[width=0.45\linewidth]{MainELGLHamasssSFR.png}
    \caption{{\em Left:} \mainsel sSFR as a function of stellar mass, color-coded by \Ha EW. We overplot as black contours the distribution of the star-forming (SF) ELG component selected with ${\rm sSFR}>10^{-11}{\rm yr^{-1}}$, weighted by EW. The dashed blue and dot-dashed red lines are the \citet{2007ApJS..173..315S} linear fits to the SDSS SF and passive galaxy populations, respectively. The side histograms show the sSFR and M$_\star$ marginal distributions of the \mainsel (green) and SF populations (black lines), and compare them to the \parent sample (grey), which has no cuts. {\em Right:} \Ha luminosity as a function of stellar mass, color-coded by sSFR, and corresponding marginal distributions, with the same colours as in the left panel. The black contours here are weighted by the sSFR.}
  \label{fig:charplot}
\end{figure*}

We cut the \parent sample at $F>2\times10^{-16}{\rm cm^{-2}s^{-1}erg}$ and ${\rm S/N}>2$ in all the six lines above. Furthermore, we remove any spurious object with non-physical flux uncertainty by limiting our selection at $\sigma_{F}<10^{-12}\,\rm s^{-1}cm^{-2}erg$, and $\rm EW\geq0\,$\AA\, in all the six lines under study. 
The resulting ELG sample, hereafter `\mainsel', is composed of 174572 emitters (about 40 per cent of the parent sample).
The characteristics of this sample are discussed in Sec.\,\ref{sec:characteristics}.

Fig.\,\ref{fig:megacorte} shows the effect of the \mainsel selection on the signal-to-noise -- $F$ plane for the \Ha line, color-coded by specific star formation rate (sSFR, star formation rate divided by stellar mass); the effect on the other emission lines, color-coded by both sSFR and EW, is shown in Fig.\,\ref{fig:megacortealllines}.
In all cases, the marginal probability distributions of the measured flux and SN are observed to decay below our adopted thresholds, suggesting that completeness would be very difficult to guarantee beyond that point.


%%%%%%%%%%%%%%%%%%%%%%%%%%%%%%%%%
% 2.3 \main properties
%%%%%%%%%%%%%%%%%%%%%%%%%%%%%%%%%

\subsection{\mainsel properties}
\label{sec:characteristics}

Here we analyse the impact of the S/N and emission-line flux cuts performed in Sec.\,\ref{sec:selections} on the sSFR, stellar mass, and EW distributions of the \mainsel sample.

Fig.\,\ref{fig:charplot}, left panel, shows the \mainsel sSFR as a function of stellar mass, color-coded by the \Ha EW. Fig.\,\ref{fig:charplotallines} shows similar plots for the rest of lines under study. 
We overplot the star-forming population selected with ${\rm sSFR}>10^{-11}{\rm yr^{-1}}$ as contours. ELGs with a high sSFR are also those with higher $\rm EW$.
Our results are in good agreement with \citet[]{2007ApJS..173..315S} fits to the SDSS SF and passive populations. On each side of the figure we display the marginal sSFR and $\rm M_\star$ distributions for the SF and \mainsel samples, and we compare them with the \parent sample (grey).
The \mainsel sample includes galaxies with relatively low values of sSFR, that will not be considered as star-forming neither in terms of their sSFR nor in relation with the so-called star formation main sequence.
We will quantify the numbers of these populations below.

The right panel in Fig.\,\ref{fig:charplot} displays the \Ha luminosity as a function of stellar mass, color-coded by sSFR. Fig.\,\ref{fig:lummassallines} shows similar plots for the rest of lines under study. Here we notice that \Ha ELGs with lower star-formation activity (i.e. ${\rm sSFR}\lesssim10^{-11}{\rm yr^{-1}}$) are also the most massive and least luminous ones, whereas SF ELGs with ${\rm sSFR}\gtrsim10^{-11}{\rm yr^{-1}}$ tend to concentrate towards the low-mass and high-luminosity end of the distribution.

These results highlight that ELGs selected with a combination of cuts in signal-to-noise and line flux, i.e. the \mainsel sample, are not equivalent to ELGs selected by using a sharp cut in sSFR or, similarly, in EW. This agrees with theoretical studies that have shown that the small-scale clustering is different for samples selected either based on SFR or emission line fluxes \citep{Violeta2020}.
In fact, the selection based on flux and S/N returns a heterogeneous population of galaxies, covering a similar range in both sSFR and stellar mass as the \parent sample.
This guarantees that the number density of galaxies, in particular the fainter ones, is preserved, maximising the completeness of the luminosity function.

%%%%%%%%%%%%%%%%%%%%%%%%%%%%%%%%%
% 2.4 incompleteness
%%%%%%%%%%%%%%%%%%%%%%%%%%%%%%%%%

\subsection{Incompleteness effects and redshift evolution}
\label{sec:incompleteness}
\begin{figure*}
\centering 
   \includegraphics[width=0.9\linewidth]{LHaMr.png}\vspace{-0.3cm}
  \caption{\mainsel \Ha luminosity as a function of the $r-$band absolute magnitude, color-coded by redshift. We compare the results of the entire sample (left column), with the lower-$z$ (central) and the higher-$z$ (right) bins. On the background we show in grey the \parent sample distribution, which has no cuts in flux nor S/N. From left to right we show the full sample ($0.02<z<0.22$), the lower-$z$ bin ($0.02<z<0.12$), and the higher-$z$ ($0.12<z<0.22$) one. The horizontal lines represent our lower completeness limits in luminosity.}
  \label{fig:LHaMr}
  \end{figure*}
Fig.\,\ref{fig:LHaMr} displays the \Ha luminosity of the\,\mainsel sample as a function of the $r-$band absolute magnitude, $M_r$, color-coded by redshift. We compare this distribution to that of the \parent sample, selected at $r_{\rm p}>17.77$. To better understand its evolution, we analyse the result in three redshift bins: the full sample at $0.02<z<0.22$, the lower-$z$ bin at $0.02<z<0.12$, and the higher-$z$ one at $0.12<z<0.22$. Fig.\,\ref{fig:LMralllines} shows similar plots for the other lines under study. 

We find that the \Ha flux cut is not independent of $M_r$ and hence from the limit $r_{\rm p}<17.77$ intrinsic to the \parent sample. A similar result is found for the other lines. The impact of such dependency is stronger as the redshift increases. In other words, when we cut in flux or S/N, we are also removing a fraction of galaxies below a certain line luminosity that varies in a non-trivial way with redshift.


Our modified $1/V_{\rm max}$ method for ELGs (see Sec.\,\ref{sec:LFcalculation}) is capable of individually correcting from flux-limited selection effects, but not from statistical correlations between the line luminosities and broadband magnitudes.
We therefore set a lower completeness limit for all emission-line luminosities in order to ensure that these correlations do not significantly affect the LF measurement. 
We set this threshold to $L=10^{39.8}\rm s^{-1}erg$ for all the emission lines in the full sample.
This value is chosen by eye, based on the completeness that the \mainsel sample shows in Figs.\,\ref{fig:LHaMr} and \ref{fig:LMralllines}.
Specifically, we set as threshold the luminosity value where the density of ELGs in these figures starts to degrade, indicated as a dotted line in Figs.\,\ref{fig:LHaMr} and \ref{fig:LMralllines}.


%%%%%%%%%%%%%%%%%%%%%%%%%%%%%%%%%
% 3 Vmax
%%%%%%%%%%%%%%%%%%%%%%%%%%%%%%%%%

\section{Volume correction}
\label{sec:LFcalculation}
\begin{figure*}
\centering 
   \includegraphics[width=0.45\linewidth]{schemeVmaxmag.png}\hfill%\vspace{0.4cm}
   \includegraphics[width=0.45\linewidth]{schemeVmaxline.png}%\vspace{-0.5cm}
  \caption{$V_{\rm max}$ computation scheme for a galaxy from the \mainsel sample (star symbol), taking into account the survey $r-$band magnitude limit $r_{\rm p}^{\rm faint}=17.77$ (left panel), and the \Ha line flux limit at $F_{\rm H\alpha}^{\rm faint}=2\times 10^{-16}\,{\rm s^{-1} erg}$ (right). For the \OII, \OIII, \Hb, \NII, and \SII lines the methodology is identical, for the corresponding flux limits (see Sec.\,\ref{sec:selections}). The lower and upper redshift limits of the survey, $z_{\rm low}=0.02$ and $z_{\rm up}=0.22$, are highlighted by dashed vertical red lines. The maximum redshifts that the galaxy can have and still be included in the sample, considering its magnitude and \Ha flux limits, as well as the faintest luminosity, are shown by dotted vertical blue lines.}
  \label{fig:Vmaxscheme}
\end{figure*}
  
The differential luminosity function is defined as the number, $N$, of galaxies per unit luminosity interval and comoving volume, $V$,  as:
\begin{equation}
    \Phi(\log L,z)=\frac{dN}{d\log L\,dV(z)},
    \label{eq:phi}
\end{equation}
where $V$ is a function of redshift.
The $1/V_{\rm max}$ estimator \citep[]{Schmidt1968, Felten1976} allows us to correct the LF from the Malmquist bias, i.e. the fact that faint objects tend to be detected only in a small volume, while bright ones are observed in the entire sample volume \citep[see e.g.][]{Weigel2016}. Other methods to estimate the galaxy LF are the C$^-$ method by \citet{Linden1971},  the parametric maximum-likelihood STY method proposed by \citet{Sandage1978}, or the Stepwise Maximum Likelihood Method \citep[SWML;][]{Efstathiou1988, Norberg2002} that does assume any functional form.

Here we focus on emission-line LFs. The galaxy counts need to include their observational incompletness, usually given as a weight. In the \parent sample we have different sources of incompleteness to take into account. In fact, the \mainsel sample is a $r-$band magnitude limited sample, on top of which we have imposed a combination of cuts in flux and signal-to-noise for the six spectral lines under study.
In this section we describe the methodology used to estimate the line LFs taking into account the incompleteness induced by the thresholds we have imposed.

In practice, Eq.\,\ref{eq:phi} is evaluated by counting the number of galaxies in each $\Delta\log{L}$ bin, $N_{k}$, and weighting it by the maximum volume $V_{\rm max}$ in which each galaxy can be observed, given the survey limits and its luminosity. In the $k-$th bin of luminosity and for a sample of $i=1,...,N_k$ galaxies we have:
\begin{equation}
    \Phi_{1/V_{\rm max}}^{k}=\frac{1}{\Delta\log{L}^k}\sum_{i=1}^{\rm N_k}\frac{1}{V_{{\rm max},\,i}}\,.
    \label{eq:Vmaxstandard}
\end{equation}
To estimate $V_{{\rm max},\,i}$ we need to determine the maximum redshift, $z_{{\rm max},\,i}$ at which a galaxy could still be observed as part of the \mainsel sample, given its observational limits. Explicitly this is:
\begin{equation}
    V_{{\rm max},\,i}=\frac{A}{3}\left(\frac{\pi}{180}\right)^2\left(D_{\rm c}^3(z_{{\rm max},\,i})-D_{\rm c}^3(z_{\rm low})\right)\,,
    \label{eq:Vmaxvolume}
\end{equation}
where $A=7300\,\rm deg^2$ is the survey area, $D_{\rm c}(z)$ is the galaxy comoving distance depending on redshift and cosmology, and $z_{\rm low}=0.02$ is the lower redshift limit of the \mainsel sample.  

We modify the standard $1/V_{\rm max}$ formulation in Eq.\,\ref{eq:Vmaxstandard} to correct the \mainsel sample from the spectroscopic, $r-$band magnitude, and luminosity selection effects. 
To correct from spectroscopic incompleteness in the SDSS sample, we weight Eq.\,\ref{eq:Vmaxstandard} by $w_{{\rm c},\,i}=c_i^{-1}$, i.e. the inverse of the SDSS spectroscopic completeness. Explicitly we have:
\begin{equation}
    \Phi_{1/V_{\rm max}}^{k}=\frac{1}{\Delta\log{L}^k}\sum_{i=1}^{\rm N}\frac{w_{{\rm c},\,i}}{V_{{\rm max},\,i}}\,.
    \label{eq:Vmaxspec}
\end{equation}
This is a small correction, as the \mainsel sample is more than 80 per cent complete in spectroscopy \citep{Blanton2003_1}. 

To correct from the limits in $r-$band, line flux and S/N, we define the maximum redshift, $z_{{\rm max},\,i}$, of a galaxy in our sample as a function of the observational cuts imposed (see Sec.\,\ref{sec:selections}):
\begin{equation}
    z_{{\rm max},\,i}={\rm min}\left( z_{{\rm max},\,i}^{\rm mag}, \,z_{{\rm max},\,i}^F, \,z_{{\rm max},\,i}^{\rm S/N},\,z_{\rm up}\right)\,,
    \label{eq:zlim}
\end{equation}
where the superscripts indicate the contributions based on magnitude (mag), flux (F), and signal-to-noise (S/N) limits, while $z_{\rm up}=0.22$ is the upper limit of the \mainsel sample. The flux and S/N are grouped vectors, $F=(F_{\rm H\alpha},\,F_{\rm [O_{II}]},\,F_{\rm [O_{III}]},\,F_{\rm H\beta},\,F_{\rm [N_{II}]},\,F_{\rm [S_{II}]})$ and $\rm S/N=(S/N_{H\alpha},\,S/N_{[O_{II}]},\,S/N_{[O_{III}]},\,S/N_{H\beta},\,S/N_{[N_{II}]},\,S/N_{[S_{II}]})$. 


As shown in the left panel of Fig.\,\ref{fig:Vmaxscheme}, the faintest $r-$band absolute magnitude that a \mainsel can have while being part of a sample limited at $r_{\rm p}^{\rm faint}=17.77$ is \citep[][]{Blanton2003_1}:
\begin{equation}
M_{r,\,i}^{\rm faint}=r_{\rm p}^{\rm faint}-{\rm DM}(z_i)-K(z_i)\,,
\label{eq:maglim}
\end{equation}
where ${\rm DM}(z)$ is the distance modulus estimated at redshift $z$ in our fiducial cosmology, and $K(z)$ is the $K-$ correction\footnote{Note that $K-$ corrections are not needed when dealing with emission-line luminosities for which the redshift is known.}. We choose not to apply any evolution correction, as this is negligible \citep[][]{Blanton2001} in the redshift range of interest, and would require optimizing the model template to our ELG selection. 

The  maximum redshift, $z_{{\rm max},\,i}$, of a galaxy in our magnitude-limited sample, is found as the root of the following equation:
\begin{equation}
M_{r,\,i}^{\rm faint}-r_{\rm p}^{\rm faint}+{\rm DM}\left(z_{{\rm max},\,i}^{\rm mag}\right)+K(z_{{\rm max},\,i}^{\rm mag})=0\,,
\label{eq:zmax}
\end{equation}
%\vgp{how to you get $M_{r,\,i}^{\rm faint}$? maybe something should be said? I understand you can get a grid of M, DM, K for each z in the grid? }
which is solved iteratively by interpolating the $M_{r,\, i}(z)$ -- redshift relation.

The faintest \Ha ELG luminosity that a galaxy can have and still be in the sample, when this is limited in line flux, is obtained in a similar manner, as shown in the right panel of Fig.\,\ref{fig:Vmaxscheme}. For a flux limit $F_{\rm H\alpha}^{\rm faint}$ (Sec.\,\ref{sec:selections}) we derive the corresponding the faintest luminosity in that line as:
\begin{equation}
L_{{\rm H\alpha},\,i}^{\rm faint}\,{\rm [s^{-1}erg]}=4\pi\,D_{\rm L}^2(z_i)\,F_{\rm H\alpha}^{\rm faint}\,, 
\end{equation}
where the luminosity distance $D_{\rm L}(z_i)=(1+z_i)\,D_{\rm c}(z_i)$ is measured in [Mpc], and the line flux in [$\rm s^{-1}Mpc^{-2}erg$]. The maximum redshift, $z_{{\rm max},\,i}$, the galaxy can have in the \Ha flux-limited sample is the root of the following equation:
\begin{equation}
\left(1+z_{{\rm max},\,i}^{F_{\rm H\alpha}}\right)\,D_{\rm c}\left(z_{{\rm max},\,i}^{F_{\rm H\alpha}}\right)-\sqrt{\frac{L_{{\rm H\alpha},\,i}}{4\pi\,F_{\rm H\alpha}^{\rm faint}}}=0\,.
\label{eq:zmaxF}
\end{equation}
This is solved by interpolating and inverting the $D_{\rm c}(z)$ -- redshift relation.
For the \OII, \OIII, \Hb, \NII, and \SII lines we adopt the same procedure with the corresponding flux limit chosen for each line. In our case we choose the same cut for all the lines: $F>2\times10^{-16}\rm{cm^{-2}s^{-1}erg}$ (see Sec.\,\ref{sec:selections}).

Finally, the faintest \Ha flux a galaxy can reach in the \mainsel sample, when this is limited in ${\rm S/N}_{\rm H\alpha}^{\rm lim}$ (Sec.\,\ref{sec:selections}), is:
\begin{equation}
F_{{\rm H\alpha},\,i}^{\rm faint}={\rm S/N}_{\rm H\alpha}^{\rm lim}\times F_{{\rm err},\,i}\,,
\end{equation}
where $F_{{\rm err},\,i}$ is the line flux uncertainty. By substituting the above expression in Eq.\,\ref{eq:zmaxF}, we obtain $z_{{\rm max},\,i}^{\rm S/N_{H\alpha}}$. Again, for the rest of the lines the procedure is identical, using fixed signal-to-noise limit in our sample: $\rm S/N>2$ (see Sec.\,\ref{sec:selections}).


%%%%%%%%%%%%%%%%%%%%%%%%%%%%%%%%%%%%%%
%4. CLASSIFICATION
%%%%%%%%%%%%%%%%%%%%%%%%%%%%%%%%%%%%%%
\section{ELG classification}
\label{sec:classification}

Strong spectral emission lines can have different origins, the most common being the gas heated by newly forming stars. Galaxies hosting super massive black holes actively accreting mass, AGN and QSOs, also present strong emission lines produced in jets and shock regions. The number density of AGN and QSOs is lower than SF galaxies, and their line ratios are different \citep[see e.g.][]{Kewley2019}. Old stellar populations can also produce strong emission lines \citep[see e.g.][]{Kennicut1992, holmes, Cid2011, Sansom2015,Byler2019,Nersesian2019,Clarke2021}.


One of the goals of this work is to understand the contribution to the LF of local ELGs classified according to the most likely origin of their emission lines. We split the \mainsel sample using two selection criteria: (i) a sharp cut in sSFR to separate star-forming (SF) from passively evolving galaxies (Sec.\,\ref{sec:sSFRselection}), and (ii) the line ratios in the BPT and WHAN diagrams (Sec\,\ref{sec:BPT}).
In Sec.~\ref{sec:results} we will study the luminosity functions for each of these ELG types.

%%%%%%%%%%%%%%%%%%%%%%%%%%%%%%%%%%%%%%
%4.1 sSFR classification
%%%%%%%%%%%%%%%%%%%%%%%%%%%%%%%%%%%%%%

\subsection{Classification using the sSFR}
\label{sec:sSFRselection}

We select star-forming galaxies as those with ${\rm sSFR}>10^{-11}{\rm yr^{-1}}$ in the \mainsel sample. These galaxies constitute $83.6$ per cent of the sample, including the volume correction.
The value chosen for this cut corresponds to the classical threshold adopted to separate SF from passive galaxies \citep[e.g.][]{Ilbert2015, Donnari19, CorchoCaballero21}.

\begin{figure*}
\centering
    \includegraphics[width=0.45\linewidth]{MainBTPNIImegacorte6lines.png}\quad
    \includegraphics[width=0.45\linewidth]{MainBTPSIImegacorte6lines.png}
    \includegraphics[width=0.45\linewidth]{Mainwhan.png}\quad\includegraphics[width=0.45\linewidth]{Mainwhancomp.png}
\caption{\NII and \SII BPT (top row) and WHAN diagrams (bottom row), for \mainsel galaxies. Galaxies in these diagrams are color-coded by their sSFR. In the BPT diagrams we overplot the \citet{Kewley2001}, \citet{Kauffmann2003} and \citet{Kewley2006} demarcation lines (black solid, dot-dashed, and thick dotted lines, respectively) separating the SF, AGN, LINER and Seyfert contributions (see Sec.\,\ref{sec:BPT}). Ambiguous objects are represented in the BPT diagrams as squares. The WHAN diagram is the same in both panels; for clarity, we show Composite ELGs on the right panel and the rest on the left. In the WHAN diagram, we indicate with horizontal lines the ${\rm EW=6}\,$\AA\, limit separating Seyferts from LINERs; and the ${\rm EW=1}\,$\AA\, limit for passive galaxies. Here we chose to relax the \citet{Cid2011} ${\rm EW=0.5}\,$\AA\, criterion to obtain better statistics for our Passive ELG sample, so their LF is not dominated by noise. 
In the WHAN diagram, we also show as a vertical line at ${\rm log([N_{II}]/H\alpha)=-0.4}$, the \citet{Stasi2006} criterion  to separate SF ELGs form the rest. Note that we do not apply this cut in our classification. The colour bar on the bottom right is fainter.}
  \label{fig:BPT}
\end{figure*}

%%%%%%%%%%%%%%%%%%%%%%%%%%%%%%%%%%%%%%
%4.2 BPT and WHAN diagrams
%%%%%%%%%%%%%%%%%%%%%%%%%%%%%%%%%%%%%%
\subsection{Classification with the BPT and WHAN diagrams}
\label{sec:BPT}

As illustrated in Fig.~\ref{fig:BPT}, we classify the origin of the \mainsel spectral lines using the emission-line ratios in the Baldwin-Phillips-Terlevich (BPT) and the EW$_{\rm{H\alpha}}$ versus \NII/\Ha (WHAN) diagnostic diagrams \citep[e.g.][]{Stasi2006, Cid2011}.


We build the BPT diagrams for the \mainsel \NII and \SII lines and adopt the demarcation criteria from \citet{Kewley2001} and \citet{Kauffmann2003} (\textquoteleft Kew01' and \textquoteleft Kau03', hereafter) to separate ELGs into SF, Composite galaxies and AGN. 
The Kew01 line marks the upper envelope of the H\,{\sc ii} region in \citet{Kewley2001} photoionization models. Above this threshold, the origin of emission lines is expected to be different from young O and B stars \citep[see also][]{Belfiore2016}. The Kau03 demarcation line is derived from an empirical relation to separate SF galaxies. Between this line and that from Kew01, the regions where emission lines originate may be due to star formation and/or other ionization sources.


For those galaxies above the Kew01 line in the BPT \SII diagram, we further split the possible origin of their emission lines using the \citet{Kewley2006} criterion (\textquoteleft Kew06', hereafter) coupled with the $\rm EW\geq6$\,\AA\, condition from \citet{Cid2011} in the WHAN diagram; i.e. the plane defined by the \Ha EW values as a function of ${\rm log([N_{II}]/H\alpha)}$. This separation allows us to better distinguish AGN candidates into Seyfert galaxies and low-ionization narrow emission-line regions \citep[LINERs;][]{Heckman1980}. 

LINERs are characterised by lower luminosities compared to Seyfert galaxies and QSOs.
It is well known that most nearby AGN with \OII, \SII or \OI emission are dominated by LINERs \citep[e.g.][]{Ho1995, Ho1997, Kauffmann2003, Kewley2006, Singh2013, Belfiore2016}. Considering the intensity of their emissions, Seyfert sources and LINERs are often referred to as \textquoteleft strong' and \textquoteleft weak' AGN, respectively \citep[see e.g.][]{Cid2011}.
On the other hand, these line ratios have also been observed in the outskirts of galaxies \citep[e.g.][]{califa}, and therefore it is unclear whether they may actually be produced by other mechanisms.

By adopting the above criteria, we finally classify the galaxies in our \mainsel sample into the following classes:
\begin{enumerate}[1),leftmargin=*,itemsep=0pt,labelsep=12pt]
    \item \textit{Star-forming} (SF): below Kau03 in \NII~BPT and Kew01 in \SII~BPT;
    \item \textit{Passive}: $\rm EW_{H\alpha}<1\,$\AA. We relax the \citet{Cid2011} criterion at $\rm EW_{H\alpha}<0.5\,$\AA\, to allow better statistics in our passive ELG sample, otherwise their LFs are dominated by noise;
    \item \textit{Seyfert} (Sy): above Kew01 in both BPT diagrams, above Kew06 in \SII~BPT, and $\rm EW_{H\alpha}\ge 6$\,\AA;
    \item \textit{LINERs}: above Kew01 in both BPT diagrams, below Kew06 in \SII~BPT;
    \item \textit{Composite}: between Kau03 and Kew01 in \NII~BPT;
    \item \textit{Ambiguous}: galaxies that either do not fall within any of the previous classifications (mostly Seyfert galaxies with $\rm EW_{H\alpha}< 6$\,\AA), or that belong to more than one class at the same time.
\end{enumerate}

%%%%%%%%%%%%%%%%%%%%%%%%%%%%%%%%%%%%%%%%%%%%%%%%%%%
%4.3 comparison
%%%%%%%%%%%%%%%%%%%%%%%%%%%%%%%%%%%%%%%%%%%%%%%%%%%
\subsection{Comparison of the classifications}
\begin{table}
    \centering
    \setlength{\tabcolsep}{3pt}
    \begin{tabular}{|c|c|c|c|}
    \hline
    BPT+WHAN & Total & \multicolumn{2}{c|}{Intersection} \\ 
    type & fraction&${\rm sSFR}>10^{-11}/{\rm yr}$ & ${\rm sSFR}\leq 10^{-11}/{\rm yr}$\\
        &          &(87\% of total) & (13\% of total)\\
    \hline
    SF& 63.0 & 100.0&0.0 \\
    Passive& 1.5 & 1.2&98.8 \\
    Seyferts& 2.9 & 80.9&19.1 \\
    LINERs& 8.8 & 10.4&89.6 \\
    Composite& 18.0 & 83.4&16.6 \\
    Ambiguous& 5.8 & 98.7&1.3\\ \hline
    
    \hline
    \end{tabular}
    \caption{{\em Second column:} volume-corrected percentages of the different types of ELGs as classified using the BPT+WHAN diagrams. {\em Last two columns:} per cent split of the total fraction for each type based on ${\rm sSFR}=10^{-11}{\rm yr^{-1}}$.}
    \label{tab:types}
\end{table}

Table\,\ref{tab:types} compares the volume-corrected percentages of galaxies, classified using the BPT+WHAN diagrams, with those that have a sSFR either above or below ${\rm sSFR}=10^{-11}{\rm yr^{-1}}$. It is clear from this table that, according to the BPT+WHAN classification, spectral emission lines originate from star-forming regions only for $63$ per cent of the \mainsel sample. Spectral emission lines are not originated in SF regions for an important fraction of ELGs with ${\rm sSFR}>10^{-11}{\rm yr^{-1}}$. The origin of these lines is likely to be shocks, as the combined total fraction of SF Seyfert, LINERs and composite ELGs is 29.8 per cent. 

In Fig.~\ref{fig:BPT} we show how the six \mainsel types distribute as a function of sSFR, in the \NII and \SII~BPT planes (top row), and in the WHAN diagram (bottom row). Composite and passive ELGs constitute 18.0 and 1.5 per cent of the total, respectively, and show lower sSFR values compared to the SF population (i.e. ${\rm sSFR}\lesssim10^{-9.8}{\rm yr^{-1}}$).
LINERs ($8.8$ per cent of the ELG) exhibit even smaller sSFR values, i.e. ${\rm sSFR}\lesssim10^{-11.4}{\rm yr^{-1}}$. Ambiguous galaxies make up 5.8 per cent of the \mainsel sample. They also feature very small sSFR values, and they tend to preferentially occupy the AGN region of the BPT diagram.
Finally, Seyfert ELGs are a mixed population in terms of sSFR. While most of them will be classified as star-forming, a non-negligible fraction of them display sSFR below our adopted threshold of $10^{-11}{\rm yr^{-1}}$.



Both panels in the bottom row of Fig.\,\ref{fig:BPT} show the same WHAN diagram. We separately show the composite ELGs in the right panel to visually distinguish them from the rest. We overplot as horizontal lines the $\rm EW=1\,$\AA\, threshold used to separate passive ELGs from the rest; and the and $\rm EW=6\,$\AA\, used to separate Seyfert ELGs from LINERs. Both the BPT and WHAN diagrams provide equivalent classifications for Seyfert galaxies and LINERs. 

In the WHAN diagrams in Fig.\,\ref{fig:BPT}, we also show a vertical solid line at $\rm log([N_{II}]/H\alpha)= -0.4$, the \citet{Stasi2006} criterion that splits SF galaxies from the rest \citep[see also][]{Cid2011}. Note that we do not apply this last condition to our \mainsel classification. We find that 19 per cent of SF galaxies selected from BPT are outside the region delimited by the \citet{Stasi2006} criterion, falling into the Seyfert region in the WHAN plane.

% sSFR vs steller mass ------------------------------
\begin{figure}
\centering
    \includegraphics[width=0.9\linewidth]{sSFRlogMstar6lines.png}\vspace{-0.3cm}
\caption{sSFR as a function of stellar mass for the different ELG contributions. We show only 10 per cent of each subset to avoid saturation. The SF population with ${\rm sSFR}>10^{-11}{\rm yr^{-1}}$ is shown as black contours. The blue and red lines are \citet[]{2007ApJS..173..315S} linear fits to the SDSS star-forming (SF) and passive (P) galaxy populations. }
  \label{fig:ssfrmass}
  \end{figure}

Fig.\,\ref{fig:ssfrmass} compares, in the sSFR -- stellar mass plane, our ELG classification based on BPT+WHAN with the one based on sSFR (black contours). Both SF ELG classifications overlap well and concentrate in the upper region of the sSFR -- stellar mass plane, i.e. at higher sSFR and lower mass values. In particular, while LINERs and passive ELGs mainly inhabit the lower tail of the distribution, towards lower sSFR values, composite and Seyfert galaxies populate the entire sSFR range. In terms of stellar mass, while SF ELGs span smaller values, down to $10^8\,\rm M_\odot$, the other ELG types concentrate above $10^{10}\,\rm M_\odot$.
  
The SF and passive \mainsel are well represented by the linear fits performed by \citet[]{2007ApJS..173..315S} to the average sSFR in bins of stellar mass for the SDSS star-forming and passive galaxy populations.

\subsection{Old populations}\label{sec:oldpops}
\begin{figure}
\centering
    \includegraphics[width=\linewidth]{D40006linesmass.png}\vspace{-0.3cm}
\caption{$\rm D_n(4000)$ break index as a function of the galaxy stellar mass, color-coded by sSFR. We compare the \mainsel sample (small dots) with its SF population selected from sSFR (contours), and with the median $\pm\sigma$ results of the SF (light blue squares), composite (brown hexagons), Seyfert (navy blue stars), LINERs (orange diamonds), and passive (red triangles) contributions selected using the BPT+WHAN diagrams. The horizontal dotted line indicates the typical separation between younger and older stellar populations. }
  \label{fig:D4000}
  \end{figure}

Galaxies that are passively evolving can present an excess of UV flux due to an old but hot stellar population, such as hot horizontal branch stars burning helium \citep[e.g.][]{phillipps2020}. 
We find that 16.4 (1.5) volume corrected per cent of the sample are passive according to the sSFR (BPT+WHAN) classification used in this study.

To better understand the contribution of old stellar populations to the \mainsel sample, we study the 4000\,\AA\,break index, or ${\rm D_n(4000)}$, as a function of stellar mass and sSFR. The $\rm D_n(4000)$ index is reddening insensitive and traces SFRs on a time scale of 300--1000 Myrs. We employ the ${\rm D_n(4000)}$ values provided in the MPA-JHU catalog. These were estimated as the ratio of the flux in the red continuum to that in the blue continuum \citep[see e.g.][]{Balogh1999, Angthopo2020}:
\begin{equation}
    {\rm D_n(4000)}=\frac{\langle F_{\rm c}^{ \rm r}\rangle}{\langle F_{\rm c}^{ \rm b}\rangle},\,\,\,\,{\rm where}
\end{equation}
\label{eq:D4000eq}
\begin{equation}
    \langle F_{\rm c}^{ i}\rangle=\frac{1}{(\lambda_2^i-\lambda_1^i)}\int_{\lambda_1^i}^{\lambda_2^1} F_{\rm c}(\lambda)\,d\lambda\,,
\end{equation}
and $(\lambda_1^{\rm b},\lambda_2^{\rm b},\lambda_1^{\rm r},\lambda_2^{\rm r})=(3850,3950,4000,4100)$\AA. 



Fig.\,\ref{fig:D4000} compares the $\rm D_n(4000)$ index as a function of the galaxy stellar mass\footnote{We have investigated the evolution of the $\rm D_n(4000)$ -- $\log{\rm M_\star}$ relation, finding no significant variation over the redshift range $0.02<z<0.22$.}, color-coded by sSFR, for the \mainsel sample and its different components.

SF ELGs, no matter if selected from sSFR or from BPT+WHAN, are fully dominated by young stellar components. Considering the error bars, their $\rm D_n(4000)$ values range between 1 and 1.7, but most of them concentrate below 1.4.
Above ${\rm M_\star}\sim10^{10.5}\rm M_\odot$, they also show some contributions from old stellar components, which are negligible (2.1 per cent) for the SF ELGs based on BPT+WHAN, but significant (24 per cent) for SF ELGs selected from sSFR.

On the other extreme, ELGs classified as LINERs or passive exhibit higher $\rm D_n(4000)$ values, mostly between 1.4 and 2. These ELGs are thus not only characterised by small sSFR values, as we have seen in the previous section, but they are also located outside the contour defined by the galaxies with sSFR$> 10^{-11}$~yr$^-1$ on the $\rm D_n(4000)$ -- $\log{\rm M_\star}$ plane.
According to their $\rm D_n(4000)$ values, 99.9 per cent of the BPT+WHAN ELGs classified as passive are dominated by an old stellar component, with their $\rm D_n(4000)$ indices ranging between 1.6 and 2. LINERs also exhibit very high $\rm D_n(4000)$ values. About 99.5 per cent of LINERs are dominated by older stars, with $\rm D_n(4000)$ between 1.4 and 2. The sources of the ionising photons in LINERS are expected to be different from star forming regions. The origins could be hot low-mass evolved stars \citep[e.g.][]{holmes}, diffuse ionised gas \citep[e.g.][]{dig}, and X-ray busters \citep[e.g.][]{xrb}. As the EW of LINERs are low, the origin of the emission lines is expected to be less energetic than AGN or shocks.

The situation is much more complex for Seyfert galaxies.
Note that the sSFR values and stellar masses of these objects cover the whole range $10^{-11.4}-10^{-9.5}\,{\rm yr^{-1}}$ and $10^{8.5}-10^{12}{\rm M_\odot}$, respectively.
51.1 per cent of Seyfert galaxies are dominated by old stellar components, with $\rm D_n(4000)\sim1.5$, and the relation between $\rm D_n(4000)$ and stellar mass is in between the trends observed for SF and passive systems.

Composite ELGs, on the other hand, are consistent with the high-mass end of the main sequence of star formation (stellar masses above $10^{11}{\rm M_\odot}$, and sSFRs in the range $10^{-10.6}-10^{-10.2}\,{\rm yr^{-1}}$).
Only 32.9 per cent of them are dominated by old stars, with $\rm D_n(4000)\sim1.5$, and they follow the same scaling relation as the SF population.



%%%%%%%%%%%%%%%%%%%%%%%%%%%%%%%%%%%
% 5. LFs
%%%%%%%%%%%%%%%%%%%%%%%%%%%%%%%%%%%%%%
  
\section{Luminosity functions}
\label{sec:results}

\subsection{{\em Main-ELG} luminosity functions}
\label{sec:LFresults}
    \begin{figure*}
\centering%\vspace{-0.3cm}
    \includegraphics[width=0.42\linewidth]{LFHasaunders.png}\hfill\quad
    \includegraphics[width=0.42\linewidth]{LFHbsaunders.png}
    \includegraphics[width=0.42\linewidth]{LFO2saunders.png}\hfill\quad
    \includegraphics[width=0.42\linewidth]{LFO3saunders.png}
    \includegraphics[width=0.42\linewidth]{LFN2saunders.png}\hfill\quad
   \includegraphics[width=0.42\linewidth]{LFS2saunders.png}
    \vspace{-0.2cm}
\caption{From top to bottom and from left to right: \Ha, \Hb, \OII, \OIII, \NII, and \SII luminosity functions of the \mainsel sample (full black dots). The contribution of ELGs classified in different ways are shown by empty coloured markers, with colours as indicated in the legend. We compare our results with several published measurements in the local Universe: ELGs from \citet{1995ApJ...455L...1G} and \citet{Gallego2002} at $z\leq0.045$, \citet{Sullivan2000} at $z<0.4$, \citet{Fujita2003} at $z=0.24$, \citet{Ly2007} at $z-0.08$, \citet{James2008} $z<0.1$, \citet{Gilbank2010} at $0.032<z<0.2$, \citet{2013MNRAS.433.2764G} at $z<0.1$, \citet{Pirzkal2013} at $z<0.5$, \citet{Vilella2021} at $z\leq0.017$; AGN from \citet{Schulze2009} at $z<0.3$, and \citet{Bongiorno2010} at $0.15<z<0.92$. We overplot the Saunders fits as thick black lines; the parameters are tabulated in Table\,\ref{tab:fitparsaunders} and were obtained considering only the points above the luminosity completeness threshold (vertical lines) established in Sec. \ref{sec:incompleteness} as $L=10^{39.8}\rm s^{-1}erg$ for all the lines of interest. The error bars are computed from 70 jackknife resamplings.}
\label{fig:LFplotsaunders}
  \end{figure*}


\begin{table*}
 \setlength{\tabcolsep}{5pt}
  \centering
 \begin{tabular}{l c c c c c}
    \toprule
    &\multicolumn{5}{c}{Saunders}\\
    &$\Phi_\star$&$\log{L_\star}$&$\alpha$&$\sigma$&$\chi^2_{\rm red}$\\
    \midrule
    &\multicolumn{5}{c}{\Ha}\\
    Full sample&5.65$\pm$0.66&40.42$\pm$0.26&-0.21$\pm$0.06&-0.70$\pm$0.01&0.5\\ 
    SF sSFR&5.48$\pm$0.30&40.23$\pm$0.31&-0.03$\pm$0.09&0.69$\pm$0.01&0.3\\
    SF BPT+WHAN&4.19$\pm$0.09&40.12$\pm$0.30&0.07$\pm$0.05&-0.70$\pm$0.01&0.9\\
    LINERs&0.35$\pm$0.19&40.35$\pm$0.62&-0.65$\pm$0.19&-0.55$\pm$0.04&1.2\\
    Composite&1.50$\pm$0.47&40.00$\pm$1.22&-0.16$\pm$0.30&-0.77$\pm$0.04&1.6\\
    Seyfert&0.07$\pm$0.03&41.19$\pm$0.52&-0.24$\pm$0.12&-0.49$\pm$0.04&1.1\\
    Passive&0.04$\pm$0.03&40.53$\pm$0.73&-1.02$\pm$0.19&0.44$\pm$0.08&1.1\\
        \midrule
        &\multicolumn{5}{c}{\Hb}\\
    Full sample&4.71$\pm$1.63&40.00$\pm$0.65&-0.38$\pm$0.16&0.73$\pm$0.02&1.5\\
    SF sSFR&4.18$\pm$1.36&40.00$\pm$0.69&-0.32$\pm$0.18&0.72$\pm$0.02&2.0\\
    SF BPT+WHAN&3.48$\pm$0.55&40.00$\pm$0.60&-0.32$\pm$0.10&-0.72$\pm$0.02&1.9\\
    LINERs&0.08$\pm$0.06&40.46$\pm$0.57&-0.97$\pm$0.17&-0.41$\pm$0.05&2.5\\
    Composite&0.75$\pm$0.54&40.10$\pm$0.94&-0.70$\pm$0.20&0.75$\pm$0.05&2.1\\
    Seyfert&0.04$\pm$0.03&40.80$\pm$1.19&-0.52$\pm$0.22&-0.50$\pm$0.11&2.7\\
    Passive&0.03$\pm$0.02&40.32$\pm$1.19&-1.07$\pm$0.28&-0.38$\pm$0.13&2.6\\
        \midrule
    &\multicolumn{5}{c}{\OII}\\
    Full sample&6.50$\pm$3.97&40.00$\pm$1.42&-0.40$\pm$0.28&0.92$\pm$0.03&0.6\\
    SF sSFR&4.62$\pm$1.83&40.02$\pm$1.33&-0.26$\pm$0.21&0.86$\pm$0.03&1.0\\
    SF BPT+WHAN&3.69$\pm$1.31&40.00$\pm$1.21&-0.25$\pm$0.23&0.87$\pm$0.03&2.1\\
    LINERs&0.16$\pm$0.05&40.81$\pm$0.27&-0.90$\pm$0.04&-0.59$\pm$0.03&0.5\\
    Composite&1.33$\pm$0.98&40.00$\pm$2.09&-0.63$\pm$0.45&0.86$\pm$0.05&1.3\\
    Seyfert&0.14$\pm$0.02&40.05$\pm$1.43&-0.03$\pm$0.01&0.74$\pm$0.02&1.8\\
    Passive&0.17$\pm$0.11&40.00$\pm$8.77&-0.59$\pm$0.48&0.63$\pm$0.08&1.0\\
        \midrule
    &\multicolumn{5}{c}{\OIII}\\
    Full sample&0.72$\pm$0.30&40.87$\pm$0.47&-0.78$\pm$0.03&-1.00$\pm$0.05&0.7\\
    SF sSFR&1.00$\pm$0.49&40.62$\pm$0.62&-0.66$\pm$0.06&0.98$\pm$0.06&1.1\\
    SF BPT+WHAN&0.33$\pm$0.26&41.04$\pm$0.91&-0.80$\pm$0.05&0.96$\pm$0.15&12.7\\
    LINERs&0.42$\pm$0.31&40.00$\pm$1.32&-0.92$\pm$0.34&0.74$\pm$0.07&4.4\\
    Composite&0.66$\pm$0.37&40.00$\pm$1.75&-0.82$\pm$0.23&-0.92$\pm$0.14&3.2\\
    Seyfert&0.13$\pm$0.05&40.00$\pm$1.52&0.32$\pm$0.29&0.73$\pm$0.06&1.7\\
    Passive&0.02$\pm$0.01&40.64$\pm$1.53&-1.11$\pm$0.31&0.37$\pm$0.20&3.0\\
        \midrule
    &\multicolumn{5}{c}{\NII}\\
    Full sample&6.21$\pm$1.19&40.00$\pm$0.48&-0.25$\pm$0.13&-0.71$\pm$0.01&0.8\\
    SF sSFR&5.10$\pm$1.27&40.00$\pm$0.74&-0.22$\pm$0.19&-0.71$\pm$0.02&1.0\\
    SF BPT+WHAN&2.35$\pm$0.58&40.47$\pm$0.36&-0.52$\pm$0.11&-0.63$\pm$0.01&3.8\\
    LINERs&0.34$\pm$0.12&40.41$\pm$0.39&-0.71$\pm$0.07&0.61$\pm$0.04&0.3\\
    Composite&1.47$\pm$1.03&40.00$\pm$0.15&-0.45$\pm$0.29&0.82$\pm$0.04&1.6\\
    Seyfert&0.15$\pm$0.01&40.00$\pm$1.03&0.10$\pm$0.08&0.72$\pm$0.04&0.3\\
    Passive&0.01$\pm$0.02&40.92$\pm$1.63&-1.29$\pm$0.18&0.40$\pm$0.26&1.5\\
        \midrule
    &\multicolumn{5}{c}{\SII}\\
    Full sample&5.71$\pm$3.22&40.00$\pm$1.22&-0.30$\pm$0.26&0.69$\pm$0.03&5.4\\
    SF sSFR&4.01$\pm$1.35&40.26$\pm$0.55&-0.43$\pm$0.16&0.65$\pm$0.02&1.4\\
    SF BPT+WHAN&2.57$\pm$0.55&40.38$\pm$0.35&-0.44$\pm$0.12&0.61$\pm$0.02&2.4\\
    LINERs&0.42$\pm$0.22&40.19$\pm$0.62&-0.69$\pm$0.18&0.62$\pm$0.04&0.5\\
    Composite&1.25$\pm$1.00&40.00$\pm$0.03&-0.60$\pm$0.29&-0.78$\pm$0.04&1.3\\
    Seyfert&0.14$\pm$0.02&40.03$\pm$1.48&0.01$\pm$0.02&-0.67$\pm$0.04&0.4\\
    Passive&0.02$\pm$0.01&40.56$\pm$0.42&-1.19$\pm$0.10&0.41$\pm$0.05&0.2\\
\bottomrule
  \end{tabular}
   \caption{Saunders best-fit model parameters. The fit is done to the measured luminosity functions in Tables\,\ref{tab:LFtable1} and \ref{tab:LFtable2}. Note that $\Phi_\star$ is given in units of ${\rm 10^3 Mpc^{-3}dex^{-1}}$, while the argument of $\log{L_\star}$ is in units of $\rm s^{-1}\,erg$.} 
 \label{tab:fitparsaunders}
\end{table*}

Fig.\,\ref{fig:LFplotsaunders} presents our \mainsel LF for \Ha, \Hb, \OII, \OIII, \NII, and \SII emission lines. Our \mainsel LF measurements are in good agreement with several published results in the local Universe \citep[e.g.][]{1995ApJ...455L...1G, Gallego2002,Sullivan2000, Fujita2003, Ly2007, James2008, Schulze2009, Gilbank2010, Bongiorno2010, 2013MNRAS.433.2764G, Pirzkal2013, Comparat2016}. 


For each measurement we overplot the best fit obtained with the functional form proposed by \citep[]{Saunders1990}:
\begin{equation} 
    \Phi(L)= \Phi_\star\left(\frac{L}{L_\star}\right)^\alpha \exp\left[-\left(\frac{\log(1+L/L_\star)}{\sqrt{2} \sigma}\right)^2\right]\,.
    \label{eq:saunders}
\end{equation}
in terms of four free parameters.
We fit the quantity $\log(\Phi(L))$, considering only the points above the luminosity completeness threshold established in Sec.\,\ref{sec:incompleteness}, represented as a vertical line in Fig.\,\ref{fig:LFplotsaunders}.
The optimal parameters for each line LF are reported in Table\,\ref{tab:fitparsaunders}.
The reduced $\chi^2$ values indicate that the Saunders model statistically provides a very good fit both to the \mainsel LFs and their different contributions.

We also fit the ELG LFs using a single Schechter function \citep{Schechter1976}:
\begin{equation} %phistar*((x/Lstar)**alpha)*np.exp(-x/Lstar)
    \Phi(L)= \Phi_\star \left(\frac{L}{L_\star}\right)^{\alpha}\exp{\left(-\frac{L}{L_\star}\right)}\,,
    \label{eq:singleschechter}
\end{equation}
and a double Schechter one \citep[e. g.][]{Blanton2005lf}:
\begin{equation} %(1./Lstar)*(phistar1*(x/Lstar)**alpha1+phistar2*(x/Lstar)**alpha2)*np.exp(-x/Lstar)
    \Phi(L)= \left[\Phi_1^\star \left(\frac{L}{L_\star}\right)^{\alpha_1}+\Phi_2^\star \left(\frac{L}{L_\star}\right)^{\alpha_2}\right]\exp{\left(-\frac{L}{L_\star}\right)}\,.
    \label{eq:doubleschechter}
\end{equation}
Their best-fit parameters and results are shown in Appendix \ref{sec:otherLFfits}. 

The reduced $\chi^2$ values in Table~\ref{tab:fitparrestschechter} indicate that a single Schechter function provides a poor fit to the observational data.
The measured line LFs do show an excess in the very bright end, as already observed by \citet{Blanton2007} and \citet{MonteroDorta2009}, who justified this excess by the presence of AGN and QSOs.

The double Schechter model statistically provides a good fit to the \mainsel LF, as shown in Table~\ref{tab:fitparrest2schechter}, but it produces a bump in the bright end that seems to suggest overfitting rather than a physical feature of the LF.
Moreover, in Fig.~\ref{fig:LFplotsaunders}, when splitting the \mainsel LF in its different components, we see no evidence that the LF can be explained as the combination of two or more Schechter functions representing distinct galaxy populations. 
On the contrary, we argue that the bright end of both the individual and the combined LFs decrease more slowly than the exponential decay assumed by the Schechter parametric form.

The exact asymptotic behaviour of very luminous galaxies is fundamental in order to make extrapolations at higher redshift, and it has profound implications on the expected duration of reionisation and the type of galaxies contributing to it \citep[see e.g.][]{Mason2015, Sharma2018}.
Therefore, we further test a double power law model \citep[e.g.][]{Pei1995} with five parameters, i.e. slightly more flexible than the Saunders function:
\begin{equation}
    \Phi(L)= \Phi_0 \left(\frac{L}{L_0}\right)^{-\alpha_0}\left[1+\left(\frac{L}{L_0}\right)^\beta\right]^{(\alpha_0-\alpha_1)/\beta}\,.
\end{equation}
As shown in Fig.\,\ref{fig:LFplotpowlaw} and Table\,\ref{tab:fitparrest2pl}, our power law fit reaches the same level of agreement with the observations as the Saunders model, both for the \mainsel population as well as its different components. Therefore, in our analysis we choose to adopt a Saunders functional form for the fit, as it performs significantly better than any Schechter model and at a similar level than a model with more free parameters. 

We further explore the evolution of the \mainsel LF results by separating the sample in two bins of redshift: low-$z$ at $0.02<z\leq0.1$ and high-$z$ at $0.1<z<0.22$. We fit both results using Saunders models and compare them to the \mainsel fits to quantify the change in slope, i.e. the $\alpha$ parameter of the fit. We find that the faint end slope gets steeper with redshift. In particular, the $\alpha$ values of the \Ha, \Hb, \OII, \OIII, \NII and \SII LFs respectively change from -0.21, -0.38, -0.40, -0.78, -0.25, -0.30 in the \mainsel sample to -0.13, -0.78, -0.29, -0.83, -0.49, -0.33 in the low-$z$ population and to -0.57, -0.61, -0.73, -0.94, -0.87, -1.05 in the high-$z$ one. 



%%%%%%%%%%%%%%%%%%%%%%%%%%%%%%%%%%%%%%%%%%%%%%%%%%%%%%%%%%%%
%LF for different components
%%%%%%%%%%%%%%%%%%%%%%%%%%%%%%%%%%%%%%%%%%%%%%%%%%%%%%
\subsection{Contributions to the luminosity functions}
\label{sec:LFcomponents}

We find that the \mainsel LFs at $z\sim0.1$ are dominated by star-forming galaxies, independently from the emission line considered.
This is true for the two classifications we have made, based on sSFR and the BPT+WHAN diagrams. 
For most spectral lines, the second contributing population is that classified as \textquoteleft composite', which could actually be mostly massive SF galaxies with weaker emission lines.
The shape of the composite component of each emission line is similar to the full and SF results, but its amplitude is about one order of magnitude lower. 

Our measurements of the LFs for the Seyfert and LINER components are in reasonable agreement with results in the literature, \citep[see e.g.][]{Bongiorno2010,Ermash2013}.
In general, Seyfert galaxies contribute significantly to the \mainsel LFs only in the bright end, while passive galaxies and LINERs are non-negligible only in the faint end.
One may notice in Fig. \ref{fig:LFplotsaunders} that the Seyfert contribution to \OIII at $10^{42}\,\rm s^{-1}\,erg$ is higher than that from composite galaxies by $\sim 1$\,dex.
For the other lines (e.g. \NII and \SII), the contribution from Seyfert ELGs is either sub-dominant or similar to that of composite galaxies.
This is somewhat expected to happen by construction, as in the BPT diagram we are requiring that these emission lines are strong for a galaxy to be assigned to the Seyfert class.


%%%%%%%%%%%%%%%%%%%%%%%%%%%%%%%%%%%%%%%%%%%%%%%%%%%%%%%%%%%%
%Summary and conclusions
%%%%%%%%%%%%%%%%%%%%%%%%%%%%%%%%%%%%%%%%%%%%%%%%%%%%%%%%%%%%
\section{Summary and conclusions}
\label{sec:disc}
We have studied the properties of emission-line galaxies (ELGs) selected from the SDSS DR7 Main galaxy sample \citep[]{Strauss2002} at $0.02<z<0.22$ (i.e. 2.4\,Gyrs). We have obtained the spectral properties of these galaxies from the MPA-JHU catalogue\footref{mpanote}. Here we only study galaxies with a line flux of $F>2\times 10^{-16}{\rm cm}^{-2}{\rm s}^{-1}{\rm erg}$ and error $\sigma_{\rm F}<10^{-12}{\rm cm}^{-2}{\rm s}^{-1}{\rm erg}$, a signal-to-noise S/N$>2$, and an equivalent width EW$\geq 0$\,\AA\,in the six lines of interest: \Ha, \Hb, \OII, \OIII, \NII, and \SII. The resulting \mainsel is composed of 174572 ELGs (see Sec.~\ref{sec:selections}). The performed cuts guarantee the line luminosity function (LF) to be complete up to certain luminosity threshold.

We have measured the \mainsel luminosity function (LF) for the \Ha, \Hb, \OII, \OIII, \NII, and \SII emission lines. To this purpose, we have developed a generalised $1/V_{\rm max}$ weighting scheme to account for the different incompleteness effects in the LF due to the sample selection: the one due to the SDSS $r$-band magnitude limit, the spectroscopic selection, and those related to the thresholds imposed to each studied spectral line in our \mainsel sample. However, we have not taken into account the effect that the correlations between the different sources of incompleteness  might have. In fact, when selecting galaxies based on emission-line flux, we are implicitly removing a fraction of objects fainter than a given $M_r$ (see Fig.\,\ref{fig:LHaMr}). Neither the standard $1/V_{\rm max}$ estimator nor our modified method are capable of correcting from this source of incompleteness.

We have fit the \Ha, \Hb, \OII, \OIII, \NII, and \SII LFs using several functional forms (Sec.~\ref{sec:LFresults}): Saunders \citep{Saunders1990}, Schechter \citep{Schechter1976}, double Schechter \cite[e.g.][]{Blanton2005lf}, and a double power law \citep[e. g.][]{Pei1995}. Globally, the smallest reduced $\chi^2$ are achieved using double power laws, however this function has five free parameters. Comparable values of reduced $\chi^2$ are obtained using Saunders models, with four free parameters. We therefore conclude that Saunders functions are the most appropriate ones to describe the emission-line LFs.


We have investigated the contributions of different ELG types to the emission-line LFs. Our \mainsel sample has been classified both according to the specific star formation rate, ${\rm sSFR}>10^{-11}{\rm yr^{-1}}$ for star-forming (SF) galaxies, and using the line ratios (Sec.~\ref{sec:classification}). In particular, we have measured the \NII and \SII~BPT diagrams, as well as the WHAN one. Using these three diagrams, we have separated the \mainsel sample into star-forming (SF), passive, LINER, Seyfert and composite galaxies. We have also used the $\rm D_n(4000)$ break index to quantify the contribution of older stellar components to the \mainsel sample.


Our main findings on the ELG types and their contributions to the line LFs are summarized below:
\begin{itemize}
    \item The \mainsel sample is dominated by star-forming galaxies, independently from how they are selected and from the specific emission line considered. Including the volume correction, we find that 83.6 (63) per cent of the sample are SF when selected from sSFR (BPT+WHAN).
    \item ELGs selected using a combination of line flux and signal-to-noise cuts are not equivalent to ELGs selected using a sharp cut in sSFR. In order to minimise the incompleteness in the faint end of their luminosity function, it is preferable to select ELGs based on line flux and S/N.  
    \item Besides the SF population, composite galaxies and LINERs are the ones that contribute the most to the ELG production below $10^{41}\,\rm s^{-1}\,erg$.
    \item The Seyfert contribution is non-negligible only in the bright end of the line LF for the \OIII and \NII lines, $L_{[\rm NII]}>10^{42}{\rm s^{-1}erg}$, $L_{[\rm OIII]}>10^{43}{\rm s^{-1}erg}$. 
\end{itemize}


The \mainsel sample can be considered as a low-redshift laboratory to test the robustness of our ELG selection methods and our ability to correct for survey incompleteness.
The ongoing DESI \citep{Schlegel2015, desi} and near future Euclid \citep{euclid, Sartoris2016}, 4MOST \citep{dejong2012} or Rubin \citep{Abell2009,2012arXiv1211.0310L} surveys will target millions of galaxies out to $z\sim2$ with strong emission spectral lines. These will be used as tracers of the dark matter field, in an attempt to build the most detailed 3D maps of the Universe to date. 
The methods used in cosmological surveys for validating different inference pipelines are based on model catalogues of galaxies, and the results of this study, together with the \Ha\,\mainsel clustering and bias results from Favole et al. in prep., can be used as guidelines to prepare these and other future science cases at higher redshifts.
A detailed comparison of the results presented here with those from a range of semi-analytic galaxy models will be instrumental in order to constrain their parameters and make realistic predictions of the statistics of the galaxy population at earlier cosmic epochs.

\section*{Data availability}
The \mainsel selections and all the results of our analysis are publicly available at \url{http://research.iac.es/proyecto/cosmolss/pages/en/dataresults.php}. 

The observational samples were selected from the SDSS NYU--VAGC (\url{http://cosmo.nyu.edu/blanton/vagc/}) matched to the MPA-JHU DR7 spectral relase (\url{http://www.mpa-garching.mpg.de/SDSS/DR7/}) to obtain the emission line properties. 

\section*{Acknowledgements}
GF is supported by a {\em Juan de la Cierva Incorporación} grant n.\,IJC2020-044343-I. GF acknowledges the SNF 175751 “Cosmology with 3D Maps of the Universe” research grant and the LASTRO group at the Observatoire de Sauverny for hosting and supporting the first stage of this project. She further thanks Andr\'es Balaguera for insightful discussion on the computational aspects of this work. 

VGP is supported by the Atracci\'{o}n de Talento Contract no. 2019-T1/TIC-12702 granted by the Comunidad de Madrid in Spain. 
YA and PC acknowledge financial support from grant PID2019-107408GB-C42 of the Spanish State Research Agency (AEI/10.13039/501100011033).
AK is supported by the Ministerio de Ciencia e Innovaci\'{o}n (MICINN) under research grant PID2021-122603NB-C21 and further thanks Dan Lacksman for the flamenco moog.
SAC acknowledges funding from {\it Consejo Nacional de Investigaciones Cient\'{\i}ficas y T\'ecnicas} (CONICET, PIP-0387), {\it 
Agencia Nacional de Promoci\'on de la Investigaci\'on, el Desarrollo Tecnol\'ogico y la Innovaci\'on} (Agencia I+D+i, PICT-2018-3743), and {\it Universidad Nacional de La Plata} (G11-150), Argentina.

Funding for the SDSS and SDSS-II has been provided by the Alfred P. Sloan Foundation, the Participating Institutions, the National Science Foundation, the U.S. Department of Energy, the National Aeronautics and Space Administration, the Japanese Monbukagakusho, the Max Planck Society, and the Higher Education Funding Council for England. The SDSS Web Site is \url{http://www.sdss.org/}. 
The SDSS is managed by the Astrophysical Research Consortium for the Participating Institutions. The Participating Institutions are the American Museum of Natural History, Astrophysical Institute Potsdam, University of Basel, University of Cambridge, Case Western Reserve University, University of Chicago, Drexel University, Fermilab, the Institute for Advanced Study, the Japan Participation Group, Johns Hopkins University, the Joint Institute for Nuclear Astrophysics, the Kavli Institute for Particle Astrophysics and Cosmology, the Korean Scientist Group, the Chinese Academy of Sciences (LAMOST), Los Alamos National Laboratory, the Max-Planck-Institute for Astronomy (MPIA), the Max-Planck-Institute for Astrophysics (MPA), New Mexico State University, Ohio State University, University of Pittsburgh, University of Portsmouth, Princeton University, the United States Naval Observatory, and the University of Washington. 


\bibliographystyle{aa}
\bibliography{main}




\begin{appendix} 
\section{Selection effects and ELG properties for all the six lines of interest}
\label{sec:appendix1}
In Fig.\,\ref{fig:megacortealllines} below we show the impact of the line flux and SN selection cuts in all six lines of interest, colour-coded by sSFR (upper 6 panels) and EW (lower 6 panels). The results of the different lines are overall consistent, with \OII spanning larger EW values compared to the rest of the lines.
\begin{figure*}
\centering 
     \includegraphics[width=0.4\linewidth]{MainELGHbmegacortelineflux6lines.png}\hfill\vspace{0.5cm}
   \includegraphics[width=0.4\linewidth]{MainELGO2megacortelineflux6lines.png}\hfill\quad
  \includegraphics[width=0.4\linewidth]{MainELGO3megacortelineflux6lines.png}\hfill\vspace{0.5cm}
  \includegraphics[width=0.4\linewidth]{MainELGN2megacortelineflux6lines.png}\hfill\quad
 \includegraphics[width=0.4\linewidth]{MainELGS2megacortelineflux6lines.png}\vspace{0.5cm}
\caption{Same result as Fig.\,\ref{fig:megacorte} for the rest of the lines of interest, color-coded by sSFR. From top to bottom and from left to right we show the \Hb, \OII, \OIII, \NII and \SII lines.}
  \label{fig:megacortealllines}
  \end{figure*}
  
\begin{figure*}
\centering 
    \includegraphics[width=0.4\linewidth]{MainELGHbmegacorteEWlineflux6lines.png}\hfill\vspace{0.5cm}
  \includegraphics[width=0.4\linewidth]{MainELGO2megacorteEWlineflux6lines.png}\hfill\quad
  \includegraphics[width=0.4\linewidth]{MainELGO3megacorteEWlineflux6lines.png}\hfill\vspace{0.5cm}
  \includegraphics[width=0.4\linewidth]{MainELGN2megacorteEWlineflux6lines.png}\hfill\quad
 \includegraphics[width=0.4\linewidth]{MainELGS2megacorteEWlineflux6lines.png}\vspace{0.5cm}
\caption{Same result as Fig.\,\ref{fig:megacortealllines} but color-coded by EW.}
  \label{fig:megacortealllinesEW}
  \end{figure*}
  
  Fig.\,\ref{fig:charplotallines} displays the \mainsel sSFR as a function of stellar mass colour-coded by EW for the six lines of interest. The contours change in each panel as they are weighted by the Ew of each line. Overall the results are all consistent. The \OII line is the one showing higher EW values, while the \OIII and \Hb EW are more concentrated towards smaller values.

  \begin{figure*}
\centering 
     \includegraphics[width=0.4\linewidth]{MainELGHbcharlineflux6lines.png}\hfill\vspace{0.5cm}
   \includegraphics[width=0.4\linewidth]{MainELGO2charlineflux6lines.png}\hfill\quad
  \includegraphics[width=0.4\linewidth]{MainELGO3charlineflux6lines.png}\hfill\vspace{0.5cm}
   \includegraphics[width=0.4\linewidth]{MainELGN2charlineflux6lines.png}\hfill\quad
   \includegraphics[width=0.4\linewidth]{MainELGS2charlineflux6lines.png}\vspace{0.5cm}
  \caption{Same result as shown in the left panel of Fig.\,\ref{fig:charplot} for the other lines of interest. From top to bottom and from left to right we show the \Hb, \OII, \OIII, \NII and \SII lines.}
  \label{fig:charplotallines}
  \end{figure*}


  \begin{figure*}
\centering 
   \includegraphics[width=0.4\linewidth]{MainELGLHbmasssSFR.png}\hfill\vspace{0.5cm}
  \includegraphics[width=0.4\linewidth]{MainELGLO2masssSFR.png}\hfill\quad
   \includegraphics[width=0.4\linewidth]{MainELGLO3masssSFR.png}\hfill\vspace{0.5cm}
   \includegraphics[width=0.4\linewidth]{MainELGLN2masssSFR.png}\hfill\quad
   \includegraphics[width=0.4\linewidth]{MainELGLS2masssSFR.png}\vspace{0.5cm}
  \caption{Same result as shown in the right panel of Fig.\,\ref{fig:charplot} for the rest of the lines. From top to bottom and from left to right we show the \Hb, \OII, \OIII, \NII and \SII lines.}
  \label{fig:lummassallines}
  \end{figure*}

In Fig.\,\ref{fig:LMralllines} we show the emission line luminosity, in the six lines of interest, as a function of the $r-$band absolute magnitude, colour-coded by redshift. From left to right we show our result in three redshift bins to better analyse their evolution: the full sample at $0.02<z<0.22$, the lower $0.02<z<0.12$, and the upper $0.12<z<0.22$ $z$ bins. For the full sample we overplot the completeness limit chosen by eye as the luminosity below which the galaxy number density falls significantly. This threshold is different for each one of the six emission lines.

This result tells us that, when selecting ELGs by cutting in line flux (i.e. in luminosity), we are implicitly removing a fraction of the sample fainter than a given $r-$band magnitude, meaning that we are making our sample incomplete in $M_r$. Our $1/V_{\rm max}$ LF estimator is not able to correct from this incompleteness effect.
  \begin{figure*}
\centering 
  \includegraphics[width=0.87\linewidth]{LHaMralllines.png}%\vspace{-1cm}
  \caption{Same result as Fig.\,\ref{fig:LHaMr} for all the six lines of interest.}
  \label{fig:LMralllines}
  \end{figure*}
  

%%%%%%%%%%%%%%%%%%%%%%%%%%%%%%%%%%%%%%%%%%%%

\section{{\em Main-ELG} luminosity function values}
\label{sec:appendixLF}
The numerical values of the \Ha, \Hb, \OII, \OIII, \NII, and \SII\,\mainsel luminosity functions and its different components are provided in Tables\,\ref{tab:LFtable1} and \ref{tab:LFtable2}.

 \begin{table*}
  \setlength{\tabcolsep}{2pt}
  \centering
 \begin{tabular}{c c c c c c c c}
    \toprule
    &\multicolumn{7}{c}{$\log\Phi(L_{\rm H\alpha}$)}\\
$\log{L_{\rm H\alpha}}$&Full sample&SF sSFR&SF BPT+WHAN&LINERs&Composite&Seyferts&Passive\\ 
\midrule
39.1&$-2.978\pm0.183$&$-3.933\pm0.057$&$-4.297\pm0.168$&$-3.590\pm0.463$&$-3.462\pm0.093$&$-$&$-3.427\pm0.085$\\
39.3&$-2.662\pm0.109$&$-3.077\pm0.032$&$-3.371\pm0.025$&$-2.992\pm0.069$&$-3.724\pm0.106$&$-5.069\pm0.217$&$-3.533\pm0.110$\\
39.5&$-2.488\pm0.085$&$-2.681\pm0.028$&$-2.837\pm0.038$&$-3.037\pm0.050$&$-3.954\pm1.489$&$-4.414\pm0.170$&$-3.583\pm0.030$\\
39.7&$-2.174\pm0.011$&$-2.353\pm0.089$&$-2.475\pm0.097$&$-3.006\pm0.030$&$-2.953\pm0.085$&$-4.391\pm0.211$&$-3.513\pm0.076$\\
39.9&$-2.148\pm0.019$&$-2.276\pm0.066$&$-2.387\pm0.028$&$-3.119\pm0.054$&$-2.919\pm0.045$&$-3.971\pm0.131$&$-3.762\pm0.065$\\
40.1&$-2.191\pm0.031$&$-2.277\pm0.055$&$-2.445\pm0.026$&$-3.341\pm0.032$&$-2.773\pm0.074$&$-3.860\pm0.085$&$-4.063\pm0.057$\\
40.3&$-2.244\pm0.049$&$-2.320\pm0.054$&$-2.434\pm0.011$&$-3.492\pm0.032$&$-2.903\pm0.041$&$-3.797\pm0.130$&$-4.146\pm0.065$\\
40.5&$-2.305\pm0.106$&$-2.360\pm0.054$&$-2.508\pm0.061$&$-3.636\pm0.056$&$-2.932\pm0.049$&$-3.943\pm0.099$&$-4.494\pm0.031$\\
40.7&$-2.353\pm0.058$&$-2.393\pm0.036$&$-2.489\pm0.023$&$-3.872\pm0.035$&$-3.140\pm0.049$&$-3.945\pm0.123$&$-4.790\pm0.034$\\
40.9&$-2.500\pm0.040$&$-2.531\pm0.040$&$-2.633\pm0.013$&$-4.092\pm0.048$&$-3.276\pm0.036$&$-3.989\pm0.140$&$-5.113\pm0.035$\\
41.1&$-2.653\pm0.043$&$-2.675\pm0.040$&$-2.772\pm0.021$&$-4.415\pm0.041$&$-3.444\pm0.029$&$-4.214\pm0.142$&$-5.384\pm0.064$\\
41.3&$-2.833\pm0.039$&$-2.851\pm0.040$&$-2.931\pm0.021$&$-4.758\pm0.048$&$-3.714\pm0.041$&$-4.341\pm0.124$&$-6.009\pm0.044$\\
41.5&$-3.044\pm0.058$&$-3.057\pm0.040$&$-3.139\pm0.020$&$-5.183\pm0.046$&$-3.901\pm0.025$&$-4.563\pm0.145$&$-6.589\pm0.036$\\
41.7&$-3.299\pm0.040$&$-3.308\pm0.042$&$-3.393\pm0.020$&$-5.565\pm0.050$&$-4.219\pm0.038$&$-4.635\pm0.143$&$-7.191\pm0.063$\\
41.9&$-3.554\pm0.052$&$-3.559\pm0.037$&$-3.635\pm0.017$&$-6.237\pm0.035$&$-4.491\pm0.043$&$-5.032\pm0.162$&$-$\\
42.1&$-3.849\pm0.050$&$-3.854\pm0.043$&$-3.932\pm0.026$&$-6.681\pm0.055$&$-4.828\pm0.052$&$-5.152\pm0.047$&$-$\\
42.3&$-4.213\pm0.052$&$-4.216\pm0.039$&$-4.296\pm0.010$&$-7.253\pm0.204$&$-5.126\pm0.053$&$-5.690\pm0.144$&$-$\\
42.5&$-4.610\pm0.041$&$-4.612\pm0.042$&$-4.696\pm0.019$&$-8.106\pm0.204$&$-5.519\pm0.037$&$-6.124\pm0.150$&$-$\\
42.7&$-5.000\pm0.045$&$-5.001\pm0.053$&$-5.096\pm0.009$&$-8.072\pm0.204$&$-5.896\pm0.077$&$-6.395\pm0.156$&$-$\\
42.9&$-5.454\pm0.021$&$-5.454\pm0.057$&$-5.558\pm0.035$&$-$&$-6.269\pm0.074$&$-6.725\pm0.133$&$-$\\
43.1&$-6.011\pm0.021$&$-6.022\pm0.022$&$-6.078\pm0.037$&$-$&$-7.115\pm0.109$&$-8.069\pm0.217$&$-$\\
43.3&$-6.407\pm0.019$&$-6.407\pm0.036$&$-6.574\pm0.057$&$-$&$-6.961\pm0.057$&$-7.793\pm0.008$&$-$\\
43.5&$-7.075\pm0.217$&$-7.075\pm0.097$&$-7.125\pm0.034$&$-$&$-8.039\pm0.217$&$-8.039\pm0.217$&$-$\\
  \end{tabular}
 \begin{tabular}{c c c c c c c c}
    \midrule
        &\multicolumn{7}{c}{$\log\Phi(L_{\rm H\beta}$)}\\
{$\log{L_{\rm H\beta}}$}&Full sample&SF sSFR&SF BPT+WHAN&LINERs&Composite&Seyfert&Passive\\ 
\midrule
39.1&$-2.197\pm0.021$&$-2.372\pm0.070$&$-2.459\pm0.064$&$-3.000\pm0.049$&$-3.086\pm0.091$&$-4.210\pm0.003$&$-3.545\pm0.059$\\
39.3&$-2.154\pm0.032$&$-2.301\pm0.035$&$-2.458\pm0.009$&$-3.066\pm0.063$&$-2.795\pm0.040$&$-3.964\pm0.018$&$-3.435\pm0.066$\\
39.5&$-2.140\pm0.006$&$-2.348\pm0.052$&$-2.384\pm0.015$&$-3.142\pm0.010$&$-2.789\pm0.047$&$-3.849\pm0.146$&$-3.863\pm0.046$\\
39.7&$-2.237\pm0.031$&$-2.313\pm0.032$&$-2.434\pm0.025$&$-3.417\pm0.045$&$-2.900\pm0.051$&$-3.831\pm0.083$&$-3.970\pm0.034$\\
39.9&$-2.315\pm0.001$&$-2.386\pm0.052$&$-2.521\pm0.054$&$-3.547\pm0.034$&$-2.977\pm0.030$&$-3.948\pm0.156$&$-4.067\pm0.054$\\
40.1&$-2.361\pm0.047$&$-2.409\pm0.040$&$-2.501\pm0.022$&$-3.746\pm0.051$&$-3.155\pm0.047$&$-3.993\pm0.110$&$-4.391\pm0.044$\\
40.3&$-2.505\pm0.048$&$-2.543\pm0.041$&$-2.639\pm0.024$&$-3.969\pm0.049$&$-3.315\pm0.035$&$-3.982\pm0.114$&$-4.739\pm0.047$\\
40.5&$-2.646\pm0.041$&$-2.676\pm0.038$&$-2.761\pm0.009$&$-4.265\pm0.046$&$-3.471\pm0.048$&$-4.257\pm0.119$&$-5.065\pm0.036$\\
40.7&$-2.830\pm0.042$&$-2.857\pm0.043$&$-2.933\pm0.019$&$-4.605\pm0.033$&$-3.700\pm0.038$&$-4.362\pm0.128$&$-5.260\pm0.065$\\
40.9&$-3.046\pm0.036$&$-3.062\pm0.039$&$-3.125\pm0.016$&$-4.997\pm0.036$&$-4.020\pm0.037$&$-4.621\pm0.132$&$-5.954\pm0.071$\\
41.1&$-3.287\pm0.040$&$-3.298\pm0.041$&$-3.368\pm0.016$&$-5.420\pm0.045$&$-4.264\pm0.036$&$-4.739\pm0.122$&$-6.447\pm0.050$\\
41.3&$-3.539\pm0.043$&$-3.547\pm0.040$&$-3.608\pm0.013$&$-5.912\pm0.029$&$-4.587\pm0.038$&$-5.030\pm0.128$&$-7.330\pm0.102$\\
41.5&$-3.828\pm0.044$&$-3.834\pm0.043$&$-3.898\pm0.014$&$-6.511\pm0.108$&$-4.875\pm0.045$&$-5.188\pm0.147$&$-7.788\pm0.204$\\
41.7&$-4.157\pm0.040$&$-4.161\pm0.045$&$-4.222\pm0.016$&$-7.386\pm0.026$&$-5.209\pm0.056$&$-5.699\pm0.175$&$-7.939\pm0.204$\\
41.9&$-4.557\pm0.042$&$-4.562\pm0.043$&$-4.630\pm0.014$&$-7.397\pm0.115$&$-5.592\pm0.028$&$-6.019\pm0.132$&$-$\\
42.1&$-4.923\pm0.041$&$-4.925\pm0.039$&$-4.991\pm0.030$&$-8.072\pm0.204$&$-5.955\pm0.070$&$-6.405\pm0.143$&$-$\\
42.3&$-5.330\pm0.059$&$-5.332\pm0.036$&$-5.405\pm0.021$&$-$&$-6.346\pm0.060$&$-6.671\pm0.136$&$-$\\
42.5&$-5.854\pm0.004$&$-5.854\pm0.045$&$-5.892\pm0.042$&$-$&$-7.111\pm0.050$&$-8.069\pm0.217$&$-$\\
42.7&$-6.253\pm0.023$&$-6.253\pm0.020$&$-6.342\pm0.052$&$-$&$-7.104\pm0.140$&$-7.598\pm0.083$&$-$\\
42.9&$-6.992\pm0.127$&$-6.992\pm0.045$&$-6.992\pm0.114$&$-$&$-$&$-$&$-$\\
43.1&$-7.354\pm0.105$&$-7.354\pm0.115$&$-7.354\pm0.075$&$-$&$-$&$-$&$-$\\
43.3&$-$&$-$&$-$&$-$&$-$&$-$&$-$\\
43.5&$-$&$-$&$-$&$-$&$-$&$-$&$-$\\
  \end{tabular}
    \caption{\Ha and \Hb luminosity functions of the \mainsel sample and their different components. All the luminosities $L$ are in units of $\rm{s^{-1}erg}$ and the LFs $\Phi(L)$ are in ${\rm Mpc^{-3}dex^{-1}}$.} 
 \label{tab:LFtable1}
\end{table*}



\begin{table*}
\setlength{\tabcolsep}{2pt}
  \centering
 \begin{tabular}{c c c c c c c c}
    \toprule
       &\multicolumn{7}{c}{$\log\Phi (L_{\rm [O_{II}]}$)}\\
$\log{L_{\rm [O_{II}]}}$&Full sample&SF sSFR&SF BPT+WHAN&LINERs&Composite&Seyferts&Passive\\ 
\hline
39.1&$-2.529\pm0.063$&$-2.657\pm0.034$&$-2.851\pm0.050$&$-3.044\pm0.079$&$-3.642\pm0.512$&$-4.610\pm0.144$&$-3.585\pm0.065$\\
39.3&$-2.269\pm0.053$&$-2.415\pm0.047$&$-2.541\pm0.043$&$-3.155\pm0.069$&$-2.980\pm0.111$&$-4.289\pm0.124$&$-3.603\pm0.088$\\
39.5&$-2.141\pm0.018$&$-2.238\pm0.048$&$-2.339\pm0.062$&$-3.449\pm0.363$&$-2.888\pm0.047$&$-4.029\pm0.152$&$-3.351\pm0.071$\\
39.7&$-2.115\pm0.059$&$-2.246\pm0.017$&$-2.371\pm0.041$&$-3.024\pm0.044$&$-2.837\pm0.050$&$-3.834\pm0.185$&$-3.492\pm0.072$\\
39.9&$-2.239\pm0.025$&$-2.372\pm0.048$&$-2.572\pm0.058$&$-3.133\pm0.056$&$-2.818\pm0.043$&$-4.019\pm0.124$&$-3.715\pm0.049$\\
40.1&$-2.279\pm0.038$&$-2.399\pm0.034$&$-2.539\pm0.018$&$-3.173\pm0.005$&$-2.990\pm0.035$&$-3.894\pm0.175$&$-3.892\pm0.040$\\
40.3&$-2.365\pm0.037$&$-2.468\pm0.038$&$-2.559\pm0.009$&$-3.331\pm0.060$&$-3.143\pm0.045$&$-3.965\pm0.164$&$-4.076\pm0.026$\\
40.5&$-2.469\pm0.034$&$-2.548\pm0.035$&$-2.652\pm0.019$&$-3.558\pm0.046$&$-3.297\pm0.043$&$-4.039\pm0.127$&$-4.189\pm0.045$\\
40.7&$-2.617\pm0.035$&$-2.684\pm0.041$&$-2.794\pm0.026$&$-3.757\pm0.028$&$-3.453\pm0.042$&$-4.039\pm0.103$&$-4.549\pm0.031$\\
40.9&$-2.773\pm0.044$&$-2.833\pm0.040$&$-2.920\pm0.018$&$-3.943\pm0.044$&$-3.692\pm0.039$&$-4.224\pm0.142$&$-4.770\pm0.035$\\
41.1&$-2.933\pm0.044$&$-2.981\pm0.042$&$-3.067\pm0.021$&$-4.213\pm0.042$&$-3.858\pm0.046$&$-4.384\pm0.129$&$-5.146\pm0.035$\\ 
41.3&$-3.143\pm0.044$&$-3.189\pm0.040$&$-3.256\pm0.015$&$-4.481\pm0.038$&$-4.222\pm0.039$&$-4.545\pm0.154$&$-5.484\pm0.061$\\ 
41.5&$-3.398\pm0.043$&$-3.435\pm0.042$&$-3.500\pm0.013$&$-4.793\pm0.039$&$-4.495\pm0.038$&$-4.759\pm0.140$&$-5.885\pm0.031$\\ 
41.7&$-3.627\pm0.052$&$-3.654\pm0.047$&$-3.712\pm0.025$&$-5.140\pm0.036$&$-4.819\pm0.037$&$-4.961\pm0.112$&$-6.375\pm0.029$\\
41.9&$-3.899\pm0.047$&$-3.920\pm0.039$&$-3.971\pm0.014$&$-5.588\pm0.039$&$-5.115\pm0.042$&$-5.318\pm0.164$&$-6.908\pm0.118$\\ 
42.1&$-4.138\pm0.059$&$-4.152\pm0.014$&$-4.192\pm0.025$&$-6.068\pm0.032$&$-5.526\pm0.038$&$-5.598\pm0.123$&$-7.318\pm0.115$\\
42.3&$-4.490\pm0.041$&$-4.503\pm0.055$&$-4.540\pm0.018$&$-6.523\pm0.050$&$-5.922\pm0.041$&$-5.965\pm0.159$&$-7.913\pm0.204$\\
42.5&$-4.823\pm0.052$&$-4.832\pm0.067$&$-4.881\pm0.023$&$-6.997\pm0.081$&$-6.218\pm0.032$&$-6.246\pm0.066$&$-$\\
42.7&$-5.095\pm0.031$&$-5.102\pm0.078$&$-5.131\pm0.062$&$-7.910\pm0.204$&$-6.770\pm0.078$&$-6.678\pm0.138$&$-$\\
42.9&$-5.597\pm0.067$&$-5.605\pm0.033$&$-5.668\pm0.057$&$-$&$-6.667\pm0.149$&$-7.133\pm0.017$&$-$\\
43.1&$-5.786\pm0.096$&$-5.795\pm0.068$&$-5.809\pm0.093$&$-8.072\pm0.204$&$-8.128\pm0.188$&$-7.371\pm0.217$&$-$\\
43.3&$-6.148\pm0.177$&$-6.148\pm0.028$&$-6.172\pm0.024$&$-$&$-7.985\pm0.188$&$-7.561\pm0.094$&$-$\\
43.5&$-6.769\pm0.005$&$-6.769\pm0.093$&$-6.901\pm0.078$&$-$&$-7.349\pm0.188$&$-$&$-$\\
\end{tabular}
 \begin{tabular}{c c c c c c c c}
    \toprule
    &\multicolumn{7}{c}{$\log\Phi (L_{\rm [O_{III}]}$)}\\
$\log{L_{\rm [O_{III}]}}$&Full sample&SF sSFR&SF BPT+WHAN&LINERs&Composite&Seyferts&Passive\\ 
\midrule
39.1&$-2.150\pm0.046$&$-2.295\pm0.041$&$-2.417\pm0.023$&$-3.022\pm0.075$&$-2.879\pm0.085$&$-5.716\pm0.217$&$-3.386\pm0.047$\\
39.3&$-2.217\pm0.076$&$-2.393\pm0.046$&$-2.604\pm0.055$&$-3.008\pm0.061$&$-2.736\pm0.023$&$-4.988\pm0.187$&$-3.480\pm0.012$\\
39.5&$-2.202\pm0.046$&$-2.329\pm0.036$&$-2.462\pm0.017$&$-3.069\pm0.066$&$-2.915\pm0.038$&$-4.494\pm0.049$&$-3.598\pm0.047$\\
39.7&$-2.234\pm0.062$&$-2.364\pm0.053$&$-2.492\pm0.041$&$-3.134\pm0.037$&$-2.973\pm0.032$&$-4.064\pm0.128$&$-3.601\pm0.115$\\
39.9&$-2.416\pm0.038$&$-2.545\pm0.022$&$-2.672\pm0.020$&$-3.348\pm0.046$&$-3.131\pm0.033$&$-3.965\pm0.094$&$-3.836\pm0.065$\\
40.1&$-2.542\pm0.053$&$-2.645\pm0.045$&$-2.782\pm0.019$&$-3.547\pm0.038$&$-3.295\pm0.038$&$-4.093\pm0.136$&$-4.145\pm0.030$\\
40.3&$-2.677\pm0.044$&$-2.774\pm0.043$&$-2.919\pm0.016$&$-3.673\pm0.051$&$-3.475\pm0.040$&$-4.000\pm0.067$&$-4.364\pm0.034$\\
40.5&$-2.785\pm0.053$&$-2.873\pm0.049$&$-3.031\pm0.003$&$-3.867\pm0.035$&$-3.619\pm0.019$&$-3.796\pm0.027$&$-4.668\pm0.048$\\
40.7&$-3.022\pm0.037$&$-3.101\pm0.049$&$-3.243\pm0.003$&$-4.202\pm0.049$&$-3.951\pm0.038$&$-3.960\pm0.144$&$-4.901\pm0.056$\\
40.9&$-3.158\pm0.043$&$-3.207\pm0.039$&$-3.339\pm0.016$&$-4.516\pm0.035$&$-4.189\pm0.036$&$-4.039\pm0.107$&$-5.307\pm0.020$\\
41.1&$-3.365\pm0.024$&$-3.406\pm0.049$&$-3.531\pm0.018$&$-4.894\pm0.038$&$-4.459\pm0.041$&$-4.246\pm0.125$&$-5.747\pm0.032$\\
41.3&$-3.565\pm0.021$&$-3.601\pm0.044$&$-3.755\pm0.009$&$-5.226\pm0.041$&$-4.711\pm0.034$&$-4.283\pm0.170$&$-6.417\pm0.077$\\
41.5&$-3.746\pm0.039$&$-3.793\pm0.029$&$-3.939\pm0.033$&$-5.698\pm0.055$&$-5.008\pm0.035$&$-4.381\pm0.159$&$-5.720\pm0.118$\\
41.7&$-3.964\pm0.016$&$-4.001\pm0.063$&$-4.129\pm0.035$&$-6.300\pm0.052$&$-5.402\pm0.030$&$-4.600\pm0.159$&$-7.247\pm0.146$\\
41.9&$-4.240\pm0.046$&$-4.285\pm0.037$&$-4.432\pm0.031$&$-6.661\pm0.080$&$-5.661\pm0.011$&$-4.795\pm0.147$&$-$\\
42.1&$-4.446\pm0.050$&$-4.466\pm0.051$&$-4.621\pm0.024$&$-7.209\pm0.115$&$-5.950\pm0.061$&$-5.019\pm0.143$&$-$\\
42.3&$-4.683\pm0.043$&$-4.702\pm0.056$&$-4.818\pm0.020$&$-7.570\pm0.022$&$-6.560\pm0.065$&$-5.357\pm0.093$&$-$\\
42.5&$-5.013\pm0.074$&$-5.034\pm0.024$&$-5.178\pm0.057$&$-$&$-6.502\pm0.112$&$-5.631\pm0.136$&$-$\\
42.7&$-5.366\pm0.097$&$-5.380\pm0.051$&$-5.498\pm0.019$&$-$&$-7.041\pm0.120$&$-6.039\pm0.061$&$-$\\
42.9&$-5.627\pm0.008$&$-5.635\pm0.038$&$-5.788\pm0.021$&$-$&$-7.220\pm0.086$&$-6.276\pm0.176$&$-$\\
43.1&$-5.945\pm0.052$&$-5.945\pm0.056$&$-6.122\pm0.036$&$-$&$-$&$-6.458\pm0.144$&$-$\\
43.3&$-6.529\pm0.123$&$-6.555\pm0.083$&$-6.750\pm0.102$&$-$&$-$&$-6.928\pm0.217$&$-$\\
43.5&$-7.348\pm0.217$&$-7.348\pm0.076$&$-7.348\pm0.147$&$-$&$-$&$-$&$-$\\
\bottomrule
  \end{tabular}
   \caption{\OII and \OIII luminosity functions of the \mainsel sample and their different components. All the luminosities $L$ are in units of $\rm{s^{-1}erg}$ and the LFs $\Phi(L)$ are in ${\rm Mpc^{-3}dex^{-1}}$.} 
 \label{tab:LFtable2}
\end{table*}

 


  \begin{table*}
  \setlength{\tabcolsep}{2pt}
  \centering
 \begin{tabular}{c c c c c c c c}
    \toprule
    &\multicolumn{7}{c}{$\log\Phi(L_{\rm [N_{II}]}$)}\\
{$\log{L_{\rm [N_{II}]}}$}&Full sample&SF sSFR&SF BPT+WHAN&LINERs&Composite&Seyfert&Passive\\ 
\hline
39.1&$-2.331\pm0.032$&$-2.494\pm0.088$&$-2.551\pm0.115$&$-3.113\pm0.010$&$-3.358\pm0.161$&$-$&$-3.590\pm0.051$\\
39.3&$-2.371\pm0.083$&$-2.433\pm0.049$&$-2.502\pm0.049$&$-3.415\pm0.097$&$-3.382\pm0.549$&$-4.839\pm0.217$&$-3.614\pm0.047$\\
39.5&$-2.193\pm0.045$&$-2.343\pm0.020$&$-2.430\pm0.034$&$-3.051\pm0.055$&$-3.016\pm0.061$&$-4.623\pm0.145$&$-3.463\pm0.083$\\
39.7&$-2.116\pm0.042$&$-2.248\pm0.040$&$-2.416\pm0.008$&$-2.994\pm0.039$&$-2.773\pm0.044$&$-3.996\pm0.145$&$-3.610\pm0.043$\\
39.9&$-2.212\pm0.007$&$-2.354\pm0.035$&$-2.517\pm0.094$&$-3.106\pm0.031$&$-2.867\pm0.037$&$-3.856\pm0.068$&$-3.617\pm0.100$\\
40.1&$-2.228\pm0.063$&$-2.323\pm0.040$&$-2.446\pm0.021$&$-3.264\pm0.015$&$-2.925\pm0.042$&$-3.958\pm0.193$&$-3.842\pm0.039$\\
40.3&$-2.312\pm0.030$&$-2.390\pm0.037$&$-2.559\pm0.024$&$-3.411\pm0.052$&$-2.937\pm0.038$&$-3.808\pm0.069$&$-4.177\pm0.048$\\
40.5&$-2.474\pm0.046$&$-2.547\pm0.044$&$-2.685\pm0.016$&$-3.602\pm0.051$&$-3.167\pm0.034$&$-3.952\pm0.144$&$-4.407\pm0.050$\\
40.7&$-2.613\pm0.039$&$-2.672\pm0.040$&$-2.826\pm0.025$&$-3.803\pm0.030$&$-3.294\pm0.038$&$-3.903\pm0.149$&$-4.634\pm0.046$\\
40.9&$-2.833\pm0.047$&$-2.891\pm0.044$&$-3.043\pm0.011$&$-4.017\pm0.051$&$-3.524\pm0.041$&$-4.163\pm0.124$&$-5.023\pm0.020$\\
41.1&$-3.049\pm0.043$&$-3.101\pm0.042$&$-3.255\pm0.022$&$-4.309\pm0.050$&$-3.757\pm0.024$&$-4.236\pm0.148$&$-5.337\pm0.047$\\
41.3&$-3.308\pm0.046$&$-3.353\pm0.039$&$-3.514\pm0.013$&$-4.620\pm0.040$&$-3.994\pm0.037$&$-4.508\pm0.126$&$-5.969\pm0.065$\\
41.5&$-3.579\pm0.034$&$-3.622\pm0.041$&$-3.805\pm0.015$&$-4.962\pm0.042$&$-4.284\pm0.030$&$-4.557\pm0.133$&$-6.389\pm0.067$\\
41.7&$-3.889\pm0.045$&$-3.920\pm0.048$&$-4.136\pm0.005$&$-5.396\pm0.043$&$-4.593\pm0.041$&$-4.727\pm0.158$&$-6.733\pm0.111$\\
41.9&$-4.277\pm0.025$&$-4.305\pm0.041$&$-4.544\pm0.013$&$-5.929\pm0.079$&$-4.909\pm0.041$&$-5.152\pm0.139$&$-7.600\pm0.120$\\
42.1&$-4.612\pm0.054$&$-4.640\pm0.052$&$-4.933\pm0.013$&$-6.256\pm0.089$&$-5.212\pm0.025$&$-5.370\pm0.136$&$-$\\
42.3&$-5.067\pm0.014$&$-5.095\pm0.033$&$-5.435\pm0.014$&$-6.913\pm0.011$&$-5.585\pm0.011$&$-5.818\pm0.124$&$-$\\
42.5&$-5.580\pm0.057$&$-5.594\pm0.033$&$-5.975\pm0.003$&$-7.585\pm0.106$&$-6.112\pm0.065$&$-6.221\pm0.163$&$-$\\
42.7&$-5.981\pm0.075$&$-6.018\pm0.071$&$-6.504\pm0.042$&$-8.106\pm0.204$&$-6.436\pm0.085$&$-6.620\pm0.166$&$-$\\
42.9&$-6.447\pm0.094$&$-6.458\pm0.056$&$-6.767\pm0.048$&$-8.072\pm0.204$&$-6.953\pm0.110$&$-7.179\pm0.019$&$-$\\
43.1&$-7.155\pm0.063$&$-7.155\pm0.131$&$-7.580\pm0.128$&$-$&$-7.579\pm0.113$&$-7.761\pm0.217$&$-$\\
43.3&$-7.391\pm0.119$&$-7.391\pm0.033$&$-7.814\pm0.204$&$-$&$-$&$-7.773\pm0.217$&$-$\\
43.5&$-$&$-$&$-$&$-$&$-$&$-$&$-$\\
  \end{tabular}
 \begin{tabular}{c c c c c c c c}
    \midrule
    &\multicolumn{7}{c}{$\log\Phi (L_{\rm [S_{II}]}$)}\\
{$\log{L_{\rm [S_{II}]}}$}&Full sample&SF sSFR&SF BPT+WHAN&LINERs&Composite&Seyfert&Passive\\ 
\midrule
39.1&$-2.514\pm0.045$&$-2.589\pm0.058$&$-2.750\pm0.049$&$-3.642\pm0.297$&$-3.576\pm0.518$&$-4.790\pm0.217$&$-3.592\pm0.061$\\
39.3&$-2.099\pm0.010$&$-2.260\pm0.072$&$-2.366\pm0.071$&$-2.957\pm0.116$&$-2.922\pm0.071$&$-4.544\pm0.167$&$-3.309\pm0.108$\\
39.5&$-2.101\pm0.071$&$-2.235\pm0.047$&$-2.361\pm0.044$&$-3.008\pm0.149$&$-2.786\pm0.045$&$-4.032\pm0.189$&$-3.556\pm0.049$\\
39.7&$-2.148\pm0.052$&$-2.265\pm0.052$&$-2.427\pm0.020$&$-3.096\pm0.210$&$-2.792\pm0.009$&$-3.802\pm0.149$&$-3.788\pm0.051$\\
39.9&$-2.249\pm0.009$&$-2.368\pm0.071$&$-2.500\pm0.062$&$-3.162\pm0.214$&$-2.912\pm0.047$&$-3.911\pm0.164$&$-3.873\pm0.072$\\
40.1&$-2.258\pm0.037$&$-2.332\pm0.018$&$-2.458\pm0.026$&$-3.386\pm0.180$&$-2.984\pm0.040$&$-3.790\pm0.132$&$-4.079\pm0.050$\\
40.3&$-2.405\pm0.049$&$-2.476\pm0.041$&$-2.583\pm0.022$&$-3.516\pm0.216$&$-3.181\pm0.041$&$-4.047\pm0.105$&$-4.327\pm0.050$\\
40.5&$-2.540\pm0.039$&$-2.598\pm0.037$&$-2.726\pm0.018$&$-3.718\pm0.199$&$-3.303\pm0.040$&$-3.919\pm0.109$&$-4.632\pm0.054$\\
40.7&$-2.701\pm0.048$&$-2.748\pm0.042$&$-2.860\pm0.014$&$-3.948\pm0.144$&$-3.518\pm0.043$&$-4.126\pm0.109$&$-4.991\pm0.042$\\
40.9&$-2.919\pm0.037$&$-2.961\pm0.042$&$-3.067\pm0.011$&$-4.221\pm0.364$&$-3.756\pm0.042$&$-4.325\pm0.135$&$-5.327\pm0.052$\\
41.1&$-3.152\pm0.039$&$-3.190\pm0.043$&$-3.295\pm0.017$&$-4.520\pm0.537$&$-4.013\pm0.042$&$-4.473\pm0.150$&$-5.821\pm0.065$\\
41.3&$-3.413\pm0.047$&$-3.449\pm0.037$&$-3.554\pm0.008$&$-4.831\pm0.537$&$-4.307\pm0.040$&$-4.598\pm0.129$&$-6.320\pm0.046$\\
41.5&$-3.711\pm0.042$&$-3.739\pm0.037$&$-3.848\pm0.015$&$-5.254\pm0.537$&$-4.659\pm0.035$&$-4.801\pm0.145$&$-6.987\pm0.081$\\
41.7&$-4.070\pm0.045$&$-4.090\pm0.041$&$-4.220\pm0.025$&$-5.756\pm0.537$&$-4.916\pm0.048$&$-5.174\pm0.124$&$-7.769\pm0.204$\\
41.9&$-4.469\pm0.034$&$-4.501\pm0.042$&$-4.629\pm0.009$&$-6.218\pm0.537$&$-5.367\pm0.016$&$-5.522\pm0.143$&$-$\\
42.1&$-4.891\pm0.037$&$-4.915\pm0.025$&$-5.060\pm0.031$&$-6.649\pm0.537$&$-5.710\pm0.063$&$-5.900\pm0.117$&$-$\\
42.3&$-5.422\pm0.035$&$-5.451\pm0.049$&$-5.642\pm0.017$&$-7.519\pm0.537$&$-6.151\pm0.025$&$-6.270\pm0.117$&$-$\\
42.5&$-5.926\pm0.024$&$-5.937\pm0.059$&$-6.151\pm0.035$&$-8.106\pm0.537$&$-6.773\pm0.096$&$-6.900\pm0.151$&$-$\\
42.7&$-6.317\pm0.078$&$-6.339\pm0.079$&$-6.764\pm0.059$&$-8.072\pm0.537$&$-6.767\pm0.146$&$-7.247\pm0.217$&$-$\\
42.9&$-6.798\pm0.023$&$-6.870\pm0.080$&$-7.041\pm0.070$&$-$&$-7.727\pm0.109$&$-7.601\pm0.217$&$-$\\
43.1&$-8.039\pm0.217$&$-8.039\pm0.204$&$-$&$-$&$-$&$-8.039\pm0.217$&$-$\\
43.3&$-$&$-$&$-$&$-$&$-$&$-$&$-$\\
43.5&$-$&$-$&$-7.876\pm0.076$&$-$&$-$&$-$&$-$\\
\bottomrule
  \end{tabular}
   \caption{\NII and \SII luminosity functions of the \mainsel sample and their different components. All the luminosities $L$ are in units of $\rm{s^{-1}erg}$ and the LFs $\Phi(L)$ are in ${\rm Mpc^{-3}dex^{-1}}$.}
 \label{tab:LFtable3}
\end{table*}


\section{Other functional forms for the LF fits}
\label{sec:otherLFfits}
Here in Tables\,\ref{tab:fitparrestschechter}--\ref{tab:fitparrest2pl} we present the best-fit parameters of the LF fits using the models beyond Saunders, as described in Sec.~\ref{sec:LFcalculation}. The corresponding results are shown in Figs.\,\ref{fig:LFplotschechter}--\ref{fig:LFplotpowlaw}. 


\begin{table*}
 \setlength{\tabcolsep}{3pt}
  \centering
 \begin{tabular}{l c c c c  }%sig(log10Phi)=sigPhi/(Phi*ln(10))
    \toprule
    \multicolumn{5}{c}{Schechter}\\
    &$\Phi_\star$&$\log{L_\star}$&$\alpha$&$\chi^2_{\rm red}$\\
    \midrule
    &\multicolumn{4}{c}{\Ha}\\
    Full sample&0.14$\pm$0.05&42.41$\pm$0.14& -0.70$\pm$0.05&13.6\\ 
    SF$_{\rm sSFR}$&0.34$\pm$0.01&42.19$\pm$0.19&-0.62$\pm$0.07&10.4\\
    SF$_{\rm BPT+WHAN}$&0.19$\pm$0.06&42.27$\pm$0.15&-0.65$\pm$0.06&18.2\\
    LINERs&0.02$\pm$0.01&41.49$\pm$0.12&-0.99$\pm$0.07&4.0\\
Composite&0.05$\pm$0.02&42.13$\pm$0.21&-0.70$\pm$0.07&12.6\\
    Seyfert&0.03$\pm$0.01&42.00$\pm$0.22&-0.38$\pm$0.08&2.0\\
    Passive&0.02$\pm$0.01&41.08$\pm$0.12&-1.05$\pm$0.11&2.4\\
        \midrule
        &\multicolumn{4}{c}{\Hb}\\
    Full sample&0.04$\pm$0.02&42.11$\pm$0.15&-0.95$\pm$0.06&21.3\\
    SF$_{\rm sSFR}$&0.02$\pm$0.01&42.24$\pm$0.17&-0.98$\pm$0.06&19.6\\
    SF$_{\rm BPT+WHAN}$&0.14$\pm$0.05&41.77$\pm$0.21&-0.79$\pm$0.08&48.2\\
    LINERs&0.02$\pm$0.01&41.09$\pm$0.07&-1.17$\pm$0.06&3.5\\
    Composite&0.02$\pm$0.01&41.87$\pm$0.18&-1.03$\pm$0.06&7.4\\
    Seyfert&0.02$\pm$0.01&41.66$\pm$0.14&-0.62$\pm$0.11&3.1\\
    Passive&0.02$\pm$0.01&40.71$\pm$0.14&-1.08$\pm$0.11&2.4\\
        \midrule
    &\multicolumn{4}{c}{\OII}\\
    Full sample&0.04$\pm$0.01&42.55$\pm$0.19&-0.90$\pm$0.04&5.3\\
    SF$_{\rm sSFR}$&0.04$\pm$0.01&42.57$\pm$0.13&-0.86$\pm$0.04&6.8\\
    SF$_{\rm BPT+WHAN}$&0.11$\pm$0.03&42.25$\pm$0.17&-0.72$\pm$0.04&17.8\\
    LINERs&0.02$\pm$0.01&41.84$\pm$0.12&-1.03$\pm$0.04&4.1\\
    Composite&0.02$\pm$0.01&42.11$\pm$0.19&-1.03$\pm$0.07&8.2\\
    Seyfert&0.02$\pm$0.01&42.25$\pm$0.15&-0.40$\pm$0.23&13.4\\
    Passive&0.02$\pm$0.01&41.31$\pm$0.16&-0.94$\pm$0.10&4.2\\
        \midrule
    &\multicolumn{4}{c}{\OIII}\\
    Full sample&0.02$\pm$0.01&42.75$\pm$0.08&-0.89$\pm$0.02&2.9\\
    SF$_{\rm sSFR}$&0.02$\pm$0.01&42.80$\pm$0.14&-0.87$\pm$0.03&5.5\\
    SF$_{\rm BPT+WHAN}$&0.02$\pm$0.01&42.55$\pm$0.19&-0.85$\pm$0.03&15.5\\
    LINERs&0.02$\pm$0.01&41.41$\pm$0.50&-1.13$\pm$0.21&38.6\\
    Composite&0.02$\pm$0.01&41.79$\pm$0.21&-1.00$\pm$0.07&10.5\\
    Seyfert&0.03$\pm$0.01&42.18$\pm$0.22&-0.31$\pm$0.09&5.7\\
    Passive&0.02$\pm$0.01&40.92$\pm$0.23&-1.05$\pm$0.14&2.9\\
        \midrule
    &\multicolumn{4}{c}{\NII}\\
    Full sample&0.13$\pm$0.05&42.02$\pm$0.15&-0.81$\pm$0.05&12.9\\
    SF$_{\rm sSFR}$&0.11$\pm$0.04&42.01$\pm$0.16&-0.82$\pm$0.07&11.1\\
    SF$_{\rm BPT+WHAN}$&0.04$\pm$0.01&42.06$\pm$0.07&-1.04$\pm$0.05&42.7\\
    LINERs&0.02$\pm$0.01&41.79$\pm$0.07&-1.03$\pm$0.05&5.6\\
    Composite&0.02$\pm$0.01&42.10$\pm$&-0.92$\pm$0.06&9.1\\
    Seyfert&0.02$\pm$0.01&42.26$\pm$0.08&-0.52$\pm$0.06&3.5\\
    Passive&0.02$\pm$0.01&41.14$\pm$0.20&-1.09$\pm$0.12&0.9\\
        \midrule
    &\multicolumn{4}{c}{\SII}\\
    Full sample&0.06$\pm$0.04&42.02$\pm$0.24&-0.94$\pm$0.09&37.4\\
    SF$_{\rm sSFR}$&0.13$\pm$0.05&41.85$\pm$0.18&-0.89$\pm$0.07&9.8\\
    SF$_{\rm BPT+WHAN}$&0.12$\pm$0.04&41.80$\pm$0.14&-0.93$\pm$0.07&33.5\\
    LINERs&0.02$\pm$0.01&41.57$\pm$0.16&-1.11$\pm$0.07&3.6\\
    Composite&0.02$\pm$0.01&41.97$\pm$0.19&-1.08$\pm$0.06&8.5\\
    Seyfert&0.02$\pm$0.01&41.91$\pm$0.19&-0.63$\pm$0.09&1.5\\
    Passive&0.02$\pm$0.01&40.96$\pm$0.11&-1.11$\pm$0.09&0.9\\
\bottomrule
  \end{tabular}
   \caption{Best-fit Schechter parameters to the measured luminosity functions shown in Tables\,\ref{tab:LFtable1} and \ref{tab:LFtable2}. Note that $\Phi_\star$ is given in units of ${\rm 10^3 Mpc^{-3}dex^{-1}}$, and the argument of $\log{L_\star}$ is in units of $\rm s^{-1}\,erg$.} 
 \label{tab:fitparrestschechter}
\end{table*}




\begin{table*}
 \setlength{\tabcolsep}{3pt}
  \centering
 \begin{tabular}{l c c c c c c }%sig(log10Phi)=sigPhi/(Phi*ln(10))
    \toprule
    \multicolumn{7}{c}{Double Schechter}\\
    &$\Phi_1^\star$&$\Phi_2^\star$&$\log{L_\star}$&$\alpha_1$&$\alpha_2$&$\chi^2_{\rm red}$\\
    \midrule
    &\multicolumn{6}{c}{\Ha}\\
    Full sample&0.22$\pm$0.05&0.002$\pm$0.01&42.09$\pm$0.05&-1.55$\pm$0.04&4.46$\pm$0.88&5.6\\ 
    SF$_{\rm sSFR}$&0.22$\pm$0.05&0.003$\pm$0.01&42.06$\pm$0.06&-1.56$\pm$0.06&4.41$\pm$0.99&5.8\\
    SF$_{\rm BPT+WHAN}$&0.19$\pm$0.05&0.002$\pm$0.01&42.06$\pm$0.06&-1.54$\pm$0.06&4.68$\pm$1.02&3.9\\
    LINERs&0.02$\pm$0.01&0.002$\pm$0.01&41.48$\pm$0.25&-1.98$\pm$0.23&-1.93$\pm$0.18&4.8\\
Composite&0.05$\pm$0.02&0.003$\pm$0.02&41.85$\pm$0.09&-1.59$\pm$0.07&3.86$\pm$1.25&7.6\\
    Seyfert&0.02$\pm$0.01&0.02$\pm$0.01&41.89$\pm$0.04&-1.30$\pm$0.05&5.48$\pm$0.99&0.7\\
    Passive&0.02$\pm$0.01&0.02$\pm$0.01&40.89$\pm$0.10&-1.86$\pm$0.15&2.08$\pm$0.76&1.6\\
        \midrule
        &\multicolumn{6}{c}{\Hb}\\
    Full sample&0.08$\pm$0.02&0.02$\pm$0.01&41.74$\pm$0.07&-1.77$\pm$0.05&4.54$\pm$1.06&6.3\\
    SF$_{\rm sSFR}$&0.05$\pm$0.02&0.02$\pm$0.01&41.85$\pm$0.06&-1.80$\pm$0.05&3.95$\pm$1.05&8.3\\
    SF$_{\rm BPT+WHAN}$&0.14$\pm$0.03&0.02$\pm$0.01&41.55$\pm$0.06&-1.67$\pm$0.06&3.38$\pm$0.97&21.1\\
    LINERs&0.02$\pm$0.01&0.02$\pm$0.01&40.96$\pm$1.16&-2.04$\pm$0.89&4.77$\pm$1.56&2.5\\
    Composite&0.02$\pm$0.01&0.02$\pm$0.01&41.58$\pm$0.07&-1.95$\pm$0.05&1.79$\pm$0.56&2.7\\
    Seyfert&0.02$\pm$0.01&0.02$\pm$0.01&41.38$\pm$1.02&-1.50$\pm$0.76&3.61$\pm$1.22&2.8\\
    Passive&0.02$\pm$0.01&0.02$\pm$0.01&40.56$\pm$0.95&-1.82$\pm$0.64&1.96$\pm$0.89&6.5\\
        \midrule
    &\multicolumn{6}{c}{\OII}\\
    Full sample&0.02$\pm$0.01&0.03$\pm$0.01&42.48$\pm$0.06&-1.88$\pm$0.04&3.53$\pm$0.87&4.0\\
    SF$_{\rm sSFR}$&0.03$\pm$0.01&0.02$\pm$0.01&42.42$\pm$0.08&-1.82$\pm$0.04&3.49$\pm$0.67&4.7\\
    SF$_{\rm BPT+WHAN}$&0.05$\pm$0.01&0.03$\pm$0.02&42.24$\pm$0.07&-1.72$\pm$0.04&5.66$\pm$1.32&15.1\\
    LINERs&0.02$\pm$0.01&0.03$\pm$0.01&41.67$\pm$0.02&-1.94$\pm$0.01&2.57$\pm$0.20&0.3\\
    Composite&0.02$\pm$0.01&0.02$\pm$0.01&41.84$\pm$0.07&-1.95$\pm$0.06&2.18$\pm$0.65&3.0\\
    Seyfert&0.02$\pm$0.01&0.02$\pm$0.01&41.85$\pm$0.11&-1.46$\pm$0.15&4.77$\pm$1.80&2.9\\
    Passive&0.03$\pm$0.01&0.02$\pm$0.01&41.08$\pm$0.13&-1.81$\pm$0.16&1.84$\pm$1.01&3.3\\
        \midrule
    &\multicolumn{6}{c}{\OIII}\\
    Full sample&0.02$\pm$0.01&0.02$\pm$0.01&42.43$\pm$0.05&-1.84$\pm$0.02&1.84$\pm$0.37&0.9\\
    SF$_{\rm sSFR}$&0.02$\pm$0.01&0.02$\pm$0.01&42.48$\pm$0.04&-1.80$\pm$0.02&1.98$\pm$0.47&2.1\\
    SF$_{\rm BPT+WHAN}$&0.03$\pm$0.02&0.02$\pm$0.01&42.42$\pm$0.10&-1.83$\pm$0.03&1.68$\pm$0.66&13.1\\
    LINERs&0.02$\pm$0.01&0.02$\pm$0.01&41.24$\pm$0.05&-2.00$\pm$0.06&5.68$\pm$1.11&2.3\\
    Composite&0.02$\pm$0.01&0.02$\pm$0.01&41.42$\pm$0.09&-1.91$\pm$0.07&2.02$\pm$0.83&6.2\\
    Seyfert&0.04$\pm$0.02&0.02$\pm$0.01&41.68$\pm$0.13&-1.06$\pm$0.12&4.89$\pm$1.97&4.1\\
    Passive&0.02$\pm$0.01&0.02$\pm$0.01&40.79$\pm$0.97& -1.85$\pm$0.32&0.96$\pm$0.12&4.8\\
        \midrule
    &\multicolumn{6}{c}{\NII}\\
    Full sample&0.06$\pm$0.02&0.02$\pm$0.01&41.96$\pm$0.08&-1.80$\pm$0.05&5.23$\pm$1.44&11.3\\
    SF$_{\rm sSFR}$&0.03$\pm$0.01&0.02$\pm$0.01&42.15$\pm$0.09&-1.92$\pm$0.08&7.12$\pm$2.43&15.6\\
    SF$_{\rm BPT+WHAN}$&0.10$\pm$0.03&0.02$\pm$0.01&41.67$\pm$0.07&-1.81$\pm$0.07&2.13$\pm$0.60&18.5\\
    LINERs&0.02$\pm$0.01&0.02$\pm$0.01&41.48$\pm$0.05&-1.88$\pm$0.04&3.72$\pm$1.05&1.8\\
    Composite&0.03$\pm$0.01&0.02$\pm$0.01&41.70$\pm$0.10&-1.77$\pm$0.07&1.79$\pm$0.79&6.8\\
    Seyfert&0.02$\pm$0.01&0.02$\pm$0.01&41.89$\pm$0.08&-1.40$\pm$0.07&4.39$\pm$1.28&1.2\\
    Passive&0.02$\pm$0.01&0.02$\pm$0.01&40.88$\pm$0.32&-2.07$\pm$0.38&0.09$\pm$2.01&2.5\\
        \midrule
    &\multicolumn{6}{c}{\SII}\\
    Full sample&0.07$\pm$0.02&0.02$\pm$0.01&41.82$\pm$0.05&-1.81$\pm$0.05&5.22$\pm$0.88&8.5\\
    SF$_{\rm sSFR}$&0.07$\pm$0.02&0.02$\pm$0.01&41.80$\pm$0.06&-1.86$\pm$0.05&5.03$\pm$1.03&4.6\\
    SF$_{\rm BPT+WHAN}$&0.11$\pm$0.02&0.02$\pm$0.01&41.62$\pm$0.04&-1.78$\pm$0.06&3.46$\pm$0.75&12.6\\
    LINERs&0.02$\pm$0.01&0.02$\pm$0.01&41.31$\pm$0.04&-1.92$\pm$0.05&2.61$\pm$0.51&1.1\\
    Composite&0.02$\pm$0.01&0.02$\pm$0.01&41.76$\pm$0.73&-1.99$\pm$0.06&-0.85$\pm$0.24&15.0\\
    Seyfert&0.02$\pm$0.01&0.02$\pm$0.01&41.57$\pm$0.12&-1.41$\pm$0.13&3.06$\pm$1.34&1.2\\
    Passive&0.02$\pm$0.01&0.02$\pm$0.01&40.80$\pm$0.09&-1.83$\pm$0.06&1.12$\pm$0.67&1.2\\
\bottomrule
  \end{tabular}
   \caption{Best-fit double Schechter parameters to the measured luminosity functions in Tables\,\ref{tab:LFtable1} and \ref{tab:LFtable2}. Note that $\Phi_1^\star$ and $\Phi_2^\star$ are in units of ${\rm 10^3 Mpc^{-3}dex^{-1}}$, while the argument of $\log{L_\star}$ is in units of $\rm s^{-1}\,erg$.} 
 \label{tab:fitparrest2schechter}
\end{table*}







\begin{table*}
 \setlength{\tabcolsep}{3pt}
  \centering
 \begin{tabular}{l c c c c c c}%sig(log10Phi)=sigPhi/(Phi*ln(10))
    \toprule
    &\multicolumn{6}{c}{Double power law}\\
    &$\Phi_0$&$\log{L_0}$&$\alpha_0$&$\alpha_1$&$\beta$&$\chi^2_{\rm red}$\\
    \midrule
    &\multicolumn{6}{c}{\Ha}\\
    Full sample&7.50$\pm$0.01&41.78$\pm$0.30&-0.05$\pm$0.03&4.90$\pm$2.45&0.58$\pm$0.19&1.1\\ 
    SF$_{\rm sSFR}$&15.93$\pm$0.03&41.79$\pm$0.36&-0.25$\pm$0.40&6.00$\pm$3.50&0.52$\pm$0.18&0.8\\
    SF$_{\rm BPT+WHAN}$&11.21$\pm$0.98&41.85$\pm$0.19&-0.24$\pm$0.21&6.00$\pm$0.89&0.53$\pm$0.10&1.3\\
    LINERs&0.09$\pm$0.03&41.28$\pm$0.27&0.76$\pm$0.19&2.45$\pm$0.87&1.35$\pm$0.54&1.1\\
    Composite&27.24$\pm$3.13&40.92$\pm$2.73&-0.78$\pm$0.62&6.00$\pm$2.43&0.47$\pm$0.36&3.3\\
    Seyfert&0.11$\pm$0.09&41.67$\pm$0.49&0.30$\pm$0.22&1.06$\pm$0.78&2.38$\pm$1.34&5.9\\
    Passive&0.02$\pm$0.01&41.09$\pm$0.41&1.17$\pm$0.20&2.23$\pm$1.01&2.03$\pm$1.42&1.2\\
        \midrule
        &\multicolumn{6}{c}{\Hb}\\
    Full sample&6.62$\pm$1.56&41.23$\pm$0.45&-0.02$\pm$0.01&5.01$\pm$3.31&0.56$\pm$0.24&1.2\\
    SF$_{\rm sSFR}$&3.44$\pm$1.02&41.59$\pm$0.83&0.08$\pm$0.06&6.00$\pm$1.33&0.53$\pm$0.35&2.4\\
    SF$_{\rm BPT+WHAN}$&14.47$\pm$5.14&41.17$\pm$0.28&-0.28$\pm$0.16&6.00$\pm$2.31&0.50$\pm$0.15&1.2\\
    LINERs&0.02$\pm$0.01&41.43$\pm$2.84&0.86$\pm$0.60&6.00$\pm$4.33&0.97$\pm$1.10&2.5\\
    Composite&0.05$\pm$0.01&41.91$\pm$4.07&0.60$\pm$0.51&6.00$\pm$3.10&0.55$\pm$0.34&2.2\\
    Seyfert&0.02$\pm$0.01&41.99$\pm$5.61&0.41$\pm$0.34&5.60$\pm$0.97&0.81$\pm$0.43&2.8\\
    Passive&0.02$\pm$0.01&41.03$\pm$1.39&0.85$\pm$0.54&5.12$\pm$0.37&1.01$\pm$0.67&2.3\\
        \midrule
    &\multicolumn{6}{c}{\OII}\\
    Full sample&48.64$\pm$3.12&40.97$\pm$2.99&-0.38$\pm$0.26&6.00$\pm$1.22&0.40$\pm$0.28&0.6\\
    SF$_{\rm sSFR}$&1.22$\pm$0.92&41.46$\pm$0.45&0.40$\pm$0.31&1.88$\pm$1.06&0.94$\pm$0.56&2.1\\
    SF$_{\rm BPT+WHAN}$&117.51$\pm$45.32&40.49$\pm$4.01&-0.97$\pm$0.73&6.00$\pm$3.44&0.42$\pm$0.34&2.3\\
    LINERs&0.02$\pm$0.01&41.80$\pm$0.58&0.84$\pm$0.11&3.45$\pm$1.61&0.95$\pm$0.33&0.6\\
    Composite&0.06$\pm$0.04&42.13$\pm$3.30&0.57$\pm$0.49&6.00$\pm$3.44&0.51$\pm$0.46&2.6\\
    Seyfert&0.13$\pm$0.10&41.20$\pm$0.48&0.03$\pm$0.01&1.90$\pm$1.63&0.94$\pm$0.64&0.8\\
    Passive&0.16$\pm$0.09&41.04$\pm$1.06&0.18$\pm$0.13&4.81$\pm$2.67&0.65$\pm$0.46&0.8\\
        \midrule
    &\multicolumn{6}{c}{\OIII}\\
    Full sample&0.02$\pm$0.01&43.16$\pm$4.77&0.67$\pm$0.15&6.0$\pm$1.92&0.46$\pm$0.30&0.6\\
    SF$_{\rm sSFR}$&0.05$\pm$0.03&42.82$\pm$5.13&0.51$\pm$0.33&6.00$\pm$2.71&0.43$\pm$0.31&1.0\\
    SF$_{\rm BPT+WHAN}$&0.02$\pm$0.01&43.15$\pm$13.49&0.66$\pm$0.37&5.99$\pm$1.27&0.45$\pm$0.23&12.3\\
    LINERs&0.18$\pm$0.07&40.77$\pm$0.20&0.87$\pm$0.16&1.51$\pm$0.28&2.21$\pm$1.10&1.4\\
    Composite&3.55$\pm$0.61&40.07$\pm$7.02&-0.17$\pm$0.14&3.25$\pm$1.93&0.58$\pm$0.46&3.1\\
    Seyfert&1.45$\pm$1.15&40.63$\pm$1.98&-1.21$\pm$0.93&2.73$\pm$1.26&0.70$\pm$0.65&1.3\\
    Passive&0.03$\pm$0.02&40.86$\pm$0.42&1.29$\pm$0.20&1.67$\pm$0.56&3.87$\pm$1.53&2.9\\
        \midrule
    &\multicolumn{6}{c}{\NII}\\
    Full sample&14.89$\pm$10.45&41.39$\pm$0.36&-0.17$\pm$0.13&6.00$\pm$0.23&0.52$\pm$0.21&1.1\\
    SF$_{\rm sSFR}$&94.87$\pm$4.58&40.81$\pm$1.27&-0.79$\pm$0.56&6.00$\pm$2.31&0.50$\pm$0.22&1.0\\
    SF$_{\rm BPT+WHAN}$&0.86$\pm$0.78&41.69$\pm$0.48&0.37$\pm$0.28&4.34$\pm$2.48&0.73$\pm$0.28&3.5\\
    LINERs&0.08$\pm$0.04&41.50$\pm$0.46&0.62$\pm$0.20&3.71$\pm$1.83&0.86$\pm$0.32&0.6\\
    Composite&0.40$\pm$0.28&41.87$\pm$1.44&0.27$\pm$0.24&6.00$\pm$4.23&0.49$\pm$0.44&2.0\\
    Seyfert&0.30$\pm$0.26&41.91$\pm$1.47&-0.18$\pm$0.16&6.00$\pm$1.45&0.56$\pm$0.46&0.5\\
    Passive&0.02$\pm$0.01&41.23$\pm$0.21&1.45$\pm$0.06&1.72$\pm$0.39&6.00$\pm$1.66&1.4\\
        \midrule
    &\multicolumn{6}{c}{\SII}\\
    Full sample&20.62$\pm$6.78&41.22$\pm$0.99&-0.26$\pm$0.19&6.00$\pm$1.95&0.54$\pm$0.49&5.1\\
    SF$_{\rm sSFR}$&1.83$\pm$0.96&41.78$\pm$0.75&0.24$\pm$0.21&6.00$\pm$0.49&0.60$\pm$0.33&1.6\\
    SF$_{\rm BPT+WHAN}$&1.41$\pm$1.04&41.66$\pm$0.49&0.24$\pm$0.17&5.10$\pm$2.94&0.68$\pm$0.24&2.2\\
    LINERs&0.07$\pm$0.05&41.57$\pm$1.08&0.58$\pm$0.48&5.14$\pm$4.21&0.73$\pm$0.47&0.9\\
    Composite&1.81$\pm$1.41&41.24$\pm$0.98&0.05$\pm$0.03&6.00$\pm$0.44&0.48$\pm$0.39&1.3\\
    Seyfert&0.07$\pm$0.02&41.42$\pm$0.29&0.29$\pm$0.23&1.84$\pm$0.89&1.29$\pm$0.64&0.4\\
    Passive&0.02$\pm$0.01&41.34$\pm$0.42&0.82$\pm$0.13&6.00$\pm$3.24&0.84$\pm$0.27&0.3\\
\bottomrule
  \end{tabular}
   \caption{Best-fit parameters of our double power law fits to the measured luminosity functions in Tables\,\ref{tab:LFtable1} and \ref{tab:LFtable2}. Note that $\Phi_0$ is given in units of ${\rm 10^3 Mpc^{-3}dex^{-1}}$, and the argument of $\log{L_0}$ is in units of $\rm s^{-1}\,erg$.} 
 \label{tab:fitparrest2pl}
\end{table*}




\begin{figure*}
\centering\vspace{-0.3cm}
    \includegraphics[width=0.45\linewidth]{LFHaschechter.png}\hfill\quad
    \includegraphics[width=0.45\linewidth]{LFHbschechter.png}
    \includegraphics[width=0.45\linewidth]{LFO2schechter.png}\hfill\quad
    \includegraphics[width=0.45\linewidth]{LFO3schechter.png}
    \includegraphics[width=0.45\linewidth]{LFN2schechter.png}\hfill\quad
    \includegraphics[width=0.45\linewidth]{LFS2schechter.png}
    \vspace{-0.2cm}
    \caption{Same results, line and marker styles as Fig.\,\ref{fig:LFplotsaunders}, but modelling the observations using a single Schechter function.}
\label{fig:LFplotschechter}
  \end{figure*}
  \begin{figure*}
\centering\vspace{-0.3cm}
    \includegraphics[width=0.45\linewidth]{LFHadoubleschechter.png}\hfill\quad
    \includegraphics[width=0.45\linewidth]{LFHbdoubleschechter.png}
    \includegraphics[width=0.45\linewidth]{LFO2doubleschechter.png}\hfill\quad
    \includegraphics[width=0.45\linewidth]{LFO3doubleschechter.png}
    \includegraphics[width=0.45\linewidth]{LFN2doubleschechter.png}\hfill\quad
    \includegraphics[width=0.45\linewidth]{LFS2doubleschechter.png}
    \vspace{-0.2cm}
    \caption{Same as previous results, this time modelled using double Schechter fit to the data, which performs better than a single Schechter function, but leads to overfitting issues (see the text for details). The points shown here have the same labels as those in the previous figure.}
\label{fig:LFplot2schechter}
  \end{figure*}
  \begin{figure*}
\centering\vspace{-0.3cm}
    \includegraphics[width=0.45\linewidth]{LFHa.png}\hfill\quad
    \includegraphics[width=0.45\linewidth]{LFHb.png}
    \includegraphics[width=0.45\linewidth]{LFO2.png}\hfill\quad
    \includegraphics[width=0.45\linewidth]{LFO3.png}
    \includegraphics[width=0.45\linewidth]{LFN2.png}\hfill\quad
   \includegraphics[width=0.45\linewidth]{LFS2.png}
    \vspace{-0.2cm}
\caption{Same as previous results, but showing a five-parameter double power law fit to the data. The points shown here have the same labels as those in the previous figures.}
\label{fig:LFplotpowlaw}
  \end{figure*}

\end{appendix}
\end{document}




