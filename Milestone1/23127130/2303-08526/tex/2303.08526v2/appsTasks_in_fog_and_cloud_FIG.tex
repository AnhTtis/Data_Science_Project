% For FOG ENVIRONMENT ONLY

\usepgfplotslibrary{groupplots}

\begin{figure}[t]
% 	\centering
\begin{tikzpicture}
%/*
\pgfplotsset{
    % scaled y ticks = false,
    width=0.23\textwidth,
    height=40mm,
    % axis on top,
    xtick={5,6,7,8,9,10},
    xticklabels={5,6,7,8,9,10},
    x tick label style={rotate=90,anchor=east,font=\tiny},   
    xmin=5,xmax=10,
    ymin=10,ymax=100,
    % xminorticks=true,
    % yminorticks=true,
    % ylabel shift={-1.5em},
    ylabel style={align=center},
    xlabel style={below=-3mm},
    every axis/.append style={font=\scriptsize},
    every axis title/.style={above right,at={(-0.05,1)}},
}     
    \begin{groupplot}[ 
        group style={
            group size=5 by 1,     % vertical sep=10mm,      % horizontal sep=4mm,
        },
    ]
    % ------------------------------------------------
    % Plot [1, 1]
    %-----------
    \nextgroupplot[
            title={(Prioriy-5 tasks)},            
            % yshift=10pt,
            xlabel={(a)},,
            ylabel={\%age of tasks},            
    ]
        \addplot[smooth,mark=o,black] table[x=noofApps,y=fog_p5]{\appTasksInFogAndCloud}; 
        \addplot[smooth,mark=x,red] table[x=noofApps,y=cloud_p5]{\appTasksInFogAndCloud}; 
    % (Relative) Coordinate at top of the first plot
    % \coordinate (c1) at (rel axis cs:0,1);
    %-----------
    % Plot [1, 2]
    %-----------
    \nextgroupplot[
            title={(Prioriy-4 tasks)},
            % yshift=10pt,
            xlabel={(b)},
            % ylabel={}
    ]
        
        \addplot[smooth,mark=o,black] table[x=noofApps,y=fog_p4]{\appTasksInFogAndCloud}; 
        \addplot[smooth,mark=x,red] table[x=noofApps,y=cloud_p4]{\appTasksInFogAndCloud}; 
    % (Relative) Coordinate at top of the second plot
    \coordinate (c2) at (rel axis cs:0,0);% I moved this to the upper right corner
    %-----------
    % Plot [1, 3]
    %-----------
    \nextgroupplot[
            title={(Prioriy-3 tasks)},
            % yshift=10pt,
            xlabel={(c)},
            % ylabel={},
            legend style={legend columns=5,fill=none,draw=black,anchor=center,align=center, font=\scriptsize},
            legend to name=fred
    ]        
        \addplot[smooth,mark=o,black] table[x=noofApps,y=fog_p3]{\appTasksInFogAndCloud}; 
        \addplot[smooth,mark=x,red] table[x=noofApps,y=cloud_p3]{\appTasksInFogAndCloud}; 
    \addlegendentry{Fog};    
    \addlegendentry{MFC};    
    % (Relative) Coordinate at top of the third plot
    \coordinate (c3) at (rel axis cs:0,0);% I moved this to the upper right corner
    %-----------    
    % Plot [1, 4]
    %-----------
    \nextgroupplot[
            % yshift=10pt,
            xlabel={(d)},
            title={(Prioriy-2 tasks)},
    ]             
        \addplot[smooth,mark=o,black] table[x=noofApps,y=fog_p2]{\appTasksInFogAndCloud}; 
        \addplot[smooth,mark=x,red] table[x=noofApps,y=cloud_p2]{\appTasksInFogAndCloud}; 
    % (Relative) Coordinate at top of the forth plot
    % \coordinate (c4) at (rel axis cs:0,1);% I moved this to the upper right corner
    
    %-----------    
    % Plot [1, 4]
    %-----------
    \nextgroupplot[
            % yshift=10pt,
            xlabel={(e)},
            title={(Prioriy-1 tasks)},
    ]             
        \addplot[smooth,mark=o,black] table[x=noofApps,y=fog_p1]{\appTasksInFogAndCloud}; 
        \addplot[smooth,mark=x,red] table[x=noofApps,y=cloud_p1]{\appTasksInFogAndCloud}; 
    % (Relative) Coordinate at top of the forth plot
    % \coordinate (c4) at (rel axis cs:0,1);% I moved this to the upper right corner
    \end{groupplot}
    
    \coordinate (c6) at ($(c2)!1.0!(c3)$);    
    \node[below] at (c6 |- current bounding box.south)
      {\footnotesize Number of Apps (in thousands)};    

    \coordinate (c6) at ($(c2)!1.0!(c3)$);    
    \node[above] at (c6 |- current bounding box.north)
      {\pgfplotslegendfromname{fred}};    
\end{tikzpicture}
\caption{Percentage of tasks (of different priorities) allocated to only FNs and to MFC environments.}\vspace{-4mm}
\label{fig:sim:PerofTskInFogCld}
\end{figure}

