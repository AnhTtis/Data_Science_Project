\documentclass[11pt]{article}
\textwidth=17.2cm
\textheight=22.2cm
\oddsidemargin= -4mm 
\topmargin -15mm
\usepackage{graphicx}
\usepackage{amssymb}
\usepackage{amscd}
\usepackage{mathrsfs}
\usepackage{longtable,lscape}
\usepackage{amsthm}
\usepackage{amsfonts}
\usepackage{amsmath}
\usepackage{bbm}
\usepackage{float}
\usepackage{url}
\usepackage{csquotes}
\usepackage{wrapfig}
\newcommand{\compl}{{\mathbb C}}
\newcommand{\real}{{\mathbb R}}
\usepackage{caption}
\usepackage{lineno}
\usepackage[title]{appendix}
\captionsetup{skip=15pt,margin=45pt,font=small,labelfont=bf}
\begin{document}
\title{\bf Hartman's effect as evidence of quantum non-spatiality
}
\author{Massimiliano Sassoli de Bianchi\vspace{0.5 cm} \\ 
\normalsize\itshape
Center Leo Apostel for Interdisciplinary Studies, \\ \itshape 
Vrije Universiteit Brussel, 1050 Brussels, Belgium\vspace{0.5 cm} \\ 
\normalsize
E-Mails: 
\url{msassoli@vub.be}
}
\date{}
\maketitle
\begin{abstract} 
\noindent We analyze the tunneling phenomenon from the viewpoint of Hartman's effect, showing that one cannot describe a tunneling entity as ``passing through'' the potential barrier, hence, one must renounce viewing it as being permanently present in space. In other words, Hartman's effect appears to be a strong indicator of quantum non-spatiality.
\end{abstract} 
\medskip
{\bf Keywords:} Tunneling, Hartman's effect, Time-delay, Non-spatiality
\\

\noindent Hartman's effect is the theoretical \cite{Hartman1962} and experimental \cite{Enders1992,Enders1993,Spielmann1994,Longhi2001} observation that the time-delay experienced by a quantum entity tunneling through a potential barrier is independent of its width, in the limit where the latter becomes increasingly larger.\footnote{In this article, we assume that the obtained experimental results have essentially confirmed the theoretical predictions, hence, we will not go into the analysis of possible criticisms about what the experiments would have actually shown; see for example \cite{Winful2006,Sokolovski2018} for two critical views.}

If we think of the tunneling process in classical terms, i.e., imagining the tunneling entity like a localized corpuscle, moving from left to right (we limit our discussion to a single dimension of space), then for a sufficiently spatially extended barrier its motion becomes apparently superluminal, in the sense that in order to account for the very large time-advance (i.e., negative time-delay) accumulated with respect to a free reference particle with same incoming speed, it is necessary to attribute to the interacting entity an effective velocity within the barrier that tends to infinity as the barrier width increases. 

More precisely, let $2a$ be the width of the potential barrier corresponding to the region where it has constant height $V_0$, with $2(a+b)$ the length of the overall interval where it has its support, including the two transition regions (see Fig.~\ref{figure1}). Then Hartman's effect is typically expressed as the following asymptotic behavior for the transmission time-delay: 
\begin{equation}
\tau_{\rm tr}=-{2a\over v}\left[1 + {\rm O}\left({1\over a}\right)\right],
\label{Hartman}
\end{equation}
where $v$ is the incoming velocity (here assumed to be positive, since the entity comes from the left). By definition of time-delay, and still reasoning here as if we were dealing with an entity of a corpuscular nature, the time $T_{\rm tr}$ the particle spends within the barrier region is given by the time-delay (\ref{Hartman}) plus the time a free reference particle of same incoming velocity spends in that same region, i.e., $T_{\rm tr}=\tau_{\rm tr}+{2(a+b)\over v}$. Hence, as $a\to\infty$, $T_{\rm tr}$ tends to a constant, which means that the effective average velocity inside the barrier, $v_{\rm eff}={2(a+b)\over T_{\rm tr}}$, tends to infinity as $a\to\infty$.
\begin{figure}[!ht]
\centering
\includegraphics[scale =0.38]{Figure1}
\caption{A potential barrier whose region of constant height $V_0$ has width $2a$, while the left and right transition regions have each width $b$, hence $V(x)$ has its support in the interval $[-a-b,a+b]$, contained in a larger interval $[-R,R]$, $R>a+b$, which is the one considered for the calculation of the time-delay. If the energy $E={1\over 2}mv^2$ of a classical particle coming from the left is strictly below the barrier height, $E<V_0$, it can only reach point $x=-a-s_0$, with $V(-a-s_0)-E=0$, before being reflected back.  
\label{figure1}}
\end{figure} 

The probability for a classical corpuscle of mass $m$ and velocity $v=\sqrt{2E/m}$, with $E<V_0$, to be transmitted through the barrier, is of course zero, hence interpreting the tunneling phenomenon using a classical analogy is not correct, since only quantum entities can be detected on the right hand side of the barrier with a non-zero probability, when the incoming energy $E$ is strictly below the barrier height $V_0$. But even if we renounce viewing the tunneling entity as a corpuscle, the question of the effective velocity of the process remains. How can a quantum entity, impinging on the barrier from the left, reappear almost instantly to its right, so that its effective velocity $v_{\rm eff}$ becomes arbitrarily height? 

We know that superluminal phenomena exist and pose no problems in physics. A famous example is the speed of the shadow of an object \cite{Leblond2006}, which is not limited by any physical principle and therefore has no upper limit. Tunneling also appears to be a superluminal phenomenon, but as for the shadow it cannot be used to produce faster-than-light signaling, considering in particular that the tunneling probability is very small.\footnote{The analysis of the notion of signaling is beyond the scope of this article. Here we limit ourselves to observing that the existence of superluminal (effective) velocities does not necessarily imply the possibility of achieving superluminal signaling; see for instance the discussion in \cite{Dumont2020}.} Nevertheless, to what extent can we say that the tunneling process correspond to a propagation through space at an effective speed greater than the speed of light?

One of the main criticisms in attaching the tunneling entity a superluminal velocity is that there would be no uncontroversial notion of tunneling time \cite{Hauge1989,Muga2008,Sassoli2012}, hence one cannot meaningfully talk of the tunneling entity's speed in the barrier region. In this regard, we can observe that there is no intrinsic arrival time operator in quantum mechanics \cite{Allcock1969,Muga2008}. However, the notion of sojourn time (also called dwell time), defined in terms of probabilities of presence, remains valid also in quantum mechanics, i.e., can be associated with a bona fide self-adjoint operator \cite{Jaworski1989}. But to associate a time-delay to a tunneling entity, one needs a more specific notion of conditional time-delay, which in turn requires a notion of conditional sojourn time, that is, a time of sojourn conditional to the fact that the entity is ultimately observed in a given state (in our case, a transmitted state). 

Unfortunately, one cannot define a quantum conditional sojourn time operator, providing a meaningful answer to the question of how much time, on average, an entity has passed inside a given region of space, say a region of radius $R$ incorporating the potential barrier (see Fig.~\ref{figure1}), conditional to the fact that it is ultimately observed in a given state. This because it is a composite question requiring to jointly answer two incompatible questions: one about the presence of the entity inside the region of interest, and one about knowing if it will be ultimately transmitted or reflected. Nevertheless, this does not mean that the definition of a meaningful notion of conditional time-delay would not be possible. Indeed, as explained in \cite{Sassoli1993,Sassoli2012}, time-delay is defined as the difference of two sojourn times -- the interaction and free sojourn times -- in the limit where the radius $R$ of the localization region tends to infinity. In this limit, the region covers the entire space, hence the question about the presence of the entity inside of it becomes compatible with the questions of whether it will be transmitted or reflected. Without going into details, this means that an unphysical negative joint probability distribution (associated with two incompatible questions) becomes positive in the time-delay limit $R\to\infty$, and therefore recovers a consistent probabilistic interpretation. As a consequence, contrary to the notion of transmission sojourn time, one can still make sense of a notion of transmission time-delay \cite{Sassoli1993,Sassoli2012,Jaworski1988}, and more generally of angular time-delay \cite{Bolle1976}, compatibly with those approaches that use a direct analysis of the evolution of the wave packets \cite{Hauge1989,Jaworski1988,Sassoli2012}.

It must be said, however, that regardless of the difficulties posed by the theoretical definition of the notion of a quantum time-delay, associated with an entity that is ultimately observed to be transmitted, from an experimental point of view the time-delays associated with the tunneling process are accessible and measured, and it is therefore necessary to be able to explain why they are superluminal in nature, when through classical, or semiclassical, reasoning we associate such processes with an effective spatial velocity $v_{\rm eff}>c$.

That being said, we now want to get to the heart of our analysis, which is to take the notion of quantum (transmission and reflection) time-delay very seriously and use the tunneling phenomenon to extract information about the nature of a quantum entity. Our reasoning uses three ingredients: (1) a careful analysis of the process of a classical entity which is reflected from the barrier; (2) the semiclassical approximation, and (3) the remarkable properties of the quantum $S$-matrix, which will allow us to deduce that what happens for reflection also happen for the transmission (tunneling) process. 

Our starting point is the calculation of the reflection time-delay experienced  by a classical corpuscle interacting with the potential barrier (see Fig.~\ref{figure1}): 
\begin{equation}
\label{potential}
V(x)=V_0
\begin{cases}
1 & |x| < a\\
h(|x|-a) & a\leq |x|\leq a+b
\end{cases}
\end{equation}
where $a>0$, and $h(s)$ is a smooth monotone decreasing function with support in the interval $[0, b]$, $b>0$, with $h(0) = 1$ and $h(b)=0$, describing the switching on and off of the barrier in space. A classical particle of energy $E<V_0$ is always reflected back from the potential barrier. To calculate the associated time-delay, one needs to calculate the difference between the time $T_{\rm re}^{\rm cl}$ it spends inside a region of radius $R>a+b$, and the time $T^{\rm cl}_0={2R\over v}$ spent in that same region by a free reference particle of same energy $E={1\over 2}mv^2$, then take the limit $R\to\infty$, which will of course be trivial here, being the potential compactly supported: 
\begin{equation}
\tau_{\rm re}^{\rm cl}=\lim_{R\to\infty}(T_{\rm re}^{\rm cl} -T^{\rm cl}_0)
\label{time-delay-classical}
\end{equation}
More precisely, a simple calculation gives for  the classical (reflection) time-delay \cite{Narnhofer1981,Sassoli2000}:
\begin{equation}
\tau_{\rm re}^{\rm cl}=\left[2\int_{-b}^{-s_0}{ds\over v(s)}-{2b\over v}\right]+\left[0-{2a\over v}\right]
\label{time-classical}
\end{equation}
where $v(s)=\sqrt{2(E-V_0h(-s))/m}$ and $s_0$ is such that $E-V_0h(s_0)=0$, so that $x=-a-s_0$ is the extreme point reached by the particle inside the potential region, before being reflected back; see Fig.~\ref{figure1}. 

Let us explain the physical content of (\ref{time-classical}). We know that the time spent by the free reference particle inside the potential region is ${2(a+b)\over v}$. The term ${2a\over v}$ describes the time spent in the region where the potential is constant, and the term ${2b\over v}$ the time spent in the two transition regions, where there is a non-zero gradient.\footnote{One can also think of the free reference particle as a particle that is elastically reflected at the origin. In both cases, it will have to cross the variable gradient transition region twice: once going in and once coming out.} Now, since the incoming energy $E$ is below the maximum height $V_0$ of the potential barrier, the particle cannot explore its region of length $2a$ where it is constant, hence, the time it spends inside that region is exactly zero, which is the reason of the $0$-term indicated in the second bracket in (\ref{time-classical}), which quantifies the time-delay produced by the zero-gradient part of the potential. On the other hand, the first bracket in (\ref{time-classical}) is the contribution to the time-delay due to the transition region, which is explored twice by the reflected particle, first when it moves from the left to the right, until it reaches the extreme point of its movement at $x=-a-s_0$, and then when it comes back from where it came, having reversed its motion. Clearly, the expression in the first bracket does not depend on the barrier width $2a$, but only on the detail of the transition region, hence we have the following asymptotic form: 
\begin{equation}
\tau_{\rm re}^{\rm cl}=-{2a\over v}\left[1 + {\rm O}\left({1\over a}\right)\right],
\label{Hartman-ref}
\end{equation}

We can see that (\ref{Hartman-ref}) is identical to (\ref{Hartman}). However, different from (\ref{Hartman}), there is no mystery here in the observed asymptotic behavior. It comes from the second bracket in (\ref{time-classical}), which is a consequence of the fact that the reflected particle never enters the (constant height) barrier region of width $2a$. Note also that it would be meaningless to use (\ref{Hartman-ref}) to speak of an effective velocity of the reflected particle inside the barrier region of width $2a$, as is clear that it never enters such region. 

What is the relevance of the above to the problem of interpreting Hartman's effect in quantum tunneling? As we said, two additional ingredients will allow us to make the connection: the fact that classical time-delays correspond to the semiclassical approximations of their quantum mechanical analogues \cite{Narnhofer1981,Berry1982,Fedoriouk1987,MartinSassoli1994}, and the existence of a remarkable relation between the transmission and reflection time-delays in quantum mechanics, which follows from the unitarity of the scattering matrix $S$ \cite{SassoliDiVentra1995}. Let us start by observing how the latter affects the quantum time-delays. For a one-dimensional problem, since there are only two directions, the scattering matrix is the $2\times 2$ matrix:
\begin{equation}
S=
\left[ \begin{array}{cc}
{\cal T} & {\cal R} \\
{\cal L} & {\cal T} \end{array} \right],
\label{ScatteringMatrix}
\end{equation}
where ${\cal T}$ is the transmission amplitude and ${\cal L}$ and ${\cal R}$ the reflection amplitudes from the left and right, respectively (for a given energy $E$). It is then straightforward to observe that the unitarity relation $S^\dagger S = I$ implies $|{\cal T}|^2 +|{\cal R}|^2 = |{\cal T}|^2 + |L|^2 = 1$ (probability conservation), as well as the relation ${\cal T}^*{\cal R} + {\cal L} ^*{\cal T} =0$, which can be equivalently written (using $|{\cal R}| = |{\cal L} |$) as the phase relation:
\begin{equation}
\alpha_{\cal T}+ {\pi\over 2} ={1\over 2}(\alpha_{\cal L} +\alpha_{\cal R}) \mod\pi.
\label{phase}
\end{equation}
Since the quantum mechanical transmission and reflection time-delays for a given energy $E$, are given by the reduced Planck constant $\hbar$ times the energy derivatives of the phases of the transmission and reflection amplitudes, respectively,  \cite{Eisenbud1948,Wigner1955,Smith1960,Sassoli2012}, i.e., $\tau_{\rm tr}=\hbar {d\alpha_{\cal T}\over dE}$, $\tau_{\rm re}^{\rm left}=\hbar {d\alpha_{\cal L}\over dE}$, $\tau_{\rm re}^{\rm right}=\hbar {d\alpha_{\cal R}\over dE}$, from (\ref{phase}) we obtain the remarkable relation: 
\begin{equation}
\tau_{\rm tr} ={1\over 2}(\tau_{\rm re}^{\rm left} +\tau_{\rm re}^{\rm right}).
\label{time-relation}
\end{equation}
In other words, the transmission time-delay is the arithmetic average of the reflection time-delays from the left and from the right, and since we have considered in our analysis a symmetric potential barrier, (\ref{phase}) simply becomes: 
\begin{equation}
\tau_{\rm tr} = \tau_{\rm re}.
\label{time-relation2}
\end{equation}

Note that (\ref{time-relation}) and (\ref{time-relation2}) are not approximations, but exact equalities, expressing a fundamental difference between quantum mechanics and classical mechanics. Indeed, in classical mechanics, for a given non-zero incoming energy $E$, the particle is always either transmitted or reflected, so relations of the above kind would be meaningless for a classical corpuscle. On the other hand, in quantum mechanics both outcomes, transmission and reflection, are always possible, hence transmission and reflection time-delays can be jointly defined for a same incoming energy $E$, and the unitarity of the evolution process then forces them to be equal, in the sense of (\ref{time-relation}) and (\ref{time-relation2}).

Now comes the last step of our reasoning. As we mentioned already, the classical expression (\ref{Hartman-ref}) also holds for a quantum process, in the (semiclassical) regime where the reflection of the incoming waves at the potential barrier boundaries can be disregarded, which is generally the case if the de Broglie wavelength $\lambda$ of the incoming entity is sufficiently small compared to the size $b$ of the gradient of the potential in the transition regions, i.e., if $\lambda \ll  b$, and this all the more so if the function $h(s)$ describing the transition regions is highly differentiable \cite{Narnhofer1981,Berry1982,Fedoriouk1987,MartinSassoli1994}.

In view of the above, it follows that equations (\ref{time-classical})-(\ref{Hartman-ref}) are good approximations for the quantum mechanical transmission time-delays, and this in itself constitutes a derivation of Hartman's effect valid for potential barriers of general shape, when the barrier's transition regions vary over distances much larger than the de Broglie wavelength of the incoming entity. However, what interests us is not the derivation in itself, but the insight it allows us to gain into the temporal aspects the tunneling phenomenon. 

Hartman's effect (\ref{Hartman}), when viewed from the perspective of a reflected entity (\ref{Hartman-ref}), is just the consequence of the fact that the latter does not penetrate the entire barrier width, but only interacts with its external transition region. Hence, in its race with a free reference entity, it has to travel a much shorter distance (roughly, shorter than $2a$), which explains its considerable time-advance (or negative time-delay). From the exact equality (\ref{time-relation2}), it then follows that, mutatis mutandis, the same must be true for the tunneling entity. Indeed, the transmission and reflection time-delays being necessarily always identical,  the mechanisms that are at their origin must also be always the same, i.e., also the time-advance experienced by the tunneling entity must be due to the fact that, in its race, it  avoids altogether the energetically forbidden region of width $2a$, and this in exactly the same way it is avoided by the reflected entity. Schematically, we thus have the situation depicted in Fig.~\ref{figure2}. 
\begin{figure}[!ht]
\centering
\includegraphics[scale =0.5]{Figure2}
\caption{Unlike the reference entity that evolves freely, the potential region for which $E-V(x)<0$ remains inaccessible both to the reflected and the transmitted (tunneling) entities. Outside of this inaccessible region, the scattering entities interact for the same amount of time, on average, with the transition regions, whether they are ultimately reflected or transmitted. In the first case, they interact twice with the left transition region (assuming they come from the left), in the second case they interact once with both transition regions, which are here identical being the barrier symmetric. For non-symmetric barriers, the same applies if one considers an average over the processes where the entities approach the barrier both from the left and from the right, as per (\ref{time-relation}).
\label{figure2}}
\end{figure} 

At this point, if we take the above analysis seriously, we are faced with a serious interpretational problem. Just as it would be absurd to think of the reflected entity as ``passing through'' the potential barrier, the same applies to the tunneling one. The difficulty with the latter is that since it approaches the potential barrier from the left, then reappears to its right hand side, and in between there is an inaccessible region that is not crossed (otherwise it would affect the resulting time-delay, making it different from the reflection time-delay), and there are no other paths to go from the left to the right of the potential barrier, one might rightly ask: How a spatial entity moving in space can suddenly skip an entire region of it, of arbitrary size, as if it did not exist, and instantly reappear on the other side of it?  What we want to emphasize is that it is one thing to think of an entity that, for unclear reasons, would be able to move through a potential region with an arbitrary effective velocity, and quite another to observe that in its evolution it is as if that region did not exist, as our analysis strongly suggests.

In our view, the conundrum has only one solution: we have to think of a quantum entity not as a spatial entity, permanently present in space, but as a non-spatial entity, only ephemerally present in space. Indeed, a non-spatial entity can in principle actualize its presence in different regions of space, and this independently of the existence of inaccessible regions separating them. Note that the notion of non-spatiality was firstly introduced by Diederik Aerts in the late eighties \cite{Aerts1990} and has been discussed in a number of works (see \cite{Aerts1998,Aerts1999} and the references cited therein), and since then other authors realized the need for its introduction, to fully explain quantum mechanics \cite{Christiaens2003,Kastner2012,Sassoli2021}. In a nutshell, non-spatiality is the observation that \cite{Aerts1999}: ``Reality is not contained within space. Space is a momentaneous crystallization of a theatre for reality where the motions and interactions of the macroscopic material and energetic entities take place. But other entities -- like quantum entities for example -- `take place' outside space, or -- and this would be another way of saying the same thing -- within a space that is not the three-dimensional Euclidean space.'' 

The analysis we have here presented adds to numerous other analyses of quantum phenomena that suggest that our reality would be fundamentally non-spatial  \cite{Sassoli2021}. Just to give two examples, quantum measurements and the Born rule can be explained by introducing ``hidden'' interactions that are genuinely non-spatial \cite{asdb2014}; also, the phenomenon of quantum entanglement, when understood as correlations created from ``hidden'' connections, requires again the latter to be non-spatial in nature \cite{asdb2016}. In this article, we offered what we think is new evidence in favor of the non-spatiality hypothesis, resulting from the analysis of the quantum tunneling phenomenon and so-called Hartman's effect.\footnote{Note that non-spatiality also forces us to update our interpretation of the very notion of sojourn time of a quantum entity in a given region of space, in the sense that a quantum sojourn time shouldn't be understood anymore as an actual time, but as a measure of the total availability of a non-spatial entity in participating to a process of actualization of its spatial localization \cite{Sassoli2012b}.}

We conclude by observing that once we take seriously the notion of non-spatiality (without falling into the trap of assuming that non-spatiality would rhyme with non-reality) an interesting question imposes itself: What would be the nature of a non-spatial entity, and how does spatiality emerge from non-spatiality? A possible answer, which is still a work in progress, is contained in the so-called `conceptuality interpretation of quantum mechanics', and refer the interested reader to \cite{Aertsetal2018} and the references cited therein.

\begin{thebibliography}{}
\setlength{\itemsep}{-.5mm}

\bibitem{Hartman1962} Hartman, T. E. (1962). tunneling of a wave packet. J. Appl. Phys. 33, pp. 3427--33.
\bibitem{Enders1992} Enders, A. \& Nimtz, G. (1992). On superluminal barrier traversal. Journal de Physique I 2, pp. 1693--1698.
\bibitem{Enders1993} Enders, A. \& Nimtz, G. (1993). Evanescent-mode propagation and quantum tunneling. Physical Review E 48, pp. 632--634.
\bibitem{Spielmann1994} Spielmann, C., Szipocs, R., Stingl, A. \& Krausz, F. (1994). Tunneling of optical pulses through photonic band-gaps. Physical Review Letters 73, p. 2308.
\bibitem{Longhi2001} Longhi, S., Marano, M., Laporta, P. and Belmonte, M. (2001). Superluminal optical pulse propagation at 1.5 $\mu$m in periodic fiber Bragg gratings. Physical Review E 64, 055602(R).
\bibitem{Hauge1989} Hauge, E. H. and St\o{}vneng, J. A. (1989). tunneling times: a critical review, Rev. Mod. Phys. 61, pp. 917--936.
\bibitem{Muga2008} Muga, J. \textit{et al.} (Eds.) (2008). ``Time in Quantum Mechanics - Vol. 1 and 2'' Lect. Notes Phys. 734 and 789, Springer, Berlin Heidelberg.
\bibitem{Sassoli2012} Sassoli de Bianchi, M. (2012). Time-delay of classical and quantum scattering processes: a conceptual overview and a general definition. Cent. Eur. J. Phys. 10(2), pp. 282--319. 
\bibitem{Sassoli1993} Sassoli de Bianchi, M. (1993). Conditional time-delay in scattering theory. Helv. Phys. Acta 66, pp. 361--77.
\bibitem{Winful2006} Winful, H. G. (2006). Tunneling time, the Hartman effect, and superluminality: A proposed resolution of an old paradox. PhysicsReports 436, pp. 1--69.
\bibitem{Sokolovski2018} Sokolovski, D. \& Akhmatskaya, E. (2018). No time at the end of the tunnel. Commun Phys 1, 47 (2018). 
\bibitem{Leblond2006} Jean-Marc L\'evy-Leblond (2006). \emph{La Vitesse de l'ombre. Aux limites de la science}. Editions du Seuil. 
 \bibitem{Dumont2020} Dumont, R. S., Rivlin, T. \& Pollak, E. (2020). The relativistic tunneling flight time may be superluminal, but it does not imply superluminal signaling. New J. Phys. 22, 093060.
\bibitem{Allcock1969} Allcock, G. R. (1969). The time of Arrival in Quantum Mechanics. Ann. of Phys. 53, pp. 253--285.
\bibitem{Jaworski1989} Jaworski, W. (1989). The concept of a time-of-sojourn operator and spreading of wave packets. J. Math. Phys. 30, pp. 1505--1514.
\bibitem{Bolle1976} D. Boll\'e, D. \& Osborn, T. A. (1976). Concepts of multiparticle time delay. Phys. Rev. D 13, p. 299. 
\bibitem{Jaworski1988} Jaworski, W. \& Wardlaw, D. (1988). Time delay in tunneling: transmission and reflection time delays. Phys. Rev. A 37, pp. 2843--2854.
\bibitem{Narnhofer1981} Narnhofer, H. \& Thirring, W. (1981). Canonical scattering transformation in classical mechanics. Phys. Rev. A 23, pp. 1688–1697.
\bibitem{Sassoli2000} Sassoli de Bianchi, M. (2000). A simple semiclassical derivation of Hartman’s effect, Eur. J. Phys. 21, pp. L21--L23.
\bibitem{Fedoriouk1987} F\'edoriouk, M. (1987). \emph{M\'ethodes asymptotiques pour les \'equations diff\'erentielles ordinaires lin\'eaires}. Editions Mir, Moscow.
\bibitem{Berry1982} Berry, M. V. (1982). Semiclassically weak reflections above analytic and non-analytic potential barriers. J. Phys. A: Math. Gen. 15, pp. 3693--3704.
\bibitem{MartinSassoli1994} Martin, Ph. A. \& Sassoli de Bianchi, M. (1994). Spin Precession Revisited. Found. of Phys. 24, pp. 1371--1378.
\bibitem{SassoliDiVentra1995} Sassoli de Bianchi, M. \& Di Ventra, M. (1995). On the number of states bound by one-dimensional finite periodic potentials. J. Math. Phys. 36, pp. 1753-1764.
\bibitem{Eisenbud1948} Eisenbud, L. (1948). PhD. thesis, Princeton University (unpublished).
\bibitem{Wigner1955}  Wigner, E. P. (1955). Lower limit for the energy derivative of the scattering phase shift. Phys. Rev. 98, pp. 145--147 (1955).
\bibitem{Smith1960}  Smith, F. T. Lifetime matrix in collision theory. Phys. Rev. 118, pp. 349--356 (1960).
\bibitem{Aerts1990} Aerts, D. (1990). An attempt to imagine parts of the reality of the micro-world. pp. 3–25, In: Mizerski, J. et al (eds.) \emph{Problems in Quantum Physics II; Gdansk '89}. World Scientific Publishing Company, Singapore.
\bibitem{Aerts1998} Aerts, D. (1998). The entity and modern physics: the creation discovery view of reality. In: E. Castellani (eds.) \emph{Interpreting Bodies: Classical and Quantum Objects in Modern Physics}, pp. 223–257. Princeton, Princeton Unversity Press (1998).
\bibitem{Aerts1999} Aerts, D. (1999). The stuff the world is made of: physics and reality. In: D. Aerts, J. Broekaert and E. Mathijs (Eds.), \emph{Einstein meets Magritte: An Interdisciplinary Reflection} (129--183). Dordrecht: Kluwer Academic.
\bibitem{Christiaens2003} Christiaens, W. (2003). Non-Spatiality and EPR-Experiments According to the Creation-Discovery View. Found Phys Lett 16, pp. 379--387.
\bibitem{Kastner2012} Kastner R. E. (2012). The Possibilist Transactional Interpretation and Relativity. \emph{Found. Phys. 42}, pp. 1094--1113 (2012).
\bibitem{Sassoli2021} Sassoli de Bianchi, M. (2021). A non-spatial reality. Foundations of Science 26, pp. 143--170.
\bibitem{Sassoli2012b} Sassoli de Bianchi, M. (2012). From permanence to total availability: a quantum conceptual upgrade. Found. of Sci. 17, pp. 223--244. 
\bibitem{Aertsetal2018} Aerts, D., Sassoli de Bianchi, M., Sozzo, S. \& Veloz, M. (2020). On the Conceptuality interpretation of Quantum and Relativity Theories, \emph{Foundations of Science 25}, pp. 5--54.
\bibitem{asdb2014} Aerts, D. \& Sassoli de Bianchi, M. (2014). The Extended Bloch Representation of Quantum Mechanics and the Hidden-Measurement Solution to the Measurement Problem. Annals of Physics 351, pp. 975--1025.
\bibitem{asdb2016} Aerts, D. \& Sassoli de Bianchi, M. (2016). The Extended Bloch Representation of Quantum Mechanics. Explaining Superposition, Interference and Entanglement. Journal Mathematical Physics 57, 122110. 

\end{thebibliography}
\end{document}
\endinput













