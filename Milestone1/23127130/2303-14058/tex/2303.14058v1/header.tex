\pdfinfoomitdate1
\pdfsuppressptexinfo-1
\pdftrailerid{}
\usepackage[utf8]{inputenc}
\usepackage[T1]{fontenc}
\usepackage{amsmath,amssymb,dsfont}
\numberwithin{equation}{section}
\usepackage{microtype}
\usepackage{graphicx,tikz,pgfplots}
\graphicspath{{images/}}
\pgfplotsset{compat=newest}
\usepackage[hyperref,amsmath,thmmarks]{ntheorem}
\usepackage{aliascnt}
\usepackage[a4paper,centering,bindingoffset=0cm,marginpar=2cm,margin=2.5cm]{geometry}
\usepackage[pagestyles]{titlesec}
\usepackage[font=footnotesize,format=plain,labelfont=sc,textfont=sl,width=0.75\textwidth,labelsep=period]{caption}

%'
%' biblatex
%'
\usepackage[backend=bibtex,maxnames=10,backref=true,hyperref=true,giveninits=true,safeinputenc]{biblatex}
\addbibresource{HofSch23.bib}

\DefineBibliographyStrings{english}{%
	backrefpage = {cited on page},
	backrefpages = {cited on pages},
}

%'
%' writes the title always in quotes.
%'
\DeclareFieldFormat[report]{title}{``#1''}
\DeclareFieldFormat[book]{title}{``#1''}
\AtEveryBibitem{\clearfield{url}}
\AtEveryBibitem{\clearfield{note}}

\titleformat{\section}[block]{\large\sc\filcenter}{\thesection.}{0.5ex}{}[]
\titleformat{\subsection}[runin]{\bf}{\thesubsection.}{0.5ex}{}[.]

%'
%' hyperref
%'
\usepackage[pdftex,colorlinks=true,linkcolor=blue,citecolor=green,urlcolor=blue,bookmarks=true,bookmarksnumbered=true]{hyperref}
\hypersetup
{
    bookmarksnumbered,
    breaklinks,
    pdfborderstyle=,
    colorlinks,
    citecolor=[rgb]{0,.65,0},
    linkcolor=[rgb]{0,0,.5},
    urlcolor=[rgb]{0,0,.5},
    pdfauthor={Authors},%ADAPT
    pdfsubject={Subject},%ADAPT
    pdftitle={Title},%ADAPT
    pdfkeywords={Keywords}%ADAPT
}
\def\sectionautorefname{Section}
\def\subsectionautorefname{Section}

%'
%' defines the pagestyle (ADAPT)
%'
\newpagestyle{headers}
{
	\headrule
	\sethead[\footnotesize\thepage][\footnotesize\sc Authors][]{}{\footnotesize\sc Linear Inverse Problems with Coders}{\footnotesize\thepage}
	\setfoot{}{}{}
}
\pagestyle{headers}

%'
%' default paragraph layout.
%'
\postdisplaypenalty= 1000
\widowpenalty = 1000
\clubpenalty = 1000
\displaywidowpenalty = 1000
\setlength{\parindent}{0pt}
\setlength{\parskip}{1ex}
\renewcommand{\labelenumi}{\textit{(\roman{enumi})}}
\renewcommand{\theenumi}{\textit{(\roman{enumi})}}

%'
%' common mathematical commands.
%'
\newtheorem{lemma}{Lemma}[section]
\def\lemmaautorefname{Lemma}

\newaliascnt{proposition}{lemma}
\newtheorem{proposition}[proposition]{Proposition}
\aliascntresetthe{proposition}
\def\propositionautorefname{Proposition}

\newaliascnt{corollary}{lemma}
\newtheorem{corollary}[corollary]{Corollary}
\aliascntresetthe{corollary}
\def\corollaryautorefname{Corollary}

\newaliascnt{theorem}{lemma}
\newtheorem{theorem}[theorem]{Theorem}
\aliascntresetthe{theorem}
\def\theoremautorefname{Theorem}

\theorembodyfont{\normalfont}
\newaliascnt{definition}{lemma}
\newtheorem{definition}[definition]{Definition}
\aliascntresetthe{definition}
\def\definitionautorefname{Definition}

\newaliascnt{assumption}{lemma}
\newtheorem{assumption}[assumption]{Assumption}
\aliascntresetthe{assumption}
\def\assumptionautorefname{Assumption}

\theorembodyfont{\normalfont}
\newaliascnt{notation}{lemma}
\newtheorem{notation}[notation]{Notation}
\aliascntresetthe{notation}
\def\notationautorefname{Notation}

\theorembodyfont{\normalfont}
\newaliascnt{example}{lemma}
\newtheorem{example}[example]{Example}
\aliascntresetthe{example}
\def\exampleautorefname{Example}

\theorembodyfont{\normalfont}
\newaliascnt{conjecture}{lemma}
\newtheorem{conjecture}[conjecture]{Conjecture}
\aliascntresetthe{conjecture}
\def\conjectureautorefname{Conjecture}

\theorembodyfont{\normalfont}
\newaliascnt{remark}{lemma}
\newtheorem{remark}[remark]{Remark}
\aliascntresetthe{remark}
\def\remarkautorefname{Remark}


\theoremstyle{nonumberplain}
\theoremseparator{:}
\theoremheaderfont{\normalfont\itshape}

%\newtheorem{remark}{Remark}






\theoremsymbol{\ensuremath{\square}}
\newtheorem{proof}{Proof}

\newcommand{\N}{\mathds{N}}
\newcommand{\Z}{\mathds{Z}}
\newcommand{\R}{\mathds{R}}
\newcommand{\C}{\mathds{C}}

% real part
\let\RE\Re
\let\Re=\undefined
\DeclareMathOperator{\Re}{\RE e}
% imaginary part
\let\IM\Im
\let\Im=\undefined
\DeclareMathOperator{\Im}{\IM m}
% support
\DeclareMathOperator{\supp}{supp}
% sign
\DeclareMathOperator{\sign}{sign}
% absolute Value
\newcommand{\abs}[1]{\left|#1\right|}
% norm
\newcommand{\norm}[1]{\left\|#1\right\|}
% set
\newcommand{\set}[1]{\left\{#1\right\}}
% inner Product
\newcommand{\inner}[2]{\left<#1,#2\right>}
% euler number
\newcommand{\e}{\mathrm e}
% imaginary unit (command \i is stored in \ii)
\let\ii\i
\renewcommand{\i}{\mathrm i}
% derivative
\renewcommand{\d}{\,\mathrm d}

%'
%' comments
%'
\newcommand{\commentO}[1]{\textcolor{red}{#1}}
%
%
%
\newcommand{\domain}[1]{\mathcal{D}{(#1)}}
\newcommand{\range}[1]{\mathcal{R}{(#1)}}

\newcommand{\bx}{{\bf{x}}}
\newcommand{\by}{{\bf{y}}}
\newcommand{\bv}{{\bf{v}}}
\newcommand{\bw}{{\bf{w}}}
\newcommand{\bz}{{\bf{z}}}
\newcommand{\bp}{{\bf{p}}}
\newcommand{\bq}{{\bf{q}}}
\newcommand{\bh}{{\bf{h}}}

%
\newcommand{\vx}{{\vec{x}}}
\newcommand{\vp}{{\vec{p}}\,{}}
\newcommand{\vq}{{\vec{q}}\,{}}
\newcommand{\vh}{{\vec{h}}}
\newcommand{\val}{{\vec{\alpha}}}
\newcommand{\vth}{{\vec{\theta}}}

\newcommand{\F}{{\mathbf{F}}}
\newcommand{\be}{{\mathbf{e}}}
\newcommand{\bY}{{\mathbf{Y}}}
\newcommand{\bD}{{\mathbf{D}}}
\newcommand{\bW}{{\mathbf{W}}}
\newcommand{\bX}{{\mathbf{X}}}
\newcommand{\bP}{{\vec{P}}}
\newcommand{\bZ}{{\mathbf{Z}}}
\newcommand{\bE}{{\mathbf{E}}}


\newcommand{\ve}{\varepsilon}

%\usepackage{showkeys}
