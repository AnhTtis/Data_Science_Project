\begin{table*}
    \begin{minipage}[t]{0.19\linewidth}
        \vspace{0pt}
        \centering
        \includegraphics[width=0.98\linewidth]{sec/image/color.pdf}
        \captionsetup{font={small}}
        \figcaption{\textbf{Color contrast.} Our synthetic data shows a similar distribution of {color contrast} with real-world dataset DUTS-TR.}
        \label{fig:color}
    \end{minipage}\hspace{5pt}
    \begin{minipage}[t]{0.19\linewidth}
        \vspace{0pt}
        \centering
        \includegraphics[width=0.98\linewidth]{sec/image/size.pdf}
        \captionsetup{font={small}}
        \figcaption{\textbf{Object size.} 
        Our synthetic data has a broader scale of salient objects (object sizes ranging from $0.1$ to $0.5$).
        }
        \label{fig:size}
    \end{minipage}\hspace{5pt}
    \begin{minipage}[t]{0.37\linewidth}
            \vspace{0pt}
            \includegraphics[width=.48\linewidth]{sec/image/ours_point.pdf}
            \includegraphics[width=.48\linewidth]{sec/image/duts_tr_point.pdf}
            \figcaption{\textbf{Center bias scatter plot for our synthetic dataset (left) and DUTS-TR (right).} DUTS-TR is object-centric, while our dataset is more diverse in center distribution and contains more hard samples.}
            \label{fig:center_bias}
    \end{minipage}\hspace{5pt}
    \begin{minipage}[t]{0.19\linewidth}
        \vspace{0pt}
        \resizebox{.98\textwidth}{!}{
          \begin{tabular}{c|cc}
            \toprule
            &DUTS-TR&Ours\\\midrule
            SC&27.1&29.9\\\midrule
            PL&2.96&2.78\\\midrule
            SD&2.31&1.91\\
            \bottomrule
          \end{tabular}
        }
        \tabcaption{\textbf{Geometry statistics,} in terms of shape complexity (SC), polygon length (PL) and shape diversity (SD). 
        }
        \label{tab:shape}
      \end{minipage}
\end{table*}



