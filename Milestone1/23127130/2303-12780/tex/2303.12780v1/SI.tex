% \documentclass[page-classic]{epl2} 
% \documentclass[12 pt]{article}
\documentclass[a4paper,10pt]{scrartcl}

\usepackage[a4paper,left=2.0cm,right=2.0cm,top=2.5cm]{geometry}
\setlength{\parindent}{0pt} 
\usepackage{enumerate}
\usepackage[shortlabels]{enumitem}

% Users of the {thebibliography} environment or BibTeX should use the
% scicite.sty package, downloadable from *Science* at
% www.sciencemag.org/about/authors/prep/TeX_help/ .
% This package should properly format in-text
% reference calls and reference-list numbers.
%
%\usepackage{scicite}
% \usepackage{times}

\usepackage{hyperref}

% Some standard mathematical notation and figure packages
\usepackage{mathrsfs}
\usepackage{xcolor}
\usepackage{amsmath}
\usepackage{amsfonts}
\usepackage{amssymb}
\usepackage{graphicx}
\usepackage{bm}
\usepackage{braket}
\usepackage{rotating}
\usepackage{geometry}

\usepackage{caption}
\usepackage{subcaption}
\usepackage[normalem]{ulem}
\usepackage{physics}
\usepackage{apacite}
%\usepackage{biblatex}
%\usepackage[numbers]{natbib}
%\bibliographystyle{eplbib}
\renewcommand*{\thefigure}{S\arabic{figure}}

%%%%%%%%%%%%%%%%% END OF PREAMBLE %%%%%%%%%%%%%%%%

\begin{document} 
% Make the title.

% \maketitle 


\noindent
\begin{center}
{\LARGE\bfseries\sffamily Supplementary Material}
\end{center}

\vspace{.5cm}

\section*{\LARGE Active particles with delayed attractions form quaking crystallites}\vspace{.3cm}
\noindent
{\large\bfseries\sffamily Pin-Chuan Chen$^1$, Klaus Kroy$^1$, Frank Cichos$^2$, Xiangzun Wang$^2$ and Viktor Holubec$^{3}$}
\vspace{.25cm}
\begin{enumerate}[label={\arabic*}] % \alph
\setlength{\itemsep}{0pt}
%\setlength{\parskip}{0pt}
\item\textit{Institute for Theoretical Physics, Leipzig University, Postfach 100 902, 04009 Leipzig, Germany.}
\\
chen@itp.uni-leipzig.de, klaus.kroy@uni-leipzig.de
\item\textit{Peter Debye Institute for Soft Matter Physics, Molecular Nanophotonics Group, Universit\"at Leipzig, 04103 Leipzig, Germany.}
\\
  cichos@physik.uni-leipzig.de, wangxiangzun@gmail.com
\item\textit{Department of Macromolecular Physics, Faculty of Mathematics and Physics, Charles Univeristy, 18000 Prague, Czech Republic}
\\
  viktor.holubec@mff.cuni.cz
\end{enumerate}

\section*{Abstract}
 The supplementary information contains a figure showing the nominal velocity field for the individual dynamical phases and the description of the supplementary videos 1-10.


% \begin{enumerate}[label=$^*$,leftmargin=*,labelsep=1pt] % \alph
% \setlength{\itemsep}{0pt}
% \item\textit{cichos@physik.uni-leipzig.de}
% %\item\href{mailto:cichos@physik.uni-leipzig.de}{\textit{cichos@physik.uni-leipzig.de}}
% \end{enumerate} 

% \textit{Peter Debye Institute for Soft Matter Physics, Molecular Nanophotonics Group, Universit\"at Leipzig, \\
% Linn\'estr.~5,  04103 Leipzig, Germany.} \medskip

% \textit{$^*$cichos@physik.uni-leipzig.de}

\vspace{0.5cm}

%This document provides supporting information to “Fully Steerable Symmetric Thermoplasmonic Microswimmers”, \textit{Journal} ..., ... (2020), http://dx.doi.org/....

\setcounter{tocdepth}{2}
%\tableofcontents

%\newpage

\section{Nominal velocity fields}

In the first row of Fig.~\ref{fig:my_label}, we show the individual particles' nominal velocities $\mathbf{F}_i$ in the lab frame. The second row of the figure depicts projections of $\mathbf{F}_i$ in the comoving, corotating frame to the radial direction from the system's center of mass:
\begin{equation}
\mathbf{F}_r^i = (\mathbf{F}_i - \dot{\mathbf{R}}_{0}) \cdot \frac{\mathbf{r}_i - \mathbf{R}_{0}}{|\mathbf{r}_i - \mathbf{R}_{0}|},
\end{equation}
where $\mathbf{R}_{0} = \sum_{i=1}^N \mathbf{r}_i/N$.
The third row of Fig.~\ref{fig:my_label} presents the tangential projections corresponding to the radial ones in the second row minus the average rotation of the system. They were calculated as 
\begin{equation}
    \mathbf{F}_\phi^i =\mathbf{F}_\parallel^i - \omega |\mathbf{r}_i - \mathbf{R}_{0}| \frac{\mathbf{F}_\parallel^i }{|\mathbf{F}_\parallel^i |},
\end{equation}
where $\mathbf{F}_\parallel^i = (\mathbf{F}_i - \dot{\mathbf{R}}_{0}) - \mathbf{F}_r^i$.


\newgeometry{left=1.6cm,bottom=0.4cm} 

    \begin{sidewaysfigure}
        \centering
        \includegraphics[width=1.1\textwidth]{SI/SI_Fig1_v2.pdf}
        \caption{1st row:  nominal (intended) velocities $\mathbf{F}_i(t)$ of the individual particles in the lab frame. The whole cluster rotates in the direction of the arrows. 2nd and 3rd row: nominal velocities in the co-moving, co-spinning frame projected on the radial and tangential directions, respectively. The colors mark the directions of the arrows (red-radial from the center of mass (COM), green-radial towards the COM, yellow-clockwise rotation around COM, and blue-counter-clockwise rotation around COM). The green circles depict the optimal single particle radius $2\delta t/\pi$. The black disc indicates the fixed target particle. Delay times $\delta t$ corresponding to the individual columns are 5.9, 7, 7.9, 8.9, and 16.8, respectively. $N=200$. The averaged values of $F_r$ and $F_phi$ (2nd and 3rd row) as a function of distance to the COM are shown in the last two rows in the main text Fig. 3. In the non-rotating phase, which is not shown, the nominal velocities of all particles point to the center, as in the 1st panel of the second row.}
        \label{fig:my_label}
    \end{sidewaysfigure}

    \newpage
    
    \restoregeometry

    \section{Supplementary videos}
    The particle colors in the videos code for the number of their nearest neighbors (from 0 to 6: deep blue, purple, steel blue, sky blue, aquamarine, orange, and yellow). The shear bands are marked with red dots. The arrows indicate the actual velocities of the particles in the co-moving, co-rotating frame. The black triangle depicts the center of mass of the system.

To make the shear bands better visible, videos 1-6 and 8-9 were made with zero noise ($D = 0$). Videos 7 and 10 show that nonzero noise ($D = 0.0136$) makes the dynamics of the system more erratic without changing its qualitative features. Videos 1-7 were recorded after the system reached a steady state. Videos 8-10 show the whole time
evolution of the system from the initial condition. In all the videos, we show $N=199$ particles, corresponding to $\rho \approx 7.43$. Except for the last three videos, all videos are sped up 3 times.
% all videos are played in the dimensionless time of the simulations. 
    
        \begin{enumerate}
        \item $\delta t=5.9$, phase II: the spinning crystallite ($D = 0$). %The crystal rotates with a uniform angular speed around the target particle. The arrows point exactly at the center of mass, as the tangential part of the velocity has been subtracted (velocties are in the comoving corotating frame).

        \item $\delta t= 7$, phase III: the quaking crystallite with tangential shear bands ($D = 0$).        
        
        % The bulk of the crystal keeps rotating uniformly, whereas the rim cannot keep up with the bulk and lags behind, creating a tangential shear band (red dots in line), which travels counter-clockwise (same as the rotation). When shearing is present, the color of the particles on its both sides changes, indicating that the close-packed structure is disturbed, and the actual velocities point in opposite directions. 

        

        \item $\delta t=7.1$, phase III: the quaking crystallite with tangential and radial shear bands ($D = 0$).%: Radial shearing become present in the bulk, roughly dividing the crystal into three even parts. Meanwhile, the rim still lags behind the bulk, and exhibits tangential shear as in $\delta t=7$.
        
        \item $\delta t=7.9$, phase IV: the ring ($D = 0$).%: The crystal expands and forms a ring around the target particles.
        \item $\delta t=8.9$, phase V: the yin-yang/blobs ($D = 0$).% The crystal breaks up into two roughly even blobs, forming a Yin-Yang structure. The orbits of the two parts are too close to the center target and thus they cannot form a larger blob like in $\delta t=16.8$ (the single blob structure will be destroyed by the center target if ever formed).
        \item $\delta t=16.8$, phase VI: the satellite ($D = 0$).% Only one single blob orbits around the center target. Once formed, the blob remains stable. Multiple shearings occur simultaneously within the bulk, and the actual velocity directions of the same domain (divided by the shear bands) align. 
        \item $\delta t=8.9$, phase V: the yin-yang/blobs ($D = 0.0136$). 
        %with nonzero noise ($D=0.0136$). The Yin-yang structure remains stable under noise, but the band shearing structure is no longer easily visible as in Video 5, and the colors and velocity arrows are destroyed due to the noise.
        \item $\delta t=8.9$, phase V: the yin-yang/blobs. Typical relaxation trajectory to the yin-yang phase from a random initial condition with $D = 0$. The video is sped up 30 times.
        \item $\delta t=8.9$, phase V: the yin-yang/blobs. Another possible relaxation path to the yin-yang phase from a random initial condition with $D = 0$. The video is sped up 30 times. 
        \item $\delta t=8.9$, phase V: the yin-yang/blobs. Typical relaxation path to the yin-yang phase from a random initial condition with nonzero noise intensity $D = 0.0136$. The video is sped up 30 times.
    \end{enumerate}

    

% \bibliography{Reference} 

    
%\newpage
%\bibliographystyle{apacite}
%\bibliographystyle{apalike}
%\bibliography{My Library2}
%\bibliography{Reference}
\end{document}






