%\chapter{Introduction} % Main chapter title

%\label{Chapter1} 

%----------------------------------------------------------------------------------------
\thispagestyle{plain}
\section{Background}
The last few decades saw the advent and flourishing of
the field of quantum information and computation.\ One of the most important classes of discoveries made in this field has to be that of quantum algorithms which provide substantial computational advantages over their classical counterparts.\ The most significant ones include 
the Deutsch-Jozsa algorithm \cite{Deutsch,Deutsch_1}, Shor’s factoring algorithm~\cite{Shor,Shor1}, the quantum search algorithms~\cite{Grover,Grover_2,Boyer,Lidar,Shenvi, Kempe} and the quantum simulation algorithms~\cite{manin, Feynman,lloyd,bernien,zhan}.\ 
The advantages of these quantum algorithms are assumed to be derived from the efficient use of quantum coherence and entanglement.\ After Grover's seminal proposal \cite{Grover,Grover_2} of his quantum search algorithm, which has been shown to be a special case of the more general \textit{amplitude amplification} algorithm~\cite{Brassard,amplitude}, an extensive amount of research effort has been directed towards implementing and studying the effects of noise on the efficiency of the algorithm in an actual quantum device.\ The experimental implementation  of the algorithm was first done using nuclear magnetic resonance techniques~\cite{Chuang_NMR}.\ Later on, the efficiency of the Grover's algorithm was studied in~\cite{Zalka} and a generalization of the algorithm for an arbitrary amplitude distribution was done in~\cite{Lidar}.\ For more works on the quantum search algorithm, see~\cite{abram,GuiLu, Kwiat,Sheng,Long,Biham3,Heinrich,Roland,Xiao} and for experimental implementations, see~\cite{Jones,Vander,Ermakov,Bhattacharya,Zhang, WaltherP,Brickman,DiCarlo,Figg}.\

\subsubsection{\textit{Grover's search on a noisy register}}
Although theoretically more efficient in comparison to its classical counterpart, an actual implementation of a quantum algorithm critically depends on the error-proof fabrication of the relevant quantum register.\ Therefore, studies on the effect of such distortions from the ideal situation, caused by decoherence and noise is important to assess the usefulness and applicability of an algorithm.\ See e.g.~\cite{Bernstein,Preskill,N&C,Barnes}.\
The effect of noise on the Grover's search algorithm
was studied in~\cite{Altaba}, which investigated 
the effect of random Gaussian noise on the algorithm's efficiency at each step.\
A perturbative method was used 
in~\cite{Azuma} to study decoherence in a noisy Grover algorithm where each qubit suffers phase-flip error independently after each step.\ 
The effect of a noisy oracle was considered in~\cite{Tu, Bae}.\ The effect of unitary noise was considered 
in~\cite{Biham} using a noisy Hadamard gate, with unbiased and isotropic noise, uncorrelated in each iteration of the Grover operators.\ An upper bound on the strength of the noise parameters up to which the algorithm works efficiently was deduced.\ 
A comparison of the effects of several completely positive trace preserving maps in the Kraus form on the efficiency and computational complexity of the algorithm was described in~\cite{Gawron}.\ The performance of the algorithm under localized dephasing was studied in~\cite{Reitzner}.\ For more discussions and further ramifications of the effect of noise on the Grover search algorithm, see~\cite{Salas,  Hasegawa, Cohn}.

\thispagestyle{plain}

\subsubsection{\textit{Non-Markovianity and memory}}
Although the Markovian treatment of open systems has been immensely successful in explaining many physical situations, it is also found that often the quantum processes at hand do not satisfy the strict conditions of Markovianity, such as the Born-Markov or the weak coupling approximation.\ With the advent of the field of quantum information and computation, quantum processors need to be made that probably require tightly-packed qubits or a long coherence time.\ These features will lead to spatio-temporal correlations in the noise due to coupling to some environmental degrees of freedom, since making a quantum device completely devoid of noise is almost unrealistic.\ A qunatum process that do not satisfy Markovianity is called non-Markovian.\ Although several approaches \cite{Guo} to define the boundary between Markovianity and non-Markovianity exist, two major directions are based on - one, CP-divisibility of the map (e.g., work by Rivas et al. \cite{RHP}), and two, information flow in and out of the system (e.g., work by Breuer et al. \cite{Breuer_Laine}).\ We will contrast these two measures in assessing the non-Markovianity in our model.
\subsubsection{\textit{Objectives of the project}}
The main objectives are as follows:
\begin{itemize}
    \item To investigate the effect of noise with memory on the efficiency of a quantum algorithm such as the Grover's quantum search.
    \item To detect non-Markovianity in our noise model by following information dynamics and CP-divisiblity of the system dynamics, and, to find out how they are affected by thermal effects.
\end{itemize}

\thispagestyle{plain}

\section{Outline}
The thesis is organised in three Parts.

\textbf{Part-I} reviews the basic concepts to be used through-out the thesis. This part is divided into two chapters.
\medskip\\
In Chapter \ref{Chapter2}, the notion of open quantum systems is introduced along with methods to describe their evolution.\ Some measures of (non-)Markovianity are presented.\ The concept of a quantum channel with memory is introduced along with an example.
\medskip\\
In Chapter \ref{Chapter3}, we briefly describe how some existing quantum algorithms can surpass classical ones and then introduce the Grover's quantum search algorithm along with a framework for analyzing its success probability.

\textbf{Part-II} analyzes the situation when the register performing Grover's algorithm is an open system. This part is divided into two chapters.
\medskip\\
In Chapter \ref{Chapter4}, we introduce our model 
of noise and analytically find the unitaries representing ``good'' noise, i.e., the noises for which the algorithm's success becomes invariant with respect to the number of noise sites.\ The effects of a memory-less noise and a Markovian-correlated noise are then compared, showing that memory in noise may improve the algorithm's efficiency.
\medskip\\
In Chapter \ref{Chapter5}, a `collisional model' is introduced that exactly reproduces the time evolution of our noisy system.\ We show that back-flow of information from the environment into the system happens for a subspace of all parameter values, but the process still remains non-Markovian for most parameter values, even when the back-flow is absent.\ We then introduce an elementary model of a thermal bath and find that increasing temperature leads to increasing information drainage, i.e., decreasing non-Markovianity of the process.
\thispagestyle{plain}

\textbf{Part-III} concludes the thesis with a summary of the outcomes of the project.
%\vspace{2cm}

\section{Publication}
\textbf{S.\ P.\ Mandal}, A.\ Ghoshal, C.\ Srivastava, and U.\ Sen, “Invariance of success probability in Grover's quantum search under local noise with memory,” (2023),  \\\href{https://doi.org/10.1103/PhysRevA.107.022427}{\textbf{https://doi.org/10.1103/PhysRevA.107.022427}}.\thispagestyle{plain}

