\chapter{Non-Markovianity \& Thermal effects} % Main chapter title
\textit{In this chapter we introduce a collisional model \cite{ciccarello,Esposito,Campbell,Salvatore,Bodor,Man} that exactly reproduces the time evolution of our system under a noise with memory (analyzed in Chapter-3).\ The unitary governing the collisions is decomposed to point out the origin of the conditional probabilities that characterize our Markovian-correlated noise.\ We define a measure of CP-divisibility for the discrete-time evolution and show that although the system's evolution is non-Markovian for most of the parameter space, the back-flow of information from the environment happens only within a subspace of all $\{p,\mu\}$ values.\ By introducing an elementary effect of temperature in the model, we find that increasing temperature leads to increasing information drainage, i.e., decreasing non-Markovianity of the process.  }
\label{Chapter5}
\section{Noisy evolution as a sequence of collisions}
\subsection{Kraus representation}
It turns out that the evolution of the noisy register as described in Section-\ref{Example} has an alternate representation.\ Starting with the same initial (walker+system) state $R_0 = \left(\frac{|g\rangle+| g'\rangle}{\sqrt{2}}\right)\left(\frac{\langle g|+\langle g'|}{\sqrt{2}}\right) \otimes |s\rangle\langle s|$, the following Kraus decomposition can be given for the evolution from $R_0$ to $R_1$
\begin{gather}
    R_1 = \sum_{\alpha=1}^{4}\ K_{\alpha}\ R_0\  K_{\alpha}^{\dagger}\label{eq:K_R1}\\ 
    K_1 =  \sqrt{p_g} \begin{psmallmatrix} G & 0\\ 0 & 0\end{psmallmatrix}, K_2 =  \sqrt{p_g} \begin{psmallmatrix} 0 & G\\ 0 & 0\end{psmallmatrix}, K_3 =  \sqrt{p_{g'}} \begin{psmallmatrix} 0 & 0\\ G' & 0\end{psmallmatrix}, K_4 =  \sqrt{p_{g'}} \begin{psmallmatrix} 0 & 0\\ 0 & G'\end{psmallmatrix}.
\end{gather}
Thus the evolution from $t=0$ to $t=1$ is given by a CP map that is also TP since $\sum_{\alpha}\ K_{\alpha}^{\dagger}\ K_{\alpha} = \mathds{1}$, but \textit{non-unital}, $\sum_{\alpha}\ K_{\alpha}\  K_{\alpha}^{\dagger} \neq \mathds{1}$. This process takes the system state to $\rho_1 = $ $\text{Tr}_{walker}\{R_1\}$ $=p_{g'}\Phi^0[\rho_0]+p_{g}\Phi^1[\rho_0]$ which is the same as we obtained before.\\
The (walker+system) states for $t\geq 2$ can be obtained likewise
\begin{gather}
    R_t = \sum_{\alpha=1}^{4}\ K_{\alpha}\ R_{t-1}\  K_{\alpha}^{\dagger} \label{eq:K_Rt}\\
    K_1 =  \sqrt{p_{g|g}} \begin{psmallmatrix} G & 0\\ 0 & 0\end{psmallmatrix}, K_2 =  \sqrt{p_{g|g'}} \begin{psmallmatrix} 0 & G\\ 0 & 0\end{psmallmatrix}, K_3 =  \sqrt{p_{g'|g}} \begin{psmallmatrix} 0 & 0\\ G' & 0\end{psmallmatrix}, K_4 =  \sqrt{p_{g'|g'}} \begin{psmallmatrix} 0 & 0\\ 0 & G'\end{psmallmatrix}.
\end{gather}
where the $p_{k|l}$ are the conditional probabilities (see Section~\ref{app:Markovian})
 \begin{numcases}{p_{k|l}=}
    (1-\mu)\,p_l + \mu, & \text{for} \quad $k=l$ \label{eq:8_2}\\
    (1-\mu)\,p_{k}, & \text{for} \quad $k \neq l$.\label{eq:18_2}
    \end{numcases}
\subsection{Unitary dilation}
Given the Kraus operators in Eqs.\ \eqref{eq:K_R1} and \eqref{eq:K_Rt}, a suitable environment can be set up for the (walker+system)'s open dynamics.\ The environment in our case turns out to be 2 two-level systems denoted `molecules' or \textit{ancillas} so that the (ancillas+walker+system) together evolves unitarily during the collisions.\ This process of constructing an unitary operator in an extended Hilbert space that reproduces the reduced dynamics of the system of interest, is referred to in the literature as \textit{unitary dilation} \cite{N&C,Preskill,vom_Ende}.\\
If we consider the ancilla systems to be in the pure state $\rho_e = |00\rangle\langle00|$ at the beginning of each collision with the (system+walker), the unitary collision operator is constructed so that the Kraus operators are given as 
\begin{equation}
    K_{\alpha} = \langle \alpha |U|00\rangle
    \label{eq:kraus_uni}
\end{equation}
where $\alpha$ corresponds to the ancilla states $\{|00\rangle,|01\rangle,|10\rangle,|11\rangle\}$.\ This ensures, as described in Eqs.\ \eqref{eq:sep_rho} and \eqref{eq:Op_sum}, that 
    $R_t =$ $\text{Tr}_{e}\{U(R_{t-1}\otimes \rho_e)U^\dagger\} =$ $\sum_{\alpha=1}^{4}\ K_{\alpha}\ R_{t-1}\  K_{\alpha}^{\dagger}$.

The ($2^{n+3}\times 2^{n+3}$) unitary operator $U_0$ governing the collision process corresponding to \eqref{eq:K_R1} and satisfying \eqref{eq:kraus_uni} has been constructed as
\begin{adjustwidth}{-.5cm}{-.7cm}
  \begin{equation}
  \begingroup % keep the change local
\setlength\arraycolsep{-2pt}
  U_0 = \begin{psmallmatrix} 
    \sqrt{p_{g}}G &0&0&0&0 &\sqrt{p_{g'}}G &0 &0\\ 
    0&0&\sqrt{p_{g}}G &0&0 &0 &0&\sqrt{p_{g'}}G \\
    0&\sqrt{p_{g}}G &0&0&\sqrt{p_{g'}}G&0 &0 &0 \\
    0&0&0&\sqrt{p_{g'}}G &0 &0&\sqrt{p_{g}}G &0 \\
    0&0&\sqrt{p_{g'}}G' &0&0 &0 &0&-\sqrt{p_{g}}G'\\
    \sqrt{p_{g'}}G' &0&0&0&0 &-\sqrt{p_{g}}G' &0 &0\\ 
    0&0&0&\sqrt{p_{g}}G' &0 &0&-\sqrt{p_{g'}}G' &0 \\
    0&\sqrt{p_{g'}}G' &0&0&-\sqrt{p_{g}}G'&0 &0 &0 
    \end{psmallmatrix}
    \endgroup
    \label{eq:Unitary_0}
\end{equation}
\end{adjustwidth}

The unitary $U$ corresponding to \eqref{eq:K_Rt}, i.e., for all times $t\geq 2$, can be similarly found to be
\begin{adjustwidth}{-.7cm}{-.7cm}
  \begin{equation}
  \begingroup % keep the change local
\setlength\arraycolsep{-2pt}
  U = \begin{psmallmatrix} 
    \sqrt{p_{g|g}}G &0&0&0&0 &\sqrt{p_{g'|g}}G &0 &0\\ 
    0&0&\sqrt{p_{g|g'}}G &0&0 &0 &0&\sqrt{p_{g'|g'}}G \\
    0&\sqrt{p_{g|g'}}G &0&0&\sqrt{p_{g'|g'}}G&0 &0 &0 \\
    0&0&0&\sqrt{p_{g'|g}}G &0 &0&\sqrt{p_{g|g}}G &0 \\
    0&0&\sqrt{p_{g'|g'}}G' &0&0 &0 &0&-\sqrt{p_{g|g'}}G'\\
    \sqrt{p_{g'|g}}G' &0&0&0&0 &-\sqrt{p_{g|g}}G' &0 &0\\ 
    0&0&0&\sqrt{p_{g|g}}G' &0 &0&-\sqrt{p_{g'|g}}G' &0 \\
    0&\sqrt{p_{g'|g'}}G' &0&0&-\sqrt{p_{g|g'}}G'&0 &0 &0 
    \end{psmallmatrix}
    \endgroup
    \label{eq:Unitary_t}
\end{equation}
\end{adjustwidth}
We can decompose the unitary $U$ into parts that act on different positions of the total (ancillas+walker+system).\ This is shown in Fig.\ \ref{fig:U_decomp}.\ We find that, $U = (\text{controlled}-\bigchi)_{e_1 s}\ (M)_{e_1 e_2 w}\ (G)_{s}$, where $\bigchi G = G'$ as was used in the previous chapter.\ The explicit expression for unitary $M$ is also given in the figure, with $A$ and $B$ being two $4\times 4$ matrices:
\begin{equation}
    A = \begin{psmallmatrix} 
    0 & \frac{\sqrt{p_{g|g}}-1}{\sqrt{p_{g'|g}}} & 0 & 0\\
    \frac{\sqrt{p_{g|g'}}}{\sqrt{p_{g'|g'}}}  & 0 & 0 & \frac{-1}{\sqrt{p_{g'|g'}}}\\
   \frac{-1}{\sqrt{p_{g'|g'}}}  & 0 & 0 & \frac{\sqrt{p_{g|g'}}}{\sqrt{p_{g'|g'}}}\\
   0 & 0 & \frac{-\sqrt{p_{g|g}}}{\sqrt{p_{g'|g}}} & 0
    \end{psmallmatrix},\quad \text{and}, \quad
    B = \begin{psmallmatrix} 
    0 & \frac{\sqrt{p_{g|g}}+1}{\sqrt{p_{g'|g}}} & 0 & 0\\
    \frac{\sqrt{p_{g|g'}}}{\sqrt{p_{g'|g'}}}  & 0 & 0 & \frac{1}{\sqrt{p_{g'|g'}}}\\
   \frac{1}{\sqrt{p_{g'|g'}}}  & 0 & 0 & \frac{\sqrt{p_{g|g'}}}{\sqrt{p_{g'|g'}}}\\
   0 & 0 & \frac{\sqrt{p_{g'|g}}+1}{\sqrt{p_{g|g}}} & 0
    \end{psmallmatrix}
\end{equation}
\subsubsection{\textit{Origin of `memory' in the noise}}
From the decomposition in Fig.\ \ref{fig:U_decomp}, the \textit{origin of `memory'} in the noise can be more clearly understood.\ At each time step, the total unitary on (ancillas+walker+system) is $U$.\ In a given time step, the system (register) first evolves under the noiseless Grover unitary $G$.\ Then, the unitary $M$ acts on (ancillas+walker); thus transforming the state $\rho_e ^1$ of the first ancilla qubit such that it then probabilistically controls the noisy $\bigchi$ evolution of the system.\\ Although the two ancillary qubits $\rho_e$ are `refreshed' at each time step, the memory of the system's noisy evolution $\bigchi$ is then passed on to the subsequent time step's $M$-evolution by the information in the walker's state $\rho_w$.\ This information is used in the next time step's $M$-evolution to evolve the state of $\rho_e ^1$ so that the controlled-$\bigchi$ in that time step is applied on the system according to the conditional probabilities $p_{i|j}$ in Eqs.\ \eqref{eq:8_2} and \eqref{eq:18_2}.
\begin{figure}[h!]
    \centering
    \captionsetup{width=\linewidth}
    \includegraphics[width=.99\linewidth]{Figures/Screenshot_164.png}
    \caption[Decomposition of collision unitary]{The unitary $U$ is shown in an equivalent three-fold decomposition that act on different locations in (ancillas+walker+system).\ The ancillas are denoted here as $\rho_e ^1$ and $\rho_e ^2$.\ The explicit forms of $M$ (gray) and controlled-$\chi$ (orange) are shown in the right.\ }
    \label{fig:U_decomp}
\end{figure}
\begin{figure}[h!]
    \centering
    \captionsetup{width=\linewidth}
    \includegraphics[width=.5\linewidth]{Figures/Screenshot_148.png}
    \caption[Collisional model for the (system + walker) time evolution]{Collisional model for the (system + walker) time evolution, Eq.\ \eqref{eq:K_R1}, \eqref{eq:K_Rt}.\ At each time step, the `ancillas' (composed of two 2-level systems) with state $\rho_e$ collides with the states $\rho_w$ and $\rho_s$.\ After each collision described by the unitary $U$, the ancillas are discarded.\ Memory effects in $\rho_s$ evolution is due to the coupling with $\rho_w$ through-out the evolution.}
    \label{fig:coll_model}
\end{figure}
\subsection{Chain of collisions}
Thus, as opposed to the application of the Grover unitary $G$ at each time step in case of the ideal Grover algorithm, the algorithm under Markovian-correlated noise (Section \ref{Example}) can instead be modeled as evolving under collisions with a walker and two `fresh' ancillas at each time step.\\
Since the time-evolution of the (system+walker) can be expressed in Kraus representation and that it has no initial correlations with the ancillas, imply that the dynamics is CPTP.\ Also, since fresh ancillas collide and evolve the (system+walker) state $R_t$ at each time step, implies the dynamics of $R_t$ is also CP-divisible and given by a \textit{discrete dynamical semigroup}
\begin{equation}
    \Phi(t,t_0)=\Phi(t-s,t_0)\circ \Phi(s,t_0); \quad t\geq s\geq t_0
\end{equation}
similar to the homogeneous composition law in \eqref{eq:semigroup} for Markovian evolution.\ Thus from the statement \ref{CP_marko}, the (system+walker)'s discrete-time dynamics can be treated as \textit{Markovian} \cite{RHP,Breuer_NM_rev}.

\section{Non-Markovian system evolution}
Although the (system+walker) state $R_t$ undergoes Markovian evolution, the reduced dynamics of the system's state $\rho_s=\text{Tr}_w \{R_t\}$ alone is non-Markovian, in general \cite{Jiang}.\ Since the walker is correlated at all times with the system, back and forth exchange of information is possible between them.\ The backflow of information from the walker into the system can then be quantified by the \textit{BLP}-measure of non-Markovianity $\mathcal{N}_{BLP}$ in Eq.\ \eqref{eq:N_BLP}.\ To reiterate, for discrete-time as in our case,
\begin{equation}
    \mathcal{N}_{BLP}(\Phi(t,t_0)) = \underset{\rho^1(t_0),\rho^2(t_0)}{\mathrm{max}} 
    \sum_{ \Delta\, D >0} \Delta\,  D(\rho^1(t),\rho^2(t))
    \label{eq:N_BLP_1}
\end{equation}
where $\Delta\,  D(\rho^1(t),\rho^2(t))$ is the change in trace distance between the time-evolved system states from time $(t-1)$ to $t$, with initial states $\rho^1(0)$ and $\rho^2(0)$.

\subsection{Quantifying information flow into system}
\label{Sec_quant_info}
\begin{figure}[h!]
\centering
    \includegraphics[width=0.86\linewidth]{Figures/Screenshot_156.png}
    \caption[Trace distance as a function of time]{Trace distance $D(\rho^1(t),\rho^2(t))$ is plotted as a function of time $t$.\ Here $\rho^1(0)=|s\rangle\langle s|$ and $\rho^2(0)$ is as in Eq.\ \eqref{eq:rho_2}.\ The subplots are for different noise probabilities $p$ and the colored curves are for different values of the memory parameter $\mu$.\ For small values of $\mu$, $D(\rho^1,\rho^2)$ decreases monotonically, whereas for $\mu = 0.75$ (say), it is not monotonic and indicates `backflow' of information.\ The BLP-measure of non-Markovianity Eq.\ \eqref{eq:N_BLP_1} is then used to make the plots in Fig.\ \ref{fig:NM_mu}.}
\end{figure}
As was shown in Section \ref{sec:N_BLP}, $\mathcal{N}_{BLP}>0$ implies there is information flow from the environment to the system.\ The measure, Eq.\ \eqref{eq:N_BLP_1}, also requires maximizing over all system initial states.\ Although the maximization seems hard to compute directly, there exist pairs of initial states, called the \textit{optimal state pairs}, that achieve this maximum.\ It has been shown in \cite{weissman} that the optimal state pairs need to be orthogonal\footnote{If the two eigenspaces containing eigenvectors with non-zero eigenvalues of any two density matrices $\rho^1$ and $\rho^2$ are orthogonal, then $\rho^1$ is defined to be \textbf{\textit{orthogonal}} to $\rho^2$; denoted as $\rho^1 \perp \rho^2$.} and to lie on the boundary\footnote{$\rho$ is an \textit{interior point} of the state space $S(\mathcal{H})$ \textit{iff} $\forall \sigma \in S(\mathcal{H}) $, $\exists \lambda > 1$ so that $((1-\lambda)\sigma + \lambda \rho) \in S(\mathcal{H})$. If $\rho$ is not an interior point, then it is defined to be on the \textbf{\textit{boundary}} $\partial S(\mathcal{H})$ of $S(\mathcal{H})$. If $\rho$ has a zero eigenvalue, then $\rho \in \partial S(\mathcal{H})$.}.\\
In our case, we will take one of the system initial states in Eq.\ \eqref{eq:N_BLP_1} to be $\rho^1(0) = |s\rangle\langle s|$, where $|s\rangle$ is the uniform superposition of all system (register) states, see Eq.\ \eqref{eq:unif_sup_}.\ Note that $\rho^1(0)$ has only one non-zero eigenvalue and the zero eigenvalue is $(N-1)$-degenerate, implying that $\rho^1(0)$ belongs to the boundary of the state space.\ Moreover, the single `eigenstate with non-zero eigenvalue' of $\rho^1(0)$ happens to be orthogonal to the only `eigenstate with non-zero eigenvalue' of another state of the system, 
\begin{equation}
    \rho^2(0) = \frac{1}{N} \begin{psmallmatrix} \mathds{1}_{\frac{N}{2}} & -\mathds{1}_{\frac{N}{2}}\\ -\mathds{1}_{\frac{N}{2}} & \mathds{1}_{\frac{N}{2}}
    \end{psmallmatrix},
    \label{eq:rho_2}
\end{equation}
which will be used in the measure, Eq.\ \eqref{eq:N_BLP_1}.\\

 \begin{wrapfigure}[21]{l}{0.48\textwidth}
\centering
    \includegraphics[width=0.99\linewidth]{Figures/Screenshot_150.png}
    \caption[$\mathcal{N}_{BLP}$ is plotted with $\mu$, for different noise probabilities $p$]{$\mathcal{N}_{BLP}$ is plotted with respect to $\mu$, for different values of noise probability $p$. For $p=0$ and 1, $\mathcal{N}_{BLP}=0,\ \forall \mu$.\ In case of $p=0.33$ and 0.67, $\mathcal{N}_{BLP}>0$ for $\mu\gtrsim0.7$.\ Plots are for $n=3$, $\sigma_x$ noise and evolution up to $t=45$. }
    \label{fig:NM_mu}
\end{wrapfigure}For $p_{g'}=p=0$, i.e., \textit{noiseless} case, the system evolves under the ideal Grover unitary $G$.\ In this case, there is no information exchange between the system and the walker.\ Thus, $\mathcal{N}_{BLP}=0$ in case of $p=0$ for all parameters.\\
For $p_{g'}=p=1$, i.e., \textit{completely noisy} case, we have $p_{g|g} = \mu$, $p_{g'|g} = (1-\mu)$, $p_{g|g'} =0$, $p_{g'|g'} = 1$ from Eq.\ \eqref{eq:8_1}.\ In this case, the same $G'$ (the noisy Grover unitary) is applied at all times.\ Thus, in case of $p=1$ also, $\mathcal{N}_{BLP}=0$ for all parameters.\\
For $0<p<1$, $\mathcal{N}_{BLP}$ becomes non-zero only after the memory parameter $\mu$ becomes greater than a certain threshold value, as shown in Figure \ref{fig:NM_mu}.\  
This implies, for intermediate values of $p$ and sufficiently high $\mu$, backflow of information occurs
from the environment into the system.\ Also, it is interesting to point out that although such backflow is not seen for low $\mu$'s in case of $0<p<1$, it \textit{does not imply Markovianity}. 



\subsection{CP-Divisibility and non-Markovianity}
\label{Sec_quant_CP}
The statement \ref{st_Cp_d} means that
\begin{gather}
    \mbox{\textit{if} there is an initial state $R$ on $\mathcal{H}_{S'}\otimes\mathcal{H}_S$} \nonumber\\ \mbox{so that the trace norm $||\big(\mathds{1}_{S'}\otimes (\Phi_t)_{S}\big)[R]||_1$ \textit{increases} temporarily,} \nonumber \\ \mbox{\textit{then} the discrete-time dynamics $\{\Phi_t\}_{t\geq0}$ is \textit{not} CP-divisible.}
    \label{impl_Cp}
\end{gather}

To show the above, we will take the initial state as $R=\frac{\mathds{1}_N}{N}\otimes (|s\rangle\langle s|-|w\rangle\langle w|)$, where $|s\rangle$ is the uniform superposition of all states of the register, $S$ and $|w\rangle$ is the marked pure state of the register.\ $S'$ is taken to be of the same dimension as $S$ and it is in the completely mixed initial state $\frac{\mathds{1}_N}{N}$.\\
We can define a \textit{rough} (in the sense that it is still dependent on the state $R$) measure that gives an idea of the CP-divisibility of this process.\ 
\begin{equation}
    \mathcal{N}_{CP} = \sum_{\mathclap{\substack{t\\ \Gamma_{t+1} >\Gamma_t}}}\ \big(\Gamma_{t+1}-\Gamma_t\big)
    \label{eq:N_cp_ga}
\end{equation}
where $\Gamma_t =\frac{1}{2}||\big(\mathds{1}_{S'}\otimes (\Phi_t)_{S}\big)[R]||_1$.\ Note that although this looks similar to Eq.\ \eqref{eq:N_BLP}, there is no maximization being performed (although ideally it should be, to be a universal measure).\ We are considering the trace distance in the $\mathcal{H}_{S'}\otimes\mathcal{H}_S$ space whereas $\mathcal{N}_{BLP}$ considers the trace distance in only $\mathcal{H}_S$ space.\ It is very important to note here, that although any \textit{non-zero} value of $\mathcal{N}_{CP}$ implies \textit{breaking} of CP-divisibility from statement \eqref{impl_Cp}, $\mathcal{N}_{CP}=0$ \textit{does not} necessarily imply that the process is \textit{CP-divisible}. \\
Here we are not using the $\mathcal{N}_{RHP}$ measure (Eq.\ \eqref{eq:N_RHP}) that was defined for continuous time evolution because the time evolution in our case is  discrete.\ But, $\mathcal{N}_{CP}$ is very similar to $\mathcal{N}_{RHP}$ and thus can be thought of as the discrete-time analogue of $\mathcal{N}_{RHP}$.\\

\begin{figure}[h!]
\centering
    \includegraphics[width=.66\linewidth]{Figures/Screenshot_161.png}
    \caption[$\mathcal{N}_{CP}$ is plotted with $\mu$ for different $p$'s]{$\mathcal{N}_{CP}$ is plotted with $\mu$ for different values of $p$.\ For $0<p<1$, we see that $\mathcal{N}_{CP}>0$ for $\mu$ above some threshold - implying breaking of CP-divisibility in those cases.\ Here $n=3$, with $\sigma_x$ noise and time evolution up to $t=20$.}
    \label{fig:N_cp}
\end{figure}
As previously discussed, in case of $p=0$ (completely noiseless) and $p=1$ (completely noisy), the processes are unitary and CP-divisibility does not make sense in these cases.\ So, we have not plotted these two cases in Figure \ref{fig:N_cp}.\ From previous section, we have also found that there is no information back-flow in these cases too.\\
In our model, we find that for $0<p<1$, the  $\mathcal{N}_{CP}$ measure is non-zero for some values of $\mu$ above a threshold that depends on the values of $p$ and $n$.\ When $\mathcal{N}_{CP}>0$ in these cases, it means that for those parameters, the discrete-time dynamics of the register is guaranteed to break CP-divisibility and thus \textit{are non-Markovian}.\\ But, we also see in Figure \ref{fig:N_cp}, that for some small values of $\mu$, the measure $\mathcal{N}_{CP}=0$.\ In these particular cases, it is \textit{incorrect to claim} that these processes are necessarily CP-divisible.\ Since our measure Eq.\ \eqref{eq:N_cp_ga} is dependent on the initial state $R$, we can only say that in these cases the measure can not detect if they break CP-divisibility or not.\\
The \textit{main message being attempted to convey} through the analysis in Sections \ref{Sec_quant_info} and \ref{Sec_quant_CP} is that we can contrast the range of parameters in which each of the non-Markovianity measures (information flow and CP-divisibility) quantify the given processes as non-Markovian.\ We notice, by comparing Figures \ref{fig:NM_mu} and \ref{fig:N_cp}, that the measure quantifying CP non-divisibility identifies a greater portion of the parameter space than that identified by the measure quantifying information back-flow as non-Markovian.\ This is in accordance with the claim made in Section \ref{sec:N_BLP}, that $\mathcal{N}_{BLP}$ is a stronger measure of non-Markovianity (in the continuous-time evolution of a system, quantifies breaking of P-divisibility of the dynamical map) than the $\mathcal{N}_{RHP}$ measure (in continuous-time case, quantifies breaking of CP-divisibility of the map).\ 

\section{Effect of an elementary thermal bath}
Instead of taking the ancillas to be in a pure state $\rho_e = |00\rangle\langle00| = \begin{psmallmatrix} 1 & 0\\ 0 & 0\end{psmallmatrix}\otimes \begin{psmallmatrix} 1 & 0\\ 0 & 0\end{psmallmatrix}$ as we did in Eq.\ \eqref{eq:kraus_uni}, we can consider them all to be initially in thermal state 
\begin{equation}
    \rho_e = \begin{psmallmatrix} z_1 & 0\\ 0 & z_2\end{psmallmatrix}\otimes \begin{psmallmatrix} z_1 & 0\\ 0 & z_2\end{psmallmatrix} \label{eq:thermal_ancilla}
\end{equation}
where $z_1=\frac{1}{1+e^{-1/T}}$, $z_2=\frac{e^{-1/T}}{1+e^{-1/T}}$ with $T$ denoting a \textit{dimensionless temperature} parameter.\ Since these ancillas collide with the (system + walker) at every time step with the same initial thermal state, they mimic a \textit{thermal bath} \cite{ciccarello}.\ Whereas an actual thermal bath is made up of a continuum of modes, in this model, the interaction happens only with small, discrete and identically prepared constituents of the \textit{elementary} thermal bath. \\
\begin{wrapfigure}[20]{l}{0.48\textwidth}
    \centering
    \includegraphics[width=\linewidth]{Figures/Screenshot_151.png}
    \caption[$\mathcal{N}_{BLP}$ is plotted with $\mu$, $p$ and Temperature $T$]{$\mathcal{N}_{BLP}$ is plotted with $\mu$, for different $p$'s.\ Colored curves in each subplot are for temperatures $T$ (inset) of the ancillas.\ Here, $n=3$, noise unitary $\sigma_x$ and the system evolution is up to $t=45$.}
    \label{fig:NM_mu_T}
\end{wrapfigure} Keeping the unitary $U$ describing the (ancillas + walker + system) collision same as in Eq.\ \eqref{eq:Unitary_0} and \eqref{eq:Unitary_t} and using the initial state of the ancillas at each collision to be $\rho_e$ in Eq.\ \eqref{eq:thermal_ancilla}, we can write the time evolution of the (walker + system) state at each time step in the following Kraus representation
\begin{equation}
    R_t = \sum_{\mathclap{\substack{\alpha,\beta \\\in \{00,01,10,11\}}}}\ K_{\alpha\beta}\ R_{t-1}\  K_{\alpha\beta}^{\dagger} \label{eq:K_Rt_1}
\end{equation}
\begin{equation}
K_{\alpha \beta} = \pi_{\beta}\  \langle \alpha |U|\beta\rangle;\end{equation}\begin{equation}\pi_{\beta}=
\begin{cases}
  z_1, & \text{for} \quad \beta =00,\\
  \sqrt{z_1 z_2}, & \text{for} \quad \beta \in \{01,10\},\\
  z_2, & \text{for} \quad \beta =11,
\end{cases}
\end{equation}
Thus, we can simulate coupling to a heat bath by making the (system + walker) collide with the thermal ancillas at every time step of the algorithm.\ 
Since a Kraus decomposition exists, the (system + walker) evolution is thus CP-divisible and as a result, Markovian, from statement \ref{CP_marko}.\ It turns out that the process is also non-unital and leads to energy dissipation to the bath.

Figure \ref{fig:NM_mu_T} indicates the parameter ranges in which there is information back-flow into the system from the environment.\\ We have used the measure Eq.\ \eqref{eq:N_BLP_1} with the initial states $\rho^1(0)=|s\rangle\langle s|$ and $\rho^2(0)$ in Eq.\ \eqref{eq:rho_2}.\ In the completely noiseless case, i.e., $p=0$, we have $\mathcal{N}_{BLP}\approx 0,\ \forall \mu, T$.\ But for $p>0$, the effect of temperature on the information back-flow is more apparent. We notice that for increasing temperature, $\mathcal{N}_{BLP}$ decreases at a given $\mu$ and $p$, indicating increase in net outflow of information from (system + walker) to the bath.\\ 
While comparing Figure \ref{fig:NM_mu_T} with Figure \ref{fig:NM_mu}, it is also interesting to note that for $p=1$, $\mathcal{N}_{BLP}\neq 0$ for some values of $\mu$ and $T$.\ See Figure \ref{fig:2d_NM} for an overview of the effects of temperature, memory and noise probability on information back-flow.


\begin{figure}[h!]
\centering
    \includegraphics[width=.99\linewidth]{Figures/Screenshot_159.png}
    \caption[Variation of $\mathcal{N}_{BLP}$ with $T$ and $\mu$, for different $p$'s]{Variation of $\mathcal{N}_{BLP}$ (color-map on the right) with temperature $T$ and $\mu$ for different values of $p$.\ $\mathcal{N}_{BLP}$ increases with increasing $\mu$ and decreasing $T$.\ Here $n=3$, noise unitary is $\sigma_y$ and evolution up to $t=30$.}
    \label{fig:2d_NM}
\end{figure}
We should note here that the measures $\mathcal{N}_{BLP}$ (Eq.\ \eqref{eq:N_BLP_1}) or $\mathcal{N}_{CP}$ (Eq.\ \eqref{eq:N_cp_ga}) directly depend on our consideration of the duration of the system's evolution.\ Since both these measures are based on trace distance increments, the value of the measure in general compounds over time, except when the trace distance is monotonically decreasing.\ Thus, although the values indicated in the figures above are not universal, a non-zero value of these measures is an obvious indicator of a non-Markovian process.\ Nevertheless, to help in comparisons, we have kept most of the other parameters equal, for example, the total number of qubits, the noise unitary and the total duration of evolution.
% \begin{figure}[h!]
%     \includegraphics[width=0.65\linewidth]{Figures/Screenshot (152).png}
%         \centering
%     \caption{}
%     \label{}
% \end{figure}