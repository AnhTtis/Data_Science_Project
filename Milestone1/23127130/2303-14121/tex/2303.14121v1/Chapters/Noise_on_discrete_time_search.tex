\chapter{Searching on a noisy register} % Main chapter title

\label{Chapter4}
 
\textit{We consider a particular type of noise in which the ideal Grover unitaries in the noiseless GSA are modified by an additional arbitrary unitary evolution of some arbitrary qubits.\ We prove that only when these noise unitaries are the Pauli matrices, the success probability of the algorithm remains unaffected by changing the number $m\geq 1$ of noisy qubits in the register.\ We verify this result in two situations: first, a noise with no time-correlations, and second, a noise that is Markovian-correlated in time.} The results \cite{mandal2021} in this chapter can also be found in \href{https://doi.org/10.1103/PhysRevA.107.022427}{https://doi.org/10.1103/PhysRevA.107.022427}.

\section{The noise model}
\label{noise}
For a large database, 
the number of iterations of the Grover operator will be large for reaching the maximal success probability (Chapter \ref{Chapter3}) resulting in accumulation of noise or fluctuations in the circuit parameters, significantly affecting the efficiency of the algorithm.
Here we model this noise by replacing the Grover operator, \(G\), in each iteration with some probability by another operator, \(G'\), that is still unitary.\
This \emph{noisy} Grover operator is given by
\begin{equation}
\begin{split}
&G' = \bigchi_m\,G \\
   & = - \bigchi_m + 2 \left( \bigchi_m|s\rangle \right)\langle s|- \frac{4}{\sqrt{N}}( \bigchi_m|s\rangle) \langle w|\\ & \qquad \qquad \qquad \qquad \qquad \qquad \qquad + 2( \bigchi_m|w\rangle) \langle w|,
\end{split} \label{eq:6}\end{equation}
where $\bigchi_m$ is a tensor product of $n$ matrices, out of which any $m$ are each a  unitary operator, $U$, on \(\mathbb{C}^2\), and the rest $(n-m)$ are two-dimensional identity operators, $\mathds{1}_2$.\ For example, it can be
\begin{equation} \bigchi_{m}=\left(U\otimes (\mathds{1}_2)^{\otimes (n-m)} \otimes U^{\otimes (m-1)}
\right).\
\label{eq:7}
\end{equation}
Here $m$ can be considered as the ``\textit{noise strength}".\ We study how this noisy Grover operator affects the success probability and whether the success probability depends on the strength of noise.\ It is a general feature that the success probability of an algorithm reduces with increase in strength of noise, as seen e.g.
in~\cite{Biham}.\ So it will be helpful to have those noise unitaries which do not decrease the success probability with increase in the noise strength.\ We can christen such noise unitaries as ``\textit{good noise}".\
In the succeeding section, we try to identify the form of 
such good noise.\
\section{Search for ``good noise"}
\label{Proof}
To find what the good noises are, we will start with 
the most general single-qubit unitary matrix (in the computational basis),
\begin{equation}
    U = \begin{pmatrix} a & b\\ -\olsi{b} e^{i\theta} & \olsi{a} e^{i\theta}
    \end{pmatrix},
\end{equation}
with $a,b\in \mathbb{C}, |a|^2+|b|^2=1 $, $\theta \in [0,2\pi)$, and $\olsi{z}$ denoting the complex conjugate of $z$.\ The good noise corresponds to the values of $a,b$ and $\theta$, for which the success probability $P_m(t) = |\langle w|$ ${G^{\prime}}^t$ $|s\rangle|^2$ remains constant on changing the value of $m$.\ We will find the good noise by eliminating the $U$'s for which $P_m(t)$ does depend on $m$.\ We will start by finding the conditions on $U$, for keeping $P_m(t=1)$ constant with changing $m$.\ We have
 \begin{smalleralign}[\normalsize]
 P_m(1) &= \lvert\langle w|G'|s\rangle\rvert ^2 \nonumber\\
&= \left\lvert\left(1-\frac{4}{N}\right)\langle w|\bigchi_{m}|s\rangle+\frac{2}{\sqrt{N}}\langle w|\bigchi_{m}|w\rangle\right\rvert^2\nonumber\\ 
&= \frac{1}{N}\Biggl\lvert \left(1-\frac{4}{N}\right)\sum_{j=1}^{N}(\bigchi_{m})_{w,j}+2(\bigchi_{m})_{w,w}\Biggr\rvert^2 \label{eq:ch}\nonumber\\
= \frac{1}{N} &\Biggl \lvert  \left(1-\frac{4}{N}\right)(a+b)^{m-q}(\olsi{a}-\olsi{b})^q+2\, a^{m-q}\olsi{a}^q\Biggr\rvert^2,\;\;\;\;\;\;\;
\end{smalleralign}
where, $\bigchi_m |s\rangle = \frac{1}{\sqrt{N}}\begin{psmallmatrix}\sum_{j=1}^{N}(\bigchi_{m})_{1,j}\\.\\.\\\sum_{j=1}^{N}(\bigchi_{m})_{N,j}\end{psmallmatrix}$. 
It can be shown that\\
\begin{equation}
    \sum_{j=1}^{N}(\bigchi_m)_{k,j} = e^{iq\theta}(a+b)^{m-q}(\olsi{a}-\olsi{b})^q \ :=\psi_q
    \label{eq:e10}
\end{equation} \\
and $(\bigchi_m)_{k,k} = e^{iq\theta}a^{m-q}\olsi{a}^q$, 
where $q \in [0,m]$ and $q$ depends on $k$. Here each $\psi_q$ appears $\left(\frac{N}{2^m}\right)\begin{psmallmatrix}
m\\q
\end{psmallmatrix}$ times in $\bigchi_m|s\rangle$.\ 
From Eq.\ \eqref{eq:ch}, we can conclude that for getting $P_{m+1}(1) = P_m(1)$, we need either $|a|=0$ or $|b|=0$.\ Thus, for satisfying the condition $P_m(t) = P_{m+1}(t)$, we cannot have both $a$ and $b$ non-zero, and therefore we get our first condition for constructing a good noise which gives the constraints, Eqs.\ \eqref{Cond-1-1} and \eqref{Cond-1-2}.\ So, a good noise needs to obey
\begin{center}
    \textit{Condition 1:} $|a|=1$ or $|b|=1$
\end{center}

with
 \begin{numcases}{U=}
  \begin{psmallmatrix} a & 0\\ 0 & \olsi{a} e^{i\theta}\end{psmallmatrix}, \quad \text{for} \quad |a|=1,\label{Cond-1-1}\\
  \begin{psmallmatrix} 0 & b\\-\olsi{b} e^{i\theta} & 0\end{psmallmatrix}, \quad \text{for} \quad |b|=1.\label{Cond-1-2}
\end{numcases}\\
%\end{eqnarray}
Hence $\bigchi_m$ has to be a \textit{generalized permutation matrix} for which
%$w' =$\\
\begin{equation}
w^{\prime}=
\begin{cases}
 w, & \text{for} \quad |a|=1,\;\;\;\;\;\;\;\\
 \Biggl\lvert N-\frac{N}{2^m}-w+2\,\Big[w\Mod{\frac{N}{2^m}}\Big]\Biggr \rvert, & \text{for} \quad |b|=1.\;\;\;\;\;\;\;
\end{cases}
\end{equation}
 and
 %{\normalsize
\begin{numcases}{\bigchi_m|s\rangle= \dfrac{1}{\sqrt{N}} \times}
    \left(\mathlarger{\sum}_{i=1}^{M}c_i \sqrt{\Delta_i}|s'_i\rangle+\alpha|w\rangle\right),&for $|a|=1$,\label{decomp_1}\;\;\;\;\\
    \left(\mathlarger{\sum}_{i=1}^{M}c_i
   \sqrt{\Delta_i}|s'_i\rangle+\alpha|w\rangle +\beta|w'\rangle\right),&for $|b|=1$,\;\;\;\;\label{decomp_2}
\end{numcases}
%}
\begin{eqnarray*}
\text{with}\quad &&|s'_i\rangle = \frac{1}{\sqrt{\Delta_i}}\underset{\{|d_i\rangle\}\neq|w\rangle,|w'\rangle}{\mathlarger{\sum}}|d_i\rangle, \quad \Delta_i = \dim(\text{Span}\{|d_i\rangle\}),\\ 
&&\langle s'_i|s'_j\rangle = \delta_{ij}, \ \ |c_i|=1=|\alpha|=|\beta|,
\end{eqnarray*}
i.e., we get the basis $\mathbb{B}=\{|s'_1\rangle,|s'_2\rangle,\ldots,|s'_{M}\rangle,|w\rangle\}$ of dimension $(M+1)$ from Eq.~\eqref{decomp_1} and $\mathbb{B}=\{|s'_1\rangle,|s'_2\rangle,\ldots,|s'_{M}\rangle,|w\rangle, |w'\rangle \}$ of dimension $(M+2)$ from Eq.~\eqref{decomp_2} respectively.\
Our goal is to find out the noise matrices $U$ for which $P_m(t) = P_{m+1}(t),\ \forall m\geq 1$.\ 
When $m$ changes, $\bigchi_m$ changes and hence the matrix elements of $G'$ change in general.\ So, we need to find the special matrices $U$ for which the matrix elements of $G'$ doesn't change when written in the basis $\mathbb{B}$.\ There are two possibilities for a matrix $U$ of the forms in Eqs.~\eqref{Cond-1-1} and~\eqref{Cond-1-2}: the two non-zero elements are either  equal (Case $(i)$), or  unequal (Case $(ii)$).\ Case $(i)$ suggests $M=1$ and 
directly leads to the constraints, which have to be satisfied by the unitary presenting the good noise, given in Eqs.~\eqref{cond-2-1} and~\eqref{cond-2-2}.\ If we have a $U$ as in Case $(ii)$, we need to put further restrictions for the success probability to stay conserved with $m$.\ For that, since $G'$ when written in $\mathbb{B}$ must not change with $m$, we do not want $\dim(\mathbb{B})$ to change with the same.\ Thus, the number of distinct $c_i$'s in Eqs.~\eqref{decomp_1} and~\eqref{decomp_2} must remain constant with $m$.\ There are total $M$ of these coefficients.\ For $m=1$, i.e., for $\bigchi_1=U\otimes \mathds{1}_{\frac{N}{2}}$ in Case $(ii)$, there are only two distinct non-zero elements in $\bigchi_1$, because $U$ has two distinct non-zero elements.\ This implies
$M=2$.\ Since $M$ should remain constant with $m$, Case $(ii)$  leads to the restrictions for the unitaries constructing good noise to be satisfied given in Eqs.~\eqref{cond-2-3}, \eqref{cond-2-4}, \eqref{cond-2-5}. So, to summarise, we have another necessary (but not sufficient) condition:
 \begin{center}
    \textit{Condition 2:} $M$ = 1 or 2  
 \end{center}
{\small
 \begin{numcases}{\text{\normalsize Thus,}\quad \bigchi_m|s\rangle =}
 %c \, \lvert s\rangle =
 c\left(\,\sqrt{\frac{N-1}{N}}|s_1'\rangle +\frac{1}{\sqrt{N}}|w\rangle\right), & for $  |a| = 1, M=1$\label{cond-2-1}\\
 c\,\left(\sqrt{\frac{N-2}{N}}|s_1'\rangle +\frac{1}{\sqrt{N}}|w\rangle+\frac{1}{\sqrt{N}}|w'\rangle\right), & for $  |b| = 1, M=1$\label{cond-2-2}\end{numcases}}
 \begin{adjustwidth}{-30pt}{-20pt}
 {\small
 \begin{numcases}{\bigchi_m|s\rangle =}
 c_1\,\left( \sqrt{\frac{N-2}{2N}}|s_1'\rangle+\frac{1}{\sqrt{N}}|w\rangle\right)+\frac{c_2}{\sqrt{2}}|s_2'\rangle, & for $ |a|=1, M=2$ \label{cond-2-3} \\
c_1\, \left(\sqrt{\frac{N-4}{2N}}|s_1'\rangle+\frac{1}{\sqrt{N}}|w\rangle+\frac{1}{\sqrt{N}}|w'\rangle\right)+\frac{c_2}{\sqrt{2}}|s_2'\rangle, & for $ |b|=1, M=2,\alpha=\beta$ \label{cond-2-4}\\
c_1\,\left( \sqrt{\frac{N-2}{2N}}|s_1'\rangle+\frac{1}{\sqrt{N}}|w\rangle\right)+c_2\,\left(\sqrt{\frac{N-2}{2N}}|s_2'\rangle+\frac{1}{\sqrt{N}}|w'\rangle\right),& for $ |b|=1, M=2,\alpha\neq\beta$ \label{cond-2-5}\end{numcases}}
\end{adjustwidth}
So, only the $U$'s that satisfy one of the Eqs.\  \eqref{cond-2-1}-\eqref{cond-2-5}, are the unitaries corresponding to the good noise for which $P(t)$ does not depend on the number of noise sites $m$.\
 It can be shown that 
 $\psi_q$ appears $(\frac{N}{2^m})\begin{psmallmatrix}
 m\\q
 \end{psmallmatrix}$ times in the column vector $\bigchi_m|s\rangle$.\ We have the following observations.\
 \begin{enumerate}
\item[(1)] If $U$ satisfies Eq.\ \eqref{cond-2-1}, then $b=0$ and $\psi_q = c, \forall q $. Solving for $a$ and $\theta$ gives $a = e^{i\phi}= \sqrt[m]{c},\ \theta = 2\phi$, i.e., $U = \sqrt[m]{c}\begin{psmallmatrix}1 & 0\\0 & 1\end{psmallmatrix} = \sqrt[m]{c}\; \mathds{1}_2$.\ 

\item[(2)] If $U$ satisfies Eq.\ \eqref{cond-2-3}, then it turns out that we need $(a)\;\psi_q = \psi_{q+2} = c_1, \forall q$ even and $(b)\;\psi_q = \psi_{q+2} = c_2, \forall q$ odd.\ That is because, $\begin{psmallmatrix}
 m\\0
 \end{psmallmatrix}+\begin{psmallmatrix}
 m\\2
 \end{psmallmatrix}+\begin{psmallmatrix}
 m\\4
 \end{psmallmatrix}+\ldots = \begin{psmallmatrix}
 m\\1
 \end{psmallmatrix}+\begin{psmallmatrix}
 m\\3
 \end{psmallmatrix}+\begin{psmallmatrix}
 m\\5
 \end{psmallmatrix}+\ldots = 2^{m-1}$, i.e., the sum of multiplicities of elements in $\bigchi_m|s\rangle$ from the set $\{\psi_q|q\ \text{even}\}$ is equal to that in case of elements from the set $\{\psi_q|q\ \text{odd}\}$.\ Since $c_1 \neq c_2$, solving $(a)$ and $(b)$ 
for $a$ and $\theta$ give the solution, $a = \sqrt[m]{c_1},\ \theta = 2\phi-\pi$.\ The solution corresponds to $c_1=-c_2$, i.e., $U = \sqrt[m]{c}\begin{psmallmatrix}1 & 0\\0 & -1\end{psmallmatrix} = \sqrt[m]{c}\;\sigma_z$.\
\item[(3)] If $U$ satisfies Eq.\ \eqref{cond-2-2}, then $a=0$ and $\psi_q = c, \forall q $.\ Solving for $b$ and $\theta$ gives $b = e^{i\phi}= \sqrt[m]{c},\ \theta = 2\phi - \pi$, i.e., $U = \sqrt[m]{c}\begin{psmallmatrix}0 & 1\\1 & 0\end{psmallmatrix} = \sqrt[m]{c} \;\sigma_x$.\
\item[(4)] If $U$ satisfies Eqs.\ \eqref{cond-2-4} or \eqref{cond-2-5}, a similar analysis as above can be performed and the solution is $U = \sqrt[m]{c}\begin{psmallmatrix}0 & 1\\-1 & 0\end{psmallmatrix} = \sqrt[m]{c}\;i\;\sigma_y$.\

 \end{enumerate}
Here $\sqrt[m]{c}$ is only a constant phase factor.\ We can see from the above discussion that the restricted set of unitary qubit evolutions that are candidates for being good noise, are the matrices $e^{i\phi}\,\mathds{1}_2$, $e^{i\phi}\,\sigma_x$, $e^{i\phi}\,\sigma_y$ and $e^{i\phi}\,\sigma_z$, for any $\phi \in [0,2\pi)$ for $t=1$.\
We will now check if this set of noise unitaries are `good noise' for all times $t>1$.\ Similar to Eq.\ \eqref{eq:ch} we can write for $t=2$,
 \begin{smalleralign}[\normalsize]
P_m(2) &= \lvert\langle w|G'\,^2|s\rangle\rvert ^2 \nonumber\\
 &= \Bigg\lvert\left(1-\frac{4}{N}\right)\left[\left(2 \langle s| \bigchi_m |s\rangle  - \frac{4}{\sqrt{N}}\langle w|\bigchi_m|s\rangle + 2\langle w|\bigchi_m |w\rangle \right)\langle w|\bigchi_m|s\rangle-\frac{1}{\sqrt{N}}\right] +\nonumber\ \ \ \  \ \ \ \\& \quad \quad \quad \quad +\frac{2}{\sqrt{N}}\left[2\langle w|\bigchi_m|s\rangle \langle s|\bigchi_m|w\rangle -1 - \frac{4}{\sqrt{N}}\langle w|\bigchi_m |s\rangle \langle w|\bigchi_m|w\rangle+2\langle w|\bigchi_m|w\rangle^2\right]   \Bigg\rvert^2, \nonumber \label{Pm2}\\ 
\end{smalleralign}
where $\bigchi_m ^2 = \mathds{1}$ for a Pauli matrix $U$.\ 
%We have used Eq. \eqref{eq:e10}. 
It can be shown that
\begin{equation}
   \langle s|\bigchi_m|s\rangle =\frac{1}{N}\sum_{k=1}^{N}\psi_q = \frac{2^{(n-m)}}{N} [(a+b)+e^{i\theta}(\olsi{a}-\olsi{b})]^m.\  
\end{equation}

Now, for $U=\sigma_x$, we have $a=0$, $b=1$, and $\theta=\pi$.\ So, $\langle s|\bigchi_m|s\rangle = 1$, $\langle w|\bigchi_m|s\rangle =\frac{1}{\sqrt{N}}$ and $\langle w|\bigchi_m|w\rangle = 0$, $\forall m$.\ Putting these values in Eq.\ \eqref{Pm2}, we can see that $P_m(2)$ is independent of $m$.\ 

For $U=\sigma_z$, we have $b=0$, $a=1$ and $\theta=\pi$. So, $\langle s|\bigchi_m|s\rangle = 0$, $(\langle w|\bigchi_m|s\rangle)^2$, $\langle w|\bigchi_m |w\rangle \langle w|\bigchi_m|s\rangle$ 
remains constant, $\forall m$.\ Therefore from Eq.\ \eqref{Pm2}, our claim for $\sigma_z$ to be a good noise also holds for $t=2$.\

In case of $U=\sigma_y$, $a=0$, $b=-i$ and $\theta=\pi$.\ So we have $\langle s|\bigchi_m|s\rangle=0$, $\langle w|\bigchi_m|w\rangle=0$ and $\langle w|\bigchi_m|s\rangle
= \frac{1}{\sqrt{N}} e^{iq\pi}\, (-i)^{m}$.\ Hence,
 \begin{smalleralign}[\normalsize]
 P_m(2)=&\ \frac{1}{N}\Big\lvert\left(1-\frac{4}{N}\right)\left[ \left(2\, \langle w|\bigchi_m|s\rangle\right)^2 +1\right]\nonumber\\ &\quad \quad \quad \quad \quad -4\lvert\langle w|\bigchi_m|s\rangle\rvert^2 +2 \Big\rvert^2\nonumber\\
    &= \frac{1}{N}\Big\lvert\frac{1}{N}\left(1-\frac{4}{N}\right)\left(2\, e^{iq\pi}\, (-i)^{m}\right)^2  - \frac{8}{N} +3 \Big\rvert^2\nonumber\\
    &= \frac{1}{N}\Big\lvert\frac{4}{N}\left(1-\frac{4}{N}\right)(-1)^{m}  - \frac{8}{N} +3 \Big\rvert^2.
    \label{eq:e24}
\end{smalleralign}
%
From Eq.\ \eqref{eq:e24} we can infer that for $U=\sigma_y$, the success probabilities $P_m(2)=P_{m+2}(2), \ \forall m$, i.e., although for any consecutive $m$, the success probability is not constant, it does remain the same for $m$ staying either odd or even.\ We have not analytically shown here if this is true for $t>2$ in case of $\sigma_y$.\ But in Fig.\ \ref{fig:s_Y}, it is shown that indeed $P_m(t)=P_{m+2}(t), \ t\geq 1, \ \forall m$.\

For any $t\geq 2$, $P_m(t)=|\langle w|G'\,^t|s\rangle|^2$ consists of the terms $\langle s|\bigchi_m|s\rangle$,\ $\langle w|\bigchi_m|s\rangle$,\ $\langle w|\bigchi_m|w\rangle$ or combinations of these terms.\ Since it turns out that for $\sigma_x$ and $\sigma_z$, $P_m(t)$ are independent of $m$, we have $P_m(t)=P_{m+1}(t), \forall m$.\

So for example, if we have total of $N=8$  elements in a search space, then it turns out that the evaluation of the success probability in case of $m = 2$ noise sites and that in case of $m=3$ noise sites will be indistinguishable if the qubits in those sites are rotated by the good noises, i.e., $U \in \{\sigma_x,\sigma_z\}$.\ The success probabilities in the cases where $m=2$, $m=4$ or $m=6$, will be exactly the same in case of $U=\sigma_y$.\ Similarly, the cases of $m=1$, $m=3$ or $m=5$ will be indistinguishable when $U=\sigma_y$.\ It is to be noted that there may be some unitary $U$, other than these Pauli matrices, which makes the success probability independent of $m$ for some particular time $t$ and not at other times.\ The Pauli matrices $\sigma_x$ and $\sigma_z$ are special in the sense that when $U$ is one of these, the success probability becomes independent of $m$, \textit{for all \(t\)}.\ 


Another important observation is that none of the conditions used above put restrictions on what the positions of the $m$ unitaries are, out of the total $n$ positions.\ The coefficients $c_i$ remain the same for any arrangement of the $m$ unitaries.\ So, as mentioned earlier in Sec.~\ref{noise}, the success probability does not depend on the positions of the qubits which evolve under the noise unitary $U\in \{\sigma_x, \sigma_y, \sigma_z\}$.\ This result is also depicted in Fig.\ \ref{fig:Comparison_sites}.

\section{Markovian-correlated noise}
\label{Example}
To illustrate the above results, we will consider the situation where the noise is Markovian-correlated in time (see the discussion in Section \ref{app:Markovian}).\ This potentially important variety of noise with memory 
has not yet been studied before in case of GSA.\ It is also to be noted here that our results are not exclusive to only this kind of noise.\ In~\cite{Biham}, they examined the effect of noise on GSA incorporating the noise in the Hadamard gate in the first step of the search algorithm.\ The noise was uncorrelated, i.e., the noisy unitary Hadamard gates were constructed in a completely arbitrary manner from a Gaussian distribution and the unitary at each step did not depend on the unitary in the previous step(s).\ In contrast, we consider a correlated noisy Grover operator which at each iteration probabilistically depends on the preceding one.\ 
It is easy to understand the situation by considering the situation where any $m$ of the $n$ qubits become connected to another degree of freedom which we call as the \textit{walker}.\ A schematic diagram is shown in Fig.~\ref{fig:Markovian probabilities}.\ The walker has two orthogonal states $|g\rangle$ and $|g'\rangle$, and when it is in $|g'\rangle$, all the $m$ qubits connected to it are rotated by a unitary $U$ and the other $(n-m)$ are left as they were.\ Thus, with each time step (iteration), application of an ideal unitary Grover operator $G$ is followed by one of the following:
\begin{enumerate}
    \item[($I$)] any $m$ out of $n$ qubits are rotated by a unitary $U$, i.e., walker is in state $|g'\rangle$, or,
    \item[($II$)] all the $n$ qubits are left untouched, i.e., walker is in state $|g\rangle$.\ 
\end{enumerate}

The transition probabilities for $|g\rangle \leftrightarrow |g'\rangle$ are determined by a dichotomous Markov chain considered in~\cite{Macchia}, and described by Eq.~\eqref{eq:2} (in Section~\ref{app:Markovian}).\ To make the situation clearer, let us assume after the $(t-1)^{\text{th}}$ Grover iteration, the $n$ qubit register is in a state given by the density matrix $\rho_{t-1}$.\ Thus, \textit{before} the \((t+1)^{\text{th}}\) iteration, the register can be in the following two possible states:
\begin{eqnarray*}
&(I)& \ \rho_t = G'\,\rho_{t-1}\,G'^{\dagger},\;\; \text{with}\;\; G'= \bigchi_{m}\,G,\\
&(II)& \ \rho_t = G\,\rho_{t-1}\,G^{\dagger}.
\end{eqnarray*}
At $t=1$, i.e., on
the first Grover iteration, the probabilities of $(I)$ and $(II)$ are determined by the initial probabilities of the walker to be in states $|g'\rangle$ and $|g\rangle$ respectively.\ These probabilities are called \textit{stationary probabilities} and are taken to be $p_{g'} = p$ and $p_g = (1-p)$ respectively.\ Here $p$ can be referred to as the \textit{noise probability}.\ 
At any later time $t$, the probabilities are determined by the $(t-1)^{th}$ iteration and the \textit{memory} parameter, $\mu$.\ That is, for $t \geqslant 2$,
    \begin{numcases}{p_{k|l}=}
    (1-\mu)\,p_l + \mu, & \text{for} \quad $k=l$ \label{eq:8}\\
    (1-\mu)\,p_{k}, & \text{for} \quad $k \neq l$.\label{eq:8_1}
    \end{numcases}
where $k$, $l$ can be $G$ or $G^{\prime}$.\ See Section~\ref{app:Markovian} for further discussions.\
\begin{figure}[h]
    \centering
    \includegraphics[width=.7\linewidth]{Figures/Markovian_transition_diagram.png}
    \caption[Schematic diagram of Markovian transition probabilities]{Schematic diagram of Markovian transition probabilities for $t \ge 2$.\ The four probabilities corresponding to  application of the ideal and noisy Grover operators are given by $p_{g|g}$, $p_{g^{\prime}|g}$, $p_{g|g^{\prime}}$ and $p_{g^{\prime}|g^{\prime}}$.\ Please see the text for details.}
    \label{fig:Markovian probabilities}
\end{figure}

Before the application of the first Grover iteration, the $n$-qubit register is in the uniform superposition state $|s\rangle$.\ Let us denote the density matrix corresponding to this state as $\rho_0 \coloneqq |s\rangle  \langle s| $.\
So, the density matrix of the composite system containing the walker and the register before applying any Grover iteration is given as $R_0 = \left(\frac{|g\rangle+| g'\rangle}{\sqrt{2}}\right)\left(\frac{\langle g|+\langle g'|}{\sqrt{2}}\right) \otimes |s\rangle\langle s|$.\ So, the state of the register $\rho_1$ after the first Grover iteration will be $\rho_1 =$ $\text{Tr}_{walker}$ $\{R_1\}$ $=p_{g}\Phi^0[\rho_0]+p_{g'}\Phi^1[\rho_0]$ and the state after the second Grover iteration, $\rho_2 =$ $\text{Tr}_{walker} \{R_2\}$ $=p_{g}p_{g|g}\ \Phi^0[\Phi^0[\rho_0]]$ $+ p_{g}p_{g|g'}\ \Phi^0[\Phi^1[\rho_0]]$ $
    + p_{g}p_{g'|g}\ \Phi^1[\Phi^0[\rho_0]]$  $+$  
    $p_{g'}p_{g'|g'}$ $\Phi^1[\Phi^1[\rho_0]]$, where $R_1=S_0R_0$ and $R_2=S\,R_1$, with $S_0$ and $S$ being the transition operators given by
\begin{smalleralign}[\normalsize]
S_0 &=\Biggl(p_g|g\rangle\langle g| \otimes \Phi^0[.] + p_{g'}|g'\rangle\langle g'| \otimes \Phi^1[.]
+p_g|g\rangle\langle g'| \otimes \Phi^0[.] + p_{g'}|g'\rangle\langle g| \otimes \Phi^1[.]\Biggr),\\
S &=\Biggl(p_{g|g}|g\rangle\langle g| \otimes \Phi^0[.]+p_{g|g'}|g\rangle\langle g'| \otimes \Phi^0[.] +p_{g'|g} |g'\rangle\langle g| \otimes \Phi^1[.] + p_{g'|g'}|g'\rangle\langle g'| \otimes \Phi^1[.]\Biggr),
\end{smalleralign}
where $\Phi^0[\rho] = G \rho \, G^\dagger$ and $\Phi^1[\rho] = G' \rho \, G'^\dagger$. Similarly, for $t\geqslant2$, we have
\begin{equation}
    R_t = S\ ^{t-1} R_1,\quad   \rho_t = \text{Tr}_{walker}\{R_t\}.
\end{equation}
The success probability, i.e., the probability to find the marked state at time $t$, is given as
\begin{equation}
    P(t) = |\langle w|\rho_t |w\rangle|.
    \label{29}
\end{equation}

This kind of correlated noise with partial memory can potentially be found in real quantum devices, and it has been shown to provide an enhancement in the transmission of classical information as compared to transmission through noisy channels without memory~\cite{Macchia}.\ We investigate the effects of this Markovian correlated noise on the GSA numerically, and the results are gathered in the  subsection below.\

\subsection{A special case}
\label{App_A}
In case of $U = \sigma_x$ from Eq.\ \eqref{eq:6}, we get  $\bigchi_m |s\rangle = |s\rangle$.\ So, we can express all the states in terms of the orthogonal basis vector set $\{|\bar{s}\rangle, |w\rangle, |w'\rangle\}$, with
\begin{equation*}
|\bar{s}\rangle \coloneqq \frac{1}{\sqrt{N-2}}\sum_{\substack{x = 1 \\ x\neq w,w'}}^{N}|x\rangle.\
\end{equation*}
The uniform superposition state $|s\rangle$ then becomes
$\begin{psmallmatrix}
  \sqrt{\frac{N-2}{N}}, & \frac{1}{\sqrt{N}}, & \frac{1}{\sqrt{N}}
\end{psmallmatrix}^{\dagger} \label{3}
$ in this basis.\ Hence $G$ and $G^\prime$ have the forms,
\begin{equation} 
G=\Biggr(\begin{smallmatrix} 
2\frac{N-2}{N}-1 & -2\frac{\sqrt{N-2}}{N} & 2\frac{\sqrt{N-2}}{N}\\
2\frac{\sqrt{N-2}}{N} & -\frac{2}{N}+1 & \frac{2}{N}\\
2\frac{\sqrt{N-2}}{N} & -\frac{2}{N} & \frac{2}{N}-1
\end{smallmatrix}\Biggr), \quad G' = \Biggr(\begin{smallmatrix} 
2\frac{N-2}{N}-1 & -2\frac{\sqrt{N-2}}{N} & 2\frac{\sqrt{N-2}}{N}\\
2\frac{\sqrt{N-2}}{N} & -\frac{2}{N} & \frac{2}{N}-1\\
2\frac{\sqrt{N-2}}{N} & -\frac{2}{N}+1 & \frac{2}{N}
\end{smallmatrix}\Biggr)
\label{eq-4.15}
\end{equation}
It is evident from the expressions above that, at least for the case $U = \sigma_x$, although changing $m$ does change the forms of the basis vectors $|w'\rangle$ and $|\bar{s}\rangle$ in the computational basis, elements of all the states or operators like $|s\rangle$ or $G'$ remain the same in 
$\{|\bar{s}\rangle, |w\rangle, |w'\rangle\}$ basis.\ Thus, increasing or decreasing the noise strength $m$ does not affect the success probability of the algorithm in case of $U=\sigma_x$ and $m\geq 1$.\


\subsubsection{Success probability for perfect memory and \texorpdfstring{$U=$}{} \texorpdfstring{$\sigma_x$}{}}
\label{perfect memory}
Here, we consider the case when $\mu = 1$, i.e., the \textit{perfect memory}. On the first iteration (i.e., $t=1$), $G$ occurs with probability $(1-p)$ and $G'$ with $p$.\ Let us assume at $t = 1$, $G$ is applied.\ Due to perfect memory, for all $t\geqslant 2$, the same operator $G$ will be applied.\ This scenario corresponds to an ideal \textit{noiseless} GSA and marked state is reached at $t \approx \frac{\pi}{4}\sqrt{N}$ \cite{N&C}.\ Instead if $G'$ is applied at $t=1$, for $t\geqslant2$ the state of the whole $n$-qubit register would be $|\psi(t)\rangle = G'\, ^t\ |s\rangle$.\ To find the explicit expression for $G'\,{}^t$, we diagonalize $G'$ to get $G'_d$ and $G'_d {}^t$.\ Thus, using the explicit form in Eq.\ \eqref{eq-4.15} for $U = \sigma_x$,
 \begin{smalleralign}[\normalsize]
    \langle w|G'\,^t|s\rangle = \langle w|XG'_d\,^t X^{-1}|s\rangle 
    =(-1)^{t+1} \frac{1}{\sqrt{N}}\, \cot\left(\frac{\theta}{2}\right)
    \times \operatorname{\mathbb{I}m}\left[\tan\left(\frac{\theta}{2}\right)\left(\tan\left(\frac{\theta}{2}\right)-i\ \right)e^{i t \theta}\right]
\end{smalleralign}
where $X$ is the diagonalizing matrix, $\theta = \cos^{-1}(\frac{2}{N})$ and $\operatorname{\mathbb{I}m}[\cdot]$ denotes the imaginary part of a complex number.\ Then the probability to find the marked state at time $t$, is given as
\begin{equation}
    P(t) = |\langle w|G'\,^t|s\rangle|^2
     = \frac{1}{N} \cos^2 (\theta t)\left(\tan\left(\frac{\theta}{2}\right)\tan(\theta t)-1\right)^2.\ \label{C6}
\end{equation}
The first maximum of $P(t)$ in Eq.\ \eqref{C6} is analytically found at $t = \left(\frac{\pi}{\theta}-\frac{1}{2}\right) \approx \left(\frac{3}{2}+\frac{8}{\pi N}\right)$.\



\subsection{Patterns of the success probability}
\label{numerics}


In this subsection, we will see that the invariance of the success probabilities in case of Pauli noise unitaries, persists irrespective of any time-correlation in the noise.\
\subsubsection{\textit{Noise without memory}}
\begin{figure}[h!]
    \includegraphics[width=0.7\linewidth]{Memoryless.png}
        \centering
    \caption[Comparison of success probabilities for different noise unitaries $U$ in case of $\mu=0$, $n=6$]{Success probabilities of GSA
    in presence of noise without any time-correlation. So, $\mu=0$ here.\ We took \(n=6\).\ $P(t)$ is on the vertical axis and time along the horizontal axis.\ The inset shows values of the noise probability $p$.\ The plots are for different \(U\) and \(m\) as displayed below each plot.}
    \label{fig:Memoryless}
\end{figure}
The case of $\mu=0$ in Eqs.\ \eqref{eq:8}, \eqref{eq:8_1}, leads to noise without any memory or time-correlation.\ So at each time step, the probability for the Grover operation to become noisy is $p_{g'}=p$.\ In Fig.\ \ref{fig:Memoryless}, we see that the success probability's evolution, $P(t)$, for a given noise probability is unchanged when the number of noisy qubits is increased from $m=1$ to $m=5$ for $U=\sigma_x$.\ We contrast this with the change in $P(t)$ in case of $U=(\sigma_x+\sigma_z)/\sqrt{2}$, i.e., the Hadamard operator.\ 
We have plotted the success probability's evolutions for $m=1,\ 2,\ 4,\ 5$ in case of the noise unitary $\sigma_y$ and $\mu=0$ in Fig.\ \ref{fig:s_Y}.\ As discussed in Sec.\ \ref{Proof}, the behaviour is exactly the same for odd noise strengths, i.e., for $m=1$ and $m=5$.\ The same is true for even noise strengths $m = 2$ and $m=4$.

\subsubsection{\textit{Noise with finite time-correlation}} The success probability $P(t)$ of GSA for non-zero (positive) memory $\mu$ and for $n=6$ qubits for two different noise unitaries are depicted in Fig.~\ref{fig:Success probabilities}.\
 \begin{figure}[h!]
    \includegraphics[width=0.6\linewidth]{s_Y.png}
    \centering
    \caption[Success probabilities for $\mu=0$, $n=7$, $U=\sigma_y$]{Success probabilities of GSA
    in presence of noise without any time-correlation, i.e., $\mu=0$.\ The plots are for \(n=7\) and $U=\sigma_y$.\
    The inset table exhibits the symbols used in the plots for noise probability $p=0.1$ and $p=0.7$.
    The different plots are for different noise strengths \(m\), as displayed below each plot.}
    \label{fig:s_Y}
\end{figure} Here we have used the form of noise as $\bigchi_m= U^{\otimes m}\otimes \mathds{1}_2 ^{\otimes (6-m)}$ with $m=1$ and $4$.\ We can observe that the success probability $P(t)$ depends on the noise probability $p$ and the memory parameter $\mu$.\ It is obvious that the success probability reduces with increasing noise probability, and we can see from all the four panels that for a very high noise probability, the oscillatory behaviour of $P(t)$ tends to vanish.\ An interesting observation from a comparison of panels $(a)$ and $(c)$ is that they support our results regarding the algorithm's behaviour under a ``good'' noise.\ It can be seen that for the Pauli matrix $U=\sigma_x$, $P(t)$ for a given $p$ and $\mu$, remains unaffected when we change the number of noise sites $m$ on which $U$ is applied.\ Whereas for a non-Pauli unitary $U=(\sigma_y+\sigma_z)/\sqrt{2}$, which was shown not to be a good noise before, the success probability $P(t)$ changes with the noise strength $m$.\ Compare panels $(b)$ and $(d)$.\ We can see that for memory-less ($\mu=0$) noise at a very high noise probability say, $p=0.7$, the search algorithm becomes completely inefficient as the original oscillating nature of success probability completely disappears.\ The GSA is thus more efficient when the noise has higher memory and thus the correlation of the noisy unitaries are found to be more beneficial than the noise without memory.
\begin{figure}[t]
    \includegraphics[width=.62\linewidth]{Success_probabilities.png}
        \centering
    \caption[Success probabilities of GSA
    in presence of noise]{Success probabilities of GSA
    in presence of noise.\ The plots are for \(n=6\).\ Here we have plotted $P(t)$ on the vertical axis and the number of Grover iterations along the horizontal axis.\
    The inset table exhibits the symbols used in the plots for different pairs of values of the noise probability $p$ and memory parameter $\mu$.\
    The different plots are for different \(U\) and \(m\), as displayed below each plot.}
    \label{fig:Success probabilities}
\end{figure}
\begin{figure}[h!]
    \centering
    \includegraphics[width=.6\linewidth]{Comparison_sites.png}
    \centering
    \caption[Success probabilities of GSA for good noises]{Case of Good noises.\ Here \(n=7\), $p=0.5$ and $\mu=0.9$. The $\chi$'s are displayed below each plot.\ The corresponding $U$'s are in the inset.}
    \label{fig:Comparison_sites}
\end{figure} 


Moreover for higher values of $p$ for which the oscillation of $P(t)$ completely vanishes, the correlated noise plays an advantageous role.\ If we compare the lines corresponding to $(p,\mu)=(0.7,0) \ \text{and} \ (0.7,1)$ in the figure, we can find that the oscillatory nature reappears for a higher memory parameter and a better success probability can be obtained for a smaller number of iterations of the operator.\ \begin{figure}[t]
    \centering
    \includegraphics[width=.7\linewidth]{Figures/Screenshot_213.png}
    \caption[Effects of memory, database size and noise probability on the algorithm's success probability]{Effects of memory, database size and noise probability on the algorithm's success probability.\ We have presented here the values of $P(t^*)$, with respect to the noise probability $p$, for different $\mu$'s.\ The algorithm is performed on $n=\log_2 N$ qubits.\ Here the noise unitary considered is $U=\sigma_x$.\ All quantities used are dimensionless.\ }
    \label{fig:P_first max}
\end{figure}
The analysis for the evolution of success probability under $U=\sigma_x$ in case of perfect memory is summarised in Section \ref{perfect memory}.\
The success probabilities for the noiseless situation ($U=\mathds{1}_2$) and the good noises from Sec.~\ref{Proof} are plotted with respect to the number of iterations of the noisy Grover operator in Fig.~\ref{fig:Comparison_sites}.\
As we have commented previously, the positions of the noisy qubits do not matter if $U$ is one of the Pauli matrices.\
It is clear from the figure that for a given $U$, $P(t)$ remains unchanged while changing the positions of the noise sites.\
Fig.~\ref{fig:P_first max} gives an overview of the effects of memory, database size and noise probability on the algorithm's success.\ Here we plot the success probabilities at their first maxima $P(t=t^*)$ with respect to the noise probability $p$ for $U=\sigma_x$.\ The effect of memory is contrasted in the three subplots.\ As observed in Fig.~\ref{fig:Success probabilities}, here also we can see that for a given value of $p$, a higher memory $\mu $ of the noise provides a higher success probability of the algorithm.\
