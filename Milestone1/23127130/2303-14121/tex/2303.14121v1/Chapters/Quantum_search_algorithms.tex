\chapter{Grover's Search algorithm} % Main chapter title

\label{Chapter3}

\section{Quantum search algorithms}
After Grover's discovery \cite{Grover,Grover_2} of his eponymous algorithm for quantum search, several new algorithms and applications have been found based on it.\ In \cite{Vazirani} it was proved that the algorithm is optimal.\ A generalization of the algorithm, called \textit{amplitude amplification}, was devised in \cite{Brassard,amplitude}.\ There is also an important variant called `spatial search' \cite{benioff,aaronson,Shenvi} in which a marked element is searched by moving between items stored in different locations.\ In this thesis, we will focus on only the original Grover's search algorithm searching for a single element from a database.


\section{Grover's algorithm}
\label{noiseless algorithm}
The algorithm begins with a search space consisting of $N=2^n$ elements with $n$ being an integer.\ The elements of the space are denoted by $x=1,2,\ldots N$ and a function $f:\{0,1\}^n \rightarrow \{0,1\}$ is defined such that for the marked element $w$, $f(w)=1$ and $\forall x\neq w$, $f(x)=0$.\ To solve this problem, a classical computer evaluates $f$ for each of the elements until it gives the value 1, i.e., the marked element is found and therefore requires $O(N)$ operations.\ The advantage of Grover's search algorithm (GSA) over the classical treatment is that, by using a sequence of unitary operations, it can find the marked element by using only $O(\sqrt{N})$ queries to $f$.\ The steps of the algorithm are described as follows and a schematic demonstration is shown in Fig.~\ref{fig:my_label}. 
\begin{figure}
    \centering
    \includegraphics[width=.75\linewidth]{Figures/Screenshot_230.png}
    \caption[Quantum circuit for Grover's search algorithm]{Grover's search algorithm. The register of $n$ qubits, each $|0\rangle$, is each first subjected to a Hadamard operation.\ The second step is the operation of Grover operator, for $t=O(\sqrt{N})$ times, which is followed by a measurement on the output state.}
    \label{fig:my_label}
\end{figure}
It starts with all qubits of an $n$-qubit register in the $|0 \rangle$ state, the eigenvector of  $\sigma_z$ operator with eigenvalue 1.\ The next step is to act on each qubit by the Hadamard operator, $H = \frac{1}{\sqrt{2}}(\sigma_x +\sigma_z)$, where $\sigma_x$ and $\sigma_z$ are Pauli operators.\ Thus the total register comes to an \textit{uniform superposition state},
 \begin{smalleralign}[\normalsize]
 |s \rangle = \left( \frac{|0\rangle +|1\rangle}{\sqrt{2}} \right)^{\otimes n} = \frac{1}{\sqrt{N}}\sum_{x = 1}^{N}|x\rangle = {\frac{1}{\sqrt{N}}}\left(\sum_{\substack{x = 1 \\ x\neq w}}^{N} |x\rangle + |w\rangle \right),
    \label{eq:unif_sup_}
\end{smalleralign}
where $|w\rangle$ is the \textit{marked state}, i.e., the state corresponding to the element we are searching for in the database of $N = 2^n$ elements.\ The state $|s\rangle$  is then acted on by the \textit{Grover operator} $G=DO$,
where $D = (2|s\rangle \langle s|- \mathds{1}_{N})$ is called the \textit{Diffuser} and $O = (\mathds{1}_{N}-2|w\rangle \langle w|)$ is the \textit{Oracle}.\ For a detailed discussion about the construction of the \textit{Diffuser} $D$, \textit{Oracle} $O$ and the Grover operator $G$, see e.g.~\cite{Kitaev,N&C}.\ The operator $G$ has the form,
\begin{equation}
    G = -\mathds{1}_N + 2|s\rangle \langle s|- \frac{4}{\sqrt{N}}|s\rangle \langle w| + 2 |w\rangle \langle w|. \label{eq:3}
\end{equation}
\begin{wrapfigure}[10]{r}{0.35\textwidth}
    \centering
    \includegraphics[width=\linewidth]{Noiseless_grover.png}
    \caption[Success probability evolution in noiseless Grover algorithm]{Grover's algorithm for $n=5$ qubits.\ The smallest \(t\) for which \(P(t)\) is maximal is at \(t=4\).}
    \label{fig:Noiseless grover}
\end{wrapfigure}It acts on successive states until the state of the $n$-qubit register $|\psi(t)\rangle = G\,^t\, |s\rangle$
reaches close enough to the marked state $|w\rangle$.\
Here $t$ stands for the number of times the Grover operator is employed after the first step, i.e., after the Hadamard operation.\ The \textit{success probability}, i.e., the probability to find the marked state after $t^{th}$ operation, is given as $P(t) = |\langle w|\psi(t)\rangle|^2$.
It can be checked that the desired result is obtained after $t=\lfloor \frac{\pi}{4}\sqrt{N}\rfloor$ operations.\
See Fig.~\ref{fig:Noiseless grover} for the profile of the success probability with time, for a database with 32 entries.\

