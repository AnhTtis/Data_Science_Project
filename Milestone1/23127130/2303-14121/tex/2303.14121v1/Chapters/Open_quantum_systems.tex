

\chapter{Open quantum systems} % Main chapter title

\label{Chapter2}
\textit{In this chapter we review the concept of open systems, dynamical maps, master equations and some measures of non-Markovianity.\ We then elucidate the concept of `information' in quantum context.\ The chapter ends with an introduction to quantum channels with memory and a Markovian-correlated noise channel.\ These ideas will be implemented extensively throughout the thesis.}
\section{What is an open system?}
\begin{wrapfigure}[13]{r}{0.37\textwidth}
    \centering
   % \captionsetup{width=\linewidth}
    \includegraphics[width=.99\linewidth]{Figures/Screenshot_136.png}
    \caption[Schematic diagram of an open System]{Schematic diagram of an open System (green) interacting with an Environment (brown); leading to an exchange of energy or information.}
    \label{fig:Open_sys}
\end{wrapfigure}A system coupled to some external degrees of freedom is called an open system \cite{Petru,Riv,Weiss,N&C}.\ The concept, being so general, has found its applications in various studies outside physics, such as in understanding growth of living organisms \cite{Zotin} or in evolutionary theory and in social sciences \cite{Luhmann}.\ In physics, the parts of the Universe $\mathfrak{U}$ that are not included in the \textit{system} $\mathfrak{S}$, could be taken as the \textit{environment} $\mathfrak{E}$. 
But it often turns out that considering a much smaller part of the Universe, $\mathfrak{A} \in \mathfrak{U}$ is enough to determine the dynamics of the open system $\mathfrak{S} \in \mathfrak{A}$, as long as $\mathfrak{A}$ itself is a \textit{closed} system obeying Hamiltonian dynamics.\\ 
If a previously closed system $\mathfrak{S}$ obeying unitary quantum evolution becomes coupled to an environment $\mathfrak{E}$, the \textit{reduced evolution} of $\mathfrak{S}$ no longer remains unitary, in general.\ Any physical quantum system can hardly ever be expected to be perfectly closed.\ As a result, its state can seldom be taken as a \textit{pure state} and a proper description is given by the \textit{density matrix formalism}, first introduced in 1927 independently by Lev Landau \cite{Landau} and von Neumann \cite{Neumann}.

% \begin{table}[h!]
%   \begin{center}
%     \caption{Table using booktabs.}
%     \label{tab:table1}
%     \begin{tabular}{l|r}
%       \toprule % <-- Toprule here
%       \textbf{System} & \textbf{Exchanged quantities}\\
%       $\alpha$ & $\gamma$ \\
%       \midrule % <-- Midrule here
%       1 & a\\
%       2 & b\\
%       3 & c\\
%       \bottomrule % <-- Bottomrule here
%     \end{tabular}
%   \end{center}
% \end{table}

\section{Density matrix formalism}

\subsection{Pure states}
In von Neumann's mathematical formalization of quantum mechanics, a state of a quantum system is a unit vector $|\psi\rangle$ (called \textit{ket} in Dirac notation) which belongs to the \textit{state space} $\mathcal{H}$, which itself is a \textit{Hilbert space} (a complex inner product space that is also a complete metric space \cite{Rudin}.)\ A linear functional (called \textit{bra} in Dirac notation) $\langle \psi |\in \mathcal{H}^*$ gives a number after acting on a ket state.\ The operators are linear functionals from $\mathcal{H}$ to $\mathcal{H}$.\ A physical \textit{observable} is an operator $A$ so that $A = A^{\dagger}$.\ The expectation value of an operator $A$ is $\langle A \rangle = \langle \psi|A|\psi \rangle$.\ One of the most important observables is the energy operator or the Hamiltonian.\\
The time evolution of a pure state $|\psi\rangle$ in \textit{Schr\"odinger picture} under a Hamiltonian $H$ is given as 
\begin{equation}
    |\psi (t)\rangle = e^{-i\, H t} |\psi (0)\rangle,
    \label{eq:Sch_pic}
\end{equation}
where $t$ is time and the reduced Plank's constant, $\hslash$, is set to 1 here and for the remainder of this work.\ The \textit{unitary time-evolution} operator on $\mathcal{H}$ is defined as $U(t,t_0) = e^{-i\,H(t-t_0)}$,  $U(t,t_0)^{\dagger}U(t,t_0)=\mathds{1}$, with $\mathds{1}$ being the identity operator on $\mathcal{H}$.\ So, Eq.\ \eqref{eq:Sch_pic} can also be written as $|\psi (t)\rangle$ $= U(t,0)$ $|\psi (0)\rangle$.\ For a closed system evolving under a time-dependent hamiltonian $H(t)$, the time-evolution operator is given as 
\begin{equation}
    U(t,t_0) = \mathcal{T}e^{-i\, \int_{t_0}^{t}ds\, H(s)}
\end{equation}
where $\mathcal{T}$ is the time-ordering operator.\\ %\textcolor{red}{[$e^{-i\, \int_{t_0}^{t}ds\, H(s)}$ e lekha jay nki direct\\ $1+ \sum_{n=1}^{\infty}\frac{1}{n!}\Big(\frac{-i}{\hslash}\Big)^n \int_{0}^{t}dt_1 \int_{0}^{t_1}dt_2 \dots \int_{0}^{t_{n-1}}dt_{n-1} \mathcal{T}[H(t_1)H(t_2) \dots H(t_n)]$ eta likhte hy ektu check kore dekhe nio.]
%}\\
In the alternative \textit{Heisenberg picture}, the operators evolve in time instead of the system's state. The evolution of an observable $A$ is given as
\begin{equation}
    A(t) = U(t,t_0)^{\dagger} A(t_0) U(t,t_0).
\end{equation}

\subsection{Mixed states}
Often it is not possible to write down the state of a system as a simple pure state, for example, when the system is a convex mixture of pure states $\{|\psi_{i}\rangle\}$, each occurring with probabilities $\{p_i\}$.\ Then, the state is given by a \textit{density matrix}
\begin{equation}
    \rho = \sum_{i} p_i |\psi_{i}\rangle\langle \psi_{i}|.
\end{equation}
So, $\rho$ is positive semi-definite and $\text{Tr}\{\rho\}=1$.\ For a pure state, $\rho = |\psi\rangle\langle \psi| \implies \text{Tr}\{\rho^2\} = 1$. Whereas for a mixed state, $\text{Tr}\{\rho^2\} < 1$.\ Expectation value of an operator $A$ for the state $\rho$ is $\langle A\rangle = \text{Tr}\{A\rho\}$.\ The space that contains $\rho$ is called the \textit{Liouville space} and the time evolution of $\rho$ is determined by the Liouville-von Neumann equation,
\begin{equation}
    \partial_t \rho = -i[H,\rho] \implies \rho(t) = U(t,t_0)\rho(t_0) U^{\dagger}(t,t_0).
    \label{eq:vN_eq}
\end{equation}

\section{Time evolution of an open system}
 The interactions \cite{Weiss} with the environment generally leads to deviation from unitary dynamics and the system is said to be evolving under \textit{quantum noise} \cite{Gardiner}. It turns out that after tracing out the environment $E$ in the Liouville-von Neumann equation \eqref{eq:vN_eq} for the whole system $\rho_{tot}$, we can write for the open system $\mathfrak{S}$ density matrix $\rho$,
 \begin{gather}
      \partial_t \rho = \mathcal{L}(t)\rho,
      \label{eq:Lindblad_0}\\
      \text{or},\ \ \rho(t) = \mathcal{T}e^{ \int_{t_0}^{t}ds\,\mathcal{L} (s)}\rho(t_0),\\
      \qquad = \Phi(t,t_0)\rho(t_0) \label{eq:Phi}
 \end{gather}
  where $\mathcal{L}(t)$ is called the \textit{lindbladian} (or, quantum \textit{liouvillian}) \cite{Petru} and is the generator for $\Phi(t,t_0)$, a superoperator representing the \textit{dynamical map} (also called \textit{quantum channel}) acting on $\rho$.
 
 \subsection{The dynamical map}
 Thus the dynamical map $\Phi(t,t_0)$ in \eqref{eq:Phi} maps the initial state of the system at time $t_0$ to the state at a time $t$.\ 
\[\begin{tikzcd}[column sep=7em]
\rho_{tot}(0) = \rho(0)\otimes \rho_E \arrow{r}{\text{evolution under}\ U} \arrow[swap]{d}{\text{Tr}_E} & \rho_{tot}(t) = U(t,0)(\rho(0)\otimes \rho_E)U(t,0)^\dagger \arrow{d}{\text{Tr}_E} \\
\rho(0) \arrow{r}{\text{dynamical map}\ \Phi} & \rho(t)= \Phi(t,0)\rho(0)
\end{tikzcd}
\]



\subsection{Kraus-Sudarshan representation}\begin{wrapfigure}[12]{l}{0.32\textwidth}
    \centering
   % \captionsetup{width=\linewidth}
    \includegraphics[width=.99\linewidth]{Figures/Screenshot_218.png}
    \caption[Initial product state of the total system]{The system $S$ and environment $E$ are initially in a product state.}
\end{wrapfigure}


If we decompose the environment's initial state $\rho_E$ as a convex mixture $\{\lambda_{\alpha}\}$ of pure states $\{|e_{\alpha}\rangle \}$,
\begin{equation}
    \rho_E = \sum_{\alpha}\lambda_\alpha |e_\alpha\rangle\langle e_\alpha|,
    \label{eq:e_dec}
\end{equation}
and insert Eq.\ \eqref{eq:e_dec} into the following expression for unitary evolution of an \textit{initially product} joint state of the system and the environment
\begin{equation}
    \rho_{tot}(t) = U(t,0)(\rho(0)\otimes \rho_E)U(t,0)^\dagger,
    \label{eq:sep_rho}
\end{equation}
we get the evolved state of the system after time $t$
\begin{equation}
    \rho(t) = \text{Tr}_E\{\rho_{tot}(t)\} = \sum_{\alpha \beta}^{}K_{\alpha \beta}(t)\rho(0) K_{\alpha \beta}^{\dagger}(t) 
    \label{eq:Op_sum}
\end{equation}
  where $K_{\alpha\beta}(t) = \sqrt{\lambda_{\beta}}\langle e_{\alpha}|U(t,0)|e_{\beta}\rangle$ are the \textit{Kraus operators} and they satisfy the \textit{completeness relation} 
  \begin{equation}
      \sum_{\alpha \beta}^{}K_{\alpha \beta}^{\dagger}(t)K_{\alpha \beta}(t) = \mathds{1},
      \label{eq:Op_sum_1}
  \end{equation}
  which guarantees that $\text{Tr}_{\mathfrak{S}}\{\rho(t)\}=1$. The representation Eq.\ \eqref{eq:Op_sum} is known as the \textit{operator-sum or Kraus-Sudarshan representation} \cite{Sudarshan,Kraus,N&C}.\\
  The dynamical map with the representation Eqs.\ \eqref{eq:Op_sum} and \eqref{eq:Op_sum_1} is a \textit{completely positive (CP) \footnote{A map $\Phi: A\rightarrow B$ is \textit{\textbf{completely positive}} (CP) if the map $(\mathds{1}_{\mathbb{C}^{m\times m}}\otimes \Phi)$ is positive\textsuperscript{2} $\forall\ m$. CP'ty guarantees that the map $\Phi$ maps the set of physically possible density matrices (positive operators) to itself even when the system is coupled to an ancilla. }\textsuperscript{,}\footnote{A map $\Phi: A\rightarrow B$ is \textit{\textbf{positive}} (P) if it maps positive-semidefinite operators ($P\geq0,\ P\in A$) to positive-semidefinite operators ($Q\geq0,\ Q\in B$).}, trace preserving (TP)} map according to the Kraus representation theorem \cite{Choi}. 
  \subsubsection{\textit{Unitality}}
  A \textit{unital map} $\Phi$ takes the maximally mixed $d$-dimensional state $\rho_{mm} = \frac{\mathds{1}}{d}$ to the same state $\Phi(\rho_{mm}) = \frac{\mathds{1}}{d}$. This implies, the Kraus operators, $K_\mu$, for the map, $\Phi$, need to satisfy $\sum_{\mu}^{}K_{\mu}(t)K_{\mu}^{\dagger}(t) = \mathds{1}$.
  A map that is not identity preserving, so that $\Phi(\frac{\mathds{1}}{d}) \neq \frac{\mathds{1}}{d}$, is called \textit{non-unital}.\\
  Unital maps describe \textit{diffusion} or decoherence. Non-unital maps describe \textit{dissipative} processes and can be related to the classical idea of dissipation as the contraction of phase space volume \cite{mata}.
  

\subsection{(Non-)Markovian continuous time evolution}
We will denote the set of positive (P) dynamical maps from a liouville space to itself as $\mathfrak{P}$, of CP maps as $\mathfrak{P}^{+}$, of trace-preserving (TP) P maps as $\mathfrak{T}$ and of TP CP maps as $\mathfrak{T}^{+}$.
\subsubsection{\textit{Divisibility of a map}}
  Let us consider a map $\Phi \in \{\mathfrak{T},\mathfrak{T}^{+}\}$.\ If there exists a decomposition $\Phi=\Phi_1 \circ \Phi_2$ so that none of $\Phi_i$ is an unitary operator, then $\Phi$ is called a \textit{divisible map} \cite{Cirac}.\\
  Otherwise,
  it is called an \textit{indivisible map}. We will denote the set of divisible maps as $\mathfrak{D}$. 
  
  \subsubsection{\textit{CP- and P-divisibility}}
  If an invertible $\Phi(t,t_0)\in \mathfrak{P}^{+}$ can be decomposed as
  \begin{equation}
        \Phi(t,t_0)=\Phi(t,s)\circ \Phi(s,t_0); \quad t\geq s\geq t_0
        \label{eq:CP_divis}
  \end{equation}
 so that $\Phi(t,s)$ is CP- (P-) map, then $\Phi(t,t_0)$ is called \textit{CP- (P-)divisible}.
  
  \subsubsection{\textit{Markovian maps}}
  It is an important special case when a dynamical map satisfies the \textit{homogeneous} composition law
  \begin{equation}
      \Phi(t_2-t_1+t_0,t_0)\circ \Phi(t_1,t_0) = \Phi(t_2,t_0); \quad t_2\geq t_1\geq t_0,
      \label{eq:semigroup}
  \end{equation}
  analogous to the classical Chapman-Kolmogorov equation.\ Then the corresponding dynamics is said to be \textit{Markovian} and \eqref{eq:Lindblad_0} leads to a \textit{master equation in the strict GKSL (Gorini-Kossakowski-Sudarshan-Lindblad)} form \cite{Gorini}:
\begin{gather}
      \partial_t \rho = \mathcal{L}\rho= -i[H,\rho] + \sum_{\nu} \mathcal{D}_{\nu}[\rho]
      \label{eq:Mark_lind}\\
      \mathcal{D}_{\nu}[\rho] 
      = \gamma_{\nu}\left( L_{\nu}\rho L_{\nu}^{\dagger} - \frac{1}{2}\{L_{\nu}^{\dagger}L_{\nu},\rho\}\right)\\
      \qquad \quad = \frac{\gamma_{\nu}}{2}\left( [L_{\nu}\rho,L_{\nu}^{\dagger}]+ [L_{\nu}\rho,L_{\nu}^{\dagger}]^{\dagger}  \right)
      \label{eq:dissipator}
\end{gather}  
  where $\mathcal{D}_{\nu}[\rho]$ is called the \textit{dissipator}, $\{\gamma_{\nu}\}$ are the positive \textit{decay rates} and $\{L_{\nu}\}$ is the set of orthonormal \textit{lindblad operators} \cite{Havel}. The time-\textit{independent} lindbladian $\mathcal{L}$ in \eqref{eq:Mark_lind} generates the one-parameter \textit{semigroup} $\Phi(t-t_0)=e^{(t-t_0)\mathcal{L}},\ \forall t\geq 0$ \cite{Lind}. Denoting the set of markovian maps as $\mathfrak{M}$, we have $\mathfrak{M}  \subset \mathfrak{D}$.\\
  The Markovian case \eqref{eq:semigroup}-\eqref{eq:dissipator} is not usually the situation in practice. The processes in which this treatment does not hold are called \textit{non-markovian} and their equations can be derived directly from the Liouville-von Neumann equation \eqref{eq:vN_eq} using the \textit{projection operator techniques}. 
  
    \subsubsection{\textit{The Nakajima-Zwanzig equation}}

  From \eqref{eq:vN_eq}, we can define the Liouville operator for the whole system dynamics $\mathcal{L}[\cdot] = i[\cdot,H]$ and the projection super-operator $\mathcal{P}$ 
  \begin{equation}
      \rho_{tot} \xmapsto[]{} \mathcal{P}\rho_{tot} = \text{Tr}_E\{\rho_{tot}\}\otimes \rho_E = \rho \otimes \rho_E
  \end{equation}
 where $\rho_E$ is a fixed environment state so that $\mathcal{P}^2 \rho_{tot} = \mathcal{P}\rho_{tot}$. An orthogonal projection operator $\mathcal{Q}$ is defined with $\mathcal{P}+\mathcal{Q}=\mathds{1}$. Operating with $\mathcal{P}$ and $\mathcal{Q}$ on the total density matrix of \eqref{eq:vN_eq}, we get the system of equations
  \begin{smalleralign}[\normalsize]
     \partial_t \mathcal{P}\rho_{tot} &=\mathcal{P}\mathcal{L}(\mathcal{P}+\mathcal{Q})\rho_{tot},\\
    \text{and}\quad \partial_t \mathcal{Q}\rho_{tot} &= \mathcal{Q}\mathcal{L}(\mathcal{P}+\mathcal{Q})\rho_{tot}.
 \end{smalleralign}
 Solving these and assuming uncorrelated initial state $\rho_{tot}(0) = \rho(0)\otimes \rho_E$ from \eqref{eq:sep_rho}, i.e., $\mathcal{P}\rho_{tot}(0)=\rho_{tot}(0),\ \mathcal{Q}\rho_{tot}(0)=0$, we get the final form for the \textit{Nakajima-Zwanzig equation} \cite{Nakajima,Zwanzig,prigogine}:
  \begin{smalleralign}[\normalsize]
    \partial_t \rho &= -i[H_s,\rho(t)]+\int_{0}^{t} d\tau\  \text{Tr}_E\{\mathcal{P}\mathcal{L}\ e^{\mathcal{Q}\mathcal{L}\tau}\mathcal{Q}\mathcal{L}\mathcal{P}\ \rho_{tot}(t-\tau)\}\\
     &=-i[H_s,\rho(t)]+\int_{0}^{t} d\tau\  \text{Tr}_E\{\mathcal{K}(\tau)\ \mathcal{P}\ \rho_{tot}(t-\tau)\}\\
     &=-i[H_s,\rho(t)]+\int_{0}^{t} d\tau\  \text{Tr}_E\{\mathcal{K}(t-\tau)\ \mathcal{P}\ \rho_{tot}(\tau)\}\label{eq:NZ_eqn}\\
   \text{where} \quad  \mathcal{K}(t) &=\mathcal{P}\mathcal{L}\ e^{\mathcal{Q}\mathcal{L}t}\mathcal{Q}\mathcal{L}\mathcal{P}
\end{smalleralign} 
is the \textit{memory kernel}.\ The equation is linear in $\rho$ but is non-local in time. The dynamics can thus account for non-markovianity and memory effects of an environment.
 
 \subsubsection{\textit{Time-dependent GKSL master equation}}
 The time non-local Eq.\ \eqref{eq:NZ_eqn} can be brought to the `canonical' form as the GKSL equation \eqref{eq:Mark_lind} using the \textit{time-convolutionless} projection operator technique \cite{Chaturvedi,Shibata} under the sufficient condition of the existence of left inverse $\Phi(t,t_0)^{-1}$ for all $t$ so that the lindbladian can be written as 
 \begin{gather}
     \mathcal{L}(t) = (\partial_t \Phi(t-t_0))\  \Phi(t,t_0)^{-1} \label{eq:TCL_lind}\\
     \implies \partial_t \rho = \mathcal{L}(t)\ \rho= -i[H(t),\rho] + \sum_{\nu} \mathcal{D}_{\nu}(t)[\rho]
 \end{gather}
 which is the time-local \textit{time-dependent GKSL} master equation with time- dependent dissipators
 \begin{equation}
     \mathcal{D}_{\nu}(t)[\rho] 
      = \gamma_{\nu}(t)\ \left( L_{\nu}(t)\rho L_{\nu}(t)^{\dagger} - \frac{1}{2}\{L_{\nu}(t)^{\dagger}L_{\nu}(t),\rho\}\right).
      \label{eq:td_diss}
 \end{equation}
 From Eq.\ \eqref{eq:TCL_lind}, the dynamical map can be written as 
 \begin{gather}
     \Phi(t,t_0) = \mathcal{T}e^{ \int_{t_0}^{t}ds\,\mathcal{L} (s)}
     \label{eq:28}\\
     \implies \Phi(t,s)\circ \Phi(s,t_0) = \Phi(t,t_0); \quad t\geq s\geq t_0,
     \label{eq:29}
 \end{gather}
 which is a time-\textit{inhomogeneous} \cite{Man_rev} composition law and differs from the time-\textit{homogeneous} Eq.\ \eqref{eq:semigroup} by being explicitly dependent on the intermediate time $s\geq t_0$ as a result of Eq.\ \eqref{eq:28}.\\
 Since the decay rates $\gamma_\nu (t)$ are time-dependent, they can even be negative temporarily without violating CP'ty of the dynamical map. But the dynamics is 
 \begin{equation}
     \mbox{CP-divisible \textit{if and only if} $\gamma_\nu (t) \geq 0,\ \forall t,\nu$\  \cite{Breuer_NM_rev}.}
     \label{CP-div}
 \end{equation}
 In \cite{Chru}, the condition for markovianity is that
 \begin{equation}
     \mbox{\textit{if} $\Phi(t,t_0)$ is CP-divisible, \textit{then} $\Phi(t,t_0)$ is \textit{markovian}.} \label{CP_marko}
 \end{equation}
 If $\Phi(t,t_0)$ is invertible, then the dynamics is 
 \begin{equation}
     \mbox{P-divisible \textit{if and only if} $\Phi(t,t_0)$ is markovian \cite{Kossa_1,Kossa_2}.}
 \end{equation}

\subsubsection{\textit{Measure of non-markovianity: CP-divisibility}}
The following statement is generally made about non-markovianity from the TD GKSL equation \eqref{eq:td_diss}
\begin{equation}
    \mbox{\textit{if} at least one $\gamma_\nu (t) < 0$, \textit{then} $\Phi(t,t_0)$ is \textit{non-markovian} \cite{hierarchy}.}
\end{equation}
Combining this with statement \ref{CP-div} implies
\begin{equation}
    \mbox{\textit{if} $\Phi(t,t_0)$ is \textit{not CP-divisible}, \textit{then} it is non-markovian.}
\end{equation}
Thus, a measure for non-markovianity can be the non-CP divisibility character of the dynamical map. This is the idea behind the \textit{Rivas-Huelga-Plenio (RHP) measure} \cite{RHP} of non-markovianity,
\begin{gather}
    \mathcal{N}_{RHP}(\Phi) = \underset{\epsilon \rightarrow 0^+}{\mathrm{lim}} 
    \int_{t_0}^{\infty}dt\  \frac{\text{Tr}\{\sqrt{Q^{\dagger}Q}\} \ -\ 1}{\epsilon}\\
    Q = (\Phi_{(t+\epsilon,t)}\otimes \mathds{1}_A)(|\Psi\rangle_{SA}\ {}_{SA}\langle \Psi|)
    \label{eq:N_RHP}
\end{gather}
where $|\Psi\rangle_{SA}$ is the maximally entangled state of the system S with an ancilla A and $Q$ is the \textit{Choi matrix} \cite{Choi} corresponding to $\Phi(t+\epsilon,t)$. \\
If $\Phi_{(t+\epsilon,t)}$ is \textit{CP}, we have Choi matrix $Q\geq 0$, i.e., $\text{Tr}\{\sqrt{Q^{\dagger}Q}\}=1$, implying $\mathcal{N}_{RHP}(\Phi)=0$. \\
Whereas if 
$\Phi_{(t+\epsilon,t)}$ is \textit{not CP}, we have $\text{Tr}\{\sqrt{Q^{\dagger}Q}\}>1$ because of the contraction property of trace distance under CP maps and thus $\mathcal{N}_{RHP}(\Phi)>0$.
Thus $\mathcal{N}_{RHP}$ may detect some P-divisible (but not CP-divisible) maps as non-markovian processes \cite{Guo,Hall,eternals}. 

 \subsubsection{\textit{Stronger measure of non-Markovianity: Information flow}}
 \label{sec:N_BLP}
 Since in a Markovian process an open system continuously loses correlations and information to the environment, a non-Markovian process can be associated with a \textit{backflow of information} from the environment and into the system.\ \begin{wrapfigure}[14]{l}{.37\linewidth}
    \centering
    \captionsetup{width=\linewidth}
    \includegraphics[width=\linewidth,frame]{Figures/Screenshot_140.png}
    \caption[Schematic evolution of trace distance for a CP map]{A schematic evolution of trace distance (purple curve) for a CP map. The initial trace distance between $\rho^1(0)$ and $\rho^2(0)$ is marked as a dashed line and is the upper bound for all $t$.\ But after the initial (red interval, left) decrease, the trace distance increases for some finite time interval (green) before again decreasing (red interval, right).}
    \label{fig:Tr_dist_NM}
\end{wrapfigure} Since two states are more distinguishable when there is more information present in the open system, information flowing in and out of it can be tracked by quantifying their distinguishability and is measured by the \textit{trace distance}
\begin{smalleralign}[\normalsize]
      D(\rho^1,\rho^2)=\frac{1}{2} \text{Tr}\left\{\sqrt{(\rho^1-\rho^2)^{\dagger} (\rho^1-\rho^2)}\right\}
      \label{eq:Tr_dist}
 \end{smalleralign}
 between any two given density matrices $\rho^1$, $\rho^2$ of the open system.\\
 It turns out that under CP dynamical maps, $D(\rho^1(t)$,$\rho^2(t))$ \textit{never exceeds} the initial $D(\rho^1(0)$, $\rho^2(0))$. But, its evolution is not monotonic, for example, see the figure \ref{fig:Tr_dist_NM}.
In the time intervals at which $D(\rho^1(t)$, $\rho^2(t))$ decreases, corresponding to loss of information from the open system to the environment.\ In the Figure \ref{fig:Tr_dist_NM}, there is also a time interval in which the trace distance increases, i.e., $\frac{dD}{dt} >0$, which corresponds to flow of information from the environment into the system. This corresponds to a non-Markovian evolution of the open system and can be quantified by the \textit{Breuer-Laine-Piilo (BLP) measure} \cite{Breuer_Laine} of non-Markovianity, 
\begin{equation}
    \mathcal{N}_{BLP}(\Phi(t,t_0)) = \underset{\rho^1(0),\rho^2(0)}{\mathrm{max}} 
    \int_{ \frac{dD}{dt} >0} dt\  \frac{d}{dt} D(\rho^1(t),\rho^2(t))
    \label{eq:N_BLP}
\end{equation}
  for the time evolution of the open system from $t_0$ to $t$.\ Since for a Markovian dynamical map, $\Phi_{M}\in \mathfrak{M}$, $D(\rho^1(t),\rho^2(t))$ is monotonically decreasing, we have $\mathcal{N}_{BLP}(\Phi_M)=0$.\ For a non-Markovian map $\Phi_{NM}$,\ $\mathcal{N}_{BLP}(\Phi_{NM})>0$.
   If a map is CP-divisible, then there is no information back-flow, but the converse may not be true \cite{Haikka,Guo}.\ Since the growth of trace distance breaks P-divisibility \cite{Breuer_NM_rev}, $\mathcal{N}_{BLP}$ \eqref{eq:N_BLP} quantifying the back-flow of information is considered a measure for \textit{strong} \cite{bernardes} or \textit{essential} non-Markovianity \cite{Chru} as compared to the $\mathcal{N}_{RHP}$ measure \eqref{eq:N_RHP} which quantifies breaking of CP-divisibility.\ There are processes that break CP-divisibility but are P-divisible \cite{Guo,Hall,eternals}.\ $\mathcal{N}_{RHP}$ can detect these processes as non-Markovian while $\mathcal{N}_{BLP}$ can not.  
   
 
 \subsection{Discrete-time evolution}
 
 Although an open system may evolve continuously in time, it may be useful in some cases to describe its evolution as `stroboscopic' or as discrete time-steps.\ 
 \begin{wrapfigure}[10]{r}{.45\linewidth}
   %\begin{figure}
    \centering
    \captionsetup{width=\linewidth}
    \includegraphics[width=\linewidth,frame]{Figures/Screenshot_262.png}
    \caption[Venn diagram for non-Markovianity]{This is a Venn diagram for \textit{weakly} and \textit{essentially} non-Markovian processes.\ It shows that they break CP- and P- divisibility respectively.}
\end{wrapfigure}Thus, a collection of CPTP maps $\{\Phi_t\}_{t\geq 0}$ can be defined so that the density matrix $\rho_s$ at time $s$ is given as $\rho_s = \Phi_s\, \rho_0$ where $\rho_0$ is the initial state and $\Phi_0=\mathds{1}$.\ 
 
 \subsubsection{\textit{CP-divisibility}}

 Similar to the continuous-time case (Eq.\ \eqref{eq:CP_divis}), the discrete-time dynamical map $\{\Phi_t\}_{t\geq 0}$ is defined to be CP-divisible \textit{if and only if} there exist CP maps $\Phi_{t,s}$ with $t>s>0$, so that $\Phi_t = \Phi_{t,s}\, \Phi_s$.
 \subsubsection{\textit{Information-decreasing maps}}
A discrete-time dynamical map $\{\Phi_t\}_{t\geq 0}$ is information-decreasing \cite{Buscemi,ujan} \textit{if and only if} $\forall$ initial ensemble of states $\{p_i,\rho^i_0\}$, we have 
\begin{equation}
    \underset{\{P^i\}}{\mathrm{\text{max}}}\big( \sum_i p_i\, \text{Tr} \{P^i\,\rho^i_s\}\big)\ \geq\ \underset{\{P^i\}}{\mathrm{\text{max}}}\big( \sum_i p_i\, \text{Tr} \{P^i\,\rho^i_{s+1}\}\big), \quad \forall s\geq 0,
\end{equation}
where the maximization is over all POVM\footnote{A \textit{\textbf{Positive Operator-valued Measure}
} (POVM) is a set of positive semi-definite hermitian matrices $\{P^i\}$ on a Hilbert space so that $\sum_i P^i = \mathds{1}$.}'s $\{P^i\}$'s over the system's Hilbert space.\\
In \cite{Buscemi}, it was shown that a discrete-time dynamical map $\{\Phi_t\}_{t\geq 0}$ is CP-divisible \textit{if and only if}
\begin{equation}
      \mbox{$\{(\mathds{1}_{S'}\otimes (\Phi_t)_{S})\}_{t\geq 0}$ is \textit{information-decreasing} for any ancilla $S'$.} \label{CP_disc}
 \end{equation}
 For an initial ensemble $\{p^1,\rho^1_0;p^2,\rho^2_0\}$, we have, $\mathrm{\text{max}}\big( \sum_i p_i\, \text{Tr} \{P^i\,\rho^i_{s+1}\}\big)$ $= \frac{1}{2}(1+D(p^1\rho^1_0,p^2\rho^2_0))$ \cite{helstrom}.\ Thus, from statement \ref{CP_disc}, we see that\\
 \textit{if} $\{\Phi_t\}_{t\geq 0}$ is CP-divisible, \textit{then} for any hermitian $R$ on $\mathcal{H}_{S'}\otimes\mathcal{H}_S$,
 \begin{equation}
     \mbox{$D\big([\mathds{1}_{S'}\otimes (\Phi_t)_{S}](R),0\big)$ is a non-increasing function of $t$.}
     \label{st_Cp_d}
 \end{equation}
 
 \subsection{Meaning of `Information'}
 Entropy measures how chaotic a system is.\ A system contains more `information' in it means it has lower entropy than a system with less information.\ The \textit{von Neumann entropy} $S$ is the quantum counterpart of the classical Gibbs entropy or the Boltzmann equation.\ For a density matrix $\rho$, it is given as 
 \begin{equation}
     S = -k_B\ \text{Tr}\{\rho\ \text{log}\rho \}.
 \end{equation}
 Thus a system `losing information' to the environment refers to the system's `increasing von Neumann entropy'.\ $S(\rho)=0$ for a pure state $\rho$. Some important properties of $S$ are listed below:
 \begin{itemize}
\item $S(\rho) = S(U\rho U^{\dagger})$ for any unitary transformation $U$.
\item $S(\sum _{i} \lambda_i\, \rho_i) \geq \sum _{i} \lambda_i\ S(\rho_i);\ \sum_i \lambda_i = 1,\ \lambda_i\geq 0$.
\item $S(\text{Tr}_{A}\{\rho\})+S(\text{Tr}_{B}\{\rho\})\geq S(\rho) \geq |S(\text{Tr}_{A}\{\rho\})-S(\text{Tr}_{B}\{\rho\})|$ for a bipartite system AB with state $\rho$.
 \end{itemize}
Apart from the von-Neumann entropy $S(\rho)$ measure of information, there also exist other measures \cite{Guo,Breuer_Laine} which depend on the particular situations at hand.\ The \textit{trace distance} measure $D(\rho^1,\rho^2)$ in Eq.\ \eqref{eq:Tr_dist} is a measure of distinguishability of two states, as discussed in Section \ref{sec:N_BLP}.\ Since more distinguishable states of a system gives more information about the states, the trace distance can thus be made an indicator of the system's information content.\ $D(\rho^1,\rho^2)$ also has the properties such as it is, like $S(\rho)$, invariant under any unitary transformation $U$ and is a contraction for any completely positive dynamical map.

 \subsection{Dynamical maps with memory}
 \subsubsection{\textit{Memoryless quantum maps}}
 If every two consecutive dynamical maps describing an open system's time evolution are independent, then the system is defined to be evolving under memoryless quantum maps.\ Thus, the Markovian dynamics, say in Eq.\ \eqref{eq:semigroup}, is an example of a memoryless quantum map or channel.\ Processes that can not be considered memoryless, are called \textit{channels with memory} \cite{Kre}.
 
 \subsubsection{\textit{Channels with memory}}
A physical map of this kind is assumed to be \textit{non-anticipatory} \cite{Man_rev}, i.e., subsequent maps do not affect previous ones.\ The part of the environment which remains coupled to the system across some consecutive time steps, is called the `\textit{memory system}' and it leads to the memory effects in the dynamics.\ Numerous examples \cite{Kre,Man_rev} of channels with memory exist.\ For example, in a \textit{localizable} \cite{Gioman} memory channel each sequential map is due to a local unitary coupling with a single multipartite correlated environment, and in a \textit{finite-memory} channel \cite{Bowman}, couplings to the memory systems last for finite times.\ 

 \subsubsection{\textit{Markovian-correlated channel}}
\label{app:Markovian}
An important example of noise with memory is the Markovian-correlated Pauli channel investigated in \cite{Macchia, Macchia_2}.\ \textit{We will consider a modification of their model in this thesis}.\ Say, we have 4 types of channels: $\phi_{i_m}$, ${i_m}={0,1,2,3}$ and they each evolve a qubit in state $\rho_{t}$ by one time step
\begin{equation}
    \rho_{t+1}=\phi_{i_t} (\rho_t) = \bigchi_{i_t}\,\rho_{t}\,\bigchi_{i_t}^{ \dagger}, \text{  with  } \bigchi_{i_t}=\sigma_{i_t}\, U
\end{equation}
where Pauli-$\sigma_i$ and $U$ produce unitary evolutions.\ The Markovian-correlated dynamical map $\Phi_T$ takes an inital state $\rho_0$ to the state
\begin{gather}
    \rho_T= \sum_{\{i_1,\ldots,i_T\}} p_{i_T|i_{T-1}}\ldots p_{i_2|i_1}p_{i_1} \  \phi_{i_T} \circ \phi_{i_{T-1}} \ldots \circ \phi_{i_1}[\rho_0]\\ 
    \text{with}\quad p_{i|j} = (1-\mu)\,p_i + \mu\, \delta_{i,j}.\
\label{eq:2} 
\end{gather}
Here $0\leq\mu\leq1$ corresponds to the \textit{relaxation time} or `\textit{memory parameter}' of the environment with $\mu=0$ for a memory-less case.