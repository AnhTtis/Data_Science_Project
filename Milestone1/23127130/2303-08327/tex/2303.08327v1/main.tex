\documentclass[lettersize,journal]{IEEEtran}
\usepackage{amsmath,amssymb,amsfonts}
\usepackage{graphicx}
\usepackage{float}
\usepackage[caption=false,font=normalsize,labelfont=sf,textfont=sf]{subfig}
\usepackage{color}
\usepackage{cite}
\usepackage{multirow}
\usepackage{algorithm}  
\usepackage{algorithmicx}  
\usepackage{algpseudocode} 
\usepackage{bm}
\usepackage[colorlinks,linkcolor=black,anchorcolor=black,
citecolor=black,urlcolor=black]{hyperref}
\newcommand{\ch}[1]{\textcolor{red}{CH: {#1}}}
\newcommand{\yb}[1]{\textcolor{blue}{{#1}}}
\newcommand{\lyb}[1]{\textcolor{cyan}{YBL:{#1}}}
\usepackage{stfloats}

\usepackage{siunitx}
\sisetup{mode=text,input-ignore={,},range-phrase = {\text{~to~}}}
\DeclareSIUnit\bps{bps}
\sisetup{range-units=single}
\sisetup{per-mode=symbol}
\sisetup{range-phrase=--}
\DeclareSIUnit\Torr{Torr}
\DeclareSIUnit\torr{Torr}
\DeclareSIUnit\sample{Sa}
\newcommand*{\circled}[1]{\lower.7ex\hbox{\tikz\draw (0pt, 0pt)%
  circle (.5em) node {\makebox[1em][c]{\small #1}};}}
\renewcommand{\algorithmicrequire}{\textbf{Input:}}
\renewcommand{\algorithmicensure}{\textbf{Output:}}
\newcommand{\init}{\textbf{Intialization:}}

\ifCLASSINFOpdf

\else

\fi

\hyphenation{op-tical net-works semi-conduc-tor}

\graphicspath{{figures/}}

\begin{document}

\title{Channel Measurement and Coverage Analysis for NIRS-Aided THz Communications in Indoor Environments}

\author{Yuanbo~Li, Yiqin Wang, Yi Chen, Ziming Yu, and Chong~Han,~\IEEEmembership{Member,~IEEE}
\thanks{
Yuanbo Li, Yiqin Wang, and Chong Han are with the Terahertz Wireless Communications (TWC) Laboratory, Shanghai Jiao Tong University, Shanghai, China (e-mail: \{yuanbo.li, wangyiqin, chong.han\}@sjtu.edu.cn).

Yi Chen and Ziming Yu are with Huawei Technologies Co., Ltd, Chengdu, China (e-mail: \{chenyi171, yuziming\}@huawei.com).
}
}
\markboth{IEEE Communications Letters, Submitted in March 2023} 
	{}
	\maketitle
	\thispagestyle{empty}

\begin{abstract}
% With abundant spectrum resources and 20~GHz+ contiguous bandwidth, the THz band (0.1-10 THz) exhibits potential to greatly meet Terabit-per-second data rate requirement in 6G and beyond wireless systems. However, 
Due to large reflection and diffraction losses in the THz band, it is arguable to achieve reliable links in the none-line-of-sight (NLoS) cases. Intelligent reflecting surfaces, although are expected to solve the blockage problem and enhance the system connectivity, suffer from power consumption and operation complexity. 
In this work, non-intelligent reflecting surface (NIRS), which are simply made of costless metal foils and have no signal configuration capability, are adopted to enhance the signal strength and coverage in the THz band. Channel measurements are conducted in typical indoor scenarios at 300~GHz band to validate the effectiveness of the NIRS. Based on the measurement results, the positive influences of the NIRS are studied, including the improvement of path power and coverage. Numerical results show that by invoking the NIRS, the power of reflected/scattering paths can be increased by more than 10 dB. Moreover, with the NIRS, over half area in the measured scenario has doubled received power and the coverage ratio for a 10~dB signal-to-noise ratio threshold is increased by up to 39$\%$.
\end{abstract}
\begin{IEEEkeywords}
 Terahertz communications, Non-intelligent reflecting surface, Channel measurement, Coverage analysis.
\end{IEEEkeywords}
\section{Introduction}
\par \IEEEPARstart{A}{bundant} thrilling applications, such as metaverse, autonomous driving, etc., are expected to be achieved with development of the sixth generation (6G) mobile communication networks~\cite{Saad2020vision}. To undertake the explosively grown data traffic, data rates in 6G need to exceed hundreds of gigabits per second and even Terabits per second~\cite{chen2021terahertz}, where one promising solution is to explore the Terahertz (THz) band. The THz band, ranging from \SI{0.1}{THz} to \SI{10}{THz}, exhibits ultra-large contiguous usable bandwidth (more than tens of GHz), which can address the spectrum scarcity and capacity limitations of current wireless systems~\cite{rappaport2019wireless}.
\par Even though the THz band offers great potentials to achieve ultra high data rates, the coverage ability of THz communications is a concern, especially in none-line-of-sight (NLoS) cases, due to the following reasons~\cite{wu2021interference}. First, with the increase of frequencies, the THz wave experiences larger free space path loss and atmospheric absorption loss compared to cmWave and mmWave bands~\cite{han2015multiray}. Second, it has been reported that the reflection and diffraction losses in the THz band are larger than those in lower frequency bands~\cite{akyildiz2018combating,Jacob2012Diffraction}. Third, THz waves can barely go through the blockages due to the very high penetration loss~\cite{eckhardt2021channel,du2021subthz}. As a result, when line-of-sight (LoS) transmission is not available, the received power could be reduced drastically where reliable THz communications might be hard to achieve. 
\par To improve the coverage ability of THz communications, either active or passive reflectors can be used to enhance the signal strength in the NLoS case. In particular, intelligent reflecting surface (IRS), is a kind of tunable metasurface, which consists of a large number of reflecting elements~\cite{9690477}. The reflecting elements, controlled by a central processor, are able to manipulate the reflecting amplitude and phase shift of the impinging THz waves, for which the THz waves can be redirected and the coverage ability of THz communications is possibly much improved. However, with the large number of reflecting surface, the usage of IRS is complex and costly.
\par In contrast, we study non-intelligent reflecting surface (NIRS) in this work, which are non-tunable surfaces made by materials that have small reflection and scattering losses, such as aluminium foils~\cite{abbasi2021ultra}. Even though several studies have revealed the effectiveness of NIRS in mmWave bands~\cite{Khawaja2020Coverage,El2022Enhancement,Maeng2022Coverage}, according to authors' best knowledge, the influences of NIRS for THz communications have not been fully analyzed in the literature. We attempt to fill this research gap, by studying NIRS in the THz band, based on measurements.
\par The contributions are summarized as follows. Two measurement campaigns are conducted in indoor corridor and hallway scenarios in \SIrange{306}{321}{GHz} in the NLoS case. In both scenarios, the THz channels are measured twice with the same layout, where the NIRS is included or excluded. To observe the effectiveness of NIRS, power of main paths and coverage situations are thoroughly analyzed and compared. Results have shown that by using the NIRS, the average power of main paths is increased by more than 10 dB and the coverage ratio in the NLoS area is increased by up to 39$\%$ for a 10~dB signal-to-noise ratio (SNR) threshold. 
\par The remainder of this letter is organized as follows. In Sec.~\ref{sec:measurement}, the measurement campaigns are introduced in detail. Moreover, the measurement results, including the power of main paths and coverage analysis, are carefully elaborated in Sec.~\ref{sec:results}. Finally, Sec.~\ref{sec:conclude} concludes the letter.
\section{Channel Measurement Campaign}
\label{sec:measurement}
\par In this section, the measurement campaigns for NIRS-aided communications are described, including the sounder system, the measurement set-up and the measurement deployment.
\subsection{Sounder System}
\par For our channel measurement, a vector network analyzer (VNA)-based channel sounder is used, whose measurable bands ranges from \SI{260}{GHz} to \SI{400}{GHz}. Furthermore, both transmitter (Tx) and receiver (Rx) modules are installed on electric carts, lifters and rotators, which support the change of the locations, heights and pointing directions of Tx/Rx. For full-fledged description of our measurement system, readers are encouraged to refer to~\cite{li2022channel,wang2022thz}.
\subsection{Our Measurement Setup}
\begin{table}[tbp]
	\centering 
	\caption{Measurement parameters} 
	\label{tab:parameters}  
	\begin{tabular}{l l}  
		\hline  
		& \\[-6pt]  
		Parameter & Values\\
		\hline
		& \\[-6pt]  
		Frequency band & 306-\SI{321}{GHz}\\

		Bandwidth & \SI{15}{GHz}\\

		Sweeping interval& \SI{2.5}{MHz}\\

		Sweeping points & 6001\\

		Maximum delay& \SI{400}{ns}\\

		Maximum path length& \SI{120}{m}\\

		Time resolution& \SI{66.7}{ps}\\

		Space resolution& \SI{2}{cm}\\

            Tx height & 2 m\\

            Rx height & 1.75 m\\
        
        Antenna gain of Tx & \SI{7}{dBi}\\
        
		Antenna gain of Rx & \SI{25}{dBi}\\

        HPBW of Tx antenna & $30^\circ$\\

		HPBW of Rx antenna & $8^\circ$\\

		Rx azimuth rotation range&[$0^\circ:10^\circ:360^\circ$]\\

		Rx elevation rotation range&[$-20^\circ:10^\circ:20^\circ$]\\

		\hline
	\end{tabular}
	\vspace{-0.5cm}
\end{table}
\par The measurement parameters are summarized in Table.~\ref{tab:parameters}, which are explained in detail as follows. The measured frequency band ranges from \SI{306}{GHz} to \SI{321}{GHz}, across a \SI{15}{GHz} wide band. Using the VNA-based channel sounder, the channel transfer functions (CTFs) are measured with a \SI{2.5}{MHz} sweeping interval and overall 6001 frequency points are measured. As a result, the maximum delay of multi-path components are \SI{400}{ns}, which corresponds to a maximum path length of \SI{120}{m}. The time resolution of our measurement system is \SI{66.7}{ps}, which indicates any multipath components (MPCs) whose length difference is larger than \SI{2}{cm} can be separated. The heights of Tx and Rx are \SI{2}{m} and \SI{1.75}{m}, respectively. Moreover, for large coverage with wide beam, the transmitter is only equipped with a standard waveguide WR2.8, which has \SI{7}{dBi} antenna gain and a $30^\circ$ half-power beamwidth (HPBW). By contrast, the Rx is equipped with a directional antenna with a \SI{25}{dBi} antenna gain and a $8^\circ$ HPBW. To capture MPCs from various directions, the Rx scans the spatial domain with $10^\circ$ angle steps, from $0^\circ$ to $360^\circ$ in the azimuth plane and $-20^\circ$ to $20^\circ$ in the elevation plane.
\subsection{Measurement Deployment}
\begin{figure}[!tbp]
    \centering
    \subfloat[] {
     \label{fig:pic_cor}     
    \includegraphics[width=0.76\columnwidth]{figures/pic_corridor.png}  
    }
    \quad
    \subfloat[] {
     \label{fig:deploy_cor}     
    \includegraphics[width=0.8\columnwidth]{figures/layout_corridor.png}  
    }   
    \caption{The measurement deployment in the corridor. (a) Pictures. (b) Bird's eye view.}
    \label{fig:corridor}
    \vspace{-0.5cm}
\end{figure}
\begin{figure}[!tbp]
    \centering
    \subfloat[] {
     \label{fig:pic_hallway}     
    \includegraphics[width=0.76\columnwidth]{figures/pic_hallway.png}  
    }
    \quad
    \subfloat[] {
     \label{fig:deploy_hallway}     
    \includegraphics[width=0.76\columnwidth]{figures/layout_hallway.png}  
    }   
    \caption{The measurement deployment in the hallway. (a) Pictures. (b) Bird's eye view.}
    \label{fig:hallway}
    \vspace{-0.5cm}
\end{figure}
\par Two measurement campaigns are conducted on the second floor of the Longbin Building in Shanghai Jiao Tong University, as shown in Fig.~\ref{fig:corridor} and Fig.~\ref{fig:hallway}. In the corridor scenario, the transmitter is placed near the room c in the corridor and remains static, while 9 Rx positions are deployed in room f, whose positions are marked in Fig.~\ref{fig:corridor} (b). Moreover, the NIRS are glued around the turning corner, as shown in Fig.~\ref{fig:corridor} (a). The NIRS is made of aluminium foils and has a size around \SI{1.2}{m}$\times$\SI{1.2}{m}.  Similarly, in the hallway scenario, the transmitter locates around the north end of the hallway a, while 12 Rx positions are selected in hallway b in the NLoS case. The NIRS with a size of \SI{1.2}{m}$\times$\SI{1.2}{m} is glued near the turning corner, on the southern wall of hallway b.
\section{Measurement Results and Analysis}
\label{sec:results}
\par In this section, the measurement results are analyzed and discussed. To begin with, the data processing procedure is introduced, including calibration and noise elimination. Moreover, based on the measurement results, propagation analysis are conducted to observe the effects of the NIRS to power of main paths. Last but not least, the coverage enhancement using the NIRS is calculated and analyzed.
\subsection{Data Processing Procedure}
\par The measured raw data includes not only effects of the THz channels, but also effects of cables, amplifiers, etc, where the latter ones need to be eliminated to obtain accurate results through the calibration process. To calibrate the raw data, two measurements are conducted, including a real measurement and a directly-connected measurement. In the real measurement, the Tx/Rx are located in certain positions and the measured results include both the system effects and the CTFs of THz channels, while in the directly-connected measurement, Tx and Rx are directly-connected through waveguides, which only includes the unwanted factors. As a result, the CTF of the THz channel can be expressed as,
\begin{equation}
    H=\frac{S_{21}^{\text{measure}}}{S_{21}^{\text{extra}}S_{21}^{\text{connect}}}
\end{equation}
where $S_{21}^{\text{extra}}$ represents influences of components due to the different set-ups when conducting real measurement and directly-connected measurement. For example, horn antennas are used in the real measurement and not used for directly-connected measurement.
\par Based on the CTFs, the channel impulse responses (CIRs) of the THz channel can be obtained through inverse Fourier transform, i.e., $h=\mathcal{F}^{-1}(H)$. Furthermore, by merging CIRs in different scanning directions into a whole matrix, the power-delay-angular profile (PDAP) can be obtained, as
\begin{equation}
    P_{i,j,k} [\text{dB}]=20\log_{10}|h_{i,j}[k]|
\end{equation}
where $h_{i,j}[k]$ denote the CIR at $k^\text{th}$ temporal sample in $i^\text{th}$ elevation scanning direction and $j^\text{th}$ azimuth scanning direction of Rx.
\par To analyze the behaviors of the multi-path components, the PDAP samples that are larger than the power threshold are regarded as MPCs. Otherwise they are regarded as noise and discarded. The power threshold is set as
\begin{equation}
    P_{\text{th}}~[\text{dB}]=\max(P_m-40, -165)
\end{equation}
where $P_m$ is the maximum of the PDAP. A \SI{40}{dB} range is set here to include the significant MPCs. Moreover, the noise floor value is set at \SI{-165}{dB} on experience, which is sufficient to filter the noise while remaining the significant MPCs.
\subsection{Propagation Analysis With/Without NIRS}
\begin{figure}[!tbp]
    \centering
    \subfloat[] {
     \label{fig:p4wx}     
    \includegraphics[width=0.8\columnwidth]{figures/wx_corridor.png}  
    }
    \quad
    \subfloat[] {
     \label{fig:p4w}     
    \includegraphics[width=0.8\columnwidth]{figures/w_corridor.png}  
    }
    \quad
    \subfloat[] {
     \label{fig:p4}
    \includegraphics[width=0.7\columnwidth]{figures/propa_corri.png}  
    }
    \caption{The PDAP and propagation analysis at Rx point 4 in the corridor. Significant MPCs are marked with stars, different color and numbers. (a) PDAP without NIRS. (b) PDAP with NIRS. (c) Propagation paths.}
    \label{fig:prop1}
    \vspace{-0.5cm}
\end{figure}
\begin{figure}[!tbp]
    \centering
    \subfloat[] {
     \label{fig:p9wx}     
    \includegraphics[width=0.8\columnwidth]{figures/wx_hallway.png}  
    }
    \quad
    \subfloat[] {
     \label{fig:p9w}     
    \includegraphics[width=0.8\columnwidth]{figures/w_hallway.png}  
    }
    \quad
    \subfloat[] {
     \label{fig:p9}
    \includegraphics[width=0.7\columnwidth]{figures/propa_hallway.png}  
    }
    \caption{The PDAP and propagation analysis at Rx point 9 in the hallway. Significant MPCs are marked with stars, different color and numbers. (a) PDAP without NIRS. (b) PDAP with NIRS. (c) Propagation paths.}
    \label{fig:prop2}
    \vspace{-0.5cm}
\end{figure}
\par To observe how the NIRS affects the propagation of THz waves, the power of main paths is compared based on the measured PDAPs, as shown in Fig.~\ref{fig:prop1} and Fig.~\ref{fig:prop2}. The variations of power in the elevation plane are omitted by adding received power from all elevation angles together. This does not affect our observation since the Tx and Rx are at similar heights and MPCs mostly propagate in the horizontal plane. Specifically, the Rx point 4 in the corridor scenario and Rx point 9 in the hallway scenario are taken as two examples, since the path power variation is the most noticeable at these two points.
\subsubsection{Rx 4 in the Corridor Scenario}
The measured PDAPs with/without NIRS and the propagation paths of main MPCs at Rx point 4 in the corridor scenario are shown in Fig.~\ref{fig:prop1}, where the influences of NIRS are mainly observed on two kinds of MPCs, namely, i) scattered MPCs from the eastern wall of room f, labeled as path 1; ii) MPCs scattered on the metal pillars and then scattered on the eastern wall of room f, labeled as path 2. Comparing the PDAPs without and with NIRS, the path loss of path 1 is decreased by \SI{19}{dB}, due to the smaller scattering loss of the NIRS. Moreover, without the NIRS, the twice scattering paths from metal pillars and the eastern wall of room f are very weak, while these MPCs become much stronger by scattering on the NIRS instead of walls, with average path loss decrease around \SI{15}{dB}.
\subsubsection{Rx 9 in the Hallway Scenario}
The measured results at Rx point 9 in the hallway scenario are shown in Fig.~\ref{fig:prop2}. There are mainly two MPCs that are obviously affected by the NIRS. The direct reflected path from the southern wall of hallway b, labeled as path 1, is enhanced by the NIRS with a signal strength increase by \SI{4}{dB}. The enhancement is not significant since this path is close to a specular reflection and the reflection loss is already small without the NIRS. In contrast, the multiple scattering path on doors and walls on sides of hallway a and hallway b, labeled as path 2, becomes much stronger, whose path loss is decreased by \SI{19}{dB}. This reveals that the scattering loss of the metal NIRS is much smaller than the concrete walls.
\subsection{Power Enhancement and Coverage Analysis With/Without NIRS}
\par To evaluate the coverage performance of THz communications with NIRS, the power enhancement and coverage ratio are calculated, as discussed as follows.
\subsubsection{Power Enhancement}
\begin{figure}[!tbp]
    \centering
    \subfloat[]{ 
    \includegraphics[width=0.4\columnwidth]{figures/cov_corridor.png}  
    }
    \subfloat[]{ 
    \includegraphics[width=0.4\columnwidth]{figures/cov_hallway.png}  
    }   
    \caption{The power enhancement by adding the NIRS. (a) Corridor. (b) Hallway.}
    \label{fig:cov}
    \vspace{-0.5cm}
\end{figure}
\par With the measured PDAP, the path loss at each Rx point is calculated by summing power of all MPCs together, as
\begin{equation}
    \text{PL}~[\text{dB}]=-10\log_{10}(\sum_{i,j,k}10^{P_{i,j,k}/10})
\end{equation}
\par Due to large time consumption of channel measurements, only limited Rx positions are measured. Therefore, to analyze the coverage situations in the whole area, the path loss in positions between adjacent Rx locations are obtained through linear interpolation. Moreover, the power enhancement using NIRS is calculated as the difference of path loss before/after adding the NIRS, where the results are shown in Fig.~\ref{fig:cov} and several observations are made as follows. 
\par First, in both the corridor and hallway scenario, received power is enhanced in most areas. Specifically, 63.6$\%$ area in the corridor and 56.8$\%$ area in the hallway obtains power enhancement more than \SI{3}{dB}. This proves the effectiveness of the NIRS. Second, the power enhancement by adding the NIRS is not uniform. At certain Rx positions, such as Rx 4 in the corridor and Rx 9 in the hallway, the power enhancement exceeds \SI{10}{dB}, while in other Rx positions, such as Rx 8 in the corridor and Rx 5 in the hallway, the received power barely changes. Therefore, it is hard to control the NIRS to enhance the received power at certain Rx locations.
\begin{figure}[!tbp]
    \centering
    \includegraphics[width=0.9\columnwidth]{figures/cov_ratio.png}  
    \caption{The measured coverage ratio in indoor environments.}
    \label{fig:cov_ratio}
\end{figure}
\subsubsection{Coverage Ratio}
\par The coverage ratio is defined as the percentage of the area where the received SNR is larger than a certain threshold. Specifically, the received SNR is calculated as
\begin{equation}
    \gamma~[\text{dB}]=\frac{P_tG_tG_r10^{-\text{PL}/10}}{FkTB}
\end{equation}
where $P_t$ is the transmit power. $G_{t/r}$ are the antenna gains of Tx/Rx. Moreover, $F$ is the noise figure of the receiver. Additionally, $k$ is the Boltzmann constant as 1.381$\times10^{-23}~\text{W}\cdot\text{S}/\text{K}$. $T$ and $B$ are the temperature and bandwidth, respectively.
\par According to the reference values in~\cite{Rikkinen2020thz}, the parameters are considered as follows. The transmit power is \SI{13}{dBm}. Moreover, the carrier frequency and bandwidth are set as \SI{315}{GHz} and \SI{15}{GHz}, which are consistent with our measurements. Additionally, the antenna gains of Tx and Rx are selected as \SI{26}{dBi}. The noise figure is set as \SI{10}{dB}.
\par The coverage ratios in indoor environments are shown in Fig.~\ref{fig:cov_ratio}, from which we can make several observations as follows. First, without NIRS, the coverage in the corridor is better than that in the hallway with low SNR thresholds ($<$\SI{5}{dB}), while opposite situation occurs with high SNR thresholds. The reason is that in the corridor, significant scattering from metal pillars results in wide coverage with low SNR. However, since the turning corner in the hallway scenario is more open, there are more area in the hallway that receives high SNR ($>$\SI{5}{dB}). Second, by including the NIRS, the coverage ratio is obviously improved, especially with a high SNR threshold. Specifically, for a \SI{10}{dB} SNR threshold, the coverage ratio is increased by $39\%$ and $10\%$ in the corridor and hallway, respectively.
\section{Conclusion}
\label{sec:conclude}
In this letter, two measurement campaigns in typical indoor scenarios are conducted to testify the effectiveness of non-intelligent reflecting surface (NIRS). 9 Rx positions in the corridor scenario and 12 Rx positions in the hallway scenario are measured. Based on the measured data, main paths in the THz channels are analyzed, where numerical results show that the signal strength of reflected/scattering paths are enhanced by more than \SI{10}{dB}. Moreover, coverage analysis are conducted to fully analyze the efficacy of NIRS. Over half area in the NLoS region benefits from NIRS by doubling the received power. Even though the power enhancement is not uniform and hard to control, the overall coverage ratio could be improved by up to 39$\%$ by adding the NIRS. In general, NIRS offers a convenient way to improve the coverage of THz communications, which can promote the development of 6G mobile networks.
\bibliographystyle{IEEEtran}
\bibliography{main}
\end{document}