\usepackage{enumerate}
\usepackage{comment}
\usepackage{amsfonts,amsmath,bbm,amssymb, amsthm,amssymb, upgreek,mathtools}
\usepackage[linesnumbered,ruled,vlined]{algorithm2e}
\setlength{\algoheightrule}{1pt} 
\setlength{\algotitleheightrule}{0.5pt}
\usepackage{float}
\usepackage{bm}

\usepackage[toc,page,header]{appendix}
\usepackage{minitoc}

\usepackage{amsfonts,amsmath,bbm,amssymb, amsthm,amssymb, upgreek,mathtools}
\usepackage{wrapfig}
\usepackage{comment}
%\usepackage[caption=false]{subfig}

\usepackage[english]{babel}
\usepackage{graphicx}
\usepackage{float, amssymb,amsmath}
\usepackage{ upgreek }
\usepackage{tcolorbox, color}
\usepackage{graphicx}
\usepackage{xcolor}
\usepackage{import}
\usepackage{paracol}

\usepackage{empheq}


% https://tex.stackexchange.com/questions/3468/an-abstract-at-the-start-of-every-chapter
\newenvironment{abstract2}{\rightskip1in\itshape}{}
\newcommand{\chapabstract}[1]{
    \begin{quote}
        \singlespacing\small
        \rule{14cm}{1pt}
        #1
        \vskip-4mm
        \rule{14cm}{1pt}
\end{quote}}


% Math Theorems
\usepackage{thmtools}
\declaretheorem[
style=plain,
name=Theorem,
numberwithin=section
]{thm}
\declaretheorem[
style=plain,
name=Proposition,
numberlike=thm
]{prop}
\declaretheorem[
style=plain,
name=Lemma,
numberlike=thm
]{lemma}
\declaretheorem[
style=plain,
name=Corollary,
numberlike=thm
]{cor}
\declaretheorem[
style=definition,
name=Definition,
numberlike=thm
]{mdef}
\declaretheorem[
style=definition,
name=Example,
numberlike=thm
]{example}
\declaretheorem[
style=definition,
name=Remark,
numberlike=thm
]{rem}
\declaretheorem[
style=plain,
name=Conjecture,
numberlike=thm
]{conj}
\declaretheorem[
style=plain,
name=Assumption,
%numberlike=thm
]{assumpt}

\newtheorem*{thm*}{Theorem}

\newcommand{\theHalgorithm}{\arabic{algorithm}}


%for creating a box around align https://tex.stackexchange.com/questions/109900/how-can-i-box-multiple-aligned-equations
\newcommand*\widefbox[1]{\fbox{\hspace{1.8em}#1\hspace{1.4em}}} 

\newtheorem{assumptionA}{Assumptions}
\renewcommand{\theassumptionA}{A}

\newtheorem{assumptionC}{Assumption}
\renewcommand{\theassumptionC}{C}

\newtheorem{assumptionB}{Assumptions}
\renewcommand{\theassumptionB}{B}

\newtheorem{remark}{Remark}

% https://tex.stackexchange.com/questions/53168/centered-box-text-inside-it-flushed-right
\usepackage{varwidth}

\usepackage{tcolorbox}

\usepackage{subcaption}

\newcommand{\jt}[1]{{\color{blue} #1}}