\subsection{Projection-based Reduced Order Models}

%\subsubsection{Full Order Models} 

We consider high dimensional parameterized dynamical systems resulting from the spatial discretization of  Partial Differential Equations (PDEs). In particular, Full Order Models (FOMs) of the form:
\begin{equation}\label{eq:FOM}
    \frac{\dd \vy}{\dd t} = \vf(\vy, t; \vmu), \qquad \vy(0; \vu) = \vy_0(\vmu)
\end{equation}
where $t \in [0,T]$ is the time variable and $T>0$ the final time, $\vmu \in \R^{n_\mu}$ denotes system parameters, $\vy = \vy(t;\vmu) \in \R^N$ is the state, $\vy_0$ is the parameterized initial condition and $\vf:\R^N \times \R_+ \times \R^{N_\mu} \to \R^{N}$ denotes the nonlinear velocity. 
In particular we are interested in the case where the dimension $N$ of the FOM \eqref{eq:FOM} is much larger than the dimension $n_\mu + 1$ of the solution manifold \[
\mathcal{M} = \{(\vy(t, \mu) \: | \: t \in [0,T], \,\vmu \in \R^{n_\mu} \}.
\] 

We study projection-based reduced order models for the FOM \eqref{eq:FOM} based on a nonlinear approximation of the solution manifold. These methods consider approximations of the solutions $\vy \in \mathcal{M}$ of the form 
\begin{equation}\label{eq:nonLinDec}
    \vy \approx \tilde{\vy} := \vg(z)
\end{equation}
where $\vg: \R^{k} \to \R^N$ is a linear or nonlinear map, often termed \textit{decoder}, and $\vz \in \R^{k}$ is the \textit{reduced order state} or \textit{latent vector}. Note that the $\vg$ generates the nonlinear manifold  
\[
\mathcal{G} = \{ \vg(\vz) \,|\, \vz \in \R^k  \}
\]  
approximation of the solution manifold $\mathcal{M}$.

%\begin{comment}
%For comparisons we will also consider the case where the decoder $\gbf$ is an affine function, leading to the approximation
%\begin{equation}\label{eq:linDec}
%    \ybf \approx \tilde{\ybf} = \gbf(z) := \ybf_{ref} + \Phibf \zbf
%\end{equation}
%where $\ybf_{ref} \in \R^{N}$ is a reference state and $\Phibf \in \R^{N \times k}$ denotes a basis matrix.
%\end{comment}

\subsection{Manifold LSPG ROM}

To obtain the Reduced Order Model (ROM) corresponding to the FOM \eqref{eq:FOM}, and the approximations \eqref{eq:nonLinDec}, we use manifold Least-Squares Petrov-Galerkin Projection (LSPG) \cite{lee2020model}.
This method consists in first applying a time-discretization method to the FOM \eqref{eq:FOM}, and then projecting the resulting residual on the trial manifold identified by the maps \eqref{eq:nonLinDec}.

A temporal discretization of \eqref{eq:FOM} on the time grid $t^0, \dots, t^{N_t}$  yields the FOM O$\Delta$E system 
\begin{equation}\label{eq:ODeltaE}
	\vr^n(\vy^n; \vy^{n-1}, \dots, \vy^{n-\ell}, \vmu) = 0 \quad\text{for}\;\; n = 1, \dots, N_t, 
\end{equation}
where $\vr^n: \R^{N\cdot(\ell+1)} \times \R^{N_\mu} \to \R^N$ is the residual, $\vy^n$ is the state at the $n$-th timestep and $\ell = \ell(n)$ is the width of the stencil of the discretization at time $t^n$. 
 
%For simplicity in this work we only consider linear implicit multisteps methods.
%For simplicity below we will discuss the Backward Euler (BE) method for time discretization, but other implicit multistep methods can also be used. 
%For example, employing a Backward Euler method we are first led to the nonlinear system of equations: 
%The Backward Euler approximation of \eqref{eq:FOM} takes the form
%\begin{equation}\label{eq:BE}
%    \vrBE^n(\vy_n; \vy_{n-1}, \vmu) := \vy_n - \vy_{n-1} - \Delta t \vf_n, \qquad \forall n \in 1, \dots, N_t
%\end{equation}
%where $\Delta t \in \R_+$ denotes the time step, $N_t$ the number of time steps, $\vy_n$ denotes the numerical approximation of $\vy(n \, \Delta t; \vmu)$ and $\vf_n = \vf(\vy_n, n \, \Delta t; \mu )$. 

Substituting then the approximation $\vy_n \leftarrow \tilde{\vy}_n$ defined in \eqref{eq:nonLinDec} and  minimizing the residual \eqref{eq:ODeltaE}, we obtain the time discrete ROM:
\begin{equation}\label{eq:projectODeltaE}
    \vz^n := \argmin_{\vz \in \R^k} \| \vr^n(\vg(\vz); \tilde{\vy}^{n-1}, \dots, \tilde{\vy}^{n-\ell(t^n)}, \vmu) \|_2^2 \quad \text{for}\;\; n = 1, \dots, N_t
\end{equation}
where $\tilde{\vy}^{n} = \vg(\vz^n)$ provides an approximation of the state $\vy^n$.
%
\subsection{Collocation-based hyper-reduction}
%
Even though the ROM states $\{\vz^n\}_n$ have small dimensions $k \ll N$, the residual $\vr$ in \eqref{eq:projectODeltaE} still scales like the dimension $N$ of the FOM and solving \eqref{eq:projectODeltaE} would incur in cost that also scales with $N$. 
Hyper-reduction methods attempt to alleviate this issue by constructing cheap approximations of the residual \eqref{eq:projectODeltaE} \cite{ryckelynck2005priori, farhat2014dimensional, lauzon2022s}.
In particular, \textit{collocation-based hyper-reduction} samples the nonlinear residual \eqref{eq:ODeltaE} at a prescribed set of indices referred to as the \textit{sample mesh}. Let $\mP_c \in \{0,1\}^{n_c \times N}$ be the sampling matrix that picks out the $n_c$ indices of the residual $\vr^n$, then the manifold LSPG \eqref{eq:ODeltaE} with collocation takes the form:
\begin{equation}\label{eq:hyperODeltaE}
    \vz^n := \argmin_{\vz \in \R^k} \| \mP_c \vr^n(\vg(\vz); \tilde{\vy}^{n-1}, \dots, \tilde{\vy}^{n-\ell(t^n)}, \vmu) \|_2^2 \quad \text{for}\;\; n = 1, \dots, N_t.
\end{equation}

 

In many cases of interest, because of the locality of the operators of the original PDE from which the FOM \eqref{eq:FOM} is derived and the locality of the space-discretization used, the states $\{ \tilde{\vy}^{j} \}_{n - \ell(t^n)}^n$ need to be evaluated only on a subset of indices. We refer to this set of indices as \textit{stencil mesh} and observe that in general its cardinality $n_s$ satisfies $n_c < n_s < N$.


Notice that in practice it is not necessary to form the full residual $\vr$ and the states $\{ \vy^j \}_{j=n-\ell}^n$ in \eqref{eq:hyperODeltaE}, as it suffices to keep track of $\vr$ on the sample mesh and of $\{\vy^j\}_{j=n-\ell}^n$ on the stencil mesh. 


