\subsection{Sample and stencil mesh generation}\label{subsec:stencil_mesh}

% TODO remove capitalization from subsections

The sample mesh indices are selected to allow the efficient and accurate approximation of the minimization problems \eqref{eq:ODeltaE} via the reduced minimization problem \eqref{eq:hyperODeltaE}. 
In GNAT \cite{carlberg2013gnat}, for example, a snapshot matrix of residuals together with a greedy method is used to select the sample indices. In this work, motivated by the recent GNAT-SNS  \cite{choi2020sns}, we use instead the solution snapshots $\mX$ with a pivoted QR method \cite{berry2005algorithm} of the POD matrix. This leads to reduced storage requirements and a reduced number of large-scale SVD performed. Algorithm \ref{algo:QRSampling} details the steps in our proposed sampling strategy (in Python notation).  

	Once the indices of the sample mesh are obtained, we generate the indices of the stencil mesh considering those indices that are needed to evaluate the residual on the sample mesh. These indices will be adjacent to the indices on the sample mesh and will be dictated by the spatial discretization used to discretize the original PDE from which \eqref{eq:FOM} is derived.
	%
%\begin{algorithm}
%\DontPrintSemicolon
%\SetAlgoLined
%\KwIn{Number of sample mesh indices $n_c$, POD matrix $\Phi}$.}
%\KwOut{Sample mesh indices $\mathcal{I}_{n_c}$.}
%	Restrict \hat{\mPhi} to \\
%    Compute the QR decomposition of $\mPhi$ with column pivoting: $\mPhi = \mQ \mR \mP^T$\\
%    ${x}_{t+1} \gets \tilde{x}_t - \alpha v_{\tilde{x}_t}$
%\Return{$x_t, G(x_t)$}\;
%\caption{{\sc Sample mesh indices selection via pivoted QR method.}}
%\label{algo:QRSampling}
%\end{algorithm}

\begin{algorithm}
\DontPrintSemicolon
%\SetAlgoLined
\KwIn{Number of sample mesh indices $n_c$, POD matrix $\mPhi$.}
\KwOut{{Sample mesh indices $\mathcal{I}_{n_c}$.}}
	${\mPhi_{n_c}} \gets \mPhi[:,:n_c]$. \\
    Compute the QR decomposition of $\mPhi_{n_c}^T$ with column pivoting: $\mPhi_{n_c}^T = \mQ \mR \mP^T$. \\
    Let $\vp\in \R^N$ the vector such that $\mP = \mI_N[:,\vp]$.\\
    %$\vu \gets [0, 1, \dots, N-1]^T$\\
  %$\vp \gets \mP^T \vu $\\
\Return{$\vp[:n_c]$.}
\caption{{\sc Sample mesh indices selection via pivoted QR method} }
\label{algo:QRSampling}
\end{algorithm}