\subsection{Sample and stencil meshes}

We use the methods described in Section \ref{subsec:stencil_mesh} to construct the sample and stencil meshes. Notice that because of the symmetries of the problem and of the initial conditions, the values of the velocities $u$ and $v$ are always identical, i.e. $u(x,y,t) = v(x,y,t)$ for all $(x,y)\in \Omega , t \in [0,T]$. For this reason, we construct a datamatrix $\mU$ by selecting only the rows corresponding to the variable $u$ in $\mY_{\text{train}}$. We then let $\mPhi^{(u)}$ be the orthogonal (POD) matrix obtained via the Singular Value Decomposition (SVD) of $\mU$. 

We apply Algorithm \ref{algo:QRSampling} with $\mPhi = \mPhi^{(u)}$  and sample mesh sizes $n_c \in \{ 50, 100, 150 \}$. Notice that the algorithm requires only the number of cells $n_c$ to be included in the sample mesh, the number of cells $n_s$ in the stencil mesh depends on the sample mesh and the discretization used. In this case, since we used stencil size of 3 for the space discretization, the inclusion of an index of a cell in the sample mesh implies the inclusion of also the indexes of the four adjacent cells in the stencil mesh. In Figure \ref{fig:meshes} we plot the stencil mesh where the yellow cells denote the cells in the sample mesh. 
 



%\begin{figure*}[ht!]
%   \subfloat[\label{mesh_nc50}]{%
%      \includegraphics[width=0.3\textwidth]{./imgs/stencil_meshes/mesh_nc50.pdf}}
%\hspace{\fill}
%   \subfloat[\label{mesh_nc100} ]{%
%      \includegraphics[width=0.3\textwidth]{./imgs/stencil_meshes/mesh_nc100.pdf}}
%\hspace{\fill}
%   \subfloat[\label{mesh_nc150}]{%
%      \includegraphics[width=0.3\textwidth]{./imgs/stencil_meshes/mesh_nc50.pdf}}\\
%\caption{\label{workflow}The overall approach. (a) figa; (b) Workflow for figb; (c) Workflow for figc.}
%\end{figure*}


%\begin{figure}
%  \begin{subfigure}{0.3\textwidth}
%    \includegraphics[width=\textwidth]{./imgs/stencil_meshes/mesh_nc50.pdf}
%    \caption{Sample mesh size $n_c = 50$} \label{fig:nc50}
%  \end{subfigure}%
%  %\hspace*{\fill}   % maximize separation between the subfigures
%  \begin{subfigure}{0.3\textwidth}
%    \includegraphics[width=\textwidth]{./imgs/stencil_meshes/mesh_nc100.pdf}
%    \caption{Sample mesh size $n_c = 100$} \label{fig:nc100}
%  \end{subfigure}%
%  %\hspace*{\fill}   % maximizeseparation between the subfigures
%  \begin{subfigure}{0.3\textwidth}
%    \includegraphics[width=\textwidth]{./imgs/stencil_meshes/mesh_nc150.pdf}
%    \caption{Sample mesh size $n_c = 150$} \label{fig:nc150}
%  \end{subfigure}
%\caption{Sample and stencil meshes for the 2D Burgers' problem. All the cells shown are in the stencil mesh, while those in yellow are the subset in the sample mesh selected via Algorithm \ref{algo:QRSampling}.} \label{fig:sample_stencil}
%\end{figure}

%\begin{figure*} 
%    \begin{subfigure}{0.333\linewidth}
%    \includegraphics[width=\linewidth]{./imgs/stencil_meshes/mesh_nc50.pdf}
%    \caption{$n_c = 50$} \label{fig:nc50} 
%    \end{subfigure}\hskip -6ex
%    \begin{subfigure}{0.333\linewidth}
%    \includegraphics[width=\linewidth]{./imgs/stencil_meshes/mesh_nc100.pdf}
%    \caption{$n_c = 100$} \label{fig:nc100} 
%    \end{subfigure}\hskip -6ex
%    \begin{subfigure}{0.333\linewidth}
%    \includegraphics[width=\linewidth]{./imgs/stencil_meshes/mesh_nc150.pdf}
%    \caption{$n_c = 150$} \label{fig:nc150} 
%    \end{subfigure}
%\caption{Sample and stencil meshes for a varying number of $n_c$ number of sample mesh points. All the cells shown are in the stencil mesh, while those in yellow are the subset in the sample mesh selected via Algorithm \ref{algo:QRSampling}.} 
%\label{fig:overall}
%\end{figure*}

\begin{figure}
      \centering
      \subcaptionbox{$n_c = 50$ ($n_s = 161$)\label{fig:nc50}}
        {\includegraphics[scale=0.5]{./imgs/stencil_meshes/mesh_nc50.pdf}}
        \hskip -6ex
       \subcaptionbox{$n_c = 100$ ($n_s = 296$)\label{fig:nc50}}
        {\includegraphics[scale=0.5]{./imgs/stencil_meshes/mesh_nc100.pdf}}
      \hskip -6ex
       \subcaptionbox{$n_c = 150$ ($n_s = 440$)\label{fig:nc50}}
        {\includegraphics[scale=0.5]{./imgs/stencil_meshes/mesh_nc150.pdf}}
      \caption{Sample and stencil meshes for a varying number of $n_c$ number of sample mesh points obtained via Algorithm \ref{algo:QRSampling}. The cells shown are those in the stencil mesh, the ones in yellow are those in the sample mesh. With $n_s$ we denote the number of cells in the stencil mesh.}\label{fig:meshes}
\end{figure}
