

We demonstrate our approach on the 2D Viscous Burgers' equation 
\begin{align*}
 \frac{\partial u}{\partial t} + \frac{1}{2} \frac{\partial u^2}{\partial x} + \frac{1}{2} \frac{\partial u v}{\partial y}  &= \frac{1}{Re} \left( \frac{\partial^2 u}{\partial x^2} + \frac{\partial^2 u}{\partial y^2} \right) \\
   \frac{\partial v}{\partial t} + \frac{1}{2} \frac{\partial u v}{\partial x} + \frac{1}{2} \frac{\partial v^2}{\partial y}  &= \frac{1}{Re}\left( \frac{\partial^2 v}{\partial x^2} + \frac{\partial^2 v}{\partial y^2} \right)\\
   (x,y) &\in \Omega = [0,1] \times [0,1],\\
   t &\in [0,2]
 \end{align*}
 with periodic boundary conditions and parameterized initial conditions
 \begin{align*}
 	u(x,y,0) = v(x,y,0) = \mu \sin(2 \pi x) \cdot \sin(2 \pi y) \chi_{[0, 0.5]^2}.
 \end{align*}
 Here $Re$ is the Reynolds number, $u, v$ denote the velocities in the $x$ and $y$ direction, $\mu$ is the scaling parameter for the initial conditions, and for a set $\mathcal{S} \subset \Omega$ we denote by $\chi_{\mathcal{S}}$ is the indicator function over the set $\mathcal{S}$.
 
 \subsection{Full-Order Model}

We use the open-source Python libary Pressio-Demoapps \footnote{https://pressio.github.io/pressio-demoapps/index.html} to solve the previous 2D Burgers' equation. The equation is discretized in space using the finite volume method on a structured orthogonal $60 \times 60$ grid with a stencil size of 3. 

We collect the FOM solutions for different values of parameters $\mu \in \mathcal{D}_{\text{train}} \cup \mathcal{D}_{\text{validation}} \cup \mathcal{D}_{\text{test}}$ using Runge-Kutta4 for time integrator with $N_t = 2000$ time steps of size $\Delta t  = 10^{-3}$. In the first set of experiments we take 
\begin{equation}\label{eq:Dtrain}
\mathcal{D}_{\text{train}} := \{0.16 , 0.208, 0.256, 0.304, 0.352, 0.448, 0.496, 0.544, 0.592, 0.64\}
\end{equation}
to construct the training set. We also consider a smaller training set where $\mu$ is taken in 
\begin{equation}\label{eq:Dtrainp}
	\mathcal{D}_{\text{train}}' := \{0.16, 0.28, 0.52, 0.64\}.
\end{equation}
We use the snapshots corresponding to $\mu \in \mathcal{D}_{\text{validation}} = \{0.4\}$ for validation. To save storage we collect the training snapshots every 4 steps, leading to a total number of 5003 training datapoints for $\mathcal{D}_{\text{train}}$ and 2001 for $\mathcal{D}_{\text{train}}'$. The test set is instead constructed by taking 34 equispaced test parameters from the range $[0.06, 0.72]$. Note that some of these parameters are outside the training interval $[0.16, 0.64]$ and will be used to study the extrapolation performance of the proposed method.


