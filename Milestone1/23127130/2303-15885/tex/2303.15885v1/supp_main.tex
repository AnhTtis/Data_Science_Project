\documentclass[journal]{IEEEtran}

\usepackage{amsmath, amsfonts, bbm, breqn, mathtools, mathrsfs, bm}
\usepackage{algorithm, algpseudocode}
\usepackage{multirow, diagbox, makecell}
\usepackage{subcaption}
\usepackage{float}
\usepackage{array}
\usepackage{stfloats}
\usepackage{graphics}
\usepackage{graphicx, adjustbox}
\usepackage{epsfig}
\usepackage{soul}
\usepackage{cite}
\usepackage[colorlinks, 
            linkcolor=red,      
            anchorcolor=red,  
            citecolor=blue,       
            ]{hyperref}
\usepackage{changes}
\usepackage[numbers]{natbib}
\usepackage{xr}
\externaldocument{Main}

\renewcommand{\thesection}{S.\Roman{section}} 
\renewcommand{\thesubsection}{\thesection.\Alph{subsection}}
\renewcommand{\theequation}{S\arabic{equation}}
\renewcommand{\thefigure}{S\arabic{figure}}
 \makeatletter
 \makeatletter \renewcommand{\fnum@table}
 {\tablename~S\thetable}
 \makeatother
 
 \makeatletter
\makeatletter \renewcommand{\fnum@figure} {\figurename~S\thefigure} 
\makeatother

\renewcommand{\bibnumfmt}[1]{[S#1]}
% citenumfont command adds S to all numbers
\renewcommand{\citenumfont}[1]{\textit{S#1}}

\begin{document}

    \title{Supplementary Material - \\Optimal Coded Diffraction Patterns for Practical Phase Retrieval}

	\author{\IEEEauthorblockN{Qiuliang Ye\IEEEauthorrefmark{1},~\IEEEmembership{Student Member,~IEEE,}
	Bingo Wing-Kuen Ling,~\IEEEmembership{Senior Member,~IEEE,}
 	 Li-Wen Wang,~\IEEEmembership{ Member,~IEEE,} and Daniel Pak-Kong Lun\IEEEauthorrefmark{1},~\IEEEmembership{Senior Member,~IEEE}}
    
    \thanks{Qiuliang Ye, Daniel Pak-Kong Lun, and  Li-Wen Wang are with the Department of Electronic and Information Engineering,
    The Hong Kong Polytechnic University,Kowloon, Hong Kong SAR, China. Bingo Wing-Kuen Ling is with the School of Information Engineering, Guangdong University of Technology, Guangdong Province, China. 
    % (email: \href{mailto:qiu-liang.ye@connect.polyu.hk}{qiu-liang.ye@connect.polyu.hk}; \href{mailto:liwen.wang@connect.polyu.hk}{liwen.wang@connect.polyu.hk}; \href{mailto:pak.kong.lun@polyu.edu.hk}{pak.kong.lun@polyu.edu.hk}).
    }% <-this % stops a space
    \thanks{ \IEEEauthorrefmark{1} Corresponding author: Qiuliang Ye (\href{mailto:qiu-liang.ye@connect.polyu.hk}{qiu-liang.ye@connect.polyu.hk}; \href{mailto:qiustander@gmail.com}{qiustander@gmail.com}) and Daniel Pak-Kong Lun (\href{mailto:enpklun@polyu.edu.hk}{enpklun@polyu.edu.hk}). }
 	 }

\maketitle

\section{Detailed Derivation of Reconstruction Algorithm}

We provide the detailed derivation of the ADMM-based reconstruction algorithm in the Appendix of the main text  in the following paragraphs. Recall that the original optimization problem defined in the main text is:
	\begin{equation}
		\begin{aligned}
			& \min _{\mathbf{x} \in \mathcal{C}^{n}} \frac{1}{2} \mathcal{D}\left(|\mathbf{z}|^{2}, \mathbf{\mathcal{X}}\right) + \frac{\alpha}{2} \mathcal{R}_1(\mathbf{d}) + \frac{\beta}{2} \mathcal{R}_2(\mathbf{\Phi}), \\
			\text { s.t. } & \mathbf{z} = \mathcal{F}(\mathbf{T} \circ \mathbf{x}),\ \mathbf{x} = \mathbf{d}\circ \mathbf{\Phi}, \ |\mathbf{\Phi}| = 1.
		\end{aligned}
	\end{equation}
	
The augmented Lagrangian function \cite{Boyd2011DistributedOA} of the nonlinear optimization problem can be formulated as follows:
	\begin{equation}
		\begin{aligned}
		\label{Eq:PRLagrangian}
			\min\limits_{\mathbf{x}, \mathbf{d}, \mathbf{\Phi}, \mathbf{z}, \xi , \zeta } \mathcal{L} &(\mathbf{x}, \mathbf{z}, \mathbf{d}, \mathbf{\Phi}, \xi, \zeta) =   \frac{1}{2} \mathcal{D}\left(|\mathbf{z}|^{2}, \mathbf{\mathcal{X}}\right) + \frac{\alpha}{2} \mathcal{R}_1(\mathbf{d}) \\ & + \frac{\beta}{2} \mathcal{R}_2(\mathbf{\Phi}) + Re\langle\xi, \mathbf{z}-\mathcal{A}(\mathbf{x})\rangle + Re\langle \zeta, \mathbf{d}\circ \mathbf{\Phi} - \mathbf{x}\rangle \\ & + \frac{r_{1}}{2} \left\| \mathbf{z} -
			\mathcal{A}(\mathbf{x})\right\|_{2}^{2}  + \frac{r_{2}}{2}\left\| \mathbf{d}\circ \mathbf{\Phi} - \mathbf{x} \right\|_{2}^{2},\\
		\end{aligned}
	\end{equation}
	where $\xi$, $\zeta$, $r_1$, $r_2$ represent the Lagrangian multipliers corresponding to $\mathbf{z}$ and $\mathbf{d}\circ \mathbf{\Phi}$, fixed penalty parameters for $\mathbf{z}$ and $\mathbf{d}\circ \mathbf{\Phi}$, respectively. $Re\langle \cdot \rangle$ denotes the projection onto the real axis. The $Re\langle \cdot \rangle$ function can be merged into the $\|\cdot\|^2$ terms. Therefore, \eqref{Eq:PRLagrangian} can be rewritten as:
	\begin{equation}
	    \begin{aligned}
	    \label{Eq:PRLagrangian2}
    	    & \mathcal{L}  (\mathbf{x}, \mathbf{z}, \mathbf{d}, \mathbf{\Phi}, \xi, \zeta) = \frac{1}{2} \mathcal{D}\left(|\mathbf{z}|.^{2}, \mathbf{y}\right) + \frac{\alpha}{2} \mathcal{R}_1(\mathbf{d}) + \frac{\beta}{2} \mathcal{R}_2(\mathbf{\Phi}) \\ &  + \frac{r_{1}}{2} \left\| \mathbf{z} - 
    		\mathcal{A}(\mathbf{x}) +  \frac{\xi}{r_1}\right\|_{2}^{2}   +\frac{r_{2}}{2}\left\|  \mathbf{d}\circ \mathbf{\Phi} - \mathbf{x}  + \frac{\zeta}{r_2} \right\|_{2}^{2}.
		\end{aligned}
	\end{equation}

    The Wirtinger calculus \cite{09WritingerCalculus} is involved in solving the $\mathbf{x}$, $\mathbf{z}$, $\mathbf{d}$, and $\mathbf{\Phi}$ sub-problems in  \eqref{Eq:PRLagrangian2} since they are all complex-valued variables. 
	
	Definition $S1$ \textit{Writinger calculus} \cite{09WritingerCalculus}: consider $\mathbb{C} \equiv \mathbb{R}^{2}$ $\{(x, y) \mid x, y \in \mathbb{R}\}$, the Wirtinger derivative is defined as the partial derivative w.r.t. $z=x+i y$ :
	$$
	\left\{\begin{array}{l}
		\frac{\partial}{\partial z}=\frac{1}{2}\left(\frac{\partial}{\partial x}-i \frac{\partial}{\partial y}\right), \\
		\frac{\partial}{\partial \bar{z}}=\frac{1}{2}\left(\frac{\partial}{\partial x}+i \frac{\partial}{\partial y}\right).
	\end{array}\right.
	$$
	For a real-valued function $f(z)$, the Wirtinger derivative w.r.t. $z$ is: $\frac{\partial f(z)}{\partial z}=\frac{1}{2}\left(\frac{\partial}{\partial x}+i \frac{\partial}{\partial y}\right)^{*}=\left(\frac{\partial f(z)}{\partial \bar{z}}\right)^{*}$. In the gradient descent-based algorithm, the update of $z$ has two directions: $\left[\begin{array}{l}\boldsymbol{z}^{\tau+1} \\ \overline{\boldsymbol{z}}^{\tau+1}\end{array}\right]=\left[\begin{array}{l}\boldsymbol{z}^{\tau} \\ \overline{\boldsymbol{z}}^{\tau}\end{array}\right]-\mu\left[\begin{array}{l}\left.(\partial f / \partial \boldsymbol{z})^{*}\right|_{z=z^{\tau}} \\ \left.(\partial f / \partial \overline{\boldsymbol{z}})^{*}\right|_{z=z^{\tau}}\end{array}\right]$. Since $z^{\tau+1}$ is conjugate to $\left(\bar{z}^{\tau+1}\right)^{*}$. It is enough to compute $z^{\tau+1}$ for updating the $z$.
	
	\subsection{Details on $\mathbf{x}$ problem}
	
	The update formula of $\mathbf{x}$ is denoted as:
	\begin{equation}
	\begin{aligned}
		\label{Eq:xproblem}
		\resizebox{\linewidth}{!}{$\mathbf{x}^{(n+1)} =  \mathop{argmin}\limits_{\mathbf{x} \in \mathbb{C}^N}\ \frac{r_{1}}{2}\left\|\mathcal{A}(\mathbf{x})-\frac{\zeta^{(n)}}{r_{1}}-\mathbf{z}^{(n)}\right\|_{2}^{2}  + \frac{r_{2}}{2}\left\|\mathbf{x}- \mathbf{d}^{(n)}\circ \mathbf{\Phi}^{(n)} + \frac{\xi^{(n)}}{r_{2}}\right\|_{2}^{2}.$}
	\end{aligned}
	\end{equation}
	
	As can be seen, the objective function is a real-valued mean square error. According to Definition $S1$, the optimization problem can be solved via Wirtinger gradient descent:
	\begin{equation}
	    \begin{aligned}
	    \label{Eq:xprobderivation}
	    &\resizebox{\linewidth}{!}{$\frac{\partial}{\partial \mathbf{x}}\ \frac{r_{1}}{2}\left\|\mathcal{A}(\mathbf{x})-\frac{\zeta^{(n)}}{r_{1}}-\mathbf{z}^{(n)}\right\|_{2}^{2}  + \frac{r_{2}}{2}\left\|\mathbf{x}- \mathbf{d}^{(n)}\circ \mathbf{\Phi}^{(n)} + \frac{\xi^{(n)}}{r_{2}}\right\|_{2}^{2}$}\\ 	&\resizebox{\linewidth}{!}{$ = r_1\mathcal{A}^H\left(\mathcal{A}(\mathbf{x})- \frac{\zeta^{(n)}}{r_{1}}-\mathbf{z}^{(n)}\right) + r_2 \left( \mathbf{x}- \mathbf{d}^{(n)}\circ \mathbf{\Phi}^{(n)} + \frac{\xi^{(n)}}{r_{2}}\right).$}\\
	    \end{aligned}
	\end{equation}
	
	When the optimization has the minimum value, the first-order derivative is equal to zero. That is, we have the following derivation:
	\begin{equation}
	    \begin{aligned}
    	&\resizebox{\linewidth}{!}{$ r_1\mathcal{A}^H\left(\mathcal{A}(\mathbf{x})- \frac{\zeta^{(n)}}{r_{1}}-\mathbf{z}^{(n)}\right) + r_2 \left( \mathbf{x}- \mathbf{d}^{(n)}\circ \mathbf{\Phi}^{(n)} + \frac{\xi^{(n)}}{r_{2}}\right) = 0$}\\
    	&\resizebox{\linewidth}{!}{$\left(r_1\mathcal{A}^H\mathcal{A} + r_2 \mathbf{I}\right)\mathbf{x} = r_1\mathcal{A}^H\left( \frac{\zeta^{(n)}}{r_{1}} + \mathbf{z}^{(n)}\right) + r_2 \left( \mathbf{d}^{(n)}\circ \mathbf{\Phi}^{(n)} - \frac{\xi^{(n)}}{r_{2}}\right)$}\\
    	&\resizebox{\linewidth}{!}{$\mathbf{x}^{(n+1)} = \left(r_1\mathcal{A}^H\mathcal{A} + r_2 \mathbf{I}\right)^{-1}\left[r_1\mathcal{A}^H\left( \frac{\zeta^{(n)}}{r_{1}} + \mathbf{z}^{(n)}\right) + r_2 \left( \mathbf{d}^{(n)}\circ \mathbf{\Phi}^{(n)} - \frac{\xi^{(n)}}{r_{2}}\right)\right],$}
	    \end{aligned}
	\end{equation}
    where $\mathbf{I}$ is the identity matrix and $\mathcal{A}^H\mathcal{A} = diag\left(\sum_l \mathbf{I}_l^H\mathbf{I}_l\right) $.
    
	%%%%%%%%%%%%%z problem%%%%%%%%%%%%%
	\subsection{Details on $\mathbf{z}$ problem}
	
	The update formula of  $\mathbf{z}$  is expressed as:
	\begin{equation}
	\label{Eq:zproblem}
		\resizebox{\linewidth}{!}{$\mathbf{z}^{(n+1)} =  \mathop{argmin}\limits_{\mathbf{z} \in \mathbb{C}^{N\times L}}\ \frac{1}{2}\mathcal{D}\left(|\mathbf{z}|^2, \mathbf{\mathcal{X}}\right) + \frac{r_{1}}{2}\left\|\mathbf{z} - \mathcal{A}(\mathbf{x}^{(n+1)})-\frac{\zeta^{(n)}}{r_{1}}\right\|_{2}^{2}$,}
	\end{equation} 
	where $\mathcal{D}$ denotes the data loss function for reducing the discrepancy between the estimated and measured intensity. Usually, the Maximum A Posterior (MAP) data loss function is used in the microscope systems to remove the Poisson noise caused by the low-light intensity measurements in mid- and high-frequency areas \cite{chang2018total}. On the other hand, the dominant disturbance is Gaussian noise for low-frequency areas with normal-light environments. Thus, Poisson and Gaussian noise coexist in the acquired Fourier intensity measurements. Thus, the Poisson-Gaussian noise is the dominant noise. In order to remove Poisson-Gaussian noise,
	we adopted the mean square error (MSE) as the data loss function, which was treated as an effective Poisson-Gaussian denoising method \cite{makitalo2012optimal}. The optimization problem of \eqref{Eq:zproblem} thus becomes:
		\begin{equation}
		\resizebox{\linewidth}{!}{$\mathbf{z}^{(n+1)} =  \mathop{argmin}\limits_{\mathbf{z} \in \mathbb{C}^{N\times L}}\  \frac{1}{2}\left\||\mathbf{z}|^2 - \mathbf{\mathcal{X}}\right\|^2 + \frac{r_{1}}{2}\left\|\mathbf{z} - \mathcal{A}(\mathbf{x}^{(n+1)})-\frac{\zeta^{(n)}}{r_{1}}\right\|_{2}^{2}.$}
	\end{equation} 
	
	However, $\left\||\mathbf{z}|^2 - \mathbf{\mathcal{X}}\right\|^2$ is for real-valued variable $|\mathbf{z}|$ while $\left\|\mathbf{z} - \mathcal{A}(\mathbf{x}^{(n+1})-\frac{\zeta^{n}}{r_{1}}\right\|_{2}^{2}$ is aimed at the complex-valued variable $\mathbf{z}$. In order to resolve this problem, we split the magnitude and phase parts and dealt with them respectively. It is because the angle of two vectors is the same when the MSE term is minimized. That is, $\angle z_{i,j} =  \angle \left(\mathcal{A}(\mathbf{x})_{i,j}^{(n+1)} + \frac{\zeta_{i,j}^{(n)}}{r_{1}}\right), i,j = 1, \dots, N\times N \times L$. Therefore, it is only necessary to optimize for the magnitude part $|\mathbf{z}|$. We thus updated $\mathbf{z}$ with the derivation:
	\begin{equation}
		\resizebox{\linewidth}{!}{$\mathbf{z}^{n+1} = \frac{1}{r_1 + 1}\left( r_1 \left|\mathcal{A}(\mathbf{x}^{(n+1)}) + \frac{\zeta^{(n)}}{r_{1}}\right| + \sqrt{\mathbf{\mathcal{X}}}\right)\angle \left(\mathcal{A}(\mathbf{x}^{(n+1)}) + \frac{\zeta^{(n)}_i}{r_{1}}\right)$}
	\end{equation}

	\subsection{Convergence Analysis}
	
	In this section, the convergence property of the reconstruction algorithm is analyzed. The algorithm converges to a saddle point that satisfies Karush-Kuhn-Tucker (KKT) conditions for the typical nonconvex optimization problem. Similarly, \cite{chang2018total} provided convergence analysis of an ADMM algorithm with total variation regularization (Theorem $ 3.4 $ in \cite{chang2018total}) for the complex-valued $ x $, though we deal with the situation of magnitude and phase regularization individually.
	
	The boundness condition for the $ \mathbf{d} $  and $ \mathbf{\Phi} $ can be satisfied with:
	\begin{equation}
		\lim \limits_{n \rightarrow +\infty} \mathbf{x}^n - \mathbf{d}^n \circ \mathbf{\Phi}^n = 0.
	\end{equation}
	Since $ \mathbf{x} $ is bounded in each iteration, the periodic phase part $ \mathbf{\Phi} $ is guaranteed as bounded. Hence the amplitude part $ \mathbf{d} $ is also bounded. The accumulation point $( \mathbf{d}, \mathbf{\Phi} )$ also satisfies the KKT condition (the other variables are referred in Theorem $ 3.4 $ in \cite{chang2018total}) since the iterative solution of two variables could be acquired through soft shrinkage algorithm \cite{Yang2009TVL1} along with the bounded $ \mathbf{x} $. Hence the proof of convergence of the reconstruction algorithm is completed. 
	
	\section{Calibration of the Hardware Setup}
	
	In this section, we calibrated the phase-only SLM using the interferometric configuration \cite{app9102012}. The interferometric approach can be used for recording and reconstructing amplitude and phase by mathematical techniques. For example, under the same direction of polarization, the interference fringes of two plane waves are described as:
	\begin{equation}
    I(x, y)=I_{1}(x, y)+I_{2}(x, y)+2 \sqrt{I_{1}(x, y) \cdot I_{2}(x, y)} \cdot \cos \Delta,
    \end{equation}
    where $I_1(x, y)$ and $I_2(x, y)$ denote the intensities on the measurement plane of two light beams, $I(x, y)$ refers to the interference pattern and $\Delta$ represents the phase difference between two beams. The interference pattern can be realized with different parts of SLM's active areas. 
    
    We applied the self-reference approach that split the SLM plane into two zones (half-half) to behave as a phase-shift zone and reference zone, respectively \cite{Bergeron:95}. The optical path of the calibration system is shown in Fig. \ref{fig:OpticalPath_calibration}. Specifically, we uploaded a series of patterns with a constant grayscale value (from $0$ to $255$) on the modulated area. Then the wavefront transmitting the region will be adjusted with a specific phase shift according to the assigned grayscale on the SLM areas. At the same time, the addressed value on the reference regions remains unchanged. To create the interference patterns effectively, we placed a pinhole board behind the SLM device, which is an opaque card with two transparent pinholes that only allow the light beam to pass through the pinholes. We carefully designed the pinhole board such that each pinhole locates at the modulated zone and reference zone, respectively.
    
    	\begin{figure}[htb]		
    		\centering\includegraphics[width=\linewidth]{Figure/Fig_Opticalpath_Calibration.pdf}
    		\caption{Configuration of the phase calibration with a reflective SLM.}
    		\label{fig:OpticalPath_calibration}
    	\end{figure}
    
    An example of the interference pattern is shown in Fig. \ref{fig:interference}(a) and the variation of phase modulation with $8$-bit gray scale values is presented in  Fig. \ref{fig:interference}(b). As can be seen, the effective phase modulation range exceeds $2\pi$ for the Holoeye SLM in the wavelength of $632nm$. In order to prevent the phase wrapping, we only utilize the phase shift with the range of $[0, 2\pi]$ that corresponds to $[30, 230]$ of the grayscale values. Then we input the corresponding phase shift of each grayscale value into a LUT file and upload to the SLM. 
    
    \begin{figure}[htb]

    \begin{subfigure}{\linewidth}	
	    \centering
        \begin{adjustbox}{width=0.7\linewidth,center}
		\centering\includegraphics[width=\linewidth]{Figure/Fig_Pattern.pdf}
    	\end{adjustbox}
        \vspace*{-4mm}
    	\caption{}
    \end{subfigure}

    \begin{subfigure}{\linewidth}	
	    \centering
        \begin{adjustbox}{width=\linewidth,center}
		\centering\includegraphics[width=\linewidth]{Figure/Fig_Calibration.pdf}
    	\end{adjustbox}
        \vspace*{-4mm}
    	\caption{}
    \end{subfigure}

    \vspace{-0.2cm}
    \caption{(a) An example of the interference pattern. (b) Experimentally calibrated intensity modulation curv in the grayscale range of $[0, 255]$ for the SLM.}    
    \label{fig:interference}
\end{figure}
	

	
% references section	
\bibliographystyle{IEEEtran}
\bibliography{Reference_supp}

\end{document}
