\documentclass[
prb,
twocolumn,
superscriptaddress,
%groupedaddress,
%unsortedaddress,
%runinaddress,
%frontmatterverbose, 
%preprint,
%preprintnumbers,
%nofootinbib,
%nobibnotes,
%bibnotes,
 amsmath,amssymb,
 aps,
%pra,
%prb,
%rmp,
%prstab,
%prstper,
%floatfix,
longbibliography,
]{revtex4-2}

% \bibliographystyle{apsrev4-2}

\usepackage[utf8]{inputenc}
\usepackage[T1]{fontenc}
\usepackage{graphicx}% Include figure files
\usepackage{dcolumn}% Align table columns on decimal point
\usepackage{bm}% bold math
\usepackage{hyperref}% add hypertext capabilities
\usepackage{comment}
\usepackage{xcolor}
\usepackage{bm}

%\usepackage[mathlines]{lineno}% Enable numbering of text and display math
%\linenumbers\relax % Commence numbering lines

%\usepackage[showframe,%Uncomment any one of the following lines to test 
%%scale=0.7, marginratio={1:1, 2:3}, ignoreall,% default settings
%%text={7in,10in},centering,
%%margin=1.5in,
%%total={6.5in,8.75in}, top=1.2in, left=0.9in, includefoot,
%%height=10in,a5paper,hmargin={3cm,0.8in},
%]{geometry}

\newcommand*{\state}{\ensuremath{\nu}}
\newcommand*{\statep}{\ensuremath{ {\nu'} }}
\setcitestyle{super}
\begin{document}

%\preprint{APS/123-QED}


\title{\vspace*{-3mm}Viscous heat backflow and temperature resonances in graphite\vspace*{-1mm}}

\author{Jan Dragašević}
\altaffiliation[\hspace*{-1mm}\vspace*{-8mm}Current address: ]{Physics Department, University of Zagreb (HR)}

\author{Michele Simoncelli}
\email{ms2855@cam.ac.uk}
\affiliation{Theory of Condensed Matter Group, Cavendish Laboratory, University of Cambridge (UK)}


\begin{abstract}
Graphite's extreme ability to conduct heat is technologically relevant for a variety of applications in electronics and thermal management, and scientifically intriguing for violating Fourier's law. Striking non-diffusive phenomena such as temperature waves---where heat transiently backflows from cooler to warmer regions---have recently been observed in experiments.
Microscopic theoretical analyses 
suggested that these phenomena emerge when momentum-conserving phonon collisions dominate over momentum-relaxing ones,  hinting that under these circumstances heat can behave hydrodynamic-like, and raising fundamental questions on the existence, possible practical detection, and technological exploitation of macroscopic hallmarks of viscous heat flow.
Here we rely on the viscous heat equations [Phys. Rev. X 10, 011019 (2020)] parametrized from first principles to show that viscous heat backflow can emerge in graphite. We predict the appearance of heat vortices in the steady-state regime, and we rationalize recent measurements of temperature waves. 
We discuss analogies and differences between the temperature waves emerging from the viscous heat equations and from the inviscid dual-phase-lag equation, showing that it is necessary to consider viscous effects to quantitatively rationalize experiments.
Finally, we show that viscous temperature waves can be driven to resonance, proposing setups for their amplification and detection.  
\end{abstract}


%\keywords{second sound, lattice cooling, Heat hydrodynamics, Computational physics}



\maketitle


Graphite is among the best conductors of heat: experiments reported a room-temperature in-plane conductivity of 2000 W/mK  \cite{schmidt_pulse_2008,balandin_thermal_2011,fugallo_thermal_2014} in bulk specimens, and even higher values in thin samples  \cite{machida_phonon_2020}. On the one hand, this extreme ability to conduct heat holds great potential for a variety of thermal-management applications in e.g. electronics and phononics  \cite{qian_phonon-engineered_2021,machida_phonon_2020,chen_non-fourier_2021}.
On the other hand, the technological exploitation of graphite's outstanding thermal properties requires achieving control over the non-diffusive behavior of heat observed in recent experiments; specifically, temperature waves where heat transiently backflows from cooler to warmer regions, have been recently measured at temperatures around 80-100 K in samples at natural isotopic abundance\cite{Huberman2019,Jeong2021}, and even up to about 200 K in isotopically pure samples  \cite{Ding2022}.

Hitherto, the theoretical investigation of these phenomena has mostly been done relying on the linearized  Peierls-Boltzmann equation (LBTE)  \cite{peierls1955quantum} and on first-principles simulations \cite{lindsay_perspective_2019,qian_phonon-engineered_2021,chen_non-fourier_2021}. 
These works have provided microscopic insights on non-diffusive heat transport phenomena in graphite \cite{fugallo_thermal_2014,app_Ding2018,guo_size_2021,Ding2022,li_reexamination_2022,huang_mapping_2022} and graphene \cite{app_Cepellotti2015,cepellotti_thermal_2016,fugallo_thermal_2014,app_Lee2015,majee_dynamical_2018,raya-moreno_hydrodynamic_2022,guo_phonon_2021,han_is_2023}, quantitatively discussing how the predominance of momentum-conserving (normal) phonons' collisions over momentum-relaxing (Umklapp) phonons' collisions can give rise to a non-diffusive, hydrodynamic-like behavior for heat, with hallmarks such as second sound \cite{app_Cepellotti2015,Huberman2019,Ding2022} and Poiseuille-like heat flow \cite{app_Lee2015,cepellotti_boltzmann_2017,sendra_hydrodynamic_2022,li_reexamination_2022,huang_observation_2022}.
However, the complexity of the microscopic (integro-differential) LBTE makes it unpractical to e.g. design thermal management in devices where non-Fourier hydrodynamic transport may appear. 
Recent research has been focused on developing and testing mesoscopic models (partial-differential equations having reduced complexity compared to the integro-differential microscopic equations) for heat hydrodynamics \cite{li_role_2018,guo_nonequilibrium_2018,shang_heat_2020,Simoncelli2020,sendra_derivation_2021} that can be parametrized from first-principles. 

Here, we employ the mesoscopic viscous heat equations (VHE) \cite{Simoncelli2020} and first-principles calculations to rationalize the conditions determining the emergence of hydrodynamic heat backflow in graphite with natural or reduced isotopic-mass disorder, in both the steady-state and transient domain. We show that it is necessary to have finite viscosity to observe backflow in the steady-state regime, while in the transient regime it is possible to have backflow also in the inviscid limit.
In particular, we show that in the limit of vanishing viscosity the VHE reduce to the dual-phase-lag equation\cite{joseph_heat_1989,tzou_unified_1995} for temperature waves (DPLE), which also encompasses Cattaneo's second-sound\cite{cattaneo1958form} equation as special case.
However, we show that transient heat backflow is significantly affected by the thermal viscosity, and 
we show that accounting for the thermal viscosity allows to rationalize the hydrodynamic relaxation timescales\cite{Jeong2021} and lengthscales\cite{Huberman2019} measured in recent experiments.
We show that temperature waves can be driven to resonance, and the consequent amplification can greatly facilitate their experimental detection.

\begin{figure*}
\vspace*{-3mm}
\includegraphics[width=\textwidth]{fig1_REV.png}
\caption{\label{fig:1_vortex}
\textbf{Signature of viscous heat backflow in graphite.} 
In-plane ($x{-}y$) heat flux (streamlines) and temperature profile (colormap) for a tunnel-chamber device made of graphite. Panel \textbf{a} (\textbf{b}) shows the solution of Fourier's equation (VHE) in the presence of a temperature gradient applied at the tunnel's boundaries ($70{\pm}25$ K at $y{=}{\pm} 2.5\mu m$), and 
considering the other boundaries as adiabatic (\textit{i.e.} $\nabla T{\cdot} \bm{\hat n}{=}0$, where $\bm{\hat n}$ is the versor orthogonal to the boundary) and, in the VHE, "slipping" ($\bm{u}{\cdot} \bm{\hat n}{=}0$).
In Fourier's case (\textbf{a}), the direction of the temperature gradient in the chamber mirrors that in the tunnel ($T_A{<}T_B$).
In contrast, the VHE (\textbf{b}) account for an additional viscous component for the heat flux---not directly related to the temperature gradient, see text---allowing the emergence of viscous backflow and temperature gradient in the chamber reversed compared to the tunnel ($T_A{>}T_B$).
Panel~\textbf{c)}, vorticity of the VHE heat flux, $\nabla {\times}\bm{Q}^{\rm TOT}$; the vorticity for Fourier's flux (not reported) is trivially zero.\\[-8mm]}
\end{figure*}

We start by summarizing the salient features of the VHE \cite{Simoncelli2020}, a mesoscopic model for thermal transport that generalizes Fourier's law to the hydrodynamic regime, given by the following partial-differential equations:\\[-6mm]
\begin{widetext}
\vspace*{-7mm}
\begin{align}
&C\frac{\partial T({\bm r}, t)}{\partial t} + \sum_{i,j=1}^{3} \alpha^{ij}\frac{\partial u^j({\bm r}, t)}{\partial r^i} - \sum_{i,j=1}^{3} \kappa^{ij} \frac{\partial^2 T ({\bm r}, t)}{\partial r^i \partial r^j} = \dot{q}({\bm r}, t),
 \label{viscous_heat_T}\\
&A^i\frac{\partial u^i({\bm r}, t)}{\partial t} +  \sum_{j=1}^{3}
\beta^{ij}
\frac{\partial T({\bm r}, t)}{\partial r^j} - \sum_{j,k,l=1}^{3} \mu^{ijkl} \frac{\partial^2 u^k({\bm r}, t)}{\partial r^j \partial r^l} = - \sum_{j=1}^{3} \gamma^{ij} u^j({\bm r}, t).\label{viscous_heat_U}\\[-10mm]\nonumber
\end{align}
\end{widetext}

In these equations, $T({\bm r}, t)$ is the local temperature, $\bm{u}({\bm r}, t)$ the local drift-velocity, and $\dot{q}({\bm r}, t)$ accounts for the possible space- and time-dependent energy exchange with an external heat source. $T({\bm r}, t)$ and $\bm{u}({\bm r}, t)$ emerge from the conservation of energy and quasi-conservation of crystal momentum in microscopic phonon collisions in the hydrodynamic regime, respectively \cite{Simoncelli2020}. The term ``quasi-conservation'' is used for crystal momentum because momentum-dissipating Umklapp collisions are always present in real materials at finite temperature---in practice the magnitude of hydrodynamic effects depends on the relative strength between normal and Umklapp collision---and the presence of Umklapp collisions is taken into account by dissipative term $-\gamma^{ij}u^j({\bm r}, t)$.
The thermal conductivity $\kappa^{ij}$ and viscosity $\mu^{ijkl}$ quantify the response of the system to a perturbation of temperature and drift-velocity, respectively~\cite{Simoncelli2020}. The coupling coefficients $\alpha^{ij}$ and $\beta^{ij}$ originate from the relation between energy and crystal momentum for phonons. All these parameters are determined exactly from the LBTE (\textit{i.e.} accounting for the actual phonon band structure and full collision matrix) and with first-principles accuracy, details are reported in Ref.~\cite{Simoncelli2020} \footnote{In order to simplify the notation, here we defined $\alpha^{ij}=W_{0j}^i\sqrt{\overline T A^jC}$, $\beta^{ij}=\sqrt{\frac{CA^i}{\overline T}} W_{i0}^j$, $\gamma^{ij}=\sqrt{A^i A^j} D_U^{ij}$, where $\overline{T}$ is the temperature at which thermal conductivity and viscosity are determined, $W_{0j}^i$ is the velocity tensor arising from the non-diagonal form of the diffusion operator in the basis of the eigenvectors of the normal part of the scattering matrix discussed in Ref.~\cite{Simoncelli2020}, and $D_U^{ij}$ the Umklapp-dissipation timescale discussed in Ref.~\cite{Simoncelli2020}.}. Finally, it is possible to show \cite{Simoncelli2020} that the temperature gradient and the drift velocity determine the total heat flux, $\bm{Q}^{TOT}=\bm{Q}^{\delta}+\bm{Q}^{D}$, where ${Q}^{\delta,i}=-\sum_j \kappa^{ij}\nabla^j T$ and ${Q}^{D,i}=\sum_j\alpha^{ij}u^j$.



The VHE encompass Fourier’s law and temperature waves \cite{joseph_heat_1989,tzou_unified_1995} as special limiting cases.
% Specifically, it can be shown that in the limit of strong crystal-momentum dissipation and negligible viscous effects ($|\mu|/|\gamma|\to 0$, where we used $|\mu|$ and $|\gamma|$ to indicate the maximum component of the viscosity and Umklapp dissipation tensors) the VHE yield Fourier's diffusive behavior \cite{Simoncelli2020}.
Specifically, it can be shown that in the limit of strong crystal-momentum dissipation and negligible viscous effects ($\mu_\mathrm{max}{\to}0$ and $[\gamma_\mathrm{max}]^{-1}{\to}0$, where $\mu_\mathrm{max}$ and $\gamma_\mathrm{max}$ are the maximum component of the viscosity and Umklapp dissipation tensors) the VHE yield Fourier's diffusive behavior \cite{Simoncelli2020}.
In contrast, we show in the Supplementary Material that in the inviscid limit ($\mu{=}0$) the VHE reduce to the dual-phase-lag  equation (DPLE) \cite{joseph_heat_1989,tzou_unified_1995} for temperature waves (which also encompasses Cattaneo's second-sound equation as a special case \cite{old_Hardy1970,cattaneo1958form}).
Differences between the viscous temperature waves emerging from the VHE and the DPLE's inviscid waves will be quantitatively discussed later.
Finally, we note that the VHE allow to describe also heat hydrodynamics in the steady-state regime, which cannot be described by Fourier's law or by the DPLE equation.

\begin{figure*}
\vspace*{-5mm}
	\centering
	\includegraphics[width=0.75\textwidth]{fig2_VHE.png}\\[-3mm]
	\caption{\textbf{Transient hydrodynamic heat backflow and lattice cooling.}
	We show the VHE predictions for the relaxation in time of a temperature perturbation (obtained applying a localized heater to a device in the time interval from 0 to 0.4 ns) in a graphitic device thermalised at 80 K the boundaries (thermalisation occurs in shaded regions, see SM~\ref{sub:modeling_realistic_thermalisation}).
	Rows show different instants in time for temperature (left column), temperature-gradient heat-flux component ($\bm{Q}^{\delta}$, central column), and drifting heat-flux component ($\bm{Q}^{D}$, right column). 
	The emergence of lattice cooling, \textit{i.e.} a temperature locally and transiently lower than the initial value $T{=}80$ K, originates from the lagging coupled evolution of $\bm{Q}^{\delta}$ and $\bm{Q}^{D}$, as discussed in the text.
    In the 2D plots, The heat flux streamlines are shown in white, while the colormap shows the magnitude of the heat flux.}
	\label{fig:fig2}
\end{figure*}


We start our analysis from the steady-state regime, solving the VHE in a device made of graphite and under conditions where the Fourier Deviation Number (FDN) discussed in Ref.~\cite{Simoncelli2020} predicts hydrodynamic deviations from Fourier law to be largest, \textit{i.e.} at an average temperature of 70 K and in a device having typical size 10 $\mu$m (natural abundance isotopic concentration). 
We show in Fig.~\ref{fig:1_vortex} that the VHE predict the emergence of viscous heat backflow in a geometry that strongly promotes gradients in the drift velocity \cite{e-vortexes}, thus of vortical heat flow.
% We highlight how the presence of viscous heat backflow can be detected via a temperature measurement\cite{menges_nanoscale_2016,cheng_battery_2022,cahill_nanoscale_2014,braun_spatially_2022,ziabari_full-field_2018}, since it yields a temperature profile in the circular chamber that is reversed compared to the temperature profile in the tunnel, a behavior completely opposite to that predicted by Fourier's law.
{We highlight that the VHE temperature profile in the circular chamber is reversed compared to the temperature profile in the tunnel, a behavior completely opposite to that predicted by Fourier's law. This is a direct consequence of the presence of viscous heat backflow, which can thus be detected via a temperature measurement\cite{menges_nanoscale_2016,cheng_battery_2022,cahill_nanoscale_2014,braun_spatially_2022,ziabari_full-field_2018}.}


To see how the vortex and consequent heat backflow in Fig.~\ref{fig:1_vortex} requires the presence of a finite viscosity to emerge, we start by noting that the behavior of the device is described by Eqs.~(\ref{viscous_heat_T},\ref{viscous_heat_U}) in the steady-state, with the source term $\dot{q}(\bm{r},t)=0$, and applying boundary conditions for temperature, $T{=}70{\pm}25$ K at $y{=}{\pm} 2.5\mu m $.
Then, note that if we consider the inviscid limit ($\mu^{ijkl}{=}0\;\forall \; i,j,k,l$), Eq.~(\ref{viscous_heat_U}) reduces to 
$\sum_{j=1}^{3}\beta^{ij}\frac{\partial T({\bm r}, t)}{\partial r^j} {=} {-} \sum_{j=1}^{3} \gamma^{ij} u^j({\bm r}, t)$; this equation can be inserted into Eq.~(\ref{viscous_heat_T}) to readily show that 
the behavior in the inviscid limit is governed by a Fourier-like (Laplace) equation, which implies zero curvature for the temperature field everywhere, prohibiting the emergence of backflow and vorticity \footnote{We note that here we are looking only at the in-plane heat flux in graphite, and the tensors $\beta^{ij}$ and $\gamma^{ij}$ can be considered diagonal and isotropic in this case, see SM.~\ref{sec:parameters_entering_in_the_viscous_heat_equations}.}.
Finally, we note that the VHE simulations in Fig.~\ref{fig:1_vortex}\textbf{b,c)} have been performed using "slipping" boundary conditions for the drift velocity (\textit{i.e.} leaving the drift-velocity component tangent to the boundary unaffected, and setting the drift-velocity component orthogonal to the boundary to zero).
We show in Sec.~\ref{sec:reflective_boundaries_and_heat_vortices} of the Supplementary Material (SM) that having at least partially slipping boundaries (corresponding to phonon-boundary scattering partially reflecting the phonon momentum \cite{ziman1960electrons,fugallo_ab_2013,ravichandran_spectrally_2018}) is necessary to observe temperature inversion due to viscous heat backflow.

Hitherto, experiments in graphite have observed heat backflowing against the temperature gradient only in the time-dependent domain, in the form of second sound\cite{Huberman2019,Ding2022} or lattice cooling\cite{Jeong2021}. The related theoretical analyses have always been performed relying on the microscopic LBTE\cite{old_Hardy1970,cepellotti_transport_2017,Huberman2019,Ding2022,Jeong2021}, or on the mesoscopic dual-phase-lag \cite{joseph_heat_1989,tzou_unified_1995,xu_thermal_2002,xu_thermal_2021} (DPLE) equation (which encompasses the well-known Cattaneo or ``second-sound'' equation\cite{cattaneo1958form,Ding2022,app_Cepellotti2015} as special case), without relying on the concept of thermal viscosity.
It is therefore natural to wonder how the viscous heat backflow emerging from the VHE behaves in the time domain, and more precisely if there is a relationship between time-dependent viscous heat backflow and transient temperature waves.
% Therefore, we perform the time-dependent simulation shown in Fig.~\ref{fig:fig2}. We consider a rectangular device that is at equilibrium at $t{=}0\;ns$ ($T{=}80$ K and $\bm{u}{=}0$ everywhere); we perturb it with a localized heater for $t_{\rm heat}{=}0.4\;ns$ ($\dot{q}(\bm{r},t){=}\mathcal{H}\theta(t_{\rm heat}{-}t) \exp\left[{-}\tfrac{(x+x_c)^2}{2\sigma_x^2}{-}\tfrac{y^2}{2\sigma_y^2}\right]$, see Eq.~(\ref{viscous_heat_T}) and note\cite{footnote_param}); then we switch off the heater and monitor the relaxation to equilibrium.
Therefore, we perform the time-dependent simulation shown in Fig.~\ref{fig:fig2}. We consider a rectangular device that is at equilibrium at $t{=}0\;ns$ ($T{=}80$ K and $\bm{u}{=}0$ everywhere); we perturb it with a heater localized at $(x_c,0)$, for $t_{\rm heat}{=}0.4\;ns$ ($\dot{q}(\bm{r},t){=}\mathcal{H}\theta(t_{\rm heat}{-}t) \exp\left[{-}\tfrac{(x+x_c)^2}{2\sigma_x^2}{-}\tfrac{y^2}{2\sigma_y^2}\right]$, see Eq.~(\ref{viscous_heat_T}) and note\cite{footnote_param}); then we switch off the heater and monitor the relaxation to equilibrium. The device is always thermalised at the boundaries ($T=80K$ and $\bm{u}{=}0$, see Ref.~\cite{braun_spatially_2022} for an example of experimental implementation of this boundary condition, further details are reported in SM~\ref{sub:lattice_cooling_boundaries}).
The evolution of the temperature field (first column) shows an oscillatory behavior, also termed ``lattice cooling''\cite{Jeong2021} because of the transient and local appearance of temperature values lower than the initial equilibrium temperature.
Such an oscillatory behavior is in sharp contrast with that predicted by diffusive Fourier's equation (see Fig.~\ref{fig:figFourier} in the SM), whose smoothing property \cite{skinner_university_nodate} implies that the evolution of a smooth, positive temperature perturbation relaxes to equilibrium 
remaining non-negative with respect to the initial equilibrium values. 
The appearance of lattice cooling from the VHE can be understood by inspecting the time evolution of the two aforementioned components of the VHE's heat flux ($\bm{Q}^{\delta}$ and $\bm{Q}^{D}$).
% Fig.~\ref{fig:fig2} shows that the heat fluxes $\bm{Q}^{\delta}$ and $\bm{Q}^{D}$ assume opposite directions during the relaxation, with the drifting flux $\bm{Q}^{D}{\propto}\bm{u}$ backflowing against the temperature-gradient flux $\bm{Q}^{\delta}{\propto}{-}\nabla T$ in the regions where temperature oscillations are observed.
{The heat-flow streamlines in the second and third column of Fig.~\ref{fig:fig2} show that the heat fluxes $\bm{Q}^{\delta}$ and $\bm{Q}^{D}$ assume opposite directions during the relaxation, with the drifting flux $\bm{Q}^{D}{\propto}\bm{u}$ backflowing against the temperature-gradient flux $\bm{Q}^{\delta}{\propto}{-}\nabla T$ in the regions where temperature oscillations are observed.}
The coupled evolution of $\bm{Q}^{\delta}$ and $\bm{Q}^{D}$ is characterized by a lagging (delayed) response in both space and time. 
The emergence of a lagging response can be seen from Eqs.~(\ref{viscous_heat_T},\ref{viscous_heat_U}), since in general the time and space derivatives of $\bm{u}(\bm{r},t){\propto} \bm{Q}^D$ and $\nabla T(\bm{r},t){\propto} \bm{Q}^\delta$ evolve according to different equations. More precisely, we show in the SM~\ref{sec:DPLE_derivation} that in the inviscid limit ($\mu{=}0$) the VHE yield exactly the lagged relationship between heat flux and temperature gradient discussed in the context of the dual-phase-lag model\cite{tzou_unified_1995}, \textit{i.e.} $\bm{Q}^{TOT}(\bm{r},t{+}\tau_Q){=}{-}\sum_j\kappa^{ij}\nabla^j T(\bm{r},t{+}\tau_T)$, where for in-plane graphite (in the following, indexes are not reported because for in-plane transport the tensor/vectors can be reduced to scalars, see SM~\ref{sec:parameters_entering_in_the_viscous_heat_equations}) $\tau_Q{=}A/\gamma$ is the time needed to create heat flux from a temperature gradient, and conversely $\tau_T{=}\kappa A/(\alpha\beta{+}\kappa\gamma)$ is the time needed to create temperature gradient from an established heat flux.
%\textcolor{red}{We show in the SM~\ref{sec:DPLE_derivation} that in the inviscid limit ($\mu{=}0$) the VHE yield exactly the lagged relationship between heat flux and temperature gradient discussed in the context of the dual-phase-lag model\cite{tzou_unified_1995}, \textit{i.e.} $\bm{Q}^{TOT}(\bm{r},t{+}\tau_Q){=}{-}\sum_j\kappa^{ij}\nabla^j T(\bm{r},t{+}\tau_T)$, where for in-plane graphite (in the following, indexes are not reported because for in-plane transport the tensor/vectors can be reduced to scalars, see SM~\ref{sec:parameters_entering_in_the_viscous_heat_equations}) $\tau_Q{=}A/\gamma$ is the time needed to create heat flux from a temperature gradient, and conversely $\tau_T{=}\kappa A/(\alpha\beta{+}\kappa\gamma)$ is the time needed to create temperature gradient from an established heat flux.}
Thus, heat backflow can be observed in the time-dependent domain as a consequence of different characteristic timescales for the evolution of $\bm{Q}^{\delta}$ and $\bm{Q}^{D}$, which allow these two heat-flux components to flow in opposite directions and yield in the inviscid limit a behavior analytically equivalent to the lagged DPLE. 
This also shows that while steady-state hydrodynamic heat backflow can emerge exclusively as a consequence of a finite thermal viscosity, time-dependent hydrodynamic backflow (\textit{i.e.} temperature oscillations) do not necessarily require viscosity to appear.
Nevertheless, we discuss in SM~\ref{sec:comparison_between_theory_and_experiments_for_lattice_cooling} how accounting for the presence of a finite thermal viscosity is necessary to obtain quantitative agreement between the relaxation timescales predicted by the VHE and those measured in recent experiments~\cite{Jeong2021}.


\begin{figure}[b]
\includegraphics[width=\columnwidth]{4_pert_time_evol_rect_new.png}
\includegraphics[width=0.9\columnwidth]{DPLE_3.pdf}\\[-3mm]
\caption{\textbf{Resonant amplification of temperature waves.} Top, two-spots periodic perturbation ($\dot{q}(\bm{r},t)$ in Eq.~\ref{viscous_heat_T}) applied to rectangular device made of graphite and having dimension  and boundary conditions as in Fig.~\ref{fig:fig2}. Bottom, resonant amplification, $A(f)/A_0$ ($A_0=\lim_{f\to 0}A(f)$), of temperature oscillations around $T{=}80 K$ predicted by the VHE (green), DPLE (blue, corresponding to VHE with zero viscosity), and Fourier's law (red, corresponding to overdamped oscillations). 
The main plot refers to graphite at natural-abundance isotopic-mass disorder (98.9\% $^{12}C$, 1.1\% $^{13}C$); the inset show the increase in resonant amplification in isotopically pure samples (99.9\% $^{12}C$, 0.1\% $^{13}C$). 
\label{fig:rect_freq} }
\end{figure}
\begin{figure*}
\vspace*{-3mm}
\includegraphics[width=\textwidth]{fig4_defF.pdf}\\[-3mm]
\caption{\label{fig:freq_cont}\textbf{Maximum resonant amplification as a function of size, temperature, and isotopic disorder.} 
The colormaps show the maximum resonant amplification (see text) as a function of the characteristic size of the rectangle\cite{footnote_legend} and of the average temperature around which the perturbation is applied: panel \textbf{a)} is the VHE in natural samples, panel \textbf{b)} is the DPLE in natural samples,  \textbf{c)} is the VHE in isotopically pure samples, panel \textbf{d)} is the DPLE in isotopically pure samples. Empty green circles have an area proportional to the hydrodynamic strength\cite{footnote_fig4} measured in experiments by \citet{Huberman2019} for natural samples, and by \citet{Ding2022} for isotopically pure samples. The red crosses show conditions under which these experiments did not observe hydrodynamic behavior.\\[-5mm]}
\end{figure*}

The temperature waves observed in Fig.~\ref{fig:fig2} have a small amplitude and consequently are difficult to be detected in experiments; still, they are expected to exhibit resonant amplification when driven with a perturbation periodic in time, and this could be exploited to facilitate their experimental detection. 
Therefore, we investigate quantitatively the behavior of the device in Fig.~\ref{fig:fig2} when driven with a periodic perturbation. 
Considering the analogies between temperature and mechanical waves, we applied to the the device in Fig.~\ref{fig:fig2} a perturbation similar to the $(1,2)$ mode of a loaded rectangular membrane, \textit{i.e.} $\dot{q}(\bm{r},t){=}\mathcal{H} [\sin(\omega t){+}1] \exp\big[{-}\tfrac{(x{+}x_c)^2}{2\sigma_x^2}{-}\tfrac{y^2}{2\sigma_y^2}\big]{+}$ $ {+} \mathcal{H} [\sin(\omega t {+}\pi){+}1] \exp\big[{-}\tfrac{(x-x_c)^2}{2\sigma_x^2}{-}\tfrac{y^2}{2\sigma_y^2}\big]$.
This perturbation is always non-negative, to account for the fact that experimental methods usually rely on lasers that inject energy in the system, creating positive temperature perturbations. 
%Then, we monitored how the amplitude of the temperature oscillation varies as a function of frequency.
% Fig.~\ref{fig:rect_freq} shows that the solution of the periodically driven VHE displays resonant behavior, and the resonant amplification, increases as isotopic-mass disorder decreases.
% We also compare the viscous resonant behavior obtained from the viscous VHE with the inviscid behavior obtained from the DPLE, finding that a finite thermal viscosity yields a weaker resonant response.
Then, we monitored how the amplitude of the temperature oscillation varies as a function of frequency ($A(f)$). 
%This is equivalent to changing the rapidity of turning the lasers on and off.
Fig.~\ref{fig:rect_freq} shows that the solution of the periodically driven VHE displays resonant behavior, \textit{i.e.} plotting the oscillation amplitude as a function of frequency we see a peak reminiscent of that observed in the frequency response of a driven underdamped oscillator. We note that the resonant amplification increases as isotopic-mass disorder decreases (see also SM~\ref{sub:rescaling}). 
We also compare the viscous resonant behavior obtained from the viscous VHE with the behavior obtained from the inviscid DPLE, finding that a finite thermal viscosity yields a weaker resonant response.
Finally, we highlight how Fourier's law completely lacks resonant response, as expected from the analogy between Fourier's law and the equations describing an overdamped oscillator.
The calculations in Fig.~\ref{fig:rect_freq} are performed around an equilibrium temperature of 80 K, \textit{i.e.} the temperature around which hydrodynamic deviations from Fourier's law are maximized in samples with isotopes at natural abundance.

Next, we investigate how the maximum resonant amplification varies as a function of device size, average temperature, and isotopic concentration. 
We performed simulations analogous to Fig.~\ref{fig:rect_freq} varying size and equilibrium temperature (Sec.~\ref{sub:rescaling}), computing for every simulation the maximum resonant amplification as $\rm{max}_f[A(f)]/A_0$ ($A(f)$ is the amplitude of the temperature oscillation in the presence of a perturbation having frequency $f$, $A_0$ is the amplitude of the temperature oscillation in the zero-frequency limit, see Fig.~\ref{fig:rect_freq}).
The results in Fig.~\ref{fig:freq_cont} show that isotopically pure samples display resonant amplification that is stronger in magnitude, and persists at higher temperatures and larger lengthscales, compared to natural samples.
Importantly, Fig.~\ref{fig:freq_cont}\textbf{a} shows that the temperatures and lengthscales at which the VHE predict the emergence of hydrodynamic resonant behavior in natural samples is in broad agreement with the temperatures and lengthscales for the emergence of heat hydrodynamics discussed by \citet{Huberman2019}. In contrast, Fig.~\ref{fig:freq_cont}\textbf{b} shows that the DPLE fails to capture the reduction of hydrodynamic behavior observed in experiments in natural samples as temperature is decreased below 100 K.
Turning our attention to isotopically pure samples (Fig.~\ref{fig:freq_cont}\textbf{c,d}), we see that experiments\cite{Ding2022} are available only at temperatures higher than 100 K, preventing us to compare VHE and DPLE with experiments in the small-size-and-low-temperature region, where VHE and DPLE display the most significant differences. In the high-temperature-and-small-size limit, where both the VHE and DPLE show a qualitatively similar reduction of the hydrodynamic signatures, experiments show a qualitative behavior similar to that obtained from the VHE and DPLE. 
Thus, our findings suggest that the thermal viscosity plays a crucial role in determining 
hydrodynamic behavior in the sub-100K limit, especially at small sizes. 

In summary, we have shown that the VHE generalize the DPLE accounting for the thermal viscosity.
We used first-principles calculations to determine the parameters entering in these equations, thus used them to study signatures of heat hydrodynamics in graphite with natural or reduced isotopic-mass disorder, in both the steady-state and time-dependent domain.
In the steady-state regime, the DPLE reduces to Fourier's law. In contrast, the VHE predict that a finite thermal viscosity can lead to the emergence of viscous heat backflow in a tunnel-chamber geometry, and an hallmark of such phenomenon is a temperature profile reversed compared to that predicted by Fourier's law.
In the time-dependent regime, we have discussed how the transient heat backflow underlying temperature oscillations
originates from a lagging coupled evolution of $\bm{Q}^{\delta}$ and $\bm{Q}^{D}$, and does not necessarily require a finite thermal viscosity to emerge. In fact, temperature oscillations emerge from the time-dependent solution of the full (viscous) VHE, but they are visible also in the inviscid limit, where the VHE analytically reduces to the DPLE\cite{joseph_heat_1989,tzou_unified_1995}.
We have quantitatively discussed how temperature oscillations are affected by the thermal viscosity, showing that there are significant differences between the viscous temperature waves emerging from the VHE and the inviscid temperature waves obtained from the DPLE.
Importantly, we have shown that it is necessary to rely on the VHE and account for the thermal viscosity to quantitatively reproduce the temperatures, timescales, and lengthscales at which hydrodynamics temperature waves have been measured in graphite at natural isotopic concentration\cite{Huberman2019,Jeong2021}.
Therefore, we have provided novel, fundamental insights on how the thermal viscosity affects the magnitude of temperature oscillations, an effect which was not considered in past DPLE-based\cite{barletta_hyperbolic_1996,xu_thermal_2002,xu_thermal_2021} works.
Overall, this work sheds light on the fundamental physics determining the emergence of hydrodynamic deviations from Fourier's law, both in the steady-state and time-dependent domain, also proposing experimental setups for their detection and control (amplification). These findings may be relevant for the technological exploitation of heat hydrodynamics in next-generation electronic and phononic devices.

\section*{Acknowledgements}
We thank Dr Miguel Beneitez for the useful discussions.
M. S. acknowledges support from Gonville and Caius College, and from the SNSF project P500PT\_203178. 
J. D. thanks Prof Hrvoje Jasak for his hospitality in Cambridge.
The conductivity and viscosity calculations were performed using the Sulis Tier 2 HPC platform, funded by EPSRC Grant EP/T022108/1 and the HPC Midlands+consortium. 


%apsrev4-2.bst 2019-01-14 (MD) hand-edited version of apsrev4-1.bst
%Control: key (0)
%Control: author (8) initials jnrlst
%Control: editor formatted (1) identically to author
%Control: production of article title (0) allowed
%Control: page (0) single
%Control: year (1) truncated
%Control: production of eprint (0) enabled
\providecommand{\noopsort}[1]{}\providecommand{\singleletter}[1]{#1}%
\begin{thebibliography}{59}%
\makeatletter
\providecommand \@ifxundefined [1]{%
 \@ifx{#1\undefined}
}%
\providecommand \@ifnum [1]{%
 \ifnum #1\expandafter \@firstoftwo
 \else \expandafter \@secondoftwo
 \fi
}%
\providecommand \@ifx [1]{%
 \ifx #1\expandafter \@firstoftwo
 \else \expandafter \@secondoftwo
 \fi
}%
\providecommand \natexlab [1]{#1}%
\providecommand \enquote  [1]{``#1''}%
\providecommand \bibnamefont  [1]{#1}%
\providecommand \bibfnamefont [1]{#1}%
\providecommand \citenamefont [1]{#1}%
\providecommand \href@noop [0]{\@secondoftwo}%
\providecommand \href [0]{\begingroup \@sanitize@url \@href}%
\providecommand \@href[1]{\@@startlink{#1}\@@href}%
\providecommand \@@href[1]{\endgroup#1\@@endlink}%
\providecommand \@sanitize@url [0]{\catcode `\\12\catcode `\$12\catcode
  `\&12\catcode `\#12\catcode `\^12\catcode `\_12\catcode `\%12\relax}%
\providecommand \@@startlink[1]{}%
\providecommand \@@endlink[0]{}%
\providecommand \url  [0]{\begingroup\@sanitize@url \@url }%
\providecommand \@url [1]{\endgroup\@href {#1}{\urlprefix }}%
\providecommand \urlprefix  [0]{URL }%
\providecommand \Eprint [0]{\href }%
\providecommand \doibase [0]{https://doi.org/}%
\providecommand \selectlanguage [0]{\@gobble}%
\providecommand \bibinfo  [0]{\@secondoftwo}%
\providecommand \bibfield  [0]{\@secondoftwo}%
\providecommand \translation [1]{[#1]}%
\providecommand \BibitemOpen [0]{}%
\providecommand \bibitemStop [0]{}%
\providecommand \bibitemNoStop [0]{.\EOS\space}%
\providecommand \EOS [0]{\spacefactor3000\relax}%
\providecommand \BibitemShut  [1]{\csname bibitem#1\endcsname}%
\let\auto@bib@innerbib\@empty
%</preamble>
\bibitem [{\citenamefont {Schmidt}\ \emph {et~al.}(2008)\citenamefont
  {Schmidt}, \citenamefont {Chen},\ and\ \citenamefont
  {Chen}}]{schmidt_pulse_2008}%
  \BibitemOpen
  \bibfield  {author} {\bibinfo {author} {\bibfnamefont {A.~J.}\ \bibnamefont
  {Schmidt}}, \bibinfo {author} {\bibfnamefont {X.}~\bibnamefont {Chen}},\ and\
  \bibinfo {author} {\bibfnamefont {G.}~\bibnamefont {Chen}},\ }\bibfield
  {title} {\bibinfo {title} {Pulse accumulation, radial heat conduction, and
  anisotropic thermal conductivity in pump-probe transient thermoreflectance},\
  }\href {https://doi.org/10.1063/1.3006335} {\bibfield  {journal} {\bibinfo
  {journal} {Review of Scientific Instruments}\ }\textbf {\bibinfo {volume}
  {79}},\ \bibinfo {pages} {114902} (\bibinfo {year} {2008})},\ \bibinfo {note}
  {publisher: American Institute of Physics}\BibitemShut {NoStop}%
\bibitem [{\citenamefont {Balandin}(2011)}]{balandin_thermal_2011}%
  \BibitemOpen
  \bibfield  {author} {\bibinfo {author} {\bibfnamefont {A.~A.}\ \bibnamefont
  {Balandin}},\ }\bibfield  {title} {\bibinfo {title} {Thermal properties of
  graphene and nanostructured carbon materials},\ }\href
  {https://doi.org/10.1038/nmat3064} {\bibfield  {journal} {\bibinfo  {journal}
  {Nature Materials}\ }\textbf {\bibinfo {volume} {10}},\ \bibinfo {pages}
  {569} (\bibinfo {year} {2011})},\ \bibinfo {note} {number: 8 Publisher:
  Nature Publishing Group}\BibitemShut {NoStop}%
\bibitem [{\citenamefont {Fugallo}\ \emph {et~al.}(2014)\citenamefont
  {Fugallo}, \citenamefont {Cepellotti}, \citenamefont {Paulatto},
  \citenamefont {Lazzeri}, \citenamefont {Marzari},\ and\ \citenamefont
  {Mauri}}]{fugallo_thermal_2014}%
  \BibitemOpen
  \bibfield  {author} {\bibinfo {author} {\bibfnamefont {G.}~\bibnamefont
  {Fugallo}}, \bibinfo {author} {\bibfnamefont {A.}~\bibnamefont {Cepellotti}},
  \bibinfo {author} {\bibfnamefont {L.}~\bibnamefont {Paulatto}}, \bibinfo
  {author} {\bibfnamefont {M.}~\bibnamefont {Lazzeri}}, \bibinfo {author}
  {\bibfnamefont {N.}~\bibnamefont {Marzari}},\ and\ \bibinfo {author}
  {\bibfnamefont {F.}~\bibnamefont {Mauri}},\ }\bibfield  {title} {\bibinfo
  {title} {Thermal {Conductivity} of {Graphene} and {Graphite}: {Collective}
  {Excitations} and {Mean} {Free} {Paths}},\ }\href
  {https://doi.org/10.1021/nl502059f} {\bibfield  {journal} {\bibinfo
  {journal} {Nano Letters}\ }\textbf {\bibinfo {volume} {14}},\ \bibinfo
  {pages} {6109} (\bibinfo {year} {2014})}\BibitemShut {NoStop}%
\bibitem [{\citenamefont {Machida}\ \emph {et~al.}(2020)\citenamefont
  {Machida}, \citenamefont {Matsumoto}, \citenamefont {Isono},\ and\
  \citenamefont {Behnia}}]{machida_phonon_2020}%
  \BibitemOpen
  \bibfield  {author} {\bibinfo {author} {\bibfnamefont {Y.}~\bibnamefont
  {Machida}}, \bibinfo {author} {\bibfnamefont {N.}~\bibnamefont {Matsumoto}},
  \bibinfo {author} {\bibfnamefont {T.}~\bibnamefont {Isono}},\ and\ \bibinfo
  {author} {\bibfnamefont {K.}~\bibnamefont {Behnia}},\ }\bibfield  {title}
  {\bibinfo {title} {Phonon hydrodynamics and ultrahigh–room-temperature
  thermal conductivity in thin graphite},\ }\href
  {https://doi.org/10.1126/science.aaz8043} {\bibfield  {journal} {\bibinfo
  {journal} {Science}\ }\textbf {\bibinfo {volume} {367}},\ \bibinfo {pages}
  {309} (\bibinfo {year} {2020})}\BibitemShut {NoStop}%
\bibitem [{\citenamefont {Qian}\ \emph {et~al.}(2021)\citenamefont {Qian},
  \citenamefont {Zhou},\ and\ \citenamefont
  {Chen}}]{qian_phonon-engineered_2021}%
  \BibitemOpen
  \bibfield  {author} {\bibinfo {author} {\bibfnamefont {X.}~\bibnamefont
  {Qian}}, \bibinfo {author} {\bibfnamefont {J.}~\bibnamefont {Zhou}},\ and\
  \bibinfo {author} {\bibfnamefont {G.}~\bibnamefont {Chen}},\ }\bibfield
  {title} {\bibinfo {title} {Phonon-engineered extreme thermal conductivity
  materials},\ }\href {https://doi.org/10.1038/s41563-021-00918-3} {\bibfield
  {journal} {\bibinfo  {journal} {Nature Materials}\ }\textbf {\bibinfo
  {volume} {20}},\ \bibinfo {pages} {1188} (\bibinfo {year} {2021})},\ \bibinfo
  {note} {number: 9 Publisher: Nature Publishing Group}\BibitemShut {NoStop}%
\bibitem [{\citenamefont {Chen}(2021)}]{chen_non-fourier_2021}%
  \BibitemOpen
  \bibfield  {author} {\bibinfo {author} {\bibfnamefont {G.}~\bibnamefont
  {Chen}},\ }\bibfield  {title} {\bibinfo {title} {Non-{Fourier} phonon heat
  conduction at the microscale and nanoscale},\ }\href
  {https://doi.org/10.1038/s42254-021-00334-1} {\bibfield  {journal} {\bibinfo
  {journal} {Nature Reviews Physics}\ }\textbf {\bibinfo {volume} {3}},\
  \bibinfo {pages} {555} (\bibinfo {year} {2021})}\BibitemShut {NoStop}%
\bibitem [{\citenamefont {Huberman}\ \emph {et~al.}(2019)\citenamefont
  {Huberman}, \citenamefont {Duncan}, \citenamefont {Chen}, \citenamefont
  {Song}, \citenamefont {Chiloyan}, \citenamefont {Ding}, \citenamefont
  {Maznev}, \citenamefont {Chen},\ and\ \citenamefont {Nelson}}]{Huberman2019}%
  \BibitemOpen
  \bibfield  {author} {\bibinfo {author} {\bibfnamefont {S.}~\bibnamefont
  {Huberman}}, \bibinfo {author} {\bibfnamefont {R.~A.}\ \bibnamefont
  {Duncan}}, \bibinfo {author} {\bibfnamefont {K.}~\bibnamefont {Chen}},
  \bibinfo {author} {\bibfnamefont {B.}~\bibnamefont {Song}}, \bibinfo {author}
  {\bibfnamefont {V.}~\bibnamefont {Chiloyan}}, \bibinfo {author}
  {\bibfnamefont {Z.}~\bibnamefont {Ding}}, \bibinfo {author} {\bibfnamefont
  {A.~A.}\ \bibnamefont {Maznev}}, \bibinfo {author} {\bibfnamefont
  {G.}~\bibnamefont {Chen}},\ and\ \bibinfo {author} {\bibfnamefont {K.~A.}\
  \bibnamefont {Nelson}},\ }\bibfield  {title} {\bibinfo {title} {Observation
  of second sound in graphite at temperatures above 100 k},\ }\href
  {https://doi.org/10.1126/science.aav3548} {\bibfield  {journal} {\bibinfo
  {journal} {Science}\ }\textbf {\bibinfo {volume} {364}},\ \bibinfo {pages}
  {375} (\bibinfo {year} {2019})}\BibitemShut {NoStop}%
\bibitem [{\citenamefont {Jeong}\ \emph {et~al.}(2021)\citenamefont {Jeong},
  \citenamefont {Li}, \citenamefont {Lee}, \citenamefont {Shi},\ and\
  \citenamefont {Wang}}]{Jeong2021}%
  \BibitemOpen
  \bibfield  {author} {\bibinfo {author} {\bibfnamefont {J.}~\bibnamefont
  {Jeong}}, \bibinfo {author} {\bibfnamefont {X.}~\bibnamefont {Li}}, \bibinfo
  {author} {\bibfnamefont {S.}~\bibnamefont {Lee}}, \bibinfo {author}
  {\bibfnamefont {L.}~\bibnamefont {Shi}},\ and\ \bibinfo {author}
  {\bibfnamefont {Y.}~\bibnamefont {Wang}},\ }\bibfield  {title} {\bibinfo
  {title} {Transient hydrodynamic lattice cooling by picosecond laser
  irradiation of graphite},\ }\href
  {https://doi.org/10.1103/PhysRevLett.127.085901} {\bibfield  {journal}
  {\bibinfo  {journal} {Phys. Rev. Lett.}\ }\textbf {\bibinfo {volume} {127}},\
  \bibinfo {pages} {085901} (\bibinfo {year} {2021})}\BibitemShut {NoStop}%
\bibitem [{\citenamefont {Ding}\ \emph {et~al.}(2022)\citenamefont {Ding},
  \citenamefont {Chen}, \citenamefont {Song}, \citenamefont {Shin},
  \citenamefont {Maznev}, \citenamefont {Nelson},\ and\ \citenamefont
  {Chen}}]{Ding2022}%
  \BibitemOpen
  \bibfield  {author} {\bibinfo {author} {\bibfnamefont {Z.}~\bibnamefont
  {Ding}}, \bibinfo {author} {\bibfnamefont {K.}~\bibnamefont {Chen}}, \bibinfo
  {author} {\bibfnamefont {B.}~\bibnamefont {Song}}, \bibinfo {author}
  {\bibfnamefont {J.}~\bibnamefont {Shin}}, \bibinfo {author} {\bibfnamefont
  {A.~A.}\ \bibnamefont {Maznev}}, \bibinfo {author} {\bibfnamefont {K.~A.}\
  \bibnamefont {Nelson}},\ and\ \bibinfo {author} {\bibfnamefont
  {G.}~\bibnamefont {Chen}},\ }\bibfield  {title} {\bibinfo {title}
  {Observation of second sound in graphite over 200 k},\ }\href
  {https://doi.org/10.1038/s41467-021-27907-z} {\bibfield  {journal} {\bibinfo
  {journal} {Nature Communications}\ }\textbf {\bibinfo {volume} {13}},\
  \bibinfo {pages} {285} (\bibinfo {year} {2022})}\BibitemShut {NoStop}%
\bibitem [{\citenamefont {Peierls}(1955)}]{peierls1955quantum}%
  \BibitemOpen
  \bibfield  {author} {\bibinfo {author} {\bibfnamefont {R.~E.}\ \bibnamefont
  {Peierls}},\ }\href@noop {} {\emph {\bibinfo {title} {{Quantum theory of
  solids}}}}\ (\bibinfo  {publisher} {Oxford university press},\ \bibinfo
  {year} {1955})\BibitemShut {NoStop}%
\bibitem [{\citenamefont {Lindsay}\ \emph {et~al.}(2019)\citenamefont
  {Lindsay}, \citenamefont {Katre}, \citenamefont {Cepellotti},\ and\
  \citenamefont {Mingo}}]{lindsay_perspective_2019}%
  \BibitemOpen
  \bibfield  {author} {\bibinfo {author} {\bibfnamefont {L.}~\bibnamefont
  {Lindsay}}, \bibinfo {author} {\bibfnamefont {A.}~\bibnamefont {Katre}},
  \bibinfo {author} {\bibfnamefont {A.}~\bibnamefont {Cepellotti}},\ and\
  \bibinfo {author} {\bibfnamefont {N.}~\bibnamefont {Mingo}},\ }\bibfield
  {title} {\bibinfo {title} {Perspective on \textit{ab initio} phonon thermal
  transport},\ }\href {https://doi.org/10.1063/1.5108651} {\bibfield  {journal}
  {\bibinfo  {journal} {Journal of Applied Physics}\ }\textbf {\bibinfo
  {volume} {126}},\ \bibinfo {pages} {050902} (\bibinfo {year}
  {2019})}\BibitemShut {NoStop}%
\bibitem [{\citenamefont {Ding}\ \emph {et~al.}(2018)\citenamefont {Ding},
  \citenamefont {Zhou}, \citenamefont {Song}, \citenamefont {Chiloyan},
  \citenamefont {Li}, \citenamefont {Liu},\ and\ \citenamefont
  {Chen}}]{app_Ding2018}%
  \BibitemOpen
  \bibfield  {author} {\bibinfo {author} {\bibfnamefont {Z.}~\bibnamefont
  {Ding}}, \bibinfo {author} {\bibfnamefont {J.}~\bibnamefont {Zhou}}, \bibinfo
  {author} {\bibfnamefont {B.}~\bibnamefont {Song}}, \bibinfo {author}
  {\bibfnamefont {V.}~\bibnamefont {Chiloyan}}, \bibinfo {author}
  {\bibfnamefont {M.}~\bibnamefont {Li}}, \bibinfo {author} {\bibfnamefont
  {T.-H.}\ \bibnamefont {Liu}},\ and\ \bibinfo {author} {\bibfnamefont
  {G.}~\bibnamefont {Chen}},\ }\bibfield  {title} {\bibinfo {title} {Phonon
  hydrodynamic heat conduction and knudsen minimum in graphite},\ }\href
  {https://doi.org/10.1021/acs.nanolett.7b04932} {\bibfield  {journal}
  {\bibinfo  {journal} {Nano Lett.}\ }\textbf {\bibinfo {volume} {18}},\
  \bibinfo {pages} {638} (\bibinfo {year} {2018})},\ \bibinfo {note}
  {publisher: American Chemical Society}\BibitemShut {NoStop}%
\bibitem [{\citenamefont {Guo}\ \emph {et~al.}(2021{\natexlab{a}})\citenamefont
  {Guo}, \citenamefont {Zhang}, \citenamefont {Bescond}, \citenamefont {Xiong},
  \citenamefont {Wang}, \citenamefont {Nomura},\ and\ \citenamefont
  {Volz}}]{guo_size_2021}%
  \BibitemOpen
  \bibfield  {author} {\bibinfo {author} {\bibfnamefont {Y.}~\bibnamefont
  {Guo}}, \bibinfo {author} {\bibfnamefont {Z.}~\bibnamefont {Zhang}}, \bibinfo
  {author} {\bibfnamefont {M.}~\bibnamefont {Bescond}}, \bibinfo {author}
  {\bibfnamefont {S.}~\bibnamefont {Xiong}}, \bibinfo {author} {\bibfnamefont
  {M.}~\bibnamefont {Wang}}, \bibinfo {author} {\bibfnamefont {M.}~\bibnamefont
  {Nomura}},\ and\ \bibinfo {author} {\bibfnamefont {S.}~\bibnamefont {Volz}},\
  }\bibfield  {title} {\bibinfo {title} {Size effect on phonon hydrodynamics in
  graphite microstructures and nanostructures},\ }\href
  {https://doi.org/10.1103/PhysRevB.104.075450} {\bibfield  {journal} {\bibinfo
   {journal} {Physical Review B}\ }\textbf {\bibinfo {volume} {104}},\ \bibinfo
  {pages} {075450} (\bibinfo {year} {2021}{\natexlab{a}})},\ \bibinfo {note}
  {publisher: American Physical Society}\BibitemShut {NoStop}%
\bibitem [{\citenamefont {Li}\ \emph {et~al.}(2022)\citenamefont {Li},
  \citenamefont {Lee}, \citenamefont {Ou}, \citenamefont {Lee},\ and\
  \citenamefont {Shi}}]{li_reexamination_2022}%
  \BibitemOpen
  \bibfield  {author} {\bibinfo {author} {\bibfnamefont {X.}~\bibnamefont
  {Li}}, \bibinfo {author} {\bibfnamefont {H.}~\bibnamefont {Lee}}, \bibinfo
  {author} {\bibfnamefont {E.}~\bibnamefont {Ou}}, \bibinfo {author}
  {\bibfnamefont {S.}~\bibnamefont {Lee}},\ and\ \bibinfo {author}
  {\bibfnamefont {L.}~\bibnamefont {Shi}},\ }\bibfield  {title} {\bibinfo
  {title} {Reexamination of hydrodynamic phonon transport in thin graphite},\
  }\href {https://doi.org/10.1063/5.0078772} {\bibfield  {journal} {\bibinfo
  {journal} {Journal of Applied Physics}\ }\textbf {\bibinfo {volume} {131}},\
  \bibinfo {pages} {075104} (\bibinfo {year} {2022})},\ \bibinfo {note}
  {publisher: American Institute of Physics}\BibitemShut {NoStop}%
\bibitem [{\citenamefont {Huang}\ \emph
  {et~al.}(2022{\natexlab{a}})\citenamefont {Huang}, \citenamefont {Guo},
  \citenamefont {Volz},\ and\ \citenamefont {Nomura}}]{huang_mapping_2022}%
  \BibitemOpen
  \bibfield  {author} {\bibinfo {author} {\bibfnamefont {X.}~\bibnamefont
  {Huang}}, \bibinfo {author} {\bibfnamefont {Y.}~\bibnamefont {Guo}}, \bibinfo
  {author} {\bibfnamefont {S.}~\bibnamefont {Volz}},\ and\ \bibinfo {author}
  {\bibfnamefont {M.}~\bibnamefont {Nomura}},\ }\bibfield  {title} {\bibinfo
  {title} {Mapping phonon hydrodynamic strength in micrometer-scale graphite
  structures},\ }\href {https://doi.org/10.35848/1882-0786/ac8f82} {\bibfield
  {journal} {\bibinfo  {journal} {Applied Physics Express}\ }\textbf {\bibinfo
  {volume} {15}},\ \bibinfo {pages} {105001} (\bibinfo {year}
  {2022}{\natexlab{a}})},\ \bibinfo {note} {publisher: IOP
  Publishing}\BibitemShut {NoStop}%
\bibitem [{\citenamefont {Cepellotti}\ \emph {et~al.}(2015)\citenamefont
  {Cepellotti}, \citenamefont {Fugallo}, \citenamefont {Paulatto},
  \citenamefont {Lazzeri}, \citenamefont {Mauri},\ and\ \citenamefont
  {Marzari}}]{app_Cepellotti2015}%
  \BibitemOpen
  \bibfield  {author} {\bibinfo {author} {\bibfnamefont {A.}~\bibnamefont
  {Cepellotti}}, \bibinfo {author} {\bibfnamefont {G.}~\bibnamefont {Fugallo}},
  \bibinfo {author} {\bibfnamefont {L.}~\bibnamefont {Paulatto}}, \bibinfo
  {author} {\bibfnamefont {M.}~\bibnamefont {Lazzeri}}, \bibinfo {author}
  {\bibfnamefont {F.}~\bibnamefont {Mauri}},\ and\ \bibinfo {author}
  {\bibfnamefont {N.}~\bibnamefont {Marzari}},\ }\bibfield  {title} {\bibinfo
  {title} {Phonon hydrodynamics in two-dimensional materials},\ }\href
  {https://doi.org/10.1038/ncomms7400} {\bibfield  {journal} {\bibinfo
  {journal} {Nature Communications}\ }\textbf {\bibinfo {volume} {6}},\
  \bibinfo {pages} {6400} (\bibinfo {year} {2015})}\BibitemShut {NoStop}%
\bibitem [{\citenamefont {Cepellotti}\ and\ \citenamefont
  {Marzari}(2016)}]{cepellotti_thermal_2016}%
  \BibitemOpen
  \bibfield  {author} {\bibinfo {author} {\bibfnamefont {A.}~\bibnamefont
  {Cepellotti}}\ and\ \bibinfo {author} {\bibfnamefont {N.}~\bibnamefont
  {Marzari}},\ }\bibfield  {title} {\bibinfo {title} {Thermal {Transport} in
  {Crystals} as a {Kinetic} {Theory} of {Relaxons}},\ }\href
  {https://doi.org/10.1103/PhysRevX.6.041013} {\bibfield  {journal} {\bibinfo
  {journal} {Physical Review X}\ }\textbf {\bibinfo {volume} {6}},\ \bibinfo
  {pages} {041013} (\bibinfo {year} {2016})}\BibitemShut {NoStop}%
\bibitem [{\citenamefont {Lee}\ \emph {et~al.}(2015)\citenamefont {Lee},
  \citenamefont {Broido}, \citenamefont {Esfarjani},\ and\ \citenamefont
  {Chen}}]{app_Lee2015}%
  \BibitemOpen
  \bibfield  {author} {\bibinfo {author} {\bibfnamefont {S.}~\bibnamefont
  {Lee}}, \bibinfo {author} {\bibfnamefont {D.}~\bibnamefont {Broido}},
  \bibinfo {author} {\bibfnamefont {K.}~\bibnamefont {Esfarjani}},\ and\
  \bibinfo {author} {\bibfnamefont {G.}~\bibnamefont {Chen}},\ }\bibfield
  {title} {\bibinfo {title} {Hydrodynamic phonon transport in suspended
  graphene},\ }\href {https://doi.org/10.1038/ncomms7290} {\bibfield  {journal}
  {\bibinfo  {journal} {Nature Communications}\ }\textbf {\bibinfo {volume}
  {6}},\ \bibinfo {pages} {6290} (\bibinfo {year} {2015})}\BibitemShut
  {NoStop}%
\bibitem [{\citenamefont {Majee}\ and\ \citenamefont
  {Aksamija}(2018)}]{majee_dynamical_2018}%
  \BibitemOpen
  \bibfield  {author} {\bibinfo {author} {\bibfnamefont {A.~K.}\ \bibnamefont
  {Majee}}\ and\ \bibinfo {author} {\bibfnamefont {Z.}~\bibnamefont
  {Aksamija}},\ }\bibfield  {title} {\bibinfo {title} {Dynamical thermal
  conductivity of suspended graphene ribbons in the hydrodynamic regime},\
  }\href {https://doi.org/10.1103/PhysRevB.98.024303} {\bibfield  {journal}
  {\bibinfo  {journal} {Physical Review B}\ }\textbf {\bibinfo {volume} {98}},\
  \bibinfo {pages} {024303} (\bibinfo {year} {2018})},\ \bibinfo {note}
  {publisher: American Physical Society}\BibitemShut {NoStop}%
\bibitem [{\citenamefont {Raya-Moreno}\ \emph {et~al.}(2022)\citenamefont
  {Raya-Moreno}, \citenamefont {Carrete},\ and\ \citenamefont
  {Cartoixà}}]{raya-moreno_hydrodynamic_2022}%
  \BibitemOpen
  \bibfield  {author} {\bibinfo {author} {\bibfnamefont {M.}~\bibnamefont
  {Raya-Moreno}}, \bibinfo {author} {\bibfnamefont {J.}~\bibnamefont
  {Carrete}},\ and\ \bibinfo {author} {\bibfnamefont {X.}~\bibnamefont
  {Cartoixà}},\ }\bibfield  {title} {\bibinfo {title} {Hydrodynamic signatures
  in thermal transport in devices based on two-dimensional materials: {An}
  \textit{ab initio} study},\ }\href
  {https://doi.org/10.1103/PhysRevB.106.014308} {\bibfield  {journal} {\bibinfo
   {journal} {Physical Review B}\ }\textbf {\bibinfo {volume} {106}},\ \bibinfo
  {pages} {014308} (\bibinfo {year} {2022})}\BibitemShut {NoStop}%
\bibitem [{\citenamefont {Guo}\ \emph {et~al.}(2021{\natexlab{b}})\citenamefont
  {Guo}, \citenamefont {Zhang}, \citenamefont {Nomura}, \citenamefont {Volz},\
  and\ \citenamefont {Wang}}]{guo_phonon_2021}%
  \BibitemOpen
  \bibfield  {author} {\bibinfo {author} {\bibfnamefont {Y.}~\bibnamefont
  {Guo}}, \bibinfo {author} {\bibfnamefont {Z.}~\bibnamefont {Zhang}}, \bibinfo
  {author} {\bibfnamefont {M.}~\bibnamefont {Nomura}}, \bibinfo {author}
  {\bibfnamefont {S.}~\bibnamefont {Volz}},\ and\ \bibinfo {author}
  {\bibfnamefont {M.}~\bibnamefont {Wang}},\ }\bibfield  {title} {\bibinfo
  {title} {Phonon vortex dynamics in graphene ribbon by solving {Boltzmann}
  transport equation with ab initio scattering rates},\ }\href
  {https://doi.org/10.1016/j.ijheatmasstransfer.2021.120981} {\bibfield
  {journal} {\bibinfo  {journal} {International Journal of Heat and Mass
  Transfer}\ }\textbf {\bibinfo {volume} {169}},\ \bibinfo {pages} {120981}
  (\bibinfo {year} {2021}{\natexlab{b}})}\BibitemShut {NoStop}%
\bibitem [{\citenamefont {Han}\ and\ \citenamefont {Ruan}(2023)}]{han_is_2023}%
  \BibitemOpen
  \bibfield  {author} {\bibinfo {author} {\bibfnamefont {Z.}~\bibnamefont
  {Han}}\ and\ \bibinfo {author} {\bibfnamefont {X.}~\bibnamefont {Ruan}},\
  }\href {http://arxiv.org/abs/2302.12216} {\bibinfo {title} {Is {Thermal}
  {Conductivity} of {Graphene} {Divergent} and {Higher} {Than} {Diamond}?}}
  (\bibinfo {year} {2023}),\ \bibinfo {note} {arXiv:2302.12216 [cond-mat,
  physics:physics]}\BibitemShut {NoStop}%
\bibitem [{\citenamefont {Cepellotti}\ and\ \citenamefont
  {Marzari}(2017{\natexlab{a}})}]{cepellotti_boltzmann_2017}%
  \BibitemOpen
  \bibfield  {author} {\bibinfo {author} {\bibfnamefont {A.}~\bibnamefont
  {Cepellotti}}\ and\ \bibinfo {author} {\bibfnamefont {N.}~\bibnamefont
  {Marzari}},\ }\bibfield  {title} {\bibinfo {title} {Boltzmann {Transport} in
  {Nanostructures} as a {Friction} {Effect}},\ }\href
  {https://doi.org/10.1021/acs.nanolett.7b01202} {\bibfield  {journal}
  {\bibinfo  {journal} {Nano Letters}\ }\textbf {\bibinfo {volume} {17}},\
  \bibinfo {pages} {4675} (\bibinfo {year} {2017}{\natexlab{a}})},\ \bibinfo
  {note} {publisher: American Chemical Society}\BibitemShut {NoStop}%
\bibitem [{\citenamefont {Sendra}\ \emph {et~al.}(2022)\citenamefont {Sendra},
  \citenamefont {Beardo}, \citenamefont {Bafaluy}, \citenamefont {Torres},
  \citenamefont {Alvarez},\ and\ \citenamefont
  {Camacho}}]{sendra_hydrodynamic_2022}%
  \BibitemOpen
  \bibfield  {author} {\bibinfo {author} {\bibfnamefont {L.}~\bibnamefont
  {Sendra}}, \bibinfo {author} {\bibfnamefont {A.}~\bibnamefont {Beardo}},
  \bibinfo {author} {\bibfnamefont {J.}~\bibnamefont {Bafaluy}}, \bibinfo
  {author} {\bibfnamefont {P.}~\bibnamefont {Torres}}, \bibinfo {author}
  {\bibfnamefont {F.~X.}\ \bibnamefont {Alvarez}},\ and\ \bibinfo {author}
  {\bibfnamefont {J.}~\bibnamefont {Camacho}},\ }\bibfield  {title} {\bibinfo
  {title} {Hydrodynamic heat transport in dielectric crystals in the collective
  limit and the drifting/driftless velocity conundrum},\ }\href
  {https://doi.org/10.1103/PhysRevB.106.155301} {\bibfield  {journal} {\bibinfo
   {journal} {Physical Review B}\ }\textbf {\bibinfo {volume} {106}},\ \bibinfo
  {pages} {155301} (\bibinfo {year} {2022})},\ \bibinfo {note} {publisher:
  American Physical Society}\BibitemShut {NoStop}%
\bibitem [{\citenamefont {Huang}\ \emph
  {et~al.}(2022{\natexlab{b}})\citenamefont {Huang}, \citenamefont {Guo},
  \citenamefont {Wu}, \citenamefont {Masubuchi}, \citenamefont {Watanabe},
  \citenamefont {Taniguchi}, \citenamefont {Zhang}, \citenamefont {Volz},
  \citenamefont {Machida},\ and\ \citenamefont
  {Nomura}}]{huang_observation_2022}%
  \BibitemOpen
  \bibfield  {author} {\bibinfo {author} {\bibfnamefont {X.}~\bibnamefont
  {Huang}}, \bibinfo {author} {\bibfnamefont {Y.}~\bibnamefont {Guo}}, \bibinfo
  {author} {\bibfnamefont {Y.}~\bibnamefont {Wu}}, \bibinfo {author}
  {\bibfnamefont {S.}~\bibnamefont {Masubuchi}}, \bibinfo {author}
  {\bibfnamefont {K.}~\bibnamefont {Watanabe}}, \bibinfo {author}
  {\bibfnamefont {T.}~\bibnamefont {Taniguchi}}, \bibinfo {author}
  {\bibfnamefont {Z.}~\bibnamefont {Zhang}}, \bibinfo {author} {\bibfnamefont
  {S.}~\bibnamefont {Volz}}, \bibinfo {author} {\bibfnamefont {T.}~\bibnamefont
  {Machida}},\ and\ \bibinfo {author} {\bibfnamefont {M.}~\bibnamefont
  {Nomura}},\ }\href {https://doi.org/10.48550/arXiv.2207.01469} {\bibinfo
  {title} {Observation of phonon {Poiseuille} flow in isotopically-purified
  graphite ribbons}} (\bibinfo {year} {2022}{\natexlab{b}}),\ \bibinfo {note}
  {arXiv:2207.01469 [cond-mat]}\BibitemShut {NoStop}%
\bibitem [{\citenamefont {Li}\ and\ \citenamefont {Lee}(2018)}]{li_role_2018}%
  \BibitemOpen
  \bibfield  {author} {\bibinfo {author} {\bibfnamefont {X.}~\bibnamefont
  {Li}}\ and\ \bibinfo {author} {\bibfnamefont {S.}~\bibnamefont {Lee}},\
  }\bibfield  {title} {\bibinfo {title} {Role of hydrodynamic viscosity on
  phonon transport in suspended graphene},\ }\href
  {https://doi.org/10.1103/PhysRevB.97.094309} {\bibfield  {journal} {\bibinfo
  {journal} {Physical Review B}\ }\textbf {\bibinfo {volume} {97}},\ \bibinfo
  {pages} {094309} (\bibinfo {year} {2018})}\BibitemShut {NoStop}%
\bibitem [{\citenamefont {Guo}\ \emph {et~al.}(2018)\citenamefont {Guo},
  \citenamefont {Jou},\ and\ \citenamefont {Wang}}]{guo_nonequilibrium_2018}%
  \BibitemOpen
  \bibfield  {author} {\bibinfo {author} {\bibfnamefont {Y.}~\bibnamefont
  {Guo}}, \bibinfo {author} {\bibfnamefont {D.}~\bibnamefont {Jou}},\ and\
  \bibinfo {author} {\bibfnamefont {M.}~\bibnamefont {Wang}},\ }\bibfield
  {title} {\bibinfo {title} {Nonequilibrium thermodynamics of phonon
  hydrodynamic model for nanoscale heat transport},\ }\href
  {https://doi.org/10.1103/PhysRevB.98.104304} {\bibfield  {journal} {\bibinfo
  {journal} {Physical Review B}\ }\textbf {\bibinfo {volume} {98}},\ \bibinfo
  {pages} {104304} (\bibinfo {year} {2018})}\BibitemShut {NoStop}%
\bibitem [{\citenamefont {Shang}\ \emph {et~al.}(2020)\citenamefont {Shang},
  \citenamefont {Zhang}, \citenamefont {Guo},\ and\ \citenamefont
  {Lü}}]{shang_heat_2020}%
  \BibitemOpen
  \bibfield  {author} {\bibinfo {author} {\bibfnamefont {M.-Y.}\ \bibnamefont
  {Shang}}, \bibinfo {author} {\bibfnamefont {C.}~\bibnamefont {Zhang}},
  \bibinfo {author} {\bibfnamefont {Z.}~\bibnamefont {Guo}},\ and\ \bibinfo
  {author} {\bibfnamefont {J.-T.}\ \bibnamefont {Lü}},\ }\bibfield  {title}
  {\bibinfo {title} {Heat vortex in hydrodynamic phonon transport of
  two-dimensional materials},\ }\href
  {https://doi.org/10.1038/s41598-020-65221-8} {\bibfield  {journal} {\bibinfo
  {journal} {Scientific Reports}\ }\textbf {\bibinfo {volume} {10}},\ \bibinfo
  {pages} {8272} (\bibinfo {year} {2020})}\BibitemShut {NoStop}%
\bibitem [{\citenamefont {Simoncelli}\ \emph {et~al.}(2020)\citenamefont
  {Simoncelli}, \citenamefont {Marzari},\ and\ \citenamefont
  {Cepellotti}}]{Simoncelli2020}%
  \BibitemOpen
  \bibfield  {author} {\bibinfo {author} {\bibfnamefont {M.}~\bibnamefont
  {Simoncelli}}, \bibinfo {author} {\bibfnamefont {N.}~\bibnamefont
  {Marzari}},\ and\ \bibinfo {author} {\bibfnamefont {A.}~\bibnamefont
  {Cepellotti}},\ }\bibfield  {title} {\bibinfo {title} {Generalization of
  fourier's law into viscous heat equations},\ }\href
  {https://doi.org/10.1103/PhysRevX.10.011019} {\bibfield  {journal} {\bibinfo
  {journal} {Phys. Rev. X}\ }\textbf {\bibinfo {volume} {10}},\ \bibinfo
  {pages} {011019} (\bibinfo {year} {2020})}\BibitemShut {NoStop}%
\bibitem [{\citenamefont {Sendra}\ \emph {et~al.}(2021)\citenamefont {Sendra},
  \citenamefont {Beardo}, \citenamefont {Torres}, \citenamefont {Bafaluy},
  \citenamefont {Alvarez},\ and\ \citenamefont
  {Camacho}}]{sendra_derivation_2021}%
  \BibitemOpen
  \bibfield  {author} {\bibinfo {author} {\bibfnamefont {L.}~\bibnamefont
  {Sendra}}, \bibinfo {author} {\bibfnamefont {A.}~\bibnamefont {Beardo}},
  \bibinfo {author} {\bibfnamefont {P.}~\bibnamefont {Torres}}, \bibinfo
  {author} {\bibfnamefont {J.}~\bibnamefont {Bafaluy}}, \bibinfo {author}
  {\bibfnamefont {F.~X.}\ \bibnamefont {Alvarez}},\ and\ \bibinfo {author}
  {\bibfnamefont {J.}~\bibnamefont {Camacho}},\ }\bibfield  {title} {\bibinfo
  {title} {Derivation of a hydrodynamic heat equation from the phonon
  {Boltzmann} equation for general semiconductors},\ }\href
  {https://doi.org/10.1103/PhysRevB.103.L140301} {\bibfield  {journal}
  {\bibinfo  {journal} {Physical Review B}\ }\textbf {\bibinfo {volume}
  {103}},\ \bibinfo {pages} {L140301} (\bibinfo {year} {2021})},\ \bibinfo
  {note} {publisher: American Physical Society}\BibitemShut {NoStop}%
\bibitem [{\citenamefont {Joseph}\ and\ \citenamefont
  {Preziosi}(1989)}]{joseph_heat_1989}%
  \BibitemOpen
  \bibfield  {author} {\bibinfo {author} {\bibfnamefont {D.~D.}\ \bibnamefont
  {Joseph}}\ and\ \bibinfo {author} {\bibfnamefont {L.}~\bibnamefont
  {Preziosi}},\ }\bibfield  {title} {\bibinfo {title} {Heat waves},\ }\href
  {https://doi.org/10.1103/RevModPhys.61.41} {\bibfield  {journal} {\bibinfo
  {journal} {Reviews of Modern Physics}\ }\textbf {\bibinfo {volume} {61}},\
  \bibinfo {pages} {41} (\bibinfo {year} {1989})}\BibitemShut {NoStop}%
\bibitem [{\citenamefont {Tzou}(1995)}]{tzou_unified_1995}%
  \BibitemOpen
  \bibfield  {author} {\bibinfo {author} {\bibfnamefont {D.~Y.}\ \bibnamefont
  {Tzou}},\ }\bibfield  {title} {\bibinfo {title} {A {Unified} {Field}
  {Approach} for {Heat} {Conduction} {From} {Macro}- to {Micro}-{Scales}},\
  }\href {https://doi.org/10.1115/1.2822329} {\bibfield  {journal} {\bibinfo
  {journal} {Journal of Heat Transfer}\ }\textbf {\bibinfo {volume} {117}},\
  \bibinfo {pages} {8} (\bibinfo {year} {1995})}\BibitemShut {NoStop}%
\bibitem [{\citenamefont {Cattaneo}(1958)}]{cattaneo1958form}%
  \BibitemOpen
  \bibfield  {author} {\bibinfo {author} {\bibfnamefont {C.}~\bibnamefont
  {Cattaneo}},\ }\bibfield  {title} {\bibinfo {title} {A form of
  heat-conduction equations which eliminates the paradox of instantaneous
  propagation},\ }\href@noop {} {\bibfield  {journal} {\bibinfo  {journal}
  {Comptes Rendus}\ }\textbf {\bibinfo {volume} {247}},\ \bibinfo {pages} {431}
  (\bibinfo {year} {1958})}\BibitemShut {NoStop}%
\bibitem [{Note1()}]{Note1}%
  \BibitemOpen
  \bibinfo {note} {In order to simplify the notation, here we defined $\alpha
  ^{ij}=W_{0j}^i\protect \sqrt {\protect \overline T A^jC}$, $\beta
  ^{ij}=\protect \sqrt {\protect \frac {CA^i}{\protect \overline T}} W_{i0}^j$,
  $\gamma ^{ij}=\protect \sqrt {A^i A^j} D_U^{ij}$, where $\protect \overline
  {T}$ is the temperature at which thermal conductivity and viscosity are
  determined, $W_{0j}^i$ is the velocity tensor arising from the non-diagonal
  form of the diffusion operator in the basis of the eigenvectors of the normal
  part of the scattering matrix discussed in Ref.~\cite {Simoncelli2020}, and
  $D_U^{ij}$ the Umklapp-dissipation timescale discussed in Ref.~\cite
  {Simoncelli2020}.}\BibitemShut {Stop}%
\bibitem [{\citenamefont {Hardy}(1970)}]{old_Hardy1970}%
  \BibitemOpen
  \bibfield  {author} {\bibinfo {author} {\bibfnamefont {R.~J.}\ \bibnamefont
  {Hardy}},\ }\bibfield  {title} {\bibinfo {title} {Phonon boltzmann equation
  and second sound in solids},\ }\href
  {https://doi.org/10.1103/PhysRevB.2.1193} {\bibfield  {journal} {\bibinfo
  {journal} {Phys. Rev. B}\ }\textbf {\bibinfo {volume} {2}},\ \bibinfo {pages}
  {1193} (\bibinfo {year} {1970})}\BibitemShut {NoStop}%
\bibitem [{\citenamefont {Aharon-Steinberg}\ \emph {et~al.}(2022)\citenamefont
  {Aharon-Steinberg}, \citenamefont {Völkl}, \citenamefont {Kaplan},
  \citenamefont {Pariari}, \citenamefont {Roy}, \citenamefont {Holder},
  \citenamefont {Wolf}, \citenamefont {Meltzer}, \citenamefont {Myasoedov},
  \citenamefont {Huber}, \citenamefont {Yan}, \citenamefont {Falkovich},
  \citenamefont {Levitov}, \citenamefont {Hücker},\ and\ \citenamefont
  {Zeldov}}]{e-vortexes}%
  \BibitemOpen
  \bibfield  {author} {\bibinfo {author} {\bibfnamefont {A.}~\bibnamefont
  {Aharon-Steinberg}}, \bibinfo {author} {\bibfnamefont {T.}~\bibnamefont
  {Völkl}}, \bibinfo {author} {\bibfnamefont {A.}~\bibnamefont {Kaplan}},
  \bibinfo {author} {\bibfnamefont {A.~K.}\ \bibnamefont {Pariari}}, \bibinfo
  {author} {\bibfnamefont {I.}~\bibnamefont {Roy}}, \bibinfo {author}
  {\bibfnamefont {T.}~\bibnamefont {Holder}}, \bibinfo {author} {\bibfnamefont
  {Y.}~\bibnamefont {Wolf}}, \bibinfo {author} {\bibfnamefont {A.~Y.}\
  \bibnamefont {Meltzer}}, \bibinfo {author} {\bibfnamefont {Y.}~\bibnamefont
  {Myasoedov}}, \bibinfo {author} {\bibfnamefont {M.~E.}\ \bibnamefont
  {Huber}}, \bibinfo {author} {\bibfnamefont {B.}~\bibnamefont {Yan}}, \bibinfo
  {author} {\bibfnamefont {G.}~\bibnamefont {Falkovich}}, \bibinfo {author}
  {\bibfnamefont {L.~S.}\ \bibnamefont {Levitov}}, \bibinfo {author}
  {\bibfnamefont {M.}~\bibnamefont {Hücker}},\ and\ \bibinfo {author}
  {\bibfnamefont {E.}~\bibnamefont {Zeldov}},\ }\bibfield  {title} {\bibinfo
  {title} {Direct observation of vortices in an electron fluid},\ }\href
  {https://doi.org/10.1038/s41586-022-04794-y} {\bibfield  {journal} {\bibinfo
  {journal} {Nature}\ }\textbf {\bibinfo {volume} {607}},\ \bibinfo {pages}
  {74} (\bibinfo {year} {2022})}\BibitemShut {NoStop}%
\bibitem [{\citenamefont {Menges}\ \emph {et~al.}(2016)\citenamefont {Menges},
  \citenamefont {Riel}, \citenamefont {Stemmer},\ and\ \citenamefont
  {Gotsmann}}]{menges_nanoscale_2016}%
  \BibitemOpen
  \bibfield  {author} {\bibinfo {author} {\bibfnamefont {F.}~\bibnamefont
  {Menges}}, \bibinfo {author} {\bibfnamefont {H.}~\bibnamefont {Riel}},
  \bibinfo {author} {\bibfnamefont {A.}~\bibnamefont {Stemmer}},\ and\ \bibinfo
  {author} {\bibfnamefont {B.}~\bibnamefont {Gotsmann}},\ }\bibfield  {title}
  {\bibinfo {title} {Nanoscale thermometry by scanning thermal microscopy},\
  }\href {https://doi.org/10.1063/1.4955449} {\bibfield  {journal} {\bibinfo
  {journal} {Review of Scientific Instruments}\ }\textbf {\bibinfo {volume}
  {87}},\ \bibinfo {pages} {074902} (\bibinfo {year} {2016})},\ \bibinfo {note}
  {publisher: American Institute of Physics}\BibitemShut {NoStop}%
\bibitem [{\citenamefont {Cheng}\ \emph {et~al.}(2022)\citenamefont {Cheng},
  \citenamefont {Ji},\ and\ \citenamefont {Cahill}}]{cheng_battery_2022}%
  \BibitemOpen
  \bibfield  {author} {\bibinfo {author} {\bibfnamefont {Z.}~\bibnamefont
  {Cheng}}, \bibinfo {author} {\bibfnamefont {X.}~\bibnamefont {Ji}},\ and\
  \bibinfo {author} {\bibfnamefont {D.~G.}\ \bibnamefont {Cahill}},\ }\bibfield
   {title} {\bibinfo {title} {Battery absorbs heat during charging uncovered by
  ultra-sensitive thermometry},\ }\href
  {https://doi.org/10.1016/j.jpowsour.2021.230762} {\bibfield  {journal}
  {\bibinfo  {journal} {Journal of Power Sources}\ }\textbf {\bibinfo {volume}
  {518}},\ \bibinfo {pages} {230762} (\bibinfo {year} {2022})}\BibitemShut
  {NoStop}%
\bibitem [{\citenamefont {Cahill}\ \emph {et~al.}(2014)\citenamefont {Cahill},
  \citenamefont {Braun}, \citenamefont {Chen}, \citenamefont {Clarke},
  \citenamefont {Fan}, \citenamefont {Goodson}, \citenamefont {Keblinski},
  \citenamefont {King}, \citenamefont {Mahan}, \citenamefont {Majumdar},
  \citenamefont {Maris}, \citenamefont {Phillpot}, \citenamefont {Pop},\ and\
  \citenamefont {Shi}}]{cahill_nanoscale_2014}%
  \BibitemOpen
  \bibfield  {author} {\bibinfo {author} {\bibfnamefont {D.~G.}\ \bibnamefont
  {Cahill}}, \bibinfo {author} {\bibfnamefont {P.~V.}\ \bibnamefont {Braun}},
  \bibinfo {author} {\bibfnamefont {G.}~\bibnamefont {Chen}}, \bibinfo {author}
  {\bibfnamefont {D.~R.}\ \bibnamefont {Clarke}}, \bibinfo {author}
  {\bibfnamefont {S.}~\bibnamefont {Fan}}, \bibinfo {author} {\bibfnamefont
  {K.~E.}\ \bibnamefont {Goodson}}, \bibinfo {author} {\bibfnamefont
  {P.}~\bibnamefont {Keblinski}}, \bibinfo {author} {\bibfnamefont {W.~P.}\
  \bibnamefont {King}}, \bibinfo {author} {\bibfnamefont {G.~D.}\ \bibnamefont
  {Mahan}}, \bibinfo {author} {\bibfnamefont {A.}~\bibnamefont {Majumdar}},
  \bibinfo {author} {\bibfnamefont {H.~J.}\ \bibnamefont {Maris}}, \bibinfo
  {author} {\bibfnamefont {S.~R.}\ \bibnamefont {Phillpot}}, \bibinfo {author}
  {\bibfnamefont {E.}~\bibnamefont {Pop}},\ and\ \bibinfo {author}
  {\bibfnamefont {L.}~\bibnamefont {Shi}},\ }\bibfield  {title} {\bibinfo
  {title} {Nanoscale thermal transport. {II}. 2003–2012},\ }\href
  {https://doi.org/10.1063/1.4832615} {\bibfield  {journal} {\bibinfo
  {journal} {Applied Physics Reviews}\ }\textbf {\bibinfo {volume} {1}},\
  \bibinfo {pages} {011305} (\bibinfo {year} {2014})}\BibitemShut {NoStop}%
\bibitem [{\citenamefont {Braun}\ \emph {et~al.}(2022)\citenamefont {Braun},
  \citenamefont {Furrer}, \citenamefont {Butti}, \citenamefont {Thodkar},
  \citenamefont {Shorubalko}, \citenamefont {Zardo}, \citenamefont {Calame},\
  and\ \citenamefont {Perrin}}]{braun_spatially_2022}%
  \BibitemOpen
  \bibfield  {author} {\bibinfo {author} {\bibfnamefont {O.}~\bibnamefont
  {Braun}}, \bibinfo {author} {\bibfnamefont {R.}~\bibnamefont {Furrer}},
  \bibinfo {author} {\bibfnamefont {P.}~\bibnamefont {Butti}}, \bibinfo
  {author} {\bibfnamefont {K.}~\bibnamefont {Thodkar}}, \bibinfo {author}
  {\bibfnamefont {I.}~\bibnamefont {Shorubalko}}, \bibinfo {author}
  {\bibfnamefont {I.}~\bibnamefont {Zardo}}, \bibinfo {author} {\bibfnamefont
  {M.}~\bibnamefont {Calame}},\ and\ \bibinfo {author} {\bibfnamefont {M.~L.}\
  \bibnamefont {Perrin}},\ }\bibfield  {title} {\bibinfo {title} {Spatially
  mapping thermal transport in graphene by an opto-thermal method},\ }\href
  {https://doi.org/10.1038/s41699-021-00277-2} {\bibfield  {journal} {\bibinfo
  {journal} {npj 2D Materials and Applications}\ }\textbf {\bibinfo {volume}
  {6}},\ \bibinfo {pages} {1} (\bibinfo {year} {2022})},\ \bibinfo {note}
  {number: 1 Publisher: Nature Publishing Group}\BibitemShut {NoStop}%
\bibitem [{\citenamefont {Ziabari}\ \emph {et~al.}(2018)\citenamefont
  {Ziabari}, \citenamefont {Torres}, \citenamefont {Vermeersch}, \citenamefont
  {Xuan}, \citenamefont {Cartoixà}, \citenamefont {Torelló}, \citenamefont
  {Bahk}, \citenamefont {Koh}, \citenamefont {Parsa}, \citenamefont {Ye},
  \citenamefont {Alvarez},\ and\ \citenamefont
  {Shakouri}}]{ziabari_full-field_2018}%
  \BibitemOpen
  \bibfield  {author} {\bibinfo {author} {\bibfnamefont {A.}~\bibnamefont
  {Ziabari}}, \bibinfo {author} {\bibfnamefont {P.}~\bibnamefont {Torres}},
  \bibinfo {author} {\bibfnamefont {B.}~\bibnamefont {Vermeersch}}, \bibinfo
  {author} {\bibfnamefont {Y.}~\bibnamefont {Xuan}}, \bibinfo {author}
  {\bibfnamefont {X.}~\bibnamefont {Cartoixà}}, \bibinfo {author}
  {\bibfnamefont {A.}~\bibnamefont {Torelló}}, \bibinfo {author}
  {\bibfnamefont {J.-H.}\ \bibnamefont {Bahk}}, \bibinfo {author}
  {\bibfnamefont {Y.~R.}\ \bibnamefont {Koh}}, \bibinfo {author} {\bibfnamefont
  {M.}~\bibnamefont {Parsa}}, \bibinfo {author} {\bibfnamefont {P.~D.}\
  \bibnamefont {Ye}}, \bibinfo {author} {\bibfnamefont {F.~X.}\ \bibnamefont
  {Alvarez}},\ and\ \bibinfo {author} {\bibfnamefont {A.}~\bibnamefont
  {Shakouri}},\ }\bibfield  {title} {\bibinfo {title} {Full-field thermal
  imaging of quasiballistic crosstalk reduction in nanoscale devices},\ }\href
  {https://doi.org/10.1038/s41467-017-02652-4} {\bibfield  {journal} {\bibinfo
  {journal} {Nature Communications}\ }\textbf {\bibinfo {volume} {9}},\
  \bibinfo {pages} {255} (\bibinfo {year} {2018})}\BibitemShut {NoStop}%
\bibitem [{Note2()}]{Note2}%
  \BibitemOpen
  \bibinfo {note} {We note that here we are looking only at the in-plane heat
  flux in graphite, and the tensors $\beta ^{ij}$ and $\gamma ^{ij}$ can be
  considered diagonal and isotropic in this case, see SM.~\ref
  {sec:parameters_entering_in_the_viscous_heat_equations}.}\BibitemShut {Stop}%
\bibitem [{\citenamefont {Ziman}(1960)}]{ziman1960electrons}%
  \BibitemOpen
  \bibfield  {author} {\bibinfo {author} {\bibfnamefont {J.~M.}\ \bibnamefont
  {Ziman}},\ }\href@noop {} {\emph {\bibinfo {title} {Electrons and phonons:
  the theory of transport phenomena in solids}}}\ (\bibinfo  {publisher}
  {Oxford university press},\ \bibinfo {year} {1960})\BibitemShut {NoStop}%
\bibitem [{\citenamefont {Fugallo}\ \emph {et~al.}(2013)\citenamefont
  {Fugallo}, \citenamefont {Lazzeri}, \citenamefont {Paulatto},\ and\
  \citenamefont {Mauri}}]{fugallo_ab_2013}%
  \BibitemOpen
  \bibfield  {author} {\bibinfo {author} {\bibfnamefont {G.}~\bibnamefont
  {Fugallo}}, \bibinfo {author} {\bibfnamefont {M.}~\bibnamefont {Lazzeri}},
  \bibinfo {author} {\bibfnamefont {L.}~\bibnamefont {Paulatto}},\ and\
  \bibinfo {author} {\bibfnamefont {F.}~\bibnamefont {Mauri}},\ }\bibfield
  {title} {\bibinfo {title} {\textit{{Ab} initio} variational approach for
  evaluating lattice thermal conductivity},\ }\href
  {https://doi.org/10.1103/PhysRevB.88.045430} {\bibfield  {journal} {\bibinfo
  {journal} {Physical Review B}\ }\textbf {\bibinfo {volume} {88}},\ \bibinfo
  {pages} {045430} (\bibinfo {year} {2013})}\BibitemShut {NoStop}%
\bibitem [{\citenamefont {Ravichandran}\ \emph {et~al.}(2018)\citenamefont
  {Ravichandran}, \citenamefont {Zhang},\ and\ \citenamefont
  {Minnich}}]{ravichandran_spectrally_2018}%
  \BibitemOpen
  \bibfield  {author} {\bibinfo {author} {\bibfnamefont {N.~K.}\ \bibnamefont
  {Ravichandran}}, \bibinfo {author} {\bibfnamefont {H.}~\bibnamefont
  {Zhang}},\ and\ \bibinfo {author} {\bibfnamefont {A.~J.}\ \bibnamefont
  {Minnich}},\ }\bibfield  {title} {\bibinfo {title} {Spectrally {Resolved}
  {Specular} {Reflections} of {Thermal} {Phonons} from {Atomically} {Rough}
  {Surfaces}},\ }\href {https://doi.org/10.1103/PhysRevX.8.041004} {\bibfield
  {journal} {\bibinfo  {journal} {Physical Review X}\ }\textbf {\bibinfo
  {volume} {8}},\ \bibinfo {pages} {041004} (\bibinfo {year}
  {2018})}\BibitemShut {NoStop}%
\bibitem [{\citenamefont {Cepellotti}\ and\ \citenamefont
  {Marzari}(2017{\natexlab{b}})}]{cepellotti_transport_2017}%
  \BibitemOpen
  \bibfield  {author} {\bibinfo {author} {\bibfnamefont {A.}~\bibnamefont
  {Cepellotti}}\ and\ \bibinfo {author} {\bibfnamefont {N.}~\bibnamefont
  {Marzari}},\ }\bibfield  {title} {\bibinfo {title} {Transport waves as
  crystal excitations},\ }\href
  {https://doi.org/10.1103/PhysRevMaterials.1.045406} {\bibfield  {journal}
  {\bibinfo  {journal} {Physical Review Materials}\ }\textbf {\bibinfo {volume}
  {1}},\ \bibinfo {pages} {045406} (\bibinfo {year}
  {2017}{\natexlab{b}})}\BibitemShut {NoStop}%
\bibitem [{\citenamefont {Xu}\ and\ \citenamefont
  {Wang}(2002)}]{xu_thermal_2002}%
  \BibitemOpen
  \bibfield  {author} {\bibinfo {author} {\bibfnamefont {M.}~\bibnamefont
  {Xu}}\ and\ \bibinfo {author} {\bibfnamefont {L.}~\bibnamefont {Wang}},\
  }\bibfield  {title} {\bibinfo {title} {Thermal oscillation and resonance in
  dual-phase-lagging heat conduction},\ }\href
  {https://doi.org/10.1016/S0017-9310(01)00199-5} {\bibfield  {journal}
  {\bibinfo  {journal} {International Journal of Heat and Mass Transfer}\
  }\textbf {\bibinfo {volume} {45}},\ \bibinfo {pages} {1055} (\bibinfo {year}
  {2002})}\BibitemShut {NoStop}%
\bibitem [{\citenamefont {Xu}(2021)}]{xu_thermal_2021}%
  \BibitemOpen
  \bibfield  {author} {\bibinfo {author} {\bibfnamefont {M.}~\bibnamefont
  {Xu}},\ }\bibfield  {title} {\bibinfo {title} {Thermal oscillations, second
  sound and thermal resonance in phonon hydrodynamics},\ }\href
  {https://doi.org/10.1098/rspa.2020.0913} {\bibfield  {journal} {\bibinfo
  {journal} {Proceedings of the Royal Society A: Mathematical, Physical and
  Engineering Sciences}\ }\textbf {\bibinfo {volume} {477}},\ \bibinfo {pages}
  {rspa.2020.0913, 20200913} (\bibinfo {year} {2021})}\BibitemShut {NoStop}%
\bibitem [{foo({\natexlab{a}})}]{footnote_param}%
  \BibitemOpen
  \href@noop {} {} ({\natexlab{a}}),\ \bibinfo {note} {{To ensure that the
  perturbation created causes variations within 10\% of the equilibrium
  temperature, we used the following parameters:
  $\mathcal{H}=0.013\tfrac{W}{\mu m^3}$, $t_{\rm heat}=0.4ns$, $x_c=5 \mu m$,
  $\sigma_x=2\mu m$, $\sigma_y=2.8\mu m$.}}\BibitemShut {Stop}%
\bibitem [{\citenamefont {Skinner}()}]{skinner_university_nodate}%
  \BibitemOpen
  \bibfield  {author} {\bibinfo {author} {\bibfnamefont {D.}~\bibnamefont
  {Skinner}},\ }\bibfield  {title} {\bibinfo {title} {Mathematical methods,
  university of cambridge},\ }\href@noop {} {\ }\BibitemShut {NoStop}%
\bibitem [{foo({\natexlab{b}})}]{footnote_legend}%
  \BibitemOpen
  \href@noop {} {} ({\natexlab{b}}),\ \bibinfo {note} {{We report the length of
  the longest side, the shortest side is 0.8 times shorter.}}\BibitemShut
  {Stop}%
\bibitem [{foo({\natexlab{c}})}]{footnote_fig4}%
  \BibitemOpen
  \href@noop {} {} ({\natexlab{c}}),\ \bibinfo {note}
  {{Refs.~\cite{Huberman2019,Ding2022} quantified the hydrodynamic strength as
  the dip depth of the TTG signal, see e.g. Fig. 1a in
  Ref.~\cite{Ding2022}.}}\BibitemShut {Stop}%
\bibitem [{\citenamefont {Barletta}\ and\ \citenamefont
  {Zanchini}(1996)}]{barletta_hyperbolic_1996}%
  \BibitemOpen
  \bibfield  {author} {\bibinfo {author} {\bibfnamefont {A.}~\bibnamefont
  {Barletta}}\ and\ \bibinfo {author} {\bibfnamefont {E.}~\bibnamefont
  {Zanchini}},\ }\bibfield  {title} {\bibinfo {title} {Hyperbolic heat
  conduction and thermal resonances in a cylindrical solid carrying a
  steady-periodic electric field},\ }\href
  {https://doi.org/10.1016/0017-9310(95)00202-2} {\bibfield  {journal}
  {\bibinfo  {journal} {International Journal of Heat and Mass Transfer}\
  }\textbf {\bibinfo {volume} {39}},\ \bibinfo {pages} {1307} (\bibinfo {year}
  {1996})}\BibitemShut {NoStop}%
\bibitem [{\citenamefont {Tu}(2008)}]{Bump}%
  \BibitemOpen
  \bibfield  {author} {\bibinfo {author} {\bibfnamefont {L.~W.}\ \bibnamefont
  {Tu}},\ }\href@noop {} {\emph {\bibinfo {title} {An Introduction to
  Manifolds}}}\ (\bibinfo  {publisher} {Springer Science + Business Media,
  LLC},\ \bibinfo {year} {2008})\ pp.\ \bibinfo {pages} {127--130}\BibitemShut
  {NoStop}%
\bibitem [{\citenamefont {Phillips}(2019)}]{Sigmoid}%
  \BibitemOpen
  \bibfield  {author} {\bibinfo {author} {\bibfnamefont {F.}~\bibnamefont
  {Phillips}},\ }\href@noop {} {\bibinfo {title} {{SmootherStep: An improved
  sigmoidal interpolation function}}},\ \bibinfo {howpublished}
  {\url{https://resources.wolframcloud.com/FunctionRepository/resources/SmootherStep/}}
  (\bibinfo {year} {2019}),\ \bibinfo {note} {[Online; first accessed
  22-July-2022]}\BibitemShut {NoStop}%
\bibitem [{\citenamefont {Chaput}(2013)}]{chaput_direct_2013}%
  \BibitemOpen
  \bibfield  {author} {\bibinfo {author} {\bibfnamefont {L.}~\bibnamefont
  {Chaput}},\ }\bibfield  {title} {\bibinfo {title} {Direct {Solution} to the
  {Linearized} {Phonon} {Boltzmann} {Equation}},\ }\href
  {https://doi.org/10.1103/PhysRevLett.110.265506} {\bibfield  {journal}
  {\bibinfo  {journal} {Physical Review Letters}\ }\textbf {\bibinfo {volume}
  {110}},\ \bibinfo {pages} {265506} (\bibinfo {year} {2013})}\BibitemShut
  {NoStop}%
\bibitem [{\citenamefont {{Simoncelli, M. and Marzari, N. and Mauri,
  F.}}(2019)}]{simoncelli2019unified}%
  \BibitemOpen
  \bibfield  {author} {\bibinfo {author} {\bibnamefont {{Simoncelli, M. and
  Marzari, N. and Mauri, F.}}},\ }\bibfield  {title} {\bibinfo {title}
  {{Unified theory of thermal transport in crystals and glasses}},\ }\href
  {https://doi.org/10.1038/s41567-019-0520-x} {\bibfield  {journal} {\bibinfo
  {journal} {Nat. Phys.}\ }\textbf {\bibinfo {volume} {15}},\ \bibinfo {pages}
  {809} (\bibinfo {year} {2019})}\BibitemShut {NoStop}%
\bibitem [{\citenamefont {Simoncelli}\ \emph {et~al.}(2022)\citenamefont
  {Simoncelli}, \citenamefont {Marzari},\ and\ \citenamefont
  {Mauri}}]{simoncelli2021Wigner}%
  \BibitemOpen
  \bibfield  {author} {\bibinfo {author} {\bibfnamefont {M.}~\bibnamefont
  {Simoncelli}}, \bibinfo {author} {\bibfnamefont {N.}~\bibnamefont
  {Marzari}},\ and\ \bibinfo {author} {\bibfnamefont {F.}~\bibnamefont
  {Mauri}},\ }\bibfield  {title} {\bibinfo {title} {Wigner formulation of
  thermal transport in solids},\ }\href
  {https://doi.org/10.1103/PhysRevX.12.041011} {\bibfield  {journal} {\bibinfo
  {journal} {Phys. Rev. X}\ }\textbf {\bibinfo {volume} {12}},\ \bibinfo
  {pages} {041011} (\bibinfo {year} {2022})}\BibitemShut {NoStop}%
\bibitem [{\citenamefont {Caldarelli}\ \emph {et~al.}(2022)\citenamefont
  {Caldarelli}, \citenamefont {Simoncelli}, \citenamefont {Marzari},
  \citenamefont {Mauri},\ and\ \citenamefont
  {Benfatto}}]{caldarelli_many-body_2022}%
  \BibitemOpen
  \bibfield  {author} {\bibinfo {author} {\bibfnamefont {G.}~\bibnamefont
  {Caldarelli}}, \bibinfo {author} {\bibfnamefont {M.}~\bibnamefont
  {Simoncelli}}, \bibinfo {author} {\bibfnamefont {N.}~\bibnamefont {Marzari}},
  \bibinfo {author} {\bibfnamefont {F.}~\bibnamefont {Mauri}},\ and\ \bibinfo
  {author} {\bibfnamefont {L.}~\bibnamefont {Benfatto}},\ }\bibfield  {title}
  {\bibinfo {title} {Many-body {Green}'s function approach to lattice thermal
  transport},\ }\href {https://doi.org/10.1103/PhysRevB.106.024312} {\bibfield
  {journal} {\bibinfo  {journal} {Physical Review B}\ }\textbf {\bibinfo
  {volume} {106}},\ \bibinfo {pages} {024312} (\bibinfo {year}
  {2022})}\BibitemShut {NoStop}%
\end{thebibliography}%


\appendix
\setcounter{figure}{0}
\renewcommand{\thefigure}{SF\arabic{figure}}

\begin{figure*}[!htb]
\includegraphics[width=0.9\textwidth]{8_heat_flux_PRX.pdf}\\[-3mm]
\caption{\label{fig:PRX}
\textbf{Effects of boundary conditions on the heat-flux profile.}
In-plane (x-y) heat fluxes in graphite, obtained solving the VHE
imposing a temperature of $80\ \mathrm{K}$ on the left side ($x=0\ \mathrm{\mu m}$) and $60\ \mathrm{K}$ on the right side ($x=15\ \mathrm{\mu m}$), assuming all boundaries at $x \neq 0\ \mathrm{\mu m}$ and $x \neq 15\ \mathrm{\mu m}$ to be adiabatic, and using no-slip boundary conditions for the drift velocity ($\bm{u}{=}\bm{0}$ at the boundary) in \textbf{a)}\cite{Simoncelli2020}, and a perfectly slipping boundary condition ($\bm{u}\cdot\bm{\hat{n}}=0$, where $\bm{\hat{n}}$ is the versor orthogonal to the boundary) in panel \textbf{b)}.
Panel \textbf{c)}, differences between the heat-flow profile along the vertical sections $x=1.5\ \mathrm{\mu m}$ and $x=9\ \mathrm{\mu m}$ for different boundary conditions.
We highlight how the lubrication-layer approach, used to model slipping boundaries in a numerically convenient way, allows to obtain results practically indistinguishable from those obtained exactly implementing the slipping boundaries. The insets in panel \textbf{c)} show schematically how the lubrication layer is implemented.
}
\end{figure*}

\section{Boundary conditions used in steady-state simulations}
\label{sec:appendix}

\subsection{Lubrication layer and slipping boundaries}
\label{sub:slipping_boundaries}


In this section we discuss the numerical techniques used to simulate different boundary conditions for the drift velocity. 
Specifically, in order to model boundary conditions different from no-slip ($\bm{u}{=}\bm{0}$ at the boundary), which correspond to boundaries that are completely dissipating the crystal momentum, we introduce a "lubrication layer", \textit{i.e.} a region in proximity of the actual boundary in which the viscosity is greatly reduced ($\mu^{ijkl}_{\rm lub}=\mu^{ijkl}/F$, where $F{\geq}1$).
By imposing a no-slip condition on the actual boundary of the device, choosing a lubrication layer width (LLW) to be small ($\lesssim5\%$ of the device's characteristic size), and varying the factor $F$ from 1 to an arbitrarily large number, one can mimic the effect of boundary conditions ranging from no-slip ($\bm{u}{=}\bm{0}$, $F{=}1$) to perfectly slipping ($F{\to}\infty$).
In fact, by reducing the viscosity in the lubrication layer, one reduces the viscous stresses due to presence of the boundary, with the limit $F{=}1$ representing no reduction and yielding the no-slip condition, and the limit $F{\to}\infty$ representing infinite reduction, \textit{i.e.} perfectly slipping boundaries.

We show in Fig.~\ref{fig:PRX} that, in analogy to standard fluid-dynamics, no-slip boundary conditions (panel \textbf{a}, also discussed in Ref.~\cite{Simoncelli2020}) yield a Poiseuille-like heat-flow profile. In contrast, using slipping boundaries yield much weaker (negligible) variations of the heat-flux profile (panel \textbf{b}).
The lubrication-layer approach is particularly advantageous from a numerical viewpoint. In fact, in standard solvers such as \textit{Mathematica} implementing perfectly slipping boundary conditions is straightforward in simple geometries such as that in Fig.~\ref{fig:PRX}, where the  Cartesian components of the drift velocity are always aligned or orthogonal to the boundaries. However, in complex geometries such as that in Fig.~\ref{fig:1_vortex},
it is not possible to find a coordinate system which is always aligned or orthogonal to the boundaries, rendering the numerical implementation of the slipping boundary conditions much more complex.
We show in Fig.~\ref{fig:PRX}\textbf{c)} that by choosing a sufficiently small LLW, and a sufficiently large value for the viscosity reduction factor ($F=10^{4}$), one obtains practically indistinguishable results by using the lubrication layer or the exact slipping boundary conditions.

Thus, the test reported in Fig.~\ref{fig:PRX} justifies the usage of the lubrication layer to simulate slipping boundary conditions in a numerically affordable way. Such an approach has been used to model the complex geometry shown in Fig.~\ref{fig:1_vortex}, in particular we chose a LLW=0.1 $\mu m$ and a viscosity reduction factor $F=1000$ for the lubrication layer.
We conclude by noting that in Fig.~\ref{fig:PRX} \textbf{a)} and \textbf{b)} the area where viscosity assumes the physical value has exactly the same size.


\section{Slipping boundaries and heat vortices}
\label{sec:reflective_boundaries_and_heat_vortices}

After having shown in Sec.~\ref{sub:slipping_boundaries} that the lubrication layer allows to model boundary conditions for the drift velocity ranging from no-slip to perfectly slipping, we use this approach to study how the inversion of the temperature gradient observed in Fig.~\ref{fig:1_vortex} depends on the boundary conditions.
\begin{figure}[b]
\includegraphics[width=\columnwidth]{device.pdf}
\caption{\label{fig:device_ll} 
\textbf{Channel-chamber geometry for observation of steady-state viscous heat backflow.}
The yellow region represents the simulation domain where all the parameters entering in the VHE assume physical values, the green area shows the lubrication-layer region where the viscosity is reduced by a factor $F$ to account for partially reflective boundaries.
We note that the width of the lubrication layer (0.1 $\mu m$), and of the opening connecting the tunnel and the chamber (0.52 $\mu m$), are represented in an exaggerated way for graphical clarity. 
The radius and opening angle used here and in Fig.~\ref{fig:1_vortex} are $R=6\ \mathrm{\mu m}$ and $\theta = 4.9672^\circ$. A temperature of 95 K (45 K) is applied at the lower (upper) opening of the tunnel, all the other boundaries are adiabatic. }
\end{figure}
Fig.~\ref{fig:device_ll} shows how the lubrication layer and external baths at fixed temperature are applied to the tunnel-chamber geometry of Fig.~\ref{fig:1_vortex}.
The large temperature gradient applied at the extremities of the tunnel creates a drift velocity, thanks to the coupling between temperature gradient and drift-velocity shown in Eq.~\ref{viscous_heat_U}.
Therefore, significant values for both the  temperature-gradient ($\bm{Q}^\delta$) and drift-velocity ($\bm{Q}^D$) heat fluxes are present at the opening of the circular chamber. 
When these heat fluxes leak into the chamber they are subject to a significant gradient; for the drift-velocity component of the heat flux, a significant gradient causes viscous shear, and the vortex forms to minimize such shear.
In contrast, as discussed in the main text, viscous forces are absent in Fourier's law, which trivially yields an irrotational flow. 


Now we want to investigate how the signature of viscous heat backflow discussed in Fig.~\ref{fig:1_vortex}, \textit{i.e.} the temperature gradient in the chamber reversed compared to the tunnel, depends on the boundary conditions.
To this aim, we look at how the temperature difference between two points in the chamber $A=(3,3.5)\mu m$ and $B=(3,-3.5)\mu m$ depends on the boundary conditions adopted.
Positive temperature difference $\Delta T=T_A-T_B$ is observed when viscous heat backflow causes a behavior analogous to Fig.~\ref{fig:1_vortex}\textbf{b}, while negative temperature difference is obtained when the system displays a behavior analogous to Fourier's law (Fig.~\ref{fig:1_vortex}\textbf{a}).
We show in Fig.~\ref{fig:slip_cont} how such temperature difference changes as a function of the viscosity reduction in the lubrication layer. We see that as the viscosity in the lubrication layer is reduced, $\Delta T$ switches from negative (Fourier-like) to positive (vortex-like), indicating that (at least partially) slipping boundaries are needed to observe the temperature inversion related to viscous heat backflow.
We highlight how in the limit of vanishingly low viscosity for the lubrication layer $\Delta T$ converges to a constant, indicating that the viscosity in the lubrication layer is so reduced that the overall effect is that of perfectly slipping boundaries.  

\begin{figure}[t]
\includegraphics[width=\columnwidth]{slip_dependence_tunnel_new.pdf}\\[-4mm]
\caption{\label{fig:slip_cont}
\textbf{Dependence of the temperature inversion due to viscous backflow on the boundary conditions.}
We plot the temperature difference $\Delta T = T_\mathrm{A} - T_\mathrm{B}$, where $A=(3,3.5)\mu m$ and $B=(3,-3.5)\mu m$, measured in the circular chamber of Fig.~\ref{fig:1_vortex} as a function of the reduction of viscosity in the lubrication layer with LLW=$0.1\mu m$.  
As the viscosity in the lubrication layer is reduced, the drifting component of the heat flow encounters less resistance when flowing tangential to the circular boundary. This analysis was performed simulating samples with isotopic-mass disorder at natural concentration.
}
\end{figure}

In summary, the results in Fig.~\ref{fig:slip_cont} suggest that smooth boundaries are needed to observe heat vortices and related temperature inversion, and conversely measurements of hydrodynamic heat transport may be use to characterize the roughness of devices' boundaries, \textit{i.e.} their capability to reflect phonon's crystal momentum.



\section{Effects of temperature, size, and isotopic disorder on viscous heat backflow}
\label{sec:effects_of_temperature_and_isotopic_disorder_on_viscous_heat_backflow}
\begin{figure}[htbp!]
\includegraphics[width=\columnwidth]{tunnel_cont_nat.pdf}
\includegraphics[width=\columnwidth]{tunnel_cont_99d9.pdf}\\[-4mm]
\caption{\label{fig:tunnel_cont}\textbf{Effects of temperature, size, and isotopic disorder on viscous heat backflow.}  The colormap shows how the temperature inversion due to viscous backflow, computed as the difference between the maximum temperature in the upper part of the chamber ($y>0\mu m$) and the minimum temperature in the lower part in the chamber ($y<0\mu m$), depends on the radius of the chamber and on the average temperature ${T}_{\rm eq}$ of the tunnel (a perturbation of $\delta T={T}_{\rm eq}\pm25 K$ is always used at the extremities of the tunnel).
The upper panel refers to samples with natural-abundance isotopic disorder, the lower panel shows isotopically pure samples.
}
\end{figure}
In this section we explore how the temperature inversion due to viscous heat backflow depends on the size of the device, the average temperature around which the temperature gradient is applied, and the presence of isotopic-mass disorder (this latter affects the parameters such as intrinsic conductivity, and viscosity of the sample).
We show in \ref{fig:tunnel_cont} that the temperature inversion due to viscous heat backflow is maximized at average temperatures around 50-70 K and at sizes around 2$\mu m$, and  in isotopically pure samples (99.9 \% $^{12}$C, 0.1 \% $^{13}$C) persists up to temperatures slightly higher than those observed for natural samples (98.9 \% $^{12}$C, 1.1 \% $^{13}$C). We note finite-size effects on the parameters entering in the VHE have been taken into account following the approach detailed in Ref.~\cite{Simoncelli2020}.


\section{Boundary conditions for time-dependent simulations}
\label{sec:boundary_conditions_for_time_dependent_simulations}

\subsection{Thermalisation lengthscale}
\label{sub:modeling_realistic_thermalisation}
In actual experimental devices thermalisation is not perfectly localized in space, but occurs over a finite lengthscale\cite{braun_spatially_2022}, \textit{i.e.} there is a smooth transition from the device interior to the thermal bath.
Therefore, we model the boundaries relying on a compact sigmoid function \cite{Bump} to smoothly connect the a thermal bath region, where temperature is fixed and drift velocity is zero (since at equilibrium phonons are distributed according to the Bose-Einstein distribution, which has zero drift velocity\cite{Simoncelli2020}), and the device's interior, where the evolution of temperature and drift velocity is governed by the viscous heat equations.

Specifically, we used the smooth-step function \cite{Sigmoid} that is ubiquitously employed in numerical calculations, \textit{i.e.} a sigmoid Hermite interpolation between $0$ and $1$ of a polynomial of order $5$:
\begin{equation}
\hspace{-1mm}f(r){=}
    \begin{cases}
        0, &  \hspace{-8mm}{\rm if\;}\hspace{5mm}d(r) < 0\\
        1 ,& \hspace{-8mm} {\rm if\;}\hspace{5mm}d(r) > 1\\
        {6[d(r)]^5{-}15[d(r)]^4{+}10[d(r)]^3\hspace*{-1mm},} & {\rm otherwise}
    \end{cases}  
    \label{eq:sigmoid}    
\end{equation}
where $d(r)=\left(\frac{r{-}R_{in}}{R_{out}{-}R_{in}}\right)$ and $r>0$, as well as $R_\text{in}>0$ and $R_\text{out}>0$. The function is second order continuous and saturates exactly to zero and one at the points $R_\text{in}$ and $R_\text{out}$, respectively. We show in Fig.~\ref{fig:sig_f} the sigmoid function~(\ref{eq:sigmoid}) for $R_\text{in}=10\mu m$ and  $R_\text{out}$ ranging from 11 to 20 $\mu m$: $R_\text{out}{\simeq}R_\text{in}$ simulates an ideal thermalisation (occurring over a negligible lengthscale), while $R_\text{out}{\gg}R_\text{in}$ simulates a very inefficient (non-ideal) thermalisation that occurs over a very large lengthscale.
\begin{figure}[b]
\includegraphics[width=\columnwidth]{3_sigmoid.pdf}
\caption{\label{fig:sig_f}
\textbf{Sigmoid function to model realistic thermalisation.}
The sigmoid function~(\ref{eq:sigmoid}) is plotted for $R_\text{in}=10\mu m$ and various values of $R_\text{out}$. 
$R_\text{out}=11\mu m$ (red) simulates a thermalisation that occurs over a very short lengthscale and is thus close to ideal, $R_\text{out}=15\mu m$ (orange) is an estimate of a realistic thermalisation length\cite{braun_spatially_2022}, and $R_\text{out}=20\mu m$ (green) represents thermalisation less efficient than in the previous cases.
}
\end{figure}
Then,  relying on the sigmoid~(\ref{eq:sigmoid}) we connected smoothly the thermal bath region, where $f(\bm r)=1$ and $T({\bm r}, t)={T}_{\rm eq}$ and $\bm{u}({\bm r}, t)=\bm{0}$, and the device's interior region, where $f(\bm r)=0$ and the evolution of $T({\bm r}, t)$ and $\bm{u}({\bm r}, t)$:
\begin{widetext}
\begin{align}
&f(\bm r)[T({\bm r}, t) - {T}_{\rm eq}]+[1-f(\bm r)]\left(C\frac{\partial T({\bm r}, t)}{\partial t} + \sum_{i,j=1}^{3} \alpha^{ij}\frac{\partial u^j({\bm r}, t)}{\partial r^i} - \sum_{i,j=1}^{3} \kappa^{ij} \frac{\partial^2 T ({\bm r}, t)}{\partial r^i \partial r^j} - \dot{q}({\bm r}, t)\right)=0,
 \label{viscous_heat_T_app}\\
&f(\bm r)u^i({\bm r}, t)+[1-f(\bm r)]\left(A^i\frac{\partial u^i({\bm r}, t)}{\partial t} +  \sum_{j=1}^{3}
\beta^{ij}
\frac{\partial T({\bm r}, t)}{\partial r^j} - \sum_{j,k,l=1}^{3} \mu^{ijkl} \frac{\partial^2 u^k({\bm r}, t)}{\partial r^j \partial r^l} + \sum_{j=1}^{3} \gamma^{ij} u^j({\bm r}, t)\right) = 0.\label{viscous_heat_U_app}
\end{align}
\end{widetext}
Details on how the non-ideal thermalisation affects the magnitude of signatures of heat hydrodynamics are reported in the next section.

\subsection{Modeling realistic thermalisation in a rectangular geometry}
\label{sub:lattice_cooling_boundaries}
In order to apply the "realistic thermalisation" boundary conditions discussed in Sec.~\ref{sub:modeling_realistic_thermalisation} to the rectangular device of Fig.~\ref{fig:fig2} we employed a smoothed rectangular domain, since we noted that smooth simulation domains yielded better computational performances (faster convergence with respect to the discretization mesh, reduced numerical noise).
The expression for the rectangular domain employed in Figs.~\ref{fig:fig2},\ref{fig:rect_freq},\ref{fig:freq_cont} is:  
\begin{equation*}
    \left(\frac{|x|}{a}\right)^{\frac{2a}{\lambda }}+\left(\frac{|y|}{b}\right)^{\frac{2b}{\lambda }}=1,
    \label{eq:recticircle}
\end{equation*}
where we used $\lambda=\frac{a}{3}$ to obtain sharp edges.

\section{DPLE as inviscid limit of the VHE}
\label{sec:DPLE_derivation}
In this section we show analytically that in the limit of vanishing viscosity the VHE\cite{Simoncelli2020} reduce to the DPLE\cite{joseph_heat_1989,tzou_unified_1995}.
We start the viscous heat equations without the viscosity and source term,
\begin{eqnarray}
C\frac{\partial T(\bm{r},t)}{\partial t} {+}\!\! \sum_{ij}\alpha^{ij}\frac{\partial u^j(\bm{r},t)}{\partial x^i}{-}\!\!  \sum_{ij}\kappa^{ij}\frac{\partial^2}{\partial x^i \partial x^j} T(\bm{r},t) {=} 0,\hspace*{6mm}\label{eq:e1}
\\
A_k\frac{\partial u_k(\bm{r},t)}{\partial t} + \sum_i \gamma^i_k u^i(\bm{r},t) + \sum_i\beta^i_k\frac{\partial T(\bm{r},t)}{\partial x^i}= 0,\hspace*{10mm}\label{eq:e2}
\end{eqnarray}
where the index $k$ denotes a generic Cartesian component.
Then we consider a setup where the tensors appearing in Eqs.~(\ref{eq:e1},\ref{eq:e2}) can be considered diagonal and isotropic (as is the case for a graphitic device with non-homogeneities exclusively in the basal plane, see Sec.~\ref{sec:parameters_entering_in_the_viscous_heat_equations}).
Thus, writing the second equation by components for the $2$-dimensional case:
\begin{eqnarray}
C\frac{\partial T}{\partial t}+ \alpha\left(\frac{\partial u_x}{\partial x} + \frac{\partial u_y}{\partial y}\right) - \kappa\left(\frac{\partial^2 T}{\partial x^2} + \frac{\partial^2 T}{\partial y^2}\right) = 0,\label{eq:d1} \hspace*{8mm}\\
A\frac{\partial u_x}{\partial t} + \gamma u_x + \beta\frac{\partial T}{\partial x} = 0,\label{eq:d2}\hspace*{20mm}\\
A\frac{\partial u_y}{\partial t} + \gamma u_y + \beta\frac{\partial T}{\partial y} = 0.\label{eq:d3}\hspace*{20mm}
\end{eqnarray}
Deriving both sides of Eq.~(\ref{eq:d2}) by $x$, and both sides of Eq.~(\ref{eq:d3}) by $y$, we obtain:
\begin{eqnarray}
\left(A\frac{\partial}{\partial t} + \gamma\right) \frac{\partial u_x}{\partial x}  = - \beta\frac{\partial^2 T}{\partial x^2} \label{eq:bf1}; \\
\left(A\frac{\partial}{\partial t} + \gamma\right) \frac{\partial u_y}{\partial y}  = - \beta\frac{\partial^2 T}{\partial y^2}\label{eq:bf2}.
\end{eqnarray}
Then, we sum Eq.~(\ref{eq:bf1}) and Eq.~(\ref{eq:bf2}), obtaining:
\begin{eqnarray}
\left(A\frac{\partial}{\partial t} {+} \gamma\right)\!\! \left(\frac{\partial u_x}{\partial x}{+}
\frac{\partial u_y}{\partial y}\right)\!
 {=} {-} \beta\bigg(\frac{\partial^2 T}{\partial x^2}{+}\frac{\partial^2 T}{\partial y^2}\bigg).\hspace*{5mm}
 \label{eq:sum_2eqs}
\end{eqnarray}
Now we notice that applying the operator $\hat{\mathcal O}=\left(A\frac{\partial}{\partial t} {+} \gamma\right)$ to both sides of Eq.~\ref{eq:d1}, and by using Eq.~(\ref{eq:sum_2eqs}), we manage to write an equation for temperature only:
\begin{equation}
\begin{gathered}
\frac{CA}{\gamma} \frac{\partial^2 T}{\partial t^2} + C\frac{\partial T}{\partial t} - \left( \frac{\alpha\beta}{\gamma} + \kappa \right) \left(\frac{\partial^2 T}{\partial x^2} + \frac{\partial^2 T}{\partial y^2}\right)  \\ -\frac{\kappa A}{\gamma}\frac{\partial}{\partial t} \left(\frac{\partial^2 T}{\partial x^2} + \frac{\partial^2 T}{\partial y^2}\right) = 0.\label{eq:DPLE_final}
\end{gathered}
\end{equation}
Eq.~\ref{eq:DPLE_final} is exactly the DPLE equation discussed by \citet{joseph_heat_1989} and \citet{tzou_unified_1995}.
We can now see the relationship between the constants appearing in the VHE, and the time lags appearing in the DPLE\cite{tzou_unified_1995} and discussed in the main text: $\tau_T = \frac{\kappa A}{\alpha\beta + \kappa\gamma}$, $\tau_Q = \frac{A}{\gamma}$, we conclude by noting that the thermal diffusivity appearing in the DPLE is  $\alpha_E = \frac{\alpha\beta + \kappa\gamma}{C\gamma}$.
It can be shown that in the limit $\tau_T=\tau_Q$ the DPLE becomes equivalent to Fourier law, while in the limit $\tau_T=0$ it becomes equivalent to Cattaneo's second-sound equation\cite{tzou_unified_1995}.



\section{Effects of viscosity on lattice cooling}
\label{sec:effects_of_viscosity_on_lattice_cooling}

\subsection{Transient heat backflow from the DPLE}
In this section we discuss how the thermal viscosity affect the temperature oscillations. 
Fig.~\ref{fig:figDPLE} shows that solving the viscous heat equations in the inviscid limit ($\mu=0$)---which is analytically equivalent to solving the DPLE equation, see Sec.~\ref{sec:DPLE_derivation}---yields temperature oscillations with a higher magnitude compared to those obtained in the case of the viscous heat equations. 
\begin{figure}[t]
	\centering
	\includegraphics[width=0.49\textwidth]{2_JPTE.png}
	
	\caption{\textbf{Transient hydrodynamic heat backflow and lattice cooling in the DPLE (inviscid) limit.}
	We show the DPLE predictions for the relaxation in time of a temperature perturbation (obtained applying a localized heater to a device in the time interval from 0 to 0.4 ns) in a graphitic device thermalised at 80 K the boundaries (thermalisation occurs in shaded regions, see SM~\ref{sub:modeling_realistic_thermalisation}).
	Rows show different instants in time for temperature (left column), temperature-gradient heat-flux component ($\bm{Q}^{\delta}$, central column), and drifting heat-flux component ($\bm{Q}^{D}$, right column). 
	The emergence of lattice cooling, \textit{i.e.} a temperature locally and transiently lower than the initial value $T{=}80$ K, originates from the lagging coupled evolution of $\bm{Q}^{\delta}$ and $\bm{Q}^{D}$. }
	\label{fig:figDPLE}
\end{figure}
Therefore, Fig.~\ref{fig:figDPLE} shows that a finite viscosity is not necessary to observe transient heat backflow, since a lagged response between temperature gradient and heat flux can emerge also in the inviscid limit, as detailed in Sec.~\ref{sec:DPLE_derivation}.
However, we will show later in Sec.~\ref{sec:comparison_between_theory_and_experiments_for_lattice_cooling} that accounting for the thermal viscous is needed to obtain quantitative agreement between the relaxation timescales predicted from theory and those measured in experiments\cite{Jeong2021}.

\subsection{Diffusive (Fourier) relaxation}
\label{sub:diffusive_}
We show in Fig.~\ref{fig:figFourier} that temperature oscillations do not emerge from Fourier's law.
This behavior was trivially expected from an analytical analysis of Fourier's  diffusive equation, 
$C\frac{\partial T}{\partial t}{-} \sum_{ij}\kappa^{ij}\frac{\partial^2 T}{\partial r^ir^y}  = 0$, whose smoothing property \cite{skinner_university_nodate} implies that the evolution of a positive temperature perturbation relaxes to equilibrium decaying in a monotonic way.
Also, within Fourier's law the heat flux has one single component proportional to the temperature gradient, so heat backflow is impossible.
\begin{figure}[t]
	\centering
	\includegraphics[width=\columnwidth]{2_Fourier.png}
	\caption{\textbf{Temporal evolution of a localized temperature perturbation according to Fourier's law.} The left column presents the temperature profile, while the right column shows the heat flux. Each row is a snapshot of the sample taken at a time indicated on the left.}
	\label{fig:figFourier}
\end{figure}


\subsection{Comparison between viscous and inviscid relaxation}
\label{sub:inviscid}
The device geometry, the transient perturbation, and the boundary conditions used in Figs. \ref{fig:fig2},\ref{fig:figDPLE},\ref{fig:figFourier}   are exactly the same. We show in Fig.~\ref{fig:compare_all} the evolution in time of the temperature in the point $\bm{r}_c{=}(5 \mu m,0\mu m$) where the temperature perturbation is centered.
We recall that we used the following temperature perturbation in Eq.~\ref{viscous_heat_T_app} and Eq.~\ref{viscous_heat_U_app}
\begin{equation}
	\dot{q}(\bm{r},t){=}\mathcal{H}\;\theta(t_{\rm heat}{-}t) \exp\left[-\tfrac{(x+x_c)^2}{2\sigma_x^2}{-}\tfrac{y^2}{2\sigma_y^2}\right],
	\label{eq:pert_LC}
\end{equation}
with $\mathcal{H}=0.013\tfrac{W}{\mu m^3}$, $t_{\rm heat}=0.4ns$,  $x_c=5 \mu m$, $\sigma_x=2\mu m$, $\sigma_y=2.8\mu m$, to ensure that the perturbation created causes variations within 10\% of the equilibrium temperature ($T_{eq}=80 K$). 

\begin{figure}[t]
	\centering
	\includegraphics[width=\columnwidth]{lattice_cooling_spot.pdf}
	\caption{\textbf{Temperature relaxation from VHE, DPLE, and Fourier's law.} 
	The temporal evolution of the temperature in the center ($5 \mu m,0\mu m$) of the external heat source (switched off at t=0.4 ns, see text for details) is shown in red for Fourier's law, in blue for the  DPLE, and in green for the  VHE. Temperature oscillations are visible both in the viscous VHE and inviscid DPLE cases (with a larger amplitude in the inviscid case), and are absent in Fourier's law.}
	\label{fig:compare_all}
\end{figure}


\subsection{Dependence of lattice cooling on the boundary conditions }
\label{ssec:dependence_of_lattice_cooling_from_boundary_conditions}
In this section we investigate how the magnitude of the viscous temperature oscillations discussed in Fig.~\ref{fig:compare_all} depends on the thermalisation lengthscale at the boundaries (see Sec.~\ref{sub:modeling_realistic_thermalisation}).
In order to quantify the magnitude of temperature oscillations, we define a descriptor to capture the ``lattice cooling stregth'' (LCS) 
\begin{equation}
	LCS= \frac{T_\mathrm{eq}-T_\mathrm{min}}{T_\mathrm{max}-T_\mathrm{min}},
	\label{eq:LCS}
\end{equation}
where $T_\mathrm{max}=\max_t[T(\bm{r}_c,t)]$ and $T_\mathrm{min}=\min_t[T(\bm{r}_c,t)]$ are the maximum and minimum temperatures, respectively, observed during the relaxation in the point $\bm{r}_c$, where the perturbation~(\ref{eq:pert_LC}) is centered.
Clearly, LCS is zero if lattice cooling does not take place, and assumes a positive value if lattice cooling occurs.
To gain insights on how  lattice cooling is affected by the thermalisation lengthscale, we fixed the 
size of the non-thermalised region (where the sigmoid function~\ref{eq:sigmoid} is zero) to $20\ \mathrm{\mu m}\times 16\ \mathrm{\mu m}$ (\textit{i.e.} exactly as in Fig.~\ref{fig:fig2}), thus we tested how the LCS emerging from the VHE varies by thermalising the boundaries with a different lengthscale, as discussed in Sec.~\ref{sub:modeling_realistic_thermalisation}.
We show in Fig.~\ref{fig:sig} that LCS assumes smaller values as the thermalisation becomes weaker. 
We highlight how lattice cooling is visible for thermalisation lengthscales as large as $10\ \mu m$ (equal to half the size of the simulated device).
In Figs.~\ref{fig:fig2},\ref{fig:figDPLE},\ref{fig:figFourier}, and in the following, we use a thermalisation lengthscale of 2 $\mu m$, a value that is expected to be a realistic representation of experimental conditions\cite{braun_spatially_2022}.

\begin{figure}[t]
\includegraphics[width=\columnwidth]{drop_dependence_line_nat.pdf}
\caption{\label{fig:sig}\textbf{Dependence of lattice cooling strength on the thermalisation lengthscale.} 
We show how LCS in a device having non-thermalised region equal to $20\ \mathrm{\mu m}{\times} 16\ \mathrm{\mu m}$ (as in Fig.~\ref{fig:fig2}) depends on the thermalisation lengthscale, \textit{i.e.} the length over which the sigmoid function shown in Fig.~\ref{fig:sig} goes from zero to one.
Even if LCS decreases as the thermalisation lengthscale increases, LCS remains appreciable even at thermalisation lengthscales as large as 10 $\mu m$.
}
\end{figure}



\subsection{Effects of isotopic disorder, average temperature, and size on lattice cooling}
\label{sub:rescaling}
In this section we discuss how isotopic-mass disorder, sample-size, and temperature affect the LCS in the VHE. 
The effect of isotopic-mass disorder is taken into account by the numerical parameters appearing in the VHE. These were computed from first principles for graphite with natural concentration of isotopes (98.9 \% $^{12}$C, 1.1 \% $^{13}$C, see Ref.~\cite{Simoncelli2020} for computational details and Table I in that reference for numerical parameters) and for isotopically purified samples (99.9 \% $^{12}$C, 0.1 \% $^{13}$C, see Sec.~\ref{sec:parameters_entering_in_the_viscous_heat_equations}). 
The effect of sample size was considered by uniformly rescaling the simulation domain~(\ref{eq:recticircle}) and the perturbation~(\ref{eq:pert_LC}). We considered the effect of finite-size and grain-boundary scattering as discussed in Ref.~\cite{Simoncelli2020}; specifically, we considered a grain size of $20 \mu m$ that is realistic for high-quality samples\cite{Jeong2021}. 
The temperature measurements were done in the same point where the perturbation was applied.
The effect of equilibrium temperature was taken into account through the temperature dependence of the parameters entering in the viscous heat equations, as shown by Table I in  Ref.~\cite{Simoncelli2020} and Table.~\ref{tab:param_graphite} at the end of this manuscript.

\begin{figure}[t]
\centering
\includegraphics[width=0.47\textwidth]{Jeong_2_comp.pdf}
\includegraphics[width=\columnwidth]{LC_isopure_sigma_VHE.pdf}
\caption{\label{fig:iso_pure} 
\textbf{Lattice cooling strength as a function of size, temperature, and isotopic disorder.}
The lattice cooling strength~(\ref{eq:LCS}) is reported as a function of the characteristic size of the rectangle and equilibrium temperature around which the perturbation is applied. 
The white cross in the upper panel represent the simulation conditions which are closest to the single-spot experiments discussed in Sec. 3 of the supplementary material of Ref.~\cite{Jeong2021}---the lack of lattice cooling predicted under these conditions agrees with the experiments.
We note that lattice cooling in natural samples  (upper panel) is weaker compared to isotopically pure samples (bottom panel).
}
\end{figure}

We show in Fig.~\ref{fig:iso_pure} that in natural samples lattice cooling is maximized at temperatures around 70-120 K and is significant in devices having size 5-20 $\mu m$. Importantly, LCS becomes negligible in devices having size larger than 30 $\mu m$.
We note that Ref.~\cite{Jeong2021} performed pump-probe experiments in graphite using spots radius equal to 6 $\mu m$ and without applying thermalisation at the boundaries (details are reported in the Supplementary Material of that reference). 
When we used a simulation setup similar to the experiments of Ref.~\cite{Jeong2021}, \textit{i.e.} $\sigma_x=6 \mu m$ in Eq.~(\ref{eq:pert_LC}) and thermalised boundaries separated by a large distance (80 $\mu m$), we did not find temperature oscillations (LCS=0), in agreement with the experiments of Ref.\cite{Jeong2021}.
We also simulated the ring-shaped geometry used by \citet{Jeong2021}, finding very good agreement between the theoretical relaxation timescales and the experimental ones, as we will show in the next section.

\subsection{Lattice cooling in a ring-shaped geometry}
\label{sec:comparison_between_theory_and_experiments_for_lattice_cooling}
In this section we simulate a ring-shaped perturbation close to the experimental setup by \citet{Jeong2021}.

We used a 15-$\mu m$-diameter ring-shape pump, with Gaussian profile having full width at half a maximum equal to $3\ \mathrm{\mu m}$ (these values were confirmed also by analyzing Fig. 1c in Ref.~\cite{Jeong2021}):
\begin{equation}
	\dot{q}(\bm{r},t){=}\mathcal{H}\;\theta(t_{\rm heat}{-}t) \exp\left[-\tfrac{(|\bm{r}|-r_{\rm{heater}})^2}{2\sigma_r^2}\right],
	\label{eq:pert_jeong}
\end{equation}
We used a circular simulation domain with diameter equal to $30\ \mathrm{\mu m}$. We simulated thermalisation with an external heat bath happening in the annular region having inner diameter $20\ \mathrm{\mu m}$ and the outer diameter equal to $30\ \mathrm{\mu m}$, using the approach to describe thermalisation with the environment in a realistic way discussed in Sec.~\ref{sub:modeling_realistic_thermalisation}.
We used the parameters for graphite at natural isotopic concentration (computed from first-principles and reported in Table I of Ref.~\cite{Simoncelli2020}), considering a grain size of $20 \mu m$ (estimated from the largest grains in Fig. S1 of Ref.\cite{Jeong2021}).
\begin{figure}[t]
\includegraphics[width=\columnwidth]{combine_JEong.pdf}
\caption{\label{fig:jeong} 
\textbf{Lattice cooling in a ring-shaped geometry.}
\textbf{a)}, time evolution of the temperature profile from the viscous heat equations. 
\textbf{b)},  evolution of the temperature measured in the center of the ring obtained from the VHE (green), DPLE (blue), or Fourier (red). Grey, normalized reflectance measured by \citet{Jeong2021} (grey); the sign of the normalized reflectance signal is indicated by the background color: red is positive and blue is negative.
The experimental relaxation timescale, \textit{i.e.} the time instant at which the normalized reflectance signal becomes lower than the initial value, is in agreement with the relaxation timescale predicted by the VHE (see text). }
\end{figure}
The ring-shaped perturbation was applied for $0.4\ ns$ and then switched off, as in the experiments. 
We report in Fig.~\ref{fig:jeong} the temporal evolution of the temperature measured in the center of the ring, computed from the viscous VHE, the inviscid DPLE, and from Fourier's law. The relaxation timescales emerging from the VHE match the experimental relaxation timescales discussed by \citet{Jeong2021}. In particular, we note that the time at which the normalised reflectance signal changes sign matches the time at which the temperature computed by the VHE changes sign. Also, the DPLE (\textit{i.e.} the VHE with $\mu=0$) predict the relaxation to be faster.
We concluded by noting that the pioneering experiments of Ref.~\cite{Jeong2021} showed a non-negligible noise in the normalized reflectance signal (see Fig.~2 in Ref.~\cite{Jeong2021}) and for this reason they focused on the sign of the normalized reflectance (shown by the background color in Fig.~\ref{fig:jeong}) rather than on the absolute values. For these reasons, we focused on comparing the times at which $T(\bm{r},t)-T_{eq}$ changes sign, and the time at which the experimental normalized reflectance changes sign.
Our results suggest that it is necessary to account for the thermal viscosity in order to obtain agreement between the experimental and theoretical time at which lattice cooling starts.


\section{Parameters entering in the viscous heat equations}
\label{sec:parameters_entering_in_the_viscous_heat_equations}
In order to simplify the notation for the paramenters entering in the viscous heat equations~(\ref{viscous_heat_T},\ref{viscous_heat_U}), we defined the following parameters: $\alpha^{ji}=W_{0j}^i\sqrt{\overline T A^jC}$, $\beta^{ij}=\sqrt{\frac{CA^i}{\overline T}} W_{i0}^j$, $\gamma^{ij}=\sqrt{A^i A^j} D_U^{ij}$. 
$C=\frac{1}{k_B \bar{T}^2 V}\sum_{\state} \bar{N}_{\state}\big(\bar{N}_{\state}+1\big)(\hbar \omega_{\state})^2$ is the specific heat, 
${W}_{0 j}^i = \frac{1}{V} \sum_{\state} \phi_{\state}^0{v}^i_{\state} \phi_{\state}^j$ is a velocity tensor that arises from the non-diagonal form of the diffusion operator in the basis of the eigenvectors of the normal part of the scattering matrix,  
$\bar{T}$ is the reference (equilibrium) temperature on which a perturbation is applied,
$A^i$ is the specific momentum in direction $i$. All these parameters were computed from first principles, see Ref.~\cite{Simoncelli2020} for computational details. The parameters for graphite with natural-abundance isotopic-mass disorder (98.9 \% $^{12}$C, 1.1 \% $^{13}$C) are reported in Table I of Ref.\cite{Simoncelli2020}. The parameters for isotopically pure samples (99.9 \% $^{12}$C, 0.1 \% $^{13}$C) are reported in Table.~\ref{tab:param_graphite}.
We conclude by noting that in this work we employed the approximation of considering the parameters entering in the VHE, DPLE, and Fourier's equation to be independent from frequency. This approximation is realistic in the MHz range considered here, as discussed by Ref.~\cite{chaput_direct_2013}.

\begin{widetext}

\begin{table*}[!htb]
  \caption{Parameters entering the viscous heat equations for isotopically purified graphite (99.9 \% $^{12}$C, 0.1 \% $^{13}$C). We report here only the in-plane components of the tensors needed to perform the calculations: 
 $\kappa^{ij}_{\rm P}=\kappa_{\rm P}\delta^{ij}$, $\kappa^{ij}_{\rm C}=\kappa_{\rm C}\delta^{ij}$ (see Refs.~\cite{simoncelli2019unified,simoncelli2021Wigner,caldarelli_many-body_2022} for details on $\kappa_{\rm C}$), $K^{ij}_{S}=K_{S}\delta^{ij}$, ${D}_{U,{\rm bulk}}^{ij}={D}_{U,{\rm bulk}}\delta^{ij}$, ${F}_U^{ij}={F}_U\delta^{ij}$, $A=A^i\;\forall \;i$, $W^j_{0i}=W^j_{i0}= W\delta^{ij}$, where the indexes $i,j$ represent the in-plane directions $x,y$ only ($i,j=1,2$ and $i\neq j$).}
  \label{tab:param_graphite}
  \centering
\resizebox{\textwidth}{!}{%
  \begin{tabular}{ccccccccccccccc}
  \hline
  \hline
  {T [K]} & 
  {$\kappa_{\rm P}\;\big[\rm\frac{W}{m{\cdot}K}\big]$} & {$\kappa_{\rm C}\;\big[\rm\frac{W}{m{\cdot}K}\big]$} & ${K_S\;\big[\rm\frac{W}{m^2{\cdot}K}\big]}$ & 
  $\mu^{iiii}_{\rm bulk}\;[\rm Pa{\cdot}s]$ & $\mu^{ijij}_{\rm bulk}\;[\rm Pa{\cdot}s]$ & $\mu^{ijji}_{\rm bulk}\;[\rm Pa{\cdot}s]$ &
    $M^{iiii}\;[\rm\frac{ Pa{\cdot}s}{m}]$ & $M^{ijij}\;[\rm\frac{ Pa{\cdot}s}{m}]$& $M^{ijji}\;[\rm\frac{ Pa{\cdot}s}{m}]$ &
    ${D}_{U,{\rm bulk}}\;[\rm ns^{-1}]$  &  ${F}_U\;[\rm \frac{m}{s}]$ &
    A $\big[\rm\frac{ pg}{\mu m^3}\big]$ &  $C$ $\big[\rm\frac{ pg}{\mu m{\cdot}ns^2{\cdot}K}\big]$ & W $\big[\rm \frac{\mu m}{ns}\big]$\\
  \hline
%T[K],            k_exactP, sma_C[W/(mK)],PrBal_k_ii[W/(m^2 K)] Exact mu:iiii,   ijij,  (ijji+iijj)/2 [Pa.S]  PrefBalmu:iiii, ijij,    (ijji+iijj)/2[Pa.S/m]   D_umk[ns^-1],Pr_Bal_Dumk_plan[m/s],A[pg/((\mu m)^3)],C[pg/((\mum)^3*ns^2*K)],W_const[\mu m/ns]]
50 & 			      3.34175e+04 & 8.17849e-05 & 2.57075e+08 &			      1.11244e-03 & 5.62103e-04 & 2.71810e-04 &			      4.39176e+02 & 1.46638e+02 & 1.46269e+02 &			      2.41659e-02 & 4.26877e+03 &			      1.53433e-04 & 1.00900e-04 & 2.72761e+00 \\
60 & 			      3.52935e+04 & 1.44119e-04 & 3.95197e+08 &			      1.09365e-03 & 5.84333e-04 & 2.51162e-04 &			      7.15578e+02 & 2.39244e+02 & 2.38166e+02 &			      4.34202e-02 & 4.59801e+03 &			      2.29498e-04 & 1.41101e-04 & 3.00647e+00 \\
70 & 			      3.24978e+04 & 2.40448e-04 & 5.59073e+08 &			      1.09700e-03 & 6.15163e-04 & 2.37329e-04 &			      1.07264e+03 & 3.59308e+02 & 3.56661e+02 &			      8.50411e-02 & 4.88642e+03 &			      3.20256e-04 & 1.84524e-04 & 3.25249e+00 \\
80 & 			      2.74850e+04 & 4.06335e-04 & 7.45726e+08 &			      1.11813e-03 & 6.50051e-04 & 2.30398e-04 &			      1.51011e+03 & 5.07190e+02 & 5.01440e+02 &			      1.68503e-01 & 5.13608e+03 &			      4.25102e-04 & 2.30997e-04 & 3.45924e+00 \\
90 & 			      2.25348e+04 & 7.19062e-04 & 9.52361e+08 &			      1.15613e-03 & 6.86615e-04 & 2.31108e-04 &			      2.02532e+03 & 6.82641e+02 & 6.71286e+02 &			      3.18608e-01 & 5.34894e+03 &			      5.43532e-04 & 2.80534e-04 & 3.62371e+00 \\
100 & 			      1.84465e+04 & 1.30478e-03 & 1.17648e+09 &			      1.21344e-03 & 7.23418e-04 & 2.41404e-04 &			      2.61392e+03 & 8.85021e+02 & 8.64331e+02 &			      5.63474e-01 & 5.52812e+03 &			      6.75030e-04 & 3.33160e-04 & 3.74625e+00 \\
120 & 			      1.27814e+04 & 3.97919e-03 & 1.66857e+09 &			      1.39850e-03 & 7.94571e-04 & 2.98592e-04 &			      3.98943e+03 & 1.36695e+03 & 1.31087e+03 &			      1.45053e+00 & 5.80204e+03 &			      9.74775e-04 & 4.47442e-04 & 3.88006e+00 \\
140 & 			      9.33971e+03 & 9.87325e-03 & 2.20653e+09 &			      1.67560e-03 & 8.59931e-04 & 4.04889e-04 &			      5.59088e+03 & 1.94459e+03 & 1.82232e+03 &			      3.02867e+00 & 5.99201e+03 &			      1.31862e-03 & 5.72512e-04 & 3.90257e+00 \\
160 & 			      7.14426e+03 & 2.01725e-02 & 2.77638e+09 &			      2.00642e-03 & 9.18230e-04 & 5.41712e-04 &			      7.37504e+03 & 2.60822e+03 & 2.38191e+03 &			      5.43535e+00 & 6.12653e+03 &			      1.70002e-03 & 7.05946e-04 & 3.85826e+00 \\
180 & 			      5.66903e+03 & 3.55200e-02 & 3.36529e+09 &			      2.33991e-03 & 9.69102e-04 & 6.83651e-04 &			      9.30496e+03 & 3.34736e+03 & 2.97640e+03 &			      8.74395e+00 & 6.22525e+03 &			      2.11243e-03 & 8.44946e-04 & 3.78035e+00 \\
185 & 			      5.37644e+03 & 4.01659e-02 & 3.51416e+09 &			      2.41895e-03 & 9.80665e-04 & 7.17548e-04 &			      9.80654e+03 & 3.54269e+03 & 3.12927e+03 &			      9.71528e+00 & 6.24592e+03 &			      2.21967e-03 & 8.80257e-04 & 3.75825e+00 \\
190 & 			      5.10779e+03 & 4.51306e-02 & 3.66336e+09 &			      2.49541e-03 & 9.91776e-04 & 7.50384e-04 &			      1.03149e+04 & 3.74191e+03 & 3.28357e+03 &			      1.07442e+01 & 6.26529e+03 &			      2.32838e-03 & 9.15719e-04 & 3.73558e+00 \\
195 & 			      4.86054e+03 & 5.04083e-02 & 3.81277e+09 &			      2.56902e-03 & 1.00244e-03 & 7.82019e-04 &			      1.08297e+04 & 3.94485e+03 & 3.43919e+03 &			      1.18304e+01 & 6.28349e+03 &			      2.43848e-03 & 9.51302e-04 & 3.71251e+00 \\
200 & 			      4.63247e+03 & 5.59919e-02 & 3.96223e+09 &			      2.63960e-03 & 1.01267e-03 & 8.12350e-04 &			      1.13505e+04 & 4.15136e+03 & 3.59604e+03 &			      1.29735e+01 & 6.30063e+03 &			      2.54990e-03 & 9.86974e-04 & 3.68920e+00 \\
220 & 			      3.87676e+03 & 8.11890e-02 & 4.55837e+09 &			      2.88966e-03 & 1.04937e-03 & 9.19634e-04 &			      1.34874e+04 & 5.01005e+03 & 4.23388e+03 &			      1.80998e+01 & 6.36031e+03 &			      3.00734e-03 & 1.13005e-03 & 3.59566e+00 \\
240 & 			      3.30876e+03 & 1.10363e-01 & 5.14699e+09 &			      3.08900e-03 & 1.07987e-03 & 1.00459e-03 &			      1.56967e+04 & 5.91420e+03 & 4.88503e+03 &			      2.40671e+01 & 6.40897e+03 &			      3.48061e-03 & 1.27276e-03 & 3.50487e+00 \\
260 & 			      2.87089e+03 & 1.42551e-01 & 5.72314e+09 &			      3.24372e-03 & 1.10492e-03 & 1.06990e-03 &			      1.79634e+04 & 6.85568e+03 & 5.54605e+03 &			      3.07981e+01 & 6.44952e+03 &			      3.96638e-03 & 1.41411e-03 & 3.41888e+00 \\
280 & 			      2.52599e+03 & 1.76706e-01 & 6.28327e+09 &			      3.36215e-03 & 1.12530e-03 & 1.11933e-03 &			      2.02756e+04 & 7.82758e+03 & 6.21450e+03 &			      3.82043e+01 & 6.48385e+03 &			      4.46204e-03 & 1.55341e-03 & 3.33828e+00 \\
300 & 			      2.24925e+03 & 2.11805e-01 & 6.82487e+09 &			      3.45227e-03 & 1.14175e-03 & 1.15650e-03 &			      2.26238e+04 & 8.82402e+03 & 6.88860e+03 &			      4.61938e+01 & 6.51324e+03 &			      4.96554e-03 & 1.69010e-03 & 3.26300e+00 \\
350 & 			      1.75530e+03 & 2.98054e-01 & 8.08784e+09 &			      3.59424e-03 & 1.16993e-03 & 1.21399e-03 &			      2.86057e+04 & 1.13924e+04 & 8.59059e+03 &			      6.81463e+01 & 6.57048e+03 &			      6.24865e-03 & 2.01775e-03 & 3.09612e+00 \\
400 & 			      1.43431e+03 & 3.74983e-01 & 9.21271e+09 &			      3.66698e-03 & 1.18574e-03 & 1.24278e-03 &			      3.46848e+04 & 1.40297e+04 & 1.03064e+04 &			      9.20012e+01 & 6.61114e+03 &			      7.55350e-03 & 2.32135e-03 & 2.95575e+00 \\

 \hline
  \end{tabular}
  }
\end{table*}

\end{widetext}
\begin{comment}
\begin{table*}[!htb]
  \caption{Parameters entering the viscous heat equations for graphite at natural isotopic concentration ($98.9\%$). We report here only the in-plane components of the tensors needed to perform the calculations: 
 $\kappa^{ij}_{\rm P}=\kappa_{\rm P}\delta^{ij}$, $\kappa^{ij}_{\rm C}=\kappa_{\rm C}\delta^{ij}$, $K^{ij}_{S}=K_{S}\delta^{ij}$, ${D}_{U,{\rm bulk}}^{ij}={D}_{U,{\rm bulk}}\delta^{ij}$, ${F}_U^{ij}={F}_U\delta^{ij}$, $A=A^i\;\forall \;i$, $W^j_{0i}=W^j_{i0}= W\delta^{ij}$, where the indexes $i,j$ represent the in-plane directions $x,y$ only ($i,j=1,2$ and $i\neq j$).}
  \label{tab:param_graphite_nat}
  \centering
\resizebox{\textwidth}{!}{%
  \begin{tabular}{ccccccccccccccc}
  \hline
  \hline
  {T [K]} & 
  {$\kappa_{\rm P}\;\big[\rm\frac{W}{m{\cdot}K}\big]$} & {$\kappa_{\rm C}\;\big[\rm\frac{W}{m{\cdot}K}\big]$} & ${K_S\;\big[\rm\frac{W}{m^2{\cdot}K}\big]}$ & 
  $\mu^{iiii}_{\rm bulk}\;[\rm Pa{\cdot}s]$ & $\mu^{ijij}_{\rm bulk}\;[\rm Pa{\cdot}s]$ & $\mu^{ijji}_{\rm bulk}\;[\rm Pa{\cdot}s]$ &
    $M^{iiii}\;[\rm\frac{ Pa{\cdot}s}{m}]$ & $M^{ijij}\;[\rm\frac{ Pa{\cdot}s}{m}]$& $M^{ijji}\;[\rm\frac{ Pa{\cdot}s}{m}]$ &
    ${D}_{U,{\rm bulk}}\;[\rm ns^{-1}]$  &  ${F}_U\;[\rm \frac{m}{s}]$ &
    A $\big[\rm\frac{ pg}{\mu m^3}\big]$ &  $C$ $\big[\rm\frac{ pg}{\mu m{\cdot}ns^2{\cdot}K}\big]$ & W $\big[\rm \frac{\mu m}{ns}\big]$\\
  \hline  
  \hline
%T[K],            k_exactP, sma_C[W/(mK)],PrBal_k_ii[W/(m^2 K)] Exact mu:iiii,   ijij,  (ijji+iijj)/2 [Pa.S]  PrefBalmu:iiii, ijij,    (ijji+iijj)/2[Pa.S/m]   D_umk[ns^-1],Pr_Bal_Dumk_plan[m/s],A[pg/((\mu m)^3)],C[pg/((\mum)^3*ns^2*K)],W_const[\mu m/ns]]
50 & 			      4.05938e+03 & 8.28006e-05 & 2.57075e+08 &			      9.64173e-04 & 5.02674e-04 & 2.28503e-04 &			      4.39176e+02 & 1.46638e+02 & 1.46269e+02 &			      2.35602e-01 & 4.26877e+03 &			      1.53433e-04 & 1.00900e-04 & 2.72761e+00 \\
60 & 			      4.71661e+03 & 1.46319e-04 & 3.95197e+08 &			      9.71961e-04 & 5.28686e-04 & 2.19176e-04 &			      7.15578e+02 & 2.39244e+02 & 2.38166e+02 &			      3.43827e-01 & 4.59801e+03 &			      2.29498e-04 & 1.41101e-04 & 3.00647e+00 \\
70 & 			      5.30227e+03 & 2.47611e-04 & 5.59073e+08 &			      9.91550e-04 & 5.60890e-04 & 2.12711e-04 &			      1.07264e+03 & 3.59308e+02 & 3.56661e+02 &			      4.90888e-01 & 4.88642e+03 &			      3.20256e-04 & 1.84524e-04 & 3.25249e+00 \\
80 & 			      5.73508e+03 & 4.30835e-04 & 7.45726e+08 &			      1.02183e-03 & 5.95234e-04 & 2.10581e-04 &			      1.51011e+03 & 5.07190e+02 & 5.01440e+02 &			      6.99024e-01 & 5.13608e+03 &			      4.25102e-04 & 2.30997e-04 & 3.45924e+00 \\
90 & 			      5.98509e+03 & 7.89339e-04 & 9.52361e+08 &			      1.06450e-03 & 6.30070e-04 & 2.14443e-04 &			      2.02532e+03 & 6.82641e+02 & 6.71286e+02 &			      9.92275e-01 & 5.34894e+03 &			      5.43532e-04 & 2.80534e-04 & 3.62371e+00 \\
100 & 			      6.05969e+03 & 1.47107e-03 & 1.17648e+09 &			      1.12367e-03 & 6.64595e-04 & 2.26760e-04 &			      2.61392e+03 & 8.85021e+02 & 8.64331e+02 &			      1.39551e+00 & 5.52812e+03 &			      6.75030e-04 & 3.33160e-04 & 3.74625e+00 \\
120 & 			      5.80748e+03 & 4.57514e-03 & 1.66857e+09 &			      1.30805e-03 & 7.31183e-04 & 2.85750e-04 &			      3.98943e+03 & 1.36695e+03 & 1.31087e+03 &			      2.62752e+00 & 5.80204e+03 &			      9.74775e-04 & 4.47442e-04 & 3.88006e+00 \\
140 & 			      5.26117e+03 & 1.12962e-02 & 2.20653e+09 &			      1.58253e-03 & 7.93315e-04 & 3.92172e-04 &			      5.59088e+03 & 1.94459e+03 & 1.82232e+03 &			      4.56142e+00 & 5.99201e+03 &			      1.31862e-03 & 5.72512e-04 & 3.90257e+00 \\
160 & 			      4.63796e+03 & 2.27877e-02 & 2.77638e+09 &			      1.91136e-03 & 8.50246e-04 & 5.28498e-04 &			      7.37504e+03 & 2.60822e+03 & 2.38191e+03 &			      7.31266e+00 & 6.12653e+03 &			      1.70002e-03 & 7.05946e-04 & 3.85826e+00 \\
180 & 			      4.04976e+03 & 3.95430e-02 & 3.36529e+09 &			      2.24432e-03 & 9.01466e-04 & 6.69837e-04 &			      9.30496e+03 & 3.34736e+03 & 2.97640e+03 &			      1.09445e+01 & 6.22525e+03 &			      2.11243e-03 & 8.44946e-04 & 3.78035e+00 \\
185 & 			      3.91393e+03 & 4.45517e-02 & 3.51416e+09 &			      2.32345e-03 & 9.13339e-04 & 7.03592e-04 &			      9.80654e+03 & 3.54269e+03 & 3.12927e+03 &			      1.19929e+01 & 6.24592e+03 &			      2.21967e-03 & 8.80257e-04 & 3.75825e+00 \\
190 & 			      3.78317e+03 & 4.98773e-02 & 3.66336e+09 &			      2.40009e-03 & 9.24833e-04 & 7.36292e-04 &			      1.03149e+04 & 3.74191e+03 & 3.28357e+03 &			      1.30975e+01 & 6.26529e+03 &			      2.32838e-03 & 9.15719e-04 & 3.73558e+00 \\
195 & 			      3.65757e+03 & 5.55111e-02 & 3.81277e+09 &			      2.47396e-03 & 9.35948e-04 & 7.67798e-04 &			      1.08297e+04 & 3.94485e+03 & 3.43919e+03 &			      1.42580e+01 & 6.28349e+03 &			      2.43848e-03 & 9.51302e-04 & 3.71251e+00 \\
200 & 			      3.53715e+03 & 6.14431e-02 & 3.96223e+09 &			      2.54485e-03 & 9.46685e-04 & 7.98006e-04 &			      1.13505e+04 & 4.15136e+03 & 3.59604e+03 &			      1.54740e+01 & 6.30063e+03 &			      2.54990e-03 & 9.86974e-04 & 3.68920e+00 \\
220 & 			      3.10557e+03 & 8.79108e-02 & 4.55837e+09 &			      2.79677e-03 & 9.85899e-04 & 9.04866e-04 &			      1.34874e+04 & 5.01005e+03 & 4.23388e+03 &			      2.08795e+01 & 6.36031e+03 &			      3.00734e-03 & 1.13005e-03 & 3.59566e+00 \\
240 & 			      2.74682e+03 & 1.18073e-01 & 5.14699e+09 &			      2.99859e-03 & 1.01939e-03 & 9.89509e-04 &			      1.56967e+04 & 5.91420e+03 & 4.88503e+03 &			      2.71090e+01 & 6.40897e+03 &			      3.48061e-03 & 1.27276e-03 & 3.50487e+00 \\
260 & 			      2.44919e+03 & 1.50905e-01 & 5.72314e+09 &			      3.15615e-03 & 1.04763e-03 & 1.05460e-03 &			      1.79634e+04 & 6.85568e+03 & 5.54605e+03 &			      3.40892e+01 & 6.44952e+03 &			      3.96638e-03 & 1.41411e-03 & 3.41888e+00 \\
280 & 			      2.20139e+03 & 1.85356e-01 & 6.28327e+09 &			      3.27760e-03 & 1.07122e-03 & 1.10391e-03 &			      2.02756e+04 & 7.82758e+03 & 6.21450e+03 &			      4.17346e+01 & 6.48385e+03 &			      4.46204e-03 & 1.55341e-03 & 3.33828e+00 \\
300 & 			      1.99379e+03 & 2.20437e-01 & 6.82487e+09 &			      3.37078e-03 & 1.09079e-03 & 1.14103e-03 &			      2.26238e+04 & 8.82402e+03 & 6.88860e+03 &			      4.99551e+01 & 6.51324e+03 &			      4.96554e-03 & 1.69010e-03 & 3.26300e+00 \\
350 & 			      1.60318e+03 & 3.05664e-01 & 8.08784e+09 &			      3.52011e-03 & 1.12596e-03 & 1.19869e-03 &			      2.86057e+04 & 1.13924e+04 & 8.59059e+03 &			      7.24549e+01 & 6.57048e+03 &			      6.24865e-03 & 2.01775e-03 & 3.09612e+00 \\
400 & 			      1.33509e+03 & 3.80953e-01 & 9.21271e+09 &			      3.59939e-03 & 1.14751e-03 & 1.22790e-03 &			      3.46848e+04 & 1.40297e+04 & 1.03064e+04 &			      9.68155e+01 & 6.61114e+03 &			      7.55350e-03 & 2.32135e-03 & 2.95575e+00 \\
 \hline
  \end{tabular}
  }
\end{table*}
\end{comment}






\end{document}
