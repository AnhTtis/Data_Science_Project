% This is samplepaper.tex, a sample chapter demonstrating the
% LLNCS macro package for Springer Computer Science proceedings;
% Version 2.21 of 2022/01/12
%
\documentclass[runningheads]{llncs}
%
\usepackage[T1]{fontenc}
% T1 fonts will be used to generate the final print and online PDFs,
% so please use T1 fonts in your manuscript whenever possible.
% Other font encondings may result in incorrect characters.
%
\usepackage{graphicx}
\usepackage{amsfonts,amssymb}

% Used for displaying a sample figure. If possible, figure files should
% be included in EPS format.
%
% If you use the hyperref package, please uncomment the following two lines
% to display URLs in blue roman font according to Springer's eBook style:
\usepackage{color}
\usepackage{epstopdf}
\usepackage{tabularx,threeparttable}
\usepackage{array}
\usepackage[hidelinks]{hyperref}
\usepackage{graphicx}
\usepackage{subfigure}
\usepackage{amssymb, amsmath, bm}
\usepackage{mathtools}
\usepackage{mathrsfs}
\usepackage{booktabs}
\usepackage{multirow}
\usepackage{cite}
\usepackage{url}
\usepackage{color}
\usepackage{dsfont}
\usepackage[table,xcdraw]{xcolor}
%\renewcommand\UrlFont{\color{blue}\rmfamily}
%

\newcommand\T{\rule{0pt}{2.4ex}}       % Top strut
\newcommand\B{\rule[-1.2ex]{0pt}{0pt}} % Bottom strut
%\newcolumntype{?}[1]{!{\vrule width #1}}
\newcommand{\tabincell}[2]{\begin{tabular}{@{}#1@{}}#2\end{tabular}}  
\def\ie{\emph{i.e.}}
\def\eg{\emph{e.g.}}
\def\etal{{\em et al.}}
\def\etc{{\em etc.}} 



\newcommand{\TODO}[1]{{\color{red}{[TODO: #1]}}}
\newcommand{\sj}[1]{{\color{red}{[SJ:#1]}}}
\newcommand{\Angie}[1]{{\color{blue}{#1}}}

\begin{document}
%
% \title{Prognosis prediction hypergraph model for Alzheimer’s disease by Information bottleneck}
\title{HGIB: Prognosis for Alzheimer’s Disease via Hypergraph Information Bottleneck}
% \title{Leveraging Multi-Modality Data with Hypergraph Information Bottleneck for Prognosis Prediction in Alzheimer's Disease}
%
\titlerunning{HGIB}
% If the paper title is too long for the running head, you can set
% an abbreviated paper title here
%

% \author{Shujun Wang\inst{1}\orcidID{0000-1111-2222-3333} \and
% Angelica I Aviles-Rivero\inst{1}\orcidID{1111-2222-3333-4444} \and
% Zoe Kourtzi\inst{2}\orcidID{2222--3333-4444-5555}
% \and
% Carola-Bibiane Schönlieb\inst{1}\orcidID{2222--3333-4444-5555}}
\author{Shujun Wang\inst{1} \and
Angelica I Aviles-Rivero\inst{1} \and
Zoe Kourtzi\inst{2} 
\and \\
Carola-Bibiane Schönlieb\inst{1}}
%
\authorrunning{S. Wang et al.}
% First names are abbreviated in the running head.
% If there are more than two authors, 'et al.' is used.
%
\institute{DAMTP, University of Cambridge, Cambridge, UK \\
\email{\{sw991,ai323,cbs31\}@cam.ac.uk}
\and
Department of Psychology, University of Cambridge, Cambridge, UK
\email{zk240@cam.ac.uk}}


%
\maketitle              % typeset the header of the contribution
%
\begin{abstract}
%Prognosis analysis of Alzheimer's disease (AD) 
Alzheimer's disease prognosis is critical for early 
%diagnosis of
Mild Cognitive Impairment patients 
%conversion 
for timely treatment to improve the patient's quality of life. Whilst existing prognosis techniques demonstrate potential results, they are highly limited in terms of using a single modality. Most importantly, they fail in considering a key element for prognosis: not all features extracted at the 
current moment
% baseline 
%Existing deep learning methods addressing this problem are limited to single modalities and ignore an essential clue: not all the features extracted from the current moment (\textit{e.g.}, captured MRI data) 
may contribute to the prognosis prediction several years later.
%, particularly under the multi-modality data setting. 
% 
To address the current drawbacks of the literature, we propose a novel hypergraph  framework based on an information bottleneck strategy (\textbf{HGIB}). Firstly, our framework seeks to discriminate irrelevant information, and therefore, solely focus on harmonising relevant information for future MCI conversion prediction (\textit{e.g.}, two years later). 
% \sj{The first drawback actually can combine with multi modality which after combination has many irrelavant information.}
% 
Secondly, our model simultaneously accounts for multi-modal data based on imaging and non-imaging modalities.
% 
HGIB uses a hypergraph structure to represent the multi-modality data and accounts for various data modality types.
%
Thirdly, the key of our model is based on a new optimisation scheme. It is based on modelling the principle of information bottleneck into loss functions that can be integrated into our hypergraph neural network.
% 
We demonstrate, through extensive experiments on ADNI, that our proposed HGIB framework
%We conduct extensive experiments on a public dataset (ADNI) for Alzheimer's disease prognosis prediction, where HGIB
outperforms existing state-of-the-art hypergraph neural networks for Alzheimer's disease prognosis. We showcase our model even under fewer labels.
%in normal supervised learning settings and settings with fewer annotations.
Finally, we further support the robustness and generalisation capabilities of our framework under both topological and feature perturbations. 
% Our results demonstrate the promise of HGIB for future research in Alzheimer’s disease prognosis prediction.

\keywords{Prognosis prediction  \and Alzheimer's disease \and Hypergraph \and Information bottleneck \and Multi-modality data}
\end{abstract}
%
%
%

\section{Introduction}
\label{sec:introduction}

Suppose that we want to \emph{fit} and \emph{validate} a model on the basis of a single dataset.  Two example scenarios are as follows:
\begin{list}{}{}
\item{\emph{Scenario 1.}} We  want to use the data both to generate and to test a hypothesis. 
\item{\emph{Scenario 2.}} We want to use the  data both to fit a complicated model, and to obtain an accurate estimate of the expected prediction error. 
\end{list}
In either case, it is clear that a naive approach that fits and validates a model on the same data is deeply problematic. In Scenario 1, testing a hypothesis on the same data used to generate it will lead to hypothesis tests that do not control the Type 1 error, and to confidence intervals that do not attain the nominal coverage \citep{fithian2014optimal}.   And in Scenario 2, estimating the expected prediction error on the same data used to fit the model will lead to massive downward bias  \citep[see][for recent reviews]{tian2020prediction,oliveira2021unbiased}.

In the case of Scenario 1, recent interest  has focused on \emph{selective inference}, a framework that enables a data analyst to generate and test a hypothesis on the same data \citep[see, e.g.,][]{taylor2015statistical}. The main idea is as follows: to test a hypothesis generated from the data, we should condition on the event that we selected this particular hypothesis. Despite promising applications of this framework to a number of problems, such as inference after regression \citep{lee2016exact}, changepoint detection \citep{jewell2022testing,hyun2021post}, clustering \citep{gao2020selective,chen2022selective,yun2023selective}, and outlier detection \citep{chen2020valid}, it suffers from some drawbacks: 
\begin{enumerate}
\item To perform selective inference, the  procedure used to generate the null hypothesis must be fully-specified in advance.  For instance, if a researcher wishes to cluster the data and then test for a difference in means between the clusters, as in \cite{gao2020selective} and \cite{chen2022selective}, then they must fully specify the clustering procedure (e.g., hierarchical clustering with squared Euclidean distance and complete linkage, cut to obtain $K$ clusters) in advance. 
\item Finite-sample selective inference typically requires an assumption of multivariate Gaussianity, though in some cases this can be relaxed to obtain asymptotic results \citep{taylor2018post,tian2017asymptotics,tibshirani2018uniform,tian2018selective}.
\end{enumerate}
Thus, it is clear that selective inference does not provide a flexible, ``one-size-fits-all" approach to Scenario 1. 

In the case of Scenario 2, proposals to de-bias the ``in-sample" estimate of expected prediction error  tend to be specialized to simple models, and thus do not provide an all-purpose tool that is broadly applicable to complex contemporary settings \citep{oliveira2021unbiased}.

\emph{Sample splitting} \citep{cox1975note} is an intuitive  approach that is broadly applicable to a variety of settings, including Scenarios 1 and 2; see the left-hand panel of Figure~\ref{fig:samplesplit_vs_datathin}. We split a dataset containing $n$ observations into two sets, containing $n_1$ and $n_2$ observations, respectively (where $n_1+n_2=n$). Then we can generate a hypothesis based on the first set and test it on the second set (Scenario 1), or we can fit a model to the first set and estimate its error on the second set (Scenario 2). Sample splitting also forms the basis for cross-validation, an important tool for a practicing data scientist \citep{hastie2009elements}. 

While sample splitting often can 
 adequately address both Scenarios 1 and 2, it also suffers from some drawbacks: 
\begin{enumerate} 
    \item If the data contain outliers, then each outlier is assigned to a single subsample. %Again, this may not be desirable.
    \item If the observations are not independent (for instance, if they correspond to a time series) then the subsamples that result from sample splitting are not independent, and so sample splitting does not provide a solution to either Scenario 1 or Scenario 2.
    \item If one is interested in drawing conclusions at a per-observation level, then sample splitting is unsuitable.  For example, if sample splitting is applied to a dataset consisting of the 50 states of the United States, then one can only conduct inference or perform validation on those states not used in fitting.
    \item If the model of interest is fit using  unsupervised learning, then  sample splitting may  not provide an adequate solution in either Scenario 1 or 2.  The issue relates to \#3 above. This is discussed in \cite{gao2020selective,chen2022selective}, and \cite{neufeld2022inference} in the context of Scenario 1. 
\end{enumerate}

In recent work, \cite{neufeld2023data} proposed an approach for \emph{convolution-closed data thinning} that addresses these drawbacks. They consider splitting, or \emph{thinning}, a random variable $X$ drawn from a convolution-closed family into $K$ independent random variables $\Xt{1},\ldots,\Xt{K}$ such that
$X=\sum_{k=1}^K\Xt{k}$, 
and $\Xt{1},\ldots,\Xt{K}$ come from the same family of distributions as $X$ (see the right-hand panel of Figure~\ref{fig:samplesplit_vs_datathin}). 
For instance, they show that $X \sim N(\mu, \sigma^2)$ can be thinned into two independent $N(\epsilon \mu, \epsilon \sigma^2)$ and $N((1-\epsilon) \mu, (1-\epsilon) \sigma^2)$ random variables that sum to $X$. 
Finally, and most critically, if  $X$ is drawn from a Gaussian, Poisson, negative binomial, binomial, multinomial, or gamma distribution, then they can thin it  \emph{even when parameters of its distribution are unknown}. Because the thinned random variables are independent, this
provides a new approach to tackle Scenarios 1 and 2:  
one thins the dataset into two independent datasets.  One then fits a model to one dataset, and  validates it on the other. 




On the surface, it is quite remarkable that one can break up a random variable $X$ into two or more {\em independent} random variables that sum to $X$ without knowing some (or sometimes any) of the parameters.  In this paper, we seek to explain the underlying principles that make this possible.  In doing so, we show that convolution-closed data thinning can be generalized so as to make it more flexible and much more widely applicable. The convolution-closed data thinning property $X=\sum_{k=1}^K\Xt{k}$ is desirable because it ensures that no information has been lost in the thinning process. However, clearly this would be equally true if we were to replace the summation by any other deterministic function.  Likewise, the fact that $\Xt{1},\ldots,\Xt{K}$ are from the same family as $X$, while convenient, is nonessential. 

Our generalization of convolution-closed data thinning is thus a procedure for splitting $X$ into $K$ random variables such that  the following two properties hold: 
$$
\text{  (i) } X=T(\Xt{1},\ldots,\Xt{K}); \text{ and  (ii) }\Xt{1},\ldots,\Xt{K} \text{ are mutually independent}.
$$
This generalization is broad enough to simultaneously encompass both convolution-closed data thinning and sample splitting. Furthermore, it greatly increases the scope of distributions that can be thinned. In the $K=2$ case, this generalized goal has been stated before \citep[see][``P1'' property]{leiner2022data} as we will describe later.  However, we are the first to develop a widely applicable strategy for achieving this goal.   Not only are we able to thin exponential families that were not previously possible (such as the  beta family), but we can even thin outside of the exponential family.  For example, generalized thinning enables us to thin $X \sim \text{Unif}(0, \theta)$ into
 $\Xt{k} \overset{\text{iid}}{\sim} \theta \cdot\text{Beta}\left(\frac{1}{K},1\right)$, for $k=1,\dots,K$, in such a way that $X=\max\{\Xt{1},\ldots,\Xt{K}\}$.




The primary contributions of our paper are as follows:
\begin{enumerate}
\item We propose \emph{generalized data thinning}, a general strategy for thinning a single random variable $X$ into two or more independent random variables, $\xo,\ldots,\Xt{K}$, without knowledge of the parameter value(s).  
%Unlike in \cite{neufeld2023data}, we recover the original random variable $X$ using the function $T(\xo,\ldots,\Xt{k})$, where $T(\cdot)$ need not be addition, and the distributions of the $\Xt{k}$'s need not be the same as $X$.  
Importantly, we show that {\em sufficiency} is the key property underlying the choice of the function $T(\cdot)$.
\item We show that it is possible to apply generalized data thinning to distributions far outside the scope of consideration of \cite{neufeld2023data}: these include the beta, uniform, and shifted exponential distributions, among others.  A summary of distributions covered by this work is provided in Table~\ref{table:maintable}.  In light of results by \cite{darmois, koopman}, and \cite{pitman_1936} (see the end of Section~\ref{sec:method}), we believe our examples are representative of the full range of cases to which this sufficiency-based approach can be applied. 
\item We show that sample splitting --- which, on its surface, bears little resemblance to convolution-closed data thinning --- is in fact based on the same principle: both are special cases of generalized data thinning with different choices of the function $T(\cdot)$.  In other words, our proposal is a direct \emph{generalization} of sample splitting. 
\end{enumerate}



We are not the first to propose generalizations of sample splitting.  Inspired by \cite{tian2018selective}'s use of randomized responses, \cite{rasines2021splitting} define what they call the $(U,V)$-decomposition, which injects independent noise $W$ to create two independent random variables $U=u(X,W)$ and $V=v(X,W)$ that together are jointly sufficient for the unknown parameters.  However, they do not describe how to perform a $(U,V)$-decomposition other than in the special case of a Gaussian random vector with known covariance.  Our generalized thinning framework achieves the goal set out in their paper, providing a concrete recipe for finding such decompositions in a broad set of examples.  Another paper with a similar goal is \cite{leiner2022data}.  They define ``data fission'', which seeks to find random variables $f(X)$ and $g(X)$ for which the distributions of $f(X)$ and $g(X) \mid f(X)$ are known and for which $X=h(f(X),g(X))$. When these two random variables are independent (which they describe as the ``P1'' property), their proposal aligns with generalized thinning.  However, like \cite{rasines2021splitting}, they do not provide a general strategy for performing P1-fission, and the only two examples they provide are the Gaussian vector with known covariance and the Poisson.

The rest of our paper is organized as follows. In Section~\ref{sec:method}, we define generalized data thinning, present our main theorem, and provide a simple recipe for thinning that is followed throughout the paper. In Section~\ref{sec:natural-exp-fam}, we consider the case of thinning natural exponential families; this section also revisits the convolution-closed data thinning proposal of \cite{neufeld2023data}, and clarifies the class of distributions that can be thinned using that approach. In Section~\ref{sec:general-exp}, we show that we can apply data thinning to  general exponential families. We consider distributions outside of the exponential family in Section~\ref{sec:outside-exp-fam}. Section~\ref{sec:counterexamples} contains examples of distributions that \emph{cannot} be thinned using the approaches in this paper; these examples provide insight into the fact that sufficiency is the key property needed for (generalized) data thinning to ``work".  We verify our results numerically in Section~\ref{sec:experiments}.  Finally, we close with a discussion in Section~\ref{sec:discussion}; derivations and additional technical details are deferred to the appendix.  


\begin{figure}
  \hspace{39mm}  Sample splitting    \hspace{10mm} Generalized data thinning 
  \vspace{-3mm}
\begin{center} 
\centering
\includegraphics[scale=0.18,trim={3cm 8cm 39cm 8cm},clip]{figures/schematics-002.png}
\includegraphics[scale=0.18,trim={3cm 8cm 40cm 8cm},clip]{figures/schematics-v3-003.png} 
\caption{\emph{Left:} Sample splitting assigns each observation into either a training set or a test set. 
\emph{Right:} Generalized data thinning, the proposal of this paper, splits each observation into two parts that are independent and that can be used to recover the original observation $T(\xo, \Xt{2})=X$. In some cases, they are drawn from the same distributional family as $X$.  
\label{fig:samplesplit_vs_datathin}}
\end{center}
\end{figure}

\setlength{\tabcolsep}{2.5pt}
\begin{table*}[t]
\small
\centering
\caption{Experimental results of proposed method.}
\begin{tabular}{lcccccccccccccccclccccc}
\hline
                              &  &               &              &  & \multicolumn{7}{c}{Total   Power Consumption {[}mW{]}}                &  & \multicolumn{2}{c}{}          &  &                                                                           &  & \multicolumn{1}{l}{}                                                            \\ \cline{6-12}
                              &  & \multicolumn{2}{c}{Accuracy} &  &
			      \multicolumn{3}{c}{Standard HW} &  &
			      \multicolumn{3}{c}{Optimized HW} &  &
			      \multicolumn{2}{c}{\#Selected} &  &
			      \multirow{2}{*}{\begin{tabular}[c]{@{}c@{}}Max
				Delay\\ Red.\end{tabular}} &  &
				\multirow{2}{*}{\begin{tabular}[c]{@{}c@{}}Voltage
				  Scaling\\ Factor\end{tabular}} &
				  \multirow{2}{*}{\begin{tabular}[c]{@{}c@{}}
				    \\ V\_SHW\end{tabular}} &
				    \multirow{2}{*}{\begin{tabular}[c]{@{}c@{}}\\
				    V\_OHW\end{tabular}}\\ \cline{3-4} \cline{6-8} \cline{10-12} \cline{14-15}
Network-Dataset               &  & Orig.         & Prop.        &  & Orig.   & Prop.     & Red.      &  & Orig.   & Prop.     & Red.      &  & Wei.            & Act.          &  &                                                                           &  &                                                                                 \\ \hline
LeNet-5-CIFAR-10              &  & 80.6\%        & 78.5\%       &  & 375.5   & 149.6   & 60.2\%  &  & 360.7    & 78.3    & 78.3\%  &  & 35            & 210           &  & 40 ps                                                                    &  & 0.71/0.8  & 10.1\%  & 5.4\%                                                                       \\
ResNet-20-CIFAR-10            &  & 91.9\%        & 89.6\%       &  & 718.9   & 361.0   & 49.8\%  &  & 663.9    & 288.3   & 56.6\%  &  & 35            & 210           &  & 40 ps                                                                    &  & 0.71/0.8    & 13.2\%  & 11.5\%                                                                     \\
ResNet-50-CIFAR-100           &  & 79.9\%        & 78.5\%       &  & 708.7   & 293.8   & 58.5\%  &  & 701.8    & 157.1   & 77.6\%  &  & 41            & 223           &  & 30 ps                                                                    &  & 0.73/0.8  & 8.1\%  & 4.2\%                                                                      \\
EfficientNet-B0-Lite-ImageNet &  & 73.8\%        & 69.7\%       &  & 21.2    & 19.3    & 9.0\%   &  & 2.4      & 1.9     & 20.8\%  &  & 50            & 236           &  & 20 ps                                                                    &  & 0.75/0.8  & 6.4\%  & 6.5\%                                                                      \\ \hline
\end{tabular}
\label{tab:results}
\end{table*}






%\section{Related Work}

\subsection{Quantization}

Quantization is one of the most effective methods to compress neural networks and to accelerate the inference of networks~\cite{deng2020model_compress_survey,nagel2021white,yuan2022ptq4vit}.
Quantizing a network means reducing the precision of the weights and activations of the network in order to make it more computationally efficient while maintaining a similar level of prediction accuracy. 
The quantized weights and activations are typically stored as integers (such as INT8 and INT4), which reduces the amount of memory required to store them.
Lower precision calculations can be performed faster than higher precision calculations, resulting in faster inference times and reduced energy in real-time applications.
There are two main types of quantization used for neural networks: Post-Training Quantization (PTQ) and Quantization Aware Training (QAT).

Post-Training Quantization (PTQ) is a quantization technique that is applied to a pre-trained neural network~\cite{migacz2017tensorrt,banner2019ptq4bit_rapid_deploy,choukroun2019lowbit_for_efficient_inference}. 
It involves quantizing the weights and activations of the network after training is complete. 
Quantization Aware Training (QAT) is a technique in which the quantization process is integrated into the training process itself~\cite{krishnamoorthi2018quantizing_whitepaper,choi2018pact,gong2019dsq,esser2020lsq}. 
During QAT, the network is trained using a combination of full-precision and lower-precision weights and activations, which helps it to learn to be more robust to the effects of quantization.
PTQ is generally simpler to implement than QAT, as it involves quantizing a pre-trained network without any additional training. 
This can be useful in situations where re-training the network with QAT would be time-consuming or impractical~\cite{nagel2021white} (e.g., when the training dataset is not available).

Some previous work has explored the stability of quantization and the influence of calibration dataset.
\cite{hubara2021ptq_with_small_calibraiton_set} explore the problems of using small calibration dataset in quantization. 
PD-Quant~\cite{liu2022pdquant} identifies a disparity between the distribution of calibration activations and their corresponding real activations, and proposes a technique for adjusting the calibration activations accordingly.
SelectQ~\cite{zhang2022selectq} shows that randomly selecting data for calibration in PTQ can result in performance instability and degradation due to activation distribution mismatch.
Although prior research has explored the impact of the calibration dataset on the performance of quantized neural networks, these investigations have only focused on a limited range of factors and their effects on the stability of quantization. 
In this paper, we present a framework that enables us to evaluate the reliability of PTQ and conduct a comprehensive analysis of the impact of various factors on the reliability of quantization.

% While previous research may have investigated the effect of the calibration dataset on the performance of quantized neural networks, the impact of on reliability of PTQ have not been fully explored. 
% In this paper, we will develop a framework to assess the reliability and comprehensively analyze the impact of these factors from a reliability standpoint.

\subsection{Reliability of Neural Network}

The reliability of deep models has become an increasingly important topic in recent years, particularly as these models are increasingly being used in critical applications such as healthcare, autonomous vehicles, and financial systems. The reliability of deep models is often measured through different dimensions, some commonly used dimensions include: (1) model performance in various situations (including performance on the existing worst categories, noise samples, and out-of-distribution samples, etc.), (2) model robustness against test-time attacks such as adversarial attacks, and (3) the quality of the model's confidence. This paper mainly focus on the first dimension and leaves other two as the future work. Therefore, here we only give a brief introduction of related works in the first dimension.

\noindent \textbf{Model reliability in the worst case:} For the first dimension, various studies have been conducted to evaluate and improve the reliability of models against worst-case scenarios involving distribution shift and data noise, among other factors\cite{OOD_survey, label_noise_survey}. In this context, the evaluation metric typically used is the worst-case accuracy among existing test categories or out-of-distribution (OOD) samples\cite{OOD_survey, dro_survey}. It has been observed that models trained conventionally with the assumption of independent and identically distributed (IID) data fail to generalize in real-world testing environments with challenges such as hardness, noise, or OOD samples\cite{co-teaching, duchi_dro}. To enhance model reliability in such settings, two common approaches have been proposed: (1) explicit employment of robust training paradigms, such as distributionally robust optimization (DRO) and learning with noisy labels (LNL), to enhance model robustness; and (2) implicit data augmentation techniques, such as Mixup\cite{zhang2018mixup}. Further literature on this subject can be found in prior surveys\cite{OOD_survey, label_noise_survey, dro_survey}. Despite the efforts made by the community, the model reliability of PTQ methods in such cases remains largely unexplored.

%\noindent \textbf{Test-time attack:} In contrast to the first dimension, model reliability against test-time attack typically emphasize the model performance over handcraft "bad" samples such as adversarial samples \cite{}. 

%\noindent \textbf{Confidence estimation:}

%There are several factors that can impact the reliability of a neural network. One key factor is the quality and quantity of the training data used to develop the model. If the training data is biased, incomplete, or otherwise flawed, this can lead to inaccurate predictions and unreliable outcomes. Additionally, neural networks can be vulnerable to adversarial attacks, where an attacker intentionally introduces small modifications to the input data to cause the model to make incorrect predictions. In some cases, these attacks may be designed to exploit weaknesses in the neural network's architecture or decision-making processes.

%Another challenge that can impact the reliability of neural networks is data drift, which occurs when the statistical properties of the input data change over time. This can happen, for example, in healthcare applications where patient populations may evolve over time, or in financial systems where market conditions may shift rapidly. If a neural network is trained on data that is no longer representative of the current environment, this can lead to errors and inaccuracies in its predictions.

%To address these challenges, researchers have developed a range of techniques for improving the reliability of neural networks. For example, uncertainty estimation methods can be used to quantify the confidence of a neural network's predictions and identify cases where the model is uncertain or has low confidence. Adversarial training techniques can be used to improve the robustness of a neural network to adversarial attacks, while domain adaptation methods can be used to adjust a neural network's parameters to new data distributions.

%In conclusion, the reliability of neural networks is a critical issue that has important implications for their deployment in real-world applications. Understanding the factors that impact the reliability of these models, and developing effective techniques for improving their robustness and accuracy, is essential for ensuring their safe and trustworthy use in a range of critical domains.

\vspace{-2mm}
\section{Proposed Framework} \label{method}
\vspace{-2mm}

\begin{figure}[!tbp]
\centering

\includegraphics[width=12cm]{figure/teaser.pdf}
\caption{Illustration of the whole workflow. (a) shows how hypergraph information bottleneck utilised to optimize the representation $Z$ to capture the minimal sufficient information within the input data $D=(G,I)$ to predict the MCI conversion label $Y$. (b) is the overall workflow integrating HGIB into the hypergraph neural network. HGNNP is a kind of hypergraph convolutional layer.}
\label{teaser}

\end{figure}

This section presents a detailed description of the crucial components of our proposed framework. Firstly, we elucidate the process of constructing the hypergraph from a given set of multi-modal data, and we delve into the specifics of the hypergraph convolution definition. Secondly, we explicate the fundamental principle of information bottleneck and integrate it into hypergraph neural networks.
The overview of the proposed method is illustrated in Fig.~\ref{teaser}.

% Introduction why using hypergraph for mult-modality learning + HGIB



% \Angie{to be updated}



\smallskip
\subsection{Hypergraph Modelling and Hypergraph Convolution}

To overcome the
challenges associated with multi-modality data ($I_1, I_2, ..., I_m$), we leverage a hypergraph
structure $G$ to represent the multi-modal features ($X_1, X_2, ..., X_m$) extracted from backbones. Subsequently, we employ a hypergraph neural network to predict MCI conversion.

% The main purpose of the proposed framework is to optimally balance the expressiveness and robustness of the learned representation of hypergraph structure data for accurate MCI conversion prediction.
% 
%Hypergraph is general framework to incorporate with multi-modality data which is common strategy for AD diagnosis.
% 
% To encourage the minimal and sufficient representation learning, we introduce  information bottleneck by applying this principle to hypergraph neural networks.
% 
%We will first elaborate hypergraph modelling and hypergraph convolution operation and then introduce the hypergraph information bottleneck.

\medskip
\noindent
\textbf{Hypergraph representation learning.}
We consider an undirected attributed hypergraph $G = (V, E, \textbf{H})$ with a vertex set $V$, a hyperedge set $E$, and an adjacency matrix $\textbf{H} \in \mathbb{R}^{|V| \times |E|}$ for hyperedge weight. 
% 
Each vertex in our hypergraph structure corresponds to a patient, while each hyperedge represents the relationship between a subset of vertices. Unlike in a graph structure, where an edge connects only two vertices, a hyperedge in a hypergraph connects multiple vertices, enabling the representation of higher-order relationships. This feature facilitates the grouping of subsets of vertices with common features or properties, enhancing the ability of the hypergraph to model complex relationships within the data.
% 
In our hypergraph structure, each vertex corresponds to a patient and each hyperedge represents the relationship between a subset of the vertices. Unlike in a graph structure, a hyperedge in a hypergraph connects multiple vertices instead of just two, allowing for the representation of higher-order relationships. This can be seen as the hyperedges grouping together subsets of vertices that have common features or properties.
% 
Specifically, the hyperedge weight between vertex $v$ and hyperedge $e$ can be defined as 
$h_{v,e}= 
\left\{ 
    \begin{array}{lc}
        1& \text{if} \  v \in e \\
        0& \text{otherwise}
    \end{array}
\right.
$.
Moreover, we denote the vertex attributes as $X$, which can be seen as a feature embedding. The input data can be represented as $D=(G, X)$. In the multi-modal setting, we assume $m$ modalities as input and denote them as $D=(G, (X_1, X_2, ..., X_m))$.
%the input data is
%can be overall 
%denoted as $D=(G, (X_1, X_2, ..., X_M))$ for M modalities
% and the hypergraph $G$.
%can represent high-order correlations of the data.

%How to generate the hypergraph 



% Given the input data $D$ with a hypergraph and corresponding embedding, 

% A crucial step in hypergraph learning is how to contruct the hypergraph structure. To do this, 
% %To generate such hypergraph structure, 
% we first obtain the feature embeddings $X$, from the given multi-modality data, using a pre-trained network backbone, as shown in Fig.~\ref{teaser}(b).
% We then use a neighbour strategy, in feature space, to generate the hyperedge groups following the same protocol as in \cite{gao2022hgnn}. Specifically, given a vertex as the centroid, its  k-nearest neighbours in the feature space can be connected by a hyperedge:
% \begin{equation}
%    E_k=\{N_{\text{KNN}_k}(v)|v \in V\}. 
% \end{equation}
% These hyperedge groups are further concatenated together to form a hypergraph for each modality data.
% To effectively utilize the multi-modality knowledge, we concat k different hypergraphs together to generate the final hypergraph. Specifically, the incidence matrices $H_k$ are concatenated directly, as $H=H_1\|H_2\|...\|H_k$.
% Then, we can feed the data D into Hypergraph Convolution Layer for further computation.



A crucial step in hypergraph learning is the construction of the hypergraph structure. To achieve this, we first obtain the feature embeddings $X=\{X_1, X_2, ...,\\ X_m\}$ from the multi-modality data using a pre-trained network backbone, as illustrated in Fig.~\ref{teaser}(b). We then employ a neighbor strategy in feature space to generate the hyperedge groups following the same protocol as described in \cite{gao2022hgnn}. Specifically, for each vertex, its $k$-nearest neighbors in the feature space are connected by a hyperedge, resulting in the set of hyperedges $E_k=\{N_{\text{KNN}_k}(v)|v \in V\}$.
% \begin{equation}
% E_k={N_{\text{KNN}_k}(v)|v \in V}.
% \end{equation}
% 
These hyperedge groups are concatenated together to form a hypergraph for each modality data. To effectively utilize the multi-modality knowledge, we concatenate $k$ different hypergraphs to generate the final hypergraph by $H=H_1\|H_2\|...\|H_k$. Then, we feed the resulting data $D$ into a Hypergraph Convolution Layer for further computation.

% To update the vertex information, we aggregate its neighbor vertex messages along the hyperpath:

\medskip
\noindent
\textbf{Hypergraph Convolution.}
We use spatial hypergraph convolution layers \cite{gao2022hgnn} for message aggregation. Messages can be passed either from vertex to hyperedge or from hyperedge to vertex using hyperpaths $P$, which is defined as $P(v_1,v_k) = (v_1,e_1,v_2,...,e_{k-1}, v_k)$.
% We utilise the Inter-Neighbor Relation $N$ of hypergraph $G$ as $N=\{ ( v,e  ) | w_{v,e}=1, v \in V, \ and\  e \in E \}$.
% 
% The vertex inter-neighbor set of hyperedge $e$ is  defined as $N_v(e)=\{v|vNe \}$ and the hyperedge inter-neighbor set of vertex is defined as $N_e(v)=\{e|vNe\}$.
% Therefore, to update the vertex information, we need to aggregate the messages from its hyperedge inter-neighbors $N_e(v)$. And the hyperedge inter-neighbor message is updated according to their vertex inter-neighbors $N_v(e)$. Such two-step message aggregation realises a closed message passing loop among vertices.
% Then the spatial hypergraph convolution layer reads:
% \begin{equation}
% h_e=w_e \cdot \sum_{v \in N_v(e)} \frac{x_v}{|N_v(e)|}, \ \ \ \ 
% y_v=\sigma \Bigg(\sum_{e \in N_e(v)} \frac{h_e}{|N_e(v)|} \cdot \Theta \Bigg),
% \end{equation}
% where $x_v$, $h_e$, and $y_v$ are the input, hidden, and output feature vectors. $w_e$ is a weight associated to hyperedge $e$, and $\Theta$ is a trainable parameter of current hypergraph convolution layer. $\sigma$ is a non-linear activation function,\textit{ e.g.}, ReLU($\cdot$). 
% 
We define the inter-neighbor relation $N$ of hypergraph $G$ as $N=\{ (v,e) | w_{v,e}=1, v \in V, \text{ and } e \in E \}$. The vertex inter-neighbor set of hyperedge $e$ is defined as $N_v(e)=\{v|vNe\}$, and the hyperedge inter-neighbor set of vertex $v$ is defined as $N_e(v)=\{e|vNe\}$. To update the vertex information, we aggregate the messages from its hyperedge inter-neighbors $N_e(v)$, and to update the hyperedge information, we use the vertex inter-neighbors $N_v(e)$.
% 
Thus, the spatial hypergraph convolution layer is defined as
\begin{equation}
f_e= \sum_{v \in N_v(e)} h_{v,e}  \cdot \frac{x_v}{|N_v(e)|}, \ \ \ \
{f}'_v=\sigma \Bigg(\sum_{e \in N_e(v)} \frac{f_e}{|N_e(v)|} \cdot \Theta \Bigg),
\end{equation}
where $x_v$, $f_e$, and ${f}'_v$ are the input, hidden, and output feature vectors. $\Theta$ is a trainable parameter of the current hypergraph convolution layer. $\sigma$ is a non-linear activation function, such as ReLU. The two-step message aggregation realizes a closed message passing loop among vertices, which enables the model to capture higher-order relationships between the vertices in the hypergraph.


\smallskip
\subsection{Hypergraph Information Bottleneck (HGIB)}
To balance the expressiveness and robustness of the model, we aim to optimize the vertex representation to capture the minimal sufficient information required for downstream tasks via the information bottleneck approach \cite{tishby2000information}. The Hypergraph Information Bottleneck (HGIB) approach, as shown in Fig. \ref{teaser}(a), is derived from the Graph Information Bottleneck \cite{wu2020graph}, which requires the node representation $Z_v$ to minimize the information from hypergraph-structured data $D$ while maximizing the information to prediction $Y$.
% 
% At the optimisation level, a major challenge for HGIB numerical realisation is
% is that the independent and identically distributed (IID) assumption of vertices are not feasible for many real-world scenarios.
% 
% Therefore, we rely on a local-dependence assumption for hypergraph-structural data: given the data related to the limited number of neighbours of vertex $v$, the data in the rest of the hypergraph is independent of $v$.
 % 
 However, a major challenge in realizing HGIB numerically is the assumption of independent and identically distributed (IID) vertices, which is not feasible for many real-world scenarios. Therefore, we rely on a local dependence assumption for hypergraph-structured data, whereby given data related to a limited number of neighbors of vertex $v$, the data in the rest of the hypergraph is independent of $v$.
 % 
 % 
 % The optimal representation follows the Markovian dependence. The representation of each vertex is updated by incorporating its neighbours with respect to the hypergraph representation $X$.
 We assume a Markovian dependence to obtain the optimal representation, whereby the representation of each vertex is updated by incorporating its neighbors with respect to the hypergraph representation $X$.
 % 
The information bottleneck seeks to optimise
 %is reduced to the following optimization:
\begin{equation}
\underset{\mathbb{P}(Z^l|D) \in \Omega }{min} \mathcal{L}_{\text{HGIB}}(X,Y;Z^l):= [-I(Y;Z^l)+\beta I(X;Z^l)],
\end{equation}
where $\Omega$ characterises the space of conditional distribution of $Z^l$ given data $D$, and $\beta$ is a balancing weight. $l$ represents the $l$-th hypergraph convolution layer. 

We now define the mutual information $I(X;Z^{l})$,  between the initial vertex embedding and updated vertex embedding, following~\cite{nguyen2010estimating}. This yield to the Cross-Entropy loss that reads:
\vspace{-2mm}
\begin{equation}
    I(Y;Z^l) \rightarrow -\sum_{v \in V} \text{CE}(Z_v^lW_{out};Y_v),
\end{equation}
% \vspace{-2mm}
where $W_{out}$ is the weight of projector to predict the MCI conversion labels.
%
%
% \subsubsection{HGIB Estimation}\\
% \\
% \textbf{Lemma 1} (Nguyen, Wainright & Jordan's bound) For any two random variables $X_1$, $X_2$, and any brivariate function $g:g(X_1, X_2) \in \mathbb{R}$, we have
% \begin{equation*}
% I(X_1, X_2) \geqslant \mathbb{E} [g(X_1, X_2)]-\mathbb{E}_{\mathbb{P}(X_1)\mathbb{P}(X_2)} [\text{exp}\left (g\left (X_1, X_2\right )-1\right )].
% \end{equation*}
% We use this lemma for $I(Y;Z^l)$ and set $g(Y,Z^l)=1+\text{log}\frac{\prod_{v \in V} Cat(Z^lW_{out})}{\mathbb{P}(Y)}$. Then $I(Y;Z^l)$ reduces to the Cross-Entropy loss by ignoring constants, \textit{i.e.,}
% $$
% I(Y;Z^l) \rightarrow -\sum_{v \in V} \text{CE}(Z_v^lW_{out};Y_v).
% $$
%
% \if 0
% \textbf{Proposition 1. }
% % 
% % \begin{equation*}
% % \underset{\mathbb{P}(X|D) \in \Omega }{min} \text{GIB} _\beta(D,Y;X):= [-I(Y;X)+\beta I(D;X)],
% % \end{equation*}
% % 
% $I(X;Z^{l})  \leqslant I(X;Z^{l-1})$.
%
% \textbf{Proof.} Since the conditional distribution of $Z_v^l$ depends only on 
% {$X,Z^{l-1},Z^l$} forms a Markov chain in this order $X \rightarrow Z^{l-1} \rightarrow Z^l$.
% According to data-processing inequality \cite{beaudry2011intuitive}, we have $I(X;Z^{l})  \leqslant I(X;Z^{l-1})$.
% \fi
%
% $I(X;Z^{l})$ measures the mutual information between the initial vertex embedding and updated vertex embedding. It has an upper bound and can be derived to a tractable objective to optimize as follow.
%
%\textbf{Proposition 1. } For any distribution $\mathbb{Q}(Z^l)$ for $Z^l$, we have 
%$I(X;Z^{l})  \leqslant  \text{KL}(\mathbb{P}(Z^l|X)\|\mathbb{Q}(Z^l))$.
%
%To specify the upper bound for $I(X;X^{l})$, we assume $\mathbb{Q}(Z^l)$ is a non-informative prior and the elements in $X^{l}$ are IID Bernoulli distributions: $Z^l = \bigcup_{i,j} \{ z_{i,j} \in  \{ 0,1  \} |z_{i,j}\overset{IID}{\sim } \text{Bernoulli(0.5)} \}$. We assume the elements in $\mathbb{Q}(Z^l)$ have a probability of 0.5. Thus the estimation of $I(X;X^{l})$ is written as
%$$I(X;X^{l})=\frac{1}{nm} \sum_{i=1}^{n}\sum_{j=1}^{m} \text{KL}\left (\text{Bernoulli}\left (z_{ij}^l\right)\| \text{Bernoulli}\left (0.5\right )\right ).$$
We then assume that the elements in $X^{l}$ are IID Bernoulli distributions: $Z^l = \bigcup_{i,j} \{ z_{i,j} \in  \{ 0,1  \} |z_{i,j}\overset{IID}{\sim } \text{Bernoulli(0.5)} \}$. So the mutual information can be defined as $I(X;Z^{l})=\frac{1}{nm} \sum_{i=1}^{n}\sum_{j=1}^{m} \text{KL}\left (\text{Bernoulli}\left (z_{ij}^l\right)\| \text{Bernoulli}\left (0.5\right )\right )$. Here, $\text{KL}$ denotes the Kullback-Leibler divergence between two Bernoulli distributions. 

\vspace{-2mm}
\subsection{Optimisation Scheme}
\vspace{-2mm}

Our main task is MCI prediction conversion. This problem is taken from the perspective of a three class prediction task (NC, MCI, and AD). We have as basis a cross-entropy loss for classification. In the medical domain, it is usual to encounter with the class imbalance problem. To address this issue, we incorporate a focal loss. We then define our overall optimisation scheme as
% 
%In this paper, we focus on the three class prediction task. So the cross entropy loss is basically applied for classification.
%Considering the class imbalance problem for the task setting, we further incorporate focal loss.
%Therefore, the final optimization loss function is the combination of CE loss, focal loss, and HGIB loss:
\begin{equation}
    \mathcal{L}_{total} = \frac{1}{|V|} \sum_{v \in V}\Bigl\{\text{CE}(P_v;Y_v) + \mu \left [-\alpha (1-P_v)^\gamma \text{log}(P_v) \right ]\Bigl\}  + \xi \frac{1}{L}\sum_{l=1}^{L} \mathcal{L}_{\text{HGIB}},
\end{equation}
where $\mu$ and $\xi$ are balancing parameters. $\alpha$ and $\gamma$ are two hyper-parameters for the focal loss~\cite{lin2017focal} in our experiments, we set their values to 2 and 0.5 respectively.
%nd been set as 2 and 0.5, respectively.
% \vspace{-3mm}
\section{Experimental Results}
\label{Experiments}


% \begin{figure}[htbp]
% \centering
% \begin{minipage}[t]{0.42\textwidth}
% \centering
% \includegraphics[width=6cm]{figure/Plot1.pdf}
% \caption{World Map}
% \end{minipage}
% \begin{minipage}[t]{0.52\textwidth}
% \centering
% \includegraphics[width=6cm]{figure/Plot3.pdf}
% \caption{Concrete and Constructions}
% \end{minipage}
% \end{figure}





% \begin{figure}[htbp]
% \centering

% \includegraphics[width=6cm]{figure/Plot3.pdf}
% \caption{World Map}

% \end{figure}










In this section, we describe in detail the set of experiments performed to validate our proposed HGIB framework for prognosis prediction of Alzheimer's Disease.

%We aim to show the effectiveness and robustness of proposed HGIB on the task of prognosis prediction of Alzheimer's Disease.
% 
 

\subsection{Dataset Description}
We conduct  our evaluation on the  Alzheimer's Disease Neuroimaging Initiative (ADNI) dataset\footnote{Data used in the preparation of this article were obtained from the Alzheimer's Disease Neuroimaging Initiative (ADNI) database (\url{adni.loni.usc.edu}).}.
% 
ADNI is a longitudinal multi-centre and multi-modality study designed for the early detection and tracking of Alzheimer’s disease. 
The dataset contains four categories: Normal Control (NC), early Mild Cognitive Impairment (EMCI), late Mild Cognitive Impairment (LMCI), and Alzheimer’s disease (AD) at the Baseline Visit while only three categories afterwards: NC, Mild Cognitive Impairment (MCI), and AD.
% 
We focus on patients diagnosed with MCI (EMCI/LMCI) at the baseline visit.
% 
The goal is to predict the diagnosis of MCI conversion within a fixed two-year window to identify whether subjects converted to NC and AD or not, based on the multi-modality data including MRI, PET, and Non-imaging information.
% 
The Non-imaging information consists of demographic, genetic, and cognitive features, \textit{e.g.,} age, gender, education, APOE4, MMSE, ADNI-MEM \cite{crane2012development}, ADNI-EF \cite{gibbons2012composite}.
% 
After the filter, we got 248 patients with complete three modalities from ADNI-2.
% 

\subsection{Dataset Pre-processing}
For the data pre-processing, all MRI volumes are processed following (1) anterior commissure (AC)-posterior commissure (PC) alignment, (2) skull stripping, (3) intensity correction, (4) cerebellum removal, and (5) linear alignment to a template MRI.
% 
The corresponding PET volumes are aligned to their MRI volume via linear registration.
% 
For the Non-imaging data, we normalise each feature in the range of [0,1] before feeding them into the network.


% All MR images were pre-processed via four steps: (1) anterior commissure (AC)-posterior commissure (PC) alignment, (2) skull stripping, (3) intensity correction, (4) cerebellum removal, and (5) linear alignment to a template MRI. Each PET image was also aligned to its corresponding MRI via linear registration. Hence, there is spatial correspondence between MRI and PET for each subject.

\subsection{Evaluation Protocol}
We  discuss the following conditions in our experiments: (1) the classification accuracy of HGIB compared with state-of-the-art hypergraph neural network methods; (2) performance comparison under different label counts of our method vs. existing techniques;
%with less annotations compared with others; 
(3) robustness analysis under two kinds of attacks. 
We follow standard protocol in the medical domain and use the area under the ROC curve (AUC), Positive predictive value (PPV \%), and negative predictive value (NPV \%) to measure the performance.
For the results, we report the average performance and standard deviation over five independent runs. 

We implemented our framework using PyTorch \cite{NEURIPS2019_9015} and DHG\footnote{\url{https://deephypergraph.com}} library on one NVIDIA A100 GPU. Our framework was optimised with the Adam algorithm for 2,000 epochs. The learning rate was set as  1× 10$^{-4}$ and decreased to 0. We set the number of neighbours as 20 when constructing the hypergraph.
% 
We empirically set the hyper-parameters $\mu$ and $\xi$ as 1 and 10 respectively.


\subsection{Results \& Discussion}

\textbf{Comparison to Other Hypergraph Neural Networks.}
We start evaluating our framework against 
%We compare our method with other 
three state-of-the-art hypergraph neural networks.
% 
HGNN~\cite{feng2019hypergraph} is a hypergraph neural network based on spectral convolution.
DHGNN \cite{jiang2019dynamic} exploits dynamically updating hypergraph structure on each layer.
% 
While HGNN+~\cite{gao2022hgnn} is an extended version of HGNN which is a general high-order multi-modal data correlation modelling framework.
% 

We report the quantitative comparison of our technique vs. existing techniques in Table
%The comparison results with the above methods are shown in Table 
\ref{tab:results-SOTA}.
In a closer look at this table, we observe that 
%It is observed that 
our HGIB framework reports the best AUC performance on NC and MCI categories and best average AUC and NPV results. 
% 
And the AUC performance of AD generates the comparable performance (0.8492) to the top value (0.8516).
% 
The highest NPV value of 76.19\% also indicates that HGIB is more reliable to identify true negative cases and avoiding false negatives.
% 
Across all the results, our framework demonstrates that
%it is demonstrating that 
utilising the information bottleneck can further improve the network prognosis prediction ability. To further support our experimental results, we ran  set of statistical tests. Firstly, we ran a non-parametric test for multiple comparisons. In particular, we use the Friedman test along with the Kendall's coefficient of concordance with 95\% confidence intervals as measure of the effect size for the Friedman test. The results are displayed in Fig.~\ref{fig2}(a). We can then conclude that there is a significant difference in performance $\chi^2_{Friedman} (3)= 5.16$. The effect size $W_{Kendall} = 0.11$ with 95\% CI.  We then performed a pair-wise comparison using the non-parametic Wilcoxon test yielding to $V_{Wilcoxon}= 17.50, 30, 40$ for DHGNN vs. HGIB, HGNN vs. HGIB and HGNN+ vs. HGIB, respectively. 

\begin{figure}[!t]
\centering
\setlength{\abovecaptionskip}{0pt}
\setlength{\belowcaptionskip}{-2pt}
\subfigure[]{\includegraphics[width=0.45\linewidth]{figure/Plot1.pdf}}
\subfigure[]{\includegraphics[width=0.53\linewidth]{figure/Plot3.pdf}}
\caption{Statistical analysis of our technique vs. 
existing ones. (a) Displays a outcome of the non-parametric Friedman test. (b) Displays pair-wise comparison performed using the Wilcoxon test.}
\label{fig2}
\end{figure}

  \begin{table*} [!t]
    \centering
    \caption{Comparison of our method (HGIB) and existing hypergraph techniques. 
    % All comparisons are run over 5 random times on the same conditions. 
    The top results are highlighted in \textbf{bold}.}
    \label{tab:results-SOTA}
    {
        % \setlength\tabcolsep{1.5pt}
        \setlength{\tabcolsep}{0.2mm}{
        \begin{tabular}{l|c|c|c|c|c|c}
            \toprule[1pt]
            
             \multirow{2}{*}{\textbf{Methods}} & \multicolumn{3}{c|}{\textbf{AUC}} &
             \multirow{2}{*}{\tabincell{c}{\textbf{AUC}\\ \textbf{average}}} & \multirow{2}{*}{\tabincell{c}{\textbf{PPV}\\ \textbf{average}}} & \multirow{2}{*}{\tabincell{c}{\textbf{NPV}\\ \textbf{average}}} \\
             % \tabincell{c}{\textbf{CIDDG}\\ \cite{li2018deep}}\\
             \cline{2-4}  & \textbf{NC} & \textbf{MCI} & \textbf{AD} &  \T \B \\
             \hline
            \T
            \textbf{DHGNN}~\cite{jiang2019dynamic} & $0.6437\pm 0.08$ & $0.5546\pm 0.09$     &$0.6841\pm 0.11$  & $0.6275$ & $50.88$  & $69.89$  \\ 
            
            \textbf{HGNN}~\cite{feng2019hypergraph} &  $0.7114\pm 0.02$ &  $0.6407\pm 0.04$   & \cellcolor[HTML]{D7FFD7}$\textbf{0.8516}\pm 0.03$  & $0.7346$ &\cellcolor[HTML]{D7FFD7}$\textbf{63.49}$  & $74.37$   \\ 

            \textbf{HGNN+}~\cite{gao2022hgnn} &  $0.7454\pm 0.04$ &  $0.6532\pm 0.07$   & $0.8325\pm 0.07$  & $0.7437$ &$55.07$  & $74.03$   \\ 

            \hline
            \textbf{HGIB} (Ours) &  \cellcolor[HTML]{D7FFD7}$\textbf{0.7504}\pm 0.03$ &  \cellcolor[HTML]{D7FFD7}$\textbf{0.6789}\pm 0.03$   & $0.8492\pm 0.05$  & $\cellcolor[HTML]{D7FFD7}\textbf{0.7595}$ &$60.98$  & $\cellcolor[HTML]{D7FFD7}\textbf{76.19}$   \\ 

            \toprule[1pt]
        \end{tabular}
    }}
\end{table*}

%\sj{add one para for the figure anlaysis statistical analysis}

  \begin{table*} [!t]
    \centering
    \caption{Comparison of our method (HGIB) and existing hypergraph techniques under different training sample settings. 
    % All comparisons are run 5 random times on the same conditions. 
    The top results are highlighted in \textbf{bold}.}
    \label{tab:results-efficient}
    {
        % \setlength\tabcolsep{1.5pt}
        % \resizebox{\textwidth}{
        \setlength{\tabcolsep}{0.5mm}{
        \begin{tabular}{l|c|c|c|c|c|c|c|c|c}
            \toprule[1pt]
            
             \multirow{2}{*}{\textbf{Methods}} & \multicolumn{3}{c|}{\textbf{80\% }} &  \multicolumn{3}{c|}{\textbf{60\% }} & \multicolumn{3}{c}{\textbf{40\% }} \\
             \cline{2-10}& \textbf{AUC} & \textbf{PPV}& \textbf{NPV} & \textbf{AUC} & \textbf{PPV}& \textbf{NPV} & \textbf{AUC} & \textbf{PPV}& \textbf{NPV} \\
             \hline
            \T
            \textbf{DHGNN}~\cite{jiang2019dynamic} & $0.7208$ & $56.08$     &$73.71$  & $0.6981$  & $53.20$ &$71.43$  & $0.6911$  & $47.05$ &$72.75$ \\ 
            
            \textbf{HGNN}~\cite{feng2019hypergraph} & $0.7346$ & \cellcolor[HTML]{D7FFD7}$\textbf{63.49}$     &$74.37$  & $0.7300$  & $55.60$ &$74.68$  & $0.7282$  & $58.84$ &$75.51$  \\ 

            \textbf{HGNN+}~\cite{gao2022hgnn} & $0.7437$ & $55.07$     &$74.03$  & $0.7425$  & $55.51$ &$74.71$  & $0.7162$  & $55.90$ &$73.62$   \\ \hline

            \textbf{HGIB} (Ours) & \cellcolor[HTML]{D7FFD7}$\textbf{0.7595}$ & $60.98$     &\cellcolor[HTML]{D7FFD7}$\textbf{76.19}$  & \cellcolor[HTML]{D7FFD7}$\textbf{0.7443}$  & \cellcolor[HTML]{D7FFD7}$\textbf{57.93}$ &\cellcolor[HTML]{D7FFD7}$\textbf{76.63}$  & \cellcolor[HTML]{D7FFD7}$\textbf{0.7393}$  & \cellcolor[HTML]{D7FFD7}$\textbf{60.81}$ &\cellcolor[HTML]{D7FFD7}$\textbf{76.22}$  \\ 

            \toprule[1pt]
        \end{tabular}
        }}
    % }
\end{table*}

\smallskip
\textbf{Performance Comparison over Different Label Counts.}
% To show the efficiency and effectiveness of the proposed method with limited labelled data, we ran another set of experiments under three different settings with different labels counts: 80\%/60\%/40\% training patients.
% % ; (2) 60\% training patients; and (3) 40\% training patients. 
% The numerical comparison results is reported in Table~\ref{tab:results-efficient}.
% % 
% We can observe that our proposed 
% %As we can see, 
% HGIB outperforms all compared techniques.
% Besides, even trained with fewer annotated data samples, we observed a slight decrease in performance compared to when the method was trained with the full set of annotated data.
% It suggests that our method has the ability to effectively learn from limited data and generalise well to the remaining testing data.
% This can be explained that HGIB can efficient use of available data and shows model ability to identify relevant features and patterns in the data, which is useful for  prognosis prediction.
% 
To demonstrate the effectiveness of our proposed method with limited labelled data, we conducted experiments with three different label counts (80\%, 60\%, and 40\% of training patients). The results are reported in Table 2, which shows that HGIB outperforms all compared techniques. We also observed that even when trained with fewer annotated data samples, our method still achieved good performance with a slight decrease compared to the full set of annotated data. This suggests that HGIB can effectively learn from limited data and generalize well to the remaining testing data by identifying relevant information (\textit{e.g.}, features and patterns), thus improving prognosis prediction.

\smallskip
\textbf{Robustness Analysis under Attacks.}
To study the robustness of the proposed HGIB under the hypergraph topology and feature tasks, we conduct experiments by (1) randomly drop 20\% hyperedges from the original hypergraph for structure attack; (2) randomly inject feature noise to the extracted feature embedding $X$ for feature attack. 
% 
Specifically, we define the mean of the maximum value of each vertex feature as $r$, random Gaussian noise as $\epsilon \sim  N(0,1)$, and feature noise ratio as $\rho$.
We then calculate noise according to $\eta=\rho \cdot r \cdot  \epsilon$. 
We set $\rho$ as 0.01 in the experiments.
The results are shown in Table~\ref{tab:results-robust}.
% 
% Our method HGIB consistently outperforms the previous state-of-the-art HGNN+ \cite{gao2022hgnn} when dropping hyperedges or adding feature noise.
% This shows that HGIB makes the model more robust for either structure or feature perturbations, and further supporting our proposed framework in terms of generalisation and robustness.
% 
The results, shown in Table 3, demonstrate that HGIB consistently outperforms the previous state-of-the-art HGNN+~\cite{gao2022hgnn} in both scenarios, indicating that HGIB enhances the model's robustness for structure and feature perturbations, supporting its effectiveness for generalization and robustness.






  \begin{table*} [!t]
    \centering
    \caption{Robustness Analysis.
    % We use two type of attacks topological and feature. 'Drop' means random drop 20\% hyperedges and Noise means noise injection  to the feature embedding. 
    The top results are highlighted in \textbf{bold} font.}
    \label{tab:results-robust}
    {
        % \setlength\tabcolsep{1.5pt}
        \setlength{\tabcolsep}{0.2mm}{
        \begin{tabular}{l||c|c||c|c||c|c}
            \toprule[1pt]
            \textbf{Methods} & \textbf{Attack} & \textbf{AUC} & \textbf{Attack} & \textbf{AUC} & \textbf{Attack} & \textbf{AUC}
              \T \B \\
             \hline
            \T
            \textbf{HGNN+}~\cite{gao2022hgnn} & - & $0.7437\pm0.05$ & Drop & $0.7155\pm0.02$ & Noise & $0.7061\pm0.04$   \\

             \textbf{HGIB} (Ours) &-  & $\textbf{0.7595}\pm 0.02$ & Drop &$\textbf{0.7452}\pm 0.02$  & Noise &$\textbf{0.7326}\pm 0.01$   \\ 
            \hline
            \toprule[1pt]
        \end{tabular}
    }}
\end{table*}
\vspace{-0.2cm}
\section{Conclusion}
\label{sec:conclusion}
We introduced \OURS{} - an intrinsically rotation-invariant model for point cloud matching. We proposed PAM~(PPF Attention Mechanism) that embeds PPF-based local coordinates to encode rotation-invariant geometry. This design lies at the core of AAL~(Attention Abstraction Layer), PAL~(PPF Attention Layer), and TUL~(Transition Up Layer) which are consecutively stacked to compose PPFTrans~(PPF Transformer) for representative and pose-agnostic geometry description. We further enhanced features by introducing a novel global transformer architecture, which ensures the rotation-invariant cross-frame spatial awareness.
%The global context is then aggregated for feature enhancement via the global transformer structure with the rotation-invariant cross-frame spatial awareness. 
Extensive experiments are conducted on both rigid and non-rigid benchmarks to demonstrate the superiority of our approach, especially the remarkable robustness against arbitrary rotations. %However, as \OURS{} does not explicitly handle the occlusion, it may fail in cases with extremely limited overlap. 
Limitations are discussed in the Appendix.

\noindent\textbf{Acknowledgment.} This paper is supported by the National Natural Science Foundation of China under Grant No. 62025208. We appreciate the help from Lennart Bastian, Mert Karaoglu, Ning Liu, and Zhiying Leng.



% \section{First Section}



% \noindent Displayed equations are centered and set on a separate
% line.
% \begin{equation}
% x + y = z
% \end{equation}
% Please try to avoid rasterized images for line-art diagrams and
% schemas. Whenever possible, use vector graphics instead (see
% Fig.~\ref{fig1}).

% \begin{figure}
% \includegraphics[width=\textwidth]{fig1.eps}
% \caption{A figure caption is always placed below the illustration.
% Please note that short captions are centered, while long ones are
% justified by the macro package automatically.} \label{fig1}
% \end{figure}

% \begin{theorem}
% This is a sample theorem. The run-in heading is set in bold, while
% the following text appears in italics. Definitions, lemmas,
% propositions, and corollaries are styled the same way.
% \end{theorem}
% %
% % the environments 'definition', 'lemma', 'proposition', 'corollary',
% % 'remark', and 'example' are defined in the LLNCS documentclass as well.
% %
% \begin{proof}
% Proofs, examples, and remarks have the initial word in italics,
% while the following text appears in normal font.
% \end{proof}
% For citations of references, we prefer the use of square brackets
% and consecutive numbers. Citations using labels or the author/year
% convention are also acceptable. The following bibliography provides
% a sample reference list with entries for journal
% articles~\cite{ref_article1}, an LNCS chapter~\cite{ref_lncs1}, a
% book~\cite{ref_book1}, proceedings without editors~\cite{ref_proc1},
% and a homepage~\cite{ref_url1}. Multiple citations are grouped
% \cite{ref_article1,ref_lncs1,ref_book1},
% \cite{ref_article1,ref_book1,ref_proc1,ref_url1}.

% \subsubsection{Acknowledgements} Please place your acknowledgments at
% the end of the paper, preceded by an unnumbered run-in heading (i.e.
% 3rd-level heading).

%
% ---- Bibliography ----
%
% BibTeX users should specify bibliography style 'splncs04'.
% References will then be sorted and formatted in the correct style.
%
\bibliographystyle{splncs04}
\bibliography{ref}
%

\end{document}
