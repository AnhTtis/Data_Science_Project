\section{Introduction} \label{introduction}

% Alzheimer’s disease (AD) is an irreversible neurodegenerative disorder, whose symptoms gradually worsen yielding to a high level of dependence. AD progresses through different stages starting from Mild Cognitive Impairment (MCI), where memory problems are highlighted, to AD. However, it is well-established that not all patients diagnosed as MCI  convert to Alzheimer’s disease. That is, patients can be categorised into MCI stables meaning that a patient retains the same diagnosis for years, and MCI progressors reflecting a change in diagnosis over the years from MCI to AD. In some cases, some MCI patients remain stable or convert back to normal (MCI non-converters).  Whilst there is no cure for AD, the development of algorithmic techniques for prognosis are of great interest in the community; as early treatment for MCI progressors is crucial for improving patients’ quality of life, which indicates the value of prognosis prediction. 

Alzheimer's disease (AD) is a progressive and irreversible neurodegenerative disorder that results in a high level of dependence. The disease progresses through different stages, starting with Mild Cognitive Impairment (MCI), characterized by memory problems, and eventually leading to AD. However, not all patients diagnosed with MCI convert to Alzheimer's disease, and some may remain stable or even revert to normal. This distinction between MCI stables, MCI progressors, and MCI non-converters highlights the need for accurate prognosis prediction.
Although there is no known cure for AD, the development of algorithmic techniques for prognosis has attracted great interest from the medical community. Early treatment for MCI progress is crucial for improving patients' quality of life, making the value of accurate prognosis prediction evident. 

% The task to predict MCI to AD conversion has been investigated in the body of literature from different perspectives. Early works have addressed this problem solely based on analysis of a region of interest to predict MCI conversion e.g.~\cite{mcevoy2009alzheimer,costafreda2011automated, davatzikos2011prediction},  different single modalities or neuropsychological evaluation~\cite{risacher2009baseline, qiu2009regional, fan2008structural}, and informative AD biomarkers~\cite{shaw2009cerebrospinal,hansson2006association}. 
% This type of data has been explored in the community, in which the most widely used technique is support vector machine (SVM) alone or in combination with other methods including Gaussian Radial functions~\cite{liu2022assessing,toussaint2012resting}.  With the advent of deep learning, a body of researchers has explored convolutional neural networks (CNNs) for MCI conversion prediction. These works used different strategies and models including transfer learning, pyramid networks~\cite{bae2021transfer, pan2020multi, shen2021heterogeneous,  syaifullah2021machine, zhu2021long}.

The literature has extensively investigated the task of predicting MCI to AD conversion from various perspectives. Early studies focused on analyzing a region of interest~\cite{mcevoy2009alzheimer,costafreda2011automated, davatzikos2011prediction}, different single modalities or neuropsychological evaluation~\cite{risacher2009baseline, qiu2009regional, fan2008structural}, and informative AD biomarkers~\cite{shaw2009cerebrospinal,hansson2006association}.
These have been widely explored using support vector machines (SVM) alone or in combination with Gaussian Radial functions~\cite{liu2022assessing,toussaint2012resting}. 
% With the advent of deep learning, researchers have explored the use of convolutional neural networks (CNNs) for MCI conversion prediction, employing various strategies and models, including transfer learning and pyramid networks~\cite{bae2021transfer, pan2020multi, shen2021heterogeneous,  syaifullah2021machine, zhu2021long}.
With the advent of deep learning, researchers have explored the use of convolutional neural networks (CNNs) for MCI conversion prediction employing transfer learning and pyramid networks~\cite{bae2021transfer, pan2020multi, shen2021heterogeneous,  syaifullah2021machine, zhu2021long}.


% Despite the promising results reported in the literature, there are three drawbacks that limit the performance of MCI conversion prediction. Firstly, the majority of existing techniques use a single modality, which mainly has been focused on using either MRI, PET, or non-imaging data (genetics, magnetoencephalography). However, early and recent studies have recognised the need to develop multi-modal techniques to extract richer information~\cite{shigemizu2020prognosis,aviles2022multi,grueso2021machine}. Multi-modal prognosis is a very challenging task due to the high homogeneity between modalities. Secondly, a major challenge is how to build stronger relations between modalities. A perfect fit for this problem is hypergraph learning, which seeks to exploit higher-order relations between the data. In this context, several hypergraph techniques have been proposed in the literature~\cite{jiang2019dynamic, feng2019hypergraph,gao2022hgnn}. Nonetheless, existing techniques have been designed to consider all information fed into the constructed hypergraph compromising the robustness and expressiveness of the model. The last limitation is how to sufficiently extract the discriminative information from the multi-modality data to facilitate the conversion prediction and discard the irrelevant information. No papers have explored this area yet for MCI conversion.

Although there have been promising results in MCI conversion prediction in the literature, there are two major drawbacks that limit their performance. 
% Firstly, existing techniques mainly rely on a single modality, such as MRI, PET, or non-imaging (genetics, magnetoencephalography) data, which may not capture the full picture of the disease progression. 
Firstly, existing techniques mainly rely on a single modality, which may not capture the full picture of the disease progression. 
% 
To address this, recent studies have emphasized the need to develop multi-modal techniques that can extract richer information~\cite{shigemizu2020prognosis,aviles2022multi,grueso2021machine}.
% 
Secondly, how to effectively extract discriminative information from the multi-modal data to facilitate conversion prediction and discard the irrelevant information from high heterogeneity multi-modality data has not been explored in the existing literature yet.
% Secondly, building stronger relations between modalities is a major challenge in multi-modal prognosis, and hypergraph learning can potentially address this by exploiting higher-order relations between the data. 
% While several hypergraph techniques have been proposed~\cite{jiang2019dynamic, feng2019hypergraph,gao2022hgnn}, they often consider all information fed into the constructed hypergraph, which may compromise the robustness and expressiveness of the model.
% Lastly, how to effectively extract discriminative information from the multi-modal data to facilitate conversion prediction and discard the irrelevant information from high heterogeneity multi-modality data has not been explored in the literature yet.

% Multi-modal prognosis is a very challenging task due to the high homogeneity between modalities.


% To address the existing problems in the literature, in this work we introduce a novel hypergraph information bottleneck strategy (HGIB) for Alzheimer's disease prognosis leveraging multi-modality data. 
In this work, we propose a novel approach, namely the hypergraph information bottleneck strategy (HGIB), to address the existing limitations in Alzheimer's disease prognosis leveraging multi-modality data.
% 
% To overcome the challenges associated with multi-modality data, 
HGIB utilises a hypergraph structure to represent the multi-modal features and harmonises various modality data types.
% 
In contrast to existing techniques, HGIB focuses on learning discriminative representations that are meaningful with minimal but relevant information, by applying the principle of information bottleneck.
% 
\textit{To the best of our knowledge, this is the first hypergraph framework specifically designed for prognosis rather than diagnosis.}
% 
% Extensive experiments conducted on a public dataset (ADNI) for Alzheimer's disease prognosis prediction demonstrate the superiority of HGIB over other state-of-the-art hypergraph neural network architectures in both normal supervised learning settings and settings with fewer annotations. Furthermore, our results reveal the robustness of HGIB under structural and feature perturbations. These findings indicate the potential of HGIB for future research in Alzheimer's disease prognosis prediction.
Experiments on the public ADNI dataset for Alzheimer's disease prognosis prediction demonstrate that HGIB outperforms other state-of-the-art hypergraph neural network architectures in supervised learning settings as well as with fewer annotations. Additionally, HGIB exhibits robustness against topological and feature perturbations, highlighting its potential for future research in this field.

% \textbf{Contributions.} 
% \Angie{to be updated at the end}
% Our contributions are three folds.
% \begin{itemize}
%     \item We propose the hypergraph information bottleneck framework designed specifically for prognosis rather than a diagnosis to facilitate relevant knowledge discovering among multi-modal data.
%     \item 
% \end{itemize}
