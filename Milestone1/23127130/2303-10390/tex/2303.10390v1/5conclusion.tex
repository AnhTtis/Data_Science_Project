\section{Conclusion}
\label{conclusion}


Our proposed HGIB demonstrated promising results in addressing the challenges associated with Alzheimer's disease prognosis leveraging multi-modal data. By utilising a hypergraph structure to represent the multi-modal features and enforcing learning discriminative representations with the help of the principle of information bottleneck, HGIB outperforms other state-of-the-art hypergraph neural network architectures in supervised learning settings and under  fewer annotations. Our results also demonstrated better robustness of HGIB under both topological and feature perturbations. The success of HGIB highlights the potential of hypergraph-based methods in the field of Alzheimer's disease prognosis prediction and sets a foundation for future research in this area.





\section{Acknowledgement}
This work was supported by grants to: 
SW from Wellcome Trust 221633/Z/20/Z;
AIAR from CMIH and CCIMI, University of Cambridge;
ZK acknowledges support from the Biotechnology and Biological Sciences Research Council H012508 and BB/P021255/1, Alan Turing Institute TU/B/000095, Wellcome Trust \\205067/Z/16/Z, 221633/Z/20/Z, Royal Society INF/R2/202107;
CBS from the Philip Leverhulme Prize, the Royal Society Wolfson Fellowship, the EPSRC advanced career fellowship EP/V029428/1, EPSRC grants EP/S026045/1 and EP/T003553/1, EP/N014588/1, EP/T017961/1,
the Wellcome Innovator Awards 215733/Z/19/Z and 221633/Z/20/Z, the European Union Horizon 2020 research and innovation programme under the Marie Skodowska-Curie grant agreement No. 777826 NoMADS,
the Cantab Capital Institute for the Mathematics of Information and the Alan Turing Institute.


% \clearpage

