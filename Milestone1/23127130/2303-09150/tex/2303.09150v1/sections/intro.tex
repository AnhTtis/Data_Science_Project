\urlstyle{tt}
\section{Introduction}\label{sec:intro}
Passwords are by far the most popular method for authentication, despite their several limitations: %
users have to remember many (unique) passwords and, in turn, they tend to choose weak, easy-to-guess passwords. 
\changed{To improve security of user passwords, several alternative tools and strategies are proposed, such as password managers and two-factor authentication mechanisms.  %
However, to secure password managers (and even while two-factor authentication is used), users are required to create and remember a secure and memorable master secret.
To that end, \emph{passphrases} are considered an alternative approach to generate memorable yet strong authentication secrets. }

\changed{Passphrases, unlike passwords, are sequences of words, for example ``correct horse battery staple''. %
They are frequently recommended for particularly sensitive scenarios---as the master secret for password managers~\cite{huth2012password}, for locking users' Cryptocurrency wallets (e.g., brainwallet~\cite{eskandari2018first}), or protecting SSH keys~\cite{ssh-keygen}.
Passphrases could also be used to create more memorable passwords. For example, mnemonics created from
passphrases can assist in memorizing complex passwords~\cite{woo2016improving}.
Thus Passphrases are a secure and user-friendly alternative to passwords~\cite{fbi-2020-passphrase}, while we (slowly) transition toward a ``password-less'' future of authentication.}

Passphrases can be user-generated 
or system-generated. User-generated passphrases are memorable, often because of their sentence/grammatical structure being
aligned with natural text~\cite{gram1, gram4} %
To that end, system-generated passphrases are proposed---such passphrases are difficult for an attacker to guess~\cite{bonneau2012quest}, but are also difficult to remember~\cite{jagadeesh2021alice, shay2012correct}.





Existing approaches to computer-generated passphrases include (a) \emph{Diceware}~\cite{reinholddiceware}, which picks random words from a given wordlist, and (b)  \textit{Template-based Diceware}~\cite{mmapCode}, which generates passphrases that adhere to a small set of pre-selected syntax template (e.g., user-adverb-verb-noun).  Shay et al. \cite{shay2012correct} showed that Diceware passphrases are 
as difficult to remember as randomly generated passwords. 
Further, our analysis reveals (\secref{ssec:systempp}) that Template-based Diceware also suffers two major drawbacks: (1) there is a fixed maximum bound on the strength of the generated passphrases, because it uses a handful of specific syntax templates, and (2) there is no systematic way to extend the space of passphrases with more templates (\secref{ssec:langmodel}), necessitating a thorough re-design of the approach. Another prior work took an alternative approach---they tried to use techniques like implicit learning \textit{post-passphrase generation} to improve memorability~\cite{jagadeesh2021alice}.

Complementary to those efforts of enhancing memorability by user-learning, we ask: \textit{Is it possible to develop a simple automated approach for producing system-generated passphrases of arbitrary length, which is memorable by abiding grammar/sentence structure, yet hard for an adversary to guess and address shortcomings of existing passphrase-generation systems?}


In this work, we answer this question affirmatively and present \system{}, which can automatically generate memorable passphrases with good security. \changed{We note that the current passphrase generation methods like Diceware implicitly trade off memorability for security. In this work, we explore this trade-off and design, build, and evaluate \system{} which attempts to provide a balanced trade-off between security and memorability.} We design \system{} using insights from a novel in-use English passphrase dataset to ensure good guessability-memorability trade-offs. 
The key contributions of this work are:
\begin{newitemize}
    \item We used heuristics to identify $\nppws$ user-chosen English passphrases %
    from prior password leaks. To the best of our knowledge, our dataset is the largest user-chosen passphrase dataset to date.\footnote{The dataset and code for this work can be found at \color{blue}{\tt\url{https://github.com/Mainack/MASCARA-passphrase-code-data}}.} Our algorithm leverages word segmentation in the noisy text to identify these passphrases. The syntactic structures of these passphrases are distinctly different from \dice---favoring memorability over guessability. %
    
    \item We created a memorability-guessability measurement framework for passphrases. Building on prior works on the memorability of natural language phrases we identified distinct and important features of a sentence %
    which affects the memorability of a passphrase. We additionally created a Monte-Carlo estimate of the passphrase guess ranks to measure the guessability of passphrases. %
    We utilize this framework to balance memorability and guessability during passphrase generation.

    \item Using our framework, we present \system{}, a novel system to automatically generate memorable yet not so easily guessable passphrases. \system{} leverages a constrained generative process by modifying a generative Markov model and explicitly considering the dimensions of memorability and guessability during passphrase generation. Our evaluation demonstrated that \system{} generated passphrases have improved security than user-generated ones and do not suffer from any of the drawbacks of the existing systems. Moreover, the user study we carried out shows that users have the highest recall rate after two days for passphrases in their preferred range with \system{}. %
    For passphrases of length 7 or less (preferred by most users) \system{} provides 1.6x--2x better recall rate than deployed systems like \dice while maintaining a less than 10\% character error rate. 
\end{newitemize}

\noindent \textbf{Limitations:} Our study has three key limitations. First, we are not providing a one-step solution to the quest of memorable passphrases. Rather \system{} takes a principled and complementary approach to enhance today's system-generated passphrases by balancing memorability and security. %
\changed{Second, as with any user study, ecological validity is hard to ensure objectively. In line with earlier work, we made a conscious and earnest effort in our experiments to not nudge our participants to choose or better remember passphrases from any specific system (to preserve the sanctity of recall values)~\cite{egelman2013ecologicalpass}. Participants did not know passphrases shown to them came from which system---our quantitative and qualitative analysis also does not indicate any bias. %
In fact, if the users believed that our study required to remember their chosen passphrases correctly and thus always choose the one which was easiest to remember and go out of their way to memorize (e.g., using pen and paper), we might not have seen the wide variance in recall values (see~\figref{fig:recall_cer_less7}).}
Finally, in line with prior work, we only consider English passphrases---it enables comparison with state-of-the-art~\cite{jagadeesh2021alice,
bonneau_linguistic_2012, bonneau_2018}. \changed{Exploring passphrases for other languages is part of our future work.}










