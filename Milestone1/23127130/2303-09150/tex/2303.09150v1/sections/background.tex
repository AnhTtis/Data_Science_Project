\section{Background and related work}
\label{sec:relwork}





\subsection{System-generated Passphrases}

\noindent Passwords are used for authenticating to critical as well as non-critical infrastructures~\cite{bonneau2012quest}. Unfortunately, several prior works~\cite{das2014tangled} 
have shown  that users tend to choose predictable passwords and reuse them across multiple accounts. 
To that end, in the past decade, passphrases have been put forward as an alternative or complementary mechanism to passwords. NIST defined a passphrase to be a \textit{memorized secret consisting of a sequence of words or other text that a claimant uses to authenticate their identity}~\cite{grassi2017draft}. Intuitively, passphrases are likely to be easier to remember than passwords (due to their closeness to natural language) for users as well as harder to guess (due to their length) for adversaries. Shay et al.~\cite{shay2012correct} demonstrated (using a user study) that even simply using passphrases consisting of three to four words can be comparable in terms of entropy with passwords generated using more-involved methods while also accounting for memorability, highlighting the utility of good passphrases. 
Today passphrases are used in password managers, cryptocurrency wallets~\cite{8285737}, and for securing ssh keys~\cite{ssh-keygen}. 

\changed{Several prior works focused on generating and remembering secure passwords (and passphrases) using techniques like contextual cues, portmanteau, or mnemonic based generation~\cite{stobert2014password, woo2020we,kuo2006human, al2015towards,joudaki2018reinforcing}. These techniques aim to associate a context (like an image) with the secret. Consequently, our work
on generating secure and memorable passphrases is complementary to such techniques of remembering secrets---they can be directly used to improve memorability of \system-generated passphrases.}

Prior works however have not systematically investigated the guessability-memorability trade-off. %
 In this work, we formalize the guessability-memorability problem  by creating a data-driven framework and leverage the framework to design  \system{} that can generate secure and memorable passphrases.





\subsection{Security and threat model}
\label{ssec:secandthreat}

\noindent Measuring security of passwords is a well-studied topic~\cite{klein1990foiling}. Earlier works considered different approaches the attacker could employ and thus used Markov models, probabilistic CFG, neural networks, etc. for the guessability estimations~\cite{wang2016targeted, pal2019beyond,
van_acker_password_2015, melicher-meter, ur-measure}. However, there is relatively less work in the domain of passphrases. 

\vspace{4mm}

\noindent \textbf{Using guess rank to measure passphrase security} Previous works found that the security of passphrases can be increased by increasing the entropy via either introducing semantic noises or increasing the wordlist size~\cite{lee2007passphrase}. However, if users are given control, they generally tend to opt for common phrases, reducing security drastically~\cite{kuo2006human}. In all of these works, the security of passphrases is generally measured through either entropy or user surveys~\cite{renyi1961measures}. 

But later, researchers have shown that \textit{guess rank} is a much more acceptable measure than entropy for measuring the security of a password~\cite{realWorldAccuracies, testingMetrics}. The guess rank of a password can be understood as the number of guesses an adversary needs to arrive at the correct password. So higher the guess rank, lower the guessability. Building upon this and earlier works on password meters~\cite{van_acker_password_2015, melicher-meter, ur-measure}, we use the guess rank metric for measuring the guessability of passphrases in our setup. Specifically, we estimate a guess rank for each passphrase in our setup--- the higher the guess rank, the higher is its resistance to external attacks. 

\vspace{4mm}

\noindent \textbf{Our adversary model with \textit{min auto} approach.} \changed{In this work, we consider a powerful offline generalized untargeted adversary.  We assume the attacker is fully aware of the passphrase generation algorithm and the dataset used in all training. The attacker can generate (offline) as many passphrases as they can (given their computational resources) for any algorithm.
The goal of the attacker is to guess the password randomly generated using an algorithm given a large guessing budget (e.g., $10^{15}$). Our offline attack~\cite{bonneau2012quest} model is stronger than an online attack setting where the attacker is limited by the number of guesses they can make. }
The primary challenge for the attacker is to generate an ordered list of passphrase guesses $\pw_{1}, \pw_{2}, ..., \pw_{n}$ to reach the target passphrase $\pw$ as early as possible (least number of guesses). The higher the guess rank of a passphrase the stronger the passphrase is (lower guessability). Given there are multiple algorithms an attacker can use to guess a passphrase we take a \textit{min auto} approach described by Ur et al.~\cite{realWorldAccuracies}. First, we used multiple password cracking algorithms ($n$-gram word and character models trained over a large corpus of passphrases generated by the system under consideration),  parameterized by a set of training data~\cite{Kelley2012, Bonneau2012}. For each algorithm, the \emph{guess rank} will be the number of incorrect guesses the particular algorithm used to arrive at the correct passphrase. Since the average guess rank for even a passphrase of medium strength is of the order of $10^{15}$ (\secref{ssec:guesseval}), running the algorithms to find the guess rank is infeasible---we therefore used Monte Carlo simulations to estimate the guess rank (\secref{ssec:guessability_old}). Finally, following the previous work by Ur et al. we simply took the minimum of all guess ranks to arrive at our estimated guess rank. Ur et al. demonstrated that taking minimum guess rank of all automated approaches (called \textit{min auto}) is a reasonable approximation of real-world cracking scenario~\cite{realWorldAccuracies}.







\subsection{Measuring memorability of passphrases}
\label{ssec:memorablemeasure}

\noindent Earlier works tried to correlate password memorability with the frequency of passwords, login durations, and even keyboard pattern~\cite{gao2018forgetting, guo2019optiwords}. Other works used methods like encoding random n-bit strings for generating memorable passwords or using chunks~\cite{ghazvininejad2015memorize}. %
All of these cases measured memorability of passwords based on user surveys instead of an automated linguistic metric~\cite{chatterjee2017typtop, woo2020we, shay2012correct}. Although useful, these works considering memorability of passwords are complementary from that of passphrases. Passphrases often contain possible linguistic properties (the order in which words are presented) which are generally absent in passwords. To that end, there is some work on the memorability of English phrases. For example, work by Danescu et. al.~\cite{danescu2012you} has tried to measure memorability of popular movie quotes using lexical distinctiveness. Some Human-Computer Interaction studies identified Character Error Rate (CER) as an acceptable measure for memorability of passphrases~\cite{mackenzie2002character, kristensson2012performance}. We build on this research, identify underlying factors affecting memorability of phrases, and consequently optimize to improve memorability of  \system{}-generated passphrases (\secref{ssec:memorability} and \secref{ssec:tuning}). Our two-part user study results demonstrated that users indeed memorize \system{} generated passphrases which leverages this model. With this background, in the next section, we start with first identifying and analyzing the state of the art methods for generating passphrases in the wild. %



































