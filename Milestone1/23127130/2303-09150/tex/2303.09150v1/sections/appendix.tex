\section{Detecting user-generated passphrases}\label{sec:inusepassphrase}

\subsection{User passphrases}

To examine the class of user passphrases, we need to have a dataset compromising user passphrases. Unfortunately, unlike passwords, where data breaches are not uncommon and a lot of which have surfaced publicly~\cite{fouriq}, there is no public dataset of passphrases available. But we notice that several password leak databases contain long passwords that could potentially be passphrases without a proper delimiter. For this, we devise a segmentation algorithm to identify passphrases from password leak datasets and construct the first user-chosen, in-use passphrase database.  We
describe in this section how we extracted the passphrases and released the code/dataset from this paper for further research on passphrases in 
\begin{center}
\small \color{blue}{\tt \url{https://github.com/Mainack/MASCARA-passphrase-code-data}}
\end{center}



\paragraph{Password leak dataset.} We use a compilation of prior data breaches that surfaced in 2018 by 4iQ security firm~\cite{fouriq}. The leaked dataset contains nearly 1.4~billion email-password pairs. Prior research on passwords has used this leak~\cite{pal2019beyond,li2019protocols}. We use this dataset to extract potential in-use passphrases.

To find passphrases in this dataset, we only consider passwords that are longer than 20 characters, or roughly 4 words (given that the average length of an English word is 4.8~characters~\cite{Norvig}). We found 5.7~million unique passwords that are longer than 20 characters. These passwords were used by 5.9~million users (identified by email addresses in the dataset) \footnote{We combined all the passwords (of any length) belonging to the same email. We further combined the emails (and the corresponding passwords) if the two emails share the same username (the part before the `@' symbol) and if they have a common password. We ignored the users with more than 1,000 passwords, as they are unlikely to be real user accounts. Such preprocessing was also done in~\cite{pal2019beyond}.} 6.2~million times in total.



\paragraph{Segmenting passwords.}\label{sec:segmenting-pws}
As these passwords are selected from a compilation of password leaks, many of them are potentially hash values ~\cite{pal2019beyond}. We remove these hash values using a heuristic-based identification algorithm. We check if the password only contains hexadecimal characters and if so, we flag them as hash values and remove them. This removed 1.9~million unique passwords. We also find many of the 20-or-more character passwords look like email addresses or have some parsing errors. We removed such 1.5~million passwords that contain an `@' symbol with a prefix and has `.' in its suffix. We
also removed 0.4~million passwords that had less than nine English letters in them, as that is the minimum number of characters necessary to form a three-word passphrase, with each word at least three characters long (See details below). This left us with 1.8~million passwords that we then test using our passphrase segmentation algorithm.
We used a standard NLP task of doing word segmentation of noisy text for segmenting passwords and creating passphrases. Initially we tried segmenting using SymSpell library~\cite{garbe2019symspell} which is parameterized by a unigram distribution $V$ (created using popular word lists~\cite{NLPch02, geo, moby}). Given a password $\pp$, it can segment to create a sequence of words $(\w_1,\w_2,\ldots,\w_\pwlen)$, such that $\prod_{i=1}^{\pwlen} \Pr[\w_i]$ is maximized. SymSpell also tolerates a specified amount of variations of the words in $V$. However, this segmentation often failed to identify proper nouns like city names or first names which are often part of a password. Thus, we devised a hybrid approach. First, we implemented a greedy strategy to find known words in a password where we searched for words from exhaustive lists of known words including names of countries and popular cities~\cite{geo}, and common first names in the US~\cite{moby}. We only consider a password as passphrase if it has at least three valid English words, each of length greater than or equal to 3. If this greedy approach fails to segment a given passphrase (likely signifying variations of words), we used SymSpell library for segmentation. 







            

    




\par
\begin{figure}[t]
  \centering
  \scriptsize
\begin{tabular}{p{0.7in}l}
  \toprule
  Type & Examples \\\midrule
  \multirow{-2}{0.7in}{Popular passphrases} & \begin{tabular}[c]{@{}l@{}}
                            \texttt{bullet-for-my-valentine}\\
                            \texttt{sponge-bob-square-pants}\\
                            \texttt{get-there-very-fast-indeed}
                            \end{tabular}    \\\cmidrule{2-2}
  \multirow{-2}{0.7in}{Unpopular passphrases} & \begin{tabular}[c]{@{}l@{}}\texttt{friendly-neighborhood-pickle} \\
                    \texttt{eddie-the-penguin-stick} \\
                    \texttt{super-looper-evil-ben}\end{tabular}    \\\midrule\midrule
  {Ineligible} & \begin{tabular}[c]{@{}l@{}}\texttt{speedtriple123456789} \\ \texttt{21101975-invalidlogin}\\ \texttt{newjob2thomapink\_socks08}\end{tabular} \\ \midrule
  \multirow{-1}{*}{Common names} & \begin{tabular}[c]{@{}l@{}}\texttt{KatherineCarrasquillo} \\ \texttt{zuleimahernandez1230} \\ \texttt{dobrovolskayatatiana}\end{tabular}             \\ \midrule
  {Non-phrasal} & \begin{tabular}[c]{@{}l@{}}\texttt{ltdjxrfgtctybz27102003} \\ \texttt{oilgurtalococsecnarf} \\ \texttt{903kingdalonsbfreitag}\end{tabular}           \\ \midrule
\end{tabular}
\caption{Figure shows different types of passwords that we analyzed. In the top row, we show the most common passphrases, as well as random samples that we were able to extract from passwords, while in the bottom three rows, we show the passwords that we do not consider as passphrase, and the categories they fall into.
}
\label{fig:ex-passphrases}
\end{figure}

\paragraph{Resulting passphrases dataset.}  Using our segmentation approach, we found~~$\nppws$ passphrases from our segmentation algorithm. We show the top three most used passphrases as well as three randomly sampled passphrases from the ones used by only one user in the top row of~\figref{fig:ex-passphrases}.




We also checked the passwords which were not considered as passphrases by our algorithm to gauge the false negative rate.  We random sampled hundred such passwords from 1.8~million passwords that are discarded by our algorithm and manully analyzed them to find three key types in this set: (a) \emph{Ineligible} (passwords having less than 3 words), (b) \emph{Common names}
(passwords that are common names, and our segmentation algorithm failed to segment them meaningfully), and (c) \emph{Non-phrasal} (passphrases that are just a mix of letters, digits, and symbols that do not segment into any meaningful sequence of words.) We show samples of these three types of passwords at the bottom three groups of rows in \figref{fig:ex-passphrases}. 

\subsection{Ecological validity of our dataset}\label{sec:ecolofical_passphrase}

\changed{As mentioned earlier, the passphrases we identified are a subset of a broader password breach dataset that is widely used in prior works and is known to contain real email-password pairs~\cite{fouriq}. According to the work by Bonneau et. al.~\cite{bonneau_linguistic_2012}, passphrases in their Amazon PayPhrase dataset (not publicly available) have linguistic properties similar to natural language. We make the same observation in our dataset too using a GPT LM-scorer, which shows user-generated passphrases that we extract are similar to natural text (\figref{fig:lm-cdf}) and some examples are given in \figref{fig:ex-passphrases}. %
We verified that the frequency distribution of the passphrase dataset mirrors that of the original breached password dataset, hinting at ecological validity of our passphrase dataset--three most frequent passphrases are used by 252 (0.26\%), 115 (0.12\%), and 102 (0.10\%) users.}












\section{Linguistic properties of in-use passphrases}\label{sec:perplexity}

\begin{figure}[t]

  \includegraphics[width=6.5cm]{figures/perplexity_new_cropped.pdf}
  \caption{CDF of logarithm of perplexity to the base 2 for \dice, \mmap and \userpp passphrases, along with natural language phrases for baseline. 
  }
  \label{fig:lm-cdf}
\end{figure}


We leveraged a natural language model GPT-2~\cite{radford2019language} to check similarity of various system-generated passphrases as well as user passphrases to the natural language~\cite{radford2019language}. GPT-2 is trained to predict the next word given a sequence of words and widely used today~\cite{exgpt2}. We sampled 3000 passphrases of similar length distribution from each system (dataset for \userpp) and compute their perplexity using GPT-2. Lower perplexity implies similarity to natural language. The distribution of the perplexity score~\cite{brown-1992-perplexity} for both the system  (\dice and \mmap) and \userpp passphrases are shown in~\figref{fig:lm-cdf}. \userpp passphrases have a very low perplexity value which is similar to the perplexity of a natural language corpus, uncovering a key reason for their memorability. On the other hand, the passphrases from \mmap claim a close second in their resemblance to natural language, almost comparable to user passphrases, which indicates a significant improvement over the passphrases generated by \dice. %

\section{Shortcoming of \mmap}\label{sec:mmap-shortcoming}

\begin{figure}[t]
  \includegraphics[width=6.5cm]{figures/mmap_length_cropped.pdf}

  \caption{Variation of mean log$_{10}$ guessrank across length for the \mmap passphrases. We observe that guessranks reached a plateau after length eight. %
  }
  \label{fig:mmap-length}
\end{figure}


We show the guessrank of an adversary who does brute force guess of the syntax rules and then wordlists on \mmap passphrases in \figref{fig:mmap-length}. We note that the guessranks of these passphrases gets saturated around length 8---guessrank of 8-word passphrase is nearly the same as that as of 13-word. The potential reason is the constraint imposed by the underlying hardcoded and extremely limited syntax rule patterns of \mmap




\section{Trade-off for \system{} parameters}\label{sec:thresholdtradeoff} 

\noindent \textbf{Constraint thresholds.} \label{ssec:constraints} We introduce $\thetaone$ and $\thetatwo$ as thresholds to be satisfied by each of the probable words from the support (at every step of generation). Words that fail in any of the two constraints are removed from the support, and then we choose the next word weighted by their conditional bigram probabilities.

The goal of these empirical thresholds is to impose constraints on a relaxed upper bound estimate for the CER score, $\score$, and a relaxed upper bound estimate for the resultant bigram probability, $\lprobbi(w_{i-1},w_{i})$. They also ensure that generated passphrases can still retain syntactic structure and flow. 

Since our CER estimate is a linear fit, and the log probability of a phrase is sum of the log probabilities of its constituent unigrams or bigrams, we estimated the score obtained at every step as a greedy approach to obtaining a sub-optimal solution, rather than having to traverse through every path from the $\start$ token exhaustively. This approach is beneficial for using a different corpus or controlling the weight of the factors influencing the CER estimate.

\textbf{Equation coefficients: }$ \{\alpha_{i}\}_{i=1}^3$ are the coefficients of the individual parameters we fit with the regression model trained over the \wiki dataset as the universal corpus (\secref{memopt}). These essentially determine the weight of each factor in the estimation of a phrase's memorability. The upper bound of the log bigram probability, $\thetatwo$, is set to a suitable fraction of the minimum log bigram probability found in the corpus. Similarly, the threshold for  intermediate CER score ($\score$), $\thetaone$, is set as fraction of maximum CER.


\section{Example of passphrases from different algorithms}\label{sec:passsample}

\begin{figure}[t]
  \centering
  \scriptsize
  \begin{tabular}[t]{lrp{2.3in}}
    \toprule
    Type & &  Samples \\\midrule
    \multirow{5}{*}{\dice}
         & 1) & \texttt{dreamscape manchuria dervish verbally} \\
         & 2) & \texttt{clay reactive smasher authentic chrome hamster} \\
         & 3)&  \texttt{spindle chemicals griminess waviness vintage stammer agenda sulphate} \\
    \midrule
    \multirow{3}{*}{\userpp}
         & 1) & \texttt{mind the fold law you should} \\
         & 2) & \texttt{just another happy ending} \\
         & 3) & \texttt{dont cry over spilt milk} \\
    \midrule
    \multirow{4}{*}{\markov}
         & 1) & \texttt{leopold arranged for some users include the war} \\
         & 2) & \texttt{during ultraviolet signals beamed}\\
         & 3) & \texttt{their home delivery and morgan suggested that more} \\
    \midrule
    \multirow{5}{*}{\mmap}
         & 1) & \texttt{when does a bellboy spike an elect but not a sidebar} \\
         & 2) & \texttt{why does Suzy grumble a redeemer}\\
         & 3) & \texttt{how does my overdone one push those violinists after their sounding} \\
    \midrule
    \multirow{4}{*}{\ours}
         & 1) & \texttt{edge bands influenced how far north south} \\
         & 2) & \texttt{stalin however was offered exclusive control those four} \\
         & 3) & \texttt{graham and republic records} \\
    \midrule
    \multirow{4}{*}{GWC}
         & 1) & \texttt{rub revisions lilo clerk apple beting} \\
         & 2) & \texttt{helios hounds binary canonized lady overflight} \\
         & 3) & \texttt{pressures broth billable playgirl raita dekko} \\
    \bottomrule
  \end{tabular}
  \caption[Passphrase samples]{Three randomly sampled passphrases from each
    group of passphrases we consider for evaluation.}
  \label{fig:samples}
\end{figure}

\figref{fig:samples} presents a set of randomly chose passphrases generated by each algorithm and the ones generated by users. 


\section{User Study instrument}
\label{sec:userstudy}





\subsection{Part 1}\label{ssec:part1}
\begin{footnotesize}

\textbf{Informed consent}

\noindent First we show the informed consent to the users. Users will see the questions below only if they indicated that they understood the requirements and agree to participate.


\begin{itemize}
\item Please enter your prolific id \_\_\_\_\_\_

\item When you login to your online accounts you often need a credential, i.e., your email/username and a secret. 

Two possible ways is creating the secret is: choosing a password (a set of characters) or choosing a passphrase (a set of words). Example of a password is "!Passw0rd!" and example of a passphrase is “correct horse battery staple”. 

You can also use a password or a passphrase as a master secret for accessing your password manager (a software that stores and manages all of your login credentials across multiple online accounts). 

Do you use passphrases as a secret for logging-in to any of your online accounts?
$\ocircle$ Yes $\ocircle$ No

 
\textcolor{gray}{if YES to use passphrases}
\item  Please briefly explain why do you use passphrases for these account(s) instead of passwords? (1-2 sentences) \hrulefill

\textcolor{gray}{if YES to use passphrases}

\item  How did you generate your passphrase(s)? Choose all that apply.

\begin{enumerate}
    \item [\ensuremath{\circ}]  Used an online tool [also write names, if you remember]: \hrulefill
    \item [\ensuremath{\circ}]  Self-generated - using a quote from a poem, movie, or book
    \item [\ensuremath{\circ}]  Self-generated - using a combination of random words
    \item [\ensuremath{\circ}]  Other: \hrulefill
\end{enumerate}


\textcolor{gray}{if NO to use passphrases}

\item Briefly explain why do you NOT use passphrases for these account(s) (and use passwords)? (1-2 sentences)

\textcolor{gray}{if NO to use passphrases}

\item If you have to use a passphrase as your master credential for a password manager, how would you generate it? Choose all that apply. 
\begin{enumerate}
    \item [\ensuremath{\circ}] Use an online tool [also write names, if you remember]: \hrulefill
    \item [\ensuremath{\circ}] Self-generated - will use a quote from a poem, movie, or book
    \item [\ensuremath{\circ}] Self-generated - will use a combination of random words
    \item [\ensuremath{\circ}] Other: \hrulefill
\end{enumerate}

\item Which of these crawls?
$\square$ Dog $\square$ Cat $\square$ Snake $\square$ Kite 




\item  For the questions below please choose the option which applies most for you
\begin{itemize}
    \item Do you write down your login credentials to remember them?\\
    Never \ensuremath{\circ} \ensuremath{\circ} \ensuremath{\circ} \ensuremath{\circ} \ensuremath{\circ} Always
    \item How often do you use the same  login credential for multiple websites?\\
    Never \ensuremath{\circ} \ensuremath{\circ} \ensuremath{\circ} \ensuremath{\circ} \ensuremath{\circ} Always
\end{itemize}

\noindent \textbf{Passphrase choice}

\item In this section, we will show you a few passphrases and ask you to choose one.  A good passphrase should be long so that others cannot guess it, but also be easy to remember so that you can enter the passphrase with minimum errors.

We are showing three passphrases and how secure they are in terms of the time it might take to guess the passphrase by an attacker. Please choose one passphrase which you prefer as your master login credential (e.g., for your password manager). %

Please do not write down the chosen passphrase.%
Try to remember it to the best extent possible. %

Please DO NOT COPY/PASTE passphrases in this study. Such actions will be detected and your task could be invalidated.

    [Passphrase 1] [Passphrase 2] [Passphrase 3]

\textcolor{gray}{Repeat question below 5 times for practicing}
\item For practicing, please enter your chosen passphrase: [CHOSEN PASSPHRASE] (4 more practices remaining)

\noindent \textbf{Post Passphrase choice questions}


\item Why did you choose this particular passphrase among the ones shown? (1 - 2 sentences): \hrulefill



\item Please select options from below which have positively affected your memorability of the chosen passphrase
    \ensuremath{\circ} Contains frequently used words.
    \ensuremath{\circ} Grammatically correct
    \ensuremath{\circ} Fewer words to remember as it contains common words
    \ensuremath{\circ} Flows like an English phrase
    \ensuremath{\circ} Other: \hrulefill

\noindent \textbf{Demographics}

\item Which age group do you belong to?
          \ensuremath{\circ} 18-30 
     \ensuremath{\circ} 31-45
     \ensuremath{\circ} 46-60
     \ensuremath{\circ} 60+

\item Which gender do you identify yourself most with?
    \ensuremath{\circ} Male 
    \ensuremath{\circ} Female
    \ensuremath{\circ} Non-Binary / Third Gender
    \ensuremath{\circ} Prefer not to say

\item What is the highest degree or level of school you have completed?
    \ensuremath{\circ} Some high school 
    \ensuremath{\circ} High school  
    \ensuremath{\circ} Some college  
    \ensuremath{\circ} Trade, technical, or vocational training 
    \ensuremath{\circ} Associate’s degree 
    \ensuremath{\circ} Bachelor’s degree
    \ensuremath{\circ} Master’s degree 
    \ensuremath{\circ} Professional degree 
    \ensuremath{\circ} Doctorate 
    \ensuremath{\circ} Prefer not to say 

\item Which of the following best describes your educational background or job field?
    \ensuremath{\circ} I have an education in, or work in, the field of computer science, engineering, or IT.
    \ensuremath{\circ} I do not have an education in, or work in, the field of computer science, engineering, or IT. 
    \ensuremath{\circ} Prefer not to say.

\end{itemize}
\end{footnotesize}

\subsection{Part 2} \label{ssec:part2}
\begin{footnotesize}

Welcome to a brief follow-up of the earlier study on passphrases that you completed. Recall that our study is on understanding utility of passphrases as login credentials for online accounts. 

Your primary task in this final part of this study is to just re-enter your chosen passphrase as you remember in this survey and answer a few questions regarding your current perception about passphrases. 
This part will take around 3 to 5 minutes of your time. You’ll be compensated \$1.00 USD for this part. Thank you for helping in our research to improve the privacy and security of users by understanding usage of passphrases.
\noindent\textbf{Questions}

\begin{itemize}

\item To start the survey please enter your Prolific ID \hfill 

\noindent \textbf{Check recall after 2 days}

\textcolor{gray}{Repeat the question below at most five times}

\item Please enter the passphrase you chose in Part 1 of this study below. (5 tries remaining). \_\_\_\_\_\_\_\_\_\_\_\_


\item What is your preferred length for a passphrase you would want to use (the number of words)?

\ensuremath{\circ} 7 or less words \ensuremath{\circ} 8-12 words \ensuremath{\circ} $>$12 words 


\item Please briefly explain your choice of preferred length for passphrases (1-2 sentences): \hrulefill

\item If you have to use a passphrase, then while choosing the passphrase, would you prefer to prioritize security or memorability? 
\\Prioritize only Memorability \ensuremath{\circ} \ensuremath{\circ} \ensuremath{\circ} \ensuremath{\circ} \ensuremath{\circ} Prioritize only Security

\item Currently, do you feel changing your chosen passphrase slightly (e.g., a few characters) would have made it more memorable for you without making it easy to guess for others?
\begin{enumerate}
    \item [\ensuremath{\circ}] No, I don’t want to change the passphrase.
    \item [\ensuremath{\circ}] Yes, I want to change my chosen passphrase to \hrulefill
\end{enumerate}

\item In brief, why do you feel your modification to the chosen passphrase will make the passphrase more memorable without making it easy to guess for others? (1-3 sentences) \hrulefill

\item After participating in this study, how likely are you to use passphrases as your login credential for some online accounts instead of passwords?\\
Not Likely At All \ensuremath{\circ} \ensuremath{\circ} \ensuremath{\circ} \ensuremath{\circ} \ensuremath{\circ} Very Likely

\item Please briefly explain your response (1-3 sentences) \hrulefill

\end{itemize}

\end{footnotesize}



\section{Comparing \system{} with human-in-the-loop passphrase generation}\label{sec:gwc}

\changed{We further 
tested the utility of \ours against passphrase generation systems which puts human in the loop (and presumably incur higher cognitive and time cost for users in exchange for more user control). Specifically, we compared the memorability of \ours generated passphrases with passphrases generated via guided word choice (GWC) method %
~\cite{blanchard-2018-gwc}.  GWC asks users to choose words for their passphrase from a list of random words shown to them.    
On our request, Blanchard et al. graciously shared the set of passphrases used in their experiment~\cite{blanchard-2018-gwc} (Implementation of GWC is unavailable). We show a sample of GWC passphrases in~\figref{fig:samples}. Thus, in this experiment we repeated part 1 of our original study (since Blanchard et al. also did not ask participants for a followup task~\cite{blanchard-2018-gwc}). A total of 37 users (recruited via mailing lists) were randomly shown either \ours generated passphrase or GWC-generated passphrases (from original paper). Then these users typed the chosen passphrases via a login screen after answering a few distracting questions. (18 users were shown GWC passphrases and 19 were shown \ours{} passphrases). A comparable 72.2\% users recalled GWC passphrases correctly in one try and 73.7\% users recalled \ours passphrases in one try. However, interestingly, the average edit distance for GWC passphrases was 2.72 which is higher than the average edit distance of \ours passphrases (1.46). Thus, although GWC provides users more control while choosing random words, in our experiments \ours-passphrases are still slightly more memorable (and less error prone) than GWC-generated passphrases.}
