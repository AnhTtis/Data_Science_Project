\section{User Study}
\label{sec:workerstudy}

\noindent We used CER to estimate the difficulty of memorizing various passphrases (in~\secref{ssec:memeval}) analytically. However, to evaluate if indeed,  final \system{} generated passphrases are memorable to the users we ran a user study. %


\paragraph{Ethical considerations.} As the primary authors' institution did not have an official ethics review board, we did not obtain any official ethics review of the study. However, we discussed extensively with our peers and followed best practices of ethical research (e.g., principles set by Belmont Report ~\cite{beauchamp2008belmont}). The study was performed with informed consent. We asked for no sensitive (e.g., actual in-use passphrases) or personally identifiable information (such as email id).  We also removed the worker ids from survey results during analysis, reducing  privacy risk to the participants.

\begin{figure}[t]
  \centering
  \small
  \begin{tabular}[t]{lccc}
    \toprule
    Model & Part1 count & Part2 count & Return rate\\
    \midrule
    \ours & 72 & 61 & 84.72\% \\
    \mmap & 78 & 63 & 80.76\% \\
    \markov & 69 & 41 & 59.42\% \\
    \dice & 82 & 50 & 60.97\% \\
    \bottomrule
  \end{tabular}
  \caption{\#participants who completed Part 1 and returned for Part 2 with return rate (across different algorithms).}
  \label{fig:parti_count}
\end{figure}

\subsection{Design and setup}
\label{ssec:design}

Our two-part survey-based user study was deployed on crowdsourcing platform Prolific, where we assigned Prolific participants at random to one of passphrase generation algorithms out of \mmap, \ours, \markov, \dice and show them a passphrase to remember. For each algorithm, we randomly generated passphrases while uniformly choosing passphrase length from one of three length ranges ($\le$ 7, 8-12, >12). We used the range of $\le 7$ as most of user-generated passphrase (in our previous dataset from \secref{ssec:langmodel}) lies in the range. We recruited participants on Prolific with a greater than 99\% approval rate for our two-part survey. We selected US Citizens who are fluent in English and are above the age of 18. All the approved participants from part 1 of the survey were invited to carry out part 2. Our study was designed based on prior work on measuring memorability of login credentials~\cite{woo2020we}. 



\textbf{Part 1.} In the first part, we asked each participant to choose from the three passphrases shown to them, all of which were generated by the same algorithm (randomly chosen for the participant) and were of the same length. %
Participants were asked not to write down or copy their chosen passphrase, and to only rely on memory. They were made to practice their chosen passphrase five times after finalizing their choice. Then, each participant was asked to authenticate twice---at the end of part 1 of the survey, and the beginning of part 2. We allowed at most five tries to authenticate in both parts of the survey. The users were asked not to paste their answers and were also assured that they would receive payment regardless of their authentication success. To distract participants before asking for authentication, the participants answered demographics, some generic questions and attention check questions. %

\textbf{Part 2.}  \changed{We invited participants who successfully finished part 1 to return and authenticate again after 48 hours from their completion time.} %
A two-day recall time interval is used and justified for most prior password and passphrase recall research~\cite{woo2020we,Kelley2012,joudaki2018reinforcing,huh2015surpass}. In fact Huh et al. argued that a good recall rate after two days potentially (empirically) signifies the practically required memorability of passphrase---thus we used the same interval~\cite{huh2015surpass}. 
\changed{However, 24\% of the users responded to the second part of the survey post 96 hours (4 days or more).}
At the end of the authentication, we provided a short survey to assess participants' perception of the chosen passphrase and how they may want to modify the passphrase. 

\changed{We paid \$0.75 to the participants who completed part 1 and \$1.00 for the participants who returned and finished their authentication in part 2. The participants took a total of 10 minutes on average to complete both parts.} The survey instrument is in Appendix~\ref{sec:userstudy}. For analysis, we used Kruskal-Wallis (KW) test and pairwise Mann-Whitney U tests to find statistically significant differences at $\alpha=0.05$~\cite{lehmann2005testing}. 
\changed{Our registration and login prompts are designed to reflect a real login system in line with a plethora of previous work~\cite{egelman2013ecologicalpass, woo2020we, ur-measure}. Like earlier work on conducting ecologically valid password study~\cite{egelman2013ecologicalpass}, we did not nudge the user to choose/remember passphrases for any specific system (e.g., \dice or \ours). Thus, we strongly believe our data (e.g., recall) also captures users' ecologically valid preferences.}






\subsection{Demographics and participants info}
\label{ssec:demograph}

A total of 310 Prolific users participated in our user study (we also ran 5 pilots in the survey development phase and updated questions to remove ambiguity according to pilot feedback). Among these participants, we detected invalid responses from 9 participants (2.9\%) who failed the attention check. After excluding those, 301 participants successfully chose the passphrases, practiced them, and answered all the survey questions properly. 

Forty-eight hours after Part 1, we emailed participants to return for the authentication in the second phase. Out of 301 participants, 217 participants returned within 24 hours of sending email, yielding a return rate of 72.1\%. Out of these participants, 202 (93.1\%) self-reported that their desired passphrase length (in number of words) is 7 or less for being able to remember in daily use. 

Of the 301 users, 70\% and 24.2\% of them reported their gender as female and male, respectively, while the rest either reported it as non-binary or they did not prefer to reveal it. Among the users who participated, the two highest age groups reported were 18-30 (53.16\%) and 31-45 (30.56\%). Also, 36\%, 29\% and 17\% of the users reported their highest qualifications as Bachelor's Degree, Some college, Master's Degree, respectively. We found no statistically significant differences across the models in their gender, age group, or highest qualification. Only 13\% participants reported working in or having education in IT or related fields.

\paragraph{Return rate is high for \ours{} due to better recall:} The distribution of our  participants across the different models and their return rate is also shown in \figref{fig:parti_count} (demographics are similar). The return rate is very high in \ours, almost 25\% higher than \dice. To further investigate, we checked the Part 1 recall rates (posted right after practicing 5 times) between the users who returned in part 2 and those who did not. We omit the detailed results for brevity, but we found that across all algorithms, users who did not return in part 2 has a significantly low part 1 recall rate. This difference is more prominent for passphrases with length 7 or less. E.g., for that passphrase length \ours have a recall rate of 92.31\% in ones who returned for part 2 and 60\% for those who did not. These results hint that the low return rate for \dice and \markov is indicative of the underlying fact that a significant fraction cannot recall the passphrases. In fact one participant mentioned for a Dice passphrase that \textit{``It was the longest and used random words''} and another mentioned for a Markov passphrase that \textit{``I'm already pretty confident I'm not gonna remember this one guys :(''}. These results indicate that some \dice and \markov passphrases are hard for users to remember. 


\subsection{Passphrase statistics}
\label{ssec:pass_stats}


\noindent \textbf{Word length.} To make sure the statistical analysis yields a proper comparison, we would like the distribution of the word lengths of the passphrases across the different models to be similar. The average word length across the passphrases chosen in part 1 among \ours, \mmap, \markov, and \dice are 9.19, 8.74, 8.62, and 9.06, respectively. We perform a KW test to see how different the underlying distributions are. The KW test fails to reject the null hypothesis ($H = 3.59, p = 0.31$), confirming that the underlying distribution is not significantly different. The distribution of the word lengths of the passphrases across returning participants for part 2 among \ours, \mmap, \markov, and \dice are 9.42, 8.81, 8.85, and 9.16, respectively. Similar to part 1, the KW test gives $p = 0.39$ ($H = 2.97$), identifying that the underlying distribution is not significantly different among part 2 participants.

{\bf Passphrase strength.} We estimate the strength of the passphrases by their guess rank, which was calculated apriori using the process mentioned in \secref{ssec:guesseval}. %
The mean log guess rank (base 10) across \ours, \mmap, \markov, and \dice are 14.80, 11.70, 14.46, and 36.49, respectively. Similarly, the median log guess rank (base 10) across \ours, \mmap, \markov, and \dice are 14.20, 12.51, 12.86, and 36.05, respectively. %
Thus, for non statistically different length distributions, the \dice and \mmap are most and least secure, respectively.

To compare the underlying distribution of the guess rank among the four algorithms, we again perform the KW test. This KW test rejects the null hypothesis ($H = 157.7, p \approx 0)$, confirming that the underlying distributions are statistically significantly different. We then perform the Mann-Whitney U test on all possible pairs, and further find that guess ranks of \textit{each pair} to be statistically significantly different. Next, we check the memorability of these passphrases using survey responses. 

\subsection{Evaluating passphrase memorability}
\label{ssec:recall_cer}

\begin{figure}[t]
  \centering
  \small
  \begin{tabular}[t]{lccc}
    \toprule
    Model & Recall & Mean CER & Median CER \\
    \midrule
    \ours & 26.23\% & 34.78\% & 35.85\% \\
    \mmap & 17.46\% & 35.44\% & 36.58\% \\
    \markov & 21.95\% & 37.84\% & 41.27\% \\
    \dice & 24.00\% & 38.49\% & 42.57\% \\
    \bottomrule
  \end{tabular}
  \caption{\% participants with successful recall, mean and median CER (as \%) while authenticating after 2 days.\ours perform best. Surprisingly \dice is close second.}
  \label{fig:recall_cer}
\end{figure}

\begin{figure}[t]
  \centering
  \small
  \begin{tabular}[t]{lccc}
    \toprule
    Model & Recall & Mean CER & Median CER \\
    \midrule
    \ours & 46.15\% & 19.10\% & 8.82\% \\
    \mmap & 28.57\% & 30.03\% & 34.48\% \\
    \markov & 14.29\% & 39.74\% & 43.59\% \\
    \dice & 23.08\% & 45.48\% & 56.41\% \\
    \bottomrule
  \end{tabular}
  \caption{\% participants who were shown passphrase length $\le$ 7, with successful recall, mean and median CER (as \%) while authenticating after 2 days. Recall of \ours is 2x of that of \dice and 1.6x of \mmap.%
  }
  \label{fig:recall_cer_less7}
\end{figure}

\paragraph{Recall and CER.} In our case, \textit{successful recall} and \textit{low CER} signify high memorability. Recall is successful if in part 2 users correctly input every character (including spaces) of the passphrase within five attempts. Also, the character error rate (CER) for a particular attempt is the edit distance between the attempt and the original passphrase divided by the number of characters in the original passphrase. We calculate CER for a user as the minimum CER across all attempts.
In part 1 (right after seeing the passphrase) \ours, \mmap, \markov, and \dice have  similar recall rates (between 50\% to 60\%) overall. However, for passphrases of length 7 or less (self-reported as preferred by more than 90\% participants), \ours has a Part 1 recall rate of 83.3\%, significantly outperforming other algorithms (60\%---65\% for this length range). 




\textbf{Recall after two days.} \figref{fig:recall_cer} shows the percentage of users who were able to successfully recall passphrases chosen in part 1 (recall rate), as well as the mean and median CER for all users after 2 days. \ours{} has the highest recall rate of 26.23\% among all algorithms (\mmap recall only 17.46\%). Surprisingly, the recall rate of \dice is 24\%, not too far from \ours. %

However, when we investigated further, we made two key observations. First, the return rate of \ours was 84.72\% and that of \dice was only 60.97\%. As we discussed before, the recall rate in part 1 for users who did not return in part 2 was significantly lower than the ones who returned. So, we already have close to 40\% users from part 1 who did not remember \dice passphrase (as opposed to less than 16\% for \ours). 

Second, for participants who returned for part 2, \figref{fig:recall_cer_less7} shows the recall rate, mean and median CER for passphrases of length 7 or less, which the majority of participants wanted to use. We hypothesize that for remembering higher-length passphrases, there are other confounding factors including strong user bias against using very long phrases for day to day use. For passphrases of length 7 of less, \ours have a recall rate of 46.15\% after two days of no forced practice and less than 10\% median CER. This recall rate is 2x higher than \dice and 1.6x higher than \mmap, which are used in real-world systems. Furthermore, the median character error rate (CER) for \ours passphrases (8.82\%) is 6-times lower than \dice (56.41\%) when the passphrases of length seven or less are considered. \changed{We further checked if the edit distances between typed passphrases and actual passphrases for \ours is statistically significantly lesser than other algorithms (lesser edit distance impliers lesser error and more memorability). We ran pairwise Mann-whitney U tests (with Bonferroni correction for multiple tests) over the edit distances for these passphrases in part 2. The edit distances for \ours passphrases is indeed statistically significantly lower than other algorithms (p < 0.005). In fact the average edit distance for \ours is 6.27 at the end of part 2 which is approximately 2.9 times less than \dice (18.61)}. So, for the desired length of passphrases (as self-reported by the users) \ours significantly improves the memorability of state of the art passphrase generation algorithms. 


























































\changed{\system{} is not a human-in-the-loop (HITL) approach. However, some prior work took a HITL approach of passphrase generation too. Just to test the utility of \system{} (even though it's not a fair comparison) we compare \system{} with a guided word choice (GWC) method by Blanchard et al.~\cite{blanchard-2018-gwc} using a separate user study (Appendix \ref{sec:gwc}). Although GWC provides users more control for choosing their random words, the user study reveals that Mascara-passphrases are still at least comparable or a little more memorable (and less error prone) than GWC-generated passphrases.}


