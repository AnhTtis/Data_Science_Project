% edited April 2006
% written by as, Feb 2006
\documentclass[12 pt,leqno]{amsart}


\topmargin=-25pt \hoffset=-60pt \voffset=-5pt \headheight=0pt
\textheight=685pt \textwidth=475pt
\usepackage[all]{xy}
\usepackage {amsfonts}
\usepackage{amsthm}
\usepackage{amssymb,amsxtra}
\usepackage{mathabx}
\usepackage{latexsym}
\usepackage{amsmath}
\usepackage{mathrsfs}
\usepackage{dsfont}
\usepackage{lmodern}
\usepackage{nicefrac,mathtools}
\usepackage{calc}
%\usepackage{comment}
%\usepackage{a4wide}
%\usepackage{showkeys}
%\usepackage{pifont}
%\usepackage{color}
%\usepackage{a4wide}


\usepackage{enumitem} %<--advanced enumerate, which handles label and ref
\setlist[enumerate,1]{label=\textup{(\arabic*)}}% ensure enumerates in theorems are upright


\pagestyle{plain}

\theoremstyle{plain}
\newtheorem{thm}{Theorem}[section]
\newtheorem{prope}[thm]{Property}
\newtheorem{propes}[thm]{Properties}
\newtheorem{lem}[thm]{Lemma}
\newtheorem{prop}[thm] {Proposition}
\newtheorem{prob}{Problem}[section]
\newtheorem{cor}[thm]{Corollary}
\newtheorem{con}{Conjecture}[section]

\theoremstyle{definition}
\newtheorem{defn}[thm]{Definition}
\newtheorem{rem}[thm]{Remark}
\newtheorem{ex}[thm]{Example}



\newcommand{\eqn}{\begin{equation}}
\newcommand{\eqne}{\end{equation}}


\newcommand{\andom}{\anal \twodens}

\newcommand*{\UMult}{\mathcal UM}
\newcommand*{\Mult}{\mathcal M}% multiplier algebra

\newcommand{\donto}{\stackrel{\tiny \mathrm{dense}}{\onto}} 

\mathchardef\mhyphen="2D %math mode hyphen



\DeclareMathOperator{\Ideals}{\mathcal I}% ideal lattice
\DeclareMathOperator{\Kish}{Aper}
\DeclareMathOperator{\dashind}{-Ind}
\DeclareMathOperator{\Aut}{Aut}
\DeclareMathOperator{\End}{End}
\DeclareMathOperator{\Iso}{Iso}
\DeclareMathOperator{\WOT}{w\text{-}}
\DeclareMathOperator{\PBij}{PBij}
\DeclareMathOperator{\PHomeo}{PHomeo}
\DeclareMathOperator{\Homeo}{Homeo}
\DeclareMathOperator{\PAut}{PAut}
\DeclareMathOperator{\iso}{iso}
\DeclareMathOperator{\orb}{o}
\DeclareMathOperator{\op}{op}
\DeclareMathOperator{\red}{r}
\DeclareMathOperator{\ess}{e}
\DeclareMathOperator{\desing}{es}
\DeclareMathOperator{\alg}{alg}
\DeclareMathOperator{\spa}{spa}
\DeclareMathOperator{\reg}{reg}
\DeclareMathOperator{\sing}{sing}
\DeclareMathOperator{\SPIso}{SPIso}
\DeclareMathOperator{\PIso}{PIso}
\DeclareMathOperator{\PSA}{PSA}
\DeclareMathOperator{\clsp}{\overline{span}}
\DeclareMathOperator{\spane}{span}
\newcommand{\Prim}{\textrm{Prim}\,}
\DeclareMathOperator{\Ker}{Ker}
\DeclareMathOperator{\sign}{sign}
\DeclareMathOperator{\hull}{hull}
\DeclareMathOperator{\supp}{supp}
\DeclareMathOperator{\Null}{\mathcal{N}}
\DeclareMathOperator{\Bis}{Bis}
\DeclareMathOperator{\E}{\mathbb{E}}
\DeclareMathOperator{\EL}{\mathbb{E}L}
%\newcommand{\Im}{\textrm{Im}\,}
\newcommand{\Int}{\textrm{Int}\,}
\newcommand{\Hilt}{\tilde{\Hil}}
\newcommand*{\Id}{\textup{Id}}%identity
\newcommand{\id}{\mathrm{id}} %also the identity
\newcommand*{\Ad}{\textup{Ad}}%conjugation by a unitary
%% Special letters
\renewcommand{\H}{\mathcal H}
\newcommand{\I}{\mathcal I}
\newcommand{\Ind}{\mathcal I}
\newcommand{\B}{\mathcal B}
\newcommand{\LL}{\mathcal L}
\newcommand{\M}{\mathcal M}
\newcommand{\G}{\mathcal G} %groupoid
\newcommand{\Gz}{{\G^{(0)}}} %groupoid
\newcommand{\TT}{\mathcal T}
\newcommand{\CC}{\mathcal C}
\newcommand{\NN}{\mathcal N}
\newcommand{\UU}{Z}%\mathcal U}
\newcommand{\Hil}{H}%\mathsf H}
\newcommand{\EE}{\mathcal E}
\newcommand{\JJ}{\mathcal J}
%\newcommand{\KK}{\mathcal K}
\newcommand{\x}{\widetilde x}
\newcommand{\q}{\widetilde q}
\renewcommand{\a}{\widetilde a}
\renewcommand{\b}{\widetilde b}
\newcommand{\OO}{\mathcal{O}}
\newcommand{\y}{\widetilde y}
\newcommand{\X}{\widetilde X}
\newcommand{\Y}{\widetilde Y}
\newcommand{\U}{\widetilde U}
\newcommand{\V}{\widetilde V}
\newcommand{\F}{\mathbb F}
\newcommand{\TDelta}{\widetilde \Delta}
\newcommand{\tdelta}{\widetilde \delta}
\newcommand{\tmu}{\widetilde \mu}
\newcommand{\TSigma}{\widetilde \Sigma}
\newcommand{\tomega}{\widetilde \omega}
\newcommand{\tal}{\widetilde \varphi}
\newcommand{\p}{\varphi}
\newcommand{\al}{\alpha}
\newcommand{\K}{\mathcal{K}}

\renewcommand{\L}{\mathcal L}

%\renewcommand{\dots}{...}

\renewcommand{\-}{^{\text{--}}}
\newcommand{\+}{^{\text{\tiny+}}}


\newcommand{\FF}{\mathcal F}
\newcommand{\A}{\mathcal A}
\newcommand{\D}{\mathcal D}
\newcommand{\C}{\mathbb C}
\newcommand{\R}{\mathbb R}
\newcommand{\Z}{\mathbb Z}
\newcommand{\N}{\mathbb N}
\newcommand{\T}{\mathbb T}
\newcommand{\bu}{\mathcal B}

\newcommand{\cst}{\ifmmode\mathrm{C}^*\else{$\mathrm{C}^*$}\fi}

\newcommand*{\into}{\rightarrowtail}
\newcommand*{\onto}{\twoheadrightarrow}

\renewcommand{\theequation}{\arabic{section}.\arabic{equation}}

%\keywords{Noncommutative topological entropy, subshift, higher-rank graph $C^*$-algebra}
\subjclass[2000]{Primary 47L10, 22A22, 46H15}

\begin{document}
\author{Krzysztof Bardadyn}
	\address{Faculty of Mathematics, University of Bia\l ystok, ul. K. Cio\l kowskiego 1M, 15-245
	Bia\l ystok, Poland}
	\email{kbardadyn@math.uwb.edu.pl}

\author{Bartosz Kwa\'sniewski}
	\address{Faculty of Mathematics, University of Bia\l ystok, ul. K. Cio\l kowskiego 1M, 15-245
	Bia\l ystok, Poland}
	\email{b.kwasniewski@uwb.edu.pl}

\author{Andrew McKee}
	\address{Faculty of Mathematics, University of Bia\l ystok, ul. K. Cio\l kowskiego 1M, 15-245
	Bia\l ystok, Poland}
	\email{a.mckee@uwb.edu.pl}



\date{\today}


\title{\small Banach algebras associated to twisted \'etale groupoids:\\
inverse semigroup disintegration
 \\
and representations on $L^p$-spaces}


\begin{abstract} 
We initiate a study of (real or complex) Banach algebras associated to twisted \'etale  groupoids $(\G,\LL)$ and to twisted inverse semigroup actions,
 which provides a general unifying framework for numerous recent papers on 
$L^p$-operator algebras and the fully developed theory of groupoid $C^*$-algebras. 
Our main result is a version of the disintegration theorem that gives a  bijective correspondence between representations of the \emph{twisted groupoid Banach algebra} $F(\G,\LL)$ and covariant representations of an inverse semigroup action. 
This powerful tool allows us to study $F(\G,\LL)$ in terms of representations of the Banach algebra $C_0(X)$ and a (twisted) inverse semigroup $S$ of partial isometries subject to some relations. 
This works best when the groupoid is ample or when the target Banach algebra $B$ is a dual Banach algebra, for example when $B=B(E)$ for a reflexive Banach space $E$. 
When $E=L^p(\mu)$ for $p\in (1,\infty)\setminus \{2\}$ these relations force the partial isometries to be \emph{spatial}, which explains why Phillips 
and others define $L^p$-analogues of Cuntz or graph algebras in terms of spatial partial isometries --- the groupoid model forces that. 

We introduce and analyze in more detail $L^p$-operator algebras $F^{p}(\G,\LL)$,  $p\in[1,\infty]$, which are universal for representations of $F(\G,\LL)$ on $L^p$-spaces, and which generalize algebras $F^{p}(\G)$ introduced recently by Gardella and Lupini. 
As more concrete examples we discuss Banach (and $L^p$-operator algebras) associated to twisted partial group actions, twisted Renault--Deaconu groupoids and directed graphs. 
We also introduce tight inverse semigroup Banach algebras that cover all ample groupoid Banach algebras.
\end{abstract}


\maketitle
% \setcounter{tocdepth}{1}
%\tableofcontents 


%
%
\section{Introduction}
\label{sec:Introduction}

The theory of Banach algebras generated by groups is a classical part of  harmonic analysis. 
In the realm of $C^*$-algebras it was extended with huge success to transformation groups, groupoids and even more general structures and actions.  
This inspired a number of attempts to generalize this theory to Banach algebras or at least to operator algebras acting on $L^p$-spaces, but it cannot be said that any general theory has arisen yet.
Symptomatically, a series of initial preprints about Banach algebra crossed products and $L^p$-operator algebras, that appeared in the 2010s, including \cite{DDW}, \cite{PhLp1}, \cite{PhLp2a}, \cite{Phillips}, remain unpublished. 
On the other hand,  Phillips's program to create an analogue of the $C^*$-theory for $L^p$-operator algebras has noticeably sparked in recent years, and   the number of significant papers on the subject is growing rapidly, see  \cite{PH}, \cite{Gardella_Lupini17}, \cite{cortinas_rodrogiez}, \cite{cgt}, \cite{Gardella_Thiel2}, \cite{Austad_Ortega}, \cite{Hetland_Ortega}.
The main motivation behind the present paper is to develop a neat theory of Banach algebras associated to twisted \'etale groupoids that allows one to prove strong results, such as simplicity and pure infiniteness, that parallel those from the $C^*$-algebra theory. 
Even in the context of $L^p$-operator algebras such results have not appeared in the literature, apart from the particular examples of Cuntz and graph $L^p$-algebras \cite{PhLp2a}, \cite{cortinas_rodrogiez}. We will pursue this in  \cite{BK}, \cite{BKM}.


In \cite{Gardella_Lupini17} Gardella and Lupini defined the \emph{full $L^p$-operator algebra} $F^p(\G)$, for $p\in (1,\infty)$ and a second countable \'etale groupoid $\G$ with a Hausdroff unit space. 
They proved an $L^p$-version of \emph{Renault's disintegration theorem} \cite[II.1.21]{Renault_book}, that gives a bijective correspondence between representations of $F^{p}(\G)$ on $L^p$-spaces for standard $\sigma$-finite Borel measures and representations of $\G$ on standard $L^p$-bundles. 
This result has a strong measure theoretical flavor, and as such works nicely with amenability, see e.g. \cite[Theorem 6.19]{Gardella_Lupini17}, but is not so suitable to study ideal structure and all the more pure infiniteness, cf.\ \cite[Problem 8.2]{Gardella_Lupini17}.
Also for Hilbert spaces Renault's disintegration theorem is known to hold for all second countable locally compact groupoids \cite[Theorem 3.1.1]{Paterson}, or even Fell bundles over these \cite{Muhly_Willimas}. 
Thus the \'etale property does not play a role in these results. For \'etale groupoids  there is another  method of disintegration that exploits the discrete part of their structure; it requires no separability assumptions and is relevant for a number of purposes.
%It manifests in the fact that
Roughly, algebras associated to an \'etale groupoid can be viewed as crossed products for (discrete) inverse semigroup actions on the unit space, and this can be used to  give a bijective correspondence between representations of the groupoid algebra and covariant representations of the inverse semigroup action. 
We call such results \emph{inverse semigroup disintegration theorems}. 
In this paper we find the right framework and prove a general inverse semigroup disintegration theorem for twisted \'etale groupoids $(\G,\LL)$ in the context of general Banach algebras over the field $\F=\R,\C$. 
This then can be specialized to particular cases such as $L^p$-operator algebras. 
In fact we believe our results are already interesting, and give new insights, for twisted groupoid $C^*$-algebras ($L^2$-operator algebras).
In particular they hold for real $C^*$-algebras  where groupoid constructions have apparently not yet been considered (see \cite{Rosenberg}, \cite{Li_book}, \cite{Schroder}). 


%
\subsection*{Twisted groupoid algebras}

We consider a general \'etale groupoid $\G$ with a locally compact Hausdorff unit space $X$. 
By a twist we mean a Fell line bundle $\L$ over $\G$, that is a one-dimensional locally trivial bundle over $\G$ equipped with a multiplication and involution satisfying natural conditions (see Section~\ref{sec:GroupoidBanachAlgebras}). 
When $\F=\C$ these are known to be equivalent to Kumjian's twists \cite{Kumjian0}. 
The space $\mathfrak{C}_c(\G,\LL)$ of quasi-continuous compactly supported sections is a $*$-algebra with operations
\[
    (f*g)(\gamma) := \sum_{d(\eta) = r(\gamma)} f(\eta^{-1})\cdot g(\eta\cdot \gamma), \qquad
    (f^*)(\gamma) := f(\gamma^{-1})^*, \qquad f,g\in \mathfrak{C}_c(\G,\LL),\ \gamma\in \G,
\]
where $r,d:\G\to X\subseteq \G$ are the range and domain maps (which are local homeomorphisms by \'etalness). 
The universal groupoid $C^*$-algebra $C^*(\G,\LL)$ is the completion of $\mathfrak{C}_c(\G,\LL)$ in a maximal $C^*$-norm $\|\cdot\|_{C^*\text{max}}$.  
Existence of this norm follows from the $C^*$-equality and the uniqueness of the $C^*$-norm on the subalgebra $C_0(X)\subseteq C^*(\G,\LL)$. Namely, the set $S(\L)$ of open bisections of $\G$ on which the bundle $\LL$ is trivial forms a wide unital inverse semigroup of bisections of $\G$.
Thus the spaces $C_c(U,\LL)$ of continuous sections with compact support in $U\in S(\L)$ form an inverse semigroup grading of $\mathfrak{C}_c(\G,\LL)$. 
In particular, every element in $\mathfrak{C}_c(\G,\LL)$ is a finite sum $\sum_{U\in F} f_U$ where $f_U \in C_c(U,\LL)$ and $F\subseteq S(\L)$. Any $C^*$-norm $\|\cdot\|$ on $\mathfrak{C}_c(\G,\LL)$ restricted to  $C_c(U,\LL)$ has to be the supremum norm $\|f\|_{\infty}=\max_{x\in U}|f(x)|$, $f\in C_c(U,\LL)$, because $f^* * f \in C_c(X)$, and therefore we have the bound
\begin{equation}\label{eq:boundness_condition_for_norm}
    \|\sum_{U\in F} f_U \|\leq \sum_{U \in F}\|f_U \|_{\infty}.
\end{equation}
For general submultiplicative norms there is no automatic boundedness condition and we need to choose one. 
We choose to build condition \eqref{eq:boundness_condition_for_norm} into our definition of the \emph{universal twisted groupoid Banach algebra} $F(\G,\LL)$ and we define it to be the completion of $\mathfrak{C}_c(\G,\LL)$ in the maximal submultiplicative norm $\|\cdot\|_{\max}$ whose restriction to $C_c(U,\LL)$, $U\in S(\L)$, coincides with the supremum norm $\|\cdot\|_{\infty}$. 
Another natural norm satisfying \eqref{eq:boundness_condition_for_norm}, considered by a number of authors, is Hahn's \emph{$I$-norm} given by $\|f\|_{I} := \max\{\|f\|_{d_*},\|f\|_{r_*}\}$, where 
\[
    \| f \|_{d_*} := \max_{x\in X}\sum_{d(\gamma)=x} |f(\gamma)| \quad \text{ and } \quad \|f\|_{r_*} := \max_{x\in X}\sum_{r(\gamma)=x} |f(\gamma)| , \qquad f\in \mathfrak{C}_c(\G,\LL) .
\]
%In general, we have $ \|f\|_{C^*\text{max}}\leq \|f\|_{I} \leq \|f\|_{\max}$. 
Both $F(\G,\LL)$ and  $F_I(\G,\LL) := \overline{C_c(U,\LL)}^{\|\cdot\|_{I}}$ are Banach $*$-algebras, and we have contractive $*$-homomorphisms $F(\G,\LL)\to F_I(\G,\LL) \to C^*(\G,\LL)$.
The hermitian structure of $F_I(\G,\LL)$, where $\LL$ comes from a groupoid cocycle, was studied recently in \cite{Austad_Ortega}.
The advantage of $F(\G,\LL)$ over $F_I(\G,\LL)$ is that $F(\G,\LL)$ is the algebra for which our general disintegration theorem works.


\subsection*{Twisted inverse semigroup crossed products}

%An inverse semigroup $S$ is a generalization of a discrete group where we allow many idempotents. 
The construction of the twisted inverse semigroup crossed products naturally generalizes  the one for discrete group actions, but is more subtle due to existance of the lattice of impotents in the inverse semigroup. 
In the realm of $C^*$-algebras this theory was initiated by Sieben~\cite{Sieben}, \cite{Sieben98}, and the correct definition in the twisted case was given by Buss and Exel~\cite{Buss_Exel}. 
We extend this theory to Banach algebras.
The main problem is to find the right notion of a covariant representation. 
Since the category of (real or complex) $C^*$-algebras with $*$-homomorphisms as morphisms is a full subcategory of Banach algebras with contractive homomorphisms as morphisms, we decided to work in the latter category. 
Thus by a \emph{representation of a Banach algebra} $A$ in a Banach algebra $B$ we mean a contractive homomorphism $\pi:A\to B$. 
For an inverse semigroup $S$ a representation is a semigroup homomorphisms $v : S\to B_1$ into contractive elements of $B$. 
Then the operators $\{v_t\}_{t\in S}\subseteq B$  are \emph{partial isometries} in the sense of Mbekhta~\cite{Mbekhta}  (which are usual partial isometries when $B$ is a $C^*$-algebra). 
An \emph{action} of $S$ on $A$ is a semigroup homomorphism $\alpha : S \to \PAut(A)$ into the inverse semigroup of partial automorphisms of $A$. 
Thus $\alpha$ consists of isometric isomorphisms $\alpha_t: I_{t^*}\to I_{t}$, $t\in S$, between (closed two-sided) ideals  of $A$,
 whose composition as partial maps is consistent with the semigroup law from $S$. 
 A \emph{covariant representation} of $\alpha$ in  $B$ can be defined as a pair $(\pi,v)$ where $\pi : A \to B$ and $v:S\to B_{1}$ are representations 
\[
    v_t\pi(a)v_{t^*} = \pi \big( \alpha_t(a) \big) , \qquad v_{tt^*}\pi(a) = \pi(a) , \qquad a\in I_{t^*} ,\ t\in S.
\]
The \emph{Banach algebra crossed product} $A\rtimes_{\alpha} S$ is a Banach algebra generated by the universal covariant representation.
In particular, every covariant representation $(\pi,v)$ in $B$ \emph{integrates} to a representation  $\pi\rtimes v : A\rtimes_{\alpha} S\to B$. 
In order for every representation  $A\rtimes_{\alpha} S \to B$ to be of the form $\pi\rtimes v$  for a unique covariant representation $(\pi,v)$ we need to impose a normalizing condition on $(\pi,v)$. 
To this end we assume that each $I_t$ has a contractive approximate unit $\{\mu_i^t\}_{i}$. 
When each $I_t$ has a unit $1_t$, the \emph{normalizing condition} is $v_{tt^*}=\pi(1_{t^*})$, for $t\in S$.
When $(B,B_*)$ is a dual Banach algebra the natural condition is $v_{tt^*} = B_*\text{-}\lim_{i} \pi(\mu_{i}^t)$, for $t\in S$, where $B_*\text{-}\lim$ indicates the limit in the weak$^*$ topology induced by the predual $B_*$ of $B$. 
This applies to the case when $B = B(E)$ for a reflexive Banach space $E$, and then  the normalizing condition is $v_t E = \overline{\pi(I_{t})E}$, for $t\in S$. 
In general the solution is to consider more general covariant representations where $v : S \to (B'')_1$ takes values in the double dual $B''$ equipped with one of the Arens products.  
We discuss these issues in detail for twisted inverse semigroup actions $(\alpha, u)$, \`{a} la Buss--Exel, in Section~\ref{sec:Inverse_semigroup_crossed_products}.

%
\subsection*{Main results}

As shown in \cite{Buss_Exel2} any  twisted inverse semigroup action $(\alpha, u)$ on the algebra $C_0(X)$ `integrates' to a twisted \'etale groupoid $(\G,\LL)$. 
Moreover, any twisted groupoid $(\G,\LL)$ arises in this way: one may take $S$ to be any wide inverse subsemigroup of $S(\LL)$, which naturally gives an action $\alpha:S\to \PAut(C_0(X))$, and then the twist $\LL$ induces a twist $u$ of $\alpha$. 
Our main result is Theorem~\ref{thm:disintegration}, which gives an isometric isomorphism
\[
    F(\G,\LL)\cong C_0(X)\rtimes_{(\alpha,u)} S
\]
by establishing a bijective correspondence between representations of $F(\G,\LL)$ and normalized covariant representations of $(\alpha,u)$. 
This can be applied to Banach algebras $F_{\EE}(\G,\LL)$ which are universal for representations of $F(\G,\LL)$ from a fixed class $\EE$.
For $p\in [1,\infty]$  we define the \emph{universal twisted groupoid $L^p$-operator algebra} $F^p(\G,\LL)$ as $F_{\EE}(\G,\LL)$, where $\EE$ is the class of all representations of $F(\G,\LL)$ on spaces $L^p(\mu)$ for some measure $\mu$.
This gives a universal model for \emph{the reduced twisted groupoid $L^p$-operator algebra} $F^p_{\red}(\G,\LL)$ given by a concrete representation  which was recently studied in \cite{Hetland_Ortega}. 
For the trivial twist, $F^p(\G,\LL)$ gives the untwisted groupoid $L^p$-operator algebra $F^p(\G)$, which as we prove agrees with the one introduced in \cite{Gardella_Lupini17} in the case $\G$ is a second countable groupoid, $\F = \C$, and $p < \infty$.
As mentioned in \cite{cgt} this is a delicate point, and solves the problem indicated below \cite[Propositon 6.5]{cgt}. It follows from the non-trivial estimates
%For any $p\in [1,\infty]$ we have 
\[
    \| f \|_{L^p} \leq \| f \|_{*d}^{1/p} \, \| f \|_{*r}^{1/q} \leq \|f\|_{I}, \qquad f\in  \mathfrak{C}_c(\G,\LL)
\]
where  $\|\cdot \|_{L^p}$ is the universal norm in $F^p(\G,\LL)$ and $p,q\in [1,\infty]$ satisfy $1/p+1/q=1$
(see Theorem \ref{thm:norm_estimates_L^p}). We also prove that 
$    F^p(\G,\LL)^{\op}\cong F^{q}(\G,\LL^{\op}) ,
$
where $F^p(\G,\LL)^{\op}$ is the opposite algebra to $F^p(\G,\LL)$ and $\LL^{\op}$ is the opposite twist to $\LL$ as defined in \cite{Buss_Sims} (see Proposition~\ref{prop:initial_on_L^p_full}). 
For $p<\infty$, $F^p(\G,\LL)$ is $F_{\EE}(\G,\LL)$ where $\EE$ consists of non-degenerate representations on spaces $L^p(\mu)$ (Corollary~\ref{cor:non-degenaracy_of_universal_norm}).
Also  we may restrict  to localizable measures $\mu$. 
Then following Phillips~\cite{PhLp1} we may define \emph{spatial partial isometries} as certain  \emph{weighted composition operators} on $L^p(\mu)$. 
They form  a natural inverse semigroup $\SPIso(L^p(\mu))$ of partial isometries in $B(L^{p}(\mu))$. 
When $p\neq 2$ we  characterize spatial partial isometries axiomatically by using the very well established notion of an $L^p$-projection, see \cite{BDEGGMM}, or hermitian projections when $\F = \C$, cf. \cite{BD}, \cite{cgt} (see our generalized version of Lamperti's Theorem~\ref{thm:spatial^partial_isometries_description}). 

Assuming that $\F=\C$ we prove%\footnote{This is the only important result that we were not able to establish this for $\F=\R$. The main issue is discussed in Remark~\ref{rem:real_representations^problem}, and its source lies in the lack of a good notion of hermitian operator in the real case.}  
that for $p \in [1,\infty]\setminus\{2\}$ the algebra $F^p(\G,\LL)$ is universal for \emph{spatial covariant representations}, i.e. covariant representations $(\pi,v)$ on $L^{p}(\mu)$ where  $v:S\to \SPIso(L^p(\mu))$ takes values in spatial partial isometries and $\pi:C_0(X)\to B(L^p(\mu))$ acts by multiplication operators (by functions from  $L^{\infty}(\mu)$).
More specifically, for $p \in (1,\infty)\setminus\{2\}$ every non-degenerate representation of $F^p(\G,\LL)$ on $L^{p}(\mu)$ comes from a spatial covariant representation, and for $p=1,\infty$ 
we have $F^p(\G,\LL) = F^p_{\red}(\G,\LL)$, the latter being defined by a spatial covariant representation (see Theorem~\ref{thm:spatial_representations_of_groupoid_algebras}).
This in particular gives a natural bijective correspondence between non-degenerate representations of $F^p(\G,\LL)$ and $F^{p'}(\G,\LL)$ for any $p,p'\in (1,\infty)\setminus\{2\}$, 
which is quite surprising 
as (the example of purely infinite graph algebras   \cite{PhLp2a}, \cite{cortinas_rodrogiez} shows) in general there is no non-zero continuous homomorphism from $F^p(\G,\LL)$ to $F^{p'}(\G,\LL)$.

The hierarchy of the discussed complex Banach algebras is summarized in the diagram below%  Figure~\ref{fig.algebras}
, where
all algebras are completions of $\mathfrak{C}_c(\G,\LL)$, in particular, $F_{d_*}(\G,\LL) := \overline{\mathfrak{C}_c(\G,\LL)}^{\|\cdot\|_{d_*}}$ and 
$ F_{r_*}(\G,\LL) := \overline{\mathfrak{C}_c(\G,\LL)}^{\|\cdot\|_{r_*}}$, the algebras in the middle column are Banach $*$-algebras, the horizontal anti-isomorphisms are given by the involution in $\mathfrak{C}_c(\G,\LL)$, and the downward arrows are representations which are identities on $\mathfrak{C}_c(\G,\LL)$:
%\begin{figure}[h]
$$
\xymatrixcolsep{1pc} \xymatrixrowsep{1pc} 
\xymatrix{
 &  &   F(\G,\LL)  \ar@{->}[d] &  &   
\\
 &  &  \ar@{->}[lldd]  \ar@{->}[ldd] F_I(\G,\LL)   \ar@{->}[ddr]  \ar@{->}[ddrr]&  &  
\\
 &  &  &  &  
\\  
F^1(\G,\LL) \ar@{=}[d] \ar@{.}[r] & F^{p} (\G,\LL) \ar@{->}[d]  \ar@{.}[r]   \ar@/^2pc/[rr]^{\stackrel{\text{anti}}{\cong}} & F^{2} (\G,\LL)= C^*(\G,\LL)   \ar@{.}[r]  & \ar@/_2pc/[ll] F^{q} (\G,\LL) \ar@{->}[d] \ar@{.}[r] & 
 F^\infty(\G,\LL)  \ar@{=}[d] 
\\
F^1_{\red}(\G,\LL) \ar@{.}[r] \ar@{=}[d]& F^{p}_{\red} (\G,\LL)  \ar@{.}[r]   \ar@/_2pc/[rr]  & F^2_{\red} (\G,\LL)= C^*_{\red}(\G,\LL)   \ar@{.}[r]  &  F^{q}_{\red} (\G,\LL)   \ar@/^2pc/[ll]_{ \stackrel{\text{anti}}{\cong}} \ar@{.}[r] & 
 F^\infty_{\red}(\G,\LL) \ar@{=}[d] 
\\
 F_{d^*}(\G,\LL)& &   & & F_{r^*}(\G,\LL) 
%\\
% &  &  \ar@{<-}[lluu]  \ar@{<-}[luu] \mathfrak{C}_0(\G)   \ar@{<-}[uur]  \ar@{<-}[uurr]&  &  &
}
$$
%\caption{}\label{fig.algebras}
%\end{figure}

\subsection*{Examples} 

In Subsections~\ref{subsect:twisted_partial_group_actions}  and \ref{subsect:Deaconu-Renault groupoids} we discuss canonical twisted inverse semigroup actions associated to a twisted partial action $(\theta,u)$ of a discrete group $G$ on a space $X$, and to a twisted Deaconu--Reanult groupoid generated by a partial local homeomorphism $h : \Delta\to X$, respectively. 
This allows us to describe representations of the associated algebras in terms of generators and relations. 
In Subsection~\ref{subsec:Banach_algebras_inverse_semigroups} we introduce universal and tight Banach algebras of an inverse semigroup $S$, which are modelled by the universal and tight groupoids $\G(S)$ and $\G_{T}(S)$. 
The new paradigm that appears here is that for general representations $v : S \to B_1$ into a Banach algebra $B$ we need to impose a condition that certain associated projections are \emph{jointly $\F$-contractive} (see Definitions~\ref{defn:jointly_contractive} and \ref{defn:representation_semigroup_contractive}). 
This condition is automatic when $B$ is a $C^*$-algebra or when $v : S \to \SPIso(L^p(\mu))$ takes values in spatial partial isometries, and therefore has not been visible in previous works.
Finally, in Subsection~\ref{subsect:directed_graphs} we define the Banach algebra $F(Q)$ of a directed graph $Q = (Q^0,Q^1, r_Q, s_Q)$ as the universal algebra for a Banach $Q$-family of partial isometries and projections. 
It is naturally a groupoid Banach algebra and an  inverse semigroup tight Banach algebra. 
Our results imply that non-degenerate representations of $F(Q)$ on $L^p$-spaces, for $p\in (1,\infty)\setminus\{2\}$ and $\F=\C$, are necessarily given by $Q$-families with values in spatial partial isometries $\SPIso(L^p(\mu))$. 
This shows that the use of spatial partial isometries in the definition of the graph $L^p$-operator algebras $F^p(Q)$ in \cite{cortinas_rodrogiez} and \cite{PhLp1} is not an assumption, rather it is forced by the groupoid model.
Specializing to the graphs with a single vertex our theory covers real Cuntz algebras \cite{Schroder} and produces Cuntz-like Banach $*$-algebras closely related to those studied by Daws and Horwath~\cite{Daws_Horwath}. 


%
\subsection*{Acknowledgements}

The first two named authors were supported by the National Science Center (NCN) grant no.~2019/35/B/ST1/02684.
The second named author would like to thank Chris Phillips, Eusebio Gardella and Hannes Thiel for sharing their expert knowledge so generously, during meetings and via e-mail. 



%
%
\section{Preliminaries}
\label{sec:Preliminaries}

%We recall here the fundamental relationship between actions of inverse semigroups on topological spaces and \'etale groupoids.
%Also we discuss here the inverse semigroup of partial isometries and representations of $C_0(X)$ on $L^p$-spaces for $p\in [1,\infty]$ and localizable measures. 
Most of the facts presented here are known to experts (though they have never been juxtaposed  together like this), so we included only sketches of (or coordinates leading to) the arguments,
when we were not able to find a reference in appropriate generality.


%
\subsection{Spatial isometries on $L^p$-spaces}

Throughout we fix $p\in[1,\infty]$ and the field $\F = \R,\C$ of real or complex numbers. 
If $(\Omega,\Sigma, \mu)$ is a measure space we write $L^p(\mu)$ for the corresponding Lebesgue space viewed as a Banach space over $\F$. 
Occasionally, to emphasize which field is used, we will write $L^p(\mu, \F)$. 
Recall that the Banach space $L^p(\mu)$ can be recovered from $\mu : \Sigma \to [0,+\infty]$, or in fact from the measure $[\mu] : [\Sigma] \to  [0,+\infty]$ defined on the Boolean algebra $[\Sigma]$ obtained from $\Sigma$ by identifying sets of measure zero.
Also if $p<\infty$, then $L^p(\mu) \cong L^p(\mu_{f})$ where $\mu_{f}(A) = \sup\{\mu(B) : A\supseteq B\in \Sigma, \mu(B)<\infty \}$ is the \emph{semi-finite} part of $\mu$, and the isomorphism is identity on integrable simple functions. 
Recall that a measure $\mu$ is called \emph{localizable} if it is semi-finite and $[\Sigma]$ is Dedekind complete.
In general, if $\mu$ is semi-finite, then $[\mu]$ extends uniquely to the Dedekind completion of $[\Sigma]$ giving a localizable measure on the completion \cite{Fremlin}. 
Using the Loomis--Sikorski Theorem we may represent this abstract measure by a concrete localizable measure space $(\overline{\Omega},\overline{\Sigma}, \overline{\mu})$ (see \cite{Fremlin}, \cite{BGL}) and again $L^p(\mu)\cong L^p(\overline{\mu})$ by the isometric isomorphism sending characteristic functions to characteristic functions. 
Thus when considering $L^p$-spaces with $p<\infty$ we may always assume the measure is localizable, or even decomposable (strictly localizable), see \cite[Corollary on page 136]{Lacey} or \cite[Theorem 7.17]{BGL}).
For $p = \infty$ the assumption that $\mu$ is localizable is also natural. For instance, $\mu$ is localizable if and only if $L^{\infty}(\mu)$ is canonically isomorphic to the dual of $L^1(\mu)$ if and only if $L^{\infty}(\mu)$ is canonically a von~Neumann algebra acting on $L^2(\mu)$ (and every abelian $W^*$-algebra is of this form), see \cite{BGL} and references therein.

\begin{defn}\label{de:SetIsomorphism}
A \emph{set isomorphism} from a measure space $(\Omega_\mu,\Sigma_\mu, \mu)$ onto a measure space $(\Omega_\nu,\Sigma_\nu, \nu)$ is a map $\Phi : \Sigma_\nu \to \Sigma_\mu$ that descends to an isomorphism  of Boolean algebras $[\Phi] : [\Sigma_{\nu}] \to [\Sigma_{\mu}]$. 
Thus, since the Boolean structure is determined by its preorder, it suffices to assume
\begin{enumerate} 
    \item\label{item:set_automorphism1} $\nu(C\setminus D)=0$ if and only if $\mu(\Phi(C)\setminus \Phi(D)) = 0$ for all $C, D\in \Sigma_{\nu}$;
        \item\label{item:set_automorphism2} for each $C\in \Sigma_{\mu}$ there is $\Phi^*(C)\in \Sigma_{\nu}$ such that $\Phi(\Phi^*(C)) = C$ up to a $\mu$-null set.
\end{enumerate} 
Condition \ref{item:set_automorphism2} gives a map $\Phi^* : \Sigma_{\mu}\to \Sigma_{\nu}$, which is a  set isomorphism that descends to an isomorphism $[\Phi^*] : [\Sigma_{\mu}] \to [\Sigma_{\nu}]$ inverse to $[\Phi]:[\Sigma_{\nu}]\to [\Sigma_{\mu}]$.
\end{defn}


We denote by $L_0(\mu)$ the algebra of $\F$-valued measurable functions identified up to equality $\mu$-almost everywhere. 
A set isomorphism $\Phi : \Sigma_\nu \to \Sigma_\mu$ uniquely defines a linear operator $T_{\Phi} : L_0(\nu) \to L_0(\mu)$ such that 
\[ 
    T_{\Phi} 1_{C} = 1_{\Phi(C)}, \qquad C\in \Sigma_{\nu},
\]
and $T_{\Phi}$ preserves (monotone) limits, cf. \cite[Proposition 5.6]{PhLp1}. 
This is an algebra isomorphism $L_0(\nu) \cong L_0(\mu)$ whose inverse is $T_{\Phi^*}$.
We call $T_{\Phi}$ a (generalized) \emph{composition operator}. 
It preserves the range of functions and in particular it restricts to an isometric isomorphism  $L^{\infty}(\nu) \cong L^{\infty}(\mu)$ \cite[Proposition 7.16]{BGL}. 
However, unless $\Phi$ preserves measures, it does not preserve the $L^p$-spaces for $p < \infty$. 
To force this, we will consider \emph{weighted composition operators} $a T_{\Phi} : L_0(\nu) \to L_0(\mu)$ where $a \in L_0(\mu)$ and $[a T_{\Phi}] \xi (x) := a(x) T_{\Phi}(\xi) (x)$.
Note that $\mu\circ \Phi$ is a measure equivalent to $\nu$. Thus if $\nu$ is localizable then 
the Radon--Nikodym derivative $\frac{\mu\circ \Phi}{\nu}$ exists, and is strictly positive $\nu$-almost everywhere, see \cite[Theorem 2.7]{Gardella_Thiel2}, \cite[Theorem 4.4]{BGL}.
Similar comments apply to the measures $\nu \circ \Phi^*$ and $\mu$.

\begin{lem}\label{le:SpatialIsometries}
Consider localizable measures $\mu$ and $\nu$ and $p\in[1,\infty]$.    
For any set isomorphism $\Phi:\Sigma_\nu\to \Sigma_\mu$ the weighted composition operator 
\[
    U_{\Phi} := \left( \frac{d\nu\circ\Phi^{*}}{d\mu} \right)^{\frac{1}{p}} T_{\Phi}
\]
is an invertible isometry $U_{\Phi} : L^p(\nu) \to L^p(\mu)$.
Moreover, the group $UL^{\infty}(\mu)$ of measurable modulus one functions $\omega \in L_0(\mu)$  embeds into the group of invertible isometries of $L^p(\mu)$ as multiplication operators. 
%\marginpar{\tiny I tried to make it more clear what $\omega$ is.}
Thus for any $\omega\in UL^{\infty}(\mu)$ the weighted composition operator $\omega U_{\Phi} : L^p(\nu) \to L^p(\mu)$ is an invertible isometry.
\end{lem}
\begin{proof}
This is straightforward and known, see, for instance, \cite[Lemmas 3.2, 3.3]{Gardella_Thiel2}. 
\end{proof}

\begin{defn}[\cite{PhLp1}]\label{de:SpatialIsometries}
We call the  isometries $\omega U_{\Phi}$ described in Lemma~\ref{le:SpatialIsometries} %above 
\emph{spatial}.
\end{defn}

\begin{prop}\label{prop:group_of_spatial_isometries} 
We have $(\omega U_{\Phi})^{-1} = T_{\Phi^*}(\overline{\omega}) U_{\Phi^*}$ and $\omega U_{\Phi} \circ  \upsilon  T_{\Psi}= \omega T_{\Phi}(\upsilon) U_{\Phi \circ \Psi}$ for all localizable measures $\mu$, $\nu$, $\eta$, all $p\in[1,\infty]$, set isomorphisms $\Phi : \Sigma_\nu \to \Sigma_\mu$, $\Psi : \Sigma_\nu \to \Sigma_\eta$ and functions $\omega \in UL^{\infty}(\mu)$, $\upsilon \in UL^{\infty}(\nu)$. 

In particular, spatial invertible isometries on $L^p(\mu)$, for localizable $\mu$ and $p \in [1,\infty]$, form a group which is isomorphic to the semi-direct product $UL^{\infty}(\mu) \rtimes \Aut([\Sigma_\mu])$
where $[\Phi] \omega =T_{\Phi}(\omega)$ for $[\Phi]\in \Aut([\Sigma_\mu])$ and $\omega\in UL^{\infty}(\mu)$.
\end{prop}
\begin{proof}
The first part follows from (the calculations in the proof of) \cite[Lemmas 3.3, 3.4]{Gardella_Thiel2}. It readily implies the second part of the assertion,
see also \cite[Theorem 3.7]{Gardella_Thiel2}.
\end{proof}

\begin{thm}[Banach--Lamperti theorem]\label{thm:Banach_Lamperti} 
Let $p \in [1,\infty] \setminus \{ 2 \}$ and $\mu , \nu$ be localizable measures. 
Then every invertible isometry from $L^p(\mu)$ to $L^p(\nu)$ is spatial.
\end{thm}
\begin{proof}  
For $p \in (1,\infty)\setminus \{2\}$, $\F = \C$, and $\mu = \nu$ this is proved in \cite[Theorem 3.7]{Gardella_Thiel2}, but the proof works for two localizable measures $\mu$ and $\nu$, and also for $\F=\R$ and $p=1$. 
In fact since $L^{\infty}(\mu) \cong L_1(\nu)'$ the case $p=1$ can be deduced from the case $p=\infty$. 
The assertion for $p=\infty$ follows from the Banach--Stone Theorem combined with the Gelfand Theorem. Indeed,  we have $L^{\infty}(\mu)\cong C(X)$ and $L^{\infty}(\nu)\cong C(Y)$ where $X$, $Y$ are compact Hausdorff (hyperstonean) spaces, and every invertible isometry $C(Y)\to C(X)$ is of the form $a T$ where $a\in C(X)$, $|a|=1$, and $T:C(Y)\to C(X)$ is an isometric algebra isomorphism. 
The Boolean algebras of idempotent elements in $C(Y)$ and $C(X)$ are canonically isomorphic to $[\Sigma_{\nu}]$ and $[\Sigma_{\mu}]$, respectively. 
Thus $T$ induces a set isomorphism $\Phi : \Sigma_\nu\to \Sigma_\mu$ such that the action of
$T_{\Phi}:L^\infty(\nu)\to L^\infty(\mu)$ on idempotents in $L^\infty(\nu)$ agrees with the action of $T$ on the corresponding idempotents in $C(Y)$.
As the idempotents are linearly dense in the considered  spaces we conclude that the isometry $L^{\infty}(\nu)\to L^{\infty}(\mu)$ corresponding to $a T$ is $\omega T_{\Phi}$ where 
$\omega\in L^{\infty}(\mu)$ is the function corresponding to $a\in C(X)\cong L^{\infty}(\mu)$. % under the isomorphism $L^{\infty}(\mu)\cong C(X)$.
\end{proof}


%
\subsection{$L^p$-operator algebras and representations of $C_0(X)$}
Throughout this paper $A$ and $B$ will be  Banach algebras, and $B(E)$ will denote the Banach algebra of bounded operators on a Banach space $E$.
\begin{defn}\label{de:Representations}
A \emph{representation of $A$ in  $B$} is a contractive algebra homomorphism $\pi : A \to B$. 
%We say $\pi$ is \emph{non-degenerate} if the closed linear span $\overline{\pi(A)B}$ is $B$. 
If $B = B(E)$ for some Banach space $E$, we call $\pi:A\to B(E)$  a \emph{representation on the space $E$}, and we say $\pi$ is \emph{non-dedegenerate}  if $\overline{\pi(A)E} = E$. 
\end{defn}

%\begin{rem} 
%Note that $\overline{\pi(A)B(E)} = B(E)$ implies $\overline{\pi(A)E}=E$, but not conversely, so the algebraic non-degeneracy is stronger than the spatial one. 
%Also if $A$ has a left approximate unit then by the Cohen--Hewitt factorization theorem it suffices to consider sets of products rather than closed linear spans.
%\end{rem}

\begin{defn}[\cite{Phillips}] \label{def:L^p_operator_algebra}
Let $p\in[1,\infty]$. 
A Banach algebra $A$ is an \emph{$L^p$-operator algebra} if there is an isometric representation $\pi : A \to B(L^p(\mu))$ for some measure $\mu$.
\end{defn}
By an \emph{approximate unit} in a Banach algebra we will always mean a contractive two-sided approximate unit. 
For an  $L^p$-operator algebra $A$,  $p\in (1,\infty)$,  with an approximate unit there is an isometric non-degenerate representation  $\pi : A \to B(L^p(\mu))$,   \cite[Theorem 4.3]{Gardella_Thiel1}.
Unlike for $C^*$-algebras there is still no axiomatic definiton of an $L^p$-operator algebra.
\begin{defn}\label{def:C*-algebra}
A \emph{$C^*$-algebra} is a Banach $*$-algebra $A$ such that $\| a^*a \| = \| a \|^2$ ($a\in A$), and if $\F=\R$ we also require $\| a^*a \| \leq \| a^*a + b^*b \|$ for all $a,b\in A$. 
We call a norm with these properties a \emph{$C^*$-norm}.
\end{defn}

\begin{rem}\label{rem:complex_the_real_C*}
It is well known that a Banach $*$-algebra $A$ (complex or real) is a $C^*$-algebra if and only if it is isometrically $*$-isomorphic to a $*$-subalgebra of bounded operators on some $L^2$-space (complex or real).
Also a real Banach algebra $A$ is a real $C^*$-algebra if and only if its complexification $A_{\C} := A + i A$ admits a (necessarily unique) $C^*$-norm that extends that of $A$, see \cite{Li_book}.
A homomorphism $A\to B$ between two $C^*$-algebras is contractive if and only if it is $*$-preserving 
(these well known facts for complex $C^*$-algebras \cite[A.5.8]{BM} also hold in the real case \cite[Proposition 5.4.1 and Theorem 5.6.4]{Li_book}). 
\end{rem}

Let $X$ be a locally compact Hausdorff space. 
The algebra $C_0(X)$ of continuous functions on $X$ which vanish at infinity, equipped with the supremum norm $\|\cdot\|_{\infty}$, is an $L^p$-operator algebra for any $p\in [1,\infty]$. 
By \cite[Theorem 5.3]{Gardella_Thiel1} a complex $C^*$-algebra $A$ is an $L^p$-operator algebra for $p\in[1,\infty)\setminus\{2\}$ if and only if it is commutative and hence of the form $C_0(X)$.
A real commutative Banach algebra $A$ is isometrically isomorphic to $C_0(X)$ for some locally compact Hausdorff space if and only if $\| a \|^2 = \| a^2 \| \leq \| a^2 + b^2 \|$ for all $a,b\in A$, see \cite{Arens}. 

\begin{prop}[\cite{Kaplansky}, see also {\cite[Theorems 9, 10]{Bns}}] \label{prop:minimality_of_sup_norm}
The supremum norm on $C_0(X)$ is minimal among all submultiplicative norms on $C_0(X)$.
\end{prop}

\begin{cor}\label{cor:minimality_of_sup_norm}
Every injective representation $\pi : C_0(X) \to B$ in a Banach algebra $B$ is automatically isometric. More generally, every representation $\pi : C_0(X) \to B$ descends to an isometric representation $\pi:C_0(X)/\ker(\pi)\to B$. 
\end{cor}
\begin{proof}
The first part follows from Proposition~\ref{prop:minimality_of_sup_norm}. 
It implies the second part, because the quotient $C_0(X)/\ker(\pi)$ is isometrically isomorphic to $C_0(Y)$ where $Y := \{ y\in X : f(y)=0 \text{ for all $f\in \ker(\pi)$} \}$. % is a closed subset of $X$. 
\end{proof}

\begin{rem}
Proposition \ref{prop:minimality_of_sup_norm} and Corollary \ref{cor:minimality_of_sup_norm} hold for arbitrary commutative real $C^*$-algebras (not necessarily of the form $C_0(X)$), see \cite[Lemma 5.1.8, Proposition 5.4.1]{Li_book}. 
\end{rem}

We now slightly extend \cite[Proposition 2.7]{cgt}, see also \cite[Proposition 2.12]{BP} and references therein, which describes hermitian operators acting on complex $L^p$-spaces.
This forces the representations of the complex algebra $C_0(X)$ on $L^p$ and $C_0$-spaces to act by multiplication operators.

\begin{prop}\label{prop:hermitian_operators} 
Assume $\F=\C$  and let $E = L^p(\mu)$ for a localizable $\mu$ and $p \in [1,\infty] \setminus \{ 2 \}$, or $E = C_0(\Omega)$ for a locally compact Hausdorff space $\Omega$.  
An element $a\in B(E)$ is hermitian (i.e. $\| e^{ita} \| = 1$ for all $t\in \R$) if and only if
$a$ is a multiplication operator by a bounded real-valued function (from $L^{\infty}(\mu,\R)$ if $E=L^p(\mu)$ and from $C_{b}(\Omega,\R)$ if $E=C_0(\Omega)$).
%Similarly, if  is , then an element in $\in B(C_0(\Omega))$ is hermitian
%if and only if  it is a multiplication operator by a real valued function from $C_0{\infty}(\mu)$.
\end{prop}
\begin{proof}
When $E = L^p(\mu)$ and $p < \infty$ this is proved in \cite[Proposition 2.7]{cgt}. 
The proof relies on the Banach--Lamperti Theorem and it also works for $p = \infty$, with the Banach--Lamperti Theorem extended as in Theorem~\ref{thm:Banach_Lamperti}. 
In fact it also works for $E = C_0(\Omega)$ with the Banach--Lamperti Theorem replaced by the Banach--Stone Theorem. % Since we could not locate a reference
We recall the argument for the `only if' part (the `if' part is straightforward).
Let $a \in B(C_0(\Omega))$ be hermitian. We may assume that $\| a \| \leq \frac{\pi}{2}$. 
The elements $u_t := e^{ita}$, $t\in \R$, form a continuous group of invertible isometries. 
By the Banach--Stone Theorem for each $t\in \R$ there is $\omega_{t} \in C_b(\Omega)$ with $|\omega_t| = 1$ and a homeomorphism $\varphi_t : \Omega \to \Omega$ such that $[u_{t}\xi](x) = \omega_t(x) \xi(\varphi_t(x))$. 
Using this, a simple calculation,  for $s,t\in \R$, gives
\[
    \| u_t - u_s \| = \max \big\{ \| \omega_t - \omega_s \|_{\infty}, 2(1-\delta_{\varphi_{t} , \varphi_s}) \big\}.
\]
Since $t \mapsto u_t$ is norm continuous and $u_0 = 1$ is the identity operator, we get that $\varphi_t = \id$ for all $t\in\R$. 
Hence each $u_t$ is the operator of multiplication by $\omega_t$. 
Since $\| a \| \leq \frac{\pi}{2}$ the spectrum of $u_1 = e^{ia}$ is contained in the half circle $\{ e^{it} : t\in [-\pi/2,\pi/2] \}$ where the holomorphic inverse $\log$ of $\exp$ is defined. 
Applying analytic functional calculus to $u_1$ we get $i a = \log(u_1)$. 
Since $m: C_b(\Omega) \to B(C_0(\Omega))) ;\ m(h)\xi (x) := h(x)\xi(x)$ is a unital homomorphism, we obtain $a = -i \log(u_1) = -i \log(m(\omega_1))= m(-i \log(\omega_1))$.
Thus $a$ is a multiplication operator by $-i \log(\omega_1)$, and since $a$ is bounded and hermitian we must have $-i \log(\omega_1)\in C_b(\Omega,\R)$.
\end{proof}

\begin{thm} \label{thm:L^p_representations of C_0(X)} 
Assume $\F=\C$ and let $E$ be as in Proposition~\ref{prop:hermitian_operators}. 
Any non-degenerate representation $\pi:C_0(X) \to B(E)$  is given by mutliplication operators. 
That is, there is a representation $\pi_0 : C_0(X)\to L^{\infty}(\mu)$ if $E = L^p(\mu)$ (respectively $\pi_0 : C_0(X) \to C_b (\Omega)$ if $E = C_0(\Omega)$), such that 
\begin{equation}\label{eq:representation_by_multiplication_operators}
    [\pi(a)\xi] (x) = \pi_0(a)(x) \xi (x), \qquad a\in C_0(X) ,\ \xi \in E.
\end{equation}
\end{thm}
\begin{proof} 
If $X$ is not compact we may extend $\pi : C_0(X) \to B(E)$ to a unital representation $\pi:C(X^+) \to B(E)$, where $X^+$ is the one-point compactification of $X$, see \cite[Theorem 4.1]{Gardella_Thiel1}. 
Then $\pi$ maps hermitian elements to hermitian ones by \cite[Lemma 2.4]{cgt}.
Hermitian elements in $C(X^+)$ are real valued functions $C(X^+,\R)$. 
Hence $\pi(C(X^+,\R))$ consists of multiplication operators by Proposition~\ref{prop:hermitian_operators}.
Since $C(X^+) = C(X^+,\R) + i C(X^+,\R)$, we conclude that there is $\pi_0 : C(X^+) \to L^{\infty}(\mu)$ (or $\pi_0 : C(X^+) \to C_{b}(X)$) satisfying \eqref{eq:representation_by_multiplication_operators}. 
%Since $\pi$ is a unital representation it follows that $\pi_0$ is a unital representation.
\end{proof}

\begin{rem}\label{rem:real_representations^problem}
We conjecture that an analogue of Theorem~\ref{thm:L^p_representations of C_0(X)} holds in the real case as well. 
If $\F = \R$ and $\pi : C_0(X)\to B(L^p(\mu))$ is a non-degenerate representation, we may pass to complexifications $C_0(X,\C)$ and $L^p(\mu,\C)$ of $C_0(X,\R)$ and $L^p(\mu,\R)$ respectively. 
Then 
\[
    \big( \pi_\C (a + i b) \big) (\xi +i \eta) := \pi(a)\xi - \pi(b)\eta + i \big( \pi(a)\eta +\pi(b) \xi \big), \qquad \xi, \eta\in L^p(\mu, \R), \ a, b\in C_0(X,\R) ,
\]
defines a non-degenerate algebra homomorphism $\pi_\C : C_0(X,\C) \to B(L^p(\mu,\C))$. 
Moreover, $\| \pi_{\C}(a) \| = \| \pi(a) \|$ for all $a\in C_0(X,\R)$ by \cite{Holtz_Karow}.
So $\pi_\C$ is contractive on every ray $e^{it}C_0(X,\R) \subseteq C_0(X,\C)$, $t\in \R$, but we do not know if it is necessarily contractive on the whole of $C_0(X,\C)$. 
Contractivenes of $\pi_\C$ on $C_0(X,\C)$ is equivalent to our conjecture.
\end{rem}

\begin{cor} \label{cor:degeneracy_of_L_infty_representations}
The complex Banach algebra $C_0(X)$ admits an isometric non-degenerate representation on $L^{\infty}(\mu)$ for localizable $\mu$ if and only if $X$ is compact.
\end{cor}
\begin{proof}
If $X$ is not compact and $\pi:C_0(X) \to B(L^{\infty}(\mu))$ is isometric and non-degenerate then, in the notation of Theorem~\ref{thm:L^p_representations of C_0(X)}, $\pi_0(C_0(X))$ is a non-unital subalgebra of $L^{\infty}(\mu)$. 
Hence $\overline{\pi(C_0(X))L^{\infty}(\mu)} = \overline{\pi_0(C_0(X))L^{\infty}(\mu)}$ is a non-trivial  (non-unital) ideal in $L^{\infty}(\mu)$,
 which contradicts the non-degeneracy of $\pi$. If $X$ is compact, then the embedding   $\pi_0:C(X)\to\ell^{\infty}(X)$
gives an isometric unital representation of $C(X)$ on $\ell^{\infty}(X)$.
\end{proof}

\begin{rem}\label{rem:C_0_and_Lindenstrauss} 
Corollary~\ref{cor:degeneracy_of_L_infty_representations} is perhaps one of the reasons why Phillips~\cite[Page 3]{Phillips} suggests that instead of $L^\infty$-algebras ``it may well be more appropriate to consider'' algebras isometrically represented on \emph{$C_0$-spaces}, i.e. spaces $C_0(\Omega)$ for a locally compact Hausdorff space $\Omega$. In fact, it is also natural to consider the even more general class of   \emph{Lindenstrauss spaces}, i.e.\ Banach spaces whose duals are isometrically isomorphic to $L^1$-spaces, see Corollary~\ref{cor:L_infty_lindenstrauss_spaces_etc} below. 
\end{rem}


%
\subsection{Inverse semigroups and \'etale groupoids}

Let $S$ be a semigroup. 
An element $t\in S$ is called \emph{partially invertible} if there is  an element $t^*\in S$ such that $t = t t^* t$ and $t^* = t^* t t^*$. 
Then $t^*t$ and $tt^*$ are idempotents, and we call $t^*$ a \emph{generalized inverse} for $t$. 
The semigroup $S$ is called an \emph{inverse semigroup} if every element in $S$ has a unique generalized inverse. 
Assume this. Then the  map $t \mapsto t^*$ is an anti-multiplicative involution, and any semigroup homomorphism between two inverse semigroups is automatically $*$-preserving. The set of idempotents
\[
    E(S) := \{ e \in S : e^2 = e \} = \{ tt^* : t \in S \} = \{ t^*t : t \in S \}
\]
is an abelian semigroup (see for instance \cite[Proposition 2.1.1]{Paterson}). 
A partial order on $S$ is defined by $s \le t$ if and only if $s = t s^* s$, if and only if there is $e \in E(S)$ with $s = t e$, $s,t \in S$,  see \cite{Paterson}. 
If $S$ has a unit $1$, then $e\in E(S)$ if and only if $e \le 1$. 

A \emph{partial bijection} on a set $X$ is a bijection $h : U \to h(U)$ between subsets $U$, $h(U)$ of $X$. % Partial bijections $\text{PBij}(X)$ form an inverse supsemigroup of  $\text{PBij}(X)$.
The set $\PBij(X)$ of all partial bijections on $X$ form an inverse semigroup with composition of $h : U \to h(U)$ and $f : V \to f(V)$ defined as the bijection $h\circ f : f^{-1}(U\cap f(V)) \to h(U\cap f(V))$. 
Then $h^* = h^{-1} : h(U) \to U$,  $h\in E(\PBij(X))$ if and only if $h = \id|_U$ for some $U \subseteq X$, and $h \leq f$ if and only if $f$ is an extension of $h$. 
Every inverse semigroup $S$ can be viewed as a subsemigroup of $\PBij(S)$, where $h_t : S_{t^*} \to S_{t}$ is given by $h_t(s) := ts$ and $S_{t} := \{ s\in S : ss^* \leq t^*t \}$ for all $t\in S$, see \cite[Proposition 2.1.3]{Paterson}.
A \emph{partial homeomorphism} on a topological space $X$ is a homeomorphism $h : U \to h(U)$ between two open subsets $U$ and $h(U)$ of $X$.
Partial homeomorphisms form an inverse subsemigroup of $\PBij(X)$ that we denote by $\PHomeo(X)$.

\begin{defn}\label{defn:inverse_semigroup_action}
An \emph{action of an inverse semigroup $S$ on a topological space $X$} is a non-degenerate semigroup homomorphism $h : S \to \PHomeo(X)$: a family  of partial homeomorphisms $h_t : X_{t^*} \to X_{t}$ such that $h_t \circ h_s = h_{t s}$ for all $s,t\in S$, and  $\bigcup_{t\in S} X_s = X$ (if $S$ is unital the latter says that the homomorphism $h$ is unital, i.e. $h_1 = \id|_X$).
\end{defn}

A groupoid  is a small category $\G$ in which every arrow is invertible. 
If $\G$ (identified with the set of arrows) is equipped with a topology  that makes taking the range, domain, composition, and inverse of arrows  continuous, we call $\G$ a \emph{topological groupoid}. 
We denote by $X \subseteq \G$ the subset of identity arrows (identified with the objects of $\G$). We call $X$ the \emph{unit space} of $\G$. 
%For an introduction to \'etale groupoids see \cite{Sims}, \cite{Exel}, \cite{Paterson}.
Let $\G$ be an \emph{\'etale groupoid} with unit space $X$, i.e. $\G$ is a topological groupoid where the range and domain maps $r,d : \G \to X \subseteq \G$ are local homeomorphisms. 
In particular, $X$ is an open subset of $\G$. 
We will  assume that $X$ is a locally compact Hausdorff space, in which case $\G$ is necessarily  locally compact and locally Hausdorff, but $\G$ is Hausdorff if and only if $X$ is closed in $\G$,  see, for instance, \cite[Lemma 2.3.2]{Sims}.
A \emph{bisection} of a groupoid $\G$ is an open set $U \subseteq \G$ such that the restrictions $r|_U$ and $d|_{U}$ are homeomorphisms onto open sets of $\G$. 
So a topological groupoid is \'etale if and only if it can be covered by bisections. 

\begin{ex}[Inverse semigroup actions from \'etale groupoids]\label{ex:actions_from_groupoids}
The set of bisections $\Bis(\G)$ of an \'etale groupoid $\G$ forms a unital inverse semigroup with multiplication and involution given by 
\[
    U \cdot V := \{ \gamma \eta : \text{$\gamma \in U,\ \eta\in V$ are composable} \},  \qquad  
    U^* = U^{-1} := \{ \gamma^{-1} : \gamma \in U \} .
\]
Then $X$ is a unit of $\Bis(\G)$. 
For each $U\in \Bis(\G)$ we have a homeomorphism 
\[
    h_U := r \circ d|_{U}^{-1} : d(U) \to r(U) ,
\]  
and these homeomorphisms define a \emph{canonical action} $h : \Bis(\G) \to \PHomeo(X)$ on the unit space $X$ \cite[Proposition 5.3]{Exel}. 
In fact we may view this action as a restriction of a partial action $\tilde{h} : \Bis(\G) \to \PHomeo(\G)$ on the groupoid $\G$, where $\tilde{h}_{U} : r^{-1}(d(U)) \to r^{-1}(r(U))$ is given by $\tilde{h}_U(\gamma)=d|_{U}^{-1}(r(\gamma))\gamma$. 
In other words, $\tilde{h}_U(\gamma) = \eta\gamma$, where $\eta$ is the unique element in $U$ that can be composed with $\gamma$. 
Using this interpretation one  sees that $\tilde{h}$ is indeed an inverse semigroup action. It restricts to the canonical action $h$ in the sense that  $h= r \circ\tilde{h}|_{X}$.
\end{ex}

%Every inverse semigroup action $h : S \to \PHomeo(X)$ factors through the canonical action $\Bis(\G) \to \PHomeo(X)$ for some \'etale groupoid $\G$. 

\begin{ex}[\'Etale groupoids from inverse semigroup actions] \label{ex:groupoids_from_actions}
For any inverse semigroup action $h : S \to\PHomeo(X)$ the \emph{transformation groupoid} $S \ltimes_{h} X$ is defined as follows (see \cite[p.\ 140]{Paterson} or \cite[Section~4]{Exel}).
The arrows of $S\ltimes_{h} X$ are equivalence classes of pairs $(t,x)$ for $x\in X_{t^*} \subseteq X$; two pairs $(t,x)$ and $(t',x')$ are equivalent if $x = x'$ and there is $v\in S$ with $v\le t, t'$ and $x \in X_{v^*}$.
The range and domain maps $r,d: S\ltimes_{h} X \rightrightarrows X$ and the multiplication are defined by $r([t,x]) := h_t(x)$, $d([t,x]) := x$, and $[s,h_t(x)] \cdot [t,x] = [s\cdot t,x]$ for $x \in X_{(st)^*}$. 
We give $S\ltimes_{h} X$ the unique topology such that the sets $U_t := \{ [t,x] : x \in X_{t^*} \}$ are open bisections of $S\ltimes_{h} X$. 
Then $S\ltimes_{h} X$ is an \'etale groupoid and the map $S\ni t \mapsto U_t \in \Bis(S\ltimes_{h} X)$ is a semigroup homomorphism. 
Composing it with the canonical homomorphism $\Bis(S\ltimes_{h} X) \to \PHomeo(X)$ we recover the action $h$. 
Composing it with the canonical homomorphism $\Bis(S\ltimes_{h} X) \to \PHomeo(S\ltimes_{h} X)$ we
get an extension $\tilde{h}$ of $h$ where 
$
    \tilde{h}_{t} : \{ [s,x] : x\in h_{s}^{-1}(X_{t^*}) \} \to \{ [s,x] : x \in h_{s}^{-1}(X_{t}) \}$ is given by $\tilde{h}_t([s,x]) := [ts,x].
$
In this way every inverse semigroup action can be viewed as coming from an \'etale groupoid, and then it can be extended to an action on this groupoid.
Conversely, for any \'etale groupoid $\G$ and any inverse subsemigroup $S\subseteq \text{Bis}(\G)$, acting canonically on $X$, we have a continous open groupoid homomorphism $S\ltimes_{h} X \ni [U,x] \mapsto (d|_U)^{-1}(x) \in \G$. 
This is an isomorphism $S\ltimes_{h} X\cong \G$ if and only if $S$ is \emph{wide} in the sense that $S$ covers $G$ and $U\cap V$ is a union of bisections in $S$ for all $U,V\in S$, see \cite[Proposition~5.4]{Exel}, \cite[Proposition 2.2]{Kwa-Meyer}. In particular, every \'etale groupoid is a transformation groupoid.
\end{ex}


%
\subsection{$L^p$-partial isometries on Banach spaces and spatial partial isometries} 
\label{subsect:Partial isometries}

Partial isometries acting on arbitrary Banach spaces were  defined by Mbekhta~\cite[Definition 4.1]{Mbekhta}, see also \cite[8.1]{cgt},
and in  Banach algebras the definition can be phrased as follows.
\begin{defn}\label{defn:partial_isometries}
A \emph{partial isometry in a Banach algebra} $B$  is partially invertible element $t$ in the semigroup $B_1$ of contractive elements, so $t \in B_1$  and there exists $t^* \in B_1$ such that $t = t t^* t$ and $t^* = t^* t t^*$. 
If $B$ is unital and complex we say that $t \in B$ is an \emph{$MP$-partial isometry} if
$t$ is contractive and admits a contractive Moore--Penrose generalized inverse $t^*\in B$, i.e.\ $t^*$ is a generalised inverse of $t$ and the idempotents $t t^*$ and $t^* t$ are hermitian.
\end{defn}

If $B$ is a $C^*$-algebra (real or complex) then a contractive generalized inverse to a partial isometry $t\in B$ is necessarily the adjoint element $t^*\in B$, see \cite[3.1, 3.3]{Mbekhta} and Remark~\ref{rem:complex_the_real_C*}, so in this case both notions in Definition~\ref{defn:partial_isometries} coincide with the usual notion of partial isometry in a $C^*$-algebra. 
%(A partial isometry $t\in B(H)$ on a Hilbert space $H$ is an operator whose restriction to the orthogonal complement of kernel $\ker(t)^\bot$ is an isometry onto its range $tH$.)
However, for many Banach spaces $E$, including $L^1(\mu)$, $c_0$, $\ell^{\infty}$, a partial isometry $t\in B(E)$ may have many contractive generalized inverses. 
On the other hand, if the Moore--Penrose generalized inverse exists it is necessarily unique. 
So the $MP$-partial isometries have uniquely determined generalized inverse $MP$-partial isometries.  

Partial isometries  usually  do not form a semigroup (the product of two partial isometries $s,t\in B$ in a $C^*$-algebra $B$ is a partial isometry if and only if the projections $s^*s$ and $tt^*$ commute, see \cite{Halmos}).
But for $p\in[1,\infty]\setminus \{2\}$ and any Banach space $E$ we introduce a natural inverse semigroup of partial isometries in $B(E)$. 
Recall that an \emph{$L^p$-projection on $E$} is an idempotent $P\in B(E)$ such that for all $\xi \in E$ we have $\|\xi\|^p = \| P\xi \|^p + \| (1-P)\xi \|^p$ when $p<\infty$ and $\|\xi\|=\max\{\|P\xi\|,\|(1-P)\xi\|\}$ when $p=\infty$, see \cite{BDEGGMM}, \cite{Agniel}.  
The $L^2$-projections on a Hilbert space are simply the orthogonal projections. 
If $p\neq 2$, then all $L^p$-projections on $E$ commute and they form a Boolean algebra $\mathbb{P}_{p}(E)\subseteq B(E)$. 
The Banach algebra $C_p(E) := \clsp \mathbb{P}_{p}(E)$  generated by $\mathbb{P}_{p}(E)$ is called the \emph{$p$-Cunningham algebra} of $E$, and we have $C_p(E) \cong C_0(X)$ where $X$ is a totally disconnected space (the Stone dual to $\mathbb{P}_{p}(E)$).
We extend the notion of $L^p$-projection to partial isometries as follows:
\begin{defn} 
An \emph{$L^p$-partial isometry} on a Banach space $E$, $p\in [1,\infty]$, is a contraction $T\in B(E)$ that has a contractive generalized inverse $T^*\in B(E)$ such that $T T^*$ and $T^* T$ are $L^p$-projections. 
We denote by $L^p\mhyphen\PIso(E)$ the set of all $L^p$-partial isometries on $E$.
\end{defn}

\begin{prop}
For $p\in [1,\infty]\setminus \{2\}$ the set $L^p\mhyphen\PIso(E)$ is an inverse semigroup whose idempotents are $\mathbb{P}_{p}(E)$.
\end{prop}
\begin{proof} 
Let  $S, T\in L^p\mhyphen\PIso(E)$ and choose the corresponding generalized inverses  $S^*, T^* \in L^p\mhyphen\PIso(E)$, so that the corresponding idempotents $TT^*$, $T^*T$, $SS^*$, $SS^*$ are in  $\mathbb{P}_{p}(E)$. 
By commutativity of these idempotents, $S^*T^*$ is a generalized inverse of $TS$.
Using the $L^p$-projection property of $SS^*$, $T^*T$, $TT^*$ and that $T^*$ is an isometry on the range of $T$ (and vice versa), for any $x\in E$, we get
\[
\begin{split}
    \|TSS^*T^*x\|^p &= \| T^*TSS^*T^*x \|^p = \|SS^*(T^*T)T^* x \|^p = \|SS^*T^*x\|^p \\
        &= \|T^*x\|^p - \|(1-SS^*)T^*x\|^p = \|TT^*x\|^p-\|T^*x - SS^* (T^*T)T^*x\|^p \\
        &= \|x\|^p - \|(1-TT^*)x\|^p - \|TT^*x - T SS^* T^*x\|^p \\
        &= \|x\|^p - \|(1-TSS^*T^*)x\|^p.
\end{split}
\]
Hence $TSS^*T^*\in \mathbb{P}_{p}(E)$. 
By symmetry we also get $S^*T^*TS\in \mathbb{P}_{p}(E)$. 
Hence $TS\in L^p\textup{-PIso}(E)$, and so $L^p\mhyphen\PIso(E)$ is a semigroup. 
Assume in addition that $T$ is the generalized inverse of $S$. 
Using that $S^*S$ is an $L^p$-projection, on one hand we have
\[
    \| TSx \|^p = \| S^*S TSx \|^p + \| (1-S^*S)TSx \|^p = \|S^*Sx\|^p + \|TSx-S^*S x\|^p .
\]
On the other hand, $\|TSx\|=\|STSx\|=\|Sx\|=\|S^*Sx\|$. 
Hence $\|TSx-S^*S x\|^p=0$ and so $TS=S^*S$. 
Similarly one shows that $ST = SS^*$. 
This implies that $T=S^*$, (cf. \cite[Propositon 2.4]{kwa-leb}), and proves the uniqueness of the generalized inverse in $L^p\mhyphen\PIso(E)$. 
%Thus $L^p\mhyphen\PIso(E)$ is an inverse semigroup.
\end{proof}

If $E=L^p(\mu)$ for a localizable measure $\mu$ and $p \in [1,\infty]\setminus \{2\}$, then $P\in B(L^p(\mu))$ is an $L^p$-projection if and only if $P$ is an operator of multiplication by a function a characteristic function, see \cite[Proposition 4.9]{BDEGGMM}
(when $\F=\C$ this can be inferred from Proposition~\ref{prop:hermitian_operators}).
In particular, we have an isomorphism of Boolean algebras $\mathbb{P}_{p}(L^p(\mu))\cong [\Sigma_{\mu}]$. 
We now discuss the relationship between $L^p\mhyphen\PIso(L^p(\mu))$ and spatial partial isometries (for $\sigma$-finite measures) introduced by Phillips~\cite{PhLp1}, \cite{Phillips}.

%\begin{ex}\label{ex:inverse_semigroup_representaion} 
%Let $p\in [1,\infty]$. 
%We may embed any inverse semigroup $S$ into $L^p\mhyphen\PIso(\ell^p(S)) \subseteq B(\ell^p(S))$ as an inverse semigroup of partial isometries.
%In fact, for any inverse semigroup action $h : S \to \PBij(X)$ we have a semigroup homomomorphism $v : S \to B(\ell^p(X))$ given by
%\[
%    v_t\xi(x) := \begin{cases} \xi \big( h_{t}^{-1}(x) \big), & x\in X_{t}, \\ 0 , & x \not\in X_{t}, \end{cases} , \qquad \xi\in \ell^p(X),\ x\in X, \ t\in S, 
%\]
%which attains values in partial isometries on $\ell^p(X)$. 
%Note that $v_t v_{t^*} =v_{tt^*}$ is the multiplication operator by the characteristic function $1_{X_t}$, and $v$ is injective if and only if $h$ is injective.
%\end{ex}
 
\begin{defn}

A \emph{subspace} of a measure space $(\Omega,\Sigma_{\mu}, \mu)$ is $(D,\Sigma_D, \mu|_D)$ where $D\in \Sigma$, $\Sigma_{D} := \{ C\cap D : C\in \Sigma\}$ and $\mu|_D := \mu|_{\Sigma_D}$. 
A \emph{partial set automorphism} of $(\Omega,\Sigma, \mu)$ is a set isomorphism between two subspaces of $(\Omega,\Sigma, \mu)$. 
So a partial set automorphism is a  map $\Phi : \Sigma_{D_{\Phi^*}} \to \Sigma_{D_{\Phi}}$ ($D_{\Phi},D_{\Phi^*}\in \Sigma$) that descends to a Boolean isomorphism  $[\Phi] : [\Sigma_{D_{\Phi^*}}] \to [\Sigma_{D_{\Phi}}]$. 
We define $\PAut([\Sigma_{\mu}]) := \{ [\Phi] : \text{$\Phi$ is a partial set automorphism} \}$ for the set of partial automorphisms of the Boolean algebra $[\Sigma_{\mu}]$. 
\end{defn}

By definition $\PAut([\Sigma_{\mu}])$ is the set of isomorphisms between ideals in the Boolean algebra $[\Sigma_{\mu}]$. It forms an inverse semigroup that can be identified with an inverse semigroup of operators on $L_0(\mu)$.
Indeed, a partial automorphism $\Phi$ defines the composition operator isomorphism $T_{\Phi} : L_0(\mu|_{D_{\Phi^*}}) \to  L_0(\mu|_{D_{\Phi}})$ whose inverse is given by $T_{\Phi^*}$. 
Identifying, $L_0(\mu|_{D_{\Phi^*}})$ and $L_0(\mu|_{D_{\Phi}})$ with subspaces of $L_0(\mu)$ we see that $\Phi$ determines uniquely a linear operator $T_{\Phi} : L_0(\mu)\to L_0(\mu)$ such that  $T_{\Phi}$ preserves (monotone) limits and
\[
    T_{\Phi} 1_{A} = 1_{\Phi(C\cap D_{\Phi^*})}, \qquad C \in \Sigma.
\]
We still call $T_{\Phi} : L_0(\mu) \to L_0(\mu)$ a (generalized) \emph{composition operator}.
For any two partial automorphisms $\Phi : \Sigma_{D_{\Phi^*}}\to \Sigma_{D_{\Phi}}$ and 
$\Psi : \Sigma_{D_{\Psi^*}} \to \Sigma_{D_{\Psi}}$ we have $T_{\Phi}\circ T_{\Psi} = T_{\Phi\circ \Psi}$, 
where $\Phi\circ \Psi : \Sigma_{\Psi^*(D_{\Phi^*} \cap D_{\Psi})}\to \Sigma_{\Phi(D_{\Phi^*} \cap D_{\Psi})}$ is a well defined partial automorphism of $(\Omega,\Sigma, \mu)$.
Also $T_{\Phi^*}$ is the unique generalized inverse for $T_{\Phi}$ among the composition operators.
In particular, composition operators on $L_0(\mu)$ form an inverse semigroup naturally isomorphic to $\PAut([\Sigma_{\mu}])$.
By the very definition of the partial automorphism $\Phi : \Sigma_{D_{\Phi^*}} \to \Sigma_{D_{\Phi}}$, the map $\mu\circ \Phi:\Sigma_{D_{\Phi^*}}\to [0,+\infty]$ is a measure on $D_{\Phi^*}$ equivalent to $\mu|_{D_{\Phi^*}}$. 
Thus we may consider the Radon--Nikodym derivative $\frac{d\mu\circ\Phi}{d\mu|_{D_{\Phi^*}}}$ as a function on $\Omega$ which is zero outside $D_{\Phi^*}$. Similar remarks concern the inverse partial automorphism $\Phi^*:\Sigma_{D_{\Phi}}\to \Sigma_{D_{\Phi^*}}$.

\begin{prop}\label{prop:spatial^partial_isometries}
Let $p\in [1,\infty]$ and $\mu$ be localizable. 
For any $[\Phi]\in \PAut([\Sigma_{\mu}])$ the weighted composition operator $U_{\Phi}   := \left( \frac{d\mu\circ\Phi^{*}}{d\mu|_{D_{\Phi}}} \right)^{\frac{1}{p}}T_{\Phi}$ is a well defined partial isometry on $L^p(\mu)$, and the map $\PAut([\Sigma_{\mu}])\ni [\Phi] \mapsto U_{\Phi}\in B(L^p(\mu))$ 
is a semigroup embedding. 
The semigroup generated by $U_{\Phi}\in B(L^p(\mu))$, $[\Phi] \in \PAut([\Sigma_{\mu}])$,  and  $UL^{\infty}(\mu)\subseteq B(L^p(\mu))$ is an inverse semigroup of partial isometries of the form
\begin{equation}\label{eq:spatial^partial_isometry}
    \omega U_{\Phi}  := \omega \left( \frac{d\mu\circ\Phi^{*}}{d\mu|_{D_{\Phi}}} \right)^{\frac{1}{p}}T_{\Phi}
\end{equation}
where $\Phi : \Sigma_{D_{\Phi^*}} \to \Sigma_{D_{\Phi}}$ is a partial set automorphism and  $\omega : D_{\Phi}\to \{ z\in \F: |z|=1 \}$ is measurable. 
The operator $\omega U_{\Phi}$ determines $\Phi$ and $\omega$ uniquely up to sets of measure zero. 
Moreover, 
\begin{equation}\label{eq:spatial^partial_isometries_relations}
    (\omega U_{\Phi})^* = T_{\Phi^*}(\overline{\omega}) U_{\Phi^*}, \qquad (\omega U_{\Phi})\circ  (\upsilon U_{\Psi})=\omega T_{\Phi}(\upsilon) U_{\Phi\circ\Psi} .
\end{equation}
%In particular, the map $\PSA(\mu)\ni T_{\Phi}\mapsto T_{1,\Phi}=\left(\frac{d\mu\circ\Phi^{*}}{d\mu|_{D_{\Phi^*}}}\right)^{\frac{1}{p}}T_{\Phi}\in \SPIso(L^p(\mu))$ is an injective 
%inverse semigroup homomorphism.
\end{prop}
\begin{proof}
This follows from  \cite[Lemma 6.12, 6.17]{PhLp1} where $\sigma$-finite measures were considered, but the arguments work for localizable measures.
\end{proof}

\begin{defn}\label{def:spatial^partial_isos}
We denote by $\SPIso(L^p(\mu))$ the inverse subsemigroup of operators \eqref{eq:spatial^partial_isometry} and call them \emph{spatial partial isometries} on $L^p(\mu)$ \cite[Definition 6.4]{PhLp1}. 
\end{defn}

\begin{rem}
The  semilattice of idempotents in $\SPIso(L^p(\mu))$ is isomorphic to $[\Sigma_{\mu}]$, as it consists of multiplication operators of idempotent elements in $L^{\infty}(\mu)$.
The group of invertible elements in $\SPIso(L^p(\mu))$ is isomorphic to $UL^{\infty}(\mu) \rtimes \Aut([\Sigma_\mu])$, see Proposition~\ref{prop:group_of_spatial_isometries}. 
\end{rem}

\begin{thm}[Banach--Lamperti Theorem for partial isometries] \label{thm:spatial^partial_isometries_description}
Let $p\in [1,\infty]\setminus \{2\}$. 
A contraction $T \in  B(L^p(\mu))$ is a spatial partial isometry if and only if it has a contractive generalized inverse $T^*\in B(L^p(\mu))$ such that both $TT^*$ and $T^*T$ are multiplication operators on $L^p(\mu)$.
Thus spatial partial isometries coincide with $L^p$-partial isometries on $L^p(\mu)$: 
\[
    \SPIso(L^p(\mu))= L^p\mhyphen\PIso(L^p(\mu)) , 
\]
and in the complex case also with $MP$-partial isometries in $B(L^p(\mu))$. 
\end{thm}
\begin{proof} 
Using \eqref{eq:spatial^partial_isometries_relations} we see that 
\[
    (\omega U_{\Phi})^* \omega U_{\Phi} = T_{\Phi^*}(\overline{\omega}) U_{\Phi^*} \circ \omega U_{\Phi}= U_{\Phi^*\circ\Phi}=U_{\id_{\Sigma_{D_{\Phi^*}}}} = 1_{D_{\Phi^*}}
\]
is the operator of multiplication by the characteristic function of the domain $D_{\Phi^*}$ of $\Phi$.  
Conversely, if $T\in  B(L^p(\mu))$ is a partial isometry with a  contractive generalized inverse $T^*\in B(L^p(\mu))$ such that both $T^*T$  and $TT^*$ are multiplication operators on $L^p(\mu)$, say by functions $1_{D^{*}}$ and $1_{D}$ respectively, then $T$ restricts to an invertible isometry  $T : L^p(\mu|_{D^*}) \to L^p(\mu|_{D})$. 
Hence $T$ is of the form \eqref{eq:spatial^partial_isometry} by Theorem~\ref{thm:Banach_Lamperti}. 
In particular, $\SPIso(L^p(\mu)) = L^p\mhyphen\PIso(L^p(\mu))$. 
If $\F=\C$ and $T\in B(L^p(\mu))$ is an $MP$-partial isometry and $T^*$ is its Moore--Penrose generalized inverse, then the idempotents $T^*T$ and $T^*T$ are  operators of multiplication by characteristic functions of some measurable sets, by Proposition~\ref{prop:hermitian_operators}.
%Hence $T\in \SPIso(L^p(\mu))= L^p\mhyphen\PIso(L^p(\mu))$. 
\end{proof}



%
%
\section{Twisted inverse semigroup actions and their crossed products}
\label{sec:Inverse_semigroup_crossed_products}


We introduce Banach algebra crossed products by twisted inverse semigroup actions, that generalize $C^*$-algebraic constructions introduced by Buss and Exel in~\cite{Buss_Exel}.
%We fix an inverse semigroup $S$. 
%In this section we generalize the consturction of define the crossed products for actions of $S$ on  Banach algebras that generalize the $C^*$-algebraic inverse semigroup crossed products from \cite{Sieben}
%In the following sections we will be interested only in actions on the commutative algebra $A=C_0(X)$, as in Example~\ref{ex:commutative_example} below.

%
\subsection{The crossed product}

By an \emph{ideal} in a Banach algebra we  always mean a closed two-sided ideal. 
A \emph{partial automorphism} on a Banach algebra $A$ is an isometric isomorphism $\alpha : I \to J$ between two ideals $I,J$ of $A$. 
The partial automorphisms on $A$ form an inverse subsemigroup of $\PBij(A)$ that we denote by $\PAut(A)$. 

\begin{defn} \label{defn:inverse_semigroup_action_on_Banach_algebra} 
An \emph{action of $S$ on a Banach algebra $A$} is a unital semigroup homomorphism $\alpha : S \to \PAut(A)$.
Thus an action of $S$ on $A$ consists of (closed, two-sided) ideals $I_t$ of $A$ and isometric isomorphisms $\alpha_t : I_{t^*} \to I_{t}$ for all $t\in S$, such that $\alpha_1 = \id_A$ and $\alpha_s \circ \alpha_t = \alpha_{s t}$ for all $s,t\in S$. 
\end{defn}

\begin{rem} 
If $A$ is a $C^*$-algebra then all ideals $I_t$ are closed under involution, and all isometric isomorphisms $\alpha_t: I_{t^*}\to I_{t}$ are necessarily $*$-preserving. 
Hence Definition~\ref{defn:inverse_semigroup_action_on_Banach_algebra} is consistent with the well established $C^*$-algebraic version \cite{Sieben}.
\end{rem}

\begin{ex} \label{ex:commutative_example}
If $A=C_0(X)$ for a locally compact Hausdorff space $X$ then $\PAut(A) \cong \PHomeo(X)$. 
Indeed, every ideal $I$ in $A$ is of the form $I = C_0(U)$ for an open set $U \subseteq X$ (we treat elements in $C_0(U)$ as functions in $C_0(X)$ that vanish outside $U$), and any partial automorphism $\alpha : I \to J$ is given by the composition $\alpha(a) = a \circ h^{-1}$ with a partial homeomorphism $h : U \to V$, where $I = C_0(U)$ and $J = C_0(V)$.
Thus inverse semigroup actions on the Banach algebra $C_0(X)$ are equivalent to inverse semigroup actions on the topological space $X$.
\end{ex}

Twisted inverse semigroup actions on $C^*$-algebras were first introduced by Sieben~\cite{Sieben98}, but a more general definition given by Buss and Exel in \cite{Buss_Exel}
  is needed for a number of problems.
To generalize this to Banach algebras it seems reasonable to assume that all ideals involved in an action have \emph{(contractive, two-sided) approximate units}. 
This in particular guarantees that the corresponding multiplier algebras have good properties and various approaches to multipliers coincide. 
We refer to \cite{Dales}, see also \cite{Daws} and references therein, for more details. 
Following \cite[Section 2]{Daws}, for a Banach algebra $A$ we define the \emph{multiplier algebra} as 
$$
    \Mult(A) := \big\{ (L,R) : \text{$L , R : A \to A$ satisfy $a L(b) = R(a) b$ for all $a,b\in A$} \big\} .
$$
Then $\Mult(A)$ is automatically a unital closed subalgebra of the $\ell^\infty$-direct sum $B(A)\oplus B(A)^{\op}$, so in particular it is a Banach algebra with the norm $\|(L,R)\| = \max\{\|L\|,\|R\|\}$. 
Also $\Mult(A)$ is complete in the strict topology, which is defined by the seminorms $(L,R) \mapsto \|L(a)\| + \|R(a)\|$ ($a\in A$). 
Every $a\in A$ yields a multiplier $(L_a,R_a)$ where $L_a(b) := ab$, $R_a(b) := ba$, and if $A$ has a contractive approximate unit the resulting  map is isometric and allows us to treat $A$ as an ideal of $\Mult(A)$. 
Then any isometric automorphism $\alpha:A\to A$ extends uniquely to an isometric automorphism $\overline{\alpha}:\Mult(A)\to \Mult(A)$. 
We denote by 
\[
    \UMult(A) := \{u\in \Mult(A): \text{$u$ is invertible and $\|u\| = \| u^{-1} \| =1$} \} ,
\] 
the group of invertible isometries in $\Mult(A)$.

\begin{defn}[{\cite[Definition 4.1]{Buss_Exel}}] \label{defn:twisted actions}  
A \emph{twisted action} of an inverse semigroup $S$ on a Banach algebra $A$ is a pair $(\alpha,u)$, where $\alpha = \{\alpha_t\}_{t\in S}$ is a family of partial automorphisms $\alpha_t : I_{t^*} \to I_{t}$ of $A$, such that each ideal $I_{t}$ contains an approximate unit and their union is linearly dense in $A$, 
and $u=\{u(s, t)\}_{s,t\in S}$ is a family of unitary multipliers $u(s, t)\in \UMult(I_{s t})$, $t,s \in S$, 
such that the following holds for all $r,s,t\in S$ and $e,f\in E(S)$:
\begin{enumerate}[label={(A\arabic*)}]
    \item\label{enu:twisted actions1} $\alpha_s\circ \alpha_t = \Ad_{u(s,t)}\alpha_{s t}$;
    \item\label{enu:twisted actions2} $\alpha_r \big( a u(s,t) \big) u(r,s t) = \alpha_r(a) u(r,s) u(rs,t)$ for $a\in I_{r^*} \cap I_{s t}$; 
    \item\label{enu:twisted actions3} $u(e,f) = 1_{e f}$ and $u(t,t^*t) = u(t t^*,t) = 1_t$, where $1_t$ is the unit of $\Mult(I_t)$; 
    \item\label{enu:twisted actions4} $u(t^*,e) u(t^*e,t) a = u(t^*,t) a$ for all \(a\in I_{t^* e t}\).
  \end{enumerate}
\end{defn}

\begin{rem}\label{rem:twisted_relations} 
All properties in \cite[Lemma~4.6]{Buss_Exel} hold for the maps and ideals involved in a twisted action on a Banach algebra as defined above. 
In particular, for all $s,t\in S$ and $e\in E(S)$ we have
\[
    I_t = I_{tt^*}, \quad \overline{\alpha}_t \big( u(t^*,t) \big) = u(t,t^*), \quad \alpha_e = \id|_{I_e}, \quad \alpha_s (I_{s^*}\cap I_t) = I_{st} ,
\] 
and $I_{s}\subseteq I_{t}$ whenever $s\leq t$. 
We will use these relations, often without warning.
\end{rem}

The authors of \cite{Sieben}, \cite{Sieben98}, \cite{Buss_Exel} consider covariant representations on Hilbert spaces. 
We propose a more general definition of representations in Banach algebras. 
Our definition works best when the target algebra has a predual, which we explain in detail in   Subsection~\ref{defn:covariant_representation_on_space} below.
Therefore in general we will use the \emph{double dual} $B''$ of a Banach algebra $B$ (which has a predual $B'$ by construction). 
Recall that $B''$ is again naturally a Banach algebra with either of the Arens products, in which $B$ sits as a $B'$-weakly dense Banach algebra. 
We will consider $B''$ equipped with the \emph{second Arens product}, which for $a,b \in B''$ is determined by the formula $a \cdot b := B'\mhyphen\lim_{\beta} B'\mhyphen\lim_{\alpha} a_\alpha b_{\beta}$
where $\{a_\alpha\}_{\alpha}$, $\{b_\beta\}_{\beta}\subseteq B$ are nets that are convergent to $a$ and $b$ respectively, in the weak$^*$ topology on $B''$ induced by $B'$. 
The first Arens product is given by $a \square b :=  B'\mhyphen\lim_{\alpha} B'\mhyphen\lim_{\beta}  a_\alpha b_{\beta}$, and $B$ is called \emph{Arens regular} if the two products coincide, 
see \cite{Dales} for more details. 
Our decision to use the second Arens product is arbitrary, and in any case we will be primarily interested in products $a v\in B$ where $a\in B$ and $v\in B''$, so that the first and second Arens  products always agree. 
Also recall that every $C^*$-algebra (complex or real) is Arens regular. 

For any representation $\pi : A \to B$ the double adjoint $\pi'' : A'' \to B''$ is a  $B'$-weakly continuous representation that extends $\pi$, and $\pi''$ is isometric whenever $\pi$ is.
In particular, for every Banach subalgebra $A\subseteq B$ we may identify $A''$ with a subalgebra of $B''$.
Also any contractive  approximate unit  $\{\mu_{i}\}$ in $A$ converges $A'$-weakly  to a left identity $1_A$ in $A''$ (and a right identity for the first Arens product in $A''$) and the map $(L,R)\mapsto L''(1_A)$ is an isometric embedding allowing us to assume that $\Mult(A) \subseteq \{ b \in A'' : \text{$ba, ab \in A$ for all $a\in A$} \} \subseteq A''$, see \cite[Proposition 2.9.16 and Theorem 2.9.49]{Dales}.
In particular, for each ideal $I$ in $A$ with an approximate unit, we will adopt the identifications $I\subseteq \Mult(I)\subseteq I''\subseteq A''$. 
  
\begin{defn}\label{defn:covariant_representation_in_algebra}
Let $(\alpha,u)$ be a twisted action of $S$  on $A$. 
A \emph{covariant representation  of $(\alpha,u)$ in a Banach algebra $B$} is a pair $(\pi,v)$, where $\pi : A \to B$ is a representation and $v : S \to (B'')_{1}$ is a map into contractive elements in $B''$,  such that: 
\begin{enumerate}[label={(CR\arabic*)}]
    \item\label{item:covariant_representation1} $v_t \pi(a) = \pi(\alpha_{t}(a)) v_t \in B$ for all $a\in I_{t^*}$;
    \item\label{item:covariant_representation2} $\pi(a)v_s v_t =\pi(au(s,t)) v_{st}$ for all $a\in I_{st}$, $s,t\in S$;
    \item\label{item:covariant_representation3} $\pi(a)v_e = \pi(a)$ for all $a\in I_{e}$, $e\in E(S)$.
\end{enumerate} 
We call $B(\pi,v) := \clsp\{ \pi(a_t)v_t : a_t\in I_{t}, \, t\in S \}$ the \emph{range} of $(\pi,v)$.
We say that $(\pi,v)$ is injective, isometric, etc. if $\pi$ has that property. 
We also say that $(\pi,v)$ is \emph{$B'$-normalized} if 
\begin{enumerate}[label={(CR\arabic*)}] \setcounter{enumi}{3}
    \item\label{item:covariant_representation4} $v_{t}= B'\mhyphen\lim_{i} (\pi(\mu_{i}^t)v_{t})$, for all $t\in S$, where $\{\mu^t_i\}_{i}$ is an approximate unit in $I_t=I_{tt^*}$.
\end{enumerate}
\end{defn}

\begin{rem}\label{rem:general_covariant_rep}
Condition~\ref{item:covariant_representation1} is equivalent to $\pi(a)v_t=v_t\pi(\alpha_{t}^{-1}(a))\in B$ for all $a\in I_{t}$, $t\in S$. 
It implies that each $v_t$ can be viewed as a ``partial mutliplier'' of $B$, and  $B(\pi,v)\subseteq B$.
Conditions~\ref{item:covariant_representation1} and \ref{item:covariant_representation3} imply that $v_{e}\pi(a)=\pi(a)$ for all $a\in I_{e}$, $e\in E(S)$.
Employing also \ref{item:covariant_representation2} we get 
\begin{equation}\label{eq:CovarianceConditionAlternate}
    v_t\pi(a)v_{t^*} = \pi \big( \alpha_t(a) u(t,t^*) \big) 
\end{equation}
for all $a\in I_{t^*}$, $t\in S$. 
Also, \ref{item:covariant_representation3} and \ref{item:covariant_representation4} imply  that $v_{e}=\pi''(1_e)$ for every $e\in E(S)$. 
%In fact, \ref{item:covariant_representation4} says that $v_t=v_t\pi''(1_{t^*})$
\end{rem} 

Notice that the elements $v_t$ of Definition~\ref{defn:covariant_representation_in_algebra} are not required to be partial isometries. The definition of the generalised inverse $v_t^*$ of $v_t$ in the proof below involves both $v_{t^*}$ and the twist $u$, so the covariance conditions \eqref{eq:CovarianceConditionAlternate} above and \ref{item:normalized_covariant_representation1} below are different. 

\begin{prop}\label{prop:normalized_twisted_rep}
A pair $(\pi,v)$ is a $B'$-normalized covariant representation of $(\alpha,u)$ if and only if $\pi : A \to B$ is a representation and $v : S\to (B'')_{1}$ takes values in partial isometries such that every $v_t$, $t\in S$, admits a contractive generalised inverse $v_t^*$, satisfying $\pi(a) v_t \in B$, for $a\in I_t$, $t\in S$, and
\begin{enumerate}
    \item\label{item:normalized_covariant_representation1} $v_t \pi(a) v_t^* = \pi(\alpha_{t}(a))$ for all $a\in I_{t^*}$, $t\in S$;
    \item\label{item:normalized_covariant_representation2} $v_s v_t = \pi''(u(s,t))v_{st}$ for all  $s,t\in S$;
    \item\label{item:normalized_covariant_representation3} $v_t v_t^* = \pi''(1_{t})$ and $v_t^* v_t  = \pi''(1_{t^*})$ for all $t\in S$.
\end{enumerate}
Moreover, for any covariant representation $(\pi,v)$ of $(\alpha,u)$ in $B$ there is a unique $B'$-normalized covariant representation $(\pi,\tilde{v})$ such that $\pi(a)v_t=\pi(a)\tilde{v}_{t}$ for all $a\in I_{t}$ and $t\in S$, namely $\tilde{v}_{t} = B'\mhyphen\lim_{i} (\pi(\mu_{i}^t)v_{t}) = v_t \pi''(1_{t^*})$, $t\in S$. 
\end{prop}
\begin{proof} 
Assume $(\pi,v)$ is a pair with the properties described in the assertion. 
For every $t\in S$ and $a\in I_{t^*}$, using \ref{item:normalized_covariant_representation3} and \ref{item:normalized_covariant_representation1} we have 
\[
    v_t \pi(a) = v_t \pi(a) \pi''(1_{t^*}) = v_t \pi(a) v_{t}^*v_t = \pi \big( \alpha_t(a) \big) v_t  ,
\] 
which is \ref{item:covariant_representation1}. 
Condition~\ref{item:normalized_covariant_representation2} immediately gives \ref{item:covariant_representation2}.
By \ref{item:normalized_covariant_representation3} we have $v_t = v_t v_t^*v_t = \pi''(1_{t}) v_t = v_t\pi''(1_{t^*})$. 
This implies \ref{item:covariant_representation4} and that for $e\in E(S)$ we get $v_e v_e= \pi''(u(e,e))v_e=\pi''(1_e)v_e=v_e$. 
Therefore $\pi''(1_e)= v_e^*v_e=v_e^*v_e v_e=\pi''(1_e) v_e=v_e$, 
and using this one gets \ref{item:covariant_representation3}.
Hence $(\pi,v)$ is a $B'$-normalized covariant representation of $(\alpha,u)$.

For the converse let $(\pi,v)$ be any $B'$-normalized covariant representation. 
For each $t\in S$ choose an approximate unit $\{\mu^t_i\}_{i}$ in $I_t = I_{tt^*}$. 
Using \ref{item:covariant_representation1} and $B'$-continuity of multiplication in $B''$ in the second variable we get
\[
    B' \mhyphen\lim_{i} \big( \pi(\mu^t_i) v_t \big) = B'\mhyphen\lim_{i} \big( v_t \pi(\alpha_{t}^{-1}(\mu^t_i)) \big) = v_t B'\mhyphen\lim_{i} \pi \big( \alpha_{t}^{-1}(\mu^t_i) \big) = v_{t}\pi''(1_{t^*}),
\]
because $\{\alpha_{t}^{-1}(\mu^t_i)\}_{i}$ is an approximate unit in $I_{t^*}$ which is $B'$-convergent to $1_{t^*}$. 
Thus $v_t = B'\mhyphen\lim_{i} (\pi(\mu^t_i) v_t) = v_{t} \pi''(1_{t^*})$. 
Also \ref{item:covariant_representation3} and \ref{item:covariant_representation4} imply that $v_e = \pi''(1_e)$ for all $e\in E(S)$. 
We put $v_t^*:=v_{t^*}\pi''(u(t,t^*)^{-1})$. % Our definition is dictated by our Disintegration Theorem.
Using \ref{item:covariant_representation1}, \ref{enu:twisted actions1} and \ref{item:covariant_representation2}, and being careful 
to use only $B'$-continuity of multiplication in $B''$ in the second variable, we get
\[
\begin{split}
    v_t v_t^* &= B'\mhyphen\lim_{i} v_t v_{t^*}\pi \big( \mu^{t}_i u(t,t^*)^{-1} \big)\\
        &= B'\mhyphen\lim_{i} \pi \big( \alpha_t \circ \alpha_{t^*} \big( \mu^t_i u(t , t^*)^{-1} \big) \big) v_t v_{t^*} \\
        &=  B'\mhyphen\lim_{i} \pi \big( u(t,t^*)  \mu^t_i u(t , t^*)^{-1} u(t , t^*)^{-1} \big) v_t v_{t^*} \\
        &= B'\mhyphen\lim_{i} \pi \big( u(t,t^*) \mu^t_i u(t , t^*)^{-1} u(t , t^*)^{-1} u(t , t^*) \big) v_{t t^*} \\
        &= B'\mhyphen\lim_{i} \pi''\big( u(t , t^*) \mu^t_i u(t , t^*)^{-1} \big) v_{t t^*}= B'\mhyphen\lim_{i} v_{t t^*} \pi''\big( u(t , t^*) \mu^t_i u(t , t^*)^{-1} \big)\\
				&=v_{t t^*}\pi''(1_{t}) = \pi''(1_{t}).
\end{split}
\]
By Remark~\ref{rem:twisted_relations} we have $\overline{\alpha}_{t^*}(u(t,t^*)^{-1}) = u(t^*,t)^{-1}$, so a similar calculation gives
\[
\begin{split}
    v_t^* v_t &= B'\mhyphen\lim_{i} v_{t^*} \pi \big(  u(t , t^*)^{-1} \mu^{t}_i\big) v_t 
        = B'\mhyphen\lim_{i} \pi \big(  u(t^* , t)^{-1} \alpha_{t^*}(\mu^t_i) \big) v_{t^*} v_t \\
        &= B'\mhyphen\lim_{i} \pi \big( u(t^* , t)^{-1} \alpha_{t^*}(\mu^t_i) u(t^* , t) \big) v_{t^* t} 
				= B'\mhyphen\lim_{i} v_{t^* t}  \pi \big(u(t^* , t)^{-1}  \alpha_{t^*}(\mu^t_i) u(t^* , t) \big) \\
        &=v_{t^* t}\pi''(1_{t^*})=  \pi''(1_{t^*}) .
\end{split}
\]
This finishes the proof of \ref{item:normalized_covariant_representation3} and shows that $v_t$ and $v_t^*$ are mutual generalized inverses. 
For $a \in I_{t^*}$ we get 
\[
\begin{split}
    v_t \pi(a) v_t^* &= v_t \pi(a) v_{t^*} \pi'' \big( u(t,t^*)^{-1} \big) = \pi \big( \alpha_t(a) \big) v_{t}v_{t^*} \pi'' \big( u(t,t^*)^{-1} \big) \\
        &= \pi \big( \alpha_t(a) \big) \pi'' \big( u(t,t^*) \big) v_{tt^*} \pi''(u(t,t^*)^{-1}) \\
        &= \pi \big( \alpha_t(a) \big) \pi'' \big( u(t,t^*) u(t,t^*)^{-1} \big) = \pi \big( \alpha_t(a) \big) ,
\end{split}
\]
which is \ref{item:normalized_covariant_representation1}.
Furthermore, $\alpha_s''(1_{s^*}1_{t})=1_{st}$ because $\alpha_s(I_{s^*}\cap I_t)=I_{st}$ and therefore
\[
    v_s v_t = v_s \pi''(1_{s^*}) \pi''(1_{t}) v_t = \pi''(1_{st}) v_s v_t = \pi''\big( u(s,t) \big)v_{st},
\]
which proves \ref{item:normalized_covariant_representation2}. 
Hence $(\pi,v)$ satisfies all properties in the assertion.

Now let $(\pi,v)$ be any covariant representation of $(\alpha,u)$ in $B$. 
As above we see that the limit $\tilde{v}_{t} := B'\mhyphen\lim_{i} ( \pi ( \mu_{i}^t ) v_{t} ) = v_t \pi''( 1_{t^*} )$ exists and does not depend on the choice of an approximate unit $\{\mu^t_i\}_{i}$ in $I_t$ ($t\in S$). 
Clearly if $(\pi,\tilde{v})$ is a $B'$-normalized covariant representation satisfying $\pi(a) v_t = \pi(a) \tilde{v}_{t}$ for all $a\in I_{t}$ and $t\in S$, then each $\tilde{v}_{t}$ must be given by such a limit, giving \ref{item:covariant_representation4}.
%Clearly  $\pi(a) v_t=B'\mhyphen\lim_{i} \psi(a\mu_i^{t})\tilde{v}_t =\pi(a)\tilde{v}_t$  for  $a\in I_{t^*}$. Thus 
For $a\in I_{t^*}$ we get  
\[
    \tilde{v}_t \pi(a) = v_t \pi''(1_{t^*}) \pi(a) = v_t\pi(a) = \pi \big( \alpha_{t}(a) \big) v_t= \lim_{i} \pi \big( \alpha_{t}(a) \big) \pi(\mu_i^{t}) v_{t} = \pi \big( \alpha_{t}(a) \big) \tilde{v}_t.
\]
This shows that $(\pi,\tilde{v})$ satisfies \ref{item:covariant_representation1} and  $\pi(a)v_t=\pi(a)\tilde{v}_{t}$ for all $a\in I_{t}$ and $t\in S$.
Using this we get $\pi(a)\tilde{v}_e =\pi(a)v_e = \pi(a)$ for all $a\in I_{e}$, $e\in E(S)$, so \ref{item:covariant_representation3} holds. 
Finally, for all $s,t\in S$ and $a\in I_{st}$ the calculation %the first equality uses \ref{item:covariant_representation1} and that $\pi(b) \tilde{v}_t = \pi(b) v_t$
\[
    \pi(a) \tilde{v}_s \tilde{v}_t = v_s \pi \big( \alpha_{s^*}(a) \big) \tilde{v}_t = \pi(a)v_s v_t = \pi \big( a u(s,t) \big) v_{st} = \pi \big( a u(s,t) \big) \tilde{v}_{st} 
\]
gives \ref{item:covariant_representation2}. 
Thus $(\pi,\tilde{v})$ is a $B'$-normalized covariant representation of $(\alpha,u)$.
\end{proof}

\begin{rem}\label{rem:unital_actions}
 If  each $I_t$, $t\in S$, is unital then a covariant representation $(\pi,v)$ is $B'$-normalized if and only if $v_t = \pi(1_t)v_t \in B$ for each $t\in S$, and Proposition~\ref{prop:normalized_twisted_rep} 
could be formulated without the use of the bidual algebra $B''$ and the extended representation $\pi''$. 
\end{rem}


\begin{lem}\label{lem:range_of_covariant_rep}
Let $(\pi,v)$ be a covariant representation of $\alpha$ in a Banach algebra $B$.
The range $B(\pi,v)$ is a Banach subalgebra of $B$. 
The spaces  $A_t = \{ \pi(a_t)v_t: a_t\in I_{t} \}$, $t\in S$, form a grading of $B(\pi,v)$ over the inverse semigroup $S$ in the sense that 
\[
    B(\pi,v) = \clsp \{ A_t : t\in S \}, \qquad A_s A_t \subseteq A_{st}, \qquad \text{$s\leq t$ implies $A_s\subseteq A_t$} ,
\]
for all $s,t \in S$. 
In fact, we have
\begin{enumerate}
    \item\label{enu:range_of_covariant_rep1} $\pi(a_s)v_s \cdot \pi(a_t)v_t = \pi \big( \alpha_s(\alpha_{s}^{-1}(a_s)a_t) u(s,t) \big)v_{st}$ and $\alpha_s \big( \alpha_{s}^{-1} (a_s) a_t \big) \in I_{st}$ for all $a_t\in I_{t}, a_s\in I_{s}$, $s,t\in S$;
    \item\label{enu:range_of_covariant_rep2} $s\leq t$ implies $I_{s}\subseteq I_{t}$ and $\pi(a)v_s = \pi \big( a u(ss^*,t)^{-1} \big) v_t$ for any $a \in I_{s}$;
%    \item\label{enu:range_of_covariant_rep3} if $B$ is a $C^*$-algebra, then $(\pi(a_t)v_t)^*=\pi(\alpha_{t^*}(a_{t}))^*v_{t^*}$  for  $a_t\in I_{t^*}$, $t\in S$.
\end{enumerate}
If in addition $A$ and $B$ are $C^*$-algebras then $(\pi(a_t) v_t)^* = \pi \big( \alpha_{t}^{-1}(a_{t}^*) u(t^*,t)^{-1} \big)v_{t^*}$ for all $a_t\in I_{t}$, $t\in S$, so $A_t^* = A_{t^*}$, and $\{A_t\}_{t\in S}$ is a saturated grading of $B(\pi,v)$ over the inverse semigroup of $S$ in the sense of \cite[Definition 6.15]{Kwa-Meyer0}, see also \cite[Definition 7.1]{Exel}.
%In particular, $B(\pi,v)$ 
%is a Banach algebra and spaces $A_t=\{\pi(a_t)v_t: a_t\in C_0(X_{t^*}), \, t\in S\} $is graded by the preoredered 
\end{lem}
\begin{proof} 
\ref{enu:range_of_covariant_rep1} For any 
$a_t\in I_{t}, a_s\in I_{s}$, $s,t\in S$, we have $\alpha_{s}^{-1}(a_s) a_t\in I_{s^*} I_{t}\subseteq I_{s^*}\cap I_{t}$ and so
$\alpha_{s}(\alpha_{s}^{-1}(a_s)a_t)\in I_{st}$ because $\alpha_s(I_{s^*}\cap I_{t})=I_{st}$, see Remark \ref{rem:twisted_relations}.
Using \ref{item:covariant_representation1} we get
\[
    \pi(a_s)v_s \cdot \pi(a_t)v_t = v_s \pi \big( \alpha_{s}^{-1}(a_s) \big) \cdot \pi(a_t) v_t = v_s \pi \big( \alpha_{s}^{-1}(a_s) a_t \big) v_t = \pi \Big( \alpha_s \big( \alpha_{s}^{-1}(a_s) a_t \big) u(s,t) \Big) v_{st}. 
\]

\ref{enu:range_of_covariant_rep2} If $s\leq t$ then $I_{s}\subseteq I_{t}$ by Remark~\ref{rem:twisted_relations}. 
Using \ref{item:covariant_representation3} and \ref{item:covariant_representation2}, for any $a \in I_{s}$ we get  $ \pi(a)v_t=\pi(a)v_{ss^*} v_t= \pi(au(ss^*,t))v_{ss^*t} = \pi(au(ss^*,t))v_{s}$. 

If $B$ is a $C^*$-algebra then $B''$ is again a $C^*$-algebra -- the enveloping $W^*$-algebra of $B$ (this also holds in the real case, see \cite[Theorem 5.5.3]{Li_book}). 
In particular, $\pi''(1_{t})v_t = v_t\pi''(1_{t^*})$ is a partial isometry whose adjoint is $\pi''(u(t^*,t)^{-1})v_{t^*}$, see (the proof of) Proposition~\ref{prop:normalized_twisted_rep}. 
Thus for $a_t\in I_{t}$, $t\in S$, we get
$
    \big( \pi(a_t) v_t \big)^* = \Big( v_t \pi''(1_{t^*}) \pi \big( \alpha_{t}^{-1}(a_t) \big) \Big)^* = \pi \big( \alpha_{t}^{-1}(a_t) \big)^* \big( \pi''(1_{t^*}) v_t \big)^* = \pi \big( \alpha_{t}^{-1}(a_t) \big)^* \pi'' \big( u(t^*,t)^{-1} \big) v_{t^*}.
$
If in addition $A$ is a $C^*$-algebra then both $\pi$ and $\alpha_{t}$ are necessarily $*$-preserving, and thus $(\pi(a_t)v_t)^* = \pi(\alpha_{t}^{-1}(a_{t}^*)u(t^*,t)^{-1})v_{t^*}$.
\end{proof}

For any twisted action $(\alpha,u)$ the subspace $\ell^1(\alpha,u) := \{ f \in \ell^1(S,A) : \text{$f(t) \in I_{t}$ for all $t\in S$} \}$ of $\ell^1(S,A)$ is a Banach algebra with multiplication
\[
    (f * g)(r) := \sum_{st = r} \alpha_s \Big( \alpha_{s^*} \big( f(s) \big) g(t) \Big) u(s,t),
\]
see (the proof of) \cite[Proposition 3.1]{Sieben98}. 
If $A$ is a $C^*$-algebra then in fact $\ell^1(\alpha,u)$ is a Banach $*$-algebra with involution
given by $f^*(t) := \alpha_{t}^{-1}(f(t^*)^*) u(t^*,t)^{-1}$, $t\in S$.
Lemma~\ref{lem:range_of_covariant_rep} (see also \cite[Proposition 3.7]{Sieben98}) implies that every covariant representation $(\pi,v)$ of $\alpha$ in a Banach algebra $B$ `integrates' to a representation  $\pi\times v : \ell^1(\alpha,u) \to B$ given by
\[
    \pi\times v(f) := \sum_{t\in S} \pi \big( f(t) \big) v_t, \qquad f \in \ell^1(\alpha,u),
\]
and $\overline{\pi\times v(\ell^1(\alpha,u))} = B(\pi,v)$ is a Banach algebra. 

\begin{defn}
For a fixed class $\EE$ of covariant representations of $(\alpha,u)$  we define the \emph{$\EE$-crossed product} for $(\alpha,u)$, denoted $A\rtimes_{(\alpha,u),\EE} S$, as the Hausdorff completion of $\ell^1(\alpha,u)$ in the seminorm
\[
    \| f \|_{\EE} := \sup \{ \| \pi\times v(f)\| : (\pi,v)\in \EE \} .
\]
The \emph{universal Banach algebra crossed product} $A\rtimes_{(\alpha,u)} S$ is the crossed product associated to the class of all covariant representations of $(\alpha,u)$.
\end{defn}

\begin{rem}\label{rem:group_crossed^products}
If $S = G$ is a  group, then  we necessarily have  $I_t = A$ for all $t\in G$, so   $\alpha:G\to\Aut(A)$,
and $A\rtimes_{(\alpha,u)}G=\ell^{1}(\alpha,u)$  is $\ell^1(G,A)$ as a Banach space. 
When $A=C_0(X)$ we study such crossed products in  \cite{BK}, and in subsection  \ref{subsect:twisted_partial_group_actions}
 below we discuss their generalizations to partial actions.
For general $A$, but when the twist $u\equiv 1$ is trivial,  $(A,G,\alpha)$ is a Banach algebra dynamical system in the sense of \cite{DDW}, and the
algebras $A\rtimes_{\alpha, \EE} G$ are examples of crossed products considered in  \cite{DDW}. 
%;the authors of \cite{DDW} consider also (Hausdorff) completions of $\ell^1(\alpha)$ in (semi)-norms that are not smaller than the $L^1$-norm of $\ell^1(G,A)$.
\end{rem}


%
\subsection{Covariant representations in dual Banach algebras and on Banach spaces}

We fix a twisted action $(\alpha,u)$ of an inverse semigroup $S$ on $A$. 
We will now discuss situations, beyond the one described in Remark~\ref{rem:unital_actions}, when in the study of algebras generated by covariant representations we do not need to pass through the bidual algebra $B''$. 

\begin{defn} \label{de:PredualNormalizedRepresentation}
Let $(\pi,v)$ be a covariant representation of $(\alpha,u)$ in a Banach algebra $B$. 
If $B$ has a predual Banach space $B_*$, we say that $(\pi,v)$ is \emph{$B_*$-normalized} if, for every $t \in S$, we have $v_{t} = B_*\mhyphen\lim_{i} ( \pi(\mu_{i}^t) v_{t} )$, for an approximate unit $\{ \mu^t_i \}_i$ in $I_t = I_{tt^*}$. 
Here $B_*\mhyphen\lim$ denotes the limit in the weak$^*$ topology on $B$ induced by $B_*$. 
\end{defn}

\begin{lem}\label{lem:B_0-normalized_is_in_B}
If $B_*' \cong B$ and $(\pi,v)$ is a $B_*$-normalized covariant representation in $B$, then $v : S \to B$ takes values in $B$ (rather than in $B''$).
\end{lem}
\begin{proof}
The limit $v_t = B_*\mhyphen\lim_{i} \pi(\mu_i^{t}) v_t$ is in $B$ as $\{ \pi ( \mu_i^{t} ) v_t \}_{i} \subseteq B_1$ and $B_1$ is $B_*$-compact.
\end{proof}

A \emph{dual Banach algebra} is a pair $(B, B_*)$ where $B$ is a Banach algebra, $B_*$ is a predual of $B$, and in addition multiplication in $B$ is separately $B_*$-continuous, see \cite{Runde}, \cite{Daws}. Examples of dual Banach algebras include all $W^*$-algebras (complex or real) with their unique preduals, and all algebras $B(E)$ of all bounded operators on a reflexive Banach space $E$, equipped with the canonical predual Banach space $E' \widehat{\otimes} E$.
By \cite[Theorem 5.6]{Ilie_Stokke}, every representation $\pi : A \to B$ into a dual Banach algebra $(B, B_*)$ has a unique extension to a representation $\overline{\pi} : \Mult(A) \to B$ which is strictly-$B_*$-continuous; it is given by  $\overline{\pi}(m) := B_*\mhyphen\lim_{i} \pi(m \mu_{i})$ for an approximate unit $\{\mu_{i}\}_i$ in $A$. 

\begin{prop}\label{prop:B_*_normalized_twisted_rep}
Assume $(B,B_*)$ is a dual Banach algebra. 
For any covariant representation $(\pi,v)$ of $(\alpha,u)$ in $B$ there is a unique $B_*$-normalized covariant representation $(\pi,\tilde{v})$ such that $\pi(a) v_t = \pi(a) \tilde{v}_{t}$ for all $a\in I_{t}$ and $t\in S$.
A pair $(\pi,v)$ is a $B_*$-normalized covariant representation of $(\alpha,u)$ in $B$ if and only if $\pi : A \to B$ is a representation and $v : S\to B_{1}$ is a map such that every $v_t$, $t\in S$, admits a contractive generalised inverse $v_t^*$, satisfying
\[
    v_t \pi(a) v_t^* = \pi \big( \alpha_{t}(a) \big), \quad v_s v_t = \pi_{st} \big( u(s,t) \big) v_{st} , \quad \text{$v_t v_t^* =\pi_t(1_{t})$ and $v_t^* v_t  =\pi_{t^*}(1_{t^*})$},
\]
where $a\in I_{t^*}$, $s,t\in S$ and $\pi_t : \Mult(I_t) \to B$ is the unique strictly-$B_*$-continuous extension of $\pi|_{I_t}$.
\end{prop}
\begin{proof} 
Follow the proof of Proposition~\ref{prop:normalized_twisted_rep} (some arguments can be simplified as the  multiplication in $B$ is $B_*$-continuous in both variables). 
\end{proof}

\begin{rem}
For any Banach algebra $B$ the pair $(B'',B')$ is a dual Banach algebra if and only if $B$ is Arens regular. %\marginpar{\tiny Maybe should give a reference for this?}
Thus Proposition~\ref{prop:normalized_twisted_rep} cannot be deduced from Proposition~\ref{prop:B_*_normalized_twisted_rep}. 
We were not able to find a nice common generalization of these two statements.
\end{rem}

\begin{rem} 
Proposition~\ref{prop:B_*_normalized_twisted_rep} implies that a $B_*$-normalized representation of 
an untwisted action $\alpha$ (an action with a trivial twist) in a dual Banach algebra $B$ is a pair $(\pi,v)$, where $\pi : A \to B$ is a representation and $v : S \to B_{1}$ is a semigroup homomorphism, such that $v_t \pi(a) v_{t^*} = \pi(\alpha_{t}(a))$ and $v_e = \pi_e (1_{e})$ for $a\in I_{t^*}$, $s,t\in S$, $e\in E(S)$.
\end{rem}
%To embrace the general situation we  weaken the notion of a predual Banach algebra:
%\begin{defn}
%We call a Banach algebra $B$ with a  predual Banach space $B_*$ is a \emph{left} (resp. \emph{right}) \emph{predual Banach algebra}
%if the product in $B$ is $B_*$-continuous in the first (resp. second) variable.
%\end{defn} 
%\begin{rem} For any Banach algebra $B$ the dual Banach space $B'$ is naturally a $B$-bimodule.
%Any predual Banach space $B_*$ for $B$ can be viewed as a subspace $B_*\subseteq B'$ of $B'$. 
%Then $(B,B^*)$ is a left (resp. right) predual Banach algebra iff  $B_*$ is a left (resp. right)  $B$-submodule of $B'$.
%Obviously, $(B, B_*)$ is a predual Banach algebra iff it is both left and right predual Banach algebra.
%For any Banach algebra $B$, $(B'',B')$ is a left (resp. right) predual Banach algebra when $B''$ is equipped with
%the second (resp. first) Arens product.
%\end{rem}

We now relate the above discussion to representations on Banach spaces. 
Note that if $\pi : A \to B(E)$ is a representation then, for each $t\in S$, $\pi$ yields a non-degenerate representation $I_t \to B(\overline{\pi(I_{t})E})$ which thus uniquely extends to a representation 
\[
    \overline{\pi}_t : \Mult(I_t) \to B(\overline{\pi(I_{t})E}) ;\ \overline{\pi}_t(m) \big( \pi(a)x \big) := \pi ( ma ) x , \qquad m\in \Mult(I_t),\ a\in I_{t},\ x\in E , 
\]
see for instance \cite[Theorem 4.1]{Gardella_Thiel1}.

\begin{defn}\label{defn:covariant_representation_on_space}
A \emph{covariant representation of $(\alpha, u)$ on a Banach space $E$} is a pair $(\pi,v)$ where $\pi : A \to B(E)$ is a representation and $v : S \to B(E)_{1}$ has the property that every $v_t$ ($t\in S$) admits a contractive generalised inverse $v_t^*$, satisfying: 
\begin{enumerate}[labelindent=40pt,label={(SCR\arabic*)},itemindent=1em] 
    \item\label{item:covariant_representation1'} $v_t \pi(a) v_{t}^* = \pi(\alpha_t(a))$ for all $a\in I_{t^*}$, $t\in S$;
    \item \label{item:covariant_representation2'} the ranges of $v_t $ and $v_t^*$ are  $\overline{\pi(I_{t})E}$ and $\overline{\pi(I_{t^*})E}$ respectively, for all $t\in S$;
    \item\label{item:covariant_representation3'} $v_s v_t = \overline{\pi}_{st}(u(s,t)) v_{st}$ for all $a\in I_{st}$, $s,t\in S$.
\end{enumerate}
Equality in \ref{item:covariant_representation3'} holds pointwise and it makes sense due to \ref{item:covariant_representation2'}.
We  say that $(\pi,v)$ is non-degenerate, injective, etc.\ if $\pi$ has that property.
\end{defn}

\begin{rem}\label{rem:space_covariant_reps} 
%A covariant representation  $(\pi,v)$ on a Banach space $E$ is a special case of a covariant representation in the Banach algebra $B(E)$ (cf. the proof of Lemma \ref{lem:normalized_twisted_rep}). 
If $\F = \C$, $E = H$ is a Hilbert space and $A$ is a $C^*$-algebra then Definition~\ref{defn:covariant_representation_on_space} is equivalent to  \cite[Definition 3.2]{Sieben98} and \cite[Definition 6.2]{Buss_Exel}. % \marginpar{\tiny I think both these references apply to the complex case only?}
\end{rem}

\begin{lem}\label{lem:covariant_reps_implies_MP^partial_isos}
Let $(\pi,v)$ be a covariant representation of $(\alpha,u)$ on a Banach space $E$. 
Then $(\pi,v)$ is a covariant representation in the Banach algebra $B(E)$ such that each $v_e\in B(E)$, $e\in E(S)$, is a  strong limit of $\{\pi(\mu_i^e)\}_{i}$, where  $\{\mu_{i}^e\}_{i}$ is a contractive  approximate unit in $I_{e}$.
\end{lem}
\begin{proof} The arguments in the beginning of the proof of Proposition~\ref{prop:normalized_twisted_rep} show that $(\pi,v)$ is a covariant representation in the Banach algebra $B(E)$. 
Then each $v_e\in B(E)$, $e\in E(S)$, is a projection onto $\overline{\pi(I_e)E}$ which commutes with elements of $\pi(I_e)$, and thus $v_e$ is a  strong limit of $\{\pi(\mu_i^e)\}_{i}$.
\end{proof}

\begin{cor} \label{cor:covariant_reps_implies_MP^partial_isos}
If $A$ is a complex $C^*$-algebra and $(\pi,v)$ is a non-degenerate covariant representation of $(\alpha,u)$ on a complex Banach space $E$ then each $v_t$, $t \in S$, is an $MP$-partial isometry on  $E$.
\end{cor}
\begin{proof} 
Since $\pi$ is non-degenerate we may extend it to the unitization of $A$ and so we may assume that $A$ and $\pi$ are in fact unital. 
Since $A$ is a $C^*$-algebra, for each $e\in E(S)$ we may choose an approximate unit $\{\mu_{i}^e\}_{i}$ in $I_e$ consisting of hermitian operators. 
Then $\{\pi(\mu_{i}^e)\}_{i}$ are hermitian operators by \cite[Lemma 2.4]{cgt}. 
As $v_e$ is a strong limit of $\{\pi(\mu_{i}^e)\}_{i}$, this implies that $v_e$ is also hermitian. 
Indeed, recall that $a\in B(E)$ is hermitian if and only if its numerical range $V(a) = \{ f(ax): x\in E,\ f\in E',\ \| f \| = \| x \| = 1 = f(x) \}$ is real. 
Hence for any $x\in E,\ f\in E'$ with $\| f \| = \| x \| = 1 = f(x)$ we get $f(v_ex) = f(\lim_i \pi( \mu_{i}^e) x)\in \R$, and so $v_e$ is hermitian. 
As $v_t v_t^* = v_{tt^*}$ and $v_t^* v_t = v_{t^*t}$, where $tt^*, t^*t\in E(S)$, for all $t\in S$, we conclude that $\{v_t\}_{t\in S}\subseteq B(E)_{1}$ are $MP$-partial isometries.
\end{proof}

\begin{lem}\label{lem:reflexive_covariant_representations}
Assume that $E$ is a reflexive Banach space, and recall that the projective tensor product Banach space $B_* := E' \mathbin{\widehat{\otimes}} E$ is naturally a predual of $B := B(E)$ making it a dual Banach algebra.
A covariant representation $(\pi,v)$  of $(\alpha,u)$ in  $B(E)$ is $B_*$-normalized if and only if $(\pi,v)$ is a covariant representation on the Banach space $E$.
% (in the sense of Definition~\ref{defn:covariant_representation_on_space}).
\end{lem}
\begin{proof}
We identify $f \otimes \xi \in E' \mathbin{\widehat{\otimes}} E = B_*$ with the functional map $B\ni b\mapsto f \otimes \xi (b) := f(b\xi)\in \F$.
Assume  $(\pi,v)$ is $B_*$-normalized, and so $\{v_t\}_{t\in S}\subseteq B(E)$ are partial isometries with the associated generalized inverses $\{v_t^*\}_{t\in S}\subseteq B(E)$, satisfying the relations described in Proposition~\ref{prop:B_*_normalized_twisted_rep}, which includes \ref{item:covariant_representation1'}. 
For $e\in E(S)$ we have $v_e = \pi_e(1_e) = B_*\mhyphen\lim\pi(\mu_i^e)$, where $\{\mu_{i}^e\}_{i}$ isan approximate unit in $I_{e}$.
Thus $f(v_e\xi) = \lim_{i} f(\pi(\mu_i^e)\xi)$ for all $(f,\xi)\in E'\times E$, which implies that for any $\xi\in E$ we have $v_e\xi\in \overline{\pi(I_{e})E}^{E'}$. 
As the weak and norm closures of convex sets coincide this means that the range of $v_e$ is $\overline{\pi(I_{e})E}$.
Thus, for each $t\in S$, we have $v_{tt^*} = \pi_t(1_t)$ is a projection onto the space $\overline{\pi(I_{t})E} = \overline{\pi(I_{tt^*})E}$. 
This implies \ref{item:covariant_representation2'} because $v_t = \pi_t(1_t) v_t$ and  $v_t^* = \pi_t(1_{t^*})v_{t}^*$. 
We also have $\pi_t(m) = \overline{\pi}_t(m) \pi_t(1_t)$ for $m\in \Mult(I_t)$, which immediately gives \ref{item:covariant_representation3'} from $v_s v_t = \pi_{st}(u(s,t))v_{st}$. 

Conversely, assume $(\pi,v)$ is  a covariant representation in the sense of Definition~\ref{defn:covariant_representation_on_space}.
By Lemma~\ref{lem:covariant_reps_implies_MP^partial_isos} each $v_{tt^*}$ is a strong limit of $\{\pi(\mu_i^t)\}_{i}$, where  $\{\mu_{i}^t\}_{i}$ is an approximate unit in $I_t = I_{tt^*}$. 
Using this, one sees that $v_{tt^*} = \pi_t(1_t)$ and $\pi_t(m) = \overline{\pi}_t(m)\pi_t(1_t)$ for $m \in \Mult(I_t)$. 
Now one readily gets the relations in Proposition~\ref{prop:B_*_normalized_twisted_rep}.
\end{proof}


\begin{cor} \label{cor:normalization_to_spatial_cov_rep}
%Let  $\alpha:S\to\PAut(A)$  be an inverse semigroup action on a $C^*$-algebra $A$ (or on a real space $C_0^\R(X)$). 
For any covariant representation $(\pi,v)$ of $(\alpha,u)$ in $B(E)$, where $E$ is a reflexive Banach space, there is a unique covariant representation $(\pi,\tilde{v})$ on $E$ such that $\pi(a) v_t = \pi(a) \tilde{v}_{t}$ for all $a\in I_{t}$ and $t\in S$ (in fact $(\pi,\tilde{v})$ is $E' \mathbin{\widehat{\otimes}} E$-normalized).
\end{cor}
\begin{proof} 
Combine Lemma~\ref{lem:reflexive_covariant_representations} and Proposition~\ref{prop:B_*_normalized_twisted_rep}.
\end{proof}

\begin{cor}\label{cor:Cstar_algebraic_inverse_semigroup_crossed^product}
Let  $(\alpha,u)$ be a twisted inverse semigroup action on a $C^*$-algebra $A$. 
Let $\EE_{C^*}$ be the class of all covariant representations of $(\alpha,u)$ in some $C^*$-algebra, and let $\EE_{\textup{Hil}}$ be the class of all covariant representations of $(\alpha,u)$ on some Hilbert space. 
Then 
\[
    A \rtimes_{(\alpha,u),\EE_{C^*}} S = A \rtimes_{(\alpha,u),\EE_{\textup{Hil}}} S ,
\]
and when $\F = \C$ this coincides with the crossed product introduced in \cite{Buss_Exel}, \cite{Sieben98}. 
\end{cor}
\begin{proof}
Since $\EE_{\textup{Hil}} \subseteq \EE_{C^*}$ we get $\|\cdot\|_{\EE_{\textup{Hil}}} \leq \|\cdot\|_{\EE_{C^*}}$. 
This is in fact equality, because for any covariant representation $(\pi,v)\in \EE_{C^*}$ in a $C^*$-algebra $B$, we may assume that the enveloping $W^*$-algebra $B''$ is faithfully represented on a Hilbert space $H$. 
Then, by Corollary~\ref{cor:normalization_to_spatial_cov_rep}, the renormalized covariant representation  $(\pi,\tilde{v})$ is a  covariant representation of $\alpha$ on $H$, so $(\pi,\tilde{v})\in \EE_{\textup{Hil}}$. 
The claim follows because Corollary~\ref{cor:normalization_to_spatial_cov_rep} implies $\pi\times v(f) = \pi\times \tilde{v}(f)$ for all $f \in \ell_1(\alpha,u)$.
\end{proof}



%
%
\section{Groupoid Banach algebras  and inverse semigroup disintegration}
\label{sec:GroupoidBanachAlgebras}

%In this paper we will consider only inverse semigroup actions on a locally compact Hausdorff space $X$. 
Let $\G$ be an \'etale groupoid with locally compact Hausdorff unit space $X$. 
By a twist over $\G$ we  mean a Fell line bundle \cite{Kumjian}.
This is equivalent to twists introduced by Kumjian in \cite{Kumjian0}, see \cite[2.5.iv]{Kumjian}.
More specifically, let $\LL$ be a  line bundle over $\G$. 
Equivalently, $\LL = \bigsqcup_{\gamma\in \G } L_\gamma$ is a topological space such that the canonical projection $\LL \onto \G$ is continuous and open, each fiber $L_\gamma \cong \F$ is a one dimensional Banach space with a structure consistent with the topology of $\LL$, in the sense that the maps 
\[
   \bigsqcup_{\gamma\in \G} L_\gamma \times L_\gamma \ni (z, w) \mapsto z + w\in \LL, \quad 
    \F\times \LL\ni (\lambda,z) \mapsto \lambda z\in \LL , \quad 
    \LL\ni z\mapsto  |z|\in \R
\]
are continuous. 
Then $\LL$ is necessarily locally trivial. 
For each $U \subseteq \G$ we write $C(U,\LL)$ for the linear space of continuous sections in the restricted bundle 
$\LL|_U := \bigsqcup_{\gamma\in U } L_\gamma$  (i.e. continuous maps $f : U \to \LL$ with $f(\gamma)\in L_\gamma$ for $\gamma \in U$).
Similarly we denote by $C_c(U,\LL)$, $C_0(U,\LL)$, $C_b(U,\LL)$, $C_u(U,\LL)$ the spaces of continuous sections with compact support, vanishing at infinity, bounded and unitary, respectively. 
Then $C_0(U,\LL)$ is a Banach space with the norm $\|f\|_{\infty} := \max_{\gamma\in U}| f(\gamma)|$. 
For any $\gamma \in \G$ there is an open neighbourhood $U$ of $\gamma$ and unitary section $f \in C(U,\LL)$, i.e. $|f(\gamma)| = 1$, for all $\gamma \in U$.
This section induces an isometric isomorphism $C_0(U,\LL)\cong C_0(U)$, as for every $g\in C_0(U,\LL)$ there is a unique element $g/f \in C_0(U)$ such that $g(\gamma) = (g/f)(\gamma) f(\gamma)$, $\gamma \in U$.

\begin{defn} %\marginpar{\tiny I started to make this definition easier to parse}
A \emph{twist over the groupoid}  $\G$ is a (locally trivial) line bundle $\LL=\bigsqcup_{\gamma\in \G } L_\gamma$ over $\G$ together 
with 
a continuous \emph{multiplication}
${\cdot}:\bigsqcup_{d(\gamma)=r(\eta)\in \G } L_{\gamma}\times L_{\eta} \to \LL$,
which is associative (whenever the product makes sense), makes the module multiplicative, and induces bilinear maps
$ L_{\gamma}\times L_{\eta}\longrightarrow L_{\gamma\eta}$; and  a continuous \emph{involution} $^{*}: \LL\to \LL$ that restricts 
to conjugate linear maps $ L_{\gamma}\to L_{\gamma^{-1}}$, preserves the module, is anti-multiplicative and $z\cdot z^*=|z|^2\in L_{r(\gamma)}\cong \F$ for every $z\in L_{\gamma}$, $\gamma \in \G$. See \cite{Kumjian} for more details.
\end{defn}

\begin{lem}\label{lem:inverse_semigroup_of_trivialtwist_bisections} 
Let $(\G,\LL)$ be a twisted groupoid. 
The set $S(\LL)$ of bisections $U$ for which the restricted bundle $\LL|_{U}$ is trivial forms a unital wide inverse subsemigroup $S(\LL)\subseteq \Bis(\G)$.
\end{lem}
\begin{proof}  
The restricted bundle $\LL|_U$ is trivial if and only if there is a continuous unitary section $c_U\in C_u(U,\LL)$. 
If $\LL|_{U}$ and $\LL|_{V}$ are trivial, and $c_U \in C_u(U,\LL),\ c_V \in C_u(V,\LL)$, then $\LL|_{UV}$ and $\LL|_{U^{-1}}$ are trivial because the convolution product $c_{U}* c_V \in C_u(UV,\LL)$ and $c_U^*\in C_u(U^{-1},\LL)$ (in a Fell line bundle the norm $| \cdot |$ is multiplicative, rather than just submultiplicative). 
Thus $S(\LL)\subseteq \Bis(\G)$ is an inverse semigroup. 
Since $\LL$ is locally trivial, $S(\LL)$ is wide.
Moreover, $\LL$ is trivial over $X$ as each fiber $L_x\cong \F$ is a one-dimensional algebra with unit $1_x$, and the unit section defined by $1(x) := 1_x$ is unitary and continuous. 
\end{proof}

According to Lemma~\ref{lem:inverse_semigroup_of_trivialtwist_bisections} we may and shall always \emph{assume that $\LL|_X = X \times \F$ is trivial}. 

\begin{ex}[Twist by a cocycle] 
The trivial bundle $\LL = \G\times \F$ with pointwise operations is a \emph{trivial twist}.
More generally, if $\sigma$ is a \emph{continuous normalised $2$-cocycle} on $\G$, that is a continuous map $\sigma : \G^{(2)} \to \{ z\in \F:  |z| = 1 \}$ satisfying 
\[
    \sigma \big( r(\gamma),\gamma \big) = 1 = \sigma \big( \gamma , d(\gamma) \big) , \qquad \sigma(\alpha,\beta)\sigma(\alpha\beta,\gamma) = \sigma(\beta,\gamma)\sigma(\alpha,\beta\gamma)
\]
for every composable triple $\alpha,\beta,\gamma \in \G$, 
then we may consider a twist $\LL_{\sigma}$ given by  the trivial bundle $\G\times \F$ with operations given by 
\[
    (\alpha,w) \cdot (\beta,z) := \big( \alpha \beta , \sigma(\alpha,\beta) wz \big) , \qquad  
    (\alpha,w)^{*} := \big( \alpha^{-1} , \overline{\sigma(\alpha^{-1},\alpha) w } \big) .
\]
Every twist based on a trivial line bundle is of this form.
\end{ex}


%
\subsection{The algebras}

Let us fix a twisted groupoid $(\G,\LL)$.
Recall our convention that $\G$ is locally compact and only locally Hausdorff. 
Thus the associated $*$-algebra will be defined on the set of quasi-continuous compactly supported sections:
\[
    \mathfrak{C}_c(\G,\LL) := \spane \{ f \in C_c(U,\LL): U\in  \Bis(\G) \} , 
\]
where we treat sections of $\LL|_U$ as sections of $\LL$ that vanish outside $U$. 
Note that $\mathfrak{C}_c(\G,\LL) = \spane\{f\in C_c(U,\LL): U\in \mathcal{C}\}$ for any cover  $\mathcal{C}\subseteq \Bis(\G)$ of $\G$, cf. \cite[Proposition 3.10]{Exel}.
In particular, we may always take $\mathcal{C} = S(\LL)$. If the groupoid $\G$ is Hausdorff, then $\mathfrak{C}_c(\G,\LL) = C_c(\G,\LL)$ is the usual space of continuous compactly supported sections. 
We define a $*$-algebra structure on $\mathfrak{C}_c(\G,\LL)$ by
\begin{equation}\label{eq:convolution_and_involution}
    (f*g)(\gamma) := \sum_{r(\eta) = r(\gamma)} f(\eta) \cdot g(\eta^{-1}\cdot \gamma), \qquad (f^*)(\gamma) := f(\gamma^{-1})^* ,
\end{equation}
where $f,g \in \mathfrak{C}_c(\G, \LL)$ \cite[Proposition 3.11]{Exel}. 
Since we assume that $\LL|_X$ is trivial, we get that $C_c(X)=C_c(X,\LL)$ is a $*$-subalgebra of $\mathfrak{C}_c(\G,\LL)$. 

The domain and range maps $d,r : \G\to X$ induce the following three submultiplicative norms on $\mathfrak{C}_c(\G)$:
\[ 
    \| f \|_{d_*} := \max_{x\in X} \sum_{d(\gamma)=x} |f(\gamma)|, \qquad 
    \| f \|_{r_*} := \max_{x\in X} \sum_{r(\gamma)=x} |f(\gamma)|, \qquad 
	\|f\|_{I} := \max\{ \|f\|_{d_*} , \|f\|_{r_*} \}, %=\max\left\{\max_{x\in X}\sum_{d(\gamma)=x} |f(\gamma)|, \max_{x\in X}\sum_{r(\gamma)=x} |f(\gamma)|\right\},
\]
see \cite[Section 2.2]{Paterson} or \cite[II, 1.4]{Renault_book}. 
The  norm $\|\cdot \|_{I}$ is called the \emph{$I$-norm} and was introduced by Hahn~\cite{Hahn}. 
It has the advantage that it is preserved by the involution $*$, while for the remaining norms we only have $\| f^* \|_{d_*} = \|f\|_{r_*}$. %(the normed algebras $(\mathfrak{C}_c(\G), \|\cdot \|_{d})$ and $(\mathfrak{C}_c(\G)^{\op}, \|\cdot \|_{r})$ are isometrically isomorphic).
The important feature of these norms is that when restricted to any subspace $C_c(U,\LL) \subseteq \mathfrak{C}_c(\G,\LL)$, where $U\in \Bis(\G)$, they coincide with the supremum norm $\| \cdot \|_\infty$.
Hence each space $C_0(U,\LL)$ embeds naturally into the completions of $\mathfrak{C}_c(\G)$ in each of these norms. 
We introduce yet another norm which is  maximal with respect to this property.

\begin{lem}
There is a submultiplicative norm on $\mathfrak{C}_c(\G, \LL)$ given by 
\begin{equation}\label{eq:projective_norm}
    \| f \|_{\max} := \inf \left\{ \sum_{k=1}^{n} \| f_k \|_\infty : f = \sum_{k=1}^{n} f_k, \ f_k \in C_c(U_k, \LL), \ U_k \in \Bis(\G) \right\} .
\end{equation}
This is the largest norm on $\mathfrak{C}_c(\G,\LL)$ which agrees with $\|\cdot\|_\infty$  on each  $C_c(U,\LL) \subseteq \mathfrak{C}_c(\G,\LL)$, 
for $U\in \Bis(\G)$.
\end{lem}
\begin{proof}
Let $f,g \in \mathfrak{C}_c(\G, \LL)$.
For any $\varepsilon>0$ there are $f_k \in C_c(U_k, \LL)$ and $g_l\in C_c(V_k, \LL)$, where $U_k, V_l\in \Bis(\G, \LL)$, such that $f = \sum_{k=1}^{n} f_k$, $g = \sum_{l=1}^{m} g_l$ and $\| f \|_{\max} + \varepsilon > \sum_{k=1}^{n} \| f_k \|_\infty$ and $\| g \|_{\max} + \varepsilon > \sum_{l=1}^{m} \| g_l \|_\infty$. 
Then $f * g = \sum_{k,l} f_k * g_l$, and $f_k * g_l\in C_c(U_k V_l,  \LL)$, therefore
\[
    \| f * g \|_{\max} \leq \sum_{k,l} \| f_k * g_l \|_\infty \leq \sum_{k,l} \| f_k \|_\infty \| g_l \|_\infty < \big( \|f\|_{\max}+\varepsilon \big) \big( \|g\|_{\max} + \varepsilon \big) .
\]
This shows that $\|\cdot\|_{\max}$ is submultiplicative. 
The proof of the triangle inequality is even simpler.  
If in addition $f \in C_c(U,\LL)$ for some $U\in \Bis(\G)$, then there is $\gamma_0 \in U$ where $|f|$ attains its maximum, and then  
\[ 
    \| f \|_\infty = | f(\gamma_0) | \leq \sum_{k=1}^{n} |f_k(\gamma_0)| \leq  \sum_{k=1}^{n} \| f_k \|_\infty < \| f \|_{\max} + \varepsilon ,
\]
which shows that $\|f\|_\infty\leq \|f\|_{\max}$. 
The converse inequality $\|f\|_{\max}\leq \|f\|_\infty$ holds by definition. 
Hence $\|\cdot\|_{\max}$ coincides with $\|\cdot\|_\infty$ on  $C_c(U,\LL)$. 
If $\|\cdot \|$ is any norm on $\mathfrak{C}_c(\G,\LL)$ that agrees with $\|\cdot\|_\infty$ on  $C_c(U,\LL) \subseteq \mathfrak{C}_c(\G,\LL)$, where $ U\in \Bis(\G)$, then for any $f=\sum_{k=1}^{n} f_k$, $f_k\in C_{c}(U_k,\LL)$, $U_k\in\Bis(\G)$, we have
$\|f\|\leq \sum_{k=1}^{n} \|f_k\|= \sum_{k=1}^{n} \|f_k\|_\infty$, which implies $\|f\|\leq \|f\|_{\max}$.
\end{proof}

\begin{rem} 
Clearly every norm $\|\cdot\|$ on $\mathfrak{C}_c(\G,\LL)$ satisfying the $C^*$-equality $\|f\|^2=\|f^*f\|$ coincides with $\|\cdot\|_\infty$ on each space $C_c(U,\LL)$, $U\in \Bis(\G)$. 
Thus the largest $C^*$-norm $\|\cdot\|_{C^*\mathrm{max}}$ on $\mathfrak{C}_c(\G,\LL)$ exists and it does not exceed $\|\cdot\|_{\max}$.
In fact, by Corollary~\ref{cor:C*-max_norm_estimate} below, we have
\[
    \| f \|_{C^*\mathrm{max}}\leq \|f\|_{I},\qquad f\in \mathfrak{C}_c(\G,\LL),
\]
(see also \cite[Lemma 3.2.3]{Sims}).
In general, the norms  $\|\cdot\|_{d_*}, \|\cdot\|_{r_*}$, $\|\cdot\|_{C^*\mathrm{max}}$ are not comparable.
\end{rem}

\begin{defn}
We call the completion $F(\G,\LL) := \overline{\mathfrak{C}_c(\G,\LL)}^{\|\cdot\|_{\max}}$ the \emph{(full) Banach algebra of the twisted groupoid} $(\G,\LL)$.
We also write $F_I(\G,\LL) := \overline{\mathfrak{C}_c(\G,\LL)}^{\|\cdot\|_{I}}$, $F_{d_*}(\G,\LL) := \overline{\mathfrak{C}_c(\G,\LL)}^{\|\cdot\|_{d_*}}$, $F_{r_*}(\G,\LL) := \overline{\mathfrak{C}_c(\G,\LL)}^{\|\cdot\|_{r_*}}$, and $C^*(\G,\LL) := \overline{\mathfrak{C}_c(\G,\LL)}^{\|\cdot\|_{C^*\mathrm{max}}}$ 
for the completions in the other norms. 
\end{defn}

\begin{rem}\label{rem:various_norms_and^projectivness}
 When $\G = G$ is a group all the  norms $\|\cdot\|_{d_*}, \|\cdot\|_{r_*}, \|\cdot\|_{I}, \|\cdot\|_{\max}$ coincide with the norm in $\ell^{1}(G)$.
In general we only have the inequalities 
\[
    \| f \|_\infty \leq \| f \|_{d_*}, \| f \|_{C^*\mathrm{max}}, \| f \|_{r_*} \leq \| f \|_{I} \leq \|f\|_{\max}. 
\]
Moreover, the norm $\|\cdot\|_\infty$ is not submultiplicative for the convolution product. 
%In the case $\G = X\times G$ is the transformation groupoid for a group action $h : G \to \Homeo(X)$ we have $F(X\times G) \cong \ell^1(G,C_0(X)) = C_0(X) \rtimes_{\alpha} G$ and for every $f\in C_c(\G)$, 
%\[    \| f \|_{d_*} = \max_{x\in X} \sum_{g\in G}|f(x,g)| , \qquad \| f \|_{r_*} = \max_{x\in X} \sum_{g\in G} |f(\varphi_{g}(x),g)| , \qquad \| f \|_{\max} = \sum_{g\in G} \max_{x\in X} |f(x,g)| .
%\] 
%\begin{ex} If $\G:=\Z_n\rtimes \Z_n$ is a discrete pair groupoid, then 
% $C_c(\G)=M_{n}(\F)$ and  for any $f=[f_{ij}]_{i,j\in \Z_n}$ we have
%$
%\|f\|_\infty=\max_{i,j\in \Z_n}|f_{ij}|$ and
%$
%\|f\|_{d_*}=\max_{i\in \Z_n}\sum_{j\in \Z_n}|f_{ij}|$, $\|f\|_{r_*}=\max_{j\in \Z_n}\sum_{i\in \Z_n}|f_{ij}|$, 
%$
%\|f\|_{\max}=\max_{k\in \Z_n}\sum_{i-j=k}|f_{ij}|.
%$ 
%\end{ex}
The involution $^*$ is easily seen to be isometric with respect to the norms $\|\cdot\|_{\max}$, $\|\cdot\|_{I}$ and $\|\cdot\|_{C^*\mathrm{max}}$. 
Hence $F(\G,\LL)$, $F_I(G,\LL)$, $C^*(\G,\LL)$ are naturally Banach $*$-algebras. In fact, when $\F=\C$,   $C^*(\G,\LL)$ is the standard $C^*$-algebra associated to
 $(\G,\LL)$ studied by a number of authors. 
\end{rem}

\begin{rem} 
If the twist is trivial, i.e.\ $\LL = \G\times \F$, then sections become $\F$-valued functions and the corresponding algebras are completions of the algebra $\mathfrak{C}_c(\G) := \spane\{f\in C_c(U): U\in  \Bis(\G)\}$.
Thus we denote them by $F(\G)$, $F_I(\G)$, $F_{d_*}(\G)$, $F_{r_*}(\G)$, $C^*(\G)$, etc. 
Formally we should write $\mathfrak{C}_c(\G, \F) = \spane \{ f\in C_c(U,\F) : U\in \Bis(G) \}$. 
The real-valued functions $\mathfrak{C}_c(\G,\R)$ form a real $*$-subalgebra in the complex algebra $\mathfrak{C}_c(\G,\C)$. 
It follows from \eqref{eq:projective_norm} that the $\|\cdot\|_{\max}$-norms on $\mathfrak{C}_c(\G,\R)$ and $\mathfrak{C}_c(\G,\C)$ are compatible. 
Thus the complex Banach algebra $F(\G, \C) = \overline{\mathfrak{C}_c(\G,\C)}^{\|\cdot\|_{\max}}$ is a complexification of the real Banach algebra $F(\G,\R) = \overline{\mathfrak{C}_c(\G,\R)}^{\|\cdot\|_{\max}}$.
Similar remarks apply to the algebras $F_I(\G)$, $F_{d_*}(\G)$, $F_{r_*}(\G)$ and $C^*(\G)$
(that $C^*(\G, \C)$ is the complexification of $C^*(\G, \R)$ follows from Corollary~\ref{cor:representations into C*-algebras} below).
%Also note that $\widecheck{\,}$ extends to  complex linear $*$-antiautomorphism 
%of $F(\G,\C)$, $F_I(\G,\C)$ and to the reduced $C^*$-algebra $C^*_{\red}(\G)$, cf. .... Hence letting $B$ be any of these algebras we get 
% yet another natural  real $*$-Banach algebras $\{f\in B: \widecheck{f}=f^*\}\subseteq B$. 
\end{rem}

All the aforementioned Banach algebras have  approximate units.

\begin{lem}\label{lem:approximate_units}
An approximate unit $\{\mu_i\}_{i}$ in $C_0(X)$ is also an approximate unit in $F(\G,\LL)$, and for any representation $\psi : F(\G,\LL)\to B$ the net $\{\psi(\mu_i)\}_{i}$ is an approximate unit in $\overline{\psi(F(\G,\LL))}$. 
In particular, if $X$ is compact then $F(\G,\LL)$ and $\overline{\psi(F(\G,\LL))}$ are unital.
\end{lem}
\begin{proof}
It suffices to prove that $\{\mu_i * f\}_{i}$ converges to $f$ in $\|\cdot \|_{\max}$ for $f\in C_c(U)$, $U\in \Bis(G)$. 
But then $\mu_i * f - f\in C_c(U)$ and therefore $\|\mu_i * f - f\|_{\max} = \|\mu_i * f- f\|_{\infty} = \max_{\gamma \in U}| \mu_i(r(\gamma))f(\gamma) - f(\gamma)|$, which obviously tends to zero.
\end{proof}

We write $A \cong B$ to indicate that two Banach algebras $A$ and $B$ are isometrically isomorphic. 
We write $A \donto B$ if there is a representation of $A$ in $B$ with dense range. 
By Remark~\ref{rem:various_norms_and^projectivness} the identity on $\mathfrak{C}_c(\G,\LL)$
extends to canonical representations 
\[
    F(\G,\LL) \donto F_I(\G,\LL) \donto  F_{d_*}(\G,\LL),\ C^*(\G,\LL),\ F_{r_*}(\G,\LL) .
\]

%
\subsection{Opposite algebras and twists}

We denote by $A^{\op}$ the \emph{opposite algebra} to an  algebra $A$, meaning that the linear structure of $A^{\op}$ is the same as that of $A$ but the multiplication is performed in the reverse order. When $A$ is a Banach algebra, then  $A^{\op}$ is a Banacha algebra with the same norm.
Fix a twisted groupoid $(\G,\LL)$. 
An \emph{opposite twist} $\LL^{\op}$ on $\G$ is defined as follows (see \cite{Buss_Sims}): 
the fibers are $L_{\gamma}^{\op} := L_{\gamma^{-1}}$ ($\gamma \in \G$), and the topology is induced by the sections
\[
    (\widecheck{f})(\gamma) := f(\gamma^{-1}), 
\]
where $f\in C_c(U,\LL)$, $U\in \Bis(\G)$; the involution on $\LL^{\op}$ is given by the involution maps $L_{\gamma^{-1}}\to L_{\gamma}$ from $\LL$ and the 
multiplication maps $ L_{\gamma}^{\op}\times L_{\eta}^{\op}\longrightarrow L_{\gamma\eta}^{\op}$  are given by $(z,w)\mapsto w \cdot z$ where $z\in L_{\gamma}^{\op}=\LL_{\gamma^{-1}}$, $w\in L_{\eta}^{\op}=\LL_{\eta^{-1}}$ and $d(\gamma)=r(\eta)$.
The map $f \mapsto \widecheck{f}$ yields a $*$-isomorphism
\[
    \mathfrak{C}_c(\G,\LL)^{\op} \cong \mathfrak{C}_c(\G,\LL^{\op}).
\]
In particular, the untwisted algebra $\mathfrak{C}_c(\G)^{\op}\cong \mathfrak{C}_c(\G)$ is self-opposite. 
It follows from the formulas for the norms $\|\cdot\|_{d_*}, \|\cdot\|_{r_*}, \|\cdot\|_{I}, \|\cdot\|_{\max}$ that the above isomorphism extends to isometric isomorphisms 
\[ 
    F(\G,\LL)^{\op} \cong F(\G,\LL^{\op}), \qquad  F_I(\G,\LL)^{\op} \cong F_I(\G,\LL^{\op}), \qquad  F_{d_*}(\G,\LL)^{\op} \cong F_{r_*}(\G,\LL^{\op}). 
\]
Thus the algebras $F(\G)$ and $F_I(\G)$ are self-opposite, while for the remaining two algebras we have only $F_{d_*}(\G)^{\op}\cong F_{r_*}(\G)$.
Let us consider a slightly more general situation.

\begin{defn} 
For a class $\EE$ of Banach spaces we define the \emph{groupoid $\EE$-Banach algebra} of $\G$ as the Hausdorff completion $F_{\EE}(\G, \LL) := \overline{\mathfrak{C}_c(\G, \LL)}^{\|\cdot\|_{\EE}}$ in the seminorm
\[ 
    \|f\|_{\EE} := \sup \{ \| \psi(f) \|: \text{$\psi$ is a representation of $\mathfrak{C}_c(\G, \LL)$ on $E$, where $E\in \EE$} \}, 
\]
where by a \emph{representation of $\mathfrak{C}_c(\G, \LL)$} we mean a $\|\cdot\|_{\max}$-contractive homomorphism. 
\end{defn}

%\begin{rem}
%For any class  $\EE$ of  Banach spaces the map $ f\mapsto\widecheck{f}$
%yields the isomorphism $F_{\EE}(\G)\cong F_{\EE}(\G^{\op})$, and if  $ f\mapsto \widecheck{f}$ is isometric in  $F_{\EE}(\G)$, then 
%we also have $F_{\EE}(\G)\cong F_{\EE}(\G)^{\op}$.
%In particular, the algebras $F(\G)$, $F_I(\G)$ and $C^*(\G)$ are isometrically isomorphic to their opposites.
%\end{rem}

If $\EE$ is a class of Banach spaces we denote by $\EE'$ the class of dual spaces to spaces in $\EE$. 
For any other class $\FF$ of Banach spaces we write $\EE\subseteq_{\iso}\FF$ if for every $E\in \EE$ there is $F\in \FF$ such that $E\cong F$. 

\begin{prop}\label{prop:duality} 
If $\EE'\subseteq_{\iso}\FF$ then the map $f \mapsto \widecheck{f}$ induces a contractive homomorphism $F_{\FF}(\G, \LL)^{\op} \donto F_{\EE}(\G, \LL^{\op})$.
If $\EE'\subseteq_{\iso} \FF$ and $\FF' \subseteq_{\iso}\EE$ then this homomorphism is an isometric isomorphism $F_{\FF}(\G, \LL)^{\op} \cong F_{\EE}(\G, \LL^{\op})$.
\end{prop}
\begin{proof}
For any representation $\pi :  \mathfrak{C}_c(\G, \LL)^{\op} \to B(E)$ the formula $\pi'(f) := \pi(\widecheck{f})'$ defines a representation $\pi' : \mathfrak{C}_c(\G, \LL^{\op}) \to B(E')$. 
If $\EE'\subseteq_{\iso}\FF$ and $E\in \EE$, then we may replace $E'$ by $F\in \FF$ such that $E'\cong F$, so that we get a representation  $\pi' : \mathfrak{C}_c(\G, \LL^{\op}) \to B(F)$. 
This implies that $\| \widecheck{f} \|_{\EE}\leq \| f \|_{\FF}$ for $f \in \mathfrak{C}_c(\G, \LL^{\op})$.
Thus the isomorphism $\mathfrak{C}_c(\G, \LL)^{\op} \cong \mathfrak{C}_c(\G, \LL^{\op})$ extends to a representation $F_{\FF}(\G, \LL)^{\op} \donto  F_{\EE}(\G, \LL^{\op})$. 
Similarly, $\FF'\subseteq_{\iso}\EE$ implies $\|\cdot\|_{\FF}\leq \|\cdot\|_{\EE}$.
Hence  $F_{\FF}(\G, \LL)^{\op} \cong  F_{\EE}(\G, \LL^{\op})$ if $\EE'\subseteq_{\iso}\FF$ and $\FF'\subseteq_{\iso}\EE$. 
\end{proof}

\begin{cor}\label{cor:reflexive_spaces_oppositie_iso}
If $\EE$ consists of reflexive Banach spaces then $ F_{\EE}(\G,\LL)^{\op} \cong F_{\EE'}(\G,\LL^{\op})$.
\end{cor}
\begin{proof}
Take $\FF = \EE'$ in Proposition~\ref{prop:duality}. 
%Note that reflexivity is defined as the evaluation map $E \to E''$ being an isomorphism is equivalent (according to wikipedia!) to the evaluation map being an isometric isomorphism. 
\end{proof}

\begin{cor} \label{cor:Hilbert_spaces_oppositie_iso} 
 $C^*(\G,\LL)^{\op}\cong C^*(\G,\LL^{\op})$ and $C^*(\G)^{\op}\cong C^*(\G)$.
\end{cor}
\begin{proof} 
Apply Corollary~\ref{cor:reflexive_spaces_oppositie_iso} to the case where $\EE$ consists of Hilbert spaces.
\end{proof}

The untwisted version of Corollary~\ref{cor:reflexive_spaces_oppositie_iso} is known  (see \cite{Gardella_Thiel15}, \cite{Gardella_Lupini17}) and Corollary~\ref{cor:Hilbert_spaces_oppositie_iso} is a special case of the main result in \cite{Buss_Sims}. 
We apply Proposition~\ref{prop:duality}  beyond reflexive spaces.

\begin{cor}\label{cor:L_infty_lindenstrauss_spaces_etc}
Let $\EE$ be the class of $L^1$-spaces, $\FF_{1}$ the class of $L^{\infty}$-spaces, $\FF_{2}$ the class of $C_0$-spaces and $\FF_{3}$ the class of Lindenstrauss spaces, cf. Remark \ref{rem:C_0_and_Lindenstrauss}.
Then
\[ 
    F_{\FF_1}(\G,\LL) \cong F_{\FF_2}(\G,\LL) \cong  F_{\FF_3}(\G,\LL) \cong F_{\EE}(\G,\LL^{\op})^{\op}.
\]
\end{cor}
\begin{proof}
We clearly have $\EE' \subseteq_{\iso}\FF_1 \subseteq_{\iso} \FF_2$ and $\FF_3' \subseteq_{\iso} \EE$. 
Also $\FF_2\subseteq \FF_3$, which follows from the Riesz--Markov--Kakutani theorem and the abstract characterization of $L^1$-spaces, see for instance \cite[page 261]{Pelczynski}. 
Thus $\FF_2'\subseteq \FF_3' \subseteq_{\iso} \EE$. 
\end{proof}



%
\subsection{From twisted groupoids to twisted actions and back}
We describe the correspondence between twisted groupoids and twisted actions of inverse semigroups on $C_0(X)$,
which implicitly can be found in \cite{Buss_Exel}, \cite{Buss_Exel2}.
Let $(\G,\LL)$ be a twisted groupoid. 
Take any wide inverse subsemigroup $S\subseteq S(\LL)\subseteq \Bis(\G)$ and let $h : S\to \PHomeo(X)$ be the canonical action. 
Then we have the inverse semigroup action $\alpha : S \to \PAut(C_0(X))$  given by $\alpha_U(a) := a\circ h_{U^*}$ for $a\in  I_{d(U)} := C_0(d(U))$,  $U\in S$.
For each $U\in S(\LL)$ take any continuous unitary section $c_U\in C_u(U,\LL)$, but if $U\subseteq X$ let $c_U(x) := (x,1) \subseteq X \times \F$, $x\in U$.  
For any $U,V\in S(\LL)$ there is a unique unitary function $u(U,V)\in C_u(r(UV))$ satisfying  
$    u(U,V) \big( r(\gamma) \big) c_{UV}(\gamma\eta) = c_U(\gamma) \cdot c_V(\eta)$,  $\gamma \in U,\ \eta \in V,\ d(\gamma) = r(\eta)$.
In fact we have
\begin{equation}\label{eq:crucial_relations_for_twisted}
    \alpha_{U}(a) = c_U* a * c_U^*, \qquad u(U,V) = c_U* c_V * c_{UV}^*, \qquad  a\in C_0(d(U)),\ U, V\in S,
\end{equation} 
with operations given by \eqref{eq:convolution_and_involution}.

\begin{lem}\label{lem:twisted_action_from_twisted_groupoid} 
For any twisted groupoid $(\G,\LL)$ the pair $(\alpha,u)$ defined above is a twisted action of $S$ on  $C_0(X)$.
\end{lem}
\begin{proof}
Since $C_0(X)$ is commutative \ref{enu:twisted actions1} is equivalent to $\alpha$ being an action. 
Axioms \ref{enu:twisted actions2}, \ref{enu:twisted actions3}, \ref{enu:twisted actions4} can be checked exactly as in the proof of \cite[Proposition 3.6]{Buss_Exel} using \eqref{eq:crucial_relations_for_twisted}. 
\end{proof}

\begin{ex} 
If the twist $\LL = \LL_{\sigma}$ comes from a $2$-cocycle $\sigma$ then $S(\LL_\sigma) = \Bis(\G)$ and for any $\eta\in U$,$\gamma\in V$, we have $u(U,V)(\eta\gamma) = \sigma(\eta, \gamma) \overline{\sigma((\eta\gamma)^{-1}, \eta\gamma)}$.
\end{ex}

Now let $(\alpha,u)$ be any twisted action of $S$ on $C_0(X)$. 
Since $C_0(X)$ is commutative $\alpha : S \to \PAut(C_0(X))$ is a semigroup homomorphism, and hence there is a semigroup homomorphism $h : S \to \PHomeo(X)$ with $\alpha_t(a) = a\circ h_{t^*}$ for $a\in I_{t^*}=C_0(X_{t^*})$, $t\in S$. 
Let $\G = S\ltimes_{h} X$ be the associated transformation groupoid.  
We define a  bundle $\LL$ over $\G$ whose elements are equivalence classes of triples $(a,t,x)$ for
$a\in C_0(X_{t})$, $x\in X_{t^*}\subseteq X$, $t\in S$, and two triples $(a,t,x)$ and $(a',t',x')$ are
equivalent if $x=x'$ and there is $v\in S$ with $v\leq t,t'$, where $x\in X_{v^*}$, and $(a \cdot u(vv^*,t))(h_{v}(x)) = (a' \cdot u(vv^*,t'))(h_{v}(x))$.  
In particular, we have a canonical surjection 
\[
    \LL \to \G ;\ [a,t,x] \mapsto [t,x] ,
\]
and we denote by $L_{[t,x]}$ the corresponding fibers. 
This makes $\LL$ a line bundle with operations defined by
\[
    [a,t,x] + [b,t,x] := [a+b,t,x], \qquad \lambda \cdot [a,t,x] := [\lambda a , t , x], \quad \big| [a,t,x] \big| := \big| a \big( h_{t}(x) \big) \big| ,
\]
and the unique topology on $\LL$ such that the local sections $[t,x] \mapsto [a,t,x]$, for $x\in X_{t^*}$, $a\in C_0(X_t)$, $t\in S$, are continuous. 
Define the partial multiplication and involution on $\LL$ by
\begin{align*}
    [a,s,h_t(x)] \cdot [b,t,x]& := [\alpha_{s}(\alpha_{s}^{-1}(a) b) u(s,t),st,x] = [a (b\circ h_{s^*}) u(s,t),st,x], \\
    [b,t,x]^* &:= [\overline{b\circ h_{t}}\cdot \overline{u(t,t^*)} , t^* , h_{t}(x)] 
\end{align*}
for all $a\in C_0(X_s)$, $b\in C_0(X_t)$, $ x\in X_{(st)^*}$, $s,t\in S$. 

\begin{prop}\label{prop:twisted_groupoid_from_twisted_action}
The pair $(\G,\LL)$ described above is a well-defined twisted groupoid. %We call the twisted groupoid associated to the twisted action $(\alpha,u)$. 
\end{prop}
\begin{proof}
This follows from \cite[Theorem 3.22]{Buss_Exel2}, as the above construction is a special case of the construction in \cite[Section 3]{Buss_Exel2} applied to a Fell bundle over $S$ associated to $(\alpha,u)$, as described in \cite[p.\ 250]{Buss_Exel}.
\end{proof}

\begin{defn}
We say that a twisted inverse semigroup action $(\alpha,u)$ \emph{models a twisted \'etale groupoid}  $(\G,\L)$ if $(\G,\L)$ is isomorphic to the twisted groupoid associated to $(\alpha,u)$ as in Proposition~\ref{prop:twisted_groupoid_from_twisted_action}.
%the \emph{twisted groupoid associated to the twisted action} $(\alpha,u)$.
\end{defn}

\begin{lem}\label{lem:twisted_groupoids_come_from_twisted_actions}
Let $(\G,\LL)$ be a twisted \'etale groupoid. 
Let $S\subseteq S(\LL)\subseteq \Bis(\G)$ be any wide inverse semigroup and $(\alpha, u)$ a twisted action of $S$ as in Lemma~\ref{lem:twisted_action_from_twisted_groupoid}, obtained from a choice of unitary sections $c_U\in C_u(U,\LL)$, $U\in S$. 
Let $(S\ltimes X, \LL_{u})$ be the twisted groupoid associated to $(\alpha, u)$, as in Proposition~\ref{prop:twisted_groupoid_from_twisted_action}. 
Then
\[
    [a,U,x] \mapsto a \big( h_{U}(x) \big) c_U \big( d|_U^{-1}(x) \big), \qquad a\in C_0(r(U)),\ U\in S,\ x\in d(U),
\]
is an isomorphism of twisted groupoids $(S\ltimes X, \LL_{u}) \cong (\G,\LL)$. 
Thus $(\alpha, u)$ models $(\G,\LL)$.
\end{lem}
\begin{proof}
The map $S\ltimes X\ni [U,x]\mapsto d|_U^{-1}(x)\in \G$ is an isomorphism of topological groupoids. 
The sections $[U,x] \mapsto [a,U,x]$ and $\gamma\mapsto  a(r(\gamma)) c_U (\gamma)$ ($\gamma \in U$, $x\in d(U)$, $a\in C_0(r(U))$, $U\in S$) determine the topology of $\LL_{u}$ and $\LL$ respectively.
Thus the map described in the assertion is a homeomorphism. 
It is straightforward to check that it preserves the algebraic operations. 
\end{proof}





%
\subsection{Integration and disintegration of representations}

We fix a  twisted action $(\alpha,u)$ of an inverse semigroup $S$ on $C_0(X)$ and let $(\G,\LL)$ be the associated twisted groupoid as described in the previous subsection. 
By Lemma~\ref{lem:twisted_groupoids_come_from_twisted_actions} every twisted \'etale groupoid arises in this way. 
It follows from the construction that the sets $U_t = \{ [t,x] : x \in X_t \}$, $t\in S$, form a wide inverse semigroup of bisections of $\G$, and for each $t\in S$ we have a linear isometric isomorphism
\begin{equation}\label{eq:coeffcients_isomorphisms}
    C_c(X_{t})\ni a_t\mapsto a_t\delta_t\in C_c(U_t, \LL) , \qquad a_t\delta_t [t,x] := [a_t , t , x],\qquad x\in X_t,
\end{equation}
and we view $(C_c(U_t, \LL), \|\cdot\|_{\infty})$ as a subspace of $(\mathfrak{C}_c(\G,\LL), \|\cdot\|_{\max})$.

\begin{lem}\label{lem:grading_of_transformation_groupoid_algebra}  
 $\mathfrak{C}_c(\G,\LL)$ is spanned by elements $a_t\delta_t$, $a_t\in C_c(X_t)$, $t\in S$, and
\begin{enumerate}
%    \item\label{enu:grading_of_transformation_groupoid_algebra1}
    \item\label{enu:grading_of_transformation_groupoid_algebra2} $a_s \delta_s \cdot  a_t\delta_t%=\alpha_{s}(\alpha_{s}^{-1}(a_s)  a_t) u(s,t) \delta_{st}
    = a_s (b\circ h_{s^*}) u(s,t) \delta_{st}$ and $(a_t\delta_t)^*=\overline{a_{t}\circ h_{t}} \cdot \overline{u(t,t^*)}\delta_{t^*}$;
    \item\label{enu:grading_of_transformation_groupoid_algebra3} $s\leq t$ implies $X_{s}\subseteq X_{t}$ and $a \delta_s = a \overline{u(ss^*,t)}\delta_t$ for any $a \in C_c(X_{s})$.
\end{enumerate}
\end{lem}
\begin{proof} 
We only explain \ref{enu:grading_of_transformation_groupoid_algebra3} as the rest is straightforward. 
Let $s\leq t$. 
Then $X_{s}\subseteq X_{t}$ by the composition law (Remark~\ref{rem:twisted_relations}) and so also $U_{s} \subseteq U_{t}$. 
For $a \in C_c(X_{s})$ we have $a \overline{u(ss^*,t)}\in C_c(X_{s})\subseteq C_c(X_{t})$ and so  $a\delta_s\in  C_c(U_s,\LL)$ and $a \overline{u(ss^*,t)}\delta_t \in C_c(U_t,\LL)$.
Take $x\in X_{t^*}$. 
If $x\in X_{s^*}$ then $[t,x]= [s,x]$ and
\[ 
    a \overline{u(ss^*,t)} \delta_t [t,x] = [a\overline{u(ss^*,t)},t,x] = [a,s,x] = a \delta_s  [t,x]
\]
by the equivalence relations defining $\G$ and $\LL$. 
If $x\notin X_{s^*}$ then $a \delta_s [t,x] = 0$ by convention, and $a \overline{u(ss^*,t)}\delta_t [t,x] = [a\overline{u(ss^*,t)},t,x]=[0,t,x]=0$ because $a(h_t(x))=0$. 
Hence $a \delta_s  = a \overline{u(ss^*,t)}\delta_t$.
\end{proof}

%cf. \cite[Propositions 3.10, 7.5, 7.6]{Exel} for the untwisted case.
\begin{prop}\label{prop:integration_of_rep}
Every covariant representation $(\pi,v)$ of $(\alpha,u)$ in a Banach algebra $B$ extends uniquely to 
a representation $\pi\rtimes v : F(\G,\LL) \to B $ such that
\[
    \pi\rtimes v (a_t\delta_t) = \pi(a_t)v_t,\qquad  a_t\in C_c(X_{t}),\ t\in S.
\]
If $B$ is a  $C^*$-algebra then $\pi\rtimes v$ is $*$-preserving.
If $(\pi,\tilde{v})$ is a $B'$-normalization (or $B_*$-normalization) of $(\pi,v)$ as in Proposition~\ref{prop:normalized_twisted_rep} (or Proposition~\ref{prop:B_*_normalized_twisted_rep}) then $\pi\rtimes v = \pi\rtimes \tilde{v}$.
\end{prop}
\begin{proof}
We claim that the formula $\pi\rtimes v(\sum_{t\in F} a_t\delta_t) = \sum_{t\in F}\pi(a_t)v_t $, where $F\subseteq S$ is  finite, yields a well-defined map $\pi\rtimes v : \mathfrak{C}_c(\G,\LL) \to B$, or equivalently that $\sum_{t\in F} a_t\delta_t = 0$ implies $\sum_{t\in F} \pi(a_t) v_t = 0$. 
This is proved in \cite[Lemma 8.4]{Exel} in the untwisted case, under the assumption that $\G$ is second countable, using measure theoretical methods. 
We prove it in general using topological tools and induction on the cardinality of $F$. 
Suppose that $\sum_{t\in F} a_t\delta_t = 0$. 
If $|F|=1$ then $\sum_{t\in F}\pi(a_t)v_t = 0$ (because $a_t\delta_t=0$ if and only if $a_t=0$). 
Now assume the claim is true for all sets with cardinality smaller than that of $F$. 
Pick any $t_0\in F$ and put $F_0 = F\setminus \{t_0\}$.
Then $\sum_{t\in F_0} a_t\delta_t = -a_{t_0}\delta_{t_0}$, so the closed support of $a_{t_0}\delta_{t_0}$, which we denote by $K$, is a compact set covered by $\{U_{t}\cap U_{t_0}\}_{t\in F_0}$. 
By construction $U_{t}\cap U_{t_0} = \bigcup_{s\leq t, t_0} U_{s}$.
So for each $t\in F_0$ we may find $s_{1,t},...,s_{n_t,t}\leq t,t_0$ such that $K\subseteq \bigcup_{t\in F} \bigcup_{i=1}^{n_t} U_{s_{i,t}}$.
Let $\{ \varrho_{s_{i,t}} \}_{i=1...,n_t ,t\in F_0}$ be a partition of unity on $K$ subordinate to this  open cover.
Then by Lemma~\ref{lem:grading_of_transformation_groupoid_algebra}\ref{enu:grading_of_transformation_groupoid_algebra3} and Lemma~\ref{lem:range_of_covariant_rep}\ref{enu:range_of_covariant_rep2} (applied to each $t$, $2 n_t$-times) we get 
\[ 
    a_{t_0}\delta_{t_0} = \sum_{t\in F_0} \left(\sum_{i=1}^{n_t} \varrho_{s_{i,t}} a_{t_0} \frac{u(s_{i,t} s_{i,t}^*,t_0)}{u(s_{i,t} s_{i,t}^*,t) } \right) \delta_t, \qquad 
    \pi(a_{t_0}) v_{t_0} = \sum_{t\in F_0} \pi \left( \sum_{i=1}^{n_t} \varrho_{s_{i,t}} a_{t_0} \frac{u(s_{i,t} s_{i,t}^*,t_0)}{u(s_{i,t} s_{i,t}^*,t) } \right) v_t .
\]
Thus $\sum_{t\in F} a_t\delta_t = \sum_{t\in F_0} b_t \delta_t$ and $\sum_{t\in F} \pi(a_t)v_t = \sum_{t\in F_0} \pi(b_t)v_t$ for some $b_t\in C_c(X_t)$ ($t\in F_0$). 
Hence the claim follows by the inductive hypothesis.
Now Lemma~\ref{lem:range_of_covariant_rep} and Lemma~\ref{lem:grading_of_transformation_groupoid_algebra} readily imply that $\pi\rtimes v:\mathfrak{C}_c(\G) \to B$ is an algebra homomorphism, which is $*$-preserving if $B$ is a $C^*$-algebra.
It is $\|\cdot\|_{\max}$ contractive because $\| \pi\rtimes v(a_t\delta_t) \| = \| \pi(a_t)v_t \| \leq \|a_t\|_{\infty}$.
Hence it extends to a representation $\pi\rtimes v : F(\G,\LL) \to B$, which is $*$-preserving if $B$ is a $C^*$-algebra.
The last statement is clear. 
\end{proof}

The following lemma will allow us to show the converse to Proposition~\ref{prop:integration_of_rep}. 

\begin{lem}\label{lem:C*_unit_rep}
Let $\psi : A \to E$ be a contractive linear map from a $C^*$-algebra into a Banach space $E$ which has a predual Banach space $E_*$. 
There is a contractive element $\psi(1_{A})\in E$ such that for every  approximate unit $\{\mu_i\}_{i}\subseteq A$ we have $\psi(1_A) = E_*\mhyphen\lim_{i} \psi(\mu_i)\in E$.
%The same holds for a bounded $\R$-linear map $\psi:C_0^\R(X)\to K$ into a real Banach space $K$.
%If $\psi$ is contractive, then $\psi(1_A)$ is  contractive, and if $\psi$ is an algebra homomorphism, then $\psi(1_{A})$ is a projection. 
\end{lem}
\begin{proof}
For simplicity we assume $\F = \C$ (the real case can be reduced to the complex case by complexification). 
We identify $E_*$ with a subspace of $E'$. 
Let $f\in E_*$. 
Then $\psi' (f) = f\circ\psi : A\to \C$ is a bounded functional. 
Hence it decomposes to $\psi'(f) = \sum_{k=0}^3 i^{k}\tau_k$ where $\tau_k : A\to \C$, $k=0,...,3$, are positive functionals.
Applying the GNS-construction to each $\tau_k$, we get representations $\pi_{k} : A\to B(H_k)$ and cyclic vectors $\omega_k\in H_k$ such that $\psi'(f)(a) = \sum_{k=0}^3 i^{k} \langle \pi_k(a) \omega_k , \omega_k \rangle$.
Since $\{\pi_k(\mu_i)\}_{i}$ is strongly convergent to the identity on  $H_k$ we conclude that  $\{ f(\psi(\mu_i)) \}_{i}$ is convergent in $\C$ to the number $c_{f,\psi} := \sum_{k=0}^3 i^{k}\|\omega_k\|^2 = \sum_{k=0}^3 i^{k} \|\tau_k\|$ that depends only on $\psi$ and $f$ (it does not depend on $\{\mu_i\}_{i}$). 
Now, since the net $\{\psi(\mu_i)\}_{i}$ is bounded in $E$, the Banach--Alaoglu Theorem says that there is a subnet $\{\psi(\mu_{i_j})\}_{j}$ with an $E_*$-limit $\psi(1_A) := E_*\mhyphen\lim_{j} \psi(\mu_{i_j}) \in E$.  
So in particular $f(\psi(1_A))=\lim_{j} f(\psi(\mu_{i_j})) = c_{f,\psi}$ for every $f\in E_*$. 
This implies that every net $\{\psi(\mu_i)\}_{i}$, where $\{\mu_i\}_{i}$ is an approximate unit, is $E_*$-convergent to $\psi(1_A)$. 
\end{proof}

\begin{thm}[Disintegration Theorem]\label{thm:disintegration}
Let $(\G,\LL)$ be a twisted \'etale groupoid and let $(\alpha,u)$ be any twisted inverse semigroup action that models it.
Then $(\pi , v) \leftrightarrow \pi\rtimes v$ establishes a bijective correspondence between representations $\psi : F(\G,\LL) \to B$ in a Banach algebra $B$ and
\begin{enumerate} 
    \item \label{enu:disintegration1} $B'$-normalized covariant representations $(\pi,v)$ of $(\alpha,u)$ in $B$;
    \item \label{enu:disintegration2} $B_*$-normalized covariant representations $(\pi,v)$ of $(\alpha,u)$ in $B$, if $(B, B_*)$ is  a dual Banach algebra;
    \item \label{enu:disintegration3} covariant representations $(\pi,v)$ of $(\alpha,u)$ on $E$, if $B=B(E)$ and $E$ is a reflexive Banach space.
%\item \label{enu:disintegration4}  
\end{enumerate}
If each $X_t$, $t\in S$, is compact (which can always be arranged when $\G$ is ample) then the pairs $(\pi,v)$ in \ref{enu:disintegration1}--\ref{enu:disintegration3} coincide with covariant representations $(\pi,v)$ of $(\alpha,u)$ in $B$ such that $v_t = \pi(1_{X_t})v_{t}\in B$ for all $t\in S$.
\end{thm}
\begin{proof}
Let $\psi : F(\G,\LL) \to B$ be a representation. 
Put $\pi := \psi|_{C_0(X)}$. 
For every $t\in S$, the map $C_c(X_{t})\ni a_t \mapsto \psi(a_t\delta_t) \in B$ extends to a linear contraction $C_0(X_{t}) \to B\subseteq B''$. 
Let $v_t\in B''$ be the  element associated to this map in Lemma~\ref{lem:C*_unit_rep}, where $E=B''$ and $E_*=B'$. 
Namely, for any approximate unit $\{\mu_i^t\}_{i} \subseteq C_0(X_t)$ we have  $v_t = B'\mhyphen\lim_{i} \psi(\mu_i^t\delta_t)$. 
Recall that we consider multiplication in $B''$ which is $B'$-continuous in the second variable.
For $a\in C_c(X_t) = C_c(X_{tt^*})$ we have $(a \delta_{tt^*}) * (\mu_i^t \delta_t) \to a\delta_t$ in $C_0(U_t,\LL)\subseteq F(\G,\LL)$. 
Thus
\[
    \pi(a)v_t = \psi( a\delta_{tt^*} ) B'\mhyphen\lim_{i} \psi (\mu_i^t \delta_t) = B'\mhyphen\lim_{i} \psi (a\delta_{tt^*} * \mu_i^t \delta_t) = \psi(a\delta_t) .
\]
For $a\in C_c(X_{t^*})$ we have $(\mu_i^t \delta_t) * (a \delta_{t^*t}) = (\alpha_t(a) \mu_i^t) \delta_t \to  \alpha_t(a) \delta_{t}$ in $C_0(U_{t^*},\LL)\subseteq F(\G,\LL)$. 
Multiplication in $B''$ is continuous in the first variable when the second variable is in $B$, so
\[
    v_t \pi(a) = B'\mhyphen\lim_{i} \psi (\mu_i^t \delta_t * a \delta_{t^*t}) = \psi \big( \alpha_t(a) \delta_{t} \big) = \pi(\alpha_t(a)) v_t .
\]
If in addition $t = e \in E(S)$ then $v_e \pi(a) = \psi(\alpha_e(a) \delta_{e}) = \psi(a \delta_{e}) = \pi(a)$.
If $a\in C_c(X_{st})$ for $s,t\in S$ then $\alpha_{s}^{-1}(a) = a\circ h_{s} \in C_c(X_{s^*}\cap X_t)$ and therefore $(a\delta_s) * (\mu_i^t \delta_t) = a (\mu_i^t \circ h_{s^*}) u(s,t) \delta_{st} = \alpha_{s}(\alpha_{s}^{-1}(a) \mu_i^t) u(s,t)\delta_{st} \to a u(s,t) \delta_{t}$ in $C_0(U_{st},\LL)\subseteq F(\G,\LL)$. 
Thus
\[
    \pi(a)v_s v_t = B'\mhyphen\lim_{i}\psi(a \delta) \psi (\mu_i^t \delta_t) = \psi \big( au(s,t) \delta_{t} \big) = \pi \big( a u(s,t) \big) v_t. 
\]
By construction $v_t=B'\mhyphen\lim_{i} \p(\mu_i^t) v_t)$. 
Hence we see that $(\pi, v)$ is a $B'$-normalized covariant representation of $(\alpha,u)$ with $\psi = \pi\rtimes v$. 

This immediately gives \ref{enu:disintegration1} as a $B'$-normalized a covariant representation $(\pi, v)$ of $(\alpha,u)$ with $\psi=\pi\rtimes v$ has to be the one we constructed above.
Similarly, we get \ref{enu:disintegration2} and \ref{enu:disintegration3} by passing to an appropraite normalization as described in Proposition~\ref{prop:B_*_normalized_twisted_rep} and Corollary~\ref{cor:normalization_to_spatial_cov_rep}.
The last part of the assertion follows from Remark~\ref{rem:unital_actions}.
\end{proof}

\begin{cor}\label{cor:disintegration}
We have a natural isometric isomorphism $F(\G,\LL)\cong C_0(X)\rtimes_{(\alpha,u)} S$.  
Moreover, if $\EE$ is a class of Banach spaces then denoting by $\EE_{\alg}$ the class of all covariant representations of $(\alpha,u)$ in Banach algebras $B(E)$, where $E\in \EE$, and  by $\EE_{\spa}$ 
the class of all covariant representations of $(\alpha,u)$ on Banach spaces $E\in\EE$, then
\[
    F_{\EE}(\G,\LL) \cong C_0(X) \rtimes_{(\alpha,u),\EE_{\alg}} S \donto C_0(X)\rtimes_{(\alpha,u),\EE_{\spa}}S .
\]
The last homomorphism is isometric, for instance, if all spaces in $\EE$ are reflexive, or if each  $X_t$, $t\in S$, is compact.
\end{cor}
\begin{proof}
Apply Theorem~\ref{thm:disintegration}.
\end{proof}

\begin{cor}\label{cor:representations into C*-algebras}
For any $C^*$-algebra $B$, a homomorphism $F(\G,\LL)\to B$ is contractive (i.e. is a representation) if and only if it is $*$-preserving.
\end{cor}
\begin{proof} 
Any representation $\psi : F(\G,\LL) \to B$ is $*$-preserving because any integrated representation is, by Proposition~\ref{prop:integration_of_rep}.
Conversely, if $\psi$ is a $*$-homomorphism it is contractive on the $C^*$-algebra $C_0(X)\subseteq F(\G,\LL)$ and therefore it is contractive on every space $C_0(U, \LL)\subseteq F(\G,\LL)$, $U\in \Bis(\G)$, because  for $f\in C_0(U, \LL)$ we have $f * f^*\in C_0(X)$: 
\[
    \|\psi(f)\|^2 = \| \psi(f) \psi(f)^* \| = \| \psi(f \cdot f^*)\| \leq \| f \cdot f^* \|_{\infty} = \|f\|_{\infty}^2 .
\]
Thus $\psi$ is contractive on $\mathfrak{C}_c(\G, \LL)$ by \eqref{eq:projective_norm}.
\end{proof}



%
%
\section{Groupoid $L^p$-operator algebras}

Fix $p\in [1,\infty]$ and let $q\in [1,\infty]$ be the number satisfying $1/p + 1/q=1$, with the convention that $1/\infty=0$. 
Take a twisted \'etale groupoid $(\G,\LL)$ with a locally compact Hausdorff unit space $X$. 
Regular representations of $\G$ on $\ell^p$-spaces (with $p<\infty$) were considered in \cite{cgt}, and of $(\G,\LL)$ (but using a different picture of the twist) in \cite{Hetland_Ortega}. 
We spell out some details using our notation. %and applying our integration result (Proposition \ref{prop:integration_of_rep}).
We denote by $\ell^p(\G,\LL)$ the Banach space of all sections of $\LL$ for which the norm $\|f\|_{p} = (\sum_{\gamma\in \G} |f(\gamma)|^{p})^{1/p}$ when $p<\infty$ and
$\|f\|_{\infty} = \sup_{\gamma\in \G}|f(\gamma)|$ when $p=\infty$, is finite.
We have an isometric isomorphism $\ell^p(\G,\LL)\cong \ell^p(\G)$. % of Banach spaces. \marginpar{\tiny Say what the isomorphism is?}

%We use  a `disintegrated form' of a regular representation of an \'etale groupoid, to define a regular covariant representaion  for an inverse semigroup action:
\begin{prop}\label{prop:regular_disintegrated} 
For each $p\in [1,\infty]$ the formula
\[
    \Lambda_p(f) \xi (\gamma) := (f * \xi)(\gamma) = \sum_{r(\eta) = r(\gamma)} f(\eta)\cdot \xi(\eta^{-1}\cdot \gamma), \qquad f \in \mathfrak{C}_c(\G,\LL),\ \xi \in \ell^{p}(\G,\LL) ,
\]
defines a representation $\Lambda_p : F(\G,\LL) \to B(\ell^{p}(\G,\LL))$, which is injective on $\mathfrak{C}_c(\G,\LL)$, and  
\begin{enumerate}
%    \item\label{enu:regular_disintegrated0} 
    \item\label{enu:regular_disintegrated1} for $p \in (1 , \infty)$ we have $\| \Lambda_p(f) \| \leq \|f\|_{*d}^{1/p} \cdot \|f\|_{*r}^{1/q} \leq \|f\|_{I}$, while $\|\Lambda_1(f)\| = \|f\|_{*d}$ and $\|\Lambda_{\infty}(f)\|= \|f\|_{*r}$, for all $f \in \mathfrak{C}_c(\G)$; 
    \item\label{enu:regular_disintegrated2} putting $F^p_{\red}(\G,\LL) := \overline{\Lambda_p(\mathfrak{C}_c(\G,\LL))}$, we have $F^p_{\red}(\G,\LL)^{\op} \cong F^{q}_{\red}(\G,\LL^{\op})$, with the isometric isomorphism induced by the map $f \mapsto \widecheck{f}$. 
    In particular, $F^p_{\red}(\G)^{\op}\cong F^{q}_{\red}(\G)$.
\end{enumerate}
\end{prop}
\begin{proof} 
It is routine to check that the formula in the assertion gives a well defined homomorphism $\Lambda_p : \mathfrak{C}_c(\G,\LL) \to B(\ell^{p}(\G,\LL))$, and $\|\Lambda_p(f)\| \leq \|f\|_{\infty}$ for any $f\in C_c(U,\LL)$, $U\in \Bis(\G)$. 
Hence $\Lambda_p$ is $\|\cdot\|_{\max}$-contractive and thus it uniquely extends to a representation of $F(\G,\LL)$, see also Remark \ref{rem:not_important_remark} below. 
For each  $\gamma \in \G$ choose a norm one element $1_{\gamma}\in L_\gamma$ and treat it as a section of $\LL$ which is zero at $\eta \neq \gamma$.
Then $\{1_\gamma\}_{\gamma\in \G}$ is a Schauder basis for $\ell^{p}(\G,\LL)$ and for any $f\in  \mathfrak{C}_c(\G)$ we have 
\[
    \| \Lambda_p(f) 1_{\gamma} \|_p = \begin{cases} 
        \left(\sum_{d(\eta) = d(\gamma)} |f(\eta \gamma^{-1})|^p\right)^{1/p}, &  \text{ if $p<\infty$} \\
        \sup_{d(\eta)=d(\gamma)}|f(\eta \gamma^{-1})|, & \text{ if $p=\infty$}.
    \end{cases}
\]
%\end{equation}
This implies that $\Lambda_p$ is injective on $\mathfrak{C}_c(\G,\LL)$. 
Also using this one sees that  $\|\Lambda_1(f)\| = \|f\|_{*d}$ and $\|\Lambda_{\infty}(f)\| = \|f\|_{*r}$.
Then the inequalities $\|\Lambda_p(f)\|\leq \|f\|_{*d}^{1/p} \cdot \|f\|_{*r}^{1/q} \leq \|f\|_{I}$ follow from the Riesz--Thorin interpolation theorem.
This proves \ref{enu:regular_disintegrated1}. 
To see \ref{enu:regular_disintegrated2} it suffices to note that under the standard isomorphism $\ell^{p}(\G,\LL)' \cong \ell^{q}(\G,\LL^{\op})$ given by the paring $\langle \xi, \eta\rangle :=\sum_{\gamma\in \G}  \xi(\gamma) \eta(\gamma)$, for $\xi\in \ell^{p}(\G,\LL)$, $ \eta\in \ell^{q}(\G,\LL^{\op})$, we have  $\Lambda^{p}(f)' = \Lambda^{q}(\widecheck{f})$ for all $f \in \mathfrak{C}_c(\G)$ 
(note that $\xi(\gamma) \eta(\gamma)\in L_{r(\gamma)}=\F$ because the bundle $\LL$ is trivial over $X$). 
\end{proof}

\begin{rem}\label{rem:not_important_remark}
Let $\tilde{h}:\Bis(\G)\to \PHomeo(\G)$ be the extension of the canonical action described in Example~\ref{ex:actions_from_groupoids}. 
The formulas 
\[
    \pi(a)\xi(\gamma) := a(r(\gamma))\xi(\gamma) , \qquad 
    v_U\xi(\gamma) :=   \begin{cases}
                            \xi(\tilde{h}_{U^*}(\gamma)), &  \gamma\in r^{-1}(r(U)), \\
                            0 , & \text{ otherwise}, % \eta\in U \text{ with }r(\gamma)=r(\eta) \gamma\not\in r^{-1}(r(U)),
                        \end{cases}
\]
where $a\in C_0(X)$, $\xi \in \ell^p(\G)$, $\gamma \in \G$, $U\in \Bis(\G)$, define a covariant representation $(\pi, v)$ of the corresponding inverse semigroup action on $C_0(X)$ on $\ell^p(\G)$,
and $\Lambda_p = \pi\rtimes v : F(\G)\to B(\ell^p(\G))$. 
More generally, if $(\alpha, u)$ is a twisted inverse semigroup action of $S$ on $C_0(X)$ and $(\G,\LL)$ is the associated twisted groupoid, then the formulas
\[
    \pi(a) \xi[s,x] = a \big( h_s(x) \big) \xi([s,x]), \qquad 
    v_t\xi[s,x] =   \begin{cases}
                        u(t,t^*s) \big( h_{s}(x) \big) \xi \big( [t^*s,x] \big), &  x\in h_{s}^{-1}(X_{t}), \\
                        0 , & \text{otherwise}, % \eta\in U \text{ with }r(\gamma)=r(\eta) \gamma\not\in r^{-1}(r(U)),
                    \end{cases}
\]
where $a\in C_0(X)$, $\xi \in \ell^p(\G)$, $x\in X_{s^*}$, $t,s \in S$, define a covariant representation of $(\alpha,u)$ such that $\pi\rtimes v : F(\G,\LL) \to B(\ell^p(\G))$ is equivalent to $\Lambda_p : F(\G,\LL) \to B(\ell^p(\G,\LL))$.
\end{rem}

\begin{defn}
We call  $F^{p}_{\red}(\G,\LL)$ the \emph{reduced $L^p$-operator algebra of $(\G,\LL)$}.
\end{defn} 

\begin{rem}
This definition is consistent with those in \cite{cgt}, \cite{Gardella_Lupini17} and \cite{Hetland_Ortega} (where it is assumed that $p<\infty$). 
Moreover, $F^{2}_{\red}(\G,\LL) = C_{\red}^*(\G,\LL)$ is the reduced $C^*$-algebra of $(\G , \LL)$, and we have $F^{1}_{\red}(\G,\LL) \cong F_{d_*}(\G,\LL)$ and $F^{\infty}_{\red}(\G,\LL)\cong F_{r_*}(\G,\LL)$ by Proposition~\ref{prop:regular_disintegrated}\ref{enu:regular_disintegrated1}.
\end{rem}

\begin{defn} 
We define the \emph{full $L^p$-operator algebra} $F^p(\G,\LL)$ as $F_{\EE}(\G,\LL)$, where $\EE$ is the class of all $L^p$-spaces.
Thus $F^p(\G,\LL) = \overline{\mathfrak{C}_c(\G,\LL)}^{\|\cdot\|_{L^p} }$, where
\[
    \|f\|_{L^p} := \sup \{ \| \psi(f) \| : \text{$\psi : \mathfrak{C}_c(\G,\LL)\to B(L^p(\mu))$ is a representation for some measure $\mu$} \}.
\]
%For $p=\infty$ we put $F^{\infty}(\G) := \overline{\mathfrak{C}_c(\G)}^{\|\cdot\|_{L^{\infty}}}$ where
% $\|f\|_{L^{\infty}}:=\sup\{\|\psi(f)\|: \psi:F(\G)\to B(L^{\infty}(\mu))  \text{is  a non-degenerate representaion and }\mu\text{ is a  localizable measure}\}$.
\end{defn}

\begin{prop}\label{prop:initial_on_L^p_full}
For any $p\in [1,\infty]$  we have $F^p(\G,\LL)^{\op}\cong F^{q}(\G,\LL^{\op})$. 
Moreover,
\begin{enumerate}
    \item\label{enu:initial_on_L^p_full2} for $f\in \mathfrak{C}_c(\G,\LL)$ the norm $\|f\|_{L^{\infty}}$ can be calculated as the supremum of $\|\psi(f)\|$ over all representations $\psi : \mathfrak{C}_c(\G,\LL) \to B(E)$, where either $E=L^{\infty}(\mu)$ for a localizable measure $\mu$, or $E$ is a $C_0$-space, or $E$ is a Lindenstrauss space; %, i.e. $E'\cong L_1(\mu)$ for some (localizable) measure $\mu$
    \item\label{enu:initial_on_L^p_full1} $\|\cdot \|_{L^{2}} = \|\cdot\|_{C^*\max}$ is the largest $C^*$-norm on $\mathfrak{C}_c(\G,\LL)$, so $F^{2}(\G,\LL) = C^*(\G,\LL)$.
\end{enumerate}
\end{prop}
\begin{proof}
If $p\in (1,\infty)$, then $F^p(\G,\LL)^{\op}\cong F^{q}(\G,\LL^{\op})$ by Corollary~\ref{cor:reflexive_spaces_oppositie_iso}.
The isomorphism  $F^{\infty}(\G,\LL)^{\op} \cong F^{1}(\G,\LL^{\op})^{\op}$ %(which is equivalent to $F_{1}(\G)\cong F^{\infty}(\G)^{\op}$)
and statement \ref{enu:initial_on_L^p_full2} follow from Corollary~\ref{cor:L_infty_lindenstrauss_spaces_etc}.
Item \ref{enu:initial_on_L^p_full1} follows from Corollary~\ref{cor:representations into C*-algebras}.
\end{proof}

\begin{prop}\label{prop:from_degenerate_to_nondegenerate_Lp}
Let $p\in [1,\infty)$. 
For any representation $\psi:F(\G,\LL)\to B(L^p(\mu))$ there is an isometric isomorphism $\Phi : \overline{\psi(F(\G,\LL))L^p(\mu)} \to L^p(\overline{\mu})$, where $\overline{\mu}$ is localizable, and then the formula $\overline{\psi}(f) = \Phi(\psi(f)|_{\overline{\psi(F(\G,\LL))L^p(\mu)}})$ ($f\in F(\G,\LL)$) defines a non-degenerate representation $\overline{\psi} : F(\G) \to B(L^p(\overline{\mu}))$ such that $\|\overline{\psi}(f)\|=\|\psi(f)\|$ for all $f\in F(\G,\LL)$. 
\end{prop}
\begin{proof}
By Lemma~\ref{lem:approximate_units}, $F(\G,\LL)$ and $C_0(X)$ have a common contractive approximate unit $\{\mu_{i}\}_{i}$. 
In particular, $\overline{\psi(F(\G,\LL))L^p(\mu)}=\overline{\psi(C_0(X))L^p(\mu)}$ and as $L^p(\mu)$ does not contain an isomorphic copy of $c_0$ (see \cite[2.6]{Gardella_Thiel1}) there is a norm one projection $P : L^p(\mu) \to \overline{\psi(C_0(X))L^p(\mu)}$ by \cite[Corollary 2.6]{Gardella_Thiel1}. 
Thus there is a measure $\overline{\mu}$, which can be chosen to be localizable, and  an isometric isomorphism $\Phi : \overline{\psi(F(\G,\LL))L^p(\mu)} \to L^p(\overline{\mu})$ by \cite[Theorem 6]{Tzafriri}. 
Also for all $f\in F(\G,\LL)$ we have $\|\psi(f)\|=\|\psi(f)P\|$ because $\|\psi(f) \xi\| = \lim_{i} \|\psi(f) \psi(\mu_{i}) \xi \| = \lim_{i} \| \psi(f) P \psi(\mu_{i}) \xi \| \leq \|\psi(f)P \| \|\xi\|$. This implies the assertion.
\end{proof}

\begin{rem}
Proposition~\ref{prop:from_degenerate_to_nondegenerate_Lp} fails when $p=\infty$ and $X$ is not compact, see Corollary~\ref{cor:degeneracy_of_L_infty_representations}.
We do not know whether it extends to $C_0$ or Lindenstrauss spaces.
\end{rem}

\begin{cor}\label{cor:non-degenaracy_of_universal_norm}
For any $p\in [1,\infty)$ and $f\in \mathfrak{C}_c(\G,\LL)$ we have 
\[
\begin{split}
    \|f\|_{L^{p}} = &\sup \left\{ \|\psi(f)\| : \text{$\psi : F(\G,\LL)\to B(L^{p}(\mu))$ is a non-degenerate representation} \right. \\
        &\qquad \qquad \qquad \qquad \qquad \qquad \qquad \qquad \qquad \quad \left. \text{and $\mu$ is a localizable measure} \right\} .
\end{split}
\]
\end{cor}


We will now estimate the universal norms $\|\cdot\|_{L^p}$ using our disintegration theorem. 
To this end, until the end of this section \emph{we fix a twisted action $(\alpha, u)$ that models $(\G,\LL)$} as in Lemma~\ref{lem:twisted_groupoids_come_from_twisted_actions}. 
Thus we have $\G = S\ltimes_{h} X$ where $h : S \to \PHomeo(X)$ is an inverse semigroup action and $\alpha_t : C_0(X_{t^*}) \to C_0(X_{t})$;\ $\alpha_t(a) = a\circ h_{t^*}$, for $a\in C_0(X_{t})$, $t\in S$.


\begin{lem}\label{lem:estimates_for_L^p} 
Let $p \in [1,\infty)$ and let $(\pi,v)$ be a covariant representation of $(\alpha, u)$ in $B(L^{p}(\mu))$ for a localizable $\mu$, where $\pi$ acts by multiplication operators. 
Then 
\[
    \| \pi\rtimes v(f) \| \leq \| f \|_{*d}^{1/p} \|f\|_{*r}^{1/q} , \qquad f\in  F(\G,\LL) .
\]
%So $\|\pi\rtimes v(f)\|\leq \|f\|_{*d}$ for $p=1$ and $\|\pi\rtimes v(f)\|\leq \|f\|_{*r}$ for $p=\infty$.
In the case $p=q=2$ the estimate $\| \pi\rtimes v(f) \| \leq \| f \|_{*d}^{1/2} \cdot \| f \|_{*r}^{1/2}$, $f\in  F(\G,\LL)$, holds for any covariant representation $(\pi,v)$ of $(\alpha,u)$ on a Hilbert space.
\end{lem}
\begin{proof}
Let $f = \sum_{t\in F} f_t \delta_t \in \mathfrak{C}_c(\G,\LL)$, where $F \subseteq S$ is finite and $f_t\in C_c(X_t)$ (see equation \eqref{eq:coeffcients_isomorphisms}).
Let $\xi \in L^{p}(\mu)$ and $\eta \in L^{q}(\mu)$. %We write $\sign(z):= \overline{z}/|z|$ for $z\in \F\setminus \{0\}$ and $\sign(0):=0$. 
For any $a\in C_0(X_{t})$ we have $\int |\pi(a)v_t \xi |^p d\mu = \int | v_t \pi (a\circ h_{t}) \xi |^p d\mu \leq \int | \pi(a\circ h_{t}) \xi |^p d\mu$ because $v_t$ is contractive. 
Also for any $a\in C_0(X)$ and any continuous multiplicative real function $\varphi$, in particular $\varphi(x)=|x|^{\alpha}$, we have $\varphi \circ (\pi(a)\xi) = \pi(\varphi \circ a) (\varphi \circ\xi)$ because $\pi$ can be identified with a representation $C_0(X) \to L^{\infty}(\mu)$, and the functional calculus applies (also in the real case \cite[1.1.5]{Schroder}).
Thus $| \pi(a) \xi \cdot \eta | = | \pi (|a|^{1/p}) \xi | \, | \pi (|a|^{1/q}) \eta |$. 
Using this and applying H\"olders inequality to the measure $\sum_{t\in F} \mu$ we get
\[
\begin{split}
    \left| \int \big( \pi \rtimes v(f) \xi \big) \eta \, d\mu \right| &\leq \int \sum_{t\in F} \big| \pi ( |f_t|^{1/p} ) v_t \xi \big| \, \big| \pi ( |f_t|^{1/q} ) \eta \big| \, d\mu \\
        &\leq \left( \sum_{t\in F} \int \big| \pi ( |f_t|^{1/p} ) v_t \xi \big|^p \, d\mu \right)^{1/p}  \left( \sum_{t\in F} \int \big| \pi ( |f_t|^{1/q} ) \eta \big|^{q} \, d\mu \right)^{1/q} \\
        &\leq \left( \int \pi \left( \sum_{t\in F} |f_t| \circ h_{t} \right) |\xi|^p \, d\mu \right)^{1/p}  \left( \int \pi\left( \sum_{t\in F} |f_t| \right) |\eta|^{q} \, d\mu \right)^{1/q} \\
        &\leq \left\| \sum_{t\in F} |f_t| \circ h_{t} \right\|_{\infty}^{1/p} \, \| \xi \|_{p} \, \left\| \sum_{t\in F} |f_t| \right\|_{\infty}^{1/q} \, \|\eta\|_{q} \\
        &= \| f \|_{*d}^{1/p} \, \|f\|_{*r}^{1/q} \, \|\xi\|_{p} \, \|\eta\|_{q}. %the \, is for spacing
\end{split}
\]
This implies that $\| \pi\rtimes v(f) \| \leq \|f\|_{*d}^{1/p} \, \|f\|_{*r}^{1/q}$.

The above argument carries over to any covariant representation $(\pi,v)$ on a Hilbert space $H$. 
Indeed, in this case $\pi$ is necessarily a $*$-homomorphism. 
Thus writing $\sign(z) := \overline{z}/|z|$ for $z\in \F\setminus \{0\}$ and $\sign(0) := 0$, and using the Cauchy--Schwarz inequality, for any $\xi, \eta\in H$ we get 
\[
\begin{split}
    \left| \langle \pi\rtimes v(f)\xi , \eta \rangle \right| &= \left| \sum_{t\in F} \Big\langle  \pi \big( |f_t|^{1/2} \big) v_t \xi , \pi \big( \overline{\sign(f_t)} |f_t|^{1/2} \big) \eta \Big\rangle \right| \\
        &\leq \left( \sum_{t\in F} \big\| \pi ( |f_t|^{1/2} ) v_t \xi \big\|^2 \right)^{1/2}  \left( \sum_{t\in F} \big\| \pi ( \overline{\sign( f_t )} |f_t|^{1/2} ) \eta \big\|^{2} \right)^{1/2} \\
        &= \left( \Big\langle \pi \Big( \sum_{t\in F} |f_t|\circ h_{t}\Big)\xi, \xi \Big\rangle \right)^{1/2} \left( \Big\langle \pi \Big( \sum_{t\in F}|f_t| \Big) \eta , \eta \Big\rangle \right)^{1/2} \\
        &\leq \Big\| \sum_{t\in F} |f_t|\circ h_{t} \Big\|_{\infty}^{1/2} \, \|\xi\| \, \Big\| \sum_{t\in F} |f_t|\Big\|_{\infty}^{1/2} \, \|\eta\| 
        = \| f \|_{*d}^{1/2} \, \|f\|_{*r}^{1/2} \, \|\xi\| \, \|\eta\|,
\end{split}
\]
which implies that $\|\pi\rtimes v(f)\|\leq \|f\|_{*d}^{1/2} \, \|f\|_{*r}^{1/2}$. 
\end{proof}

\begin{cor}\label{cor:C*-max_norm_estimate}
We have $\| f \|_{C^*\max} = \| f \|_{L^2} \leq \| f \|_{*d}^{1/2} \, \|f\|_{*r}^{1/2} \leq \|f\|_{I}$ 
for any $f\in \mathfrak{C}_c(\G,\LL)$.
\end{cor}
\begin{proof}
Clearly, we have $\| f \|_{C^*\max} = \|f\|_{L^2}$ as every $C^*$-algebra embeds into $B(L^2(\mu))$ for some $\mu$. 
The estimates follow from  Lemma~\ref{lem:estimates_for_L^p} and Corollary~\ref{cor:disintegration}.
\end{proof}

In order to apply Lemma~\ref{lem:estimates_for_L^p} for $p\neq 2$ we will use Theorem~\ref{thm:L^p_representations of C_0(X)}.
Therefore until the end of this section we will mainly consider the case when $\F=\C$.  
If an analogue of Theorem~\ref{thm:L^p_representations of C_0(X)} holds in the real case (see Remark~\ref{rem:real_representations^problem}) then the remaining results in this section also go through in the real case.

\begin{thm}\label{thm:norm_estimates_L^p}  
 Assume $\F=\C$. 
For any $p\in [1,\infty]$ we have 
\[
    \| f \|_{L^p} \leq \| f \|_{*d}^{1/p} \, \| f \|_{*r}^{1/q} \leq \|f\|_{I}, \qquad f\in  \mathfrak{C}_c(\G,\LL).
\]
Moreover, $\| f \|_{L^1} = \| f \|_{L^1_{\red}} = \|f\|_{*d}$ and $\| f \|_{L^{\infty}} = \| f \|_{L^\infty_{\red}} = \|f\|_{*r}$, and so $F^{1}(\G,\LL) = F^{1}_{\red}(\G,\LL) = F_{*d}(\G,\LL)$ and $F^{\infty}(\G,\LL) = F^{\infty}_{\red}(\G,\LL) = F_{*r}(\G,\LL)$. 
\end{thm}
\begin{proof}
By Corollary~\ref{cor:C*-max_norm_estimate} we may assume that $p\neq 2$. 
For $p<\infty$ the first part of the assertion follows from the first part of Lemma~\ref{lem:estimates_for_L^p} combined with Corollary~\ref{cor:non-degenaracy_of_universal_norm} and  Theorem~\ref{thm:L^p_representations of C_0(X)}. 
This also implies the case $p=\infty$ because $F^{\infty}(\G,\LL)^{\op} \cong F_{1}(\G,\LL^{\op})$ by Proposition~\ref{prop:initial_on_L^p_full}.
The second part follows from Proposition~\ref{prop:regular_disintegrated}\ref{enu:regular_disintegrated1}
\end{proof}

We now fix a localizable measure space $(\Omega, \Sigma,\mu)$ and discuss covariant representations of $\alpha$ that are given by data coming from the space $(\Omega, \Sigma,\mu)$. 

\begin{defn}\label{defn:covariant_representation_spatial}
A covariant representation of $(\alpha,u)$ on $L^p(\mu)$, $p\in [1,\infty]$, as in Definition~\ref{defn:covariant_representation_in_algebra}, is called \emph{spatial} if $\pi : C_0(X) \to B(L^p(\mu))$ acts by multiplication operators on $L^p(\mu)$ and $v : S \to \SPIso(L^p(\mu))$ takes values in the inverse semigroup of spatial partial isometries (Definition~\ref{def:spatial^partial_isos}).
\end{defn}

%Using Theorems~\ref{thm:L^p_representations of C_0(X)}, \ref{thm:spatial^partial_isometries_description} and Ando's description of contractive projections in $L^p$-spaces for $p\in (1,\infty)\setminus \{2\}$, see \cite{Ando66}, \cite{Bernau_Lacey} we get the following. 

\begin{prop} \label{prop:covariant_rep_is_necessarily_spatial}
 Assume $\F=\C$.
If $p\in (1,\infty) \setminus \{2\}$ then every non-degenerate covariant representation $(\pi,v)$ of $(\alpha,u)$ on $L^p(\mu)$ is spatial. 
%So up to an isometric isomorphism of the $L^p$-space every covariant representation of $\alpha$ on $L^p$-space is spatial.
\end{prop}  
\begin{proof} The map $v$ takes values in $\SPIso(L^p(\mu))$ by Corollary~\ref{cor:covariant_reps_implies_MP^partial_isos} and Theorem~\ref{thm:spatial^partial_isometries_description}, and
 $\pi$ acts by multiplication operators  by Theorem~\ref{thm:L^p_representations of C_0(X)}.
\end{proof}

In what follows we use the notation of Subsection~\ref{subsect:Partial isometries}.

\begin{defn}\label{defn:covariant_triple_spatial}
A \emph{spatial covariant triple} $(\pi_0,\Phi, \omega)$ for $(\alpha,u)$ and $\mu$ consists of 
a representation $\pi_0 : C_0(X)\to L^\infty(\mu)$, 
an inverse semigroup $\Phi = \{ [\Phi_s] \}_{s\in S}\subseteq \PAut([\Sigma])$ of partial set automorphisms $\Phi_{s} : \Sigma_{D_{s}} \to \Sigma_{D_{s^*}}$,
$D_{s}, D_{s^*}\in \Sigma$, $s\in S$, satisfying $[\Phi_s]\circ [\Phi_t]=[\Phi_{st}]$ for $s,t\in S$,
and a family (cocycle) $\omega = \{\omega_{s}\}_{s\in S}$ of partial unimodal maps $\omega_s\in UL^{\infty}(\mu|_{D_{s^*}})$, $s\in S$. 
These must satisfy: 
\begin{enumerate}[labelindent=40pt,label={(SCR\arabic*)},itemindent=1em]
    \item\label{enu:covariant_representation_spatial1} $T_{\Phi_t}(\pi_0(a)) = \pi_0(\alpha_t(a))$ for all $a\in C_0(X_t)$, $t\in S$; %(\emph{covariance});
    \item\label{enu:covariant_representation_spatial2} $\pi_0(\mu_i^e) 1_{D}\nearrow 1_D$, for every  measurable $D\subseteq D_e$ with $\mu(A)<\infty$, every positive  approximate unit $\{\mu_i^e\}_{i} \subseteq C_0(X_e)$ and all $e\in E(S)$;% (\emph{projection non-degeneracy});
    \item\label{enu:covariant_representation_spatial3} $\pi_0(a)\omega_{s}T_{\Phi_s}(\omega_t) = \pi_0(au(s,t))\omega_{st}$ $\mu$-almost everywhere, for $s,t \in S$, $a\in C_0(X_{st})$. % (\emph{cocycle}).
\end{enumerate}
We say that $(\pi_0, \Phi, \omega)$ is \emph{non-degenerate} if $\pi_0(\mu_i) 1_{D}\nearrow 1_D$, for every measurable $D$ with $\mu(D)<\infty$ and  a positive  contractive approximate unit $\{\mu_i\}_{i}
\subseteq C_0(X)$. 
\end{defn}

\begin{prop} \label{prop:spatial_cov_reps22222}
 For  each $p\in [1,\infty)$ we have a bijective correspondence between spatial covariant representations $(\pi, v)$ of $(\alpha , u)$ on $L^{p}(\mu)$ and spatial covariant triples $(\pi_0, \Phi, \omega)$ for $(\alpha , u)$ and $\mu$. This correspondence is given by 
\[ 
    \big( \pi(a)\xi \big) (\omega) = \pi_0(a)(\omega) \xi(\omega), \quad v_{t} := \omega_t  \left( \frac{d\mu \circ \Phi_{t*}}{d\mu|_{D_{\Phi_{t^*}}}} \right)^{\frac{1}{p}} T_{\Phi_{t}}, \quad a\in C_0(X),\ \xi \in L^p(\mu),\ t\in S.
\]
The representation $(\pi, v)$ is non-degenerate if and only if the corresponding triple $(\pi_0,\Phi, \omega)$ is non-degenerate.
\end{prop}
\begin{proof}
A representation $\pi : C_0(X)\to B(L^p(\mu))$ acting by multiplication operators is equivalent to
a homomorphism $\pi_0 : C_0(X)\to L^\infty(\mu)$. 
By Proposition~\ref{prop:spatial^partial_isometries}, every map $v : S \to \SPIso(L^p(\mu))$ is given by  $v_{t} = \omega_t (\frac{d\mu\circ\Phi^{*}}{d\mu|_{D_{\Phi^*}}})^{\frac{1}{p}} T_{\Phi_{t}}$, $t\in S$, where $\Phi = \{ [\Phi_s] \}_{s\in S} \subseteq \PAut([\Sigma])$ and $\omega_t\in UL^{\infty}(\mu|_{D_{t^*}})$, $t\in S$. 
The final statement of Proposition~\ref{prop:spatial^partial_isometries} (and uniqueness of the presentation in Banach-Lamperti theorem, see Proposition \ref{prop:group_of_spatial_isometries}) tells us that $\pi(a) v_s v_t = \pi (au(s,t)) v_{st}$ holds for all $s,t \in S$, $a\in C_0(X_{st})$ if and only if   \ref{enu:covariant_representation_spatial3} in Definition~\ref{defn:covariant_triple_spatial} holds and $\Phi=\{[\Phi_s]\}_{s\in S}\subseteq \PAut([\Sigma])$ is an inverse semigroup. 
Condition \ref{enu:covariant_representation_spatial2} in Definition~\ref{defn:covariant_triple_spatial} 
is equivalent to the equality $\overline{\pi(C_0(X_e))L^{p}(\mu)} = L^{p}(\mu|_{D_e})$ for all $e\in E(S)$.    
Assuming this, condition \ref{enu:covariant_representation_spatial1} in Definition~\ref{defn:covariant_triple_spatial} is equivalent to $v_t \pi(a) v_{t^*} = \pi(\alpha_t(a))$  for all $a\in C_0(X_{t})$, $t\in S$. 
This gives the assertion.
\end{proof}

\begin{thm}\label{thm:spatial_representations_of_groupoid_algebras}
 
Let $(\G,\LL)$ be a twisted groupoid and let $(\alpha,u)$ be any twisted inverse semigroup action that models $(\G,\LL)$.
\begin{enumerate}
    \item \label{enu:spatial_representations_of_groupoid_algebras1} For each $p\in [1,\infty]\setminus\{2\}$ there is a spatial covariant representation $(\pi,v)$ of $\alpha$ on $L^p(\mu)$, for some localizable measure $\mu$, such that $\pi\rtimes v : F^p(\G,\LL)\to B(L^p(\mu))$ is isometric. If  $p< \infty$ then $\pi$ can be chosen to be non-degenerate.
    \item \label{enu:spatial_representations_of_groupoid_algebras2} For $p\in (1,\infty) \setminus\{2\}$ every non-degenerate representation of $F^p(\G,\LL)$ on $L^p(\mu)$, for some localizable $\mu$, is of the form $\pi\rtimes v$ for a spatial covariant representation $(\pi,v)$ of $(\alpha,u)$. This gives  a bijective correspondence between non-degenerate representations of $F^p(\G,\LL)$ on $L^p(\mu)$ and non-degenerate spatial covariant triples for $(\alpha,u)$ and $\mu$.
    \item\label{enu:spatial_representations_of_groupoid_algebras3} For any $p, p'\in (1,\infty)\setminus\{2\}$ we have a bijective correspondence between non-degenerate representations of $F^p(\G,\LL)$ and $F^{p'}(\G,\LL)$ on $L^p$ and $L^{p'}$-spaces, respectively. This correspondence matches $\pi_p : F^p(\G,\LL)\to B(L^p(\mu))$ and $\pi_{p'} : F^{p'}(\G,\LL) \to B(L^{p'}(\mu))$, given by 
\[
    \pi_p (a_t \delta_t) = \pi_0(a_t) \omega_t \left( \frac{d\mu\circ\Phi_{t^*}}{d\mu|_{D_{\Phi_{t^*}}}} \right)^{\frac{1}{p}} T_{\Phi_{t}}, \qquad  
    \pi_{p'} (a_t\delta_t) = \pi_0(a_t) \omega_t \left( \frac{d\mu\circ\Phi_{t^{*}}}{d\mu|_{D_{\Phi_{t^*}}}} \right)^{\frac{1}{p'}} T_{\Phi_{t}},
\]
    where $a_t\in C_c(X_{t}), t\in S$, and $(\pi_0,\Phi, \omega)$ is a non-degenerate spatial covariant triple for $(\alpha,u)$ and a localizable measure $\mu$.
\end{enumerate} 
\end{thm}
\begin{proof} 
\ref{enu:spatial_representations_of_groupoid_algebras2} By Theorem~\ref{thm:disintegration}\ref{enu:disintegration3} every representation of $F^p(\G,\LL)$ on  $L^p(\mu)$ is of the form $\pi\rtimes v$ for a covariant represenation $(\pi,v)$. 
If $\mu$ is localizable, and the representation is non-degenerate, then $(\pi,v)$ is necessarily spatial by Proposition~\ref{prop:covariant_rep_is_necessarily_spatial}, and so it is given by a spatial covariant triple by Proposition~\ref{prop:spatial_cov_reps22222}. 
This proves \ref{enu:spatial_representations_of_groupoid_algebras2}.
Item \ref{enu:spatial_representations_of_groupoid_algebras3} follows readily. %from \ref{enu:spatial_representations_of_groupoid_algebras2}. 

\ref{enu:spatial_representations_of_groupoid_algebras1} If $p\in (1,\infty)\setminus\{2\}$ the assertion follows from \ref{enu:spatial_representations_of_groupoid_algebras2} and Proposition~\ref{prop:from_degenerate_to_nondegenerate_Lp}. 
For $p\in \{1,\infty\}$ we get \ref{enu:spatial_representations_of_groupoid_algebras1} by Theorem~\ref{thm:norm_estimates_L^p}, because regular representations are given by spatial covariant representations. 
%Item \ref{enu:spatial_representations_of_groupoid_algebras3} follows readily from \ref{enu:spatial_representations_of_groupoid_algebras2}. 
\end{proof}


%
%
\section{Examples}

%
\subsection{Crossed products by twisted partial group actions} 
\label{subsect:twisted_partial_group_actions}

Twisted partial actions of a discrete group on $C^*$-algebras and their crossed products were gradually introduced in \cite{Exel_circle}, \cite{McClanahan}, and \cite{Exel_twisted_partial}, and the literature on that subject is now vast, see \cite{Exel_book}. 
We will consider only partial actions on commutative Banach algebras $A=C_0(X)$, which are equivalent to partial actions on the space $X$.

\begin{defn}
A \emph{partial action of a discrete group} $G$ on a locally compact Hausdorff space $X$ is a map $\theta : G\to \PHomeo(X)$ such that $\theta_1 = \id_X$, the partial homeomorphism $\theta_{ts}$ extends $\theta_{t} \circ \theta_s$, and $\theta_{t}^{-1} = \theta_{t^{-1}}$ for all $t,s\in G$. 
\end{defn}

We fix a partial action $\theta : G \to \PHomeo(X)$ of a group $G$. 
Note that in general this is not the same as an action of the group $G$ viewed as an inverse semigroup. 
Namely, denoting by $X_t$ the domain of $\theta_{t}$ we have homeomorphisms $\theta_t : X_{t} \to X_{t^{-1}}$ ($t\in G$) such that $\theta_{s^{-1}}(X_t\cap X_{s^{-1}}) = X_{s}\cap X_{ts} \subseteq X_{ts}$, so that the composition $\theta_{t}\circ\theta_s$ is defined on a subset of the domain of $\theta_{ts}$, rather than on all of $X_{ts}$, for all $s,t\in G$. 
However, $\theta$ gives rise to an action  the universal inverse semigroup $S(G)$ introduced in \cite{Exel^partial_vs_inverse_semigroup}.
By definition $S(G)$ is the universal semigroup generated by elements $[t]$, where $t\in G$, subject to the relations
\[ 
    [s] [t] [t^{-1}] = [st][t^{-1}], \qquad 
    [s^{-1}] [s] [t] = [s^{-1}][st], \qquad 
    [t][1] = [t], \qquad 
    [1][t] = [t],
\]
for all $s,t\in G$.
Then $S(G)$ is a unital inverse semigroup, $G\ni t\mapsto [t]\in S(G)$ is a unital semigroup embedding, and $[t]^* = [t^{-1}]$ for all $t\in G$, see \cite{Exel^partial_vs_inverse_semigroup}.
Moreover, putting $e_t := [t^{-1}][t]$ for $t\in G$ every $\overline{t}\in S(G)$ has a \emph{standard form}:
\begin{equation}\label{eq:standard_from}
    \overline{t} = e_{t_1} \cdots e_{t_n} [t_0], \qquad t_0, \ldots , t_n \in G,
\end{equation}
where the $t_i$ are mutually distinct and different from $1$ and $t_0$ for $1\leq i< j\neq n \in \N$, and this form is unique up to the order of the $e_{t_i}$s.
By \cite[Theorem 4.2]{Exel^partial_vs_inverse_semigroup} we have a bijective correspondence between partial actions of $G$ and actions of $S(G)$. 
Thus there is a unique action $[\theta] : S(G) \to \PHomeo(X)$ such that $[\theta]_{[t]} = \theta_t$ for $t\in G$. 
Namely, if $\overline{t}$ is of the form \eqref{eq:standard_from} we put $X_{\overline{t}} := \bigcap_{k=0}^n X_{t_k}$, and then
\begin{equation}\label{eq:theta_extended}
    [\theta]_{\overline{t}} = \theta_{t_0}|_{X_{\overline{t}^{-1}}} : X_{\overline{t}^{-1}} \to X_{\overline{t}}.
\end{equation}
This can be extended to the twisted case as follows, cf. \cite[Theorems 4.2, 4.3]{Sieben98}. 

\begin{defn}
A \emph{twist of the partial action $\theta$}, or of the associated algebraic partial action $\alpha : G\to \PAut(C_0(X))$, where $\alpha_t(a) := a\circ h_{t}^{-1}$, is a family of maps $u(s,t) : G \times G \to C_u(X_{s}\cap X_{st})$ ($s,t\in G$) satisfying $u(1,t) = u(t,1) = 1$ for $t\in G$, and
\[
    \alpha_r \big( a u(s,t) \big) u(r,st) = \alpha_r(a ) u(r,s) u(rs,t), \qquad a \in C_0(X_{r^{-1}}\cap X_{s}\cap X_{st}),\ r,s,t\in G.
\]
\end{defn}

We fix a twist $u = \{u(s,t)\}_{s,t\in G}$ of $\theta$.
There is a unique twisted action $([\alpha], [u])$ of $S(G)$ such that $[\alpha]_{[t]} = \alpha_t$ and $[u]([s],[t]) = u(s,t)$ for $s,t\in G$. 
Namely, $[\alpha] : S(G)\to \PAut(C_0(X))$ is the algebraic action associated to $[\theta] : S(G) \to \PHomeo(X)$, and if $\overline{t }= [t_0] e_{t_1} \cdots e_{t_n}$  and $\overline{s} = e_{s_1} \cdots e_{s_m}[s_0]$ are in standard form, then $[u](\overline{s},\overline{t})$ is the restriction of 
$u(s_0,t_0)$ to $X_{\overline{s} \overline{t}} \subseteq X_{s_0}\cap X_{s_0 t_0}$.

\begin{defn}
The \emph{transformation groupoid for $\theta$} is  $\G_{\theta} := \bigsqcup_{t\in G} \{t\}\times X_{t^{-1}}$, equipped with the product topology inherited from  $G\times X$, where the range and domain maps and the multiplication are defined by 
\[
    r(t,x) := h_t(x) , \qquad 
    d(t,x) := x, \qquad 
    \big( s, h_t(x) \big) \cdot (t,x) := (s t,x) , \qquad s,t \in G,\ x \in X_{t^{-1}} . 
\]
The twist associated to $u$ is $\LL_{u} = \F\times \G_{\theta}$ with the product topology, obvious Banach space structure in the fibers $\F \times (t,x)$, and multiplication and involution given by
\[
    (a, s,h_t(x)) \cdot (b, t,x) := \big( a b u(s,t) (h_{st}(x) ), s t,x \big) , \qquad 
    (b,t,x)^* := \big( \overline{b} \cdot \overline{u(t,t^{-1})}(x), t^{-1},h_{t}(x) \big) .
\]
\end{defn}

\begin{lem}\label{lem:isomorphism_of_transform_groupoids}
The twisted groupoids $(\G_{\theta},\LL_u)$ and $(S(G)\ltimes_{[\theta]} X, \LL_{[u]})$ associated to $(\theta,u)$ and $([\theta],[u])$ are naturally isomorphic.
\end{lem}
\begin{proof}
Note that if $\overline{t} = e_{t_1} \cdots e_{t_n}[t_0]$ and $\overline{s} = e_{s_1} \cdots e_{s_m}[s_0]$ are in a standard form \eqref{eq:standard_from} then $\overline{s} \leq \overline{t}$ if and only if $t_0 = s_0$ and $\{t_1, \ldots ,t_n\}\subseteq \{s_1, \ldots ,s_m\}$.
Hence in view of \eqref{eq:theta_extended} we have $[\overline{t},x] = [[t_0],x]$ for any $x\in X_{\overline{t}^{-1}}$.
Using this we get that the map $\G_{\theta}\ni (t,x) \mapsto [[t],x]\in S(G)\ltimes_{[\theta]} X$ is bijective. 
Now it is straightforward to see that this is in fact an isomorphism of topological groupoids, and 
that it extends to an isomorphism of twists $\LL_u \cong \LL_{[u]}$ where $\LL_{[u]}$ is described just before Proposition~\ref{prop:twisted_groupoid_from_twisted_action}.
\end{proof}

The space $\ell^1(\alpha,u) := \{ f \in \ell^1(G,C_0(X)): f(t)\in C_0(X_{t}), t\in S \}$ with operations 
\[
    (f * g)(r) := \sum_{st=r} \alpha_s \Big( \alpha_s^{-1} \big( f(s) \big) g(t) \Big) u(s,t) , \qquad 
    f^*(t) := \alpha_{t}^{-1} \big( a (t^{-1})^* \big) u(t^{-1},t)^*,
\]
for $f,g \in \ell^1(\alpha,u)$ and $s,t,r\in G$, is a Banach $*$-algebra which we now show is a \emph{Banach algebra crossed product}. 

\begin{thm}\label{thm:partial_actions}
Let $(\theta,u)$ be a twisted partial action of a group $G$ on $X$. 
We have natural isometric isomorphisms 
\[
    \ell^1(\theta,u)\cong F(\G_{\theta},\LL_u)\cong F(S(G)\ltimes_{[\theta]} X, \LL_{[u]})\cong C_0(G)\rtimes_{[\alpha],[u]} S(G) .
\]
\end{thm}
\begin{proof}
The isometric isomorphisms $F(\G_{\theta},\LL_u)\cong F(S(G)\ltimes_{[\theta]} X, \LL_{[u]})\cong C_0(G)\rtimes_{[\alpha],[u]} S(G)$ follow from Lemma~\ref{lem:isomorphism_of_transform_groupoids} and Corollary~\ref{cor:disintegration}. 
It is straightforward to check that the map $C_c(\G_{\theta},\LL_{u}) \ni \hat{f} \mapsto f \in \ell^{1}(\theta,u)$ where $f(t)(x) := \hat{f}(t^{-1},x)$ is an injective $*$-homomorphism that extends to a representation $\Psi : F(\G_{\theta},\LL_u) \donto \ell^{1}(\theta,u)$.
Also one readily checks that every covariant representation of $(\pi,v)$ of $([\theta],[u])$ defines a representation of $\ell^{1}(\theta,u)$ via $\pi\rtimes v(a)=\sum_{t\in G} \pi(a(t))v_{[t]}$. 
Disintegrating the identity representation $F(\G_{\theta},\LL_u)\to F(\G_{\theta},\LL_u)$ we get a covariant representation $(\pi,[v])$ of $([\theta],[u])$ such that the associated representation $\pi\rtimes v:\ell^{1}(\theta,u)\to F(\G_{\theta},\LL_u)$ is inverse to $\Psi$. 
Thus $\ell^1(\theta,u)\cong F(\G_{\theta},\LL_u)$.
\end{proof}

The $C^*$-envelope (maximal $C^*$-completion) of $\ell^1(\alpha,u)$ is the $C^*$-algebraic crossed product as in \cite{Exel_twisted_partial}. 
In the untwisted case ($u\equiv 1$), these crossed products were defined in \cite{McClanahan}, \cite{Sieben} and representations of $\ell^1(\alpha)$ on Hilbert spaces were characterised as appropriate representations of $\alpha$ in \cite{Quigg-Raeburn}, and as representations of a groupoid algebra in \cite{Abadie}. 
Theorem~\ref{thm:partial_actions_representations} allows one to generalize these results to representations of $\ell^1(\alpha,u)$ in Banach algebras.
Namely, all representations of $\ell^1(\alpha,u)$ come from covariant representations of $([\alpha],[u])$, which can be described in terms of $(\theta, u)$. 
As an illustration we give more details in the untwisted case. 

\begin{defn}
A \emph{partial representation} of the group $G$ in a Banach algebra $B$ is a map $v : G \to B_1$ such that,  for all $s,t \in G$,
\[
    v_{s} v_{t} v_{t^{-1}}=v_{st}v_{t^{-1}} , \qquad v_{s^{-1}} v_{s} v_{t}=v_{s^{-1}}v_{st} , \qquad v_{t} v_{1} =v_{t} , \qquad v_{1} v_{t} = v_{t}.
\]
\end{defn}

\begin{rem}\label{rem:partial_reps_vs_reps}
By definition of $S(G)$ we have a bijective correspondence between partial representations $v : G\to B_1$ of $G$ and representations (semigroup homomorphisms) $[v] : S(G) \to B_1$, given by $[v]_{[t]} = v_t$. 
Thus each $v_t$ is a partial isometry in $B$, and if $B$ is a $C^*$-algebra (complex or real), then a map $v : G\to B_1$ is a partial representation if and only if
$v_{s} v_{t} v_{t^{-1}}=v_{st}v_{t^{-1}}$, $v_{t}^*=v_{t^{-1}}$, $v_{t} v_{1}=v_t$, for all $s,t\in G$.
%and then $v$ is non-degenerate if and only if $B$ is unital and $v_1$ is the unit in $B$.
%This follows from the fact that partial isometries in $C^*$-algebras have unique generalized inverses.
\end{rem}

\begin{defn}\label{defn:covariant^partial_representation_in_algebra}
A \emph{covariant representation of the partial action $\theta$ in a Banach algebra $B$} is a pair $(\pi,v)$ where $\pi : A \to B$ is a representation and $v : G \to (B'')_{1}$ is a partial representation, satisfying 
\[
    v_t \pi ( a\circ\theta_{t} ) = \pi(a) v_{t} \in B \quad \text{ and }\quad  v_{t} v_{t^{-1}} \pi(a) = \pi(a), \qquad a\in C_0(X_{t}) ,\ t\in G .
\]
A \emph{covariant representation of $\theta$ on a Banach space $E$} is a pair $(\pi,v)$ where $\pi : A \to B(E)$ is a representation and $v : G \to B(E)_1$ is a partial representation such that
\[
    v_{t} \pi ( a\circ \theta_{t} ) v_{t^{-1}} = \pi(a) , \quad \text{ and } \quad  v_{t} v_{t^{-1}} E = \overline{\pi(C_0(X_{t}))E}, \qquad a\in C_0(X_{t}) ,\  t\in G . 
\]
\end{defn}

\begin{lem}
For every covariant representation $(\pi, v)$ of the partial action $\theta : G \to \PHomeo(X)$ there is a unique covariant representation $(\pi, [v])$ of $[\alpha] : S(G) \to \PAut(C_0(X))$ in $B$, where $v_t=[v]_{[t]}$, $t\in G$. 
Every $B'$-normalized (or $B_*$-normalized if $(B,B_*)$ is a dual Banach algebra) covariant representation of $[\alpha]$ in $B$ is of this form. 
This  gives a bijective correspondence between covariant representations of $\theta$ and $[\alpha]$ on a Banach space $E$.
\end{lem}
\begin{proof} 
Let $(\pi, v)$ be a covariant representation of $\theta$ in a Banach algebra $B$. 
Then $v$ extends uniquely to a representation $[v]$ of $S(G)$ in $B''$ by Remark~\ref{rem:partial_reps_vs_reps}.
Thus $[v]$ satisfies \ref{item:covariant_representation2}. 
For any $\overline{t}$ in the form \eqref{eq:standard_from} we have $[v]_{\overline{t}} = v_{t_1^{-1}} v_{t_1^{-1}} \cdots v_{t_n^{-1}} t_{t_n^{-1}} v_{t_0}$.
Every idempotent in $S(G)$ is of the form $\overline{e} = e_{t_1}....e_{t_n}$, and for $a \in C_0(X_{\overline{e}}) = C_0(\bigcap_{k=0}^n X_{t_k})$, using covariance of $(\pi, v)$ $n$-times, we get $[v]_{\overline{e}}\pi(a) = v_{t_1^{-1}} v_{t_1^{-1}} \cdots v_{t_n^{-1}} t_{t_n^{-1}} \pi(a) = \pi(a)$. This gives \ref{item:covariant_representation3}. 
Similarly, $\pi(a)[v]_{\overline{e}} = \pi(a)$.
Thus for any $a\in C_0(X_{\overline{t}}) = C_0(\bigcap_{k=0}^n X_{t_k})$ we get
\[
    [v]_{\overline{t}} \pi ( a\circ\theta_{t} ) = [v]_{\overline{e}} v_{t_0} \pi ( a\circ\theta_{t} ) =[v]_{\overline{e}} \pi(a) v_{t_0} = \pi(a) v_{t_0} = \pi(a)[v]_{\overline{e}} v_{t_0} = \pi ( a\circ\theta_{t} ) [v]_{\overline{t}}.
\]
This gives \ref{item:covariant_representation1}.
Thus $(\pi, [v])$ is a covariant representation of $[\theta]$.
Also if $B = B(E)$ and $(\pi, v)$ is a covariant represenation on $E$, 
one readily sees that $(\pi, [v])$ is a covariant representation of $[\theta]$ on $E$. 

Conversely, if $(\pi, [v])$ is any covariant representation of $[\theta]$ then putting $v_t := [v]_{[t]}$, $t\in G$, the pair $(\pi, v)$ satisfies the conditions in Definition~\ref{defn:covariant^partial_representation_in_algebra} except that $v$ might not be a partial representation. 
It will be whenever $[v]$ is a representation of $S(G)$, and this is always the case when $(\pi, [v])$ is $B'$-normalized, or $B_*$-normalized if $(B,B_*)$ is a dual Banach algebra, or if  $(\pi, [v])$  is a covariant representation on a Banach space $E$. 
\end{proof} 

\begin{thm}\label{thm:partial_actions_representations}
Let $\theta : G \to \PHomeo(X)$ be a partial action of a discrete group.
\begin{enumerate} 
    \item \label{enu:partial_actions_representations1} Every representation of $\ell^1(\theta)$ in a Banach algebra $B$ is given by the formula $\pi\rtimes v(a) = \sum_{t\in G} \pi(a(t))v_t$ for a covariant representation $(\pi,v)$ of $\theta$ in $B$. If $B$ is a dual Banach algebra or if each $X_t$, $t\in G$,  is compact, then $v$ can be chosen to take values in $B$.
    \item \label{enu:partial_actions_representations2} If $E$ is a Banach space and either each $X_t$, $t\in G$, is compact or $E$ is reflexive, we have a bijective correspondence between representations of $\ell^1(\theta)$ on $E$ and covariant representations of $\theta$ on $E$.
    \item \label{enu:partial_actions_representations3} If $\F=\C$, $p\in (1,\infty)\setminus \{2\}$, and $\mu$ is a localizable measure, we have a bijective correspondence between non-degenerate representations of $\ell^1(\theta)$ on $L^p(\mu)$ and pairs $(\pi,v)$ where 
        \begin{itemize}
            \item[(a)] $\pi : C_0(X)\to B(L^p(\mu))$ is a non-degenerate representation acting by multiplication operators on $L^p(\mu)$ such that $v_{t} \pi ( a\circ \theta_{t} ) v_{t^{-1}} = \pi(a)$ and $v_{t} L^p(\mu) = \overline{\pi(C_0(X_{t}))L^p(\mu)}$ for all $a\in C_0(X_{t})$ and $t\in G$;
            \item[(b)] $v : G \to \SPIso(L^p(\mu))$ satisfies $v_{1}=1$, $v_{s} v_{t} v_{t^{-1}} = v_{st}v_{t^{-1}}$, $v_{t}^* = v_{t^{-1}}$  for  $s,t \in G$.
        \end{itemize}
\end{enumerate}
\end{thm}
\begin{proof}
In view of Theorem~\ref{thm:partial_actions} and Lemma~\ref{lem:isomorphism_of_transform_groupoids}, items \ref{enu:partial_actions_representations1} and \ref{enu:partial_actions_representations2} follow from Theorem~\ref{thm:disintegration}, while \ref{enu:partial_actions_representations3} follows from Theorem~\ref{thm:spatial_representations_of_groupoid_algebras}\ref{enu:spatial_representations_of_groupoid_algebras2}.
\end{proof}




%
\subsection{Banach algebras associated to inverse semigroups} 
\label{subsec:Banach_algebras_inverse_semigroups}

The universal $\F$-algebra $\F S$ for an inverse semigroup $S$ consists of formal finite $\F$-linear combinations of elements in $S$, and is equipped with the multiplication induced from $S$. 
This algebra is universal for semigroup homomorphisms of $S$. 
Wanting to keep at least a hint that $S$ is more than just a semigroup, we will consider here completions of $\F S$ such that the commutative subalgebra $\F E$, generated by the idempotents $E := E(S)$, is  a uniform Banach algebra, which boils down to an assumption of \emph{joint contractiveness} of certain projections.
When applied to tight representations this will give us anoter description of all Banach algebras associated to ample groupoids.
 
We start by recalling the groupoid models for $\F S$ and $FE$ from \cite{Steinberg}. 
Recall that the semigroup $E$ is a semilattice: the `meet' and `preorder' are defined by $e\wedge f := e f$ and $e\leq f$ if and only if $ef=e$.
We write $e<f$ if $e\leq f$ and $e\not=f$. 
If $E$ has a zero, we say $e$ and $f$ are orthogonal if $ef = 0$.
The set  $\widehat{E}\subseteq \{0, 1\}^E$  of all non-zero homomorphisms $\phi : E \to \{0,1\}$ is a totally disconnected locally compact Hausdorff space in the product topology of the Cantor space $\{0, 1\}^E$ (which is the topology of  pointwise convergence). 
In fact, putting $Z(e) := \{ \phi \in \widehat{E} : \phi(e) = 1 \}$, the sets $Z(e) \setminus \bigcup_{f \in F} Z(f)$, where $e\in E$ and $F\subseteq eE$ is finite, form a basis of compact-open sets for $\widehat{E}$. 
We call $\widehat{E}$ the \emph{spectrum} of $E$. 
A map $\phi : E\to \{0,1\}$ is in $\widehat{E}$ if and only if its support $\supp(\phi) := \{ e\in E : \phi(e)=1\}$ is a \emph{filter} of $E$, i.e. a non-empty, upward-closed and downward directed and subset of $E$. 
There is a natural action $h : S \to \PHomeo(\widehat{E})$ given by 
\[
    h_{t} : Z(t^*t)\to Z(tt^*) ;\ h_{t}(\phi)(e) := \phi ( t^*et ) , \qquad t\in S ,\ e\in E ,\ \phi\in \widehat{E} 
\]
(see \cite[Proposition 10.3]{Exel}). 
The corresponding transformation groupoid $\G(S) := S \ltimes_{h} \widehat{E}$ is ample. 
It is covered by compact open bisections $U_{t} = \{ [t,e] : e\in Z(t^*t) \}$, $t\in S$. 
The associated Steinberg algebra is
\[
    A(\G(S)) := \spane \{ 1_{U}: \text{$U\in \Bis(\G(S))$ compact open} \} = \spane \{ 1_{U_t} : t\in S \} \subseteq C_c(\G(S)).
\]
By a homomorphism from the semigroup $S$ to an algebra $B$ we mean a semigroup homomorphism $v:S\to B$ into the multiplicative semigroup of $B$.

\begin{prop}[{\cite[Theorem 5.1]{Steinberg}}] \label{prop:Steinberg_algebra_of_inverse_semigroup} 
The map $t \mapsto 1_{U_t}$ determines an isomorphism $\F S\cong A(\G(S))$. 
Thus for any $\F$-algebra $B$, the relation $\psi(1_{U_t}) = v_t$, $t\in S$, establishes a bijective correspondence between algebra homomorphisms $\psi : A(\G(S))\to B$ and homomorphisms $v : S \to B$.
\end{prop}

We have the following commutative subalgebra of $A(S)$:
\[
    A(\widehat{E}) := \spane \{ 1_{U} : \text{$U\subseteq \widehat{E}$ compact open} \} = \spane \{ 1_{Z(e)} : e\in E \} \subseteq C_c(\widehat{E}) \subseteq C_c(\G(S)).
\]
Proposition~\ref{prop:Steinberg_algebra_of_inverse_semigroup} applied to $E$ gives a bijective correspondence between algebra homomorphisms $\pi : A(\widehat{E}) \to B$ and homomorphisms $v : E \to B$. 
We now aim to characterise injectivity of $\pi$ in terms of $v$. 
For finite $F\subseteq E$ we write $\bigwedge{F} := \bigwedge_{e\in F} e= \prod_{e\in F}e$.

\begin{lem}[Inclusion-exclusion principle]\label{lem:inclusion-exclusion}
Let $v:E\to B$ be a homomorphism into an algebra $B$. 
For any finite $F\subseteq E$ and $F_0\subseteq F$ put
\begin{equation}\label{eq:general_projections_to_be_contractive}
    P_{F_0}^F := \prod_{f\in F\setminus F_0} (v_{\bigwedge{F_0}} - v_{\bigwedge{F_0}\wedge f}) = \sum_{F_0\subseteq G\subseteq F} (-1)^{|G\setminus F_0|} v_{\bigwedge{G}} ,
\end{equation}
with the convention $P_{F}^F=v_{\bigwedge{F}}$ and $P_{\emptyset}^F=0$. 
Then $\{ P_{F_0}^{F} \}_{F_0\subseteq F}$ are mutually orthogonal idempotents such that 
$v_e = \sum_{e\in F_0\subseteq F} P_{F_0}^F$, $e\in F$.
\end{lem}
\begin{proof}
This can be proved by induction on $|F|$. 
If $|F| = 1$ the assertion is trivial.
Suppose the assertion holds whenever $|F| < n$, and fix $F$ with $|F| = n$. 
Fix also $e_0\in F$ and let $H=F\setminus\{e_0\}$. 
For any $F_0\subseteq F$ we then have 
\[
    P_{F_0}^F = \begin{cases} 
                    P_{F_0}^H - P_{F_0}^H v_{e_0} , & e_0\not\in F_0, \\
                    P_{F_0 \setminus \{ e_0 \} }^H v_{e_0}, & e_0 \in F_0,\ F_0\neq\{e_0\}, \\
                    v_{e_0} - \sum_{H_0 \subseteq H} P_{H_0}^H , & F_0 \neq \{e_0\}.
                \end{cases}
\]
By the inductive hypothesis $\{ P_{H_0}^{H} \}_{H_0\subseteq H}$ are mutually orthogonal and $v_e = \sum_{e\in H_0\subseteq H} P_{H_0}^H$ ($e\in H$). 
Using this and the above relations it is readily checked that $\{ P_{F_0}^{F} \}_{F_0\subseteq F}$ are also mutually orthogonal, and $v_e = \sum_{e\in F_0\subseteq F} P_{F_0}^F$, $e\in F$.
\end{proof}

%\begin{lem}\label{lem:empty_cylinder}
%$Z(e) \setminus \bigcup_{f \in F} Z(f)\neq \emptyset$ for all $e\in E$ and $F\subseteq eE\setminus\{e\}$.
%\end{lem}
%\begin{proof}
%If $f<e$ for all $f\in F$, then .
%\end{proof}

\begin{cor}\label{cor:injectivity_representation_A(E)} 
Let $\pi:A(\widehat{E})\to B$ be an algebra homomorphism and let $v : E\to B$ be the corresponding semigroup homomorphism.
Then $\pi$ is injective if and only if $\prod_{f\in F} (v_e - v_f) \neq 0$ for every $e\in E\setminus \{0\}$ and finite $F\subseteq eE\setminus\{e\}$.
\end{cor}
\begin{proof} 
Idempotents \eqref{eq:general_projections_to_be_contractive} corresponding to idempotents $\{ 1_{Z(e)} \}_{f\in F} \subseteq A(E)$ are characteristic functions of basic cyinder sets $Z \big( \bigwedge F_0 \big) \setminus \bigcup_{f\in F\setminus F_0} Z\big( \bigwedge F_0\wedge f \big)$. 
Hence, by Lemma ~\ref{lem:inclusion-exclusion}, every element in $A(E)$ is a finite linear combination of mutually orthogonal idempotents of the form $1_{Z(e) \setminus \bigcup_{f \in F} Z(f)}$, where $e\in E\setminus\{0\}$ and $F\subseteq eE\setminus\{e\}$, which are mapped by $\pi$ to mutually orthogonal idempotents of the form $\prod_{f\in F} (v_e - v_f)$. 
Moreover, $1_{Z(e) \setminus \bigcup_{f \in F} Z(f)}\neq 0$ because $\phi(f)=1$ if and only if $f\geq e$ defines $\phi\in Z(e) \setminus \bigcup_{f \in F} Z(f)$. 
Hence $\pi$ is injective if and only if it is non-zero on elements of the form $ 1_{Z(e) \setminus \bigcup_{f \in F} Z(f)}$.
\end{proof}

For representations in Banach algebras we need an analytic condition.

\begin{defn}\label{defn:jointly_contractive}
Let $B$ be a  Banach algebra over $\F$ and let $M\subseteq \F$ be a set. 
We say that a collection of mutually orthogonal idempotents $\{P_i\}_{i\in F}\subseteq B$ is \emph{jointly $M$-contractive} if 
\begin{equation*}\label{eq:jointly_contractive}
    \| \sum_{i\in F} a_i{P_i} \| \leq \max_{i\in F} |a_{i}|, \qquad a_i\in M,\ i \in F .
\end{equation*}
\end{defn}

\begin{rem}\label{rem:types_of_projections}
The above concept embraces a number of other notions existing in the literature. 
For instance, if $B$ is a unital Banach algebra then $\{P,1-P\}$ are jointly $\{0,1\}$-contractive if and only if $\{ P,1-P \}\subseteq B_1$, which by definition means that $P$ is \emph{bicontractive} \cite{Bernau_Lacey2}. 
If $\F = \C$ and $\T = \{ z\in \C : |z|=1 \}$, then $\{P,1-P\}$ are jointly $\T$-contractive if and only if $zP +w (1-P)$ are isometries (contractive invertible elements with contractive inverses), for all $z,w\in \T$, which by definition means that $P$ is \emph{bicircular}, see \cite{Stacho_Zalar}. 
Every bicircular projection is bicontractive but not conversely. 
In fact, a projection $P$ is bicircular if and only if it is hermitian \cite{Jamison}.
Moreover, for any $p\in [1,\infty]$ and any Banach space $E$, mutually orthogonal $L^p$-projections (in the sense considered in Subsection~\ref{subsect:Partial isometries}) $\{P_i\}_{i\in F}\subseteq B(E)$ are jointly $\F$-contractive (the Banach algebra they generate embeds isometrically into $c_0(F)$). 
%If $\{P_i\}_{i \in F}\subseteq B(E)$ is a finite family of pairwise orthogonal $L^p$-projections then they are jointly $\F$-contractive.
%Moreover, if $P,Q$ are commuting $L^p$-projections, then $P\wedge Q=PQ$, $P\vee Q=P+Q-PQ$ and $P-P\wedge Q$ are $L^p$ projections.
\end{rem}
%\begin{proof}
%For $p=\infty$ note that  $\|\xi\|=\max\{\max_{i\in F}\|P_i\xi\|, \|\prod_{i\in F} (1-P_i) \xi\|\}$. This implies  
%$\|\sum_{i \in F}  a_i P_i\xi\|\leq\max_{i \in F} \|a_i P_i\xi\|\leq \max_{i\in F} |a_{i}| \|\xi\|$. 
%While for $p<\infty$ we have  $\|\xi\|^p=\sum_{i\in F}\|P_i\xi\|^p + \|\prod_{i\in F} (1-P_i) \xi\|^p$ and hence
%\begin{align*}
%\|\sum_{i \in F}  a_i P_i\xi\|^p\leq \max_{i\in F} |a_{i}|^p\sum_{i \in F} \|P_i \xi\|^p=
%\max_{i\in F} |a_{i}|^p \|\sum_{i \in F}  P_i\xi\|^p \leq \max_{i\in F} |a_{i}|^p \|\xi\|^p.
%\end{align*}
%Thus $\|\sum_{i\in F} a_i{P_i}\|\leq \max_{i\in F} |a_{i}|$ for all $a_i\in \F, i \in F$. 
%\end{proof}

\begin{defn}\label{defn:representation_semigroup_contractive}
A \emph{representation of the inverse semigroup $S$ in a Banach algebra} $B$ is a semigroup homomorphism $v : S \to B_1$ such that for every finite $F\subseteq E$ the idempotents $\{ P_{F_0}^F \}_{F_0\subseteq  F}$ given by \eqref{eq:general_projections_to_be_contractive} are jointly $\F$-contractive.
\end{defn}

\begin{lem}\label{lem:representation_characterization}
Let $v:S\to B_1$ be a zero-preserving semigroup homomorphism into the contractive elements in a Banach algebra $B$, and let $A := \clsp \{ v_{e} : e\in E \} \subseteq B$. 
The following are equivalent:
\begin{enumerate}
    \item\label{enu:representation_characterization1} $v$ is a representation of $S$ in $B$;
    \item\label{enu:representation_characterization2} the map $1_{Z(e)}\mapsto v_{e}$ extends to a representation $\pi:C_0(\widehat{E})\to A\subseteq B$; 
    \item\label{enu:representation_characterization3} $A\cong C_0(X)$ for a closed subset $X\subseteq \widehat{E}$;
    \item\label{enu:representation_characterization4} $A$ embeds isometrically into a (real or complex) $C^*$-algebra.
\end{enumerate}
The above equivalent conditions always hold when $B=B(K)$ and $v|_E$ takes values in the $L^p$-projections on a Banach space $K$, for some fixed $p\in[1,\infty]\setminus \{2\}$. 
In particular, any homomorphism $v : S \to \SPIso(L^p(\mu)) \subseteq B(L^p(\mu))$, where $p\in[1,\infty]$, is a representation of $S$, and any $*$-homomorphism $v : S \to B$ into a (real or complex) $C^*$-algebra is a represenation of $S$.
%If these equivalent conditions hold, then $\pi:C_0(\widehat{E})\to A$ in \ref{enu:representation_characterization2} is isometric
%iff $\prod_{f\in F} (v_e -v_f) \neq 0$ for every $e\in E\setminus \{0\}$ and finite $F\subseteq eE\setminus\{e\}$.
\end{lem}
\begin{proof} 
\ref{enu:representation_characterization1}$\Rightarrow$\ref{enu:representation_characterization2}  
By Proposition~\ref{prop:Steinberg_algebra_of_inverse_semigroup}, the map $1_{Z(e)}\mapsto v_{e}$ extends to an algebra homomorphism $\pi : A(\widehat{E}) \to A\subseteq B$ given by $\pi(\sum_{e\in F} a_e 1_{Z(e)}) = \sum_{e\in F} a_e P_{e}$, for finite $F\subseteq E$ and $a_e\in \F$, $e\in F$.  
Let $\{ P_{F_0}^{F} \}_{F_0\subseteq  F}$ be the orthogonal idempotents associated to $F$ in Lemma~\ref{lem:inclusion-exclusion}.
The corresponding idempotents associated to $e\mapsto 1_{Z(e)}$ are $\{1_{Z(\bigwedge F_0)\setminus \bigcup_{f\in F\setminus F_0} Z(\bigwedge F_0\wedge f)} \}_{F_0\subseteq  F}$.
So there are $b_{F_0}\in \F$ ($F_0\subseteq F$) such that 
\[
    \sum_{e\in F} a_e 1_{Z(e)} = \sum_{F_0\subseteq F} b_{F_0} 1_{Z\big(\bigwedge F_0\big) \setminus \bigcup_{f\in F\setminus F_0} Z\big(\bigwedge F_0\wedge f\big)} = 0 , \quad \text{ and }\quad
    \sum_{e\in F} a_e P_{e} = \sum_{F_0\subseteq F} b_{F_0} P_{F_0}^{F}. 
\]
The joint $\F$-contractivness of $\{P_{F_0}^{F}\}_{F_0\subseteq  F}$ implies 
\[
    \Big\| \sum_{e\in F} a_e P_{e} \Big\| \leq \max_{P_{F_0}^{F} \neq 0} | b_{F_0} | \leq \max_{Z\big(\bigwedge F_0\big) \setminus \bigcup_{f\in F\setminus F_0} Z \big( \bigwedge F_0\wedge f \big) \neq \emptyset } | b_{F_0} | = \Big\| \sum_{e\in F} a_e 1_{Z(e)} \Big\|_{\infty}.
\]
Hence the map $\pi:A(\widehat{E})\to A\subseteq B$ is contractive and so it extends (uniquely) to a representation $\pi : C_0(\widehat{E}) \to A\subseteq B$, because $A(\widehat{E})$ is dense in $C_0(\widehat{E})$. 

Implication \ref{enu:representation_characterization2}$\Rightarrow$\ref{enu:representation_characterization3} follows from Corollary~\ref{cor:minimality_of_sup_norm} and \ref{enu:representation_characterization3}$\Rightarrow$\ref{enu:representation_characterization4} is trivial. 
Condition~\ref{enu:representation_characterization4} allows us to assume that $A\subseteq B(H)$ for a Hilbert space $H$. 
Then $v_{e}$, $e \in E$, are orthogonal projections on $H$ (they are contractive idempotents), so the  idempotents \eqref{eq:general_projections_to_be_contractive} are orthogonal projections (they are self-adjoint idempotents). 
In particular, idempotents \eqref{eq:general_projections_to_be_contractive} are mutually orthogonal $L^2$-projections, and so they are jointly $\F$-contractive.
This shows \ref{enu:representation_characterization4}$\Rightarrow$\ref{enu:representation_characterization1}.
\end{proof}

\begin{defn}
We define the \emph{universal inverse semigroup Banach algebra} $F(S)$ as the completion of $\F S$ in the norm 
\[
    \| {\textstyle \sum_{t\in F}} a_f t \|_{\max} := \sup\{ \| {\textstyle \sum_{t\in F}} a_f v_t \| : \text{$v:S\to B$ is a representation in a Banach algebra $B$} \} .
\]
For $p\in[1,\infty]$ we define the $L^p$-Banach algebra $F^p(S)$ in a similar way, as the completion of $\F S$ in the maximal norm for representations in $L^p$-operator algebras.
\end{defn}

\begin{thm}\label{thm:inverse_semigroups_Banach_representations}
We have an isometric isomorphism $F(S) \cong F(\G(S))$ which descends to $F^p(S) \cong F^p(\G(S))$ for $p\in [1,\infty]$.
In particular, the relations $\psi(1_{U_t}) = v_t$, $t\in S$, establish a bijective correspondence between representations 
$\psi : F(\G(S)) \to B$ and $v : S \to B_1$ in a Banach algebra $B$. 
Moreover, $\pi := \psi|_{C_0(\widehat{E})}$ is isometric if and only if for every $e\in E$ and every finite $F\subseteq eE\setminus\{e\}$ we have $\prod_{f\in F} (v_e -v_f) \neq 0$.
\end{thm}
\begin{proof}
By the last part of Theorem~\ref{thm:disintegration} every representation of $F(\G(S))$ in $B$ is of the form $\pi\rtimes v$ where $\pi  :C_0(\widehat{E}) \to B$ is a representation and $v  :S\to B_1$ is a semigroup homomorphism such that $\pi(1_{Z(e)})=v_e$, $e\in E$,
 and $v_t\pi(a)v_{t^*} = \pi(a\circ h_{t^*})$ for all $a\in C(Z(t^*t))$ ($t\in S$). 
In particular, Lemma~\ref{lem:representation_characterization} implies that $v:S\to B_1$ is a representation of the inverse semigroup $S$.
Conversely, if $v:S\to B_1$ is a representation of the inverse semigroup $S$, then by Proposition~\ref{prop:Steinberg_algebra_of_inverse_semigroup} there is a unique algebra homomorphism  $\psi : A(\G(S))\to B$ such that $\psi(1_{U_t}) = v_{t}$, $t\in S$. 
Putting $\pi:=\psi|_{A(\widehat{E})}$, we have $v_t \pi(a )v_{t^*} = \pi(a\circ h_{t^*})$ for all $a\in 1_{Z(t^*t)} A(\widehat{E})$, $t\in S$. 
By Lemma~\ref{lem:representation_characterization}, $\pi : A(\widehat{E}) \to B$ extends to a representation $\pi : C_0(\widehat{E})\to B$ and hence the latter condition holds for all $a\in C(Z(t^*t)) = \overline{1_{Z(t^*t)} A(\widehat{E})}$.
Thus $\psi$ extends to a representation $\psi = \pi\rtimes v : F(\G(S))\to B$. 
This gives the first part of the assertion. 

For the second part recall that basic non-empty open sets are of the form  $Z(e) \setminus \bigcup_{f \in F} Z(f)$ for $e\in E$ and finite $F\subseteq eE\setminus\{e\}$.
Moreover, for the corresponding representations we have $\prod_{f\in F} (v_e -v_f) = \pi(1_{Z(e) \setminus \bigcup_{f \in F} Z(f)})$. 
Thus if $\pi$ is injective then $\prod_{f\in F} (P_e -P_f)\neq 0$, see also Corollary~\ref{cor:injectivity_representation_A(E)}. 
If $\pi$ is not isometric then by Corollary~\ref{cor:minimality_of_sup_norm}, $\ker\pi  =C_0(U)$ for a non-empty open set $U\subseteq \widehat{E}$.
Hence $\pi$ vanishes on a non-empty set of the form $1_{Z(e) \setminus \bigcup_{f \in F} Z(f)}$, which implies $\prod_{f\in F} (P_e -P_f)= 0$.
\end{proof}

\begin{cor}
For any inverse semigroup $S$ we have $C_0(\widehat{E})\cong F(E)\cong \overline{\F(E)}^{\|\cdot\|_{\max}}\subseteq F(S)$. 
%So $\|\cdot\|_{\max}$ is the maximal submultiplicative norm on $\F S$ that extends the minimal submulitiplicative norm on $\F E$. 
\end{cor}

\begin{cor} 
Let $p\in [1,\infty]$.
If  $\F=\C$ and $p\neq 2$, then $F^p(S)$ is a universal $L^p$-operator algebra for semigroup homomorphisms into spatial partial isometries: $F^p(S) \cong \clsp \{ V_t : t\in S \}$ for a (universal) semigroup homomorphism $V : S \to\SPIso(L^p(\tilde{\mu}))$, and any semigroup homomorphism $v : S\to\SPIso(L^p(\mu))$ extends to a representation $v\rtimes S : F^p(\G(S)) \to B(L^p(\mu))$, where $v\rtimes S(V_t) = v_t$, for $t\in S$.
\end{cor}
\begin{proof}
Combine Theorems~\ref{thm:inverse_semigroups_Banach_representations} and \ref{thm:spatial_representations_of_groupoid_algebras}.
\end{proof}

\begin{rem}
When $S$ has a zero it is natural to consider zero-preserving homomorphisms. 
The corresponding universal algebras $\F_0 S$, $F_0^p(S)$, $p\in[1,\infty]$, and $F_0(S)$, called \emph{contracted algebras of $S$}, are quotients of $\F S$, $F^p(S)$, $p\in[1,\infty]$, and $F(S)$ respectively by the one-dimensional space generated by $0\in S$.
The \emph{contracted groupoid} $\G_0(S)$ modeling these algebras is the transformation groupoid for the $S$-action restricted to the closed $S$-invariant set $\widehat{E}\setminus \{1\}$. 
\end{rem}

The algebras constructed above are much too big for a number of purposes.
To explain this consider a totally disconnected locally compact Hausdorff space $X$. 
The set  $E:= \{ 1_{U} : \text{$U\subseteq X$ compact open} \}$ of idempotents in $C_0(X)$ is naturally 
a \emph{Boolean ring} (a \emph{generalized Boolean algebra} in the nomenclature of \cite{Steinberg}). 
Here we do not assume Boolean rings are unital, so they are equivalent to structures $(R,0,\vee, \wedge, \setminus)$ where $\setminus$ is a binary operation (relative complement). 
The Stone spectrum of the Boolean ring $E$ can be naturally identified with the space $X$, but the spectrum $\widehat{E}\setminus\{1\}$ of $E$ as a semillatice is much larger. 
Thus $C_0(X)$ is merely a quotient of the contracted algebra $F_0(E)$.  
The main idea behind the work of Exel on tight representations, see %\cite{Exel_tight}, 
\cite{Exel}, is to fix that and exploit the Boolean ring structure. 
We adapt here Exel's theory, see also \cite{Donsig_Milan}, to the (non-unital) Banach algebra context, and characterise injectivity of the relevant representations. 
To this end, we fix again a semilatice $E$ of idempotents in an inverse semigroup $S$.

\begin{defn} 
A \emph{cover} of $e\in E$ is a finite set $F\subseteq eE$ such that for every nonzero $z\leq e$ we have $zf\neq 0$ for some $f\in F$.
We say that a map $v : E\to R$ into a Boolean ring $R$ is \emph{tight} if $v_e - \bigvee_{f\in F} v_f = \bigwedge_{f\in F} (v_e\setminus v_f)$ is zero whenever $F$ covers $e\in E$.
A semigroup homomorphism $v : S\to B$ into an $\F$-algebra $B$ is \emph{tight} if $\prod_{f\in F} (v_e - v_f) = 0$ for every cover $F$ of $e\in F$ 
(equivalently $v|_E$ is tight as a map into a Boolean ring of idempotents in $\spane \{ v_e : e\in E \} \subseteq B$).
\end{defn}

\begin{rem}
If $E$ has a zero, then $\emptyset$ covers $0$ and so $v_0=0$ for every tight map $v:E\to R$ into a Boolean ring $R$.
If $E$ is a Boolean ring, then a semigroup homomorphism $v:E\to R$ is a Boolean ring homomorphism if and only if $v$ is tight.
Indeed, if $F$ covers $e$ then $e=\bigvee F$ as otherwise $e\setminus \bigvee F\leq e$ would be a nonzero element which is orthogonal to $F$. 
This equality has to be preserved by $v$ if it is a ring homomorphism and so it then has to be tight.  
Conversely, if $v$ is tight, then for $e,f\in E$ the set $\{e,f\}$ covers $e\vee f$ and $\{e\setminus f, e\wedge f\}$ covers $e$, which implies $v_e \vee v_ f= v_{e\vee f}$ and $v_{e\setminus f}=v_e\setminus v_{f}$. 
Hence $v$ is a Boolean ring homomorphism.
\end{rem}

By  \cite[Theorem 12.9]{Exel}, the set
\[
    \widehat{E}_{T} := \{ \phi : E \to \{0,1\} \ \text{is a tight homomorphism} \}
\]
is a closed subset of $\widehat{E}$. 
In fact, $\widehat{E}_{T}$ is the closure of the set of $\phi\in \widehat{E}$ whose support $\supp(\phi ) = \{ e\in E : \varphi(e) = 1 \}$ is an ultrafilter. 
We call $\widehat{E}_{T}$ the \emph{tight spectrum} of $E$.
When $E=E(S)$ is the semilatice of idempotents in an inverse semigroup $S$, then $\widehat{E}_{T}$ is invariant for the $S$-action on $\widehat{E}$, and hence the corresponding tranformation groupoid $G_T(S) := S\ltimes \widehat{E}_{T}$, called the \emph{tight groupoid of $S$} is the restriction of $\G(S)$ to the set $\widehat{E}_{T}$. 
 By \cite[Corollary 2.14]{Steinberg_Szakacs} the isomorphism from Proposition~\ref{prop:Steinberg_algebra_of_inverse_semigroup} factors to the isomorphism 
\begin{equation}\label{eq:tight_algebra}
    A(\G_T(S)) \cong \F S/ \Big\langle \prod_{f\in F} e - f : \text{$F$ covers $e\in E$} \Big\rangle,   
\end{equation}
and this algebra is universal for tight semigroup homomorphisms. 

\begin{defn} 
The \emph{tight Banach algebra} $F_T(S)$ is the completion of the algebra \eqref{eq:tight_algebra} in the maximal norm induced by tight representations $v : S\to B$ in Banach algebras $B$.
Similarly, for $p\in[1,\infty]$, we define the \emph{tight $L^p$-Banach algebra} $F^p_T(S)$, using tight representations in $L^p$-operator algebras. 
\end{defn}

\begin{thm}\label{thm:tight_inverse_semigroups_Banach_representations}
The somorphism \eqref{eq:tight_algebra} extends to isometric isomorphisms $F_T(S)\cong F(\G_T(S))$ and $F^p_T(S)\cong F^p(\G_T(S))$ for $p\in [1,\infty]$.
In particular, we have a bijective correspondence between representations $\pi:F(\G_T(S))\to B$ and tight representations $v:S\to B_1$ in a Banach algebra $B$.
Moreover, $\pi$ is isometric on $C_0(\widehat{E}_T)$ if and only if $\prod_{f\in F} (v_e -v_f) \neq 0$ for every $e\in E$ and finite $F\subseteq eE$ that does not cover $e$.
\end{thm}
\begin{proof} 
One may mimic the proof of Theorem~\ref{thm:inverse_semigroups_Banach_representations}, using that the algebras in \eqref{eq:tight_algebra} are universal for cover-to-join maps. 
In particular, for the second part note that basic non-empty open sets in $\widehat{E}_T$ are of the form  $Z(e) \setminus \bigcup_{f \in F} Z(f) \cap \widehat{E}_T$ for $e\in E$ and finite $F\subseteq eE\setminus\{e\}$. 
Such a set is empty if and only if $F$ is a cover of $F$.
Indeed, if $F$ is not a cover, then there is $0\neq z\leq e$ with $zf=0$ for all $f\in F$, and taking
any ultrafilter containing $z$ the corresponding $\psi$ is in $Z(e) \setminus \bigcup_{f \in F} Z(f)$.
\end{proof}
En passant we slightly improve the main result of \cite{Exel_recon}.
\begin{cor}\label{cor:disconnected_are_tight}
Let $X$ be a totally disconnected locally compact Hausdorff space and let $E$ be a semilattice consisting of compact open sets that separate the points of $X$ and cover $X$. 
The pairing $\phi(e) = 1_{e}(x)$, for $x\in X$, $\phi\in \widehat{E}_{T}$, $e\in E$, gives a homeomorphism $X\cong \widehat{E}_{T}$.
\end{cor}
\begin{proof} 
The map $E\ni e \mapsto 1_e\in C_0(X)$ is a tight representation, and so it gives rise to a representation $\pi : C_0(\widehat{E}_{T}) \to C_0(X)$ which is surjective by the Stone--Weierstrass Theorem. 
To see that $\pi$ is isometric note that $\pi( 1_{Z(e) \setminus \bigcup_{f} Z(f)} |_{\widehat{E}_{T}} ) = \prod_{f\in F} (1_e - 1_f) = 1_{e\setminus \bigcup  F}$ for every $e\in E$ and finite $F\subseteq eE$.
So if $\prod_{f\in F} (1_e -1_f) = 0$, then $e = \bigcup F$ which means that $F$ is a cover of $e$.
%The latter implies  $1_{Z(e)\setminus\bigcup_{f}Z(f)}|_{\widehat{E}_{T}}=0$.
Hence $\pi$ is isometric by Theorem~\ref{thm:tight_inverse_semigroups_Banach_representations}. 
It is immediate that the homeomorphism dual to the isomorphism $C_0(\widehat{E}_{T}) \cong C_0(X)$ is of the described form.
\end{proof}



\begin{cor}\label{cor:realization_of_ample_groupoids}
Let $\G$ be an ample groupoid, and let $S\subseteq \Bis(\G)$ be a wide inverse semigroup of compact open bisections that separate points in $\G$. 
We have a natural isomorphism $\G\cong \G_T(S)$ of topological groupoids, and so we have a natural isomorphism $F_T(S)\cong F(\G)$.
\end{cor}
\begin{proof}  
Recall that $\G_T(S)$ is the transformation groupoid for the action $\hat{h} : S \to \PHomeo( \widehat{E}_{T} )$, where 
$  \hat{h}_{t} : \hat{Z}(t^*t) \to \hat{Z}(tt^*)$,   
$\hat{Z}(t^*t) := \{\phi \in \widehat{E}_T: \phi(t^*t)=1\}$ and $\hat{h}_{t}(\phi)(e) := \phi(t^*et)$ for  $t\in S$, $e\in E$, $\phi\in \hat{Z}(t^*t)$. 
Similarly $\G$ is the transformation groupoid for the action 
\[
    h:S\to \PHomeo(X) ;\ h_t = r\circ d|_{t}^{-1}: t^*t\to tt^*, \qquad t\in S .
\]
Note that for $e\leq t^*t$ we have $h_t(e)=tet^*$. 
Let $\Psi:\widehat{E}_{T}\to X$ be the homeomorphism from Corollary \ref{cor:disconnected_are_tight}.
Since $\psi \in \widehat{E}_{T}$ if and only if $1_{t^*t}(\Psi(\phi))=1$ we see that $\Psi:\hat{Z}(t^*t)\to t^*t$. 
Moreover, for $\phi\in \hat{Z}(t^*t)$ and $e\in E$,
we have
\[
    1_{e} \Big( \Psi \big( \hat{h}_{t}(\phi) \big) \Big) = \hat{h}_{t}(\phi)(e) = \psi(t^*et) = \phi \big( h_t^{-1} (e) \big) = 1_{h_t^{-1}(e)}(\phi) = 1_e \Big( h_t \big( \Psi(\phi) \big) \Big) .
\]
Thus $\Psi$ intertwines the $S$-actions and so the transformation groupoids are isomorphic.
\end{proof}



%
\subsection{Banach algebras associated twisted Deaconu--Renault groupoids}
\label{subsect:Deaconu-Renault groupoids}

We fix a local homeomorphism $\varphi:\Delta\to X$  between open subsets $\Delta$, $\varphi(\Delta) \subseteq X$ of a locally compact Hausdorff space $X$. 
Thus $(X,\varphi)$ is a singly generated dynamical system (SGDS) in the sense of \cite{Renault2000}. 
Put $\Delta_0 = X$ and inductively define $\Delta_{n} := \varphi^{-1}(\Delta_{n-1})$, the natural domain of the partial local homeomorphism $\varphi^n$ ($n\in \N$).
The \emph{Deaconu--Renault groupoid} associated to $(X,\varphi)$ is an \'etale, amenable, locally compact Hausdorff groupoid, see \cite{Renault2000}, where
\[
    \G(X,\varphi) := \{ (y,n-m,x): n,m\in \N_0,\ x\in \Delta_{n},\ y\in \Delta_{m},\ \varphi^{n}(x)=\varphi^{m}(y) \} ,
\]
and the groupoid structure is given by 
\[
    (z,n,y) (y,m,x) := (z,n+m,x) , \qquad (y,n,x)^{-1} := (x,-n,y) ,
\]
and the topology is inherited from $X\times \Z \times X$. 
The groupoid $\G(X,\varphi)$ is \'etale as it has a  basis for the topology consisting of bisections of the form
\begin{equation}\label{eq:basic_bisections_renault}
    \UU(V,n-m, U) := \{ (y,n-m,x): (y,x)\in V\times U, \ \varphi^{m}(y) = \varphi^{n}(x) \} ,
\end{equation}
where $U\subseteq \Delta_n$ and $V\subseteq \Delta_m$ are open sets such that $\varphi^n|_U$ and $\varphi^m|_V$ are injective and $\varphi^n(U) = \varphi^m(V)$. 
We identify the unit space $Z(X,0,X) = \{ (x,0,x) \in X\times \Z\times X\}$ of $\G(X,\varphi)$ with $X$ via the map $X\ni x\longmapsto (x,0,x)\in Z(X,0,X)$. 
Thus the range and domain maps are given by $r(y,n,x) = y$ and $d(y,n,x) = x$.  
%The groupoid $\G(X,\varphi)$ can be viewed as a universal (the largerst) transformation groupoid for inverse semigroup actions generated by  restrictions $\varphi|_{U}:U\to \varphi(U)$ where $U$ is an open set $U\subseteq X $ and $\varphi|_{U}$ is injective (a partial homeomorphism), cf. \cite{Renault2000}. In particular, there is also 
%a universal (the smallest) such groupoid called groupoids of germs $Germ(X,\varphi)$. 
%The groupoids $\G(X,\varphi)$ and  $Germ(X,\varphi)$ coincide if and only $\varphi$ is \emph{topologically free}, i.e. for each $n\in \N$ the set $\{x\in \Delta_n:\varphi^{n}(x)=x\}$ has
%empty interior, see \cite[Proposition 2.3]{Renault2000}.

%\begin{lem}
%Let $Z(V,N,U)$ and $Z(W,M,Y)$ be bisections of the form \eqref{eq:basic_bisections_renault}, so $N=n-m$ and $M=k-l$ where $n,m,k,l\in N_0$ and 
%$\varphi|_{V}^{m}$, $\varphi|_{U}^{n}$, $\varphi|_{W}^{l}$, $\varphi|_{Y}^{k}$ are injective. Then
%$Z(V,N,U)^{-1}=Z(U,-N,V)$ and 
%$$
%Z(V,N,U) \cdot Z(W,M,Y)=Z( 
%$$
%$$
%\UU(V,N,U) \cdot \UU(W,M,Y)=\UU(V',N+M, Y') 
%$$
%\end{lem}

\begin{prop}\label{prop:Renault-Deaconu_semigroup}
The collection $S(X,\varphi)$ of bisections \eqref{eq:basic_bisections_renault} forms a wide inverse semigroup of bisections of $\G(X,\varphi)$ whose canonical action is given by the formula 
\[
    \UU(V,n-m,U) \mapsto h_{\UU(V,n-m,U)} = (\varphi|_{V}^{m})^{-1} \circ \varphi|_{U}^{n}.
\]
In particular, we have a natural isomorphism $S(X,\varphi)\ltimes X\cong \G(X,\varphi)$.
If $X$ is totally disconnected then the collection $S_c(X,\varphi)$ of bisections \eqref{eq:basic_bisections_renault} with compact open $U$, $V$ is also a wide inverse semigroup, and so  $S_c(X,\varphi)\ltimes X\cong \G(X,\varphi)$. 
\end{prop}
\begin{proof}
Let $\UU(V,N,U)$ and $\UU(W,M,Y)$ be bisections of the form \eqref{eq:basic_bisections_renault}, so $N=n-m$, $M=k-l$ where $n,m,k,l\in \N_0$,  $\varphi|_{V}^{m}$, $\varphi|_{U}^{n}$, $\varphi|_{W}^{l}$, $\varphi|_{Y}^{k}$ are injective and $\varphi^m(V)=\varphi^n(U)$, $\varphi^l(Y)=\varphi^k(W)$. 
%Let $\UU(V,n-m,U)$ and $\UU(W,k-l,Y)$ be bisections of the form \eqref{eq:basic_bisections_renault}. 
Clearly, $\UU(V,N,U)^{-1}  \UU(U,-N,V)$ and $h_{\UU(V,N,U)} = (\varphi|_{V}^{m})^{-1}\circ \varphi|_{U}^{n} : U\to V$ is a homeomorphism.
Moreover,
\[
    \UU(V,N,U) \cdot \UU(W,M,Y) = \UU(V',N+M, Y') ,
\]
where $Y'=(\varphi|_{Y}^{k})^{-1}\circ \varphi^l(U\cap W)$ and $V'=(\varphi|_{V}^{m})^{-1} \circ \varphi^{n}(U\cap W)$. 
More specifically, if $n \geq l$ then $\varphi^{k+n-l}$ is injective on $Y'$ and $h_{\UU(V',N+M, Y')}=(\varphi|_{V'}^{m})^{-1}\circ \varphi|_{Y'}^{k+n-l}$ maps $Y'$ onto $V'$.
If $n\leq l$, then $\varphi^{m+l-n}$ is injective on $V'$ and  $h_{\UU(V',N+M, Y')}=(\varphi|_{V'}^{m+l-n})^{-1}\circ \varphi|_{Y'}^{k}$ maps $Y'$ onto $V'$.
Thus $S(X,\varphi)$ is an inverse semigroup.
It is wide because it forms a basis for topology of $\G(X,\varphi)$.
If the sets $U$, $V$, $W$, $Y$ above are compact open, then also $Y'$ and $V'$ are compact open. 
Hence $S_c(X,\varphi)$ is an inverse semigroup. 
It forms a basis for the topology of $\G(X,\varphi)$ when $X$ is totally disconnected.
\end{proof}

\begin{rem} 
Let $S(\varphi)$ be the unital inverse subsemigroup of $\PHomeo(X)$ generated by restrictions $\varphi|_{U}\in\PHomeo(X)$ to open sets $U\subseteq X$ where $\varphi|_{U}$ is injective. 
Proposition~\ref{prop:Renault-Deaconu_semigroup} implies that $S(\varphi)$ consists of partial homeomeomorphisms of the form $(\varphi|_{V}^{m})^{-1} \circ \varphi|_{U}^{n}$, and we have the canoncial semigroup epimorphism $h:S(X,\varphi)\onto S(\varphi)$. 
A look at the germ relation in \cite{Renault2000} shows that the groupoid of germs $\textup{Germ}(X,\varphi)$ defined there is canonically isomorphic to $S(\varphi)$. 
So we have a canonical groupoid epimorphism $\G(X,\varphi)\cong S(X,\varphi)\ltimes X \onto S(\varphi)\ltimes X\cong \textup{Germ}(X,\varphi)$. 
By \cite[Proposition 2.3]{Renault2000} this is an isomorphism if and only if $\G(X,\varphi)$ is topologically free, which holds if and only if $\varphi$ is \emph{topologically free} in the sense that for each $n\in \N$ the set $\{x\in \Delta_n:\varphi^{n}(x)=x\}$ has empty interior.
\end{rem}

As shown in \cite{Kumjian_Li} $C^*$-algebras associated to twisted topological graphs are modeled by  twisted groupoids of the form $(\G(X,\varphi),\LL)$.
We now describe general representations in Banach algebras associated to such groupoids. 

%Assuming further that $\varphi$ is \emph{proper}, for each $y\in X$ the preimage $\varphi^{-1}(y)
%$ is finite and the map $X\ni y\mapsto |\varphi^{-1}(y)|\in \N_0$ is continuous (locally constant),  see \cite[Lemma 2.2]{BroRaeVit}.
%Under this assumption the formula
%\begin{equation}\label{equ:classical_transfer}
%L(a)(y)= \frac{1}{|\varphi^{-1}(y)|}\sum_{x\in \varphi^{-1}(y)} a(x), \qquad a\in C_0(\Delta),
%\end{equation} 
%defines a linear contractive map $L:C_0(\Delta)\to C_0(\varphi(\Delta))$ satisfying $a L(b)=L( a\circ \varphi  b)$ for $a\in C_0(\varphi(\Delta))$, $ b\in C_0(\Delta)$.
%The map $L$ is a classical \emph{transfer operator}  for $\varphi$.

%Fix $p\in [1,\infty)$ and $q\in (1,\infty]$ with $1/p+1/q=1$.
%\begin{defn} 
%By a \emph{covariant representation }of $L$ on a space $L^p(\mu)$ we mean a triple $(\pi, S,T)$ where 
%$\pi:C_0(X)\to B(L^p(\mu))$ is a non-degenerate representation  and $S,T\in B(L^p(\mu))_1$ are mutually generalized inverses to each other 
%satisfying
%\begin{enumerate}
%\item $\pi(L(a))=S\pi(a) T$,  for $a\in C_0(\Delta)$
%\item  $T\pi(a)=\pi(a\circ \varphi )T$ for $a\in C_0(\varphi(\Delta))$
%\item  for any  $ a\in C_c(\Delta)$  and any partition of unity
%$\{v_i\}_{i=1}^n\subseteq C_0(\Delta)$ on $\overline{\supp(a)}$  subordinated to a finite cover $\{U_i\}_{i=1}^n$
% such that
%  $\varphi|_{U_i}$ is injective for every $i=1,...,n$,  we have
%$$
%\pi(a)\sum_{i=1}^n \pi(u_i^{1/p}) TS \pi(u_i^{1/q})  =   \pi(a) \qquad \text{ where }u_i(x):=|\varphi^{-1}(\varphi(x))|v_i(x).
%$$ 
%Here we assume that $u_i^{1/\infty}=u_i^0=1$.
%\end{enumerate}
%\end{defn}
%We relate the groupoid $L^p$-operator algebras of  to crossed products by transfer operator under the assumption that 
%$\varphi$ is proper. 
%./ $$
%L(a)=tat^*, \,\, a\in C_0(\Delta), \qquad a\sum_{i=1}^n u_i^{K} tt^* u_i^{K}  =   a, \,\, a\in C_c(\Delta_{\reg})
%$$

\begin{thm}\label{thm:Renault-Deaconu_groupoid}
Let $\LL$ be any twist over $\G(X,\varphi)$. 
Let $S \subseteq S(X,\varphi)\cap S(\LL)$ be any wide inverse semigroup of bisections where the twist $\LL$ can be trivialized (one may always take $S=S(X,\varphi)\cap S(\LL)$) and consider the action
\[
    \alpha : S \to \PAut(C_0(X)) ;\ \alpha_{\UU(V,n-m,U)}(a) := a \circ (\varphi|_{U}^{n})^{-1} \circ \varphi|_{V}^{m} , \qquad a\in C_0(U) . 
\]
For $t\in S$ pick any $c_t\in C_u(t,\LL)$, unless $t\subseteq X$, in which case put $c_t(x):=(x,1)\subseteq X\times \F$. 
For $s,t\in S(\LL)$ put $u(s,t):=c_s* c_t * c_{st}^*$. 
Then $(\alpha,u)$ is a twisted action of $S$ and we have a natural isomorphism
\[
    F(\G(X,\varphi),\LL)\cong C_0(X)\rtimes_{(\alpha,u)} S.
\]
Thus every representation $\psi:F(\G(X,\varphi), \LL)\to B$ is determined by the formula
\begin{equation}\label{eq:cor:Renault-Deaconu1}
    \psi (a\delta_{\UU(V,N,U)} ) = \pi(a)v_{\UU(V,N,U)}, \qquad a\in C_c(V),
\end{equation}
for a covariant representation $(\pi,v)$ of $(\alpha,u)$. 
In particular,
\begin{enumerate}
    \item \label{enu:Renault-Deaconu_groupoid1} If $\F=\C$, $p\in (1,\infty)\setminus \{2\}$ and $\mu$ is a localisable measure, then \eqref{eq:cor:Renault-Deaconu1} establishes a bijective correspondence between non-degenerate representations $\psi:F^p(\G(X,\varphi)) \to B(L^p(\mu))$ and pairs $(\pi,v)$ where $\pi:C_0(X)\to B(L^p(\mu))$ is a non-degenerate representation acting by multiplication operators, $v:S\to \SPIso(L^p(\mu))$ is a semigroup homomorphism satisfying $v_{\UU(V,n-m,U)}\pi(a) = \pi( a\circ (\varphi|_{U}^{n})^{-1}\circ \varphi|_{V}^{m})v_{\UU(V,n-m,U)}$ ($a\in C_0(U)$), and $v_{\UU(U,0,U)}$ is a projection onto $\overline{\pi(C_0(U))L^p(\mu)}$ for any open $U\subseteq X$.

    \item\label{enu:Renault-Deaconu_groupoid2} If $S=S_c(X,\varphi)$ consists of compact open bisections (so $X$ is totally disconnected), then $F_T(S)\cong F(\G(X,\varphi))$ and the relation $\psi(\delta_{t})=v_{t}$ ($t\in S$) yields a bijective correspondence between representations $\psi : F(\G(X,\varphi))\ to B$ and tight representations $v:S\to B$ of $S$ in a Banach algebra $B$. 
\end{enumerate}
\end{thm}
\begin{proof}
Both $S(X,\varphi)$ and $S(\LL)$ are inverse semigroups that form a basis for the topology of $\G(X,\varphi)$ which is downward directed by inclusion, see Lemma~\ref{lem:inverse_semigroup_of_trivialtwist_bisections}. 
Hence $S(X,\varphi)\cap S(\LL)$ has the same properties and therefore is a wide inverse semigroup. 
The corresponding pair $(\alpha,u)$ is an example of a twisted action from Lemma~\ref{lem:twisted_action_from_twisted_groupoid} which models $(\G(X,\varphi),\LL)$ by Lemma~\ref{lem:twisted_groupoids_come_from_twisted_actions}. 
Hence the first part of the assertion follows from Corollary~\ref{cor:disintegration}.
Point \ref{enu:Renault-Deaconu_groupoid1} follows from Theorem~\ref{thm:spatial_representations_of_groupoid_algebras}\ref{enu:spatial_representations_of_groupoid_algebras2} and \ref{enu:Renault-Deaconu_groupoid2} follows from Corollary~\ref{cor:realization_of_ample_groupoids}.
\end{proof}



%
\subsection{Banach algebras associated to directed graphs}
\label{subsect:directed_graphs}

Let $Q= (Q^0,Q^1, r_Q, s_Q)$ be a \emph{directed graph} (sometimes also called a \emph{quiver}). 
So $Q^0$ is the set of vertices, $Q^1$ is the set of edges and $r_Q, s_Q : Q^{1}\to Q^0$ are the range and source maps. 
The vertices in $Q_{\textup{reg}} := \{ v\in Q^0: 1<|r_Q^{-1}(v)|<\infty\}$ are called \emph{regular} and the remaining $Q_{\textup{sing}} := Q^0\setminus Q_{\textup{reg}}$ are called \emph{singular}.
We denote by  $Q^n$ ($n>0$) the set of finite paths $\mu=\mu_1 \cdots \mu_n$, where $s_Q(\mu_i) = r_Q(\mu_{i+1})$ for all $i=1,\ldots ,n-1$. 
Then $|\mu|=n$ stands for the length of $\mu$ and $Q^* = \bigcup_{n=0}^\infty Q^n$ is the set of all finite paths (vertices are treated as paths of length zero). 
We denote by $Q^\infty$  the set of infinite paths and put $Q^{\leq \infty} := Q^*\cup Q^{\infty}$. 
The maps $r_Q,s_Q$ extend naturally to $Q^*$ and $r_Q$ extends also to $Q^\infty$.  
We recall the defintion of the Leavitt path algebra associated to $Q$, see \cite{AAM}.

\begin{defn} 
A \emph{$Q$-family in an algebra $B$} (over $\F=\R,\C$) is a pair $(P,T)$ where $P=\{P_v\}_{v\in Q^0}\subseteq B$ consists of pairwise orthogonal idempotents and $T=\{T_e, T_e^*\}_{e\in Q^1}\subseteq B$
consists of elements where $T_e^{*}$ is a generalized inverse of $T_e$ ($e\in Q^1$) satisfying 
\begin{enumerate}[label={(CK\arabic*)}]
    \item $T_e^*T_e=P_{s(e)}$ and $T_e T_e^*\leq P_{r(e)}$ for all $e\in Q^{1}$;
    \item $P_v=\sum_{e \in r^{-1}(v)} T_e T_e^*$ for all $v\in Q^{0}_{\reg}$.
\end{enumerate} 
\end{defn}

Let $(P,T)$ be a $Q$-family in $B$. 
For $\mu = \mu_1 \cdots \mu_n \in Q^{*}$ we put $T_\mu := T_{\mu_1} \cdots T_{\mu_n}$ and $T_\mu^* := T_{\mu_n}^*\cdots T_{\mu_1}^*$. 
Then the elements $T_{\mu}T_{\nu}^*$ form an inverse semigroup with multiplication from $B$ and so $\spane\{T_{\mu}T_{\nu}^*: \mu,\nu \in Q^*\}$ is a subalgebra of $B$ generated by $P\cup T$.
In fact, it is a $*$-algebra with the involution determined by $(T_{\mu}T_{\nu}^*)^* = T_{\nu}T_{\mu}^*$ ($\mu,\nu \in Q^*$). 
By definition the \emph{Leavitt path algebra} $\L(Q)$ is the algebra generated by a universal $Q$-family $(p,t)$. 
Thus 
\[
    \L(Q) = \spane\{t_{\mu}t_{\nu}^*: \mu,\nu \in Q^*\} ,
\]
and for any $Q$-family $(P,T)$ in $B$ there is a unique algebra homomorphism $\psi_{(P,T)} : \L(Q)\to B$ such that $\psi_{(P,T)}(p_v)=P_v$, $\psi_{(P,T)}(t_e)=T_e$ and $\psi_{(P,T)}(t_e^*)=T_e^*$, for $v\in Q^0$, $e\in Q^1$. 
We refer to \cite{AAM} for more details. 
The elements $\{t_{\mu}t_{\nu}^*: \mu,\nu \in Q^*\}$ form an inverse semigroup that can be described in terms of $Q$ as follows: put  $S_{Q}=\{(\mu,\nu)\in Q^* \times Q^*:s_Q(\mu) = s_Q(\nu)\}\cup \{0\}$ and 
define multiplication in $S_Q$ by the formula
\[
    (\mu,\nu)(\alpha,\beta) :=  \begin{cases}
                                    (\mu \alpha',\beta) & \text{if $\alpha = \nu \alpha'$}, \\
                                    (\mu, \beta \nu') & \text{if $\nu = \alpha \nu'$}, \\
                                    0 &\text{otherwise.}
                                \end{cases}%\label{eqn:Inverse Semigroup}
\]
Then the involution is given by $(\mu,\nu)^* =(\nu,\mu)$. 
It is  known to experts (and not hard to see) that tight representations of $S_{Q}$ in are in bijective correspondence with  $Q$-families. 
We deduce this from our results. 
The groupoid model for $\L(Q)$ (isomorphic to the tight groupoid $\G_T(S_Q)$) can be described as follows \cite[Examples 2.1, 3.2]{Clark_Sims}. 
We put $Q^*_{\mathrm{sing}} := \{ \mu\in Q^*: s_Q(\mu) \in Q_{\mathrm{sing}}\}$. 
For any $\eta\in Q^*\setminus Q^0$ let ${\eta}Q^{\leq \infty} := \{ \mu = \mu_1 \cdots \in Q^{\leq \infty} : \mu_1 \cdots \mu_{|\eta|} = \eta \}$ and for $v\in Q^0$ put ${v}Q^{\leq \infty} := \{\mu \in Q^{\leq \infty}: r_Q(\mu)=v\}$.
The \emph{boundary space} of $Q$ \cite[Section 2]{Webster} is the set 
\[ 
    \partial Q := Q^\infty\cup Q^*_{\mathrm{sing}} 
\]
equipped with the topology generated by the `cylinders' $Z(\mu):=\partial Q\cap {\mu}Q^{\leq \infty}$, $\eta \in Q^*$, and their complements. 
In fact, the sets $Z(\mu) \setminus \bigcup_{e \in F} Z(\mu e)$, where $\mu\in Q^*$ and $F\subseteq {s(\mu)}Q^1$ is a finite set of edges, form a basis of compact-open sets for the Hausdorff topology on $\partial Q$ \cite[Section 2]{Webster}. 
The one-sided \emph{topological Markov shift} associated to $Q$ is the map $\sigma:\partial Q\setminus Q^0 \to \partial Q$ defined, for $\mu=\mu_1\mu_2 \cdots \in \partial Q\setminus Q^0$, by the formulas  
\[
    \sigma(\mu) := \mu_2\mu_3 \cdots \ \text{ if $\mu \notin Q^1$}, \qquad \sigma(\mu) := s(\mu_1) \  \text{if $\mu=\mu_1 \in Q^1$} .
\]
This is a partial local homeomorphism on $\partial Q$. 
By definition the \emph{groupoid of the graph} $Q$ is the Deaconu--Renault grooupoid $\G_Q := \G(\partial Q,\sigma)$ of $\sigma$. 
Thus
\[
    \G_Q = \{ (\mu x, |\mu|-|\eta|, \eta x) : \mu, \eta \in Q^*, \ x\in \partial Q,\ s_Q(\mu) = s_Q(\nu) = r_Q(x) \} 
\]
is an ample Hausdorff groupoid with the topology generated by the `cylinders' $Z(\mu,\nu) := \{ (\mu x, |\mu|-|\eta|, \eta x) \in \G_Q \}$, for $\mu, \eta \in Q^*$ with $s_Q(\mu)=s_Q(\nu)$, and their complements. 
In fact, the sets $Z(\mu,\nu) \setminus \bigcup_{\alpha \in F} Z(\mu \alpha)$, where $\mu, \eta \in Q^*$ with $s_Q(\mu)=s_Q(\nu)$ and $F\subseteq {s(\mu)}Q^*$ is finite, form a basis of compact open bisections for $\G_Q$. 
Moreover, every compact open set in $\G_Q$ is a finite disjoint union of these basic sets.
The associated Steinberg algebra is 
\[
    A(\G_Q) := \spane \{ 1_{Z(\mu,\nu)} : \mu, \eta \in Q^*,\ s_Q(\mu)=s_Q(\nu) \} .
\]
Note that $A(\G_Q)\subseteq \mathcal{C}_c(\G_Q)$ is a dense $*$-subalgebra of $F(\G_Q)$, and $A(\partial Q) := \spane\{ 1_{Z(\mu)} : \mu\in Q^* \} \subseteq C_c(\partial Q)$ is a dense $*$-subalgebra in $C_0(\partial Q)$.

\begin{prop}[\cite{Clark_Sims}]\label{prop:Leavitt_groupoid_model}
There is an algebra isomorphism $\Psi:\LL(Q)\to A(\G_Q)$ that maps $t_{\mu}t_{\nu}^*$ to $1_{Z(\mu,\nu)}$ to , $\mu,\nu\in Q^*$.
It is uniquely determined by $\Psi(p_v) = 1_{Z(v)}$, $\Psi(t_e)=1_{Z(e,s_Q(e))}$ and $\Psi(t_e^*)=1_{Z(s_Q(e),e)}$, for $v\in Q^0$, $e\in Q^1$. 
In particular we have an isomorphism $ A(\partial Q) \cong \spane\{t_{\mu}t_{\mu}^*: \mu \in Q^*\}$ where $1_{Z(\mu)}\mapsto t_{\mu}t_{\mu}^*$.
\end{prop}

\begin{lem}\label{lem:Webster_lemma}
For any $Q$-family $(P,T)$ in an algebra $B$ and any finite $F\subseteq Q^*$ the elements
\[
    P^F_\mu := \prod_{\mu\mu'\in F\setminus \{\mu\}} T_\mu T_\mu^*-  T_{\mu\mu'} T_{\mu\mu'}^*, \qquad \mu \in F, 
\]
(with the convention $P^F_\mu:= T_\mu T_\mu^*$ if the product is over the empty set)
are mutually orthogonal idempotents such that $T_{\mu}T_{\mu}^* = \sum_{\mu\mu'\in F\setminus \{\mu\}}P^F_\mu$, so they span the algebra $\spane \{ T_{\mu}T_{\mu}^* : \mu \in F \}$.
If $B$ is a Banach algebra, then the map $(A(\partial Q), \|\cdot\|_{\infty}) \rightarrow \spane\{T_{\mu}T_{\mu}^*: \mu \in Q^*\}\subseteq B$; $1_{Z(\mu)}\mapsto t_{\mu}t_{\mu}^*$ is contractive if and only if for every finite $F\subseteq E^*$ the idempotents $\{P^F_\mu\}_{\mu \in F}$ are jointly $\F$-contractive (Definition~\ref{defn:jointly_contractive}).
\end{lem}
\begin{proof}
The first part follows from \cite[Lemma 3.1]{Sims_Webster}, see also \cite[Lemma 3.4]{Webster}.
This corresponds to the fact every element $f\in A(\partial Q)$ can be written in the form $f=\sum_{\mu \in F} a_{\mu} 1_{Z(\mu, F)}$ where $Z(\mu, F):=Z(\mu) \setminus \bigcup_{\mu' \in F\setminus \{\mu\}} Z(\mu\mu')$ are pairwise disjoint compact open sets for all $\mu\in F$ and a fixed finite set $F\subseteq Q^*$. 
Then $\|f\|_{\infty}=\max_{\mu\in F} |a_{\mu}|$, and the map in the second part of the assertion sends $f$ to $\sum_{\mu \in F} a_\mu{P^F_\mu}$, so it is contractive if $\sum_{\mu \in F} a_\mu{P^F_\mu}\leq \max_{\mu\in F} |a_{\mu}|$ for all choices of $F$ and $a_\mu$. 
\end{proof}

\begin{defn}\label{defn:Banach_space_Q_family}
A \emph{Banach $Q$-family} is a $Q$-family $(P,T)$ in a Banach algebra $B$ where $P\cup T\subseteq B_1$ and for any finite $F\subseteq Q^*$ the idempotents $\{P^F_\mu\}_{\mu \in F}$ are jointly $\F$-contractive (Definition~\ref{defn:jointly_contractive}). 
A \emph{spatial $Q$-family on $L^p(\mu)$}, where $p\in [1,\infty]$ and $\mu$ is localizable, is a $Q$-family $(P,T)$ in the algebra $B(L^p(\mu))$ where $P\cup T\subseteq \SPIso(L^p(\mu))$ are spatial partial isometries (then $T_e^*$ is the adjoint of $T_e$ in the sense of the inverse semigroup $\SPIso(L^p(\mu))$). 
%We say that  $(P,T)$  is \emph{non-degenerate} if idempotents in $P$ sum up strongly to identity on $K$. 
We define the \emph{universal graph Banach algebra} $F(Q)$ of $Q$ to be the completion of $\LL(Q)$ in 
\[
    \| b \|_{\max} = \sup \{ \psi_{(P,T)}(b): \text{$(P,T)$ is a $Q$-family in a Banach algebra} \}.
\]
For $p\in [1,+\infty]$ we define the \emph{graph $L^p$-operator algebra} $F^p(Q)$ to be the completion of $\L(Q)$ in the norm $\|b\|_{L^p}=\sup\{\psi_{(P,T)}(b): \text{$(P,T)$ is a $Q$-family on an $L^p$-space} \}$. 
\end{defn}

\begin{rem}
Lemma~\ref{lem:representation_characterization} implies that a spatial $Q$-family on $L^p(\mu)$ is a Banach $Q$-family in $B(L^p(\mu))$, and any $Q$-family $(P,T)$ in a $C^*$-algebra $B$ consisting of contractive elements is a Banach $Q$-family in $B$.
\end{rem}

\begin{thm}\label{thm:representations_of_graph_algebras}
For any directed graph $Q$ we have natural isometric isomorphisms $F(Q)\cong F(\G_Q)\cong F_T(S_Q)$.
In particular: 
\begin{enumerate}
    \item\label{enu:representations_of_graph_algebras1} for any Banach algebra $B$ the relations $\psi(p_v)=P_v=V_{(v,v)}$, $\psi(t_e)=T_e=V_{(e,s(e))}$ and $\psi(t_e^*)=T_e^*=V_{(s(e),e)}$, for $v\in Q^0$, $e\in Q^1$, establish bijective correspondences between representations $\psi:F(Q) \to B$, Banach $Q$-families $(P,T)$ in $B$, and tight Banach algebra representations $V:S_Q\to B_1$.

    \item\label{enu:representations_of_graph_algebras2} for every $p\in [1,+\infty]$ the isomorphism $F(Q)\cong F(\G_Q)$ descends to an isometric isomorphism $F^p(Q)\cong F^p(\G_Q)$ and $F^p(Q)$ is a  universal Banach algebra for spatial $Q$-families, i.e.
\[
    \| b \|_{L^p} = \sup\{\psi_{(P,T)}(b): \text{$(P,T)$ is a spatial $Q$-family on an $L^p$-space} \}.
\]
    If $\F=\C$ and $\mu$ is localizable, then non-degenerate representations of $F^p(Q)$ on  $L^p(\mu)$ arise from non-degenerate spatial $Q$-families on $L^p(\mu)$.
\end{enumerate}
\end{thm}
\begin{proof}
The collection $S := \{ Z(\mu,\nu) : \mu, \eta \in Q^*, s_Q(\mu)=s_Q(\nu)\} \cup \{\emptyset\}$ forms a wide inverse semigroup of compact open bisections in $\G_Q$ that separates the points in $\G_{Q}$.
Thus $\G_Q\cong \G_T(S)$ is the tight groupoid of $S$ by Corollary~\ref{cor:realization_of_ample_groupoids}. 
We have natural semigroup isomorphisms $S\cong S_Q\cong \{t_{\mu}t_{\nu}^*: \mu,\nu \in Q^*\}$ from  Proposition~\ref{prop:Leavitt_groupoid_model}. 
Thus by Theorem~\ref{thm:tight_inverse_semigroups_Banach_representations} and Lemmas~\ref{lem:representation_characterization} and \ref{lem:Webster_lemma} the relations in \ref{enu:representations_of_graph_algebras1} establish bijective correspondences between the corresponding objects. 
For a fixed $\mu$ and $p\in [1,\infty]$ they restrict to bijective correspondences between spatial $Q$-families $(P,T)$ on $L^p(\mu)$, tight homomorphisms $V:S \to\SPIso(L^p(\mu))$ and representations  $\psi:F(Q)=F(\G_Q) \to B(L^p(\mu))$ such that $\psi(C_0(\partial Q))$ consists of multiplication operators.
This and Theorem~\ref{thm:spatial_representations_of_groupoid_algebras} gives \ref{enu:representations_of_graph_algebras2}. 
\end{proof}

\begin{rem}
Theorem~\ref{thm:representations_of_graph_algebras} implies that our definition of the algebras $F^p(Q)$ agrees with the one given in \cite{cortinas_rodrogiez} for $p\in [1,\infty)$. 
We also have $F^p(Q)\cong F^q(Q)^{\op}$ for $p,q\in [1,\infty]$ with $1/p+1/q=1$, by Proposition~\ref{prop:initial_on_L^p_full}.
\end{rem}

\begin{rem}[Banach--Cuntz algebras] 
For any $n>1$ consider the graph $Q_n$ with a single vertex $v$ and $n$ edges $e_1,...,e_n$. 
Then our $F(Q_n)$ is a Banach algebra analogue of the Cuntz $C^*$-algebra $F_n$.
Daws and Horv\'{a}th~\cite{Daws_Horwath} proposed a different construction: the \emph{Daws--Horv\'{a}th algebra} $\mathcal{DH}_n$ is by definition the quotient of the Banach $*$-algebra $\ell^{1}(S_{Q_{n}}\setminus \{0\})$ by an ideal $J$ generated by the element $\delta_{(v,v)} - \sum_{i=1}^{n} \delta_{(e_i,e_i)}$.
Thus it is the universal Banach algebra generated by a contractive $Q_n$-family, or equivalently the completion of $\LL(Q_n)$ in the largest submutliplicative norm $\|\cdot\|_{DH}$ such that $\|t_{e_i}\|_{DH}=\|t_{e_i}^*\|_{DH}=\|t_{v} \|_{DH}=1$ for $i=1, \ldots ,n$.
As shown in \cite{Daws_Horwath}, $\mathcal{DH}_n =\overline{\LL(Q_n)}^{\|\cdot\|_{DH}}$ is naturally a simple purely infinite Banach $*$-algebra.
Thus we have an injective $*$-homomorphism $\mathcal{DH}_n \donto F(Q_n)$ which is the identity on $\LL(Q_n)$. 
It seems unlikely that this homomorphism is surjective. 
To prove this one could try to show that there are no bounded traces on $F(Q_n)$, cf. \cite[Section 4]{Daws_Horwath}.
\end{rem}




\bibliographystyle{plain}
\bibliography{bibliographygroupoidbanachalgebras}


\end{document}


%old bibliography ;eft here just in case

%\begin{thebibliography}{fulllllll}

\bibitem[Aba04]{Abadie}
F. Abadie, \emph{On partial actions and groupoids}, Proc. Amer. Math. Soc., \textbf{132}
(2004), 1037--1047.

\bibitem[AAM17]{AAM}
G. Abrams, P. Ara, M. S. Molina, Leavitt Path Algebras. Lecture Notes in Mathematics, Vol. 2191. (2017) London: Springer

\bibitem[Agn21]{Agniel}
V. Agniel,  
\emph{$L^p$-projections on subspaces and quotients of Banach spaces}
Adv. Oper. Theory \textbf{6}, 38 (2021).

%\bibitem[A-D97]{A-D} C. Anantharaman-Delaroche, \emph{Purely infinite
%    {$C^*$}-algebras arising from dynamical systems}, Bull. Soc. Math. France
%    \textbf{125} (1997), 199--225.

%\bibitem[And66]{Ando66}
%T. Ando, \emph{Contractive projections in $L^p$ spaces}, Pacific J. Math. \textbf{17} (1966)
%391--405.

\bibitem[AGP02]{AGP} P.~Ara, K.~R.\  Goodearl, and E.~Pardo,
{\emph{$K_0$~of purely infinite simple regular rings}},
K-Theory {\textbf{26}}(2002), 69--100.

\bibitem[Are04]{Arens}
R. Arens, \emph{Representation of  $^*$-algebras} Duke Math. J. \textbf{14}(2) (1947), 269--282.

\bibitem[AO22]{Austad_Ortega}
A. Austad, E. Ortega, \emph{Groupoids and Hermitian Banach *-algebras}
Int. J.  Math.  \textbf{33}, No. 14, (2022) 2250090. 

\bibitem[BKM24]{BKM}
K. Bardadyn, B. Kwa\'sniewski, A. Mckee,  \emph{Banach algebras associated to  twisted \'etale groupoids II:
 simplicity and pure infiniteness} in preparation

%\bibitem[BL74]{Bernau_Lacey}
%S. Bernau,  H. Lacey, \emph{The range of a contractive projection on an $L^p$-space},
%Pacific J. Math. \textbf{53} (1974), 21--41.

\bibitem[Bla98]{Bl3} B.~Blackadar,
{\emph{K-Theory for Operator Algebras}}, 2nd ed.,
MSRI Publication Series~{\textbf{5}},
Cambridge University Press, Cambridge, New York, Melbourne, 1998.

\bibitem[BDE77]{BDEGGMM} 
E. Behrends, R. Danckwerts, R. Evans, S. G\"obel, P. Greim, K. Meyfarth,  W.
M\"uller, $L^p$-structure in real Banach spaces, Lecture Notes in Mathematics, Vol. 613,
Springer-Verlag, Berlin-New York, 1977.


\bibitem[BL77]{Bernau_Lacey2} 
S. J. Bernau, H. E. Lacey
\emph{Bicontractive projections and reordering of $L_p$-spaces}, Pacific J. Math.
\textbf{ 69} (1977), No. 2, 291--302.

\bibitem[BGL21]{BGL} 
 D. P. Blecher, S. Goldstein, L.E. Labuschagne, \emph{Abelian von
Neumann algebras, measure algebras and $L^\infty$-spaces}, Expositiones Mathematicae (2022),


\bibitem[BM04]{BM} D.~P.\  Blecher and C.~Le Merdy,
\emph{Operator Algebras and their Modules: An Operator Space
  Approach}, London Mathematical Society Monographs, New Series, no.~30.
Oxford Science Publications. The Clarendon Press, Oxford University Press, Oxford, 2004.

\bibitem[BP19]{BP} 
D. P. Blecher and N. C. Phillips, \emph{$L^p$-operator algebras with approximate identities, I}, Pacific J. Math. \textbf{303} (2019), 401--457.




\bibitem[Bon54]{Bns} F.~F.\  Bonsall,
{\emph{A minimal property of the norm in some Banach algebras}},
J.~London Math.\  Soc.\  {\textbf{29}}(1954), 156--164.


\bibitem[BD71]{BD}
F. F. Bonsall and J. Duncan, Numerical ranges of operators on normed spaces and of elements of normed algebras, London Mathematical Society Lecture Note Series 2, Cambridge University Press, London-New York, 1971.

%\bibitem[BCFS14]{BCFS}
%J. Brown, L. O. Clark, C. Farthing, A. Sims, \emph{Simplicity of algebras associated to
%\'etale groupoids}. Semigroup Forum \textbf{88} (2014),  433--452. 

%\bibitem[BFPR21]{BFPR} J. H. Brown, A. H. Fuller, D. R. Pitts, S. A. Reznikof, \emph{Graded
%$C^*$-algebras and twisted groupoid $C^*$-algebras}, New York J. Math. 27 (2021), 205--252.

%\bibitem[BRV10]{BroRaeVit}
%N. Brownlowe, I. Raebrun,  S. T. Vittadello, \emph{Exel's crossed product for non-unital $C^*$-algebras}, Math. Proc. Camb. Phil. Soc. \textbf{149} (2010), 423--444.

\bibitem[BE11]{Buss_Exel} 
A. Buss,  R. Exel, 
\emph{Twisted actions and regular Fell bundles over inverse semigroups},
Proc. London Math. Soc. (3) \textbf{103} (2011) 235--270.

\bibitem[BE12]{Buss_Exel2} 
A. Buss,  R. Exel,  \emph{Fell bundles over inverse semigroups and twisted \'etale groupoids}, J. Operator Theory
\textbf{67} (2012), 153--205.

\bibitem[BS21]{Buss_Sims} 
 A. Buss,  A. Sims, \emph{Opposite algebras of groupoid $C^*$-algebras}. Isr. J. Math. \textbf{244} (2021), 759--774. 
%\bibitem[Raa21]{Raad}  A. Raad, Existence and Uniqueness of Inductive Limit Cartan Subalgebras in Inductive Limit $C^*$-algebras, PhD thesis (2021)

\bibitem[CGT19]{cgt}
Y. Choi, E. Gardella, H. Thiel, \emph{Rigidity results for $L^p$-operator algebras and applications}, Arxiv preprint math.OA/1909.03612, 2019.


%\bibitem[CEPSS19]{CEPSS}
% L. O. Clark, R. Exel, E. Pardo, A. Sims, C.
%Starling, \emph{Simplicity of algebras associated to non-Hausdorff groupoids},
%Trans. Amer. Math. Soc. \textbf{372} (2019), no. 5, 3669--3712.

\bibitem[CS15]{Clark_Sims}
L.O. Clark, A. Sims, \emph{Equivalent groupoids have Morita equivalent Steinberg algebras}, J. Pure Appl. Algebra, \textbf{219}(6) (2015), 2062--2075.

\bibitem[CR19]{cortinas_rodrogiez}
G. Corti\~nas and M. E. Rodriguez, \emph{$L^p$-operator algebras associated with oriented graphs},
J. Operator Theory \textbf{81} (2019),  225--254.

%\bibitem[Cun81]{Cuntz}
%J. Cuntz, \emph{$K$-theory for certain $C^*$-algebras}, Annals of Math., 113 (1981), 181--197.

\bibitem[Dal00]{Dales}
H. G. Dales, Banach Algebras and Automatic Continuity, Clarendon Press, Oxford, 2000.


\bibitem[Daw10]{Daws} 
M. Daws, \emph{Multipliers, self-Induced and dual Banach algebras},
Dissertationes Math. \textbf{470} (2010), 62 pp.

\bibitem[DH22]{Daws_Horwath} 
M. Daws, B. Horv\'ath,
\emph{A purely infinite Cuntz-like Banach $*$-algebra with no purely infinite ultrapowers,
} J. Funct. Anal.,
\textbf{283} 2022, 109488

\bibitem[DM14]{Donsig_Milan} A. P. Donsig,  D. Milan,
\emph{Joins and covers in inverse semigroups and tight $C^*$-algebras},
Bull. Aust. Math. Soc. \textbf{90} (2014), 121--133.


\bibitem[DDW11]{DDW} S.~Dirksen, M.~de Jeu, and M.~Wortel,
{\emph{Crossed products of Banach algebras. I}},
preprint (arXiv:1104.5151v2)

%\bibitem[DE05]{D-E}
%M. Dokuchaev, R. Exel, 
5\emph{Associativity of crossed products by partial actions, enveloping actions and partial representations}
%Trans. Amer. Math. Soc., \textbf{357} (2005), 1931--1952.

%\bibitem[DWZ]{DWZ}
%A. Duwenig, D. P. Williams, J. Zimmerman
%\emph{Renault's $j$-map for Fell bundle $C^*$-algebras}, J. Math. Anal. Appl.
%\textbf{516}  (2022), 126530.


%\bibitem[Dix51]{Dixmier} 
%J. Dixmier, \emph{Sur certains espaces consideres par M. H. Stone}, Summa Bras. Math., \textbf{2} (1951), 151--181.

\bibitem[Exe94]{Exel_circle} R. Exel,
\emph{Circle actions on $C^*$-algebras, partial automorphisms, and a generalized Pimsner-Voiculescu exact sequence},
J. Funct. Anal., \textbf{122} (2) (1994),  361--401.

\bibitem[Exe97]{Exel_twisted_partial}
R. Exel, \emph{Twisted partial actions a classification of $C^*$-algebraic bundles}, Proc. London
Math. Soc \textbf{74} (3), (1997), 417--443.

\bibitem[Exe98]{Exel^partial_vs_inverse_semigroup}
 R. Exel, \emph{Partial actions of groups and actions of inverse semigroups}, Proc. Amer. Math.
Soc. \textbf{126} (1998),  3481--3494. 



\bibitem[Exe08]{Exel} R. Exel, \emph{Inverse semigroups and combinatorial $C^*$-algebras}, Bull. Braz. Math. Soc. (N.S.) \textbf{39}
(2008), no. 2, 191--313.

\bibitem[Exe09]{Exel_tight}
R. Exel, \emph{Tight representations of semilattices and inverse semigroups}
Semigroup Forum, \textbf{79} (2009), 159--182. 


\bibitem[Exe17]{Exel_book} R. Exel, 
Partial dynamical systems, Fell bundles and applications, Mathematical Surveys and
Monographs, vol. \textbf{224}, Amer. Math. Soc., Providence, RI, 2017

\bibitem[ELQ02]{ELQ} R.~Exel, M.~Laca,  J.~Quigg
{\emph{Partial dynamical systems and $C^*$- algebras generated by partial isometries.}}
 J. Operator Theory, (2002)  \textbf{47}(1), 169--186.
\bibitem[Fre02]{Fremlin}
 D. H. Fremlin, Measure theory. Vol.2, Torres Fremlin, Colchester, 2002





\bibitem[GL17]{Gardella_Lupini17}
E. Gardella,  M. Lupini, \emph{Representations of \'etale groupoids on $L^p$-spaces},
Adv. Math. \textbf{318} (2017), 233--278.

\bibitem[GT15]{Gardella_Thiel15}
E. Gardella, H. Thiel, \emph{Group algebras acting on $L^p$-spaces}, J. Fourier Anal. Appl. \textbf{21} (2015), 1310--1343.

\bibitem[GT20]{Gardella_Thiel1}
E. Gardella, H. Thiel, \emph{Extending representations of Banach algebras to their
biduals}, Math. Z. 294 (2020), 1341--1354.


\bibitem[GT21]{Gardella_Thiel2}
E. Gardella, H. Thiel, \emph{Isomorphisms of algebras of convolution operators}, to appear in Ann. Sci. \'Ecole Nor. Sup. (2021)


\bibitem[Hah78]{Hahn}
P. Hahn, \emph{The regular representations of measure groupoids}
Trans. Amer. Math. Soc. \textbf{242} (1978), 35--72.


\bibitem[HW70]{Halmos}
P. R. Halmos, L. J. Wallen, \emph{Powers of partial isometries}, Indiana Univ. Math. J. \textbf{19},
(1970), p. 657--663.

\bibitem[HO22]{Hetland_Ortega}
E. V. Hetland, E. Ortega, \emph{Rigidity of twisted groupoid $L^p$-operator algebras}, preprint 2022, arXiv:2209.11447






\bibitem[HK05]{Holtz_Karow}
O. Holtz and M. Karow, Real and Complex Operator Norms, 2005, 13 pages. preprint
http://arxiv.org/abs/math/0512608.


\bibitem[IS08]{Ilie_Stokke}
M. Ilie and R. Stokke, \emph{Weak$^*$-continuous homomorphisms of Fourier-Stieltjes algebras},
Math. Proc. Cambridge Philos. Soc. 145 (2008), 107-120.


\bibitem[Jam07]{Jamison}
J. E. Jamison,  \emph{Bicircular projections on some Banach spaces}, Linear Algebra Appl. \textbf{420} (2007), no. 1, 29--33.

\bibitem[Kap49]{Kaplansky}
I. Kaplansky, \emph{Normed algebras}, Duke Math. J., \textbf{16} (1949), 399--417.

\bibitem[Kum86]{Kumjian0} A. Kumjian, \emph{On $C^*$-diagonals}, Canad. J. Math. \textbf{38} (1986), 969--1008.

\bibitem[KL17]{Kumjian_Li}
A. Kumjian, H. Li, \emph{Twisted topological graph algebras are twisted groupoid $C^*$-algebras}  J. Operator Theory \textbf{78}, no. 1 (2017), 201--25.

\bibitem[KL20]{kwa-leb} B. K. Kwa\'sniewski, A. V. Lebedev, \emph{Variational principles for spectral radius of
 weighted endomorphisms of $C(X,D)$},  Trans. Amer. Math. Soc. 373, No 4 (2020),  2659--2698.


%  \bibitem[KM18]{Kwa-Meyer_studia}
% B. K. Kwa\'sniewski, R. Meyer, \emph{Aperiodicity, topological freeness and pure outerness:
%from group actions to Fell bundles} Studia Math. \textbf{241}, no. 3 (2018) 257--303.

  \bibitem[KM20$_{1}$]{Kwa-Meyer0}
 B. K. Kwa\'sniewski, R. Meyer, \emph{Stone duality and quasi-orbit spaces for generalised $C^*$-inclusions}, Proc. Lond. Math.
Soc. (3) 121 (2020), no. 4, 788--827.


%\bibitem[KM20$_{2}$]{KwaMeyer}  B. K. Kwa\'{s}niewski,  R. Meyer,
%\emph{Noncommutative Cartan $C^*$-subalgebras},
%Trans. Amer. Math. Soc. \textbf{373} (2020), no. 12, 8697--8724

%  \bibitem[KM21]{Kwa-Meyer}  
%	 B. K. Kwa\'{s}niewski,  R. Meyer,
%\emph{Essential crossed products by inverse semigroup actions: simplicity and pure infiniteness}, Doc. Math. 26 (2021), 271--335.


%  \bibitem[LS96]{Laca-Spielberg}   M. Laca, J. Spielberg, \emph{Purely infinite 
%$C^*$-algebras from boundary actions of discrete groups}, J. reine
%angew. Math. 480 (1996), 125–139.

\bibitem[Li03]{Li_book}
B. Li, Real operator algebras, World Scientific, Singapore 2003.


%  \bibitem[JR00]{Jolissaint-Robertson}  
% P. Jolissaint and G. Robertson, \emph{Simple purely infinite $C^*$-
%algebras and $n$-filling actions}. J. Funct. Anal.
%\textbf{175} (2000), no. 1, 197--213.

  \bibitem[Lac74]{Lacey}  
H. E. Lacey, The isometric theory of classical Banach spaces, Grundlehren der Math.
Wissenschaften 208, Springer, Heidelberg, 1974.

%  \bibitem[Lai74]{Lai}  
%H. C. Lai, 
%\emph{Multipliers of a Banach algebra in the second conjugate algebra as an idealizer},
%Tohoku Math. J. (2) \textbf{26} (1974), 431--452. 

% \bibitem[Ma22]{Ma} 
%X. Ma, \emph{Purely infinite locally compact Hausdorff \'etale groupoids and their $C^*$-algebras},
% IMRN,  \textbf{2022}  (2022),  8420--8471.

\bibitem[Mbe04]{Mbekhta}
M. Mbekhta  \emph{Partial isometries and generalized inverses}, Acta Sci. Math.,  \textbf{70}  (2004),  767-781.

\bibitem[McM95]{McClanahan}
K. McClanahan
\emph{$K$-theory for partial crossed products by discrete groups},
J. Funct. Anal., \textbf{130} (1) (1995),  77--117.


\bibitem[MW08]{Muhly_Willimas}
P. S. Muhly,  D. P. Williams, \emph{Equivalence and disintegration theorems for Fell bundles and their $C^*$-algebras}, Dissertationes Math. (Rozprawy Mat.) \textbf{456} (2008), 1--57.

%\bibitem[Nek04]{Nekrashevych}
%V. Nekrashevych \emph{Cuntz--Pimsner algebras of group actions}, J. Operator Theory \textbf{52} (2004),
%223--249.
%\bibitem[Nek04]{Nekrashevych2}
%V. Nekrashevych \emph{$C^*$-algebras and self-similar groups}, J. Reine Angew. Math. \textbf{630} (2009), 59--123.

\bibitem[Pat99]{Paterson} A. L. T. Paterson, Groupoids, inverse semigroups, and their operator algebras, Progr. Math., vol. 170, Birkh\"auser
Boston Inc., Boston, MA, 1999.

\bibitem[PB9]{Pelczynski}
A. Pe\l{}czy\'nski,  Cz. Bessaga, Some aspects of the present theory of Banach
spaces. In S. Banach, Oeuvres, volume II, Travaux sur l'analyse fonctionelle, pages 223--302.
PWN – Editions Scientiques de Pologne, Warsaw, 1979.


\bibitem[Phi12]{PhLp1} N.~C.\  Phillips,
{\emph{Analogs of Cuntz algebras on $L^p$~spaces}},
preprint (arXiv: 1201.4196).

\bibitem[Phi13$_a$]{PhLp2a} N.~C.\  Phillips,
{\emph{Simplicity of UHF and Cuntz algebras on $L^p$~spaces}},
preprint (arXiv: 1309.0115).

\bibitem[Phi13$_b$]{Phillips} N.~C.\  Phillips,
\emph{Crossed products of $L^p$ operator algebras and the $K$-theory of Cuntz algebras on $L^p$ spaces},
preprint (arXiv: 1309.6406).

\bibitem[PH15]{PH} S.~Pooya and S.~Hejazian,
\emph{Simplicity of reduced $L^p$~operator crossed products
  with Powers groups},
J. Operator Theory \textbf{74} (2015), no. 1, 133--147.

\bibitem[QR97]{Quigg-Raeburn}
J. C. Quigg and I. Raeburn, \emph{Characterisations of crossed products by partial actions}, J. Operator Theory, \textbf{37} (1997), 311--340.

%\bibitem[RSY03]{RSY03} I. Raeburn, A. Sims and T. Yeend, \emph{Higher-rank graphs
%and their $C^*$-algebras}, Proc. Edinb. Math. Soc. \textbf{46}
%(2003), 99--115.

%\bibitem[RSY04]{RSY04} I. Raeburn, A. Sims and T. Yeend, \emph{The $C^{\ast }$-algebras of finitely aligned higher-rank graphs}, J. Funct. Anal.
%\textbf{213} (2004), 206--240.

%\bibitem[RS20]{Rainone-Sims}
%T. Rainone,  A. Sims, \emph{A dichotomy for groupoid $C^*$-algebras}, Ergod. Theory Dyn. Sys. \textbf{40}
%(2020), 521--563.

\bibitem[Ren80]{Renault_book} J.\,Renault, 
 A groupoid approach to $C^*$-algebras, Lecture Notes in Mathematics, vol. 793,
Springer-Verlag, New York, 1980. 

%\bibitem[Ren97]{Renault_Fourier}
%J. N. Renault, \emph{The Fourier algebra of a measured groupoid and its multipliers}, J. Functional
%Anal. \textbf{145}(1997), 455--490.

\bibitem[Ren00]{Renault2000} J. Renault,
 \emph{Cuntz-like Algebras}, Proceedings of the 17th International
Conference on Operator Theory (Timisoara 98), The Theta Fondation (2000), 371--386.

\bibitem[Ren08]{Re} J.\,Renault,  \emph{Cartan subalgebras in {$C^\ast$}-algebras}, Irish Math. Soc. Bull. \textbf{61}(2008), 29--63.


\bibitem[Ros16]{Rosenberg} J. Rosenberg, \emph{Structure and applications of real $C^*$-algebras}, Contemporary Math., vol \textbf{671}, Amer. Math. Soc, Providence, RI, 2016,  235--258.

\bibitem[Run01]{Runde}
V. Runde, \emph{Amenability for dual Banach algebras}, Studia Math. \textbf{148} (2001), 47--66.

\bibitem[Sch93]{Schroder}
Herbert Schr\"oder, $K$-theory for real $C^*$-algebras and applications, Pitman Research Notes in Mathematics Series, vol. 290, Longman Scientific \& Technical, Harlow; and John Wiley \& Sons, Inc., New York, 1993.

\bibitem[Sie97]{Sieben} N. Sieben, \emph{$C^*$-crossed products by partial actions and actions of inverse semigroups},  J. Austral. Math. Soc. Ser. A \textbf{63} (1997), 32--46.

\bibitem[Sie98]{Sieben98}
 N. Sieben, \emph{$C^*$-crossed products by twisted inverse semigroup actions}, J. Operator Theory \textbf{39} (1998)
361--393.

\bibitem[Sim20]{Sims}
A. Sims, 
 Hausdorff \'etale groupoids and their $C^*$-algebras,   `Operator algebras and dynamics: groupoids, crossed products and Rokhlin dimension' (F. Perera, Ed.) in Advanced Courses in Mathematics. CRM Barcelona, Birkh\"auser, 2020. [http://arxiv.org/abs/1710.10897]

\bibitem[SW10]{Sims_Webster}
A. Sims, S. B. G. Webster, \emph{A direct approach to co-universal algebras associated to directed graphs}, Bull. Malays. Math. Sei. Soc. (2) \textbf{33} (2010), 211--220. 


\bibitem[SZ04]{Stacho_Zalar}
L. L. Stacho, B. Zalar, Bicircular projections and characterization of Hilbert spaces, Proc. Amer. Math. Soc. \textbf{132} (2004)
 3019--3025.

\bibitem[Ste10]{Steinberg}
B. Steinberg, \emph{A groupoid approach to discrete inverse semigroup algebras}, Adv. Math., \textbf{223} (2010), 689--727. 


\bibitem[SS21]{Steinberg_Szakacs}
B. Steinberg, N. Szak\'acs,
\emph{Simplicity of inverse semigroup and \'etale groupoid algebras},
Adv. Math., \textbf{380} (2021), 107611,


%\bibitem[Suz17]{Suzuki}
%Y. Suzuki, \emph{Construction of minimal skew products of amenable minimal dynamical systems},
%Groups Geom. Dyn. \textbf{11} (2017), no. 1, 75--94.

\bibitem[Tza69]{Tzafriri}
L. Tzafriri, \emph{Remarks on contractive projections in $L^p$-spaces}, Israel J. Math. \textbf{7}
(1969), 9--15.
\bibitem[Web14]{Webster}
S. B. G. Webster, \emph{The path space of a directed graph}, Proc. Amer. Math. Soc., \textbf{142}(1) (2014), 213--225.


\end{thebibliography}





