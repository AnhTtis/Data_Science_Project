% Use only LaTeX2e, calling the article.cls class and 12-point type.

\documentclass[12pt]{article}

% Users of the {thebibliography} environment or BibTeX should use the
% scicite.sty package, downloadable from *Science* at
% http://www.sciencemag.org/authors/preparing-manuscripts-using-latex 
% This package should properly format in-text
% reference calls and reference-list numbers.

\usepackage{scicite}

\usepackage{times}
\usepackage{graphicx}
\usepackage{float}
\usepackage{amsmath}
\usepackage{upgreek}
\usepackage{multicol}

% The preamble here sets up a lot of new/revised commands and
% environments.  It's annoying, but please do *not* try to strip these
% out into a separate .sty file (which could lead to the loss of some
% information when we convert the file to other formats).  Instead, keep
% them in the preamble of your main LaTeX source file.


% The following parameters seem to provide a reasonable page setup.

\topmargin 0.0cm
\oddsidemargin 0.2cm
\textwidth 16cm 
\textheight 21cm
\footskip 1.0cm


%The next command sets up an environment for the abstract to your paper.

\newenvironment{sciabstract}{%
	\begin{quote} \bf}
	{\end{quote}}



% Include your paper's title here

\title{Generation of cold polyatomic cations by reactive two-body ion-atom collisions} 


% Place the author information here.  Please hand-code the contact
% information and notecalls; do *not* use \footnote commands.  Let the
% author contact information appear immediately below the author names
% as shown.  We would also prefer that you don't change the type-size
% settings shown here.

\author
{
	Wei-Chen Liang$^{1\ast}$, Feng-Dong Jia$^{1\ast}$, Fei Wang$^{2}$, Xi Zhang$^{1}$, \\Yu-Han Wang$^{1}$, Jing-Yu Qian$^{1}$, Xiao-Qing Hu$^{3}$, \\Yong Wu$^{3}$, Jian-Guo Wang$^{3\dag}$, Ping Xue$^{2\ddag}$, and Zhi-Ping Zhong$^{1,4\S}$
	\\
	\footnotesize{$^{1}$School of Physical Sciences, University of Chinese Academy of Sciences, }\\
	\footnotesize{Beijing, 100049, People's Republic of China}\\\\
	\footnotesize{$^{2}$State Key Laboratory of Low-Dimensional Quantum Physics, }\\ \footnotesize{Department of Physics, Tsinghua University, }\\
	\footnotesize{Beijing, 100084, People's Republic of China}\\\\
	\footnotesize{$^{3}$Institute of Applied Physics and Computational Mathematics, }\\
	\footnotesize{Beijing 100088, People's Republic of China}\\\\
	\footnotesize{$^{4}$CAS Center for Excellence in Topological Quantum Computation, }\\
	\footnotesize{University of Chinese Academy of Sciences, }\\
	\footnotesize{Beijing, 100190, People's Republic of China}\\
	\\
	\normalsize{$^\dag$E-mail:  wang\_jianguo@iapcm.ac.cn.}\\
	\normalsize{$^\ddag$E-mail:  xuep@tsinghua.edu.cn.}\\
	\normalsize{$^\S$E-mail:  zpzhong@ucas.ac.cn.}\\
	\\
	\normalsize{$^\ast$These authors contributed equally to this work.}\\
}

% Include the date command, but leave its argument blank.

\date{}



%%%%%%%%%%%%%%%%% END OF PREAMBLE %%%%%%%%%%%%%%%%



\begin{document} 
	
	% Double-space the manuscript.
	
	\baselineskip24pt
	
	% Make the title.
	
	\maketitle 
	
	
	
	% Place your abstract within the special {sciabstract} environment.
	
	\begin{sciabstract}		
		
		The frontiers of chemical research have shifted from all-neutral cold chemistry to charged-neutral cold chemistry with the development of laser cooling and charged particle trapping technologies. One of the goals of cold chemical reactions is the formation  of cold molecules containing three or more atoms. However, no polyatomic cations have been generated in a laboratory through reactive ion-atom collisions. Here, we report the first creation of a series of cold ($\leq$ mK) polyatomic cations $^{87}$Rb$_\text{N}^+$ (N = 2, 3,…, 18) in the two-step CW-laser photoionization of laser-cooled $^{87}$Rb atoms in the ion-atom hybrid trap. These polyatomic cations occur in a large number ($\sim 10^6$ for $^{87}$Rb$_\text{N}^+$, N = 2,3,…,7) thanks to the experimental realization of spatial overlap of a dense ionic cloud and an atomic cloud as well as the storage of charged reaction products. We have for the first time directly observed these reaction products by combining time-of-flight mass spectrometry (TOF-MS) and resonant-excitation mass spectrometry (REMS). By controlling the interaction time of ion-atom collisions and the storage time of ions, we have experimentally validated the cascaded generation and dissociation characteristics of this series of polyatomic cations. The techniques for assembling and observing homonuclear polyatomic cations developed by us are applicable to any experiment involving an ion-atom hybrid trap. Our work paves the way for the investigation of cold-controlled ion-atom chemistry and the evolution of the interstellar medium.
		
	\end{sciabstract}
	
	
	
	
	% In setting up this template for *Science* papers, we've used both
	% the \section* command and the \paragraph* command for topical
	% divisions.  Which you use will of course depend on the type of paper
	% you're writing.  Review Articles tend to have displayed headings, for
	% which \section* is more appropriate; Research Articles, when they have
	% formal topical divisions at all, tend to signal them with bold text
	% that runs into the paragraph, for which \paragraph* is the right
	% choice.  Either way, use the asterisk (*) modifier, as shown, to
	% suppress numbering.
	
	%	\section*{Introduction}\label{sec-intro}
\begin{multicols}{2}
	
	Cold ion-atom systems in an ion-atom hybrid trap are a promising platform for studying cold ion-atom interactions and chemical reactions in the mK regime and below \cite{rmp2019,heazlewood_2021,puri_reaction_2019}(see Fig.\ref{fig1}(a) for a simplified image). The long-range potential between an ion and an atom is a $\propto r^{-4}$ (where $r$ is the internuclear distance between ion and atom) polarization potential. as shown in Fig.\ref{fig1}(c). Such a potential is expected to have a large ion-atom interaction cross section\cite{krukow_reactive_2016}, which is advantageous for research in such areas as cold controlled chemistry \cite{deiglmayr-2012,tomza-2015,puri-2017,kilaj-2018,tomza-2017,dorfler_long-range_2019,puri_reaction_2019,kas-2019}, many-body physics\cite{cote-2000,cote-2002,casteels_polaronic_2011,bissbort-solid-2013}, quantum information processing \cite{doerk-gate-2010}, and quantum simulation\cite{bissbort-solid-2013}. Moreover, cold ion-atom chemistry dominates the evolution of matter in the universe due to the low density and temperatures (on the order of several Kelvin and below) of typical interstellar clouds. The simplest polyatomic molecular ion H$_3^+$ is the most abundant molecular ion in the universe, and plays a key role in interstellar chemistry because it is the start of ion-neutral reactions in the interstellar medium\cite{geballe_key_2006}. 
	
	Molecular ions, which can be produced through reactive ion-atom collisions, offer distinct advantages in terms of their ease of trapping, detection, and possession of additional rotational and vibrational degrees of freedom when compared to atomic ions. Using cold ion-atom hybrid systems, experiments have produced diatomic molecular cations through reactive two-body ion-atom collisions or ion-atom-atom three-body collisions at higher atomic densities\cite{dieterle_transport_2021,rellergert_measurement_2011,hall_light-assisted_2011,hall_ion-neutral_2013,hall_light-assisted_2013,sullivan_role_2012,harter_single_2012,krukow_energy_2016,krukow_reactive_2016,dieterle_inelastic_2020,mohammadi_life_2021}. As for reactive two-body collisions, conservation of momentum requires the existence of an external field\cite{rmp2019}. Therefore, molecular ions can theoretically be produced via spontaneous ion-atom radiative association, photoassociation, and magnetoassociation\cite{rmp2019}. In order to realize magnetoassociation experimentally, the ultracold s-wave regime of ion-atom collisions is needed. This requires far more sophisticated experimental techniques than that are available in current research. To date, cold molecular ions experimentally produced by spontaneous two-body ion-atom radiative association in cold hybrid ion-atom systems\cite{rmp2019} are RbCa$^+$\cite{hall_ion-neutral_2013,hall_light-assisted_2011}, RbBa$^+$\cite{hall_light-assisted_2013}, CaYb$^+$\cite{rellergert_measurement_2011}, and CaBa$^+$\cite{sullivan_role_2012}. Ion-atom-atom three-body collisions play a role in  Rb$_2^+$\cite{dieterle_inelastic_2020,dieterle_transport_2021}, RbBa$^+$\cite{krukow_energy_2016,mohammadi_life_2021}, in which a single ion in a Paul trap was immersed in a cloud of ultracold atoms. 
	
	For polyatomic ions, although some polyatomic cations have been created in a lab\cite{jacox-polyatomic-2003,boustani-cluster-book,brechignac_charge_2000}, no polyatomic cations are generated in a laboratory through reactive ion-atom collisions. Given the rapid decline in production rate as the number of atoms increases, the likelihood of polyatomic cations being generated through multi-body non-sequential association in a typical magneto-optical trap (MOT) is improbable due to the low atomic density. The reaction product molecular ions and atoms collide to form molecular ions containing one more atom, and this process continues to obtain a series of polyatomic ions as long as there are enough remaining atoms and molecular ions as reaction products. In fact, the rates of reactive two-body ion-atom collisions are usually explained using the classical Langevin-capture model\cite{langevin,levine-2009}. Only the reduced mass of the product, atomic polarizabilities, and ionic charge are related to the Langevin rate coefficient for two-body ion-atom reaction\cite{rmp2019,langevin,levine-2009}. It should be noted that the ions here can be atomic ions, diatomic molecule ions, or polyatomic ions. Each subsequent reaction occurs based on the reaction products formed in the previous step, as illustrated in Fig.\ref{fig1}(d). The key of tackling this problem requires creating an ion-atom mixture with a large number of ions and atoms. Besides, additional challenge in investigations is the detection and identification of this group of polyatomic molecular ions. With the presence of these difficulties, there is no experimental observation of such sequential ion-atom reaction yet.
	
	Here, we report the creation of a series of polyatomic cations $^{87}$Rb$_\text{N}^+$ (N = 2, 3,…, 18) from a cold mixture of $^{87}$Rb$^+$ ions and $^{87}$Rb atoms. $^{87}$Rb$^+$ ions were continuously produced in CW-laser photoionization of cold $^{87}$Rb atoms in an ion-atom hybrid trap. The spatial coincidence of the ionic cloud and the atomic cloud is ensured by photoionization in a magneto-optical trap (MOT), while the implementation of an ion trap facilitates the storage of the reaction products. As a result, we have efficiently produced and stored a large number of ion-atom reaction products, which results in the cascaded creation of polyatomic cations. We used the combination of time-of-flight mass spectrometry (TOF-MS) and resonant excitation mass spectrometry (REMS) to discriminate these reaction products and directly observed the production of a large number ($\sim 10^6$  $^{87}$Rb$_\text{N}^+$, N = 2,3,…,7) of polyatomic cations $^{87}$Rb$_\text{N}^+$ (N = 2, 3,…, 18) as reaction products. Moreover, by changing the interaction time of ion-atom collisions and the trap time of ions, we observed a sequential generation from diatomic cations to polyatomic cations and the sequential dissociation of polyatomic cations. 
	
	
	%	\section*{Experimental protocol}\label{sec-exp}
	
	
	
	%	\section*{Results}\label{sec-result}
	
	\section*{Production of polyatomic cations by sequential ion-atom collisions}\label{sec-create}
	
	Our experiments begin with the preparation of a mixture of cold $^{87}$Rb atoms and $^{87}$Rb$^+$ ions. The detailed description of our apparatus can be found in our previous study\cite{lv_measurement_2017}. In brief, the ion-atom hybrid trap comprises an Rb standard  magneto-optical trap (MOT) and a mass-selective linear Paul trap (LPT), which are spatially concentric and combined in a polyhedral non-magnetic stainless-steel cavity. A simplified diagram of this device is shown in Fig.\ref{fig1}(a). The magneto-optical trap (MOT) was firstly turned on and cold $^{87}$Rb atoms were loaded to steady state. $^{87}$Rb$^+$ ions were produced through two-step CW-laser photoionization of cold $^{87}$Rb atoms, thus a mixture of cold $^{87}$Rb atoms and $^{87}$Rb$^+$ ion in the MOT was prepared. The cooling/trapping beams serve as the first excitation laser concurrently, while the CW diode laser serves as the second excitation laser, i.e., the ionizing laser, which photoionizes the $^{87}$Rb atoms in the 5$P_{3/2}$ states. The number of ions was counted by the microchannel plate (MCP) by switching off the voltage on the end-cap ring electrode closer to the MCP, the trapped ions were pushed to the MCP and time-of-flight (TOF) spectrum of the ions was recorded by an oscilloscope. 
	
	The TOF spectra obtained in our study are presented in Fig.\ref{fig2}(a), showcasing two distinct types. The first type displays a solitary, broad peak, indicating a shorter duration of ionizing light. The second type, however, reveals the presence of a distinct secondary peak (peak 2) alongside the primary $^{87}$Rb$^+$ peak (peak 1), indicating the substantial formation of $^{87}$Rb$_2^+$ in the mixture of cold $^{87}$Rb atoms and $^{87}$Rb$^+$ ions. It should be noted that peak 1 is narrower and stronger, while peak 2 is wider, corresponding to the reactant $^{87}$Rb$^+$ ions. To determine the absolute count of ions, we have employed the atomic absorption imaging method in conjunction with rate equations of ions and atoms\cite{liang-fit-2022}, showing that there were $\sim10^7$ ions in both peak 1 and 2. Notably, although the generation of polyatomic species may occur in a cascading manner as discussed above, The TOF spectrum can only distinguish up to two spectral peaks, it may be due to the strong overlap of spectral peaks. This assertion is corroborated by the simulated TOF spectrum obtained using the charged particle tracing module in COMSOL Multiphysics\textregistered\cite{comsol}, where it is evident that the broad peak 2 contains more than one ionic species(see Appendices for detail). We will explore methods for directly observing these reaction products in the subsequent section.
	
	In contrast to atomic ions, molecular ions can dissociate. To analyze this phenomenon, we investigated the evolution of the total number of ions and the number of reaction products (peak 2) as a function of hold time in the ion trap. This was done under the condition that the magneto-optical trap (MOT) and ionizing laser were turned off, while the ion trap remained on. As depicted in Fig.\ref{fig2}(b), both the total number of ions and the number of reaction products can be accurately fitted with a single exponential function. Moreover, the rate of decrease in the number of reaction products is faster than that of the total number of ions. While the limited trapping capability of the ion trap accounts for the decreasing total number of ions, another mechanism is responsible for the decreasing of the number of reaction products. As illustrated in Fig.\ref{fig2}(c), the dissociation lifetime of reaction products is shorter when the total number of ions is higher. This means the collision between ions leads to the dissociation of reaction products. Therefore, more ions lead to more collisions and the reaction products dissociate faster. 
	
	The collision energy, which determines the outcome of ion-atom collisions, is influenced by the relative kinetic energy between reactant ions and atoms. While the temperature of the MOT atom is limited to the Doppler cooling limit temperature, which is $\sim$0.2 mK for $^{87}$Rb, the ions stored in the ion trap and those in the MOT which is generated through photoionization of cold atoms exhibit significantly different temperatures (kinetic energy). Based on energy and momentum conservation principles, the initial temperature of $^{87}$Rb$^+$ ions generated by photoionization is estimated to be a few mK. And the maximum temperature for the trapped $^{87}$Rb$^+$ in the ion trap is approximately $10^3\sim10^4$ K as the ion trap's well-depth is approximately 0.7 eV. Our experimental observations indicate that ion-atom collisions primarily occur with ions that have not yet diffused into the ion trap during the photoionization of cold atoms. As shown in Fig.\ref{fig2}(d), after the MOT, the ion trap, and the ionizing laser worked together for 500 ms, we turned off the ionizing laser, and the intensity of the peak 2 is measured as a function of the hold time in the ion trap in both cases where MOT is turned on or off. Obviously, the peak 2 intensities of the two cases are almost equal, which means that the ions in the ion trap almost do not react with atoms. Consequently, the collision energy between $^{87}$Rb$^+$-$^{87}$Rb collisions approximates the kinetic energy of $^{87}$Rb$^+$ ions in the MOT. According to the fitting of experimental data with rate equations\cite{liang-diff-2023,wang-atomion-2022}, the diffusion rate of $^{87}$Rb$^+$ ions in the MOT is approximately 1.0/s, corresponding to a diffusion rate of about 1.0 mm/s and a temperature of $\sim$5.2 mK. Moreover, it is crucial to understand that the collision energy is not entirely converted into the kinetic energy of the reaction products. Therefore, in this study, the collision energy of ion-atom reactions lies in the range of a few mK. Additionally, by controlling the switch of the ionizing laser, we can control the occurrence and cessation of ion-atom collisions, as these collisions predominantly involve ions generated through the photoionization of cold atoms rather than those stored in the ion trap.
	
\end{multicols}
	
	\begin{figure}[htbp]%
		\centering
		\includegraphics[width=\textwidth]{./figures/fig1.jpg}
		\caption{Experimental setup and concepts. (a) Schematic diagram of our ion-atom hybrid trap containing an ionizing laser. The ion trap was operated at a radio frequency (RF) of $\Omega_{trap}=2\pi\times$ 550 kHz with an amplitude of $V_{trap}$ = 140 V. (b) Cold ion-atom mixtures produced by the two-step CW-laser photoionization of laser-cooled $^{87}$Rb atoms in a magneto-optical trap (MOT). (c) Schematic diagram of polyatomic cations generation for $^{87}$Rb referred to Ref.\cite{cote-2002} based on the Langevin-capture model\cite{langevin,levine-2009}. (d) Schematic diagram of the generation of a series of polyatomic cations $^{87}$Rb$_\text{N}^+$ through sequential two-body ion-atom collisions.}
		\label{fig1}
	\end{figure} 
	
	\begin{figure}[htbp]%
		\centering
		\includegraphics[width=0.45\textwidth]{./figures/fig2a.eps}
		\includegraphics[width=0.45\textwidth]{./figures/fig2b.eps}
		\includegraphics[width=0.43\textwidth]{./figures/fig2c.eps}
		\includegraphics[width=0.47\textwidth]{./figures/fig2d.eps}
		\caption{(a) Typical time-of-flight distribution of ion counts on the microchannel plate (MCP); one has two distinct peaks at the reaction time of 750 ms,  i.e., peak 1 and peak 2 in the order of the flight-of-time, while the other appears to have only one broad peak at the reaction time of 100 ms. The time zero is set to the moment when the voltage on the end-cap ring electrode closer to the MCP turns off. (b) The total ion signal and  the intensity of peak 2 including $^{87}$Rb$_2^+$, $^{87}$Rb$_3^+$,$^{87}$Rb$_4^+$,$\ldots$ as a function  of the hold time in the ion trap without cold atoms and the ionizing laser. Measurements were performed under the conditions that the reaction time is 500 ms, the wavelength of the ionizing laser is 478.8 nm, and its intensity is about 265.25 mW/cm$^2$. Here, the time zero is set to the moment that both the ionizing laser and MOT are turned off. (c) The lifetime of $^{87}$Rb$_\text{N}^+$ (N = 2, 3,…, 18), i.e., ions in peak 2 as a function of the total number of ions. The number of the total ions was varied by controlling the duration of the ionizing laser. (d) The intensity of peak 2 in the TOF spectrum was measured as a function of the hold time with/without the presence of cold atoms by switching the MOT on or off. Here hold time is the duration of holding the ions in the ion trap for a predetermined period by switching off the ionizing laser. The time zero is set to the moment the ionizing laser is switched off. These measurements were taken after the same reaction time of 500 ms, i.e., the time that MOT, LPT, and the ionizing laser work together. The wavelength and intensity of the ionizing laser are  478.8 nm and 265.25 mW/cm$^2$, respectively.}
		\label{fig2}
	\end{figure} 
	
\begin{multicols}{2}
	
	\section*{Direct detection of polyatomic cations}\label{sec-observe}
	
	We aim to investigate whether the observed apparently isolated peak 2 in time-of-flight mass spectrometry (TOF-MS) comprises contributions from multiple ion species. As discussed above, the Langevin capture model offers the potential for generating polyatomic molecular cations, as depicted in Fig.\ref{fig1}(d). The position of a peak ( time-of-flight ) in TOF-MS is approximately inversely proportional to the square of the corresponding ion's mass, while the width of the peak increases with the increase of the ion's mass. Consequently, it is plausible that ion signals from a series of polyatomic cations exhibiting precise changes in the atomic number of a single rubidium atom may overlap, leading to the appearance of an apparently isolated and broad peak in TOF MS. We have employed two approaches, combining time-of-flight mass spectrometry with resonant-excitation mass spectrometry, to directly observe these reaction products.
	
	By altering the intensity ratio, particularly by quenching the signal with higher intensity, it becomes possible to separate the ostensibly isolated peak into multiple discernible peaks, thus facilitating their identification. The work in Ref.\cite{drakoudis-iontrap-2006,schuessler-iontrap-1969} demonstrates that mass selective resonant excitation can be employed to suppress undesirable ions from the ion trap. Notably, when the additional radio frequency (RF) resonates with the radial secular frequency of a particular ion species in the ion trap, the ions are heated and subsequently leave the ion trap. The theoretical radial secular frequency of $^{87}$Rb$_\text{N}^+$, denoted as $f_{theor.}$($^{87}$Rb$_\text{N}^+$), is calculated as 109.2/N kHz using the Mathieu equations\cite{cohen}. As a result, the strong signal of $^{87}$Rb$_2^+$ is quenched by applying a 58.4 kHz additional RF field, after which time-of-flight mass spectrometry measurements were conducted. As depicted in Fig.\ref{fig3}(a), the broad peak 2 splits into six distinguishable narrow peaks upon the excitation and expulsion of $^{87}$Rb$_2^+$. By comparing the time-of-flight ratios of these identifiable peaks with the squared mass ratio of $^{87}$Rb$_N^+$, we readily identify these peaks as $^{87}$Rb$_2^+$, $^{87}$Rb$_3^+$, $^{87}$Rb$_4^+$, $^{87}$Rb$_5^+$, $^{87}$Rb$_6^+$, and $^{87}$Rb$_7^+$. Furthermore, the intensities of $^{87}$Rb$_3^+$, $^{87}$Rb$_4^+$, $^{87}$Rb$_5^+$, $^{87}$Rb$_6^+$, and $^{87}$Rb$_7^+$ decrease as the atomic number (N) in $^{87}$Rb$_\text{N}^+$ increases, which  is attributable to the fact that the Langevin ion-atom reaction-rate coefficient is proportional to the inverse square root of the reduced mass ($\mu$) of the colliding particles\cite{rmp2019,levine-2009}. The numerical order of these ions amounts to $10^6$, a sufficiently large value to enable cascade reactions. The fact that he intensity of $^{87}$Rb$_3^+$ is higher than that of $^{87}$Rb$_2^+$ means that intensity of $^{87}$Rb$_2^+$ is weakened. The reason why the molecular Ionic strength does not disappear may be the existence of the second harmonic frequency and ions in  the MOT are not governed by the dynamics of the ion trap. As the value of N in $^{87}$Rb$_\text{N}^+$ increases, the width of the corresponding peak broadens and the intensity diminishes. Molecular ions $^{87}$Rb$_\text{N}^+$ with N$\geq$ 7 cannot be distinguished in the TOF spectrum. To observe these polyatomic cations containing more than seven atoms, we combine resonant-excitation mass spectrometry (REMS) with time-of-flight mass spectrometry (TOF-MS) by sweeping the frequency of the additional RF while employing TOF-MS to determine the total number of ions. The underlying principle is as follows: when the additional RF resonates with the resonance frequency of an ion in the ion trap (determined by macroscopic motion), the ion absorbs the radio frequency, heats up, and subsequently exits the ion trap. This leads to a reduction in the total number of ions, resulting in a dip in the measured resonant-excitation spectrum. The measured resonant-excitation spectrum is shown in Fig.\ref{fig3}(b). Here, we can clearly see the production of polyatomic cations $^{87}$Rb$_\text{N}^+$ (N = 2, 3,…, 18). Note that since the preparation paths of these reaction products are related (shown in Fig.\ref{fig1}(b)), driving out a certain ionic species will also affect other ionic species. Therefore, the positions of some dip signals deviated from the resonant frequencies of the ionic species. We found that their positions agree with the $^{87}$Rb$_\text{N}^+$ (N = 2, 3,…, 18) second harmonic frequencies, which come from ac-side-effect heating caused by perturbing the motion of ions in the ion trap\cite{sivarajah-iontrap-2012,goodman-iontrap-2012}. The method we developed for directly observing charged products from different ion species in an ion trap is straightforward and efficient, suitable for any experiment using an ion-atom hybrid trap, and not constrained by the type of ion. 
	
\end{multicols}	
	
	\begin{figure}[htbp]%
		\centering
		\includegraphics[width=0.45\textwidth]{./figures/fig3a.eps}
		\includegraphics[width=0.54\textwidth]{./figures/fig3b.eps}
		\caption{Direct detection of polyatomic cations. (a) $^{87} $Rb$^+$, $^{87} $Rb$_2^+$, $^{87} $Rb$_3^+$,  $^{87} $Rb$_4^+,\ldots$, $^ {87} $Rb$_7^+$  were directly observed in the time-of-flight (TOF) mass spectrum with the help of resonant-excitation mass spectrometry. The inset expands the view in the TOF range of 60-140 $\upmu$s. Here, the reaction time is 350 ms. After the designated reaction time has passed, the MOT and ionizing laser are turned off, and the additional RF with amplitude $V_{RF}$ = 4 V and frequency 58.4 kHz is applied while the LPT remains on. This process spans 50 $\upmu$s, after which ions are extracted onto the MCP to obtain the TOF spectrum. (b) Resonant-excitation mass spectrum of ions. The dashed line represents the theoretical resonant frequency of polyatomic cations. Here, the reaction time is 750 ms. After the designated reaction time has passed, the MOT and the ionizing laser are turned off, and the additional RF with amplitude $V_\text{RF}$ = 4 V is applied for 1 ms while the LPT remains on at each frequency point. Ions are then extracted onto the MCP to obtain the TOF spectrum. The dashed lines represent the theoretical resonant frequencies of different ion species. Here, the wavelength and the intensity of the ionizing laser are 478.8 nm and 265.25 mW/cm$^2$, respectively. }
		\label{fig3}
	\end{figure} 

\begin{multicols}{2}
	
	\section*{Cascade production and dissociation of polyatomic cations}\label{sec-cascade}
	
	Now we aim to experimentally validate the production and dissociation of polyatomic cations in a cascade manner, as predicted by the classical Langevin-capture model\cite{langevin,levine-2009}. In a cascade production process, the production time of polyatomic cations is expected to vary, which may result in cascade features in the associated dissociation kinetics.
	
	To validate the cascade production of polyatomic cations, we conducted measurements of REMS after varying reaction times to determine the production time of polyatomic cations. The reaction time is  referred to as the duration of photoionization. Fig.\ref{fig4}(a) and (b) display REMS at reaction times of 90 ms, 100 ms, and 105 ms. Notably, the resonant frequencies for polyatomic cations $^{87}$Rb$_\text{N}^+$ (N = 2, 3,…, 18) generally aligned with the dip signals observed in our REMS, as depicted in Fig.\ref{fig4}(d). Dip signals without markers represented the second harmonic frequencies of polyatomic cations. The presence of second harmonic frequencies is a consequence of ac-side-effect heating caused by ion trap perturbations of ion motion\cite{sivarajah-iontrap-2012,goodman-iontrap-2012}. Additionally, each signal of polyatomic cations exhibited varying intensities, suggesting different structural stabilities due to diverse molecular structures. This relationship between N in $^{87}$Rb$^+$N and the intensity of its signal warrants further exploration through a coupled rate-equation model. Notably, at a reaction time of 90 ms, no observable dip signal was detected within the excitation frequency range of 5$\sim$80 kHz. However, at a reaction time of 100 ms, a distinct broad dip signal near the resonant frequency of $^{87}$Rb$_2^+$ was observed, indicating the production of $^{87}$Rb$_2^+$ at that time. As the reaction time increased to 105 ms, the dip signal progressed toward lower extra excitation frequencies, indicating the production of molecular ions with an increased number of atoms. This indicates that when the reaction time grows from 100 ms to 105 ms, $^{87}$Rb$_2^+$ appears first, followed by $^{87}$Rb$_\text{N}^+$ (N = 3, 4,…, 17) at 5 ms. This indicates that the generation of $^{87}$Rb$_2^+$ and $^{87}$Rb$_\text{N}^+$ (N = 3, 4,…, 17) occurs sequentially.
	
	In our investigation of polyatomic ion dissociation, we measured REMS after varying hold times to determine the dissociation time of polyatomic cations. The hold time referred to the duration ions were held in the ion trap without the presence of the magneto-optical trap (MOT) and ionizing laser following a reaction of 750 ms. In Fig.\ref{fig4}(c), the dip signals of $^{87}$Rb$_\text{N}^+$ (N = …, 5,…, 18) at a reaction time of 750 ms were clearly identified. However, when the MOT and ionizing laser were switched off after 750 ms, and the ions were held in the ion trap for 200 ms, some polyatomic cations disappeared while the signals of others with smaller numbers of atoms became intensified. For example, at a 200 ms hold time, the signal of $^{87}$Rb$_{18}^+$ vanished, while the signals of $^{87}$Rb$_{15}^+$ and $^{87}$Rb$_{17}^+$ became clearer. As the hold time increased to 300 ms, the signals of   heavy polyatomic cations $^{87}$Rb$^+$N such as $^{87}$Rb$_{15}^+$ and $^{87}$Rb$_{17}^+$ weakened, while the signals of lighter polyatomic cations like $^{87}$Rb$_8^+$ became more prominent (Fig.\ref{fig4}(d)). Additionally, the second harmonic frequency signals of polyatomic cations also evolved, as evident in the signal of $^{87}$Rb$_{11}^+$, which appeared at a 200 ms delay and disappeared at a 300 ms delay. These findings indicate a sequential dissociation process, in the 0-200 ms hold time, $^{87}$Rb$_{18}^+$, $^{87}$Rb$_{17}^+$, $^{87}$Rb$_{16}^+$ gradually dissociate into  $^{87}$Rb$_{17}^+$, $^{87}$Rb$_{16}^+$, $^{87}$Rb$_{15}^+$, respectively, then when the hold time comes to 300 ms, the signals of heavy polyatomic cations such as $^{87}$Rb$_{17}^+$ and $^{87}$Rb$_{11}^+$ gradually disappear and only the signals of lighter polyatomic cations remain.
	
\end{multicols}	
	
	\begin{figure}[htbp]%
		\centering
		\includegraphics[width=0.45\textwidth]{./figures/fig4a.eps}
		\includegraphics[width=0.45\textwidth]{./figures/fig4b.eps}
		\includegraphics[width=0.45\textwidth]{./figures/fig4c.eps}
		\includegraphics[width=0.45\textwidth]{./figures/fig4d.eps}
		\caption{Cascade features in the production and dissociation processes of polyatomic cations. (a) Comparison between the reaction time of 90 ms and the reaction time of 100 ms. (b) REMS at the reaction time of 105 ms. (c) Comparison between the REMS obtained with no hold time and 200 ms of hold time after the reaction time of 750 ms. (d) Comparison between the REMS obtained with 200 ms and 300 ms of hold time after the reaction time of 750 ms. The dashed line represents the theoretical resonant frequency of polyatomic cations and the blue box illustrates the second harmonic frequency of Rb$_{11}^+$ as an example. Here, the wavelength and the intensity of the ionizing laser are 478.8 nm and 265.25 mW/cm$^2$, respectively. And the duration of addition RF is 1 ms. In (c) and (d), the hold time means after the reaction time of 750 ms, the ions are held in the ion trap for a predetermined duration without the presence of MOT and ionizing laser. }
		\label{fig4}
	\end{figure} 
	
\begin{multicols}{2}
	
	\section*{Conclusion}\label{sec-dis}
	
	 In summary, a significant abundance of cold polyatomic cations, denoted as $^{87}$Rb$_\text{N}^+$ (N = 2, 3,…, 18), was successfully generated within an $^{87}$Rb-$^{87}$Rb$^+$ mixture. This mixture was created through a two-step continuous wave (CW) laser photoionization process of laser-cooled $^{87}$Rb atoms confined in an ion-atom hybrid trap. The formation of these polyatomic cations occurred via sequential two-body ion-atom collisions, as opposed to non-sequential multi-body ion-atom radiative association, mainly due to the exponential decline in the likelihood of their generation with increasing atom numbers involved. The collision energy of ion-atom collisions is estimated not to be higher than millikelvin magnitudes since the kinetic energy of the atomic ions produced during the photoionization of laser-cooled $^{87}$Rb atoms inherited that of their parent atoms. A straightforward and efficient method was developed to directly observe these aforementioned reaction products. This approach effectively combines time-of-flight mass spectrometry and resonant-excitation mass spectrometry, rendering it adaptable to experiments conducted within an ion-atom hybrid trap. Moreover, our investigation successfully demonstrated the occurrence of cascade reactions leading to the formation of these reaction products, with distinct cascade features evident within the associated dissociation processes. The implications of this work extend beyond the confines of our study, offering a plethora of research opportunities. For instance, our findings offer comprehensive insights into the dynamics of cold ion-atom interactions and chemical reactions, thereby enabling investigation into intricate many-body systems. The manipulation, detection, and trapping of polyatomic molecular cations, which possess additional rotational and vibrational degrees of freedom compared to atomic ions, are inherently easier. Consequently, these polyatomic species find relevance in precision measurement and quantum information sciences. This series of polyatomic cations forms an atomically precise metal cluster, which holds significant import in the study of physics from the few-body to many-body perspectives.
	
\end{multicols}
	
	\section*{Acknowledgments}
	
	%\bmhead{Acknowledgments}
	
	The authors express their deep appreciation of Dr. Xin-Yu Luo (Max-Planck-Institute for Quantum Optics) and Prof. Zhen-Sheng Yuan (University of Science and Technology of China) for our fruitful discussions.
	
	This study was supported by the National Key Research and Development Program of China (Grant Nos. 2017YFA0402300 and 2017YFA0304900), the Beijing Natural Science Foundation (Grant No. 1212014), Fundamental Research Funds for the Central Universities, the Strategic Priority Research Program (Grant No. XDB28000000) of the Chinese Academy of Sciences., specialized research fund for CAS Key Laboratory of Geospace Environment (Grant No. GE2020-01), and National Natural Science Foundation of China (Grant Nos. 61975091, 61575108).
		
	% Your references go at the end of the main text, and before the
	% figures.  For this document we've used BibTeX, the .bib file
	% scibib.bib, and the .bst file Science.bst.  The package scicite.sty
	% was included to format the reference numbers according to *Science*
	% style.
	
	%BibTeX users: After compilation, comment out the following two lines and paste in
	% the generated .bbl file. 
	
	\bibliography{polyatomic}
	
	\bibliographystyle{Science}
	
	\section*{Methods}
	
	%	\subsection{Methods}
	
	
	\subsection{Simulation of the ion time-of-flight spectrum using COMSOL Multiphysics\textregistered}
	
	The ion time-of-flight (TOF) spectrum was simulated using the charged particle tracing module in COMSOL Multiphysics\textregistered\cite{comsol} based on our ion-atom hybrid apparatus\cite{lv_measurement_2017}. The simulations were performed considering Coulomb interactions among ions and between ions and the electrical field of the ion trap. Ions were released from a $r$ = 1 mm spherical area at the center of the trap, and the ionic cloud evolved for 500 ms and then was extracted to obtain a simulated TOF spectrum, as discussed in Ref.\cite{liang-fit-2022}. Here, two kinds of ionic clouds were respectively simulated: 10$^5$ $^{87}$Rb$^+$ + 5$\times$10$^4$ $^{87}$Rb$_2^+$ + 3$\times$10$^4$ $^{87}$Rb$_3^+$  + 2$\times$10$^4$ $^{87}$Rb$_4^+$, and 10$^5$ $^{87}$Rb$^+$ + 10$^5$ $^{87}$Rb$_2^+$, to illustrate the differences in the TOF spectrum with or without the presence of polyatomic molecular ions. The results are shown in Fig.\ref{figs1}. Obviously, the peak 2 in the simulated TOF including $^{87}$Rb$_3^+$ and $^{87}$Rb$_4^+$ descends much slower and is closer to the experimental TOF signal compared to the peak 2 in the simulated TOF without $^{87}$Rb$_3^+$ and $^{87}$Rb$_4^+$. Therefore these results suggest the signals of $^{87}$Rb$_\text{N}^+$ are included in the peak 2 of the experimental TOF spectrum.
	
	\begin{figure}[h]%
		\centering
		\includegraphics[width=0.8\textwidth]{./figures/figs1.eps}
		\caption{Comparison between experimental and simulated time-of-flight (TOF) mass spectrum by COMSOL Multiphysics\textregistered\cite{comsol}. Experimental TOF spectra were measured in the photoionization process of our ion-atom hybrid trap. The wavelength and the intensity of the ionizing laser were 447 nm and 157.48 mW/cm$^2$, respectively. The duration of photoionization was 500 ms. The difference between the two simulation curves is that one molecule ion only contains $^{87}$Rb$_2^+$, and the other contains $^{87}$Rb$_2^+$, $^{87}$Rb$_3^+$, $^{87}$Rb$_4^+$. In the simulation, 10$^5$ $^{87}$Rb$^+$ + 5$\times$10$^4$ $^{87}$Rb$_2^+$ + 3$\times$10$^4$ $^{87}$Rb$_3^+$  + 2$\times$10$^4$ $^{87}$Rb$_4^+$(red curve) and 10$^5$ $^{87}$Rb$^+$ + 10$^5$ $^{87}$Rb$_2^+$(blue curve) were added with an initial temperature of 20 mK.}
		\label{figs1}
	\end{figure} 
	
	
	% For your review copy (i.e., the file you initially send in for
	% evaluation), you can use the {figure} environment and the
	% \includegraphics command to stream your figures into the text, placing
	% all figures at the end.  For the final, revised manuscript for
	% acceptance and production, however, PostScript or other graphics
	% should not be streamed into your compliled file.  Instead, set
	% captions as simple paragraphs (with a \noindent tag), setting them
	% off from the rest of the text with a \clearpage as shown  below, and
	% submit figures as separate files according to the Art Department's
	% instructions.
	
	
	\clearpage
	
\end{document}

