
\section{Task Specification}    \label{sec:task_specification}

As specified earlier, the methodology focusses on a transportation task.
The goal is to place a payload next to a previously placed payload, or 
mechanical constraint, as illustrated in Fig. 
\ref{fig:discrete_states}.


%%%%%%%%%%%%%%%%
% assumptions? %
%%%%%%%%%%%%%%%% 
\subsection{Assumptions and Knowledge}

The task specification of the use case investigated in this work
makes following assumptions:
\begin{itemize}
    \item The payload is not fragile and the environment can be used to 
    mechanically dampen vibrations, without damaging the payload or the 
    environment.
    
    \item The world model information, such as the position of 
    objects already placed in the workspace, is known to the controller.
    
    \item Motor brakes can be individually activated and controlled.
    
    \item The top motor is positioned over the previously placed 
    payload, such that the current payload can be swung over it.

    \item Grasping the payload is out of the scope of this paper
\end{itemize}  

\noindent The control approach makes use of the following 
knowledge in the task specification: 
\begin{itemize}
    \item The payload has a significant mass, thus
    gravitational force can be used to ensure cable tension. 

    \item The execution can be split up in a lifting, transporting, 
    dropping and (optional) fine-positioning state.
\end{itemize}

\subsection{Energy measures}
The total electric power consumed by a single motor is given by: 
\begin{equation}
\begin{gathered}
    P_{el} = P_{mech} + P_{mech,loss} + P_{el,loss} 
    = V I_a,
\end{gathered}
\end{equation}  
with $P_{mech}$ the mechanical power, 
$P_{mech,loss}$ the mechanical losses due to friction in the motor and 
transmission, $P_{el,loss}$ the electrical losses of the motor, $V$ the 
voltage applied to the motor, and $I_a$ the armature current. 

The electrical losses are typically dominated by the copper losses 
$P_{cu,loss}$, determined by the motor resistance $R_a$ and armature
current $I_a$: 
\begin{equation}
    \begin{gathered}
        P_{cu,loss} = R_a^{} I_{a}^{2}.
    \end{gathered}
\end{equation}
The energy consumption can be found by integration of the
power over time $t$:
\begin{equation}
\label{eq:el_energy}
    E_{el} = \int_{0}^{t} P_{el} \mathrm{d}t.
\end{equation}
