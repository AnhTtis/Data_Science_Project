\section{Experimental Results}                   \label{sec:results}
The proposed approach was experimentally compared against 
a sequence of PTP motions, each with a trapezoidal velocity profile.

The trajectories corresponding to both methods are shown in 
Fig. \ref{fig:trajectories}.
The PTP-trajectory performs a vertical lift up, followed by a horizontal 
transportation, a vertical drop slightly above the ground, and finally a 
horizontal and vertical movement that finishes the placement. 
Parts of the trajectory that are similar 
to the end criteria of the proposed method are marked with 
circled numbers in the figure
The PTP-trajectory does not make use of a contact with the ground to 
detect when the drop has ended, it goes straight from 
\circled{3} to \circled{5}.

It should be noted that the side motors are not intended for servo 
applications, causing jittery motions at low speeds.
This caused a chattering effect when the PTP controller was used, 
since an exact position needs to be followed, resulting in higher 
jerks (e.g. between \circled{1} and \circled{2}).
The proposed controller does not rely on position tracking, 
and as such does not induce this chattering.

\begin{figure}[h!]
    \centering
    \includegraphics[width = 0.68\linewidth]{Figures/background.eps}
    \caption{Executed trajectory of the proposed and the PTP controller.}
    \label{fig:trajectories}
    \vspace{-6mm}
\end{figure}

\noindent
The consumed energy is directly related to the total 
execution time. The lift and drop state are executed at the same 
speed, since these are similar and interchangeable parts in 
both trajectories. The other parts of the PTP-trajectory are  
scaled such that the same total execution time is achieved, 
and the maximum motor velocity is not exceeded. 
The energy consumption of each motor is given by eq. 
\eqref{eq:el_energy}, and is depicted in Fig. 
\ref{fig:energy_consumption}. Table \ref{table:energy} gives the 
consumed energy of each motor during each subtask, and the 
cumulative energy consumed by all motors.

\begin{figure*}[h!]
    \vspace{-4mm}
    \centering
    \begin{subfigure}[b]{0.455\linewidth}
        \centering
        \includegraphics[width = \textwidth]{Figures/energy_proposed_method.eps}
        \caption{Proposed controller}
        \label{fig:energy_swing}
    \end{subfigure} 
    \begin{subfigure}[b]{0.455\linewidth}
        \centering
        \includegraphics[width = \textwidth]{Figures/energy_ptp.eps}
        \caption{PTP controller}
        \label{fig:energy_PTP}
    \end{subfigure}
    \caption{Energy consumption corresponding to the trajectories 
    of Fig. \ref{fig:trajectories}, for each motor, as well as for 
    the whole system.
    }
    \label{fig:energy_consumption}
\end{figure*}

\begin{table}[h]
\vspace{-4mm}
\caption{Energy consumption during each subtask, and the 
cumulative total.}
\label{table:energy} 
\centering
\setlength\tabcolsep{4.55pt}
\begin{tabular}{|l|l|c|c|c|c|c|c|}
    \hline
    \multicolumn{2}{|c|}{\multirow{2}{*}{}} &
    \multicolumn{6}{|c|}{Subtask energy consumption [J]} \\
    \cline{3-8}
    \multicolumn{2}{|c|}{} &
    \circledtable{1} \!$\rightarrow$\!\!\!
    \circledtable{2} 
    &
    \circledtable{2} \!$\rightarrow$\!\!\!
    \circledtable{3}  
    &
    \circledtable{3} \!$\rightarrow$\!\!\!
    \circledtable{4} 
    &
    \circledtable{4} \!$\rightarrow$\!\!\!
    \circledtable{5} 
    &
    \circledtable{5} \!$\rightarrow$\!\!\!
    \circledtable{6} 
    &
    \circledtable{6} \!$\rightarrow$\!\!\!
    \circledtable{7} \\
    \hline
    \multirow{2}{*}{Top} & 
    PTP & \raisebox{-.6pt}{183.87} & \raisebox{-.6pt}{162.81} & 
        \multicolumn{2}{|c|}{\raisebox{-.6pt}{94.76}} & 
        \raisebox{-.6pt}{26.46} & \raisebox{-.6pt}{18.80} \\  
    \cdashline{3-8}
    & Swing & \raisebox{-1.1pt}{199.06} & \raisebox{-1.1pt}{7.16} & 
        \raisebox{-1.1pt}{76.89} & \raisebox{-1.1pt}{27.97} & 
        \raisebox{-1.1pt}{9.10} & \raisebox{-1.1pt}{11.15} \\
    \hline
    \multirow{2}{*}{Left} & 
    PTP & \raisebox{-.6pt}{62.85} & \raisebox{-.6pt}{18.08} & 
        \multicolumn{2}{|c|}{\raisebox{-.6pt}{21.59}} & 
        \raisebox{-.6pt}{24.05} & \raisebox{-.6pt}{24.45} \\  
    \cdashline{3-8}
    & Swing & \raisebox{-1.1pt}{71.22} & \raisebox{-1.1pt}{0.17} & 
        \raisebox{-1.1pt}{0} & \raisebox{-1.1pt}{0} & 
        \raisebox{-1.1pt}{0} & \raisebox{-1.1pt}{0} \\
    \hline
    \multirow{2}{*}{Right} & 
    PTP & \raisebox{-.6pt}{5.90} & \raisebox{-.6pt}{47.77} & 
        \multicolumn{2}{|c|}{\raisebox{-.6pt}{27.95}} & 
        \raisebox{-.6pt}{6.40} & \raisebox{-.6pt}{3.99} \\  
    \cdashline{3-8}
    & Swing & \raisebox{-1.1pt}{6.27} & \raisebox{-1.1pt}{14.98} & 
        \raisebox{-1.1pt}{0.01} & \raisebox{-1.1pt}{3.56} & 
        \raisebox{-1.1pt}{0.15} & \raisebox{-1.1pt}{0.71} \\
    \hline
    \hline
    Total  & 
    PTP & \raisebox{-.6pt}{263.35} & \raisebox{-.6pt}{489.81} & 
        \multicolumn{2}{|c|}{\raisebox{-.6pt}{616.02}} & 
        \raisebox{-.6pt}{651.34} & \raisebox{-.6pt}{674.54} \\  
    \cdashline{3-8}
    (Cumulative)
    & Swing & \raisebox{-1.1pt}{276.54} & \raisebox{-1.1pt}{298.85} &
        \raisebox{-1.1pt}{375.74} & \raisebox{-1.1pt}{407.27} & 
        \raisebox{-1.1pt}{416.52} & \raisebox{-1.1pt}{428.38} \\
    \hline 
\end{tabular}
\vspace{-4mm}
\end{table}
\vspace{-4mm}

\noindent
During \circled{1} \!$\rightarrow$ \circled{2} energy is injected to 
overcome gravity. 
During \circled{2} \!$\rightarrow$ \circled{3} the biggest 
difference in energy consumption occurs. This is 
mainly because the top motor, which consumes the most power, is braked 
in the proposed method. In the proposed method no force is
generated in the left motor from this point onward. Either gravity 
produces tension in the corresponding cable for the rest of the motion, 
or the motion occurs in the half plane where this motor is disabled.

In this particular experimental setup, the weight of the payload cannot overcome the friction of 
the top motor, due to its high gear ratio. Thus, it still needs to be 
powered in order to drop the payload during \circled{3} 
$\!\rightarrow$ \circled{4} of the proposed method. 
However, by using the brakes of the right motor to control the drop, 
instead of actuating the motor, no power is consumed by that joint 
compared to the PTP method.

For the PTP controller, the brakes 
are continuously energised and consume a constant power. In the 
proposed control method, the brakes are only energised when it is 
necessary (Fig. \ref{fig:brake_time}). The consumed energy is calculated by 
multiplication of the on-time of the brake, with its 
power consumption (Table \ref{table:brake_energy}).

\begin{figure}[!h]
\vspace{-4mm}
\centering
\includegraphics[width = 7.2cm ]{Figures/brake_time.eps}
\caption{Time during which each brake is active in the proposed 
        method. The left brake is omitted from this figure since 
        it is always off.
        } 
\label{fig:brake_time}
\vspace{-4mm}
\end{figure}

\vspace{-4mm}
\vspace{-4mm}

\begin{table}[!h]
\vspace{-4mm}
\centering
\setlength\tabcolsep{6pt}
\caption{Brake energy consumed for the 
    PTP- and the proposed controller.}
\label{table:brake_energy}
\begin{tabular}{|l|c|c|c||c|}
    \hline
    \multirow{2}{*}{} & \multicolumn{4}{|c|}{Brake energy consumption [J]}  \\ 
    \cline{2-5} 
    & Left & Right & Top & Total \\
    \hline
    PTP & 104.37 & 104.37 & 170.79 & 379.54 \\
    \hline
    Swing & 103.73 & 94.03 & 97.60 & 295.36 \\
    \hline
\end{tabular}
\vspace{-4mm}
\end{table}

\noindent
As indicated in Tables \ref{table:energy} and \ref{table:brake_energy}, 
the total energy consumed by the robot with the proposed approach is 
$723.74J$ compared to $1054.08 J$ of the PTP approach, which is a gain 
of more than $30\%$.