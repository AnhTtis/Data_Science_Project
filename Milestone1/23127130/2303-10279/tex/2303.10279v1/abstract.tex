\begin{abstract}
This paper focusses on the energy-efficient control of a cable-driven
robot for 
tasks that only require precise positioning at few points in their
motion, and where that accuracy can be obtained through contacts. This
includes the majority of pick-and-place operations.

Knowledge about the task is directly taken into account when specifying 
the control execution.
The natural dynamics of the system can be exploited when there is a 
tolerance on the position of the trajectory. 
Brakes are actively used to replace standstill torques, and as passive 
actuation.
This is executed with a hybrid discrete-continuous controller. 
A discrete controller is used to specify and coordinate between subtasks,  
and based on the requirements of these specific subtasks, specific, robust, 
continuous controllers are constructed. This approach allows for less stiff 
and thus saver, and cheaper hardware to be used.
For a planar pick-and-place operation, it was found that this results 
in energy savings of more than $30\%$. However, when the payload 
moves with the natural dynamics, there is less control of the followed 
trajectory and its timing compared to a traditional trajectory-based 
execution.
Also, the presented approach implies a fundamentally different way to specify 
and execute tasks.

\keywords{
task-specific control
\and cable-driven parallel robot
\and passive brake control 
\and pick-and-place
\and natural constraints
}
\end{abstract}
