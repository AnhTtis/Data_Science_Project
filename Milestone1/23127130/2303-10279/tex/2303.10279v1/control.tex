\section{Control Strategy} \label{sec:control_strategy}
The controller is of the hybrid continuous-discrete type.
The task is split up in subtasks which are executed 
with specific continuous controllers. Monitors are used to 
trigger transitions in the discrete control, implemented 
as a finite state machine (FSM). 

\subsection{Continuous control} \label{sec:continuous_control}
At the lowest level, each of the motors is 
either controlled by a velocity PID, or a current controller. 
The latter offering more opportunities for energy savings, at the cost 
of a higher control design effort.
Depending on the subtask and the joint, inverse kinematics are 
used to construct a Cartesian velocity controller, 
or a joint current controller.
The layout of the cable robot is illustrated in Fig. 
\ref{fig:InverseKinematics}. 

\begin{figure}[h]
    \vspace{-4mm}
    \centering
    \includegraphics[width = 5.5 cm]{Figures/InverseKinematics.eps}
    \caption{Kinematics of a planar, redundant cable robot. In the 
    depicted scenario,  
    the right actuator (depicted in blue) is not being driven.}
    \label{fig:InverseKinematics}
    \vspace{-4mm}
\end{figure}

\vspace{-6mm}
\subsubsection{Cartesian controller}
The top motor act as a hoist, and is situated such that it can 
always deliver the majority of the gravity compensating force,
and hence can take the role of a hoist.
Depending on which side of the vertical plane (Fig. 
\ref{fig:InverseKinematics}) the payload is situated, only the side motor that is 
in the same half plane will be able to deliver a force that can 
counteract gravity. The other one is not used for the 
manipulation.
This reduces the redundancy to zero, meaning that a unique 
inverse kinematic solution exists, with two complementary discrete 
modes. 

The control input of the robot is the velocity of the non-redundant 
motors which 
is related to the rate of change of their cable length 
$\dot{d_i} = n_i r_i \dot{\theta_i}$. With $n_i$, $r_i$, and
$\dot{\theta_i}$ respectively the gear ratio, the drum radius, and 
the velocity of motor $i$.
In the following, the symbol $x_j$ and $x_j'$ signify parameter $x$ 
of respectively the driven and non-driven side motor.
The Jacobian can easily be derived from the kinematics. The resulting 
inverse kinematic equations are given by:

\begin{equation}
    \begin{bmatrix}
        \dot{d_1} \\ 
        \dot{d_j} 
    \end{bmatrix}
    = 
    \begin{bmatrix}
        \dfrac{-\mathrm{sin}(\alpha_j)}{\mathrm{sin}(\alpha_1-\alpha_j)} 
        & \dfrac{\mathrm{sin}(\alpha_1)}{\mathrm{sin}(\alpha_1-\alpha_j)} \\[14pt]
        \dfrac{\mathrm{cos}(\alpha_j)}{\mathrm{sin}(\alpha_1-\alpha_j)}
        & \dfrac{-\mathrm{cos}(\alpha_1)}{\mathrm{sin}(\alpha_1-\alpha_j)}
    \end{bmatrix}^{-1}
    \begin{bmatrix}
        V_x \\
        V_y
    \end{bmatrix}
    ,
\end{equation}
% 
% \noindent
when the movement is in the `push' direction of the driven side motor 
($\dot{l_i}>0$), gravity is used as the driving force of the 
motion. 
To ensure that the non-driven side cable does not slack, a  
current, just slightly larger than the static friction, is 
maintained in the non-driven side motor when the motion is along 
the $-\dot{d_j}'$ direction. Otherwise gravity ensures cable tension, and 
the motor is not powered.

\vspace{-2mm}
\subsubsection{Current controller}
The current controller is used when one of the actuators is braked.
Braking an actuator implies a geometrical
constraint, such that the end effector has to be on a spherical 
surface with a radius that is determined by the cable length. 
This reduces the mobility of the end effector to 1 DoF in a plane, 
where the other joints can be used to move along the circular constraint. 
This results in simple kinematics (assuming the braked cable 
remains tensioned).

This method is used when the desired motion is in the direction of 
gravity, and thus the control can occur passively. E.g., when dropping 
the payload the side motor brake is controlled to avoid a holding 
torque (Fig. \ref{fig:BrakeDrop}). 

\begin{figure}[h]
    \vspace{-4mm}
    \centering
    \includegraphics[width = 5.2cm]{Figures/BrakeDrop.eps}
    \caption{Dropping the payload by passive control. Red circle segments
    indicate the side motor is braked. Orange segments indicate that the 
    side motor is passive, allowing the cable to move freely due to gravity.}
    \label{fig:BrakeDrop}
    \vspace{-4mm}
\end{figure}

\vspace{-6mm}
\subsection{Discrete control}   \label{sec:discrete_control}
The aforementioned continuous control is implemented for each of the 
specific subtasks by means of a higher level discrete controller. 
Figure \ref{fig:discrete_states} illustrates the different subtasks. 
Each circled number represents when the end criteria monitor of one of 
the discrete states should trigger, which serves as a signal for the 
discrete controller to go to the following state.      
The high level control consists of the states illustrated in Fig. 
\ref{fig:discrete_states}

\begin{figure}[h]
    \vspace{-2mm}
    \centering
    \begin{subfigure}[b]{0.542\linewidth}
        \centering
        \includegraphics[height=3.5cm]{Figures/DiscreteStates1.eps}
    \end{subfigure}
    \begin{subfigure}[b]{0.45\linewidth}
        \centering
        \includegraphics[height=3.5cm]{Figures/DiscreteStates2.eps}
    \end{subfigure} 
    \caption{Different subtasks to be executed. The arrows indicate the motion 
        in each subtask. The numbers 
        indicate when a monitor triggers that indicates the end 
        of the current subtask. 
        Cables depicted in green are actuated, in orange can passively 
        move, in red are constrained by a brake.  
        }
    \vspace{-4mm}
    \label{fig:discrete_states}
\end{figure}

\vspace{-8mm}
\subsubsection{Lifting state:}
Monitor \circled{1} indicates the system is operational and the first subtask can be executed.
The load is lifted upwards with a 
feed-forward Cartesian velocity input.   
When the end effector has reached a certain height $h_d$, monitor 
\circled{2} 
triggers, signalling the discrete controller to go to the `swing state'.
$h_d$ is such that the payload does not collide with previously 
placed payloads, nor the ground during the swing state. 
This height can be determined by, e.g., a camera system and 
is assumed to be known in this paper. 
During this state, energy is injected into the payload and stored as 
gravitational energy. 

\vspace{-2mm}
\subsubsection{Swing state}
Releasing the side motor power causes the payload to execute a free 
fall. However, the top motor brake is  activated, 
constraining the motion along a circle, resulting in a pendulum-like motion. 
During this motion, the direction of the end position is monitored. 
After half a period, when the end effector reaches its apex, 
monitor \circled{3} triggers. This causes the brake of the other side 
motor to activate, holding the payload steady. 
A small amount of energy is consumed by a single 
side motor to maintain cable tension. By keeping the brake of the  
top motor active, no holding torque (and thus no energy) is required.

\vspace{-2mm}
\subsubsection{Drop state}
After the swing, the load is dropped to the ground in the vicinity 
of the target position by applying braking actions (Fig. \ref{fig:BrakeDrop}).
During the dropping motion, the current of the top motor is monitored.
A sudden discrepancy  
of this value can be used to detect an impact force. This is used to 
detect the collision when the payload touches the ground, triggering 
monitor \circled{4}. 
Which in turn switches to the fine-positioning state.

\vspace{-2mm}
\subsubsection{Fine-positioning state}
A sequence of actions is performed to place the payload at the target 
position. First the load is slightly lifted upwards until the cable 
length $l_2$ is such that the payload can not collide with the ground. 
This triggers monitor
\circled{5}. 
Which causes the top motor to brake and the side 
motor to release power, causing a free swing motion, at very low speed 
until the payload collides with the 
previously placed payload, triggering monitor
\circled{6}.
Afterwards the payload is moved straight down with a Cartesian controller, 
until it collides with the ground triggering monitor
\circled{7}, terminating the fine-positioning and the task execution.
To lift the payload a small amount of energy is injected, which is also true 
for the final drop since this is done with active actuation. 
 