\section{Experimental Set-up}          \label{sec:experiment_setup}
These concepts were validated on 
a parallel cable-driven robot (Fig. \ref{fig:setup}). 
The system is able to manipulate a payload of $14$kg
in the vertical plane, with a workspace of $1\times 1$m 
through a 3 DoF redundant actuation. 

A $750$W Beckhoff AM8032 motor with a gear ratio of 70, controlled
by a Beckhoff AX5203 drive, is positioned at the top such that it 
mostly compensates the gravitational force of the payload. 
It has a normally-on electromagnetic brake, allowing discrete braking 
actions. A power of $P_{b,top}=11$W is required to deactivate the brake.
The cable is clamped and wound around a $4.2$cm radius drum that is 
connected at the shaft output. 

The payload is connected at both sides to a $188$W BLDC motor 
(Fig. \ref{fig:bldc}). Both motors are controlled through a VESC 
drive, an open-source hardware project \cite{VESC}. 
The motors have an internal gear ratio of $8$, and are 
connected by a timing belt and pulley system with an additional 
gear ratio of $30/16$ to a rotary shaft with a radius of $8$mm.
The cables are directly wound around this shaft, and guided along 
pulleys towards the side of the payload (Fig. \ref{fig:setup} and 
\ref{fig:bldc}).
The motors are equipped with normally-off electromagnetic 
brakes, instead of normally-on due to long supplier lead times on 
the latter. However, in an industrial application the robot
would be equipped with normally-on brakes. Thus, the brakes will 
be treated as normally-on during the energy consumption analysis.
A power of $P_{b,left}=P_{b,right}=8$W
is necessary to activate the brakes.

\begin{figure}[h]
    \vspace{-4mm} 
    \centering
        \includegraphics[width = 0.58\linewidth]{Figures/bldc.eps}
    \caption{BLDC motor, brake and timing belt transmission 
        used for the side actuation.}
    \vspace{-4mm}
    \label{fig:bldc}
\end{figure}

\noindent
All the used cables are made from Dyneema SK78 with a diameter of $1$mm. 
These cables are lightweight, have a low working stretch ($<1\%$), 
and can carry up to $1.95$kN. 
The current $I_a$ and voltage $V$ supplied to the motors are 
directly obtained from the drives. 
% 
The control feedback relies on the position of the end effector. 
In most cable robots this can be derived from the encoder 
positions, but since cable tension cannot be ensured during the 
free swing of the payload, a camera with a sensor size of 
$1920\times 1280$p is used. 
This allows direct end position tracking of the end effector at 
a rate of $150$Hz with a resolution of $\pm 1$mm.  