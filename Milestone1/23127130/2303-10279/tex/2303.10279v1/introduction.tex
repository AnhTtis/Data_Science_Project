
\section{Introduction}    \label{sec:introduction}
% | Problem statement | %
Traditionally the execution of robot manipulator tasks is not 
focussed on limiting the energy consumption, but rather on speed and 
precision. 
This often results in robots that are more precise, and thus 
stiffer, heavier, and more
expensive, than strictly necessary for the task.
In addition, increasing energy costs and a growing need for more 
sustainability push for more energy-efficient solutions.
Practical applications of tasks that can also be accomplished with 
less accurate robots are numerous in industry, e.g. palletising, truck 
unloading, box stacking, etc. 
These types of pick-and-place tasks only require high precision 
at the start and end of the execution, but not for the gross of
the motion.
These tasks are the main focus of this paper and will 
henceforth be referred to as `transportation tasks'.

This paper demonstrates that by adapting the definition of a task 
in a smart way, by utilising knowledge about the system and task directly
in the control execution,  
a more energy-efficient control can be achieved using simple control 
strategies that do not require accurate modelling nor a high 
computational load. 
To illustrate these strategies, a robot manipulator was built to 
manipulate relatively high payloads of 10 to 100kg (Fig. \ref{fig:setup}). 
For the majority of the task execution, high precision 
is not necessary, and natural constraints \cite{Eppner2015} can be 
used to achieve the required precision. 
By using cables as actuation, the moving mass of the 
robot is minimal, and lowering the 
inherent energy consumption. 

\begin{figure}[ht]
    \vspace{-4mm}
    \centering
    \includegraphics[width = 6.4cm]{Figures/setup.eps}
    \caption{Cable-driven parallel robot that was used in this work. 
            The actuated cables are highlighted with blue lines.
            All motors are equipped with a brake.}
    \label{fig:setup}
    \vspace{-4mm}
\end{figure}

\noindent
Research covering non-stiff manipulators for such applications
typically attempts to mimic the general purpose robot requirements 
and applications, actively trying to eliminate vibrations induced by 
its natural dynamics \cite{Malzahn2014}. This work argues that a 
better way is to compete on specific application use cases by 
making them more energy efficient, and 
not attempt to replace industrial robots altogether.

% | Literature Situation | %
The energy consumption can be minimised  
by optimising the executed trajectory, by optimising the 
execution time \cite{Pellicciari2013}, or the followed path in 
space \cite{Paes2014}. However, the robot is still constrained 
to follow a point-to-point (PTP) trajectory that might 
differ from the natural dynamics. 

Attaching springs to a SCARA robot in order to store energy in between 
cycles of a task has shown to significantly increase
the energy efficiency \cite{Goya2012}.
The natural dynamics of a robot with series elastic actuators (SEA) were 
also used to achieve end effector velocities that were much higher than 
a stiff robot \cite{Haddadin2012}, limiting the required power of the motors.
While these works show promising results, the concept of moving with the
natural dynamics has not been explored extensively in robotic manipulation.

To hold a constant position during control, usually a standstill torque is 
generated in the actuators, continuously consuming power.    
Instead brakes could be used to achieve the same effect.
Commercial robots are already equipped with normally-on brakes, 
consuming a constant power to deactivate the brakes. 
Thus, utilising the brakes in the control has a double positive 
effect on the energy consumption. 
The downside being that it introduces a time delay to (de)activate 
them, and the user usually cannot individually control the brakes 
of an industrial robot.
Brakes have already been used to passively control robots
that experience an external force
\cite{Hirata2011,Andreetto2016}. 
Braking a joint implies a geometric constraint, discretely changing
the natural dynamics of the system. 
This constraint can be easily incorporated in the kinematic control. 

A common challenge in cable robots is the redundancy resolution 
\cite{Lamaury2013,Oh2007}. As will be explained in section 
\ref{sec:continuous_control}, the method described 
in this work will solve this by selectively disabling the motors.

% | Contributions | %
The task context associated with the application determines 
the requirements and constraints throughout the task execution.
For transportation tasks, these requirements and 
constraints are subject to change during the operation. 
For example, the placement or insertion of a payload typically 
requires a higher precision than the transportation in free space.  
This knowledge can be taken into account to split the task into 
multiple simple subtasks with different requirements.
For these subtasks, specific controllers and mechatronics can be 
developed that focus on their robust execution, and 
their specific requirements.
Monitors can be used to track the trends of the continuous 
execution, and coordinate the discrete switching between the controllers.
% 
Thus, for flexible robots
it makes sense to focus on such tasks 
that do not have strict precision requirements, or where this 
precision does not need to originate directly from the control.

Summarising, the contributions of this paper are the following: 
\begin{itemize}
\item 
        Exploiting the natural dynamics of the system for higher 
        energy-efficiency task execution, while still fulfilling the 
        required position precision of the task. 
        
\item 
        Using insights about the task and the mechanical 
        system to develop simple, robust continuous controllers  
        with realtime task execution monitors that feed into  a discrete task execution.

\item
        Achieving the required task precision by making use of natural 
        constraints of the environment, or artificially induced constraints 
        on the cable lengths by means of braking. 
\end{itemize}
