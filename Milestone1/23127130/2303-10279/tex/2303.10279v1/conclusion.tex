\section{Discussion and Conclusion}            \label{sec:conclusion}
In this paper a novel energy-efficient control method was introduced. 
This method is built upon simple principles, and only uses the 
kinematics of the system. It does not rely on complex models and 
does not need extensive quantitative identification. 
The dynamic effects in the motors and cables could be neglected.  
However, more effort is required to
specify the task in the control execution. While this might 
be beneficial for repetitive tasks with high cycles, for 
task with low cycles, the added 
effort might not outweigh the energy gains.  

Based on experimental results of a transportation task, 
the total energy consumption was $31\%$ 
lower with the proposed method compared to a conventional PTP controller,
on exactly the same hardware and software setup. 
The current implementation is not yet optimised, so it is expected 
that the energy can still be lowered if the execution is optimised.
This would however require knowledge of the dynamics of the system, 
and thus a more in-depth identification.
The principles of this method can be used for any task that have 
similar tolerances on path following and timing as the (subtasks) 
of the described use case.

The validation set-up was built to a large extend with low-cost 
hardware that was readily available. 
The jittery behaviour of the side motors (especially at low speeds) 
currently limit the achievable 
tracking behaviour.
But even with this non-ideal hardware, the task could be reliably 
executed.

In future work additional actuation 
will be added such that the top motor can be moved horizontally,
and the swing can occur wherever in the workspace, giving more 
flexibility in the execution. 